% vim: keymap=russian-jcukenwin
%%beginhead 
 
%%file 26_12_2019.fb.lesev_igor.1.padenie_rossii.cmt
%%parent 26_12_2019.fb.lesev_igor.1.padenie_rossii
 
%%url 
 
%%author_id 
%%date 
 
%%tags 
%%title 
 
%%endhead 
\subsubsection{Коментарі}
\label{sec:26_12_2019.fb.lesev_igor.1.padenie_rossii.cmt}

\begin{itemize} % {
\iusr{Елена Ланская}

Вы, почему-то, Игорь, не заметили, что Украина сама становится страной-изгоем.
И, если бы ее не использовали в качестве палки, дразнящей русского медведя, то
от этой страны уже давно остались рожки да ножки. Историческое прошлое есть у
всех.. К сожалению, государство во многом зависит от власти. Далеко ходить не
надо.. Украина наглядный пример..! Так, и, в России были предатели.. Горбачев и
Ельцин. РФ.., страна огромнейшая. Там трудно порядок наводить. Но.., они
стараются, борятся! Насколько это возможно.Если бы еще и "ЕР" сместить, вообще
хорошо бы было. С колен Россия поднимается..! Крымский мост чего стоит..!
Сейчас начинается новый мегапроект..порт СМП, еще.. мост через Лену..! Да..,
много чего. В отличие от Украины.. Да, и Запад, все больше, начинает в сторону
Москвы разворачиваться !


\iusr{Игорь Лесев}
Пост был не об Украине. В Чаде, думаю, еще хуже.

\iusr{Andrei Raschuk}
Надо было опус назвать «Величие Украины»

\iusr{Игорь Лесев}
это название я для продолжения приберег

\iusr{Тимур Филоненко}
\textbf{Игорь Лесев} Интересно будет почитать.

\iusr{Елена Ланская}
Тролли из-под плинтуса выползать начали..

\iusr{Polina Zorina}
Журналюги умеют плести кружева из слов. Художники крупными мазками. Докажут без проблем что дважды два - пять, а земля плоская.

\iusr{Марина Прохорова}

Жаль, не на мове написано. Патриоты не оценят. И кстати, витрину сделали.
Польшу. И не для России, а как раз для Украины. Чтобы беспрепятственно грабить.
Инжой!

\begin{itemize} % {
\iusr{Вячеслав Савченко}
патриоты умеют читать на двух минимум.
Зато сразу заметно, как текст заценили "патриоты" соседей, вот по первому же каменту.
Это еслишо.

\iusr{Natasha Varetskaya}
какая-то что-ли заказуха... или нахлынула любовь к неньке на почве сравнения... или только мне одной это кажется?

\iusr{Марина Прохорова}
\textbf{Наталья Варецкая} Иногда надоедает сидеть в дерьме. Выбраться не удалось - просто поменяли сорт. Подсознательно хочется, чтобы рядом было такое же дерьмо. Не так обидно выходит (

\iusr{Natasha Varetskaya}
как-то не совсем ожидала этого от автора.
даже стыдновато.
этот пассаж вызывает недоумение

\iusr{Игорь Лесев}
\textbf{Наталья Варецкая} не стоит тратить ваш стыд на меня, а то мне неловко от такой жертвы

\iusr{Natasha Varetskaya}
\textbf{Игорь Лесев} если бы все начиналось и заканчивалось кокетством - был бы прекрасный, но дурноватый мир, а так - что-то пошло не так.
вот честно - я вовсе не ожидала, что Вы пойдете в эти сравнения, как Булонский лес.

\iusr{Natasha Varetskaya}
\textbf{Игорь Лесев} простите за ожидания. больше не буду.
\end{itemize} % }

\iusr{Евгений Отовчиц}

Да всё там точно такое же. Те же либералы-соросята в правительстве. Орешкины
всякие. Благодаря "бюджетному правилу" Россия платит дань в пользу западной
финансовой модели - вкладывает доходы в ценные бумаги США, обеспечивая своим
ВВП доллар. Вместо того, чтобы вкладывать в свою экономику. И, конечно,
газопроводы достроят. Чем больше Россия продаст своих недр - тем больше
субсидирует западный финансовый центр.

У нас Гонтарева - тварь и грабительница, которая обвалом гривны сделала всех
нищими - у них Набиуллина идентичным обвалом рубля "спасла" отечественного
производителя и экспортёра. Мы открываем рынок земли по требованию МВФ - Россия
проводит пенсионную реформу по рекомендациям МВФ. Но мы должники МВФ, а Россия
нет. Логика?..

В России вполне компрадорская модель управления. Они ОЧЕНЬ много платят дани на
запад. Но при этом пытаются быть минимально независимыми политически.

Наша отечественная элита, взращенная в грантовских инкубаторах, вообще не видит
такого государства, как Украина. Языки, законы - это всё всего лишь средство
как можно дольше находиться при должности, напаковаться - и домой. В Лондон,
Тоскану или Париж.

Две колонии. А политика, война и конфликты - ширма для прикрытия донорства.

\iusr{Александр Власенко}

Мы о России больше знаем, чаще следим за российскими медиа, где каждый фак-ап
отображается. Поэтому, мы не можем объективно оценить современную Россию.
Абберация близости.

\begin{itemize} % {
\iusr{Дарина Зарицкая}
Именно. В Китае, Штатах и Индии может и чаще всякое происходит. Но мы же не рассматриваем их под микроскопом, как Россию)

\iusr{Игорь Лесев}

а разве Россия к нам географически ближе, чем Белоруссия или Венгрия? Но в
Белоруссии я вот даже без гугла премьера назвать не могу... Вот это и есть
иррациональная зацикленность на ру-повестке от крайне негативной до позитивной

\end{itemize} % }

\iusr{Василий Январев}

Не согласен - РФ непривлекательна для соседей в первую очередь из за отсутствия
идеи, подходящей для всех. Что может в идеологическом плане предложить РФ?
Суверенную демократию с ее установками из 19-го века - самодержавие,
православие, народность. Кстати, как ни странно, в самой РФ это как то
работает, но вот за ее пределами никого не возбуджает. А вот идея большевиков
построения царства Божьего на земле была весьма привлекательной для многих.
Например, прототип Павла Корчагина он из Западной Украины, да. Так что с РФ
дружат либо те, кому некуда деваться от враждебного окружения - Армения,
например, или семейка Асадов. Либо за деньги - как хитрый Бацька. Но оба
варианта так себе - вынужденные союзы не бывают крепкими, а дружба за деньги
заканчивается когда человеку предложат (а иногда просто посулят) сумму
побольше.

\begin{itemize} % {
\iusr{Олеся Бабина}
Дело в том что реально НИКТО НИЧЕГО идеологически предложить не может. Ни Россия ни Штаты. Живите и жрите, причем каждый за себя это нынешняя идеология.

\iusr{Тимур Филоненко}
\textbf{Олеся Бабина} Почему же. Штаты через Голливуд успешно продают Американскую мечту.

\iusr{Олеся Бабина}
\textbf{Тимур Филоненко} ну так она в этом и заключается.

\iusr{Тимур Филоненко}
\textbf{Олеся Бабина} ну она вполне конкретная. Приезжай к нам, ебаш с утра до ночи и будешь счастлив. Вполне идея.
\end{itemize} % }

\iusr{Наталья Закерничная}
Спасибо за правду

\iusr{Полина Дьяченко}

Их элита зажралась пуще нашей. Деньги там бОльшие крутятся. Только было сегодня
в ленте у скольких росдепутатов иностранное гражданство. Все тоже что и у нас


\iusr{Тимур Филоненко}
У одного попа даже недвигу нашли в Испании)

\iusr{Вячеслав Савченко}
весьма неплохой текст и мысли. Ощущение, что Вы читали мой камент к предыдущему тексту )))

\iusr{Правдивый Виталий}
Костя Белоусов, прокомментируйте пожалуйста, расставьте правильно акценты.)

\begin{itemize} % {
\iusr{Костя Белоусов}
\textbf{Правдивый Виталий} 

прочитал, про всё правильно написано, нам все это мешает стать кем то
большими...

Что возражать против аварий и технологических проблем???

А против силы работает только сила, это я про историю с запретом работы судов
трубоукладчиков, все слушаются независимо от права и законности приказа...

Как то достроить достроят трубопровод - это проект дешёвого газа для Германии,
её конкурентоспособности, так как прямая подача дешевле транзита по суше.

На Дальнем Востоке построен завод крупнокаркасного судостроения типа
южнокорейских верфей, лет через "дцать" построят и российский трубоукладчик.

Про "тянущихся" к России изгоях - не думаю, что это верно, Россия сама их
использует для каких-то целей в геополитике, это вернее...

А любовь между странами это вообще "моветон", ее не бывает - в реальности есть
страх, уважение, презрение... говоря о нелюбви к России Лесев слишком сильно
эмоционален)

Я работаю с технологиями - автоматизация, мне ли не знать все проблемы???

Но я знаю и другое - развитие в России зависит от необходимости, пока не
грянет, не перекрестимся да и Сибирь поможет как всегда ресурсами... я спокоен -
дороги строят и дураков тоже!


\iusr{Правдивый Виталий}
\textbf{Костя Белоусов} , спасибо за обстоятельный ответ!

\iusr{Костя Белоусов}
\textbf{Правдивый Виталий} мне непонятна Ваша точка зрения? Она схожа с точкой зрения Лесева? Или Вы более спокойно относитесь к победам Украины над Россией в войне вообще и за за транзит в том числе ?
С учётом того, что деньги за транзит всё равно как-то заложены в цену продажи, большим или меньшим процентом...

\iusr{Правдивый Виталий}
\textbf{Костя Белоусов} , мне нравится как пишет автор, и про Украину тоже.
Очень остро!)

\iusr{Костя Белоусов}
\textbf{Правдивый Виталий}, а зачем читать, то что и так известно, он что даёт советы?

\iusr{Костя Белоусов}
\textbf{Правдивый Виталий} или в книгах про это не пишут и Вы живёте в изоляции?)

\iusr{Правдивый Виталий}
\textbf{Костя Белоусов} , мы не живем в изоляции не отрицаем очевидное, не говорим, что мы великие, в книгах пишут про вечное, а «зачем читать», хороший вопрос.
Каждый сам на него ответит.

\iusr{Костя Белоусов}
\textbf{Правдивый Виталий}, мне нравится читать мнения людей, но...только до цикла дискуссии, т. е. первые их мысли, когда понимаешь как человек живёт и дышит, а не как умеет изворачиваться, например! @igg{fbicon.hand.victory} ️ ))

\iusr{Natasha Varetskaya}
\textbf{Костя Белоусов} супер сказано. я готова спорить с Лесевым но половине высказываний. но не хочу. он обидчивый. как мне кажется. нельзя поднимать кундалини Украине Россией.
Это как мин. смешно.
Ну, и автор должен изучить Русский Дух и Русский Бунт - местами бессмысленный и беспощадный.... но, только местами.

\iusr{Костя Белоусов}
\textbf{Наталья Варецкая} кундалини России это наверное Сибирь, а вот что является кундалини Украины...возможно антиРоссия? это их питает, даёт смыслы?

\iusr{Natasha Varetskaya}

так как я здесь, в Украине, но по духу и крови- русский человек и советский
ребенок - мне есть чем анализировать происходящее. так во - ДА, это - анти
Россия, это комплексы младшего брата/сестры... комплексы НЕ центра принятия
решений, страшная хитрость и воровитость элит и не только элит..... анекдот про
"бурячок посажу" - именно вмасть...

зависть. страшная зависть - как нац. черта... можно много чего еще описать....
уже одного отношения к Донбассу довольно.

и именно нечто, в чем есть и элементы зависти и злорадства я встретила в тексте
автора. думала он выше такого ракурса.


\iusr{Правдивый Виталий}
\textbf{Наталья Варецкая} , анализ - так себе, возможно круг общения такой.Такие черты есть, но не у подавляющего большинства.
Воровать и садить буряк- разные вещи.
Не центр принятия решений, это скорее западные области из-за своей истории.
Жители центра и юго- востока обладают вполне имперским ) мышлением.
Отношение к Донбассу сформировано пропагандой.
У Вас прослеживаются те же эмоциональные моменты, которые Вы порицаете.

\iusr{Костя Белоусов}
\textbf{Правдивый Виталий} думаю, что теперь у вас Западная Украина везде - более или менее... Именно поэтому нет имперского, мультинационального, толлераннтного контента в политике и общественном мнении. Первое видно по решениям власти, второе - по тишине и неактивности противников неконституциональных решений по АТО, языковой политике и в отношении инакомыслящих...

\iusr{Правдивый Виталий}
Костя ,Белоусов , Вы питаетесь пропагандистским продуктом.)
И путаете народ и власть.

\iusr{Костя Белоусов}
\textbf{Правдивый Виталий} я ничем не питаюсь, а вижу результат по факту 5 лет.
Путать народ с властью не надо верно, но за все годы после 2014, народ не проявил себя так же активно, как тогда, когда произошёл майдан... более нехорошие вещи произошли по вине властей за 5 лет, чем то что вменили в вину Януковичу в 2013-14, значит народ достоин власти!

\iusr{Правдивый Виталий}
\textbf{Костя Белоусов} , народ достоин власти- это так!
Но мне тяжело обсуждать с Вами процессы происходящие в нашем обществе.
По ряду причин.
Вам же смешно, когда я пишу о России пропагандистским клише.
Вы здесь не живете.
Вы живете в стране, где подавлены все гражданские свободы.
Ваша критика нам ничем не помогает.
Вы нас проигрываете, потому, что не гибкие.
Из Ваших постов и комментов не понятно, Вы друг Украины, или люто ее ненавидите( при этом конечно-же презирая всех, кто не любит Россию.)
Вообщем,ничего нового, к сожалению.

\iusr{Костя Белоусов}
\textbf{Правдивый Виталий}, зачем Вам это, где я живу, подавлена ли у нас свобода и прочее?
Я говорю о факте нарушений прав человека на обучение на языке родителей, на неконституонные действия на майдане и после...
Я кажется писал, что подобное майдану происходило в 90-е в России? Это не смешно, а Вам?
Советы не работают для всех, не только для Украины, общества сами доходят до всего ...об этом же написал некто Суслов в посте о прошлом и будущем Украины и МВФ. Это мне не смешно, а Вам?
И не надо обсуждать со мной, напишите это для себя, пусть все прочтут...

\iusr{Правдивый Виталий}
\textbf{Костя Белоусов} , ну, если : Суслов пишет, это аргумент, то и : Лесев пишет- тоже аргумент.

\iusr{Костя Белоусов}
... я вчера спросил Ваше мнение?
За всё время Вы так и не ответили.
Моё мнение по России я высказал: про технологическую немощь - правда, эмоции излишние, я спокоен - мы уже крестимся и Сибирь поможет ресурсами

\iusr{Правдивый Виталий}
\textbf{Костя Белоусов} , мы в тяжелом, вы в нелегком положении.
Вы и все кто с вами проигрывают.
Потому что коррупция!
Она убивает все!
Нет науки, нет технологий, нет инноваций.
Это длинные деньги, у вас и у нас так не играют, все в офшор, детей на Запад.
А когда нет хлеба, людям дают зрелища.

\iusr{Natasha Varetskaya}
\textbf{Правдивый Виталий} НЕТ. я, как носитель имперских энергий прекрасно понимаю о чем ВЫ. Но Вы не о ТОМ. целью вашего спича было поставить мне подножку... но "то такое".. как говорят в Одессе.. проще не обратить внимание.
думаю, вот лучший ответ на реплику "от\%ебитесь от рашки" - :" Fox Renard да не отъе\#утся. Они же ведь и есть эта самая рашка, ее самая лучшая часть!
это метаморфоза. но только на поверхности.

\iusr{Natasha Varetskaya}
\textbf{Правдивый Виталий} 

особенно понравилось ваше клише: "Вы живете в стране, где подавлены все
гражданские свободы...." товарисчи! идите смотреть телевизор дальше......  в
стране - даже не в трикстере, кем пыталась быть Украина говорить о гражд.
свободе - это песец.... Ну здесь все на месте: - есть граждане, есть свобода,
есть закон. НО ВСЕ ОТДЕЛЬНО и никак не соединяется. НИКАК.  это ,
естественно,новая полифония.

\iusr{Natasha Varetskaya}
\textbf{Правдивый Виталий}

чтобы подвести итог скажу так: русские по-настоящему любят Россию ЛЮБОЙ и
ВСЕГДА. это то, что невозможно объяснить... это тот русский дух, который
человек унесет с собой ТУДА. вот честно честно - Украина - не моя Родина,я
родилась в России - в Мурманской области - еще при Союзе и в 7 лет меня
привезли в Украину. Понятно, времени у меня было предостаточно.... но я не вижу
здесь настоящей любви украинцев к своей стране... слышу и вижу пи.деж... по
темам вокруг да около... вижу сокращение населения за счет эмигрантов(!) и
"заробитчан".... но любят свой дом , свою улицу лишь старики... носители все
той-же имперской бациллы...  страной своей хочется гордиться - и это норма.

\iusr{Костя Белоусов}
\textbf{Правдивый Виталий} детей на Запад, деньги в оффшор?

Ни один чиновник и большинство бизнесменов уже несколько лет не держит деньги и
несовершеннолетних детей на Западе - первым запрещено по закону, вторые страдая
из-за политики давления Запада сами перестали хранить там грязные деньги.

Коррупция убивает нас, но она всегда была в России, сейчас её не больше чем в
Советском Союзе или в царской России.

Частный бизнес не ворует, госкорпорации воруют и их представителей сажают,
но... параллельно что-то строится - не мытьем так кататьем тупые и умные,
вороватые и не очень люди, что то делают и на пользу обществу - уровень жизни
выровнялся несмотря на низкий рост экономики, ходят новые поезда по новым
дорогам, с грехом пополам летаем в космос и на superjet... но ресурсы есть, а с
деньгами можно многое...

\iusr{Правдивый Виталий}
\textbf{Наталья Варецкая} , абсолютно кристальная, незамутненная ненависть ко всему украинскому.
Это имеет определение- шовинизм.
Не вижу вариантов для дискуссии.
Сочувствую Вашему пребыванию в настолько чуждой среде, надеюсь есть достаточные мотивы.
Подножку Вам не ставил, Вы сами с этим отлично справляетесь.

\iusr{Правдивый Виталий}
\textbf{Костя Белоусов} , такое впечатление, как прочёл передовицу газеты «Правда».

\iusr{Костя Белоусов}
\textbf{Правдивый Виталий}))
Вы когда то читали газету Правда?
Обычно в жизни апологеты чего либо становятся самыми большими антагонистами этого...
Я никогда не вспоминаю "советы" , потому, что не состоял, не участвовал, не ценил... да и ладно, мы сами тут, вы и некий МВФ - сами там...
А всего то я хотел спросить про Ваше мнение о роли МВФ и Евроинтеграции на их условиях в будущем страны Украина )))!!! 
@igg{fbicon.hand.victory}

\iusr{Правдивый Виталий}
\textbf{Костя Белоусов} , МВФ институция по зачистке конкурентных рынков.
Ни к какой евроинтеграции они нас не ведут.
Тот же Лесев в предыдущем посте четко описал ситуацию у нас.
Мы папуасы, которыми управляют белые люди , через набсоветы госкомпаний.
Это вкратце.
Газету «Правда» читал и даже изучал( в армии)).

\iusr{Костя Белоусов}
\textbf{Правдивый Виталий}, 

не читал, но и так всё понятно...
Есть свидетельства такого же отношения Германии 1939 - 40 года к Румынии.
Произошло присоединение Молдавии, румынский правитель стал спрашивать совета и помощи у Германии, там ответили уступите, но больше не беспокойтесь, у нас хорошая промышленность, у вас хорошие с. х. угодия, занимайтесь с. х. обеспечивайте Германию, а пром. продукцией и защитой мы вас обеспечим...так всё и решилось тогда.

\end{itemize} % }

\iusr{Сергей Бердинский}

Хм.... Вроде всё правильно, но ощущение, что опять таки Украина надувает щеки и
делает важный вид, потому что в России ещё хуже! Но это не 2 миллиона россиян
работают в Украине и не Украина обеспечивает Европу газом. А СП-2 все равно
когда-нибудь построят, так же как "Силу Сибири". Да в любой стране падают
самолёты (военные и гражданские), тонут военные корабли, танки
переворачиваются, Там где они есть ))) А ещё у россиян есть пассионарность.
Такая конечно уёби....щная, как для среднего европейца, но есть ! У украницев
же её нет, несмотря на Киевскую Русь, королеву Анну в Париже, географический
центр Европы в Рахове, самые красивые деньги в Европе и борщ как мировое
наследие ЮНЕСКО. Поэтому долгоиграющий проект Украина-не Россия провалился в
первую очередь из-за самих украинцев, у которых оказались не совсем те
руки-которые-ничего-не-крали и конечно же "весь-мир-нам-должен"!

\begin{itemize} % {
\iusr{Сергей Киселев}
\textbf{Сергей Бердинский} потому что нет никаких "украинцев"

\iusr{Сергей Бердинский}
\textbf{Сергей Киселев} А кто есть?

\iusr{Сергей Киселев}
\textbf{Сергей Бердинский} русские

\iusr{Тимур Филоненко}
\textbf{Сергей Киселев} Это во влажных мечтах донбасят? Даже если и не было никаких украинцев, вы их создали 5 лет назад. Посмотрите опросы (причем разных соцгруп) - 92\%!

\iusr{Сергей Киселев}
\textbf{Тимур Филоненко} зависит от того, что по телевизору скажут. Скажут что они русские, будут русскими. Крым и Луганск в пример

\iusr{Тимур Филоненко}
\textbf{Сергей Киселев} Но вы же прекрасно понимаете, что говорить не будут? И если исходить из такой логики, получается если Крымчанам и луганчанам говорить что они украинцы - они ими станут? Вы переоценваете телек.

\iusr{Natasha Varetskaya}
\textbf{Тимур Филоненко} вы не переживайте- Зе энд ко примут префектуры и закон о земле и абсолютно законно лнр и днр перейдут либо сами к себе или к россии. нах власти этот донбасс с такими пассионариями... власть пока вытанцовывается, надев памперсы от нациков и конченных атошников.
так что.. все в итоге - русские. но это - метаморфоза. думаю, на таком уровне вы не мыслите.

\iusr{Сергей Киселев}
\textbf{Тимур Филоненко} телек говорит то что ему предпишет тот кто силовым путем контролирует территорию.
Будет контроль Москвы силовой - все будут русскими

\iusr{Тимур Филоненко}
\textbf{Сергей Киселев} Самоидентификация сильнее танков. Вам ли не знать.

\iusr{Сергей Киселев}
\textbf{Тимур Филоненко} только не на Украине

\iusr{Ирина Слюсарева}
\textbf{Тимур Филоненко}, а у вас с самоидентификацией прямо отлично, да. Столько вою про Крым и Донбасс, с чего бы, ведь нация едына.

\iusr{Александр Власенко}

Русские Центральной и Центрально-Черноземной России, белорусы из Восточной
Белоруссии и украинцы Юго-Восточной и Центральной Украины настолько близки друг
другу, что самоидентификация их очень гибкая, и может менятся даже в течение
одного дня. Сегодня в Украине выгодно называться украинцем, оттого процент
титульной нации в опросах взлетел за 90\%. Не значит, что завтра не будет
по-другому.


\iusr{Лампопус Красотусов}
\textbf{Александр Власенко} украинца нельзя купить. Его можно только арендовать.

\end{itemize} % }

\iusr{Maksim Kammerer}

для анализа России пока Вы жидковаты.... поверхностно.... @igg{fbicon.smile} 

\begin{itemize} % {
\iusr{Natasha Varetskaya}
очень поверхностно, даже обидно за автора.

\iusr{Игорь Лесев}
как вы аргументированы, сразу виден твердый стул

\iusr{Maksim Kammerer}
\textbf{Игорь Лесев} cтул-то диденький, иожно сказать - колченогий:)но и автору,. как кажется, не стоит обижаться:)
\end{itemize} % }

\iusr{Валерий Архипов}
Увы, все так и есть  @igg{fbicon.frown} 

\iusr{Павло Бубенко}

То, что россия прекратит своё мерзкое существование, рано или попозже -
сомнений не вызывает. Все империи ожидал и ожидает распад.Проблема может быть
лишь в том, что при своей кончине она может потащить за собой многих, в первую
очередь, своих географических соседей!

\begin{itemize} % {
\iusr{Полина Дьяченко}
\textbf{Павло Бубенко} а можете так же эмоционально сказать про мерзкую Британскую империю?

\iusr{Павло Бубенко}
\textbf{Полина Дьяченко} можно - её давно уже - нет!

\iusr{Ирина Слюсарева}
Павло Бубенко, да неужели. Если бы это было так очевидно, вряд ли вы так верещали.

\iusr{Александра Валуйская}
\textbf{Павло Бубенко} а что там по поводу "великой американской империи"?))))
\end{itemize} % }

% -------------------------------------
\ii{fbauth.koreckij_aleksej.odessa.ukraina}
% -------------------------------------

Насчёт трубопроводов, всяких там Северных и прочих потоках.

Северный поток 2 они обязательно достроят, это просто такая игра США в политику
'Кнута и пряника" и "Хорошего, плохого, злого".

Другое дело, что стоимость обоих потоков десятки миллиардов и вовсе не рублей.
И все проекты газопроводов РФ последних лет, включая "Силу Сибири" больше
политика, чем экономика.

Эти затраты они никогда не окупят, во всяком случае в обозримом будущем, а
вероятно вообще. Учитывая норвежскую трубу в Польшу и амбициозный проект США
обеспечить 1/3 европейского потребления своим сжиженным газом в ближайшие 10
лет. А ведь есть еще и Катар, и альтернативные источники понемногу
эволюционирующие с каждым годом.

Но это традиции доставшиеся РФ еще от СССР, где политика всегда погоняла
экономикой и дебит не сходился с кредитом.

При этом, мы Украина, умудряемся смотреться смешно и убого даже на фоне РФ. Уж
этого таланта у нас не отнять.

Только мы умеем так искренне кричать - "Ми Велика Нація", одевая вышиванку на
очередной идентичный праздник. Так упорно верить в собственные догматы, не веря
собственным глазам. Так гордиться своим патриотизмом и идентичностью при этом
разворовывая все до чего только можно дотянуться при малейшей возможности.

Только мы всерьез можем сказать - "Ми нація воїнів", не замечая десятилетие
отрицательной демографии и стремительно стареющее население, на фоне
олигархического феодализма и почти полного отсутствия граждан в классическом
понимании.

Украина другая крайность, но не меньший европейский парадокс, существующий вне
логики и здравого смысла.

\begin{itemize} % {
\iusr{Марина Прохорова}
Да ладно, всё это сформулировано задолго до. "Что ты смотришь на сучок в глазе брата твоего, а бревна в твоём глазе не чувствуешь" (Матф.7:3).

\iusr{Всеволод Саксонов}
\textbf{Алексей Корецкий}
Ну вот откуда вы такие берётесь....?
Читаете статью в каком нибудь Обозревателе и начинаете вещать с умным видом?
"Сила Сибири" поставляет в Китай 38 млрд кубов, а проектная мощность 61 млрд! Хотя бы поинтересуйтесь, куда остальное пойдет. И что такое
ЗапСибНефтехим,
что такое Амурский ГПЗ, и какие планы по газофикации Дальнего Востока.

\iusr{Алексей Корецкий}
\textbf{Марина Прохорова}

У автора не просто про брата и сучок.

Речь идет именно о зацикленности на РФ и ирациональной вере, присущей многим
украинцем. При этом процесс перманентного "развала" у Российской Империи, СССР
и РФ продолжается уже второе столетие.

А самое главное, ни наличие веры, ни ее отсутствие, ни бардак у соседей никак
не способствует построению собственного успешного государства и обретению
государственности де факто.

\iusr{Алексей Корецкий}
\textbf{Всеволод Саксонов}

Если вы действительно узкопрофильный специалист в этой области, напишите о
себестоимости проекта, стоимости обслуживания и прокачки, амортизации и
сегодняшних ценах по которым покупает газ КНДР.

И на основе этих данных посчитайте предполагаемый период хотя-бы
самоокупаемости.

Мего-пректы реализуют либо с целью самоокупаемости и прибыли в будущем, либо
ради стимуляции собственной промышленности и внутреннего спроса рассчитывая
получить в будущем прибыль с сопутствующих отраслей, либо просто потому что
могут себе это позволить ради политических и социальных дивидендов.

Но бывает также самый запущенный вариант, когда амбиции, политические и
социальные цели есть, есть даже возможности, но они не соответствуют получаемым
реальным результатам и самой экономической обоснованности в целом.

\iusr{Ирина Слюсарева}
\textbf{Алексей Корецкий}, 

а вы специалист по схождению дебета с кредитом? Тогда в курсе ли вы, что
золотовалютные резервы РФ сейчас превысили 400 (четыреста) миллиардов
(миллиардов, да) долларов - это, конечно, оттого, что в России считать не
умеют. И все время политику путают с экономикой, да.


\iusr{Алексей Корецкий}
\textbf{Ирина Слюсарева}
А кто де спорит?
Вы внимательно читайте весь комментарий, а не только одну строку.
К тому же было уже почти 600 миллиардов пр ценам 10-ти летней давности, пока не пробежал севернай мохнататый лисиц.
И я писал, о том, что РФ делает хоть что-то, пока Украина только надувает щеки ждет "падения Российской стены".


\end{itemize} % }

\iusr{Oksana Kudrya}
Автор в приступе чёрной меланхолии. ))))

\iusr{Марина Прохорова}

Не хотелось бы никого обидеть, но список российских провалов, аварий и
катастроф сразу показался мне подозрительно знакомым. Я его УЖЕ читала,
буквально на днях. Вы не поверите, у зомбиБабетты, в телеграме. Пожалуйста,
прямая цитата

"Babchenko, [25.12.19 15:22]
Но что мне в России нравится, так это морской флот.
Купили в США крейсер "Варяг" - утопили нахуй.
Купили в Дании "Челюскин" - утопили нахуй.
Купили в Германии "Адмирал "Нахимов" - утопили нахуй.
Купили в Чехословакии "Булгарию" - утопили нахуй.
Купили в Финляндии плавучий док ПД-50, самый длинный в России и один из крупнейших в мире - и док утопили, и "Адмирала "Кузю", которого спиздили у Украины, наебнули! Вот это был реально страйк, признаю. Это да. Редко кто так может. П-профессионализм.
Купили в Финляндии буровую платформу "Кольская" - утопили нахуй.
Купили в Китае трубоукладчик "Академик "Черский" - специально, чтобы прокладывать "Северный поток-2" - если кто не знал, у "Газпрома" есть свой трубоукладчик, но "Газпром" ссыт его использовать, боясь попасть под санкции уже всем "Газпромом" - наебнули весь "Северный поток!"
Этот страйк даже круче, чем ПД-50.
Мне кажется, это у них такая тактика. По выматыванию противника. Покупать у него корабли - и топить, покупать - и топить. И когда корабли у проклятых буржуинов и закончатся - тут и сказочке конец.
Так и победим".
И знаете, этот оппозиционный флёр (с) весьма неважно пахнет ((

\iusr{Олег Хавич}
Я тебе больше скажу: Абхазия и ПМР ложат на Москву ***, просто молча, в отличие от Бацьки.

\iusr{Polina Zorina}

При желании тот же журналист может надергать других фактов издругих историй и
представить РФ передовой технической державой. Блядство людей пишущих. Об этом
мы должны всегда помнить, к написанному относиться критически и не попадать под
обаяние красиво сплетенного текста

\begin{itemize} % {
\iusr{Марина Прохорова}

В РФ есть примеры и провалов, и успехов. У Украины нет ничего. Здесь не падают
самолёты на испытаниях, потому что ничего не строят и не испытывают. Здесь
ракетоносец тихо гниёт у причальной стенки Николаевской верфи и проблема только
в том, что он не нужен даже китайцам. Здесь не могут построить мост через Днепр
в столице, но уже 20 раз доказали, что Крымский мост построить невозможно.
Здесь строят пирамиду ГКО, но почему-то верят, что в этот раз пронесёт. И так
далее. Просто каждый выбирает на свой вкус )

\end{itemize} % }

\iusr{Всеволод Саксонов}

7 мая 2016г Штурмовик Корпуса морской пехоты США AV-8B Harrier рухнул в
Атлантический океан вблизи побережья штата Северная Каролина.

19 мая 2016г. Рухнувший на Гуаме американский бомбардировщик В-52 перед
падением на землю загорелся.

2 июня 2016г. Разбился второй за день американский истребитель.

26 мая 2016г. Истребители F-18 ВМС США столкнулись над Атлантикой у побережья
штата Северная Каролина.

21 апреля 2017 г.

Упал в море истребитель атомного авианосца ВМС США, идущего к Кореям.

12 ноября 2018 г.

Реактивный бомбардировщик ВМС США F/A-18 потерпел крушение возле острова
Окинава в Тихом океане.

31 июля 2019 г.

Многоцелевой истребитель F/A-18E Super Hornet разбился в среду около базы ВМС
США в Калифорнии. Об этом сообщается в Twitter ВМС США.

Про Боинг 737-Мах и ежегодные пожары в Калифорнии писать уже лень.

\begin{itemize} % {
\iusr{Дарина Зарицкая}
Нее, это не интересно) Нам интересно а шо там в России))
\end{itemize} % }

\iusr{Филипп Ильин-Адаев}

Игорь, не хочу расстраивать, но изнутри видится все иначе. Провели Олимпиаду
(блестяще), провели Чемпионат мира по футболу (все в восторге), спасли Сирию от
бармалеев, вернули Крым, возродили армию (реально ею гордимся), построили мост
(никто у вас не верил), проложили скоростную автодорогу между Москвой и
Питером, из Москвы сделали один из лучших европейских городов, метро в столице
роют так, словно золото ищут (каждый год 15-20 новых станций; уже в Подмосковье
вышли), научились делать свои сыры и вино и так далее. Дай бог, чтоб и дальше
без майданов, революций и войн.

И да, Северный поток-2 построят, без вариантов.

\begin{itemize} % {
\iusr{Евгений Водовозов}
\textbf{Филипп Ильин-Адаев} изнутри всегда по другому. Изнутри смотришь - свет, а снаружи - просто жопа)

\iusr{Игорь Лесев}
меня ничего из приведенного списка не расстраивает

\iusr{Елена Самойленко}
Да, пенсий на пять лет лишили. Глядишь и вовсе отменят. Но олимпиаду провели.

\iusr{Филипп Ильин-Адаев}
\textbf{Елена Самойленко}, ну а что делать, если население стареет? Молодежи мало, получателей пенсий много. Во всем мире с этой проблемой сейчас сталкиваются.
\end{itemize} % }

% -------------------------------------
\ii{fbauth.vodovozov_evgenij.kiev.ukraina.doneck.muztorg}
% -------------------------------------

Странный ты человек, Игорь. То прикармливаешь, прикармливаешь адептов
потусторонних.. а потом вдруг - хрясь по физиономии таким текстом!!!) И
читаешь, и думаешь - толи это тактика такая, толи просто стебёшься)

Но текст зачётный)

\begin{itemize} % {
\iusr{Natasha Varetskaya}
неа... это гайсин из автора полез, а не тактика.

\iusr{Natasha Varetskaya}
То Ли - это отдельно.
музторг это оценит. будьте бдительны. вата наблюдает.

\iusr{Евгений Водовозов}
\textbf{Наталья Варецкая} вата следит за тобой)
\end{itemize} % }

\iusr{Сергей Киселев}

Россия сейчас переживает свой Золотой век. А никак не падение

\begin{itemize} % {
\iusr{Олег Хавич}
Это "золотой дождь"  @igg{fbicon.smile} 

\iusr{Сергей Киселев}
\textbf{Олег Хавич} можно и так сказать

\iusr{Матвей Кублицкий}
\textbf{Сергей Киселев} учитывая золотые резервы РФ именно
\end{itemize} % }

\iusr{Валерий Песецкий}
Мощно и точно)

\iusr{Ros Misiura}

Вообще говоря, не падают самолёты только в тех странах, где они не летают.
Особенно новые образцы, ещё сырые. И происшествия на флоте и во всех сферах -
это нормальное следствие активности (если процент происшествий не зашкаливает).
У американцев был такой истребитель чудесный F 104 Starfighter, неофициально
называемый widowmaker в ВВС. Его статистика небоевых потерь была чудовищна. Это
навскидку для примера.

\iusr{Yan Pronin}

Автора взломали? Вот, например, фондовый рынок мордора за последний год - номер
один по росту среди emerging markets.

\ifcmt
  ig https://scontent-frx5-1.xx.fbcdn.net/v/t1.6435-9/80424476_10114945068390080_2953374071422713856_n.jpg?_nc_cat=110&ccb=1-5&_nc_sid=dbeb18&_nc_ohc=2XI9Xr2Iq4gAX_-eAdt&_nc_ht=scontent-frx5-1.xx&oh=9ef74286ba5558954369f9986f4864dc&oe=61A84F7D
  @width 0.4
\fi

\iusr{Ros Misiura}

Для тех, кто лайкнул мой комментарий, скажу, что с основным посылом автора
вполне согласен.

\iusr{Vitaliy Lukyanenko}

В принципе, такой же ряд можно построить для США. Суть не в национальных
успехах и неудачах, наверно. Суть, видимо, в том, что геополитическая
конструкция Ялта - Хельсинки демонтирована и идёт дикий поиск своих мест в
неопределенном мире.


\iusr{Василий Стоякин}

На самом деле Россия всегда такой была. Просто в истории как-то не обращают
внимания на проебы, а восхищаются тем, когда русские, наконец, начинают их
исправлять. Это нормальное состояние. Что характерно - величие от этого не
страдает, хотя сосед да, сложный.

\begin{itemize} % {
\iusr{Матвей Кублицкий}

проебы упомянутые Игорем - обычное сопровождение техногенного развития. ну
кроме спорта. очень смешно как украинцы судят о России с упорством не желая
видеть что в ней происходит.

\iusr{Василий Стоякин}
\textbf{Матвей Кублицкий} Тут есть еще такой момент как неполная информированность. Проебы Запада заметны меньше - далеко, туман...
\end{itemize} % }

\iusr{Владимир Сорокин}

Россия , также как и Киргизия с Украиной, обломки некогда вполне перспективного
проекта под названием СССР. Люди похерившие свой собственный проект для того
чтобы стать мальчиками на побегушках в проекте чужом не могут быть успешными по
определению. При этом Россия, при всей её нынешней слабости и
несамодостаточности, всё же пытается сменить тренд и вернуться от нынешнего
лакейского проекта к проекту собственного , суверенного, развития. Это сложно,
но иного пути Россия не видит. Ну а Украина если хочет пусть продолжает
подмахивать добрым белым господам.

\iusr{Олег Хавич}
\textbf{Игорь Лесев}, 

есть ещё один важный момент (это лишь дополнение к посту, с которым согласен на
100\%): российские СМИ специально акцентируют внимание на
аварийно-катастрофической составляющей того кромешного пи*деца, в который
погружается РФ - и сразу же появляются комментаторы типа "а в США тоже негров
вешают... ну, самолёты падают".

Однако куда вжнее тот факт, что 2/3 граждан РФ живут в бедности, из которой не
видно выхода, при этом от четверти до трети - в крайней нищете. Зарплату в
10-12 т.р. на руки и пенсию в 7-10 т.р. получают десятки миллионов россиян. Да,
коммуналка там чуток дешевле, но зато продукты дороже и зима длиннее, чем вна,
и традиции огородничества и гастарбайтерства нет - так что ситуация во многих
регионах РФ хуже, чем даже в ЛДНР.

А если учесть, что на значительной части территории РФ (Кавказ, нацреспублики
Поволжья и Сибири) русские являются дискриминируемым этносом (хотя частно
составляют большинство), то сгоревший "Кузя" - далеко не главная проблема
дорогих (на самом деле нет) россиян...

\begin{itemize} % {
\iusr{Евгений Отовчиц}
\textbf{Олег Хавич} на Кавказе, Поволжье и Сибири русским вообще делать нечего.
Примерно, как галичанам в Киеве.

\iusr{Олег Хавич}
\textbf{Евгений Отовчиц}, вот только галичане в Киеве задают дискурс, а русские в перечисленных регионах - либо заложники, как в Чечне или Дагестане, или граждане третьего сорта, как в Татарстане или Тыве.

\iusr{Евгений Отовчиц}
\textbf{Олег Хавич} да, только в этом разница.
Но там-то как раз закономерно. Как инородный элемент. Это у нас почему-то Галиция хозяйничает. Ах, да, их же демократическая империя крышует))

\iusr{Матвей Кублицкий}
какое удивительное незнание того что происходит в России. все знания - строго по прозападным сми где рассказывается как мы плохо живем

\iusr{Олег Хавич}
\textbf{Matvey Kublitsky}, принципиально использую для изучения российского пи*деца только официоз или провластные СМИ.

\href{https://rg.ru/2019/04/17/reg-skfo/urozhency-dagestana-ubili-muzhchinu-za-otkaz-tancevat-lezginku.html}{%
Уроженцы Дагестана убили мужчину за отказ танцевать лезгинку, rg.ru, 17.04.2019%
}

\iusr{Олег Хавич}
\textbf{Matvey Kublitsky}, в других странах это события, в РФ - явление.

\iusr{Матвей Кублицкий}
\textbf{Олег Хавич} может сравним науку? или финансы? или количество нищих?

\iusr{Олег Хавич}
\textbf{Matvey Kublitsky}, по КОЛИЧЕСТВУ нищих РФ в несколько раз опережает б/УССР. Поэтому давайте лучше сравним проценты  @igg{fbicon.wink} 

\iusr{Олег Хавич}
\textbf{Matvey Kublitsky}, науки в РФ не существует, как и наукоёмкой промышленности. Финансы в РФ есть, пока Центробанк РФ финансирует США.

\iusr{Матвей Кублицкий}
\textbf{Олег Хавич} 

ну давайте цифры глянем. Итак по уровню бедности. в россии 20 миллионов на 146
млн населения. На украине 14 миллионов на 42 млн. Пролет полный. Наука - там
просто букет того что есть только в России. я уж не сравниваю с украиной.
прямой пример - ВУзы в списке ведущих вузов мира - там сразу всё видно.
наукоемкая промышленность - вы вообще доступа не имеете к инету? всего один
пункт назову - в космос люди амеры летают строго на российских ракетах.

\iusr{Матвей Кублицкий}
\textbf{Олег Хавич} и расскажите откуда эта глупость про финансирование ЦБ РФ американцами?

\iusr{Евгений Отовчиц}
\textbf{Матвей Кублицкий} 

...а смартфонами россияне пользуются исключительно китайскими и американскими.

Ту стоит 30 миллионов долларов, а аналогичный Боинг 100 миллионов. Но почему-то Боинги покупают в разы больше.

Вам нравится сравнивать Россию с Украиной, успокаивает? Мне, например, не особо
утешительно то, что я живу лучше, чем спившийся алкаш из соседнего дома.

Впрочем, не признавать очевидных вещей и считать себя полноценным игроком на
мировой арене, а не сырьевым придатком - ваше полное право. Считайте.

Раньше таких лечили, а сегодня свобода, демократия и либерализм!))

\iusr{Олег Хавич}
\textbf{Matvey Kublitsky}, не американцы финансируют ЦБ РФ, а ЦБ РФ финансирует ФРС США - это такая форма дани. Пока финансирует - РФ позволяют иметь подобие финансовой системы. Но, судя по последним санкциям, это ненадолго.

\iusr{Матвей Кублицкий}
\textbf{Евгений Отовчиц} я вас оч удивлю. смартфонами китайскими пользуются все страны. если берем смартфон за вершину прогресса - то в мире только три развитых страны - сша китай и юж корея. даже японцы оч отсталые

\iusr{Матвей Кублицкий}
\textbf{Олег Хавич} 

вы бы проверенными источниками пользовались. чтоб с вами было интересно
дискутировать. а то как с ребенком. Инфо для вас: золотовалютные резервы цб рф
- 540 млрд баксов. из них непосредственно в ценные бумаги сша вложено 10
млрд.... вот такое содержание сша))))))

\iusr{Матвей Кублицкий}
\textbf{Евгений Отовчиц} еще по науке - есть вещи, которые разработаны только в России и нет ни в каких странах мира вообще. и таких вещей оч много: атомные ледоколы, переработка ядерного топлива, запуски космических ракет с людьми, гиперзвуковые ракеты. это я влет написал. не задумываясь. а если начать искать направленно - полно

\iusr{Олег Хавич}
\textbf{Matvey Kublitsky}, ага, ракеты с ядерным двигателем и другие мультики Сказочного  @igg{fbicon.wink} 

\iusr{Матвей Кублицкий}
\textbf{Олег Хавич} ага. прикинь - по большинству мультиков уже все испытания прошли и они на боевое дежурство встают. и амеры вовсю уже кинулись догонять.

\iusr{Матвей Кублицкий}
\textbf{Олег Хавич} а по остальным пунктам значит даже и сказать нечего?

\iusr{Олег Хавич}
\textbf{Matvey Kublitsky}, ага, особенно последние испытания в Североморске были "успешными", упокой Господи души их участников.

\iusr{Матвей Кублицкий}
\textbf{Олег Хавич} у вас подобная техника точно в ближайшие сто лет не появится

\iusr{Матвей Кублицкий}
\textbf{Олег Хавич} ибо науку уже под откос отправили

\iusr{Олег Хавич}
\textbf{Matvey Kublitsky}, я не русофоб, поэтому РФ с Украиной и Сомали не сравниваю.

\iusr{Матвей Кублицкий}
\textbf{Олег Хавич} правильно. что там сравнивать. РФ развивается. украина и сомали - нет. все высокотехнологичные производства летят под откос.

\iusr{Евгений Отовчиц}
\textbf{Матвей Кублицкий} дело не в удивлении, а в технологиях. Просто следует признать, что вы, мы - весь мир! - колонии. США, Китая.
А не выпрыгивать из штанов.
Можно сколько угодно надувать щёки, но есть факт - контракт на прокачку газа на 5 лет и 3 миллиарда долларов живыми деньгами. И это сдача стране, которая даже контуры государственности имеет очень призрачные.
Я ж не буду вам рассказывать, что это Украина или Нефтегаз дожали Газпром. Но очевидно же, что хозяева приняли решение - ваши и наши власти быстро выполнили.
Поднимаемся с колен))

\iusr{Матвей Кублицкий}
\textbf{Евгений Отовчиц} ну по полочкам - сначало дождитесь 3 млрд а потом уже заявите о победе. во вторых - именно щас и идет борьба чтоб не быть колонией сша - поэтому и санкции. украина эту борьбу полностью проиграла. на сегодня в мире три независимых государства: сша, китай и россия

\iusr{Евгений Отовчиц}
\textbf{Матвей Кублицкий} смотрите - и я, и Олег не являемся свидомыми патриотами. Зато вас постоянно клинит - "а вот у вас!"...
Это признак промытых мозгов. Я стараюсь мыслить категориями геополитики, а не письками с молдаванами меряться.
А то ведь какая-то Калифорния экономически всю великую Россию уделывает. С ракетами и атомными ледоходами.
Только какой смысл сравнивать? Окей, выяснили, что у соседа 78 тараканов, а у вас всего 63. Всё, счастливы?

\iusr{Евгений Отовчиц}
\textbf{Матвей Кублицкий} я вам так скажу - если вы мыслите в координатах государств и границ - то нам сложно будет вести диалог))
Потому что государства - это выдумка 20-го века)))
Это вам кто хочешь скажет - от генетиков до историков)

\iusr{Матвей Кублицкий}
\textbf{Евгений Отовчиц} просто все можно познать строго в сравнении. вы же делаете вброс информации и упорно избегаете сравнения. а они показывают что вы не правы считая РФ амерской колонией. вот тот же договор на транзит - а разве не все кричали что газпрому выгоднее прокачивать газ через украину?

\iusr{Матвей Кублицкий}
\textbf{Евгений Отовчиц} на счет государств - вообще не понятно что вы сказали. вам привести определение государства?

\iusr{Евгений Отовчиц}
\textbf{Матвей Кублицкий} ну давайте сравним. Давайте сравним пенсионные реформы РФ и Украины. Нам навязал МВФ и вам навязал МВФ. Нам деньги от них нужны - Россия от МВФ "независима". Более того, в пенсионной реформе не нуждалась.
В чём разница реформ в обоих государствах?))

\iusr{Евгений Отовчиц}
\textbf{Матвей Кублицкий} нет, расскажите мне, сколько государств было до Первой мировой войны, а сколько после?
А после Второй мировой?

\iusr{Матвей Кублицкий}
\textbf{Евгений Отовчиц} нам мвф в принципе ничего не может навязать - мы от них не зависим вообще. разница - в размерах пенсий и возрасте выхода. в украинцы уже прям щас вовсю выходят в 65 лет. в росси это произойдет в 28 году.

\iusr{Евгений Отовчиц}
\textbf{Матвей Кублицкий} я не понимаю вашего стремления сравнивать. Я где-то сказал, что Украина успешнее или более суверенна, чем Россия?
Наоборот. Одинаковое говно. Только у вас ещё нефть и газ.
Ну давайте я поеду из Киева в Гельзенкирхен или Хельсинки и будем сравнивать.

\iusr{Евгений Отовчиц}
\textbf{Матвей Кублицкий} да, вам МВФ ничего навязать не может, а пенсионная реформа и девальвация валют - синхронно с Украиной.
Так у нас МВФ и "путин напал". А у вас что? Избыток суверенности?))

\iusr{Матвей Кублицкий}
\textbf{Евгений Отовчиц} девальвация валют - это тоже мвф спланировал? а кризис 2008 года кто спланировал?

\iusr{Евгений Отовчиц}
\textbf{Матвей Кублицкий} да, тоже МВФ. По первому каналу не сказали?
Я понял. У вас просто не принято афишировать управление МВФ. В таком случае у нас в стране намного честнее;)

\iusr{Матвей Кублицкий}
\textbf{Евгений Отовчиц} нет у нас никакого мвф. напридумывали и верите в эти суперзаговоры. вам дают команды - вы на долговом крючке висите. как россии то они что то указать могут? хоть что-то

\iusr{Олег Хавич}
\textbf{Евгений Отовчиц}, вот видите, у персонажа всё отлично: МВФ нет, величие есть. Оставьте его  @igg{fbicon.wink} 

\iusr{Евгений Отовчиц}
\textbf{Олег Хавич} это ж надо, как украинское иго, украинские попы и аристократия перепрошила мозги соседям! Их заветная мечта - взять на довольствие всю Украину. Ну, может, за исключением Галичины сейчас.
А так - фантомные боли до сих пор, придите до нас княжити. Киев - мать городов русских. И величаво пахать в своих сибирях на благо украинской шляхты и олигархии.
И больше ничего в жизни не нужно.
Кстати, по Крыму, наглухо дотационному, отлично видно.

\iusr{Олег Хавич}
\textbf{Евгений Отовчиц}, зато какая возможность пи*дить миллиарды на "развитии Крыма" появилась! Вот и с остальной б/УССР - та же причина экспансионизма, киевские попы и учёные XVII века тут ни при чём.

\iusr{Евгений Отовчиц}
\textbf{Олег Хавич} ничего, скоро на восстановлении Донбасса попилят.

\iusr{Natasha Varetskaya}
\textbf{Евгений Отовчиц} а что или кт не говно?

\iusr{Полина Дьяченко}
\textbf{Матвей Кублицкий} Про финансировпние может и не правда. Но Вы не хотите знать правду? ЦБ РФ курируют западные спецы. Об этом много написано. Погуглите. Многому удивитесь

\iusr{Евгений Отовчиц}
\textbf{Наталья Варецкая} понятия не имею. Мне в принципе современное сословное общество претит.
Но есть страны с приличным уровнем жизни. Финляндия, например. Финляндия - не говно.

\iusr{Natasha Varetskaya}
очень жаль Украину. Россию не жаль. она не слабая. она - такая. а вот Украину жаль... что-то здесь изначально не так...

\iusr{Natasha Varetskaya}
\textbf{Евгений Отовчиц} ответ принимается.

\iusr{Евгений Отовчиц}
\textbf{Полина Дьяченко} как это неправда? Бюджетное правило уже отменили?)) Кажется, пока не разрешали;)

\iusr{Полина Дьяченко}
\textbf{Евгений Отовчиц} ой, сильно подробности уже не помню)


\end{itemize} % }

\iusr{Алла Непомнящая}

ого сколько информации я вычитала и в посте и в комментах о провалах России))
Ну жду второй пост об успехах. Ну чисто по канонам журналистики же. В Украине
проще - не дел, нет провалов.. ну и успехов тоже.)

\iusr{Матвей Кублицкий}

даже в лом дискутировать. для информации - дочь лаврова живет в москве. у
большинства современных российских правителей - тоже. да и офшоров за границей
ни у кого из чиновников не осталось. а техногенные катастрофы происходят во
всех странах - где разрабатывают новую технику и эксплуатируют её. особый
пример - смех украинцев над российскими мультиками. мультики то реальностью
оказались. Главное - не смотреть что Россия уже сделала - а то очень неудобно
получается в сравнении с Украиной


\iusr{Natasha Varetskaya}

понравился анализ Олега Хавича. из Варшавы так ласково пересчитывать пенсии угнетенных россиян....
ребята... делайте со своими жизнями и странами что-то дельное... так противно
видеть как вы натираете свои косяки перекрученными фактами из жизни другой
страны, которая идет своим путем. СВОИМ. херовая страна - НЕ СРАВНИВАЙТЕ. и
будет вам счастье и интеграция.

\begin{itemize} % {
\iusr{Natasha Varetskaya}

После выкопанных морей, постройки пирамид и рождения Христа примитивные
состязания с другими народами в размерах построенных кораблей, небоскребов и
мостов- для великой нации не.комильфо, блядь.

Сегодня- очередной рекорд.

"Одновременно заиграли 146 трембит и рогов".

"..и рогов". Почему меня это не удивляет?

Может, потому, что локация рождения этого второго по значимости события после
пуска андронного коллайдера- главный донор "активистов" для обоих майданов?
Рогов там должно быть в избытке. Как и копыт.

Пауза между очередной, блядь, вылазкой на майдан и сезоном сбора яблок в Польше
отлично заполняется игрой на трембитах и рогах. На рогах друг друга.

Если музы, гоняющие тараканов в головах этих смерековых Паганини, требуют дозу
адреналина, музыканты на месяц-другой обычно отправляются на восток "немного
поубивать", как призывал их известный земляк.

Andrey Nedoves

\iusr{Олег Хавич}
\textbf{Наталья Варецкая}, дейстивтельно, это ж не крысу из навоза вылепить  @igg{fbicon.wink} 

\href{https://regnum.ru/news/society/2814877.html}{%
В Якутии слепили огромную крысу из навоза, regnum.ru, 23.12.2019%
}

\iusr{Natasha Varetskaya}
\textbf{Олег Хавич} ну, если это вы используете как аргумент и даже если в "шутку".... все одно - слабо.

\iusr{Олег Хавич}
\textbf{Наталья Варецкая}, я Вам сейчас расскажу шутку, которую Вы не знаете  @igg{fbicon.wink} 
Гуцульские программисты доказали, что восемь трембит составляют один трембайт.
С наступающими праздниками. Надеюсь, Вы относитесь к той части россиян, которая живёт на уровне Польши.

\iusr{Natasha Varetskaya}
\textbf{Олег Хавич} я наполовину русская и на четверть - полька( фамилия, если "что") а живу я в украине. живу по -разному, так как занимаюсь искусством. НО, когда люди начинают закрывать свои дыры Россией - мне не то чтобы смешно.. я понимаю, что нахожусь среди олигофренов.

\iusr{Олег Хавич}
\textbf{Наталья Варецкая}, понятно, то есть мастурбация на несуществующее "величие России" - это просто реакция замещения. Сочувствую, но ничем помочь не могу.

\iusr{Всеволод Саксонов}
\textbf{Олег Хавич}
ВВП по ППС на душу населения за 2018 год от МВФ:
Польша- 30 966
РФ - 27 961
Казахстан- 26 549
Беларусь- 18 480!!!
Украина- 9 129
И вас с Наступающими!))

\iusr{Олег Хавич}
\textbf{Всеволод Саксонов}, отнимите от показателей РФ хотя бы Москву - получите уровень жизни Румынии и Албании. А если ещё миллионники отнять - то и до Бангладеш будет недалеко.

\iusr{Natasha Varetskaya}
\textbf{Олег Хавич} батенька, "величие России" - как вы выразились, это на уровне ДНК.. УМОМ Россию не понять, тем более не русским умом..
это то, чего в вас видимо нет... поэтому присутствуют действия не проработанного второго контура он называется " анально - территориальный" - это когда человек-примат бегает и швыряется какашками, может и кучу навалить - так он выражает свое право на территорию.
НЕ будьте как приматы - ЖИВИТЕ ЦИВИЛИЗОВАННО. и вам праздников!

\iusr{Всеволод Саксонов}
\textbf{Олег Хавич} Беларусь, которая в два раза богаче Украины, тоже отнимать?

\iusr{Natasha Varetskaya}
\textbf{Всеволод Саксонов} ввп польши - это за счет Хавичей. ( ссори Олег Хавич)

\iusr{Олег Хавич}
\textbf{Всеволод Саксонов}, Беларусь сама у РФ отнимает - 10-15 миллаирдов долларов в год.

\iusr{Natasha Varetskaya}
\textbf{Всеволод Саксонов} вы верите цифрам МВФ?

\iusr{Natasha Varetskaya}
\textbf{Олег Хавич} Я же выше написала - у вас не проработанный 2-й контур. . поэтому примеры о навозе. вас воспитывали женщины без мужей/мужчин?

\iusr{Олег Хавич}
\textbf{Наталья Варецкая}, навоз в дискуссии был исключительно российский, это не ко мне. А папа мой прожил на 20 лет дольше мамы - но, к сожалению, родители давно покойны.

\iusr{Всеволод Саксонов}
\textbf{Олег Хавич} вот это новое общественное пространство в городе Дивногорск (Красноярский Край) с населением в 30т. человек. От МКАДА 4 126 км.

\ifcmt
  ig https://scontent-frx5-1.xx.fbcdn.net/v/t1.6435-9/80556289_2680709328634285_2063246725006491648_n.jpg?_nc_cat=105&ccb=1-5&_nc_sid=dbeb18&_nc_ohc=TaXfRF9sFI4AX-Z3yji&_nc_ht=scontent-frx5-1.xx&oh=01d2187ce1225481d0681b83b33f39d4&oe=61A8D8B3
  @width 0.4
\fi

\iusr{Natasha Varetskaya}
\textbf{Олег Хавич} вы выбрали аргумент с навозом. это характеризует.

\iusr{Олег Хавич}
\textbf{Наталья Варецкая}, да, но это характеризует Россию.

\iusr{Natasha Varetskaya}
\textbf{Всеволод Саксонов} и сейчас запостить движуху по Украине... опозориться чтобы... ну и чтобы трембит стало больше...

\iusr{Олег Хавич}
\textbf{Всеволод Саксонов}, 

хороший пример! На этой стройке колониальная администрация украла 30-50\%
бюджета (может, и 80\% - Красноярский край же). А санитарки и учителя
Дивногорска и дальше получают по 15-20 т.р. зарплаты.


\iusr{Natasha Varetskaya}
\textbf{Олег Хавич} ну вы же вроде не глупый человек.. пишете по - русски довольно грамотно.... запостить вам картины одесского художника Ройтбурда, который их писал фекалиями или сами загуглите?
( да, и он залихватски принял майдан, за что получил должность директора ОХМ)

\iusr{Олег Хавич}
\textbf{Наталья Варецкая}, Ройтбурд, в отличие от меня, знает только русский  @igg{fbicon.wink} 

\iusr{Natasha Varetskaya}
\textbf{Олег Хавич} ну как вы можете говорить о з/п в Рф, ПОСЛЕ ЗАРПЛАТ В УКРАИНЕ? бляяя... капец же полный...

\iusr{Всеволод Саксонов}
\textbf{Олег Хавич} не преуменьшайте, украли все 100\%))
Олег, не стесняйтесь, выкладывайте заказ-наряды, счета, платежные поручения, я тут же перешлю в Следственный комитет.... Они же у вас есть, вы же не просто так трепитесь,
не так ли?

\iusr{Natasha Varetskaya}
\textbf{Олег Хавич} вы о языке?

\iusr{Natasha Varetskaya}
\textbf{Всеволод Саксонов} зачет

\iusr{Олег Хавич}
\textbf{Всеволод Саксонов}, "Чтобы убедиться в тухлости яйца, не обязательно съедать его полностью" (Бернард Шоу).

\iusr{Всеволод Саксонов}
\textbf{Олег Хавич} вы знаете, я яйца не часто ем, предпочитаю овощи и фрукты....Какие виды на урожай клубники в Польше в этом году?)

\iusr{Natasha Varetskaya}
почему-то очень смешно:
"Зеленский будет отмечать Новый год вместе с семьёй и Богданом в карпатской резиденции «Синегора», где когда-то любил охотиться австрийский император"

\iusr{Олег Хавич}
\textbf{Всеволод Саксонов}, не знаю, в этом всего несколько дней осталось. Сегодня супермаркеты закрыты в связи с Рождеством, а завтра в 8 утра, уверен, проблем с клубникой в Польше не будет.

\iusr{Всеволод Саксонов}
\textbf{Олег Хавич} ну и Слава богу! Значит и ВВП Украины немного подрастёт!)

\iusr{Natasha Varetskaya}
\textbf{Олег Хавич}

\href{https://zaxid.net/u_lvovi_patsiyenti_psihlikarni_priyshli_z_vertepom_do_deputativ_oblradi_n1495216}{%
У Львові пацієнти психлікарні прийшли з вертепом до депутатів облради, zaxid.net, 24.12.2019%
}

\iusr{Natasha Varetskaya}
в результате мутации в стране развивается хорошо забытый старый вид искусства

\iusr{Фролов Руслан}
В смысле "из Варшавы"?

\end{itemize} % }

\iusr{Марина Прохорова}

Всё, расходимся. Автор просто раньше всех применил закон о медиа, существующий
пока в виде проекта. А там чётко сказано: О России ничего, кроме плохого )))))

\begin{itemize} % {
\iusr{Александра Валуйская}
\textbf{Марина Прохорова} неприятно удивил((( просто сшиб с ног - прямой в голову((((
\end{itemize} % }

\iusr{Natasha Varetskaya}

\href{https://strana.news/news/241412-zakon-o-media-kak-sluhi-naroda-budut-zakryvat-sajty-i-shtrafovat-za-novosti-o-putine.html}{%
Быстрая блокировка сайтов и запрет цитировать Путина. "Страна" узнала, что по итогу вписали в законопроект про медиа, %
strana.ua, 26.12.2019%
}

\iusr{Владимир Чабановский}
Пьяный сегодня что-ли? Абсолютную херню написал. Никакой связности и логики. Даже комментировать противно.

\begin{itemize} % {
\iusr{Игорь Лесев}
и надо же, поборол брезгливость и таки высрался

\iusr{Владимир Чабановский}
\textbf{Игорь Лесев} Дурак вы , Игорь, однако, а какое-то время прикидывался нормальным. Прискорбно.
\end{itemize} % }

\iusr{Дмитрий Луговой}
М...да. Такое впечатление, что на страницу Максим Кухар (Maxim Kukhar)
заглянул вместо Лесева. Не читали такого? Тогда загляните и прослезитесь.  @igg{fbicon.wink} 

\iusr{Александр Каревин}
Картинка зачётная. Я так понял: украинская красавица в объятиях русского медведя?  @igg{fbicon.smile} 

\begin{itemize} % {
\iusr{Игорь Лесев}
да, запечатлен акт зоофилии
\end{itemize} % }

\iusr{владимир фельдман}
Какой-то грязненький пост. Не хочется верить, что Игорь Лесев способен на заказуху, но очень похоже на это.

\begin{itemize} % {
\iusr{Алиса Лесева}
\textbf{владимир фельдман} Как правда, так сразу заказуха. А жаль!!!

\iusr{владимир фельдман}

Очень уважаю Игоря Лесева, всегда делаю перепосты его публикаций. А это пост
мне не нравится он односторонний, заангажирован на России. А зачем, что
хорошего в этой полуправде? Не буду даже это доказывать.


\iusr{Александра Валуйская}
\textbf{Алиса Лесева} извините, но сегодня - это был удар поддых((( и как минимум, нечестно и необъективно (((
\end{itemize} % }

\iusr{Костя Белоусов}
\textbf{Виталий Правдивый}

\url{https://glavred.info/opinions/10130288-kreditory-i-fondy-sozhrut-ukrainu-esli-ona-ne-pomenyaet-svoyu-politiku.htm}

Вот здесь нет эмоций и протест майдана был не только мечтой, но и глупостью...
Об этом Янукович вам хотел, но не успел рассказать в 2014...

\begin{itemize} % {
\iusr{Правдивый Виталий}
\textbf{Костя Белоусов} , спасибо, но страница не открывается.(

\iusr{Костя Белоусов}
\textbf{Правдивый Виталий}
Я увидел
Попробуйте самостоятельно
Главред

\ifcmt
  ig https://scontent-frx5-2.xx.fbcdn.net/v/t1.6435-9/81434939_3388298827878808_581740525535625216_n.jpg?_nc_cat=109&ccb=1-5&_nc_sid=dbeb18&_nc_ohc=5BpW-IvVoM0AX-uPz4J&_nc_ht=scontent-frx5-2.xx&oh=9d5eb190208b27eaeb5acc5333c01b8a&oe=61A96A47
  @width 0.4
\fi

\iusr{Костя Белоусов}
\textbf{Правдивый Виталий}, подпишитесь под этим тестом или майдан это только протест, а не глупая мечта?

\iusr{Правдивый Виталий}
\textbf{Костя Белоусов} , а Вас касаются Ваши посылы про хватание за язык?
Что это такое - глупая мечта?
Есть какая-то антитеза- пример умной мечты?
Я не смогу Вам обьяснить , что такое Майдан.
Когда у вас начнётся- поймёте сами.

\iusr{Костя Белоусов}
\textbf{Правдивый Виталий}, майдан это мечта о лучшей жизни, я по прежнему так считаю...
Мечта оказалась глупой.
Умная мечта это иметь троих детей и хорошую жену и работу, майдан в этом не помощник, как и великий октябрь 1917!
У нас начнётся нечто такое же? У нас уже было и не такое - расстрел 1993 года, война в Чечне 1994-2002... теракты 2002 и 2005 годы...

\iusr{Костя Белоусов}
\textbf{Правдивый Виталий} я хотел только узнать, не жаль Вам, что Янукович не успел вам всем рассказать о том, о чем написано в посте некоего Суслова???

\iusr{Правдивый Виталий}
\textbf{Костя Белоусов} , мы уже с Вами говорили о нем.
Жаль, не жаль...
Надо было остаться и рассказать.

\iusr{Костя Белоусов}
\textbf{Правдивый Виталий} вооружённые люди убивали своих и чужих, кому как не этому человеку было все известно тогда это???
Это Вы и другие ничего не знали и верили и себе и толпе... как большевикам в 1917

\iusr{Правдивый Виталий}
\textbf{Костя Белоусов} , он был главой государства с силовыми структурами.
Имея волю мог бы не допустить убийств.
Просто заниматься госстроительством затратно и хлопотно.
То ли дело паковаться.

\iusr{Костя Белоусов}
\textbf{Правдивый Виталий} это его моральное уродство не могло быть поводом для той акции ради которой некие негодяи стреляли в своих и чужих и той лжи после акции в течение 5 лет от которой пострадали 10-ки тысяч людей

\iusr{Правдивый Виталий}
\textbf{Костя Белоусов} , я не оправдываю подлость, убийства и ложь.
Мы чувствуем себя обманутыми. И Россией в том числе.
Но вопрос же в другом.
Как жить дальше с учётом всего этого?

\iusr{Костя Белоусов}
\textbf{Правдивый Виталий}, 

обмануты все - Вы, я, планы, надежды, людей лишили родного языка, люди лишились
жилья, работы... много погибло. .. все то что было причиной этого до сих пор
выдаётся за неизбежность и пока ставятся лайки под постами о неизбежности
майдана это будет продолжаться, к сожалению...

\end{itemize} % }

\iusr{Костя Белоусов}

\href{https://glavred.info/opinions/10130288-kreditory-i-fondy-sozhrut-ukrainu-esli-ona-ne-pomenyaet-svoyu-politiku.html}{%
Кредиторы и фонды сожрут Украину, если она не поменяет свою политику, gravred.info, 26.12.2019%
}

\iusr{Костя Белоусов}
Теперь получилось, Виталий это для Вас!

\iusr{Правдивый Виталий}
\textbf{Костя Белоусов} , спасибо, я уже Ваш пост видел.

\iusr{Mike Smith}
Это от души))

\iusr{Александра Валуйская}
Игорь, что это было???? Не зашло, от слова совсем((( можешь банить, конечно, твое право... Но, сегодня это точно - мимо(((

\iusr{Ирина Иваницкая}

Не тонут подлодки у тех, у кого их нет. Не горят авианосцы у тех, у кого их
нет. Ну и дальше - истребители пятого поколения. строительство трубопровода и
т.д и т.п. И не мешают и не "ставят палки в колеса" тем, кто из себя ничего не
представляет.

\iusr{Александр Ежов}
Всё правильно написано. И от этого обидно втрой-впятерне.

\iusr{Fox Renard}

Хосподя, украинцы, отъе.лись бы вы уже от клятой рашки. ОК, всё у них плохо,
везде просрали. Ну постройте свою державнисть, покажите мастер-класс.

\begin{itemize} % {
\iusr{Игорь Иващенко}
\textbf{Fox Renard} да не отъе\#утся. Они же ведь и есть эта самая рашка, ее самая лучшая часть!

\iusr{Natasha Varetskaya}
\textbf{Игорь Иващенко} а вот вы правы.
\end{itemize} % }

\iusr{Игорь Иващенко}
Прекрасно пишите, б\#ядь прекрасно! Новый Андерсен для долбо\#бов!

\iusr{Александр Власенко}

Конечно, из Украины взгляд на то, как живет Россия, будет априори искаженный. В
украинских медиа о России пишут либо плохо, либо никак. И если говорить о
мейнстримовых украинских телеканалах и журналах, то это тренд всех лет
независимости, а не с 2014 года. Поэтому, судить о России лучше по фактам, а не
по новостной ленте. Безусловно, Россия проигрывает США и Западу по
привлекательности, так как там выше уровень жизни, Голливуд создает красивую
картинку, там красивые и уютные города и выше зарплата. Но за Россию играет
более близкая ментальность, что проявляется и в лучшей усваиваемости
проивзодимого Россией культурного продукта. Язык - большое дело. И потому
российские сериалы были в Украине очень рейтинговыми, сейчас, кто может,
смотрит их по спутниковой тарелке или в интернете (хотя по качеству они - не
Нетфликс и не НВО). Дети, живущие в Интернете, по-прежнему обитают в общем
пространстве, которое можно назвать русскоязычным и включающем также
Белоруссию, Казахстан и русскоязычное население Прибалтики. Все их "мемасики" и
приколы - общие, слово "доширак" украинским подросткам известнее, чем "мивина"
(если только ее не покупают родители), и в этом очень большая "мягкая сила"
России. Для меня же важно, что Россия пытается и постепенно входит в клуб
стран, способных воплощать большие инфраструктурные проекты, которые Украине
уже, увы, не под силу (и это, пожалуй что, навсегда). Да, со скрипом. Да, порой
через тернии к звездам, как всегда у нас было. Да, порой очень по-русски, чтобы
сначала создать проблемы на ровном месте, а потом вспомнить, что в жизни всегда
есть место подвигу. Да, большой ценой и дорого. Но многим ли странам это под
силу? И Крымский мост на фоне 20-летнего недостроя Подольско-Воскресенского
моста в Киеве это отличие подчеркивает. В конце концов, в современной Украине в
качестве ориентира избрали Польшу, а не США. Ибо всем понятно, что
американского или немецкого уровня жизни достичь в обозримом будущем нереально.

\begin{itemize} % {
\iusr{Галина Дегтярева}
\textbf{Александр Власенко} Мне кажется, многие не поняли автора. Он и хочет, чтобы Россия как можно быстрее вошла в клуб этих стран, способных воплощать большие инфраструктурные проекты ,без создания не нужных проблем, чтобы не было потом необходимости совершать подвиги. Поэтому и написал об этих проблемах.
\end{itemize} % }

\iusr{Fox Renard}

Лично я для себя вывод уже сделал. Когда украинцы окончательно просрут своё
государство и его богатства, после чего, сирые и убогие, поползут на восток,
чтобы попросить просто тупо пожрать, хотя бы накормить детей, наступает очень
тонкий момент, исключительно важная грань. В этой ситуации в интересах РФ будет
НИЧЕГО НЕ ЗАБЫТЬ И НИЧЕГО НЕ ПРОСТИТЬ. После чего вытолкать всю эту
приблудившуюся шоблу штыками и прикладами назад в Украину, за поребрик. И это
будет самая важная инвестиция РФ в своё будущее.


\iusr{Игорь Потысьев}
В России две беды - дураки и дороги, а в Украине три - дураки , дороги и Россия

\iusr{Svitlana Gunning}

Интересует только в плане исчезновения в ее нынешнем виде. Пока Россия - это
такое абсолютное зло, которое лезет и лезет везде , где его не просят.
Интересует точно так же как изобретение лекарств от рака. Потому что понимаешь
- надо чтобы это зло как-то рассосалось .... а так - та на фига она надо ???

\begin{itemize} % {
\iusr{Курганская Виктория}
ага только последствия 17-го мы знаем.

\iusr{Курганская Виктория}

это на грани отчаяния, от невозможности повлиять на ситуацию. Видеть и не
делать, участвовать и бездействовать, понимать, что если платишь военный налог,
то воюешь. ничего не делать - окажемся в фавелах, что делать? просто жить в
своем домике уже не дают.

\end{itemize} % }

\iusr{Василий Лыбедской}

Шизофрения у вас в голове. Меньше надо новости читать, меньше будет подводных
лодок тонуть. Просто "сводка мировых СМИ" какая-то у вас получилась.


\iusr{Сергей Киселев}
А что за авианосец сгорел? Что за бред?

\iusr{Павло Тарасюк}
Дуже гарний пост  @igg{fbicon.smile} 

\iusr{Игорь Лесев}
\textbf{Павло Тарасюк} все думал, а где же Паша)))

\iusr{Долгов Олег}
Игорь с помощью этого поста решил проверить количество подписчиков из России.


\end{itemize} % }

