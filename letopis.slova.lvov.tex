% vim: keymap=russian-jcukenwin
%%beginhead 
 
%%file slova.lvov
%%parent slova
 
%%url 
 
%%author 
%%author_id 
%%author_url 
 
%%tags 
%%title 
 
%%endhead 
\chapter{Львов}
\label{sec:slova.lvov}

%%%cit
%%%cit_pic
%%%cit_text
Самое интересное, что практически все украинские националисты прекрасно знают
польский язык, и даже их украинская речь переполнена самыми явными полонизмами
сверх всякой меры. Это предполагает то, что корни этих националистов следует
искать не в Киеве или Чернигове, а исключительно в пределах Польши, не будем
забывать, что \emph{Львов}, рассадник любого украинского национализма, практически до
самого последнего времени был чисто польским городом. И даже кратковременно
пребывая в составе Австрийской империи (что-то около 150 лет), считался самими
австрийцами польским, несмотря на то, что ими же был переименован в \emph{Лемберг}
%%%cit_title
\citTitle{Почему современную Украину назвали «окраиной», а не более престижно - Киевской Русью?}, 
Исторический Понедельник, zen.yandex.ru, 22.02.2021 
%%%endcit

%%%cit
%%%cit_pic
%%%cit_text
Такую аналогию проводит ученый из США в связи с нашим скандалом.  Американский
историк, писатель, специалист по истории Украины Тарик Сирил Амар, бывший
сотрудник Центра изучения истории \emph{Львова} пишет: «Слава Украине» является
частью приветствия на которое отвечают «Героям слава». Это приветствие
исполнителей Холокоста и массовых этнических чисток во время Второй мировой (и
нет, это не вымысел российской информационной войны, а исторический факт).
Аргумент, что это приветствие сейчас используется в безвредных патриотических
целях – это высокопарный нонсенс. Почему? Потому, что новое использование
подразумевает, что те массовые убийцы, которые его использовали, не сделали
ничего плохого, чтобы не применять это приветствие
%%%cit_title
\citTitle{Представьте, что немцы опять решили использовать \enquote{Зиг Хайль!}}, 
Эдуард Долинский, strana.ua, 12.06.2021
%%%endcit

%%%cit
%%%cit_head
%%%cit_pic
%%%cit_text
Сегодня хочется немного поностальгировать по тем временам, когда по миру можно
было перемещаться относительно свободно, когда слово «корона» ассоциировалась у
большинства людей исключительно с головным убором монарха, а главной заботой
россиян было «чего там у соседей» (тут я изменил крылатую фразу дабы никого не
обидеть). Именно в поисках ответа на этот животрепещущий вопрос я и решил
провести две недели отпуска в 2017 году в (на) Украине, и первой моей
остановкой был \enquote{великий и ужасный} город \emph{Львов}!
%%%cit_comment
%%%cit_title
\citTitle{Поехал во Львов, когда многие боялись туда ехать, весь риск ради архитектуры. Показываю, что скрывается за фасадами.}, 
Нетуристический Путеводитель, zen.yandex.ru, 07.04.2021
%%%endcit

%%%cit
%%%cit_head
%%%cit_pic
\ifcmt
tab_begin cols=3
  width 0.3
  pic https://img.strana.ua/img/article/3410/kak-vo-lvove-22_main.jpeg

  pic https://strana.ua/img/forall/u/10/85/d500_dsc_1897_60d6f0630b50c.jpg

  pic https://strana.ua/img/forall/u/10/85/666554.png
tab_end
\fi
%%%cit_text
\enquote{\emph{Львов} традиционно дискутирует, насколько корректно во время московской
агрессии проводить даже самый лучший джазовый фестиваль за деньги российского
олигарха и одного из столпов режима Путина. А я задумался: разве приезд во
\emph{Львов} \enquote{джазового спонсора} Фридмана и его ручкания-обнимания с Садовым
не соответствует лозунгу партии \enquote{Самопомощь} \enquote{Свой к своему за
своим}}, - написал на своей странице в Facebook депутат \emph{Львовского} облсовета от
\enquote{Громадянськой позиции} Владимир Гирняк.  А известный своими
скандальными высказываниями львовский телеведущий Остап Дроздов посвятил целую
передачу \enquote{Раша-фестивалю}, как он назвал фест, а также разразился
длинным гневным пост в Facebook.  Самих \emph{львовян}, которые ходят на фест, по
обыкновению пренебрежительно назвал \enquote{\emph{львивцями}} (от слова
\enquote{вивци}) и \enquote{предателями}
%%%cit_comment
%%%cit_title
\citTitle{Как во Львове националисты снова взбунтовались из-за российского спонсора джазового фестиваля}, 
Александра Харченко, strana.ua, 29.06.2021
%%%endcit

%%%cit
%%%cit_head
%%%cit_pic
\ifcmt
tab_begin cols=2
  caption В соцсетях сокрушаются о том, что молодежь деградирует. Скриншот 

  pic https://img.strana.ua/img/article/3420/za-den-rozhdenija-81_main.jpeg

  pic https://lvov.strana.ua/img/forall/u/11/33/%D0%B4%D0%BF%D0%BC.png
tab_end
\fi
%%%cit_text
Во \emph{Львове} девушки студенческого возраста отпраздновали день рождения
одной из своей компании на Мемориале памяти героев Небесной сотни. После
публикации видео веселья в соцсетях полиция \emph{Львовской} области открыла на
них уголовное дело.  Видео уличной вечеринки с фейерверком участница компании
опубликовала на своей странице в Инстаграм, после чего  его заметили подписчики
паблика \enquote{Вторая частная мемарня} и распространили в Фейсбуке и
Твиттере. В ролике девушки обнимаются и танцуют под песню \enquote{18} в
исполнении российского блогера и музыканта Nechaev
%%%cit_comment
%%%cit_title
\citTitle{За день рождения на фоне мемориала Небесной сотни во Львове на студенток завели дело}, 
Наталья Полулях, lvov.strana.ua, 05.07.2021
%%%endcit
