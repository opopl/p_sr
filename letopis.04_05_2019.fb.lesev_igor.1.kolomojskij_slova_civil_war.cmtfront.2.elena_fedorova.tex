% vim: keymap=russian-jcukenwin
%%beginhead 
 
%%file 04_05_2019.fb.lesev_igor.1.kolomojskij_slova_civil_war.cmtfront.2.elena_fedorova
%%parent 04_05_2019.fb.lesev_igor.1.kolomojskij_slova_civil_war
 
%%url 
 
%%author_id 
%%date 
 
%%tags 
%%title 
 
%%endhead 

\begin{cmtfront}
\normalsize\uzr{Елена Фёдорова}

Тут такая заковыка. Сразу обозначу. Я - из Днепропетровска. Семья проживала в
нем с конца 1944 ода, сразу после освобождения города. Их отправили в теплушках
восстанавливать город. Конкретно - Горный институт и Криворожские ГОКи. Корни -
дореволюционный Санкт-Петербург, профессор Санкт-Петербургской Горной
Академии,. Дед же был проректором Днепропетровского Горного института. Второе -
город изначально звался Екатеринослав и историю его основания Вы, вероятно,
знаете, ну или легко найдете в Википедии. Как мы, выросшие в русской культуре
восприняли события 13-14 г? как посягательство на свою идентичность. культуру,
этническую принадлежность. Кто посягнул - неонацисты. При полной поддержке США
и ЕС. Вот мы и поднялись защищать страну от распада, упадка и уничтожения. На
кого мы могли рассчитывать? На своих же по культуре и идентичности - на
русских, где бы они ни находились. Защищать Украину? Да, готовы от внешнего
агрессора - поляков, американцев и прочих натовцев, или кому там теоретически
может взбрести в голову. И от сил, возжелавших устроить геноцид. Я уверена,
пройдет время, будут названы причины, пружины, заинтересанты,
выгодоприобретатели госпереворота, все будет разложено по полочкам. Состоятся
суды. И вот тогда выяснится, что именно русские Украины, которых сейчас
убивают, выдавливают из страны, над которыми глумятся и унижают, именно они
спасли Украину и сохранили ее государственность.
	
\end{cmtfront}

