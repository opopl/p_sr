% vim: keymap=russian-jcukenwin
%%beginhead 
 
%%file 09_12_2020.news.ru.ukraina_ru.ischenko_rostislav.1.kiev_pora_vzryvat
%%parent 09_12_2020
 
%%url https://ukraina.ru/opinion/20201209/1029903835.html
 
%%author Ищенко, Ростислав
%%author_id ischenko_rostislav
%%author_url 
 
%%tags kiev,politics,opinion
%%title Киев: пора взрывать
 
%%endhead 
 
\subsection{Киев: пора взрывать}
\label{sec:09_12_2020.news.ru.ukraina_ru.ischenko_rostislav.1.kiev_pora_vzryvat}
\Purl{https://ukraina.ru/opinion/20201209/1029903835.html}
\ifcmt
  author_begin
   author_id ischenko_rostislav
  author_end
\fi

\ifcmt
pic https://cdn1.img.ukraina.ru/images/102990/46/1029904637.jpg
\fi

\begin{leftbar}
  \bfseries
В течение последнего года риторика украинских властей по отношению к ДНР/ЛНР,
да и по отношению к России стала значительно жёстче
\end{leftbar}

Украинские политики всех уровней, цветов и размеров, не скрываясь, открытым
текстом говорят, что Минские соглашения выполнять они не будут. Если же кто-то
хочет, чтобы Минские соглашения выполнялись, надо безоговорочно признать их
украинскую трактовку. Зеленский и его министр иностранных дел даже пытались
пугать (непонятно кого) выходом Украины из Минских соглашений, если трактовки
Киева не будут положены в основу обсуждаемых Трёхсторонней контактной группой
проектов реализации Минска.

В общем, Украина постепенно, но довольно быстро перешла от своей любимой
(продолжавшейся пять лет) игры «я так вижу» и «объясните мне, как это
выполнять», которая предполагала, что Украина как партнёр — ничего не
понимающий дурак, но конструктивный (искренне желающий понять) дурак, к полному
отрицанию возможности какого-либо конструктива. Изменение своей позиции Киев
объяснял тем, что раньше, мол, вынужден был пойти на заключение Минских
соглашений, чтобы остановить проигранную им войну и не допустить возобновления
боевых действий.

При всём моём скептическом отношении к умственным способностям украинских
чиновников и политиков я всё же уверен, что они не рассчитывают в случае
возобновления боевых действий выиграть войну. Наоборот, если в 2015-2017 годах
украинские военные эксперты усиленно готовились захватить Донбасс «по
хорватскому сценарию», то с тех пор, как Путин предостерёг от авантюр,
способных стоить Украине государственности, даже самые большие оптимисты в
Киеве не надеются на возможность проведения против Донбасса успешной силовой
акции.

Можно было бы предположить, что в Киеве надеются на то, что приход
администрации Байдена резко увеличит поддержку Украины со стороны Запада.
Однако даже человеку, далёкому от актуальных политических проблем, понятно, что
администрация Обамы вступала в конфронтацию с Россией в значительно лучших
условиях, чем те, в которых уже сейчас находится администрация Байдена. У США
тогда не было жёсткого противостояния с Китаем. Сейчас же даже такой
патентованный ястреб, как Хиллари Клинтон, в программной речи, указывая, на что
следует обратить внимание Байдену в Белом доме, предостерегает от слишком
глубокой конфронтации с Россией, замечая, что отношения с Китаем прошли точку
возврата и оперативно восстановить их будет невозможно. Поэтому она не
рекомендует Байдену воевать на два фронта, советуя ограничиться на российском
направлении скорее шагами демонстративными, чем существенными.


Ситуация для США тем тяжелее, что ЕС также почувствовал ослабление американской
гегемонии и задумался о самостоятельности. Прямо конфликтовать с США Европа не
будет, но и таскать для Вашингтона каштаны из огня, беспричинно ссорясь с
Россией, тоже. Так что возложить ответственную миссию сдерживания Москвы на
союзников тоже не выйдет. Соответственно и Украине не приходится рассчитывать
на то, что Запад коллективно или США самостоятельно пойдут в защите Киева
дальше дипломатических протестов и объявления очередного пакета недействующих
санкций.

Тем не менее Киев явно провоцирует военный конфликт, причём не только с
Донбассом, но и с Россией. Как уже было сказано, Минские соглашения Украиной
де-факто дезавуированы и публично заявлено о намерении вернуть Донбасс (кстати,
и Крым) путём занятия жёсткой и бескомпромиссной позиции, если надо —
подкреплённой силовым ресурсом. И это не слова. Террористические обстрелы
городов Донбасса украинской артиллерией продолжаются, ДРГ через линию
разграничения ходят, попытки «отжима» нейтралки периодически возобновляются,
военнослужащие и мирные жители Донбасса периодически гибнут. А на днях была
предотвращена попытка вывоза на Украину похищенного в России гражданина РФ. При
этом ССО (Силы специальных операций) Украины оказывали похитителям силовую
поддержку, вступив в бой с российскими пограничниками.

Подобного рода пограничные провокации Москва может в любой момент посчитать
поводом, достаточным для начала боевых действий в порядке индивидуальной
самообороны от неспровоцированной агрессии. Ну а пока там будут проведены все
предусмотренные Уставом ООН меры по восстановлению мира, от Украины может уже и
мокрого места не остаться.

В Киеве этого не понимают? Уверен, что понимают. Россия неоднократно наглядно
показывала, как она умеет наказывать наглецов. Саакашвили может поделиться
богатым опытом. Благо, он под рукой — в Киеве находится.


На что рассчитывают украинские политики? На то, что им нет альтернативы? Это
правда. Даже весьма условно договороспособные украинские деятели, во-первых,
практически не имеют шансов на приход к власти, во-вторых, не имеют
сколько-нибудь конструктивной программы урегулирования украинско-российских
противоречий. Все их предложения сводятся к тому, чтобы Россия в одностороннем
порядке сняла санкции, помогла Украине восстановить власть над Донбассом,
стабилизировала украинскую государственность, взамен же Киев даже Крым
российским признать не готов. То есть гипотетическая смена украинской власти в
результате вооружённого конфликта ничего для России не меняет, да и не факт,
что власть сменится, как раз наоборот.

Украинские политики столкнулись с серьёзной внутренней дестабилизацией. Причём
дестабилизируют Украину не гипотетические «пророссийские силы», давно загнанные
в подполье, уничтоженные или изгнанные из страны. Дестабилизацию вызывает
борьба внутри самой украинской власти за ограниченный и постоянно сокращающийся
ресурс.

Традиционно в таких случаях власть прибегает к испытанному способу консолидации
общества, именуемому «маленькая победоносная война». Победоносную войну Украина
вести не может — ни маленькую, ни большую. Зато Украина может вести войну
вообще. В 2014-2015 году Порошенко, спекулируя тезисом войны с Россией, удалось
стабилизировать свою власть, несмотря на два подряд катастрофических поражения.
Почему бы не попробовать ещё раз?

На деле Донбасс киевским властям совсем не интересен и не нужен. Но и
отказаться они от него они не могут: внутриполитические конкуренты тут же
обвинят в предательстве. В Киеве также отдают себе отчёт, что завоёвывать
большую Украину России надо ещё меньше, чем маленькую Грузию в 2008 году.
Потерять ещё кусок территории и пару миллионов населения там тоже не боятся.
Население в Донбассе всё равно преимущественно бандеровской власти нелояльное,
территория, подконтрольная Украине,  разорена семилетней войной и хозяйничаньем
«военно-гражданских администраций» и просто силовиков. Зарабатывать на
контрабанде через линию разграничения руководители ВСУ и СБУ могут независимо
от того, где эта линия проходит (под Донецком или под Бердянском).

Зато потерявшее страх и возмущающееся постоянным падением уровня жизни
население можно опять пугать русскими танками, которые вот-вот ворвутся в Киев,
Луцк и Львов. У Запада можно вновь просить деньги «на войну с Россией». Под тем
же предлогом можно попытаться отложить выполнение особенно непопулярных
требований МВФ. Война ведь, о какой свободной продаже земли может идти речь.
Вначале дайте денег на победу, а затем посмотрим.

Конечно, есть риск, что войска ДНР/ЛНР больше не остановятся (обещали же, что
Минска-3 не будет). Но для особо ушлых в этом есть своя прелесть. Украина уже
обобрана до нитки и начинает создавать больше проблем, чем приносить прибылей.
Есть, конечно, такие жадные люди, как Порошенко, и такие простодушные, как
Ахметов, которые до последнего будут ждать, что и на этот раз пронесёт. Есть
проблема Коломойского, которому пока с Украины некуда бежать. Но часть
полуразорённых олигархов видит в «российской оккупации» не опасность, а
возможность. Это форс-мажор, который спишет все их обязательства и позволит
спокойно доживать жизнь на оставшиеся от состояния несколько десятков миллионов
переведённых в наличку долларов где-нибудь в Британии или в Швейцарии. По
крайней мере, они так думают.

Так же думают многие наворовавшиеся генералы и старшие офицеры ВСУ и других
силовых структур. Сидя на Украине, можно попасть либо под раскулачивание своими
же товарищами, которые позже пробились к корыту и обнаружили, что всё украдено
до них, либо под народный бунт. А если отправиться в политическую эмиграцию, то
можно даже правительством в изгнании поработать. Думаю, что и наиболее
сообразительная часть грантоедов из числа журналистов, экспертов и прочих
«общественников» совсем не прочь бросить полуобглоданный труп страны и
перебраться куда-нибудь подальше от ответственности за соучастие в её убийстве.
Тем более есть шанс примкнуть к эмигрантским политическим структурам, которые
обязательно будут созданы и будут финансироваться Западом для того, чтобы
продолжать создавать проблемы России. На всех, конечно, западной
«благотворительности» не хватит, но кому-то повезёт.

В общем, по мере осознания сторонниками нынешнего режима бесперспективности
украинского проекта как в целом, так и для них лично война на Востоке,
независимо от её формата и от объёма неизбежного поражения, начинает казаться
им не просто не опасной, но единственным выходом из безвыходной ситуации. Чем
очевиднее украинский тупик, тем для большего количества адептов режима внешняя
война (или что-то, что можно за неё выдать) становится спасительной и желанной.

На оставленные беглецами места у корыта с удовольствием рванутся нижестоящие.
Как показывает опыт того же Третьего рейха, находились идиоты, которые до
последнего дня (даже когда Гитлер был уже мёртв, а фронт рухнул) думали только
о карьере и боролись за должности. Им и отвечать, если что.

Конечно, риск есть. И до сих пор украинские политики и их обслуга старались
этого риска избегать. Кто-то ведь всегда не успевает удрать, а кому-то, как
Остапу Бендеру с румынскими пограничниками, не везёт с принимающей стороной. Но
с каждым днём риск неконтролируемого распада Украины становится всё больше,
перекрывая остальные риски.

Думаю, не случайно поведение Украины на международной арене становится тем
провокационнее, чем чаще киевские политики и эксперты публично констатируют
распад страны, деградацию экономики и невозможность преодолеть кризис своими
силами. Кто в полной, кто в не полной мере, но постепенно все начинают
осознавать безвыходность ситуации. На этом фоне война (какого бы результата от
неё ни ждали) начинает казаться далеко не худшим решением проблемы.

\url{https://youtu.be/SD_usPMazNw}

Это, конечно, не ход по правилам, а банальный переворот доски. Но украинские
политики неоднократно доказывали, что для них правил не существует, а мнение
того же Запада их интересует, только пока он деньги даёт и безопасность им
обеспечивает. Если же эти функции Запад выполнять больше не способен, а он не
способен, то каждый сам за себя и спасается как может и как умеет.

У украинских руководителей богатый опыт поджогов и взрывов складов накануне
сулящей проблемы ревизии. В их глазах страна ничем, кроме размеров, от склада
не отличается. Поскольку она разворована, а ревизия близко, опыт подсказывает,
что пора взрывать.

