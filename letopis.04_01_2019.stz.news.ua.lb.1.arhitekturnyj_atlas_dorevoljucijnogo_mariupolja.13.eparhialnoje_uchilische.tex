% vim: keymap=russian-jcukenwin
%%beginhead 
 
%%file 04_01_2019.stz.news.ua.lb.1.arhitekturnyj_atlas_dorevoljucijnogo_mariupolja.13.eparhialnoje_uchilische
%%parent 04_01_2019.stz.news.ua.lb.1.arhitekturnyj_atlas_dorevoljucijnogo_mariupolja
 
%%url 
 
%%author_id 
%%date 
 
%%tags 
%%title 
 
%%endhead 

\subsubsection{Епархиальное училище}

В 1911 году, право называться самым большим в городе зданием перешло к
Епархиальному женскому училищу. Выполнено оно по проекту городского архитектора
Нильсена. Интересно, что мариупольский зодчий воплощал этот масштабный проект
одновременно с не менее сложным строительством городского водопровода. Вот так
плодовитость! После революции Епархиальное училище, конечно, было закрыто.
Известно, что во время гражданской войны в нём работал клуб, где проводились
вечера для выздоравливающих красноармейцев. Позже в стенах расположился 238-й
стрелковый полк, на его основе в 1939 году развернули 80-ю стрелковую дивизию,
трагически сгинувшую во время войны. Бойцом этой дивизии был советский
кинорежиссер Григорий Чухрай. Немцы использовали корпуса как штаб и
комендатуру. В октябре 1941 года отсюда на расстрел к траншеям Агробазы они
отправили тысячи евреев Приазовья. В 1951 году здание восстановил и занял
Ждановский металлургический институт (сейчас ПГТУ), чтобы остаться здесь до
наших дней. Университет оказался очень ответственным квартиросъёмщиком, старое
Епархиальное училище содержится в хорошем состоянии.

\ii{04_01_2019.stz.news.ua.lb.1.arhitekturnyj_atlas_dorevoljucijnogo_mariupolja.13.eparhialnoje_uchilische.pic.1}

\emph{Ещё одна большая группа архитектурных сооружений – административные здания и
государственные учреждения. На рубеже Х ІХ – ХХ веков в Мариуполе, можно
сказать произошла индустриальная революция. Её прямым следствием стал
стремительный рост населения. Благодаря трудовой миграции количество жителей
Мариуполя почти удвоилось всего за 16 лет в период с 1897 по 1913 годы. Эти
обстоятельства создали необходимость создания новых и расширения существующих
гражданских, а также государственных институтов для обеспечения хозяйственной
деятельности города.}

\ii{04_01_2019.stz.news.ua.lb.1.arhitekturnyj_atlas_dorevoljucijnogo_mariupolja.13.eparhialnoje_uchilische.pic.2}
\ii{04_01_2019.stz.news.ua.lb.1.arhitekturnyj_atlas_dorevoljucijnogo_mariupolja.13.eparhialnoje_uchilische.pic.3}
\ii{04_01_2019.stz.news.ua.lb.1.arhitekturnyj_atlas_dorevoljucijnogo_mariupolja.13.eparhialnoje_uchilische.pic.4}
\ii{04_01_2019.stz.news.ua.lb.1.arhitekturnyj_atlas_dorevoljucijnogo_mariupolja.13.eparhialnoje_uchilische.pic.5}
\ii{04_01_2019.stz.news.ua.lb.1.arhitekturnyj_atlas_dorevoljucijnogo_mariupolja.13.eparhialnoje_uchilische.pic.6}
