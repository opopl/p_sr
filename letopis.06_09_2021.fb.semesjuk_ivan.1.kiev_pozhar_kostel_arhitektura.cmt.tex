% vim: keymap=russian-jcukenwin
%%beginhead 
 
%%file 06_09_2021.fb.semesjuk_ivan.1.kiev_pozhar_kostel_arhitektura.cmt
%%parent 06_09_2021.fb.semesjuk_ivan.1.kiev_pozhar_kostel_arhitektura
 
%%url 
 
%%author_id 
%%date 
 
%%tags 
%%title 
 
%%endhead 
\subsubsection{Коментарі}

\begin{itemize} % {
%%%fbauth
%%%fbauth_name
\iusr{Марина Марченко}
%%%fbauth_url
%%%fbauth_place
%%%fbauth_id
%%%fbauth_front
%%%fbauth_desc
%%%fbauth_www
%%%fbauth_pic
%%%fbauth_pic portrait
%%%fbauth_pic background
%%%fbauth_pic other
%%%fbauth_tags
%%%fbauth_pubs
%%%endfbauth
 

\obeycr
Ох, як смачно ви все це описали.
Ще зранку хотіла відгукнутися.
А саме про Полісся.
Довелося сидіти там прямо у глибинці, де навіть автобусів нема, пів мясяця з італійцями.
Ну точно Терра Інкогніта - така незаймана природа - мішаний красивенний ліс, озера(нехай у кар'єрах), люди такі класні, щедрі і зараз дзвонять запрошують по гриби.
Із для мене дивного було почуто про нєшуточну війну між житомирськими та рівненьськими за ліс.
Наші(житомирькі) не пускають рівненьських до себе, бо кажуть, що ті знищили все у себе із-за бурштину, тепер до них лізуть збирати ягоди та гриби. І там війна серйозна.
Як вони люблять ліс, милуються їм.
Але є проблеми - алкоголізм дійсно зашкалює, вплодь до бабусь і через одного рак шлунку, Чернобиль.
\restorecr

%%%fbauth
%%%fbauth_name
\iusr{Дмитро Руденко}
%%%fbauth_url
%%%fbauth_place
%%%fbauth_id
%%%fbauth_front
%%%fbauth_desc
%%%fbauth_www
%%%fbauth_pic
%%%fbauth_pic portrait
%%%fbauth_pic background
%%%fbauth_pic other
%%%fbauth_tags
%%%fbauth_pubs
%%%endfbauth
 

Ця пожежа, однозначно є символічною, тому найбільше хотілося б щоб ми правильно
зрозуміли цей символізм і відповідно відреагували, бо вже так набридло
виправляти наслідки власної байдужості до сакральних речей.


%%%fbauth
%%%fbauth_name
\iusr{Олекса Манн}
%%%fbauth_url
%%%fbauth_place
%%%fbauth_id
%%%fbauth_front
%%%fbauth_desc
%%%fbauth_www
%%%fbauth_pic
%%%fbauth_pic portrait
%%%fbauth_pic background
%%%fbauth_pic other
%%%fbauth_tags
%%%fbauth_pubs
%%%endfbauth
 

До речі, європейці до середини 19 сторіччя дупля не відбивали що то за готика
така. Для них це будо потворство, несмак і груда почорнівших камінюк, яка
заважає порпатись на городі і пасти там гусей. Покі не з’явився Ежен
Віоле-ле-Дюк, не почав реставрувати собори і замки, писати книги і статті та не
пояснив популярно публіці, що це взагалі таке. І що це має неабияку культурну
цінність. І зараз, завдяки йому, у масовій свідомості сформована ця зв’язка
Готика=Європа=Сердьновіччя=Культура. До нього її не було, все це, що будували
сторіччями цілими містами і цехами, руйнувалось і зносилось тільки так.

\begin{itemize} % {
%%%fbauth
%%%fbauth_name
\iusr{Оксана Цюпа}
%%%fbauth_url
%%%fbauth_place
%%%fbauth_id
%%%fbauth_front
%%%fbauth_desc
%%%fbauth_www
%%%fbauth_pic
%%%fbauth_pic portrait
%%%fbauth_pic background
%%%fbauth_pic other
%%%fbauth_tags
%%%fbauth_pubs
%%%endfbauth
 
\textbf{Olexa Mann} Українці почали розбиратися в неоготиці, а от в романському стилі ніяк, сумно, що руїни десятинної в Києві не мають жодного мистецького, культурного, історичного значення для освічених людей нашого краю, бо вже касапи там свій рпцшний храм збудували. В центрі Києва і ніхто не переймається . Сумно все

%%%fbauth
%%%fbauth_name
\iusr{Raisa Prisetska}
%%%fbauth_url
%%%fbauth_place
%%%fbauth_id
%%%fbauth_front
%%%fbauth_desc
%%%fbauth_www
%%%fbauth_pic
%%%fbauth_pic portrait
%%%fbauth_pic background
%%%fbauth_pic other
%%%fbauth_tags
%%%fbauth_pubs
%%%endfbauth
 
\textbf{Оксана Цюпа} будівництво рпц шних церков- то чиста пролітика, проникнення в свідомість і життя люду. рпц споруда біля Ейфелевої вежі і всі приміщення її - це чисто фсб штука.. І вона вже має вплив на пересічних французів- бо то "духовний" дом. Там і оперні співаки і всякі пеци виступають. сКазати, шо наші цього не розуміють? сумніваюсь... Простіше мати архітекторів-запроданців совіцького пошибу і владу таку ж. яка нічого не тільки за кордоном, а й вдома не робить. ВІргіше, трудиться в потілиця проти України. З культури починається все, з радіо на стіні, з музеїв, з історіїї - з поваги до кожної збереженої цеглинки старовини..

%%%fbauth
%%%fbauth_name
\iusr{Valerii Zema}
%%%fbauth_url
%%%fbauth_place
%%%fbauth_id
%%%fbauth_front
%%%fbauth_desc
%%%fbauth_www
%%%fbauth_pic
%%%fbauth_pic portrait
%%%fbauth_pic background
%%%fbauth_pic other
%%%fbauth_tags
%%%fbauth_pubs
%%%endfbauth
 
\textbf{Olexa Mann} ну чого ж це, Віктор Гюґо долучився і Вальтер Скот, Фюстель де Кулянж

%%%fbauth
%%%fbauth_name
\iusr{Volodymyr Kolbasa}
%%%fbauth_url
%%%fbauth_place
%%%fbauth_id
%%%fbauth_front
%%%fbauth_desc
%%%fbauth_www
%%%fbauth_pic
%%%fbauth_pic portrait
%%%fbauth_pic background
%%%fbauth_pic other
%%%fbauth_tags
%%%fbauth_pubs
%%%endfbauth
 
\textbf{Olexa Mann} Кьольнський собор добудували вже при Пруссії після Наполеона. Перед тим він стовбичив з однією баштою і краном на ній аж з Середньовіччя. Кран перейшов прямо з полотен 15 ст на фотографії

%%%fbauth
%%%fbauth_name
\iusr{Максим Горбань}
%%%fbauth_url
%%%fbauth_place
%%%fbauth_id
%%%fbauth_front
%%%fbauth_desc
%%%fbauth_www
%%%fbauth_pic
%%%fbauth_pic portrait
%%%fbauth_pic background
%%%fbauth_pic other
%%%fbauth_tags
%%%fbauth_pubs
%%%endfbauth
 
\textbf{Olexa Mann} це якось трохи спрощений погляд, адже саме готичні собори були центром помазання монархів на трон.

%%%fbauth
%%%fbauth_name
\iusr{Олекса Манн}
%%%fbauth_url
%%%fbauth_place
%%%fbauth_id
%%%fbauth_front
%%%fbauth_desc
%%%fbauth_www
%%%fbauth_pic
%%%fbauth_pic portrait
%%%fbauth_pic background
%%%fbauth_pic other
%%%fbauth_tags
%%%fbauth_pubs
%%%endfbauth
 

\textbf{Maxim Gorban} це відбулось після втрати функціональної актуальності готичних споруд для громади.

%%%fbauth
%%%fbauth_name
\iusr{Pavlo Salyga}
%%%fbauth_url
%%%fbauth_place
%%%fbauth_id
%%%fbauth_front
%%%fbauth_desc
%%%fbauth_www
%%%fbauth_pic
%%%fbauth_pic portrait
%%%fbauth_pic background
%%%fbauth_pic other
%%%fbauth_tags
%%%fbauth_pubs
%%%endfbauth
 
Я так розумію, що повернення готики в архітектуру відбувалося паралельно до зародження національної ідентичності і концентрацією уваги на оцій всій Вальтерскотівщині

%%%fbauth
%%%fbauth_name
\iusr{Pavlo Salyga}
%%%fbauth_url
%%%fbauth_place
%%%fbauth_id
%%%fbauth_front
%%%fbauth_desc
%%%fbauth_www
%%%fbauth_pic
%%%fbauth_pic portrait
%%%fbauth_pic background
%%%fbauth_pic other
%%%fbauth_tags
%%%fbauth_pubs
%%%endfbauth
 
Неоготичні будівлі були популярні серед української національносвідомої буржуазії кінця 19ст. Наприклад неоготичні будинки Терещенків
\end{itemize} % }

%%%fbauth
%%%fbauth_name
\iusr{Дмитро Різниченко}
%%%fbauth_url
%%%fbauth_place
%%%fbauth_id
%%%fbauth_front
%%%fbauth_desc
%%%fbauth_www
%%%fbauth_pic
%%%fbauth_pic portrait
%%%fbauth_pic background
%%%fbauth_pic other
%%%fbauth_tags
%%%fbauth_pubs
%%%endfbauth
 
За Кривий Ріг стало обідно.

\begin{itemize} % {
%%%fbauth
%%%fbauth_name
\iusr{Olga Garbuzyuk}
%%%fbauth_url
%%%fbauth_place
%%%fbauth_id
%%%fbauth_front
%%%fbauth_desc
%%%fbauth_www
%%%fbauth_pic
%%%fbauth_pic portrait
%%%fbauth_pic background
%%%fbauth_pic other
%%%fbauth_tags
%%%fbauth_pubs
%%%endfbauth
 
\textbf{Дмитро Різниченко} Маю пару розкішних фотоальбомів про Кривий Ріг. Люди старались, вибирати найкращі об'єкти. Так от, єдине, що візуально точно ідентифікує Кривий Ріг, це відкриті кар'єри. Все інше може бути будь-де. Якби не напис, нізащо не вгадаєш, де знаходиться цей будинок. От і все
\end{itemize} % }

%%%fbauth
%%%fbauth_name
\iusr{Маріанна Перепелиціна}
%%%fbauth_url
%%%fbauth_place
%%%fbauth_id
%%%fbauth_front
%%%fbauth_desc
%%%fbauth_www
%%%fbauth_pic
%%%fbauth_pic portrait
%%%fbauth_pic background
%%%fbauth_pic other
%%%fbauth_tags
%%%fbauth_pubs
%%%endfbauth
 
Ну для мене не тільки готика. Але здебільшого погоджуюсь. Сумно.

%%%fbauth
%%%fbauth_name
\iusr{Євгенія Чуприна}
%%%fbauth_url
%%%fbauth_place
%%%fbauth_id
%%%fbauth_front
%%%fbauth_desc
%%%fbauth_www
%%%fbauth_pic
%%%fbauth_pic portrait
%%%fbauth_pic background
%%%fbauth_pic other
%%%fbauth_tags
%%%fbauth_pubs
%%%endfbauth
 
А чи не є історичні об'єкти, особливо культові споруди, апогеєм безсенсовності?

\begin{itemize} % {
%%%fbauth
%%%fbauth_name
\iusr{Іван Семесюк}
%%%fbauth_url
%%%fbauth_place
%%%fbauth_id
%%%fbauth_front
%%%fbauth_desc
%%%fbauth_www
%%%fbauth_pic
%%%fbauth_pic portrait
%%%fbauth_pic background
%%%fbauth_pic other
%%%fbauth_tags
%%%fbauth_pubs
%%%endfbauth
 
\textbf{Євгенія Чуприна} ні


%%%fbauth
%%%fbauth_name
\iusr{Іван Семесюк}
%%%fbauth_url
%%%fbauth_place
%%%fbauth_id
%%%fbauth_front
%%%fbauth_desc
%%%fbauth_www
%%%fbauth_pic
%%%fbauth_pic portrait
%%%fbauth_pic background
%%%fbauth_pic other
%%%fbauth_tags
%%%fbauth_pubs
%%%endfbauth
 
\textbf{Євгенія Чуприна} Це занадто широкий маневр, тут треба одразу починати зі смислу культури як такої у виживанні виду


%%%fbauth
%%%fbauth_name
\iusr{Іван Семесюк}
%%%fbauth_url
%%%fbauth_place
%%%fbauth_id
%%%fbauth_front
%%%fbauth_desc
%%%fbauth_www
%%%fbauth_pic
%%%fbauth_pic portrait
%%%fbauth_pic background
%%%fbauth_pic other
%%%fbauth_tags
%%%fbauth_pubs
%%%endfbauth
 
\textbf{Nataliya Lyutenko} дарма, бо релігія відповідала за цілеутворення, зшивала суспільства зсередини. Людство без чогось більшого ніж воно само у свідомості, хай віртуального, це мавпи. Врешті культура вона уся по суті є віртуальністю, тому релігії як її еманація є природним і необхідним явищем, просто навіть для того, щоби не почуватися в космосі самотньо


%%%fbauth
%%%fbauth_name
\iusr{Іван Семесюк}
%%%fbauth_url
%%%fbauth_place
%%%fbauth_id
%%%fbauth_front
%%%fbauth_desc
%%%fbauth_www
%%%fbauth_pic
%%%fbauth_pic portrait
%%%fbauth_pic background
%%%fbauth_pic other
%%%fbauth_tags
%%%fbauth_pubs
%%%endfbauth
 
\textbf{Jake Ferguson} ну, просто вони як професіонали розводили науку і свої філософічно уявлення по кутках.
\end{itemize} % }

%%%fbauth
%%%fbauth_name
\iusr{Alexey Teplinsky}
%%%fbauth_url
%%%fbauth_place
%%%fbauth_id
%%%fbauth_front
%%%fbauth_desc
%%%fbauth_www
%%%fbauth_pic
%%%fbauth_pic portrait
%%%fbauth_pic background
%%%fbauth_pic other
%%%fbauth_tags
%%%fbauth_pubs
%%%endfbauth
 

стародавня архітектура Києва жалюгідна, включаючи пафосну Софію. як дійде до
того технологія, то її варто буде вийняти зі свого місця та перенести на якісь
живописні пагорби передмістя, щоби до неї їздили жалюгідні пафосні внтурішні
туристи, а на її місці навпроти Хайяту збудувати щось сучасне - чи то ТЦ, чи
просто суспільнокорисний паркінг  @igg{fbicon.smile} 

\begin{itemize} % {
%%%fbauth
%%%fbauth_name
\iusr{Іван Семесюк}
%%%fbauth_url
%%%fbauth_place
%%%fbauth_id
%%%fbauth_front
%%%fbauth_desc
%%%fbauth_www
%%%fbauth_pic
%%%fbauth_pic portrait
%%%fbauth_pic background
%%%fbauth_pic other
%%%fbauth_tags
%%%fbauth_pubs
%%%endfbauth
 
заскладний сарказм

%%%fbauth
%%%fbauth_name
\iusr{Olga Lashch}
%%%fbauth_url
%%%fbauth_place
%%%fbauth_id
%%%fbauth_front
%%%fbauth_desc
%%%fbauth_www
%%%fbauth_pic
%%%fbauth_pic portrait
%%%fbauth_pic background
%%%fbauth_pic other
%%%fbauth_tags
%%%fbauth_pubs
%%%endfbauth
 
\textbf{Alexey Teplinsky} Дідько, аж до ТЦ я вірила і кипіла!  @igg{fbicon.thumb.up.yellow}  @igg{fbicon.face.tears.of.joy} 

%%%fbauth
%%%fbauth_name
\iusr{Оксана Цюпа}
%%%fbauth_url
%%%fbauth_place
%%%fbauth_id
%%%fbauth_front
%%%fbauth_desc
%%%fbauth_www
%%%fbauth_pic
%%%fbauth_pic portrait
%%%fbauth_pic background
%%%fbauth_pic other
%%%fbauth_tags
%%%fbauth_pubs
%%%endfbauth
 
\textbf{Іван Семесюк} Але, для тих, хто в темі, зрозумілий  @igg{fbicon.frown} 

%%%fbauth
%%%fbauth_name
\iusr{Andrij Danylenko}
%%%fbauth_url
%%%fbauth_place
%%%fbauth_id
%%%fbauth_front
%%%fbauth_desc
%%%fbauth_www
%%%fbauth_pic
%%%fbauth_pic portrait
%%%fbauth_pic background
%%%fbauth_pic other
%%%fbauth_tags
%%%fbauth_pubs
%%%endfbauth
 
\textbf{Alexey Teplinsky} блискучий сарказЬм, побудований на антитезі!

%%%fbauth
%%%fbauth_name
\iusr{Andrij Danylenko}
%%%fbauth_url
%%%fbauth_place
%%%fbauth_id
%%%fbauth_front
%%%fbauth_desc
%%%fbauth_www
%%%fbauth_pic
%%%fbauth_pic portrait
%%%fbauth_pic background
%%%fbauth_pic other
%%%fbauth_tags
%%%fbauth_pubs
%%%endfbauth
 
\textbf{Olga Lashch} перефразовуючи з гумором заскорузлий мокшанський псевдореволюційний текстографічний лейтмотив: "...ой, браття й сестри, щось кипить мій розум розгніваний!"
\end{itemize} % }

%%%fbauth
%%%fbauth_name
\iusr{Kseniia Dorozhkina}
%%%fbauth_url
%%%fbauth_place
%%%fbauth_id
%%%fbauth_front
%%%fbauth_desc
%%%fbauth_www
%%%fbauth_pic
%%%fbauth_pic portrait
%%%fbauth_pic background
%%%fbauth_pic other
%%%fbauth_tags
%%%fbauth_pubs
%%%endfbauth
 

А в мене печатка, що ви сортуєте і оцінюєте клас міста за власними критеріями
наявності певних архітектурних ансамблей. Якщо їх нема, та й ще степ, чи ліси і
болота, то все пізда. Ваша вибірковість і ваше уявлення про власну країну і
себе фантазійні, і все частіше абсурдні. Прийміть вже реальні дані і з ними
співіснуйте. А свій арбалет чи які там у вас гномські приколи, залиште для
комік кону.

Ви розфокусовані на деталі, які незмінні і не хочете це визнавати. А ви маєте
почуватися комфортно на одинці із собою в будь якій точці країни. Але що мені
цікаво, так досліджувати як цей ваш фантазійний сумбур всі по різному
інтерпретують і головне активно підтримують, на рівні якихось деревлянських
рефлексів хоббітів. Ви в мене так і зазначаєтеся як сектор бентежних гномів і
мрійливих хоббітів. Яка готика, полісся то і є ваше все, енергія землі, річки й
болота, а ви з цим боретеся, із самим собою.


%%%fbauth
%%%fbauth_name
\iusr{Marina Daniluk-Yarmolaeva}
%%%fbauth_url
%%%fbauth_place
%%%fbauth_id
%%%fbauth_front
%%%fbauth_desc
%%%fbauth_www
%%%fbauth_pic
%%%fbauth_pic portrait
%%%fbauth_pic background
%%%fbauth_pic other
%%%fbauth_tags
%%%fbauth_pubs
%%%endfbauth
 

Про самовичавлювання. Прогнозую, що буде активно освоюватись Чернігівський
напрямок. Там симпатично, досить швидко до Києва і там завше жили люди до
різанини Другої світової і голодоморів

\begin{itemize} % {
\iusr{Катерина Ісаєнко}
 
\textbf{Marina Daniluk-Yarmolaeva} да, і купа пустих сіл, зокрема Козелеччина

\iusr{Marina Daniluk-Yarmolaeva}
 
\textbf{Катерина Ісаєнко} а Ніжин це моя любов

\iusr{Катерина Ісаєнко}
 
\textbf{Marina Daniluk-Yarmolaeva}  @igg{fbicon.heart.blue}
@igg{fbicon.heart.yellow} це да, є таке, і моя @igg{fbicon.heart.suit}
@igg{fbicon.rocket} 
\end{itemize} % }

\iusr{Tamara Tkachenko}

Цікаво що в тому середовищі ,яке знищує Київ , головне мати квартиру з
краєвидом на-Софію, Андієвську церкву, Лавру-тому вони туляться своїми *нео* до
них ,як гієни. Свідомість десь на рівні - ну нащо автентичність -вона брудна,
потребує збереження і о який жах (!!!)-розумового навантаження ,якщо можна
побудувати свій Діснейленд в окремому селі за високим парканом. Поняття про
місто не існує взагалі .До речі щоб мати краєвид на Андріївську церкву Людмила
Кучма облаштувала окреме вікно і для цього знесли будинок 19 ст.

\begin{itemize} % {
\iusr{Людмила Горделадзе}
\textbf{Tamara Tkachenko} Влучно
\end{itemize} % }

\iusr{Yaroslav Lazor}
Готика - це католицизм)
А Київ - місто православне)

\begin{itemize} % {
\iusr{Іван Семесюк}
\textbf{Yaroslav Lazor} з іншого боку мене це не дуже обходить, бо я не відчуваю себе православним, в костелах мені найкраще на серці


\iusr{Іван Семесюк}
\textbf{Yaroslav Lazor} ви так говорите наче бароко вигадали православні


\iusr{Іван Семесюк}
\textbf{Yaroslav Lazor} православі освоїли, на щастя, ієзуїтську освіту.
\end{itemize} % }

\iusr{Daria Ozerna}
є, біля Софії чи під нею басейн спорткомплексу Софійський

\begin{itemize} % {
\iusr{Віталій Корніяка}
\textbf{Daria Ozerna} там ще й у комплексі з підземним паркінгом, здається.
\end{itemize} % }

\iusr{Illya Piven}

Прикро, та готика не є рідною для України попри всі лютеранські кірхи,
побудовані німецькими колоністами. Якщо якийсь Регенсбург без готичної
архітектури втрачає сенс, то типова Полтава цілком може продовжувати існування.

\begin{itemize} % {
\iusr{Іван Семесюк}
Архітектурно Київ зараз відтворює азійську сучасну забудову, як в місті Астана. В якомусь смислі столиця глибинної Орди

\iusr{Illya Piven}
Але безперечно, місто треба рятувати. Там десь в Ісландії був якийсь чи то комік, чи то рокер, який став мером міста і щось там десь дивовижно подолав. Його успіх необхідно повторити

\iusr{Yaroslav Lazor}
\textbf{Ілля Півень} ну чому ж? Львів до великої пожежі був цілком собі готичним. Катедра будувалась як типово готична (це вже пізніше йому надали барокових форм)

\iusr{Illya Piven}
\textbf{Yaroslav Lazor} Є така думка, що кожна держава у своїх містах намагається відтворити столицю. Римляни будували скрізь амфітеатри і терми, московити - тюрми і обеліски, а в Штатах ціла купа Бродвеїв (знаю, приклад не дуже). Так от в цьому ключі, Львів зовсім не схожий на Київ. Тому що східніше від Дніпра, то більше Львів сприймається як закордон.

\iusr{Yaroslav Lazor}
\textbf{Ілля Півень} Львів і відтворював столицю - Відень))
А зараз якимись елементами відтворює Київ (хоча, більше протиставляє. Київ сприймається, як конкурент).

\iusr{Illya Piven}
\textbf{Yaroslav Lazor} Ну от, а в Відні наші не сиділи. Принаймні так міцно, як хотілося б.

\iusr{Svitlana Chub-Krywuzka}
\textbf{Yaroslav Lazor} Львів відтворював Відень, але найсильніше враження ''я у Львові'' створює, що цікаво, Зальцбург.
От прямо викапаний Львів, за атмосферою, за стилем. Львів, тільки здорової людини. Ідеально чистий Львів. Зразок, до якого Львову треба прагнути.
Відень схожий, але в ньому забагато пафосу а-ля Париж.
\end{itemize} % }

\iusr{Anatoly Vexlarsky}
забудовник діє зухвало, бо безкарно.
усе почалося з Софії - хілтон у охоронній зоні. з того моменту як снігова куля росте і набирає швидкості

\iusr{Яр Кордон}

Це усюди таке. Після того, як Львів виґлянцували і зробили з нього новорічну
ялинку для туристів, один старий поляк сказав щось на кшталт "бовдури ви, бо
колись до Львова їздили якраз за отією екзистенційною патріархальністю серед
старовини, а тепер тут таке ж як і всюди".

\begin{itemize} % {
\iusr{Yurco AciMov}
\textbf{Jar Kordon} дуже хитро жити з ґлянцем, а їхати за старовиною з екзистенцію до нас  @igg{fbicon.wink} 
\end{itemize} % }

\iusr{Dmytro Lazutkin}
Смисл Кривого рогу - під землею. Там і руда, і трамвай як метро. І кар’єрна безодня як антипод кар’єрного зростання як ідеї.

\iusr{Оксана Стаценко}

Вичавилася з Києва у передмістя в 2009 році (в Бучу). Тоді чоловік доїжджав від
будинку до офісу в Києві (біля цирку) за 15 хв. Зараз - більше години. І ці
роки передмістя бурхливо/злочинно забудовується. Ірпінь, Буча, Блиставиця,
Гостомель, Ворзель - ліси вирубано відсотків на 70, дороги перевантажені, за
парковки конфлікти, в садок не потрапиш, в класах по 40 дітей, на велосипеді не
проїдеш, комунікації перевантажені і регулярно дають збої. І зараз
спостерігаються тренд, як з передмістя жителі починають мігрувати ще далі. А
щодо пустого Полісся - буваю у друзів в Малинському районі, в селі з порожніми
хатами (більше половини покинуті). То які там історії буття місцевих - трилери
можна знімати.

\begin{itemize} % {
\iusr{Володимир Ковальчук}
\textbf{Oksana Statsenko} Буча - це фактично той же Київ))
\end{itemize} % }

\iusr{Іван Лєкарєв}
Просто на сто відсотків згоден з автором даного посту, сучасні люде не шарять в навколишньому просторі, основна ціль життя - спродати все що заробив за життя щоб купити дітям кватирку і уєбанській багатоповерхівці десь під Глевахою чи Чабанами...

\iusr{Тетяна Цибко}
У мене такі самі відчуття про Львів. Тими ж словами можна ситуацію описати. Він теж вже лускає від нових житлових кварталів , втулених у кожну шпарину межи старою забудовою. І вузькі вулички, і автомобільний колапс. А розвиток міста заключається у відкритті нових «криївок». У Львові стало некомфортно.(((

\iusr{Nemo Skal}

Я випадково вчора прочитав, що в Хорватії є таке собі місто Хум.

Воно визнане найменшим містом в світі.

Там живе лише декілька десятків людей, воно має лише дві вулиці, в довжину десь
150м, в ширину десь 50м.

І при цьому там 2 (дві) церкви.

Ось стаття, яка мені вчора попалася на очі:

\url{https://www.wandererscompass.com/destinations/hum-croatia/}

А ось стаття з Вікіпедії:

\url{uk.wikipedia.org/wiki/Хум}

Мені здається, що існування в Європі такого міста має пряме відношення (хоча й
неочевидне) до пожежі в київському костьолі, до самого Києва в його
теперішньому стані, та до України загалом.

\iusr{Mikolaj Szyszkin}
По Польщі чітко видно якраз цю залежність

\iusr{Pavlo Volodymyrovych}
\textbf{Іван Семесюк} ти вчергове згадуєш нещасний Кривий Ріг )
Але, думаю, якби ти його відвідав то деякі аспекти минулого трохи б змінили твоє ставлення.

\begin{itemize} % {
\iusr{Іван Семесюк}
\textbf{Pavlo Volodymyrovych} я там бував, тому й згадую, інакше не згадував би

\iusr{Kermit Noshimuri}
Я там народився та виріс. Підтримую Семесюка. Майбутне Кривого Рогу - парк індустріальних розваг накшталт Детройта чи Атлантік Сіті. Лишилося лише перебудувати мою рідну шахту "Родіна" в казіно та дати їй поетичну назву типу ТрампРодінаТауер.
\end{itemize} % }

\iusr{Роберт Боб Тибурак}
для мене як для старого католика пожежа в будь-якому костелі є нехорошим символізмом. але пост в точку

\iusr{Світлана Філімоненко}

Я не знаюся на мистецтві, але Ви точно вказали причину, з якої мені не хочеться
відвідувати столицю: Київ щоразу втрачає частку своєї душі. І це так прикро й
гірко( Вже боюся за Софію, за Видубицький монастир, за Троїцьку надбрамну
церкву та інші місця, де відчуваються смисли...


\iusr{Марина Черевань}
Найменш заселений простір - Олешківські піски. Повний комфорт(((

\iusr{Pavlo Mosiienko}

Субурбанізація це позитивний процес. Жити і утримувати маєток в передмісті не
дешево! Це добра ознака коли заможні мешканці переїздять за місто))

\begin{itemize} % {
\iusr{Александр Куриленко}
\textbf{Pavlo Mosiienko} Це негативний процес з точки зору навантаження на дорогах. Тупо всі на жоповозки пересідають

\iusr{Pavlo Mosiienko}
\textbf{Олександр Куриленко} те що всі сідають на жоповозки це теж хороша тенденція) зубожіння і геноцид відміняється))) не той рівень щоб порівнювати з явищем субурбанізації
\end{itemize} % }

\iusr{Pavlo Salyga}

Не можна зараз готику будувати. Вийде пшик. Ось тут глянь, як це у Городецького
вийшло \url{https://m.youtube.com/watch?v=_6vJYchHBBs&t=189s}


\iusr{Andrii Melnyk}
дотичне
Нещодавно знову в судовому порядку скасовано ДБН Б.2.2-12:2019 «Планування і забудова територій» в частині вимог щодо висотності у селах, СМТ та маленьких містах.
Кажуть, київські забудовники, забудовники інших великих міст всіляко боролися за такі обмеження...

\iusr{Volodymyr Tarnavskyi}
Це, до речі, ще одна наша схожість із запоребриком, коли є столиця, тільки столиця і нічого окрім столиці.

\iusr{Victor Brylew}
Все що вище хатинки біля вишневого садка не відповідає українській айдентиці і має бути знесено. Бо набудують а потім усілякі зайди кажуть: "майже як Польща" чи "в нас в Відні таке ж". Все знести і засаджати вишневими садками.

\iusr{Олег Котенко}
Давно вже вичавлюються. Вишневе, Крюківщина, Гатне - суцільна стройка, ЖК будуються десятками + приватні будинки. Одеською трасою доїхати до Глевахи - це тупо квест.

\iusr{Артем Комисаренко}
Взагалі-то, проєкт побудувати підземний паркінг на Михайлівській площі від Кличка був, але поки що не наважуються, бо ж ясно, що це срака всьому культурному шару.

\begin{itemize} % {

\iusr{Іван Семесюк}
\textbf{Artem Komisarenko} жесть

\iusr{Максим Горбань}
\textbf{Артем Комісаренко} від Кличка? ))) Там ідея збудувати парковку вітає десь з 2010 року від різного штибу "забудовників"

\iusr{Roman Melnyk}
Зі сторони провулку Рильського у дворах в сторону Софії (до межі огорожі) щось таки колись суркісята нарили (фітнес-центр і паркінг)....скандал був років 15 назад

\iusr{Артем Комисаренко}
\textbf{Maxim Gorban} Кличко її виголошував. А що там раніше з Космосом було, то вже не цікаво, бо він полетів.

\iusr{Оксана Цюпа}
\textbf{Артем Комісаренко} а чо, оно на подолі скільки битва йде проти забудови ТЦ, і митців там не видно, бо усім хочеться себе з готикою асоціювати, хай і з нео, а не з 9-10 ст


\iusr{Іван Семесюк}
\textbf{Оксана Цюпа} думаю може укласти список чого ж корисного для суспільства я за життя наробив, бо іноді трапляється "сидить, ніхуя не робе, тільки пиздить". Це точно не справедливо

\iusr{Tamara Tkachenko}
\textbf{Roman Melnyk} Так вони все добудували і позалазили куди тільки дістались  @igg{fbicon.smile} 
\end{itemize} % }

\iusr{Оксана Цюпа}

Та, ладно, оно, на руїнах десятинної мацкаліки церкву збудували і ніхто з
митців не хмикнув навіть. Мабуть, це не їхня історична тяглість до европейської
цінностей. Сумно

\begin{itemize} % {
\iusr{Іван Семесюк}
\textbf{Оксана Цюпа} двоє митців, якщо згадаєте, ходили під тюремним строком за спробу підпалу тієї халупи. Обидва архітектори, витягали і робили розголос усією громадою. Забулися?


\iusr{Іван Семесюк}
\textbf{Оксана Цюпа} людина не може бути ефективною в усьому. Неможливо качати ситуацію щомиті на усіх напрямках.


\iusr{Іван Семесюк}

В цілому мені не подобається взятий вами, пані, тон. Як на ринку, їй бо, але ж я не торгуюся і виправдовуватися не маю, наче я щось вкрав з ятки.

\iusr{Оксана Цюпа}
\textbf{Іван Семесюк} ок, зараховано. Двойко

\iusr{Оксана Цюпа}
\textbf{Іван Семесюк} , а диви, оно на відновлення оргАну вже збирають усім миром. Чому така слабка реакція на те, що нам мало б бути важливішим в плані культури та ідентичності


\iusr{Іван Семесюк}
\textbf{Оксана Цюпа} у кого саме слабка?

\iusr{Оксана Цюпа}
\textbf{Іван Семесюк} у тих, хто, для прикладу , на орган гроші дає. У частини активного суспільства, мабуть. Як ти бачиш, чому одні події визивають такі реакції швидкі, а от інші події , практично, не зачіпають суспільство. Я пригадую, яка була хвиля в укрсегменті фб щодо пожежі в соборі Паризької богоматері , а своє хвилює менше, чому ? ( Дурацькі ці питання задовбують )

\iusr{Марина Литвинович}
\textbf{Оксана Цюпа} Оксано, реакція була, коли будувалося оте одоробло, просто у той час не був такий розповсюджений фейсбук, мабуть. А потім, більщість киян усе ж - таки є прихожанами православної церкви, а з них більшість і досі ходять до МП. Більш догматичної, застійної структури і не знайдеш зараз. Воно сидить павуком на свідомости своїх прихожан, не дозволяючи їм думати самостійно. Тому вони 1. Не здатні відрізняти істині цінности від гівняних новоділів, 2. Не вважають цю споруду чимось небезпечним, а навпаки впевнені у її доречності, ще й захищатимуть своє одоробло, спробуй хтось проти неї виступити. А костьол більше все ж сприймається не як культова, а як культурна точка, томуі кинулися гроші збирати...

\end{itemize} % }

\iusr{Олександр Бахно}

Це теорія про те, що Україна православна і тому не може відірватись від
Константинополя як частини Азії? Чи про латинку, прийняття якої зробить нас
ближче до європейської цивілізації? Заплутали зовсім...

\begin{itemize} % {
\iusr{Іван Семесюк}
\textbf{Олександр Бахно} На цю мить Україна - це переважно протестантська
країна, тому що дієвих, організованих, активних християнських громад, отже й
реальних вірян, саме реальних членів церкви як громади у практичному вимірі,
значно більше саме баптистів та іншх протестантських конгрегацій ніж
православних. Це було дуже помітно коли в Києві відбувся з'їзд протестантських
церков України кілька років тому. Це була така люта кількість людей в центрі
Києва, що жоден православний хресний хід ніколи навіть й близько не наближався
до неї. Центр Києва був в транспортному колапсі, особливо біля Палацу Спорта.
Також дійсно активними саме у громадському смислі є греко-католики, себто
християни приналежні Риму, ба більше національні життя українських діаспор
закордоном значною мірою відбуваються саме завдяки унійним осередкам, а не
православним. Тому Україна - це скоріше країна західної традиції, якщо говорити
про реальну живу релігійну практику, ніж східної, яка часто і давно виглядає
музейною формальністю і язичницькими Гробками.

\iusr{Виталий Проскура}
\textbf{Іван Семесюк} Але ж православні більш є...ті. І ця риса компенсує багато чтого

\iusr{Julia Halanyuk}
Шеб календар перевести і греко католиків реформувати на кшталт римо- і можна жити
\end{itemize} % }

\iusr{Yurco AciMov}
Синдром мігранта плюс старість)) - це не про пожежу.
Як бабусі люблять стару свою совєцьку "архітектуру" коли машин було мало і магазинів-кіосків і дітки по пів-ночі не сиділи на штучних футбольних шкільних майданчиках з колонками-мобілками))
І одягалися скромніше і дітей в єжових рукавіцах дєржалі - нє баловалі...

\begin{itemize} % {
\iusr{Іван Семесюк}
\textbf{Yurco AciMov} цікаве відзеркалення
\end{itemize} % }

\iusr{Alexey Teplinsky}

без жартів, козир Києва це модерн! який теж прагнуть спаплюжити, але його
захищати правильніше, хоча би тому, що його цінність у світовому вимірі
непорівнянно вища за ту ж Софію чи простигосподи "готичний костьол" 19/20-го
ст.

\begin{itemize} % {
\iusr{Іван Семесюк}
\textbf{Alexey Teplinsky} Софія - архітектурна і взагалі культурна перлина світового рівня і значення. Кажу не як турист, а як людина з профільною освітою. Тобто ви помиляєтеся.

\iusr{Kyjeslav Borščahiveć}
\textbf{Alexey Teplinsky} справжній козир Києва - це його дика мішанина стилів. Це і фішка, і покарання міста

\iusr{Ievgen Gerasymenko}
Скоріше уєбан-еклектика

\iusr{Alla Tkalina}
Alexey Teplinsky Козир Києва - це засклений балкон, і що характерно саме йому в будь-якому районі Києва нічого не загрожує

\iusr{Oksana Pavlovska}
\textbf{Alexey Teplinsky} модерн, про який ви кажете як про козир Києва, давно під загрозою знищення, а основний козир Києва - це хтонічного вигляду ЖК, які ростуть та множаться як дурні гриби після дощу.
\end{itemize} % }

\iusr{Alex A Sinyavsky}
От нахіба таке писати? Зараз всі у Полісся попруться. )

\iusr{Костянтин Ворона}
Що казати, коли у людей досі готичне мислення? Від нього і страждають готичні будівлі.
Про Полісся не згоден. За 50 км від столиці стоять завод кока-коли і елеватори, а болота та ліси вже давно замінені полями соняшника, кукурудзи та сої

\begin{itemize} % {
\iusr{Костя Смирнов}
\textbf{Kostyantyn Vorona} полісся величезне є і дикі місця. А ще полісся це, да, болота і ліси ні на болотах, ні в лісах соняхи не ростуть. Але правда. Був я років 15 тому біля одного невеличкого озерця недалеко від овруцького кряжу. Красота дика природа з дикими людьми.в лісі біля села кладовище.... Навкруги плоско, а кладовище горбочок наче. Ще подумав "невже це якійсь давній курган". Аж тут в ютюбчику надибав Залізняка, так він каже, що дійсно там такі кладовища на давніх курганах. Вирішив ще раз туди з'їздити... Попередньо, щоб згадати маршрут, поліз у Гугл карти. Фіг там! З того озерця піщаний кар'єр зробили. І кладовище те побачив, але воно тепер чомусь не в лісі, а голий пагорб.
\end{itemize} % }

\iusr{Natali Blashchyshena}

Щодо Кривого Рогу так...у мене тут серце не тьохкає навіть коли ходжу по місцям
дитинства... У нас в КР були гарні будинки, якась історія, але замість цього
всього великі ями- кар'єри, зона обрушення. Все цікаве ще при СРСР перерили.

Архітектури цікавої практично немає. А те що будується банальні коробки, і такі
ж банальні церкви. Житлового будівництва немає зовсім, воно і ясно - місто по
трохи вмирає.


\iusr{Artem Kovtun}

Мені здається наш Kyiv town перетворюється на Kyiv city, а всі сіті трохи менш
національні, трохи менш затишні, трохи менш камерні. Поки ще Київ вночі спить,
а коли перевалить за 5 млн постійного населення, вже спати не буде.


\iusr{Irina Gladuniak}
будь-яка талановита архітектура є печаткою цивілізації))

\iusr{Павло Громов}
Багатопланове, суто українське спостереження. Нам не притаманне "міське" життя, архітектура... тільки - воля). Мабуть, єдине українське місто - Полтава. А, взагалі, зараз вважаю треба перенести столицю у Харків, загроза захвату Бєлгорода, та, й він динамічніший, й в сенсі просування, не буду тут) цивілізації

\iusr{Alisa Lisniak}
В Дніпрі ситуація така ж сама, історичного славного Катеринославу лишається все менше, і все навкруги перетворюється та урбаністично-безлике ніщо, не маюче коріння та розуміння, звідки воно і нащо треба(...

\iusr{Мария Норенко}
Не люблю надавати символізму трагедіям і не надаю у цьому випадку теж. Але сучасна "забудова" Києва нервує дико.

\iusr{Alla Tkalina}
такі проекти були і є - в зоні Софійського заповідника ці варвари будували фітнесцентр з басейном, і до речі, невідомо, чим це скінчилось. \url{https://www.radiosvoboda.org/a/894845.html}

\iusr{Евгений Разуваев}

а що до бароко? і так, це ж дійсно просто про готику та цивілізацію. тільки
тепер зрозумів, чому барса на викликає ніяких майже почуттів, як місто

\begin{itemize} % {
\iusr{Ксения Птица}
\textbf{Евгений Разуваев} а Саграда прям соцреализм, да? @igg{fbicon.face.tears.of.joy} 

\iusr{Евгений Разуваев}
Ксения Птица ну, така собі модернова готіка, ага  @igg{fbicon.laugh.rolling.floor}  але ж, це скоріше виключення з суцільно модернового міста

\iusr{Ксения Птица}
\textbf{Евгений Разуваев} Барса прекрасна) просто не твое, наверное)

\iusr{Евгений Разуваев}
\textbf{Ксения Птица} себя в городе и я прекрасно ощущаю, но вот «город в себе» уж точно не мой.
\end{itemize} % }

\iusr{Олександр Невисевич}
Полісся дійсно класне, в лісах трапляються закинуті/затоплені гранітні кар’єри, милота)

\iusr{Александр Куриленко}
От прям відчував, що ви не любите архітектуру модернізму. Якщо ми хочемо колонізувати схід, то нам треба зробити естетику панельних районів українською, бо в деяких містах нічого більше нема!

\iusr{Геннадий Цупин}
не готика, але наш український стиль у архітектурі початку ХХ сторіччя. теж непогано \url{https://www.youtube.com/watch?v=6VOeXHc7IsQ}

\iusr{Oleksander Galushka}
Тоді міста були для людей, зараз для рабойобів, яким все одно де жити, аби тільки можна було на жоповозці дістатисть до роботи і щоб сільпо було недалеко

\begin{itemize} % {
\iusr{Yurco AciMov}
\textbf{Oleksander Galushka} хрущовки і брєжнєвкі?

\iusr{Oleksander Galushka}
\textbf{Yurco AciMov} новобудови

\iusr{Александр Куриленко}
\textbf{Oleksander Galushka} Ні пандусів, ні ліфтів, ні освітлення. Оце так для людей...
\end{itemize} % }

\iusr{Іван Ткаченко}
У Києві є Дніпро. Звісно, водосховища гальмують, але неспроможні до решти зафільтрувати голоси.

\iusr{Іхь Бін Я-я}
Такі ж почуття... Я навіть будинок собі будую з високим гострим дахом.

\iusr{Володимир Ковальчук}
Святе місце пустим не буває. Впевнений, що зовсім скоро порожні території України заселять прийшлі люди. Суть багатьох "реформ" - якраз для цього.

\end{itemize} % }
