% vim: keymap=russian-jcukenwin
%%beginhead 
 
%%file 25_10_2020.stz.news.ua.mrpl_city.1.zrazkovyj_teatr_grani
%%parent 25_10_2020
 
%%url https://mrpl.city/blogs/view/zrazkovij-teatr-grani
 
%%author_id demidko_olga.mariupol,news.ua.mrpl_city
%%date 
 
%%tags 
%%title Зразковий театр "Грані"
 
%%endhead 
 
\subsection{Зразковий театр \enquote{Грані}}
\label{sec:25_10_2020.stz.news.ua.mrpl_city.1.zrazkovyj_teatr_grani}
 
\Purl{https://mrpl.city/blogs/view/zrazkovij-teatr-grani}
\ifcmt
 author_begin
   author_id demidko_olga.mariupol,news.ua.mrpl_city
 author_end
\fi

Незважаючи на закінчення міського фестивалю \emph{\enquote{Маріуполь театральний – 2020}} ми
продовжуємо знайомитися з тими колективами цього заходу, які мають свою історію
і унікальні особливості, але відомі не всім маріупольцям. Один з таких
колективів – \emph{Зразковий театр \enquote{Грані}}, юні актори якого готові дивувати і
приємно вражати маріупольців своїми виставами та творчими задумами.

\ii{25_10_2020.stz.news.ua.mrpl_city.1.zrazkovyj_teatr_grani.pic.1}

Театр \enquote{Грані} розпочав свою діяльність у Міському палаці культури у 2007 році
на базі Народного театру-клубу \enquote{Діалог}. У зв'язку зі зміною репертуару та
вікового складу театр-клуб \enquote{Діалог} було реорганізовано у Зразковий театр
\enquote{Грані}. Заснувала і організувала колектив для подальшої роботи \emph{\textbf{Данілова Наталя
Петрівна}}, чудова педагогиня, яка ставила вистави українською мовою і зуміла
прищепити дітям любов до державної мови.

Головні цілі театру: \emph{прищепити любов учасникам колективу до театрального
мистецтва та розвинути акторську майстерність та естетичні смаки у дітей.}
Основна вікова категорія колективу – діти і підлітки від 6-ти до 16-ти років.
Наразі у колективі бере участь 20 юних акторів. На жаль, карантин вплинув на
колектив, але діти поступово повертаються до театру.

За 2016–2019 роки театр підготував і поставив чотири вистави: \emph{\enquote{Мишлі – Шишлі},
\enquote{Як Настенька трохи Марою не стала}, \enquote{Золоті стріли}, \enquote{День народження Вовка
або хиже Зайча}, новорічний мюзикл \enquote{Буратіно}, \enquote{Все може Дід Мороз}, \enquote{Новий рік
в Простоквашино}.}

Юні часники колективу постійно беруть участь у районних, міських святах,
фестивалях, театралізованих заходах та працюють на КСН району.

\ii{25_10_2020.stz.news.ua.mrpl_city.1.zrazkovyj_teatr_grani.pic.2}

На базі театру було проведено багато майстер-класів, один з них за темою:
\emph{\enquote{Клубний заклад як центр розвитку народної творчості та альтернативного
молодіжного мистецтва}}. За підготовку і участь у заході керівниця театру Наталя
Петрівна Данілова і театр відзначені Подякою Донецького обласного
навчально-методичного центру культури. Учасники колективу і самі беруть участь в
майстер-класах від викладачів зі сценічної мови чи режисерів, що допомагає
підвищити майстерність та підготовку акторів.  

З 2020 року режисеркою театру стала молода і енергійна \emph{\textbf{Захарова Марина
Едуардівна}}, завдяки якій театр почав модернізуватись та змінювати програму
навчання. За освітою Марина Едуардівна режисер, тому одразу вирішила
урізноманітнити репертуар. Зокрема, крім дитячих вистав  заплановано ставити і
дорослі спектаклі. Режисерка вважає, що дорослій групі (15 – 16 років) буде
цікавіше попрацювати із серйозною драматургією. Сьогодні актори вивчають нові
форми і напрями театрального мистецтва (працюють над створенням перфомансу та
гепенінгу). Цікаво, що діти почали вивчати і нові предмети. Сьогодні у них в
арсеналі такі серйозні дисципліни як: \emph{ритміка, вокал, пластика, сценічні рухи,
сценічна мова (українською та російською)}. Планів у колективу багато. Зокрема,
театр намагатиметься й надалі брати активну участь у міських, обласних та
всеукраїнських фестивалях. Водночас ведеться підготовка до створення
онлайн-театру, що зараз є, безперечно, затребуваним і актуальним. Також Марина
Едуардівна планує творчі колаборації – співпрацю з іншими театрами в нашому
місті. Так, найближчим часом можна буде побачити результати спільної праці
театру \enquote{Грані} та \emph{Народного театру пісні \enquote{Forte music}}. Марина Захарова
наголошує, що необхідно навчити дітей працювати з різними формами – і театр
ляльок, і буфонада, і класична драматургія – дозволять зрозуміти повною мірою,
що таке театральне мистецтво.

\ii{25_10_2020.stz.news.ua.mrpl_city.1.zrazkovyj_teatr_grani.pic.3}

В колективі Зразкового театру \enquote{Грані} панує тепла атмосфера, взаєморозуміння та
підтримка, тому маленькі та юні актори поспішають на репетиції і ставляться до
театру як до другого дому.

\ii{25_10_2020.stz.news.ua.mrpl_city.1.zrazkovyj_teatr_grani.pic.4}

А для того, щоб приєднатися до цього унікального і яскравого колективу
необхідно пройти онлайн-реєстрацію, прикріпити всі необхідні документи, після
чого колектив з радістю прийме ще одного учасника, який незабаром може стати
частиною їхньої дружньої та творчої сім'ї.

