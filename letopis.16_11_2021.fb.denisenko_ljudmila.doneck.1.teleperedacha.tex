% vim: keymap=russian-jcukenwin
%%beginhead 
 
%%file 16_11_2021.fb.denisenko_ljudmila.doneck.1.teleperedacha
%%parent 16_11_2021
 
%%url https://www.facebook.com/LED65/posts/4802673623116594
 
%%author_id denisenko_ljudmila.doneck
%%date 
 
%%tags dnr,donbass,doneck,medicina,televizor,tv,zdorovje,zhizn
%%title Последние съёмки передачи "Если хочешь, будь здоров! С Людмилой Денисенко"
 
%%endhead 
 
\subsection{Последние съёмки передачи \enquote{Если хочешь, будь здоров! С Людмилой Денисенко}}
\label{sec:16_11_2021.fb.denisenko_ljudmila.doneck.1.teleperedacha}
 
\Purl{https://www.facebook.com/LED65/posts/4802673623116594}
\ifcmt
 author_begin
   author_id denisenko_ljudmila.doneck
 author_end
\fi

У меня сегодня были последние съёмки передачи "Если хочешь, будь здоров! С
Людмилой Денисенко" на ТК Юнион. Завтра доснимаем два сюжета в любимом
ресторане семейных традиций "Оливье 80" и всё...

Передачи будут до января, потом каникулы до февраля, а там... 

\ifcmt
  ig https://scontent-frx5-1.xx.fbcdn.net/v/t39.30808-6/256910784_4802640989786524_8883955223820177295_n.jpg?_nc_cat=100&ccb=1-5&_nc_sid=730e14&_nc_ohc=orlqqwIcC_8AX80lFMZ&_nc_ht=scontent-frx5-1.xx&oh=6fba0fac59d3cda67395b2c496e48679&oe=61A50444
  @width 0.4
  %@wrap \parpic[r]
  @wrap \InsertBoxR{0}
\fi

Я ухожу в творческий отпуск. И буду думать, хочу ли я вернуться. С одной
стороны, я просто наслаждаюсь общением с нашими донецкими докторами, умными,
харизматичными, красивыми, талантливыми! Приглашаю и наших корифеев медицины, и
молодое наше будущее. Чтобы Донецк не чувствовал себя обделённым, понимал, что
у нас есть перспектива в медицине. 

Я закончила наш Донецкий государственный медицинский институт (сейчас
Национальный университет) имени М. Горького в 1988 году. С красным дипломом.
Гордилась и горжусь быть его выпускницей, а теперь ещё и горжусь своими
коллегами - выпускниками нашей родной Alma Mater. И хочу, чтобы и дончане
гордились ими, и знали, что есть в Донецке, кому доверить своё здоровье. 

С другой стороны, я устала. Не стану озвучивать причины усталости, не хочу
раскрывать "кухню" телевидения... Но...

Мыслей и стремлений много! Меня вдохновляет поддержка коллег, принимавших
участие в моей передаче (напомню, авторской, темы и сценарии придумываю сама,
пытаюсь вникнуть в каждую тему глубоко, чтобы разговаривать с профессионалами
на одном языке)  и тех, кого только мечтаю "заполучить" в спикеры, отзывы моих
друзей и зрителей дают силы. 

Но сомнений масса...

Продолжать или хватит? Есть мысли сделать ток-шоу, а не просто беседу со
спецами, хочется какого-то "движа"...

Вот и остаются вопросы: 

Нужно ли это мне и моему любимому городу? Сформулируйте мне хотя бы пару
причин, чтобы продолжать. Деньги (это, реально, смешно) и популярность (я уже
не в том возрасте, что мне это нужно) не предлагать!  @igg{fbicon.wink} 

\ii{16_11_2021.fb.denisenko_ljudmila.doneck.1.teleperedacha.cmt}
