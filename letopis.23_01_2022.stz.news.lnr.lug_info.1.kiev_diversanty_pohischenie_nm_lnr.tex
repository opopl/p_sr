% vim: keymap=russian-jcukenwin
%%beginhead 
 
%%file 23_01_2022.stz.news.lnr.lug_info.1.kiev_diversanty_pohischenie_nm_lnr
%%parent 23_01_2022
 
%%url https://lug-info.com/news/kievskie-diversanty-pohitili-voennosluzhashego-narodnoj-milicii
 
%%author_id news.lnr.lug_info
%%date 
 
%%tags diversia,donbass,kiev,lnr,nm_lnr,pohischenie,ukraina,vojna
%%title Киевские диверсанты похитили военнослужащего Народной милиции
 
%%endhead 
 
\subsection{Киевские диверсанты похитили военнослужащего Народной милиции}
\label{sec:23_01_2022.stz.news.lnr.lug_info.1.kiev_diversanty_pohischenie_nm_lnr}
 
\Purl{https://lug-info.com/news/kievskie-diversanty-pohitili-voennosluzhashego-narodnoj-milicii}
\ifcmt
 author_begin
   author_id news.lnr.lug_info
 author_end
\fi

Киевские диверсанты похитили военнослужащего Народной милиции. Об этом сообщил
официальный представитель оборонного ведомства Республики Иван Филипоненко.

\enquote{22 января в ходе патрулирования участка местности в районе Светлодарской дуги
диверсионной группой ВСУ был похищен военнослужащий Народной милиции ЛНР. В
ходе разбирательства установлено, что военнослужащий передал по радиосвязи о
подозрительной активности в лесополосе и не дожидаясь подкрепления осуществил
проверку местности, чем, несомненно, подставил себя под удар, однако, вероятно,
предотвратил дальнейшее продвижение диверсантов противника в глубину
Республики}, - сказал он.

\ii{23_01_2022.stz.news.lnr.lug_info.1.kiev_diversanty_pohischenie_nm_lnr.pic.1}

Напомним, в плену киевских силовиков уже более 100 суток находится похищенный
украинскими диверсантами наблюдатель представительства ЛНР в Совместном центре
по контролю и координации режима прекращения огня (СЦКК) Андрей Косяк.

Диверсионно-разведывательная группа ВСУ захватила наблюдателя представительства
ЛНР в СЦКК утром 13 октября 2021 года в зоне разведения сил и средств у
Золотого. После того как Киев не выполнил требования Республики о немедленном
возвращении наблюдателя, глава ЛНР Леонид Пасечник заявил о том, что
продолжение диалога с Киевом в рамках Минского формата до освобождения Косяка
является бессмысленным. Представительство ЛНР в СЦКК до освобождения офицера
прекратило контакты с украинской стороной и ограничило ряд маршрутов
передвижения наблюдателей СММ.

27 декабря 2021 года диверсионно-разведывательная группа Сил специальных
операций ВСУ захватила военнослужащего Народной милиции ЛНР. Украинские
силовики и СМИ распространяли ложную информацию о том, что он якобы
самостоятельно перешел на сторону Киева.

Власти Украины начали силовую операцию против Донбасса в апреле 2014 года.
Урегулирование конфликта базируется на Комплексе мер по выполнению Минских
соглашений, подписанном 12 февраля 2015 года в белорусской столице участниками
Контактной группы и согласованном с главами стран - участниц \enquote{нормандской
четверки} (Россия, Германия, Франция и Украина). Обеспечить выполнение
положений документа стороны конфликта призвал Совет безопасности ООН, который
одобрил документ резолюцией № 2202 от 17 февраля 2015 года.

Комплекс мер предусматривает прекращение огня, отвод тяжелых вооружений от
линии соприкосновения, начало диалога о восстановлении социально-экономических
связей Киева и Донбасса, а также реформу конституции Украины с целью
децентрализации и закрепления \enquote{особого статуса отдельных районов Донецкой и
Луганской областей}.
