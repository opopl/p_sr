% vim: keymap=russian-jcukenwin
%%beginhead 
 
%%file 10_08_2020.stz.news.ua.mrpl_city.1.anatolij_lomakin_misto_pochyn_dyt_majdanchyk
%%parent 10_08_2020
 
%%url https://mrpl.city/blogs/view/mariupolets-anatolij-lomakin-misto-pochinaetsya-z-dityachogo-majdanchika
 
%%author_id demidko_olga.mariupol,news.ua.mrpl_city
%%date 
 
%%tags 
%%title Маріуполець Анатолій Ломакін: "Місто починається з дитячого майданчика"
 
%%endhead 
 
\subsection{Маріуполець Анатолій Ломакін: \enquote{Місто починається з дитячого майданчика}}
\label{sec:10_08_2020.stz.news.ua.mrpl_city.1.anatolij_lomakin_misto_pochyn_dyt_majdanchyk}
 
\Purl{https://mrpl.city/blogs/view/mariupolets-anatolij-lomakin-misto-pochinaetsya-z-dityachogo-majdanchika}
\ifcmt
 author_begin
   author_id demidko_olga.mariupol,news.ua.mrpl_city
 author_end
\fi

\ii{10_08_2020.stz.news.ua.mrpl_city.1.anatolij_lomakin_misto_pochyn_dyt_majdanchyk.pic.1}

Багато містян знають \emph{Анатолія Михайловича Ломакіна} як автора публікацій,
присвячених історії маріупольського колективу сучасного бального танцю. Але
діяльність маріупольця охоплює зовсім різні сфери і основна тема його статей,
пов'язана з дітьми та дитинством. Чоловік наголошує, що все в нашому житті
починається саме з дитинства (характер, уподобання, моральні переконання), тому
діти потребують більшої уваги від дорослих, ніж вони її отримують.

Сьогодні Анатолій Михайлович працює монтувальником сцени у ПК \enquote{Молодіжний}.
Водночас готує блоги на сайт MRPL.CITY і продовжує доглядати за дитячим
майданчиком, який був створений саме завдяки ініціативі та зусиллям нашого
героя.

Анатолій Михайлович народився у селищі Гугліно і вчився в 43 школі, де вчитель
російської мови та літератури \emph{Олександр Васильович Шабанець} зумів прищепити
хлопцю любов до книг. У школі займався стінгазетами: виступав і редактором, і
журналістом. З дитинства хотів займатися письменницькою діяльністю, але життя
диктувало свої правила, адже батьки наполягали на отриманні технічної
спеціальності.

Перша публікація Анатолія була надрукована в інститутській багатотиражці
\enquote{Радіосигнал} Таганрозького радіотехнічного інституту, де він навчався  з 1968
року.

Після інституту чоловік не тільки одружується, але й знаходить для себе нове
хобі – бальні танці та бере активну участь у створенні колективу сучасного
бального танцю \emph{\textbf{\enquote{Феєрія}}}. Незабаром Анатолій починає публікувати статті про
\enquote{Феєрію} в \enquote{Приазовському робочому}.

24 роки (з 1974 року) пропрацював на \enquote{Азовмаші} інженером-конструктором (від
рядового до провідного інженера бюро).

\ii{10_08_2020.stz.news.ua.mrpl_city.1.anatolij_lomakin_misto_pochyn_dyt_majdanchyk.pic.2}

У 1989 році Анатолія відправляють у відрядження для надання допомоги потерпілим
від землетрусу, що стався у Вірменії. Він відправився до вірменського міста
Ванадзор (до 1992 року – Кіровокан). Вперше в житті чоловік сам попросив
відправити його у відрядження, йому потрібен був час для переосмислення
власного життя. Цього ж року його сім'я розпадається. Захоплення бальними
танцями розвиває перша дружина Анатолія, Наталія Ломакіна, за допомогою дітей
Тетяни і Георгія. Згодом захоплення створює \href{https://mrpl.city/blogs/view/feeriya-dans-dinastiya-tantsorov}{танцювальну династію}.%
\footnote{\enquote{Феерия-данс}: мариупольская династия танцоров, Анатолий Ломакин, mrpl.city, 25.11.2018, \par%
\url{https://mrpl.city/blogs/view/feeriya-dans-dinastiya-tantsorov}
}

Писати Анатолій Михайлович продовжував постійно. 1990-ті роки пов'язані з
публікаціями про туристичні зльоти в багатотиражці \enquote{Жданівський
машинобудівник}.

Переломним для маріупольця став 1995 рік, коли він приймає хрещення і починає
ходити до баптистської церкви, де згодом стає проповідником. Підготовка
проповідей сприяла розвитку літературних здібностей Анатолія  і дохідливості
викладу думки.

Запам'яталася маріупольцю його участь в міжнародному русі скаутів і поїздка до
міжнародного табору \enquote{Єврокемп-2003} до Фінляндії, де чоловік отримав купу
незабутніх вражень.

Несподівана смерть зятя додала проблем з підтримки багатодітної сестри Олени і
її сімох малолітніх дітей. Поява в новій сім'ї дітей Олексія та Сергія, а потім
і онуків привела до того, що проблема дитячого дозвілля та дитячих майданчиків
завжди була в полі зору Анатолія.

У 2011 році чоловік починає брати активну участь у благоустрої пустиря поруч з
церквою \enquote{Благодать} та отримує допомогу від церкви і керівництва. КСН
\enquote{Машинобудівник}. Згодом дитячий майданчик перетворюється в Місто Дитинства, а
потім в \emph{\textbf{Острівець Дитинства \enquote{Вітер Надії}}} на 26 кварталі. На дитячому
майданчику Анатолій Михайлович не тільки підходив до роботи творчо і намагався
робити все своїми руками, але й залучав помічників членів церкви, батьків і
дітей, КСН та депутатів. З підручних засобів зробив багато розважальних
об'єктів (вантажівка, трактор, катер тощо) для малечі. Він розуміє, що місто
для маленьких маріупольців починається з дитячого майданчика, розташованого
поруч з будинком, тому радіє, що зараз ситуація змінилася на краще і для дітей
робиться набагато більше. Важливо щоб дитина розуміла, що у своєму рідному
місті їй безпечно, цікаво і комфортно. А у своєму \href{https://mrpl.city/blogs/view/moe-i-nashe-na-detskoj-ploshhadke}{%
подвір'ї діти повинні знаходити все для свого дозвілля та всебічного розвитку}.%
\footnote{МОЕ и НАШЕ на детской площадке, Анатолий Ломакин, mrpl.city, 30.03.2018, \par%
\url{https://mrpl.city/blogs/view/moe-i-nashe-na-detskoj-ploshhadke}
}

На прохання секретаря КСН \enquote{Машинобудівник} Анатолій Михайлович почав писати
історію створення дитячого майданчика, яка була видана у 2015 році друкарнею
\enquote{Приазовського робочого} за клопотанням Миколи Токарського в в трьох
примірниках під назвою \emph{\enquote{Подаруйте своїй дитині Острівець Дитинства}}. Цього ж
року Ломакін взяв участь у Всеукраїнському конкурсі народних ідей \enquote{Чудовий двір
– чудове місто} і переміг в номінації \enquote{Зелений парк-продовження моєї вулиці}.

\ii{10_08_2020.stz.news.ua.mrpl_city.1.anatolij_lomakin_misto_pochyn_dyt_majdanchyk.pic.3}

Після весілля старшого сина Георгія і появи онучки Олі стосунки з першою сім'єю
покращилися настільки, що з боку можна подумати, що це одна велика родина.
Сьогодні у маріупольця троє онуків і тема дитинства для нього залишається
основною, оскільки проведення часу з дітьми він вважає і важливим, і корисним.
Свою письменницьку діяльність Анатолій теж спрямовує на створення публікацій,
присвячених темі дітей. Зокрема, завдяки співпраці з \emph{Сайтом батьків Маріуполя},
регулярно друкується в розділі \enquote{Творчість з дітьми і для дітей} на сторінці
\enquote{Дитячий майданчик}.

З червня 2018 року Анатолій починає публікації про життя і творчість керівника
колективу сучасного бального танцю \enquote{Феєрія}, про сам колектив і виникнення
творчої танцювальної династії. А вже у 2019 р. за матеріалами публікацій у
письменника вийшла книга \textbf{\emph{\enquote{Феєрія-данс}}} – нариси про життя танцювального
колективу. Книга мала на меті на прикладі життєпису конкретного колективу і
його керівника висвітлити виникнення і розвиток бального танцю в Маріуполі.

У блогах MRPL.CITY  (публікується з 2017 року) готує статті про танцювальні
колективи та їхніх керівників. Наразі публікується цикл \textbf{\enquote{Незвичайний
\enquote{Вернісаж}}}, присвячений народному колективу сучасного бального танцю та його
керівникам Марині і Віктору Юшинам.

У Маріуполі Анатолій Михайлович любить проводити час на морі та своїй дачі
(кінець Волонтерівки). Але найбільше полюбляє приходити на дитячий майданчик,
до створення якого він має безпосереднє відношення. Досі доглядає за майданчиком
і радіє, що Острівець дитинства \enquote{Вітер Надії} став улюбленим місцем
відпочинку для багатьох маленьких маріупольців.

\begingroup
\em
\textbf{Хобі:} полюбляє працювати з деревом.

\textbf{Улюблена книга:} Біблія.

\textbf{Улюблений фільм:} \enquote{Доярка з Хацапетівки} (2007 рік).

\textbf{Порада маріупольцям:} \enquote{Більше уваги приділяйте дітям!}.
\endgroup
