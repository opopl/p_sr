% vim: keymap=russian-jcukenwin
%%beginhead 
 
%%file slova.titushka
%%parent slova
 
%%url 
 
%%author 
%%author_id 
%%author_url 
 
%%tags 
%%title 
 
%%endhead 
\chapter{Титушки}

%%%cit
%%%cit_head
%%%cit_pic
\ifcmt
  pic https://storage1.censor.net/images/6/5/d/d/65ddf06ed573d50909d25b5111b78a6c/650x434.jpg
	width 0.4
	caption Артилерист Максим Баглєй
\fi
%%%cit_text
Але потрапив на війну я з четвертої спроби. Тобто чотири рази моя мама плакала,
бо її син їде воювати, але потім він повертався назад (Сміється). Строкову
службу я не проходив - морозився, тобто був уклоністом. Під час Майдану певний
час хворів на грип, але встиг з Самообороною поганяти \emph{тітушок}. А потім якось
мені подзвонила знайома і сказала, що, Макс, в Росії приймають рішення Ради
Федерації про те, щоб вводити війська. Напевно, у нас буде війна. Я почув це і
реально стало трохи не собі - таке неприємне почуття в роті, наче мідну дротину
смокчеш. Кажу їй, що яка там війна - побрязкають зброєю і все заспокоїться
%%%cit_comment
%%%cit_title
\citTitle{Артиллерия: Артилерист Максим Баглєй: \enquote{Розповіді, що от коли
прийдуть до мене додому, тоді я буду воювати, я не сприймаю. Коли вони
прийдуть аж додому, ти отримаєш кулю в живіт... Тому краще їх зупиняти десь
там}}, 
Віка Ясинська, censor.net, 21.06.2021
%%%endcit

