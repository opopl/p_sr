% vim: keymap=russian-jcukenwin
%%beginhead 
 
%%file 04_11_2017.stz.news.ua.mrpl_city.1.pestrye_zametki_pro_avtomobili
%%parent 04_11_2017
 
%%url https://mrpl.city/blogs/view/pestrye-zametki-pro-avtomobili
 
%%author_id burov_sergij.mariupol,news.ua.mrpl_city
%%date 
 
%%tags 
%%title Пестрые заметки про автомобили
 
%%endhead 
 
\subsection{Пестрые заметки про автомобили}
\label{sec:04_11_2017.stz.news.ua.mrpl_city.1.pestrye_zametki_pro_avtomobili}
 
\Purl{https://mrpl.city/blogs/view/pestrye-zametki-pro-avtomobili}
\ifcmt
 author_begin
   author_id burov_sergij.mariupol,news.ua.mrpl_city
 author_end
\fi

Как-то вылазит из своего \enquote{Рено} бывший боцман одного из судов торгового флота,
а ныне пенсионер Саша, и ворчит: \enquote{Все жалуются - денег нет, а машину поставить
негде}. Действительно, где парковаться? Прямо у входа синего цвета \enquote{Деу} стоит,
за ней – красный старенький \enquote{Форд}, дальше - Серегин китайский бежевого цвета
драндулет, ему навстречу – \enquote{Фиат уно} с помятым боком, а поперек дороги
устроился на подзарядку \enquote{Ниссан} с электрическим двигателем – гордость  соседа
Алешки... Человека  родом с середины ХХ века разнообразие автомобильного \enquote{стада}
поражает. А ведь еще есть \enquote{Волги} и \enquote{Лады}, \enquote{Таврии} и \enquote{Волыни}, румынские
\enquote{Дачии}, чешские \enquote{Шкоды}, шведские \enquote{Вольво}, корейские \enquote{Хюндаи} и японские
\enquote{Мазды}. Ой, надо остановиться, иначе вся заметка будет состоять из
перечисления племени \enquote{четырехколесных}, которые снуют туда-сюда, заполонив все
пространство дорог и стоянок.

\ii{04_11_2017.stz.news.ua.mrpl_city.1.pestrye_zametki_pro_avtomobili.pic.1}

Лучше поделиться пестрыми заметками на тему Мариуполь и автомобили. В
Мариупольском краеведческом музее хранится фотография, датированная 20 апреля
1912 года, запечатлевшая  эпизод  из Дня белого цветка в Мариуполе, во время
которого собирали средства для лечения больных туберкулезом. Для нашей темы
фотография интересна тем, что на ней изображен автомобиль. Он украшен бумажными
цветами. На переднем сиденье рядом с водителем восседает Давид Александрович
Хараджаев, самый богатый мариуполец того времени, общественный деятель и щедрый
благотворитель. На заднем сиденье - две дамы. Кто они? Неизвестно. Снимок этот,
наклеенный на серого цвета паспарту,  – материальное подтверждение того, что в
нашем городе в 1912 году по меньшей мере один автомобиль уже был.

Из детства. Середина 40 – начало 50-х. Торговая улица. По ней движется
транспортный поток на \enquote{Азовсталь} и с \enquote{Азовстали}. Направление движения
автомобилей, подвод и линеек (варианта легкового гужевого средства
передвижения) зависит от времени суток. Утром – вниз больше, вечером – вверх
больше. Полуторки – ГАЗ-АА Горьковского автозавода, трехтонки ЗиС-5, окрашенные
в защитный цвет, редко появляющийся, наверное, единственный в городе автобус
Ярославского автозавода. Иногда с чувством собственного достоинства уверенно
следовали полученные по ленд-лизу во время войны американские \enquote{Студебекеры} и
\enquote{Шевроле}, изредка проскакивали трофейные \enquote{Мерседесы}, BMW, \enquote{Опели}. Всеобщее
внимание обращал на себя танк, превращенный в грузовик. С него была снята
башня, а вместо нее установлен большой кузов, сваренный из стальных листов.
Хорошо, что проезжая часть улицы сохраняла еще добротную брусчатку из местного
гранита. Проложенный позже асфальт вряд ли бы выдержал натиск гусениц боевой
машины.

Заведующая читальным залом профсоюзной библиотеки комбината имени Ильича Оксана
Чеботаревская установила, что  в феврале 1952 г. в актовом зале заводского
клуба им. Карла Маркса состоялись два концерта эстрадного оркестра под
управлением и при участии Леонида Утесова, в концерте участвовала также дочь
Леонида Осиповича - Эдит. Концерты прошли с огромным успехом. А это к чему? А
сейчас поймете. Уже не вспомнить, когда впервые в Мариуполе появились \enquote{Победы},
массовое производство которых началось в 1946 году. Скорее всего, в начале 1947
года. Одним из первых, если не самый первый,  в нашем городе приобрел такую
машину страстный охотник и любитель всяческих технических новшеств Уар Ефимович
Мальченко. Какая связь между концертами Утесова в нашем городе и мариупольским
автомобилистом-любителем? Дело в том, что Уар Ефимович возил на своей \enquote{Победе}
Леонида Осиповича Утесова и его дочь из гостиницы \enquote{Спартак} в клуб им. Карла
Маркса и обратно. Чем очень гордился.

Что еще вспомнилось на автомобильную тему? Ну, например, что на редких
автобусных маршрутах ходили автобусы ЛАЗы. Были они тесноваты, двигатель у них
чадил. Цена билета зависела от количества остановок, которые должен был
проехать пассажир. Понятно, что билеты продавал кондуктор. И только 1 февраля
1965 года на городских автобусных маршрутах был установлен единый тариф – 5
копеек. Также были введены проездные билеты на месяц стоимостью 3 рубля 75
копеек. Эти тарифы оставались неизменным около двадцати пяти лет.
