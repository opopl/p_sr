%%beginhead 
 
%%file 30_09_2021.fb.fb_group.mariupol.nekropol.1.volontery_otkrytie_mariupol_nekropol
%%parent 30_09_2021
 
%%url https://www.facebook.com/groups/278185963354519/posts/609945693511876
 
%%author_id fb_group.mariupol.nekropol,arximisto
%%date 30_09_2021
 
%%tags 
%%title Волонтеры сделали неожиданное открытие в Мариупольском Некрополе
 
%%endhead 

\subsection{Волонтеры сделали неожиданное открытие в Мариупольском Некрополе}
\label{sec:30_09_2021.fb.fb_group.mariupol.nekropol.1.volontery_otkrytie_mariupol_nekropol}
 
\Purl{https://www.facebook.com/groups/278185963354519/posts/609945693511876}
\ifcmt
 author_begin
   author_id fb_group.mariupol.nekropol,arximisto
 author_end
\fi

\vspace{0.5cm}
\textbf{Волонтеры сделали неожиданное открытие в Мариупольском Некрополе}

Кому принадлежит самая величественная усыпальница Мариупольского Некрополя?
Долгие годы считалось, что ее воздвиг знаменитый купец и гласный думы Дмитрий
Пиличев или Александр Гозадинов из рода митрополита Игнатия.

Открытие волонтеров оказалось сенсационным – в полу усыпальницы они обнаружили
массивную мраморную плиту ... Анастасии Исидоровны Оксюзовой (1826-1903)!

Чтобы окончательно разрешить загадку, в эту субботу волонтеры планируют
завершить уборку усыпальницы и приглашают всех желающих принять участие.

\#новости\_архи\_города

Заброшенная усыпальница с колоннами и фронтоном, украшенным цветной плиткой,
является самым известным сооружением Мариупольского Некрополя. Десятилетиями
краеведы пересказывали предание, что оно было построено Дмитрием Дмитриевичем
Пиличевым (1848-1913), купцом 2-й гильдии и гласным городской думы в 1893-97
гг.

Альтернативная версия - ее владельцем был Александр Гозадинов из рода
митрополита Игнатия, крупный землевладелец, председатель Мариупольской уездной
земской управы в 1891-1916 гг.

Две недели назад Владимир Николаевич Пиличев, потомок старинного греческого
рода Пиличевых, рассказал волонтерам, что в начале 1960-х годов он видел в
усыпальнице плиту с надписью \enquote{Пиличев} (см.
\url{https://cutt.ly/EEPva40}
\footnote{\url{https://www.facebook.com/groups/278185963354519/posts/598547001318412}}).

\ifcmt
  ig https://i2.paste.pics/PKB22.png?trs=1142e84a8812893e619f828af22a1d084584f26ffb97dd2bb11c85495ee994c5
  @wrap center
  @width 0.9
\fi

Чтобы разрешить эту вековую загадку, 18 сентября волонтеры приступили к уборке
усыпальницы. Оказалось, что пол скрывается под полуметровым слоем из земли,
битого кирпича и бутылок. Когда-то он был выложен орнаментом из плитки разных
цветов (см. фото).

После расчистки трети площади волонтеры обнаружили массивную плиту из черного
шлифованного мрамора. Когда-то она провалилась в склеп, подземную часть
усыпальницы (или ею попытались закрыть дыру в потолке склепа...). Надпись на
плите обескуражила всех: \textbf{\em Анастасия Исидоровна Оксюзова} (1826 – 1903)...

\textbf{Кто такие Оксюзовы и кем была Анастасия Оксюзова?}

Основателем этого старинного греческого рода был Юрий Иванович Оксуз
(1741-1811), цеховой ремесленник. Хоть некоторые его потомки стали купцами, но
они уступали по знатности и богатству другим известным греческим родам (тем же
Чентуковым или Поповым).

Внук родоначальника Федор Иванович Оксюзов (р.1808 г.) в 1840-х годах был
купцом и гласным Мариупольской городской думы.

Купцами также стали его правнук \textbf{Харитон Васильевич} (р.1845) и
праправнук \textbf{Павел Иванович} (нач. XX в.) (см. генеалогическое древо
Оксюзовых среди фото к публикации).

В начале XX века Харитон Васильевич, скорее всего, владел домом на Итальянской,
44 и роскошным домом на углу Николаевской и Торговой (нынешний бизнес центр
\enquote{Столетний}). Василий Гиацинтов арендовал его для Реального училища в 1906
году.

Но основной бизнес Харитона Оксюзова располагался в немецкой колонии Грюнау
(нынешняя Розовка). Он торговал там мануфактурными и бакалейными товарами.
Скорее всего, он там жил постоянно, а в Мариуполь приезжал по делам.

У Павла Ивановича Оксюзова была своя лавка-магазин на центральной
Екатерининской улице (нынешнем проспекте Мира). В 1917 году он возглавлял
городское коммерческое общество.

А вот Анастасия Исидоровна – вообще не из рода Оксюзовых. Скорее всего, она
была второй женой Василия Ивановича Оксюзова (1807-1870) и матерью Харитона и
его брата Кирилла Васильевича (р.1839). Так, среди крестников некой Варвары
Юрьевой в 1870 году в церковной метрике значатся – \emph{\enquote{мариупольский мещанин
Кирилл Оксюзов и мать его Анастасия Исидорова}}. Но вот какой была ее девичья
фамилия – мы не знаем...

Интересно, что мариупольские Оксюзовы дружили с семейством Каракашей, соседями
Архипа Куинджи. Один из Каракашей с ним учился и благодаря его воспоминаниям мы
имеем хоть какое-то представление о детстве знаменитого художника.

\textbf{А что же Пиличевы?}

Ответ – интригующий. Дмитрий Пиличев прекрасно знал Харитона Васильевича
Оксюзова, сына Анастасии Исидоровны! В 1885 году в Ильинской церкви Темрюка он
крестил его дочь Марину, а в 1891 году – сына Савву! А в 1883-м и 1886-м в
Успенской церкви Мариуполя он крестил вместе с женой Харитона детей
мариупольского мещанина Саввы Бабека...

Может быть, Анастасия Исидоровна из рода Пиличевых? Нет - в этом роду не было
ни исидоров, ни анастасий...

\textbf{Как же разгадать эту загадку?}

Неужели именно Оксюзовы воздвигли усыпальницу? И старожилы ошибались?

Приглашаем всех желающих в эту субботу, в полдень на завершение очистки пола
усыпальницы. Возможно, мы найдем еще одну плиту – уже Пиличевых... (подробности
мероприятия – в анонсе на странице \enquote{Архи-Города} \url{https://cutt.ly/5EAcWo4}\footnote{\url{https://www.facebook.com/events/284269723535281/284269733535280}}).

\ifcmt
  ig https://i2.paste.pics/PKB79.png?trs=1142e84a8812893e619f828af22a1d084584f26ffb97dd2bb11c85495ee994c5
  @wrap center
  @width 0.9
\fi

\ifcmt
  ig https://i2.paste.pics/PKB7E.png?trs=1142e84a8812893e619f828af22a1d084584f26ffb97dd2bb11c85495ee994c5
  @wrap center
  @width 0.9
\fi

\ifcmt
  ig https://i2.paste.pics/PKB7G.png?trs=1142e84a8812893e619f828af22a1d084584f26ffb97dd2bb11c85495ee994c5
  @wrap center
  @width 0.9
\fi

Мы надеемся, что к нам присоединится и Владимир Николаевич Пиличев...

Если у вас есть свои предположения и объяснения – обязательно поделитесь со
всеми!

\textbf{Благодарность волонтерам-исследователям}

Открытие стало возможным благодаря самоотверженному труду волонтеров Илья
Луковенко, \href{https://www.facebook.com/profile.php?id=100017225554315}{Євген Григорович Сандруцький}\footnote{\url{https://www.facebook.com/profile.php?id=100017225554315}}, Maryna Holovnova, Александр и \href{https://www.facebook.com/natalie.shya}{Наталия
Шпотаковская}\footnote{\url{https://www.facebook.com/natalie.shya}}

\textbf{Благодарность историкам и краеведам}

Мы с большим удовольствием благодарим исследователей, которые разыскали и
предоставили нам факты об Оксюзовых: Раису Петровну Божко, зам. директора
Мариупольского краеведческого музея, эксперта по генеалогии
\href{https://www.facebook.com/oleksii.resh}{Oleksii
Resh}\footnote{\url{https://www.facebook.com/oleksii.resh}}, краеведа
\href{https://www.facebook.com/helga.buzlami}{Helga
Buzlami}\footnote{\url{https://www.facebook.com/helga.buzlami}}. В церковных
метриках копался Андрей Марусов, директор ГО \enquote{Архи-Город}...

\ifcmt
  ig https://i2.paste.pics/PKB5U.png?trs=1142e84a8812893e619f828af22a1d084584f26ffb97dd2bb11c85495ee994c5
  @wrap center
  @width 0.9
\fi

\ifcmt
  ig https://i2.paste.pics/PKB5X.png?trs=1142e84a8812893e619f828af22a1d084584f26ffb97dd2bb11c85495ee994c5
  @wrap center
  @width 0.9
\fi

%\ii{30_09_2021.fb.fb_group.mariupol.nekropol.1.volontery_otkrytie_mariupol_nekropol.cmt}
