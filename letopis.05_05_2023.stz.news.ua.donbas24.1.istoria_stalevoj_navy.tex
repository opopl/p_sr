% vim: keymap=russian-jcukenwin
%%beginhead 
 
%%file 05_05_2023.stz.news.ua.donbas24.1.istoria_stalevoj_navy
%%parent 05_05_2023
 
%%url https://donbas24.news/news/istoriya-stalevoyi-navi-zaxisnicya-mariupolya-rozpovila-pro-vesillya-na-azovstali-ta-rosiiskii-polon-video
 
%%author_id news.ua.donbas24,prokopchuk_elina.mariupol
%%date 
 
%%tags 
%%title Історія сталевої Нави — захисниця Маріуполя розповіла про весілля на Азовсталі та російський полон (ВІДЕО)
 
%%endhead 
 
\subsection{Історія сталевої Нави — захисниця Маріуполя розповіла про весілля на Азовсталі та російський полон}
\label{sec:05_05_2023.stz.news.ua.donbas24.1.istoria_stalevoj_navy}
 
\Purl{https://donbas24.news/news/istoriya-stalevoyi-navi-zaxisnicya-mariupolya-rozpovila-pro-vesillya-na-azovstali-ta-rosiiskii-polon-video}
\ifcmt
 author_begin
   author_id news.ua.donbas24,prokopchuk_elina.mariupol
 author_end
\fi

\ii{05_05_2023.stz.news.ua.donbas24.1.istoria_stalevoj_navy.pic.front}

\begin{center}
\em\color{blue}\Large\bfseries
Майже рік Валерія Суботіна провела у неволі. Зараз дівчина вчиться жити далі з
пам'яттю про загиблого чоловіка
\end{center}

%\ifcmt
  %ig https://i2.paste.pics/59d19041615b19e5e0e683c95dcdc59d.png
  %@wrap center
  %@width 0.9
%\fi

Історія кохання Валерії та Андрія Суботіних зворушила всю Україну. В цей час,
поки світ спостерігав за непохитністю захисників Маріуполя, на Азовсталь падали
бомби, а місто палало від ворожого вогню, у фортеці героїв було місце
найщирішим почуттям. Перебуваючи у пеклі війни, прикордонник Андрій зробив із
фольги обручку своїй коханій. Вона відповіла \enquote{так}, і 5 травня, в день
народження полку \enquote{Азов}, закохані одружилися. У пари було небагато часу разом —
вже 7 травня, виконуючи бойове завдання, Андрій Суботін загинув... А на Валерію
ще чекали нові випробовування — дні в оточеній Азовсталі, здача в російський
полон і майже рік у неволі. 10 квітня захисниця Маріуполя нарешті \href{https://donbas24.news/news/nava-povernulasya-z-polonu-zvilnili-zaxisnicyu-mariupolya-valeriyu-subotinu}{повернулася в Україну}.

Історію, сповнену болю, кохання та світлої пам'яті, вона \href{https://www.youtube.com/watch?v=SuTVEPisKJQ}{розповіла} ТСН.

\textbf{Читайте також:} \href{https://donbas24.news/news/ruyini-zamist-budinkiv-rozbiti-azovstal-ta-centr-google-onoviv-karti-mariupolya-foto}{%
Руїни замість будинків, розбиті \enquote{Азовсталь} та центр: Google оновив карти Маріуполя (ФОТО), %
Наталія Сорокіна, donbas24.news, 26.04.2023%
}

\subsubsection{Андрій з'явився, як диво}

Валерія народилася та зростала у селищі під Шахтарськом. У цьому куточку
Донбасу, який потім окупували росіяни та назвали \enquote{исконно русской землей},
споконвіку всі говорили українською, співали народних пісень та шанували
традиції. Маленька Лера теж говорила українською з близькими, але що таке
велика українська родина, ніколи не знала. З життя рано пішли її батьки, Леру
виховували дідусь та бабуся, які поховали власних синів — батька та дядька
дівчинки, які обидва були військовими льотчиками. Дідусь Лери багато хворів та
згодом помер, тож єдиною близькою людиною для дівчини довгий час залишалася
бабуся.

\begin{leftbar}
	\begingroup
		\bfseries
\qbem{У мене практично не було дитинства — не вистачало тепла та любові, було
недостатньо коштів. І коли мене ображали, я уявляла, що Бог сидить на
великому стільці, у нього велика ряса, і я, така маленька, заповзаю під
той стілець і лягаю спати. Бог зверху, і ніхто мене не може скривдити},
— згадує через багато років Валерія.
	\endgroup
\end{leftbar}

Потім був переїзд до Маріуполя, кар'єра у Маріупольському державному
університеті. Валерія Суботіна шість років працювала викладачем, стала
кандидаткою наук із соціальних комунікацій, писала вірші та до повномасштабної
війни встигла видати власну збірку поезій \enquote{Квіти та зброя}.

\ii{05_05_2023.stz.news.ua.donbas24.1.istoria_stalevoj_navy.pic.1}

У 2015 році Валерія залишила викладацьку діяльність та працювала у прес-службі
полку \enquote{Азов}, а з 2017 року — очолила пресслужбу прикордонного загону Донецької
області. Там вона працювала до 2020 року, і саме там на неї чекало кохання.

\begin{leftbar}
	\begingroup
		\bfseries
\qbem{Андрій з'явився, як диво, — згадує Валерія. — Він працював у нашому
прикордонному загоні, був юристом, а я — пресофіцером. Я закохалася з
першого погляду! Це людина, з якою мені було максимально легко. Я
завжди мріяла зустріти доброго чоловіка, це така рідкість в світі! І
тоді мені здалося, що він мені був даний за все це дитинство, за все
горе, яке було в моєму житті. Я була з ним і дитиною, і жінкою, і була
нереально щасливою}.
	\endgroup
\end{leftbar}

\ii{05_05_2023.stz.news.ua.donbas24.1.istoria_stalevoj_navy.pic.2.valeria_ta_andrij}

\textbf{Читайте також:} 

\href{https://donbas24.news/news/fotograf-jevgen-maloljetka-peremig-na-world-press-photo-zuri-vrazili-svitlini-mariupolya}{%
Фотограф Євген Малолєтка переміг на World Press Photo — журі вразили світлини Маріуполя, %
Еліна Прокопчук, donbas24.news, 20.04.2023%
}

\subsubsection{\enquote{Розкопуйте Наву, вона жива}}

З 2020 року до повномасштабної війни Валерія не служила у війську, але коли
Росія вторглася на наші землі, дівчина за власним бажанням мобілізувалася і
знову увійшла до пресслужби легендарного полку \enquote{Азов}.

Валерія згадує: тоді з міста поїхала б лише за однієї умови. Вона хотіла
вивезти з нього бабусю, а потім повернутися у Маріуполь. Та бабуся відмовилася,
і Валерія одразу зателефонувала Калині, заступнику командира полку \enquote{Азов}. Він
сказав \enquote{приїзди}. Так дівчина опинилася на Азовсталі. Андрій теж не поїхав з
міста з прикордонниками і залишився захищати його з \enquote{Азовом}.

15 квітня Валерія отримала поранення — на бункер, де вона знаходилася з
побратимами, впала авіабомба.

\begin{leftbar}
	\begingroup
		\bfseries
\qbem{Я почула голоси своїх побратимів: \enquote{Нава там, треба відкопувати, Нава
жива}. Мене почали розкопувати, тягти, щось говорили про ноги. Я
подумала, що в мене немає ніг, бо я їх не відчувала. Виявилося, що їх
просто привалило, — згадує Нава. — Мене схопили Орест та Калина,
понесли на руках у шпиталь, з голови текла кров. Був сильний обстріл, і
виявилося, що майже всі навколо мене мертві. Ніхто не розумів, як так
виявилося, що я була жива, це було диво}.
	\endgroup
\end{leftbar}

\textbf{Читайте також:} 

\href{https://donbas24.news/news/film-pro-mariupol-peremig-na-kinofestivali-u-ssa}{%
Фільм про Маріуполь переміг на кінофестивалі у США, Еліна Прокопчук, donbas24.news, 07.04.2023}

\subsubsection{71-й день війни став днем весілля}

Ще у мирному Маріуполі Андрій хотів одружитися. Валерія чекала, коли відросте
волосся — мріяла зробити гарну зачіску, коли стане нареченою. Все змінила війна
— пропозицію коханий зробив Валерії з обручкою із фольги. Там, у підземеллях
Азовсталі, Валерія говорила, що ще буде час одружитися... Але, звісно, погодилася
на пропозицію і вони розписалися.

71-й день повномасштабного вторгнення став для пари днем весілля. Вони обрали
знакову дату — 5 травня — день народження полку \enquote{Азов}.

\begin{leftbar}
	\begingroup
		\bfseries
\qbem{Не було якогось розпису, обміну клятвами, просто було щастя, що він прийшов}, — згадує дівчина.
	\endgroup
\end{leftbar}

\ii{05_05_2023.stz.news.ua.donbas24.1.istoria_stalevoj_navy.pic.3}

Андрій Суботін загинув 7 травня, а Валерія дізналася про це 9-го. Та досі
дівчина відчуває, що він поруч. Тому розповідає усім, яким він був. Як сміливо
дбав про інших. Як носив їжу дітям у бункері. Як допомагав евакуювати поранених
побратимів. Як рятував життя інших і не боявся втратити своє.

\begin{leftbar}
	\begingroup
		\bfseries
\qbem{Андрій мені постійно снився у Таганрозі, я не знала, де було його тіло.
І одного разу мені наснилося, що він в Маріуполі. Він сказав — \enquote{мені
тут добре, я тут з хлопцями}}, — згадує Валерія Суботіна.
	\endgroup
\end{leftbar}

\subsubsection{Смерть була б легшою за полон}

\begin{leftbar}
	\begingroup
		\bfseries
\qbem{Для мене смерть була б набагато легшою, ніж все те, що довелося
пережити. Це була б можливість бути з моїм Андрієм. А я пішла в полон
до людей, які в мене його забрали, які забрали в мене все. Це було
найважче}, — розповідає Нава.
	\endgroup
\end{leftbar}

Досі дівчина не відчула, що повернулася додому. Вже не боїться прокидатися
зранку, але що б не робила, думає, як прокидаються в полоні наші захисники, що
вони їдять, яку мову чують.

\begin{leftbar}
	\begingroup
		\bfseries
\qbem{Не можу повернутися з полону, поки вони там}, — говорить захисниця Маріуполя.
	\endgroup
\end{leftbar}

Перед поверненням в Україну не було ані речей, ані грошей. Валерія разом з
іншими захисницями думали: як жити далі, як виходити у порожнечу.

\begin{leftbar}
	\begingroup
		\bfseries
\qbem{Ми вийшли і зрозуміли, як Україна зустрічає своїх. Нас нагодували, дали
багато одягу, я не розуміла, навіщо мені два чи три светри! Ми співали
гімн, до нас бігло стільки людей}, — говорить про свої перші радісні
спогади Валерія.
	\endgroup
\end{leftbar}

\ii{05_05_2023.stz.news.ua.donbas24.1.istoria_stalevoj_navy.pic.5}

Бабуся завжди вчила дівчину виглядати гарно. І весь час у полоні Нава мріяла
лише про гребінець для волосся! Разом з подругою вони уявляли, як бабуся
зустріне онуку і посварить за сивину. \enquote{Навіть тюрма — не привід ходити так}, —
сказала б обов'язково жінка.

Вже після виходу з полону Валерія дізналася, що бабуся померла в окупованому
Маріуполі. А коли відкрила сумку, яку їй зібрала мама Андрія, побачила такий
омріяний гарний гребінець.

\begin{leftbar}
\bfseries
\emph{\enquote{Зараз в мене є батьки Андрія, які мене безмежно люблять. Такої любові я
ніколи не відчувала. Мені дуже б хотілося помінятися з Андрієм місцями,
щоб вони могли любити його живим. Але я намагаюся жити далі}}, —
говорить Нава.
\end{leftbar}

Після виходу з полону дівчина одразу зробила татуювання зі словом \enquote{сталева}.
Розповідає: це про всю Азовсталь, про всіх наших сталевих, про весь \enquote{Азов} —
хлопців і дівчат, яких не зламати.

Наразі на честь річниці полку \enquote{Азов} та річниці весілля Валерії та Андрія про
пару вийшла пісня \enquote{Сталеве кохання}. Леся Шиденко та Григорій Попович написали
слова, а виконала її Марія Бурмака. Пісня стане саундтреком документального
фільму Станіслава Сукненко \enquote{Азовсталь прихисток кохання}.

\href{https://archive.org/details/video.29_04_2023.tsn.staleva_nava_interview_jake_prosto_rozryvaje}{%
Відео: Сталева НАВА: відверте інтерв'ю, яке просто РОЗРИВАЄ..., ТСН, 29.04.2023%
}%
\footnote{\url{https://archive.org/details/video.29_04_2023.tsn.staleva_nava_interview_jake_prosto_rozryvaje}} %
\footnote{\url{https://www.youtube.com/watch?v=SuTVEPisKJQ}}

\ifcmt
  ig https://i2.paste.pics/PRY2B.png?trs=1142e84a8812893e619f828af22a1d084584f26ffb97dd2bb11c85495ee994c5
  @wrap center
  @width 0.8
\fi

Нагадаємо, раніше Донбас24 розповідав, що в Україні \href{https://donbas24.news/news/v-ukrayini-moze-zyavitisya-den-vsanuvannya-zaxisnikiv-mariupolya}{\emph{може з'явитися}}%
\footnote{В Україні може з’явитися День вшанування захисників Маріуполя, Тетяна Веремєєва, donbas24.news, 24.03.2023, \par\url{https://donbas24.news/news/v-ukrayini-moze-zyavitisya-den-vsanuvannya-zaxisnikiv-mariupolya}}
День вшанування захисників Маріуполя.

Ще більше новин та найактуальніша інформація про Донецьку та Луганську області
в нашому телеграм-каналі Донбас24

ФОТО: з відкритих джерел

%\ii{05_05_2023.stz.news.ua.donbas24.1.istoria_stalevoj_navy.txt}
