% vim: keymap=russian-jcukenwin
%%beginhead 
 
%%file 23_12_2021.fb.fb_group.story_kiev_ua.1.gde_vy_sushite_belje.cmt
%%parent 23_12_2021.fb.fb_group.story_kiev_ua.1.gde_vy_sushite_belje
 
%%url 
 
%%author_id 
%%date 
 
%%tags 
%%title 
 
%%endhead 
\zzSecCmt

\begin{itemize} % {
\iusr{Сергій Огородник}
у нас і сьогодні люди сушать на вулиці

\begin{itemize} % {
\iusr{Марина Соболевская}
\textbf{Сергій Огородник} , якщо нема загазованного повітря. Бо ми вже на балконах боїмось сушити білизну..

\iusr{Сергій Огородник}
\textbf{Марина Соболевская} нам з повітрям пощастило. Від проспектів далеко, поряд - Виноградарський ліс.

\iusr{Марина Соболевская}
\textbf{Сергій Огородник} , от коли можна позаздрити віддаленому від великих магістралей життю.)
\end{itemize} % }

\iusr{Николай Гребенкин}
Было время: и на чердаках снимали. Бедность потому что...

\iusr{Марина Соболевская}
\textbf{Николай Гребенкин}, представляю, как это обидно хозяйке.. такой урон..

\iusr{Татьяна Вакуленко}

На гвоздяке, на чердаке)

На нашем стареньком патриархальном Сырце, на длинной улице Олены Телиги, а раньше
Коротченко, а еще раньше Ново-Окружной... До сих пор во дворах полощатся алые и
прочие разноцветные паруса пододеяльников и простыней и сигнальные флажки
всяких милых детских вещичек!

Но панталоны и бюстгалтеры все также прячут от чужих глаз, и это правильно, я считаю).

Мой папа, светлая ему память, учил вешать белье на балконе простынями наружу и
последний ряд лицом в комнату. А уж между ними, внутри прятать нижнее белье)

\begin{itemize} % {
\iusr{Марина Соболевская}
\textbf{Татьяна Вакуленко} , оказывается, были и наработки, как лучше развешивать белье!) Спасибо за подробности!)

\iusr{Татьяна Вакуленко}
\textbf{Марина Соболевская} о-о-о, сколько было патриархальных условностей...

\iusr{Вікторія Святненко}
\textbf{Татьяна Вакуленко} и мой дедуля меня учил «прятать флаги» @igg{fbicon.face.grinning.sweat} 

\begin{itemize} % {
\iusr{Марина Соболевская}
\textbf{Вікторія Святненко} , я в восторге от вашего дедули.)

\iusr{Вікторія Святненко}
\textbf{Марина Соболевская} даааа! Дедуля у меня был ещё тот юморист)

\iusr{Марина Соболевская}
\textbf{Вікторія Святненко} , как он вас учил? Расскажите?)

\iusr{Вікторія Святненко}
\textbf{Марина Соболевская} я поступила в институт и переехала жить к нему в частный дом. Зимой дедуля просил расчистить снег, ну а я студентка! В одну лопату стежку почищу и убегать! А он в след: «Я тебе что!!! Канатоходеееец? А ну ка нормально чисти!!!!»

\iusr{Марина Соболевская}
\textbf{Вікторія Святненко} , гениальный дедушка! Это же надо придумать! Канатоходец!)))

\iusr{Вікторія Святненко}
\textbf{Марина Соболевская} 

так и с бельём... Из под тишка показывает на серое вывешенное белье соседки и
говорит: «Витуля, это не белье, такое стыдно вывешивать на всеобщее обозрение!
Накрайняк, для улицы держи новый комплект постельного и нательного! Что б
думали «какая хозяйкаааа!»

\iusr{Вікторія Святненко}
\textbf{Марина Соболевская} ой, Мариночка! А его фраза: «я тебе что, лесоруб??? Что ты мне столько борщу насыпала???»

\iusr{Марина Соболевская}
\textbf{Вікторія Святненко} , а почему лесоруб?)

\iusr{Вікторія Святненко}
\textbf{Марина Соболевская} бо вони важко працюють і багато їдять @igg{fbicon.laugh.rolling.floor}{repeat=3} 

\end{itemize} % }

\end{itemize} % }

\iusr{Татьяна Петрюк}

 @igg{fbicon.laugh.rolling.floor} Вечный кадр ... Это Подоле улице Еленовская
 @igg{fbicon.face.grinning.big.eyes} ( не удержалась и засняла исторический
 кадр)

\ifcmt
  ig https://scontent-frx5-1.xx.fbcdn.net/v/t39.30808-6/269814219_2552458711552741_6760950649870775704_n.jpg?_nc_cat=110&ccb=1-5&_nc_sid=dbeb18&_nc_ohc=vic5Y-uu6zsAX-GU2gP&_nc_ht=scontent-frx5-1.xx&oh=00_AT-Dt6JxYU9P9ByLCebWNQS2Ea-mxkJWb8gqRPh-Dkk75w&oe=61CEA73C
  @width 0.4
\fi

\iusr{Марина Соболевская}
\textbf{Татьяна Петрюк} , может это тайный знак?)

\iusr{Вікторія Святненко}

Ой, Спасибо большое! С таким удовольствием прочитала! Да, да, застали и мы в
80-е такие столбы с веревками... а тетушка на Большой Житомирской вывешивала
белоснежное белье во внутреннем дворике... ах, как же оно пахло с мороза....


\iusr{Марина Соболевская}
\textbf{Вікторія Святненко} , искренне благодарна за доброе внимание и ваши воспоминания!)

\iusr{Валерий Стрельцов}
А, Ваш папа случай не с Татарки.

\iusr{Татьяна Вакуленко}
\textbf{Валерий Стрельцов} нет, мы евбазовские. Воровского 6.

\iusr{Катя Бекренева}
Нам с квартирой повезло в том смысле, что до дороги далековато, и дом во дворе. Постельное бельё с мороза это что-то)

\iusr{Марина Соболевская}
\textbf{Катя Бекренева} , искренне за вас радуюсь!)  @igg{fbicon.tulip} 

\iusr{Николай Бурчин}
Пахнет наволочка снегом, где-то капает вода..  @igg{fbicon.face.nerd} 

\begin{itemize} % {
\iusr{Марина Соболевская}
\textbf{Николай Бурчин} , философское видение.)

\iusr{Николай Бурчин}
\textbf{Марина Соболевская} , это ж из песни)
\end{itemize} % }

\iusr{Lidia Bualo}
На чердАке, на гамАке, чтоб не свиснули собаки

\iusr{Марина Соболевская}
\textbf{Lidia Bualo} , отлично! Уже третий вариант стихотворения.)

\iusr{Yurii Koval}
Саме так і казали: \enquote{Сохнєтє}.

\begin{itemize} % {
\iusr{Марина Соболевская}
\textbf{Yurii Koval} , так это не шутка?)

\iusr{Yurii Koval}
\textbf{Марина Соболевская} Єврейский говір.
\end{itemize} % }

\iusr{Evgen Zachepylenko}
У них в Венеции то же сушат ...

\ifcmt
  ig https://scontent-frt3-1.xx.fbcdn.net/v/t39.30808-6/269811084_1055423678365065_813436908500964475_n.jpg?_nc_cat=104&ccb=1-5&_nc_sid=dbeb18&_nc_ohc=20iNGTck2OkAX-_ho0R&_nc_ht=scontent-frt3-1.xx&oh=00_AT-FcFW4x05tMT_w5XCVR1i2rCmYqedcyMX8mwGNo1X5Ng&oe=61CF08ED
  @width 0.3
\fi

\begin{itemize} % {
\iusr{Татьяна Николаева}
\textbf{Evgen Zachepylenko} 11трусов и 2 майки - странная коллекция )

\begin{itemize} % {
\iusr{Evgen Zachepylenko}
\textbf{Татьяна Николаева} итальянцы наверно часто меняют такой народ...

\ifcmt
  ig https://scontent-frt3-1.xx.fbcdn.net/v/t39.30808-6/269808432_1055429585031141_8086476427554166830_n.jpg?_nc_cat=104&ccb=1-5&_nc_sid=dbeb18&_nc_ohc=ZFqV7LjdhbUAX_nq46t&_nc_ht=scontent-frt3-1.xx&oh=00_AT-eZwBtTFnxJ1bhxN5bH4vT-hW-ndmKVydmwarHVynD0Q&oe=61CDF804
  @width 0.3
\fi

\iusr{Татьяна Николаева}
\textbf{Evgen Zachepylenko} все постирала  @igg{fbicon.laugh.rolling.floor} 

\iusr{Evgen Zachepylenko}
\textbf{Татьяна Николаева} таки ДА

\iusr{Al Pol}
\textbf{Evgen Zachepylenko} 

Однажды у меня в доме была в гостях прелестная юная дева. В то время наш
коттеджный поселок ещё продолжал застраиваться и рядом на участке вовсю шли
работы по сооружению фундамента соседского дома.

Дева имела привычку разгуливать по дому с бокалом шампанского в костюме
Евы..... В задумчивости подошла к большому окну и наблюдая за рабочими
предалась раздумьям...

Я в это время после жарких утех задремал...

Разбудил меня дикий вопль. Работяга на соседском участке грохнул себя по ноге
200-килограммовой трамбовкой.... Засмотрелся бедолага на юную, нагую деву в
окне....

Женская красота - страшная сила ....

\end{itemize} % }

\iusr{Катя Семенкова}
\textbf{Evgen Zachepylenko} 

романтично потягивая кофе, а над тобой развиваются лёгкие труселя
@igg{fbicon.face.smiling.eyes.smiling}{repeat=3} 

\begin{itemize} % {
\iusr{Татьяна Николаева}
\textbf{Катя Семенкова} всегда можно посмотреть нарравление и силу ветра  @igg{fbicon.wink} 

\iusr{Evgen Zachepylenko}
\textbf{Татьяна Николаева} итальянская девушка устала от стирки

\ifcmt
  ig https://scontent-frt3-2.xx.fbcdn.net/v/t39.30808-6/269822623_1055455211695245_7074469432369854602_n.jpg?_nc_cat=101&ccb=1-5&_nc_sid=dbeb18&_nc_ohc=4MtHzstjDVwAX8amLHS&_nc_ht=scontent-frt3-2.xx&oh=00_AT8xn9YH87yMU2w-qBnrVF21Sv2jvAXJht6oDodvqnXoqw&oe=61CE88B2
  @width 0.3
\fi

\end{itemize} % }

\iusr{Марина Соболевская}
\textbf{Evgen Zachepylenko} , супер! И милиция их не гоняла!)))

\end{itemize} % }

\iusr{Владимир Баваровский}

У нас во дворе (Верхний Сырец) были бабушки, на которых оставляли белье. Они
четко сидели и смотрели за ним. И нас гоняли чтоб мы не бегали и не исгадили))
А вот зимой хозяйки как-то сами следили. Мама всегда сушила на открытом
балконе. Но в начале 2000х как-то после сушки она сняла и встряхнула простынь,
а от нее столб пыли( Больше у нас на балконе белье не сохло и веревок нет
@igg{fbicon.face.grinning.smiling.eyes}  А вот труселя с панталонами у нас
всегда на кухне на натянутой по диагонали веревке))

\begin{itemize} % {
\iusr{Lesya Paschenko-Startseva}
\textbf{Владимир Баваровский} Простите, вопрос не по теме: а что такое Верхний Сырец( живу на Щусева,36)?

\begin{itemize} % {
\iusr{Владимир Баваровский}
\textbf{Lesya Paschenko-Startseva}, 

массивы возле парка детской жд и выше в сторону парка нивки. Это условное
разделение. Больше в период подростковых разборок) \enquote{А, ты с вехнего Сырца? А мы
с нижнего) Вот тебе в дыню!}  @igg{fbicon.laugh.rolling.floor} 


\iusr{Владимир Баваровский}
\textbf{Lesya Paschenko-Startseva}, 

вот, кстати, кажется по Щусева, оно и делилось. Я точно уже не скажу. Я жил с
самого края между парками и мне было это не важно, а вот друзья обитавшие в
районе 97й школы на Телиги постоянно имели столкновения и им это было
принципиально. Знаю только, что нежелательно было без взрослых соваться дальше
Щусева) Если пройти по Грекова до техникума (или что там сейчас),то дальше
можно было схлопотать.


\iusr{Тома Храповицкая}
\textbf{Владимир Баваровский} Помню, помню... не участвовала, но слышала от мальчишек.

\iusr{Lesya Paschenko-Startseva}
\textbf{Владимир Баваровский} Ничего себе! Я про такое и не подозревала! Я выросла на Лукьяновке, училась на Татарке(СШ1). Спасибо за разъяснение.
\end{itemize} % }

\iusr{Марина Соболевская}
\textbf{Владимир Баваровский} , идеальный вариант! И белье высохнет, и охрана надежная!)

\begin{itemize} % {
\iusr{Владимир Баваровский}
\textbf{Марина Соболевская}, 

да! До сих пор помню, как баба Сима ходит по двору \enquote{руки в боки}, в
одной руке выбивалка для ковра и гоняет нас от белья) А нам ведь там медом
намазано @igg{fbicon.face.grinning.smiling.eyes}  Не соображали, что чумазые
пачкаем белье @igg{fbicon.face.grinning.smiling.eyes} 


\iusr{Марина Соболевская}
\textbf{Владимир Баваровский} , прямо увидела эту картинку! Спасибо!)
\end{itemize} % }

\end{itemize} % }

\iusr{Леонид Гапон}
Мы в духовке на вегёвке и немножко на чегдак, чтобы не было багдак.

\begin{itemize} % {
\iusr{Ли Бесс}
\textbf{Леонид Гапон} очень по-одесски ))

\iusr{Наталья Борчанинова}
\textbf{Ли Бесс} И по-белоцерковски!!!)))

\iusr{Ли Бесс}
\textbf{Наталья Борчанинова} та да )) надо ж понимать согт людей )))

\iusr{Марина Соболевская}
\textbf{Леонид Гапон} , спасибо за еще один вариант!)
\end{itemize} % }

\iusr{Al Pol}

Бельё, высушенное на морозе имеет неповторимый аромат. Никакая сушильная машина
так не высушит.

Вспоминаю улицы Неаполя вблизи ж/д вокзала, где прямо через всю улицу натянуты
веревки для сушки белья .... Там и трусы, и всякое другое бельё никого не
смущает.

\begin{itemize} % {
\iusr{Марина Соболевская}
\textbf{Al Pol} , теперь и я люблю Италию! За свободу в развешивании белья!)

\begin{itemize} % {
\iusr{Al Pol}
\textbf{Марина Соболевская} 

Италия это целый мир ! Флоренция, Неаполь, Рим, Милан, Венеция .... Обожаю
эти города .... Хотя у небольших городов своя прелесть ... Сорренто,
Монтекатини, Лидо-ди-Есоло, Амальфи, Позитано - просто сказка !

Однажды в Неаполе меня \enquote{обидели} .... В каком-то бутике изысканно-вежливый,
прилизанный продавец мне сказал на убогом английском: Dear sir, you are
wasting your time, in our stores there are no clothes of your size.

Несчастный не понял, что этим он сделал мне комплимент.

\iusr{Марина Соболевская}
\textbf{Al Pol}, спасибо! Мне ужасно интересно!)

\iusr{Al Pol}
\textbf{Марина Соболевская} 

Иногда, путешествуя, я снимаю короткие видеоролики, которые выкладываю в
YouTube..... Не всегда, но часто ...

Испания, Италия, Греция, Южная Франция, Эмираты, Норвегия, Черногория, Швеция, и т. д.

Могу скинуть ссылку на мой видеоролик об Италии....

\iusr{Марина Соболевская}
\textbf{Al Pol} , здесь нельзя делать ссылки. Можно в личку.)

\iusr{Al Pol}
\textbf{Марина Соболевская} 

Отправил ссылку в личку.... я редко пользуюсь этим сервисом, возможно что-то
пошло не так. Пишите, если не получите.

\end{itemize} % }

\iusr{Вера Мезенцева}

Волшебное чувство восхищения после каждого вывешивания во дворе частного
дома, это было в юности и почему то запомнился этот запах единственного
стирального порошка в те времена Лотос.

\begin{itemize} % {
\iusr{Марина Соболевская}
\textbf{Вера Мезенцева} , 

удивительно! У вас запах порошка, как триггер, возвращает вас мысленно в картинки из прошлого.)

\iusr{Al Pol}
\textbf{Марина Соболевская} 

Позволю себе вмешаться. Действие запахов в качестве триггера, запускающего
определенные эмоции, почувствовал на себе не однажды.

Вы, наверное, не знаете такие духи, как БЫТЬ МОЖЕТ. В конце 60-х, и в 70-х
прошлого века это был очень популярный аромат. Многие девушки пользовались
этими духами, в том числе и предмет моего пылкого, юношеского обожания....моя
первая любовь ....

С тех пор прошло более тридцати лет и, однажды, я почувствовал аромат БЫТЬ
МОЖЕТ.... Что со мной произошло, описать словами невозможно. Откуда появились
эти духи через столько лет - история заслуживающая отдельного повествования.

Потом была другая девушка и духи ШЕХЕРЕЗАДА ... И снова через много лет встреча
с запахом .... И снова эмоциональный шторм ....
\end{itemize} % }

\end{itemize} % }

\iusr{Марина Соболевская}

Весной 2016 года Мэрия инициировала закон о публичном пространстве, согласно
которому сушить белье на балконах будет запрещено — даже под угрозой штрафов.
Кроме того, инициатива должна была избавить город от уродливых балконов,
которые портят внешний вид города, и излишества рекламных билбордов.

Инициаторы проекта, заммэра Виталия Кличко Алексей Резников, обещали вынести
его на рассмотрение Киеврады в сентябре.)

\begin{itemize} % {
\iusr{Vladimir Popoff}
\textbf{Марина Соболевская} 

а про уродские кондиционеры там нет инициативы? Чтобы убрать их нафиг с
фасадов, хотя бы старых, зданий.


\iusr{Марина Соболевская}
\textbf{Vladimir Popoff} , честно говоря, я не знаю, чем закончилась эта история.)

\iusr{Владимир Баваровский}
\textbf{Марина Соболевская}, увы( Ничем

\iusr{Марія Рябоконенко}
\textbf{Марина Соболевская} 

Так ведь и советской власти было запрещено сушить белье на балконе, как пишет
автор \enquote{выше ватерлинии}. На центральных улицах за это штрафовали.

\end{itemize} % }

\iusr{Людмила Мудренко}
В 50-х годах у нас на Подоле сушили во дворе и охраняли.
Мама повесила мое поатьеце и рейтузы на чердаке. Утром было пусто.

\begin{itemize} % {
\iusr{Марина Соболевская}
\textbf{Людмила Мудренко} , вот да. Пишут, что в 50-ых послевоенных воровство было обычным делом.

\iusr{Вера Мезенцева}
\textbf{Людмила Мудренко} Ужасно
\end{itemize} % }

\iusr{Валентина Козачук}

Всегда сушила белье на балконе, дом стоит внутри массива, дороги далеко., но и
в ванной есть веревки, очень удобно досушивать белье и полотенца после душа. А
когда жили в хрущевке в Ново-Беличах, то часто сушили на улице на веревках,
воровства не было, так как все друг друга знали, многие работали на одном
предприятии, а зимой на снег всегда выносили коврики, дорожки и чистили снегом

\begin{itemize} % {
\iusr{Марина Соболевская}
\textbf{Валентина Козачук} , спасибо за ваш рассказ!)

\iusr{Володимир Денисов}
\textbf{Валентина Козачук} я свою Бабулю переконав років 7 тому, краще здати у чистку.... до того кожну зиму -« онучек, сніг впав, треба паласи почистити»
Але ... після снігу запах інший...
\end{itemize} % }

\iusr{Наталья Борчанинова}

В 80-е я жила на Березняках над о. Тельбин на 13 этаже. Балкон был не
застеклён. Вернулась с пляжа, пополоскала купальник (раздельный) и, когда
вывешивал сушить на верёвке, прищепка соскользнула в руке, трусы в другой руке
тоже соскользнули, не прищеплены были, и.... полетели вниз на чей - то балкон,
может быть и соседнего подъезда)))...

\begin{itemize} % {
\iusr{Марина Соболевская}
\textbf{Наталья Борчанинова} , сочувствую.) Нашлись? Или с концами?

\iusr{Наталья Борчанинова}
\textbf{Марина Соболевская} 

Второй вариант))) С концами...

Ни на деревьях, которые росли под домом, ни на земле их не было. А под моим
балконом было 12 балконов моего стояка и столько же смежных балконов соседнего
подъезда...

Объявлений о такой \enquote{пропаже} в подъездах я не писала)))

\iusr{Марина Соболевская}
\textbf{Наталья Борчанинова} , купальник- это важно. ) Потеря ужасная. Очень вас понимаю.

\iusr{Наталья Борчанинова}
\textbf{Марина Соболевская} 

А я тогда представила \enquote{картину}, когда хозяйка того балкона нашла
женские трусы и вела допрос мужу)))

\iusr{Марина Соболевская}
\textbf{Наталья Борчанинова} , это шедевральная ситуация! Представляю!)))
\end{itemize} % }

\iusr{Надежда Владимир Федько}

1970-1972-й... Подільський райвідділ міліції... Щотижнева нарада дільничих
інспекторів... Нараду проводить керівник служби дільничих інспекторів,
заступник начальника райвідділу... Підполковник Ш-ов.

Одне із стратегічних завдань - боротися з вивішенням білизни на балконах по вулицях (перелік вулиць)!

Керівна настанова - не ініціатива міліції (у дільничих багато більш важливих
проблем), а директивна вказівка райкому партії! Київ - зразкове місто!

\begin{itemize} % {
\iusr{Марина Соболевская}
\textbf{Надежда Владимир Федько} , отлично! Спасибо за подтверждение!)

\iusr{Надежда Владимир Федько}
\textbf{Марина Соболевская} 

В ті роки я працював у карному розшуку... Дільничі мого району обслуговування
постійно жалілися, що боротьба з \enquote{трусами на балконі} відволікає їх від
боротьби із злочинністю.

До речі, в ті роки була категорія злодіїв, які спеціалізувалися на крадіжках
білизни у дворах, з балконів 2 та верхнього поверхів, з горищ будинків.


\iusr{Марина Соболевская}
\textbf{Надежда Владимир Федько} , благодарю от всей души! Гениально про жалобы!! А про воровство. Все логично..
\end{itemize} % }

\iusr{Mariia Lashkevych}
Тиріли і котлети з пательні, що на примусі поки кухарка за чимось до кімнати йшла

\begin{itemize} % {
\iusr{Марина Соболевская}
\textbf{Mariia Lashkevych} , котлеты свежепожаренные, видимо, так вкусно пахли, что трудно было удержаться.))
\end{itemize} % }

\iusr{Елена Баскова}

1980 - спокойно сушили белье на балконах и никаких запретов не встречали.
помню еще соседей, сушивших свое белье на улице (балконов у них, видимо, небыло)

\begin{itemize} % {
\iusr{Владимир Баваровский}
\textbf{Елена Баскова}, смотря где.

\iusr{Марина Соболевская}
\textbf{Елена Баскова} , верно. Вы их уже не застали.)

\iusr{Марина Соболевская}
\textbf{Елена Баскова} , посмотрите коммент - \textbf{Надежда Владимир Федько}.) В 1970-ых милицию обязывали бороться с бельем на балконах.))

\begin{itemize} % {
\iusr{Елена Баскова}
\textbf{Марина Соболевская} я вижу коммент. Я пишу про 80-е.

\iusr{Олена Харченко}
\textbf{Марина Соболевская} якщо на хрещатику, то можливо. А на воскресенці не чіпали. Боролись зі застекленими балконами

\iusr{Раиса Лузан}
\textbf{Олена Харченко} 

да, было такое. Мы жили на Печерске на первом этаже в доме , который выходил
окнами на райком партии и в 73 году застеклили лоджию. Через какое-то время
папу вызвали в райком и сказали снять застекление. Не знаю, что ответил им
папа, но стекла остались. Потом постепенно на всех 16-ти этажах соседи начали
стеклить лоджии, также как у нас. Смотрелось очень гармонично, как будто так и
было при строительстве дома.

\end{itemize} % }

\iusr{Наталья Борчанинова}

А вот в начале 80-х на оз. Тельбин проводились международные соревнования по
катанию на водных лыжах... До начала соревнований с жильцов дома, у которых
окна выходили на озеро, участковый или какие-то \enquote{товарисчи} взяли расписку,
чтобы не вывешивали белье в дни соревнований...

\begin{itemize} % {
\iusr{Марина Соболевская}
\textbf{Наталья Борчанинова} , очень интересное событие! Спасибо!)

\iusr{Наталья Борчанинова}

Сейчас живу в г. Белая Церковь. Белье с соседями сушим на площадках с проволок
для сушки белья. Дом \enquote{хрущовка}, 88 квартир. Живу на 1 этаже. Летом сохнет
белье на солнце, а зимой на морозе!

За веревки соседи не ругаются, даже те, кто живут на верхних этажах, тоже
предпочитают сушить во дворе. Краж давно нет...

\end{itemize} % }

\end{itemize} % }

\iusr{Анна Цесорева}

А який чудовий спектакль з такою назвою я дивилась в театрі Колесо на
Андріївському узвозі! Хто ще не бачив - рекомендую \enquote{Одного разу на Подолі, або
гдє ві сохнітє бєльйо}

\iusr{Марина Соболевская}
\textbf{Анна Цесорева} , дякую за рекомендацію!)

\iusr{Ксенія Ковальська}

Трохи не в тему, проте дотично. Згадайте, як вибивали тріпачкою килими на
спеціально встановленій конструкції зі зварених труб. Це влітку. А взимку, коли
випадав сніг, з усіх квартир виносили килими, притрушували їх сніжком,
вибивали, топтали, змітали сніг віником і несли в квартиру разом з килимом
запах снігу. Минулої зими жодного килима на снігу я з вікна багатоповерхівки не
бачила.

\begin{itemize} % {
\iusr{Владимир Баваровский}
\textbf{Ксенія Ковальська}, 

это потому что уже ковров почти не осталось @igg{fbicon.smile}  Однако как раз
в прошлом году на Сырце, где живут родители, все дворы были в следах чистки
ковров и даже часть парка успели изгадить
@igg{fbicon.face.grinning.smiling.eyes}  Так что традиции неискоренимы.

\begin{itemize} % {
\iusr{Марина Соболевская}
\textbf{Владимир Баваровский} , сейчас помыть ковер- 120 грн. кв. метр.)

\iusr{Владимир Баваровский}
\textbf{Марина Соболевская}, я знаю, но папа брыкается и ни в какую)

\iusr{Ксенія Ковальська}
\textbf{Владимир Баваровский} дійсно, килимів на душу населення поменшало. Але в старих районах ще є. Я свій здавала чистити на фірму.

\iusr{Марина Соболевская}
\textbf{Ксенія Ковальська} , а за яку ціну? Й чи задоволені?)

\iusr{Ксенія Ковальська}
\textbf{Марина Соболевская} не пам'ятаю ціну. Задоволена. Зараз згадаю назву фірми. Здається \enquote{Мокра справа} чи \enquote{Мокрое дело}.

\iusr{Марина Соболевская}
\textbf{Ксенія Ковальська} , дякую!)
\end{itemize} % }

\iusr{Марина Соболевская}
\textbf{Ксенія Ковальська} , так.) Моя матуся переймалась в ті роки, як снігу не було, як килимки мити?)
\end{itemize} % }

\iusr{Nadia Matura}
Не слямзили воровки

\iusr{Марина Соболевская}
\textbf{Nadia Matura} , верно. Просто мне так запомнилось, ошибочно.)

\iusr{Татьяна Бартновская}

Замечательно! Сейчас живу в частном доме в селе и сушу белье всегда во дворе, на воздухе! Даже зимой, когда руки с прищепками за секунду примерзают к верёвке. И всегда, где бы я не жила, и дома, в Киеве, всегда сушилка белье на улице! И в двухтысячных охраняла его. Все точно так как в этом замечательном рассказе!!! А этот стишок - родом из нашей любимой Одессы!  @igg{fbicon.face.smiling.eyes.smiling}{repeat=3}  @igg{fbicon.face.blowing.kiss}{repeat=3} 

\begin{itemize} % {
\iusr{Марина Соболевская}
\textbf{Татьяна Бартновская} , с удовольствием прочла ваш комментарий! Думаю, одесские стишки передавались цыганской почтой.) А сушка белья на природе - постепенно забывается. Так радуюсь за вас, за вашу возможность продлить удовольствие!)

\begin{itemize} % {
\iusr{Larysa Karbysheva}
\textbf{Марина Соболевская} В старых районах ещё есть место во дворах. Много деревьев, между ними веревки натягивают..

\iusr{Марина Соболевская}
\textbf{Larysa Karbysheva} , но охранять, я так понимаю - еще надо.)

\iusr{Larysa Karbysheva}
\textbf{Марина Соболевская} Не знаю.... Из Окна вижу как развешивают. Охранников нет, может 1-ый этаж?

\iusr{Марія Рябоконенко}
\textbf{Марина Соболевская} Сестра живёт на Отрадном. Когда не приеду, во дворе сушится белье. Веревки постоянно натянуты, их не снимают.
\end{itemize} % }

\end{itemize} % }

\end{itemize} % }
