% vim: keymap=russian-jcukenwin
%%beginhead 
 
%%file 04_03_2020.stz.news.ua.mrpl_city.1.knyga_istoria_odnogo_dnja
%%parent 04_03_2020
 
%%url https://mrpl.city/blogs/view/istoriya-odnogo-dnyakniga-znahidka-pro-mariupol
 
%%author_id demidko_olga.mariupol
%%date 
 
%%tags istoria,kniga,mariupol,mariupol.istoria
%%title "Історія одного дня" – книга-знахідка про Маріуполь
 
%%endhead 
 
\subsection{\enquote{Історія одного дня} – книга-знахідка про Маріуполь}
\label{sec:04_03_2020.stz.news.ua.mrpl_city.1.knyga_istoria_odnogo_dnja}
 
\Purl{https://mrpl.city/blogs/view/istoriya-odnogo-dnyakniga-znahidka-pro-mariupol}
\ifcmt
 author_begin
   author_id demidko_olga.mariupol
 author_end
\fi

\ii{04_03_2020.stz.news.ua.mrpl_city.1.knyga_istoria_odnogo_dnja.pic.1}

В Маріуполі є безліч унікальних історичних споруд, культурних пам'яток та
рідкісних книг, історія і походження яких відомі далеко не всім його мешканцям.
Мабуть, саме тому я вирішила познайомити вас, дорогі читачі, з одним із
найбільш цікавих видань міста. Прочитавши цю книгу, я була у неймовірному
захваті від її стилю та змісту. Як-то кажуть, Гоголь відпочиває. Йдеться про
\enquote{Історію одного дня}, надруковану у 1901 році в маріупольській друкарні Л.
Шпарбера.

Автор книги – російський громадський діяч і політик, член III Державної думи
від Катеринославської губернії \textbf{Петро Валерійович Каменський}. Він займав посаду
почесного мирового судді Бахмутського округу (з 1881) і Маріупольського повіту
(з 1891 року), був гласним Маріупольського повітового і Катеринославського
губернського земських зборів, а також Маріупольським повітовим головою
дворянства (1899–1913). Оскільки в повіті практично не було дворянства, свою
діяльність переважно спрямовував в інтересах сільського населення: українських
та російських селян, колишніх німецьких колоністів, селян-греків і
євреїв-землевласників. Був відомим лектором і публіцистом. Якщо переглянути
список творів Павла Валерійовича виникне враження, що це доволі серйозний
лектор, теми якого торкалися релігії, благодійності, освіти, боротьби з
бідністю... Але водночас цьому благородному діячу вдавалося писати з гумором та
легкою іронією, адже саме так написана книга, присвячена подіям, що відбулися в
Маріуполі \textbf{6 квітня 1863 року}.

За один день доволі кмітливий і завзятий чоловік \textbf{Григорій Ільяшенко} (він же
головний герой) вирішив провчити колишнього члена грецького суду Маріуполя
\textbf{Логафетова} за зловживання довіреною йому владою. Виявивши неабиякі акторські
здібності, він представився уповноваженим імператора і розпорядився \enquote{за грабежі
та вбивство і взагалі за всі зловживання позбавити (Логофетова) всіх прав стану
зі засланням в алтайські заводи у вічні працівники, а маєток його продати з
публічного торгу і задовольнити всіх боржників...}. Маріупольська влада,
налякана \enquote{начальницькою} особою, завзято виконала всі вказівки, заарештувала
Логафетова і навіть поголила йому півголови. Але, на жаль, благородна справа Г.
Ільяшенка було розкрита... Цей короткий переказ – ніщо в порівнянні з самою
книгою, яка дозволяє краще зрозуміти побут, відносини між різними народами,
особливості життя в Маріуполі наприкінці XIX ст. І якщо вірити автору, ця
неймовірна історія не була вигадкою. Дата мене вразила найбільше, адже саме 6
квітня народився мій синочок і я подумала, що це не випадковість. Можливо,
настав час представити це рідке і унікальне видання широкому загалу
маріупольців. І саме мені потрібно трошки для цього постаратися. Книга
зберігається в Маріупольському краєзнавчому музеї.

На щастя, мені вдалося отримати відсканований примірник (звісно, за окрему
плату). Читати її не так легко, роздрукувати неможливо, дуже розмитий текст,
тому ознайомлюватися можна лише на комп'ютером. Ще й сторінки доволі потерті,
але кому сьогодні легко? Набравшись терпіння і ентузіазму, я вирішила
адаптувати мову до більш сучасної і найбільш цікаві уривки представити у своїх
блогах. А ідей, пов'язаних з цим виданням безліч. Зокрема, музей неодноразово
пропонував поставити на маріупольській сцені виставу за твором П. Каменського.
Думаю, дійство було б захоплюючим. Я пропоную ознайомитися з уривками з книги
мовою оригіналу. Якщо будуть відступи, переходитиму на українську. Намагалася
підібрати найбільш цікаві фрагменти і дуже сподіваюся, що ця книга знайде свого
читача в нашому місті, адже вона незаслужено забута і мало кому відома...

\ii{04_03_2020.stz.news.ua.mrpl_city.1.knyga_istoria_odnogo_dnja.pic.2}

\enquote{\em Около чисел 20-х марта 1863 года в г. Бердянске в гостинице \enquote{Белого лебедя}
Ореховской 3-й гильдии купец Поддубня, торговавший товарами, рассказывал двум
своим знакомым Мазину и Колосовскому, о приключившихся ему несчастиях в г.
Мариуполь. В числе слушателей находился еще третий слушатель, неизвестный
Поддубне, но также внимавший рассказчику, назывался он \textbf{Григорий Власов
Ильяшенко}. Это был молодой человек 33 лет, блондин, ничего особенного из себя
не представляющий, никаких, как говорится в паспортах, особых примет не
имевший. Он был уроженец города Николаева; в описываемое время проживал с женой
в г. Бердянск, где занимался частно чертежными работами местного архитектора.
До прибытия в Бердянск Ильяшенко находился на службе в севастопольской
инженерной командировке морской строительной части чертежником и был награждён,
как это удостоверяется официальными документами, бронзовой медалью на
Андреевской ленте. 

Рассказ Поддубни не был связный. Он начал с объяснения, в чем состояло в городе
Мариуполе особое греческое управление, но его объяснения были малоуяснительны,
ибо они сводились  к построению одних и тех же слов: каторжные порядки,
проклятое греческое царство и т. п. Понятно, что эти повторяющиеся фразы,
переплетенные с ругательства и никак не знакомыми с учреждением, которым
подчинялось в то время греческое население. На самом же деле, ознакомление с
этими учреждениями не представляется делом сложным...}

Далі автор на кількох сторінках ознайомлює читача з історією Маріуполя, а саме:
з заснуванням міста греками, з грецьким судом, особливостями його роботи,
компетенціями, складом...

\enquote{\em Из дальнейшего разговора можно было узнать, что в конце 50-х и начале 60-х
годов членом греческого суда состоял некий обыватель города Мариуполя
\textbf{Логафетов}, он ведал полицейскую часть, был таким образом, администратором,
имевший своей задачей способствовать благоустройству мариупольского греческого
округа и его обывателей. В этакой своей роли он ничего почётного не оставил. Но
зато, как полицейский чин, руководивший обходами  во время ярмарок в Мариуполе,
Логафетов приобрел широкуюю известность. Молва соединяла его имя с целым рядом
мелких и крупных преступлений. Знаменитые логофетовские обходы надолго
сохранились в памяти потомства Мариуполя, которые обычно состояли в разврате,
кутежах и обжорстве...}

Можливо, читач вже зрозумів, що саме через Логафетова постраждав Поддубня. Далі
автор дуже докладно наводить всі скоєні злочини Логафетовим та його загоном,
які намагалися уважно \enquote{стежити за порядком} в Маріуполі... А от, що відбувалося
потім та чому вирішив Григорій Ільяшенко помститися за Поддубню, ви дізнаєтеся
у наступному блозі, що вийде трохи пізніше... 
