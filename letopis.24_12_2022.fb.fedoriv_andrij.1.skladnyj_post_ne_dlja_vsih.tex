% vim: keymap=russian-jcukenwin
%%beginhead 
 
%%file 24_12_2022.fb.fedoriv_andrij.1.skladnyj_post_ne_dlja_vsih
%%parent 24_12_2022
 
%%url https://www.facebook.com/andriy.fedoriv/posts/pfbid02xXgEev59QFvLRoi3D9NbRwWeTqjVrhY83YQTnHGzQmbdx8VSRedpE7pDQ4gfNFhPl
 
%%author_id fedoriv_andrij
%%date 
 
%%tags 
%%title Складний пост. Не для всіх
 
%%endhead 
 
\subsection{Складний пост. Не для всіх}
\label{sec:24_12_2022.fb.fedoriv_andrij.1.skladnyj_post_ne_dlja_vsih}
 
\Purl{https://www.facebook.com/andriy.fedoriv/posts/pfbid02xXgEev59QFvLRoi3D9NbRwWeTqjVrhY83YQTnHGzQmbdx8VSRedpE7pDQ4gfNFhPl}
\ifcmt
 author_begin
   author_id fedoriv_andrij
 author_end
\fi

Складний пост. Не для всіх. 

Сьогодні день народження Дарини Жолдак. Моєї першої дружини, що раптово залишила цей світ більше 10 років тому. 

Я знаю, скільки людей зараз втрачає тих, кого так любить. 

Ми всі втратили наше минуле життя 24 лютого. 

Тому я наважився написати  кілька речей, поступове розуміння яких психологічно
допомогло мені свого часу.  Я маю надію, комусь зараз це теж допоможе. Хоч
трішки.

1. Все має початок й має кінець. Коли ти знаходишся в епіцентрі трагедії, тобі
здається, що це назавжди. Це не так. Все закінчиться. Рано чи пізно. Так чи
інакше. Почнеться життя «після». В це зараз важко повірити, але є всі шанси, що
нове життя буде прекрасним. 

2. Головна задача не дати відчаю паралізувати вас. Треба діяти. Рухатись
вперед. Дбати про себе й рідних. Приймати рішення. Часто не буде ідеального й
доведеться обирати між поганим й поганим. Треба обирати. Переживання, сльози,
рефлексію ідеально  відкласти на момент, коли всі важливі кроки зроблено. У вас
точно буде на це достатньо часу. Але не зараз. Зараз потрібно просто діяти.
Навіть якщо для цього потрібні надзусилля. 

3. Світ не справедливий. Він не працює за логічною схемою «я хороша людина - ми
хороша родина - значить в нас все має бути добре». Добро й зло в постійному
русі й завжди виникають точки концентрації зла та несправедливості - йдуть
найкращі й найрідніші. Це безкінечно боляче й несправедливо. Майже кожна
людина, кожне покоління, кожна нація проходить через власні трагедії. Це закон
природи. Не варто питати «чому саме ми?». Не варто очікувати від світу
справедливості. Треба ставитись до всього як до випробування. Страшного
екзамену,  який треба скласти якнайкраще й йти вперед. Робити що маєш й най
буде, що буде. 

4. Життя людини - найвища цінність. Зробіть все можливе, будь які надзусилля,
щоб вберегти себе, близьких та рідних. Ніяких думок про самогубство!!!! Навіть
коли здається, що жити далі немає сенсу. Сенс життя в тому, щоб продовжувати
жити. Всі додаткові сенси прийдуть. 

5. Не сумуйте за матеріальним чи за комфортом. Все це все одно нам не належить.
Прийшло - пішло - знову прийде. Як і ми. Прийшли, побули, полетимо далі.
Берегти треба свою душу, совість, свою гідність й стосунки з тими, кого любиш й
цінуєш. Не робити те, за що соромно. Навіть в критичні моменти. 

6. Не сумуйте за минулим. Його вже немає.  Залиште від нього тільки світло.
Якщо є час,  запишіть зараз в нотатки чи зошит максимально детально те, що для
вас в ньому було важливо. В усіх дрібницях. Люди, звички, опишіть свою кімнату,
події, слова. Напишіть про те як жила ваша родина. Які були традиції. Пам’ять
не надійна. Через 5-7 років все що зараз яскраве буде заховано в тумані. А так
ви зможете його зберегти. 

7. Відпускайте. «Дякую що в мене були такі прекрасні люди й такі моменти. Я
люблю Вас. Я відпускаю вас...». 

8. Що справді важливо - ваш особистий контент - фото, відео, дрібниці. Коли
буде час - розберіть старі фото, створіть папки з важливими для вас фото чи
відео з телефону. Зробіть до них підписи. Це відволікає та заспокоює. Прекрасна
медитація. Можливо це єдине, що варто зберігати та передати дітям. Подбайте про
це. 

9. Не тримайте горе в собі! Воно руйнівне. Розмовляйте. Не бійтеся розповідати
про свою історію, свій  біль, свій страх раптовим знайомим, або близьким людям,
або психологам. Через слова, через повторення, через пояснення ви утилізуєте
свої переживання. Ви можете вести щоденник й записувати туди те, що
відбувається з вами зараз, свої переживання. Це допомогає. 

10. Якщо ви будете намагатися побудувати своє життя як безперервну лінію минуле
- сьогодні - майбутнє, у вас не вийде, бо забагато хорошого але вже недосяжного
в минулому, трагічного в сьогодні та невизначеного в майбутньому. Тому задача
тепер - прожити один день. З ранку до ночі. Без минулого й майбутнього. Ми
народжуємось, коли прокидаємось, щоб померти, коли засинаємо. Один день = одне
життя. Спробуйте прожити його якнайкраще. Спробуйте навіть дозволити собі
посміхнутися. Навіть якщо все трагічно. Не думайте про щастя. Думайте про
радість. Маленькі краплі  радості. Й одного дня... все буде добре.

