% vim: keymap=russian-jcukenwin
%%beginhead 
 
%%file 27_07_2021.fb.olivec_vladimir.1.marshrutka_mova
%%parent 27_07_2021
 
%%url https://www.facebook.com/permalink.php?story_fbid=3025413351062325&id=100007810686548
 
%%author 
%%author_id olivec_vladimir
%%author_url 
 
%%tags jazyk,marshrutka,mova,ukraina,ukrainizacia,zakon
%%title Отже наявне ще одне порушення - неволодіння громадянином України державною мовою!
 
%%endhead 
 
\subsection{Отже наявне ще одне порушення - неволодіння громадянином України державною мовою!}
\label{sec:27_07_2021.fb.olivec_vladimir.1.marshrutka_mova}
 
\Purl{https://www.facebook.com/permalink.php?story_fbid=3025413351062325&id=100007810686548}
\ifcmt
 author_begin
   author_id olivec_vladimir
 author_end
\fi

Цікаво, чому дотримання 30 статті "Закону про забезпечення функціонування
української мови як державної" контролює дедалі більше "знайомих Лариси Ніцой",
а 36 - ніхто?

25 липня став свідком принципового порушення цієї статті водієм маршрутного
таксі ТОВ "Фастівавтотранс". Чоловік не просто грубіянив пасажирці і намагався
збити з неї 24 гривні за проїзд однієї зупинки по Києву, посилаючись на
наявність відеокамери над його робочим місцем. Він також обслуговував абсолютно
усих іноземною мовою навіть після інформування про незаконність цього й прохань
перейти на державну. Особливо цікаво, що водій підтвердив фіксацію камерою і
звуку в салоні, отже був абсолютно певний у безкарности порушення закону.
Української він, за його словами, не вивчав у жодному навчальному закладі й
письмові документи в своєму АТП заповнює з допомогою словника! Отже наявне ще
одне порушення- неволодіння громадянином України державною мовою! Як він міг
бути допущений до рейсу?!

З цими питаннями звернувся до керівника підприємства Костянтина Олександровича
Федуна. Той спершу пообіцяв провести роз'яснювальну роботу з водієм. Але на
запитання чому не  з усіма водіями їхнього величезного парку почав
виправдовуватися:і робота важка, і стресова, і зарплата низька через що водії
масово їдуть за кордон тощо, але найголовніше- він реально не знав про 36
статтю "Закону про забезпечення функціонування української мови як державної"!
Звісно директор спитав чи наявні порушення на маршрутах не обслуговуваних
"Фастівавтотрансом". Можливо, справді змова, адже безліч порушень наявні досі
по АТП усієї України!

\ifcmt
  tab_begin cols=2
		width 0.3

     pic https://scontent-cdt1-1.xx.fbcdn.net/v/t1.6435-9/224305398_3025413287728998_6042660151326926309_n.jpg?_nc_cat=105&ccb=1-3&_nc_sid=8bfeb9&_nc_ohc=-eRx8JmadmEAX8BR0VH&_nc_ht=scontent-cdt1-1.xx&oh=532d322a7a6202a0595c99cde5ef5483&oe=612C5564
		width 0.35

     pic https://scontent-cdt1-1.xx.fbcdn.net/v/t1.6435-9/224106363_3025413327728994_6517844181664015970_n.jpg?_nc_cat=109&ccb=1-3&_nc_sid=8bfeb9&_nc_ohc=fSxxtU9G4-8AX--1zkA&_nc_ht=scontent-cdt1-1.xx&oh=2a65e99978cd59b6eae1726c23921f0e&oe=612C2CA9
		width 0.2

  tab_end
\fi

Словом, я щиро сподіваюся, що на третій рік чинности мовного закону
роз'яснювальні роботи будуть проведені керівниками усих АТП країни, але це
залежить винятково від нас!

Чому? Та тому що за 2 роки я виявився єдиним, хто звернувся до пана Федуна з мовною скаргою!!!!!!!!......

А що там на ваших перевізниках, брати українці та сестри українки?

Зміни починаються з нас!

\ii{27_07_2021.fb.olivec_vladimir.1.marshrutka_mova.cmt}
