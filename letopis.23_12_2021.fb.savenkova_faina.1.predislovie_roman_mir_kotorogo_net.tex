% vim: keymap=russian-jcukenwin
%%beginhead 
 
%%file 23_12_2021.fb.savenkova_faina.1.predislovie_roman_mir_kotorogo_net
%%parent 23_12_2021
 
%%url https://www.facebook.com/permalink.php?story_fbid=439417627679175&id=100048328254371
 
%%author_id savenkova_faina
%%date 
 
%%tags chelovek,kniga,literatura,serdce
%%title Предисловие к нашему новому роману "Мир которого нет"
 
%%endhead 
 
\subsection{Предисловие к нашему новому роману \enquote{Мир которого нет}}
\label{sec:23_12_2021.fb.savenkova_faina.1.predislovie_roman_mir_kotorogo_net}
 
\Purl{https://www.facebook.com/permalink.php?story_fbid=439417627679175&id=100048328254371}
\ifcmt
 author_begin
   author_id savenkova_faina
 author_end
\fi

Предисловие к нашему новому роману \enquote{Мир которого нет} замечательного писателя и
переводчика Елены Хаецкой. Спасибо @igg{fbicon.grin} 

Там, где время становится пространством.

\ii{23_12_2021.fb.savenkova_faina.1.predislovie_roman_mir_kotorogo_net.pic.1}

Традиция сказок о ремесленниках и о механизмах очень старая (хотя, конечно,
далеко не такая старая, как, напри мер, традиция сказок о животных). Можно даже
предполо жить, что именно сказки о ремесленниках наиболее близ ки современному
читателю. Ведь эта традиция зародилась в городах, там, где большим авторитетом
обладали ремеслен ники – свободные люди, создающие различные удивитель ные
вещи, подчас с полным основанием именуемые «шедев рами» («шедевр» – это
изделие, которое подмастерье изго тавливает в качестве выпускного экзамена при
аттестации на звание мастера). Механизмы с их удивительным устройством и
таинственными для непосвященного принципами работы завораживают и таким образом
имеют полное основание стать сюжетообразующим элементом сказки.

Мы знаем примеры подобных сказок – наверное, наиболее известным является
«Городок в табакерке» В. Ф. Одоевского, где мальчик Миша чудесным образом
попадает внутрь музыкальной шкатулки и узнает принцип ее работы, обща ясь с
живыми деталями этого механизма.  Но, разумеется, о чем бы ни писал писатель и
как бы при чудливо ни выглядели его персонажи – будь то говорящие механизмы,
роботы или инопланетяне, – он всегда пишет о людях. И в новом романе Александра
Конторовича и Фаины Савенковой «Мир, которого нет» читатель без труда угадает
стоящие за живыми часовыми механизмами человече ские образы: будильники и
брегеты, хронометр и бедолага таймер, у которого и циферблата-то нет... Будильник
– тру дяга, Брегет – красавчик, Хронометр – распорядитель... тут даже и гадать-то
не приходится. Взаимодействие главных героев, мальчика и девочки (брата и
сестры), с этими суще ствами – это не столько о постижении устройства работы ча
сового механизма (у мальчика Миши в «Городке в табакерке» была именно такая
цель в первую очередь), сколько о поиске и осмыслении человечности,
чрезвычайно важной теме для подросткового возраста.

А насколько может быть человечен человек с механическим сердцем в груди?
Человек, который отчасти – механизм? Человечнее ли, чем он, «настоящие» люди,
способные без зазрения совести наживаться на бедах своих соседей, на болезни
чужих детей? Где «заканчивается» механизм и на чинается человек? Помните, в
«Волшебнике Изумрудного города» Железный Дровосек не сразу перестал быть
человеком: он постепенно заменял части своего тела на железные, но так и не
понял, с какого момента он превратился в... в то существо, каким мы его знаем. И
наличие любящего сердца не делает ли его настоящим человеком? А если все
обстоит ровно наоборот – как у одного из главных персонажей книги, которую вам
предстоит прочитать, – что если ты полностью человек, но сердце у тебя
искусственное, – тогда как считать? Человек ты или, может быть, монстр?

Мы-то знаем ответ, как знают его и авторы романа «Мир, которого нет»:
единственное мерило в подобном вопросе – человечность.

И в простеньком будильнике может жить чья-то душа, и прекрасный брегет может
оказаться пустой игрушкой, однако в любом творении человеческих рук мы должны
уважать людскую работу, потому что это уважение – уже наша личная
ответственность.

Роман дает много интересных тем для обсуждения, ведь несмотря на
приключенческую, фантастическую оболочку он содержит довольно глубокие мысли и
провоцирует на серьезные вопросы.

И один из этих вопросов, на сей раз даже не социально-нравственный, а чисто
научный, – это вопрос о соотно шении времени и пространства. В романе
сакрализируются часы, точно измеренное время представлено как фактор, от
которого зависит ни больше ни меньше правильное функци онирование этого мира.
Да, конечно, первобытный, антич ный и даже средневековый человек жили в мире,
где вре мя измерялось очень приблизительно, летом дневные часы были длиннее,
зимой – короче, а уж что говорить о мину тах! И ничего, функционировал как-то
мир, в котором жили все эти люди... 

Но все чаще человек задумывался над тем, в какой точке сходятся время и
пространство и какое влияние искажение времени может оказывать на искажение
пространства. На этом крайне интересном фантастическом «па радоксе» построен
сюжет романа. И, надо сказать, у авторов получилось представить эту тему
достаточно отчетливо, без каких-то лишних наукообразных рассуждений и в то же
время вполне на уровне добротной научной фантастики.

Вообще к какому жанру отнести роман – НФ или фэнтези

– тут можно поспорить, поскольку четких, однозначных критериев для определения
жанра так и не выработано. Можно вообще считать, что «Мир, которого нет»
принадлежит к почтенному жанру утопии. На это, кстати, намекает и название
романа.

Вот как описан мир механизмов, в который попадают наши герои:

«Этот мир был напрочь лишен всех изъянов и проблем.  Все работали, выполняя
своё предназначение... Не существовало проблем с жильём – каждый прибывший
получал свой дом. Ему предоставлялась работа и время для отдыха.
Удовлетворялись все текущие потребности – которых, по правде сказать, и было-то
не особенно много...»

Но на то и щука в озере, чтобы карась не дремал: появле ние в утопическом
механистическом мире живых ребят вно сит свои коррективы – и в мир, и в сюжет.

В путь, читатель! Нас ждет сонный городок в Калужской области. Не следует,
впрочем, обольщаться – на чердаках старых домов таких сонных городков может
обнаружиться нечто очень и очень неожиданное.

Елена Хаецкая, писатель, переводчик
