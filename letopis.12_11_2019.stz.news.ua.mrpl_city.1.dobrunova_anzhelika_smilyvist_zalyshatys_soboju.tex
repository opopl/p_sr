% vim: keymap=russian-jcukenwin
%%beginhead 
 
%%file 12_11_2019.stz.news.ua.mrpl_city.1.dobrunova_anzhelika_smilyvist_zalyshatys_soboju
%%parent 12_11_2019
 
%%url https://mrpl.city/blogs/view/dobrunova-anzhelika-majte-smilivist-zalishatisya-soboyu
 
%%author_id demidko_olga.mariupol,news.ua.mrpl_city
%%date 
 
%%tags 
%%title Добрунова Анжеліка: "Майте сміливість залишатися собою!"
 
%%endhead 
 
\subsection{Добрунова Анжеліка: \enquote{Майте сміливість залишатися собою!}}
\label{sec:12_11_2019.stz.news.ua.mrpl_city.1.dobrunova_anzhelika_smilyvist_zalyshatys_soboju}
 
\Purl{https://mrpl.city/blogs/view/dobrunova-anzhelika-majte-smilivist-zalishatisya-soboyu}
\ifcmt
 author_begin
   author_id demidko_olga.mariupol,news.ua.mrpl_city
 author_end
\fi

\ii{12_11_2019.stz.news.ua.mrpl_city.1.dobrunova_anzhelika_smilyvist_zalyshatys_soboju.pic.1}

Наша героїня - неймовірно талановита та багатогранна особистість, ліричні та
витончені вистави якої для багатьох маріупольців стали справжнім відкриттям.
Режисерка-постановниця театру, викладачка Маріупольського коледжу мистецтв
\textbf{Анжеліка Арганівна Добрунова} нещодавно завоювала серця одразу всіх
маріупольських театралів вдалою та унікальною виставою \enquote{Біла
ворона}.\footnote{\url{https://mrpl.city/events/view/belaya-vorona}} Такої
естетичної насолоди містяни не отримували давно... І поки розкуповуються білети
на першу в Маріуполі рок-оперу, пропоную ближче познайомитися з чарівною
жінкою, без творчості якої сьогодні складно уявити мистецьке життя Маріуполя.

Народилася Анжеліка в Самарській області. Мама мала білоруське походження,
батько – кавказьке. Зустрілися батьки в Самарі, де й одружилися. А незабаром
після народження донечки сім'я переїхала до Горлівки. Батьки у нашої героїні
дуже талановиті. Мама зіткнулася з війною, тому не отримала належної освіти.
Вона завжди була дуже артистичною, а тато – справжній романтик. Дітям вони
ніколи нічого не забороняли. Завжди у всьому підтримували Анжеліку, допомагали
всебічно розвиватися.

\textbf{Читайте також:} \emph{Сам себе режиссер: видеоролики профессионалов и аматоров Мариуполя покажут на большом экране}%
\footnote{Сам себе режиссер: видеоролики профессионалов и аматоров Мариуполя покажут на большом экране, mrpl.city, 08.11.2019, \par%
\url{https://mrpl.city/news/view/sam-sebe-rezhisser-videoroliki-professionalov-i-amatorov-mariupolya-pokazhut-na-bolshom-e-krane}
}

Спочатку дівчина вступила до Донецького будівельного технікуму на архітектурну
спеціальність. Однак одразу зрозуміла, що це не її професія. Закінчивши
технікум, до речі, з червоним дипломом, Анжеліка вступила до Харківського
інституту культури (спеціальність \enquote{режисура драми}). На той час вона вже
виступала у Горлівському народному театрі \enquote{Юність}, де потрапила в студію до
головного режисера театру, талановитого педагога \textbf{Костянтина Володимировича
Добрунова}. Це була найважливіша зустріч в їхньому житті. Колектив театру був
досить сильним, з мудрими викладачами. Народний театр виступав на рівні
міського, що вплинуло на подальше отримання статусу муніципального. Спочатку
Анжеліка Арганівна прийшла в горлівський театр, щоб стати актрисою. Саме там
вона зіграла велику кількість різнопланових ролей, адже Костянтин Володимирович
їй повністю довіряв. Оскільки навчалася наша героїня на денному відділенні, це
забирало багато часу, тому вона вирішила перевестися до Київського інституту
культури (який тепер теж академія) на заочне навчання. Річ у тому, що на одному
з фестивалів на неї звернув увагу столичний педагог, він і запросив Анжеліку до
себе на курс. Після цього більше часу з'явилося для роботи в горлівському
театрі. За пропозицією Костянтина Добрунова вона стала викладати в одній зі
студій.

\ii{12_11_2019.stz.news.ua.mrpl_city.1.dobrunova_anzhelika_smilyvist_zalyshatys_soboju.pic.2}

Стати режисером запропонував теж Костянтин Володимирович, який вірив в Анжеліку
часом навіть більше, ніж вона сама. Він сформував її як творчу особистість,
знайшовши спосіб відкрити її, дати особистій природі вийти, став для неї
творчим батьком. Їхні стосунки будувалися на суцільній довірі, повазі,
помноженими на два. Причому найкращим у цьому було те, що спочатку відбулося
їхнє духовне єднання. Коли розумієш, що людина, яка знаходиться поруч з тобою,
– частина самого тебе і ви – дві половинки одного цілого, то далі ніякі
труднощі у стосунках, ніякі розбіжності вже неважливі. Разом вони створили
спочатку творчий союз, який згодом перетворився на міцну і щасливу сім'ю.

\ii{12_11_2019.stz.news.ua.mrpl_city.1.dobrunova_anzhelika_smilyvist_zalyshatys_soboju.pic.3}

До Маріуполя приїхали на початку 2000-х, оскільки в драматичному театрі якраз
2000 року Костянтину Володимировичу запропонували спочатку обійняти посаду
режисера-постановника, а через кілька місяців він став головним режисером.
Анжеліка Арганівна переїхала на два роки пізніше, і була відразу ж прийнята на
посаду режисера-постановника. Без перебільшення можна вважати, що вся творча
біографія жінки пов'язана з Костянтином Добруновим, який і зараз залишається в
її серці. Вона часто згадує його настанови та поради. Завдяки енергійній та
невтомній діяльності талановитого митця маріупольський театр отримав не тільки
велику кількість нових вистав, а ще й поповнився молодими та яскравими
артистами. Сьогодні місто зобов'язане Костянтину Володимировичу тим, що в
Маріупольському коледжі мистецтв є спеціальність \enquote{акторська майстерність} і
зараз коледж продовжує готувати молоді кадри для театру.

\ii{12_11_2019.stz.news.ua.mrpl_city.1.dobrunova_anzhelika_smilyvist_zalyshatys_soboju.pic.4}

\textbf{Читайте також:} \emph{С большим пульсирующим сердцем: в Мариуполе прошел гала-концерт инклюзивного фестиваля творчества}%
\footnote{С большим пульсирующим сердцем: в Мариуполе прошел гала-концерт инклюзивного фестиваля творчества, mrpl.city, 04.11.2019, \par%
\url{https://mrpl.city/news/view/s-bolshim-pulsiruyushhim-serdtsem-v-mariupole-proshel-gala-kontsert-inklyuzivnogo-festivalya-tvorchestva-foto-video}}

\ii{12_11_2019.stz.news.ua.mrpl_city.1.dobrunova_anzhelika_smilyvist_zalyshatys_soboju.pic.5}

Донька Анжеліки і Костянтина – \textbf{Наталя} – сама вирішала стати актрисою. Тато її
навіть відмовляв. Але, коли Костянтин Володимирович побачив старанність
Наталки, її сумлінність і відповідальність, визнав, що вона дійсно талановита,
та погодився, що донька обрала правильний шлях. Все ж таки гени нікуди не
подінеш.

\ii{12_11_2019.stz.news.ua.mrpl_city.1.dobrunova_anzhelika_smilyvist_zalyshatys_soboju.pic.6}

Ліричні й витончені вистави Анжеліки Добрунової (\enquote{Анна Марина}, \enquote{Таємниця
Дітріх}, \enquote{Едіт Піаф} \enquote{Танець на двох}, \enquote{Зіта і Гіта}, \enquote{Наполеон і Корсиканка},
\enquote{За двома зайцями} тощо) є найулюбленішими для багатьох глядачів Маріуполя.
Визначальною рисою Добрунової-майстра було і залишається прагнення до яскравої
поетичної образності, до поєднання глибокого психологізму з вдалою
публіцистичністю. Режисерка обирає ті вистави, які створює для чогось, важливо,
щоб тема могла запалити. Буває, що виставу обирає не вона, адже частими є і
замовлення від адміністрації театру, але і в цьому випадку Анжеліка Арганівна
намагається зробити тему близькою для себе. Важливо не \emph{\textbf{\enquote{виходити на сцену з
холодним серцем}}}! Ці слова Костянтина Володимировича часто повторює Анжеліка
Добрунова.

\ii{12_11_2019.stz.news.ua.mrpl_city.1.dobrunova_anzhelika_smilyvist_zalyshatys_soboju.pic.7}

Як я наголосила на початку нарису, остання вистава режисерки \enquote{Біла ворона}
викликала у глядачів неабиякий захват. І це не дивно, адже в рок-опері є все
для справжнього успіху: і надважлива тема та ідея п'єси, і неабияка
майстерність виконавців, і злагодженість роботи всього колективу. Вже під час
перегляду відчувалося, що це перемога для театру та режисера-постановника. Сама
Анжеліка Арганівна задоволена всіма, хто взяв участь у виставі. Бажання
поставити \enquote{Білу ворону} на маріупольській сцені виникло у директора театру
\textbf{Кожевнікова Володимира Володимировича}, і завдяки майстерності
режисера-постановника ця постанова міцно увійшла в репертуар театру. Режисерці
хотілося уникнути прямолінійності, вона намагалася зробити все на образах, для
чого обрала метафоричну мову. Було вирішено створити середньовіччя, яке
розповість і про сучасність. У виставі є дуже багато підтекстів. Можливо, саме
тому кожний глядач знаходить у постанові щось важливе і близьке для себе...

\ii{12_11_2019.stz.news.ua.mrpl_city.1.dobrunova_anzhelika_smilyvist_zalyshatys_soboju.pic.8}

Анжеліка Арганівна має чітке розуміння, що всіх акторів, з якими доводилось
працювати, вона по-справжньому любить і цінує, тому намагається давати
підтримку, яку вони відчують. Актори дуже залежні від режисера, проте у кожного
є свій творчий світ, який слід пізнавати та поважати. Головне в роботі
режисера, на думку нашої героїні, заразити своєю ідеєю і запалити актора,
занурити його у свій задум, тільки за таких умов на виставу чекає успіх. 

\ii{12_11_2019.stz.news.ua.mrpl_city.1.dobrunova_anzhelika_smilyvist_zalyshatys_soboju.pic.9}

У Маріуполі найбільше Анжеліка Арганівна любить центр міста і все, що
знаходиться поруч з театром. Подобається в місті приморський спокій, якась
незалежність. Важливо, що Маріуполь ні на кого не дивиться. Є в місті своє
особливе зачарування, наголошує мисткиня.

\ii{12_11_2019.stz.news.ua.mrpl_city.1.dobrunova_anzhelika_smilyvist_zalyshatys_soboju.pic.10}

\ii{insert.read_also.demidko.kozhevnikov}

\textbf{Хобі:} відкривати для себе нові місця і міста.

\textbf{Улюблені книги:} \enquote{Тріумфальна арка} Еріха Марії Ремарка, філософські романи Людмили Улицької, \enquote{Майстер та Маргарита} Михайла Булгакова, \enquote{Маленькі трагедії} Олександра Пушкіна, \enquote{Жарт} Мілана Кундери.

\textbf{Улюблений фільм:} \enquote{Службовий роман}. До речі, улюбленою актрисою є Аліса Фрейндліх.

\textbf{Порада маріупольцям:} 

\begin{quote}
\em\enquote{Всім бажаю мати сміливість залишатися собою. Бути уважнішими один до одного!}.
\end{quote}

\emph{Фото Лева Сандалова та Євгена Сосновського.}
