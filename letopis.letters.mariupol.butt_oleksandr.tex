% vim: keymap=russian-jcukenwin
%%beginhead 
 
%%file letters.mariupol.butt_oleksandr
%%parent letters
 
%%url 
 
%%author_id 
%%date 
 
%%tags 
%%title 
 
%%endhead 

Ок, понял, спасибо! группу видел, вроде даже подписан, сейчас проверю... но я
больше всего сейчас смотрю группу Мариуполь Довоенный... Сочувствую, и знаю,
через что всем вам пришлось пройти... А насчет времени... Вы говорите, что
время его еще не пришло... Знаете. Я боюсь наверное сказать не то, боюсь сейчас
задеть Вас, но дело в том, что к сожалению люди... в общем, все мы смертны. И
иногда хочешь что то сделать, но просто не успеваешь, увы... Это касается в
равной степени и меня, и Вас, и всех людей на Земле вообще. Потому что у Бога
на нас свои планы, в которые Он нас не посвящает. Поэтому... я не могу Вам
советовать конечно, как поступать с Вашим архивом. Но... все, что касается
Мариуполя. абсолютно бесценно. И... я бы на Вашем месте не откладывал
публикацию архива, по вышеуказанным причинам, но... решать Вам, конечно. Моя
позиция здесь такова, что чем быстрее люди будут публиковать, все, что они
могут опубликовать, все, что они знают про Мариуполь и что они хотели бы
сказать миру про Мариуполь, тем оно лучше. Вообще говоря, есть такое, проблема
книжек про Мариуполь. Ну то есть проблема фактического отсутствия книжек - и
про довоенный Мариуполь, и про блокадный. В практической плоскости, если идешь
в магазин, то есть какие угодно книжки, но только не про Мариуполь (поправка -
появилась небольшая книжка Надежды Сухоруковой, плюс есть еще книжка
(Шишацкий), которую в принципе не проблема купить, - но общей картины пока что
это не сильно меняет). Ну то есть. Например, если брать довоенный Мариуполь, то
идешь на Петровку или в книжный - то про Киев, в меньшей степени Львов, Одессу.
и т.д. - десятки, сотни книжек - бери и выбирай, какие хочешь. А вот про
Мариуполь - нету просто. А если их нет - значит люди не читают, не знают. В
общем, здесь большая проблема, как я вижу, потому что освобождение Мариуполя -
это ведь не только чисто военный аспект, это также вопрос духовного,
культурного возрождения, проблема вовлечения всей остальной Украины в процесс
возрождения Мариуполя, а наличие либо отсутствие книжек в свободном доступе -
это как раз маркер того, насколько успешно или неуспешно идет дело в этом
направлении. Парадокс - в интернете, фейсбуке - если покопаться - бездна
информации про Мариуполь, а вот книжек - очень мало. Заходишь в магазин -
видишь толстые тома про голод, архивы всяческие, на 1000 страниц, даже вот
видел недавно полный архив Малороссийского приказа в Книгарня Є - это все про
события уже 200 лет давности, наверное. А вот про то, что было совсем недавно
(не обязательно про войну, про мирный Мариуполь тоже нужны книжки) - совсем
мизер либо нет вообще.
