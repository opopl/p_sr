% vim: keymap=russian-jcukenwin
%%beginhead 
 
%%file 16_10_2021.fb.chistopolcev_oleg.1.istoria_park_shevchenko_mova_rebenok.cmt
%%parent 16_10_2021.fb.chistopolcev_oleg.1.istoria_park_shevchenko_mova_rebenok
 
%%url 
 
%%author_id 
%%date 
 
%%tags 
%%title 
 
%%endhead 
\subsubsection{Коментарі}

\begin{itemize} % {
\iusr{Sergiu Devdyk}
Але це батьки так ''ліплять''...

\begin{itemize} % {
\iusr{Олег Чистопольцев}
\textbf{Sergiu Devdyk} так. Але якби окрім батьків навколо був українськомовний простір, то наврядче б він так сказав

\iusr{Sergiu Devdyk}
\textbf{Олег Чистопольцев} , більшості зараз діти черпають з інету , а це чи контролюють батьки. Таке...

\iusr{Андрій Дворніченко}
\textbf{Sergiu Devdyk} 

типовий випадок мій 9річний син. У школі тільки класний керівник розмовляє українською.
Зі мною спілкується українською і раз на тиждень на пласті.
Мультики та фільми дома дивимось дубльовані українською.
Книжки зараз майже усі українською.
Але.
Між собою діти спілкуються російською, більшість слів, бо батьки російськомовні 95 відсотків.
У родині у нас із мамою також російською.
Тренера по айкідо, роботетехніці, плаванню, шахам також російськомовні.
І ютубконтент майже увесь, що малий дивиться інколи московитський  @igg{fbicon.frown} 
По футболу намагається українською розмовляти і репетитор англійської також намагається українською спілкуватись.
Українізація Дніпра тільки на початку  @igg{fbicon.frown} 

\iusr{Sergiu Devdyk}
\textbf{Андрій Дворніченко} 

вчора, малював з хлопцями карту, в мене їх двоє..., карту як з московії іде газ
по нашій території, і що московити зараз роблять. В кінці сказав , шо моск@лі
на скоро нас ввалять. Треба було бачити погляд старшого. З старшим в мене
легше, там фільтрація мосслів. + Розуміння свій , наш, чужий - вже ''привито''.
А от з меншим, там в розмові дуже багато мосслів, і все через ютюб..., оточення
у нас все на наші мові говорить(регіон такий).

Тому я включаю тяжку ''артилерію'' , вбитих мамів і татів московитами на сході,
показую дітей які лишились, але це коли ловлю, що в ютюбі дивиться московський
контент....

А також їм показую і розповідаю про наших Великих українців, про наші Перемоги,
про нашу історію, не ділячи українців по теренах .

Що зрозумів, що наш - чужий діє, і дитина сама оберає нашу, свою сторону....

\end{itemize} % }

\iusr{Ганна Собко}
жахлива історія!
і не соромно цьому манкурту просити "на руССкам"!
От що,, жахливо!
Поки їм не буде стрьомно визнавати, що не знають мови- ми програємо!

\begin{itemize} % {
\iusr{Олег Чистопольцев}
\textbf{Гандзя Собко} питання до дорослих
\end{itemize} % }

% -------------------------------------
\ii{fbauth.kuharchuk_anatolij.ukraina}
% -------------------------------------



На таке зазвичай відповідаю: що не розумію іншомовних , так як в Україні єдина
мова українська

\end{itemize} % }
