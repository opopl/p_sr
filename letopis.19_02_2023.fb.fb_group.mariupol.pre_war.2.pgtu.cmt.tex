% vim: keymap=russian-jcukenwin
%%beginhead 
 
%%file 19_02_2023.fb.fb_group.mariupol.pre_war.2.pgtu.cmt
%%parent 19_02_2023.fb.fb_group.mariupol.pre_war.2.pgtu
 
%%url 
 
%%author_id 
%%date 
 
%%tags 
%%title 
 
%%endhead 

\qqSecCmt

\iusr{Алла Алимова}

Старый корпус. Там учили \enquote{Теорию пластичности металла}

\iusr{Вячеслав Приходченко}

Мой родной ПГТУ, первое, а до самой войны, последнее место моей работы. Долгое
время были \enquote{Тяжмаш} и \enquote{Азовсталь}.

\begin{itemize} % {
\iusr{Viktor Lensky}
\textbf{Вячеслав Приходченко}. 

Мне очень нравится эта подписка и у меня есть чем поделиться с участниками. Но
я никак не пойму, как я могу добавить в неё свои фотки.

\iusr{Вячеслав Приходченко}
\textbf{Виктор Ленский} Витя, привет. Просто публикуй фото и тебя пригласят в группу.

\iusr{Viktor Lensky}
Да я, вроде уже в группе!

\end{itemize} % }

\iusr{Елена Аврамова}

Моя альма-матер!❤

\begin{itemize} % {
\iusr{Олег Присяжнюк}
\textbf{Елена Аврамова} І моя також !

\iusr{Tetiana Mukolaivna}
\textbf{Елена Аврамова} и моя альма-матер!
\end{itemize} % }

\iusr{Алексей Сапрыкин}

Моя кафедра промышленной теплотехники была в этом любимом корпусе...

\iusr{Alexandr Pliner}

А я закінчував ще ЖМІ у далекому 1977 році. Бувало на радянських демонстраціях
спочатку шов ЖМІ (Ждановський металургійний інститут), а за ним - ЖМУ(
Жданівське медичне училище).
