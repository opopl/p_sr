% vim: keymap=russian-jcukenwin
%%beginhead 
 
%%file 20_10_2021.fb.bilchenko_evgenia.3.chto_vy_mozhete
%%parent 20_10_2021
 
%%url https://www.facebook.com/yevzhik/posts/4360711703963906
 
%%author_id bilchenko_evgenia
%%date 
 
%%tags bilchenko_evgenia,rossia,rusmir,ukraina
%%title БЖ. Что вы можете
 
%%endhead 
 
\subsection{БЖ. Что вы можете}
\label{sec:20_10_2021.fb.bilchenko_evgenia.3.chto_vy_mozhete}
 
\Purl{https://www.facebook.com/yevzhik/posts/4360711703963906}
\ifcmt
 author_begin
   author_id bilchenko_evgenia
 author_end
\fi

БЖ. Что вы можете

"Что мы можем?" - самая распространенная фраза, которую я слышу от украинских
несогласных. Когда закрывали Россотрудничество, собиравшие по десять лайков в
сети своими проектами по вязанию пожилых людей крючком, в то время, как весь
Донбасс качало от рэпера Хаски, "Русское лето" в Воронеже сотрясали 25/17, а
МХАТ нес знамя Станиславского и Чехова так, что содрогалась красавица
драматургия Европы, вопрошали: "Что мы можем?"

\ifcmt
  ig https://scontent-lga3-2.xx.fbcdn.net/v/t1.6435-9/246781956_4360711643963912_5214610396075749570_n.jpg?_nc_cat=108&ccb=1-5&_nc_sid=8bfeb9&_nc_ohc=hzKVGzwnv7sAX9aTxVV&_nc_ht=scontent-lga3-2.xx&oh=e60506def7b2892de7ef6d4ed09f9192&oe=619532A7
  @width 0.4
  %@wrap \parpic[r]
  @wrap \InsertBoxR{0}
\fi

"Что мы можем?" - спрашивали они меня и тогда, когда я приносила им на хвосте
Эдичку и "Некоторые не попадут в ад", заходила туда в худи и джинсах и собирала
залы по новому прочтению Макса Волошина без унылого пацифизма. "Вы же нас под
СБУ подводите". Да, ребятня, я - сама боец, я и себя подвожу, кто не может,
отойди. Нормальное такое сопротивление без риска? Это как? 

"Как вы могли давать слово бывшей майдановке?" - орали бедной главе
Россотрудничества, моей покойной подруге, в меня поверившей, благонравные
небывшие и небудущие женщины, вяжущие крючком во имя спасения русского мира уже
много лет подряд, проклиная мой вечер под рок-гитары совместно с МХАТ, где
орали для молодежи русский поэтический авангард с яростным напором победителя,
а не с шамканьем ностальгирующей жертвы. Лайков это набрало немного больше, чем
их  вязание крючком. 

"Надо защищать не интересы России, а интересы Украины," - сказал мне один из
самых "пророссийских" моих друзей, а что делать мне, если я их не делю? Другой
товарищ долго определял свою идентичность "русский", чтобы в корень "Русь" не
вполз, не приведи Бог, корень "Росс". Третий принц сказал, что украинская
антифа оппозиция - на то и заукраинская, чтобы не быть зароссийской. То есть -
это опять полюса Фа: Украина и Россия. Яйца поменяли профиль. Один мой любимый
поэт решил отделить русский от россиян и продвинуть оный как исключительное
достояние Украины в условиях конфликта со "страной властелина колец". Короче,
мужу нравятся ноги жены, но голову надо срезать, на голове перхоть.

"Что мы можем?", - написали мне бывшие члены кружка по вязанию крючком,
умнейшая и тончайшая интеллигенция времён Ельцина и Горба, которая несла свет
культуры диалога, демократии и толе млин рантности вмережку с хрустом
французской булки в Россотрудничестве, пока стерненки не закрыли его благостную
толерантную калитку одной ногой.

"Что мы можем?", - спрашивают меня правозащитные структуры, которые, как
правильно сказала Oxana Chelysheva , на Украине являются сервисными
приложениями государства. Я могу объяснить, почему оппозиция карманная. В
американской колонии типа Чили, коей является наша страна, государство и
оппозиция кормятся из одного источника. Имя ему - транснациональный глобализм.
Государство кормится из правого соска, оппозиция - из левого. В странах типа
России оппозицию кормит Запад, а государство   - само себя, может себе
позволить, тут же мамка одна - Запад просто меняет маски.

"Что мы можем?" - вы можете, ребята, одно: без подмоги - только добровольно
лишить себя всего, пойдя на акт демонстративного неповиновения (по типу
жаннад'аркина смишного ритуального самоуничтожения, осмеиваемого вами за
наивность), самовозгореться по вопросу о России, но не оправдаться за нее и не
извиниться, тем самым показав другим, что так можно. За остальное бла-бла по
языку, диалогу and so on, здесь не сильна бьют, на этой полуправде (у вас она
называется "виважена позиція") можно кучеряво мотаться по грантам,  форумам,
курортам и эфирам. 

Так что ответ таков: таки, да, парни, вы ничего не можете. На домне можете
играть, правда: "во-первых, это красиво"...

"Девушка, я ваш принц на белом коне. - Вижу. А где принц?" \textbf{\#чтомыможемшоу}
