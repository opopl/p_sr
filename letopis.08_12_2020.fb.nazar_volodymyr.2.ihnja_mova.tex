% vim: keymap=russian-jcukenwin
%%beginhead 
 
%%file 08_12_2020.fb.nazar_volodymyr.2.ihnja_mova
%%parent 08_12_2020
 
%%url https://www.facebook.com/permalink.php?story_fbid=750219808917764&id=100017892540365
 
%%author Назар, Володимир
%%author_id nazar_volodymyr
%%author_url 
 
%%tags 
%%title Їхня мова не дає можливості організувати логічний зв'язок слів у реченні.
 
%%endhead 
 
\subsection{Їхня мова не дає можливості організувати логічний зв'язок слів у реченні.}
\label{sec:08_12_2020.fb.nazar_volodymyr.2.ihnja_mova}
\Purl{https://www.facebook.com/permalink.php?story_fbid=750219808917764&id=100017892540365}
\ifcmt
	author_begin
   author_id nazar_volodymyr
	author_end
\fi

\ifcmt
pic https://scontent.fiev6-1.fna.fbcdn.net/v/t1.0-9/130721566_750218958917849_906246596392174961_n.jpg?_nc_cat=101&ccb=2&_nc_sid=730e14&_nc_ohc=GoiYvyWzcm0AX97me6h&_nc_ht=scontent.fiev6-1.fna&oh=e8d39e527c437b81ee87ceb7b53931cb&oe=5FF62BE9
fig_env wrapfigure
width 0.5
\fi

Їхня мова не дає можливості організувати логічний зв'язок слів у реченні. Потім
самі слова у тій мові наповнені негативом. Не лікарня, де лікують, а
"больніца," де "болєют" і так без кінця. Українською важіль, бо ним можна
"підважити" щось важке. А що можна зробити російським "ричагом"- невже
"ричать"? чи "ригать"?! На українському ліжку можна лежати. А що робити на
російській "кроваті" ? Невже "кровіть"?! Ніякої логіки. А ще - саме через мову
формується агресивний психотип."Куда прьошся!?" Так "мамаша" кричить на свою
дитину. Але саме в народі "разгулялась" загадчівая русская" душа. Саме
"загадчівая" - від слова "гадіть". Щось на кшталт звичного діалогу: =" Нахрєна
дохрєна нахрєнячіл? Разхрєнячівай, нахєр! - Охрєнєл? Какого хєра
разхрєнячівать? - А тебе не по-хєр? - Да по-хєр."= Дійсно "велікий і могучій
язик", як в корови!
