%%beginhead 
 
%%file 16_03_2023.fb.kipcharskij_viktor.mariupol.1.r_k_tomu__den_21__16
%%parent 16_03_2023
 
%%url https://www.facebook.com/permalink.php?story_fbid=pfbid02vF329orDM6LSZJi1uY6EE4g8Zf8pobuWd3gTz6s7SADvP6V9aAeEuR69oseMYECil&id=100006830107904
 
%%author_id kipcharskij_viktor.mariupol
%%date 16_03_2023
 
%%tags mariupol,mariupol.war,dnevnik
%%title Рік тому: День 21. 16-03-22. Середа
 
%%endhead 

\subsection{Рік тому: День 21. 16-03-22. Середа}
\label{sec:16_03_2023.fb.kipcharskij_viktor.mariupol.1.r_k_tomu__den_21__16}

\Purl{https://www.facebook.com/permalink.php?story_fbid=pfbid02vF329orDM6LSZJi1uY6EE4g8Zf8pobuWd3gTz6s7SADvP6V9aAeEuR69oseMYECil&id=100006830107904}
\ifcmt
 author_begin
   author_id kipcharskij_viktor.mariupol
 author_end
\fi

Рік тому:

День 21. 16-03-22. Середа.

День ювілеїв: третій тиждень \enquote{гарячої} війни і другий тиждень без світла і
зв'язку.

0:20. Декілька дуже сильних вибухів з тремтінням стін та вигинанням вікон. З
жодного з вікон не видно ані диму, ані вогню. 

\ii{16_03_2023.fb.kipcharskij_viktor.mariupol.1.r_k_tomu__den_21__16.pic.1}

0:30. Вибухи трохи слабші за попередні - нібито від нас 

2:50. Дуже, дуже гучно - будинок хитає.

6:30. Пішли заводити Епіку. Від акумуляторів вона не завелася. Причепили до
Калини і я потягнув. У приватному секторі, не доїжджаючи калюжі крові вона
завелася. Розвернулися, аби не їхати туди, де були \enquote{прильоти}, під'їхали обома
машинами під під'їзд і почали виносити речі. Паковали машини впереміж: не
дивилися, чиї речі - наші чи дітей, а складали туди, де було вільне місце.
Паралельно якісь речі роздавали. Пуско-зарядний пристрій віддали Віталію.
Володя - батько Жені - попросив щось почитати і все ходив за мною, аби я не
забув. Я пропонував йому зайти і взяти самому, що захоче, але він відмовився.
Виніс торбу детективів. Оля дала йому мед, варення, залишок лимона для його
онука - сина Жені.

Невістка виглядає своїх батьків. Перепитує, що я сказав її батькові. Плаче. 

Оля з невісткою різали чи рвали якусь наволочку на  білі стрічки, аби пов'язати
на ручки дверей машин. Син скотчем ліпить на Епіку написи "Дети". В'яжемо
стрічки. Оля Ш. допомогає - в'яже свої з новенького простирадла.

Свати так і не прийшли. Напередодні сват казав, що вони не вирішили остаточно
чи поїдуть, але казав, що прийдуть провести нас і попрощатися. Невістка
боїться, що з ними щось сталося.

Я роздавав з багажника Калини \enquote{речі подвійного призначення}: шматки тканини
оливкового кольору, військовий наплічник, тощо.

Сватів все ще нема. Запропонував їхати комусь із сусідів: \enquote{Та ми не готові..} 

На задні сидіння Калини почали вантажити якісь речі без розбору.

9:45. (Скоріше за все, ми поїхали раніше, бо Драм знищили близько десятої).
Попрощалися з сусідами і поїхали:

Гонди - 7-й проїзд - Мазая - Куінджі - Миколаївська - Грецька - Підгірна -
Перша Слобідка - Приморський бульвар - Портовська - Великою Азовською на гору -
до кінця селища Моряків, аж за кінцеву - Мелекінська траса - Портовське (на
повороті картонка з написом від руки) - Мангуш.

28-ма школа стоїть нібито ціла: у правому крилі вибито вікно - мабуть лізли
мародери.

Біля автовокзалу стоять обгорілі автобуси. На Куінджі кілька вирв, обірвані
трамвайні дроти, уламки. Їхали дуже обережно, аби не пошкодити колеса вихляли з
одного боку дороги на інший, тому я не дивися на будинки.

\ii{16_03_2023.fb.kipcharskij_viktor.mariupol.1.r_k_tomu__den_21__16.pic.2}

Вирішили не їхати до Драму: якщо там буде скупчення машин, то раптом наліт -
утвориться тиснява. Краще зачекаємо десь на виїзді.

На перехресті Грецька-Миру подивився у бік Драму: за ним чорний дим - горить чи
то Дитячий світ (ПУМБ), чи то  гастроном Донецьк, чи то ЦУМ.

На Слободці перед нами дуже повільно їде вантажівка з кунгом (будкою). Повземо
за нею, бо вузька дорога не дозволяє випередили її. На Примбулі попрощалися з
морем...

На виїзді з Моряків вантажівка поїхала грунтовкою вздовж поля, а потім стала.
Ми, боячись мін, не стали розвертатися, а порачкували до останнього повороту. 

На блок-пості на виїзді з міста стоїть згоріла і вже руда від іржі БМП. 

Поки ми рачкували, вперед нас проїхало кілька машин: взагалі нас затисли - усі
намагалися швидше виїхати. Довелося \enquote{створити аварійно-небезбечну ситуацію} -
нагло стати поперек дороги і дати можливість виїхати сину.

На блок-посту перед Мангушем повзли годину і сорок хвилин. Вимикаю двигун -
майже одразу стає холодно, хоч і світить сонце. Вже майже доповзли до
блок-посту, як під'їхав російський військовий з автоматом (вони геть усі з
автоматами): "Чого ви копирсаєтесь! Горілка чекати не любить! Ось як треба!' і
почав махати рукою, аби проїжджали. Втім, одну з автівок за нами таки затримали
- я бачив по дзеркалах.

З'явився зв'язок - на телефон посипалися сповіщення \enquote{Вам ...надцять разів
телефонували...}.

Поїхали на Урзуф, але там дорога дуже погана, тож вирішили їхати кращою і
коротшою (бензину ж обмаль) на Токмак. На цьому виїзді дуже велика черга на
блок-пості. Довго стояли  без руху. Машини вистигли. Син зателефонував друзям і
йому порадили заночувати у Юр'ївці у Наді - знайомої з мотокросу.
Зателефонували їй. Вона вдома і зустріне нас на в'їзді, аби ми не плутали
селом.

(Набагато пізніше я дізнався, що саме того дня на виїзді з Мангушу автівку з
моїм товаришем з кінця 1970-х років по команді із спортивного орієнтування
Юрієм і його дружиною Людмилою розчавив російський БТР. Юрій помер у лікарні, а
Людмила з потрощеними ногами вижила... А якби ми поїхали не до Юр'ївки, а до
Мангушу???).

Надя нас зустріла, провела до свого дому і метнулася у магазин. У магазин!!! Це
ж так просто, але ми про таке вже забули: останні дні ми чули лише про те, що
магазини грабують...

14:00. Світло!!! Вода!!! Щоправда, немає газу, тому немає опалення. Надя
нагодувала нас щойно звареною (на кухні, а не на вогнищі (!) електроплита)
вермішеллю з тефтелями і напоїла гарячим чаєм. Зі свіжим хлібом!!! Ми спочатку
довго нюхали хліб, насолоджуючись забутим запахом...

Надя онукам дала великі порції, тож син доїв за онуком, а я за онукою - в
Маріуполі робив вигляд, що їсти не хочу...

Накололи дрова і запалили камін на другому поверсі - там стало тепло і
потягнуло на сон...

Надя поїхала зустрічати чергових гостей: Дмитро повернувся із Запоріжжя по
старших родичів і вивіз їх з Маріуполя. Завтра він поїде у Маріуполь ще за
кимось. Люди привезли пса Хаскі, якого закрили у сараї і молодого кота Семена,
який обнюхав усіх і почав гратися з онукою - нападати на неї.

По телевізору (два тижні без нього!) жахливі новини. 

(Помилка: спочатку знищили Драм, а Нептуні - після нього).

Десь за годину після того, як ми поїхали, розбомбили Нептун. Якби ми
залишилися, скоріше за все діти з онукою саме в цей час пішли б за щоденним
\enquote{дитячим пайком}. А в Нептуні до того ж приймали педіатри і інші лікарі і біля
нього стояли люди, очікуючи сигарети, їжу та новини... 

Ще десь за годину скинули дві півтонних бомби на Драмтеатр, у якому було більше
тисячі людей (напередодні волонтери отримали замовлення їжу для 1200 людей!),
а з заднього боку стояла польова кухня, біля якої стояло багато людей, очікуючи
на гарячу воду... 

Кажуть, що загинуло щонайменше 600 людей. У тому числі й ті, кого перевезли з
пологового будинку, дитячого та інших відділень розбомбленої Третьої лікарні...

16:30. Насолоджуємося тишою і спокоєм...

Пішов у туалет. Зайшов. Поліз у кишеню за ліхтарем. Отямився - вийшов і
увімкнув світло...

Невістка вимила волосся і висушила його феном: є електрика і працює
бойлер-водогрій...

18:50. Прийшло відчуття величезної провини перед тими, хто залишився...

Телефонував родичам - усі запрошують до себе. Написав у фейсбуці \enquote{Ми живі!} 

Вже майже у ночі зателефонував друг-рибалка Едік він помітив, що мій профіль
знову з'явився у фейсбуці і прочитав пост. (Перед виїздом ми повидаляли з
телефонів соцмережі, ніби це могло допомогти при догляді, або як потім почали
казати \enquote{фільтрації}...) Едік розповів, як вони їхали (вони поїхали 15-го), як
ночували у полі під Василівкою: у \enquote{сірій зоні} поліцейські повели їх
грунтовками, бо напередодні біля Степногорська колону накрили Градами. Настала
темрява і колона зупинилася, бо вночі без світла їхати вузенькою дорогою по
греблях було б самогубством, а їхати зі світлом означало б накликати на себе
обстріл Градами. Заборонили виходити з машин - скрізь міни.

За ніч конденсат від дихання у машині став кригою і наутро вони ледве відкрили
двері. Вони прошкрябали щілину у кризі на лобовому склі і так і поїхали, поки
скло не розмерзлося... Едік вимикав двигун на будь-якому схилі, бо бензину було
обмаль і купити не було де...

Написалося таке:

\obeycr
Полночь... Не спится... Лежу в тишине....
Разные мысли звенят в голове:
Что же ты бросил дом и друзей?
Вдруг им не хватает молитвы твоей?
Мы вместе деревья из балки таскали,
Мы "свой" павильон в детсаду защищали,
Валили акации в том детсаду....
И вот я я уехал...
Они - в том аду...
\smallskip
Мой разум все время твердит и твердит:
Мне за шестьдесят, и я - инвалид.
Ну что бы я сделал, оставшись в аду?
В копчёной кастрюле готовил еду
И байки травил  у костра во дворе?
И книжки давал почитать детворе?
Разумные вещи мой мозг говорит
Такого же хрена  сердце болит?
\smallskip
Мы были дружны, как большая семья.
Простите, соседи, что вас бросил я.
\restorecr

1 - дворова кухня. Фото сина.
2 - наш двір. Фото дружини.
3 - "звільнений" Нептун... Фото з інтернету.

%\ii{16_03_2023.fb.kipcharskij_viktor.mariupol.1.r_k_tomu__den_21__16.cmt}
