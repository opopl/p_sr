% vim: keymap=russian-jcukenwin
%%beginhead 
 
%%file 20_10_2021.fb.bilchenko_evgenia.4.prosjba_pomosch.cmt
%%parent 20_10_2021.fb.bilchenko_evgenia.4.prosjba_pomosch
 
%%url 
 
%%author_id 
%%date 
 
%%tags 
%%title 
 
%%endhead 
\subsubsection{Коментарі}

\begin{itemize} % {
\iusr{Сергей Николаев}

Не переживайте Евгения, Вы же доносы на батюшек не писали, как некоторые
депутаты, ну те которые лежали как живые.)))Так, что будете жить вечно)).Надеюсь
Вы поняли о чём я, хотя пост то Вы быстро удалили. Но смрад дерьма от Ваших речей
остался.

\begin{itemize} % {
\iusr{Евгения Бильченко}
\textbf{Сергей Николаев} 

ничего не поняла, ничего не писала, ничего не удаляла, не поняла ни слова из
сказанного. Ну, кроме слова "смрад речей моих" - довольно бодлеровская поэтика.
Вы бы излагали как-то внятнее, всем же интересно, не только мне.

\iusr{Сергей Николаев}
\textbf{Евгения Бильченко} лечись попрошайка))))

\iusr{Евгения Бильченко}
\textbf{Сергей Николаев} 

И опять мимо. Вот так вы все, оппоненты: если гадость сказать, так всегда
пожалуйста, а за слова свои ответить - зась да нит. Я искренно желаю вам, чтобы
вас не настигла карма: многие заболели из насмехающихся. Если не трудно,
объясните мне про мои доносы на батюшек, а то они интересуются, заодно и про
депутатов проясните. Я ж не лекарь осеннему обострению, чтобы с бредовых
ассоциаций смыслы выуживать. Так что извольте-с.

\iusr{Мирослава Александровна Бердник}

\textbf{Евгения Бильченко} , не вступай в дискуссию с ботом. Все батюшки помнят, как ты ПЦ на международных площадках в 2018-м году, когда были репрессии, защищала

\iusr{Евгения Бильченко}
\textbf{Мирослава Александровна Бердник} Так это, похоже, не наш батюшка, это что-то другое)
\end{itemize} % }

\iusr{Сергей Николаев}

Вы Евгения, лживый человечек. Так, что искренне надеюсь, что получите по
заслугам. Скрин к сожалению не делал Да и мне он ни к чему.

\begin{itemize} % {
\iusr{Евгения Бильченко}
\textbf{Сергей Николаев} 

очень зря, хоть перескажите, где ж я лгу, всем интересно, да и мне тоже. А по
заслугам верующий человек всегда получить рад: жизнь воспринимается им как
череда ошибок и покаяний, мужества и расплаты. Так что, вы не робейте,
перескажите. Можете даже скрины делать и собирать, это интересное для многих
занятие, я слышала). Есть такая пословица: "Не рой другому яму". Иисус тоже
говорил на эту тему. Но раз вы такой неверующий, то хоть скринами позабавьте. А
нет скринов, расскажите мне про депутатов: тянет мою душу на вторую серию про
агента Купера. Ну? Смелее.

\end{itemize} % }

\iusr{Евгения Бильченко}
Ну, то где ж я в борщ вам наклала, дражайший Сергей Николаев ?)) Всем же интересно.

\iusr{Мирослава Александровна Бердник}

Держись, заяц!

\end{itemize} % }
