% vim: keymap=russian-jcukenwin
%%beginhead 
 
%%file 09_03_2022.stz.news.ua.radiosvoboda.1.rosia_ce_zlo
%%parent 09_03_2022
 
%%url https://www.radiosvoboda.org/a/viyna-bizhentsi-volonter-ahresor-henotsyd/31744296.html
 
%%author_id tereschuk_galina
%%date 
 
%%tags 
%%title Те, що робить Путін, Росія ‒ це зло ‒ доброволець зі США
 
%%endhead 
 
\subsection{Те, що робить Путін, Росія ‒ це зло ‒ доброволець зі США}
\label{sec:09_03_2022.stz.news.ua.radiosvoboda.1.rosia_ce_zlo}
 
\Purl{https://www.radiosvoboda.org/a/viyna-bizhentsi-volonter-ahresor-henotsyd/31744296.html}
\ifcmt
 author_begin
   author_id tereschuk_galina
 author_end
\fi

«Не втікай. Захищай!» – написано на великому рекламному щиті перед воротами у
пункті пропуску «Шегині» (українсько-польський кордон).

\ii{09_03_2022.stz.news.ua.radiosvoboda.1.rosia_ce_zlo.pic.1}

Два тижні тут не припиняється автомобільний рух і нескінченний потік людей на
пішому переході. Переважно у черзі стоять жінки з дітьми. Дітей дуже багато.
Малеча ‒ загорнута на руках у матерів, старші дітлахи пересувають валізи чи
торбини. Чимало людей взяли своїх собак.

Люди приїжджають до пункту пропуску, що на Львівщині, з різних регіонів
України. До пішого переходу, а відкриті вони по всьому українсько-польському
кордоні, добираються самостійно або ж організованим транспортом зі Львова. Щоби
дійти до паспортного контролю, розповідають, простоюють у черзі від 5 до 7
годин.

\ii{09_03_2022.stz.news.ua.radiosvoboda.1.rosia_ce_zlo.pic.2}
