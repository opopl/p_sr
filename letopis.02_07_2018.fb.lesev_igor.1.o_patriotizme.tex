% vim: keymap=russian-jcukenwin
%%beginhead 
 
%%file 02_07_2018.fb.lesev_igor.1.o_patriotizme
%%parent 02_07_2018
 
%%url https://www.facebook.com/permalink.php?story_fbid=1978990638798696&id=100000633379839
 
%%author_id lesev_igor
%%date 
 
%%tags obschestvo,patriotizm,ukraina
%%title О патриотизме
 
%%endhead 
 
\subsection{О патриотизме}
\label{sec:02_07_2018.fb.lesev_igor.1.o_patriotizme}
 
\Purl{https://www.facebook.com/permalink.php?story_fbid=1978990638798696&id=100000633379839}
\ifcmt
 author_begin
   author_id lesev_igor
 author_end
\fi

О патриотизме.

В современной Украине много хороших, да и просто нейтральных понятий,
откровенно паскудятся. Волонтер. Активист. Журналист. И патриот.

Все эти слова в украинских реалиях стали новоязом. Казалось бы, что может быть
плохого в любви к Родине, традициям, культуре да и вообще, к своей
идентичности? Но в Украине любое светлое начинание губится в
гиперболизированной чрезмерности. А затем то, что изначально имеет значение со
знаком «+» девальвируется в самом гнусном виде.

\ifcmt
  ig https://scontent-frt3-1.xx.fbcdn.net/v/t1.6435-9/36485321_1978990435465383_5489142219182964736_n.jpg?_nc_cat=107&ccb=1-5&_nc_sid=730e14&_nc_ohc=bvauPs5xGQsAX_cFPXp&_nc_ht=scontent-frt3-1.xx&oh=49aa71f9e5b822813a0bb058f6dcc1f5&oe=61BA266B
  @width 0.4
  %@wrap \parpic[r]
  @wrap \InsertBoxR{0}
\fi

Любовь к Родине в украинской действительности трансформируется в серию
иррациональных культов, с непременно навязчивыми обрядами и их крайне
агрессивной защитой. «Мойте руки перед едой». Полезная, очевидная и вполне
нейтральная фраза. Но если ее возьмут в оборот украинские патриоты, они начнут
ломиться в офисы и выявлять тех, кто не помыл руки перед едой. Откроют сайт
«Рукотворец» и будут выставлять на нем личные данные подонков, не помывших
руки. Сложат речевки о мытье рук и будут кассирам в маркетах бросать мелочь в
лицо, обвиняя тех в незнании этих речевок.

В результате, для одной части общества «патриот» становится синонимом слова
«идиот». А те, кто считает себя патриотами, подменяют содержание иррациональной
и раздражающей окружающих формой. В результате, общество пребывает в
перманентном конфликте. Одни пытаются донести до несознательных светлые чувства
любви к Родине в том виде, в каком они это понимают. Другие отбиваются от всего
этого дерьма и просят гадить на своем огороде.

Так украинское общество становится некомфортным для проживания, а также крайне
непродуктивным. Украина не созидает внутри, а сама себя разрушает. Чистой воды
византийщина с ее уличными махачами партий «синих» и «зеленых». Мордобой ради
самого мордобоя. С институциональным разрушением государственности,
естественно.

Да, а выигрывают от всего этого исключительно политические временщики. Мы их
всех знаем поименно. Они – первые патриоты и любители Родины. Чаще в
неестественной форме, но это уже издержки. Другого способа, чтобы сравнительно
спокойно пилить и выводить в офшоры нацбогатство, не сталкивая при этом
«патриотов» и «непатриотов», пока что не существует. Если не будет взаимной
ненависти внутри общества, ТАК пилить уже не получится.

Как все это закончится? Вовсе не обязательно «национальной катастрофой». Хотя
бы потому, что мы УЖЕ живем в режиме национальной катастрофы. Просто еще
очередные минус 20 лет в развитии. И второе потерянное поколение подряд. А так,
ничего принципиально нового из того, что уже было в мире. Немножко гражданки,
как в Нигерии. Чуточку диктаторства, как в Уганде. Толика сегрегации, как в
ЮАР. Ну и экономический ориентир, как в Танзании.

\ii{02_07_2018.fb.lesev_igor.1.o_patriotizme.cmt}
