% vim: keymap=russian-jcukenwin
%%beginhead 
 
%%file 03_12_2021.fb.zaborin_dmitrij.1.den_ukr_borscha.cmt
%%parent 03_12_2021.fb.zaborin_dmitrij.1.den_ukr_borscha
 
%%url 
 
%%author_id 
%%date 
 
%%tags 
%%title 
 
%%endhead 
\subsubsection{Коментарі}
\label{sec:03_12_2021.fb.zaborin_dmitrij.1.den_ukr_borscha.cmt}

\begin{itemize} % {

\iusr{Serhij Vasylchenko}

На фото схоже на борщівник сосновського - гадость родом з Кавказу привезли до
нас десь в 40-х минулого століття.

\href{https://iz.ru/news/350529}{%
Как спастись от \enquote{мести Сталина}, Богдан Степовой, iz.ru, 09.07.2009%
}

\begin{multicols}{2}
В Москве началась тотальная война с борщевиком - мясистым растением, сок
которого вызывает ожоги и аллергические реакции. Во всех округах созданы
бригады по вырубке смертельно опасной травы, завезенной с Кавказа. Префектуры
просят жителей повысить бдительность и немедленно сообщать им по телефону обо
всех известных случаях произрастания борщевика. "Известия", уточнив пароли и
явки, отправились на войну со смертельно опасным сорняком

Борщевик появился в окрестностях Москвы в 1950-х годах - этим двух-трехметровым
растением с белыми зонтиками советские агрономы хотели накормить подмосковных
коров. Да только животные не захотели его есть. Плантации распахали. Но
борщевик крепко уцепился за нашу почву и скоро оккупировал все Подмосковье.

Областные колхозники еще во время развенчания культа личности дали
неискоренимому сорняку кличку "месть Сталина". Перелома в борьбе с пережитком
прошлого после этого, правда, не случилось.

- В парке Сходненская чаша есть две поляны, - дает вводную бригаде истребителей
главный специалист Красногорского спецлесхоза Виктор Гульбин. - Растение
ядовитое, поэтому надо работать в резиновых перчатках. Зеленную массу собирать
в мешки - позднее ее отправят на мусоросжигательный завод грузовиками. Техникой
безопасности не пренебрегать: из-за ожогов от борщевика за последние два года
погибло несколько детей в возрасте 2-3 лет. А в этом году подполковник
Кантемировской дивизии был доставлен с ожогом 70\% тела - отмахивался от мух и
слепней листом борщевика, приняв его за лопушиный. Военного откачали, но
состояние пациента было тяжелым.

После строгого инструктажа рабочие-истребители берут в руки косы, лопаты,
натягивают перчатки и спускаются по крутому откосу вглубь лесопарка. По мере
удаления от цивилизации растительность становится все более раскидистой и
сочной.

- Чем дальше в лес, тем толще борщевик, - мрачно шутим при виде поляны,
поросшей трехметровой травой.

Мужчины вооружаются секаторами, чтобы первым делом срезать цветки и сложить их
в мешки. На руку одного из них падает капля. Мужчина начинает нервничать.

- Где мыльный раствор! - кричит он. Коллеги приносят пластиковую баклажку с
мыльной водой и промывают руку.

- Не щиплет? - беспокоятся они. И ставят диагноз: - Ампутации удалось избежать.

На зачистку полянки площадью 20 на 30 метров уходит около часа, после чего
рабочие устраивают перекур и ждут грузовика, который должен забрать ядовитую
траву. Тема для разговора есть - два дня назад по всем телеканалам
рассказывали, как налетчики в Подмосковье ограбили инкассаторскую машину с 2,5
млн рублей. Оказывается, это была зарплата сотрудников Красногорского
спецлесхоза.

- Да там только водитель и пожилая бухгалтерша ехали, - возмущаются рабочие. -
Что они могли сделать? В водителя выстрелили из травматического пистолета,
женщина все сама отдала. А вот мы теперь сидим без копейки денег.

- Тьфу, - грустно, но дружно сплевывает бригада, показывая свое отношение к
произошедшему.

Впрочем, долго рассусоливать не приходится - на опушке слышится сигнал
старенького "ЗИЛа", который повезет борщевик на утилизацию. Рабочие взваливают
мешки на плечи и начинают погрузку. Обе поляны от сталинского наследия
зачищены.

P.S.

Рано или поздно бойцам "зеленого фронта" придется идти в поход еще на одного
врага. В Москве увлечены заменой старых почв, со всей их живностью и
растениями, на привозные. Грунт попадает в Москву вместе с семенами непривычных
для столицы растений. По мнению ученых, именно таким способом наши лужайки
заселяет конопля. И похоже, что это не последняя столичная дурь.

\textSelect{Римма Карписонова, главный специалист Главного ботанического сада РАН, доктор
биологических наук: \enquote{Это агрессор, а агрессора уничтожают}}

В 1950-е годы в Институте кормов Коми АССР усиленно искали культуры для
крупного рогатого скота. После ряда экспериментов местные специалисты и
остановились на Heracleum sosnowskyi, борщевике Сосновского, который
произрастал в субальпийских лугах Кавказа. Выбор был не случаен - это крупное,
хорошо растущее растение, в стеблях которого много витаминов. Борщевик начали
насаждать в колхозах средней полосы России, засеивая целые поля. Цветет он
красиво, поэтому следом его начали культивировать в городских парках. Но вскоре
выяснилось, что борщевик ядовит и что растение-агрессор, занимающее все
свободные места. Он быстро стал настоящей бедой, особенно для Подмосковья. А
бороться с агрессором, как известно, можно одним способом - уничтожать.

Но борщевик оказался живучим. Гербициды его не берут. Даже такой сильный, как
раундап, уничтожающий практически все известные сорняки, ему нипочем. Через год
плантации восстанавливались на старом месте, уже обработанном раундапом. На
данный момент существует единственный способ борьбы с борщевиком - это
выкапывание. Такой способ ужасно затратен и трудоемок, но ничего лучше пока не
придумали.

Для подростков и взрослых людей борщевик мало опасен, а вот за маленькими
детьми надо внимательно следить, объяснять им, что к этому растению лучше не
подходить. Если сок борщевика все же попал на кожу, это место надо тут же
промыть водой. Не помешает и обращение к врачу.
\end{multicols}

\iusr{Ирена Березник}

А при чем тут сорняк? «каждый готовит его дома сам у себя». Готовят сорняк?!
«это их повседневная еда и питье». Сорняк? Я понимаю так, что не покупают в
трактирах и харчевнях борщ, как еду, потому что готовят его дома. сами! Так же
как мы сейчас. Не идем есть в столовую то, что можем приготовить дома.

\begin{itemize} % {
\iusr{Дмитрий Заборин}
\textbf{Ирена Березник} как вы ловко делаете стратегические выводы из одной цитаты! любо-дорого

\iusr{Дмитрий Заборин}

Грюневег размышлял о ценах на борщевик, Ирина. У речки Борщаговки. Тогда везде
рос съедобный сорт этой травы и вот ее варили и кушали. А как "вы сейчас"
первое не варили, поскольку не было ни красной свеклы, ни помидоров, ни
картошки.

\iusr{Ирена Березник}
\textbf{Дмитрий Заборин},

во всяком случае из цитаты просматривается именно блюдо в готовом виде, а не
растение. Может, цитата неудачная.


\iusr{Дмитрий Заборин}
\textbf{Ирена Березник} может стоило бы открыть книгу? а то вы прям как Вятрович действуете

\iusr{Ирена Березник}
\textbf{Дмитрий Заборин} ,и я не говорю, что оно (блюдо) выглядело, как сейчас. Возможно, это блюдо готовилось из борщевика, но говорится именно про готовое блюдо, а не про растение!

\iusr{Александр Алексеевич}
\textbf{Дмитрий Заборин} Странно...
Что, может и Бандеры не было?

\iusr{Ирена Березник}
Не понимаю смеха. Кто-то видит в цитате речь о растении,а не о готовом блюде?

\iusr{Ирена Березник}
\textbf{Дмитрий Заборин},

я исхожу исключительно из приведенной вами цитаты. Покажите мне место, где там
говорится о растении, а не о готовом блюде, пофиг-из чего оно приготовлено?!

\iusr{Denis Dunaev}
Фототоксичность Править

Листья и плоды борщевика богаты эфирными маслами. Прикосновение к растениям
некоторых видов может вызывать раздражение и ожог кожи за счёт того, что все их
части содержат фуранокумарины — вещества, резко повышающие чувствительность
организма к ультрафиолетовому излучению. Самые сильные ожоги борщевик вызывает
при соприкосновении с кожными покровами в ясные солнечные дни. Но достаточно и
непродолжительного и несильного облучения солнцем участка кожи, испачканного
соком растения. Как правило, на поражённых участках кожи возникает ожог второй
степени (пузыри, заполненные жидкостью). Время проявления ожога — от нескольких
часов до нескольких суток. Особая опасность заключается в том, что
прикосновение первое время не даёт никаких неприятных ощущений. Борщевик также
является контактным и дыхательным аллергеном и имеет сильный запах, похожий на
керосин, который ощущается уже на расстоянии около пяти метров. Сок при
попадании в глаза может привести к слепоте[14]. Отмечены случаи потери зрения
детьми, которые играли с полыми стеблями растения как с телескопами

\iusr{Ирена Березник}
\textbf{Denis Dunaev},

и к чему это? Хоть как-то с цитатой в посте соотносится? Я знаю,что такое
борщевик.

\iusr{Дмитрий Заборин}
\textbf{Ирена Березник} 

Ваш приказной тон предполагает, что я что-то должен. А это не так. Сходите к
старине Гуглу и спросите, где лежит Мартин Груневег (отец Венцеслав): духовник
Марины Мнишек: Записки о торговой поездке в Москву в 1584–1585 гг. /
Составитель А. Л. Хорошкевич. М.: Памятники исторической мысли, 2013. - 384 с.
– стр. 161

\iusr{Ирена Березник}
\textbf{Дмитрий Заборин}, 

какой приказной тон? Вы о чем? Я же написала,что целиком ориентируюсь на вашу
цитату и исключительно на нее. И все выводы только от ее прочтения.

\iusr{Сергей Харичев}
\textbf{Дмитрий Заборин} Пытаетесь просветить? Думаю, это бесполезно: бездельникам на такие мелочи всегда не хватает времени

\iusr{Ирена Березник}
\textbf{Дмитрий Заборин},

я написала что-то смешное? Ок, может, я чего-то не вижу? Не нужно отсылать меня к
гуглу, покажите в приведенной вами цитате место, где речь идет конкретно о
растении, а не о блюде?



\iusr{Дмитрий Заборин}
\textbf{Ирена Березник} 

мне надоело объяснять, что обширный текст Груневега пересказан коротко, а в
данной цитате главное слово - \enquote{русские}, а не \enquote{борщ}. Хотите прочитать, что
речь про траву, а не блюдо - откройте книгу, я даже номер страницы дал. И не
выносите людям мосх.

\iusr{Геннадий Валериевич}
\textbf{Ирена Березник} Почему борщ так называется?

\iusr{Александр Мовчан}
\textbf{Дмитрий Заборин} Точно. Картошку только в 18 веке завезли. И садить её крестьян принуждали. Сопротивлялись они.
\end{itemize} % }

\iusr{Виталий Скиба}

С таким же успехом можно и праздновать День казачьей похлебки- когда козаки на
чайках уходили в поход они засыпали в горшки пшено, сверху заливали смальцем, а
после просто подогревали на огне, даже в море


\iusr{Вячеслав Хомяков}
Мiцний такий укрiпчик)))

\iusr{Vladimir Green}
Мдя.... )))

\iusr{Александр Карпец}
Не, ну а шо!
Не 8 Марта ж им праздновать! Будут праздновать борщ... Ещё надо День вареника, День холодца, День самогона и так далее по списку...)

\begin{itemize} % {
\iusr{Сергей Семенов}
\textbf{Александр Карпец} главный праздник - день сала.

\iusr{Александр Карпец}
\textbf{Сергей Семенов} Не, ну это само собой. Здесь даже обсуждать нечего )

\iusr{Зоя Бондаровская}
\textbf{Александр Карпец} Вот! Голосую за День вареника! С капустой!
\end{itemize} % }

\iusr{Martin Žour}
А выходной будет?  @igg{fbicon.wink} 

\begin{itemize} % {
\iusr{Дмитрий Заборин}
ну а как праздновать без выходного? фестиваль, поляны по центральной улице, по сто грамм. все как положено

\iusr{Martin Žour}
\textbf{Дмитрий Заборин} Бл..может попробовать в Чехии протолкнуть день кнедлика...


\iusr{Дмитрий Заборин}
отличная мысль, но нет ключевой проблемы: все и так знают, что \enquote{кнедлик наш} ))

\iusr{Martin Žour}
\textbf{Дмитрий Заборин} Не пройдёт ...тут ключевая проблема  @igg{fbicon.laugh.rolling.floor}  ,кнедлик из немецкого knodl...

\iusr{Юрий Ткачук}
\textbf{Martin Žour} а печенэ вэпровэ колено?
\end{itemize} % }

\iusr{Светлана Обидейко}
Хочу заметить, что этот самый борщевик и по сию пору растут вдоль Борщаговки. Которую зачем-то переименовали в Нивку.
Только очень редко нормальный - все больше встречается жгучий...

\iusr{Artem Ilyin}
Недавно приготовили блюда месопотамской кухни по рецептам с клинописных табличек:

\href{https://techno.nv.ua/popscience/priyatnogo-appetita-uchenye-rasshifrovali-chetyre-drevneyshih-recepta-v-istorii-50054528.html}{%
Приятного аппетита. Ученые расшифровали четыре древнейших рецепта в истории, %
Антон Ходоренко, %
techno.nv.ua, 20.11.2019%
}

\begin{multicols}{2}
Международная команда ученых, кулинарных историков и специалистов по клинописи
воссоздала четыре древнейших известных рецепта, найденных на клинописных
табличках.

Три древнейшие таблички датируются старовавилонским периодом, около 1730 г. до
н.э., четвертым — неовавилонским периодом, то есть на тысячу лет позднее,
отмечают авторы работы.

На каждой табличке есть несколько рецептов. Одна из трех самых старых
перечисляет коллекцию из 25 рагу, в основном записаны ингредиенты с краткими
инструкциями по приготовлению пищи. Две других таблички более подробны. Каждая
из этих табличек в течение тысячелетий страдала от повреждений, что усложняло
задачу приготовления подлинных древних вавилонских блюд.

Таблички экспонировались годами, но старые переводы клинописи нуждались в
переосмыслении, поскольку раньше артефакты считали медицинскими рекомендациями.
Кулинарные эксперты и специалисты по клинописи объединились, чтобы определить
некоторые из трав и других ингредиентов в рецептах, а затем методом проб и
ошибок они были воссозданы.

Команда из Гарварда хотела как можно ближе придерживаться рецептуры
оригинальных блюд. Это довольно амбициозная цель, учитывая, насколько скудны
некоторые из сохранившихся инструкций. Наличие ингредиентов также было
проблемой.

Авторы исследования подготовили три рецепта, все с одной таблички: два рагу из
ягненка — одно со свеклой, другое — с молоком и булочками — и вегетарианский
рецепт, обогащенный пивным хлебом.

\href{https://www.youtube.com/watch?v=qfqhJNUtiww}{%
Interdisciplinary team cooks 4000-year old Babylonian stews at NYU event, %
youtube, 13.06.2018%
}

\end{multicols}

\ifcmt
  tab_begin cols=3,no_fig,center

     pic https://i2.paste.pics/1413da566720b03694245502c76c0e76.png
		 pic https://i2.paste.pics/156572abf1a06fb9eb4f17635159f61f.png
		 pic https://i2.paste.pics/2b3443a0e9f7b36202f249d9f119c58c.png

  tab_end
\fi

\begin{multicols}{2}

Of the few thousand written clay tablets that survived from the ancient
Babylonian kingdom to today, only four are known to contain recipes. Those four
— which are about 4000 years old — reside at the Yale Babylonian Collection in
Sterling Memorial Library. Recently, two Yale researchers and colleagues from
other universities teamed up to cook three of these ancient recipes at "An
Appetite for the Past," an event at New York University that featured
historical recipes from across the world.

The team consisted of:

From Yale: Agnete Lassen, Associate Curator at the Yale Babylonian Collection
and Chelsea Graham, Digital Imaging Specialist at the Institute for the
Preservation of Cultural Heritage

From Harvard: Gojko Barjamovic, Senior Lecturer on Assyriology; Patricia Jurado
Gonzalez, Visiting scholar from the Basque Culinary Center in San Sebastián,
Spain; and Pia Sörensen, Senior Preceptor in Chemical Engineering and Applied
Materials and Nawal Nasrallah, culinary historian, author, and chef

Разнообразие ингредиентов, сложная подготовка и персонал, который готовит эти
блюда, позволяют предположить, что они предназначались для королевского дворца
или храма. В то время немногие повара умели читать клинопись, поэтому рецепты,
скорее всего, были записаны «для истории».

«Это событие дало нам возможность по-настоящему общаться с людьми того времени.
Повторяя некоторые процессы, которые они использовали бы для приготовления этих
рецептов, вы чувствуете себя ближе к культуре и людям, это помогает нам
рассказать их историю», — признают авторы исследования.

\end{multicols}

\iusr{Микола Лисенко}
,, Чернобыльский,, укроп!

\iusr{Валерий Одинцов}

В Украине надо какой-то день объявить днём Дурака ! Ветровичу, как главному
празднующему поручить составить список себе подобным и пусть празднуют ! Народ
думаю поможет составить список!

\iusr{Владимир Викторыч}
Нельзя их лишать последней радости ...

\iusr{Александр Климанский}
Зря они предложили. Кто нибудь попробует кулинарить из того, что найдет.

\iusr{Andrii Privaltsev}
Забавно.
В последнее время, когда читаю новости о событиях на Украине, все чаще вспоминаю: "Когда поганому кобелю делать нехер, он яйца лижет"... Ах, там ещё и суки...

\iusr{Николай Бабаджанян}
О! Даже могу угадать где этот шедевр был снят ))

\end{itemize} % }
