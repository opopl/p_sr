% vim: keymap=russian-jcukenwin
%%beginhead 
 
%%file 15_10_2021.vk.lgaki.1.anna_volvak_festival_saratov
%%parent 15_10_2021
 
%%url https://vk.com/wall-972076_32380
 
%%author_id lgaki
%%date 
 
%%tags lgaki,rossia,festival,multfilm,donbass,kultura,lnr,lugansk,saratov.rossia
%%title Анна Волвак - фестиваль в Саратове 
 
%%endhead 
\subsection{Анна Волвак - фестиваль в Саратове}
\label{sec:15_10_2021.vk.lgaki.1.anna_volvak_festival_saratov}

\Purl{https://vk.com/wall-972076_32380}

\ifcmt
 author_begin
   author_id lgaki
 author_end
\fi

Студентка третьего курса кафедры художественной анимации нашей Академии Анна
Вольвак вернулась из Саратова, где работала в составе жюри VIII Открытого
фестиваля-конкурса детского и юношеского кино «Киновертикаль».

— В прошлом году Анин авторский мультфильм «Что у кошки для чего» стал
победителем фестиваля в номинации «Свободный полет», и поэтому в нынешнем году
ее пригласили войти в состав жюри и лично присутствовать на фестивале, —
рассказала педагог кафедры и руководитель анимационной студии «Юла» нашей
Детской академии искусств Анна Вегера.

К слову, Анна Вольвак – выпускница «Юлы». А «Что у кошки для чего» она создала
всего за три месяца еще на первом курсе!

— В команду жюри набираются победители конкурса прошлого года. Участвовать в
работе этой команды – трудоемкий, но интересный опыт (Аня с коллегами
просмотрела 157 конкурсных работ! - ред.)! — говорит студентка. — Помимо эмоций
от самой поездки, фестиваль тоже оставил море впечатлений. Мегагостеприимные
организаторы, новые знакомства, премьера анимационного фильма Гарри Бардина
«Песочница»... В общем, пища для мозгов на долгое время вперёд! Спасибо
фестивалю и организаторам за впечатления, возможности и вдохновение на новые
работы!

Ждем новых мультфильмов Ани.

Фото из личного архива Анны Вольвак.

А мультик Ани \enquote{Что у кошки для чего} смотрите прямо тут! Он милейший!
