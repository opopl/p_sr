% vim: keymap=russian-jcukenwin
%%beginhead 
 
%%file 11_12_2020.news.world.cbc_ca.1.astrazeneca_sputnik_v
%%parent 11_12_2020
 
%%url https://www.cbc.ca/news/health/covid-19-vaccines-astra-zeneca-sputnik-v-1.5837992
 
%%author 
%%author_id 
%%author_url 
 
%%tags sputnik_v,covid,vaccine,astrazeneca
%%title AstraZeneca to test combining its COVID-19 vaccine with Russian one
 
%%endhead 
 
\subsection{AstraZeneca to test combining its COVID-19 vaccine with Russian one}
\label{sec:11_12_2020.news.world.cbc_ca.1.astrazeneca_sputnik_v}
\Purl{https://www.cbc.ca/news/health/covid-19-vaccines-astra-zeneca-sputnik-v-1.5837992}

Clinical trial announced to check for better efficacy of vaccine combination

\ifcmt
  pic https://i.cbc.ca/1.5819144.1606491748!/fileImage/httpImage/image.JPG_gen/derivatives/16x9_780/healthcare-coronavirus-astrazeneca-northkorea.JPG
  caption AstraZeneca says it will explore with Russia's Gamaleya Institute, which developed Sputnik V, whether two vaccines based on a common-cold virus could be successfully combined. (Dado Ruvic/Reuters)
  width 0.5
  fig_env wrapfigure
\fi

AstraZeneca is to start clinical trials to test a combination of its
experimental coronavirus vaccine with Russia's Sputnik V shot to see if this
can boost the efficacy of the British drugmaker's vaccine, Russia's public
investment fund said on Friday.

Trials will start by the end of the year and Russia wants to produce the new
vaccine jointly if it is proven to be effective, said the RDIF wealth fund,
which has funded Sputnik V.

AstraZeneca said it was considering how it could assess combinations of
different vaccines, and would soon begin exploring with Russia's Gamaleya
Institute, which developed Sputnik V, whether two vaccines based on a
common-cold virus could be successfully combined.

It did not give further details. However, its Russian arm said it would start
to enroll adults aged 18 and older for the trial.

\begin{itemize}
				\item \href{https://www.cbc.ca/news/canada/covid19-vaccine-rollout-plans-canada-1.5836262}{Here's the COVID-19 vaccine rollout plan, province by province}
				\item \href{https://www.cbc.ca/news/world/russia-vaccine-caution-1.5833611}{Russia's Sputnik V vaccination program has started, but it's facing resistance}
\end{itemize}

The cooperation between one of Britain's most valuable listed companies and the
state-backed Russian research institute highlights the pressure to develop an
effective shot to fight the pandemic, which has killed over 1.5 million people.

The move is likely to be seen in Moscow as a long-awaited vote of confidence by
a Western manufacturer in Sputnik V, which the Russian defense ministry alleged
on Friday was the target of a foreign-backed smear campaign.

Sputnik's Russian developers say clinical trials, still under way, have shown
it has an efficacy rate of over 90 per cent, higher than that of AstraZeneca's
own vaccine and similar to those of U.S. rivals Pfizer and Moderna.

Some Western scientists have raised concerns about the speed at which Russia
has worked, giving the regulatory go-ahead for its vaccines and launching
large-scale vaccinations before full trials to test Sputnik V's safety and
efficacy have been completed. Russia says the criticism is unfounded.

AstraZeneca, once seen as a frontrunner in the vaccine race, is preparing
further tests to confirm whether its shot could be 90 per cent effective,
potentially slowing its rollout.

Its average efficacy rate was 70.4 per cent in interim late-stage data — which
prompted the developers of Sputnik V to suggest trying to combine the two
vaccines.

\subsubsection{2 vaccines better than 1?}

RDIF head Kirill Dmitriev called AstraZeneca's acceptance of the proposal "an
important step towards uniting efforts in the fight against the pandemic."

Kate Bingham, chair of Britain's vaccine task force, said this week that the
U.K. would start trials next year using combinations of different kinds of
vaccine for the initial and booster vaccinations, in the hope that a
"mix-and-match" approach might maximize the immune response.

WHO officials give an update on coronavirus vaccine efforts:

\ifcmt
  pic https://thumbnails.cbc.ca/maven_legacy/thumbnails/1022/559/2020-12-11.jpg
  caption The World Health Organization is urging countries to help fill a funding gap of \$4.3 billion to help buy COVID-19 vaccines for poor- and middle-income countries. 1:06
  width 0.5
  fig_env wrapfigure
\fi

Both projects are using harmless adenoviruses as vehicles, or vectors, to carry
genetic instructions into the body to prompt cells to produce antibodies, an
approach that has previously been used in an Ebola vaccine.

One challenge of such a method is that the immune system could attack the
vector and, in particular, neutralize the second booster shot that is now an
important feature of the leading vaccine candidates to protect against
COVID-19.

\begin{itemize}
\item \href{https://www.cbc.ca/news/health/covid-19-vaccine-oxford-astra-zeneca-lancet-1.5832798}{AstraZeneca/Oxford, once a frontrunner in COVID-19 vaccine race, has work to do to catch up}
\end{itemize}

Using different viral vectors for the two shots is one approach that
researchers, including at the Gamaleya Institute, have pursued. Combining
vaccines from different developers could be another.

AstraZeneca did not mention immunity against the viral vector as an issue in
its statement on Friday.

The firm and its partner Oxford University have used a harmless adenovirus
found only in monkeys to ensure that people receiving the shot had not
previously been exposed to the vector and developed an immune response against
it.

Russian officials have not always been complimentary about the British vaccine.

When AstraZeneca paused a clinical trial in September due to the unexplained
illness of a volunteer, Kremlin spokesman Dmitry Peskov told reporters that
Sputnik V was more reliable because it was based on an adenovirus found in
humans, whereas the British candidate was a "monkey vaccine."

The new partnership may draw scrutiny after Britain said in July that hackers
backed by the Russian state were trying to steal COVID-19 vaccine and treatment
research from academic and pharmaceutical institutions around the world. The
Kremlin rejected the allegations.

\begin{itemize}

\iusr{Richar Gibbs}

Canada has jumped the gun on the pfizer vaccine. The mRNA method needs -70
freezing complications, increased cost and lipid nanoparticles are more unknown
variables added to the mix. It will not encourage the hesitant. Harmless
adenoviruses have been used since the smallpox vaccine and have more mileage.
It would make sense to de-politicize this work and media Russophobia is largely
to blame. Don't have to be a scientist to see that. 

\iusr{Reyter Frunze}

Funny stuff. This can be seen as pretty much confirmation the much maligned (in
the corporate media) Russian vaccine is the only safe and effective game in
town. Those Russian hackers who have been so interested in getting info about
western vaccines must feel pretty stupid about now. What an epic waste of time! 

\iusr{Thomas Wayne}

Guinness Beer and Smirnoff Vodka mixed together? why not... 
\end{itemize}
