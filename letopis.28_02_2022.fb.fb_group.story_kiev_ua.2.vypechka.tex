% vim: keymap=russian-jcukenwin
%%beginhead 
 
%%file 28_02_2022.fb.fb_group.story_kiev_ua.2.vypechka
%%parent 28_02_2022
 
%%url https://www.facebook.com/groups/story.kiev.ua/posts/1871383056391820
 
%%author_id fb_group.story_kiev_ua,dubinina_oksana
%%date 
 
%%tags 
%%title Друзья, у меня хорошая идея - давайте делиться хорошими рецептами простой выпечки
 
%%endhead 
 
\subsection{Друзья, у меня хорошая идея - давайте делиться хорошими рецептами простой выпечки}
\label{sec:28_02_2022.fb.fb_group.story_kiev_ua.2.vypechka}
 
\Purl{https://www.facebook.com/groups/story.kiev.ua/posts/1871383056391820}
\ifcmt
 author_begin
   author_id fb_group.story_kiev_ua,dubinina_oksana
 author_end
\fi

Совсем недавно я писала о нашем вкуснющем киевском хлебе. Друзья, у меня
хорошая идея. 

Пока в магазинах перебои с хлебом и выходить на улицу не рекомендуется, давайте
делиться хорошими рецептами простой выпечки, которую смогут сделать даже те,
кто не очень любит выпекать своими руками. 

Кстати, домашний хлеб не портится и очень даже вкусный. Его можно завернуть в
чистую хлопковую ткань и хранить в холодильнике довольно долго. А запах свежего
хлеба порадует наших домашних. 

\ii{28_02_2022.fb.fb_group.story_kiev_ua.2.vypechka.pic.1}

У многих есть хлебопечки, у кого же их нет, можно использовать любые формы для
выпечки, а еще просто формировать шарик из теста и надрезать его острым ножом,
как нашу любимую киевскую ПАЛЯНИЦЮ, чтоб она улыбалась.

*До речі, це те саме слово, що ніяк не можуть вимовити окупанти.))

***Бабушкин хлеб.

Состав: тёплая вода 400гр(можно напополам с молоком или сывороткой), масло
подсолн. 2ст.л., соль 2ч.л, сахара 1,5ст.л., пшеничной муки 3 полных стакана,
свежих около 1/3 пачки (если сухих дрожжей 4ч.л.)

Замешиваем в муку воду и остальные ингредиенты, подсолнечное масло поверх теста
для того, чтоб лучше подходило. Тесто должно быть живое, не жёсткое. Даём
подойти минут 30-40 и отправляем в горячую духовку (180градусов). Через минут
35-40 (зависит от духовки)  хлеб готов.

***Оладьи. Самые воздушные оладьи выходят из теста, в котором нет яиц. В кефир,
ряженку или йогурт (грамм 100-150) кладём ½ ч.л. соды, подождём пока сода
«погасится» и добавляем муки столько, чтоб была консистенция, как густая
сметана, щепотка соли и ст.л. сахара, чуть ванильки. Пусть тесто чуть постоит.
Жарим на разогретой сковороде на подсолнечном масле. Ням-ням. С чаем и вкусным
вареньем.

А ще я вітаю всіх з Масляною, адже завтра вже починається весна. А до Масляної
млиці – чудовий смаколик!

P.S. Если знаете короткие, хорошие и проверенные рецепты, делитесь)

\ii{28_02_2022.fb.fb_group.story_kiev_ua.2.vypechka.cmt}
