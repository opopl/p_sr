% vim: keymap=russian-jcukenwin
%%beginhead 
 
%%file 17_10_2017.stz.news.ua.mrpl_city.1.istoria_odniei_schaslyvoj_budivli
%%parent 17_10_2017
 
%%url https://mrpl.city/blogs/view/istoriya-odniei-shhaslivoi-budivli
 
%%author_id demidko_olga.mariupol,news.ua.mrpl_city
%%date 
 
%%tags 
%%title Історія однієї щасливої будівлі
 
%%endhead 
 
\subsection{Історія однієї щасливої будівлі}
\label{sec:17_10_2017.stz.news.ua.mrpl_city.1.istoria_odniei_schaslyvoj_budivli}
 
\Purl{https://mrpl.city/blogs/view/istoriya-odniei-shhaslivoi-budivli}
\ifcmt
 author_begin
   author_id demidko_olga.mariupol,news.ua.mrpl_city
 author_end
\fi

Безперечно, і в Маріуполі є будівлі, які мають щасливу долю. У них є цікава
історія, гарна архітектура, а, головне, вони добре збережені, користуються
попитом і навіть є прикрасою власного міста. Саме таким щасливим будинком є
бізнес-центр \enquote{Столєтній}, розташований на вулиці Миколаївській 7/17.

\ii{17_10_2017.stz.news.ua.mrpl_city.1.istoria_odniei_schaslyvoj_budivli.pic.1}

Завдяки дослідженням краєзнавців Л. Яруцькому, С. Бурову стало відомо, що
побудований будинок на початку ХХ століття. Першим господарем будівлі був пан
Оксюзов. Будинок виконував дуже важливу функцію, адже саме в ньому розмістилося
перше в Маріуполі реальне училище, завдяки якому маріупольці змогли отримувати
технічні спеціальності.

Справа в тому, що засновані Феоктистом Хартахаєм ще в 1876 р. чоловіча та
жіноча гімназії давали лише класичну освіту, після якої маріупольці могли стати
лікарями чи адвокатами. Проте на початку ХХ століття разом з будівництвом
залізниці, порту і двох металургійних заводів в Маріуполі все більше зростає
потреба в підготовці кадрів технічного спрямування. Відкриття в місті реального
училища стає необхідністю.

Ідея відкриття училища з'явилася у викладача чоловічої та жіночої гімназій В.
Гіацинтова та репетитора П. Бєляєва. Це питання було піднято на засіданні
міської думи 26 травня 1906 року і, завдяки клопотанню батьківського комітету
маріупольської Олександрівської чоловічої гімназії, було розглянуто позитивно.

\ii{17_10_2017.stz.news.ua.mrpl_city.1.istoria_odniei_schaslyvoj_budivli.pic.2}

Проте головна заслуга у відкритті училища належить Василю Івановичу Гіацинтову.
Він закінчив Санкт-Петербурзький історико-філологічний інститут. З 1901 по
1906-ті роки викладав історію і географію в маріупольській Олександрівської
чоловічій гімназії, а з 1904 року паралельно викладав і в Маріїнській жіночій
гімназії. Водночас Василь Іванович був гласним (депутатом) Маріупольської
міської думи, головою опікунської ради Маріїнської жіночої гімназії. А ще йому
дуже пощастило з дружиною, адже Надія Олександрівна була донькою дуже відомої в
Маріуполі людини Олександра Гозадінова, голови Маріупольської земської управи
(очолював її 25 років), одного з найбільших землевласників у повіті, який
опікувався цілим рядом маріупольських шкіл. Можливо, саме завдяки видатному
тестю Василю Івановичу вдалося втілити у життя ідею з училищем.

14 липня 1906 р. в місцевій газеті \enquote{Мариупольская жизнь} з'я\hyp{}вилося
оголошення про те, що в 1906 – 1907 навчальному році в м. Маріуполі буде
функціонувати приватне реальне училище, що відкриває В. І. Гіацинтов.

\ii{17_10_2017.stz.news.ua.mrpl_city.1.istoria_odniei_schaslyvoj_budivli.pic.3}

Це був середній навчальний заклад з практичною спрямованістю; основу викладання
становили предмети природничо-наукових і математичних циклів. Замість
давньогрецької і латинської мов викладалися німецька і французька. Програма
навчання була різноманітною. Так, історія зберегла звіт В.І. Гіацинтова про
екскурсію учнів Маріупольського реального училища і учениць Маріїнської жіночої
гімназії до Києва, Санкт-Петербурга, Москви.  Випускниками реального училища
стали відомі земляки маріупольців – грецький поет Георгій Костоправ і творець
двигуна танка Т-34 Костянтин Челпан.

\ii{17_10_2017.stz.news.ua.mrpl_city.1.istoria_odniei_schaslyvoj_budivli.pic.4}

Помер Василь Іванович Гіацинтов у 1918-му році. Маріупольське реальне училище
проіснувало до 1922 року, і було закрито через відсутність коштів на опалення.
З вересня 1922 року в коридорах і класах будинку №7/17 звучали не тільки голоси
хлопчаків, але й дівчаток. Тепер це була 3-тя трудова школа. Вчили хлопців
колишні викладачі реального училища і маріупольських гімназій.

Під час гітлерівської окупації у будинку нібито знаходилася якась казарма, а
при відступі німці спалили будівлю. Лише на початку 60-х років минулого
століття щойно утворений трест \enquote{Ждановжілстрой} відновив будівлю. У ньому
розмістилося управління трестом. На початку 90-х років житлове будівництво
різко пішло на спад. І трест \enquote{Ждановжілстрой} припинив існування. У будівлі
розмістилися різноманітні організації. 

\ii{17_10_2017.stz.news.ua.mrpl_city.1.istoria_odniei_schaslyvoj_budivli.pic.5}

Нещодавно історична будівля була дбайливо відреставрована завдяки зусиллям
сім’ї Сапарових. Сьогодні у будівлі розташовані важливі соціальні та політичні
установи і організації, а саме: Міжнародний комітет Червоного Хреста,
Управління земельних відносин, Центр психосоціальної адаптації, Секретаріат та
Організаційний відділ Маріупольської міської ради. 

\ii{17_10_2017.stz.news.ua.mrpl_city.1.istoria_odniei_schaslyvoj_budivli.pic.6}

26 грудня 2016 р. на будинку був встановлений вуличний годинник з церковним
боєм. Основний елемент годинника \enquote{Мирна дзвіниця}. Годинник з церковним
боєм був спеціально замовлений у республіці Польща. Складається з годинника
діаметром 1,5 метра, з коригуванням часу через сигнал GPS, а також зі
спеціальним відлитим бронзовим дзвоном, з нанесеним написом: \enquote{Маріуполь
з вірою і надією на мир}.

Поруч з будинком встановлена найбільша лавочка у місті, присвячена
мандрівникові, який повернувся або повертається до свого будинку, міста,
країни. На табличці біля лавочки є цікаве зауваження: \enquote{Звичайно ж,
ніхто не буде проти, якщо на цьому місці опиниться людина, яка і не думала
кудись їхати, а подорожує виключно в межах рідного Маріуполя}.

\ii{17_10_2017.stz.news.ua.mrpl_city.1.istoria_odniei_schaslyvoj_budivli.pic.7}

Ну що ж, дорогі маріупольці, якщо раптом вирішите подорожувати вулицями рідного
міста, не забудьте про щасливу будівлю, яка вже давно стала прикрасою старої
частини міста і розташована на вулиці Миколаївській 7/17.

\ii{17_10_2017.stz.news.ua.mrpl_city.1.istoria_odniei_schaslyvoj_budivli.pic.8}
\ii{17_10_2017.stz.news.ua.mrpl_city.1.istoria_odniei_schaslyvoj_budivli.pic.9}
