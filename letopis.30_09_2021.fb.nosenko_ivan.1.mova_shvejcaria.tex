% vim: keymap=russian-jcukenwin
%%beginhead 
 
%%file 30_09_2021.fb.nosenko_ivan.1.mova_shvejcaria
%%parent 30_09_2021
 
%%url https://www.facebook.com/vanekzp/posts/4296972640423802
 
%%author_id nosenko_ivan
%%date 
 
%%tags federacia,jazyk,mova,shvejcaria,ukraina,ukrainizacia,unitarnost
%%title В Швєйцарії чєтірі ґасударствєнніх і нічєво
 
%%endhead 
 
\subsection{В Швєйцарії чєтірі ґасударствєнніх і нічєво}
\label{sec:30_09_2021.fb.nosenko_ivan.1.mova_shvejcaria}
 
\Purl{https://www.facebook.com/vanekzp/posts/4296972640423802}
\ifcmt
 author_begin
   author_id nosenko_ivan
 author_end
\fi

@igg{fbicon.circle.red}  "В Швєйцарії чєтірі ґасударствєнніх і нічєво" - часто
можна почути від прихильників російської мови.

@igg{fbicon.circle.red}  Саме цю тему хотілось би детально розкласти по
поличках. Щоб пересічному українцю було чим аргументувати перед натовпом
невігласів.

@igg{fbicon.circle.red}  Багато хто з борців за російську будуть здивовані
коли взнають що з 4 офіційних державних мов у Швейцарії, немає Швейцарської
мови. 

Саме так, ви не помиляєтесь в Швейцарії немає Швейцарської мови. Ось вони 4 -
Німецька 62.6\%, Французька 22.9\%, Італійська 8.2\%, Ретороманська 0.5\%.

 @igg{fbicon.circle.red}  На питання мені "Ну почєму вот в Швєйцарії чєтірі ґасударствєнніх і нічєво"
- в мене одразу контр запитання "Дай відповідь яка різниця між Федеративністю і
Унітарністю, а потім я поясню чому у них так, а в нас інакше. І поясню чому нам
не підходить те що в них".

Ну звісно як це буває "руцкій брат" інтелектом сильно не обдарований і одразу
вам відповіді не надасть. Але ви надати зможете.

 @igg{fbicon.circle.red}  Чому я піднімаю питання походження устрою державності - тому що в мові це
важливо.

1. Федеративність - це форма державного устрою, за якої вищі територіальні
одиниці держави мають певну юридично визначену політичну самостійність.
Складові частини федерації — це своєрідні державоподібні утворення, які
називають суб'єктами федерації, а територія федерації складається з територій
її суб'єктів

2. Натомість Унітарна держава — єдина цілісна держава, територія якої
поділяється на адміністративно-територіальні одиниці, що не мають статусу
державних утворень і не володіють суверенними правами.

 @igg{fbicon.circle.red}  Чому це так важливо в порівнянні мовних питань між Швейцарією та Україною? 

Тому що Швейцарія — це федеративна держава — добровільне об’єднання трьох
різних мовних регіонів, кожний із яких складається з низки кантонів. Кантони до
моменту створення 1848 року Швейцарської федеральної держави були повністю
суверенними державами зі своїми власними кордонами, армією і валютою. Це
зумовлене географічним положенням Швейцарії. Її землі розташовані на перехресті
великих військових походів з півночі на південь та з заходу на схід, що
змушувало населення до відповідної поведінки.

 @igg{fbicon.circle.red}  Тобто умовно кажучи якби наша держава історично склалася з України, Італії,
Німеччини та Франції - думаю у нас також були б 4 офіційні мови. 

Але наша історія інша. Ми не склалися федеративно в кантони, з різних мовних,
культурно-ментально-політичних народів в один. росія та Україна не "Адін
народ".

На нас просто напала росія, окупувала, вела сотні років знищення. Насаджувала
свої правила, свою культуру, мову, тощо. 

Присутність російської мови в Україні - насаджена. Вона на відміну від
Швейцарії не зобумовлена союзами народів, в одну Федерацію. Наш народ - один.
Український. І крапка. Нацменшини є, корінні народи є, завезені росіяни також
є. Але переробляти для них кращі умови робити якісь держави чи напівдержави,
державні мови, федеративні ділянки в такий складний час для України ніхто не
буде.

 @igg{fbicon.circle.red}  Ситуацію в Україні не можна порівнювати з ситуацією у Швейцарії. В Україні
всі знають українську мову. А мовні спекуляції мають під собою політичний
ґрунт. Ними грають для руйнування одного з атрибутів державності, яким є мова.
А для "особливо обдарованих" — це абсурдне забезпечення права на незнання в
державі державної мови.

Тому не шановні російськомовні, якщо ви в силу низьких розумових здібностей, не
можете та/або не хочете опанувати мову держави в якій живете - це не говорить
про те що держава повинна йти на поступки, аби заохочувати ваше невігластво.

\textbf{\#мова}
