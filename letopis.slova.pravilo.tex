% vim: keymap=russian-jcukenwin
%%beginhead 
 
%%file slova.pravilo
%%parent slova
 
%%url 
 
%%author 
%%author_id 
%%author_url 
 
%%tags 
%%title 
 
%%endhead 
\chapter{Правило}

%%%cit
%%%cit_head
%%%cit_pic
%%%cit_text
Сергей Фурса написал длинный пост о том, что у украинцев не нужно ничего
спрашивать, потому что они ни в чем не разбираются. Это очень понятный и
рукопожатный пост – не спрашивать же, в самом деле, мнения людей, которые живут
в \emph{неправильных} регионах, разговаривают на \emph{неправильном} языке, голосуют за
\emph{неправильных} политиков, празднуют \emph{неправильные} праздники и имеют \emph{неправильных}
предков?  Но вслух об этом можно было и не говорить. Тем более что большинство
«реформ» в стране и так вполне официально предусматривают передачу прав и
полномочий международным структурам, которые «лучше знают как надо»
%%%cit_comment
%%%cit_title
\citTitle{В этой Украине живут какие-то неправильные украинцы}, Вячеслав Чечило, strana.ua, 07.07.2021
%%%endcit
