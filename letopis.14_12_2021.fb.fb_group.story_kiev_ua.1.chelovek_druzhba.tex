% vim: keymap=russian-jcukenwin
%%beginhead 
 
%%file 14_12_2021.fb.fb_group.story_kiev_ua.1.chelovek_druzhba
%%parent 14_12_2021
 
%%url https://www.facebook.com/groups/story.kiev.ua/posts/1818634351666691
 
%%author_id fb_group.story_kiev_ua,kirkevich_viktor.kiev
%%date 
 
%%tags cennosti,chelovek,druzhba
%%title Нет ничего дороже человеческого общения и многолетней крепкой дружбы
 
%%endhead 
 
\subsection{Нет ничего дороже человеческого общения и многолетней крепкой дружбы}
\label{sec:14_12_2021.fb.fb_group.story_kiev_ua.1.chelovek_druzhba}
 
\Purl{https://www.facebook.com/groups/story.kiev.ua/posts/1818634351666691}
\ifcmt
 author_begin
   author_id fb_group.story_kiev_ua,kirkevich_viktor.kiev
 author_end
\fi

Нет ничего дороже человеческого общения и многолетней крепкой дружбы. На склоне
лет - это ощущаешь всё острее и острее. Популярный поэт Евгений Винокуров
писал: «Жизнь оставь мне на старость, хотя бы для шашек друга». Мне повезло –
меня жизнь бросала в разные части не Земного шара, а только города Киева. Может
быть это и сыграло свою роль, хотя, скорее, это моя общительность и активная
жизненная позиция! И теряя старых друзей (за последний месяц – ПЯТЬ), я
приобретаю новых! И в шашки я не играю! Обычные традиционные застолья, где
главное беседы, иногда споры, но всегда уважение, почёт и понимание. Хотя и
бывает противоположные позиции и взгляды. Расскажу о нашей компании киевлян,
которой без малого 65 лет... Это не наши годы, а продолжительность нашей дружбы с
10-летнего возраста. Началось всё со школы № 94, которая располагается и сейчас
на Маяковского (Ольгинской). С Сережей Сторожевым мы там познакомились в 1955
году и сидели за одной партой. С Вовой Ходоровским, Толей Редько и Толей
Добровольским на пару лет позже... Главное в другом. Мы дружим по сей день и
находим возможности встречаться, но не в последнее время из-за напасти, которая
обрушилась на всё Человечество. У нас опасный возраст! Для «короны», а не для
женщин!

\begin{multicols}{2} % {
\setlength{\parindent}{0pt}

\ii{14_12_2021.fb.fb_group.story_kiev_ua.1.chelovek_druzhba.pic.1}
\ii{14_12_2021.fb.fb_group.story_kiev_ua.1.chelovek_druzhba.pic.2}
\end{multicols} % }

Наши регулярные встречи, где-то раз в месяц, продолжились после 2010 года,
когда я учился ходить после паралича конечностей, вызванного повреждением 2-го
шейного позвонка. Однокашники стали навещать меня и сложилась теплая компания:
музыкант Володя Ходоровский, инженер Толя Добровольский, лингвист Толя Редько и
полицейский Сережа Сторожев. О нём скажу особо – полковник, один из лучших
оперативников страны, он в последние годы преподавал оперативное дело в
Академии.

В нашей компании бывали и другие: одноклассники, друзья-приятели, но костяк -
наша пятерка, несмотря на встречающиеся небольшие размолвки был крепок, как и
напитки, которые мы употребляли. Единственное, что у нас было общее – мы не
курили. Это шутка! Нас объединяла многолетняя дружба и взаимопонимание! И ещё
любовь к Киеву, где мы все родились! Несмотря на почтенный возраст – мы хотели
жить… И вот у самого молодо выглядевшего и подтянутого, как–никак полковник,
Сергея Сторожева 11 декабря отказало сердце!

Больно, мучительно больно, но мы не можем его проводить в последний путь... У
каждого со здоровьем свои проблемы. Я после сильной травмы ноги – хожу только с
ходунками по комнате... Но я уверен, что мы соберемся с силами, встретимся и
помянем НАШЕГО БОЛЬШОГО ДРУГА!!!

\ii{14_12_2021.fb.fb_group.story_kiev_ua.1.chelovek_druzhba.cmt}
