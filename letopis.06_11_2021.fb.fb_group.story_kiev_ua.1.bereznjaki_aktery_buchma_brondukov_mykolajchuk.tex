% vim: keymap=russian-jcukenwin
%%beginhead 
 
%%file 06_11_2021.fb.fb_group.story_kiev_ua.1.bereznjaki_aktery_buchma_brondukov_mykolajchuk
%%parent 06_11_2021
 
%%url https://www.facebook.com/groups/story.kiev.ua/posts/1791576754372451
 
%%author_id fb_group.story_kiev_ua,sirota_tatjana.kiev
%%date 
 
%%tags akter,film,gorod,kiev,kiev.bereznjaki,kino,kultura,ukraina
%%title Березняки – актеры - Бучма, Брондуков и Миколайчук
 
%%endhead 
 
\subsection{Березняки – актеры - Бучма, Брондуков и Миколайчук}
\label{sec:06_11_2021.fb.fb_group.story_kiev_ua.1.bereznjaki_aktery_buchma_brondukov_mykolajchuk}
 
\Purl{https://www.facebook.com/groups/story.kiev.ua/posts/1791576754372451}
\ifcmt
 author_begin
   author_id fb_group.story_kiev_ua,sirota_tatjana.kiev
 author_end
\fi

Березняки – тихий, спокойный район, где живут десятилетиями, и многие местные
жители знакомы друг с другом.

Казалось бы, чем может быть примечателен этот небольшой жилой массив, кроме
того, что его можно считать курортным?

А ведь в этом районе проживали трое известных актёров.

\begin{multicols}{2} % {
\setlength{\parindent}{0pt}

\ii{06_11_2021.fb.fb_group.story_kiev_ua.1.bereznjaki_aktery_buchma_brondukov_mykolajchuk.pic.1}
\ii{06_11_2021.fb.fb_group.story_kiev_ua.1.bereznjaki_aktery_buchma_brondukov_mykolajchuk.pic.2}
\ii{06_11_2021.fb.fb_group.story_kiev_ua.1.bereznjaki_aktery_buchma_brondukov_mykolajchuk.pic.2.cmt}

\ii{06_11_2021.fb.fb_group.story_kiev_ua.1.bereznjaki_aktery_buchma_brondukov_mykolajchuk.pic.3}
\ii{06_11_2021.fb.fb_group.story_kiev_ua.1.bereznjaki_aktery_buchma_brondukov_mykolajchuk.pic.3.cmt}

\ii{06_11_2021.fb.fb_group.story_kiev_ua.1.bereznjaki_aktery_buchma_brondukov_mykolajchuk.pic.4}
\ii{06_11_2021.fb.fb_group.story_kiev_ua.1.bereznjaki_aktery_buchma_brondukov_mykolajchuk.pic.4.cmt}

\ii{06_11_2021.fb.fb_group.story_kiev_ua.1.bereznjaki_aktery_buchma_brondukov_mykolajchuk.pic.5}
\ii{06_11_2021.fb.fb_group.story_kiev_ua.1.bereznjaki_aktery_buchma_brondukov_mykolajchuk.pic.5.cmt}

\ii{06_11_2021.fb.fb_group.story_kiev_ua.1.bereznjaki_aktery_buchma_brondukov_mykolajchuk.pic.6.skver}
\ii{06_11_2021.fb.fb_group.story_kiev_ua.1.bereznjaki_aktery_buchma_brondukov_mykolajchuk.pic.6.cmt}

\ii{06_11_2021.fb.fb_group.story_kiev_ua.1.bereznjaki_aktery_buchma_brondukov_mykolajchuk.pic.7}
\ii{06_11_2021.fb.fb_group.story_kiev_ua.1.bereznjaki_aktery_buchma_brondukov_mykolajchuk.pic.7.cmt}

\end{multicols} % }

Улица Амвросия Бучмы, 8 - обычная 16ти этажка, построенная в 70-ом году
прошлого столетия... Здесь-то и жили со своими семьями известные актеры
Борислав Брондуков (с 1970 по 1985 год) и Маргарита Криницына (с 1970 по 2005
год).

Борислав Николаевич Брондуков - Народный артист Украинской ССР и первый лауреат
Государственной премии Украины имени Александра Довженко. 

Снялся более чем в 100 фильмах.

Известен, в основном, по эпизодическим ролям, где, как правило, выступал в
амплуа алкоголиков, проходимцев или недотёп, которые получались у него очень
достоверными и обаятельными. Это алкоголик Федул (\enquote{Афоня}, 1975), инспектор
Лестрейд (сериал \enquote{Приключения Шерлока Холмса}, 1979—1986), незадачливый жених
(«Гараж», 1979), фальшивый капитан Колбасьев (\enquote{Мы из джаза}, 1983), ковбой
(\enquote{Человек с бульвара Капуцинов}) и многие другие.

Борислав Николаевич отказывался от центральных ролей, если они ему не
нравились, и всегда мог, по его словам, из любого эпизода \enquote{сделать конфету}.

Несравненная Проня Прокоповна — Маргарита Криницына, по сути, стала актрисой
одной роли, хотя на её счету более семидесяти ролей.

Все дальнейшие роли у актрисы были хоть и запоминающимися, хоть и в очень
популярных фильмах, но все как одна эпизодическими — \enquote{Свадьба в Малиновке},
\enquote{Будни уголовного розыска}, \enquote{Зелёный фургон}, \enquote{Рождённая революцией}, \enquote{Одинокая
женщина желает познакомиться} и другие.

Одной из лучших театральных работ актрисы стал спектакль \enquote{Женщина с цветком и
окнами на север} в Киевском Театре-студии киноактёра. Роль Аэлиты имела
огромный успех у зрителей, о ней много писали критики. Но даже после этого
новых ролей на сцене не последовало, и Маргарита Васильевна ушла из театра.

Лишь в конце 1990-х годов, по достоинству оценив талант неповторимой актрисы,
настоящей королевы комедии, на неё посыпались награды, звания, её стали
приглашать в телешоу, создавать о её жизни и творчестве фильмы ("Ах, не
говорите мне про любовь", 2002). Но актриса была уже больным человеком, очень
плохо передвигалась.

23 августа 1999 года на Андреевском спуске в Киеве торжественно был открыт
памятник, в бронзе увенчавший героев фильма \enquote{За двумя зайцами} Проню и Свирида
Голохвостого. Герои Маргариты Криницыной и Олега Борисова навсегда застыли на
месте событий — напротив Андреевской церкви.

Дом под номером 5 по бывшей улице Серафимовича - здесь проживал Иван
Миколайчук, лауреат Государственной премии Украинской ССР им. Шевченко
(посмертно).

Судьба отмерила Ивану Миколайчуку короткую жизнь – 46 лет.

На его творческом счету – 34 кинороли, девять написанных сценариев, две
режиссерские работы. Еще больше осталось в планах, поскольку долгое время ему
просто не давали сниматься или снимать. 

В кино дебютировал еще студентом, в курсовой режиссерской работе Леонида Осыки
\enquote{Двое}.

Роли молодого Тараса Шевченко в фильме \enquote{Сон} и Ивана Палийчука в \enquote{Тенях забытых
предков} принесли Миколайчуку всеобщее признание.

Его называли лицом и душой украинского поэтического кино, аристократом духа,
блестящим самородком.

Иван Миколайчук был кинозвездой 60-70-х годов. Он был особенный, народный,
настоящий.

В его лице у украинской нации есть мировой бессмертный позитивный герой,
который пробуждал национальный дух украинцев.

В 1979-м Иван Миколайчук снял свой фильм \enquote{Вавилон ХХ}. Яркий, наполненный
фантастическими и, в то же время, реальными образами фильм вобрал в себя все
лучшее, что мог дать Иван Миколайчук - сценарист, режиссер, актер. Последней,
незаконченной работой Ивана Миколайчука стали \enquote{Небылицы об Иване}. В жизни
этого величайшего актера, режиссера и сценариста было все, чтобы стать легендой
не только национального, но и мирового кинематографа.

Улица на Березняках, где он проживал, и сквер, что расположен вдоль этой
улицы, сегодня носят его имя.


\ii{06_11_2021.fb.fb_group.story_kiev_ua.1.bereznjaki_aktery_buchma_brondukov_mykolajchuk.cmt}
