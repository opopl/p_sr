% vim: keymap=russian-jcukenwin
%%beginhead 
 
%%file 02_05_2021.fb.arestovich_aleksei.1.hristos_pas_ha
%%parent 02_05_2021
 
%%url 
 
%%author 
%%author_id 
%%author_url 
 
%%tags 
%%title 
 
%%endhead 
\subsection{In Nomine Patris Et Filii ET Spiritus Sancti}
\label{sec:02_05_2021.fb.arestovich_aleksei.1.hristos_pas_ha}
\Purl{https://www.facebook.com/alexey.arestovich/posts/4288737871190260}

\ifcmt
  pic https://scontent-bos3-1.xx.fbcdn.net/v/t1.6435-9/179951711_4288736934523687_3258163524308427305_n.jpg?_nc_cat=108&ccb=1-3&_nc_sid=8bfeb9&_nc_ohc=17P0M9Vs7uUAX85RK1m&_nc_ht=scontent-bos3-1.xx&oh=875907e557fc9df4bc2c46970af13918&oe=60B750B4
\fi

Сегодня Пасха у христиан восточного обряда. 

Однажды я общался с очень хорошим человеком (исламским радикалом), на тему содержания их и нашей веры.

Он мне говорит:

- Ожесточенность нашего сопротивления миру во многом объясняется тем, что
боимся потерять душу, растворенную в "Кока-коле" (тм).  Этим объясняется и наша
насторожённость и прямой запрет на спекулятивную философию и теологию.

К индустриальному барьеру 1500 года, подошло шесть цивилизаций -
восточно-православная, китайская, Османский халифат (шире - исламская),
цивилизации Центральной и Южной Америки (ацтеки, майя, инки), индийская и
западно-христианская. 

Наибольшие шансы взять индустриальный барьер и создать Модерн, были у китайской
цивилизации и Османского халифата.

Обе они к этому моменту уже совершили свои Великие географические открытия.

Однако, новый мир создала цивилизация, у которой формально шансов было едва ли
не меньше, чем у всех других - западно-христианская.

Тайна вечной силы Запада заключалась в специфике развития христианской теологии
и социальной доктрины. 

А тайна этих практик, в свою очередь состоит в христианской онтологии.

Дело в том, что христианская онтология это онтология Абсолютного Субъекта,
воплощенного в человеке.

А это, в свою очередь, означает абсолютную ответственность за все, что
делается, логику абсолютного действия.  

Логика абсолютного действия вызывает к жизни стратегию крайностей

- через попытки доведение любой цивилизационной тенденции до ее абсолюта. 

Если, другие цивилизации человечества пытаются разумно сочетать традицию и и
прогресс, или просто противостоять прогрессу для сохранения традиции,
западно-христианская цивилизационная лодка всегда закладывала галсы на крайних
положениях руля.

Стратегия специфическая: руль вправо - несколько десятков миллионов человек за
борт, руль влево - ещё десятки миллионов. 

Стратегия давала огромные возможности и выдающиеся достижения, творила новое,
создавала иное. 

В конце концов, «народ христианский», как столетиями называли себя европейцы,
решили «...обойтись без этой гипотезы», отказались от Христа.

Произошло это при Просветителях, которых когда-нибудь обязательно переименуют в
Затемнителей: Разум был объявлен Божественным, а Бога (видимо, с целью убрать
конкурента) обьявили «несуществующим».

Лодка поплыла по инерции, в туман заблуждений.

Через триста лет, к середине 20 века Европа выдохлась. 

Запад сегодня - уже не тот Запад, что брал цивилизационные барьеры один за
другим: - кволое, мягкотелое общество на грани вырождения. 

Пока ещё держатся США, секрет которых как раз в повышенной религиозности большой массы народа и лидеров. 

То, чего боялись мусульмане, произошло с Западом - он растворился в
«Кока-коле», феминизме, популизме и всем том, от чего пытается сохранить себя
исламская цивилизация.

Но есть тайна. 

Тайна эта заключена в том, что в нашей онтологии Вечного Человека, глубоко,
очень глубоко, в самом фундаменте представлений об этом мире, есть такая вещь,
как Воскресение-при-этой-жизни.

Мы можем позволить себе чрезвычайно рискованное поведение, венчурные стратегии,
потому что мы знаем: 

- нас можно убить, но через три дня мы вернемся.

И начнем снова.

Воскресение принципиально выделяет нас среди других братьев по человечеству.

В своих рискованных стратегиях, мы прибегли к наиболее рискованной - попытались
обойтись без Бога, как такового.

Но, трехсотлетний грандиозный эксперимент по отмене Бога завершается.

Блудные дети уже возвращаются домой, и новое вино готово налиться в новые мехи.

Мы не можем представить, как и в каком образе Вечный Человек вернется к нам снова. 

Но, я знаю наверняка:

- в чем-то абсолютно новом, неожиданном, ином; в чем-то - вечно том самом, что
две тысячи лет назад шел босиком горячими дорогами Иудеи. 

Христос воскрес и мы воскреснем. 

Вечная история вечно нова и будет новая история - чтобы повториться.
