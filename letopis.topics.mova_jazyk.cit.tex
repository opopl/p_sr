% vim: keymap=russian-jcukenwin
%%beginhead 
 
%%file topics.mova_jazyk.cit
%%parent topics.mova_jazyk
 
%%url 
 
%%author_id 
%%date 
 
%%tags 
%%title 
 
%%endhead 

%%%cit
%%%cit_head
%%%cit_pic
%%%cit_text
Особливість українських реалій в тому, що саме МОВА тривалий час заступила
собою відсутність Української Держави. Питання мови було питанням влади і
панування. Не комунікації. Саме звідси сотні указів заборон української мови з
боку держав окупантів: Московії, Польщі, Румунії, Угорщини, що через фізичне і
моральне знищення українців намагалися панувати на споконвіку наших етнічних
землях.  В основі нашої розвідки – сучасний терор Москви проти української мови
– та її жертви і Герої водночас. Не було б тут російської мови як наслідку
історичної окупації України Московією – не було б сьогодні війни, бо Путін не
мав би кого захищати. Він каже про це сам: "Наголошу, що ми були змушені
захищати російськомовне населення на Донбасі, змушені були відреагувати на
прагнення людей, що живуть в Криму, повернутися до складу РФ" (Путін, 12 жовтня
2016) [3].  Тобто словосполука "російськомовне населення" – це "частина воєнної
доктрини Путіна" [1]. У нас НЕМА російськомовного населення. Є етнічні росіяни
– нащадки окупантів, сьогоднішні окупанти і з'яничарені українці – жертви
московського терору.  Що маємо сьогодні на окупованих теренах?
%%%cit_comment
%%%cit_title
\citTitle{МОВА СЬОГОДНІ – ЦІНА ЖИТТЯ}, 
Ірина Фаріон, blogs.pravda.com.ua, 09.11.2021
%%%endcit

%%%cit
%%%cit_head
%%%cit_pic

\ifcmt
  tab_begin cols=3
     %pic https://umoloda.kyiv.ua/img/content/i140/140521w540.webp
		 pic https://1news.com.ua/wp-content/uploads/2020/11/7-11.jpg
     pic https://avatars.mds.yandex.net/i?id=c5dfc0dded249d4c6fa31763d72eaa1e-3936338-images-thumbs&n=13
		 pic https://img.strana.news/img/article/3436/aleksandr-dolzhenkov-raskritikoval-62_main.jpeg
  tab_end
\fi
%%%cit_text
Суддя Конституційного суду України Ігор Сліденко запідозрив екснардепа 8-го
скликання від "Опозиційного блоку" Олександра Долженкова у цитуванні воєнної
доктрини президента Росії Володимира Путіна.  Свої претензії з цього приводу
суддя озвучив під час розгляду закону про українську мову на предмет
конституційності за поданням 51 парламентаря.  Зокрема, Сліденка обурило
формулювання політика про "російськомовне населення" України.  "Чи відомо вам,
що термін "російськомовні громадяни" є частиною воєнної доктрини Путіна?" –
звернувся суддя до Долженкова та процитував заяви лідера країни-агресорки про
нібито необхідність захисту "російськомовного населення" в окупованому Криму та
на Донбасі.  Колишній нардеп, зі свого боку, заявив, що йому про це нібито
нічого невідомо, оскільки у своїх претензіях він керується лише українським
законодавством та Конституцією України. Після цього слово знову взяв Сліденко
%%%cit_comment
%%%cit_title
\citTitle{В КСУ вважають термін «російськомовне населення» цитуванням воєнної доктрини Путіна}, 
Оксана Середюк, umoloda.kyiv.ua, 07.07.2020
%%%endcit
