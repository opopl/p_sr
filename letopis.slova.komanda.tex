% vim: keymap=russian-jcukenwin
%%beginhead 
 
%%file slova.komanda
%%parent slova
 
%%url 
 
%%author 
%%author_id 
%%author_url 
 
%%tags 
%%title 
 
%%endhead 
\chapter{Команда}
\label{sec:slova.komanda}

%%%cit
%%%cit_head
%%%cit_pic

\ifcmt
  tab_begin cols=3
   	 width 0.3

     pic https://storage.lug-info.com/cache/5/e/583deb92-0e9e-4833-a8ab-5a80b96584d9.jpg/w1000h616
		 width 0.24

     pic https://storage.lug-info.com/cache/4/4/b0618e81-14da-4138-9ff1-1a1e5507f58c.jpg/w1000h616
		 width 0.28

		 pic https://storage.lug-info.com/cache/3/4/cbcfc12d-37a4-45ff-bfdb-3b82835fca07.jpg/w1000h616
		 width 0.2

  tab_end
\fi

%%%cit_text
\emph{Команда} из Ровеньков стала призером военно-исторического форума
\enquote{Александровский стяг}, который состоялся в Санкт-Петербурге. Об этом
ЛИЦ сообщил председатель общественной организации \enquote{Патриотическая
ассоциация Донбасса} (ПАД) Алексей Селиванов \enquote{В Санкт-Петербурге
завершился ежегодный, одиннадцатый по счету, Всероссийский молодежный
военно-исторический форум \enquote{Александровский Стяг} имени святого
благоверного князя Александра Невского. Традиционно в форуме приняли участие
команды Народных Республик Донбасса, ЛНР представили \emph{команды} ПАД и
\enquote{Гвардия} Ровеньковской гимназии №1 имени Шонина}, - сказал Селиванов
%%%cit_comment
%%%cit_title
\citTitle{Луганский Информационный Центр – Команда из Ровеньков стала призером
военно-исторического форума \enquote{Александровский стяг}}, , lug-info.com,
04.07.2021
%%%endcit

%%%cit
%%%cit_head
%%%cit_pic
\ifcmt
tab_begin cols=2
	width 0.4
  pic https://strana.ua/img/forall/u/0/92/%D1%80%D0%B0%D0%B7%D0%B3%D1%80%D1%83%D0%B7%D0%BA%D0%B0(1).jpg

	pic https://strana.ua/img/forall/u/0/34/207941474_2503568363110874_3886265068086152222_n.jpg
tab_end
\fi
%%%cit_text
Выйдя из поезда Петр Алексеевич весь в черном и \emph{команда} направились в
новоиспеченный Нью-Йорк (переименованный поселок Новгородское). По дороге он
заехал в штаб 10-й горно-штурмовой бригады и передал бригадный комплекс, сетку
и пять генераторов, упомянув, что купили их в \enquote{Розетке} с большой скидкой.  На
месте лидер \enquote{Евросолидарности} показал, как умеет организовывать работу и стал
командовать выгрузкой свертков и коробок из машины
%%%cit_comment
%%%cit_title
\citTitle{Гетьман в Нью-Йорке. Как Порошенко устроил черешня-тур на Донбасс}, 
Оксана Малахова, strana.ua, 06.07.2021
%%%endcit

%%%cit
%%%cit_head
%%%cit_pic
%%%cit_text
І в цьому головна роль належить найбільш авторитетній людині в \emph{команді} –
її головному тренеру Андрієві Шевченку. Якщо би він повірив, що спільне
прийняте рішення (наприклад, щодо вивчення української мови) може якісно
збільшити згуртованість та ефективність \emph{команди}, та прийняв відповідне
рішення, ми б мали якісно кращу \emph{команду} та якісно кращий футбол на
чемпіонатах, причому вже за декілька місяців!  Залишається побажати
\emph{команді} прислухатись до цієї пропозиції! Ну, й зараз якнайшвидше
відновитись після виснажливого марафону, проаналізувати отриманий досвід,
провести роботу над помилками і почати готуватись до нових викликів!
%%%cit_comment
%%%cit_title
\citTitle{Чи можна примусити збірну України розмовляти державною мовою?}, 
Євген Лапін, www.pravda.com.ua, 06.07.2021
%%%endcit
