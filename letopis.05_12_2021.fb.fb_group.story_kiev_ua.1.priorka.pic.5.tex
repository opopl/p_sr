% vim: keymap=russian-jcukenwin
%%beginhead 
 
%%file 05_12_2021.fb.fb_group.story_kiev_ua.1.priorka.pic.5
%%parent 05_12_2021.fb.fb_group.story_kiev_ua.1.priorka
 
%%url 
 
%%author_id 
%%date 
 
%%tags 
%%title 
 
%%endhead 

\ifcmt
  ig https://scontent-frx5-1.xx.fbcdn.net/v/t39.30808-6/264682859_964745664115136_3369506995756398970_n.jpg?_nc_cat=100&ccb=1-5&_nc_sid=b9115d&_nc_ohc=C3hf6140RKUAX8YBT2Z&_nc_ht=scontent-frx5-1.xx&oh=a4cbbaba538f548a40e688a2fd57d68d&oe=61B1CC41
  @width 0.4
\fi

\iusr{Maksym Oleynikov}

Весна 1983 року. Вулиця Мостицька. Праворуч приватні будинки №№ 6 і 8, знесені
у 1987р. За новою 9-поверхівкою видно Покровську церкву.

\iusr{Ksenia Guzeeva}
Да .
Козья тропинка была ..
А сейчас дале представить сложно, глядя на нынешнюю улицу .

\iusr{Богдан Король}

Праворуч якраз багатоповерхові будинки "готельного" типу №№ 6 і 8. Приватні
будинки були іншої нумерації. До речі, в будинку №6 (другий від церкви)
народився та виріс Олімпійський Чемпіон України - Жан Беленюк!
