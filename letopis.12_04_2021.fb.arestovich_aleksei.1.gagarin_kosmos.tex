% vim: keymap=russian-jcukenwin
%%beginhead 
 
%%file 12_04_2021.fb.arestovich_aleksei.1.gagarin_kosmos
%%parent 12_04_2021
 
%%url https://www.facebook.com/alexey.arestovich/posts/4231784113552303
 
%%author 
%%author_id 
%%author_url 
 
%%tags 
%%title 
 
%%endhead 
\subsection{Космос - Отказ - Путь - Идем - Надежда}
\label{sec:12_04_2021.fb.arestovich_aleksei.1.gagarin_kosmos}
\Purl{https://www.facebook.com/alexey.arestovich/posts/4231784113552303}

\ifcmt
  pic https://scontent-bos3-1.xx.fbcdn.net/v/t1.6435-0/p180x540/172628691_4231888986875149_5681543353046567731_n.jpg?_nc_cat=111&ccb=1-3&_nc_sid=8bfeb9&_nc_ohc=Hj79zHA1LN8AX-yXu5N&_nc_ht=scontent-bos3-1.xx&tp=6&oh=f125bf44779962108b2c7784cdf26f3a&oe=609C4BA4
\fi

- Шестьдесят лет назад родилась великая надежда и великая гордость за
человечество и Человека, когда маленькие былинки полетели в лучах света за
пределы колыбели.

В 1975 она начала угасать: коммерческий космос, ближний космос, орбитальный
космос.

Ночь беспамятства длилась почти сорок лет.

Нет, проблема не в том, что мы не могли лететь. 

И даже не в том, что отказывались.

Проблема в том, что мы оправдывали Отказ. 

Пятидесятилетие космонавтики, десять лет назад, мы всей колыбелью встречали без
особой надежды, больше с сожалениями.

Но сегодня, есть чем гордиться снова.

Мы возвращаемся к дальней пилотируемой космонавтике.

Не страны - люди.

Снова полетят маленькие коробочки в бескрайние океаны света, понесут в ладонях
людское - к звёздам.

А значит, мы ещё на Пути.

- Идём.
