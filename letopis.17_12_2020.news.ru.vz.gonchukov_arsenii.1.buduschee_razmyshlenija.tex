% vim: keymap=russian-jcukenwin
%%beginhead 
 
%%file 17_12_2020.news.ru.vz.gonchukov_arsenii.1.buduschee_razmyshlenija
%%parent 17_12_2020
 
%%url https://vz.ru/opinions/2020/12/17/1075269.html
 
%%author Гончуков, Арсений
%%author_id gonchukov_arsenii
%%author_url 
 
%%tags mysli,buduschee
%%title Будущее – не то место, где нужно искать надежду
 
%%endhead 
 
\subsection{Будущее – не то место, где нужно искать надежду}
\label{sec:17_12_2020.news.ru.vz.gonchukov_arsenii.1.buduschee_razmyshlenija}
\Purl{https://vz.ru/opinions/2020/12/17/1075269.html}
\ifcmt
	author_begin
   author_id gonchukov_arsenii
	author_end
\fi
\index[rus]{Мысли!Будущее – не то место, где нужно искать надежду, Арсений Гончуков, 17.12.2020}

\ifcmt
  pic https://img.vz.ru/upimg/experts/4296_26.jpg
  caption Арсений Гончуков, режиссер
  width 0.3
  fig_env wrapfigure
\fi

В Сети гуляет карикатура 1962 года итальянского художника Вальтера Молино под
названием «Жизнь в 2022 году», где люди стоя едут в маленьких, похожих на
тележки, автомобильчиках, каждый в своем. Сегодня это выглядит пронзительно:
вместо всеобщего единения человечества в XXI веке, художник увидел лишь
индивидуализм и людей, упакованных по лоточкам, как куриные грудки в
супермаркете. Как оказалось, автор был недалек от истины.

Баталии с ковидоотрицателями, полицейские рейды, скандал в театре, когда
женщину без маски выводит полиция. Первое убийство в общественном транспорте
из-за конфликта на «масочной почве». Петиции родителей против онлайн-обучения
(как будто в школе было лучше!). В Израиле старушку на улице штрафанули за то,
что она, отогнув маску, отпила кофе. Раньше над россиянами иронизировали
иностранцы: мол, живете стиснув зубы, ни одной улыбки. Толпа в масках уравняла
всех, она стала еще более безликой и оттого еще более бездушной. Парень с
девушкой целуются на эскалаторе. Люди смотрят с откровенной злостью: а про
родителей своих вы подумали, они же могут умереть от вируса!

Беда и смерть подступают все ближе, прорывая газетную бумагу заголовков и
вторгаясь в жизнь. Друг остался без работы, стреляет деньги. Два соседних
магазинчика закрылись. Плакатик «Аренда, недорого» с одной стороны отклеился и
жалко болтается на ветру. Технологии стараются компенсировать всеобщий развал.
Бешеными темпами растет рынок IT-продуктов, чуть ли не на 40\% увеличили число
подписчиков онлайн-кинотеатры. Цветет онлайн-образование. Выросла
интернет-торговля. Продажи хлебопечек бьют рекорды. Хотя и продажи беговых
дорожек тоже. Ну а врачи уже готовятся в ближайший год бороться с массовым
ожирением и гиподинамией. Разве такого будущего мы хотели? 

Жаль книг, которые не доехали до ярмарок. Книжные магазины тоже жалко. Жаль
сорванные отпуска. Жаль, что мой восемнадцатилетний сын, только что поступивший
во ВГИК, не увидел исторических аудиторий великого кинематографического вуза.
Его долгожданное студенчество – это многочасовой зум и красные глаза. Жаль
Москву, которая в последние годы упорно воплощала в жизнь концепции открытых
объединяющих горожан пространств. Жаль все эти уютные дворики, террасы,
лавочки, модные парки, которые оказались ровно тем, что в пандемию
противопоказано и вредно…

Чуткая культура, столько веков занятая мечтой о светлом будущем, теперь решает
конфликт настоящего и грядущего радикально и просто – лучше не рисковать. Почти
трехчасовой фильм Кристофера Нолана «Довод» до какого-то момента сохраняет
интригу: чего же от нас хочет будущее, чего ждет? Но очень быстро становится
понятно, что будущее хочет одного – уничтожить нас. Единственный вариант выжить
– вернуться в прошлое, развернуть мир вспять, потому что в будущем еще хуже.
Предсказание режиссера, снимающего, казалось бы, бодрый экшен, настолько
мрачно, что кажется, будто Нолан – проповедник запрещенной секты, призывающей в
дремучем лесу конец света. Будущее – это ад, утверждает профессиональный
голливудский оптимист. 

\ifcmt
  pic https://img.vz.ru/upimg/m10/m1075269.jpg
  width 0.5
  fig_env wrapfigure
\fi

Выход только один, но как ни парадоксально очень российский. Мысль Нолана о
неизбежности возвращения как нельзя лучше подходит российскому кинематографу.
Собственно говоря, наше кино уже давно трудолюбиво воплощает эту идею. Писатели
и режиссеры уже лет 20 топчутся у станочка, на котором вытачивают поделки
строго по лекалам прошлого. Что ни сериал, то про сталинские репрессии и синие
околыши, что ни книга, то про лихие девяностые, что ни фильм, то про Цоя и еще
раз про Цоя, и нет этому конца. Уничтожив целую страну ради светлого будущего,
мы тут же начали воспроизводить ее во всех ностальгических деталях. Еще
немножко поднажмем и таки увидим на прилавках колбасу по два двадцать. 

\index[rus]{Мысли!Будущее!Став настоящим, будущее безвозвратно изменяет нам, 17.12.2020}

Чем дальше живет цивилизация, чем выше взятые ею технологические высоты, тем
очевидней простая истина: став настоящим, будущее безвозвратно изменяет нам. И
как после этого ему верить? Как доверять ему свои фантазии и надежды, уповая на
лучшее? Как опираться на извечное завтра, которое светило человечеству со
времен Платона? Кажется, будущее – это наша главная моральная потеря, самая
ценная вещица, которую мы так бережно хранили в сумочке повседневности. Увы,
эту сумочку срезал карманник в полупустом метро.

Будущее сегодня явно не то место, где нужно искать надежду. Мы уже ничего не
хотим рисовать на белом листе завтрашнего дня. Напротив, мы хотим ограничить
время сегодняшним вечером и просто жить, не заглядывая в завтра. Может быть это
не так уж плохо? 

У современного фантаста Теда Чана есть рассказ «Выдох», где герои живут не в
огромном мире, как думают сами, а в какой-то загадочной емкости, напоминающей
баллон с газом. Сначала осознание того, что огромный мир лишь часть чего-то
большего, расстраивает, но к этой мысли быстро привыкаешь. Во всяком случае, у
Чана лучше, чем у Нолана, потому что в этом будущем есть продолжение.

Похоже, мы входим в эпоху футуристического минимализма, где будущее,
разоблачившее себя как финансовые пузыри последних кризисов, надо до предела
сдуть. Чем меньше мы наполняем надеждой фантом грядущего, тем больше
вероятность, что оно наступит. Это вопрос не мечты, а простого выживания. К
счастью, русским людям выживать, стиснув зубы, не впервой. Это можно делать без
улыбок, все равно за маской не разглядеть.

\begin{itemize}
\iusr{вап странник} 53 минуты назад  

...ага, давайте вернёмся в прекрасное прошлое - во времена чернобыльского
взрыва, когда весь СССР надолго встал на дыбы, десятки тысяч людей было
эвакуировано из зоны поражения навсегда, а черноморское побережье и большое
Сочи в тот год пустовали из-за массового опасения радиационного заражения от
поверхностных вод!

Блииин, эти вечно стонущие мАсквичи уже достали!

\iusr{Мимо Проходил Джизак} 3 часа назад  

Так мы уже сейчас последние могикане здравого смысла и образования среди белых
людей, китайцы и прочие образованные азиаты то же стремительно глупеют под
воздействием морока западных свобод.

\end{itemize}


