% vim: keymap=russian-jcukenwin
%%beginhead 
 
%%file 17_07_2020.fb.lnr.4
%%parent 17_07_2020
 
%%endhead 
\subsection{Азербайджан пригрозил Армении ракетным ударом по АЭС}
\label{sec:17_07_2020.fb.lnr.4}
\url{https://www.facebook.com/groups/LNRGUMO/permalink/2852561501522072/}
  
\vspace{0.5cm}
{\ifDEBUG\small\LaTeX~section: \verb|17_07_2020.fb.lnr.4| project: \verb|letopis| rootid: \verb|p_saintrussia|\fi}
\vspace{0.5cm}

В Минобороны Азербайджана предупредили Армению, что в случае атаки способны
нанести ответный удар по Мецаморской атомной электростанции.

Ранее в СМИ и социальных сетях появились предположения, что Ереван мог бы
атаковать Мингечевирское водохранилище в Азербайджане.

В Баку отметили, что располагают современными средствами противовоздушной
обороны и противник не сможет запустить ракету по их стратегическому объекту.

"В то же время армянская сторона не должна забывать, что новейшие <...>
системы, имеющиеся на вооружении нашей армии, с высокой точностью способны
нанести удар по Мецаморской АЭС, а это обернется для Армении огромной
трагедией", — заявил РИА Новости глава пресс-службы военного ведомства
полковник Вагиф Даргяхлы.

Начавшееся 12 июля боестолкновение на армяно-азербайджанской границе
продолжается четвертый день в сопредельных районах — Товузском и Тавушском,
граничащих также с Грузией и находящихся в нескольких сотнях километров от
непризнанного Нагорного Карабаха, где сейчас ситуация спокойная.

В месте обстрелов находятся боевые посты, дислоцированные близ села Мовсес. По
информации Баку, погибли 11 азербайджанских военных, включая генерала.
Армянская сторона заявила о двух жертвах и пяти раненых.

Азербайджан и Армения перекладывают вину за обстрелы друг на друга.
Международное сообщество призывает стороны к диалогу. Российский МИД заявил,
что готов оказать Баку и Еревану содействие для стабилизации обстановки.
  
