% vim: keymap=russian-jcukenwin
%%beginhead 
 
%%file 25_06_2023.stz.site.mariupol.retrogorod.1.kamennye_baby_ukrainskih_stepej.0.intro
%%parent 25_06_2023.stz.site.mariupol.retrogorod.1.kamennye_baby_ukrainskih_stepej
 
%%url 
 
%%author_id 
%%date 
 
%%tags 
%%title 
 
%%endhead 

В прошлом веке, когда массово распахивали нетронутые степи, в одно украинское
село приехала экспедиция археологов. Их интересовал высокий древний курган, на
вершине которого стояла каменная баба.

Когда начались раскопки, из села пришла женщина с ребенком и просила ученых
бабу не трогать, а оставить на месте, жителям села. Когда изваяние все-таки
извлекли из земли, у его подножия нашли множество монеток, и даже украшений.
Эти вещи приносили в знак поклонения каменным бабам. Среди старинных, давно
ушедших в историю денег кочевников лежали монеты Российской империи. А еще –
крестики...

Эту историю о священном изваянии рассказали мне участница археологической
экспедиции МГГУ Ольга Демидко. Каменные бабы – тема научной работы Ольги. Эту
тему она выбрала по своему желанию – \enquote{никто не задавал}, и уже собрала большое
количество материала о каменных изваяниях. Особенно много Ольга знает о
половецких бабах.

\begin{center}
  \em\color{blue}\bfseries\Large
\qbem{Знает каждый пастух, прогоняющий стадо,
Что оставить овцу перед идолом надо}. (Низами, азербайджанский поэт 12 века)
\end{center}

Каменные бабы – символы степи, стражи степи. Так много их было найдено, всего
около двух тысяч. Но, говорят мариупольские историки, не совсем ясно – в чем
истинное предназначение баб. Просто идолы? Символы предков, по сути тоже идолы?
Надгробия с изображением покойника? Иначе почему у половецких баб ярко выражены
индивидуальные черты?

В музее заповедника Хомутовская степь мне рассказывали, что негоже каменную
бабу с кургана снимать, а тем более устанавливать где-нибудь во дворе, на
пороге. И даже в музее. Это же надгробие, это по сути запечатленный в камне
покойник. В Мариуполе археологи заявили, что такие мысли - чушь полная.
Дискуссия продолжается.
