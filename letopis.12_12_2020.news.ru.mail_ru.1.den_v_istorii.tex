% vim: keymap=russian-jcukenwin
%%beginhead 
 
%%file 12_12_2020.news.ru.mail_ru.1.den_v_istorii
%%parent 12_12_2020
 
%%url https://news.mail.ru/society/44421351/?frommail=1
 
%%author 
%%author_id 
%%author_url 
 
%%tags 
%%title День в истории: 12 декабря
 
%%endhead 
 
\subsection{День в истории: 12 декабря}
\label{sec:12_12_2020.news.ru.mail_ru.1.den_v_istorii}
\Purl{https://news.mail.ru/society/44421351/?frommail=1}

День Конституции, крупнейшая железнодорожная катастрофа во Франции и другие
события этого дня.

\subsubsection{День Конституции}

\ifcmt
pic https://retina.news.mail.ru/pic/87/f0/image44421351_f8e943d5b89e59a7d77e107259aba4d7.jpg
\fi

12 декабря 1993 года всенародным голосованием была принята Конституция
Российской Федерации. Она вступила в силу 25 декабря. С 1994 года 12 декабря
было объявлено государственным праздником. Принятию Основного закона
предшествовал продолжительный Конституционный кризис, длившийся больше года.
Его кульминацией которого стало вооруженное столкновение 3−4 октября 1993 года
в центре Москвы и возле телецентра «Останкино».

4 июля 2020 года вступили в силу значительные изменения в Основной закон, за
которые россияне проголосовали 1 июля.

\subsubsection{Родилась Мария Тюссо}

\ifcmt
pic https://retina.news.mail.ru/pic/02/6a/image44421351_300565b735ffde36bf98d7b2661d4b65.jpg
\fi


12 декабря 1761 года в Эльзасе родилась Мари Гросгольц, ставшая известным
скульптором и создавшая в Лондоне легендарный Музей мадам Тюссо. Его копии в
дальнейшем появятся по всему миру.

Мари научилась искусству изготовления восковых скульптур у доктора Филиппа
Вильгельма Куртиуса, у которого ее мать работала экономкой. Куртиус делал
анатомические модели из воска, а позже стал изготавливать портреты на заказ.

Мари прославилась своими работами во Франции, но едва не погибла на гильотине
во время Французской революции. Женщина оставила мужа, который злоупотреблял
алкоголем, и перебралась в Лондон, где вскоре открыла постоянную выставку своих
работ.

\subsubsection{Крупнейшая железнодорожная катастрофа во Франции}

\ifcmt
  pic https://retina.news.mail.ru/pic/ec/64/image44421351_de00caf164270540c72d109ce1bcfbc1.jpg
\fi

В ночь с 12 на 13 декабря 1917 года возле французского селения
Сен-Мишель-де-Морьен произошла крупнейшая железнодорожная катастрофа в истории
Франции. Воинский эшелон, в котором с итальянского фронта возвращалось около
тысячи солдат, на спуске в альпийскую долину из-за плохих тормозов разогнался
до слишком высокой скорости, после чего сошел с рельсов. В крушении официально
погибли около 700 человек, что делает его крупнейшим в XX веке.

\subsubsection{Родился Владимир Шаинский}

\ifcmt
  pic https://retina.news.mail.ru/pic/b6/ab/image44421351_c43c0dc8faad28ea4348b12e913d37f0.jpg
\fi


12 декабря 1925 года родился будущий композитор и народный артист РСФСР
Владимир Шаинский. Мальчик появился на свет в Киеве в еврейской семье. Он
занимался по классу скрипки при Киевской консерватории, а во время Великой
Отечественной войны продолжил музыкальное образование в Ташкентской
консерватории — в Узбекистан эвакуировали его семью. После войны Шаинский
переехал в Москву, где работал в оркестре Леонида Утесова, преподавал и работал
в эстрадных оркестрах.

Наибольшую же известность Шаинскому принесли созданные им мелодии для
художественных и мультипликационных фильмов: «Анискин и Фантомас», «Завтрак на
траве», «Школьный вальс», «Финист — ясный сокол», «Пока бьют часы», «Шапокляк»,
«Катерок», «Крошка Енот», «Трям! Здравствуйте!», «Площадь картонных часов» и
многие другие. Именно он создал песни Чебурашки, крокодила Гены и старухи
Шапокляк из фильмов Романа Качанова.

