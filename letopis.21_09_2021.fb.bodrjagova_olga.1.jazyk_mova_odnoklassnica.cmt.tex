% vim: keymap=russian-jcukenwin
%%beginhead 
 
%%file 21_09_2021.fb.bodrjagova_olga.1.jazyk_mova_odnoklassnica.cmt
%%parent 21_09_2021.fb.bodrjagova_olga.1.jazyk_mova_odnoklassnica
 
%%url 
 
%%author_id 
%%date 
 
%%tags 
%%title 
 
%%endhead 
\subsubsection{Коментарі}

\begin{itemize} % {
\iusr{Тетяна Рибачук}
Оля, так тримати!!!
\ifcmt
  ig https://scontent-yyz1-1.xx.fbcdn.net/v/t39.1997-6/p480x480/91521538_1030933857302751_5093925307199520768_n.png?_nc_cat=1&ccb=1-5&_nc_sid=0572db&_nc_ohc=wxCpbfgwU4YAX_ftOnk&_nc_ht=scontent-yyz1-1.xx&oh=a9bc083bf7dfa6a09d13f8d2aca81066&oe=6150F2B7
  @width 0.1
\fi

\iusr{Veronika Zhmurko}
"Всі ті, для кого немає різниці, переходять на мову тих, для кого різниця таки є."-чудово.

\iusr{Yaroslava Muha}

Летіли на Кіпр цього року,ми з малою 4 роки. В літаку за нами сидить мама з
дівчинкою років 8. Ну малим сумно, вони між собою розмовляли, гралися,
обмінювалися цукерками і іграшками. Потім стоїмо разом в черзі зробити плр
тест, мама цієї дівчинки питає звідки ми такі українськомовні, і дуже
дивується, що ми з Києва:

-Кієв же рускоговорящий!

 · Ответить · 16 ч.
\iusr{Tetyana Nikolaeva}

Оля! Це - шикарно, бо це життя більшості української інтелігенції, що ділиться
на три типи: ті, що продовжують плисти в заданому в радянські часи напрямку,
переконуючи що «велікій і могучій»; ті, що «какая разніца», «ми ведь за
Україну»; та частина, що усвідомила жити по- советскі(маючи саме радянську
освіту і виховання), це означає не мати власної країни ні тепер, ні в
перспективі. Останній шлях - найважчий, але у ньому є сенс і усвідомлення, що
своїми переконаннями і прикладом ти сприяєш побудові справжньої національної
держави! @igg{fbicon.flag.ukraina} Особливо це надихає, коли ти - класний фахівець в своїй ( не
освітній) галузі!

\iusr{Нина Скопич}

\ifcmt
  ig https://scontent-yyz1-1.xx.fbcdn.net/v/t39.1997-6/p480x480/91521538_1030933857302751_5093925307199520768_n.png?_nc_cat=1&ccb=1-5&_nc_sid=0572db&_nc_ohc=wxCpbfgwU4YAX_ftOnk&_nc_ht=scontent-yyz1-1.xx&oh=a9bc083bf7dfa6a09d13f8d2aca81066&oe=6150F2B7
  @width 0.1
\fi


\iusr{Галина Дмитренко}

Ольго Іванівно!
Я захоплююся Вами!

\iusr{Тетяна Попова}
шкода що допис тільки для друзів. вартий поширення

\begin{itemize} % {
\iusr{Olga Bodriagova}
Ніби поширила зараз  @igg{fbicon.smile} 

\iusr{Тетяна Попова}
\textbf{Olga Bodriagova} нема кнопочки поширити. бо запис з приватністю для друзів. треба змінити приватність на «для всіх»

\ifcmt
  ig https://scontent-yyz1-1.xx.fbcdn.net/v/t1.6435-9/242571686_10220728148779745_5113413373708059939_n.jpg?_nc_cat=110&_nc_rgb565=1&ccb=1-5&_nc_sid=dbeb18&_nc_ohc=_5VhrdFCIvQAX-QzjfW&_nc_ht=scontent-yyz1-1.xx&oh=a6ed9eea810bab7769691e552949afdf&oe=616F68D0
  @width 0.3
\fi

\iusr{Olga Bodriagova}
Я змінила, має працювати

\ifcmt
  ig https://scontent-yyz1-1.xx.fbcdn.net/v/t1.6435-9/242642167_4501495863241275_783145461986312312_n.jpg?_nc_cat=105&_nc_rgb565=1&ccb=1-5&_nc_sid=dbeb18&_nc_ohc=d-IuSunVJboAX-GjiQq&_nc_ht=scontent-yyz1-1.xx&oh=5243d7df725bf97f10259fb7e2933e1b&oe=617217D9
  @width 0.3
\fi

\end{itemize} % }

\iusr{Тетяна Попова}
ппкс

\iusr{Аліна Мельник}
Вчора до нас звернулись незнайомі в готелі: О, ви говорите українською, точно
наші! А звідки? З Києва? О то точно наші в квадраті ))))

\begin{itemize} % {
\iusr{Тетяна Попова}
\textbf{Аліна Мельник} і я в отелі так зраділа почувши мову. бо довкола лунає московинська. мої «свої» виявились з Вінниці.
\end{itemize} % }

\iusr{Olena Derevska}
Додам у вашу скарбничку принизливих цитат: "Вьі же женщина, зачем вам ето
нужно, как вам не стьідно". (2012 рік)

\begin{itemize} % {
\iusr{Olga Bodriagova}
Саме так, за цими всіма фразами я відчувала саме такий підтекст: від «зачем вам это нужно?» аж навіть до «как вам не стыдно».

\iusr{Kateryna Hnoieva}
\textbf{Olga Bodriagova} останнім часом відповідаю одним словом "хочу"!)
Діє на 100\%!!! @igg{fbicon.face.wink.tongue} 
\end{itemize} % }

\iusr{Марина Хаперська}

Ми живемо в Харкові, і найчастіше однолітки запитують в мого сина-підлітка:

\begin{itemize}
  \item 1. Ви што і дома разговаріваєтє па-украінскі?
  \item 2. Тєбя радітєлі заставляють гаваріть на мовє?
\end{itemize}

\begin{itemize} % {
\iusr{Olga Bodriagova}
В Києві це непросто, а в Харкові це взагалі надзвичайно важко, потрібно мати неабияку внутрішню силу, щоб триматися своєї позиції. Особливо серед однолітків, які можуть бути дуже жорстокими

\iusr{Марина Хаперська}
\textbf{Olga Bodriagova} дякую, ми звикли

\iusr{Olga Bodriagova}

\ifcmt
  ig https://scontent-yyz1-1.xx.fbcdn.net/v/t39.1997-6/p480x480/91521538_1030933857302751_5093925307199520768_n.png?_nc_cat=1&ccb=1-5&_nc_sid=0572db&_nc_ohc=wxCpbfgwU4YAX_ftOnk&_nc_ht=scontent-yyz1-1.xx&oh=a9bc083bf7dfa6a09d13f8d2aca81066&oe=6150F2B7
  @width 0.1
\fi

\end{itemize} % }

\iusr{Галина Глазова}

В мене старша сестра запитала:" ты,что перешла на украинский?", на це я
відповіла,що ми живемо в Україні, і повинні знати державну мову. Так і
розмовляли - вона російською, а я українською, трохи з помилками, але вчитися
ніколи не пізно.

\iusr{Тамара Максимик}
Пані Олю, побільше б таких, як Ви... Дякую!!!

\iusr{Ольга Мякота}
Ото біда що україномовні переходять на російську.
А з питань мене завжди смішить . А дома ви як розмовляєте ?

\iusr{Інна Некрашевич}

Для мене володіння українською, даруйте на слові, як тест на розвиненість. Моя
мама переїхала в Київ з Хмельниччини після школи і була повністю зросіщена. Ми
з братом, звісно, теж спілкувалися російською - оточення має великий вплив. Але
після 2014 року моя мама свідомо почала розмовляти українською мовою, ми з
братом теж. І коли кажуть, що важко розмовляти українською, то я завжди думаю
про те, чи легко моїй мамі почати в 62 роки розмовляти українською в Києві
після того, як вона все свідоме життя розмовляла російською. Але це тільки
питання розвитку, свідомості, розуміння того, що, можливо, їй вже ці зміни і не
потрібно у такому віці, але є ще діти і онуки, для розвитку яких ти мусиш
змінюватися не залежно від твого віку. Коли починаєш думати не про себе як
відокремлену особистість, а починаєш себе відчувати частиною нації, частиною
країни, розвитку цієї країни, розвитку майбутнього, то питання якою мовою
розмовляти не виникає.

\begin{itemize} % {
\iusr{Тамара Трофімішина}
\textbf{Inna Nekrashevish} 

Дивно, що розмовляти українською важко людині, яка
від народження була в українськомовному середовищі і розмовляла українською
мовою... Мені - 64, я з російськомовної сім'ї киян, мама - москвичка, школа,
вуз, робота, друзі- російськомовні, сім'я також. Але 14 рік виправив все : було
б бажання. Але ще є що виправляти і у себе, і в сім'ї... Все буде Україна!!!
\end{itemize} % }

\iusr{Надія Захарчук}

В 15-му, однокакашнік при зустічі сказав:" Да брось.. тєбє нє ідьот". Подумала
тоді.. ах ти ж курва, і тобі аж бігом "зайдьот", як жерти захочеш.. тиснемо,
наближаємо той час, коли без української в Україні вони будуть почуватись, як
без російської в росії.


\iusr{Українською Мовою Будь Ласка}

Якщо вам кажуть московськоязичні, що немає дискримінації українців в Україні за
мовною ознакою, запропонуйте їм на тиждень перейти з московської на українську.
Нехай відчують на власній шкурі як це.

\iusr{Павло Рафальський}

Завжди кажу на зауваження "а какая разница, на каком языке говорить?" - "якщо
нема різниці, то давайте по-українськи")))

\begin{itemize} % {
\iusr{Yakovenko Anatoliy}
\textbf{Павло Рафальський} ...давайте українською!
\end{itemize} % }

\iusr{Анна Винокурова}
\#мова\_має\_значення

\iusr{Ігор Костогриз}

@igg{fbicon.100.percent}

\iusr{Валентина Чорномаз-савицька}

\ifcmt
  ig https://scontent-yyz1-1.xx.fbcdn.net/v/t39.1997-6/p480x480/91521538_1030933857302751_5093925307199520768_n.png?_nc_cat=1&ccb=1-5&_nc_sid=0572db&_nc_ohc=wxCpbfgwU4YAX_ftOnk&_nc_ht=scontent-yyz1-1.xx&oh=a9bc083bf7dfa6a09d13f8d2aca81066&oe=6150F2B7
  @width 0.1
\fi

\iusr{Valentina Kubik}

Дуже добре розкладено по пунктах саме чому, ми знаємо російську ящику. - томущо
союз не був союзом. Була кремлівська окупація.  @igg{fbicon.face.angry} 

\iusr{Віктор Шевчик}

\ifcmt
  ig https://scontent-frx5-2.xx.fbcdn.net/v/t39.1997-6/s168x128/17639224_1652591041433963_4031073656246370304_n.png?_nc_cat=1&ccb=1-5&_nc_sid=ac3552&_nc_ohc=LYkBBm5L7qUAX8SFgb-&_nc_ht=scontent-frx5-2.xx&oh=9af970290406a094be4ec51c5a4d28c5&oe=614F8BDF
  @width 0.1
\fi

\iusr{Olena Vermizova}

Для мене найбільш показовим був випадок, коли пише мені з подивом якось у ФБ
мій знайомий: \par "мені ( ім'я нашого спільного знайомого ) сказав, що ти тепер лише
українською розмовляєш". Ну ніби пішов слух селом, що я людину вбила


\iusr{Пасічник Тетяна}

Оля Вам респект!!! Я щаслива, що живу у Львові. Тут рідко почуєш
російську(правда останнім часом побільшало -переселенці, або туристи) і сказати
що мені було б некомфортно навколо

російське середовище - це нічого не сказати. Просто уникаю його, особливо
останіх 8 років.

\begin{itemize} % {
\iusr{Любов Пунь}
\textbf{Пасічник Тетяна}
Останній раз розмовляла російською коли була на практиці в ( тоді ще) Ленінграді на 4 чи 5-му курсі.
Це було в 1987 чи 1988 . І. ВСЕ ! Навіть чути її не хочу !
\end{itemize} % }

\end{itemize} % }
