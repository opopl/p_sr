% vim: keymap=russian-jcukenwin
%%beginhead 
 
%%file people.zelenskii_vladimir_prezident_ukr.cit
%%parent people.zelenskii_vladimir_prezident_ukr
 
%%url 
 
%%author_id 
%%date 
 
%%tags 
%%title 
 
%%endhead 

%%%cit
%%%cit_head
%%%cit_pic
%%%cit_text
\emph{ПреЗЕдент Украины} на этой неделе явно с удовольствием вернулся к своему
любимому делу шоуменства, выступив на климатическом саммите в Глазго. Но годы,
проведенные в кресле главы "державы", явно дали о себе знать: выступал он
уныло.  Да и текст экс-игрецу на рояле писали, похоже, примерно тем же, чем он
на нем играл. Все действо напоминало анекдот про студента, который сумел
выучить лишь билет про строение блохи и пытался на этой блохе вытянуть весь
экзамен, избегая ответов на конкретные вопросы и делая невероятные допущения.
В речи \emph{Зеленского} такой "блохой" стала "российская агрессия". Подстроив
изрядно всем поднадоевший жупел под тематику мероприятия, в Офисе президента
сочинили страшилку про "экобомбы", которые "агрессоры" якобы заложили в Крыму и
Донбассе. В частности, по словам \emph{Зеленского}, уникальные флора и фауна
полуострова находятся под угрозой из-за размещения российской военно-морской
базы, а в Народных Республиках происходят затопления шахт и деградация почв. О
том, почему природе Крыма не угрожала украинская база и как украинские власти
задолго до войны закрыли шахты в Донбассе, \emph{Владимир Александрович} почему-то
умолчал
%%%cit_comment
%%%cit_title
\citTitle{НЕДЕЛЯ ГЛАЗАМИ ЭКСПЕРТА: Ротация блокираторов, экобомбист и комедия Гайдая}, 
, lug-info.com, 07.11.2021
%%%endcit

%%%cit
%%%cit_head
%%%cit_pic
%%%cit_text
При сохранении всех этих тенденций \emph{Зеленский} потеряет всякие шансы на
переизбрание. Выборы в Раду в 2023 году "слуги" проиграют, будет сформирована
новая коалиция, она выдвинет премьером условного Разумкова, который в роли
"новой надежды нации" победит Зе на выборах президента. После чего и
\emph{Зеленского}, и все его окружение, и весь творческий коллектив "Квартала" начнут
беспощадно догонять. Причем бежать им будет особо некуда. И это еще не самый
худший для \emph{Зеленского} вариант, так как, не исключено, что он и не до сидит до
конца срока – если будет подходящий повод, то его снесут еще раньше.  Может ли
\emph{Зеленский} избежать этой участи? Теоретически да.  У него есть три варианта как
поломать гибельный для себя инерционный сценарий
%%%cit_comment
%%%cit_title
\citTitle{"Будут беспощадно догонять". Главред "Страны" Игорь Гужва о трех сценариях будущего для Зеленского}, 
, strana.news, 06.11.2021
%%%endcit

%%%cit
%%%cit_head
%%%cit_pic
%%%cit_text
Если хотите увидеть человека, сожжённого властью и демонами тщеславия -
посмотрите пресс-конференцию. Это и обида, и мстительность, и хамство. И это
все искренне. Человек находится в состоянии сильной дезадаптации.
Кстати, по поводу стратегии, с помощью которой можно достаточно уверенно
уничтожать Зеленского. Точнее, заставить его самоповреждаться. Она очень
проста. Нужен человек, который станет Злом для Президента (типа, как и Ткач или
Бутусов). Человек, который вызовет на себя ненависть и желание мести. Этот
человек и будут каналом в психику Зеленского. Владимир Александрович не сможет
его забыть и будет сжигаем неудовлетворимым чувством мести. Далее этому
человеку нужно упорно говорить в глаза Зеленскому обвинения. Упорно и стойко.
Зная, что обычная стратегия ОП Зеленского - фантазировать без привязки к
реальности. И Зеленский не выдерживает - меняет позицию и дает признательные
показания. Это просто удивительно, насколько легко проламывается
самоуверенность. Но просто потому, что Зеленский уже давно выгорел. Дальше
будет хуже
%%%cit_comment
%%%cit_title
\citTitle{Зеленский психологически полностью разрушился – главный итог пресс-конференции экватора президента}, 
Олег Хомяк, analytics.hvylya.net, 27.11.2021
%%%endcit

%%%cit
%%%cit_head
%%%cit_pic
%%%cit_text
Уже не йдеться про другий термін президентства Зеленського. Брехня вбила всі
електоральні перспективи. Але, однозначно, в разі масових протестів, ніхто не
вийде на \enquote{антимайдан}. Зеленський – не Янукович. Розколу країни нема і не буде.
Зеленський нікому не потрібен, крім \enquote{Слуг народу} в парламенті та жмені
чиновників.  Але, чи є перспективи в країни з таким президентом? Чи є в нас
надія на завтрашній день? І до речі, про що ще брехав і бреше нам президент?
Які ще оборудки Єрмака покриває? Не хочеться дізнатися про це, коли вже буде
пізно.  Опозиційним політичним партіям треба визначатися. Ні. Не з тим, що
робити з Зеленським. Тут усе зрозуміло: він – гальмо і ганьба для країни. З ним
треба прощатись. Дружно. Урочисто. На Майдані.  Визначатись потрібно з єдиним
кандидатом у президенти – людиною рівновіддаленою від традиційних центрів
впливу, без негативного рейтингу, компетентною і бездоганною, з морального
погляду. Певен, це реальне завдання.  А Зеленському скажімо: "прощавай". Годі
вже. Край!
%%%cit_comment
%%%cit_title
\citTitle{Скажімо Зеленському \enquote{Прощавай!} – дружно, голосно, на Майдані}, 
Валентин Бушанський, gazeta.ua, 29.11.2021
%%%endcit
