% vim: keymap=russian-jcukenwin
%%beginhead 
 
%%file topics.ukraina.rossia.ugroza_napadenia.preface.3.slova
%%parent topics.ukraina.rossia.ugroza_napadenia.preface
 
%%url 
 
%%author_id 
%%date 
 
%%tags 
%%title 
 
%%endhead 
\clearpage
\subsubsection{Слова, слова, слова...}

Касательно мнимого вторжения России в Украину, и словесном потоке, его
сопровождающем.  Знаете, забавно за этим наблюдать. Вроде с осени 2021 пишут,
что вот-вот уже сейчас вторгнется, а сейчас уже февраль 2022 года, - и
поразительно! - в Киеве спокойно и тихо. Спокойно работает метро, ездят машины,
ходят пешеходы. Влюбленные парочки рассматривают Днепр, а компания подростков
что-то там увлеченно обсуждает.  Как же так? Уже столько написано, уже как бы
ясно - что все это - один гигантский информационный пузырь, а до сих пор
находятся люди, которые во всю эту чушь неистово верят. На границе страшный
враг, он обязательно хочет нас уничтожить, злобный карлик в Кремле день и ночь
не спит, думая, как бы захватить бедную несчастную Украиночку. В чем же дело?
Почему постоянная ложь и нагнетание со стороны Западных сми до сих пор проходит
в Украине? Наверное, тут есть несколько причин:

\begin{itemize} % {
\item в информационном обществе 21 века ценность слова как такового, да, именно
Слова, - стала очень низкой, и вообще отсутсвующей. Люди в фейсбуке,
телеграме и СМИ пишут что попало и как попало.  А в обычной
жизни как? Если ты что-то пообещал, своему другу например, - то
надо исполнить обещанное, - а то можно и друга потерять, нажив
вместо него врага. Это называется по-простому отвечать за свой
базар, или же, пацан сказал, - пацан сделал.
\end{itemize} % }
