% vim: keymap=russian-jcukenwin
%%beginhead 
 
%%file 31_07_2021.fb.panchenko_diana.1.oko_za_oko
%%parent 31_07_2021
 
%%url https://www.facebook.com/permalink.php?story_fbid=1973323666152572&id=100004248730374
 
%%author 
%%author_id panchenko_diana
%%author_url 
 
%%tags future,nenavist,obschestvo,strana,ukraina,uvazhenie
%%title «Древний закон «око за око» приведёт к тому, что все останутся слепыми»
 
%%endhead 
 
\subsection{«Древний закон «око за око» приведёт к тому, что все останутся слепыми»}
\label{sec:31_07_2021.fb.panchenko_diana.1.oko_za_oko}
 
\Purl{https://www.facebook.com/permalink.php?story_fbid=1973323666152572&id=100004248730374}
\ifcmt
 author_begin
   author_id panchenko_diana
 author_end
\fi

«Древний закон «око за око» приведёт к тому, что все останутся слепыми».

Последнюю неделю я очень много думала о том, к что спустя 7 лет после Майдана
мы все дальше и дальше друг от друга. Что ещё может сплотить всех этих людей?
Как найти точки соприкосновения? 

«Нам с ними не по пути», «о чем с ними разговаривать», «мы разные»..

Пусть в меня летят тысяча камней, но я считаю, что компромис искать нужно. 

Но есть два условия. Первое - взаимное уважение. Второе - воля, в первую
очередь тех, кто руководит страной. 

Мы не сможем найти компромисс, пока одни презирают русскоговорящих и запрещают
русский язык. А другие   отказываются понять и принять тех, кто по-своему
понимает  историю и память. 

Да, разговоров  о единстве и «сшивании» страны много. Но почти все, «сшиватели»
отказываются видеть дальше своего носа. 

Страны просто не будет, если дальше рвать ее на части. Да, мы слишком разные,
но в этом наша прелесть и сила. 

Уверена, Украине нужна настоящая децентрализация. С возможностью каждого
региона выбирать своих героев. 

Тотальная украинизация не приведёт ни к чему хорошему. Только отдалит и озлобит
тех, кому она не по душе. 

Украине нужен сильный закон об оппозиции. С которой нужно наконец считаться. 

А политическим силам, предоставляющим условный Юго-восток, стоит наконец понять
- распри и козни в итоге погубят всех. Вместо того, чтобы ковать все новые
политические проекты, заранее обречённые  на провал, нужно сплотиться вокруг
силы, которая сегодня пользуется наибольшей поддержкой Юго-востока. 

И напоследок. Я на собственной шкуре ощутила, как высок градус неприятия и
ненависти в нашем обществе. Я увидела, как много гречи во всех нас. Искренне
прошу прощения у всех, чьих ожиданий не оправдала. Но я кое-что поняла. Большие
поступки - это всегда о благоразумии и компромиссах. И мне есть к чему
стремиться.


\ii{31_07_2021.fb.panchenko_diana.1.oko_za_oko.cmt}
