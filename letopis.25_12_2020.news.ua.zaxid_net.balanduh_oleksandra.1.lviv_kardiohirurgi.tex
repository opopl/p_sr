% vim: keymap=russian-jcukenwin
%%beginhead 
 
%%file 25_12_2020.news.ua.zaxid_net.balanduh_oleksandra.1.lviv_kardiohirurgi
%%parent 25_12_2020
 
%%url https://zaxid.net/lvivski_kardiohirurgi_vpershe_samostiyno_peresadili_sertse_n1512396
 
%%author 
%%author_id balanduh_oleksandra
%%author_url 
 
%%tags lviv,hirurgia
%%title Львівські кардіохірурги вперше самостійно пересадили серце
 
%%endhead 
 
\subsection{Львівські кардіохірурги вперше самостійно пересадили серце}
\label{sec:25_12_2020.news.ua.zaxid_net.balanduh_oleksandra.1.lviv_kardiohirurgi}
\Purl{https://zaxid.net/lvivski_kardiohirurgi_vpershe_samostiyno_peresadili_sertse_n1512396}
\ifcmt
	author_begin
   author_id balanduh_oleksandra
	author_end
\fi

\begin{leftbar}
  \begingroup
    \em\Large\bfseries\color{blue}
    Донором для 30-річного чоловіка стала 42-річна львів’янка
  \endgroup
\end{leftbar}

\ifcmt
pic https://zaxid.net/resources/photos/news/202012/1512396.jpg?202012281715&fit=cover&w=960&h=540&q=100
caption У лікарні швидкої медичної допомоги цього року пересадили чотири серця. Останню трансплантацію львівські кардіохірурги провели самостійно
\fi

У п’ятницю, 25 грудня, у львівській Клінічній лікарні швидкої медичної допомоги
львівські кардіохірурги вперше самостійно пересадили серце. І виконав
трансплантацію завідувач віділення кардіохірургії та трансплантації Центру
серця та судин лікарні Роман Домашич разом із колегою з Інституту
трансплантації серця МОЗ України Гаврилом Ковтуном, повідомив директор лікарні
Олег Самчук.

Кардіохірург Роман Домашич раніше асистував відомому кардіохірургу Борису
Тодурову і Гаврилу Ковтуну на п’яти трансплантаціях серця, які проводили у
Ковелі та Львові.

\ifcmt
pic https://zaxid.net/resources/photos/news/202012/1512396_2092991.jpg?202012251626&fit=cover&w=720&h=540&q=65
caption Серце львів'янки продовжує битися в грудях цього чоловіка з Мукачева
\fi

«Роман Домашич пересадив серце у Клінічній лікарні швидкої медичної допомоги
міста Львова. Дев’ята за рік трансплантація серця в Україні щойно завершилася.
І виконав її наш заввіділенням кардіохірургії та трансплантації Центру серця та
судин Роман Домашич разом із заввідділенням трансплантації серця Інституту
серця МОЗ України Гаврил Ковтун. Звісно, спільно із командами спеціалістів двох
наших закладів», – повідомив директор лікарні Олег Самчук.

Серце пересадили 30-річному чоловікові з Мукачева. Його власне серце після
пневмонії було у критичногму стані. За останній рік воно зупинялося шість
разів. Чоловік, батько двох маленькх дітей, отримав шанс на повноцінне життя.

\ifcmt
pic https://zaxid.net/resources/photos/news/202012/1512396_2092986.jpg?202012251551&fit=cover&w=720&h=540&q=65
caption Кардіохірурги готують серце до пересадки
\fi

«Це вже четверта трансплантація у Львові. Особлива гордість за те, що цього
разу операцію виконав завідувач відділення кардіохірургії ЛШМД Роман Домашич.
Донором стала жінка, яка потрапила в лікарню з тяжким інсультом», – повідомив
мер Львова Андрій Садовий.

Після того, як у 42-річної львів’янки лікарі констатували смерть мозку, мама,
чоловік і син погодилися на донорство, щоб серце рідної людини продовжило
битися.

У новому році хірурги Клінічної лікарні швидкої медичної допомоги планують
проводити пересадку печінки і легень.



