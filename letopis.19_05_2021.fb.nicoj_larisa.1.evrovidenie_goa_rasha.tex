% vim: keymap=russian-jcukenwin
%%beginhead 
 
%%file 19_05_2021.fb.nicoj_larisa.1.evrovidenie_goa_rasha
%%parent 19_05_2021
 
%%url https://www.facebook.com/larysa.nitsoi/posts/4312185065480259
 
%%author 
%%author_id 
%%author_url 
 
%%tags 
%%title 
 
%%endhead 
\subsection{Ницой Лариса - Україна, Україна, Україна! Господи, рашка така
гидотна пройшла, невже нашу пісню не візьмуть? Ну де справедливість?}
\Purl{https://www.facebook.com/larysa.nitsoi/posts/4312185065480259}

- Го-о-о-о-л! - пролунало несамовито-радісне на пів кварталу.  Насправді, то
був не "гол", але вигук лунав дуже схожий, коли, затамувавши подих, ми
слідкували за екраном телевізора, у якому озвучували переможців першого
півфіналу Євробачення. 

\ifcmt
  pic https://scontent-iad3-1.xx.fbcdn.net/v/t1.6435-9/187651820_4313445715354194_4756820403444352634_n.jpg?_nc_cat=102&ccb=1-3&_nc_sid=8bfeb9&_nc_ohc=DhUoYHDHUT0AX8h8NXi&_nc_ht=scontent-iad3-1.xx&oh=95ae00bfd6700ec1c3c8d304a6f2421b&oe=60CA9D6B
\fi


Називають першого переможця - не Україна. Другого, третього, шостого - не
Україна. З кожним наступним напруга зростає. Серце завмирає. Лишилося одне
місце. Одне. Хто?! Хто цей останній щасливчик? Стискаєш кулачки. Замружуєшся.

Шепчеш: \enquote{Україна, Україна, Україна! Господи, рашка така гидотна пройшла, невже
нашу пісню не візьмуть? Ну де справедливість?}

Ведучі все тягнуть інтригу. Серце калатає все гучніше. Останнє місце. Нужбо, ну ж бо...

- Україна! - кажуть ведучі.
- А-а-а! - заволали ми як скаженні, чи "о-о-о!", чи що там лунало серед ночі з
вікон нашої квартири, схоже на "го-о-ол". Ми вболівали за Україну.

Кажуть, не всім українцям подобається наша пісня. Ви що, смієтеся? Хіба про
таке говорять, коли наші вже там і вже змагаються? 

Пофіг, люди, подобається Вам, чи ні пісня або виконавець. Навчіться, нарешті,
підтримувати своїх. Пофіг, що це не відповідає вашому смаку. Пізно вже
сперечатися про смаки. Пісня там. Україна там. Навчіться, нарешті,
підтримувати своє. І вболівати за УКРАЇНУ. 

А ще мені дуже сподобалася Литва, дружня до нас країна. Другий рік поспіль таке
крутецьке в них. І Норвегія з ангелом нічогенька. Усі інші пісні і країни
непогані, але відстали від нас хтозна на скільки років, бо з такою піснею і з
таким платтям за царя гороха ще лорачка, коли була українкою, їздила на
Євробачення. А тепер вони вирішили нас цим здивувати. 

--------

Якщо у фіналі якісь придурки в Україні голосуватимуть за рашкину учасницю, і
євробаченські ведучі на весь світ озвучать, що Україна дала раші таку то
кількість балів для їхньої перемоги (на сьомому році їхньої війни з нами) - усі
телевізори тих придурків погорять від ганьби. Не встигнете й квакнути.

Oleg Barinov

Як кому, а мені сподобалась пісня! І виконання прекрасне! Не порівняти з
мартьошкай!)))

Любов Бурак

І не тільки телевізори погорять. Ми їх чіпувати будемо, чіп у вухо, жовтий.

Вероніка Токар

Мені спочатку пісня не сподобалась, а потім як затягнуло!.. Виконання достойне,
ще й українською мовою❤️Дуже вболіваю за наших!

Еміль Дубров

Тоді схід і південь залишиться без телевізорів 100\%!

Larysa Nitsoi

Еміль Дубров ну й нехай.

Ksenia Ksenina

Я з Херсонщини, і ніколи б не проголосувала за Росію. Вам не здається, що ви
самі розділяєте народ? Ваше "Схід і Південь" звучить точно так же, як путінське
"Юговосток".

Ksenia Ksenina

Еміль Дубров можливо на виборах у нас опзж займає перше місце, але це приблизно
20 (+/-) відсотків від тих, хто прийшов на вибори. Про які 100\% ви говорите?
Вибачте, але це виглядає як розпалювання ворожнечі. "Ось ми тут такі правильні,
а там - 100\%". Якщо Ви порядна людина, Вам треба подумати і перепросити.

Еміль Дубров

Ksenia Ksenina... Даруйте, пані Ксеніє, але схоже на те, що це Ви розпалюєте
ворожнечу! Я ніде й словом не обмовився, що ті, хто підтримають рашу є наші
вороги, просто нещасні люди, які зазомбовані російською пропагандою оскільки
дивляться в більшості своїй супутникове ТБ, тобто - російський контент,
оскільки український закодований і за нього потрібно платити.

Що стосується порядності, то я на Вашій сторінці не побачив жодної публікації,
яка б пробуджувала в людині почуття патріотизму і закликала бути єдиними... Тож
перш ніж звинувачувати когось в непорядності і давати поради, зверніться до
себе.

Ksenia Ksenina

Еміль Дубров Ви розділяєте людей? З Ваших дописів це витікає безсумнівно. Про
100 відсотків Ви писали. А те, що порядність вимірюється публікаціями на своїй
сторінці в фейсбук, для мене це новина. Дякую, не знала. Як легко, виявляється,
бути патріотом.

Olena Slyusarenko

Та як може не подобатися наша тисячолітня веснянка, якою наші пращури закликали
весну??? Це магія світосприйняття і світотворення!!!! Той, хто цього не знає,
не розуміє - це відсталі в своєму розвитку люди, бо у нас всіх крали історію,
релігію, наше минуле !!!! Любити своє можна тоді, коли багато знаєш! Любити
своє - це неабияка праця над собою, історією, духовністю!! Просто подобається
чи не подобається - тут не правильно, бо більшість наших людей виховані на "
пєснях савецкава врємєні", а деякі і досі заслуховуються кіркоровськими зайками
і думають, що це добре. Не просто думають, а впевнені, що вони "прадвінутиє" в
музичному плані!!! Хтось дуже мудрий сказав, що " дурень ніколи не визнає, що
він дурень. Це лише розумна і освічена людина, яка частенько володіє декількома
мовами, може вважати, що знає дуже мало"

Valik Marushchak

Бггии, у тебе Рінат Кузьмін голова ТСК Верховної Ради (!!) і сєпар Аксьонов,
воістину "Моторола в Раді" - а ці убожиські реєстрові "націоналісти", на
утриманні Кремля, постять котиків і Євробачення.

Не уявляю, як би волала ця неадекватна, якби таке сталось при Порошенку. А тут
- ні пари з вуст, нема команди з москви, видно.

Nataliia Boguslavska

\enquote{Рашка гидотна пройшла}, бо їх по всьому світу як саранчі багато.

Oleg Gryniw

Голосуватимуть за рашу, не сумніваюся.
