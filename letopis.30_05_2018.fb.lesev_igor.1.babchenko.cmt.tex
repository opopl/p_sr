% vim: keymap=russian-jcukenwin
%%beginhead 
 
%%file 30_05_2018.fb.lesev_igor.1.babchenko.cmt
%%parent 30_05_2018.fb.lesev_igor.1.babchenko
 
%%url 
 
%%author_id 
%%date 
 
%%tags 
%%title 
 
%%endhead 
\subsubsection{Коментарі}

\begin{itemize} % {
\iusr{Дмитрий Серба}
Аминь

\iusr{Виктория Павликова}

Да пусть бы писал что хотел, мало кого это интересовало, если только бывших
коллег, это очень узкий круг... но деньги он собирать умел и собирал их ловко,
круг благодарных овечек был весьма не узок, даже его последние сборы - все
видят последний пост, но никто не смотрит предпоследний, а там накидали
прилично, судя по плюсикам в ленте и озвученным суммам... а там где деньги, там
всегда присутствует посторонний интерес, и если кроме говорить зубов у тебя
нет, то на такие легкие деньги всегда найдется хищник быстрее и хитрее. Не надо
светить деньгами - совет родительский исполняться должен истово.


\iusr{German Gorozhanski}

Как те самописцы оранжевого цвета у самолётов зовутся, "черный ящик"? Жаль, что
у человеков не бывает самописцев. Прокрутил плёночку и все стало на свои места:
кто, как, зачем... Ан, нет.

А от чёрного ящика никому ещё отделаться не удалось.

И пусть Бог продлит наши годы.


\iusr{Светлана Йойна}
Аминь.

\iusr{Юрий Ткачук}
"...обнимается с кемеровскими детишками...", аж мороз по коже...

\iusr{Ирина Шабанова}
Верите, про существование Бабченко только вчера и узнала.

\iusr{Олег Резник}
Игорь, спасибо. Понятно, сдержанно и в полном объёме изложено событие. Больше про это можно ничего и не читать.

\iusr{Матвей Кублицкий}
У нас не принято радоваться чьей-то смерти, но плакать по нему никто не станет

\begin{itemize} % {
\iusr{Игорь Лесев}
не обобщай. Радоваться или не радоваться смерти зависит от внутренного воспитания, а не от принадлежности к какому-то народу

\iusr{Матвей Кублицкий}
\textbf{Игорь Лесев} согласен.

\iusr{Матвей Кублицкий}
\textbf{Игорь Лесев} но он - бабченко - радовался

\iusr{Игорь Лесев}
и заметь, был русским

\iusr{Матвей Кублицкий}
\textbf{Игорь Лесев} русский или украинец - суть субъективные понятия, не имеющие ничего общего с понятием нация

\iusr{Игорь Лесев}

давай еще гегелевское различие понятий народ и нация разбирать будем... Был
Бабченко со своими критериями морали. И есть все остальные, которые после его
смерти могут попытаться не дегуманизироваться


\iusr{Матвей Кублицкий}
\textbf{Игорь Лесев} 

меня больше повеселило каким идиотом был сегодня в ООН климкин. сразу после
слов Небензи, что щас украина начнет вешать это убийство на кремль - климкин
тут же сделал это...


\iusr{Игорь Лесев}
ну это у нас утомительная норма. Если бы он погиб в ДТП, тоже нашли б русский след

\iusr{Матвей Кублицкий}
\textbf{Игорь Лесев} какой я вежливый утром был. Климкина всего лишь идиотом назвал... @igg{fbicon.face.tears.of.joy}{repeat=3} А оказалось что он идиот в кубе....

\iusr{Матвей Кублицкий}
\textbf{Игорь Лесев} щас самая смешная часть марлезонского балета будет. Заказчика убийства будут колоть на связь с Кремлем. И колонут же... одна нестыковка останется: российские спецслужбы, имеющие один из лучших спецназов в мире, искали через посредника киллера в Киеве @igg{fbicon.face.tears.of.joy}{repeat=3} 
\end{itemize} % }

\iusr{Serhiy Vorobey}

До відома, він воював за Росію у обидвох Російсько-Чеченських війнах. Також
висвітлював окупацію Грузинських територій з про-російського боку.

\begin{itemize} % {
\iusr{Игорь Лесев}
я знаком с его био. И к чему это "до видома"?

\iusr{Иван Мартынюк}
Значить засланий козачок до Києва.

\iusr{Tanya Kolpakova}
\textbf{Иван Мартынюк}, следовательно, казачка раскрыли и ликвидировали

\iusr{Иван Мартынюк}
\textbf{Tanya Kolpakova} все може бути.
\end{itemize} % }

\iusr{Вениамин Иблздикоев}
Цитируя его же: Земля стекловатой...

\iusr{Марина Иванова}
Бояться следует, Гонопольмкому, Киселеву(Евгению) да и Шустеру. Всяко может быть.

\end{itemize} % }
