% vim: keymap=russian-jcukenwin
%%beginhead 
 
%%file letters.kipcharskij
%%parent letters
 
%%url 
 
%%author_id 
%%date 
 
%%tags 
%%title 
 
%%endhead 

Доброго дня! Щиро дякую! Щодо викладання, власне кажучи, ця лінка на файл
приватна, її можуть бачити лише ті люди, які мають цю лінку. Щодо дозволу на
коментарі і т.д. Тут моя позиція така.  І взагалі щодо авторських прав на все
те, що лежить у відкритому доступі.  Справа в тому, що це все не є якимись
приватними сімейними фотографіями, які лежать в закритому доступі. Це все,
спогади, щоденники - неймовірно цінні свідчення про жахливу трагедію, трагедію
війни, яка зачепила всіх нас, і насправді це все, і ваші спогади, і коментарі,
все, до найменших деталей, потрібно всім нам, всьому Українському Народу, все
це потрібно Україні, а не мені персонально. Потім. Ми живемо в інформаційній
ері, де все, абсолютно все, що викладається публічно, може бути використовано
будь-ким у будь-який спосіб. Тому я думаю, що всі люди, що пишуть коментарі або
публікують пости, повинні бути свідомі щодо того, що іх слова або фото можуть
потім з'явитись де завгодно і коли завгодно і у який завгодно спосіб. Тому... з
однієї сторони, тут є декілька сторін. З однієї сторони, моя мотивація в тому,
що наша історія, особливо щодо цих надзвичайно трагічних подій повинна бути
записана, не повинна бути втрачена, і повинна бути відображена у максимально
доступній формі для подальшого збереження і розповсюдження. На жаль, те що на
фейсбуці знаходиться - може бути втрачено будь-якої миті - через (1) цензурну
політику фб (2) фб приватна компанія, яка може впасти і розвалитись і сама по
собі. Колись ера фб почалась, і вона може закінчитися коли завгодно і доволі
несподівано, приблизно так само як лопаються банки або компанії, так само фб
може лопнути і все, що зберігається тут, може пропасти назавжди. І Ваші
спогади, і спогади Світлани - все може випаруватись просто так, якщо не
докласти зусиль до збереження їх в надійному вигляді...  Так.. тобто з однієї
сторони моя мотивація записувати все що пишеться важливого, для України, заради
нашої Перемоги, заради покарання злочинців, і заради наступних поколінь.  З
іншої сторони, я хочу співпрацювати і дійсно допомагати людям, а не просто
брати тексти. Тому тут потрібен баланс між тим, наскільки потрібно питати
дозволу у конкретних людей, і тим, наскільки ці або інші пости важливі для
України. І я так чи інакше спочатку записую, що бачу а потім вже питаю дозволу.
Як це пишеться латиною, veni vidi scripsi - прийшов, побачив, записав. Тому...
щодо коментарів, я думаю, питати окремого дозволу на кожен коментар, це вже
занадто. А щодо того, щоби дописати подяки і т.д. немає ніяких проблем,
насправді моя роль тут лише як переписувача, і все. От колись Нестор написав
Повість Времьяних Літ.  Оригінал втрачений, і все, що дійшло до нас, це вже
переписування, які робили безвісні монахи у своїх келіях. І я так само, можна
сказати, такий собі трохи монах, мотивація така сама )))  І в принципі можна
зробити який завгодно дизайн, додати що завгодно, шрифти, посилання, навіть
зробити книжку на 1000 сторінок (хоча треба буде чекати годину-дві-три, поки
компіляція не завершиться) і т. д.  Якщо Ви вже мали досвід з TeX/LaTeX, то я
думаю Ви і самі знаєте, що це неймовірно потужна і гнучка система, створена
дійсно геніальним Дональдом Кнутом у свій час, за допомогою якої можна зробити
все що завгодно, навіть альманах по розв'язуванню шахових задач (до речі, я дуже люблю шахи 
і я недавно прочитав, що у вас в Маріуполі є багаторазовий чемпіон з шахів - що
дійсно фантастично!).  Так що можна включити туди все що побажаєте, додати
титульну сторінку, вступ, моя роль тут технічна в основному. Я можу просто
відносно швидко все зшивати докупи в печатний і зручний вигляд, от і все (хоча
за цією швидкістю роботи стоїть багато попередньої праці на протязі років по
розробці відповідних методів). А щодо спогадів інших людей, звичайно
надсилайте, буде дуже добре. Я записую зараз по можливості і моїм силам, все що
є важливим. До речі, програма дозволяє легко комбінувати записи у будь-який
спосіб, і в моїх планах зробити пізніше розбивку по кожному дню. І ось ще файл
нижче (ще ніде не викладався, лінка приватна, (то вже інші спогади - поки що
ніде не викладені, але я з цим особливо не поспішаю, оскільки тема вся ця
досить чутлива, я це розумію )
