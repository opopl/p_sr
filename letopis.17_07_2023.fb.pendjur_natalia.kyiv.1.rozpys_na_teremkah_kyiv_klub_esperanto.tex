%%beginhead 
 
%%file 17_07_2023.fb.pendjur_natalia.kyiv.1.rozpys_na_teremkah_kyiv_klub_esperanto
%%parent 17_07_2023
 
%%url https://www.facebook.com/natalia.pendiur/posts/pfbid02B3iLYKuJS4YGLg3ft77vhbQdJWtNLhZm7RVu7nK7NfydFeU63aKbJGaTi5cYR448l
 
%%author_id pendjur_natalia.kyiv
%%date 17_07_2023
 
%%tags 
%%title Цей розпис на Теремках ми зробили разом з Київським клубом есперанто
 
%%endhead 

\subsection{Цей розпис на Теремках ми зробили разом з Київським клубом есперанто}
\label{sec:17_07_2023.fb.pendjur_natalia.kyiv.1.rozpys_na_teremkah_kyiv_klub_esperanto}

\Purl{https://www.facebook.com/natalia.pendiur/posts/pfbid02B3iLYKuJS4YGLg3ft77vhbQdJWtNLhZm7RVu7nK7NfydFeU63aKbJGaTi5cYR448l}
\ifcmt
 author_begin
   author_id pendjur_natalia.kyiv
 author_end
\fi

Цей розпис на Теремках ми зробили разом з Київським клубом есперанто. 

Я саме проходила вишкіл для цивільних в Українському легіоні. Тоді відчула на
собі, наскільки це виснажлива праця - розумово й фізично - бути у війні,
ухвалювати швидкі рішення й робити справу незважаючи на біль чи страх. 

Після заняття з інструкторками-парамедиками я й обрала ідею:

розпис присвячено жінкам у війні. 

Перша постать перевозить рікою кулі символічного світла. Це волонтерка. Її
обличчя - це обличчя волонтерки Насті з Ірпеня, яка опікується собаками. Її
фото ви точно пам'ятаєте. 

Друга жінка із символічною зброєю спостерігає обстановку і готова захищатися
будь-якої миті. У ній є риси парамедика Хельги.  

Третя постать тримає пташку: ми вкладали в цю частину малюнка вдячність
медикам, які зберігають людське життя. 

Звісно, вийшло далеко не ідеально. Проте наша вдячність безмежна. 

вул. Конєва, 7, Київ

Нумо вже малювати! 

Якщо хочете розписати щось старе, обдерте - ви знаєте, куди звертатися!
