% vim: keymap=russian-jcukenwin
%%beginhead 
 
%%file 06_11_2020.news.ru.bbc.1.death.zhvaneckii
%%parent 06_11_2020
 
%%url https://www.bbc.com/russian/news-54833848
%%author 
%%tags 
%%title 
 
%%endhead 

\subsection{Умер Михаил Жванецкий, король смеха и печальный мудрец}
\Purl{https://www.bbc.com/russian/news-54833848}

6 ноября 2020
Жванецкий

\ifcmt
pic https://ichef.bbci.co.uk/news/800/cpsprodpb/DFCF/production/_115259275_0683da0a-3d4b-44d1-abf4-4444b177b349.jpg
caption Умер Михаил Жванецкий, король смеха и печальный мудрец. Автор фото, Vyacheslav Prokofiev/TASS
\fi


{\bfseries
Умер писатель-сатирик Михаил Жванецкий. Ему было 86 лет. Из творцов российской
культуры последних десятилетий Жванецкий по всеобщей любви и известности стоял,
пожалуй, рядом с Владимиром Высоцким. Его смех порой был смехом сквозь слезы:
да, человек смертен, да, мир несовершенен, поделать с этим ничего нельзя -
остается только относиться ко всему с юмором.
}

"Вот и все. Ты ушел от нас навсегда. Невосполнимая утрата. Друг мой, ты
незабываем. Ты всегда в моем сердце", - написала певица Алла Пугачева в
"Инстаграме" в посте с фотографией Жванецкого.

Президент Украины Владимир Зеленский (Жванецкий был народным артистом
Украины) написал в "Твиттере": "Как говорил Михаил Жванецкий, ничего
страшного, если над тобой смеются. Значительно хуже, когда над тобой
плачут. Сегодня такой горький день. Соболезнования всем, кто его любил и
знал. Я - среди пылких поклонников силы мысли Михаила Михайловича. Его
сатира стала для многих мудростью".

IFrame
Для просмотра этого контента вам надо включить JavaScript или использовать
другой браузер
Подпись к видео,

Михаил Жванецкий о начале карьеры, чувстве юмора и секрете успеха

Если в 1950-х властителями дум в России были поэты, то 1980-е стали
временем расцвета разговорного юмора и сатиры. На эстраде и на телевидении
блистали Геннадий Хазанов, Григорий Горин, Аркадий Арканов, Михаил
Задорнов, Евгений Петросян, Лион Измайлов, Михаил Евдокимов, дуэт
Ширвиндт-Державин. Но на этом небосводе талантов ярче всех сияла звезда
Жванецкого.

%* "Может, что-то в консерватории подправить?" Лучшие монологи Михаила
%Жванецкого
%* Михаил Жванецкий: хорошо ли вы знаете его афоризмы?
%* Михаил Жванецкий: все равно я за новую жизнь
%* Умер Роман Карцев - артист, знакомый по фразе "Вчера по пять, но
%большие"

\subsubsection{Писатель Жванецкий}

Его отличало великолепное чувство языка. Он умел играть словами, ставить их в
неожиданные сочетания и выворачивать наизнанку: "Лучше с любовью заниматься
трудом, чем с трудом заниматься любовью". Или сказать совершенно обычную фразу,
а после секундной паузы добавить нечто, от чего зал взрывался хохотом: "Крейсер
под моей командой не войдет в нейтральные воды. Из своих не выйдет".

\ifcmt
pic https://ichef.bbci.co.uk/news/412/cpsprodpb/96E8/production/_106023683_tass_31805259.jpg
caption Портфель Жванецкого - такой же неотъемлемый атрибут, как кепка Ленина и сигара Черчилля. Автор фото, Вячеслав Прокофьев/ТАСС
\fi

До Жванецкого бытовало мнение, что писать хорошие тексты и представлять их
на сцене - разные профессии. Его предшественник и учитель, великий Аркадий
Райкин, был актером: играл, переодевался, перевоплощался.

Единственным инструментом Жванецкого было слово. Когда требовалось дать
ему определение, неизменно и совершенно справедливо говорили: "писатель
Михаил Жванецкий".

Он не актерствовал подчеркнуто и нарочито и доказал, что это
необязательно. Просто выходил с кожаным портфелем, доставал листки формата
А4 и читал ровным голосом, а публика стонала от восторга.

Миниатюры Жванецкого несли глубокую житейскую мудрость и, как ни странно,
неизбывную печаль. Он был философ: человек смертен, мир несовершенен,
поделать с этим ничего нельзя, остается только относиться ко всему с
юмором.

Знаменитое: "Я не стал этим и не стал тем. Ничего не получу в смокинге, в
прожекторах, в Каннах. Времени уже не хватит", - это ведь крик души во
время кризиса среднего возраста.

\subsubsection{Последний из одесситов}

По возрасту Михаил Жванецкий был последним выдающимся представителем
"одесского периода" русской культуры, давшего Ильфа и Петрова, Утесова,
Бабеля, Катаева, Олешу. Он родился в "городе у моря" 6 марта 1934 года в
семье врачей. По его собственным словам, "и не знал, что можно печататься
и выступать". Окончил институт инженеров морского флота и семь лет работал
в одесском порту механиком по кранам.

В годы "оттепели" страна не сделалась свободной, но перестала быть
угрюмой. То было время взрывного расцвета юмора: в кино, на эстраде, в
молодежных театрах миниатюр, ставших предтечей КВН, в стенгазетах.

Еще во время учебы Жванецкий начал писать миниатюры и монологи и создал
самодеятельный театр "Парнас-2". Там же выступали его будущие многолетние
партнеры Роман Карцев и Виктор Ильченко.

В 1963 году во время гастролей в Одессе Ленинградского театра миниатюр
молодой автор представился его руководителю Аркадию Райкину, который взял
его произведения в свой репертуар, а спустя несколько месяцев пригласил в
театр завлитом.

Сотрудничество двух выдающихся людей продолжалось шесть лет. При этом
народ не знал ни того, что многие бывшие у всех на слуху тексты, в том
числе "Авас", "В греческом зале", "Век техники", созданы Жванецким, ни
вообще имени последнего.

Таково было условие Райкина. Великий артист не опускался до присвоения
чужого труда, а просто обходил вопрос молчанием. Публика же знала
"миниатюры Райкина" и в дальнейшее не углублялась.

"Наступил момент, когда подобное положение Жванецкому показалось обидным.
Входя в троллейбус и слыша реплики собственного сочинения, ему хотелось
обратить внимание пассажиров на то, что автор-то вот он", - признался
Райкин в мемуарах, но поведения не изменил.

Впоследствии Райкин и Жванецкий отдавали должное друг другу, но в
воспоминаниях и высказываниях обоих сквозила недосказанная публично обида.

\subsubsection{После Райкина}

Довольно скоро друзья стали спрашивать Жванецкого, почему бы ему не начать
самостоятельную карьеру. Мешал, по его словам, недостаток уверенности.

В 1970 году Жванецкий, Карцев и Ильченко ушли от Райкина в Одесскую
филармонию, где быстро прославились в качестве, как говорили тогда,
артистов разговорного жанра.

Жванецкий стал проводить все больше времени в Москве и в 1983 году
перебрался туда окончательно, основав собственный театр миниатюр, который
возглавлял до конца жизни.

Во время перестройки он включился своим творчеством в политику, и не было
сомнений, на чьей стороне его симпатии. Говорили, что вклад Жванецкого в
развенчание "совка" не меньше вклада самого Горбачева.

В 1990-х годах он несколько сбавил обороты, замечая, что охотников
высмеивать новую российскую власть и без него довольно, а восхвалять и
поддакивать он не умеет.

\subsubsection{Дежурный по стране}

В одном из интервью писатель сказал, что в современности ему больше всего
нравятся "неограниченные возможности и разнообразие - после того унылого
однообразия, в котором мы жили"..

С 2002 года Михаил Жванецкий вел программу "Дежурный по стране" на
телеканале "Россия-1".

Он выступал против содержания под арестом юриста ЮКОСа Светланы Бахминой и
подписал открытое письмо деятелей культуры в защиту участниц Pussy Riot. В
2015 году минкульт Украины включил его в "белый список" российских
артистов и интеллектуалов.

В июне того же года на церемонии вручения премии "ТЭФИ" Жванецкий прочитал
монолог, в котором юная сотрудница телевидения поучает маститого
писателя-юмориста: "Это телевидение, это не для умных, дед. Для рейтинга в
лоб надо!".

Российские СМИ сообщали, что шутки Жванецкого вызвали недовольство
присутствовавшего в зале телевизионного начальства.

Прощальным выходом писателя стало большое выступление в честь его 85-летия
в московском концертном зале имени Чайковского.

В октябре 2020 года он объявил, что больше не станет появляться перед
публикой. "Стареть нужно дома", - сказал от его имени журналистам
представитель Жванецкого.

Михаил Жванецкий был народным артистом России, кавалером ордена "За
заслуги перед Отечеством" IV и III степени, почетным гражданином Одессы и
президентом Всемирного клуба одесситов. В 2009 году власти города
переименовали бывший Комсомольский бульвар в бульвар Жванецкого.

Писатель не любил распространяться о своей личной жизни. Известно, что его
первый брак продлился недолго, а впоследствии он имел, по одним сведениям,
четырех, по другим - пять детей от женщин, с которыми не состоял в
зарегистрированных отношениях.

Изумительные афоризмы Жванецкого остались с нами навечно. Давайте вспомним
некоторые из них:

\begin{itemize}
\item * Виден свет в конце тоннеля, но только вот тоннель, сука, не кончается!
\item * Возьми за правило прерывать беременность еще в период знакомства.
\item * Имей совесть и делай что хочешь.
\item * Ты такой, какая женщина возле тебя, ты в ее вкусе.
\item * В историю трудно войти, но легко вляпаться.
\item * Вы видели человека, который никогда не врет? Его трудно увидеть, его
же все избегают.
\item * Лысина - это полянка, вытоптанная мыслями.
\item * У нас чего только может не быть. У нас всего может не быть. У нас чего
только ни захочешь, того может и не быть.
\item * Не нарушайте мое одиночество и не оставляйте меня одного!
\item * Счастье - это увидеть туалет и успеть до него добежать.
\item * Вначале было Слово.... Cудя по тому, как развивались события дальше,
Слово было непечатным.
\item * Если вам говорят, что вы многогранная личность - не обольщайтесь.
Может быть, имеется в виду, что вы гад, сволочь и паразит
одновременно.
\item * Не водите машину быстрее, чем летает ваш ангел-хранитель.
\item * Мыслить так трудно, поэтому большинство людей судит.
\item * Милиционер - не бандит, от него спасения нет.
\item * Что значит - выпить? Он не понимает, что он говорит. Я не понимаю, что
я говорю. Но мы понимаем друг друга.
\item * Друзья познаются в беде, если, конечно, их удается при этом найти.
\item * Сегодня слова: "Есть на телевидении одна хорошая передача..." -
напоминают донос.
\item * Танки ходят по улицам вместо троллейбусов. Для защиты населения от
населения. (о путче ГКЧП)
\item * И когда я слышу: поверьте мне как министру! - не верю именно как
министру! (о последствиях Чернобыля)
\item * Бюрократы и приспособленцы бессмертны. Мы улучшали их породу, съедая
самых слабых.
\item * Что охраняешь, то и имеешь.
\item * Одно неосторожное движение - и ты отец.
\item * Алкоголь в малых дозах безвреден в любом количестве.
\item * Мало знать себе цену - надо еще пользоваться спросом.
\item * Ничто так не ранит человека, как осколки собственного счастья.
\item * Нормальный человек в нашей стране откликается на окружающее только
одним - он пьет. Поэтому непьющий все-таки сволочь.
\item * Порядочного человека можно легко узнать по тому, как неуклюже он
делает подлости.
\item * Хочешь всего и сразу, а получаешь ничего и постепенно.
\end{itemize}

