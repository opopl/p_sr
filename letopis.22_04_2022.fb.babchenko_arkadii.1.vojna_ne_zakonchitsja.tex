% vim: keymap=russian-jcukenwin
%%beginhead 
 
%%file 22_04_2022.fb.babchenko_arkadii.1.vojna_ne_zakonchitsja
%%parent 22_04_2022
 
%%url https://www.facebook.com/babchenkoa/posts/4535847306515467
 
%%author_id babchenko_arkadii
%%date 
 
%%tags 
%%title Война не закончится через месяц. И через два
 
%%endhead 
 
\subsection{Война не закончится через месяц. И через два}
\label{sec:22_04_2022.fb.babchenko_arkadii.1.vojna_ne_zakonchitsja}
 
\Purl{https://www.facebook.com/babchenkoa/posts/4535847306515467}
\ifcmt
 author_begin
   author_id babchenko_arkadii
 author_end
\fi

Война не закончится через месяц. И через два. И даже через пять, скорее всего,
не закончится. Пентагон говорит, что еще год. Что ещё как минимум до зимы.
Пентагону верю теперь, как себе. 

Но я думаю, даже больше. 


Сколько? Я не знаю. 

Но не завтра и не на следующей неделе точно. 

Российское народонаселение переживает сейчас ровно ту же стадию, которую оно
переживало между Первой и Второй чеченскими войнами. Первую чеченскую
российское общество - тогда оно еще было обществом - не приняло совершенно.
Лейтмотивом было «нет войне». 

Через два с половиной года - даже меньше, от подписания мира в Хасавюрте
тридцать первого августа девяносто шестого до начала второй серии в октябре
девяносто девятого в Чабан-Махи и Кара-Махи - главным лейтмотивом
народонаселения - общества там уже не было - стало «мочить в сортире». 

Если в девяносто пятом вид горящих русских солдат на улицах Грозного вызывал
одно лишь желание - прекратить эту бойню - то в девяносто девятом эти же кадры
вызывали у них уже только желание отмщения. Во вторую войну российская армия
входила в Чечню с одним только желанием - мстить. 

Наши мальчики не могли умереть в Грозном просто так. Это не могло быть
напрасно. 

А, значит, их всех надо убить. 

Замочить в сортире. 

Сейчас там происходит то же самое. Все, кто мог бежать, дезертировать,
отказаться - бежали, дезертировали, отказались. Их больше нет. Остались только
те, кто готов подыхать. За Леху, за Пашку, за Сашку. Которого убили проклятые
бандеровцы. За ВДВ. За нас, за вас и за спецназ. За Русь усрусь.

И теперь они будут сплачиваться вокруг все той же самой идеи - наши мальчики не
могли погибнуть под Киевом просто так. 

А значит, их всех надо убивать. 

Пусть попробуют, конечно. 

Но стадия там сейчас такая. 

Потом-то они своего царя, безусловно, скинут. Вокруг того сплотились в 1914, а
в 1918 они же его и расстреляли. Против этого тоже с вилами и голой жопой
попрут. Когда годами жрать будет нечего, задницу будут вытирать газетой
«Правда», туалетная бумага будет только в колбасе, а колбаса - только у
верхушки партактива. Это неизбежно. Штыки в землю, наплявать, наплявать,
надоело ваявать - всё это, обязательно, будет. 

Но не сейчас. 

Сейчас они сплачиваются вокруг идеи отмщения.

А в середине этой идеи - Путин. 

Следовательно, они сплачиваются вокруг него. 

Простая геометрия, ничего не поделаешь.

Если Путин решит все же провести мобилизацию в ближайший год - у него с этим не
будет вообще никаких проблем. 

У него уже сейчас нет никаких проблем с пополнением личного состава. В
военкоматах полно желающих подписать контракт. Срочники готовы переходить на
контракт через одного. То же самое было и во Вторую Чечню. Всего лишь два года
назад вся страна была в шоке от происходящего. А как только объявили набор - в
военкоматах были толпы. Толпы! Я это видел своими глазами. 

Да, идеей была месть, но она была все же на втором плане. А на первом были
деньги. Восемьсот шестьдесят долларов в месяц. Сумма по тем временам очень
большая. 

Если Путин будет платить сейчас хотя бы тысяч пятьдесят - от желающих поехать
мочить бандеровцев у него отбоя не будет. 

Поверьте мне. 

Я знаю.

И никакие переговоры сейчас не имеют вообще никакого значения. Вы вообще можете
не обращать на них внимания. Все эти встречи Арахамии с Мединским - это просто
белый шум. Най буде, конечно, но сейчас они абсолютно безрезультативны. 

Переговоры могут иметь результат только в двух случаях. 

Либо одна из сторон не способна больше продолжать боевые действия. 

Либо обе стороны достигли паритета сил и не способны больше ни продвинуться, ни
продолжать наносить урон противнику. 

Первый вариант - вариант Финляндии. Да, финны за три месяца Зимней войны
накрошили сто пятьдесят тысяч орков. Но через три месяца они не могли больше
вести войну. 

А СССР, несмотря на такие гигантские потери - мог. 

Поэтому финны вынуждены были сесть за стол переговоров и отдать территории. 

У нас ситуация пока другая. Да, орков накрошили двадцать тысяч. Но.

И Украина и Россия боевые действия: 

\begin{itemize}
  \item а - вести могут
  \item в - вести хотят
  \item с - вести мотивированы.
\end{itemize}

А, значит, всё будет решаться на поле боя. 

Вы все думаете, что российская армия слабо мотивирована, что там одни утырки,
над которыми мы ржем - и в значительно мере это действительно так и есть - но
вот в чем проблема. 

Простейшим не нужна сложная мотивация. 

Простейшим достаточно мотивации простой. 

И простая мотивация в голове простейших - работает. 

Хотите, скажу, какая у них мотивация? 

«Х@хлы чота развыёбывались, надо им пизды дать»

Вот и вся их мотивация. 

Всё! Больше там нет ничего! 

И она работает преотлично.

Посмотрите фотографии Бучи и Ирпеня после орды. Посмотрите, что они пишут своим
говном на стенах. Именно это. «Х@хлы пидарасы». «Х@хлы, сдохните». 

Вот и все, что простейшим нужно иметь в голове, чтобы царь толпами отправлял их
закидывать врага своими обгорелыми трупами. 

И это работает. 

Это всегда там работало. 

И всегда будет. 

Большевики провели прекрасный эксперимент. О, да. Просто снимаю шляпу.
Эксперимент удался на миллион процентов. Если десятилетиями - а до этого
веками, столетиями - убивать все, что хоть на вершок возвышается над пляшкой
«Жигулевского» - вы получите то, что получили. 

Страну АУЕ. Страну «Сдохни ты сегодня, а я завтра». 

Одну сплошную Колыму, один сплошной Дальлаг от моря до моря. 

Где половина зеки, половина вертухаи.

Страну, в которой своя-то жизнь не стоит ни копейки, не то, что чужая. 

Что вы хотите от страны, в которой десятилетиями главным смыслом жизни за
Уралом было взять велосипедные цепи, арматуру, заточки - и идти убивать тех,
кто живет в соседнем районе? 

Химммаш на Тяжмаш. Встречаемся посередине. 

А потом скорая растаскивает трупы с проткнутыми животами и пробитыми головами. 

ТАК ЧЕГО ВЫ ОТ НИХ ЖДЕТЕ, КОГДА ОНИ С ЭТИМ МИРОВОЗЗРЕНИЕМ ПРИШЛИ В УКРАИНУ?

Не в соседний район.

В чужую стану.

Приказ убивать людей в Буче исходил от Путина, или нет? Когда вы поднимаете эту
тему, даже когда вы участвуете в таких спорах - вы искажаете реальность. Потому
что подоплекой этого является все тот же поиск хороших русских. Ну, они же не
сами это сделали, это им приказ такой пришел.

Нет! 

Сами! 

Им не надо никакого приказа!

Им не надо никакого приказа, чтобы убивать и пытать людей!

В Чечне я два раза не дал расстрелять людей. Один раз просто человека, который
шел мимо нашего блокпоста. Его схватили, привязали к дереву, оставили так на
всю ночь зимой в горах, а утром потащили расстреливать. Я просыпаюсь - к дереву
привязан человек в полубессознательном состоянии. Сейчас расстреляем. Да вы
чего, охренели что ли? Драка. Дошло до хватания за оружие. Не дал убить
человека.

Второй раз когда из города под обстрелом уходила группка женщин. Под белым
флагом. Вася-сапожник поднял автомат. «Там боевики в середине, они их прячут».
Я задрал его ствол в небо. Очередь ушла в облака. Ты охуел что ли? Это женщины!
Женщины! Бабы, старухи и дети! Да пусть хоть сто боевиков - пусть уходят! В его
глазах было непонимание. Я был предателем тогда в его представлении. 

Ни в первом, ни во втором случае никаких приказов пытать человека и
расстреливать женщин - не было. 

Вася-сапожник - тихий, спокойный, добрый паренек. Умные глаза. Безотказный
товарищ. Надежный друг. 

Этот тихий умный паренек по своей инициативе спокойно убил бы полтора-два
десятка женщин - и так и не смог понять, почему я не дал ему этого сделать. 

Просто шли, просто шмальнул, просто убил. Всё. 

И никакого Путина не надо. 

Точность формулировок влечет за собой точность осознания реальности. Не надо
никаких эвфемизмов. Не надо искать новых терминов. Не надо никакого слова
«рашизм». Даже «орки» и «Мордор», столь часто употребляемые мною - это так,
игра слов, попытка найти эрзац-замену. 

Но вещи надо называть своими именами. 

А свое имя у происходящего в России только одно. 

Фашизм. 

В Россию пришел фашизм. Россия - фашистская страна. Не «как фашистская», не
\enquote{как Германия} - а именно «фашистская». Фашизм там был всегда, но
теперь он оформился окончательно и стал и государственной идеологией, и
национальной идеей. 

На украинскую землю пришли фашисты. 

Точка. 

У некоторых из них могут быть умные глаза.

Но, в отличии от Германии, добивать фашизм в его логове в Москве в этот раз
никто не собирается. Никто не будет оккупировать Россию. Никто не будет
устанавливать там оккупационные администрации и вводить план Маршала по
денацификации. Не будет генерала Макартура, который разработает и блестяще
претворит в жизнь план перезагрузки Японии. Не будет «пойнт Чарли» и кающегося
Вилли Брандта. 

Пока, по крайней мере. 

Пока все закончится тогда, когда стороны либо достигнут паритета, и остановятся
там, где остановятся, либо тем, что российские войска будут выкинуты с
территории Украины и уберутся восвояси. 

Но сама Россия при этом никуда не денется. 

И фашизм в России никуда не денется. 

Пока будет Путин, там будет фашизм. 

Путин будут еще лет десять-пятнадцать. 

Значит, еще лет десять-пятнадцать на границе Украины будет существовать
фашистское государство с ядерным оружием и мотивацией мочить ххлв в сортире. 

И никуда от этого не деться. 

Все только начинается. 

Да и даже после Путина... Там никогда ничего не менялось независимо от строя.
Улус ли Джучи. Удельные ли княжества. Царизм ли. Большевизм ли. Сталинизм ли.
Путинизм ли. Или относительно травоядный ельцинизм. 

Три раза в истории эта страна было относительно более-менее демократична -
первый раз при Лжедмитрии, которого они же сами за эту свободу и убили. Потом с
февраля 1917 по октябрь 1917.  И с 1991 года по 1993. Всё остальное время она
всегда, как неваляшка, возвращалась в свое исходное состояние. 

Ну, вы и так все понимаете, не буду растекаться мысью по древу.  

И надо быть к этому готовым. Чтобы не сломаться психологически от невозможности победы в ближайшее время и обещанных шашлыков. 

Потому что излишний, необоснованный оптимизм, опасен точно так же, как
предательская зрада. 

Я вот сейчас читаю Горбулина. Он пишет, что за всю историю войн более слабая
армия побеждает более сильную примерно в тридцати процентах случаев. Но при
одном условии - она защищает свою землю. 

Это как раз наш случай. 

Украина победит. Безусловно. Однозначно. Я в этом не сомневаюсь вообще ни
секунды. 

Но путь до победы будет долгим. 

И вот именно к этому и надо быть готовым. 

В рамках проекта \enquote{Журналистика без посредников}.

К сожалению, пока я живу только на ваши донаты. 

Подписывайтесь на меня на Патреон. Спасибо:

\url{https://www.patreon.com/babchenko}

Мой телеграм: \url{https://t.me/s/babchenko77}

Мой ютуб: \url{https://www.youtube.com/watch?v=BsYJYQMdxkg}
