% vim: keymap=russian-jcukenwin
%%beginhead 
 
%%file slova.idea
%%parent slova
 
%%url 
 
%%author 
%%author_id 
%%author_url 
 
%%tags 
%%title 
 
%%endhead 
\chapter{Идея}
\label{sec:slova.idea}

%%%cit
%%%cit_head
%%%cit_pic
%%%cit_text
Почему в Украине нет объединяющей \emph{идеи}.  За годы независимости в Украине
уже почти полностью исчезла крупная промышленность и страна стремительно
превращается в аграрно - сырьевой придаток Запада. Через 5-10 лет процесс будет
полностью завершен. Такому неиндустриальному, батрачно - сервисному типу
экономики соответствует феодально - поместная политическая система, которая уже
практически сложилась в Украине.  Однако феодальная система, в основе которой
лежит сословное неравенство и сословные привилегии, нуждается в некой
объединяющей идее, которая позволяет держать все сословия вместе и наполнить их
совместное существование каким - то смыслом, чтобы избежать межсословной резни.
В Российской империи такой \emph{объединяющей идеей} были самодержавие,
православие, Москва - Третий Рим, имперскость, державность и т.п.  В СССР,
который был еще более жестким сословным обществом, людей объединяли с помощью
химеры коммунизма, ну и КГБ, конечно
%%%cit_comment
%%%cit_title
\citTitle{Объединяющая идея европейского рая в Украине уже не работает / Лента соцсетей / Страна}, 
Андрей Головачев, strana.ua, 06.07.2021
%%%endcit

%%%cit
%%%cit_head
%%%cit_pic
%%%cit_text
\emph{Идея} прекрасно работала и достигла своей кульминации во время Революции
Достоинства и последующего дарованного Европой безвиза. И затем, после
повышения коммунальных тарифов до рыночного уровня \emph{идея} стала идти на спад.
Народ хотел ЕвроРая, но так, чтобы тарифы остались советскими и чтобы не надо
было декларировать все свои доходы и уплачивать с них налоги. Ну, а после
заявления Ж. К. Юнкера о том, что в \enquote{ближайшие 25 лет Украина не не имеет
шансов вступить в ЕС}, \emph{идея} и вообще потеряла свой объединяющий потенциал. Все
удовлетворились одним безвизом, - богатые чтобы свободно путешествовать по
Европе, а бедные чтобы там батрачить без визовых хлопот. На этом \emph{евроидея} себя
исчерпала, так как проводить тяжелые, серьезные и болезненные еврореформы,
народ не изъявил желания
%%%cit_comment
%%%cit_title
\citTitle{Объединяющая идея европейского рая в Украине уже не работает / Лента соцсетей / Страна}, 
Андрей Головачев, strana.ua, 06.07.2021
%%%endcit

%%%cit
%%%cit_head
%%%cit_pic
%%%cit_text
Ладно, с той стороны у них «раковая опухоль». Но элитке ведь похуй даже на
своих воинов.  Да, есть \emph{идейные}, которые видят разрешение конфликта в
военном реванше.  Но \emph{идейные} в 19-м с треском проиграли. Люди в 19-м
проголосовали за мир.  А лидер \emph{идейных} сейчас селфится на Донбассе рядом
с табличкой «Нью-Йорк», а затем рванет в свой особняк в Испании и продолжит
туристический сезон от Галапагос до Мальдив.  Но теперь и нью-элитка стала
такой же. Она назовет двести девятнадцать причин, почему нельзя на Донбассе
развести войска.  Они каждый день будут выпускать на эфиры прилизанных спикеров
и тереть нам о «безопасности», но при этом не сделать ничего, чтобы добиться
безопасности хотя бы для своих же военных
%%%cit_comment
%%%cit_title
\citTitle{Наши ребята гибнут там, где у власти нет ни стратегической, ни тактической цели}, 
Игорь Лесев, strana.ua, 08.07.2021
%%%endcit

%%%cit
%%%cit_head
%%%cit_pic
%%%cit_text
«Може, для когось це звучить парадоксально, але того дня Україна святкувала
свій День Незалежності. І навіть «Голос України», газета Верховної Ради,
зазначила це на своїх шпальтах. Але до дніпровської влади не дійшла ця
\emph{ідея}, і ті, хто намагався реалізовувати Декларацію про державний
суверенітет на практиці, сприймалися як чужі. Цей день став знаковим, після
якого почалася хвиля демонстрацій, протестів, масових мітингів. Він засвідчив
прагнення українців до незалежності», – зауважив історик
%%%cit_comment
%%%cit_title
\citTitle{«Нас називали безумцями». 30 років тому над містом Дніпром вперше підняли синьо-жовтий прапор}, 
Юлія Рацибарська, www.radiosvoboda.org, 16.07.2021
%%%endcit

%%%cit
%%%cit_head
%%%cit_pic
%%%cit_text
Сама \emph{идея} Государства берёт верх над всем разнообразием иных типов устройства
человеческого общества своею законченностью и самодостаточностью. И становится
своеобразным «веретеном», на которое неспешно наматывается нить социальной
жизни. Покачивающимся, как метроном, и незаметно отмеряющим время на фоне
привычно застывшей человеческой будничности.  В \emph{идее} Государства прослеживается
связь времён. Глубоко ощущается мимолётность земного бытия любого человека и
ещё что-то вечное, объединяющее, стоящее за этой мимолётностью жизни. В этой
\emph{идее} чувствуются извечные поиски родства со всем миром, мечта о единении,
непрекращающаяся эстафета жизни. И, одновременно, общечеловеческое начало,
помогающее раскрыть национальную сущность человеческого характера, его
особенности в общей гармонии мира
%%%cit_comment
%%%cit_title
\citTitle{Замысел украинского государства: социальность, самодостаточность, независимость. Пятая часть}, 
Акулов-Муратов В. В., analytics.hvylya.net, 18.10.2021
%%%endcit

%%%cit
%%%cit_head
%%%cit_pic
%%%cit_text
Слово, или \emph{идея}, по-моему мнению, является одной из основных движущих сил
развития всей человеческой цивилизации, начиная с эпохи каменных топоров. Еще в
те древние времена идея пойти набить морду соседнему племени заставляла
первобытных мужчин, презрев опасности, отправляться за десятки, а то и сотни
километров от своего обжитого стойбища. Не всегда такие походы завершались
успешно, но это не умаляет влияние самой \emph{идеи} как их движущей силы. С ростом
сознания общества и его членов росли и \emph{идеи}, ими овладевающие. Простые
похождения к соседям с десятком товарищей превратились в великие походы через
полконтинента многотысячных армий Александра Македонского, с его \emph{идей} единой
всемирной державы, и туменов Чингисхана, воплощающих в жизнь грандиозную
концепцию универсальной империи. \emph{Идея-фикс} открытия пути в Индию сподвигла
Колумба на открытие Америки, а опережающие время \emph{идеи} Леонардо да Винча лежат в
основе многих разделов современной медицины и техники.
Практически всегда в основе социальных потрясений, глобальных исторических
событий, важных географических открытий, великих научных достижений, прорывных
технологий лежит \emph{идея}. Она является центральной категорией в философии Платона,
легшей в основу Западной цивилизации. Основополагающие \emph{идеи} учения Конфуция до
сих пор оказывают исключительное влияние на образ жизни и сознание
полуторамиллиардного китайского этноса
%%%cit_comment
%%%cit_title
\citTitle{Украина после Зеленского}, 
Sergius Ak, hvylya.net, 07.12.2021
%%%cit_url
\href{https://hvylya.net/analytics/243091-ukraina-posle-zelenskogo}{link}
%%%endcit
