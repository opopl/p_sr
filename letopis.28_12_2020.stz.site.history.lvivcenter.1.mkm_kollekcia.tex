% vim: keymap=russian-jcukenwin
%%beginhead 
 
%%file 28_12_2020.stz.site.history.lvivcenter.1.mkm_kollekcia
%%parent 28_12_2020
 
%%url https://www.lvivcenter.org/updates/mariupol-uma
 
%%author_id site.history.lvivcenter
%%date 
 
%%tags 
%%title Колекція Маріупольського краєзнавчого музею в Міському медіаархіві
 
%%endhead 
 
\subsection{Колекція Маріупольського краєзнавчого музею в Міському медіаархіві}
\label{sec:28_12_2020.stz.site.history.lvivcenter.1.mkm_kollekcia}
 
\Purl{https://www.lvivcenter.org/updates/mariupol-uma}
\ifcmt
 author_begin
   author_id site.history.lvivcenter
 author_end
\fi

28.12.2020

В межах проєкту \enquote{(Роз)архівування
(пост)індустрії}\footnote{\url{https://www.lvivcenter.org/researches/unarchiving-post-industry}}
Міський медіаархів Центру оцифрував \textbf{1955 негативів з Маріупольського
краєзнавчого музею}. Тепер на нашому сайті можна
побачити\footnote{\url{https://uma.lvivcenter.org/uk/collections/147/photos}},
як був обладнаний мартенівський цех, який захисний одяг носили робітники і як
вони проводили час після роботи.

\ifcmt
  ig https://www.lvivcenter.org/wp-content/uploads/2020/12/20_03_Mariupol_Museum_72_00002-1920x1281.jpg
  @wrap center
  @width 0.9
\fi

Фотографії зробив \textbf{Павло Кашкель} у період 1949-1960 років. Спершу він був
слюсарем на судоремонтному заводі, а у вільний час фотографував. Згодом він
почав працювати фотографом у місцевій газеті \enquote{Приазовский рабочий}, під час
Другої світової став воєнним фотографом, після неї повернувся у редакцію. Для
газети він фотографував робітників на заводах, будівництвах, події у палацах
культури, жінок-інженерок та жінок на тяжкій фізичній роботі, читання
пропагандистських брошур у бібліотеках, відпочинок у санаторіях та зустрічі з
агітаторами, стажування індійських фахівців на радянському заводі тощо.

Фотографії створюють привабливий та ідеалізований образ ро\hyp{}бітничого життя,
відповідають тогочасній тенденції героїзувати працю і естетизувати промислові
ландшафти. На території заводів видно агітаційні елементи, наприклад, на
подвір'ї одного з них був пам'ятник Леніну. Деякі з них сьогодні сприймаються
більш проблемно: наприклад, праця із сіркою в брусках без спеціального захисту
або будівництво посеред морозної зими.

Світлин, які висвітлюють приватне життя, небагато. Однак серед них є зображення
дружини сталевара Шкуропата під час приготування їжі, спостереження за сонячним
затемненням 30 червня 1954 року, художня виставка, оздоровчі процедури у
фотарії, дитяче свято у \enquote{Азовсталі}.

Ця колекція не показує буття промислового міста всебічно, бо фотографії, які до
неї входять, робили на замовлення редакцій чи для музею заводу. Вони
репрезентували дійсність так, як це було вигідно для панівної ідеології,
відповідно до соцреалізму як провідного жанру мистецтва. Можливо, багато кадрів
не пройшли цензуру і не збереглися для нас. Однак ті, що є, свідчать про свій
час візуально та ідейно.

Ми будемо доповнювати колекцію новими матеріалами і заохочуємо стежити за
оновленнями та вивчати образи індустріального Маріуполя.
