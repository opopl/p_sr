% vim: keymap=russian-jcukenwin
%%beginhead 
 
%%file 06_12_2021.fb.fb_group.story_kiev_ua.1.dorogi_kotoryje_my_vybirajem
%%parent 06_12_2021
 
%%url https://www.facebook.com/groups/story.kiev.ua/posts/1813117818885011/
 
%%author_id fb_group.story_kiev_ua,manilova_svetlana.kiev.ukraina
%%date 
 
%%tags gorod,kiev,vybor,zhizn
%%title ДОРОГИ, КОТОРЫЕ МЫ ВЫБИРАЕМ
 
%%endhead 
 
\subsection{ДОРОГИ, КОТОРЫЕ МЫ ВЫБИРАЕМ}
\label{sec:06_12_2021.fb.fb_group.story_kiev_ua.1.dorogi_kotoryje_my_vybirajem}
 
\Purl{https://www.facebook.com/groups/story.kiev.ua/posts/1813117818885011/}
\ifcmt
 author_begin
   author_id fb_group.story_kiev_ua,manilova_svetlana.kiev.ukraina
 author_end
\fi

Вчера, когда я ехала в такси, снова вспомнила о том, о чем писала в группе
почти год назад. Решила повторить свой пост.

ДОРОГИ, КОТОРЫЕ МЫ ВЫБИРАЕМ.

Долго думала, почему я так неплохо отношусь к городскому транспорту. Недавно
нашла ответ, который мне кажется наиболее вероятным. Я была
студенткой-вечерницей, много проводила времени в транспорте, добираться
приходилось не только на работу, но и с работы на занятия, а с занятий,
соответственно, домой. Если учесть, что жила я на Лукьяновке, работала на
Левобережной, училась в районе ВДНХ, то можно сделать выводы. Пребывание утром
в переполненном, как правило, транспорте, конечно, не доставляло особого
удовольствия, зато возвращение домой поздним вечером (занятия заканчивались в
21.50) было приятным, несмотря на усталость и желание поскорее быть дома.
Дорога с места учебы, а именно с механико-математического факультета КГУ,
корпус которого во времена моей учебы находился, как и сейчас, возле ВДНХ,
предопределяла разные варианты. Можно было пойти направо, а можно и налево от
корпуса. Первый вариант предполагал возможность доехать до центра на 11 или 12
троллейбусе, остановка которых была возле выставки. Во втором случае можно было
воспользоваться 38 маршрутом автобуса, пройдя от корпуса в противоположном
направлении, конечная остановка которого была на площади Богдана Хмельницкого.
И всегда, выходя из здания, мы с девочками решали, какой выбираем путь. Почему
с девочками, а не с мальчиками? Потому что у нас было всего 2 или 3 мальчика в
группе. Дорога была долгой, и мы успевали обсудить все, что не получалось на
переменах. А обсуждать всегда было что. Мы все работали, многие уже имели семьи
и детей. 

\ifcmt
  ig https://scontent-lhr8-1.xx.fbcdn.net/v/t39.30808-6/264996133_1553118405051802_3398754261551525456_n.jpg?_nc_cat=111&ccb=1-5&_nc_sid=825194&_nc_ohc=tDfa5jXlM7UAX8p6OVL&_nc_ht=scontent-lhr8-1.xx&oh=004e21207af88f44f0688edef602d7be&oe=61B4A113
  @width 0.4
  %@wrap \parpic[r]
  @wrap \InsertBoxR{0}
\fi

Вечерний Киев... Какое же это было удовольствие рассматривать город, хотя уже
казалось, что все давно знакомо. Мы знали каждую остановку и, можно сказать,
каждый дом. Годы многое стерли из памяти. Если я раньше без труда могла назвать
все остановки в правильной последовательности, то сейчас едва вспомню несколько
названий. Был у нас еще один запасной вариант (чтоб порадовать себя @igg{fbicon.smile} ) на
случай успешной сдачи зачетов, экзаменов и по другим поводам, когда хотелось
попасть поскорее домой. Такси! Тогда мы с девочками в количестве четырех
человек садились в пустое "зеленоглазое такси" и отправлялись в путь. Водитель,
выслушав наши адреса, предлагал свою траекторию движения и развозил нас по
домам. Конечно, этим видом услуг мы пользовались нечасто, но вскладчину
получалось недорого, да и все мы были работающими студентками. 

У меня есть мечта весной поехать в те края и проделать знакомый путь на
городском транспорте. Конечно, многое изменилось за эти годы. К сожалению,
девочек собрать не смогу, да и мальчиков- тоже, потому что жизнь разбросала
многих из нас по разным странам и даже континентам...

\ii{06_12_2021.fb.fb_group.story_kiev_ua.1.dorogi_kotoryje_my_vybirajem.cmt}
