% vim: keymap=russian-jcukenwin
%%beginhead 
 
%%file slova.pokorenie
%%parent slova
 
%%url 
 
%%author 
%%author_id 
%%author_url 
 
%%tags 
%%title 
 
%%endhead 
\chapter{Покорение}
\label{sec:slova.pokorenie}

%%%cit
%%%cit_head
%%%cit_pic
%%%cit_text
Вот не имеет слово дно множественного числа. Очень бы пригодилось для новейшей
истории законотворчества нынешнего состава моноРады. Когда-то, при треклятом
совке, в УССР \emph{покоряли} вершины. Науки, искусства, технического совершенства. Я
еще застал такое \emph{покорение}. Теперь иное. Теперь \emph{днища покоряют}. От одного - к
другому. Закон о коренных народах из этих \enquote{\emph{покорений}}. И далеко, увы не
последнее. Стадно ведь не стыдно. Или как, \emph{моноднопокорители}?
%%%cit_comment
%%%cit_title
\citTitle{Когда-то в Украине покоряли вершины - науки, искусства, техники / Лента соцсетей / Страна}, 
Валерий Песецкий, strana.ua, 04.07.2021
%%%endcit
