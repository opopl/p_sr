% vim: keymap=russian-jcukenwin
%%beginhead 
 
%%file 16_01_2022.stz.news.ua.zaxid.1.figurnoje_katanie_ukraina
%%parent 16_01_2022
 
%%url https://zaxid.net/ukrayintsi_na_chempionati_yevropi_z_figurnogo_katannya_n1533960
 
%%author_id slipchenko_katerina
%%date 
 
%%tags figurnoje_katanie,sport,ukraina
%%title Українці на чемпіонаті Європи з фігурного катання
 
%%endhead 
\subsection{Українці на чемпіонаті Європи з фігурного катання}
\label{sec:16_01_2022.stz.news.ua.zaxid.1.figurnoje_katanie_ukraina}

\Purl{https://zaxid.net/ukrayintsi_na_chempionati_yevropi_z_figurnogo_katannya_n1533960}
\ifcmt
 author_begin
   author_id slipchenko_katerina
 author_end
\fi

\begin{zznagolos}
У Таллінні завершилась континентальна першість
\end{zznagolos}

У Таллінні наближається до завершення чемпіонат Європи з фігурного катання. Усі
змагання відбулись, залишились лише показові виступи.

Українські танцюристи Олександра Назарова та Максим Нікітін завершили змагання
на 10-му місці, повторивши свій результат минулої європейської першості. Те, що
їх вдалося потрапити до чільної десятки, дозволить Україні наступного року мати
у танцях на льоду двох представників.

\ii{16_01_2022.stz.news.ua.zaxid.1.figurnoje_katanie_ukraina.pic.1}

Назарова та Нікітін тренуються у Харкові у Галини Чурилової. Проте програми
ставить їм російський тренер Олександр Жулін. Пара колись тренувалась у нього в
Москві, а у 2014 році перебралась до США до Ігоря Шпільбанда, тренувалась у
француза Фаб’єна Бурза, а потім знову повернулась до Жуліна, але вже «одною
ногою». Однак тренер, що в недавньому інтерв’ю заявив, що задоволений
зовнішньою політикою Путіна, виводить наших фігуристів на лід.

\ii{16_01_2022.stz.news.ua.zaxid.1.figurnoje_katanie_ukraina.pic.2}

До слова, днями Жулін опинився у центрі чергового скандалу назвавши
американського фігуриста Тімоті ЛеДюка, що стане першою небінарною людиною на
зимовій Олімпіаді, «виродком» і додав, що вибачатись не планує. На цих іграх
саме він стоятиме біля бортика під час виступу українців.

Назарова-Нікітін не раз жалілися в інтерв’ю на погані умови тренування у
Харкові. Попри те, що їх програми насичені оригінальними цікавими елементами,
повторити свій найкращий результат – 9 місце у 2017 році вони так і не
повторили. Після ритм-танцю пара була 12-ю й піднялась вище не так завдяки
власному катанню, як завдяки помилкам суперників.

Дебютанти європейської першості Марія Голубцова та Кирило Білобродов фінішували
17-ми. Танцюристи, для яких цей сезон став першим серед дорослих, показали, що
уже мають власний стиль та добрі перспективи.

Одиночник Іван Шмуратко фінішував 12-м. Після того, як він завершив коротку
програму 8-м хотілося б, звичайно, щоб він опинився у чільній десятці та приніс
Україні ще одну квоту на наступний чемпіонат. Проте фігурист не впорався зі
стрибками та посів лише 12-ту сходинку. Проте для українських
чоловіків-одиночників це найкращий результат з 2002 року. «Коли чемпіон
Європи-1993 Дмитро Дмитренко посів 11-те місце в Лозанні, Шмураткові було менше
місяця», – повідомляє «Суспільне».Та й для самого Івана, який посів на минулому
чемпіонаті 22-ге місце, теперішній результат можна вважати прогресом.

Нагадаємо, що спортивна пара Софія Голіченко та Артем Даренський посіла 15-те
місце. Для них цей чемпіонат був дебютним.

У змаганнях жінок Україна участі не брала через те, що наша спортсменка Марія
Андрійчук отримала позитивний тест на коронавірус.

До слова, під час чемпіонату усі спортсмени ту усі журналісти мали здавати тест
щодня. Саме тому зі змагань знімались ще кілька учасників. Дехто вирішив просто
не їхати на чемпіонат, щоб не ризикувати перед Олімпіадою.

Таким чином серед 12-ти медалей 9 (серед них усі золоті) отримали росіяни.

