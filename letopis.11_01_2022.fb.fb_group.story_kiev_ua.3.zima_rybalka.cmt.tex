% vim: keymap=russian-jcukenwin
%%beginhead 
 
%%file 11_01_2022.fb.fb_group.story_kiev_ua.3.zima_rybalka.cmt
%%parent 11_01_2022.fb.fb_group.story_kiev_ua.3.zima_rybalka
 
%%url 
 
%%author_id 
%%date 
 
%%tags 
%%title 
 
%%endhead 
\zzSecCmt

\begin{itemize} % {
\iusr{Вадим Ляшенко}
Прекрасный фотограф, в 1987 году издал Київ. Погляд через століття

\iusr{Оксана Дубинина}
\textbf{Вадим Ляшенко}
Да, Вадим, Вы правы!
Я дуже вдячна Сергію, що знайшов чудову ілюстрацію зимової рибалки.

\iusr{Татьяна Петрюк}
Ну ....рассказ от настоящей рыбачки, а ни как от дочки рыбака  @igg{fbicon.face.grinning.squinting} !!!

\begin{itemize} % {
\iusr{Оксана Дубинина}
\textbf{Татьяна Петрюк} у меня терпения для рыбалки маловато, но в курсе @igg{fbicon.thinking.face}  @igg{fbicon.face.grinning.smiling.eyes}{repeat=2} 

\iusr{Татьяна Петрюк}
\textbf{Оксана Дубинина} по таланту рыбацких баек Вы превосходите бывалого рыбака  @igg{fbicon.face.grinning.squinting} ... так думаю

\iusr{Александр Савенко}
Татьяна ты молодец.
\end{itemize} % }

\iusr{Лидия Щепкина}

О! Собираться на рыбалку, у папы был целый ритуал... Сам выливал себе мармышки,
тщательно собирал свой рыбацкий чемодан, это я его так называла, но вот рыбу не
ел, не любил... Папы нет уже с 1982 года, сейчас видимо это увлечение передалось
моему сыну, и невестка моя тоже азартный рыбак.. @igg{fbicon.face.smiling.eyes.smiling} 

\begin{itemize} % {
\iusr{Оксана Дубинина}
\textbf{Лидия Щепкина} жещина-рыбачка - большая редкость!

А ритуал сборов на рыбалку это ж магическое действо ))
@igg{fbicon.face.smiling.sunglasses}
@igg{fbicon.face.grinning.smiling.eyes}{repeat=2} 

\iusr{Галина Гурьева}
Подтверждаю)))Мой папа тоже сборы превращал в ритуал! Забыть обед дома~это пожалуйста, а вот рыбацкое \enquote{причандалля}~никогда)))
\end{itemize} % }

\iusr{Раиса Карчевская}
Оксана!
Рассказ профессионального рыбака со знанием дела. Написан очень интересно,
получила огромное удовольствием Спасибо большое

\iusr{Оксана Дубинина}
\textbf{Раиса Карчевская}  ☺ ️  @igg{fbicon.face.grinning.smiling.eyes} благодарю))

\iusr{Ирина Иванченко}

Рыбалка! \enquote{Как много в этом звуке для сердца русского слилось! Как много в нём
отозвалось} Я всей душой люблю рыбалку, хотя бывала только в раннем детстве.

Благодарю, Оксана, дивный пост - море моих эмоций и новой информации - вот,
где, скажите, где б ещё узнала о \enquote{гендерных} подробностях определения мотыля -
прям захотелось срочно в бой...сорри, на лёд с мармышкой и \enquote{спецбутербродом от
зайчика} ! Спасибо, очень классный пост !

\begin{itemize} % {
\iusr{Оксана Дубинина}
\textbf{Ирина Иванченко} 

Ирочка, я сама удивлялась этой дивной фантазии!! С ужасом представляя, что
кто-то мог бы так проверять @igg{fbicon.face.grinning.smiling.eyes}{repeat=2}
Но, как говорится, из песни слов не выбросишь! Вот такие побасенки))

\iusr{Ирина Иванченко}
\textbf{Оксана Дубинина} ,ну, да, на \enquote{жирность} его опрелелить - то проще!

\iusr{Оксана Дубинина}
\textbf{Ирина Иванченко}  @igg{fbicon.face.zany}  та да)) жуть!)
\end{itemize} % }

\iusr{Ирина Иванченко}

\ifcmt
  ig https://scontent-frx5-2.xx.fbcdn.net/v/t39.30808-6/271663008_624822705491979_1331580436417846380_n.jpg?_nc_cat=109&ccb=1-5&_nc_sid=dbeb18&_nc_ohc=SBn7DIqPOFQAX8sAFze&_nc_ht=scontent-frx5-2.xx&oh=00_AT9DDCL7ut1_yojVeMymB8dxIj5vfGE8yxGCrG3oLnPYGw&oe=61E3CE31
  @width 0.2
\fi

\iusr{Татьяна Петрюк}

По комментариям понимаю ; попала в женский рыбацкий клуб
@igg{fbicon.beaming.face.smiling.eyes}  @igg{fbicon.laugh.rolling.floor} 

\iusr{Ольга Красникова}

Как всё точно описано, прямо точь- в-точь, мой папуля в
кожухе, шапке-ушанке, валенках, деревянный сундук на плече, и непременно, с уловом, и
гостинцем \enquote{от зайчика}, только в виде термоса с чаем, с мороза особенно
вкусным). Папа был заядлым рыбаком, и даже такая мелочь, как провалиться на
льду, не могла его испугать, ведь для него рыбалка была-стиль жизни)... Рыбы у нас
в семье было всегда море в виде таранки, ухи и жареной рыбы. Правда, чистить всю
эту радость зимнего клёва, в виде окуней и ершей, я не очень любила, если
честно, зато кушать одно удовольствие). К моему большому сожалению, мой папа умер
тоже на рыбалке, как настоящий рыбак он доплыл в ледяной воде до берега, но,
сердце не выдержало.

\begin{itemize} % {
\iusr{Оксана Дубинина}
\textbf{Ольга Красникова} 

верно, всё у нас похоже. Вы тоже понимаете, что мы все, особенно наши мамы,
волновались и ждали дома своих рыбаков, удержать же их было невозможно! И в
ледяной воде купались, ужас..

Машины даже под лед уходили.

Очень-очень жаль, что так случилось с Вашим папой. Надо же какая судьба...
\end{itemize} % }

\iusr{Анна Земко}

Дуже цікаво!

\iusr{Eugene Kovalchuk}
Тільки не чухонь, а чехонь, чехоня.

\begin{itemize} % {
\iusr{Геннадий Дмитренко}
\textbf{Eugene Kovalchuk} не чехоня, а чахлая

\iusr{Eugene Kovalchuk}
\textbf{Геннадий Дмитренко} то вже вульгаризм.  @igg{fbicon.smile} 

\iusr{Геннадий Дмитренко}
\textbf{Eugene Kovalchuk} то не вульгаризм, то рыбацкий сленг

\iusr{Оксана Дубинина}
\textbf{Eugene Kovalchuk} да, спасибо! Чехонь. Как-не задумалась)) исправлю

\iusr{Юрий Спичак}
\textbf{Eugene Kovalchuk} чухонь тоже есть, только живет она северней и не в воде.))
\end{itemize} % }

\iusr{Людмила Власенко}

Была морозная зима, Днепр давно был скован толстым льдом. Ехала с 5-ти летней
дочкой с \enquote{ёлки}. Выехали на мост метро и ребёнок на весь вагон (как мне
тогда показалось): мама, смотри сколько пингвинов.

Людей в вагоне было не очень много, но все устремили свой взгляд в ту сторону,
куда смотрел ребёнок.

Мои объяснения она не принимала, пока ближе к берегу не увидела рыбака, который
со своим \enquote{чемоданчиком} шёл по льду в сторону \enquote{стаи пингвинов}.
Дочка уже взрослая. Когда зима, холодно, любим вспоминать этот случай. Жаль,
что Днепр уже много лет не замерзает


\iusr{Дмитрий Мурин}
Шикарный пост.
Реально захотелось половить зимней рыбы)

\iusr{Alexandr Savitsky}

\ifcmt
  ig https://scontent-frx5-1.xx.fbcdn.net/v/t39.30808-6/271746681_4871704769562836_6981253494031392371_n.jpg?_nc_cat=105&ccb=1-5&_nc_sid=dbeb18&_nc_ohc=HEmYEiEr1IYAX_85bzT&_nc_ht=scontent-frx5-1.xx&oh=00_AT_cAQP6p6dG-JwmQFsujVgmssSJIA6DMaFRXZlf_ezlJw&oe=61E3CE5B
  @width 0.4
\fi

\iusr{Геннадий Дмитренко}

Отличный пост. вкусно и насыщенно, здесь все за рыбалку, все за душевный покой.
Сочувствую коллеге продолжающему ползти по суше, знакомо, по себе знаю, это
страх. Кстати металлические ящики были гораздо легче и практичнее чем
деревянные (деревянные намокали), изготавливались в основном из дюралюминия,
кустарно, позже появились в магазинах, но стоили дорого. И еще, сеточка с
грузком с которой течение вымывает корм, это не самосвал, самосвал это нечто
другое. Спасибо за статью.

\begin{itemize} % {
\iusr{Оксана Дубинина}
\textbf{Геннадий Дмитренко} 

спасибо!))

насчет «самосвала» спорить не буду, это с папиных слов. Еще его расспрошу.
Может это и я чего напутала... ☺ ️  @igg{fbicon.face.grinning.smiling.eyes} 

\iusr{Геннадий Дмитренко}
\textbf{Оксана Дубинина} 

\enquote{Самосвал} - металлическая пирамидка, заполненная кормом, чаще всего
изготовленная из жести, ( консервной банки) дно которой при погружении в водоем
открывается и корм сбрасывается,. сваливается в место лова. Ваш папа знает,
уверен.

\end{itemize} % }

\iusr{Воробей Виталий}
Чухонь... @igg{fbicon.laugh.rolling.floor}{repeat=4} 

\iusr{Оксана Дубинина}
\textbf{Воробей Виталий} чехонь)) был прокол ☺ ️ )) я уже исправила.

\iusr{Александр Асатуров}
я и сейчас их пингвинами называю.... сидят себе в 5 метрах от промоин.....

\iusr{Татьяна Кислова}
Даааа, прям за душу взяло... Всегда восторгалась зимними рыбаками. Вот где охота пуще неволи  @igg{fbicon.smile} 

\iusr{Александр Мазур}
Получил море удовольствия, как будто действительно побывал на зимней рыбалке 🐟

\iusr{Inatova Irina}

Какие чудесные воспоминания! Я родом не из Киева и у нас зимняя рыбалка была не
в чести, но до сих пор помню как папа брал меня на летнюю рыбалку. Как здорово
было!

\iusr{Люда Турчина}

Я их называю - \enquote{пингвины}, мне с 16 этажа этих \enquote{малость ненормальных} хорошо
видно. Никогда не пойму, как можно рисковать своей жизнью, когда на улице плюс
10, сверху льда лужи, а они пруться на край льдины, чтобы поймать какого-то
ершика. А потом нормальный человек, рискуя жизнью, должен их спасать. А часто
просто погибает из-за таких чокнутых. Рассказ великолепен, чувствуется любовь к
своему папе. Но я таких любителей рискнуть своим и чужим здоровьем называю - \enquote{
тихо помешаные !}

\begin{itemize} % {
\iusr{Виктор Задворнов}
\textbf{Люда Турчина} 

не-а, \enquote{безумству храбрых поем мы песню}. Перед увлеченными людьми всегда снимал
головной убор. Лыжники, конькобежцы чешут даже по льду - красота. Рыболовы -
другая \enquote{каста}. Тут ради \enquote{чекушки-поллитрухи} на морозе можно и домочадцев
оставить: ни чешуи, ни плавника, зато романтика. Помнится, один железнодорожный
профсоюз даже Нептуна приглашал на озерцо. Ну это, чтобы массовый праздник
организовать. А потом - уха. Клев не ахти какой, но \enquote{пару-тройку} кг \enquote{хвостов}
можно организовать из ближайшего, как они тогда назывались, универсама - весьма
об'единяющее профсоюз блюдо уха под неограниченное количество горячительных
напитков. Водителю автобуса не наливали.

\iusr{Оксана Дубинина}
\textbf{Люда Турчина} 

Вы правы, есть такое. До сих пор помню, как мама смотрела в окно и подходила к
двери, выглядывала, где там наш задержавшийся «рыбачила»)

\end{itemize} % }

\iusr{Dedekalo Iryna}

Спасибо! Очень напомнило детство. Отец был заядлым рыбаком, и каждый раз когда
он собирался зимой на рыбалку просилась с ним, но мне отказывали ( зимняя
рыбалка не для девочек). Мне оставалось ждать подарков от зайчика и чистить
ершей, окуней ..Зато летом какая радость, было ловить карасиков, окуней, щучек,
краснопеерок и других.

\iusr{Alena Torun}

Спасибо за шикарный рассказ! Крепкого здоровья и долголетия Вашему папе!

\iusr{Оксана Дубинина}
\textbf{Alena Torun} спасибо!@igg{fbicon.heart.red}

\iusr{Галина Гурьева}

Спасибо, Оксана! Мой папа тоже был рыбак. Все один к одному, только без
подарков от зайчика, не прижилось это у нас)) А подкормить мотыля теплым
супчиком, перебирать его перед рыбалкой и \enquote{выкупать} (промыть)? Мама
была в шоке))) И рыбацкий ящик до сих пор служит ~ инструменты там лежат!


\iusr{Оксана Дубинина}
\textbf{Галина Гурьева}  @igg{fbicon.face.smiling.hearts} 

\iusr{Андрій Пацьора}

-25 вполне комфортно переносится, если сухо и солнечно. Намного хуже в -1 при
ветре и сырости.

\iusr{Оксана Дубинина}
\textbf{Андрій Пацьора}
Закалка!))
Согласна, вот сегодня -14, но солнечно, прекрасная погода! Хорошего нам всем дня!)

\iusr{Сириэль Риверстар}

А я таскала и поедала у папы «рыбную кашу». Мама считала пшенку «мусорной». И
готовила только рис и гречку. Но вкуснее ничего нет!

\iusr{Зіна Бублей}

Рыбалка - романтика... А вот когда рыбаки возвращаются с зимней рыбалки,
пытаясь разместиться своей дружной компанией в общественном транспорте с этими
ящиками, плешнями, красными от мороза носами, то мама-дорогая...

\iusr{Оксана Дубинина}
\textbf{Зіна Бублей} один зимний рыбак - два посадочных места!)) @igg{fbicon.face.grinning.smiling.eyes}{repeat=2} 

\iusr{Татьяна Сирота}

Спасибо за прекрасный пост о зимней рыбалке.
Вспомнилось... Телефонный разговор между моим мужем и папой:
Папа - \enquote{Коля, как там ледовая обстановка?}
Муж - \enquote{Дед,завтра едем на }броню\enquote{. Собирайся.
Папы давно нет...
А муж сегодня на рыбалке.
Надеюсь на вечернюю уху!

\iusr{Андрей Соломатин}

Больные люди, что за удовольствие не пойму.

\begin{itemize} % {
\iusr{Оксана Дубинина}
\textbf{Андрей Соломатин}
у каждого свои радости.
Я думаю, они вполне здоровы @igg{fbicon.wink} 
Но тот, кто проводит зимние денёчки на воздухе поздоровее будет тех, кто проводит дни у телевизора)
\end{itemize} % }

\iusr{Svitlana Plieshch}

Помню, ехала от Осокорков на Березняки вдоль Днепра. В автобус с мрачными,
уставшим после работы людьми шумно зашёл такой Дед Мороз в шубе с ящиком,
румяный, говорливый, энергичный. Такой контраст: бледные и уставшие в автобусе
и он, 70-летний, бодрый, шумный, весёлый, прямо жизнь настоящая вошла. Всех
развеселил и оживил прибаутками.


\iusr{Виктория Зайцева}

У нас на Азовском море раньше тоже рыбаки были. Море так промерзало, что мы
ходили гуляли по льду, метров 300 -400 от берега. Рыбаки сидели дальше, там, на
расстоянии 2 км от берега ледокол всегда прокладывал канал между портом и
заводом \enquote{Азовсталь}. Вот ближе к нему рыбаки и старались сесть. Но мы боялись
так далеко ходить. А по льду ещё носились лодки с парусами и коньками на днище.
Интересно так. Не помню, как они назывались. Мне нравилось смотреть, какие они
выкрутасы на льду делали.

\begin{itemize} % {
\iusr{Boris Yablukov}
\textbf{Виктория Зайцева} буер, А канал для агломерата из Керчи, очевидно.

\iusr{Виктория Зайцева}
\textbf{Boris Yablukov} Не только. Но в основном для него.

\iusr{Дмитро Піталов}
\textbf{Виктория Зайцева} , нужно чтоб снова промерзло и ледоходом снесло рашистский мост в Крым!

\iusr{Виктория Зайцева}
\textbf{Дмитро Піталов}  @igg{fbicon.face.grinning.squinting} Согласна!
\end{itemize} % }

\iusr{Ассоль Грей}

У нас в детстве, во дворе парень жил, так он сам смастерил такое: к санкам
приделал самодельный парус и рассекал на Киевсом водохранилище по льду!

\iusr{Людмила Краснюк}

Очень яркий живой рассказ о зимней рыбалке !  @igg{fbicon.thumb.up.yellow} прочла с большим удовольствием!
! Спасибо Вам, Оксана, и знатному рыболову Вашему папе ! @igg{fbicon.heart.red}Будьте здоровы и
счастливы ! @igg{fbicon.hearts.two} 


\iusr{Зоя Луценко}
Спасибо, за рассказ! Как будто, про моего папочку написали, все так и было!

\iusr{Denis Dolgov}
шикарный очерк, @igg{fbicon.heart.red}
и анекдот смешной)

\iusr{Сергій Кучерявенко}
Очень грамотно с рыбацкой точки зрения! Прекрасно! @igg{fbicon.thumb.up.yellow} 

\iusr{Оксана Дубинина}
\textbf{Сергій Кучерявенко}  ☺ ️ мерси))

\iusr{Олег Комаров}
Я в марте на киевском море загорал

\iusr{Оксана Дубинина}
\textbf{Олег Комаров} прохладно, однако)))

\iusr{Игорь Сирадчук}

Зима, вечір, на льоду рибаки ловлять рибу. Кльову нема, а по сусідству в
палатці шум, гам, дівочий сміх і так кілька годин. Кльову нема. Під ранок в
палатці все стихло. Кльову нема. І тут з палатки вибігає мужик і кричить:
Сьогодні кльова не буде, кльово було вчора!

\iusr{Vladimir Tkachev}
Спасибо за отличный рассказ!!!)))

\iusr{Андрей Журавлев}

Оксана - текст отличный! Спасибо. От себя добавлю - я сам рыбак, но на зимней
рыбалке был только один раз. Не понравилось. Но вот вспомнил конец 80-х... То
ли 1988, то ли 1989... Тогда в Киев приехало несколько сотен английских и
американских студентов, приехали на практику по русскому языку. В общем,
познакомился я с одним англичанином. Оказалось - тоже рыбак. Пару раз возил я
его на Десну и на Днепр, рыбачили, пили водку, ну и т.д. Потом, когда ударили
морозы, ехали мы с ним в метро, через мост. Он увидел рыбаков на льду. Чтобы
было понятно - зимняя рыбалка в Англии отсутствует как явление. Он меня
спрашивает, мол а кто там на льду, на реке сидит? Я на голубом глазу отвечаю,
мол это пингвины. Их завезли из Антарктиды лет 10 назад, ну вот они и
размножились... Кадр просто офигел: - Что, правда???!!! - Конечно правда!

Он пару дней ходил под впечатлением...


\end{itemize} % }
