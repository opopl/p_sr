% vim: keymap=russian-jcukenwin
%%beginhead 
 
%%file 02_12_2020.news.ru.lenta_ru.mozzhuhin_andrei.1.pribaltika.v_strane_sovetov
%%parent 02_12_2020.news.ru.lenta_ru.mozzhuhin_andrei.1.pribaltika
 
%%url 
 
%%author 
%%author_id 
%%author_url 
 
%%tags 
%%title 
 
%%endhead 
\clearpage
\subsubsection{В Стране Советов}
\label{sec:02_12_2020.news.ru.lenta_ru.mozzhuhin_andrei.1.pribaltika.v_strane_sovetov}

В книге вы указываете, что «жестокая сталинская советизация» настроила жителей
Прибалтики против новой власти.

Я убеждена, что если бы не жестокость тоталитарного сталинского режима, то
советизация Прибалтики в предвоенный период прошла бы гораздо мягче и
спокойнее. И реакция ее жителей на приход нацистов была бы совершенно иной.

\ifcmt
tab_begin cols=2
	caption Книга «Прибалтика. 1939-1945 гг. Война и память», РГАСПИ

	pic https://icdn.lenta.ru/images/2020/10/20/15/20201020155440480/pic_a8083072578dd234fcaddae15485a3f3.jpg
	caption Демонстранты проходят мимо полпредства СССР с приветствием товарищу Сталину и Советскому правительству. Таллин, 5 июля 1940 г. 

	pic https://icdn.lenta.ru/images/2020/10/20/15/20201020155444769/pic_6a293a4c1dd3b31a3e5491fdeafd277b.jpg
	caption Шведская молодежь деревни Свиби. Остров Вормси (Ормсё), Эстония, 1940 г.
tab_end
\fi

\lenta{Но чем были вызваны массовые репрессии против местного населения в
мае-июне 1941 года, которые закончились всего лишь за пять дней до начала
Великой Отечественной войны?}

Мера устрашения. Причем поначалу раздражение людей было вызвано не
политическими или идеологическими причинами, а неуклюжей и абсолютно
безграмотной экономической политикой советской власти в Прибалтике.
Национализация, на которую так надеялся местный пролетариат, прошла безобразно
— рабочие от нее ничего не получили. Предприятия стали работать хуже, так как
их прежние хозяева были в лучшем случае отстранены, в худшем — сразу
репрессированы. После грабительского для местного населения перехода на
советскую валюту зарплаты стали существенно меньше, а цены выше.

Шоком для жителей республик Прибалтики стало и первое столкновение с
ассортиментом и качеством ввозимой из СССР продукции, особенно одеждой и
другими товарами народного потребления. Как говорится в одном документе того
времени, эти изделия «собирали у витрин злорадствующие по адресу советской
власти толпы». После этого Москве пришлось пойти на попятную и «дать указание
союзным наркоматам и главкам о том, чтобы производство изделий потребительского
характера по прибалтийским республикам не подчинять союзной стандартизации».

Одновременно с этим раздражение вызвал ажиотажный спрос на местную продукцию
среди приехавших — для советских людей такое изобилие, продукты и одежда такого
качества были чудом. Тем не менее уровень жизни населения советской Прибалтики
очень быстро и резко упал, что породило среди людей естественное недовольство.
А советская власть сталинского образца реагировала на это единственным
известным ей способом: репрессиями и высылками в Сибирь.

\lenta{Можно ли все эти репрессивные сталинские акции трактовать как геноцид литовцев,
латышей и эстонцев? В современной Прибалтике именно так их и называют.}

Геноцид — это целенаправленное уничтожение какого-либо одного народа. Но
сталинский террор в предвоенной Прибалтике, как и во всем остальном СССР, был
интернационален.

Он затронул представителей самых разных народов, в том числе и местных русских,
и в неменьшей степени, чем представителей титульных народов. Если судить по
документам, то, например, в Латвии в пропорциональном отношении больше всех
пострадали евреи (11 процентов из пяти процентов ее предвоенного населения), а
в Литве — евреи и поляки (соответственно 12,5 процента из 9,6 процента и 11,4
процента из 10,3 процента предвоенного населения).

\lenta{Как вы считаете, помогла ли советизация Прибалтики в 1939-1940 годах улучшить
военно-стратегическое положение СССР и подготовиться к будущей войне с
Германией?}

Советское руководство исходило именно из этого постулата. Вопрос в том,
насколько это удалось реализовать. В критические для нашей страны первые недели
гитлеровского нашествия в советской Прибалтике происходили шокирующие события.
Значительная часть населения этих республик буквально стреляла в спину
отступающим частям Красной армии и всячески помогала стремительному наступлению
вермахта.

\ifcmt
pic https://icdn.lenta.ru/images/2020/10/20/15/20201020155824161/pic_b7aaedaa31c4ae8d288e2802bd5b3c1f.jpg
caption Военный парад в Таллине. 7 ноября 1940 г.  Книга «Прибалтика. 1939-1945 гг. Война и память», РГАСПИ
fig_env wrapfigure
width 0.5
\fi

Уже в первые дни Великой Отечественной войны немецкая авиация уничтожила
значительную часть самолетов, базировавшихся на наших авиабазах в Прибалтике.
Хаотичная эвакуация из Таллина по Балтийскому морю в августе 1941 года, о
которой в книге тоже рассказывается, стала настоящей трагедией.


Здесь нужно еще учесть, и это хорошо видно по документам, что в предвоенный
период многие мероприятия по укреплению обороны территорий прибалтийских
республик, особенно их побережья, были фактически провалены. Причин тут
несколько: и плохая организация этой работы — в частности, отсутствие внятных
планов ее проведения и механизмов осуществления, и откровенный саботаж со
стороны местного населения, и тот факт, что прибалтийская промышленность, «со
скрипом» переходившая на советские рельсы, не смогла быстро и качественно
обеспечить оборонные нужды СССР.

