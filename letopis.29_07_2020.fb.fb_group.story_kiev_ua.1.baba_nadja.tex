% vim: keymap=russian-jcukenwin
%%beginhead 
 
%%file 29_07_2020.fb.fb_group.story_kiev_ua.1.baba_nadja
%%parent 29_07_2020
 
%%url https://www.facebook.com/groups/story.kiev.ua/posts/1415437408653056
 
%%author_id fb_group.story_kiev_ua,zhuravljov_andrej.kiev
%%date 
 
%%tags kiev
%%title КИЕВСКИЕ ЗАРИСОВКИ ИЛИ БАЙКИ ВРЕМЕН ЗАСТОЯ: БАБА НАДЯ
 
%%endhead 
 
\subsection{КИЕВСКИЕ ЗАРИСОВКИ ИЛИ БАЙКИ ВРЕМЕН ЗАСТОЯ: БАБА НАДЯ}
\label{sec:29_07_2020.fb.fb_group.story_kiev_ua.1.baba_nadja}
 
\Purl{https://www.facebook.com/groups/story.kiev.ua/posts/1415437408653056}
\ifcmt
 author_begin
   author_id fb_group.story_kiev_ua,zhuravljov_andrej.kiev
 author_end
\fi

Это небольшой полудокументальный рассказ. Возможно, кто то из читателей учился
в киевской школе №87 и помнит героиню рассказа)

КИЕВСКИЕ ЗАРИСОВКИ ИЛИ БАЙКИ ВРЕМЕН ЗАСТОЯ: БАБА НАДЯ.

Надежда Сергеевна была весьма заметной личностью в сфере среднего образования
города Киева в совдеповские времена. Она была директором школы №87. Школа эта
до сих пор находится в центре Киева, в самом начале бульварной части улицы
Антоновича, возле парка им. Шевченко. 

Эта школа была с очень-очень углубленным изучением английского языка, что само
по себе было не очень характерно для пуританской советской системы образования.
Английский там изучался с первого класса, по шесть дней в неделю, начиная с 5
класса, в определенные дни недели, было по два урока в день и количество этих
дней постепенно увеличивалась, а в старших классах добавлялись такие предметы,
как \enquote{Технический перевод} и \enquote{Военный перевод}. 

Каждый класс был разбит на три группы, в каждой группе -
свой учитель. Поэтому учителя английского были самой многочисленной
педагогической диаспорой среди остальных коллег. Выпускники этой школы по
советским меркам очень прилично владели языком Шекспира и могли почти свободно
поступить в киевский или даже в московский инъяз или университет им. Шевченко.
Естественно, при определенном везении и при наличии демографической дыры у
наследников партийной и советской \enquote{элиты}. Да-да, не удивляйтесь - в советское
время для поступления в престижный ВУЗ одних знаний было недостаточно. Вопреки
современному мифу о всеобщей и равной доступности к высшему образованию в СССР.
Поступление гарантировалось детям из \enquote{непростых} советских семей (чаще всего)
или при уплате соответствующей мзды в волосатую лапу (более редкий случай).

Надежда Сергеевна в сей \enquote{англоязычный} мир не особо вписывалась, особенно
учитывая, что она была ярым приверженцем советской власти и не терпела
инакомыслия. Ее раздражало, что многочисленные учителя английского на переменах
разговаривали друг с другом на языке \enquote{вероятного противника}. Так тогда в
советской прессе и военной доктрине называли США. Директриса очень
комплексовала по этому поводу: \enquote{Людмила Трохимівна (завуч по английскому),
скажіть вашим підлеглим розмовляти між собою на рідній мові! Я їх не розумію!}
Но в те времена, уже не сталинские, учителя английского языка на претензии бабы
Нади почти не обращали внимания и лишь только согласно кивали головами, но
делали все по своему.

Не удивительно, что в общем и целом Надежда Сергеевна была лютая тетка, такая,
сталинской закалки и, естественно, с партбилетом в кармане.

Кстати вспомнилось, что тогдашние партийцы на руководящих должностях были
своего рода близнецами. Я имею ввиду их высказывания, поступки и взгляды на
окружающую действительность. Все они, начиная от Генсека ЦК КПСС Леонида Ильича
Брежнева и заканчивая парторгом школы или небольшого заводика, периодически
делали уморительные оговорки. 

Чемпионом по этим самым оговоркам (причем
чемпионом с огромным отрывом) был конечно же товарищ Брежнев. Его словесные
ляпы были очень популярны в народе и служили поводом для многочисленных шуток и
анекдотов. И можно предположить, что партийные деятели более мелкого масштаба
делали всевозможные оговорки специально, стараясь соответствовать \enquote{патрону}.
Свой скромный вклад в это дело, по мере сил вносила и Надежда Сергеевна. Во
время поездки в Болгарию с группой школьников, Надежда Сергеевна выдала
очередной руководящий перл. Прибыв вместе с ребятами поездом на вокзал в Софию,
она сказала: \enquote{Ольга Іванівна! Хтось із нас (сопровождающих было всего двое -
директорша и Ольга Ивановна, учительница английского) залишиться з дітьми, а ви
ідіть та подивіться з якої колії відправляється наша електричка} (группе надо
было попасть в Бургас).

Когда директорша, выпятив грудь шла по школьным коридорам, наступала гробовая
тишина. Не слыхать было даже воробьиного чириканья и жужжания насекомых.
Учителя вытягивались по стойке смирно, а школьные раздолбаи-старшеклассники
быстро прятались по мужским туалетам. Правда мужские туалеты директрису не
смущали - она широко распахивала дверь в отхожее место и проникновенным тоном
провозглашала: \enquote{Курці, виходьте!}. И стояла возле открытых дверей в туалет,
пока не раздавался звонок на урок. 

Выбегающих Надежда Сергеевна запоминала
чекистской памятью, и далее на родителей раздолбаев отправлялись кляузы на
работу, в местные партийные органы. Доходило дело даже до того, что из-за кляуз
директора школы некоторых родителей лишали очереди на получение \enquote{бесплатной}
квартиры. Она похлеще любого вояки-замполита следила за прическами учеников
независимо от их возраста. 

И если, в ее понимании, прическа была длиннее на пару сантиметров от
пресловутого советского полубокса (прическа \enquote{полубокс} - можно сказать
почти лысый мальчик с челкой), она тут же говорила пойманному: \enquote{Негайно
підстригтися!}. Не выполняющие такие команды ученики получали в табелях
заниженные оценки по поведению, истории и обществоведению (которые преподавала
Надежда Сергеевна), и почему-то по физкультуре. Физрук Иван Захарович очень
злился по этому поводу, так как оценки \enquote{нерадивым} ученикам
исправлялись за его спиной. Иван Захарович  был мировой мужик, возрастом где-то
в районе \enquote{полтинника}, сухощавый, поджарый, ростом метр в кепке, он
прославился тем, что когда-то на школьных танцах, в субботний вечер, сам-один
наглухо раскидал компанию пятерых блатных лет 20-25ти, проникнувших в школу
\enquote{поснимать тёлок} и начавших драку с десятиклассниками. Он директоршу
называл не иначе как \enquote{эта сука}. 

Но физрук, конечно же, был исключением. Остальные учителя ее просто боялись. И
кличка у Надежды Сергеевны была иронично-уважительная – баба Надя.

Так бы и была она бабой Надей до самой пенсии, если бы с ней не познакомилась
Роза Абрамовна, бабушка одного из учеников. Директорша как-то раз, на
родительском собрании, попыталась наставить бабушку Розу на \enquote{путь
истинный}.

Вот тут и прозвучало имя Клары Цеткин (одна из деятельниц международного
рабочего движения в начале ХХ столетия). В виде комплимента от бабы Нади - мол
\enquote{ви, Роза Абрамівна чимось схожі на світоч революції - товарища Клару
Цеткін}.  Тут, конечно, директриса оплошала. Отпустив этот «оригинальный»
комплимент, Надежда Сергеевна заработала себе еще одно прозвище. На следующий
день бабушка Роза спросила внука, вернувшегося из школы: «Внучек, как там эта
ваша малахольная Клара Цыцкин? Еще живая или уже лежит из своим лысым
биндюжником у Мавзолее?». Новое \enquote{погоняло} разошлось по школе за
отрицательное время, и теперь в школе за глаза \enquote{красную директоршу}
называли баба Надя или Клара Цыцкин. В зависимости от ситуации.

\ii{29_07_2020.fb.fb_group.story_kiev_ua.1.baba_nadja.cmt}
