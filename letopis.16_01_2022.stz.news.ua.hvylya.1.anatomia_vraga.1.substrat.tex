% vim: keymap=russian-jcukenwin
%%beginhead 
 
%%file 16_01_2022.stz.news.ua.hvylya.1.anatomia_vraga.1.substrat
%%parent 16_01_2022.stz.news.ua.hvylya.1.anatomia_vraga
 
%%url 
 
%%author_id 
%%date 
 
%%tags 
%%title 
 
%%endhead 

\subsubsection{Субстрат}

Московское царство, Российская империя, СССР или современная Российская
Федерация - представляют собой один и тот же государственный фундамент, который
в зависимости от исторической конъюнктуры менял свои названия, оставаясь
неизменным и монолитным в форме государственного управления, принципах
конструкции общественной идеологемы и экспансионистском мировоззрении.

Россия - это империя континентального типа с 800 - летней традицией своей
государственности и когда Вы будете слышать залихватские россказни о \enquote{орде}, о
\enquote{тьме} и \enquote{свете}, необходимо четко понимать то, что на самом деле лежало в
основе становления московской (российской) государственности, что является
источником ее эндогенной силы и что является сильными сторонами данной
государственной модели. Понимая эти базовые, с моей точки зрения, довольно
простые вещи, можно начинать дискуссию о том, как именно выстраивать
реалистичную стратагему противостояния с Кремлем за возможность сохранения
нашей украинской державности.
