% vim: keymap=russian-jcukenwin
%%beginhead 
 
%%file 18_01_2022.fb.menendes_enrike.1.arestovich.cmt
%%parent 18_01_2022.fb.menendes_enrike.1.arestovich
 
%%url 
 
%%author_id 
%%date 
 
%%tags 
%%title 
 
%%endhead 
\zzSecCmt

\begin{itemize} % {
\iusr{Сергей Перевозчиков}
У вас тоже каждое заявление скандальное.

\begin{itemize} % {
\iusr{Елена Максюченко}
\textbf{Сергей Перевозчиков} вы хоть поняли , что написали? Энрике не гос служащий

\iusr{Энрике Менендес}
\textbf{Сергей Перевозчиков} У меня нет скандальных заявлений. Только правда, которая некоторым колет глаз

\iusr{Елена Максюченко}
\textbf{Энрике Менендес} ну пусть это его оценочное суждение, хрен с ним, пусть скандальное даже. Но сраавнивать мягкое и теплое? Господи прости, это как городовой стал музыкальным критиком

\iusr{Сергей Перевозчиков}
\textbf{Энрике Менендес} вот и у \textbf{Алексей Арестович} тоже правда, которая колет глаз.

\iusr{Сергей Перевозчиков}
\textbf{Елена Максюченко} а какая разница? Скандальность заявления не зависит от должности.

\iusr{Елена Максюченко}
\textbf{Сергей Перевозчиков} слова понятны все, в куче, смысла ноль. При чем здесь Энрике до Аоестовича

\iusr{Сергей Перевозчиков}
\textbf{Елена Максюченко} Вам просто не дано понять, не расстраивайтесь.
\end{itemize} % }

\iusr{Михаил Омский}
Да, ладно. А кто там теперь будет зажигать? Озвучивать волю нации?

\iusr{Остап Петренко}
\textbf{Михаил Омский} , Дмытро Ярош! Ты доволен, россиянин?

\iusr{Никита Синицин}

В моем понимании, если на в общем-то серьезной роли требуется быть или
притворяться шутом, это плохо говорит о состоянии власти и общества

\begin{itemize} % {
\iusr{Энрике Менендес}
\textbf{Никита Синицин} 100\%

\iusr{Андрій Баранець}
\textbf{Энрике Анатольевич Менендес} пан Енрике я додався в друзі

\iusr{Энрике Менендес}
\textbf{Андрій Баранець} Хорошо )
\end{itemize} % }

\iusr{Олександра Пеліховська}

В последнее время Вы как-то начали (по ощущениям) не то чтобы
радикализироваться, но повышать градус. Это и грустно (надо же испытывать некие
чувства, которые усиливаются), и интересно. Но это также констатирует, что
время теряется((

\begin{itemize} % {
\iusr{Энрике Менендес}
\textbf{Олександра Пеліховська} 

Время идёт. И моё тоже. Я вижу, что никто не собирается возвращать мой регион,
который медленно умирает. Что мне делать?

\iusr{Олександра Пеліховська}
\textbf{Энрике Менендес}

Тут вопрос - что вообще можно сделать(. Если раньше моей любимой вещью в JCS
Ллойда Уэббера была ария Иуды, то последние годы часто слушаю Гефсимань -
именно потому что ответа нет.

\url{https://youtu.be/0rEVwwB3Iw0}
\end{itemize} % }

\iusr{Sergiy Gerasimenko}

Что ж ты будешь делать, и опять к России никаких воззваний. Хотя имеем простой
до тошноты факт: это российские войска присутствуют на украинской территории, а
не наоборот.

\begin{itemize} % {
\iusr{Энрике Менендес}
\textbf{Sergiy Gerasimenko} Так что тогда мешает запустить прямые переговоры с РФ?

\iusr{Sergiy Gerasimenko}

Возведенный со стороны России в ранг экзистенционального масштаб конфликта, из
которого истекает единственная логика таких переговоров: капитуляция и отказ
Украины от субъектности. Игра с нулевой суммой не предполагает никаких
компромиссов. А россияне еще приговаривают: \enquote{Радуйтесь, что пока с нулевой,
дальше сумма для вас будет отрицательная}.

\iusr{Константин Токар}
\textbf{Sergiy Gerasimenko} 

Типичная позиция Украины - даже в анекдотах - игнорирование интересов и мнений других.

\iusr{Остап Петренко}
\textbf{Константин Токар} , да, и дальше будем игнорировать, и что?

\iusr{Константин Токар}
\textbf{Остап Петренко} и тогда придётся вас уничтожить. Бешеное животное убивают.

\iusr{Sergiy Gerasimenko}
\textbf{Константин Токар} 

Нет, как раз это позиция россиян - проецировать на других то, чем ты
руководствуешься в своих собственных действиях. Но есть нюанс: это российские
войска находятся на территории Украины, а не ВСУ на территории РФ.

\iusr{Sergiy Gerasimenko}
\textbf{Константин Токар} 

\enquote{и тогда придётся вас уничтожить. Бешеное животное убивают.} -
действительно, что же мешает запустить переговоры с РФ?!

\iusr{Константин Токар}
\textbf{Sergiy Gerasimenko} Я только могу повторить - надо учитывать интересы не только свои.

\iusr{Остап Петренко}
\textbf{Константин Токар} , 

тогда пойдут вереницы 200-х в вашу Рашу, а это на фоне повышения цен,
обнищания, недовольства б\_дломассы, населющей Россию, на фоне эпидемии ковида,
на фоне ненависти друг к другу наций и народностей населяющих вашу рашу ещё
неизвестно как на самой России скажется. А Украине помогут западные партнёры:
нет, воевать они не будут, зато оружием, перекрыть небо, разведданные. И ещё:
украинцы возненавидят Россию ещё больше!

\iusr{Остап Петренко}
\textbf{Константин Токар} , повторяй не повторяй, всё равно получишь х@й!

\iusr{Константин Токар}
\textbf{Остап Петренко} и всё это будет происходить над Украиной. Просто удивительно, до чего глупыми бывают люди.

\iusr{Остап Петренко}
\textbf{Константин Токар} , 

но всё это перекинется на Россию! И всему цивилизованному миру наконец-то
представится случай раздавить гадину! Кремлебот, ваши страшилки в Украине уже
не пугают, смени методичку!

\iusr{Константин Токар}
\textbf{Остап Петренко} 

гадина это вы, вы же то самый бот с методичками, которым называете меня.
Удивляюсь что Энрике не заблокировал вас до сих пор, ну а я заблокирую.
Противно.

\iusr{Natasha Semergey}
\textbf{Sergiy Gerasimenko} 

если российские войска находятся на территории Украины почему руководство
Украины заявляет что российские войска собираются вторгнуться? Вы уж
определитесь - она есть на территории или ее еще нету  @igg{fbicon.smile}  а то у вас шизофрения
какая то получается

\iusr{Остап Петренко}
\textbf{Natasha Semergey} , бабка, а что твои карты говорят, есть ли войска РФ на территории Украины? Давай-ка, раскинь карты, и мозги

\iusr{Natasha Semergey}
\textbf{Остап Петренко} диду, чи з глузду зъихав? Яки российськи вийска? @igg{fbicon.face.grinning.squinting} 

\iusr{Александр Малюк}
\textbf{Natasha Semergey} было бы даже странно, если бы карты правду показали.  @igg{fbicon.smile} 

\iusr{Остап Петренко}
\textbf{Александр Малюк} , ну она же этим на жизнь зарабатывает, людей дурит

\end{itemize} % }

\iusr{Дмитрий Есин}
Не торопитесь, он может запросто оказаться в вашем лагере

\iusr{Энрике Менендес}
\textbf{Дмитрий Есин} В моём, это в каком?

\iusr{Виктория Пасечник}

не уверена что нам нужен вообще какой-либо переговорный формат. другие
приличные страны тоже только делают вид, что переговаривают

\begin{itemize} % {
\iusr{Энрике Менендес}
\textbf{Виктория Пасечник} 

Да, но война не у них. Нам нужен прямой формат Киев-Москва, нам нужен прямой
диалог с Л/ДНР. Хватит уже прятаться за расплывчатыми формулировками.

\iusr{Виктория Пасечник}
\textbf{Энрике Анатольевич Менендес} я пытаюсь вспомнить, велись ли переговоры между германией и ссср в период с 1941 по 45й гг...

\iusr{Кирилл Лебедев}
\textbf{Виктория Пасечник} Если бы фашисты не уничтожали бы мирное население и не вели бы войну на уничтожение, то такие переговоры велись бы. Наполеон отправил не одно письмо Александру I.

\iusr{Vladislav Baturskiy}
\textbf{Виктория Пасечник}, попытайтесь вспомнить, была ли объявлена война между Германией и СССР или там тоже было что-то непонятно-гибридное.

\iusr{Остап Петренко}
\textbf{Vladislav Baturskiy} , так и Россия не объявляла, всё время говорит а это не мы, а вы докажите, а их там нет

\iusr{Vladislav Baturskiy}
\textbf{Остап Петренко}, так и Украина не объявляла. У Украины война с Россией для внутреннего потребителя и только. Официально никакой войны, что вы. Одна торговля.
\end{itemize} % }

\iusr{Victoria Strakhova}

Знаешь, немного задело. А потом думаю, ты занимаешься пропагандой - а есть
реальные люди из Луганской области, которые благодарны и за уголь и за попытки
решить экологические проблемы. Хорошо, что фейс бук блогерство сопоставимо с
картонным трафаретом, который ты упомянул.

\begin{itemize} % {
\iusr{Андрей Колосов}
\textbf{Victoria Strakhova} 

я - реальный человек из Луганска, если есть вопросы - с удовольствием для вас отвечу на них.

\iusr{Энрике Менендес}
\textbf{Victoria Strakhova} 

Вика, в общем я только ради тебя сделал оговорку про хороших людей. Потому что
тебя я знаю и уважаю. В остальном я остаюсь при своем мнении. И это не
пропаганда, а наша суровая реальность. Мой регион медленно умирает, а мне,
вместо решения проблемы предполагают попытки решения, вместо самого решения. И
моя претензия тут конечно не к составу ТКГ, а к ОП и к Владимиру
Александровичу. Я на 100\% сейчас уверен, что реальной цели урегулировать
конфликт у Киева нет. Ты и сама знаешь, какие установки по переговорам.

\iusr{Victoria Strakhova}
\textbf{Андрей Колосов} из Луганска или Луганской области?

\iusr{Victoria Strakhova}
\textbf{Энрике Анатольевич Менендес} 

поскольку я как раз знаю и установки, и позици и материалы - потому и обидно.
Не все ж может идти вовне

\iusr{Энрике Менендес}
\textbf{Victoria Strakhova} В любом случае, я надеюсь, что ты не принимаешь мои уколы на свой счёт.

\iusr{Андрей Колосов}
\textbf{Victoria Strakhova} из Луганска, но ситуация всей области мне понятна и известна...

\iusr{Victoria Strakhova}
\textbf{Андрей Колосов} 

в домах, обслуживаемых Лутэс (это контролируемая территория - есть тепло,
горяча вода и нет веерных отключений)

\iusr{Андрей Колосов}
\textbf{Victoria Strakhova} 

Это так. Но следует иметь ввиду, что контролируемая территория Луганской
области обслуживается единственной ТЭС - это именно Луганская ТЭС (ЛуТЭС) в г.
Счастье, огромная мощность которой используется частично, преимущественно - в
Луганской области. Это плюс, но есть и минус - в случае неполадок на ЛуТЭС или
отсутствии угля (что теоретически возможно), контролируемая часть Луганской
области столкнется с нехваткой электроэнергии.

\iusr{Victoria Strakhova}
\textbf{Андрей Колосов} 

так вот я и договариваюсь со своим визави в РФ о поставах угля на лутэс. И она
записывает попаснянский водоканал, который поставляет воду в ОРЛО

\iusr{Андрей Колосов}
\textbf{Victoria Strakhova} ..вы именно имеете к этому отношение? Тогда не все понятно...

\iusr{Андрей Колосов}
\textbf{Victoria Strakhova} 

...следует понимать, что вы в ТКГ? \enquote{И справедливо поблагодарить моего визави по
ТКГ - Сергея Назарова, который помог в данном вопросе.} Тогда это могло бы
стать только лишь началом разговора.... Добавлю (может это и неуместно здесь),
что нам доподлинно известно, кто влияет на поставки угля в Счастье, это,
конечно же, бенефициар ДТЭК! лично!


\iusr{Андрей Колосов}
\textbf{Victoria Strakhova} ..все это правильно! Но...десятки вопросов в ведении ТКГ блокируются нашей стороной...

\iusr{Андрей Колосов}
\textbf{Victoria Strakhova} 

Теперь то я, наконец, понял ваше непринятие критики Энрике. Но если говорить об
основной задаче ТКГ, то он конечно же прав на 99,999\% (за вычетом поставки угля
на ЛуТЭС и воды в ОРЛО), что ТКГ саботирует Минские соглашения, естественно, по
заданию руководства страны!

\end{itemize} % }

\iusr{Александр Минский}
А я помню его не то что адекватные заявления, а признания )

\href{https://vesti.ua/politika/270558-arestovich-priznal-chto-vral-ukraintsam-ob-ato-s-2014-hoda}{%
\enquote{На становление наций я смотрю, как солдат на вошь}: Арестович признал, что врал украинцам об АТО с 2014 года, %
vesti.ua, 21.12.2017%
}

\iusr{Игорь Степанов}

Читать тебя - большая работа! Продираться сквозь джунгли манипуляций, чтобы
выловить крохотного кузнечика смысла. И за этот труд ты просто обязан нас
поддерживать. Мы ведь твой рейтинг читаемости поддерживаем, причем сегодня -
бескорыстно! Дать свой аккаунт на Патреоне? А то совсем поиздержался.

\begin{itemize} % {
\iusr{Энрике Менендес}
\textbf{Игорь Степанов} Можете не читать. В чём проблема?

\iusr{Игорь Степанов}
Могу. А ты можешь не писать. В чем проблема? В ленте мне часто попадаются эти
тексты, волей-неволей читаю.
\end{itemize} % }

\iusr{Василь Мамелін}
Спасибо! Можете не «работать» и ересь не писать

\iusr{Григорий Перепелицын}

Если крыса бежит по швартову на берег - это верный признак, что корабль начал
принимать забортную воду еще стоя у стенки...

\iusr{Остап Петренко}
\textbf{Григорий Перепелицын} , чайка ходит по песку, моряку несёт тоску

\iusr{Александр Лапин}
\textbf{Энрике Менендес} , 

это тот случай, когда подходит поговорка «умер Максим, да и х..р с ним». Люся
никогда не была локомотивом Минского процесса (да и любого другого тоже), и от
её словесного поноса ничего не зависело. Она там оказалась по тому же
кумовскому принципу, что и СерЛещ в набсовете железной дороге - вовремя
подлизала зебанатам. Дали человеку подзаработать. Потом поняли, что от её
пердежа в эфире рейтинг нашего Наполеончика не то, что не растёт, а ещё и
снижается, и снизили ораторке размер пайки. А она обиделась и ушла. Самая
главная интрига в этой новости, кто подберёт это поюзанное отребье теперь  @igg{fbicon.wink} 

\href{https://youtu.be/UOBggWPw7GY}{%
Алексей Арестович — Люсенька, youtube, 04.04.2015%
}

\iusr{Вадим Щетинин}
\textbf{Энрике Менендес}, 

так как вы носитесь с возможностью войны, тоже ставит под сомнение Вашу
адекватность. Хотя Арестович и говорящая голова.

И кто такие порядочные люди в нашей делегации, это кому лизнули, если не
секрет?

\iusr{Кирилл Полевой}

Странно, что Энрике не заметил, что в подозрении Порошенко прокуроры
употребляют Луганская Народная Республика, Донецкая Народная Республика

\iusr{Тимур Чудутов}

Не то, чтобы я был поклонником Арестовича, но как-то в эфире на Укрлайфе он
высказал одно соображение, которое на меня сильно повлияло.

Если очень кратко, то весь происходящий кошмар можно интерпретировать как
проектную конкуренцию. Проект это образ желаемого будущего. Как будем в
долгосрочной перспективе жить мы, наши дети, наши внуки и правнуки. Та картина
будущего, которую мы хотим, ради которой готовы идти на жертвы.

Однако есть несколько конкурирующих проектов. Будущее может быть разным. Кто-то
видит привлекательной картину возрождения СССР, для кого-то привлекательно
общемировая либеральная глобализация с минимизированными государствами. Кто-то
хочет национальное консервативное государство. Кто-то хочет восстановления
Российской империи, а кто-то хочет чтобы РФ распалась на несколько более
комфортных образований. И так далее.

Проблема в том, что проекты эти конкурируют за воображение людей. И в нагрузку
к образу желаемого будущего человек получает картину остронежелаемого будущего.
Человек в своем воображении преувеличенно воспринимает не только достоинства
своего проекта и преуменьшает его недостатки, но и преувеличенно воспринимает
недостатки нежелаемого проекта и игнорирует его достоинства.

Собственно поэтому нормальные люди начинают в буквальном смысле убивать своих
близких. Я знаю несколько случаев, когда по разные стороны фронта оказались
друзья детства и близкие родственники. Ужас в том, что чем лучше человек, чем
он идеалистичнее, тем на более серьезные зверства готов.

Надо понимать, что Арестович ощущает себя идеологическим бойцом на фронте
конкуренции проектов. Ради торжества своего проекта он готов делать очень
сомнительные заявления. Собственно в том же интервью на укрлайфе он прямо про
это и говорил.

Конкуренция проектов - дело в принципе хорошее. Если бы еще научиться вести ее
так, чтобы люди не гибли - вообще было бы замечательно.

\end{itemize} % }
