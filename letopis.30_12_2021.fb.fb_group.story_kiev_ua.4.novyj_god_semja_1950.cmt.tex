% vim: keymap=russian-jcukenwin
%%beginhead 
 
%%file 30_12_2021.fb.fb_group.story_kiev_ua.4.novyj_god_semja_1950.cmt
%%parent 30_12_2021.fb.fb_group.story_kiev_ua.4.novyj_god_semja_1950
 
%%url 
 
%%author_id 
%%date 
 
%%tags 
%%title 
 
%%endhead 
\zzSecCmt

\begin{itemize} % {
\iusr{Ольга Кирьянцева}
Спасибо! И Вас поздравляю с Новым годом! Будьте здоровы и счастливы.  @igg{fbicon.heart.sparkling} 

\iusr{Раиса Канаева}

\ifcmt
  ig https://scontent-mxp1-1.xx.fbcdn.net/v/t39.1997-6/s168x128/34849193_231467377438008_5173102921927098368_n.png?_nc_cat=1&ccb=1-5&_nc_sid=ac3552&_nc_ohc=nxbTzTFT0fkAX98CYk1&tn=lCYVFeHcTIAFcAzi&_nc_ht=scontent-mxp1-1.xx&oh=00_AT8UrLEvmbxtstSbjROgq13UFdJilXeHgHeMAOT8AUwyDQ&oe=61D83454
  @width 0.1
\fi


\iusr{Вікторія Святненко}

Ах, как замечательно Вы все описали... С наступающим Новым Годом Вас! И пусть
всегда этот праздник будет связан с детством и прекрасными воспоминаниями @igg{fbicon.heart.red}

\iusr{Aleksandr Mitryaev}

\ifcmt
  ig https://scontent-mxp1-1.xx.fbcdn.net/v/t39.30808-6/270380712_640063474017403_5108039823257123614_n.jpg?_nc_cat=107&ccb=1-5&_nc_sid=dbeb18&_nc_ohc=k2h5OsT0du0AX-LR0N4&_nc_ht=scontent-mxp1-1.xx&oh=00_AT_i_7zaBdmQse9J9dgehQaOf9ldPVagqHV9AWX-g06PsA&oe=61D8C548
  @width 0.3
\fi

\begin{itemize} % {
\iusr{Анатолий Крупин}

Необходимо напомнить, что у автора воспоминания о 50-х, а не о 90-х годах.
Приведенный на снимке универсам искусственно действительно был пуст, чтобы
усилить ненависть и озлобленность граждан к своей стране. Странно все это
конечно, и это при наличии Госплана. Хотя все шло по задуманному сценарию
развала. Нужно было все же на снимке показать обычный магазин 50-х и что на
полках его было. Да и в те годы Новогодний стол никогда не был пуст, есть еще
свидетели тому, что было. Многим конечно хотелось после лаптей, нищеты и полной
безграмотности все и сразу и сейчас, особенно тем, кто попал из грязи прямо в
князи... Но сам праздник Нового Года тогда был светлым и радостным, а не так
как сейчас с безрадостным ожиданием неизвестности. Время в пути и все меняется,
но во все времена были любители подкинуть в бочку с медом ложку дегтя и это им
доставляло удовольствие. Поэтому, как всегда, как во все времена, как и в те
далекие 50-е, хотелось бы пожелать: С Наступающим!

\begin{itemize} % {
\iusr{Vladimir Sterling}
\textbf{Анатолий Крупин} вот как сегодня выглядят супермаркеты в Европе, - тоже искусственно созданный дефицит для переформатирования общества:

\ifcmt
  ig https://scontent-frt3-1.xx.fbcdn.net/v/t39.30808-6/270371615_1095631101251521_4660248314217479297_n.jpg?_nc_cat=104&ccb=1-5&_nc_sid=dbeb18&_nc_ohc=JZoRLfP1R5kAX-uioyJ&_nc_ht=scontent-frt3-1.xx&oh=00_AT_lQOuYoSs7mwPYQg3V8G632gfPm08-Uq9b2lWmL3G-sQ&oe=61D7AD6B
  @width 0.3
\fi

\end{itemize} % }

\iusr{Лида Еремейкина}
\textbf{Aleksandr Mitryaev} где это такие корзинки и тележки?

\iusr{Ольга Почивалова}
\textbf{Aleksandr Mitryaev} это в 50-е уже были корзины на колесах!? Ну даем! @igg{fbicon.face.grinning.squinting} 

\iusr{Vladimir Sterling}
\textbf{Aleksandr Mitryaev} в совке было, как сейчас в США и Англии)))

\ifcmt
  ig https://scontent-frt3-1.xx.fbcdn.net/v/t39.30808-6/270342160_1095631227918175_545032637678070100_n.jpg?_nc_cat=108&ccb=1-5&_nc_sid=dbeb18&_nc_ohc=IRqdNI73fRkAX-m7QYH&_nc_ht=scontent-frt3-1.xx&oh=00_AT9URcMtJqC8zsZ_rPQqG1tQNv5cc-4Xm5zzM2cr7JoTAA&oe=61D8FCC7
  @width 0.3
\fi

\begin{itemize} % {
\iusr{Ольга Почивалова}
\textbf{Vladimir Sterling} 

блин, это правда, звоню подруге пару дней назад (Калифорния), спрашиваю \enquote{куда
пойдете на каникулах, по магазинам?} Она отвечает, что уже все скупили,
магазины уже пустые. Изначально владельцы закупают одинаковое кол-во товара,
чтобы все распродать


\iusr{Vladimir Sterling}
\textbf{Ольга Почивалова} это фото за месяца два до Рождества)))
\end{itemize} % }

\end{itemize} % }

\end{itemize} % }
