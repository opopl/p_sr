% vim: keymap=russian-jcukenwin
%%beginhead 
 
%%file 05_02_2023.fb.fb_group.mariupol.pre_war.4.azovstal.cmt
%%parent 05_02_2023.fb.fb_group.mariupol.pre_war.4.azovstal
 
%%url 
 
%%author_id 
%%date 
 
%%tags 
%%title 
 
%%endhead 

\qqSecCmt

\iusr{Тетяна Катрич}

Я - донька металурга, який після війни все життя працював на
\enquote{Азовсталі}. Не раз горів, працював доменщиком.

Пишаюся тобою, могутній красень-завод!!!

\begin{itemize} % {
\iusr{Ольга Нуждина}
\textbf{Тетяна Катрич} и мой отец доменщиком был, только на Ильича.

\iusr{Nadezhda Nabok}

38 років пропрацювала на комбінаті \enquote{АЗОВСТАЛЬ}, він в моєму серці. \enquote{И заводская
проходная, что в люди вывела меня...}. Дуже боляче.
\end{itemize} % }

\iusr{Чурилович Антоха}

Я на Азовстале отработал 16.5 лет в Рельсобаллочном цеху !!!

\iusr{Олег Присяжнюк}

Мій батко будував стан 3600, а потім аж до самої пенсії працював на Азовсталі в
конверторному цеху машиністом крана.

\begin{itemize} % {
\iusr{Наталья Сергеева}
\textbf{Олег Присяжнюк} коллега!!!
\end{itemize} % }

\iusr{Ольга Орел}

А ми все жалілись, що він забруднює повітря.Оно так, тільки зараз хай би він
пихкав,але був...І було б наше мирне довоєнне життя

\iusr{Сергей Сергеев}

Какая была мощь, хоть мы и ворчали на него, наш дорогой \enquote{Азовсталь}...

\iusr{Татьяна Тищенко}

На предпредпоследнем фото - вид с \enquote{Азовстали} на наш Морской бульвар, от
которых мало что осталось.

Там всю жизнь работали 3 поколения и моей семьи, с первых лет завода и до его
последних дней работы.

Благодарю за фото!

\begin{itemize} % {
\iusr{Олег Присяжнюк}
\textbf{Татьяна Тищенко} Будь ласка!
\end{itemize} % }

\iusr{Ирина Кринфельд}

💔Разбитое сердце Мариуполя😢...

\iusr{Сергей Христодулов}

Я сєрж тоисть Серёга площядь Кирова франция кахорс. У меня ні чого ні ябомж. Пр
метал,, 109 40

\iusr{Александр Лазаренко}

Больно...

\iusr{Galina Gerich}

Как мы мариупольцы ругались на Азовсталь... Загрязнял воду, воздух, но давал
работу людям. А теперь только боль утраты и гордость за непокорность врагу.
Строили же, умели. Освободим Марик и будет новый, передовой Азовсталь!

\iusr{Наталья Авлиякулова}

Кому-то вонял....! Теперь, рады бы были и этой вони! Лижбы как раньше..... Мы
все дома💔😭

\iusr{Елена Лозовская}

Нет больше нашего завода. Когда он был, всем мешал. Теперь его нет и его не
хватает.

\iusr{Ирина Лаврова}

Всякий раз, когда ехали с дачи домой на электричке, то, завидев из окна завод
уже радовались, что вот еще 1 - 2 станции и вокзал. Уже Мариуполь. И, когда
едешь в пассажирском поезде, то при виде завода думаешь, что вот уже скоро
приедешь, скоро будешь дома. Лично мне завод не мешал и не так уж сильно и
дымил. Это был знак того, что скоро буду дома. А с приходом \enquote{асвабадителей} и
непонятно, когда же все вернуться домой.

\iusr{Vladimir Tereschenko}

Мой прадед, спасаясь от раскулачивания, приехал из Запорожской обл. в Мариуполь на
своей подводе со своими лошадьми на строительство Азовстали. Таких власть не
трогала. Где-то на правом берегу Кальмиуса он соорудил себе землянку. Семья
же, потомственные хлеборобы, жила в Володарском районе. Однажды нас, сельских
школьников, в 60-х годах возили на экскурсию в Азовсталь. Грандиозное зрелище.

\iusr{Serg Zvyagin}

R.I.P.

\iusr{Світлана Федоровська}

Моя мама працювала колись на Азовсталі. Тато відробив там \enquote{горячій} стаж в
рельсобалкє і ще на пенсії працював. Його орки вбили. Брат працював там. І я.
Там познайомилася з чоловіком. Я не можу, сльози ллються, біль пекучій.

\iusr{Наталья Бандурко}

\ifcmt
  igc https://i.paste.pics/584fcfb1d28b73b80c5b88d616984cee.png
	@width 0.2
\fi

\iusr{Георгий Панин}

Чума!

\iusr{Наталья Сергеева}

Полжизни отработала в конвертерном цехе. Завод не обещал чистейший воздух и
безвредное производство! И график работы был не самый лучший! Но работу свою
любила. Были свои трудности. Зато там, я встретила самых лучших людей в моей
жизни. Своих друзей, которые со мной по жизни!!! Горжусь, что \enquote{Азовсталь} часть
моей жизни!!!

\iusr{Наталья Комиренко}

А Ахметов акт таки скотина бешеная))сейчас живу возле завод Сименс. Бывает
вечером вижу с трубы белый дым, как дома.. но ни пыли, ни запаха.. Воздух как в
лесу свежий. И лакокрасочный завод у меня рядом огромный))

\iusr{Виталий Витвицкий}

Мой.родной.завод

\iusr{Оля Оля}

Всё верно, как многие здесь сказали, раньше нам вроде мешали эти выбросы, дым с
завода...

И как-бы это глупо не звучало!

Но сейчас этого так не хватает... 😢

Завод замер, и город и ним "спит"... 😧

И вот пришло время, когда так не хватает нам того, чем раньше жили, дышали,
радовались и ругались....😒
