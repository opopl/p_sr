% vim: keymap=russian-jcukenwin
%%beginhead 
 
%%file 15_12_2021.stz.news.ua.den.1.nuzhen_lenin
%%parent 15_12_2021
 
%%url https://day.kyiv.ua/uk/blog/polityka/nam-nuzhen-novyy-lenyn
 
%%author_id slivko_jevgenija
%%date 
 
%%tags 
%%title «Нам нужен новый Ленин!»
 
%%endhead 
\subsection{«Нам нужен новый Ленин!»}
\label{sec:15_12_2021.stz.news.ua.den.1.nuzhen_lenin}

\Purl{https://day.kyiv.ua/uk/blog/polityka/nam-nuzhen-novyy-lenyn}

\ifcmt
 author_begin
   author_id slivko_jevgenija
 author_end
\fi

«Она говорит на украинском, но похоже, она  нормальная…», - прошепотів до своєї
обраниці чоловік у нічному купе «Київ-Маріуполь», в якому мені довелося
провести близько 14-ти годин.

Після цієї фрази можна було б закутатися в колючу ковдру, увімкнути навушники й
сподіватися, що ніхто не плюне у мій ранковий чай чи пляшку з водою, яка стояла
на дорожньому столику.

Проте, бажання піти в контрнаступ виявилося сильнішим: ми з вимушеним сусідом
почали довгу розмову про мову, патріотизм, політику й віру. Словом, намагалися
зачепити усі теми, які в пристойному товаристві прийнято замовчувати.

Як з'ясувалося пізніше, мужчина їхав зі своєю дружиною та сином додому, в Марік
(Маріуполь – ред.). У сусідньому купе пили пиво їхні куми і друзі. Атмосфера
була досить домашньою: аромати «півка і сушоной рибкі».

З системи координат вибивався тільки мій мовленнєвий стиль, який від початку
налякав подружжя та їхніх близьких, але подорожні поводили себе досить
ввічливо: намагалися пригостити мене своїми яствами, а коли отримали відмову,
то не стали глибоко закочувати очі і демонстративно мовчати.

Навпаки, зацікавлено продовжували розмову під звуки свіжовідкоркованих пляшок.

Опівночі  майже всі «близькі й далекі» цієї родини об'єдналися навколо наших
місць: хтось просто намагався примоститися у тісному коридорі, аби долучитися
до розмови.

Дискусія розгорталася, як гостросюжетний сіквел. Удар на себе прийняв якраз
батько сімейства: моряк, активіст і корінний маріуполець  Дмитро. Спочатку він
захоплено розповідав про менталітет європейців, про їхній принцип братерства,
толерантності та взаємопідтримки, а потім з сумом почав говорити про українців:
«Понимаешь, наши совершенно не умеют объединяться. Сколько работаю с
иностранцами и постоянно наблюдаю одну и ту же картину – они, не зная имени,
готовы помочь, идентифицируя друг друга только по национальности. Наши же
наоборот, если услышат, что на борту есть украинец, даже расстроятся...».

З виразу обличчя Дмитра було помітно, що його це по-справжньому ображає. Як
громадянина та й просто як звичайного чоловіка.

Душевні муки цього незнайомого мужчини на рахунок самоідентифікації українців
насправді дуже наболіле питання як для абстрактних громадян, так і для
конкретного Дмитра:

\headTwo{Чому ж українці здатні об’єднуватися тільки «перед розстрілом»?}

Приблизно таке питання було поставленим у нічному купе.

А справа в наступному. Українці (не всі) просто не усвідомлюють себе як націю і
один народ. Звинувачувати себе, Дмитра чи своїх родичів не варто, бо для таких
наслідків були створені усі історичні умови.

Я б розділила ці умови на три основні фактори, які й сформували сучасне
самоусвідомлення  українців.

Перший фактор – це той стереотипний образ українця,  який був сформованим у
класичній літературі геніальними Шевченком, Лесею Українкою, Іваном Франком.
Доповнювали таку самоідентичність козацькі ритуали, фольклор, звичаї та
традиції, які тісно перепліталися з релігійними уявленнями «давнього
стереотипного» українця.

Про це наш умовний Дмитро з Маріуполя не чув, бо його уявлення про світ
формувалися на зовсім інших «китах»: на імперській російській культурній
традиції, та на репресивному комунізмі, який формувався на основі ідентичності
«радянської людини».

Якраз «радянська людина» досить чітко означена і досить агресивно
репрезентована в усіх шарах теперішнього соціуму.

Така категоричність адептів комунізму пов’язана з тим, що його позитивні
атрибути – уявне соціальне благополуччя і особистісний успіх – для значної
кількості громадян України закорінені в їхньому власному досвіді.  Принаймні,
такими українцям уявляються роки до Перебудови з позицій нинішнього
стресово-нестабільного сьогодення, позбавленого віри у справедливість для
більшості населення.

Прикметно, що «синдром втоми» від змін супроводжує життя найбільш незахищених
верств населення, тоді як колишня номенклатура й у нових умовах зберегла своє
домінуюче становище.

У ході продовження розмови Дмитро став яскравим прикладом якраз «втомленого»
громадянина, який видав наступне: «Нам нужен новый Ленин или, как минимум,
Лукашенко!».

Чи могло з його уст пролунати інакше, коли мова йде не стільки про політичні
передумови чи елементарну неосвітченість, скільки про психологічну реакцію
людей, що опинилися в умовах боротьби за виживання і не мають іншого досвіду
інтелектуального (ідейного) самоорганізації, окрім комуністичної риторики.
«повернення» до радянської ідентичності, що для більшості таких «дмитрів» є
своєрідною формою психологічного «притулку». Тобто захищеності.

Говорячи про конкретний приклад у потязі, б’юся об заклад, що більшість
сучасних політиків відмахнеться від перерахованих факторів та наслідків, які
сформували у суспільстві те, що ми зараз називаємо «совком», але чи маємо ми
право засуджувати тих, хто був сформованим в абсолютно інших соціальних та
політичним умовах?

Все що повинні надати сучасні політики східним регіонам – це безумовну
підтримку та діалог аби вони нарешті перестали почувати себе «покинутими» як
індивідуально, так і територіально. Звичайно, це забіг на дуже довгу дистанцію
з «незручними» людьми, з якими треба працювати десятками років, а сучасні
політичні еліти не уявляють себе на посадах більше двох років, тому й не
ініціюють ніякого бажання заглиблюватися в глибокі кризи ідентичності таких
родин як Дмитрова.

А це вже якраз і відповідь на питання – чому на території України дотепер війна
(Росія само собою).

Євгенія СЛІВКО, політичний радник
