% vim: keymap=russian-jcukenwin
%%beginhead 
 
%%file 02_12_2022.fb.ohmatdet.bolnica.1.35_operacij_divchynka
%%parent 02_12_2022
 
%%url https://www.facebook.com/ndslohmatdyt/posts/pfbid0DUhBtkg8MnLpufmEKBB3RnrWQegZmNSdjosiDvEfFrkwxL2PKyUcEr6UoYxnGk9al
 
%%author_id ohmatdet.bolnica
%%date 
 
%%tags 
%%title 35 операцій за 8 місяців лікування
 
%%endhead 
 
\subsection{35 операцій за 8 місяців лікування}
\label{sec:02_12_2022.fb.ohmatdet.bolnica.1.35_operacij_divchynka}
 
\Purl{https://www.facebook.com/ndslohmatdyt/posts/pfbid0DUhBtkg8MnLpufmEKBB3RnrWQegZmNSdjosiDvEfFrkwxL2PKyUcEr6UoYxnGk9al}
\ifcmt
 author_begin
   author_id ohmatdet.bolnica
 author_end
\fi

⚡️35 операцій за 8 місяців лікування: з Охматдиту виписали 21-річну пацієнтку,
яка отримала тяжкі бойові травми в березні у Чернігові⚡️

На виписці Марина не стримує сліз та обіймає кожного лікаря, який був поруч з
нею останні 223 дні — саме стільки дівчина лікувалась в Охматдиті. Вона
отримала поранення 19 березня у Чернігові, коли вийшла з дому та потрапила під
російський обстріл.💔

Дівчина отримала осколкові поранення у живіт, руку та шию. У результаті тяжких
уражень внутрішніх органів, Марина потребувала складного лікування. В Чернігові
їй була надана первинна лікарська допомога — перший етап лікування. Далі
дівчину перенаправили до Охматдиту, де виконали складні реконструктивні
оперативні втручання, що допомогли швидко відновитись і поставити дівчину на
ноги.🙏🏻

«Пацієнтці провели ургентну хірургію: наклали нову кишкову стому, провели
часткову реконструкцію вже наявної стоми, яку поставили у Чернігові, прибрали
нежиттєздатні тканини. Марина потребувала неодноразових хірургічних втручань.
Дівчина поступила в тяжкому стані. Але вже за день після першої операції, вона
почала самостійно пити і їсти»,— згадує хірург відділення ургентної хірургії
Роман Жежера .💪🏻

Далі до наших хірургів доєдналися ортопеди, які провели 27 хірургічних втручань
на руці Марини.🙌🏻

«Лікування вогнепального поранення руки поділялось на декілька етапів: спершу
потрібно було загоїти рани та закрити дефекти шкіри, потім — шляхом
дистракційного остеогегезу відновити 4-х сантиметровий дефект ліктьової кістки.
Далі — відновлення сухожилків-згиначів пальців правої кисті та серединного
нерва. А останнім етапом є довготривалий період реабілітації для повноцінного
відновлення функції кисті»,— розповідає лікар ортопед-травматолог Іван Черняк
\href{https://www.facebook.com/ivan.s.cherniak}{Ivan Cherniak}.⚡️

Для повного відновлення функцій руки, дівчини знадобиться принаймні рік.

На виписці Марина підготувала подарунки для своїх лікарів. Лівою рукою вона
створила 2 картини, які тепер висять у відділеннях ургентної хірургії та
ортопедії і травматології.💛

\ii{02_12_2022.fb.ohmatdet.bolnica.1.35_operacij_divchynka.orig}
\ii{02_12_2022.fb.ohmatdet.bolnica.1.35_operacij_divchynka.cmtx}
