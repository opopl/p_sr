% vim: keymap=russian-jcukenwin
%%beginhead 
 
%%file 20_07_2020.fb.lnr.1
%%parent 20_07_2020
 
%%endhead 
\subsection{Киев признал использование наемников в Донбассе}
\url{https://www.facebook.com/groups/LNRGUMO/permalink/2862789930499229/}
  
\vspace{0.5cm}
{\small\LaTeX section: \verb|20_07_2020.fb.lnr.1| project: \verb|letopis| rootid: \verb|p_saintrussia|}
\vspace{0.5cm}

Только за последние сутки украинские силовики четырежды обстреляли окрестности
Донецка. В зоне поражения оказались сразу несколько населенных пунктов. И
подобные обстрелы территории ДНР не прекращались всю неделю. Более того, Киев
направил туда диверсантов, среди которых были иностранные наемники. Часть из
них попала на минное поле, так что назад вернулись не все. 

Украинские власти после этого обвинили донбасских военных в убийстве врача, не
зная, что на теле погибшего была видеокамера. Что снятые ею кадры позволили
прояснить в его биографии и в биографии других воюющих на Украине иностранцев?

Вот район под населенным пунктом Зайцево – это ДНР – в который с 12-го на 13
июля попыталась проникнуть украинская диверсионная группа. А дальше произошли
события, на основе которых разгорелась самое настоящее информационное сражение.

Первая группа украинских военных шла в ДНР с целью совершить диверсию. Эдуард
Басурин, официальный представитель военных ДНР, предполагает, что это был
вариант "Сафари", – так украинские военные называют охоту на людей.

Командир украинской диверсионной группы подорвался на мине, его бросили и ушли.
Украинская сторона попросила перемирия, чтобы забрать тело.

В ДНР объявили, что погибший – американец Шон Фуллер, известный своими
радикальными националистическими взглядами, воевавший на стороне Украины. Но
почти сразу на украинском телевидении появился человек, назвавший себя
Фуллером, с интервью и руганью в адрес Донбасса.

"Мы добились того, чтобы украинская сторона признала, что за нее воюют
наемники. Вот это самое главное было", – сказал Басурин.

В сторону позиций ДНР для эвакуации первого своего погибшего пошла вторая
украинская группа, но совсем в другое место, не оговоренное перемирием. Еще
один боец подорвался на мине.

Украинский комбриг Палац тут же обвинил народную милицию ДНР в расстреле мирных
медиков, шедших под белым флагом и в белых жилетах с крестом. Но военные ДНР
захватили камеру второго погибшего украинского рейнджера.

Боец народной милиции ДНР, позывной Абдулла, сам не раз ходил в тыл противника
и также подорвался на мине. Мы попросили его прокомментировать трофейные кадры:
"Вот он побежал. На гору бежал быстрее лани. Судя по этому видео, этот человек
сейчас сам себя и убьет". От страха.

Этим вторым погибшим оказался гражданин Эстонии Николай Ильин. Он умудрился
взять на разведвыход все свои документы и телефон, где масса его видеозаписей и
фотографий.

"У него переписка есть, как он говорит, с американскими друзьями, которых он
зовет из Америки сюда приезжать, таких же националистов: у вас уникальный шанс
побывать в боевой обстановке, у вас уникальная возможность загнать этих
"свиней". Это мы "свиньи", – сказал Эдуард Басурин.

Вот гражданин Эстонии Ильин с американцем Родригесом Джастиным, они из одного
взвода, вместе воевали против Донбасса. Вот кадры боевой подготовки из
телефона, вот кадры украинского канала "1+1", где Ильин и другие иностранцы не
скрывают, что они участвуют в боевых действиях. Кстати, эстонца Ильина называли
Медик, но это был лишь позывной. На последнее задание он шел без носилок,
аптечку нес лишь для себя. Также у него с собой были компас, радиостанция,
запасные батареи к прибору ночного видения и документы.

В конце концов украинские военные отправляют к позициям ДНР третью
разведгруппу. И третий боец подрывается на мине.

- Получается, что один из них был ранен и они бросили раненого?  - Да, когда
подошли к телу, он уже мертвый был. Сказали, наверное: мы за тобой вернемся,
полежи немножко, сейчас мы сил соберем. И бросили его, – сказал офицер армии
ДНР.

В итоге погибших украинцев пошли забирать военные Донбасса.

История оказалась не такой уж и запутанной. Не было бы нарушений Минских
соглашений, не было бы погибших украинских солдат.

Под Зайцево пришла украинская диверсионная группа, командир погиб. Запросили
перемирие для того, чтобы "вытянуть" тело, обманули, пошли не туда, снова погиб
человек, еще погиб человек. 

Командование украинское начало врать, какие-то версии выдвигать. Опровергли, и
в итоге-то Донбасс не просто обстреливают из пушек и минометов – против
Донбасса Киев ведет интенсивную информационную войну.
  
