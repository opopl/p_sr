% vim: keymap=russian-jcukenwin
%%beginhead 
 
%%file 26_07_2022.stz.news.ua.donbas24.1.mrpl_motanka_istoria_iljustratorky.txt
%%parent 26_07_2022.stz.news.ua.donbas24.1.mrpl_motanka_istoria_iljustratorky
 
%%url 
 
%%author_id 
%%date 
 
%%tags 
%%title 
 
%%endhead 

Наталія Сорокіна
Маріуполь,Україна,Мариуполь,Украина,Mariupol,Ukraine,Анастасія Пономарьова,Маріупольська мотанка,Творчість,date.26_07_2022
26_07_2022.natalia_sorokina.donbas24.mrpl_motanka_istoria_iljustratorky

Як створювалася «Маріупольська мотанка»: історія талановитої ілюстраторки
(ФОТО)

Наразі серія нараховує сім робіт, доповнених знімками маріупольського фотографа

Анастасія Пономарьова створила серію надзвичайних ілюстрацій, що поєднують
образи традиційних українських ляльок-мотанок та фото знищеного й окупованого
росіянами Маріуполя. Яким було життя до війни, куди вдалося евакуюватися та
звідки прийшла ідея «Маріупольської мотанки», жінка розповіла Донбас24.

«Найстрашніше було розбудити сина зі словами, що почалася війна»

До повномасштабного військового вторгнення рф життя маріупольчанки Анастасії
Пономарьової було дуже насиченим. Вона працювала викладачкою географії та
дитячим ілюстратором, мала освітній центр, координувала безкоштовні курси
програмування для дітей. 24 лютого жінку розбудили гучні вибухи. Світла в
помешканні вже не було, тож разом з близькими вона поїхала до сестри. Однак і
там вже було небезпечно: росіяни обстрілювали район з «Градів».

Наступною зупинкою для Пономарьових став центр Маріуполя, де вони оселилися в
друзів. Проте в місті надовго не затрималися: спочатку евакуювалися до Дніпра,
а за певний час — у Дрогобич.

«Я не брала ані фарби, ані якісь мирні речі, крім з десятка книжок сина.
Здавалося, що не буде можливості жити „нормальним“ життям, малювати, працювати.
Найстрашніше для мене було розбудити сина зі словами, що почалася війна», —
пригадує Анастасія.  

Життя в евакуації: волонтерство, курси для дітей та творчість

Діставшись безпечного місця, маріупольчанка не змогла сидіти на місці та «бити
байдики». У Дрогобичі жінка зібрала команду волонтерів, які проводили
терапевтичні майстер-класи для дітей. Разом з психологами створила й адаптувала
методики, що допомагають подолати наслідки війни у маленьких українців: за ними
можуть працювати волонтери у Запоріжжі та Дніпрі.

Окрім того, Анастасія відкрила безкоштовні курси з програмування для дітей та
курси комп'ютерної грамотності для третього віку, а разом з Галиною Вдовиченко
запустила проєкт лікувальних казок на листівках. Попри активну діяльність,
маріупольчанка залишила в своєму житті місце й для творчості:

«Зараз є проєкт «Місто, в яке варто повертатись»: я малюю для маріупольців їхнє
місто знову квітучим, наприклад, під їх запит малюю зруйновані будівлі знову
квітучими... Ну і, звісно, другий проєкт — «Маріупольська мотанка», — розповідає
Анастасія.

«Перша мотанка мені наснилася»

Історія створення серії ілюстрацій «Маріупольська мотанка» — дещо містична та
символічна. За словами Анастасії Пономарьової, ніколи в житті їй не снилися
малюнки, проте образ першої ілюстрації прийшов до неї саме в «обіймах Морфея»,
а подальші «картинки» ставали яскравішими.

«Мені наснилася сама картина — мотанка на фоні зруйнованого Маріуполя, і такі
дві реальності «зшиті», — ділиться спогадами творча жінка.

Створюючи першу ілюстрацію, Анастасія відчула потребу доповнити її фотографією
Маріуполя. Невдовзі з міста Марії евакуювався фотограф Євген Сосновський з
архівом робіт. Він дозволив жінці їх використовувати. В якості першого
«доповнення» для ілюстрації став знімок з квартири самого Євгена: цікаво, що в
помешканні фотографа також раніше стояла мотанка.

Наразі серія «Маріупольська мотанка» налічує сім ілюстрацій. Робота над нею
триває, тож ілюстрацій точно буде більше: адже болючих моментів, пов'язаних із
війною, ще багато.

"Маріуполь — то місто Марії, і мотанка — то для мене такий образ берегині,
оберегу Маріуполя, це наче ота ниточка червона, яка нас всіх тримає і зшиває
разом", — зазначає Анастасія, підкреслюючи, що не вона обрала образ мотанки, а
він її.

Нагадаємо, переселенка з Маріуполя проводить майстер-класи для земляків зі
створення ляльок-мотанок.

Найсвіжіші новини та найактуальнішу інформацію про Донецьку й Луганську області
також читайте в нашому телеграм-каналі Донбас24.

Фото: Анастасія Пономарьова
