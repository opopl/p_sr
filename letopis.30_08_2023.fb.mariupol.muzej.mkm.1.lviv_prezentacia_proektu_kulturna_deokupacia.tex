%%beginhead 
 
%%file 30_08_2023.fb.mariupol.muzej.mkm.1.lviv_prezentacia_proektu_kulturna_deokupacia
%%parent 30_08_2023
 
%%url https://www.facebook.com/100093184796939/posts/pfbid0EKYbgxMQPPmJTSUrT8MKn4b7N6eqrMSbgVkD8C6uuMZSdMC9CphZtbCT82p99Tj6l
 
%%author_id mariupol.muzej.mkm
%%date 30_08_2023
 
%%tags 
%%title Львів - презентація проєкту "Маріуполь. Культурна деокупація"
 
%%endhead 

\subsection{Львів - презентація проєкту \enquote{Маріуполь. Культурна деокупація}}
\label{sec:30_08_2023.fb.mariupol.muzej.mkm.1.lviv_prezentacia_proektu_kulturna_deokupacia}

\Purl{https://www.facebook.com/100093184796939/posts/pfbid0EKYbgxMQPPmJTSUrT8MKn4b7N6eqrMSbgVkD8C6uuMZSdMC9CphZtbCT82p99Tj6l}
\ifcmt
 author_begin
   author_id mariupol.muzej.mkm
 author_end
\fi

✨️😊25 серпня стало визначним днем для культурної спільноти Маріуполя, як
своєрідний проміжний підсумок в роботі галузі з формування стратегії
відродження Культури українського Маріуполя після його деокупації. Незважаючи
на розпорошеність по різних містах, працівники культури продовжують
підтримувати своє місто важливими та актуальними проєктами.

✅️У Львові на презентацію проєкту \enquote{Маріуполь. Культурна деокупація} зібрались
представники маріупольських театрів та філармонії, бібліотек та музеїв, палаців
культури та художніх шкіл. Надзвичайно приємно було відчути підтримку
львівських, київських, луганських колег, та навіть діячів культури з закордону.

🔅 Художники, архітектори, хореографи, літератори та книговидавці долучились до
відкритого діалогу, і це свідчить наскільки багатогранною та живою залишається
наша сфера.

Звісно, ще багато суперечливих проблемних питань потребують вирішення. Але
діячі культури їх відчувають, реагують та спільними зусиллями шукають шляхи
подолання. І в цьому сенсі приємно відчувати себе частиною сильної команди –
команди Культури Маріуполя.

❗️\enquote{Кожен крок має значення} – цей вислів став своєрідним слоганом зустрічі. За
півроку існування проєкту таких кроків було зроблено безліч. У вигляді
діалогових платформ, виставок, арт-проєктів тощо. І кожен з подібних кроків
наближає до головної мети – об'єднання маріупольців заради продовження життя
міста, збереження його культури та спадщини. Сьогодні цей процес відбувається
відкрито, у співпраці. Кожен охочичий може поділитись своїми ідеями та
долучитись до реалізації культурних проєктів.

➡️ З ініціативами, які можуть бути реалізовані в рамках музейної роботи, звертайтесь на пошту: \url{mamu1920@ukr.net}

Детальніше про проєкт \enquote{Маріуполь. Культурна деокупація} читайте тут:

\url{https://www.facebook.com/groups/mkultura/permalink/3597130320516140}
