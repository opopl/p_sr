% vim: keymap=russian-jcukenwin
%%beginhead 
 
%%file 15_01_2022.fb.antarctic_center_ukraina.1.velyke_prybyrannja
%%parent 15_01_2022
 
%%url https://www.facebook.com/AntarcticCenter/posts/291330363030394
 
%%author_id antarctic_center_ukraina
%%date 
 
%%tags antarktida,ukraina
%%title ВЕЛИКЕ ПРИБИРАННЯ: УКРАЇНСЬКІ ПОЛЯРНИКИ ВІДКОПУЮТЬ СТАНЦІЮ ПІСЛЯ РЕКОРДНОГО ГРУДНЕВОГО СНІГОПАДУ
 
%%endhead 
 
\subsection{ВЕЛИКЕ ПРИБИРАННЯ: УКРАЇНСЬКІ ПОЛЯРНИКИ ВІДКОПУЮТЬ СТАНЦІЮ ПІСЛЯ РЕКОРДНОГО ГРУДНЕВОГО СНІГОПАДУ}
\label{sec:15_01_2022.fb.antarctic_center_ukraina.1.velyke_prybyrannja}
 
\Purl{https://www.facebook.com/AntarcticCenter/posts/291330363030394}
\ifcmt
 author_begin
   author_id antarctic_center_ukraina
 author_end
\fi

ВЕЛИКЕ ПРИБИРАННЯ: УКРАЇНСЬКІ ПОЛЯРНИКИ ВІДКОПУЮТЬ СТАНЦІЮ ПІСЛЯ РЕКОРДНОГО
ГРУДНЕВОГО СНІГОПАДУ

Пригадуєте, ми в грудні розповідали, що попри початок антарктичного літа біля
\enquote{Вернадського} постійно хурделить і рівень снігового покриву став рекордним за
останні 20 років? Зокрема, 24 та 25 грудня його висота була найбільшою – аж 3
метри!

\ii{15_01_2022.fb.antarctic_center_ukraina.1.velyke_prybyrannja.pic.1}

З нового року снігопади майже припинилися, але ж тепер нашим полярникам
необхідно відкопати всю інфраструктуру станції @igg{fbicon.face.flushed}  Тож у
них почалися активні фізичні тренування на свіжому повітрі:)

Так, зимівники та учасники сезонної експедиції уже кілька тижнів визволяють
територію \enquote{Вернадського} від снігового полону. Майже відкопали житловий
корпус, капличку та прохід до другого причалу Джету. Також для проведення
запланованих робіт з модернізації станції розчистили металоконструкцію біля
дизельної для переміщення вантажів, територію за сміттєвим майданчиком для його
розширення, місце для прокладання нової паливної лінії та фундамент крану для
його гідроізоляції.

\ii{15_01_2022.fb.antarctic_center_ukraina.1.velyke_prybyrannja.pic.2}

Попереду на полярників чекає ще одне грандіозне \enquote{копання}: треба звільнити
площу навколо складу для його розширення.

До речі, до роботи долучилися й французькі гості, які нещодавно приїхали на
\enquote{Вернадський} для проведення досліджень.

\ii{15_01_2022.fb.antarctic_center_ukraina.1.velyke_prybyrannja.pic.3}

Побажаємо всій команді, щоб подальше снігоприбирання було легким та швидким, а
вже очищені ділянки не притрушував новий сніжок @igg{fbicon.face.smiling.halo} 

Дякуємо за фото Артему Ігнатенку, Олександру Книжатку, Олександру
Милашевському, Яну Бахмату та Анатолію Андреєву
