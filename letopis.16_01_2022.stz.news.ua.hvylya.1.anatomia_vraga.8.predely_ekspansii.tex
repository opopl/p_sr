% vim: keymap=russian-jcukenwin
%%beginhead 
 
%%file 16_01_2022.stz.news.ua.hvylya.1.anatomia_vraga.8.predely_ekspansii
%%parent 16_01_2022.stz.news.ua.hvylya.1.anatomia_vraga
 
%%url 
 
%%author_id 
%%date 
 
%%tags 
%%title 
 
%%endhead 

\subsubsection{Пределы экспансии}
\label{sec:16_01_2022.stz.news.ua.hvylya.1.anatomia_vraga.8.predely_ekspansii}

Исторические факты свидетельствуют о том, что московская экспансия в их
своеобразной борьбе с \enquote{хаосом} останавливалась лишь там, где ее ограничивали
обстоятельства непреодолимой силы в виде географических препятствий либо же \textbf{в
виде государственных моделей с высоким уровнем внутренней организации и
возможностями своей политической инициативы.} Именно поэтому расширение на запад
было для Москвы не таким успешным, как это происходило на севере, востоке и
юге, где в большинстве случаев, за исключением, пожалуй, Китая, Москва имела
дело с десятками этносов, не имевшими устойчивой государственности.

Западная территория многоуровневой и \enquote{выборной} свободы всегда была не менее
опасна для московской формы \enquote{единоначальной стабильности} чем восточный
\enquote{фронтир хаоса}, однако, западные государственные модели имели более высокий
уровень общественной организации, чем Астраханское или Казанское ханства,
якуты, чукчи, узбеки или племена северного Кавказа, поэтому экспансия Москвы на
запад шла, по объективным причинам, гораздо медленнее.

В тех случаях, когда соседняя государственная модель или объединение государств
показывали высокий уровень общественной организации и субъектности, Москва не
могла проводить эффективную политику экспансии. Ослабленная Ливония, на которую
напал Иван IV (Грозный) в 1558 году, была почти завоевана Москвой, однако,
столкнувшись с хорошо организованной силой в виде Речи Посполитой и королем
Стефаном Баторием, экспансию пришлось свернуть и вскоре уже московские войска с
трудом сдерживали контрэкспансию объединенного польско-литовского государства,
проиграв в итоге двадцатипятилетнюю Ливонскую войну и потеряв на длительный
исторический период уже завоеванные было территории.

Территория Московского царства, как и территория современной Украины, долгое
время было открыта для регулярных и опустошительных походов татар Крымского
ханства - этого осколка Золотой Орды, который нес \enquote{хаос} с юго-запада.
Столетиями московиты прятались за \enquote{засечными чертами}, платили дань крымским
ханам, а хан Девлет Гирей еще в XVI веке доходил до Москвы и жег ее. Крымское
ханство было небольшим государством но его армия и вассальные отношения с
мощным экспансионистом Османской империей политически усиливали позиции
Бахчисарая. Очень долгое время именно крымские татары были важным фактором
геополитического воздействия Стамбула на Московское царство. Однако, как только
Османская империя организационно ослабла, Российская империя Екатерины II в
результате двух кровопролитных войн, смогла осуществить масштабную экспансию на
юг, оккупировав Крым, уничтожив государственность крымских татар, отобрав
северное Причерноморье и выйдя на Кубань, заложила основы для последующей
кавказской экспансии Москвы.

После войны с Швецией 1808-1809 годов, русские относительно легко, почти на сто
с лишним лет отобрали у ослабленного Стокгольма всю Финляндию, включив ее в
свое политическое пространство, отправив Швецию в \enquote{зрительный зал мировой
геополитики}, однако в 1939-1940 годах XX века высокая государственная и
общественная организация финского общества не позволили сталинскому СССР
навязать Финляндии свое понимание \enquote{единоначальной стабильности}. Финны, военным
методом, уложив в мерзлую землю 150 000 советских солдат, ярко и очень доступно
показали Москве величину материально-ресурсной цены, которую придется заплатить
за борьбу с \enquote{хаосом} на их территории. Соприкоснувшись с реальностью, прагматик
Сталин решил на время остановить экспансию, сконцентрировавшись на подготовке к
\enquote{освободительному походу} в Европу. Последующая \enquote{финляндизация}, имевшая место
после Второй Мировой Войны, так же показывает правильное понимание реальности
прагматичной финской элитой, ставившей в приоритет политический суверенитет и
современная экономически успешная Финляндия является подтверждением этой
\enquote{позорной} стратегии.

Речь Посполитая, в разгар внутриполитического кризиса в середине XVII века,
вызванного шляхетским латифундизмом, который у нас принято называть «повстанням
Хмельницького», утратила внутреннюю организацию. Как следствие, ослабла и
потерпела поражение от Москвы. По результатам войны (1654-1667) Речь Посполитая
потеряла Левобережную Украину и Смоленск. Более того, чуть позже, по Вечному
миру 1686 года, за 146 тысяч рублей, в качестве компенсации, Московское царство
выкупило у поляков Киев, который надолго - до 1991 года XX века года оставался
в культурном, политическом и идентичностном пространстве Москвы, как \enquote{мать
городов Русских}.

Сталинский СССР в 50 - е годы XX века, после Второй Мировой Войны продолжал
расширять горизонт экспансии своего влияния в Восточной и даже Центральной
Европе, пока не уперся в жесткие ответные инициативы стран коллективного
запада. После Фултонской речи Черчилля, в Европе от Киля до Триеста вырос
\enquote{железный занавес} и воплощением его стала Берлинская стена,
разделявшая два мира. Москва, увеличившая свои \enquote{мировоззренческие}
границы от Северной Кореи и коммунистического Китая до Чехословакии и Германии,
остановилась лишь тогда, когда США продемонстрировали порог ядерной цены
невосполнимого ущерба, которую придется уплатить Кремлю и \enquote{Карибский
кризис} стал тем \enquote{моментом истины}, после которого, на время, Москва
сбавила свой экспансионистский напор \enquote{единоначальной стабильности}.

