% vim: keymap=russian-jcukenwin
%%beginhead 
 
%%file 07_07_2023.stz.news.ua.donbas24.1.svitlana_otchenashenko_birthday.2.tvorche_osobyste_zhyttja
%%parent 07_07_2023.stz.news.ua.donbas24.1.svitlana_otchenashenko_birthday
 
%%url 
 
%%author_id 
%%date 
 
%%tags 
%%title 
 
%%endhead 

\subsubsection{Творче і особисте життя}

У репертуарі актриси низка яскравих ролей: Аркадіна, Раневська, Аббі Патнем. Та
ключовою для себе Світлана Іванівна вважала роль Марії Каллас (\enquote{Майстер Клас},
\enquote{Страсті по Марії}). Загалом, про зіграних персонажів Світлани Отченашенко
можна сказати, що всі вони сильні, неоднозначні та незабутні.

\ii{07_07_2023.stz.news.ua.donbas24.1.svitlana_otchenashenko_birthday.pic.6}

Народна артистка дуже любила Маріуполь і вважала його найріднішим містом. Саме
у Маріуполі, куди вона приїхала з Полтавської області, щоб почати працювати в
театрі, відбувалися найважливіші події в її житті. Тут вона познайомилася зі
своїм чоловіком Харабетом Юхимом Вікторовичем, скульптором і медальєром,
заслуженим діячем мистецтв України, який підтримував у всьому кохану жінку. Тут
народила і виховувала сина. Завдяки енергійній діяльності актриса змогла
зберегти історичну театральну спадщину Маріуполя, надихнувши телеканал \enquote{Сигма}
на створення унікального циклу, присвяченого історії та людям Маріупольського
театру.

\ii{07_07_2023.stz.news.ua.donbas24.1.svitlana_otchenashenko_birthday.pic.7}

Попри те, що Світлана Отченашенко була актрисою далекого провінційного театру,
їй вдалося стати відомою в столичних професійних колах. Водночас вона була
літературним критиком. Крім цього, на рахунку актриси ціла низка моноспектаклів
(вечори поезії програми віршів Анни Ахматової, Марини Цвєтаєвої, \enquote{Мати} за
однойменним твором Олександра Довженка тощо). У актриси був досвід і в кіно.
Вона зіграла бабу Зою у фільмі \enquote{Зачароване кохання} (2008 рік).

\textbf{Читайте також:} \emph{Унікальні факти про театральну культуру Приазов'я}%
\footnote{Унікальні факти про театральну культуру Приазов'я, Ольга Демідко, donbas24.news, 27.03.2023, \par%
\url{https://donbas24.news/news/unikalni-fakti-pro-teatralnu-kulturu-priazovya-do-vsesvitnyogo-dnya-teatru}
}

% svitlana+juhim
\ii{07_07_2023.stz.news.ua.donbas24.1.svitlana_otchenashenko_birthday.pic.8}

% ---- 5 fotos: 9-13
% performance scene microfon hands
\ii{07_07_2023.stz.news.ua.donbas24.1.svitlana_otchenashenko_birthday.pic.9}
% performance scene hands
%\ii{07_07_2023.stz.news.ua.donbas24.1.svitlana_otchenashenko_birthday.pic.10}
% svitlana lamp table sitting book
\ii{07_07_2023.stz.news.ua.donbas24.1.svitlana_otchenashenko_birthday.pic.11}
% child foto juhim
\ii{07_07_2023.stz.news.ua.donbas24.1.svitlana_otchenashenko_birthday.pic.12}
% ozherelje
\ii{07_07_2023.stz.news.ua.donbas24.1.svitlana_otchenashenko_birthday.pic.13}
% ---------------------------------

\ii{07_07_2023.stz.news.ua.donbas24.1.svitlana_otchenashenko_birthday.pic.14}

\begin{leftbar}
	\begingroup
\emph{\enquote{Дорогі мої маріупольці, всі українці, збережіть живою вашу душу. Не пускайте
туди ненависть і злість, адже зруйнована душа це набагато гірше зруйнованого
тіла!}}, — наголошувала Світлана Іванівна ще до повномасштабного вторгнення рф в Україну.
	\endgroup
\end{leftbar}

Коли рф щодня бомбила Маріуполь, народна артистка залишалася у своїй квартирі.
Вона не захотіла покидати рідне місто і не могла подумати, що будівля театру, в
якій вона пропрацювала 61 рік, може бути знищена. Померла Світлана Отченашенко
7 квітня 2022 року після життя в блокадному Маріуполі внаслідок набряку легенів
у 2 міській лікарні Маріуполя.

Нагадаємо, раніше Донбас24 розповідав, що Луганський обласний театр представив
Україну на Міжнародному фестивалі \enquote{New Wave Theatre} в Румунії.%
\footnote{Луганський обласний театр представив Україну на Міжнародному фестивалі \enquote{New Wave Theatre} в Румунії, %
Тетяна Веремєєва, donbas24.news, 05.07.2023, \par\url{https://donbas24.news/news/luganskii-oblasnii-teatr-predstaviv-ukrayinu-na-miznarodnomu-festivali-new-wave-theatre-v-rumuniyi-foto}}

\emph{Найсвіжіші новини та найактуальнішу інформацію про Донецьку й Луганську області
також читайте в нашому телеграм-каналі Донбас24.}

\emph{Фото: з особистого архіву Світлани Отченашенко}
