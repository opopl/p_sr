% vim: keymap=russian-jcukenwin
%%beginhead 
 
%%file 01_01_2022.stz.news.lnr.lug_info.1.zabeg
%%parent 01_01_2022
 
%%url https://lug-info.com/news/bolee-50-luganchan-uchastvovali-v-novogodnem-zabege-i-vodili-horovod-u-glavnoj-elki-lnr
 
%%author_id 
%%date 
 
%%tags horovod,novyj_god,prazdnik,lugansk,lnr,donbass,zhizn,zabeg,jolka
%%title Более 50 луганчан участвовали в новогоднем забеге и водили хоровод у главной елки ЛНР
 
%%endhead 
\subsection{Более 50 луганчан участвовали в новогоднем забеге и водили хоровод у главной елки ЛНР}
\label{sec:01_01_2022.stz.news.lnr.lug_info.1.zabeg}

\Purl{https://lug-info.com/news/bolee-50-luganchan-uchastvovali-v-novogodnem-zabege-i-vodili-horovod-u-glavnoj-elki-lnr}

Фото: Марина Сулименко / ЛИЦ

Более полусотни луганчан приняли участие в легкоатлетическом забеге,
посвященном празднованию Нового года и Рождества. Об этом передает
корреспондент ЛИЦ.

\ii{01_01_2022.stz.news.lnr.lug_info.1.zabeg.pic.1}

В мероприятии приняли участие как профессиональные спортсмены, так и просто
активные горожане. Забег стартовал из квартала Жукова. Собравшиеся пробежали по
центральным улицам города, сделав остановку у главной новогодней елки
Республики, где выпили чай, пообщались и поводили хоровод, после чего вернулись
к отправной точке.

\ii{01_01_2022.stz.news.lnr.lug_info.1.zabeg.pic.2}

Один из организаторов мероприятия, старший учитель отделения велосипедного
спорта Луганского высшего училища физической культуры Артем Пивоваров
рассказал, что участники пробежки преодолели дистанцию в 16 км.  

\ii{01_01_2022.stz.news.lnr.lug_info.1.zabeg.pic.3}

\enquote{Мы проводим это мероприятие уже четвертый раз. Для участия зарегистрировались
более 50 человек разных возрастов. Пандемия нам не помешала проводить эти
пробеги, и хоть в малом количестве, но мы их проводим. Я уверен, что
традиционный забег 1 января будет жить и дальше. Люди собираются здесь ради
общей идеи}, - добавил организатор.

\ii{01_01_2022.stz.news.lnr.lug_info.1.zabeg.pic.4}

Участник спортивного мероприятия Андрей Бельский отметил, что традиционный
новогодний забег способствует популяризации здорового образа жизни.

\enquote{Спорт, мне кажется, стал интереснее людям, чем выпивать и все остальное. Люди
тянутся к физическим нагрузкам для укрепления здоровья, стремятся к таким
полезным мероприятиям. Поучаствовать в этом - одно удовольствие}, - сказал
луганчанин.

Луганская спортсменка Ия Данько рассказала, что участвует в забеге третий год
подряд.

\enquote{Я занимаюсь спортивным ориентированием, мне нравится бегать. Раньше мы вместе
с мужем вдвоем устраивали пробежки в новогоднюю ночь. Сейчас мы так не делаем,
встречаем праздник дома. Но если такое мероприятие проводят, то почему бы не
поучаствовать и не поддержать традицию?} – отметила девушка.
