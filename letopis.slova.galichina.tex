% vim: keymap=russian-jcukenwin
%%beginhead 
 
%%file slova.galichina
%%parent slova
 
%%url 
 
%%author 
%%author_id 
%%author_url 
 
%%tags 
%%title 
 
%%endhead 
\chapter{Галичина}
\label{sec:slova.galichina}

%%%cit
%%%cit_pic
\ifcmt
  pic https://avatars.mds.yandex.net/get-zen_doc/3557661/pub_5f78449671c44f0829b959da_5f79bc4a8d3ae5589b962369/scale_1200
	caption Русское воеводство на карте средневековой Польши
\fi
%%%cit_text
И еще один интересный факт: \emph{Галичина} с самого начала польской оккупации в XIV
веке и до самого раздела Польши в конце XVIII века (то есть почти 400 лет)
имела административное название Русское воеводство Польского царства, в
котором, естественно, жили только русские, и центр этого воеводства находился в
Львове. В \emph{Галичину} это воеводство переименовали австрийцы, которым эта западная
русская область досталась в результате раздела Польши, они же и переименовали
русских в «украинцев», чтобы те забыли свои русские корни – Австрийская империя
всегда была врагом империи Российской, в том числе и в культурном плане.
Русский язык, пусть даже в исковерканном полонизированном состоянии, был
запрещен, и теперь все «украинцы» должны были разговаривать только на немецком
%%%cit_title
\citTitle{Почему современную Украину назвали «окраиной», а не более престижно - Киевской Русью?}, 
Исторический Понедельник, zen.yandex.ru, 22.02.2021 
%%%endcit

