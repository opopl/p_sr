% vim: keymap=russian-jcukenwin
%%beginhead 
 
%%file 12_04_2021.fb.mironenko_vladimir.1.gagarin_titov
%%parent 12_04_2021
 
%%url https://www.facebook.com/droid5.5/posts/3864210706982099
 
%%author 
%%author_id 
%%author_url 
 
%%tags 
%%title 
 
%%endhead 
\subsection{Сегодня большой день, великий день. Один из тех дней, которыми человечество оправдывает своё существование}
\label{sec:12_04_2021.fb.mironenko_vladimir.1.gagarin_titov}
\url{https://www.facebook.com/droid5.5/posts/3864210706982099}

\ifcmt
  pic https://scontent-bos3-1.xx.fbcdn.net/v/t1.6435-0/p526x296/172224327_3864210040315499_2071653916767855073_n.jpg?_nc_cat=101&ccb=1-3&_nc_sid=730e14&_nc_ohc=LvYfRyINkPUAX9BH69I&_nc_ht=scontent-bos3-1.xx&tp=6&oh=d3043dd4066e1493a175aa99e4b12790&oe=60995E10
\fi

Сегодня большой день, великий день. Один из тех дней, которыми человечество
оправдывает своё существование. День, в который разумная материя в нашем,
хомосапиенсовском лице, вытянула себя на новый виток развития.

Поздравляю всех с этим прекрасным праздником. Ниже — отрывок из моего
последнего романа «Небеса преображённые». Путь этого текста, уже понятно,
тернист; однако, вне зависимости от того, как сложится литературная судьба его,
я уверен, что дух диктовал его мне не просто так.

Примите как мой праздничный подарок.

*   *   *

В ночь перед полётом ни один из друзей не мог заснуть. Оба, конечно же,
отказались от снотворного. Ночевали в одной комнате. Лежали брёвнами, стараясь
не ворочаться, и даже не шевелиться — к кроватям были подключены датчики,
фиксировавшие любое движение. Датчики разместили якобы тайком от будущих
космонавтов, но перед грандиозными событиями, маячившими даже не на горизонте,
а высоко над головой, многое из тайного становилось явным.

Время от времени Герман мечтал — разумеется, о космическом прорыве. Например,
сейчас Гагарин станет вертеться, как юла, а утром его посчитают недостаточно
выспавшимся для важной миссии. Или, скажем, в последний момент у Юрия обнаружат
дырку на костюме. Или, например, по пути к ракете споткнётся и ногу подвернёт…

На этой мысли Титов устыдился.

— Нет, — сказал он себе. — Такого желать нельзя. Нужно оставаться человеком, в космосе или на земле. Земля, в конце концов, тоже летит в космосе, а вот человеком на ней живёт не каждый. И раз уж его выбрали, пусть Юре повезёт.
— Слышь, Гагарин, — сказал он тихо. — Удачи тебе завтра.
— Спасибо, — так же тихо отозвался Юра.

В три часа ночи вошёл Королёв, постоял у постелей изображавших сон притворщиков
и, ничего не сказав, ушёл.

Вдали отсюда Хрущёву тоже не спалось. Никита Сергеевич мечтал о том, как после
гагаринского анабазиса наступит Коммунизьм и не будет больше зим на земле.
Тогда он наконец отойдёт от дел, сделает себе соломенный брыль и отправится в
Калиновку, спрятавшись от коллег и охраны у старинного друга по фамилии
Пироженко.

Не спал генерал Каманин, закуривая одну за другой и представляя улыбчивого
деревенского паренька в холодной и чужой, внешней бесконечности.

Из причастных лишь Леонид Ильич Брежнев хорошенько выпил коньяка и мирно храпел
на белых простынях, видя во сне шаловливых русалок, заманчиво дразнящих его с
берега Цемесской бухты.

*   *   *

Повторно Королёв зашёл к Гагарину с Титовым в пять тридцать утра, в компании врача.

— Хватит уже притворяться, — сказал он. — Подъём.
— Как спалось? — спросил доктор.
— Как учили! — подскочил Гагарин.
— Спасибо, всё хорошо, — вежливо, но отстранённо ответил Титов.

Завтракали невкусной, но вдохновляющей смесью из космонавтских тюбиков. После
друзей стали облачать в космическое — Титова первого, поскольку он, вероятнее
всего, никуда не полетит, а значит, пусть дольше изнывает и потеет в тяжёлых
доспехах. Сначала термобельё, затем мягкий лазоревый комбинезон, затем
ярко-оранжевый скафандр. Шлемофон, гермошлем с гордой надписью «СССР», замки,
проводки, трубочки, застёжки…

Герман не мог не ощутить горечи: ведь всё это снаряжение, в которое его сейчас
так тщательно обряжают, скоро придётся снять несолоно, как говорится, хлебавши.
И снова он пристыдил себя: думай, дескать, прежде всего о деле.

Те, кто наряжал Гагарина, протянули к нему листики — за автографами. «Ну что
вы, что вы, товарищи, зачем же это нужно», — безропотно расписываясь, бормотал
скромный Юрий.

Сергей Павлович в своей характерной манере сразу посоветовал ему привыкать к
звёздному статусу: с этого дня, мол, задолбёшься так расписываться, без
некролога тебя уже никак не хлопнуть.

Автобус с космонавтами прибыл на стартовую площадку. Провожающие
высокопоставленные товарищи сгрудились на ней взволнованной могучей кучкой.
Нежные лучи рассветного солнца ласково упали на ответственные лица.

— Какое жизнерадостное солнце! — проблеял, как показалось желчному, но наблюдательному Титову размягчённый величием момента первый космонавт.
Герман вспомнил, как Королёв демонстрировал отобранным лётчикам кабину будущего космического корабля. «Разрешите ознакомиться?» — спросил тогда Гагарин, и, получив утвердительный ответ, сначала снял ботиночки, а потом уже полез юрким Юркою по лестнице в люк. Никто другой из парней ботинки снимать не стал.
Однако на этот раз первый космонавт тоже не стал стягивать обувь.
— Вот так, товарищ Гагарин, — сказал генерал Каманин. — Полтора часа полёта. Снаряжаться на старт нужно больше времени: как минимум два. А уж потом годами будут тебе это вспоминать. Вот такая, понимаешь, теория относительности Шекспира.
— Шинельку новую справил? — спросил космонавта Королёв. — Скоро Первому секретарю будешь рапортовать. Так мол и так, был в космосе, бога не видел…
— А вдруг увидит, Сергей Павлович? — подхватил академик Келдыш. — Наверняка не можем знать, ведь человек там окажется впервые.
— Ну нахер, — ответил Главный Конструктор. — После того, что мы здесь воротим, бог если бы даже и был, то показываться не стал бы…
Троекратно расцеловаться напоследок не удалось. Гагарин и Титов кое-как похлопали друг друга по шлемам. Железный человек Сергей Павлович Королёв, наблюдая действо, утратил демонстративную обыденность поведения и украдкой смахнул предательскую слезу.
Автобус медленно повёз Титова прочь от старта. Помощники сопроводили в бункер его, второго, имевшего все шансы стать первым, и приступили к раздеванию. Словно бы сдирают кожу, подумал претендент на полёт, и в третий раз устыдился: этот образ показался ему чересчур драматичным.
Шлем, перчатки, завязки, застёжки, трубки, шланги летели с Германа, как листья с клёна.
— Стартует, — объявил кто-то извне, из-за волнения крякая по-утиному, и техники кинулись прочь от Титова наблюдать исторический запуск.
В своём полуспущенном снаряжении Герман остался совершенно один.
— Забыли меня, — вслух сказал он, постояв так с минуту, и сам запрыгал на наблюдательный пост. Там, снаружи, пронзительно, как по покойнику, заходясь в визге и лязге, уже принялись завывать по улетающему  космонавту Юрке, закачивая топливо в камеры сгорания, стальные глотки насосов…
*   *   *
Вскоре после того, как Юра стартовал, Никита Сергеевич позвонил маршалу Малиновскому и спросил:
— В каком там у вас Гагарин звании? Старлей?
— Так точно, товарищ Хрущёв. Старший лейтенант.
— Мало, — отрезал Никита. — Надо повысить.
На том конце провода Малиновский недовольно и влажно засопел.
— Капитана можем дать, Никита Сергеевич, — довольно кисло ответил он.
— Какого капитана?! Минимум майора давайте.
Некоторое время Малиновский отнекивался, обречённо и ритуально. На каждом новом витке разговора аргументы Никиты Сергеевича носили всё более матерный характер. 
— Ну, майор так майор, — наконец согласился маршал. 
Так старший лейтенант Гагарин вернулся из космоса майором.
*   *   *
Человек создан для того, чтобы выходить за пределы; для того, чтобы превзойти самого себя. Так говорила Василина; так говорил Великий Лысый Никита.
Свершилось нечто очень важное. Нечто о человеке, каким он должен быть, да что там занудное словечко «должен», каким он есть, если он человек. Не только о том, кто побывал в космосе, но обо всех, кто готовил его визит.
Это навсегда в истории: бодрый бравый Гагарин марширует к трибуне, не обращая внимание на развязавшийся шнурок, утирает набежавшую вдруг слезу Никита Сергеевич, чистые люди шестидесятых, исполненные радости упоительной и заразительной, как мечта о Коммунизьме, машут шариками и флажками.
Редко преисполняющийся видовой солидарности, я бы, наверное, появись там, орал и восторгался вместе с ними. Человек вернулся из космоса, подумать только, из космоса. Какая сцена! Жаль, Королёва здесь нет, всё ещё засекреченного. Вернее, и он здесь, но в сторонке, за кадром, незримый дух науки.
И вечно мрачный Косыгин во время гагаринского рапорта торжественно и почтительно стаскивает с безобразной головы шапку, как и полагается поступить перед человеком, видевшим богов. 
Перед Юркой-космонавтом. О чём он там подумал, сказав своё последнее, обрубающее «Поехали»? Что пронеслось пред светлыми очами его в решающий тот момент? Отчаянные надежды, свирепые мечты, изнурительные тренировки, тонкая струйка чахнущего родника в деревне Клушино, старая холщовая рубаха с родным запахом отцовского пота и хохочущая девчонка в платье колокольчиком, спазмы долга, пароксизмы патриотизма, трепет перед полётом, буддийская лысина Хрущёва и долгий прощальный взгляд Главного Конструктора.
