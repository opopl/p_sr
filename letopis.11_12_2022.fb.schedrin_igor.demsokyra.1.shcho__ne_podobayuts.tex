%%beginhead 
 
%%file 11_12_2022.fb.schedrin_igor.demsokyra.1.shcho__ne_podobayuts
%%parent 11_12_2022
 
%%url https://www.facebook.com/permalink.php?story_fbid=pfbid03697xV321n1TQXjEuwGcYJuvT9PaxThRGr4DSisc23GyQSqAUF6h3D1Sm42MHafYjl&id=100013467931016
 
%%author_id schedrin_igor.demsokyra
%%date 11_12_2022
 
%%tags srach,psihologia,chelovek,obschestvo
%%title Що, не подобаються срачі?
 
%%endhead 

\subsection{Що, не подобаються срачі?}
\label{sec:11_12_2022.fb.schedrin_igor.demsokyra.1.shcho__ne_podobayuts}

\Purl{https://www.facebook.com/permalink.php?story_fbid=pfbid03697xV321n1TQXjEuwGcYJuvT9PaxThRGr4DSisc23GyQSqAUF6h3D1Sm42MHafYjl&id=100013467931016}
\ifcmt
 author_begin
   author_id schedrin_igor.demsokyra
 author_end
\fi

Що, не подобаються срачі? Я знаю один фокус, як зменшити їхню кількість. Цей
фокус чудово ілюструє старий анекдот.

Цирк. Конферанс‘є:

- А зараз виступлять наші чудові акробати!

Голос з залу:

- Акробати - пдрси!

Всі засмучені. Якось виступили, конферанс‘є оголошує наступний номер:

- А тепер на сцені жонглери!

Голос з залу:

- Жонглери - пдрси!

Всі ще більше засмучені. Жонглери відпрацювали, конферанс‘є оголошує знову:

- А тепер смертельний номер, вперше у світі: стрибок без страховки з-під куполу
цирку у стакан з водою!

Голос з залу: 

- Пдрси, куди ви мне тягнете?!

Так от. Кожного разу, коли ви критикуєте якусь групу в цілому (порохоботи,
зелеботи, банківські співробітники, лікарі, волонтери, ліберали, військові,
євреї, і так далі) - ви породжуєте тупий срач і проблеми особисто для себе в
майбутньому. Бо кожний читач, який натягне на себе (а таких може бути дуже
багато), сприйме за особисту образу. 

Але ж ви не хотіли його образити, ви писали В ЦІЛОМУ, бо побачили якийсь пост
представника цієї групи, і просто висловили свою чесну думку! 

Це ваше право, звичайно, життя - супермаркет, але попереду каса, і я не знаю
жодної людини, яка б не заплатила за необдуману публічну критику соціальних
груп або спільнот. Вас запам'ятають, вам будуть мститись люди, яких ви навіть
не хотіли образити.

А що ж робити? А от що. Або критикуйте конкретну людину (а не групу, яку ця
людина БУЦІМТО представляє) - або критикуйте групу, але будьте готові до війни
з кожним її представником, незалежно, подобається він вам чи ні. Як от з
росіянами, наприклад. 

Це стосується і критики «влади». Люди, які критикують «владу» в цілому, роблять
повну дурню, абсолютно не потрібну, їхня критика ніколи ні до чого не
призводить, і навіть робить гірше, бо дискредитує Україну як державу, а
конкретних негідників при владі абсолютно не зачіпає. Якщо взялися критикувати
- назвіть конкретний випадок, назвіть конкретних винних, скажіть, що цей
випадок змушує не довіряти партії такій-то, або меру такому-то. Це єдиний
різновид працюючої критики у демократичній державі.

Є й інші лайфхаки, як уменшити кількість дурних суперечок - себто, таких, що не
мають під собою фактичного конфлікту, а породжені суто невмінням комунікувати.
Залишайтеся на зв'язку, буду писати ще.
