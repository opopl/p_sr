% vim: keymap=russian-jcukenwin
%%beginhead 
 
%%file 30_01_2022.stz.news.ua.strana.1.narkomany_vtorzhenie
%%parent 30_01_2022
 
%%url https://strana.news/news/374154-zelenskij-hovorit-chto-putin-ne-napadet-reaktsija-zapada.html
 
%%author_id minin_maksim
%%date 
 
%%tags narkomania,rossia,ugroza,ukraina,vtorzhenie,zapad,zelenskii_vladimir
%%title "Они там все наркоманы?" Как Запад реагирует на заявления Киева о том, что "вторжения России" не будет
 
%%endhead 
 
\subsection{\enquote{Они там все наркоманы?} Как Запад реагирует на заявления Киева о том, что \enquote{вторжения России} не будет}
\label{sec:30_01_2022.stz.news.ua.strana.1.narkomany_vtorzhenie}
 
\Purl{https://strana.news/news/374154-zelenskij-hovorit-chto-putin-ne-napadet-reaktsija-zapada.html}
\ifcmt
 author_begin
   author_id minin_maksim
 author_end
\fi

В западных СМИ все чаще поднимается тема о расхождениях между Киевом и
Вашингтоном по вопросу \enquote{российского вторжения}. 

\ii{30_01_2022.stz.news.ua.strana.1.narkomany_vtorzhenie.pic.1}

Пока в США на официальном уровне раздувают эту тему, Киев начинает все громче
заявлять, что проблема преувеличена. И признаков скорого нападения на Украину
нет.

Причем опровержения подготовки России ко вторжению идут от украинской власти
все в более резких тонах.

Эту позицию начали замечать и крупные мировые медиа, которые раньше писали о
\enquote{вторжении}, как о решенном деле. А особенно явственно тема несогласия с
Байденом прозвучала на встрече Зеленского с западными журналистами. 

И это противоречие - когда \enquote{жертва вторжения} говорит, что угрозы нет - может
кардинально поменять всю эту историю. Причем не только глобально, но и лично
для Зеленского.

\enquote{Страна} разбиралась, какие последствия будут у раскола между США и Украиной по
вопросу \enquote{нападения России}.

\subsubsection{Как на Западе реагируют на опровержения Зеленского}

Впервые о том, что расхождения существуют, заявил телеканал CNN в своем сюжете
по следам телефонного разговора Зеленского и Байдена 27 января.

Журналисты заявили, что президент США якобы пугал украинского коллегу скорым
нападением на Киев. А Зеленский не соглашался и говорил, что все не так
однозначно. 

Позже и в Белом доме, и в Офисе президента этот инсайд опровергли, а CNN даже
удалила свой сюжет. Но уже 28 января выпустила новый - с говорящим заголовком
\enquote{Напряженность между Вашингтоном и Киевом растет по мере затягивания
украинско-российского кризиса}. 

Агентство сообщило, что Зеленский остался недоволен реакцией США на угрозу
вторжения России. При этом Байден считает, что Киев недооценивает эту самую
угрозу.

\enquote{Почему вы затеваете ссору с США? Почему вы допускаете утечки и искажение
информации, когда США - один из ваших единственных настоящих друзей? И зачем
просить больше оружия, если вы говорите, что угроза не выросла?} – приводит CNN
слова американского чиновника.

На следующий день - после пресс-конференции Зеленского западным СМИ - с
материалом о \enquote{зраде} со стороны Киева вышло издание Politico. Заголовок -
\enquote{Почему \enquote{неизбежное} бесит Зеленского}.

Журналисты пишут: Киев опасается, что США намеренно преувеличивают российскую
угрозу. И делает это для заключения договоренностей с Москвой, например, по
Донбассу (об этом, кстати, еще на прошлой неделе сообщала \enquote{Страна}).

В статье говорится, что изменение позиции президента Украины по российской
угрозе связано с \enquote{растущим раздражением} в отношениях с США.

\enquote{Действия Зеленского также вызвали вопросы о том, почему США отправляют Украине
оружие и деньги, если Киев не слишком обеспокоен масштабным вторжением?} -
задается вопросом издание в день пресс-конференции, которую дал Зеленский
западным СМИ, чтобы сбить волну военных слухов, распространяемых Западом.

Один из последних аналогичных выпадов сделало 30 января издание Washington
Post. Оно пишет, что западных чиновников встревожил публичный скептицизм
Зеленского и его команды относительно \enquote{вторжения}.

\enquote{Есть раскол среди его административных советников и того, что он может слышать
от своих военных и разведывательных служб}, - сказал правительственный чиновник
США, специализирующийся на делах России. \enquote{Я думаю, что Зеленский и его
политическая команда работают, исходя из собственного набора приоритетов, и они
не обязательно совпадают с приоритетами разведки и военных}, - заявил он
изданию.

То есть Washington Post намекает, что украинский президент саботирует данные
разведки о \enquote{вторжении} ради каких-то своекорыстных целей. 

\subsubsection{\enquote{Зеленский некомпетентен}}

Как видим, американские медиа наконец-то заметили заявления Украины о том, что
вторжения не будет. Но вопросы в СМИ звучат не к Белому дому и практически не
ставят его трактовок под сомнение. А наоборот - рисуют Зеленского возмутителем
спокойствия. И даже грозят остановить поставки оружия, если тот продолжит
дезавуировать \enquote{линию партии}. 

\raggedcolumns
\begin{multicols}{2} % {
\setlength{\parindent}{0pt}

\ii{30_01_2022.stz.news.ua.strana.1.narkomany_vtorzhenie.pic.1.guzhva}


И это достаточно тревожный знак для украинского президента. 

"Ранее западные СМИ практически игнорировали тот факт, что об угрозе
\enquote{нападения Путина} говорят в основном в США и в Британии. А Киев эту
угрозу опровергает и, более того, считает ее преднамеренным раздуванием паники
и \enquote{подставой} со стороны западных партнёров, так как из-за этого
Украина несёт экономические потери.

Но после пресс-конференции Зеленского, который хоть и путано, но донёс
иностранным журналистам мысль о несогласии с Байденом относительно реальности
угрозы вторжения (собственно, это и была главная цель мероприятия), делать вид,
что ничего не происходит, для западных СМИ было уже невозможно.

И стали появляться публикации о недовольстве Зеленского нагнетаемой на Западе
паникой, а также о подозрениях Киева, что эта паника раздувается дабы принудить
украинские власти к уступкам по Минским соглашениям. То есть все то, о чем
\enquote{Страна} писала последние недели.

Теперь интересно, какой выход может быть из этой ситуации.

С точки зрения логики, после того как главная \enquote{жертва} потенциального
\enquote{вторжения России} заявила, что в угрозу этого вторжения не верит,
накал нагнетания в западных СМИ должен снизиться. Ведь глупо кричать
\enquote{Путин нападет на Украину}, если сама Украина говорит: \enquote{Путин
не нападет}.

\end{multicols} % }

Судя по всему, на такой эффект и рассчитывает Зеленский.

Но эффект может быть и другой: на Западе объявят, что Зе просто ослеп и не
видит угрозы. И вообще некомпетентен. А потому по-прежнему нужно бить во все
колокола. Причём ещё громче бить.

И эту тему очень охотно многие подхватят и внутри Украины", - пишет главный
редактор \enquote{Страны} Игорь Гужва.
