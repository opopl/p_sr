% vim: keymap=russian-jcukenwin
%%beginhead 
 
%%file 10_01_2022.fb.fb_group.story_kiev_ua.1.tegeran
%%parent 10_01_2022
 
%%url https://www.facebook.com/groups/story.kiev.ua/posts/1836015679928558
 
%%author_id fb_group.story_kiev_ua,aleksejenko_aleksandr.kiev
%%date 
 
%%tags 
%%title Как надо дорожить каждой минутой, каждым мгновением жизни
 
%%endhead 
 
\subsection{Как надо дорожить каждой минутой, каждым мгновением жизни}
\label{sec:10_01_2022.fb.fb_group.story_kiev_ua.1.tegeran}
 
\Purl{https://www.facebook.com/groups/story.kiev.ua/posts/1836015679928558}
\ifcmt
 author_begin
   author_id fb_group.story_kiev_ua,aleksejenko_aleksandr.kiev
 author_end
\fi

Совсем недавно Украина помянула сбитый два года назад над Тегераном наш Боинг,
летевший в Борисполь. Мне вспомнились свои личные события, поскольку тогда я с
семьей находился совсем рядом. Об этом я и поделюсь...

В Дубай мы прилетели поздно вечером в Рождество. А поутру сын ошеломил меня
новостью:

- А ты знаешь, что в Иране разбился наш Боинг? Мы звонили в Киев.  Знакомые
авиаторы говорят – сбили его под Тегераном!..

Еще не было никаких официальных сообщений, а только домыслы и рассуждения
политиков и людей несведущих. Но друзья, работающие в авиакомпании,
сообщили, что в Борисполе всем было понятно с самого начала –  самолет
сбили. Как мог упасть самый новый в авиакомпании Боинг да с таким опытным
экипажем?!

Даже допуская, что с техникой все случается, но при любом отказе есть еще
время бороться за жизнь самолета и пассажиров. А тут яркая вспышка на
ночном небе – и все: тишина в эфире...

После этой печальной новости мы все задумались о том, что и сами могли
попасть «под раздачу». Наш-то самолет  тоже длительное время летел над
Ираном. Да еще и над неспокойным Курдистаном. А через неделю нам нужно
будет возвращаться тем же путем.

Однако, за время, что мы провели в Дубаях, все разрешилось.
Обстоятельства трагедии стали известны, Иран признал свою вину в
случившейся катастрофе, напряженность спала, и авиалайнеры продолжили
летать по утвержденным маршрутам.

Но людей и самолет уже не вернуть. За Боинг, думаю, деньги еще худо-бедно
компенсируют. А с людьми как? Остаются только человеческое горе и наша
скорбь о невинно убиенных...

Однако я хотел поговорить о другом.

В интернете я наткнулся на рассказ киевлянки, которая тем ранним утром
тоже улетала из Тегерана, но в другом направлении. Регистрация ее
собственного рейса и рейса в Киев проходили на соседних стойках.

Ради любопытства – «а вдруг кого знакомого встречу» – она подошла к
киевскому рейсу.

Сейчас уже известно, что в том рейсе из пассажиров было всего лишь две
украинки, а все остальные являлись гражданами других стран. В подавляющем
большинстве это были иранцы. Или этнические иранцы, но с канадскими
паспортами.

Естественно, никого знакомых она не встретила. Но ее внимание привлекла
одна сценка, которая отложилась в памяти.

Некая супружеская пара перепаковывала чемоданы. Такое часто бывает, когда
вес надо правильно распределить по чемоданам, чтобы не превышать
разрешенную ручную кладь. И вот они открыли свои саквояжи и озабоченно,
переругиваясь, перекладывают вещи.

Обычная житейская ситуация. Наверняка, многие наблюдавшие эту сценку, тут
же выкинули ее из головы.  Ведь никто в тот момент не ведал, что вскоре
случится.

Наша пассажирка наверняка тоже забыла бы об этой обычной аэропортовской
ситуации, если бы позже не узнала о катастрофе. И в памяти сразу же
вспыхнул ночной эпизод из тегеранского аэропорта. Два супруга тратят
время на никчемные разговоры, в то время как им осталось жить чуть более
часа. И она ужаснулась от понимания того, что все в этом мире вершится
непредсказуемо и фатально...

Так вот, для меня эта пронзительная картина стала самой щемящей историей
среди обстоятельств авиакатастрофы под Тегераном. Я будто воочию
представил себе банальную ситуацию перед вылетом, раздражение супругов, а
позже, уже в самолете, вероятное примирение, потом их спокойные мысли о
полете и пересадке в Киеве...

А возможно, они так и сидели в своих креслах, недовольные друг другом. И
не ведали, что время, которое им было отпущено на земле, улетучивалось
невозвратно с каждой секундой... А они продолжали сердиться. И вдруг  –
удар!

Наверняка, мгновенная смерть. Но каких-то важных слов никто из этих двух
самых близких людей сказать своей половинке не успел. Самых важных. Да и
никто из всех пассажиров и экипажа. Все закончилось...

Вот в этой связи я и думаю, как надо дорожить каждой минутой, каждым
мгновением жизни. Как важно проживать каждый день как последний! Не
упускать времени для того, чтобы обнять любимых, сказать доброе слово
родным и поцеловать своих детей и внуков...

Иначе может случиться так, что ты этого не успеешь сделать. Надо об этом
помнить всегда. Каждый день. Каждый час. Пока жив.

\ii{10_01_2022.fb.fb_group.story_kiev_ua.1.tegeran.cmt}
