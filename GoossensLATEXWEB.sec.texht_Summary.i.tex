\isubsec{texht_Summary}{Summary}

This chapter has shown how \texht can be used to translate \ \LaTeX\  documents 
into a HTML files, with a very extensive set of facilities to configure the results. 
The strength of this system is that it uses \ \LaTeX itself to read the file, permitting a 
far greater range of BTEX constructs (such as complex macros) to be handled than 
most other translators. 

Because much \texht's work is done by hooks in a  \LaTeX\ style file, it
is relatively easy to change it to generate different markup. In
Appendix B.2 on page 404 we look at how to make the system generate XML,
and we give some concrete examples of a \ \LaTeX\  to XML translator,
including MathML, in 8.2.3.2 on page 382. 
