% vim: keymap=russian-jcukenwin
%%beginhead 
 
%%file 26_11_2021.fb.lukjanenko_sergej.pisatel.1.zolotoj_vek
%%parent 26_11_2021
 
%%url https://www.facebook.com/permalink.php?story_fbid=1288466521670114&id=100015203342628
 
%%author_id lukjanenko_sergej.pisatel
%%date 
 
%%tags chelovek,humanity,istoria,obschestvo
%%title Золотой Век
 
%%endhead 
 
\subsection{Золотой Век}
\label{sec:26_11_2021.fb.lukjanenko_sergej.pisatel.1.zolotoj_vek}
 
\Purl{https://www.facebook.com/permalink.php?story_fbid=1288466521670114&id=100015203342628}
\ifcmt
 author_begin
   author_id lukjanenko_sergej.pisatel
 author_end
\fi

Золотой век начался году примерно в 1887, когда в головах читателей (а вовсе не
на Бейкер-стрит) поселился Шерлок Холмс, восстанавливающий справедливость для
сильных и слабых мира сего. Некоторые, впрочем, считают, что это случилось в
1868, когда лорд Гленарван отправился в кругосветное путешествие - спасать
капитана Гранта. Или на два года раньше, когда Раскольников задался вопросом:
тварь ли он дрожащая.

Но в любом случае золотой век начался в головах людей гораздо раньше, чем в
реальной жизни.

И несмотря на все революции, войны, голод, преступления - золотой век местами
почти возник.

В Европе и США в конце пятидесятых. И в СССР, как бы странно это ни прозвучало
для некоторых. И даже в многочисленных странах третьего мира. Возникли права
человека, международные кодексы и правила, защита меньшинств всякого рода.
Возникла избыточность - тоже не везде, но возникла. Стало можно учиться -
просто так. Отдыхать и развлекаться - всем. Потреблять знания, калории, галоны
и киловатты. Меряться пышностью Олимпиад. Содержать профессиональных
бездельников - от спортсменов странных видов спорта и до творцов никому
непонятных перформансов.

Золотой век закончился в 1991 году, вместе с распадом СССР, когда "история
закончилась" и напрягаться стало не нужно. Или в 2001, когда упали башни
всемирного торгового центра. Или в 2019, когда появились первые новости про
ковид.

Но в любом случае то, что начиналось как идея и фантазия, вначале закончилось
трагическими символами реального мира. А уже потом начало умирать в головах.
Там, внутри нас, ещё живёт мир без границ (которого в реальности не было),
всеобщая декларация прав человека (которая не соблюдалась), примат личного над
общественным (что всегда заставляло улыбаться мудрый древний Восток).

Мир меняется. С века сползает последняя позолота, и что под ней - тяжелый
свинец или трухлявое дерево, мы пока не знаем. Не надо начищать сползшую
позолоту, быстрее сотрёте. Привыкайте жить в новом мире.

Кому сколько достанется.
\ii{26_11_2021.fb.lukjanenko_sergej.pisatel.1.zolotoj_vek.cmt}
