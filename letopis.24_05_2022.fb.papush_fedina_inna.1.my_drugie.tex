% vim: keymap=russian-jcukenwin
%%beginhead 
 
%%file 24_05_2022.fb.papush_fedina_inna.1.my_drugie
%%parent 24_05_2022
 
%%url https://www.facebook.com/inna.inness.1/posts/1865345023660642
 
%%author_id papush_fedina_inna
%%date 
 
%%tags mariupol
%%title Мы другие. Мы отличаемся. Даже от других украинцев из неоккупированных городов 
 
%%endhead 
 
\subsection{Мы другие. Мы отличаемся. Даже от других украинцев из неоккупированных городов}
\label{sec:24_05_2022.fb.papush_fedina_inna.1.my_drugie}
 
\Purl{https://www.facebook.com/inna.inness.1/posts/1865345023660642}
\ifcmt
 author_begin
   author_id papush_fedina_inna
 author_end
\fi

Мы другие. Мы отличаемся. Даже от других украинцев из неоккупированных городов.
Психологи говорят ПТСР. Убеждают,  с этим можно справиться. Пока, ни у кого не
получается. 

Тот, кто выбрался из Марика - грезит этим городом, болеет им, тоскует по
нему. Пишет посты, выкладывает фотографии, ковыряет раны палкой. Это очень
больно. Больнее, чем сидеть в подвале. 

Тогда мы знали, что нужно выжить. Любой ценой. Или умереть, но только сразу.
Без ранений, болезней и попадания под завалы. Мы молились об избавлении или
счастливой смерти. Или - или. 

Чтобы прямое попадание. Или многотонная бомба, насквозь с девятого по первый
этажи.  Без мук. На муки нет сил и отваги. Когда ты кричишь от боли -  тебя
боятся и перестают любить. Не знают как помочь. Хотят, чтобы замолчал.

Мы другие. Наша планета -  Мариуполь. Мертвый город у Азовского моря.

Как я могу объяснить степень страха?  Ужас потерь, тоску по раздолбанному
вдребезги городу: по  мертвым девятиэтажкам, по убитым  близким, по холодному
мрачному подвалу .Как это можно рассказать за пять минут? 

Как объяснить, что мы - мариупольцы - другие. Мы хотим обратно в город,
которого больше нет. Разве это может понять - не мариуполец? 

У нас навсегда отобрали счастье. Я жду стадию принятия. Один хороший человек
сказал, что она будет. Я обязательно приму все, что у меня отняли. Я смирюсь и
не стану пробивать головой стену и звать на помощь Бога, когда ненависть
зашкаливает и хочется отомстить. Вернуться в город  и отомстить тем, кто нас
убивал. Тем, кто не давал шансов, кто жал на гашетку в самолёте, кто наезжал
танком на наши  живые души. 

Кто-то  из нас  научился  смеяться, радоваться мелочам, кормить бездомных
котиков, высаживать цветы в чужой стране, обзаводиться новыми  знакомыми и не
отвечать на вопросы о боли и страхе. Если мы улыбаемся - это не значит, что мы
стали прежними . Мы - другие. 

Наша боль разрослась до космических масштабов, но мы ее прячем от чужих глаз.
Наш страх - это не только Мариуполь. Это Харьков, Северодонецк, Николаев,
Одесса и другие украинские города. Мы, мариупольцы, знаем, как это безнадежно,
когда на  глазах умирает твой город, в котором все ещё остались люди. Наши
люди. Мариупольцы. 

Так сегодня выглядит мертвый Мариуполь
