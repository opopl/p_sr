% vim: keymap=russian-jcukenwin
%%beginhead 
 
%%file 12_04_2021.fb.kiev_old_fotos.1.my_kievljane_fotoalbom
%%parent 12_04_2021
 
%%url https://www.facebook.com/groups/923672854777788/posts/1178664519278619/
 
%%author 
%%author_id kiev_old_fotos
%%author_url 
 
%%tags foto,gorod,istoria,kiev,kievljane,ukraina
%%title Мы и есть родственники, мы — КИЕВЛЯНЕ!
 
%%endhead 
 
\subsection{Мы и есть родственники, мы — КИЕВЛЯНЕ!}
\label{sec:12_04_2021.fb.kiev_old_fotos.1.my_kievljane_fotoalbom}
\Purl{https://www.facebook.com/groups/923672854777788/posts/1178664519278619/}
\ifcmt
 author_begin
   author_id kiev_old_fotos
 author_end
\fi

Где-то далеко-далеко на полке мрачной и неприступной антресоли, в маленьком
пыльном чемоданчике, уютно устроился бархатный альбом — старинный друг,
хранитель семейных воспоминаний, запечатлённых на монохромных фотокарточках.

Старик, совсем ещё недавно даривший тепло и уют, собирая за круглым столом
семью, давно забыт. Его больше не приглашают на праздники, не открывают
скрасить долгий зимний вечер под треск камина в приглушённом полумраке восковых
свечей.. нет больше надобности в нем. Да и мы своими семьями давно уже не
собираемся, как прежде, вместе. Так вышло. Новый век диктует нам свои порядки,
и нет в этом ничего плохого. Социальная жизнь проникает в каждый дом
посредством гаджетов, когда-нибудь и их забудут, но не сейчас. 

Сейчас я предлагаю всем нам создать семейный альбом «Киевских историй». ХXI век
— век скоростей, и по сегодняшним меркам мы давно уже знаем друг друга, а с
некоторыми и вовсе как родственники. Мы и есть родственники, мы — КИЕВЛЯНЕ!

«Киевские истории» открывают группу, «Старі фотографії» — семейный альбом
сообщества, в котором каждый участник сможет разместить частичку своей души,
подарив вторую жизнь семейным воспоминаниям, спрятанным и давно забытым где-то
далеко в старом чуланчике на картонных страничках доброго старичка.

Так давайте же соберёмся за круглым семейным столом «Киевских историй» и
поделимся своими историями в фотографиях.

\begin{itemize}
\iusr{Павел Пауль}
Всегда что-то находится. Попытается искать и соответствовать.

\iusr{Ольга Кириенко}

Спасибо большое, очень хорошая и добрая группа, группа для души!

\iusr{Liliya Vickers}

На фото в цій групі мають бути присутні Киів і кияни, вірно я розумію?

\iusr{Hélg Smith}

Liliya, не обязательно.
\iusr{Liliya Vickers}

Hélg Smith тоді було б добре додавати у тексті поста не тільки рік зйомки, але
й місце. До речі, більшість дописувачів так і робить за власним бажанням 😉
Інші, мабуть, чекають на ваше «розпорядження» 😊

\iusr{Hélg Smith}

Liliya, кто хочет, тот добовляет без особых распоряжений. 😉

\iusr{Viktoria Koroleva}

Дякую щиро що прийняли до вашої цікавої групи, шкода, що раніше не знала про неї
\end{itemize}
