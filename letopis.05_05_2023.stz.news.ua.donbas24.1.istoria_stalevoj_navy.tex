% vim: keymap=russian-jcukenwin
%%beginhead 
 
%%file 05_05_2023.stz.news.ua.donbas24.1.istoria_stalevoj_navy
%%parent 05_05_2023
 
%%url https://donbas24.news/news/istoriya-stalevoyi-navi-zaxisnicya-mariupolya-rozpovila-pro-vesillya-na-azovstali-ta-rosiiskii-polon-video
 
%%author_id news.ua.donbas24,prokopchuk_elina.mariupol
%%date 
 
%%tags 
%%title Історія сталевої Нави — захисниця Маріуполя розповіла про весілля на Азовсталі та російський полон (ВІДЕО)
 
%%endhead 
 
\subsection{Історія сталевої Нави — захисниця Маріуполя розповіла про весілля на Азовсталі та російський полон (ВІДЕО)}
\label{sec:05_05_2023.stz.news.ua.donbas24.1.istoria_stalevoj_navy}
 
\Purl{https://donbas24.news/news/istoriya-stalevoyi-navi-zaxisnicya-mariupolya-rozpovila-pro-vesillya-na-azovstali-ta-rosiiskii-polon-video}
\ifcmt
 author_begin
   author_id news.ua.donbas24,prokopchuk_elina.mariupol
 author_end
\fi

\begin{center}
\em\color{blue}\Large\bfseries
Майже рік Валерія Суботіна провела у неволі. Зараз дівчина вчиться жити далі з
пам'яттю про загиблого чоловіка	
\end{center}

Історія кохання Валерії та Андрія Суботіних зворушила всю Україну. В цей час,
поки світ спостерігав за непохитністю захисників Маріуполя, на Азовсталь падали
бомби, а місто палало від ворожого вогню, у фортеці героїв було місце
найщирішим почуттям. Перебуваючи у пеклі війни, прикордонник Андрій зробив із
фольги обручку своїй коханій. Вона відповіла \enquote{так}, і 5 травня, в день
народження полку \enquote{Азов}, закохані одружилися. У пари було небагато часу разом —
вже 7 травня, виконуючи бойове завдання, Андрій Суботін загинув... А на Валерію
ще чекали нові випробовування — дні в оточеній Азовсталі, здача в російський
полон і майже рік у неволі. 10 квітня захисниця Маріуполя нарешті \href{https://donbas24.news/news/nava-povernulasya-z-polonu-zvilnili-zaxisnicyu-mariupolya-valeriyu-subotinu}{повернулася в Україну}.

Історію, сповнену болю, кохання та світлої пам'яті, вона \href{https://www.youtube.com/watch?v=SuTVEPisKJQ}{розповіла} ТСН.

\textbf{Читайте також:} \href{https://donbas24.news/news/ruyini-zamist-budinkiv-rozbiti-azovstal-ta-centr-google-onoviv-karti-mariupolya-foto}{%
Руїни замість будинків, розбиті \enquote{Азовсталь} та центр: Google оновив карти Маріуполя (ФОТО), %
Наталія Сорокіна, donbas24.news, 26.04.2023%
}

\subsubsection{Андрій з'явився, як диво}

Валерія народилася та зростала у селищі під Шахтарськом. У цьому куточку
Донбасу, який потім окупували росіяни та назвали \enquote{исконно русской землей},
споконвіку всі говорили українською, співали народних пісень та шанували
традиції. Маленька Лера теж говорила українською з близькими, але що таке
велика українська родина, ніколи не знала. З життя рано пішли її батьки, Леру
виховували дідусь та бабуся, які поховали власних синів — батька та дядька
дівчинки, які обидва були військовими льотчиками. Дідусь Лери багато хворів та
згодом помер, тож єдиною близькою людиною для дівчини довгий час залишалася
бабуся.
