% vim: keymap=russian-jcukenwin
%%beginhead 
 
%%file 11_06_2021.fb.nicoj_larisa.1.ukraina_sbornaja_mova_jazyk
%%parent 11_06_2021
 
%%url https://www.facebook.com/nitsoi.larysa/posts/931705110729195
 
%%author Ницой, Лариса
%%author_id nicoj_larisa
%%author_url 
 
%%tags futbol,jazyk,komanda.sbornaja,kultura,mova,pesnja,showbiz,sport,ukraina
%%title Це просто треш якийсь. Спочатку Збірна України з футболу публікує плейлист з російськими піснями
 
%%endhead 
 
\subsection{Це просто треш якийсь. Спочатку Збірна України з футболу публікує плейлист з російськими піснями}
\label{sec:11_06_2021.fb.nicoj_larisa.1.ukraina_sbornaja_mova_jazyk}
\Purl{https://www.facebook.com/nitsoi.larysa/posts/931705110729195}
\ifcmt
 author_begin
   author_id nicoj_larisa
 author_end
\fi

Це просто треш якийсь. Спочатку Збірна України з футболу публікує плейлист з
російськими піснями. Тоді каже, що то не їхній плейлист - і публікують новий,
уже їхній, куди не увійшла жодна українська пісня. Українські футболісти
демонструють публічно свою московитість - а тоді обурюються, що московити їм
вказують на форму... 

Українські футболісти! На тлі усього світу, який ще недавно співав українською
мовою \enquote{Плакала}, а тепер українською мовою співає \enquote{Шум}, який поклав на лопатки
всі світові рейтинги, ВИ НЕ ВНОСИТЕ у свій плейлист жодної української пісні -
ВИ ДІЄТЕ ЯК кончені МОСКОВИТИ. Тож, не дивуйтеся, що московія вам забороняє
форму. Ви їхні піддані. Ви їхнє продовження. Ви їхні - з усіма \enquote{потрохами,
тобто, мозгами}. Ви їхні. Вони вас так сприймають. Сприймають як своїх. Чому б
їм не вказати вам, що вам носити? От вони і вказують. 

Скажіть, дорогі українські вболівальники. От Ви там раптівки влаштовуєте на
сайті УЄФА з гаслом \enquote{Слава Україні}. Я рада, ви молодці. 

А чи задумувалися ви, що Українські борці, кому належить гасло \enquote{Слава
Україні} виступали проти засилля московії і всього московського в Україні? Тому
московія проти цього гасла. І сьогодні не просто модно в Україні так вітатися.
А ради того, щоб це гасло стало звичайним привітанням в Україні, було покладено
мільйони українських життів у боротьбі проти московії в тому числі. Тоді не
було інтернету і телебачення, і борці гасла \enquote{Слава Україні}
розповсюджували листівки, в яких закликали українців не говорити московською
мовою, не співати московських пісень, не переймати московських традицій, не
поширювати московське по Україні, не сприяти культурній московській окупації...

Що ми маємо сьогодні. Ви, захищаєте \enquote{Слава Україні} і водночас слухаєте
російськомовні пісні, дивитеся російськомовні ютуби, серіали, фільми, відоси,
ви підписуєте себе в соцмережах московською мовою, ви пишете дописи московською
мовою, ви поширюєте котиків з написами московською мовою, ви вітаєте один
одного картинками з новим годам і дньом раждєнія московською мовою, ви
спілкуєтеся між собою москвоською мовою, ваші діти вам пишуть есемески
московською мовою, ви наймаєте аніматорів, дідів морозів з московською мовою і
цей список можна продовжувати до безкінечности - ВИ, захищаючи \enquote{Слава Україні},
є водночас  ПОШИРЮВАЧАМИ московської експансії в Україні.

Чи не вбачаєте ви деяку НЕСТИКОВКУ в цьому? 

Так от, я Вам відповім. Нестиковка таки є. І називається вона амбівалентністю,
тобто, роздвоєнням. Не нормально бути одночасно поширювачами московитського  -
і Гасло борців з московитським вважати своїм рідним, відстоюючи його. 

Оце роздвоєння, стикування нестиковки у свідомості відбувається тому, що ми, я,
ви, наші друзі, ми всі, українці - травмований народ.

І ваша роздвоєність - це яскравий приклад нашого травматизму. Травму отримали
не ми, а попередні покоління українців, але цей посттравматичний синдром ПТСР
передався нам, бо ми виховані носіями ПТСР, і тому теж є носіями ПТСР. Він
нікуди не зник і діє сьогодні у вигляді роздвоєння. 

Я рада, що ви відстоюєте гасло \enquote{Слава Україні}, однак, чи не пора нам визнати,
що ПТСРи у нас таки є, на наше життя вони таки впливають, тож пора взятися за
їхнє подолання.

Як? Двома шляхами. 

\begin{itemize}
\item 1. Особисто ваш шлях. Свій простір звільняти від усього, що є московитською
мовою. Зробити над собою зусилля. Не поширювати котиків з російськими написами.
Виправити ваші імена на українські в соцмережах. Просити родичів і друзів не
присилати вам картинок \enquote{сдньомражденія}, а лише з Днем народження. Не слухати
московитською, не дивитися московитською. Вимагати від акторів і музикантів не
створювати український продукт московитсвькою...

\item 2. Шлях гуманітарної політики нашої країни. Робити все те саме, але на
державному рівні. Звільняти український простір від продуктів московитською
мовою, наповнюючи українським.
\end{itemize}

Почніть із себе. Бо поки ви будете публічно демонструвати ваші особисті
московитські вподобання (наприклад, плейлисти) - доти і будуть московити нас
вважати своїм продовженням і роздавати вам вказівки. А ви їм будете доказувати,
що ви слухаєте московитською, дивитеся московитською, говорите московитською,
думаєте московитською, пишете московитською - але не московити? Авжеж.
