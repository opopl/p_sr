% vim: keymap=russian-jcukenwin
%%beginhead 
 
%%file 26_12_2021.fb.fb_group.story_kiev_ua.4.pechal_po_korabliku.cmt
%%parent 26_12_2021.fb.fb_group.story_kiev_ua.4.pechal_po_korabliku
 
%%url 
 
%%author_id 
%%date 
 
%%tags 
%%title 
 
%%endhead 
\zzSecCmt

\begin{itemize} % {
\iusr{Вадим Горбов}
Спасибо. За передышку.

\iusr{Людмила Козак}
Очень сожалею за теми домиками. Такие уютные. Как внутри не знаю. Не приходилось побывать.

\iusr{Anatoliye Anatoliy}
Спасибо. Там на Бажова жила хорошая знакомая нашей семьи

\iusr{Александр Асатуров}
Бажова + БВС это уже просто Дарница

\begin{itemize} % {
\iusr{Галина Полякова}
\textbf{Александр Асатуров} Конечно, Дарница! Но для нас, старых киевлян, это еще и Соцгород. Хоть и без Кораблика.

\iusr{Александр Асатуров}
\textbf{Галина Полякова} ну я киянин молодой) но Кораблик помню. У вас хорошо получилось, спасибо!

\iusr{Нина Светличная}
\textbf{Александр Асатуров} Ні, Соцмістечко.

\iusr{Александр Асатуров}
\textbf{Нина Светличная} хай так
\end{itemize} % }

\iusr{Алла Парадня}

Не знаю, кто там высказывает недовольство, я с большим удовольствием и
интересом читаю Ваши миниатюры и даже боюсь пропустить что-то. В любом рассказе
всегда есть что-то от личности рассказчика, но меня это не смущает. Личность
устраивает.)

\iusr{Елена Корниенко}

Ну у нас в Дарнице есть ещё один такой райончик со старыми \enquote{немецкими} домиками.
И большими \enquote{сталинскими} домами.

И был когда-то гастроном ,который все называли тэмп. Не знаю почему, но был. И
молочный, где продавали в поллитровых бутылочках куфир, сливки по 51 коп., иногда
сметану в баночках, но в основном развесную, как и творог. Молоко бывало и в
литровых, потом в треугольничках. Ох, как давно.

\iusr{Дмитрий Даен}

Соцгород однообразен в отличие от Дорогожицкой и парка детской ЖД. Я не говорю
об осколках детства: Дмитриевской, Воровского, Сенном рынке, Львовской площади,
Артёма, Обсерваторной. Нельзя сваять за 60 лет продолжение
полуторатысячелетнего города. Особенно, если прожил в нем всю осознанную (67
лет из 70-ти) жизнь. A propos, хрущевки строились на Дорогожицкой, а 2-х и 3-х
этажные сталинки - на Грекова, Максима Берлинского, Щусева. Так что не только
Соцгород.

\begin{itemize} % {
\iusr{Галина Полякова}
\textbf{Дмитрий Даен} 

Есть ли смысл сравнивать жилой район с парком детской ЖД? Соцгород уникально
спланирован: детские садики, спрятанные подальше от проезжей части, уютные
скверики, тоже удаленные от улиц, просторные дворы со спортивными и детскими
площадками. Я ни в коей мере не умаляю достоинств названных Вами улиц. Но они
формировались веками. А соцгород был построен в одночасье и по конкретному
плану. В этом его особенность и прелесть. А \enquote{хрущовок} в Киеве полным-полно.
2-х этажных \enquote{сталинок} я, честно, говоря знаю очень немного. Это по большей
части все же немецкие постройки.

\begin{itemize} % {
\iusr{Дмитрий Даен}
\textbf{Галина Полякова} Не будем спорить. Каждому - своё. С наступающим и всего наилучшего!

\iusr{Natala Vashchuk}
\textbf{Galina Poliakova} 

это вы плохо знаете Сырец и район вокруг Щусева *вернее - справа и слева от
Щусева, 4-5 кварталов от Телиги к станции метро Сырец): двухэтажные домики,
дворы с погребами, детские сады внутри района, микорайоны там так и строились -
по одному плану практически в одночасье

\iusr{Галина Полякова}
\textbf{Natala Vashchuk} 

Даю Сырец я плохо знаю. Но я ведь ничего плохого ни об одном районе Киева не
сказала. Я написала о Кораблике!

\iusr{Natala Vashchuk}
\textbf{Galina Poliakova} 

не-не, я не в качестве обвинения  @igg{fbicon.face.rolling.eyes}  просто я вот Соцгород очень поверхностно
знаю, а Сырец хорошо, вы наборот, я написала только о том, что знаю @igg{fbicon.grin}  оценивать
районы, какой плохой, какой хороший и в голову бы не пришло. Я так думаю, во
многих районах есть одинаковые микрорайоны по времени застройки и типовым
планам

\iusr{Дмитрий Даен}
\textbf{Галина Полякова} 

Давайте закроем дискуссию. Всё я знаю хорошо. Уж точно не хуже Вас. Хочу
отметить, что на Дорогожицкой 13, нынче Парково-сырецкая в доме, где почта, я
прожил с 9-ти лет до 14-ти. Район строился на моих глазах. Впрочем я Вам ничего
не пытаюсь доказать. Ну не повезло мне полюбить Соцгород. Простите, бога ради.

\iusr{Галина Полякова}
\textbf{Natala Vashchuk} 

Вы, безусловно, правы. Но, полагаю, для того и существует эта группа: мы можем
познакомить друг друга с различными уголками Киева.

\iusr{Lola Madino}
\textbf{Дмитрий Даен} Дорогожицька так і є Дорогожицькою.
Парково-Стрецька - це колишня Шамрила. І пошта саме на цій вулиці, а Дорогожицька до неї перпендикулярна.

\iusr{Дмитрий Даен}
\textbf{Галина Полякова} У 1960 році там була Дорогожицька. Нема про що сперечатися.
\end{itemize} % }

\iusr{Ольга Морозова}
\textbf{Дмитрий Даен} 

Полутаратысялетния история состоит из шестидесятилетних историй, каждая из них
является продолжением. И последние 30 лет трагическое продолжние.

\iusr{Natala Vashchuk}
\textbf{Galina Poliakova} именно:)

\iusr{Нина Светличная}
\textbf{Дмитрий Даен} 

Вы ошибаетесь, Соцгород очень разнообразен касательно архитектуры. Здесь
сочетание классических хрущевок и хрущевок улучшенных - с большимы коридорами,
8-ми метровыми кухнями. Здесь сочетаються \enquote{сталинки} и дома по спецпроектам,
которые не бросаются в глаза - снаружи обычная 9-ти этажка, а планировка -
ого-го! Ну, и сразу послевоенные дома по совершенно необычным проектам. И,
конечно же, высотки \enquote{от сан саныча} на украденных стадионах.

\begin{itemize} % {
\iusr{Дмитрий Даен}
\textbf{Нина Светличная} 

Ничего я не ошибаюсь. 40 лет назад - пустынная улица Строителей с хилыми
деревцами. Архитектура... Также как проспект Мира, улица Гагарина и
Ленинградская площадь. Что может уютнее однообразия?


\iusr{Галина Полякова}
\textbf{Дмитрий Даен} 

А Вы сверните с улицы в дворы. И увидите, что задумано было очень неплохо.
Много сквериков, детских и спортивных площадок, много зелени... Конечно,
архитектурных изысков Вы не найдете, Разве что остатки фонтанов во дворах
\enquote{сталинок}.

\iusr{Нина Светличная}
\textbf{Дмитрий Даен} Ну ви і порівняли!
\end{itemize} % }

\end{itemize} % }

\iusr{Наташа Семечева}

\ifcmt
  ig https://i2.paste.pics/e9cd4f9ab2f7603492285c7cf674084d.png
  @width 0.2
\fi

\iusr{Нина Светличная}

Мы жили рядом - Бажова, 1, бывший ведомственный дом. Поэтому Кораблик был совсем
рядом. Жаль, что уничтожили, кому мешал(

\begin{itemize} % {
\iusr{Галина Полякова}
\textbf{Нина Светличная} 

Время меняет многое. Но мне кажется, что новые хозяева магазина могли бы
сохранить сам кораблик, даже видоизменив его. От этого они бы только выиграли.

\end{itemize} % }

\iusr{Наташа Семечева}

В Соцгород попала после жизни на метро \enquote{Арсенальная}, с немеренным количество
машин. Просмотрела много квартир по городу и вдруг предложили посмотреть
квартиру на Красноткацой, квартира понравилась, но хозяева нашли размен. Но мне
уже понравился район, метро рядом, до центра 15 минут. И квартира на Бажова
неплохая в сталинке. Живу тут около тридцати лет, о Печерске не жалею, и
однообразности не ощущаю, Рынок, новые торговые центры, магазины район
становится современным, да и кино постоянно снимают

\iusr{Ольга Глуздань}

\ifcmt
  ig https://i2.paste.pics/2c0ca05ee4c0fc38ec344ceee0efd6d3.png
  @width 0.2
\fi

\iusr{Мария Кар}

Я родилась на Соцгороде, на ул. Краковской, в 5ти этажной сталинке. Одни окна
выходили на Краковскую, вторые - во двор - отличный уютный двор, с площадкой,
деревьями, стадионом, на котором зимой заливался каток. И гастрономы назывались
кодами - например, 33ий, сейчас уже и не вспомню. Парочка была на Строителей,
на Краковской, и один вон тот, на Бажова. Намуглу Краковской и б-ра Труда были
обувный магазин и фотоателье, а во дворах - Галантерея. Множество магазинов
было по Попудренко. Кстати, в фотоателье сейчас частный загс, эта деталь меня
очень порадовала, фотографии там получаются красивые и светлые.


\iusr{Олена Голяницька}

Директор Галантереи жила в моем подъезде, в доме рядом с этим магазином.
Магазин-№28 располагался на углу Краковской и Строителей, №32 на углу
Строителей и Попудренка, ТЭЦевский на Строителей, ближе к магазину Тканей и 14
гастроном..-пересечение ул. Красноткацкой и ул. Бажова. А еще популярный
Молочный на Строителей и рядом аптека. В аптеке большая банка с пиявками, мы
детьми бегали на них посмотреть. Давно там не живу, однако пару раз видела свой
дом в нескольких фильмах-сериалах. \enquote{Беги, не оглядывайся} и \enquote{Ничего не
случается дважды}.

\begin{itemize} % {
\iusr{Мария Кар}
\textbf{Олена Голяницька} 

точно, 32 и 28 Гастрономы. А Галантерея была во дворе одного из домов на углу
Строителей, она была на крыльце с лестницей.


\iusr{Олена Голяницька}
\textbf{Мария Кар} 

..да, на крыльце... я родилась и жила в доме рядом.. оранжевый такой с шпилем на
крыше... замечательный такой дом был, двор дружный, детей много, лавочки,
старушки каждую весну на газонах цветики сажали, зимой в большом дворе дворник
нагребал огромные сугробы, мы по этим сугробам прыгали, устраивали снежные
крепости, все друг друга знали, друг другу помогали... замечательное детство...

\end{itemize} % }

\iusr{Denis Kostjuk}

Откуда у всхех такое устойчивое мнение что \enquote{аварийный} строили пленные немцы?
Читайте историю кто конкретно и как строил. А не -\enquote{одна баба казала}.

\begin{itemize} % {
\iusr{Натали Натали}
\textbf{Denis Kostjuk} немцы, немцы.... Не весь Соцгород, но очень много....

\iusr{Denis Kostjuk}
\textbf{Натали Натали} Именно пленных там небыло  @igg{fbicon.smile} 

\iusr{Наташа Гресько}
\textbf{Натали Натали} по проектам немецким строили. вот точно ли пленные? помню, и о румынах шла речь - не только о немцах.

\iusr{Натали Натали}
\textbf{Наташа Гресько} да, просто привыкли говорить «пленные немцы«, имея ввиду всех пленных.
\end{itemize} % }

\iusr{Serhii Ivantsov}

Очень понравилось сочетание: «обычный советский гастроном» и «продавщицы -
милые приветливые тетки». Это Вы о чём, простите?

\begin{itemize} % {
\iusr{Галина Полякова}
\textbf{Serhii Ivantsov} О гастрономе и о продавщицах. А Вы что подумали?

\begin{itemize} % {
\iusr{Serhii Ivantsov}
\textbf{Галина Полякова} в обычном советском гастрономе - милые приветливые продавщицы? Это называется-когнитивный дисонанс.

\iusr{Галина Полякова}
\textbf{Serhii Ivantsov} 

Вы хотите сказать, что все советские продавщицы были стервами? Мне 65 лет. Из
них, как легко подсчитать, я прожила в СССР чуть более половины. В магазины
ходила ежедневно. Вредных, злобных продавщиц встречала раз пять за 35 лет. Не
более. Продавец - это профессия, работа, а не черта характера. Они тоже могут
уставать, болеть, расстраиваться, нервничать, поскольку они тоже люди. А если к
человеку обратиться по-доброму, то Вам так же и ответят. Попробуйте и Вы не
пожалеете.

\iusr{Світлана Святюк}
\textbf{Serhii Ivantsov} 

Да-да... милые продавщицы сейчас только в супермаркетах... просто очень милые..
@igg{fbicon.laugh.rolling.floor} 

\iusr{Serhii Ivantsov}
\textbf{Галина Полякова}

я чуть-чуть моложе, но, что такое «советские продавщицы» - знаю не понаслышке.
« Вас много-я одна», «дома будешь умничать» , «бери, что дают» - пожалуйста,
не надо приукрашивать совок, это было именно так, в тех самых «милых советских
гастрономах». Бывали, конечно, исключения. Например, мама моего товарища
работала в гастрономе на неплохой должности. Один раз вместе с ним там
появился-и все продавщицы стали добрыми и приветливыми.

\iusr{Lola Madino}
\textbf{Serhii Ivantsov} 

а вам не здається, що це саме до вас були фрази? Може інші люди спілкувались з
тими ж самими продавчинями по-іншому?

\iusr{Галина Полякова}
\textbf{Serhii Ivantsov} 

Зло за пустые прилавки срывали на продавцах. Я вовсе не идеализирую СССР. но
нельзя же поливать дерьмом все и всех. Неужто, после 1991 года все в один миг
переродились и стали лучше? Зачем огульно клеймить всех? Мы ведь с Вами тоже
оттуда, не так ли? И вспомните мультик про крошку Енотика: улыбнитесь
приветливо.

\iusr{Serhii Ivantsov}
\textbf{Lola Madino} не здалося. Найчастіше, це були відповіді на ввічливі прохання та слушні зауваження з боку покупців.

\iusr{Serhii Ivantsov}
\textbf{Галина Полякова} 

сейчас прямо расплачусь от жалости к бедным несчастным советским продавцам, на
которых срывали злость... это была закрытая каста, попасть в сферу услуг или
торговли для простых людей-практически невозможно было. Они себя прекрасно
чувствовали даже, и особенно - при пустых прилавках. И в грош не ставили тех,
кто был с другой стороны этих прилавков. Если, конечно, это не была
какая-нибудь родственная каста. Или номенклатура.

\iusr{Галина Полякова}
\textbf{Serhii Ivantsov} 

Неправда. Абсолютная неправда. В гастрономах, в универмагах была жуткая текучка
кадров. Работать было очень тяжело. Продавцом мяса - да! Устроиться можно было
только по большому блату. Равно как и приемщиком стеклотары. Почему-то у нас
очень полярное отношение к СССР: одни рыдают от умиления тогдашним
благополучием, другие проклинают почем зря. А истина все же посередине.


\iusr{Serhii Ivantsov}
\textbf{Галина Полякова} 

не хочу развивать эту дискуссию, исключительно из уважения к Вам и вашей
позиции. Замечу только, если истина-посредине, то и неправда не может быть
абсолютной...

\end{itemize} % }

\end{itemize} % }

\iusr{Ирине Вильчинская}
какой теплый и \enquote{вкусный} рассказ об одном из уголков Киева...

\iusr{Юрий Блохин}
Спасибо действительно теплый рассказ.

\iusr{Людмила Викторовна}
Спасибо за рассказ. Душевно.

\iusr{Елена Аксенова}

На Станкозаводе ещё остались домики, которые строили пленные немцы, на
Кулибина. А рассказ очень теплый и душевный, пишите, жду.

\iusr{Галина Полякова}
\textbf{Елена Аксенова} Спасибо!

\iusr{Ирине Вильчинская}

а еще такие теплые, душевные рассказы, как бы приоткрывают дверцу в других
сердцах и люди, откликаясь, начинают тоже вспоминать и делиться... и вьется
веревочка добра откуда-то изнутри, из глубин памяти, вызывая легкую и немного
грустную улыбку...

\iusr{Arcadia Olimi}

Спасибо, тепло, уютно, ностальгия  @igg{fbicon.heart.beating} 

\iusr{Oksana Brauner}

В 80-е годы там в интерьере гастронома был довольно крупный барельеф или даже
скульптурная композиция с кораблем под парусами - поэтому и кораблик.

\begin{itemize} % {
\iusr{Ольга Кизуб}
\textbf{Oksana Brauner} И маленький бассейн (как две -три ванны) с живой рыбой, которой и торговали тут же.
\end{itemize} % }

\iusr{Iryna Naidonova}
Благодарю,! Детство пройшло в этом районе.

\begin{itemize} % {
\iusr{Алла Волонтирець}
\textbf{Iryna Naidonova}
Моё - тоже...

\iusr{Iryna Naidonova}
\textbf{Алла Волонтирець}

\ifcmt
  ig https://scontent-frx5-2.xx.fbcdn.net/v/t39.1997-6/s168x128/47270791_937342239796388_4222599360510164992_n.png?_nc_cat=1&ccb=1-5&_nc_sid=ac3552&_nc_ohc=pX-qMR-3rC8AX9PaeKS&_nc_ht=scontent-frx5-2.xx&oh=00_AT843omcsd5WMnLZc94LcrerU4h-ejj4j-3uQD5Rl3dyWg&oe=61D0D7F6
  @width 0.1
\fi

\end{itemize} % }

\iusr{Alla Zgurzhnitsky}

Моя бабушка получила квартиру на Краковской в 56 году. Я часто бывала там. Я
помню время, когда были пески и сосны, никакого БВС, и метро. На углу Бажова
был гастроном с отделом соков. Мы с бабушкой загорали на песке, а потом шли
пить сок. Я очень любила этот район. Немецкие домики были на аварийном посёлке,
потом построили метро Комсомольская

\begin{itemize} % {
\iusr{Алла Волонтирець}
\textbf{Alla Zgurzhnitsky}
Да, именно на аварийном посёлке!
В один из таких домов принесли меня, новорожденную...
\end{itemize} % }

\iusr{Наталия Кипаренко}

А я помню кораблик! Этот гастроном чем-то отличался- всегда казалось,что
продукты там свежее и разнообразнее. Это было тогда, когда сделали ремонт- вот
тогда и появилась стеночка из мозаики с корабликом. Детство...


\iusr{Євген Александров}
спасибо

\iusr{Tatyana Smirnova}

Интересно и с любовью Вы пишите о Киеве, его улицах и районах. И о Дарнице
тоже. Пишите исторические посты. Познавательно.

\iusr{Галина Полякова}
\textbf{Tatyana Smirnova} Спасибо на добром слове!

\iusr{Виктория Зайцева}

В Мариуполе тоже есть район, который пленные немцы строили, там все дома или в
два этажа, или в три. Трехэтажные обычные сталинки. А вот двухэтажные очень
интересные. У них эркеры на фасаде, по бокам огромные не то балконы, не то
лоджии, я не знаю как их назвать. Величиной с комнату. А чердачные окошки
круглые. И эти домики сразу выпадают из стандарта советской застройки. Только
уже когда я переехала в частный дом в этом районе, я узнала, что их немцы
строили. У нас эти микрорайоны носят прозаичное название \enquote{Квартал 25, Квартал
26 и 27.} В двухэтажных домиках деревянные лестницы, правда фонтанов у нас нет.
Сухое Приазовье, вода редкость. Но дворы и улицы засажены тоже акцией, и весной
необыкновенный аромат разливается по всем кварталам. Наши кварталы, судя по
всему, снесут не скоро.

\begin{itemize} % {
\iusr{Алла Волонтирець}
\textbf{Viktoria Zaytseva}
И это счастье, что снесут не скоро...
В этих домах особенная аура, где соседи знают друг друга, дружат...
На глазах соседей выросло три поколения...
Это как большая семья!
\end{itemize} % }

\iusr{Мария Маслова}

Очень тронул ваш рассказ, даже слезы на глаза навернулись. Хотя живу на Левом
берегу лишь с 1995-го (первые пять лет жизни провела на Ярославом Валу, потом
еще 33 года – на Чоколовке). Но на БВС жила моя свекровь, и мы часто ходили к
ней и по Бажова, и по другим улочкам, начиная с середины восьмидесятых прошлого
столетия. Мозаичного кораблика не помню, но хорошо помню уют этого района и
вообще тот, прежний уют, поэтому для меня ваш Кораблик с большой буквы –
понятен и много значит.

\iusr{Лариса Олейникова}

Спасибо, всколыхнули воспоминания! Всегда любила этот район, хотя жила по
другую сторону метро \enquote{Дарница} возле г-цы \enquote{Братилава}. Но с 1996 по 2007
работала на Попудренко, почти на углу Бажова. Исходила Соцгород вдоль и
поперёк, знакомые жили на углу Бажова-Краковская, т.ч. всё знакомо, и
\enquote{Кораблик} помню с пироженками и соком. А магазин \enquote{Ткани} на Строителей! Мне
кажется, что туда приезжали со всего города.

\begin{itemize} % {
\iusr{Alla Zgurzhnitsky}
\textbf{Larisa Oleynikova} никак не могла вспомнить название улицы попудренко. Она идёт параллельно метро.

\begin{itemize} % {
\iusr{Лариса Олейникова}
\textbf{Alla Zgurzhnitsky} Да! И тянется аж за метро Черниговская вдоль зданий НИИ ( правда, не знаю, что сейчас в этих зданиях)

\iusr{Alla Zgurzhnitsky}
\textbf{Larisa Oleynikova} я ещё помню лес на месте метро. Грибы собирали там
\end{itemize} % }

\iusr{Natalia Zabrovski}
\textbf{Лариса Олейникова} 

и обувной магазин был замечательный на Строителей, мама любила забегать туда,
хотя жили мы по другую сторону м. Дарница. Какие далёкие времена детства.

\begin{itemize} % {
\iusr{Alla Zgurzhnitsky}
\textbf{Natalia Zabrovski} я тоже несколько раз попадала там на импортную обувь.

\iusr{Natalia Zabrovski}
\textbf{Alla Zgurzhnitsky} да, чешская, югославская. Очень хорошо помню этот магазин и ткани, а ещё рядом была Фотография и мы с подругой по дороге домой с выпускного вечера, уже утром, там сфотографировали, получились ещё те красавицы, но память осталась!
\end{itemize} % }

\end{itemize} % }

\iusr{Елена Мельникова}

Я выросла на улице Строителей 22/14, она была тихая и посередине с бульваром на
котором росли каштаны. На другой стороне возле обувного магазина был киоск,
где продавал мороженое дядя Йося. вкусное

\iusr{Ирина Иванченко}
Очень милая, уютная зарисовка, спасибо

\iusr{Ольга Лубягина}
Спасибо. Очень хорошо знаю милый, уютный соцгород с детства, но, к сожалению, всё течёт, всё меняется...

\iusr{Alisa Brekhova}

Я к репетитору по математике ходила на Соцгороде, который жил в таком
замечательном уютном домике, построенном немцами. Помню гостиную, как мне тогда
казалось, очень просторную и огромный обеденный стол за которым мы занимались.
Мы быстренько решали все задачи, а потом минут 15 до конца урока он доставал
альбомы с репродукциями различных художников и рассказывал мне о них.
Математика была его профессия, а художественное искусство- хобби. Как давно это
было!

\iusr{Evgeniy Maslov}

Родился и прожил 23 года в том самом доме, где был гастроном \enquote{Кораблик}. Это БВС 17.

\enquote{Корабликом} он назывался потому, что ещё в старом гастрономе, до ремонта, там
находился макет парусного корабля а точнее - его носовая часть и часть
такелажа. После очередного ремонта этот кораблик убрали и оставили тот, о
котором написано в посте.

В доме ещё находился прокат и пункт приёма стеклопосуды. А также пункт зарядки
сифонов газированной водой. Такой же пункт был и недалеко на ул. Строителей...

Но это - уже совсем другая история...

\begin{itemize} % {
\iusr{Алла Волонтирець}
\textbf{Evgeniy Maslov}
Да, да!
Именно макет корабля!
\end{itemize} % }

\iusr{Леонид Красиловский}

Да, был у нас тоже офис на Углу Попудренко и Строителей в глубине возле
разрушенного детского сада ( там ещё кино снимали лет 15 назад про бандитов ...)

И вообще вдоль метро между «Черниговской» и «Дарницей» по Попудренко стоят
построенные немцами двухэтажки - давненько там не был

\iusr{Люсянка Балашова}
ну, здорово же.. пишите ещё

\end{itemize} % }
