% vim: keymap=russian-jcukenwin
%%beginhead 
 
%%file 18_02_2020.fb.fb_group.story_kiev_ua.1.okonchanie_elegii_detstva
%%parent 18_02_2020
 
%%url https://www.facebook.com/groups/story.kiev.ua/posts/1276874229176042/
 
%%author_id fb_group.story_kiev_ua
%%date 
 
%%tags chelovek,detstvo,gorod,kiev
%%title Окончание "Элегий детства"
 
%%endhead 
 
\subsection{Окончание \enquote{Элегий детства}}
\label{sec:18_02_2020.fb.fb_group.story_kiev_ua.1.okonchanie_elegii_detstva}
 
\Purl{https://www.facebook.com/groups/story.kiev.ua/posts/1276874229176042/}
\ifcmt
 author_begin
   author_id fb_group.story_kiev_ua
 author_end
\fi

Окончание "Элегий детства"

А вот третья соседка этой квартиры, Зоя, была несклонна к творческому полету
мысли, фантазиям и прочей восторженной дребедени. Её ум был рассудителен,
практичен и расчетлив.

\ifcmt
  tab_begin cols=3

     pic https://scontent-frt3-1.xx.fbcdn.net/v/t1.6435-9/86603418_3004311666269087_4885073932496928768_n.jpg?_nc_cat=102&ccb=1-5&_nc_sid=b9115d&_nc_ohc=yN9eWRcZZ5sAX8h2vb8&_nc_oc=AQmY1RU67AR8UzXp1qOgtNsoOJqz5TqtE-C3579Ja1Dx7CD6w5EC2BCCA6wNCLW2UF8&_nc_ht=scontent-frt3-1.xx&oh=898b5c0744ef0f58313f676dae0349dc&oe=61B2E9C2
		 @caption Моя мама Валюша

     pic https://scontent-frt3-2.xx.fbcdn.net/v/t1.6435-9/86696658_3004312056269048_4873978550207643648_n.jpg?_nc_cat=103&ccb=1-5&_nc_sid=b9115d&_nc_ohc=40rFKDtZnEcAX_NW1x_&_nc_ht=scontent-frt3-2.xx&oh=cefd3669ea248183670d51fb3ea90ace&oe=61B46580

		 pic https://scontent-frt3-2.xx.fbcdn.net/v/t1.6435-9/86696658_3004312056269048_4873978550207643648_n.jpg?_nc_cat=103&ccb=1-5&_nc_sid=b9115d&_nc_ohc=40rFKDtZnEcAX_NW1x_&_nc_ht=scontent-frt3-2.xx&oh=cefd3669ea248183670d51fb3ea90ace&oe=61B46580

  tab_end
\fi

Несколько лет после окончания Великой войны Зоин муж прослужил в освобожденной
Германии. Возвращались оттуда служивые люди кто с чем – роялями, мебелью,
шубами. Но не все. Володя, муж Валюши, пройдя инженером заграждений и
разминирования в составе 1-й железнодорожной бригады 3-го Балтийского фронта,
расчищая пути от фашистских мин наступающим на запад советским войскам, поседев
в неполные двадцать пять лет, привез домой, в город детства на Волге, три
пузырька цветной туши, несколько наборов карандашей фирмы «Pelikan», да
качественную готовальню. Всё это пригодилось молодому офицеру-слушателю
Военно-Транспортной Академии в Ленинграде. Его отец написал сыну: «Притащишь
фашистское барахло – ты мне не сын.»

\ifcmt
  tab_begin cols=3

     pic https://scontent-frx5-1.xx.fbcdn.net/v/t1.6435-9/86261825_3004312279602359_378855745170440192_n.jpg?_nc_cat=100&ccb=1-5&_nc_sid=b9115d&_nc_ohc=EkXnNz1GLoIAX8qhPIx&_nc_ht=scontent-frx5-1.xx&oh=b2bf0a9ede2119b26da9877c92aa1526&oe=61B23381
		 @caption Не знаю, правда, что в стаканчике...)))

     pic https://scontent-frt3-1.xx.fbcdn.net/v/t1.6435-9/86728225_3004313156268938_8309025610374578176_n.jpg?_nc_cat=106&ccb=1-5&_nc_sid=b9115d&_nc_ohc=04R4o7hjxMEAX9aXoAy&tn=lCYVFeHcTIAFcAzi&_nc_ht=scontent-frt3-1.xx&oh=a8a2b1adcb13a8e802ac645e16613799&oe=61B380CB

		 pic https://scontent-frx5-2.xx.fbcdn.net/v/t1.6435-9/86665842_3004314256268828_111201036915441664_n.jpg?_nc_cat=109&ccb=1-5&_nc_sid=b9115d&_nc_ohc=5X46iBu8XgsAX-Zb6_k&_nc_ht=scontent-frx5-2.xx&oh=a22eed97dd2f79ef5819d56eb1981e1e&oe=61B5ACD9

  tab_end
\fi

Но, мы немного отвлеклись от описываемых событий.

В те годы бытовая техника была из разряда диковинок.

Радиоприемник и телефон были, пожалуй, единственными представителями
технического прогресса в сфере быта. В конце 50-х забелели круглыми боками
холодильники «Днепр». Автомобиль считался роскошью, уж телевизоры и вовсе были
экзотической диковинкой. Счастливыми обладателями ящичка с экраном, величиной с
сигаретную пачку и огромной, похожей на аквариум, линзой, были и Зоя с
супругом. Ящичек стоял на почетном возвышении в четырнадцатиметровой комнате.
Семья Голубятниковых – муж, Зоя, два сына, гордилась им, как личной заслугой.
Линза стояла перед телевизором на изогнутой специальной подставке, и на день
плотно упаковывалась в шикарный китайский чехол из плотного черного шелка,
расшитый яркими павлинами и драконами.

\ifcmt
  tab_begin cols=2

     pic https://scontent-frt3-1.xx.fbcdn.net/v/t1.6435-9/86722585_3004314422935478_7230979005078306816_n.jpg?_nc_cat=102&ccb=1-5&_nc_sid=b9115d&_nc_ohc=AF-uq892gpcAX_BpRz-&_nc_ht=scontent-frt3-1.xx&oh=b082d32374bd35f2f44828adc30eaed0&oe=61B1FCDF
		 @caption Тут прообраз наряда Зоиньки (справа), описываемый в рассказе) — в городе Киев.

     pic https://scontent-frx5-1.xx.fbcdn.net/v/t1.6435-9/86717662_3004315096268744_2329552761990938624_n.jpg?_nc_cat=105&ccb=1-5&_nc_sid=b9115d&_nc_ohc=uhwp-6-7m2YAX8djwaJ&_nc_ht=scontent-frx5-1.xx&oh=0c9b9daad378edde007b80330da66a21&oe=61B54A57
		 @caption Моя подруга с самого детства и по сей день Ирина нарисовала мне мой дом со словами :"Вот тебе картинка, смотри каждый день и не ной о Заньковецкой!" Что я иделаю вот уже 12 лет. Висит в рамочке на стене)

  tab_end
\fi

Вечерами все домочадцы усаживались за круглым столом, глава семьи начинал
священнодействие. Со словами: «Ну-ка, ну-ка, что у нас там?», чехол снимался,
мягкой фланелькой протирались линза и крохотный экранчик. Включали телевизор,
потом долго крутили и переставляли с места на место рожки настольной антенны.
Наконец исчезали хрипение, свист, снежные хлопья и мутные волны на экране, в
голубом мерцающем свете появлялись знакомые, почти родные, Ниночка, Валечка,
Анечка и красавец Игорь, молодые и элегантные. «Добрый вечер, товарищи!
Начинаем программу передач Центрального телевидения.» Это были, ставшие потом
легендой телевидения, Нина Кондратова, Валентина Леонтьева, Анна Шилова и Игорь
Кириллов.

Всё было интересно, обсудить увиденное Зое надо было с кем-нибудь немедленно,
во время просмотра. А Ваня, после третьей ежевечерней рюмочки, начинал сладко
дремать. Сыновья, немного послушав новости, отвлекались игрой с деревянными
солдатиками, и Зое становилось скучно. Вот тогда она и приняла решение
приглашать соседей на просмотр вечерних телепередач.

Много лет спустя, в пронзительном фильме Никиты Михалкова «Пять вечеров» я
увижу картинку из детства – неловко, стесняясь, в комнату Тамары просачивается
пожилая супружеская пара и замирают перед голубым мерцающим окошечком.

Зоинька приглашала в гости Валюшу с дочкой, соседку с мужем и приятельницу,
имевшую взрослого сына, живших на первом этаже - за круглым столом усаживались
человек семь. И на этом сходство с фильмом заканчивается, ибо практичный
Зоинькин ум подсказывал : соседи - соседями, дружба – дружбой, но телевизор
электроэнергию мотает, детальки разные потихоньку изнашиваются, лапочки всякие
накаляются, проще говоря, расходы по эксплуатации аппарата налицо. Кто же их
компенсировать-то будет? А зрители на что? На тумбочке с газетами, около двери,
была поставлена коробка из-под духов «Красный мак». Зоя объявила – отныне все
по гривенничку платят, а с маленькой Иришки – пятачок. Поворчали, поосуждали
Зоину скаредность соседушки, но – деваться некуда, телевидение уже начинало
свой засасывающий марафон. Тихо звякали монетки в душистой коробочке, несколько
пар счастливых глаз вглядывались в голубой полумрак заэкранной жизни. Я не
помню практически ничего из тех передач, но одно событие осталось в памяти.
Потом, годы спустя, я увидела фильм «Старые друзья» и вспомнила, что сцена с
бешено несущимся табуном произвела на маленькую девочку такое впечатление, что
несколько ночей я плакала и кричала, оказываясь во сне под копытами страшных
лошадей.

Пройдет почти четыре десятка лет и Зоино ноу-хау вернется к нам в образе
«видеосалонов» – полутемных подвальчиков, бывших колясочных, дворницких,
мусорных дворовых павильонов с отвратительно показывающими телевизорами,
набором пошлых и глупых фильмов, гнусавым из-за бельевой прищепки на носу,
голосом переводчика. Детям тоже предоставлялась скидка…

Праздники в 60-е очень любили. Люди оттаивали душой от стылого ужаса войны,
хотелось жить, радоваться, веселиться. Первомай и 7 ноября отмечали наряду с
днями рождений и новогодними праздниками. Накрывали столы, приглашали гостей,
тосты были извечно-человеческие: за здоровье, за любовь, за детей, и, с
несколько может и политическим подтекстом, но такой воистину мудрый – за Мир во
всем мире!

Запахи праздников – восковой дух мастики для паркета, чесночно-мускатный
бабуниного холодца, ванильно-сладостный маминого наполеона. Раздвигался большой
круглый стол, хрустела скатерть крахмально-острыми складками, появлялись
«кобальтовый» столовый, на 12 персон,  сервиз и хрустальные фужеры с
дочками-рюмочками. Проигрыватель «Серенада» и черные, еще бьющиеся, пластинки с
мелодиями танго, фокстротов, вальсов ожидали гостей.

И пусть всё, кроме хлеба и воды в сифоне, надо было «достать по блату» или
получить «в пайке» – майонез и шпроты, копченую колбасу и лимоны, баночку
зеленого горошка и хорошую селедочку, баночку хрена и апельсины, но…

Такие ощущения веселья и праздника в последующие, всё более сытные и
бездефицитные, времена, почувствовать становилось все сложнее и реже.

Закуски, горячее, тосты, потом танцы – мама с папой скользят в ритме
«Рио-Риты», танцуют гости вальсы и танго, кавалеры кружат дам, дети завистливо
вздыхают. Шейков еще не придумали, а уж современная трясучка обернулась бы
вызовом кареты «психиатрички». Все это было вечером, а днем отправлялись в
парк. Теперь он Мариинский, а тогда был даже названием праздничен –
Первомайский.

Выездные «буфеты», где отпускали воды и ситро, пиво в бумажных стаканчиках, в
таких же стаканчиках – мороженное с деревянной палочкой-ложечкой, павильон с
играющими в шахматы и духовой оркестр на летней эстраде – атмосфера радости и
счастья. И как тут удержаться от флирта, романчика, ухаживаний? Вот такие знаки
внимания и начал оказывать нашей знакомой даме-коммерсанту Зоиньке сосед Сергей
Федорович. С женой Лидой, дочкой и сыном жили они этажом ниже Зои. Квартира
такого же расположения, те же четыре семьи в четырех комнатах, коммунальная
кухня с четырьмя хозяйками, двумя плитами и тремя кошками. Служил Сергей
Федорович на железной дороге, а значит, носил форменную тужурку с блестевшими
пуговицами и фуражку с кокардой.

Ах, что творит форма (не беда, что и не военная) с пылким женским сердцем!

Добавьте высокую, статную фигуру, темные глаза цвета чая и бархатный бас. Был
Сергей Федорович «щирим козаком», пел украинские песни и говорил «спiвучою
мовою». Зоинька была взята в плен быстро и бесповоротно, этакий блиц-криг
Амура. Об этом сладостном пленении тут же было сообщено подруге Валюше,
обсуждались стратегия и тактика ведения действий, наряды, украшения и
конспирация. Проще всего было с нарядами. Несколько лет после окончания Великой
войны Зоин муж прослужил в освобожденной Германии. Возвращались оттуда служивые
люди кто с чем – роялями, мебелью, шубами. Не все.

Володя, муж Валюши, пройдя инженером заграждений и разминирования в составе 1-й
железнодорожной бригады 3-го Балтийского фронта, расчищая пути от фашистских
мин наступающим на запад советским войскам, поседев в неполные двадцать пять
лет, привез домой, в город детства, три пузырька цветной туши, несколько
наборов карандашей фирмы «Pelikan» да качественную готовальню. Всё это
пригодилось молодому офицеру-слушателю Военно-Транспортной Академии в
Ленинграде. Мой дед отец написал тогда своему сыну: «Притащишь фашистское
барахло – ты мне не сын.»

Но, мы немного отвлеклись от описываемых событий.

Из послевоенной Германии Зоинькой было привезено немало бархатных и шелковых
платьиц, ажурное белье и шифоновые блузочки, замшевые и кожаные лодочки на,
невиданных еще в те годы, шпильках, бусы «под жемчуг», клипсы, браслеты и
множество другого крама и бижутерии. Большая часть этих богатств постепенно
уплывали на «толкучку» (вещевой рынок по-теперешнему) в Беличах, на выручку
покупались другие нужные вещи, в частности, телевизор. Но самые красивые платья
Зоинька берегла, одевала по праздникам. Она была дамочкой аккуратненькой,
миниатюрной, росту и полутора метров не было, личиком походила на молоденькую
хорошенькую обезьянку, но Голос… Голос был громок и пронзителен, речь –
образная, с цветистыми оборотами, эмоции непосредственны и, порой, несдержанны.

После нескольких праздничных вечеров с многозначительными переглядываниями,
волнующими танго, незаметными пожатиями локотка и коленочки под скатертью, было
решено активизировать действия.

СВИДАНИЕ! От этого слова сладко трепетало Зоинькино сердечко. Вот посудите сами
– семья, дети, кухонные баталии с соседками, «видеосалон» – быт, быт, быт
засасывал. А тут – ОНО, свидание.

Наитщательнейшим образом обсуждался с подружкой Валюшей наряд. Выбрано было
черное панбархатное платье с глубоким декольте на груди и спине, замшевые
лодочки с пряжечкой в виде подковки, низка жемчуга, лаковая сумочка и кружевные
перчатки. Какие-то диковинные духи были срочно приобретены у знакомой
спекулянтки. К бою готова! Завораживающе красива, нарядна, благоуханна,
блистательна!

Вот только одно уточнение. Всё это было уместно в обстановке праздника, а с
обычным вечером, посреди рабочей недели, как-то резко диссонировало. Но, что
поделаешь? Ведь выходные и праздничные дни семейные люди проводят у родимых
очагов. Назначено было на среду, на шесть часов, чудным майским вечером у
булочной, что была на углу. Многие обитатели улочки жили здесь подолгу,
квартиры не разменивались, все держались за центр, и, практически все, были
знакомы друг с другом. Поэтому у спешащих домой, после обычного трудового дня,
сограждан фигурка Зоиньки в обворожительном облачении вызывала некоторое ммм...
недоумение.

«Наши люди в таком виде в булочную не ходят!» – перефразируем мы известную
цитату.

Ох, как же неуютно ощущала себя Зоинька на углу, в закатных лучах солнца,
раскланиваясь направо и налево с вопрошающе глядящими жителями улочки.

«А Германа всё нет!» Промаявшись почти час, вызвав нездоровый интерес у
полутора десятков соседей, пунцовая, возбужденная, злющая Зоинька смачно
выругалась про себя, плюнула, едва не попав на замшевую лодочку, и… направилась
к своему парадному.

А в тот самый момент с другой стороны к этому же парадному молодцевато подшагал
наш Сергей Федорович. Настроение у него было отличное, только что в
«гадюшничке» соседнего дома он опрокинул пару рюмочек красненького, был весел и
игрив. Конечно, бедолага и думать забыл о том, вскользь оброненным за
праздничным столом, приглашении-полушутке. При виде расфуфыренной
красавицы-соседки с, так шедшим ей, гневным румянцем, реакция его была стара,
как подлунный мир. «Зоя! А куды це вы, така нарАдна?» – пророкотал бархатный
бас…

Среди наших читателей могут оказаться дети до 16-ти, поэтому дословно изложить
суть Зоинькиного ответа на поставленный вопрос, не представляется возможным.
Но, даю вам честное пионерское, одесские биндюжники почерпнули бы для себя
много новых словообразований.

Давно нет в живых почти всех героев этих забавных историй, нет коммунальных
квартир, телевизоров с линзами и «докторской» пристойного вкуса. Есть память
сердца, память детства, а, значит, живы и все эти люди. Ибо человек жив до тех
пор, пока его помнят

Киев, май 2008

все фотографии, использованные для иллюстраций - из личного архива семьи
Петровых.

Рисунок  моего искреннего друга Ирины Грушецкой (светлая ей память...)

"Заньковецкой 7" (акварель, тушь) март 2008

\ii{18_02_2020.fb.fb_group.story_kiev_ua.1.okonchanie_elegii_detstva.cmt}
