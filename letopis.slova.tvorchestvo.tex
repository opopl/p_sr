% vim: keymap=russian-jcukenwin
%%beginhead 
 
%%file slova.tvorchestvo
%%parent slova
 
%%url 
 
%%author 
%%author_id 
%%author_url 
 
%%tags 
%%title 
 
%%endhead 
\chapter{Творчество}
\label{sec:slova.tvorchestvo}

%%%cit
%%%cit_head
%%%cit_pic
%%%cit_text
Після такого одкровення українська влада постала перед засадничим вибором: 1)
або цілком в дусі ОУН (чи якщо завгодно Лукашенка) почати репресії проти
"колаборантів" та "іноземних агентів" і в такий спосіб остаточно перетворитись
на українську версію "русского мира" (тільки з протилежним геополітичним
вектором), або 2) запропонувати українцям іншу ідентичність, в якій ключовими
будуть не мова, віра та ненависть до москалів, а \emph{творчість},
інклюзивність, конкурентоспроможність та відкритість до світу. Ідентичність
майбутнього, а не минулого.  І найгірший вихід з цієї дилеми – це заховати
голову в пісок і не зробити жодного вибору
%%%cit_comment
%%%cit_title
\citTitle{Фіаско етнічної українізації}, 
Генадій Друзенко, analytics.hvylya.net, 28.07.2021
%%%endcit

%%%cit
%%%cit_head
%%%cit_pic
\ifcmt
  pic https://avatars.mds.yandex.net/get-zen_doc/50509/pub_60fe7b15d6464c5966da3e99_60fe7bab51197c3f2183635e/scale_1200
  width 0.4
\fi
%%%cit_text
Прекрасная задача, потому что она нерешаемая, но только не для русских. Для
русских людей это всего лишь неопределенность, сказать научным языком –
Квантовая запутанность. Да-да, вольно или не вольно мы попадаем в состояние
Квантовой суперпозиции и ищем кошку в темной комнате, пока не поймем, что сами
же являемся \emph{Творцами} и тогда мы находим (створяем) то, что нам необходимо.
Поэтому, когда мы уходим туда, куда Макар телят не гонял, находим своего
неуловимого «Белого бычка» и начинаем создавать новые нерешаемые задачи
%%%cit_comment
%%%cit_title
\citTitle{Русский Путь - решение нерешаемых задач}, Вестник, zen.yandex.ru, 26.07.2021 
%%%endcit
