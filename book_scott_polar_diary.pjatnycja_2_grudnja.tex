% vim: keymap=russian-jcukenwin
%%beginhead 
 
%%file pjatnycja_2_grudnja
%%parent body
 
%%url https://www.facebook.com/ScottPolarDiary/?notif_id=1606935611996386&notif_t=page_post_liker_invite&ref=notif
 
%%author 
%%author_id 
%%author_url 
 
%%tags 
%%title 
 
%%endhead 
\section{П’ятниця, 2 грудня}
\Purl{https://www.facebook.com/ScottPolarDiary/?notif_id=1606935611996386&notif_t=page_post_liker_invite&ref=notif}

День великого лиха. З четвертої ранку вітер став різко дужчати і невдовзі ми
вже йшли під самими тільки марселями, клівером і стакселем. Вітер був сильний і
море одразу ж розбурхалося. Скоро корабель почав глибоко пірнати і набирати
багато води через завітряний борт. Отс та Анкінсон з перемінною допомогою від
інших весь час намагалися утримати поні на ногах. По верхній палубі почали
розсипатися баки з бензином, фураж тощо; найбільше біди наробили неприв’язані
мішки з вугіллям, які море підіймало й жбурляло у прив’язані скрині. \enquote{Ти знаєш,
як старанно ми все закріпили, але жодне кріплення не вистояло б довго під
натиском цих вугільних мішків, вони лупили, як тарани. Нічого не лишалося,
окрім як стати до бою зі злом, і ледь не всі члени команди годинами гарували на
головній палубі, скидаючи мішки з вугіллям за борт, перев’язуючи скрині з
бензином тощо, і роблячи це настільки вправно, наскільки це взагалі було
можливо за таких складних і небезпечних обставин. Море безупинно накривало
людей хвилями, а подеколи й повністю їх затоплювало. У такі миті в боротьбі за
любе життя їм доводилося хапатися бодай за щось, аби їх не змило за борт; а з
урахуванням того, що довкола проносилися мішки з вугіллям й відкріплені скрині,
постійно була небезпека, що їх винесе в море разом із тим, за що вони
схопилися.

\ifcmt
pic https://scontent.fiev6-1.fna.fbcdn.net/v/t1.0-9/127494075_102841751672641_1621375401127981436_n.jpg?_nc_cat=111&ccb=2&_nc_sid=730e14&_nc_ohc=U2O3IqrPktUAX-3FyaD&_nc_ht=scontent.fiev6-1.fna&oh=b09108963b945130071fbf10a74d349c&oe=5FECADDE
caption П’ятниця, 2 грудня, День великого лиха
\fi

Щойно ми відновили якусь подобу порядку, як винятково потужна хвиля розірвала
кріплення і роботу довелося почати наново} \footnote{Уривки в лапках взято з особистих листів Скотта.}. 
Усю ніч море й вітер лютували,
а корабель пірнав дедалі безладніше; ми опустили паруси, залишивши тільки
грот-марсель і стаксель, зупинили рушії та лягли в дрейф, але толку від того
небагато. Спереду, де Отс та Аткінсон працювали всю ніч, раз у раз доходили
вісті, що поні падають. Далі було гірше, значно гірше – з машинного відділення
повідомили, що помпи забилися і вода піднялася понад ґратами.

Від тої миті, близько четвертої ранку, машинне відділення опинилося у центрі
уваги. Вода надходила, попри всі наші старання. Лешлі, по шию в бурхливій воді,
хоробро й затято чистив помпи. Якийсь час здавалося, що донкою та трюмною
помпою вдасться взяти воду під контроль, але надія ця виявилася скороминущою:
п’ять хвилин помпування незмінно призводили до однаково результату – помпи
знову засмічувались.

Справи були кепські. Кількість набираної води при такому розхитуванні була
небезпечною. \enquote{Ми знали, що зазвичай корабель набирає небагато води, але ми
також знали, що значна частина води, що затоплювала верхню палубу, стікалася
вниз; вона струмами проривалася крізь палуби. Корабель сильно навантажений,
багато йому не треба, аби пірнути під воду; а за таких обставин трапитися могло
будь-що}. Ручна помпа втягувала лічені краплі, а до її всисної труби дістатися
було неможливо; піднявшись, вода дійшла до котла, стала нагріватися і зрештою
стала такою гарячою, що ніхто не зміг би працювати за всисною трубою. Вільямс
змушений був визнати поразку. Що було робити? На якусь мить ситуація видалася
зовсім безпросвітною. Море неначебто здіймалося вище, ніж будь-коли; воно
поглинало завітряний борт і ют бурхливою зеленою водою; корабель хитався з боку
на бік; величезний шмат фальшборта віднесло геть. Трюмна помпа залежить від
головного рушія. Аби скористатися нею, необхідно було піти вперед. І саме в
такі моменти могутні хвилі били з завітряного боку; знову і знову поручні від
фок-щогли до грот-щогли вкривалися суцільним шаром закрутистої води, яка
жбухала на корму і ют. Одного разу й мені довелося стояти по пояс у воді біля
поручнів юту. 

Видовище на палубі було спустошливе, а в машинному відділенні вода, хоча її там
було й небагато, розливалася по всій підлозі так, що справляла страхітливе
враження.

Еванс організував два загони для роботи із відрами; матросів поставили невпинно
працювати за забитими помпами, – на той час більше ми нічого не могли вдіяти;
подумати тільки, ці потуги – наш єдиний засіб не дати кораблю потонути,
фактично це боротьба за його порятунок! І це може видатись дивиною, та все ж
зусилля принесли плоди: вервечка відер, яка рухається ось уже чотири години, і
помпи, що теж відводять якийсь дріб’язок, змогли зупинити наступ води, а то й
навіть трохи зменшити її рівень.

Тим часом ми обмізкували, як дістатися до всисної труби помпи: у перегородці
машинного відділення зроблять отвір, вугілля між нею та шахтою насоса
приберуть, і зроблять отвір у шахті. З таким огромом води на борту неможливо
відкрити люк над шахтою. Ми ще не вибралися з халепи, але у мене вже
зароджується надія, як і має бути, коли службу здійснюють так чудесно. Офіцери
й матроси за тяжкою працею співають шанті. У виснажливій спеці за котлом
Вільямс намагається зробити отвір у перегородці. Жоден не занепав духом. Вчора
потонув один собака, один поні здох і ще два у поганому стані, вони теж, схоже,
довго не протягнуть. \enquote{Час від часу сильні хвилі збивають собак за борт, і
тільки ланцюг їх рятує. Мірз із помічниками постійно стежить, аби ці знедолені
тварини не удушились, і намагається знайти їм кращий прихисток – завдання майже
безнадійне. Одного сіромаху знайшли вже задушеним, іншого так вгатило хвилею об
борт, що ланцюг порвався і він щез у воді; якимсь дивом наступна хвиля вернула
його назад на борт, і тепер він цілий і неушкоджений}. Буря завдала нам великих
збитків, але я відчуваю, що все буде гаразд, якщо тільки ми дамо раду із водою.
Ще одного собаку щойно змило за борт – халепа! Дякувати Богові, буря відступає.
Хвилі й досі стоять горами, але кораблю вже не так важко, як було. Молюся, щоби
ще до світанку ми змогли знову стати під вітрила.

