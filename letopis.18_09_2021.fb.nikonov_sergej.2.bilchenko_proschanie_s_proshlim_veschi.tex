% vim: keymap=russian-jcukenwin
%%beginhead 
 
%%file 18_09_2021.fb.nikonov_sergej.2.bilchenko_proschanie_s_proshlim_veschi
%%parent 18_09_2021
 
%%url https://www.facebook.com/alexelsevier/posts/1583965825282060
 
%%author_id nikonov_sergej,bilchenko_evgenia
%%date 
 
%%tags bilchenko_evgenia,chelovek,nacionalizm,obschestvo,zhizn
%%title БЖ. Прощание с прошлым: вещи
 
%%endhead 
 
\subsection{БЖ. Прощание с прошлым: вещи}
\label{sec:18_09_2021.fb.nikonov_sergej.2.bilchenko_proschanie_s_proshlim_veschi}
 
\Purl{https://www.facebook.com/alexelsevier/posts/1583965825282060}
\ifcmt
 author_begin
   author_id nikonov_sergej,bilchenko_evgenia
 author_end
\fi

Ещё один пост Евгении Витальевны Бильченко. 

Евгения Бильченко

БЖ. Прощание с прошлым: вещи.

По вечерам, выпив гидазепам, и, готовясь ко сну, я облегчённо вздыхаю, запустив
в мир свои анонимные статьи, научные работы, стихи, эссе, повести и романы...
Украинцы все так же хвалят меня под чужими именами, потому что есть, за что:
качество я клиентам гарантирую, помощь оказываю не тупую, а с консультациями и
лекциями, учу их писать, как мастер. Хотя и корректурой с копирайтерствои не
брезгую.

\ii{18_09_2021.fb.nikonov_sergej.2.bilchenko_proschanie_s_proshlim_veschi.top}

Вчера один парень вечером толкнул меня в живот. Угодил в яичник с кистой,
болит. Их было трое, смеялись они на суржике. Я не убежала, а докурила спокойно
сигарету, вспомнила фильм про Цоя с заточкой в животе, и меня охватил грешный
азарт. Что, если я получу нож в живот, свидетелей не будет, а те русские, что
превзошли украинцев в пожелании мне смерти, меня, наконец, простят?

Но потом я поняла, что, чем больше жертв я приношу здесь в Киеве ради России,
тем больше они меня там не прощают и ненавидят. Я пожалела их и подумала: может
быть, я тригерю их тем, что напоминаю им об их пассивности? Но, конечно, был у
меня момент деморализации и разочарования в русском мире, а сейчас Я ЛЮБЛЮ ЕГО
ЕЩЕ БОЛЬШЕ.

\ii{18_09_2021.fb.nikonov_sergej.2.bilchenko_proschanie_s_proshlim_veschi.bottom}

Готовность умереть за идеал у меня есть. Но вечером после трудов я и мои
плюшевые зверьки из разных стран озирают мои владения: вот картины, подарки
читателей со всего мира, вот рукописи десятков дисциплин и сотен лекций за
двадцать лет, каждая для меня - дитя... Вот - мое педагогическое призвание,
которым гордился дед. Вот - мои научные мысли, летящие по больной стране под
чужими именами...

Вот - премия "Лучший преподаватель года" от секретаря Сергея Стерненко Виктора
Андрущенко, ректора института национальной славы "Солнце и тьма Путина на
Украине: образовательный аспект". Я посмотрела на диплом, который у меня был
соблазн выкинуть, но маленькая птичка из Горловки, подаренная мне дончанами в
благодарность за честность, прострекотала:

-Насринасринасри!

И насраланасраланасрала я глубоко.

\ii{18_09_2021.fb.nikonov_sergej.2.bilchenko_proschanie_s_proshlim_veschi.cmt}
