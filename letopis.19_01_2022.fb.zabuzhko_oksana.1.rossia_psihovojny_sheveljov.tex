% vim: keymap=russian-jcukenwin
%%beginhead 
 
%%file 19_01_2022.fb.zabuzhko_oksana.1.rossia_psihovojny_sheveljov
%%parent 19_01_2022
 
%%url https://www.facebook.com/oksana.zabuzhko/posts/472833594208123
 
%%author_id zabuzhko_oksana
%%date 
 
%%tags infvojna,japonia,psihologia,putin_vladimir,rossia,ukraina,zapad
%%title ПРО ПСИХОЛОГІЧНІ ВІЙНИ, РОСІЮ, ЗАХІД, ЯПОНІЮ І ШЕВЕЛЬОВА
 
%%endhead 
 
\subsection{ПРО ПСИХОЛОГІЧНІ ВІЙНИ, РОСІЮ, ЗАХІД, ЯПОНІЮ І ШЕВЕЛЬОВА}
\label{sec:19_01_2022.fb.zabuzhko_oksana.1.rossia_psihovojny_sheveljov}
 
\Purl{https://www.facebook.com/oksana.zabuzhko/posts/472833594208123}
\ifcmt
 author_begin
   author_id zabuzhko_oksana
 author_end
\fi

ПРО ПСИХОЛОГІЧНІ ВІЙНИ, РОСІЮ, ЗАХІД, ЯПОНІЮ І ШЕВЕЛЬОВА. Трохи втомилась
заспокоювати своїх західних друзів (у них усіх panic disorder чи щось близько
того). Зокрема пояснювати, що це й є головна зброя Кремля - індукований психоз:
замість жити своє життя, творити-будувати-мислити-розвиватись, ви спускаєте
енергію й невідшкодовний час свого життя в чорний унітаз \enquote{планів Путіна}: він
диктує вам свій порядок денний. На фіга ж йому якось іще \enquote{нападати}, коли він і
без зайвих затрат, надурняк може вас розчудесно паралізувати якою-небудь одною
ексцентричною новиною, і ви ні про що більш не будете думати цілими днями?..

(Колись на моїх очах у такий спосіб було паралізовано ціле покоління
інтеліґенції - покоління моїх батьків, оте, котре починаємо називати \enquote{Задушеним
Відродженням} - і я знаю, наскільки для нас досі - досі!!! - \enquote{невидимими} й
незначними здаються ці форми психологічної війни, хоча в КГБ ними займались
цілі департаменти: досі \enquote{Арештовану Коляду} ми бачимо - обшук-арешт-тюрма, це
\enquote{понятно}, а неможливість, десятиліттями, для десятків тисяч людей займатись
улюбленою працею - це вже \enquote{не так понятно}, і ніби й \enquote{не так страшно} - не
повісили ж, як Івасюка, не зарубали, як Горську, не \enquote{залікували} в тюрмі до
смерти, як Марченка й Снєгірьова! - ну а вже коли людина роками тільки тим і
зайнята, що \enquote{грає в шахи} з КГБ, навіть на прогулянці з рідною дитиною
обдумуючи стратегії захисту й нападу, то така людина просто зникає з усіх
радарів, навіть в очах своїх близьких:

- Що ти кажеш, татку?

- Нічого, нічого, доцю, це я думаю, як я з одним дядьом буду розмовляти... 

І так дівчинка звикає, що татко завжди зайнятий. Іноді, коли під час прогулянки
його внутрішній монолог проривається уривчастим бурмотінням, вона перепитує:

- Це ти думаєш, як з одним дядьом будеш розмовляти?..

- Так, так...

І тільки через пів століття , вирісши, дівчинка розуміє, що її теж тоді було
окрадено: вона теж жертва психологічної війни - андроповські технології
\enquote{непомітно} відгризли і в неї колосальний шмат ресурсу, якого вона недоотримала
через те, що її батьки в цей час мусили готуватись до наступного допиту і,
замість спогадів, писати скарги в ЦК КПРС - тодішній еквівалент
\enquote{кол-центру}...)

А сьогодні Путін - продукт \enquote{андроповської школи} - застосовує цю технологію вже
в масштабі цілих країн. І всі ведуться. 

Цілий тиждень повторюю на Захід на різні лади, телефоном і на письмі: люди, не
годуйте вампіра! Готуйтесь проткнути його кілком, але, ради всього святого, не
панікуйте, - не розмовляйте (подумки) з Путіним, розмовляйте з своїми дітьми!

І на цьому тлі - зненацька, як ковток свіжого повітря, розповідь про реакцію
японських ЗМІ: єс!!! оце воно! Відчуйте різницю (лінк див. у коменті)

\enquote{Банзай} там уже чи не \enquote{банзай})), а психологічно - єдино грамотна реакція: та,
що забезпечує концентрацію сил, а не їх розпорошення. Аж я, мимоволі
облизнувшись, вкотре подумала про геніальність Юрія Шевельова, з його прогнозом
щодо ролі Японії в цивілізації 21-го ст. (хто читав наше Листування, той в
темі, хто не читав - стисло: Шевельов захоплювався Японією тому, що (крім Пд.
Кореї) це, казав, єдина країна, якій вдалась модернізація на традиційній
основі: невикористаний резервний шанс людства на майбутнє, коли Захід остаточно
виснажиться...)

Учімось. Читаймо. Озброюймось.
