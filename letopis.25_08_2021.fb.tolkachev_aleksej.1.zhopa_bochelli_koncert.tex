% vim: keymap=russian-jcukenwin
%%beginhead 
 
%%file 25_08_2021.fb.tolkachev_aleksej.1.zhopa_bochelli_koncert
%%parent 25_08_2021
 
%%url https://www.facebook.com/oleksiy.tolkachov/posts/4563216923697984
 
%%author Толкачев, Алексей
%%author_id tolkachev_aleksej
%%author_url 
 
%%tags kiev,koncert,nezalezhnist,ukraina
%%title Идите в жопу с своим Бочелли!
 
%%endhead 
 
\subsection{Идите в жопу с своим Бочелли!}
\label{sec:25_08_2021.fb.tolkachev_aleksej.1.zhopa_bochelli_koncert}
 
\Purl{https://www.facebook.com/oleksiy.tolkachov/posts/4563216923697984}
\ifcmt
 author_begin
   author_id tolkachev_aleksej
 author_end
\fi

Идите в жопу с своим Бочелли!

Хотел написать про горькое послевкусие от 30-й годовщины Независимости. Но
получилось слишком много букв. 

Поэтому ограничусь только одним очень неприятным эпизодом. 

Когда проанонсировали концерт Андреа Бочелли в Киеве, я невероятно обрадовался. 

Во-первых, мечтал его послушать вживую. 

Во-вторых, такое культурное мероприятие может быть исключительной изюминкой
торжеств по случаю 30-й годовщины Независимости Украины.. 

В газете KyivPost, как и во многих других информационных ресурсах было завлено,
что выступление состоится на Площади Конституции перед Мариинским дворцом в
17.00.

ВХОД СВОБОДНЫЙ. 

"Хм..." - подумал я.  Как-то мало места оставили между сценой и дворцом для
всех желающих. Но ведь анонсировали! 

За полтора часа до начала концерта я прибыл к Мариинскмоу парку, чтобы занять
место. Но тут я увидел то, чему бы стоило Януковичу поучиться. 

Весь периметр Мариинского парка был вплотную оцеплен военными грузовиками КРАЗ,
между машинами невозможно было даже протиснуться. Дестяки машин! Улица
Грушевского заблокирована грузовиками. 

Только вдумайтесь - Нацгвардией оцепили весь парк! Весь квартал! 

Да будь такое оцепление в 2014 году, хрен бы кто в парламенту прорвался!

Оказалось, что заявленный концерт - только для узкого круга небожителей и
приближённых, которых пускали строго по пригласительным. 

Весь остальной народ, прибывший на концерт, после такого плевка мог разве что
умыться. 

Женщина, приехавшая из Запорожья, сидела на тротуаре и плакала. 

И я чуть не плакал. 

Мне было больно, что на 30-м году Независимости в Украине всё так же
торжествует идеология садизма и издевательства над людьми. 

Сначала проанонсировать, а потом ограничить, запретить и нае@\%ть. Ведь не для
холопов честь. Каждый сверчёк - знай свой шесток. 

К сожалению, власти всё так же упиваются своим исключительным положением и
статусом, своими привелегиями, самоутверждаясь за счёт унижения простых людей. 

Во всём этом нет Любви! 

Как и не было её в организации парада на Крещатике, где тоже не подумали о
людях, которые более 2 часов стояли на солнцепёке и ждали начала движения
техники. 

Как и не было Любви в отсутсвии чёткого плана мероприятий для людей. Полный
провал коммуникации! А сколько хороших мероприятий прошло в безызвестности?  

Нет Любви и в том, что многих заслуженных людей оставили за бортом празднований
и торжественных приёмов. Как поётся в новой песне Ярмака - "Україна цінує лиш
мертвих"...

Парад Ветеранов вот уже который год оказался настоящей отдушиной... Здесь всё
искренне и с большой любовью друг к другу и к Украине. 

Так что идите в жопу со своим Бочелли. Народу положено Сердючку слушать. 

Всем спасибо. 

Мне полегчало. 

На видео - оцепление Мариинского парка и квартала 24 августа перед концертом
Андреа Бочелли.
