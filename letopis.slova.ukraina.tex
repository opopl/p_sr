% vim: keymap=russian-jcukenwin
%%beginhead 
 
%%file slova.ukraina
%%parent slova
 
%%url 
 
%%author 
%%author_id 
%%author_url 
 
%%tags 
%%title 
 
%%endhead 
\chapter{Украина}
\label{sec:slova.ukraina}

Так и живем в \emph{Украине}: удочек нет у власти для простых \emph{украинцев},
и рыбы нет для страждущих, \textbf{Еще никому из президентов не удалось объединить
Украину}, Александр Гончаров, strana.ua, 24.05.2021

Євробачення 2021: \emph{Україна} посіла п'яте місце, beztabu.net, 23.05.2021

\emph{Украину} нужно лишить права оценивать выступления участников от
страны-агрессора: это аморально, Игорь Кондратюк, obozrevatel.com, 23.05.2021

Уберите российский сапог и \emph{украинская} культура побьет все мировые
рекорды, Елена Кудренко, obozrevatel.com, 23.05.2021

\emph{Украинские} туристы подрались в самолете: мама младенца озвучила свою
версию конфликта, Марина Петик, obozrevatel.com, 24.05.2021

Не просто \emph{Україна}, а \emph{Україна}-Русь. Як Михайло Грушевський захищає
державні кордони?  radiosvoboda.org, 23.05.2021

Навіть попри те, що Путін не почав нову фазу завоювань в \emph{Україні} –
останній епізод кілька тижнів тому змусив людей у Вашингтоні і Києві
понервувати,
radiosvoboda.org, 24.05.2021

«Дайте нам грошей!»: большая \emph{украинская} мечта о великой халяве, from-ua.com, 24.05.2021

«Евровидение» в Европрессе: Италия — победитель, Германия и Британия  «пасут
задних», а \emph{Украину} назвали «кошмарной танцевальной угрозой», sharij.net,
24.05.2021

Зато выступление GoA от \emph{Украины} с песней «Шум» назвали одним из лучших
выступлений вечера,
sharij.net, 24.05.2021

Паралимпийская сборная \emph{Украины} заняла первое место на чемпионате Европы,
завоевав 37 золотых медалей, sharij.net, 23.05.2021

Кладбище империй: Чему научила США и \emph{Украину} война в Афганистане,
sharij.net, 24.05.2021

Для \emph{Украины}, где миллионы людей с разных сторон, Холодная война долго не сможет быть холодной,
\textbf{Дело Протасевича - еще один повод раздуть Холодную войну}, Денис Жарких, strana.ua, 24.05.2021

Вот я читаю посты некоторых российских патриотов, и не нахожу особой разницы
между ними и патриотами \emph{украинскими},
\textbf{Дело Протасевича - еще один повод раздуть Холодную войну}, Денис Жарких, strana.ua, 24.05.2021

Тобто. вже рік тому можна було бачити, що Північний потік 2 для \emph{України}
- програна справа, \textbf{Наш МИД живет в каком-то выдуманном мире}, Владимир
Воля, strana.ua, 24.05.2021

У келіях Печерського монастиря запалав світильник \emph{української} культури. Саме
тут бере свій першопочаток давня \emph{українська} література, художнє мистецтво,
медицина. Нестор-літописець – перший історик \emph{України}-Руси, автор Повісті, що є
головним джерелом вивчення \emph{української} історії, Агапій – перший відомий лікар,
Аліпій – перший живописець...
radiosvoboda.org, 06.05.2017

Шановні туристи, ми не заперечуємо вагомості внеску цього київського духовного
центру до скарбниці всіх східнослов'янських культур, але наголошуємо, що Лавра
– це передусім феномен \emph{української} культури».
radiosvoboda.org, 06.05.2017

Відвідувачі можуть запитати, чому у тексті Повісті Нестора Літописця (мова
твору церковнослов'янська) древня обитель іменується лише Печерською. Один із
найкращих знавців давньоруських літописів професор Василь Яременко стверджує –
це лише доказ того, що розмовною мовою автора була \emph{українська}. Саме тому
«\emph{українська} лексика ллється суцільним потоком у Повісті: жито, колодязь,
подружжя, туга, печера…» – аргументує професор.
radiosvoboda.org, 06.05.2017

\emph{Украина} была одной из главных составных частей Советского Союза — чрезвычайно
развитой в экономическом отношении, многолюдной и комфортной республикой.
\emph{Украинцы} по сути являлись второй имперской нацией, а выходцы из республики со
своими многочисленными группами поддержки порою оказывались на самых вершинах
власти, \textbf{Надоела Украина? Ежедневное «иди и смотри»}, Константин Кеворкян, ukraina.ru, 24.05.2021

Скоро в \emph{Украину} ворвется лето: синоптики заявили о \enquote{шикарной} погоде и сделали детальный прогноз, obozrevatel.com,
25.05.2021

На реках и водоемах с начала года утонуло более 300 \emph{украинцев}: причины жуткой
статистики, obozrevatel.com, 25.05.2021

\emph{Україна} і ЄС готують рішення про припинення польотів до і над Білоруссю.
Якими можуть бути інші санкції? radiosvoboda.org, 25.05.2021

«Це незграбно і підло. Винуватці зриву спецоперації \emph{українських}
спецслужб мають бути встановлені і покарані. А спроби їх «відмазати» марні», –
написав у твітері колишній міністр, radiosvoboda.org, 25.05.2021

\emph{Украина} может стать главным карателем Белоруссии, vz.ru, 25.05.2021

Сама \emph{Украина} попадет в патовую ситуацию, ведь для нее полеты над Россией
уже и так закрыты, а теперь закроется небо еще и над Белоруссией. Облетать
\emph{украинцам} придется много и долго, vz.ru, 25.05.2021

Важной является позиция \emph{Украины}, которая очень часто действует в русле западных политиков,
vz.ru, 25.05.2021

Уж если борщ не \emph{украинский}, Царит не наша правота – Расколот мир! В нём нет
единства, Повсюду мрак и темнота, odnarodyna.org, 25.05.2021

Бойцы, хранитель святыни, Над нами реет вечный стяг!  Неспевший славу \emph{Украине}
Державе нашей – лютый враг, odnarodyna.org, 25.05.2021

Эй, Макаревич, дорогуша, Ты лоб, пожалуйста, не морщь!  Ты \emph{украинцам} плюнул в
душу, А заодно попал нам в борщ!
odnarodyna.org, 25.05.2021

Но независимой державой \emph{Украйне} быть уже пора: И знамя вольности
кровавой, Я подымаю на Петра, Александр Пушкин, Полтава

Під впливом «Історії Русів» писали свої твори на \emph{українську} тематику
Кіндрат Рилєєв, Микола Гоголь, Тарас Шевченко. Використовував «Історію Русів» і
Пушкін, пишучи «Полтаву». Саме з цього \emph{українського} твору з'явилися і в
творі Пушкіна мотиви героїчної \emph{української} історії, мотиви боротьби
\emph{України} за незалежність, radiosvoboda.org, 25.05.2021

Храня суровость обычайну, Спокойно ведал он \emph{Украйну}, Молве, казалось, не
внимал, И равнодушно пировал, Александр Пушкин, Полтава

На вашому YouTube каналі одне із рекордсменів за переглядами – відео
\enquote{Чи матюкались давні \emph{українці}?}. Давайте розставимо крапки над і
щодо лайок в \emph{українській} мові. Чи винен в цьому умовний колективний
путін?  pravda.com.ua, 25.05.2021

Остап \emph{Українець}: Твоя індивідуальна \emph{українська} мова не має бути такою, як
вимагає неіснуючий \enquote{золотий стандарт}, pravda.com.ua, 25.05.2021

Часом ви дозволяєте собі говорити страшні речі. Про те, що \emph{українська}
мова не наймилозвучніша в світі, що вона не походить безпосередньо від
фінікійського письма, що лайки прийшли в неї не виключно як запозичення з
російської, а \emph{українською} не лише з мальвами розмовляють. Чому ви досі
не в списку ворогів нації? pravda.com.ua, 25.05.2021

Якщо розбирати, яке слово де-факто прийшло з \emph{української}, а яке
запозичене, вийдуть дуже прикольні речі. Вийде, що слово \enquote{щур} в
\emph{українській} мові скоріш за все з російської, а російське слово
\enquote{крыса} скоріш за все з \emph{української}, просто свого часу ми ними
помінялися, pravda.com.ua, 25.05.2021

Верховный суд \emph{Украины} принял решение о том, что в Ровенской области настоятели
двух захваченных храмов УПЦ останутся проживать вместе со своими семьями в
церковных домах, fraza.com, 25.05.2021

Сладкие парочки: Россия и \emph{Украина}, mikle1.livejournal.com, 21.11.2009

Даже я понял, что он лично виноват в том, что Россия до сих пор не развалилась
на демократические Чечни и Запопинские республики Окрайны, Китайны и прочая.
Еще он виноват в \emph{украинском} голодоморе 1932 года, монголо-татарском
нашествии и проигрыше Мазепы в 1709 году. Змею, укусившую вещего Олега тоже
тренировал он, mikle1.livejournal.com, 21.11.2009

Такие вот культы личности с обратными знаками. 2 страны, 2 личности, 2 системы
ценностей и 2 результата. Все смелые, прогрессивные, правильные, модные,
продвинутые, революционные и прочая говорят, что лучше \emph{украинский}
вариант, mikle1.livejournal.com, 21.11.2009

Не возвращайся в \emph{Украину}!  Угроза жизни дышит в спину, В затылок дышит
подлецо – Фашизма подлое лицо! Юнна Мориц, www.owl.ru

Василь Шкляр: «\emph{Українською} заговорять навіть в Африці, якщо без неї неможливо
буде повноцінно жити», globlvillage.com, 15.07.2020

Тим часом, \emph{Україна} вже не вперше у своїй історії опинилася сам на сам з
найкровожерливішим агресором. Соромно бачити, як плазує перед Росією Франція,
як стримано поводиться Німеччина, наче вони досі не оговталися від давніх
поразок у війнах з цим монстром, Василь Шкляр, globlvillage.com, 15.07.2020

Зеленский уже и не скрывает, что он является марионеткой в руках иностранцев и
все делает в их интересах, а не интересах \emph{Украины} и \emph{украинцев}.
Вот честно и открыто рассказывают как будут продавать землю иностранцам о чем
ранее молчал, Ростислав Кравец, strana.ua, 25.05.2021

\enquote{Руслан и Людмила} в вольном пересказе первого премьер-министр
\emph{Украины} Витольда Фокина. Цікаво. Чудово, www.skadtalk.cc, 21.06.2020

\enquote{Господи милосердний, \emph{Україна} – це не Раша}, Ірина Фаріон,
obozrevatel.com, 16.06.2019

Недогромадяни. Як колаборантки викидали \emph{українські} паспорти і нарвалися
на неприємності, \url{https://www.youtube.com/watch?v=yMFZEZLmnec} 21.05.2021,

Викинула \emph{український} паспорт – отримала проблеми. Чому колаборантів
варто позбавити права голосу? - Сергей Стерненко,
\url{https://www.youtube.com/watch?v=DR-IeF3SKA0}, 20.05.2021

Цей пам'ятний день було запроваджено Організацією \emph{українських} націоналістів на
честь наших Героїв, щоб ми, \emph{українці}, про них не забували. Щоб від покоління до
покоління, до наших дітей і внуків передавалася \emph{українська} традиція, віра,
патріотизм, виховання, національна пам'ять та національні Герої, – наголосив
Олег Тягнибок, zz.te.ua, 25.05.2021

\enquote{Ты на \emph{Украине} сейчас находишься! Ты понял? Ты понял меня? Ты на
\emph{Украине} находишься!}, Менти настільки \enquote{переатестувались}, що
підзабули уроки Майдану, Євген Дикий, gazeta.ua, 23.05.2021

Але ні, ми тут, в \emph{Україні}, для чогось робили Революцію гідності - зокрема для
того, щоб подібні нащадки НКВД не мали жодного шансу безкарно вбивати нас.
Здається, за 7 років вони настільки \enquote{переатестувались}, що підзабули уроки
Майдану, Євген Дикий, gazeta.ua, 23.05.2021

Заявляю, что я нахожусь в \emph{Украине} и не собираюсь прятаться от
правосудия. Я и дальше буду участвовать в законных следственных действиях и
добиваться справедливости как для себя, так и для всех \emph{украинских}
избирателей, доверивших мне высокое звание народного депутата \emph{Украины}»,
— подчеркнул оппозиционный политик, fraza.com, 25.05.2021

Єдине «тішить» у цій ситуації, що й \emph{українці} дали «достойну відповідь»
слов'янам. \emph{Українські} глядачі лише проголосували за виступ від однієї
слов'янської країни – Росії. Хоча Росію (що показово!) представляла співачка,
яка не є слов'янкою. Але ж \emph{українці} без «братньої Росії» жити не можуть – хоча
й воюють із нею. Тому й дали співачці Маніжі від Росії 4 бали. Добре, що не 12, 
Про слов'янську солідарність? Петро Кралюк, day.kiev.ua, 24.05.2021

Участник кампании по вакцинации публичных людей от коронавируса в \emph{Украине} не
может найти вторую дозу препарата, strana.ua, 25.05.2021

В половине регионов \emph{Украины} приостановлена вакцинация от Covid-19,
strana.ua, 24.05.2021

Ковидные качели. В \emph{Украине} снова пошел рост числа новых зараженных. За
сутки плюс 3 395 человек, strana.ua, 26.05.2021

... и в целом, в \emph{Украине} властям можно всё, \textbf{Украина легализовала
сомалийские действия государства в воздухе}, Елена Лукаш, strana.ua, 26.05.2021

- Александр Гарриевич, не \enquote{в \emph{Украине}}, а \enquote{на
\emph{Украине}}. Это важно. \enquote{В \emph{Украине}} говорят
\emph{украинские} сепаратисты, а сторонники триединого русского мира говорят
\enquote{на \emph{Украине}}, Мак Сим, zen.yandex.ru, 08.04.2021

Ржу и плачу. У меня, в Беларуси, 15 суток за это. У немцев, наверное, не
меньше. На \emph{Украине} какая то собственная Европа?
Игорь Правлоцкий, комментарий к видео, \textbf{Посадки
самолётов от Порошенко до Лукашенко. Что нужно понимать по скандалу с
задержанием Протасевича}, Олеся Медведева, strana.ua, 24.05.2021

Предположительно, президент Беларуси под территорией, \enquote{где угробили 300
человек} имеет в виду \emph{Украину} - 17 июля 2014 года, в небе над Донбассом был
сбит самолет, который следовал из Амстердама в Куала-Лумпур, strana.ua, 26.05.2021

\enquote{Кто платил подонку, который убивал людей в братской Украине?} -
отметил Лукашенко, \textbf{\enquote{Убивал людей в братской Украине}. Лукашенко
впервые прокомментировал задержание Протасевича}, strana.ua, 26.05.2021

Зеленский готовит закон о поражении \emph{Украины} в войне России, Владимир
Воля, strana.ua, 26.05.2021

В \emph{УССР} не было школы и садика, где дети не учили бы \emph{украинский}
фольклор, Андрей Манчук, strana.ua, 26.05.2021

Богдана була носієм дуже важливих цінностей. Не тільки журналістських, але
цінностей ідентифікації людини до \emph{українського}, до \emph{українських}
тем, до \emph{української} позиції, до \emph{українського} вільнодумства, з
точки зору несення своїх персональних цінностей гідности, \textbf{У Києві
презентували книгу спогадів про журналістку Радіо Свобода Богдану Костюк},
radiosvoboda.org, 26.05.2021

\emph{Украинец} установил рекорд, внедрив с свое тело восемь чипов. С их помощью он
заводит авто и открывает квартиру, strana.ua, 26.05.2021

Іван Франко – титан \emph{українського} слова і \emph{української} думки,
Михайло Цимбалюк, censor.net.ua, 26.05.2021

Ми мусимо навчитися чути себе \emph{українцями} — не галицькими, не
буковинськими \emph{українцями}, а \emph{українцями} без офіціальних кордонів.
І се почуття не повинно у нас бути голою фразою, а мусить вести за собою
практичні консеквенції. Ми повинні — всі без виїмка — поперед усього пізнати ту
свою \emph{Україну}, всю в її етнографічних межах, у її теперішнім культурнім
стані, познайомитися з її природними засобами та громадськими болячками і
засвоїти собі те знання твердо, до тої міри, щоб ми боліли кождим її частковим,
локальним болем і радувалися кождим хоч і як дрібним та частковим її успіхом, а
головно, щоб ми розуміли всі прояви її життя, щоб почували себе справді,
практично частиною його, \textbf{Одвертий лист до галицької молодежі}, Іван
Франко, 1905

Не забувайте, що ми досі в Галичині жили з національного погляду крайнє
ненормальним життям. Велика більшість нашої нації лежала безсильна,
закнебльована, а ми, маленька частина, мали свободу рухів і слова. І нам іноді
здавалося, що ми – вся \emph{українська} нація, що ми її чільні ряди, її
репрезентанти перед світом. Тепер, коли на російській \emph{Україні} не
сьогодні то завтра повстануть десятки таких центрів, якими тепер являються
Львів та Чернівці, ся наша передова роля скінчилася, \textbf{Одвертий лист до
галицької молодежі}, Іван Франко, 1905

Якщо це Русь, то Русь, а не \emph{Україна}. Якщо це Московія, то Московія, а не
Росія, \textbf{Життя нації. Використання топонімів Русь, Московія, Україна,
Росія}, zrada.org, 12.03.2021

\emph{Украина} в топ-5 стран мира по количеству сомнительных бот-сетей в Фейсбуке, 
strana.ua, 26.05.2021

Число смертей от коронавируса в \emph{Украине} перевалило за 50 000 человек с начала
эпидемии, strana.ua, 27.05.2021

Соцопросы регулярно показывают, что у нас процентов 70 считают - \emph{Украина}
движется в неправильном направлении. Причем, в какой-то момент этот процент
социологи фиксируют при всех президентах \emph{Украины}, при всех Кабминах и Верховных
Радах. То есть, \emph{украинцы} из раза в раз выбирают во власть людей, которыми потом
недовольны, \textbf{Протасевич нам важнее, чем состояние собственной экономики}, Владислав Михеев, 
strana.ua, 27.05.2021

Возмущаться бардаком на соседнем участке, подглядывая за соседом в дыру забора,
для \emph{украинцев} почему-то намного органичнее, чем навести порядок на своем
огороде, \textbf{Протасевич нам важнее, чем состояние собственной экономики}, Владислав Михеев, 
strana.ua, 27.05.2021

\emph{Україна} – не США, проте ми країна, яка воює і відстоює свою
незалежність. Це те, що притягує людей в армію, \textbf{Чому українські
військові йдуть з армії},  pravda.com.ua, 27.05.2021

Падение капитальных инвестиций в \emph{Украине} - просто беспрецедентное,
Александр Гончаров, strana.ua, 27.05.2021

14 травня в \emph{Україні} вперше відзначили День пам'яті \emph{українців}, які
рятували євреїв під час Другої світової війни, \textbf{«Через страх перед
німецькою владою»}, Андрій Усач, zaxid.net, 26.05.2021

\emph{Украина} не смогла нажиться на авиационной блокаде Беларуси, Михаил
Чаплыга, strana.ua, 27.05.2021

\enquote{Мне в голову приходит госпожа Меркель и ее походы за продуктами. Мне
хочется, чтобы \emph{Украина} была как Германия}, - добавила она, \textbf{\enquote{Не
может пройти 30 метров}. На вокзале Киева ради удобства Порошенко перегнали
поезд на другую колею}, strana.ua, 27.05.2021

Первые 224 \emph{украинца} получили уколы еще незарегистрированной в стране
вакцины Johnson \& Johnson, strana.ua, 27.05.2021

\emph{Пенсионер} получает в Украине в среднем 3500 грн в месяц, а распорядители
пенсий по всей \emph{Украине} по 1800 баксов.  Я уверен, что ни в Польше, ни в
Литве государственный клерк не получает таких зарплат. А сколько же тогда
получает директор департамента? А сколько трудяги центрального аппарата?, -
удивляется киевский предприниматель Павел Себастьянович, который на своей
странице в Фейсбуке поделился скрином декларации чиновницы, \textbf{Было 12
тысяч, стало 30. Как при Зеленском пошли в рост зарплаты чиновников, опережая
частный сектор}, strana.ua, 26.05.2021

Госдума России призвала мир осудить тотальную \emph{украинизацию}, strana.ua,
20.01.2021

Терористи вважають Лукашенка негідним. А 30\% \emph{українців} - найкращим
президентом, Сергій Фурса, gazeta.ua, 25.05.2021

На Донбассе снайпер террористов убил \emph{украинского} воина, obozrevatel.com,
27.05.2021

Папа Римский обеспокоен эскалацией насилия на востоке \emph{Украины},
strana.ua, 18.04.2021,

В этом законе сегодня вы закладываете очень серьезную бомбу против командиров
\emph{украинских} Вооруженных Сил. С большим трудом мы вытащили сержанта Маркива,
которого схватили, задержали за рубежом. По этому закону, который вы сейчас
проголосуете, могут задержать его командира роты, его командира батальона и
командующего Нацгвардии, \textbf{До 15 лет за участие в боевых действиях на Донбассе. Кого будут судить по закону о военных преступлениях}, strana.ua, 27.05.2021

Запад говорит \emph{Украине}: \enquote{Ребята, может, вы, наконец, приберетесь
дома}, hvylya.net, 27.05.2021

Сегодня Петр Порошенко напомнил, что он чемпион \emph{Украины} по неявке на
допросы, \textbf{Навстречу солнцу и прочь от допроса. Как Порошенко уехал от
СБУ в клубничный тур на Донбасс}, strana.ua, 27.05.2021

\emph{Украинский} политический пиар – бессмысленный, беспощадный и обязательно
неадекватный, - пишет журналист Вячеслав Чечило, \textbf{Навстречу солнцу и
прочь от допроса. Как Порошенко уехал от СБУ в клубничный тур на Донбасс},
strana.ua, 27.05.2021

Оно ведь как? Все «патриоты» \emph{Украины} жаждут призвать к ответу режим
Лукашенко, но чужими руками и за чужой счет. На что у союзного интуриста
бюджетов не предусмотрено, то с радостью оплатит \emph{украинский}
налогоплательщик. Хочет он того или нет.  Хорошо устроились, дорогие борцы!
\textbf{\emph{Украинский} патриот жаждет бороться с режимом Лукашенко - но чужими
руками}, Максим Могильницкий, strana.ua, 27.05.2021

\emph{Украина} - как феодальная Англия времен шерифа Ноттингема, только без Робин
Гуда, Алексей Кущ, strana.ua, 27.05.2021

Ще у 2017 році Верховна Рада звернулась до Конгресу США з проханням надати
\emph{Україні} статус основного союзника поза НАТО, але, як відомо, наша країна
цей статус так і не отримала. По – перше, США не схильні підписувати угоду, яка
би вимагала від них прийти на допомогу \emph{Україні}. По – друге, американська
еліта не хоче надавши цей статус \emph{Україні}, підштовхнути Росію до Китаю,
\textbf{Битва за \emph{Україну}. Хто винен, та що робити?}, Стефан Закревський,
hvylya.net, 27.05.2021

Битва за \emph{Україну}. Хто винен, та що робити? hvylya.net, 27.05.2021

Співробітники компанії з країни-агресора працюватимуть в \emph{Україні} «за
обміном», \textbf{«А як \emph{українською} «дружба?»: агенція, що розробила бренд
Ukraine NOW, привезла до Києва російських рекламників}, kyiv.media, 27.05.2021

В США расследуют вмешательство \emph{Украины} в президентские выборы-2020 на
стороне Трампа, strana.ua, 28.05.2021

\emph{Украинские} дипломаты сейчас пытаются помочь с возвращением \emph{украинцам}, которые
лечатся в Беларуси и \enquote{застряли} там из-за прекращения авиасообщения,
\textbf{Кулеба объяснил, почему Киев не отзывает посла из Минска после ареста
Протасевича}, strana.ua, 28.05.2021

Сытник заявил, что директора НАБУ должны выбирать иностранцы. «Для сохранения
институциональной независимости Национального антикоррупционного бюро
\emph{Украины}», \textbf{Послушать Сытника, так \emph{украинскому} народу
нельзя доверять государство}, Вячеслав Чечило, strana.ua, 28.05.2021

Олег Скрипка назвал \enquote{не \emph{украинцами}} тех \emph{украинских}
артистов, которые поют песни на русском, strana.ua, 28.05.2021

А есть те, что ездят в Россию. Пусть зарабатывают, но для меня они не являются
\emph{украинцами}, они просто резиденты \emph{Украины}, они не разговаривают на \emph{украинском},
они не находятся в мировоззрении \emph{украинском}, они не живут в \emph{украинском} мире.
Они живут в паттернах, в постсовках. Я иногда, может, неправильно говорил на
этот счет или не так интерпретировали мои слова, что таких людей нужно
изолировать, - сказал Скрипка, strana.ua, 28.05.2021

\enquote{Это резиденты \emph{Украины}, это не \emph{украинские} артисты. Это
люди, которые живут на этой территории, но они не \emph{украинцы}. Это чисто
моя личная точка зрения. Для меня этот вопрос не является вопросом. Люди тут
живут, но они являются частью русского мира}, - сказал Олег Скрипка, strana.ua,
28.05.2021

И этим заявлением Костржевский, как говорится, испортил всю малину премьеру.
Тот ведь рассказывал, какие прибыли \emph{Украина} получит после того, как весь мир
обложит авиасанкциями Белоруссию, а оказалось, что вместо гипотетических
прибылей мы уже имеем реальные убытки. То есть санкциями \emph{Украина} выстрелила
себе в ногу, strana.ua, 28.05.2021

Як може \emph{Україна} зупинити терористів, які захопили Раду ООН з прав
людини, lb.ua, 26.05.2021

\enquote{Зеленский имитирует Путина. Он считает Путина успешным и хочет быть таким же...
Мы видим параллели. Мы понимаем, куда идет наша страна и куда идет
\emph{Украина}: туда же}, - заявил российский главред, dialog.ua, 28.05.2021

\enquote{Медведчук - человек плоть от плоти} ру***ого мира... Свои три канала
(ZIK, NewsOne и \enquote{112 Украина} - ред.) он превратил в рупор Путина. Они
с утра до вечера занимались тем, что подрывали \emph{украинскую} власть,
dialog.ua, 28.05.2021

За перші чотири місяці 2021-го в \emph{Україні} сталося 126 підліткових
самогубств або спроб – Денісова, radiosvoboda.org, 28.05.2021

Як хочеться потрапити у світ без Русской общины \emph{Украины}, Русского блока
та їм подібних московських шавок.  Як хочеться почути живих Драгоманова та
Ключевського; тих людей, які займались історією як наукою, а не як повією,
\textbf{Забыть все = Забить на всё. Як хочеться усе забути}, zrada.org,
24.07.2010

Я не даремно говорив про те, що націонал-патріоти перетворилися на
\emph{український} різновид гаїтянських тонтон-макутів - секретну службу, що
тероризувала населення, сіючи жах і беззаконня (при цьому прості люди вірили,
що тонтон-макути - живі мерці, зомбі).  Схоже на те, що \emph{українські}
націоналісти почали активно переймати практику вуду (найбільше розповсюджену
саме на Гаїті).  Ляльку Путіна голками ще не тикають (покищо), але макет Кремля
вже спалили))), \emph{Украинские} националисты начали активно перенимать
практику вуду, Константин Бондаренко, strana.ua, 28.05.2021

Беларусь в ответ на введение запрета на авиасообщение, запретила поставки 95
бензина в \emph{Украину}. Теперь за тупость нашей власти как обычно заплатят
простые \emph{украинцы}, ведь повышение цен ударит именно по простым людям,
strana.ua, 28.05.2021

100 гривен в день за проезд в \emph{киевском} городском транспорте?  Это на
наших глазах становится суровой реальностью, Александр Карпец, strana.ua,
28.05.2021

\emph{Украина} - страна кривых зеркал и фискального каннибализма, Вячеслав
Черкашин, strana.ua, 28.05.2021

В \emph{Украине} явно готовятся схемы по массовой скупке земель, strana.ua,
29.05.2021

В \emph{Украину} возвращаются даже не девяностые, а Средневековье, Андрей
Манчук, strana.ua, 29.05.2021

Нет, это не девяностые - это гораздо глубже и хуже. \emph{Украинское} село
возвращается к архаическим корням, со всеми вытекающими отсюда последствиями,
которые будут все чаще напоминать о Средневековье, \textbf{В \emph{Украину}
возвращаются даже не девяностые, а Средневековье}, Андрей Манчук, strana.ua,
29.05.2021

Смелость, проявляемая в слабоумии и отваге, ведет к разрушению, Именно она
характерна для \emph{украинской} власти, Виктор Скаршевский, strana.ua, 29.05.2021

Зеленский в этой борьбе проиграет, потому что он абсолютно одинок в этом
эпическом противостоянии. Его никто не поддерживает в ТАКОЙ борьбе: ни Запад,
ни политические игроки, ни \emph{украинское} общество, \textbf{Порошенко -
живое олицетворение коррупции}, Андрей Головачев, strana.ua, 29.05.2021

Разложение силовой и правоохранительной системы ставит крест на каких либо
попытках реформировать \emph{Украину} в правовом и демократическом направлении, так
как приводит к узурпации власти Президентом, strana.ua, 29.05.2021

Позже \emph{украинца} обнаружили. Ему потребовалась медицинская помощь. Выяснилось,
что он долгое время провел на глубине 40 метров, а потом слишком резко всплыл
на поверхность. Это вызвало азотное отравление крови - Кессонную болезнь,
\textbf{Черный день на Красном море. Как в Египте погиб бывший глава
\emph{украинской} разведки и что о нем известно}, strana.ua, 28.05.2021

Сначала спасатели забили тревогу, когда \emph{украинец} вовремя не поднялся на
поверхность, а когда он появился, оказалось, что ему нужна медицинская помощь.
Спасти его не удалось.  Собрали подробности гибели экс-главы \emph{украинской}
разведки, а также факты из его биографии, strana.ua, 28.05.2021

Если посмотреть на карту \emph{Украины}, то, найдя на ней Днепр, можно увидетьи
один из его притоков — реку Десну.  Любой человек, хоть сколько-нибудь знакомый
с азами географии, скажет, что Десна — левый приток Днепра. Но слово «десна» на
древнерусском означает «правая». А такое выражение, как «сесть одесную
хозяина», означало буквально сесть справа от хозяина.  В современном
сербохорватском языке — одном из близких родственников русского — «десна» также
означает «правая». Так почему же тогда Десна является левым притоком Днепра?
Дело в том, что восточные славяне, осваивая и называя эту реку, двигались вверх
по Днепру, а при этом Десна действительно располагалась справа, \textbf{Когда
лево находится справа?}, vogrugsveta.ru

Усі вже, здається, помітили скандал щодо «\emph{української} російської мови»,
а дехто навіть встиг взяти в ньому участь. Багатьох обурила сама постановка
питання. Мовляв, якась \emph{українська} російська мова? В \emph{Україні} не
може бути нічого російського. Ці люди роблять вигляд, що в природі існують так
звані «\emph{українські} етнічні землі» – чітко окреслена плюс-мінус до
кілометра територія, де здавна мешкає автохтонне \emph{україномовне} й
\emph{українокультурне} населення, тобто етнічні \emph{українці}. Всі інші на
цій території є чужинцями або навіть ще гірше – чужинцями-колонізаторами,
\textbf{\emph{Українська} російська цибуля} Павло Зуб'юк, zaxid.net, 15.04.2021

\emph{Украина} напоминает фильм про Ивана Васильевича, когда Жорж Милославский
угнал народ на войну, чтобы те не поняли, что у власти самозванцы,
\textbf{Анекдот дня (за \emph{Украину})}, Крымский кот, zen.yandex.ru,
27.05.2021

\emph{Украина} объявила, что откроет Чернобыль для туристов. Говорят, это как
Диснейленд, только двухметровая мышь - настоящая!  \textbf{Анекдот дня (за
\emph{Украину})}, Крымский кот, zen.yandex.ru, 27.05.2021

Мне всегда нравилось \emph{украинское} народное творчество. Нежный, ласковый
язык.  Песни на \emph{украинском} - сама нежность. Звучат всегда душевно и с
особым трепетом.  Ну, конечно же, зависит от исполнителя.  Пелагея, исполняющая
\enquote{Цвiте терен} - лучшее, что могло произойти.  Хочу поблагодарить за
такой потрясающий дуэт: очень рада за Веронику Сыромля. Молодец, девочка,
ангельский голос. В тандеме с ее наставницей - звучит до глубины души.
Пересматриваю который раз.  Эту же песню они вдвоем исполняли на открытии
Крымского моста в 2018 году. Дуэт Пелагеи и Вероники просто великолепен, очень
гармоничен, голоса красиво перекликаются, одинаковы в исполнении по силе и
чистоте, \textbf{\emph{Украинская} песня в исполнении Пелагеи}, КИНОТЕРАПИЯ,
zen.yandex.ru, 27.05.2021

В Киеве считали, что самоопределятся крымчане обособленно не имеют права и
точка. Именно тогда лидер УНСовцев Корчинский, изрёк ту саму фразу: «Крым будет
\emph{украинским} или безлюдным» — устраивая марши устрашения в Севастополе и
других городах, \textbf{Аннексия на аннексию?}, Дмитрий Жук, zen.yandex.ru,
05.05.2021

Аналогічно й у випадку СРСР. Фактично після закінчення експеременту з
коренізацією в СРСР утворилася негласна ієрархія народів, у якій українці були
«другі серед рівних». Якщо точніше – \emph{українці} були майже росіянами. Звісно,
\emph{українська} мова і культура розвивалися суто в межах дозволеного державою
невидимого ґетта, але пересічний етнічний \emph{українець} мав кар'єрні перспективи не
набагато гірші, ніж етнічний росіянин, 
\textbf{\emph{Українська} російська цибуля} Павло Зуб'юк, zaxid.net, 15.04.2021

\enquote{Проверяли канал для \enquote{взрослого} оружия}. Кто и зачем вез в
\emph{Украину} через Румынию тысячи револьверов, strana.ua, 29.05.2021

А по поводу заворота самолёта \emph{украинскими} покровителями \enquote{Азова}
и Протасевичей, закрытия ими целых изданий и каналов, убийства Бузины и угроз
писавшим об ультраправых журналистах -- все молчат, \textbf{История с
Протасевичем объемно показывает, кто чего стоит}, Роман Подолян, strana.ua,
29.05.2021

Современная \emph{Украина} — вполне состоявшееся государство, причем
состоявшееся на антироссийской основе. Оно сумело внести раскол в народ
\emph{Украины} и ещё попортит нам много крови и по Донбассу, и по Крыму, и по
многим другим вопросам. И за всё это \emph{Украине} надо будет не только
предъявлять претензии – за всё это её надо будет наказывать, \textbf{«Грехи»
1990-х и борьба за \emph{Украину}: Киев не должен уйти от ответственности},
regnum.ru, 19.05.2021

У России на \emph{украинском} направлении есть два разных, хотя и
взаимосвязанных дела: защита интересов страны перед лицом враждебной политики
нынешнего \emph{украинского} государства и «борьба за \emph{Украину}», о
которой тоже сказал депутат Затулин, \textbf{«Грехи» 1990-х и борьба за
\emph{Украину}: Киев не должен уйти от ответственности}, regnum.ru, 19.05.2021

Что же до современной \emph{Украины}, так это вполне состоявшееся государство,
оно состоялось на антироссийской основе, оно сумело внести нужный ему раскол в
народ \emph{Украины}, в среду православных на \emph{Украине}, оно ещё очень
много попортит нам крови и по Донбассу, и по Крыму, и по отношениям с Европой,
США, Турцией, государствами на пространстве бывшего СССР, \textbf{«Грехи»
1990-х и борьба за \emph{Украину}: Киев не должен уйти от ответственности},
regnum.ru, 19.05.2021

По поводу такого традиционного блюда, как борщ, власти \emph{Украины} сломали
немало копий, пытаясь представить его исключительно своим национальным
достоянием.  \emph{Украинский} МИД положил немало сил и энергии на протесты,
чтобы заклеймить позором издания или зарубежные рестораны, назвавшие его
русским. Тем не менее рецептов этого блюда сотни, и они даже в традиционной
\emph{украинской} кухне существенно отличаются в зависимости от региона. Есть
борщ полтавский и черниговский, полесский и винницкий.

С этой мыслью парни мечутся уже добрые два года. Как сделать так, чтобы СМИ
писали о власти или хорошо, или ничего? Сама формулировка намекает, что власть
в \emph{Украине} не совсем жива, но вопрос стоит именно так, \textbf{Проект
антиолигархического закона нацелен на удушение СМИ}, Сергей Лямец, strana.ua,
29.05.2021


Все происходящее в \emph{украино}-российских и \emph{украино}-белорусских
отношениях часто не имеет смысла. Ну, то есть, рационально объяснить эту череду
выстрелов себе в ногу невозможно. Однако, на удивление, все становится на свои
места, если принять за аксиому, что цель – максимальный отрыв \emph{Украины} от
«русского мира», \textbf{\emph{Украину} хотят любой ценой оторвать от русского
мира}, Вячеслав Чечило, strana.ua, 29.05.2021

В \emph{Украине} идет бесконечная борьба с ветряными мельницами, Павел Себастьянович,
strana.ua, 29.05.2021

Клімкін: Американський батальйон біля Одеси був би гарантією безпеки \emph{України}, 
radiosvoboda.org, 29.05.2021

Столица отмечает день рождения. За свою тысячелетнюю историю город повидал
немало. В разное время Киевом правили Рюриковичи, татары, литовцы, поляки,
советы. Но все равно город остался истинно \emph{украинским}. Шарм улиц, уют
двориков и даже царь-балконы делают Киев неповторимым. Богатая история оставила
для нас многочисленные памятники архитектуры, культуры и природы, obozrevatel.com, 29.05.2021

Але чому в нашому суспільстві немає культу успішного бізнесмена, який створює
геніальний продукт? Доблесного військового, який героїчно захищає
\emph{Україну}?  Просунутого хіміка, який робить приголомшливі відкриття у
найсучаснішій лабораторії? Актора, кумира мільйонів, який заробляє небосяжні
суми? \textbf{Запитання від дитини}, Андрій Любка, day.kiev.ua, 28.05.2021

Брехню про участь Протасевича у боях на Донбасі як «бойовика-найманця» одним із
перших почав поширювати \emph{український} блогер Анатолій Шарій, надавши як
доказ обкладинку часопису «Чорне сонце» зі світлиною молодого усміхненого
чоловіка у формі батальйону «Азов» з автоматом, \textbf{Тріумф негідників},
day.kiev.ua, 27.05.2021

Сегодняшние реалии российско-\emph{украинских} отношений совсем не похожи на
содружество, а современная \emph{Украина} не совсем похожа на независимое
государство и внутренне расколота. В таких обстоятельствах договориться о
содружестве практически невозможно. \enquote{Переформатирование} \emph{Украины}
и её населения продолжается с большим усердием. Каковы же будут результаты,
каким будет наше будущее? - \textbf{Россия и \emph{Украина} - утраченное
Содружество}, Igor Novikov, zen.yandex.ru, 29.05.2021

И новая киевская власть, и местные \emph{украинские} силовики заняли тогда
странную выжидательную позицию. Чего же они ждали? Фактически местным и
приезжим революционерам предоставлялась свобода действий по принуждению всех
несогласных с ними крымчан к безоговорочному повиновению, любыми способами.
Военнослужащие \emph{Украины}, обязанные защищать \emph{украинских} граждан,
расслабленно наблюдали за происходящим из-за ограды своих расположений. Вот в
таком состоянии их и застали внезапно появившиеся россияне... \textbf{Как
\emph{Украина} свой Крым не защитила}, Igor Novikov, zen.yandex.ru, 18.03.2021

И почему-то именно жители современной \emph{Украины} умудряются преподносить
сюрпризы.  Помните такую группу как «Вопли Видоплясова»? Должны бы помнить. Как
минимум у них был один большой хит, который крутили везде. Называлась та песня
«Весна» и в своё время она изрядно пошумела. Хотя поколение постарше может
вспомнить и песню «Танцы». Собственно, на мой взгляд «ВВ» первая группа,
которая популяризировала \emph{украинский} язык на просторах России. Хотя
играть они и вовсе начинали ещё до того как \emph{Украина} с Россией отделились
границами, \textbf{Олег Скрипка. Человек, забывший родину}, soullaway
soullaway, zen.yandex.ru, 24.05.2021

«\emph{Украинцы} ждут неизбежного — когда \emph{Украина} распадется», Аннотация: В сети
обсудили заявление депутата \emph{украинского} парламента Нестора Шуфрича о том, что
за все санкции \emph{украинской} власти в отношении Белоруссии поплатится простой
\emph{украинский} народ, regnum.ru, 29.05.2021

\emph{Український} Стоунхендж. На Дніпропетровщині поряд із майбутньою забудовою
знайшли стародавній кромлех: що з ним буде? radiosvoboda.org, 29.05.2021

Так что \emph{украинцам} просто необходимо учиться быть гражданами одной
страны, \textbf{Восстановить единство \emph{украинской} нации будет непросто},
Олесь Доний, glavred.info, 23.11.2019

\emph{Украинское} общество взбудоражили слова Сивохо о том, что нужно просить
прощение у жителей Донбасса. Данное утверждение вполне могло понравиться части
людей, живущих на оккупированных территориях, а также части тех, кто покинул
Донбасс, как вынужденный переселенец, но не получил ожидаемого тепла и внимания
со стороны государства, \textbf{Восстановить единство \emph{украинской} нации
будет непросто}, Олесь Доний, glavred.info, 23.11.2019

Эта истина, которая для кого-то звучит отвлечённо, для нас, живущих на
\emph{Украине}, составляет часть нашей жизни. \emph{Украинский} церковный
раскол, углублённый и благословлённый патриархом Варфоломеем в 2018 году
продолжает приносить горькие, преступные плоды, \textbf{Канонический бунт
Фанара: время мягких решений церкви заканчивается}, odnarodyna.org, 26.05.2021

Поёшь на русском? Ты не \emph{украинец}! \emph{Украинский} музыкант, фронтмен
известной группы «Вопли Видоплясова» Олег Скрипка назвал артистов
\emph{Украины}, которые поют песни на русском языке «не \emph{украинцами}».
Настоящие \emph{украинские} певцы – это те, кто, во-первых, говорят, пишут
песни и поют на мове, а, во-вторых, не ездят в Россию. Ты можешь сколько угодно
любить свою страну \emph{Украину}, но Олег Скрипка не будет тебя считать
\emph{украинским} певцом. О музыкальных талантах, кстати, речь не идёт,
\textbf{Олег Скрипка: \emph{украинский} певец не может петь по-русски},
odnarodyna.org, 29.05.2021

Голая девушка попала в кадр в прямом эфире \emph{украинского} канала во время
включения редактора \enquote{Дождя}. Видео, strana.ua, 29.05.2021

«Кремль спотворює історію регіону, щоб легітимізувати ідею про те, що
\emph{Україна} і Білорусь є частиною «природної» сфери впливу Росії, –
розвінчують путінський міф в Chatham House. – Історично неправильно
стверджувати, що Росія, \emph{Україна} і Білорусь коли-небудь складали єдине
національне утворення. Останні дві країни насправді мають політичні та
культурні корені в європейських за своєю суттю структурах, таких як Велике
князівство Литовське», radiosvoboda.org, 30.05.2021

Верю, что миро-творческий и воссоединительный процессы в \emph{Украине} еще
впереди. И нынешний дурдом, замешанный на травмах войны, страхах, эгоизме,
близорукости и безответственности, мы переживем достойно, \textbf{Мир и
воссоединение с Донбассом еще впереди}, Андрей Ермолаев, strana.ua, 30.05.2021

Биткоин потребляет в год больше электроэнергии, чем \emph{Украина}, Анатолий
Амелин, strana.ua, 30.05.2021

У добу Русі-\emph{України}, за середньовіччя графіті сприймали як безпосереднє
звернення до святих і до Всевишнього по допомогу, radiosvoboda.org

Вот такая сегодня у нас тяжелейшая ситуация не только на долговом рынке, а и в
целом в экономике \emph{Украины}. И надо честно признать: благополучие рядовых
налогоплательщиков тает буквально на глазах, Александр Гончаров, strana.ua,
30.05.2021

Михайло Слабошпицький – член Національної спілки письменників \emph{України},
лауреат національної премії імені Тараса Шевченка (2005), за роман-біографію
«Поет із пекла». Серед його робіт – документальна, публіцистична та біографічна
проза, а також численні твори для дітей. Його син Мирослав є відомим
\emph{українським} кінорежисером, дочка Іванна – журналісткою,
radiosvoboda.org, 30.05.2021

\emph{Украинский} сериал: почему у них каша в голове, Сериал \enquote{Папик}
наглядно показывает, почему у \emph{украинцев} каша в голове. И это не вопрос -
это утверждение, ЗВЕЗДУЛЬКИ, zen.yandex.ru, 01.05.2021

Однако, слов из песни не выкинешь и мы, русские, глядя на \emph{Украину} и
\emph{украинцев} всё чаще приходим к выводу, что там люди словно с ума
посходили, ЗВЕЗДУЛЬКИ, zen.yandex.ru, 01.05.2021

Некоторые в счастливом экстазе ежегодно празднуют день независимости
\emph{Украины}, другие воспринимают этот день как личную и общественную
трагедию, лишившую их очень многого, \textbf{О независимости \emph{Украины}:
как не обмануться в мире обмана}, ukraina.ru, 29.05.2021

Це европа, безвиз, безгаз, безбензин, безвакцин, беззавод, безземель, безнадёг,
бездонбасс, безкрым, безбожие...\footnote{Украина} Sergey Stus, комментарий,
\textbf{Потерять яйца в сражении с Беларусью}, Анатолий Шарий, youtube,
30.05.2021 

Живу в \emph{Украине}, рад искренне за Бацьку который не зассал и не дал
заднюю, и за санкции в нашу сторону, тем больше \emph{украинцев} прозреет от
ничтожности нашей конченной власти, и быстрее мы их сметем, Дмитрий СМ,
комментарий, \textbf{Потерять яйца в сражении с Беларусью}, Анатолий Шарий,
youtube, 30.05.2021

Я видел много идиотов, и слышал разных чудаков. Но столько сколько \enquote{в}
\emph{Украине}, ещё не видел дураков. В долгах погрязнув как в болоте,
обворовав самих себя. Вопят, орут всё о свободе, в своих грехах других виня!
комментарий, \textbf{Потерять яйца в сражении с Беларусью}, Анатолий Шарий,
youtube, 30.05.2021

Белорусы, если вы читаете мой коммент знайте: мы адекватные \emph{украинцы}
против этого гребанного маразма который творит наша вонючая, наркоманская
власть!  комментарий, \textbf{Потерять яйца в сражении с Беларусью}, Анатолий
Шарий, youtube, 30.05.2021

Только великие мастера школы \emph{Майданлинь} способны ударить сами себя по
яйцам пяткой, комментарий, \textbf{Потерять яйца в сражении с Беларусью},
Анатолий Шарий, youtube, 30.05.2021

\emph{Украинская} власть давно не имеет яиц, у нее только рабочая дупа,
комментарий, \textbf{Потерять яйца в сражении с Беларусью}, Анатолий Шарий,
youtube, 30.05.2021

Что такое жизнь под \emph{украинскими} обстрелами? Это когда вечером 1 июня, в
День защиты детей, у мемориалов погибшим детям Донбасса в небо взмывают сотни
бумажных фонарей, чтобы осветить путь ангелам. Ведь малышам, запускающим
фонарики, сложно объяснить, почему у этих ангелов отняли их короткую жизнь,
лишив возможности повзрослеть и увидеть мир на нашей Родине. Теперь они могут
только наблюдать с небес и плакать вместе со взрослыми, утешая их, Фаина
Савенкова, facebook.com, 30.05.2021

На \enquote{двери} расположен QR-код, который при переходе ведет на фото с
обрисованным ранее Офисом президента и надписью \enquote{Президент
\emph{Украины} лох}, strana.ua, 30.05.2021

\emph{Украине} необходима деконструкция национального мифа – чтобы строить свое
будущее не на архаической полотняной сорочке и не на «исконно
\emph{украинском}» борще – а на развитии общедоступного образования, науки и
высокотехничного производства, \textbf{Косоворотка и вышиванка должны сближать,
а не разделять народы}, Андрей Манчук, strana.ua, 30.05.2021

«Страна»: популярне і проросійське медіа в \emph{Україні}, radiosvoboda.org,
30.05.2021

Новинний вебсайт «Страна.ua» – серед найпопулярніших медіа в \emph{Україні},
radiosvoboda.org, 30.05.2021

Співаючий далекобійник. Як \emph{український} хорист став противником Путіна,
зіркою YouTube та потрапив в Канни, pravda.com.ua, 28.05.2021

В'ятрович попереджає про наміри влади скасувати норми про \emph{українську} мову
фільмів і преси, umoloda.kyiv.ua, 29.05.2021

\enquote{А от для Зеленського і проросійських олігархів – власників телеканалів
– це замах на \enquote{святе}. Адже серіальчики і фільми в телевізорі мають
бути на общєпонятном язикє, а \emph{українська} мова, цитуючи скандальну
продюсерку \enquote{1+1} годится только для комедий}, - додає обранець,
umoloda.kyiv.ua, 29.05.2021

На його думку, \enquote{Зеленський і його слуги вирішили реалізувати одну з
головних цілей Москви у гібридній війні проти \emph{України} – забетонувати
русифікацію \emph{українського} медіапростору}.  \enquote{Кожен, хто голосуватиме за
ці анти\emph{українські} закони, відверто працює на Кремль, якими б байками про
\enquote{збитки від \emph{української} мови} це не прикривалося. Наш обов'язок –
зупинити цей нахабний російський реванш}, – завершив В'ятрович,
umoloda.kyiv.ua, 29.05.2021

\emph{Украине} снова спокойно не спится. Подавай им теперь изменения в правилах
голосования на Евровидении. Естественно дело в России. На \emph{Украине}
вообще нет других причин как наша страна и еë граждане.  В этот раз их тонкая
натура не может смириться с тем фактом, что на международном конкурсе они
вынуждены оценивать участницу из страны-агрессора. Им это кажется странным,
\textbf{Евровидение для \emph{Украины} не просто конкурс}, Мила Белая,
zen.yandex.ru, 24.05.2021

Между прочим, политика политикой, а выступление \emph{Украины} мне понравилось.
Уж точно больше Манижи. Интересная яркая исполнительница, хороший голос,
постановка номера интересная, музыка запоминающаяся.  \enquote{Я считаю
аморальным оценивать что-либо и кого-либо, кто представляет страну-агрессора!
Для меня лично агрессор всегда на последнем месте независимо от вида
соревнования!} – член \emph{украинского} жюри Игорь Кондратюк. А как же
искусство и спорт вне политики? Разве такие мероприятия не призваны
продемонстрировать единство и перемирие. Для чего они ещё созданы?
\textbf{Евровидение для \emph{Украины} не просто конкурс}, Мила Белая,
zen.yandex.ru, 24.05.2021

Є \emph{українці}, які чекають \enquote{Спутник V}. Що з ними робити – ребус
для Ляшка, Сергій Грабовський, gazeta.ua, 27.05.2021

А є ж іще у YouTube т.зв. «Перший козацький»; як слушно зауважив один
журналіст, «Телеканал «Інтер» у порівнянні з «Першим козацьким» – це просто
вісник \emph{українського} націоналіста»... Звичайно, чинна влада не господарює у
YouTube. Але виникають запитання. Скажімо: чи звертався Офіс президента чи
якась інша інстанція до адміністрації YouTube з проханням прикрити «Перший
незалежний» чи бодай припинити його трансляцію під марками трьох уже
заблокованих для \emph{українського} користувача телеканалів? Адже, як на мене,
йдеться про пряме знущання не лише з президента Зеленського та РНБО, а й із
\emph{Української} держави. Чи, може, зверталися, проте якось ніяково, без
наполегливості? \textbf{«Заблоковані» телеканали та антидержавна пропаганда}, day.kiev.ua, 18.05.2021

Широкий і вольний був їм шлях на \emph{Україну}. Літня спека застелила його на
долоню курявою. Сонце пекло з гарячого неба. Курява посіла на семінаристів,
обліпила їм лиця так, що вони не впізнавали один одного. Піт котився з їх
потьоками і, помочивши чорну куряву, пописав їх лиця довгими смужками,
\textbf{Хмари}, Іван Нечуй-Левицький

Сніг за шиворот, в Карпати ми летим, На літній резині і майже без бензини, І
весело, бо я тут не один — На лижі їде ціла \emph{Україна}!  \textbf{Буковель},
Кузьма Скрябін, Повне зібрання творів, Харків, 2019

Але, як і революціонери, що використовували нову форму Інтернету для об'єднання
та усунення ворога, Росія тепер використовувала мережі, щоб розірвати
\emph{Україну} на частини, \textbf{Війна лайків. Зброя в руках соціальних
мереж}, П. В.  Сінґер, Емерсон Т. Брукінґ, Харків, 2019

Важливим прецедентом цього стала \emph{Україна}. Кількість негативних
російськомовних новин про \emph{Україну} зросла вдвічі, а потім утричі. Етнічні
росіяни всередині \emph{України} невдовзі збурилися проти активістів, що
скинули проросійський уряд, \textbf{Війна лайків. Зброя в руках соціальних
мереж}, П. В. Сінґер, Емерсон Т. Брукінґ, Харків, 2019

Зеленский создает в \emph{Украине} \enquote{университет будущего} для \enquote{людей
будущего}, strana.ua, 01.06.2021

Зеленский предложил превратить \emph{украинские} киностудии в музеи, strana.ua,
28.05.2021

Рост ВВП в \emph{Украине} за последние десятилетия унизительно низок,
strana.ua, 31.05.2021

В \emph{украинском} магазине продают книгу участника ОУН о том, почему евреев следует
называть ж@дами, strana.ua, 07.02.2021

А мне вот даже чуточку жаль авторов подобной макулатуры. Ведь чушь про «дело
своего народа» они выдают на-гора искренне. И сами рады бы послужить, да только
народа у них нет. Немецкий этим недугом давно переболел и на «расово верных»
бумагомарателей из \emph{Украины} глядит с отвращением. Как на прокаженных
глядит и ничего общего с такими иметь не желает, \textbf{Несчастные мерзавцы,
эти авторы книжек во славу героев СС}, Максим Могильницкий, strana.ua,
31.05.2021

Но \emph{украинский} пример говорит об обратном. Страна, в которой господствует
идеология \emph{украинского} национализма (идеология же), катится в пропасть с
той же скоростью, с которой Россия рвётся к звёздам, \textbf{Непойманный
украинский крокодил}, Ростислав Ищенко, ukraina.ru, 31.05.2021

Часть, даже очень важная, не может жить и развиваться без целого. А целое, даже
без очень важной части, может функционировать, зачастую не менее эффективно,
чем с ней. \emph{Украина} была очень важной частью России, не менее, а
возможно, и более важной, чем ноги для лётчика-истребителя. И на этом основании
решила, что она способна одна не просто выжить, но жить лучше, чем вместе,
\textbf{Непойманный украинский крокодил}, Ростислав Ищенко, ukraina.ru,
31.05.2021

Вы, звезда, Анатолий. Главный человек в \emph{Украине}. Пиарят, вас, столько говорят,
не забывают. Это здорово. Мы с вами. Шарий супер!
комментарий, \textbf{Опять Шария достали. Что дальше?} Анатолий Шарий, youtube.com, 31.05.2021

Гончаренку снова пошел в одно место. Удачи Шарийцам, потому что с этим
беспределом нужно заканчивать.  Я переживаю за нормальных \emph{украинцев}, они ни в
чем не виноваты,
комментарий, \textbf{Опять Шария достали. Что дальше?} Анатолий Шарий, youtube.com, 31.05.2021

Мне вот интересно: а здоровые в \emph{украинском} политикуме остались?
комментарий, \textbf{Опять Шария достали. Что дальше?} Анатолий Шарий, youtube.com, 31.05.2021

\emph{Украина} - равняйся на Шария! Анатолий, ты ЗНАМЕНИТ на Родине,
комментарий, \textbf{Опять Шария достали. Что дальше?} Анатолий Шарий, youtube.com, 31.05.2021

\begin{itemize}
\item Мы отключили Крыму свет и воду, чтобы вернуть доверие крымчан. 
\item Мы обстреливаем Донбасс, чтобы они увидели, что \emph{Украина} за мир. 
\item Мы закрываем телеканалы, чтобы все знали,что у нас свобода слова. 
\item Мы проводим факельные шествия со свастикой на знаменах, чтобы все увидели, что фашистов у нас нет.
\item Мы не сажаем убийц, чтобы все знали, что на \emph{Украине} есть правосудие.
\item Мы запретили русский язык и школы, чтобы все увидели, что \emph{Украина} едина.
\item Мы узурпировали власть, чтобы все увидели что Янукович узурпатор.
\end{itemize}
комментарий,  Olga Vpalto, \textbf{Опять Шария достали. Что дальше?} Анатолий Шарий, youtube.com, 31.05.2021

а еще - голодомор - это геноцид \emph{украинцев}, а 400 трупов в день из-за \enquote{ми ведемо
перемовини} - это норма. Ты, сука, должен любить \emph{Украину}! Иначе, чемодан,
вокзал... Польша,
комментарий, Александр, \textbf{Опять Шария достали. Что дальше?} Анатолий Шарий, youtube.com, 31.05.2021

\emph{Украинцы} вас всех устраивает происходящие в стране?? Может пора импичмент?
комментарий, \textbf{Опять Шария достали. Что дальше?} Анатолий Шарий, youtube.com, 31.05.2021

Представьте, что Анатолий, нелегально проехался по \emph{Украине}, наснимал роликов и все это выложил в сети. Кондрашка хватила бы многих:))),
комментарий, \textbf{Опять Шария достали. Что дальше?} Анатолий Шарий, youtube.com, 31.05.2021

Та поезда похоже \emph{Украине}, если действительно за зе 20\% проголосовали бы...... Я
думаю за 3года зелька не то еще придумает, как собственный народ обобрать,
Тимошенко Юлька уже что-то там проорала, про то шо пенсии хотят отобрать....
Нууу терпите дальше, молодцы \emph{Украины}, в правильном направлении движитесь,
комментарий, \textbf{Опять Шария достали. Что дальше?} Анатолий Шарий, youtube.com, 31.05.2021

Толик, мы ждем тебя дома, в \emph{Украине}! Хватить любить Родину в Европе,
комментарий, \textbf{Опять Шария достали. Что дальше?} Анатолий Шарий, youtube.com, 31.05.2021

Практичний досвід \emph{України} у боротьбі із систематичними російськими
пропагандистськими та дезінформаційними наративами бере свій початок у 2013
році, і за цей час \emph{українські} експерти, неурядові та державні установи
розробили низку інструментів та стратегій для подолання цих викликів,
radiosvoboda.org, \textbf{Російська підривна діяльність, Білорусь та Крим –
серед тем другої зустрічі \emph{Українсько}–чеського форуму}, 31.05.2021

Чеські та \emph{українські} історики працюють разом, щоб зробити доступними
\emph{українські} архіви. Наприклад, чеські історики отримали доступ до
\emph{українських} архівів КҐБ (на відміну від подібних архівів у Росії), що
дозволило виявити нові та невідомі факти про те, що сталося в Чехословаччині
після 1945 року, наприклад, з громадянами Чехословаччини, викраденими в СРСР, 
radiosvoboda.org, \textbf{Російська підривна діяльність, Білорусь та Крим –
серед тем другої зустрічі \emph{Українсько}–чеського форуму}, 31.05.2021

Пенсионерка в Первомайске подорвалась на \emph{украинском} снаряде во время работы на огороде,
miaistok.su, 31.05.2021

Опять рост после снижения. Новых зараженных в \emph{Украине} за сутки стало
больше на 2 137 человек, strana.ua, 01.06.2021

Нас плавно превращают в сырьевую колонию без науки, технологий и индустрии, Так
происходит мягкий захват Украины, Александр Гончаров, strana.ua, 01.06.2021

\enquote{Бандеровец} - \emph{украинский} нацистский коллаборационист. Как
Netflix перевёл цитаты из \enquote{Брата} и \enquote{Брата-2}, strana.ua,
01.06.2021

Выкладываю кусок стенограммы нашей беседы с Вадим Аристов. Он дал яркую и
доступную картинку о том, что врачи уже поняли о ковиде и почему \emph{Украине} грозят
локдауны до 2023 года, \textbf{\emph{Украине} нужно понять, что ей грозят локдауны до 2023 года}, strana.ua, 01.06.2021

Интервью Зеленского немецкому изданию - жалобная просьба о помощи, У него все
виноваты, и только \emph{Украина} стеной стоит за Европу! ... все в мире должны
защищать \emph{Украину}, но они какие-то хрупкие (ЕС, НАТО), не вооружают
\emph{Украину}, не отказываются от Северного потока-2 и даже не признают Россию
стороной конфликта ... \emph{Украина} защищает, \emph{Украина} – стена
безопасности этой философии существования, европейской цивилизации, прав и
свобод человека;,  Виктор Скаршевский, strana.ua, 01.06.2021

Третьи в мире по детской порнографии и рост изнасилований. Как в \emph{Украине}
\enquote{защищают детей}, strana.ua, 01.06.2021

Сотни тысяч для главы набсовета, 8 тысяч - для инженера. Какие зарплаты платят
на \enquote{\emph{Укр}зализныце}, strana.ua, 01.06.2021

Нажали на кулёк. Как в \emph{Украине} запретили пластиковые пакеты и кого за
них оштрафуют, strana.ua, 01.06.2021

За предыдущие несколько лет команда Нафтогаза добилась почти невозможного -
значительной отсрочки достройки \enquote{Северного потока-2}. Однако до победы еще
далеко и бой \emph{Украины} против российской геополитической трубы не завершен,
\textbf{Галантерейщик и кардинал. Сцена третья.  Апофеоз абсурда}, strana.ua, Валентин Землянский, 01.06.2021

В \emph{Україні}, станом на 2019 рік, діяли 282 вищих навчальних заклади, де здобували освіту 1 300 000 студентів, 
\textbf{Щодо вчорашнього указу про \enquote{Президентський університет}}, Костянтин Матвієнко, pravda.com.ua, 01.06.2021

Адже молодь залишає \emph{Україну} не лише тому, що їй нерідко пропонують за кордоном
якісніше і безкоштовне навчання, але і кращі життєві перспективи, що
ґрунтуються на ліпших за \emph{українські} суспільно-державних взаєминах.  Наразі, ані
Академія Наук, ані окремі визначні \emph{українські} вчені не наважилися прямо вказати
на волюнтаристський та безперспективний підхід у питанні заснування нового
університету. Може б президенту було б правильним, перш, ніж підписувати явно
сирий Указ, зустрітися з науковцями задля консультацій?
\textbf{Щодо вчорашнього указу про \enquote{Президентський університет}}, Костянтин Матвієнко, pravda.com.ua, 01.06.2021

Ось вже два роки Зелеенський є Президентом \emph{України}...і пробуде ще принаймі 3
роки...Бо вірогідно ще виберуть його ЛЮДИ на 5 років... Погані, скажете,
люди... А Ви що краще? Хто з критиків має суспільний високий рейтинг довіри в
поваги в суспільстві. Ніхто в т.ч і К Матвієнко...Чому критика? Якби то були
комуністи сказав, що ідеологія...А так або жаба даве або втратили гроші або
заробляють гроші, як Маітвієнко... Щодо РІШЕННЯ Президента... Має право...Хто
хоче теж мати право, то йдіть на вибори - хай Вас виберуть...І будете свій ВИШ
влаштовувати...або виливати золотий батон...
коментар, Віктор Совщак, \textbf{Щодо вчорашнього указу про \enquote{Президентський університет}}, 
Костянтин Матвієнко, pravda.com.ua, 01.06.2021

\enquote{Революція вихідного дня}, як іронічно називають численні протести у Російській
Федерації, позатим викликала великий інтерес в \emph{українському} експертному
середовищі, та й у суспільстві загалом. Це перед усім зумовлено війною, яку
росіяни розв'язали проти \emph{України} 7 років тому. Зміна влади у країні-агресорі
теоретично може мати своїм наслідком припинення війни і відновлення
територіальної цілісності нашої країни у її законних кордонах, визнаних
світовим співтовариством. Однак, зараз є цілком доречним згадати вираз
Володимира Винниченка про те, що \enquote{вся російська демократія, включно з
опозицією, закінчується на \emph{українському} питанні}. Тому за територіальну
цілісність України та невтручання росіян в \emph{українські} справи ще доведеться
поборотися, як з нинішньою, так і з наступною російською владою,
\textbf{Знову \enquote{Україна не \emph{Росія}}, проте багатьом дуже хочеться стерти цю відмінність},
Костянтин Матвієнко, pravda.com.ua, 02.02.2021

Докорінною відмінністю \emph{українських} революцій від, того, що нині бачимо у
сусідів, полягає у неодмінній наявності парламентської опозиції, яка мала
високий рівень політичної та організаційної спроможності, доступ до ЗМІ, та, що
дуже важливо, була наділена парламентським імунітетом, без чого ніякі намети на
Майдані Незалежності встановити було б неможливо,
\textbf{Знову \enquote{Україна не \emph{Росія}}, проте багатьом дуже хочеться стерти цю відмінність},
Костянтин Матвієнко, pravda.com.ua, 02.02.2021

Хотите разобраться с этим вопросом?:) Тогда сначала нужно определение война
кого с кем? Война России с \emph{Украиной}? нет, ее нет. И опять же, нужно
определиться с тем, что вкладывается в \enquote{\emph{Украина}} и
\enquote{Россия}. Если под \enquote{\emph{Украина}} понимать текущую и
предыдущую власть (полностью управляемую из-за океана), то война есть. А если
страну, людей, и в моем понимании будущее этих людей, то ее не может быть в
принципе.  Схематично так – власть захвачена иностранной державой, которая
развязала конфликт с Россией (текущей властью) с целью ее ослабления и
перемещения поближе границ НАТО.  Какова должна быть позиция гражданина
\emph{Украины}? углублять конфликт, положившись на дядю Сэма? или подумать а во имя
чего граждан власть посылает умирать? вспомните Вьетнам. И нашу олигархию:)
коментар, Mike\_Kharkov, \textbf{Знову \enquote{Україна не \emph{Росія}}, проте багатьом дуже хочеться стерти цю відмінність},
Костянтин Матвієнко, pravda.com.ua, 02.02.2021

\begin{itemize}
\item Как-то жалко такое читать, с обязательными \enquote{решениями съезда} в
        предисловии (\enquote{зумовлено війною, яку росіяни розв'язали проти України 7
        років тому}), клеймлением империалистических агрессоров и их
        пособников, о роли партии и лично (ЮВТ), о великом народе (\enquote{бачимо
        кардинально вищий рівень зрілості тодішнього українського суспільства у
        порівнянні з нинішнім російським}) и не прикрытое ничем тут видим, а
        тут не видим.
\item Можно и покороче – \enquote{Відіграв свою важливу роль доброзичливий до
        революціонерів нейтралітет олігархів – мобільні оператори не вимикали
        телефонного зв'язку та інтернету в київському середмісті в ході
        революцій, так само працювали банкомати та інші сервіси.} Этого
        достаточно, и для полной честности бы добавить про сговор олигархии с
        иностр. державами для взаимной пользы от доения населения, названный
        революциями. И опять ассоциации с пафосом 1917. Не было штурма Зимнего,
        ребята, не было. Но опять пишут учебники.
\end{itemize}
коментар, Mike\_Kharkov, \textbf{Знову \enquote{Україна не \emph{Росія}}, проте багатьом дуже хочеться стерти цю відмінність},
Костянтин Матвієнко, pravda.com.ua, 02.02.2021

Що чекає на охорону здоров'я в \emph{Україні} у 2018 році, або як з'їсти слона? 
Тимофій Бадіков, pravda.com.ua, 11.01.2018

Із початком нового року в \emph{Україні} офіційно розпочалася медична реформа.
Напередодні законодавчий старт змінам в охороні здоров'я дали Президент і Уряд.
При цьому, всі ми добре розуміємо, що \emph{українська} медицина не зміниться за одну
мить, тільки за підписами посадових осіб. Зміни в системі охорони здоров'я
\emph{України} – це поетапний комплекс дій та завдань для всіх сторін. Від
злагодженості їхньої роботи, від наявності стратегії та вміння реагувати на
виклики, та, насамперед, від нас самих – громадськості, залежить, чи досягне
\emph{українська} медицина європейського рівня, чи продовжить падіння у прірву,
Що чекає на охорону здоров'я в \emph{Україні} у 2018 році, або як з'їсти слона? 
Тимофій Бадіков, pravda.com.ua, 11.01.2018

Для \emph{української} системи охорони здоров'я 2018 рік обіцяє бути цікавим та насиченим подіями,
\textbf{Що чекає на охорону здоров'я в \emph{Україні} у 2018 році, або як з'їсти слона?} 
Тимофій Бадіков, pravda.com.ua, 11.01.2018

Цілком у тренді явного тепер союзу політиків з колишньої Партії регіонів та
Блоку Петра Порошенка і Народного фронту є вчорашнє рішення Шевченківського
районного суду Києва, яким відмовлено юридично встановити факт агресії Росії
проти \emph{України},
\textbf{Роби те, що слід, і нехай станеться те, що статися має!}
Костянтин Матвієнко, pravda.com.ua, 13.05.2016

\begin{itemize}
\item Минуло понад 30 років відтоді, як \emph{українська} стала державною, та
        тиждень від запровадження обов'язкової \emph{української} мови у сфері
        послуг. Ресторани, спортзали та магазини по всій країні навіть у
        соцмережах починають писати рекламні тексти \emph{українською}. Вата ж
        – іще більше горлати по Медведчуківським каналам про
        \enquote{нелюдські} утиски \enquote{корінного} російськомовного
        населення.
\item Які проблеми сьогодні існують у російської мови? Ніяких. Взагалі. Мені їх навіть вигадати складно, щоб звучало правдоподібно.
\item Які проблеми сьогодні існують в \emph{української} мови? Чимало, як на
державну. Досі у нас немає якісного культурного продукту, який збирав
би більше аудиторії, ніж російський.
\end{itemize}
\textbf{Чи є різниця, якою мовою говорити?} Андрій Білецький, pravda.com.ua, 22.01.2021

Вата постійно говорить, що захищати \emph{українську} не потрібно тому, що російська
мова з'явилася внаслідок \enquote{конкуренції}.  Ніби \emph{українці} самі
\enquote{обрали} її як свою рідну, як президента-клоуна у 2019 році. А тепер їй
для чогось потрібен \enquote{захист}.  Та пригадаймо шкільний курс історії –
російську на \emph{Українських} землях захищали найбільше, знищуючи
\enquote{конкурентів}. Взяти хоча б Валуєвський Циркуляр та Емський Указ, які
забороняли \emph{українську} мову та дозволяли знищувати тих, хто її
популяризував. Перераховувати усі 134 задокументовані факти утисків
\emph{української} мови немає потреби. Їх легко можна знайти у шкільних
підручниках з літератури та історії \emph{України},
\textbf{Чи є різниця, якою мовою говорити?} Андрій Білецький, pravda.com.ua, 22.01.2021

Що ж робити тим, хто любить \emph{Україну}, але все життя говорить російською? Все
просто – вчіть державну мову. За 30 років Незалежності це можна було зробити
без проблеми. Звичайна людина може за рік навчитися вільно говорити будь-якою
мовою. За півроку – якщо живе у мовному середовищі.  Чи є різниця, якою мовою
говорити? Так, є. Мова – це природній та культурний кордон. Він окреслює ареал
проживання народу. Він єднає нас із попередніми поколіннями \emph{українців}. Це – те,
що треба захищати завжди,
\textbf{Чи є різниця, якою мовою говорити?} Андрій Білецький, pravda.com.ua, 22.01.2021

Похоже, что приходит конец \emph{рАсейской} космонавтике. Дырка в МКС,
кончилась жратва, а потом и туалет сломался. Все признаки, что пипец не за
горами.  А в то время, пока \emph{Раша} заклеивает скотчем дырки на МКС и
просит у американцев еду, Украина развивает космические технологии:
\enquote{Американская компания-разработчик ракетно-космической техники Firefly
Aerospace Inc., собственником которой является украинский бизнесмен Максим
Поляков, успешно провела предполетные испытания своей первой ракеты Alpha.} (С)
коментар, \textbf{Чи є різниця, якою мовою говорити?} Андрій Білецький, pravda.com.ua, 22.01.2021

Чим більше людей буде розмовляти \emph{українською}, тим швидше ми проженемо окупанта,
тому хто не згоден нехай їдуть геть з країни, а якщо хочуть жити та розвиватись
тут, то будьте ласкаві спількуйтеся державною мовою,
коментар, \textbf{Чи є різниця, якою мовою говорити?} Андрій Білецький, pravda.com.ua, 22.01.2021

Ви для початку поясніть це своїм активістам Нацкорпусу, які хіба що крім
західної \emph{України} переважно російськомовні. А то якісь подвійні стандарти...
коментар, \textbf{Чи є різниця, якою мовою говорити?} Андрій Білецький, pravda.com.ua, 22.01.2021

Кто убил 17 миллионов \emph{украинцев}?, Юрий Гуленок, hvylya.net, 01.06.2021

Кто виноват в том, что миллионы \emph{украинцев} умерли преждевременно, миллионы не
родились, миллионы батрачат вдали от семей, надеясь уехать из этой несчастной
страны навсегда?  Ответ общеизвестен: во всех наших бедах виноваты те
избиратели, которые каждый раз избирают во власть воров, негодяев, невежд и
клоунов,
\textbf{Кто убил 17 миллионов \emph{украинцев}?}, Юрий Гуленок, hvylya.net, 01.06.2021

\emph{Украина} вчера стала третьей в Европе по смертям от коронавируса, Елена Вьюн, strana.ua, 02.06.2021

Ровно семь лет назад \emph{Украина} навсегда потеряла Донбасс, - Помните женщин с оторванными ногами у здания Луганской ОГА? 
Андрей Головачев, strana.ua, 02.06.2021

Когда я посмотрел кадры женщин с оторванными ногами, ползающих в лужах своей
крови, перед входом в здание Луганской ОДА после авиационного удара \emph{украинских}
ВВС я написал следующие слова: \enquote{Все! Донбасс ушел. Забудьте! Сейчас уже
нет силы, которая могла бы возвратить Донбасс} К сожалению, по прошествии 7 лет
я вынужден признать, что мой прогноз оказался верным,
\textbf{Ровно семь лет назад \emph{Украина} навсегда потеряла Донбасс, - Помните женщин с оторванными ногами у здания Луганской ОГА?} 
Андрей Головачев, strana.ua, 02.06.2021

Через 10 лет население \emph{Украины} составит не больше одной сотой населения Китая,
Зато все мы будем в вышиванках, Юрий Касьянов, strana.ua, 02.06.2021

Население Китая сегодня - 1,4 млрд человек, население Земли - 7,9 млрд человек.
Население \emph{Украины} - 0,04 млрд человек, \textbf{Через 10 лет население
\emph{Украины} составит не больше одной сотой населения Китая, Зато все мы
будем в вышиванках}, Юрий Касьянов, strana.ua, 02.06.2021

Якщо ти живеш і працюєш в \emph{Україні}, є \emph{українцем}, поважаєш традиції та інших
людей, то ти за замовчуванням маєш бути патріотом. І говорити про виховання
патріотів – це нонсенс. Але в \emph{Україні} ситуація склалась трішки по-іншому.
Оскільки є багато людей, які ненавидять все \emph{українське}, але живуть і працюють
тут. Це проблема, але не потрібно, реагуючи на це, починати виховувати їхніх
антиподів. Це призводить до конфліктів та нерозуміння процесів. Тоді боротьба
за щось стає самоціллю, а результатом ніколи не буде розвиток, 
\textbf{Союз – мертвий, комсомол – живий, Тиск замість розвитку! Що хочуть від \emph{української} молоді?},
Станіслав Безушко, zaxid.net, 31.05.2021

Передовсім проблема лежить у площині цінностей. У більшості \emph{українців} вони
насправді відсутні. Цінності заміняються поведінковими моделями. Ми сприймаємо
за цінність набуту модель поведінки і вважаємо її головним маяком у житті. І це
стосується не тільки бабусь та батьків, а й молоді. Молодь має модель, але не
розуміє її суті. Старше покоління (якщо рівень освіти та критичного мислення
дозволяє) здатне оцінити всі ризики й загрози. Молодь, не розуміючи звідки
взялась така модель поведінки, просто копіює її та не заглиблюється в суть,
\textbf{Союз – мертвий, комсомол – живий, Тиск замість розвитку! Що хочуть від \emph{української} молоді?},
Станіслав Безушко, zaxid.net, 31.05.2021

Термін «передова радянська молодь» поступово трансформувався у «найкращі
представники \emph{української} молоді». Замість «виховувати молодь в дусі комунізму»
тепер виховують «патріотичну молодь». І такі порівняння можна продовжувати дуже
довго,
\textbf{Союз – мертвий, комсомол – живий, Тиск замість розвитку! Що хочуть від \emph{української} молоді?},
Станіслав Безушко, zaxid.net, 31.05.2021

Перший. Ні, \emph{Україна} не стала незалежною завдяки Бандері та \emph{УПА}.
Це може не подобатися, але це факт. Ідеологія інтеґрального націоналізму
бандерівців робила ставку на один спосіб звільнення \emph{України} –
загальнонаціональне визвольне повстання. Насправді незалежність \emph{України}
стала можливою передусім через шкурне, еґоїстичне бажання еліт \emph{УРСР}
роздерибанити майно республіки. Тобто держава \emph{Україна} створена не
патріотичними повстанцями, а радянськими пристосуванцями. Але так, наявність в
історії Бандери та \emph{УПА} дещо вплинула на особливості суспільної думки та
радянської гуманітарної політики в Галичині і, меншою мірою, на Волині,
\textbf{П'ять цікавих фактів – від \emph{УПА} до АТО}, Павло Зуб'юк, zaxid.net, 01.06.2021

Так виглядає, що час «рожевих окулярів» давно закінчився. Більшість \emph{українців}
знають хто є ворогом, який напав на півдні і сході \emph{України}. Є навіть бачення і
певні кроки в побудові сучасних збройних сил, які зможуть дати відсіч зовнішній
агресії. Проте ми повинні усвідомити, що акцент на відбитті такої агресії і
навіть на поверненні захоплених територій є лише частиною правильної зовнішньої
політики. Навіть побудова успішної економічної країни не вирішить усіх проблем
із РФ. Ми повинні усвідомити і змиритися з тим, що нам і в майбутньому
прийдеться жити із нашим екзистенційним якщо навіть не ворогом то точно
екзистенційним опонентом. І можливості позбутися цього у нас не має,
\textbf{\emph{Росія} – вічний супротивник України}, Роман Кізима, zaxid.net, 28.05.2021

Маючи постійного сильного опонента і навіть ворога під боком нам потрібно
враховувати той факт, що зовнішні фактори безумовно будуть впливати і на
внутрішні. І чим більш успішною буде \emph{Україна}, тим більшою буде протидія зі
Сходу. Нам не вдасться ігнорувати цивілізаційну зовнішню загрозу, закрившись у
власній шкаралупі. І приклад Ізраїлю у постійному ворожому оточенні лише
підтверджує цей факт. Будьмо сильні, успішні та найголовніше – проінформовані.
А проінформовані – значить озброєні...
\textbf{\emph{Росія} – вічний супротивник України}, Роман Кізима, zaxid.net, 28.05.2021

Останні декілька днів український політично ангажований facebook-сегмент рясніє
фотографіями депутатів однієї з політичних сил (як мінімум п'ятеро), які по
черзі несуть на мужніх чоловічих і жіночих плечах автомобільну шину. Все це
відбувається на фоні мітингу підтримки колишнього президента від кримінального
переслідування. Як відомо проти колишнього гаранта Конституції відкрито близько
10 (!) кримінальних справ. Починаючи з доволі серйозних, типу можливих
корупційних призначень і завершуючи доволі екзотичними, на кшталт «справи
Томосу»,
\textbf{Автомобільна шина як символ \emph{українського} бумерангу}, Роман Кізима, zaxid.net, 20.06.2020

Три місяці щеплень позаду. Чи стає \emph{Україна} ближчою до колективного
імунітету від коронавірусу?, Ольга Кириленко, pravda.com.ua, 02.06.2021

Про те, наскільки далека від цього \emph{Україна} говорить її місце у \enquote{загальному
заліку} країн за відсотковим охопленням щепленням – 132 місце зі 178, між
Нікарагуа та Соломоновими островами. Дуже далеко навіть після сусідніх Росії та
Білорусі, які на початку пандемії відверто ігнорували карантинні заходи.  УП
розповідає, що говорить наука про колективний імунітет від ковіду та чи
досяжний він для \emph{України} з нинішніми темпами вакцинації,
\textbf{Три місяці щеплень позаду. Чи стає \emph{Україна} ближчою до колективного
імунітету від коронавірусу?}, Ольга Кириленко, pravda.com.ua, 02.06.2021

Почему разваливается \emph{украинская} семья?, Иван Пургин, from-ua.com, 02.06.2021

На этот вопрос дают ответ психологи, считающие, что в \emph{Украине} материальные
проблемы просто доминируют над остальными, зачастую скрывая их. Они изначально
закладываются во многих браках, когда к своей будущей «половинке»
предъявляются, в первую очередь, именно материальные требования. Особенно это
касается женщин, мечтающих о принце на белом «мерседесе». Но они забывают, что
на чисто денежных отношениях семья не строится. «Если человек продал себя в
брак, то однажды его могут выбросить и купить другого», - гласит современная
народная мудрость,
\textbf{Почему разваливается \emph{украинская} семья?}, Иван Пургин, from-ua.com, 02.06.2021

Поднять уровень жизни \emph{украинцев} хотя бы втрое, до нижнего европейского, уже
непосильная задача для современной \emph{Украины}. Для этого потребовались бы не
просто колоссальные реформы, но и полная смена всей власти и политической
концепции – а этого просто не позволят сделать. Но если произойдет какое-то
чудо, форс-мажор, и однажды \emph{украинцы} проснутся в стране со средним доходом
1000-1200 евро на человека, то их радость будет недолгой. Потому что они просто
перейдут на более высокий потребительский уровень (выше, чем сейчас в \emph{Украине},
но еще ниже, чем в Германии или Франции), где люди испытывают такую же нехватку
денег. Только их будет не хватать уже не на детское питание или планшет, а на
что-то более существенное,
\textbf{Почему разваливается \emph{украинская} семья?}, Иван Пургин, from-ua.com, 02.06.2021

Впрочем, многие \emph{украинцы} об этом даже не догадываются. Потому что если наша
власть в чем и преуспела, так это в умении «замыливать» глаза и «запудривать»
мозги иллюзиями «перемог». И если «загуглить» статистику браков и разводов в
\emph{Украине}, то 95\% полученных результатов будут ссылаться на данные Министерства
юстиции. А они выглядят не просто ободряюще, а действительно «переможно»! Так,
например, согласно Минюсту, в 2019 году на 237 тысяч зарегистрированных браков
пришлось всего 38 тысяч разводов, то есть около 16\%. О таких показателях в
Европе и не мечтают! Кажется, что можно смело развалиться в кресле и начать
витийствовать о семейных ценностях \emph{уникальной нации},
\textbf{Почему разваливается \emph{украинская} семья?}, Иван Пургин, from-ua.com, 02.06.2021

Так от, пригадався мені цей апокриф після ознайомлення зі змістом пояснювальної
записки нещодавно опублікованого законопроєкту під номером 5554, в якій
стверджується, що обов'язкове використання \emph{української} мови при
демонстрації на \emph{українському} телебаченні російськомовних серіалів
\enquote{призведе до руйнування кінематографічної та телевізійної галузі від
чого, найбільших втрат зазнає \emph{український} глядач.},
\textbf{Микола Лукаш - \emph{Київ} - Черга - Дівчата - СРСР - Легенда}, Філіп Іллєнко, facebook.com, 31.05.2021

Воно там трохи колись пострибало на \emph{українському} телебаченні, а потім
його забули.  То треба ж про себе нагадати хоч якось, от воно й вищить.
Звичайний представник «гонимого» народу без країни, роду та племені, що за
гроші і в церкві пердне,
коментар, \textbf{Микола Лукаш - \emph{Київ} - Черга - Дівчата - СРСР - Легенда}, Філіп Іллєнко, facebook.com, 31.05.2021

Сегодня исполняется семь лет авиаудару по Луганску, во время которого погибли
восемь человек. 2 июня 2014 года ракеты \emph{украинского} штурмовика Су-25
обстреляли здание областной администрации в самом центре города. Это случилось
неподалеку от детского садика, сквера и жилых многоэтажек.  \emph{Украина} до сих пор
официально не признала своей роли в этой трагедии. Никаких дел не заведено и
расследований не ведется. Хотя другой авиации, кроме \emph{украинской}, над Луганском
не было. На седьмую годовщину авиаудара его снова обсуждают в \emph{Украине}. Звучат
мнения, что случившееся стало точкой невозврата в конфликте на Донбассе,
\textbf{\enquote{Точка невозврата для Донбасса}. Семь главных вопросов об авиаударе \emph{ВСУ} по Луганску 7 лет назад},
Максим Минин, strana.ua, 02.06.2021

5 марта в Луганске выбрали первого \enquote{народного губернатора} Александра
Харитонова. А через месяц противники Майдана взяли штурмом здание областного
СБУ, которое превратили в свой штаб. Это произошло 6 апреля - в день, когда
началась Антитеррористическая операция, объявленная исполняющим обязанности
президента \emph{Украины} Турчиновым. Через две недели \enquote{губернатором} объявили
бывшего десантника Валерия Болотова, а 27 апреля провозгласили \enquote{ЛНР}. Далее под
контроль сепаратистов стали переходить как объекты в Луганске, так и другие
города области. А 12 мая прошел \enquote{референдум} об отделении от \emph{Украины}, 
\textbf{\enquote{Точка невозврата для Донбасса}. Семь главных вопросов об авиаударе \emph{ВСУ} по Луганску 7 лет назад},
Максим Минин, strana.ua, 02.06.2021

Далее \emph{украинские} власти запустили версию о взорвавшемся кондиционере. По словам
замгенпрокурора Николая Голомши, сторонники \enquote{ЛНР} пытались сбить
украинский самолет из переносного зенитно-ракетного комплекса. Но попали в
здание ОГА - а именно, в кондиционер. Правда, судя по фото здания до и после
обстрела, на месте попадания ракеты не было кондиционера. Он находился двумя
окнами левее и был целым после взрыва,
\textbf{\enquote{Точка невозврата для Донбасса}. Семь главных вопросов об авиаударе \emph{ВСУ} по Луганску 7 лет назад},
Максим Минин, strana.ua, 02.06.2021

Университет Зеленского закончится очередным полетом в космос. В стране нет тех,
кто учил бы молодых \emph{украинцев} по-новому, Дмитрий Раимов, strana.ua, 02.06.2021

Эти новости - \enquote{первые ласточки} того безрадостного будущего, в котором правящие
элиты разных стран попытаются \enquote{размыть} ответственность за принятие
управленческих, правосудных, и иных решений, от которых может зависеть жизнь и
смерть, \enquote{перепоручив} её всевозможным роботам и \enquote{искусственному интеллекту}.
Конечно же, в реальности ни о какой \enquote{диктатуре машин} и \enquote{технологической
сингулярности} речи не идёт. Управлять всё равно будут люди. Но управляющий
слой надёжно прикроет себя от возмущения масс, переложив в их сознании
ответственность за принятие всех важных решений на машины. Кстати, именно
поэтому в \emph{Украине} так любят говорить о цифровизации,
\textbf{В Ливии квадрокоптер самостоятельно принял решение убить человека}, Даниил Богатырев, 02.06.2021

\enquote{Дальше Ткаченко обложит налогами PornHub}. Сеть обсуждает идею \enquote{слуг народа} \emph{украинизировать} Netflix и YouTube,
Виктория Венк, strana.ua, 14.11.2019

Закрытие телеканалов и ущемления по языку. Как в \emph{Украине} нарушают права человека. Анализ доклада ООН,
Максим Минин, strana.ua, 02.06.2021

2 июня 2014 года многие луганчане считают «точкой невозврата»: в этот день
гибридные формирования России штурмовали Луганский погранотряд. Среди горожан
всю весну распространяли слухи о якобы едущем в город «Правом секторе» с
Майдана – тот так и не прибыл, зато \emph{украинские} военный части начали штурмовать
вооруженные отряды, подконтрольные России. Ко второму июня все административные
здания в городе уже были захвачены, и вокруг погранзаставы началась настоящая
война с выстрелами и взрывами. Однако, в интерпретации пропагандистов 2 июня
войну, наоборот, развязали «\emph{украинские} каратели», нанеся авиаудар по зданию
Областной госадминистрации,
\textbf{«Никто нам не вернет украденные годы»: как в \emph{Луганске} вспоминают захват города},
Донбас.Реалії, Записки з окупації, radiosvoboda.org, 02.06.2021

\underline{Валентина, пенсионерка: Неужели в Луганске не было милиции?}
– Уже в мае повсюду по городу стояли вооруженные люди. В окнах государственных
зданий были мешки с песком. На зданиях висели не \emph{украинские} флаги. Вывозилась
документация. Вопрос – почему был такой бардак? Да потому что всё было
проплачено. Неужели в Луганске не было милиции? Почему тогда люди ходили с
оружием? Откуда взялись эти казаки, причем сразу и везде? Еще тогда были
отключены все \emph{украинские} каналы на телевидении. Вместо них показывали
\emph{аниукраинские} ролики, пропаганда – жесточайшая. Людям попросту промывали мозги.
А правду узнать было неоткуда. Поэтому такая поддержка России,
\textbf{«Никто нам не вернет украденные годы»: как в \emph{Луганске} вспоминают захват города},
Донбас.Реалії, Записки з окупації, radiosvoboda.org, 02.06.2021


\underline{Виктор, пенсионер: страну пишу только с маленькой буквы},
– Точка невозврата наступила, когда жгли коктейлями и били различными
предметами ментов, вплоть до убийства. Когда боевики Парубия вели отстрел с
гостиницы по «сотне» и ментам. И когда людей сожгли заживо в Доме профсоюзов в
Киеве. Затем был Дом профсоюзов в Одессе, горотдел милиции в Мариуполе... Думаю
этого достаточно, чтобы не любить \emph{«украину»}. Возможно, у вас возник вопрос,
почему я пишу с маленькой буквы? Я \emph{эту страну} с 2014 года пишу только с
маленькой буквы,
\textbf{«Никто нам не вернет украденные годы»: как в \emph{Луганске} вспоминают захват города},
Донбас.Реалії, Записки з окупації, radiosvoboda.org, 02.06.2021

\underline{Кристина, работник культуры: мы все в анабиозном сне} – До сих пор
помню события тех дней.  Мы понимали и видели, как работал отлаженный механизм
захвата города.  Осознала, что началось страшное, когда поздно вечером услышала
автоматные очереди. Тогда на городке завода ОР захватывали военную часть. Было
очень страшно, всю ночь рыдала, понимая, что жизнь теперь изменится, и это
война. Все, что происходило в городе после этого, похоже на сон. Мы понимали и
видели, как работал отлаженный механизм захвата города. Но до последнего
верили, что всё скоро закончится. Во время обстрелов было особенно страшно,
погибали люди, а сбитый самолет... Я слышала, как он взорвался! После этого мы
на время выскочили из города и вернулись только в октябре. Вернулись жить
дальше в свою квартиру, смириться было сложно с обстоятельствами. Мы понимаем,
что ничего хорошего в оккупации не может быть, но живем дальше. А сколько нас
таких! Даже многие соседи в доме, рьяно радующиеся русскому миру, сегодня хотят
возврата \emph{Украины}. Город жив, но он как в анабиозном сне,
\textbf{«Никто нам не вернет украденные годы»: как в \emph{Луганске} вспоминают захват города},
Донбас.Реалії, Записки з окупації, radiosvoboda.org, 02.06.2021

Колишній очільник УІНП, народний депутат \emph{України} від «Європейської
солідарності» Володимир В'ятрович вважає, що «декомунізувати» герб заважає
відсутність політичної волі. «Було б бажання, вже давно гроші б знайшли і
зробили», – сказав В'ятрович у коментарі Радіо Свобода. Володимир Бірчак,
керівник академічних програм у Центрі досліджень визвольного руху, один із
авторів пакету законів про «декомунізацію», нагадав у коментарі Радіо Свобода,
що «як символіка тоталітарного режиму герб має бути демонтованим», а уряд має
виконати норму закону,
\textbf{«Перекрити тризубом»: що заважає декомунізувати герб СРСР на монументі «Батьківщини-матері» у Києві}?
Ірина Штогрін, radiosvoboda.org, 01.06.2021

Заграничные белорсские патриоты составили революционный план Перемога, Но опыт
\emph{Украины} показывает, что за Перемогой всегда идет Зрада, Дмитрий Василец,
strana.ua, 03.06.2021

Интервью Зеленского отражает слабость \emph{украинской} внешней политики,
Никому не интересна страна, которая всегда попрошайничает, Виктор Суслов,
strana.ua, 03.06.2021

И главный вопрос: кому адресованы эти жалобы и какой в них смысл? Ну разве что
в очередной раз попугать Европу, что \enquote{Путин обязательно нападет. Ну до 12-16
сентября - точно}.  Так все равно денег не дадут и в \enquote{буржуинство} не возьмут.
А продолжат дальше разбирать Украину \enquote{на запчасти}.  На повестке дня -
\enquote{раздача} украинской земли и природных ресурсов,
\textbf{Интервью Зеленского отражает слабость \emph{украинской} внешней политики,
Никому не интересна страна, которая всегда попрошайничает}, Виктор Суслов,
strana.ua, 03.06.2021

Для граждан \emph{Украины} и России смысл этого диалога объяснять не нужно. Но для
иностранцев его решили разъяснить. И в разъяснении аналогом слова
\enquote{бандеровец} стала фраза \enquote{\emph{Украинский} нацистский коллаборант}. В
результате поднятого вокруг такого \enquote{перевода} скандала новые субтитры
стали политически нейтральными и непонятными для абсолютного большинства
иностранцев: \enquote{бандеровец} стал \enquote{banderite}. И некоторые
политики записали это себе как \enquote{перемогу}. При этом главной
\enquote{зрады} никто не заметил.  Как известно, фильм \enquote{Брат-2} в
\emph{Украине} запрещен. И тем не менее, фильм появился на \enquote{Нетфликсе}, а
\emph{украинские} политики не только ничего не сделали для снятия фильма с платформы,
но и прорекламировали \enquote{новинку} для максимально широкой отечественной
аудитории,
\textbf{Операция \enquote{Антиолигарх}, антибандеровский \enquote{Нетфликс},
МИД \emph{Украины} против \enquote{БелАвиа}. Итоги \enquote{Страны}},
strana.ua, 03.06.2021

Я президент цієї федерації. У нас є осередки в багатьох містах \emph{України}: Київ,
Харків, Дніпро, Одеса, Миколаїв, Запоріжжя, Івано-Франківськ, Львів і в містах
по областях. Змагання ми проводимо кілька разів на рік, але, наприклад,
минулого року їх взагалі не було через ковід і карантин. Ще не змагалися в 14
році, бо тоді було не до цього. Зараз ми збираємо близько сотні учасників - це
для аматорського спорту дуже великий показник. Не завжди навіть олімпійські
види спорту збирають стільки людей,
\textbf{Доброволець Олександр Воробєй: \enquote{Війна не змінює людину, вона
просто розкриває її сутність. Якщо людина була гандоном, то після фронту вона
стане махровим п\#дарасом}},
Віка Ясинська, censor.net.ua, 02.06.2021

Зараз ми міняємо приміщення – і я хочу, щоб ми робили 400 ножів на місяць. Для
\emph{України} це нормальна кількість. Бо якщо в принципі говорити про партії
ножів, які у нас є, – в основному це Китай. А наші вироби майже повністю
зроблені з український матеріалів: наприклад, сталь, робота теж українська.
Називаються наші ножі \enquote{Blade Brothers Knives}, бо мій клуб називається
\enquote{Blade Brothers}.  Дизайн лого нам зробив дизайнер Сергій Снурник,
\textbf{Доброволець Олександр Воробєй: \enquote{Війна не змінює людину, вона
просто розкриває її сутність. Якщо людина була гандоном, то після фронту вона
стане махровим п\#дарасом}},
Віка Ясинська, censor.net.ua, 02.06.2021

Я пройшов метрів десять і полегшено зітхнув. Голову не повертав, боявся проклять...
Іду далі, бачу чоловіка, одягнутий в хороше пальто, розмовляє по дорогому
телефону. Поряд з ним ідуть два \enquote{тітушки}, молоді хлопці, в тактичному одязі.
Один з них так подивився на мене, що моя рука автоматично полізла в сумку за
пістолетом. На щастя, не довелося його використовувати, тим більше, поблизу
стояли кілька хлопців з Національної гвардії \emph{України}.  Я дійшов до площі, де
стояла імровізована сцена, звідки кричали пригноблені вожді опзж, серед яких
\enquote{ошивався} комуніст симоненко,
\textbf{Чому партія \enquote{опзж}, морить своїх людей голодом?}, 
Михайло Ухман, censor.net.ua, 03.06.2021

Вони говорили кілька хвилин, у співрозмовника рабіновича час від часу світився телефон, вартістю 30 тисяч гривень.
А поряд, під парканом - сиділи прості люди, яких збирали по всій \emph{Україні}.
- Я жрать хочу, б..ть, - кричала одна старша жіночка. - Уже трі чіса нас дєржат.
- Подожді, сейчас наорутса із сцени і покормлят нас. Обіщалі же - промовив
чоловік, який стояв біля неї в дешевому одязі і розбирав телефон - жабку, яка:
\enquote{Заглючила, бля..ь}!
Що і хто кому обіцяв, я не знаю. Довелося іти по своїм справам. Але те, що
керівництво опзж використовує простих громадян в своїх цілях - це сто
відсотків. А вони, як стало баранів ідуть за тими, хто навіть не може їх
нагодувати...
Так і здохните рабами,
\textbf{Чому партія \enquote{опзж}, морить своїх людей голодом?}, 
Михайло Ухман, censor.net.ua, 03.06.2021

Якщо почуєте від когось, що в \emph{Україні} \enquote{громадянський конфлікт},
а не війна з Росією - розкажіть і покажіть цю історію. Якщо це не допоможе -
бийте в голову, стріляйте в коліна, адже перед вами може бути, як мінімум
сепаратист,
\textbf{Російські офіцери в \emph{українському} полоні},
Михайло Ухман, censor.net.ua, 17.05.2021

Група ворожих розвідників висунулася в напрямку наших позиції, не зустрівши
опору, зуміла підійти впритул і була виявлена \emph{українським} воїном - Вадимом
Пугачовим, який отримав важкі поранення в ході перестрілки і згодом помер.
Наші бійці зуміли затримати Єрофиєва і Олександрова, які потім дали важливі для
нас свідчення.  У 2016 році відбувся суд, росіян визнали винними і дали
тюремний термін. Проте через деякий час, їх обміняли на так звану політичну
бранку - Надію Савченко. Яка згодом відкрито почала працювати з ворогами нашого
народу, адже її полон і перебування в \emph{Україні} після визволення - це суцільна
трагікомедія, через яку загинули десятки \emph{українських} воїнів, 
\textbf{Російські офіцери в \emph{українському} полоні},
Михайло Ухман, censor.net.ua, 17.05.2021

О скандалах с изгнанием из Киевского оперного театра им. Шевченко руководителя
балетной труппы Дениса Матвиенко и с изгнанием из Полтавского
музычно-драматычного им. Гоголя главрежа Алексея Коломийцева писалось,
говорилось и показывалось много – истории известные и абсолютно архетипичные
для \emph{Украины}. Можно еще вспомнить инновационный театр Андрея Жолдака – он тоже
где-то в творческих \enquote{бегах}.  Иначе в \emph{Украине} НЕ БЫВАЕТ,
\textbf{Мариинский-2 и украинская культура: Почему \emph{Украина} – это фабрика человеческого мяса? (обн.)},
Ірина Славінська, pravda.com.ua, 03.05.2013

А вот в \emph{Украине} нет ни западного представления об \enquote{общем благе}, ни российского
имперского дискурса. Поэтому \emph{Украина} – это питомник талантов, которые могут
реализоваться только где-то далеко. \emph{Украина} – это ферма уникального
человеческого мяса,
\textbf{Мариинский-2 и украинская культура: Почему \emph{Украина} – это фабрика человеческого мяса? (обн.)},
Ірина Славінська, pravda.com.ua, 03.05.2013

\emph{Украинские} олигархи вкладываются в футбольные стадионы и футбольные клубы. Или
в высокохудожественные выставки с намотанными кишками и заспиртованными
коровьими головами. Но не в театры. А зачем? Ведь функции театра, а особенно
цирка, в \emph{Украине} неплохо исполняет Рада, телеканалы и т.д.  Едва ли не главная
борьба в XXI веке – это борьба с свино-человечеством, с обыдливанием человека,
с превращением его в тупое животное-потребителя. \emph{Украина}, кажется, с этого
фронта уже давно дезертировала,
\textbf{Мариинский-2 и украинская культура: Почему \emph{Украина} – это фабрика человеческого мяса? (обн.)},
Ірина Славінська, pravda.com.ua, 03.05.2013

\emph{Украина} как всегда – производитель ценнейшего человеческого материала (в плане
оперных голосов – не нужны дополнительные рекомендации). Однако ни одного
бренда, ни одного успешного проекта. А советские институции и форматы –
одряхлели и обветшали.  Попытка превратить Одесскую оперу в театр мирового
класса, которой несколько лет назад занимался режиссер Сергей Проскурня,
разбилась о тупизну и коррупционность одесских жлобов и киевских чиновников,
\textbf{Мариинский-2 и украинская культура: Почему \emph{Украина} – это фабрика человеческого мяса? (обн.)},
Ірина Славінська, pravda.com.ua, 03.05.2013

Власти \emph{Украины} начали силовую операцию против Донбасса в апреле 2014 года.
Урегулирование конфликта базируется на Комплексе мер по выполнению Минских
соглашений, подписанном 12 февраля 2015 года в белорусской столице участниками
Контактной группы и согласованном с главами стран - участниц \enquote{нормандской
четверки} (Россия, Германия, Франция и \emph{Украина}). Документ, в частности,
предусматривает прекращение огня и отвод тяжелых вооружений от линии
соприкосновения,
\textbf{Киев препятствует попыткам ЛНР выстроить отношения 
с государствами мира – Дейнего}, lug-info.com, 03.06.2021

\emph{Украинская} русофобия становится все более радикальной. Марков поделился инструментом для борьбы с ней,
ukraina.ru, 03.06.2021

З давніх-давен українців відзначало особливе ставлення до жінок, і це в часі
сягає не століть, а тисячоліть. Так стверджує історія та археологія – наші
праматері були берегинями родини, роду, племені й усього народу.  Виважене і
шанобливе ставлення українців до жіноцтва було поширене серед усіх соціальних
станів і за Київської Русі, і після її Хрещення. Права жіноцтва були вписані
до \enquote{Руської Правди} – зведення тогочасних законів Київської Русі, виконаного
Ярославом Мудрим на початку XІ століття,
\textbf{Шанобливе ставлення до жіноцтва в \emph{Україні}},
Марія Гуцол, slovoprosvity.org, 17.05.2021

Архітектурний проєкт у стилі модного тоді конструктивізму мав і \emph{українські}
риси, якими відомий зодчий Василь Кричевський завжди доповнював і прикрашав
свої споруди. Щоправда, авторам довелося змінювати початковий задум через брак
коштів, і їхнє творіння значно відрізнялося від первісного проекту, тож
архітектори були незадоволені,
\textbf{Знакові будівлі центру столиці}, Надія Наумова, slovoprosvity.org, 17.05.2021

Але в час політичних репресій деяким мешканцям знаменитої споруди як ворогам
народу довелося переходити на інші, некомфортні квартири. За багато років
будинок відбив складний шлях \emph{української} літератури та її творців, про нього
можна написати роман, як це зробив Юрій Тріфонов (\enquote{Дом на набережной}),
\textbf{Знакові будівлі центру столиці}, Надія Наумова, slovoprosvity.org, 17.05.2021

Уночі не спали, бо саме після півночі чи вдосвіта з під'їздів виводили ворогів
народу, тих, хто не хотів іти проти своєї совісті. А ті, хто уникнув арешту,
упокорилися, рішуче стали на шлях оспівування соціалістичної дійсності,
продовжували жити і писати тут свої твори. У деяких згадано і цей будинок:
\enquote{Коли у червні почалась війна…} Л. Первомайського, \enquote{У Києві, де
вгору гнеться вулиця...} С. Голованівського, \enquote{Наш добрий дім} В.
Малишка.  Тут були написані визначні, різні за тематикою твори \emph{української}
літератури: \enquote{Мир хатам, війна палацам} Юрія Смолича, \enquote{Собор}
Олеся Гончара, переклад \emph{українською мовою} поеми Шота Руставелі \enquote{Витязь
у барсовій шкірі} Миколи Бажана, \enquote{Гуси-лебеді летять} Михайла
Стельмаха, поезії \enquote{Знову цвітуть каштани... (Київський вальс)},
\enquote{Пісня про рушник} Андрія Малишка,
\textbf{Знакові будівлі центру столиці}, Надія Наумова, slovoprosvity.org, 17.05.2021

Маємо зворушливі спогади про Євгена Плужника – талановитого поета, який вірив у
ленінські ідеї та Жовтневу революцію. Він не мав власного кутка і жив у приймах
у своєї дружини в кімнаті реквізованого \enquote{прибуткового} будинку на вулиці
Прорізній. Плужник одним із перших членів Спілки письменників \emph{України} вступив
до кооперативу і заплатив гроші за житло. Його так хвилювала думка про те, чи
увійдуть улюблені меблі в кімнати, що він, вимірявши їх розміри, побіг у свою
ще неофіційну квартиру. Повернувся веселий і повідомив дружині: \enquote{Стане, стане!}
Мав туди заселитися 11 грудня. Але вже дещо відчував і наготував про всяк
випадок вузлик з усім необхідним. Євгена \enquote{взяли} в кімнаті на Прорізній 4
грудня 1934 р. на очах у дружини та її сестри,
\textbf{Знакові будівлі центру столиці}, Надія Наумова, slovoprosvity.org, 17.05.2021

У дворі в невеликому будинку, що поволі розвалюється, можна було б створити
філію Музею літератури, або віддати приміщення Музею історичного центру міста
Києва. Або ж створити літературну кав'ярню, де відвідувачі могли б згадати курс
\emph{так званої української радянської літератури},
\textbf{Знакові будівлі центру столиці}, Надія Наумова, slovoprosvity.org, 17.05.2021

Не важно, что именно для каждого из нас сегодня является олицетворением этого
страха, кого именно каждый из нас сегодня видит в качестве угрозы - Америку,
Кремль, \emph{Украину}, гомосексуалистов или турков, \enquote{развратную} Европу,
пятую колонну или просто начальника на работе или полицейского у входа в метро.
Важно - осознаем ли мы, до какой степени наши сегодняшние личные страхи, личное
ощущение внешней угрозы - в реальности являются лишь призраками прошлого,
существование которого мы так боимся признать,
\textbf{Бред сумасшедшего! Как с этим жить?}, Владимир Яковлев, news24ru.net, 03.06.2021

Объекты Северного потока-2 уже переводят в тестовый режим. Как-то не помогают
протесты \emph{украинской} патриотической общественности..., Александр
Рябоконь, strana.ua, 03.06.2021

Словом, \emph{Украина} стреляет себе не просто в ногу, а в голову из-за какой-то
недоросли... по фамилии... как его?.. а, Протасевич!  И кстати, обратите
внимание: несмотря ни на что, Лукашенко, как бы к нему не относиться,
продолжает называть \emph{Украину} братской страной...,
\textbf{Европа впала в иррациональный истеричный лукашизм}, Александр Карпец,
strana.ua, 03.06.2021

Зе-стандарты в сфере СМИ совершенно не оценили в ООН. Там считают, что
\emph{Украина} нарушает международные права человека, Елена Лукаш, strana.ua,
03.06.2021

Перепись населения \emph{Украины} становится все более и более актуальной.
\emph{Украинцев} уже 34 млн.?  Мы, UIF, получили данные очередной социологии, согласно
которой 88\% \emph{украинцев} имеют договора с семейными врачами, \textbf{Простая
логика подсказывает, что нас осталось уже 34 миллиона...}, strana.ua, Анатолий
Амелин, 03.06.2021

Необхідність в \emph{українській} ГТС залишається актуальною.  Зараз вона прокачує до
40 мільярдів кубів газу. Для того, щоб наша труба стала рентабельною, треба
збільшити прокачку газу хоча б до 60 млрд кубів.  Рівень зношеності системи
складає близько 70\%. Та й до того ж за останні 7 років газотранспортній
системі завдали немалої шкоди 5 великих аварій... Отож, влада має зосередитися на
модернізації власної ГТС і домовитися з Росією на збільшення прокачки газу.
Цілком реально вийти на перспективу щонайменше 100 млрд кубів газу. Тоді в
бюджет \emph{України} зайде значно більше 3 млрд дол. за транзит російського
блакитного палива територією нашої країни до Європи. Була б на це політична
воля \emph{української} влади,
\textbf{Наша газотранспортная система может быть рентабельной. Все зависит
только от политической воли \emph{украинской} власти}, Виталий Журавский,
strana.ua, 03.06.2021

Вечно можно смотреть на три вещи – как горит огонь, как течет вода, и как
\emph{украинские} президенты занимаются деолигархизацией.  Царь внёс в Раду
законопроект об олигархах. И это не попытка покончить с этим позорным явлением,
которое четверть века тянет вниз нашу страну, которую по прежнему контролирует
десяток семей, выбирающих президентов, формирующих парламент, правительство и
власть на местах,
\textbf{Закон об олигархах придуман не для борьбы с олигархами}, Светлана Крюкова,
strana.ua, 03.06.2021

Чому «Євробачення» – таємна зброя «м'якої сили» \emph{України}, radiosvoboda.org, 02.06.2021

Із чим найбільше асоціюють \emph{Україну}? Усі три останні десятиліття \emph{української}
державності відповіді на це запитання часто були депресивно безрадісними: від
Чорнобильської ядерної катастрофи до бідності, корупції і війни. Ці негативні
стереотипи створили додаткові перешкоди для країни, яка намагається скинути
спадщину століть, проведених під іноземною владою, і витворити власний,
незалежний міжнародний образ,
\textbf{Чому «Євробачення» – таємна зброя «м'якої сили» \emph{України}}, radiosvoboda.org, 02.06.2021

Друга перемога \emph{України} на «Євробаченні» прийшла через 12 років за дуже
відмінних обставин, але стала так само історичною. У час, коли \emph{Україна}
входила в третій рік неоголошеної війни, яку веде проти неї путінська Росія,
кримськотатарська співачка Джамала здобула перемогу на «Євробаченні» в травні
2016 року своєю баладою «1944», що так западає в пам'ять – у ній розповідається
історія радянських депортацій кримських татар часів Другої світової війни,
\textbf{Чому «Євробачення» – таємна зброя «м'якої сили» \emph{України}}, radiosvoboda.org, 02.06.2021

Крім двох перемог на «Євробаченні», \emph{Україна} також здобула два «срібла» і
одну «бронзу». Найбільш пам'ятним із цих майже переможних виступів був,
безумовно, чудово шалений номер \emph{українського} комедійного дреґ-виконавця
Вєрки Сердючки «Dancing Lasha Tumbai», який узяв друге місце 2007 року,
\textbf{Чому «Євробачення» – таємна зброя «м'якої сили» \emph{України}}, radiosvoboda.org, 02.06.2021

Номер запам'ятався сумішшю німецьких, англійських, російських і \emph{українських}
слів, шапкою на голові Вєрки, що нагадувала дискотечну кулю, і насмішливим
поданням – вони стали найсуттєвішими елементами виступу на «Євробаченні», які
гарантували \emph{українському} виконавцеві визначне місце в «залі слави» конкурсу.
Британська газета Guardian оголосила цей виступ «найкращою піснею, яка не
перемогла на «Євробаченні». І через чотирнадцять років любителі «Євробачення»
досі копіюють костюм Вєрки, який уже став канонічним, під час
вечірок-маскарадів на честь щорічного конкурсу,
\textbf{Чому «Євробачення» – таємна зброя «м'якої сили» \emph{України}}, radiosvoboda.org, 02.06.2021

За його словами, тільки так можна покласти край штучно створеному Путіним
конфлікту з \emph{Україною} та уникнути нової великомасштабної війни в Європі.
Мельник вважає, що взяття \emph{України} під захист НАТО може відлякати Росію
від нових воєнних вторгнень,
\textbf{Посол України в Німеччині: \enquote{Божевільна ціль Путіна – знищити \emph{Україну}}},
radiosvoboda.org, 03.06.2021

\emph{Украиноозабченные} так часто настаивают на том, что \emph{Украина} – это
Европа, что для них большим удивлением станет новость – Европа это все, что до
Урала. Потому, если смотреть географически, то город Донецк – вне сомнений
Европа! Мало того, некоторые европейские начинания вполнеприживаются в Донецке,
\textbf{Донецк – Европа!}, Сергей Лебедев (Лохматый), voskhodinfo.su, 02.06.2021

Колектив Національного культурного центру \emph{України} у м. Москві вітає з Міжнародним днем захисту дітей!,
\textbf{Наші діти!}, ukrcentr.ru, 01.06.2021

Відкриваючи концертну програму творчого колективу, зі словами вдячності до
гостей вечора звернулася художній керівник та диригент капели, начальник служби
з питань культури і діаспори НКЦУ, заслужений працівник культури \emph{України} та
Росії, голова РГО «Культурно-просвітницький центр \emph{українців} у м. Москві»
Вікторія Скопенко.  Протягом вечора у виконанні \emph{Української} народної хорової
капели Москви та її солістів прозвучала \emph{українська духовна музика}, обробки
народних пісень, етнографічні та вокально-хореографічні композиції, авторські
твори, поезія Тараса Шевченка, \emph{Лесі Українки},
\textbf{Пісенне джерело натхнення}, ukrcentr.ru, 28.05.2021

Внаслідок російської агресії на сході \emph{України} загинули 240 дітей, -
\emph{Україна} в ОБСЄ, day.kiev.ua, 03.06.2021

В \emph{Украине} от Covid-19 умерло втрое больше людей, чем указано в официальной статистике - исследование,
Юлия Супрун, strana.ua, 07.05.2021

Реестр олигархов не решит важнейших проблем нашей экономики. Он не поднимет
\emph{Украину} с уровня Сомали и Зимбабве, где она оказалась. Мы быстро
скатились до уровня Сомали и Зимбабве, и законопроект об олигархах уже не
спасет,
\textbf{Реестр олигархов не решит важнейших проблем нашей экономики},
Александр Гончаров, strana.ua, 04.06.2021

Схема государственных переворотов и без него была понятна. Из рассказанного
Протасевичем новое — разве что детали, суть же государственных переворотов и
роль в них вот этих всех Протасевичей и так понятна. В \emph{Украине} — ровно то же
самое происходило, но более успешно для организаторов переворота 2014,
\textbf{Сотрудничество с властью и слезы - это выбор самого Протасевича},
Александр Скубченко, strana.ua, 04.06.2021

Америка поняла, что может добиться своего от \emph{Украины} за гроши,
Алексей Кущ, strana.ua, 04.06.2021

Проблема \emph{Украины} заключается не в том, что нам не \enquote{платят} за выполнение
\enquote{коллективной разнарядки}. Проблема в том, что запад нащупал дешевый способ
добиваться от нас необходимого. Он работает с \enquote{янычасрким корпусом} за гранты,
предоставляет легитимизацию коррупционных активов нашим \enquote{Ылитам} и забрасывает
им подачки в виде кредитов. Обходится это в разы дешевле,
\textbf{Америка поняла, что может добиться своего от \emph{Украины} за гроши},
Алексей Кущ, strana.ua, 04.06.2021

Происходит когнитивный диссонанс: пока у нас есть \enquote{еврооптимисты}, нам не
видать Европы, так как возможность с их помощью влиять на страну, дешевле для
Брюсселя, чем план по включению \emph{Украины} в ЕС. Зачем обещать больше, если
\emph{украинцы} согласны на мизер?,
\textbf{Америка поняла, что может добиться своего от \emph{Украины} за гроши},
Алексей Кущ, strana.ua, 04.06.2021

Так как военный сбор не является целевым и просто повышает налоговое бремя на
доходы \emph{украинских} граждан, необходимо его отменить. Очевидно, что
\emph{украинские} граждане самостоятельно нашли бы более эффективное применение
своих заработанных денег в сумме \$5 млрд, чем государство, которое отобрало их
в виде военного сбора,
\textbf{Настало время отменить военный сбор},
Виктор Скаршевский, strana.ua, 04.06.2021

Рада против осуждения нацистов, откровения Протасевича и посла \emph{Украины} в Германии. Итоги \enquote{Страны},
strana.ua, 04.06.2021

«Ми звернемося до події, яка ділить історію \emph{України} на період «до» та «після» і
ділить все \emph{українське} суспільство. Це Майдан», – повідомив він. Азаров, Царьов,
Олійник, О'Браян і Боннел були представлені як «прямі учасники або свідки
подій»,
\textbf{«Азаров та Царьов у Раді безпеки ООН»: що насправді відбулося і в чому маніпуляція?},
Донбас.Реалії, radiosvoboda.org, 03.06.2021

\emph{Україна}, розповіла російська делегація, відповідальна за масові вбивства в Одесі та Маріуполі,
\textbf{«Азаров та Царьов у Раді безпеки ООН»: що насправді відбулося і в чому маніпуляція?},
Донбас.Реалії, radiosvoboda.org, 03.06.2021

Євромайдан призвів до появи – цитата Василя Небензі – «жахливих форм
націоналізму», він реалізований руками радикальних націоналістів, після
Євромайдану \emph{Україна} припинила бути толерантною та культурно розмаїтою
країною, у ній забороняють православ'я,
\textbf{«Азаров та Царьов у Раді безпеки ООН»: що насправді відбулося і в чому маніпуляція?},
Донбас.Реалії, radiosvoboda.org, 03.06.2021

Поява такого транспорту в перший рік війни говорила про жалюгідний на той
момент стан \emph{української} армії. Але головний висновок із цієї історії все
ж інший.  – Можна пишатися нашим народом, – каже підполковник, військовий
журналіст Сергій Камінський. – У світі ще не було настільки масштабного
волонтерського руху, прагнення захистити своїх бійців. Настільки різноманітної
техніки, яку переробляли для фронту, я ніде не бачив.  Навесні 2021 року
Камінський випустив енциклопедію \enquote{Народні панцерники}, в якій зібрав
детальну інформацію про 75 бронемобілів, зроблених аматорськими
конструкторськими бюро,
\textbf{Шушпанцери і бандеромобілі. Героїчна історія саморобних броньовиків у війні за український Донбас},
Євген Руденко; Дмитро Ларін, pravda.com.ua, 04.06.2021

Після драматичних боїв за Іловайськ у 2014 році до рук бойовиків потрапило
щось: чи то машина з постапокаліптичних фантазій \enquote{Шаленого Макса}, чи то
іржавий Tesla CyberTruck.  На відео сепаратистів зберігся захоплений КамАЗ
добровольчого спецбатальйону міліції \enquote{Дніпро-1}, наглухо захований під
незграбним панциром із зварених металевих листів.  \enquote{Чо они курят?! Дикий тюнинг
\emph{по-укропски}}, – описував ворожий голос за кадром цього мутанта,
\textbf{Шушпанцери і бандеромобілі. Героїчна історія саморобних броньовиків у війні за український Донбас},
Євген Руденко; Дмитро Ларін, pravda.com.ua, 04.06.2021

На тлі повчальної історії пікапа Януковича-молодшого особливими виглядають
приклади, коли прості \emph{українці} добровільно віддавали фронту все, що у них є.  У
вересні 2014 року 63-річний пенсіонер-водій Валерій Крижов з Кривого Рога з
позивним \enquote{Пахан} відправився воювати у складі батальйону \enquote{Кривбас} на своєму
МАЗі,
\textbf{Шушпанцери і бандеромобілі. Героїчна історія саморобних броньовиків у війні за український Донбас},
Євген Руденко; Дмитро Ларін, pravda.com.ua, 04.06.2021

Незабаром в \emph{українському} шушпанцері бойовики знайшли особисті документи
Крижова.  – Вони визначили номери телефонів його та членів родини. Дзвонили
йому та доньці, пропонували йти воювати на бік квазіреспубліки. Казали, що
відремонтували машину і будуть добре платити. Але далі Валерій служив у
морській піхоті. Подарованим \enquote{Правим сектором} джипом Ford за п'ять місяців на
Світлодарській дузі він намотав 57 тисяч кілометрів та вивіз майже сто
поранених \emph{українських} захисників, – ділиться Сергій Камінський,
\textbf{Шушпанцери і бандеромобілі. Героїчна історія саморобних броньовиків у війні за український Донбас},
Євген Руденко; Дмитро Ларін, pravda.com.ua, 04.06.2021

Словничок освіченого \emph{українця}. Як говорити та писати про рак,
Юлія Саліженко, pravda.com.ua, 04.06.2021

Повернімося до новин від адекватних сусідів. Скажімо, всі в курсі, що Польща –
наш стратегічний партнер й «адвокат \emph{України}» в НАТО та ЄС. Тоді
порівняймо, що ми читаємо про життя польського суспільства в \emph{українських}
медіа, і що пишуть польські ЗМІ,
\textbf{«Вікно в Європу»}, Назар Кісь, zaxid.net, 03.06.2021

Очередным примером \emph{украинского кривосудия} стало дело Сергея Стерненко —
одесского радикала, обвиняемого в похищении и пытках человека, а по факту
получившего всего год условно. Жаловаться на \emph{украинскую Фемиду} глупо:
мало ли убийц (а Стерненко обвиняют ещё и в убийстве), террористов, военных
преступников гуляют на свободе, в то время как независимые журналисты или
оппозиционно настроенные пользователи соцсетей подвергаются реальным гонениям,
\textbf{Казус Стерненко. Европейская «правда» как символ лжи}, 
Константин Кеворкян, ukraina.ru, 04.06.2021

На \emph{Украине} действует ультраправый режим, лишь слегка прикрытый фиговым листком
неких демократических процедур. В целях самосохранения режим продолжит
целенаправленно оправдывать «своих» и преследовать «чужих» (под «чужими»
подразумеваются те, кто хоть с какой-то степенью эффективности борется за
освобождение \emph{Украины} от колониальной зависимости от Запада).  Весьма
показательно в этом отношении, что на освобождении радикала Стерненко
настаивали иностранные посольства, которые видят в молодом убийце весьма
перспективного лидера ультраправых, через которых во многом осуществляется
запугивание и управление \emph{Украиной}. И это далеко не единичный случай прямой
поддержки ультраправых радикалов якобы «демократическим Западом»,
\textbf{Казус Стерненко. Европейская «правда» как символ лжи}, 
Константин Кеворкян, ukraina.ru, 04.06.2021

Данилов пригрозил отобрать гражданство \emph{украинцев}, которые попали под
санкции США, strana.ua, 04.06.2021

Врагам Америки пересмотрят гражданство \emph{Украины}. Главные решения нового СНБО,
Максим Минин; Игорь Рец, strana.ua, 04.06.2021

Мать убитого в 2015 году писателя Олеся Бузины побывала в студии ток-шоу «Право
на владу» телеканала «1+1». Валентина Павловна рассказала о своих бесконечных
хождениях по инстанциям, чтобы добиться справедливого расследования
преступления, и о том, как дело намеренно затягивают. Подчёркнутая любезность
ведущей Натальи Мосейчук противоестественно контрастировала с редакторской
подводкой, в которой зрителям сообщили о «скандально известном публицисте с
\emph{украиноненавистническими} взглядами...»,
\textbf{«Я не живу, я тень». Мать Олеся Бузины рассказала, как адвокаты обвиняемых затягивают дело об убийстве её сына},
buzina.org, 04.06.2021

Сегодня еще одна страшная дата новейшей истории \emph{Украины}. Семь лет назад
\emph{украинский} военный самолет нанес ракетный удар по центральной площади
большого европейского города Луганск. Тогда погибли восемь и были ранены
десятки мирных жителей.  В сквере перед обстрелянным зданием областной
госадминистрации нашли две неразорвавшиеся авиационные ракеты. Общее количество
выпущенных при залпе боеприпасов составило около 20 штук.  Вокруг сквера
расположены жилые многоэтажные дома. За зданием ОГА находится детский сад. В
самом сквере на расстоянии 30-40 метров от места взрывов расположена детская
площадка, на которой находились дети.  Это военное преступление, как и многие
другие, до сих пор не расследовано, виновные не наказаны,
\textbf{2 июня 2014 года — Луганск. Все это было безумно дико и страшно},
Алексей Романов, buzina.org, 06.02.2021

Это преступление сыграло огромную роль в развязывании войны. Жители Луганщины,
восточной \emph{Украины} увидели, насколько жестокой и лживой является новая
\emph{украинская} власть.  Именно тогда многие решили взять в руки оружие.  Для
большей части луганчан этот ряд воронок ракетного «привета» из Киева стал
чертой, за которой исчезли все надежды на диалог.  Пока военные преступления не
будут расследованы, а виновники не будут наказаны, мира на востоке \emph{Украины} не
будет.  Но ни прошлые, ни нынешние киевские власти на это не пойдут. Прошлые
захватили власть в результате переворота, нынешние их духовные наследники. И
все они либо замалчивали, либо защищали и оправдывали преступления против
мирных жителей.  Но военные преступления не имеют срока давности.  Наверное,
именно поэтому \emph{Украина} до сих пор не ратифицировала Римский статут —
международный договор, которым был создан Международный уголовный суд в Гааге,
\textbf{2 июня 2014 года — Луганск. Все это было безумно дико и страшно},
Алексей Романов, buzina.org, 06.02.2021

І не треба про мовні квоти на радіо. Це все у порожнечу, як у порожнечу були
колись Кобзон, Ротару та Самоцветы з піснею \enquote{Мой адресс Советский Союз}. І
винуваті у цьому не діти, які не хочуть слухати патріотичні пісні, а \enquote{торчать}
під російські матюки. Дітей не обманеш – вони завжди відчувають коли з ними
щирі, а коли ні. Дитина ніколи не повірить \enquote{патріоту}, який проголосивши
офіційну промову \emph{українською}, за мить переходить на російську, як це робить
переважна більшість \emph{українських} політиків, депутатів, артистів, телеведучих,
\textbf{Російський реп може зруйнувати Україну},
Микола Несенюк, gazeta.ua, 02.06.2021

Тушите свет, гасите долг: электричество станет не по карману большинству \emph{украинцев}!,
Иван Пургин, from-ua.com, 04.06.2021

Коммунальный коллапс, либо социальный взрыв: что ждет \emph{украинцев} в ближайшем будущем,
from-ua.com, 04.06.2021

У \emph{украинской} молодежи снижается поддержка европейского курса,

В \emph{Украине} все это было уже много раз и, очевидно, еще будет не раз. При этом
многие положат свои жизни за \enquote{восстановление исторической
справедливости}, со скрипом прокручивая \emph{Украину} по очередному кругу.  Пока
\emph{Украина} пребывает в феодальном застое она не имеет будущего и поэтому всю свою
энергию тратит на воспроизводство идеальной \emph{Украины} прошлого и постоянного
переименования улиц,
\textbf{Украина не должна тратить энергию на воспроизводство идеального прошлого},
Андрей Головачев, strana.ua, 05.06.2021

Феодально-сословное общество подобно бабочке в янтаре,- оно застывает в
неизменном виде на столетия. Так Европа застыла без развития в феодализме с 6
по 16 века. Тысячу лет без всякого развития! Только Библию перечитывали
по-разному. Как всякое феодальное общество, \emph{Украина} не имеет будущего и поэтому
вынуждена жить мифами прошлого. Будущее для \emph{феодальной Украины} это всегда лишь
попытки воспроизвести некое прекрасное прошлое. Поэтому \emph{украинцы} постоянно
копаются в истории, стараясь найти там пример той \emph{идеальной Украины}, которую
надо воспроизвести и тогда настанет счастье,
\textbf{Украина не должна тратить энергию на воспроизводство идеального прошлого},
Андрей Головачев, strana.ua, 05.06.2021

Нынешнее состояние многие критикуют и считают его неудачным. А раз так, то весь
вопрос заключается лишь в том чтобы определить тот момент, когда \emph{Украина}
свернула с пути истинного и надо просто вернуться на этот путь. Ведь была де
когда то \emph{Украина} \enquote{лучшая, более правильная} \emph{Украина}!!  Поэтому, некоторые
считают что \emph{Украины} ошиблась еще при Хмельницком, когда потеряла связь с
Европой и сейчас надо просто восстановить историческую справедливость и
посносить все памятники \enquote{изменнику Хмельницкому, который продался москалям}. А
некоторые, наоборот, что казацкая \emph{Украина} была идеальной и стараются всех
одеть в шаровары и жупаны,
\textbf{Украина не должна тратить энергию на воспроизводство идеального прошлого},
Андрей Головачев, strana.ua, 05.06.2021

Почему в Украине постоянно переименовывают улицы?  Если коротко, то потому что
\emph{Украина} это феодальное общество. И как всякое феодально-сословное
общество \emph{Украина} не развивается и не может развиваться, потому что
инновационный процесс в сословном обществе невозможен в принципе. Поэтому
\emph{Украина} уже многие десятилетия просто ходит по кругу и каждый раз
воспроизводит сама себя. Смешно, но даже ВВП на душу за годы независимости не
вырос, а долги бюджета стали уже преддефолтными,
\textbf{Украина не должна тратить энергию на воспроизводство идеального прошлого},
Андрей Головачев, strana.ua, 05.06.2021

А если в \emph{Украине} снова будут доминировать идеи что хорошая
\emph{Украина} это \emph{Украина}, как младший брат России, то улицы будут
снова называться именем Хмельницкого, Ватутина и Гагарина. Если в обществе
победит точка зрения, что идеальная \emph{Украины} была \emph{Украиной}
Бандеры, то соответственно улицы будут именоваться именами героев УПА. А кое
где в западных районах восстанавливают сейчас и австрийские названия. Ведь не
просто так, кто то ностальгирует по \emph{Украине} в составе А-В, как идеальной
модели, \textbf{Украина не должна тратить энергию на воспроизводство идеального
прошлого}, Андрей Головачев, strana.ua, 05.06.2021

Який тільки народ не адаптував сюжет стародавньої казки про відомих \emph{українцям}
братів Правду і Кривду. Проте найкраще, здається, це вдалося зробити в
англосаксонському світі, де цю моралізаторську казку назвали «Засліплення
правди фальшивістю». І тут йдеться не тільки про маніпуляції та перебріхування.
Мова йде також про тих, хто облудою і фальшем забирає в інших здатність бачити,
чути і розуміти. Хоча у відомій казці є безліч незрозумілих для сучасної людини
поведінкових практик, а сюжет містить багато спрощень, основна мораль зрозуміла
– неправдою щасливим не станеш. Але цього разу нам важливо зосередитися на
тому, що одному з братів за допомогою цинічних маніпуляцій вдалося виманити у
брата його коржа. Ще цікавішим є те, чому брат Правда погодився на те, щоб його
скалічили до напівживого стану,
\textbf{Правда і Кривда}, Василь Расевич, zaxid.net, 04.06.2021

Мне тут напомнили о цитировании несколько лет назад весьма примечательного
\emph{украинского} политика. Последний \emph{украинский} гетман, он же адьютант Николая
Второго, он же простой русский человек Павел Скоропадский. Который в мемуарах
слово государь писал с большой буквы Г.  Процитирую снова. Эпоха Порошенко
сменилась на времена Зеленского, но ничего не изменилось: \enquote{В старой России
единственная область, где \emph{украинство}, и то под сильной цензурой, разрешалось, —
это театр ... }, 
\textbf{За Порошенко пришел Зеленский, но в украинской политике ничего не поменялось}, Максим Войтенко, strana.ua, 05.06.2021

Українська правда почему-то подала это как праздник предоставления американцами
вакцин Киеву, но вот прямая цитата советника по национальной безопасности
Салливана: «Наш підхід віддає пріоритет іншим частинам світу, включно з
країнами з низьким рівнем вакцинації та тими, які зіткнулись з терміновою
кризою, такими як Західний Берег і Сектор Гази, Україна, Косово, Ірак та інші»,
\textbf{Америка поставила Украину в ряд с сектором Газа, Ираком и Косово},
Надежда Сасс, strana.ua, 05.06.2021

Группа французских сенаторов после визита в Киев в конце мая этого года
обратилась с официальным запросом к министру иностранных дел Франции с просьбой
высказать позицию по поводу неонацистского движения в Украине, а именно
деятельности таких организаций как вооруженное подразделение \enquote{Азов}.
Сенаторы Натали Гуле, Жан-Пьер Мог и Жоэль Геррио из комиссии иностранных дел
Сената требуют от главы французского МИДа выступить по украинскому вопросу в
Сенате.  Сенаторов возмутило то, что в центре столицы \emph{Украины} торгуют
нацистскими символами и записывают в бойцы \enquote{Азова},
\textbf{Французские сенаторы потребовали от МИД страны реакции на ситуацию с ультраправыми в Украине. Что это значит?},
strana.ua, 05.06.2021

Окремим напрямком дослідження було визначення соціолінгвістичного стану
ураження \emph{українських} військовослужбовців (досліджували активність у соцмережах
представників двадцяти бригад ЗСУ) з метою вироблення пропозицій військовому
відомству щодо своєчасного проведення випереджальних заходів протидії
негативному комунікаційно-контентному впливу на особовий склад \emph{ЗСУ},
\textbf{\emph{Росія} здійснює глобальну інформаційну «гібридно-месіанську агресію»: як її нейтралізувати?},
radiosvoboda.org, 05.06.2021

Конечно, меня эти все вопросы доводили до белого каления, хоть я старалась
этого и не показывать. Во-первых, бежали мы из Киева не по своей воле, а после
внесения на сайт «Миротворец» (запрещён в РФ), призывов Ницой через
\emph{украинские} СМИ к нашему физическому уничтожению, на которые откликнулись
парни из С14, визитов \emph{СБУ} к мужу на работу и очень пристального внимания
со стороны \emph{украинских} СМИ.  Во-вторых, мы действительно могли выбрать
любую страну ЕС, так как на тот момент у меня на руках было по крайней мере три
зарегистрированных заявления в прокуратуру Киева, несколько заявлений в
полицию, официальное обращение к офицерам ООН и ОБСЕ в связи с травлей нашей
семьи, составленные с помощью Института правовой защиты им. Ирины Бережной,
\textbf{Ярославль глазами Киевлянки},
Светлана Пикта, odnarodyna.org, 03.06.2021

Мы не искали улучшения своего материального положения, мы искали безопасности и
Родины.  Мы не рассчитывали на какие-то «плюшки» вроде материнского капитала,
потому что все дети были рождены на \emph{Украине}, а закон в этом отношении имеет
двойное прочтение: могли дать, могли не дать. Мы хотели, чтобы дети получали
нормально образование, а нам больше бы не приходилось оглядываться, выходя из
подъезда: не поджидают ли нас отморозки, которым дали команду «фас»,
\textbf{Ярославль глазами Киевлянки},
Светлана Пикта, odnarodyna.org, 03.06.2021

Очень, прямо очень сильно меня порадовали детские площадки с прорезиненными
покрытиями и новенькими «горками», но особенно поразил детский сад. Он был
похож на сказочный замок! Внутри оранжереи, бассейн, музыкальные залы,
гимнастические, хореаграфические комнаты, богатые детские площадки. Особенно
умилило баскетбольное поле для самых маленьких. Причём попали мы в этот садик
не за взятку, как это было принято на \emph{Украине}, а по электронной очереди.  Когда
мы сменили район, садик стал попроще, но тоже с бассейном,
\textbf{Ярославль глазами Киевлянки},
Светлана Пикта, odnarodyna.org, 03.06.2021

Кстати, о взятках. После \emph{Украины} первое, что окатывает приезжего, как
холодный душ, – это совсем другое отношение к закону. Его здесь, конечно,
обходят, но уж особо дерзкие и гораздо более осторожно. И нередки случаи, когда
и они попадаются и сидят. «Неприкосновенных», которые творят беспредел и им за
это ничего не будет, я лично не встречала. А о том, чтобы на бытовом уровне
дать взятку врачу, воспитателю, учителю, чиновнику – вообще можно забыть,
\textbf{Ярославль глазами Киевлянки},
Светлана Пикта, odnarodyna.org, 03.06.2021

Одно из слабых мест в Ярославской области – медицина.  Поликлиники переполнены,
новые строятся очень медленно, попасть к некоторым специалистам можно только
через некоторое время. Однако же все прививки делаются регулярно, вакцина
всегда проверенная, чаще всего российская, которой я доверяю. Кроме того, по
ОМС можно бесплатно делать плановые операции, которые в \emph{Украине} бы
стоили круглую сумму. Необычно, что раз в два-три года надо проходить
медосмотры взрослым, причём с онкомаркерами и анализами, которые без ОМС
влетели бы в копеечку,
\textbf{Ярославль глазами Киевлянки},
Светлана Пикта, odnarodyna.org, 03.06.2021

А встретились мы в \enquote{Одноклассниках}, лет этак 35 спустя. Галя на фото была в
вышиванке. Писала мне на \emph{украинском} (искусственно созданном) языке. Убеждала
меня, что \emph{Украина}, стала наконец свободной от засилья русских. Я пытался
возражать, но куда там. Потом выдала, что Бандера - их настоящий герой, борец
за свободу. А однажды объявила, что первая конституция была написана на \emph{Украине}
в 17 веке. Надо же - страны как таковой нет, а конституция есть. Разошлись мы с
Галей идейно и переписка прекратилась. А недавно она в соцсетях снова
засветилась - стоит в обнимку с двумя офицерами - участниками АТО, увешанных
орденами. Ордена эти, видимо, за убийства детей и мирных жителей были получены.
Слышал от знакомых, что вышла замуж уже в третий раз - куда делась ее
стеснительность. Перековалась Галя, стала идейной националисткой,
\textbf{Как примерная \emph{украинская комсомолка} Галя стала стала непримиримой националисткой},
Greg, zen.yandex.ru, 06.06.2021

На школьном вечера она читала стихотворение \enquote{Комсомолец}, там в финале
звучит фраза - \enquote{Вышел один и сказал куренному: \enquote{Я комсомолец,
стреляй!}} (куренной - это полковник петлюровских войск). А начинается стих так
(мой перевод с \emph{украинского}): \enquote{Бой отгремел, желто-синее знамя
затрепетало над станцией вновь...}. Понятно, что речь идет о Гражданской войне
на \emph{Украине}. Читала Галя с пафосом, с выражением, довольно неплохо. Ей долго
аплодировали,
\textbf{Как примерная \emph{украинская комсомолка} Галя стала стала непримиримой националисткой},
Greg, zen.yandex.ru, 06.06.2021

Згадалося: чому бідні, бо дурні; чому дурні, бо бідні. Софіївська площа
заставлена розмальованими зайцями. Організатори виставки дали пафосну
конференцію, похвалилися супер ідеєю. На моє питання у фб: яке стосування це
має до \emph{українського} Великодня, вони, організатори, відповіли, що
європеїзують \emph{Україну}. (Правда? - саркастично запитую я)...  Поруч з
Софією є і виставка великих мальованок, які організатори вперто і безграмотно
називають писанками, формуючи фейклорні суспільні уявлення замість того, щоб
популяризувати фольклорні, 
\textbf{Великодні зайці на столичній Софіївській площі: фольклор чи фейклор?},
Анжеліка Рудницька, glavcom.ua, 05.04.2018

Взвесив все «за» и «против», Светлана и Вадим, наконец, приняли верное решение
— уехать в Россию. В конце октября Светлана трогательно попрощалась со своими
\emph{украинскими друзьями}. А 20 декабря она родила двойню – Тихона и Марию,
\textbf{Ярославщина ждет отважных киевлян}, rweek.ru, 19.01.2018

Светлане очень понравился Ярославль: – Древний город с потрясающей тысячелетней
историей. Видно, что о городе заботятся. Поразили порядок и чистота.  Одно из
первых блюд для семьи, которое приготовила Светлана для своей семьи в
ярославской квартире, был, конечно же, борщ. Правда, она призналась, что
пришлось искать в магазинах свеклу. Восторг у нее вызвал шоколад «Россия –
щедрая душа» – Вкуснотища, – отозвалась о нем Светлана. Записи жителей России
и \emph{Украины} в чате шли сплошным потоком. Люди радовались за нее и ее семью,
поздравляли. Вместе с тем, некоторые интересовались, будет ли Светлана
возвращаться \emph{на Украину}?  – Пока на \emph{Украине} творится такое безумие, проводится
«госдеповский» эксперимент, возвращаться не думаем,
\textbf{Мужественные киевляне стали нашими земляками}, rweek.ru, 20.07.2018

Земля, кров, честь. \emph{Українська нація} – кровно-духовна спорідненість минулих,
теперішніх і прийдешніх поколінь. На основі багатотисячолітнього досвіду наших
предків – трипільців, аріїв, сколотів, скитів, антів, русів, \emph{українців} – ми на
своїй землі будуємо і втілюємо у життя наше бачення майбутнього свого народу та
світу в цілому. Служіння добру і справедливості є нашою честю.  Наше ставлення
до людей інших національностей є прихильним до тих, хто разом із нами бореться
за \emph{українську} національну державу та допомагає у цій боротьбі; нейтральним – до
тих, хто не заважає нашій боротьбі за право бути господарями своєї долі на
своїй землі; ворожим – до тих, хто протидіє справі \emph{українського} національного
відродження і державотворення,
\textbf{Програма національного визволення та державотворення}, pravyysektor.info

В \emph{Україні} християнські релігійні організації як і держава відстоюють інтереси
народу, але є самостійними, тому створення єдиної Помісної Христової Церкви,
безумовно, є справою \emph{українських} християнських церков. Римська та Візантійська
імперії, у боротьбі за світове домінування, розділили віручих, але це не
повинно жодним чином впливати на віруючих в \emph{Україні} через 1000 років. Можливо,
що саме досвід створення єдиної Помісної Христової Церкви у сопричасті з Римом
і Константинополем стане тим першим кроком, який допоможе примирити усі світові
конфесії. Створення єдиної \emph{української} християнської церкви є не просто
випробуванням для церков в \emph{Україні}, це є саме тією історичною місією
\emph{українських} священнослужителів і народу, втілення якої приведе до об'єднання
християн у всьому світі, починаючи з \emph{України},
\textbf{Програма національного визволення та державотворення}, pravyysektor.info

Кримінальна відповідальність за зневажливе ставлення до \emph{української} мови
(штучна мова, суміш польської та російської і т.п.) – один рік позбавлення
волі, виправні роботи, або штраф. Для іноземних громадян передбачити заочне
засудження за цією статтею. Експертизу подібних публічних висловлювань
необхідно доручити Інституту \emph{української мови} НАН \emph{України} з урахуванням
контексту висловлювання,
\textbf{Програма національного визволення та державотворення}, pravyysektor.info

Якщо будь-які юридичні особи усіх форм власності (крім дипломатичних
представництв і організацій, що мають дипломатичний статус) обслуговують
громадян \emph{України} на території \emph{України} (крім онлайн послуг) іноземними мовами
без їхньої згоди, то на такі юридичні особи накладається штраф у розмірі 100
мінімальних заробітних плат (крім державних органів виконавчої влади, народних
депутатів усіх рівнів, комунальних органів влади, бюджетних установ, де
відповідальність є персональною),
\textbf{Програма національного визволення та державотворення}, pravyysektor.info

При проведенні тендерів чи будь-яких інших державних чи комунальних закупівлях
– уся технічна і супровідна документація на товари, що експортуються в
\emph{Україну}, повинна бути виключно \emph{українською мовою} (крім випадків,
коли послуги надаються поза межами \emph{України}). Забороняється закупляти за
державні кошти програмне забезпечення без інтерфейсу, технічної документації і
допомоги \emph{українською мовою},
\textbf{Програма національного визволення та державотворення}, pravyysektor.info

На місці слабкості і невиразності ми маємо створити постійно діючий фактор
ширення в суспільстві високих естетичних, духовних та моральних цінностей і
формування національної етичної та естетичної свідомості \emph{українців}.
Маємо виробити в кожному \emph{українцеві} духовно-світоглядний імунітет проти
агресивних російського за західного культурних впливів,
\textbf{Програма національного визволення та державотворення}, pravyysektor.info

Обмеження загального виборчого права в \emph{Україні} з метою запобігання популізму
(політичного аферізму).  Загальне виборче право безумовно стало одним із
найважливіших досягнень ХХ століття. Однак, практика його використання без
обмежень призвела до вкрай негативних тенденцій, а саме - популізму. До влади
приходять не кращі сини народу, а фактично аферисти, які обіцянками вводять
народ в оману, відверто обманюють. І така ситуація спостерігається не лише в
\emph{Україні}, а й у всьому світі,
\textbf{Програма національного визволення та державотворення}, pravyysektor.info

Нардеп Евгений Мураев вспомнил о тех, кто отмечает День журналиста в тюрьме - и
в \emph{Украине} таких людей немало.  "По-разному придется праздновать
причастным к нынешнему Дню журналиста. Кто-то отмечает его в тюрьме и власть,
которая их туда посадила, утверждает, будто там вовсе не журналисты и сидят не
за выполнение своего долга. В это же самое время поглаживая по голове и давая
сахарок тем, кто лижет руку и преданно смотрит в глаза. Но зато мы знаем, что
данный сорт тружеников СМИ к журналистам не относится совершенно. Так же, как и
многочисленные \enquote{блоггеры}, \enquote{эксперты}, \enquote{райтеры} и копипастеры,
\textbf{\enquote{У кого теперь поднимется рука поздравить Бабченко?} Как в сети отмечали День журналиста},
strana.ua, 06.05.2018

Так догонит ли \emph{Украина} Россию по уровню средних зарплат?  Можно уверенно
сказать, что да.  Примерно 10\% \emph{украинцев} будут получать (и уже получают)
больше, чем все население РФ в среднем.  И это уже наша проблема – ведь эти 10\%
и есть основа нашего политического режима: что прошлого, что нынешнего.  Это
именно те 10\%, которые рукоплещут повышению тарифов и фразам о детях
\enquote{не того качества}.  Это именно те 10\%, которые клеймят позором
\enquote{патерналистов и совков}.  Это те 10\%, которые готовы наступать на
свободу человека, если она не подкреплена личной \enquote{монетизацией}: ограничить
право голоса для тех, кто не платит налоги, или отменить пенсии.  Мы получили
смысловую пропасть между 10\%, которые зарабатывают на остальном народе, и этими
90\% населения,
\textbf{В Украине утвердилась модель скрытого политического апартеида}, Алексей Кущ, strana.ua, 06.05.2021

Слабоумие и отвага в исполнении секретаря СНБО Данилова. Секретарь СНБО Данилов
полагает, что Германия и Франция частично ответственны за аннексию Крыма, так
как не дали \emph{Украине} ПДЧ по НАТО.  «Должны ли за это нести
ответственность Германия и Франция? Я считаю, что должны», - заявил секретарь
СНБО.  Это слабумие и отвагу нужно расценивать только как очередное публичное
признание факта внешнеполитического банкротства зе-команды,
\textbf{Вся Зе-политика заключается в двух действиях: позорить и предавать},
Елена Лукаш, strana.ua, 06.05.2021

Неслабое такое противостояние намечается. 20 млн \emph{украинцев}, живущих
менее, чем на 4000 грн в месяц против миллиона зажравшихся чиновников,
накопивших в своих маетках около триллиона долларов кэша. История знает массу
примеров, как такие противоречия разрешались. Жгите дальше, зеленинцы!, 
\textbf{Власть хочет добраться до каждой сотки и каждого квадратного метра},
Павел Себастьянович, strana.ua, 06.05.2021

Вопрос не в государстве \emph{Украина} — с ним уже всё понятно, но в сохранении
на этой территории остатков здравого смысла и человечности, без чего
возрождение цивилизованной жизни окажется невозможно здесь в принципе,
\textbf{Русский вопрос. И нацистский ответ}, Константин Кеворкян, ukraina.ru, 01.06.2021

Тем временем, Тищенко уже хочет более тесного диалога с венгерской стороной,
причем сразу на серьезном уровне. По информации издания, он обратился в
посольство с просьбой встретиться с послом Иштваном Ийдярто. Правда, непонятно,
в желаемом статусе председателя группы дружбы или как простой ее участник,
каковым он до сих пор указан на парламентском сайте...  Как сообщают
\enquote{Стране} источники в дипкорпусе, скандал с руководством в
украинско-венгерской группе не остался без внимания посольства.  Собеседник
среди участников группы считает, что на венгров ситуация произведет невыгодное
\emph{Украине} впечатление,
\textbf{\enquote{Ты же Оболонский, куда ты прешься?}. Почему Шуфрич послал Тищенко с трибуны Рады на венгерском языке},
strana.ua, 06.05.2021

Он стоял на краю огорода перед белым полем, уходившим покатым, широким языком
вниз, а потом так же плавно поднимавшимся вверх, к Ждановке. Там, где лежал
заснеженный горизонт, прятались укрепления \emph{украинской} армии. Разглядеть
их отсюда Сергеич не мог. Далеко, да и зрение оставляло желать лучшего. Справа
уходила туда же, вверх по пологому подъему, местами густая, местами жидкая
лесополоса ветрозащиты. Правда, вверх начинала она уходить только от излома
земли, а до поворота на Ждановку высажены были деревья ровной линией вдоль
грунтовки, которая сейчас смирно лежала под снегом, так как с начала военных
действий никто по ней не ездил. А когда до весны 2014-го ездили, то доехать
могли и до Светлого, и до Калиновки,
\textbf{Серые Пчелы}, Андрей Курков

Увы, но \emph{Украина} давно уже превратилась в неизлечимого шизофреника с массой
навязчивых идей. И никакие лекарства здесь уже не помогут. Однако, наблюдая за
очередным бенефисом г-на Кулебы, все же хочется напомнить высказывание Сергея
Лаврова, который ещё в 2014-м году, ставя на место все того же невменяемого
Дещицу, припечатал: «у банкрота нет прав что-либо требовать, он может только
просить».  Впрочем, вряд ли власти \emph{Украины} осознают, насколько глупо и жалко
выглядят на фоне подобных заявлений. Поэтому ждать от них адекватности —
бессмысленная трата времени.  И в этой связи только и остаётся, что занять
место в первом ряду и наблюдать за тем, как соседняя страна с маниакальным
упорством пытается повторить трюк, приписываемый унтер-офицерской вдове и
постараться не разориться на попкорне,
\citTitle{Украина как унтер-офицерская вдова - Голос Севастополя - новости
Новороссии, ситуация на Украине сегодня}, , voicesevas.ru, 06.06.2021

Допускаю, что вскоре мы станем свидетелями душераздирающего флешмоба «не дай
белорусу!» Вероятно, найдутся желающие покидать яйцами и зеленкой в белорусское
посольство.  Возможно, \emph{украинские} радикалы даже песенку про Лукашенко
споют. Ту самую, которую они поют про Путина. Не знаю, станет ли Кулеба (по
примеру Дещицы) подпевать, но если будет — то ничуть не удивлюсь. Потому как
подобные напевы очень повышают концентрацию гидности и положительно влияют на
чувство собственного величия, которое в такие моменты приобретает буквально
вселенские размеры,
\citTitle{Украина как унтер-офицерская вдова - Голос Севастополя - новости
Новороссии, ситуация на Украине сегодня}, , voicesevas.ru, 06.06.2021

Краеугольный камень \emph{украинской} пропаганды - весьмирснами. \emph{Украина}
- форпост на пути азиатомосковских орд, а запад надёжный тыл \emph{незалэжной}
на пути русских агрессоров.  Всё бы хорошо, но эта действительсть существует
лишь в мечтах свидомых патриотов, реальность несколько иная... Совсем иная,
кардинально иная...  Запад, страны западной Европы и Северной Америки предельно
прагматичны, обидно прагматичны для \emph{Украины},
\textbf{Очередная зрадаперемога}, Мак Сим, zen.yandex.ru, 06.06.2021

У-Га-Га. Вангую, скоро \emph{Украина} всё же будет получать именно европейский
газ (в смысле из Европы). Прагматичные европейцы с легкостью обдурят
\emph{украинскую} глупышку, получат ВЕСЬ газовый транзит, а потом ещё и начнут
в \emph{незалэжную} еще реэкспорт. А \emph{Украина} ведёт себя как глупая, доверчивая
девица, верящая что её вот-вот возьмут замуж (ЕС, НАТО), хотя она же должна
быть жёсткой, циничной, никому не верящей бабой Ягой, к чьим ушам лапша в
принципе не прилипает. Мля, ей же 300000 лет, ума должно быть больше чем в
России газа, а её запад раз за разом как малолетку разводит... ,
\textbf{Очередная зрадаперемога}, Мак Сим, zen.yandex.ru, 06.06.2021

«А как же \emph{украинская разведка}? Лучшая служба в мире, которая ловит шпионов с
фотоаппаратами выпуска 70-х годов прошлого века? Разве они тоже бессильны? Ай
да «Темнейший!» — пишет Валерий Р.,
\citTitle{«Он везде!» — «Никто не знает, где Путин появится завтра»}, Оксана Переможенко, regnum.ru, 06.06.2021

Крым безвозвратно утерян для \emph{Украины}, и всё, что от него осталось для Киева, —
лишь контур. Об этом в воскресенье, 6 июня, заявил российский сенатор Алексей
Пушков, комментируя форму футболистов \emph{украинской} сборной для Евро-2020, на
которой был изображен контур страны с Крымским полуостровом в ее составе.
«\emph{Украина} напрасно решила нанести контур Крыма на футболки игроков своей
сборной. Вернуть Крым это ей не поможет. Крым достался \emph{Украине} по ошибке и был
утрачен ею по глупости. И теперь уже он для нее безвозвратно утерян. От Крыма
остался для \emph{Украины} лишь контур», — написал российский политик в своем
Telegram-канале,
\citTitle{Пушков прокомментировал форму футболистов Украины с контуром Крыма}, ,iz.ru, 06.06.2021

В городе Запорожье на \emph{Украине} задержали мужчину, который угрожал взорвать
торговый центр и требовал \$1 млн и вертолет. Об этом 7 июня сообщила
Национальная полиция \emph{Украины},
\citTitle{Угрожавший взорвать ТЦ на Украине потребовал вертолет и \$1 млн}, , iz.ru, 07.06.2021

Новость, которая лично меня сильно радует. Госагентство туризма анонсировала
возвращение «Ракет». Глава госоргана Марьяна Олеськив пообещала, что по Днепр
вскоре снова будут курсировать скоростные теплоходы «Ракета». Первым планируют
запустить маршрут «Киев-Канев» с остановками в других городах. Предварительно,
путь составит около двух часов. Проблема в том, что из флота в 150 таких
кораблей времен СССР в \emph{Украине} сейчас осталось только 4 судна на подводных
крыльях. И те курсируют в Николаеве. Это ж их построить заново надо где-то
будет. Но я с удовольствием проедусь по маршрутам детства,
\citTitle{Большая посадка, беспритульный Голос, отступающий коронавирус...}, Денис Безлюдько, strana.ua, 07.06.2021

Продолжается скандал с \emph{украинской} футбольной формой. УЕФА утвердила новый
дизайн с контурами карты Украины с оккупированными территориями Донбасса и
Крыма. Кроме этого, на футболках присутствует лозунг «Слава \emph{Украине}» (он, кста,
с 2018 используется) и с изнанки «Героям слава». У мышебратьев, ессесно, стало
подгорать. Особенно, учитывая, что они из-за допинговых скандалов себе
нормальную форму позволить не могут. Да и успехи на футбольном поприще у РФ,
скажем, так себе. Если еще и проиграют сборной \emph{Украины}, то отапливать всю
Сибирь можно будет подгоранием пуканов,
\citTitle{Большая посадка, беспритульный Голос, отступающий коронавирус...}, Денис Безлюдько, strana.ua, 07.06.2021

Обсуждают новую форму \emph{украинской} футбольной сборной, с вышитым лозунгом
\enquote{Слава \emph{Украине} - Героям слава}. Как известно, это вариация
слогана, который был утвержден во время второго конгресса Организации
\emph{украинских} националистов - в сочетании с арийским поднятием правой руки.
Потому что конгресс проходил в 1939 году, в городе Риме, под отческим
покровительством Муссолини,
\citTitle{Презентованы футболки сборной Украины с лозунгом националистов}, Андрей Манчук, strana.ua, 07.06.2021

В посольстве США прокомментировали новую форму сборной Украины по футболу,
представленную 6 июня.  Посвященный форме пост появился в Фейсбуке
дипломатического представительства в понедельник, 7 июня.  \enquote{Нам нравится новая
форма. Слава Украине!} - гласит пост, отмеченный тэгом \enquote{Крым - это Украина}.  А
в комментариях пресс-служба посольства добавила: \enquote{Мы знаем, как приятно иметь
новый комплект}. Причем этот комментарий сопровождается снимком ликующего
Алекси Лаласа - звезды футбола в США 1990-х, защитника сборной команды страны,
\citTitle{Посольство США одобрило новую форму сборной Украины}, Наталья Полулях, strana.ua, 07.06.2021

Для того щоб зрозуміти чому \emph{Україна} опинилась в такому жахливому стані,
обов'язково потрібно детально проаналізувати основні події у центрі Києва
восени 2013 – взимку 2014 років, бо як написав Ігор Губерман: \enquote{В пучине наших
бедствий спят корни всей кручины, мы лечимся от следствий, а нас е@ут причины}.
Безумовно що заяви Януковича про євроінтеграцію були блефом, за допомогою якого
він намагався отримати від Росії фінансову допомогу, необхідну йому для
зміцнення авторитарного режиму. Кремлівський ідеолог Владислав Сурков вважав,
що на території \emph{України} в 2013 році вже не існує держави, бо міністерства та
силові структури знаходились під контролем неофіційних кураторів – смотрящих.
Фактично, Янукович перетворив \emph{Україну} в вотчину для свого родинного клану, що
дуже не сподобалось олігархам, які намагались послабити позиції \enquote{сім'ї} за
рахунок Революції Гідності. Представники великого бізнесу були дуже не
задоволенні діями та амбіціями старшого сина президента – Олександра Януковича,
який \enquote{розбудовуючи сім'ю} почав контролювати МВС, Податкову службу, Нацбанк,
Міністерство природніх ресурсів. Амбіції сина Януковича викликали занепокоєння
основних олігархічних \emph{українських} груп, які врешті – решт перейшли в
контрнаступ, скориставшись Революцією Гідності,
\citTitle{Про Революцію Гідності та її наслідки}, Стефан Закревський, analytics.hvylya.net, 06.06.2021

Якщо \emph{українці} дозволяють геополітичним гравцям використовувати \emph{Україну} в їхніх
інтересах, то цю країну ніхто не буде поважати і дбати про неї. На слабких не
зважають, їх повчають, принижують і використовують,
\citTitle{Про Революцію Гідності та її наслідки}, Стефан Закревський, analytics.hvylya.net, 06.06.2021

Для цього ветерани та медійники створили подію у Фейсбук і запросили інших
ветеранів, волонтерів та журналістів, які були в у місті Щастя, Луганської
області через сторінку \enquote{Луганський фронт Батальон Айдар}. Саме ця сторінка була
центральним інформаційним та координаційним органом щодо організації та
проведення цього заходу.  Зазвичай в нашій країні відзначають лише трагедії і
мало згадають про успішні бойові операції. А це конче необхідно для формування
бойового духу \emph{українців}. Ось саме тому і був задуманий цей масштабний захід,
щоб привернути увагу ЗМІ та почати змінювати цю традицію,
\citTitle{Український сегмент Facebook під контролем росіян. Що з цим робити?}, Виталий Кулик, analytics.hvylya.net, 07.06.2021

Це не єдиний випадок, коли адміністрація Facebook без попередження та права на
оскарження видаляє акаунти українських користувачів, які загрожують російській
політиці в Україні.  Так, паралельно з \enquote{Айдаром} було видалено
більшість сторінок \enquote{Правого Сектору}. Постраждала навіть сторінка
\enquote{Офтальмологічний центр Іріс}, де айдарівець Олексій Онасенко був
адміном.  Раніше за схожих обставин було видалено сторінку проекту
\enquote{Блеск и нищета \enquote{русского мира}} (радіопрограми, яка виходить
на українському радіо) (нова сторінка та особисті акаунти його авторів: Дмитра
Левуся та Олега Лісного,
\citTitle{Український сегмент Facebook під контролем росіян. Що з цим робити?}, Виталий Кулик, analytics.hvylya.net, 07.06.2021

По-третє, творити свою мережу. Якщо ти не контролюєш правила, мережа контролює
тебе. Отже знову на порядку денному (як в 2013-2014 рр) – створення \emph{української}
мережі, здатної стати помітною альтернативою Facebook в \emph{Україні}. Я розумію, що
це потребує астрономічних інвестицій та лобістських зусиль. Ніхто поки не каже
про заміщення Facebook якимось аналогом \enquote{інтронету} (за
північнокорейським зразком). Можна піти в ТікТок, Twitter, Інстаграм чи
Телеграм, але їх власники можуть грати в ту ж гру, що і Facebook. Нам потрібні
альтернативні відкриті мережі, де ми – користувачі, впливаємо на правила
співтовариства,
\citTitle{Український сегмент Facebook під контролем росіян. Що з цим робити?}, Виталий Кулик, analytics.hvylya.net, 07.06.2021

Учит ли это чему-то \emph{Украину}? И если учит, то в чем состоят ключевые
собственно \emph{украинские интересы}, а в чем налет фэн-шуя, который выглядит
красиво, но не всегда ощущается приложенным к нужному месту.  Есть ли вообще у
\emph{Украины} видение, о чем договариваться с Россией, если уж начинать
договариваться?  Чтобы что-то увидеть впереди, надо увидеть и трезво оценить
то, что есть сейчас.  Несложно просчитать, что будет реально маячить за
российской позицией на переговорах, что бы ни произносилось формально,
\citTitle{О судьбе украинского эксперимента и стратегическом здравомыслии}, Дмитрий Пастернак-Таранушенко, analytics.hvylya.net, 04.06.2021

Мы вас научим родину любить!  За последние сто лет власть в \emph{Украине} менялась
раз двадцать, а то и тридцать. Ломался один строй и строился другой, приходили
и изгонялись оккупанты и их пособники, происходили революции и перевороты,
свергались «режимы», клеймились «культы», к власти приходили разные
политические силы. И все они (ну, почти все) били себя в грудь и назывались
патриотами, пытаясь учить других «правильно» любить родину. Вот только
представления о родине у них были часто противоположные, да и представления о
любви к ней тоже.  Однако при всей их разнице формула патриотизма у них была
одинаковая и до сих пор остается той же самой: за веру, царя и отечество!
Именно в таком порядке: на первом месте стоит идеология, на втором власть или
партийные вожди, и только на третьем само отечество и интересы его граждан,
\textbf{«За веру, царя и отечество!»: разновидности украинского патриотизма},
Виктор Дяченко, from-ua.com, 14.04.2016

Ще однієї диктатури у центрі Європи ніхто не допустить! Тим більше диктатури не
освідченої!  Так що привіт де олігархізації від Зе!  Цій владі краще прямо
сказати: ми ховаємо ключове питання крадіжки \emph{Української Землі} з її
багатствами у \emph{українського народу} за примарою закону про олігархів, за примарою
референдуму про олігархів ... за усякою х....ю, а тепер і до обіцянки посадити
мільйон, сорі, мільярд дерев добралися!  От тут, нехай тільки попробує не
посадить!!!  Дерева або є, або їх немає! Але ж як перерахувати!!! От то
бреше... без обмежень,
\citTitle{Еще одной диктатуры в центре Европы никто не допустит!}, Марина Ставнийчук, strana.ua, 08.06.2021

Он рассказал, что во время телефонного разговора с Байденом тот дал ему
\enquote{прямые сигналы} против газопровода.  \enquote{Я действительно думал,
что, когда дело дошло до второго \enquote{Северного потока}, США оставались,
так сказать, последним форпостом. Мы понимаем, что только США способны
остановить это строительство. Знаете, мы поднимали эту тему впервые во время
разговора с президентом Байденом, это был телефонный разговор, и я получил все
сигналы. Это были прямые сигналы, и я был очень этому рад}, – вспомнил
Зеленский.  Но дальнейшее вызвало у президента \emph{Украины}
\enquote{неприятное удивление}.  \enquote{Я всегда говорил, да и все эксперты с
дипломатами единогласно считали, что Байден знает \emph{Украину} лучше всех
других президентов, а значит разбирается во всех вопросах и, самое главное,
понимает все риски безопасности. Вот почему, опять же, мы были очень неприятно
удивлены}, – сказал \emph{украинский} президент,
\citTitle{Зеленский интервью Axios – зрада по Северному потоку, НАТО и встрече с Путиным}, Максим Минин, strana.ua, 07.06.2021

ПРО ПРОЕКТ.  \enquote{В Україні дедалі складніше вести бізнес}, \enquote{медицина занепадає},
\enquote{розумні, молоді та перспективні тікають за кордон} – ці скиглення уже набридли
всім.  Кожен українець по-своєму прагне незалежності та хоче досягати успіхів.
До Дня Незалежності \enquote{Українська правда. Життя} та комунікаційна агенція \enquote{ВАРТО}
запускають новий проект \enquote{Варто пишатись}.  У ньому ми розповімо про українські
компанії, яким вдається втілювати свої амбітні плани, змінювати життя людей на
краще не лише в Україні, а й за її межами.  \enquote{Варто пишатись} – це проект про
досягнення українського бізнесу в різних галузях економіки.  Це проект про нас,
українців. Про те, що нам варто пишатися собою й тими, хто робить нашу країну
сильнішою, докладаючи зусиль для її незалежності.  Партнер проекту –
комунікаційна агенція \enquote{ВАРТО}, Варто пишатися, pravda.com.ua

Вчора на болотах Московії в районі Кремля почалася істерика, яка все ще не
заспокоїлася. І, судячи з усього, розриває їх не по-дитячому. Причина в тому,
що вчора презентували нову форму збірної \emph{України} з футболу. Форма ефектна й
гарна. З тисненням карти України. А на карті, як і належить, є Крим, який
анексувала Росія, і є ОРДЛО, яке окупувала Росія. І Кремлю це чомусь не
сподобалося. Красти наші землі подобається, вбивати наших громадян подобається,
фінансувати підконтрольних терористів їм подобається і дуже подобається
вивозити з окупованих територій заводи та підприємства у РФ. Тож ці мародери
перехвилювалися й у них почався словесний понос, суміщений з театром абсурду,
\citTitle{Нова форма збірної України з футболу - реакцію Росії аналізує Борислав Береза}, , fakty.ua, 06.07.2021

Форма в национальных цветах – желто-голубом – продумана. Правда, в
представленном экземпляре больше желтого почему-то. Хотя вот прочитала, что
испанская фирма Joma, которая занималась разработкой дизайна спортивной формы
для футболистов \emph{украинской} сборной, на всякий случай выпустила три
варианта: желтый, голубой и белый. Интересно, к чему бы это? Ну, голубой –
понятно, чтобы доказать свою толерантность. Желтый – чтобы притягивать
солнечную энергию и победу? А белый – на случай, если придется сдаваться? Шучу,
шучу, \emph{украинская сборная} нацелена только на победу! Они уже «победили»
своего «ворога и агрессора» и «вернули Крым и Донбасс»! Во всяком случае именно
это у них на футболках начертано!,
\citTitle{Форма не главное – главное содержание! И в спорте тоже...}, Мысли Бабы Яги, zen.yandex.ru, 07.06.2021

Когда \emph{укры} будут играть в европах, пусть играют в чем хотят. Но ведь они
работают с дальним прицелом: четверть финал (если пройдут) будут играть в
С-Петербурге! Приедут про фашистские болельщики, с речевками:
\enquote{\emph{слава украине - героям слава}}, зигхалями, факельными шествиями
и т.п.провокациями. И что, проглотим? А может и дальше будем глотать
\enquote{москаляку на гиляку}, \enquote{москалив на ножи}? Это в российском
МИДе считают, что спорт вне политики( их эпопея с отстранением и санкциями в
отношении российских спортсменов ничему не научила). Запад давно использует
спорт, как самый эффективный рычаг давления на молодежь россии, а мы все
считаем, что это детские шалости. Нам плюют в лицо, лишив флага и гимна,
сначала на ОИ, я затем на всех ЧМ., а наши чиновники все надеются, что это
временно и в 2023 году с нас снимут санкции. Не надейтесь, не снимут.
Соплежуйство, примиренчество, порождают безнаказанность, а затем и
вседозволенность,
\citComment{Kovalev},
\citTitle{Форма не главное – главное содержание! И в спорте тоже...}, Мысли Бабы Яги, zen.yandex.ru, 07.06.2021

Любаша I Мира и здоровья Вам и нашей \emph{Украине}! Вы Любовь, а я Надежда. Будем
жить и ждать Веру в светлое будущее нашей страны Удачи Вам Дорогой Человек! С
любовью и уважением Надежда 7 8 лет 2. Харьков,
\citComment{Надія Капліна}, 
\citTitle{Cpoчнo! PE3KOE oбpaщeниe Tитapeнкo к 3eлeнcкoмy ШOKИPOBAЛO Kиeв!},
youtube.com, 07.06.2021

В \emph{УССР} он занимал наивысшую для писателя должность - 12 лет был главой Союза
писателей \emph{Советской Украины}. Но после распада Союза Гончар поддержал
независимость \emph{Украины}, стал критиковать коммунистов, Россию, выступать за
\enquote{дерусификацию} и единый государственный язык - \emph{украинский}.  Пост-советские
взгляды Олеся Гончара компактно изложены в его \enquote{Дневниках} - третьей части,
которая охватывает период после распада Союза и почти до смерти писателя. Они
были изданы в 2004 году,
\citTitle{Донбасс как раковая опухоль. Есть ли это в дневниках Олеся Гончара},
Максим Минин; Екатерина Терехова, strana.ua, 08.06.2021

Блогерша из Чугуева Харьковской области Яна Коваленко обозвала \emph{украинских
военнослужащих} \enquote{тварями}, \enquote{наркоманами} и \enquote{алкашами} и
заявила, что их не за что уважать. \enquote{Та блин, это позорище. А что
сказать хорошего, то, что они вонючие, грязные, обрыганные, ходят бухие по
городу или обдолбанные. Где этот военнослужащий, которого можно уважать за
что-то.  Против кого они воюют. Не трогайте меня по этой теме. Я размотаю
любого и каждого. Ненавижу этих тварей, просто ненавижу. По-другому я и сказать
не могу. И мы им платим налоги. Вот этим нахлебникам, которые сидят и нихера не
делают.  Таким, блин, ублюдкам, которые не хотят работать. Наркоманам и
алкашам, которые идут туда}, - говорит девушка,
\citTitle{Блогерша Яна Коваленко обозвала украинских военнослужащих тварями}, Владислав Бовтрук, strana.ua, 08.06.2021

Как узнала \enquote{Страна} из собственных источников в патрульной полиции,
один из посетителей заправки был избит до крови из-за маски.  Конфликт
произошел между двумя клиентами заправки, которые стояли рядом у прилавка. На
одном из них была плотно надета на лицо медицинская маска, и ему не
понравилось, что стоящий по соседству мужчина стянул свою маску на подбородок.
Он сделал замечание и тут же поплатился за это. Его оппоненту пришелся не по
душе менторский тон визави, и он без лишних слов избил его прямо при персонале
и других клиентах заправки.  Как ни странно, никто не вступился за
потерпевшего. Только когда он упал, а пол вокруг залила кровь из разбитого
лица, нападавший сам прекратил драку и спокойно ушел по свои делам.  Полиция
начала расследования по факту нанесения побоев, агрессивного противника
правильного ношения масок сейчас разыскивают.  Ему грозит по ч.1 ст.296
Уголовного кодекса \emph{Украины} (хулиганство) до пяти лет ограничения
свободы. Если у потерпевшего будут диагностированы переломы, то действия
нападавшего также подпадут под ч.1 ст.122 УК \emph{Украины} (причинение
телесных повреждений средней тяжести). В этом случае ответственность более
жесткая - до трех лет тюрьмы,
\citTitle{В Киеве за замечание о маске был избит клиента заправки}, Дмитрий Войко, kiev.strana.ua, 08.06.2021

Я бы стал президентом Украины только для того, чтобы под конец каденции сменить
гимн.  Не в смысле \enquote{поставить задачу креативной группе}, а принять
страну под \enquote{Ще не вмерла} и сложить полномочия под \enquote{Живи,
Україно, прекрасна і сильна}.  Причём так, чтобы это не звучало, как анекдот. И
потом спокойно доживать в прекрасном и сильном государстве. Домино, рыбалка... ,
\citTitle{Хороший президент - тот, который оставляет после себя прекрасную и сильную страну}, Стас Косаренко, strana.ua, 08.06.2021

Вот они и выдумывают фейки о членстве \emph{Украины} в НАТО.  В \emph{Украине} неправильно
подали заявления по итогам телефонного разговора между президентами \emph{Украины} и
США Владимиром Зеленским и Джо Байденом. В заявлении \emph{украинской} стороны
изначально говорилось, что Байден «подчеркнул... важность предоставления
\emph{украинскому} государству плана действий по членству в НАТО». Хотя Байден ничего
такого не говорил.  З.Ы. Сейчас сообщение о беседе Зеленского с Байденом на
сайте ОП исправлено. Ложечки нашлись.  Манипуляции с информацией – ключевое
умение для \emph{украинского} медиа-менеджера, да. Для этого этих людей в ОП на работу
и брали. Но в Вашингтоне явно не оценили,

Вот они и выдумывают фейки о членстве \emph{Украины} в НАТО В Украине
неправильно подали заявления по итогам телефонного разговора между президентами
Украины и США Владимиром Зеленским и Джо Байденом. В заявлении украинской
стороны изначально говорилось, что Байден «подчеркнул... важность
предоставления \emph{украинскому государству} плана действий по членству в
НАТО». Хотя Байден ничего такого не говорил.  З.Ы. Сейчас сообщение о беседе
Зеленского с Байденом на сайте ОП исправлено. Ложечки нашлись.  Манипуляции с
информацией – ключевое умение для украинского медиа-менеджера, да. Для этого
этих людей в ОП на работу и брали. Но в Вашингтоне явно не оценили,
\citTitle{Офис президента Украины опозорился с фейком о разговоре Байдена с Зеленским}, Вячеслав Чечило, strana.ua, 08.06.2021

Недавнее заявление Зеленского о том, что в ближайшие три года в \emph{Украине}
высадят миллиард новых деревьев, а площади лесов увеличат на миллион гектар,
взбудоражило общественность. Напомним: активные \enquote{посадки} он пообещал
на форуме \enquote{Украина 30 Экология}, и в тот же день, 7 июня, появился указ
президента о мероприятиях по сохранению и восстановлению лесов.  С лесами в
\emph{Украине} пробема, и это в общем-то не секрет. Как и то, что идут
несанкционированные вырубки ценных реликтовых лесов в Карпатах, а наша страна,
несмотря на мораторий, активно отгружает лес-кругляк на мебельные производства
в Европу.  То есть леса спасать однозначно надо. Но сработает ли президентская
идея и реально ли вообще высадить миллиард деревьев за в три года - вопрос.  В
соцсетях идут активные обсуждения - сколько деревьев в секунду нужно
высаживать, чтобы выйти на озвученный Зеленским миллиард. И в какую сумму в
итоге это обойдется бюджету,
\citTitle{Реально ли посадить миллиард деревьев от Зеленского}, Людмила Ксенз, strana.ua, 08.06.2021

\emph{Украина} – страна парадоксов. Как у большинства соотечественников не возникает
когнитивный диссонанс до сих пор не понимаю. По опросу КМИС большинство
украинцев (54,5\%) против выдвижения Зеленского на второй срок, «за» - 37\%. При
этом, действующий президент гарантировано выходит во второй тур выборов с
хорошим отрывом. По данным того же опроса, Зеленский набирает в первом туре
27,3\% голосов. Его вечный оппонент Порошенко – 14,6\%. По второму туру не
опрашивали, но я думаю, что шестой президент сможет повторить свой триумф
двухлетней давности,
\citTitle{Порошенко примазался, Зеленский сажает, украинцы недовольны...}, Денис Безлюдько, strana.ua, 08.06.2021

На днях секретарь Совета нацбезопасности и обороны \emph{Украины} Алексей Данилов
заявил, что в \emph{Украине} не осталось ни одного вора в законе. \enquote{Страна} уже писала
о том, что эта информация не соответствует действительности: несколько
отечественных воров, которые попали в санкционные списки, продолжают жить в
стране.  Есть проблемы и с другими ворами: например, на днях на Львовщине суды
выпустили на свободу задержанных СБУ 28 мая во Львовской области воров в законе
Виктора Панюшина по кличке Витя Пан и Реваза Катамадзе по кличке Магало. У
обоих оказались документы, позволяющие им жить в \emph{Украине}. То есть по факту
попадание в списки СНБО еще не означает выдворение из страны.  Как видим, суды,
мягко говоря, игнорируют решения Совета национальной безопасности.  Но даже
если когда-нибудь \emph{Украине} и удастся выдворить всех без исключения воров в
законе, криминальная ситуация от этого вряд ли изменится. Так как воры еще со
времен советской власти освоили популярные ныне методики \enquote{дистанционки} и
\enquote{удаленки},
\citTitle{Что изменили в криминальном мире санкции СНБО против воров в законе}, Александр Сибирцев, strana.ua, 08.06.2021

\emph{Україна} однією з перших скасувала авіасполучення з Білоруссю. Але відновити
рейси зараз вимагають \emph{українці}, у яких у сусідній країні запланована операція
пересадки внутрішніх органів.  В \emph{Україні} таких людей більше ніж три сотні, усім
їм держава вже сплатила за лікування,
\citTitle{Українці, які чекають на трансплантацію органів у Білорусі, просять
відновити авіасполучення між країнами (відео)}, Настоящее Время,
www.radiosvoboda.org, 08.06.2021

Недавно в \emph{Украине} вспыхнул очередной скандал, связанный с высказываниями
отечественных блогеров о ситуации в стране. На этот раз в неприятную историю
попала Яна Коваленко, блогерша из города Чугуев Харьковской области. Девушка на
своей странице в Instagram назвала \emph{украинских военнослужащих} «тварями»,
«наркоманами» и «алкашами». Кроме того, она заявила, что \emph{украинских
военных} не за что уважать.  Буквально через несколько часов Яне Коваленко уже
пришлось извиняться за сказанное, поскольку ей начали звонить с неизвестных
номеров и обвинять в оскорблении всей украинской армии.  Редакция «Шарий.net»
рассказывает об очередном Instagram-скандале, связанном с подавлением
инакомыслия в стране,
\citTitle{«Ботоксное животное» и «тухлодырое сосалище»: как патриоты накинулись на блогершу из-за видео против военных}, , sharij.net, 08.06.2021

Бывшая народный депутат Елена Бондаренко заявила, что будет делать адвокатский
запрос из-за визита Службы безопасности Украины (СБУ).  Об этом она рассказала
в комментарии \enquote{Стране}.  \enquote{Сейчас ничего не известно, но будем
делать запрос для того, чтобы понимать, что им надо. Тем более это не первая
такая их ходка, они делают их эпизодически. Последняя ходка была год назад,
идентичная ситуация. Они не сказали, по какому делу, не показали повестку.
Развернулись и ушли}, - сообщила экс-народная депутатка.  Основной причиной
визита Бондаренко назвала общественную деятельность и активную позицию, которая
\enquote{приносит дискомфорт тем, кто ее слышит},
\citTitle{Елена Бондаренко рассказала о деталях визита СБУ}, Владислав Бовтрук, strana.ua, 08.06.2021

Но пост о другом! \emph{Украина} умнее и хитрее! Мы то уже давно празднуем то,
за что поляки на беларусов взъелись!  Наша дата - 14 октября. День
присоединения \emph{Западной Украины} к \emph{УССР}. И ниче... поляки молчат.  Просто
мы хитро совместили это с Покровой, днем козацтва и защитника \emph{Украины}.
Учитесь товарищи беларусы нашей хитрости!,
\citTitle{Украинцы знают, как не обидеть поляков}, Михаил Чаплыга, strana.ua, 09.06.2021

Возможно, британских экспертов покусали \emph{украинские} коллеги?, Дмитрий Василец, strana.ua, 09.06.2021

Поэтому лучше вообще не ходить в провинциальный \emph{украинский театр}.  Мы
живем в типичном гидеборовском обществе спектакля. Реальность мало кого
интересует.  Пленки Бигуса тоже абсолютно постановочны - не зависимо от их
подлинности.  И абсолютно постановочна реакция так называемого - абсолютно
постановочного! - \enquote{гражданского общества} на них.  Есть категории
людей, позиционирующих себя как ломов с \enquote{безупречной нравственной
позицией}.  Их возмущение моральным обликом Порошенко после обнародования
записей велико, а Медведчук вызывает вообще запредельное отвращение. Но при
этом они забывают простую вещь - Порошенко и Медведчук тоже существуют в
области постановочного \enquote{знака}.  Публика, купившая билеты на их
спектакли и аудитория, апплодирующая постановке Бигуса - это разные целевые
аудитории. Это другая публика.  Хочешь выиграть в казино - купи себе казино.
Хочешь ставить спектакли - купи театр.  Но лучше вообще не ходите в
\emph{провинциальный украинский театр} и не апплодируйте дешевым
\emph{украинским актерам},
\citTitle{Мы живем в обществе, превращенном в спектакль / Лента соцсетей / Страна}, Владислав Михеев, strana.ua, 09.06.2021

Уполномоченный по защите украинского языка Тарас Креминь направил в Нацсовет
жалобу на три телеканала из-за вещания в прайм-тайм на русском языке. При этом
у нарушителей объем украинского языка \enquote{не дотягивал} всего 3-5\% до \enquote{нормы} в
75\% эфирного объема Об этом говорится в публикации на странице Креминя в
Facebook Секретариат Уполномоченного по защите государственного языка
осуществил оперативный мониторинг вечернего эфира (прайм-тайм с 18:00 до 22:00)
четырех украинских телеканалов с 17 по 23 мая: \enquote{ICTV}, \enquote{1 + 1}, \enquote{Интер} и
\enquote{Украина}.  Кремень напомнил, что на украинском должно быть не менее 75\% от
общей продолжительности передач и/или фильмов в промежутке времени между 18.00
и 22.00, (а не от общей продолжительности эфирного контента в целом), поэтому
выборка производилась именно за такой временной промежуток,
\citTitle{Мовный омбудсмен проверил Интер, 1+1, ТРК Украина и ICTV на нормы украинского языка}, Игорь Рец, strana.ua, 09.06.2021

\citEntry{%
У лютому-березні в \emph{Україні} анонсували створення двох центрів, які, на перший
погляд, мають спільне поле роботи – Центр протидії дезінформації при Раді
національної безпеки і оборони та Центр стратегічних комунікацій та
інформаційної безпеки при Міністерстві культури.  Перший очолила Поліна
Лисенко, другий – Любов Цибульська. Для обох інформаційні атаки – не щось
нове, але кожна дотикалась до них по-своєму
}{%
  \citTitle{(Поки що) без грошей, штату і приміщень. Як нові центри при РНБО та Мінкульті планують боротися за правду}, 
  Ольга Кириленко, www.pravda.com.ua, 09.06.2021
}

Взагалі, німці, в перші дні свого панування, намагалися показати народу, що
вони дбатимуть про нього не так, як більшовики. Але це тільки в перші дні. Дуже
швидко виявилося, що нацисти від сталіністів нічим не кращі. Почалися вже скоро
і нацистські репресії. Особливо це стосувалося \emph{українських} євреїв. У
Березані їх було не так і багато, а в Новозибкові на Стародубщині єврейська
громада була численна, і тут жахливу трагедію єврейського Голокосту моїм
родичам довелося побачити зблизька. Бабусі моїй, Ніні Терентіївні Остапчук,
було тоді 19 років, і закінчила вона перед війною перший курс математичного
факультету новозибківського педучилища. Розповіді її про те, як нацисти
розстрілювали євреїв у Новозибкові, навіки закарбувалися в моїй пам'яті
страшними, страхітливими картинами,
\citTitle{Відступ радянських військ з України 1941 року. Вони знищували матеріальні цінності}, 
Ігор Роздобудько, www.radiosvoboda.org, 08.06.2021

\citEntry{%
Такого у нас не наблюдалось с начала 90-х.  Вангую...  Осенью \emph{Украину} ждет
невиданный по масштабам кризис, глубину которого можно сопоставить только с
началом 90-х прошлого века. Тогда распалась государственность СССР и шло
строительство новой, украинской государственности. Теперь, похоже пришла
очередь за распадом \emph{украинской} государственности, которая последние семь лет
держалась исключительно на внешних и внутренних долгах.  Фактически речь о
государственности в долг в самом широком смысле этого слова. От экономики до
безопасности. Что имеем на сегодня в результате бездарной деятельности власти
от Порошенко и Зеленского?
}{%
\citTitle{На глазах нарастает грандиозный кризис украинской государственности}, Валерий Песецкий, strana.ua, 09.06.2021
}

Акванавты в степях \emph{Украины} (политологический фантазм). \emph{Украинский политик} -
явление очень своеобразное, и, можно сказать, уникальное. И эволюционирующее).
В \enquote{первом поколении}, в уже далекие 90-е, выходцы из номенклатуры,
\enquote{красно-директорских кабинетов} и диссидентской среды сочетали в себе качества
\enquote{землеробов} и \enquote{помещиков}. Такое \enquote{диссоциативное раздвоение личности} во
многом было обусловлено совковым комплексом \enquote{очередников за карьерой}. Ведь
длительное время карьера в \emph{Украине} (вернее, в УССР) рассматривалась лишь как
трамплин для дальнейшего успеха в советских центрах (вершина - работа в
Москве). Партийцы и комсомольцы, директора и артисты, журналисты и инженеры
ждали своих назначений и приглашений. А для этого нужно было \enquote{заработать}
карьеру - показателем, результатом, яркой индивидуальностью. Поэтому - работа и
ожидание, \enquote{очередь}. Но Независимость резко изменила этот сложившийся порядок
вещей. \enquote{Землеробская} повседневность вдруг сменилась брезжущим шансом стать
\enquote{помещиком} без очереди. Дикая борьба за право обрести положение \enquote{помещика} и
даже стать местным \enquote{Джорджем Вашингтоном}, \enquote{розбудувати державу} и создать без
очереди страну по своему образу и подобию, породила то самое поколение, которое
уже к концу 1990-х издавало многотысячными тиражами \enquote{Золотые страницы
украинской элиты}, создавало именные блоки на выборах, строило дворцы на
окраинах нищих городов и училось \enquote{говорить попроще} в ТВ-роликах,
\citTitle{Современные украинские политики - акванавты теплой ванны}, Андрей Ермолаев, strana.ua, 09.06.2021

Знаменитая дамба, сооруженная из известной субстанции и палок по
\emph{древнеукским} технологиям разрушается. \emph{Украина}, которая в 2014
году решила наказать крымчан за переход в состав России, рассчитывала оставить
полуостров без воды. По замыслам \emph{украинских} стратеХов водная блокада
должна была заставить жителей Тавриды передумать, снова полюбить
\emph{незалежную} всем сердцем.  Так же с целью максимально усложнить жизнь
новым гражданам РФ, устроить им еще энергетическую блокаду, были подорваны
опоры ЛЭП, но РФ решила эту проблему.  Транспортная блокада полуострова со
стороны Киева привела к появлению крымского моста, что в итоге лишь укрепило
связи между большой РФ и Крымом. Блокировка поставок продовольствия с
\emph{Украины} в Крым просто лишило \emph{украинских} сельхозпроизводителей
лакомого рынка сбыта, а их российские конкуренты не остались в накладе,
\citTitle{Блокада Крыма дала течь...}, Мак Сим, zen.yandex.ru, 08.06.2021

\emph{Українське суспільство} перебуває у стані шоку багато поколінь. Постійно
проживаємо якусь катастрофу. Катастрофа покоління наших батьків – Чорнобиль,
одразу по тому – розпад Радянського союзу, розповіла в інтерв'ю журналу
\enquote{Країна} культурологиня, директора \enquote{Мистецького Арсеналу} Олеся
Островська-Люта.  8 червня стартував онлайн продаж квитків на 10-й Міжнародний
фестиваль \enquote{Книжковий Арсенал}, який відбудеться 23-27 червня. Через пандемію
COVID-19 вхід на фестиваль можливий лише у певний часовий проміжок за слотами.
\enquote{На Книжковий Арсенал класи ніколи не приходять. Немає порозуміння між освітою
і сферою культури, – пояснює Островська-Люта. – У німецькому Ляйпциґу всі школи
в період проведення книжкового ярмарку отримують безкоштовні квитки. На
фестивалі маса дітей, вони сидять всюди під стінами, гортають книжки. В \emph{Україні}
школи дуже обмежені у своїй можливості відвідувати такі події. Це вважається
зламом навчального процесу. Та залежить великою мірою від ініціативи вчителя і
батьків. Але це не є частиною природного освітнього процесу.},
\citTitle{Книжковий Арсенал: суспільство перебуває у стані шоку}, gazeta.ua, 09.06.2021


УССД. \emph{Українська} Соборна Самостійна Держава — держава \emph{української} нації ("усіх
і мертвих, і живих, і ненароджених \emph{українців}, в \emph{Україні} і поза її межами
сущих") на предківських, визначених Богом і освоєних незліченними поколіннями
українців, землях.  Держава — дієвий механізм забезпечення усіх без виключення
потреб \emph{української нації} для її повноцінного Буття у часі просторі. \enquote{Своя хата,
в якій Своя \emph{українська Правда} і Сила, і Воля}. Заснована на властиво
\emph{українських традиціях}: права і урядування, духовності і культури, звичаїв і
цінностей, морально-етичних і естетичних засад, співіснування між собою та з
іншим світом.  Самостійна — не обмежена іншими державами, союзами держав,
світовими економічними корпораціями та іншими понаддержавними утвореннями, у
своєму волевиявленні і життєвих потребах, внутрішній і зовнішній політиці.
Соборна — окрім державної єдності усіх \emph{українських} етнічних територій є,
образно висловлюючись внутрішнім морем, до якого стікаються усі українські ріки
з усіх-усюд планети Земля. Духовним центром єднання усіх \emph{українців}, де б вони
не мешкали і водночас джерело духовності з якого черпатиме свою життєву наснагу
\emph{українство} нинішнє і майбутнє.  \emph{Українська} — не німецька (за німецьким
зразком), не італійська, не іспанська, не польська, не московська і не будь-яка
инша, а саме \emph{українська} — не тільки декларативно й атрибутивно; за буквою, але
за своєю найглибшою суттю, за своєю природою,
\citTitle{Нова Держава, за яку боремось — яка?}, , pravyysektor.info, 24.05.2017

Мій дід Петро Антонович Машков, був уродженцем міста Коломия на
Івано-Франківщині, але з 30-х років проживав в Лубнах на Полтавщині. В 1940
році його призвали на службу до РКЧА. Мій дід пройшов війну з першого до
останнього дня, закінчивши її у званні капітана артилерії. Петро Антонович був
на половину поляком, по матері, і вільно володів польською мовою. З огляду на
це, його, як кадрового офіцера, після закінчення війни направили у військо
польське, яке створювала совєтська влада в окупованій Польщі. Історія, яку я
хочу розповісти, сталася саме в цей період його життя. Одно разу, мій дід на
виклик відбув до штабу дивізії, а коли повернувся назад - знайшов вбитим свого
друга і побратима разом з яким вони пройшли всю війну. Як виявилось, поки він
їздив до штабу, в частину завітав заступник міністра оборони Польщі, генерал
Свірчевський, один із тих що готували операцію \enquote{Вісла} по \enquote{розукраїненню} краю.
Із-за якоїсь дурниці, по своєму самодурству Свірчевський застрелив дідового
друга. Мій дід був людиною честі, і керований праведним гнівом він схопився
наздоганяти Свірчевського щоб помститись за побратима. Але від'їхавши трохи
далі в гори, дід побачив розгромлений кортеж, вбитого генерала і його охорону.
Ті хто їх атакував були там же на місці - це були бандерівці. Дід підняв зброю
вгору і помахав руками, на знак того, що не буде стріляти. Бандерівці відповіли
зустрічним мирним вітанням. За кілька хвилин вони роз'їхались у різні сторони.
До останнього дня дід вважав, що тим, що вони ліквідували Свірчевського  -
врятували його. Адже якби йому вдалось наздогнати генерала - він би сам вступив
у бій, і неминуче був би за це розстріляний совєтською владою.  Все життя мій
дід Петро лишався \emph{українським патріотом}. Вже у 90-х роках він свідомо голосував
за блок \enquote{Національного фронту}, до п'ятірки якого входила Ярослава Стецько,
мотивуючи це своєю історію з життя, коли бандерівці допомогли йому, офіцеру
РКЧА, ліквідувати ворога України, \citAuthor{Ігор Мосійчук}, telegram, 09.05.2021

\enquote{Дух одвічної стихії, що зберіг нас від татарської повені, поставив нас
\emph{українців} на грані двох світів творити нове життя}. Тому перед нами, окрім
національної мети є ще й відповідальна місія — стати для навколишнього світу
прикладом творення національних держав вільних народів. Для збереження їх
національної унікальності на розмаїтій мапі національних культур. Стати оазою
серед пустелі, де панують вітри денаціоналізації, бездуховності, занепаду і
виродження. Наша боротьба не тільки за \emph{українську Україну}, але й за світ без
національного пригнічення, без імперій чи то територіальних чи фінансових. За
світ національних держав і взаємовигідного добросусідства, у якому однією з
найбільших чеснот буде плекання свого рідного, а зазіхання на чуже — гідне
осуду. З Богом до перемоги! Бо якщо не ми, то хто? Якщо не тепер, то коли?  А
коли з нами буде Бог і ми з Богом, то не важливо, хто стане проти нас!  Слава
\emph{Україні}!,
\citTitle{Нова Держава, за яку боремось — яка?}, , pravyysektor.info, 24.05.2017

В мировой рейтинг из 1300 лучших вузов мира, вошли 8 \emph{украинских}
университетов.  Это рекорд. В прошлом году было шесть. Правда, после 500
позиции. Итак: Харьковский национальный Университет им. Каразина покинул
топ-500 (511-520 место), Киевский национальный университет имени Тараса
Шевченко (601-650), Харьковский политехнический институт (651-700), Киевский
политехнический институт им. Сикорского (701-750), Сумской государственный
университет (701-750), Национальный университет Львовская политехника
(801-1000), Национальный университет Киево-Могилянская академия (1001-1200),
ЛНУ им. Франко (1001-1200). Мой нархоз опять не попал. Возглавил список
Массачусетский технологический институт. На втором месте - Оксфорд, а третье
место разделяют американский Стэнфорд и британский Кембридж.  А в этот день
ровно год назад, Киев взлетел сразу на 44 позиции в рейтинге самых дорогих
городов мира. На 106 место из 209 городов,
\citTitle{Всеобщая зелень, победа Вирастюка, университетский рейтинг...}, Денис Безлюдько, strana.ua, 09.06.2021

В \emph{Украине} хотят отменить льготы на проезд в общественном транспорте.
Нищеброды, слезайте. Приехали. "В \emph{Украине} могут отменить льготный проезд в
общественном транспорте. Местной власти хотят предоставить возможность
самостоятельно определять сумму, которую они готовы выплачивать льготникам в
качестве компенсации.  Соответствующий законопроект на заседании 9 июня
утвердил Кабмин. Документ предусматривает окончательную монетизацию льготного
проезда в транспорте, а также вносит ряд изменений в действующее
законодательство по льготному проезду.  Так, вопрос о том, будет ли льготный
проезд для ветеранов, чернобыльцев, инвалидов и других категорий граждан,
которые сейчас право на него имеют, после принятия нового закона окажется в
подвешенном состоянии,
\citTitle{Местным властям хотят дать право устанавливать и отменять льготы на проезд в транспорте}, 
Андрей Манчук, strana.ua, 10.06.2021

По мнению Путина, никакой реакции европейских организаций на данный
законопроект не будет. \enquote{С \emph{Украиной} строятся отношения таким
образом, что она рассматривается как антипод России. [А потому] \emph{Украину}
предпочитают не трогать... И в том числе не замечать свастики, с которой ходят
неонацисты по городам Украины}.  Владимир Путин высказал мнение, что принятие
законопроекта найдет серьезное отражение в \emph{украинском} обществе. По его
словам, согласно переписи населения советского периода, в \emph{Украине}
проживало много разных национальностей. \enquote{Никто же не захочет быть
людьми второго сорта! Я уже не говорю про дискриминацию по языку и другим
составляющим нормальной жизни человека}, – говорит Путин.  Вывод он делает
такой: \enquote{Это приведет к тому, что сотни тысяч, а может и миллионы,
вынуждены будут либо уехать, не желая быть людьми второго сорта, либо начнут
себя переписывать как-то по-другому. Это мощный, серьезный удар по русскому
народу в целом. Это не может быть для нас безразличным},
\citTitle{Путин интервью – о вступлении Украины в НАТО, коренных народах, встрече с Зеленским. Главное}, 
Екатерина Терехова, strana.ua, 10.06.2021

«Усі тварини рівні, але деякі рівніші за інших» (за тв. Джорджа Орвелла «Скотоферма»).
Близько місяця назад, з ініціативи прем'єр-міністра Шмигаля та за підтримки
МВС, до \emph{Верховної Ради України} Урядом подано \emph{україноненависний},
антихристиянський та антилюдський законопроект 5488. Цей законопроект є
небезпечним не лише для християн та традиційної \emph{української сім'ї}, але й
для будь-якого тверезомислячого \emph{українця}, патріота своєї держави. За
своєю \emph{україноневисною} суттю його можна прирівняти до законів 16 січня
2014 року. Він, фактично, є клоном попередніх законів №3316 /2-3/, які були
відкликані завдяки активній позиції свідомих \emph{українців}.  5488 нібито має
боротися з «дискримінацією» та «нетерпимістю» на законодавчому рівні. Що ж це
таке, на думку авторів цього ЛГБТ-закону? \enquote{нетерпимість — відкрите,
упереджене, негативне ставлення стосовно категорії осіб, відмінних за такими
ознаками, як раса, колір шкіри, політичні, релігійні та інші переконання,
статева приналежність, вік, інвалідність, етнічне та соціальне походження,
громадянство, сімейний і майновий стан, сексуальна орієнтація, гендерна
ідентичність, місце проживання, мова, або іншими ознаками},
\citTitle{Законопроект 5488 – Уряд «ЗА» дискримінацію українців}, , pravyysektor.info, 06.06.2021

Тут у кремлівських пропагандонів пердаки запалали.  Гарантую, що тільки
закінчимо з кривавим диктатором, тираном та узурпатором Лукашенко і Білорусь
стане вільною зразу візьмемося за вас. Підтримаємо Хабаровськ, підтримаємо всі
поневолені Москвою народи які прагнуть свободи, і обов'язково прийдемо за
головним кремлівським карликом-уродцем а подорозі будьте певні перетопимо у
власному лайні всіх путінських пропагандонів починаючи звичайно з соловьова.
До відома, соловйових...  Ви мабуть там в Московії зовсім того кукуньою
поїхали... Тричі кремлівські спецслужби готували і вчиняли на мене замахи, в
результаті одного з них теракту в центрі Києва загинуло двоє людей в тому числі
випадковий перехожий, ваші найманці вбивають моїх друзів та соратників як на
фронті так і на мирній території \emph{України}, і навіть в Європі, ваш слідком
оголосив мене в розшук за вигаданим звинуваченням а тепер дивуєтесь: чому я з
соратниками і далі боремося з вами?  Не дивуйтесь! Ми вас кремлівських бля\#ей
та ваших маріонеток на кшталт Лукашенка заганятимемо по цілому світу, поки не
витруємо та не видавимо як тарганів. І да, в Білорусі є \emph{наші українці}, як
журналісти так і патріоти які допомагають повсталим білорусам! Вони допомагали
нам на Майдані та на фронті а ми звичайно допомагаємо їм в боротьбі з
диктатором і скоро ви це відчуєте та побачите!  А на сам кінець скажу, що в
Білорусь наші соратники потрапили транзитом через РФ, через ваш дірявий кордон.
Ви ж всі діряві і за 1000 рублів маму на панель поставите, а не тільки
\emph{українських} патріотів в Білорусію транзитом через свій кордон пропустите!  Без
жодної поваги до соловйових, Народний депутат \emph{України} VIII скликання Ігор
Мосійчук,
Ігор Мосійчук, telegram, 13.08.2020

В «Укрзализныце» призвали \emph{украинского} боксера Александра Усика больше не
фотографироваться на железнодорожных путях. Об этом пресс-служба компании
сообщила в Facebook.  «Известный боксер Александр Усик, видимо, забыл о
правилах поведения на объектах железной дороги и сел на рельсы ради
оригинальной фотографии для соцсетей. Александр, среди железнодорожников тоже
много Ваших фанов, но правила важно соблюдать. Берегите себя и нервы
машинистов!» — говорится в сообщении,
\citTitle{УЗ попросила Усика не фотографироваться на железнодорожных путях (фото)}, , sharij.net, 10.06.2021

Заседание участников Контактной группы, прошедшее сегодня в режиме
видеоконференции, завершилось без подвижек и отличилось абсурдностью выходок
представителей \emph{Украины}. Об этом сообщила пресс-секретарь делегации ЛНР в
Контактной группе Мария Ковшарь.  \enquote{Заседание Контактной группы 9 июня
завершилось без подвижек, но отличилось абсурдностью выходок \emph{Украины}}, -
сказала она.  Ковшарь отметила, что \enquote{обсуждение вопросов политического
урегулирования неприятно удивило: глава \emph{украинской} делегации (Леонид)
Кравчук заявил, что \emph{Украина} не собирается обсуждать политические вопросы
в рабочей подгруппе},
\citTitle{Луганский Информационный Центр — Заседание Контактной группы
отличилось абсурдностью выходок Украины – делегация ЛНР}, , lug-info.com,
09.06.2021

Результати тесту про COVID-19 і документ про вакцинацію, що пред'являються на
кордоні з \emph{Україною}, повинні містити достатню інформацію, довідка про
щеплення російською вакциною права на в'їзд не дає. Джерело: Державна
прикордонна служба, речник відомства Андрій Демченко у коментарі УП.  Деталі:
Держприкордонслужба отримала роз'яснення від Міністерства охорони здоров'я та
просить громадян врахувати інформацію про те, як мають бути оформлені такі
документи, та якими вакцинами мають бути імунізовані громадяни.  Зокрема, за
останні два дні в пунктах пропуску через державний кордон \emph{України}
спостерігаються випадки, коли з-за кордону  намагаються в'їхати іноземці з
документами про імунізацію вакциною \enquote{Спутник V},
\citTitle{Прикордонники України завертають росіян, вакцинованих \enquote{Спутником}}, , www.pravda.com.ua, 10.06.2021

Як повідомляють РИА Новости, в УЄФА дійшли висновку, що \enquote{конкретне поєднання
цих двох гасел (\enquote{Слава \emph{Україні}! – Героям слава!} – ред.) явно вважається
політичним за своєю природою, що має історичне і мілітаристське значення}.
Тому в УЄФА заявили, що \enquote{цей конкретний слоган на внутрішній стороні футболки
має бути вилучений для використання в матчах змагань УЄФА}. В УЄФА відзначили,
що карта \emph{України}, на якій зображений Крим, правил не порушує,
\citTitle{УЄФА вимагає прибрати \enquote{Героям слава} з української форми – ЗМІ}, , www.pravda.com.ua, 10.06.2021

Шістдесят років тому, 10 червня 1961 року, померла Катерина Білокур. «Якби ми
мали художницю такого рівня майстерності, то змусили б заговорити про неї цілий
світ», – сказав про Білокур дин із найвидатніших митців XX століття Пабло
Пікассо. Між тим, останні роки через важкі умови життя, Катерина Білокур майже
не малювала. І на момент своєї смерті, як всі \emph{українські селяни} в СРСР,
не мала навіть паспорта. Квіти, намальовані Катериною Білокур, упізнають
одразу, бо вони немовби левітують у повітрі і випромінюють світло. «Вона бачить
душі квітів», казали про неї. Радіо Свобода зібрало деякі факти із життя
Катерини Білокур,
\citTitle{\enquote{«Була, щоб залишитися у квітах»: до дня пам'яті Катерини Білокур}}, Ірина Штогрін, www.radiosvoboda.org, 10.06.2021

Нужна ли \emph{Украине} очередная революция?, Денис Жарких, youtube.com, 09.06.2021

Временного поверенного в делах \emph{Украины} в Москве Василия Покотило в десять утра
вызвали в МИД РФ. Ему заявили решительный протест по поводу нарушения Киевом
своих обязательств по Венским конвенциям 1961-го и 1963 года.  \enquote{В очередной раз
мы потребовали от \emph{украинской стороны} принять все необходимые меры по
недопущению подобных провокаций, обеспечить условия для нормальной работы и
неприкосновенности посольства и генконсульства России на \emph{Украине}, а также
безопасность их сотрудников}, - заявила пресс-секретарь российского МИД Мария
Захарова,
\citTitle{МИД РФ вызвал украинского поверенного из-за событий с радикалами, не давшими возложить цветы Пушкину}, 
Эллина Либцис, strana.ua, 10.06.2021

%%%join
\enquote{Предложение сборной: если запретят футболки с 
\enquote{Героям слава!}, выйти на поле в такой форме}, - гласит пост нардепа.
Причем к нему прикреплена фотография советника министра внутренних дел \emph{Украины}
Зоряна Шкиряка, на спине которого красуется огромная татуировка с трезубцем,
мордой волка и надписью по-английски \enquote{Слава \emph{Украине}! Героям слава!}.
В постскриптуме Вятрович спросил у Шкиряка, не против ли он такой идеи. 
\enquote{Категорически за!} - ответил в комментариях советник Арсена Авакова,
\citTitle{Вятрович предложил украинским футболистам сделать тату, как у Шкиряка}, Наталья Полулях, strana.ua, 10.06.2021

А в качестве нашей ответной меры я бы предложил использовать в спортивной форме
России карту Российской империи, которая полностью включала всю современную
территорию \emph{Украины}. Ведь до 1917 года это были просто обычные русские губернии:
Черниговская, Полтавская, Харьковская, Киевская, Волынская, Подольская. Даже
слово «Украина» не использовалось, а \emph{по-украински}, то есть, на смеси русского и
польского языков, говорили только в глухих западных деревнях!,
\textbf{Жириновский об ответе Украине: «Я бы предложил использовать в нашей форме карту Российской империи»},
www.sport-express.ru, 08.06.2021
%%%endjoin

\emph{Проукраїнські} громадяни вважають головною причиною нелюбові, яка отруює соціум,
природну агресію та ущербну озлобленість російських зайдів закодованих
монголо-татарським культурним геномом. А також вони вважають, що нелюбов
нащадків орди та угрофінських племен походить від екзистенційної нелюбові до
\emph{українського}. Мабуть, саме тому автохтони намагаються вплинути на чужих через
апеляцію до природних принад території, що знаходить своє відображення у
слоганах «\emph{Любіть Україну}» винесених на борди з \emph{українськими краєвидами},
\citTitle{Нелюбов та некраса української незалежності}, Анатолій Якименко, analytics.hvylya.net, 10.06.2021

Висновок. Перед \emph{українцями} стоїть кілька складних питань, відповіді на які
неможливо позичити у архаїчній парадигмі розуміння духовності як віри у Бога і
навіть віри Богу, відшукати у логоцентричній моделі модерну чи безсуб'єктній та
безоб'єктній парадигмі постмодерна. Все показує на те, що для того, аби
вирішити важку проблему безумовної метафізичної любові потрібно зробити
нетрадиційний крок, об'єднавши матерію та науку з містикою. З великою
впевненістю на сьогодні можна сказати, що духовністю у сучасному нерелігійному
світі є відчуття, яке має матеріальний субстрат, що знаходиться у нервовій
тканині. Це відчуття сумління, яке сучасна людина навчилася притлумлювати перед
матеріальними орієнтирами нав'язаними цивілізацією. Рухаючись за цим відчуттям
та потрапляючи у дуже складні та небезпечні ситуації людина долає труднощі і
підвищується у свідомості, долаючи шлях від страху, бажання, гордості тощо
досягаючи рівня любові та просвітлення,
\citTitle{Нелюбов та некраса української незалежності}, Анатолій Якименко, analytics.hvylya.net, 10.06.2021

\emph{Українці}, як завше, поміж світами. Вільнолюбні, ще в часи Київської Русі мали
кілька імен — одне, церковне, інше — мирське. Прізвища поєднували з
прізвиськами, й тут я не омину розповісти, що мого батуринського прадіда, на
ім'я Мефодій, усі звали Красивий, хоч то й не було його прізвище, радше
естетичний факт. Проте, є в нас і щось східне, коли ввічливо згадуємо батька,
іменуючи людину Петром Олександровичем чи Тетяною Василівною. Молодь все
частіше відмовляється від таких форм, а я завжди відмовляюся від інших — коли
співрозмовник, забуваючи про ім'я, просто гукає / жартує  чи й офіційно
звертається самим лиш по батькові. Василівна і Петрівна, Миколайович і Петрович
— це точно ані вияв поваги до предків, ані до самого адресата, назвати якого на
ім'я якось і забули. Тож скільки б ви не гукали до мене — Леонардівна, не
озирнуся,
\citTitle{Як ти називаєшся?}, Анна Данильчук, day.kyiv.ua, 10.06.2021

Другой важный момент – \emph{украинцы} в нашем собственном массовом сознании и
пропаганде не считаются здесь, в \emph{Украине}, коренным народом. У нас говорят об
\emph{украинском народе}, как о вобравшем в себя разные народы. А если кто-то говорит
об \emph{украинском} народе как коренном, как отдельной нации, то это уже \enquote{фашисты}.
То есть нам еще нужно разобраться в самих себе. По большей части мы говорим как
о коренных о тех народах, которые пострадали от различных ассимиляций,
репрессий, которые являются вымирающими. Но коренным является народ, который
является автохтонным, таким, что вырос на этой земле. А разве \emph{украинцы} здесь не
выросли?,
\citTitle{Как законопроект о коренных народах Украины выбил почву из-под ног Москвы - Главред}, 
Григорий Перепелица, opinions.glavred.info, 10.06.2021

\enquote{Прощание состоялось в подразделениях, которые защищают территориальную
целостность \emph{Украины}, а также в пункте постоянной дислокации. Командование
бригады и собратья Карины выражают искренние соболезнования семье и близким}, –
сказано в сообщении,
\citTitle{На Донбассе погибла 22-летняя военнослужащая ВСУ: стало известно имя Героини. Фото}, 
Карина Бондаренко, plus.obozrevatel.com, 10.06.2021

Добрий вечір, малятка. Любі хлопчики та дівчатка.  УЕФА обязал сборную
\emph{Украины} убрать с формы на чемпионате Европы лозунг «Героям слава», даже
с учетом того, что его не видно. Без изменений оставили наличие на форме
сборной карты с Крымом, поскольку действующие границы государства с
оккупированным полуостровом признаны ООН, а также приветствие «\emph{Слава
Украине}», которое было разрешено еще в 2018. А вот соцчетание обеих лозунгов,
по мнению европейских футбольных чиновников, носит политических и
военизированный характер. Президент \emph{Украинской} ассоциации футбола Андрей
Павелко срочно вылетел в Рим для переговоров по сохранению дизайна формы. Не
думаю, что они будут успешными. Бабло всегда побеждает, а у русских, у которых
подгорело из-за нашей формы, его объективно больше. Поменяли бы просто на
«Путин хуйло»,
\citTitle{Скандал с формой, яйца от Гордона, электронные больничные...}, Денис Безлюдько, strana.ua, 10.06.2021

%%%cit
%%%cit_pic
%%%cit_text
Почему так туго доходит до власти простая истина, сказанная Бжезинским еще в
2014, что НАТО ничего хорошего не даст \emph{Украине}, а вступление \emph{Украины} в Альянс
только обострит существующий конфликт с Россией и будет раскалывать \emph{украинцев}.
Отцепитесь вы уже от НАТО и от ЕС.  Никто и никуда \emph{Украину} не примет. Особенно
с вами.  Стыдно слушать как вас посылают все кому не лень.  Попробуйте поискать
новую «фишечку».  Очередной томас, например
%%%cit_title
\citTitle{Сколько еще раз лидеры НАТО должны послать Зеленского?}, Елена Лукаш, strana.ua, 11.06.2021 
%%%endcit

%%%cit
%%%cit_pic
%%%cit_text
Петр Порошенко должен наконец объяснить обществу почему он привез из Минска то,
что больше напоминает капитуляцию \emph{Украины}.  Владимир Путин в своих публичных
выступлениях уже откровенно «троллит» заявляя, что представители «ЛДНР» не
хотели подписывать Минские соглашения, но на этом якобы настояла украинская
сторона. \emph{Украина} не может годами находиться в подвешенном состоянии из-за
того, что \emph{украинцы} же не знают, как реально заключались Минские договоренности,
как там появились подписи представителей ДНР и ЛНР, кто их в конце концов
писал. Какую роль при этом сыграли контакты между Порошенко и Медведчуком?
%%%cit_title
\citTitle{Порошенко должен наконец объяснить Украине свое поведение в Минске}, Андрей Золотарев, strana.ua, 11.06.2021
%%%endcit

%%%cit
%%%cit_pic
%%%cit_text
Очень часто, а то и вовсе постоянно, доводится слышать о том, что якобы \emph{Украина}
всю историю своего существования в качестве колонии разных империй, стремилась
получить независимость. Это ей, наконец, удалось сделать в 1991 году, а до
этого чуть ли не 800 лет украинцы боролись за свободу от колониальной
зависимости.  Напомним, что Древнерусское государство, основанное Рюриком в IX
веке, современные \emph{украинцы} тоже считают \emph{Украиной}, называя его Киевской Русью,
правда, неизвестно, на каком основании. Видимо, они думают, что раз тогда
столицей Руси был Киев, то сегодня вся Русь, включая Новгород, Москву, Рязань,
Брест-Литовск и даже Владивосток с кучей всех остальных русских и белорусских
городов, должна принадлежать \emph{современной Украине}. Но не будем о Киевской Руси,
а будем о независимости \emph{Украины}, которой она якобы добивалась со времен
Средневековья
%%%cit_title
\citTitle{Украина 800 лет подряд хотела отдаться кому-то в управление, этот вектор продолжается и сегодня}, 
Исторический Понедельник, zen.yandex.ru, 04.06.2021
%%%endcit

%%%cit
%%%cit_pic
%%%cit_text
Кстати, дело Пилипа Орлика не угасло с его смертью. В истории \emph{Украины} было
много других «гетманов», мыслителей и составителей всяческих «\emph{украинских}
конституций», которые собирались отдать земли \emph{Украины} то под наполеоновское
управление во время Отечественной войны, то под британское во время Крымской,
то под германское во время Первой Мировой, то под гитлеровское во время Великой
Отечественной. И сегодняшняя ситуация не сильно изменилась. С 1991-го по
2014-й между \emph{украинскими} политиками велись постоянные споры о том, кому
повыгодней продаться – Европе или России? России продаться было тогда выгодней,
но вот грянул 2-й Майдан, и оказалось, что надо срочно продаваться Америке,
потому что Европа от Украины в ужасе открестилась – ни ЕС, ни НАТО даже
пары-тройки старых трамваев и танков из своего бюджета \emph{свидомой Украине} не
выделили!
%%%cit_title
\citTitle{Украина 800 лет подряд хотела отдаться кому-то в управление, этот вектор продолжается и сегодня}, 
Исторический Понедельник, zen.yandex.ru, 04.06.2021
%%%endcit

%%%cit
%%%cit_pic
%%%cit_text
Да и Америка что-то менжуется, не хочет брать \emph{Украину} под свое полное
управление, хотя сначала такие планы были, но не по сеньке шапка оказалась.
Была надежда на Турцию, затем на Китай, но и тут не выгорело ничего.  Так кому
же сегодня \emph{Украине} отдаться в управление? Под чей протекторат писать новую
Конституцию?
%%%cit_title
\citTitle{Украина 800 лет подряд хотела отдаться кому-то в управление, этот вектор продолжается и сегодня}, 
Исторический Понедельник, zen.yandex.ru, 04.06.2021
%%%endcit

%%%cit
%%%cit_pic
%%%cit_text
Русский президент Украины Леонид Кучма в своей нашумевшей книге «\emph{Украина} – не
Россия» написал, что русские отличаются от \emph{украинцев} тем, что 1000 лет «живут
по понятиям», а \emph{украинцы} очень законопослушны и принимают любую власть, лишь бы
она была законной. Правда, реальная действительность этого не подтверждает –
«по понятиям» живут и русские, и \emph{украинцы}
%%%cit_title
\citTitle{Русские живут «по понятиям», а украинцы чтут закон и любят власть. Что тут не так?}, 
Исторический Понедельник, zen.yandex.ru, 29.05.2021
%%%endcit

%%%cit
%%%cit_pic
%%%cit_text
\enquote{... Також Виконком затвердив офіційний футбольний статус гасел \enquote{Слава \emph{Україні}!} і
\enquote{Героям слава!}, які вже багато років є привітанням для мільйонів наших
вболівальників у рідній країні, і по всьому світу. На всіх матчах наших
збірних!  Унікальний національний футбольний код, який містять в собі ці
атрибути, об'єднав усіх \emph{українців}, з різних регіонів \emph{України} й наших земляків з
різних країн і континентів. Ми зобов'язані й маємо за честь оберігати його і
наповнювати новою силою єднання! Вітаю всю футбольну \emph{Україну} з цим історичним
рішенням!} - повідомив Павелко у Facebook
%%%cit_title
\citTitle{Україна і Євро-2020. Затвердили гасло українського футболу}, , gazeta.ua, 11.06.2021
%%%endcit

%%%cit
%%%cit_pic
%%%cit_text
\enquote{Руководитель \emph{украинской} компании «Нафтогаз» Юрий Витренко
утверждает, что завершение строительства газопровода «Северный поток – 2» якобы
может привести к началу полномасштабной войны}.  Я бы не относился с иронией к
очередному бреду наших \enquote{защитников национальных интересов}.
Спровоцировать войну - вполне по силам этим ребятам, которые будут объяснять
это для своих примерно так - \enquote{нас вынудили - иначе Северный поток 2 не
остановить}, а для мирового общественного мнения и объяснять ничего не надо -
войну развязать может только страна-агрессор, и это не \emph{Украина}
%%%cit_title
\citTitle{Глава Нафтогаза Витренко пугает войной после запуска Северного потока-2}, Михаил Погребинский, strana.ua, 11.06.2021
%%%endcit

%%%cit
%%%cit_pic
%%%cit_text
Писательница и борец за \emph{украинский язык} Лариса Ницой заявила, что Украина может
исчезнуть из-за равнодушного отношения ее граждан к истории и выдающимся
деятелям. Своим мнением поделилась в Facebook.  \enquote{Как исчезнет \emph{Украина}. Это
будут не танки, как на востоке. Украину убьет наше равнодушие к \emph{украинскому}.
\emph{Украина} когда-то исчезнет так, как исчезает уже сегодня. Постепенно и
незаметно. Она исчезнет так, как тает на поляне куча снега}, - предупредила она
%%%cit_title
\citTitle{Ницой рассказала, что \enquote{убьет} Украину}, , strana.ua, 11.08.2019
%%%endcit


%%%cit
%%%cit_pic
%%%cit_text
Скандал вокруг лозунга \enquote{Героям слава!} вышел далеко за пределы футбола.
Попытка запретить этот лозунг на форме сборной \emph{Украины} является далеко
не футбольной и не спортивной.  Важна реакция \emph{украинского общества} на
это событие, а она была очень правильной. \emph{Украинцы} возмутились и стали
выражать свое недовольство в письмах, комментариях, заметках, пытаясь
достучаться до УЕФА.  При этом попытку заставить убрать с футболок
\emph{украинских} спортсменов лозунг \enquote{Героям слава!} УЕФА пытался
представить как компромисс и шаг навстречу \emph{Украине} – мол, и так останутся слова
\enquote{Слава Украине!} и очертания территории нашего государства, включая
Крым. Но непонятно, почему мы вообще здесь должны говорить о каких-то
компромиссах, почему вообще этот вопрос стал предметом дискуссии. Этот момент
не требует никаких дискуссий и компромиссов
%%%cit_title
\citTitle{Почему взбесилась Россия из-за надписи \enquote{Героям слава!} на форме украинской сборной - Главред}, 
Владимир Вятрович, opinions.glavred.info, 11.06.2021
%%%endcit

%%%cit
%%%cit_pic
%%%cit_text
И вот опять в 2021-м к нам возвращается радость соревнования. Но сборная
\emph{Украины} не была бы сборной \emph{Украины}, если бы не затеяла свои игры с футбольной
формой. Нарисовали контур \emph{Украины}, в который внесли Крым, и написали два
лозунга: «слава \emph{Украине}» и «героям слава». Хорошо, что у нашей федерации
футбола хватило смелости и настойчивости потребовать от УЕФА убрать с формы
футболистов бандеровский клич. У УЕФА хватило ума с этим согласиться и
запретить: «После дальнейшего анализа было принято решение, что конкретное
сочетание двух фраз считается исключительно политическим по природе, имеет
историческое и милитаристское значение». Это значит, что не быть на форме
\emph{украинских} футболистов нацистским лозунгам
%%%cit_title
\citTitle{Ни героев, ни славы. УЕФА запретил лозунги на форме футболистов Украины}, 
Моя Поляна, zen.yandex.ru, 11.06.2021
%%%endcit

%%%cit
%%%cit_pic
\ifcmt
  pic https://avatars.mds.yandex.net/get-zen_doc/1936915/pub_60bf450a2b92e05b9bce0bd1_60bf482ffaeaaf49e4143718/scale_1200
  caption Ну, чем не армия батьки Махно, чистая «Свадьба в Малиновке»
\fi
%%%cit_text
Ну, чем не армия батьки Махно, чистая «Свадьба в Малиновке». Солдаты
перечисляют средства на карточки офицеров, среди состава идет разделение на
«хороших» и «нехороших», законодательство \emph{Украины} трактуется как захотела левая
нога командира, удивительно, что они не издают свои личные законы. Но больше
всего посмешило, что на позициях 1-й и 2-й линии обороны ВСУ смотрят российские
каналы: РТР, ТНТ, НТВ, а \emph{украинские каналы} смотреть нет возможности, плохое
вещание.  И конечно, неутешительные выводы: таких не возьмут в НАТО, надо
поднимать уровень морально-психологического состояния войск.  Могу сказать
армейским руководителям \emph{Украины}, ваш морально-психологический уровень почти
догнал доблестные силы НАТО. Судя по тому, как британские солдаты себя ведут на
территории Эстонии, все в порядке в \emph{ВСУ}
%%%cit_title
\citTitle{Украинская армия \enquote{пьет и воюет} по натовским стандартам. А
\enquote{Боширов и Петров}, возможно, разлагают британских солдат в Эстонии}, 
Моя Поляна, zen.yandex.ru, 08.06.2021
%%%endcit

%%%cit
%%%cit_pic
\ifcmt
  pic https://avatars.mds.yandex.net/get-zen_doc/3606239/pub_60b49d56b86a7b140d750d16_60b4ade6a471d16dacf1088a/scale_1200
  caption А нам мясо Бог послал! Оно не вредно. Сашхен и Альхен

  pic https://avatars.mds.yandex.net/get-zen_doc/3700776/pub_60b49d56b86a7b140d750d16_60b4ae762174cf46a929be31/scale_1200
  caption Классика вечна! Олег Табаков
\fi
%%%cit_text
Есть такая категория политиков, которые думают, что никто не услышит их речи в
мировом информационном океане, они стараются сделать громкие заявления, но
как-то стыдливо, вроде что-то сказал, все дома оценили, а там, может, и не
услышат, пропустят в потоке новостей.  Это президент \emph{Украины} – В. Зеленский, и
много выступает, и везде ездит, вроде что-то заключает, но старается, чтобы
лишний раз не заметили, чтобы лишний раз не услышали, какой-то он \enquote{тихий}
политик, который мне напоминает голубого воришку Альхена из \enquote{Двенадцати
стульев} Ильфа и Петрова. А нынешняя \emph{Украина} является олицетворением 2-го дома
Старсобеса: \emph{украинцы}, как пугливые старушки, одетые во все серое, живут по
распорядку, придуманному \enquote{новой} властью, мирятся с тем, что заведующий-завхоз
с олигархами брательниками их обкрадывает и надувает, эти \enquote{руководители} тащат
все, что плохо лежит, никто им не указ. Законы \emph{Украины} очень напоминают
развешанные по всему дому указания, как в столовой, только это не законы – это
агитация, в которой тебе кратко объяснят, почему мясо есть вредно
%%%cit_title
\citTitle{Застенчивая улыбка Зеленского похожа на улыбку \enquote{голубого воришки} Альхена. 
\enquote{Ты кому Украину продал, жулик?!}}, Моя Поляна, zen.yandex.ru, 01.06.2021
%%%endcit

%%%cit
%%%cit_pic
%%%cit_text
Кто же тогда такой Остап Бендер в новой политической литературе? Конечно, это
представители ЕС, Америки, Турции. Они пришли в этот серый для них \emph{украинский}
мир для того, чтобы решать исключительно свои вопросы, они, конечно, все видят,
они даже готовы выслушать настойчивые, но негромкие жалобы \emph{украинцев-старушек},
но, не более. У них другие задачи, все, что им надо, это стул под названием
\emph{Украина}, в котором спрятаны драгоценности. И вот звучит громкий тост Остапа
Бендера за обеденным столом: «– Пью за ваше коммунальное хозяйство! –
воскликнул Остап»
%%%cit_title
\citTitle{Застенчивая улыбка Зеленского похожа на улыбку \enquote{голубого воришки} Альхена. 
\enquote{Ты кому Украину продал, жулик?!}}, Моя Поляна, zen.yandex.ru, 01.06.2021
%%%endcit

%%%cit
%%%cit_pic
%%%cit_text
Так само за двадцять років у топ-політиці не знайшлося державних мужів, які
поставили б питання руба перед Юлією Тимошенко. Про її зв'язок з
топ-корупціонером Павлом Лазаренком. Про те, чому вона роками боялася летіти до
США? Як за нею закріпилося прізвисько «газова принцеса»? Звідки в неї гроші на
зміну нарядів від Louis Vuitton, які вона свого часу міняла буквально щодня? Що
спонукало її в обхід президента Ющенка підписати кабальний для \emph{України}
десятирічний газовий контракт з Газпромом? Чому Янукович так ніколи і не
знайшов її рахунків, на які Росія нібито справно перераховувала те, що їй
належалося за здачу \emph{українських} державних інтересів? Не знайшлося й тих, хто з
цими питаннями звернувся б до виборців Юлії Володимирівни. Не запитав, чому
вони, знаючи про всі ці діяння ЮВТ, так щиро її підтримували і голосували за
неї?,
%%%cit_title
\citTitle{В боротьбі за себе - ZAXID.NET}, Василь Расевич, zaxid.net, 11.06.2021
%%%endcit


%%%cit
%%%cit_pic
\ifcmt
  pic http://img.lug-info.com/cache/9/d/1623440644_607.jpg/1000wm.jpg
  width 0.2
\fi
%%%cit_text
\emph{Украинские} диверсанты, проникшие на территорию ЛНР, убили пятерых защитников
Республики. Об этом сообщил офицер пресс-службы управления Народной милиции ЛНР
Иван Филипоненко. \enquote{11 июня диверсионной группой отдельного центра
специальных операций \enquote{Запад} (бывший 8-й полк специального назначения
ССО ВСУ) на наблюдательном посту Народной милиции в районе населенного пункта
Голубовское ЛНР были хладнокровно убиты пятеро защитников Республики}, - сказал
он.  Филипоненко отметил, что \enquote{военнослужащие застрелены выстрелами в
голову с использованием специального вооружения, при этом один из них убит
выстрелом, сделанным из снайперской винтовки иностранного производства,
остальные расстреляны в упор \emph{украинской} диверсионной группой, проникшей на
территорию поста}
%%%cit_title
\citTitle{Луганский Информационный Центр — Киевские диверсанты убили пятерых
защитников ЛНР в районе Голубовского – Народная милиция}, , lug-info.com,
11.06.2021
%%%endcit

%%%cit
%%%cit_pic
%%%cit_text
Конечно же, никакое \enquote{почтение к современным защитникам \emph{Украины}} здесь ни при
чём. Лозунг этот имеет только одно значение: это лозунг \emph{украинских}
националистов/национал-шовинистов. Он не может быть истолкован двояко.  А на
форме футболистов он был размещён не для того, чтобы \enquote{показать миру} что-то, а
для того, чтобы в очередной раз дискриминировать тех граждан \emph{Украины}, которые
не разделяют националистические убеждения. Государственные и футбольные
функционеры в очередной раз говорят нам: - \enquote{Вам тут не рады! Это наша,
бандэровская страна! И рано или поздно, мы вас всех выгоним либо убъём!} В этой
связи, неплохо бы вспомнить, что продолжение упомянутого лозунга звучит как: -
\enquote{Слава нации! Смерть врагам!} Под \enquote{врагами}, конечно же, понимаются все, кто не
бандэровец. То есть, все русские, православные, \enquote{русскоязычные}, и т. п.
граждане Украины. А также - венгры, румыны, болгары, и прочие немногочисленные
на территории нашей страны этнические группы.  В общем, \enquote{неправильным}
гражданам \emph{Украины}, которых здесь под 60\%, в очередной раз напомнили, что они
здесь \enquote{никто}. И это, как и любые другие формы разжигания межнациональной,
межконфессиональной, и т. п. розни, заслуживает осуждения
%%%cit_title
\citTitle{Лозунг националистов на форме сборной - способ дискриминации граждан}, 
Даниил Богатырев, strana.ua, 11.06.2021
%%%endcit

%%%cit
%%%cit_pic
%%%cit_text
ШАНОВНІ КОЛЕГИ. Запрошуємо Вас на чергову нараду в ZOOM, яка відбудеться в
четвер 13 травня о 20:00. Над Ненькою \emph{Україною} нависла смертельна загроза
широкомасштабної агресії. Не менш загрозливим для нації є запровадження
внутрішніми ворогами з липня місяця цього року ринку землі – \emph{української}
території.  Слід об'єднатися і створити такі сили, які нікому не дозволять
позбавити український народ його історичної і Богом даної Землі, а також не
дозволять загарбати \emph{Україну}.
ПОРЯДОК  ДЕННИЙ:
\begin{itemize}
\item 1. Об'єднання громадських організацій України – єдина запорука наших
перемог з зовнішніми і внутрішніми загрозами.  (Виступ громадських діячів від
областей \emph{України}).
\item 2. Вироблення шляхів та методів боротьби за збереження \emph{України} – доповідає
Голова \emph{Всеукраїнського} об'єднавчого координаційного центру – академік Петро
Таланчук.
\item 3. Звернення Конференції до громадського активу  всіх областей \emph{України} -
«Зупинимо введення із 1 липня  Закону про торгівлю землею
сільськогосподарського призначення!».
\end{itemize}
Голова \emph{Всеукраїнського} координаційного центру академік Петро Таланчук. Голова Конгресу професор Михайло Бугаєць
%%%cit_title
\citLink{https://t.me/uanarod/186}, Український Народ, telegram, 13.05.2021
%%%endcit


%%%cit
%%%cit_pic
%%%cit_text
Если кто-то во всем мире и знает об \emph{Украине}, то только как о территории,
населенной лентяями и олигархами. И это не мое мнение, а мнение европейских и
американских политиков, которые давно уже не хотят иметь с \emph{Украиной} никакого
дела, включая и финансовые подачки. Но они продолжают цепляться за \emph{Украину} по
собственной глупости. \emph{Украина} им еще откликнется огромными проблемами,
относительно которых нынешние проблемы – это просто пыль
%%%cit_title
\citTitle{Почему великороссы стали такими сильными, а малороссы остались такими слабыми?}, 
Исторический Понедельник, zen.yandex.ru, 11.06.2021
%%%endcit

%%%cit
%%%cit_pic
%%%cit_text
Если хорошо подумать, и, так сказать, позреть в корень, то можно сразу же
понять, что понятие \emph{«Украина»} - это все же достаточно обидное название для
страны, собравшей в себе множество земель, населенных другими народами, и даже
Киевской Руси никогда не принадлежавших, и особенно в более поздние периоды
\emph{Украины}. \emph{Украина} – это недвусмысленно означает «окраина», этого не отрицают и
сами украинцы. Вот, например, Белоруссия – это реальное отражение не только
исторического, но и современного значения страны, но страна-империя какой-то
Окраиной именоваться как бы не должна.  Термин \emph{«украина»} («окрайна») - это
древнерусское слово, и применялось не только к территории современной \emph{Украины},
но и к другим регионам, особенно на не освоенных севере и востоке Руси. Так, в
начале II-го тысячелетия на Руси существовали \emph{«украины»} окская и псковская, - а
где Ока и Псков относительно \emph{Украины}, можно посмотреть на карте. Так что
никакого особого, то есть уникального смысла современное название \emph{Украины} не
имеет
%%%cit_title
\citTitle{Почему современную Украину назвали «окраиной», а не более престижно -
Киевской Русью?}, Исторический Понедельник, zen.yandex.ru, 22.02.2021
%%%endcit

%%%cit
%%%cit_pic
\ifcmt
  pic https://avatars.mds.yandex.net/get-zen_doc/3323369/pub_5ff4cffafe4e686f6a93f46c_5ff4da39f906b168726011d9/scale_1200
  caption В.Егоров \enquote{Каганы рода русского}. В этой книге описывается вся история киевских князей, ставших каганами
\fi
%%%cit_text
Может, кто не знает, но киевский князь Владимир был каганом, то есть татарским
(хазарским) царем. А вообще-то этим титулом венчали не только хазарских и
татарских царей, но и киргизских, уйгурских и даже монгольских. Да-да, в
далёкой Монголии. Каганом, кстати, был и сам Чингисхан, потому что «каган» и
«хан» - это одно и то же, просто по-разному произносится. Так что Владимир
Святославович был настоящим ханом, что вообще ни в какие \emph{украинские ворота} не
лезет
%%%cit_title
\citTitle{Почему киевский хан Владимир Креститель считается украинским князем, когда он даже славянином не был?},
Исторический Понедельник, zen.yandex.ru, 05.01.2021 
%%%endcit

%%%cit
%%%cit_pic
%%%cit_text
Так что еще непонятно – \emph{украинским} ли царем был князь Владимир, или
монгольским. Был ли титул кагана у русских, то есть московских царей? Не
припомню что-то. А вот у \emph{украинских} царей, то есть великих киевских князей,
этот титул был, причем не только у Владимира.  Ну ладно, с титулом разобрались,
теперь разберемся с происхождением. Достоверно известно, что родной прадедушка
нашего героя был Рюрик. А кем по происхождению был сам Рюрик? Никто не знает.
Но точно, что не \emph{украинцем}, потому что наверняка он пришел в Новгород с севера
или запада, а не с юга или востока. Так что \emph{украинцем} Рюрик быть никак не мог
%%%cit_title
\citTitle{Почему киевский хан Владимир Креститель считается украинским князем, когда он даже славянином не был?},
Исторический Понедельник, zen.yandex.ru, 05.01.2021 
%%%endcit

%%%cit
%%%cit_pic
%%%cit_text
Vladimir S, Если был \emph{украинский язык} и украинская нация, да еще, не дай
Бог, \emph{украинское государство}, тогда дело за малым предъявите факты.  К
примеру, с 11 века сохранилось 5 книг на русском языке, где много раз
упоминается государство РУСЬ и РОССИЯ. Чем дальше, тем больше сохранившихся
книг. Тысячи сохранившихся страниц, миллионы упоминаний Руси. Первое упоминание
России в иностранных документах в главе 9, Трактата как управлять империей
византийского императора Константита 7 Багрянородного. Надписи на чугунных
пушках РУСЬ и РОССИЯ ХРАНЯЩИЕСЯ ВОЗЛЕ ЗДАНИЯ АРСЕНАЛА. Сохранились монеты
Новгородского князя с надписью: Владимир (не Владисвит.укр) на столе а это его
серебро и другие монеты. Первая монета с надписью Россия рубль 1654 года.
Последняя гривна Руси 1726 год Санкт-Петербург. Договора с другими
государствами, ярлыки монгольских ханов на уйгурском языке.  А что с
\emph{Украиной}, 22 раза упоминается в русских летописях и только два раза из
них географически к месту при условии правильного исправлении ошибок. Если
ошибки исправить не правильно, то вместо \emph{Украины} Гондурас получится. И
потом никто не говорил, что окраина это государство, а не футбольная команда.
Как могла существовать \emph{украинская нация} и \emph{Украинское государство}
ни с кем не воюя, не отливать пушек, не чеканить монеты, не издавать книг,
летописей и без упоминаний в иностранных документах.  Факты: первая книга на
\emph{украинском языке} 1856 год П. Кулиш \enquote{Записки о Южной Руси}
(согласитесь, неудачное название для первой \emph{украинской книги}). Первый
человек \emph{украинской национальности} попал в акты переписи населения Киева
в 1917 году
%%%cit_comment
Алексей Четин
%%%cit_title
\citTitle{Кто основал Москву? Вы даже удивитесь - ее основал великий киевский князь в XII веке и заселил ее украинцами}, 
Исторический Понедельник, zen.yandex.ru, 18.02.2021
%%%endcit

%%%cit
%%%cit_pic
%%%cit_text
Vladimir S, Разница большая. Русь была, а \emph{Украины} как политического субъекта
никогда не было это географическое название местности. Даже в документах по
воссоединению \emph{Украины} с Россией, \emph{Украина} не упоминается ни разу. До Первой
мировой войны \emph{Украина} была размером с Киевскую область. Разные народы, которые
были крепостными в разных государствах коммунисты собрали в одну автономию и
под угрозой увольнения заставили учить единый, искусственно созданный,
украинский язык, чтоб всему миру показать \enquote{Дружбу народов}. С таким же успехом
коммунисты могли назвать автономию Слобожанщиной. И искали бы теперь слобожанцы
себя в древних русских летописях и наверняка бы нашли, поменяв в каком то слове
две-три буквы. Предки \emph{украинцев} были русскими крепостными холопами, поскольку
не умели воевать на четыре столетия выпали из мировой истории с 1447 (привелей
Казимира 4) по 1848 год (Австрийская аграрная реформа) и не надо примазываться
к литовцам, полякам, русским и запорожским казакам. Вы к ним никакого отношения
не имеете. Холопы...
%%%cit_comment
Алексей Четин
%%%cit_title
\citTitle{Кто основал Москву? Вы даже удивитесь - ее основал великий киевский князь в XII веке и заселил ее украинцами}, 
Исторический Понедельник, zen.yandex.ru, 18.02.2021
%%%endcit

%%%cit
%%%cit_pic
%%%cit_text
А сам Великий князь знал, что он Русь \emph{украинцами} заселяет?
%%%cit_comment
Александр
%%%cit_title
\citTitle{Кто основал Москву? Вы даже удивитесь - ее основал великий киевский князь в XII веке и заселил ее украинцами}, 
Исторический Понедельник, zen.yandex.ru, 18.02.2021
%%%endcit

%%%cit
%%%cit_pic
%%%cit_text
Автор. Ты бы матчасть поучил что-ли. В 1147 году Юрий Долгорукий еще НЕ БЫЛ
киевским великим князем. На тот момент он был ростовско-суздальским князем и
киевский стол он занял только в 1149 году через 2 года после основания Москвы.
И во Владимире Юрий Долгорукий никогда не княжил, потому что Владимир был стал
стольным городом в 1158 году - уже после смерти Долгорукого. И что там у тебя
за \enquote{украинцы} дивные в 12 веке?
%%%cit_comment
Александр
%%%cit_title
\citTitle{Кто основал Москву? Вы даже удивитесь - ее основал великий киевский князь в XII веке и заселил ее украинцами}, 
Исторический Понедельник, zen.yandex.ru, 18.02.2021
%%%endcit

%%%cit
%%%cit_pic
%%%cit_text
Интересно Киевские князья в то время догадывалась, что они \enquote{\emph{Украинцы}} и что
Киевское княжество это оказывается \enquote{\emph{Незалежная}}??? 
%%%cit_comment
vadim malahov
%%%cit_title
\citTitle{Кто основал Москву? Вы даже удивитесь - ее основал великий киевский князь в XII веке и заселил ее украинцами}, 
Исторический Понедельник, zen.yandex.ru, 18.02.2021
%%%endcit

%%%cit
%%%cit_pic
%%%cit_text
Увы, но таков наиболее вероятный сценарий для \emph{Украины}.  Путин начертил красные
линии.  Каковы теперь возможны сценарии для \emph{Украины}?  Итак, Путин предельно
ясно дал понять, что попытка втягивать \emph{Украину} в НАТО или попытка де факто
строить в \emph{Украине} американские базы ( вне отношений с НАТО), будет означать
войну.  И у Запада нет сомнений, что Путин не шутит. Что именно так и будет.
Путин сознательно сделал столь обязывающее и предельно ясное заявление, потому
что хотел задать границы и фон, на котором будет обсуждаться вопрос \emph{Украины}
на встрече с Байденом. Какие в этой связи возможны сценарии для \emph{Украины}?
%%%cit_title
\citTitle{Путин начертил красные линии по Украине}, Андрей Головачев, strana.ua, 12.06.2021
%%%endcit

%%%cit
%%%cit_pic
%%%cit_text
Российская фигуристка Анастасия Скопцова, выступающая в танцах на льду в паре с
Кириллом Алёшиным, поделилась ожиданиями от стартовавшего чемпионата Европы по
футболу ... Желаю успеха всем нашим ребятам. Могут выстрелить парни из
европейских лиг — Алексей Миранчук, Саша Головин. Алексей поиграл в Италии год,
Саша — два во Франции. Это совсем другой уровень по сравнению с РПЛ. Надо
понимать, против каких команд они выходили. Это огромный опыт, который скажется
и на их игре за сборную. Прямо верю в ребят! И отдельно буду болеть за
\emph{украинца} Александра Зинченко. Он мне очень нравится в «Манчестер Сити».
Болела за «горожан» в недавнем финале Лиги чемпионов. Нравится этот клуб и его
тренер — я без ума от Пепа Гвардиолы. Поэтому буду поддерживать Зинченко», —
приводит слова Скопцовой Sport24
%%%cit_title
\citTitle{Российская фигуристка Скопцова: на Евро-2020 отдельно буду болеть за украинца Зинченко}, 
Яна Метлёва, www.championat.com, 12.06.2021
%%%endcit

%%%cit
%%%cit_pic
%%%cit_text
А мы на 75 лет старшего брата сделали \emph{украинский} концерт! Две песни,
третья, конечно, Пидманула, а первая, как понимаете, Нiчь, а вторая Чернмшина.
Да в венках, в спидницях с фартуками! Месяц готовились! А еще пионерскую
линейку с тремя отрядами и рапортами сделали! Вот это были потом воспоминания!
В нас много кровей намешано, но фамилия \emph{украинская}, дача у брата в с.
\emph{Украинка}! Брат и дядя слезы вытирали под наши песни!
%%%cit_comment
елена чупрова
%%%cit_title
\citTitle{5 задушевных украинских песен, которые пели наши родители, а теперь поем мы}, 
Кино Вояж И Не Только, zen.yandex.ru, 07.06.2021
%%%endcit

%%%cit
%%%cit_pic
%%%cit_text
Во мне тоже трохи \emph{украинской крови} е, песни обожаю, все эти пять песен хорошо
знаю. К самым известным можно добавить ЧЕРЕМШИНА И МЕСЯЦ НА НЕБЕ
%%%cit_comment
Марина Астраханцева
%%%cit_title
\citTitle{5 задушевных украинских песен, которые пели наши родители, а теперь поем мы}, 
Кино Вояж И Не Только, zen.yandex.ru, 07.06.2021
%%%endcit

%%%cit
%%%cit_pic
\ifcmt
  pic https://avatars.mds.yandex.net/get-zen_doc/1348874/pub_60be8c24942ceb0598e0db47_60be92f52b92e05b9b61c336/scale_1200
  caption Вот такая карта на футболке. Источник: Яндекс.Картинки
\fi
%%%cit_text
Второе несомненно лучше. Видимо \emph{на Украине} совсем отчаялись что когда то
Крым вернется к ним назад и решаются на отчаянные и бессильные поступки в виде
Крымской платформы или... карты \emph{Украины} с Крымским полуостровов на
футболках их футбольной сборной
%%%cit_title
\citTitle{Какая разница на каких футболках Крым, главное он в составе России}, 
Добрый Человек, zen.yandex.ru, 11.06.2021
%%%endcit

%%%cit
%%%cit_pic
%%%cit_text
Светлана Светлана да, радость моя, нацистам в \emph{Украине} я делаю именно
песца: вы уже столько месяцев стараетесь уволить, а меня не увольняют, доказать
профнепригодность, а у меня - Асademia Premium за международый индекс
цитирования и десятки интервью в неделю, лишить связей, а благодаря вам у меня
теперь - друзья в Европарламенте и мировой науке, лишить залов, а люди на
квартирнике под потолок забиваются, как при Башлачёве в СССР, лишить дохода, а
мне помогают от Вашингтона до Сибири. Ну, как вы заставите меня не любить
Россию? Грузом 200. Так вы меня глорифицируете. Смерти и обструкции я не боюсь.
Ещё что придумаете? Понимаете, что я - одна из немногих на \emph{Украине}, кто
конкретно не стал перед вами, бандеровцами, на колени? И не стану. Тип Зои.
Сделать ничего нельзя
%%%cit_title
Евгения Бильченко, facebook, 12.06.2021
%%%endcit

%%%cit
%%%cit_pic
%%%cit_text
  
Первые \enquote{\emph{украинцы}} появились в середине 19в в модных столичных салонах, а
столицей тогда был Санкт-Петербург. В Европе тогда появилась мода из подданных
империй вычленять национальности; придумывать им языки и национальную одежду.
Следующим ход для появления \enquote{\emph{украинцев}} сделали преподаватели-поляки в
Харьковском универе. К концу 19в самый рассадник \emph{украинства} был в этом городе.
Ну а массово \enquote{\emph{украинцы}} появились как по щелчку в 1914 - с началом Первой
мировой - с единственной целью развала Российской Империи и выхода ее из войны
%%%cit_comment
Григорий Клюев
%%%cit_title
\citTitle{Что должен знать каждый русский}, 
Школа Экзорцистов, zen.yandex.ru, 12.06.2021
%%%endcit

\emph{Украина}, типа, борется с фейками и ложными новостями, но сама живет в
искаженном фейковом пространстве, которое ее губит. Начнем с того, что \emph{Украина}
начинала независимость с невыгодных стартовых позиций - столько драгоценного
сырья, сколько имела Россия, у нее не было. Была промышленность, но она не была
заточена под рынок. Короче, необходимо было оценить ситуацию и как-то
выворачиваться из этого положения.  Но украинское руководство повелось на
мантры, что Запад нам поможет, а рынок все сам сделает, грохнули
промышленность, добивают энергетику и распродают землю. За короткий срок
\emph{Украина} стала белой Африкой, где стреляют, порядка нет и выхода из кризиса
тоже. И весь этот ужас и безысходность изрядно поливается пропагандой, что мы,
мол Европа, у нас тут свобода и небывалый экономический подъем, ну, ждем
буквально с дня на день.  Вся эта гордыня привела к тому, что все соседи живут
богаче \emph{Украины}. В современном мире бедная страна остается независимой только в
том случае, если она никому не нужна. Россия богаче \emph{Украины}, а потому
стремление пограничных регионов в нее, то есть, сепаратизм, экономически
детерминирован. И этого сепаратизма больше, чем желания России что-то себе
забрать
%%%cit_title
\citTitle{Зеленский продолжает раскалывать Украину по целому ряду признаков}, 
Денис Жарких, strana.ua, 13.06.2021
%%%endcit

%%%cit
%%%cit_pic
%%%cit_text
Тем, кто утверждает, что лозунг \enquote{Слава Украине! Героям Слава!} не имеет
отношения к ОУН, отвечает бывший глава \emph{Украинского} института национальной
памяти Владимир Вятрович: \enquote{Символику ОУН (меч-тризуб, красно-черный
флаг) поднимают нынешние защитники \emph{Украины} в войне с Россией. Гимн ОУН
«Зродились ми великої години» стал маршем Вооруженных сил \emph{Украины}, а
организационное приветствие «Слава \emph{Україні}! Героям слава!» с прошлого года -
официальное приветствие новой \emph{украинской} армии}.  Из интервью
\enquote{Укринформ} от 04.02.2019
%%%cit_title
\citTitle{Вы еще утверждаете, что лозунг \enquote{Слава Украине! Героям Слава!} не имеет отношения к ОУН?}, 
Эдуард Долинский, strana.ua, 13.06.2021
%%%endcit

%%%cit
%%%cit_head
E...q украинский язык 
%%%cit_pic
%%%cit_text
В своих Instagram-историях Маргарита пожаловалась на требования к курсовой
работе и нецензурно высказалась об \emph{украинском языке}, заявив, что часть
материалов она и так перевела с английского, а теперь ей придется делать еще
один перевод.  "Пишу курсовую работу на е**чем \emph{украинском языке}. Вот понимаю,
что \emph{украинская власть} – она меня ущемляет, ущемляет мои права. Что значит – я
не могу писать свою работу, в которой мои научные заключения, мои выводы, в
которой описано, как я проводила эту работу, почему я не могу писать ее на
русском языке... Нравится \emph{украинский язык}? Переводите, пожалуйста. Почему вы
насилуете меня, мою психику и мои нервы", – сказала Маргарита

%%%cit
%%%cit_head
%%%cit_pic
%%%cit_text
\emph{Украинцы} достаточно умные люди, чтобы сделать любую глупость...  Я, как
бы помягче это сказать, недоумеваю, глядя на позицию наших властей по отношению
к творящихся в/на \emph{Украине} событий, процессов и явлений направленных
только на одно - как бы по гаже плюнуть в сторону России. Иных целей на текущий
момент у \enquote{настоящих} \emph{украинцев} нет
%%%cit_comment
Edouard Agadjanian
%%%cit_title
\citTitle{Когда власть в стране принадлежит команде КВН}, 
Мак Сим, zen.yandex.ru, 05.06.2021
%%%endcit

%%%cit
%%%cit_head
%%%cit_pic
\ifcmt
  pic https://avatars.mds.yandex.net/get-zen_pictures/3403730/584714379-1622956392144/orig
  caption Специальный Путин для Зеленского
\fi
%%%cit_text
Выступления, заявления Зелинского и его команды достойны пластмассового кивина
за политический юмор. Коллеги по цеху из КВН и др шоу угорают когда \emph{укрполитики}
несут чушь с серьезным выражением физиономии
%%%cit_comment
Майоров
%%%cit_title
\citTitle{Когда власть в стране принадлежит команде КВН}, 
Мак Сим, zen.yandex.ru, 05.06.2021
%%%endcit

%%%cit
%%%cit_head
%%%cit_pic
%%%cit_text
Примечательно, что власти \emph{Украины} пытаются заставить всех забыть о годовщине
инцидента. Почему так происходит, ведь это прекрасный повод напомнить о
\enquote{российской агрессии} и \enquote{зверствах сепаратистов}? А все дело в том, что вина
\emph{украинского} командования в уничтожении полусотни украинских же военнослужащих
очевидна, более того, ее доказал даже \emph{украинский} же суд. Еще в марте 2017 года
за служебную халатность, повлекшую гибель подчиненных, к семи годам заключения
был приговорен бывший начштаба – первый заместитель командующего \enquote{АТО} генерал
Виктор Назаров. Однако генерал сначала занял глухую оборону на юридическом
фронте, а затем и вовсе перешел в судебное наступление. В итоге в конце мая
2021 года решением кассации \emph{Верховного суда Украины} решение об осуждении
Назарова было отменено, а само дело было закрыто в связи с отсутствием в
действиях обвиняемого состава преступления
%%%cit_comment
%%%cit_title
\citTitle{Луганский Информационный Центр – НЕДЕЛЯ ГЛАЗАМИ ЭКСПЕРТА: Саммит в Женеве, разрыв артерии и Ил забвения}, 
, lug-info.com, 20.06.2021
%%%endcit


%%%cit
%%%cit_head
%%%cit_pic
%%%cit_text
Якщо уважно придивитися до путінської Росії за останні років двадцять, то можна
легко зауважити відродження боротьби ще сталінського зразка на всіх фронтах і
напрямках. Тому дивно, що ця запекла боротьба і викидання на неї без
перебільшення валютних трильйонів майже залишається поза увагою аналітиків.
Дивно, що не виникає бажання предметно розібратися, за що борються і проти кого
воюють? Що бережуть і чим жертвують?  Але, переводячи погляд з Росії на
\emph{Україну}, виникає дивне відчуття, що \emph{українці} багато в чому наслідують росіян.
Щоправда, відбувається це з меншим ресурсом, зміненими зовнішніми орієнтирами і
відчутною \emph{«українською»} специфікою. У зв’язку із цим виникає запитання до обох
народів, чому вони дозволяють правлячим елітам витрачати велетенський ресурс не
на розвиток країни і покращення їхнього життя, а на суцільний деструктив і
боротьбу зі Заходом?
%%%cit_comment
%%%cit_title
\citTitle{Подвійний ворог України - ZAXID.NET}, 
Василь Расевич, zaxid.net, 18.06.2021
%%%endcit

%%%cit
%%%cit_head
%%%cit_pic
%%%cit_text
Захыснык \emph{украинского} языка Кремень напомнил нам Конституцию: \enquote{В 2019 году на
выполнение 10 статьи Конституции, которая требует обеспечить всестороннее
развитие и функционирование \emph{украинского} языка во всех сферах общественной жизни
на всей территории \emph{Украины}, наконец был принят \emph{Закон Украины} \enquote{Об обеспечении
функционирования \emph{украинского языка} как государственного}. Битва за язык
продолжается и сейчас}
%%%cit_comment
%%%cit_title
\citTitle{Защищать украинский язык можно и без угнетения русского / Лента соцсетей / Страна}, 
Александр Скубченко, strana.ua, 28.06.2021
%%%endcit


%%%cit
%%%cit_head
%%%cit_pic
%%%cit_text
У 1843 році в \emph{Україні} офіційно почав діяти \enquote{Звід законів Російської імперії},
який формально залишався основним законом до лютого 1917 року. У цей похмурий
період \emph{українські} громадські діячі створили кілька проєктів конституцій.
\enquote{Заповідна державна грамота}, вона ж \enquote{Руська правда} декабриста Павла Пестеля
1824 року, \enquote{Начерки Конституції Республіки} кирило-мефодіївця Георгія
Андрузького 1847 року, \enquote{Вольный Союз – Вільна Спілка} Михайла Драгоманова 1884
року, \enquote{Основний Закон самостійної \emph{України}} Миколи Міхновського 1905 року – ось
неповний список. За ними можна відстежити еволюцію української державної думки
– від обласництва в неподільній Росії через суб'єкт федерації до повної
незалежності
%%%cit_comment
%%%cit_title
\citTitle{Українській Конституції – 2300 років}, 
Сергій Громенко, gazeta.ua, 26.06.2021
%%%endcit

%%%cit
%%%cit_head
%%%cit_pic
%%%cit_text
Другой путь - выбросить за борт иных, отгородиться заборами от \enquote{чужих},
покрасить всех в один цвет и одеть в одну сорочку, жить с \enquote{одинаковыми}. Верить
в мудрость гетмана, ждать \enquote{сигналов} и \enquote{мудрых решений} сверху.  Ну,
собственно, к чему и скатываемся. Правда, сколько останется в такой \emph{Украине} от
\emph{Украины-1990} - огромный вопрос...
%%%cit_comment
%%%cit_title
\citTitle{Весь путь нашей независимости - путь потерь, бесконечных конфликтов и даже войны / Лента соцсетей / Страна}, 
Андрей Ермолаев, strana.ua, 30.06.2021
%%%endcit

%%%cit
%%%cit_head
%%%cit_pic
%%%cit_text
\emph{Украина} никогда не имела своей государственности и в разные времена разные
части территории нынешней \emph{Украины} управлялись разными (историки насчитывают
больше 10) внешними центрами силы. Для тех, кто были частью Российской империи
(Советского Союза) языковая и культурная идентичность с русским народом
естественна и неоспорима. Так же как и неоспорима особость \emph{Западной Украины},
которая стала частью Советского Союза только в 1939 году, а раньше никогда не
была \enquote{под Москвой}. В некоторой мере про особость можно сказать и о части
окатоличеной поляками части \emph{Украины}. \emph{Украина} остаётся лоскутным одеялом,
расколотой страной, о чём свидетелствуют сотни опросов общественного мнения.
Сейчас мы свидетели попыток гомогенизировать страну по лекалам \emph{Западной
Украины}. Отсюда - НАТО, ЕС, атака на русский язык и православие и т.п.
%%%cit_comment
%%%cit_title
\citTitle{Украина остается лоскутным одеялом, расколотой страной / Лента соцсетей / Страна}, 
Михаил Погребинский, strana.ua, 01.07.2021
%%%endcit

%%%cit
%%%cit_head
%%%cit_pic

\ifcmt
  tab_begin cols=3
     width 0.3

     pic https://avatars.mds.yandex.net/get-zen_doc/5290304/pub_60dcfebf828fb32711387e20_60dd0da8d08a1a0b513bb043/scale_1200

     pic https://avatars.mds.yandex.net/get-zen_doc/5235248/pub_60dcfebf828fb32711387e20_60dd0db64f91615070480489/scale_1200

     pic https://avatars.mds.yandex.net/get-zen_doc/5327700/pub_60dcfebf828fb32711387e20_60dd0dc3104ca22aeabe2575/scale_1200

  tab_end
\fi

%%%cit_text
\obeycr
  - Алло, слава Украине, это ЖЭК?
  - Героям слава, да, это ЖЭК.
  - У нас в квартире очень холодно, слава Украине, когда будет тепло?
  - Летом! Героям слава.
\restorecr
%%%cit_comment
%%%cit_title
\citTitle{Анекдот дня (об, в, на, за, про Украину) | Крымский кот | Яндекс Дзен}, 
Крымский Кот, zen.yandex.ru, 01.07.2021
%%%endcit


%%%cit
%%%cit_head
%%%cit_pic
%%%cit_text
Из всего этого можно сделать вывод, что \emph{Украине} пообещали, дали некие гарантии
по поводу невозможности ввода СП-2, если Киев будет придерживаться
антироссийской политики. Чубатые хлопцы все горшки побили с Россией на потеху
США, западу, ЕС, ну а теперь \emph{украинский Табаки} понимает, что его обманули, о
нем вообще не думали, о него вытерли ноги. Наверное, в \enquote{форпосте запада против
России} чувствуют не просто обиду и разочарование, а ощущают себя круглыми
дураками, ведь их \enquote{тыл}, вся Европа и США спокойно торгуют, сотрудничают с
Москвой, чему СП-2 свидетель. Ну а \emph{Украине} остаётся лишь обнести свои границы
рвом по всему периметру, потому что кругом зрада.  Ладно, с \emph{Украиной} то более
менее понятно, \emph{никто и не ожидал от неё здравомыслия}. Но, видать, польские
политики подхватили в Киеве некий вирус, по другому объяснить последнее
заявление Варшавы по СП-2 не могу
%%%cit_comment
%%%cit_title
\citTitle{Вой коллективного Табаки | Мак Сим | Яндекс Дзен}, 
Мак Сим, zen.yandex.ru, 29.06.2021
%%%endcit

%%%cit
%%%cit_head
%%%cit_pic
%%%cit_text
После того, как автор победного гола матча \emph{Украина}-Швеция на Евро-2020 Артем
Довбик на пресс-конференции отвечал на русском языке, на него обрушились с
критикой националисты.  Нападающего сборной \emph{Украины}, который побил
рекорд легендарного Мишеля Платини, раскритиковала скандальная писательница
Лариса Ницой, которая и ранее называла \emph{украинских футболистов}
\enquote{кончеными московитами}. А языковой омбудсмен Креминь потребовал от
футболистов общаться на \emph{украинском} даже после завершения карьеры
%%%cit_comment
%%%cit_title
\citTitle{Зеленский о русских и русском языке. Что будущий президент говорил в 2014 году}, 
Екатерина Терехова, strana.ua, 03.07.2021
%%%endcit

%%%cit
%%%cit_head
%%%cit_pic
%%%cit_text
Щоправда, відмінність між ними таки є. Тільки в росіян \emph{Україна} за
означенням є безперспективною і приреченою залишитися в орбіті Росії, а в
прихильників п’ятого президента та його ручних медіа – в \emph{України} нема
майбутнього, якщо на її чолі не стоїть Порошенко. Небезпека цієї ситуації
полягає в тому, що реально гігантський інформаційний потенціал задіяний на те,
щоб показати, що \emph{Україна} завалиться. І в тому, що \emph{«українська»}
складова цього потенціалу є не менш руйнівною. Бо намагається роз’їдати
ситуацію зсередини
%%%cit_comment
%%%cit_title
\citTitle{Вболівальники поразки - ZAXID.NET}, Василь Расевич, zaxid.net, 25.06.2021
%%%endcit


%%%cit
%%%cit_head
%%%cit_pic
%%%cit_text
\emph{Украина} умерла в 1991 году при развале огромной страны. Теперь это и
вовсе квазиобразование, раковая опухоль, которую необходимо как можно скорее
удалять, иначе метастазы приведут к страшным последствиям для окружающих.  СТУК
МОЛОТКА.  С 1 июля \emph{Украина} официально пошла с молотка: вступил в силу
закон, отменяющий запрет на продажу земель сельскохозяйственного назначения.
Официально теперь владельцы сельскохозяйственных наделов смогут свободно их
продать, а любой желающий гражданин \emph{Украины} — купить
%%%cit_comment
%%%cit_title
\citTitle{Луганский Информационный Центр – НЕДЕЛЯ ГЛАЗАМИ ЭКСПЕРТА: 
Пеленг дозволенного, штампы внешнего управления и выходной бастард}, , lug-info.com, 04.07.2021
%%%endcit

%%%cit
%%%cit_head
%%%cit_pic
%%%cit_text
Полную передачу под внешнее управление \emph{Украины} можно считать одобренной
законопроектом № 3711-д, который требовали принять послы стран G7 и
представительство ЕС на \emph{Украине}. Согласно документу, решающую роль в
определении состава Высшего совета правосудия и Высшей квалификационной
комиссии судей будут играть иностранные эксперты. То есть \emph{украинских судей}
будут назначать иностранцы, а, значит, \emph{украинская Фемида} приказала долго жить,
и последняя из трех ветвей власти теперь тоже в руках Запада.  Был принят и ряд
других законов, которые в конечном итоге обернутся большой бедой для \emph{украинцев}
и вызовут проблемы в отношениях Киева с соседними государствами. Самый
чудовищный из них – закон о коренных народах, из числа которых исключили
русских
%%%cit_comment
%%%cit_title
\citTitle{Луганский Информационный Центр – НЕДЕЛЯ ГЛАЗАМИ ЭКСПЕРТА: 
Пеленг дозволенного, штампы внешнего управления и выходной бастард}, , lug-info.com, 04.07.2021
%%%endcit

%%%cit
%%%cit_head
%%%cit_pic
%%%cit_text
Запад с нами, но то не точно!  Бар у бассейна. Сижу, пью пиво. Рядом со мной
шумная компания англоязычных граждан. Судя по говору - американцы. Возраст
20-25, визуально.
Вдруг рядом плюхается в воду чей-то ребёнок и меня с этой всей компанией
накрыло брызгами. Посмеялись, перекинулись парой фраз по поводу смешного
ребёнка и сидим дальше. Но англоязычной барышне было видимо скучно в компании
своих друзей и она решила завести лёгкий трёп.
\begin{itemize}
  \item - А ты откуда, местный? (Интересуется у меня)
  \item - Нет, из \emph{Украины} (Удивлённый тем, что я похож на обитателя Кипра, отвечаю ей)
\end{itemize}
Девушка одобрительно кивнула головой и продолжила.
\begin{itemize}
  \item - А у вас там сейчас снег, наверное?
\end{itemize}
%%%cit_comment
%%%cit_title
\citTitle{Для рядового американца мы - это какое-то Бурунди / Лента соцсетей / Страна}, 
Василий Апасов, strana.ua, 05.07.2021
%%%endcit

%%%cit
%%%cit_head
%%%cit_pic
%%%cit_text
Ми отримали незалежність, але ми не отримали Україну. \emph{Українську Україну}, де
була б «в своїй хаті своя правда і сила, і воля», як писав наш пророк Тарас
Шевченко. Тобто де була б своя мова, своя історія, свої герої. І ми сьогодні не
сперечалися б, кому ставити пам’ятники та чиї імена мають носити наші вулиці.
Спочатку треба було сотворити націю, з населення, з народу \emph{України} сотворити
український народ. Немає народу Росії, Франції або Німеччини. Є російський
народ, французький, німецький... Щоб бути \emph{українцем}, треба трохи потрудитися.
Бо якщо ви живете в державі, назва якої \emph{«Україна»}, значить, треба знати і
поважати її цінності, історію, традиції, мову
%%%cit_comment
%%%cit_title
\citTitle{Герой України, народний артист Анатолій Паламаренко: «Буде мова — буде нація, буде нація — буде держава»}, 
Ліна Тесленко, www.umoloda.kiev.ua, 02.07.2021
%%%endcit

%%%cit
%%%cit_head
%%%cit_pic
%%%cit_text
Концентруючи стільки уваги та енергії на протистоянні Росії, можна забути, що є
тільки один шлях по-справжньому виграти в цій боротьбі – збудувавши сильну
державу, прийнявши і усвідомивши нашу \emph{українську ідентичність} в усій її
суперечності консерватизму і сучасності.  Серед експатів, що густо замешкали в
Києві після Революції Гідності, часто чулася прогресивна для
\emph{українського} суспільства думка: слід позбавити Росію монополії на
російську мову, привласнивши її. Адже російська в різних країнах відрізняється
так само, як і англійська
%%%cit_comment
%%%cit_title
\citTitle{Українці не розуміють одне одного не через мову, а через небажання слухати, чути і сприймати}, 
Юлія Мендель, www.pravda.com.ua, 07.07.2021
%%%endcit

%%%cit
%%%cit_head
%%%cit_pic
%%%cit_text
В пятницу 16 июля в \emph{Украине} вступают в силу новые нормы закона об \emph{украинизации}.
Который на днях как раз признали соответствующим Основному закону - так решил
Конституционный суд.  Это не значит, что закон теперь нельзя менять. И такие
попытки были, но провалились.  Так, "слуги народа" хотели, но не так и не
решились вынести на рассмотрение Рады поправки, которые бы откладывали, к
примеру, обязательный \emph{украинский дубляж} фильмов на телевидении. Он стартует как
раз в июле. Еще с мая фракция президента пыталась эти поправки протянуть, но
после обвинений в зраде со стороны националистов каждый раз отступала.  И, по
итогу, с 16 июля новые \emph{украинизаторские} нормы вступают в силу.  "Страна"
рассказывает, что они меняют в жизни \emph{украинцев}
%%%cit_comment
%%%cit_title
\citTitle{\enquote{Операция Ы} на мове и экзамены для чиновников. Что украинизируют с 16 июля}, 
Максим Минин, strana.ua, 15.07.2021
%%%endcit

%%%cit
%%%cit_head
%%%cit_pic
%%%cit_text
Нужно понимать, что в основе данной антироссийской идеологии лежат не интересы
своего народа, а интересы тех, кого нынешняя власть пытается защитить собой.
Власть надеется, что ситуация враждебности между Россией и Европой/США будет
продолжаться бесконечно долго, ведь в таком случае антироссийское государство
якобы будет необходимо Западу ‒ его будут финансировать, вооружать, но рано или
поздно от его услуг откажутся.  Но нужно ли сегодня Европе государство, главной
задачей которого будет сдерживание России? Ответ очевиден – нет. В противном
случае \emph{Украина} получала бы намного больше поддержки и никакого
«Северного потока ‒ 2» не было. Но \emph{украинская власть} успешно игнорирует
реальность, считая, что ситуация именно такая, какую ей удобно представлять.
Это и есть главная и широко транслируемая властью безрассудная романтическая
«хотелка»
%%%cit_comment
%%%cit_title
\citTitle{О будущем украинского и русского народов}, 
Виктор Медведчук, strana.ua, 15.07.2021
%%%endcit

%%%cit
%%%cit_head
%%%cit_pic
%%%cit_text
Внимательно читаем и… не находим ничего нового, то есть такого, что бы не было
бы нам известно о его отношении к нашей стране и без его исторической
беллетристики. Путин просто и понятно подводит историческую базу (в собственной
интерпретации) под следующие тезисы:
\begin{itemize}
\item 1. \emph{Украины} не существует.
\item 2. \emph{Украинцев}, как нации, не существует.
\item 3. \emph{Украинского языка} не существует, он, якобы, всего лишь диалект русского
языка, который возник вследствие «разделения единого народа (Западом, понятное
дело) и возникшей обособленности». Правда, в отношении языка трудно однозначно
понять президента РФ: то его не существует, то он (Путин), ближе к концу
статьи, его уже уважает. Вы как-нибудь определитесь уже, Владимир Владимирович
(хотя, может быть действительно, сам писал).
\item 4. Все вышеперечисленное (\emph{Украина}, \emph{украинцы и украинский язык}) – возникли «в
среде польской элиты и некоторой части малороссийской интеллигенции …
Укреплялись представления об отдельном от русского украинском народе…
Австро-венгерские власти подхватили эту тему…» В общем, \emph{Украина}, \emph{украинцы} и
\emph{украинский язык} придуманы властями Австро-Венгрии с целью ослабить Российскую
империю перед Первой Мировой Войной.
\item 5. \emph{Современная Украина} – квази-государственное, искусственное образование,
созданное Западом в рамках проекта «анти-Россия», доказавшее свою
несостоятельность во всех сферах деятельности. Следует понимать, что под
термином «анти-Россия», Владимир Владимирович понимает любое независимое
демократическое государство вблизи границ РФ
\end{itemize}
%%%cit_comment
%%%cit_title
\citTitle{«Историческое единство русских и украинцев» и внешнеполитические метания Украины}, 
Игорь Балута, analytics.hvylya.net, 28.07.2021
%%%endcit
