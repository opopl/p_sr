% vim: keymap=russian-jcukenwin
%%beginhead 
 
%%file 07_01_2021.stz.news.ua.mrpl_city.1.prostir_shahmatna_koroleva
%%parent 07_01_2021
 
%%url https://mrpl.city/blogs/view/mobilnij-interaktivnij-prostir-shahmatna-koroleva-1
 
%%author_id demidko_olga.mariupol,news.ua.mrpl_city
%%date 
 
%%tags 
%%title Мобільний інтерактивний простір "Шахматна королева"
 
%%endhead 
 
\subsection{Мобільний інтерактивний простір \enquote{Шахматна королева}}
\label{sec:07_01_2021.stz.news.ua.mrpl_city.1.prostir_shahmatna_koroleva}
 
\Purl{https://mrpl.city/blogs/view/mobilnij-interaktivnij-prostir-shahmatna-koroleva-1}
\ifcmt
 author_begin
   author_id demidko_olga.mariupol,news.ua.mrpl_city
 author_end
\fi

5 січня у Міському палаці культури (Центр культури Лівобережний) відбулася
презентація нового креативного простору \enquote{Шахматна королева}. Реалізація цього
проєкту дозволить залучити якомога більше населення до гри в шахи, а також
розповсюдити інформацію про наявність та діяльність творчих колективів у Палаці
культури.

\ii{07_01_2021.stz.news.ua.mrpl_city.1.prostir_shahmatna_koroleva.pic.1}

Проєкт \enquote{Шахова Королева} центру культури Лівобережний став одним з переможців
міської програми \enquote{Громадський бюджет}. Це театральна постановка, прикрасою якої
стали величезні шахові фігури.

Автори цієї ідеї – беззмінний керівник Клубу міського палацу культури шахового
мистецтва \enquote{Паламед} \emph{\textbf{Олег Дмитрович Темірбек}} та керівники творчих колективів
Міського палацу культури на чолі з художньою керівницею \emph{\textbf{Тетяною Живолугою}}. Як
наголошує директорка Міського палацу культури (ЦК \enquote{Лівобережний}) \emph{\textbf{Тетяна
Обєдкова}}, це рухливий проєкт, який дозволить з одного боку долучити людей до
гри в шахи, але водночас це й іміджевий для Палацу проєкт, адже відбувається
популяризація всіх жанрів. Колективи, що братимуть участь у виставі, постійно
змінюватимуться.

Завдяки мобільності постановку можна буде показувати у приміщеннях та на
відкритих майданчиках. В рамках проєкту маріупольці побачили виставу
\emph{\textbf{\enquote{Абсолютний таск}}}, що означає рекордний ідейний задум в шаховій композиції,
який вже перевершити неможливо, хіба що повторити. До речі, це єдина постановка
центру культури за період пандемії. Сценарій до вистави написала художня
керівниця Міського палацу культури Тетяна Живолуга.

\ii{07_01_2021.stz.news.ua.mrpl_city.1.prostir_shahmatna_koroleva.pic.2}

Тетяна Володимирівна розповідає, що \enquote{перші рядки взяла з інтернету, а потім
якось воно само полилося, тому що була дуже драйвова синергія команди}. Художня
керівниця розповіла й про зміст танцювальної вистави... Навколо королеви
змінився світ і вона не розуміє, що відбувається, іноді здається, що серце
виснажене, сил вже зовсім не залишилося, а біль все продовжується і
продовжується. Чорні атакують, але королева своєю монаршою рукою дає сигнал
зупинитися і тоді її військо збільшується вдвічі, адже до білих приєднуються і
чорні. Живолуга наголошує, що театралізація досить проста, але атмосфера
неймовірна.

Вже декілька років при Центрі культури існує \emph{Клуб любителів шахового мистецтва
\enquote{Паламед}}. Його керівник Олег Темірбек активно пропагує гру в шахи серед
маріупольців. Чоловік виступив консультантом при підготовці театралізації. Олег
Дмитрович стверджує, що шахи є корисними для людини будь-якого віку. Це дуже
корисна справа, яка розвиває діаметрально протилежні якості та сприяє вихованню
особистості. Театральна постанова \emph{\textbf{\enquote{Абсолютний таск}}} – це тільки частина
проєкту. До заходів його реалізації також увійшли:

\ii{07_01_2021.stz.news.ua.mrpl_city.1.prostir_shahmatna_koroleva.pic.3}

- популяризація шахового мистецтва з використанням професійних навчальних
технік серед різних верст населення;

- створення самостійних культурних подій (майстер-класи, лекції, ігрові
програми, театралізовані інтерактивні шоу з долученням глядачів як на
відкритому повітрі, так і в приміщеннях;

- використання \enquote{Мобільного простору} як окремої самодостатньої видовищної
локації на міських святах (День міста, Новорічна кампанія, День Європи, тощо);

- пошук активної молоді для створення нових клубів, організацій  шахового
напряму;

- створення молодіжного шахового клубу на базі Клубу \enquote{Паламед} КУ \enquote{Міський
Палац культури}.

З огляду на цей унікальний і корисний проєкт можна вважати, що 2021 рік для
маріупольців починається досить потужно. А Вистава \enquote{Абсолютний таск} може стати
яскравою візитівкою ЦК \enquote{Лівобережний} та заряджати містян на високі результати
і нові рекорди.

\ii{07_01_2021.stz.news.ua.mrpl_city.1.prostir_shahmatna_koroleva.pic.4}
