% vim: keymap=russian-jcukenwin
%%beginhead 
 
%%file people
%%parent other
 
%%url https://sites.google.com/site/svetickusnerova/pidgotovka-do-zno-ta-vstupnih-ispitiv/osobi-istoria-ukraieni
%%author 
%%tags 
%%title 
 
%%endhead 

\chapter{Люди Руси}

\Purl{https://sites.google.com/site/svetickusnerova/pidgotovka-do-zno-ta-vstupnih-ispitiv/osobi-istoria-ukraieni}

Особи історія України

Василій (Володимир) Романюк (1925—1995) Український суспільно-релігійний діяч, єпископ неканонічної УПЦ-КП, в 1993-1995 рр. її предстоятель з титулом«Патріарх Київський і всієї Русі-України». Диссидент, член УХГ. Виступав за створеннянезалежної Української Православної Церкви з центром у Києві, піднесеної до статусу Патріархату. 1995 смерть «чорний вівторок».

Микола Руденко (1920–2004) Український поет, письменник, учасник Другої світової, правозахисник, один із засновників і голова Української Гельсінської групи, неодноразово заарештований і засуджений за «антирадянську агітацію і пропаганду», перебував на примусовому психіатричному лікуванні, автор заборонених поетичних творів «Всесвіт у тобі», «Хрест», «Я вільний».лауреат Державної премії ім. Т. Шевченка (посмертно). Наприкінці 60-х вийшли 3 збірки його поезій, поема про голод 1933 р., запроторений у психіатричну лікарню, ув’язнений в 1977 р.

Яновський Юрій 1902-1954 Один із найвизначніших романтиків в українській літературі першої половини XX століття. Редактор журналу «Українська література», військовий журналіст. Ветеран Армії УНР. У 1947 р. було звинувачено в прихильності до «українського буржуазного націоналізму». Роман "Вершники", «Київські оповідання» .

Корнійчук Олександр 1905-1972 пмсьменник, Голова Верховної Ради Української РСР, Виступив на пленумі СПУ в 1947 р. проти „буржуазних націоналістів” М. Рильського, Ю. Яновського, Л. Смілянського, О. Довженка, І. Сенченка. Написав лібрето для опери Костянтинв Данькевича "Богдан Хмельницький".

Сенченко Іван 1901=1975 — український письменник та журналіст. Належав до літературних організацій «Плуг» і «Гарт», потім до ВАПЛІТЕ (1925—1928). Автор замовних нарисів про концтабірне життя Біломорсько-Балтійського каналу.

Лисенко Трохим на серпневій сесії Всесоюзної сільгоспакадемії ім. Леніна Всесоюзної сільгоспакадемії ім. Леніна 1948 року, було остаточно розгромлено радянську генетику, яка ще в середині 30-х рр. минулого століття посідала чільне місце у світі. Радянській і українській науці було завдано удару, від якого вона так і не оговталася

Поляков Ілля генетик, Наукові праці присвячено експериментальній ботаніці, теоретичній біології, дарвінізму, історії біології.

Делоне Лев 1891-1969, укр. генетик і цитолог; доктор біології; праці з цитології, генетики, каріосистематики рослин; перший в Україні здійснив штучний мутагенез у рослин.

Гришко Микола Учений у галузі генетики і селекції рослин, фундатор Національного ботанічного саду НАН України. У 1948 р. звільнений від виконання обов'язків голови відділу сільсько-господарських наук АН УРСР і позбавлений можливості займатися генетикою.

Лебедев Сергій на початку 50-х років XX ст. науковці АН УРСР під його керівництвом створили першу в СРСР і Європі електронно-обчислювальну машину «МЕОМ».

Вернадський Володимир 1918-1919, першим президентом Академії наук. Ідеї лягли в основу нових провідних напрямків сучасної мінералогії , геології , гідрогеології , визначив роль організмів у геохімічних процесах. Для його діяльності характерні широта інтересів, постановка кардинальних наукових проблем, наукове передбачення. «Біосфера» 1911

Хмара Степан  - видав 7-й та 8-й номери "Українського вісника" у 1974 році.

Караванський Святослав - український мовознавець, поет, перекладач,журналіст, автор самвидаву. Багатолітній в'язень концтаборів СРСР 1944–1960, 1965–1979. Член ОУН. член Української гельсинської групи (1976), автор «Словника українських рим». З 1979 р. — в еміграції. Дружина Ніна Строката член УГГ, засуджена у 1972 р.

Кандиба Іван співзасновник Української робітничо-селянської спілки (1959)., за що отримав 15-річне ув’язненняЧлен Української гельсинської групи (1976).

Горинь Михайло (1930 р.н.) – У 1965 заарештовано і в 1966 засуджено до ув'язнення на шість років. Після звільнення працював на різних роботах у Львові. Видавав «Бюлетень Гельсінської групи». У грудні 1981 заарештований і засуджений до 10 років ув'язнення в таборах особливо суворого режиму. У 1987 – повертається до Львова, бере участь у виданні «Українського вісника», у створенні УГС, УРП. У 1990-1994 – народний депутат України, голова підкомісії Комісії Верховної Ради України з питань державного суверенітету. У 1990-1992 – заступник голови, співголова НРУ. У 1992-1995 – голова УРП. З жовтня 1995 – почесний голова УРП.

Левицький Орест 1919-1921 Президент Української (Всеукраїнської) академії наук. історик, етнограф, письменник. Член Історичного товариства імені Нестора-Літописця, Київського товариства старожитностей і мистецтв.

Липський Володимир ботанік, 1922-1928 Президент Української (Всеукраїнської) академії наук. директор Одеського ботанічного саду.

Паладін Олександр 1946-1962 Президент Української (Всеукраїнської) академії наук.

Некрасов Віктор український, радянський письменник і дисидент. Російською мовою - «В окопах Сталінграда».

Стельмах Михайло Загальне визнання принесли письменникові романи «Велика рідня», «Кров людськане водиця», «Хліб і сіль»

Бучма Амвросій 1891-1957 актор і режисер, актор театру «Березіль». На Одеській кінофабриці зіграв першу роль — Тараса Шевченка в однойменному фільмі 1926 року.

Ужвій Наталія радянська акторка театру і кіно.

Глущенко Микола 1901–1977 живописець, який 20 років прожив у Європі, де працював шпигуном радянської розвідки під агентурним псевдонімом “Ярема”. Працюючи на радянську розвідку, був одним з тих, хто заздалегідь (у січні 1940-о року) поінформував СРСР про те, що фашистська Німеччина готує напад. З 1944-о року оселяється в Києві.

Хлебанов Дмитро 1907-1987 український радянський композитор

Дерегус Михайло 1904–1997 живописець та графік. Найвідоміші графічні серії роботи Дерегуса: “Катерина” (офорт, монотипія, 1936–1938), “По дорогах війни” (офорт з акватинтою, 1943), “Українські народні думи і історичні пісні” (офорт, м’який лак, акватинта, вугілля, 1947–1950).

Шовкуненко Олександр 1884–1974 1920-30 рр. акварелі. У роки Другої Світової війни він пише портрети героїв, письменників, поетів, художників, артистів, вчених: П. Тичини, архітектора В. Заболотного, академіків О. Палладіна, О. Богомольця, С. Ковпака (1945) та ін.

Маланчук Валентин секретар ЦК КПУ з питань ідеології, науки та культури, «невтомний борець з українським націоналізмом»

Гершензон Сергій Протягом 1937–1948 рр. був завідувачем кафедри генетики і дарвінізму Київського університету. Через гоніння на генетику не міг представити свої дослідження, тому Нобілівську премію 1975 р. отримали американеці за синтез вірусної ДНК на матриці зараженої поліедрозом РНК.

Апанович Олена 1919 -2000 історик, досліджувала українське козацтво. 12 вересня 1972 року була звільнена за політичні погляди. Формально «за скороченням штатів», а фактично за дисиденство в науці. 1994 року поновлена на роботі як така, що звільнена «за політичними мотивами».

Брайчевський Михайло 1924-2001 читав лекція у Клубі творчої молоді «Сучасник»; 1966 року написав трактат «Приєднання чи возз'єднання?». Робота спочатку поширювалася «самвидавом», а 1972 року вийшла брошурою в Канаді. За це вченого звільнили з роботи. Продовжуючи писати «в шухляду», Михайло Брайчевський створив у ці роки низку монографічних досліджень і десятки статей, частину з яких не опубліковано й досі.

Компан Олена Доктор історичних наук (з 1965 року). Дружина письменника Івана Сенченка. 1972 року було звільнено з роботи «за пропаганду буржуазно-націоналістичних ідей і дружні взаємини з репресованими дисидентами».

Севрук Галина художник, Співавторка (разом з Опанасом Заливахою й Людмилою Семикіною) знищеного в травні 1964 року вітражу в Київському університеті. 1968 року виключена з Спілки художників України за підписання «Листа-протесту 139» на захист репресованих.

Заливаха Опанас художник у творчості якого провідним був образ гуцульської мадонни. Член Клубу творчої молоді «Сучасник» (1959—1964). 1964 разом з Аллою Горською отримала замовлення зробити у вестибюлі Київського університету вітраж до ювілею Тараса Шевченка. У день відкриття робота була вщент розбита. Заарештований 1965 р. із забороною малювати в таборах. Нагороджений Державною премією України імені Т. Г. Шевченка.

Іллєнко Юрій оператор визначного українського кінофільму “Тіні забутих предків”, режисер, а точніше сказати автор кінофільмів “Криниця для спраглих”, “Вечір на Івана Купала”, “Білий птах з чорною ознакою”, “Легенда про княгиню Ольгу”, “Лебедине озеро. Зона”, “Молитва за гетьмана Мазепу”. Творчі здобутки митця заслужено поставили його в ряд визначних кінематографістів світу. Осика Леонід кіно

Драч Іван член Киїівського Клубу творчої молоді «Сучасник» (1959—1964). На IX з'їзді письменників України (червень 1986 р.) заявляв, що «ми навчилися говорити досить відверто про правду, як ми жили і живемо». 1989 р. ініціатор Народного руху України за перебудову. 1990 р. голова Народного Руху.

Білодід Іван мовознавець, досліджував мови художньої літератури. З 1957 по 1962 роки працює на посаді міністра освіти УРСР. За його головування завершилось видання одинадцятитомного тлумачного «Словника української мови» .

Курчатов Ігор Під його керівництвом створені: у 1944 році — перший радянський циклотрон; в 1949 і 1953 роках — атомна і термоядерні бомби; у 1954 побудована перша в світі атомна електростанція.

Попович Павло перший український льотчик-космонавт 1962

Тютюнник Григір 1931-1980 молодший брат Григорія, автора роману «Вир». У 1966р. надруковано першу книжку «Зав’язь». Відома новела «Три зозулі з поклоном»

Тютюник Григорій 1920-1961 Творчий доробок митця складає збірка оповідань «Зоряні межі» (1950), повість «Хмарка сонця не заступить» (1957). Вже після смерті письменника світ побачив його збірку поезій воєнного часу «Журавлині ключі» який опубліковано 1963 року. Роман «Вир» посідає особливе місце в історії українського письменства.

Мейтус Юлій 1903-1997 український композитор юдейського походження. вважається фундатором модерної української опери, він є автором 17 різноманітних за жанрами опер (з них 4 у співавторстві): побутові («Украдене щастя»), історико-епічні («Ярослав Мудрий»), романтично-казкові («Донька вітру», «Лейлі і Меджнун»), героїчні («Молода гвардія», «Абакан»). Значне місце в його творчості займає вокальна музика — він є автором близько 300 романсів .

Майборода Георгій  композитор, голова правління Спілки композиторів України (1967-1968).  Опери «Милана», «Тарас Шевченко».

Тетяна Яблонська,

Катерина Білокур,

Марія Приймаченко

Вертов Дзиґа режисер і сценарист перших в Україні звукових фільмів «Симфонія Донбасу» та «Фронт» перший український звуковий кінофільм. Знятий 1930 року метром світового кіноавангарду

Ремесло Василь винайшов всесвітньо відомі сорти пшениці «Миронівська»

Ященко Леопольд композитор диригент

Миколайчук Іван актор, написав 9 сценаріїв та здійснив 2 режисерські роботи «Білий птах з чорною ознакою» і «Вавілон» .

Роговцева Ада

Микола Вінграновський український письменник-шістдесятник, режисер, актор, сценарист та поет, член Киїівського Клубу творчої молоді «Сучасник» (1959—1964)

Євген Гуцало письменник, журналіст, поет і кіносценарист. Ввійшов до плеяди українських письменників-шістдесятників.

Павло Загребельний Письменник, журналіст, автор романів «Первоміст», «Роксолана», «Я, Бог­дан», сценарію до фільму «Ярослав Мудрий »

Тихий Олекса сільський учитель і письменник, автор «Роздумів про українську мову і культуру в Донецькій області», тричі засуджений, востаннє — як співзасновник Української гельсинської групи (1976).

Бердник Олесь (1926-2003) поет, письменник-фантаст, учасник Другої світової, вперше заарештований 1949. Культовий роман "Зоряний Корсар" (вийшов у 26 країнах світу), виключений із спілки письменників. Член Української гельсинської групи (1976), арештований. 1987 р. Вийшов друком роман "Вогнесміх", написаний в ув'язненні.

Дмитро Гнатюк, Євгенія Мірошниченко, Анатолій Солов'яненко

Олександр Білаш, Ігор Шамо опера

Танюк Лесь 1938-2016 - режисер театру та кіно, 1960 р. очолив клуб творчої молоді у Києві, через переслідування з боку радянської влади перехав до Москви. 1989 р. один з фундаторів в Україні історико-просвітницького товариства «Меморіал».

Мороз Валентин - У 1969 р. повернувся з в’язниці та відновив активну правозахисну д1яльність . 1970 р., його було заарештовано вдруге за есе «Мойсей i Датан», «Хроніка опору», «Серед снігів» i звинувачено в антирадянській пропаганді, засуджено на 14 років. Написав працю «Репортаж із заповідника ім. Берії».

Мороз Олександр  У Раді 1990 рокуочолював групу група 239 («За суверенну Радянську Україну») Жовтень 1991 установчий зʼїзд СПУ, її голова.  Голова Верховної Ради України у 1994–1998 та 2006–2007 роках. 1994 Конституційна комісія  разом з президентом Л.Кучмою. Учасник Помаранчевої революції та коаліції «Сила народу» НУ, БОТ, СПУ. Перейшов в Антикризову коаліцію разом з ПР та КПУ (з 2007 р. перейменована в Коаліцію Національної Єдності).  

Тереля Йосип український містик, мученик, сучасний пророк та ясновидець із Західної України, письменник, греко-католицький дисидент, 1982 р. Комітет захисту УГКЦ, в'язень совісті, політімігрант з СРСР, жив у Торонто.

Гель Іван релігійний дисидент 1987 р. Комітет захисту УГКЦ, правозахисник, член Української гельсинської спілки 1988.




Агапіт-чернець Києво-Печерського монастиря, відомий руський лікар

Алімпій православний святий, іконописець, чернець Києво-Печерського монастиря.

Амет-Хан Султан — національний герой кримсько-татарського народу. Радянський льотчик-ас, учасник Другої світової війни. 30 червня 1945 року за успішне проведення 603 бойових вильотів, збиті особисто 30 літаків противника різних типів і 19 літаків ворога, збитих в групових боях, удостоєний звання двічі Героя Радянського Союзу. З 1947 до 1971 року — льотчик-випробувач.

Амосов Микола (1913–2002) Учений у галузі медицини та біокібернетики, громадський діяч, академік Національної академії наук України (1969) та Академії медичних наук України (1993), лауреат Ленінської премії (1961), Державної премії УРСР (1978, 1988) і Державної премії країни в галузі науки і техніки (1997). Директор Інституту серцево-судинної хірургії (1983-1988)

Ангел Євген керівник повстанського руху проти більшовиків у  на Чернігівщині та Сумщині. 1918 р. організував у Конотопі «Курінь смерті» імені Кошового Івана Сірка. Євген Ангел виступав з радикальних позицій націонал-самостійництва, він вважав, що ворогів України треба знищувати. За це Володимир Винниченко критикував його, звинувачуючи в антисемітизмі та русофобії.  Конфлікти з Директорією через свавілля отамана. Нетривалий союз з більшовиками провалився. Увійшов до складу ревкому повстанців (отаман Зелений), спільні дії з Петлюрою та УГА. Продовжив разом з Зеленим боротбу проти Денікіна. Загинув наприкінці 1919 року. 

Андрій Боголюбський володимиро-суздальський князь, у 1169 році зруйнував Київ та вивіз Вишгородську ікону Божої Матері, як символ духовного центру Київської Русі.

Анна-дочка Ярослава Мудрого, королева Франції в ХІ ст. друга дружина французького короля Генріха I Капета, мати Філіпа І. В історії цієї країни вона залишилася як прабабця майже 30 французьких королів. Привезла Реймське Євангеліє на якому до революції 1793 року присягали на вірність французькі королі.

Антонов Олег (1906-1984) Авіаконструктор, автор понад 200 наукових праць з питань планеризму, літакобудування. Транспортний Ан-8 першим у світі зміг перевозити до 11 тонн вантажу на великі відстані. Пасажирський Ан-10 став рекордсменом за кількістю перевезених пасажирів (удостоєний Великої Золотої медалі на Всесвітній виставці у Брюсселі). Перший реактивний літак Ан-72 (1977), велетні Ан-22 («Антей», 1965) і Ан-124 («Руслан», 1982) стали новим етапом розвитку світової авіації.

Антонов-Овсієнко Володимир командувач радянськими військами під час війни більшовиків із УНР Центральної Ради, брав участь у розробленні плану першої війни РСФРР з УНР, командував наступом більшовиків на Україну 

Антонович Володимир історик, ідеолог хлопоманства , згодом лідер та ідейний натхненник Київської громади. Започаткував систематичні археологічні дослідження на території України, під час роботи в Київській археографічній комісії підготував до видання 9 томів «Архіву Південно-Західної Росії» . співавтор видання «Історичних пісень малоросійського народу» з примітками Володимира Антоновича та Михайла Драгоманова.  Ініціатор разом з Олександром Барвінським політики "Нової ери" та переїзду Михайла Грушевського до Львова, позаяк саме він був науковим наставником останнього. Автор терміну "Україна-Русь" 

Апостол Данило  1727-1734 Після скасування І Малоросійської колегії гетьманом стає Д. Апостол. Діяльність гетьмана визначалась «Рішительними пунктами» 1728р. з метою врегулювання земельних справ протягом 1729-1731рр. провів генеральне слідство про маєтності; зумів домогтися скасування багатьох обмежень та утисків української торгівлі; уперше встановив бюджет державних видатків. За сприяння гетьмана у 1728р. Кодифікаційна комісія розпочала створювати звід законів «Права за якими судиться малоросійський народ» (1743р.)

Артем (Ф.Сергєєв) очолив Народний секретаріат у грудні 1917 р. – уряд новоствореної радянської України у Харкові

Аскольд напівлегендарний київський князь, варязького походження («Повість минулих літ» Нестора), нашадок Києя (атохтонна теорія). здійснював походин на Візантію, прийняв хрещення. Вбитий Олегом у 882 році.

Атей скіфський цар 4 ст. до н.е., об‘єднав племена у державу, друкував монети.

Аттіла цар гунів за якого гунський союз племен досяг найбільшої могутності серед. 5 ст. Після поразки в Каталаунській битві (451 p.) та смерті Аттіли (453 р.) гунський союз племен розпався.

Багалій Дмитро автор «Історії Слобідської України» входив до Київської громади; від 1881 року брав участь у роботі Історичного товариства Нестора літописця.   

Бажан Микола (1904–1983) Український письменник, філософ, громадський діяч, перекладач, поет. За ініціативою і під керівництвом М. Бажана видана Українська Радянська Енциклопедія. Автор великої історичної поеми «Сліпці». ктивно сприяв реабілітації письменників під час відлиги.

Базилович Іоанникій «Короткий нарис фундації Федора Коріятовича» чернець Мукачівського монастиря, написав першу наукову працю з історії Закарпаття.

Бандера Степан (1909-1959) 1934 року заарештований, засуджений до смертної кари, згодом заміненої на довічне ув'язнення. Звільнений 1939 року. 1941 року був заарештований німецькою окупаційною владою, після відмови скасувати Акт відновлення Української держави ув'язнений у концентраційному таборі Саксенгаузен. Після звільнення жив в Інсбруку, Зеєнфельді, Мюнхені. вбитий 15 жовтня 1959 року в Мюнхені агентом КДБ Б. Сташинським. 

Бантиш-Каменський Дмитро 1822, Москва 4 томи «История Молой России» Основа праці батька– архівіста. Перша наукова праця з історії України від найдавніших часів до 18 ст. Переконливо засвідчив багатовікову тяглість української історії, зробив наголос на періоді, коли Україна була автономною.

Барінський Олександр (1847–1926) Один з лідерів народовського руху. За сприяння В. Барвінського 1880 р. у Львові відбулося перше українське народне віче. У 1886 р. започаткував «Руську історичну бібліотеку» – перше серійне видання фахової істо ричної літератури українською мовою. Саме О. Барвінський розпочав пропагувати термін «Русь-Україна» в історичній літературі. Депутат галицького сейму й австрійського парламенту. Разом з В.Антоновичен ініціатар політики "нової ери" 1890-1894 рр. 

Барвінський Володимир (1850–1883) Один з організаторів й ідеологів народовського руху, найвизначнішою справою життя якого було заснування та видання найавторитетнішої галицько-української газети «Діло»; рідний брат Олександра Барвінського. 

Батий (1227-1255) монгольський хан, онук Чингісхана, засновник Золотої Орди.

Бачинський Андрій мукачівський єпископ, ініціатор навчання буковинських студентів у створеній греко-католицькій семінарії у Львові.

Бачинський Юліан у 1895 р. вийшов друком публіцистичний твір під назвою «Україна ірредента» («Україна уярмлена»), у якому він з позиції соціалістичних ідей обґрунтовував історичну правомірність боротьби українців за окрему самостійну державу. Член РУРП та один з керівників УСДП, член ЗУР у травні 1915 р. 1933 р. повернувся в УСРР, репресований.

Берест Олексій (1921–1970) Герой України, лейтенант Червоної Армії, що встановив, разом з М. Єгоровим та М. Кантарією, Прапор Перемоги на даху німецького Рейхстагу о 21.50 30 квітня 1945.

Беретті Вікентій (1781–1842) архітектор класицизму, будівля Київського університету.

Беринда  Памво (працював у Львівській друкарні та школі, з 1619 у Лавру, Найвизначніша праця— друкований український словник «Лексикон словенський» (1627).

Биков Леонід (1928-1979) Український актор, режисер і сценарист, заслужений артист РРФСР (1965), народний артист УРСР (1974). У числі найліпших ролей Леоніда Бикова в кіно можна назвати Богатирьова («Дорога моя людина»), Акішина («Добровольці»), Альошки («Альошкова любов»), Гаркуші («На семи вітрах»). Биков прославився як режисер картин «В бій ідуть лише старі» та «Ати-бати, йшли солдати».

Боберський Іван (1873—1947) Громадський діяч, педагог, один із засновників молодіжних спортивно-пожежних патріотичних товариств «Сокіл», «Січ», «Пласт» член Бойової управи УСС, член Головної української ради, автор підручників із фізичного виховання.

Боголюбов Микола (1909—1992) — математик і механік, фізик-теоретик, засновник наукових шкіл з нелінійної механіки і теоретичної фізики, член фізико-математичного відділу ВУАН, автор квантової теорія поля.

Богомолець Олександр (1881-1946) Український патофізіолог. Заснував у Києві Інститут експериментальної біології і патології та Інститут клінічної фізіології. Президент АН України з 1930 р. Під його керівництвом створено багатотомник «Основи патологічної фізіології». З ініціативи Олександра Богомольця 1941 р. відкрито перший у світі диспансер для боротьби з передчасним старінням, на базі якого 1958 р. створено Інститут геронтології. Президент АН України 1930-1946.

Боженко Василь 1917–1921 років, учасник боротьби за встановлення радянської влади  на Єлисаветградщині. Боженко обирається до Тимчасового виконавчого комітету по організації Київської ради робітничих депутатів, тоді ж вступає у більшовицьку партію. Січневе повстання проти УЦР. В період Гетьманату вийшов в «нейтральну зону».1918 командир Таращанського полку, воював проти Директорії. За взяття Києва 5 лютого 1919 р. нагороджений червоним прапором. Помер у 1919 р.

Білозерський Василь член Кирило-Мефодіївського товариства, найпалкіший пропагандист доцільності видання в Петербурзі українського часопису «Основа». 

Бож-князь антського військового союзу у IV ст., загинув у боротьбі з готами.

Божко Юхим український повстанець-отаман (Катеринославщина), командир збройного формування «Запорозька Січ», яке виникло під час антинімецького та антигетьманського повстання; пізніше — офіцер армії УНР, командир 2-ї піхотної дивізії. Мав свою принципову позицію, був прихильником П.Болбочана. Мав конфлікти з С.Петлюрою. В кінці листопада 1919 збирався перейти на бік більшовиків, але, за ігнорування наказів Волоха, на початку грудня 1919 був вбитий.

Бойчук Михайло (1882—1937) Живописець, засновник школи художників-монументалістів, поєднував риси візантійського живопису з традиціями українського малярства, автор декорацій для вистав «Молодого театру» Л.Курбаса.

Болбочан Петро 1917 р. командир Республіканського полку, 1918 р. командир найбільш боєздатного і дисциплінованого з’єднання української армії – Запорізького корпусу. «запорожці» відіграли ключову роль у здобутті Києва в лютому 1918-го. та здійснили Кримський похід в березні–квітні. Розстріляний за наказом С.Петлюри. Посол України у Відні В’ячеслав Липинський на знак протесту подав у відставку, при цьому звинувативши уряд у некомпетентності. В. Антонов-Овсієнко: «допоки був живий Болбочан, він Україну не віддав би». 

Боплан Гійом де  французький інженер і військовий картограф, з початку 1630-х до 1648 р. перебував на польській службі, приймав участь у будівництві Кодака, склав першу карту України.

Борецький Іов  (1560-1631) ректор Львівської братської школи, засновник Київського братства та братської школи, київський митрополит після відновлення православної ієрархії у 1620 р.

Борис, Гліб-перші святі на Русі. Сини Володимира Великого, підступно вбиті братом Святополком

Боровець Тарас (псевдоніми: Тарас Бульба, Чуб, Ґонта) (1908-1981) Український військовий і політичний діяч. Народився на Волині. У 1930 р. заснував організацію «Українське національне відродження». Кілька разів Т. Боровця заарештовували польські служби. У 1939—1941 pp. брав участь в організації націоналістичного підпілля на Поліссі та Волині. На початку радянсько-німецької війни організував та очолив партизанські підрозділи «Поліської Січі». За деякими даними в листопаді 1943 р. був заарештований гестапо у Варшаві й ув'язнений до концтабору Саксенгаузен. Звільнений у вересні 1944 р. З 1948 р. жив в еміграції в Канаді. Видавав журнал «Меч і воля». Помер у Торонто. Автор спогадів «Армія без держави».

Боровиковський   Володимир в далекому Петербурзі  ніколи не забував про Батьківщину, залишався типовим полтавчанином, дотримувався українських звичаїв та мріяв повернутися додому.

Бортнянський Дмитро  Композитор. Закінчив першу (1729р.) в Україні музичну школу в Глухові. Автор чотирьох десятків хорових концертів «Алкід», «Квінт Фабій», «Сокіл», «Син-суперник»

Брюховецький  Іван 1663-1668 гетьман Лівобережної України, обраний на Чорній раді, уклав Московські статті. Перший здійснив поїздку до Москви, отримав титул боярина.

Бут (Павлюк) Пало  - керівник повстання 1637 р. Битва під Кумейками (біля Черкас) і Боровицею (біля Чигирина) поразка козаків. Переговори, посередник Адам Кисіль. Павлюк страчений.

Василенко Микола історик, член ТУП. За Гетьманату голова Ради міністрів, міністр освіти, голова Державного Сенату. Взяв участь у відкритті українських університетів у Києві та Кам'янці-Подільському. Від 1920 року — співробітник Української академії наук (УАН). 

Васильківський Сергій пейзажний жанр, «небесний» художник, «Сторожа запорізьких вольностей». У його картинах переважали українські мотиви: «Ранок (Отара в степу)», «Дніпрові плавні», «Ранок на Дніпрі». Пейзаж у творчості С. Васильківського органічно поєднувався з історичними сюжетами з доби козацтва («Козачий пікет», «Запорожець у розвідці», «Козаки в степу», «Козача левада») або з побутовими картинами («Козак в степу», «Ярмарок у Полтаві»). Сучасники називали С. Васильківського «небес ним» художником через особливе ставлення митця до неба. Уславився С. Васильківський і в історичному жанрі. Найвизначнішим його твором на історичну тему вважають картину «Сторожа запорозьких вольностей» («Козаки в степу»).  

Ватутін Микола командував військами Воронезького, Південно-Західного та І Українського фронтів. Брав участь у Сталінградській битві та битві на Курській дузі. Його війська визволяли Бєлгород, Харків, Київ, форсували Дніпро. Загинув під час бойових дій. Один із керівників визволення Києва 6 листопада 1943 р. Помер у квітні 1944 р. від поранення, яке завдали українські воїни УПА.

Вахнянин Анатоль перший голова «Просвіти» 1868 р., педагог і композитор автор першої західноукраїнської опери «Купало». Він створив багато хорових і сольних вокальних творів, аранжував народні пісні, ініціював створення Вищого музичного інституту імені М. Ли сенка у Львові, був його першим директором.  

Вербицький Михайло  (1815–1870), автор музики Державного гімну «Ще не вмерла України...» Йому належать понад сорок духовних композицій. М. Вербицький створив музику до більш ніж двадцяти театральних п’єс, є автором дванадцятьох симфонічних увертюр-симфоній, музика яких виразно утверджувала національний стиль. 

Вериківський

Верьовка Григорій (1895–1964) Український композитор і хоровий диригент, педагог. Найвідоміші твори: «Ой, як стало зелено», «Ой чого ти земле, молодіти стала», «І шумить, і гуде», «Клятва», «Дівчата з Донбасу», «Пісня про Волго-Дон», «Шахтарочка» тощо. Серед творів великої форми кантата «Ми ковалі своєї долі» на слова поета Павла Тичини.

Виговський  Іван гетьман України 1658-1659. Уклав Гадяцький договір з Річчю Посполитою - українські землі Руське князівство. 1659 Конотопська битва.

Винниченко Володимир один з найпопулярніших письменників і драматургів. Першим в українській літературі створив галерею образів революціонерів, які діяли у складних психологічних ситуаціях. 

Вишиванний Василь (Вільгельм  Габсбург-Лотрінґен ) племінник останього австрійського імператора Карла І. Після Бреського миру прийшов в Україну, як командир легіону УСС. Його вважали одним з неофіційних претендентів на український трон у разі утворення монархічного ладу.   

Вишенський  Іван письменник-полеміст, основоположник жанру сатири в українській літературі.

Вишневецький Дмитро  (Байда) – черкаський і канівський староста, засновник першої Запорізької (Хортицької) Січі. Походи на Кри (Аслам-Керман, Очаків) і турецькі фортеці. Після зруйнування Хортицької січі, перенесена на о. Тамаківка. Перейшов на службу до Московського царя: походи на Крим та в Молдаві. «Пісня про Байду».

Вишня Остап (П.Губенко) 

Вінітар-готський король, онук Германаріха. У IV ст. воював з антами. Захопив у полон антського князя Божа та наказав стратити його разом з іншими антськими старійшинами

Вірський Павло (1905–1975) Український танцівник і хореограф, Народний артист СРСР (1960), художній керівник (1955–1975) Ансамблю танцю УРСР.

Вітовський Дмитро (1887–1917) український політик і військовик, полковник Легіону Українських Січових Стрільців, полковник УГА, Державний секретар військових справ ЗУНР. 

Вітовт 1392—1430 великий князь литовський, виступив проти Кревської унії, але пізніше уклав Городельську. Брав участь в Грюнвальдській битві. Ліквідував автономію руських князівств.

Вовчок Марко (Марія Олександрівна Вілінська) письменниця, авторка першої української соціальної повісті «Інститутка», у якій створено реалістичні образи кріпаків

Возняк Михайло 1881-1954 літературознавець і фольклорист, академік АН УРСР за спеціальністю літературознавство. Особливо вагомим є внесок Возняка у вивчення біографії і творчості Івана Франка.   

Волобуєв Михайло (1903–1972) Український економіст 1930-тих років. Науковий працівник науково-дослідного інституту ВУАН у Харкові. Автор концепції економічної самодостатності УССР. на початку 1928 р в журналі «Більшовик України» виступив що Україна колонія Росії, самостійна економіка, частина світової. Жертва сталінського терору.

Володимир Великий здійснив релігійну реформу (980 – культ Перуна, 988 – хрещення Русі). Побудував захисні «змієві вали», провів судову реформа «Устав земляний» (усні закони), друкував монету з зображенням тризуба, побудував Десятинну церкву. Приєднав Червенські землі (закінчення формування території Русі).

Володимир Мономах ініціатор Любецького з´їзду 1097 р. мета: вотчини, припинити усобиці, спільна боротьба проти половців, походи проти половців, «Статут (Устав) Володимира Мономаха», «Повчання дітям».

Володимир Ольгердович-перший Київський князь із династіїГедиміновичів (1362-1394 рр.); син Великого князя Литовського Ольгерда. Охрещений за православною традицією, Володимир Ольгердович не був сприйнятий в Києві як чужинець і швидко порозумівся з місцевим боярством

Володимирко засновник Галицького князівства у сер. ХІІ ст., галицький князь, батько Ярослава Осмомисла.

Вовк Федір український антрополог, етнограф, археолог, археограф, музеєзнавець, видавець та літературознавець. Автор першої загальної науково обґрунтованої концепції антропологічного складу українців.

Волох Омелян в 1919 році — член партії боротьбистів. Гайдамацький Кіш Слобідської України створений Симоном Петлюрою, який у червні 1918 року полк перетворено в Гайдамацьку бригаду якою командував отаман Омелян Волох. Взяв участь в арешті полковника Петра Болбочана (січень 1919 р.), у єврейському погромі в Проскурові (лютий 1919 р.), та ін. Грудень 1919 р. — очолюючи Гайдамацьку бригаду, захопив скарбницю самоліквідованої УНР і в січні 1920 перейшов на бік червоних військ. 1920 — вступив до Комуністичної партії. 1933 року був заарештований по справі УВО. 

Волошин (Августин 1874-1945) 1920 року під безпосереднім впливом А. Волошина постає товариство «Просвіта», У 1938 році очолив уряд Підкарпатської Русі. Після угорської окупації Закарпаття змушений був емігрувати. Упродовж 1939—1945 pp. живу Празі, був ректором Українського вільного університету. У травні 1945 року радянські спецслужби заарештували А. Волошина й перевезли до Москви, де він і помер у тюрмі.

Гамалія Микола  мікробіолог і епідеміолог, з І. Мечниковим створили першу вітчизняну бактеріологічну станцію, що успішно застосовувала запобіжні вакцини й сироватки для лікування інфекційних хвороб.

Ганькевич Микола перший голова і засновник УСДП,  Заступник голови Головної Української Ради (ГУР) у 1914—1915 рр. і віце-президент Загальної української ради у Відні 1915—1918 рр. 1918 р. — член Української національної ради. 

Галущинський Михайло (1878-1931) Перший командант легіону Українських січових стрільців, згодом - військовий секретар УСС.  У 1923—1931 рр. - очільник товариства «Просвіта».

Галшка Гулевичівна українська шляхтичка, подарувала землі для Київського братства, на якій зараз знаходиться Києво-Могилянська академія.

Гедимін-литовський князь, який перший розпочав приєднання руських земель до Литовського князівства у ХIV ст.

Германаріх-готський король, який у IV ст. утворив державу у Північному Причорномор’ї. Вів боротьбу з антами, племінний союз яких очолював князь Бож

Геродот-давньогрецький історик V ст. до н.е., який детально описав життя скіфів

Глушков Віктор (1923-1982) український вчений, піонер комп'ютерної техніки, цифрові машини «Київ», «Дніпро», «Промінь», «Мир», автор фундаментальних праць у галузі кібернетики, математики і обчислювальної техніки, ініціатор і організатор реалізації науково-дослідних програм створення проблемно-орієнтованих програмно-технічних комплексів для інформатизації, комп'ютеризації і автоматизації господарської і оборонної діяльності країни. Глава наукової школи кібернетики. Дійсний член АН СРСР, АН УРСР. Почесний член багатьох іноземних академій. ЗасновникІнституту кібернетики  НАН України носить його імʼя, який і очолював до 1982 р. Помер в Москві куди був перевезений для лікування. 

Гоголь Микола Збирав український фольклор, використав при написанні своїх творів. Популяризація українських традицій та звичаїв.

Гомер-легендарний давньогрецький поет, створив «Іліаду» та «Одіссею», у якій уперше згадуються кіммерійці

Гонта Іван  - один з керівників гайдамаків під час Коліївщини 1768 р. 

Гончар Олесь (1918–1995) Український прозаїк, публіцист, громадський і політичний діяч, учасник Великої Вітчизняної війни, голова правління Спілки письменників України, автор трилогії «Прапороносці», романів «Тронка», «Собор» (1968р.) та ін. 

Горська Алла (1829–1970) художниця «шестидесятниця», була одним із організаторів Клубу творчої молоді «Сучасник» (1959—1964). 1968 Горська підписала «Лист-протест 139» на ім'я керівників КПРС і радянської держави з вимогою припинити незаконні процеси. 1870 р. її вбито за невідомих обставин.

Граве Дмитро творець першої великої математичної школи в Україні, член фізико-математичного відділу ВУАН 

Гребінки Євген (1812-1848) учень Ніжинської гімназії. Активний учасник Петербурзької громади, приятелював із Шевченком, брав участь у викупі його з кріпацтва. Українські байки: «Ведмежий суд», «Віл», «Рибалка», «Вовк і Огонь».

Греков Олександр  генерал-хорунжий Армії УНР. Змінив М.Омеляновича-Павленка на чолі УГА. Командував під час Чортківської офензиви.

Грінченко Борис письменник, один із фундаторів «Братства тарасівців»; автор фундаментальних наукових праць, перших підручників з української мови й літератури, зокрема «Рідного слова» - книжки для читання в школі. Автор 3 томного «Словника української мови».

Григор’єв Матвій  (1884–1919) Один з лідерів Української Національної Революції, видатний український військовий та громадсько-політичний діяч доби Громадянської війни 1917—1921 рр . Керівник антибільшовицького повстання на Півдні України та командуючий об’єднаною Українською повстанською армією (1919), самопроголошений Гетьман України (1919). Деякий час співпрацював з більшовиками. Убитий Н.Махно.

Григоренко Петро пройшов війну, мав звання генерал-майора. За виготовлення листівок про розстріли робітників у Новочеркаську, Теміртау, Тбілісі, опинився у психушці. Було звільнено, але позбавлено пенсії, звання понижено до рядового, роботи. Протестує проти вторгнення до Чехословаччини, виступає на захист кримських татар. За це довелося заплатити п’ятьма роками психушки. Приїзд Ніксона допомогли визволити. Співучасник створеня УГС позбавлення громадянства СРСР, виїхав до США.

Григорович-Барський Іван - український архітектор доби бароко 18 ст.

Грушевський Михайло 1894 р. «Кафедра всесвітньої історії з особливим наголосом на історії Східної України», за рекомендацією В.Антоновича, очолив у 28-років. 1897 р. очолив НТШ, 1898 р. у Львові вийшов перший том десятитомної «Історії України-Русі». Входив до УНДП ( Є. Левицький, І.Франко). Голова УЦР. з 1923 академік ВУАН, професор Київського університету. 1931 р. переїхав до Москви, проходив по справі «керівника Українського націоналістичного центру». Автор понад 2000 наукових праць. 

Гулак-Артемовський Петро (1790–1865) байки, «Пан та собака» 1818 р, етико-філософське послання «Справжня добрість». Ректор Харківського університету 1841-1848 р.

Гулак-Артемовський поставив першу оперу «Запорожець за Дунаєм»  1862 р.

Гулак Микола адвокат, член радикального крила Кирило-Мефодіївського товариства, автор рукопису під назвою «3акон божий» складається зі 109 параграфів .

Данило Галицький (1219-1264) відновив єдність Галицько-Волинського князівства, переміг тевтонців у битві під Дорогочином у 1238 р, та поляків і угорців під Ярославом у 1245 р. Боровся з монголо-татарами: брав участь у битві на Калці 1223р.; у 1253 р. коронований Папою у Дорогочині. Переніс столицю в Холм, заснував Львів.

Данькевич Костянтин український радянськийкомпозитор, піаніст, 1956–1967 роках — голова правління Спілки композиторів України. Опера «Богдан Хмельницький» (лібрето B. Василевської та О. Корнійчука, 1950) занала критика під час Жданівщини. 1978 року присуджено Державну премію УРСР імені Тараса Шевченка за оперу «Богдан Хмельницький»  

Дерев’янко Кузьма  (1906–1945) Український радянський військовий діяч, Герой України, генерал-лейтенант. 2 вересня 1945 р. від імені радянського Верховного Головнокомандування приймав капітуляцію Японії. За кілька днів після атомного бомбардування міст Хіросіма та Нагасакі генерал Дерев’янко відвідав радіоактивну місцевість та склав детальний звіт з фотографіями.

Дзюба Іван (1829–1970) Український літературознавець, критик, громадський діяч, дисидент радянських часів, Герой України, академік НАНУ, співзасновникНародного Руху України, головний редактор журналу «Сучасність» (1990-ті рр.), член редколегій наукових часописів «Київськастаровина», «Слово і час». Автор праці «Інтернаціоналізм чи русифікація?». Творчий наробок близько 400 наукових праць.

Добрянський Адольф (1817–1901)  «будитель» на Закарпатті, активний прихильник москвофільства, член Головної Руської ради 1848 р. від Закарпаття.

Довбуш Олекса  - керівник руху опришків 1738-1745 рр. 

Довженко Олександр (1894–1956) Письменник, кінорежисер, художник-ілюстратор, один із засновників української кінематографії, працювавна Одеській кіностудії, Київській кінофабриці, створив фільми «Арсенал», «Звенигора», «Земля»,«Поема про море», «Україна в огні», автор книги «Зачарована Десна». 

Довженко Олександр (1894-1956) Фундатор української національної школи кінематографії, геніальний кінорежисер, чия картина «Земля» увійшла до 12 найкращих стрічок «усіх часів і народів», назвали Гомером XX століття. Добровільно вступив до Армії УНР. За своє творче життя поставив 14 ігрових і документальних фільмів, написав 15 літературних сценаріїв і кіноповістей, дві п'єси, автобіографічну повість, понад 20 оповідань і новел, ряд публіцистичних статей і теоретичних праць, присвячених питанням кіномистецтва.

Долинський Лука  вихованець Віденської академії мистецтв, львівський митець (родом з Білої Церкви) вдало поєднував релігійний і світський живопис, зокрема портрет. Портрет князя Лева Даниловича. Кінець 18 ст. (портрет з уяви) 

Донцов Дмитро (1883 — 1973) Український літературний критик, публіцист, філософ, політичний діяч, засновник теорії інтегрального націоналізму. Автор роботи «Націоналізм». Перший голова СВУ.

Дорошенко Дмитро історик, засновник «Просвіти» на Катеринославщині. Член Союзу земст та міст Південно-Західного фронту, член Центральної Ради, міністр закордрнних справ Української Держави. 

Дорошенко Михайло  - гетьман реєстрових козаків, брав участь у повстанні Марка Жмайла 1625 р. Уклав Куруківську угоду. Воювавз з татарами та турками під час одного з боїв поблизу Бахчисарая загинув. Згадується у пісні “Ой на горі, тай женці жнуть…”.

Дорошенко Петро  «сонце руїни», 1665 р гетьман Правобережної України. 1668 – став гетьманом обох берегів України. 1672 Бучацький мирний договір (тур-пол) «Україна держава», 1676 передав булаву Самойловичу.

Духнович Олександр «будитель» на Закарпатті, активний прихильник москвофільства. літературні твори, праці з педагогіки, філософії, історії. Мрія - створення фундаментальної праці з історії закарпатських русинів. Статті «Становище русинів в Угорщині», ідея об‘єднання, 1849 р. у «Зорі галицькій». У 1850–1852 рр. 3 літературні альманахи «Поздоровлення русинів».

Я Русин был, есмь и буду,

Я родился Русином,

Честный мой род не забуду,

Останусь его сыном».

Єфименко Олександра  авторка популярної «Історії українського народу»; перша жінка в Росії та Україні, якій 1910 р. Харківський університет надав ступінь почесного доктора наук

Єфремов Сергій (1876—1939) Громадсько-політичний і державний діяч, літературний критик, історик літератури, член Наукового товариства ім. Т. Шевченка у Львові, один із засновників УРП і Товариства українських поступовців, член Української Центральної Ради, секретар міжнаціональних справ Генерального Секретаріату УЦР, віце-президент ВУАН (1922—1928 рр.), засуджений як один із керівників СВУ.

Желябов Андрій діяч революційного народництва в Росії та Україні; член «Землі і волі», після її розколу один з керівників «Народної волі», співорганізатор замахів на Олександра II 

Жмайло  Марко керівник козацько-селянського повстання 1625 р. Битва біля м.Кирилова біля Курукового озера. Війська запорізького та польських військ (С.Конецпольський). Розкол серед козаків, переговори (Михайло Дорошенко). Підписана Куруківська уода.

Жук Андрій (1880-1968) член УСДРП, заарештований а потім  виїхав за кордон. У серпні 1914 переїхав до Відня. 1914–1918 — один із засновників і чільних членів Союзу Визволення України, Головної Української Ради (1914—1915), Загальної Української Ради (1915—1916, від Наддніпрянської України), Боєвої управи УСС (1915—1920). Перебував на дипломатичній службі в посольстві Української Держави, згодом УНР у Відні та радник МЗС УНР. Виступав проти Варшавського договору.  

Заболотний Данило (1866—1929) український мікробіолог, епідеміолог, Президент ВУАН (1928–1929), засновник Інституту мікробіології та епідеміології в Києві. Опублікував понад 200 праць, присвячених головним чином вивченню трьох інфекційних хвороб — чуми, холери (у співпраці із Савенком) й сифілісу. Першим у світовій науці дослідив шляхи поширення чуми і запропонував ефек­тивні засоби боротьби проти цієї хвороби

Залозецький Володимир (1884–1965) Український (буковинський) політик, громадський діяч, дипломат, мистецтвознавець, меценат. став одним із засновників Української Національної Партії Буковини, був її президентом. Обирався сенатором парламенту Румунії, згодом був у ньому єдиним українським депутатом. Брав активну участь у діяльності Міжнародного Конгресу Національних Меншин, був обраний його президентом. У цій ролі у липні 1937 року виступив у Палаті громад британського парламенту, зокрема, підняв питання Голодомору, влаштованого більшовиками в УСРР.

Заньковецька Марія актриса, одна з основоположників українського національного театру, у 1908 р. урочисто відзначила 25-річчя сценічної діяльності 

Затонський Володимир (1888-1938) очолив делегацію радянської УНР для переговорів у Бересті-Литовському. під час Радянсько-польської війни. Галицький ревком,  Народний комісар освіти УСРР  до і після українізації. Поплічник Павла Постишева у знесенні пам'яток церковної архітектури в Україні. 1934 р. був головою лічильної комісії під час виборів Сталіна. Репресований.

Зелений отаман (Терпило)Данило   1886-1919 1918 р. у Трипіллі Дніпровську дивізію (2,5 тис. осіб). Разом із загонами С.Петлюри та січових стрільців Є.Коновальця у грудні 1918 р. оволодіває Києвом. Став главою “Придніпровської республіки” в січні 1919 р. переходить на бік більшовиків. Але в квітні 1919 висунуте гасло: “Ради без комуністів”. Остаточно загони отамана були знищені в серпні 1919 р., отамана було смертельно поранено. 

Зизаній  Лаврентій – український вчений, автор «Граматики словенської» і «Лексиси» (перший словник, теорія віршування).

Іван Калита 1325—1340 московський князь почав збір Руських земель.

Івасюк Володимир (1949-1979) Український композитор і поет. Герой України (2009, посмертно). Один із основоположників української естрадної музики (поп-музики). Автор 107 пісень, 53 інструментальних творів, музики до кількох спектаклів. Професійний медик, скрипаль, чудово грав на фортепіано, віолончелі, гітарі, майстерно виконував свої пісні. Неординарний живописець. Автор пісень «Червона рута» й «Водограй».

Ігор (912-945) перший з династії Рюриковичів, програв похід проти Візантії у 941 р, через використання «грецького вогню». 943 році другий вдалий похід. Вперше на кордонах Русі з‘являються печеніги. Загинув під час збору данини з древлян.

Ігор Святославович 1178 - 1202 новгород-сіверський князь згадується у «Слові о полку Ігоревім» його похід на половців 1185 р.

Іжакевич Іван 

Ілларіон перший митрополит слов'янського походження, письменник, автор «Слова про закон і благодать» - першої писемної пам‘ятки присвяченої діянням Володимира Великого та Ярослава Мудрого.

Кавалерідзе Іван Український скульптор, кінорежисер, драматург, сценарист, художник кіно. В Ромнах пам’ятник Т. Шевченку. Як кінорежисер поставив фільми «Злива» (1929), «Перекоп» (1930), «Коліївщина» (1933), «Прометей» (1936),Наталка Полтавка (1936), «Запорожець за Дунаєм» (1937), «Григорій Сковорода» (1958), «Повія» (1961; за твором Панаса Мирного). 

Каганович Лазар (1893–1991) Радянський партійний і державнийдіяч, один із найближчих прибічників Й. Сталіна, генеральний секретар ЦК КП(б)У 1925—1928 рр., противник українізації, виступав проти політичної лінії українських націонал-комуністів О. Шумського і М. Хвильового, прихильник проведення суцільної колективізації. Перший секретар ЦК КП(Ь)У у 1947 році.

Казимір ІІІ польський король (1333–1370), після смерті Юрія ІІ Болеслава перебрав титул «Король Русі», почав захоплення Галицько-Волинського князівства.

Калнишевський  Петро - останній кошовий отаман війська Запорізького низового. Вірою і правдою служив царату. Після зруйнування Січі його відправлено на соловки.

Камерон Чарлз шотланський архітектор за його проектом споруджено  палац Кирила Розумовського в Батурині 1799-1803 

Каразін Василь винахідник, агроном і метеоролог, створив першу метеорологічну станцію, ініціатор відкриття Харківського університету 1805 р.

Квітка-Основ‘яненко Григорій основоположник художньої прози («Маруся»), п‘єси «Сватання на Гончарівці», «Шельменко-денщик», бурлескно-реалістична повість Маркевич «Конотопська відьма». Співзасновник гуртка «харківських романтиків».

Квітка-Основ’яненко Григорій (1778–1843) засновник сучасної української прози. 1812 р. директор харківського театру. Участь у видавництві «Украинсого весника». 1833 р. «Супліка до пана іздателя» - «маніфест» на захист укр. літератури. «Пан Халявський», «Українські дипломати», «Маруся», п’єси «Сватання на Гончарівці»,  «Шельменко – волосний писар». повість «Сердешна Оксана».

Кериченко Олексій – перший секретар КП(б)У України 1953-1957. Ставленик Берії. Після відставки курує КДБ. Олексій Кириченко є замовником вбивства Степана Бандери.

Кирпонос Михайло радянський військовий діяч часів Великої Вітчизняної війни. Генерал-полковник. Герой Радянського Союзу. Керівник Південно-Західного фронту СРСР. Загинув у вересні 1941 р. під час оборони Києва німецько-фашистських загарбників.

Кішка Самійло - 1600-1602 козацький гетьман очолював походи в Молдавію, Семиграддя, Ліфляндію. Змальований у думі "Самійло Кішка", де він врятував невільників з турецької галери.

Кобилянська Ольга  в новелах «Природа», «Меланхолійний вальс», «Некультурна» тонко відтворила злам у суспільній свідомості, пов’язаний зі зміною становища жінки, її новими соціальними ролями. Героїні О. Кобилянської – «аристократки духу», які прагнуть досягти ідеалу надлюдини, не звертаючи уваги на суспільні стереотипи. 

Ковпак Сидір Один з організаторів партизанського руху в Україні. Учасник Першої та Другої світових війн. У роки громадянської війни очолив у с. Котельва партизанський загін, який боровся проти австро-німецьких окупантів та денікінців, воював на Східному фронті. У роки війни з фашистськими загарбниками партизанське з'єднання під командуванням С. Ковпака провело 5 рейдів по тилах ворога. Керівник Карпатського рейду, що розпочався 12 червня 1943 р. Автор книги спогадів «Від Путивля до Карпат». 

Кожедуб Іван радянський військовий діяч, льотчик-ас часів Другої світової війни, найбільш результативний льотчик-винищувач в авіації союзників. Тричі Герой Радянського Союзу. Маршал авіації.

Конашевич-Сагайдачний Петро  - гетьман України 1616-1622, очолив походи на Туреччину. Здобули фортеці на Чорному морі: Перекоп, Ізмаїл, Кілію, Акерман,Трапезунд, Синоп, Кафу (1616), Стамбул. Впорядкував козацьке військо, підтримував православну церкву та культуру (1615 р. вступив разом з усіма козаками до Київськоо братства, за його сприяння 1620 р. відновлена православна ієрархія). 1621 р. під час Хотинської битви, зупинив просування турків у Європу, підням міжнародний авторитет козацтва.

Кондратюк Юрій (1897–1941) Український, радянський вчений-винахідник, один із піонерів ракетної техніки й теорії космічних польотів. Автор так званої «траси Кондратюка», якою подорожували на Місяць космічні кораблі «Аполлон». Також він вперше обґрунтував економічну доцільність вертикального злету ракет, створення проміжних баз під час польотів, гальмування у верхніх шарах атмосфери, використання сонячної енергії космічними апаратами тощо.

Кониський Олександр  діяч Полтавської та Київської громад, організатор недільних шкіл у 1863 р. було заслано до Вологди, потім – до Тотьми (нині місто у Вологодській обл. Ро сійської Федерації). Він є автором гімну «Боже великий, єдиний, нам Україну храни», першим активним самостійником, пропагандистом ідей української культурної окремішності , був одним із фундаторів Літературного Товариства ім. Т. Шевченка у Львові (1873). Разом з  В. Антоновичем ініціатор створення у 1897 р.  Загальної української безпартійної організації (ЗУБО). «Можна любить Україну так, як її любив Кониський, – але більше, як він, – любить не можна» (О. Лотоцький).

Коновалець Євген керівник Галицько-Буковинського куреня Українських січових стрільців, одного з найдієздатніших частин Армії УНР полковник Армії УНР, З дозволу гетьмана формує Окремий загін січових стрільців. 1918 року підтримали Директорію УНР у повстанні проти влади П. Скоропадського 1920 р. голова УВО, 1929 року голова Проводу Організації Українських Націоналістів. 1938 року Є. Коновалець загинув у Роттердамі (Нідерланди), відкриваючи поштовий пакет, у якому був вибуховий пристрій, переданий йому агентом радянських спецслужб. 

Коріатовичі -династія литовсько-руських (українських) князів (одна з ліній Гедиміновичів), що правили на Поділлі та Закарпатті. Першим представником династії Коріятовичів був Коріят-Михайло Гедимінович, що князював у Новогрудку та був братом великого литовського князя Ольгерда.

Коссак Григорій (Гриць) (1882–1939) —3 серпня 1914 р. його призначено отаманом другого куреня Легіону Українських січових стрільців. У 1917–1918 рр. — заступник командира, а згодом командир вишколу УСС. Після підписання Договору між командуванням УГА та командуванням Добровольчої армії в 1919 р. з группою старшин і стрільців покинув армію та переїхав у Закарпатську Україну. Згодом жив в еміграції в Австрії та на Закарпатті. У 1924 р. повернувся до УСРР. 1939 р. був розстріляний.

Косинка Григорій 

Косинський Криштоф  - керівник козацько-селянського повстання 1591-1593 рр. Здобув Білу Церкву (помста), приєдналися міщани та селяни. Взяли Трипілля, Богуслав, Переяслав, Волинь, Поділля (підтримка місцевих жителів). 1593 р. битва під П‘яткою з ополченням магнатів і шляхти.

Косіор Станіслав (1889–1939) Більшовицький партійний і радянський діяч, Генеральний Секретар та Перший Секретар ЦК КП(б)У, один з організаторів Голодомору в Україні 1932–1933. Репресований.

Костенко Ліна (1930 р.н.) Українська письменниця-шістдесятниця, поетеса. Перші книги Ліни Костенко «Проміння землі» (1957), «Вітрила» (1958),«Мандрівки серця» (1961) були новим словом в українській поезії. На початку 1960-х рр. Ліна Костенко брала участь у літературних вечорах Клубу творчої молоді. 1965 р. Костенко підписала лист-протест проти арештів української інтелігенції.

Костомаров Микола «Богдан Хмельницький», «Руїна», «Мазепа і мазепинці» створив фундаментальні праці переважно з історії Київської Русі та Гетьманщини. Після збірки Максимовича став патріотом.

Котляревський  Іван український письменник, поет, драматург, громадський діяч. Стояв біля джерел сучасного українського театру. Автор п‘єс «Москаль-чарівник» і «Наталка Полтавка». Поема «Енеїда» (1798) стала першим в українській літературі твором, написаним народною мовою. Вважається засновником сучасної української літератури.

Коцюбинський Михайло письменник, майстер соціально-психологічної новели. У повісті «Фата Моргана» відтзорив широку панораму суспільно-політичних подій 1905-1907 рр. в українському селі.

Коцюбинський Юрій радянський партійний та державний діяч. Входив з грудня1917 р. до складу першого радянського уряду України - заступник, а потім народний секретар військових справ. Боровся за встановлення влади більшовиків в Україні 

Кравс Антін генерал-четарь Украинской галицкой армии. Весною 1919 р., безпосередньо перед уведенням Диктатури ЗУНР (в зв'язку з польським наступом ген. Галлера), військовими колами Галицької армії висувалась пропозиція призначити саме Кравса Диктатором як успішного генерала, оскільки Петрушевич не мав військової освіти. В серпні 1919 року — командир (осередньої ІІІ-ї) армійської групи у Київській наступальній операції обох (ГА та Дієвої армії) українських армій, з якою зайняв Київ 30 та утримував його до 31 серпня 1919 (змушений був залишити столицю через непорозуміння в наказах з Головним отаманом військ УНР Петлюрою).

Кравчук Леонід (1934 р.н.) Український політичний діяч. Перший Президент України після здобуття нею незалежності (1991—1994 рр.), Голова Верховної Ради України у 1990—1991 роках, Народний депутат України у 1990—1991 рр. та 1994—2006 роках, Герой України (2001 р.).

Крилов Микола математик, механік, 

Кримський Агатангел вчений-мовознавець і сходознавець. Відомий поліглот (вільно володів майже 60 мовами). Його наукові праці - вагомий внесок у дослідження проблеми поход­ження й розвитку української мови.

Кричевський Василь (1873–1952) автор проекту будинку Полтавського земства, низки проектів державних і приватних будівель, Меморіального музею біля могили Т.Шевченка . піонер українського модерну, мотиви українського бароко, народного дерев’яного будівництва, виразом якого був орнамент. 

Кричевський Федір  (1879-1947) 1917 р. один із засновників і перший ректор Української академії мистецтва. У 1927 написав славетний триптих «Життя», який  1928 р. на Міжнародному конкурсі у Венеції, де мав великий успіх, «став справжньою сенсацією». Монументальний триптих «Життя» є найяскравішим зразком українського модернізму з елементами ар нуво (архітектура в стилі модерн) та українського релігійного живопису. Залишися в окупації.  Засланий до Сибіру. Не писав на радянську тематику.

Крушельницька Соломія (1872—1952) Співачка, солістка Львівського оперного театру, виступала на оперних сценах театрів у Петербурзі, Кракові, Варшаві, Парижі, Мілані, виконала близько 60 партій в операх «Запорожець за Дунаєм», «Пікова дама», «Мадам Батерфляй» та ін., виступала с концертами, присвяченими пам’яті Т. Шевченка. 

Кук Василь (Василь Коваль, Юрко Леміш, Ле, Медвідь; 1913-2007) Генерал-хорунжий, головнокомандувач УПА з 1950 (після загибелі Р. Шухевича). 1954 року був заарештований і відсидів у радянських в'язницях і таборах, не дочекавшись суду, 6 років. Останні роки проживав у Києві.

Кукольник Василь директором головного педагогічного інституту в Петербурзі, Кукольник читав курси права, фізики, хімії, сільського господарства

Куліш Пантелеймон (1819–1897) «батько українського правопису» (абетка). Перший український історичний роман «Чорна рада», збірка віршів «Досвітки», поема «Україна»,  про події від часів князя Володимира до гетьмана Богдана Хмельницького. Переклад українською мовою Біблії, поем Дж.Байрона та драм В.Шекспіра.

Курбас Лесь (1887—1937) Режисер-новатор, театральний діяч, один із засновників «Молодого театру», театру «Березіль», на сцені яких поставив вистави «Чорна пантера і білий ведмідь» В. Винниченка, «Мина Мазайло» М. Куліша та ін., «поєднував українське театральне мистецтво з європейською культурою.

Кучма Леонід (1938 р.н.) Кандидат технічних наук, працював заступником генеральногоконструктора КБ «Південне», генеральним директором ракето будівного концерну «Південний машинобудівний завод», перебував на посаді прем’єр-міністра України 1992—1993 рр., двічі обирався Президентом України (1994 р., 1999 р.).

Лебідь Миколо 1910-1998 один із лідерів ОУН-УПА. Перший начальник Служби безпеки ОУН (1940—1941). Організатор замаху на міністра внутрішніх справ Польщі Броніслава Перацького (1934). 

Лев Данилович 1264-1301 переніс столицю до Львова, приєднав Закарпаття/

Лев ІІ, Андрій І-правителі Галицько-Волинського князівства 1308-1323 рр.

Левинський Іван (1851–1919) за його проектами та інших архітекторів у Львові споруджено будівлі страхового товариства «Дністер» 1905–1906 рр. , бурси Українського педагогічного товариства, академічних дому, гімназії та ін.

Левицький Дмитро найкращий європейський портретист, роботи в Луврі, Женевський музей тощо. Боровиковський за наказом Катерини ІІ змушений був переїхати до Петербурга. портрети земляків, які служили на високих посадах у Петербурзі,– Дмитра Трощинського, Івана Безбородька та ін. Всього за своє життя майстер зробив близько 160 портретів. Його пензлю належать також виконані на рідній землі розписи церков у Миргороді, Кибинцях та Романівці.

Левицький Кость (1859–1941) голова Головної української ради (ГУР) та ЗУР. Перший очільник Тимчасового держаного секретаріату, уряду ЗУНР. 

Левицький Євген один з ініціаторів створення  РУРП та у 1899 р. Української національно-демо кратичної партії (УНДП) – основної політичної сили українства в краї. Під час Першої світової війни 1914—1918 років  за дорученням Союзу Визволення України вів організаційну і просвітницьку роботу серед полонених українців у німецьких таборах. 

Леонтович Микола створив оригінальні композиції на основі фольклору («Щедрик», «Ду дарик», «Козака несуть»), хорові поеми, опери.  

Лизогуб Федір голова Ради міністрів Української держави П.Скоропадського

Липинський Вʼячеслав - приєднання до національного руху відомих спольщених шляхетських родин. (як колись В.Антонович), обгрунтував ідею територіального патріотизму, яка показала шлях до консолідації українців у поліетнічну політичну націю. Разом з Сергієм і Володимиром Шеметами утворив УДХП. Автор теорії українського монархізму.

Липківський Василь (1864–1937) Український релігійний діяч, церковний реформатор, проповідник, педагог, публіцист, письменник і перекладач, борець за автокефалію українського православ’я, творець та перший митрополит Київський і всієї України відродженої у 1921 році Української Автокефальної Православної Церкви 1919–1927 рр. — останній настоятель та доглядач Софії Київської.

Лисенко Микола – український композитор, піаніст, диригент, педагог, збирач пісенного фольклору, громадський діяч. Учасник «Старої громади». Організатор першої в Україні Музично-драматичної школи. Опери «Різдвяна ніч», «Утоплена», «Енеїда», «Наталка Полтавка», перші дитячі опери «Коза-дереза», «Пан Коцький». 

Лук’яненко Левко  (1927 р.н.) Український дисидент часів СРСР, член УГГ, політик та громадський діяч, народний депутат України. Герой України. Письменник. Засновник «Української Робітничо-Селянської Спілки» 1959 р. Автор Акту про Незалежність України.

Лучкай Михайло «Історія карпатських русинів» та «Граматики слов‘янської мови» 1830 висловлювався за церковнослов‘янську мову в її карпатській редакції, як літературну мову закарпатських українців.

Любарт 1340-1384 володів Волинню, претендував на Галичину. Переніс столицю до Луцька, побудував замок Любарта.

Любченко Панас (1897-1937) член УЦР, належав до самоліквідованої у 1920 р. партії УКП (боротьбистів). Причетний до організації голоду. Виступив громадським звинувачувачем на процесі «СВУ», від 1934 до 1937 рр. - голова Раднаркому УСРР. Покінчив життя самогубством.

Людкевич Станіслав створив монументальну кантату-симфонію «Кавказ» на слова Тараса Шевченка написав , він також організував видання «Артистичного вісника» – першого українського мистецтвознавчого журналу. 

Лятошинський Борис 1894-1968 один із основоположників модернізму в українській класичній музиці.  Третя симфонія, «Слов’янський концерт» для фортепіано з оркестром

Лянге Антон  німець  пейзажеві майже цілком присвятив свою діяльність у Львові, вихованець Віденської академії мистецтв  

Мазепа  Іван 1687-1709 гетьман України. Уклав Коломацькі статті, 1701 p. I. Мазепа видав указ про дводенну панщину для селян, 1700-1704 повстання Семена Палія на Правобережжі, придушив та об’єднав Україну. 1700-1721 Північні війна між Росією і Швецією. 1708 р перейшов на бік Карла ХІІ. Покровитель культури, під його керівництвом було споруджено 12 храмів, реставровано багато памятокчасів Київської Русі. "Мазепинське бароко".

Майборода Платон (1918-1989) Український композитор, народний артист УРСР (з 1968 року), лауреат Державної премії УРСР імені Т. Шевченка (1962), народний артист СРСР (з 1979 року). Автор численних пісень і хорів, обробок народних пісень, а також ліричних пісень — в тому числі «Рушничок», «Якщо ти любиш», «Ми підем де трави похилі», «Київський вальс». В творчому доробку також симфонічна поема «Героїчна увертюра», вокальна-симфонічна поема «Тополя» (слова Тараса Шевченка), музика до драм. спектаклів, а також до фільмів.

Максимович Михайло видання українських пісень 1827 р. Перший ректор Київського університету. У фольклорі вбачав поетичний вияв народних почуттів і прагнень і таким чином хотів ознайомити широкий загал з національним характером українців.

Малиновський (Родіон 1898–1967) полководець. Маршал Радянського Союзу (1944 р.), двічі Герой Радянського Союзу (1945 р., 1958 р.). Міністр оборони СРСР (1957—1967 рр.). Війська під командуванням генерала армії брали участь у низці наступальних операцій 1943 і 1944 рр. — Запорізькій, Нікопольсько-Криворізькій, Березнегувато-Снігурівській, Одеській і Яссько-Кишинівській. Відзначився провальною Харківською операцією та підписанням миру з Румунією. 

Мануїльський Дмитро (1883–1959) Радянський державний і партійний діяч, заступник голови Раднаркому УРСР, народний комісар закордонних справ УРСР (1944 р.), голова делегації УРСР на міжнародній конференції у Сан-Франциско (1945 р.) та на Паризькій мирній конференції (1947 р.).

Марченко Валерій (1947–1984) Літературознавець, перекладач, правозахисник, автор листів-протестів проти тоталітарного режиму, противник русифікації, неодноразово заарештований і засуджений за «антирадянську агітацію і пропаганду», автор книги «Я не маю ні дому, ні вулиці».

Мартос Іван скульптор  Пам‘ятник градоначальнику та генерал-губернатору А. де Рішельє в Одесі. 1828 

Махно Нестор (1888–1934) Український повстанський отаман, один з лідерів анархістського руху. Влітку 1918 р. розпочав боротьбу проти влади П. Скоропадського, у середині грудня 1918 р. він уклав угоду з керівництвом Директорії УНР, яка виявилась недовготривалою. У січні 1919 р. розпочав боротьбу проти денікінців, військ Директорії і Антанти. Кілька разів ішов на зближення з більшовиками. З кінця листопада 1920 р. до серпня 1921 р. Махно вів боротьбу проти більшовицької влади. 

Мирний Панас  (Панас Рудченко) – автор соціально-психологічних романів і повістей. Правдивим літописом українського села є його роман (у співавторстві з І. Біликом) «Хіба ревуть воли, як ясла повні?». 

Мельхіседек Архімандрит (світське ім'я Матвій Карпович Значко-Яворський)  ігумен Мотронинського монастиря  благосовив Коліївщину. Чутки благословіння Синоду, війська на допомогу. 

Меленський Андрій (1766–1833) архітектор класицизму, набув своєрідного забарвлення. 1799–1829 рр. був головним міським архітектором Києва. У зв’язку з поверненням Києву 1802 р. прав міського самоврядування за його проектом було споруджено перший на українських землях «Пам’ятник на честь поновлення магдебурзького права», визначив Хрещатик як у майбутньому головну вулицю міста, створив новий центр губернської адміністрації на Печерську, збудував на Хрещатику перший в місті будинок театру, забудова центру Подолу після пожежі 1811 р. тут спорудили Контрактовий дім і перебудували Гостиний двір,  мавзолей-пам’ятник князю Аскольдові. 

Мельник Андрій (1890–1964) Український військовий і політичний діяч, полковник Армії УНР, співзасновник УВО, один із лідерів Організації українських націоналістів, голова ОУН-М, після смерті Є. Коновальця — головаПроводу ОУН, прихильник відновлення української державності. 

Мечников Ілля біолог, мікробіолог, імунолог, почесний член Петербурзької академії наук, лауреат Нобелівської премії (1908), відкрив явище фагоцитозу, разом з М. Гамалією створив бактеріологічну станцію в Одесі, займався проблемами боротьби з інфекційними захворюваннями. 

Микола 3 збірки пісень («Малоросійські пісні» 1827, «Українські народні пісні» (1834), «Збірник українських пісень» 1849) 5-томна «История Малороссии» 1842-1843 Головним джерелом історичних дослідження була «Історія Русів» та праці Д.Бантиш-Каменського. Розглядав роль козацької старшини, як еліти суспільства, яка відстоювала автономію козацької держави.

Михайло Глинський український магнат, очолив повстання за відновлення автономії руських земель у 1508 р.

Міндовг-засновник литовської держави у ХІІІ ст.

Міхновський Микола (1873—1924) У український політичний і громадський діяч, адвокат, публіцист, перший ідеолог українського націоналізму та організатор війська. Ідеолог і провідник Братства тарасівців, засновник УНП (1902), один із лідерів УДХП, член Братства самостійників. Автор брошури «Самостійна Україна» (1900), підготував проєкт Конституції (1905). Організатор товариства «Український військовий клуб імені гетьмана Павла Полуботка» та полку ім. Б.Хмельницького. Очололив повстання самостійників проти Другого Універсалу. 

Многогрішний Дем’ян  1668-1672 Наказний гетьман 1668р., з 1669р. гетьман Лівобережної України підписав з Московією Глухівські статті. Створив наймане військо - компанійців. В наслідок змови проти Д.Многогрішного в 1672р. гетьмана відправлено до Москви, а згодом на заслання до Сибіру.

Могила  Петро київський митрополит домігся визнання православної ієрархії у «Статтях для заспокоєння руського народу» у1632 р., провів релігійну реформу, уклав «Требник», заснував Київський колегіум.

Могильницький Іван 1819 р. «Граматика язика словено-руського» Греко-католицький священик. Став засновником у 1816 р. у Перемишлі "Товариства священиків" (разом з Михайлом Левицьким), що стало початком відродження у Галичині. Створив першу граматику української мови в Галичині

Морачевський Пилип автор першого перекладу  Євангелія Нового Заповіту сучасною українською мовою 1863 р. Привід до Валуєвського циркуляру.

Мстислав Великий - син Володимира Мономаха, правитель Київської Русі 1125-1132 рр. Зберіг єдність руських земель. Після смерті у 1132 р. Київська Русь остаточно розпадається на окремі князівства

Мстислав-син Володимира Великого, правив на Русі разом з братом Ярославом з 1019 р. по 1036 р.

Мудрий Василь (1893–1966) Український громадський і політичний діяч, журналіст. Брав участь в організації Львівського таємного українського університету. В період 1921–1931 рр. — член Головної управи товариства «Просвіта». Був одним із засновників Українського національно-демократичного об’єднання (УНДО).

Мурашко Олександр  імпресіонізм, жанровий живопис, «Дівчина в червоному капелюсі». Чи не єдиний з українських митців, який поєднав надбання французького імпресіонізму й ґрунтовну реалістичну школу Петербурзької академії мистецтв, створивши свій неповторний стиль. 

Муха-керівник першого селянського повстання на українських землях у 1490-1492 рр.

Нагірний Василь (1848–1921) архітектор. У 1883 p. створив у Львові кооперативне торговельне підприємство «Народна торгівля», яке стало першим споживчим кооперативом у західноукраїнських землях. Член у 1885 р. політичного товариства «Народна рада» Був першим головою «Сокола» 1894 р. Період розквіту цього товариства пов’язують із співпрацею зі знаним педагогом Іваном Боберським.

Наливайко Северин  - керівник козацько-селянського повстання 1594-1596 рр. 1594 р. поверталися з походу. Приєдналися до них реєстрові козаки (Григорій Лобода) Пройшли Поділля, Волинь, Полісся, Білорусь. Притік селян та міщан. Мета створити республіку. Повстанці перемогли під Білою Церквою. В урочищі Солониця поляки оточили табір повстанців. Протиріччя між реєстровцями (Лобота та Шаула) та Наливайком, призвели до поразки.

Нарбут Георгій (1886–1920) Український художник-графік, ілюстратор, автор перших українських державних знаків (банкнот і поштових марок). Один з засновників і ректор Української Академії Мистецтв.

Наумов Михайло Герой Радянського Союзу, в роки радянсько-німецької війни один з керівників партизанського руху в Україні, начальник штабу оперативної групи партизанських загонів Сумської області; командир партизанського кавалерійського з'єднання.

Нестор літописець, чернець Києво-Печерського монастиря, автор «Повісті минулих літ».

Нечуй-Левицький Іван письменник, який першим в українській прозі розповів про життя українськоїї нтелігенції («Причепа», «Хмари», «Над Чорним морем»), торкаючись болючої проблеми її зросійщення. Він утвердив в українській літературі жанр соціальної повісті, зразками якої є перелічені твори («Бурлачка», «Микола Джеря», «Кайдашева сім’я» та ін.). 

Озаркевич Іван священик  започаткував основи українського професійного театру на теренах підавстрійської України. У травні 1848 р. аматорський театр у Коломиї, співзасновником і керівником якого він був, поставив першу в Галичині українську виставу – «Наталку Полтавку». І. Озаркевич адаптував п’єсу до життя покутських українців під назвою «На милування нема силування». За словами І. Франка, «І. Озаркевичеві хотілося подати Котляревського галичанам не в оригіналі, а в підгірськім кептарі». У жовтні того самого року І. Озаркевич повторив цю виставу у Львові на З’їзді руських учених. Власне ця подія, на думку І. Франка, започаткувала український театральний рух у Галичині. 

Олег (882-912) київський князь, утворення держави Київська Русь, здійснив у 907, 911 рр. походи на Візантію (прибив щит на ворота Константинополя), походи на Кавказ та Каспій(проти хозар) та проти угорців.

Олелькович Михайло, Ольшанський Іван, БельськийФедір-руські князі, які у 1480-1481 рр. намагалися здійснити заколот з метою скинути литовське панування на українських землях і приєднати їх до Московської держави. План не здійснився

Олесь Олександр (Кандиба), Заживши слави короля української лірики, він створив чимало віршів громадянського звучання, у яких революційні гасла поєднав із людськими почуттями і клопотами буднів «Айстри», «З журбою радість обнялась». Батько поета та політичного діяча Олега Ольжича. Жив в єміграції у Празі. 2017 р. перепохований у Києві.

Ольга (945 – 964(969) впорядкування данини: «устав», «урок», «оброк», створення адміністративно-податкових одиниць – погостів +судова влада, започаткувала державні мисливські угіддя (податки з продажу хутра. меду, віску). Прийняла християнство. Стосунки з Візантією та німецьким імператором Оттоном І.

Ольгерд 1345—1377 великий князь литовський, внаслідок перемоги у битві на Синіх водах проти татар приєднав Чернігово-Сівещину, Київщину, Подіддя, Переяславщину, боровся за Волинь.

Омелянович-Павленко Михайло  Генерал-полковник Армії УНР. У груднi 1918 - червні 1919 pp. був головнокомандувачем УГА. 1919 р. за наказом С.Петлюри очодив Перший зимовий похід. 3 листопада 1920 р. перебував у Ta6opi інтернованих вояків у Польщі, згодом - в еміграції. 

Омельченко Андрій сотник студентського військового формування УНР під час битви під Крутами 19(16) січня 1918 р. 

Оріховський Станіслав  (1513-1566) письменник полеміст, прихильник ідей М.Лютера, завзято виступав на захист народної мови і руського обряду. Одним з перших висунув ідею єдності слов'янських народів. Підтримував унію.

Орлай Іван медик, випускник Львівського та Віденського університетів. Був ученим секретарем Медичної академії, багато наукових праць з історії медицини. Історію Закарпатської України 9-18 ст. очолював Ніжинську гімназію та Рішельєвський ліцей. Звинувачений у вільнодумстві, помер під час слідства.

Орлик Пилип . гетьман обраний за межами України. Автор документа «Пакти й конституції законів і вольностей Війська Запорозького» (першої Конституції на українських землях .1710р.) В 1711р. при допомозі Туреччини здійснив похід на Правобережну Україну. Після невдалого походу перебрався спочатку до Швеції, потім до Німеччини, а згодом до Франції. З 1722р. змушений був переїхати до Туреччини де і прожив останніх 20 років.

Осиповський Тимофій математик, ректор Харківського унів. 1813-1820 , спостереження небесних явищ, тритомний «Курс математики»

Остроградський Михайло очолював кафедру математики в паризькому коледжі, зсновник Петербурзької наукової математичної школи, член академій наук Нью-Йорка, Італії, Росії. Дружні стосунки з Т.Шевченком.

Острозький Василь-Костянтин  - «некоронований король Русі» , приймав участь в обговоренні Люблінської та Берестейської уній, де захищав права православних. Був претендентом на польський і московський престоли (вів родословну від Рюриковичів).Розбудував Острозький замок. Створив у 1576 р Острозьку слов'яно-греко-латинську академію. Надруковано перший повний текст Біблії церковнослов’янською мовою 1581 р. - «Острозька Біблія». Під час повстання Криштофа Косинського у 1591—1593 рр., у вирішальній битві під П’яткою військо Островського нанесло нищівної поразки повсталим, виступив проти повстання Северина Наливайка у 1594—1596 рр.

Острозький Костянтин Іванович  один із найвидатніших полководців 16 століття, провів 35 битв, із них програв тільки дві. приймав участь на посаді литовського гетьмана (очільника війська) у війнах з проти кримських татар, нападаючи на них коли ті поверталися із здобиччю, завдяки цьому відбулося заселення Київщини та Поділля. Також брав участь у походах на Московське князівство, переміг у битві під Оршею (1514).

Острянин Яків  - керівник повстання 1638 р. Повстання тривало на Лівобережжі. Розбили поляків під Голтвою. Оточення Я.Острянина. Поразка козаків під Лубнами й Жовниром, перехід кордону Московської держави. Перша хвиля колонізації Слобощанщини.

П’ятаков Георгій 1918  секретар ЦК КП(б)У. В листопаді 1918 П. увійшов до складу Української Революційної Військової Ради (Й. Сталін, В. Затонський і В. Антонов-Овсіенко), яка розробила план і провела підготовку до вторгнення російських більшовицьких військ в Україну. голова тимчасового робітничо-селянського уряду в Україні з листопада 1918 р. по січень 1919 р. Організатор масових розстрілів у Криму в 1920 р. 

Павлик Михайло (1853–1915)  співзасновник разом з І.Франком Русько-української радикальної партії (РУРП), її голова 1898-1914. Член НТШ. 

Павловський Олексій 1818 «Граматика малороссийського наречия» Друкована граматика живої української мови.

Павлюк (Бут) Пало  - керівник повстання 1637 р. Битва під Кумейками (біля Черкас) і Боровицею (біля Чигирина) поразка козаків. Переговори, посередник Адам Кисіль. Павлюк страчений.

Палладін Олександр український біохімік. Президент Академії наук Української РСР(1946–1962), академік АН УРСР і АН СРСР. Засновник української школи біохіміків. 

Параджанов Сергій (1924—1990) Український і вірменський кінорежисер, народний артист УРСР, режисер фільму «Тіні забутих предків», учасник акцій протесту проти масових арештів інтелігенції у 1960-ті рр., заарештований і засуджений до тривалого ув’язнення, лауреат Державної премії ім. Т. Шевченка (посмертно).

Пархоменко Олександр член РСДРП з 1904 року. В червні 1917 року призначений начальником штабу Червоної гвардії Луганська. 1919 — начальник більшовицького гарнізону Харкова, потім у Першій кінній армії. 1921 загинув. 1942 — знято фільм «Олександр Пархоменко».

Патон Борис (1918 р.н.) Видатний вчений, фахівець у галузі зварювання, металургії та технології металів; доктор технічних наук, професор, академік, президент НАН України, перший відзначений званням Герой України. Автор багатьох фундаментальних досліджень і створених на їх основі високих технологій, організатор науки, державний і громадський діяч.

Патон Євген (1870-1953) Видатний учений у галузі електрозварювання та мостобудування, академік АН УРСР (з 1929 р.). У 1929 р. організував в АН України кафедру інженерних споруд, на базі якої створено Інститут електрозварювання (1934 p.). Протягом 1934—1953 pp. — директор цього інституту. Під його керівництвом винайдено спосіб автоматичного швидкісного зварювання, який відіграв визначну роль у технічному розвитку людства.

Петлюра Симон (1879–1926) Громадсько-політичний і державний діяч, публіцист, член РУП і УСДРП, автор статті «Війна і українці», член Української Центральної Ради, секретар військових справ першого українського уряду — Генерального Секретаріату.

Петровський Григорій (1878-1958)  З березня 1919 до літа 1938 року — голова ВУЦВК Всеукраїнський Цкнтральний виконавчий комітет - реальний орган влади («всеукраїнський староста»). Единий з парткерівництва уникнув репресій. Причетний до організації голоду.

ПетрушевиЄвген ч (1863–1940) Український громадсько-політичний діяч, президент і диктатор (верховний військово-політичний зверхник в часі війни) Західноукраїнської Народної Республіки (ЗУНР).

Пимоненко Микола художник-реаліст «співець українського села», побутовий жанр, «Святочне ворожіння», «Весілля в Київській губернії», «Жнива», «Проводи рекрута».  . 

Підкова Іван  козацький ватажок. Здійснював походи на Молдавію та проголосив себе молдавським володарем 1577 — 1578. За вимогою султана був схоплений та страчений за наказом С.Баторія.

Плетенецький  Єлисей архімандрит Києво-Печерської лаври, засновник Лаврської друкарні 1615 (1616 «Часослов»).

Полуботок Павло  Наказний гетьман здійснив реформу суду (Генеральний суд стає колегіальним, установив порядок подання апеляцій); звертався зі скаргами в Сенат на порушення Малоросійською колегією українських законів і традицій. У 1723р. представники старшинської опозиції в козацькому таборі на р.Коломак склали - Коломацькі чолобитні, коли ці чолобитні отримав Петро І, то наказав ув’язнити Полуботка. В 1724р. Павло Полуботок помер у Петропавлівській фортеці.

Попович Омелян член Української Національної Ради від Буковини, очолив Крайовий комітет на Буковині  у 1918 р., який провів у Чернівцях Буковенське народне віче за приєднання до ЗУНР.

Порошенко Петро (1965 р.н.) Український державний та політичний діяч, підприємець, мільярдер, п’ятий Президент України.

Постишев Павло (1887–1939) Радянський державний діяч, член Політбюро ЦК КП(б)У, одночасно перший секретар Харківського, а згодом Київського обкому Компартії, один із організаторів голодомору 1932—1933 рр. в Україні, сприяв згортанню українізації, організації політичних процесів, винищенню української інтелігенції. Репресований.

Потій Іпатій  Київський і Галицький митрополит, визначний письменник-полеміст, який палко обстоював унію. Один із ініціаторів Берестейської унії 1596 р. Після смерті митрополита М.Рогози (1599) став Київським греко-католицьким митрополитом.

Прокопович Феофан  у 1707-1708рр. уклав курс лекцій з арифметики та геометрії для Києво-Могилянської академії. В стінах цього закладу виступив з промовою перед студентами і професорами «Про заслуги й користь фізики». У 1705р. написав історичну драму «Володимир» присвячену гетьману Івану Мазепі.

Птуха Михайло (1884-1961) український статистик і демограф, економіст, організатор науки, засновник Інституту Демографії АН УРСР – першої в світі науково-дослідної установи з демографії.

Пулюй Іван фізик та електротехнік, був першим винахідником рентгенівської трубки, завдяки якій зробив фотографію скелета людини; винайшов багато електротехнічних при­ладів. Один з перших перекладачів Біблії 1905 р. разом з П.Кулішем, перший декан першого в Європі електротехнічного факультету 

Раковський Християн (1873–1941) Політичний і державний діяч УСРР, член КП(б)У, голова Раднаркому УСРР 1919 —1923 рр., прихильник розширення політичної та економічної самостійності УСРР, виступав проти «автономізму Й.Сталіна, посол СРСР у Великій Британії та Франції.

Ревуцький Лев протягом 1944 р. — 1948 р. очолював Спілку композиторів України. 

Рильський Максим (1895–1964) Український поет, перекладач, публіцист, громадський діяч, академік АН України. Його обурення проти ідейно-політично ї та літературної атмосфери, що панувала тоді в СРСР врешті закінчилася арештом НКВС у 1931 р. Після цього Рильський проголосив активне сприйняття радянської дійсності, завдяки чому він єдиний з неокласиків урятувався від сталінського терору і був зарахований до числа офіційних радянських поетів. Його творчість поділилась на два річища — офіційне та ліричне, в останньому йому вдавалося створити незалежні від політики, суто мистецькі твори, які пережили його.

Рогоза Михайло  київський митрополит, після 1596 перший митрополит греко-католицької церкви.

Розумовський Кирило  1750-1764 У 1750р. російська імператриця Єлизавета затвердила на гетьманство Кирила Розумовського, однак лише у 1751р. гетьман прибуває до Глухова де старшинська рада обирає його гетьманом. До складу гетьманщини відходить Запорозька Січ і Київ. Протягом 1760-1763рр. провадилась судова реформа, наслідком якої Гетьманщина мала перетворитись з військової на цивільну державу. Гетьман провадив реформи в армії та освіті; почав розбудовувати Батурин, де планував відкрити університет. В 1763р. імператриця Катерина II отримала прохання старшини закріпити гетьманство за родом Розумовських. Це було однією з причин ліквідації гетьманства (1764р.)

Роман Мстиславович (1199-1205) волинський князь, об‘єднав Галицько-Волинське князівство, поширив вплив на Київ, в літописі згадується, як «самодержавець всієї Руської землі».

Романчук Юліан  (1842–1932) Член-засновник «Просвіти», голова «Народної ради», депутат Галицького сейму (1883–1895) та австрійського парламенту (1891–1897, 1901–1918). Із 1910 р. – його віце-президент. 1894 р. виступив проти політики "Нової ери". Член УНДП.

Русин Павло (1470-1517)   - перший гуманістичний поет України. 1509 видав збірку своїх поезій «Пісні Павла Русина з Кросна» 

Рутський Йосиф-Вельямін  митрополит греко-католицької церкви, шукав способів порозумітися з православними, заснував Василіанський чернечий орден.

Рюрик-правитель варязького племені, якого у 862 р. запросили на князювання новгородці. Засновник династії руських правителів – Рюриковичів

Сабуров Олександр (1908–1974) Радянський військовий діяч, генерал-майор військ НКВД, Герой Радянського Союзу, начальник Житомирсь­кого обласного штабу партизанського руху, член нелегального ЦК КП(б)У.

Савченко Ігор радянський, український кінорежисер, сценарист, театральний педагог.  Лауреат трьох Сталінських премій. Фільми «Гармонь», «Дума про козака Голоту», «Вершники», «Вершники» (1939), «Богдан Хмельницький» (1941; за О. Корнійчуком), «Третій удар» і «Тарас Шевченко» (закінчений учнями режисера після його смерті у 1951); «Партизани в степах України» (1942; за О. Корнійчуком) та ін.. До сценарію фільму, ним написаного, офіційна цензура поставила вимогу про необхідність збереження епізоду, у якому відображено реакцію Т. Шевченка на революцію у Франції. 

Садовський (Тобілевич Микола) (1856—1933) Український актор, режисер, один із засновників українського професійного театру, виконавець драматичних ролей у п’єсах І. Карпенка-Карого «Безталанна», М. Гоголя «Ревізор», М. Старицького «Тарас Бульба». 1906 р. в Полтаві перший український стаціонарний театр, 1907 переїхав до Києва.

Самойлович Іван  1672-1687 Гетьман Лівобережної України з 1672р. підписав з Московією Конотопські статті. Після Бучацького договору здійснив похід на Правобережжя, в 1674р. Івана Самойловича проголосили на козацькій раді в Переяславі гетьманом обох боків Дніпра.. був ініціатором так зв. «Великого згону» 1678р. - примусове переселення на лівий берег Дніпра. Формуючи козацьку аристократію започаткував бунчукове товариство (до складу якого входили сини козацької старшини). Був фундатором численних соборів і церков (Троїцький собор Густинського монастиря під Прилуками).

Самокиш Микола  засновник батального живопису в Україні. Найвідоміші його картини – «Бій Максима Кривоноса з Ієремією Вишневецьким» та «Бій Богуна з Чарнецьким під Монастирищем в 1653 р.». Протягом 1901–1908 рр. разом із С. Васильківським М. Самокиш писав монументальні панно для інтер’єру будинку Полтавського земства. 

Сверстюк Євген (1928 р.н.) український письменник, філософ, гоголезнавець, головний редактор православної газети «Наша віра», президент Українського ПЕН-клубу. Доктор філософії. Автор одного з найважливіших текстів українського самвидаву — «З приводу процесу над Погружальським». Політв’язень радянського режиму.

Свидригайло (1430-32, князь Руський до 1440, Волинський до 1452) боровся за права православного населення з Вітовтом та Сигізмундом. Відновив удільний устрій Руських земель.

Світличний Іван (1829–1992) Літературознавець, мовознавець, літературний критик, поет, перекладач. Через свою громадянську позицію зазнав переслідувань з боку репресивних органів тоталітарної радянської влади. Лауреат Державної премії України імені Т. Г. Шевченка.

Святополк-син Володимира Великого, розпочав міжусобні війни з братами, підступно вбив своїх братів Бориса та Гліба

Святослав (964-972) похіди на Волську Булгарію, розгромив Хазарський каганат. Два болгарські походи (другий поразка під Аркадіополем), 968 напад печенігів на Київ, на шляху додому вбито. Поділив землі між синами (уділи).

Севрюк Олександр член української делегації у Бересті (став головою уряду наприкінці переговорів, підписав договір) , посол УНР у Берліні. 1919 член делегації УНР на Паризькій мирній конференції. 

Симоненко Bасиль (1935–1963) Український поет, прозаїк, журналіст, представник покоління «шістдесятників», «лицар українського відродження», автор збірок поезій «Тиша і грім», «Земне тяжіння», лауреат Державної премії ім. Т. Шевченка (посмертно).

Сігізмунд 1432-1440 великий князь литовський, представляв інтереси католиків, боровся проти Свидригайла, переміг у Вількомирській битві 1435 р. Вбитий руським шляхтичем.

Сірко Іван  Неодноразовий кошовий отаман Запорозької Січі. Провів понад 60 битв проти військ Османської імперії, Кримського ханства та ногайських орд і жодного разу не зазнав поразки. В 1672р. Сірка було ув’язнено й відправлено до Сибіру. За його звільнення клопотався польський король Ян III Собеський. Першим в історії військового мистецтва переправився через Сиваш (Гниле море). Російський художник І.Рєпін у своєму творі «Запорожці пишуть листа турецькому султану» зобразив Івана Сірка в центрі картини.

Скілур-цар Малої Скіфії у II ст. до н. е.

Сковорода Григорій Філософ, гуманіст,просвітитель, поет, педагог. Працював викладачем етики та поетики в Харківському та Переяславському колегіумах. На його думку людина може стати щасливою пізнавши самого себе, а також слід займатись у житті тим, що людині природно відповідає. Автор творів «Сад божественних пісень», збірник байок «Басни харковскія ». пісню Сковороди «Всякому городу нрав і права» використав І.Котляревський у п’єсі «Наталка Полтавка». Йому належить вислів: «Світ ловив мене , але не спіймав».

Скоропадський  Іван 1708-1722 Після антимосковського повстання Мазепи стає гетьманом. При гетьманові призначався царський резидент Ізмайлов, який мав «явні і таємні статті» згідно яких йому надавалось право здійснювати контроль за гетьманом та урядом України. В цих статтях землі Війська Запорозького називались Малоросійським краєм. В 1722р. царським указом на українських землях розпочала свою діяльність Малоросійська колегія на чолі з бригадиром Вельяміновим. Цей крок російського імператора змусив І.Скоропадського передати усю повноту влади наказному гетьману П.Полуботку.

Скоропадський Павло (1873–1945) Український державний і політичний діяч, генерал-майор, флігель-ад’ютант російського імператора Миколи ІІ, учасник Першої світової війни, командуючий Першим українським корпусом, почесний військовий отаман Вільного козацтва, гетьман Української Держави (1918 р.)

Скрипник Микола (1872–1933) Радянський державний діяч, один із засновників КП(б)У, голова першого радянського уряду України — Народного Секретаріату, народний комісар освіти УСРР (1927—1933 рр.), «батько українізації», прихильник політики українізації, звинувачений у «націоналістичному ухилі».

Сліпий Йосип (1892-1984) Український церковний діяч, ректор Львівської духовної семінарії, патріарх УГКЦ, звинувачений у «ворожій діяльності проти УРСР» і засуджений до 8 років ув’язнення (1945 р.), звільнений з ув’язнення й висланий за межі СРСР завдяки виступам відомих політичних діячів країн світу на його захист (1963р.). 

Смаль-Стоцький Степан 1885 р. він стає завідувачем кафедри української мови і літератури у Чернівецькому університеті. Згодом його обирають деканом філософського факультету. У 1891 р. він очолив політичне товариство «Руська рада» і перетворив його на найвпливовішу українську організацію на Буковині. У 1892 р. його обрано послом Буковинського крайового сейму, на якому постійно обговорювали життєво важливі проблеми українців. 

Смотрицький  Мелентій письменник полеміст, виступав проти унії, пізніше підтримав. Автор підручник старослов'янської мови «Граматика словенська» (1619 p.).

Смотрицький Герасим  перший ректор Острозької Академії, редактор «Острозької Біблій», полеміст, автор «Ключ царства небесного» (1587) та «Календаря римського нового».

Соколовський Дмитро  (1894 — 1919) один з керівників повстанського руху проти більшовиків в 1919 році на Волині, Київщині та Поділлі. Ввійшов до складу створеного в Трипіллі на Київщині ревкому повстанців, головою якого став Олександр Грудницький, військовим комісаром був отаман Зелений (Данило Терпило), отаман Євген Ангел.  Під Києвом, простягнувшись на кілька десятків кілометрів, утворився «Зеленівський фронт». Наступ на Київ з півночі розпочав отаман Соколовський разом із загонами отамана Струка, а отаман Зелений трохи пізніше почав наступ з півдня. Після поразок отамана Зеленого, Дмитро Соколовський повертається в Горбулів, де знову бореться з більшовиками. Планував наступ на Київ на зустріч військам Петлюри та УГА  але був вбитий.

Соленик Карпо відомий актор 19 ст.. Своє життя він присвятив служінню саме українському театру, відмовившись виступати на російській імператорській сцені.  

Сосюра Володимир (1897–1965) Український письменник, поет-лірик, автор понад 40 збірок поезій, широких епічних віршованих полотен (поем), роману «Третя Рота», козак Армії УНР. Належав до низки літературних організацій того періоду — «Плуг», «Гарт», «ВАПЛІТЕ» та ін.

Сошенко Іван художник реаліст, «Портрет бабусі М.Чалого», «Жіночий портрет», жанрові твори «Хлопчики-рибалки», «Продаж сіна на Дніпрі».

Срезневський Ізмаїл 1833-1838 рр. 6 випусків «Запорожской старины» Дослідник української старовини. Студент, а згодом викладач Харківського університету. Разом з Г.Квіткою-Основ‘яненком став засновником гуртка «харківських романтиків». Провідник ідеї слов‘янської єдності (панславізму).

Сталін Йосип (1879–1953) Радянський державний діяч, генеральний секретар ЦК РКП(б) з 1922 р., домігся абсолютної влади в партії й державі, організатор «радянської одернізації», терору голодом, масових репресій, голова Державного комітету оборони, Верховний головнокомандувач під час Великої Вітчизняної війни.

Станкевич Євген (1942 р.н.) Український композитор, голова Національної спілки композиторів України (з 2005 р.), заслужений діяч мистецтв УРСР (1980), народний артист УРСР (1986), Герой України (2008). Автор опери «Коли цвіте папороть» і Rustici, а також музики для більш ніж 100 фільмів.

Стаханов Олексій (1906–1977) 30 на 31 серпня 1935 року (друга п´ятирічка) вибійник шахти «Центральна-Ірміне» в Кадіївці застосував новий метод роботи, перевизив норму у 14,5 раза., зачинатель масового робітничого руху за підвищення продуктивності праці й досягнення високих виробничих показників (стахановського руху) у 1930-х рр.

Степанів Олена (1892-1963) хрунжа,  командир стрілецької чети в легіоні Українських січових стрільців. перша в світі жінка, офіційно зарахована на військову службу у званні офіцера; четар Української Галицької Армії. Історик. Закінчила Віденський університет, захистила дисертацію. 1949-56 рр. перебувала в радянських таборах.

Стефан Баторій 1576—1586 король Польщі, збільшив козацький реєстр до 600 чол., надав територію навколо м. Трахтемирова, створив 6 полків, надав клейноди.

Стефаник Василь західноукраїнський письменник, неперевершений майстер соціально-психологічної новели. Однією з тем творчості була вимушена еміграція галицьких українців до Канади і США. Член Покутської трі́йці — умовне об'єднання трьох українських письменників Леся Мартовича, Василя Стефаника та Марка Черемшини.  

Стефанович Яків (Дмитро Найда)член народницького руху. В околицях Чернігова використовуючи віру селян в «доброго царя»,  зачитували «царський маніфест», повставати проти панів. Створили у Чигиринському повіті селянську організацію «Таємна дружини». 1877 р. члени «Чигиринської змови» заарештовані.  

Стецько Ярослав скликав Національні збори, які 30 червня 1941 р. проголосили Акт проголошення Української Держави й обрали Ярослава Стецька прем'єром Українського Державного правління (у віці 29 років). Стецька заарештували і запроторили до концтабору Заксенхаузен, де він перебував до вересня 1944 р. У 1945 р. крайова конференція ОУН-Б обирає членом бюро Проводу ОУН, до якого належали Степан Бандера і генерал Роман Шухевич. Голова Організації українських націоналістів у 1968-1986 рр., президент Антибільшовицького блоку народів у 1946-1986 рр.

Стражеско Микола (1876–1952) Український терапевт, доктор медицини (1904), професор (1907), заслужений діяч науки УРСР (1934), академік АН УРСР (1934) та академік АН СРСР (1943). Автор більш ніж 100 наукових праць, присвячених різноманітним питанням клініки і лікування внутрішніх хвороб. Вперше у світі дав розгорнутий опис різних клінічних форм інфаркту міокарда і виявив його основну патогенетичну ланку.

Строкач Тимофій (1903—1963) — організатор партизанських формувань в Україні в роки німецько-радянської війни, генерал-лейтенант. До початку війни служив у прикордонних військах. Учасник оборонних боїв 1941 за Київ і Москву. В 1942—1945 pp. — начальник Українського штабу партизанського руху (УШПР). Під його керівництвом УШПР перетворив партизанські з´єднання та загони України на фактор стратегічного значення. Після війни — заступник народного комісара НКВС УРСР (1945—1946), міністр внутрішніх справ УРСР (1946—1956), начальник Головного управління прикордонних військ, заступник міністра МВС УРСР (1956—1957).

Струк Ілько   брав участь у звільненні Арсеналу від більшовицьких заколотників. Підвладне йому військо офіційно називалося Першою повстанчою армією УНР. В квітні 1919 Ілько Струк разом з отаманами Бурлакою та Зеленим  повстанням проти більшовиків на Волині, Київщині та Поділлі.   Воював проти більшовиків разом з об'єднаною армією УНР-УГА, Денікіним, наступом на Київ українсько-польських військ. 1920 повернуся додому в Чорнобиль де очолив антибільшовицьке повстання. Ілько Струк так сформулював свою позицію: «Мій висновок: я партизан і повстанець своєї неньки-У країни й оборонець її від грабіжників… Не був, не буду й на думці не маю бути на боці ворогів своєї Батьківщини. Знищу всіх, хто хоч у душі на їх боці…»  Радянська пропаганда писала: «Ярый националист… один из наиболее крупных организаторов кулацких банд…» «Гидкий шкуродер»… «Один из наиболее выдающихся главарей банд… ярый украинец». Коли Визвольні змагання завершилися поразкою, Струку (орієнтовно взимку 1922 — 23 років) вдалося емігрувати до Польщі. 

Ступка Богдан (1941-2012) Український актор театру і кіно, лауреат Шевченківської премії (1993), Народний артист УРСР (1980), Народний артист СРСР (1991), Герой України (2011). Загалом у Ступки — близько 100 ролей у кіно та понад 100 на сцені. У переліку ролей Б.Ступки — багато історичних постатей: гетьмани Іван Брюховецький («Чорна рада»), Іван Мазепа («Молитва за гетьмана Мазепу»), Богдан Хмельницький («Вогнем і мечем»)

Стус Василь (1938–1985) Український поет, літературознавець, літературний критик, представник «шістдесятників», учасник акції протесту у київському кінотеатрі «Україна» 1965 р., правозахисник, член Української Гельсінської групи, неодноразово заарештований і засуджений за «антирадянську агітацію і пропаганду», помер у таборі для політв’язнів, автор поетичної збірки «Дорога болю», 

Сулейман І 1520-1566 султан Османської імперії в період її найвищого розквіту. Дружиною Сулеймана I була Роксолана.

Сулима Іван  кошовий отаман у 1635 р зруйнував Кодак. 1639 році відбудовано.

Сулькевич Сулейман (Матвій) Олександрович  5з червня по  листопад 1918 р. при підтримці німецької окупаційної влади займав посади прем'єр-міністра, міністра внутрішніх справ та міністра військових справ Кримського Крайового Уряду. Через виведення німецьких військ з півострова, загрозливу економічну ситуацію та незадоволення кримських політичних кіл залишив свою посаду. 

Сухомлинський Василь  ( 1948-1970) видатний український педагог, засновник гуманістичної, новаторської педагогіки. З 1948 по 1970 рік — директор Павлиської середньої школи. Кандидат педагогічних наук, член-кореспондент АПН СРСР. Автор низки педагогічних праць: «Серце віддаю дітям». 

Тацит, Пліній Старший, Птоломей, Прокопій Кесарійський, Йордан, Маврикій Стратег…-римські, візантійські та готські автори, які вперше згадують давніх слов’ян (венедів, антів, склавинів)

Теліга Олена (дівоче прізвище — Шовгенів) (1907-1942) поетеса у 1926 році вийшла заміж за сотника армії УНР М. Телігу. 1932 року вступила до Організації українських націоналістів. Після розколу 1940 року залишилася в ОУН(М). 1941 року разом з чоловіком переїхала до Львова. Того ж року разом з похідними групами ОУН вирушила до Києва, де організувала Спілку українських письменників, редагувала літературно-мистецький додаток до газети «Українське слово» — часопис «Литаври». 7 лютого 1942 року заарештована й разом з чоловіком розстріляна в Бабиному Яру.

Терещенко Михайло (1886—1956)  підприємець,  Депутат IV Державної думи. Родині Терещенків тогочасний Київ завдячував дитячим притулком, лікарнею, двома гімназіями та двома училищами, кількома корпусами Політехнічного інституту. Терещенки збудували Музей старовини і мисте цтва (тепер Національний художній музей України), Троїцький народний дім (нині Театр оперети). У Глухові для дітей бідних батьків і сиріт благодійники заснували міське училище, для якого було збудовано приміщення з бібліотекою Надавав допомогу Київському міському музею, Київській консерваторії (1912 р.); був членом «Російського Червоного Хреста». У 1915—1917 рр. керував Київським військово-промисловим комітетом, був членом Всеросійського військово-промислового комітету. У березні-травні 1917 р. — міністр фінансів у Тимчасовому уряді Росії, у травні-жовтні 1917 р. — міністр закордонних справ.

Терпило (отаман Зелений)Данило   1886-1919 1918 р. у Трипіллі Дніпровську дивізію (2,5 тис. осіб). Разом із загонами С.Петлюри та січових стрільців Є.Коновальця у грудні 1918 р. оволодіває Києвом. Став главою “Придніпровської республіки” в січні 1919 р. переходить на бік більшовиків. Але в квітні 1919 висунуте гасло: “Ради без комуністів”. Остаточно загони отамана були знищені в серпні 1919 р., отамана було смертельно поранено. 

Тетеря Павло  - При Б.Хмельницькому брав участь у підписанні українсько-російського договору. Перший гетьман Правобережної України 1663-1665. Взяв участь у поході польського короля

Тимошенко Семен (1895–1970) Радянський військовий діяч, генерал, учасник Великої Вітчизняної війни, командуючий військами Південно-Західного фронту після загибелі М.Кирпоноса, представник Ставки Верховного головнокомандування. У 1939 р. керував «визволенням Західної України».

Тичина Павло (1891–1967) Український поет, перекладач, публіцист, громадський діяч. Автор збірок «Плуг» (1920) і «Вітер з України» (1924). Переломною у творчості поета вважається збірка «Чернігів» (1931 р.), після якої Тичина став «офіційним» автором. Відомі твори: «Григорій Сковорода» (1939), Похорон друга» (1942), «Творча сила народу», «Геть брудні руки від України» (1943) та інші.

Трильовський Кирило (1864 — 1941) Громадсько-політичний діяч, адвокат, організатор молодіжних спортивно -пожежних патріотичних товариств «Сокіл», «Січ», «Пласт», голова Бойової управи УСС, член Головної української ради, автор історичних нарисів.

Тропінін Василь – син кріпака, шедеври Портрет українця», «Молодий український селянин», «Дівчина з Поділля», «Пряла».

Труш Іван Краківська академія, імпресіонізм,  пейзажний жанр, 350 портретів, Видатний портретист, майстер пейзажу та побутового жанру. Опанувавши в Краківській академії мистецтв засади імпресіонізму, митець сприяв поширенню в Україні техніки західних модерністських течій. Івана Франка.  Лесі Українки Малював його художник з натури 1900 р. Цей портрет вважають одним з найкращих прижиттєвих портретів Лесі Українки.  

Трясило Тарас керівник козацько-селянське повстання 1630 р. 15 травня під Переяславом бій «Тарасова ніч» розгромили військо С.Конецпольського, повністю винищив його охорону — «Золоту роту». Підписана Переяславська угода.

Тухачевський Михайло під час Польсько-радянської війни 1920 року очолював Західний фронт похід на Варшаву. Розстріляний у 1837 р.

Тютюнник Юрій (1891–1930) Український військовий діяч, генерал-хорунжий армії УНР. 1917 року організував у Звенигородці кіш (батальйон) вільного козацтва, став його командиром (отаманом). У лютому 1919 р. частини Тютюнника об’єдналися з загонами Матвія Григор’єва, а сам Тютюнник став головою його штабу. Очолив Перший та Другий зимові походи. В кіно зіграв самого себе. Репресований.

Удовиченко Олесандр (1887-1975) генерал-полковник Армії УНР.  В червні 1919 р. захопив Камʼянець, початок Камʼянецької доби Директорії (до листопада 1919 р.).  У берені 1920 р. сформував і очоли Залізну дивізію, брав участь у спільному з поляками поході на Київ. Із 1924 р. мешкав у Франції, військовий історик.

 Федоров Олексій (1901–1989) організатор партизанського руху, двічі Герой Радянського Союзу, активний учасник «рейкової війни» (очолюваний ним загін підірвав 549 ешелонів), начальник Чернігівського обласного штабу партизанського руху Після розформування з´єднання з 1944 до 1957 р. працював першим секретарем Херсонського, Ізмаїльського та Житомирського обкомів КП(б)У. Був міністром соціального забезпечення УРСР.

Федорович Іван  першодрукар надрукував у Львові у 1574 р. Апостол і перший український друкований підручник для навчання грамоти - "Буквар", а у 1581 р. Острозьку Біблію.

Федькович Юрій  (1834–1888) "буковинський кобзар" засновником нової української літератури на Буковині, перший збирач фольклорних матеріалів на Буковині й організатор зби рання фольклору. Представник народовців Буковини, Редактор «Буковини» – першої української політичної газети на Буковині (почала виходити 1885 р.). У 1862 р. вийшла друком дебютна збірка поезій митця, 1876 р. у Києві надруковано першу збірку прозових творів.

Феодосій, Антоній-засновники Києво-Печерського монастиря у 1051 р.

Філатов Володимир (1875–1956) Радянський науковець, офтальмолог , хірург, винахідник, поет, художник, мемуарист російського походження. Засновник та перший директор, з 1936 року по 1956 рік, Інституту очних хвороб і тканинної терапії НАМН України. За весь період життя Філатов написав близько 460-ти наукових праць та монографій. Разом з тим, він займався громадською та політичною діяльністю. Прославився захисником церковних пам’яток від наруги правлячим режимом.

Фіоль Швайпольт у 1491 р. видав кирилицею у краківській друкарні "Осьмигласник", "Тріодь цвітна", "Часословець".

Франко Іван (1856–1916) Український поет, учений, громадський діяч, член Наукового товариства ім. Т. Шевченка, один із засновників Української радикальної партії, та член УНДП, автор віршів «Каменяр», «Вічний революціонер», повісті «Захар Беркут», драми «Украдене щастя». Блискучий майстер поетичної творчості, автор соціально-психологічних та історичних повістей, письменник-драматург і учений-літературознавець, праці якого стали етапними в розвитку української критики.

Фрунзе Михайло командував Східним фронтом проти О.Колчака та Туркестанським фронтом. 1920 командувач Південного (Кримського) фронту разом з Н.Махном розбив П.Врангеля.    Призначений командувачем збройними силами України і Криму, розбив Н.Махна та Ю.Тютюника, придушував селянські повстання. 

Хаджі-Гірей-засновник Кримського ханства у 1449 р., перший хан правлячої династії Гіреєв

Ханенко Богдан (1849—1917) Колекціонер української старовини і творів мистецтва, археолог, меценат. Провадив на власні кошти археологічні розкопи, видав збірник «Древности Поднепровья», був членом Археологічної комісії, Історичного товариства Нестора Літописця і інших наукових товариств. Член Державної ради. Зібрав велику колекцію творів мистецтва і бібліотеку. Зіграв головну роль у заснуванні Київського художньо-промислового й наукового музею.

Хвильовий (Фітільов) Микола (1894—1933) Письменник, публіцист, один із засновників Спілки пролетарських письменників «Гарт», Вільної академії пролетарської літератури (ВАПЛІТЕ), автор творів «Іван Іванович», «Ревізор», «Україна чи Малоросія», ініціатор літературної дискусії під гаслом «Геть від Москви» (1925—1928 рр.).

Хвойко Вікентій-український археолог, який у 1890-х рр. виявив перші пам’ятки трипільської культури

Хмельницький Юрій  - гетьман України 1659-1663. Уклав Переяславські статті (1659, Росія), Слободищенський трактат (1660, Річ Посполита), здійснив Чигиринські походи (1677, 1678). У 1667р. султан Магомет IV вручив йому булаву з титулом «князя Сарматії та України, володаря Війська Запорозького»

Цегельський Лонгин  1914—1918 роках був членом усіх провідних українських організацій у Галичині — Головної української ради, Загальної української ради, Бойової управи УСС, Союзу визволення України,  Української національної ради ЗУНР-ЗОУНР. Автор договіру з Директорією УНР про об'єднання ЗУНР і УНР в єдину Українську державу. Зачитав його 22 січня біля пам'ятника Богдану Хмельницькому у Києві.

Церетелев Микола 1819 р. збірка «Опит собрания старинніх малороссийских песен». Грузинський князь, дослідник і видавець українських народних дум. Понад десять років збирав і досліджував український фольклор.

Черняхівський Іван   (1906–1945) Радянський полководець, генерал армії, двічі Герой Радянського Союзу. З червня 1941 р. — на фронтах Великої Вітчизняної війни. У листопаді 1943 р. армія генерала взяла участь у Київській наступальнійоперації, в ході якої було звільнено столицю України. Надалі 60-та армія відзначилась у Житомирсько-Бердичівській, Рівненсько-Луцькій і Проскурово-Чернівецькій операціях.

Чехівський Володимир прем'єр-міністр УНР Директорії. Один із засновників УАПЦ.  Виступав за союз з більшовиками, проти Антанти.

Чикаленко Євген (1861—1929) Громадсько-політичний і культурний діяч, меценат, активний член «Старої громади», один із засновників УДП і Товариства українських поступовців, член Української Центральної Ради, фінансував видання українських газет «Громадська думка», «Рада».

Чингізхан-засновник монгольської держави у ХІІІ ст., великий монгольський хан

Чорновіл В’ячеслав (1937—1999) Український політичний і державний діяч, журналіст, 1970 створений журнал «Український вісник» до 1972 року вийшло 6 номерів. Правозахисник, член Української гельсінської групи (1976 р.), неодноразово засуджений за «наклепницьку діяльність на радянський суспільний лад»до ув’язнення, голова Народного Руху України, народний депутатнезалежної України, автор книг «Лихо з розуму», «Правосуддя чи рецидиви терору». 

Чубинський Павло дослідник української етнографії та фольклористики, автор віршів «Ще не вмерла України...». Під його керівництвом зібрано та видано громадівцями  7 томів матеріалів етнографічно-статистичної експедиції, проведеної в 1869–1871 рр. у Південно-Західному краї

Шашкевич Маркіян Греко-католицький священик. Член «Руської трійці». У1836 р. підготував підручник для молодших школярів – «Читанку», написану живою українською мовою.

Шевченко Тарас (1814–1861). У 1840 р. в Петербурзі «Кобзар», а 1841 р. –поема «Гайдамаки». 1843–1845 рр. збірка поезій «Три літа».

Шелест Петро (1908-1996) Український радянський і партійний діяч, представник націонал-комуністичної лінії у керівництві УРСР, перший секретар ЦК КП(б) У в 1963—1972 рр., відстоював економічні інтереси України, підтримував розвиток української мови й культури, став ініціатором запровадження Шевченківської премії.

Шептицький Андрей (1865 1944) Церковний, культурний та громадський діяч, митрополит Української греко-католицької церкви, будував храми; на власні кошти утримував Академічний дім, шпиталь, Національний музей; допомагав студентам, закладам освіти; -взяв дозвіл у папи на відкриття єпископств у Канаді й США.член Української Національної Ради ЗУНР, прихильник створення незалежної соборної Української держави. Засудив терористичну діяльність ОУН, але підтримав Акт відновлення Української держави 30 червня 1941 року. 

Шліхтер Олександр 1931 голова Всеукраїнської асоціації марксистсько-ленінських інститутів (ВУАМЛІН) 

Шраг Ілля - очільник Укранської громади у І Державній думі 1906 р., член Української Центральної Ради. Також культурний діяч, відомий підтримкою класиків української літератури. Ідеолог чернігівської «Просвіти».

Штейнгель Федір Депутат І Державної думи Російської імперії. Член ТУП,  1915 р. очолив Комітет Південно-Західного фронту Всеросійського союзу земств і міст, Генеральний секретар торгівлі й промисловості Центральної Ради. Посол Української Держави у Берліні. 

Штернберг Василь  одним з найближчих його друзів Т.Шевченка. Виконав офортне зображення сліпого кобзаря з хлопчиком-поводирем на фронтисписі Кобзаря 1840 р. Найкращими зразками пейзажу є картини  «Садиба Г. Тарновського в Качанівці» (1837), «Переправа через Дніпро під Києвом», «Вітряки в степу». 

Шумський Олександр (1890—1946) Радянський державний діяч, член КП(б)У, народний комісар освіти УСРР (1924—1927 рр.), прихильник політики українізації, звинувачений у «націоналістичному ухилі» (шумськізмі).

Шухевич Роман (псевдонім Тарас Чупринка) (1907-1950) Український військовий та політичний діяч. Генерал-хорунжий. Входив до галицького крайового проводу ОУН (Б). Дотримувався тактики революційних дій. Один з організаторів «Карпатської Січі». З 1942 р. — головнокомандувач УПА. З березня 1943 р. — військовий референт проводу ОУН. Голова Секретаріату Української Головної визвольної ради (УГВР 1944 ). У березні 1950 р. загинув у бою зі співробітниками радянських органів безпеки біля с. Білогорща, поблизу Львова. 

Щепкін Михайло  (1788–1863) актор харківської трупи, І.Котляревський ініціював викуп видатного актора з кріпацтва. Був провідним актором полтавської трупи і першим українським професійним актором, для якого І. Котляревський спеціально написав роль виборного Макогоненка («Наталка Полтавка») та козака Чупруна («Москаль-чарівник»)

Щербицький Володимир (1918–1990) Радянський і партійний діяч, представник проімперської лінії у керівництві УРСР, перший секретар ЦК КП(б)У в 1972—1989 рр., підпорядковував економічні інтереси України союзним потребам, сприяв русифікації, боровся з дисидентським рухом, намагався приховати Чорнобильську катастрофу.

Щорс Микола (1895–1919) Підпоручик царської армії, учасник Першої світової війни. З 1918 р. — член партії більшовиків. У лютому 1918 р. очолив червоногвардійський загін на Чернігівщині – Богунський полк, з осені 1918 р. – червоний командир 1-ї Української дивізії,  до складу якої, крім Богунського, входив також і Таращанський полк.  6 лютого 1919 року Богунський полк під командуванням Щорса узяв Київ.  

Юра Гнат (1888—1966) Український театральний режисер, актор театру і кіно. Засновник і режисер Державного драматичного театру ім. І.Франка. Народний артист УРСР, Народний артист СРСР (1940).

Юрій Дрогобич навчався в Кракові, став доктором та ректором Болонського університету. Написав "Прогностик 1483 року" - твір про розташування небесних тіл. Був професором медицини та мав титул "королівського лікаря".

Юрій І Львович 1301-1308 втратив Закарпаття, перейшло до Угорщини створив окрему Галицьку церковну митрополію.

Юрій ІІ Болеслав 1308-1340 останній князь з династії Романовичів, оточив себе іноземцями та католиками.

Ющенко Віктор (1954 р.н.) Державний і політичний діяч України, заступник голови правління Агробанку України, голова Національного банку України (1993 р.), перебував на посаді прем’єр-міністра України 1999—2001 рр., обирався Президентом України (2004 р.).

Яворницький Дмитро (1855—1940) Український історик, археолог, етнограф, дослідник історії українського козацтва, автор тритомної праці «Історія запорізьких козаків», укладач зібрання історичних джерел «До історії Степової України».

Ягайло -литовський князь, у часи правління якого відбулася Кревська унія 1385 р. - об’єднання. Великого Князівства Литовського та Королівства Польського на основі шлюбу литовського князя Ягайла та польської принцеси Ядвіги. Ягайло охрестився за католицьким обрядом

Ядвіга 1384-1399 королева Польщі, за Кревською унією вийшла заміж за литовського князя Ягайла, у 1387 р. приєднала Галичину.

Янукович Віктор (1950 р.н.) Український державний діяч, політик, Президент України (з 25 лютого 2010). 22 лютого 2014 року визнаний Верховною Радою України таким, що самоусунувся. Утім, сам Віктор Янукович досі вважає себе чинним главою держави. В.Януковича оголошено в розшук в рамках кримінальної справи за масові вбивства мирних громадян.

Ярослав Мудрий – 1019-1054 боровся з братами за київський стіл, правив разом з Мстиславом (поділ по Дніпру), розбив печенігів у 1036 р, побудував Софію Київську, уклав письмовий звід законів «Руська правда»

Ярослав Осмомисл 1153—1187 галицький князь, який об‘єднав всі галицькі землі, розквіт князівства. Згадується у «Слові о полку Ігоревім».

Ярославичі – триумвірат Ізяслав – Київ, Святослав – Чернігів, Всеволод – Переяслав. 1072 р. «Правда Ярославичів». 1068 р поразка від половців на р. Альті.


