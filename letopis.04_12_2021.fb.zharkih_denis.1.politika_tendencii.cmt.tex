% vim: keymap=russian-jcukenwin
%%beginhead 
 
%%file 04_12_2021.fb.zharkih_denis.1.politika_tendencii.cmt
%%parent 04_12_2021.fb.zharkih_denis.1.politika_tendencii
 
%%url 
 
%%author_id 
%%date 
 
%%tags 
%%title 
 
%%endhead 
\subsubsection{Коментарі}

\begin{itemize} % {
\iusr{Владимир Епифанов}
что значит альтернативный . и откуда миллиарды

\iusr{Людмила Осадчая-Комиссарчук}
\textbf{Владимир Епифанов} А миллиарды, вы видели. Состоятельный человек, но это не Петро!!!

\iusr{Александр Ста́лив}

Для начала следует учиться общаться по-человечески. Самое простое и действенное
- доброта. Вот и всё.

Всё остальное само собою естественно приложится в достатке.

А без доброты между собой - всё остальное впустую.

Неужели тридцать лет это не показали?

Думайте. Мозги включайте и думайте.

Вот оно выше написано, простое решение.

Вы его видите?

Жизнью движет доброта. Она - суть человечности.

Иначе невозможно.

Отраслевое хозяйствование (экономика) это доброта человечности.

Без доброты человечности всё остальное бессмысленно и впустую.

И чтобы настроить (запрограммировать) каждого человека на доброту, для этого
необходимо существенное время и, главное, меры и мероприятия. Их следует
продумывать и готовить.

А так всё - просто бла-бла-бла...

Я это здесь, в Фейсбук, писал десять лет назад.

И более двадцати лет назад - докладывал по месту тогдашней своей службы в СБУ
простой, доходчивый, убедительный проект реформы органов государственной
безопасности и одновременно всего отраслевого хозяйствования (экономики)
Украины.

Не дошло. Ни двадцать лет назад. Ни десять лет назад. Ни все эти годы. "А воз и
ныне там". Пока не сложилось. Всему своё место в своё время. Это главное
правило (норма) Вселенной и жизни в Ней.

Запомните: нарушая правила (нормы) законов естествознания, в том числе и и
правила (нормы) общения и отношений, - вы, каждый и все, ничего хорошего не
достигните. Видите же, что не достигли.

Думайте. Выбор за каждым.

Доброта движет жизнью. Это помните.

Жизненности вам, всем и каждому.

С особым почтением.

\begin{itemize} % {
\iusr{Александр Ста́лив}
Прочитайте внимательно:

\ii{link.04_12_2021.fb.staliv_aleksandr.1.upravljat_i_spravljatsja}

\end{itemize} % }

\iusr{Александр Ста́лив}

\ifcmt
  ig https://scontent-frt3-1.xx.fbcdn.net/v/t39.30808-6/262713301_4602793186481629_3881170254784353525_n.jpg?_nc_cat=106&ccb=1-5&_nc_sid=dbeb18&_nc_ohc=-tr0ZSG0o3sAX_rcH2i&_nc_ht=scontent-frt3-1.xx&oh=b44b21d653a51172e73f8e24bdf3cd0b&oe=61B01EF4
  @width 0.4
\fi

\iusr{Александр Ста́лив}

Медведчук не умеет подчиняться и, значит, он не справится. Он и не справлялся
никогда. Существование и деятельность ОПЗЖ - одно из насилий в Украине.
Диктатура. От диктатуры счастья нет. Вот и всё. Выше я написал о главном, без
которого всё остальное бессмысленно (без связи мыслей - без логики) и впустую.

\begin{itemize} % {
\iusr{Ирина Дубик}
\textbf{Александр Ста́лив} Ага, последние 8 лет одно процветание, какое там насилие.. Главное в управлении страной - это профессионализм, а доброта - это из другой оперы.

\iusr{Людмила Осадчая-Комиссарчук}
\textbf{Александр Ста́лив} ..
набор слов...
!

\iusr{Саша Тихилов}
Ты почти как сталин по форме. А вот по содержанию, наверно внучок бандеры. Угадал

\iusr{Александр Ста́лив}
\textbf{Ирина Дубик} Постарайтесь это понять, пригодится.

\ifcmt
  ig https://scontent-frt3-1.xx.fbcdn.net/v/t39.30808-6/262713301_4602793186481629_3881170254784353525_n.jpg?_nc_cat=106&ccb=1-5&_nc_sid=dbeb18&_nc_ohc=-tr0ZSG0o3sAX_rcH2i&_nc_ht=scontent-frt3-1.xx&oh=b44b21d653a51172e73f8e24bdf3cd0b&oe=61B01EF4
  @width 0.4
\fi

\iusr{Александр Ста́лив}
\textbf{Людмила Осадчая-Комиссарчук} Да, согласен, когда разучились или же не научились думать, то "набор слов".

\iusr{Александр Ста́лив}
\textbf{Саша Тихилов} Да, пожалуйста, как Вам угодно.

\iusr{Ирина Дубик}
\textbf{Александр Ста́лив} 

Постарайтесь понять истину, что доброта уместна при напевании колыбельной. А в
управлении страной исключительно профессионализм и мудрость. Я лучше пойду к
злому, но профессиональному хирургу, чем к просто доброму.

\iusr{Александр Ста́лив}
\textbf{Ирина Дубик} Там, где отсутствует доброта, там отсутствует мудрость и, стало быть, отсутствует и профессионализм.
Как же сочувствую родной Украине, насколько же отсутствует определённость жизни у населения. Поэтому и война, потому что зло "лучше", чем доброта. Это же надо так утратить и не приобрести определённость жизни.
Думайте, насколько Вы вообще потеряны для себя и общества с таким взглядом. Вокруг разруха. И она - следствие зла.
Всего доброго.

\iusr{Ирина Дубик}
\textbf{Александр Ста́лив} Слушайте, вы порядком надоели со своим "думайте" и нравоучениями. Увольте от вашего общества.
\end{itemize} % }

\iusr{Юрий Лукшиц}

Много слов, но мало конкретики. За всё хорошее против всего плохого. Конкретика
- это упразднение института президентства, региональный русский язык в Украине
через референдум, отмена декоммунизации, люстрации и насильственной
украинизации, внесение изменений в закон о децентрализации, Минские соглашения
на всеукраинский референдум и т.д. В предложениях Медведчука везде нужно
ставить вопросы "как", "каким образом?".

Восстановление законности - каким образом? Возврат к прагматизму - как?
Социальная справедливость - что в это вкладывается, какие законы, инициативы и
т.д.? При этом некоторые инициативы утопичны и не могут быть реализованы. Очень
странно будет выглядеть восстановление контактов с РФ на фоне военной
интервенции и аннексии Крыма. Скорее, восстановливать их можно будет после
урегулирования всех проблем между двумя государствами.

\iusr{Анатолий Старковский}

Хорошо сказано, "а пока мы в плену собственных иллюзий...". Украина и украинцы
вот уже 30 лет находятся в плену собственных иллюзий, что сытая, спокойная,
красивая жизнь на халяву в составе Союза, будет продолжаться и на вольных
хлебах незалежного существования, но что то пошло не так и оказалось, что
"житница" после 30 лет незалежности оказывается не в состоянии прокормить сама
себя, имея самые плодородные земли на планете. Иллюзии, иллюзии, иллюзии
незалежных на вечную халяву, по истечении 30 летнего незалежного существования
превратились в пшик потому, что оторвавшись от смачной российской титьки
халявщики оказались ни на что не способны ни управлять государством, ни
руководить огромным промышленным потенциалом, доставшимся от Союза и не смогли
сохранить агро промышленный комплекс. Увы и ах господа панове!


\iusr{Олег Хавич}
Медведчук платит за подобный бред - это воистину хорошая тенденция для райтеров!  @igg{fbicon.wink} 

\iusr{Анатолий Дранков}
Бред-Это продолжать жить как сейчас по лекалам запада.

\iusr{Константин Петухов}

Гладко было на бумаге, но забыли про овраги. Упоротых нацистов куда денете? Да,
их меньшинство. Но они вооружены, имеют боевой опыт и поддерживаются силами
которые стремятся втянуть Россию в войну. Думаете они просто так исчезнут? Нет.
Они будут защищаться и атаковать. Кто им будет противодействовать? Единственная
сила, которая это сможет сделать это ЛДНР. А это полномасштабная гражданская
война и развал государства на 3-4 части. Вот и получается, как ни крути, это
госудаство обречено... Точка невозврата пройдена...

\begin{itemize} % {
\iusr{Сергей Лобач}
\textbf{Константин Петухов} все эти вопросы можно было порешать в Беловежской пуще. Но увы.

\iusr{Lara Grigoryeva}
\textbf{Константин Петухов} 

и почему вы думаете, что нацистов меньшинство. Новое поколение, за 30 лет
независимости выросло по новой националистической идеологии, по новым
учебникам, им здорово промыли мозги и они с ясельного возраста точно знают, что
все беды от России

\iusr{Петр Вдович}
\textbf{Константин Петухов} ...верно, поэтому Медведчука и автору пора бы в своих публикациях исходить из другой реальности.
Предложение Медведчука дельные, но для думающих людей.
Бандерлогов меньшинство ..да, но в ихних руках все силовые органы, власть и помощь запада, а также что важно...образование.
Ихние ряды пополняться, в.то время как сторонников связи с Россией уменьшается.
Точка невозврата пройдена.
Это нужно понять и принять.

\end{itemize} % }

\iusr{Lara Kozhukhar}
С нацистами добротой не справиться! Их уже пожалел в свое время Хрущев, на нашу голову, расхлебываем теперь его доброту.....

\iusr{Юрий Бураков}
Денис, о чем хорошем, будем говорить, когда в едином государстве будем с Россией. Больше неё о чем

\iusr{Наталья Асуховская}
 @igg{fbicon.thumb.up.yellow}{repeat=4} 

\iusr{Владимир Гугняк}
Главное на выборах не забывать, что под нац. лозунгами грабят страну и каждого из нас. И по делам оценивать, а не словам верить.

\iusr{Сергей Адаменко}
Восстанавливать или нет правовое государство имеет полномочия только гражданин Янукович В. Ф.

\iusr{Wlad Sankin}

Благие пожелания. Деградация зашла слишком далеко, да так, что не Украина
"нормализируется", а уже и Запад "укранизируется". Вот в Сербии правда сейчас
экономический бум, ей похоже удаётся та пресловутая многовекторность. Наверное
потому что там и националистический сектор "пророссийский". А Украину разъела,
разложила изнутри бандеровщина. И для Украины не Сербия, а Хорватия с её
фашистским прошлым образец для подражания (с которой у России сейчас кстати
неплохие отношения). А связи с Россией у когда-то неразрывно братской Украины с
каждым годом безвозвратно теряются. В общем, Медведчук - это для историков,
чтобы на каком-нибудь уроке потом в школе какой-нибудь другой страны задать
вопрос - а были у Украины все-таки в тот момент альтернативные пути развития?


\iusr{Максим Мороз}

Без показательных посадок и расстрелов вся эта программа - очередная болтовня
за всё хорошее. Их то в программе и нет)


\iusr{Tanya Myakysheva}

Боюсь, что уже никто и ничего не спасёт Украину, Слишком далеко всё это зашло.
Страна распадется на несколько частей. И это лишь вопрос времени...


\iusr{Yuriy Storchak}

Иллюзии состоят в том, что ремонт государственности Украины как самостоятельной
единицы позволит ей начать успешно развиваться.

\iusr{Николай Гончаров}

Денис, два примера из Природы матушки. На фруктовое дерево напали вредители что
нужно делать? Вы на даче посадили грядку огурцов, её начала забивать сорная
трава, что вы будете делать? Так и Украина, была большое фруктовое дерево, на
которое напали вредители. Фамилии вредителей я не называю их все давно знают.

\iusr{Игорь Красиков}

Говорит-то он правильные вещи. Но беда в том, что тот же Зе ДО выбора говорил
правильные вещи, а ПОСЛЕ стал делать совершенно противоположные.

Почему вы считаете, что в политике есть другие люди? Тем более среди олигархов?
Просто эмоционально — "наш благородный разведчик выше их мерзкого шпиона"?...


\iusr{Кира Берестенко}

Он очень умен, образован, не беден - все это чуждо нашему обывателю, который в
большинстве, независимо от его убеждений.


\iusr{Женя Зубчевский}

Денис, я Вас уважаю, ну успокойтесь по поводу Медведчука, если вы не знаете или не
помните кем он был при Кучме, то я помню!!!!! они это начали и этим
закончили. Поезд давно ушёл, также как труса, и предателя Симоненко!!!

\iusr{Stella Patriks}
а где это выступление ?? покажите ..

\iusr{Gottfrid Lenz}

Небандеровская Украина - не нужна. Нет для Украины никакого смысла в том, чтобы
быть незалежной и небандеровской. Именно бандеровщина является смыслом
украинской незалежности, потому что в составе РФ она невозможна. А если Украина
небандеровская, то проект украинской незалежности закрывается, просто потому
что у него отсутствуют цели и задачи, и Украина становится Юго-Западным
федеральным округом.

Что касается господина Медведчука, то он точно такой же украинский националист
как и все остальные украинские "политики". Единственная разница между ним и
каким-нибудь Порошенко в том, что Порошенко хочет чтобы украинскую незалежность
оплачивал в основном Запад, а Медведчук хочет чтобы ее оплачивала в основном
РФ. Никакой другой разницы между ними нет. Соратник и патрон Медведчука
Янукович для незадежности сделал намного больше чем многие горлопаны в
вышиванках.

\iusr{Олег Дрога}

Медведчук искусный сказочник, таже энергетика что и у Путина, - вешай
лапшу, оболванивай - нас окружили и прикарманивай ресурсы!, - верить
человеку, создававшему с Кучмой олигархическую модель, это как выпить Пургену и
надеяться пукнуть, а по итогу понос @igg{fbicon.face.smirking} 

\iusr{Саша Тихилов}

За 30 лет незалежности из процветающей усср, страна превратилась в
квазигосударство бандерштадт. Разрушено практически все - машиностроение,
энергетика, инфраструктура, здравоохранение, образование, культура. Зато
огромные достижения в сфере русофобской пропаганды , здесь доктор Гёббельс мог
бы гордится своими идейными внучками. Перспектив у этой колониальной
территории нет абсолютно никаких. Произошли необратимые изменения и возврата к
союзу с остальными славянскими народами России и Белоруссии уже не будет.
Население не может из своей среды выдвинуть на пост руководителя страны
этнического славянина, пусть хоть он будет трижды внучок бандеры. Почему то
второй раз подряд выбирают ментально и этнически чуждого Украине джидая, это
говорит о глубоком кризисе этноса, такое невозможно себе представить в любой
мало- мальски уважающей себя стране, где есть чувство национальной гордости и
самосознания

\iusr{Noela Nola}

Совершенно верно, самая главная задача: восстановить правовое государство. Без
этого все остальное вообще невозможно.

\iusr{Александр Коняев}

На последних презедентских выборах так называемая оппозиция не смогла выдвинуть
единого кандидата и народ вынужден был голосовать за плесень лишь бы против
пороха


\end{itemize} % }
