% vim: keymap=russian-jcukenwin
%%beginhead 
 
%%file 25_12_2021.fb.fb_group.story_kiev_ua.2.pediatria
%%parent 25_12_2021
 
%%url https://www.facebook.com/groups/story.kiev.ua/posts/1826254187571374
 
%%author_id fb_group.story_kiev_ua,poljakova_galina.kiev
%%date 
 
%%tags istoria,kiev,medicina,pediatria,vrach
%%title Педіатрія в Києві
 
%%endhead 
 
\subsection{Педіатрія в Києві}
\label{sec:25_12_2021.fb.fb_group.story_kiev_ua.2.pediatria}
 
\Purl{https://www.facebook.com/groups/story.kiev.ua/posts/1826254187571374}
\ifcmt
 author_begin
   author_id fb_group.story_kiev_ua,poljakova_galina.kiev
 author_end
\fi

Вікна вагонів київського метро і вікна маршруток обліплені об’явами з
проханнями про допомогу хворим дітям. Ці заклики лунають в теленовинах, ми
бачимо їх в інтернеті. Сповнені болю і сподівання слова крають серце. 

\ii{25_12_2021.fb.fb_group.story_kiev_ua.2.pediatria.pic.1}

Педіатри. Їм ми довіряємо найдорожче, на них ми дивимося з надією. Дай їм,
Боже, здоров’я. Та й усім іншим лікарям. І побільше. Бо без них нам аж ніяк. 

Педіатрія зовсім не нова галузь медицини, проте не така вже й давня, як,
скажімо, хірургія з її славнозвісними цирульниками. Або ж військово-польова
хірургія, чиї підвалини були покладені ще у римських легіонах. Педіатри
з’явились значно пізніше. 

Перша в світі дитяча лікарня відкрилася 1802 року в Парижі. У 1846 році
невістка Миколая I, Фредеріка Шарлотта Марія Вюртемберзька, у заміжжі Олена
Павлівна, в пам’ять про чотирьох померлих маленьких донечках заснувала клініку
для  малолітніх дітей. Потім ще одна дитяча лікарня булі відкрита 1852 року в
Лондоні. А за три роки американці заснували у Філадельфії свою дитячу клініку.

А як же складалася ситуація в Києві? Справи теж потроху просувалися, проте не
так швидко, як бажалося. Потреба у створенні окремого курсу дитячих хвороб та
окремої кафедри була очевидною. Вчена Рада Університету Святого Володимира з
травня 1885 року терпляче і наполегливо зверталася до можновладців з
відповідним поданням. Нарешті 12 серпня 1889 надійшов довгоочікуваний лист за
номером 6300: 

«Пану Ректору Університету Св. Володимира.

Маю честь повідомити Вашу Високоповажність, що пан Міністр Народної Просвіти
пропозицією від 31 минулого липня за № 13112 призначив з 1 серпня цього року
Головного лікаря Дитячої лікарні Св. Ольги в Москві доктора медицини Чернова
екстраординарним професором університету Св. Володимира по кафедрі дитячих
хвороб»

Звичайно, кафедрі потрібна клініка, а надто для такої, адже професійна
підготовка таких фахівців неможлива без практичних занять. Проте на це не було
грошей! І допомога не забарилася: 1891 року власним коштом дружини міського
Голови Олени Петрівни Демідової (уродженої княжни Трубецької) була створена
дитяча клініка. А 1895 року було зібрано кошти на відкриття першого в місті
пологового будинку на базі Олександрівської лікарні.

«Попереду нього іде чеснота» - це про Лазара Ізраїлевича Бродського. Але ж який
він має стосунок до педіатрії? Безпосередній! Завдяки його щедрості в Києві
з’явилися університетська акушерська клініка, лікарня для хронічно хворих
дітей, протитуберкульозний санаторій, бактеріологічний інститут. Лазар
Бродський став одним з засновників і першим головою правління Товариства
боротьби з інфекційними хворобами. 

Пам’ятаємо з вдячністю! Тепер таких людей вже немає?

\ii{25_12_2021.fb.fb_group.story_kiev_ua.2.pediatria.cmt}
