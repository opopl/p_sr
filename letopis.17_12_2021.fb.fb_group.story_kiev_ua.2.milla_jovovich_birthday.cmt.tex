% vim: keymap=russian-jcukenwin
%%beginhead 
 
%%file 17_12_2021.fb.fb_group.story_kiev_ua.2.milla_jovovich_birthday.cmt
%%parent 17_12_2021.fb.fb_group.story_kiev_ua.2.milla_jovovich_birthday
 
%%url 
 
%%author_id 
%%date 
 
%%tags 
%%title 
 
%%endhead 
\zzSecCmt

\begin{itemize} % {
\iusr{Ольга Кирьянцева}

МЫ отмечаем? Спасибо большое! Чуть было не пропустила ТАКОЙ день!
@igg{fbicon.wink}  @igg{fbicon.face.grinning.smiling.eyes} 

\iusr{Вадим Горбов}
\textbf{Ольга Кирьянцева} в 16-00 нальём и выключим «5-й элемент».

\iusr{Осіпчук Микола}
Вітаємо з Днем Народження!!!

\iusr{Тамара Ар}

Ну, мы, простите, не отмечали такое событие! Одно дело день рождение знаменитых
людей Украины, сделавших для страны многое, список большой

\begin{itemize} % {
\iusr{Вадим Горбов}
\textbf{Тамара Ар} Прощаем.) Ваш список тоже имеет право на жизнь. У нас демократия и плюрализм мнений и героев.

\begin{itemize} % {
\iusr{Тамара Ар}
\textbf{Вадим Горбов} 

у нас демократия? Кажется, вы ошиблись,,,,,,,а список, да, большой, причём все
не только принесли свой талант, славу, труд (не правда ли, забытые понятия?) в
свою страну, но и известны заграницей

\iusr{Тамара Ар}
\textbf{Вадим Горбов} наш список живёт в делах до сих пор, к примеру, работа Патона, вы ж по мосту ездите, скорее всего

\iusr{Вадим Горбов}
\textbf{Тамара Ар} 

Вам меня не переспорить. За границей все знают знаменитого врача киевлянина
Владимира Зеленко. А вы о нем даже не слышали. И Даяна Шмуэля и Самуэля Крамера
и Финланда Максвелла, все они великие уроженцы одного маленького городка. А вы
всё продолжаете рассказывать про Труд и Славу.

\iusr{Тамара Ар}
\textbf{Вадим Горбов} 

зачем мне перечислять тех известных талантливых людей, чьи день рождения сейчас
не празднуют?,,,,,,,,,,,пожалуйста, отмечайте, \enquote{Обитель зла} вам в помощь!

\ifcmt
  ig https://scontent-frx5-1.xx.fbcdn.net/v/t39.1997-6/s168x128/65857085_2295443063867936_2186483780703748096_n.png?_nc_cat=105&ccb=1-5&_nc_sid=ac3552&_nc_ohc=HLdRKcC3NAEAX87Yo_B&_nc_ht=scontent-frx5-1.xx&oh=00_AT_QmLazFXJvwakaQeKtQjzQqYudIj8kSvlbLJXM4QicGg&oe=61D1A860
  @width 0.1
\fi

\iusr{Тамара Ар}
\textbf{Вадим Горбов} Спокойных выходных!

\iusr{Вадим Горбов}
\textbf{Тамара Ар} Леонида Броневого отмечайте. Я только за.

\iusr{Тамара Ар}
\textbf{Вадим Горбов} Труд не в почете, единственный новый мост никак не достроят!
\end{itemize} % }

\iusr{Тамара Ар}
\textbf{Віта Ілейко} У вас?

\end{itemize} % }

\iusr{Тамара Ар}
Врочем, известной американской актрисе Йовович желаем здоровья

\iusr{Светлана Наливайчук}

З днем народження вітаємо! Будемо знати, що народилася в один день з моїм
чоловіком  @igg{fbicon.smile} 

\iusr{Виктория Лунная}
Серьезно??!! Все киевляне отмечают такой важный праздник?

\begin{itemize} % {
\iusr{Вадим Горбов}
\textbf{Виктория Лунная} Где в тексте написано что все киевляне??!! Вы персонально можете отмечать День Варвары или Шабат. Мне это безразлично, у нас демократия.

\iusr{Виктория Лунная}
\textbf{Вадим Горбов}  @igg{fbicon.face.tears.of.joy} 

\iusr{Ольга Кирьянцева}
\textbf{Вадим Горбов} скорее бы уже учредили праздник Человеков с ЧЮ. Вот напьёмсуууу  @igg{fbicon.thinking.face}  @igg{fbicon.wink} 
\end{itemize} % }

\iusr{Анатолий Габдрахимов}
Как говорили у нас раньше на Лукьяновской и Валах: \enquote{Ну и что?}

\begin{itemize} % {
\iusr{Вадим Горбов}
\textbf{Анатолий Габдрахимов} Сейчас на Лукьяновской говорят: Салям Алейкум, а на Валах - шалом. Ну и что?

\begin{itemize} % {
\iusr{Анна Дубницкая}
\textbf{Вадим Горбов} Ну на Лук'янівській ще кажуть: \enquote{Доброго ранку,дня} і т.д. Ну і шо?

\iusr{Вадим Горбов}
\textbf{Анна Дубницкая} Это вы там на кладбище собак выгуливаете?

\iusr{Анна Дубницкая}
\textbf{Вадим Горбов} У нас немає собак, ми котів любимо. А на кладовищі не гуляємо, ходимо на гору.

\iusr{Вадим Горбов}
\textbf{Анна Дубницкая} я це мав на увазі. Через цвинтар на гору. Сам я туди ходжу за краєвидами.

\iusr{Анна Дубницкая}
\textbf{Вадим Горбов} Краєвиди гарні.

\iusr{Вадим Горбов}
\textbf{Анна Дубницкая} і єнергетика
\end{itemize} % }

\end{itemize} % }

\iusr{Ирина Середа}
Замечательная,талантливая,красавица и очень добрая

\iusr{Dimitri Statnikov}
Что она может помнить уехала отсюда маленькой девочкой..

\iusr{Сергей Константинович}
Я предлагаю этот день сделать выходным, такое событие нельзя оставить без внимания

\begin{itemize} % {
\iusr{Вадим Горбов}
\textbf{Сергей Константинович} Такая ваша петиция должна набрать 25 тысяч подписей и ее рассмотрят.

\iusr{Сергей Константинович}
\textbf{Вадим Горбов} уже подал

\iusr{Лариса Кушниренко}
\textbf{Сергей Константинович} Можно и улицу назвать ее именем. @igg{fbicon.face.happy.two.hands} 

\iusr{Сергей Константинович}
\textbf{Лариса Кушниренко} зачем улицу, надо назвать район,

\iusr{Лариса Кушниренко}
\textbf{Сергей Константинович} Ой! Это уже занадто. @igg{fbicon.face.happy.two.hands} 
\end{itemize} % }

\iusr{Анна Серебрякова}

\ifcmt
  ig https://scontent-frt3-1.xx.fbcdn.net/v/t39.30808-6/268355114_1850449858675788_756199488991853240_n.jpg?_nc_cat=106&ccb=1-5&_nc_sid=dbeb18&_nc_ohc=2atdXCkWRLkAX8vzNvW&_nc_oc=AQn9VP5TbtyLdC3dhs5J6qf9hmE04jn73bjT6tX3pDuMvjyWf2kKjpowX1rfkHLt2k0&_nc_ht=scontent-frt3-1.xx&oh=00_AT9p4K2aFbWKM390PjcoocRUsNPHkC68SSSmgGT9nsEv3Q&oe=61D107B5
  @width 0.3
\fi

\iusr{Игорь Гаврилов}
Жили на Русановке. Сама Йович рассказывала.

\begin{itemize} % {
\iusr{Вадим Горбов}
\textbf{Игорь Гаврилов} Топ-блогер Вишняков и его команда SMM не могут ошибаться. Сегодня они написали в Фейсбуке , что на улице Богомольца. )

\begin{itemize} % {
\iusr{Тамара Ар}
\textbf{Вадим Горбов} Следующая версия - на Борщаговке

\iusr{Вадим Горбов}
\textbf{Тамара Ар} Софиевской

\iusr{Ксения Погорлецкая}
\textbf{Вадим Горбов} Конкретизировать номер дома в котором в ТЕ годы можно арендовать квартиру они не хотят? Кол-во домов на Богомольца можно по пальцам пересчитать, в части из них коммуналки были. Нужно быть ОЧЕНЬ далеким от реальности и жизни на близлежащей улице чтобы такое писать.
\end{itemize} % }

\end{itemize} % }

\iusr{Vilyam Safyan}

Не считаю ее выдающейся актрисой, мягко говоря... Кроме того, не слышал чтоб она
или ее мама поддержали Украину во время войны... Отмечайте, ваше дело, а мы не
будем..))

\begin{itemize} % {
\iusr{Александр Иванович Лавров}
\textbf{Vilyam Safyan} 

В боевиках она играет очень здорово! Я не очень люблю этот жанр, но когда Милла...
- смотрю, ведь она наша.

\iusr{Vilyam Safyan}
\textbf{Александр Иванович Лавров} 

Кроме того что не смотрю боевики, не считаю ее нашей... Уверен, и она так не
считает тоже..))))


\iusr{Valerii Popilevych}
вам так только кажется. она поддержала Украину

\begin{itemize} % {
\iusr{Vilyam Safyan}
\textbf{Valerii Popilevych} 

Да? Есть какие то доказательства? Извинюсь с удовольствием... Реально поддержали
Украину Макаревич, например, или Ахеджакова... Это очевидно.. Из западных звёзд
вот Шон Пенн недавно приезжал, слышал что даже был на передовой.... Больше не
знаю примеров... К сожалению....

\iusr{Valerii Popilevych}

я запомнил что она тогда высказывалась в фейсбуке или в инсте уже не помню
говорила что она с Украиной в ети дни. помню она во времена Майдана тоже
поддерживала. говорила что он с народом. так что не надо ля ля

\iusr{Svetlana Kozlova}
\textbf{Valerii Popilevych} сейчас она чудно поменяла позицию.

\iusr{Valerii Popilevych}
\textbf{Svetlana Kozlova} хм... а почему Вы так решили??
\end{itemize} % }

\end{itemize} % }

\iusr{Iryna Tymoshok}
Українка!!! Чудова...

\begin{itemize} % {
\iusr{Олег Курилов}
\textbf{Iryna Tymoshok} 

Ивините что поправляю Вас. Но во всех интервью на данную тему, она говорила,
что не считает себя украинкой от слова совсем. Просто уточнил. А то что она -
Чудова, полностью с Вами согласен  @igg{fbicon.smile} 

\begin{itemize} % {
\iusr{Alex Maler}
\textbf{Олег Курилов} якби була українкою, то б ходила у віночьку прісядкою!

\iusr{Iryna Tymoshok}
\textbf{Олег Курилов} 

я не гонюсь за буквальностью, про она родилась в Киеве, я слышала, как она
пыталась петь песни на украинском, вот и приятно называть ее
украиночкой... просто люблю ее и не претендую на...


\iusr{Тетяна Водоріз}
\textbf{Alex Maler} Такой дешевый сарказм

\iusr{Олег Курилов}
\textbf{Iryna Tymoshok} 

Я вас понял, просто не все родившиеся в Киеве, автоматически становяться
украинцами.  @igg{fbicon.smile} . Мила, если не ошибаюсь, как раз именно пела на украинском, но
на подобный вопрос о национальности она однозначно отвечала, что она считает
себя русской. Хотя... я считаю это совсем не важно, кто там по национальности.
главное что вам этот человек нравиться! Кроме того её можно назать землячкой  @igg{fbicon.smile} 

\end{itemize} % }

\iusr{Alex Maler}
\textbf{Тетяна Водоріз} вибачайте, меншовартість

\end{itemize} % }

\iusr{Iryna Tymoshok}
 @igg{fbicon.hand.ok}{repeat=3} 

\iusr{Александр Брушко}

Другие источники сообщают, что до отъезда в США семья жила в Киеве на
Борщаговке ул. Ромена Роллана 17.....

\begin{itemize} % {
\iusr{Евгений Скрипка}
\textbf{Александр Брушко} , может Гавела улица)) ану гляньте на карте

\iusr{Александр Брушко}
\textbf{Евгений Скрипка} Тогда была Ромена .. сейчас Кольцова 17возле 131школы

\iusr{Ксения Погорлецкая}
\textbf{Александр Брушко} Это больше похоже на правду чем на Богомольца.
\end{itemize} % }

\iusr{Наталья Калмыкова}
Ох, какие вы все ядовитые!

\iusr{Игорь Ялоза}

\ifcmt
  ig https://i2.paste.pics/a1b8330637b279957fddfeb7bf3d8769.png
  @width 0.2
\fi

\iusr{Игорь Ялоза}

\ifcmt
  ig https://i2.paste.pics/6545cd6e7398d4d9b7b888820f00818d.png
  @width 0.2
\fi

\iusr{Павел Левченко}
УМНИЦА, КРАСАВИЦА,
СПАСИБО РОДИТЕЛЯМ! !!

\iusr{Tatiana Plachotnaya}
А мама успела имя поменять на Галину?

\iusr{Леся Фандралюк}

Пардон, но мой отец дружил с отцом Миллы. Детство она провела на улице Ромена
Роллана

\begin{itemize} % {
\iusr{Tatiana Tania}
\textbf{Леся Фандралюк} на Борщаговке получается.

\iusr{Леся Фандралюк}
\textbf{Tatiana Tania} Да, кажется в доме 17или, возможно, 18
\end{itemize} % }

\iusr{Ирина Шусть}
Вяликая киевлянка @igg{fbicon.face.grinning.smiling.eyes}{repeat=3} 

\begin{itemize} % {
\iusr{Татьяна Данилова}
\textbf{Ирина Шусть} В вашем возрасте зависть и злость - плохое качество! Морщин будет больше!

\iusr{Ирина Шусть}
\textbf{Татьяна Данилова}
Никакой зависти!
Видимо, она свойственна вам?
Просто, с неё киевлянка, как с меня амстердамка!
\end{itemize} % }

\iusr{Юрій Кіптілий}
Зовсім здуріли

\iusr{Татьяна Данилова}
\textbf{Юрій Кіптілий}

\ifcmt
  ig https://scontent-frt3-1.xx.fbcdn.net/v/t39.1997-6/p480x480/17632925_2041012122792928_7296182939413381120_n.png?_nc_cat=108&ccb=1-5&_nc_sid=0572db&_nc_ohc=Bw9mLT_i9gAAX9Ewur8&_nc_ht=scontent-frt3-1.xx&oh=00_AT8EpRustBlQPCsNUk4K1rVQv8_J6T7In-jD3I4RQ8Gmig&oe=61D152FD
  @width 0.2
\fi

\iusr{Александр Муратов}
А Богдана Йововича я знал как студентк Киевског мед. имнститута. Мы,
преподаватели знали, что он из Югославии.

\begin{itemize} % {
\iusr{Alex Maler}
\textbf{Александр Муратов} ...и понимали его по-югославски...

\iusr{Slava Fuksman Klyashtorny}
\textbf{Александр Муратов} так он и был из Югославии.

\iusr{Татьяна Ушакова}
\textbf{Александр Муратов} Я училась с ним в одной группе

\iusr{Александр Муратов}

Чна занятияхс группами с ним не втречался, но встречл в Мрофкрпусе вне заняий,
особо не контактовал. Намомните, пожалуйста, в какие годы, и на каком факульте
в с нм учились. Я помню, что это было давно, но помню.

\end{itemize} % }

\iusr{Ирина Костюченко}
Вы ещё и улицу в честь неё назовите

\begin{itemize} % {
\iusr{Татьяна Данилова}
\textbf{Ирина Костюченко} 

А плохое качество для женщины зависть и злость на весь мир!
@igg{fbicon.thumb.down.yellow}  @igg{fbicon.face.womiting} 

\iusr{Наталия Галицкая}
\textbf{Татьяна Данилова} 

при чем здесь зависть ? Надо гордится своим народом, а не притягивать за уши
всех, кто живёт за рубежом, но этнические украинцы. А наши СМИ послушать то,
все в мире придумали украинцы и колесо и т. д. и тому подобное. И вообще Иисус
Христос был гуцулом. Так, что не в злости и зависти дело. А если вы
празднуете День Рождения Милы, то это ваше личное дело.

\iusr{Ирина Костюченко}
\textbf{Татьяна Данилова} причём тут зависть?

\iusr{Дима Бузовский}
\textbf{Ирина Костюченко} Есть давно, Милославская в честь ее
\end{itemize} % }

\iusr{Вадим Сандино}

Гвинтокрил, Пятый элемент, Черный квадрат... Но пришел гегемон и имеем мост и
метро на Троещину каждые выборы!

\iusr{Олена Медведева- Прицкер}

Очаровательная женщина Милла Йовович, известная актриса в Голливуде. Пожелаем
ей успехов и новых ролей. Так хочется пойти в кино и посмотреть что- то
душевное.

\begin{itemize} % {
\iusr{Slava Fuksman Klyashtorny}
\textbf{Olena Medvedyeva-Pritsker} в Голливуде таких, как она, сотни.

\begin{itemize} % {
\iusr{Yuriy Austin}
\textbf{Slava Fuksman Klyashtorny} 

Нет, это не правда. Милка очень своеобразная спортивная и удивительно красивая.
Таких и на всём земном шаре единицы, не говоря уже о небольшой голливудской
тусовке.


\iusr{Slava Fuksman Klyashtorny}
\textbf{Yuriy Austin} 

это правда. Ее уже очень давно не снимают в значительных ролях в Голливуде.
Собственно говоря, в значительных ролях никогда и не снимали. Она с подругой
попыталась открыть дизайнерское ателье по пошиву женской одежды, но это у них
не получилось. Говорить о том, что в мире мало таких, вообще смешно. И в
голливудскую небольшую тусовку стремятся таланты со всего мира.


\iusr{Yuriy Austin}
\textbf{Slava Fuksman Klyashtorny} 

Уроды там, сейчас, в Голливуде. Даже разговаривать о них противно. Ну кроме
Клинта Иствуда и Вогта, который папа Джоли. Даже моя любимая прежде Сигурни
несёт такую херню, что и её слушать противно. Я не говорил что Милка хорошая
актриса. Я писал что она красивая. Ну а сейчас ей 46-ть, это очень плохой
возраст для киноактрисы.

\end{itemize} % }

\end{itemize} % }

\iusr{Нина Харченко}

Галины Логиновой @igg{fbicon.laugh.rolling.floor}{repeat=3}, автор явно ошибся...

\iusr{Нина Харченко}

Ещё и была моей соседкой @igg{fbicon.laugh.rolling.floor}{repeat=3}, я прожила
на Богомольца 5 всю жизнь почти..

\iusr{Evg Prol}
Та ну її! Російським змі говорить що вона русская, українським - українка

\begin{itemize} % {
\iusr{Олег Курилов}
\textbf{Evg Prol} да ну, и второе сможете доказать?  @igg{fbicon.smile}
Возможно вы немного ошиблись... или я что то пропустил....

\begin{itemize} % {
\iusr{Evg Prol}
\textbf{Олег Курилов} неумение искать информацию очень типично для подставок для красных шариков. Ютуб: Milla Jovovich in Ukraine 2

\iusr{Олег Курилов}
\textbf{Evg Prol} 

Посмотрел и? Там же была шутка! Вы же видели в каком контексте это говорилось!
Это всё что смогли найти? Печально. 

Вот навскидку: 

Милла: \enquote{Меня воспитывала
русская мама, она была знаменитой актрисой в СССР. Первые книги, которые я
прочитала, были на русском языке. Я воспитывалась в атмосфере русской
классической театральной школы. И современное кино стоит на этих же принципах.
Кроме того, я всегда помню, что в моих жилах течет русская кровь}. 

На вопрос французского издания Purple о том, чувствует ли она себя русской,
выехав из России в возрасте 5 лет, Милла ответила: \enquote{Разумеется. Я по-прежнему
общаюсь на русском. Разговариваю на нем со своей дочерью. Читаю ей русские
стихи и рассказы. Мои корни очень важны - они делают меня той, кто я есть}. 

\enquote{Я - русская. Я выросла у русских родителей, у меня много русских друзей, моя лучшая
подружка - русская. Я понимаю, что выросла в Америке, - но всё равно чувствую
себя русской... и, конечно, мне Россия очень близка и дорога. Я люблю русский
юмор, русскую кухню, русские песни, русские церкви}. 

И вот интервью УКРАИНСКОЙ версии журнала Vogue Милла сказала, что не считает
себя украинкой: \enquote{Когда на съемке Оля спросила Миллу, ощущает ли она
себя украинкой, актриса моментально протянула: \enquote{Не-е-ет!}} Не
соглашается с теми газетами, которые назвали Миллу \enquote{украинской
актрисой}, и ее мать, актриса кино Галина Логинова: \enquote{Мои родители -
русские, из Тулы. Милла - полурусская-получерногорка, но, я думаю, по своему
происхождению она считает себя русской}. 

В 2017 году Милла отметила в одном из интервью, что ее рождение на Украине было
\enquote{случайным}. Попробуйте такое же найти в другом смысле. Я не смог. А то, что вы
привели это убого. И выдавать шутку, за чистую монету, как это показательно для
порохоботиков.  @igg{fbicon.frown}  По этому не стоит на Миллу навешивать ярлыки.


\iusr{Evg Prol}
\textbf{Олег Курилов} не удивлен ответом. Но повторяться не буду. Sapienti, как говорится, sat.

\iusr{Олег Курилов}
\textbf{Evg Prol} Что и требовалось доказать. Аргументов у вас нет. Кому надо сами откроют и посмотрят, что за видео Вы дали, и в каком там контексте и как там в виде шутки всё это подавалось. Желаю здоровья!
\end{itemize} % }

\iusr{Валентина Прибутько}
\textbf{Evg Prol} а як що це і так? То чому ну її

\begin{itemize} % {
\iusr{Evg Prol}
\textbf{Валентина Прибутько} 

Тому що насправді вона себе ідентифікує як росіянку. У в Україні називала себе
українкою щоб сподобатися українцям. Вона любить росію, вона їй \enquote{дуже близька і
дорога}. Почитайте її цитати у вікі.

\end{itemize} % }


\end{itemize} % }

\iusr{Alla Kenya}

\ifcmt
  ig https://i2.paste.pics/1674264ba1c97c234f950e59d65c5b28.png
  @width 0.2
\fi

\iusr{Марина Ткаченко}
Соседка

\iusr{Галина Копыл}
В тексте неточность. Не Татьяны Логиновой, а Галины.

\end{itemize} % }
