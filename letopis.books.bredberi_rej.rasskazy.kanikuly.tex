% vim: keymap=russian-jcukenwin
%%beginhead 
 
%%file books.bredberi_rej.rasskazy.kanikuly
%%parent books.bredberi_rej.rasskazy
 
%%url 
 
%%author_id 
%%date 
 
%%tags 
%%title 
 
%%endhead 

\section{Каникулы}

День был свежий — свежестью травы, что тянулась вверх, облаков, что плыли в
небесах, бабочек, что опускались на траву. День был соткан из тишины, но она
вовсе не была немой, ее создавали пчелы и цветы, суша и океан, все, что
двигалось, порхало, трепетало, вздымалось и падало, подчиняясь своему течению
времени, своему неповторимому ритму. Край был недвижим, и все двигалось. Море
было неспокойно, и море молчало. Парадокс, сплошной парадокс, безмолвие
срасталось с безмолвием, звук со звуком. Цветы качались, и пчелы маленькими
каскадами золотого дождя падали на клевер. Волны холмов и волны океана, два
рода движения, были разделены железной дорогой, пустынной, сложенной из
ржавчины и стальной сердцевины, дорогой, по которой, сразу видно, много лет не
ходили поезда. На тридцать миль к северу она тянулась, петляя, потом терялась в
мглистых далях; на тридцать миль к югу пронизывала острова летучих теней,
которые на глазах смещались и меняли свои очертания на склонах далеких гор.

Неожиданно рельсы задрожали.

Сидя на путях, одинокий дрозд ощутил, как рождается мерное слабое биение,
словно где-то, за много миль, забилось чье-то сердце.

Черный дрозд взмыл над морем.

Рельсы продолжали тихо дрожать, и наконец из-за поворота показалась, вдоль по
берегу пошла небольшая дрезина, в великом безмолвии зафыркал и зарокотал
двухцилиндровый мотор.

На этой маленькой четырехколесной дрезине, на обращенной в две стороны двойной
скамейке, защищенные от солнца небольшим тентом, сидели мужчина, его жена и
семилетний сынишка. Дрезина проходила один пустынный участок за другим, ветер
бил в глаза и развевал волосы, но все трое не оборачивались и смотрели только
вперед. Иногда, на выходе из поворота, глядели нетерпеливо, иногда печально, и
все время настороженно — что дальше?

На ровной прямой мотор вдруг закашлялся и смолк. В сокрушительной теперь тишине
казалось — это покой, излучаемый морем, землей и небом, затормозил и пресек
вращение колес.

— Бензин кончился.

Мужчина, вздохнув, достал из узкого багажника запасную канистру и начал
переливать горючее в бак.

Его жена и сын тихо глядели на море, слушали приглушенный гром, шепот, слушали,
как раздвигается могучий занавес из песка, гальки, зеленых водорослей, пены.

— Море красивое, правда? — сказала женщина.

— Мне нравится, — сказал мальчик.

— Может быть, заодно сделаем привал и поедим?

Мужчина навел бинокль на зеленый полуостров вдали.

— Давайте. Рельсы сильно изъело ржавчиной. Впереди путь разрушен. Придется
ждать, пока я исправлю.

— Сколько лопнуло рельсов, столько привалов! — сказал мальчик.

Женщина попыталась улыбнуться, потом перевела свои серьезные, пытливые глаза на
мужчину.

— Сколько мы проехали сегодня?

— Неполных девяносто миль. — Мужчина все еще напряженно глядел в бинокль. —
Больше, по-моему, и не стоит проходить в день. Когда гонишь, не успеваешь
ничего увидеть. Послезавтра будем в Монтерее, на следующий день, если хочешь, в
Пало Альто.

Женщина развязала ярко-желтые ленты широкополой соломенной шляпы, сняла ее с
золотистых волос и, покрытая легкой испариной, отошла от машины. Они столько
ехали без остановки на трясучей дрезине, что все тело пропиталось ее ровным
ходом. Теперь, когда машина остановилась, было какое-то странное чувство,
словно с них сейчас снимут оковы.

— Давайте есть!

Мальчик бегом отнес корзинку с припасами на берег. Мать и сын уже сидели перед
расстеленной скатертью, когда мужчина спустился к ним; на нем был строгий
костюм с жилетом, галстук и шляпа, как будто он ожидал кого-то встретить в
пути. Раздавая сэндвичи и извлекая маринованные овощи из прохладных зеленых
баночек, он понемногу отпускал галстук и расстегивал жилет, все время озираясь,
словно готовый в любую секунду опять застегнуться на все пуговицы.

— Мы одни, папа? — спросил мальчик, не переставая жевать.

— Да.

— И больше никого, нигде?

— Больше никого.

— А прежде на свете были люди?

— Зачем ты все время спрашиваешь? Это было не так уж давно. Всего несколько
месяцев. Ты и сам помнишь.

— Плохо помню. А когда нарочно стараюсь припомнить, и вовсе забываю. — Мальчик
просеял между пальцами горсть песка. — Людей было столько, сколько песка тут на
пляже? А что с ними случилось?

— Не знаю, — ответил мужчина, и это была правда.

В одно прекрасное утро они проснулись и мир был пуст. Висела бельевая веревка
соседей, и ветер трепал ослепительно белые рубашки, как всегда поутру блестели
машины перед коттеджами, но не слышно ничьего «до свидания», не гудели уличным
движением мощные артерии города, телефоны не вздрагивали от собственного
звонка, не кричали дети в чаще подсолнечника.

Лишь накануне вечером он сидел с женой на террасе, когда принесли вечернюю
газету, и даже не развертывая ее, не глядя на заголовки, сказал:

— Интересно, когда мы ему осточертеем и он всех нас выметет вон?

— Да, до чего дошло, — подхватила она. — И не остановишь. Как же мы глупы,
правда?

— А замечательно было бы… — Он раскурил свою трубку. — Проснуться завтра, и во
всем мире ни души, начинай все сначала!

Он сидел и курил, в руке сложенная газета, голова откинута на спинку кресла.

— Если бы можно было сейчас нажать такую кнопку, ты бы нажал?

— Наверно, да, — ответил он. — Без насилия. Просто все исчезнет с лица земли.
Оставить землю и море, и все что растет — цветы, траву, плодовые деревья. И
животные тоже пусть остаются. Все оставить, кроме человека, который охотится,
когда не голоден, ест, когда сыт, жесток, хотя его никто не задевает.

— Но мы-то должны остаться. — Она тихо улыбнулась.

— Хорошо было бы. — Он задумался. — Впереди — сколько угодно времени. Самые
длинные каникулы в истории. И мы с корзиной припасов, и самый долгий пикник.
Только ты, я и Джим. Никаких сезонных билетов.

Не нужно тянуться за Джонсами. Даже автомашины не надо. Придумать какой-нибудь
другой способ путешествовать, старинный способ. Взять корзину с сэндвичами, три
бутылки шипучки, дальше, как понадобится, пополнять запасы в безлюдных
магазинах в безлюдных городах, и впереди нескончаемое лето…

Долго они сидели молча на террасе, их разделяла свернутая газета.

Наконец она сказала:

— А нам не будет одиноко?

Вот каким было утро нового мира. Они проснулись и услышали мягкие звуки земли,
которая теперь была просто-напросто лугом, города тонули в море травы-муравы,
ноготков, маргариток, вьюнков. Сперва они приняли это удивительно спокойно,
должно быть потому, что уже столько лет не любили город и позади было столько
мнимых друзей, и была замкнутая жизнь в уединении, в механизированном улье.

Муж встал с кровати, выглянул в окно и спокойно, словно речь шла о погоде,
заметил:

— Все исчезли.

Он понял это по звукам, которых город больше не издавал.

Они завтракали не торопясь, потому что мальчик еще спал, потом муж выпрямился и
сказал:

— Теперь мне надо придумать, что делать.

— Что делать? Как… разве ты не пойдешь на работу?

— Ты все еще не веришь, да? — Он засмеялся. — Не веришь, что я не буду каждый
день выскакивать из дому в десять минут девятого, что Джиму больше никогда не
надо ходить в школу. Всё, занятия кончились, для всех нас кончились! Больше
никаких карандашей, никаких книг и кислых взглядов босса! Нас отпустили, милая,
и мы никогда не вернемся к этой дурацкой, проклятой, нудной рутине. Пошли!

И он повел ее по пустым и безмолвным улицам города.

— Они не умерли, — сказал он. — Просто… ушли.

— А другие города?

Он зашел в телефонную будку, набрал номер Чикаго, потом Нью-Йорка, потом Сан-
Франциско. Молчание. Молчание. Молчание.

Все, — сказал он, вешая трубку.

— Я чувствую себя виноватой, — сказала она. — Их нет, а мы остались. И… я
радуюсь. Почему? Ведь я должна горевать.

— Должна? Никакой трагедии нет. Их не пытали, не жгли, не мучали. Они исчезли и
не почувствовали этого, не узнали. И теперь мы ни перед кем не обязаны. У нас
одна обязанность — быть счастливыми. Тридцать лет счастья впереди, разве плохо?

— Но… но тогда нам нужно заводить еще детей?

— Чтобы снова населить мир? — Он медленно, спокойно покачал головой. — Нет.
Пусть Джим будет последним. Когда он состарится и умрет, пусть мир принадлежит
лошадям и коровам, бурундукам и паукам Они без нас не пропадут. А потом когда-
нибудь другой род, умеющий сочетать естественное счастье с естественным
любопытством, построит города, совсем не такие, как наши, и будет жить дальше.
А сейчас уложим корзину, разбудим Джима и начнем наши тридцатилетние каникулы.
Ну, кто первым добежит до дома?

Он взял с маленькой дрезины кувалду, и пока он полчаса один исправлял ржавые
рельсы, женщина и мальчик побежали вдоль берега. Они вернулись с горстью
влажных ракушек и чудесными розовыми камешками, сели, и мать стала учить сына,
и он писал карандашом в блокноте домашнее задание, а в полдень к ним спустился
с насыпи отец, без пиджака, без галстука, и они пили апельсиновую шипучку,
глядя, как в бутылках, теснясь, рвутся вверх пузырьки. Стояла тишина. Они
слушали, как солнце настраивает старые железные рельсы. Соленый ветер разносил
запах горячего дегтя от шпал, и мужчина легонько постукивал пальцем по своему
карманному атласу.

— Через месяц, в мае, доберемся до Сакраменто, оттуда двинемся в Сиэтл.
Пробудем там до первого июля, июль хороший месяц в Вашингтоне, потом, как
станет холоднее, обратно, в Йеллоустон, несколько миль в день, здесь
поохотимся, там порыбачим…

Мальчику стало скучно, он отошел к самой воде и бросал палки в море, потом сам
же бегал за ними, изображая ученую собаку.

Отец продолжал:

— Зимуем в Таксоне, в самом конце зимы едем во Флориду, весной — вдоль
побережья, в июне попадем, скажем, в Нью-Йорк. Через два года лето проводим в
Чикаго. Через три года — как ты насчет того, чтобы провести зиму в Мехико-Сити?
Куда рельсы приведут, куда угодно, и если нападем на совсем неизвестную старую
ветку — превосходно, поедем по ней до конца, посмотрим, куда она ведет. Когда-
нибудь, честное слово, пойдем на лодке вниз по Миссисипи, я об этом давно
мечтал. На всю жизнь хватит, не маршрут — находка…

Он смолк. Он хотел уже захлопнуть атлас неловкими руками, но что-то светлое
мелькнуло в воздухе и упало на бумагу. Скатилось на песок, и получился мокрый
комочек.

Жена глянула на влажное пятнышко и сразу перевела взгляд на его лицо. Серьезные
глаза его подозрительно блестели. И по одной щеке тянулась влажная дорожка.

Она ахнула. Взяла его руку и крепко сжала.

Он стиснул ее руку и, закрыв глаза, через силу заговорил:

— Хорошо, правда, если бы мы вечером легли спать, а ночью все каким-то образом
вернулось на свои места. Все нелепости, шум и гам, ненависть, все ужасы, все
кошмары, злые люди и бестолковые дети, вся эта катавасия, мелочность, суета,
все надежды, чаяния и любовь. Правда, было бы хорошо?

Она подумала, потом кивнула.

И тут оба вздрогнули.

Потому что между ними (когда он пришел?), держа в руке бутылку из-под шипучки,
стоял их сын.

Лицо мальчика было бледно. Свободной рукой он коснулся щеки отца, там где
оставила след слезинка.

— Ты… — сказал он и вздохнул. — Ты… Папа, тебе тоже не с кем играть.

Жена хотела что-то сказать.

Муж хотел взять руку мальчика.

Мальчик отскочил назад.

— Дураки! Дураки! Глупые дураки! Болваны вы, болваны!

Сорвался с места, сбежал к морю и, стоя у воды, залился слезами.

Мать хотела пойти за ним, но отец ее удержал.

— Не надо. Оставь его.

Тут же оба оцепенели. Потому что мальчик на берегу, не переставая плакать, что-
то написал на клочке бумаги, сунул клочок в бутылку, закупорил ее железным
колпачком, взял покрепче, размахнулся — и бутылка, описав крутую блестящую
дугу, упала в море.

Что, думала она, что он написал на бумажке? Что там, в бутылке?

Бутылка плыла по волнам.

Мальчик перестал плакать.

Потом он отошел от воды и остановился около родителей, глядя на них, лицо ни
просветлевшее, ни мрачное, ни живое, ни убитое, ни решительное, ни отрешенное,
а какая-то причудливая смесь, словно он примирился со временем, стихиями и
этими людьми. Они смотрели на него, смотрели дальше, на залив и затерявшуюся в
волнах светлую искорку — бутылку, в которой лежал клочок бумаги с каракулями.

Он написал наше желание? — думала женщина.

Написал то, о чем мы сейчас говорили, нашу мечту?

Или написал что-то свое,пожелал для себя одного,чтобы проснуться завтра утром —
и он один в безлюдном мире, больше никого, ни мужчины, ни женщины, ни отца, ни
матери, никаких глупых взрослых с их глупыми желаниями, подошел к рельсам и
сам, в одиночку, повел дрезину через одичавший материк, один отправился в
нескончаемое путешествие, и где захотел — там и привал.

Это или не это? Наше или свое?..

Она долго глядела в его лишенные выражения глаза, но не прочла ответа, а
спросить не решилась.

Тени чаек парили в воздухе, осеняя их лица мимолетной прохладой.

— Пора ехать,- сказал кто-то.

Они поставили корзину на платформу. Женщина покрепче привязала шляпу к волосам
желтой лентой, ракушки сложили кучкой на доски, муж надел галстук, жилет,
пиджак и шляпу, и все трое сели на скамейку,глядя в море,- там, далеко, у
самого горизонта, поблескивала бутылка с запиской.

— Если попросить — исполнится? — спросил мальчик. — Если загадать — сбудется?

— Иногда сбывается… даже чересчур.

— Смотря чего ты просишь.

Мальчик кивнул, мысли его были далеко.

Они посмотрели назад, откуда приехали, потом вперед, куда предстояло ехать.

— До свиданья, берег, — сказал мальчик и помахал рукой.

Дрезина покатила по ржавым рельсам. Ее гул затих и пропал. Вместе с ней вдали,
среди холмов, пропали женщина, мужчина, мальчик.

Когда они скрылись, рельсы минуты две тихонько дребезжали, потом смолкли. Упала
ржавая чешуйка. Кивнул цветок.

Море сильно шумело.
