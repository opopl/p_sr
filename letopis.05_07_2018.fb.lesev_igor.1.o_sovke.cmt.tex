% vim: keymap=russian-jcukenwin
%%beginhead 
 
%%file 05_07_2018.fb.lesev_igor.1.o_sovke.cmt
%%parent 05_07_2018.fb.lesev_igor.1.o_sovke
 
%%url 
 
%%author_id 
%%date 
 
%%tags 
%%title 
 
%%endhead 
\subsubsection{Коментарі}
\label{sec:05_07_2018.fb.lesev_igor.1.o_sovke.cmt}

\begin{itemize} % {
\iusr{Mike Schachmann}

да. моим папе и маме не повезло. дедушке с бабушкой тоже. они прожили жизнь в
этой казарме. из которой было не выбраться. совок. чего наводить тень на
плетень

\begin{itemize} % {
\iusr{Юрий Ливанов}
Это они вам так говорили @igg{fbicon.thinking.face} 

\iusr{Игорь Лесев}

а еще ваши родители наверняка получили какое-то образование и оставили вам
какое-то наследство, плюс впитали в вас какие-то ценности. Последние, похоже,
помогают вам строить счастливое настоящее

\iusr{Mike Schachmann}

это как то даже не смешно. образование. ценности "строить". Вам бы, простите,
точно де то в райком. или обком. было до черта всяких комов. Вы бы не пропали

\iusr{Игорь Лесев}

я не скучаю за райкомом и даже обкомом, более того, даже не застал их в силу
юного возраста, когда не стало Союза. А вот вы очень удачно воюете со своим
прошлым. Наверное, одерживаете постоянно победы. Молодцом

\iusr{Mike Schachmann}

зато я. застал. поэтому Вам и толкую. что Вы бы не пропали. не в упрек Вам. у
комуняк это была первейшая метода. невозможно было носки нормальные купить. а
они толковали про образование. знаете "им про фому, а они за ерему'. я ни с кем
не воюю. это Вы меня с кем то спутали. ладно. спасибо что отвечали.

\iusr{Alexei Prolov}
\textbf{Mike Schachmann} согласен с вами. Совок он и есть совок. Из песни слов не выкинешь, даже если в то время жили и вы сами, и ваша семья.

\iusr{Матвей Кублицкий}
\textbf{Mike Schachmann} главного не понял. пост совсем не об совке....

\iusr{Елена Круть}
\textbf{Mike Schachmann} 

Привет колбасной эмиграции! Друзья из Израиля говорят, что в Израиль уезжают
патриоты, в Америку - энергичные... и только в Германию "колбасники", предавшие своё
страшное прошлое. Конечно, убивали евреев не страшный совок, а милая немчура. Но
вам там комфортно, носки купить можно, колбасы, опять таки, - надцать сортов.

\end{itemize} % }

\iusr{Ян Пругло}
Прекрасно.

\iusr{Михаил Подоляк}

как всегда ёмко, отлично, хлёстко... очень полезное чтение для тех, кто умён,
умеет думать чуть больше, чем о собственной ж..пе)... но, увы, не для
современных наследников тех самых хунвэйбинов, правда разговаривающих на ином
языке)))

\begin{itemize} % {
\iusr{Игорь Лесев}

кстати, пример хунвэйбинов как раз и наводит на мысль, что не все потеряно...
если дикие китайцы из 60-х, выбрасывавшие из окон преподавателей и министров,
но по итогу построили экономику №1 в мире, может и у нас что-то получится? Ну
или не получится. Но посмотреть все равно ведь интересно)

\iusr{Михаил Подоляк}
\textbf{Игорь Лесев} нет)... у нас чуть иные хунвэйбины... те были кроманьонцами), а значит развились от низшего к высшему... у нас неандертальцы), тупиковая ветвь...

\iusr{андрей кислов}
\textbf{Игорь Лесев} Раскаявшиеся бывшие хунвэйбины перед памятником убитой ими во время «культурной революции» учительницы.Тогда в Китае пострадало 100 миллионов человек, а 1 миллион человек погибли. Фото 2014 г.

\ifcmt
  ig https://scontent-lhr8-1.xx.fbcdn.net/v/t1.6435-9/36661726_10155485169185969_2929304099738353664_n.jpg?_nc_cat=107&ccb=1-5&_nc_sid=dbeb18&_nc_ohc=A4ipiisjgxwAX9oBMRK&_nc_ht=scontent-lhr8-1.xx&oh=19c7434c767c26fbdd2ae9effed19d7a&oe=61B8A6C6
  @width 0.4
\fi

\end{itemize} % }

\iusr{Матвей Кублицкий}

вот поэтому и учу историю. и оч удивляюсь людям, считающим что в прошлом было
хорошо или плохо. наивные. для них и сегодняшний мир черно-белый

\iusr{Елена Несветайлова}

"Контрастуты" вообще "люди-мисочники".)))

\begin{itemize} % {
\iusr{Игорь Лесев}
сложно, но на всякий случай лайкнул)

\iusr{Елена Несветайлова}
Предусмотрительность - признак если не мудрости, то зрелости.)
\end{itemize} % }

\iusr{Алексей Селиванов}

Россия, Русь - наше Отечество. А не "карла-марла", которая в центре картинки
кощунственно пародирует Господа.

\begin{itemize} % {
\iusr{Игорь Лесев}
грешно спорить с гусаром
\end{itemize} % }

\iusr{Владислав Клочков}

На всякий случай Игорь сказал, что если вы решили вырваться из советской
скорлупы, то вы идиот. И у вас непременно есть какой-либо памятник. Ну так,
иначе хаоса не миновать...

\begin{itemize} % {
\iusr{Матвей Кублицкий}
Мне показалось что Игорь предложил не срать на своё прошлое и своих предков

\iusr{Владислав Клочков}
А что предки? Титул. А образные ряды без унитаза уже перевелись?

\iusr{Матвей Кублицкий}
\textbf{Владислав Клочков} ваши предки - титул? я так понимаю - вам пофигу что про вас скажут ваши дети...

\iusr{Владислав Клочков}

Если даже мои предки не больше титула, я буду заботиться о том, что обо мне
подумают мои потомки. И уж точно заботиться буду, не возводя род в центр

\iusr{Матвей Кублицкий}
\textbf{Владислав Клочков} данный пост именно об этом. если вы не будете уважать своих предков - ваши дети никогда не будут уважать вас. это что-то типа одного из законов мироздания

\iusr{Владислав Клочков}

Уважать буду того, кто не даст повода к обратному. Безотносительно того, кто
когда родился и чьих фрикций усердием. Законов мироздания не формулирую и
другим этого не передоверяю. О чем пост, определил в первом своем комменте

\end{itemize} % }

\iusr{Михайло Бойченко}

Люстрація цінностей  @igg{fbicon.wink} 

\iusr{Елена Круть}

Сказать-согласна, это ничего не сказать. Это мои мысли и ощущения, только идеально изложенные....

\end{itemize} % }
