% vim: keymap=russian-jcukenwin
%%beginhead 
 
%%file slova.ideologia
%%parent slova
 
%%url 
 
%%author 
%%author_id 
%%author_url 
 
%%tags 
%%title 
 
%%endhead 
\chapter{Идеология}
\label{sec:slova.ideologia}

%%%cit
%%%cit_head
%%%cit_pic
\ifcmt
  pic https://img.strana.ua/img/article/3401/pervyj-vitse-spiker-69_main.jpeg
	caption Первый заместитель Дмитрия Разумкова возводит коттедж в Белогородке, фото: Изым Каумбаев, \enquote{Страна} 
\fi
%%%cit_text
Один из главных \emph{идеологов} партии \enquote{Слуга народа}, первый вице-спикер и
представитель президента в Верховной Раде Руслан Стефанчук скоро может отметить
новоселье в своем шикарном поместье под Киевом.  Журналисты \enquote{Страны} побывали в
селе Белогородка, что в 30 км от Киева и разыскали новенький особняк
заместителя Дмитрия Разумкова. Эта загородная недвижимость проходит по
декларации чиновника, как недостроенная усадьба. На местности видим
основательный коттедж на два этажа, выполненный в английском стиле: выдержанный
и чопорный
%%%cit_comment
%%%cit_title
\citTitle{Первый вице-спикер Руслан Стефанчук заканчивает строительство роскошного особняка в английском стиле}, 
Варвара Квитка; Владислав Бовтрук, strana.ua, 24.06.2021
%%%endcit

%%%cit
%%%cit_head
%%%cit_pic
\ifcmt
  pic https://scontent-bos3-1.xx.fbcdn.net/v/t1.6435-9/140313177_3566865490015202_1803062037982538616_n.jpg?_nc_cat=104&ccb=1-3&_nc_sid=8bfeb9&_nc_ohc=Bb9PvSNH994AX-PU1Dy&_nc_ht=scontent-bos3-1.xx&oh=2734364df86f7ce267c990ca12e3ac2e&oe=60B15B7F
	width 0.4
\fi
%%%cit_text
Процветает человеконенавистническая \emph{идеология}. Ксенофобия по отношению к
собственному населению достигла военного и гражданского апогея.
Националистическая цензура овладела не только медиа, но и наукой.
Цивилизационная память уничтожается на всех уровнях...  Вот о чем надо
говорить. А этот закон - лишь звено в цепи, очередной опухший лимфоузел в теле
онкобольной американской колонии. И стыдливые полумеры в оценках ситуации
посредством попытки отстоять... хотя бы свою систему знаков - недостаточны: так
мы сами себя обрекаем на лузерство, на выпрашивание того, что и так нам
принадлежит по праву
%%%cit_comment
%%%cit_title
\citTitle{БЖ. О русском языке без банальности}, 
Евгения Бильченко, facebook, 18.01.2021
%%%endcit


%%%cit
%%%cit_head
%%%cit_pic
%%%cit_text
Чи знаєте ви, що під час реконструкції будівлі Рейхстагу було виявлено написи
солдатів Червоної армії? Серед іншого й дискусійний, але, можливо,
найвідвертіший тогочасний напис: "Я їб…в Гітлера в зад". Її вирішили зберегти,
як і решту.  Адже немає в історії нічого ганебнішого, ніж інфантильні спроби
зображати тиранію \emph{ідеологією} \enquote{непорочної} жорстокості,
приховуючи розмаїття людського свавілля, падіння героїв, тріумф полеглих, увесь
цей \enquote{трах} у голові п’яного солдата, який, геть забувши про мирне
життя, заліз на німкеню, яка волала від жаху і, можливо, була побита, щоб потім
ще роками заливати оковитою свій посттравматичний синдром, не вміючи розповісти
про те, що скоїла з ним війна, відштовхуючи власну родину, гиркаючи на матір;
солдата, в якого, можливо, за всеньке його життя випала єдина нагода вкласти
всю свою виплекану на вже Другій світовій війні огиду до нацизму в ті кілька
слів на стіні Рейхстагу
%%%cit_comment
%%%cit_title
\citTitle{Українці не розуміють одне одного не через мову, а через небажання слухати, чути і сприймати}, 
Юлія Мендель, www.pravda.com.ua, 07.07.2021
%%%endcit
