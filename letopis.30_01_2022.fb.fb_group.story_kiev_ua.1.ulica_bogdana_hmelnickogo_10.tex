% vim: keymap=russian-jcukenwin
%%beginhead 
 
%%file 30_01_2022.fb.fb_group.story_kiev_ua.1.ulica_bogdana_hmelnickogo_10
%%parent 30_01_2022
 
%%url https://www.facebook.com/groups/story.kiev.ua/posts/1850547748475351
 
%%author_id fb_group.story_kiev_ua,borozenec_anatolij.kiev
%%date 
 
%%tags gorod,kiev,ulica.kiev.bogdana_hmelnickogo
%%title Вул. БОГДАНА ХМЕЛЬНИЦЬКОГО, 10
 
%%endhead 
 
\subsection{Вул. БОГДАНА ХМЕЛЬНИЦЬКОГО, 10}
\label{sec:30_01_2022.fb.fb_group.story_kiev_ua.1.ulica_bogdana_hmelnickogo_10}
 
\Purl{https://www.facebook.com/groups/story.kiev.ua/posts/1850547748475351}
\ifcmt
 author_begin
   author_id fb_group.story_kiev_ua,borozenec_anatolij.kiev
 author_end
\fi

Вул. БОГДАНА ХМЕЛЬНИЦЬКОГО,10

На початку 1900-х це був один із прибуткових будинків купця Фрідріха
Міхельсона. А ще раніше тут знаходився двоповерховий будинок архітектора
Л.Станзані. Потім садибу придбало Київське літературно-артистичне товариство.
Планували звести будівлю з великим концертною залою. Не склалося. Їхні фінанси
проспівали печальні романси і садибу перекупив Міхельсон.

\raggedcolumns
\begin{multicols}{3} % {
\setlength{\parindent}{0pt}

\ii{30_01_2022.fb.fb_group.story_kiev_ua.1.ulica_bogdana_hmelnickogo_10.pic.1}

\ii{30_01_2022.fb.fb_group.story_kiev_ua.1.ulica_bogdana_hmelnickogo_10.pic.2}
\ii{30_01_2022.fb.fb_group.story_kiev_ua.1.ulica_bogdana_hmelnickogo_10.pic.3}
\ii{30_01_2022.fb.fb_group.story_kiev_ua.1.ulica_bogdana_hmelnickogo_10.pic.4}
\ii{30_01_2022.fb.fb_group.story_kiev_ua.1.ulica_bogdana_hmelnickogo_10.pic.5}

\end{multicols} % }

За рік з хвостиком - у 1904-му - за проектом архітектора Е.Братмана був
зведений у цегляному модерновому стилі цей п’ятиповерховий будинковий комплекс
у вигляді каре з двором-колодязем всередині. Через незвичний декор кияни
називали його «Мереживним» («Кружевным»). Але декор зробили тільки на фасадній
частині. Тоді жартували, мовляв, будинок красивий, але всередині пустий.

Нічого собі пустий! У сенсі декору з внутрішньої дворової частини можна
погодитися. Натомість будівля здивувала різноманіттям внутрішньої
функціональної наповненості. Якщо б Міхельсона спитали тоді: «Кто-кто в
теремочке живет?», то його відповідь вразила б.

Крім житлових квартир, у \enquote{теремочек} з підвальними приміщеннями вмістилися:
ресторан, аптека, магазини, Правління шляхів сполучення округу, Клуб
автомобілістів, Товариство мистецтва та літератури з театральним училищем при
ньому, фортепіанна фабрика І. Мекленбурга, склад тютюнових виробів Г.Денс...

На другому та третьому поверхах розміщалася приватна жіноча гімназія Олександри
Титаренко. Її закінчила майбутня зірка МХАТу Алла Тарасова.

Тут також було Технічне товариство «Трикутник», яке представляло в Києві
«Товариство Російсько-Американської гумової мануфактури «Трикутник». За високу
якість продукції воно мало право маркувати вироби державним гербом. Я так
розумію Товариство налагодило виробництво гумових калош у Санкт-Петербурзі, а
тут була його київська контора, склади, магазини.

Не знаю, який сенс вкладали засновники в назву Товариства, тому трохи
пофантазую... А чи не мало воно якесь відношення до масонів? Тим паче, що мало
російсько-американське походження. Основне виробництво – в Санкт-Петербурзі, де
за царських часів тих масонів було, як гуталіну в дядька Кота Матроскіна з
гуталінової фабрики ) До того ж, трикутник є масонським символом. Часто
зображувався з іншим – Всевидючим оком.

Тут же, аби не так кидалося в очі, назву «Треугольникъ» вписали в арочне
напівколо над таким же вікном. Символ Всевидючого ока могли замінити як вікно,
так і зображення відбитка підошви калоші. Ну, а поряд із підошвою можна
побачити уже й фігуру трикутника. Всередині його щось зображено, але, на жаль,
фото не дає можливості розгледіти, що саме.

У радянський період з фасаду будівлі зникли балкони. При чому знесли їх якось
тупо. Навіть на думку організаторам і виконавцям не спало замаскувати дверні
отвори під віконні. Так і залишили – балконів немає, а балконні двері є. Тому
під цим будинком краще не співати під гітару: «Вийди, серденько, на балкон.
Заспіваю тобі серенаду» )
