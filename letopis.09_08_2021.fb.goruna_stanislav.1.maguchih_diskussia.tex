% vim: keymap=russian-jcukenwin
%%beginhead 
 
%%file 09_08_2021.fb.goruna_stanislav.1.maguchih_diskussia
%%parent 09_08_2021
 
%%url https://www.facebook.com/stanislav.horuna.3/posts/1879508002260146
 
%%author Горуна, Станислав
%%author_id goruna_stanislav
%%author_url 
 
%%tags __aug_2021.maguchih.foto.olimpiada.lasickene,goruna_stanislav,maguchih_jaroslava,olimpiada.tokio,sport,ukraina
%%title Хочу прояснити контекст дискусії, яка розгорнулася в попередньому пості
 
%%endhead 
 
\subsection{Хочу прояснити контекст дискусії, яка розгорнулася в попередньому пості}
\label{sec:09_08_2021.fb.goruna_stanislav.1.maguchih_diskussia}
 
\Purl{https://www.facebook.com/stanislav.horuna.3/posts/1879508002260146}
\ifcmt
 author_begin
   author_id goruna_stanislav
 author_end
\fi

Хочу прояснити контекст дискусії, яка розгорнулася в попередньому пості.

Своїм дописом я хотів зупинити потік негативу, який пішов на 19 річну Ярославу
Магучіх. Насправді мені це вдалось, бо весь негатив перейшов на мене))

Що ж, завдання виконано. А тепер давайте спробуємо поговорити нормально.

Є спорт як сфера функцій держави, і є спорт як діяльність атлетів і тренерів. 

Для мене спорт: спортзал, напрацювання техніко-тактичних завдань, покращення
фізичних кондицій, підготовка і виїзд на змагання, тренувальні збори, спаринги,
харчування, зважування, виступи, перемоги. Цьому процесу я віддаю себе і свої
сили. Впевнений, що і 19-річна Ярослава Магучіх також. Там немає місця навіть
для існування політики, тим більше, там немає умислу впливати політично,
використовуючи спортивну впізнаваність. 

Спорт, як сфера функцій держави - це справи державні, тут не те, що є місце
політиці, це і є політика. Це те, що працює навколо безпосередньої спортивної
діяльності, те , що її забезпечує. Це імідж і престиж держави, це справа
функціонування Міністерства спорту та інших профільних державних і комунальних
структур та організацій. Це як міжнародні відносини, так і внутрішні відносини
Мінспорту з федераціями. Це зарплати, призові і стипендії. Це діяльність цілого
держапарату.

Це площина, в якій я не працюю.

Це не те, чим займаюсь я, чи Ярослава Магучіх, чи Олег Вєрняєв, чи будь-який
інший діючий спортсмен. Чи не так,  Андрій Демчук?

Привітання на п’єдесталі  суперниць/суперників обіймами - це звичний ритуал у
всіх видах спорту у всіх країнах.

Це традиційний прояв поваги. 

Я, власне, на цьому і наголошував, але більшість побачило лише заголовок «спорт
поза політикою» і почали трактувати за своїми вподобаннями. 

Міжнародна арена - це наслідки складної щоденної роботи атлета. І поведінка
атлета на міжнародній арені також регламентована. Особливо, коли атлет
представляє рідну державу - він як ніколи мусить поводитись ввічливо і коректно
щодо інших атлетів, тренерів чи функціонерів. Церемонія нагородження
спортсменів - це традиційний ритуал, під час якого відбувається урочиста
фіксація результатів змагань. Є писані і неписані правила поведінки атлетів на
подіумі. Ярослава Магучіх все зробила, як від неї очікувалось. А ігнорування
інших атлетів на подіумі більше б розцінювалось як негідна поведінка
спортсмена.

Я усвідомлюю всю складність і гіркоту подій останніх 8 років. Я жодним чином не
хотів знехтувати чи применшити жертву наших бійців, наших героїв, на благо і
захист України! Я вдячний їм, що маю можливість займатись улюбленою справою і
також приносити користь і перемоги нашій державі!
