% vim: keymap=russian-jcukenwin
%%beginhead 
 
%%file 27_11_2020.news.ua.strana.romanova_maria.1.pirozhok_pravyy_sektor
%%parent 27_11_2020
 
%%url https://strana.ua/news/303130-ihor-pirozhok-intervju-natsionalista-iz-pravoho-sektora-kotoryj-sidit-v-rossii-za-ekstremizm.html
 
%%author Романова, Мария
%%author_id romanova_maria
%%author_url https://strana.ua/hgraph/authors/80.html
 
%%tags 
%%title "Правый сектор создала СБУ". Что рассказал майдановец Пирожок, сидящий за решеткой в России
 
%%endhead 
 
\subsection{\zqq{Правый сектор создала СБУ}. Что рассказал майдановец Пирожок, сидящий за решеткой в России}
\label{sec:27_11_2020.news.ua.strana.romanova_maria.1.pirozhok_pravyy_sektor}
\Purl{https://strana.ua/news/303130-ihor-pirozhok-intervju-natsionalista-iz-pravoho-sektora-kotoryj-sidit-v-rossii-za-ekstremizm.html}
\ifcmt
	author_begin
   author_id romanova_maria
	 author_url https://strana.ua/hgraph/authors/80.html
	author_end
\fi
\index[names.rus]{Пирожок, Игорь!Правый Сектор}

\ifcmt
	 pic https://strana.ua/img/article/3031/30_main.jpeg
	 caption Игорь Пирожок. Кадр из видео 
\fi

Вчера российские СМИ опубликовали интервью с бывшим членом \zqq{Правого сектора}
Игорем Пирожком.\Furl{https://strana.ua/news/251786-v-rossii-posadili-pirozhka-iz-pravoho-sektora-26-fevralja-2020.html} 

С февраля он отбывает срок в РФ за экстремизм - вербовку сторонников. 

У Пирожка пестрая националистическая и преступная биография. В 90-е он
проживал в Москве и был одним из организаторов \zqq{концлагеря}. В нем
насильно содержались бомжи, рабочая сила которых использовалась бесплатно.

Попутно Пирожок входил в неонацистскую организацию \zqq{Легион Вервольф}.
Идеология сводилась к \zqq{окончательному решению еврейского вопроса
радикальными методами}. В рамках группировки украинец получил срок за
подготовку теракта в Ярославской области. После отсидки отправился в
Украину, в родную Одессу. Там получил еще один срок - за распространение
порнографии. 

С началом Евромайдана компетенции Пирожка пригодились, карьера пошла в
гору. Мужчина стал членом \zqq{Правого сектора}, занялся самообороной
Бердичева Житомирской области. 

На базе самообороны функционировали структуры, которые крышевали
незаконную добычу января. Также, утверждает Пирожок, он помогал с
продвижением людей от \zqq{Радикальной партии Ляшко}, занимался люстрацией и
другими вопросами. 

Об этом всем этом Игорь Пирожок рассказал изданию \zqq{Лента.ру}.\Furl{https://lenta.ru/articles/2020/11/26/pirozhok_ukr/} \zqq{Страна}
пересказывает интервью.

\subsubsection{Концлагерь в Москве и самооборона в Бердичеве}

Игорь Пирожок (носил псевдоним Роман Чирка) начал рассказ с того, что с
Евромайданом в страну пришла и анархия. Бердичев, по его словам, - город
четырех культур: польской, русской, украинской и еврейской.

\zqq{Мы видели, что творилось. Регулярно происходили налеты на небольшие
города - молодые люди на автобусах заезжали и громили все, грабили
магазины. Майдановцы, антимайдановцы - неважно, кто оплатил их услуги.
Когда такая банда помогает устроить рейдерский захват и лишает вас места
работы, ее политические пристрастия второстепенны}, - говорит Пирожок. 

В связи этим каждый взял понятное ему направление. Про его историю с
\zqq{Легионом \zqq{Вервольф} знали. Поэтому было решено, что я беру под себя всех
наших правых. Я и взял. Но я не был их лидером, скорее политическим
советником}, - рассказывает мужчина. 

По \zqq{Вервольфу} Пирожок действительно был известной личностью в
определенных кругах. Например, в Интернете есть видео собрания этой
группировки. Все вокруг в свастике, он зигует. 

\ifcmt
tab_begin cols=2
	caption Пирожок зигует
	pic https://strana.ua/img/forall/u/10/91/%D0%BF%D0%B8%D1%80%D0%BE%D0%B6%D0%BE%D0%BA2.jpg
	pic https://strana.ua/img/forall/u/10/91/%D0%BF%D0%B8%D1%80%D0%BE%D0%B6%D0%BE%D0%BA1.jpg
tab_end 
\fi


По организации \zqq{концлагеря} в Москве он также был знаменит. Российские СМИ
сообщали, что там было совершено не одно убийство. Сам Пирожок хвастался
на камеру отрубленными ушами мужчины - одного из своих \zqq{рабов}. Кадры были
показаны по телевидению в передаче \zqq{Совершенно секретно} в 1993 году. 

Под руководством этого человека, которого приняли в ряды \zqq{Правого
сектора}, и была организована самооборона Бердичева.

На все въезды и выезды Пирожок и Ко ставили блокпосты. Следили за
обстановкой, переговаривалась друг с другом с помощью рации - \zqq{в случае
проблем в нужные места выезжало подкрепление}. Чужаков в городе не
пускали. С местной полицией сотрудничали. Связь с криминалитетом
поддерживали. 

В общем, альтернативная власть. 

\ifcmt
pic https://strana.ua/img/forall/u/10/91/pic_1e4021509ed17ae5328ec489dea8aadc.jpg
caption Самооборона Бердичева. Фото: Facebook
\fi

В рамках \zqq{Правого сектора} Пирожок организовал отряд \zqq{Хорты}. Ее сиделец
описывает как созидающую и соответствующую принципам двуязычного народа
(имеется в виду украинский и русский языки). 

\zqq{Одно из первых дел, которое сделали наши парни, - восстановили и
отреставрировали все памятники советским воинам в районе. Мы были против
идеологического противостояния}, - говорит Игорь Пирожок. 

\subsubsection{Кто создал и финансировал \zqq{Правый сектор}}

Для российской аудитории Игорь Пирожок рассказывает: организация \zqq{Правый
сектор} называется так лишь потому, что находилась на Евромайдана справа.
А позже к ней присоединились ультрас киевского \zqq{Динамо}, которые на
стадионе также занимают правую часть. 

По словам Пирожка, \zqq{Правый сектор} был создан Службой безопасности
Украины. \zqq{Он до сих пор продолжает работать на спецслужбы против министра
внутренних дел Украины Арсена Авакова}, - говорит заключенный в интервью. 

О том, кто есть кто в \zqq{Правом секторе}, националист рассказывает
следующее: \zqq{Дмитрий Ярош - это же кум Валентина Наливайченко, бывшего
главы СБУ при президенте Викторе Ющенко, крестный отец его детей.
Наливайченко создал организацию \zqq{Тризуб} и создал Яроша. Это маленькая
маргинальная организация с хорошей крышей и эксклюзивной информацией}. 

На вопрос о том, кто финансировал \zqq{Правый сектор}, Пирожок
отвечает: \zqq{вездесущие американцы и НАТО}. Но хватало и представителей
местного бизнеса.

Также поддерживал националистов, говорит Пирожок, в том числе
Янукович: \zqq{Его очень впечатлили события после убийства журналиста Георгия
Гонгадзе в 2000 году, которые фактически свалили Леонида Кучму с поста
президента, и ему не нужны были вот такие гражданские выступления. Нужно,
чтобы улица была за \zqq{Тризубом} }.

По словам Пирожка, радикалы очень полезны для нанесения точечного удара:
\zqq{И неважно, кто они - ультралибералы, ультраправые или ультралевые. До
того как Майдан перерос в государственный переворот, он несколько месяцев
плясал. Люди приезжали, уезжали, слонялись без толку. Они бы до сих пор
там плясали - бессмысленный протест}.

Об успехе Евромайдана он рассуждает так: \zqq{Чтобы он во что-то вылился,
нужно было его радикализировать - поджечь сотрудника правоохранительных
органов, разбить кому-то голову, кому-то получить пулю. Они и добились
того, чего хотели}. 

В итоге пути Пирожка и \zqq{Правого сектора} разошлись. Он вынужденно покинул
и эту организацию, и движение \zqq{Хорты}. Мол, благословение на работу от
Яроша ему не давали. 

\ifcmt
pic https://strana.ua/img/forall/u/10/91/%D1%85%D0%BE%D1%80%D1%82%D1%8B.jpg
caption Пирожок с юными членами движения "Хорты Бердичев". Фото: Facebook
\fi


\subsubsection{О крышевании добычи янтаря и пиаре партии Ляшко}

После того, как Игорь Пирожок ушел из \zqq{Правого сектора}, он остался
работать в местной житомирской политике. Он был в организации по
содействию люстрации, оставался в совете при губернаторе, входил в
коллегию по кадастру.

\zqq{Хватало дел с землей. Житомир же самая перспективная область. У нас там
ресурсов... Янтаря столько, что вашей Калининградской области и не
снилось! А еще гранит, щебенка, песок и самые большие в Европе запасы
титана и ильменита. За них и теперь идет война}, - рассказывает Пирожок. 

На крышевание нелегальной добычи янтаря после активной фазы войны на
Донбассе начали поглядывать атошники - хотели забрать под себя. 

\zqq{В янтарный бизнес активно лезли все эти общественно-гражданские отделения
добровольческих батальонов. Надо понимать, что добробаты вышли из
\zqq{движений}. Нормальные добровольцы шли тогда служить в Вооруженные силы
Украины, а добровольческие батальоны - это же бандитские военизированные
формирования}, - говорит Пирожок. 

По словам радикала, тогда в Житомирской области началась схватка за ресурсы.
\zqq{Когда я вмешался, у меня хватало сторонников. В областном совете было
шесть моих депутатов, которых мы протолкнули через \zqq{Радикальную партию}
Олега Ляшко. На базе самообороны в Бердичеве продолжали работать созданные нами
структуры, не дававшие развернуться чужакам на нашем янтаре. Потом мы
противостояли введению новых тарифов. Не пускали чиновников из Киева}, -
отмечает он. 

В качестве примера Игорь Пирожок рассказывает: \zqq{Когда к нам прислали из
Киева нового начальника милиции, мои ребята пришли и выгнали его пинками.
Мы не держали власть, но мы ее корректировали}. 

В 2015 году Пирожок был задержан винницкой СБУ. Это, по словам,
предшествовал конфликт с местным олигархом Александром Ревегой.

\zqq{Раньше он был депутатом от Партии регионов, но к тому времени, как
водится, находился уже в Блоке Петра Порошенко. И вот он пообещал, что
меня посадит. Поругались мы из-за земли и в целом из-за его попыток
ставить своих людей у нас на должности}, - объяснил он. 

\ifcmt
pic https://strana.ua/img/forall/u/10/91/%D0%B7%D0%B0%D0%B4%D0%B5%D1%80%D0%B6%D0%B0%D0%BD%D0%B8%D0%B5(1).jpg
caption Задержание Пирожка. Фото: Онлайн Житомир
\fi

В итоге его арестовали и отправили под домашний арест. Потом \zqq{выпустили погулять}. За это время он \zqq{успешно провел выборы для партии Ляшко}. 

\zqq{Им я
помогал, поскольку можно было с минимумом голосов провести в облсовет
своих людей}, - рассказывает Пирожок. 

\subsubsection{Пирожок в Москве}

Суд приговорил его в Украине к трем годам. Но до апелляции он \zqq{мог гулять
свободно}. Этим он воспользовался. Уехал на авто в Приднестровье, а оттуда в
Москву. Он был уверен, что России опасаться нечего, ведь там он уже отбыл
\zqq{ходку}. 

Через пару лет после переезда в Москву его задержали за новое преступление.
\zqq{Правый сектор} - запрещенная организация в России. Дома в Москве у Пирожка
прошел обыск. Нашли символику, обмундирование, литературу. Было возбуждено дело
по статье \zqq{Организация деятельности экстремистской организации} - за
вербовку людей.

Задержание было в апреле 2019 года. Он получил четыре года тюрьмы. На
свободу может выйти досрочно по УДО или планово через три года.

Пирожок предполагает, что после этой отсидки он будет экстрадирован в
Украину, где опять отправится в тюрьму уже за правонарушения в Украине. 


