%%beginhead 
 
%%file 30_08_2021.fb.fb_group.mariupol.nekropol.1.volontery_otkryli_dve_drevnie_plity_xix_veka
%%parent 30_08_2021
 
%%url https://www.facebook.com/groups/278185963354519/posts/589598428879936
 
%%author_id fb_group.mariupol.nekropol,marusov_andrij.mariupol
%%date 30_08_2021
 
%%tags 
%%title Волонтеры открыли в Мариупольском Некрополе две древние плиты XIX века
 
%%endhead 

\subsection{Волонтеры открыли в Мариупольском Некрополе две древние плиты XIX века}
\label{sec:30_08_2021.fb.fb_group.mariupol.nekropol.1.volontery_otkryli_dve_drevnie_plity_xix_veka}
 
\Purl{https://www.facebook.com/groups/278185963354519/posts/589598428879936}
\ifcmt
 author_begin
   author_id fb_group.mariupol.nekropol,marusov_andrij.mariupol
 author_end
\fi

\vspace{0.5cm}
\textbf{Волонтеры открыли в Мариупольском Некрополе две древние плиты XIX века}

\#новости\_архи\_города

Начало осеннего исследовательского сезона-21 в Мариупольском Некрополе
оказалось богатым на открытия. На древнем \enquote{Хараджаевском} участке волонтеры
обнаружили две старинные плиты греческого рода Сахаджи. Предположительно, обе
датируются 1880-х годами.

Эти находки окончательно подтвердили уникальный характер участка. На нем
сохранились усыпальницы сразу пяти родов приазовских греков, включая склепы и
памятники Александра Хараджаева, городского головы в 1860-х годах, Спиридона
Гофа, гласного городской думы, и его брата Гаврила, построившего дом редакции
\enquote{Приазовского рабочего} и т.д.

Здесь же находится шестнадцать древних безымянных плит – самое большое
количество среди всех участков Некрополя.

Первая плита, обнаруженная 21 августа, пострадала от неизвестных
гробокопателей. Они ее перевернули, наткнулись на кирпичное основание и
забросили плиту в заросли сирени. 

На лицевой стороне массивной гранитной плиты оказалась надпись: \emph{\enquote{Здесь покоится
прах Анастасии Сахаджи}}. По данным сайта Azovgreeks, в роду Сахаджи было две
Анастасии, обе с отчеством \enquote{Кирилловна}. Первая родилась в 1838 году, а вторая
– в 1902 году. Судя по материалу и исполнению плиты и надписи, под ней, скорее
всего, похоронена Анастасия Кирилловна 1838- года рождения...

Вторая плита явно использовалась \enquote{повторно} в 20 веке. Ее внешняя поверхность
была покрыта штукатуркой, и волонтеры не обращали на нее внимание, думая, что
это плита советского периода.

Лишь благодаря наблюдательности волонтерки Наталии Шпотаковской (Наталия
Шпотаковская) удалось выяснить, что эта плита когда-то принадлежала Ивану
Гавриловичу Сахаджи (1804-1883 гг.). В 20 веке кто-то из мариупольцев ее
\enquote{присвоил}, передвинул и приспособил для своей могилы.

Примечательно, что буквально в пяти шагах находится плита отца Ивана
Гавриловича – Гавриила Кирилловича Сахаджи. Она была \enquote{пере-открыта} волонтерами
в апреле 2020 года и на сегодня является древнейшей датированной плитой
Некрополя (1834-35 гг.). Надпись на ней выполнена на греческом!

Таким образом, на сегодня на \enquote{Хараджаевском} участке найдены родовые
усыпальницы пяти старинных родов греков-перво\hyp{}поселенцев: Сахаджи, Хараджаевых,
Гофов, Пиличевых, Чакириди, а также сестер Гозадиновых, родственниц митрополита
Игнатия.

Шестнадцать безымянных древних плит на участке, скорее всего, принадлежали
грекам-первопоселенцам из этих же родов (см. карту участка). 

\begin{leftbar}
По мнению Андрея Марусова, директора ГО \enquote{Архи-Город}, Хараджаевский участок
Мариупольского Некрополя – это первый кандидат на музеефикацию и
полномасштабное благоустройство. Ведь, кроме памятников греческого
наследия, на нем находятся могилы многих выдающихся мариупольцев
(послевоенного мэра Мариуполя Скидского, родственников фотографа Михаила
Улахова, историка и провидника ОУН Николая Фененко, преподавателя
Мариинской женской гимназии Ивана Александровича, директора послевоенного
\enquote{Азовстальстроя} Александра Потлова и т.д.). 

Еще парочка таких ливневых лет, и плиты снова будут занесены землей, а
расчищенное волонтерами пространство поглотят заросли. Поэтому уже сейчас
необходимо прокладывать дорожку по \enquote{Хараджаевской} тропе, а плиты поднимать
и ставить на надежное основание. У волонтеров отсутствуют ресурсы для
реализации этой задачи. Необходима помощь мэрии и меценатов... 
\end{leftbar}

\begin{minipage}{0.9\textwidth}
Мы искренне благодарны авторам открытий – волонтерам Александру и Наталье
Шпотаковским, Илья Луковенко\footnote{\url{https://www.facebook.com/profile.php?id=100004930356528}} и 
Valentina Evseenko\footnote{\url{https://www.facebook.com/valentina.evseenko.7}}.

Мы будем рады предоставить более подробную информацию (контакты
\enquote{Архи-Города}: тел. 096 463 69 88, arximisto@gmail.com).
\end{minipage}
