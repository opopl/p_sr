% vim: keymap=russian-jcukenwin
%%beginhead 
 
%%file 17_12_2022.fb.skytalinska_oksana.kiev.volonter.dietolog.1.tyzhden
%%parent 17_12_2022
 
%%url https://www.facebook.com/O.Skytalinska/posts/pfbid02VwJNhE7V13inWomBEpn8xcFSiZvUDTYZTJnsfHCHox3vwLHa9LUCF3pNJ6w9NMfLl
 
%%author_id skytalinska_oksana.kiev.volonter.dietolog
%%date 
 
%%tags 
%%title Минулий тиждень був багатий на обстріли та наукові події
 
%%endhead 
 
\subsection{Минулий тиждень був багатий на обстріли та наукові події}
\label{sec:17_12_2022.fb.skytalinska_oksana.kiev.volonter.dietolog.1.tyzhden}
 
\Purl{https://www.facebook.com/O.Skytalinska/posts/pfbid02VwJNhE7V13inWomBEpn8xcFSiZvUDTYZTJnsfHCHox3vwLHa9LUCF3pNJ6w9NMfLl}
\ifcmt
 author_begin
   author_id skytalinska_oksana.kiev.volonter.dietolog
 author_end
\fi

Минулий тиждень був багатий на обстріли та наукові події.

Між бахканням ППО і відключеннями світла писалися доповіді, консультувалися
пацієнти, і закуповувалася термобілизна на фронт.

Кожного вечора лягаєш спати і не знаєш, чи прокинешся вранці. Такі реалії. Але
це все - ніщо у порівнянні з тим, як нашим захисникам на фронті. Ми витримаємо
все.

Незважаючи на спорожнілі не освітлені вулиці, наша Асоціація дієтологів України
провела гарну освітню подію, присвячену дієтології.

Я вибрала одну із своїх улюблених тем: \#нейрохарчування. 

І поєднала її з темою жіночого здоров'я. Ми, \#жінки, надзвичайно вразливі.
Такими нас роблять наші жіночі гормони, точніше, стани, коли їх стає мало і
критично мало.

І ця тема стосується не лише гінекологів та ендокринологів, вона значно ширша,
бо жінки дуже часто не йдуть до лікаря, списують свій дискомфорт на стрес,
втому і багато іншого.

Тому до теми періменопаузи та менопаузи мають бути залучені фахівці різних
спеціальностей: гінекологи, ендокринологи, кардіологи, неврологи, дієтологи,
тренери, психологи, косметологи. І всі ці фахівці мають бути інформованими не
лише у своїй сфері.

❗️Харчування, режим прийому їжі, кількість та склад формує наш організм,
напрямок обміну речовин, забезпечує роботу імунної системи, рівень
детоксикації, швидкість розвитку запалень та порушення, які виникають при не
здоровому харчуванні.

❗️Для жінок різного віку мають бути свої правила складання раціону. Це важливо!
Не достатньо говорити про здорове харчування та про Тарілку здорового
харчування, бо те, що підходить 30-річній жінці, для 60-річної не принесе
користі і може бути навіть шкідливим. 

❗️Це стосується і харчування і фізичних навантажень. Схуднути для 35-річної -
це зовсім не те, що для 55-річної. 

І ця інформація має поширюватися всюди. 

Хотілося, щоб і наше Міністерство охорони здоров'я звернуло увагу на тему
особливостей жіночого харчування.

Принаймні, я обіцяю писати про це багато.

Якщо Ви слухали та дивилися мою презентацію, напишіть, що би ще хотіли почути?

P.S. Я була дуже щасливою, коли побачила і обійняла своїх колег, незламних
українських лікарів ❤️

Обіймаю всіх!

Слава Україні!

\ii{17_12_2022.fb.skytalinska_oksana.kiev.volonter.dietolog.1.tyzhden.orig}
\ii{17_12_2022.fb.skytalinska_oksana.kiev.volonter.dietolog.1.tyzhden.cmtx}
