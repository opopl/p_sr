% vim: keymap=russian-jcukenwin
%%beginhead 
 
%%file letters.mariupol.demidko_olga.9
%%parent letters.mariupol.demidko_olga
 
%%url 
 
%%author_id 
%%date 
 
%%tags 
%%title 
 
%%endhead 

21:35:29 15-10-23
... добрался сейчас между прочим уже и до Стоминой ) а Вы чо думали, я шутил насчет собрания сочинений... и вообще, я сам себе удивляюсь, знаете, потому что я никогда не мог подумать, что я буду заниматься тем, чем я сейчас занимаюсь, то есть Мариуполем, причем в таком необычном ракурсе! потому что новости про Мариуполь в основном из негатива какого то состоят... но я то их не читаю... для меня Мариуполь - живой, да! И чем больше я им занимаюсь, тем он живее, ярче, красочней, интереснее становится! А касательно Киева... да,  Киев - вообще удивительный Город! Тут происходит то, чего нигде больше не случается! Странный, необычайный Город Киев! Как же мне повезло все-таки, что я киевлянин!!! И голуби мне сами на балкон залетают, и мариупольские кролики из ниоткуда сами собой вылазят на четвертом этаже Музея Города Киева! ... но что  насчет Стоминой, как однако актуально звучат ее стихи, ейбогу... Ми – немов дві половини одного міста.
Без тебе для мене це місто не має змісту,
Це місто не має сенсу і глибини.
Без мене руйнуються в ньому дахи і стіни,
Без мене воно перетвориться на руїни,
Немов після землетрусу або війни.
Проте, якщо ми удвох, все іде як треба:
Всі сходи за першим бажанням ведуть у небо,
Всі двері – у щастя, всі хвіртки – у ті сади,
Де квітне кохання, й співають про нього птиці,
Де на діамант перетворюються дрібниці,
Де мешкає ніжність і всюди її сліди...
