% vim: keymap=russian-jcukenwin
%%beginhead 
 
%%file 30_08_2021.fb.ivanova_viktoria.doneck.1.sonet_gorod_doneck
%%parent 30_08_2021
 
%%url https://www.facebook.com/permalink.php?story_fbid=889518594984124&id=100017779756354
 
%%author_id ivanova_viktoria.doneck
%%date 
 
%%tags donbass,doneck,gorod,peace,poezia,stihi,vojna
%%title Сонет любимому городу
 
%%endhead 
 
\subsection{Сонет любимому городу}
\label{sec:30_08_2021.fb.ivanova_viktoria.doneck.1.sonet_gorod_doneck}
 
\Purl{https://www.facebook.com/permalink.php?story_fbid=889518594984124&id=100017779756354}
\ifcmt
 author_begin
   author_id ivanova_viktoria.doneck
 author_end
\fi

Сонет любимому городу 

\begin{multicols}{2}
\obeycr
Поднялось весеннее солнышко, по листве мягко льётся свет.
Нам семь лет отчаянно хочется, этих слов «Войны больше нет»!!!!!!!!!!!!
Чтобы жизнь разделить на мирную, до войны и военных лет.
\smallskip
Пусть Донецк заживет по-старому: без обстрелов и разных бед.
Город- труженик, город- мученик! Я, как дочь, за тебя молюсь!
Ты же выстоишь? Ты же справишься? Ты же выдержишь этот груз?
\smallskip
Укрепила война, сплотила нас и проверила прочность уз...
Показала нам цену каждому.... Кто остался тут, тот не трус!!!
Все мы ждем, очень ждем, надеемся, что наступит войне конец!
\smallskip
Город - крепость, в которой молится в унисон миллион сердец...
Пусть окопы травой покроются, а в степи зацветет чабрец!
Тишиною поля укроются, перестанет свистеть свинец...
\smallskip
Город-труженик, город-мученик! Я люблю тебя, мой Донецк!
Я пишу тебе, я дышу тобой... на тебе терновый венец....
\restorecr
\end{multicols}

(с)

\begin{itemize} % {
\iusr{Red Fly Fox}
Ну вот, теперь реву как дура...

\iusr{Виктория Иванова}
\textbf{Виктория Летаева} я тоже сопли по морде размазала


\iusr{Наталья Гольденберг}
Добавьте "поделиться", pls!

\begin{itemize} % {
\iusr{Виктория Иванова}
\textbf{Наталья Гольденберг} ща

\iusr{Виктория Иванова}
\textbf{Наталья Гольденберг} сделано

\iusr{Наталья Гольденберг}
\textbf{Виктория Иванова} спасибо!
\end{itemize} % }

% -------------------------------------
\ii{fbauth.logvinova_dina.doneck.dnr.jenakievo}
% -------------------------------------



Прекрасное стихотворение! Возьму его в свою поэтическую копилку... Надо бы
автора узнать. Я попробую. Если Вы, Виктория, узнаете, напишите, пожалуйста...

\begin{itemize} % {
\iusr{Виктория Иванова}
\textbf{Дина Логвинова} обязательно. Попробую спросить напрямую, его скинули в рабочий чат

\iusr{Дина Логвинова}
Автора нашла. Некто Марина Перкова. Но лучше уточнить... Вам СПАСИБО.

\iusr{Виктория Иванова}
\textbf{Дина Логвинова} , люблю такие стихотворения - слова простые, но до мурашек и слез

\iusr{Дина Логвинова}
Это действительно высокая поэзия. Потому что от сердца...

\iusr{Виктория Иванова}
\textbf{Дина Логвинова} безусловно
\end{itemize} % }

\iusr{Юрий Афендулов}
Правильные стихи и думаю пророческие.

\end{itemize} % }
