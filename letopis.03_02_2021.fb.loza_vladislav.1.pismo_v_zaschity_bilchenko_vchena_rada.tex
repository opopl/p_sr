% vim: keymap=russian-jcukenwin
%%beginhead 
 
%%file 03_02_2021.fb.loza_vladislav.1.pismo_v_zaschity_bilchenko_vchena_rada
%%parent 03_02_2021
 
%%url https://www.facebook.com/vladyslavloza/posts/1646848002186322
 
%%author Лоза, Владислав
%%author_id loza_vladislav
%%author_url 
 
%%tags bilchenko_evgenia
%%title Відкритий студентський лист на захист професора Більченко
 
%%endhead 
 
\subsection{Відкритий студентський лист на захист професора Більченко}
\label{sec:03_02_2021.fb.loza_vladislav.1.pismo_v_zaschity_bilchenko_vchena_rada}
\Purl{https://www.facebook.com/vladyslavloza/posts/1646848002186322}
\ifcmt
 author_begin
   author_id loza_vladislav
 author_end
\fi

Вчена рада факультету філософії та суспільствознавства НПУ ім. Драгоманова
провела засідання у справі професора Більченко. Рішення: тимчасово відсторонити
від викладання без зняття з посади. У справу залучать невідомих
\enquote{експертів} та \enquote{представників держави}, які проаналізують
дискурс Більченко на предмет ідеологічної крамоли. Таким чином адміністрація
університету прагне перекласти відповідальність за остаточне рішення на третю
сторону. 

Звільнити \enquote{по-тихому} не вийшло: надто сильним виявився спротив
студентів та контр-кампанія у медіа.

З одного боку, можна це розуміти як тактичну перемогу, з іншого – як
бюрократичну хитрість, яка унеможливлює пряме втручання адвоката й дає нашим
опонентам час для перегрупування та вироблення менш сміховинної позиції. 

На Вченій раді нам вдалося, попри шалений спротив адміністрації, озвучити наш
Відкритий студентський лист на захист професора Більченко, який зібрав 44
підписи (а не 25, як озвучували опоненти під час засідання). Вважаємо перемогою
вже той факт, що наша позиція прозвучала, нехай і в обстановці шизофренічного
залякування і кафкіанського структурного абсурду. 

Аргументи доносчиків від професури містили: безграмотні та поверхові
квазі-методологічні звинувачення пост-єльцинських шизоїдних бабусь, які на
лекціях говорять про походження культури від колонізації Землі рептилоїдами, у
тому, що Більченко поєднує науку та ідеологічні тренди; закиди щодо
невідповідності її наукових підходів \enquote{паспорту спеціальності} з боку
представників академічної спільноти, для яких \enquote{паспортна} культурологія
зводиться до феноменології кока-коли у серіалі \enquote{Як я зустрів вашу маму};
заклики втрутитись і \enquote{прошерстити}, адресовані представникам спецслужб, із
подальшим перерахуванням статей Кримінального та Адміністративного кодексу, які
нібито були розтоптані; сентиментальні звинувачення у нехтуванні мораллю та
професійною етикою й у нападі на сакральну символіку постмайданних режимів;
слізне апелювання до травматичної пам'яті столітньої та тисячолітньої давнини,
строго у рамках націоналістичного наративу; довгі інтелектуалізовані виступи з
посиланнями на Гусерля, де доводилось, що \enquote{критика тоталітарних тенденцій
ліберал-демократії} та \enquote{політична пропаганда, неприпустима для професора} – не
одне й те ж. 

Протягом усього засідання нам, студентським активістам, які
прийшли зачитати документ на захист свого академічного наставника, посилено
намагались стулити пельку на підставі внутрішніх нормативів засідання. 

У розпалі дискусії у відповідь на моє цитування рядків Галича проти цькування
Пастернака (\enquote{Мы поимённо вспомним всех, кто поднял руку}) якийсь
професор істерично заволав, щоб йому не погрожували. Кілька викладачів,
посилаючись на травматичний досвід своїх предків, які \enquote{пройшли Сибір},
погрожували звільненням, якщо Більченко лишиться в університеті (мабуть,
огидний сентиментальний донос в рамках панівної ідеології на боці більшості
став для них способом солідаризації з подвигом предків-нонконформістів).

Результати голосування: 1 проти, 1 утримався, решта (не згадаю числа) за.
Зразкове поєднання класичної та посткласичної тоталітарної логіки: зовнішня
форма партійного зібрання (з притаманними їй одностайністю та репресуванням
альтернативних голосів), внутрішній зміст шизоїдного постмодерного дискурсу (із
м'якою стигматизацією альтернативних голосів як \enquote{некомпетентних} та
\enquote{ідеологічних}). Знавці антитерористичного й карного законодавства,
пост-єльциністи, СБУшні феноменологи, креативщики, малоросійська номенклатура й
травматизовані бандерівці проти академічної науки в Україні – що це, клінічний
портрет неоліберального університету чи бестіарій кінця історії на наших сумних
теренах?

На наступний день відбулась дискусія між активістами та адміністрацією. Нашу
позицію можна підсумувати зверненням, яке було зачитане на цій зустрічі: 

"Ви керуєтесь бажанням захистити і зберегти факультет. Це благородне прагнення
керівництва, і ми його поважаємо. Проте в даній ситуації єдине, що можна
зробити для захисту факультету – це захистити Більченко. Будь які компромісні
рішення розв'язують руки відверто темним силам, чия мета – дестабілізація і
розвал суспільства. В рамках класичної тоталітарної логіки можна було
розрізнити прапорці, за які не можна переступати. 

Ваша помилка полягає в тому,
що вам здається, ніби ці прапорці зараз існують. Мовляв, принесемо Більченко в
жертву, щоб задобрити злих демонів. На жаль, ми всі зараз серед цих прапорців.

Неможливо переплисти за буйки, коли буйки всюди. Якщо ви демонструєте
слабкість, ви потураєте злу. Наступним можете бути ви або будь хто з
факультету. Будь-що може бути витлумачене як антиукраїнська позиція і звернене
проти вас і всіх нас. Хто наступний? Охранительство не врятує нікого, лише
переконає колективного стерненка у власній безкарності. Якщо ви приймаєте
рішення на користь факультету, захистіть Більченко, захистіть нас. Не потурайте
невігласам і злочинцям."

Наша позиція почута не була: адміністрація лишилась при своєму
\enquote{поміркованому} рішенні (яке, з одного боку, може запустити маховик
подальших чисток з боку радикалів, з іншого – звужує нам простір для
маневрування, підвішуючи ситуацію в повітрі). Натомість, нас, багаторазово
перекручуючи нашу позицію та маніпулюючи фактами, звинуватили у \enquote{лівому
радикалізмі} та \enquote{розколюванні факультету}. А ще терапевтично закликали
не \enquote{підбурювати} студентів, не розхитувати ці \enquote{незміцнілі уми}.

Отже, ситуація застигла в невизначеності. Опонентам: спротив буде
продовжуватись, доки нашого вчителя повністю не реабілітують. Адміністрації:
якщо реабілітації не відбудеться, ми запустимо усі ті сценарії, які згадано у
нашому Відкритому листі, в якості останніх заходів проти академічної цензури.
На цьому поки все.

\begin{verbatim}
  #стопцькуваннюБЖ  #наукабезцензури
\end{verbatim}

\ii{03_02_2021.fb.loza_vladislav.1.pismo_v_zaschity_bilchenko_vchena_rada.cmt}
