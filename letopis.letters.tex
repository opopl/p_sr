% vim: keymap=russian-jcukenwin
%%beginhead 
 
%%file letters
%%parent body
 
%%url 
 
%%author_id 
%%date 
 
%%tags 
%%title 
 
%%endhead 

\ii{letters.27_01_2023}
\ii{letters.06_02_2023}
\ii{letters.14_02_2023}
\ii{letters.18_02_2023}
\ii{letters.to_olena_sugak}
\ii{letters.to_mihail_miroshnichenko}
\ii{letters.to_bereza}
\ii{letters.suhorukova}
\ii{letters.dedova}
\ii{letters.leonid}

\ii{letters.comments}
\ii{letters.rojz_svitlana}
\ii{letters.bludsha_maryna}
\ii{letters.krutenko_maryna}
\ii{letters.sosnovsky}
\ii{letters.rassadina}
\ii{letters.kipcharskij}
\ii{letters.kutnjakov_sergij}
\ii{letters.kutnjakova_maria}

\ii{letters.jusypenko_svitlana}

\ii{letters.mariupol.natali_mihajlovskaja}
\ii{letters.mariupol.olena_vitko}
\ii{letters.mariupol.butt_oleksandr}
\ii{letters.mariupol.fedorova_alevtina}
\ii{letters.mariupol.stomina_oksana}
\ii{letters.mariupol.marfljuk_svitlana}
\ii{letters.mariupol.katja_karnauh}
\ii{letters.mariupol.ponomareva_anastasia}
\ii{letters.mariupol.demidko_olga}

\ii{letters.prepare}

% https://www.facebook.com/permalink.php?story_fbid=pfbid02zMFAg4BunBSszYsWkeX6YoRtgZwnajJHTzQEZmmqwF5kcYaJ2vfC7ZAdxijkFXTRl&id=100006825902516
о! і тут зайчик! А я нещодавно був в Музеї Історії Міста Києва, там на
четвертому поверсі стоїть той самий Зайчик, що був на Театральній Площі в
Маріуполі в 2019 році, цілий-цілісінький ))) всього його обфоткав ))) його
автор - Тетяна Ніколаєва. А історію цього кролика я ще не дослідив, але це один
із зайчиків, який приїхав в Маріуполь із Києва, це він раніш експонувався на
Софійській Площі



\ii{letters.bondarchuk_svitlana}

%https://www.facebook.com/permalink.php?story_fbid=pfbid0sbMsAukawJAnZeA8fdWBvWhfxh4pusncRbwfHWN2sQYRxNnjbGn4FkQ5fDmWm4ajl&id=100087641497337

Надіє, дуже важлива книга! Мені, якщо чесно, як людині, що на щастя не бачила
наочно всього цього пекла, хоча я багато читав і читаю постів про Маріуполь, і
я відчуваю ваш біль, який так, відчуваєш, навіть якщо всього цього не бачив,
важко коментувати, оскільки подумки питаєш себе, що і як варто казати людям, що
все це пережили. Але... я мушу зізнатись в такому. Ще до війни я бачив вже
приблизно все те, що було в Маріуполі... у фільмі. Польському фільмі. Фільм
називається Місто 44 (Город 44). Страшний фільм насправді, і я в той момент
зовсім не думав, що всі ті жахіття, що були показані у тому фільмі, колись
будуть повторені вже в Україні. І цей фільм про трагедію Варшавського
повстання. Так, трагедія такого ж масштабу...  Хоча інша країна, інша історична
епоха, інші конкретні обставини. Так, у 1944 році, коли Варшава була під
німецькою окупацією, поляки підняли повстання проти нацистів, яке було в
результаті надзвичайно жорстоко придушене. Надзвичайно жорстоко. Місто було в
вогні, зруйновано, близько 100 000 поляків було в результаті вбито. Поляки
дотепер пам'ятають про це. Але... Варшава... якщо ходиш вулицями цього міста -
красивого, гарного міста... не відчуваєш, що тут колись коїлось... Так
заведено, що Міста мають внутрішню, так, внутрішню велику Силу... яка все
подолає і переможе! І так! В кінці фільму - якщо знайдете час колись подивитись
- дуже зворушливо. По-перше... (якщо подивититесь, побачите самі... - все
розказувати наперед напевне не треба... )  І по друге, що стосується власне
Міста Варшава. Кінець фільму... Палаюче знищене місто зникає... І постає
Варшава наших днів... І оцей перетворення із палаючого міста у сучасне мирне
усміхнене європейське місто колись мене надзвичайно зворушив, аж до сліз... І
так само Маріуполь. Місто залікує свої рани, так само як і маріупольці. І
колись, через десятиліття те, що відбувалось, залишиться лише у книжках та у
відео та фільмах, та пам'яті народу. Але приклад поляків дуже показовий. Вже
скільки років пройшло, а вони зняли цей фільм зовсім недавно, хоча стільки
років вже пройшло! Але вони пам'ятають, так, і дотепер, і будуть пам'ятати
завжди! 

от ви кажете, що зоряні війни, то є нудно... знаєте. А я почну здалека... з
Природи. Ось природа... виходите на вулицю... вдихаєте свіже повітря...
радієте... Сонечко світить... чи не задумувались ви про те, як все прекрасно,
взаємопов'язано між собою. Сонце світить - дощ ллє. Земля всмоктує в себе
вологу, і дає життєдайні сили, щоби прекрасна квіточка - скажімо, десь у Києві
на Співочому Полі - або у бабусі в селі -  проросла. Потім з'являються люди,
висаджують квіти, створюють парки, як от парк Кейкенхов в Голландії, або ж
дитячий парк Веселка в Маріуполі - справжня міні Голандія у прекрасному місті
біля Моря... Так от. Сонце світить, вітер віє... дощ ллє... Саме по собі Сонце
- то є величезна розжарена куля, на якій немає ніякого життя, там панує лише
Вогонь і Смерть... Так, Величезний Страшний Вогненний Океан під назвою Сонце,
розміром із тисячу таких планет як Земля... Але... світло від Сонця досягає
Землі, і дає в результаті життя прекрасній тендітній квіточці... Одне - велике,
неймовірно велике, дуже гаряче і смертоносне - дає в результаті життя чомусь
дуже малому, тендітному, крихкому... Так... В Природі, у Всесвіті все
взаємопов'язане... мале пов'язане з великими... мертве - з живим, гаряче - з
холодним... А як можна назвати те, что все це уособлює собою? Одні кажуть -
Матір-Природа, інші кажуть - Бог, Творець. А в цьому фільмі центральним
лейтмотивом проходить ідея Сили... Сили, що все об'єдную між собою... Дуже
філософський фільм насправді. І показовий діалог з передостанньої, восьмої серії
фільму (Останні Джедаї)... 

Таймкод - 49-10

Люк - тянись до неї своєю душею...  що ти бачиш?
Рей - острів... Життя... Смерть та тлін... початок нового Життя... тепло...
холод... мир... насилля... 
Люк - а що їх об'єднує?
Рей - баланс... і енергія... Сила...
Люк - а в тобі самій...
Рей - і в мені самій... та ж сама Сила... 
Люк - це і є твій урок... Сила не є власністю джедаїв...
ідея, що зі смертю джедая вмирає Світло, є хибною... ти відчуваєш це!? 

Ну от. У Всесвіті все пов'язано. Як колись писав англійський поет-романтик Вільям Блейк, 

Бачити світ у зерні піску, Небо — у квітці синій, В своїй долоні безкрайність усю І вічність в одній годині.

А Сила - то є назва тому, що все об'єднує
в одне ціле, і далекі галактики, і розжарене Сонце, і тендітну квіточку десь у бабусі на городі.
Ось у цьому власне і є основна ідея фільму, ідея Сили, так, ідея Сили.

%https://t.me/maripol_hope/1291

Дуже влучно сказано про останню крапку, що поставить Життя! Бо дійсно, так і
буде! Життя переможе Смерть. І щоби це сталось насправді... потрібно...
навчитись смерть перемагати у собі, в душі, в серці... перемагати відчай,
розпач, відчуття безнадії... Розумію, що для маріупольців це важко зараз... Я -
киянин... дивлюсь все це зі своєї сторони... Але... знаєте. Я от десь недавно
прочитав у одного автора - він каже про нескінченний постійний потік фото  в
телеграмі про зруйнований Маріуполь. Я скажу так... Це звичайно дійсно так.
Але... Інтернет - така штука... що тут ти насправді сам вибираєш, що ти можеш
дивитись... Я, коли заходжу в телеграм або фейсбук, звичайно, також бачу багато
фото або історій поганих про Маріуполь. Але... також! Нескінченний, дійсно
реально нескінченний потік радісних, гарних фото, історій про Маріуполь на
різних ресурсах та в абсолютно різних ракурсах! З яких на нас дивиться саме
Життя! Так, саме Життя, бурхливе невгамовне Життя, а не Смерть! Більше 200
років Життя, акумульованого у цих прекрасних та різноманітних фото чудового
міста біля моря, так - 200 років - бо фото, відео, історії, враження - це
акумуляція, сумарний ефект величезної Духовної Сили Міста, що стоїть біля Моря
вже більше двох століть! у порівнянні з цим роком смерті... Що ж переможе
врешті решт? 200 років Життя, чи рік (два-три) занепаду та смерті? Я думаю,
Життя! Так, Життя! Українське Життя в Українському Маріуполі! 💛💙💛💙

% 23.02.2023
% https://t.me/maripol_hope/1307

Доброго ранку! 

Щодо книги - то є велика подія. Обов'язково прочитайте, книжка сама по собі вас
не вб'є!  Так, можливо, буде боляче читати, але це лише книжка. Її завжди можна
відкласти в сторону, почитати пізніше, повернутись, перегорнути сторінки,
закрити, подумати, уявити, згадати...  потім знову повернутись до тих чи інших
рядків... Книжка - то є велика духовна річ, яка дійсно здатна лікувати душевні
рани! Але... у мене тут є такі думки. Щодо повернення у звільнений Маріуполь.
Скажу таке, це напевне не дуже вам сподобається, що я скажу. Зауважу наперед,
що я ніколи не був у Маріуполі, і тому... як це парадоксально не звучить, він
для мене такий самий живий - на фото, відео, як і до війни. Відсторонена
позиція, так сказати, тут допомагає це усвідомлювати.  Так, я співчуваю
маріупольцям душею і серцем, але я також знаю - що справжній Маріуполь живий,
хоча він і залишився таким лише у пам'яті та у фото та відео. А щодо звільнення
Маріуполя. Ви мрієте про повернення, уявляєте як все буде, як ЗСУ звільнять
місто... от можна уявити собі, ...  найоптимістичний варіант... російські
війска розбиті вщент, путін вмер або його вбито, військова перемога України
здобута, росія підписує принизливий для себе мир, навіть погоджується на
репарації, і виводить свої війська взагалі... Ви купляєте квиток... радісно
їдете на поїзді Київ-Маріуполь, наприклад... і знаєте. тут на вас чекає
смертельна небезпека. Абсолютно і без жартів.  А чому... Тому що ви приїдете в
напів-мертве місто, в місто-примару. ЗСУ може звільнити місто фізично, але
звільнити місто в душі і серці - то є справа персональна кожного, справа
духовна, інтимна, тут ніякий хаймарс не допоможе. Так от... Повернетесь... В
місто, яке стало Кладовищем, і в місто, яке є лише Тінню Маріуполя. В місто,
яке зараз насичене невидимою страшною духовною отрутою... І це є страшна
небезпека для душі, для серця... Для тих, хто вживу знав його зовсім іншим...
Оскільки це може просто вбити на місці... як тільки ви приїдете туди,
пройдетесь вулицями, дійдете до тих місць, які колись робили вас щасливими... і
усвідомите на повну і всерйоз, що повернення до того минулого, яким ви його
пам'ятаєте, вже ніколи не буде... В таких випадках людина може просто вмерти на
місці від атомного вибуху у серці та душі, або зійти з розуму і назавжди...
Так, ваше найрідніше місто може вас вбити, якщо ви душевно, духовно не станете
як слід сильними, готовими до цього повернення !  Так що...  Звільнення,
повернення Маріуполя у душі, у серці повинне передувати справжньому поверненню
у реальності.  І чим більше спогадів, чим більше всього, що вас поєднує із
Маріуполем, тим страшніша ця небезпека, і тим більша ймовірність, що без
відповідної розумової і душевної підготовки, духовної роботи, мертве отруєне
місто вас просто вб'є на місці, як удар блискавки в дім без громовідводу.

%Дякую.. 

%Я теж про це думаю! Додому хочется, але.. Я вирішила після деокупації поїхати
%на тиждень, подивитися, придивитися ЯК мені, дитині .. Ми зможемо там жити!? А
%який контингент навколо буде? Не відразу повертатися. Можливо там буде жахлива
%енергетика💔😣 Але як то кажуть ми не перші .. багато міст відновили після війн
%і зараз це квітучі міста!


Доброго дня! Щодо контингенту, відповідь на коментар вище... як там буде... не
в цьому річ, абсолютно не в цьому. А в Ваших персональних відносинах, так,
інтимних душевних відносинах із Вашим Містом... Розумієте. Неважливо насправді,
який там буде контингент на момент вашого приїзду у звільнене місто. Головним
тут є ваша зустріч із Містом один-на-один.  І так... щодо контингенту.
Припустимо, що... ось...  Маріуполь звільнено, більше того, всі зрадники та
колаборанти або заарештовані, або просто повтікали...  Зібрався цілий потяг
друзів, маріупольців, всі щасливі, радісні, нарешті ми їдемо додому!!! Поїзд
прибуває на розгромлений вокзал...  перші гнітючі думки... руйнування усюди,
незважаючи на те, що місто вже наше, і гордо майорить вже наш найгарніший у
світі прапор над будівлею мерії... потім, ви всі розходитесь... або гуляєте
разом...  спогади, минулі жахи оживають... ось тут було оце, ось тут було це.
Тут у дворі поховані мої друзі, там заживо згоріла бабуся... Тут був будинок із
муралом...  нема вже ні будинку, ні муралу, ні тих людей, що там жили....
Розумієте...  І Місто починає вас душити, всю вашу прекрасну компанію.
Безжалісно душити, безжалісно бомбардувати вашу душу, ваше серце. Починає
душити всіх вас поодинці, жорстко, безжалісно розбиває вашу єдність... Береться
за кожного, і починає роздирати ваші душі, ваші серця на частини... Мертве,
зруйноване - хоча і звільнене місто, але ще наскрізь отруєне місто... береться
за кожного маріупольця, і починає творити над ним (нею) свій Страшний Суд...
Спогадів, з'єднаних із жорсткою реальністю, все більше і більше, як смертоносна
сніжна лавина в Карпатах зимою, вони сходять на вас, зносячи все на своєму
шляху... вони починають душити вас звідусіль...  Так!  Пекло, через яке ви
колись пройшли, і з якого ви вибрались живими, і яке зараз залишилось лише у
книжці Надії та ваших спогадах та фото і відео в інтернеті, повертається
вживу... В результаті три можливі варіанти - або ви берете квиток і наспіх
тікаєте з міста щодуху, щосили, щоби туди більше ніколи не повернутись... або
ж...  Місто просто вбиває вас насмерть...  через інфаркт, інсульт, просто
вбиває на місці... або ж.. Ваша Внутрішня Сила, так, Ваша Духовня Броня - дає
відсіч демонам темряви і відчаю... (бажаний варіант) Ви виходите переможцем
врешті решт... І Місто відпускає свої сталеві єжові рукавиці від вас, і каже
вам - залишайся тут, друже. Залишайся тут, я, Українське Місто Маріуполь,
найгарніше місто біля моря у світі, буду з тобою скільки ти забажаєш. Залишайся
тут, вкладуй свою Душу, своє Серце в мене, в Місто Маріуполь, і я тобі віддячу
сповна! Будуть тобі і фонтани, і новий парк Веселка, і новий ДрамТеатр, і
радісні веселі новорічні трамваї, і просто трамваї, і тролейбуси, і може навіть
метро! все буде, не сумнівайся! Ось так я це все бачу...

""Я згодна Тому і кажу треба поїхати, а так ми просто фантазуємо" ... -  а що
ви будете робити з тим, що цим мертвим містом гуляють тисячі і десятки тисяч
невинних загублених душ, душ людей, померлих жахливою смертю, що день і ніч
шукають умиротворення і спокою, але не знаходять його? Що ви будете робити,
коли ці невинно вбиті - а їх тисячі, десятки тисяч прийдуть до вас в гості,
заполонять всю вашу душу, почнуть розривати її на частини... абсолютно не
питаючи вашого дозволу? Розумієте, тут немає жартів. Все це дуже-дуже серйозно.
Місто не впустить до себе назад тих, хто не готовий до цього. Я взагалі вважаю,
зі своєї сторони, що на даний момент (можливо це звучить дуже зухвало для вас,
але я просто кажу зараз, що думаю) лише декілька людей із усіх маріупольців
переживуть своє повернення до Маріуполя, витримають свою першу зустріч із
мертвим отруєним Містом. Наприклад, це пані Надія і пані Наталія. А чому. Пані
Надія написала книжку, вклала величезну кількість сил, душі і серця. Пані
Наталя так само - дуже сильна жінка, стільки пережила, вона витримає, вона
стільки зусиль вкладає в проєкт голоси мирних, стільки зусиль докладає, щоби
зберегти всі ці історії. Так! Книжка та пости, проєкт, душа і серце - вкладені
в усе це - це і є та духовна броня, яка дасть цим конкретним людям достатній
захист проти смертоносних обіймів мертвого Міста... А щодо інших
маріупольців... я не знаю. Я бачу по коментарям, що навпаки, багато людей
тримають свій біль всередині... Мої судження тут лише по тому, що я бачу як
стороння, але небайдужа людина... Справжнє повернення у Маріуполь - це ваш
духовний Еверест. Якщо ви не подолаєте спочатку цю гору всередині себе, тобто
не подолаєте, хоча б частково свій біль, свої страхи, свої спогади, не
збалансуєте більш-менш свою душу... поки ви на безпечній відстані від Міста, то
справжня зустріч із зруйнованим мертвим Містом-Кладовищем буде для вас просто
фатальною.

Доброе утро! Ох! Надеюсь я еще не надоел всем тут со своими мыслями! Знаете...
сейчас как раз годовщина... не спится совсем... итак... вы хотите приехать
туда, наверное припасть к тому, что осталось еще от дома... посидеть,
поплакать... правильно я понял? чтобы вот так сидишь день у дома и плачешь... а
потом как бы слезы наверное должны пройти, вы думаете, и станет легче?
знаете... если бы Ваш Город был цел... если бы живая ткань Города как Живого
Организма не была бы так жестоко разрушена... если бы дело было только лишь в
одном Вашем доме... ну вот скажем... как в Днепре... ракета попала в Ваш дом.
Предположим, что вы жили когда-то в том доме, а потом - уехали. Бывает так по
жизни... А вы далеко... Дом разрушен верно, погибли Ваши друзья, знакомые,
частичка Вашего детства. Вы с ужасом об этом узнали. Все - дома Вашего детства
нет. Но! Все остальные дома целы, Город цел. Набережная цела, офисные здания,
магазины, университеты, школы, все работает. И вы приезжаете издалеко... дом
уже сносят... А вы стоите недалеко, в Днепре... и прощаетесь... Туда-сюда снуют
люди... Ездят машины. Город живой. Конкретно ваш дом умер - но Город живой. В
этом случае это бы получилось. Вы бы приехали - поплакали. И попрощались бы.
Понимаете. Потому что живая энергия Города, который живет по прежнему, помогла
бы Вам пережить все это. А здесь... все не так. Дело то не только в том, где
жили Вы. Дело также в соседнем доме. Дело также в доме Натальи Дедовой или
Надежды Сухоруковой, дело также в убитых тысячах, дело в сотнях варварски
уничтоженных домов дело в душах, которых никто не похоронил, не помянул как
следует.  Дело в том, что тот Город, который должен быть живой, и который вы
помните, по факту мертв. Понимаете, не просто конкретно ваш дом уничтожили, а
уничтожен ваш живой Город. И это колоссальная негативная психическая и душевная
энергия, которая будет убивать наповал тех, кто не позаботится зараннее о том,
как противостоять этому. Как вирус, знаете, убивает, если не вколоть себе
вакцину... Да... Прощаться, выплакивать нужно не только ваш дом, а вообще весь
тот Город, в котором вы жили! Вы говорите... приедете, поплачете... А Вы не
думали, что, приехав... вы сядете или упадете как камень... чтобы больше
никогда не встать? ... Понимаете... был живой Город. А сейчас его место заняла
его Тень... И вот эта Тень... да, Тень Города... она ждет вас, да, там на месте
ждет вас... чтобы, когда вы приедете... забрать к себе навсегда, чтобы сожрать
вас целиком... Именно там... Вот знаете... бойтесь вот этого - именно там на
том клочке, потому что этот клочок может стаь для вас могилой по настоящему...
Ну ладно. Насчет выплакать. Выплакивать нужно начинать прямо сейчас! А лучший
способ наверное... Прочтете книгу Надежды - а потом... ну возьмите и напишите
тоже книжку... и про ваш дом... и про ваш живой довоенный Город, про то, чем вы
занимались, чем жили, про ваших друзей. И так далее!  Вам же есть что
вспомнить, не так ли? Так мало сейчас книжек про Мариуполь!  Начинайте
вспоминать, что с вами было в том, живом Городе!  Вспоминайте, вспоминайте,
записывайте, записывайте, как следует распишитесь. Выплачьте свой Город через
Слово!

%09:41:13 25-02-23
Доброго ранку! Так, звісно добре поспати... Але ось ранок, час вставати!  Що я
хотів сказати щодо війни та взагалі. От Ви недавно написали, що Ваша мама не
вірить у ракетну атаку і читає книжку Надії. Але спочатку... відкрив я ленту,
тут одна людина пише - можливо Ви цей допис теж вже бачили -

" ... Захоплююся мамою. В річницю війни вона, знаходячись за сотні кілометрів від
дому, в незнайомій країні, незнайомій квартирі, з невідомістю що далі і як
далі, вже планує, як вона повернеться на роботу в дитячий садок Маріуполя щоб
допомагати відновити, щоб знову висадити квіти, щоб знову оживити рідні стіни,
рідні вулиці. Людей не оживити тільки.  А ще вона боїться того, що «оживити» не
вийде, тих, хто сліпий, глухий зараз, хто не зрозумів досі. Але ця її Надія в
мене викликає захоплення. І це в той час, як в будинку її мами, теж, до речі,
Надії, російські солдати сплять, їдять, може дивляться наші родинні фото,
топчуть ногами килими, візерунки яких я в дитинстві обводила пальчиком щоб
швидше заснути ... "

І також, щодо Вашої мами - Ви написали - тільки Ваша мама не вірить у ракетну
атаку. Я так скажу.  Розумієте. Ви вірите у ракетну атаку, і не можете заснути,
все чекаєте. І нічого не стається. Дивно, правда!  А я не вірю у ракетну атаку,
і спокійно вчора гуляв Містом Києвом. До речі, навіть заїхав на Петрівку (),
купив собі книжок!  Розумієте... в ракетну атаку вірити не треба, вірити можна
в Бога. Вірити можна в дітей, в себе, в свою Країну, в Перемогу України!
Насправді, кожного дня може бути атака, і зараз, коли я пишу, і завтра, і через
тиждень.  Не треба тут чіплятись до дат. А щодо Києва та дому. Ви втратили свій
дім у Маріуполі, були змушені переїхати, Київ став зараз Вашим новим домом.
Звичайно, Вас тягне додому в Маріуполь, Ви день і ніч читаєте новини про
Маріуполь... але... Скажіть...  Наскільки Ви добре знаєте Київ, так,
тисячолітній Вічний Город Київ? Чи були Ви наприклад у Софійському Соборі або ж
чи знаєте Ви про Бабин Яр, місце страшної трагедії масштабу Маріуполя? Чи
спускались Ви коли небудь із свічкою у печери Лаври, або ж чи роздивлялись коли
небудь фрески на стінах Володимирського Собору? Поки Ви тут, чи знайшли Ви час
прошло піти і просто посидіти подивитись на Дніпро, подумати, ось так... Просто
йдеш дивишся на Дніпро і про все на світі забуваєш...  Чи маєте Ви вдячність
Києву, так, Києву як Місту, що дав Вам прихисток? Розумієте... У Києві зараз
мир. Так, взагалі то йде війна.  Але в Києві зараз мир. І Київ...  все, що
зараз відбувається. Вже бачив за свою історії багато разів.  Його багато раз
спалювали... тут коїлись страшні людські трагедії... падіння 1240 року,
громадянська війна, окупація, Бабин Яр, Чорнобиль, розстріл на Грушевського...
Місто все це пам'ятає. Але, знаєте. Київ таке місто, що воно відкриває свої
двері всім, хто приїздить до нього. Це дуже щедре, неймовірно щедре Місто. І
воно дає можливість жити у ньому людям, яким в принципі абсолютно нецікавий ні
сам Київ, ні його історія. Але... Матір Городам Руським немовби каже... друже,
жити тут скільки хочеш, роби що хочеш... але все ж таки, я маю надію...  що
коли небудь ти відволікнешся від своїх справ, і поки ти живеш тут, ти почнеш
відкривати для себе Місто... повір, друже, воно того варте! Якось так...

%08:14:38 25-02-23
а еще они говорят друг друг, когда прощаются - дуй! дуй дуй! (doei ) когда я в
свое время такое слышал, у меня всегда была ассоциация с "дуй отсюдова" )) до
свидания же звучит tot ziens (тот зинс) кстати, интересно, что дуй - это
производное от глагола doen -то есть, делать, работать. То есть, буквально, как
бы, прощаясь, голландцы друг друг желают работать дальше. Неудивительно,
кстати, потому что голландцы - очень трудолюбивая нация, закаленная постоянным
противоборством с суровой природой, а также соседством с гораздо более мощными
и сильными странами-соседями рядом (Германия, Франция). Удивительная страна
Голландия! Такая маленькая, но такая сильная, мощная, ухоженная страна! У
них... знаете, есть пословица... Бог создал Землю... а голландцы Голландию.
Кстати, Вы никогда не задумывались, что любой язык несет как бы структуру
мировозрения людей, народа, который на нем говорит. Вот например, Франция. Язык
красивый, изысканный, там даже прямое недвусмысленное приглашение девушке
переспать звучит красиво и так по французски! А вот Голландия. Простой язык.
Звучит грубо, не очень красиво на ухо. Практичный простой язык без излишеств.
Точно так же как и голландцы. Простые прямые открытые и очень практичные люди.
Английский - туманный Альбион. Вежливость, сдержанность, учтивость, уважение к
индивидуальности и личности... загадочность, намеки... коварство, лживость,
лицемерие... Слова пишутся и произносятся с тысячами исключений, слова как бы
сохраняют по возможности свою индивидуальность... Грамматика то простая, а вот
слова сами по себе - их произношение и написание - ну уж никак не хотят вести
себя по одному правилу! Прям как люди! И хорошие, и плохие стороны английского
национального характера, все это отображается в языке... Ну и немецкий. Строгий
четкий язык, более строгий с точки зрения грамматики, чем украинский или
русский. А ведь немцы так известны своей упорядоченностью, педантичностью...
Как то я в свое время изучал немного немецкий. Потом понял... трудное это дело,
хотя и интересно. Все эти приставки, которые улетают в конец... Как бы знаете,
воспитывается вежливость к собеседнику, терпение... потому что пока человек не
закончит свою речь, не станет окончательно понятно, что же он именно имеет в
виду (улетевшая в конец приставка  ab/auf/an может поменять весь смысл... ) Вот
так. Ну а возвращаясь к Голландии, ниже фото из цветочного парка Кейкенкоф )


%14:45:09 25-02-23
%Natalya Dedova
%Kiev Fortress доброго ранку! Про маму моєї колеги - читала.

%По Києву - гуляла. Тільки що зі свічкою - ні.

%Щодо ударів. Знаєте, що таке - історична пам'ять? А знаєте, що таке три тижні
%щоденного безперервного вбивства та виживання? Я та мій син досі присідаємо,
%коли сусід зверху щось кидає. І для мене сирени - це не просто щось там за
%вікном. І попередження про ракетний удар - не я його вигадала. І я б не вірила
%в ракетну атаку, якби з 2014 не жила в стані війни. Ось в чому справа.

%А щодо Києва. Так, прихистив. І це прекрасно. І за це - дякуємо. Але. Я живу в
%чужій квартирі, сплю на чужому дивані. Мій син спить взагалі на кухні. Адже не
%хоче спати зі мною. І сплачую я за квартиру майже всю свою зарплату. І працюю в
%нашій родині - тільки я. 24/7.

%А щодо Бога. Після масового знищення людей, зокрема маріупольців, я не розумію,
%де в той час був Бог. І чому карлик досі живий. А мої близькі - ні. Якось я
%порахувала. Дуже приблизно. Мінімум 40 людей із мого оточення - загинули.

%А про Бабин Яр - знаю. Влітку 2021 ми останнього разу приїздили всією родиною
%до Києва. І ходили саме туди. І я читала кожну табличку. І вдивлялася в кожне
%обличчя. І думала, не дай Бог..... А воно - ось як вийшло.


Доброго дня! Дякую! Знаєте... я ризикую, що я пишу надто багато слів, ризикую
Вас втомити та прогнівати... Але я все ж таки допишу. Бог... він одному дає
одне, іншому - інше. Не нам Його судити, не нам міряти Його діяння. Він нас
створив по Своїй Подобі та Образу, і Він може кожного з нас знищити будь-якої
миті. Кожного з нас, і божевільного карлика, і Вас, і мене. Де Він був під час
того, що відбувалось у Маріуполі... Я відповім так, щодо своїм уявленням про
Нього. Він був у тому самому місці, коли стався Бабин Яр, коли було знищено
більше 100 000 киян, євреїв, комуністів та безпартійних. І де стався розстріл
на Майдані, Він також був недалеко. І Він був там, де було придушене Варшавське
повстання 1944 року, яке по трагічності має такий розмір, як і Маріуполь, і по
якому вже в наші дні був знятий фільм Місто 44. Він був у тому самому місці,
коли розстрілювали Розстріляне Відродження, і Він також був у тому самому
місці, де... Ви і та Ваш син врятувались, а Ваш чоловік та ваші друзі та просто
сусіди - на жаль, загинув. І є поруч з нами, усюди, і вдень і вночі, і буде
завжди, во Віки Віків... І знаєте. Сьогодні день народження Лесі Українки. Вона
з десяти років хворіла на туберкульоз, все життя страждала від цього.  Ось
такий у неї є вірш.. Ви його напевне знаєте. Без надії таки сподіваюсь. Contra
spem spero

Гетьте, думи, ви хмари осінні!
То ж тепера весна золота!
Чи то так у жалю, в голосінні
Проминуть молодії літа?

Ні, я хочу крізь сльози сміятись,
Серед лиха співати пісні,
Без надії таки сподіватись,
Жити хочу! Геть, думи сумні!

Я на вбогім сумнім перелозі
Буду сіять барвисті квітки,
Буду сіять квітки на морозі,
Буду лить на них сльози гіркі.

І від сліз тих гарячих розтане
Та кора льодовая, міцна,
Може, квіти зійдуть - і настане
Ще й для мене весела весна. 

Я на гору круту крем'яную
Буду камінь важкий підіймать
І, несучи вагу ту страшную,
Буду пісню веселу співать.

В довгу, темную нічку невидну
Не стулю ні на хвильку очей -
Все шукатиму зірку провідну,
Ясну владарку темних ночей. **
Так! я буду крізь сльози сміятись,
Серед лиха співати пісні,
Без надії таки сподіватись,
Буду жити! Геть, думи сумні!

% 18:09:21 25-02-23
% #Мариуполь #Надежда Chat
Добрый день! Важные темы затронули... Чудесные фото прекрасного удивительного
Города! А насчет темы... психологии того, во что превратились россияне.
Знаете... вот иногда хочется найти какое то ключевое слово... которое бы как то
емко передавало суть... Матрица. Когда был снят такой фильм, Матрица. Про то,
что наш мир как бы на самом деле Матрица... Здесь, конечно, всего лишь
аналогия, воображаемое понятие, но суть довольно близка... Орки создали себе
свою Матрицу, да, мощное информационное поле - Смысловое Когнитивное Поле,
которое их целиком поглотило, и которое управляет их мыслями, поведением.
Смыслы, элементы этой Матрицы - это Слова, Предложения, Конструкции, ну.. типа
хохлы-продались-америке, россия-великая-страна, путин-президент-мира,
нацисты-бандеровцы ляляля, и понимаете. Эта Матрица, Виртуальный Мир - это...
Раковая Опухоль Интернета... Они... создали эту Матрицу, а теперь Матрица
сожрала их мозги целиком, и живет своей жизнью, питаясь их мозгами. Знаете,
такой невидимый ментальный вирус, ментальный информационный вирус массового
поражения... У меня в этом смысле был в прошлом году довольно интенсивный опыт
изучения поведения российских зомби в соцсети одноклассники. Специфика той сети
в том, что там ты вплотную сталкиваешься лоб в лоб с этими зомби, там можно
писать что хочешь и как хочешь, и в этой соцсети - так она устроена виртуальное
общение наиболее приближено по динамике к реальному человеческому общению.
Понимаете... Я уже так данных насобирал наверное на диссертацию... И в конце
концов видишь, насколько же безумно смертоносным может быть Интернет...
насколько он многократно усиливает влияние пропаганды, ложных бредовых идей...
Да! Скажу я... состояние российского общества в плане пропаганды - это промывка
мозгов похлеще, посильнее будет, чем во времена Геббельса! А почему это так -
отдельная большая тема... Вот, чтобы начинать все это понимать в таком
основательном ключе, есть книжки... ну например такие...

%20:32:26 25-02-23
Спасибо! Ох! А касательно зомби, конкретно насколько они безумны. Они реально
безумны, и это можно показать на примере... Вот реальный пример из моих данных
из ок. Щас туда зашел, порылся в своих постах там, достал. Тут сначала
небольшое пояснение. Я под эту сеть специально понаписывал текстов - штук 60
наверное, которые бы били по этой самой Матрице в целом. Как можно сильнее били
по зомби. То есть, насыщенный сильный текст, зараннее подготовленный, который
бьет по ключевым точкам российской пропаганды в мозгу этих безумцев, бьет
конкретно, персонально. И ты приходишь в эту сеть, и начинаешь в чатиках
методично их бить, бить и еще раз бить. Вплотную берешься и начинаешь их
трусить Словом, просто выносишь им мозги, целенаправленно и систематично. И
тут... оказывается, что вместо живых людей с Волей и Разумом... это зомби,
просто какие то жалкие мешки с дерьмом в голове вместо мозгов и разума.  И эти
зомби... когда их так атакуешь... они пытаются защищаться... и одним из методов
самозащиты у них было переделка моих текстов под свои смыслы. Ну то есть - если
в тексте россия виновата - они заменяли на Украину, российские нацисты - меняли
на "усраинский (хохляцкий) нацист" - меняли что виноват, кто прав.  И
отправляли назад переделанный текст. Но... на выходе у них реально получался
уже бред! Понимаете, бред, даже с точки зрения российской пропаганады! Безумие
и шизофрения, так сказать, выделенные в концентрированной форме! Ну... а ниже
фото - мой изначальный текст и безумная переделка зомби.

% https://ok.ru/profile/586956477134/statuses/155969245918670
Леся Українка 💛 💙 💛 💙 💛 💙 💛 💙        

Слава Україні, друзі!!! Вибачте, я в останній час заглибився в інші справи, і тут перестав так активно писати. Вибачте, що не всім відповідаю вчасно... Вчора було річниця дуже сумних подій, які докорінно змінили все наше життя... І війна все йде і йде... Ну а сьогодні - так, сьогодні... хочу нагадати, що у цей день, 25 лютого 1871, народилася Леся Українка - наша геніальна поетеса.

Ось трохи нагадаю нижче...

Леся Українка (справжнє ім’я — Лариса Косач) народилася 25 лютого 1871 р. у місті Новоград-Волинському. Вона була дочкою поетки та письменниці Олени Пчілки та племінницею філософа, політика та письменника Михайла Драгоманова, який значною мірою вплинув на формування її світогляду. До 13 років Леся мешкала на Волині — у Луцьку, Ковелі та у батьківському маєтку в с. Колодяжному, пізніше — у Гадячі, а від 1884 р. — у Києві. Тут вона долучилася до літературної групи «Плеяди» та до «Літературно-артистичного товариства».

Від 10 років Леся хворіла на туберкульоз кісток, тому навчалася вдома, проте часто виїздила на консультації та лікування до Криму, Болгарії, Італії, Австрії, Німеччини, Литви, Швейцарії, Єгипту, Греції, Польщі, Грузії. Саме у цих мандрівках вона написала багато своїх віршів і навіть цілі поетичні цикли, як-от: «Подорож до моря» (1888), «Кримські спогади» (1893), «Кримські відгуки» (1897), «Весна в Єгипті» (1910), «З подорожньої книжки» (1911).

Леся навчилася читати у 4 роки, у 5 років почала компонувати невеликі музичні п’єси, у 9 років написала свій перший вірш «Надія» і у 13 років уперше опублікувала свій вірш «Конвалія» (у львівському журналі «Зоря»). У 14 років вона видала свою першу книгу — збірку перекладів українською мовою оповідань Миколи Гоголя, у 19 років для своїх молодших сестер написала підручник «Стародавня історія східних народів», а свою першу збірку віршів «На крилах пісень» видала у 22-річному віці.

... так! ну а нижче одна з моїх улюблених поезій Лесі Українки - насправді, відома річ дуже! - але тим не менш! 

Contra Spem Spero!

Гетьте, думи, ви хмари осінні!
То ж тепера весна золота!
Чи то так у жалю, в голосінні
Проминуть молодії літа?

Ні, я хочу крізь сльози сміятись,
Серед лиха співати пісні,
Без надії таки сподіватись,
Жити хочу! Геть, думи сумні!

Я на вбогім сумнім перелозі
Буду сіять барвисті квітки,
Буду сіять квітки на морозі,
Буду лить на них сльози гіркі.

 І від сліз тих гарячих розтане
Та кора льодовая, міцна,
Може, квіти зійдуть — і настане
Ще й для мене весела весна.

Я на гору круту крем'яную
Буду камінь важкий підіймать
І, несучи вагу ту страшную,
Буду пісню веселу співать.

В довгу, темную нічку невидну
Не стулю ні на хвильку очей —
Все шукатиму зірку провідну,
Ясну владарку темних ночей.

Так! я буду крізь сльози сміятись,
Серед лиха співати пісні,
Без надії таки сподіватись,
Буду жити! Геть, думи сумні!

Так що... якщо у вас погано на душі... не сумуйте! Читайте твори, поезії Лесі, і це вам безумовно підніме настрій і окрилить душу в такі тяжкі і буремні для усіх нас часи! Все буде добре! Все буде Україна! Слава Україні! 💛 💙 💛 💙 💛 💙 💛 💙        


% 13:36:40 26-02-23
% https://t.me/maripol_hope/1317
знаете, не совсем так, если говорить как бы более четко. У них есть как бы и
мнение, и позиция, которое они со злобой и остервенением пихают в разных
местах, даже вот в этом чатике могут залезть. Но... это "мнение", эта
"позиция"... являются ложными. А почему. А потому что эти так называемые мнение
и так называемая позиция являются исскуственно вживленными посредством
многолетней массовой и продуманной психологической обработки. Вот кстати... Я
не знаю, насколько я смогу здесь объяснить... но в физике есть понятие
минимума. То есть, любая система физическая, как например, маятник... или
намного более сложная система, как человеческий белок, стремится, согласно
уравнениям, занять наиболее низкое, выгодное с точки зрения энергии, положение.
Проблема в том, что этих минимумов может быть много, очень много. Есть
основное, истинное положение, и есть куча других минимумов, в которых система
также может застрять. И чем сложнее система, тем больше этих ложных,
неправильных минимумов. Это в природе мы можем наблюдать в форме стекла. - те
же самые молекулы, атомы, но они образуют не метал или кристалл (нижайшее
положение, наиболее устойчивое), а именно стекло. Это кстати возможно
объясняет, почему стекло так легко разбить, в отличие от металла... То же самое
насчет людей. Эти внешне как бы люди застряли в ложном положении, они по факту
зомби, и соответствуют определению зомби - человек, потерявший свою собственную
Волю, Разум, ставший в результате послушным исполнителем все того, чем его
напичкали его хазяева. Жалкое зрелище, очень жалкое, на самом деле! Стеклянные
люди... зомби-стекло... А в биологии кстати, когда белок застряет в ложном
минимуме, он ломается и перестает выполнять свою основную биологическую
функцию, превращаясь в так называемый прион, что в итоге приводит (накопление
прионов в мозгу) к развитию таких болезней, как болезнь Альцгеймера.

% https://www.facebook.com/groups/2850411311922693/posts/3105488239748331
Добрый день! Спасибо, стихи берут за душу! А знаете... я киевлянин. Никогда в
Мариуполе не был. Сейчас довольно много читаю про Мариуполь. И книжку Надежды
сейчас читаю. И я скажу так... Автор, то есть вы, как бы немного со стороны
пишет, когда я вернусь, я тебя (то есть Город) не узнаю... когда-нибудь... А я
так думаю. Возвращение в Город на самом деле зависит в первую очередь от вас. И
восстановление, и тот вид, который Город примет, когда он будет восстановлен, и
снова засияет всеми красками, зависит в первую очередь от тех людей, которые
его более всего любят - то есть, мариупольцев. И отстраивать, и определять его
новый облик, и вдувать в Город снова Душу, снова Жизнь, снова возвращать будете
вы, мариупольцы, и не какой то сторонний дядя или чиновник. И... у Города есть
два стороны. Есть физическая, в реальности, разрушенная, в оккупации. Ужасная
сторона. Полу-мертвый Город-Кладбище, Место невообразимой Трагедии. Это что в
реальном физическом пространстве. Там, где мурал Милана снесена, и там, где
погибло ужасной смертью десятки тысяч человек. И есть Город Счастливый, Живой,
Радостный, который остался в ваших сердцах, душах, в памяти. Остался в книжках,
в записях постов в Фейсбуке. И сейчас я бы хотел немного напомнить об этом
Городе. Этот Город нынче в Духовном Пространстве. Но он также реален, абсолютно
реален. Стоит только осознать это как следует. Потому что например, мурал
Милана... физически он уничтожен... но как образ... он сохранен... и это не
проблема его снова воссоздать, когда придет время! И этот... Город. Духовный
Счастливый Мариуполь. Нужно возвращать себе, вовсю и целиком. Через Сердце,
Душу, сохранение памяти, книжек... через мечты и мысли о будущем Мариуполя, о
том, каким Вы его видите в том виде, когда он полностью заблистает снова в
Украине. Да! Рашисты уничтожили Мариуполь физический. Но Мариуполь духовный -
они уничтожить не могут... А вот сохранение Мариуполя в пространстве духовном -
это и есть ответственность в первую очередь мариупольцев... и если этим не
заниматься... с любовью, усердием... да, собирать материалы, записывать,
сохранять! да... если память про Мариуполь Мирный, Счастливый, Веселый
исчезнет, сотрется, распылится... то... Мариуполь будет потерян уже навсегда...
Не дайте же этому произойти! Вот... ниже фото книжки, вы ее знаете. Так вот. Ее
нигде нет. Ее невозможно найти в Украине в книжных магазинах или в библиотеках.
Вот это и называется затирание, умирание памяти. Прекрасная книжка, люди
столько усилий вложили в нее, но... ее нигде нет в продаже сейчас, а почему...

Книги — морська глибина / Хто в них пірне аж до дна / Той, хоч і труду мав досить / Дивнії перли виносить

спасибо! да, меня зовут Иван, если что. Я программист. живу в Киеве. Я про это
читал, это очень грустно, очень грустно... и слово грустно, может быть, это не
то, что нужно сказать... очень сочувствую, всем Сердцем, и это не пустые слова,
поверьте! Да... я намедни по магазинам и на Петровке искал... искал эту
книжку... ее не нашел... нашел другое, тоже важное - а вот ее не нашел....
вообще я думаю, что тут есть проблема для меня... потому что слова человека, не
прошедшего через эти ужасы, через которые прошли мариупольцы... по любому будут
резать слух и сердце вам... извините... но знаете... я думаю, есть файлы. есть
книжки у кого то, остались экземпляры на руках. Я имею в виду вообще принцип,
понимаете. Не лично мой интерес - что вот лично якобы мне это нужно. Это нужно
Украине, и это нужно всем нам. Да... Есть Книжка. Физически... Те конкретные
экземпляры сгорели, да. Но! У кого то возможно есть на руках. Возможно, есть
файлы оригиналов, и их можно распечатать снова. Более того, и самое важное -
да, самое важное - живы творцы идеи, живы авторы, живы те, кто рисовал рисунки.
Понимаете. Эта Книжка реально сейчас жива по прежнему, хотя те экземпляры и
сгорели. Жива в ее творцах, жива, потому что она была издана совсем недавно,
жива память про нее. И единственное, что нужно - чтобы она снова появилась в
книжных магазинах в достаточном количестве! Это Воля и Желание... Да, Воля это
сделать. Кинуть клич - а давайте друзья, издадим ее снова! Ну вот возьмем и
издадим снова, наперекор всему! И знаете... я боюсь сейчас оказаться в позиции,
у Вас может создаться впечатление... что я обвиняю Вас или что то еще. Упаси
Бог! Я просто очень очень люблю книжки... Я фанат книжек. Знаете, вот Иван
Франко очень хорошо сказал про книжки... Книги - морська глибина / Хто в них
пірне аж до дна / Той, хоч і труду мав досить / Дивнії перли виносить - И мне
больно. Когда бывает так... что книжка, очень хорошая книжка может умереть...
Тем более, когда ее вполне можно сейчас спасти, особенно потому что это очень
нужная, важная книжка и для Украины, и особенно для Мариуполя и мариупольцев, и
конечно для детей. И знаете... я вспомнил сейчас историю... про армян. Это -
древняя нация, и на их судьбу выпало множество испытаний и горестей. Против них
был устроен геноцид в свое время... Так вот... У них в Ереване есть Матенадаран
- музей Книжки... И знаете, что они спасали в первую очередь... они спасали
книги...

%14:07:30 27-02-23
Добрый день! Я тоже читаю Вашу книжку, все никак не дочитаю... столько всего...
уже взял карандаш... записываю походу, чтобы не забыть... не пропустить... а
насчет Книжки, памяти... Как память выглядит про события более давние, не менее
трагичные... У меня есть книжка Голод 1946-1947 Коллективная Память... В ней
более 1000 страниц. Это толстая объемная книжка. Это - колоссальный труд многих
людей, копания в архивах, собирания историй. Но это было уже давно. А что
насчет Мариуполя, это все сейчас... И всем... кто пережил, нужно писать вместе
вашу Книжку. И она тоже будет на тысячи, многие тысячи страниц, много томов. И
это сделать легче, потому что вы - очевидцы - живы. Тем более, что есть
колоссальное количество фото, видео, постов. И... касательно еще фото. Насчет
сохранения памяти о Мариуполе на самом деле еще очень мало сделано. Персонально
насчет Вас, Надежда -  Ваша книжка - это подвиг, это колоссальная работа, если
брать в контексте одного человека, и если знать, через что Вам пришлось
пройти... Это реально Ваш Духовный Подвиг. Но в глобальном контексте - увы, это
очень мало. И много сейчас говорят, что нужно донести правду про Мариуполь по
всему миру. А я думаю так. Очень важно, чтобы в первую очередь вся Украина об
этом узнала как следует, на обыкновенном, бытовом уровне, на уровне походе в
книжный магазин или на Петровку. Чтобы заходишь в магазин - и целая полка
уставлена книгами про Мариуполь. И про войну, и про мир. И толстые фолианты про
трагедию, с датами, фото... и красивые альбомы, и необыкновенная книжка про
мозаики Ивана... И просто художественная литература о Мариуполе... Много всего
надо! Много! Очень много! А насчет фото. Смотрите. Самая толстая книжка - это
про голод. Страшная трагедия, которая случилась уже давно. Выше - все, что у
меня есть на руках про Мариуполь на данный момент. А выше - сбоку - сама Жизнь.
Горшок у меня на кухне стоит, я его поставил сбоку. Да, сама Жизнь. Снаружи
светит Солнце, ездят машины, Киев живет своей жизнью... Ну а у меня на кухне
растет в горшке крошечный зеленый Листок, который побеждает смерть, который
растет, несмотря на все, что написано в этих книжках.

да, видео... само по себе впечатляет, бьет в психику... но... позвольте
возразить, оно, как по мне, не очень подходит на интерпретацию войны, и какое
то оно не очень оптимистичное... уж извините, имхо для меня Украина - это явно
не одинокий крошечный саженец посреди каких-то непонятных рельсов и камней, а
россия - явно не мощный поезд... Как то уныло и неоптимистично это видео, уж
извините... Это вы (создатели видео) что ли, все величие Украины, весь
несокрушимый дух Украины и ее народа представляете в виде какого-то крошечного
несчастного саженца посреди камней и рельсов неизвестно где, а рашку в виде
огромного тяжелого поезда?? Это вы с такими видео, что ли, побеждать
собрались?? А насчет Мариуполя... Если уж на то пошло... то знаете, какая
аналогия? Прекрасное Дивное Озеро... Из него вылили воду и расплескали...
Сейчас на месте озера непонятно что... Но капельки то никуда не исчезли... они
тянутся друг к другу... тянутся друг и другу... и неизбежно снова соберутся
вместе, и Озеро воскреснет! И для меня также, например, Украина - это мощное
Дерево... да... На него налетел порыв ветра... достаточно мощный, яростный
ветер... Дерево погнуло туда-сюда, много веток поламалось... но... Наше
Украинское дерево имеет мощные корни... тысячелетние корни! неиссякаемый Дух,
Сила, Воля, Славные Деяния Предков, вот наши Корни! Живительная Сила Украинской
Земли! Да, Дерево Украины живо и будет жить дальше, расти дальше... ветер же
скоро стихнет и исчезнет... россия же, если уж брать аналогию с деревом... то
это трухлявое гнилое дерево, которое скоро рухнет... И кроме того, это видео с
поездом не дает ответа на вопросы - что же будет дальше с саженцем? и два,
поезд уехал, но он остался таким же тяжелым поездом, который может вернуться
сколько угодно раз, не так ли? А кроме того, саженец... поезд его не сбил, а
вот какой нибудь случайный... человек... скажем... будет перелазить... и
случайно так задавит ногой, не так ли? И все! Нет саженца... Тяжелый массивный
поезд его не сбил... а человек растоптал...  Впечатление вообще такое,
признаюсь, не хочу никого обидеть... что может быть это видео даже рашисты
придумали, потом запустили так хитро в наши ресурсы... они в последнее время
очень даже так поднаторели на запуске разных ИПСО... Ну ладно. Вот видео,
которое мне нравится в миллион раз больше, хоть сейчас и не Рождество!
https://www.youtube.com/watch?v=5BSEuMDoZ40

%08:49:09 28-02-23
Доброе утро! Рискую разгневать здесь уважаемых участников чата.. но еще немного
добавлю. Много говорят про Победу, Перемогу. Что вот будет перемога, я сделаю
то. Я сделаю это. Сейчас вот например книжку читать не буду, потому что мне
страшно, а после Перемоги - обязательно. И что вот Перемога будет обязательно.
При этом... знаете, какая ошибка делается. Перемога, Победа понимается
исключительно как военная победа. Ну то есть как бы ЗСУ все освободят, и будет
сразу же Перемога. Поверьте, это не так. Не будет перемоги, даже если ЗСУ
войдут туда, сюда. Не будет. Потому что одних военных побед мало, мало. Потому
что...  Нужна еще Духовная Победа. И это дело уже индивидуальное, дело каждого,
задача каждого. Тут за спиной ЗСУ не спрячешься, и ЗСУ тут никак не помогут.
Победа над своим отчаянием, победа над внутренней грызней. Победа над унынием,
победа над чувством безнадеги, победа над чувством, что ты все потерял и что
уже ничего не будет, как раньше. Это все - демоны, которые тянут на дно,
которые убивают душу, и их тоже нужно побеждать. И побеждать прямо здесь и
сейчас, не дожидаясь каких нибудь хороших новостей с фронта, или не дожидаясь,
пока Байден приедет в Киев, и что то нам пообещает (вот помню, когда я увидел
про Байдена в телеграме, то страшно удивился - Байден в Киеве!? не может быть!
- посмотрел, потом... засунул телефон в карман, и забыл об этом на полдня, и
уже вечером проверил, что да, таки Джо был в Киеве, и звуки сирены не особо его
смутили... проснулся видно в нем ковбойский дух! ). И кстати, насчет ЗСУ, вот
это как раз очень хороший пример Духовной Победы. Потому что им очень тяжело.
Они рискуют жизнью каждый день, они теряют своих лучших друзей на войне, они
сражаются ради Украины зачастую в поистине ужасных условиях. И Наши Воины - это
реально пример Настоящей Духовной Победы. И полк Азов - особенно. 83 дня они
держали оборону Азовстали против превосходящих сил врагов. Их подвиг реально
уже вошел в историю навсегда! Это подвиг Геркулесов, подвиг Гераклов, подвиг
300 спартанцев!  Скажите, полк Азов разве похож на тот несчастный одинокий
саженец посреди рельсов? Нет, ясно не похож, ни на секунду! Ну ладно.  Вот еще
одно видео, давнее уже, но ничего! Эта девушка в ЗСУ, она воюет. Она рискует
жизнью каждый день ради всех нас, дай ей Бог здоровья и счастья на многие
многие года, но разве похожа она на тот несчастный жалкий саженец, на который
наезжает огромный поезд? https://www.youtube.com/watch?v=VOlXHknc0eg

Доброе утро! И Вам спасибо! С первым днем Весны! Хотел бы про такое написать.
Про Интернет. Я смотрю, у Надежды на данный момент приблизительно 4200
подписчиков. Я читал не раз у разных авторов, мариупольцы жалуются, что
Мариуполь ушел из информационной повестки, про него не пишут как следует. Вот
эти 4200 подписчиков - это про это самое. В масштабах Интернета - это вообще ни
о чем, если вспомнить, что у мерзких тварей соловьева или симоньян по миллиону
или сотне тысяч подписчиков. Эти каналы сеют вражду, безумие, и все это
получает миллионы просмотров сразу же. Касательно украинских тг-каналов, самые
известные каналы тоже имеют под миллион подписчиков, впрочем. А в этом
телеграм-канале, у меня такое впечатление, в основном только мариупольцы. И
увы, получается так. Надежда пишет очень много, очень старается. Но аудитория -
мала, и аудитория в основном тех людей, которые все это сами пережили. Друзья,
так вообще не должно быть, это не только ваша беда, это беда всей Украины, это
вообще беда всего человечества, также как беда всего человечества Холокост, а
не только евреев. О том, что вы пережили, действительно должен узнать весь мир,
по настоящему. И этот телеграм канал должен иметь намного больше, намного
больше, десятки тысяч, сотни тысяч, подписчиков, и также таких подписчиков,
которые непосредственно к Мариуполю отношения не имеют, таких как я . Но...
количество подписчиков зависит в том числе от активности в интернете уже тех
людей, которые подписались. То есть, если у вас есть друзья, а у тех друзей -
тоже есть друзья - и так далее. То нужно писать, делиться, просить, чтобы
подписывались. Рекламировать вовсю - хотя может быть это слово вам срежет слух
- как это вообще язык поворачивается говорить "рекламировать канал",  когда
речь идет о таких невообразимо трагических событиях, но интернет так работает,
если про это не рассказать, не порекомендовать другим людям - то эти другие
люди сюда не придут, сами по себе. Хотите, чтобы про Мариуполь узнало больше
людей, чтобы у этого канала было намного-намного больше подписчиков? Делитесь с
друзьями, и с не-мариупольцами, делитесь, распространяйте. Не стесняйтесь. Если
вы хотите, чтобы о Мариуполе узнали вся Украина, весь мир, то одни Надежда или
Наталья за вас всю работу не сделают, это ваша коллективная забота по
сохранению вашей памяти. Одна небольшая книжка Надежды - хотя, конечно, очень
важная книжка - память об этих ужасных событиях не сохранит в глобальном мире,
не будьте слишком оптимистичны насчет этого, книжек нужно десятки, сотни,, на
сотни, тысячи страниц, чтобы это произошло, чтобы память была надежно передана
в следующие поколения. И эти книги нужно писать как можно скорее, пока память
жива, пока вы все живы, потому что потом это будет сделать намного-намного
сложнее. Потому что пока человек жив - он может рассказать, поделиться. А когда
человек умирает - не существует никакого способа спросить его, все, что он
знал, видел, пережил, он уносит с собой в могилу навсегда... И очень
несправедливо, что какие то твари безумные имеют по сто тысяч подписчиков, или
даже под миллион, а очень светлые, очень хорошие люди, такие как Надежда всего
лишь 4200!

%22:16:48 01-03-23
гарні, дуже гарні фото! А ці рядки я вже читав у Вашій книзі... Знаєте...
Читаєш, обдумуєш, купа думок... І от щодо фрази "жити лише сьогодні". От не
погоджусь, а чому. Я розумію, що Ви писали ці рядки в дуже важких обставинах,
але хотілось бі прокоментувати... Був такий психолог Віктор Франкл. У свій час
він пережив нацистські концтабори, у нього нацисти вбили всю сім'ю, відібрали
все... І в концтаборі, поки він там сидів, він спостерігав, занотовував, як
себе вели інші його колеги по нещастю... Потім, після війни, він опублікував
свої спостереження, написав книжку, Людина в пошуках справжнього сенсу.
Психолог у концтаборі. Так от, щодо фрази жити лише сьогоднішнім. Вона може
легко ввести в оману і може призвести до хибних висновків щодо того, як слід
далі жити... Насправді, Людина - це така Істота, яка не живе лише сьогодні, і
абсолютно... не повинна жити лише сьогоднішнім днем, в будь-яких... навіть
найтрагічніших обставинах... Людина живе в минулому... через свої спогади,
пам'ять... Людина живе у даний момент... і Людина живе Майбутнім. Через свої
мрії, смисли, чим вона живе внутрішньо. Всередині кожної Людини, її душі... є
Вісь... Минуле-Сьогодні-Майбутнє... і чим ця ВІсь є довша, міцніша всередині
Людини... тим воно краще... Так от. Незважаючи на те, що Франкл втратив все,
він дожив до 94 років... І... він помітив, що ті люди... які в концтаборі... не
втрачали свої мрії... мріяли про своє майбутнє... які не втрачали осмисленість
свого перебування в концтаборі... вони не ламались психологічно... вони знали,
для чого вони живуть, навіть якщо ти чекаєш, що завтра тебе відведуть у газову
камеру і все... А от ті... які жили лише тою жахливою дійсністю, яка була
навколо, - коли в тебе відібрали все, і ти не знаєш, скільки тобі залишилось
жити, - і власне кажучи, ти вже не розумієш і не знаєш, нащо тобі взагалі жити,
якщо все навкруги все так жахливо, - навпаки, ламались і вмирали досить
швидко... Ось так.


а я вже зрозумів! Ви в Черкасах, вірно... Але не в Києві або Черкасах тільки
справа... а в тому, що є насправді... а насправді... є таке... Сталось щось
таке, страшне... що забрало багатьох ваших близьких, друзів... розкидало вас по
всьому світу... По Україні, по іншим країнам, по всьому світу, дійсно. І ви
всі... весь час намагаєтесь повернутись... відкрити двері, що зараз зачинені.
Вперто намагаєтесь відкрити зачинені наглухо двері... Я думаю так. Всьому свій
час, і ці Двері відчиняться для вас знову... Але зараз ці двері зачинені, а ви
відправились у подорож. Звичайно... ви не просили цієї подорожі, але факт є
факт. Хто в Київ переїхав, хто в Черкаси, хто в Львів, хтось у Файне Місто
Тернопіль, де теж є ДрамТеатр, і як маріупольці кажуть, він досить таки схожий
на маріупольський... Але в Тернополі є озеро, але немає моря... і хтось зараз
сидить у далекому карпатському селі, а хтось зараз порівнює Одеське море, і
зітхаючи, каже... все не те, все не те... Знаєте, ви як голуби... так,
маріупольські голуби... Біля ДрамТеатру... Літали, ходили коло театру вдома...
а потім взяли і розлетілись.... так, звісно, ви обов'язково повернетесь у своє
Рідне Гніздо, у свій час... але... зараз ви в польоті, ви залітаєте в різні
міста, різні країни... І поки ви у подорожі... потрібно пізнавати ці місця,
незнайомі для вас спочатку, начебто чужі, але насправді - свої. Бо то є все
Україна, а Україна, в свою чергу, пов'язана з усім світом, бо в усьому світі на
Різдво звучить Щедрик. А Маріуполь - це Україна, так було... так є... і так
буде завжди... Так от, якщо ви в Києві, зайдіть, поговоріть із Орантою в
Софійському Соборі.., бо Божа Матір - фреска Нерушима Стіна - вже 10 сторіч
дивиться на Київ...і існує легенда, що... поки є Оранта, доки буде стояти й
Київ... Подивіться на графіті у соборі... не поспішаючи... уявіть собі... чим
жили древні кияни... там до речі є навіть графіті про смерть Ярослава
Мудрого... прямо в центрі собору, наскільки я памятаю. потім... можна піти на
фунікулер - вниз - потім... Поділ... а там далі стоїть Сковорода... а потім
Андріївський Узвіз... піднялись знову... Пейзажна Алея... далі... далі...
далі... Львівська Площа, Ярославів Вал... там є місце з чудовими булочками,
називається Ярославна... А якщо у Львові - облазьте весь Львів... якщо у
Карпатах - підіть у похід... Якщо у Житомирі... там є музей космонавтики.
Полтава ж - це колиска мови... Еней був парубок моторний, і хлопець хоч куди
козак... Харків - Сковорода, якому у минулому році виповнилось 300 році..
Розумієте... Потрібно всмоктувати енергію, дух... тих місць... де ви є... щоби
потім принести часточку всього цього богатства назад, додому... І це варто
робити, дуже варто робити, поки ви там, а не вдома, знаєте, бо вам потрібно
дуже-дуже багато сил, щоби зустрітись знову з Маріуполем вживу... Потрібен
духовний щит, духовна броня...   Бо там зараз Мертве Місто, Місто-Кладовище,
Місто Крику, Місто Жаху... І воно просто вб'є вас... Пекло, з якого вам
пощастило вибратись, оживе для вас знову на повну, і поховає вас заживо там на
місці і вже назавжди... Закричить на вас тисячами голосів жаху. відчаю,
смерті... Так, вдома вас очікує смертельна небезпека... Тому... потрібно
готуватись... добре підготуватись до повернення...  

% Scorpions - White Dove (перевод субтитры)
% https://www.youtube.com/watch?v=GQT1yPIyk8Q

%15:30:55 02-03-23
все втрачає? Людина все втрачає, коли вона опиняється в Могилі. Бо тоді дійсно
вже нічого зробити не можна. А у Вас є Життя. Більш того, Ви маєте здоров'я,
маєте телефон, щоби сюди писати. Так от. Хочете вірте - хочете ні. Те, що було,
вже сталося, і це неможливо змінити, неможливо повернутись в минуле, і щось
пофіксити, на те воно і минуле... А те, що у Вас буде далі в житті - залежить
якраз від Вас. Отой Франкл, про якого я писав раніш, він теж втратив абсолютно
все - друзів, сім'ю, дім, все абсолютно. Тим не менш, він дожив до 94 років,
проживши цікаве і насичене життя. Жах наступного дня? А де той жах? Жах той є
лише в вашій голові, і тільки. В Києві зараз, наприклад, немає ніякого жаху.
Звичайно, йде війна в Україні, але там, де я знаходжусь, немає ніякого жаху
взагалі, від слова зовсім. Дім у двох валізах? Ваш дім не у двох валізах, ваш
дім - є уся Україна. Для кого ж Сосюра писав в Києві свого вірша у далекому
буремному 1944 році, коли ще йшла жахлива війна, і Україна була знекровлена,
спустошена... любіть Україну, як Сонце любіть, як вітер і трави, і води, В
годину щасливу, У Радості Мить, Любіть у Годину Негоди?

%19:17:10 02-03-23
да здесь дело не только в телевизоре. Я уже писал про Матрицу - мощное
информационное поле, ментальный вирус, паразитирующий на мозгах миллионов
людей. Понимаете. Это целенаправленное, вдумчивое, учитывающее слабые стороны
психики любого человека, систематическое, многолетнее взращивание ненависти к
соседнему народу, умноженное на практически неограниченные возможности
Интернета,  который многократно усиливает любой бред, который льется из
телевизора, потому что все, что льется из телевизора, также дублируется во всех
главных тг-каналах, а эти каналы имеют сотни тысяч подписчиков. Уж извините,
что могу задесь ваши чувства, но врага нужно изучать в том числе в
информационном пространстве, вдумчиво собирать информацию, изучать психологию.
Потому что в итоге усилия в информационном пространстве, в том числе по сбору
информации об поведении и образе мышления наших врагов могут спасти чьи то
конкретные жизни, скажем даже так. Так что... вот например пример из моего
архива чатиков в ок, это же реально полностью безумные зомби. Их никто не
бомбит, они сидят где нибудь в Красноярске или Новосибирске в тепле и уюте, но
они... просто брызжут какой то безумной бессмысленной злобой, желая уничтожить
всю Украину целиком. Обратите внимание  на стиль того, как они пишут.
Деградация психики, деградация мышления налицо

є така штука, Інтернет Архів - Машина Часу - Way Back Machine,
https://archive.org Там можна файли, документи, відео зберігати, і
передивлятись його там. Я думаю, що ніхто їх видаляти не буде, так, як це зараз
зробив ютюб або фейсбук, тому що місія цього сайту - якраз збереження всього,
що є в Інтернеті, і він в собі вже зберігає напевне майже трильйон
різноманітних ресурсів. Реєструєте аккаунт і завантажуєте що завгодно. Для
таких надзвичайно важливих речей, як цей фільм, завжди варто мати резервні
посилання. А ось як виглядає на цьому сайті відео Щедрика 2021 в Маріуполі,
можливо ви його вже бачили десь
https://archive.org/details/mariupol.schedryk.2021

%15:18:52 03-03-23
Доброго дня, пані Надія! От щойно дочитав Вашу книжку... Дуже-дуже багато є
думок... Напевне, прийдеться ще пару раз перечитувати... І потім якось напишу
більш докладно... Але... хотілось би якось з чого почати, щоби не забути...
Сподіваюсь, Вам буде цікаво те, що я напишу, особливо оскільки Ви зараз живете
в Німеччині, наскільки я розумію... Ось, уривок щодо життя в Німеччині... що
вулиці німецьких міст чужі, весь час повертаєшся в Маріуполь... я розумію, це,
так... Але... Місто... то не є тільки фізичний простір, то є також духовний...
Крім того, інша країна, інші звичаї... все інше... А щодо Німеччини. Моє
персональне знайомство з Німеччиною сталось вже давно, колись... через
книжку... так... я прочитав книжку Тім Талер, або Проданий Сміх, видавництво
Київ Веселка 1991, український переклад, дуже гарний український переклад,
просто неймовірний! А написав її свого часу німецький письменник Джеймс Крюс...
і книжка ця насправді не тільки для дітей, а й для дорослих, тому що... містить
дуже багато мудрих думок в собі... І я... рекомендую її всім почитати, якого
віку б ви не були... Але... щодо вуличок німецьких міст... Там теж історія
починається з напевне таких самих вулочок, якими Ви можливо гуляєте прямо
зараз... І сподіваюсь... якщо знайдете час почитати... можливо ці тихі німецькі
вулички, які мовчать для вас... одного дня раптом заговорять, яскраво, весело,
цікаво! Розкажуть Вам неймовірно багато цікавого! Так! А щодо маріупольского "І
шо"... так повірте, в Києві так теж говорять, я от так доволі часто кажу )))

%16:14:52 03-03-23
Доброго дня! Неймовірне фото! Дівчатка з голубами просто агонь! ...І мені от
знаєте, що зараз згадалось... досить давня вже книжка німецького письменника
Джеймса Крюса - видавництво Веселка Київ 1991 - там є три різні твори - Тім
Талер, або проданий Сміх, Флорентіна, та Мій Прадідусь, Герої та я. Так от...
Тут я кажу якраз за другий твір - про веселу дівчинку Флорентіну та її подружку
Боббі, які весь час встряють у різні неймовірні пригоди, із голубами в тому
числі... І ось уривочок... Почитайте, якщо будете мати час... Ну чим не
Маріупольська Флорентіна із подружкою? )

Пригода Маленького Мене - 1

Трохи такий собі крик душі, не судіть надто строго...

%21:15:09 03-03-23
Щойно дочитав книжку Nadia Sukhorukova дуже багато вражень, думок, запитань... сподіваюсь, скоро буду в змозі написати більш докладно... А зараз... Скажу так. Я дуже люблю книжки. Різноманітні книжки. І наукові, і художні, і історичні. Різноманітні книжки. Я - просто такий собі звичайнісінький київський книжковий маньяк. Маньяки бувають різні - бувають такі, що дійсно є маньяки, на совісті яких дійсно багато багато людського горя і смертей, навіть маньяки в кубі, як от той мерзенний огидний карлик на болотах,  а є маньяки книжкові, або маньяки шахові, або маньяки музикальні, або хтозна які ще маньяки. Такі собі бєзобідні маньяки, які нікого не чіпають, але просто є закоханими у щось. Так от, я - книжковий маньяк. Люблю Книжки. От добре колись Іван Франко написав про книги - 

Книги - морська глибина, 
Хто в них пірне аж до дна, 
Той, хоч і труду мав досить, 
Дивнії перли виносить 

І... щодо Маріуполя... Я змушений констатувати на даний момент, що ситуація з книжками про Маріуполь просто жахлива... Розумієте. По-перше, сталась жахлива трагедія, загинуло багато людей, Маріуполь було зруйновано, понівечено... Страшна людська трагедія... Але книжок про це знаєте... як от кажуть.. раз-два и обчелся... От я живу зараз в Києві - Столиці Україні, я - киянин. В Столицю завжди стікаються найкращі інтелектуальні сили країни, так вже сталось - без образ щодо інших міст, Київ - це не тільки Місто, де знаходяться Верховна Рада або Адміністрація Президента, це також - Інтелектуальний, Культурний, Духовний Вузол України. Тут можна цілу дисертацію написати зараз щодо підтримки цього твердження, але... я думаю, всі це і так знають... І так було завжди... Від часів Київської Русі... коли у Києві була заснована перша Бібліотека, і був написаний перший Літопис - Повість Времьяних Літ... аж до сьогодення... І зокрема, як наслідок всього цього.. В Києві більше всього книгарень, бібліотек - тут, зокрема, розташована Бібліотека Вернадського, де до речі знаходяться абсолютно унікальні манускрипти, -  так от, у Києві... найбільший вибір того, що можна купити з книжок в Україні, від букіністичних видань Булгакова, навколо творчості якого весь час точаться якісь неймовірні запеклі дискусії або навіть давно забутих як страшний сон і проклятих Лєніна/Сталіна/Енгельса до найсучасніших найактуальніших та наймодерновіших письменників, як то Сергія Жадана або ж Максима Кідрука. Будь-що, на будь-який смак! Взяв руки в ноги, поїхав собі, знайшов, що тобі треба, і читаєш!

І в принципі, якось вже можна очікувати, що трагедія Маріуполя після одного року, як все це сталось, повинна була б вже бути відображена у книжковому вигляді як слід...

Але от... Щодо Маріуполя...

Якщо находиш у якусь книгарню у Києві... Книгарня Є, або ж якась інша книгарня... або сідаєш на метро і ідеш до Метро Почайна, а потім - блукаєш собі Книжковим Ринком Петрівка - 
то... питаєш книжки про Маріуполь... і виявляється... хм... нема книжок про Маріуполь. Їх просто не-ма. Є тоненька книжка Шишацького Люди - і все, фактично... Якщо хочеш купити про Київ - будь-ласка, сотні книжок на будь-який смак... Але... про Маріуполь по факту нічого немає... І от це є якраз дуже боляче, персонально для мене... тому що 

(1) трагедія Маріуполя, збереження памяті про все це вимагає публікації цілих великих томів... на кшталт книги Колективна Память про Голод 1946-1947 - більше тисячі сторінок, . Так! Потрібно багато багато томів на сотні, тисячі сторінок і все це потрібно писати прямо зараз і негайно! - поки живі свідки, поки память ще жива!

(2) Маріуполь мирний, довоєнний, щасливий, веселий, заслуговує так само, щоби про нього було багато книжок, різноманітних книжок - художніх, туристичних, історичних, щоби можна було прийти в книгарню і без проблем купити щось про Маріуполь... 

І от. Я щойно дочитав книжку пані Надії. Вона зокрема згадує, що її чоловік Віктор Сухоруков написав книжку Пригода Маленького Мене - я так розумію, про мирний Маріуполь. Мені стало дуже цікаво, про що ж ця книжка... і я почав шукати... шукав, шукав... але гугл вперто видає мені знаєте що... того самого Сухорукова з фільму брат-1,2. Насправді, це просто жах... тому що мені абсолютно не потрібен той Віктор Сухоруков з фільму брат-1,2 мені потрібен Віктор Сухоруков, чоловік Надії, що написав оту саму книжку.  Але... бездушний Гугл нічого не знає про мої переживання, і вперто сує мені абсолютно непотрібні мені речі. Так! Врешті-решт я щось знайшов... Але... я не знаю, як виглядає ця книжка, нічого про неї взагалі не знаю, по факту. Далі... Є така книжка... Неймовірні пригоди Маріка та Марічки - веселий путівник Україною - одним із авторів є Оксана Стомина наскільки я розумію... із тих зображень в інтернеті, які мені вдалось знайти... Ось... зображення книжки є - а от самої книжки вже немає... Парадокс і тільки! Далі - сама книжка пані Надії - так, вона вже з'явилась у декотрих книгарнях... але... обмеженим тиражем... і сама по собі книжка - дуже потрібна книжка - не змінює загальної ситуації... далі - монументальний Маріуполь - автор Іван Станіславський - більш дороге і красиве видання - але... його ніде немає... (на щастя, в мене ця книжка є, пощастило деякий час тому... - неймовірна гарна книжка! ) Далі - маріупольський краєзнавець Сергій Буров - він загинув, на жаль - його книжок теж ніде немає. Звичайних географічних карт про Маріуполь - немає. От Карт України, Києва, Львова - до біса. А Маріуполя - немає. Туристичні путівники, художня література - про все завгодно, все це є, до біса. А от про Маріуполь - немає! Ну добре.
Я продовжу... пізніше... тут багато треба писати... а поки що скріни моїх (марних) пошуків знайти ту саму книжку Пригода Маленького Мене. 

Розумієте... Так.. Десь є ця книжка... на руках у Надії, я так розумію. Але... в Інтернеті - її немає, по факту. Немає не тільки наприклад, якогось піратського скану в вигляді pdf-файлу, немає навіть детального опису, немає фото - обкладинки... Ії просто немає! Є одна згадка в нашому ресурсі (скрін), і все... І більш того, вилазить постійно отой самій огидлий актор Сухоруков... А якщо її немає в Інтернеті - значить - її немає у свідомості людей, масово... немає у свідомості людей читаючих, тих, що зазвичай зараховують до інтелектуальної еліти нації, - і звісно, немає у свідомості людей пересічних, звичайних громадян... її немає... і ніхто її вже не прочитає... А шкода. Дуже гарна і цікава книжка, я так здогадуюсь!

Дуже дякую, я радий, що мій пост так швидко відгукнувся у Вашій Душі і Серці!
але... я думаю... обіцяти треба не особисто мені, бо я в принципі... особливо є
ніхто... просто киянин, таких у нас в Києві, навіть таких програмістів, що
люблять читати книжки не тільки по джаві або пітону (це все мови
програмування), как пруд пруди, как грязи... а Маріуполю, так, Вашому Коханому
Маріуполю... це Вашому Місту треба обіцяти... і виконувати обіцяне... ...я так
думаю. І насправді, вас багато. Вас дуже багато! Неймовірно багато! і... якщо
ви - маріупольці - так, зараз розкидані по світу, багато з вас у розпачі через
втрачене Місто, через втрачене щастя, втрачених друзів, батьків, дітей... у
журбі... - але... якщо ви станете згуртованими, дійсно згуртованими, так! це не
буде проблемою надрукувати скільки завгодно книжок про Маріуполь! І більше
того, це буде однією із передумов вашого справжнього, на повну повернення у
Маріуполь... Бо окрім чисто військового звільнення, потрібне і духовне,
культурне звільнення Маріуполя... бо Залужний може обіцяти все що завгодно - і
я навіть впевнений, що він виконає свою обіцянку... але, оживити Маріуполь,
зробити його таким же живим, радісним, як він був до вторгнення... він
(Залужний) ніяк не зможе, це явно не його парафія... І... от знаєте, я згадав
Майдан... навіть у мене книжка є... хоча безпосередньо активної участі не
приймав... ну... так вже сталось. Так от... У Майдану було дуже гарне гасло -
Разом Нас Багато, Разом Нас не Подолати, пам'ятаєте, так?

... і щодо ще книжок... Тут справа власне не в тому, що важко видавати, а в
тому... що ви, маріупольці, як спільнота поки що розкидана... Розумієте.
Маріуполь навсправді дуже болить багатьом українцям, і не лише українцям, вони
б усі із великим задоволенням читали б книжки про Маріуполь... Але... оскільки
книжок немає... то... їх і не читають, логічно ж, так? Ось наприклад, є фейсбук
группа Враження UA - величезна кількість постів, стільки всього люди читають! А
про Маріуполь нема постів (може десь там й є, але я не бачив...) Є пости про
фантастику, є пости про українську літературу - а от про Маріуполь немає... Ось
наприклад дівчинка купила книжку (скрін знизу) - потім прочитала - потім пише
відгук... Була б книжка в неї про Маріуполь, щось красиве, цікаве про мирний
Маріуполь, історії, художня література, навіть Ваша книжка... - вона б написала
відгук про Маріуполь, а так вона пише відгук про щось інше... А чому нема? А
тому що нема книжок. А нема книжок - Маріуполь просто зникає з інформаційного
простору. І так... він помер фізично... Але... зараз є ще страшніша загроза...
духовної смерті Міста, духовного забуття... Ну а відсутність Маріуполя в
книжковому просторі... призводить до того, что звичайні пересічні люди нічого
не читають про Маріуполь... (окрім постійних телеграм новин - що зруйновано це,
зруйновано те - а ці новини, звісно, особливо ніяк надихати на високі матерії
не можуть, тільки підігрівають лють і бажання помсти), і... це в свою чергу
призводить у маріупольців до тяжких думок, що про них забули... А цей весь
негатив в свою чергу виснажує, висмоктує сили... виходить таке собі зачароване
коло... - як це не прикро казати мені, а Вам може болісно чути... І вибачте,
може трохи якось змішано написав

доброго ранку! насправді... це дуже печально... і тут треба зауважити таке (1)
відбудова, повернення Маріуполя - то є справа усієї України, бо самотужки, ви
самі, ніколи не повернетесь до Маріуполя, поки ви самі по собі, вам не
вистачить сил не те щоби його відбудувати як слід, а й просто фізично
повернутись туди, бо вбите Місто має страшну негативну енергію - мертве Місто
почне вас просто реально вбивати на місці (2) щодо тої жіночки... ну,
повірте... далеко не всі українці такі, є багато таких, які навпаким, Маріуполь
і все, що з ним, дуже-дуже болить, навіть якщо вони там ніколи не були (3) ви
всі живете в тому світі. про що написана книга... теж тут не можу погодитись,
бо... за всіх напевне точно відповідати не треба. Є маріупольці, які дотепер
так, живуть, так як ви описали, не знаходячи собі ніде місця, і це дуже дуже
сумно... але є й такі маріупольці, які змогли перемогти розпач, тугу у собі, і
це навіть видно, знаєте, по тому що вони пишуть у соцмережах. Я, звичайно, не
можу залізти всередину душі людини і от так прямо стверджувати, що ось ця
людина ось перемогла в собі всіх цих демонів, а ось ця - ні... Але... Ось
наприклад, дуже показовий приклад однієї людини з Маріуполя (Анна Качалова),
яка намалювала ось такі малюнки (нижче скрін)... Ну і... Щодо маріупольців в
цілому. Є велика небезпека в тому, що ви замкнетесь в собі на своєму горі, і
станете така собі каста замкнена... тобто... уявно кажучи... почнете на повну
вірити в те, що... як ви кажете... два світи... ось є нормальний світ.. а ось
ми... ми там все втратили... ми вже назавжди будемо іншими... а я в це - не
вірю, абсолютно. Ви - такі самі! Незважаючи на втрату (тимчасову) свого Міста,
ви так само заслуговуєте на Щастя, на те, щоби бути дійсно сильними, так само
відчувати Радість від Життя, як і усі інші люди, я в цьому абсолютно певен! Ось
пані Надія написала в книжці про маріупольців як про гордих і сильних людей,
так от... є питання у мене... маріупольці - це горді, об'єднані та сильні духом
люди, які здатні перемагати своїх внутрішніх демонів та робити разом великі
справи... от як, наприклад, публікація книжок, або щось набагато більше... чи
це розпорошені по всьому світі люди, які вже навіки приречені жити у світі
своїх минулих жахів? Відповідь на це питання, як не дивно, залежить від самих
маріупольців...

а щодо книжки пані Надії... безумовно, неймовірно важлива книжка. Але... це -
щоденник, який писався у дуже важких умовах... він передає весь жах того, що
відбувалось, так, це дуже цінне свідчення, передане в сильній художній формі.
Але як дороговказ того, як треба жити далі, як повернутись та відбудувати
Маріуполь, як знову стати щасливими і дихати повітрям на повні груди - він
точно не підходить. І там є купа тверджень, з якими я точно не погоджуюсь...
Але... я розумію, в яких умовах все це писалось... Так що... потрібні нові
книжки, багато нових книжок... як і перевидання старих книжок... як і товсті
томи, що зафіксують на папері тисячами сторінок для історії на повну весь той
жах... так і нові, абсолютно нові, життєствердні книжки про Маріуполь, які
будуть надихати на великі справи, так!

%09:05:52 04-03-23
заємно ) на жаль... я це не можу зробити зараз... тому що. Тут є багато
причин... одна з них, наприклад... тому що я також зареєстрований в ок, і там я
активно воював з рашистами, і там я писав на них просто неймовірно жахливі
речі. Просто матюкався мама не горюй. Але... якщо моя мама або тато
дізнаються... а Інтернет така штука, що будь-яка інформація неймовірно швидко
поширюється... то це їм дуже не сподобається, хоча вони звісно теж ненавидять
росію. Вони дуже інтелігентні люди, і мій стиль для них напевне буде аж
занадто. Тому... поки що так, вибачайте...



%14:17:31 04-03-23
от дивіться. Пані Nadia Sukhorukova напевне це теж буде цікаво До речі. Якщо
зайт на сайт, і подивитись скільки є місць вільних, бачу, що більшість місць на
цю виставу є вільні. Зал невеликий, але... ажіотажу ніякого немає. Всього 5
днів лишилось до вистави, більшість місць не викуплені. І це - Київ. Столиця
України... І я певен, що в Києві зараз багато маріупольців, які переїхали...
ось наприклад, як пані Natalya Dedova але... чому мало хто купляє квитки, я от
дивуюсь. Невже актори і режисер дарма старались?... Знаєте... акторам і
режисеру... буває, що вистава не вдається, і глядачі невдоволені... Хоча у нас
дуже душевна публіка ходить в театри, і я ніколи не чув, щоб колись кого
освистували... Так, таке буває, рідко... коли вистава не вдається і це відчутно
по реакції зали... але... коли 3/4 зали пусті... це набагато більш боляче...
грати перед пустими стільцями... особливо коли така неймовірно важлива тема,
тема Маріуполя... більш того, зроблена маріупольським театром...

%17:02:23 04-03-23
так... ризикую розгнівати вас... але не гнівайтесь, будь-ласка. Пишу як думаю,
від душі (1) щодо рівня трагедії і розуміння іншими. Ви до речі не згадали
жителів Бучі. От бучанці я думаю зрозуміли б Вас краще, ніж скажімо я. Також...
і Харків... і інші міста, Суми, Чернігів, Миколаїв, також Херсон - то є теж
страшна людська трагедія... І я думаю... не варто надто мірятись тим, у кого
трагедія більша... Крім того, власне саме поняття "зрозуміти" - походить від
слова Розум... розумієте... пропустити біду іншого через Розум і Серце... і є
завжди деякий максимум того, наскільки інша людина може взагалі зрозуміти біду
іншої людини, яка б та людина не була хорошою або чуйною... Так вже створена
Людина... І насправді, щоби зрозуміти - тобто, пропустити через Розум і Серце
... те, що сталось з Вами, не потрібно повторювати на мені, або кому іншому...
- не дай Боже. Але я розумію вас, так. Наскільки я взагалі можу розуміти Вашу
біду, я розумію, тому я і читаю, записую... Просто за покликом серця і душі. А
щодо всього того, що з Вами там було, повірте, в Києві було теж саме, але
більше 70 років тому. Так, на бульварі Тараса Шевченка спокійно собі валялись
трупи, а жінки варили прямо у дворах їжу на вогні. І перед Бабиним Яром був ще
Київський Котел. Місто було оточене... і більше ніж 600 000 червоноармійців
було оточене і взято в полон. Так от... Київ - це місце жахливої трагедії
масштабу Маріуполя або більше. І якщо Маріуполь - це Озеро Крові, то Київ -
беручи всю його історію докупи - то Велике Море Крові, Горя, Трагедій і
Нещасть. Можна ще тут згадати про Чорнобиль... І взагалі кажучи, важко знайти
на мапі Землі інше таке місто, навіть Маріуполь.. - яке б мало таку ж трагічну
долю, яку має Київ... Але... якщо Ви пройдетесь Хрещатиком, або зайдете в
метро... ви цього не знайдете всього... Місто зберігає свій біль у книгах,
щоденниках... Місто заховало свій біль, свій розпач, свою тугу, всередину
себе... назовні ж ніщо особливо не вказує на те, які жахливі страшні речі тут
колись коїлись... І до речі, якщо ви колись гуляли Хрещатиком - це все,
відбудова по новому після війни. Так! Кияни власноруч відбудовували своє кохане
Місто, що лежало у руїнах... камінь за камінчиком повертали у Місто Життя... А
щодо Хрещатику... До війни все це виглядало зовсім по іншому. Так... а фото...
це з книжки про Бабин Яр. Та інше фото... з книжки архіви окупації... мене
вразило друге фото (жінка з котлом), оскільки подібне фото з котлом (але без
жінки у кадрі) я побачив у коментарях до посту Наталі (там, де вона просила
розповісти свої історії)... і трупи на фото не видно (вони зверху, фб може не
пропустити, так що... я обрізав, але залишив підписи до фото... )

%https://www.facebook.com/kidruk/posts/pfbid0p8K5GMf585mrA9nAgjPC5wYC9yoEoZNCDavoM6eBDnscvVGdPUR1v3NkfsievFwHl?comment_id=211383961437461

Доброго вечора, Максе! Колись із дуже великим задоволенням прочитав Вашу книжку
про острів Пасхи. Вона в мене вдома є. Дуже класна книжка із глибокими
роздумами, отримав справжнє задоволення колись від прочитаного! Також я є
шанувальником творчості Тура Хєйердала, і колись давно... навіть побував у його
музеї в Осло, в Контікі-Музеї... Бачив і плот, і човен Ра-2, все там облазив як
слід...  Але... то було вже давно, і зараз не про це річ... І на жаль, останню
вашу творчість ще не встиг перечитати, вибачте. хоча і бачив Вашу книжку у
книгарнях... І я сам - киянин... Але тут я пишу не про Київ, а щодо іншого,
щодо Маріуполя. Справа у тому... що в Києві майже немає книжок про Маріуполь.
Практично немає, окрім декількох найменувань. І це неймовірно прикро... тому
що, по перше, сталась жахлива трагедія, яка вимагає якнайскорішої публікації
щоденників, свідчень, фото... - потрібні товсті томи, де все це би фіксувалось
би. На кшталт Книги Колективна Пам'ять, Голод 1946-1947, яка є в мене вдома, і
в якій більше ніж 1000 сторінок.  Але... всього цього немає щодо Маріуполя.
Звичайно, є в телеграмі, фейсбуці, але всі ці свідчення є неймовірно
розпорошені, і більш того, під загрозою зпрямого нищення і забуття, оскільки як
відомо, фейсбук є дуже ненадійне середовище для публікацій, і тим більше - для
довгострокового зберігання памяті про всі ці жахи, та про злочини росіян.
Достатньо одного натискання клавіші, щоби пост або аккаунт, той або інший,
просто зник, і назавжди... Як це сталось наприклад минулого року із аккаунтом
маріупольської журналістки Надії Сухорукової, в якому вона викладала уривки із
свого щоденника (він щойно був опублікований у вигляді невеличкої книжки, до
речі) А по друге... Маріуполь потребує не тільки військового звільнення, але
також духовного, культурного звільнення. Зараз... багато маріупольців в
розпачі... їхні домівки знищено, їхнє кохане місто знищено та перебуває в
окупації... Напів-мертве Місто... Місто-Кладовище... І потрібно також зберігати
пам'ять про мирний, щасливий, радісний Маріуполь! І це можна зробити також,
оскільки на даний момент є велика кількість фото, відео, але знову ж, таки, все
це розпорошене, розкидано... Я вже всім цим почав займатись - систематизацією,
записом доступної інформації. Як все це виглядає, можна подивитись ось тут... Я
поступово викладаю матеріали у готовому до перегляду і друку вигляді
https://archive.org/details/@kyiv_chronicler  Також тут
https://t.me/kyiv_fortress_1 і також на моїй фб сторінці... Але то є окрема
величезна робота, якої я займаюсь у свій вільний час (я програміст). Крім
того... щодо проблеми культурного звільнення Маріуполя... відродження
Маріуполя. Потрібні нові книжки про Маріуполь... нові, цікаві книжки... багато
книжок. Тому я і написав зараз Вам. Бо мене це питання дуже мучить, хоча я й не
маріуполець, а киянин... Але Маріуполь - це Україна, і це є справа всієї
України не тільки звільнити Маріуполь війсковим шляхом, але також оживити це
прекрасне Місто, зробити його таким же радісним, веселим, живим, яким воно було
всього навсього рік з гаком тому... Тому... дуже сподіваюсь, Ви знайдете час
подумати над цим, може у Вас з'явиться натхнення написати Роман про
Маріуполь... Або щось інше, на Ваш розсуд... І я теж думаю писати книжку про
Маріуполь...  Бо це дійсно дуже-дуже потрібно зараз! Маріуполь повинен знову
з'явитись на повну в інформаційному просторі країни, і також в книжковому,
духовному, культурному просторі! Маріуполь, так, прекрасне Місто Марії,
заслуговує на набагато більше, ніж практично повна відсутність книжок про це
найпрекрасніше і водночас настільки трагічне Місто у книгарнях Києва та
України! Я вірю, це можна виправити, так!
https://www.facebook.com/kiev.fortress/posts/pfbid02Wr6DEVCFtH6j2cJQF5umzWuztTXVAM149px4EHQqBuSzvNePHPSqfZXt1UmbFSckl

%09:40:14 05-03-23
Доброе утро, Леонид! Да, напишу больше, как и собирался. Я заметил, Вы написали
в группе. Я Вас очень понимаю, поверьте. Вы видите то, что не видят другие, Вы
хотите, чтобы люди не замыкались в своем горе, в своем отчаянии, и я тожу этого
хочу. Но... здесь у меня есть мысли, которыми я хочу поделиться. (1) касательно
группы, и того, что там постят. Ваша поэзия очень сильная, и рвет душу, это да!
И несомненно нужно писать стихи, и другие посты, так, как это Вы видите! Но...
к сожалению, часто бывает так, что всегда есть конкретные люди, со своими
тараканами и прибамбасами, и с этим приходится считаться. Эта группа... да, я
Вас поддержал лайком, у Вас очень хорошие, сильные стихи, и очень хорошие
намерения... но. Эта группа, как я это вижу, создавалась все-таки с упором на
фото мирного Мариуполя. И мне понятно раздражение модераторов группы, и мне
кажется... не в обиду будет сказано, что Вы там многого не добьетесь, несмотря
на все Ваши добрые намерения. Все-таки, группы всегда создаются под какую-то
конкретную тему, и если в группу писать все что угодно - даже если эти
публикации абсолютно хорошие и патриотические - то размоется сам смысл этой
группы, и модераторы - конкретные люди, и у них тоже есть какое то свое личное
мнение по этому поводу... Так... далее. Насчет фото в этой группы... Я не
совсем понимаю Вашего разочарования этой группой, Вы это интерпретируете как
эмигрантскую тоску. На самом деле, это конечно так, с одной стороны, но это
хорошая попытка - эта группа - как то начать собирать людей заново вместе, на
самом деле, потому что по настоящему людей может собрать вместе какие то
светлые вещи, даже та же память о мирном Мариуполе... или же мечта, видение,
как надо будет отстраивать Мариуполь, а не вместе пережитые ужасы. Так вот...
Вы сказали, что эта группа - конденсат страдания и разрушительной тоски. Я с
этим не согласен, нет. Во-первых, эта группа содержит ценнейшие фотографии,
впечатления, переживания людей, которые жили в этом Городе. Очень красивые
фото, и воспоминая тоже очень красивые! И это особенно важно для тех людей, как
я, которые никогда не были в Мариуполе, и которые могут узнать о Мариуполе
только прочитав о Городе у тех, кто там жил, учился, работал, в общем, от самих
мариупольцев. Поэтому для меня - эта ценнейшая вообще группа. Да... конечно, я
вижу много тоски, я вижу как люди тоскуют... Но это всего лишь комментарии, это
отпечаток душевных переживаний людей, спасшихся от ужасов и смерти. А фото
говорят сами за себя, да, на фото говорит сам Город, да, прекрасный необычайный
Город Мариуполь! Да! Мариуполь жив! И его нужно собирать по кирпичикам назад.
Военные делают свое дело, и в свое время освободят Город, как Залужный
обещал... Но... Залужный не в состоянии оживить Город, не в состоянии вдохнуть
в него жизнь заново, это явно не его парафия... Это дело мариупольцев, и это
дело Украины, и всех украинцев, вдохнуть Жизнь, Дух, Радость заново в Город...
Это задача большая, культурная, духовная, душевная... и для воплощения этой
цели в жизнь как раз и нужна Память про Мирный Счастливый Мариуполь. Поэтому я
и лажу по этой группе, собираю по частям все что могу собрать, я уже ссылки
кидал - заходите, смотрите, подписывайтесь на телеграм канал. Далее.. Мариуполю
нужны Книжки! Да... много много книжек. Я живу в Киеве и вижу, что книжек про
Мариуполь практически нет, или очень мало (да, вышла книжка Надежды
Сухоруковой, но это не меняет общей картины, более того, эта книжка довольно
мрачная, как по мне... ну неудивительно, - это дневник, фиксирующий все те
ужасы, очевидцем которых была автор, - и эта книжка на дает ответов на вопрос -
а как же собственно говоря жить дальше, как построить заново свое Счастье, как
собраться вместе снова, и как снова вернуться в свой родной Город). Так вот.
Книжек про Мариуполь очень очень мало. И я как раз занимаюсь всем этим. Я хочу,
чтобы Мариуполь начал возрождаться духовно, культурно. а для этого нужно (1)
сберечь память как про мирный, так и про блокадный Мариуполь. И здесь очень
очень работы впереди, и это нужно делать прямо здесь и сейчас, потому что все,
что в фб, инстаграме, телеграме, пока это не перенесено в печатный вид и не
сохранено надежно, может быть потеряно навсегда в любой момент, достаточно лишь
нажать кнопку. Поэтому, еще... насчет этой группы... цель должна быть в
сохранении и систематизации и публикации всего, что пишут люди в этой группе.
Эта группа - очень очень ценная! И да... насчет книжек о Мариуполе... Вот,
кстати, вчера написал об этом писателю Максу Кидруку, может быть это
подействует

%21:09:40 05-03-23
доброго вечора! дякую! так, ці книжки (Асєєва) я бачив в книгарнях, Маріуполь -
Останній Форпост тільки десь онлайн, але все це ще не читав... треба буде
почитати... ох! а щодо власної долі... ніколи наперед всього не передбачиш...
знаєте... кияни в Києві 1941 року... теж нічого не передбачували такого
особливого... євреї, яких потім мільйонами знищували... теж якось ні про що
особливо не задумувались... Те, що сталось, вже повторювалось в історії
багато-багато разів в різних епохах... Жили собі люди. жили, не тужили... І...
тут раптом налітає грім, блискавка, наступає повний капец... Знаєте, як море?
Ось Азовське море... Білосарайська коса, я фоточки дивився з групи Маріуполь
Довоєнний... (автор Олена Сугак https://www.facebook.com/lenabidulia ). Там
була якраз фотка про те, як така величезна хмара лізе (ось фото знизу +
текст)... Ну, Добре... я взагалі то вірю в Бога... так от. Христос колись
сказав таке... Не складайте собі скарбів на Землі, де міль та іржа знищать їх,
де злодії можуть вдертися й викрасти їх. Краще збирайте скарби для себе на
Небі, де ні міль, ні іржа не понівечать їх, де злодії не вдеруться й не
викрадуть їх... Бо де скарб твій, там буде й серце твоє!


Доброе утро! Спасибо, интересно! Но... знаете... меня все эти подробности
насчет того... это агитка или нет... не очень волнуют. Сказки - не сказки. Я
кстати уверен, что этот ролик - правда, несмотря на Ваши аргументы. Чутьем
вижу. И я кстати из тех людей, которые привыкли головой думать, в принципе -
было бы желание, мог бы по всем этим аргументам пройтись глубоко, и во всем
разобраться. Но... мне это сейчас не нужно. Да.... Меня волнует Сила Духа,
Стремление к Победе. В общем, духовные вещи. И этот ролик - вообще то
замечательная вещь, которая здорово поднимает дух, когда его смотришь.
Понимаете... но Вы смотрите со своей колокольни... И... дело наверное даже не в
деталях, правда этот конкретный ролик или нет, Вы как специалист в военном
деле, служивший в армии, и имеющий опыт в этом деле (не зря же все сейчас
конкретно так расписали - очень здорово! ), - все равно переспорите меня,
программиста, который в этом деле вообще не разбирается... а знаете, вот в чем.
Этот текст, что я привел, был написан в ответ на пост с видео, где есть
маленький саженец посреди рельсов, и через него сверху проезжает огромный
поезд, проезжает, проезжает, потом проехал, страшно, и все. И в конце подпись -
365 дней незламности. И это видео - вообще полная депресуха - но людям
нравится, они питаются этим. Ейбогу, это так печально, это так грустно смотреть
на это. Потому что Украина - это никак не одинокий саженец посреди камней и
рельсов, а расея - уж явно никакой не мощный бронированный поезд, который мчит
с огромной скоростью, и который всегда может вернуться... Понимаете. Видео про
поезд, - я где то его могу найти, - может быть рашисты вообще запустили, потому
что это весьма похоже на такую тонкую психологически-информационную операцию.
Понимаете, многие люди, особенно из Мариуполя, к сожалению, довольно слабы в
психологически-информационном поле, так легко подхватывают и начинают
распространять какую-то откровенную депрессивную вещь, не пытаясь как следует
вдуматься, а что же они именно распространяют


%20:27:18 06-03-23
сочувствую Вам очень... думал, что сказать... а скажу так... (1) Язык до Киева
доведет... это еще древние знали... когда то давным давно... в седые времена...
люди не имели ни телефонов, ни машин... но они имели Язык... они спрашивали
дорогу, как мне пройти в Киев? как мне добраться в Стольный Град Киев? Как мне
добраться до Софии, или же до Подола? и им охотно отвечали... пройди туда,
потом сюда, потом будет лес, такой темный лес!  ты туда не иди, а обойдешь вот
так... потом будет чудесная речка, а там вот мост - а потом холм, а потом поле,
а потом там будет дорога, будешь по ней идти день, и потом ты уже увидишь наш
прекрасный Золотой Город Киев ))) в общем, надеюсь вы поняли, что я имею в
виду... спрашивайте, спрашивайте, не стесняйтесь... и найдете дорогу. Это же
Киев, тут полным полно людей... И что-то знаете, есть очень такое глубокое в
этой пословице... Память о старых-старых временах, когда все было иначе, чем
сейчас! Да, я когда еще в школе киевской учился, читал про Киевскую Русь, все
время мне казалось очень несправедливым, что тот, древний Киев, был уничтожен,
случилась тогда катастрофа 1240 года... Я это свое воспоминание, впечатление
школьное очень даже хорошо помню! Да, язык до Киева доведет... так что
спрашивайте, не стесняйтесь... Так, дальше... (2) тут на фб кстати есть
замечательная группа Киевские Истории, там во-первых все время какие то
замечательные и интересные истории выкладываются, я сам все время что то новое
узнаю, кроме того, что штук 200+ этих историй записал себе в свое время (в
таком же формате, как Вы видели), а во-вторых, регулярно объявляются экскурсии,
так что если Вы присоединитесь к экскурсии, то точно не потеряетесь, это раз, а
два, познакомитесь с интересными людьми... А три. Вы всегда можете
путешествовать в одиночку по Киеву... просто открыв Книгу... Вот мне например
очень нравится творчество Павла Загребельного, у него есть роман Смерть в Киеве
- рекомендую... А начинается этот роман такими словами, наверное, моя самая
любимая фраза о Киеве, щас...  
 
Київ був повен лагідного сяйва. Воно лилося згори, із спокійного осіннього
неба, високо знесеного над київськими горами, знизу піднімалося йому навстріч
сяйво зелене, а між зеленим і високоголубим тихо плавало золото соборів, легко
лягала поміж ними перша пожовклість листя, і неначе в душу вхлюпувалося оте
м'яке світіння, і відчувалося, що, входячи в цей Город, мовби стаєш безсмертним...

понимаете, этот Город... таких Городов больше нет на Земле... 
потому что этот Город - сама Вечность...

знаете, я вот думаю, мне зачастую трудно что то советовать людям, которые
пережили такое... думаешь, а вообще это к месту или нет... ну ладно, скажу...
знаете, я уверен, все это у Вас пройдет, перестанете бояться и сирен, и
метро... и знаете... я вот недавно был в Софии... знаете... это совершенно
особое место... вот если вам тяжело на метро... может, запланируйте, там, через
месяц-два или когда... когда решитесь... может быть... так чтобы взять такси
туда-сюда, чтобы вас отвезти туда-сюда без мучения по городу... в общем, это
уже Вам решать, сообразно Вашему бюджету... Это просто такое место... в котором
можно находиться вечно... знаете... вот там Оранта смотрит... уже 10 веков
почти... а вот тут графити с надписью про смерть Ярослава Мудрого... ходишь.
думаешь... спокойно так, почти никого нет, кроме тетушек, которые сидят и
смотрят за всем...  смотришь на мозаики, фрески, графити, тихо так... никуда не
спешишь... даже воздух какой-то такой... знаете, сладкий такой... и я вот там
был недавно, просто передаю свои впечатления (и скоро фотки выложу отдельно у
себя). Собор то сам небольшой сам по себе... но место совершенно особое...

Доброе утро! Я тут хотел бы еще поделиться... относительно Киева. касательно
понятия киевлянин. Киевлянин - само по себе - понятие очень своеобразное. И
каждый его трактует как бы по своему. Особенно есть такое забавное понятие
коренной киевлянин, которое по мне, ну, довольно бессмысленное, когда его
применяют по отношению к конкретным живым людям. Но... как по мне.  Киев -
настолько огромен, не столько пространственно, сколько исторически и
культурно... он же протянут не только географически, но также уходит вглубь
веков, что... для меня - понятие киевлянин - это понятие культурно-духовное,
понятие привязанности, любви к Городу, а не только понятие прописки, права
рождения, или того, что там дескать, я там типа киевлянин в каком то поколении.
Потому что... потому что, как я уже сказал, Киев насколько безумно огромен
культурно и исторически... что мы все по отношению к нему - на самом деле
гости... Единственные коренные киевляне - я думаю, знаете... это... Святые
Угодники в пещерах, ну и конечно, в духовной сфере, Архангел Михаил,
Богородица... потом, самые разные литературные персонажи, исторические деятели
прошлого - вот это, коренные киевляне, да, потому что они уже навечно вросли в
Киев своими достижениями, творчеством и так далее. Ну и... вот Вы мне уже
открыли с неожиданной стороны Киев - и для меня Вы уже киевлянка, потому что
вкладываете душу, сердце в Киев. Конечно, если вы вдруг начнете говорить, что
вы киевлянка, потому что вы уже наделали кучу фоток, и вы дескать любите Киев
больше чем другие, потому что вам об этом сказал один киевлянин-программист...
вас точно не поймут, потому что киевлянин - это не то понятие, которым нужно
как то сильно хвалиться, что ли... меряться своим киевлянством, это явно был бы
дурной тон... Но... Киев - Город Открытый для всех, Киев - это Мать, которая
готова обнять каждого, кто в этот Город приезжает... не зря же еще в седые
времена его называли Мати Градом Роуским. Конечно, Киев Вам заменить Мариуполь
наверное не сможет... но... сейчас в Киеве мир. Конечно, вообще то идет в
Украине ужасная страшная война, но конкретно в Киеве - мир. И он уже дает Вам
силы и будет давать силы жить, радоваться и дальше, потому что Киев - это
щедрый Город... у него внутренней, духовной силы, хватит на всех! сколько
хочешь, просто бери себе сколько хочешь, Киеву не жалко... Как я уже написал в
своем посту... Киев - это Духовный Узел... Он соединяет в себе Мариуполь и
Львов... Харьков и Ужгород... Донецк и Тернополь... Одессу и Чернигов.. Юг и
Запад, Север и Восток сходятся в Киеве... Вот как гимн, як тебе не любити,
Киеве Мий! В нем собственно кратно и красиво выражено все то, что я вам сказал
выше! Замечательный гимн, кстати, в прошлом году ему уже 60 лет исполнилось! А
вот еще одна песня про Киев, менее известная, наверное, исполнение коллектива
Сонях Дворца Юношества (тот, что возле Аллеи Славы)
https://www.youtube.com/watch?v=BjGIOUV3va8

%10:48:07 07-03-23
Так, звісно Київ Вам не замінить Маріуполя... я теж так думаю... щодо каштанів,
ось цікаво, ніколи не задумувався над таким питанням, так що прям не знаю, що
відповісти! ))  а щодо українського Маріуполя. Він ВЖЕ Вас чекає, повірте. Він
ВЖЕ Вас чекає... Але... всьому свій час. Маріуполь... знаєте. Він... живий,
абсолютно. Тільки... він... затаївся, завмер. Розумієте. На деякий час випав із
часу-простору, застиг... щоби потім знову розцвісти, засяяти, розквітнути по
новому... Але... Маріуполь, знаєте, так... чекає... і йому потрібні
маріупольці, що повернуться з новими враженнями від усього світу. Начебто
так... Питає... ти повернувся до мене. Де ти був, що ти бачив, за той час, поки
я (Маріуполь) перебував сам по собі, у фото, пам'яті? Чи був ти у Києві, а коли
був там, що ти встиг побачити... чи бачив ти голубів та озера та парки (як це
бачити та фотографуєте Ви), або ж Софію, або ж Володимирський Собор, або ж
Величний Дніпро...? Якщо ти був в Парижі, чи побував ти в Луврі або в абатстві
Сен-Дені, де поховані всі королі Франції і дворянство різних часів? Якщо в
Італії, чи ходив ти по музеями, чи бачив ти падаючу Пізанську Вежу, або ж
Місто-Квітку Флоренцію? Чи розповідав ти по світам іншим людям, яке чудове
неймовірне місто Маріуполь? За той час, поки ти був далеко від мене
(Маріуполя), чи написав ти книжку про мене тощо... І чи зберіг, записав ти свою
історію, так, щоби всі могли про це дізнатись... про блокадний Маріуполь... Чи
пізнав ти інші світи, чи завів ти друзів там або там, щоби їх можна було потім
запросити у гості до себе додому? І Маріуполь буде питати строго-строго... тому
що ті, що тільки і думають про втрачений Маріуполь, весь час, перебуваючи у
журбі і розпачі, забуваючи про все інше... їм буде на жаль нічого відповісти на
ці питання... і Місто їх буквально вб'є на місті або вижене геть...
Розумієте... якщо я приїду в Маріуполь - навіть якщо я приїду в розтрощене,
зруйноване Місто... Місто-Кладовище... одразу після звільнення - Місто нічого
зі мною не зробить... тому що я стороння людина. Так... мені буде дуже важко
дивитись на все, можливо, я заплачу, і добрячи таки заплачу, але і тільки... А
от для маріупольців це буде такий собі страшний суд... Але добре... ось доречі,
ще штука про голубів. Недавно я бачив ось таку фотку. Як на мене, оці дівчатка
чудово пасують до героїв книжки Джеймса Крюса про Флорентіну (таке собі
дівчисько, що бігає по місту з голубами і з своєю подружкою, весь час встряє в
якісь пригоди )...

%15:53:18 09-03-23
Так это ж Ваш выбор, не мой ) мне то что.  Да... а Шевченко тоже как то
проживет и без вашего упоминания. Так что... Хотите - варитесь дальше в своей
печали, ненависти, собирайте самые плохие новости, публикуйте, насыщайтесь
праведным гневом и т.д. И по факту - Вы уже проиграли рашистам, потому что они
сделали Вас заложником, рабом своих безумных деяний... Знаете... Каждый - сам
кузнец своего счастья. Потому что только сам Человек решает, и никакой путин
ему не в силах помешать.. Только лишь сам Человек и решает... Знаете, как у
Тима Талера. Научи меня смеяться - Спаси Мою Душу. Упс. Наверное, щас точно
забаните )))

%21:49:39 09-03-23
Доброго вечора, пані Тетяно! Я бачу Ви (1) вдодобали мій коментар щодо Шевченка
(2) були у дискусії з пані Надією щодо мого коментаря. Мушу сказати, що я не
буду коментувати слова пані Надії щодо мене, моїх начебто погроз смерті і т.д.
По-перше, це приватна переписка, і я не мав ніякого бажання цим ділитись з ким
ще, крім пані Надії, і не збираюсь ділитись для свого виправдовування. Але
оскільки пані Надія, підняла це питання, і в дуже різких тонах, я скажу так. Я
дійсно написав пані Надії в персональних повідомленнях - після того, як вона
мене видалила з друзів на фб - досить своє різке ставлення до різних речей, але
про такі божевільні речі, як побажання смерті, я звісно навіть і подумати не
міг. Навпаки, пані Надія - неймовірно сильна жінка, і я звісно не міг і не буду
бажати їй смерті, навпаки, я бажаю, щоби вона жила довгим щасливим життям, я
щиро їй бажаю бути по справжньому сильною і незламною, і я щиро бажаю їй
повернутись у свій Український Маріуполь. Більш того, я прочитав вже всю книжку
пані Надії, і навіть прорекламував її в декількох місцях (і пані Надія це
бачила, і навіть сказала мені дякую). Але... на жаль, пані Надія пройшла у
минулому через дуже важкі випробування, як і взагалі всі маріупольці, і це
можливо стало причиною її такої неймовірно різкої і викривленої реакції, так що
я не ображаюсь. Їй можливо хотілось першою написати про Шевченка (я бачу, вона
вже написала, зобразивши Тараса з автоматом), а я їй в цьому завадив, ну... не
знаю... Зараз війна, люди на взводі, як мовиться... А щодо пані Наталії, я
нещодавно скинув на ноут для хлопця на зборі, який вона сама відкрила... (і
досить таки непогану сумму...), і це далеко не все... Але пані Наталія того вже
не пам'ятає... Ну, добре, я якось переживу...  Так, для доповнення картини...
Щиро Ваш, Іван. PS. А щодо моєї начебто пустої сторінки, вона далеко не пуста,
містить (1) мої власні пости та думки (2) мою роботу по Літопису Війни, яку я
веду як програміст у свій вільний час. А взагалі то я киянин. Ось якось так.
Крім того, я граю в шахи, і сьогодні до речі, крім дня народження Шевченка,
також день народження Боббі Фішера - американського чемпіона світу з шахів
(якщо Ви раптом цікавитесь шахами...)

%23:24:26 09-03-23
Доброго вечора! Дякую за розуміння. Напевне, не варто зараз дуже багато писати,
але трохи допишу. Підтримка не повинна означати потакання слабкостям. Підтримка
повинна бути дієва, продумана, та така, що врешті решт дозволить їх витягнути
із цього пригніченого стану, повернути їх у стан таких самих людей, як Ви або
я... Я от персонально не хочу бачити слабких маріупольців, які весь час у
розпачі і печалі, і які так несамовито відбрикуються (скочуючись на жаль до
відвертої неправди щодо інших) від щирих порад та добрих справ, які для них
роблять інші небайдужі люди (як це було у чаті...). Я хочу бачити сильних
духом, дійсно незламних Людей, що здатні творити дива! Бо попереду
багато-багато дуже важкої роботи, яка буде потребувати багато-багато розумових
і духовних сил! Розумієте... Всім зараз важко, дуже важко. І херсонцям, і
харків'янам, і полтавцям, і маріупольцям, і одеситам, і жителям Миколаєва...
усім українцям. Важко Воїнам ЗСУ, що зараз боронять Бахмут, важко енергетикам,
важко лікарям, важко патріотам в окупації, важко волонтерам, важко тим, важко
сим, Буданову або Залужному от теж важко. Він (Залужний) можливо зараз по 20
годин на добу працює, сидить в ГенШтабі, сидить сидить думає, планує... І я
персонально проти того, щоби маріупольці - при всій моїй повазі та сердечному
відношенню до них як до талановитих та чудових людей (які на жаль зазнали
страшної трагедії) - перетворились на таку собі печальну замкнену вибрану
касту, які завжди в печалі, які думають тільки про зруйнований Маріуполь, не
помічаючи нічого навкруги, які не бачать ніякого світла попереду, яких ніхто не
розуміє ну і тому подібне. А така небезпека дійсно існує, і від цього нікому
добре точно не буде... Надобраніч!

%07:53:42 10-03-23
Доброго ранку! Дякую! Дивіться... щодо психології. Тут справа не стільки в
психології, скільки... банально в тому, що маріупольці мало спілкуються з
іншими українцями, + Україна начебто забула про Маріуполь в масовому
інформаційному полі. Це видно (1) по відсутності книжок про Маріуполь - (2)
видно по темам, які зараз в телеграмі або фб. Звичайно, про Маріуполь пишуть
теж, - але, беручи до уваги болючість цієї теми, - це неймовірно мало. І... я
не хочу сказати, що Маріуполь наприклад важливіше за Бахмут, тому що Бахмут то
є теж дуже важливо. але... ось наприклад, Грузія. Весь час зараз мусолять оту
Грузію. Там зараз начебто у них їх Майдан. Грузія - то є дружня для нас країна,
звісно. Але... наші внутрішні найболючіші питання повинні бути точно
важливішими, і це повинно бути чітко видно по тому, як люди пишуть. А у нас
зараз навпаки все. Грузія.. безліч інших тем... а про Маріуполь дійсно
практично не пишуть і не цікавляться. Фото групи Маріуполь - довоєнний.
Коментарі практично одних маріупольців - як все було гарно, як шкода... чат
пані Надії, сторінка пані Надії - практично теж саме. Тому вони й варяться в
своєму соку, і тому так агресивно віднеслись до мене, в тому числі. Тому що
вони вже звикли, що вони весь час в якійсь вже печалі, а тут хтось вліз зі
своєю думкою... Посмів тут якісь позитивні думки штовхати. Я думаю, тут не
потрібен якийсь специфічний тонкий психологічний підхід, тут потрібно просто
спілкування, спілкування, і ще раз спілкування. Просто інші українці дійсно
повинні знайти час та натхнення просто почати писати та говорити про Маріуполь,
особливо про Маріуполь мирний, радісний, щасливий. Так, потрібно дуже багато
спілкування. А українців зараз масово цікавлять інші теми, от і все. Виходить
зачароване коло. Українці не цікавляться Маріуполем - тому що немає книжок, про
Маріуполь мало хто пише в тг, фб - маріупольці замикаються в собі - немає сил
писати ті самі книжки - а якщо немає книжок, значить, ніхто про це не читає, не
цікавиться ...

а щодо єдності... от дивіться. Єдність зараз багато людей розуміють як
об'єднання навколо люті, ненависті до ворога, єдність навколо важких думок про
минуло. Але... на цьому грунті єдності не збудуєш. Єдність можна збудувати
навколо Любові до чогось, навколо Цінностей, навколо Візії Майбутнього, навколо
якихось спільних творчих зусиль. До речі. Ось щодо Тараса. Його зараз масово
зображають як такого собі могутнього воїна зі снайперською гвінтивкою. Ось пані
Надія витерла мій коментар, викинула мене, натомість написала теж про Тараса з
гвинтівкою. Але... Як на мене, це просто знущання над пам'яттю Поета та
Художника. У нас що, в історії не було Видатних Воїнів, Воєначальників,
Святослава Хороброго, Максима Кривоноса, Богдана Хмельницького, Петра
Конашевича Сагайдачного, Байди Вишневецького, багатьох-багатьох інших... нащо
усюди пхати Шевченка у військовій формі. Шевченко - це поет та художник, так,
геній українського народу, і його Слово живе у Серцях Мільйонів. Але він точно
ніколи не був воїном, і більш того, військова служба для нього у свій час - то
було взагалі тяжке покарання для нього.

а то звісно. Я просто маю на увазі... що... у кожної нації, особливо нації
великої - як от українці - є свої Поети та Письменники (Шевченко, Леся
Українка, Олександр Олесь, Ліна Костенко інші інші інші...), є свої Філософи та
Мислителі (Ілларіон, Сковорода, багато багато інших), є науковці (Мечников,
Амосов, Богомолець, Боголюбов, інші, інші, інші), є свої Державні Діячі
(Володимир Великий, Ярослав Мудрий, Богдан Хмельницький, Святослав, інші інші)
є свої Воїни - як в історії, так і вже в сучасності. І кожному своє. Мені от
наприклад, неймовірно важко уявити собі в реальному житті Шевченка, Лесю
Українку або ж Сковороду із снайперською гвинтівкою, бо їхня зброя - Сильне
Могутнє Слово, Зброя Духовна, а не фізична. Для цього є Святослав або Максим
Кривоніс або хтось ще. Розумієте. Ми - дуже багата нація на таланти, і у нас є
безліч інших визначних постатей в різних областях, окрім Шевченка.

Так, військова тема. Але... знаєте. На війні - війна, а миру - мир. Що я маю на
увазі. Є місця, де є війна, і це абсолютно природньо там мати сильні почуття -
як от ненависть до ворога, лють, помста. Це є абсолютно природньо. Там - війна.
І є місця, де зараз мир. Наприклад, в Києві. І там, де є мир... потрібно
знаходити час та натхнення, щоби черпати енергію миру, розумієте. Я досить
часто бачу таке. От наші військові. Їм дуже важко, але вони духовно виглядають
міцніше, ніж багато людей в тилу, яким безпосередньо вже нічого не загрожує
(окрім того, що ракета може влучити в них під час оцих ракетних атак)...Так,
люди знаходяться в тилу, але духовно вони набагато слабші, ніж ті люди, яки
безпосередньо воюють. Набагато слабші, тому що... замість того, щоби черпати
свою енергію з миру, передаючи її потім тим, хто потребує цієї енергії - вони
насичуються поганими новинами, фото, спогадами...  і т.д. Розумієте. Перемога
потрібна не лише чисто військова, а також Перемога Духовна, Душевна, оскільки
війна йде не тільки фізично, а йде війна Слів, Смислів, Війна Когнітивна,
Ідеологічна... Так от... Духовна, Душевна Перемога -  це вже задача кожного,
задача персональна. Тут ЗСУ в цьому ніяк не допоможуть, це явно не їх парафія

ось до речі... можете подумати. Я недавно бачив ось такий ролик, можете його
знайти - величезний поїзд наїжджає на рослинку, потім проїзжджає, в кінці
надпис - 365 днів незламністі. Багато людей його перепостили, не усвідомлюючи,
яку депресивну маячню вони поширили. Скоріш всього, це росіяни запустили в наш
інфопростір. А чому я так вважаю. А тому що звісно, Україна - то є ніяка не
одинока рослинка посеред каміння та рельсів, а московія - точно не величезний
важкий поїзд. А люди це постять і навіть радіють - кажуть - о! це точно про
нас! А насправді - це є типовий приклад психологічно-інформаційної
спецоперації, які ворог намагається провести у нашому інформаційному просторі.

справедливо відреагувала на російську??? Це що взагалі таке? Надія сама
російськомовна, щоденник свій вона вела російською, її пости російською, так, і
вона і зараз багато пише російською, як і багато учасників чату. Так що Ваш
коментар щодо російської взагалі недоречний... Вона просто втратила контроль
над собою - що є свідоцтво психологічної слабкості - і почала нести маячню, от
і все.

а щодо відсутності фото та ін, знаєте, може це й краще так. Тому що дивлячись
на останню неймовірно неадекватну реакцію... хоча я все розумію... і не
ображаюсь... знаєте... ділитись ще своїми фото, нащо? щоб мене взагалі з'їли з
потрохами?

а щодо української. Це неймовірно важливо, звісно, зараз говорити, писати
українською. Багато писати, говорити, усюди. Мова - без сумніву є код нації,
Духовний Стержень України. Але коли люди починають вилізати із своїх штанців, і
починають шалено вдавати із себе тих, ким вони не є... на це дивитись... якось
не дуже приємно, знаєте.

Тримаймось, так! Але... переможемо ми... в тому випадку, якщо переможемо.
Вибачте за тавтологію, але історія знає купу прикладів, коли народи або просто
зникали під тиском обставин та більш могутніх сусідів, або ж зазнавали
неймовірно страшних поразок... Тому... перемогу ще треба виборювати і
виборювати...  Кожен день... А по-друге... ми ВЖЕ найкращий народ! Прямо зараз
і тут )) Ось доречі, я зараз був на Майдані, ховали Дмитра. Це просто
неймовірно, скільки людей прийшло. І малі діти - майже немовлята.., і старі, і
побратими, я бачив також літню людину на костилях... людина сама прийшла,
розумієте... і ніхто навіть не подумав, що це може бути страшно, що якщо
раптом, що ракети і т.д. І начебто, там десь збоку я бачив нашого Президента
(але може мені це здалось тільки... )

%21:07:48 11-03-23
Натали Михайловская
А если б нашелся человек, который бы помог нам издать альбом с фотографиями нашего Мариуполя?? И был у нас фотохудожник Виктор Дедов, он погиб в марте при обстреле города. Его жена, Наталья Дедова смогла выехать из города.. У Викт… See more

Іван Іван
Натали Михайловская 

добрый день. Я из Киева, я программист, вот как раз таким занимаюсь. Издать
альбом физически я вряд ли Вам помогу, это вопрос уже конкретного договора с
издательством, либо заказа на печать в какой то конкретной конторе, а вот с
электронной версией, которую можно будет потом распечатать в конторе или
разослать друзьям и по всему свету - нет проблем вообще. Я уже кучу постов
записал для себя про Мариуполь, и для меня не будет проблемы технической
составить электронный альбом с какими угодно фото Мариуполя, хоть на десять,
хоть на сто страниц. А насчет увидет весь мир - нужно размещать такие альбомы в
таких местах, где их никто не удалит, и все смогут увидеть, например на ресурсе
Машина Времени - Интернет Архив - https://archive.org/details/@kyiv_chronicler
(пока что там только копии некоторых постов про Мариуполь - и фотоальбомы в
постах других людей - в том числе)

%21:04:41 11-03-23
спасибо, без проблем. Моя здесь роль только в технических навыках, оформлении и
т.д. В принципе, это может делать любой человек при желании, тут нет никаких
особых тайн. Технология, которую я использую, называется LaTeX. И... Просто я
уже набил на этом руку, и поэтому делаю такие штуки очень быстро. А примеры
моей работы уже выложены по ссылке выше (конечно, это не единственный вариант
оформления, Латех позволяет делать оформление какой угодно степени сложности).
А насчет фото Виктора Дедова. (1) на самом деле фото Виктора не принадлежат
лично Наталии, эти фотографии абсолютно бесценны и принадлежат Городу
Мариуполю, мариупольцам и вообще всей Украине, при всем уважении к памяти
Виктора и сочувствии к Наталии и ее личной трагедии потери Виктора, как и
трагедии всех вас. Потому что... эти фото не лично-семейные, это фото
прекрасного Города, в создание и построение которого вложило сил и души
огромное число людей в разные времена. Это моя личная позиция насчет авторских
прав вообще любых фото Города Мариуполя (2) лично у меня с Дедовой в последнее
время возникли некоторые разногласия, в общем, комментировать не буду здесь,
поэтому, возможно, если Вы будете обсуждать этот вопрос с ней, она Вам может
наговорить всяких разных вещей насчет меня... но, здесь вопрос не в каких то
личных разногласиях, а в том, чтобы память о Мариуполе, как мирном, так и
блокадном, стала доступной как можно большему числу людей, да, вопрос состоит
именно об этом, а так же о будущем Мариуполя, о возрождении Мариуполя... Это
необычайно нужные вопросы глобальной важности для мариупольцев и вообще для
всей Украины, и я буду счастлив помочь Вам в этом с техническо-информационной
стороны

%08:03:51 14-03-23
Доброго дня ) дуже гарний дайджест ) але є таке побажання, якщо дайджест йде по
дням, то можна було б зробити такі собі вставки щодо культури та історії, як
української, так і всесвітньої.. (день в історії) посилання до того, чим є той
або інший день у світі... тобто... наприклад, 12 березня - народився
Вернадський, а помер Іов Борецький... 13 березня - Макаренко та Андрухович,
також, в 1961 в Києві сталась Куренівська трагедія, 14 березня - це день
Українського Добровольця, так... але... крім того, в цей день народились...
Альберт Ейнштейн, Олекса Новаківський, та Амвросій Бучма, померли... Стівен
Хокінг... а день... також міжнародний день числа пі, якось так ) (і ще...
Нестеров та Руденко здійснили в цей день перший авіапереліт з Києва до Одеси в
1914, а в 1990 в Стрию був піднятий перший український прапор)

%12:37:22 17-03-23
тут щодо мапи... (1) незрозуміло, звідки вона взялась взагалі, де першоджерело,
хто робив якесь дослідження (підрахунки - напевне, з даних офіційного перепису
тощо), а чи це просто хтось намалював на колінці і потім запустив в мережу (2)
а без надійності першоджерела (дослідження) робити взагалі якісь узагальнюючі
висновки якось передчасно (3) щодо прізвищ і т.д. Сучасна Історія України
творилась і твориться конкретними людьми, яких ні в яки статистичні мапи з
прізвищами запихнути неможливо... Де от тут наприклад на мапі Сергій Нігоян,
чиє обличчя зараз дивиться з муралу на Михайлівську площу в Києві, або ж
Мустафа Найєм, чий заклик збиратись із друзями на Майдані в 2013 році призвів
до Майдану-2, тим самим назавжди змінивши Україну, або ж Ольга Харлан, наша
знаменита шаблістка і т.д. ... Історія України та її майбутнє твориться
конкретними Людьми, а не прізвищами...

%21:41:22 17-03-23
%draft chat nadezhda
Добрый вечер. Меня здесь упомянули. Во-первых, меня зовут Иван, и я это уже не
раз говорил. А во вторых... Я так понял уже после общения с Надеждой, что в
общем то мариупольцы хотят остаться наедине сами со своей болью, и не хотят,
чтобы кто-то со стороны им сочувствовал, предлагал какие то советы и так далее.
Я это так понял, уж извините. Так что ничего добавить здесь по сути не могу,
могу только пожелать всем присутствующим здесь мира и спокойствия в душе, и
скорейшего избавления от вашей боли.

%12:47:28 18-03-23
% https://www.facebook.com/profile.php?id=100073246549131
% Oleksandr Baranovskyi

Доброго дня, пане Олександр! Випадково натрапив на Ваш допис щодо LaTeX. Щодо
мене - активно користуюсь LaTeX. Сам я киянин, програміст, живу в Києві.
Використовую LaTeX у своєму проєкті Літопис Війни - власне кажучи, запис
різноманітних постів, в основному в фб, потім систематизація по авторам, темам.
Це треба так чи інакше робити, бо фб - дуже ненадійне середовище для зберігання
пам'яті. Цим всім я вже займаюсь більше року, у свій вільний час... Як все це
виглядає, можна подивитись ось тут https://t.me/kyiv_fortress_1 або тут
https://archive.org/details/@kyiv_chronicler. Останнім часом в основному
займаюсь темою Маріуполя. До речі. щодо Маріуполя та його культурно-духовного
відродження, ось мій пост від 3 березня,
https://www.facebook.com/ivan.ivan.kyiv/posts/pfbid02X7PFBDFiFcNY9Ge7mY2XDvykevGa8KWRUyXKK5MPFsH8NyFSstBCeZPb2GoCQ4mkl.
З повагою, Іван.

%15:57:12 24-03-23
Добрий день! Можливо, комусь цікаво. Я займаюсь архівацією, збереженням важливих
фб-постів. Під це є власний проєкт Літопис Війни. Сам я програміст, живу в
Києві, мене звати Іван. Якщо є потреба - можу заархівувати як копії будь-які
пости в печатному вигляді (з коментарями, скрінами, html-оригіналом). Як це
виглядає (останнім часом займаюсь постами про довоєнний мирний Маріуполь, бо це
особливо є важливим), можна подивитись  тут  https://t.me/kyiv_fortress_1 або
тут https://archive.org/details/@kyiv_chronicler

%23:22:44 27-03-23
дякую! так, на жаль, не бачу Вашого допису щодо вистави в музеї... можливо, фб
щось там заглючив... шкода! тітоньки... мені якось запам'ятались тітоньки в
Софії Київській... навіть не знаю чим... просто... як би так сказати... сидять
там цілий день в дуже особливому місці, а людей майже немає... тиша, спокій
навкруги... а вони сидять собі... і оберігають Вічність... і самі вже трохи
здаються Вічними... Хранительки Вічності... а щодо вистав... от у мене є
улюблений театр в Києві - навчальний театр Універу Карпенка-Карого, біля
Львівської Площі. Грають звісно студенти, але грають просто агонь ))

%08:37:29 25-04-23
до речі. Вчора я був в тому районі, де ваша кафедра і універ. Так от. Я побачив
там щось дуже цікаве. Якщо Ви зайдете в книгарню Є навпроти універу
архітектури, підете до кінця в закуток зліва від каси і переберете книжки про
мистецтво, можливо, Ви щось знайдете там неймовірно цікаве і важливе для себе і
для інших маріупольців також...  Але поспішайте, книжки мають властивість бути
проданими...
