% vim: keymap=russian-jcukenwin
%%beginhead 
 
%%file 02_12_2020.news.ru.lenta_ru.mozzhuhin_andrei.1.pribaltika.reichs_kommisariat_ostland
%%parent 02_12_2020.news.ru.lenta_ru.mozzhuhin_andrei.1.pribaltika
 
%%url 
 
%%author 
%%author_id 
%%author_url 
 
%%tags 
%%title 
 
%%endhead 
\clearpage
\subsubsection{Рейхскомиссариат «Остланд»}
\label{sec:02_12_2020.news.ru.lenta_ru.mozzhuhin_andrei.1.pribaltika.reichs_kommisariat_ostland}

Почему после оккупации Прибалтики летом 1941 года Германия не позволила
литовцам, латышам и эстонцам иметь свои государства или хотя бы автономии, в
отличие от хорватов и словаков, которые вообще были «расово неполноценными»
славянами?

Как я уже говорила, Гитлер смотрел на прибалтийские народы с опасливым
презрением. Поэтому даже того подобия государственности, что получили от него
хорваты и словаки (для уроженца Австро-Венгрии те были «своими славянами»),
нацисты никогда бы не разрешили латышам, эстонцам и литовцам. Те надеялись, что
немцы помогут восстановить их независимость, но у Берлина имелись иные планы.
Даже на статус колонии Прибалтика не имела шансов претендовать. Ее территория
вошла в состав рейхскомиссариата «Остланд» (в переводе с немецкого — «восточная
земля») с административным центром в Риге.

Национальные флаги были запрещены, национальные языки — тоже. Радиовещание и
театры — только на немецком. Разумеется, прибалты не должны были иметь и своей
интеллигенции — музеи, университеты, библиотеки либо закрывались, либо
разорялись. Нацисты строго запретили в Прибалтике использовать слова «свобода»,
«независимость» и «государственность» даже в агитационных целях, когда они
набирали добровольцев в Латышский и Эстонский легионы СС.

Кстати, они появились только в феврале 1943 года, после поражения вермахта под
Сталинградом, а легионеры присягали на верность лично Гитлеру. До этого немцы
разрешали создавать в Прибалтике только парамилитарные вооруженные
формирования, выполнявшие вспомогательные и карательные функции в тылу, в
прифронтовой полосе и на фронте, в том числе на территории РСФСР и Белоруссии.

\lenta{Какую политику Третий рейх проводил в оккупированной Прибалтике? Там была
тотальная германизация?}

Да, и немцы это очень четко формулировали, совершенно не скрывая свои
намерения. Так, в «Меморандуме» Розенберга о вехах активности в отношении
Латвии, Литвы, Эстонии, территории которых воспринимались как будущая
территория немецкого расселения, говорилось следующее:

\begin{leftbar}
				\em
				\enquote{…Обеспечить отток значительных слоев интеллигенции, особенно латышской, в
центральные русские области (позже речь стала идти и о Сибири — Ю.К.),
затем приступить к заселению Прибалтики крупными массами немецких
крестьян… чтобы через одно или два поколения присоединить эту страну,
уже полностью онемеченную, к коренным землям Германии. В этом случае,
видимо, нельзя обойтись и без перемещения значительных по численности
				расово-неполноценных групп населения Литвы за пределы Прибалтики}
\end{leftbar}

Или вот из специально подготовленной Гиммлером для Гитлера записки про
обращение с местным населением Прибалтики: дробить на этнические группы,
искусственно разжигать национальную рознь. Для прибалтов не должно существовать
высших школ, «вполне достаточно 4-классной народной школы». Главное, чему учить
там детей — простой счет, самое большее до 500, умение расписаться, а также
знать божественную заповедь о том, чтобы «повиноваться немцам, быть честными,
старательными и послушными».

\ifcmt
tab_begin cols=3,env=longtable
	caption Книга «Прибалтика. 1939-1945 гг. Война и память», Центральный архив ФСБ России

	pic https://icdn.lenta.ru/images/2020/10/20/16/20201020160847925/pic_b22dedc919254dd19925fc9181c3de23.jpg
	caption Распоряжение генерального комиссара г. Риги Дрехслера и коменданта латвийской окружной охраны генерал-майора Вольфсбергера о запрете помощи советским военнопленным (на немецком и латышском языках). 1 марта 1943 г.

	pic https://icdn.lenta.ru/images/2020/10/20/16/20201020161105859/pic_9976de9719d669e3378fcb32bc64f651.jpg
	caption Схема газовой камеры. Из заявления П. Гурвича в Чрезвычайную республиканскую комиссию по расследованию злодеяний немецко-фашистских захватчиков о деятельности немецких институтов в Риге. 6 декабря 1944 г.

	pic https://icdn.lenta.ru/images/2020/10/20/16/20201020161109241/pic_98f701c9a9ab3104ab97682c3cee4e1a.jpg
	caption Схема кормушки для вшей на руке человека. Из заявления П. Гурвича в Чрезвычайную республиканскую комиссию по расследованию злодеяний немецко-фашистских захватчиков о деятельности немецких институтов в Риге. 6 декабря 1944 г.
\fi

Основная же масса населения восточных территорий, которая будет признана
«расово-неполноценной», должна, по замыслу Гиммлера, представлять собой
руководимую арийцами рабочую силу и поставлять Германии рабочих для
использования на черных работах.

\begin{leftbar}
	\large
	Часть — причем, меньшая, — представителей балтийских народов подлежала
				онемечеванию. Большинство — ассимиляции или уничтожению
\end{leftbar}

Все это прописано в документах. Наиболее подходящими для онемечивания Розенберг
считал эстонцев, в гораздо меньшей степени — латышей и литовцев. Некоторые
послабления были сделаны, когда потребовалась мобилизация: их признали более
«ариизированными», чтобы бросить на передовую.

Можно выделить несколько главных принципов оккупационной политики нацистов в
Прибалтике. Это тотальное онемечивание, насаждение национальной вражды среди
местных народов (особенно титульных наций к русским и евреям), превращение
региона в аграрно-промышленный придаток Третьего рейха с выкачиванием из него
людских и материальных ресурсов и, разумеется, «окончательное решение
еврейского вопроса» руками местных коллаборационистов.

