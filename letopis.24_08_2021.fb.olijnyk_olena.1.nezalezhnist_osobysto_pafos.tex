% vim: keymap=russian-jcukenwin
%%beginhead 
 
%%file 24_08_2021.fb.olijnyk_olena.1.nezalezhnist_osobysto_pafos
%%parent 24_08_2021
 
%%url https://www.facebook.com/ol.olijnyk/posts/4298597556885924
 
%%author_id olijnyk_olena
%%date 
 
%%tags nezalezhnist,pafos,ukraina
%%title Отже. Що особисто я отримала від Незалежності. (утилітарненько, так. Але буде патос))
 
%%endhead 
 
\subsection{Отже. Що особисто я отримала від Незалежності. (утилітарненько, так. Але буде патос))}
\label{sec:24_08_2021.fb.olijnyk_olena.1.nezalezhnist_osobysto_pafos}
 
\Purl{https://www.facebook.com/ol.olijnyk/posts/4298597556885924}
\ifcmt
 author_begin
   author_id olijnyk_olena
 author_end
\fi

\#ДеньНезалежностіУкраїни

Отже. Що особисто я отримала від Незалежності.
(утилітарненько, так. Але буде патос))

Дивним чином, але розфікціонізувалося для мене поняття Батьківщина. Причому
якось легко і одразу, то ж мабуть я її потребувала, хоч і не дуже здогадувалася
про це.

\ifcmt
  pic https://scontent-frt3-2.xx.fbcdn.net/v/t1.6435-9/238749898_4298591310219882_4610637796951386983_n.jpg?_nc_cat=101&_nc_rgb565=1&ccb=1-5&_nc_sid=8bfeb9&_nc_ohc=z7AH5TNYCUkAX-A7Sp9&_nc_ht=scontent-frt3-2.xx&oh=201ba0aab25e349818a16c3f0c0a2bd3&oe=615F2C38
  @width 0.4
  %@wrap \parpic[r]
\fi

Савєйський Саюз, де я народилася, виросла й добряче встигла пожити, такого
запиту задовольнити не міг за означенням. Не те, щоб я тоді свідомо сприймала
його, як окупацію, хоча окремі відчуття були. А просто він для мене був "страна
Фуфляндія". Невідповідність на кожному кроці декларованого реальності мені
впадала у вічі, скільки себе пам'ятаю. Ну, таку вже у'їдливу і прискіпливу
вдачу мала, що поробиш. І з тієї от "рОдіни", що про неї у вуха дзижчали у
школі й скрізь, невідповідність лізла чи не в першу чергу. Фікція.

\ifcmt
  ig https://scontent-frt3-1.xx.fbcdn.net/v/t1.6435-9/236045033_4298702973542049_123636212442140992_n.jpg?_nc_cat=102&ccb=1-5&_nc_sid=730e14&_nc_ohc=Y8ZFXiEaxtgAX8g-f20&_nc_ht=scontent-frt3-1.xx&oh=74f24ca1c40804ca2d4fa873cc8ef86a&oe=6162F2A1
  @width 0.4
  %@wrap \parpic[r]
  @wrap \InsertBoxR{2}
\fi

А Україна для мене так, була й тоді... але що ж то воно було, коли ніби є, а
ніби й нема. Щось красиве, але химерне. А Батьківщина химерою не має бути.

Втім, тоді я ще довгий час цього всього не усвідомлювала по-справжньому. Не
складалося з нагодами. Та ні, загалом їх, вірогідних нагод, було купа, і кожна
могла спрацювати, але не спрацьовувала. Аж до 2004 року. Про це писати довго,
хоч я вже й писала деінде, але не важко здогадатися, що йдеться про важкий
тупий предмет на ймення "Янукович", що в нього влучило яйце, а він в свою чергу
влучив у мене, падаючи 😩

(це про роль особистості в історії, да, знов))

От тоді нарешті прийшло живе осяяння, що в мене весь цей час була моя, от саме
моя Патріа, яка вже тут і зараз може знову залишитися в минулому. І з цим треба
негайно щось робити, бо капець вже ось він.

Із цим відчуттям я живу відтоді майже безперервно. Майдан і гаряча фаза війни з
ерефією його тільки посилили, а вибори 2019 загалом лише додали бридкого
присмаку в роті. Ну, це звісно оглядаючись на ті події із сьогодні, 24 серпня
2021 року, що з цією датою я нас всіх і вітаю, адже ми принаймні ще живі,
трясця...

А от за останнє дякую тим, кого зокрема вже нема серед нас, живих 🙏 😣.     R.I.P.

І тим, хто живий. \#ЗавдякиТобіНезалежність

Цінуймо, хороші люди, що маємо, поки маємо його. Бо жити можна і в іншій
країні, поки є своя, куди можна повернутися. А якщо вже нема куди, то ти ніхто
і звати ніяк. І куди б не занесло, скрізь ти це відчуєш.

Отакий настрій станом на тепер 😶😬

Менше з тим, Слава Україні!
