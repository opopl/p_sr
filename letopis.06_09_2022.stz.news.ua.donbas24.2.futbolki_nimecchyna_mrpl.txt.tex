% vim: keymap=russian-jcukenwin
%%beginhead 
 
%%file 06_09_2022.stz.news.ua.donbas24.2.futbolki_nimecchyna_mrpl.txt
%%parent 06_09_2022.stz.news.ua.donbas24.2.futbolki_nimecchyna_mrpl
 
%%url 
 
%%author_id 
%%date 
 
%%tags 
%%title 
 
%%endhead 

Ольга Демідко (Маріуполь)
Маріуполь,Україна,Мариуполь,Украина,Mariupol,Ukraine,date.06_09_2022
https://donbas24.news/news/yak-u-nimeccini-stvoryuyutsya-futbolki-prisvyaceni-mariupolyu-detali
06_09_2022.olga_demidko.donbas24.futbolki_nimecchyna_mrpl

Як у Німеччині створюються футболки, присвячені Маріуполю — деталі

Відтепер можна допомогти вразливим верствам України, придбавши унікальні
футболки з принтами «Маріуполь» та «Азовсталь»

5 вересня на своїй сторінці у Facebook маріупольська художниця Олена
Украінцева, яка наразі перебуває у Німеччині, повідомила, що долучилася до
нового цікавого проєкту.

«Мені знов пощастило. Минулого тижня долучилася до роботи зі створення малюнка
про Маріуполь, який друкуватимуть на футболках. Разом з Anna Hilmenna з ідеї та
ескіза вибудували ось таку красу, такий спогад про наше квітуче місто та його
завод, що був біллю у мирні часи, та став останнім форпостом, надією, місцем
сили на початку повномасштабної війни. Працювала над зображенням з любов'ю до
міста, з великою шаною до його захисників», — зазначила Олена.

Олена Украінцева та художниця Hilmenna створили ескіз. Запросив маріупольську
художницю приєднатися до створення малюнків для унікальних футболок харків'янин
В'ячеслав Овсєєнко, який до війни займався навчанням людей з порушенням зору.
Наразі він разом з трьома дітьми перебуває в Німеччині та є координатором зі
створення серії футболок з популярним логотипом «Харків — Залізобетон». Чоловік
вирішив, що футболки про Маріуполь та його захисників стануть не менш
популярними.

«Сподіваюся, що ця війна посприяє змінам в нашій країні на краще. А зараз, поки
вона триває, необхідно робити все можливе, щоб допомагати тим, хто цього
потребує», — наголосив В'ячеслав.

Футболку можна придбати за 25−30 євро чи за власним бажанням заплатити більше.
Отримані кошти підуть на допомогу евакуації маріупольців, харків'ян та
підтримають вразливі верстви населення. Також найближчим часом планується
придбання автомобілів для виконання завдань евакуації та забезпечення для
АТО/ТРО. Зараз створюється сайт, на якому можна буде переглянути всі футболки
та обрати ту, що сподобається найбільше.

«Ми розглядаємо варіант друку в Україні, якщо знайдемо, як їх перевозити до
Європи. Підтримка вітчизняного виробника також важлива», — додала Олена
Украінцева.

Незабаром на футболках з'являться принти і з Херсоном та Миколаєвом. З
маріупольським принтом вже встигли продати декілька футболок. Замовити футболку
можна після заповнення гугл-форми. 

Нагадаємо, що екстрених лікарів перепрофілювали на допомогу при хімічних та ядерних катастрофах.

Ще більше новин та найактуальніша інформація про Донецьку та Луганську області в нашому телеграм-каналі Донбас24

ФОТО: Олени Українцевої.
