% vim: keymap=russian-jcukenwin
%%beginhead 
 
%%file 24_12_2021.stz.news.ua.radiosvoboda.1.podrostki_tiktok_psihologia
%%parent 24_12_2021
 
%%url https://www.radiosvoboda.org/a/pidlitky-sotsial'ni-merezhi-problemy/31622649.html
 
%%author_id 
%%date 
 
%%tags 
%%title Телеграм, Дискорд, ТікТок: соціальні мережі і дружба, що кажуть підлітки? 
 
%%endhead 
\subsection{Телеграм, Дискорд, ТікТок: соціальні мережі і дружба, що кажуть підлітки?}
\label{sec:24_12_2021.stz.news.ua.radiosvoboda.1.podrostki_tiktok_psihologia}

\Purl{https://www.radiosvoboda.org/a/pidlitky-sotsial'ni-merezhi-problemy/31622649.html}

(Рубрика «Точка зору»)

Тетяна Кісєльова

Нас періодично накривають полемічні навали. Час написання цієї статті якраз
припав на пік дебатів про вакцинацію, проте цей шквал уже починає вщухати.
Суджу по тому, як один мій знайомий, який відмовлявся вакцинуватися, писав у
соцмережах про попрання його особистої свободи, але врешті-решт зробив щеплення
проти COVID-19. А тепер він пише таке: «Вас ще турбує щеплення? Подивіться, як
ростуть ціни!».

Ми з головою поринаємо у хвилі ескалованого протистояння, втрачаючи з поля зору
те, про що ніколи не розкаже маніпулятивна пропаганда. У мене, як і в багатьох
батьків, зараз з’явилися нові хвилювання, пов’язані з локдауном та дистанційним
навчанням.

\ii{24_12_2021.stz.news.ua.radiosvoboda.1.podrostki_tiktok_psihologia.pic.1}

\subsubsection{Діти у гаджетах, а батьки тривожаться}

У період дистанційного навчання діти практично весь час проводили перед
комп’ютером або смартфоном, і ця звичка збереглася навіть після виходу з нього.
У холодну пору року спілкувалися переважно через онлайн. Здається, що підлітки
втрачають навичку дружити.

Адже, коли діти перебувають у колективі, вони разом граються, вчаться довірі та
співпраці, складають уявлення про спільність, у них виникає почуття симпатії
один до одного.

Для підлітків спілкування з друзями особливо важливе. Брак його може впливати
на психологічний розвиток та перспективу здобуття професії.

Зараз роботодавці, крім освіти та кваліфікації, цінують ще й соціальні
компетенції. Це вміння налагоджувати стосунки і мати розвинутий емоційний
інтелект, тобто розуміти почуття інших людей.

Від того, як діти зараз дружать між собою, залежатиме і те, наскільки
згуртованим буде наше майбутнє суспільство.

Але раптом я подумала: у скаргах дорослих на те, що діти весь час у гаджетах,
немає нічого нового. Мені стало цікаво, чи згодні самі діти з тим, що це –
проблема?

\subsubsection{Послухаємо підлітка}

Вирішила спитати про дружбу та виклики пандемії COVID-19 у п’ятнадцятилітнього
сина. У підлітках бачу гостре відчуття справедливості, альтруїзм, бурхливу
енергію та бажання зробити світ кращим. Вони вже здатні на дорослі судження,
проте вільні від стереотипів. Іноді питаю у сина поради чи думки, відкрито
обговорюю те, що хвилює, як ось і зараз.

Слухати подкаст: Розмова з підлітком

Син охоче ділиться своїми переживаннями. Відчувається, що для нього це важлива
тема. Розказує, що й раніше багато спілкувався з друзями через Телеграм і
Діскорд. Тому з настанням локдауну майже нічого не змінилося. В нього є
віртуальні друзі для переписки, навіть в інших країнах. Це, скоріше, знайомі.

У сина бували випадки, коли друзі його зраджували. Каже, що спочатку було
сумно, а потім через деякий час стало все одно. Він перестав з ними
спілкуватися, просто забув, от і все. Можливо, і сам зраджував. Але робив це не
навмисно, так вийшло.

Найбільше цінує у людях щирість. Це те, що сприяє дружбі. Також каже, що знайти
справжнього друга не легко. Проте, має давнього друга, з яким вони
потоваришували в школі дев’ять років тому.

На думку сина, дружбі між людьми сприяють спільні інтереси та схожі характери,
тоді легко дійти згоди. Однак, така схожість може бути не на користь. Він та
його друг обидва вперті, й іноді вперто стоять на своєму. Тоді беруть перерву і
не бачаться.

\begin{zznagolos}
«З другом спілкуюсь дуже часто і дуже агресивно. Я можу йому сказати все, що
думаю про нього, і так само він мені. Але не вважаю, що це погано. Зараз усі
друзі так роблять – проявляють грубість один до одного. Нам все рівно, що ми
кажемо одне одному, ми не ображаємося. Іноді обзиваємо один одного заради
жарту».
\end{zznagolos}

Завдяки цій розмові з сином дізналася, що на жаль, у їх взаєминах присутня
агресія та зневага, нехай навіть хлопці і не ображаються один на одного. Таке
зізнання від сина засмутило мене. Їхнє спілкування проходить у мене на очах.
Майже впевнена, що така грубість з’явилась не так давно. Можливо, не тільки
через спілкування в Інтернеті, але й через дорослішання.

\subsubsection{Батьки та вихователі за баланс між реальним та віртуальним}

Вирішила також поговорити з кількома своїми знайомими батьками та вихователями.
Це дозволило глибше зрозуміти проблему та побачити її під іншим кутом зору.

Не тільки підлітки зараз потерпають від браку живого спілкування, але й
дошкільнята також. Особливо на прикладі маленьких дітей добре видно, що ті мало
бувають у колективі разом з іншими малятами. Мають мало навичок вирішення
конфліктів, наприклад, коли хтось забирає іграшку.

Хоча зараз набагато більше можливостей і каналів спілкування, ніж колись, на
обсяг спілкування це не впливає. Здається, що завдяки Інтернету, друзів у
віртуальному світі повинно бути багато. Але віртуальні дружні стосунки
насправді не глибокі, зазвичай побудовані навколо гри, відеоблогера чи короткої
віртуальної любові.

Підлітки легко пристосовуються до нової реальності обмеженого спілкування.
Зручно і комфортно, коли тобі не потрібно людині дивитися в очі, коли неможливо
розпізнати правду чи фальш через міміку та інтонацію. Достатньо тільки
правильно підібрати емодзі.

І якщо потім підлітки зустрічаються в реальному світі, то і тоді продовжують
говорити про віртуальні речі. Віртуальний світ для них стає реальним, а
реальний – тривожним. Розучуються дивитися в очі, підкреслювати хороші риси
інших, спілкуватися по телефону і бути фізично витривалими.

\subsubsection{«Діти Covid-19» пізнають цінність дружби}

Такі спостереження за стосунками між дітьми співпадають з результатами
дослідницького проєкту “Діти Covid-19”, під час якого вивчався вплив
коронавірусу на життя дітей 9-16 років. Серед досліджуваних сфер були також їх
взаємини з друзями та однокласниками/цями.

У підсумковому звіті цього проєкту зазначається, що під час пандемії діти
значно більше почали користуватися смартфонами, комп’ютерами та планшетами, які
також є засобом спілкування. Проте, їм для спілкування з друзями потрібно
фізично бути біля них, в той час як для знайомих чи однокласників достатньо
чатів.

Найважливішим уроком карантину для старших підлітків стала переоцінка дружби.
Завдяки пандемії вони змогли дізнатися, хто для них є справжніми друзями. Це
ті, хто зміг зробити над собою зусилля і почати спілкуватися онлайн. Несправжні
друзі відпали.

\begin{zznagolos}
Виявляється, «...для дітей 9-16 років дружба має вищу цінність за звичайне
спілкування з однолітками, й саме перша потребує контакту наживо. Отже, брак
живого якісного спілкування – це те, що викликало тривогу та фрустрацію дітей».	
\end{zznagolos}

\subsubsection{Як в умовах «дистанційки» навчати дружити?}

Добре, що з'являються подібні дослідження. Суспільство, а також ми, дорослі –
батьки та вихователі – не можемо залишити дітей один-на-один з їх
переживаннями. Не слід бути осторонь, спостерігаючи, як діти тихо сидять удома.
Можемо допомогти їм стати більш соціальними.

Як? Граючи з ними настільні ігри, ведучи душевні розмови за чашкою чаю,
віддаючи їх на колективні заняття спорту, музики, танців, образотворчого
мистецтва тощо. Це допоможе їм зберегти баланс між реальним світом і
віртуальним та пережити локдаун та дистанційне навчання.

Хоча ми самі вже більше проводимо часу удома, варто дозволяти дітям запрошувати
їх друзів до себе. Створювати для цього умови, облаштовувати кімнату чи
куточок, де вони можуть поговорити, погратися чи навіть подивитися мультики.

Також ми, батьки, можемо об’єднувати наші зусилля. У молодшої доньки був
період, коли вона та дві інші її подруги не могли “поділити” одна одну. Бувало
так, що дві з них хотіли спілкуватися тільки між собою і не бажали приймати
третю. Тоді ми з мамами дівчаток одразу приходили на допомогу.

Ми активно розбирали цю ситуацію між собою та з доньками, пояснювали, як це
неприємно, коли тебе не беруть у гру, і що колись це може статися з ними також.
Розказували, що треба остерігатися образити іншого, вчили турботі про ближнього
і навіть розігрували сценку. І зараз продовжуємо спілкуватися, разом гуляємо та
ходимо в кіно, а потім обговорюємо сюжет.

Діти хочуть повернутися до школи і в соціум, хоча самі вони можуть
стверджувати, що їм вистачає спілкування. Вони більше люблять спілкуватися
вживу, ніж у переписці чи по телефону, і це дає надію на те, що з часом
ситуація виправиться.

Діти завжди будуть винаходити нові способи, щоб реалізувати свою потребу у
компанії. Навіть під час глибокого локдауну вони в соціальних мережах активно
контактували з друзями, що на той час певною мірою замінювало їм живе
спілкування.

\subsubsection{Як суспільство може зарадити?}

Суспільство також має допомогти. Під час експертної дискусії \enquote{Згуртованість
суспільства через освіту та просвіту} обговорювалось питання про обов’язковий
компонент освіти, який би формував довіру між людьми та бажання до
співробітництва. Цій компетентності навчають учителів, щоб ті зуміли розвинути
таку систему відносин у дитячому колективі.

В умовах нової реальності дистанційної освіти дітям поки що не викладають
якогось особливого предмету, як математика чи історія, який би розвивав у них
навики дружби, а значить будував майбутнє згуртоване суспільство.

На мою думку, потрібно змалку навчати дітей таким якостям та етичним нормам, як
наприклад, доброті, правдивості, щирості та іншим чеснотам. Зрештою,
розказувати, що між дружбою та агресією немає нічого спільного. У школі має
бути предмет, під час якого діти би вивчали ці якості людської особистості. Так
само, як наприклад, на фізиці вони вивчають якості металів та інших речовин.
Маємо про це подумати гуртом.

(Стаття підготовлена в рамках конкурсу СторіМейкер-2021 «COVID-19: випробування
людяності в епоху пандемії»).

Тетяна Кісельова – координаторка відділу зовнішніх зв'язків Громади бахаї
України
