% vim: keymap=russian-jcukenwin
%%beginhead 
 
%%file 09_05_2022.fb.rudyk_daria.kiev.1.vchora_djadju_vbylo
%%parent 09_05_2022
 
%%url https://www.facebook.com/daria.rudyk/posts/7395598523844141
 
%%author_id rudyk_daria.kiev
%%date 
 
%%tags 
%%title Вчора Дядю вбили....
 
%%endhead 
 
\subsection{Вчора Дядю вбили....}
\label{sec:09_05_2022.fb.rudyk_daria.kiev.1.vchora_djadju_vbylo}
 
\Purl{https://www.facebook.com/daria.rudyk/posts/7395598523844141}
\ifcmt
 author_begin
   author_id rudyk_daria.kiev
 author_end
\fi

— дітьонка, ти їла? Ану ходи сюди! Тут мама Тамара передала твоє блюдо. Моїм
блюдом була смажена бараболя з солоним огірком. В банці, замотана рушником.

І я швидко знімала бронік і наминала поки нікуди не позвали терміново їхати на
300го.

Він  ніколи не сварився ні з ким, не підвищував голосу, ніколи не жалівся,
ніколи не нарікав і щиро переживав що діти воюють, а так би не мало бути.

\ii{09_05_2022.fb.rudyk_daria.kiev.1.vchora_djadju_vbylo.pic.1}

Зимою 15го в Старобільську було дуже сніжно і холодно.

Ми приїхали туди помитися, до мами Тамари. Вона знову насмажила картоплі для
\enquote{Даринки}, зварила каву, поставила цукерки...

І я сиділа з мокрою головою у рушникові ніби в чалмі, а вона сміялася і все
питала чи мені не холодно....

\ii{09_05_2022.fb.rudyk_daria.kiev.1.vchora_djadju_vbylo.pic.2}

В хаті запалили грубку, Дядя сів за стіл біля неї і вони розказали свою
неймовірну історію любові. 

Я тоді ними дуже вразилася і написала її у фб. І підкріпила фотографією....

Згодом, перезнайомившись з майже всією родиною і з деякими родичами, я без
вагань звонила по потребі, заїздила в гості, завжди була рада десь перетнутися
з ними...

Бо це справді чудові люди... Дуже добрі.

Від них світило якимось таким особливим світлом, що я ніяк не могла зрозуміти
як це можливо, зберегти всередині стільки радості маючи таке непросте життя....

Вчора Дядю вбили....

Мама Тамара, його дружина, померла близько місяця тому, виїхавши з
Старобільська з дочками і внуком. Онкологія.

А вчора російські свині вбили Дядю... Коли він допомогав евакуювати
поранених...

І від вчора я не знаю як толком зібратися

Бо неможливо.

Яке тут примирення? Яке never again? Яка тут найвища цінність? 

—  людське життя? українці щодня протягом 9 років віддають свої, бо є мразота
яка шантажує увесь світ своїм газом і ядеркою... І воно ніби не можна
захоплювати, вбивати і руйнувати, але коли і тебе є ядерка, то можна. 

Я не знаю.

Має бути пам'ять про те, як ніколи не забути, ніколи не йти на поступки з тими,
хто ницо і підло бреше на весь світ, скидає бомби на дитячі евакуаційні
автобуси, пологові будинки, руйнує пам'ятки архітектури, нищить все, заради
хворобливого годуванняня своїх амбіційних комплексів. І тішитися тим,  шо у
нього, начебто, є сила.

Ви всі будете відповідати.

І так, в цьому світі існує помста, якої не уникнути навіть у бункері. І вона—
найстрашніша.

Бувай, мама Тамара і Дядя....

Ви зараз разом... Певно, Дядя приніс тобі квіти, як тоді... І ви зараз
цілуєтесь....

%\ii{09_05_2022.fb.rudyk_daria.kiev.1.vchora_djadju_vbylo.eng}
