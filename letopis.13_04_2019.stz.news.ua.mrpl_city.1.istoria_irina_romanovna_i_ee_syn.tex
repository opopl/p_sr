% vim: keymap=russian-jcukenwin
%%beginhead 
 
%%file 13_04_2019.stz.news.ua.mrpl_city.1.istoria_irina_romanovna_i_ee_syn
%%parent 13_04_2019
 
%%url https://mrpl.city/blogs/view/istoriya-irina-romanovna-i-ee-syn
 
%%author_id burov_sergij.mariupol,news.ua.mrpl_city
%%date 
 
%%tags 
%%title История: Ирина Романовна и ее сын
 
%%endhead 
 
\subsection{История: Ирина Романовна и ее сын}
\label{sec:13_04_2019.stz.news.ua.mrpl_city.1.istoria_irina_romanovna_i_ee_syn}
 
\Purl{https://mrpl.city/blogs/view/istoriya-irina-romanovna-i-ee-syn}
\ifcmt
 author_begin
   author_id burov_sergij.mariupol,news.ua.mrpl_city
 author_end
\fi

\ii{13_04_2019.stz.news.ua.mrpl_city.1.istoria_irina_romanovna_i_ee_syn.pic.1}

По понедельникам, ближе к вечеру приходила к нам в гости \textbf{Ирина Романовна
Бирюкова} – Оришка, как на украинский лад называла ее бабушка. Она заметно
хромала и, вероятно, при ходьбе без палочки обходиться не могла. Была роста
небольшого, щупловатая, черты лица, несмотря на небольшую впалость щек, были
приятны. Гостья казалась очень пожилой женщиной. Это подчеркивалось и сединой
на висках, и темным одеянием, и белым, туго накрахмаленным платком, повязанным
под подбородком. Но сейчас, на расстоянии прожитых лет, понимаешь, что было ей
не более пятидесяти. Ирина Романовна была уроженкой одного из сел Курской
губернии, однако, попав еще девочкой в среду украинцев, поселившихся в
Мариуполе, восприняла их мову, которая стала для нее родной. Хлеб насущный она
зарабатывала шитьем телогреек и последующей продажей их на рынке. Для молодых
читателей приведем синонимы этой нехитрой одежонки военных и послевоенных лет:
стеганка, душегрейка, ватник. К слову, ее муж, \textbf{Николай Прокофьевич}, также был
портным.

\textbf{Читайте также:} 

\href{https://archive.org/details/20_10_2018.sergij_burov.mrpl_city.istoria_portnihi_mariupolja}{%
История: портнихи Мариуполя, Сергей Буров, mrpl.city, 20.10.2018}

У нее были печальные глаза, и что-то не помнится, чтобы гостья когда-нибудь
смеялась. Криво улыбнуться, это – да. Такое случалось, когда Ирина Романовна
намекала об очередном \enquote{амурном} похождении ее незадачливого супруга. Как-то
бабушка обронила незлобно, мол, Оришка приходит к нам, чтобы выговориться. И ее
плавную неторопливую речь действительно никогда не прерывали. Разговор обычно
начинался с \enquote{обзора} товаров и цен на базаре. Она сетовала, например, что
севрюжка подорожала, да и совсем мало ее, а сулы много, но уж слишком большая
для двоих. Рассказывала, что было много любимого мариупольцами винограда под
местным названием \enquote{Березка}, как торговалась с хозяйкой селянской курицы, но
так и не удалось ей сбить цену, пришлось покупать без скидки. Сообщала, кого
встретила из знакомых по пути к нам. Подробным, как правило, было ее
повествование о поездке в село Самсоново, куда она каждый год ездила, чтобы
подлечить азовскими грязями больную ногу.

Но подростка более интересовали рассказы Ирины Романовны о ее единственном сыне
\textbf{Косте}. Каким он был славным ребенком. Как, будучи второклассником, научил ее
читать, ведь ей не пришлось ходить в школу. Она гордилась его успехами в школе
и с удовольствием ходила на родительские собрания, где ее Костю хвалили за
примерное поведение и прилежную учебу, а однажды сам директор поведал, что
сочинение по русской литературе направлено в область на конкурс. И это
достижение возникло не на пустом месте. Костя с раннего детства был завзятым
читателем. Был завсегдатаем книжных развалов на базаре. Купив какую-нибудь
книгу (а это были главным образом дореволюционные издания с \enquote{ятями} и \enquote{ерами}
после конечных согласных), он сразу приступал к чтению. Постепенно в доме
собралась библиотека, где с творениями Пушкина, Гоголя, Лермонтова, Тургенева,
Достоевского, Льва Толстого соседствовали произведения Мережковского,
Мордовцева и даже забытого сейчас всеми Боборыкина. И, конечно же, романы
Александра Дюма, Фенимора Купера, Майн Рида, Луи Буссенара, Луи Жакальо. Ирина
Романовна вновь и вновь перечитывала некоторые из этих книг, в основном романы
Мордовцева. Иногда, придя в гости, пересказывала недавно прочитанное. Но такое
бывало не так уж и часто.

\textbf{Читайте также:} 

\href{https://mrpl.city/news/view/mariupolets-v-komande-s-mrplcity-izdal-knigu-ob-osushhestvlenii-mechty-foto}{%
Мариуполец в команде с MRPL.CITY издал книгу об осуществлении мечты, Ганна Хіжнікова, mrpl.city, 11.04.2019}

После нескольких обращений Ирина Романовна позволила читать книги ее сына при
обязательном условии самого бережного обращения с ними. Это было счастье вместе
с авторами романов пережить приключения в Океании или стране львов,
познакомиться с канадскими охотниками и капитаном Сорви-голова, сражаться с
индейцами, совершать путешествия в Индию, Берег Слоновой Кости, Австралию,
Гвиану, Бразилию. И вместе с тем получать знания о природе, народах и их
обычаях разных стран почти всех континентов земли. Повзрослев, довелось
прочесть \enquote{Трех мушкетеров}, \enquote{Графа Монте-Кристо}, \enquote{Королеву Марго}. А позже –
исторические романы Даниила Мордовцева \enquote{Державный плотник}, \enquote{Великий раскол},
\enquote{Лжедмитрий}, \enquote{Тень Ирода}, \enquote{Иосиф в стране фараона}. Книги Кости были
интересны не только своим содержанием. Большую часть томов он тщательно
отреставрировал, заново переплел, утраченные страницы бисерными буковками
переписал из взятых у друзей неповрежденных экземпляров и переписанные страницы
вклеил в нужное место, вытравил чернильные пятна. Видна была трепетная любовь
Кости к книге...

* * *

Штабной писарь выстукивал на пишущей машинке: 

\begin{quote}
\enquote{Именной список безвозвратных
потерь начальствующего и рядового состава 509-го Истребительного
противотанкового артиллерийского полка с 25 по 30 апреля.

Бирюков Константин Николаевич, лейтенант, и.о. командира батареи, кандидат в
члены ВКП(б), место и год рождения – г. Мариуполь, 1923 г., призван
Мариупольским РВК – 1941 г. убит от артобстрела 26 апреля 1943 г. Место, где
похоронен - г. Ленинград, северо-западная окраина Авиагородка, могила №2. Отец:
Бирюков Николай Прокофьевич, г. Мариуполь, ул. Котовского, д. 8. Извещение не
выслано ввиду оккупации территории противником. Других потерь за данный период
не было. Командир полка майор Пудов, начальник штаба майор Анисов. 30 апреля
1943 г.}.
\end{quote}

\textbf{Читайте также:} 

\href{https://archive.org/details/22_06_2018.sergij_burov.mrpl_city.mariupol_22_iunja_1941_goda}{%
Мариуполь: 22 июня 1941 года, Сергей Буров, mrpl.city, 22.06.2018}
