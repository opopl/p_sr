% vim: keymap=russian-jcukenwin
%%beginhead 
 
%%file 11_02_2017.stz.news.ua.mrpl_city.1.u_ozera_domaha
%%parent 11_02_2017
 
%%url https://mrpl.city/blogs/view/u-ozera-domaha
 
%%author_id burov_sergij.mariupol,news.ua.mrpl_city
%%date 
 
%%tags 
%%title У озера Домаха
 
%%endhead 
 
\subsection{У озера Домаха}
\label{sec:11_02_2017.stz.news.ua.mrpl_city.1.u_ozera_domaha}
 
\Purl{https://mrpl.city/blogs/view/u-ozera-domaha}
\ifcmt
 author_begin
   author_id burov_sergij.mariupol,news.ua.mrpl_city
 author_end
\fi

Во времена, отдаленные от нас более чем столетием, крещенскую купель в
Мариуполе устраивали, как  ни странно,  не на море, а на озере Домаха, теперь
не существующем. Это озеро было результатом заиливания одного из рукавов устья
Кальмиуса.  В 20-х годах прошлого века озеро вовсе засыпали. Так вот,
дальновидные аборигены нашего  города  смекали, что неустойчивый приморский
ветер может отогнать льдину с  крестообразной прорубью  далеко в море в самый
неподходящий момент, и все труды по ее приготовлению обратятся в прах.  Вот
почему они готовили место  для крещенского молебна именно на Домахе. Там всегда
был лед крепок, поскольку  крещенские морозы в ту эпоху были не пустым
словосочетанием.

\ii{11_02_2017.stz.news.ua.mrpl_city.1.u_ozera_domaha.pic.1}

Иордань, прорубь для освящения воды в честь памяти Крещения Господнего,  была
заботой Александр\'ова. Он же с момента образования достаточно крепкого льда
устраивал на Домахе каток. Каток, который вечерами освещался керосиновыми
лампами, где небольшой оркестр играл вальсы  и другую музыку, подходящую  для
скольжения по льду. Здесь можно было за  весьма умеренную плату взять напрокат
ботинки с коньками. Для девиц – \enquote{снегурки} с закрученными бубликом спереди
полозьями, для кавалеров – \enquote{нурмисы}, те же \enquote{снегурки}, только со срезанными
под углом полозьями.

Кто же такой Александр\'ов? Он был известен как цирюльник. Цирюльник, который не
только умел  подстричь шевелюры, усы и бороды  местным франтам, но и соорудить
на головках модниц прически \enquote{по последней парижской или венской моде}. Кто что
пожелает. Но славен был мариупольский куафер не только  искусством  украшения
девиц и дам. Кроме того, он умел вправить вывихнутую ногу или руку,
безболезненно извлечь из глаза  соринку, дать дельный совет старушке,
страждущей от головной боли. Словом, не побоимся возвышенных  слов, он являлся
местным Эскулапом.

Александр\'ов слыл оригиналом. Например, в субботу в четвертом часу дня он вешал
замок на дверь своей парикмахерской, находившейся на Екатерининской улице рядом
с базаром.  Вдохнув свежий воздух, чинно шел к соседнему  погребку, над входом
которого была укреплена черного цвета вывеска. На ней неумелой рукой была
выведена \enquote{золотыми} буквами надпись: \enquote{Греческія вина}. Едва заслышав
спускающегося в погребок цирюльника, половой спешил к нему навстречу со
словами: \enquote{Сколько бутылочек сегодня изволите-с заказать?} Вновь прибывший
посетитель изображал поднятыми пальцами количество желаемых вместилищ
санторийского вина...

Финалом субботы для Александр\'ова был вызов двух извозчиков. Один из них вез
самого цирюльника, а второй - его шляпу. Кавалькада из двух ландо, миновав
Харлампиевский собор и церковь св. Екатерины, сворачивала на Бондарную  улицу,
по ней спускалась до Малофонтанной и по ней уже достигала цели, где нашего
героя ожидали чада и домочадцы. Между тем, существовал более короткий  путь от
погребка  \enquote{Греческія вина} до александро́вского  двора. Нужно было пройти вдоль
базара, спуститься по тщательно обтесанным из известняка ступеням и тут же
оказаться у своей усадьбы. Но не таков  был  Александр\'ов.

Увлекшись грандиозной личностью местного оригинала, оказалось забытым озеро
Домаха.  Итак, на иордань, которая устраивалась на Домахе  трудами героя нашего
повествования, в Крещение Господня народ шел с первыми лучами скудного зимнего
солнца. Шли с хоругвями,  с церковным хором, с иереями и причтом. От давно
ушедших в иной мир довелось слышать, что чаще всего святил воду протоиерей отец
Павел Щербина, просвещенный богослов, добросердечный пастырь, законоучитель
Мариупольской Александровской мужской гимназии, удостоенный ордена  св. Анны II
степени за труды на ниве просвещения. Шел молебен, священник правил службу, хор
пел в нужных местах, молящиеся прикрывали ладошками трепещущие огоньки свечек.

Наконец наступал самый зрелищный момент – омовение в освященной воде. Несколько
мужчин разного возраста раздевались в сторонке до костюма Адама, гуськом шли
словно по тлеющим углям за идущим впереди Александр\'овым к проруби. Он первым
бросался в ледяную воду, трижды погружался в нее с головой, с каждым разом
накладывая на себя крестное знамение. Затем выбирался на лед. Тотчас на его
раскрасневшееся голое тело набрасывали шубу. Находились угодливые доброхоты,
которые хватали комом вещи цирюльника, бросали в пролетку рядом с уже
восседавшим там Александр\'овым. Дома его ждал обильный праздничный стол и
водочка, настоянная на лимонных корочках. Стопку-другую этого \enquote{бальзама} он
позволял себе откушать только по двунадесятым праздникам...

Давно это было. Очень давно.
