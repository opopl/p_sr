% vim: keymap=russian-jcukenwin
%%beginhead 
 
%%file 16_12_2021.stz.news.lnr.lug_info.2.ajkido
%%parent 16_12_2021
 
%%url https://lug-info.com/news/aktivisty-proveli-trenirovku-i-peredali-sladosti-podopechnym-luganskogo-tehnikuma-internata
 
%%author_id 
%%date 
 
%%tags sport,donbass,lnr,lugansk,ajkido
%%title Активисты провели тренировку и передали сладости подопечным Луганского техникума-интерната
 
%%endhead 
\subsection{Активисты провели тренировку и передали сладости подопечным Луганского техникума-интерната}
\label{sec:16_12_2021.stz.news.lnr.lug_info.2.ajkido}

\Purl{https://lug-info.com/news/aktivisty-proveli-trenirovku-i-peredali-sladosti-podopechnym-luganskogo-tehnikuma-internata}

Спортсмены республиканской Федерации айкидо и джиу-джитсу совместно с депутатом
Молодежного парламента ЛНР Юрием Рагулиным посетили Луганский техникум-интернат
и передали подарки его подопечным. Об этом с места события передает
корреспондент ЛИЦ.

\ii{16_12_2021.stz.news.lnr.lug_info.2.ajkido.pic.1}

\enquote{Молодежный парламент совместно с Федерацией айкидо и джиу-джитсу проводит для
вас мероприятие в преддверии Дня святого Николая. Сегодня мы хотим показать
что-то интересное для вас – это и приемы айкидо, и джиу-джитсу. Надеюсь, вам
понравится}, - обратился Рагулин к присутствующим на мероприятии юношам и
девушкам.

\ii{16_12_2021.stz.news.lnr.lug_info.2.ajkido.pic.2}

Участники спортивной федерации провели показательное выступление и открытую
тренировку, в ходе которой все желающие могли отработать со спортсменами
различные приемы самообороны.

\ii{16_12_2021.stz.news.lnr.lug_info.2.ajkido.pic.3}

\enquote{Мы хотели показать ребятам, что такие виды спорта как айкидо и джиу-джитсу
развиваются в Республике, что они популярны как среди юношей, так и среди
девушек. Было приятно видеть, что ребята выходили на татами и пробовали делать
приемы, им было интересно, они задавали вопросы, старались правильно выполнять
техники}, - рассказал вице-президент Федерации айкидо и джиу-джитсу Игорь Рек.

\ii{16_12_2021.stz.news.lnr.lug_info.2.ajkido.pic.4}

Инструктор спортивной федерации Виктория Ященко отметила, что самые юные
спортсмены подготовили программу своего выступления самостоятельно.

\ii{16_12_2021.stz.news.lnr.lug_info.2.ajkido.pic.5}

\enquote{Как только дети услышали о том, что мы едем в интернат, сразу проявили
инициативу и были очень рады такой возможности. Наши ученики всегда принимают
участие в мероприятиях, для них это показатель их уверенности и командного
духа}, - подчеркнула она.

В конце праздника подопечные техникума-интерната получили сладкие подарки, а
сотрудники учреждения поблагодарили активистов за подготовленную программу и
предложили почаще приезжать в гости.

Православные чтят Николая Угодника дважды в год: зимой – 19 декабря и весной –
22 мая. Святой Николай – помощник страждущих, чуткий к бедам людей,
благотворитель и чудотворец.

Согласно легенде, святой Николай приходит к детям в ночь с 18 на 19 декабря и
приносит подарки тем, кто хорошо себя вел весь год. Непослушные малыши утром
находят вместо подарка веточку или палочку. Подарок от Николая обычно оставляют
под подушкой или в башмачках, которые ребенок специально на ночь ставит на
подоконник.

