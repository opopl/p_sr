% vim: keymap=russian-jcukenwin
%%beginhead 
 
%%file 15_09_2021.fb.nikonov_sergej.1.bilchenko_rusmir_neft_vechnost
%%parent 15_09_2021
 
%%url https://www.facebook.com/alexelsevier/posts/1581807385497904
 
%%author_id nikonov_sergej,bilchenko_evgenia
%%date 
 
%%tags bilchenko_evgenia,rossia,rusmir,vechnost'
%%title БЖ. Нефть пахнет вечностью
 
%%endhead 
 
\subsection{БЖ. Нефть пахнет вечностью}
\label{sec:15_09_2021.fb.nikonov_sergej.1.bilchenko_rusmir_neft_vechnost}
 
\Purl{https://www.facebook.com/alexelsevier/posts/1581807385497904}
\ifcmt
 author_begin
   author_id nikonov_sergej,bilchenko_evgenia
 author_end
\fi

К своим репостам Евгении Витальевны я никогда не даю комментариев, не реагирую
на комментарии пользователей, не ставлю отметок про отношение. Этот текст
помечен ей как важный.

Евгения Бильченко

БЖ. Нефть пахнет вечностью

Я поняла одну восхитительную вещь: русский мир - это океан. Там все едино,
милосердно, целостно, великодушно. Там нет никакой личности, потому что
личность - это океан. Там нет никакого океана, потому что океан - личность. Там
младенец Егор, злой ранимый Егор, поет Летова на руках у Богородицы, а
Достоевский качает на ладошке его маленького Христа. Это нельзя убить, как
нельзя убить душу. Русских расставили, рассадили, развели, вручили им в руки
флаги ненависти и заставили кричать друг на друга и убивать друг друга,
делиться на правых и левых, белых и красных, атеистов и набожных, прощающих и
непрощаюших.

\ifcmt
  ig https://scontent-frx5-1.xx.fbcdn.net/v/t1.6435-9/241825317_1581805058831470_8249639821336519839_n.jpg?_nc_cat=111&ccb=1-5&_nc_sid=730e14&_nc_ohc=YtSRpVPUJHIAX_NTcIr&_nc_ht=scontent-frx5-1.xx&oh=0cc5f62f3099ceb54a82bab1254f714c&oe=6165F6B5
  @width 0.4
  %@wrap \parpic[r]
  @wrap \InsertBoxR{0}
\fi

Нет всемирнее и всечеловечнее русского мира, он вбирает в себя все. Он никогда
не умирает, потому что погибает за победу, даже, если победа - за седьмой
горой. Он идёт на гору, зная, что цепкие руки тьмы тянут его вниз, но гора под
ним обрывается, и руки тьмы летят в тартары. В нем нет суицида - только крест.
Нет веревки на шее - только принятие муки и смерти за идею. Никогда идея не
служит ему рабом, он сам себе идея и высота, в нем нет никакой идеи, он сердцем
чует. Он - вечно за други своя, даже, если други - ломают его своими руками.
Раздираемый по окраинам, он отрастает в них же, ибо в гетто он настоящий.

Истерия Настасьи Филипповны - вечное - "Не сдюжу!". Вечное карбышевское:
"Сдюжу". Его людей учат ломать свой дом. А дом - сам человек, растет и ширится,
небо ему крыша, земля - паркет деревянный, дедушкин. Какая свобода в отцовских
руках его. Как не нужны грамоты, ордена и медали ему. Как сладко страдание за
него. Никакого лукинга не знает он, лишь лицо. Никакого стиля не знает он, лишь
лик. Скоморошество его рыжее, схима его черная, нежность его бирюзовая...
Падения и взлеты, рогалики его, рулетки его. Жизнь его - это бытие у смерти.
Смерть его - это воскресение, рыбалка, книга, пруд, Пасха...

Ощущаете ли вы это так, когда булочки в печи, когда капучино в чашке, когда
толстовка на плечах, когда пуля в виске, когда гонят и травят вас за него,
когда былины его тониками ломанымм текут, медом его дактиля по губам и
подбородку, по дождю и солнцу? Жидким Маяковским - магмою - да по мрамору
Державина...

Разве что-то с этим сравнится? Ничего. Спасибо, Господи Наш, Иисус Христос, за
богатство, которым одарил Ты, Сын Человеческий, сердце раненое, предсмертное,
бесконечное!
