% vim: keymap=russian-jcukenwin
%%beginhead 
 
%%file 12_01_2022.stz.news.ua.hvylya.1.globalnyj_strah_infodemia
%%parent 12_01_2022
 
%%url https://hvylya.net/analytics/245217-globalnyy-strah-i-infodemiya
 
%%author_id dacjuk_sergіj,news.ua.hvylya
%%date 
 
%%tags strah,obschestvo,internet,chelovek,psihika,psihologia
%%title Глобальный страх и Инфодемия
 
%%endhead 
\subsection{Глобальный страх и Инфодемия}
\label{sec:12_01_2022.stz.news.ua.hvylya.1.globalnyj_strah_infodemia}

\Purl{https://hvylya.net/analytics/245217-globalnyy-strah-i-infodemiya}
\ifcmt
 author_begin
   author_id dacjuk_sergіj,news.ua.hvylya
 author_end
\fi

\begin{zznagolos}
Глобальный Страх вернулся к человечеству в последние годы во многих видах.
Инфодемия стала основным способом поддержки и усиления глобальных по своей
природе ядерного, коронавирусного и климатического страхов.	
\end{zznagolos}

Глобальный Страх вернулся к человечеству в последние годы во многих видах.
Инфодемия стала основным способом поддержки и усиления глобальных по своей
природе ядерного, коронавирусного и климатического страхов.

\subsubsection{Глобальный страх}

Массовый страх — это страх в человеческих сообществах, приблизительно
оцениваемый по происхождению, непредсказуемый по направлениям распространения,
слабо прогнозируемый по периодам усиления и затухания и контролируемый с
усилиями, в особых, специально создаваемых условиях.

\ii{12_01_2022.stz.news.ua.hvylya.1.globalnyj_strah_infodemia.pic.1}

Массовый страх может быть скомпенсированным, то есть понятным по происхождению,
теоретически отрефлексированным, договорным образом оцениваемым и политически
регулируемым при согласии на такое регулирование; и нескомпенсированным, то
есть анонимным, малорефлексивным, оцениваемым в ситуации недоверия к оценкам и
регулируемым в ситуации сопротивления такому регулированию.

Глобальный страх — это испытываемый глобально, то есть во всех
государствах-странах, во всех нациях, во всех группах и индивидуально,
независимое от пола, расы и классовой принадлежности массовый страх
относительно того или иного очевидного или непосредственно ожидаемого явления,
представляющего непреодолимую угрозу для существования Всечеловечества.

Экономические войны, на которых сосредоточено столько медиа-внимания, вообще не
представляют глобальной опасности. Мировые войны с применением обычного или
химического оружия без применения ядерного оружия существенной глобальной
опасности не представляют, поскольку не ведут к тотальному уничтожению
Всечеловечества. Ядерное и Биологическое оружие, эпидемии, климатические
изменения, технологическая катастрофа и космическая катастрофа могут
представлять глобальную опасность для Всечеловечества.

Существуют также гипотетические глобальные угрозы — космическая война между
внеземными колониями (или даже пришельцами) и цивилизацией планета Земля, а
также войны с иноэкзистенциями: кибер-война с участием роботов или мощного
искусственного интеллекта, виртуальная война на поглощение телесно
закрепленного человеческого сознания, война со структурно-экспансивными
образованиями, например, экспансия «серой слизи».

Скомпенсированный Глобальный страх имеет теории интерпретации, экспертные
оценки и измерения, осмысленную информационную поддержку, политические и
экономические стратегии, договора, проекты, программы, направленные на
изменения, минимизирующие страх за счет такой компенсации, ведущие к его
рутинизации и тем самым к преодолению, а затем и к изживанию и тем самым
полному исчезновению.

Нескомпенсированный Глобальный страх ведет к отказу от мышления и включает
реакционные инструменты работы с будущим. Выход на стратегии, компенсирующие и
преодолевающие страх сущностно, основательно и предельно, в этом случае
невозможен. Поэтому к стратегированию в ситуации нескомпенсированного
Глобального страха способны отдельные индивиды и микрогруппы. Однако даже такое
маргинальное стратегирование не может оказать существенное влияние на
компенсацию Глобального страха.

Глобальный страх компенсируется в мифах, религиях и идеологиях, а затем
преодолевается экзистенциально и изживается транзистенциально. Экзистенциально
скомпенсированный Глобальный страх преодолевается в религиях, оперирующих
представлениями об индивидуальной и групповой смерти, за счет церковной работы
с массовым сознанием внутри представления о спасении.

Также Глобальный страх может быть скомпенсирован за счет идеологизации сознания
и преодолен за счет политически организованной деятельности индивидов, групп,
стран и целого мира.

Транзистенциально Глобальный страх избывается через организацию бесстрашных
ойкумен и пространств коммуникации, где происходят преобразования
начал-оснований, ориентаций-установок, пределов-границ Всечеловечества как
такового.

Мифы, мировые религии и мировые идеологии были сосредоточены в основном на
страхе индивидуальном и групповом. Мировые религии в контексте страха конца
мира и наказания за грехи предлагали спасение как способ компенсации или
преодоления страха. Однако мировые религии, хоть и называются мировыми, имеют
преимущественно локальное распространение.

Поэтому впервые всеобщий страх появляется в марксизме, где всемирное отчуждение
труда порождает ресентимент классовой революции и диктатуры пролетариата,
которая порождает страх невовлеченных в идеологию групп, причем не обязательно
привилегированных классов капиталистов и помещиков, но также интеллектуалов,
разночинцев, менеджеров, верных присяге военных и т.п.

Рефлексивная цель марксизма — создать, с одной стороны, ресентимент у
пролетариата, а, с другой стороны, нескомпенсированный массовый страх у
привилегированных классов.

Так возникает нескомпенсированный страх мировой революции, который на практике
оказался локальным страхом, классово структурированным, преимущественно в бурно
развивающихся странах, причем в ситуации мирового кризиса. Такой локальный, то
есть неглобальный, в большинстве регионов мира, страх возникает в связке Первая
мировая война — Революция в России 1917 года и ряд последовавших за ней
европейских революций.

Даже Первая мировая война не испугала так сильно, как Революция-1917. Между
Первой и Второй мировой войной именно страх коммуно-социалистической революции
мотивировал германские привилегированные классы на повторение попыток военного
разрешения ситуации мирового кризиса в виде Второй мировой войны.

Однако наличие так называемых Убежищ от локального страха позволяло избегать
Глобального страха. Иначе говоря, где-нибудь в Латинской Америке, в
значительной части Африки, в Австралии, Швейцарии, Португалии, Швеции или даже
в Турции и Испании вполне можно было спрятаться от страха революции и страха
войны.

Опять же локальный страх возникает в связи с изобретением ядерного оружия.
Ядерный, преимущественно локальный, страх сохранялся с 60-х годов до начала
90-х годов ХХ века. Это опять-таки Глобальный страх лишь в предельном выражении
ядерной угрозы, то есть в случае применения большей части ядерных арсеналов их
собственниками, потому что Латинская Америка, Африка и Австралия надеялись, что
ограниченная ядерная война затронет их меньше всего.

Концепция ядерного паритета стала основой равновесия ядерных угроз, поэтому
ядерное оружие подсчитывается, оценивается экспертным образом и сравнивается
для установления ядерного паритета в так называемом Ядерном клубе — клубе стран
с прогнозируемым применением ядерного оружия.

Концепция ядерного сдерживания суть политика сдерживания, не позволяющая никому
получить ядерное преимущество, а также геополитика ограничений, блокирующая
право претендентов-стран с непредсказуемой политикой на производство ядерного
оружия и вступление в Ядерный клуб.

Концепция взаимного гарантированного уничтожения суть рефлексивное объяснение и
своеобразное преодоление Ядерного Глобального страха в ситуации ядерного
паритета в закрытом Ядерном клубе: после успешного первого удара разрушительный
ответный удар неизбежен, противники вряд ли пойдут на конфликт, так как это
гарантированное уничтожение обеих сторон.

Иначе говоря, Ядерный Глобальный страх компенсировался сохранением ядерного
паритета в мире и ограниченно преодолевался за счет поддержки закрытого
характера Ядерного клуба. При этом закрытый характер Ядерного клуба не был
соблюден, у самих участников Ядерного клуба всегда был соблазн нарушить ядерный
паритет, особенно, когда неравномерность мирового цивилизационного развития
вынуждала их использовать Ядерное оружие как последний мощный инструмент
цивилизационного равновесия. То есть Ядерный Глобальный страх никогда не был ни
изжит, ни преодолен.

У Глобального ядерного страха есть свой индикатор — Часы Судного дня (англ.
Doomsday Clock) — проект журнала Чикагского университета «Бюллетень
учёных-атомщиков», начатый в 1947 году создателями первой американской атомной
бомбы. Периодически на обложке журнала публикуется изображение часов, с часовой
и минутной стрелкой, показывающих без нескольких минут полночь. Время,
оставшееся до полуночи, символизирует напряжённость международной обстановки и
прогресс в развитии ядерного вооружения. Сама полночь символизирует момент
ядерного катаклизма.

Первый эксперимент применения Глобального ядерного страха — Карибский кризис.
Первое сознательное глобальное производство Глобального страха связано с
событиями 11 сентября 2001 года — атака на башни-близнецы Всемирного Торгового
Центра и на Пентагон в США. Недолгое по времени и весьма локальное событие
имело глобальное влияние, породило свои теории, политические последствия:
Патриотический Акт США и войну США в Афганистане, продолжавшуюся 20 лет, — как
компенсации страха. Также событие имело экономические последствия и культурное
влияние.

Эпидемия коронавируса и Климатический кризис порождают первые по-настоящему
нескомпенсированные Глобальные страхи. То есть Глобальные страхи это такие, от
которых по всему миру нет Убежищ для спасения, для которых нет адекватных
теорий интерпретации, доверительных экспертных оценок и адекватных измерений,
осмысленной информационной поддержки, а также политических и экономических
стратегий, договоров, проектов, программ, направленных на изменения и изживание
страха.

Таким образом Глобальный страх в отличие от локального страха суть страх
повсеместный, всевременный, анонимный, понятийно обозначенный, но в
практическом плане с не вполне ясным источником угрозы, от которого нигде не
спасешься.

Особенно интересна в этом плане Австралия, которая, несмотря на дистанционное
участие во Второй мировой войне (нападение на флот и даже бомбежку городов
японцами), очень мало сталкивалась с локальным страхом. Однако Австралия
полноценно столкнулась с Глобальным Коронавирусно-эпидемическим страхом и
оказалась к нему не готова. Поэтому «Австралийский психоз» в ситуации эпидемии
коронавируса должен быть исследован очень пристально.

Как таковой «Глобальный страх» должен быть введен в Википедию и исследован.

\subsubsection{Инфодемия}

Особым качеством или атрибутом Глобального страха является глобальная
инфодемия. При этом инфодемия сама по себе не порождает страх, она всего лишь
является попыткой контроля и использования Глобального страха, который
распространяется по миру в значительной степени стихийно.

Инфодемия (контаминация слов «информация» и «эпидемия») — быстрое и далеко
идущее распространение точной и неточной информации о чем-либо, например,
эпидемия, катастрофа, война, сопровождающее широко распространяемое и длительно
удерживаемое состояние Глобального страха.

Статья «Инфодемия» присутствует в Википедии на английском, немецком,
французском, испанском, китайском, арабском, португальском и некоторых других
языках, но на начало 2022 года отсутствует на всех славянских языках. Славяне
до сих пор не рефлексируют свой Глобальный страх хотя бы на уровне понятия
«инфодемия», не говоря уже о понятии «ресентимент», которого избегают в
публичной риторике, и эта рефлексивная отсталость весьма красноречивая.

Термин «инфодемия» обычно приписывают журналисту и политологу Дэвиду Роткопфу,
который придумал его для описания ситуации, когда «несколько фактов, смешанных
со страхом, предположениями и слухами, быстро распространяемые по всему миру с
помощью современных информационных технологий», влияют на экономику, политику и
безопасность.

Признаки инфодемии:

1) глобальный характер тематического государственно-корпоративного
цензурирования;

2) недоверие к науке и ученым, которые на всем протяжении инфодемии вынуждены
удерживать границу между сохранением своего авторитета в сообщаемых знаниях и
предъявлением незнания, неисследованного и неясного;

3) избирательное использованием авторитета экспертов (например, компетентными в
эпидемической ситуации являются вовсе не медики, а эпидемиологи, в то же время
медики компетентны в способах индивидуального лечения больных);

4) недоверие к выгодополучателям — в ситуации коронавирусной эпидемии это
правительство, фармацевтические компании, медико-диагностические центры, СМИ и
коммуникационные платформы под влиянием правительств и
корпораций-выгодополучателей;

5) трансляция в СМИ прямых объяснений и оценок инфодемического содержания
журналистами и другими неспециалистами, применение абсолютных оценок без
сравнения с оценками похожих данных;

6) нагнетание отрицательных эмоций, разжигание агрессии и стимулирование
социального размежевания в ситуации
неопределенности-неизученности-неисследованности — в ситуации коронавирусной
эпидемии это противостояния так называемых масочников и антимасочников, так
называемых вакцинаторов и антивакцинаторов.

7) неинформационные признаки: глобальное ограничение и нарушение прав,
закрепленных в национальных конституциях и основных правовых документах ООН.

Глобальная инфодемия существовала и во время Ядерного Глобального страха —
просто она выступала в виде идеологической войны или пропаганды и
контрпропаганды. Ядерная угроза от империалистов США была составляющей
советской пропаганды, а Ядерная угроза от империи зла СССР была составляющей
западной пропаганды.

Именно поэтому инфодемию тогда нельзя было отделить от пропаганды и исследовать
как отдельное явление, сопровождающее Глобальный ядерный страх. В известном
смысле идеологии служили определенным способом компенсации страха — за счет
создания в массовом сознании ресентименального обвинения врага-противника в
производстве причин страха.

Инфодемия проявляет свои чистые сущностные качества именно когда она
глобальная, а не когда является пропагандой отдельных больших стран — например,
пропаганда в социал-националистической Германии, пропаганда в
коммуно-социалистическом СССР или даже западная либерально-демократическая
пропаганда времен Холодной войны. Глобальная инфодемия впервые обнаруживается,
обозначается и описывается именно внутри Глобального
коронавирусно-эпидемического страха.

Инфодемия — это попытка перенаправления, манипуляции или даже просто
ограниченного влияния на массовый страх со стороны потенциальных
выгодополучателей. В конечном счете, у инфодемии нет конкретного
субъекта-инициатора. Однако от инфодемии как способа
контроля-управления-использования Глобального страха есть выгодополучатели со
стороны постепенно осознающих свой интерес и перспективу государств, корпораций
и общественных групп.

Инфодемия становится продолжительной, когда выгодополучатели осознают свою
выгоду и начинают умышленно поддерживать инфодемию. Скажем CNN поддерживает
инфодемию в ситуации эпидемии коронавируса точно так же, как Фейсбук, немецкие,
французские или российские СМИ.

По мере того, как это явление будет изучено, создание инфодемически управляемых
Глобальных страхов может оказаться гуманитарными технологиями.

Скорее всего, инфодемия в ситуации коронакризиса меньше коснулась Китая,
Швеции, Беларуси и Великобритании, хотя это и требует отдельного исследования и
научных доказательств.

Выход из инфодемии состоит в аргументированной коммуникации. Обратите внимание
— аргументированная коммуникации это такая, где приводится не только научная
или официальная точки зрения, а разные точки зрения со своей аргументацией. Ибо
инфодемия подрывает влиятельность научной и официальной точек зрений.

Наверное, для ученых будет весьма неожиданным или даже шокирующим узнать, что в
процессе инфодемии они являются не носителями знаний, а носителями именно
научной точки зрения. И не потому, что в ситуации инфодемии и Глобального
страха СМИ переполнены квазинаучными сообщениями. А потому, что в вопросах
экзистенциального выбора между несвободной жизнью и свободой с риском смерти
компетенция ученых не является преобладающей, поскольку это свободный
экзистенциальный выбор, а не научный выбор внутри некоторого научного знания. В
этом смысле доминирование официальной точки зрения или научной точки зрения,
которые, например, продвигают социальные сети, контрпродуктивно.

Инфодемия имеет весьма неожиданные последствия для потери или отказа от
мышления и запрета на мышление в ситуации Глобальных страхов. Инфодемия
порождает особый инфодемический дискурс, например, «противостояние вакцинаторов
и антивакцинаторов», хотя на самом деле речь идет о противостоянии выступающих
за законную, добровольную и информированную вакцинацию и выступающих за
незаконную, принудительную и малоинформативную вакцинацию, то есть
противостояние между свободой и принуждением.

Рациональная несостоятельность принудительной вакцинации не мешает применять ее
государствам, занимать колеблющуюся позицию ООН и обогащаться корпоративным
выгодополучателям.

Причем, люди в своей массе могут выступать за компенсации и даже за наказания
по поводу дискриминации и рабства в прошлом и одновременно — за
политико-фармакологическое рабство и принуждение в настоящем по поводу
коронавирусной эпидемии. Таким образом, инфодемия плодит социальную шизофрению,
подрывающую всякую возможность рациональности.

Эмоционирования, флеш-мобы солидарности, фрики и массовые протесты не просто
ничего не значат для преодоления или компенсации массового страха, а, наоборот,
затрудняют их. Активизм отказавшихся от мышления масс это всего лишь часть
инфодемии. Компенсация и преодоление массовых страхов, поскольку происходят
через изменение массового сознания, требует мощного, на пределе сложности
эпохи, рефлексии и мышления элит.

\subsubsection{Последствия Глобального Страха и Инфодемии}

Индуктором нынешней ситуации явился Ядерный Глобальный страх, который нарастает
с 2010 года. В 2018 году индикатор Ядерного Глобального страха был на отметке
«за 2 минуты», точно такой же, как в 1953 году во время испытания термоядерных
бомб США и СССР. Даже во время Карибского кризиса индикатор был меньше («за 7
минут»). По состоянию на 2020 индикатор Ядерного Глобального страха — «за 1
минуту 40 секунд». Еще никогда за все время оценки угрозы Ядерной войны мы не
были к ней так близко.

В настоящее время мы живем в ситуации трех Глобальных страхов: Ядерного страха,
Коронавирусно-эпидемического страха и Климатического страха. Эти наложенные
друг на друга Глобальные страхи поддерживают и индуцируют друг друга. То есть
появившийся Глобальный Коронавирусно-эпидемический страх в 2020 году позволил
индуцироваться Глобальному климатическому страху и усилится Глобальному
ядерному страху.

Невозможность установления равновесия трех Глобальных страхов обеспечивают
стихийно соперничающие между собой имперский авторитаризм, национал-шовинизм и
либеральная демократия. Стихия неравномерного мирового развития, наложенная на
стихию неуправляемых и неконтролируемы массовых страхов, создает условия для
Глобальной войны — по содержанию: ядерной, биологической, экологической и
кибернетической; по сторонам противостояния: между государствами, корпорациями
и общественными группами самоопределения.

В ситуации нынешних Глобальных страхов при потере-отказе-запрете на мышление
доминирует образное творчество, а не теоретическая рефлексия. Фильмы о массовых
смертях — 11 сезонов сериала «Ходячие мертвецы» и 7 сезонов «Бойтесь ходячих
мертвецов», а сериалы «Игра в кальмара» и «Зов ада» становятся лидерами
просмотров 2021 года. Глобальный страх начинает нерефлексивно доминировать в
образном творчестве.

Очень хочется патетически воскликнуть: неужели нынешняя молодежь всю жизнь
проживет в страхе за свою жизнь? Это будет жалкая жизнь. Прямо по
Салтыкову-Щедрину: «Жил — дрожал, и умирал — дрожал». Такой
нескомпенсированный, непреодоленный и неизбывный многокомпонентный Глобальный
страх — прямой путь к фашизму.

Глобальный страх при посредстве инфодемии сильно блокирует способности
мышления, воображения, рационального рассуждения и принятия рациональных
решений. Поэтому в состоянии Глобального страха весьма трудно и порой вообще
невозможно вести аргументированную коммуникацию по поводу причин страха и о
самой ситуации, в которой возник страх.

Концептуальные обеспечения Глобального коронавирусно-эпидемического страха и
Глобального климатического страха по большому счету отсутствуют. Несмотря на
продолжающуюся коронавирусную эпидемию человечество до сих пор не имеет
доказанных представлений о происхождении эпидемии, имеет неконтролируемое и
непредсказуемое по направлениям распространение эпидемии, слабо прогнозируемое
по периодам усиления и затухания, полноценную глобальную инфодемию и массовые
протесты против неконституционных способов регулирования карантина и
ограничения прав.

Проблема инфодемии в том, что ее не могут не только рефлексировать или
оценивать, но даже не могут распознать именно те, кто подвержен инфодемическому
страху. Находящиеся в страхе воспринимают инфодемическое нагнетание страха как
адекватную информационную поддержку. Избыточность количества и эмоциональности
говорения по инфодемической теме замечается, но воспринимается как оправданные
и желательные.

В ситуации Глобального страха не срабатывают апелляции к Конституции, к правам
и свободам. То есть люди или человечество в ситуации Глобального страха
становятся податливыми не только к ограничению их прав, но и к тотальному
порабощению. Как говорит Агамбен, люди переходят в состояние непрекращающегося
чрезвычайного положения.

Референдумы в ситуации Глобального страха о поддержке решений правительства по
ограничению конституционных прав не решают проблему неконституционности этих
ограничений, поскольку сама конституция принимается исключительно в состоянии
свободы, бесстрашия и достоинства общества перед лицом вечности и всемирности.
Это весьма интересная проблема инициативы права — принимать или изменять
правомочные решения нужно не только в здравом уме и твердой памяти, но и не под
давлением страха, безнадежности или депрессии.

Понятно, что с тремя Глобальными страхами национальные государства уже не
справляются. Государства, словно позабыли, что их поддерживают, потому что они
обеспечивают устойчивую законность, гуманитарную стабильность и предсказуемость
изменений. Когда государства перестают обеспечивать базовые условия
безопасности, они становятся не нужны. Кому нужно государство, которое
бесконечно запугивает и создает непредсказуемость?

Для Глобальных страхов средством контроля и управления является выход
Глобального Права за пределы национальных конституций в вопросах ограничения
гражданских прав, ограничений непосредственной социальной коммуникации,
тотального контроля передвижения людей, ограничения определенных типов
деятельностей.

Однако глобальное правительство не может возникнуть в ситуации многих
глобальных страхов, потому как мировая элита сама испугалась до дрожи в
коленях. Вполне возможно, что роль глобального правительства возьмут на себя
корпорации и искусственный интеллект.

Компенсация и преодоление Глобальных страхов возможно не только через мировую
войну, но также через цифровой фашизм и кибер-фашизм, которые не следует
путать. Цифровой фашизм это когда вводятся тотальные контроль и принуждение со
стороны правительств и корпораций при помощи цифровых технологий. А
кибер-фашизм — это когда контроль и принуждение перекладывается на
искусственный интеллект.

Для Глобальных страхов средствами контроля и управления являются не разного
рода договора и соглашения, а первое полноценное Всечеловеческое
надгосударственное и надкорпоративное управление или же цифровой и
кибер-фашизм.

Перед нами стоит несколько важных вопросов. Способны ли отдельные микрогруппы
удержать человеческое достоинство за счет спонтанных мышления-воли-веры и
вывести Всечеловечество из Глобального страха к экзистенциальному
преобразованию? Способна ли мировая элита сохранить остатки рационального
мышления и не ввергнуть мир ни в войну, ни в фашизм? Способны ли отдельные люди
преодолеть Глобальные страхи и услышать призывы к позитивным смыслам и
перспективам отдельных микрогрупп?

Это тот выбор, который уже стоит перед нами: Глобальная война или
цифровой/кибер-фашизм или Всечеловечественное управление миром.
