%%beginhead 
 
%%file 24_01_2023.fb.kazarin_pavel.1.menya_inogda_prosyat
%%parent 24_01_2023
 
%%url https://www.facebook.com/kazarin.pavel/posts/pfbid02aqp9DhYKYUNeETabLNwnqRwGs31VF8yqKZczHZB1uR5ikCYzwXAtbde1KyTdJENGl
 
%%author_id kazarin_pavel
%%date 24_01_2023
 
%%tags prognoz,analiz,pobeda,vojna,vojna.2022
%%title Меня иногда просят дать комментарий о войне
 
%%endhead 

\subsection{Меня иногда просят дать комментарий о войне}
\label{sec:24_01_2023.fb.kazarin_pavel.1.menya_inogda_prosyat}

\Purl{https://www.facebook.com/kazarin.pavel/posts/pfbid02aqp9DhYKYUNeETabLNwnqRwGs31VF8yqKZczHZB1uR5ikCYzwXAtbde1KyTdJENGl}
\ifcmt
 author_begin
   author_id kazarin_pavel
 author_end
\fi

Меня иногда просят дать комментарий о войне. 

Спрашивают про обстановку на фронте. Про западное оружие. Про перспективы
наступления. Единственное, что мне остается – отвечать «я не знаю». 

Армия иерархична. И чем меньше звезд на твоих погонах – тем меньше ты знаешь о
ситуации. На моих погонах звезд нет вовсе – а потому уровень моего армейского
кругозора нередко меньше вашего. Потому что на чтение новостей время остается
не всегда. 

Вдобавок, спрашивают о процессах, а отвечать приходится о людях, которые рядом.
Любое шапкозакидательство фонит пренебрежением к их жизням. Отдает бахвальством
и высокомерием. Наверное поэтому я осторожен в оценках и консервативен в
прогнозах. 

Внутри нашей страны успел сложиться консенсус. Разговор о сценариях победы
всегда упирается в параллели со Второй Мировой. Когда враг не просто повержен,
но еще и переформатирован. Когда не только выход на границы, но еще репарации,
не только физическое разоружение, но еще и ментальное. Мы закономерно мечтаем
про окончательную победу – ту, по итогам которой агрессор переходит в новое
качество и не представляет угрозы в будущем. 

Но всегда есть риск стать заложником собственных ожиданий. Даже выход на
границы 91-го года не означает однозначного конца войны. Россия может сесть в
глухую оборону, отказаться от переговоров и начать зализывать свои военные
раны. А тем временем российских детей в школах продолжат учить тому, что
«принадлежащие их стране» города Херсон, Донецк, Луганск, Запорожье и
Симферополь «оккупированы фашистами». И в этом сценарии у нашей нынешней войны
обязательно будет продолжение. 

Оптимистом легче быть сегодня, но реалистам легче придется завтра. Войны в
Боснии и в Хорватии длились по четыре года. Война в Корее – три. Французы
сражались во Вьетнаме восемь лет – столько же, сколько и американцы. Вполне
вероятно, что Путин воспринимает итог этой войны как свое политическое
наследство – а потому у него нет ни единого резона идти на попятный. 

На этом фоне довольно сильно раздражают те, кто пытается торговать надеждой на
скорую победу. Все те, кто продают «заговор в Кремле», «онкологию у Путина» и
«восстание в российских регионах». Нам очень хочется «Бога из машины» и те, кто
зарабатывают себе имиджевый капитал на войне, – не особо церемонятся с
заголовками. С той лишь разницей, что эффект плацебо в наших условиях не
работает. Ваша вера в победу имеет мало смысла, если вы победу никак не
приближаете. 

«Этот дождь надолго» и к этому лучше привыкнуть. Ставка на спринтерский рывок
всегда выходит боком на стайерской дистанции. А потому даже в армии не так уж
важен твой первоначальный порыв. Тот самый, что заставляет тебя надеть форму и
взять в руки оружие. Потому что после первых недель героических селфи тебя
непременно догоняют быт и усталость. И единственное, что обретает смысл в этот
момент – это твоя готовность «тянуть». 

Твоя готовность тянуть службу и повседневность. Бытовую неустроенность и
рутину. Физические нагрузки и выгорание. Неизбежную бюрократию и стресс. Этого
всего нет на плакатном изображении войны – да и сами мы тут очень далеки от
плакатности. В каждом из нас уйма недостатков и при ближайшем рассмотрении
довольно просто разглядеть изъяны. Но если мы не помещаемся в чей-то шаблон, то
проблема не в нас, а в шаблоне. 

Может поэтому мне куда больше по душе осторожные оценки и консервативные
прогнозы. Те, что не торгуют предчувствием скорого триумфа. Те, что не твердят
о скором крахе российского режима. Сосредоточенные и злые на этой войне имеют
больше шансов, чем расслабленные и оптимистичные. Поэтому я люблю
сосредоточенных и злых. 

Я не знаю, когда украинские флаги появятся в Крыму. Не могу оценить степень
устойчивости российской экономики. Понятия не имею, какими будут границы России
через пять лет. А еще я довольно скептично отношусь к формуле «вернутся ребята
– наведут порядок». Вера в подобное отдает инфантильным желанием переложить на
нас ответственность, а я по-прежнему не теряю надежду, что война заставит
страну повзрослеть.  

Нам и так во многом везет. Не нужно придумывать чудеса – достаточно тех, что
уже случились. Наша армия сражается так, как не приходилось за последние сорок
лет ни одной другой. Наш тыл – несмотря на ракеты – продолжает жить в условиях
довоенного быта. «Карточная система», «продуктовый паек» и «гуманитарная
катастрофа» не даны абсолютному большинству в ощущениях и остаются чем-то из
документальных фильмов на Нетфликсе. А условия западной финансовой помощи дают
осторожную надежду на реформы в стране. 

Мои знакомые иногда пишут, что на их внутреннем календаре по-прежнему февраль
двадцать второго. В этом подходе чудится мечта о возвращении к довоенному. Не
уверен, что готов согласиться с подобным. Мы пережили слишком многое, чтобы это
можно было обнулить. У нас за спиной – лишь первый год полномасштабной войны и
мы не знаем, сколько таких лет будет впереди. Все, что происходит с нами – это
не нарушение нормы. Это и есть новая норма. 

Жизнь становится проще в тот момент, когда ты понимаешь, что проще она не
станет.
