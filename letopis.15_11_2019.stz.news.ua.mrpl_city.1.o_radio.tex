% vim: keymap=russian-jcukenwin
%%beginhead 
 
%%file 15_11_2019.stz.news.ua.mrpl_city.1.o_radio
%%parent 15_11_2019
 
%%url https://mrpl.city/blogs/view/ko-dnyu-rabotnikov-radio-televideniya-i-svyazi-o-radio-v-mariupole-v-voennye-gody
 
%%author_id burov_sergij.mariupol,news.ua.mrpl_city
%%date 
 
%%tags 
%%title Ко Дню работников радио, телевидения и связи: о радио в Мариуполе в военные годы
 
%%endhead 
 
\subsection{Ко Дню работников радио, телевидения и связи: о радио в Мариуполе в военные годы}
\label{sec:15_11_2019.stz.news.ua.mrpl_city.1.o_radio}
 
\Purl{https://mrpl.city/blogs/view/ko-dnyu-rabotnikov-radio-televideniya-i-svyazi-o-radio-v-mariupole-v-voennye-gody}
\ifcmt
 author_begin
   author_id burov_sergij.mariupol,news.ua.mrpl_city
 author_end
\fi

Вскоре после начала Великой Отечественной войны было предписано населению сдать
радиоприемники в специально указанные места \enquote{для временного хранения до
окончания военных действий}. В довоенные годы далеко не каждая семья имела
возможность обзавестись чудом техники в виде фанерного ящика, содержащего
внутри себя таинственно мерцающие оранжевым накалом шесть - семь внушительного
размера электронных ламп. Ящика, оборудованного круглой ручкой, вращая которую,
можно было \enquote{поймать} Москву или Киев, Тбилиси или Ленинград, а если повезет, то
услышать сквозь треск эфира позывные радио Праги или Кенигсберга. Для тех, кому
удавалось приобрести приемник, он становился, пожалуй, самой ценной вещью в
доме, предметом особой гордости и заботы.

\ii{15_11_2019.stz.news.ua.mrpl_city.1.o_radio.pic.1.svd9}

\textbf{Читайте также:} 

\href{https://mrpl.city/blogs/view/mariupol-fm-antitila-bude-sin}{Mariupol FM: Антитіла. Буде син, mrpl.city, 30.10.2018}

Можно себе представить, с какими сердечными терзаниями и печалью сносили свои
радиосокровища жители старой части Мариуполя в городской театр. Он стоял там,
где сейчас на проспекте Мира возвышается многоэтажный дом с отделением
Сбербанка на первом этаже. Ильичевцы доставляли свои 6Н-1, СВД-М, СВД-9 и
другие произведения советской радиопромышленности в клуб им. Карла Маркса, чье
здание, построенное еще в дореволюционные годы, сохранилось до сих пор. Не
думайте, что после этой акции мариупольские жилища и помещения разнообразных
учреждений напрочь лишились изобретения талантливого инженера Попова. Оно
присутствовало в виде радиорепродукторов — \enquote{тарелок}, представлявших собой
конусы, склеенные из тонкого, но довольно прочного черного картона. Эти
устройства должны были круглосуточно находиться во включенном состоянии. За
этим строго следили представители власти \enquote{на местах} — квартальные. Ведь
городская радиотрансляционная сеть использовалась не только для удовлетворения
информационных и идеологических потребностей населения, но также — может, это
было главным — для оповещения о налетах вражеской авиации...

8 октября 1941 года, когда немецкие войска заняли почти весь Мариуполь, местный
радиоузел продолжал транслировать Москву и передавать местные сообщения. Лишь к
вечеру репродукторы умолкли. Но, как оказалось, ненадолго. Оккупанты не
замедлили воспользоваться сетью радиоточек для передачи своих приказов. Приказы
были разные, но текст их заканчивался всегда одинаково с солдатской четкостью и
определенностью: \enquote{За невыполнение — расстрел}. И расстреливали... В начале осени
1943 года части и соединения Красной Армии неумолимо приближались к нашему
городу, 6 сентября гитлеровцы, готовясь к отступлению, приступили к
планомерному сожжению Мариуполя. Многое погибло в огне, в том числе и
трансляционная радиосеть. Однако в первые же дни освобождения радиофикаторы
приступили к ее восстановлению. Сейчас, правда, трудно объяснить нашим
современникам, каких невероятных усилий это стоило, но факт остается фактом: 5
ноября 1943 года в сохранившихся после пожара домах мариупольцев заговорили
репродукторы.

\textbf{Слушайте также:} \href{https://mrpl.city/radio.html}{радио Mariupol FM}

В те уже отдаленные годы они не только сообщали новости, но и помогали
ориентироваться во времени. Для этого необязательно было ждать, когда диктор
сообщит, который час и сколько минут в столице. Поднаторевшие радиослушатели по
тематике передач сразу определяли, когда нужно идти на работу или еще
куда-нибудь по неотложным делам. В отведенное время вклинивалось: \enquote{Работает
радиостанция РВ-26 \enquote{Сталино-Донбасс}}, и диктор читал областные новости и
сообщения. Периодически Мариупольский радиоузел давал объявления горисполкома
по проводной сети... Окончилась война, ушли в историю сводки Совинформбюро,
трансляции вечерних салютов в честь воинских соединений, освобождавших города
от коричневой чумы. А радиоточки еще долго оставались главным каналом
информации для мариупольцев, да и не только для них.
