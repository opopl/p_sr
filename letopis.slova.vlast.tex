% vim: keymap=russian-jcukenwin
%%beginhead 
 
%%file slova.vlast
%%parent slova
 
%%url 
 
%%author 
%%author_id 
%%author_url 
 
%%tags 
%%title 
 
%%endhead 
\chapter{Власть}
\label{sec:slova.vlast}

Что бы ни делала нынешняя \emph{власть}, даже при самом благоприятном раскладе
видим неправильное движение в правильном направлении.  Вроде бы, на знамена
поднимаются адекватны идеи. Но они настолько через одно место реализовываются,
что вызывают когнитивный диссонанс.  Вот и в этом случае – надо уметь настолько
удобно подставиться под критику.  Это как в боксе выйти с опущенными руками и
будто на блюдечке подставить свою челюсть сопернику. Взяли интересную идею — и
разработали идиотскую ее реализацию.  Реструктуризации долгов – правильная
идея, она обсуждается во всём мире. Абсолютно не правы те, кто критикует ее.
Они или защищают интересы финансовых спекулянтов, или просто не читают ведущих
экономистов мира. А те говорят, что реструктуризация долгов для бедных стран в
условии нынешнего кризиса — фактически предопределенная задача. Кредиторы
должны будут согласиться на более мягкие или более жесткие условия
реструктуризации, умеряв свою жадность,
\citTitle{Власть вроде бы озвучивает здравые идеи - но реализовывает их по-идиотски}, Алексей Кущ, strana.ua, 09.06.2021


%%%cit
%%%cit_head
%%%cit_pic
%%%cit_text
Отчуждение \emph{власти}. После вчерашнего молитвенного стояния вурующих УПЦ
под стенами парламента, президент презрительно назвал их \enquote{бабушками в
кепочках ОПЗЖ}, перепутав два разных мероприятия, проходивших в тот день. И ещё
более презрительно сказал \enquote{опять что-то в церкви не в порядке},
предположив, что все проблемы можно решить \enquote{раздав бабушкам денег}.
Ранее он не менее презрительно говорил о том, что в стране, якобы, \enquote{нет
проблем с русским языком}, а его предшественник действовал ещё проще, втупую
объявив половину населения страны \enquote{агентами Кремля}
%%%cit_comment
%%%cit_title
\citTitle{Зеленский позволил себе презрительно высказаться о протестующих прихожанах УПЦ}, 
Даниил Богатырев, strana.ua, 16.06.2021
%%%endcit

%%%cit
%%%cit_head
%%%cit_pic
%%%cit_text
На самом деле, всё это - звенья одной цепи отчуждения \emph{власти} от граждан
страны.  От кого-то она отчуждается идеологически. От других - по признаку веры
либо атеизма. От третьих - через желание раз за разом шарить в их кошельке. От
четвёртых - через стремление \enquote{отжать} их землю и квартиры.  В этой
ситуации, ума не приложу, для кого эта \emph{власть} может быть
\enquote{своей}. Вероятно, кроме назначенных ею чиновников, таких людей не
осталось
%%%cit_comment
%%%cit_title
\citTitle{Зеленский позволил себе презрительно высказаться о протестующих прихожанах УПЦ}, 
Даниил Богатырев, strana.ua, 16.06.2021
%%%endcit

%%%cit
%%%cit_head
%%%cit_pic
%%%cit_text
\emph{Власть} Украины выбрала путь национальной резервации, поскольку этот путь просто
легче, хоть он и губительный. Сегодня в стране бушует правовой нигилизм. Закон
в целом воспринимается не выразителем интересов общественной безопасности и
развития, а досадным недоразумением, которое направлено на ограничение воли
народа. Украинцы не доверяют судам, Верховной Раде, полиции, органам
безопасности и \emph{власти} в целом.  Но если депутатам непонятно, какую страну мы
строим, судьям ‒ кого и за что судить, а полиции ‒ кого и от кого защищать, то
правовое государство нам не построить. Мы, граждане Украины, должны запустить
политический процесс строительства правового государства. При этом это должен
быть именно отечественный внутренний процесс, с политическими дискуссиями,
широким спектром мнений. Украинцы должны стать субъектами в формировании своих
прав и свобод и перестать быть подопытными кроликами для иностранных советников
%%%cit_comment
%%%cit_title
\citTitle{О будущем украинского и русского народов}, 
Виктор Медведчук, strana.ua, 15.07.2021
%%%endcit

%%%cit
%%%cit_head
%%%cit_pic
%%%cit_text
Серед вимог було надати приміщення Народному Руху, а також припинити
переслідування всіх «неформальних громадських організацій». \emph{Влада}
погодилася на певні на поступки.  Знаковий день.  Начальник Південно-східного
міжрегіонального відділу Українського інституту національної пам'яті Ігор
Кочергін каже: 16 липня 1991 року було знаковим днем для українців. Перший і
єдиний раз річницю ухвалення Декларації про державний суверенітет в Україні
відзначали як День Незалежності
%%%cit_comment
%%%cit_title
\citTitle{«Нас називали безумцями». 30 років тому над містом Дніпром вперше підняли синьо-жовтий прапор}, 
Юлія Рацибарська, www.radiosvoboda.org, 16.07.2021
%%%endcit

%%%cit
%%%cit_head
%%%cit_pic
%%%cit_text
Лучший правитель — тот, кто мечтает о \emph{власти}, и кто ею тяготится. Лучший
эксперт — тот, кто не спешит высказывать свое мнение, навешивать ярлыки и вести
вас к лучшей жизни, — потому что осознает ответственность за каждое сказанное
им слово. Такого эксперта еще нужно поискать. Поэтому, конечно, гораздо проще
спросить у гугла или фейсбука. Они-то всегда имеют наготове не один десяток
ответов.  И, вероятно, лучшая стратегия в мире, где экспертные мнения
загораются как лампочки на новогодней гирлянде — смотреть на все это издали, ни
в коем случае не поднося к глазам. Не бежать очертя голову туда, где горит ярче
всего, даже, если горит вашим любимым светом. И сомневаться. Всегда и во всем.
Особенно когда нет никаких причин для сомнения
%%%cit_comment
%%%cit_title
\citTitle{Фраза.юа - авторский взгляд на жизнь}, Варвара Фалеева, fraza.com, 19.10.2021
%%%endcit

%%%cit
%%%cit_head
%%%cit_pic
\ifcmt
tab_begin cols=2
	pic https://strana.news/img/forall/u/0/36/2021-10-28_16h16_01.png
  pic https://strana.news/img/forall/u/0/36/2021-10-28_16h19_58.png
tab_end
\fi
%%%cit_text
Неизвестно, знает ли об этом Денис Шмыгаль. Но это и не столь важно.
\emph{Власть} продолжает навязывать украинцам взгляд на историю, который не
просто противоречит реальности, но и оскорбляет память миллионов погибших и
сражавшихся украинцев.  И внушительные колонны людей, которые каждый год
выходят почтить память Победы каждый год на 9 мая - показатель того, что далеко
не всем украинцам эта идеология по душе.  Поэтому в соцсетях уже обсуждают
очередное "поздравление" от украинской \emph{власти}
%%%cit_comment
%%%cit_title
\citTitle{Высадка союзников в Ужгороде. Кого Зеленский и Шмыгаль поблагодарили за освобождение Украины от нацистов}, 
Анна Копытько, strana.news, 28.10.2021
%%%endcit

%%%cit
%%%cit_head
%%%cit_pic
\ifcmt
  pic https://avatars.mds.yandex.net/get-zen_doc/4350071/pub_618312bce8eeee6665ae858c_618312f737011142658acd1c/scale_1200
  @width 0.4
\fi
%%%cit_text
А вот с писательством у них сложилось хорошо. Удивительно ведь, правда? Я про
чудесное овладение украинским языком. Родились в молдавском городе Дубоссары, в
русскоязычной семье. Десять лет учились в русской школе в русскоязычном
городке, потом четырнадцать лет жили в России. И вдруг – украинские писатели.
Плодовитые, надо признать, писатели.  При этом они еще и очень успешные
издатели. С ними считаются, к ним прислушиваются где-то там - на самом верху, в
высших эшелонах украинской \emph{власти}. Капрановы ратовали за введение
полного запрета на ввоз книг из России.  И такой запрет был введен. Серьезных
конкурентов теперь у братьев-книгоиздателей на украинском рынке нет
%%%cit_comment
%%%cit_title
\citTitle{Украинские писатели-братья}, Украинский русский, zen.yandex.ru, 04.11.2021
%%%endcit

%%%cit
%%%cit_head
%%%cit_pic
%%%cit_text
Анна Герман, бывшая спикер и советник президента Виктора Януковича, дает свой
взгляд на события того периода.  Она считает, что \emph{власть} не отказалась
от подписания Ассоциации, а сделала паузу, о чем прямо и заявляли чиновники. Но
проблема была в том, что приостановка случилась очень резко и неожиданно для
всех - хотя еще вчера власть говорила, что подписание - вопрос решенный.  "Мы
делали все, чтобы выторговать лучшие условия для Украины... Но такие решения
требуют подготовки общественного мнения. Если бы  мы раньше начали говорить
людям, что это соглашение невыгодно Украине, общественное мнение было бы лучше
подготовлено. Оппозиция же имела мощные рычаги пропаганды", - говорит Герман.
По ее словам, в администрации президента периодически подкармливали отдельных
участников протеста. "Я, когда ехала на работу, брала в свою машину замерзших
парней из Львова. Говорила им: ребята, захотите погреться и покушать -
приходите к злочинной владе".  Герман рассказывает, как у Януковича угощали
майдановцев пирожками.  "Все говорят о Виктории Нуланд, которая вышла с
пирожками. Но первая-то с пирожками вышла я! Мы заказали корзину пирожков, а
потом пошли и к солдатам, которые охраняли АП, и на сторону Майдана - в знак
того, что нам надо примиряться. Это граждане нашей страны. И те наши, и те
наши", - говорит Анна Николаевна
%%%cit_comment
%%%cit_title
\citTitle{Как убивали на Майдане и что он дал стране. Воспоминания участников событий 8 лет спустя}, 
Юлия Колтак, strana.news, 21.11.2021
%%%endcit
