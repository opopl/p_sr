% vim: keymap=russian-jcukenwin
%%beginhead 
 
%%file 12_06_2023.stz.news.ua.donbas24.1.mariupol_kamerna_filarmonia_proekt_zagybli_mytci
%%parent 12_06_2023
 
%%url https://donbas24.news/news/mariupolska-kamerna-filarmoniya-dolucilasya-do-projektu-prisvyacenomu-zagiblim-mitcyam
 
%%author_id demidko_olga.mariupol,news.ua.donbas24
%%date 
 
%%tags 
%%title Маріупольська камерна філармонія долучилася до проєкту, присвяченому загиблим митцям
 
%%endhead 
 
\subsection{Маріупольська камерна філармонія долучилася до проєкту, присвяченому загиблим митцям (ВІДЕО)}
\label{sec:12_06_2023.stz.news.ua.donbas24.1.mariupol_kamerna_filarmonia_proekt_zagybli_mytci}
 
\Purl{https://donbas24.news/news/mariupolska-kamerna-filarmoniya-dolucilasya-do-projektu-prisvyacenomu-zagiblim-mitcyam}
\ifcmt
 author_begin
   author_id demidko_olga.mariupol,news.ua.donbas24
 author_end
\fi

\begin{center}
\Large\bfseries\em\color{blue}
У Києві пройшов концерт-присвята загиблим артистам України
\end{center}

\ifcmt
  ig https://i2.paste.pics/ed6c2800db9463d9106c23e0200a6e84.png
  @wrap center
  @width 0.9
\fi

10 червня у колонній залі Національної філармонії України відбувся
концерт-присвята загиблим артистам \enquote{Янголи мистецтва}. Це окремий проєкт
народної артистки України Ольги Чубаревої, з яким Національна філармонія
України приєдналася до міжнародного марафону пам'яті Василя Сліпака. Концерт
проходив за участю камерного оркестру \enquote{Ренесанс} Маріупольської камерної
філармонії під керівництвом заслуженого діяча мистецтв України, художнього
керівника і головного диригента Василя Крячка. 

\textbf{Читайте також:} \href{https://donbas24.news/news/u-kijevi-vidbulosya-proshhannya-z-geroyicnim-mariupolcem}{У Києві відбулося прощання з героїчним маріупольцем (ВІДЕО), Ольга Демідко, donbas24.news, 06.06.2023}

І камерний оркестр \enquote{Ренесанс} Маріупольської камерної філармонії, і оркестр, і
солістки — народна артистка України Ольга Чубарева (сопрано) та лауреатка
Міжнародного конкурсу Тетяна Жук-Сєдова (орган) — підкорили всіх присутніх
віртуозною грою та своєю майстерністю.

\begin{leftbar}
  \begingroup
    \bfseries
{\color{blue}\em\enquote{Ми маємо вшанувати пам'ять всіх наших героїв, героїв-митців, інших
професій всіх людей, чиї життя забрала ця страшна, жорстока, нелюдська
війна. Я хочу, щоб ми всі разом з вами помолилися за здоров'я і життя
тих людей, які зараз боронять нашу Україну. Ми обов'язково переможемо.
Україна буде вільною. Вона буде щасливою і процвітаючою. Вічна шана
героям-митцям, вічна шана всім воїнам! Герої не вмирають}}, — з такими
словами зі сцени звернулася до всіх глядачів Ольга Чубарева. 
  \endgroup
\end{leftbar}

\textbf{Читайте також:} \href{https://donbas24.news/news/mariupolska-aktrisa-z-ukrayinskimi-virsami-vistupaje-v-nimeckix-mistax}{%
Маріупольська актриса з українськими віршами виступає в німецьких містах, Ольга Демідко, donbas24.news, 08.06.2023}

\ifcmt
  tab_begin cols=1,no_fig,center,separate
     pic https://donbas24.news/storage/news/12wgocznndqwzdry.jpg
     pic https://donbas24.news/storage/news/wttbvqlktdjlhiok.jpg
  tab_end
\fi

Між музичними номерами звучали короткі історії, присвячені артистам, поетам,
музикантам, літераторам, культурологам, соціологам, краєзнавцям України, які
загинули після 24 лютого. Ведучий, український актор театру та кіно \textbf{Олександр
Рудько} назвав імена і відомих маріупольців, які загинули у блокадному місті чи
на ролі бою. Це \emph{Богдан Слющинський, Наталя Харакоз, Сергій Буров, Даня
Подибайло, Єлизавета Очкур та Соня Амельчакова}.

\begin{leftbar}
  \begingroup
    \bfseries
{\em\color{blue}\enquote{На жаль, війна забирає найкращих синів України. І це дуже прикро. Бо
після Перемоги саме вони мали б відновлювати культуру та духовність
нашої країни. Досить символічним стало те, що саме маріупольський
оркестр виступав на концерті. Культура окупованого міста Марії не
вмирає та працює заради відновлення фізичного тіла міста. Адже культура
є тим єдиним ідентифікатором, за яким ми точно знаємо, чи є
маріупольцями}}, — наголосила директорка Департаменту
культурно-громадського розвитку Маріупольсьскої міської ради Діана
Трима.
  \endgroup
\end{leftbar}

\textbf{Читайте також:} 

\href{https://donbas24.news/news/zustrinemosya-na-drami-mariupolciv-rozculilo-proniklive-video-pro-ridne-misto-video}{%
\enquote{Зустрінемося на драмі}: маріупольцям призначили побачення в українському звільненому місті (ВІДЕО), %
Еліна Прокопчук, donbas24.news, 26.05.2023%
}

\ifcmt
  ig https://i2.paste.pics/6b5150df9caf1f966af4fa89261007a2.png
  @wrap center
  @width 0.9
\fi

\href{https://youtu.be/D0xOY1ya7os}{Концерт-присвята загиблим артистам \enquote{Янголи мистецтва}, Донбас Новини, youtube, 12.06.2023}

У фіналі концерту прозвучало Адажіо Томазо Альбіноні в інтерпретації Ольги
Чубаревої, однієї з найкращих, за версією міжнародної премії Adagio Words,
виконавиць цього твору. Цей важливий концерт-реквієм заплановано представити
публіці знову через місяць. 

\begin{leftbar}
  \begingroup
    \bfseries
{\em\color{blue}\enquote{Зараз у мене запланована поїздка в Берлін з окремим проєктом, після чого
знову повернуся в Київ. Думаю, у серпні знову представимо цей проєкт,
який сьогодні відгукується у серці кожного з нас}}, — зазначив
заслужений діяч мистецтв України, художній керівник і головний диригент
Василь Крячок.
  \endgroup
\end{leftbar}

\ifcmt
  tab_begin cols=1,no_fig,center,separate
     pic https://donbas24.news/storage/news/b4t4fs8gqnehqg3l.jpg
     pic https://donbas24.news/storage/news/cs4inkbleawpfnwh.jpg
     pic https://donbas24.news/storage/news/xnsok2y6zixhudvy.jpg
  tab_end
\fi

Раніше \href{https://donbas24.news}{Донбас24} розповідав, що захисник \enquote{Азовсталі} возз'єднався з пухнастим
другом.

\begingroup
\em
Ще більше новин та найактуальніша інформація про Донецьку та Луганську області
в нашому \href{https://t.me/donbas24}{телеграм-каналі Донбас24}.

ФОТО: з архіву Донбас24 та з архіву Едуарда Межула 
\endgroup

\ii{insert.author.demidko_olga}

