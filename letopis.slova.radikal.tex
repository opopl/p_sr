% vim: keymap=russian-jcukenwin
%%beginhead 
 
%%file slova.radikal
%%parent slova
 
%%url 
 
%%author 
%%author_id 
%%author_url 
 
%%tags 
%%title 
 
%%endhead 
\chapter{Радикал}
\label{sec:slova.radikal}

%%%cit
%%%cit_head
%%%cit_pic
\ifcmt
  tab_begin cols=2
     pic https://strana.ua/img/forall/u/0/36/2021-08-09_10h58_25.png
     pic https://strana.ua/img/forall/u/0/36/photo_2021-08-09_09-30-47.jpg
  tab_end
\fi
%%%cit_text
Не смолчал и обвиняемый в убийстве \emph{радикал} Сергей Стерненко, который считает,
что "спорт вне человечности".  "Младший сержант ВСУ Ярослава Магучих обнималась
с капитаном вооруженных сил страны-агрессора РФ Марией Ласицкене.  Коллеги
Марии убивают коллег Ярославы, но ее, похоже, это не беспокоит.  Так же как не
беспокоит тот факт, что Ласицкене - доверенное лицо Путина.  Просто спорт вне
человечности.  Между тем на фронте еще одного нашего воина убили русские
оккупанты", - сообщил Стерненко в Facebook
%%%cit_comment
%%%cit_title
\citTitle{\enquote{Поддерживают дискурс вражды}. Как власти кошмарят олимпийскую призерку Магучих за фото с россиянкой}, 
Анна Копытько, strana.ua, 09.08.2021
%%%endcit

%%%cit
%%%cit_head
%%%cit_pic
\ifcmt
  pic https://strana.news/img/forall/u/0/0/photo_2021-10-27_23-28-58.jpg
  @width 0.4
\fi
%%%cit_text
В среду, 27 октября, в Сумах застрелили \emph{радикала} из "Правого сектора"
Александра Иванину. Инцидент произошел вечером, возле дома погибшего и на
глазах у его супруги.  Об этом сообщает "Общественное".  "Около 9 часов Саша
вместе с женой Аней подъехали на автомобиле к своему подъезду, Саня вылез из
машины, полез в багажник за вещами. В это время неподалеку, на лавочке, сидел
какой-то мужчина. Как описывает свидетель, увидел. Саню, достал оружие - и
начал стрелять
%%%cit_comment
%%%cit_title
\citTitle{В Сумах на глаза у жены убили члена "Правого сектора". Фото, видео 18+}, 
Эллина Либцис, strana.news, 27.10.2021
%%%endcit

%%%cit
%%%cit_head
%%%cit_pic
%%%cit_text
Таким образом, нынешняя слава Тараса Шевченко – это результат специальной
идеологической кампании революционеров в 1920-е годы в рамках насильственной
украинизации юго-западной части Руси-России, когда создавали «Украину», как
отдельное государственное образование и «украинский народ», как отдельный от
русского народа этнос. Тогда срочно понадобились кумиры «украинского народа»,
вспомнили и про Шевченко, так он был бы лишь одним из многих представителей
русской интеллигенции, родом из Малороссии. А с 1991 года эта информационная
кампания приняла уже более \emph{радикальный}, антирусский характер. Шевченко сделали
кумиром украинских нацистов, хотя в реальности он был сторонником панславизма –
создания единой славянской державы, включая западных и южных славян
%%%cit_comment
%%%cit_title
\citTitle{Шевченко без украинизма}, , topwar.ru, 09.03.2019
%%%endcit
