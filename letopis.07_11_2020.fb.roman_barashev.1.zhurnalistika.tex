% vim: keymap=russian-jcukenwin
%%beginhead 
 
%%file 07_11_2020.fb.roman_barashev.1.zhurnalistika
%%parent 07_11_2020
 
%%url https://www.facebook.com/roman.barashev/posts/2157474861051648
%%author 
%%tags 
%%title 
 
%%endhead 

\subsection{Занятие - ЖУРНАЛИСТИКА ДЛЯ ШКОЛЬНИКОВ}
\label{sec:07_11_2020.fb.roman_barashev.1.zhurnalistika}
\Purl{https://www.facebook.com/roman.barashev/posts/2157474861051648}
\Pauthor{Барашев, Роман}
\index[cities.rus]{Киев!Студия!Журналистика для школьников}

Конечно, у вас может сложиться впечатление, что на занятии ЖУРНАЛИСТИКА ДЛЯ
ШКОЛЬНИКОВ мы только тем и занимались, что песни на стихи Михаила Пляцковского
пели, но это не так.

Анастасия выступила с самопрезентацией, удивив нас распахнутой миру душой и
тем, что в свои 15 уже зарабатывает в Инстаграме, и мне надобно теперь его
освоить;

Жанна обещала познакомить со своими стихотворениями;

Артём написал и зачитал историю своей любимой песни "Моя бабушка курит трубку";

Елена написала заметку о посещении женского и мужского монастырей Киевской
области, а ее мама настоятельно рекомендовала посмотреть переворачивающий
сознание фильм "Где ты, Адам?"

Мария, к сожалению, не пришла, приболела.

Марк тоже охрип, но прислал свои извинения и рецензию на книгу "В стране
невыученных уроков", которую мы прочли всем нашим дружным коллективом.

Словом, завалили меня ребятки чтивом. Когда
редактировать-дорабатывать-публиковать? И фильм же надо посмотреть.  Но такой
это, признаюсь честно, приятный завал! Не было такого радостного завала прежде.
