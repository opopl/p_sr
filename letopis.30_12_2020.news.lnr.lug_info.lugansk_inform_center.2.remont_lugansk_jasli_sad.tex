% vim: keymap=russian-jcukenwin
%%beginhead 
 
%%file 30_12_2020.news.lnr.lug_info.lugansk_inform_center.2.remont_lugansk_jasli_sad
%%parent 30_12_2020
 
%%url http://lug-info.com/news/one/stroiteli-v-2020-godu-proveli-remont-kirovskogo-kolledzha-i-luganskogo-yasli-sada-minzhkkh-63373
 
%%author 
%%author_id lugansk_inform_center
%%author_url 
 
%%tags 
%%title Строители в рамках госпрограммы провели ремонт кировского колледжа и луганского ясли-сада
 
%%endhead 
 
\subsection{Строители в рамках госпрограммы провели ремонт кировского колледжа и луганского ясли-сада}
\label{sec:30_12_2020.news.lnr.lug_info.lugansk_inform_center.2.remont_lugansk_jasli_sad}
\Purl{http://lug-info.com/news/one/stroiteli-v-2020-godu-proveli-remont-kirovskogo-kolledzha-i-luganskogo-yasli-sada-minzhkkh-63373}
\ifcmt
	author_begin
   author_id lugansk_inform_center
	author_end
\fi

\ifcmt
  pic http://img.lug-info.com/cache/d/a/(765)_4.jpg/w620h420wm.jpg
  width 0.4
\fi

Строители Республики в рамках государственной программы "Поддержание в
надлежащем состоянии объектов жилого фонда и социальной сферы ЛНР на 2020-2022
годы" провели ремонт зданий Кировского транспортного техникума и луганского
ясли-сада № 75. Об этом на брифинге в ЛуганскИнформЦентре сообщила временно
исполняющая обязанности министра строительства и жилищно-коммунального
хозяйства ЛНР Татьяна Буряк.

"Помимо объектов многоквартирного жилого фонда, ремонтно-восстановительные
работы также осуществлялись на социально значимых объектах. В частности, это
было государственное учреждение "Кировский транспортный техникум" и луганское
дошкольное образовательное учреждение "Ясли-сад комбинированного типа № 75".
Основные работы были выполнены в 2020 году, а завершающие, отделочные,
планируется реализовать уже в 2021 году", - сказала она.

Временно исполняющая обязанности министра уточнила, что здание ясли-сада
получило значительные повреждения в результате обстрелов со стороны киевских
силовиков "и только в рамках действующей госпрограммы детское учреждение будет
полностью восстановлено".

"В процессе проведения дополнительных восстановительных работ специалисты уже
выполнили полное усиление строительных конструкций здания, капитальный ремонт
фасада, ремонт водосточной системы, замену оконных блоков, устройство отмостки
и общее благоустройство территории, включая укладку тротуарной плитки. Впереди
выполнение внутренней отделки помещений", - добавила Буряк.

Ранее глава ЛНР Леонид Пасечник сообщал, что администрации городов и районов
Республики выполнили\Furl{http://lug-info.com/news/one/administratsii-vypolnili-osnovnuyu-chast-programmy-po-kapremontu-sotsobektov-pasechnik-59830} основную часть мероприятий, запланированных
Государственной целевой программой "Поддержание в надлежащем состоянии объектов
жилого фонда и социальной сферы Луганской Народной Республики на 2020–2022
годы", по капитальному ремонту подведомственных социальных объектов.

ЛуганскИнформЦентр — 30 декабря — Луганск



