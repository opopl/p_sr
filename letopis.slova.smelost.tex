% vim: keymap=russian-jcukenwin
%%beginhead 
 
%%file slova.smelost
%%parent slova
 
%%url 
 
%%author 
%%author_id 
%%author_url 
 
%%tags 
%%title 
 
%%endhead 
\chapter{Смелость}
\label{sec:slova.smelost}

%%%cit
%%%cit_head
%%%cit_pic
%%%cit_text
Дмитро Кулеба заявив, шо Україна не буде визнавати домовленості між Байденом і
Путіним, якщо вони стосуватимуться України, але будуть укладені за спиною в
України.  А що, \emph{сміливо}. Нагадує одеський анекдот про
чоловіка-підкаблучника, який вирішив одного разу підняти бунт проти сімейного
диктату дружини. Він прийшов і з порогу тоненьким тремтячим фальцетом заявив: -
Все!!! Мені це набридло! Зараз я піду і нап‘юся з друзями! А потім піду на
стриптиз! А потім познайомлюся з молодими симпатичними дівчатами!!! І знаєш, де
я буду ночувати???  Дружина спокійно відповіла: - Знаю. В морзі
%%%cit_comment
%%%cit_title
\citTitle{Украина ведет себя, как подкаблучник из одесского анекдота}, 
Константин Бондаренко, strana.ua, 17.06.2021
%%%endcit

%%%cit
%%%cit_head
%%%cit_pic
%%%cit_text
И это еще - в лучшем случае.  Президент Украины Зеленский спросил.  Кандидаты в
канцлеры ФРГ ему ответили.  Команда Зе будет долго еще объяснять свою \emph{смелость}
и теле-видео-субъектность. Политики - упражняться в нато/ненато. Эксперты -
жевать давно пережеванное.  А нам с этим жить дальше. C этими очевидными
ответами. C этой президентской безответственностью. C этой жвачкой.  И, судя по
рефлексам и реакциям \enquote{субъектных игроков} с улицы Банковой, закончится такая
игра в границах \enquote{люблинского треугольника}, в размерах Малой Украины. В лучшем
случае.  Ex nihilo nihil fit
%%%cit_comment
%%%cit_title
\citTitle{Большая геополитическая игра Зеленского закончится в размерах Малой Украины / Лента соцсетей / Страна}, 
Андрей Ермолаев, strana.ua, 28.06.2021
%%%endcit

