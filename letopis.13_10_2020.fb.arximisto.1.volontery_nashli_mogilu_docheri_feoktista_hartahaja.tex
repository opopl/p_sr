%%beginhead 
 
%%file 13_10_2020.fb.arximisto.1.volontery_nashli_mogilu_docheri_feoktista_hartahaja
%%parent 13_10_2020
 
%%url https://www.facebook.com/arximisto/posts/pfbid02Y5s7EfTPMsUBhfYY9JrnpmLiLmge62R2VXE3n6tj5itMFeRgU41PzMp8WNgHjjyUl
 
%%author_id arximisto
%%date 13_10_2020
 
%%tags 
%%title Волонтеры нашли в Некрополе могилу дочери Феоктиста Хартахая
 
%%endhead 

\subsection{Волонтеры нашли в Некрополе могилу дочери Феоктиста Хартахая}
\label{sec:13_10_2020.fb.arximisto.1.volontery_nashli_mogilu_docheri_feoktista_hartahaja}

\Purl{https://www.facebook.com/arximisto/posts/pfbid02Y5s7EfTPMsUBhfYY9JrnpmLiLmge62R2VXE3n6tj5itMFeRgU41PzMp8WNgHjjyUl}
\ifcmt
 author_begin
   author_id arximisto
 author_end
\fi

Волонтеры нашли в Некрополе могилу дочери Феоктиста Хартахая

Участники волонтерского проекта по восстановлению Мариупольского Некрополя в
воскресенье обнаружили могилу Анны Феоктистовны Рудевич, дочери Феоктиста
Хартахая, основателя первых гимназий в Мариуполе, первого историка и
просветителя приазовских греков.

\#новости\_архи\_города

Могила находится в советской части Некрополя – недалеко от юго-западных ворот и
стены бывшего завода \enquote{Электробытприбор}. В одной ограде похоронены Анна
Феоктистовна Рудевич (1876-1951), ее муж Владимир Викторович Рудевич
(1870-1930), его сестра Ольга Викторовна Рудевич (1877-1964) и их родственница
Анна Тихоновна Херсонская (1917-1971).

У Феоктиста Хартахая и Софии Ковальской было трое детей: сын Феоктист (1880 -
?) и две дочки Анна и Екатерина (1878 - ?), по данным историков-краеведов
Степана Калоерова, Льва Яруцкого и Сергея Бурова. О судьбе сына Феоктиста
ничего неизвестно.

Екатерина Хартахай вышла замуж за Леонида Попова, студента Харьковского
университета. У них был сын Валентин (1901 - ?) и дочь Елена (1898 – ?). Их
судьба неизвестна.

Потомки Анны Хартахай и Владимира Рудевича – самая изученная ветвь рода
Феоктиста Хартахая. Супруги преподавали в гимназиях, которые он основал. Сын
Виктор умер в годы большевистской революции. У Сергея (1901-1968) и Ольги
Рудевич (1899-1990; в замужестве Кафери) были дети, они пережили все бедствия
XX века.

Рудевичи жили в Мариуполе по адресу улица Итальянская, 91, вместе с Ольгой
Викторовной Рудевич. Лев Яруцкий, выдающийся мариупольский краевед, успел
пообщаться с Ольгой в уже далеком 1962 году%
\footnote{%
Гимназия: Основание, Лев Яруцкий, Старый Мариуполь, 22.07.2011% 
, \par\url{http://old-mariupol.com.ua/gimnaziya-osnovanie}%
, \par\url{http://old-mariupol.com/gimnaziya-osnovanie}%
, \par Internet Archive: \url{https://archive.org/details/22_07_2011.lev_jaruckij.old_mariupol.gimnazia_osnovanie}%
}

Находка могилы Рудевичей стала возможной благодаря подсказке Раисы Петровны
Божко, заместителя директора Мариупольского краеведческого музея, как сообщил
Андрей Марусов, директор ГО \enquote{Архи-Город}. Чтобы найти могилу самого Феоктиста
Хартахая, мы стали искать его потомков и обратились к краеведам, историкам и
старожилам. 

По словам Раисы Божко, она общалась с потомками Рудевичей более десяти лет
назад. Они упоминали, что могила Анны Хартахай и Владимира Рудевича находится
где-то у стен завода. Поэтому на выходных волонтеры Александр Шпотаковский,
Дейниченко Елена и Геннадий тщательно обследовали именно эту территорию. К
счастью, работники Комбината коммунальных предприятия уже очистили ее
значительную часть от зарослей...

Мы обращаемся ко всем с просьбой помочь в поиске потомков Феоктиста Хартахая и
Рудевичей. У них также могут быть фамилии Кафери, Херсонские, Худолий, Поповы,
Шифановы... Возможно, они обладают информацией, которая поможет найти могилу
Феоктиста Хартахая и выяснить судьбу всех его потомков...  

Контакты: arximisto@gmail.com, +38096 463 69 88.

Фото Анны Феоктистовны Рудевич (Хартахай) взято c сайта azovgreeks.com.

\#люди\_мариупольского\_некрополя
