% vim: keymap=russian-jcukenwin
%%beginhead 
 
%%file 22_04_2022.fb.kolyhaev_igor.herson.mer.1.herson
%%parent 22_04_2022
 
%%url https://www.facebook.com/kolykhaev.igor/posts/5133070176786375
 
%%author_id kolyhaev_igor.herson.mer
%%date 
 
%%tags 
%%title Итак, ситуация в Херсоне на сегодняшний день выглядит так
 
%%endhead 
 
\subsection{Итак, ситуация в Херсоне на сегодняшний день выглядит так}
\label{sec:22_04_2022.fb.kolyhaev_igor.herson.mer.1.herson}
 
\Purl{https://www.facebook.com/kolykhaev.igor/posts/5133070176786375}
\ifcmt
 author_begin
   author_id kolyhaev_igor.herson.mer
 author_end
\fi

Говоря вам правду, которая порой может и не нравиться, я хочу одного: чтобы вы
трезво оценивали ситуацию и четко понимали, что происходит вокруг. Потому что
обладая только ДОСТОВЕРНОЙ информацией, можно принимать правильные решения. 

\ii{22_04_2022.fb.kolyhaev_igor.herson.mer.1.herson.pic.1}

Итак, ситуация в Херсоне на сегодняшний день выглядит так.

1. Зеленых коридоров для эвакуации всё так же нет. Мы просим об этом с первого
дня оккупации. Но, поскольку положительного ответа нам по-прежнему не дали, я
не могу гарантировать безопасность для тех, кто выезжает из города на свой
страх и риск.

2. Из-за военных действий город сильно «просел» финансово по доходам в бюджет.
Мы недополучили налог на доходы физлиц в размере 60,5 млн грн, акцизного сбора
– 10,0 млн грн, платы за землю – 9,8 млн грн, налога на недвижимое имущество –
10,7 млн грн, единого налога – 20 млн грн, других налогов и сборов – 11,5 млн
грн. При этом у города есть фактические расходы почти 138 млн грн. – на
зарплату (95 млн грн), оплату коммунальных услуг и энергоносителей (9,6 млн),
траты на содержание КП, обеспечивающих жизнедеятельность нашего города, – почти
20 млн грн. и т.д.

Поэтому мы обратились к Премьер-министру Украины Денису Шмыгалю с просьбой
рассмотреть возможность дополнительной дотации для Херсона, чтобы перекрыть
дефицит. Город должен жить и работать. Несмотря на глухую оккупацию, мы обязаны
выжить и обеспечивать жизнедеятельность Херсона. Это сейчас – задача номер
один.

3. Из города уехали больше 100 учителей и свыше 3000 детей, но большая часть
всё равно осталась с нами. Онлайн обучение продолжается. Хочу обратиться к
ребятам-школьникам. Пожалуйста, помогите нам, взрослым. Мы все сейчас настолько
заняты глобальными задачами, что не хватает времени быть для вас примерами
идеальных родителей. Очень вас прошу: занимайтесь самообразованием – читайте,
смотрите правильные фильмы исторического, документального, научного содержания.
Сегодня это – ваш фронт, и ваш вклад в общую Победу. Нашей стране необходимо
грамотное, образованное поколение будущего, которое будет помогать отстраивать
Украину и наши города! С вашей помощью Херсон расцветёт. 

4. Не спрашивайте о подробностях резонансных событий, деталей которых у меня
нет и быть не может. Я так же, как и вы, жду комментариев от правоохранительных
органов. Воздерживаюсь от сплетен и вам рекомендую.

Никогда в жизни мы ещё так не нуждались друг в друге, как в эти дни. И ещё
никогда мы не были так близки. Я верю вам, горжусь вами и переживаю за каждого.
Мы все, действительно, стали одной большой семьёй в это тяжелое время. 

Скажите, что ещё у нас может быть дороже, чем наши семьи, родная земля и люди,
живущие на ней!
