% vim: keymap=russian-jcukenwin
%%beginhead 
 
%%file 13_12_2021.stz.news.ua.vovka.1.tochka_nolj
%%parent 13_12_2021
 
%%url https://vovka.com.ua/?p=10285
 
%%author_id blog_obo_vsem_i_niachem
%%date 
 
%%tags 
%%title Точка ноль
 
%%endhead 
\subsection{Точка ноль}
\label{sec:13_12_2021.stz.news.ua.vovka.1.tochka_nolj}

\Purl{https://vovka.com.ua/?p=10285}

\ifcmt
 author_begin
   author_id blog_obo_vsem_i_niachem
 author_end
\fi

\begin{multicols}{2} % {
\setlength{\parindent}{0pt}

Этот пост из серии «Заметки на полях роликов», однако из-за особой важности
выходит под персональным названием.

Владимир Александрович в пятницу вывалил на Ютьюбчике кагбэ отвлекающий видос,
про какую-то левую хрень типа жила-была страна «Киевская Русь — Украина». Форма
правления — демократия через вече, на котором женщины князей выбирали, а потом
отзывали если чо там им не нравилось.

Короче малопонятная лютая дичь, в которой даже разбираться лень.

А сам тем временем пригласил к себе одну журношлистку с одного савсем жэ не
олигархического телеканала и надавал ей интервью.

И вот в этом интервью стало уже совсем не смешно.

Само интервью вышло каким-то скомканным шоле. Сидит человек, пытается родить
умную мысль, и тутраз, и просто перескакивает на отвлеченную тему. И так 22
минуты с хвостиком.

Но в нем прозвучали две вещи, которые кагбэ намекают, что Украина опять
приближается к точке ноль.

Для тех, кто не знает шо это, кратенько объясню.

Точка ноль, это звено в цепи событий, которое обрушает привычное мироустройство
и далее жизнь идет по совершенно новому сценарию. Характерной особенностью
точки ноль является то, что нет возможности вернуться обратно. Мало того, чем
больше будут прилагать усилий по возвращению, тем хуже будет результат в
последствии.

В этом ее главное отличие от точки невозврата. Они похожи, но это не тождества.
Точка невозврата не предопределяет будущее, она перечеркивает прошлое. Точка
ноль ставит четкий коридор, по которому уже нет развилок, вправо, влево, вбок.

Чтобы долго не размазывать философскую кашу по тарелке, наглядный пример — это
супружеская измена.

Жила семья, все было хорошо.

Затем кто-то сходил налево. Это была точка невозврата, но не точка ноль. Точка
невозврата, потому как это только вопрос времени, когда эта измена повторится
вновь, ну и ясно, шо это уже не семейная жизнь, а форменное лядство. Но кагбэ
выбор еще есть, не большой но все же.

А вот точка ноль — это когда вторая половина узнает об этом факте.

Жить как раньше не получится. Можно конечно с покер-фейсом делать вид, что
ничего не произошло, но результат в таком случае будет еще разрушительнее.

Возвращаясь к интервью, замечу что оба события болтаются вокруг референдумов.

Причем один из них никакой хрен не проводил, а следовало бы.

Вторым жэ Владимир Александрович с загадочным лицом пугал, и так не сильно
отчехляющую по жизни тетку, а тут еще такое. Так шо она пропустила это мимо
ушей, и вот сильно зря.

Могла бы оказать услугу стране и выспросить, а шо конкретно он имел в виду. Ибо
я не уверен, шо понял его правильно, но если таки да, то проводить такой
референдум категорически нельзя, по моему ошибочному мнению.

Но начнем с менее важного.

Если кто не следит за ходом выяснения кто кому Рабинович, между Вашингтонским
обкомом и башнями Кремля, так я могу ошибочно предположить, шо одна из тем
видеоконференции касалась пресловутого «Расширения НАТО на Восток».

Опять жэ, для тех кто не знает шо это.

Есть версия, что когда-то, еще при развале совка, в НАТО пообещали, правда
якобы на словах, не расширяться вплотную к границам русских. Но кагбэ это
обещание, если оно вообще было, старательно игнорируется, шо очевидно,
например, по троебалтам.

Было оно, не было, установить истину не представляется возможным. Горбачев в
привычной для себя манере переобувается на лету, то яростно подтверждая, то
яростно жэ опровергая сам факт таких договоренностей. В НАТО отрицают, русские
утверждают.

Все как всегда, в принципе.

Нутакот требование русских «письменно» зарубить вступление Украины и Грузии в
Альянс с моей точки зрения тупо дипломатическая уловка.

В НАТО уже подустали повторять, шо они в гробу видали Украину с ее проблемами.

И дело не в противостоянии с русскими. Дело в самой Украине.

Обозначенных Альянсом 2\% бюджета от ВВП, выделяемых на оборону и так не все
страны-участницы вытягивают. Греция, Эстония, США, Франция, Польша, Турция и
Румыния, вроде, но наверное кого-то не вспомнил. А их там 28 стран, если чо.

И в этих странах (кроме Турции) не фантики, а вполне себе деньги, по отношению
к которым украинская гривна ну кагбэ в пролете.

Даже сейчас, выпрыгивая из трусов и сдавая в долговое рабство еще не рожденных
украинцев, бюджет Украины на оборону чутка больше, чем у Бельгии. Зато площадь
Бельгии, которую надо оборонять, это шота там 5\% от площади Украины.

Вопчем понятная математика. Это даже без учета коррупции.

Такое счастье даром никому не надобно на содержании, даже если бы не было
никакой агрессии и аннексии.

Именно поэтому, как мне показалось, «письмо трех» в 2008 году было встречено ну
скажем так с прохладой со стороны Франции и Германии.

Ну и в результате этих предъяв получается дипломатическая растяжечка.

Письменно пообещать не жениться — это плевок в сторону невесты, которую уже
отматросили, но заявление в ЗАГС еще не подавали.

Отсутствие такого письменного обещания — это слив диалога в тупик, ну типа: «О
чем нам говорить, если вы свои жэ слова не можете на бумаге изложить и
расписаться?»

По словам Владимира Александровича сонный дедуля утверждает, шо вступление в
НАТО это исключительно воля украинского народа.

А вот как раз народ этот никто и не спросил.

Референдума не было, хотя вон в тех странах, которые уже в НАТЕ, референдумы
проходили. И в некоторых по нескольку раз, если чо.

В (на) Украине этот момент решили пропустить в виду неоднозначности
волеизъявления. А че их спрашивать, этих людишек? Че они там ваще понимают?

Раися шо? Атакуэ? Атакуэ! Значит нада в НАТУ, она нас защитит.

Поменяли Конституцию и готово. Встречай нас Альянс, мы уже идем.

Так вас же не звали! Да пофигу, мы уже настроились.

И плевать, шо по Конституции источником власти в (на) Украине есть народ.
Единственным кстати источником, других не обозначено.

Но это еще не самый цимес.

Как я уже писал, Владимир Александрович вращая глазами затулил, шо он «не
исключает референдума относительно Донбасса в целом».

И вот тут я припух конкретно. Че ща началось, то?

Дело в том, что когда-то чуть ли не единственный умный человек среди этих
селюков и клептоманов Вячеслав Черновол осторожно высказывался за перекроение
устройства страны. Очень осторожно и он в частности указывал, что пока еще не
время.

Суть его идеи сводилась к замене территориально-административного устройства с
унитарной формы на федеративную.

Кому интересно можете вот тут ознакомиться:

\href{https://www.youtube.com/watch?v=yhx3hXitop0}{%
Вячеслав Чорновол о федерализации Украины, История новейшего времени, youtube, 06.02.2017%
}

\end{multicols} % }

\ifcmt
  tab_begin cols=3,no_fig,center

     pic https://i2.paste.pics/25e070e33c10025186a748ca0115b907.png
		 pic https://i2.paste.pics/cb9c30274ebc3e37650cbee533f95b09.png
		 pic https://i2.paste.pics/73e6f7eb5d9ea427f180842bcbc1c421.png

  tab_end
\fi

\begin{multicols}{2} % {
\setlength{\parindent}{0pt}

Такой референдум чисто теоретически имеет право быть. Если назрела ситуация в
стране, если есть запрос у людей, если есть смелость у власти.

Но чтобы жить как Швейцария, нужно быть Швейцарией. Это там по любому чиху
пилят референдум. И страна представляет собой кантоны, в которых ВНЖ для
иностранцев из одного кантона не гарантирует получение такого же ВНЖ в другом в
автоматическом режиме путем замены карточки.

Короче Украина не Швейцария и никогда ею не будет, поэтому все эти разговоры в
пользу бедных. Они ушли вместе с Черноволом, и как мне кажется, его именно за
это и грохнули, хотя я могу и ошибаться.

А референдум по отдельно взятому Донбассу это ваще шо? Все очень напоминает
сепаратизм. Только уже со стороны человека, занимающего пост президента.

И если только поставят так вопрос, то все. Туши свет, сливай воду. Это будет
очередная точка ноль. Даже если только заикнуться.

На сейчас жители неподконтрольных территорий демонизированы и загнаны под
плинтус.

Ну типа вот они сами козлы виноваты, Путина звали, флагами махали. И ваще они
не украинцы.

Ну как не украинцы? Паспорта украинские? Значит украинцы.

Флагами махали и Путина звали отдельные Иванов, Петров и Сидоров, а не все
поголовно жители Донецкой и Луганской областей.

И вот эти вот товарищи, если следствие докажет, а суд вынесет решение, могут
называться сепаратистами, с последующей реакцией.

А не все население, просто потому что они тугие, кривые, косые или как там их
еще называют.

Никто не обязан любить свое государство. Это нигде не прописано. Нет в
Конституции статьи «Каждый гражданин обязан любить вышиванку, Бандеру и
Украину, патаму шо это и есть Украина».

Это личное дело отдельно взятого гражданина.

А вот государство наоборот обязано делать все, что написано в Конституции для
всех граждан Украины, без исключения. За это наемные менеджеры, вот эти все
чинуши, депутаты, менты, судьи и прочая шваль, получают зряплату, которую ему
платит народ Украины.

Да, да, именно народ скидываясь в бюджет. Эти деньги не с неба туда падают.
Даже кредиты, которые все равно придется отдавать тому же народу. Их не спишут,
с большой долей вероятности.

Так не работает «вон те, какие-то не такие, давайте мы их прореферендим».Через
очень короткий промежуток времени есть неиллюзорный шанс получить еще 100500
таких же «не таких» граждан.

Почему я говорю именно про очередную точку ноль? Потому что одна уже была,
только на нее не обращают внимания. Причем сознательно игнорируя.

Ща проиллюстрирую.

19 февраля 2014 года была захвачена воинская часть на территории города Львова.
Мало того, были захвачены здания СБУ и МВД.

Вот пруф.

\end{multicols} % }

\href{https://www.youtube.com/watch?v=GQ0Lmm-mTmU}{%
Погромы во Львове: захватили ОГА, военныую часть, МВД, СБУ и - Чрезвычайные новости, 19.02, %
Надзвичайні новини. ICTV, youtube, 19.02.2021
}


\ifcmt
  tab_begin cols=3,no_fig,center

     pic https://i2.paste.pics/7634810a392dd878e5b02671bd4ec9da.png
		 pic https://i2.paste.pics/c14bcfe6e97fa53a420a00c4dde73137.png
		 pic https://i2.paste.pics/12908dd5ad211b07e2bfdd8a60a105d3.png

  tab_end

  tab_begin cols=4,no_fig,center

		 pic https://i2.paste.pics/e0da648d4d784880d59bdbcb39273fed.png
		 pic https://i2.paste.pics/81f70ab6796159cbdb0ea24d3aa07527.png
		 pic https://i2.paste.pics/eaf37c8873c1c8ed53d9e7632d31768d.png
		 pic https://i2.paste.pics/0a96ebba945e8673b50214794a22c771.png

  tab_end
\fi

\begin{multicols}{2} % {
\setlength{\parindent}{0pt}\em
\headTwo{Комментарии к видео}

\iusr{Кирилл Егоров}

Всем Колхозникам привет :)
Как сказал Руслан: 
-А вот и эти тропки, откуда идут те самые ножки... 

\iusr{Bra-za Bro}

Ага, \enquote{колхозник}- это \enquote{дело-то добровольное}@igg{fbicon.hand.victory} вот и я забрел,
захотелось \enquote{освежить память} )  \enquote{ностальгирую по-тихой}))) 

\iusr{Микола Киевский}

С чего всё началось и кто подал пример можно посмотреть по дате публикации.
Далее события развивались как снежный ком, с Запада на Юго-Восток. Но одних
назвали активистами или патриотами, а других терористами, хотя разница эта как
между  шпионом и разведчиком - зависит относительно какой стороны брать. Одним
помогает РФ, другим США... Главное отличие в том, против кого были применены
ВСУ, а против кого нет. 

\begin{itemize} % {
\iusr{валерий сегун}
Согласен на все 100, привет из Донецка
\iusr{Bra-za Bro}
 @валерий сегун  Привет и Уважуха Донбассу из Риги! @igg{fbicon.hand.victory}
\iusr{валерий сегун}
 @Bra-za Bro  Риги Привет из Донбасса 

\iusr{Олег Ковальчук}

Потому что двойные стандарты. Захваты ОГА и силовых структур происходили не
только во Львове. В середине февраля 2014 года тоже самое  происходило в
Ивано-Франковске, Тернополе и др. городах западной украины. А 22 февраля 2014
года на украине произошел насильственный захват власти. Верховная рада, нарушив
достигнутые договоренности между президентом и лидерами оппозиции, изменила
конституцию, сменила руководство парламента и МВД. А на следующий день
исполняющим обязанности президента назначили Турчинова. Массовые акции протеста
начались на юго-востоке Украины в конце февраля. И только 12 апреля в Донецке
были заняты админздания. Затем кровавый пастор начал АТО на востоке украины. А
это видео \enquote{мирная} и \enquote{законопослушная} украина должна показывать своим
гражданам, чтобы помнили где все начиналось. А еще в УК есть Статья 341. Захват
государственных или общественных зданий или сооружений, которая не отменялась.
И не было на западе украины по этому поводу никакого АТО. Вот такая вот
неудобная и горькая правда. 

\iusr{liiisi}
Во Львове вешали польские флаги? Вот и вся разница

\iusr{Irina M.}

@liiisi  гос. переворот карается по  уголовной статье, а сепаратизм может даже
рассматриваться через референдум, а потому преступлением не является. Тем
более, в условиях переворота.

\iusr{Maikl Тритон}
@liiisi  В Киеве вешали флаги Евросоюза. А в Крыму униженные войсковые
украинские кричали \enquote{Америка с нами!}

\iusr{Александр Бусель}

@liiisi  а в Киеве не было флагов Евросоюза  @igg{fbicon.face.tears.of.joy} ?
На некоторых видео мелькали даже американские.

\iusr{Evgeniy Sidorenko}

@Александр Бусель  в Киеве на майдане мелькали и российские флаги. И что с
этого? Твоя теория вижу резко потеряла смысл да?) разрыв шаблона. 

\iusr{Evgeniy Sidorenko}

С одной небольшой разницей дружок. Те кто захватывал в западной части Украины,
против них открыли уголовные дела во первых, а во вторых они не кричали за
отделение и разделение страны. В отличии от даунбаса.

\iusr{Evgeniy Sidorenko}

@валерий сегун  те что захватывали в западной части Украины, они преступники в
том что делали беспредел. А вы преступники в том что начали отделять
территорию Украины, это коллаборационизм. И когда ты смотришь на них, то ты
ещё больше преступник чем они. Так как их преступление только в том что они
захватывали обл здания с требованиями вывести беркут с Киева, а вы звали
Россию к нам с войной. Помните об этом всегда, предатели. 

\iusr{валерий сегун}

@Evgeniy Sidorenko  Я сам из Славянска и видел как это все у нас происходило, а
Россию стали звать на помощь тогда когда армия и правый сектор начали стрелять
по нам и применять все свое вооружение против мирных людей, так что не был ты у
нас и не знаешь как было на самом деле

\iusr{Evgeniy Sidorenko}

@валерий сегун  ой. Как здорово что ты мне попался. Ану давай, конкретно, а
какого числа стали стрелять по тебе в славянске? Назови число и год

\iusr{Evgeniy Sidorenko}
 @валерий сегун  сейчас я тебя разносить буду лжепропагандист

\iusr{Evgeniy Sidorenko}

@валерий сегун  ты главное не сливайся. Какого числа, как ты говоришь армия и
мистический правый сектор начали стрелять по тебе? 

\iusr{Evgeniy Sidorenko}

@валерий сегун  что ты там молчишь российский бот? Какого числа по тебе начали
стрелять? И какого числа был проведён так называемый фейковый референдум за
разделение страны? И какого числа группа Гиркина захватила Славянск? Жду
внимательно ответов. Не сливайся

\iusr{Maikl Тритон}

@Evgeniy Sidorenko  Народ свободного Донбасса не захотел жить с теми, кто
творил беспредел, в одном государстве и ушел. А Вы всё про ТЕРРИТОРИЮ. О ЛЮДЯХ
НАДО ДУМАТЬ!

\iusr{Evgeniy Sidorenko}

@Maikl Тритон  такого нет понятия народ Донбасса, не несите чушь. Это граждане
Украины. А попытки отделения страны, призыв другой страны и фейковые
референдумы это уголовная статья в любой стране за сепаратизм. За это судят
как в Украине, так и в России и в любой стране. Когда в России хотели создать
УНР, уральскую народную республику, там всех посадили и закрыли. Почему не
дали этого сделать? Ещё такой вопрос. Съезд, так называемый силлитер, ещё
провели на Донбассе в 2006 году за создание ДНР, с флагами и так далее.
Внимание вопрос. 2006 год, какие события или как вы говорите беспредела были
тогда? Ведь уже в 2006 году, местные коллаборанты проводили подобные съезды за
отделение. И как быть тем людям, которые являются жителями Донбасса и они за
Украину, но были вынуждены выехать из-за войны? Или остаются там. Что с ними
делать?? 

\iusr{валерий сегун}

@Evgeniy Sidorenko скажи дату когда готовились добровольцы, Гиркин зашел 14
апреля а стрелять начали в первых числах

\iusr{валерий сегун}
 @Evgeniy Sidorenko  Гиркин зашел 14 апреля стрелять начали в первых числах
\iusr{Evgeniy Sidorenko}

@валерий сегун  внимание. Словил бота на лжи. Гиркин захватил с оружием
славянск 12 апреля. В это же день были проведены фейковые  референдумы за
незаконное отделение от страны. Боевых действий ещё никаких не было и эти
референдумы провели без проблем. Также был убит местный депутат Александр
Рыбак. После того как гиркин с оружием захватил Славянск, начались боевые
действия. После того как освободили Славянск, там больше не стреляют. Как и в
Северодонецке

\iusr{Evgeniy Sidorenko}

@валерий сегун  Гиркин захватил Славянск 12 апреля. Это он лично рассказал в
интервью. Также в интервью он сказал что захват планировался ещё до
референдумов, сразу после Крыма. 

\iusr{Evgeniy Sidorenko}

@валерий сегун  в данном интервью, пропагандистскому каналу RT Россия, в
программе Антонимы, Гиркин утверждает что именно он начал войну на востоке
Украины. Так же Гиркин утверждает что именно он получал приказы от советника
Путина Суркова. В данном интервью Гиркин говорит о том что никто не хотел
воевать, и если бы не он, горячей фазы войны не наступило бы, и митинги
закончились бы без агресии на Донбассе. Также Гиркин утверждает что операция
по захвату Донбасса готовилась за долго любых боевых действий и референдума.
Наслаждайся лживый бот из Славянска

\url{https://youtu.be/lwRU8liw_Io} 

\iusr{Evgeniy Sidorenko}

@Maikl Тритон  странно, а вот Гиркин в программе Антонимы, каналу RT Россия,
Маргариты симонян, говорит обратное. Что именно он начал войну на Донбассе.
Именно он получил приказы от советника Путина Суркова, по захвату Донбасса.
Также Гиркин утверждает что если бы не его группа, которая захватила Славянск
и начала войну, после чего они захватили Донецк, война бы не началась. Гиркин
утверждает что -«митинги на Донбассе закончились бы ничем» так сказать
беззубые митинги просто бы закончились. Это прямая речь Гиркина. Человека
который, как он сказал в интервью, планировал захват Донбасса. Наслаждайся 

\url{https://youtu.be/lwRU8liw_Io} 

\iusr{Александр Бусель}

@Evgeniy Sidorenko  кому ты эти сказки рассказываешь, как нах уголовные дела,
у убийцы на свободе, до сих пор не расследовали убийства на майдане в Одессе.

\iusr{Evgeniy Sidorenko}

@Александр Бусель  о, дружок, давай по поводу убийтсв. И так, кто был
начальник Одесской милиции в 14 году, когда произошли эти события? Это был
Фучеджи. На кадрах столкновений в Одессе он идёт рядом с пророссийскими
митингующими, которые вооружены палками и щитами. Теперь внимание вопрос. Что
бы Украине полноценно расследовать это дело, они просят Россию выдать Фучеджи,
который был начальником одесской милиции в то время и который не остановил
этот беспредел. Сейчас он сбежал в Россию и скрывется там. Почему Россия его
не выдаёт для расследование, балабол ты. Второй момент. Начальник главного
управления одесского областного управления МЧС Украины, который должен
отвечать за тушение пожара, а мы знаем что пожарные намерено не приехали во
время тушить это дом, учитывая что они были в 10 минутах езды, сбежал в Крым и
стал там заместителем главы. Экс-начальник главного управления одесского
областного управления МЧС Украины Владимир Боделан, занимавший эту должность
во время трагедии 2 мая в местном Доме профсоюзов, стал заместителем главы
Симферопольского района Крыма. Почему Россия прячет у себя преступников и не
выдаёт их доя того что бы Украина могла расследовать дело по 2 мая. Главный
милиционер Одессы и главный начальник МЧС, сбежали и Россия их не выдаёт.
Вопрос - почему? Отвечай дикая обезьянка.. 

\iusr{Александр Бусель}

@Evgeniy Sidorenko  да ты шо, а воинская часть, отделения милиции, это как
понимать? Так вот дружок, никакого разрыва шаблона у меня нет. Ты настолько
туп, что видишь следствие, а не видишь причины. Если бы волнения со стороны
западной Украины и разагнали бы майдан, был бы на месте Крым, Доннбасс, не
было бы небесной сотни, трагедии в Одессе, Мариуполе. Не было бы войны и 12-13
тысяч погибших, ещё больше раненых, не было бы миллионов исковерканых судеб.
Не закрылось бы столько предприятий, гораздо меньше гастров, а сейчас дети без
родителей, папа в Польше, мама в Италии, для тебя это нормально? Не подняли бы
настолько тарифы, не поднялись бы так цены на продукты. Кем сейчас стала
Украина, она не то что промышленностью может похвастаться, а даже, как
аграрная в жопе, вы сейчас курицу и свинину закупаете в стране агрессоре,
другую сельхоз продукцию, пенсионеры вымирают от холода и голода, коммуналка
выше пенсий, треть рабочего населения гастры. Вместо одного вора посадили
другого, про клоуна вообще молчу, по городам бегают радикалы и кошмарят народ.
У вас сейчас свобода, только цензура хуже, чем при СССР, запрещают всё, что
как-то связано с РФ, если у тебя иная точка зрения, ты сразу сепар и
кремлебот, без всяких референдумов открыли рынок земли, в Карпатах уже с
космоса видны проплешины. Вот такие пироги Вася.

\iusr{Александр Бусель}

@Evgeniy Sidorenko  Мариуполь 9-е мая, и не нужно рассказывать про мифическое
ДРГ, которое в количестве трёх человек захватило целый райотдел милиции, есть
видео с мобильных телефонов, а не ваша сказка на укр ТВ. 

\iusr{Александр Бусель}
 @Evgeniy Sidorenko  в Мариуполе эта босота была 9-го мая
\iusr{Александр Бусель}
 @Evgeniy Sidorenko  

а что же они восемь лет ждали  @igg{fbicon.face.tears.of.joy} ? А не конституционный переворот это законно,
захватывать административные здания, отделы милиции, воинскую часть, ведь это
не только во Львове было, это законно. Давай, разбивай меня в пух и прах.

\iusr{Александр Бусель}

@Evgeniy Sidorenko  а вот тебе про Мариуполь и мифическое ДРГ, только это не
сказка вашего ТВ, а снято на мобильные телефоны обычных граждан.
\url{https://youtu.be/Pmp0AaItuwU}

\iusr{Александр Бусель}

@Evgeniy Sidorenko  тобишь без его персоны невозможно раследование, это если
преступник не пойман, то не возможно следствие, ох ты и отмазку нашёл, где
следствие по майданной сотне, почему ваш политик приезжал и забирал
снайперское оружие, ведь было это видео в ютубе. Честно забыл фамилию. 

\iusr{Александр Бусель}
@Evgeniy Sidorenko  ахаха, ещё и комментарии удалили  @igg{fbicon.face.tears.of.joy} .

\iusr{Александр Бусель}
@Evgeniy Sidorenko  пробую ещё раз про Львов \url{https://youtu.be/GQ0Lmm-mTmU}

\iusr{Александр Бусель}
@Evgeniy Sidorenko  притих, ну давай аргументы, судя по аве, тебе тогда лет 10-12-ть было. 

\iusr{Evgeniy Sidorenko}

@Александр Бусель  легко я тебя сейчас развалю. Смешно читать твой бред.
Распятого мальчика среди этих «трёх» человек в мариуполе не было? 

И так поехали. Я так и не увидел ответа по поводу 2 мая, клоун. Ты смотрю
любитель задавать свои вопросы по шаблону, а отвечать когда ты будешь. Я тебе
повторю. Почему начальник одесской милиции Фучеджи и начальник МЧС одесской
области Боделан свалили в Россию? Эти люди ответственны за события второго мая.
Фучеджи как начальник милиции, который не остановил беспредел и начальник
пожарной части, которые не приехали тушить пожар. Давай клоун, ответь мне. Ты
же хочешь суда по 2 мая? Почему Россия не выдаёт их для расследования события,
что бы они дали показания? Ответ будет клоун или нет?

Идём дальше. То что ты переходишь уже на оскорбления, это очень хорошо,  ты уже
там в панике. Продолжим теперь за войну. Военная операция и АТО началось 2014
года, 12 апреля. С захвата Гиркиным Славянска. Ответь мне на вопрос. Когда ты
кричишь за тех митингующих что были на майдане, и тех кто в западной части
захватывал местные ОГА, они выдвигали требования за отделение территории? Они
преступники в том что захватывали обл администрации. А вы преступники в том,
что мало того что вы захватывали обл администрации, вы ещё и начали проводить
фейковые референдумы у себя. Тебе напомнить когда в апреле были проведены
референдумы за отделение страны? 6 апреля, вы провели этот фейковый референдум.
Знаешь как во всех странах называют тех, кто выступает за отделение частей
страны и присоедините к другой? Сепаратизм. В любой стране, сепаратизм это
уголовная статья. Даже в России. Почему Россия подавила создание УНР, Уральской
народной республики и всех посадили? Вопрос не задавал? Так чем же вы лучше? По
закону Украины и любой страны вы (те кто был активным участником за отделение
страны)  сепаратисты. Идём дальше. До того как Гиркин начал войну на Донбассе
12 апреля, никаких военных действий не было. А были беззубые протесты. В
интервью RT Россия, в программе Антонимы, Маргариты Симонян, Гиркин заявляет
что он собирал бойцов в Ростове, пересекал границу с оружием и ехал захватывать
Донбасс. Это лично говорит Гиркин. Также Гиркин утверждает что получал команды
от секретаря Путина Суркова. Что ответишь за это? Украина начала войну? Или
местные коллаборанты с россиянами? А как же силлитер в 2006 году на Донбассе,
могу тебе видео скинуть прям тут в ютубе, как был съезд местных сепаратистов с
флагами ДНР. Это ещё раз повторю был 2006 год. Тогда были майданы и захваты на
западе Украины? Или ты это клоун не знаешь? По поводу экономик не смеши меня
тут вообще. Посмотри какая минимальная зарплата в Украине и какая минимальная
зарплата в России , и ты сильно удивишься что уже в Украине минималка выше чем
в России. С учетом войны против России. О свободе слова даже смешно читать твой
бред. Берёшь своими кривыми руками пишешь в Гугл, рейтинг свободы слова в
странах. И посмотри на какому месте Украина, и на каком месте Россия.
Подсказка, Россия ниже Зимбабве. Страна, в которой сажают даже за репост
картинок. Как так? Также Украина в мировом рейтинге коррупции ниже чем Россия. 

\iusr{Evgeniy Sidorenko}

@Александр Бусель   в данном интервью, пропагандистскому каналу RT Россия, в
программе Антонимы, Гиркин утверждает что именно он начал войну на востоке
Украины. Так же Гиркин утверждает что именно он получал приказы от советника
Путина Суркова. В данном интервью Гиркин говорит о том что никто не хотел
воевать, и если бы не он, горячей фазы войны не наступило бы, и митинги
закончились бы без агресии на Донбассе. Он утверждает что митинги закончились
также как и везде, в Харькове, Одессе, и так далее. Также Гиркин утверждает
что операция по захвату Донбасса готовилась за долго до любых боевых действий
и референдума. Наслаждайся

\url{https://youtu.be/lwRU8liw_Io} 

\iusr{Evgeniy Sidorenko}

 @Александр Бусель  наслаждайся клоун, я уже написал. Жду ответов а не твоё балабольство
\iusr{Evgeniy Sidorenko}

@Александр Бусель  ах да, путешок, забыл добавить. Ану покажи мне видео, с
телефона как ты говоришь, где в мариуполе захватили сдание 3 человека ваших
сепаратистов. Хочу посмотреть какой ты балабол

\iusr{Александр Бусель}

@Evgeniy Sidorenko  так давай скидывай насчёт 2006-го, а ещё будь добр и
ответь на мои вопросы, ещё раз повторяю, до майдана Донбасс пытался отделится?
Мы крымчани были более настроенны, но даже мы не начинали вести сепаратистские
движения, ибо вас считали единым народом. Ну давай, показывай, а то пистеть не
мешки ворочать. И именно где Доннбасс во всём порыве хотел, а не сборище
десяти человек. Обкакался ты знатно  @igg{fbicon.face.tears.of.joy} .

\iusr{Evgeniy Sidorenko}

@Александр Бусель  олень, комментарии ютуб удаляет потому что ты материться как
сапожник, а это запрещено ютубом. Но ты клоун все думаешь что это страшный
правый сектор или гос деп 

\iusr{Evgeniy Sidorenko}

@Александр Бусель  давай. Начнём с 2006 года. Идёт? Что бы не писать все в
одну кучу. Я тебе отвечаю по поводу 2006 и кстати не только в 2006 был. Ещё
были. Сейчас тебе все найду. А ты мне ответишь на вопрос, почему Россия у себя
сервант начальника одесской милиции, ответственного за события 2 мая. И почему
Россия скрывает у себя Боделана, начальника МЧС, который не тушили пожар 

\iusr{Александр Бусель}
 @Evgeniy Sidorenko  олень, покажи хоть нескольких матов, а вот тебе сказка про
 троих из дрг \url{https://youtu.be/soKabj67pzs}
\iusr{Александр Бусель}
 @Evgeniy Sidorenko  так давай доказательства, жду.
\iusr{Evgeniy Sidorenko}

@Александр Бусель  сейчас тебе все доказательства будут. Вангую сразу что ты
будешь выть что то все фейки @igg{fbicon.face.tears.of.joy}  по поводу матов клоун ты тупой не у меня
спрашивай а у ютуба. Так как он удаляет твои коменты из-за мата, олигофрен ты
блин 

\iusr{Александр Бусель}
 @Evgeniy Sidorenko  я тебе про Мариуполь два скинул, ну.
\iusr{Александр Бусель}
 @Evgeniy Sidorenko  что притих снова унтерменш?
\iusr{Evgeniy Sidorenko}
 @Александр Бусель  ты можешь ещё раз подробно за мариуполь. Я ничего не понимаю какой бред ты пишешь и что ты утверждаешь. Что мариуполь никто не захватывал а украинские военные просто приехали туда людей хуярить или что? Ты хоть сам определись уже что ты хочешь мне написать.
\iusr{Maikl Тритон}
 @Evgeniy Sidorenko  

А вот Герой Украины Надежда Савченко говорит, что на Украине идёт гражданская
война - \url{https://www.youtube.com/watch?v=A8L9KVFwE6Q} 


\end{itemize} % }

\end{multicols} % }
