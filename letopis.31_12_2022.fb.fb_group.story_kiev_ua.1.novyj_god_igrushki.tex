% vim: keymap=russian-jcukenwin
%%beginhead 
 
%%file 31_12_2022.fb.fb_group.story_kiev_ua.1.novyj_god_igrushki
%%parent 31_12_2022
 
%%url https://www.facebook.com/groups/story.kiev.ua/posts/2105285746334882
 
%%author_id fb_group.story_kiev_ua,levickaja_natasha.kiev
%%date 
 
%%tags 
%%title Сегодня наступает Новый год
 
%%endhead 
 
\subsection{Сегодня наступает Новый год}
\label{sec:31_12_2022.fb.fb_group.story_kiev_ua.1.novyj_god_igrushki}
 
\Purl{https://www.facebook.com/groups/story.kiev.ua/posts/2105285746334882}
\ifcmt
 author_begin
   author_id fb_group.story_kiev_ua,levickaja_natasha.kiev
 author_end
\fi

Сегодня наступает Новый год. Очень странное чувство - нет радости, предвкушения
праздника, но есть одно желание одинаковое у всех нас - мы все желаем МИРА и
окончания войны,  чтобы наши города снова сияли ярким светом и праздничной
иллюминацией, а по улицам ходили люди с радостными лицами в предпраздничной
суете.

Этот пост был  в группе год назад, когда мы даже представить не могли, что
принесет нам 2022 год...

Хотя  бы ненадолго отвлечёмся от войны, политики, переживаний, ненависти, от
страшных реалий, и пустим в душу немного позитива и тепла из воспоминаний
детства, когда живы были наши родители, бабушки и дедушки и про войну мы знали
только из кино.

МИРА нашей Украине!❤️💛💙

НОВОГОДНИЕ ИГРУШКИ

Ностальгия по детству и теплые воспоминания о новогодних семейных праздниках
вызывают у нас желание хотя бы ненадолго вернуться в прошлое, в детство!
Наверное, поэтому у нас  такое трепетное отношение к старым новогодним
игрушкам.

Все они  хранят бесценную память  о чём-то  очень  близком. О тех, кто был
рядом с нами в те далёкие времена - вот наши молодые папа и мама, которые
украшают с нами ёлку, наши бабушки и дедушки, запах хвои и мандарин,
предвкушение и ожидание чуда!

Конечно, времена меняются и  последние годы в украшении наших ёлок стал
преобладать современный стиль - шары, атласные банты…

И безусловно, монохромные шары и атласные банты на ёлке – это красиво! Но это
холодная красота, а в старых игрушках теплая и душевная. Они из прошлого и
хранят дух того  далёкого  времени, когда мы верили в новогодние чудеса!

Мы давно уже повзрослели, возмужали, огрубели душой, взгляды на жизнь стали
циничнее, однако в каждом из нас взрослом осталось что-то детское, и  где-то в
далеком уголке нашей души все ещё теплится маленькая надежда на чудо, которое
все мы ждём накануне каждого Нового года!

\obeycr
"Старые игрушки... просто раритет.
Облупились где-то, сколько же им лет?
Кукуруза, летчик, часики, луна...
Дед Мороз, хлопушки, бусы, бахрома...
Шишки и фонарик, звездочка внутри.
Вот олень картонный, и рога... целы!
Сказочное царство в доме у меня.
Это - детство... Память... Целая страна!
Есть, конечно, лучше..., но не заменить
То, что было в детстве. А в душе... щемит."
\smallskip
(И.Воробьёва)
\restorecr

\ii{31_12_2022.fb.fb_group.story_kiev_ua.1.novyj_god_igrushki.orig}
\ii{31_12_2022.fb.fb_group.story_kiev_ua.1.novyj_god_igrushki.cmtx}
