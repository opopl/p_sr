% vim: keymap=russian-jcukenwin
%%beginhead 
 
%%file 18_02_2021.fb.fb_group.story_kiev_ua.2.puscha_vodica_tramvaj_marshrut.cmt
%%parent 18_02_2021.fb.fb_group.story_kiev_ua.2.puscha_vodica_tramvaj_marshrut
 
%%url 
 
%%author_id 
%%date 
 
%%tags 
%%title 
 
%%endhead 
\zzSecCmt

\begin{itemize} % {
\iusr{Андрей Надиевец}
Детства маленький трамвай, ты меня не забывай..

\iusr{Nina Gerasymenko}
Любимий маршрут!

\iusr{Елена Зозуля}
Трамвайчик который дарил счастье!!

\iusr{Валентина Горобец}

\ifcmt
  ig https://i2.paste.pics/b543e68b7f3a67c72a3b389edc88a06f.png
  @width 0.2
  @name scr.million
\fi

\iusr{Дмитрий Мойсеенко}
ходит и ходит, все нормально

\iusr{Елена Константинова}

Обожаю этот маршрут. Живу неподалеку (площадь Шевченко). Часто с мужем гуляем в
лесу. Особенно любим зимнюю прогулку. И когда встречаем трамвайчик, то
радуемся, как дети.


\iusr{Nata Komashko}

\ifcmt
  ig https://i2.paste.pics/9269591b117f529680379d9389bcecaf.png
  @width 0.2
  @name scr.super
\fi

\iusr{Елена Мирошниченко}
Ходит в юность.

\iusr{Елена Гордзялковская}

С 1957 года несколько раз в разные годы меня вывозили в Пущу-Водицу на дачу
летом. Когда я с бабушкой утром ходила за хлебом в магазин, мы по дороге
собирали немного грибов и кружку земляники. Вместо бассейна бабушка выставляла
на солнышко детскую ванночку с водой. В жару это было очень приятно.


\iusr{Владимир Бахтурин}

Обожаю ездить летом на трамвайчике в Пущу... как в детстве. Уже в начале 70-х
тут бегали чешские трамваи со скоростью поезда, как мне тогда казалось. Ведь по
центру ездили, вернее ползали с грохотом и скрипом, резким дёрганым разгоном и
таким же торможением, жёлтые \enquote{черепахи} МТВ...

А я любил красные чешские трамвайчики... огромное удовольствие получал от езды
на них... и Пущу я безмерно люблю...!!


\iusr{Мария Елинская}

Рідна Пуща! Чудоаий маршрут через ліс та вулички курорту! Стукіт коліс та звуки
музики це той шарм, що залишився а пам'яті, бо теперішні цегляні мури змінили
облік Пущі!

\iusr{Lara Ilich}

Два года тому назад я решила прокатить 6 ти летнего внука, который приехал в
гости из другой страны, на этом трамвайчике. В Пущу поехали на машине, а там
сели в трамвай. Малышу очень хотелось смотреть вперед и это увидел
вагоноважатый. Необыкновенный человек. Он взял внука в кабину, посадил его,
показал все рычаги и кнопки и провез через всю рощу. А потом попросил вернуться
в вагон, чтобы никто не увидел в кабине постороннего да еще ребенка. На мои
благодарности только и сказал, что не забыл о мальчишеских желаниях. Можете
себе представить, что с тех пор мой внук считает, что в Киеве лучше всего на
свете!

В тот же день он впервые в жизни увидел на озере в Пуще лёд!

\ifcmt
  ig https://scontent-frt3-1.xx.fbcdn.net/v/t1.6435-9/152005658_3798248930218611_8024967104636442862_n.jpg?_nc_cat=104&ccb=1-5&_nc_sid=dbeb18&_nc_ohc=wQF42ImKi68AX_eq6uG&_nc_ht=scontent-frt3-1.xx&oh=00_AT_sQfkbxvlbgiEzxxb7uf8MV_2BxzcXGDVtxKSMBvXwFw&oe=61E073C2
  @width 0.4
\fi

\begin{itemize} % {
\iusr{Владимир Бахтурин}
\textbf{Lara Ilich} мечтал, чтобы меня так прокатили!
Как много ребёнок впечатлений получил, - представляю.

\ifcmt
  ig https://scontent-frx5-2.xx.fbcdn.net/v/t39.1997-6/s168x128/10734344_610917765700644_325078055_n.png?_nc_cat=1&ccb=1-5&_nc_sid=ac3552&_nc_ohc=OAqk2kz_Y-gAX8yvC0Y&_nc_ht=scontent-frx5-2.xx&oh=00_AT9CasV_gtB9IZP7SfNd7yudvEx1175qIM65nx8vnegKLA&oe=61BF4B05
  @width 0.1
\fi

\begin{itemize} % {
\iusr{Lara Ilich}
\textbf{Владимир Бахтурин} да, это не аттракцион карусельный за деньги

\iusr{Липовец Наталья}
\textbf{Lara Ilich} есть много детей, которые никогда не катались на атракционах. Все в мире относительно.  @igg{fbicon.face.smiling.eyes.smiling} 
\end{itemize} % }

\iusr{Липовец Наталья}
\textbf{Lara Ilich} вот это Человек вам встретился! Ведь как просто сделать
счастливым внука и (еще более счастливой) его бабушку!

\iusr{Marina Ganopolska}

Пуща любимейшее место с детства. Пуща-это лето, запах хвои, запах воды в озере, ее
болотно-бутылочный цвет. В Пуще папа научил меня грести, мы всегда брали лодку на
прокат. Пуща-это праздник, который начинался с 12го трамвая, в который
садились, иногда с боем, на площади Шевченко. Дорога трамваем через лес была
отдельным удовольствием.

И В моей судьбе Пуща сыграла судьбоносную роль-там случайно встретилась со
своим бывшим одноклассником, который стал моим мужем

\begin{itemize} % {
\iusr{Lara Ilich}
\textbf{Marina Ganopolska} Марин, это Сонин сын

\iusr{Marina Ganopolska}
\textbf{Lara Ilich} какой чудесный!

\iusr{Marina Ganopolska}
Я так и подумала
\end{itemize} % }

\end{itemize} % }

\iusr{Олег Русанов}
Все футбол смотрят!

\ifcmt
  ig https://i2.paste.pics/79a823c277461015abc3bd81fd98aafd.png
  @width 0.1
\fi

\iusr{Виктория Хорликова}

\ifcmt
  ig https://i2.paste.pics/95212a6ba870665f1b93ea5688473c5e.png
  @width 0.1
\fi

\iusr{Ludmila Teslenko-ponomarenko}

Мой любимый маршрут трамваем с конечной на Подоле до площади Шевченко, номер
19. Ездила на нем на станцию Полесье, а дальше пригородным автобусом к нам на
дачу. Ездила и наоборот, с \enquote{Полесья} до Подола. Трамвай неторопливо пыхтит мимо
молодого Виноградаря, бывшей знаменитой \enquote{Птички}, улицы Викентия Хвойки, потом
долго и неторопливо по старому Подолу, доезжает наконец до Контрактовой площади
и там на конечной высаживает пассажиров. Люблю его за неторопливую
основательность. К сожалению, дальше в саму Пущу ни разу не выбралась еще. Сам
Подол, Андреевский спуск, Контрактовая площадь, очень долго был моим ежедневным
и любимым маршрутом. С конечной остановки 27 трамвая до Березняков, где я
выходила и топала домой.

\begin{itemize} % {
\iusr{Тетяна Леонова}
\textbf{Ludmila Teslenko-ponomarenko} ,класс... и успевали и помечтать, и отдохнуть, и насладиться видами за окошком.

\iusr{Ludmila Teslenko-ponomarenko}
\textbf{Тетяна Леонова} даа, пожалуй и за это тоже! И еще за то, что трамвай долго крутится по Подолу...
\end{itemize} % }

\iusr{Светлана Гаврилко}
Ездили осенью.такая красота
Наверное, в любое время года это всегда потрясающе!!!!

\iusr{Olga Koteneva}
Когда мы жили на Красной площади, мама работала на 7 линии, в санатории им. 1-го мая.
Каждый день в 6 утра садилась в 12 маршрут и ехала в Пущу.
Историй, случавшихся по дороге, было множество.
То дерево упадет и перекроет пути, то лось выйдет и стоит себе, рассматривая трамвай...

\begin{itemize} % {
\iusr{Липовец Наталья}
\textbf{Olga Koteneva} лось??  @igg{fbicon.face.astonished}{repeat=2} 

\iusr{Olga Koteneva}
\textbf{Липовец Наталья} Лось. Это были 60- начало 70-х и в Пуще ещё жили лоси

\iusr{Диана Ишунина}
\textbf{Olga Koteneva} Лось еще недавно выходил из Пущи по утру к Синему Озеру, жители Виноградаря фотки выкладывали
\end{itemize} % }

\iusr{Eli Na}
Рядом жили. Ездили на великах

\iusr{Тетяна Леонова}
Краса незрівнянна !

\ifcmt
  ig https://scontent-frx5-2.xx.fbcdn.net/v/t39.1997-6/s168x128/70089142_2227712840660529_772462753087488000_n.png?_nc_cat=1&ccb=1-5&_nc_sid=ac3552&_nc_ohc=x2KE056eZAoAX_GMLa_&_nc_ht=scontent-frx5-2.xx&oh=00_AT_mZJfQHX8J2hsiowmeM3fTzeKryqcOBp8VauDP3bIf-Q&oe=61BF5240
  @width 0.1
\fi

\iusr{Mike Bux}

Пуща густо усыпана останками солдат и разного рода боеприпасов с ВОВ. Высоко на
деревьях весят разорванные \enquote{катюши}. В 70 е весной туда бегал с лопатой.

\begin{itemize} % {
\iusr{Дмитрий Мойсеенко}
\textbf{Mike Bux} не нагоняйте страху, все совсем не так, маршрут безопасен

\iusr{Mike Bux}
\textbf{Дмитрий Мойсеенко} да нет он конечно безопасен. Но очень интересное место для поисковиков

\iusr{Mike Bux}
Я с лесником был знаком. Приносили ему на пасху водки а яиц он он за это показывал где копать. Оружия там было видимо не видемо.

\iusr{Липовец Наталья}
\textbf{Mike Bux} наверное, вы стали геологом?))
\end{itemize} % }

\iusr{татьяна гордиенко}
Ходил экскурсионный трмвмйчик
Очень приятно было прокатиться

\iusr{Елена Царовская}
\textbf{татьяна гордиенко} и сейчас ходит

\iusr{Ника Орлова}
Оболонский район

\begin{itemize} % {
\iusr{Сергей Оборин}
\textbf{Ника Орлова} 

Раньше был Шевченковский, теперь Подольский, по крайней мере Виноградарь. Я
поселился на Виноградаре, возле Синего озера в 1983 году. И мы, всей семьей
ходили в лес, аж до маршрута трамвая. Интересно было наблюдать среди сплошного
зеленого пейзажа сосен, - ярко красное пятно трамвая, следить за тем, как он
приближался издалека, увеличиваясь в размерах, становился огромным, проносился
с грохотом по рельсам мимо, и исчезал за поворотом. Это, каждый раз, было,
как-бы отдельным рассказом, воспоминанием, запомнившимся на всю жизнь.


\iusr{Ника Орлова}
\textbf{Сергей Оборин} ,Пуща теперь Оболонский район

\iusr{Сергей Оборин}
Возможно. Не спорю.
\end{itemize} % }

\iusr{О. Зк}

\ifcmt
  ig https://scontent-frt3-1.xx.fbcdn.net/v/t1.6435-9/151777132_2888070374770117_6082380386245968064_n.jpg?_nc_cat=102&ccb=1-5&_nc_sid=dbeb18&_nc_ohc=CiEegyUsv7kAX_t731U&_nc_ht=scontent-frt3-1.xx&oh=00_AT8zjBa4R8GLW-rO3eZf9_ZmBmn8KJKjVgP3HAYfl2I0mA&oe=61DFAF89
  @width 0.4
\fi

\iusr{Галина Баутина-Тимошенко}
Трамвай из Сказки

\iusr{Люба Опанасенко}

\ifcmt
  ig https://scontent-frx5-2.xx.fbcdn.net/v/t39.1997-6/s480x480/14130018_1775288519378519_561823984_n.png?_nc_cat=1&ccb=1-5&_nc_sid=0572db&_nc_ohc=Ap4kSE78EnIAX82ViBt&_nc_ht=scontent-frx5-2.xx&oh=00_AT8fgo-6x3fwoF2V9w3WelBvPEXD-xOQw0952oYa73bwoA&oe=61BF68A1
  @width 0.2
\fi

\iusr{Юлия Кривенко}

Поездка в Пущу для нас детей всегда была в радостьСкорость с которой трамвай
несся по лесу. В младшей школе(124) мы с классом с родителями выбирались на
пикники Вкуснее сосисок на костре и батона с вареньем не было ничего на свете
Сейчас все чаще поездка в Пущу, это визит к 4м поколениям родственников, которые
нашли покой на Пущанском кладбище.

\begin{itemize} % {
\iusr{Петр Кузьменко}
\textbf{Юлия Кривенко} мои родители покоятся на Пущанском кладбище...

\iusr{Юлия Кривенко}
\textbf{Петр Кузьменко} Пусть земля будет пухом

\iusr{Юлия Кривенко}
Когда буду там, передам от вас привет Они услышат
\end{itemize} % }

\iusr{Наталия Миронова- Ставицкая}
В любую погоду и время года - это самая красивая в Киеве дорога

\iusr{Ирина Даниленко}
Приезжайте... катайтесь с удовольствием...

\iusr{Марія Омельченко}

\ifcmt
  ig https://i2.paste.pics/9876d2c3d5c2128bb9a68ba28f6d0a30.png
  @width 0.2
\fi

\iusr{Ирина Даниленко}

\ifcmt
  ig https://scontent-frx5-2.xx.fbcdn.net/v/t39.1997-6/s168x128/70089142_2227712840660529_772462753087488000_n.png?_nc_cat=1&ccb=1-5&_nc_sid=ac3552&_nc_ohc=x2KE056eZAoAX_GMLa_&_nc_ht=scontent-frx5-2.xx&oh=00_AT_mZJfQHX8J2hsiowmeM3fTzeKryqcOBp8VauDP3bIf-Q&oe=61BF5240
  @width 0.1
\fi

\iusr{Ната Пазио}

\ifcmt
  ig https://scontent-frt3-1.xx.fbcdn.net/v/t1.6435-9/150590713_892212134874309_53833130701965975_n.jpg?_nc_cat=102&ccb=1-5&_nc_sid=dbeb18&_nc_ohc=204VvVHELY4AX8ijXuE&_nc_ht=scontent-frt3-1.xx&oh=00_AT_BSg8oRGXwUuO3ObwROn39aVRUJ3Wv3d6TTkrSpD5RAA&oe=61DFECC5
  @width 0.4
\fi


\iusr{Оксана Грибанова}
Чудова Пуща!

\iusr{Вера Иванова}
Красивейшее место в Киеве @igg{fbicon.heart.with.ribbon} 

\iusr{Георгий Майоренко}

Славные места! Но зимой там был только один раз. Друг отмечал на территории
санатория День рождения. Сюжет холодный и снежный, а воспоминания теплые!


\iusr{Светлана Петровская}
Один из любимых маршрутов Города

\ifcmt
  ig https://scontent-frx5-2.xx.fbcdn.net/v/t1.6435-9/152298834_928224251320360_160173149032650108_n.jpg?_nc_cat=109&ccb=1-5&_nc_sid=dbeb18&_nc_ohc=eGFVPUK3KfUAX-6_gJo&_nc_ht=scontent-frx5-2.xx&oh=00_AT-S0h7C8sUoaH76pHoSv_4mWKDuiWOTXyDOQi6BlgoJfQ&oe=61DF8861
  @width 0.4
\fi

\begin{itemize} % {
\iusr{Валентина Николаевна}

Пуща - это Оболонский район. Очень красиво и зимой и летом. Особенно если
ездить туда отдохнуть. Но жить в Пуще очень тяжело. Транспортной развязки нет.
Трамваи едут через раз. Не доберешься ни на работу ни с работы.

\iusr{Петр Кузьменко}
\textbf{Валентина Николаевна} раньше Пуща была Подольским районом.

\iusr{Валентина Николаевна}
\textbf{Петр Кузьменко} а сейчас Оболонский. Я живу в Пуще.
\end{itemize} % }

\iusr{Валентина Николаевна}

\ifcmt
  ig https://scontent-frt3-1.xx.fbcdn.net/v/t39.1997-6/p240x240/19876780_1900029023543218_2877948282426884096_n.png?_nc_cat=104&ccb=1-5&_nc_sid=0572db&_nc_ohc=pL7bty9fkQAAX9hHpfj&tn=lCYVFeHcTIAFcAzi&_nc_ht=scontent-frt3-1.xx&oh=00_AT9yAvgVXWZEAYy3YbeBK_6t1saMGwzL5kLhlswXhA1dNA&oe=61BED387
  @width 0.1
\fi

\iusr{Марина Чичкан}

Когда хочется отдохнуть душой, мы садимся на 12 трамвайчик и едем в Пущу. Это
тот случай, когда поездкой в общественном транспорте ты наслаждаешься! Потом
выйти у парка, пройтись до озера, посмотреть на уток) По мостику - на другой
берег, и дальше идти и идти, пока не устанешь). Доходим до кольца трамваев,
заходим в милое семейное кафе, чтобы подкрепиться и выпить чаю. Теперь можно
сесть на трамвайчик и доехать обратно до площади Шевченко.

\begin{itemize} % {
\iusr{Анна Литвинчук}
\textbf{Марина Чичкан}  

@igg{fbicon.face.grinning.smiling.eyes}  Вот Вы просто написали мой рассказ ☹️
опередили  @igg{fbicon.face.grinning.smiling.eyes} и кафешка \enquote{лето}
ЗАМЕЧАТЕЛЬНАЯ!!

\iusr{Марина Чичкан}
\textbf{Анна Литвинчук} Да!!! Это как награда в конце прогулки! @igg{fbicon.heart.suit}

\iusr{Анна Литвинчук}
\textbf{Марина Чичкан}  @igg{fbicon.100.percent} 

\iusr{Анна Литвинчук}
А дети так ждут конца прогулки, чтоб зайти в кафе и потискать зверюшек )

\iusr{Ирина Иванченко}
\textbf{Марина Чичкан}, 

наслаждаешься, если не торопишься, и если рядом не присядут дурно пахнущие
\enquote{вольные} граждане, поэтому, при всём уважении, лучше машиной.

\end{itemize} % }

\iusr{Екатерина маковецкая}
а помоему Пуща это уже Оболонский район... может я путаю, конечно

\iusr{Генрих Киевский}
\textbf{Екатерина маковецкая} да это так и было так

\iusr{Татьяна Задорожная}

\ifcmt
  ig https://scontent-frt3-1.xx.fbcdn.net/v/t1.6435-9/152327713_787743285158721_8519741385621870928_n.jpg?_nc_cat=107&ccb=1-5&_nc_sid=dbeb18&_nc_ohc=_xvTrOf1O0YAX9K2OIE&_nc_ht=scontent-frt3-1.xx&oh=00_AT8aPDuFAIEdsnHANhbdQgJCon6ZNfzl7AcVluIVNmZafQ&oe=61E05188
  @width 0.4
\fi

\iusr{Наталия Новикова}

\ifcmt
  ig https://scontent-frx5-1.xx.fbcdn.net/v/t1.6435-9/151269171_984839278711346_3524813502642212867_n.jpg?_nc_cat=105&ccb=1-5&_nc_sid=dbeb18&_nc_ohc=H1UQ1OjJ-tcAX9IeN5L&_nc_ht=scontent-frx5-1.xx&oh=00_AT9TqTJVkF9BM7CZA_TDFOyVs3YhgDnHqpjHTQcOccsrkQ&oe=61E12F78
  @width 0.2
\fi

\iusr{Ирина Белова}
Казка

\iusr{Татьяна Задорожная}

\ifcmt
  tab_begin cols=2,no_fig,center,height=0.1

     ig https://scontent-frt3-1.xx.fbcdn.net/v/t1.6435-9/151754225_787743335158716_6119061744083457515_n.jpg?_nc_cat=104&ccb=1-5&_nc_sid=dbeb18&_nc_ohc=-X5ywNwFxMkAX9RjWlD&_nc_oc=AQkqHaHLwaslgJuooAKZDk5okqm314RkbUtxnKnNSVGU2G-QX9Vasq5CWpNi74eF62I&_nc_ht=scontent-frt3-1.xx&oh=00_AT-d2XNHjUlqCi8GVwRZQMFaxaH2o9yr5y7H-sf9ovl-LA&oe=61E1093D
     %@width 0.2

     ig https://scontent-frt3-1.xx.fbcdn.net/v/t1.6435-9/151229037_787743378492045_3389543458162467566_n.jpg?_nc_cat=104&ccb=1-5&_nc_sid=dbeb18&_nc_ohc=IHkQrfJUG14AX9O9iHG&_nc_ht=scontent-frt3-1.xx&oh=00_AT-O2V-5NjlMcO7ef0MIiX8grKa9oiI_jVWK_8EfSE7RsQ&oe=61DE3287
     %@width 0.3

  tab_end
\fi

\iusr{Марина Чичкан}

\ifcmt
  ig https://scontent-frt3-1.xx.fbcdn.net/v/t1.6435-9/151263790_3824730104246195_8158905126650029679_n.jpg?_nc_cat=104&ccb=1-5&_nc_sid=dbeb18&_nc_ohc=0B2EP654AuQAX8Z2W7q&_nc_ht=scontent-frt3-1.xx&oh=00_AT_h2wDe536ZRcdN_b5AvGT8YZhV5WUf1crQgGRRA7RMmw&oe=61DDFFA5
  @width 0.4
\fi

\iusr{Раиса Зарицкая}

\ifcmt
  ig@ name=scr.super
  @width 0.2
\fi

\iusr{Татьяна Задорожная}

\ifcmt
  ig https://scontent-frt3-1.xx.fbcdn.net/v/t1.6435-9/152074675_787743518492031_767664242604708062_n.jpg?_nc_cat=104&ccb=1-5&_nc_sid=dbeb18&_nc_ohc=1-_hqFOIFCYAX-TjYWR&_nc_ht=scontent-frt3-1.xx&oh=00_AT-vqoHIBdVlDzg-4AEAGQ5VvrxP10YvTAkvmMII1qmArQ&oe=61DE17A6
  @width 0.3
\fi

\iusr{Elena Wolak}

\obeycr
Родная Пуща Водица!
Сколько же там проведено замечательных дней))
Шашлыки, купание, лес, прогулки как летом так и зимой)
Ностальгия!
Я сейчас так далеко((
\restorecr

\iusr{Татьяна Литвина}
Это трамвайный маршрут в живую сказку....

\iusr{Elena Wolak}

\ifcmt
  ig https://i2.paste.pics/05a6a3414f8555f44b4c189baf595a78.png
  @width 0.2
\fi

\iusr{Татьяна Першина}
Пуща зараз Оболонський район

\iusr{Людмила Малыгина}

\ifcmt
  ig https://scontent-frx5-2.xx.fbcdn.net/v/t39.1997-6/p480x480/105941685_953860581742966_1572841152382279834_n.png?_nc_cat=1&ccb=1-5&_nc_sid=0572db&_nc_ohc=Kq8EgqfccboAX_dXae-&_nc_ht=scontent-frx5-2.xx&oh=00_AT9k3Id6lPfMvYfF-TSijDEBQead0zG0wW3fswGyScMcew&oe=61BFE64B
  @width 0.2
\fi

\iusr{Лариса Сергієнко}

\ifcmt
  ig https://scontent-mxp1-1.xx.fbcdn.net/v/t1.6435-9/151581296_216473500204305_5369293591901897818_n.jpg?_nc_cat=102&ccb=1-5&_nc_sid=dbeb18&_nc_ohc=KWbyT__GzQ8AX8xjxyC&_nc_ht=scontent-mxp1-1.xx&oh=00_AT_DO4Jc_FGj6fthSklZt2aGescSc7s3duleH1A5uvlQdw&oe=61DEC443
  @width 0.4
\fi

\begin{itemize} % {
\iusr{Петр Кузьменко}
\textbf{Лариса Сергієнко} где сейчас церковь Святого Николая на воде, я раньше ловил лещей...

\iusr{Лариса Сергієнко}
Рыбаки на Днепре, с моста Патона ( сердце замирает )

\ifcmt
  ig https://scontent-mxp1-1.xx.fbcdn.net/v/t1.6435-9/151353347_216475456870776_8186232941228050465_n.jpg?_nc_cat=110&ccb=1-5&_nc_sid=dbeb18&_nc_ohc=EtqtvIvP7B8AX9vD8lV&_nc_ht=scontent-mxp1-1.xx&oh=00_AT8KcFG0Y82JkGPLkyrZdqoRrqdkyL0d5LXZSKzrQn4xrQ&oe=61E08145
  @width 0.4
\fi

\end{itemize} % }

\iusr{Лариса Сергієнко}

\ifcmt
  ig https://scontent-mxp1-1.xx.fbcdn.net/v/t1.6435-9/152018514_216474263537562_42597055105329274_n.jpg?_nc_cat=109&ccb=1-5&_nc_sid=dbeb18&_nc_ohc=_gBZqTXsADYAX-2YhqW&_nc_ht=scontent-mxp1-1.xx&oh=00_AT_M-R7xF8WP6i8vi2pceYkU7bKSUx_aU3M1wjn8tBaPaw&oe=61E16E95
  @width 0.4
\fi

\iusr{Rosa Müller}
А ходят такие трамвайчики ещё по Киеву?

\iusr{Наталия Удовченко}
Да, конечно. И этот маршрут в Пущу есть.

\iusr{Marina Starobinskaja}
Детство @igg{fbicon.heart.red}

\iusr{Анна Гречаник}
В Киеве везде красиво! Каждый район - по своему красив! Я выросла на Никольской Слободке, для меня самое лучшее воспоминание, частный дом, зимой запах горящих дров, печь, летом - чужие сады!

\iusr{Юрий Мулькин}
Пуща-Оболонский район

\iusr{Петр Кузьменко}
\textbf{Юрий Мулькин} уже пояснили мне. Благодарю.

\iusr{Татьяна Сирота}

Пуща 2 января 2021 года.
На улице весенняя погода: солнце и +7

\ifcmt
  ig https://scontent-mxp1-1.xx.fbcdn.net/v/t1.6435-9/150937119_427770248532644_5623379438093070365_n.jpg?_nc_cat=111&ccb=1-5&_nc_sid=dbeb18&_nc_ohc=EOuY6ZrgYqoAX8PIGy3&_nc_ht=scontent-mxp1-1.xx&oh=00_AT9gvCG9aSk6GQjMTLKcs-qbAdq3gB8KbTdAjMWyefwWsg&oe=61E09B22
  @width 0.4
\fi

\iusr{Ольга Федорченко}

А я там никогда не была.... Хотя родилась там в роддоме. Кто то знает где был
роддом в 19 81?

\begin{itemize} % {
\iusr{Леонтина Матиящук}
\textbf{Ольга Федорченко} . 

Роддом был на 1-й линии, в больнице под лесом, в основном там рожали те, у
которых было подозрение на какую- нибудь инфекцию. Но рожали все, кому
удавалось туда попасть. Во время ВОВ там был Немецкий госпиталь). Позже в эту
больницу перевели детей с костным туберкулезом , где они очевидно и сейчас.


\iusr{Марина Чепак}
\textbf{Ольга Федорченко} 

Я в этом роддоме рожала сынулю в 1984 году т.к. за год до родов переболела
гепатитом. По сравнению с другими роддомами людей было совсем мало и внимания
было много. Я рожала в пятницу, а следующие роды были в четверг. Когда
регистрировали сына, спрашивали как записать место рождения сына -Киев или пос.
Пуща-Водица.

\iusr{Ольга Федорченко}
\textbf{Марина Чепак} а меня записали по месту прописки моей мамы. Это было село в Киевской обл. и теперь как то так в паспорте и указано.Говорит там в Пуще было хорошо рожать)

\iusr{Марина Чепак}
\textbf{Ольга Федорченко}

\ifcmt
  ig https://scontent-mxp1-1.xx.fbcdn.net/v/t39.1997-6/s168x128/17640308_1652591141433953_2515677274297073664_n.png?_nc_cat=1&ccb=1-5&_nc_sid=ac3552&_nc_ohc=2s3ZVplaAZwAX_ZMOpi&_nc_ht=scontent-mxp1-1.xx&oh=00_AT_VJNZYIq4gw3gwo8C-zcmUg1BBS_VAfuszibYv3DCMCg&oe=61BF772E
  @width 0.1
\fi

\end{itemize} % }

\iusr{Іванна Братусь}
Люблю

\iusr{Геннадий Федоровский}
80-90-е дуже добре пам'ятаю))

\iusr{Людмила Марченко}
Пуща - это красота неземная, лес, грибов немеро..........!!!

\ifcmt
  ig https://i2.paste.pics/1b2549c7957ff8ca1bb5d7df28071ab9.png
  @width 0.2
\fi

\iusr{Хана Гуфельд}


\ifcmt
  tab_begin cols=2,no_fig,center

     ig https://scontent-mxp1-1.xx.fbcdn.net/v/t1.6435-9/150967006_3810827845700578_3963819681582380780_n.jpg?_nc_cat=103&ccb=1-5&_nc_sid=dbeb18&_nc_ohc=wmAl6_NSqVoAX_LuzW4&_nc_ht=scontent-mxp1-1.xx&oh=00_AT-MtbNKmNMGf19E7Hsh4k_ObkHkTseJGrV6sK_kPfHc7Q&oe=61DF7605

     ig https://scontent-mxp1-1.xx.fbcdn.net/v/t1.6435-9/150698638_3810828142367215_6719120777540153843_n.jpg?_nc_cat=105&ccb=1-5&_nc_sid=dbeb18&_nc_ohc=JdqKoUDalG0AX9W8x_h&_nc_ht=scontent-mxp1-1.xx&oh=00_AT-c_IpkDlFJ_x9MSbsJ6wpf7T9qvFgL2n8yq7-sIwKG-w&oe=61DDD29A

  tab_end
\fi

\iusr{Светлана Дудникова}
Це, щось як щастя. Це потрібно людям! Це життя!

\iusr{Lidiya Vasylivna}

Взимку ніколи не їздила в Пущу, а от в теплу пору року їздили часто. Завжди
враження були чудові! Люди збиралися навколо озер, розкладали свої стільці і
столики і проводили там весь день. Природа неймовірна! Повітря чисте! А на 3-й
лінії був санаторій для ветеранів, і ми їздили туди провідувати батька. Восени
там було багато грибів, батько збирав купу опеньок і мені був клопіт їх
готувати. Батьків вже немає, а спомини залишились...

\iusr{Алла Тихонова}
Сказочный трамвай!

\ifcmt
  ig https://scontent-mxp1-1.xx.fbcdn.net/v/t1.6435-9/151602628_3814592965255779_8012180657502428678_n.jpg?_nc_cat=108&ccb=1-5&_nc_sid=dbeb18&_nc_ohc=gEACjH1bxeQAX_pRUkx&_nc_ht=scontent-mxp1-1.xx&oh=00_AT8a4lGNxG3gkvo-RXKv8Lv2mXgYptqUkgyzsZ8iqoOPNg&oe=61DE41B5
  @width 0.6
\fi

\iusr{Люба Потемкина}

В 1938-39 т я училась и жила в Пуще Водице в Лесной школе. Сейчас там находится
Санаторий винного ведомства. УЖЕ не помню какая это линия, но воспоминания самые
теплые. Иногда мы на кухне спрашивали картошку и в лесу, на заднем дворе,
разжигали костер и пекли картошку. Это такая вкуснятины!!! А однажды через много
десятилетий мне захотелось побывать в здании. Пришлось ,даже, сдать паспорт
чтоб пропустили на территорию. Конечно многое изменилось, но сколько у меня было
волнения. Например, у нас была цвет точная клумба с вкусными маленькими
цветочными, а теперь там фонтан. Деревья стали большими, конечно, они меня не
помнят. А Вдруг кто нибуть из живых откликнется. ??!!?

\begin{itemize} % {
\iusr{Леонтина Матиящук}
\textbf{Люба Потемкина} . Санаторий Киевского военного округа находится на 11-й линии, а Лесная школа - на 5-й линии, напротив парка. Мой брат там находился после войны , его отец погиб во время ВОВ. Была ли эта школы там до войны не знаю.

\iusr{Irina Andreeva}
\textbf{Люба Потемкина} Я была в этой лесной школе 1970-1971 году.Училась в 4 классе.

\iusr{Люба Потемкина}
ИРОЧКА АНДРЕЕВА, СПАСИБО, ЧТО ОТКЛИКНУЛИСЬ. Конечно, моих сверстников, наверное, уже нет А я там училась в 3 классе.

\iusr{Неман Иванов}

Санаторно-лесная школа на 5 линии (трамвай на повороте огибает школьный стадион).

Напротив центрального входа расположен вход в парк \enquote{Пуща-Водица}.

На параллельной улице, \enquote{за спиной} школы находился детский санаторий \enquote{Пионер},
тоже довольно культовое место в памяти многих советских людей. Чего только
стоит танцплощадка в большой деревянной беседке. А развалины старого барского
особняка стояли до начала восьмидесятых. Позднее там был \enquote{Артек}.

Репутация Пущи-Водицы довольно пёстрая: послевоенный уголовный мир превратил её
в сплошные \enquote{малины}, как в подмосковной Марьиной роще. Но уже к концу
пятидесятых здесь всё больше пансионатов, санаториев да пионерских лагерей.

В киевской Пуще летом 41го была открыта первая в СССР школа подготовки бойцов
истребительных отрядов. После оставления Киева большинство \enquote{ястребков}
пополнили партизанско - диверсионные отряды.

В мае 1998 года в санатории \enquote{Пуща озёрная} (от 14 линии вправо, мимо кладбища,
через мост) гостили представители МВФ, решалась судьба украинской экономики.
Здесь же часто проходят киносъёмки фильмов и сериалов, как украинских, так и
других стран.

\iusr{Татьяна Плишко}
\textbf{Neman Ivanov} дякую, дуже цікаві факти з історії Пущі-Водиці.

\iusr{Люба Потемкина}
\textbf{Неман Иванов} очень интересная история Пуща-Водица

\iusr{Неман Иванов}

Можно добавить к истории Пущи-Водицы, что кинотеатр \enquote{Барвинок} на картах Киева
был под первым номером, позже в нём размещался цыганский театр.

А лесную Каланчу великолепный киевский писатель Всеволод Нестайко увековечил в
своей повести о школьниках \enquote{Одиниця з обманом}. В главе \enquote{Макароніна}.

\end{itemize} % }

\iusr{Алла Тихонова}

\ifcmt
  ig https://scontent-mxp1-1.xx.fbcdn.net/v/t1.6435-9/152177911_3814593221922420_2767949131644872249_n.jpg?_nc_cat=104&ccb=1-5&_nc_sid=dbeb18&_nc_ohc=ihyDamEuqXoAX8wqD7m&_nc_ht=scontent-mxp1-1.xx&oh=00_AT_Ekcio_jdD1j0u40_67QhmQ47YUf5EwCN6v1jLY7cvrg&oe=61E068E0
  @width 0.6
\fi


\iusr{Люба Потемкина}
Нужно читать Военного ведомства.

\iusr{Александр Муратов}

Приятно вспомнить эту трамвайную линию. Первый рах по ней проезжал в 1950-х.
Она еще существет,? Её не продали на металлом при незалежных властях, как
многие другие трамвайные маршруты Киева?

\begin{itemize} % {
\iusr{Anyakina Nataliya}
\textbf{Александр Муратов} нет. Не продали. Всё нормально. Иногда ездим в Пущу. Даже есть теперь ещё одна остановка в лесу — по требованию.

\iusr{Диана Ишунина}
\textbf{Александр Муратов} Существует. И езжу, и пешком хожу с Виноградаря аж до 14-й линии( конечная) купаться на Сапсаев Став.

\iusr{Татьяна Сирота}
\textbf{Александр Муратов} Всё осталось... Как и раньше... Трамвай номер 12 с Контрактовой площади
\end{itemize} % }

\iusr{Валентина Степанова}
Я люблю Київ увесь!!!!!

\iusr{Жанна Главацкая}
Пуща это детство, любимый и самый лучший на земле пионерский лагерь \enquote{Огонёк}.

\iusr{Гура Валентина}
Відразу згадала, як ми, з кумою, по грибочки їздимо.  @igg{fbicon.grin} 

\iusr{Ната Бузань}

\ifcmt
  ig https://i2.paste.pics/add2c75203ced02c53284ea7c1ffdd4b.png
  @width 0.2
\fi

\iusr{Елена Янчук}
Це Оболонський район

\iusr{Olga Koteneva}
\textbf{Елена Янчук} раніше це був Подільський

\iusr{Lena Zabelina}
Для меня-лето
14-я линия, п/л Электрон
Красота, детство, радость, тепло

\iusr{Kravets Irina}
Пуща- Водица - Красная площадь.
Красивый маршрут .
Ещё туда же ходил самый длинный трамвайный маршрут Киева - 25. До Ж. д. вокзала.

\iusr{Светлана Молчанова}
Дуже люблю Пущу-Водицю. Дуже багато теплих спогадів.

\ifcmt
  ig https://i2.paste.pics/307562d2fd411abbed06d0437d593b8d.png
  @width 0.2
\fi

\iusr{Елена Майкут}

\ifcmt
  ig https://scontent-mxp1-1.xx.fbcdn.net/v/t1.6435-9/150743145_1330202307356540_2719249880274137234_n.jpg?_nc_cat=100&ccb=1-5&_nc_sid=dbeb18&_nc_ohc=4JgtEiQ4jjIAX_4RcoJ&_nc_ht=scontent-mxp1-1.xx&oh=00_AT_iSJl8T2XdV8_Hno-G58zr9oYNLSUp59KsA8OQYbEuYQ&oe=61DF6811
  @width 0.4
\fi

\iusr{Наталия Королева}

Мы с подругами любим туда ездить. Я даже стихотворение написала.

\headCenter{Вам из города в сказку?}

\begin{multicols}{2} % {
\setlength{\parindent}{0pt}
\obeycr
Вам из города в сказку? 
Так это недолго и в принципе просто по сути:
\smallskip
Контрактовая площадь (былинный Подол),
На трамвайчик 12 — и в путь!
\smallskip
Там шепчутся сосны, сверкают озера,
Искрится песок золотой...
\smallskip
Из шума столичного катимся споро
Туда, где релакс и покой.
\smallskip
К окну приникая в полдневной истоме,
Случайно вдруг выхватит взгляд
\smallskip
Резные наличники старого дома
И чей-то запущенный сад.
\smallskip
Тоннель из зелёных деревьев мне снится
И неодолимо влечёт,
\smallskip
Ведь сказка ожившая — Пуща Водица
Скучает, тоскует и ждёт...
\smallskip
Сентябрь 2019 — 19.05.2020
\restorecr
\end{multicols} % }

\iusr{Юрий Шевченко}
Пуща-Водица - Оболонский ныне район.

\iusr{Світлана Житкова}
В Пущу на озера ездили очень многие киевляне. Запах леса в Пуще не повторим.
.

\iusr{Липовец Наталья}

Я не коренная киевлянка, мы родом из Кропивницкого, теперь уже несколько лет по
прописке киевляне @igg{fbicon.face.smiling.eyes.smiling}  В этом году впервые летом попала на пляж в Пуще Водице.
Красота невероятная! Очень люблю трамваи за их неспешность, ловлю ощущение
мира, спокойствия, размеренности. Подол, Контрактовая, весь старый город -
любимейшие места прогулок! С огромным удовольствием прочла воспоминания
киевлян. Эта группа вдохновляет меня исследовать красивейшие уголки города,
такого древнего и удивительного! Всем участникам моя огромная благодарность!
 @igg{fbicon.face.smiling.eyes.smiling}{repeat=3} 

\iusr{Ирина Медына}

Это Пуща. Лес, слава богу, ещё остался, но так много понастроено \enquote{дворцов} за
каменными высоченными заборами, а резные домики разрушаются, и никому до них
нет дела... Пока ещё сказочная красота природы во все времена года, но -
спешите наслаждаться! Неизвестно, насколько лет это сохранится...


\iusr{Николай Левицкий}

\ifcmt
  ig https://scontent-frx5-2.xx.fbcdn.net/v/t39.1997-6/s168x128/93027172_222645632401274_7176243611145601024_n.png?_nc_cat=1&ccb=1-5&_nc_sid=ac3552&_nc_ohc=5bx1DLdOCosAX-JHrek&_nc_ht=scontent-frx5-2.xx&oh=00_AT9L9KwenvrusFTaOx0ZBpyoAamfe-x0DggfBA2MNyBbzA&oe=61BE3A70
  @width 0.1
\fi

\iusr{Anna Dobrovynska}
Там, де живе душа. Відчуття щастя і все життя попереду. @igg{fbicon.face.smiling.eyes.smiling} 
На жаль, вже більше року не була у рідному місті...

\iusr{Юлія Ігорівна}
Обожаю Пущу, в любое время года там божественно красивая природа. Живу не далеко, на Куреневке.
\end{itemize} % }
