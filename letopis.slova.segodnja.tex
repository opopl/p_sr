% vim: keymap=russian-jcukenwin
%%beginhead 
 
%%file slova.segodnja
%%parent slova
 
%%url 
 
%%author_id 
%%date 
 
%%tags 
%%title 
 
%%endhead 
\chapter{Сегодня}
\label{sec:slova.segodnja}

%%%cit
%%%cit_head
%%%cit_pic
\ifcmt
  tab_begin cols=3
     pic https://img.strana.news/img/article/3631/evromajdanu-let-chto-77_main.jpeg
     pic https://strana.news/img/forall/u/0/92/%D1%82%D0%B0%D1%82%D1%8C%D1%8F%D0%BD%D0%B0(1).png
		 pic https://strana.news/img/forall/u/0/92/%D0%BA%D1%83%D0%BB%D1%8C%D1%87%D0%B8%D1%86%D0%BA%D0%B8%D0%B9.png
  tab_end
\fi
%%%cit_text
\emph{Сегодня}, 21 ноября, исполняется 8 лет со дня начала Евромайдана.  Протесты,
которые развернулись из-за срыва подписания Соглашения об ассоциации с
Евросоюзом, перешли к требованиям досрочного ухода с поста Виктора Януковича.
И быстро скатились к насилию.  Погибли десятки протестующих и силовиков, что
резко накалило все противоречия внутри страны.  После событий в Киеве началась
война, в ходе которой погибли более десятка тысяч украинцев. Утраченными
оказались территории Крыма и частично Донбасса, где до сих пор идут боевые
действия.  То есть Майдан изменил ход истории Украины.  "Страна" поговорила с
участниками тех событий, которые были по разные стороны баррикад. Они
рассказали, что происходило в те дни в Киеве и как они оценивают итоги Майдана
\emph{сегодня}
%%%cit_comment
%%%cit_title
\citTitle{Как убивали на Майдане и что он дал стране. Воспоминания участников событий 8 лет спустя}, 
Юлия Колтак, strana.news, 21.11.2021
%%%endcit
