% vim: keymap=russian-jcukenwin
%%beginhead 
 
%%file slova.moskva
%%parent slova
 
%%url 
 
%%author 
%%author_id 
%%author_url 
 
%%tags 
%%title 
 
%%endhead 
\chapter{Москва}

Болтовня о \emph{промосковскости} и \enquote{пропутинскости} Лукашенко есть полная ерунда,
поскольку все эти годы бацька только то и делал, что балансировал между \emph{Москвой}
и Западом, всячески уворачивался от цепких объятий \emph{Москвы} и лично Путина, с
которым у него очень непростые отношения, пытаясь сохранить максимум
самостийности, но не разрушить при этом, по украинскому образцу, экономику и
социалку, во многом зависящую от рынков бывшего Союза и вообще от традиционных
связей. Один только герб Советской Белоруссии чего стоит на фоне российского
триколора!,
\textbf{Европа впала в иррациональный истеричный лукашизм}, Александр Карпец,
strana.ua, 03.06.2021

Колектив Національного культурного центру України у м. \emph{Москві} вітає з Міжнародним днем захисту дітей!, 
\textbf{Наші діти!}, ukrcentr.ru, 01.06.2021

27 травня в Національному культурному центрі України у м. \emph{Москві} відбувся
концерт лауреата міжнародних конкурсів Української народної хорової капели
\emph{Москви}. На барвисту концертну програму «Пісенне джерело натхнення» завітали
шанувальники й друзі Української народної хорової капели \emph{Москви}, які протягом
багатьох років відвідують численні музичні вечори в Культурному центрі України
та, насамперед, щомісячний Український музичний салон. Відкриваючи концертну
програму творчого колективу, зі словами вдячності до гостей вечора звернулася
художній керівник та диригент капели, начальник служби з питань культури і
діаспори НКЦУ, заслужений працівник культури України та Росії, голова РГО
«Культурно-просвітницький центр українців у м. \emph{Москві}» Вікторія Скопенко,
\textbf{Пісенне джерело натхнення}, ukrcentr.ru, 28.05.2021

Конечно, мне было безумно страшно переезжать. Младшим двойняшкам на тот момент
исполнилось полгодика, третья дочь тоже была кнопкой, второй было 10, старшему
только исполнилось 12, вся эта «команда» была непригодна для бегства, мягко
говоря. Мы не могли взять с собой ничего, кроме детей, трое из которых ехали в
колясках,  и нескольких чемоданов с их же одеждой: чтобы нанять безумно дорогую
грузоперевозку, нужно было хотя бы знать, где мы будем жить. На пересадке с
самолёта на самолёт в Минске мы так долго заполняли декларации на всех семерых,
что чуть не опоздали на второй борт до \emph{Москвы}. Добрались до Ярославля
одним днём. Нам чудом удалось в тот же день подписать договор аренды
квартиры... и никогда не забуду, как я в этот первый день лежала в темноте на
голом матрасе и чужой подушке, рядом сопели младшие, а я смотрела широко
открытыми глазами в потолок. Скоро 40 лет, а приходится мне, многодетной
матери, всё начинать с нуля,
\textbf{Ярославль глазами Киевлянки},
odnarodyna.org, 03.06.2021

В Смутное время монастырь выдержал 16-месячную осаду польско-литовских войск,
подошедших к нему в сентябре 1608 года. В конце 1609 года в осаждённом
монастыре началась цинга, во время которой умерло свыше двух тысяч человек.
Людей, способных защищать монастырь осталось около 200. В январе 1610 года
осада была снята русскими войсками. Монастырские укрепление быстро восстановили
и при этом стены надстроили в высоту и увеличили в ширину, а башни обрели
дошедший до наших дней вид. Последний раз враги были под стенами монастыря в
1618 году, во время похода на \emph{Москву} польского королевича Владислава,
после чего наступило время процветания монастыря. В 1689 году в монастыре
укрывался спасшийся из \emph{Москвы} бегством Пётр I, и из него он уехал уже
полновластным правителем. При Петре I в обители появилась великолепная барочная
трапезная с храмом преподобного Сергия Радонежского, так называемая Трапезная
церковь. Это самая необычная церковь, которую я видел,
\citTitle{Место силы}, Ленинградский Наблюдатель., zen.yandex.ru, 03.06.2021

\emph{Москва} не может такого простить, потому что это наносит страшный удар по ее
государство- и нациеобразующей мифологии. Поэтому Россия всячески будет
дискредитировать такой законопроект о коренных народах. Ведь Путин мировым
лидерам говорил, что Украина – это \enquote{русская земля}, а украинцы и русские – это
\enquote{один народ}, тем более, что украинцев никогда не было и нет, есть только
\enquote{русские люди}, \enquote{русский язык}, да и вообще там, где Украина, \enquote{Русью пахнет}. А
тут вдруг оказывается, что русские не являются коренным народом... ,
\citTitle{Как законопроект о коренных народах Украины выбил почву из-под ног Москвы - Главред}, 
Григорий Перепелица, opinions.glavred.info, 10.06.2021

