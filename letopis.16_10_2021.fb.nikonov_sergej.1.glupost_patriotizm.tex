% vim: keymap=russian-jcukenwin
%%beginhead 
 
%%file 16_10_2021.fb.nikonov_sergej.1.glupost_patriotizm
%%parent 16_10_2021
 
%%url https://www.facebook.com/alexelsevier/posts/1604854733193169
 
%%author_id nikonov_sergej
%%date 
 
%%tags deti,glupost,nacionalizm,patriotizm,shkola,ukraina
%%title Публикую репост, чтобы глупость каждого была видна. Нас этим хотят приучить к патриотизму
 
%%endhead 
 
\subsection{Публикую репост, чтобы глупость каждого была видна. Нас этим хотят приучить к патриотизму}
\label{sec:16_10_2021.fb.nikonov_sergej.1.glupost_patriotizm}
 
\Purl{https://www.facebook.com/alexelsevier/posts/1604854733193169}
\ifcmt
 author_begin
   author_id nikonov_sergej
 author_end
\fi

Публикую репост, чтобы глупость каждого была видна. Нас этим хотят приучить к
патриотизму. Думают впечатлить.Благодарят родителей за патротическое
воспитание. Там есть просьба в тексте о репосте чтоб враги увидели. Мол,
убедить, что мы стойкие, боремся с противником, поколение растет патриотов
Украины, сочувствующее раненым.  Хотят с помощью репоста убедить. Но китайская
пословица гласит: "Тысячу способов узнать легко, одного результата добиться —
трудно". И вот результат в моем случае, так как я вижу следующее. Ну и какое
поколение у нас растет вообще? У нас на Украине в городе Первомайск смертность
в два больше, чем рождаемость. И так повсюду по стране. Благодаря политике того
режима, который ВСУ защищают наряду с территорией страны. И что у нас враги. А
то, что они такие же люди как мы с нашим гражданством в основном, это не
покажут. Как и разведданных про регулярное присутсствие русских войск вообще.
Не то, что составляющих большинство в рядах противной стороны. Чему, чему, а
мышлению у этих взрослых учиться не стоит. Ещё раз какое поколение?
Великолепно, что это поколение любит все украинское и сочувствует раненым,
уважают тех, кто по внушению взрослых или по их собственному мнению, отдает
жизнь за Украину. Это наша страна и это Человечность и Любовь к стране с
большой буквы. Но, по факту, за что раненым сочуствовать? Пропаганда врала и
врет так неумело, что простой логики достаточно, чтобы взрослые ей не верили,
но пошли в АТО, думали, что защищать. Разве что, как к жертвам войны.  Да, шли
защищать земли от проекта Новороссии, земли, которые не поднялись на войну
против режима Турчинова и Порошенко. Это выбор этих земель. А ЕЩЁ НЕОСОЗНАННО
НАПАДАТЬ НА ДОНЕЦК И ЛУГАНСК. КОТОРЫЙ И НЮХОМ СЛЫШАТЬ ОБ ЭТОЙ ВЛАСТИ НЕ ХОЧЕТ.
Это их выбор..

Поколение растет по человечески хорошее, отличное, любимое мной, но обманутое
и не наученное думать учителем, школой, государством и родителями. Родители
воспитали патриотов, а научили ли они думать  и понимать других людей? Это
родители организовали. А они подумали, что с той стороны таких раненых нету?
Что у нас Минские соглашения? Откуда тогда раненые,  если соглашения есть? А,
наверное только враги нарушают... А Интернета и размышлений нет у взрослых?
Почему учительница не использует Интернет, чтобы дети общались с детьми из
Донецка и Крыма? И ТАМ ДЕТИ ВЕЛИКОЛЕПНЫ. И МОЖНО ПРОСТО ЗАПРЕТИТЬ ОБЩАТЬСЯ ПРИ
КОНТАКТАХ НА ТЕМУ ВОЙНЫ. ВАЖЕН КОНТАКТ. НО К РАНЕНЫМ ОНА ВЕДЕТ, А ТУДА НЕТ. И
ещё дети во Львове в 2014 ходили на мирные детские площадки. А дети в Горловке
сидели в подвалах. Они держат в рукуах книги. А деревенский мальчик из
фронтового поселения -  снаряд. Так может не стоит сочувствовать этим раненым
или сочувствовать не только. Кстати, это не Баканов на видео в очках? Тогда
это мероприятие - так, хорошая картинка для лукавого и кровавого обмана.
Кровавого потому, что говорят: "продолжайте воевать!" С вымышленным врагом...
Автор поста хочет, чтобы это  видели "враждебные соседи". Для нее. Чтож,
русские, белорусы и непокорные жители Донбасса смотрите, какими не должно быть
государство и школа, учитесь Человечности у этих прекрасных наших украинских
детей и учитесь мышлению, сопоставлению фактов с моральными установками. Разум
должен быть взвешенным, а мораль чувственной и прекрасной. На основе разума. А
Вы, украинцы, избагайте таких школ и учителей или учите детей сами. Пусть они
общаются с этими великолепными первоклашками. Но осторожно ибо рядом с этими
детьми обманутые или гнилые взрослые. Сдадут, настучат. Кстати, есть такой
термин 1937 - "враг народа". Его я видел в постах патриотов. И них
декоммунизация, как борьба с тоталитаризмом.  Себе перечат. Не с
тоталитаризмом они борятся, а с иным взглядом на жизнь. Ибо в нормальном,
мирном, стабильном обществе взрослым детям из радикальных организаций места
нет.

\url{https://www.facebook.com/nataliiayusupova/posts/2066955560128637}

\ii{16_10_2021.fb.nikonov_sergej.1.glupost_patriotizm.cmt}
