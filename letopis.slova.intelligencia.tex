% vim: keymap=russian-jcukenwin
%%beginhead 
 
%%file slova.intelligencia
%%parent slova
 
%%url 
 
%%author 
%%author_id 
%%author_url 
 
%%tags 
%%title 
 
%%endhead 
\chapter{Интеллигенция}

%%%cit
%%%cit_head
Судний день для української інтелігенції
%%%cit_pic
%%%cit_text
Насправді плівки генерала Кондратюка (більш відомі як \enquote{плівки Медведчука} чи
\enquote{плівки Бігуса}) – це судний день для \emph{української інтелігенції}. Вона настільки
довго і наполегливо плекала ідентичність коштом етичності, що тепер не може
оговтатись від того, яке лайно ховалось під оспіваною ними вишиванкою. І за
інерцією досі виправдовує негідника тільки тому, що він – \enquote{свій}.  \enquote{Сучий син,
але свій сучий син} – погана максима для тих, хто претендує бути совістю і
розумом нації. Олександр Доній слушно наголошує, що єдиний шанс перемогти для
слабших – це мати за собою моральну правду. Саме завдякі моральній перевазі
української справи українство стало вибором для багатьох, хто не був етнічним
українцем: від Липинського і Фітільова до Жизневського і Нігояна.  Втративши
моральну перевагу у нашій борні проти русского мира, в якому \enquote{ніщо не правда і
все можливе}, в якому \enquote{русские бабы еще нарожають}, ми виймаємо наріжний камінь
з української справи. Без моральної переваги вона просто розсипається і
перетворюється на \enquote{техас мають грабувати техасці}. От тільки як відомо техаська
вольниця тривала недовго...
%%%cit_comment
%%%cit_title
\citTitle{Трагедія української інтелігенції}, Генадій Друзенко, 
analytics.hvylya.net, 14.06.2021
%%%endcit

%%%cit
%%%cit_head
Час повертати на прапор моральну правду української справи
%%%cit_pic
%%%cit_text
Українській \emph{інтелігенції} на часі пригадати мудрі слова одного зі своїх
найкращих представників Бориса Антоненка-Давидовича: «зло, навіть під
українським національним прапором, виголошене українською мовою, все ж
лишається злом...».  Зацикленість на національній ідентичності зіграла з
\emph{українською інтелігенцією} злий жарт. Виявилось, що під вишиванками може чудово
ховатись моральний розклад та абсолютна безпринципнцість. Час повертати в
політику мораль. Час боротись не тільки за своє, а за етичну вищість цього
свого перед чужим. Час повертати на прапор моральну правду української справи.
Але це апріорі неможливо, доки \emph{українська інтелегенція} не визнає, що 2019 вона
помилилась...
%%%cit_comment
%%%cit_title
\citTitle{Трагедія української інтелігенції}, Генадій Друзенко, 
analytics.hvylya.net, 14.06.2021
%%%endcit

