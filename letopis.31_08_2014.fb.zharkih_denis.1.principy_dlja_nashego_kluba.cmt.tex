% vim: keymap=russian-jcukenwin
%%beginhead 
 
%%file 31_08_2014.fb.zharkih_denis.1.principy_dlja_nashego_kluba.cmt
%%parent 31_08_2014.fb.zharkih_denis.1.principy_dlja_nashego_kluba
 
%%url 
 
%%author_id 
%%date 
 
%%tags 
%%title 
 
%%endhead 
\subsubsection{Коментарі}

\begin{itemize} % {
\iusr{Алексей Столяров}

Разница в том, что Сталин не был властью. Он был как раз управленцем.

\iusr{Алексей Большаков}

Для соблюдения этого принципа власть не должна прийти с помощью военного
переворота. Т.к. при подобном сценарии уже изначально человеческая жизнь теряет
свою ценность. Если люди готовы жертвовать своими гражданами и при этом
лицемерно захватить результаты такого кровавого переворота без делегации им
прав самим народом, который совершил переворот. Разве это только я видел, что
сам майдан в начале был не доволен, что три богатыря проталкивали себя как
главные переговорщики и вещатели гласа народа. Реально они таковыми не были, но
сам народ не смог четко сформулировать свои требования и выдвинуть из своих
масс новых лидеров. А три богатыря проявили настойчивость, изворотливость и
терпение. Т.е. была совершена афера. Так чего же от этой власти ждать? Она и
продолжает преследовать свои корыстные интересы жертвуя при этом опять таки
человеческими жизнями. При этом с помощью пропаганды эта власть сформировала
новое чучело врага и уже многие из народа ведутся на эту ложь и совсем забыли
зачем стояли на майдане. Но я продолжаю верить в наш народ и думаю, что это
временное умопомрачение. Он опомнится и встряхнет прихлебателей. Выборы должны
расставить все по местам. Если мы изменились, то изберем новых и ни одна старая
партия с старыми лицами не пройдет в парламент. Мы получим новый и он
сформирует проукраинское правительство и мы засучив рукава от слов перейдем к
реальному преобразованию страны.


\iusr{Лариса Шумило}
Вчера почла статью с мыслью, что вся Украина - это майдан

\iusr{Денис Жарких}
Алексей, какие еще выборы? Они ж через два месяца состоятся. Там уже все расписано.

\iusr{Светлана Рыжова}

...не словом, а делом утвержаться во власти. двигателем должна быть искренняя
любовь к людям и к своей стране... основа- справедливость и человеколюбие.
Знание законов развития общества и истории необходимо для анализа происходящего
и планирования будущего. Очень важно всегда дружить с наукой Логикой... уффф..
как сложно управлять государством....
\iusr{Алексей Большаков}

Я понимаю. Сейчас новые технологии выборов, когда через сми нам расписывают
всех возможных кандидатов. Но там никогда нет наших народных. Но народ же
должен поумнеть и хотя бы раз взять и проголосовать не по этим рейтингам. Ну
хоть бы за Рабиновича или Богомолец. А как по другому? Надежда всегда есть). До
объявления результатов выборов...)))


\iusr{Денис Жарких}
А нам не хватает Рабиновичей во власти? Вон его радио постоянно его расхваливает.

\iusr{Ольга Чуксина}

Согласна. Человек и его жизнь в последнее время вообще обесценились. Лозунг
"Украина понад усе", что это? Сколько еще людей должно погибнуть, чтоб пришло
это "понад усе"? Еще зимой заметила, что у людей отсутствует инстинкт
самосохранения.


\iusr{Алексей Большаков}

Денис, какое Ваше предложение. Ведь все старые намного хуже новых. У нас и так
законодательно очень сложно даже участвовать в выборах. Он по крайней мере не
был и говорит сам уйдет, если не выполнит обещаний. Пока сам народ не поймет
чего хочет и не будет выбирать сознательно, ну хотя бы половина, ничего не
произойдет!

\iusr{Денис Жарких}
Нужно усовершенствовать украинскую политическую систему.

\iusr{Игорь Семичастный}
Принципы хорошие, лично для меня они приемлемы. Я с ними согласен полностью.

\iusr{Алексей Большаков}

Как, если парламент не народный. Они что ли усовершенствовать будут. Я Леониду
Черновецкому предложил реформы для страны на конкурсе. Он от моих реформ как от
ладана....

\iusr{Денис Жарких}
Нужно было план предлагать. А Вы с бумажками ходили..

\iusr{Алексей Большаков}
Я, к сожалению, очень ограничен в хождении. А план я и предлагал. Не было Денис бумажек)))

\iusr{Денис Жарких}
И он его выкурил?

\iusr{Алексей Большаков}
Вы шутите, но он его по иному использовал)))

\emph{Вадим Богачук}

Полностью поддерживаю. Этот принцип приоритета личности, ее выбора, а не
"Украина - понад усе" отчасти содержится в ст. 3 Конституции Украины. При том,
что понятие "здоровье" должно учитывать благополучие не только физическое, но и
душевное и духовное (как записано в прамбуле ключвого документа ВООЗ).


\iusr{Елена Елена}

Согласна, сбежав от АТО, попала в Москву, было воскресенье, многолюдн. на Кр. пл, наша
невойна и там дает о себе знать, приятно удивилась, что так тщательно в этой
стране меня охраняли, где сутки назад хотели убить, но той стране я родила
детей, отдала 30лет работы во благо, а для этой я просто русская ,когда-то
родившаяся а Ростове

\iusr{Ирина Бабенко}

Первому принципу - Да) А вы думаете , такая система может поддаться коррекции
?)) Мы же видим, что воля протестующих безнаказанно подавляется самыми жесткими
средствами. В Киеве пропаганда, в результате - много злых и чокнутых, на
востоке - вообще геноцид. Украинская полит.система прямыми наводками наносит
удары по жилым кварталам. Возможности сесть за стол переговоров были же, но
дискутировать Петро не собирается

\iusr{Елена Елена}

А зачем теперь Ю-В эти перегов? разрушил жизнь людей Наполеон херов, а теперь
понял АТО не то

\iusr{Ирина Бабенко}

Теперь да , уже и сами не хотят. Но раньше -то можно было. когда были за
автономию в составе Украины. Лен, наполеоном у нас ВВП называют ))

\iusr{Елена Елена}

А я указала какой именно, это сразу сзади от ВВП

\iusr{Жанна Худа}

Алексей, не нужно переоценивать мыслительные способности людей, за последние
полгода я прозрела как можно легко ими управлять, особенно когда это толпа. Да
и выборы та же история, достаточно месяц-полтора 2 разв в неделю вывешивать
рейтинги и кандидат-лидер уже нарисовался и большинство уже не сомневается что
выбирает мессию, главное надавить на нужные точки у электората. Спецам за
формирование "правильного" общественного мнения сотни тысяч долларов платят.

\iusr{Sergey Serous}

вот докаркаемся мы, хунта, хунта, а ведь реально, чтобы вывести украину из
текущего (во все стороны) дерьма нужна будет настоящая хунта, к власти должен
будет придти человек с рукой в ежовой рукавице, который введет строгий режим и
железный порядок. никакой воли и майданов, всех в форму и пайки. для внешнего
врага (как же без него) выбрать, что-то более реальное чем громадная Россия,
Польшу например. Ввести военный коммунизм.


\iusr{Александр Кузнецов}
Осталось опять собрать вече и пригласить варягов. Вот только захочет ли кто?

\iusr{Sergey Serous}
Славяне не от хорошей жизни пригласили Рюрика, а чтоб не перебить друг дружку ...

\iusr{Александр Кузнецов}
А мы, типа, дружный и сплоченный коллектив? И жить в нашей стране - сплошной рай. Зае...ли наши князья престол делить.

\iusr{Светлана Турчин}

Пока престол будет существовать в таком виде, как сегодня, всегда найдутся
охочие его делить. Надо нагрузить на этот престол огромную гору обязательств,
ответственности и общественного контроля, чтоб каждый желающий вступить сто раз
задумался, а надо ли оно ему.

\iusr{Лада Куликовская}
Все верно

\iusr{Снежана Киевская}
С первым принципом согласна на 100\%.

\iusr{Игорь Лобода}
Согласен.


\end{itemize} % }
