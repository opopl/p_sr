% vim: keymap=russian-jcukenwin
%%beginhead 
 
%%file 29_11_2021.fb.fb_group.story_kiev_ua.1.albom_gitlera_hudozhnik_gluschenko
%%parent 29_11_2021
 
%%url https://www.facebook.com/groups/story.kiev.ua/posts/1808040202726106
 
%%author_id fb_group.story_kiev_ua,fedjko_vladimir.kiev
%%date 
 
%%tags gluschenko_nikolaj.hudozhnik.ukr,hitler_adolf,isskustvo,kiev,sssr
%%title Художник Микола Глущенко і «Альбом Гітлера»
 
%%endhead 
 
\subsection{Художник Микола Глущенко і «Альбом Гітлера»}
\label{sec:29_11_2021.fb.fb_group.story_kiev_ua.1.albom_gitlera_hudozhnik_gluschenko}
 
\Purl{https://www.facebook.com/groups/story.kiev.ua/posts/1808040202726106}
\ifcmt
 author_begin
   author_id fb_group.story_kiev_ua,fedjko_vladimir.kiev
 author_end
\fi

Художник Микола Глущенко і «Альбом Гітлера».

***

Навесні 1975 року автору цих рядків зателефонував Володимир Павлович Цельтнер,
відповідальний працівник апарату Спілки художників УРСР, і повідомив, що до
75-річчя М. П. Глущенко [1] готується ювілейна виставка. Мені і художнику
Борису Харіку доручається зробити каталог його робіт.

Сонячним ранком наступного дня ми прогулювалися біля висотної будівлі на
Хрещатику в очікуванні Майстра. Хоча я і був знайомий з Миколою Петровичем десь
з 1971 року, але в його майстерню ще жодного разу не був запрошений,
зустрічалися, в основному, в Спілці художників на Львівській площі або на
вернісажах.

Микола Петрович підійшов до нас з величезним букетом квітів, явно призначеним
для написання чергового етюду. Привіталися і на ходу розмовляючи, рушили в
майстерню, яка розташовувалася на дванадцятому поверсі будинку по вул.
Хрещатик, 25, в горищному приміщенні будівлі (сам Майстер називав її мансардою,
на французький манер).

Майстерня була невеликою, але дуже високою, метрів сім-вісім до стелі.
Екзотичне переплетення труб опалення навівало асоціації з лабораторією
алхіміка... Антресолі в кілька поверхів, забиті завершеними полотнами в рамах.
На двох стінах – почорнілі полотна фламандських і нідерландських живописців, а
також німецьких і французьких імпресіоністів; пожовклі від часу гравюри,
виконані в традиціях Клода Лоррена і Джованні Тьєполо. Старовинний малюнок
оголеної натурниці, кілька керамічних плиток українського народного майстра
Бахматюка. Біля вікна журнальний столик і два досить обдертих крісла. Біля
стіни, в кілька рядів – картини, більшість на підрамниках, без рам. Всі
повернені живописом до стіни. Колекція писанок на полиці, і дуже багато
зарубіжних книг та журналів з мистецтва. В центрі кімнати старий мольберт, на
який під час сеансу перегляду робіт кріпилася прекрасна французька рама XVIII
століття.

Поки ми оглядалися, Микола Петрович зварив міцну каву, відкрив коробку
шоколадних цукерок та гарний коньяк. За кавою ми з Борисом розповіли про
поставлене нам завдання по створенню каталогу робіт Майстра, я підготував
фотоапаратуру для зйомки.

***

Глущенко писав етюд, енергійно і впевнено кидаючи на полотно широкі яскраві
мазки, а я фотографував його за роботою з різних точок, щоб потім вибрати
найкращі знімки, що найбільш повно відображають Майстра, який творить…

Так минуло кілька годин, протягом яких ми неодноразово смакували кавою з
коньяком і вели розмови про мистецтво. У бесіді я згадав, що завершую роботу
над каталогом одного відомого художника-графіка, який, перебуваючи в похилому
віці і знесилений хворобами, знаходився в цей час на лікуванні у Феофанії.
Кожного дня, виходячи на прогулянку, він малював на стіні одного з лікарняних
об'єктів відверто порнографічні малюнки. Після того, як графік повертався у
палату охорона ці малюнки замазувала, а художник під час наступної прогулянки
малював нові... Оскільки цей художник-графік був дуже іменитий, то ніяких
виховних заходів до нього застосувати було неможливо. Прізвище цього художника
я не називаю з етичних причин, але той, хто знайомий з художнім світом Києва
середини 70-х років, без особливих зусиль зрозуміє, про кого я говорю.

І тут Микола Петрович несподівано сказав, що ім'ярек був прекрасним художником
до тих пір, поки не зайнявся політикою, а зайнявшись політикою – помер як
художник. Помовчавши з пів-хвилини, Глущенко продовжив: «Якби Гітлер не
зайнявся політикою, світ мав би прекрасного архітектора і художника» [2]. Це
було настільки несподівано, що ці слова Майстра врізалися в мою пам'ять, і я
дослівно пам'ятаю їх до сих пір. У той час мої пізнання про Адольфа Гітлера
обмежувалися коментарями замполіта про період Великої Вітчизняної війни, під
час моєї служби в армії, і фільмом «Звичайний фашизм» Михайла Ромма. Побачивши
здивування на наших обличчях, Глущенко розповів про акварелі і архітектурні
проекти Гітлера, які він бачив на виставці в Кремлі [3], а на завершення
приголомшив визнанням, що Гітлер подарував йому альбом факсимільних репродукцій
своїх акварелей. Продовжуючи свою розповідь, Микола Петрович розповів, що в
1940 році він відповідав за радянську художню виставку, що експонувалася в
Берліні. Під час церемонії закриття виставки до нього підійшов Ріббентроп і
сказав, що Фюреру дуже подобаються роботи Глущенко, і Гітлер вважає його
найкращим пейзажистом Європи. Тому Гітлер вирішив подарувати Глущенко альбом з
репродукціями своїх робіт і своїм автографом. Після цього Ріббентроп вручив
Глущенко альбом, на першій сторінці якого була дарчий напис: «Ніколасу Глущенко
Адольф Гітлер» і дата. А від імені Міністерства закордонних справ його
нагородили Почесною грамотою.

Через деякий час після повернення в СРСР до Глущенко з'явився високопоставлений
співробітник держбезпеки, який передав прохання Сталіна: дати альбом для
ознайомлення членам Політбюро і Сталіну. Через деякий час альбом через того ж
співробітника НКВД (прізвище Глущенко не назвав) повернули. Вдруге альбом
зажадали після початку війни, нібито для ознайомлення з ним Кукриніксів. Через
тривалий час альбом повернувся до власника. Про це Глущенко розповідав так:

«Ніч... Гучний і вимогливий стукіт у двері... Відкриваю... Біля дверей
полковник НКВС і два автоматники... Перша думка – все, арешт! Ноги ватяні, очі
застилає туман... Полковник простягає пакет і каже: «Наказано повернути!
Розпишіться!»... Беру пакет, розписуюся, полковник козиряє, повертається і
йде... За ним слідують автоматники».

Більше альбомом, за словами Глущенка, ніхто ніколи не цікавився.

***

При нашій наступній зустрічі Микола Петрович показав нам цей альбом!

Глущенко вважав подарунок дуже цінним і зберігав альбом все життя!

Після смерті Майстра альбом зник! Зникли також і почорнілі полотна фламандських
і нідерландських живописців, а також німецьких і французьких імпресіоністів;
пожовклі від часу гравюри, виконані в традиціях Клода Лоррена і Джованні
Тьєполо. 

Ціна цим творам мистецтва – мільйони доларів!

Хоча Гітлер і є одіозною фігурою, в той же час, альбом репродукцій його робіт
на сьогодні коштує за мільйон доларів! Були спроби розшукати зниклий альбом. 

Пошуками «Альбому Гітлера» займався Віктор Тригуб, головний редактор журналу
«Музеї України», про що розповідав у своїх публікаціях.

Дещо про долю «Альбому Гітлера» повідала і журналістка Катерина Каменська у
публікації «Картины Глущенко, учившего рисовать Гитлера, хотят отсудить у
Украины».

Одразу зауважу, що твердження «Глущенко навчав Гітлера малювати» явно придумане
журналісткою заради ефектності статті – Глущенко і Гітлер до 1940 року ніколи
не зустрічалися! Гітлер познайомився з творчістю Миколи Глущенка побачивши його
картини у виставковій галереї вже після від’їзду Миколи Петровича у Францію.

Щодо ж долі «Альбому Гітлера» після смерті Глущенко, то Каменська розповідає
таке:

«Куда подевался знаменитый альбом с акварелями фюрера?

«Комсомолка» нашла человека, который держал в руках альбом с акварелями
Гитлера, которые фюрер подарил Глущенко. Профессор академии художеств Украины
Михаил Александрович Криволапов, в послевоенные годы работавший начальником
управления изобразительных искусств и главным экспертом Минкульта УССР, часто
общался с Глущенко по работе, позже они стали приятелями.

- Видите этот пейзаж? Мне его Николай Петрович подарил на 40-летие. Он был
гениальным художником, остроумным и очень обаятельным человеком, - вспоминает
профессор Криволапов.

- Николай Петрович что-нибудь рассказывал о своей разведдеятельности?

- Нет, что вы! Даже когда мы устраивали застолье, ни разу не проговорился.
Кремень-человек был.

- Говорят, у него уроки рисования брал даже Гитлер…

- Сначала они оба учились у одного австрийского профессора Ашбе. Вполне
вероятно, Николая Петровича Гитлеру тот же профессор и порекомендовал…

Кстати, после смерти Николая его жена - Марья Давыдовна - попросила меня
сохранить альбом акварелей Гитлера. Помню, когда я взял его - он просто жег мне
руки! От этих акварелей исходила какая-то зловещая энергетика!

Альбом был небольшой, размером с книгу. Там было более 30 рисунков - в основном
наброски архитектурных элементов, мосты разные. Чувствовалось, что автор любит
монументальные вещи.

Я побежал к министру культуры Алексею Романовскому: «Что делать с этим
альбомом?» В итоге его отдали секретарю ЦК по идеологии Александру Капто. А его
тогда собирались переводить в Москву. Он и забрал альбом. А что дальше с
акварелями случилось – неизвестно».

***

На початку 1990-х, коли розвалився СРСР, з журнальної публікації я дізнався, що
Микола Глущенко з 1920-х років працював на радянську розвідку (оперативний
псевдонім «Ярема») і був дуже результативним розвідником [4]. 

***

Посилання:

[1]. Мико́ла Петро́вич Глу́щенко (4 (17) вересня 1901, Новомосковськ, Україна — 31
жовтня 1977, Київ, Українська РСР, СРСР) — український радянський художник
(автор портретів і краєвидів), радянський розвідник (псевдо «Ярема»). 

У 1918 закінчив комерційне училище в м. Юзівка (нині Донецьк). Був
мобілізований до Добровольчої армії Антона Денікіна. Разом із нею відступив за
кордон і був інтернований на території Польщі. Після втечі з таборів для
полонених дістався Німеччини.

Опановував основи мистецтва в Школі-студії Ганса Балушека (Берлін). У 1920–1924
рр. продовжив навчання в Берлінській вищій школі образотворчого мистецтва у
Шарлоттенбурзі. Як виходець із України мав матеріальну підтримку від
української еміграції різних політичних орієнтацій. Навчання оплачував гетьман
П. Скоропадський, кошти на прожиття надавав колишній представник УНР в
Німеччині Роман Смаль-Стоцький, а першу персональну виставку робіт допоміг
організувати в Берліні Володимир Винниченко.

У 1925 році Микола Глущенко перебрався з Берліна до Парижа, відкрив художнє
ательє на вулиці Волонтерів, 23, яке відвідували представники української та
російської еміграції.

До 1936 мешкав у Франції та Іспанії. Займався живописом, робив портрети діячів
прокомуністичної орієнтації (Ромен Роллан, Анрі Барбюс та інші). У столиці
Франції проходили його виставки. 

У 1936-му, отримавши довгоочікуваний дозвіл на повернення в СРСР, виїхав із
дружиною до Москви. Перед від'їздом із Парижа дістав від Володимира Винниченка
листа такого змісту: «Ваша религия, Николай Петрович, ваше самое дорогое и
„святое“ — это материальный интерес… Я пишу вам по-русски. Пишу так, потому что
в глубине души я не считаю вас настоящим украинцем. Называете вы себя украинцем
не так, как должен называть себя член угнетенного коллектива, а только тогда и
там, где это вам выгодно…».  

[2]. В цілому, по ряду оцінок, Гітлером як художником було створено близько
3400 картин, ескізів та малюнків.

[3]. Політичне, економічне і військове зближення СРСР і Німеччини у 1938-1939
рр., завершене підписанням 23 серпня 1939 року Угоди про ненапад між Німеччиною
та Радянським Союзом, розгром та окупація Польщі, не оминули своєю увагою і
культурне життя обох країн. На початку 1940 року відомому радянському художнику
Миколі Глущенко було доручено налагодити контакти з головним референтом з
культурних зв'язків штабу Ріббентропа – Петером Клейстом і запропонувати йому
обмін виставками художників в Москві і Берліні. Під час перебування Клейста в
Москві Глущенко був дуже люб'язно їм прийнятий. У бесіді Клейст показав свою
обізнаність з творчістю художника, а також виявив серйозну зацікавленість в
налагодженні культурних зв'язків Третього Райху з Радянським Союзом. 

Задум вдалося реалізувати досить швидко. Виставка німецького образотворчого
мистецтва в Москві пройшла успішно. У посольстві Німеччини відбувся обід, куди
Клейст запросив Глущенко та інших радянських діячів культури. Тут же він
познайомив Миколу Петровича зі своїм заступником Шутте і вони обговорили деталі
організації в Берліні виставки «Народна творчість СРСР».

Вже 17 квітня Глущенко виїхав до Берліна на відкриття радянської виставки.
Експозиція мала великий успіх. В останній день в виставковий зал завітало вище
керівництво Третього Райху. Очолював делегацію міністр закордонних справ Йоахім
фон Ріббентроп. 

Після короткої промови про зміцнення і розширення культурних зв'язків між
Німеччиною і Радянським Союзом Ріббентроп, звернувшись до Миколи Петровича,
подякував за активну роботу в організації радянської експозиції та сприяння в
популяризації робіт німецьких художників в СРСР. Потім Ріббентроп сказав, що
Фюрер, як живописець, високо цінує талант Глущенко і вважає його одним з кращих
пейзажистів Європи. А на згадку про перебування в Берліні і в знак подяки за
внесок у розвиток дружніх відносин між Німеччиною і Радянським Союзом дарує
альбом з факсимільними репродукціями своїх акварелей. Від свого імені
Ріббентроп вручив Глущенко Почесну грамоту.

[4]. Згідно різних публікацій, радянська розвідка одержала від Глущенко цілком
таємні креслення на 205 видів військової техніки, зокрема авіаційних моторів
для винищувачів.

***

З рукопису спогадів «Погляд у минуле», написаних для особового фонду Федька В.Ф
у Державному архіві Київської області.
