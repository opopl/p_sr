% vim: keymap=russian-jcukenwin
%%beginhead 
 
%%file 26_02_2022.fb.baumejster_andrej.kiev.filosof.1.proshla_esche_odna_noch.cmt
%%parent 26_02_2022.fb.baumejster_andrej.kiev.filosof.1.proshla_esche_odna_noch
 
%%url 
 
%%author_id 
%%date 
 
%%tags 
%%title 
 
%%endhead 
\zzSecCmt

\begin{itemize} % {
\iusr{Sergey Titov}
Андрей, дорогой, береги себя! Ангела Хранителя

\iusr{Андрей Баумейстер}
\textbf{Sergey Titov} Серёжа, спасибо!

\iusr{Виктор Каган}
Держу кулаки, Андрей!

\iusr{Лора Чернышова}

В бомбоубежище мы разговариваем на двух языках. Никого не волнует мой русский
или русский ребят из Донецка, слава богу. Подвал - лучшее уравнивающее и
мультикультурализующее средство  @igg{fbicon.smile} 

\iusr{Лев Малхазов}

Русский язык - это язык Пушкина, а не Путина. Но я прекрасно понимаю чувства
людей, отказывающихся на нём говорить. Тем не менее, если Украина его сохранит,
а не отдаст, это будет ещё одна победа. Победа великодушия.

\begin{itemize} % {
\iusr{Наталія Єпіфанова}
\textbf{Lev Malkhazov} зберігати щось вороже це не великодушно, це нечистоплотно

\iusr{О. Олег Кобель}
\textbf{Lev Malkhazov} він збережеться, язик. В бібліотеках.
\end{itemize} % }

\iusr{Яна Прозорова}

Благодарю за ваши слова! Важные слова поддержки! Надеюсь, что так и будет. Верю
в Большую Украину. @igg{fbicon.heart.red}

\iusr{Елена Колтунович}
Мужественно! Благодарю!
Поддерживаю!
Ращу людей, способных построить большую Украину!

\iusr{Iryna Pukhta}

Я би сказала, що це Росія присвоїла собі частину української історії і на цьому
грунті збудувала собі міф \enquote{великої Росії}. А ще мені страшенно цікаво, які саме
смисли і цінності російської культури такі дорогоцінні для вас, що ви готові за
них триматися до останнього? \enquote{Тварь я дрожащая иль право имею?} - ось основний
меседж російської культури, який зараз демонструє фсбешна російська влада (не
один путін) в україні.

\iusr{Kira Savy}

Благослови вас Господь за эти слова! Мир уже не будет прежним, война в каждом
доме и в каждом сердце. Это час испытания веры, мужества,
человечности, час выбора. Мы молимся за вас неустанно.

\iusr{Татьяна Конькова}

Нет ничего более утешительного, чем то, что Вы пишете. Я стыжусь писать слова
поддержки, хотя с 2014 выходила в поддержку Украины.

Удивляюсь тому, что многие украинцы все ещё не ненавидят нас всех. Как можно
найти в сердце, полном страха и ненависти к захватчикам, место для понимания и
сострадания к нам - тем, кто ненавидит путинский режим и осуждает агрессию
против Украины? Восхищаюсь украинцами, плачу, молюсь о вашей победе и свободе.

\iusr{Лора Чернышова}
\textbf{Татьяна Конькова} спасибо

\iusr{Любовь Терехова}

Андрій Олегович, даруйте, зараз би не чекала, що час говорити про російську
мову й культуру. Стоїть питання про виживання української культури й нас як
українців. Росіяни й так, появляючись в кожному чаті, переводять все на себе й
отримають оплески як герої тільки тому, що пишуть, що вони окремо од Путіна й
проти війни і все - всі в восхіщєнії. Я два дні пишу відповідь спільній е-мейл
розсилці міжнародної олімпіади з філософії, де всі восхіщаються росіянкою, яка
написала, що вона проти війни, а україняцям вона каже \enquote{say to forgive}.
Нормально? Даже не \enquote{ask to forgive}.

І всі ой, яка молодець, ти так бідна страдаєш. Цитують Канта і Фрейда, стєнають
про долю інтелектуалів. І тільки одна людина з понад 100 у цьому чаті написала,
а як там українки? Написала в особисті, по-тихому. Бо велика і могуча руська
культура не сильно пострадає, вони дуже добре вміють робити з себе жертв, і
робиби все про себе.

\iusr{Елена Постольник}
Поддерживаем!
Молимся!
Верим в большую Украину!

\iusr{Alexey Burov}
Мы скоро встретимся в победившей Украине, Андрей, да поможет ей Бог. God bless you.

\iusr{Любовь Терехова}
Ось, будь ласка, про те, що українська зараз мова безпеки:
\url{https://www.facebook.com/100001515094776/posts/5058411000886020/}

\iusr{Игорь Чумак}

Андрей Олегович, спасибо вам за здравые смыслы ваших меседжей. Эта трагедия,
которую мы сейчас переживаем, переплавила нас всех разных в единый национальный
сплав, и заставила нас осознать ценность свободы во всех её проявлениях.

Уверен, её вкус мы уже никогда не забудем.

\iusr{Ольга Квасницкая}
Берегите себя! Вы нужны нам! Нужны Украине! Молимся

\iusr{Lara Alia}

Андрей, хоть мы с вами и не читали Эммануэля Левинаса, но его ассиметричная
этика произвела на меня сильное впечатление. Поэтому последующие дни сделаю
себе практику ассиметричной этики. Это не просто. Но увидеть свет в другом
возможно только через Его Свет. Я очень надеюсь. Дай Бог нам всем любви и сил,
чтобы сохранить человеческое достоинство.

\iusr{Igor Manannikov}
Все верно!

\iusr{Строцев Дмитрий}
Спасибо! Храни Господь!

\iusr{Наталья Прижилевская}
Спасибо, Андрей Олегович!
Слава Украине и ее Героям!

\iusr{Liubomyr Fedoriv}

Нет! Вьі ошибаетесь, профессор! Русская культура и русская речь и все русское в
ответе за то что происходит, ибо они отвечают за тех, кого воспитали!!! А о том
что \enquote{культура без границ} путин так же говорил и решил присвоить нас себе!
По-моему, смортя на бои за окном, пора пересмотреть свою концепцию.

\begin{itemize} % {
\iusr{Maria Lypych}
\textbf{Liubomyr Fedoriv} в ідеалі можлива «культура без границ», але з таким гібридним сусідом-ворогом на мою думку це нереально...

\iusr{Liubomyr Fedoriv}
\textbf{Maria Lypych} якщо цим гаслом прикриваються варвари, то це велика небезпека!

\iusr{Maria Lypych}
\textbf{Liubomyr Fedoriv} так, велика небезпека

\iusr{Maria Lypych}
\textbf{Liubomyr Fedoriv} і це не надзусилля говорити українською, перший час помічається, а потім привичка...
\end{itemize} % }

\iusr{Vladimir Zelinsky}
Благослови Бог Украину и Вас!

\iusr{Kate Mozart}

має відбутися квантовий скачок, а не ось це «російська культура має бути
увібрана» - у вас постколоніальний дискурс, з якого треба звільнятися. ми
вільні незалежні з власною самобутньою культурою. нам не треба ніякі асиміляції
чи щось таке. підтримка росіян - пшик, який нічого зараз не дає. де та
підтримка була всі ці роки? і хто це вбиває нас зараз? не путін же власноруч

\begin{itemize} % {
\iusr{Шарлотта Хмельницька}
\textbf{Kate Mozart}, 

дякую Катю, що змогла інтелектуально йому пояснити, що він застряв у якомусь
давньому мисленні. А то вже хотіла тут матюкать))


\iusr{Ната Руст}
\textbf{Sharlotta Khmelnytska} 

на жаль, претензія на інтелектуальність, не більше.. якщо про квантовий скачок,
який відбувається саме зараз, а не той, що не стався за 30 років, то не менше
половини тих, хто бере в ньому участь - російськомовні. нікого ще не зіпсувало
знання ще однієї мови. російська на наших теренах - це просто так вже склалося,
а не відкрита хвіртка для зла. зло заходить, користуючись будь-якою мовою. хоча
б через неповагу і ігнор до культурных здобутків геніїв, які користалися
російською, бо нею були навчені в свій час, хоч і були і є нашими земляками.
ігнор безперечних культурних цінностей німецькою - чого ні? вони теж нашої
крові досить попили. якої б ваги не була наша культурна самобутність - вона
пісчинка у порівнянні зі світовим культурним багажем. слава Україні - Героям
слава - СМЕРТЬ ворогам!

\iusr{Шарлотта Хмельницька}
\textbf{Ната Руст}, 

мова - зброя. Зараз єдина можливість відрізнити ворога від свого - мова.

Першими окупували російськомовний Крим, Донецьк і Луганськ.

І не треба зараз тут строїти з себе найрозумніших і казати, що ми ігноруємо
культурні здобутки російськомовних. Ніхто нікого ніколи не притісняв! Це
абсолютна риторика окупанта.

І так, знання ще однієї мови точно зайвим не буде. І маніпулювати культурними
цінностями теж зараз не рекомендую (і не тільки зараз)

\iusr{Ната Руст}
\textbf{Sharlotta Khmelnytska} 

1) зараз мова - це тест, не треба плутати.. 

2) російськомовні регіони - прикордонні, тому і першими, це просто  @igg{fbicon.beaming.face.smiling.eyes}  

3) ніхто нікого не притісняв - про це і не йшлося, будьте уважніші, пані.
йшлося про самодостатність української культури. то таких взагалі нема. сила
будь-якої культури в мультикультурності, про це йшла мова. 

4) де ви вгледіли маніпуляції культурними цінностями? і шо ви за цабе, шоб шось
мені \enquote{не рекомендувати} - маніпулюєте погрозами?  @igg{fbicon.beaming.face.smiling.eyes} 

\iusr{Ната Руст}
\textbf{Sharlotta Khmelnytska} 

це мовою Героїв, до речі. і на випадок, якщо в пані вже закінчилися кліше на
кшталт \enquote{риторика окупанта} і \enquote{мова - це зброя}, то можемо і інтелектуально
поматюкатися  @igg{fbicon.face.smiling.sunglasses} 

\end{itemize} % }

\iusr{Віталій Козаренко}

я прекрасно прожил в пензе и новосибирске 10лет и жена у меня оттуда зпнимался
спортом носился по всей росии но во всех компаниях один разговор -нам руским
нужен пиночет.ненависть со страхом к кавказцам евреям а душу их радует
Александр 3 которого европа боялася и это все с висшим образованием люди


\iusr{Maria Lypych}
Молимось за Україну, українців і наших Воїнів!  @igg{fbicon.hands.pray} 

\iusr{Tanya Pachina}

Полностью согласна, я русская по рождению, полностью, до последней клеточки -
украинка, по духу! Для меня важно, чтобы русский язык, мой родной, литература,
культура, как часть большой Украины, не были выкинуты и потеряны для новой
Украины! Зачем же выбивать кирпичик из нашего фундамента  @igg{fbicon.face.smiling.eyes.smiling}  Слава Украине!


\iusr{Рикита Марина}

Теперь только новая Украина. Теперь с русскими, русским языком и русской
культурой - это то, что нас глубоко объединяет. Славься Украина! Храни, Боже,
украинский народ.

\iusr{Natalya Kokoreva}

Андрей, вопрос к Вам, как к человеку думающему. Все ругают Путина. Но все
молчат про Зеленского, про то, что творила украинская власть со своим народом
десятилетия. Насколько нужно было быть недальновидным и примитивным лидером,
чтобы махать тряпкой с надписью НАТО перед носом России и не понимать, к чему
приведут все эти действия. Даже сейчас, безоговорочно понимая, что силы РФ и
Украины в войне не равны, а точнее не сопоставимы от слова совсем, вместо
переговоров о мире, Зеленский берет деньги и оружие у США и хочет продолжать
войну. И никто не пишет про сумасшедшего клоуна. Вот что это?

\begin{itemize} % {
\iusr{Maria Lypych}
\textbf{Natalya Kokoreva} 

українська влада просто не займалась своїм народом десятиліття, але нічого з
ним поганого не робила... А що хорошого робила російська влада зі своїм народом?
У мене до вас 2 запитання - чому український лідер не може «махати тряпкою з
написом НАТО», чому він має боятися заявляти про свої наміри. І друге, чому
весь світ зараз підтримує Украіну, а не Росію, і ніхто не називає клоуна, а
називає когось іншого якось інше?...

\iusr{Natalya Kokoreva}
\textbf{Maria Lypych} тому, что у нас с вами страны немножечко отличаются по масштабу... и соответственно, по задачам... и по интересам к нам со стороны сша... весь мир - это вы загнули... только сша и ее сателлиты...

\iusr{Maria Lypych}
\textbf{Natalya Kokoreva} 

і що що наші держави різні по масштабах і задачах? В України вже давно була
орієнтація на НАТО і ЄС, ми маємо такі наміри і цілі, при чому тут масштаби і
задачі?

Про весь світ я не загнула... не тільки США підтримують, шукайте повну інформацію...

\iusr{Natalya Kokoreva}
\textbf{Maria Lypych} 

читайте про НАТО. Про отношения России и НАТО, про цели и задачи НАТО. Историю
читайте, смотрите, что происходит с Болгарией, Латвией, Литвой и так далее...
сами думайте...

\iusr{Maria Lypych}
\textbf{Natalya Kokoreva} 

я знаю про історію росіі і НАТО, але це росіі проблеми... хай спочатку
навчаться бути цивілізованим як сучасний світ, варвари!!!


\iusr{Михайло Цехош}
\textbf{Natalya Kokoreva} если Россия неадекватная тварь которой есть разница какая суверенная страна в какой оборонный союз вступает то это бешеное животное стоит прикончить.

\iusr{Natalya Kokoreva}
\textbf{Maria Lypych} конечно, варвары. И мы Украину развалили, отток населения более 20 млн чел, война на своей территории 8 лет. А мы - варвары. Все логично.

\iusr{Maria Lypych}
\textbf{Natalya Kokoreva} 

те які в України проблеми це наша справа, зараз мова йде про агресію і війну,
ми на нікого не нападаємо, на нас нападають. Про війну на своїй території 8
років - це ви пишете брєд і пропаганду, це не внутрішній конфлікт у нас, а
гібридна війна росіі, ще скажіть про Крим український як він опинився
російським...

Влада росіі це є орда, варвари, а ви їх підтримуєте замість того щоб зрозуміти
і стати на протест.

Не маю бажання з вами переписуватися, витрачати даром енергію, коли не бачу
результату, ... відчуваю як ви глибоко переконані в іншому... подумайте...

\end{itemize} % }

\iusr{Наталія Єпіфанова}
Слава героям, Слава Україні! Позбавляймось всього російського!

\iusr{Tatiana Polianichka}

Русский язык им не принадлежит! Ми вільна країна, ми переможемо. Росія, якщо у
вас є душа, прокидайся! Хоча б 1\% від вашого населення вийде на вулиці - це вже
більше мільйону! Ви змогли у 91, зможете і зараз!

\iusr{Себастьян Тегза}

Просто переходимо на українську. І все. Без зайвих пояснень і розмов про язик.
Розвиваємо українське і збагачуємо світову культуру культурою українською! Інші
культури нехай розвивають інші.


\iusr{Оксана Голец}

Если будет оккупация и новый вариант \enquote{Советской Украины}, то русский язык
перестаёт быть языком Пушкина, он, прежде всего, будет языком Путина. В такой
ситуации я принципиально перестану им пользоваться.

\iusr{Ната Руст}
запрету подлежит язык ненависти.. а то чем мы тогда отличаемся от них, запрещающих нас, с нашим языком?

\iusr{Taras Klok}

Ніхто з українців не заперечує російської мови. Наш гімн написаний українською
мовою, тому варто більше писати і говорити українською!

\iusr{Игорь Гулак}

Нам нужно просто присвоить себе то, что итак по праву наше. Простая математика,
был сначала Киев, из него вышли все вокруг. Не нужно подчинять себя идее
\enquote{великого} русского всего на свете. Освоить, разобраться и иметь в арсенале.
Сбросьте комплекс меншовартості. И вот тогда все будет волшебно.

\iusr{Наталя Вернидуб}

Русская культура зараз себе показує в повній мірі. Вибачте, це все в толстом,
достоевском, пушкине. і ось вони проявляються. Кусская культура, іді назуй

\iusr{Ines Pura}

А стріляли у вам під вухом носіі отоі культури та цивілізаціі. І не вмовляйте
себе і не кажіть, що ви мудро бачите глибше, ніж ми, не професійні філософи

\iusr{Игорь Гулак}

Лично Вы Андрей Олегович одни из кузнецов и этой победы! Сегодняшней и будущей.
Вас слушают миллионы, в том числе и я. Спасибо Вам огромное за Ваш труд!

\iusr{Anna Yampolskaya}
Берегите себя!!!

\iusr{Виктория Ланжар}
Держимся @igg{fbicon.heart.red}

\iusr{Павел Керинский}

Ещё совсем недавно я бы согласился с вашей формулировкой, но границы эти
проходят между двумя юрисдикциями, между обществами, которые формулируют смыслы
и отстаивают своё право на реальность и действительность мира с его
многообразием культур. Русский язык всегда останется ассоциированным с соседней
Украинскому обществу юрисдикцией, какое бы название она не носила (Московия,
Россия, Великороссия, РСФСР). И смыслы сформулированные на русском языке в этой
соседней юрисдикции продолжат доминировать в русскоязычной среде, именно
потому, что обладают этой ассоциативной связью не с репрессиями и войнами, а
\enquote{большой} культурой и правовой традицией. Украина не антироссия, и русский язык
в Украине возможен как бытовой и поэтический, но не как государственный, или
как язык политики, как язык философии и культуры. Русским языком можно
наслаждаться эстетически, как католической-польской архитектурой в Украине, или
австрийской академической музыкой, да много чем ещё, что стало наследием на
украинской земле и питает наш ум и глаз - их стоит беречь и хранить, но беречь
и хранить на свой манер, по своему вкусу, делать их не частью, а материалом для
настоящей и будущей национальной украинской культуры.

\iusr{Светлана Карассёва}
Спасибо, Андрей. За ясность стойкость и великодушие @igg{fbicon.heart.red}

\iusr{Elena Kamenarova}

Not good. Your problem is not Russia or Putin. And you know it. If you don't,
you will learn it hardly as Balkans.

The world is not "democracy", it is full of liars. And you will learn it.

Don't delude yourself.  @igg{fbicon.face.disappointed} 

\iusr{Svetlana Sahlfeld}

Андрей, спасибо Вам, что в такой сложной ситуации вы поддерживаете нас. Сейчас
очень не хватает рациональности! Молюсь за Вас и Ваших близких. Берегите себя,
пожалуйста.

\iusr{Ines Pura}
Перший раз в житті я не з вами. Це якраз боротьба цивілізацій, де не ідеться про іх об»єднання, а навпаки!!!

\iusr{Анна Матиенко}
Храни нас Господь!

\iusr{Натали Дугнист}
Поддерживаю!!!! @igg{fbicon.heart.red}{repeat=3}
И благодарю.

\iusr{Людмила Куклева}
Новое это неизвестное внутреннее СКа

\iusr{Oleksandr Tverdomed}
Согласн с Вами, Андрей!

\iusr{Tatiana Khoptynska}
Все так! Спасибо!

\iusr{Vlad Solo}

На востоке теперь Тартария - я не могу ее отождествлять с Толстым, Булгаковым,
Тютчевым. Слушайте их дипломатов, руководитей - это не русские, это урки,
уголовники. Сегодня настоящие былинные русичи в Украине - они будут создавать
новую европейскую русскость, если у сегодняшних киевских князей хватит мудрости
не выплескивать ребенка вместе с водой, хоть она и очень мутна и кровава.

\iusr{Natalia Artemenko}
Держимся вместе!

\iusr{Ирина Семенчук}

В тому - то і справа що діячі руской культуры підтримують Путіна ! Гергієву і
Мацуєву заборонили виступати в Карнегі Холл і попередили в Ля Скала ........
Євреї до сьогодення не грають і не слухать Вагнера..........


\iusr{Юра Чопко}
Тримайтеся, Київ! Уся Україна з вами!

\iusr{Любовь Терехова}

Я дуже добре пам'ятаю репортаж із Грузії, де російськомовна грузинка описувала,
як не знала куди бігти в укриття, коли почали рватися ракети. Вона хотіла
запитати дорогу й зрозуміла, що не може питати російською, хоч грузинську знає
погано, але просто не може чути російську.

Саме зараз російська - це мова ракет і привід рятувати російськомовних.
Відіб'ємося й російську заберемо, але зараз таке боляче читати

\iusr{Анна Яковлева}
Полностью согласна и поддерживаю. Спасибо

\iusr{Ines Pura}
І Украіна буде вільна !!! Тут ви праві

\iusr{Анжелика Орлицкая}

Дорогой Андрей, спасибо за слова в сторону русских людей и русской культуры.
Большинство нас против войны! Силы вам, мужества всей Украине! ХРАНИ ВАС БОГ!

\iusr{Ирина Купчук}
@igg{fbicon.flag.ukraina} @igg{fbicon.heart.red} @igg{fbicon.hands.pray}{repeat=4} 

\iusr{Almantas Stankūnas}

Dorogoj Andrii Baumeister. Mi v Litve gordimsia geroicheskoj borboj Ukrainskogo
naroda za vashu i nashu svobodu. Delius s vami robotai moej zheni, kotoraja est
profesionalnij chudosznik - VISTOJIT!

\ifcmt
  ig https://scontent-frx5-1.xx.fbcdn.net/v/t39.30808-6/274720620_5049352565111059_111945879601524371_n.jpg?_nc_cat=100&ccb=1-5&_nc_sid=dbeb18&_nc_ohc=Jayrh6lSvbgAX9_FDPC&_nc_ht=scontent-frx5-1.xx&oh=00_AT9z_EE4aZcNyrFzlZ2QVzp-PfbWlTBSDBmSD6depSo63Q&oe=62215772
  @width 0.3
\fi

\iusr{Олеся Серова}
Да

\iusr{Михаил Аркадьев}
Да


\end{itemize} % }
