% vim: keymap=russian-jcukenwin
%%beginhead 
 
%%file 23_07_2021.fb.stepanov_farid.1.kiev_gorod_ljuteranskaja
%%parent 23_07_2021
 
%%url https://www.facebook.com/StepanovFarid/posts/611339423174540
 
%%author Степанов, Фарид
%%author_id stepanov_farid
%%author_url 
 
%%tags gorod,istoria,kiev,progulka,ulica.kiev.ljuteranskaja
%%title Лютеранская – улица вакханалии разных архитектурных стилей
 
%%endhead 
 
\subsection{Лютеранская – улица вакханалии разных архитектурных стилей}
\label{sec:23_07_2021.fb.stepanov_farid.1.kiev_gorod_ljuteranskaja}
 
\Purl{https://www.facebook.com/StepanovFarid/posts/611339423174540}
\ifcmt
 author_begin
   author_id stepanov_farid
 author_end
\fi

\index{Лютеранская, улица!Киев}

ТИХАЯ УЛИЦА.

Или: по Лютеранской, через Анненковскую и Энгельса к … Лютеранской

Люблю Киев за то, что не перестаёт удивлять, за то, что каждый день Город
открывает мне всё новые и новые свои тайны и истории.

Казалось бы, сотни раз исхоженная вдоль и поперёк Лютеранская. Что может быть
там нового? Ан нет. Оказывается, еще много тайн хранят её дворы и ещё много
историй могут рассказать стены её домов. И тут только одна проблема: как
вместить все эти истории и тайны в одном рассказе. Да так, чтобы рассказать
побольше и не утомить читателя.

\ifcmt
  tab_begin cols=3

     pic https://scontent-cdt1-1.xx.fbcdn.net/v/t39.30808-6/223166555_611338166507999_5639742923842932635_n.jpg?_nc_cat=109&ccb=1-5&_nc_sid=8bfeb9&_nc_ohc=_VHBmU5S2pEAX8sWfq1&_nc_ht=scontent-cdt1-1.xx&oh=8cb53ddf55949fd839502ee9ab7b80c3&oe=6124AA04

     pic https://scontent-cdt1-1.xx.fbcdn.net/v/t39.30808-6/223338703_611338216507994_8249439404189932854_n.jpg?_nc_cat=105&ccb=1-5&_nc_sid=8bfeb9&_nc_ohc=u2H6kyQtEbwAX90IKff&_nc_ht=scontent-cdt1-1.xx&oh=44b7f817f1b4819b50b79bb64dd98be4&oe=6123E5F2

		 pic https://scontent-cdg2-1.xx.fbcdn.net/v/t39.30808-6/222922445_611338256507990_6631294304782097873_n.jpg?_nc_cat=102&ccb=1-5&_nc_sid=8bfeb9&_nc_ohc=fJ4z9-yfeKoAX9_RgUD&_nc_ht=scontent-cdg2-1.xx&oh=45f174bd53c0a1074d28649fe23c3a5a&oe=61251D8A

  tab_end
\fi

А рассказать есть что. Не верите на слово? Да пожалуйста.

Лютеранская – улица вакханалии разных архитектурных стилей: неоромантизм и
модерн, неоренессанс и конструктивизм. И это всё несмотря на сложный рельеф
местности.

А возникла Лютеранская благодаря пожару 1811 года, превратившему Подол в
сплошное пепелище. Жители Подола расселились по всему городу. А немецкая община
облюбовала Графскую гору. Пусть вас не вводит в обман название. В те годы это
была дикая и мало заселённая местность, которая соединяла аристократические
Липки и Крещатик. В ливни потоки воды, сбегавшие по Графской горе, превращали
Крещатик в непроходимое болото. Поэтому, немцам отдали эту местность без
сожаления. Всё равно других желающих не нашлось.

\ifcmt
  tab_begin cols=3

     pic https://scontent-cdg2-1.xx.fbcdn.net/v/t39.30808-6/222664305_611338306507985_6303642306472496357_n.jpg?_nc_cat=111&ccb=1-5&_nc_sid=8bfeb9&_nc_ohc=aP7QcVdw4sAAX-9Zl07&_nc_ht=scontent-cdg2-1.xx&oh=7d67d38eb3dd349975068a4babefbbb0&oe=6125578E

     pic https://scontent-cdt1-1.xx.fbcdn.net/v/t39.30808-6/222664305_611338346507981_5871758591562775802_n.jpg?_nc_cat=109&ccb=1-5&_nc_sid=8bfeb9&_nc_ohc=qhTzdLlabv8AX9ajNPl&_nc_ht=scontent-cdt1-1.xx&oh=0fcbd78df99d789408b1fb087a139b38&oe=61252B50

		 pic https://scontent-cdg2-1.xx.fbcdn.net/v/t39.30808-6/222712343_611338396507976_3726892659122840036_n.jpg?_nc_cat=102&ccb=1-5&_nc_sid=8bfeb9&_nc_ohc=Z0cpJvJvlNAAX9Lzz8e&_nc_ht=scontent-cdg2-1.xx&oh=6734108a36f971891951b06cd2659ef9&oe=6123DFF3

  tab_end
\fi

В разные годы на Лютеранской жили уже маститый Максим Горький и юный Константин
Паустовский. На Лютеранской каждый дом – настоящая история, неразрывно
связанная с историей Города. Лютеранская кирха, которая и дала название самой
улице. Советской властью превращенная в Дом воинствующих атеистов – куда же без
них. Легендарный Дом плачущей вдовы – жемчужина киевской архитектуры.
Окруженный множеством легенд и небылиц.

На Лютеранской в 1914 году проходила запись на сеанс одновременной игры в
шахматы с будущим великим гроссмейстером Хосе Раулем Капабланка. Сам сеанс
проходил на Крещатике 1. На 30 досках он сражался с киевскими шахматистами.
Параллели с Остапом Ибрагимовичем и Нью-Васюками считаю некорректными. Матч
закончился победой гостя: 25 – 5.

\ifcmt
  tab_begin cols=3

     pic https://scontent-cdg2-1.xx.fbcdn.net/v/t39.30808-6/222977744_611338459841303_6495090077182606404_n.jpg?_nc_cat=104&ccb=1-5&_nc_sid=8bfeb9&_nc_ohc=byvoFM50zdYAX9PVj4y&_nc_ht=scontent-cdg2-1.xx&oh=84f9572967f31022fdc22968d50046f4&oe=61254069

     pic https://scontent-cdt1-1.xx.fbcdn.net/v/t39.30808-6/222929470_611338523174630_4790581350298765897_n.jpg?_nc_cat=101&ccb=1-5&_nc_sid=8bfeb9&_nc_ohc=bVLmjEjo090AX-uO0zo&_nc_ht=scontent-cdt1-1.xx&oh=b4a6ba8e5ae4a02c42b0c52aa8d763b2&oe=612488CD

		 pic https://scontent-cdt1-1.xx.fbcdn.net/v/t39.30808-6/223470499_611338566507959_5088446842464333582_n.jpg?_nc_cat=110&ccb=1-5&_nc_sid=8bfeb9&_nc_ohc=dTqn_LbmdtoAX9B6Hsc&_nc_ht=scontent-cdt1-1.xx&oh=33fc01a4ec1fbcce68519dcaa08ab737&oe=61253599

  tab_end
\fi

В начале Лютеранской сохранилась старинная брусчатка – единственная в центре
Города. Там же находилось известное в Киеве фотоателье «Фотография Владимира …
Высоцкого». В этом ателье в 1887 году была сделана одна из самых известных
фотографий Леси Украинки. На Лютеранской жил Лесь Курбас и провёл детство
будущий философ Николай Бердяев.

\ifcmt
  tab_begin cols=4

     pic https://scontent-cdg2-1.xx.fbcdn.net/v/t39.30808-6/223813560_611338619841287_4117149773876083041_n.jpg?_nc_cat=108&ccb=1-5&_nc_sid=8bfeb9&_nc_ohc=KI7-XAsfAUgAX8IwL7R&tn=lCYVFeHcTIAFcAzi&_nc_ht=scontent-cdg2-1.xx&oh=94a42228c7413802260cf730501a8615&oe=61243F75

     pic https://scontent-cdt1-1.xx.fbcdn.net/v/t39.30808-6/222712283_611338669841282_8363740212295914870_n.jpg?_nc_cat=109&ccb=1-5&_nc_sid=8bfeb9&_nc_ohc=gOJMo8S5-nUAX8PNdOz&tn=lCYVFeHcTIAFcAzi&_nc_ht=scontent-cdt1-1.xx&oh=162e13362282410accec235f6a6ff3c1&oe=61245A44

		 pic https://scontent-cdt1-1.xx.fbcdn.net/v/t39.30808-6/221967140_611338729841276_6898519201102145412_n.jpg?_nc_cat=103&ccb=1-5&_nc_sid=8bfeb9&_nc_ohc=DUq5Iu2mVU0AX-O4cZZ&_nc_ht=scontent-cdt1-1.xx&oh=45889f80c0a1e9fd39a23af2f08097f4&oe=6123EF4E

     pic https://scontent-cdg2-1.xx.fbcdn.net/v/t39.30808-6/222926004_611339019841247_8160385118121305583_n.jpg?_nc_cat=111&ccb=1-5&_nc_sid=8bfeb9&_nc_ohc=pjmV4sGHGqwAX_QjYIM&_nc_ht=scontent-cdg2-1.xx&oh=90e9646394d55cd60100650ba37ce6a3&oe=612500CD

  tab_end
\fi

Потомки Гетмана Ивана Сулимы открыли на Лютеранской известную в Киеве
богадельню, а местные жители, с удивлением, рассматривали двугорбых верблюдов,
на которых крымские татары привозили в Киев чёрный виноград «Осман» и
продолговатые крымские яблоки. На Лютеранской же жили караимы Моисей и Соломон
Коген – основатели Киевской табачной фабрики. Именно они заказали Владиславу
Городецкому Караимскую кенасу на Ярославовом Валу. На углу
Круглоуниверситетской и Лютеранской стоит красивый дом, построенный на участке,
который продала жена композитора Игоря Стравинского – Екатерина Носенко.
Уроженка Киева. Мимо этого дома на Банковую до 1938 ходил трамвай.

\ifcmt
  tab_begin cols=4

     pic https://scontent-cdg2-1.xx.fbcdn.net/v/t39.30808-6/223303409_611338776507938_4289041472638629663_n.jpg?_nc_cat=104&ccb=1-5&_nc_sid=8bfeb9&_nc_ohc=SbvbDms6cdEAX_BI5a6&_nc_ht=scontent-cdg2-1.xx&oh=ef1c303b5fbd98e4471ed6073fc78b18&oe=6123887B

     pic https://scontent-cdg2-1.xx.fbcdn.net/v/t39.30808-6/222625066_611338829841266_6156754142131288952_n.jpg?_nc_cat=111&ccb=1-5&_nc_sid=8bfeb9&_nc_ohc=v9jwLBCXHpgAX-TvwX7&_nc_ht=scontent-cdg2-1.xx&oh=987e5ce1bf74b1c06be2c4ef20dce5ce&oe=612429D9

		 pic https://scontent-cdg2-1.xx.fbcdn.net/v/t39.30808-6/223338913_611338879841261_915346149224805677_n.jpg?_nc_cat=111&ccb=1-5&_nc_sid=8bfeb9&_nc_ohc=JAUknZy4L3YAX_Vx7kD&_nc_ht=scontent-cdg2-1.xx&oh=adbb1914e1e9f82b766d73b6e037a6ac&oe=6124DE64

     pic https://scontent-cdg2-1.xx.fbcdn.net/v/t39.30808-6/222987589_611338949841254_1093790355540164236_n.jpg?_nc_cat=108&ccb=1-5&_nc_sid=8bfeb9&_nc_ohc=PZXxnOvE6MMAX_lBM0R&_nc_oc=AQkSDP33vV7ig6t61yOwd3jB1PKmZqH0FoXRT3A3C4JPad4mSYJ3ETTw3ElBeWa8OSI&_nc_ht=scontent-cdg2-1.xx&oh=46bf84e11f2297b4960f7fb843f0ab24&oe=6123843B

  tab_end
\fi

А сегодня по Лютеранской бродил я. Любуясь и восхищаясь. Этим восхищением и
делюсь с вами.

PS:

Рядом с Лютеранской ещё одна известная киевская улица – Шелковичная. Вы знаете,
что на Шелковичной плодоносят шелковицы? Еще можно успеть, пока городские
голуби всё не склевали. Но это уже совсем другая история. Расскажу вам её в
следующий раз.

Продолжение следует…

