%%beginhead 
 
%%file 08_10_2021.fb.fb_group.mariupol.nekropol.1.usypalnica_pilicheva_oksjuzovoj_v_nekropole_zagadka_tajna
%%parent 08_10_2021
 
%%url https://www.facebook.com/groups/278185963354519/posts/614587516381027
 
%%author_id fb_group.mariupol.nekropol,arximisto
%%date 08_10_2021
 
%%tags 
%%title Усыпальница Пиличева-Оксюзовой в Некрополе: загадка разгадана, но тайна осталась
 
%%endhead 

\subsection{Усыпальница Пиличева-Оксюзовой в Некрополе: загадка разгадана, но тайна осталась}
\label{sec:08_10_2021.fb.fb_group.mariupol.nekropol.1.usypalnica_pilicheva_oksjuzovoj_v_nekropole_zagadka_tajna}
 
\Purl{https://www.facebook.com/groups/278185963354519/posts/614587516381027}
\ifcmt
 author_begin
   author_id fb_group.mariupol.nekropol,arximisto
 author_end
\fi

\vspace{0.5cm}
\textbf{Усыпальница Пиличева-Оксюзовой в Некрополе: загадка разгадана, но тайна
осталась}

\#новости\_архи\_города

Волонтеры разгадали вековую загадку Мариупольского Некрополя: его самая
величественная усыпальница принадлежит одновременно Анастасии Оксюзовой
(1826-1903) и купцу Дмитрию Пиличеву (1848-1913).

Но осталась тайна – почему не-родственники и не-супруги были похоронены вместе?

Приглашаем всех желающих в эту субботу на окончательную уборку усыпальницы.

================

На прошедших выходных волонтеры продолжили уборку усыпальницы с колоннами в
Некрополе и практически сразу же обнаружили массивную черную плиту \emph{Дмитрия
Дмитриевича Пиличева} (1848-1913), купца 2-й гильдии и гласного городской думы в
1893-97 гг. Она лежит прямо на полу усыпальницы. Ее сильно изуродовали вандалы.
Рядом с ней хорошо сохранилась плитка с орнаментом.

Ранее волонтеры открыли в усыпальнице плиту Анастасии Исидоровны Оксюзовой. Как
удалось выяснить историкам и краеведам, она была второй женой Василия Ивановича
Оксюзова (1807-70) и матерью купца Харитона Васильевича Оксюзова.

Харитон Оксюзов жил и торговал преимущество в Грунау, бывшей немецкой колонии
(нынешняя Розовка Запорожской области). И Пиличев, и Харитон с Анастасией
прекрасно знали друг друга – вместе крестили детей и выступали поручителями на
свадьбах.

\textbf{Но почему они похоронены вместе?}

А вот этот вопрос остается без четкого ответа. Как правило, в одной могиле или
усыпальнице хоронили и хоронят только близких родственников. Насколько мы
знаем, Пиличевы и Оксюзовы не были кровными родственниками. Да, Дмитрий Пиличев
крестил детей Харитона Оксюзова, сына Анастасии Исидоровны. Но у него было трое
собственных детей, и он был достаточно богат, чтобы выстроить усыпальницу для
своего рода...

Неясным остается и дата возведения усыпальницы – после смерти Анастасии
Оксюзовой или после смерти Дмитрия Пиличева?

\textbf{Анастасия Оксюзова – из рода Пичахчи!}

При этом, удалось выяснить, что Анастасия Исидоровна происходила из рода
мариупольских Пичахчи. Ее отцом был Сотир Давидович Пичахчи (р.1800 г.) (в то
время приазовские греки часто использовали имя \enquote{Исидор} как
\enquote{Сотир}). К такому предположению независимо друг от друга пришли
волонтерка и краевед Helga Buzlami\footnote{\url{https://www.facebook.com/helga.buzlami}}, а также Раиса Петровна Божко, замдиректора
Мариупольского краеведческого музея, и краевед Сергей Катрич.

\textbf{Что дальше?}

В эту субботу в полдень волонтеры соберутся в Некрополе, чтобы завершить
очистку и уборку усыпальницы (под мусором остается еще около трети площади ее
пола). Приглашаем всех желающих! Остается шанс, что мы обнаружим еще плиты...

\textbf{Благодарность волонтерам-исследователям}

Мы благодарим за самоотверженный труд волонтеров Елена Сугак, Андрей Клепиков\footnote{\url{https://www.facebook.com/profile.php?id=100004296615030}},
Илья Луковенко и Maryna Holovnova\footnote{\url{https://www.facebook.com/m.holovnova}}.

Мы также благодарим историков и краеведов, которые помогли разыскать информацию
об Анастасии Исидоровне и Харитоне Васильевиче Оксюзовых.
