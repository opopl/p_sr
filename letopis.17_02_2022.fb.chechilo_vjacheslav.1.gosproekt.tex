% vim: keymap=russian-jcukenwin
%%beginhead 
 
%%file 17_02_2022.fb.chechilo_vjacheslav.1.gosproekt
%%parent 17_02_2022
 
%%url https://www.facebook.com/vyacheslav.chechilo/posts/4982088585186696
 
%%author_id chechilo_vjacheslav
%%date 
 
%%tags rossia,strana,ugroza,ukraina,vtorzhenie
%%title Наш государственный проект нежизнеспособен
 
%%endhead 
 
\subsection{Наш государственный проект нежизнеспособен}
\label{sec:17_02_2022.fb.chechilo_vjacheslav.1.gosproekt}
 
\Purl{https://www.facebook.com/vyacheslav.chechilo/posts/4982088585186696}
\ifcmt
 author_begin
   author_id chechilo_vjacheslav
 author_end
\fi

Анонсированное вторжение не состоялось, все пожали плечами и типа проехали.
Однако проблема никуда не делась. Россия своей позиции не изменила, США своей
позиции не изменили, Германия как играла в свои игры, так и играет, тут как
сходили с ума, так и сходят. 

Украина по-прежнему камень преткновения в отношениях между Западом и РФ,
территория нестабильности, потенциальная зона боевых действий. Нет, это не
означает, что завтра обязательно начнется война. Может и не начнется. Но нужно
отдавать себе отчет в том, что теперь самый благоприятный для нас вариант – это
«серая зона» между Западом и Востоком. Место без инвестиций, стабильности,
будущего. Но об этом даже вслух мало кто из экспертов решается сказать, потому
что это «зрада», «московские нарративы» и так далее. Проще рассказать про «весь
мир с нами», «подадим заявку в ЕС через 10 лет», получить гонорар и вывезти
семью подальше отсюда. 

Наш государственный проект нежизнеспособен – он кость в горле у соседей, он
нестабилен, разваливается изнутри, держится на нагнетании ненависти,
патриотической истерике, на «московской угрозе». И именно нам нужно что-то с
этим делать. Ждать, пока вопрос решат за нас силой или еще как-то, долго,
дорого и страшно.
