%%beginhead 
 
%%file 18_04_2023.fb.demidko_olga.mariupol.1.vidkryttja_vystavky_vpysani_v_istoriu_zrujnovani_vijnoju
%%parent 18_04_2023
 
%%url https://www.facebook.com/100009080371413/posts/pfbid084c4xCv4Zwizxy4d9gaz2rrTB5AEougzXqrCMgQbqvuKHvo9w1a2oR9GorpdzXPul
 
%%author_id demidko_olga.mariupol
%%date 18_04_2023
 
%%tags 
%%title Відкриття виставки "Вписані в історію - зруйновані війною"
 
%%endhead 

\subsection{Відкриття виставки \enquote{Вписані в історію - зруйновані війною}}
\label{sec:18_04_2023.fb.demidko_olga.mariupol.1.vidkryttja_vystavky_vpysani_v_istoriu_zrujnovani_vijnoju}

\Purl{https://www.facebook.com/100009080371413/posts/pfbid084c4xCv4Zwizxy4d9gaz2rrTB5AEougzXqrCMgQbqvuKHvo9w1a2oR9GorpdzXPul}
\ifcmt
 author_begin
   author_id demidko_olga.mariupol
 author_end
\fi

18 квітня у Національному музеї історії України з нагоди Дня пам'яток історії
та культури  відбулося відкриття виставки \enquote{Вписані в історію - зруйновані
війною}, організованої \href{\urlmkulturaIA}{Департамент культурно-громадського розвитку
Маріупольської міської ради} на чолі з \href{\urlDianaTrymaMariupolIA}{Diana Tryma}. Росія нахабно та цинічно
знищила архітектурні будівлі та монументальне мистецтво \href{\urlMariupolIA}{Маріуполя}, що мають
багату та унікальну історію... 

Вона і зараз продовжує знищувати маріупольські
пам'ятки культури та архітектури, приховуючи сліди власних злочинів... На
відкритті виступив Міністр культури та інформаційної політики України
Олександр Ткаченко, який наголосив, що \enquote{у 2021 році Маріуполь був визнаний
Великою культурною столицею \href{\urlUkrainaIA}{України}, а сьогодні росіяни свідомо знищують
культурну спадщину міста}. Вкотре дивитися на світлини мого рідного \href{\urlMariupolIA}{Маріуполя}
до повномасштабного вторгнення рф та після - мені особисто було дуже важко...💔
Так, архітектуру можна зруйнувати, але наша ідентичність, культура і пам'ять
залишаться.  Рада була бачити на виставці щирих маріупольців, які продовжують
зберігати та популяризувати культуру \href{\urlMariupolIA}{Маріуполя}. 

Дякую \href{\urlSosnovskyMariupolIA}{Євгену Сосновському} та \href{\urlIvanStanislavskiiMariupolIA}{Івану
Станіславському} за ваші фото! Сьогодні вони набули ще більшої значущості та
важливості. До речі, у музеї можна побачити і виставку \enquote{Азовсталь: нові сенси}.
Всім рекомендую відвідати обидві виставки, які чесно і без ретуші показують усі
жахіття війни, а також передають героїзм і незламність нашого народу.

\ii{insert.author.demidko_olga}
