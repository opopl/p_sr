% vim: keymap=russian-jcukenwin
%%beginhead 
 
%%file 08_03_2019.stz.news.ua.mrpl_city.2.olena_osypenko
%%parent 08_03_2019
 
%%url https://mrpl.city/blogs/view/mariupolchanka-olena-osipenko-zakohana-i-viddana-profesii
 
%%author_id demidko_olga.mariupol,news.ua.mrpl_city
%%date 
 
%%tags 
%%title Маріупольчанка Олена Осипенко: закохана і віддана професії
 
%%endhead 
 
\subsection{Маріупольчанка Олена Осипенко: закохана і віддана професії}
\label{sec:08_03_2019.stz.news.ua.mrpl_city.2.olena_osypenko}
 
\Purl{https://mrpl.city/blogs/view/mariupolchanka-olena-osipenko-zakohana-i-viddana-profesii}
\ifcmt
 author_begin
   author_id demidko_olga.mariupol,news.ua.mrpl_city
 author_end
\fi

До 8 березня я вирішила підготувати розповідь про маріупольчанку, закохану і
віддану професії, від якої напряму залежить жіноче здоров'я. \textbf{Олена Іванівна
Осипенко} вже 33 роки працює аку\hyp{}шером-гінекологом, вражає своєю працьовитістю,
щирістю і бажанням постійно займатися самовдосконаленням.

\ii{08_03_2019.stz.news.ua.mrpl_city.2.olena_osypenko.pic.1}

Олена народилася в Маріуполі в родині медиків. Її батьки були лікарями, тому
вона закохалася в медицину з дитинства, про іншу професію не могло бути й мови,
хоча батьки на цьому не наполягали, навпаки – пропонували різні варіанти.
Закінчила школу № 41 із золотою медаллю. Паралельно навчалася в музичній школі
7 років. Оскільки далі мріяла займатися тільки медициною, то й почала свій шлях
з Маріупольського медичного училища, який закінчила з червоним дипломом. 2,5
роки проходила навчання в усіх пологових будинках Маріуполя, де Олена отримала
навички акушерки. А коли прийшов час розподілу після медучилища, то вона
попросила направити її в пологовий будинок № 1 (нині Перинатальний центр), там
і почала свою медичну кар'єру. Працюючи в цьому пологовому будинку,
продовжувала закохуватися в професію. Але найголовніше – тут вона зустріла
своїх перших вчителів в житті й професії.

\medskip
%\begin{minipage}{0.95\textwidth}
\ii{insert.read_also.demidko.stomina}
%\end{minipage}
\medskip

Працюючи акушеркою, Олена Іванівна пройшла через всі відділення, що існували
тоді в пологовому будинку. Але найцікавішою виявилася робота в операційній. Там
вона працювала операційною сестрою, анестезисткою, поки лікарі-корифеї не
сказали їй, що вона вже досить вміє і пора б їй вступати до медичного
інституту. Юна Олена послухала старших колег і вступила до Казанського
медичного інституту здобувати вищу медичну освіту. Вчитися їй було дуже легко,
оскільки книги батьків вона вивчила ще в школі. Закінчивши інститут в 1986 році
- відразу ж в інтернатуру, і знову в Маріуполь – у 1-й пологовий будинок, за
яким дуже сумувала, як за співробітниками, так і за акушерством загалом.

\ii{08_03_2019.stz.news.ua.mrpl_city.2.olena_osypenko.pic.2}

Захоплює, що загальний стаж роботи Олени Іванівни складає 45 років! Спочатку
вона працювала в пологовому будинку №1, а з 2013 року – в Перинатальному
центрі. Річ у тому, що у 2000 році в Україні було поставлено завдання створити
Перинатальні центри на базі найбільших і пристосованих для цього пологових
будинків. У Маріуполі вибір припав саме на пологовий будинок № 1. Його
необхідно було переобладнати відповідно до вимог перинатальних центрів,
підготувати персонал, закупити обладнання і підключити апаратуру. З усією
роботою вдалося впоратися завдяки організаторським здібностям і вмілому
керівництву Головного лікаря \textbf{Комлєва Костянтина Валерійовича}. Завідувачі
відділень і всі співробітники, не припиняючи працювати в штатному режимі, не
закривалися на ремонт. Робота Перинатального центру полягає в тому, щоб були
створені умови для прийняття пологів не тільки звичайних, але й передчасних,
складних, з патологіями, а головне – це виходити, вилікувати дітей, народжених
передчасно, або з якимись особливостями розвитку. Тепер це - клініка, в якій
повинні працювати тільки висококваліфіковані спеціалісти, грамотні, витривалі,
сміливі й закохані у свою справу, які вміють працювати на складній апаратурі,
спілкуватися з особливими матусями та їхніми сім'ями, вміти реабілітувати
передчасно народжених, допомагати їм вижити. Олена Іванівна наголошує, що люди
довіряють акушерам-гінекологам найдорожче – своїх дітей, дружин, дочок, сестер,
онучок, тому права на помилку просто немає. \emph{\enquote{Ми працюємо як МНС по секундах і
по хвилинах. А, щоб так працювати, потрібно постійно вчитися і тренуватися}}, -
зазначає наша героїня. Саме з метою вдосконалювати свої знання і вміння в
Маріуполі на базі Перинатального центру було створено ситуаційний центр, де з
2018 року лікарі відпрацьовують всі навички надання допомоги, ведення пологів,
реанімації тощо.

\ii{08_03_2019.stz.news.ua.mrpl_city.2.olena_osypenko.pic.3}

\ii{insert.read_also.burov.doktor_praskovja_smirnaja}

Останні 5 років Олена Іванівна веде \enquote{Школу відповідального батьківства}, або
спрощено – \enquote{Школу матерів}. Ці заняття є не тільки необхідними для матусь і
їхніх родин, вони є одним з критеріїв роботи Перинатального центру. Важливість
таких \enquote{шкіл} прописана в наказах пологових будинків і Перинатальних центрів
всіх провідних європейських країн (Швейцарія, Німеччина, Бельгія та ін.).

Наша героїня наголошує, що за останні 30 років підхід до ведення пологів в
медицині значно змінився, з'явилися нові знання, а отже й нові підходи, сучасна
апаратура. Все проводиться відповідно до вимог міжнародних протоколів. Головне
- стати на облік у зв'язку з вагітністю, не нехтувати заняттями, спілкуванням з
лікарями та акушерками. А якщо вагітна перебуває в стінах Перинатального центру
– то відразу ж познайомитися зі школою, а далі Олена Іванівна веде кожну жінку
окремо, готує її до пологів індивідуально, якщо в цьому є необхідність. Головне
для неї – відчувати кожну \enquote{матусю}, зняти всі страхи, розвіяти погані \enquote{міфи}
про пологи, навчити правильній поведінці. Вона порівнює заняття в Школі матерів
з водінням автомобіля в \enquote{автошколі}.

\ii{insert.read_also.dkm.berkova}

Олена Іванівна дуже різнобічна людина. Вона з дитинства любить Маріуполь, має
багато улюблених місць: набережну, море, вулиці старого Маріуполя, драмтеатр,
Міський сад. Щодня невтомно займається самовдосконаленням, відточує свою
майстерність у викладанні мамам, чоловікам, їхнім сім'ям. Оскільки до кожної
людини потрібен індивідуальний підхід, вона вивчає книги з психології. Також
вивчає комп'ютерну грамотність та англійську мову. Намагається встигати за
всіма нововведеннями, що відбуваються в медицині. Олена Іванівна зізнається, що
у неї з дитинства \enquote{синдром відмінниці}, тому отримувати нові знання вважає не
просто корисним заняттям, а й справжньою необхідністю. Її хобі – це робота, яку
вона любить в усіх проявах. Девізом Олени є слова видатного поета М.
Заболоцького \emph{\enquote{не дозволяй душі лінуватися}}. Дуже любить готувати, пекти пироги,
а особливо варити український борщ за всіма правилами, для неї це ціле
таїнство. Якщо зробити все як слід, то повинен вийти справжній комплекс
мультивітамінів, наголошує наша героїня. Олена колекціонує книги, висловлювання
великих людей, журнали, літературу. Незважаючи на розквіт інтернету, для неї
книги – це духовна їжа, без них вона не може жити. До речі, за медичну
бібліотеку Перинатального центру відповідає також Олена Іванівна, напевно було
складно не помітити її любові до книг, тому цю відповідальну місію поклали на
неї. І чоловік, і донька пишаються своєю дружиною і мамою, вони з розумінням
ставляться до великого навантаження своєї найбільш важливої і рідної Жінки.

\ii{08_03_2019.stz.news.ua.mrpl_city.2.olena_osypenko.pic.4}

У Олени Іванівни є багато ідей і пропозицій. Зокрема, їй хотілося б передавати
свої знання маріупольцям, тому вона сподівається, що найближчим часом буде
створений окремий медичний проект, завдяки якому можна буде нести знання в
маси.

\textbf{Читайте також:} \emph{\enquote{ЗОЛУШКИ} В ЗАКЛЮЧЕНИИ: ПРЕОБРАЖЕНИЕ ПОДОПЕЧНЫХ МАРИУПОЛЬСКОЙ КОЛОНИИ}%
\footnote{\enquote{ЗОЛУШКИ} В ЗАКЛЮЧЕНИИ: ПРЕОБРАЖЕНИЕ ПОДОПЕЧНЫХ МАРИУПОЛЬСКОЙ КОЛОНИИ, \url{mrpl.city.tilda.ws/zolushki}}

\textbf{Улюблені книги Олени Іванівни:} \enquote{Віднесені вітром} Маргарет Мітчелл, \enquote{Таїс Афінська} Анатолія Єфремова, \enquote{Історія Вітчизни}.

\textbf{Улюблені фільми:} \enquote{Іван Васильович змінює професію}, \enquote{Іро\hyp{}нія долі, або З легкою парою!}, серіал \textbf{\enquote{Жіночий лікар}}.

\ii{08_03_2019.stz.news.ua.mrpl_city.2.olena_osypenko.pic.5}

\textbf{Курйозний випадок з життя:} За довгий час роботи лікарка бачила багато
унікальних випадків, згадує навіть, як привезли жінку вже народжувати, яка не
здогадувалася, що вагітна. Після успішного кесаревого розтину народилася
дитина, ще й немаленька – 4 300 кг. Попри переконання жінки, що вона не може
бути вагітною, все ж новонародженому була невимовно рада.

\textbf{Порада маріупольцям:} \enquote{Хочу порадити всім нашим маріупольцям брати
участь і підтримувати всі зміни й перетворення в місті. Не бути песимістами,
вірити в краще, будувати, творити своє майбутнє! Без нас самих само собою
нічого не буває в житті. За все потрібно боротися, завжди досягати. Маріуполь –
це наш Еверест і тільки нам самим його будувати, берегти, підкорювати. Ну і,
звичайно, здоров'я, оптимізму та миру всім нашим маріупольцям і процвітання
Україні!}
