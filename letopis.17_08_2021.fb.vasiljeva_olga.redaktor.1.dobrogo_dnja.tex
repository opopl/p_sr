% vim: keymap=russian-jcukenwin
%%beginhead 
 
%%file 17_08_2021.fb.vasiljeva_olga.redaktor.1.dobrogo_dnja
%%parent 17_08_2021
 
%%url https://www.facebook.com/olha.vasylieva.litred/posts/178793030902259
 
%%author_id vasiljeva_olga.redaktor
%%date 
 
%%tags filologia,jazyk,mova,slovo,ukrainizacia
%%title Зізнавайтеся, ви кажете "Доброго дня!"?
 
%%endhead 
 
\subsection{Зізнавайтеся, ви кажете \enquote{Доброго дня!}?}
\label{sec:17_08_2021.fb.vasiljeva_olga.redaktor.1.dobrogo_dnja}
 
\Purl{https://www.facebook.com/olha.vasylieva.litred/posts/178793030902259}
\ifcmt
 author_begin
   author_id vasiljeva_olga.redaktor
 author_end
\fi

Зізнавайтеся, ви кажете "Доброго дня!"? 😉

Наболіла помилка, яка лунає звідусіль 😐 Зі мною так вітаються щодня в усіх
закладах, на що я відповідаю "Добрий день" і бачу в очах того, хто вітається,
непохитну впевненість у тому, що він знає мову краще за мене 😁

\ifcmt
  ig https://scontent-frt3-1.xx.fbcdn.net/v/t1.6435-9/238897227_178792997568929_2645755548427768220_n.jpg?_nc_cat=104&ccb=1-5&_nc_sid=8bfeb9&_nc_ohc=ee5ibLNO-nwAX-MM_C5&_nc_ht=scontent-frt3-1.xx&oh=a3c137b55c9a640ddfa9e54e468f6fe8&oe=615E4D93
  @width 0.4
  @wrap \parpic[r]
\fi

Багато хто тепер думає, що українською тільки так і треба вітатися. Очевидно,
що ця форма виникла під впливом "Доброго ранку!" і мовці її підхопили, бо такої
немає в російській мові. І сталося це всього 10-12 років тому.

Проте українці віками віталися "Добрий день!" і "Добрий вечір!". Саме форми в
називному відмінку наявні і в народній творчості, і у творах письменників.
Натомість "Доброго дня! і "Доброго вечора!" майже відсутні і там, і там☝🏻Вони
відгонять штучністю ще й тому, що люди часто вітаються так для буцімто
культурнішого (читай улесливішого) звучання, забуваючи, що вітальні формули
самі по собі є побажаннями. У різних піснях ми чуємо "Добрий день вам, люди
добрі!", "Добрий вечір тобі, пане господарю!" і так далі.

А побажання "Гарного вам дня!", "Вдалого вам дня!" говорять на прощання, а не
при зустрічі.

⚠️ Тож запам'ятайте! Правильно казати "Доброго ранку!", але "Добрий день!" та
"Добрий вечір!".  Етикетні формули – це стійкі сполуки слів (фразеологічні
одиниці), які не можна видозмінювати, інакше порушимо мовну норму☝🏻 Тепер ви
знаєте.  📖 У тексті посилаюся на матеріали Олександра Пономарева, Анатолія
Жилавого, Олександра Авраменка.
