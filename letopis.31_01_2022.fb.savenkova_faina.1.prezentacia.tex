% vim: keymap=russian-jcukenwin
%%beginhead 
 
%%file 31_01_2022.fb.savenkova_faina.1.prezentacia
%%parent 31_01_2022
 
%%url https://www.facebook.com/permalink.php?story_fbid=461905162097088&id=100048328254371
 
%%author_id savenkova_faina
%%date 
 
%%tags donbass,kniga,literatura,savenkova_faina,vojna
%%title Презентация - «Стоящие за твоим плечом»
 
%%endhead 
 
\subsection{Презентация - «Стоящие за твоим плечом»}
\label{sec:31_01_2022.fb.savenkova_faina.1.prezentacia}
 
\Purl{https://www.facebook.com/permalink.php?story_fbid=461905162097088&id=100048328254371}
\ifcmt
 author_begin
   author_id savenkova_faina
 author_end
\fi

Вчера прошла презентация нашего с Александром Конторовичем романа «Стоящие за
твоим плечом». Да, пока это только онлайн-встреча, но я рада, что нам с
Александром Конторовичем удалось собрать людей объединенных сопереживанием
судьбе Донбасса и способных чувствовать его боль. Тех, кто рассказывает правду
о такой ужасной вещи, как война. Это необходимо, чтобы достучаться до людей.
Даже с помощью романов, ведь пока мы слышим лишь то, что рассказывает одна
сторона конфликта. Но я рада, что люди в Европе начинают понимать, что не все,
что им говорят СМИ и политики – правда. И теперь наши тихие голоса могут быть
услышанными. Спасибо фонду «Сперанза» и всем моим друзьям. Спасибо, любимая
Италия.

\ii{31_01_2022.fb.savenkova_faina.1.prezentacia.pic.1}

А это слова о презентации моего соавтора Александра Конторовича: «Мы долго к
этому готовились, и вот, наконец... Произошла онлайн-презентация нашей книги в
Италии. Увы, по Интернету, хотя хотелось бы и воочию. Но ковид и всё прочее...
такой возможности не дают! Зато, где можно сразу увидеть людей, которые живут в
самых разных местах страны? Пожалуй, только в таком вот виде общения.

И даже языковый барьер не стал непреодолимым препятствием – как-то всё быстро
устроилось. Вполне продуктивно пообщались, сколько успели – рассказали, жаль,
что времени было недостаточно...

Договорились выслать несколько книг в Италию, узнали кое-что о планируемых
переводах нашей книги...

А кот, который нас в конце разговора поприветствовал – это необыкновенно! И
очень трогательно!»
