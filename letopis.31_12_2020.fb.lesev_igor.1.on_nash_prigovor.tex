% vim: keymap=russian-jcukenwin
%%beginhead 
 
%%file 31_12_2020.fb.lesev_igor.1.on_nash_prigovor
%%parent 31_12_2020
 
%%url https://www.facebook.com/permalink.php?story_fbid=3848660235165051&id=100000633379839
 
%%author_id lesev_igor
%%date 
 
%%tags obschestvo,politika,poroshenko_petr,ukraina,zelenskii_vladimir
%%title Он наш приговор
 
%%endhead 
 
\subsection{Он наш приговор}
\label{sec:31_12_2020.fb.lesev_igor.1.on_nash_prigovor}
 
\Purl{https://www.facebook.com/permalink.php?story_fbid=3848660235165051&id=100000633379839}
\ifcmt
 author_begin
   author_id lesev_igor
 author_end
\fi

Он наш приговор

Да, яркий вышел год, друзья. Вот вроде бы, за тобой поле с говном размером с
Семипалатинский ядерный полигон. И все это после Порошенко. 300 лет срать, а
все равно и половину не загадить. Беспроигрышная ситуация. Вот как любой
освободитель после Освенцима. А поди ж ты, Зеленский смог. Смог насрать больше
самого Порошенко, и все это в уходящем текущем году.

\ifcmt
  ig https://scontent-frt3-2.xx.fbcdn.net/v/t1.6435-9/134352158_3848659795165095_4846389628552103970_n.jpg?_nc_cat=101&ccb=1-5&_nc_sid=730e14&_nc_ohc=PRMJryVe2-YAX89XIJp&_nc_ht=scontent-frt3-2.xx&oh=a1f34e8fd31325b36e21133e0adce7a6&oe=61A3C282
  @width 0.4
  %@wrap \parpic[r]
  @wrap \InsertBoxR{0}
\fi

Меня буквально на днях один ютубер в переписке спрашивал. Ну ты же полтора года
назад за Зе сам топил, сам призывал голосовать, жалеешь теперь? А ведь вопросы
по своей постановке сами по себе говно. Это как поклонники СССР жалеют о его
распаде. Все что не работает, то не едет. Украина Порошенко также не работала и
не ехала. Поэтому, любой, кто раскаивается, что голосовал за Зе, делает это
совершенно напрасно. Мы все устроили свой легитимный Майдан, когда на выборах
закрыли расистское человеконенавистническое говно.

Был ли тогда Зе, или Тимошенко, или Дохлая Собака – вопрос вторичный. Порошенко
и его секта – это была беда Украины. И Украина в своей массе эту беду
отрыгнула. Точка.

Другое дело, что после освободительной отрыжки мы столкнулись с этой самой
Вторичностью. И вот здесь окончательно в 20 году произошел натуральный шок.
Зеленский стал гаже Порошенко.

\begin{cmtfront}
\uzr{Глеб Бобров}

Все хорошо написано, Игорь, только за скобками анализа остались жители
Донбасса, коих при ЗЕ продолжают расстерливать также, как и при Порохе,
плюс остальная Украина, которой глубоко насрать на седьмой год нашего
геноцида. Уже сегодня мы, жители Донбасса, живем с простым пониманием:
"Вы мрази, господа! Будьте вы прокляты, вне зависимости от своих
политических предпочтений и территориальной принадлежности от Днепра до
Стрыя. Гражданин Украины в нашем понимании уже не человкек, а упырь".
Вот, что вот с этим вы будете делать, Лесев когда скините Зе???
	
\end{cmtfront}

А что такое «стать гаже Порошенко»? Это значит стать его поломанной версией.
Давайте прямо, Порошенко и его пятилетка ведь особо никогда не разочаровывала.
Он был мерзок для одних и сектантски-обожествляем для других. Классическая
макиавеллиевская хрень. Разделяй и властвуй. Тути и хутси. Именно при Порошенко
моей маме дважды делали замечания, что та разговаривает в общественных местах
на русском. В маркете и в аптеке. Женщине-пенсионеру, которая ни сотрудник, ни
кассир, а обычный покупатель. То была Украина Порошенко. Мы это должны помнить.
Людям тогда стреляли говном прямой наводкой прямо в мозг. Забыть такое – значит
унизить самого себя.

И все же Зеленский сумел Порошенко переплюнуть. Порошенко строил апартеид, но
обосрался с географией и ресурсами. Но его гольфкар хотя бы ездил по Львовской
и примыкающим областям. И все было предельно понятно. Отличная конституция
новояза из «1984» со священным армовиром. Страна без полутонов. Черное и белое.
Украинцы и сепары. Новини и кремлевская пропаганда. Это было авторское чучхе, в
котором 22 июня Советский Союз вместе с гитлеровской Германией нападал на
Украину.

\begin{cmtfront}
\uzr{Татьяна Букланова}

Эта зеленая плесень оставит после себя на украинской земле выжженное поле, ей
ничего здесь не жалко: ни людей, ни страну, т.к. уже давно застолбила
для своих семейств участки в лондонах- испаниях- италиях и пр.
	
\end{cmtfront}

И вот Зеленский образца 2020 года. И это была мерзкая трансформация. Вот как в
«Мухе» 1986 года, только еще отвратительней. Мы ведь – те 73\% антипорошенковцев
– не ждали от Зе откровений. Нет, никто не против был Сингапура или хотя бы
какой-то заштатной Словакии. Сюрприз так сюрприз. И все же мы не ждали от него
чуда. Сменил маньяка и вот и все чудо, спасибо и на этом. Дальше как-то сами.

Но гадость Зеленского в том, что он стал сам Порошенко, только в его поломанной
версии. А ведь это и есть главное предательство. Мы выбирали Непорошено, а
выбрали Зепорошенко-2.

Порошенко пришел к власти по итогам государственного переворота и последующих
сомнительных по степени легитимности для огромной части страны выборов. В
итоге, его легитимность опиралась на поддержку западных структур, которые стали
демонтировать архитектуру украинского суверенитета. При Порошенко появились все
эти НАБУ, НАПКи и прочие прослойки. Но еще раз, персонально для Порошенко это
было условием политического выживания. Зеленский же сходу отдал всю реальную
власть агентам иностранного влияния.

При Зеленском под завесой карантина был принят закон о приватизации земли. И
уже летом следующего года украинскую землю начнут массово скупать иностранные
банки.

При Зеленском соросня со статуса содержанок западных посольств и фондов перешла
в статус иждивенцев из нашего бюджета. Вся эта мразота стала обильно
харчеваться за счет бюджета в одной из самых бедных стран мира. Украина – это
совсем не Европа, это тропическая Африка. Вот вам чуть статистики. Средняя
зарплата в Уганде 328 долларов. В Танзании 381. В Кении 567. В Гане 350. Как
думаете, сколько у нас кассир получает в «Ашане» или АТБ? Зарплата кассира –
это и есть индикатор заработка по стране. И вот в этой Уганде-Украине у нас
сидит Лещенко с зп в полляма при убыточном УЗ. Или чучело Смилянский с зп в 2
миллиона, когда его почтальоны получают меньше 5 тысяч. Меньше, чем кассиры в
Гане или Уганде. Вот это все – Украина Зеленского 2020.

При Зеленском произошло большое и циничное воровство на дорогах. Уникальная
афера этого года, названая «Большой стройкой». Даже при Порошенко до такой
пиар-аферы не додумались. Он торжественно открывал какие-то садики,
амбулатории, средние школы… А оказывается можно было открывать «ремонт дорог».
Вот чтобы понимали. Ремонт дорог – это базовая функция государства. Такая же,
как выплата пенсий. У нас есть «Укравтодор», который при любой власти получает
десятки миллиардов бюджетных денег. Но при Зеленском вдруг эту функцию назвали
«достижением».

При этом, достижение ведь очень специфическое. Качество, как и реальную
протяженность дорог никто толком не проверит. Вот я ездил в Гайсин к маме при
Януковиче по одесской трассе – там постоянно что-то ремонтировалось. Ездил при
Порошенко – то же самое. Ездил три недели назад при Зеленском – такие же
сужения, и такие же разбитые участки на трассе. Но именно при Зе дежурный
ремонт назвали «Большой стройкой». И это за счет ковид-фонда.

Ну и тот самый ковид. Возможно, это самое значимое событие года, которое
показало всем нам какая же эта власть – говно. Это наша лакмусовая бумажка. Нас
делили по языку, вере, идентификации, а на самом деле у нас сорт украинцев
только один – богатые и бедные. В стране Зеленского заболевший бедняк может
рассчитывать только на свой иммунитет и помощь близких. А страна – это мародер.
Здесь ничего нет бесплатного, доступного и для людей. Вообще, вся Украина – это
страна только для чиновников и воров. А каждый чиновник – это уже вор. 62%
бюджета нашей страны идет на обслуживание внешнего долга, силовиков и
чиновников. И оставшиеся 38\% - это то, что будет только сокращаться для
неблагодарного быдла.

И вот итог. Зеленский заполз в колею Порошенко. Выстраивая антисоциальную
модель как бы независимой страны под внешним управлением, он может идти только
в одной парадигме – внутреннем противостоянии. В 2020 году это сводится к
тезису – «Россия во всем виновата». Но в 2021 этого уже будет недостаточно. У
тебя нет точек роста экономики, а значит и источников наполнения бюджета. И
чтобы сбивать социальный негатив, нужно искать врага. И он будет найден.
Внутри. Чучхе без врага не работает.

И все же, все же, все же… Нет, все не так плохо. Напротив, мы – простые люди –
ищем, спотыкаемся, обжигаемся, но все более четко формулируем запросы на
справедливое общежитие. Да, этот год в стратегическом измерении Украина
просрала. Мы ни в чем не выросли и ничего не приобрели. ВВП сократилось, почти
300 тысяч людей ушло в минус, мы всем должны на внешних рынках и нет никаких
предпосылок, что что-то изменится к 31 декабря 2021 года. Но мы – все кто не
присматривает домик в Хорватии и квартиру в Лондоне – притираемся друг к другу.
Мы вырабатываем коллективный иммунитет. Да, год крысы уходит, а Зеленский
остается. Но также и остается наше коллективное желание избавится от этого
говно-разочарования. Ах, сколько у нас еще их будет. Но мы все ближе
приближаемся к формуле ответственности за наши шальные выборы. С наступающим
Новым годом, друзья!

\ii{31_12_2020.fb.lesev_igor.1.on_nash_prigovor.cmt}
