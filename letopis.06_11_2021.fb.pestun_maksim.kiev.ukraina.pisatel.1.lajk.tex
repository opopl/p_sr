% vim: keymap=russian-jcukenwin
%%beginhead 
 
%%file 06_11_2021.fb.pestun_maksim.kiev.ukraina.pisatel.1.lajk
%%parent 06_11_2021
 
%%url https://www.facebook.com/kiev.poetry/posts/3050286321908195
 
%%author_id pestun_maksim.kiev.ukraina.pisatel
%%date 
 
%%tags chelovek,internet,lajk,obschestvo,psihologia
%%title Товарищи, лайки это не простое дело. Относитесь к нему со всей ответственностью!
 
%%endhead 
 
\subsection{Товарищи, лайки это не простое дело. Относитесь к нему со всей ответственностью!}
\label{sec:06_11_2021.fb.pestun_maksim.kiev.ukraina.pisatel.1.lajk}
 
\Purl{https://www.facebook.com/kiev.poetry/posts/3050286321908195}
\ifcmt
 author_begin
   author_id pestun_maksim.kiev.ukraina.pisatel
 author_end
\fi

Лайк – это новое, но прочно вошедшее в жизнь понятие для каждого современного
человека.

По крайней мере, точно для тех, кто читает эти строки.

Лайк – это как маленький знак качества. Раньше такие знаки могло ставить только
государство, а теперь каждый отдельно взятый гражданин.

Что, конечно, поднимает его значительность в своих же глазах.

Вот раньше-то оно как было?..

Встанешь пораньше, выйдешь за утренней газетой. Прочтешь за завтраком о том,
что в этом году страна и правительство сотворили во благо народа.... Нет, это
передовица... пропускаем. А, вот интересное! Помощь братскому мадагаскарскому
народу увеличена на семнадцать с половиной процентов. И что? Как ты выскажешь
свое к этому отношение?! Разве что пролитым кофейным напитком?..

\ifcmt
  ig https://scontent-mxp1-1.xx.fbcdn.net/v/t39.30808-6/253421160_3050286201908207_2903247979105278878_n.jpg?_nc_cat=105&ccb=1-5&_nc_sid=730e14&_nc_ohc=mSp6ORdPP7cAX9eG4Rt&_nc_ht=scontent-mxp1-1.xx&oh=c2ed11f7a1fbfa559acc75756038ce29&oe=618CA686
  @width 0.4
  %@wrap \parpic[r]
  @wrap \InsertBoxR{0}
\fi

Или вот еще. Бухгалтер седьмой моторизованной колонны четвертого треста
«Стройкабельпускмонтаж» напился пьяным и обзывал свою жену нецензурными словами
прямо в присутствии соседей. Соседи получили бесплатное развлечение, а
бухгалтер лишился тринадцатой зарплаты. И что? Как ты мог отреагировать?

А теперь все просто. Целых пять вариантов лайков. От восхищения до негодования.

Правда, иногда сложно выбрать, какой поставить. Вообще проблема выбора бывшего
советского человека всегда ставит в тупик. Раньше пришел в магазин и просишь:
масло, мыло, молоко, вино, конфеты, - и никто не задает тебе дурацких вопросов,
типа, молоко какой жирности или вино какой страны вы предпочитаете?!

А сейчас – понравился тебе пост, сидишь, думаешь... Можно, конечно, поставить
сердечко, но правильно ли тебя поймет товарищ? Вдруг ты со школы изменился
кардинальным образом!

Или вот со смеющейся рожицей. Ее можно ставить, и когда понравилось, и когда
наоборот...

Но тут тонкая грань – могут обидеться, если автор ничего смешного в виду не
имел!

Опять же – возмущенная рожица...

Представьте, написал человек статью о том, что всех сегодня волнует. O
медицине, бюджете, застройках... Обличал, приводил факты. Ты со всем согласен и
тоже возмущен. И ставишь красную рожу. И автору не всегда понятно, ты кому эту
рожу показал?! Те, про кого статья, ее не увидят, значит, ему?..

Самый универсальный значок — это рожица с открытым в удивлении ртом. Ее можно
смело ставить везде. Она может выражать, как крайнюю степень восхищения, так и
удивления. Например, тупостью написанного. Но автор этого не поймет, особенно
во втором случае...

Лайки также бывают разные по качеству: одни лайки ставят из вежливости или из
жалости, мол, человек напрягался, умную мысль писал, картинку искал,
старался... И даже если мысль не очень умная, можно иногда поставить –
приободрить, может, в следующий раз получится лучше... Авансом, так сказать!

Иногда лайк ставят просто хорошему человеку. Соседу, например, одолжил ведь до
получки сто гривень на пиво. А твой лайк дает понять, что ты помнишь. И не
важно, что ты уже на пенсию вышел...

А бывает так, кто-то ставит всякие фотки: собачек там, кошечек и прочую
живность - ну как тут не лайкнуть такую милоту?..

Встречаются еще принципиальные товарищи: «Вижу с...ки – ставлю лайк!»

Попадаются и ответственные лайкатели: помню, лайкал меня этот товарищ на
прошлой неделе, надо лайкнуть в ответ, а то некрасиво как-то.

Иногда же лайк действительно выражает твое ощущение от прочитанного, прямо
представляешь себе, как сжимаешь кулак и выставляешь вверх большой палец. Ну до
чего же здорово!

Это самые правильные лайки. Они выражают твое истинное отношение к
прочитанному. Правда, отличить их от прочих трудно, но те, кто пишут, иногда
могут!

А еще иногда бывает, что лайки жизнь спасают. Вот лайкала кого-то старушка
каждый день, и не важно что. То фотографию в пионерской форме, то в горах
висящего на последнем карабине, то шутку про карбюратор (ты уже и сам забыл,
что это), то анекдот на португальском... а тут перестала. Самое время бить
тревогу, писать общим друзьям, которых ты никогда не видел. Вдруг человеку так
плохо, что аж не до лайков! Глядишь, и помощь подоспеет.

Иногда лайки приводят к настоящей любви, ты – ее, она – тебя...
полайкали-полайкали, встретились за чашечкой кофе, глядишь, и новая семья
образовалась. Ячейка общества!

А даже если и не образовалась, то все равно время даром не пропало!

А сколько новых друзей можно найти таким образом! А где их еще искать?

В школу ты уже не ходишь, в студенческий лагерь не пускают по возрасту (если
они вообще еще существуют), в троллейбусе и метро не ездишь последние сорок
лет... Так где ж друзей-то новых брать?

А тут прочел интересный пост, поставил свой знак под ним, прям как у собачек
(простите не сдержался), потом еще и еще, глядишь, и тебя лайкнут в ответ.
Тогда можно переходить к комментариям. А там и до личных сообщений недалеко...

Тут еще такое понятие появилось – лайкозависимость.

Вот поставила фоточку «янаморе» и каждые пять минут смотришь, кто не лайкает.
Именно так! С теми, кто лайкает, понятно. Им положено! Но вот Светка из 117
квартиры не лайкнула, ничего-ничего... Я ее тоже с ее «новым» не лайкну. Память
у меня хорошая. А если еще Нинку и Машку подговорю из второго подъезда не
лайкать, так вообще придется ей жениха менять. Чтоб знала, как невнимательность
к моему новому купальнику проявлять!

Бывают еще отозванные лайки.

Вот лайкнула ты, например, борщ, который одноклассница выставила, и ничего, что
там одни буряки, сметаной замаскированные, а она твое «мясо по-французски» -
нет! Все, получай отозванный лайк! А ведь ты целых сорок минут ждала...

В общем, товарищи, лайки это не простое дело. Относитесь к нему со всей
ответственностью!

\ii{06_11_2021.fb.pestun_maksim.kiev.ukraina.pisatel.1.lajk.cmt}
