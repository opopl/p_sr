% vim: keymap=russian-jcukenwin
%%beginhead 
 
%%file people.skovoroda_grigorij_filosof.cit
%%parent people.skovoroda_grigorij_filosof
 
%%url 
 
%%author_id 
%%date 
 
%%tags 
%%title 
 
%%endhead 

%%%cit
%%%cit_head
%%%cit_pic
%%%cit_text
Кілька років тому посол України в Словенії пан Михайло Бродович показував мені
в столиці цієї країни пам’ятник Сковороді. Невеличка й вишукана скульптура
встановлена в парку в центрі Любляни, а поруч із нею – пам’ятники видатним
представникам інших народів. Побачити як символ України саме Сковороду було
свіжо й незвично, адже зазвичай за кордоном українці ставлять пам’ятники
Шевченкові.  На моє запитання пан Бродович відповів, що про Шевченка словенці
знають (і пам’ятник йому в одному з містечок є), а от Сковорода став для них
відкриттям.  Адже коли вони бачать дати життя 1722-1794 й підпис «український
філософ», то мимоволі думають: ого, у ті древні часи вже існували не тільки
українці, а й українська філософія! Це докорінно змінює стереотип про Україну,
ламає російський наратив про «націю, якої ніколи не існувало», створює нову
парадигму мислення про нашу країну. Сковорода відкриває іншу перспективу на
Україну, де Шевченко стає не творцем нації, а її будителем від сну, Григорій
Савич поглиблює нашу історію й слугує живим підтвердженням глибшого зв’язку з
багатостолітньою традицією.  І через це, безперечно, постать і творчість
Сковороди ідеально надається для культурної дипломатії й представлення України
закордоном. Але водночас і українцям давно варто уважніше придивитися до
Сковороди, відкрити не тільки його зображення на великій купюрі і хрестоматійні
вірші, а й інші твори, захоплюючу й небанальну біографію (Григорій Савич ще й
довгенько прожив – 71 рік!), контекст його епохи і місце філософа в
європейській системі координат
%%%cit_comment
%%%cit_title
\citTitle{300 років Григорію Сковороді}, 
Андрій Любка, day.kyiv.ua, 19.11.2021
%%%endcit
