% vim: keymap=russian-jcukenwin
%%beginhead 
 
%%file 11_04_2021.fb.berdnik_miroslava.2.gagarin_amerikanec
%%parent 11_04_2021
 
%%url 
 
%%author 
%%author_id 
%%author_url 
 
%%tags 
%%title 
 
%%endhead 

\subsection{Лёгким движением руки Гагарин превращается... превращается Гагарин... в американца}

С возмущением прочитала, что \enquote{большинство американцев не считают Гагарина первым космонавтом}.

Тут же хотела возразить, что в апреле 2011 года Почтовая служба ООН выпустила
серию юбилейных марок, посвященных 50-й годовщине полета Юрия Гагарина в
космос. Блоки марок, на которых были изображены Гагарин в скафандре,
космические корабли и станции, планеты Солнечной системы, были оформлены
американским художником-дизайнером Питером Боллингером.

Согласно принятой в том же в 2011-м году резолюции ГА ООН, 12 апреля отныне
будет отмечаться как Международный день полета человека в космос.

Вот эта марка на фото 1.

Но решила вооружиться ссылками и пошла на ту страницу ООН, где была ранее
размещена информация и онемела! - там - страница 404!

Зато по поиску ``50th anniversary of human space flight''  вместо Гагарина теперь
\enquote{Колумбия}!!! (фото 2) 

Кстати полёт \enquote{Колумбии}, американцы целенаправленно осуществили 12 апреля 1981 года.


\ifcmt
  pic https://scontent-frx5-1.xx.fbcdn.net/v/t1.6435-9/171790749_1185417578577157_7009507313038073658_n.jpg?_nc_cat=100&ccb=1-3&_nc_sid=730e14&_nc_ohc=hlWUnXtnsdwAX_H3lmm&_nc_ht=scontent-frx5-1.xx&oh=7cb68b6b3143676d8f113d23d08b271b&oe=6099DA8B

	pic https://scontent-frt3-1.xx.fbcdn.net/v/t1.6435-9/172041320_1185417605243821_5402984318789522661_n.jpg?_nc_cat=102&ccb=1-3&_nc_sid=730e14&_nc_ohc=skr6GRz8EGkAX8XBPrB&_nc_ht=scontent-frt3-1.xx&oh=dea15ea0bce38e70ca84122991c1e500&oe=609886F2
\fi

