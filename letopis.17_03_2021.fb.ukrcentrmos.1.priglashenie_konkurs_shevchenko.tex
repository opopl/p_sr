% vim: keymap=russian-jcukenwin
%%beginhead 
 
%%file 17_03_2021.fb.ukrcentrmos.1.priglashenie_konkurs_shevchenko
%%parent 17_03_2021
 
%%url https://www.facebook.com/ukrcentrmos/posts/3854938971264753
 
%%author Національний культурний центр України у м. Москві
%%author_id ukrcentrmos
%%author_url 
 
%%tags konkurs,kultura,literatura,moskva,poezia,rossia,shevchenko_taras,ukraina,ukrcentr.moskva
%%title Запрошення - читання творів Шевченка - 19ий всеросійський конкурс
 
%%endhead 
 
\subsection{Запрошення - читання творів Шевченка - 19ий всеросійський конкурс}
\label{sec:17_03_2021.fb.ukrcentrmos.1.priglashenie_konkurs_shevchenko}
\Purl{https://www.facebook.com/ukrcentrmos/posts/3854938971264753}
\ifcmt
 author_begin
   author_id ukrcentrmos
 author_end
\fi

Шановні друзі!

Запрошуємо Вас взяти участь у ХІХ Всеросійському конкурсі на краще читання поетичних творів Тараса Шевченка серед дітей та молоді.

Відбірковий тур Конкурсу проходитиме безпосередньо за місцем проживання учасників.

Фінальний тур відбудеться 27 березня в Національному культурному центрі України у м. Москві.

\ifcmt
  pic https://scontent-lga3-2.xx.fbcdn.net/v/t1.6435-0/p526x296/161514476_3854938714598112_6314660398308561443_n.jpg?_nc_cat=108&ccb=1-3&_nc_sid=730e14&_nc_ohc=_ZPbvis50coAX9Zfhf3&_nc_ht=scontent-lga3-2.xx&tp=6&oh=3ea3e899ab08649cb6152fb61e28acd4&oe=60D74426
	width 0.4
\fi

До участі в конкурсі запрошуються учні українських недільних шкіл, середніх
загальноосвітніх навчальних закладів, студенти вузів, молодь із регіонів Росії:

\begin{itemize}
  \item - І група – 3-6 років;
  \item - ІІ група – 7-12 років;
  \item - ІІІ група – 13-16 років;
  \item - IV група – 17-22 роки;
  \item - V група – 23-35 років.
\end{itemize}

Учасники віком понад 35 років мають можливість представити поетичний твір Тараса Шевченка поза конкурсною програмою, надіславши заявку на участь.

Заявки на участь потрібно надіслати до 25 березня на електронну адресу: \url{ukrcenter_info@mail.ru}

У заявці необхідно зазначити: прізвище, ім’я, по батькові учасника; дату
народження; навчальний заклад (ПІБ директора/ректора та контактні дані);
клас/курс; місцезнаходження навчального закладу (село, місто, область, край,
республіка); назву твору Тараса Шевченка.

Мешканці інших регіонів Російської Федерації можуть взяти участь у конкурсі заочно, надіславши відеозапис виступу.
