% vim: keymap=russian-jcukenwin
%%beginhead 
 
%%file 28_01_2017.stz.news.ua.mrpl_city.1.pozhary
%%parent 28_01_2017
 
%%url https://mrpl.city/blogs/view/pozhary
 
%%author_id burov_sergij.mariupol,news.ua.mrpl_city
%%date 
 
%%tags 
%%title Пожары
 
%%endhead 
 
\subsection{Пожары}
\label{sec:28_01_2017.stz.news.ua.mrpl_city.1.pozhary}
 
\Purl{https://mrpl.city/blogs/view/pozhary}
\ifcmt
 author_begin
   author_id burov_sergij.mariupol,news.ua.mrpl_city
 author_end
\fi

\ii{28_01_2017.stz.news.ua.mrpl_city.1.pozhary.pic.1}

Пожары — большое несчастье. И были они с тех пор, как люди начали добывать
огонь. Но и до этого молнии поджигали леса и степи, где обитали наши
первобытные пращуры, огонь вынуждал их покидать насиженные места. Заглянем в
прошлое, чтобы узнать, как обстояло дело с пожарами в Мариуполе. Единственный
случай  утраты общественного здания от огня в XVIII веке зафиксирован в широко
используемом краеведами сборнике \enquote{Мариуполь и его окрестности}. Там на стр.127
написано: \enquote{Для жителей Кафайского квартала преосвященный Игнатий в 1780 году
заложил, основал и устроил Феодоро-Стратилатовскую церковь. Она при жизни еще
митрополита сгорела и не возобновлялась. Она находилась там, где в настоящее
время духовное училище}. Современный адрес этого здания – Митрополитская,
37/52.  Были, конечно, в прошлом и другие, как говорят пожарные, возгорания
домов, быть может, не так уж редко. Но глобальных пожаров в Мариуполе до
середины ХХ века, от которых страдали города центральной России, местные
хроники  не зафиксировали. Да и понятно — в наших краях издавна и жилые
строения, и общественные здания сооружались из камня и самана, материалов
негорючих. К тому же располагались дома друг от друга на приличном расстоянии.

\ii{28_01_2017.stz.news.ua.mrpl_city.1.pozhary.pic.2}

Коль скоро речь идет о пожарах, следует отметить, что наш город с давних пор
имел свою команду борцов с огнем. Ее подворье находилось на Итальянской улице
(теперь дом №40). 25 марта 1911 года в Мариуполе начал действовать водопровод,
спроектированный и построенный под надзором гражданского инженера В. А.
Нильсена. Составной частью водопровода была существующая до сих пор
водонапорная башня, возведенная на самой высокой точке города того времени.  На
самом верху ее была устроена будка для дозорного, который денно и нощно
наблюдал за городом – нет ли где пожара. Если таковой случался, то дозорный
остервенело звонил в колокол. И происходило то, что описано С. Я. Маршаком: \enquote{И
вот с пожарного двора команда выезжала. Тревожный звон будил народ, дрожала
мостовая. И мчалась с грохотом вперед команда удалая}. 

\ii{28_01_2017.stz.news.ua.mrpl_city.1.pozhary.pic.3.zhenskaja_gimnazia}

Уже с конца XIX века в Мариуполе распространилось страхование от огня. Об этом
свидетельствуют так называемые огневые знаки — жестяные таблички с названием
страховых обществ, прикрепленные на стенах домов. Их можно и сейчас встретить
на старых особняках в исторической части города. Кстати, на фасаде дедушкиного
дома, который выходил на Торговую улицу, была прикреплена овальная пластинка из
жести, по краю ее была выдавлена надпись \enquote{Первое россійское страховое
общество}, а в центре красовались цифры \enquote{1827}. Местные \enquote{всезнайки} утверждали,
мол, табличка эта свидетельствует, что дом деда Петра построен в начале XIX
века, при этом клялись \enquote{век свободы не видать}. Ошибались ребята...

\ii{28_01_2017.stz.news.ua.mrpl_city.1.pozhary.pic.4.birzha_truda}

5 сентября 1943 года начались бои за освобождение нашего города. В этот же день
был расклеен на улицах приказ полевого коменданта Мариуполя генерала Гофмана, в
котором предписывалось населению уходить на Запад. 6 сентября гитлеровцы начали
жечь жилые дома, школы, больницы, общественные здания. Делали это оккупанты с
помощью огнеметов. С детских лет запомнились погорелки на Торговой улице:
почтовое отделение на углу Митрополитской, школа №3, (до 1918 года - Реальное
училище) на Николаевской, магазины Марспо, бывшие Адабашевские. На проспекте
Республики сохранились только считаные одноэтажные постройки да  особняк, в
котором находится в наши дни редакция газеты \enquote{Приазовский рабочий}, на
противоположной стороне - помещение бывшего кондитерского кафе. От театра,
Дворца пионеров и октябрят, бывшей гостиницы \enquote{Континенталь} (сейчас - Дворец
молодежи) и других домов остались только прокопченные стены. Сплошные руины на
улице Артема. Множество сожженных домов на Итальянской улице. К этому
печальному списку нужно добавить преданные огню почти все школы.  Сгорели
здание бывшей Мужской Александровской гимназии на Георгиевской улице
(Мариупольский колледж), управление 238-го стрелкового полка (1-й корпус ПГТУ).
Был уничтожен огнем и комплекс зданий городской больницы №3, а также многие
другие жилые и общественные здания, перечисление которых заняло бы много места.
Ветеран 221-й Мариупольско-Хинганской стрелковой дивизии И. С. Смык, исследуя
обстановку, предшествующую освобождению города, пришел к выводу, что
стремительные действия против врага  передо­вого  подвижного отряда 221-й
дивизии полковника Блажевича обеспечили быстрое освобождение Заводского
(Ильичевского, ныне Кальмиусского) района. Поэтому оккупанты не успели
уничтожить большую часть жилых строений, в том числе и некоторые многоэтажные
дома, а  близлежащие  поселки  вообще остались нетронутыми. 

\ii{28_01_2017.stz.news.ua.mrpl_city.1.pozhary.pic.5.vodonapornaja_bashnja}

В Мариуполе есть, по крайней мере, два мистических места, где пожары
происходили неоднократно. Одно из них на углу Митрополитской и Греческой улиц,
там, где превратилась в пепел упоминавшаяся выше Феодоро-Стратилатовская
церковь, на месте которой было построено в 1881 году Духовное училище. Его
здание не успели сжечь гитлеровцы, но в конце ХХ века оно, как бы само по себе,
сгорело. На углу Больничной улицы (теперь она носит имя Архипа Куинджи) и улицы
Георгиевской стояла частная женская гимназия Остославской. Гимназия была
испепелена  вихрями Гражданской войны. Постепенно предприимчивый народ
растаскал обгорелые кирпичи для своих нужд, на освободившемся месте в начале
30-х годов был построен трехэтажный дом. Что в нем было до Великой
Отечественной войны, не удалось узнать. Но достоверно известно, что во время
оккупации нацисты устроили в этом здании биржу труда. Отсюда шли повестки
мариупольским юношам и девушкам явиться для отправки на рабский труд в
Германию. 6 сентября 1943 года дом стал едва ли не первой жертвой фашистов,
предавших его огню. После войны здание было восстановлено и передано МВД. Что
произошло с Мариупольским  городским управлением милиции 9 мая 2014 года,
читатели сами знают. 
