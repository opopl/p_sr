
\isubsec{4_4_How_texht_works}{How \texht\ works}

It takes three phases to translate a source document into hypertext (see
\refsec{4_6_1_The_translation_process} on page 184): a compilation of the source by the \ \TeX\
program into DVI code, a manipulation of the DVI code by the
\verb|tex4ht| program, and a processing of looseend tasks required for
completing the translation. 

\isubsubsec{From_Latex_to_dvi}{From \ \LaTeX\  to DVI }

\LaTeX\ requires two compilations of a source file by \ \TeX\ to
establish crossreferences, and \texht might require a third compilation
to get all the hypertext links in place. On rare occasions when a
tabular environment is used, with many cells being merged, more
compilations might be needed to let the system work out how the cells
should look. 

When \ \LaTeX\  loads the \texht package, it loads the file \verb|tex4ht.sty| and looks 
at just a few lines there. Then it records a request for loading the file again at a later 
time, when it will scan the rest of the file. The second loading takes place when the 
\verb|\begin{document}| code is encountered, at which time the requests made by the 
package options are also honored. 

Since \texht enters into the picture only when \verb|\begin{document}|
is reached, some earlier user definitions might not get the full
attention of \texht, unless they are redefined in a configuration file.
For instance, \texht would have difficulties introducing HTML tags for
the superscript of a macro \verb|\newcommand{\x}{a^{b}}| which was
defined before the start of the the document environment. 

\isubsubsec{From_DVI_to_HTML}{From DVI to HTML}

DVI is a page description language that includes instructions for
specifying what content should go at which location in a print-oriented
medium. HTML, on the other hand, is a structure-oriented language with
little regard to layout issues. 
 
%%page page_191                                                  <<<---3

Consequently, in many respects, a translation from DVI to HTML is a
backward process, having to reconstruct information that might have been
lost in the translation from \ \LaTeX\  to DVI. This backward process
may well fail, if the DVI code results from a source document that
places things outside the normal stream; \verb|\hspace{-O.6em}| is a
possible example. 

During translation from DVI to HTML, font calls are processed using
virtual hypertext fonts (\refsec{4_6_7_The_font_control_files} on page 190). If these are
missing or are inappropriate, new ones can be composed by the user
without too much effort. 

\isubsubsec{Other_matters}{Other matters}

The last phase of the translation turns its attention to the production
of bitmap pictures from DVI code. To that end, \texht relies on tools
available for the current platform and that might also offer more than
one route for producing the pictures. 

The creation of pictures is the bottleneck of the translation process
and may take a long time to complete. Some shortcuts might be taken to
speed up the process; for instance, if bitmaps from earlier compilations
are already available, the system-dependent utilities may allow them to
be reused. 
