% vim: keymap=russian-jcukenwin
%%beginhead 
 
%%file 24_08_2021.fb.lukjanenko_sergej.pisatel.1.dusheschipatelnyje_istorii
%%parent 24_08_2021
 
%%url https://www.facebook.com/permalink.php?story_fbid=1226346307882136&id=100015203342628
 
%%author Лукьяненко, Сергей
%%author_id lukjanenko_sergej.pisatel
%%author_url 
 
%%tags chelovek,dusha,literatura,rusmir
%%title Как правильно рассказывать душеспасительные... то бишь душещипательные истории?
 
%%endhead 
 
\subsection{Как правильно рассказывать душеспасительные... то бишь душещипательные истории?}
\label{sec:24_08_2021.fb.lukjanenko_sergej.pisatel.1.dusheschipatelnyje_istorii}
 
\Purl{https://www.facebook.com/permalink.php?story_fbid=1226346307882136&id=100015203342628}
\ifcmt
 author_begin
   author_id lukjanenko_sergej.pisatel
 author_end
\fi

Как правильно рассказывать душеспасительные... то бишь душещипательные истории?

Этот древний и прекрасный жанр раньше был знаком лишь прихожанам тех церквей,
которым достался пылкий и не лишенный талантов проповедник. Ныне же он вполне
себе живёт в интернете. Вы наверняка встречали эти истории - обычно в перепосте
друзей, чаще всего под замком, а возможно и сами, будучи в меланхолии или
подпитии, делали перепост очередной истории в духе:

"Однажды маленькая девочка сказала отцу..."

или

"Один мужчина много и тяжело работал, чтобы..."

ну или

"Одна кружка была навеки прикована..."

Хм. Кажется, последнее уже где-то было...

В чём секрет этого вечно молодого жанра?

В том, разумеется, что все люди сентиментальны. Даже суровый киллер, мрачно
караулящий жертву со снайперской винтовкой на раскалённой крыше, может пустить
слезу, глядя на любимую аглаонему в горшке. Даже поручик Ржевский может
заплакать, сняв за три рубля голодную гимназистку.

Что уж говорить о нас?

Мы, конечно, стесняемся в повседневной жизни сентиментальности. Но организм у
человека устроен так, что периодически слёзы из глаз выпускать надо. И для
этого душещипательные истории - само то!

Что нужно написать в душещипательной истории, чтобы она и впрямь всех защипала и её стали репостить?

Перечисляю.

\begin{itemize}
  \item 1. Ключевые положительные персонажи: Бог, ангел, мать, отец, ребенок, священник, Добрый Человек, собака, кошка.
  \item 2. Ключевые отрицательные персонажи: злые люди (в любом виде, на ваш выбор - бандит, чиновник, депутат, врач, блогер, священник, мать, отец, ребенок, Злой Человек). Бог, ангелы, собаки и кошки отрицательными персонажами не бывают!
  \item 3. Ключевое действие - тоска, печаль, страдание, болезнь.
  \item 4. Момент катарсиса - мудрая мысль, высказанная одним из положительных персонажей.
  \item 5. Смысл катарсиса - всё совсем не так, как думал Положительный Персонаж, всё гораздо лучше - или гораздо хуже. Но - справедливость восстановлена!
\end{itemize}

(Кстати, идеальный пример душещипательной истории для самых маленьких - сказка
про Зайчика, у которого была избушка лубяная, Лису с её недолговечной ледяной
избушкой и Петуха, который в данном случае выступает в роли ангела-хранителя,
восстанавливающего справедливость).

Ну что, поехали? Потренируемся?

"Однажды Бог посмотрел на Землю и увидел страдающую Мать. Страдание её было так
глубоко, а рыдания так пронзительны, что Бог послал своего Ангела, чтобы тот
узнал причину горя и устранил проблему.

Ангел спустился с небес, предстал пред рыдающей женщиной и спросил:

- Почему ты плачешь, дитя?
(N.B. Ангелы и Бог всегда обращаются к людям как к детям!)

- Как же мне не плакать? - воскликнула Мать. - Отец моего ребёнка - Священник и
Добрый Человек. Но он не разрешает ребёнку завести ни кошку, ни собаку!

Ангел разгневался и отправился к отцу ребёнка.

- Почему ты запрещаешь Матери завести Ребёнку домашнее животное? - спросил он.

- Может быть Злой Человек навёл тебя на такие мысли? Может быть Бандит,
Чиновник, Депутат, Блогер или другой Священник рады детским слезам?

Отец упал на колени и сказал:

- Я и сам всей душой хотел бы завести кошечку! Но у ребёнка аллергия, астма и
атопический дерматит! Врач сказал, что от животного ему станет ещё хуже!

- О, глупый человек! - воскликнул Ангел. - Если бы ты знал, какая у нас,
Ангелов, аллергия на вас, людей! Но мы же терпим её ради любви!

Отец вскочил, потрясённый столь глубокой мыслью и отправился в зоомагазин. Там
он купил ребёнку черепашку и все были счастливы.

Ангел же схватился за голову и, со слезящимися от аллергии глазами вернулся на
Небеса..."

Простите, кажется в конце я немного сменил жанр.

Но в целом, надеюсь, вы теперь сможете при желании рассказать свою собственную
душещипательную притчу.

\ii{24_08_2021.fb.lukjanenko_sergej.pisatel.1.dusheschipatelnyje_istorii.cmt}
