% vim: keymap=russian-jcukenwin
%%beginhead 
 
%%file 06_01_2022.fb.fb_group.story_kiev_ua.1.vladimirskij_sobor
%%parent 06_01_2022
 
%%url https://www.facebook.com/groups/story.kiev.ua/posts/1834795360050590
 
%%author_id fb_group.story_kiev_ua,denisova_oksana.kiev.ukraina.gid
%%date 
 
%%tags blok_aleksandr,kiev,poezia,sobor.kiev.vladimirskij,stihi,vasnecov_viktor.hudozhnik
%%title Зашла сегодня утром во Владимирский собор
 
%%endhead 
 
\subsection{Зашла сегодня утром во Владимирский собор}
\label{sec:06_01_2022.fb.fb_group.story_kiev_ua.1.vladimirskij_sobor}
 
\Purl{https://www.facebook.com/groups/story.kiev.ua/posts/1834795360050590}
\ifcmt
 author_begin
   author_id fb_group.story_kiev_ua,denisova_oksana.kiev.ukraina.gid
 author_end
\fi

Зашла сегодня утром во Владимирский собор. В храме было очень нарядно, но
почему-то пустынно и тихо. Я очень люблю этот собор, но особенно люблю
Богородицу Виктора Васнецова. И, глядя на неё, всегда вспоминаю своё любимое
стихотворение у Александра Блока.

\ii{06_01_2022.fb.fb_group.story_kiev_ua.1.vladimirskij_sobor.pic.1}

\raggedcolumns
\begin{multicols}{2} % {
\setlength{\parindent}{0pt}

\obeycr
Девушка пела в церковном хоре
О всех усталых в чужом краю,
О всех кораблях, ушедших в море,
О всех, забывших радость свою.
\smallskip
Так пел ее голос, летящий в купол,
И луч сиял на белом плече,
И каждый из мрака смотрел и слушал,
Как белое платье пело в луче.
\smallskip
И всем казалось, что радость будет,
Что в тихой заводи все корабли,
Что на чужбине усталые люди
Светлую жизнь себе обрели.
\smallskip
И голос был сладок, и луч был тонок,
И только высоко, у царских врат,
Причастный тайнам, — плакал ребенок
О том, что никто не придет назад.
\restorecr
\end{multicols} % }

Блок писал это стихотворение не в Киеве, но мне оно всегда вспоминается именно
во Владимирском соборе.
