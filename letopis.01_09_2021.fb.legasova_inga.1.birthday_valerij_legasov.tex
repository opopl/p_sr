% vim: keymap=russian-jcukenwin
%%beginhead 
 
%%file 01_09_2021.fb.legasova_inga.1.birthday_valerij_legasov
%%parent 01_09_2021
 
%%url https://www.facebook.com/inga.legasova/posts/3974399076005608
 
%%author_id legasova_inga
%%date 
 
%%tags birthday,chernobyl,legasov_valerij.akademik.likvidator.chernobyl,sssr
%%title Сегодня, 1 сентября , в День знаний , родился мой папа, академик Валерий Алексеевич Легасов
 
%%endhead 
 
\subsection{Сегодня, 1 сентября , в День знаний , родился мой папа, академик Валерий Алексеевич Легасов}
\label{sec:01_09_2021.fb.legasova_inga.1.birthday_valerij_legasov}
 
\Purl{https://www.facebook.com/inga.legasova/posts/3974399076005608}
\ifcmt
 author_begin
   author_id legasova_inga
 author_end
\fi

Сегодня, 1 сентября , в День знаний , родился мой папа, академик Валерий
Алексеевич Легасов. Родился он в 1936 году, значит , сегодня, в первый
сентябрьский день 2021 года ему могло исполниться 85 лет , и вся наша семья
вечером после трудового дня села бы отмечать его юбилей. 

Мы и отметили его юбилей в семье. Только без него и без мамы, светлая им обоим
память. Вспоминали такие папины качества, как правдивость, честность,
порядочность, неподкупность, преданность своей семье и родной стране, верность
тем идеалам, которые он считал ценными...  

\ifcmt
  pic https://scontent-frx5-1.xx.fbcdn.net/v/t1.6435-9/241068633_3974598959318953_1991697697182762792_n.jpg?_nc_cat=105&ccb=1-5&_nc_sid=8bfeb9&_nc_ohc=jS2XPNjGGEcAX9fL6c9&_nc_ht=scontent-frx5-1.xx&oh=4a0f2e601440af6e1ba3ddfda35ea1ed&oe=615ADCAE
  width 0.4
\fi

И почему-то именно сегодня мне вспомнился пятисерийный сериал "Чернобыль",
снятый, к сожалению, не нашими российскими кинематографистами, а американскими.
Его весь мир смотрел, затаив дыхание. И несмотря на то, что в этом сериале папа
мой был показан героем, каковым он на самом деле и был, но у меня тот
прогремевший на весь мир сериал оставил очень неприятный осадок в душе и на
сердце. Почему ? Потому что вся фабула оного сериала шла вразрез с той самой
честностью, под девизом которой прожил свою жизнь мой отец. Хотя именно
приверженность правде была заявлена как лейтмотив всех пяти серий. 

"Мой долг рассказать об этом" - так называлась папина статья о чернобыльских
событиях, опубликованная в 1988 году в газете "Правда". 

Наверное , теперь мой долг рассказать о том, как была перемешана правда с
неправдой в сериале "Чернобыль" с единственной целью заработать как можно
больше денег на трагическом сюжете, мимоходом пнув Советский Союз, на
территории которого в 1986 году произошла эта "катастрофа планетарного
масштаба", как назвал ее отец. 

Есть такой метод запутывания зрителя - показав несколько реально правдивых
кадров и сюжетов, щедро разбавить их кадрами и сюжетами лживыми. Страшен он
тем, что непосвященные верят как правдивым, так и ложным мыслям, картинкам и
историям. Причём, именно ложным сюжетам отводится основная роль
провозглашателей истины. Именно эти ложные сюжетцы и застревают в сознании
зрителей и формируют их отношение к произошедшему. Мне кажется важным
рассказать об этом, так как уходят из жизни люди - свидетели тех событий
35-летней давности, некому скоро будет оспорить отснятый и показанный далеко
недокументальный материал. 

Начнём с того, что с неправды начинается весь сериал "Чернобыль" : отец показан
одиноким, несчастным, живущим в какой-то необустроенной грязной квартирке. Так,
наверное, создатели сериала представляли себе быт советского ученого,
занимавшегося стратегическими научными задачами. Однако же ,на самом деле, отец
был счастливо семеен, жил в большом выделенном ему государством доме с любимой
и любящей женой, со счастливо семейными любящими детьми, с любимыми внуками и
даже с любимой собакой . В доме было радостно, часто людно, весело и очень
дружелюбно. 

Вместо реального академика Легасова, очень энергичного, деятельного и
решительного, показан какой-то рефлексирующий слабовольный и всего опасающийся
человек. Достаточно сравнить последние документальные кадры сериала, на которых
виден мой отец,  с тем образом, который слепили авторы "Чернобыля", чтобы ясно
понять, что "профессор Легасов" в фильме и настоящий академик Легасов в
реальной жизни - это две большие разницы, как говорят в Одессе. 

Не был отец и разработчиком реакторов РБМК, как зачем-то сгоряча упомянули в
сериале, не подумав хорошенько. Видимо, забыли загуглить, что слава создания
этих реакторов принадлежит академику Доллежалю. А академик Легасов, или как
его, любя, в силу его молодости называли коллеги, -  "академик Валера" был
довольно известным в учёных кругах физико-химиком, и был серьёзным специалистом
в ректификационной очистке и химии гексафторидов тяжелых металлов , в
исследованиях коррозионной устойчивости различных материалов в среде
гексафторидов; отец стал первооткрывателем в новой области химии - химии
соединений благородных газов и получил в своей лаборатории десятки новых
уникальных веществ, имевших широкое практическое использование. Он исследовал
ксенон-органические соединения и применение плазмы для синтеза новых
соединений, использовал оригинальные техники синтеза химических соединений в
неравновесных условиях , предложил и широко внедрил плазмохимический метод
синтеза высших фторидов тяжёлых металлов с применением "инертных" фторагентов
(четырехфтористого углерода, гексафторида серы) , основанный на явлении
взрывающихся проводников ; отец развил новое направление , связанное с
получением и применением высокоинтенсивных потоков электроотрицательных и
электроположительных частиц в технологии неорганических материалов , что
позволило осуществить химический синтез с высоким градиентом температуры в
условиях сильно неравновесной плазмы. Академиком Легасовым были предложены,
разработаны и освоены новые высокоэффективные методы синтеза благородных газов
(радиационный, фотохимический, плазмохимический) и технологии их производства.
Отец внедрил искровой плазмотрон, УФ- и ВЧ-генераторы атомарного фтора. Им были
найдены катализаторы процесса термической диссоциации фтора и подробно изучена
фотохимия сжиженного фтора. Разработанный отцом каталитический процесс
получения атомарного фтора сделал технологию синтеза термодинамически
неустойчивого, но очень важного дифторида криптона и других неустойчивых
соединений намного безопаснее и производительнее. Легасов впервые осуществил
каталитический синтез гексафторида ксенона, исследовал термодинамику и кинетику
процессов образования соединений благородных газов, изучил термодинамику
системы ксенон-фтор и предложил оригинальный метод определения основных
кинетических констант газофазных химических реакций в неизотермических
условиях. Он определил кинетические константы химических процессов образования
и разложения ди-, тетра- и гексафторидов ксенона, а также с помощью
экспериментальных данных и математических методов нашёл со своими коллегами
оптимальные условия синтеза индивидуальных фторидов ксенона. Благодаря работам
лаборатории Легасова были развиты плазмохимические методы генерации и
применения интенсивных потоков атомарного фтора для синтеза эндотермических
фторидов и окислителей , используемых в различных технологиях, а также
исследовано взаимодействие указанных потоков с матрицами твёрдых благородных
газов при криогенных температурах , различными методами определены коэффициенты
использования атомарного фтора. Впервые в науке им была исследована кинетика
процессов окисления ксенона, кислорода, трифторида азота фтором на поверхности
сильных кислот Льюиса , лимитируемых переносом заряда от донора к акцептору.
Выявлено каталитическое влияние акцепторов фтор-иона на реакции фторирования
неорганических и органических соединений дифторидом ксенона в неводных
растворителях. Впервые в мире была изучена кинетика и предложен механизм
гидролиза дифторида ксенона. Им были исследованы взаимодействия фторидов
ксенона и криптона со всеми элементами периодической системы , оксидами ,
солями, реакции комплексообразования, фторирования , замещения и присоединения
, окислительно-восстановительные , что имело фундаментальное значение для химии
благородных газов. 

Выполненные отцом и его коллегами исследования заложили научные основы и
помогли разработать методы количественного улавливания радиоактивных изотопов
благородных газов , что необходимо для решения экологической проблемы ядерной
энергетики и радиохимических производств всего мира. 

За все эти многочисленные работы отцу неоднократно присуждались государственные премии СССР . 

Поэтому с великим удивлением, помню, смотрели мы те кадры вышеупомянутого
сериала "Чернобыль" в которых актёр, игравший Щербину (умнейшего, кстати,
человека и прекрасного талантливого организатора) кричал на "профессора
Легасова" и грозил, что выкинет его из вертолета. На самом деле, встретившись с
Легасовым  в составе первой правительственной комиссии в подмосковном аэропорту
Внуково, Щербина сразу понял, что может полностью рассчитывать на отца, и
относился к нему с глубоким уважением и доверием. И , конечно же, никакого
конвоя (как это показано в сериале) ни к отцу, ни к кому бы то другому не
приставляли. Никто его не арестовывал, нигде не запирал. Эта дичь, выданная за
правду, ни малейшего отношения к правде не имеет . 

Да и на вертолете правительственная комиссия не летела из Москвы до Киева,
натурально. Это сколько же по времени им лететь - то пришлось? Не сопоставили
авторы сериала скорость движения вертолёта и расстояние от Москвы до Киева. 

Что за странная идея возникла в воспалённых от желания подзаработать деньжат на
чернобыльском сюжете мозгах  сценаристов -  показать какие-то долгие партийные
заседания, совещания, разговоры под портретами и бюстами Ленина на тему о том
"что делать?" ?!?! 

Не было ничего подобного в реалии - авария  случилась в ночь с 25 на 26 апреля
1986 года . Утром 26 апреля уже была срочно сформирована советским
правительством первая правительственная комиссия, в которую были включены
Щербина в качестве главы комиссии , Легасов как научный эксперт, министр
энергетики СССР Майорец, зам министра здравоохранения СССР Воробьев, зам
председателя Госатомэнергонадзора СССР Сидоренко , зам генерального прокурора
СССР Сорока, руководитель одного из подразделений КГБ СССР Щербак, зам
председателя Совета Министров Украины Николаев, Предселатель Киевского
облисполкома Плющ. 

В два часа дня 26 апреля эта комиссия вылетела в Киев , откуда на машинах была
быстро доставлена к месту аварии в город Припять. И уже вечером 26-го апреля
отец попросил , чтобы ему дали возможность облететь на вертолете несколько раз
зону взрыва , что он и сделал, нахватав смертельную дозу радиации . Кроме того,
он в тот же вечер несколько раз на предоставленном военными бронетранспортере
подъезжал к четвёртому разрушенному энергоблоку, так как сам должен был понять,
работает ли реактор . Это было принципиально важно понять, чтобы быстро принять
правильные решения. В сериале же неправдиво показано, что он посылал людей
осматривать четвёртый энергоблок. Никого он никуда не посылал. Тем более , на
верную гибель . 

Очень разумно и точно действовал Щербина как руководитель первой
правительственной комиссии. Он разделил всех участников на шесть групп: 

\begin{itemize}
  \item 1) первая группа занималась определением причин аварии (Мешков); 
  \item 2) вторая группа должна была организовать все дозиметрические изменения в районе Припяти и в близлежащих районах (Абагян) ; 
  \item 3) группе гражданской обороны было поручено осуществить быструю эвакуацию людей из зоны аварии (генерал Иванов) ; 
  \item 4) четвёртая группа занималась организацией порядка нахождения в поражённой зоне людей (генерал Бердов); 
  \item 5) пятая группа занималась всем комплексом медицинских мероприятий (Воробьев); 
  \item 6) шестая группа была создана для локализации аварии . Эту группу возглавил академик Легасов.
\end{itemize}

Все группы с самого начала и вплоть до завершения всех работ по ликвидации
последствий аварии действовали четко, быстро и слаженно . Поэтому никакой
критики не выдерживают кадры сериала о нерасторопности, медлительности и
нерешительности ликвидаторов .

Ложь в сериале и то, что якобы правительственной комиссией принимались решения
"не выпускать из Припяти лбдей", "оцепить весь город , чтобы из него никто не
уехал", "обрезать телефонные линии". Циничная ложь . Ложь также и в том, что
устами сериального  "профессора Легасова" говорится "не стоит беспокоиться". 

А правда в том , что Легасов после того, как облетел на вертолете разрушенный
реактор, сказал Щербине в присутствии всех членов правительственной комиссии о
том, что беспокоиться о людях как раз стОит, и что необходимо их немедленно
эвакуировать  из Припяти . И Легасова поддержал Сидоренко. Именно их обоих и
послушался Щербина, результатом чего были снятые украинской властью со всех
рейсов в Киеве и в окрестных городах автобусы, колонны которых всю ночь с 26 на
27 апреля шли в Припять для эвакуации жителей этого города. 27 апреля с 12 до
16 часов практически все жители были из Припяти вывезены. И сам процесс
эвакуации был по-настоящему трагедийным: люди вынуждены были покидать свои
обжитые дома и квартиры, не взяв даже самые необходимые вещи - разрешалось
брать с собой только документы и деньги . Домашних животных приходилось
оставлять - от сердца отрывали люди своих питомцев , кошек и собак. 

После эвакуации жителей начались тяжёлые будни ликвидаторов аварии. И вместо
того , чтобы показать, как на самом деле слаженно и дружно работали все
советские министерства , ведомства и учреждения, задействованные в ликвидации
последствий аварии, сериал ложно демонстрирует пьянство, свинство , мздоимство,
нежелание говорить правду согражданам и мировому сообществу. 

В фильме неоднократно показана несуществовавшая в природе женщина-физик (якобы
олицетворяющая собой собирательный образ учёных), которая убеждает "профессора
Легасова" рассказать правду о происходящем. А он якобы боится эту правду
излагать . Бред сивой кобылы , простите мне мой "французский". Реального
академика Легасова не надо было убеждать говорить правду , ибо правда была для
него важна как воздух . Правдивость была его естественным состоянием. Это он
сам всех призывал не скрывать никаких фактов , не утаивать ничего . Это он
предложил создать оперативную информационную группу журналистов, которая  могла
бы освещать все происходящее в Припяти честно и беспристрастно. Жаль , что это
его предложение не было одобрено тогдашним первым лицом страны. 

И если уж рассказывать эту историю  правдиво (на что претендуют авторы
блокбастера "Чернобыль"), то тогда уж надо обозначить , что женщина рядом с
Легасовым действительно была. И советы давала , и поддерживала, и встречала его
с любовью , когда он накоротко возвращался домой , чтобы честно рассказать о
происходящем коллегам и снова вернуться туда , в Припять . Этой женщиной была
моя мама, Маргарита Михайловна Легасова. Его жена, его подруга, его коллега и
единомышленница, его муза. 

Отдельное "спасибо" создателям сериала, воткнувшим в него чисто голливудский
слезовызывающий приёмчик - "профессор Легасов" идёт на приём к Горбачёву и
просит у него разрешения "убить трёх человек". Речь о водолазах. Этот сюжетец
оказался вообще за гранью разумного : во-первых, меньше всего отец был склонен
в ситуации опасности просить у кого бы то ни было разрешения на те или иные
действия в тот момент , когда надо было именно действовать . Они все там
действовали, а не разглагольствовали; во-вторых, Горбачёв сидел в Москве и
никаких решений не мог принимать издалека , да и отношений у отца с ним никаких
не могло быть и не было; и в-третьих (и в-главных), выбирались на те или иные
действия только добровольцы , которые лезли в пекло не потому, что им
предлагались деньги за героизм или не потому, что их под дулами автоматов
заставляли  (как это опять -таки неправдиво показано в сериале), а потому , что
в тот момент в людях и просыпались герои - они шли на риск осознанно, хорошо
понимая, что рискуют своим здоровьем и своими жизнями. Все они - герои.
Настоящие герои , а не голливудские. Вечная светлая память героям- ликвидаторам
последствий аварии на Чернобыльской АЭС .

Советский Союз с честью справился с последствиями Чернобыльской аварии. 

Об этом надо было снимать большой серьёзный сериал как руководство к действию
на случай, если где-то такая ситуация повторится , упаси Господь. 

Причём, снимать такой сериал надо не заморским, а российским киномастерам. 

Но Чернобыльская тема по зубам только очень сильному мастеру. Киногению. Почему
? Потому что ликвидаторами этой катастрофы были очень масштабные личности, и
рассказать об их подвиге сможет только очень масштабный киномастер. Академик
Легасов от кино.
