% vim: keymap=russian-jcukenwin
%%beginhead 
 
%%file 01_12_2021.fb.fb_group.story_kiev_ua.1.familnaja_lozhka.cmt
%%parent 01_12_2021.fb.fb_group.story_kiev_ua.1.familnaja_lozhka
 
%%url 
 
%%author_id 
%%date 
 
%%tags 
%%title 
 
%%endhead 
\zzSecCmt

\begin{itemize} % {
\iusr{Оксана Дубинина}

Георгий, ничего себе! Каждый абзац Вашей публикации тянет на отдельный роман о
судьбе человека. Очень интересно, а еще чудо, как Вы сохраняете историю своего
рода. Спасибо!


\iusr{Георгий Майоренко}

Благодарю. Ещё очень важно встретить добрых людей на пути, которые подскажут и
помогут в поиске. И некоторые из этих добрых людей в группе \enquote{КИ}. Здесь я даже
родственников нашел. Так-что отдельная благодарность создателям группы и
модераторам. И с первым днём зимы!

\iusr{Татьяна Гурьева}

Блеск) как интересно читать о таких людях, поднимающих империю. Вот именно род
укрепляет человека, многое было завязано на этом. Славная история)

\begin{itemize} % {
\iusr{Георгий Майоренко}
\textbf{Татьяна Гурьева} 

Спасибо. Сейчас читаю воспоминания, переписку некоторых из тех, кого упомянул.
И таки да, интеллектуалы были высшей пробы! Причем, без айфонов, интернета.
Сейчас это сложно даже представить!

\iusr{Татьяна Гурьева}
\textbf{Георгий Майоренко} это называется порода)

\begin{itemize} % {
\iusr{Георгий Майоренко}
\textbf{Татьяна Гурьева} И колоссальный труд. Ведь многие аристократы жизнь \enquote{прожигали}. И такое было.
\end{itemize} % }

\iusr{Татьяна Гурьева}
\textbf{Георгий Майоренко} да всяко было. Главное, что в сухом остатке)

\iusr{Кретов Андрей}

Спасибо огромное за такой интереснейший материал!!!

О В. Шульгине знал, читал его мемуары и книгу, написанную после его
насильственного возвращения в Союз. Знал и что его племянник был в Раде, а вот
остальное для меня откровение. О богатейших Оболонских знал, а вот о
Жлобах... Спасибо!

\begin{itemize} % {
\iusr{Георгий Майоренко}
\textbf{Кретов Андрей} Да, слово жлобы звучит грубо. Но из песни слов не выкинешь! ))

\iusr{Rimma Turovskaya}
\textbf{Георгий Майоренко} Может быть, это сейчас звучит грубо, а в те времена это слово имело совсем другой смысл.

\iusr{Георгий Майоренко}
\textbf{Rimma Turovskaya} 

Есть версия, что в польском варианте это звучало не Жлобы - Погорельские, а
Злобы - Погоржецкие. Так что, возможно, эти люди были либо прижимистыми, либо
злобными. Хотя иногда употребляют слово жлоб по отношению к здоровяку - амбалу.

Выходили из избы
здоровенные жлобы...

В. Высоцкий

\iusr{Rimma Turovskaya}
\textbf{Георгий Майоренко} Да, вариантов много.

\iusr{Rimma Turovskaya}
\textbf{Георгий Майоренко} 

Георгий, я хотела спросить, неужели все эти материалы о своих предках Вы нашли
в архивах? Наверно, это потому, что среди них много выдающихся личностей?

\iusr{Георгий Майоренко}
\textbf{Rimma Turovskaya} 

Дело в том, что мне достался по наследству неплохой домашний архив. Да, так
совпало, что некоторые родственники были людьми незаурядными. Но эту тему я
начал подробно исследовать два года назад. До этого - все не было времени, не
доходили руки.


\iusr{Георгий Майоренко}
\textbf{Rimma Turovskaya} 

А с архивами (Киева и не только) сотрудничаю достаточно активно. А ещё в сборе
материала помогают многие замечательные люди. В том числе и из группы КИ. Под
Новый год подниму за них бокал.

\end{itemize} % }

\end{itemize} % }

\end{itemize} % }
