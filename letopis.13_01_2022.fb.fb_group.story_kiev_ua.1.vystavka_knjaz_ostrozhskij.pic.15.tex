% vim: keymap=russian-jcukenwin
%%beginhead 
 
%%file 13_01_2022.fb.fb_group.story_kiev_ua.1.vystavka_knjaz_ostrozhskij.pic.15
%%parent 13_01_2022.fb.fb_group.story_kiev_ua.1.vystavka_knjaz_ostrozhskij
 
%%url 
 
%%author_id 
%%date 
 
%%tags 
%%title 
 
%%endhead 

\ifcmt
  ig https://scontent-frx5-1.xx.fbcdn.net/v/t39.30808-6/271776768_4722124667894444_5307401373428340235_n.jpg?_nc_cat=105&ccb=1-5&_nc_sid=b9115d&_nc_ohc=8J2w02dJ0tsAX8KnZaF&_nc_ht=scontent-frx5-1.xx&oh=00_AT9a0_Z_1c6LgnUOVvyU8CmMxBbQaKloCl0pDZugY3fXAA&oe=61E59CEE
  @caption @label 15
\fi

\iusr{Igor Poluektov}

Крім цієї мапи на виставці представлена одна з моїх улюблених мап 1613 року,
яка містить назву «Україна» (Vkraina; позначаючи так Київщину і землі
Подніпров‘я), була розроблена за участі Острозьких і кращих інтелектуалів
Острозької школи.  Я вже двічі описував цю мапу. Перший раз роки два тому.
Другий раз нещодавно, для книги, яка готується до друку і сьогодні була
відправлена до редактора на вичитку.  А ви поки можете почитати невичитаний
варіант 1/4 тут:
