% vim: keymap=russian-jcukenwin
%%beginhead 
 
%%file 24_01_2021.fb.ipf_npu.1.mova_zakon
%%parent 24_01_2021
 
%%url https://www.facebook.com/IPFNPU/posts/699142444118037
 
%%author Інженерно-педагогічний факультет НПУ ім. М. Драгоманова
%%author_id ipf_npu
%%author_url 
 
%%tags mova,ukraina,ukrainizacia,universytet_dragomanova,zakon
%%title Всім привіт! Поговоримо?
 
%%endhead 
 
\subsection{Всім привіт! Поговоримо?}
\label{sec:24_01_2021.fb.ipf_npu.1.mova_zakon}
 
\Purl{https://www.facebook.com/IPFNPU/posts/699142444118037}
\ifcmt
 author_begin
   author_id ipf_npu
 author_end
\fi

\verb|#непіар| \verb|#нпубезвати| 

Всім привіт! Поговоримо? Про мовний закон сказано багато і сумніву, як на мене,
не підлягає.

Так повинно бути, бо ми живемо та навчаємося в країні, в якій основна мова
УКРАЇНСЬКА. Але мені ніхто не забороняє говорити іншими мовами…

Не хочеться поговорити про людей, які осуджують мовний закон та не хочуть
навіть прочитати його до кінця, щоб зрозуміти суть цього закону. Особливо мене
дивують люди, які не читаючи закону, роблять висновки, а ще більше мене дивують
люди, які читають закони, щоб заявити про те, чого в нашій країні нема і бути
не може, бо Україна - вільна держава, а відстоювати це варто, навіть цим
постом. 

Я студент Інженерно-педагогічний факультет НПУ ім.М.Драгоманова!

Голова Інформаційного сектору НПУ імені М.П. Драгоманова. 

Я повністю підтримую ст. 30 Закону України «Про забезпечення функціонування
української мови як державної», хто б що не говорив або писав! Моя позиція -
відкрита!

З вами був ваш © Тарас Кошеленко

\ifcmt
  ig https://scontent-lga3-1.xx.fbcdn.net/v/t1.6435-9/141697118_699142417451373_642882353997294200_n.jpg?_nc_cat=106&ccb=1-3&_nc_sid=8bfeb9&_nc_ohc=a-vq7XqGdPIAX9Pse5E&_nc_ht=scontent-lga3-1.xx&oh=cd91ec890083c60ba71eb4c53c388f41&oe=6125FFEF
  width 0.4
\fi
