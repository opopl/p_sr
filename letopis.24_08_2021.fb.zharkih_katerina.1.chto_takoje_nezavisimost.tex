% vim: keymap=russian-jcukenwin
%%beginhead 
 
%%file 24_08_2021.fb.zharkih_katerina.1.chto_takoje_nezavisimost
%%parent 24_08_2021
 
%%url https://www.facebook.com/kate.zharkih.5/posts/6643834138975967
 
%%author Жарких, Катерина
%%author_id zharkih_katerina
%%author_url 
 
%%tags nezalezhnist,strana,ukraina
%%title ЧТО ТАКОЕ НЕЗАВИСИМОСТЬ СТРАНЫ?
 
%%endhead 
 
\subsection{ЧТО ТАКОЕ НЕЗАВИСИМОСТЬ СТРАНЫ?}
\label{sec:24_08_2021.fb.zharkih_katerina.1.chto_takoje_nezavisimost}
 
\Purl{https://www.facebook.com/kate.zharkih.5/posts/6643834138975967}
\ifcmt
 author_begin
   author_id zharkih_katerina
 author_end
\fi

ЧТО ТАКОЕ НЕЗАВИСИМОСТЬ СТРАНЫ? 

(Ссылка на блог в первом комментарии)

Для меня - в первую очередь это самостоятельность принятия решений, которые
выгодны стране и ответственность, которую несёт сама страна после их принятия.
это значит проводить реформы не потому, что нам советует кто-то извне (другая
независимая страна, которая действует в своих же интересах), а исходя из
потребности собственного народа. И когда что-то не выходит - не обвинять другие
государства, а как взрослому человеку нести последствия, как хорошие так и
плохие. 

\href{https://www.youtube.com/watch?v=Ec6Ops5Y57Y}{%
30 лет независимости Украины: а есть ли, что праздновать?, Екатерина Жарких, %
youtube, 23.08.2021%
}

\ifcmt
  pic https://scontent-cdg2-1.xx.fbcdn.net/v/t1.6435-9/240408941_6643834072309307_4079480692040025138_n.jpg?_nc_cat=107&ccb=1-5&_nc_sid=8bfeb9&_nc_ohc=a5BIcEwXth8AX8aFxaJ&_nc_ht=scontent-cdg2-1.xx&oh=c80e17375148221f330d3e0ccbb43b81&oe=614F7766
  width 0.4
\fi

Независимость - это когда деньги государства не уходят из страны, а люди не
уезжают в ближнее да и дальнее зарубежье искать лучшей судьбы или хотя бы
подработку. В независимой стране человек может найти себе реализацию, он будет
нужен этому государству.   

Точно также, независимое государство находит своё место и роль в мире среди
других государств. 

Независимость страны - это отсутствие страха граждан перед агрессивным
меньшинством, которые могут запугать, избить и поставить на колени за другое
мнение, одежду и язык. В независимой стране можно смотреть любые каналы и
ходить в любую церковь, петь на улицах песни не на государственном языке и не
боятся, что тебя заставят униженно извиняться на камеру. В независимой стране
ты не боишься, что за пост в фесбуке против власти тебя будут проследовать, а
за публичное выступления посадят. А может даже и убьют. В независимой стране
действует закон и порядок и те, кто нарушает и закон и конституцию, то есть
убийцы и преступники, сидят в тюрьме, а не продолжают гулять по улицам. В
независимом государстве существует монополия на насилие. 

В независимом государстве быть журналистом и доносить людям положение вещей -
безопасно для здоровья и жизни. Точно также, как и свободно высказывать свою
политическую позицию. 

В независимом государстве обязательно присутствует народовластие, равные права
в управлении страной и защитой своих прав и интересов для всех граждан страны
без исключения , а не только для избранной группы лиц по какому любо признаку.

Независимая страна заканчивает войну, прекращая убивать своих граждан, а не
руководствуется интересами особей, которые на этой войне зарабатывают. Не
продает землю. Не уничтожает конституционный суд. Не отдаёт власть иностранным
банкам. 

Государство, которое считает себя независимым вкладывает в развитие экономки. В
предприятия, которые производят готовый продукт. А не существует только как
сырьевой придаток чужой страны. 

Независимое государство не просит милостыню у других стран, не выступает
обьектом переговоров. Самостоятельно принимает решение исходя из своих
потребностей и выгоде. На международной арене не пытается присосаться к кому-то
продавая интересы своего народа, а имеет политическую волю продавливать свои
интересы. И с теми, с кем эти интересы сходятся сотрудничает. 

А теперь посмотрите на Украину и ответьте. Разве мы независимы? 

Пишите в комментариях.
