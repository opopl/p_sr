% vim: keymap=russian-jcukenwin
%%beginhead 
 
%%file 30_10_2020.fb.max_buzhanskii.1.dek_selo
%%parent 30_10_2020
 
%%url https://www.facebook.com/permalink.php?story_fbid=1784390338392104&id=100004634650264
%%author max buzhanskii
%%tags max buzhanskii
%%title Горе пришло в каждое село
 
%%endhead 

\subsection{Горе пришло в каждое село}

\Purl{https://www.facebook.com/permalink.php?story_fbid=1784390338392104&id=100004634650264}
\Pauthor{Бужанский, Максим}

Как отменили проверку деклараций? Не могли они так поступить с НАЗК!! - кричат
друг другу люди, обнимаясь, и размазывая по лицам слезы.

Да лучше бы ещё в двадцать раз тарифы на газ подняли, чем безвиз отнять!- воет
баба Клава, последний раз выезжавшая из села проводить Щорса в поход.

Ну и что, что неконституционный, зато послу Литвы нравился!!- криком кричит
единственный в селе соросенок Аристарх.

Раньше был физруком Степаном, после Майдана переименовали, когда мода на
соросят пошла.

Почему не поднимают авиацию в воздух??!!- возмущается почтальон Захаровна,-
уйдут же коррупционеры!!!!

Эх, зря Макарыч ядерное оружие отдал, сейчас бы мы их всех превентивно разом и
накрыли бы.

Так я не понял, не есть мне фуа гра на берегах Роны?- роняет на пол бутылку с
молоком механизатор, осознавший, наконец, чудовищную угрозу, вставшую перед
державой.

В некоторых антикоррупционных фондах замироточил лик Шабунина!- кричит
Аристарх, - слезы текут по лицу, и там, где они падают на землю, сразу же
всходит конопля, упомянутая в законопроекте д-б-ёб/2020/пп!!!

Верите в такое?

Нет?

И я не верю.

Не верю, что демонтаж системы шантажа и внешнего давления как то скажется на
антикоррупционных процессах, которых никто отродясь пока что в глаза не видел.

Как и предыдущего рейтинга Слуги Народа, прощание с которым состоялось в
прошлое воскресенье.

Официальное прощание, я имею ввиду.

Согласитесь, очень странно дать людям, проголосовавшим против Порошенко
продолжение политики Порошенко и не получить результат Порошенко?

Дали, и получили.

Понимаете, дело ведь не в ухудшении или улучшении ситуации, которая как
трухлявое бревно и не подает признаков жизни.

Дело в отношении.

В каждом унижении, которое было щедро выплеснуто в лицо избирателю, в каждом
назначенном порохоботе и соросенке, в каждом несдвинутом ни на миллиметр
порошенковском законе,  и каждом забытом обещании.

Нет, никуда не делся избиратель и ни к кому не сбежал, остальные то не
перестали быть самими собой.

Он просто посмотрел на предложенные кандидатуры, равнодушно, без обиды уже
даже, пожал плечами, и провел воскресенье так, как хотел.

Остальные партии тоже не приросли так, чтобы это вызывало воодушевление.

Проблем то две, отсутствие идей и отсутствие желания выполнять обещанное.

У тех политсил, у которых это обещанное ещё может привлечь интерес.

И если кто то думает, что это как то повлияло на пафос, доносящийся со всех
сторон, то нет, в этом мы твёрдо чемпионы мира.

Особенно, когда стало понятно, что у нас есть своя вакцина против короновируса.
Её ещё никто не видел, но она есть, это прям чувствуется в воздухе.  Какая она?
Дикий вопрос.  Бездоганна.

С этим нельзя спорить, потому что её никто не видел, согласитесь.

Тем временем, пока мы наслаждаемся мыслью об обладании ею, всё живое вокруг
наглухо закрывает нежизненно необходимую инфраструктуру.

Германия, Франция, Испания уходят в жёсткий карантин, я так думаю, имея целью
немного снизить нагрузки на больницы.

Но у нас второй тур выборов мэров, поэтому, хоть динозавры восстань из мёртвых,
всё будет работать.

Но я бы готовился к кризису.

К нему всегда надо готовиться, а мы всегда не готовы.

Вот и поговорим, можем ли мы уже покачать пробиркой с вакциной в Совбезе ООН
перед носом Колина Пауэлла, как он любит, или ещё неделю подождем.

Сегодня, в 21.00, на Украина 24, Народ Против.

Включайте, вспомним те старые добрые времена, когда НАБУ крушило коррупционеров
тысячами в сутки) На фото - Днепр сегодня, завидуйте, киевляне, с вашим
лондонским туманом!))
