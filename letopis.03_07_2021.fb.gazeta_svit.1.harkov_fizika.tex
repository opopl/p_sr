% vim: keymap=russian-jcukenwin
%%beginhead 
 
%%file 03_07_2021.fb.gazeta_svit.1.harkov_fizika
%%parent 03_07_2021
 
%%url https://www.facebook.com/GazetaSvit/posts/119303300376703
 
%%author Gazeta Svit
%%author_id gazeta_svit
%%author_url 
 
%%tags cern,fizika,harkov,jazyk,mova,ukraina,ukrainizacia
%%title ЯК НАЗИВАЄТЬСЯ СТОЛИЦЯ УКРАЇНСЬКОЇ ФІЗИКИ? Або дещо про громадянську відповідальність ученого
 
%%endhead 
 
\subsection{ЯК НАЗИВАЄТЬСЯ СТОЛИЦЯ УКРАЇНСЬКОЇ ФІЗИКИ? Або дещо про громадянську відповідальність ученого}
\label{sec:03_07_2021.fb.gazeta_svit.1.harkov_fizika}
 
\Purl{https://www.facebook.com/GazetaSvit/posts/119303300376703}
\ifcmt
 author_begin
   author_id gazeta_svit
 author_end
\fi

ЯК НАЗИВАЄТЬСЯ СТОЛИЦЯ УКРАЇНСЬКОЇ ФІЗИКИ?

Або дещо про громадянську відповідальність ученого.

Автор пише це, розуміючи: що такі думки, напевно, не будуть сприйняті кимось із
колег. А дехто, навіть поділяючи позицію автора, подумає, що в українській
науці є багато значно гостріших проблем – і матиме рацію. І все ж, сподіваюся,
що ці міркування змусять когось замислитися над речами, які досі просто
проходили повз його увагу. І тоді автор вважатиме своє завдання виконаним.

\ifcmt
  pic https://scontent-cdt1-1.xx.fbcdn.net/v/t1.6435-9/212239911_119303257043374_6444615254285642706_n.jpg?_nc_cat=110&ccb=1-5&_nc_sid=730e14&_nc_ohc=ox2eoUpRBHMAX8U96xp&_nc_ht=scontent-cdt1-1.xx&oh=fbe755ab4bc80d6b90dddfbc11e9b702&oe=614B2D9C
  width 0.4
\fi

Отже, на початку червня автор брав участь у міжнародній конференції, на
високому рівні організованій одним відомим харківським академічним інститутом.
Присвячено її було фізиці конденсованого стану й низьких температур. У
відповідності до реалій пандемічного часу відбувалася конференція в змішаному
режимі – свою доповідь автор (як і більшість інших учасників з-поза Харкова)
робив онлайн. Мовою конференції, як і годиться для форуму такого рівня, була
англійська.

Фізика високих енергій не належить до поля фахової діяльності автора. Однак,
побачивши анонсовані в програмі конференції міжнародні майстер-класи ЦЕРН,
автор вирішив не втратити нагоди почути оглядову доповідь, присвячену вступові
до Стандартної моделі. Робив доповідь відомий харківський фізик, що активно
працює на Великому адронному колайдері (і має один із найвищих в Україні
показників цитованості).

Представляючи цю доповідь, ведучий (англійською) оголосив, що через широкий
характер і зорієнтованість на молодь, зроблено її буде «рідною мовою» («native
language») аудиторії. А вже доповідач, сказавши кілька вступних речень доброю
англійською, розтлумачив міжнародному співтовариству, що ця рідна мова в
Україні – російська, що саме нею зрозуміліше буде слухати школярам – і перейшов
на неї. А на його англомовних слайдах знизу скрізь стояло: Kharkov, хоч
програма конференції й стверджувала, що вона відбувається в місті, яке
по-англійському називається Kharkiv…

І кілька іноземних учених високого рівня, присутніх при цьому, відразу ж
дістали предметний урок: російська пропаганда має рацію, рідною мовою українців
насправді є російська, а всі ті кампанії «Kyiv not Kiev», які героїчно
проводять наші дипломати вкупі з волонтерами – то щось безумовно чуже
«автентичним юкам».

Не говоритиму про те, що, чинячи так, і доповідач, і організатори конференції
порушили відразу два закони – про державну мову і про освіту (добрі наміри,
закладені в українські закони такого штибу, насправді нівелюються
необов’язковістю їх виконання). 

Натомість наголошу на іншому: хоча де факто значна частина університетських
лекцій у Харкові далі звучить російською (і МОН на це традиційно заплющує очі –
бо інакше довелося б визнати, що реальна ситуація  відрізняється від
декларованої), мовою середньої освіти навіть на Слобожанщині є вже майже на
100\% українська. І якщо лекцію про Стандартну модель таки слухали якісь реальні
старшокласники з Мерефи, Змієва чи Люботина, то російська мова доповіді
(помножена на англійську – слайдів) явно не полегшила для них сприйняття
матеріалу...

Отже, за підтримки ЦЕРН у незалежній Україні силами місцевих науковців було
організовано  й здійснено ще один маленький акт русифікації юних українців. 

В аспекті майбутнього фізики це може призвести до двох різних наслідків. Якщо
надалі держава таки проводитиме своє мовне законодавство в дію, і в
Каразінському університеті українська мова таки посяде помітне місце і в
фізичних спеціальностях теж, то частина молодих людей, для яких українська
зусиллями різних «популяризаторів науки» лишиться принципово чужою, шукатиме
способів учитися далі в Москві, або в близькому Бєлгороді... 

А якщо «каразінські» фізики й надалі залишаться переважно російськомовними –
вони ризикують повторити долю своїх одеських колег. Як відомо, в Одесі (колись
теж славній фізичними традиціями) десь із п’ять років тому припинив існування
окремий фізичний факультет ОНУ імені І.Мєчнікова. Причина проста: не стало
студентів. Адже на фізику вступали переважно не «гонорові» одесити (їх більше
приваблювала юридична академія), а скомні хлопці й дівчата з україномовної
провінційної Одещини, Миколаївщини, Вінничини, Кіровоградщини. І за нових
реалій вони почали обирати інші, україномовні виші в інших містах, нехтуючи
ОНУ, де фізфак далі лишався майже всуціль російськомовним...

Не зупинятимуся на тій очевидній обставині, що російська давно вже не є
міжнародною мовою науки. Років із десять тому на організованому Єврокомісією
форумі автор змушений був погодитися з думкою французького експерта (росіянина
з походження): для науки пост-радянського простору ця мова відіграє відверто
негативну роль, оскільки прив’язує українських, білоруських чи казахських
учених до невеликого вже сьогодні «російськомовного гетто», не випускаючи їх на
загальносвітові обшири, перепусткою куди сьогодні є лише англійська.

Тому в незалежній державі, якою є Україна, поруч із міжнародною мовою науку
може обслуговувати ще й своя національна (=державна) мова, але в жодному разі
не мова колишньої метрополії, яка вже сім років веде проти України неоголошену
війну. 

Лишається сподіватися, що Національна академія наук пошле колись зрозумілий
сигнал керівникам харківських академічних установ: ті, хто обрав за мету -
розбудовувати українську науку, повинні (незалежно від походження, рідної чи
розмовної мови, політичних уподобань) виявляти громадянську лояльність до свого
народу і своєї держави. Навіть попри абсолютно неприпустиме ставлення цієї
держави до власних інтелектуальних еліт (це – тема окремої розмови, бо таке
ставлення теж конче потрібно змінювати). 

А до цієї громадянської лояльності безумовно належить і розуміння того, що «the
native language for Ukrainians is Ukrainian», а дотеперішня (і, сподіваюся,
майбутня так само) столиця української фізики називається англійською таки
«Kharkiv».

Максим СТРІХА,

доктор фізико-математичних наук, професор
