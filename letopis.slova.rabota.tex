% vim: keymap=russian-jcukenwin
%%beginhead 
 
%%file slova.rabota
%%parent slova
 
%%url 
 
%%author 
%%author_id 
%%author_url 
 
%%tags 
%%title 
 
%%endhead 
\chapter{Работа}
\label{sec:slova.rabota}

%%%cit
%%%cit_head
%%%cit_pic
\ifcmt
tab_begin cols=2
	caption Сайчук заявил, что \enquote{русскоязычность Шевченко} обесценила для него победу Украины в сложном матче. Фото: Прямой

  pic https://img.strana.ua/img/article/3412/rabotnik-kanala-poroshenko-51_main.jpeg
	width 0.55

	pic https://strana.ua/img/forall/u/11/52/%D0%97%D0%BD%D1%96%D0%BC%D0%BE%D0%BA_%D0%B5%D0%BA%D1%80%D0%B0%D0%BD%D0%B0_2021-06-30_%D0%BE_17.25_.07_.png
	width 0.35

tab_end
\fi
%%%cit_text
\emph{Работник} телеканала \enquote{Прямой}, принадлежащего экс-президенту, лидеру
\enquote{Евросолидарности} Петру Порошенко, Андрей Сайчук раскритиковал главного
тренера сборной Украины Андрея Шевченко за то, что тот говорил на
пресс-конференции после победы над сборной Швеции на русском языке.
Соответствующий пост он опубликовал в своем Facebook в среду, 30 июня
%%%cit_comment
%%%cit_title
\citTitle{Работник канала Порошенко раскритиковал Шевченко за русский язык}, , strana.ua, 30.06.2021
%%%endcit

%%%cit
%%%cit_head
%%%cit_pic
%%%cit_text
Следует обратить особое внимание на так называемый «центр предотвращения
дезинформации» (ЦПД), который под патронатом СНБО и сочинил указанный
«глоссарий». Руководитель этой мутной организации, некая Полина Лысенко
заявила, что \emph{работа} центра содержит в себе три этапа — сбор информации, ее
анализ и реагирование. Самое интересное касается реагирования.  «Реагированием
для центра является сотрудничество с правоохранительными органами. В случаях,
когда во время сбора и анализа информации мы видим, что есть признаки
уголовного производства, то сообщаем об этом правоохранителям и просим их
отреагировать», — поведала Лысенко
%%%cit_comment
%%%cit_title
\citTitle{Краткий словарь грантоедов под редакцией СНБО / Лента соцсетей / Страна}, 
Александр Карпец, strana.news, 26.10.2021
%%%endcit
