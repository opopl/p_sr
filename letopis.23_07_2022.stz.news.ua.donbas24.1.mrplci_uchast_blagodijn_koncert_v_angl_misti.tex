% vim: keymap=russian-jcukenwin
%%beginhead 
 
%%file 23_07_2022.stz.news.ua.donbas24.1.mrplci_uchast_blagodijn_koncert_v_angl_misti
%%parent 23_07_2022
 
%%url https://donbas24.news/news/mariupolci-vzyali-ucast-v-organizaciyi-ta-provedenni-blagodiinogo-koncertu-v-angliiskomu-misti
 
%%author_id demidko_olga.mariupol,news.ua.donbas24
%%date 
 
%%tags 
%%title Маріупольці взяли участь в благодійному концерті в англійському місті
 
%%endhead 
 
\subsection{Маріупольці взяли участь в благодійному концерті в англійському місті}
\label{sec:23_07_2022.stz.news.ua.donbas24.1.mrplci_uchast_blagodijn_koncert_v_angl_misti}
 
\Purl{https://donbas24.news/news/mariupolci-vzyali-ucast-v-organizaciyi-ta-provedenni-blagodiinogo-koncertu-v-angliiskomu-misti}
\ifcmt
 author_begin
   author_id demidko_olga.mariupol,news.ua.donbas24
 author_end
\fi

\ii{23_07_2022.stz.news.ua.donbas24.1.mrplci_uchast_blagodijn_koncert_v_angl_misti.pic.front}

\begin{center}
  \em\color{blue}\bfseries\Large
22 липня в англійському місті Вітчерч відбувся благодійний концерт на
підтримку біженців України, в організації та проведенні якого взяли
участь маріупольці
\end{center}

Українці, які виїхали закордон, намагаються не сидіти склавши руки, а
допомагати своїй країні та співгромадянам, які опинилися у важкому
матеріальному становищі. Зокрема, маріупольська хореографиня і танцівниця,
керівниця дитячої танцювальної студії \enquote{Імпульс} ПК \enquote{Молодіжний} \href{https://www.facebook.com/apaniotova}{\emph{Анна Паніотова}}%
\footnote{\url{https://www.facebook.com/apaniotova}}
та викладачка англійської мови і громадська діячка Маріуполя \href{https://www.facebook.com/profile.php?id=100000962395016}{\emph{Олена Попова}}%
\footnote{\url{https://www.facebook.com/profile.php?id=100000962395016}}
долучилися до організації та проведення благодійного концерту на підтримку
біженців України, який пройшов 22 липня в англійському місті Вітчерч. Анна та
Олена — дві маріупольчанки, які опинилися разом в Англії — познайомилися з
\href{https://www.facebook.com/profile.php?id=100008256349355}{\emph{Оленою Батовською}},%
\footnote{\url{https://www.facebook.com/profile.php?id=100008256349355}} докторкою мистецтвознавства, професоркою Харківського
національного університету мистецтв імені І. П. Котляревського. Саме вона
створила хор \enquote{Калина}. Олена з Анною в Маріуполі займалися танцями і виступали
в студії танців фламенко \enquote{ROSAS PARA MARIA}. Жінки вирішили об'єднати зусилля і
створити спільний колектив. Українки почали активну діяльність у Вінчестері, де
всі вони і проживали. Там вони змогли заявити про себе і взяти участь в
декількох концертах.

\ii{23_07_2022.stz.news.ua.donbas24.1.mrplci_uchast_blagodijn_koncert_v_angl_misti.pic.1}

Концерт, що відбувся у Вітчерчі має досить високий рівень. Він був
організований спільно з Британським фольклорним музичним клубом (місто Вітчерч)
та колективом українок, який було створено у Вінчестері.

\begin{leftbar}
\emph{\enquote{Witchurch folk club — це такий англійський клуб, який ґрунтується на
англійській фольклорній музиці. Ось організатори цього клубу
запропонували мені станцювати український народний танець. Я
запропонувала дівчатам, які співають взяти участь в цьому благодійному
заході і вони з радістю погодилися}}, — розповіла Анна Паніотова.
\end{leftbar}

\ii{23_07_2022.stz.news.ua.donbas24.1.mrplci_uchast_blagodijn_koncert_v_angl_misti.pic.2}

Танцівницею та ведучою на концерті була Олена Попова, якій вдалося весь вечір
тримати увагу багаточисельної публіки. Жінки підготували один спільний номер,
який отримав назву \emph{\enquote{Віночок}}. Спочатку його готували до Івана Купала. Але саме
в цьому номері вдалося дуже вдало поєднати хореографію та вокал.

\begin{leftbar}
\emph{\enquote{Ми не хочемо називати українців, які опинилися в інших країнах, біженцями.
Ми називаємо себе гостями з України. Дуже радіємо, що за кордоном
можемо приносити користь власній державі та відчувати себе корисними
завдяки таким заходам}}, — наголосила Олена Попова.
\end{leftbar}

\ii{23_07_2022.stz.news.ua.donbas24.1.mrplci_uchast_blagodijn_koncert_v_angl_misti.pic.3}

На концерті були представлені українські та британські музиканти і танцюристи.
Саме завдяки цьому вдалося представити насичену програму та різнопланову
музику. Адже зі сцени лунала і народна, і класична, і сучасна музика. Проте
найголовнішим на цьому заході все ж \emph{\textbf{стала підтримка і
популяризації української культури.}} На концерті були представлені:

\begin{itemize} % {
\item Гімн України (виконував Хор \enquote{Калина});

\item Танок з прапором (виконували Анна та Олена);

\item Ой у лузі червона калина (Хор \enquote{Калина});

\item Чарівна скрипка (Ноелла);

\item Ой у вишневому садку (Лала);

\item Несе Галя воду (Олена та Ноелла);

\item Танок Анни;

\item Купальські пісні (Хор \enquote{Калина}).
\end{itemize} % }

\href{https://archive.org/details/video.23_07_2022.donbas_novyny.vystup_ukrainciv_v_vincherchi}{%
Відео: Виступ українців у Вінчерчі, Донбас Новини, 23.07.2022}%
\footnote{%
\url{https://www.youtube.com/watch?v=OPfeElSPIG0}, \par%
Internet Archive: \url{https://archive.org/details/video.23_07_2022.donbas_novyny.vystup_ukrainciv_v_vincherchi}%
}

\begin{leftbar}
\emph{\enquote{Дуже важливо, що в цьому концерті брали участь дві сторони: і британська, і
українська. Це був спільний, платний концерт. Білет коштував 15 фунтів. Було
продано близько 150 квитків. Зала була переповнена. Це дуже велике досягнення і
дуже гарний результат, адже гроші від цього заходу підуть на підтримку
українців. Після концерту люди підходили, дякували, фотографувалися. Це був
справжній фурор}}, — поділилася Олена Попова.
\end{leftbar}

\begin{leftbar}
\emph{\enquote{Захід вийшов інтернаціональний. Публіка дуже підтримувала, зустріла тепло,
було багато оплесків. Ми й досі знаходимося під сильними враженнями}} — додала
Анна Паніотова.
\end{leftbar}

Концерт проходив у старовинній церкві у Вінчерчі, що додавало заходу особливої
атмосфери. У Анни та Олени ще багато нових ідей та планів. Можливо, наступний
захід буде проведено у Лондоні і стане ще більш масштабним.

\ii{23_07_2022.stz.news.ua.donbas24.1.mrplci_uchast_blagodijn_koncert_v_angl_misti.pic.4}

Нагадаємо, раніше Донбас24 розповідав, як \href{https://donbas24.news/news/yak-morpixi-zaxishhali-mariupol-istoriya-odnogo-z-muznix-voyiniv}{морпіхи захищали Маріуполь}.%
\footnote{Як морпіхи захищали Маріуполь: історія мужнього воїна, Наталія Сорокіна, donbas24.news, 22.07.2022, \par\url{https://donbas24.news/news/yak-morpixi-zaxishhali-mariupol-istoriya-odnogo-z-muznix-voyiniv}}

ФОТО: з особистих архівів Анни Паніотової та Олени Попової.

\ii{insert.author.demidko_olga}
%\ii{23_07_2022.stz.news.ua.donbas24.1.mrplci_uchast_blagodijn_koncert_v_angl_misti.txt}
