% vim: keymap=russian-jcukenwin
%%beginhead 
 
%%file 12_11_2020.fb.roman_barashev.1.sinichkin_den
%%parent 12_11_2020
 
%%url https://www.facebook.com/roman.barashev/posts/2162487600550374
%%author 
%%tags 
%%title 
 
%%endhead 

\subsection{12 ноября – Синичкин день}
\Purl{https://www.facebook.com/roman.barashev/posts/2162487600550374}
\Pauthor{Барашев, Роман}

\index[rus]{Синичкин День}

12 ноября – Синичкин день. На Руси издавна приметили: в это время птицы
перебираются поближе к человеческому жилью, чтобы пережить зимний холод и
голод.

А в детско-взрослой газете «АЙДА!» -- свежая заметка 9-летнего корреспондента
студии «ЖУРНАЛИСТИКА ДЛЯ ШКОЛЬНИКОВ» (РЦНК, Киев) Елены ЛЬВИЦЫ.

\subsubsection{О МОНАСТЫРЯХ, ЖИВОТНЫХ и ОЛЬШАНКАХ}
\Pauthor{Львица, Елена}

Недавно мы с мамой посетили два монастыря Киевской области. Женский монастырь
«Отрада и утешение» находится в Ольшанке. Спасо-Преображенский мужской
монастырь -- в Нещерове.

\ifcmt
pic https://scontent.fiev22-1.fna.fbcdn.net/v/t1.0-9/124833761_2162486907217110_7111676768517325291_o.jpg?_nc_cat=100&ccb=2&_nc_sid=730e14&_nc_ohc=fEUppjyI6pYAX_0C2z7&_nc_ht=scontent.fiev22-1.fna&oh=6411602d91333218ba04a2dcd8a20f1a&oe=5FD4B5A7

pic https://scontent.fiev22-1.fna.fbcdn.net/v/t1.0-9/125071808_2162486950550439_5958915834125954993_o.jpg?_nc_cat=106&ccb=2&_nc_sid=730e14&_nc_ohc=8tO_1sfD0msAX8sVnCh&_nc_ht=scontent.fiev22-1.fna&oh=60d58e51a17e77026d7c67408b8baff1&oe=5FD30488

pic https://scontent.fiev22-1.fna.fbcdn.net/v/t1.0-9/125093314_2162486977217103_1473633567779209947_o.jpg?_nc_cat=108&ccb=2&_nc_sid=730e14&_nc_ohc=e3F4q5v68kgAX-FI3La&_nc_ht=scontent.fiev22-1.fna&oh=0df50c9d47c5de423833cbc80b08592f&oe=5FD2C773

pic https://scontent.fiev22-1.fna.fbcdn.net/v/t1.0-9/125175002_2162492223883245_3982184636668688933_o.jpg?_nc_cat=103&ccb=2&_nc_sid=730e14&_nc_ohc=BBJcUeu_-p8AX-v9I7M&_nc_ht=scontent.fiev22-1.fna&oh=a0e0f467ca674e9d81067fcd90c8376a&oe=5FD2FCA2
\fi

В мужском монастыре организован настоящий приют для котов – их тут больше ста,
за ними ухаживают, кормят и лечат. Даже продают. А прихожане привозят мурлыкам
корм. 

Здешние коты приветливые, дают себя погладить. Есть и совсем маленькие котята.

Налюбовавшись монастырскими подопечными, мы спустились в подземный храм, который был создан 250 лет назад. 

И в женском монастыре в Ольшанке -- много животных: курицы, петухи, индюки, бык, корова и телята. 

Как-то сюда привезли корову, которая никак не хотела возвращаться из пастбища в
хлев. Уж и десять человек тащило её, заталкивало, а она упёрлась и всё. Но тут
у кого-то в коровнике зазвонил мобильный, заиграла классическая музыка. И
корова успокоилась и на этот звук пошла. С тех пор в обители «Отрада и
утешение» и коров утешают классикой.

А еще тут есть "помидорник" – теплица, где выращивают овощи. Также монахини
собирают травы, из которых готовят «Ольшанский чай». Создают и расписывают
керамические изделия – чашки, блюдца, подсвечники.

Ольшанка – слово какое тёплое, правда? Оказывается, это устаревшее название
птицы зорянки, её ещё называли малиновкой.  Я заглянула в Википедию и
поразилась: сёла с таким названием есть и в России, и в Белоруссии, и в
Казахстане, и даже в Турции. В Украине -- 8 рек Ольшанок, одна из них – приток
Днепра, Ольшанками называются 20 украинских и 27 российских сел.

Елена Львица, 9 лет (помогал Роман ГЕОГРАФ)
