% vim: keymap=russian-jcukenwin
%%beginhead 
 
%%file slova.semja
%%parent slova
 
%%url 
 
%%author 
%%author_id 
%%author_url 
 
%%tags 
%%title 
 
%%endhead 


Народ Бутана сплотился как одна \emph{семья}. Король постоянно ездил во все районы,
чтобы лично оценить ситуацию и оказать моральную поддержку и успокоить воинов
на ковидной передовой, \textbf{Ковиду не победить сплоченный и счастливый
народ. Даже если денег не так много}, Владимир Спиваковский, strana.ua,
01.06.2021

%%%cit
%%%cit_pic
\ifcmt
  pic https://strana.ua/img/forall/u/0/34/photo_2021-06-18_12-08-14(1).jpg
	caption Влад Яма рассказал \enquote{Стране}, что будет помнить о великом танцоре Григорие Чапкисе.
\fi
%%%cit_text
Григорий Чапкис родился в Румынии в 1930 году. Он рассказывал, что фамилия
\emph{семьи} была Чаопеску, но после эвакуации во время Второй мировой войны,
когда \emph{семья} осела в Киеве, они стали Чапкисами.  В столице Чапкисы осели
случайно.  Возвращались из эвакуации в Казахстане в Кишинев, но по дороге их
высадили, поскольку \emph{вся большая семья} заболела брюшным тифом. Это
произошло спустя всего два месяца после того, как из Киева выбили немцев. В
интервью Чапкис рассказывал, что видел, как вешали недобитых фашистов,
принимавших участие в убийствах. Он участвовал в восстановлении Крещатика,
который стоял в руинах.  Артист рассказывал, что танцевал с трех лет. В
\emph{семье} было одиннадцать детей и, чтобы помочь родителям прокормить всех,
он выходил на улицу, клал шапку и танцевал для прохожих. Это стало хорошей
закалкой.  Позже он связал свою жизнь с танцами. Рассказывал, что в 15 лет
танцевал гопак в кремлевском театре, когда его подхватил и посадил себе на
колени Сталин.  В 16 лет был зачислен в знаменитый ансамбль имени Вирского, где
танцевал двадцать шесть лет
%%%cit_comment
%%%cit_title
\citTitle{Чапкис умер - причина смерти, биография, жена}, 
Оксана Малахова, strana.ua, 13.06.2021
%%%endcit

%%%cit
%%%cit_head
%%%cit_pic
\ifcmt
  pic https://img.strana.ua/img/article/3405/futbolist-kalinin-rasskazal-52_main.jpeg
	caption Игрок объяснил, почему сделал российский паспорт. Фото: business-gazeta.ru
\fi
%%%cit_text
\enquote{Я не знаю, почему народ у нас такой... Столько ненависти!} - сказал Калинин.
По его словам, хейтеры сыплют такими угрозами, которые он сам и врагу бы не
сказал. Футболист также объяснил, почему решился на смену гражданства: ему
было бы трудно ездить в Крым к \emph{бабушке и матери}.  \enquote{Для меня \emph{моя семья} - это
главное в жизни, поэтому я выбрал Россию}, - заметил защитник
%%%cit_comment
%%%cit_title
\citTitle{Футболист Калинин рассказал об угрозах украинцев после смены гражданства}, 
Наталья Полулях, strana.ua, 25.06.2021
%%%endcit

%%%cit
%%%cit_head
%%%cit_pic
%%%cit_text
Итак, благодаря везению и предусмотрительности удалось сохранить существование
\emph{Семей} в тайне. Это было к лучшему — можете быть уверены, что Пророки
постарались бы любыми путями заполучить секрет долголетия.  Как таковые,
\emph{Семьи} отстранились от участия в событиях, приведших ко Второй
Американской Революции. Но многие члены \emph{Семей} были членами подполья,
пользовались полным доверием в Кабале и участвовали в сражении,
предопределившем падение Нового Иерусалима. Воспоследовавший период
дезорганизации дал нам возможность изменить возраст тех из нас, кто за
прошедшее время стал подозрительно старым. В этом весьма помогли наши
братья-долгожители, которые, будучи членами Кабала, заняли ключевые
государственные посты в период Реконструкции.  На собрании \emph{Семей} в 2075
году, в год принятия Общественного Договора, многие высказывались за то, что
пришла пора без обиняков объявить о своем существовании, так как гражданские
свободы были полностью восстановлены. Но большинство с этой точкой зрения в то
время не согласилось... возможно потому, что сохранение конспирации стало
привычкой. Но постепенное и неуклонное возрождение культуры, доброжелательности
и хороших манер, разумная ориентация обучения, возрастающее уважение к свободе
личности и ее правам, стабильно имеющие место вот уже на протяжении пятидесяти
лет, вселили в нас надежду на то, что наш час пробил и мы спокойно можем
объявить о своем существовании человечеству, заняв среди людей подобающее нам
место необычного, но тем не менее уважаемого меньшинства
%%%cit_comment
%%%cit_title
\citTitle{Дети Мафусаила}, Роберт Хайнлайн
%%%endcit
