% vim: keymap=russian-jcukenwin
%%beginhead 
 
%%file 23_07_2021.fb.fb_group.story_kiev_ua.1.uslyshano_v_kieve_4
%%parent 23_07_2021
 
%%url https://www.facebook.com/groups/story.kiev.ua/posts/1713099712220156/
 
%%author Киевские Истории
%%author_id fb_group.story_kiev_ua
%%author_url 
 
%%tags gorod,kiev,poslovica,slovo,ukraina,uslyshano_v_kieve
%%title Услышано в Киеве - ЧАСТЬ ЧЕТВЕРТАЯ
 
%%endhead 
 
\subsection{Услышано в Киеве - ЧАСТЬ ЧЕТВЕРТАЯ}
\label{sec:23_07_2021.fb.fb_group.story_kiev_ua.1.uslyshano_v_kieve_4}
 
\Purl{https://www.facebook.com/groups/story.kiev.ua/posts/1713099712220156/}
\ifcmt
 author_begin
   author_id fb_group.story_kiev_ua
 author_end
\fi

ЧАСТЬ ЧЕТВЕРТАЯ-ПОСЛЕДНЯЯ.

Комменты для первых трех частей поста уже закрыты.

За 4 дня обсуждений, тема «Услышано в Киеве» получила две с лишним тысячи
комментов и собрала больше сотни фраз и выражений.

Всем спасибо!

Комменты этой части не закрываю. Если будет что-то стоящее, буду вставлять по
мере накопления, пока админам не надоест одобрять изменения.

Опять повторяюсь – у меня собраны только фразы и выражения. Названия, клички,
прозвища не рассматриваются!

Открывайте свои темы. Вот Sergej Strakhov только открыл тему «НАШИ КИЕВСКИЕ
АЛЬМА-МАТЕРИ» и сразу посыпалось. Приобщайтесь!

Итак, мой и ваш, уже совсем не короткий список:

\begin{itemize}
\item  «Аз ох-н-вей» (идиш/иврит) -  когда (хочется сказать) «ох!» и «вей!». Первое междометие употребляется не от хорошей жизни, второе же слово означает «горе». 
\item  «А Менделе герой мит бритва» - о человеке, который храбрый только с ножом в руке. Евбазовское.
\item  А он крутой шнарант – от фамилии хозяина дома №63 на углу Межигорской и Еленовской. Он был знаменит тем, что сдавал временную жилплощадь торгашам. Там заключались сделки насчет купли-продажи оптовых партий товаров и останавливались разношерстные аферисты. Именно поэтому впоследствии слово «шнарант» стало олицетворять авантюриста, прохиндея, нечистого на руку дельца. Когда дом Шнаранта сносили, в стенах нашли тайники с деньгами, расписками, а рабочие обнаружили в подвале шкатулку с золотыми монетами.
\item  «А Сара хочет негра» – в ответ на «Я хочу…».
\item  «А там вся аристократия липовая» — ироническое наи¬менование владельцев усадеб на Липках. В стари¬ну это выражение не имело того саркастическо¬го смысла, который вкладывается в него теперь.
\item  «Ах ты старая кошолка» (от кошеля) – Евбазовское приветствие.
\item  «Біля Лева печінка дешева» - рядом с Самсоном был базарчик.
\item  «Будете в Киеве- заходите в Лавру» – так киевляне приглашали в гости, но своего адреса не давали.
\item  «Ваш борщ такий смачний, що у цицьках пече. Так виньте цицьки из борщу» – без комментариев.
\item  «Вей из мир» (идиш) – горе мне.
\item  «Він мов гівно з просом» – він гливкий як в’ялений інжир. 
\item  «В инвалидку за бананами» – пойти в магазин для инвалидов.
\item  «В Киеве и кирпич не краснеет» – поговорка богомольцев. 
\item  «В Киеве одни стены святы, а люди поганы» - поговорка богомольцев.
\item  «Вона так кричить, аж ВИРЯЧИЛА (выпучила) очи» - без комментариев.
\item  «В свинячий голос» - когда что-то делается не вовремя или с опозданием.
\item  «Встретимся на шариках» - на улице Мечникова у дома с шарами на парапете. 
\item  «Встретимся у пеликанов» - у памятника пеликанам  на Лютеранской.
\item  «Встретимся у Черного моря» -  в парке Шевченко у фонтанчика.
\item  «Вус трапилось» (идиш/укр.) – что случилось?
\item  «Выпить лампочку коньяку» -  выпить из плошки для иллюминации.
\item  «Вы посмотрите на эту дарницкую дворянку» – возглас возмущения евбазовских торговок.
\item  «Гейн какен вайтер» (идиш – иди какать подальше) – ласковый посыл.
\item  «Гиб а кик, шо робыться» (идиш/укр.) – Взгляни, что творится.
\item  «Глухих повезли» - если человек не расслышал и переспросил.
\item  «Граф Алексей Бобринский - «А кто это будет делать - Бобринский?», "Теж мені, графиня Бобринська!"», и пр. - на месте Щорса, стоял памятник графу Бобринскому, который много в жизни успел сделать. Вот оттуда и пошло.
\item  «Грузи на Бессарабку» – динамовско-стадионное.
\item  «Дай мне кецик» - кусочек чего-либо съестного. Евбазовское.
\item  «Двойная половинка» - первые кофейни на Крещатике и на Заньковецкой, в которых готовили аппаратный кофе, в 80-е назывались кафетериями и были оснащены автоматами, настроенными на стандартную порцию кофе на стакан воды. Напиток получался так себе. Для кофеманов работница кафетерия вручную отмеряла две порции кофейного порошка и половинку порции воды. Получалось полстакана (разливали в граненые стаканы) кофе, носившего название «двойная половинка». 
\item  «Дрек мит фефер» (идиш– гавно с перцем) – крученый пройдоха. 
\item  «Душить мерзавчика» – пить водку тайком в общественных местах.
\item  «Езжай на Бесарабку, купи себе петуха и морочь ему голову» - без комментариев.
\item  «Если хочешь поработать ляжь поспи и все пройдет» - без комментариев.
\item  «Жизнь дала трещину еду на Троещину» - без комментариев.
\item  «Забегалась, как Ицикова сучка» - без комментариев.
\item  «Закрой душу» - одень шарф.
\item  «Їхати на ковбасі»  - У старих трамваях був ззаду маленький майданчик, на який ставали чоловіки, а тримались за драбину, яка вела на дах. Отак їхали від зупинки до зупинки.
\item  «Иди на Евбаз курочек ловить» – ласковый посыл.
\item  «Пойти в Яму» - пойти за продуктами в низину в магазинный ряд между ул. Бастионной и Струтинского (Болсуновской). 
\item  «Из под пятницы – суббота» - если из-под рубашки видно было майку или из-под пиджака - подол рубашки - это считалось очень неряшливо.
\item  «Их вейс?» (идиш) – откуда я знаю?
\item  «Как ты мене дорог» - говорилось если треплют  нервы.
\item  «На городі бузина, а в Києві дідько» - що ти дурню верзеш.
\item  «На ней бэрэт, мышиный цвет и юпочка в полосочку» – модница.
\item  «Наружное образование» – умение одеваться и держать себя на людях.
\item  «Насипь мені миску борщу» - без комментариев.
\item  «Не делай мне вырванные годы» - без комментариев. 
\item  «Не дрек мне копф» (идиш/рус.) - не гамни мне голову. 
\item  «Не звони, метличка, метлик фартовый» – сленг проституток про богатого клиента.
\item  «Не люблю, коли вмер і дивиться» – про ненадежного человека, который может изменить  свое решение.
\item  «Нема лаве - нема кохання» – без денег – никуда.
\item  «Не питай кума, чому заплакані очі» - без комментариев.
\item  «Не швиндай» – не слоняйся без цели.
\item  «Нi з глови слова - нi з дупи перду» – не с головы, не с жопы.
\item  «Ні кує, ні меле» – без комментариев.
\item  «Ну ты и босяк» – так звались голоногие изгои, артельная подольская «босая команда» временные грузчики на Притыке (пристани) еще при царе. Они спали там босиком и на подошвах у них мелом были написаны цены, за которые они нанимались.
\item  «Одевай курточку и иди в Дарницу» –  из Адъютанта.
\item  «Он из бичевых (аристократов)» – из польских помещиков, которые ездили с бичами (кнутами). 
\item  «Опять за рыбу гроши» – ну сколько можно?
\item  «Пани мої дрібнесенькі , а воші як біб» - без комментариев.
\item  «Партач (a parte) подольский» – кустарь-одиночка.  
\item  «Пенициллинка (кирпичка, дрожжевой, ДШК) задула» – потянуло запашком.
\item  «Підожды» - без комментариев.
\item  «Плевать с лаврской колокольни» - без комментариев.
\item  «Позовите городового - пусть он отведёт меня в сумасшедший дом» – в конфликтных ситуациях местного значения.
\item  «Поедем на эмочке?» – на такси.
\item  «Прошу, пані, до мешкані на скляночку хербати» - приходите в гости на чай.
\item  «Пся крев, а скільки гонОру» - пренебрежительно о заносчивом человеке.
\item  «Ривка така медікована» – такая хитрая соседка.
\item  «Розумному лихо, дурному радість» - без комментариев.
\item  «Рупь или в морду?» – предложение босяка к кавалеру гуляющей парочки.
\item  «Рубай компот - он жирный» - без комментариев.
\item  «Самый цимес» – великолепно.
\item  «С Короленка видна Ленка» – с Короленка, 15 (с милиции) видна Сибирь…
\item  «Сиди жди, поки, жаба цицки дасть» – жди, мечтай - может дождешся.
\item  «Так далеко, як до Києва рачки» – о чем-то далеком.
\item  «Та, ладно…» - не верю, обманываете.
\item  «Танцуем от печки» - из лексикона знаменитого киевлянина Соломона Шкляра, героя известной песни, который, вообще-то, был парикмахером и работал на Бибиковском бульваре. Однако решил, что сможет преуспеть и совершенно неожиданно открыл в доме Пфалера на большой Васильковской, 10 школу танцев. Нанятые учителя танцев преподавали, а Шкляр, как сказали бы сейчас, занимался конферансом. Он объяснял фигуры танца, беспрестанно шутил, много комментировал, давал рекомендации, короче, создавал настроение. 
\item  «Танька (или любое другое имя) на базаре семечками торгует» - в ответ на уничижительное обращение.
\item  «Та тож кожемяцкие аристократы» - без комментариев.
\item  «Твои штаны меня в Киеве держат» - без комментариев.
\item  «Тебе на Берковцах прогулы ставят» – тебя заждались.
\item  «То цирульник из-за канавы» – без комментариев.
\item  «Тю…» – без комментариев.
\item  «Тиць-гриць» - что-то, сказанное не в тему.
\item  «Ума нет- на Бесарабке не купишь» - без комментариев.
\item  «Устроить пасовку» – прогулять уроки.
\item  «Усюсяный-ухвысяный» (совр. уси-пуси) - Это означало, что младенец "любленный-голубленный".
\item  «Химині кури, а Мотрині яйця» – спорный вопрос.
\item  «Хочу на копки-баранки» – посади меня на плечи.
\item  «Цирк на дроті» — о нелепой ситуации.
\item  «Цьом, цьом, Пацю в срацю»  - о ком-то избалованном.
\item  «Через духовку на Куреневку» – не в том направлении.
\item  «Чорти шо и с боку бантик» - на какую-то нелепость.
\item  «Что на Шулявке конь - на Липках кошка» – без комментариев.
\item  «Ша, селяне, земля ваша» – не шумите за столом.
\item  «Шоб Вы мне были здоровы» -без комментариев.
\item  «Шоб ты жил напротив своего дома» – лукьяновское выражение про Деда Лукьяна (тюрьму).
\item  «Шо Вам сказать…» – нет слов.
\item  «Шо за фойле штык» (идиш) – что за дурная шутка?
\item  «Щас, возьму разбег с Бессарабки» – выражение торговок Сенного рынка на выпады насчет цены товара.
\item  «Що з возу впало, то пропало» – тысячелетняя киевская поговорка с Андреевского спуска, где была таможня.
\item  «Що ти швендяєш?» – чего гуляешь?
\item  «Это форменное надувательство» – киевские мясники «надували» мясо для улучшения внешнего вида;
\item  «Это что за шарашкина контора» – контора строительного подрядчика Шарашкина, неофициальным девизом которого было: «Главное подрядиться», т.е. взять аванс. Называлась в противовес конторе Льва Гинзбурга.
\item  «Я Вас умоляю» -  ну, что Вы.
\item  «Яхна ты такая» – грубая, не образованная еврейка. Так ул. Ярославскую называли Яхнославской.

\end{itemize}
