% vim: keymap=russian-jcukenwin
%%beginhead 
 
%%file 03_04_2021.fb.rostovcev_oleg.1.stalin_izbranie_partia_gensek
%%parent 03_04_2021
 
%%url https://www.facebook.com/groups/492322934446416/permalink/1428508030827897/
 
%%author 
%%author_id 
%%author_url 
 
%%tags 
%%title 
 
%%endhead 

\subsection{3 апреля 1922 г. Избрание Иосифа Сталина генеральным секретарем ЦК РКП(б).}
\Purl{https://www.facebook.com/groups/492322934446416/permalink/1428508030827897/}


\ifcmt
  pic https://scontent-amt2-1.xx.fbcdn.net/v/t1.6435-9/168291007_476118540191980_666474931104706764_n.jpg?_nc_cat=105&ccb=1-3&_nc_sid=825194&_nc_ohc=obiuP7iBzpAAX-1bSVz&_nc_ht=scontent-amt2-1.xx&oh=ebf6a34738d871c68caa9305d161c29e&oe=609058B5
  width 0.4
\fi

На Пленуме ЦК РКП(б) 3 апреля 1922 года Сталин был избран в Политбюро и Оргбюро ЦК РКП(б), а также Генеральным секретарём ЦК РКП(б). Первоначально эта должность означала лишь руководство аппаратом партии, а в качестве лидера партии и правительства всеми продолжал восприниматься Председатель СНК РСФСР Ленин.
С 1922 года, ввиду болезни, Ленин фактически отошёл от политической деятельности. Внутри Политбюро Сталин, Зиновьев и Каменев организовали «тройку», основанную на противодействии Троцкому. Все три партийных лидера на тот момент совмещали целый ряд ключевых постов. Зиновьев возглавлял влиятельную Ленинградскую парторганизацию, одновременно являясь председателем Исполкома Коминтерна. Каменев возглавлял Московскую парторганизацию и одновременно также руководил Советом Труда и Обороны, объединявшим ряд ключевых наркоматов. С отходом Ленина от политической деятельности именно Каменев стал чаще всего председательствовать вместо него на заседаниях Совнаркома. Сталин же объединял руководство одновременно Секретариатом и Оргбюро ЦК, возглавляя также Рабкрин и наркомнац.
В противовес «тройке», Троцкий возглавлял Красную армию на ключевых должностях наркомвоенмора и предреввоенсовета.
В сентябре 1922 года Сталин впервые ярко проявил свою склонность к традиционному российскому великодержавию. Согласно поручению ЦК, он, как нарком по делам национальностей, подготовил свои предложения по урегулированию отношений Москвы с советизированными национальными окраинами бывшей Российской империи. Сталин предложил план «автономизации», в частности Грузия должна была оставаться в составе Закавказской республики. Этот план встретил ожесточённое сопротивление на Украине, и особенно в Грузии, и был отвергнут под давлением лично Ленина. Окраины вошли в состав советской федерации на правах союзных республик со всеми атрибутами государственности, впрочем, в условиях однопартийной системы фиктивными. Из названия самой федерации было устранено слово «Российский», и вообще географические наименования.
В конце декабря 1922 — начале января 1923 годов Ленин продиктовал «Письмо к съезду», в котором дал критические характеристики своим ближайшим соратникам по партии, в том числе Сталину, предложив снять его с должности генерального секретаря. Ситуация усугублялась тем, что в последние месяцы жизни Ленина имела место личная ссора Сталина с Крупской Н.К.
Письмо было оглашено среди членов ЦК накануне XIII съезда РКП(б), проходившего в мае 1924 года. Сталин подал в отставку, однако она не была принята. На съезде письмо огласили каждой делегации, однако по итогам съезда Сталин остался на своей должности.
