% vim: keymap=russian-jcukenwin
%%beginhead 
 
%%file 08_02_2020.news.ua.obozrevatel.1.evgen_dykyy.emancipacia
%%parent 08_02_2020
%%url https://www.obozrevatel.com/ukr/society/nezruchna-emansipatsiya.htm
%%tags evgen dykyy,mova,ukraine
 
%%endhead 

\subsection{Незручна емансипація}

\url{https://www.obozrevatel.com/ukr/society/nezruchna-emansipatsiya.htm}
\index[rus]{Дикий, Євген}

Жили собі чорношкірі американці. Білі пани спершу привезли їх рабами, а потім
під час однієї із панських міжусобиць самі ж і звільнили. Хтось із білих трохи
про це шкодував, хтось навпаки – трохи соромився рабовласництва предків та
втішав свою совість тим, що ж таки звільнили.

Жили вони поруч, але не разом, всім було відведено свій час, свій простір, свої
можливості – у білих одні, у чорних помітно менші, але ж вже не рабство, тож
хай і тому радіють. І всім так було ніби зручно. Аж якось одна чорношкіра
кравчиня відмовились бути зручною – не звільнила у автобусі місце, відведене
"тільки для білих". Її арештували та оштрафували. Але її приклад повторили
сотні чорношкірих. А далі тисячі. Вони стали нечемними, некомфортними та
незручними – і так із "нігерів" та "чорних" стали Афроамериканцями. І на цьому
їхнє рабство нарешті дійсно скінчилось.

Нещодавно дві українки відмовились бути зручними і терпіти у салоні показ
москальських серіалів. Їм так само довелося залишити автобус, як колись
афроамериканським активісткам. Нардеп від малоросійської більшості вже подав на
них заяву до поліції. Це нормальний розвиток подій – коли хтось сам
звільняється із рабства, це завжди непокоїть "білих панів" - колонізаторів, та
викликає у них бурхливу реакцію. Бо їм вперше в житті стає некомфортно та
незатишно.

Українка – стоматолог попередила, що буде спілкуватись з клієнтурою виключно
нашою мовою, і хай клієнти вирішують, їх це влаштовує, чи слід пошукати собі
іншого доктора. Реакція "білих панів" не забарилася, притому комплекс "панівної
раси" несподівано повилазив навіть з дуже інтелігентних та ніби зовсім не
"ватних" людей – аякже, чергова аборигенка відмовилась бути для них зручною та
комфортною! Як вона посміла! Абсурд звинувачень зашкалив – від звичних штампів
про "зоологічну русофобію" та "шовінізм" до кумедних зойків про "порушення
клятви Гіпократа" та "відмову у порятунку хворого" - ніби йдеться про
невідкладну допомогу та реанімацію, а не про дуже недешеву ринкову послугу, яку
замовляють наперед, і де наявний вибір із сотень пропозицій.

Шлях із рабства до рівності – він тільки такий. Перший крок – перестати бути
зручним для "білих панів", натомість стати зручним для себе. І це слід нарешті
збагнути нам, українцям. Ми надто звикли бути зручними, м’якими, поблажливими
до тих, хто чомусь досі вважає себе "вищою расою" та зверхньо ставиться до
"аборигенів" (хоч часто навіть не усвідомлює цього). Ми настільки звикли до
цього, що вважаємо ознакою інтелігентності переходити на російську, коли наш
співбесідник не дає собі праці бодай мінімально вивчити українську. Так, ми ж
дійсно всі добре знаємо мову "старшого брата" - ну то чого б не піти людині
назустріч? Тільки от цей рух за унікальними випадками виявляється рухом
одностороннім, і наш крок назустріч не лише не викликає зустрічного кроку, а
навпаки – закріплює звичний порядок речей, де попри декларовану рівність
насправді "одні тварини рівніші за інших" (с). Такий само звичний порядок, як
зарезервовані у автобусі "місця для білих".

Ми – аборигени, які досі не вийшли із резервації, відведеної нам "білими" на
землі наших предків. Межі цієї резервації днями публічно озвучила продюсерка
"плюсів", яка без жодної задньої думки повідомила, що нашою мовою зніматимуть
лише комедійні серіали, для всього іншого у нашій мови виявляється "важко
знайти потрібну тональність". Вона здається дійсно не хотіла нікого образити.
Просто такий кордон резервації, встановлений Імперією ще пару століть тому:
тубільцям дозволяється користуватись "хохляцкім нарєчієм" у комедіях, бо воно
на панське московське вухо смішно звучить, і загалом бути таким собі
колективним "денщиком Шельменком" при "гаспадінє ахвіцерє – насітєлє високай
культур-мультур". Якщо ж хтось із тубільців прагне говорити про розумні речі,
чи про серйозні почуття – теж без проблем, але виключно "на гарадском язикє".

Пам’ятаєте, у грудні минулого року зграя підлітків, вихованих на "культурє
русскаго міра" насмерть забила Артема Мірошніченка, який відмовився говорити
"на нармальнам язикє"? Гадаю, пасажири, які мовчки дивились на
дівчат-волонтерок за вікном автобуса, та переводили погляд на екран з
москальськими мелодрамами, не вважають себе схожими на малолітніх убивць із
Бахмута. Вони нікого не били, їм просто зробили незручно, і вони захистили свій
комфорт. А інтелігентна київська публіка, що обурюється "неполіткоректною"
стоматологинею, мабуть образилась би на порівняння з "ширнармасою" із
маршрутки. Вони ж не слухають "шансон" і не люблять Путіна, їм просто також
зухвала аборигенка наступила на звичний комфорт колонізаторів. Зовні вони такі
різні – привітна продюсерка "Плюсів", яка "шукає потрібну тональність" і
знаходить її лише в "язикє", обурені нечемною докторкою інтелігентні
"кієвлянчікі", забичені глядачі московського "мила" у провінційному бусіку, і
зрештою виховані з дитинства на "блатнякє" та "Братє-2" малолітні бахмутські
кати. Але їх всіх єднає одне – вони по інший бік паркану культурного та мовного
гетто, відведеного Московією нам, українцям.

Вибачайте, пані і панове, але ми вже пережили той час, коли просто вижити, не
бути фізично знищеними, та десь в глибині резервації бодай якось зберегти мову
предків було для нас спершу майже недосяжною мрією, а потім величезним
досягненням. Ми досягли цього вкрай дорогою ціною, заплатили мільйонами життів
та поколіннями жертви найкращих. Ми вижили, і повертаємо собі землю предків –
по праву тубільного населення. Повертаємо разом з мовою предків, разом з чесною
та неспотвореною колонізаторами історією. Ми не хочемо нікого гнобити – лише
прагнемо нарешті емансипації. Рівності не на словах, а в щоденному житті. І
так, через це ми стаємо незручними. Ми більше не бажаємо стояти у переповненому
автобусі, мовчки дивлячись на зарезервовані місця "тільки для білих". І комусь
це може видатись некомфортним, а ми – нечемними та невихованими. Терпіть та
звикайте – тубільці виходять з резервації та хочуть собі рівних прав та поваги.
Не більше – але і не менше.

Наша емансипація – не проти когось, український націоналізм завжди був
інклюзивним, відкритим для всіх. У якому б віці не згадав мову предків
зросійщений "гарадской маларос" - ласкаво просимо до свого коріння. Якого б
етнічного кореня людина не схотіла б розділити з нами радість спілкування
українською – ласкаво просимо до гурту, як колись на Січі, ми приймаємо всіх
охочих, і не ділимо українців ні на "сорти", ні на "раси". Достатньо просто
схотіти стати одним із нас – і виявити мінімальну повагу до тубільних мови та
історії. Але – виявити цю повагу, і не формально, а щиро. Ніхто не дорікне ні
акцентом, ні початковою бідністю словника – навпаки, допоможемо, підтримаємо,
вітатимемо. Бо це все – не про шовінізм чи "чистоту раси", а лише про
припинення кількасотлітньої дискримінації. Тим же, для кого досі все українське
є меншовартісним, "провінційним", нижчим за "високую русскую культуру", з
кожним днем ставатиме на нашій землі дискомфортніше. І в якийсь момент таки
доведеться обрати – сприйняти аборигенів за рівних, чи скласти валізи і
повертатись до метрополії. Ми з кожним днем ставатимемо все більш незручними
для самозваних "білих панів". Така вже вона штука – емансипація. Як там пишуть
на парканах, "вибачте за тимчасові незручності".
