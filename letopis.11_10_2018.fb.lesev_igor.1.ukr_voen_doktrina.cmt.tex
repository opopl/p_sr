% vim: keymap=russian-jcukenwin
%%beginhead 
 
%%file 11_10_2018.fb.lesev_igor.1.ukr_voen_doktrina.cmt
%%parent 11_10_2018.fb.lesev_igor.1.ukr_voen_doktrina
 
%%url 
 
%%author_id 
%%date 
 
%%tags 
%%title 
 
%%endhead 
\subsubsection{Коментарі}

\begin{itemize} % {
\iusr{Оксана Высоцкая}
Утащу себе. Очень уж доходчиво и образно.

\iusr{Gavrilova Irina}
ждали.. скучали

\begin{itemize} % {
\iusr{Игорь Лесев}
ничего, скоро новая ходка)) думаю, уже где-то в загашнике хранится мой коммент, "нарушающий нормы сообщества"))

\iusr{Gavrilova Irina}
\textbf{Игорь Лесев} давя только в этой статье насчитала десяток нарушений

\iusr{Валентина Гаташ}
Классный анализ.
\end{itemize} % }

% -------------------------------------
\ii{fbauth.janvarev_vasilij.kiev.ukraina}
% -------------------------------------

Про Карабах забыли - вполне себе "африканский" конфликт. Несмотря на то, что
"горячая" фаза закончилась в 1994 году, перестрелки (в том числе
артиллерийские) идут постоянно. Последняя - 22 сентября. Ну а сказать, что
азербайджанцы или армяне тупые у меня как то язык не поворачивается, да.

\begin{itemize} % {
\iusr{Игорь Лесев}

та не, там даже близко интенсивность конфликта не сравнится... сейчас
азербайджанцы активизировались по ряду причин - дружба/жвачка с русскими и
борзостью самих армян, которые чуток начали путать в какой угол им бежать... но
с тем, что происходит на Донбассе это никак несопоставимо

\iusr{Василий Январев}
\textbf{Игорь Лесев} 

Интенсивность, конечно же, меньшая чем на Донбассе. Но зато продолжительность
впечатляет. И самое печальное, что выхода из сложившегося положения для обеих
стран (так чтобы не потерять лицо) в обозримой перспективе не просматривается.

\iusr{Матвей Кублицкий}
\textbf{Василий Январев} именно поэтому РФ и предлагает замораживать конфликт. Чтобы ушло поколение воевавших людей с их ненавистью к врагу. Именно такой план внедряли в Осетии и Абхазии. И именно этот план похерил Саакашвили. С предсказуемым итогом.

\iusr{Василий Январев}
\textbf{Матвей Кублицкий} Так дело в том, что в Карабахе он заморожен давно, но желания у сторон воевать как то не поубавилось. Всплеск боевых действий в апреле 2016 года это очень ярко показал.

\iusr{Матвей Кублицкий}
\textbf{Василий Январев} вот и надо дальше ждать когда стороны смогут нормально переговоры вести. Кстати, президенты обеих стран на той неделе впервые пообщались вместе.

\iusr{Василий Январев}
\textbf{Матвей Кублицкий} ну может быть лет через 30 будет как на Кипре. У меня товарищ недавно там был, говорит линия разграничения весьма условна. Да, есть заграждения, КПП, какие то солдатики там стоят, но народ спокойно туда сюда шастает и никто никого уже не режет.

\iusr{Матвей Кублицкий}
\textbf{Василий Январев} вот тогда и можно начать переговоры. Но не раньше. В любом случае худой мир лучше доброй ссоры. Главное - чтоб народ не погибал.

\iusr{Сергей Белашко}
\textbf{Василий Январев} 44 года - два поколения.
\end{itemize} % }

\iusr{Сергей Сысоев}

Банят не только Вас, но и наших, пророссийских.

И, кстати, ругань в сторону фб абсолютно аналогичная Вашим словам. Просто под
копирку.

Разница только в том, что наши уверены, что ФБ играет (подыгрывает) на Вашей,
украинской стороне. У Вас, думается, мнение диаметрально противоположное.

А на самом деле, как мне кажется, оружие в этой борьбе сходно что с Вашей
стороны, что с нашей, потому и результаты похожи:

\begin{itemize}
  \item  пролезть в друзья
  \item  внимательно изучить контент (а может уже иметь кальку тех, кто ща ним следил)
  \item  перманентно или же по команде писать жалобы, дабы "френд" из бана не вылезал.
\end{itemize}

Вот и меряемся х..., ну в общем, меряемся!  @igg{fbicon.face.confused} 

Сами создаем себе перемоги "плаксивым доносом", радуемся им как дети, и, наоборот, грустим, когда "забанили нашего".

Ну а виноват то во всем фэйсбук конечно!!!

Ну и конечно (это уже мое мнение) он украинской стороне подыгрывает!...

 @igg{fbicon.smile} 
Как то так....

\begin{itemize} % {
\iusr{Игорь Лесев}

виноват именно ФБ, потому что допускает подобный алгоритм действий... при этом
он не универсальный. В соцсети куча ботов и просто идиотов, позволяющих себе
все что угодно, любые оскорбления и их не трогают. И есть "на особом счету",
которые уходят в бан на 30 дней за давно забытое и никого не оскорбляющее слово

\iusr{Сергей Сысоев}
В общем, наверное Вы правы.
Действительно есть каста неприкасаемых.

\iusr{Вячеслав Баталин}

это еще что. Меня забанили как боза фоты билетов (моих) с тутуру (сервис
продажи авиабилетов), за то что в моей ленте появился человек, которого уже
забанили, ну и за прошлогодние высказывания о стране южнее Белгорода, это уже
стало хорошим тоном

\end{itemize} % }

\iusr{Вадим Зицер}
Игорь, а где вас еще читать можно?
Не хочется на фб надеяться..

\begin{itemize} % {
\iusr{Сергей Киселев}
Даблируйте в ВК. Там нет цензуры
\end{itemize} % }

\iusr{Anna Vladimirovna}
о... наканец-та.

\iusr{Andrei Mazur}
С возвращением... Годный текст

\iusr{Сергей Сысоев}

А текст, кстати говоря, хороший!  Заставляет задуматься и очень многое здорово
подмечено, хоть я и "с другой стороны".  Про стрельбу на границе с курской
областью очень точно подмечено: ваши власти воры, но не идиоты!  При попытке
стрельбы на границе с курской областью мы (даже с учетом небольшого 46
миллиардного бюджета) создали бы в кратчайшие сроки э....  Ну в общем в Киеве
был бы запах гари!  И много, много коробочек с орденами....

\iusr{Матвей Кублицкий}

Коррупционный мотив войны на Донбассе - не последний, но и не первый. Первый -
это полная зависимость нынешней киевской власти от США. А вот тем как раз
выгодно в подбрюше России тлеющий костер держать. И им как то пох что дровами в
этом костре украинцы

\iusr{Елена Джабраилова}
Ну, и не забываем о катастрофическом снижении численности населения...

\iusr{Сергей Киселев}

Если быть циниками, то вялотекущая война выгодна прежде всего России.  Война
отягощает проблемами киевскую хунту и сдерживает вал репрессий против русских и
против Православной Церкви Украины.

\begin{itemize} % {
\iusr{Матвей Кублицкий}
и в чем конкретно интерес России?

\iusr{Сергей Киселев}
\textbf{Матвей Кублицкий} отягощение проблемами киевского режима. Сдерживание давления на русский язык и православие.

\iusr{Сергей Киселев}

Цитата из поста

"Война на Донбассе в этом плане очень африканская. Мы тратим на оборонку по
нашим меркам непростительно дикие деньги. Это деньги проедания. Деньги, которые
не становятся потом новыми деньгами. У нас стремительно сокращаются оборонные
заказы за рубежом. Мы не решаем геополитических и даже тактических задач этими
тратами. Мы сокращаем свои трудовые ресурсы, сокращая тем самым и ВВП. Наконец,
мы воюем на своей территории. И уничтожаем свою же инфраструктуру, на которую
же и претендуем"

\iusr{Матвей Кублицкий}

давление на русский язык и православие - чисто украинские проблемы. прямого
вреда России не несут. а вот обнищание украинского народа несет прямой
экономический вред России. именно поэтому Россия против войны

\iusr{Василий Январев}
\textbf{Сергей Киселев} Чушь. Наоборот давление усиливается ибо вялотекущий военный конфликт дает очень удобный образ внутреннего врага.

\iusr{Сергей Киселев}
\textbf{Василий Январев} усиливается. Но пять лет уже выиграли. А там посмотрим. Время и демография играют против хунты.

\iusr{Василий Январев}
\textbf{Сергей Киселев} Демография играет и против РФ, если Вы не в курсе. Да, и не стоит употреблять слов, значение которых Вам незнакомо.

\iusr{Сергей Киселев}
\textbf{Василий Январев} население России растет. С демографией у нас вопрос решаемый.

\iusr{Сергей Киселев}
\textbf{Матвей Кублицкий} обнищание украинцев крайне выгодно для России.

\iusr{Василий Январев}
\textbf{Сергей Киселев} 

\href{https://www.vedomosti.ru/politics/articles/2018/09/19/781380-ubil}{%
В России ускорилась убыль населения, vedomosti.ru, 19.09.2018%
}

\iusr{Матвей Кублицкий}
\textbf{Василий Январев} это без учета миграции. с миграцией рост постоянный

\iusr{Василий Январев}
\textbf{Матвей Кублицкий} "Миграционный прирост лишь на 46,1\% компенсировал потери населения"

\iusr{Сергей Киселев}
\textbf{Василий Январев} упростим выдачу украинцам вида на жительство, и миграционный прирост компенсирует убыль.
Население растет почти десять лет подряд. Небольшое уменьшение сейчас не проблема.

\iusr{Матвей Кублицкий}
\url{https://ru.wikipedia.org/wiki/Население_России}

\iusr{Василий Январев}
\textbf{Сергей Киселев} Во-первых, уже. Во-вторых, этот поток как раз и иссяк.

\iusr{Сергей Киселев}
\textbf{Василий Январев} поток подрегулируют

\iusr{Матвей Кублицкий}
\textbf{Василий Январев} просто зайдите на сайт статуправления и гляньте. рост населения россии идет с 2010 года.

\iusr{Василий Январев}
\textbf{Матвей Кублицкий} Но с 2018 года уже не идет - в заметке последние данные Росстата

\iusr{Сергей Киселев}
\textbf{Василий Январев} статистики еще нет по 18 году.

\iusr{Василий Январев}
\textbf{Сергей Киселев} Вот скажите честно, Вы действительно дурак или просто прикидываетесь? 

\url{http://www.gks.ru/free_doc/doc_2018/info/oper-08-2018.pdf}

\iusr{Сергей Киселев}
\textbf{Василий Январев} а вы что, будущем живете? Мы вообще-то в 2018 году еще живем.

\iusr{Матвей Кублицкий}
\textbf{Василий Январев} все правильно сказано - смотреть надо по итогам года. вот и глянем.... щас то что обсуждать 2018 год?

\iusr{Сергей Сысоев}

Спор из разряда у вас плохо, а у вас еще хуже.

Тем, кто "за Украину агитируют":

все верно, с демографией у нас в России не блеск. Действительно в России есть
проблемы.

Но любые проблемы надо примерять к конкретному процессу, сравнивать!

Так вот наши российские проблемы с демографией не идут ни в какое сравнение с
Вашими, украинскими.

У вас все тоже с соотношением смертности и рождаемости, только в разы больше!!!

Кроме этого, поток на выезд у Вас просто угрожающий.

С нашим потоком на выезд в относительных величинах не сравним. Так что это тоже
вам, украине, в минус.

Ну и наконец "объем воды в сосудах" - кол- во населения у нас и у вас - разное.
При конфликте на истощение наш запас прочности много (на пару порядков) выше,
чем украинский.

А потому, прав то как раз Сергей Киселев.

Мы можем особо ничего не делать, а просто ждать, пока тлеющий конфликт на
востоке Украины медленно, но верно будет поджирать людей, ресурсы, веру.

Даже в отсутствии войны у украины огромная проблема: кризис инфраструктуры.

На его решение нужны огромные денежные ресурсы.

Так вот нам, России, нужно просто не давать украине ресурса развития.

В условиях, когда и страны запада не горят желанием это делать, для украины это
в перспективе фатально.

Так что Киселев прав: надо отягощать проблемами киевский режим.

\iusr{Сергей Киселев}
По оценкам киевских прохунтовских экспертов Украина теряет ежегодно 500-600 тыс. населения. При общем населении около 35 млн.

\iusr{Василий Январев}
\textbf{Сергей Сысоев} 

1. Никаких "ресурсов развития" от РФ Украина уже не имеет. 

2. Конфликт малой интенсивности не критичен для существования государства. 

3. Да, будет, хотя в принципе уже есть не блещущая ничем аграрно-сырьевая
страна. Ну примерно как Румыния. В чем фатальность то?


\iusr{Сергей Киселев}
\textbf{Василий Январев} просто не хватит моб. ресурса держать такую большую территорию и такую протяженнную границу с сильным врагом.

\iusr{Василий Январев}
\textbf{Сергей Киселев} С каким врагом, уточните плиз?

\iusr{Сергей Киселев}
С Россией

\iusr{Сергей Сысоев}
\textbf{Василий Январев}, 

по моему, вы хитрите, а на самом деле все понимаете.
Ну если очень хочется подробностей, то пожалуйста.

1. Инфраструктура у вас дышит на ладан. Ведение военного конфликта, как
абсолютно верно было подмечено Киселевым, связывает и распыляет в прах финансы,
что необходимы бвли бы на те самые трубы, провода, электростанции и, прости
господи, дороги.

2. Про цену на газ я вам ничего рассказывать не буду: сами знаете.

Тут интересен вопрос транзита.  Украина его прокекала.  Потеря транзита это не
только потеря платы за транзит, это еще и существенные, огромные проблемы с
газоснабжением украинских городов.  Это ведь только на бумаге у вас реверс газа
из Европы.  На самом деле Вы как брали газ из магистральных транзитных труб,
так и берете! И только потом за счет байпаса делаете виртуальный "change"-
реверс.  Снижение транзита означает, что из 4 ваших труб заполненной останется
1.  А города, висящие на трех оставшихся трубах, откуда сосать будут.... газ?!
Или останется только "глагол"?  Уменьшение транзита влечет за собой крупные
неприятности с обычным газораспределением внутри Украины.  Нужно строить НОВЫЕ
газопроводы. Интерконнекторы.  А денег нет.  И не будет.

3. Единственный ресурс пока еще не сильно подорожавший, - это ээ.

Но и она подорожает с июля 2019 года до 3 раз (!!!) - см. Закон о рынке ээ.  Ну
и самое главное: АЭС - основа для дешевой ээ в Украине вырабатывают ресурс.  И
продлить его просто так - не даст МГТ.  А кто вам новые АЭС строить будет?

4. Деградация энергосистемы по газу, углю и ээ ведет к умиранию примышленности
как таковой.  Для аграрной сверхдержавы не нужно много людей.  Ну миллинов 5
человек. Ну, допустим, еще миллиона 2-3 будут работающим на земле "Щеневзмерлу"
петь, чтобы патриотизьм не спадал.  А остальные люди чем заниматься будут?
Падение производства влечет к снижению пенсионного обеспечения, здравоохранения
и образования.  Эти проблемы у вас уже есть в ПОЛНЫЙ РОСТ!

\iusr{Василий Январев}
\textbf{Сергей Киселев} То есть если я правильно понял, Вы сейчас утверждаете что Россия и Украина являются врагами?

\iusr{Сергей Киселев}
\textbf{Василий Январев} несомненно. Если украинцы не воспринимают Россию как врага, то это их проблемы.
Для России Украина - враг. Жестокий и непремиримый.

\iusr{Сергей Киселев}
Украина это Антироссия

\iusr{Сергей Сысоев}

Люди будут покидать украину просто чтобы выжить.
Для решения всех указанных вопросов нужны деньги.
Инвестиции.
Запад их делать не будет.
Ему это не надо.
Нам вы тоже не нужны.
Если вас используют как оружие против нас, то нам надо все сделать, чтобы ваша экономическая жизнь "безвременно прервалась".
А для этого нам достаточно просто создавать фон и условия к тому, чтобы у вас не происходило улучшение.
Тот самый маленький тлеющий конфликт вам в помощь.
Сам по себе он не критичен.
Критичны деньги.
А их нет и не будет, пока есть конфликт.
Пример с вашей гтс - как раз об этом.
Европейцам проще вас обойти, чем ввязываться.
И так будет совсем, пока конфликт тлеет.

\iusr{Василий Январев}
\textbf{Сергей Киселев} Все понял, вопросов больше не имею.

\iusr{Василий Январев}
\textbf{Сергей Сысоев} И Вам спасибо за содержательный ответ.

\iusr{Матвей Кублицкий}
\textbf{Сергей Киселев} для России Украина - просто соседнее государство с кучей проблем. Это украинская власть называет Россию врагом, но никак не наоборот

\iusr{Сергей Киселев}
\textbf{Матвей Кублицкий} вы ошибаетесь. Просто российская власть умнее. Не кричит Украина - враг!
А действует. Неспешно, но неумолимо.
Строятся потоки, подтягивается армия к границе. И т.д. и т.п.

\iusr{Матвей Кублицкий}
\textbf{Сергей Киселев} может тогда напишите какие у России причины считат Украину врагом?

\iusr{Сергей Сысоев}
\obeycr
Вы Матвей, не правы.
Я воспринимаю Украину и ее народ как единое целое.
Украина сейчас всеми силами гадит нам.
Но такие как Вы мне рассказыааете, что народ то нам дружественен.
А знаете: я как житель России, такую дружественность в гробу вилал!
Народ несет ответственность за свое правительство.
Если украина гадит, то это должно получить адекватный ответ с нашей стороны.
Если народ украины найдет в себе силы сменить власть, тогда посмотрим, а пока...
Если что то ведет себя как противник, выглядит, как противник, и действует как противник, то это и есть ПРОТИВНИК!
И не важно что там на борту написано,....
\restorecr

\iusr{Василий Январев}
\textbf{Матвей Кублицкий} Ну почему же просто соседнее? По крайней мере один житель Орла и один житель Москвы только что высказали точку зрения что мы являемся врагами. Наверное они не одиноки, последняя социология от Левада центра говорит, что таких в России 55\%.

\iusr{Сергей Киселев}
\textbf{Матвей Кублицкий} причина одна - Украина инструмент для разрушения России, украинцы - оккупанты на русской земле.

\iusr{Василий Январев}
\textbf{Сергей Киселев} Какой перл!

\iusr{Василий Январев}
\textbf{Сергей Сысоев} Чем гадит то? Или Украина должна сказать спасибо за Крым или за Гиркина в Славянске и казаков и чеченов, которые помогали разжигать огонь войны в Донецке? Чем Украина вам так нагадила то до февраля 2014 года?

\iusr{Матвей Кублицкий}
Угомонитесь со своей ненавистью. всё проходит. пройдет и эта власть на Украине. от народа нынче мало что зависит. Петя на днях пошуровал в составе ЦИК - так что считать будут не зависимо от голосования.

\iusr{Сергей Киселев}
\textbf{Матвей Кублицкий} нет никакой ненависти, я люблю Украину, и её народ. Много раз бывал, неплохо понимаю мову.

\iusr{Сергей Киселев}
Но это государство должно быть уничтожено. Так или иначе.

\iusr{Матвей Кублицкий}
\textbf{Сергей Киселев} ну и что тогда рассуждать про беды украинцев. нынче наша задача молча наблюдать за происходящим. Украина не будет уничтожена. Украина - это прежде всего географически просто земля - куда ж она денется? возможен лишь отскок еще нескольких регионов.

\iusr{Сергей Сысоев}

\obeycr
Миноточку, Василий!
Пересмотрите кадры майдана 2013-14 годов.
"Москаляку на гиляку " - помните?
И это еще ДО крыма!!!
Переводить надо или сами знаете?!
Или может они провидцы были и потому кричали?
 @igg{fbicon.smile} 
Весь переворот был осуществлен ИЗНАЧАЛЬНО АНТИРОССИЙСКИМИ силами.
И раз они победили, то совершенно очевидно, что свою антироссийскую политику будут продавливать.
Нынешние принятые законы это подтверждают.
\restorecr

\iusr{Василий Январев}
\textbf{Матвей Кублицкий} 

Уже не будет - никто не хочет оказаться в новом ОРДЛО. С окончательно вставшей
промышленностью, пенсиями по 3 тысячи рублей и главным работодателем в виде
"народной милиции" где люди за 15 косарей согласны лезть в окопы..

\iusr{Сергей Киселев}
\textbf{Матвей Кублицкий} я имею ввиду уничтожение УГ в нынешнем виде. Украина и ее население конечно никуда не денется. Да и государственность останется. Но должна произойти перезагрузка. С передачей полномочий регионам

\iusr{Сергей Сысоев}

Где ж это вы вставшую промышленность то увидели в ДНР И ЛНР?
Уголь, что вы покупаете, оттуда!
 @igg{fbicon.smile} 

\iusr{Сергей Сысоев}
Кстати про пенсии в 3 тысячи рублей.
Это у нас в гривнах то 1500 гривен примерно.
А какая самая массовая пенсия в стране победившего майдана?!
 @igg{fbicon.smile} 

\iusr{Матвей Кублицкий}
\textbf{Василий Январев} про промышленность ЛДНР - это в основном сказка укросми. В реальности эта промышленность сохранила связи с Россией и щас отлично встраивается в эту действительность. И - главное - ЛДНР не отягощенны огромными кредитами

\iusr{Василий Январев}
\textbf{Сергей Сысоев} 1. Москаляку на гилляку - это меня, а не Вас. Я в Киеве живу. 2. Есть большая разница между намерениями (даже если они очевидны) и действиями. Это Вам любой участковый объяснит. Сам Майдан означал просто новый оранжевый период, который бы бесславно закончился на следующих выборах. 3. Промышленность ОРДЛО за редким исключением стоит - это есть факт. И основной приток миграции в РФ именно оттуда - людям сложнее заработать.

\iusr{Сергей Сысоев}

По данным укрстата украина увеличила покупки антрацита у россии на 60\%.
Это лишние 500 млн\$.
Угадайте, в какую "неработающую" промышленность они пошли?
 @igg{fbicon.smile} 

\iusr{Василий Январев}
\textbf{Матвей Кублицкий} Матвей, Лугансктепловоз который входит в состав российского Трансмашхолдинга стоит мертво в 2015 года.

\iusr{Матвей Кублицкий}
\textbf{Василий Январев} вам перечислить украинские предприятия, которые тож встали? обвал промышленности произошел и на Украине и в ЛДНР. Но я написал что ситуация для промышленности ЛДНР более выигрышная

\iusr{Василий Январев}
\textbf{Матвей Кублицкий} Совсем наоборот. В ЛДНР полный трындец. Никакой рос.рынок для них не открылся. Как то полуподпольно торгуют углем и стальным полуфабрикатом. И все.

\iusr{Сергей Сысоев}
Василий!
Я вас там еще про пенсии в Украине спросил.....
Про размер самой массовой,..

\iusr{Сергей Киселев}
\textbf{Сергей Сысоев} наладить жизнь и запустить промышленность в ЛДНР для России не такая уж сложная задача. Она будет решена в ближайшее время.

\iusr{Сергей Сысоев}
Т.е. вы уголь днр теперь подпольно покупаете?
 @igg{fbicon.smile} 

\iusr{Василий Январев}
\textbf{Сергей Киселев} Товарищ, перестаньте молоть ересь. Почему же за три года никто ничего не запустил? Решена в ближайшее время. Клоун.

\iusr{Матвей Кублицкий}
\textbf{Василий Январев} я почему то уверен что просить у вас ссылку на неукраинские сми по поводу развала промышленности ЛДНР бессмыслено

\iusr{Сергей Киселев}
Почему украинцы всегда переходят на личные оскорления?

\iusr{Василий Январев}
\textbf{Сергей Сысоев} Средняя пенсия в переводе на ваши где то 6800. Уголь по документам российский, если там есть с неподконтрольных территорий, то наверное немного. Есть проблема с документами.

\iusr{Василий Январев}
\textbf{Сергей Киселев} Может быть потому что Вы идиот?

\iusr{Матвей Кублицкий}
\textbf{Василий Январев} на 1 января 2018 года средняя пенсия на Украине 1828 гривен - что явно намного меньше 6800 наших

\iusr{Сергей Сысоев}
Вы врете Василий.
Средняя пенсия НЕ ОТРАЖАЕТ реальную пенсию.
Из 12 млн пенсионеров почти 8 млн (!!!) получает минимальную пенсию в 1500 с небольшим гривен.
Это данные вашей статистики!
Т.е. почти те же самые 3 тыс рублей, что вас так ужасают в лднр.
Только еще и тарифы лошадиные!

\iusr{Сергей Сысоев}
Средняя, она и депутатскую равняет.
На самом деле 70\% получает 1542 гривны кажись.
Врет Василий и не краснеет!

\iusr{Матвей Кублицкий}
\textbf{Василий Январев} И российский уголь у вас сплошь донецкий. документы - вообще не проблема

\iusr{Василий Январев}
\textbf{Матвей Кублицкий} На самом деле проблема. Газпром почему то не берет газ у Черноморнефтегаза.

\iusr{Василий Январев}
\textbf{Сергей Сысоев} Так в ЛДНР 3 тысячи - это СРЕДНЯЯ

\iusr{Матвей Кублицкий}
\textbf{Василий Январев} это совсем другая ситуация. не передергивайте

\iusr{Матвей Кублицкий}
\textbf{Василий Январев} по пенсиям ЛДНР - не забывайте размер коммуналки

\iusr{Василий Январев}
\textbf{Матвей Кублицкий} А думаете другие поставщики хотят так подставляться?

\iusr{Сергей Сысоев}
\obeycr
Не везут его с кузбаса- дорого.
Только с ДНР
Василий то это знает конечно, но стесняется признать!
Но его кстати еще и в Польшу реализуют.
Министр внешних сношений Климкин задал польским партнерам вопрос: как же так?!
Ответ был ферический :
Поставка в Польшу угля из ДНР не несет вреда польской политике и экономике!
Украина свободна!
\restorecr

\iusr{Матвей Кублицкий}
\textbf{Василий Январев} какие поставщики и как подставятся поставляя уголь с ЛДНР на Украину?

\iusr{Василий Январев}
\textbf{Матвей Кублицкий} На санкции можно внезапно налететь.

\iusr{Василий Январев}
\textbf{Матвей Кублицкий} Собственно именно поэтому Газпром и не берет газ у Черноморнефтегаза.

\iusr{Матвей Кублицкий}
\textbf{Василий Январев} налететь на санкции в России? вы уверены что у нас эти санкции кого-то интересуют?

\iusr{Сергей Сысоев}

Поставки осуществляются через госрезерв.
А ему санкции, как рыбке зонтик!
 @igg{fbicon.smile} 

\iusr{Сергей Сысоев}
Кстати, Василий!
А чего ж Польша то все никак не налетает!?
И турция?!
Шо такое?!
 @igg{fbicon.smile} 

\iusr{Матвей Кублицкий}
\textbf{Василий Январев} а мне вот интересно стало: какой газ должен забирать Газпром у Черноморскнефтегаза с условием отсутствия газопровода???

\iusr{Василий Январев}
\textbf{Матвей Кублицкий} Нет, почему в России? Все угледобывающие компании активно работают на мировом рынке в т.ч. привлекают там кредитные средства. Кому нужно пятно на репутацию?

\iusr{Матвей Кублицкий}
\textbf{Василий Январев} в чем это пятно будет выражено? уголь официально закуплен на территории Украины и поставляется на территорию Украины? Или Украина уже официально отказалась от ЛДНР?

\iusr{Василий Январев}
\textbf{Матвей Кублицкий} Блокада с начала прошлого года.

\iusr{Матвей Кублицкий}
\textbf{Василий Январев} с украинской стороны блокада. с российской нет блокады

\iusr{Сергей Сысоев}
Правильно.
И именно с этого момента вырос импорт угля из России на 60 %
Вы думаете он не из ДНР?
 @igg{fbicon.smile} 

\iusr{Василий Январев}
\textbf{Матвей Кублицкий} Ну так а потом надо этот уголь кому то перепродать, чтобы обратно ввезти на территорию Украины. Проще через каких то югоосетин продать в Турцию

\iusr{Матвей Кублицкий}
\textbf{Василий Январев} проще продать его как российский Украине. чисто географически

\iusr{Василий Январев}
\textbf{Матвей Кублицкий} Думаю, что часть наверное как то так и прогоняют через какие то прокладки. Но далеко не весь.

\iusr{Сергей Сысоев}
Блажен, кто верует!
 @igg{fbicon.smile} 

\iusr{Матвей Кублицкий}
\textbf{Василий Январев} да какая разница как это оформлено? какая то российская фирма имеет шахты в России и заодно перекупает уголь с донбасса. Ничего незаконного

\iusr{Василий Январев}
\textbf{Матвей Кублицкий} Если бы все было так просто. Потому что у предприятий на ОРДЛО есть владелец, который эти шахты потерял и покупка угля сейчас напрямую будет выглядеть банально как скупка краденого.

\iusr{Матвей Кублицкий}
\textbf{Василий Январев} покупателю в России плевать как это выглядит с точки зрения Украины. Чисто бизнес. ничего личного. Еще раз - никаких нарушений ни российских законов, ни международных

\iusr{Василий Январев}
\textbf{Матвей Кублицкий} Так потребителю в России этот уголь не нужен - своего хватает. Его надо перепродать за границу. А как вы укажете в документах? Что это уголь куплен у предприятия, принадлежащего компании которая официально заявила что у нее отжали эти шахту? По закону это скупка краденого.

\iusr{Матвей Кублицкий}
\textbf{Василий Январев} а зачем указывать у кого куплен уголь? в международной торговли нет такого требования. продавец продает российский уголь - украинский, польский, турецкий покупатель покупает

\iusr{Василий Январев}
\textbf{Матвей Кублицкий} Тогда придется фальсифицировать свою статистику указывая завышенные показатели добычи на своих предприятиях. Я уже не говорю о необходимости внезапно отвечать на вопрос, а почему у вас в этой партии химсостав угля так сильно отличается.

\iusr{Сергей Сысоев}

\obeycr
Наивный человек...
Да никто этого и не указывает!
Ставится ооо рога и копыта и дается сертификат.
В сертификате указано, что это уголь ростовского угольного бассейна.
Дело в том, что с Донбасом это один и тот же бассейн, а потому уголь мало отличен по хим свойствам.
И все.
Но даже если уголь и идентифицируют, как это было в Польше, то украина опять свободна.
Польша официально сообщила, что ее безопасности этот уголь не несет вреда, а потому им "по барабану".
Климкин рвал и метал, но кого его мнение волнует.,,..
\restorecr

\iusr{Матвей Кублицкий}
\textbf{Василий Январев} зачем фальсифицировать то? так и укажут в своей отчетности в российские налоговые органы - купили уголь на Украине и продали на Украину

\iusr{Василий Январев}
\textbf{Сергей Сысоев} Как же у Вас плохо с русским языком..... @igg{fbicon.frown} 

\iusr{Сергей Сысоев}
Это не с языком, а с освещенностью и тряской: с работы еду.

\iusr{Василий Январев}
\textbf{Матвей Кублицкий} Вы только что предложили этим достойным людям вот так взять и во всем признаться. Поэтому и толкают через всякие югоосетинские конторы.

\iusr{Матвей Кублицкий}
\textbf{Василий Январев} в чем признаться? они купили официально уголь у украинского предприятия. купили законно с точки зрения российского законодательства

\iusr{Василий Январев}
\textbf{Матвей Кублицкий} Матвей, смотрите. Они купили у кого? Допустим шахта №1 бис - это предприятие, добывшее уголь. Кому оно принадлежит? Компании Иван-Царевич. Компания Иван-Царевич год назад официально заявила, что у нее отжали эту шахту какие то люди с автоматами. То есть уголь куплен у людей захвативших шахту у законного владельца. Теперь понятно?

\iusr{Матвей Кублицкий}
\textbf{Василий Январев} мне это давно понятно - это вам не понятно. российскому покупателю плевать что и как происходит на Украине. он обязан соблюдать исключительно российское законодательство. То что у вас кто-то у кого-то украл - его не волнует.

\iusr{Василий Январев}
\textbf{Матвей Кублицкий} А получить от Иван-Царевича иск он не боится, не? Причем по заранее проигранному делу, если официально укажет в документах шахту №1 бис?

\iusr{Матвей Кублицкий}
\textbf{Василий Январев} а иск куда будет подан то?

\iusr{Василий Январев}
\textbf{Матвей Кублицкий} Думаю, что во все возможные суды.

\iusr{Матвей Кублицкий}
\textbf{Василий Январев} вы мне хоть один то назовите

\iusr{Василий Январев}
\textbf{Матвей Кублицкий} начнут по месту регистрации покупателя, дальше видно будет

\iusr{Матвей Кублицкий}
\textbf{Василий Январев} то есть в российский арбитраж? а что укажут в качестве доказательства? например, откуда возьмут оригиналы документов на поставку? наш арбитраж копии не принимает

\iusr{Василий Январев}
Если бы можно было так просто этим заниматься, то никто бы не крутил схемы через Южную Осетию. И это уголь - в принципе ерунда, сырье. А намертво стоящий ХТЗ. Полетела же вся украинская сертификация. Кому теперь нужны эти трубы без сертификата? И чей то никто в РФ не говорит, а давайте в Харцызске трубы купим, дешевле же будет. Ну да, дешевле. Только ворованные и без сертификата.

\iusr{Матвей Кублицкий}
\textbf{Василий Январев} я же писал уже про проваленные предприятия донбасса. они провалены также как и украинские предприятия. только возможности у них шире

\iusr{Василий Январев}
\textbf{Матвей Кублицкий} Ну как же шире то? Газопроводы в России строят? Строят. Трубы для газопроводов на ХТЗ делать можно? Еще как. Почему же их все эти годы никто не берет???

\iusr{Матвей Кублицкий}
\textbf{Василий Январев} потому что пока своих труб хватает. потом тихонко купят тот же ХТЗ подешевле и их трубы внезапно также окажутся российскими. пока не время. никто в воюющий донбасс деньги не вложит.

\iusr{Василий Январев}
\textbf{Матвей Кублицкий} Боюсь, что туда вообще никто ничего не вложит. Как в Приднестровье. Много туда вкладывают? Да ничего.

\iusr{Матвей Кублицкий}
\textbf{Василий Январев} если бы у приднестровья была граница с россией - там бы давно уже все винодельные предприятия россияне выкупили. как в той же абхазии и осетии

\end{itemize} % }

\iusr{Сергей Киселев}
Утащил к себе

\iusr{Марина Прохорова}
Ну вы ещё спросите, в чём смысл существования нынешней укродэржавы. Кроме антироссийских санкций ))

\iusr{Олег Резник}

Сильно! Чувствуется мощный прорыв накопившихся дум. Бан, получается способствует
творчеству. С содержанием согласен 100\%, спасибо за публикацию.!

\iusr{Сергей Бухтияров}

Соскучился! Спасибо! А предположение на  @igg{fbicon.100.percent}  верное, так как и попыток прекратить
этот конфликт нет и не предвидится. Да ещё подарка Ющенко в Минск сватают! Так
что ужас без конца.

\iusr{Сергей Бухтияров}
Простите: читать - придурка!

\iusr{Станислав Бочкур}
За этот пост тоже, кстати, могут забанить.

\end{itemize} % }
