% vim: keymap=russian-jcukenwin
%%beginhead 
 
%%file 10_03_2017.stz.news.ua.mrpl_city.1.istoria_direktor_mmk_im_iljicha_nikolaj_gavrilenko
%%parent 10_03_2017
 
%%url https://mrpl.city/blogs/view/istoriya-nikolaj-gavrilenko
 
%%author_id burov_sergij.mariupol,news.ua.mrpl_city
%%date 
 
%%tags 
%%title История: директор ММК им. Ильича Николай Гавриленко
 
%%endhead 
 
\subsection{История: директор ММК им. Ильича Николай Гавриленко}
\label{sec:10_03_2017.stz.news.ua.mrpl_city.1.istoria_direktor_mmk_im_iljicha_nikolaj_gavrilenko}
 
\Purl{https://mrpl.city/blogs/view/istoriya-nikolaj-gavrilenko}
\ifcmt
 author_begin
   author_id burov_sergij.mariupol,news.ua.mrpl_city
 author_end
\fi

\ii{10_03_2017.stz.news.ua.mrpl_city.1.istoria_direktor_mmk_im_iljicha_nikolaj_gavrilenko.pic.1}

Чем больше отдаляешься от детства, тем чаще всплывают в сознании имена, фамилии
и события далекого прошлого.  Так недавно вспомнилась фамилия  Гавриленко. О
нем впервые случилось  услышать от папы, который работал на заводе имени
Ильича. Детская память не удержала содержание его слов, но остался образ
большого, основательного человека, которого все слушают, хотя он и не повышает
голоса. Позже, конечно, стало ясно, что речь шла о Николае Георгиевиче
Гавриленко. А теперь захотелось узнать о нем больше. В изданиях, касающихся
истории комбината имени Ильича, и в подшивках газет, и в интернете не так уж
много сведений о нем, к тому же порой они противоречат друг другу. Но с помощью
добрых людей удалось познакомиться с Людмилой Николаевной Буйневич, дочерью
Николая Георгиевича, которая рассказала много интересного об отце, а некоторые
факты биографии еще и уточнила.

Итак, Николай Георгиевич Гавриленко родился 12 мая 1910 года в Мариуполе, в
поселке Волонтеровка.  Его отец был  рабочим завода. Николай был старшим сыном,
кроме него были еще брат и сестра. С мальчишеских лет помогал родителям
обрабатывать земельный участок. Кстати, любовь к земле, садоводству у него
осталась на всю жизнь. Лозы, привитые им на старые кусты, давали на даче в
Донецке урожай раньше, чем в Мариуполе. В семнадцать лет поступил на завод
имени Ильича. Работал чернорабочим. Вероятно,  учился на рабфаке, так как в
1931 году был принят на первый курс Харьковского инженерно-строительного
института. Но, проучившись год, перевелся в Ленинградский политехнический
институт на прокатное отделение. Шел 1932 год. Украина, и Харьков в том числе,
-  в тисках голода. Может, это и было причиной перевода в город на Неве? А
может, желание учиться металлургии? В Ленинградском политехе он подружился на
долгие годы с однокурсником Фролом Козловым, в будущем партийным и
государственным деятелем, секретарем ЦК КПСС.

В 1936 году молодой инженер-прокатчик возвратился на родной завод. Его
направили в сортопрокатный цех. Как он работал? Ответ - в книге Д.Н.
Грушевского \enquote{Имени Ильича}: \enquote{Значительно перевыполняли план в сортопрокатном
цехе смены молодых инженеров - комсомольцев Александра Гармашева и Николая
Гавриленко}. Перед самой войной Николай Георгиевич был переведен в Горьковскую
область на Кулебакский металлургический завод начальником сортопрокатного цеха.
Предприятие как раз начало переходить на выпуск брони и сборку корпусов легких
танков. Знания инженера Гавриленко очень пригодились. Но работать там пришлось
недолго. Началась Великая Отечественная война. Николая Георгиевича направили на
Магнитогорский металлургический комбинат, где он возглавил один из прокатных
цехов. На комбинате, с самого начала ориентированного  на выпуск \enquote{гражданских}
сортов металла, предстояло освоить  выплавку броневых марок стали, а также
прокатку броневого листа. Почему понадобился именно ильичевский прокатчик? Да
потому что производство броневого проката на заводе имени Ильича было освоено
еще с дореволюционных лет и с тех пор постоянно и целенаправленно
совершенствовалось. К этим работам был причастен и Гавриленко. Персонал
комбината был переведен на казарменное положение, в цехах работали и жили. Лишь
изредка разрешалось на очень короткое время повидаться с семьей. Людмила
Николаевна рассказывала, как она приняла своего отца за \enquote{чужого дядю} и
расплакалась, когда он пришел домой, чтобы увидеть жену Галину Ивановну и дочь
да заодно и помыться. Людмиле было тогда чуть больше года.

Николай Георгиевич с семьей вернулся в Мариуполь только в 1945 году. В
биографических справках о нем значится: \enquote{В 1945—1949 годах — начальник
броневого отдела, заместитель главного инженера Мариупольского
металлургического завода}. Совмещал ли он две должности или занимал
последовательно,  -  неизвестно. В любом случае обе были очень ответственны. За
выдающиеся заслуги перед страной Указом Президиума Верховного Совета СССР от 7
июня 1947 года коллектив Мариупольского завода имени Ильича был награжден
орденом Ленина. Более пятисот рабочих и инженерно-технических работников
предприятия  удостоились правительственных наград. Среди других орденом Ленина
был награжден Н.Г. Гавриленко. Забегая вперед, нужно сказать, что за
самоотверженный труд на Магнитогорском металлургическом комбинате Николай
Георгиевич был награжден орденом Красной Звезды, а в последующие годы  двумя
орденами Трудового Красного Знамени, орденом \enquote{Знак Почета}, многочисленными
медалями. 31 декабря 1949 года он был назначен директором завода, сменив на
этом посту А.Ф. Гармашева. Ему  предстояло  развивать на заводе наряду с
металлургическим производством и машиностроительные цеха. Интересная деталь. 12
марта 1950 года Н.Г. Гавриленко был избран депутатом Верховного Совета СССР как
директор Мариупольского... машиностроительного завода. 

В юбилейном издании  \enquote{ОАО \enquote{Мариупольский металлургический комбинат имени Ильича. 110 лет}} 
перечислены объекты, введенные в эксплуатацию за время работы
Николая Георгиевича Гавриленко директором завода. Это трубоэлектросварочный
стан \enquote{650}, стан спирально-шовных труб, толстолистовой стан \enquote{1250}, доменная
печь № 2, водогрязелечебница, освоена технология массового производства
толстых листов  для судостроения. Но самое главное, Николай Георгиевич
обратился в вышестоящие инстанции с предложением реконструировать действующий
завод и расширить его за счет строительства объектов второй очереди. В книге
Ю.Я. Некрасовского \enquote{Огненное столетие. 1897-1987} приведена выдержка из
воспоминаний Н.Г. Гавриленко. Вот она с небольшими сокращениями: \enquote{В 1954-56 гг.
возникла необходимость увеличить производство чугуна, стали, проката и труб.
Стал вопрос о возможности быстрого расширения их производства на действующих
заводах, располагающих необходимыми территориями, энергетическими ресурсами или
возможностью их расширения. А также кадрами. Заводом, располагающим этими
возможностями по проработке \enquote{Укргипромеза}, оказался Мариупольский завод имени
Ильича. Проработка проектного института показала возможность его расширения.
Учитывая это, а также наличие мощных строительных организаций в Мариуполе,
правительством было принято решение немедленно приступить к проектированию и
расширению завода. Немаловажную роль в принятии решения о реконструкции завода
сыграло отношение к нему заместителя председателя Совмина СССР Ф.Р. Козлова}.
Николай Георгиевич скромно умолчал, что развитие завода было лично его идеей.
Не каждому творцу доводится увидеть осуществление своей мечты, Гавриленко свою
- увидел. Он увидел на родном заводе аглофабрику, строй из пяти доменных печей,
ново-мартеновский и кислородно-конвертерный цеха, слябинг, стан \enquote{1700}, цех
холодного проката. Но это было потом.

В 1957 году Н.Г. Гавриленко был назначен первым заместителем Председателя
Донецкого Совнархоза Украинской ССР. Работая на этом посту, он внес
значительный вклад в развитие тяжелой индустрии Донбасса, и в частности завода
имени Ильича. После упразднения совнархозов, в 1965 году  был назначен первым
заместителем министра черной металлургии Украинской ССР. Им был создан в
Днепропетровске научно-исследователь\hyp{}ский институт, который он возглавил,
завершив работу в министерстве. В этом учреждении он работал до выхода на
заслуженный отдых в 1990 году.

Николай Георгиевич Гавриленко ушел из жизни 12 февраля 1999 года. Делегация
ильичевских металлургов приняла участие в похоронах, отдав последний долг
бывшему директору завода.
