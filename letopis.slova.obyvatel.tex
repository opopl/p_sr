% vim: keymap=russian-jcukenwin
%%beginhead 
 
%%file slova.obyvatel
%%parent slova
 
%%url 
 
%%author 
%%author_id 
%%author_url 
 
%%tags 
%%title 
 
%%endhead 
\chapter{Обыватель}
\label{sec:slova.obyvatel}

%%%cit
%%%cit_head
%%%cit_pic
%%%cit_text
Общественный дискурс после унизительного поражения в Армении, естественно,
задавали \enquote{патриоты}. Весь эфир и все СМИ были заняты обсуждением причин
поражения, поиска виновных и жажды немедленного реванша. Это создавало иллюзию
, что ВСЕ общество волнуют именно эти вопросы. Но, правда, заключается в том,
что \emph{обывателя}, который составляет большинство избирателей, эти вопросы, по
настоящему мало волнуют. Воевать среднестатистический обыватель не хочет,
территориальные приобретения или потери его не интересуют, державнического
инстинкта у него нет. \emph{Обывателя} в первую очередь и главным образом волнуют
только вопросы мира, благосостояния, работа, пенсия, здоровье, дети, цены,
социальная помощь, общественный порядок и т. д. Но сразу после поражения эти
темы не могли быть предметом дискурса - тон задавали \enquote{патриоты}. Патриотов в
любой стране, как правило, не более 5\%, но они крайне активны и поэтому
всегда создаётся впечатление что они отражают мнение большинства общества. Но
это совсем не так. Точнее, это совсем не так
%%%cit_comment
%%%cit_title
\citTitle{В Украине совершенно свободна та нише, которую в Армении успешно использует Пашинян / Лента соцсетей / Страна}, 
Андрей Головачев, strana.ua, 05.07.2021
%%%endcit
