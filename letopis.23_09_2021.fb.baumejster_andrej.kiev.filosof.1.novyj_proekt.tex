% vim: keymap=russian-jcukenwin
%%beginhead 
 
%%file 23_09_2021.fb.baumejster_andrej.kiev.filosof.1.novyj_proekt
%%parent 23_09_2021
 
%%url https://www.facebook.com/andriibaumeister/posts/4274593185995541
 
%%author_id baumejster_andrej.kiev.filosof
%%date 
 
%%tags cennosti,filosofia,future,myshlenie,obschestvo,obsuzhdenie,ukraina
%%title Настало время начать новый проект, о котором я давно думал и который давно планировал
 
%%endhead 
 
\subsection{Настало время начать новый проект, о котором я давно думал и который давно планировал}
\label{sec:23_09_2021.fb.baumejster_andrej.kiev.filosof.1.novyj_proekt}
 
\Purl{https://www.facebook.com/andriibaumeister/posts/4274593185995541}
\ifcmt
 author_begin
   author_id baumejster_andrej.kiev.filosof
 author_end
\fi

Настало время начать новый проект, о котором я давно думал и который давно
планировал. Наконец, в этот субботний вечер (25 сентября), могу с радостью
пригласить всех желающих! 

\ifcmt
  pic https://external-lga3-1.xx.fbcdn.net/safe_image.php?d=AQEwnefoFekgJ8ue&w=500&h=261&url=https%3A%2F%2Ffs-thb02.getcourse.ru%2Ffileservice%2Ffile%2Fthumbnail%2Fh%2Fd6bc98a98c5817c7652f8e0e2b2e9190.jpg%2Fs%2F600x%2Fa%2F65895%2Fsc%2F167&cfs=1&ext=jpg&_nc_oe=6ebc7&_nc_sid=06c271&ccb=3-5&_nc_hash=AQHiAPZFBs05XqY8
  @width 0.7
\fi

\url{https://otium.academy/baumeister_panorama/payment}

Какие вопросы меня волнуют в этом проекте? 

- Что происходит с миром? Какие события сегодня определяют картину ближайшего
будущего? Мы свидетели стремительной ценностной революции? Какие идеи и
ценности идут на смену традиционным европейским или "западным" ценностям? Как
меняется мировой порядок? Как происходит перераспределение центров влияния и
силы? Что ждет национальные государства в ближайшем будущем? Какие концепты и
нарративы будут определять мировую повестку дня в ближайшие десятилетия? Как
ведущие мировые СМИ интерпретируют ключевые события и о чем они умалчивают? 

- Мне, как философу и свидетелю времени важно находить ответы на эти вопросы. Я
анализирую ведущие мировые СМИ (газеты и журналы), слежу на новейшей
политической литературой. И мне, как и многим из вас, не хватает независимой и
объективной аналитики. И совсем недавно, в беседе с Евгением Голубом, у меня
родилась идея. А почему не проводить в конце каждого месяца специальный
аналитический обзор главных событий и важных фактов мировой жизни? Как по
версии ведущих мировых СМИ, так и по нашей с Евгением, собственной версии? Если
этим не занимается нынешнее TV, почему самим не занятая этим важным и
чрезвычайно интересным делом? 

И мы решили начать. Премьера состоится уже в эту субботу, 25 сентября, в 18.00
по Киеву. 

Что это будет? Какой формат? 

- Это закрытая трансляция в Zoom. Почему такое решение? Чтобы участники были
заинтересованы именно в серьезном и содержательном разговоре. Без характерной
для последнего времени блогерской поверхностности. Мы будем после каждого
выпуска сбрасывать группе участников также ссылки и материалы, для дальнейшего
осмысления главных событий месяца. Это будут ведущие западные СМИ (журналы,
газеты, книги), высказывания политиков и интеллектуалов, значимые события и
факты. Как вы прекрасно понимаете, такие выпуски требуют специальной и
серьезной подготовки. Каждый день, на протяжении каждого месяца, мы (я и
Евгений), будем отслеживать важнейшие публикации и готовить аналитику для
очередного выпуска. Будем благодарны за помощь в отслеживании таких материалов.
Это сделает нашу аналитику более информативной и разноплановой. 

- Каждая встреча будет состоять из двух частей:

\begin{itemize} % {
\item - Вначале аналитика. Я анализирую важнейшие факты и события месяца.
Помогает мне в этом Евгений Голуб. Наш анализ будет опираться на публикации
и сообщения ведущих мировых СМИ. Это будет программа первого часа.  

\item - Вторая часть - ответы на вопросы и дискуссия. Вы можете задавать свои
вопросы, а также высказывать свои критические мнения по тем или иным
событиям. Это программа второго часа (час-полтора). Программа в целом будет
длиться 2,5 часа. 
\end{itemize} % }

- Почему я решился начать этот проект (несмотря на свой и так уже максимально
плотны график)? Потому что сегодня ощущается большой дефицит такой аналитики.
Сегодня очень важно манипуляциям и пропаганде противопоставлять попытки
прояснения и понимания того, что происходит с нами и с миром. Мне, как
философу, следует этим заниматься по долгу призвания и профессии. И если я не
нахожу тех образцов, в которых остро нуждаюсь, значит, нужно делать это самому.

Нет, надеюсь не самому, а с помощью единомышленников. Поэтому искренне
приглашаю к участию в общем деле! Надеюсь, это будет дискуссионный клуб
единомышленников, в котором важна именно разность позиций и разность взглядов
(но, конечно, аргументированных, опирающихся на четкие основания). Из этой
разности, из этого дружеского "пространства несогласия" только и может
рождаться истина.
