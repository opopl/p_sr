% vim: keymap=russian-jcukenwin
%%beginhead 
 
%%file slova.zhizn
%%parent slova
 
%%url 
 
%%author 
%%author_id 
%%author_url 
 
%%tags 
%%title 
 
%%endhead 
\chapter{Жизнь}
\label{sec:slova.zhizn}

Самый потрясающий наш разведчик — это Дмитрий Быстролетов: его \emph{жизнь}
похожа на авантюрный роман, в котором чего-чего, а приключений хватало,
\textbf{«Разведка уродует души и \emph{жизни}» Советский разведчик добывал
секреты по всему миру. За что его сослали в лагеря?} lenta.ru, 29.05.2021

Почну здалеку. Якось так сталося, що останнім часом в стрічці мені стали
траплятися прекрасні мотивуючи дописи про те, як потрібно перестати боротися за
життя і почати просто \emph{жити}. Автори порівнюють звичку пручатися
обставинам, які неможливо змінити, чи не з вадами, через які людина не може
насолоджуватися \emph{життям}. Таких дописів все більше. Можливо, це викликано
якоюсь внутрішньою втомою. Для людини, яка останнім часом тільки те і робить,
що протистоїть: окупантам, обставинам, постійним особистим втратам це дуже
приваблива пропозиція - припинити спротив та сприймати все як є. Але таке
можливо тільки, якщо проблеми перебільшені, а звичка пручатися скоріше нагадує
вередування з приводу чи без. Не можна насолоджуватися \emph{життям}, коли саме
\emph{життя}, гідність, або свобода під загрозою. Я звикла боротися, тому що
знаю, що безнадія - це не об'єктивна реальність. Це суб'єктивний стан готової
до поразки душі, \citTitle{Попереду виснажлива боротьба. Мені потрібна ваша
підтримка та допомога}, Лариса Волошина, gazeta.ua, 09.06.2021

Life is not a Battle to Win, but a Melody to Sing,
\textbf{Two Steps From Hell - Secret Melody},
youtube, 19.02.2018

%%%cit
%%%cit_head
%%%cit_pic
%%%cit_text
Интересно тут, правда, другое. Новые хозяева \emph{жизни} заимели карманные армии и
ловят рыбку в мутной воде, точно зная, что государство на их «шалости» глядит
сквозь пальцы.  Думают, что схватили Бога за бороду, но по факту держат волка
за уши. Стоит лишь самую малость зацепить старших товарищей за живое и вот уже
«товарищ майор» потрошит ваши кубышки, тайники и схроны.  Не расслабляйтесь,
дорогие «патриоты». Рано или поздно коса найдет на камень и кому-то
влиятельному захочется вытащить из-под полы ваши грязные секретики. И вот тогда
«госбезопасность», которая глядит сквозь пальцы, но очень внимательно,
припомнит ваши имена
%%%cit_comment
%%%cit_title
\citTitle{Не расслабляйтесь, дорогие патриоты! Рано или поздно, коса найдет на камень}, 
Максим Могильницкий, strana.ua, 20.06.2021
%%%endcit

%%%cit
%%%cit_head
%%%cit_pic
\ifcmt
  pic https://strana.ua/img/forall/u/0/36/2021-07-03_16h02_16.png
  width 0.4
\fi
%%%cit_text
\enquote{На самом деле, плохо. Потому что мы выступали за Украину, и получилось, что
отвечали на русском языке. Думаю, что это ненормально. Может быть удобнее, но
есть понятия того, что мы \emph{живем} в Украине, нужно отвечать на нашем, на
украинском языке. Понятное дело, есть какие-то свои удобства, но все-таки в
этом плане лучше было бы для меня ответить на украинском языке. Нужно было себя
перебороть и ответить на украинском}, - был категоричен молодой человек,
который ответил нам на русском и признался, что думает тоже на русском языке.
Но пока только идет к тому, чтобы общаться на украинском языке
%%%cit_comment
%%%cit_title
\citTitle{Что говорят украинцы о пресс-конференциях футболистов на русском языке. Опрос Страны}, 
Антонина Белоглазова, strana.ua, 03.07.2021
%%%endcit


%%%cit
%%%cit_head
%%%cit_pic
%%%cit_text
Нардеп от партии «Европейская солидарность», представитель украинской делегации
в Парламентской ассамблее Совета Европы Алексей Гончаренко обозвал председателя
Госдумы РФ Петра Толстого «павлином». Сообщение он выставил в Telegram-канале.
«Толстой — это клинический идиот, который в своей \emph{жизни} умеет только две вещи —
целовать Путину одно место и писать доносы. Петрушка, ну выпустили тебя в
Европу поглазеть немного, но что так перья распустил? Ишь павлин прям. Ходит
такой на эфирах выступает, важный. Всерьез тебя никто не воспринимает. Так, шут
гороховый, да и все», — написал Гончаренко
%%%cit_comment
%%%cit_title
\citTitle{Делегат Украины в ПАСЕ: Толстой — клинический идиот, умеющий только писать доносы и целовать Путину одно место}, 
, sharij.net, 08.07.2021
%%endcit

%%%cit
%%%cit_head
%%%cit_pic
\ifcmt
  tab_begin cols=2
     caption Вчитель Геннадій Іванченко

     pic https://life.pravda.com.ua/images/doc/1/d/1d472b8-ivanchenko-1.jpg
     caption Геннадій Іванченко із Приазов'я вже 34 роки працює в школі, а понад 20 років очолює маріупольський осередок національної скаутської організації \enquote{Пласт}. Фото: Андрій Болотнов

     pic https://life.pravda.com.ua/images/doc/2/b/2b19364-ivanchenko-2.jpg
     caption Своїх учнів Геннадій постійно заохочує до соціально важливих проєктів. Фото: Андрій Болотнов

  tab_end
\fi
%%%cit_text
Події 2013-2014 років повністю змінили не тільки \emph{життя} країни, а й кожного
українця. Я зрозумів, що не можу залишатися осторонь того, що відбувається у
нашій державі.  Активно став займатися волонтерською діяльністю, більше уваги
на предметах \enquote{Захист України} та фізична культура почав приділяти патріотичному
вихованню дітей та молоді, адже патріотизм має бути нормою в школі.  \enquote{У мене
чітка позиція з цього питання – національно свідома та фізично здорова молодь},
– каже Геннадій Іванченко
%%%cit_comment
%%%cit_title
\citTitle{Не бачили України без війни: як учителі в прифронтових зонах інтегрують дітей в українське суспільство}, 
Жанна Дуфіна; Дар'я Городецька, life.pravda.com.ua, 09.07.2021
%%%endcit

%%%cit
%%%cit_head
%%%cit_pic
%%%cit_text
Культуролог Евгения Бильченко, которую уволили из Национального педагогического
университета имени М. П. Драгоманова, ликвидировав целую кафедру,
прокомментировала это решение "Стране".  "Мои ощущения - это фантомные боли. Я
уже давно знала, что получу письмо о ликвидации кафедры. Это единственный
приемлемый способ меня убрать. Ибо невозможно доказать профнепригодность или
аморалку", - считает Бильченко.  Однако она не скрывает, что переживает
увольнение болезненно.  "Мне очень больно, что на кону оказалось все дело моей
\emph{жизни}. Я поднимала культурологов с 2004 года, я провела все
лицензирования и аккредитации, я открыла магистратуру, у меня были аспиранты и
руководство аспирантурой, вся теория... И они готовы этим всем пожертвовать?
Ради политики? Это похоже на какую-то истерию. Или агонию? Я относительно себя
смирилась и принимаю любые преследования. Но вот так обойтись со студентами, с
аспирантами... Это непостижимо жестоко. Но разве этой стране нужны
фундаментальная наука, теория, критика, академизм? Мы только что увидели, что,
кроме лакеев и менеджеров, не нужен уже никто", - констатировала Евгения
Бильченко
%%%cit_comment
%%%cit_title
\citTitle{\enquote{Это похоже на истерию или агонию}. Культуролог Бильченко прокомментировала увольнение из вуза}, 
Анастасия Товт, strana.ua, 21.07.2021
%%%endcit


%%%cit
%%%cit_head
%%%cit_pic
%%%cit_text
Уже майже тридцять років Україна сидить у почекальні. В очікуванні переходу до
кращого \emph{життя}, де будуть дотримуватися умови соціальної справедливості і
верховенства права. Найближчою моделлю такого життя є західний світ. А у
випадку України – Європейський Союз. Чому ж період очікування долучитися до
цієї системи для України розтягнувся аж на тридцять років? І ще до кінця не
відомо, що з того всього вийде. Чому одним колишнім республікам СРСР вдалося, а
Україні – ні. У чому причина постійного зволікання, а то й відкладання
остаточного «рішення» по Україні?
%%%cit_comment
%%%cit_title
\citTitle{Затрималися в почекальні}, 
Василь Расевич, zaxid.net, 16.07.2021
%%%endcit

%%%cit
%%%cit_head
%%%cit_pic
\ifcmt
  tab_begin cols=3

     pic https://storage.lug-info.com/cache/0/3/7be277e6-850a-4aca-b747-e6efcbbeb9a9.jpg/w1000h616%7Cwm

     pic https://storage.lug-info.com/cache/9/e/2c124897-6f64-4e5b-a6b9-437a6086e9be.jpg/w1000h616%7Cwm

		 pic https://storage.lug-info.com/cache/3/8/c41f982c-e286-42a1-b127-2b66c45246a0.jpg/w1000h616%7Cwm

  tab_end
\fi
%%%cit_text
"ЯЩИК УОРДА". Флорариум пришел в нашу \emph{жизнь} около 170 лет назад. Английский
медик и биолог Натаниэль Уорд изучал, как гусеница через куколку становится
бабочкой, он проводил свои исследования в закрытых емкостях по типу аквариумов.
Ученый заметил, что в емкостях прорастают папоротники и различные растения,
которые попадают с грунтом. 170 лет назад в Лондоне уже был достаточно
загазованный воздух и Уорд заметил, что уличные и комнатные растения из-за
загазованности, холодного и влажного климата растут намного хуже, а в
аквариумах чувствуют себя хорошо. Он сделал первую тепличку, посадил туда
цветы, с этого момента произошла революция в оранжерейном и комнатном
цветоводстве. Стеклянные емкости для выращивания и транспортировки растений
стали называть "ящиком Уорда". Именно при помощи их морем в Лондон приехали
первые тропические растения из Австралии, Новой Зеландии. "Ящик Уорда" за много
лет претерпел изменения, в нынешнем представлении – это флорариум, то есть
аквариум, в котором высажены растения
%%%cit_comment
%%%cit_title
\citTitle{Помощник флориста рассказала об особенностях и этапах создания флорариума своими руками}, 
, lug-info.com, 24.07.2021
%%%endcit

%%%cit
%%%cit_head
%%%cit_pic
\ifcmt
  pic https://img.strana.ua/img/article/3461/umer-chlen-hkchp-39_main.jpeg
  width 0.4
	caption В позднем СССР харьковчанин курировал космическую отрасль. Фото: m.5-tv.ru 
\fi
%%%cit_text
Скончался советский хозяйственный и политический деятель, министр общего
машиностроения СССР и последний член Государственного комитета по чрезвычайному
положению (ГКЧП) Олег Бакланов. Ему было 89 лет.  О смерти Бакланова сообщил в
своем Телеграм-канале глава "Роскосмоса" Дмитрий Рогозин.  "Сегодня ушел из
\emph{жизни} Олег Дмитриевич Бакланов, руководивший нашей ракетно-космической
отраслью и возглавлявший Министерство общего машиностроения СССР. Олег
Дмитриевич дружил с моим отцом, часто бывал в нашем доме. Мои искренние
соболезнования его родным и близким!" - сказано в записи Рогозина
%%%cit_comment
%%%cit_title
\citTitle{Умер последний из остававшихся в живых член ГКЧП, запустивший в космос "Буран}, 
Наталья Полулях, strana.ua, 28.07.2021
%%%endcit

%%%cit
%%%cit_head
%%%cit_pic
%%%cit_text
Хоча це не означає, що базари варто відвідувати тільки в цікавих і туристичних
містах. Зовсім навпаки: наприклад, дуже часто у маленьких українських
райцентрах, де навіть красиве фото зробити ніде, саме базар і є найцікавішою
точкою, місцевим Вавилоном, де змішалися всі і все, де вирує \emph{життя}, крики і
пристрасті. Такий базар починає роботу ще вдосвіта і перед обідом уже порожніє,
після чого \emph{життя} в містечку знову провалюється в сонну зажуру. І оживе воно
лише тоді, коли рано-вранці сторож прийде відімкнути замок на базарній брамі
%%%cit_comment
%%%cit_title
\citTitle{Базар і цвинтар – найважливіші точки в місті. Між ними й треба шукати його серце}, 
Андрій Любка, gazeta.ua, 26.07.2021
%%%endcit

%%%cit
%%%cit_head
%%%cit_pic
%%%cit_text
Их \emph{жизненное} кредо – примкнуть к сильному. Позавчера они были с
коммунистами, потому что сила была  за коммунистами. Вчера –  с как бы
демократами (по аналогичной причине). Сегодня –  с русофобами-бандеровцами. А с
кем будут завтра, зависит от того, кто на Украине будет сильным завтра.  Форму
они меняют не потому, что что-то переосмыслили (они ничего не могут
переосмыслить, ибо имеют пластилиновый мозг). И не потому, что поменяли
убеждения – нет у них убеждений. Или, правильнее сказать, все их убеждения
заключаются в том, чтобы подстроиться под власть. Не под конкретного
Зеленского, Порошенко и прочих (те все тоже пластилиновые), а под реальную
власть, центр которой, по всей видимости, не на Украине находится.  Признак
пластилинового украинца – это именно готовность подстроиться под победителя,
независимо от того, кто этот победитель. Мы же все это знаем из истории и видим
в современности
%%%cit_comment
%%%cit_title
\citTitle{ПЛАСТИЛИНОВЫЕ УКРАИНЦЫ}, Александр Каревин, facebook, 25.10.2021%
%%%endcit

%%%cit
%%%cit_head
%%%cit_pic
%%%cit_text
Гааль Дорник, родом из провинции, ни разу в \emph{жизни} не бывал на Тренторе, видел
его только в гипервизионных фильмах и передачах. Пару раз ему посчастливилось
посмотреть грандиозные трансляции с коронации нового Императора и с открытий
заседаний Галактического Совета. Но и прожив всю \emph{жизнь} на Синнаксе, маленькой
планетке, обращавшейся вокруг звезды на окраине Голубого Потока, он не был
нецивилизованным изгоем. В те времена оторванных от цивилизации мест уже не
существовало.  В Галактике насчитывалось около двадцати пяти миллионов
обитаемых миров, и все они входили в состав Империи, столицей которой был
Трентор.  Для Гааля теперешнее путешествие менее всего означало конец юности и
учебы.  Прежде он бывал в космосе только на экскурсиях и раз посетил
единственный спутник Синнакса с целью сбора данных о метеорном потоке,
необходимых ему в работе над диссертацией. Месяц назад он получил степень
доктора математики, а через две недели – приглашение от великого Гэри Селдона
прибыть на Трентор для участия в разработке проекта.  Итак, в свое первое
настоящее космическое путешествие он улетал на многие миллионы миль и световых
лет от дома...
%%%cit_comment
%%%cit_title
\citTitle{Основание}, Айзек Азимов
%%%endcit

%%%cit
%%%cit_head
%%%cit_pic
%%%cit_text
Уравнения Селдона говорили ему, что Империя, достигнув вершины, должна пасть, и
что пролетит тридцать тысячелетий человеческой \emph{жизни} в нищете и агонии, прежде
чем из руин поднимется Вторая Империя. Однако, если урегулировать некоторые
существующие условия, этот интервал можно уменьшить до одной тысячи лет.  Для
обеспечения этого Селдон организовал две колонии ученых, Которые он назвал
«Основаниями». Он намеренно поместил их в «противоположных концах Галактики».
Первое Основание, которое сконцентрировалось на физической науке, было
организованно у всех на глазах. Существование же другого, Второго Основания –
мира психоисториков и «ментальных» ученых было покрыто тайной
%%%cit_comment
%%%cit_title
\citTitle{Кризис основания}, Айзек Азимов
%%%endcit

%%%cit
%%%cit_head
%%%cit_pic
%%%cit_text
І зненацька відкривається мені, що \emph{життя} не обмежено тісним земним
кругом, що це лиш трагічна омана, що я ніби й умираю щоразу, але й не вмираю —
\emph{живу} тисячі років, безкінечно й безперервно, і \emph{живу} не в небесах,
бо небо це завжди тільки жахлива порожнеча, і немає там для нас ні надій, ні
страхів, — все тільки тут, на землі, у вічному лоні людства, в цій тричі
благословенній і стократно проклятій обителі і юдолі, де ми ростемо,
розквітаємо, радіємо, величаємось, страждаємо і животіємо.  Безсмертне не тіло
— ця комбінація елементів, які природа довільно може тасувати, мов божевільний
гравець карти в колоді тарот, і не душа, що так ніколи й не вийде з крейдяного
кола загадкового царства уяви, — безсмертна наша сутність, наше назначення,
наша судьба і відчай.  Довідавшись про це, диктатори полопаються від сміху. А з
ними й кати.  Але чому тільки диктатори й кати мають радіти своєму безсмертю? А
ми з вами?
%%%cit_comment
%%%cit_title
\citTitle{Тисячолітній Миколай}, Павло Загребельний 
%%%endcit

%%%cit
%%%cit_head
%%%cit_pic
%%%cit_text
Захопившись, Борис оповів мені таке, що стократ перевершило мою першу
зацікавленість. \emph{Життя} його сім’ї, друзів, знайомих так вразило мене, що я
вивудив з нього все можливе, бо вже не міг лишитися осторонь, збагнувши
причетність цих подій до революційного перелому в космогонії, гносеології,
технології, в безлічі відгілків великої науки. Тепер мені годі заспокоїтись,
доки не викладу всього в книзі. Форма викладу фантастична, феєрична, проте це
книга про найголовніші проблеми науки, пізнання, \emph{життя}: звичайно, все це
найголовніше для мене, а може, — і я був би щасливий тим, — і для читача.
Доктор Боголо ще буде в Києві досить довго, доки не сформується творча група
«Резонансу Еволюції». Може, мені пощастить особисто з ним поговорити і
прояснити таємницю — звідки, чому у аборигена «дикого» острова виникли такі
прогресивні еволюційні концепції?
%%%cit_comment
%%%cit_title
\citTitle{Вогнесміх}, Олесь Бердник
%%%endcit

%%%cit
%%%cit_head
%%%cit_pic
\ifcmt
  pic https://lgaki.info/wp-content/uploads/2021/10/Zaharov-v-eburge.jpg
  @width 0.4
\fi
%%%cit_text
Приз за лучший (!) короткометражный фильм на 32-м Открытом фестивале
документального кино «Россия» получил студент 5-го курса кафедры кино-,
телеискусства нашей Академии Владимир Захаров. Его кино «Реанимация
\emph{жизни}», отмеченное этой авторитетной наградой, — работа, за которую
летом на сессии автор получил «отлично».  — Герой фильма — детский реаниматолог
Алексей Завальский. Он работает в санавиации детской краевой больницы
Краснодара, — рассказал Владимир. — А в свободное от спасения детских жизней
время Алексей пишет картины. Для него это — выход эмоций, который позволяет
поддерживать баланс душевного спокойствия. Работы продает, и половина от
вырученной суммы всегда идет на помощь детям, нуждающимся в лечении и
реабилитации. Вторая половина — на покупку материалов.  Открытый фестиваль
документального кино «Россия» — ежегодный международный кинофорум, проходящий с
1988 года в Свердловске (ныне — Екатеринбург). И приз на нем — большая честь,
настоящее признание. И мы от души поздравляем с ним Владимира и его педагога,
старшего преподавателя кафедры кино-, телеискусства Валентину Семеновну
Курчину!
%%%cit_comment
%%%cit_title
\citTitle{Приз за «Реанимацию жизни»}, , lgaki.info, 22.10.2021
%%%endcit

%%%cit
%%%cit_head
%%%cit_pic
%%%cit_text
Еще одно важное событие в культурной \emph{жизни} Донецка состоится осенью. На
1 октября запланировано открытие музея при Донецкой государственной
академической филармонии. Это мероприятие будет сопровождать вернисаж «Модные
классики».  В музее будут представлены личные вещи, фотографии, музыкальные
инструменты артистов, в разные годы работавших в Донецкой филармонии, а это без
преувеличения настоящие звезды советской и российской эстрады. Посудите сами:
Валерий Ободзинский, Владимир Высоцкий, Давид Тухманов, Тамара Миансарова,
Михаил Шуфутинский, Ким Брейтбург и Александр Градский, в раззые годы работали
и гастролировали от Донецкой филармонии.  Без сомнения, музей станет новым
центром притяжения культурной \emph{жизни} столицы Донбасса
%%%cit_comment
%%%cit_title
\citTitle{Донецк отметит День города культурными событиями}, , voskhodinfo.su, 26.08.2021
%%%endcit

%%%cit
%%%cit_head
%%%cit_pic
%%%cit_text
Лови на грубого звіра — то не забавка, то боротьба тяжка, не раз кровава, не
раз на \emph{життя} і смерть. Тури, медведі, дики — се небезпечні противники; стрілами
з луків рідко кому удасться повалити такого звіра; навіть рогатиною, яку
кидалось на противника при відповіднім приближенню, нелегко дати йому раду. Тож
остатньою і рішучою зброєю було важке копіє, яким треба було влучити противника
зблизька, власноручно, з цілою силою, відразу. Схиблений удар — і \emph{життю} борця
грозила велика небезпека, коли йому не вдалось в остатній хвилі сховатись у
безпечну криївку і добути меча або тяжкого топора для своєї оборони
%%%cit_comment
%%%cit_title
\citTitle{Захар Беркут}, Іван Франко
%%%endcit

%%%cit
%%%cit_head
%%%cit_pic
%%%cit_text
Они не отрываясь смотрели друг другу в глаза, и обоих охватило неожиданное
волнение. Роберу показалось, что в голосе Изабеллы прозвучал тайный зов. Он
вдруг почувствовал какое-то странное смятение, его сковывала, смущала
собственная сила, и он боялся наделать неловкостей. Вблизи голубые глаза
Изабеллы под полукружием каштановых бровей казались еще прекраснее, еще
бархатистей казалась кожа, еще соблазнительнее похожие на пушистый персик щеки.
Меж ее полуоткрытых губ белели ослепительные зубы. Внезапно Робера охватило
желание посвятить свою \emph{жизнь}, свои дни, свое тело и душу этим губам,
этим глазам, этой хрупкой королеве, которая сейчас вдруг стала такой, какой
была на самом деле, - юной девой; его влекло к ней, и он не умел выразить это
страстное и неукротимое влечение. Знатные женщины были не в его вкусе, и не в
его натуре было разыгрывать из себя галантного кавалера
%%%cit_comment
%%%cit_title
\citTitle{Железный король}, Морис Дрюон
%%%endcit

%%%cit
%%%cit_head
%%%cit_pic
%%%cit_text
\emph{Життя} Гоголя простежене й розписане до хвилин. Не моє завдання знов його
розписувати. Ось він народився і ріс, бачив приниження людське (свого батька
перед магнатом Трощинським і своїх кріпаків перед рідним батьком), бачив велич
духу (дитяча зустріч з Г. Державшим у маєтку В, Капніста), пройшов сціллу й
харібду полтавського училища і Ніжинської гімназії вищих наук, пережив батькову
смерть, при звістці про яку хотів покінчити з собою, зазнав осмішнень,
зневажань, образ і несправедливостей, але нарешті йому 19 років, він їде до
столиці з метою "ни одной минуты короткой жизни своей не утерять, не сделав
блага", відважно й відчайдушно кидається він у вир \emph{життя}, бо "невидимая сила
натолкнула меня, предчувствие вошло в жизнь мою"
%%%cit_comment
%%%cit_title
\citTitle{Три долі. Гоголь, Шевченко, Чехов}, Павло Загребельний
%%%endcit

%%%cit
%%%cit_head
%%%cit_pic
%%%cit_text
— Мы даем вам эту возможность, — сказал Корнуэлл. — Но предупреждаю вас, сэр,
что причины ваших действий должны быть основательными.  — Вероятно, мне следует
начать с самого начала, — сказал Сторож. Это нужно было сделать сразу. Моя раса
очень древняя и возникла на планете в центре галактики. Задолго до появления
человека, может быть, задолго до того, как \emph{живое} существо на вашей планете
выползло из моря, мы создали великую цивилизацию. Я знаю, сэр Марк, что вы в
растерянности
%%%cit_comment
%%%cit_title
\citTitle{Зачарованное паломничество}, Клиффорд Саймак
%%%endcit

%%%cit
%%%cit_head
%%%cit_pic
%%%cit_text
Полемика же на означенную тему продолжается вот уже почти 2 столетия, то
обостряясь, то затухая. Очередная актуализация анти-кирилличной публицистики
наметилась на сегодняшнем этапе «гибридной войны». Мотивируется сейчас
означенная метаморфоза не только необходимостью всестороннего сближения со
странами Запада и унификацией с ними своего образа \emph{жизни}, но и
потребностью радикального разрыва с русским миром. Апологетам «денонсации»
можно посоветовать предложить ещё более «решительную» меру: полный переход на
английский язык и никакого двуязычия с любым из восточнославянских
лингвистических субъектов.  С этой точки зрения у украинской мовы (как и у всех
её диалектов-субязыков) есть серьёзный недостаток. Она и все её «наречия» очень
похожи на русский язык.  Это обстоятельство отметил известный украинизатор
А.Погребной на руб. XX-XXI вв. Анатолий Григорьевич тогда обьяснил причину
неудачи украинизации нашей страны в 1990 е гг. слишком большой схожестью между
собой собственно-украинского (виленско-полтавского) и русского
(киево-владимирского) языков
%%%cit_comment
%%%cit_title
\citTitle{Кириллица VS латиница: что лучше? / Статьи}, 
Александр Абакумов, fraza.com, 17.09.2021
%%%endcit

%%%cit
%%%cit_head
%%%cit_pic
%%%cit_text
Совершенно неясен и полностью запутан вопрос о происхождении \emph{жизни} на
Земле. Дело доходит до того, что один из ведущих биологов современности Ф.
Крик сравнительно недавно пытался возродить вариант старинной гипотезы
панспермии (корни которой восходят еще к учению отцов церкви о "зародышах
жизни").  Неприемлемость гипотезы панспермии видна хотя бы из того, что
\emph{жизнь} есть категория историческая, а отнюдь не вечная, как считал С.
Аррениус. Ее не могло быть на ранних этапах эволюции Вселенной, когда не
существовало ни звезд, ни галактик, ни даже тяжелых элементов. Поэтому не уйти
от ответа на вопрос: как же \emph{живое} произошло от \emph{неживого}? Нелепо
для этого искать вместо первобытной Земли какие-то другие космические объекты с
совершенно неясными физическими условиями.  Столь плачевное состояние этой
проблемы объясняется тем простым обстоятельством, что других форм \emph{жизни}
во Вселенной (кроме земной) мы не знаем.  Поэтому возникает важный вопрос о
распространенности \emph{жизни} во Вселенной. Не следует, однако, впадать в
черный пессимизм. Мы, астрономы, возлагаем большие надежды на орбитальный
оптический телескоп с диаметром зеркала 2,4 м, который начнет работать через
год. Есть основания полагать. что с его помощью удастся обнаружить ближайшие к
Солнцу планетные системы. Что касается внеземной жизни, то есть надежда
обнаружить ее по тем преобразованиям, которые она в процессе своей эволюции
осуществляет в атмосферах материнских планет (вспомним происхождение кислорода
в земной атмосфере)
%%%cit_comment
%%%cit_title
\citTitle{Существуют ли внеземные цивилизации?}, И. С. Шкловский, Земля и Вселенная, 1985
%%%endcit

%%%cit
%%%cit_head
%%%cit_pic
%%%cit_text
Сбор малоизвестных людей, которым случайным образом очень повезло в \emph{жизни} и
которых в украинской политологической тусовке называют, не иначе, как
депутатами КиевРады накануне праздника-годовщины освобождения Киева от
немецко-фашистских захватчиков(!) зачем-то переименовал улицу Николая Кузнецова
в улицу Олеся Бабия. Специально сообщаю для этих случайных людей, не имеющих
ничего общего с киевлянами и их проблемами, по поводу и без поющими гимн, а
также песню "Батько наш Бандера", что Николай Кузнецов это легендарный
разведчик, герой Советского Союза который гениально, путем сложнейших операций
и с \emph{риском для жизни}, организовал ликвидацию 11 крупных чинов оккупационной
немецкой администрации Украины, включая генералов - чем нанес невосполнимый
урон оккупантам и приблизил победу народа СССР, а также освобождение
территории УССР. Николай Кузнецов долгое время был наиболее ярким примером для
подражания специалистов органов госбезопасности, не только СССР и по его
биографии сняты не только документальные но и художественные фильмы. Николай
Кузнецов погиб смертью храбрых в неравном бою с отрядом УПА под Львовом
%%%cit_comment
%%%cit_title
\citTitle{Почему в Киеве не должна быть улица в честь Кузнецова? / Лента соцсетей / Страна}, 
Евгений Хурсин, strana.news, 08.11.2021
%%%endcit

%%%cit
%%%cit_head
%%%cit_pic
%%%cit_text
Врага создать непросто. Нельзя взять и просто объявить кого-то врагом. Выбор
врага суть весьма сложное действие, которое осуществляется рефлексивно.  Чтобы
случился враг, расхождения с некоторой другой стороной должны быть доведены до
фундаментального уровня: бытие одной стороны отрицает бытие другой стороны,
будущее одной стороны отрицает будущее другой стороны.  Вопросы различия языка,
культуры, экономики, технологии, психологии для вражды не годятся, потому как
не достигают онтологического уровня. Такие различия всегда вторичны.  Враждуют
онтологии и перспективы как способы \emph{жизни}. Враждуют религии и идеологии.
Враждуют цивилизации, классы, корпорации, партии, семьи-кланы и отдельные
индивиды
%%%cit_comment
%%%cit_title
\citTitle{Враг}, Сергей Дацюк, analytics.hvylya.net, 18.11.2021
%%%endcit
