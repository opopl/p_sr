%%beginhead 
 
%%file 10_11_2021.fb.mariupol.turystychne_misto.1.ploscha_svobody_voseny
%%parent 10_11_2021
 
%%url https://www.facebook.com/mistoMarii/posts/pfbid05nBfpv3HQcXVhwnFcbxfFv77ayE7D1sXCpfJhT6JukW29mGVwe55J4rNjT8xAmwRl
 
%%author_id mariupol.turystychne_misto
%%date 10_11_2021
 
%%tags 
%%title Площа Свободи у Маріуполі восени
 
%%endhead 

\subsection{Площа Свободи у Маріуполі восени}
\label{sec:10_11_2021.fb.mariupol.turystychne_misto.1.ploscha_svobody_voseny}

\Purl{https://www.facebook.com/mistoMarii/posts/pfbid05nBfpv3HQcXVhwnFcbxfFv77ayE7D1sXCpfJhT6JukW29mGVwe55J4rNjT8xAmwRl}
\ifcmt
 author_begin
   author_id mariupol.turystychne_misto
 author_end
\fi

Площа Свободи у Маріуполі восени.

🍂Перша в Україні та Європі інтерактивна площа, яка одночасно існує і в віртуальному, і в реальному світі.

🌠Щовечора скульптури голубів проєктують на площу лазерно-світлове шоу. 

🍁Проєкції змінюються в залежності від пори року. Наразі на поверхні площі
кружляють листочки, взимку – з'являться сніжинки, навесні \enquote{розквітнуть} квіти,
а влітку \enquote{порхатимуть} метелики. 

⚽️Крім сезонних елементів для трансляції на площі розроблені святкові та
урочисті проєкції. Так,  під час Євро 2020 з'явилась символіка чемпіонату,
тризуб та надписи на підтримку Збірної України з футболу.

🩺За допомогою проєкцій Маріуполь долучається до всесвітніх світлових акцій на
підтримку соціально важливих проєктів. Так,  до Дня орфанних лазерні проєкції
транслювали символи за спеціальним світловим сценарієм у блакитному, зеленому,
рожевому кольорах.

👐А під час акції \enquote{16 Днів активізму проти ґендерно обумовленого насильства в
Україні} площу Свободи розфарбували помаранчевим світлом та транслювали символи
цієї дати – долоні.

💛До Дня закоханих на площі з'являються серця. А в період зимових свят можна
створити не тільки дельфінчика, а й новорічну прикрасу.

🇺🇦На початку кожної години на площі можна спостерігати світлове шоу, під час
якого всі області об'єднуються в єдину квітучу Україну.

🐬Останні 20 хвилин кожної години на площу проєктуються дельфіни. Кожен
відвідувач може отсканувати QR-код та у застосунку створити власного дельфіна і
\enquote{випустити} його на площу Свободи.

🏆Проєкт освітлення площі Свободи отримав нагороду на міжнародному конкурсі
світлового дизайну Lighting Design Awards, який проходить в Лос-Анджелесі.
