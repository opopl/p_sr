% vim: keymap=russian-jcukenwin
%%beginhead 
 
%%file 20_10_2021.fb.fb_group.respublika_lnr.1.kiev_peregovory_sryv.cmt
%%parent 20_10_2021.fb.fb_group.respublika_lnr.1.kiev_peregovory_sryv
 
%%url 
 
%%author_id 
%%date 
 
%%tags 
%%title 
 
%%endhead 
\subsubsection{Коментарі}

\begin{itemize} % {
\iusr{Lybov Terskih}
Там не с кем разговаривать

\begin{itemize} % {
\iusr{Alexander Stáliv}
\textbf{Любовь Терских} 

Сами подумайте: жизнь осуществляется от приятного, от противного - выживание.

Как поступают с маленькими и взрослыми детьми?

Стремятся найти общий язык и донести суть, чтобы запомнили и руководствовались.

То же самое следует и со взрослыми. Разницы нет. В том числе и с преступниками.
Искать и находить общий язык.

Самое важное - самому завсегда по-человечески.

"Доброе слово, оно и кошке приятно".

С особым почтением.
\end{itemize} % }

\emph{Alexander Stáliv}
\textbf{Любовь Терских} 

Здесь и завсегда главное - самому себя правильно (нормально) вести,
по-человечески и по-людски, по-свойски, по-доброму благородно, участливо и
счастливо, жизненно относиться.

Тогда всегда есть с кем разговаривать.

Просто важно быть человеком. Добрым. Приветливым.

Именно вследствие потребительской недочеловечности и существует межусобная
звероподобная, хищническая, слабоумная (дементивная) вражда и борьба,
бойня-война, существует человеконенавистничество и живодёрство, смертоубийства
и разруха хозяйствования (экономики) государства.

Сами себя по неопределённости и недомыслию поставили в эти бесчеловечные
отношения. Сами.

Отношение необходимо и следует изменить. Отношение.

Иначе самоистребление и самоизничтожение населения Украины, в том числе и
здесь, в Луганске продолжится.

Зло - не спутник жизни.

Зло - могильщик.

Спутник жизни - доброта. Простая доброта. Доброта человечности.

В зле и нападках никогда не достичь общего языка и не договориться ни с кем ни
о чём. Это невозможно.

Именно и только в доброте, в искренней сердечной доброте осуществляется жизнь,
все её события.

В зле - выживание.

Задумайтесь об этом, каждый читающий эти строки.

Тридцать лет назад нам злоумышленно извне разрушили Советский Союз и продолжили
рушить Его части - бывшие союзные республики. И продолжают.

Кто осуществляет этот беспредел?

Заокеанские и западноевропейские финансовые мошенники, так называемые
"финансовые воротилы", "мировое правительство", "глобалисты".

Именно с их подачи по всему земному шару свирепствует современный корон@вирyс и
гибнут сотни тысяч людей. Как гибли и гибнут в межусобных войнах, в том числе и
здесь.

И отношение "там не с кем разговаривать" - не годится. Ибо не по-человечески и
не по-людски, не по-свойски, не по-доброму, с недовольством и с неприязнью.
Ничего хорошего от этого самому так относящемуся и другим людям рядом не будет
- только хуже.

Вон, какая разруха вокруг. И усугубляется.

И не следует на Россию рассчитывать, Она и так в пределах своего полномочия
помогает. Следует самим по-человечески и по-людски, по-свойски, по-доброму
благородно, участливо хозяйствовать. Иначе - невозможно.

По-взрослому зрело следует ко всему относиться. Ответственно.

В том числе и в затронутом выше обстоятельстве. Принять достаточные меры, чтобы
обеспечивать безопасность деятельности уполномоченных представителей, учитывая
прошлые ошибки и оплошности. Раз такое случилось, значит, не учли.

И разрешать это и подобные затруднения необходимо и следует не требованиями, а
запросами и просьбами, как то определено природой.

"Доброе слово, оно и кошке приятно".

Вот из этого и следует исходить, а не из обид и недовольства, зла.

Зло, о нём написал выше, зло - это не спутник жизни, зло - это могильщик.
Спутник жизни - добро, доброта душевная.

И экономика и политика, в том числе и международная экономика и политика - это
сполна доброе дело. Сполна, стопроцентно доброе дело.

Как и дипломатия - орудие (инструмент) внешнеполитической деятельности - это
тоже, само собою естественно - сполна, стопроцентно доброе дело.

Не давить и не напирать, что кто-то что-то "должен...", "должны...", а просто
общаться и говорить по-человечески добро. И всё.

Доброта все врата и двери открывает.

Главное здесь и завсегда - воля, выдержка, действия без перенапряжения. Сильно
там, где расслаблено. И там, где расслаблено, там и доходит то, что есть честь.
Там совесть просыпается, та, которая спит или дремлет. Там суть межусобной
справедливости и равенства, равноправия в общении и отношениях, равновесия
(баланса).

В этом направлении естественно необходимо и следует двигаться.

Иного пути нет.

Всё иное, там, где обида, неприязнь, зло и нападки, там - непутёвость.

И важно знать разницу и руководствоваться. И удача не пременёт - прибудет. Сто
процентов.

Все мы - люди. И между собой и по отношению ко всему нам естественно необходимо
и следует поступать по-людски и по-человечески, по-свойски, по-доброму,
участливо и счастливо, жизненно.

Берите, пожалуйста, и руководствуйтесь, живите и здравствуйте счастливо, родные
мои, родные наши.

Жизненности вам, всем и каждому.

С особым почтением.

.............................................................

К сведению. \textbf{Влад Дейнего}

\end{itemize} % }
