% vim: keymap=russian-jcukenwin
%%beginhead 
 
%%file 11_01_2020.fb.fb_group.story_kiev_ua.1.esse_elegii_detstva.cmt
%%parent 11_01_2020.fb.fb_group.story_kiev_ua.1.esse_elegii_detstva
 
%%url 
 
%%author_id 
%%date 
 
%%tags 
%%title 
 
%%endhead 
\subsubsection{Коментарі}

\begin{itemize} % {
\iusr{Ирина Петрова}
Не судите строго за слог. Я не профессиональный литератор, я инженер-сантехник. @igg{fbicon.wink} 

\begin{itemize} % {
\iusr{Светлана Манилова}

Спасибо, Ирина! И за прекрасный слог, в том числе! @igg{fbicon.wink} Не раз убеждалась, что
умение красиво и трогательно излагать мысли и чувства никак не связаны ни с
образованием, ни со специальностью и другими моментами.


\iusr{Ирина Петрова}
\textbf{Светлана Манилова} наверное, Вы правы, Света, когда пишешь о трогающем душу, как-то и слова находятся)

\iusr{Олена Темнікова}
\textbf{Ирина Петрова} что Вы! Напротив! Очень интересно и профессионально! Хотелось бы продолжения.

\iusr{Ирина Петрова}
\textbf{Олена Темнікова} есть) будет)

\iusr{Марина Лабунец}
Классно написано! Большое спасибо!

\iusr{Олена Темнікова}
\textbf{Ирина Петрова}  @igg{fbicon.thumb.up.yellow} 

\iusr{Ирина Петрова}
\textbf{Марина Лабунец} спасибо) наверное, когда не придумываешь, получается сносно)
\end{itemize} % }

\iusr{Наталія Крюкова}

\ifcmt
  ig https://scontent-frx5-2.xx.fbcdn.net/v/t39.1997-6/s480x480/68872623_2164190107205758_7121938469856739328_n.png?_nc_cat=1&ccb=1-5&_nc_sid=0572db&_nc_ohc=XK1DXEdmXJMAX-79ijL&tn=lCYVFeHcTIAFcAzi&_nc_ht=scontent-frx5-2.xx&oh=1bc88039a7625b9fee15c50fcb878ff9&oe=61932C87
  @width 0.2
\fi

\iusr{Elena Zhaley}

Спасибо, чудесный рассказ и прекрасные фотографии с комментариями.

Получила огромное удовольствие.

\begin{itemize} % {
\iusr{Ирина Петрова}
Благодарю! Ничто так не радует тщеславное сердце аффтора, как похвала Читателя)
\end{itemize} % }

\iusr{Алена Бондарева}
Ирина, невероятно! Спасибо огромное! Рассказ - просто восхитительный и... очень вкусный  @igg{fbicon.monkey.see.no.evil}  ☺ ️  @igg{fbicon.face.happy.two.hands} 

\begin{itemize} % {
\iusr{Ирина Петрова}
Благодарю! Очень тронута! Есть и продолжения, постепенно опубликую)

\iusr{Алена Бондарева}
\textbf{Ирина Петрова} очень, очень ждём!  ☺ ️  @igg{fbicon.face.happy.two.hands} 

\iusr{Ирина Петрова}
\textbf{Алена Бондарева}  @igg{fbicon.face.relieved} 
\end{itemize} % }

\iusr{Ирина Архипович}
Спасибо!!! Просто прелесть!!!  @igg{fbicon.hand.ok} Хотелось бы продолжения!!!  @igg{fbicon.face.happy.two.hands}  @igg{fbicon.heart.eyes} 

\begin{itemize} % {
\iusr{Ирина Петрова}
\textbf{Irina Arhipovich} Благодарю. Продолжение будет)
\end{itemize} % }

\iusr{Александр Венге}
Читал Ваш рассказ, а фоном была едва различимая музыка. Благодарю.

\begin{itemize} % {
\iusr{Ирина Петрова}
\textbf{Александр Венге} да, мне кажется, это была "Мелодия" Скорика. Приятно от похвалы)
\end{itemize} % }

\iusr{Пани Елена}
Прекрасно!)

\begin{itemize} % {
\iusr{Ирина Петрова}
\textbf{Пани Елена} дякую, душа моя)
\end{itemize} % }

\iusr{Лида Баженова}
Спасибо. Очень интересно

\begin{itemize} % {
\iusr{Ирина Петрова}
\textbf{Лидия Баженова} спасибо! Так и думалось, что киевляне вникнут в мелодию)
\end{itemize} % }

\iusr{Светлана Юшина}
Интересные воспоминания, редкие фотографии. Спасибо!

\begin{itemize} % {
\iusr{Ирина Петрова}
\textbf{Svetlana Yushina} Спасибо! Папа любил фотографировать. Мы много гуляли с ним)
\end{itemize} % }

\iusr{Svitlana Yeresova-Bilovir}
Написано потрясающе, спасибо!

\begin{itemize} % {
\iusr{Ирина Петрова}
\textbf{Світлана Єресова} спасибо за теплый отклик!

\iusr{Svitlana Yeresova-Bilovir}
\textbf{Ирина Петрова} пообещайте написать еще, ну пожалуйста

\iusr{Ирина Петрова}
\textbf{Світлана Єресова} да, продолжение есть. Надо только фотографии обработать. У меня Элегии в виде общего текста с фотографиями. Надо выбрать отдельно фото. Сделаем) мне самой приятно вспоминать те времена)
\end{itemize} % }

\iusr{Larysa Lytvyn}

\ifcmt
  ig https://scontent-frx5-2.xx.fbcdn.net/v/t39.1997-6/s480x480/47472862_1846683478763869_5271603212267290624_n.png?_nc_cat=1&ccb=1-5&_nc_sid=0572db&_nc_ohc=M-_OootvpKIAX9ojE0M&_nc_ht=scontent-frx5-2.xx&oh=ab6e85b6f278bbf9230bd4508a59385a&oe=6192FC67
  @width 0.2
\fi

\begin{itemize} % {
\iusr{Ирина Петрова}
\textbf{Larysa Lytvyn}  @igg{fbicon.face.happy.two.hands} 
\end{itemize} % }

\iusr{Татьяна Черник}

Спасибо, очень хороший рассказ, оттуда из детства ... А у нас коммуналка была
рабочая от з-да "Арсенал" на Печерске, ул. Киквидзе. Тоже есть, что вспомнить
...

Ещё раз благодарю автора за воспоминания о том времени, было хорошо ...!!!

\begin{itemize} % {
\iusr{Ирина Петрова}
\textbf{Татьяна Черник} приятно, что напомнила и Вам детство
\end{itemize} % }

\iusr{Юлія Конопко}
Левашовская, Меринговской - это из такого детства) Очень душевно. Спасибо

\begin{itemize} % {
\iusr{Ирина Петрова}
\textbf{Юлія Конопко} да, бабуня моя так и до конца жизни, до конца 70-х говорила: на Левашовской)
\end{itemize} % }

\iusr{Анатолий Габдрахимов}

Тёплые воспоминания о детстве и юности. .. Насколько помню, дирижабль был над
Крещатиком поднят на ноябрьские праздники 1967 года.

\begin{itemize} % {
\iusr{Ирина Петрова}
\textbf{Anatoliy Gabdrakhimov} 

да, на ноябрьские поднимали каждый год, я это помню с ранних лет, т. е. года с
61-го, вот про Первомай не помню... мне кажется, что не было. Хотя, могу
ошибаться)

\iusr{Анатолий Габдрахимов}
\textbf{Ирина Петрова} Вот на 9 мая поднимали, по моему, в 1970г.

\iusr{Ирина Петрова}
\textbf{Anatoliy Gabdrakhimov} может быть)

\iusr{Irena Visochan}
\textbf{Анатолий Габдрахимов} Да, и я помню этот день!

\iusr{Татьяна Савич}
\textbf{Анатолий Габдрахимов} дирижабль поднимали со старого двора, там сейчас метро крещатик еще с середины 50тых

\iusr{Александр Шукевич}
Всегда помню Дирижабль, на праздники, с Детства, где -то, с 1961года ..

\iusr{Olena Ivanenko}
Да, бывало... тоже помню не раз и в детстве, и в школьные годы...
\end{itemize} % }

\iusr{Альбина Орел}

Спасибо. Вспомнила и детство, и Киев 60-х, и соседей по коммуналке... Каждое
10-летие имеет свой уклад, свою атмосферу, это и быт, в котором жили, и одежда,
которую носили, и город, каким он был тогда... И складывается эта атмосфера из
таких вот фрагментов...  @igg{fbicon.hearts.two} 

\iusr{Ирина Петрова}
Да, это абсолютно верно. Спасибо за мысленное соучастие)

\iusr{Татьяна Петрова}
Спасибо. Очень по-доброму написано. Вернулась в 60-70 годы нашего Киева. Жду продолжения.

\begin{itemize} % {
\iusr{Ирина Петрова}
\textbf{Татьяна Петрова} Тёзка уважаемая, конечно, будет продолжение)

\iusr{Алла Гузий}

Прочла с огромным удовольствием. ВМЕСТЕ с героем побывала на энакомых улицах и
в легендарном колбасном магазине. описание комуналки и ее жильцов до боли
знакомо нам, поколению 50х. Спасибо вам. ждем продолжения.

\iusr{Ирина Петрова}
\textbf{Алла Гузий} спасибо за теплые слова) продолжение уже есть, в теме можно найти

\end{itemize} % }

\iusr{Валентина Турунцева}

Ирина! Огромное спасибо. Прекрасный рассказ!!!!!!

\begin{itemize} % {
\iusr{Ирина Петрова}
\textbf{Валентина Турунцева} Так приятно читать добрые слова! Спасибо!

\iusr{Валентина Турунцева}

\ifcmt
  ig https://scontent-frx5-2.xx.fbcdn.net/v/t39.1997-6/s168x128/46949788_1074619346032831_6110986696902311936_n.png?_nc_cat=1&ccb=1-5&_nc_sid=ac3552&_nc_ohc=bcMbgyras1MAX_JRdjW&_nc_ht=scontent-frx5-2.xx&oh=a24ca76bac42dc9c3fc6bfccbe91ee09&oe=61927DB7
  @width 0.2
\fi

\end{itemize} % }

\iusr{Tetyana Vincent}
Спасибо, ждем продолжения!

\begin{itemize} % {
\iusr{Ирина Петрова}
\textbf{Tetyana Vincent} будет)
\end{itemize} % }

\iusr{Maksym Oleynikov}

Все чудово, окрім однієї неточності: ні, не "построил он много добротных и
удобных домов с красивыми лепными фасадами". Коли професор Мерінг помер у
1878р., його нащадки садибу продали акціонерному товаиству, яке розпланувало її
на дрібніші ділянки і розпродало по 1тис. рублів за десятину. От лише тоді (
починаючи з 1896р.) і зявляються вулиці і будинки, про які Ви пишете. А Мерінг
за життя лікував людей, але не будував прибуткові будинки - його садиба була
просто парком зі ставком...

\begin{itemize} % {
\iusr{Ирина Петрова}
\textbf{Maksym Oleynikov} так, це трішечки художня прикраса, бо розтлумачувати хто та коли будував було б довго. Уважні та освічені читачі одразу побачать нестиковочку та підкажуть авторові, за що афффтар буде щиро вдячний)
\end{itemize} % }

\iusr{Люсянка Балашова}
Душевный рассказ... словно киноновелу посмотрела

\begin{itemize} % {
\iusr{Ирина Петрова}
\textbf{Люсянка Балашова} чорно-білу на такому тріскучому кінопроекторі)))) радію, що люди читають)
\end{itemize} % }

\iusr{Слоницький Віктор Юрійович}
С 50-х помню дирижабль

\ifcmt
  ig https://scontent-frx5-2.xx.fbcdn.net/v/t39.1997-6/s168x128/70072536_494809517752882_4258995926988750848_n.png?_nc_cat=1&ccb=1-5&_nc_sid=ac3552&_nc_ohc=pPzdCv9LlKQAX9rvbQz&tn=lCYVFeHcTIAFcAzi&_nc_ht=scontent-frx5-2.xx&oh=d0495d4897850e4d78b274bde5ff0701&oe=619384F1
  @width 0.2
\fi

\begin{itemize} % {
\iusr{Александр Венге}
\textbf{Слоницький Віктор Юрійович} а они над всем центром вечно висели. Никто не знает почему.

\iusr{Олена Андурова}
Для защиты от неожиданного нападения вражеской авиации

\iusr{Анатолий Габдрахимов}
\textbf{Александр Венге} Не помню, чтобы в 60-х над Крещатиком вечно висели дирижабли. ..

\iusr{Александр Венге}
\textbf{Anatoliy Gabdrakhimov} в 50х

\iusr{Татьяна Гурьева}
А я подумала о том, как мы все любили это время, но оно прошло мимо, и мы его не задержали...

\iusr{Maksym Oleynikov}
\textbf{Олена Андурова} 

Дирижабль із спортмайданчика школи № 117 запускали на свята, на тросі під ним
кріпили прапор, у вечірній час це з землі підсвічувалось прожектором, дирижабль
"зависав" переважно над площею Калініна. . Ніякого нападу ворожої авіації у ті
святкові дні ніхто не очікував  @igg{fbicon.beaming.face.smiling.eyes} , та й взагалі аеростати у системі ППО у ті
роки вже були повністю витіснені зенітними ракетами... 😋

\iusr{Maksym Oleynikov}
\textbf{Анатолий Габдрахимов} 

Тільки на свята - 7 листопада і 1 травня. Я жив у ті роки на вул.Заньковецької,
5/2 і все це було у мене перед очима.

\iusr{Ирина Петрова}
\textbf{Александр Венге} весь час вони, вочевидь, були під час війни. Звісно, що у 60-х то була святкова атрибутика)

\iusr{Ирина Петрова}
\textbf{Maksym Oleynikov} так, такий червоний прапор у яскравому промені...

\iusr{Ирина Петрова}
\textbf{Татьяна Гурьева} затримати час - неможливо... та, може й непотрібно? В кожному віку свої радощі та втіхи

\iusr{Olga Step}
А на прапорі я пам'ятаю ще були профілі Маркса, Енгельса, Леніна і навіть якийсь час Сталіна.

\iusr{Ирина Петрова}
Чесно кажучи, профілів не пам'ятаю, на початку 60-х сталіна точно не було

\iusr{Olga Step}
После 20 съезда Сталина не стало, а остальные какое-то время ещё вывешивались.

\end{itemize} % }

\iusr{Владимир Новицкий}

Отличный рассказ. Мне понравился, есть что вспомнить, я тоже держал голубей,
тогда это было почти а каждом дворе.

\begin{itemize} % {
\iusr{Ирина Петрова}
\textbf{Vladimir Novitsky} 

маю надію, що турманів Ви підганяли веселим посвістом) та Ваш інтерес до
голубів не був ... "гастрономічним" ))))

\end{itemize} % }

\iusr{Olena Ivanenko}

Спасибо огромное за интересное и трогательное воспоминание... а еще преклоняюсь
перед мастерским слогом...

\begin{itemize} % {
\iusr{Ирина Петрова}
\textbf{Olena Ivanenko} щиро дякую! Такі слова надають втіху)
\end{itemize} % }

\iusr{Natasha Levitskaya}

Спасибо, прекрасный рассказ! Так трогательно, душевно о том времени, о детстве,
в котором не было идеологии и всего того, чем сейчас характеризуют те времена,
потому что детство - это только светлые и тёплые воспоминания! И Киев был
добрым, уютным, гостеприимным и тёплым городом!

\begin{itemize} % {
\iusr{Ирина Петрова}
\textbf{Natasha Levitskaya} 

так, я завжди намагаюсь відокремити ідеологію та повсякденне життя тодішнє.
Звісно, є що сказати і про політичне забарвлення, але навіщо? Всі все знають та
пам'ятають, все вже перемелене, перегризене, перероблене... оце й хочеться
просто повернутись у тихі спогади дитинства

\end{itemize} % }

\iusr{Ирина Скотарь}
У Вас талант писателя! Жду ещё рассказов!

\begin{itemize} % {
\iusr{Ирина Петрова}
\textbf{Irina Skotar} Старший брат у меня профессиональный литератор, прозаик, с Литературным институтом в "кармане"))) возможно, какой-то шаловливый ген попал и в мою хромосомку))))

\iusr{Ирина Скотарь}
Талант без тяжёлой работы не проявится сам по себе. Вы - умничка!

\iusr{Ирина Петрова}
\textbf{Irina Skotar} спасибо)
\end{itemize} % }

\iusr{Николай Гайдай}

Какой слог, какой небезталанный взгляд и воздушная лёгкость!!! Расчудесно...

\begin{itemize} % {
\iusr{Ирина Петрова}
\textbf{Mykola Gaid} шановний, дякую Широ, Ви мене так нахвалили, шо аж зашарілась) не очікувала від товариства стільки добрих слів)

\iusr{Николай Гайдай}
Але ж це дійсно того варте! А далі буде?

\iusr{Ирина Петрова}
Так, буде, трішечки пізніше, але це тільки частинка циклу Елегії дитинства)
\end{itemize} % }

\iusr{Ольга Dzhun}
Супер!

\iusr{Ирина Петрова}
\textbf{Ольга Dzhun} дякую!)

\iusr{Владимир Владимирович}
Спасибо

\iusr{Ирина Петрова}
\textbf{Vladimir Vladimirovich} Спасибо, очень приятно)

\iusr{Наталия Андреева}
Спасибо! Замечательно!

\begin{itemize} % {
\iusr{Ирина Петрова}
\textbf{Наталія Андрєєва} дякую! Приємно)

\iusr{Наталия Андреева}

Я теж киянка. Але про таку вулицю не чула. Хоча зрозуміла, де вона була. Моє
дитинство - це Лівий берег. Прочитала з величезним задоволенням. Цікаво!

І читається легко! Дякую!

\end{itemize} % }

\iusr{Tatiana Tokarenko Priluchnaya}

Вибачте, але я не зрозуміла "аппетитные краснобокие троллейбусы". Вам хотілося
їх з'їсти?!

\begin{itemize} % {
\iusr{Ирина Петрова}
\textbf{Tatiana Tokarenko Priluchnaya}  @igg{fbicon.face.tears.of.joy}  так, вони мені яблучка нагадували, напевно)
\end{itemize} % }

\iusr{Лариса Быковская}
Такого прекрасного рассказа давно не читала! Талантливо написали. Спасибо!

\begin{itemize} % {
\iusr{Ирина Петрова}
\textbf{Лариса Быковская} спасибо! Приятно, сердцу автора такие слова, как бальзам)))
\end{itemize} % }

\iusr{Наталья Костецкая}

Прекрасный слог, легкий, прямо чеховский! Жили рядом, на Николаевской (Карла
Маркса). Погрузили в детство!

\begin{itemize} % {
\iusr{Ирина Петрова}

Очень приятно, что соседи находятся. Тогда Вы можете просто пошагово
представить путь Наумчика от Колбасного до дома со сверточком в вощеной
бумажке. Мимо молочки на Заньковецкой, мимо аптеки на углу Станиславского,
правда? Когда пишешь о милом сердцу предмете (конечно, о детстве, а не
Наумчике @igg{fbicon.face.grinning.sweat} ), тогда легко побираются нужные слова.

\iusr{Наталья Костецкая}
\textbf{Ирина Петрова} 

Ирина, конечно! Всё правда! И Колбасный, Молочный, и аптека... В Вашем доме жили
и живут наши большие друзья, семья проректора сельхозакадемии. Бывала в Вашем
доме и я. Помню очереди за хлебом в 1964 году. Ранним утром бежали с подругой из
дома на Крещатике, 25 на спортивную площадку Вашей школы 117, а бабушки уже
занимали очередь. Мы поступали в институты тем летом и перед репетиторами
заряжались энергией!

\iusr{Ирина Петрова}

О! Я, конечно, прекрасно тоже знаю, о какой семье речь! Со старшим сыном мы
одногодки. Правда, мы учились в разных школах, он - в 86-й, а я в 94-й)


\iusr{Наталья Костецкая}
\textbf{Ирина Петрова} Мы ещё и в одной школе 94 учились! Только я гораздо раньше.

\iusr{Ирина Петрова}
\textbf{Наталья Костецкая} 

ух ты! Нееет, не сильно уж и гораздо) я закончила в 73-м, но, очень много помню
и знаю старших классов) кто были Ваши одноклассники?

\iusr{Наталья Костецкая}
\textbf{Ирина Петрова} 

У нас разница 10 лет. Мы учились в 11-ом классе, а Вы в первом. Скорее всего у
нас и учителя были разные. Одним из одноклассников был Виктор Киркевич.
Наверняка Вы его знаете по книгам о Киеве.


\iusr{Ирина Петрова}
\textbf{Наталья Костецкая} 

наверное, вместе с Вами училась Света Смирнова? Мы, первоклашки, дарили цветы
выпускникам на последнем звонке в мае 1964) и, наверное, Вы помните физрука Сан
Тиныча, нет?

\end{itemize} % }

\iusr{Геннадий Бабарика}

Спасибо! Ваше эссе необходимо в качестве \enquote{прививки} порядочности давать
рекомендовать всем горожанам!!! Необходимо \enquote{инфицировать} любовью и уважением к
истории города, прошлому, ценить и беречь!

\begin{itemize} % {
\iusr{Ирина Петрова}
\textbf{Геннадий Бабарика} 

дякую! Я цілком згодна, що треба знати історію свого міста, вулиці. Людина без
минулого, як реп'ях без коріння, несе його вітер по всіх усюдах((((

\iusr{Геннадий Бабарика}

\textbf{Ирина Петрова} 

Можливо, ( дуже, на це сподіваюсь) саме нашому поколінню дісталось (вийшло)
передати нащадкам пам'ять про втрачене, сплюндроване, все краще що було, та про
що мріяли в минулому. Хай знають наше ставлення, та формують своє, попри
складнощі та труднощі сьогодення.

\end{itemize} % }

\iusr{Татьяна Гурьева}
как хорошо написано, с душой)

\iusr{Ирина Петрова}
Танечка, спасибо! Добрые слова от коллеги по перу очень приятны!

\end{itemize} % }
