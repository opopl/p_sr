% vim: keymap=russian-jcukenwin
%%beginhead 
 
%%file slova.banderovec
%%parent slova
 
%%url 
 
%%author 
%%author_id 
%%author_url 
 
%%tags 
%%title 
 
%%endhead 
\chapter{Бандеровец}
\label{sec:slova.banderovec}

Почему-то современные \emph{бандеровцы}, которым принадлежит вся власть в западных
регионах страны, очень стесняются собственной истории. Надписи на памятниках
жертвам \enquote{буржуазных националистов} замазывались цементом, целые страницы
прошлого вырваны с мясом, чтобы напоказ гордиться незначительными эпизодами.
Например, про период бурной любви с немцами и деятельное участие националистов
во всех мероприятиях по укреплению оккупационной власти принято отнекиваться и
яростно доказывать, что такого не было никогда. Но так же яростно убеждать в
войне \enquote{на два фронта}, для которой примеров как-то маловато,
\textbf{Современные бандеровцы любят говорить, что их предки воевали на два фронта},
Дмитрий Заборин, strana.ua, 06.05.2021

Если только не считать войной такие эпизоды, как убийство пяти сержантов 1-й
гв. танковой бригады в с. Иваниковка Богородчанского района. Дочь хозяина хаты,
где спали бойцы, навела \emph{бандеровцев}, которые застрелили четверых, а пятого
ранили. Он пытался бежать, но его догнали и добили лопатами. После всех кинули
в скотомогильник и забросали землей.  За это гражданку Ганну Ковтык
\enquote{отправили в Сибирь} - так в документе 1985 года. Правда, к тому
моменту она оттуда уже вернулась, как ни странно при лютой жестокости совитив,
о которой тоже модно ввернуть что-нибудь эдакое,
\textbf{Современные бандеровцы любят говорить, что их предки воевали на два фронта},
Дмитрий Заборин, strana.ua, 06.05.2021

Посмотрите внимательно: видите контур «незалежной»? Так там и Крым, и Донбасс
на старом месте – как было до госпереворота, пардон, до «революции гидности». А
еще, наябедничаю, там и «кричалки» \emph{бандеровские}, которые возродил майдан
2014 года и закрепил на государственном уровне, шелком вышиты. Да-да: «Слава
Украине! Героям слава!» - как и положено настоящему арийско-патриотическому
государству (правда, почему-то второй лозунг, очень похожий на лозунг
нацистской Германии, спрятан в тылу, на спине, да еще и с изнанки вышит)),
\citTitle{Форма не главное – главное содержание! И в спорте тоже...}, Мысли Бабы Яги, zen.yandex.ru, 07.06.2021

Привычка всюду лепить \emph{бандеровщину} выходит боком.  УЕФА потребовал
убрать с формы Украины слоган «Героям слава».  Наши излишне патриотические
друзья привыкли тулить свою бандеровщину куда нужно и куда не нужно. Внутри
страны это работает на ура – тут предпочитают не связываться. Но за границей,
оказывается, можно получить по наглой сельской морде. Теперь ждем истерии
вплоть до уровня президента.  Панове, отозвать сборную с Евро слабо? А как же
«принципы»?,
\citTitle{УЕФА потребовал убрать с формы сборной Украины слоган Героям слава}, 
Вячеслав Чечило, strana.ua, 10.06.2021

%%%cit
%%%cit_pic
%%%cit_text
И получилось, что номинал главы ОУН Евгения Коновальца - две гривны.  Хорошая
инициатива Наубанка оценить в гривне героев, которым слава. Две гривни -
номинал Коновальца. В 1930-х, напомню, он во главе ОУН тесно сотрудничал с
НСДАП и Третьим Рейхом, а в 1934 ОУН перенесла свою штаб-квартиру в Берлин и
вошла в состав Гестапо на правах особого отдела. Присвоить Коновальцу звание
Героя Украины предлагает правящая партия Слуга народа в лице нардепа Александра
Алексийчука, о чем направлен запрос президенту, поддержанный 241 депутатом СН.
- Возьмите две \emph{бандеры} на сдачу. - та оставьте себе
%%%cit_title
\citTitle{Нацбанк решил выпустить двухгривневую монету с Евгением Коновальцом}, 
Елена Дьяченко, strana.ua, 12.06.2021
%%%endcit

%%%cit
%%%cit_pic
%%%cit_text
Аби завершити епоху подвійних стандартів та тотальної брехні в українській
політиці, нам потрібний свій історичний трибунал. Нам потрібний публічний суд
над українським Петеном. Який, до речі, зробив для величі Франції набагато
більше, аніж Порошенко – для слави України.  Бо не повернувши мораль у
політику, ми приречені залишатися сталіністами. Навіть якщо проголосимо
червоно-чорний прапор державним, заборонимо згадувати СРСР і перейменуємо усі
вулиці на честь \emph{Степана Бандери}
%%%cit_title
\citTitle{У політику треба повернути мораль. Інакше ми приречені залишатися сталіністами}, 
Геннадій Друзенко, gazeta.ua, 14.06.2021
%%%endcit

