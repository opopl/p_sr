% vim: keymap=russian-jcukenwin
%%beginhead 
 
%%file slova.territoria
%%parent slova
 
%%url 
 
%%author 
%%author_id 
%%author_url 
 
%%tags 
%%title 
 
%%endhead 
\chapter{Территория}
\label{sec:slova.territoria}

%%%cit
%%%cit_pic
%%%cit_text
Бюрократии нужна \emph{территория} и население потому что она извлекает из этого свой
доход. Чем больше территория страны и е население, - тем выше ее доход. Самая
богатая бюрократия сейчас в Китае, Индии, США и России. Борьба за квадратные
километры своей \emph{территории} всегда объявляется священной войной и вся
пропагандистская машина государства работает на обоснование необходимой
жертвенности всей нации во имя этого принципа. Жертвует, правда, не вся нация
а, главным образом, беднейшие ее слои, а обеспеченная часть, в основном,
занимается теоретическим обоснованием этой самой священности войны. Так было
всегда и во всех странах. Последний яркий пример - это война за Карабах.  Между
тем, нет ничего более бессмысленного и антигуманного в современном мире чем
борьба, и тем более война, за \emph{территориальную целостность}. Потому что все
больше экономистов начинают осознавать тот факт, что странам с меньшей
территорией легче поднять свой уровень жизни чем странам с большой \emph{территорией}
%%%cit_comment
%%%cit_title
\citTitle{Мы живем в мире, когда размер перестал иметь значение}, 
Андрей Головачев, strana.ua, 13.06.2021
%%%endcit
