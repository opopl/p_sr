% vim: keymap=russian-jcukenwin
%%beginhead 
 
%%file 11_11_2018.fb.lesev_igor.1.po_vyboram_v_ldnr.cmt
%%parent 11_11_2018.fb.lesev_igor.1.po_vyboram_v_ldnr
 
%%url 
 
%%author_id 
%%date 
 
%%tags 
%%title 
 
%%endhead 
\subsubsection{Коментарі}
\label{sec:11_11_2018.fb.lesev_igor.1.po_vyboram_v_ldnr.cmt}

\begin{itemize} % {
\iusr{Виталий Васильевич}

\ifcmt
  ig https://scontent-frx5-2.xx.fbcdn.net/v/t39.1997-6/s168x128/851586_126361977548599_392107290_n.png?_nc_cat=1&ccb=1-5&_nc_sid=ac3552&_nc_ohc=buNrRcYxWj8AX_y1PR6&_nc_ht=scontent-frx5-2.xx&oh=40a338d833cc62bfe074b57b61a915ab&oe=61991D31
  @width 0.05
\fi

\iusr{Виталий Васильевич}
Нам плохо и им плохо

\iusr{Igor Maximov}

Как бессмысленные? Это "восстание" двух областей. Несогласие, вот уже четыре
года, с сумасшествием центральной власти. Так же, как у нас в Крыму. Но мы
реально хотели в Россию. Вы хотите - верьте, хотите - не верьте, мне глубоко
все равно. Очень жалко ЛДНР, страдают люди и гибнут. Но при, допустим,
вхождении в Украину их там ждет зачистка полная. Вы вот даже не возражайте. Это
понятно. ПОЭТОМУ. Выборы для них совершенно необходимы. Как и для хунты было
важно легализовать президента хунты - Порошенко. Кому рассуждать о законности.
Ваш законный президент - Янукович. Согласны на него? Не согласны? А ЛДНР не
согласны на вашего кровавого шоколадного. Извините многобукв.

\iusr{Сергей Сысоев}

Скрестить бобра с тумбочкой - это сильно!
Пусть наши будут бобром!
 @igg{fbicon.smile} 

\iusr{Василий Стоякин}

Минские соглашения не предполагают никаких переговоров Киева с ЛДНР. Это все
кремлевские мечтания. Неизменность позиции Киева тут логична и полностью
соответствует минским соглашениям. Хотя им, конечно, не надо было вообще
участввать в форматах, где присутствуют представители республик.

\begin{itemize} % {
\iusr{Игорь Лесев}

Чего это "не предполагают"? Хотя все сроки Минска давно уже вышли и они в
принципе ничего не предполагают... но это такое. А вообще, "в особых районах"
там должны пройти выборы. И какая разница, как эти "особые районы" называются?
Хоть Швамбрания.

\iusr{Василий Стоякин}
\textbf{Игорь Лесев} У них есть вполне конкретные названия. вот, например, новоазовский

\iusr{Игорь Лесев}
Особый Новоазовский район

\iusr{Василий Стоякин}
\textbf{Игорь Лесев} И он в том числе. Игорь, нельзя называть районом что заблагорассудится. Район как термин закреплен в конституции. Т.е. речь идет именно об административных единицах, входящих в две области. в них выборы должны быть. А ЛДНР в области не входят, в них и выборы не нужны, и переговариваться с ними не о чем

\iusr{Игорь Лесев}
\textbf{Василий Стоякин Киевский} ох уж эти законники-буквоеды с социологическим образованием... вот будут скоро самые законные выборы в Верховную Раду, баллотируйся и там будешь рассуждать... а здесь извольте-с, умные комментарии не приветствуются

\iusr{Матвей Кублицкий}
\textbf{Василий Стоякин Киевский} четкое украинское видение у вас. А потом удивление - а нас то за шо.... пункт 4 минских соглашений как раз о начале переговоров о модальности выборов. не начали переговоры - ваша проблема. время было полно.

\iusr{Василий Стоякин}
\textbf{Матвей Кублицкий} Я не виноват, что минские соглашения написаны с украинской позиции. Что до переговоров, то переговорщики должны быть приемлемы для обеих сторон. США нашли в себе силы не предлагать для переговоров Мельничука и Семенченко. Россия - нет. Хотя ей предлагали

\iusr{Матвей Кублицкий}
\textbf{Василий Стоякин Киевский} Россия не нашла в себе силы не предложить Мельничука и Семенченко??? удивили. не знал что российские власти этих двух мудаков поддерживают

\iusr{Василий Стоякин}
\textbf{Матвей Кублицкий} Вы поняли, что я написал

\iusr{Матвей Кублицкий}
\textbf{Василий Стоякин Киевский} конечно. я и написал - ЧТО я понял...

\iusr{Елена Максюченко}
Василий, особоенно тронуло про логичность, что в принципе, невозможно, куда ни глянь, Игорь все четко написал, плюсую
\end{itemize} % }

\iusr{Матвей Кублицкий}

про выборы в ЛДНР весело читать - ибо в марте след года пройдут выборы на
Украине - и по сравнению с ними выборы на Донбассе окажутся сверхчестными и
прозрачными. Уже щас в статье Игоря видно двойное отношение к одному и тому же.
донбасские - пластилиновые, а у Киева старший товарищ. К выборам на Донбассе
будут допущенны согласованные с Москвой - а разве кто-то сможет пройти на
выборах на Украине без согласования Госдепа?

\begin{itemize} % {
\iusr{Игорь Лесев}
может, но не очень далеко. Ну и самое главное - а нас точно также любят бесплатно старшего брата, как на Донбассе любят своего.

\iusr{Матвей Кублицкий}
\textbf{Игорь Лесев} эту мысль я и высказал - всё одинаково. а уж если задуматься, что такое демократия в чистом виде - понимаешь что все выборы - хорошо срежиссированный спектакль в любой стране

\iusr{Oleksiy Kanishchev}
Отвалите уже от Украины, идите спасать очередных детей в Гондурасе. Вы обсуждаете наши выборы, у вас даже таких в РФ нет.
\end{itemize} % }

\iusr{Михаил Мищишин}

Наиболее сбалансированный анализ из тех, что попались на глаза. Но надо держать
перед глазами то, с чем мы имеем дело. Это видно по реакции Украины. Вначале:
Минские соглашения не выгодны Украине, Россия их не выполняет, мы их будем
выполнять в той последовательности, в какой посчитаем нужным, Минские
соглашения - ошибка. В общем, Минские соглашения - это то, о чем много говорят,
но никто не обращает на них внимания. И вдруг: та-та-та дам! Выборы пускают под
откос Минские соглашения!:). Все неважное вдруг стало важным:).Эта
фантасмагорическая реальность предполагает или подход с юмором, трудно всерьез
обсуждать то, что никто не выполняет. Или надо называть вещи своими именами. А
это сейчас никому не нужно нигде. Всех устраивает игра в Минские соглашения,
которые не выполняются.

\begin{itemize} % {
\iusr{Игорь Лесев}

Вы все несколько усложняете, наделяя нашу верхушку некими признаками интеллекта
и сообразительности. Если завтра, предположим, РФ вдруг согласится выплатить
Украине проигранный суд по газовому иску, в Киеве тут же будут искать какой-то
подвох и хитросплетения. Реакция киевский властей на любые действия/бездействия
наших соседей неизменна - всегда быть против.

\iusr{Михаил Мищишин}
\textbf{Игорь Лесев} 

Да. Именно так. Это был способ остановить большую бойню. Но как идти дальше и
сделать ее невозможной впредь - пока что никто не знает. Поэтому соглашения нам
то нужны, то не нужны, то они правильные, то нет. Не мы решаем, что будет, и
что должно быть на Донбассе, увы. Вот эту простую штуку мы все часто упускаем
из виду и анализируем ситуацию так, будто бы от нас что-то там зависит.

\end{itemize} % }

\iusr{Igor Maximov}

Выборы состоялись. На 16-00 явка 70+ процентов. На фото - голосующие досрочно
военнослужащие. Это сделано для того, чтобы не оголять фронт в воскресенье. ЕС
заявил, что выборы не признает. В республиках полно других забот. Тем временем
Трамп в Париже не заметил Петра Алексеевича. Драма.

\iusr{Михайло Бойченко}
Очень гигантская и всегда тупо неизменна - Игорь, да тьі новьій Гоголь, епрст @igg{fbicon.face.tears.of.joy}{repeat=3} 

\end{itemize} % }
