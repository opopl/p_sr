% vim: keymap=russian-jcukenwin
%%beginhead 
 
%%file 10_01_2022.stz.news.lnr.lug_info.1.mir_donbass
%%parent 10_01_2022
 
%%url https://lug-info.com/news/novogodnij-blic-voenkor-dmitrij-stesin-nicego-krome-mira-uze-slucitsa-ne-mozet
 
%%author_id news.lnr.lug_info
%%date 
 
%%tags donbass,vojna,mir,ukraina
%%title Новогодний блиц. Военкор Дмитрий Стешин: "Ничего, кроме мира, уже случиться не может"
 
%%endhead 
\subsection{Новогодний блиц. Военкор Дмитрий Стешин: \enquote{Ничего, кроме мира, уже случиться не может}}
\label{sec:10_01_2022.stz.news.lnr.lug_info.1.mir_donbass}

\Purl{https://lug-info.com/news/novogodnij-blic-voenkor-dmitrij-stesin-nicego-krome-mira-uze-slucitsa-ne-mozet}
\ifcmt
 author_begin
   author_id news.lnr.lug_info
 author_end
\fi

\begin{zznagolos}
Известный российский военкор Дмитрий Стешин в традиционном новогоднем
блиц-интервью ЛИЦ делится своим видением итогов прошедшего года и прогнозом на
год наступивший.	
\end{zznagolos}

- Как вы оцениваете положение и ситуацию в Республиках Донбасса к началу 2022
года?

- Когда приезжаешь в Донбасс, не можешь сказать точно: я уже в России или еще в
России?

\ii{10_01_2022.stz.news.lnr.lug_info.1.mir_donbass.pic.1}

Меня поразили две детали быта. Оформившийся утренний и вечерний трафик -
совершенно непривычное явление для последних семи лет. Одни объясняли его тем,
что людей прибавилось, другие – прибавилось машин на дорогах. В любом случае
это означает не запустение, а оживление. Вторая деталь – рекламные плакаты
\enquote{Отдых в Египте, вылет через Ростов}. Разумеется, процентов 70\%
жителей Донбасса не могут себе этого позволить, но какой-то спрос появился, и
это опять знак нормальной, привычной жизни, пусть и несущей отпечаток войны.

- Что считаете наиболее значимыми достижениями Луганской и Донецкой Народных
Республик и каковы были самые серьезные проблемы в ушедшем году?

- Промышленность и только промышленность - мерило всего в Донбассе. Весной я с
ужасом читал про забастовки на шахтах и останавливающиеся заводы. Было ясно,
что первая попытка реанимировать металлургию закончилась крахом. Самое обидное,
что это произошло вопреки логике рынка, на самом пике цен. Но Россия в
очередной раз показала, что Донбасс ей нужен. Летом зашел инвестор с деньгами.
Я был на нескольких заводах и в одном рудоуправлении – эти потраченные деньги
можно увидеть и потрогать руками. Я видел, что эти предприятия работают, и на
них стали возвращаться люди, сбежавшие от безденежья. Новый инвестор, кстати,
начал с того, что закрыл долги по зарплатам, потом поднял их и выписал всем 20
тысячам сотрудников премии. И, судя по указу о снятии украинских квот и
таможенных ограничений, признании сертификатов и допуске продукции Донбасса к
сфере госзакупок, Кремль положительно оценил работу инвестора и людей,
запустивших заводы в ЛНР и ДНР. Я постоянно получал смс-сообщения с вакансиями.
Кадровый голод – новая проблема Донбасса, обратная сторона такого святого дела,
как выдача российского гражданства в упрощенном порядке. Но это решается в
отличие от траншей на передовой, где я с ужасом увидел растущие древесные грибы
– так давно стоит фронт, так давно ничего не менялось.

- Ваш прогноз развития ситуации вокруг Донбасса в 2022 году?

- Россия как модератор идущих процессов демонстрирует две шапки, и в каждой
лежат яйца. В одной шапке – возрождение экономики - сильнейший раздражитель для
Украины. Во второй шапке - агрессивная риторика и демонстрация готовности к
силовым действиям, к долгожданному \enquote{распрямлению горбатых}. На фронтах
тихо, не жужжит (беспилотный летательный аппарат) \enquote{Байрактар},
демонстрируя доблесть турецких маркетологов. На Украине поняли, что перегнули
палку. Особенно забавно было наблюдать, как интересанты и покровители начали
отползать от Украины один за другим, причем обязательно сообщая публично, что
за Киев воевать не будут.  Мне совершенно твердо обещали \enquote{интересный
год}. И закончится он миром. Как заметила жительница одного поселка в
\enquote{серой зоне}, после того, что здесь творилось семь лет, ничего, кроме
мира, уже случиться не может. Пусть так и будет.
