% vim: keymap=russian-jcukenwin
%%beginhead 
 
%%file 30_10_2020.fb.news.ua.pravda.1.grecia_turcia_zemletrus
%%parent 30_10_2020
 
%%url https://www.pravda.com.ua/news/2020/10/30/7271817/
%%author 
%%tags 
%%title 
 
%%endhead 

\subsection{Грецію і Туреччину сколихнув потужний землетрус, є значні руйнування}

\Purl{https://www.pravda.com.ua/news/2020/10/30/7271817/}
П'ятниця, 30 жовтня 2020, 15:04

Сильний землетрус магнітудою 6,7-7 бала за шкалою Ріхтера стався в п'ятницю
біля східного Егейського острова Самос.

Про це повідомляє "Європейська правда" з посиланням на Ekathimerini.

Землетрус стався о 13.51 за місцевим часом, всього в 19 кілометрах на північний
північний захід від столиці острова, на глибині 10 кілометрів.

Місцеві ЗМІ на Самосі повідомляють про пошкодження будівель і дорожньої мережі,
в той час як з сусідньої Туреччини надходять повідомлення про те, що кілька
будівель обвалилися в приморському місті Ізмір.

Європейський середземноморський сейсмологічний центр (EMSC), тим часом, дав
землетрусу магнітуду 7 балів.

\ifcmt
pic https://img.eurointegration.com.ua/images/doc/b/7/b78afad-68729.jpg
\fi

Як повідомляє Daily Sabah, землетрус був досить сильним, щоб люди в паніці
повискакували на вулиці в країні, яка бачила потужні землетруси, що забрали
життя тисяч людей в минулому.

Міністр внутрішніх справ Туреччини Сулейман Сойлу заявив, що в районах Байракли
і Борнова Ізміра обвалилися шість будівель. Міністр також повідомив, що з
сусідніх провінцій, включаючи Ушак, Денізли, Маніса, Баликесір, Айдин і Мугла,
надійшли повідомлення про незначні пошкодження.

Телевізійні кадри та відео, що циркулюють в соціальних мережах, показали, як
люди відчайдушно прибирають уламки деяких зруйнованих будівель, щоб дістатися
до людей, що опинилися під завалами.

Про загиблих і постраждалих поки даних не надходило.
