% vim: keymap=russian-jcukenwin
%%beginhead 
 
%%file 10_01_2022.fb.zabuzhko_oksana.1.chudo_na_dnipri.cmt
%%parent 10_01_2022.fb.zabuzhko_oksana.1.chudo_na_dnipri
 
%%url 
 
%%author_id 
%%date 
 
%%tags 
%%title 
 
%%endhead 
\zzSecCmt

\begin{itemize} % {
\iusr{Volodymyr Viatrovych}

Нас було багато завдяки тому, що тривалий час Майдан тримався ненасильницьких
засобів боротьби. Саме завдяки цим ненависним деяким радикалам «ліхтарикам» і
«танцям». Такий Майдан приваблював не лише світ, створюючи позитивний образ
протесту, але й інших українців готових долучитися до нього. Радикалізація
протесту звужує його мобілізаційні можливості.

Але з іншого боку Майдан набравши розгону як ненасильницький спромігся дати
силову відповідь на насильство влади. І це сталося не надто рано (Казахстан) і
не надто пізно (Білорусь). Ми справді пройшли по лезу.

\begin{itemize} % {
\iusr{Mila Oliynyk}
\textbf{Volodymyr Viatrovych} білоруси теж були ненасильницькі з рожевими поні. І?

\iusr{Volodymyr Viatrovych}
\textbf{Mila Oliynyk} прочитайте комент до кінця. В ньому і про готовність відповісти на силові дії влади.

\iusr{Maria Zayats}
\textbf{Volodymyr Viatrovych} так, але і люди в своїй більшості вийшли якраз реагуючи на насильство над студентами зі сторони режимної влади януковича – хто зна чи вийшло б стільки людей, якби не трагічні події 30 листопада.

\iusr{Volodymyr Viatrovych}
\textbf{Maria Zayats} 

насильство 30.11 мало викликати негайну силову відповідь Майдану за планом
влади. І це мало стати кінцем Майдану. Але цього не сталося. І навіть спроби
«пришвидшити» перетікання протесту 1.12 теж не вдалися. Тому коли Майдан таки
відповів силою це не змінило позитивного ставлення до нього з боку світу та
українців

\iusr{Мирослав Логийко}
\textbf{Volodymyr Viatrovych}

 @igg{fbicon.thumb.up.yellow}  але все ж таки без парламенту з проукраїнськими та проєвропейськими
парламентарями, які не втекли, а стали представляти Майдан у Раді, нічого б не
вийшло.

\begin{itemize} % {
\iusr{Volodymyr Viatrovych}
\textbf{Miroslav Logiyko} згоден. І опозиція в парламенті вчасно і вдало зіграла свою роль

\iusr{Мирослав Логийко}
\textbf{Volodymyr Viatrovych} 

В 2014-му році у України вийшло не тільки завдяки майдану і добровольцям, а й
завдяки Парламенту України.

У Парламенті України були проукраїнські та проєвропейські політики та партії,
які втримали державу і заклали фундамент нової.

Саме Парламент України, після втечі януковича і його влади, оголосив початок
проведення антитерористичної операції.

Парламент України провів вибори і легітимізував владу.

І тепер саме парламент України є ціллю путіна-зеленського.

ОПЗЖ вже заявили про свої наміри кинути мандати, а це в свою чергу дозволяє
зеленському розпустити парламент.

\url{https://youtu.be/Kccmy-TqmgM}

Зеленський з його монобільшістю та проросійскою підтримкою це підхватить.

\url{https://m.facebook.com/story.php?story_fbid=4932810226762610&id=100001010458434}

Цього разу Путін не повторить своєї помилки 2014-го.

Главный вывод по событиям в Казахстане - что бы Западу было кому помогать, в
том числе оружием, нужна организованная оппозиция в стране и лидер у этой
оппозиции.

Запад не будет поставлять орудие и помощь разроздненым группам сопротивления.

Именно этого сейчас добивается путин/зеленський - убрать лидера оппозиции, как
объединяющую фигуру, и уничтожить парламентаризм как таковой, что бы не было
альтернатив.

Уже год про это пишу:

\url{https://m.facebook.com/story.php?story_fbid=4932810226762610&id=100001010458434}
\end{itemize} % }

\iusr{Ntina Ntoubrova}
\textbf{Volodymyr Viatrovych} 

Треба ще й врахувати, що це для українців другий майдан (а для білорусів -
перший), і що в нас на відміну від росіян та казахів - дуже компактна територія
без значних перешкод в пересуванні із наспунктів навіть прикордоння - до Києва.
Ну і наостанок - на нас перших цю спецоперацію випробували.

\begin{itemize} % {
\emph{Marina Mischenko}
Ntina Ntoubrova Третій. Перший був " Революція на граніті", голодування студентів.Після першого Майдана був офіційно прийнятий Блакитно- жовтий Державний прапор
\end{itemize} % }

\end{itemize} % }

\end{itemize} % }
