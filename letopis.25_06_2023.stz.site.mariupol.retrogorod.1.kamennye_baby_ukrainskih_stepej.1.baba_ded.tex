% vim: keymap=russian-jcukenwin
%%beginhead 
 
%%file 25_06_2023.stz.site.mariupol.retrogorod.1.kamennye_baby_ukrainskih_stepej.1.baba_ded
%%parent 25_06_2023.stz.site.mariupol.retrogorod.1.kamennye_baby_ukrainskih_stepej
 
%%url 
 
%%author_id 
%%date 
 
%%tags 
%%title 
 
%%endhead 

\subsubsection{Почему \enquote{баба}? Потому что \enquote{дед}}

Тюркское слово \enquote{балбал} - означает деда, предка. Каждое племя кочевников
оставляло свои изваяния в степи. Племена уходили, а бабы оставались стоять на
курганах или перед ними.

Древнейшие в человеческой истории каменные бабы, в принципе даже называться
бабами не имеют права – слишком мало в них от человека. Ни рук, ни ног, ни
украшений. Это просто каменные стелы – и на отшлифованной поверхности вырезан
контур головы… Это подобия, символы человека.

По данным научной работы Ольги Демидко, в районе города Мариуполя более ста
древних антропоморфных, то есть человекоподобных, каменных скульптур.

Самые древние из этих каменных баб - две стелы ямной культуры и одна –
катакомбной культуры. Все это еще бронзовый век человечества, а культуры
называются по типу захоронений. Киммерийских скульптур в Мариуполе не
сохранилось, хотя загадочные киммерийские племена когда-то населяли здешние
места. О них вообще известно мало.

Скифская скульптура только одна. Пять баб создали печенеги – те самые
кочевники, что пустили киевского князя Святослава на столовые приборы. Три бабы
высекли из камня торки. 28 половецк, муж и жен поровну И около девяноста баб –
половецкие. Три из них украшают вход в краеведческий музей.

\emph{\enquote{Они суровы и жестоки, Их бусы — грубая резьба}}

1919 год, война, голод, тиф. А поэт Велимир Хлебников пишет стихотворение
\enquote{Каменная баба}. Все истуканы видятся ему девушками, с \enquote{булыжной грудью}, с
\enquote{каменным уставом} любви.

В реальности каменные бабы встречаются как мужчины, так и женщины – по фигуре
обычно ясно. Есть и такие, которые \enquote{непонятно...}. Известна лишь одна мадонна –
скульптура матери и ребенка в Луганской области.

По внешнему виду можно определить, \enquote{откуда баба пошла} - разные племена
по-разному высекали изваяния. И, видимо, вкладывали разный смысл. Редкостны
скифские бабы – по-видимому, мифические герои, мужчины с бородой. Черты их лиц
вырезаны грубо и усредненно. Все лица скифских баб, отличие от половецких
изваяний, одинаковы.

Большинство баб, что найдены в нашем крае - сделаны половцами. Тут уже
разнообразие черт, одежды и украшений. Усы, бороды, косы и сережки, ожерелья и
гривны на шее, как у мужчин, так и у женщин. Бабы-мужчины в кафтанах с поясами
и пряжками, бабы с саблями и кинжалами. Рука с чашей для жертвоприношения,
птица, сидящая на руке... Бабы с прическами и зеркальцами...

Только не забывайте, что все каменное.

Стоит баба – воин. Сидит баба – аристократ, так полагают ученые. Женщины тоже
изображались стоя и сидя. Видимо, истуканов еще и раскрашивали, чтобы сделать
лица по-настоящему живыми. Представляете чувства всадника, который в бескрайней
степи встречает священное изваяние, что смотрит в вечность живыми глазами. Он
считает истукана живым и могущественным... Точнее, мертвым – и поэтому еще более
могущественным...

Шли века, менялись верования и религии. Кочевников сменили крестьяне, в степи
стали города, взвились новые флаги, проложили железную дорогу. Но люди в степи
продолжали бережно, с уважением и поклонением относиться к каменным стражам,
даже приносить им небольшие жертвы.

Это отношение не исчезло и теперь.

И сейчас, нет-нет, да и найдут где-нибудь фермеры в земле каменную бабу. Или
археологи – на раскопках кургана. Не так давно нашли в селе Староласпа
Тельмановского района.

Бабы в заповеднике \enquote{Каменные могилы}, как подтвердили мне Ольга Демидко и
Вячеслав Забавин, \enquote{не местные}. В заповедник они свезены со всей окрестной
степи, когда распахивали целинную степь под посевы, ровняли курганы.

По этому поводу есть предание, о правдивости которого судить не берусь. Когда
привозили баб в заповедник, их было так много, что некоторые местные жители
этим пользовались. \enquote{Свистнут} парочку истуканов – и в фундамент строящегося
дома. Говорят, что жильцов этих домов потом преследовали болезни...

Эти каменные изваяния, хоть и пришли к нам из невероятной древности, далеко не
вечны. Особенно уязвимы те, что сделаны из песчаника. Сейчас им неуютно в
задымленном воздухе, под кислотными дождями. И они уходят от нас – черты лица
стираются за 30-40 лет до неузнаваемости.

Сохранение каменных баб – отдельная и серьезная тема.

Мария Королёва. Первоначально материал опубликован на Надо.ua
