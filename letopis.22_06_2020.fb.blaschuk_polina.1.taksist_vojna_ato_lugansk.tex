% vim: keymap=russian-jcukenwin
%%beginhead 
 
%%file 22_06_2020.fb.blaschuk_polina.1.taksist_vojna_ato_lugansk
%%parent 22_06_2020
 
%%url https://www.facebook.com/polina.blashchuk/posts/2937179116335565
 
%%author 
%%author_id 
%%author_url 
 
%%tags 
%%title 
 
%%endhead 
\subsection{Їхала з цікавим таксистом - Ветеран АТО - Телефон не дала}
\label{sec:22_06_2020.fb.blaschuk_polina.1.taksist_vojna_ato_lugansk}
\Purl{https://www.facebook.com/polina.blashchuk/posts/2937179116335565}

Їхала з цікавим таксистом. Молодий хлопець. Ветеран АТО. Проукраїнський, але російськомовний, бо родом з Луганську. Мали цікаву розмову. 
- Я разочаровался в войне.
- Ну у війні важко не розчаруватися.
- И побратимы мои разочаровались. Мы думали, что это что-то изменит, а вернулись, я пошел в полицию работать - все по-прежнему. Вот уволился недавно.
- Так, це правда, зміни швидко не робляться, є багато ще над чим працювати. 
- Еще и Зеленский... Видели, что творит?
- А ви від нього щось іншого очікували?
- Честно, ожидал. Я за него хотел голосовать, но не смог: открепительный лист не взял. 
- Вибачте, при всій повазі, але ви не надто добре аналізували Зеленського, зразу було зрозуміло, що він не здатен. 
- Так а кого было выбирать? Порошенко же ничего не сделал!
Я мовчу. Тримаю паузу. Потім повторюю його ж слова:
- Нічого не зробив?
- Хотя да... Армию поднял... Это да... Мы же босые, даже без оружия уходили, потом наладилось. И коалицию с Европой создал против России...
- Томос, безвіз, закон про мову, - додаю я, - у нього безліч провтиків, але з того, що ми мали на вибір, - це був кращий варіант, погоджуєтесь?
- Ну если подумать, то выходит так... Вот только закон про мову я не поддерживаю!
- Чому?
- Мне нравится украинский, я умею на нем говорить, но когда навязывают, заставляют - это бесит, понимаете, у меня протест!
- Просто цікаво, а коли і хто вам нав‘язував мову? Я з вами говорю українською, але хоч слово сказала про вашу російську? Може, колись вас відмовились обслуговувати в кафе через російську? Чи як ще змушували?
У хлопця на обличчі гамма емоцій, йде активне перебирання думок і спогадів. І тут раптом він вперше усвідомлює, що ніхто ніколи не змушував його говорити українською:
- Та нет... не было такого никогда... Но... Но ведь закон!
- За законом ви в житті маєте право говорити тою мовою, яка вам зручна. І продавці мають право переходити на російську, якщо покупець просить. Лише в держустановах треба українською. Ви ж погоджуєтесь з цим?
- Ну да, это правильно...
- Тоді виходить, що ніхто ніколи вам не нав‘язував українську, але у вас стійка думка, що це відбувається. Як гадаєте, звідки це у вас?
- Не знаю...
- А ви знаєте, що гасло «они обижают и ущемляют русскоязычных» - це російська пропаганда?
- Да, точно, российская пропаганда... Ээээ... Так я что, попался на российскую пропаганду?
- Виходить, так.
- Я никогда об этом так не думал... А вы всегда говорили на украинском?
Решту дороги я давала йому майстер-клас, як перейти на українську. Обговорювали серіали з українським перекладом - зацікавився, сказав, спробує дивитись. 
На прощання хлопець попросив у мене телефон (убер зараз не показує телефони пасажирів). А я що? Щиро подякувала за службу в АТО, а телефон не дала. Бо при всій повазі до ветеранів, голосував-то він за Зеленського))

