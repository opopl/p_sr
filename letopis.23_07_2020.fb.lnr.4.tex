% vim: keymap=russian-jcukenwin
%%beginhead 
 
%%file 23_07_2020.fb.lnr.4
%%parent 23_07_2020
 
%%endhead 
\subsection{Наступление в Донбассе официально запрещено}
\label{sec:23_07_2020.fb.lnr.4}
\url{https://www.facebook.com/groups/LNRGUMO/permalink/2869413616503527/}

\vspace{0.5cm}
{\small\LaTeX section: \verb|23_07_2020.fb.lnr.4| project: \verb|letopis| rootid: \verb|p_saintrussia|}
\vspace{0.5cm}

МИД ДНР опубликовал перечень дополнительных мер по усилению и контролю
действующего бессрочного режима прекращения огня, передаёт корреспондент
"Донбасс Сегодня".

Решение о принятии данных мер было принято представителями республик Донбасса и
Украины в ходе сегодняшних переговоров группы по безопасности в рамках ТКГ:

- Запрет на наступательные и разведывательно-диверсионные действия, а также
запрет на использование любых видов летательных аппаратов сторон;

- Запрет на применение огня, включая снайперский;

- Запрет на размещение тяжелого вооружения в населенных пунктах и их
окрестностях, в первую очередь на объектах гражданской инфраструктуры, включая
школы, детские сады, больницы и помещения, открытые для общественности;

- Эффективное применение дисциплинарных мер за нарушение режима прекращения
огня и уведомление о них координатора ТКГ, который уведомляет об этом всех ее
участников;

- Создание и задействование координационного механизма по реагированию на
нарушения режима прекращения огня при содействии СЦКК в действующем составе;

Ответный огонь в случае наступательных действий* допускается только в том
случае, если он открыт по приказу соответствующего руководства Вооруженных сил
Украины и руководства вооруженных формирований ОРДЛО после безуспешной попытки
использовать вышеупомянутый координационный механизм. Об издании таких приказов
будет уведомлена ТКГ;

Вышеупомянутые меры не могут быть полностью или частично дезавуированы никакими
другими приказами, в том числе секретными.

Отмечено, что под наступательными действиями подразумеваются какие-либо попытки
изменить позиционное размещение войск, которое сложилось на момент принятия
данного заявления, включая дополнительное инженерное оборудование позиций,
и/или любое продвижение или перемещение вооруженных лиц в сторону противника, в
том числе в целях проведения разведывательных или диверсионных действий.

