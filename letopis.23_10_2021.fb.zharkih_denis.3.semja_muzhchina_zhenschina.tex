% vim: keymap=russian-jcukenwin
%%beginhead 
 
%%file 23_10_2021.fb.zharkih_denis.3.semja_muzhchina_zhenschina
%%parent 23_10_2021
 
%%url https://www.facebook.com/permalink.php?story_fbid=3105600929653303&id=100006102787780
 
%%author_id zharkih_denis
%%date 
 
%%tags chelovek,muzhchina,obschestvo,semja,woman
%%title Отношения мужчины и женщины в семье
 
%%endhead 
 
\subsection{Отношения мужчины и женщины в семье}
\label{sec:23_10_2021.fb.zharkih_denis.3.semja_muzhchina_zhenschina}
 
\Purl{https://www.facebook.com/permalink.php?story_fbid=3105600929653303&id=100006102787780}
\ifcmt
 author_begin
   author_id zharkih_denis
 author_end
\fi

Мне всегда противно слушать, когда мужик, а тем более политик, говорит, что
основное назначение мужчины добывать еду, мясо, которое он приносит своей
женщине в виде добычи. А она, женщина, должна ждать его в пещере с этим самым
мясом и не высовываться, поскольку он ее царь и бог. И вообще, женщина выбирает
мужика по количеству мяса, которое он может ей принести. 

Ну, во-первых, стоит посмотреть на того, кто это говорит. Если отпустить такого
мачо в джунгли/тайгу/лес без автоматической винтовки с оптическим прицелом, а с
ножом или лучше с дубиной , то он не только никакого мяса не принесет, но и,
скорей всего, сам этим мясом для хищников станет.Хищники они что, они понтов не
понимают. И почему-то их не пугает ни бритая голова с двойным подбородком, ни
большой живот, ни пропитый голос. Наоборот, все это вызывает у хищников
аппетит. Не обязательно такие люди толсты, бывают и весьма мелкие особи, но
сверкающие глаза и пропитый голос для них, скорее, правило. 

Во-вторых, всех женщин они принимают за таких же примитивных существ, как они
сами. Нет, есть конечно особи женского пола, которые оценивают мужчин
исключительно по горам мяса, пачкам денег и понтам. Но даже эти особи смотрят
на таких вот мачо, как на жертв, а вовсе не, как на господ. И оторвав от этих
субъектов свою долю мяса, они с удовольствием делятся им  с менее примитивными
мужчинами или, по крайней мере, не такими заносчивыми. 

Что уж говорить о нормальных женщинах, которые не склонны отдавать себя всю за
гору мяса. Назначение мужчины вовсе не мамонтов убивать, эти-то назывались
неандертальцами и вымерли весьма давно. А быть отцом, воином, защитником. Но
все-таки важнее всего первое. У нас, как в повседневной жизни, так и в политике
не хватает именно отцов. Отцов-основателей, отцов-командиров, отцов нации.
Сплошные людоеды и неандертальцы. Естественно хорошо откормленные и
неприспособленные к нормальной жизни. 

В своей жизни я видел всего одного человека, которому такое вот неандертальское
поведение было органично. Увидел я его недалеко от детской площадки, где гулял
с маленькой Катькой. Площадка находилась возле леса. И тут из этого леса
появляется человек-не человек, волк-не волк, а какой-то снежный человек,
одетый, как браток. Вот, чтобы вы поняли, не гопник, и именно браток высокой
пробы, с какой-то лихостью и дьявольским эстетизмом. Меня поразили его глаза,
это были глаза дикого зверя, хозяина джунглей/тайги/леса. Я служил на казанских
зонах срочную, видел всякое, но это был ужас и красота в одном лице. Более
того, в нем не было ничего блатного, криминального, а именно что-то
неандертальское и опасное, звериное и жестокое. Блатные, они как бы жизнью
обиженные, мстящие за свое унижение,там либо масса комплексов, либо желания
задавить собой, осознание своей силы. Но именно осознание, что вот ты пахан, в
авторитете, не просто так тут. А тут нет никакого осознания - одна дикая
животная энергия. При этом нет никакой рисовки, позы, демонстрации - так
выходит тигр из тайги или лев из саванны. 

Он отыскал среди мамаш свою женщину с ребенком и взглядом, не проронив ни
слова, заставил ее быстро собираться и идти домой. Также взглядом он оценил
всех других молодых мамаш. Это был реальный взгляд альфа-самца, тут не было
никакого заигрывания и куртуазности. Он неторопливо ощупал взглядом каждую
молодую женщину,если нравилась, его ноздри раздувались, а желваки начинали
играть. После примерил каждого мужчину на предмет крепости. Когда он осматривал
меня, то у меня было ощущение, что слегка толкнули кулаками в грудь, проверить
не упаду ли. При этом мне доводилось смотреть в глаза убийцам и даже известному
казанскому людоеду, но таких ощущений я не испытывал. Пожалуй что-то похожее я
испытал, когда в лесу на меня вышло два волка. Я не знал, что они могут быть
такими большими и вести себя совершенно не как собаки. Вот также один волк, тот
что был побольше и поближе ко мне, внимательно меня осмотрел, на предмет "Что
ты тут делаешь в моем лесу?", а потом решил не связываться, и пошел своей
дорогой, все же явно дав понять, кто тут хозяин. 

Они удалялись, он, женщина и ребенок. В этом было что-то первобытное. Он шел
отдельно от нее, скорее, конвоировал, а на ребенка не обращал совершенно
никакого внимания. Тут он все же бросил взгляд на малыша, сидевшего в коляске,
и его глаза слегка увлажнились, при этом ни один мускул на лице не дрогнул. Он
даже не прищурился, не моргнул, и уж, тем более, не улыбнулся. Но по лицу его
пробежала какая-то первобытная нежность, которая тут же исчезла.

Когда он удалился я спросил, кто это. Мне сказали, что это владелец нескольких
ларьков пришел за женой. Они недавно сюда приехали. Дело было в середине 90-х.
Прошло немного времени, и я узнал, что этого странного и яркого человека до
полусмерти избили рэкетиры. Он остался инвалидом на всю жизнь. Больше я его не
видел. Вот у меня есть стойкое убеждение, что только этот человек мог реально
жить по пещерным законам, только такой выжил бы в тайге или среди людоедов. Но
он не бил себя в грудь, не выпендривался, и вообще редко что-то говорил. Он был
реально зверем, его и убивали, как зверя, но он выжил. Жена осталась с ним.
Забрала его куда-то в село, там они и жили. Она не была зверем. Просто любящей
женщиной.

2017

\ii{23_10_2021.fb.zharkih_denis.3.semja_muzhchina_zhenschina.cmt}
