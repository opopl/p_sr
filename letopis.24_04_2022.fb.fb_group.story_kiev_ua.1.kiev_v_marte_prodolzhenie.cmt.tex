% vim: keymap=russian-jcukenwin
%%beginhead 
 
%%file 24_04_2022.fb.fb_group.story_kiev_ua.1.kiev_v_marte_prodolzhenie.cmt
%%parent 24_04_2022.fb.fb_group.story_kiev_ua.1.kiev_v_marte_prodolzhenie
 
%%url 
 
%%author_id 
%%date 
 
%%tags 
%%title 
 
%%endhead 
\zzSecCmt

\begin{itemize} % {
\iusr{Tatyana Katunina}

Как знакомо. Моя запоребриковая родня описана один в один. И про нацистов, и
про биолаборатории

\begin{itemize} % {
\iusr{Наталья Денисова}
\textbf{Татьяна Катунина}

Совершенно верно!

Когда я попросила рассказать мне подробнее про биолаборотории (может кто-то
работал там или рядом? ) то мне ответили просто и незатейливо: лечите голову...
Даже придумать что то фантазии не хватает....

\iusr{Tatyana Katunina}
\textbf{Наталья Денисова}, а я просто загуглила \enquote{биолаборатории россия} и сбросила им рекламные ссылки по россии

\iusr{Александр Алексеенко}
\textbf{Татьяна Катунина} телевизор у них один на всех...
\end{itemize} % }

\iusr{Наталья Наталья}
Подписываюсь под каждым словом. Все так и есть

\iusr{Андрей Кошевой}
Двое суток, по 2 часа стояли на заправках, что бы получить заветные 20 литров бензина

\iusr{Олена Чепега}

Ой, как будто мои чувства и мысли описали @igg{fbicon.face.pensive}. Все точно так, и про новости, и
про общение. Какие-то фантастические мысли приходят в голову, что им вместе со
Спутником ещё что-то вкололи. Они все повторяют одно и тоже. Даже моя родная
дочь и зять в Крыму. Ну, что ж, мы все переживём, а их прозрение когда нибудь
будет страшным

\begin{itemize} % {
\iusr{Наталья Денисова}
\textbf{Олена Чепега}

Вам тоже приходила в голову эта странная идея про Спутник?

Я ничем не могла объяснить почему они несут одну и ту же чушь. И разница одна-у
вас дочь с зятем, а у нас родители.

И все говорят одно и то же....

Как под копирку....

\iusr{Олена Чепега}
\textbf{Наталья Денисова} Да, иногда можно поверить в самые нереальные вещи. Хотя, может по современным технологиям, они уже стали реальностью

\iusr{Tatyana Katunina}
\textbf{Олена Чепега}, 

если им хотя бы на время отключить телевизор- становятся более вменяемыми (мне
так кажется). Поэтому первыми должны судить Скабееву, Эрнста и ко

\iusr{Ира Титова}
Наталья Денисова Всё-таки, это телевизор! Соловьевы, Скабеевы и прочий сброд долго промывал им мозги!

\iusr{Ира Титова}

Так близко и понятно,, как-будто про меня написали! Только, у меня ещё
развилась ,,паранойя,, ,уходя из квартиры, я не закрывала дверь! Было
страшно, если вдруг, что со мной произойдет, как моя любимая кошка будет в
запертой квартире?

\end{itemize} % }

\iusr{Валентина Барановская}

Всё так и есть. И мы более 12 часов ехали заветных чуть больше трёхсот км в
известном направлении в начале марта. И в остальном также.

\iusr{Valery Le-ra}

И кстати, многие в первые дни рванули на свои дачи под Киевом, что оказалось
для некоторых роковой ошибкой

\begin{itemize} % {
\iusr{Анна Войнова}
\textbf{Valery Le-ra} , это так. Наша дача под Бородянкой. Еле оттуда выбрались(

\iusr{Valery Le-ra}
\textbf{Анна Войнова} мы тоже, Богдановка, Броварской район

\iusr{Анна Войнова}
\textbf{Valery Le-ra}, думали, едем в безопасное место,подальше от города...

\iusr{Valery Le-ra}
\textbf{Анна Войнова} аналогично, у нас во дворе в Киеве упал кусок сбитой ракеты, у полдома окна повыбивало

\iusr{Галина Калинина}
\textbf{Анна Войнова} мои знакомые - тоже...

\iusr{Наталия Фурто}
\textbf{Valery Le-ra} Так і є. Ми залишилися у Києві, а наші родичі поїхали на дачу по Житомирській трасі. Натерпілися там. Та й ми за них хвилюватися, бо зв'язку не було. Але, дякувати Богові, вони вже вдома.

\iusr{Valery Le-ra}
\textbf{Наталия Фурто} у наших родичів дом біля Житомирської траси, вони там були в окупації з дітьми, які з Ірпіня, діти змогли виїхати ризикуючи життям, бо невістка була вагітна, а так жах, що вони розповідали, суцільний жах, але живі
\end{itemize} % }

\iusr{Tatyana Pavlovskaya}

Под копирку написанная наша история! Нас больше донимали ни бомбежки, ни страх
войны, а эвакуированные наши дети, которе истерили и в 4 утра, и в 12 ночи по
3 раза в день несколько недель подряд

\iusr{Vlad Volik}

Такая же история про родственников. Когда мне мой двоюродный брат, живущий на
дальнем востоке, заявил, что Украина собирается создать мол ядерное оружие и,
что он верит Путину... Я понял, что все.... дальше общаться смысла я не вижу!

\iusr{Виктория Ушакова}

Я безмерно благодарна моим детям за то, что приняли наше решение остаться.
Догадываюсь по некоторым моментам, что у них кошки на душе скребут до сих пор.
У меня свой домашний кошмар, связанный с отъездом. В первый день мы ещё кое-как
собирались в дорогу. Окончательный разговор, что мы остаёмся, был утром второго
дня. Рюкзаки, приготовленные в дорогу, три кошачьи переноски, мешочек с кормом
так и стоят в углу комнаты. Один только взгляд на этот склад сразу обрушивает и
без того неважное настроение. Пришлось затянуть все пледом и ещё сверху вещей
набросать(( а родня из той страны да-вся взяла и закончилась. Спасибо, что
молча.

\iusr{Елена Томашевская}
Аналогично  @igg{fbicon.frown} 

\iusr{Наталья Денисова}

Спасибо, Александр!

Как то спокойнее становится на душе, когда узнаешь, что ещё и другие люди
переживают те же чувства!

Могу подписаться под каждым вашим словом!

\iusr{Valery Le-ra}
Кто то бы нам рассказал год назад, какие здесь будут Киевские истории весной 2022 года

\iusr{Полина Полякова}

Я ещё пытаюсь толерантно и корректно отвечать своим одноклассникам из Литвы.
Но, с каждым днём понимаю, мы чужие люди. О какой дружбе идёт речь?

\begin{itemize} % {
\iusr{Chris Smith}
\textbf{Полина Полякова} Always remember that there is a big difference between friends and acquaintances.
Glory to Ukraine
Glory to our heroes.
@igg{fbicon.flag.ukraina}{repeat=2}@igg{fbicon.heart.red}{repeat=2}
\end{itemize} % }

\iusr{Irina Bonatska}
а я так же приняла решение остаться в Киеве

\iusr{Polina Verbytska}

С детьми, не маленькими, уточняю @igg{fbicon.face.smiling.eyes.smiling}  двумя кошками и собакой приняли решение не
уезжать. Полагая, что Киев будет воевать и защищаться до последнего. Таки есть.
 @igg{fbicon.face.happy.two.hands} продолжаем жить и радуемся изменяющемся ежедневно городу.  @igg{fbicon.hands.pray}  Любим тебя,
Киев! @igg{fbicon.heart.sparkling} 

\iusr{Inna Konyukhova}

Я благодарна судьбе что прожила 20 лет в любимой многострадальной Украине. Шлюб
развалился... Уехала к маме, больной и никому ненужной, в РФ... Здесь уже 10 лет. С
закрытым ртом и кипящими мозгами. И с периодически накатывающем отчаянным
желанием наложить на себя руки...

Украинцы, дорогие, победите! Христа ради! Не дайте этой сволочи лезть дальше с
их треклятым \enquote{русским миром}. Я верю, все буде Украiна! @igg{fbicon.flag.ukraina}

\begin{itemize} % {
\iusr{Наталья Денисова}
\textbf{Inna Konyukhova} спасибо, родненькая!

\iusr{Aleksandra Rudnitskaya}
\textbf{Inna Konyukhova} ООО! Вас не съела телевизионная пропаганда. Уважаю!

\iusr{Aleksandra Rudnitskaya}

Тезка! Спасибо за ваши рассказы. Повторюсь, но история как моя - под копирку.
Смущалась, когда знакомые ужасались, что не иду в убежище. Мой аргумент: не
хочу быть засыпанной в подвале воспринимался как -то ...не очень. Смущалась,
что не реагирую на отчаянные призывы Кличко. Оказалась, нас таких много.
И, думаю, все люди в возрасте, выпустившие детей. Да и хвостик мой любимый
жалко.

\iusr{Наталья Денисова}
\textbf{Александра Рудницкая}

У нас с мужем то же.

Он наотрез отказался спускаться в убежище. Мы так и остаёмся. Только подстилку
собаки отодвинули подальше от окон. А как психика приспособилась! При первых
сигналах воздушной тревоги я начинаю засыпать....

Уж выспалась...
\end{itemize} % }

\iusr{Виктория Лунная}

У мене батько росіянин, точніше народився на кордоні Білорусії і Росії. І не
один з моїх родичів з російської сторони не підтримає путлера! І в каталажці
були за свої погляди, і штрафи на них накладали. Але ні разу, ні одної миті
вони не казали що путлер герой і правий! Я не відмовлюсь від своїх рідних.

\begin{itemize} % {
\iusr{Наталья Денисова}
\textbf{Виктория Лунная}

Як це втішає!

Невже ж таки і там є не зомбовані люди. Бережіть їх, бо вони те тільки ваші
рідні, вони унікальні.

Може вони зможуть стати тою \enquote{вакциною} що врятує хоча б наших родичів в росіі.

\iusr{Marina Morneva}
\textbf{Виктория Лунная} вам повезло

\iusr{Katrya Bevzuk}
\textbf{Виктория Лунная} 

якже вам поталанило! У нас не так. Рідня чоловіка в Росії повністю за путлєра.
Моя сестра (українка) проживає 45 років в Латвії, теж за рашу і путлєра.

\end{itemize} % }

\iusr{Anastasia Luchkova}

Как все грамотно и детально и правдиво!! Хочется этот текст вставить в голову
этим зомби!

\iusr{Александр Алексеенко}

Пытался. Зомби они и есть зомби. Их даже смерть близких не останавливает.
Просто мы все пишем историю. А что написано пером, то...

\iusr{Катерина Парфьонова}

Я добре пригадую, як в перший і другий день оці біженці скуповували продукти)))

Пост тупо ні про що

\begin{itemize} % {
\iusr{Valery Le-ra}
\textbf{Катерина Парфьонова} таке враження, що ви читати не вмієте

\iusr{Катерина Парфьонова}
\textbf{Valery Le-ra} навпаки. Вмію.
І оце перекручування реальності вражає
Ми казали вісім років - не користуйтесь салютами...
Що чули у відповідь?
А тепер страшно...

\iusr{Наталия Фурто}
\textbf{Катерина Парфьонова} Люди були налякані, що тут смішного? У моїх знайомих 8 річна дитина бачила, як ракета влетіла у будинок його друга, і друг з мамою потрапив у лікарню. У других знайомих дитина 9 років до сих пір щодня приходить до бомбосховища і сидить там увесь день у шоковому стані. Це у Києві. Ви вважаєте це смішно?

\begin{itemize} % {
\iusr{Катерина Парфьонова}
\textbf{Наталия Фурто} ми в Києві.
..
Війна триває вже вісім років.
Раптом налякались?

\iusr{Наталья Денисова}
\textbf{Наталия Фурто}

Наталія, не хвилюйтесь, то така провокація ( може й не свідома, може такий
характер у людини). В мене онук отримав психологічну травму після обстрелів.
Батьки і не відразу помітили. Дитина не кричит, не плаче.... Тільки любов'ю і
відходили.

Кому це може бути смішним???

\iusr{Ира Титова}
\textbf{Катерина Парфьонова} 

Так и не читали бы! Пост не для вас и вам его не понять! Можно подумать, что Киев,
за последние лет 70, кто-то обстреливал! И детей наших, нужно было
,обязательно, всех вывозить из города! Тем более, никто не знал, куда попадет
следующая ракета или это будет авиабомба, не Дай Бог, потому что и сейчас
возвращаться, в Киев, рано!

\iusr{Катерина Парфьонова}
\textbf{Ира Титова} то не вертайтесь.
Чи ви не пригадуєте події 2013-2014?
Але ж в телевізорі постріли не такі страшні, правда?))

\iusr{Александра Пучкова}
\textbf{Катерина Парфьонова} Ви, як та змія сичите спокійно і вкрадливо...... Залиште цей тон для своїх.

\iusr{Valery Le-ra}
\textbf{Катерина Парфьонова} 

ви в цій групі дуже нагадуєте мені росіян, які теж аргументують, що ми на це
заслуговуємо, бо Донбас 8 років бомбили. Так і ви тут намагаєтесь донести, що
війна вже 8 років йде, а ми її не помічали. Звідки така впевненість? У кожного
своя історія і своя біль

\end{itemize} % }

\end{itemize} % }

\iusr{Олена Андурова}

После первых дней налетов, которые невестка просидела с малышами в подвале, за
два часа собрали детей с маленькими внуками (2 и 4 года),посадили в свою
машину и отправили в Карпаты. Ехали через Винницу почти сутки. На Житомирской
трассе их друзья, уехавшие почти сразу, до Житомира ехали 16 часов. Уехавших позже
расстреливали на трассе. Сами остались. Возле дома детей была попытка порыва по
поспекту Победы. Сильно стреляли. Тоже радовалась, что успели их отправить. В
подвалы не спускались. Решили, что лучше (если можно так сказать) погибнуть на
воздухе, чем медленно умирать в заваленном подвале. Первый месяц ходили везде
втроем. Муж, я и собака. Боялась оставаться дома без мужа (вдруг с ним что-то
случится, а меня рядом нет)или оставлять собаку дома. А вдруг мы погибнем, а
собака мучительно умрет одна от голода и жажды в запертой квартире. О себе не
думала. В парадном из 28 квартир осталось три, где кто-то жил. Оставшиеся
делились продуктами с очень старой одинокой соседкой. Сообщали друг другу, где
появлялся хлеб или яйца. В центре подвоза первое время никакого не было. В аптеку
простояла более 4 часов. Делились медикаментами. С запоребриковыми знакомыми и
родней та же история. Троюродная сестра исчезла из сети и даже не
поинтересовалась, как мы. Одногруппница, доцент технического ВУЗа в
Екатеринбурге, еще лет 5 назад, бывшая адекватной, звонила и просила приехать к
ней, чтобы спастись от нацистов. Никакие доводы не действовали.

\iusr{Мария Бойчук}

Так! Світ навиворіт! Марна справа спілкуватися з хворими людьми. Їх треба
лікувати... А ліки лише одні: знищити московську орду! І тільки разом з усім
світом ми можемо припинити існування імперії зла. Тож молимося за нашу
Перемогу, за наших Захисників і за Президента!

\iusr{Наталья Денисова}

Ещё раз хочу поблагодарить Вас, Александр.!

И жду продолжения...

Простое описание событий, изложение своих мыслей без лозунгов и истерик...
Такое общение помогает нам сберечь душевное здоровье во время этой безумной
войны.

Это так тяжело...

И как я рада, что нас много!

Единомышленников.

\iusr{Julia Panchul}
зато дороги Зе построил для путинских танков1

\begin{itemize} % {
\iusr{Ирина Иванченко}
\textbf{Julia Panchul} ,пані хвора на печінку або на голову?

\iusr{Valery Le-ra}
\textbf{Ирина Иванченко} 

а ви лікар, що діагнозами розкидаєтесь? І це правда, замість того щоб
фінансувати армію і ракетні програми, гроші розкрадались на великих
крадівництвах, бо була політика перестати стріляти, розводити війська і
подивитись в очі путіну. Тепер вся країна туди дивиться

\iusr{Lyuba Mukutska}
\textbf{Valery Le-ra} 

адміни таке люблять. бо мирумир. а Ви розпалюєте ворожнечу. Бубочка і КО святі й
безгрішні. Цікаво скільки часу знадобицься. щоб закрити коменти незгодним.

\iusr{Valery Le-ra}
\textbf{Люба Микицька} якщо люди не хочуть бачити причини та їх наслідки, і далі голову в пісок хавають
\end{itemize} % }

\iusr{Катерина Базаренко}
Во второй ехали 12 часов, вместо привычных 6.

\iusr{Оксана Головань}

Да, мы выехали в 8:40. Расстояние в 486 км (от дома до дома), которое в
нормальных условиях можно было проехать за 6-6,5 часов, мы преодолевали 10
часов. В Гостомеле стали в очередь на заправку, стояли 45 минут, но успели
залиться до полного. Это были самые первые часы, когда ещё было топливо и
наливали столько, сколько нужно было. На заправке решили перекусить, поскольку
с собой еду не брали, а дорога дальняя. Муж предложил пожевать на парковке,
чтобы не на ходу. Но я, зная о том, что рядом военный аэропорт, сказала, что
это не лучшее место для стоянки, уж лучше поесть в пути. Через 20 минут после
того, как мы выехали из Гостомеля, нам позвонили друзья и сообщили, что
Гостомель бомбят. Тогда я поняла, что мы все сделали правильно....

В плане взаимодействия с родственниками у нас абсолютно зеркальная ситуация.
Пожилые бабушка и дедушка мужа, которые за последние 10-12 лет с Печерска
никуда не выезжают, регулярно нам звонили и говорили: \enquote{Не занимайтесь
фигнёй и едьте домой, тут тихо}.

Ровно через месяц после начала, наш район подвергся плотному огню градов. И в
очередной раз я поняла, что мы все сделали правильно!

Сейчас мы находимся в условно безопасном месте, в Киев возвращаться не
планируем..... Пришли к выводу, что родину будет приятнее любить из-за границы.
Мы просто хотим, чтобы наши дети жили в безопасности. Дай Бог дождаться того
момента, когда границы откроют и мы сможем уехать.....

\iusr{Nata Nata}

Вы пишете про дом в Киеве, из которого ракета вырвала кусок и «как он устоял и
не сложился?» это потому что он монолитно-каркасный. Самая прочная технология.
Складываются панельки (дай Бог больше ни одной не сложиться  @igg{fbicon.hands.pray} ). Я раньше
вычитывала, какой дом прочнее при землетрясении. Получилось похожая
устойчивость домов что к землетрясению, что к частичному вырыванию дома прости
Господи

\iusr{Nata Nata}

Я уехала в первый час. И уже в первый час было полно машин на дорогах. Только я
не на Львов. Я южнее уехала. И когда у меня пошла пустая дорога в полях я
подумала: вот тут мир, вот тут ничего не может быть. Потому что я мыслила, что
рашистам интересны Киев, Донбасс. Но это не так... та местность в полях, слава
Богу, осталась в порядке. Но вот правда, я ехала среди полей и думала: тут их
ничего не интересует, тут благодать, тут мир, который сегодня есть уже не везде

\iusr{Людмила Васьковська}
Как все верно! Это наши мысли и суждения. Спасибо.

\iusr{Olga Pikalova}
Киев?(((


\iusr{Елена Медун}

Мы коренные киевляне и не только коренные ... Все в одной лодке... Александр
всё под копирку ...моя история только я подалась панике и уехала...

\iusr{Инна Кравченко}

Жители \enquote{великой}, уверены что Украина без трусов ходит и вооружается, им пахать
не надо, вот и пришли воровать, природные ископаемые вывозить, пусть сами в
сибирь свою едут икру и кедровые орешки кушать - витаминов в мозгу (если он есть)
не хватает, пусть лечатся, Украина россиянам не здравница. Слава Украине @igg{fbicon.heart.growing} 

\iusr{Ольга Юровских}

Я сразу приняла решение - не ехать! Досери говорю - бери детей и в путь! Но мы
остались все. Тревожный рюкзак собрали (я только свои документы туда положила),
в общем коридоре стулья поставили, плед положили. Но спали в своих кроватях.
Поначалу было жутко (особенно когда начали ракеты мимо пролетать), а потом
привыкли. Квартира угловая, дочка то дело бегала от окна к окну. Брат из
Севастополя - мы Вас освободим! От чего? Говорю по-русски, кто меня унижает?
Дала отпор больше звонков нет. Но я верю, что Киев не сдадут и буду вс Городе
ждать нашу Победу.

\iusr{Любовь Королькова}
Такое впечатление, что вытащили мои мысли и ощущения и опубликовали...Один в один... @igg{fbicon.smile} 

\iusr{Iryna Domashova}
Дякую за чудову і відверту розповідь. Все так знайоме.....

\iusr{Irina Svistunenko}
Я тоже в Николаеве так живу. Чтоб заснуть, надо закинуться валерьянкой...

\end{itemize} % }
