% vim: keymap=russian-jcukenwin
%%beginhead 
 
%%file 22_06_2018.stz.news.ua.mrpl_city.1.mariupol_22_iunja_1941_goda
%%parent 22_06_2018
 
%%url https://mrpl.city/blogs/view/mariupol-22-iyunya-1941-goda
 
%%author_id burov_sergij.mariupol,news.ua.mrpl_city
%%date 
 
%%tags 
%%title Мариуполь: 22 июня 1941 года
 
%%endhead 
 
\subsection{Мариуполь: 22 июня 1941 года}
\label{sec:22_06_2018.stz.news.ua.mrpl_city.1.mariupol_22_iunja_1941_goda}
 
\Purl{https://mrpl.city/blogs/view/mariupol-22-iyunya-1941-goda}
\ifcmt
 author_begin
   author_id burov_sergij.mariupol,news.ua.mrpl_city
 author_end
\fi

\ii{22_06_2018.stz.news.ua.mrpl_city.1.mariupol_22_iunja_1941_goda.pic.1}

В памяти четырехлетнего мальчонки на всю жизнь закрепилась картина. Он
раскатывает на трехколесном велосипеде по длинной веранде. Папа в белой
рубашке-сеточке склонился над чертежной доской. Рейсшина время от времени
шуршит по большому листу бумаги. Мама принесла стеклянную кринку с молоком,
поставила на подоконник. Пахнет клубничным вареньем. Это бабушка варит его в
летней кухне. В углу под самым потолком примостилась черная тарелка
репродуктора. На несколько мгновений она умолкла, а затем из ее недр четким
голосом было произнесено несколько фраз. Взрослые притихли, вслушиваясь в
звуки, издаваемые репродуктором. Потом кто-то другой долго говорил. Этот кто-то
еще продолжал, а женщины начали плакать. Мужчины перебрасывались время от
времени нескoлькими фразами. Понятым осталось лишь одно слово – \enquote{война}...

Как-то в преддверии очередной даты начала самой кровопролитной войны в истории
нашей страны пришла мысль спросить у знакомых старожилов Мариуполя, как им
запомнился страшный день – воскресенье 22 июня 1941 года. \textbf{Грета}, ей было шесть
лет: \enquote{Я играла во дворе. Вдруг открылось окно и папа крикнул: \emph{\enquote{Война началась.
Иди в дом}. Но я как-то не придала значения этому и продолжала беззаботно
бегать}}. 

\textbf{Валентине}, только что окончившей первый класс, события этого дня хорошо
закрепились в памяти главным образом потому, что через два дня должны были
отмечать ее именины и именно 22 июня были отменены. Папа, начальник одного из
цехов \enquote{Азовстали}, был на работе. Из репродуктора сказали нечто такое, что мама
сразу заволновалась, забрала ее и двухлетнего братика - и все трое побежали к
соседям. Там они сидели до самой ночи, пока за ними не пришел вернувшийся с
завода папа.

Ровесник Валентины Георгий рассказал следующее: \emph{\enquote{За несколько месяцев до начала
войны отец купил радиоприемник СВД-9. Его ручки-махо\hyp{}вички и особенно зеленый
глазок на передней панели притягивали меня, как магнит. Мне позволили самому
включать приемник. Были каникулы, и я часами слушал радиопередачи. Сражения
происходили где-то в Европе, но для меня все это оставалось абстрактным.
Недавно закончилась война с Финляндией, отец с нее вернулся домой, сама она
была кратковременной. Казалось, что и эта, сейчас объявленная, скоро завершится
победой непобедимых красноармейцев}}.

\textbf{Алла}, которой исполнилось девять лет, запомнила, что в тот день у них были
гости, после сообщения по радио они о чем-то шепотом переговаривались с
родителями. И родители, и гости были обеспокоены тем, что произошло. Но ей
самой не показалось, что произошло нечто страшное.

Семья десятилетней \textbf{Эллы} жила в большом дворе с несколькими одноэтажными жилыми
строениями. Почему-то репродуктор был подвешен на дерево. Под деревом возились
дворовые дети. Вдруг как-то разом люди вышли из квартир. Они слушали молча.
Стало необъяснимо тревожно. Притихла и детвора.

\textbf{Игорю} было 22 июня 1941 года чуть больше четырнадцати лет. О нападении
фашистской Германии на Советский Союз он узнал, как и его родители, только
вечером. Просто в их квартире были выключены и репродуктор, и приемник. 

На год старше Игоря - \textbf{Анатолий}. Он занимался гимнастикой в спортивном кружке
Мариупольского дворца пионеров и октябрят. В тот роковой день он вместе с
товарищами устанавливал на эстраде Городского сада спортивные снаряды. Ребята
готовились к выступлению перед отдыхающими. В метрах тридцати от эстрады на
столбе был укреплен громкоговоритель. Вдруг возле него начал собираться народ.
Людей становилось все больше и больше. Спортсмены прислушались. Говорил
Заместитель Председателя Совнаркома и Нарком иностранных дел Молотов. Когда
стало ясно, что речь идет о начале войны, руководитель кружка Александр
Иванович вполголоса сказал: \emph{\enquote{Я погиб}}. И действительно, он был мобилизован и
вскоре был убит в одном из первых для него боев.

Дети ощутили в полной мере, что такое война несколько позже. По их судьбам она
прошлась тяжелым катком. Грету, Аллу и Эллу гитлеровцы лишили отцов. Грета
помнит, как срочно, впопыхах эвакуировали госпиталь, в котором работала ее
мама. Как их поезд, обозначенный красными крестами, по ходу следования не раз и
не два бомбили немецкие самолеты.

Валентина никогда не забудет отправления их поезда, состоящего из пяти
паровозов и четырех вагонов-теплушек, от проходных ворот завода \enquote{Азовсталь} 8
октября 1941 года, когда немецкие мотоциклисты уже подъезжали к мосту через
Кальмиус. 

Георгий познал, что это такое несколько месяцев жить в тесноте вагонов, наскоро
переоборудованных для перевозки людей, когда поезд не столько двигался по
маршруту Мариуполь – Кузнецк, сколько пропускал составы с войсками и
вооружением. Выросший в тепле южного города, он испытал суровую сибирскую зиму.

Алле, Элле, Игорю и Анатолию довелось пережить оккупацию – это был страх,
грабежи немецких и особенно румынских солдат, недоедание, бомбежки, сожжение
родного города. Из этих четырех детей больше всего досталось от войны Игорю.
Ему пришлось прятаться, чтобы не угнали в Германию. В 1944 году его,
семнадцатилетнего парнишку, только что окончившего девятый класс, призвали в
армию. В августе 1945 года он вступил в бой с японцами на полях Маньчжурии...

Вероятно, самым первым в нашем городе узнал о начале войны врач-отоларинголог
\textbf{Лев Михайлович Карасик}. Лет десять назад от его дочери Наталии Львовны довелось
узнать следующее. Лев Михайлович уже после окончания гимназии прилично владел
немецким и французским языками, последний усовершенствовал, обучаясь во Франции
в университете Нанта. А английский язык выучил уже самостоятельно. Чтобы не
потерять языковых навыков, по утрам слушал иностранные передачи. Рано утром,
включив свой приемник, он услышал страшную весть. Радио Афин сообщило на
английском языке, что фашистская Германия напала на Советский Союз...

Итак, в полдень 22 июня 1941 года прозвучало радиообращение Заместителя
Председателя Совнаркома и Наркома иностранных дел Молотова. Оно начиналось
словами: \textbf{\enquote{Советское правительство и его глава тов. Сталин поручили мне
сделать следующее заявление: сегодня, в 4 часа утра, без предъявления
каких-либо претензий к Советскому Союзу, без объявления войны, германские
войска напали на нашу страну, атаковали наши границы во многих местах и
подвергли бомбежке со своих самолетов наши города Житомир, Киев, Севастополь,
Каунас и некоторые другие, причем убито и ранено более двухсот человек. Налеты
вражеских самолетов и артиллерийский обстрел были совершены также с румынской и
финляндской территории...}}. 

А завершалось обращение так: \textbf{\em\enquote{Правительство Советского Союза выражает твердую
уверенность в том, что все население нашей страны, все рабочие, крестьяне и
интеллигенция, мужчины и женщины отнесутся с должным сознанием к своим
обязанностям, к своему труду. Весь наш народ теперь должен быть сплочен и един,
как никогда. Каждый из нас должен требовать от себя и от других дисциплины,
организованности, самоотверженности, достойной настоящего советского патриота,
чтобы обеспечить все нужды Красной Армии, флота и авиации, чтобы обеспечить
победу над врагом. Правительство призывает вас, граждане и гражданки Советского
Союза, еще теснее сплотить свои ряды вокруг нашей славной большевистской
партии, вокруг нашего Советского правительства, вокруг нашего великого вождя
тов. Сталина. Наше дело правое. Враг будет разбит. Победа будет за нами}}.

В этот же день были подписаны Указы Президиума Верховного Совета СССР: \enquote{Об
объявлении в отдельных местностях СССР военного положения} (это касалось и
Мариуполя, как города, входящего в состав Украинской ССР), \enquote{О мобилизации
военнообязанных по Ленинградскому, Прибалтийскому особому, Западному особому,
Киевскому особому, Одесскому, Харьковскому, Орловскому, Московскому,
Архангельскому, Уральскому, Сибирскому, Приволжскому, Северокавказскому и
Закавказскому военным округам} (его действие распространялось и на наш город,
поскольку он входил в Харьковский военный округ), \enquote{Об утверждении Положения о
военных трибуналах в местностях, объявленных на военном положении, и в районах
военных действий}.

К концу дня 22 июня в Мариуполе по городу были расклеены листовки с текстом:
\emph{\enquote{Приказ Мариупольского военного комиссариата. Президиум Верховного Совета Союза
Советских Социалистических Республик объявил мобилизацию Рабоче-Крестьянской
Красной Армии и Военно-Морского Флота. Первый день мобилизации - 23 июня 1941
года}}. В этом документе, в частности, было сказано, что мобилизации подлежат
военнообязанные, родившиеся с 1905 по 1918 год включительно. 

На следующий день потянулись военнообязанные мариупольцы к Городскому
военкомату. Он располагался в доме № 2 на улице ΙΙΙ Интернационала. (Сейчас ей
возвращено исконное имя – Торговая.) Их становилось все больше и больше,
сначала они заполнили небольшой бульвар, а потом и проезжую часть улицы. Среди
мобилизованных военнообязанных, кроме мужчин, были и женщины: врачи,
медицинские сестры, фармацевты, телеграфистки и телефонистки, а также
выпускницы аэроклубов – летчицы и парашютистки.

А в кабинетах директоров мариупольских заводов в присутствии парторгов были
вскрыты конверты с надписью: \enquote{План мероприятий на особый период}. С этого
момента производство на предприятиях переходило на военные рельсы. Начались
работы по светомаскировке...

\textbf{Читайте также:} 

\href{https://mrpl.city/blogs/view/istoriya-mariupol-9-maya-1945-goda}{%
История. Мариуполь: 9 мая 1945 года, Сергей Буров, mrpl.city, 08.05.2017}
