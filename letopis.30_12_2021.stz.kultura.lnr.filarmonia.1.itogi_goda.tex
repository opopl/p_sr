% vim: keymap=russian-jcukenwin
%%beginhead 
 
%%file 30_12_2021.stz.kultura.lnr.filarmonia.1.itogi_goda
%%parent 30_12_2021
 
%%url http://filarmonia64.com/?p=21902
 
%%author_id kultura.lnr.filarmonia
%%date 
 
%%tags 
%%title Луганская академическая филармония подвела итоги уходящего 2021 года
 
%%endhead 
\subsection{Луганская академическая филармония подвела итоги уходящего 2021 года}
\label{sec:30_12_2021.stz.kultura.lnr.filarmonia.1.itogi_goda}

\Purl{http://filarmonia64.com/?p=21902}
\ifcmt
 author_begin
   author_id kultura.lnr.filarmonia
 author_end
\fi

Учреждение культуры провело 318 концертов, которые посетили около 40 тысяч
зрителей. Сольные программы, литературно-музыкальные композиции для детей и
родителей, органные, джазовые и камерные концерты радовали публику на
протяжении всего года.   

\ii{30_12_2021.stz.kultura.lnr.filarmonia.1.itogi_goda.pic.1}

По традиции в сентябре академический симфонический оркестр филармонии принял
участие в Международном Крымском музыкальном фестивале. Коллектив под
управлением заслуженного деятеля искусств ЛНР Александра Щурова выступил на
одной сцене с такими известными артистами, как пианисты Михаил Воскресенский,
Татьяна Сафонова, виолончелист Глеб Степанов, скрипач Юрий Ревич. В рамках
фестиваля также выступил струнный квартет симфонического оркестра. Елена
Ганцева, Алёна Петрунина, Елена Авершина и Елена Порошистая представили
произведения С. Рахманинова и Э. Шоссона.

Несмотря на трудности, связанные с эпидемиологической обстановкой, в Республику
приезжали с гастролями солисты и дирижёры. Так, на сцену филармонии вышли
дирижёр Владимир Заводиленко (Донецк), пианист Михаил Илющенко (Донецк),
пианистка Валентина Лисица (США), скрипач Даниил Коган (Москва), пианист
Александр Ключко (Москва), органист Александр Хмарный (Донецк), а также артисты
из Сибири — мультиинструменталист Пётр Ткаченко и вокалист Алексей Сухов.

Не менее насыщенной была конкурсная деятельность. Артисты филармонии приняли
участие в более чем 20 очных и заочных творческих состязаниях, которые
проходили в Пекине, Токио, Пизе, Вене, Грозном, Кизилюрте, Москве,
Санкт-Петербурге, Ростове-на-Дону, Туле. Кроме того, заслуженный артист
Ингушетии, заслуженный артист ЛНР Руслан Буханцев, заслуженный артист ЛНР
Сергей Чуйков и дирижёр Александр Щуров становились членами жюри различных
конкурсов.

В уходящем году была расширена география акции по культурному обслуживанию
жителей сельской местности «Искусство без границ», которая проходит при
поддержке Министерства культуры, спорта и молодёжи Луганской Народной
Республики. Солисты и коллективы побывали в населенных пунктах
Славяносербского, Антрацитовского, Лутугинского и Краснодонского районов
Республики.

Филармония также продолжила реализацию культурологического проекта «\#Искусство
дома», инициированного Министерством культуры, спорта и молодёжи и
Министерством связи и массовых коммуникаций ЛНР. Были созданы телевизионные
версии концертных программ академического симфонического оркестра, ансамбля
эстрадной и джазовой музыки «Сombo», ансамбля песни и танца «Раздолье», а также
вокалистов музыкального лектория.

Финальным аккордом года стали новогодние музыкальные представления для детей и
родителей, покорившие как юных, так и взрослых зрителей, а коллектив филармонии
получил благодарность учреждений, которые посмотрели сказку. Так, администрация
общеобразовательной школы №1 имени профессора Льва Михайловича Лоповка в
благодарственном письме отметила яркость и содержательность действа.

Впереди — новые планы и новые программы. Луганская академическая филармонии
ждет своих зрителей уже в новом году!
