% vim: keymap=russian-jcukenwin
%%beginhead 
 
%%file 19_01_2022.yz.italomania_i_gastromania.1.etruski_russkie
%%parent 19_01_2022
 
%%url https://zen.yandex.ru/media/italy/etrusski-eto-russkie-ili-net-poprobuem-razobratsia-alfavit-tochno-pohoj-61e80efaf9f75d0fb45a8f17
 
%%author_id yz.italomania_i_gastromania
%%date 
 
%%tags alfavit,etruski,pismennost',russkie
%%title Этрусски - это русские или нет?
 
%%endhead 
 
\subsection{Этрусски - это русские или нет?}
\label{sec:19_01_2022.yz.italomania_i_gastromania.1.etruski_russkie}
 
\Purl{https://zen.yandex.ru/media/italy/etrusski-eto-russkie-ili-net-poprobuem-razobratsia-alfavit-tochno-pohoj-61e80efaf9f75d0fb45a8f17}
\ifcmt
 author_begin
   author_id yz.italomania_i_gastromania
 author_end
\fi

Древний спор о происхождении этрусков не могли разрешить на протяжении
нескольких веков. Учёные пытались найти ответы, изучая культуру, историю,
обычаи и языки. Но не так давно в дело включились генетики и похоже их
исследования положат конец догадкам о том, кто же всё-таки этруски – русские
или итальянцы?

\ii{19_01_2022.yz.italomania_i_gastromania.1.etruski_russkie.pic.1}

\subsubsection{Что вообще известно об этрусках?}

Этруски – довольно загадочная цивилизация, о которой сохранилось не так много
прямых фактов. Известно лишь, что этот могущественный народ жил ещё в доримской
Италии начиная с 9 века до н.э.

Их язык и культура вызывают неподдельный интерес у археологов и историков,
заставляя искать всё новые доказательства их принадлежности к той или иной
культуре.

\ii{19_01_2022.yz.italomania_i_gastromania.1.etruski_russkie.pic.2}

\begin{zznagolos}
В основном информация об их традициях, обычаях и религии дошла до нас благодаря
рассказам древних римлян.
\end{zznagolos}

Современные исследования полагают, что родиной этрусков были Причерноморские
степи. Но по каким-то причинам (возможно, голод или неурожай) им пришлось
покинуть родные края в начале бронзового века и двинуться на север Италии.

\subsubsection{Какими были этруски?}

Мужчины носили короткие волосы и коротко стригли бороду, любили устраивать
зрелищные бои и париться в бане. Женщины же напротив имели длинные волосы,
которые носили распущенными и тщательно за ними ухаживали. 

\begin{zznagolos}
Что интересно, в отличие от римлянок, этрусски обладали большей свободой и даже
могли вступать в бой с мужчинами.
\end{zznagolos}

В качестве одежды выступала накидка или рубашка, а в качестве обуви – сандалии.

Особо эта цивилизация преуспела в строительстве и медицине (среди найденных
предметов при раскопках были обнаружены даже зубные протезы). А в 8 веке до
нашей эры начало стремительно развиваться ювелирное дело, что подтверждают
найденные многочисленные украшения из драгоценных камней в виде колец,
браслетов и булавок.

\ii{19_01_2022.yz.italomania_i_gastromania.1.etruski_russkie.pic.3}

Особый интерес этот народ проявлял к силам природы, которым устраивались
жертвоприношения, а также загробному миру, о чём говорят роскошно украшенные
саркофаги и гробницы.

Известно, что этруски и римляне были союзниками. А период, когда Этрурия
перестала быть независимой и до момента получения этрусками римского
гражданства (в 89 году до нашей эры), назывался этрусско-римским.

\subsubsection{А как же язык?}

Этруски имели свой алфавит и успешно его использовали, хотя до нашего времени
сохранились в основном лишь погребальные надписи. И хотя по поводу языка ещё
много споров и разногласий, мысль о том, что этрусский язык стоит у истоков
русского, имеет место. Напишите, что думаете на этот счёт в комментариях!

\begin{zznagolos}
Например, этрусское слово «cena» – это есть наше русское «цена», причем в том же значении. Или «ati» – отец.	
\end{zznagolos}

А по поводу алфавита есть много доказательств, что до кириллицы русские писали
слова именно на латинице, но так, как они звучат в русском (такое часто можно
встретить и сейчас, когда русские живут за границей и пользуются только
английской раскладкой, но пишут при этом русский текст). 

\ii{19_01_2022.yz.italomania_i_gastromania.1.etruski_russkie.pic.4}

\begin{zznagolos}
Кстати, этот феномен могут понять и прочесть только русские, потому что любому
иностранцу такое прочтение даётся нелегко.
\end{zznagolos}

Однако, это лишь одно из предположений, которое осложняется тем, что этруски в
письме не различали звонкие и глухие согласные, поэтому возможно читались эти
слова вовсе не так, как мы их сейчас произносим.

\ii{19_01_2022.yz.italomania_i_gastromania.1.etruski_russkie.pic.5}

Но один польский учёный Тадеуш Воланский все же перевёл многочисленные
этрусские тексты на русский по такой схеме, доказав тем самым, что все они
являются славянскими. Но правда за это был подвергнут гонениям со стороны
католической церкви.

А итальянский исследователь Себастьяно Чампи был настолько увлечен изучением
этрусской цивилизации, что потратил всю жизнь, пытаясь расшифровать их алфавит
и хотя бы один текст. В конце своих дней он так и записал в дневнике:
«Этрусский язык вымер и надписи прочесть невозможно».

\ii{19_01_2022.yz.italomania_i_gastromania.1.etruski_russkie.pic.6}

Но факт остаётся фактом – слова, над которыми бились великие умы всего мира, с
лёгкостью понимаются только русскими, стоит лишь прочесть их на латинице. Но
это не точно =)

Ещё одним фактом, указывающим на славянские корни этрусков является и то, что
славяне действительно жили в Италии. Во втором тысячелетии до н.э. эти земли
населяли опичи, любичи, воличи и другие славянские племена.

\subsubsection{Открытие генетиков}

И вот, спустя столько веков и попыток определить принадлежность этрусков, в
дело вступили генетики.

Недавние исследования, проводимые в Германии и Италии, похоже наконец помогли
найти ответ на вопрос, терзающий учёных всего мира – кто же такие этруски?

Всего было изучено более 80 ДНК человек, живших в разные периоды от 800-го года
до нашей эры до 1000-го года нашей эры. 

\begin{zznagolos}
И все они оказались коренными поселенцами Апеннин, а вовсе не переселенцами с Ближнего Востока, как считалось ранее.
\end{zznagolos}

Дело в том, что ранее уже были попытки сравнивать ДНК жителей современной
Тосканы с представителями Ближнего Востока и результаты исследований
действительно показали связь их митохондрий с нынешним анатолийским населением.
Это и явилось подтверждением миграции этрусков.

\ii{19_01_2022.yz.italomania_i_gastromania.1.etruski_russkie.pic.7}

Но более точные изучения ДНК людей, проживающих в этой местности на разных
этапах позволили доказать, что приток мигрантов с Ближнего Востока был, но
случилось это уже после падения этрусской цивилизации.

Как бы то ни было, этруски – это уникальная, невероятно развитая для тех времён
цивилизация, которая имела очень богатую культуру и её изучение наверняка
преподнесёт ещё немало сюрпризов.

Разумеется в таком коротком материале не вместить и части информации об
этруссках. Да и часть информации дошла до наших времён мягко говоря искажённой.
Напишите в комментариях, что вы думаете об этруссках их принадлежности и их
происхождении?

\ii{19_01_2022.yz.italomania_i_gastromania.1.etruski_russkie.cmt}
