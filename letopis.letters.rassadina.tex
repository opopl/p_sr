% vim: keymap=russian-jcukenwin
%%beginhead 
 
%%file letters.rassadina
%%parent letters
 
%%url 
 
%%author_id 
%%date 
 
%%tags 
%%title 
 
%%endhead 

%11:14:32 31-03-23
Ви просто неймовірна Людина!! Дуже дякую за пост!! До речі, я
бачив там зверху дискусію щодо запису та публікації. Це дійсно необхідно
робити, і я цим займаюсь зі свого боку. Я програміст. живу в Києві. Займаюсь
систематизацією, записом та архівацією різноманітних публікацій в фб та взагалі
в інеті. У мене є проект Літопису Війни. Щодо збереження та публікацій - у мене
є власний метод (заснований на системі підготовки публікацій LaTeX - це те, що
використовують науковці як де факто стандарт публікацій в світі фізики та
математики та інших наук) та власне програмне забезпечення, яке я використовую.
Як все це виглядає, можете подивитись у мене на сторінці, я якраз виставив
архівовану копію цього поста (там в біо фб-сторінки також є посилання на мій
телеграм канал та акаунт в Інтернет-Архіві із збереженими постами). Щодо
книжки. Із технічної сторони, я вже майже рік тому зробив книжку на 200
сторінок про Харків під час війни (посилання є в телеграм каналі, в самому
початку, як прогорнете назад). Якщо що, не проблема зробити книжку з усіх ваших
постів + оформити згідно побажанням. Я цим всім (літопис, запис постів і т.д.)
займаюсь вже роки два, і вже добрячи набив на цьому руку. На запис Вашого поста
у мене вчора вийшло дві години (разом із усім, повний запис всього)...
Сподіваюсь, Вам це буде цікаво, і я зможу Вам чимось в цій архіважливій справі
допомогти!

%11:28:17 31-03-23
добре ) так, пишіть! це дуже важливо, щоби світ дізнався правду!
