% vim: keymap=russian-jcukenwin
%%beginhead 
 
%%file 26_02_2022.fb.fb_group.story_kiev_ua.2.scho_u_vas_novogo.cmt
%%parent 26_02_2022.fb.fb_group.story_kiev_ua.2.scho_u_vas_novogo
 
%%url 
 
%%author_id 
%%date 
 
%%tags 
%%title 
 
%%endhead 
\zzSecCmt

\begin{itemize} % {
\iusr{Марина Волков-я}
Дякуємо на доброму слові.

\iusr{Сергей Векслер}
\textbf{Марина Волков-я}
Ми разом, яхад יחד

\iusr{Татьяна Ткаченко}

\ifcmt
  tab_begin cols=2,no_fig,center,resizebox=0.6

     pic https://scontent-lhr8-1.xx.fbcdn.net/v/t39.30808-6/274795436_474882614301180_1768050601192728370_n.jpg?_nc_cat=110&ccb=1-5&_nc_sid=dbeb18&_nc_ohc=tjlTMl9vDgQAX88RBtN&_nc_ht=scontent-lhr8-1.xx&oh=00_AT_S1p2oTcer3TdDzAExDbMoAXLmMtXqOsWsDddUwZs0mA&oe=622BB637

		 pic https://scontent-lhr8-1.xx.fbcdn.net/v/t39.30808-6/274822231_474883110967797_3831853601389638681_n.jpg?_nc_cat=109&ccb=1-5&_nc_sid=dbeb18&_nc_ohc=zFmWpUekAr8AX-YjmBE&_nc_oc=AQklfTXBMiE8_lihwYDbXn39NLAzmRPMcoWIkgqhhQ2U6I2LUy2zosxKvipt5S-vZNY&_nc_ht=scontent-lhr8-1.xx&oh=00_AT9AlGhbvUVgkRQdV1UhqNXR7LDAEX5M_Mv8sKiYhGzIzA&oe=622AADF4

  tab_end
\fi

\iusr{Екатерина Егорова}

Дякуємо, рідні! Ми вами пишаємось!

\iusr{Max Gopencko}

Весь світ із Україною, окрім окремих недоумків, які знаходяться під впливом
кремлівської пропаганди.

\iusr{Михайлина Голуб}

Пане Сергію! Велика вам подяка. Я півтора роки у Беер-Шеві. Тут моя молодша
донька із сім'єю. Старша - в Єрусалимі. Онук навчається в Празі. Здається, ми в
безпеці. Але наше серце стікає кров'ю. В Україні рідні, друзі, учні. Я теж пишу
вам українською. Завжди я вважала себе людиною російської культури, але зараз
мені страшно. Я не можу збагнути, що мої улюблені мною культурні цінності
посміють привласнити собі нелюди. Мені страшно, що би відчував мій батько
зараз. Він, 18-річний хлопчик, захищав Москву і своє 18-річчя зустрічав на
госпітальній койці. Щоб зараз сказав мій дядько - заслуженний тренер України,
кореспондент газети \enquote{Советский Спорт}, коли, може, діти його московьких
друзів гатять зараз по його харківському дому.

Я вірю, перемога буде за нами. Ми всі - єврейські діти України. Слава Україні!

\begin{itemize} % {
\iusr{Сергей Векслер}
\textbf{Михайлина Голуб}

Щодо мови і культури.

Велика німецька культура, культура Канта, Моцарта, Ніцше, Баха, Гьоте і Гейне.
Деякий час це була мова гітлера і гебельса. Деякий час ...

Зараз, нажаль, час путіна. Але це не зменшує мову і культуру Лихачова,
Сахарова, Євтушенко, Бердяєва, Толстого, Достоєвського, Пушкіна, Ахматової.

Гітлери і путіни приходять і уходять, залишаючи пляму. Велике залишається.

\iusr{Михайлина Голуб}
\textbf{Сергей Векслер} 

Я згодна з вами, шановний пане Сергію. Але в нас всіх болить серце. Я згадую
слова мого головного в житті поета - Євгенія Євтушенка і десь у півсвідомості
навіть минає думка: яке щастя, що він не бачить того, що діється зараз. Я
розумію, якби він був з нами, він засудив би цей злочин російських нелюдів, як
засуджують його кращі люди Росії. Але оця пляма, про яку ви пишите, вона кровав
й болюча. І важко жити нам, для яких російська мова, російська культура
рідні....

\end{itemize} % }

\iusr{Сергей Векслер}

לא דובר רוסית.
הפוסטים שלי באוקראינית
כי אֵין לִי אֶרֶץ אַחֶרֶת
גַם אִם אַדְמָתִי בּוֹעֶרֶת

\iusr{Evgeny Mekhanik}

Я також з Києва.

У Києві мешкають батьки та сестра з родиною. Батьки залишись на Шулявці.

Тримаються та намагаються заскопокоїти нас... Але, серце не на місці, та ви
самі усе знаєте.

Тримайтеся, будь ласка! Ми з вамі як тільки можливо. Перемога буде за нами.
Слава Україні!

\iusr{Evgeny Mekhanik}

Я також з Києва.

У Києві мешкають батьки та сестра з родиною. Батьки залишись на Шулявці.

Тримаються та намагаються заскопокоїти нас... Але, серце не на місці, та ви
самі усе знаєте.

Тримайтеся, будь ласка! Ми з вамі як тільки можливо. Перемога буде за нами.

Слава Україні!

\iusr{Виктор Задворнов}
\textbf{Evgeny Mekhanik} Героям слава! Смерть ворогам!

\iusr{Сергей Векслер}

\ifcmt
  ig https://scontent-lhr8-2.xx.fbcdn.net/v/t39.30808-6/274782906_10225216725058940_5972537590979195998_n.jpg?_nc_cat=104&ccb=1-5&_nc_sid=dbeb18&_nc_ohc=B3UpgGo5AFwAX9EJ_6C&_nc_ht=scontent-lhr8-2.xx&oh=00_AT_aUkA1s6mz78rjP6h7ZSsRiSOjpxVlb6g9b1XBfSPSDQ&oe=622B0A90
  @width 0.3
\fi

\iusr{Сергей Векслер}

Автор

1.3.22

Підійшов вчора до нього, до давнього друга - довго чекав на цю зустріч. В
Єрусалимському ботанічному саду прокидається після зимової сплячки каштан,
київський каштан у Єрусалимі. Набухають бруньки, вузлики майбутньої весни
дивляться в небо, посміхаються завтрашньому дню.

Неподалік побачив пару, що гуляє, єрусалимці. Помітивши, що я фотографую,
підійшли ближче, глянули вони на табличку з написом на івриті і розділили мою
радість, прочитавши на ній слово ערמון, каштан:

- Каштан в фєвралє, - сказав чоловік своїй жінці. - побачив я і їх увагу і небайдужість..
- Київський каштан у єрусалимському лютому" - відповів я, саме українською мовою.
- І нашою мовою не фєвраль, а люты - підтвердили вони.
- Ви з Білорусії?
– Так, але ми з Україною разом.

Ми разом.

І на зміну лютому ось-ось прийде березень та квітень. І про них, про весняні
місяці, обов'язково розповість каштан, єрусалимсько-київський.

У Іерусалімскім батанічным садзе прачынаецца пасля зімовай спячкі каштан,
кіеўскі каштан у Іерусаліме. Набрыняюць ныркі, вузельчыкі будучай вясны
глядзяць у неба, усміхаюцца заўтрашняму дню.

\ifcmt
  ig https://scontent-lhr8-1.xx.fbcdn.net/v/t39.30808-6/274877262_10225222351999610_615472696244564369_n.jpg?_nc_cat=110&ccb=1-5&_nc_sid=dbeb18&_nc_ohc=Avnh8r7-WLsAX-rn4dM&_nc_ht=scontent-lhr8-1.xx&oh=00_AT8YMQ7x36R5iqu5SCXpuL1REWfH5MLHaKMIrcICkzvxCg&oe=622BBB44
  @width 0.3
\fi

\iusr{Сергей Векслер}

Вчора зустрів на площі Кікар Сафра в Єрусалимі ДРУЗІВ. Тії, кого знаю по
фейсбуку, але побачив уперше. Вони прийшли підтримати Україну і сказати НІ
путіну. Москва, Дніпро, Київ, Санкт-Петербург, Ухта, Харків, Якутськ, Одеса,
Минськ.

Стільки дивовижних світлих людських облич! Жовтий і блакитний. Напроти -
єрусалимський бар, якій більше не зветься Рutin.

\iusr{Сергей Векслер}

1.3.22

Що у вас на думці?

Треба щось таке пригадати зараз, щоб було нібито епітафією закритому в
Єрусалимі путіну.

Подумав знайти світлини, зроблені років п'ятнадцять тому. Бар Putin був
розташовуваний в іншому місці, не на центральній єрусалимській вулиці Яфо, а в
маленькому провулку поруч з Русским Подворьем, більшість якого Хрущов віддав
Ізраїлю за апельсины.

Були сфотографовані окремі моменти. Навіть завітав у внутреннее пространство.
Побачив тоді світлини у півтемряві - славный путь из питерской подворотни к
кремлёвскому креслу.

Ні, не буду шукати ці світлини, хоча знайти їх зовсім неважко сьогодні, завдяки
інтернету.

Тільки нагадаю напис на вході. Гумор минулих років, який актуальний - саме для
нього - у сьогоденн. Пам'ятаю ті слова, що були написані на вході у старий бар
путін. Такі вони:

\enquote{НЕ ЛОМАЙТЕ ДВЕРЬ. ЕСЛИ ВАМ НЕ ОТКРЫЛИ СРАЗУ, ТО НЕ ОТКРОЮТ НИКОГДА}

Сьогодні це про ту двір, яка не відчинена для путлера.

\end{itemize} % }
