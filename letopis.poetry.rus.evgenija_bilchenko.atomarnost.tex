% vim: keymap=russian-jcukenwin
%%beginhead 
 
%%file poetry.rus.evgenija_bilchenko.atomarnost
%%parent poetry.rus.evgenija_bilchenko
 
%%url https://t.me/bilchenko_z/118
%%author 
%%tags 
%%title 
 
%%endhead 

\subsubsection{БЖ. Атомарность}
\Purl{https://t.me/bilchenko_z/118}

Каково это: быть "креативной элитой", миллениальской тлёй?
Сцепись со своей квартирой, как с офисною землёй.
Сиди перед монитором, закрывая глаза картин, -
Цифровой генерации глаз слепой, тотальнейший карантин.

Новое хуторянство недвижимых абстрактных тел.
Дигитальные технологии, которые ты хотел
Заполучить как средство борьбы со своей судьбой...
Радуйся, ты - победитель над правом собственным быть собой.

Ты - Иона во чреве кита, но кит твой - машинозверь.
Он - бесчувственный механизм калькуляции . От потерь
До новых приобретений - три линка и две строки
В добавление к комментарию, что длиннее твоей руки.

Впрочем, и рук-то уже не надо: стилус заменит две.
Не ходи на работу, не размножайся в мокрой густой траве.
Не дыши ей в рот: она - просто вирус. Больше - не человек.
Всяк недочеловек да имеет здесь свой отсек.

Каково это - быть? Просто быть и помнить память, которой нет.
Помнить вот эти вот: "Буря мглою..." и "Закройте глаза газет!"
Когда от Цветаевой - в горле ком, от картошки - горчит в кишках.
Когда грузом двести твоя любовь несёт тебя на руках

В живую влажную дедовщину песен, медалей, слёз.
Она несёт тебя так, как ты тоже её бы нёс,
Если б не эти стальные, сложно устроенные киты...

Помнишь ли ты, о, помнишь ли ты, как просто - нести кресты?

11 октября 2020 г.
