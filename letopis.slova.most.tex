% vim: keymap=russian-jcukenwin
%%beginhead 
 
%%file slova.most
%%parent slova
 
%%url 
 
%%author 
%%author_id 
%%author_url 
 
%%tags 
%%title 
 
%%endhead 
\chapter{Мост}
\label{sec:slova.most}

%%%cit
%%%cit_pic
%%%cit_text
Жизнь князя Владимира, крестившего Русь в 988 году, была бурной, неспокойной и
неоднозначной. И в мир оной он тоже отошел отнюдь не с улыбкой смирения на
устах. До последнего момента в нем кипела энергия. Тело уже не слушалась, но
мозг продолжал свою кипучую деятельность.  Летопись сохранила для нас
предсмертную фразу, произнесенную князем. Он не молился, не каялся, он отдавал
команду: \enquote{Требите путь и \emph{мостите мост}.  Куда собирался
отправляться Владимир? Для чего ему нужно было расчищать дорогу и строить
\emph{мосты}? Все просто. Он ни много ни мало собирался в очередной раз отправиться в
военный поход
%%%cit_comment
%%%cit_title
\citTitle{Мостите мост. Что означали последние слова князя Владимира?}, 
Русичи, zen.yandex.ru, 09.06.2021
%%%endcit

%%%cit
%%%cit_head
%%%cit_pic
%%%cit_text
Очень прикольно с «крымским \emph{мостом}», вместо которого надо говорить
«незаконно построенный \emph{мостовой} переход между временно оккупированной АР
Крым и РФ через Керченский пролив». Что-то подсказывает, что говорить все равно
будут «крымский \emph{мост}».  Вместо «особого статуса Донбасса» — «особый
порядок местного самоуправления ОРДЛО», что, очевидно, навеяно недавними
юродивыми высказываниями одного отечественного неадекватного деятеля о том, что
никакого Донбасса, дескать, вообще не существует в природе, и это название
якобы выдумала кремлевская пропаганда
%%%cit_comment
%%%cit_title
\citTitle{Краткий словарь грантоедов под редакцией СНБО / Лента соцсетей / Страна}, 
Александр Карпец, strana.news, 26.10.2021
%%%endcit

%%%cit
%%%cit_head
%%%cit_pic
%%%cit_text
Российская пропаганда может разбиваться в лепешку, но факты... Тем, кто еще вчера
в Крыму орал, что придет Россия и жизнь наладится, сегодня остается оценивать
на вкус собственные локти. Потому что это Крым сейчас могу получить массу
интересных инфраструктурных объектов для привлечения туристов. И реконструкцию
аэропорта. В составе совсем не богатой, но делающей ставку на развитие Украины.
Сейчас все это получает Херсонщина. А «русский мир» не может предложить ничего,
кроме «камней с неба» и угроз показать всему миру ядерный перпел. Кстати, если
бы в Крым не пришел «русский мир», то строить \emph{мост} не было бы никакой нужды. Ну
про воду и так всем очевидно. И какое государство демонстрирует неполную
состоятельность? 
%%%cit_comment
%%%cit_title
\citTitle{И какая страна теперь 404?}, Кирилл Сазонов, censor.net, 28.10.2021%
%%%endcit

%%%cit
%%%cit_head
%%%cit_pic
\ifcmt
  tab_begin cols=3
     pic https://avatars.mds.yandex.net/get-zen_doc/167204/pub_61485ea2800cd44002069062_61485fc48a833355e737f2fd/scale_1200
     pic https://avatars.mds.yandex.net/get-zen_doc/229502/pub_61485ea2800cd44002069062_614861709d150d6120ca7ce5/scale_1200
		 pic https://avatars.mds.yandex.net/get-zen_doc/4766103/pub_61485ea2800cd44002069062_614860c19d150d6120c8e892/scale_1200
  tab_end
\fi
%%%cit_text
Начнём прогулку по одному из самых древних российских городов с одного из самых
новых и масштабных городских объектов. 1400-метровый Муромский \emph{мост}
через Оку - один из новейших в России. Он открыт в 2009 году и соединяет
Владимирскую и Нижегородскую области. В 2013 году \emph{мост} назван самым
красивым мостом в России, заняв 1-е место в конкурсе, проводимым Федеральным
дорожным агентством. До строительства этого \emph{моста} перебраться через Оку
в Муроме можно было только по понтонному летнему \emph{мосту}, а зимой на
пароме.  Что за техника, не знаю. Видимо, проводились какие-то военные учения.
Муром - город монастырей. Мы на территории одного из них,
Спасо-Преображенского. Чисто, уютно, аккуратно. Снимать не запрещают (кстати, в
других монастырях города - тоже, что для меня стало приятной неожиданностью)
%%%cit_comment
%%%cit_title
\citTitle{Муром. Прогулка по одному из старейших городов России}, 
Уникальная Россия, zen.yandex.ru, 21.09.2021
%%%endcit
