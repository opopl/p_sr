% vim: keymap=russian-jcukenwin
%%beginhead 
 
%%file 29_04_2022.fb.solovjov_mikita.harkov.demsokyra.1.hronika
%%parent 29_04_2022
 
%%url https://www.facebook.com/Mikita.Solovyov/posts/7461759520561063
 
%%author_id solovjov_mikita.harkov.demsokyra
%%date 
 
%%tags 
%%title Хроника Харькова, 29-е апреля
 
%%endhead 
 
\subsection{Хроника Харькова, 29-е апреля}
\label{sec:29_04_2022.fb.solovjov_mikita.harkov.demsokyra.1.hronika}
 
\Purl{https://www.facebook.com/Mikita.Solovyov/posts/7461759520561063}
\ifcmt
 author_begin
   author_id solovjov_mikita.harkov.demsokyra
 author_end
\fi

Хроника Харькова, 29-е апреля. 

Обстрелы опять очень неравномерные. Сегодня кроме стандартных, сильно
обстреливали ХТЗ. По Салтовке примерно как всегда. На Пятихатках и Жуковского
легче, чем предыдущие дни. То ли это просто удачный день, то ли это новая
реальность после освобождения Русской Лозовой. По другим районам города
серьезных обстрелов или не было, или просто у меня нет информации. 

По коммуналке есть для меня лично очень хорошая новость. Сегодня Синегубов
написал, что уже готовятся начинать работу по восстановлению теплоснабжения.
Почему это важная и хорошая новость? Потому что свидетельствует о расширении
горизонтов планирования. Городские и областные власти уже понимают, что война
сама собой быстро не рассосется, что нужно учиться обеспечивать критическую
инфраструктуру в условиях войны. Иначе был бы большой соблазн оставить вопрос с
отоплением на \enquote{после войны}. Но под обстрелами готовиться к отопительному
сезону будет намного тяжелее, тем более что заметная часть инфраструктуры
повреждена. А значит, надо начинать готовиться уже скоро, в идеале сейчас. Да,
продолжаются работы по восстановлению водо-, газо- и электроснабжения. Сегодня
еще в несколько домов восстановили подачу газа. 

По области главным событием однозначно является Русская Лозовая. Сегодня стало
понятно, что оттуда не просто выбили русских, а смогли там закрепиться. Как и в
Кутузовке с Момотово на восточном направлении. То есть даже в условиях
тяжелейшей битвы на Донбассе идет расширение пояса безопасности вокруг
Харькова. Фактически, остается только одно направление, по которому русские
стоят почти вплотную к городу. Это северо-восток, линия Циркуны-Тишки-Липцы.
Если удастся продвинуться еще немного по двум другим направлениям и выбить
русских из Липцев, то обстрелы Харькова сильно уменьшатся. Дальше нас будут
доставать только дальнобойным. 

Но основные бои в области идут на изюмском направлении. Там же и наиболее
плотные обстрелы. Развернутых данных по происходящему там у меня нет. Знаю
только, что там нашим очень тяжело, но они держатся. 

Еще раз хотел сказать несколько слов об эвакуации. Я все так же никого не
призываю эвакуироваться из относительно спокойных районов Харькова, тут каждый
решает сам. Но есть места, из которых эвакуироваться на мой взгляд необходимо.
Так меня очень радует, что эвакуация объявлена из наиболее пострадавших
микрорайонов. Хотя ведется она и медленнее, чем мне бы хотелось, но ведется. И
точно так же, эвакуация нужна из населенных пунктов совсем возле линии фронта и
на направлениях наиболее вероятного наступления русских. Например, сегодня из
Русской Лозовой эвакуировали порядка 600 человек. За что отдельное огромное
спасибо Кракену. 

А знаете, какая сейчас самая большая проблема с эвакуацией? Люди категорически
отказываются уезжать. Из Пятихаток, с Северной Салтовки и Горизонта. Сегодня
вот из Русской Лозовой. И я просто обращаюсь ко всем с просьбой. Если вы живете
в настолько пострадавших или опасных местах, пожалуйста, эвакуируйтесь. Если у
вас есть живущие там друзья и родственники, постарайтесь уговорить их
эвакуироваться. Я точно знаю, что есть варианты куда перехать в относительно
комфортные условия. Я точно знаю, что волонтеры обеспечат переезд и помогут с
размещением. Варианты есть. Но не отказывайтесь, когда вам предлагают такие
варианты. Я понимаю, что уезжать из дома очень не хочется. Что это очень тяжело
психологически, особенно возрастным людям. Что страх переезда в незнакомое
место может быть сильнее страха обстрелов и даже оккупации. Но помогите
волонтерам позаботиться о вашей безопасности!

Харьков стоит!

Слава Украине!

Низкий поклон нашим защитникам!

\#ХроникаХарькова

\ii{29_04_2022.fb.solovjov_mikita.harkov.demsokyra.1.hronika.cmt}
\ii{29_04_2022.fb.solovjov_mikita.harkov.demsokyra.1.hronika.cmtx}
