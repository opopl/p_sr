% vim: keymap=russian-jcukenwin
%%beginhead 
 
%%file 23_01_2022.stz.news.ua.luck.visnyk.1.malchik_300_knigi_chtenie
%%parent 23_01_2022
 
%%url https://visnyk.lutsk.ua/news/ukraine/regions/volyn/68558-10-richniy-khlopchik-z-lutska-prochitav-300-knizhok/
 
%%author_id news.ua.luck.visnyk,golovij_oksana
%%date 
 
%%tags chtenie,deti,kniga,kultura,literatura,luck,ukraina
%%title 10-річний хлопчик з Луцька прочитав 300 книжок
 
%%endhead 
 
\subsection{10-річний хлопчик з Луцька прочитав 300 книжок}
\label{sec:23_01_2022.stz.news.ua.luck.visnyk.1.malchik_300_knigi_chtenie}
 
\Purl{https://visnyk.lutsk.ua/news/ukraine/regions/volyn/68558-10-richniy-khlopchik-z-lutska-prochitav-300-knizhok/}
\ifcmt
 author_begin
   author_id news.ua.luck.visnyk,golovij_oksana
 author_end
\fi

10-річний лучанин Володя Каліш обожнює читати книжки. 

А ще хлопчик разом із мамою веде відеоблог на YouTube, у якому ділиться
враженнями від прочитаного та знайомить підписників з літературними новинками,
пише ІА \enquote{Волинські Новини}. 

\ii{23_01_2022.stz.news.ua.luck.visnyk.1.malchik_300_knigi_chtenie.pic.1}

«Обожнюю читати. Читаю книги різних авторів, різних видань та з різноманітними
сюжетами, – розповідає хлопчик. – Детективи, фентезі, історія – мені все до
вподоби. Комікси я не надто полюбляю, але читаю і їх, цікаві трапляються.
Багатьох вони приваблюють тим, що коротенько розповідають якісь захопливі
історії, ще й малюнки яскраві».

10 січня Володя з мамою святкували невеличкий ювілей – їхньому блогу
виповнилося п’ять років. Хлопчик зізнається: ідея була мамина, а він радо
долучився.

\ii{23_01_2022.stz.news.ua.luck.visnyk.1.malchik_300_knigi_chtenie.pic.2}

«Книжки ми читаємо разом. Хоча іноді буває, що я чимось зайнята, і син мене
переганяє. Потім каже: «Я прочитав 70 сторінок, тепер наздоганяй мене», –
розповідає мама хлопчика Наталка. – Зазвичай огляди записуємо одразу після
прочитання, як то кажуть, першим дублем. Коли ще свіжі враження, без підготовки
– тоді й емоція яскравіша, щиріша».

«Аудиторія у нас не надто велика – близько 900 підписників. Але ми віримо, що
їх з часом буде значно більше, – сподівається юний блогер. – Мама записує наші
блоги на свій телефон, купила триногу, аби якість запису була кращою. А тато
нам подарував мікрофони. Хоча зараз ми їх не використовуємо, бо мені трохи не
сподобався звук».

Володя додає: вони з мамою старанно ведуть статистику прочитаних книг. І в
листопаді минулого року записали огляд вже трьохсотої книжки. Каже, часто
підписники в коментарях просять прочитати й розказати про те чи те видання.

«Крім звичайних оглядів, ми робимо ще й новорічні випуски, в яких розповідаємо
про топ-10 книжок, що їх прочитали за рік, – провадить Володя. – А ще
визначаємо найкращий твір. Цього року наш гран-прі – книзі про Олексу Довбуша
Василя Карп’юка. Її першу й другу частини я вже прочитав. Книгу написано в дуже
цікавому стилі, до того ж, у ній класні ілюстрації. А перше місце ми віддали
книзі Лариси Ніцой «Дві бабуськи в незвичайній школі, або Скарб у візку».

Перший випуск відеоблогу мама й син записали, коли Володі було п’ять років.

«Тоді я й навчився читати. А так мені дуже багато читала мама, – усміхається. –
Перша книжка, яку я сам прочитав, – «Веселі пригоди Кицика і Мицика» Юхима
Чеповецького. Дуже люблю котиків!»

Хлопчик каже: за п’ять років його літературні смаки дещо змінилися.

«От, наприклад, остання книжка, яку прочитав минулого року, – «Викрадачі снігу»
Анастасії Нікуліної. Вона вийшла зовсім недавно. Насправді українського фентезі
не так багато, а ця книга ну дуже класна, вона навіть увійшла в наш топ-10, –
ділиться книголюб. – Але знаєте, моя любов до котиків не минула. Я їх так само
обожнюю».Володя каже: про те, ким стане в майбутньому, ще не замислювався.

«Але останнім часом мені часто пророкують кар’єру телеведучого, – знизує
плечима. – Ну, побачимо. Поки що й надалі вестиму наш із мамою відеоблог».

\ii{23_01_2022.stz.news.ua.luck.visnyk.1.malchik_300_knigi_chtenie.pic.3}

Доки журналісти розмовляли з Володею у Волинській обласній бібліотеці для
юнацтва, він встиг навибирати собі цілий оберемок новеньких книжок на
відкритому перегляді україномовних видань. А дві з них устиг прочитати й одразу
ж узявся записувати черговий випуск свого блогу для YouTube.

«Атож, треба ж поспішати, бо он скільки ще всього цікавого маю. Перечитаю, про
все розкажу і принесу назад у бібліотеку – хай інші читають. Бо це – круто», –
усміхнувся нам на прощання.

Оксана ГОЛОВІЙ

Фото Олександра ДУРМАНЕНКА
