% vim: keymap=russian-jcukenwin
%%beginhead 
 
%%file 14_05_2022.fb.basakin_aleksej.harkov.newsroom.1.meshok.cmt
%%parent 14_05_2022.fb.basakin_aleksej.harkov.newsroom.1.meshok
 
%%url 
 
%%author_id 
%%date 
 
%%tags 
%%title 
 
%%endhead 
\qqSecCmt

\begin{itemize} % {
\iusr{Оксана Говорунова}

Вот да, настолько разные посты от разных людей, что хрен поймёшь, как оно там.
Но я склоняюсь к хорошим и позитивным и что там все будет хорошо и медленно
пакую свой рюкзак  @igg{fbicon.grin} домой

\begin{itemize} % {
\iusr{Алексей Басакин}
\textbf{Оксана Говорунова} 

более тихо, чем было. А так - каждый сам оценивает риски. Сосед по-прежнему
ебанутый.

\iusr{Оксана Говорунова}
\textbf{Алексей Басакин} 

ну он никуда не денется, увы. Надо скинуться и купить купол и жить поживать
@igg{fbicon.thumb.up.yellow} 

\iusr{Ольга Белецкая}
\textbf{Оксана Говорунова} 

вы должны понимать, что магазины, аптеки и больницы рпботают ограничено. В
парках и лесах гулять нельзя. (1. Мины, 2. Не желательно кучковаться)

Бдительность наше всё!

А! Ну и не дергаться когда не надо. И падать на землю когда надо.

Велкам хоум!

\iusr{Оксана Говорунова}
\textbf{Ольга Белецкая} 

в парке Шевченко и Горького мин нет, по утрам в Шевченко народ бегает. В леса я
особо и без войны не ходила и сейчас нет необходимости. Кроме похода в лес дел
полно будет. Все едят, значит в магазинах есть еда. Или привезут добрые люди.
Аптеки и больницы тоже дело такое-лекарство дома осталось, а болеешь не каждый
день и болеть за границей намного сложнее. Кучковаться тоже нет времени, тут
квест будет бензина накружить. Ну а как себя вести под обстрелами я в курсе
потому, что побывала.

\iusr{Ольга Белецкая}
\textbf{Оксана Говорунова} думаете очереди на заправке очень отличаются от очередей за гуманитаркой?
Или парк Горького ни разу не обстреливался?
Магазин еды и магазина с трусами не одно и тоже....
Как же здорово, что вы на позитиве, главное что бы без илюзий.

\iusr{Оксана Говорунова}
\textbf{Ольга Белецкая} 

я в парк Горького не хожу, у меня Шевченко за углом. Все, что происходит я вижу
из первых источников. Очередь за гуманитаркой мне не сильно нужна, мои друзья
покупают и развозят свою и хорошую и я им тоже деньгами помогаю, они ещё и
кормят наших соседей которые остались и следят за домом. Я на позитиве потому
что два месяца из Праги делаю что в моих силах для города @igg{fbicon.hands.raising}  вот бензин
тревожит, это да. Но надеюсь когда я приеду и этот вопрос уже будет решён. А
вот прислала мамы соседка фото из класс на алексеевке-вроде есть еда были б
деньги.

\end{itemize} % }

\iusr{Елена Борс}

А я у Клас сьогодні додзвонилася із скаргою.

Дістало. Мовчала довго. З початку війни чомусь різко скінчилася українська мова
на касах. При чому, демонстративно навіть якось - я українською, а мені у
відповідь російською. Сьогодні вранці аж вкурвило, що я хлопцю кажу:

- Ви принципово ігноруєте українську?

- А мнє так удобнєє.

Я попередила, що поскаржуся. Дівчина, що прийняла дзвінок, виявилася дуже
нашою, на щастя. Сподіваюся, подіє.

\iusr{Віктор Козоріз}

Єдине виправдання для бухариків, якщо вони бухали на радощах за наші ЗСУ і за
нашу Перемогу.

\iusr{Dmitriy Viktorovich}

К 500му вообще открываться ультимативные сверхспособности))

\begin{itemize} % {
\iusr{Алексей Басакин}
\textbf{Dmitriy Viktorovich} столь серьезных требований не было:)

\iusr{Dmitriy Viktorovich}
\textbf{Алексей Басакин}

\iusr{Dmitriy Viktorovich}
\textbf{Алексей Басакин} поверь на слово))
\end{itemize} % }

\end{itemize} % }
