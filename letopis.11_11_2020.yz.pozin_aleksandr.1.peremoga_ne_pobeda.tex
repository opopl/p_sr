% vim: keymap=russian-jcukenwin
%%beginhead 
 
%%file 11_11_2020.yz.pozin_aleksandr.1.peremoga_ne_pobeda
%%parent 11_11_2020
 
%%url https://zen.yandex.ru/media/id/5cc88c35eb28ac00aea480f4/peremoga-ne-pobeda-liubitelskii-opyt-sravnitelnoi-lingvistiki-5fac549bd59a05137b9a7828
 
%%author_id yz.pozin_aleksandr
%%date 
 
%%tags 
%%title ПЕРЕМОГА не ПОБЕДА: любительский опыт сравнительной лингвистики
 
%%endhead 
\subsection{ПЕРЕМОГА не ПОБЕДА: любительский опыт сравнительной лингвистики}
\label{sec:11_11_2020.yz.pozin_aleksandr.1.peremoga_ne_pobeda}

\Purl{https://zen.yandex.ru/media/id/5cc88c35eb28ac00aea480f4/peremoga-ne-pobeda-liubitelskii-opyt-sravnitelnoi-lingvistiki-5fac549bd59a05137b9a7828}

\ifcmt
 author_begin
   author_id yz.pozin_aleksandr
 author_end
\fi

Сегодня открываю новую рубрику на своем канале: \enquote{УкраЙна: реальность и мифы,
правда и вымысел}. 

\begin{zznagolos}
\obeylines
Есть слова – словно раны, слова – словно суд, - 
С ними в плен не сдаются и в плен не берут. 
Словом можно убить, словом можно спасти, 
Словом можно полки за собой повести.
Вадим ШЕФНЕР
\end{zznagolos}

\ii{11_11_2020.yz.pozin_aleksandr.1.peremoga_ne_pobeda.pic.1}

Ноябрь уж на дворе, а то значит, что на УкраЙне вступила в силу статья 57
\enquote{Закона о функционировании украинского языка как государственного.} В общем,
власти той страны решили ещё раз обокрасть своих граждан, на сей раз в
гуманитарном вопросе. Ибо украинский язык беден! И пусть я буду совсем не
толерантен, но попробую то вам продемонстрировать на простом примере.

Сразу предупреждаю, что в вопросах языкознания я полный дилетант. Посему  это
скорее эмпирический эксперимент, на уровне ощущений. Но вряд ли кто будет
отрицать, что слова различаются не только по значению, но и по эмоциональной
энергетике, которую они несут. Слова как бы рассказывают нам о предмете, не
только обозначая его, но и характеризуя предмет своим произношением, звучанием.

Слова «ГРОМ» и «БОМБА» вряд ли вызовут сонливость, напротив, слова «ПРОСТЫНЯ» и
«ПОСТЕЛЬ» не подходят для призыва, окрика, решительного действия или сильного
чувства. За эмоциональный фон слова отвечает «ПСИХОЛОГИЧЕСКАЯ» сторона языка,
изучаемая СЕМАНТИКОЙ.

Если верно утверждение, что в языке каждого народа есть слова, которые
характеризуют мироощущение и характер нации, то «ПОБЕДА» безусловно – одно из
таких. В этом слове заключён секрет непобедимости русских, их ярость
сопротивления, умение достичь результата через сверхусилие, способность
бороться до конца, даже без надежды на успех. В звонком и хлёстком,
жизнеутверждающем слове «ПОБЕДА» - торжество Русской идеи на просторе Евразии,
душа нашего народа. Это слово-песня, слово-клич.

Украинская «ПЕРЕМОГА» суть то же, что и «ПОБЕДА». Однако какие отличные
ассоциации оно вызывает! И это более чем странно, ведь в корне этого слова
заключена немалая сила. А родственно оно таким звучным и сильным словам, как
«МОЩЬ», «МОГУЩЕСТВО», «МОЧЬ», «ДОПОМОГА», утверждающему глаголу «МОГУ»,
прилагательному-эквиваленту силы «МОЩНЫЙ». Но в слове «ПЕРЕМОГА» сей могучий
корень теряется, получается не жизнеутверждающий призыв, а какая-то словесная
жвачка, нерешительная и осторожная тягучка. Словно произносящий это слово не
может решиться поставить не то, что восклицательный знак, а и точку.  В этой
неопределённости и недосказанности – нерешительность, боязнь определиться до
конца, нежелание встать на конкретную точку зрения и принять окончательное
решение. Все эти качества свойственны самому смыслу УКРАИНСТВА, характерны для
его дуалистичного нрава.

\ii{11_11_2020.yz.pozin_aleksandr.1.peremoga_ne_pobeda.pic.2}

Какие же особенности словообразования, сочетания букв и слогов делают эти слова
такими разными по эмоциональной насыщенности? Попробуем разобраться. Первое,
что бросается в глаза, слово «ПОБЕДА» короче, а, следовательно, более
энергоёмко. Если принять информацию, заключённую в слове за 100\%, то на три
слога «ПОБЕДЫ» приходится по 33\% информации, а на четыре слога «ПЕРЕМОГИ»
только по 25 процентов. При пересчете на звуки информационная насыщенность
слова «ПОБЕДА» возрастает ещё более. На 7 звуков сего слова приходится почти по
14.5\% информации. А в слове «ПЕРЕМОГА» аж 10 звуков, на которые падает по всего
10\% энергетической информации.

Но энергетический потенциал различных звуков неодинаков. Есть звуки проходные,
а есть эмоционально ударные, придающее слову характерную мелодику, строй,
энергетику. Сложные гласные, а слогов с ними в слове «ПЕРЕМОГА» 50\% (в слове
«ПОБЕДА» их меньше, всего 33\%), трудноваты для восприятия и воспроизведения,
поэтому практически не несут эмоциональной нагрузки. Исключение тут составляет
звук «Ё», но он то, как раз звук-изгой в украинском языке, для него даже нет
соответствующей буквы. Сила гласных звуков – в их открытости и протяженности. И
самыми «звучными» гласными являются звуки «А» и «О». Это вообще элементарные
звуки, самые простые, на уровне бессознательного крика, поэтому они являются
самыми энергетически ёмкими. Это не только самые первые звуки, которыми
овладело человечество на пути к речи, они являются начальными и для каждого
ребёнка в отдельности, когда он, вынутый из материнской утробы, впервые глотнув
кислорода, издаёт «А-А-А-А...».  Недаром первые в нашей жизни слова «ПАПА» и
«МАМА». Колоссальную энергетику таит и звук «О». Вспомните, как выразил автор
«Слова о полку Игореве...» ощущение тревоги, нависшую опасность над русским
воинством: «Дремлет Оль-гО-вО, хО-рОб-рО-е гнез-дО». Эта фраза в тексте как
набат, подобна звону колокола. Звуками «А» и «О» мы можем выражать целую гамму
эмоций и чувств, как то утверждение, восхищение, негодование, удивление, боль.
И в слове «ПОБЕДА» и в слове «ПЕРЕМОГА» эти звуки есть. Но в слове «ПО-бе-ДА»
слогов с простыми гласными 2/3, а в слове «пе-ре-МО-ГА» только половина. Значит
эмоциональное насыщение слова «ПОБЕДА» значительно выше.

И интонационно эти «кричащие» слоги расположены стратегически грамотно, в
начале и конце слова. Особо впечатляет супермощная концовка слова «ПОБЕДА»,
заканчивающееся утверждающим, безапелляционным, ярким, ликующим «ДА»!  Пафосный
характер такого окончания – во всём смысловом ряду подобных слов. Вслушайтесь:
«правДА», «звезДА» и, конечно «побеДА». Слово «ПЕРЕМОГА» заканчивается на слог
«ГА». Это южнорусское окончание по идее соответствует северорусскому «ДА» и
должно создавать яркую эмоциональную концовку. Тем более, что это типичное
индоевропейское сочетание звуков: «ГА», «ГО», «ГЕ». Но когда на базе
южнорусского языка образовывался современный украинский, он значительно
оторвался от своей начальной первоосновы и потерял многие смыслы. Поэтому слоги
со звуком «Г» оказались в загоне и произносятся без необходимого нажима, а то и
вовсе заменяются. За примерами далеко ходить не надо, междометие «ЭЙ»,
произносимое в малорусском наречии как «ГЕЙ», было заменено на невразумительное
«АГОВ», слово, этимология которого неясна. Хотя междометия во всех славянских
языках звучат примерно одинаково и трудно представить украинца, кричащего
вдогонку «АГОВ». Свидетельством отказа укроязыка (не путать с малороссийским
наречием) от славянского корня служит забытый боевой клич славян «ГЕЙ,
СЛАВЯНЕ!»

Ну, вот мы и перешли к анализу согласных звуков, а ведь именно они задают строй
и лад слову, отвечают за его эмоциональный фон. На заре исторического бытия
людей именно согласные и составляли собственно СЛОВО. Гласные звуки
использовались лишь в качестве связки между согласными. А в начальной
письменности и вовсе не писались. В арабском, если не ошибаюсь, до сих пор не
пишутся. И, естественно, наиболее энергоёмкими являются звонкие согласные. В
слове «ПЕРЕМОГА» их фактически нет. О звонком звуке «Г» мы уже рассуждали.
Можно добавить только, что этот звук звонкий только в великорусском диалекте.
Особенности малороссийского говора таковы, что звук «Г» не звонкий, а скорее
Г-лухой, ближе к «Х». Звук «Р», конечно, не глухой, он просто непарный, как и
«М». Могут возразить, что данный звук «Р-ыкающий», означающий хищное рычанье:
«Р-Р-Р...». Это не означает агрессивности, утверждения, а напротив, даже
некоторую неуверенность. Рычащий хищник своим рыком старается отпугнуть, после
этого он, как правило не нападает. Когда хищник охотится, он нападает молча,
чтобы не спугнуть жертву, как Гитлеровская Германия на СССР. Как видим в слове
«ПЕРЕМОГА» отсутствуют согласные звуки, способные вызвать эмоциональный подъём,
мобилизующая сила слова невелика.

Совсем иная картина вырисовывается при семантическом анализе слова «ПОБЕДА».
Секрет огромного эмоционального воздействия «ПОБЕДЫ» в том, что из трёх
согласных слова две – энергетически сильные, звонкие, яркие. Звук «Б», пожалуй,
самый сильный в нашем алфавите. Это звук взрыва, грохота, набата: «БОМ», «БУХ»,
«БОМБА». Не менее сильным является и звук «Д». Он тоже, как и звук «Б», звук
колокола, набата, только звенящий, дзинькающий: «ДЗИНЬ-ДОН». А ещё он уДарный
звук: «УДАР». Он Дарящий звук: «ДАР». И, самое главное, он роДной звук,
придающий осмысленность подвигу, означающий самое дорогое и сокровенное, что
стоит защищать и за что стоит умирать: «ДОМ», «РОДИНА». Буква даже по форме
напоминает роДной Дом. Получается, что по согласным осмысленность слова
«ПЕРЕМОГА» близка нулю, в отличие от слова «ПОБЕДА», в котором преобладают
ударные согласные.

Теперь окинем взглядом оба слова и оценим интонационное расположение звуков,
его соответствие значению слова. В слове «ПЕРЕМОГА» сильные звуки расположены
сумбурно, бессистемно, их расположение не служит увеличению эмоционального
восприятия слова. Первые два слога остались вообще без энергетического
насыщения, один энергетически сильный звук расположен в третьем слоге и только
последний слог несёт хоть какую-то энергетику: «пе-ре-мО-ГА». Всего лишь 30\%
сильных звуков, меньше 50\% энергетически ёмких слогов. «ПОБЕДА» же есть Победа,
из этого слова не выбросить ни одной буквы, как слова из песни не выкинуть, всё
на месте! Первый слог вытягивает набатный звук «О», во втором расположен
взрывной, колокольный звук «Б» и замечательная ударная концовка, зовущая на Бой
за Дом: «пО-Бе-ДА»! Это почти 60\% сильных звуков и 100\% энергетически ёмких
слогов.

Но самое интересное в том, что «ПЕРЕМОГА» не есть «ПОБЕДА». Это разные по
смыслу слова. Победа – сокрушительный и полный разгром врага, не оставляющий
ему никаких шансов на реванш, окончательное поражение противника, вплоть до
полного уничтожения. «Перемога» же всего лишь достижение превосходства в
соревновании, борьбе, турнире. В русском языке ей больше соответствует слово
«ПРЕВОЗМОГАНИЕ», Поэтому «ПОБЕДА» - превосходная, высшая степень «ПЕРЕМОГИ», её
окончательный и бесповоротный результат.

Семантика слова «ПОБЕДА» имеет мистический, почти сакральный смысл, в ней –
секрет непобедимости русских, их магической способности переломить ситуацию,
добиться успеха  в самых неблагоприятных обстоятельствах. Эмоциональную  силу
«ПОБЕДЫ» не раз использовали в плакатах, лозунгах, призывах и песнях, это
придавало фразам чеканную поступь русского солдата, оказывало колоссальное
психическое воздействие на людей:

- Всё для  фронта, всё для Победы!

- Враг будет разбит, Победа будет за нами!

Не зря самая любимая нашим народом, НАРОДОМ-ПОБЕДИТЕЛЕМ, песня о войне это
«ДЕНЬ ПОБЕДЫ!»

Данная статья была написана в 2013 году. И напечатана в украинском
еженедельнике \enquote{Время UA} в мае 20013г. 

\ii{11_11_2020.yz.pozin_aleksandr.1.peremoga_ne_pobeda.cmt}
