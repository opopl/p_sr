% vim: keymap=russian-jcukenwin
%%beginhead 
 
%%file 29_06_2022.stz.news.ua.donbas24.1.kamjani_polovecki_baby_oberigajut_vid_bombarduvanj
%%parent 29_06_2022
 
%%url https://donbas24.news/news/kamyani-polovecki-babi-donbaski-idoli-oberegi
 
%%author_id demidko_olga.mariupol,news.ua.donbas24
%%date 
 
%%tags 
%%title Кам'яні половецькі баби на Донеччині оберігають жителів від бомбардувань
 
%%endhead 
 
\subsection{Кам'яні половецькі баби на Донеччині оберігають жителів від бомбардувань}
\label{sec:29_06_2022.stz.news.ua.donbas24.1.kamjani_polovecki_baby_oberigajut_vid_bombarduvanj}
 
\Purl{https://donbas24.news/news/kamyani-polovecki-babi-donbaski-idoli-oberegi}
\ifcmt
 author_begin
   author_id demidko_olga.mariupol,news.ua.donbas24
 author_end
\fi

Археологічні артефакти, що знаходяться на території Донбасу, мають власну
історію і для багатьох можуть стати справжнім відкриттям. Проте сьогодні вони
почали відігравати нову, ще більш значущу роль. Так, з початком
повномасштабного вторгнення росії в Україну у донбаських селах їх почали більше
шанувати. Мешканці сіл вважають, що вони завжди мали особливе значення, адже це
пам'ятники сакрального мистецтва, яким поклонялися наші далекі предки. А
сьогодні \emph{жителі сіл приходять до них з проханнями про захист як себе, так і
своїх близьких.}

\begin{leftbar}
\emph{\enquote{Ці ідоли давно захищають мою сім'ю, а кам'яна баба, до якої я постійно
ходжу, врятувала моїх рідних під час бомбардувань в Маріуполі. Я щодня приходжу
до цієї скульптури та прошу її захистити близьких у такий складний час. Це не
просто статуї! Вони мають особливу силу. Я це перевірила на собі багато разів.
Потрібно просто правильно їх просити. У мене багато знайомих, які теж останнім
часом стали до них приходити частіше. Вони дійсно допомагають}}, — розповіла
мешканка смт Тельманове (нині Бойківське) пані Юлія.
\end{leftbar}

\ii{29_06_2022.stz.news.ua.donbas24.1.kamjani_polovecki_baby_oberigajut_vid_bombarduvanj.pic.1}
\ii{29_06_2022.stz.news.ua.donbas24.1.kamjani_polovecki_baby_oberigajut_vid_bombarduvanj.pic.2}

\subsubsection{Історія кам'яних баб}

Не всім відомо, що половецькі скульптури, які в народі мають назву \enquote{кам'яні
баби} і які сьогодні стали для багатьох оберегами, мають унікальне значення і
дають можливість більше дізнатися про середньовічних кочовиків — половців, що
справили величезний вплив на всі сторони економічного, соціально-політичного і
культурного життя Київської Русі.

Згідно з літописами половці вперше з'явилися біля кордонів Русі в 1055 р. і,
уклавши мирний договір, пішли в степ. З цього моменту почалася двохсотлітня
епопея ворожнечі і дружби цього кочового народу з Руссю.

\ii{29_06_2022.stz.news.ua.donbas24.1.kamjani_polovecki_baby_oberigajut_vid_bombarduvanj.pic.3}

Головним джерелом для вивчення історії, побуту, культури половецького народу
були і залишаються їхні статуї — кам'яні баби. Термін \enquote{кам'яні баби} (baba) в
перекладі з тюркського говору — пращур. Також вчені припускають, що слово
\enquote{баба} походить від \enquote{pählaban}, що перекладається з перських діалектів як
\enquote{богатир, атлет}.

На Донбасі число кам'яних баб може наближатися до 1000. На частині з них можна
виявити певні етнографічні риси і, відповідно, приналежність до конкретного
народу. Половецькі статуї (XI — XIV ст.) виготовлені майстрами в місцевих
кар'єрах з граніту, піщанику, вапняку.

\subsubsection{Семантика кам'яних баб}

Фігури (що стоять або сидять) відтворюють канонізований образ чоловіків і жінок
з певним набором прикмет: головним убором, зачіскою, одягом, взуттям,
прикрасами, зброєю. На голові у чоловіків-воїнів, одягнутих в каптан,
півсферичний шолом, а жіночі фігури увінчані капелюхом. Цікаво, що не лише у
жінок, а деколи і у чоловіків підкреслені голі груди — символ сили. Всі статуї
обов'язково тримають біля живота судину в руках. Виходячи з вірувань половців,
він наповнений священним напоєм. Судина символізувала і присутність духа
померлого. З точки зору анімістичних уявлень тюрків, шаман міг укласти душу
померлого в кам'яну статую. Етнографічні відомості дозволяють збудувати
сценарій давніх обрядів \enquote{за участю} статуй. Виготовлена до дня поминань, статуя
встановлювалася біля дерев'яного житла, що стоїть на кам'яній опорі (огорожі),
яка будувалася в першу чергу.

\ii{29_06_2022.stz.news.ua.donbas24.1.kamjani_polovecki_baby_oberigajut_vid_bombarduvanj.pic.4_6}

Статуя була уособленням померлого, який немов був присутнім серед учасників
тризни на власних поминках. Пізніше проводили повторні поминання, після яких
статую розбивали, звільняючи душу померлого, а житло спалювали. Після нього
залишалася огорожа, яку закидали камінням. Так була використана частина ідолів
і на Донбасі, більшість з них мають значні дефекти: втрачена голова, пошкоджені
руки, ноги.

Обличчя деяких жіночих скульптур облямовують \enquote{роги} — вони, як вважають учені,
виконували не тільки роль декору, але й мали магічне значення. Високі
\enquote{капелюхи} з \enquote{рогами} були найбільш святковим вбранням жінок. \enquote{Роги} мали і
смислове значення. Вони були, ймовірно, символом дівчини-нареченої. Наречена у
половців ототожнювалася з бараном, а наречений і його дружки — з вовками,
зобов'язаними наздогнати дівчину. Фігури чоловіків — із зброєю і військовим
спорядженням — це зображення вождів, а сидячі статуї відображують старійшин
родів. У половців був свій пантеон божеств, який також відбивався в статуях.
Наприклад, жіночу статую половці пов'язували з богинею Умай, чоловіча фігура
ототожнювалася з верховним божеством Кам. Половці були язичниками і не
виключено, що вони присвячували скульптури саме пантеону божеств.

Водночас зображення жінок несли різне семантичне навантаження: мати, знатна
пані і навіть войовниця. Є серед статуй одна, яка отримала назву \enquote{Чорнухинської
мадонни} (смт Чорнухине, Луганська область). На ній зберіглося чітке зображення
дитини.

\ii{29_06_2022.stz.news.ua.donbas24.1.kamjani_polovecki_baby_oberigajut_vid_bombarduvanj.pic.7}

\subsubsection{Місце і значення половецьких скульптур в історії та культурі людства}

З історичних джерел до нас дійшли відомості, що кам'яних ідо\hyp{}лів завжди шанували
та боялися. Не даремно класик персидської поезії Нізамі, спираючись на
розповіді своєї дружини-половчанки, відзначав, що всі племена половців
згинаються перед цією єдиною у своєму роді скульптурою, вершник залишає перед
нею стрілу, пастух, який заведе до неї отару, опускає перед нею вівцю. Дійсно,
і на території Приазов'я у підніжжя половецьких скульптур знаходили кістки
баранів, а один раз — кістки дитини — дівчинки. Жіночі статуї обожнювали, їх
боялися, у них просили допомоги. І все ж таки не можливо визначити остаточно,
кому вони ставилися. Відомо, що мешканці Донбасу і до війни вірили у здатність
скульптур оберігати та приносити удачу. Так, у 2011 р. жителі смт Тельманове
(нині Бойківське) не захотіли передавати бабу до музею, боялися, що після цього
від них відвернеться удача і влітку не підуть дощі, які так потрібні для
вдалого врожаю.

\ii{29_06_2022.stz.news.ua.donbas24.1.kamjani_polovecki_baby_oberigajut_vid_bombarduvanj.pic.8}
\ii{29_06_2022.stz.news.ua.donbas24.1.kamjani_polovecki_baby_oberigajut_vid_bombarduvanj.pic.9}

Цікавим є той факт, що кам'яним бабам половців присвячені пісні, картини,
фільми — вони вже давно стали надбанням культури людства.

Сьогодні гостро стоїть проблема збереження половецьких скульптур, які стали
оберегами. На жаль, час руйнує безцінні пам'ятки минулого, якими багаті землі
Донбасу. Необхідно вжити всіх заходів для подальшого збереження половецької
культури, яка уособлює звичаї, релігійні уявлення та неповторне мистецтво вже
неіснуючого народу і є складовою частиною історії України і Донбаського краю.

\ii{29_06_2022.stz.news.ua.donbas24.1.kamjani_polovecki_baby_oberigajut_vid_bombarduvanj.pic.10}

Також, раніше Донбас24 розповідав про \href{https://archive.org/details/27_06_2022.olga_demidko.donbas24.bilosarajskij_majak_symvol_nadii_bezpeky}{\emph{історію найстарішого маяка Північного Приазов'я.}}%
\footnote{Білосарайський маяк: символ надії і безпеки, Ольга Демідко, donbas24.news, 27.06.2022, %
\par\url{https://donbas24.news/news/bilosaraiskii-mayak-simvol-nadiyi-i-bezpeki}, \par%
Internet Archive: \url{https://archive.org/details/27_06_2022.olga_demidko.donbas24.bilosarajskij_majak_symvol_nadii_bezpeky}%
}

