% vim: keymap=russian-jcukenwin
%%beginhead 
 
%%file 27_08_2017.stz.news.ua.mrpl_city.1.istoria_den_shahtera_v_mariupole
%%parent 27_08_2017
 
%%url https://mrpl.city/blogs/view/den-shahtera-v-mariupole
 
%%author_id burov_sergij.mariupol,news.ua.mrpl_city
%%date 
 
%%tags 
%%title История: День шахтера в Мариуполе
 
%%endhead 
 
\subsection{История: День шахтера в Мариуполе}
\label{sec:27_08_2017.stz.news.ua.mrpl_city.1.istoria_den_shahtera_v_mariupole}
 
\Purl{https://mrpl.city/blogs/view/den-shahtera-v-mariupole}
\ifcmt
 author_begin
   author_id burov_sergij.mariupol,news.ua.mrpl_city
 author_end
\fi

Как известно, наш Мариуполь не имеет прямого отношения ни к шахтам, ни к
шахтерскому труду. Разве что когда-то в курортной зоне находился санаторий
центрального комитета профсоюза угольщиков. Здания его и по сей день украшают
Приморский бульвар, но отдыхают ли там шахтеры – вопрос. Почему же все-таки и в
нашем городе (он еще тогда назывался город Жданов) отмечался шахтерский
праздник? Понятно, что в Донецке День шахтера праздновался широко. Во-первых,
в самом большом зрительном зале города собирался народ на торжественное
собрание, где говорились пространные речи, вручались знамена шахтам-победителям
соревнований, а передовикам-шахтерам - награды. На стадионе \enquote{Шахтер} начинался
грандиозный концерт с участием знаменитых артистов.

То ли по желанию городских властей, то ли по инициативе организаторов концерта
в областном центре встреча с мастерами театра, кино и эстрады повторялась в
нашем городе на стадионе \enquote{Металлург}, позже переименованном в \enquote{Азовсталь}. Кто
же  радовал мариупольцев своим искусством? Подшивки местных газет, хранящихся в
Центральной  библиотеке им. В.Г. Короленко, дают возможность установить имена и
коллективы, которые услаждали слух и душу наших земляков своим мастерством
вокала или игрой на музыкальных инструментах.

Итак, погрузимся в историю. 29 августа 1969 года на стадионе \enquote{Металлург} для
жителей нашего города был устроен большой праздничный концерт в честь Дня
шахтера. В нем приняли участие оперные певцы народный артист СССР Юрий  Гуляев,
народная артистка СССР Евгения Мирошниченко, солист Киевского театра оперы и
балета им. Т. Шевченко  Анатолий Соловьяненко, украинский оперный и эстрадный
певец Константин Огневой, а также необыкновенно популярные в то время эстрадные
исполнители Тамара Миансарова, Лариса Мондрус, Валерий Ободзинский. В
представлении также участвовали  вокальный квартет \enquote{4Ю}, Омский народный хор.
Смешил публику талантливый украинский артист театра и кино Андрей Сова.

25 августа 1972 года на стадионе \enquote{Азовсталь}  состоялся большой концерт,
посвященный Дню шахтера, в котором участвовали знаменитый диктор Всесоюзного
радио Юрий Левитан, обожаемая всей страной подлинно народная артистка Клавдия
Шульженко, заслуженная артистка РСФСР, лауреат двух Государственных  премий
Зоя Федорова, артистка театра и кино  Римма Маркова, звезда эстрады Екатерина
Шаврина, уже отстраненный от выступлений по Центральному телевидению, но еще
допущенный к гастрольным концертам Вадим Муллерман, мастера парного конферанса
Лев Миров и Марк Новицкий. А также Борис Владимиров и Вадим Тонков в образе
старушек Авдотьи  Никитичны и Вероники Маврикиевны, представители циркового
искусства.

30 августа 1975 года как обычно стадионе \enquote{Азовсталь} в концерте, посвященном
Дню шахтера и сорокалетию стахановского движения, пели обожаемые народом Юрий
Богатиков, Лев Лещенко, Геннадий Белов, миниатюры совсем молодого Геннадия
Хазанова вызывали гомерический смех. Пела и Алла Пугачева, в то время она имела
лишь один почетный титул – лауреат конкурса \enquote{Золотой Орфей}. Интересно, что в
газетном отчете ее назвали после почти забытых теперь эстрадных певиц Нины
Бродской, Светланы Резановой, Аллы Абдаловой.

27 августа 1981 года на стадионе \enquote{Азовсталь} в концерте, посвященном Дню
шахтера, выступили популярный  актер театра и кино, народный артист СССР,
лауреат двух Государственных премий Борис Андреев,  любимец  публики,  актер
театра и кино народный артист СССР, лауреат Государственной премии СССР
Евгений Леонов, диктор Всесоюзного радио, народный артист СССР Юрий Левитан,
народная артистка РСФСР известная эстрадная певица Ольга Воронец,  молдавская
солистка-вокалистка Мария Кодряну, популярная  джазовая и фолк-певица, актриса
кино Ирина Понаровская, другие артисты. Вел концерт Олег Марусев.

Конечно, здесь приведены воспоминания далеко-далеко не обо всех концертах
мастеров искусств, которые выступали перед публикой приморского города.  Были и
другие встречи такого рода на стадионах, в Летнем театре Городского сада и в
драматическом театре, во дворцах культуры. 
