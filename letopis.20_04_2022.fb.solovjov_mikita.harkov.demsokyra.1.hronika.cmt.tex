% vim: keymap=russian-jcukenwin
%%beginhead 
 
%%file 20_04_2022.fb.solovjov_mikita.harkov.demsokyra.1.hronika.cmt
%%parent 20_04_2022.fb.solovjov_mikita.harkov.demsokyra.1.hronika
 
%%url 
 
%%author_id 
%%date 
 
%%tags 
%%title 
 
%%endhead 
\zzSecCmt

\begin{itemize} % {


\iusr{Roman Frolov}
Сутки или около того не бьіло воздушной тревоги.

\iusr{Оксана Масалітіна}
\textbf{Roman Frolov} и у нас в Киеве

\iusr{Виталий Буняев}
Ну да, 35-40\% населения, похоже....

\iusr{Тетяна Уразова}
Даёшь метро!!!

\iusr{Людмила Колесник}
Спасибо!

\iusr{Тетяна Уразова}
Если пустят метро -я вернусь!

\iusr{Назар Гнатович}

як можна запускати метро, якщо воно де-факто саме безпечне місце в місті?
Наступні руйнації будинків вимагатимуть надання прихистку новим постраждалим

\begin{itemize} % {
\iusr{Tetyana Sfandex}
\textbf{Назар Гнатович} 

це ж очевидно - для безпечного пересування містом  @igg{fbicon.shrug} 🏻 метро годиться як
тимчасове бомбосховище, але не годиться як житло.

\iusr{Andriy Trushevskyi}

Назар Гнатович, в Києві метро працювало і продовжувало виконувати функції
укриття - поїзди їздили у човниковому режимі по одній колії. А на іншій
залишалися потяги для розміщення людей

\iusr{Назар Гнатович}
\textbf{Андрій Трушевський} коли запрацювало метро в Києві, то вже не було таких масових руйнацій будинків, як у нас

\iusr{Назар Гнатович}
\textbf{Tetyana Sfandex} логічно, а тепер запропонуйте аналогічне безпечне місце замість метро для тих, хто там зараз?

\iusr{Надія Лабик}
\textbf{Назар Гнатович} Польща?

\iusr{Назар Гнатович}
\textbf{Надія Лабик} тобто Тєрєхов має сказати всім тим, хто в метро шуруйте в Польшу?)

\iusr{Надія Лабик}
\textbf{Назар Гнатович} 

тим, в кого зруйновані квартири, та чиї райони постійно обстрілюють, краще
виїхати з міста. До Польщі, Румунії, на Західну Україну чи просто до Дніпра.
Ніхто не знає, скільки ще буде йти війна, і жити невизначений термін в метро -
така собі ідея.

\iusr{Назар Гнатович}
\textbf{Надія Лабик} 

ну міський голова ніколи не зробить такий заклик, бо він тоді продемонструє
свою слабкість. В ідеалі він має запропонувати альтернативу і сказати: "Шуруйте
з метро в інше обладнане надійне бомбосховище за такою адресою. Однак у нашому
місті такого нема(

\iusr{Roman Frolov}
\textbf{Назар Гнатович} 

Метро ж не може розмістити усіх.

Більшість населення розташована надто далеко щоб туди оперативно сховатись

Але у Харкові залишается півмільона людей - та не всі сидять по домах чи
сховищах, а транспорт зараз у нас відсутній. Відстані великі - якщо на роботу
треба то або таксі або пішки. Метро б тут дуже допомогло.

\iusr{Назар Гнатович}
\textbf{Roman Frolov} 

у нас совкове калічне місто без бомбосховищ і Тєрєхов не ризикне вигнати з
метро людей. Куди він їх відправить? До школи? Завтра туди прилетить снаряд і
що потім йому робити? З його вини загинули люди - всюди казатимуть))

\iusr{Roman Frolov}
\textbf{Назар Гнатович} так у метро вже не помістяться нові бажаючі ночувати там.

\iusr{Назар Гнатович}
\textbf{Roman Frolov} так дотепер усіх, хто втратив житло розміщують у метро

\iusr{Микита Соловйов}
\textbf{Назар Гнатович} 

Отвечал уже примерно 100500 раз. Потому что самое безопасное место в Харькове
во Львове или Польше. Если ставить во главу угла безопасность и более ничего,
то в принципе нет никакого смысла сидеть в Харькове. А для любой осмысленной
деятельности в городе запуск самого надежного, эффективного и безопасного
общественного транспорта нужно как воздух.

\iusr{Назар Гнатович}
\textbf{Микита Соловйов} ну в принципі згоден, що в даній ситуації проблема в Тєрєхові, він мав би прямим текстом їм указати шлях на Захід і запустити метро

\iusr{Roman Frolov}
\textbf{Назар Гнатович} 

є ще потяги безкоштовні кожен день, якщо виїхало 2\textbackslash 3 міста то виходить не усіх
у метро розміщують. Просто у метро годують, не треба у черзі за гуманітаркою
стояти декілька годин або їхати кудись шукати прихисток.

\iusr{Назар Гнатович}
\textbf{Roman Frolov} це слабкість Терехова, я як раз з його позиції це коментував

\iusr{Светлана Прокопенко}
\textbf{Назар Гнатович} а хто сказав, що людей потрiбно виганяти.
У столицi того не робили, але метро працюэ.

\iusr{Tetyana Sfandex}
\textbf{Назар Гнатович} практично вся Західна Україна, плюс деякі села Полтавської області

\iusr{Назар Гнатович}
\textbf{Tetyana Sfandex} ну не всюди безкоштовно, подекуди треба й платити за житло

\iusr{Tetyana Sfandex}
\textbf{Назар Гнатович} так і в Харкові житло не було безкоштовне.

\iusr{Назар Гнатович}
\textbf{Tetyana Sfandex} хто жив у власній квартирі могли роками ні за що не платити - значить жити теоретично можна безкоштовно

\iusr{Tetyana Sfandex}
\textbf{Назар Гнатович} комуналка і всякий ремонт все одно.
Якщо шукати безкоштовно, можна щось знайти, без комфорту. За комфорт треба платити.
«Безкоштовно» це завжди за рахунок когось іншого, шкода що не всі ставлять собі питання, за рахунок кого саме.

\iusr{Nina Homa}

Що за дурниці Ви пишете? Хто живе у власній квартирі повинен оплачувати
квартплату, комунальні послуги (електрика, газ, водопостачання), ремонти і тому
подібне. \enquote{Безкоштовно} - це значить, що хтось злісний не платник за спожиті
послуги і товари. Дуже дивна точка зору.

\iusr{Назар Гнатович}
\textbf{Nina Homa} я пишу про реалії, у мого сусіда борг за тепло 33000. Мені слід мораль йому прочитати й засудити?))

\end{itemize} % }

\iusr{Попков Максим}
Сьогоди в день так тіхо було... але к ночи почалося.

\iusr{Timofeeva Natalia}

слышала мнение что они как то отслеживают скопление сигналов мобильных
телефонов и туда бьют - очереди за гуманитаркой к примеру.. это я по поводу
освячения пасок

\iusr{Viktoria Belyavtseva}
Героям Слава!

\iusr{Наталья Акименко}

По мнению Терехова в метро 15 ООО постоянных жителей приходящих только ночевать
не считают. И вот вопрос неужели нельзя как-то их организовать? Думаю ответ в
головняке, пусть сидят, а то думай лишний раз как всё проконтролировать, а так
можно и репортаж из метро снять и волонтёры их всем обеспечат и для горвластей
меньше проблем, вот сезон фонтанов открыть это важнее важного

\begin{itemize} % {
\iusr{Roman Frolov}
\textbf{Наталья Акименко} ну может хоть фонтанами вьіманят ненадолго из метро  @igg{fbicon.smile} 

\iusr{Назар Гнатович}
\textbf{Наталья Акименко} які в харкові ще є підземні споруди, здатні витримати бомбовий удар?

\iusr{Tetyana Sfandex}
\textbf{Назар Гнатович} 

їм туди привозять їжу явно з західних регіонів або взагалі гуманитарку з-за
кордону. Чим годувати 15 тисяч людей, щодня возячи продукти за тисячі
кілометрів, може краще один раз перевезти цих людей на ці тисячі километрів, і
годувати їх там?  @igg{fbicon.face.rolling.eyes} 

\iusr{Marianna Markova}
\textbf{Tetyana Sfandex} саме так, й гостро стає питання здоров’я тих, що мешкають у метро.

\iusr{Микита Соловйов}
\textbf{Наталья Акименко} По описаниям волонтеров, людей заметно меньше.

\iusr{Roman Frolov}
\textbf{Микита Соловйов} 15 тыс на 30 станций получается в среднем по 500 человек на каждой

\iusr{Микита Соловйов}
\textbf{Roman Frolov} Спасибо, я арифметикой владею )

\iusr{Roman Frolov}
\textbf{Микита Соловйов} 

Сорри, не дописал сразу все мысль. Я не уверен что даже столько можно
разместить на большинстве станций, так что я согласен что меньше (и думаю что
не все там согласны долго сидеть и часть все же уезжают или возвращаются наверх
понемногу)

Только те, кто лишился всего и у кого теперь с психикой проблемы. Другая группа
- просто те, для кого метро и бесплатная кормежка это лучше чем та жизнь, что
они вели.

Мы можем бесконечно рассуждать, а сможем ли как-то повлиять на решение мера и
ОВА, что работающее метро лучше чем только убежище, надеюсь что скоро будут
рассматривать такой вопрос все же. Там же и технические вопросы уже начинаются
скорее всего - как сохранить работоспособное состояние путей и подвижного
состава.

\iusr{Микита Соловйов}
\textbf{Roman Frolov} Повлиять на решение я сейчас вижу возможность только публичными методами.

\iusr{Микита Соловйов}
\textbf{Roman Frolov} А, и поговорите с психологами. Мне вот пятеро сказали одно и тоже. У сидящих в подземелье больше месяца, психологические проблемы начинаются у всех. Ну как минимум, у подавляющего большинства.

\end{itemize} % }

\iusr{Igor Prykhodko}

...а у нас как раз Песах заканчивается. И жизнь общины, хотя и поредевшей,
продолжается. Тоже повод для денацификации...

\iusr{Yulia Denisyuk}

Люди из метро не уезжают, потому что им некуда ехать и не за что там жить
(учитывая, что уезжать прийдется на несколько месяцев). Не все так просто, как
вы, Никита, пишите.

Люди надеются, что скоро закончатся обстрелы, и Они смогут вернутся в свои
дома. Или вообще некуда возвращаться. Но поехать неизвестно куда и с пустым
карманом, чтобы в итоге там стать бомжами, многие не решаются. По поводу
заграницы - многие не хотят уезжать без мужей и разрушать свои семьи.

Это причины, которые я слышала.

\begin{itemize} % {
\iusr{Sergiy Ryabykin}
\textbf{Yulia Denisyuk} 

Якби їх вивезли напр. в дома відпочинку в Полтавській обл - чим не рішення? Так
само годували б, зато на повітрі і без обстрілів.

\iusr{Ольга Белецкая}
\textbf{Yulia Denisyuk} 

ну вот допустим, война будет длиться еще 10 месяцев. И что, люди будут год жить
в метрло? А те у кого, как выше писалось, вообще нет дома те будут жить до
постройки нового дома?

А медсестра добраться с холодной горы в 17 больницу не сможет. Все эти условные
2 года стройки....

Так что ли?

\iusr{Валерия Величко}
\textbf{Ольга Белецкая} 

нет конечно, закончится война и будет возможность заняться, этих жителей
поместят в гостиницы и ТП, привезут модульные домики, организуют городки и тд

\iusr{Tetyana Sfandex}
\textbf{Валерия Величко} 

модульні будинки вже зараз будуються (деякі побудовані) у Львові та інших
містах. Війна не закінчиться ні завтра ні через тиждень. Ольга має рацію,
скільки років люди збираються жити в метро?

\iusr{Tetyana Sfandex}
\textbf{Yulia Denisyuk} обстріли скоро не закінчаться.

«Через два тижні» ми всі чули ці 2 місяці. Воно може тривати ще рік або більше.
Може з меншою інтенсивністю, але найближчим часом немає підстав очікувати що
закінчиться.

\iusr{Marianna Markova}
\textbf{Yulia Denisyuk} 

ну дивиться, волонтери зара підхоплюють людей на усіх вокзалах Західної
України. харчами та прихистком на кілька днів забезпечують (так, це школи,
садочки чи гуртожитки, але там все-дно краще ніж у метро). Це дає час на
\enquote{оговтатися та прийняти рішення}. Я знаю що з постійним поселенням все
складніше, але таки розселяють людей, причому безоплатно. Й так, для
пенсіонерів та сім’ях з дітьми виїхати у Західну Україну чи може за кордон -
то тимчасовий гарний вихід, ліпше ніж пересижувати в метро.

\iusr{Микита Соловйов}
\textbf{Юлия Денисюк} 

Так вот нет, скоро обстрелы не закончатся. Это то самое пережидание, которое
было вполне понятным в первые неделю-две, но которые уже становятся абсолютно
нерациональными через почти два месяца. Так что или принять этот риск и
вернуться в дома. Или оценив этот риск уезжать в более спокойные места. Но в
любом случае, получить помощь (а ее сейчас предоставляют для эвакуации более
чем) и принять на себя ответственность за себя. Что касается тех, кому некуда
возвращаться, то тем более вопрос что делать в метро. Они ждут там когда что?

\iusr{Roman Frolov}
\textbf{Валерия Величко} в Харькове есть модульный городок, там люди 8 лет прожили и он пришел в негодность, их уже собирались выгонять оттуда

\iusr{Валерия Величко}
\textbf{Roman Frolov} 

значит будут новые, жаловаться друг другу не время, надо объединяться и
действовать, добиваться, сейчас людей оставшихся без жилья много, выбрать совет
и уполномочить их заниматься конкретно программой жилья

\iusr{Ольга Белецкая}
\textbf{Валерия Величко} 

в Харькове вопрос временного жилья не актуален. Снять можно и сейчас, ждать
пока построят или дадут что то бесплатное можно годами.

Кто вот прям СЕЙЧАС отался без жилья и живет в метро надеясь на модульный домик
послезавтра, мягко говоря вызывает сомнения в своей адекватности.

\end{itemize} % }

\iusr{Тетяна Сторожко}
Дякую за інформацію! Рідне місто, тримайся @igg{fbicon.hands.pray} 

\iusr{Светлана Белоус}
без метро, блин, полнаяж.... па! МЕТРООО!

\iusr{Светлана Белоус}

недавно был хвалебный пост, как туда гуманитарные кровати привезли. а следующий-
театры там дают постановки. суперпиар! а горожане пешком и автостопом по
полгорода катаются. и не считают чем-то опасным. выбора- то нет... и еще... в
военное, да и в любое другое, хозяин таксопарка должен получить разрешение на
работу в городе? и кто его должен дать? и за какие бабки? разве не опасно в
городе? прилеты, прилеты... кто на этом заработал? город полон такси ,только у
кого есть средства на них за мизерную зарплату каждый день ездить? вопросы сами
возникают...

\iusr{Татьяна Логвинова}
Я бы сказала, что церкви лучше ВСЕГДА обходить по большому радиусу.

\iusr{Yulia Vepritskaya}
Всё так.

\iusr{Анастасія Шиян}
Дякую, що пишете.

\iusr{Володимир Скорик}

Я теж за курсування метра, але є підозра, що не хочуть запускати його для того,
аби не почали повертатися масово в місто. Вже і так, знаю, почали повертатися,
але ще дуууже рано.

\iusr{Elena Borisova}
А вот некоторые \url{https://www.facebook.com/100005755987411/posts/2137959229739256/?d=n}

\end{itemize} % }
