% vim: keymap=russian-jcukenwin
%%beginhead 
 
%%file 10_04_2017.stz.news.ua.mrpl_city.1.gigantskie_pysanki_iz_mariupolja_ukrasili_kiev
%%parent 10_04_2017
 
%%url https://mrpl.city/news/view/gigantskie-pisanki-iz-mariupolya-ukrasili-kiev-foto
 
%%author_id news.ua.mrpl_city
%%date 
 
%%tags kiev,mariupol,pysanka
%%title Гигантские писанки из Мариуполя украсили Киев (ФОТО)
 
%%endhead 
 
\subsection{Гигантские писанки из Мариуполя украсили Киев (ФОТО)}
\label{sec:10_04_2017.stz.news.ua.mrpl_city.1.gigantskie_pysanki_iz_mariupolja_ukrasili_kiev}
 
\Purl{https://mrpl.city/news/view/gigantskie-pisanki-iz-mariupolya-ukrasili-kiev-foto}
\ifcmt
 author_begin
   author_id news.ua.mrpl_city
 author_end
\fi

10 квітня 2017 в 09:32

Центр столицы украсили 7 писанок из Мариуполя, некоторые из которых в высоту
достигали одного метра, передаёт MRPL.CITY.

На Софийской и Михайловской площадях в Киеве в рамках всеукраинского фестиваля
Писанок 2017 выставлено 585 расписанных яиц, среди которых гигантские писанки
из Мариуполя.

\ii{10_04_2017.stz.news.ua.mrpl_city.1.gigantskie_pysanki_iz_mariupolja_ukrasili_kiev.pic.1}

Как сообщила куратор проекта от Мариуполя – руководитель фолк-клуба \enquote{Макошь}
Светлана Мешкова-Давиденко, в этом году наш город на фесте кроме неё
представляли восемь художников: Подтыканова Наталья, Зоренко Оксана, Чебанова
Татьяна и четверо юных художников - Мешкова Анастасия, учащаяся 7-А класса
школа №10, Мешков Сергей, учащийся 1-Г класса школы №10, Подтыканов Петр,
учащийся 6-А класса и Подтыканов Владислав, учащийся 2-А класса  школы №61. Они
представили для участия в арт-перформансе семь своих произведений.

Светлана Мешкова-Давиденко сообщила, что для участия в выставке были
реализованы три направления росписи: традиционное, детское и авторское.

Самыми большими по размеру были три авторские писанки метровой
высоты.Участникам проекта бесплатно выдавались три лёгкие заготовки из
стекловолокна и краски. После окончания росписи и фотосессии авторы отослали
свои работы почтой в Киев.

\ii{10_04_2017.stz.news.ua.mrpl_city.1.gigantskie_pysanki_iz_mariupolja_ukrasili_kiev.pic.2}

Как уточнила Светлана Мешкова-Давиденко, открытие выставки планировалось на
субботу, 8 апреля. Но, в связи с ненастной погодой, его перенесли на сегодня.

Участники фестиваля выразили надежду, что подобное мероприятие будет
организовано в Мариуполе: \enquote{Хотелось, чтобы в нашем городе проводилось что-то
подобное фестивалю в Киеве, так как подобные мероприятия очень сближают}.
