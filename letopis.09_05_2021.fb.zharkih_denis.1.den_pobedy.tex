% vim: keymap=russian-jcukenwin
%%beginhead 
 
%%file 09_05_2021.fb.zharkih_denis.1.den_pobedy
%%parent 09_05_2021
 
%%url https://www.facebook.com/permalink.php?story_fbid=2983601281853269&id=100006102787780
 
%%author 
%%author_id zharkih_denis
%%author_url 
 
%%tags den_pobedy,kiev,prazdnik,prazdnovanie,ukraina,zelenskii_vladimir
%%title Непопулярные мысли - День Победы
 
%%endhead 
 
\subsection{Непопулярные мысли - День Победы}
\label{sec:09_05_2021.fb.zharkih_denis.1.den_pobedy}
 
\Purl{https://www.facebook.com/permalink.php?story_fbid=2983601281853269&id=100006102787780}
\ifcmt
 author_begin
   author_id zharkih_denis
 author_end
\fi

Непопулярные мысли. 

Не знаю, как где, а я в праздновании Дня победы в Киеве почувствовал усталость
с двух сторон. Ну, это мои личные ощущения. И нацыки какие-то общипанные и
немногочисленные, и ватники какие-то усталые и озабоченные.  У тех, что шли с
цветами, усталая решимость, для них идеологические противники просто назойливые
мухи, элемент ландшафта. Те, что традиционно пришли мешать и обвинять в
предательстве, за столько лет проорали себе горло, что оно как-то дальше не
орется.  

Если пару лет назад вполне могли быть кровавые провокации, то теперь война
дышит каждому в затылок, и положить себя на алтарь этой войны не спешит никто.

Все серьезно, комедия подходит к концу. Мода на камуфляж, берцы и
патриотические вопли уходит. Остаются проблемы, которые могут привести к
дальнейшей крови, но кураж заканчивается. Война это работа, к работе господа
патриоты не приучены. Зато ватники пообвыклись к ним, все меньше боятся, все
больше презирают. Это не драматическое презрение, достойное полотна Бондарчука
старшего, а презрение работяги к местной шпане, которая спокойно жить не дает.

А политической шпане начинает доходить ее роль, что чуть что хозяин пошлет на
пушечное мясо. И нет никакого тыла. Эти мрачные люди не полюбят их ни за что, а
ненавидеть давно устали. Между собой нацыки давно переругались и тоже тихо
ненавидят друг друга. 

Обвинять ватников в пособничестве России тоже не выходит. В России выходить
возлагать цветы, скорее, поддерживать власть, в Украине однозначно
протестовать. 

Ватникам требуется намного больше мужества возлагать цветы, чем нацыкам
осквернять могилы. И это мужество у народа никуда не делось, а вот кураж
нацыков спадает. Из народа они ушли, а в начальство не попали - типичная доля
полицая, который предал своих, но был предан сам. 

Примерно такая же усталая ненависть ждет Зеленского. Он ведь тоже ни туда, и ни
сюда. Он ничей, никому тут не нужен, не смешной, не теплый, а просто досадное
недоразумение, которому скоро придет конец.

\ii{09_05_2021.fb.zharkih_denis.1.den_pobedy.cmt}
