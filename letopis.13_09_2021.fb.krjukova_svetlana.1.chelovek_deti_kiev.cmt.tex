% vim: keymap=russian-jcukenwin
%%beginhead 
 
%%file 13_09_2021.fb.krjukova_svetlana.1.chelovek_deti_kiev.cmt
%%parent 13_09_2021.fb.krjukova_svetlana.1.chelovek_deti_kiev
 
%%url 
 
%%author_id 
%%date 
 
%%tags 
%%title 
 
%%endhead 
\subsubsection{Коментарі}

\begin{itemize} % {

\iusr{Зинаида Маслова}
Светлана, спасибо большое. Всегда с удовольствием читаю Ваши зарисовки. Действительно - уютно и тепло.

\iusr{Svetlana Kryukova}
\textbf{Зинаида Маслова}  @igg{fbicon.heart.eyes} 

\iusr{Яна Бойко}
Так уютно и тепло пишите! Хочется перечитывать вслух  @igg{fbicon.face.relieved} 

\iusr{Svetlana Kryukova}
\textbf{Яна Бойко}  @igg{fbicon.heart.red}

\iusr{Denys Bezliudko}
Бросай журналистику, иди в писатели! )

\begin{itemize} % {
\iusr{Svetlana Kryukova}
\textbf{Denys Bezliudko} а лучше сразу в мечтатели 	@igg{fbicon.face.smiling}

\iusr{Denys Bezliudko}
\textbf{Svetlana Kryukova} в философы. Ты хотела сказать в философы )
\end{itemize} % }

\iusr{Анни Атикетпеш}
Здорово!
Крошки достаются воробьям, крупные куски птицам по- крупнее, а птицы покрупнее попадаются уже Орлам)

\iusr{Лидия Сазонова}
Света, спасибо Вам большое за рассказ.Очень хорошо написано.

\iusr{Marta Krivska}
Благодарю!

\iusr{Элина Слободянюк}
Ужасное кафе. Шум жуть. Лучше вглубь зайти: one love. Или в Боссано подняться

\begin{itemize} % {
\iusr{Стас Шведюк}
\textbf{Элина Слободянюк} мозги вправь прежде что то писать уже не знаешь куда подкатить лиж бы не работать
\end{itemize} % }

\iusr{Леша Сокол}
@igg{fbicon.face.smiling}

\iusr{Инна Иванова}
А потом стайка голубей взлетает и благодарно метит своих кормильцев.

\iusr{Вячеслав Кедровский}

Как всегда хорошо написано тэнкс. Ходили бы себе Светлана Крюкова да
наговаривали на диктофон. Сколько бы зарисовок красивых получилось. Аудио
выложила ежедневно и все дела да людям удобнее. А экономия времени? А время это
чё? Правильно. А лучше стрим 7/24. Чё я бы посмотрел. Думаю не я один такой.
Нравитесь-то многим да многим и не нравитесь. Те тоже будут смотреть.


\iusr{Sergei Borisov}
Ну тут немного ...: "...крошишь мелкими кошками..." Скорее - "крохами".

\end{itemize} % }
