% vim: keymap=russian-jcukenwin
%%beginhead 
 
%%file 11_11_2021.fb.fb_group.story_kiev_ua.2.pamjat_rahlina_eleonora.cmt
%%parent 11_11_2021.fb.fb_group.story_kiev_ua.2.pamjat_rahlina_eleonora
 
%%url 
 
%%author_id 
%%date 
 
%%tags 
%%title 
 
%%endhead 
\subsubsection{Коментарі}

\begin{itemize} % {
\iusr{Natasha Levitskaya}

Спасибо, Георгий, за этот пост памяти о благородных и душевно-богатых
Киевлянах!

К сожалению, таких людей уже нет, их уже не производят...

Чем больше читаешь о Леоноре Натановне, чем больше узнаешь от людей, которым
повезло встречаться и общаться с нею, тем больше восхищаешься этой невероятной
женщиной, которая так любила свой Город, знала его и дарила свои знания людям.

Очень понравился один комментарий, написанный 6 лет назад о ней: "Лена
настолько приподнимала нас над обыденностью, что порой мы чувствовали себя как
бы персонажами Шагаловских картин!"

В свете недавних и горячих споров в нашей группе о переименовании улиц, о
героях, о новоявленных друзьях из далёких стран, чьими именам называют нынче
улицы Киева, хочется надеяться, что всё-таки там "наверху" найдутся разумные
люди и в Киеве, не на окраинах среди новоделов, появятся улицы знаменитых
киевлян, и поколение тик токов и Инстаграм будет знать хот бы так о великих
людях города.

Светлая и добрая память Вам, Элеонора Натановна!

\begin{itemize} % {
\iusr{Rimma Turovskaya}
\textbf{Natasha Levitskaya} Те, которые переименовывают, не любят Киев и не знают наших великих земляков.
\end{itemize} % }

\iusr{Maria Rasin}

Потрясена ! Натан Рахлин - фамилия с которой я прожила всю жизнь в Киеве и
такие унижения, чтобы повесить эту несчастную доску таким именам ? Стыд и позор
! Люблю свой город, но после некоторых историй настоящее отравление .....


\iusr{Валентина Татаржинская}

Дуже цікаве оповідання. Я згадала це прізвище деригента. А його історії раніше
не знала. Дякую


\iusr{Алик Старшой}
Спасибо за такой пост....

\iusr{Наталия Руденко}
Несправедливо, когда величие зависит от бездарности...

\iusr{Наталия Калатозишвили}

\ifcmt
  ig https://scontent-frx5-2.xx.fbcdn.net/v/t39.1997-6/p480x480/91521538_1030933857302751_5093925307199520768_n.png?_nc_cat=1&ccb=1-5&_nc_sid=0572db&_nc_ohc=E-EmWZR18LIAX_MAwr4&_nc_ht=scontent-frx5-2.xx&oh=703a04bbbfef9f21c4879806c99d36e0&oe=61B1D8F7
  @width 0.2
\fi

\iusr{Елена Мельникова}
Потрясена! Сколько в городе не очень нужных памятников, а патриоту доску нельзя было открыть!

\iusr{Ольга Ткачук}
обычный НЕОБЫКНОВЕННЫЙ человек, спасибо что хранили Киев @igg{fbicon.rose} 

\iusr{Olga Lys}
До сліз...

\iusr{Валентина Рябчук}

\ifcmt
  ig https://scontent-frx5-2.xx.fbcdn.net/v/t39.1997-6/s168x128/17634213_1652591098100624_731967241620291584_n.png?_nc_cat=1&ccb=1-5&_nc_sid=ac3552&_nc_ohc=JUTvezKrB9oAX8MUi61&_nc_ht=scontent-frx5-2.xx&oh=b53ed3d1c5d31123920e94b0733e5723&oe=61B21237
  @width 0.1
\fi

\iusr{Валентина Рябчук}
Светлая память....

\iusr{Марина Волович}
Большое спасибо за этот пост, за интервью. Была на ее экскурсии по Подолу. Осталась память на всю жизнь . Будем помнить .

\iusr{Ирина Кедрова}

Спасибо, Жора! Вернулась в юность... Очень хорошо помню и экскурсии и лекции
её. Это было великолепно! \enquote{Заразила} она меня так, что со временем я закончила
курсы экскурсоводов. Правда, екскурсии проводила только для учеников и
знакомых @igg{fbicon.face.smiling.eyes.smiling} 

\begin{itemize} % {
\iusr{Георгий Майоренко}
\textbf{Ирина Кедрова} А ты должна помнить и ее сына Игоря Лысенко. Он же бы когда-то на презентации моей книги.

\iusr{Андрей Саенко}
\textbf{Георгий Майоренко} Ой, кого там только не было - всех разве упомнишь?

\iusr{Георгий Майоренко}
\textbf{Андрей Саенко} И ты там был! Мед-пиво пил!

\iusr{Андрей Саенко}
\textbf{Георгий Майоренко} ...и сладко кушал, и песни слушал.

\iusr{Георгий Майоренко}
\textbf{Андрей Саенко} И пророчествовал!
\end{itemize} % }

\iusr{Ирина Иванченко}

Я не знала, к сожалению, Элеонору Рахлину как экскурсовода, но мне очень по
душе её стихи. Талантливая женщина. Спасибо за рассказ.


\iusr{Наташа Драпушко}

Я была на экскурсии по делу Бейлиса, у нее дома на квартирнике. Возле дома
дежурил дворник и спрашивал куда идете. Ответ: день рождения у Елены Натановны.


\iusr{Людмила Білоус}

Була знайома з Елеонорою Натанівною, незадовго до її смерті передала до
створеного нею Фонду Натана Рахліна цінні архівні документи. Що з цим Фондом і
документами зараз невідомо, адже не стало вже і сина - Ігоря Лисенка, який
проживав у Криму.

\iusr{Валентина Можар}
Дякую, що описали її особисте життя, не знала.
Але на її екскурсії ходила, було цікаво.
Царство Небесне! !!!

\iusr{Лариса Давиденко}
Дякую вам за чудовий допис!!!

\iusr{Iryna Kovalenko}
Какие у нее были экскурсии!!! Час стоишь, слушаешь, как секунда!!! Уникальный человек!

\iusr{Роза Щупак}

Спасибо большое за эту историю. Много лет ходила с Элеонорой Натановной по
Киеву и просто посещала те лекции,которые она проводила. Она была настоящий
патриот Киева и настоящий ЧЕЛОВЕК! Все события по установке доски-памяти я
помню не по наслышке.

\iusr{Kateryna Vinn}
Дякую за історію про Київ і киянку. Не даремно я тут живу.

\iusr{Виктория Крижановская}
Вечная память. Спасибо всем кто помнит.

\iusr{Ірина Громова}
Дякую за цю цікаву історію. Світла пам'ять чудовій людині

\iusr{Людмила Бабич}

Благодарю за достойный экскурс в жизнь замечательной женщины! Светлая Память!
Жаль, что столь ничтожная нынешняя \enquote{элита} ничего не ценит, особенно историю
Киева и его достойных жителей...


\iusr{Ольга Лубягина}
Спасибо! Помню её великолепные экскурсии. Потрясающий знаток истории. Киева. Светлая память

\iusr{Юля Рясенчук}
Спасибi. Цiкаво.

\iusr{Inna Maistrouk}

Прекрасный человек. Но неужели доска может стоить как квартира,? ..жулики..
тогда многие потеряли квартиры общаясь с такими вот делками..Киев родной родные
киевляне

\iusr{Наташа Котляренко}

Она права! Она заразила и меня болезнью - интерес к истории Киева. Помню ее
экскурсии - она много курила папиросы и чтобы не сорить, прикурив, спичку
засовывала в фильтр и так докуривала. Странная, и потому талантливая была
женщина. Память о ней в Киеве сохранится надолго....... @igg{fbicon.hands.applause.yellow}{repeat=3} 

\iusr{Валентина Можар}
\textbf{Наташа Котляренко} 

Дійсно так було. Я була теж здивована, що вона купила папіроси, але не зважаючи
на це, лекції в КЛІО та екскурсії по Києву були захоплюючі, про кожен будинок,
про героїв з твору Булгакова і т.д.

\iusr{Людмила Бобошко}


\iusr{Татьяна Миронова}

Какой замечательный рассказ. Сохранить память о выдающихся согражданам -святое
дело

\ifcmt
  ig https://i2.paste.pics/963984023fddb2118b0e31cc92a2feba.png
  @width 0.4
\fi

\iusr{Татьяна Миронова}
Какой замечательный рассказ. Сохранить память о выдающихся согражданам -святое дело

\iusr{Марина Бурд}

Много раз ходила с ней по городу. Экскурсии были просто волшебные. И все как
написано - вдруг звонит телефон и кто то сообщает - тогда то и тогда то Рахлина
ведёт.


\iusr{Людмила Тищенко}

В конце 80-х я была всего лишь 2 раза на ее лекциях по архитектуре, уникальная,
запоминающаяся личность. Светлая память!

\iusr{Кретов Андрей}

Огромнейшее спасибо!!! Трогательный рассказ о прекрасной, талантливой и неравнодушной женщине и ее гениальном отце.
Я не коренной киевлянин, поэтому многое для меня - откровение.
Это рассказ не только о подвижничестве и жертвенности личности, но и (увы!) о преступном и подлом равнодушии власти.
Вспоминаешь невольно Миколу Зерова и его слова про "наші підлі і скупі часи"...

\iusr{Альбина Орел}

...сразу слышу её голос, в руках неизменная Прима, в которую она втыкает спичку, в глазах огонек: 'Сейчас я вам такое расскажу!"
Мне повезло, я была в группе, где Рахлина читала лекции и проводила экскурсии по Киеву. Остались конспекты...
Светлая память! @igg{fbicon.hands.pray}  @igg{fbicon.hearts.two} 

\iusr{Світлана Куликова}

\ifcmt
  ig https://i2.paste.pics/0f9383e7c8e52035d4804ae1e0193fd2.png
  @width 0.4
\fi

\iusr{Олег Верхоградов}

Да. Интересная женщина. Доводилось видеть ее неоднократно и бывать у нее в
квартире по работе. Я тогда работал в Фонде гуманитарного развития "Коллегиум".


\iusr{Marina Lentz}
Какая чудесная и насыщенная жизнь.

\iusr{Sem Zilman}

Ната́н Григо́рьевич Ра́хлин (отец!) был много лет Дирижёром Симфонического
Оркестра Украины, а Первой скрипкой был — Натан Кушнир (мой дядя, он жил на
Бессарабке)...

Семья Рахлиных:

Дочь — Элеонора Натановна Рахлина (1934—2006), экскурсовод, краевед

Внук — Игорь (Григорий) Альбертович Лысенко (род. 1964), украинский скульптор.
Член Национального союза художников Украины (с 1994).

Двоюродный брат — Леопольд Матвеевич Рахлин (1898—1994), учёный-медик (терапевт
и кардиолог), доктор медицинских наук.

\begin{itemize} % {
\iusr{Георгий Майоренко}
\textbf{Sem Zilman} 

А как сложилась судьба Леопольда Матвеевича Рахлина? Он тоже в Киеве жил?
Остались ли потомки? Светлая память вашему дяде Натану Кушниру. Интересно, он
оставил какие-то дневники, воспоминания? Раз жил на Бессарабке, то, получается,
в самом сердце города. И, несомненно, многое повидал.


\iusr{Sem Zilman}
\textbf{Георгий Майоренко} К сожалению, я не люблю много печатать — я спец. риторического жанра  @igg{fbicon.smile} 
Если есть желание, то мы можем пообщаться в личке Messenger голосом?
Я киевлянин в 3-х поколениях; я родился в Александр. больнице и сначала жил в Липках, а потом — около Золотых Ворот... From 1989 - Austria, Italy, USA >>> and I visited 32 countries.

\iusr{Георгий Майоренко}
\textbf{Sem Zilman} Я перезвоню.

\iusr{Sem Zilman}
\textbf{Георгий Майоренко} OK!  @igg{fbicon.smile} 

\iusr{Sem Zilman}
\textbf{Георгий Майоренко} Почему мне ЗАПРЕЩЕНО ставить свой пост в этой группе?

\iusr{Георгий Майоренко}
\textbf{Sem Zilman} Я даже не в курсе. Я здесь такой же гость, как и вы. Это надо спросить у модераторов.
\end{itemize} % }


\end{itemize} % }
