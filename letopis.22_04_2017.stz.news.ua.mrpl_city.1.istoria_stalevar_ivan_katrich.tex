% vim: keymap=russian-jcukenwin
%%beginhead 
 
%%file 22_04_2017.stz.news.ua.mrpl_city.1.istoria_stalevar_ivan_katrich
%%parent 22_04_2017
 
%%url https://mrpl.city/blogs/view/istoriya-stalevar-ivan-katrich
 
%%author_id burov_sergij.mariupol,news.ua.mrpl_city
%%date 
 
%%tags 
%%title История: сталевар Иван Катрич
 
%%endhead 
 
\subsection{История: сталевар Иван Катрич}
\label{sec:22_04_2017.stz.news.ua.mrpl_city.1.istoria_stalevar_ivan_katrich}
 
\Purl{https://mrpl.city/blogs/view/istoriya-stalevar-ivan-katrich}
\ifcmt
 author_begin
   author_id burov_sergij.mariupol,news.ua.mrpl_city
 author_end
\fi


Иван Климентьевич Катрич принадлежит к тому периоду истории Мариуполя, когда на
первых полосах газет портреты сталеваров, в том числе и его, не были редкостью.
Люди эти заслуживали внимания и почета, поскольку занимались конкретным и,
главное, полезным делом. Наверное, с той поры и отложилась в памяти фамилия
Катрич. А подробно довелось узнать о нем от внука Ивана Климентьевича.

Так что последующее повествование будет в значительной мере пересказом
воспоминаний Сергея Владимировича Катрича со вставками из истории комбината им.
Ильича. Иван Климентьевич родился 15 мая 1915 года в семье потомственных
хлеборобов. Его малая родина – село Федоровка Володарского района. В 1931 году,
после окончания семилетки, Иван Катрич отправился в Мариуполь. Его приняли на
завод им. Ильича в мартеновский цех №2. Ему повезло – наставником у него был
сталевар от Бога Максим Махортов, работавший еще в мартеновском цехе завода
\enquote{Русский Провиданс}. Кстати, именно у Максима Васильевича с 1933 года стал
работать первым подручным Макар Мазай, прославившийся впоследствии как один из
инициаторов стахановского движения.  В этом же цехе работали знаменитые
сталевары Никита Пузырев и Иван Шашкин, они щедро делились своим опытом с
молодежью.

В 1934 году Иван Катрич женился на Паше Ярмоленко, приехавшей из Белоруссии в
Мариуполь на строительство второго трубопрокатного стана в новотрубном цехе
завода имени Ильича. Иван Климентьевич с женой Прасковьей Михайловной и Макар
Никитич с супругой Марфой Дмитриевной дружили семьями. У них было много общего
- и крестьянские корни, и жажда деятельности, и комсомольский задор, присущий
их поколению, и, конечно же, работа в одном цехе. Эти дружеские отношения не
прекратились и после того, как слава Макара Мазая вознеслась до всесоюзного
уровня...

22 июня 1941 года. Война! Фронт стремительно продвигался вглубь страны. В
начале августа началась постепенная эвакуация оборудования цехов завода им.
Ильича на восток. Когда было объявлено, что настало время отправки на Урал
персонала мартеновских цехов, Катрич решил отправить Прасковью Михайловну с
детьми, - а это были Владимир шести лет, Валентина четырех лет  и Виталий,
которому не исполнилось и года, - в Федоровку.  Наскоро собрались. Иван
Климентьевич повез семью на свою родину. Попрощавшись с родными, он тут же
поспешил в обратный путь в Мариуполь. Ему нужно было поспеть к шести часам
утра, к отправлению эшелона, отбывающего на восток. Он успел прибежать к
назначенному времени, даже несколько раньше, но...  Железнодорожная  платформа
была пуста. Уже после освобождения Мариуполя от оккупантов Катрич узнал, что
составы с работниками завода были отправлены на два часа прежде назначенного
срока.  По этой же причине остались на оккупированной территории и Макар Мазай,
и Иван Лут, удостоенный впоследствии звания Героя Социалистического Труда, и
некоторые другие работники завода. Иван Климентьевич вернулся в Федоровку. Даже
в отдаленном селе сталевару было опасно оставаться на виду. Он вырыл подальше
от села землянку на берегу речушки Каратыш, где и провел все время, пока в
наших краях хозяйничали гитлеровцы.

Как известно, Мариуполь был освобожден 10 сентября 1943 года. А 12 сентября
Катрич с семьей был уже в Мариуполе. Здесь будет уместно привести прямую речь
Сергея Владимировича Катрича: \enquote{Был развернут полевой военкомат. Дедушка одним
из первых отправился записываться воевать. Он уже получил винтовку и стоял в
строю. В это время бабушка прибежала с детьми и стала уговаривать: - Иван, куда
ты идешь, ты же сталевар, сейчас завод будут восстанавливать... И тут лейтенант,
который занимался записью, сказал: - Подожди, кто тут сталевар? - Вот, Иван
Катрич, известный сталевар, с Мазаем работал.  Лейтенант приказал дедушке сдать
винтовку и отправиться в мартеновский цех}... 

12 сентября на завод им. Ильича срочно прибыла группа руководителей во главе с
директором предприятия Александром Фомичом Гармашевым. Перед их глазами
предстала ужасная  картина - разрушенные цехи и жизненно важные объекты, в том
числе и электростанция – \enquote{сердце} завода. Без промедлений приступили к
восстановительным работам. На следующий день, 13 сентября, этим делом было уже
занято более 10 тысяч мариупольцев.  Среди них - и Иван Климентьевич. 20
сентября  1943 года электростанция завода дала ток. 10 октября 1943 года первую
плавку стали после восстановления выпустила мартеновская печь №5. Люди работали
с невероятным напряжением. Рабочий день – 12 часов, а если было нужно, то и
больше, выходных почти не было. Все понимали – это нужно для Победы.

На первых восстановленных сталеплавильных агрегатах завода начали работать те
из работников, которым удалось выжить в условиях оккупации, в том числе и Иван
Климентьевич. Постепенно стали прибывать специалисты из эвакуации. Одна за
другой восстанавливались мартеновские печи, наращивалось производство, менялись
руководители, а Катрич оставался на своем посту у мартеновских печей. За
самоотверженный труд при восстановлении мартеновского производства на заводе
им. Ильича Иван Климентьевич Катрич был награжден медалями \enquote{За доблестный труд
в Великой Отечественной войне 1941-1945 гг.} и \enquote{За трудовую доблесть}. Позже
ему было присвоенное звание \enquote{Заслуженный металлург Украины}.

Сергей Владимирович вспоминал: \enquote{Дедушка был высокого роста, улыбчивый,
доброжелательный к людям. Его авторитет был непререкаем. Все с уважением к нему
относились. Никогда нецензурной брани от него никто не слышал. Он общался ровно
со всеми}.

Уже зрелым человеком Иван Климентьевич поступил в вечерний металлургический
техникум и окончил его за три года до получения права на заслуженный отдых. И
сделал это для того, чтобы подать пример молодежи. Владимир Иванович Подрезов,
человек заслуженный и авторитетный, инженер-металлург, мастер мартеновского
производства, а позже преподаватель ПТУ в разговоре с автором этих строк
поставил Ивана Климентьевича в один ряд с такими прославленными мартеновцами,
как Герои Социалистического Труда Иван Андреевич Лут, Владимир Павлович
Клименко, Иван Дмитриевич Черняк. Он же отметил, что сталевар Катрич в свое
время был назначен старшим мастером, минуя должность мастера. Сколь
ответственна роль старшего мастера печного пролета мартеновского цеха станет
ясно из следующего случая.

22 февраля 1962 года первая 650-тонная мартеновская печь Новомартеновского цеха
завода им. Ильича была сдана строителями в эксплуатацию. Для того чтобы начать
в ней варить сталь, оставалось наварить подину – ванну из спекшегося порошка
магнезита, в которой собственно и происходит выплавка металла. Операция эта
чрезвычайно ответственна, поскольку даже незначительное  нарушение монолитности
подины приводит к тяжелейшей аварии. Расплавленный металл проедает подину и в
считанные минуты выливается под печь, сметая все, что попадается на пути. На
уборку застывших \enquote{козлов} приходится тратить много тяжелейшего труда и массу
времени. Вот почему именно старшему мастеру Катричу, известному своим
искусством наварки подин печей, было поручено сделать подину первой 650-тонной
мартеновской печи. С задачей Иван Климентьевич блестяще справился...

Иван Климентьевич ушел из жизни в июне 1978 года после тяжелой скоротечной
болезни. Ему было всего 63 года. Достойным продолжателем его дела  стал сын
Владимир. Цитата из газетного очерка: \enquote{Коллектив бригады сталевара Владимира
Катрича с первой мартеновской печи имени Германа Титова из месяца в месяц
добивается высоких произ­водственных показателей. На её счету десятки тонн
сверхплановой стали. Сталеплавильщики внесли немалый вклад в успех всего
коллектива предприятия, завоевавшего приз имени Макара Мазая}. Таких отзывов о
работе Владимира Ивановича было множество.
