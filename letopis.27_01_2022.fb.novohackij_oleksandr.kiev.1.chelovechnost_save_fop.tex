% vim: keymap=russian-jcukenwin
%%beginhead 
 
%%file 27_01_2022.fb.novohackij_oleksandr.kiev.1.chelovechnost_save_fop
%%parent 27_01_2022
 
%%url https://www.facebook.com/oleksandr.novokhatskyi.7/posts/975326796709138
 
%%author_id novohackij_oleksandr.kiev
%%date 
 
%%tags chelovechnost,obschestvo,ukraina
%%title Современное украинское «публично-информационное» пространство лишено человекости и человечности
 
%%endhead 
 
\subsection{Современное украинское «публично-информационное» пространство лишено человекости и человечности}
\label{sec:27_01_2022.fb.novohackij_oleksandr.kiev.1.chelovechnost_save_fop}
 
\Purl{https://www.facebook.com/oleksandr.novokhatskyi.7/posts/975326796709138}
\ifcmt
 author_begin
   author_id novohackij_oleksandr.kiev
 author_end
\fi

На фоне массового психоза с прапорцями и вализками естественным образом
теряется событие, которое показывает нашу реальность. Я про смерть участника
акций «Save ФОП». Отношение к этому событию ярчайший маркер состояния так
называемого «общественного пространства» ...

Смерть человека всегда самая большая утрата. Ведь ценность человеческой жизни,
это то, что так яростно декларируют все современные радетели порядка и прочих
биоограничений в купе с биотеррором. Однако реакции практически ноль. Реакция в
стиле – ну, ведь сам виноват ...

Современное украинское «публично-информационное» пространство лишено
человекости и человечности. Лишено напрочь. Вот тут и проявлено во всём своём
великолепии полная «красота» людожерского подхода к пониманию общественного
пространства сожития. Это прямое следствие принятия формулы – «государство
источник всего» ...

ФОПы лишние в этой системе отношений. Они как «нелюбимое дитя» и «объект
грабежа» со стороны машины государственного насилия. Они тот пласт человеческой
среды, который объявлен вне законов государственной этики в эпохе
постсоветского восприятия правильности. Ведь всё в этом пространстве против
человека самостоятельного, само-обеспеченного, мало нуждающегося в так
называемой защите государства.

Государственной машине (легитимного) насилия противны и неприятны люди, которые
не зависят от чиновников или государственных навязчивых услуг. И.
государственная политика насилия щедро поддерживается так называемыми «простыми
людьми», которые молчаливо согласны с тем, что государство обязано кого-нибудь
грабить. И согласны они потому, что живут за счёт тех самых ограбленных. Они не
столько свидетели и молчащие сторонники насилия, они соучастники. Соучастники
всех преступлений бездушной машины чиновников над живыми людьми.

Особенность нашего пространства, в котором мы живём совместно, в том, что на
самом деле основой мировоззрения невероятного большого количества людей тут
есть абсолютно иной принцип возникновения пространства отношений. Нам не
государство проявляет поле отношений. У нас государство есть следствием нашей
общности. Не государство формирует общество, а наоборот. Люди тут формируют
поле отношений, порядок взаимодействия и границы позволенного. И эти люди
нуждаются в своём государстве, которое охраняет и бережёт именно эти, людские
порядки, а не безумные высеры фашистских фантазий отдельных чиновничьих
сволочей.

Украина – это мир ФОПов. Это мир предпринимателей и живых людских сущностей.
Этот мир для людей свободных духом и крепких в своей убеждённости. Это мир
уникального подхода к ценности жизни. И в этом мире нет и не будет места
людожерским машинам насилия и втаптывания в грязь людей, ради призрачного
достатка других.

Сегодня только проблема в постановке задачи и понимании своей идентификации
теми самыми ФОПами. Это их мир, и они оплачивают своё право на него, своей
кровью и своими жизнями.

Поверьте, этому цирку со страшными клоунами, осталось уже совсем чуть-чуть. Как
вы будете смотреть в глаза людям, которые на своих плечах вынесут новый мир для
всех? Какое место вы в этом мире займёте? Ведь большинству, таки придётся
становиться теми же ФОПами. 

Подумайте об этом, даже если я пишу «непонятным для масс» языком ...

Посмотрите на ситуацию не через мои слова, а через ваше человеческое сердце.
