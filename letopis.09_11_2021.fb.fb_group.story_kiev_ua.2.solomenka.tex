% vim: keymap=russian-jcukenwin
%%beginhead 
 
%%file 09_11_2021.fb.fb_group.story_kiev_ua.2.solomenka
%%parent 09_11_2021
 
%%url https://www.facebook.com/groups/story.kiev.ua/posts/1794109247452535
 
%%author_id fb_group.story_kiev_ua,nasteckij_vjacheslav.kiev
%%date 
 
%%tags gorod,kiev,solomenka.kiev
%%title Соломенка
 
%%endhead 
 
\subsection{Соломенка}
\label{sec:09_11_2021.fb.fb_group.story_kiev_ua.2.solomenka}
 
\Purl{https://www.facebook.com/groups/story.kiev.ua/posts/1794109247452535}
\ifcmt
 author_begin
   author_id fb_group.story_kiev_ua,nasteckij_vjacheslav.kiev
 author_end
\fi

В марте 2020 г. в Киеве ввели локдаун. Я и ещё одна наша одноклассница,
независимо друг от друга, стали думать, что бы позитивное предпринять в это
невесёлое время. Остановились на встрече одноклассников. Мы учились в киевской
восьмилетней 60-й  школе с 1972 г. по 1980 г. Класс был дружный, но как-то
после окончания ни разу мы не встретились. А тут 40-летие окончания, и локдаун
же не навсегда. В общем, начали через фейсбук и по памяти собирать народ.
Создали группу на фейсбуке. Попросили ребят сбрасывать туда фотографии и
предложения по встрече. И мне показалось, что хорошо бы ещё сделать несколько
роликов-коллажей из наших школьных и послешкольных фотографий. А ещё лучше на
песню о нашем Соломенском районе. Нашёл я на фейсбуке группу Киевские истории,
а в ней комментарии Леонида Духовного. С детства помнил его песню \enquote{А
без Подола Киев невозможен}. Написал ему в мессенджере и спросил, нет ли у него
песни о Соломенке. Он ответил примерно в стиле Леонида Утёсова: \enquote{Вы
хочете песен - их есть у меня}. 

Из переписки:

"Здравствуйте, Леонид! Вы - автор "А без Подола Киев невозможен". Можно
поинтересоваться Вашими песнями о Соломенке с любительской целью?

Вячеслав, действительно,  в \enquote{Киевском цикле} есть песня \enquote{Соломенка}. Ну, как
вам сказать, - она очень личная. Дело в том, что в 50-е года я учился в
Судостроительном Техникуме, великолепное здание которого только только было
возведено на пустыре около Воздухофлотского Шоссе на противоположной стороне от
входа в КАЛКАУ. Сейчас, там, по моему какое-то корабельное  НИИ. Добро,
Вячеслав, вы - мой земляк Для меня это немаловажно. \enquote{Соломенка} была написана
где-то в 80-х годах. Однако, у меня, вот,  появился вопрос:  вы пишите не для
комерческих целей. Добро, а для каких, если не секрет, конечно?

я хочу попросить знакомых сделать пару роликов на темы встречи выпускников
своего класса. кот. планируем скоро провести. и для музыкального оформления
нужны такие 2-3 песни, учитывая. что многие живут уже в других районах, а
ностальгию по району нашей школы хорошо бы пробудить.

Понял вас. Вот слова песни

Соломенка  (вариант 1)          1961г. (?)
(на мотив «Демеевки»)

\obeycr
Встречай, Вокзал – ворота городские:
Крик поездов и суетный перрон.
И проводник  вещает : «Місто Київ!»,
А как по мне,- Соломенский район   
Соломенский район, вы б только знали,
Как много исторического в нём,
В одном депо здесь двое написали
«Наш паровоз вперёд летит!» - споём:
Пр.  Наш паровоз вперёд летит –
Так где же остановка ?
И, если нет уже пути
К чему тогда винтовка?
Бежит трамвай соломенский на гору,
Людей качает в такт и вразнобой.
Я у окна зубрил конспекты там и впору
Считать трамвай читальнею-избой
Хоть и не Пушкин, но свой срок лицейский
На «пять» тут отмотал, что твой пиит.
Тут ВУЗы грозные -  военнный, милицейский,
Тут даже Штаб Расстреллевский стоит ?!
\restorecr

\obeycr
Пр.   Соломенка, Соломенка, Соломенка ,
Кому район, ну, а кому музей,
Соломенка ,Соломенка, Соломенка,
Ты лучших подарила мне друзей !
Соломенка без клёнов невозможна.
\restorecr

\obeycr
Я тоже посадил их, сколько смог,
И рощицы зелёные,возможно,
Здесь оттого мелькают вдоль дорог
Клеймо судьбы – залысины и нимбы
Судьба-злодейка, карточный расклад...
\restorecr

Нет, не забыть Соломенку, где б ни был:

Здесь в юность возвратиться был бы рад!.

\obeycr
Пр.  Соломенка, Соломенка, Соломенка,
Года лихие в имени твоём !
Соломенка, Соломенка, Соломенка,
Давайте, братцы, за неё нальём !
\restorecr

Давайте, братцы,  выпьем и споём! Leonid

Мой поклон вашим школьным друзьям, землякам моим.

Огромное спасибо Вам, Леонид!

Вы отправили скажите, пожалуйста, а может. она у Вас есть в аудиозаписи и Вы
мне можете прислать такой файл в МП-3?

К сожалению, запись этой песни хранится у меня на старых плёночных кассетах . Я
должен их как-то оцифровать. Но руки не доходят. Ну, раз возникла
необходимость, попытаюсь этим заняться. У Сэма Рубчинского должна быть
аудиозапись моего прощального концерта 18 июня 1992 г. Не помню, пел я её там.
Но Демеевку точно пел и мелодии там очень похожи. Можете её, музыку,
использовать и в \enquote{Соломенке}.  Добра и здоровья."

После этого я обратился к Георгию Майоренко и он мастерски её напел. Каюсь, что
фоторяд я подобрал неудачный, а Георгий был занят, и тогда обратился к Геннадию
Банникову и тот ещё её напел."

Эти ролики поспособствовали тому, чтобы наша встреча 30 мая 2020 г. состоялась.
А потом уважаемый Леонид подарил ещё одну песню о Караваевских банях. И в том
же году были ещё 3 встречи, по иронии судьбы, по одной в каждое время года.

А вот и ролики:

исполняет Георгий Майоренко (участник нашей группы):
