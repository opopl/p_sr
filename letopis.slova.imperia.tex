% vim: keymap=russian-jcukenwin
%%beginhead 
 
%%file slova.imperia
%%parent slova
 
%%url 
 
%%author 
%%author_id 
%%author_url 
 
%%tags 
%%title 
 
%%endhead 
\chapter{Империя}
\label{sec:slova.imperia}

%%%cit
%%%cit_head
%%%cit_pic
%%%cit_text
Может, стоит больше внимания обращать на примеры нашего величия?
\emph{Имперскости} нам не хватает. Тут в связи с мовосрачами, поднятыми
Ларисой и Вахтангом, вот что хочу сказать.  \emph{Имперскойсти} нам не хватает.
Величие собственное не осознаём. Всё нам кажется, что кто-то нас использовал.
Что не мы строили СССР, а какие-то марсиане строили его, а Украину
использовали. Что не мы были \emph{Российской империей}, Речью Посполитою и
Княжеством Литовским, а эти \emph{империи} строили какие-то чужинци, а
украинцев порабощали и пили кровь ихнюю.  А как зайдёшь в какой-то музей
польский или литовский - то там вся украинская знать в одном ряду с польской и
литовской знатью. Как-то не похожи князья наши стародавние на подчинённых и
порабощенных... А первыхе лица \emph{Российской империи} и СССР - ну украинцев
больше, чем литовцев, казахов и таджиков
%%%cit_comment
%%%cit_title
\citTitle{Почему нам все время кажется, что нас кто-то использовал? / Лента соцсетей / Страна}, 
Павел Себастьянович, strana.ua, 01.07.2021
%%%endcit

%%%cit
%%%cit_head
%%%cit_pic
%%%cit_text
Така \enquote{віра} в російську \enquote{демократію} завжди породжувала нескінченні трагедії
для українців. Це вже траплялося і в часи новітньої історії, насаджуючи таку
маріонеткову не \enquote{українську} владу (на кшталт білоруської), яка б забезпечила
поглинання України цією варварською \emph{імперією}.  Відкриті спроби завершувалися
Майданами, оскільки дух свободи і віри в непереможність ідеї \enquote{своєї правди в
своїй хаті} був і залишається найвищою цінністю Українського народу. В Україні
і поза її межами мають знати всі, що ніщо і ніхто не може позбавити Народ цього
священного і суверенного права. Будь-яка українська влада є життєздатною, коли
відчуватиме цю спраглість народної Волі, не зраджуватиме Україну, сповідуватиме
у Слові і Чині Українську Ідею. Усе, що поза цими цінностями, є брехнею,
\enquote{блекотою}, приводячи таких правителів лише до забуття та народного осуду.
Культурна, політична, економічна інтеграція з тими, хто успадкувавши найгірші
риси ординського \enquote{Кипчака}, де завжди панує хан-диктатор і суспільний
менталітет рабської залежності, не приніс житейських благ жодному народу
%%%cit_comment
%%%cit_title
\citTitle{Правда в Рідному Слові}, Георгій Філіпчук, slovoprosvity.org, 12.07.2021
%%%endcit

%%%cit
%%%cit_head
%%%cit_pic
%%%cit_text
А ви – малороси? Навіть якщо ваші предки – етнічні росіяни, то ви мали би вже
давно зукраїнізуватися в Україні, а не російщити українців. Це не природно. І
настільки не природно, що вас треба називати незукраїнізованими росія­ нами
(малоросами), а не «російськомовними» українцями. Ви обурюєтеся, що я
категорично не даю вам права почувати себе «російськомовними українцями». Це
не я не даю права. Просто такого не існує. Птах чоловічої статі з обручкою
орнітолога на нозі не стає завдяки обручці одруженою людиною. Щур, який
народився у хліві, не стане вепром завдяки місцю народження. Баба з базару, яка
торгує яблуками, виноградом, бананами та гранатами не стане військовим, бо вона
має кавказькі гранати. Скрипалем не станеш, бо вмієш посмикати себе по
причандалах, тому що скрипка має смичок. Якщо ви хочете, аби вас називали
українцями, то заговоріть українською а російську покиньте назавжди, бо
\emph{Імперія Зла} вбиває тепер щодня українців а частину території – окупує
%%%cit_comment
%%%cit_title
\citTitle{НЕ ІСНУЄ РОСІЙСЬКОМОВНИХ УКРАЇНЦІВ. КРАПКА}, Przemysław Lis-Markiewicz Profil Prywatny, %
facebook, 22.10.2021
%%%endcit

%%%cit
%%%cit_head
%%%cit_pic
%%%cit_text
Первая галактическая \emph{империя} рухнула. Она разлагалась в течении
столетий, и только один человек полностью осознал этот факт.  Это был Хари
Селдон, последний великий ученый Первой \emph{Империи}, который
усовершенствовал психоисторию – науку, описывающую человеческое поведение
математическими уравнениями.  Индивидуальный человек – существо
непредсказуемое, но реакции человеческой массы, считал Селдон, могли быть
вычислены статистически.  Чем больше масса, тем большей точности можно
добиться. А объем человеческой массы, с которой работал Селдон, был не меньше,
чем население всех миллионов обитаемых миров Галактики
%%%cit_comment
%%%cit_title
\citTitle{Кризис основания}, Айзек Азимов
%%%endcit

%%%cit
%%%cit_head
%%%cit_pic
%%%cit_text
І все ж настав той день, коли, плачучи й ридаючи, сіла царівна в ромейську
неоковирну кувару, щоб попливти до темних варварів, хижих умів, тупих душ, в
дикість і безвір’я. Але то тільки так мовиться, що сіла в кувару й попливла.
Бо вже перед тим послано в Рим, щоб повідомити \emph{імператора} й папу про те,
скільки пресвітерів і слуг віри супроводжує Анну, щоб хрестити Русь, яка
дружина супроводжує сестру \emph{імператорів}, щоб захистити при потребі віру й
честь, які дарунки відправлено київському князеві: мечі, кольчуги, лати,
золоті ланцюги, срібні й золоті тканини, фібули, сосуди.
Все це пливло на ромейських незграбних кораблях-куварах, волочилося по морю
повільно й тяжко, ми насилу стримували свої меткі легенькі ладьї, щоб не
вирватися наперед і не завдати образи високій особі. Берегом, не відстаючи від
морського походу, супроводжувало нас кінне ромейське військо з повозами,
шатрами, припасами на всі забаганки Анни, коли набридне їй море і зволить вона
зійти на берег для спочинку
%%%cit_comment
%%%cit_title
\citTitle{Тисячолітній Миколай}, Павло Загребельний 
%%%endcit

%%%cit
%%%cit_head
%%%cit_pic
%%%cit_text
Тут, щоправда, є деякі питання до Путіна з Лавровим. Зрозуміло, що ринок зброї
побудований на цинізмі, але невже не можна було використати безліч спільних
тем, які пов'язують ердоганівську Туреччину з путінською Росією, щоб не
допустити продажу такої ефективної зброї Україні, проти якої Путін веде війну?
Ці запитання могли б поставити прихильники зміцнення \emph{імперії}, якби у
фашистській державі, побудованій Путіним, диктатору та його найближчому
оточенню можна було б ставити безсторонні запитання. Утім, добре, що цей режим
сам обрубав усі канали комунікації, що прирікає його на постійні помилки та
загальну неадекватність і, зрештою, робить неминучою його загибель
%%%cit_comment
%%%cit_title
\citTitle{Росія отримала два щигля по носі від України}, 
Ігор Яковенко, gazeta.ua, 29.10.2021
%%%endcit
