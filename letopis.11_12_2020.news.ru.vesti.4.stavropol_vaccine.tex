% vim: keymap=russian-jcukenwin
%%beginhead 
 
%%file 11_12_2020.news.ru.vesti.4.stavropol_vaccine
%%parent 11_12_2020
 
%%url https://www.vesti.ru/article/2497720
 
%%author 
%%author_id 
%%author_url 
 
%%tags 
%%title Первая партия вакцины от COVID-19 поступила в Ставропольский край
 
%%endhead 
 
\subsection{Первая партия вакцины от COVID-19 поступила в Ставропольский край}
\label{sec:11_12_2020.news.ru.vesti.4.stavropol_vaccine}
\Purl{https://www.vesti.ru/article/2497720}

\ifcmt
pic https://cdn-st1.rtr-vesti.ru/vh/pictures/xw/307/777/7.jpg
\fi

Первая большая партия вакцины уже в регионе. Вакцинировать будут жителей трёх
городов – это Пятигорск, Невинномысск и Ставрополь. Первыми инъекцию сделают
врачам, которые работают в тесном контакте с ковидными больными. На очереди –
учителя, соцработники, правоохранители и все, у кого хронические заболевания.
Сегодня первые 200 доз поступили в Ставрополь, передает ГТРК "Ставрополье".\Furl{https://stavropolye.tv/}

\begin{leftbar}
	\begingroup
		\em "В воскресенье, 13 декабря, поступает третья партия вакцины, это 1400
				доз, тоже будет распределена между медорганизациями, в Ставропольском
				крае определена 41 организация, где будет проходить вакцинация. Пока
				это 9 пунктов", – пояснили замминистра здравоохранения Наталья
				Звягинцева.
	\endgroup
\end{leftbar}

Ранее глава региона Владимир Владимиров подчеркнул, что вакцинация – только
добровольная, а за принуждение губернатор призвал наказывать. Кроме того, чтобы
не создавать очередей на прививку, записаться можно будет через портал
госуслуг. Для тех, кто живет в отдаленных населенных пунктах, будет организован
транспорт туда и обратно. Всего в Ставропольском крае планируется привить от
коронавируса до 1,5 миллиона человек.

По данным на 11 декабря, пятницу, на Ставрополье с начала пандемии
коронавирусом заболели более 30 262 человека, выздоровели 26 874 заболевших,
615 пациентов умерли. Российская вакцина от коронавируса "Спутник V" была
зарегистрирована ровно 4 месяца назад, 11 августа. Ее разработали в
Национальном исследовательском центром эпидемиологии и микробиологии имени Н.
Ф. Гамалеи Минздрава РФ.
