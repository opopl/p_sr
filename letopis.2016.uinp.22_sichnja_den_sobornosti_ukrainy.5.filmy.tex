% vim: keymap=russian-jcukenwin
%%beginhead 
 
%%file 2016.uinp.22_sichnja_den_sobornosti_ukrainy.5.filmy
%%parent 2016.uinp.22_sichnja_den_sobornosti_ukrainy
 
%%url 
 
%%author_id 
%%date 
 
%%tags 
%%title 
 
%%endhead 

\subsubsection{Документальні фільми}

\textbf{\em Орієнтовний перелік документальних фільмів, які відображають події Української
революції 1917 – 1921 років}

\begin{center}
%\tiny
%\footnotesize
%\small
\begin{longtable}{|l|p{8cm}|p{1cm}|p{5cm}|}
\hline
\textbf{№} & \textbf{Назва картини} & \textbf{Рік випуску} & \textbf{Режисер} \\
\hline
1. & \enquote{Акт Злуки: відтворення історичної правди} & 2011 & Тарас Каляндрук \\
\hline
2. & \enquote{Апельсинова долька} & 2004 & Ігор Кобрин \\
\hline
3. & \enquote{Герої України. Крути. Перша Незалежність} & 2014 & Сніжана Потапчук \\
\hline
4. & \enquote{Київ. Епоха переворотів} (з циклу \enquote{Історії міста}) & 2011 & Віталій Загоруйко \\
\hline
5. & \enquote{Легіон. Хроніка Української Галицької Армії 1918–1919} & 2015 & Тарас Химич \\
\hline
6. & \enquote{Непрощені. Симон Петлюра} & 2007 & Віктор Шкурін \\
\hline
7. & \enquote{Обличчя купюри. Михайло Грушевський} & 2008 & Віталій Загоруйко, Сергій Братішко \\
\hline
8. & \enquote{Свято Злуки. Політика пам'яті} & 2011 & \\
\hline
9. & \enquote{Українська революція. Втрачена держава} & 2007 & Сергій Братішко \\
\hline
10. & \enquote{Українська революція. За спогадами Всеволода Петріва} & 2012–2013 & Іван Канівець \\
\hline
11. & \enquote{Холодний Яр. Воля України – або смерть!} & 2014 & Галина Химич \\
\hline
12. & \enquote{Хроніки української революції} & 2008 & Сергій Братішко \\
\hline
\end{longtable}
\end{center}
