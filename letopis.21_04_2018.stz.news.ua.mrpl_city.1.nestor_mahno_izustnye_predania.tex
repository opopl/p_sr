% vim: keymap=russian-jcukenwin
%%beginhead 
 
%%file 21_04_2018.stz.news.ua.mrpl_city.1.nestor_mahno_izustnye_predania
%%parent 21_04_2018
 
%%url https://mrpl.city/blogs/view/nestor-mahno-izustnye-predaniya
 
%%author_id burov_sergij.mariupol,news.ua.mrpl_city
%%date 
 
%%tags 
%%title Нестор Махно: изустные предания
 
%%endhead 
 
\subsection{Нестор Махно: изустные предания}
\label{sec:21_04_2018.stz.news.ua.mrpl_city.1.nestor_mahno_izustnye_predania}
 
\Purl{https://mrpl.city/blogs/view/nestor-mahno-izustnye-predaniya}
\ifcmt
 author_begin
   author_id burov_sergij.mariupol,news.ua.mrpl_city
 author_end
\fi

\ii{21_04_2018.stz.news.ua.mrpl_city.1.nestor_mahno_izustnye_predania.pic.1}

\textbf{Нестор Иванович Махно} – личность героическая и трагическая одновременно и
потому, наверное, противоречивая в оценках историков, литераторов и
кинематографистов, очень часто конъюнктурных. Вместе с тем, как у всякого героя
из народа, его имя обросло своеобразным фольклором. В нем были и небылицы,
часто переплетающиеся причудливым образом, где он или рыцарь без страха и
упрека, на белых одеждах которого нет ни единого пятнышка, либо жестокий и
непредсказуемый в своих поступках тиран. Свой вклад в устное народное
творчество о Несторе Ивановиче внесли и мариупольцы. Автор этих строк позволил
себе положить на бумагу то, что довелось узнать от своих земляков о батьке
Махно и махновцах.

Вероятно, впервые о батьке, а вернее, о представителях его воинства довелось
услышать от бабушки. Она, демонстрируя свою находчивость, рассказала историю,
как в дом деда зашло несколько махновцев. Один из них заметил висевшую на стене
лубочную картинку с изображением казака Кузьмы Крючкова, героя Первой мировой
войны, георгиевского кавалера, нанизавшего на пику нескольких кайзеровсих
солдат. \enquote{А, золотопогонник!} – воскликнул один из анархистов, выхватил
револьвер и стал прицеливаться в картинку. Реакция бабушки оказалась
мгновенной: \enquote{Сынок, зачем расстреливать казака? Он и так повешен}.
Присутствующие при этой сцене рассмеялись, погорячившийся воин крестьянской
армии засунул оружие в кобуру. Потоптавшись некоторое время в комнате, хлопцы
ушли восвояси.

Еще одна история, услышанная в детстве. Ее поведала наша соседка, Мария
Михайловна Федорова, царствие ей небесное. У Марии Михайловны была сестра,
которая работала в кондитерском заведении на Екатерининской улице. Когда
Мариуполь заняли деникинцы, ей стал оказывать знаки внимания  молодой офицер.
Перед изысканным ухаживанием статного юноши, опоясанного скрипящими портупеями,
с позолоченными погонами на плечах и Георгиевским крестом на груди, с
подношением роз и приглашениями в ресторан мариупольская красавица не смогла
устоять. Завязалась любовь. Однажды прошел слух, что к нашему городу
приближается батька Махно. На следующий день к девушке пришел офицер. \enquote{Если ты
меня любишь, возьми этот конвертик, подсыпь порошок из него в торт, а когда
этот главарь банды войдет в город, преподнеси торт ему}. Девушка в точности
выполнила все, что поручил ей обожатель. Но в последний момент, когда Махно уже
готов был взять подношение, сестра Марии Михайловны разрыдалась, бросила торт
себе под ноги и закричала: \enquote{Нестор Иванович, его есть нельзя - там крысиный
яд!} Батька оценил поступок мариупольчанки и одарил ее золотой брошью. Но на
этом история не закончилась. Прошло какое-то время, власть в Мариуполе
переменилась. В город вновь зашли части добровольческой армии генерала
Деникина, а с ними и ухажер героини нашего повествования. Он нашел свою
возлюбленную и... собственноручно ее застрелил. Здесь нужно остановиться.
Родственник несчастной девушки, прочитав этот очерк, возмутился. Мол, на самом
деле несчастная с появлением деникинцев в Мариуполе выпила ацетон и скончалась
в страшных мучениях.

А это семейное предание поведал доцент ПГТУ Сергей Сергеевич Данилов. Его
дедушка, доктор Илья Ильич Данилов (кстати скажем, что он был в числе
основателей и первый главный врач мариупольской городской больницы), в свое
время построил в глубине двора на Екатерининской улице особняк, нынешний адрес
этого строения – проспект Ленина, 29 (пр. Мира.- Прим. ред.). Так вот, в этот
особняк зашли группа махновцев, не говоря ни слова, они забрали шесть стульев
из гостиного гарнитура хозяина и ушли. Прошло не так уж много времени, как в
дом Данилова вновь вошел махновец (в нем был узнан один из экспроприаторов
стульев) и строго произнес: \enquote{Ты – доктор? Бери инструменты и иди за мной}. Илья
Ильич, прихватив докторский саквояж, отправился за своим \enquote{повелителем}. Идти
пришлось совсем недолго, завернув на Греческую улицу, он был введен в дом,
который некогда стоял наискосок от здания Мариинской женской гимназии, нашим
современникам известной как школа № 1. Посреди большой жарко натопленной
комнаты на диване, застеленном огромным  ковром, восседал среди подушек Нестор
Иванович. Доктор заметил, что он был сильно простужен, а еще ему бросились в
глаза… стулья из его гарнитура. Осмотрев больного, Илья Ильич дал выпить
больному нужные лекарства и остался ждать, пока они подействуют. Только
глубокой ночью у Махно кризис миновал - температура спала, ему стало намного
лучше. \enquote{Доктор, сколько я должен вам за визит?} - \enquote{Нисколько, Нестор Иванович.
Распорядитесь только, чтобы мне вернули мои стулья}. После этого краткого
диалога  Илья Ильич, сопровождаемый группой хлопцев со стульями в руках,
вернулся домой. Трудно передать словами, какие переживания были перенесены
домочадцами врача за часы его отсутствия, а тем более, радость его счастливого
возвращения.

Приходилось читать и слышать, что Махно сурово карал тех из своего воинства,
кто грабил селян и малообеспеченных мещан. Даже сам расстреливал мародеров
перед строем в назидание другим. Но, видимо, будучи вдохновленными призывом
\enquote{Грабь награбленное!}, махновцы позволяли себе \enquote{потрясти} богатеньких, на их
взгляд, людей. Стоило Секлетинье Герасимовне Дьяченко услышать имя Махно или
его сподвижников, как она разражалась изощренными проклятиями с добавлением
самых грубых слов, услышанных ею некогда от молотобойцев. Праведный гнев
Секлетиньи Герасимовны имел веские основания – ей пришлось под угрозой
наставленного на нее маузера отдать махновцам золотишко, \enquote{накованное} ее мужем
– кузнецом Иваном Исидоровичем -  тяжким трудом у горна и наковальни.

Не наилучшие воспоминания оставили сподвижники Нестора Ивановича и в памяти
Альбины – вдовы Иоганна Вайца, чеха, занимавшегося при жизни торговлей скотом в
Приазовье... Тачанки махновцев неслись со стороны Таганрога по Торговой улице.
Одна из них резко остановилась у приметного двухэтажного дома, где жила Альбина
со своими тремя сыновьями и дочерью и ее родная сестра Екатерина с детьми и
мужем -  колбасником Войтехом Карасеком. Несколько парней, обвешанных оружием и
гранатами, стремительно обежали все комнаты и в одной из них (когда-то она
служила кабинетом Вайцу), обнаружили два кресла, обитые красным сафьяном. Тут
началось страшное для чешки-аккуратистки. На глазах у Альбины незваные гости
стали вырезать острыми ножами полосы сафьяна из кресел и тут же начали
приторачивать их к своим папахам и фуражкам...

Владелец самого большого магазина головных уборов Яков Сапальский, наслышанный
о \enquote{подвигах} анархистов в Екатеринославе, где громили почем зря магазины, решил
применить собственную тактику. Когда кортеж тачанок во главе с автомобилем, в
котором сидел Махно, стал приближаться к шапочному заведению, его хозяин вышел
навстречу, держа в обеих руках по нескольку каракулевых шапок. Самую лучшую из
них он преподнес Нестору Ивановичу, а остальные раздал его приближенным. Нестор
Иванович отдал должное щедрости мариупольского предпринимателя и произнес
следующие слова: \enquote{Торгуй спокойно, тебя мои хлопцы не тронут}. И действительно,
никогда не трогали.

Но были и такие мариупольцы, которые с теплотой вспоминали Нестора Ивановича.
Как-то у парадного входа в дом № 39 по проспекту Ленина довелось услышать такую
историю от уроженца Мариуполя, а затем жителя столицы Эстонии Таллина Иосифа
Михайловича Цехановского. Он рассказал, что в пору его детства (а оно пришлось
на годы Гражданской войны) в первом этаже дома была кондитерская Жозефа. Над
кондитерской находился бильярдный зал. В один из своих набегов на Мариуполь
Нестор Махно устроил в этом зале свой штаб. Мальчишки, и в том числе наш
земляк, с нетерпением ждали у парадного каждый выход вождя восставших
крестьянских масс. С момента появления \enquote{Батьки} в проеме двери и до того
момента, пока он садился в открытый автомобиль, орава семи-, восьмилетних
сорванцов скандировала: \enquote{Слава батьке Махно! Слава батьке Махно!} В ответ
Нестор Иванович одобрительно кивал головой, кто-нибудь из его свиты подавал
\enquote{торбочку}, из которой Махно пригоршней выбрасывал детворе кон­феты. Иногда
вместо конфет фигурировали \enquote{керенки} - обесцененные банкноты Временного
правительства.

Автор этих строк надеется, что читатель не будет слишком строго судить его за
написанное. Ведь его миссия заключалась лишь в том, чтобы зафиксировать
изустные сказания о Несторе Ивановиче Махно и его сподвижниках, услышанные им в
Мариуполе на протяжении своей довольно продолжительной жизни.

\underline{Справка:}

\enquote{Махно Нестор Иванович (1888-1934), один из главарей мелкобуржуазной
контрреволюции на Южной Украине в Гражданскую войну, анархист. В 1921 году
бежал в Румынию}.

(\emph{Советский энциклопедический словарь, Москва, Советская  энциклопедия, 1981})

\begin{quote}
\enquote{Махно Нестор Иванович (1888-1934), российский деятель анархо-крестьянского
движения в 1918-21 на Южной Украине в Гражданскую войну. Возглавляемое Махно
движение (общая численность непостоянна — от 500 человек до 35 тыс. человек)
выступало под лозунгами \enquote{безвластного государства}, \enquote{вольных советов}, вело
вооруженную борьбу против германского нашествия, белогвардейцев, а затем и
против Советской власти. Движение ликвидировано Красной Армией. Махно в 1921
эмигрировал} (\emph{Большая энциклопедия Кирилла и Мефодия. Электронная версия 2006
года})
\end{quote}
