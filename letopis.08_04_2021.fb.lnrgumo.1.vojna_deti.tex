% vim: keymap=russian-jcukenwin
%%beginhead 
 
%%file 08_04_2021.fb.lnrgumo.1.vojna_deti
%%parent 08_04_2021
 
%%url https://www.facebook.com/groups/LNRGUMO/permalink/3578635385581343/
 
%%author 
%%author_id 
%%author_url 
 
%%tags 
%%title 
 
%%endhead 

\subsection{После смерти школьника Меркель пригласили в Донецк}
\Purl{https://www.facebook.com/groups/LNRGUMO/permalink/3578635385581343/}


\ifcmt
  pic https://external-bos3-1.xx.fbcdn.net/safe_image.php?d=AQFkY6mS58MRKZAi&w=500&h=261&url=https%3A%2F%2Fdonbasstoday.ru%2Fwp-content%2Fuploads%2F2021%2F04%2FSkrinshot-07-04-2021-113425.jpg&cfs=1&ext=jpg&ccb=3-4&_nc_hash=AQGkqvVApIZTOCI_
\fi


Когда-то в той, далёкой, довоенной жизни мы не понимали, какое это счастье –
засыпать и просыпаться в тишине, прогуливаться по скверам и аллеям города. И
спешить домой, потому что по тебе там скучают любящие люди, а не потому, что
ровно в 13:30 начинается обстрел.

А до этого времени тебе после работы еще нужно забежать в магазин. Ведь хоть и
«не хлебом единым…», но и им тоже жив человек. И эта постоянная мысль: «Успеть
добежать (ходить тогда мы разучились), успеть спрятаться в родных стенах!».
УСПЕТЬ!!!

Кроме войны, тем для разговора просто не было: «Куда «прилетело»?», «Все ли
живы?», «Какие молодцы коммунальщики – всё исправили-залатали за какие-то
сутки!..».

Мы разучились планировать свою жизнь. И научились различать приземляющиеся
снаряды: мины, «Грады», «Точки-У».

Сегодня моя память закрывается наглухо, когда я пытаюсь вспоминать те страшные
дни, часы и моменты – когда вжимаешься в кровать, пытаясь превратиться в
маленькую точку, в которую невозможно попасть; когда губы, несмотря на то, что
тебе уже за 50, вместе с молитвой шепчут: «Мамочка!»; когда приходишь в родную
школу не для того, чтобы войти к детям в светлый класс, а для того, чтобы
убрать в очередной раз выбитые взрывной волной стёкла…

Но никогда не забудется самый страшный день – 26 ноября 2014 года. День, когда погиб ученик нашей школы Русов Никита. Обычный мальчишка. По-юношески во всём пытался дойти до сути, учился легко, любил футбол (а кто ж из мальчишек его не любил и не хотел играть в «Шахтёре»?). Вот и бежал на тренировку после школы…
Когда солдат идёт защищать Родину, он понимает, чем рискует. Он может погибнуть, это страшно, но такова его мужская работа и солдатская доля… А когда гибнут дети – это неправильно, больно и страшно вдвойне! …В тот момент у меня родилось вот такое письмо:
«Госпожа Меркель. Я думаю, что Вы как высокообразованный и эрудированный человек наслышаны о гостеприимстве славян (я имею в виду русских, украинцев, белорусов). Так вот, я, человек, на данный момент имеющий паспорт гражданки Украины, хотела бы пригласить Вас в гости. Просто в гости. Поверьте, никакой политики.
Приезжайте в наш замечательный город Донецк. Город миллиона роз. Он и сегодня очень красивый, несмотря на все усилия наших бывших сограждан превратить его в руины и груды мусора.
Здесь, как и в любом городе мира, живут совершенно разные люди: богатые и не очень, доверчивые и хитрые… И всё же больше здесь замечательных людей: добрых, отзывчивых, умеющих дружить и любить, несмотря ни на что или даже вопреки этому чему-то...
Нас не испортили война и некоторый недостаток продуктов. Мы по-прежнему рады гостям, любим и умеем их принимать. Приезжайте! Особых разносолов в сложившейся ситуации не обещаем, но голодной не оставим. Зрелища же организует для Вас украинская армия.
Вполне возможно, в день Вашего приезда украинская армия устроит праздничный салют из всех видов вооружения (всё же Вы не рядовая гражданка Германии). Надеюсь, Вас порадуют букеты из «Гвоздики», «Сирени», «Гиацинтов» и «Тюльпанов». Конечно, мы не можем повлиять на погодные условия, и в этот день вероятны осадки в виде «Градов». «Ураган» не входит в основную программу, но его можно заказать дополнительно у украинских военных.
Приезжайте с семьёй. Можно пригласить друзей.
С нетерпением жду Вашего приезда.
Ирина Я.».
В памяти вот всплыло ещё одно событие. В школе праздновали День Победы. Пришли ветераны Великой Отечественной войны, бойцы армии ДНР, воины-афганцы. В тот момент, когда дети выступали на сцене, неподалеку на полигоне начались учения: взрывы, выстрелы. Было очень громко. Взрослые люди вскочили с мест, а дети на сцене даже бровью не повели. Хорошо это или плохо? Скорее, грустно. Но, увы, – факт.
Мне рассказывала бабушка, что в годы Великой Отечественной все были очень дружны, все соседи спешили на помощь друг другу. Тогда в это не очень верилось, вызывало улыбку. Но сегодня я точно знаю: вокруг меня живут замечательные люди, отзывчивые, добрые. Они не пройдут мимо, не оставят в беде.
И иногда у меня возникает мысль: а может быть, для того, чтобы понять, кто мы и какие мы, нужно было пройти через ужас войны?..
Стилистика и пунктуация автора сохранена
Ирина Я. учитель школы № 46,
г. Донецк, микрорайон Азотный,
Куйбышевский район
В проекте «Как я встретил начало войны» каждый житель Донбасса может рассказать, как именно изменила война его жизнь, что произошло в его судьбе с началом боевых действий в Донбассе. Необходимо, чтобы весь мир узнал о тех тревожных днях 2014 года, когда началась гражданская война.
Вы можете отправить свою историю нам на почту: pismo@donbasstoday.ru
