% vim: keymap=russian-jcukenwin
%%beginhead 
 
%%file 20_05_2019.fb.lesev_igor.1.zelenskii_jarko.cmt
%%parent 20_05_2019.fb.lesev_igor.1.zelenskii_jarko
 
%%url 
 
%%author_id 
%%date 
 
%%tags 
%%title 
 
%%endhead 
\subsubsection{Коментарі}

\begin{itemize} % {
\iusr{Nathalie Khod}
Вот точно что хомячки.

\iusr{Александр Рябоконь}
Наоборот) всплеск эмоций - просто всплеск. В любом случае рейтинг Зе покатится вниз. На Юго-Востоке в первую очередь.

\begin{itemize} % {
\iusr{Игорь Лесев}

у всех рейтинг когда-то катится вниз, даже у самого крутого сериала... Ключевой
вопрос - это запас времени. И именно с этим Зе пока что успешно справляется

\iusr{Александр Рябоконь}
\textbf{Игорь Лесев} запас времени, вероятно, уже известен) 2 месяца +/-

\iusr{Сергей Бердинский}
Если не будет судов и посадок коллективного свинарчука - то и раньше!

\iusr{Евгений Отовчиц}
\textbf{Александр Рябоконь} 

я считаю, что Слуга народа наберёт на выборах больше вчерашних 40\%, которые
дают социологи. После такой инаугурации, после таких требований к Раде - может
быть 50\%, а может и 60\%. Это натурально превращается в третий тур выборов
Президента, и людям реально не впадло и третий раз послать всех заслуженных
буррито в унитаз.

\iusr{Александр Рябоконь}
\textbf{Евгений Отовчиц} 

да не вопрос) считать-то можно. Вся электоральная история Украины есть сплошной
попыткой кого-то куда-то слить или послать. Всплески периодически бывают разной
силы. Результат противоположен ожиданиям. Особенно это очевидно, когда волна
идёт на спад.
\iusr{Михаил Мищишин}

\textbf{Александр Рябоконь} Мне тоже кажется, что времени у Зеленского пару месяцев, и отток избирателей юго-востока уже пошел. Просьбами усилить санкции США против России свой рейтинг в тех краях не поднимешь. По сути, это все то же,что говорили Климкин и Порошенко. Со всеми вытекающими для рейтингов Зеленского.

\iusr{Александр Рябоконь}
\textbf{Михаил Мищишин} да) но это пока только говорит) месяца через 2-3 ЗеКастрюли робко и тихонько будут массово роптать "а где же? Что же? Как же?"... А потом, санкции - это не только политика) это вполне себе собственные свинарчуку, таскающие неликвид втридорога через прокладки)) ибо больше неоткуда взять.
\end{itemize} % }

\iusr{Сергей Киселев}
Всё по полочкам

\iusr{Елена Ланская}

Правый сектор, в лице Тарасенко, уже проаел "красные линии"в президентской
деятельности Зе. Где, "мова .. запретная тема. Остается в преоритете..! И,
остается война на Донбассе ! Необходимо начать усиленные военные действия.
Иначе, это будет сдача интересов Украины". К тому же, Зеленский уже встретился
с представителями конгресса и попросил их оказывать воздействие на агрессию
России, в виде новых санкций. Так что, Вы, Игорь, абсолютно верно, как всегда,
все константировали.


\iusr{Арина Родионова}

Блин... Ну, может я и хомячок) я сегодня радовалась) так уже надоела эта
грусть-печаль, хотелось простой щенячьей радости) ну могу себе позволить?)) И
даже сентиментально всплакнуть?) (боже! где я и где сентиментальность!)). А
разочаровываться я решила что буду потом)) а может и не буду?)

Да что же это со мной?) Старею, наверное))

По делу)

От роспуска ВР я тоже не в восторге, ведь эти дармоеды закон о выборах точно не
проголосуют и опять зайдут в раду продажные мажоритаршики, и опять все
по-новой...

Остальное тоже не проголосуют, зачем им?все равно уже никуда они не войдут...
Но на этом Зеленский хорошо хайпонул. И только чтобы увидеть эти унылые
ошеломлённый рожи стоило за него голосовать и стОит многое ему простить)

О драконах... Есть фильм, "Пересмешница", кажется, недавно, в пол глаза, фоном
смотрела), фэнтэзи. Так вот там боролись-боролись с кровавым диктатором и в
итоге одна из победительниц захотела им стать... Но другая победительница это
просекла и застрелила ее... из лука, стрелой. ... Кто у нас будет кто -
посмотрим....

\iusr{Сергей Бердинский}

Да нет, Юля - не птица Феникс. Ей походу уже сказали - помалкивать. Тут другое
назревает... Вова хороший актер и 95 Квартал - молодцы, но - где команда профи?
Кто будет управлять державою? Вернувшийся Портнов или Лукаш-советник?
Телепузики Комаровский и Пальчевский? Или может, Корогодский с Меламудом???
Show must go on!

\begin{itemize} % {
\iusr{Евгений Отовчиц}
\textbf{Сергей Бердинский} у вас есть конкретные претензии к квалификации Портнова?
\end{itemize} % }

\iusr{Руслан Питовранов}
Шикарный слог и аргументы

\iusr{Дмитрий Коломийченко}

ЗЕро должен был отвечать на прокат с 19 числом. Оставлять Раду - признавать
политическое поражение на старте. Ну а у ПАПа на этот случай не было варианта
Б. Оказалось, что у него не Блейнхаймский дворец, а хижина дяди Пети. 5 лет
полудиктатуры и всего лишь хижина. Отступать некуда, позади Рошен.

\iusr{Михаил Мищишин}

При Зеленском Украина окажется еще более расколотой. Как Америка при Трампе или Франция при Макроне.

\begin{itemize} % {
\iusr{Олег Резник}
А можно ли ее " склеить" обратно при наших делах? Не в теории, а на практике.

\iusr{Михаил Мищишин}
\textbf{Oleg Rieznik} 

Уже трудно. Слишком много сил и энергии было вложено в разделение страны и
народа. И слишком много заинтересованных стран в этом.

\iusr{Евгений Отовчиц}

Все можно склеить. Благосостояние страны + государственная нетерпимость к
национализму. В Австралии попробуй начать рассказывать о том, что ты расовый
немец, англичанин, русский или абориген - до 7 лет с конфискацией. Все -
граждане Австралии. С равными правами и обязанностями.

\iusr{Дарина Зарицкая}
\textbf{Евгений Отовчиц} ух ты, вот бы у нас так, как в Австралии, задолбали титульные.
\end{itemize} % }

\iusr{Павло Бубенко}

Толково, Игорь! Будем посмотреть на всех означенных Вами как известных
персонажей, так и на тоже хорошо известных восточных, западных, центральных и
прочих хомячков!

\iusr{Иван Васильев}
Но ведь многие люди верят в то что Земля плоская... Факты и реалии их не волнуют, и это нормально...

\iusr{George Sizov}
Весело и креативно ... Почти как у Зеленского ...

\iusr{Василий Стоякин}
Как всегда - прекрасно. Единственно, что Зе, как раз - за Бандеру и против Путина. Но избирателям нравится думать, что нет

\iusr{Сергей Белашко}
Полностью согласен с предыдущим комментатором.

\iusr{Марина Прохорова}

Юля себя трахнула ещё раньше, из атомного пистолета ) И добила томосом. А в
остальном... Если раньше Украина распадалась медленно и печально, то теперь
будет быстро и весело

\iusr{Oleksandr Novokhatskyi}

Отлично изложена ось "либо пэтя, либо юля". Так оно, собственно, и есть.
В любом случае, результаты этих парламентских выборов будут "неожиданными" для всех "традиционных" политиков.

\iusr{Оля Окольникова}

Гройсман промову Зеленськоого сприйняв буквально і сам вирішив подати у відставку. От бачите, а не такий і дурний у нас прем’єр.

\iusr{Михаил Гарягин}

Хороший анализ поверхностных визуальных эффектов, но не все так просто
ребятушки, Зеленского готовили очень заранее и наверняка, значит он должен
провести в жизнь что-то очень серьезное и ошибок или просчётов быть не может
... Разыграно все так что помешать ему выиграть президентскую гонку не смог бы
ни один внешний и внутренний противник, безупречная работа ... Вот чья, скоро
узнаем ...

\end{itemize} % }
