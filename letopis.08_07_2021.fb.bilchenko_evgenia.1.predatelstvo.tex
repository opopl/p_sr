% vim: keymap=russian-jcukenwin
%%beginhead 
 
%%file 08_07_2021.fb.bilchenko_evgenia.1.predatelstvo
%%parent 08_07_2021
 
%%url https://www.facebook.com/yevzhik/posts/4043656635669416
 
%%author Бильченко, Евгения
%%author_id bilchenko_evgenia
%%author_url 
 
%%tags bilchenko_evgenia,poezia,predatelstvo
%%title БЖ. Предательство
 
%%endhead 
 
\subsection{БЖ. Предательство}
\label{sec:08_07_2021.fb.bilchenko_evgenia.1.predatelstvo}
\Purl{https://www.facebook.com/yevzhik/posts/4043656635669416}
\ifcmt
 author_begin
   author_id bilchenko_evgenia
 author_end
\fi

БЖ. Предательство.
Я - солдат и не знаю слов церемоний. Я буду резким.
О, моя королева, зачем Вам были мои подвески?
Зачем были эти клятвы, шуршание юбок в зале,
Где Вы за меня ручались, где Вы за меня "сказали"?
Разве просил я Вас о защите, о гербе на моей бумаге?
Я делал то, во что верил. А верил я в Родину, в силу шпаги
И в Вас, моя королева. Я не думал о Вашем чине.
Я видел в вас женщину. Человека. Так меня обучили
Мои первопредки, мои никем не сломленные гасконцы.
Когда из луны выходило солнце, я видел луну и солнце.
Но вам оставалась одна луна - сломленная ограда,
Витое чугунное ожерелье из классицизма сада.
Говорят, что женщине - розы, Родине - меч свой, а честь - себе же.
Моя королева, мы - не на бирже, не на диване бежевом.
Оставьте улыбки для будуара, а уста - для воззваний веских...
Да, и на всякий случай, вот...
Я привез Вам Ваши подвески.
8 июля 2021 г. 
Илл.: кадр из фильма "Д'Артаньян и три мушкетера".

\ifcmt
  pic https://scontent-lga3-1.xx.fbcdn.net/v/t1.6435-9/213725706_4043656595669420_1190002804909227099_n.jpg?_nc_cat=107&ccb=1-3&_nc_sid=8bfeb9&_nc_ohc=9lvPltYsWqoAX89cwit&_nc_oc=AQkXXEZ024h4E-Yn608U2h8_XphYYAZ8IKmyuZYN6eUcZmOLBeJfdBrXuz-YiZizDfU&_nc_ht=scontent-lga3-1.xx&oh=72670bd1c6dcc3328bec2cf20e23bd17&oe=60EC426E
\fi
