% vim: keymap=russian-jcukenwin
%%beginhead 
 
%%file 08_12_2021.fb.lesev_igor.1.strategicheskaja_pobeda_zelenskii
%%parent 08_12_2021
 
%%url https://www.facebook.com/permalink.php?story_fbid=4857197647644633&id=100000633379839
 
%%author_id lesev_igor
%%date 
 
%%tags biden_joe,donbass,geopolitika,kitaj,krym,putin_vladimir,ukraina,zelenskii_vladimir
%%title Стратегическая победа Зеленского
 
%%endhead 
 
\subsection{Стратегическая победа Зеленского}
\label{sec:08_12_2021.fb.lesev_igor.1.strategicheskaja_pobeda_zelenskii}
 
\Purl{https://www.facebook.com/permalink.php?story_fbid=4857197647644633&id=100000633379839}
\ifcmt
 author_begin
   author_id lesev_igor
 author_end
\fi

Стратегическая победа Зеленского

В школе на особо скучных уроках с соседом по парте я играл в словесную игру.
Нужно было по очереди называть слово, при этом максимально нелепое, составляя
одно бесконечное предложение. Обычно выходило что-то вроде «корова трахнула
ветку» или «училка нюхает борщ».

\ifcmt
  ig https://scontent-mxp1-1.xx.fbcdn.net/v/t39.30808-6/265201909_4857196734311391_4404586559303008698_n.jpg?_nc_cat=109&ccb=1-5&_nc_sid=730e14&_nc_ohc=QMoTlIpgyJgAX-ZDZ25&_nc_ht=scontent-mxp1-1.xx&oh=4044f1acaba7e9123e0b6e1faa058b26&oe=61B66E93
  @width 0.4
  %@wrap \parpic[r]
  @wrap \InsertBoxR{0}
\fi

У ботофермы Офиса сейчас похожая по сложности задача. Переговоры Байдена и
Путина нужно описать, как «стратегическая победа Зеленского», а по итогу все
равно выходит «корова трахнула Вову» и «Блогер нюхает не борщ».

Отметьте, что русская экспертная тусовка изначально к этим переговорам была
настроена скептически. Путин в самом начале и вовсе залупил, давайте
побыстрячку, у меня тут после еще одна протокольная встреча.

Но в истории что-то прорывное, революционное и судьбоносное не происходит
каждую неделю, год и даже десятилетие. Да, случаются события, которые
перелопачивают очень многое. Если говорить о нас, то здесь есть две ключевые
даты – 1991 и 2014 год. В первом случае появилась Украина, а во втором Украина
стала принципиально другой.

Но какие видимые изменения за этот же период времени произошли, скажем, в
Китае? А вроде бы и никаких. Попробуйте нащупать отличие Китая 1985 года от
Китая, скажем, 1990 года, и перенесите его мысленно в Китай 1995. Те же
коммунисты у власти, какие-то имена генсеков меняются, все тот же Дэн Сяопин
над ними, ну чуть студентоту подавили на Тяньаньмэнь. Но что поменялось
принципиально?

А это именно тот случай, когда вода точит камень. В 1991 году Китай – 10-я
экономика мира, а в 2021-м – уже давно как вторая, при этом с минимальным
отставанием от первой. И это все не вписывается в какое-то одно ключевое
событие. Но вдруг так стало, что уже и в Нигерии, и в Бразилии при подаче
новостей почти в каждом выпуске что-то, а таки говорится о Китае.

И вот вчерашняя встреча Байдена и Путина. Вроде бы принципиально и ни о чем.
Пресс-релизы Кремля и Белого дома вообще как армянское радио. Где-то даже в
чем-то совпадают, но никакой даже околоконкретики.

И все же, уже можно зафиксировать ряд ключевых моментов, которые всеми
сторонами воспринимаются по умолчанию.

Первое – Крым. О нем уже даже не чешут. На Западе осталась словоблудливая
конструкция «уважение территориальной целостности» и какие-то дежурные
заявления в годовщину присоединения. И, собственно, всё. Во всем мире претензии
по Крыму к России приблизительно такие же, как претензии по Пуэрто-Рико к
Штатам. Иногда только обдолбанные наркоманы с целью личного пиара, ну и чтобы
децл украсть на организации, проводят какие-то платформы, чтобы потом вся
страна ржала над историями в стиле «бегущего мальчика по минным полям Крыма».
Кстати, кто-то знает, чем этот «постоянно действующий офис Крымской платформы»
занимается?

Второе – Донецк и Луганск де-факто интегрированы в Россию. Мир спокойно
проглотил ру-паспортизацию региона, за которой последовала рутинная
экономическая унификация. По сути, ЛДНР – это продолжение Ростовской области с
«особым статусом».

Последнее окно возможностей у Украины было в 19-м. Но у нас тут визжали о
«раковой опухоли» и неприемлемости региона с особым статусом в составе расово
чистой Украины. Ну не захотели и не захотели. Зато русские не отказались.

И заметьте, как сменилась за эти несколько лет мировая риторика. Там требуют,
чтобы Россия не взяла чего-то большего у Украины, а не вывела чего-то там из
ЛДНР. Байден угрожает России «страшными санкциями» не за Донецк и Луганск, а за
то, чтобы еще не было Мариуполя, Краматорска и Мелитополя.

Третье – экономическое взаимодействие России и Запада. У нас это традиционно
сводят к узкопрофильной теме «Северный поток-2», который сам по себе важен, но
он точно не ключевой пуп этих отношений. «Севпоток-2» - это индикатор этих
отношений. По итогу разговора Байдена и Путина стало ясно, что потоку быть. А
значит быть и торговым отношениям, которые по-прежнему для России остаются
ключевыми. Все друг с другом торгуют и пилят выгоду.

И это не только правильно, но и нравоучительно. В мире не так много
шизофренических стран, которые устраивают экономическую блокаду своему же
региону, а потом таращат глаза и удивляются, почему законы физики и логики
действуют вопреки расистским хотелкам.

Четвертое. Россия таки добилась дискуссии о не расширении НАТО. Да, это пока
только дискуссия. Но мельком гляньте на карту Европы, чтобы понять, о ком это
дискуссия. Белоруссия – это по факту уже Россия, а значит остается только
«вторая Франция». Собственно, «Украина никогда не будет в НАТО», и какая цена
этой опции, на ближайшее время становятся центральной темой между Москвой и
Вашингтоном.

Вот это краткое изложение этого дежурного разговора об Украине без участия
Украины. Блогеру старик Джо обещал позвонить в четверг. Передаст привет,
возможно, даже не спутает имя.

А в Офисе пока могут строчить посты о стратегической перемоге. Ребята все-таки
плотно поработали. В начале месяца упредили государственный переворот. Теперь
вот отразили неминуемое вторжение России. Спасли Харьков от окружения.
Предотвратили высадку морского десанта в Одессе. Не дали врагу осквернить
могилу цадика Нахмана в Умани. Да и куда было отступать, когда за спиной
оставался только Лемберг?

\url{https://t.me/Lesev_Igor}

\ii{08_12_2021.fb.lesev_igor.1.strategicheskaja_pobeda_zelenskii.cmt}
