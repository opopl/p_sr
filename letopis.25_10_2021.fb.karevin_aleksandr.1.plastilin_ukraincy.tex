% vim: keymap=russian-jcukenwin
%%beginhead 
 
%%file 25_10_2021.fb.karevin_aleksandr.1.plastilin_ukraincy
%%parent 25_10_2021
 
%%url https://www.facebook.com/permalink.php?story_fbid=3036459559926677&id=100006879888184
 
%%author_id karevin_aleksandr
%%date 
 
%%tags chelovek,obschestvo,ukraina
%%title ПЛАСТИЛИНОВЫЕ УКРАИНЦЫ
 
%%endhead 
 
\subsection{ПЛАСТИЛИНОВЫЕ УКРАИНЦЫ}
\label{sec:25_10_2021.fb.karevin_aleksandr.1.plastilin_ukraincy}
 
\Purl{https://www.facebook.com/permalink.php?story_fbid=3036459559926677&id=100006879888184}
\ifcmt
 author_begin
   author_id karevin_aleksandr
 author_end
\fi

ПЛАСТИЛИНОВЫЕ УКРАИНЦЫ

Вчера один мой ФБ-друг (с которым я и в реале знаком) опубликовал пост под
названием «растерянность». Там про то, что «Путин не знает, что делать с
Украиной», в чём недавно сам и признался. «Растерянность Путина видна как
никогда» и т.п.

\href{https://buzina.org/golos-naroda/4027-plastilin.html}{%
ПЛАСТИЛИНОВЫЕ УКРАИНЦЫ, buzina.org, 26.10.2021%
}

\ifcmt
  pic https://buzina.org/images/stories/articles/2021/2021-10-26-12-52-54.jpg
  @width 0.6
\fi

Ну, то такое. Ничего нового. Я на другое обратил внимание. Автор поста
утверждает, что «Украина уже давно переродилась». Стала, дескать, русофобской и
дальше будет ещё хуже. А что с этим делать не знает ни Путин, ни вообще никто. 

В общем, пост полный уныния, который, однако, получил кучу «лайков» и
перепостов.

Спору нет – ситуация хреновая. Очень хреновая. Не согласен я тут с тем, что
Украина (а точнее её жители, ибо речь тут о них) переродилась.

Да не перерождались они! Просто многие украинцы (в смысле – жители Украины) –
пластилиновые. И всегда были такими. Из них, как из пластилина, можно лепить
всё, что угодно. Да и лепить особо не надо. Они сами принимают форму, нужную
тому, кто сильнее. У них это автоматически происходит. Во всяком случае,
впечатление такое, что автоматически.

Их жизненное кредо – примкнуть к сильному. Позавчера они были с коммунистами,
потому что сила была  за коммунистами. Вчера –  с как бы демократами (по
аналогичной причине). Сегодня –  с русофобами-бандеровцами. А с кем будут
завтра, зависит от того, кто на Украине будет сильным завтра.

Форму они меняют не потому, что что-то переосмыслили (они ничего не могут
переосмыслить, ибо имеют пластилиновый мозг). И не потому, что поменяли
убеждения – нет у них убеждений. Или, правильнее сказать, все их убеждения
заключаются в том, чтобы подстроиться под власть. Не под конкретного
Зеленского, Порошенко и прочих (те все тоже пластилиновые), а под реальную
власть, центр которой, по всей видимости, не на Украине находится.

Признак пластилинового украинца – это именно готовность подстроиться под
победителя, независимо от того, кто этот победитель. Мы же все это знаем из
истории и видим в современности. 

Говорю о пластилиновых украинцах, поскольку речь идёт о жителях Украины. Но
таких (пластилиновых) полно и в РФ. Да и в любой другой стране.

Так было и так, скорее всего, будет дальше. И где здесь повод для уныния?

Да – этих пластилиновых человечков много, даже – большинство. Ну и что?
Повторюсь: так всегда было. Но не всегда наша страна пребывала в такой,
извините за выражение, заднице. Не от пластилиновых это зависит. Просто надо
учесть, что они есть. И что они пластилиновые.

Они - пластилиновые, а мы - нет. Мы не подстраиваемся под сиюминутного
победителя. У нас есть стержень. И если завтра мы станем силой, то
пластилиновые мгновенно подстроятся под нас.

Проблема в том, что у нас сейчас не получается стать сильными. То есть проблема
в нас, а не в них. Но у нас хотя бы есть шанс стать силой. И именно потому, что
мы не пластилиновые. А у них (пластилиновых) такого шанса нет.

Поэтому не надо устраивать всеобщий плач по «переродившейся стране». Она не
переродилась. И не переродится, пока есть мы.

\begin{cmtfront}
\uzr{Виталия Козина}

Вы правы, у нас есть шанс, но прежде всего, мы должны поставить цели и
элементарно написать план действий. Только план не встроиться тушкой или
чучелком в чужую повестку. Нужно осознать свои приоритеты и интересы и
использовать свой потенциал для блага страны, а не кучки олигархов! Вся
патетика вокруг пластилиновости, просто, ни о чем! Вопрос, как всегда, ключевой
у кого собственность и кому принадлежит страна? Пока, это будет территорией
которую грабит капитал, причём не принципиально, свой или чужой, так и будем
искать сильное плечо, куда облокотиться.

\uzr{Ирина Буторина}

я не против братского Союза, но считаю называть нас одной нацией - это обижать
и украинцев и белоруссов, которые претендуют на самостийность.

\uzr{Арсений Родына}

Это всего-навсего демократия. Она везде "пластилиновая". В России её
значительно меньше, поэтому там больше настоящей свободы. Вы, кстати, тоже
(извините за напоминание) вели себя как "пластилиновый", когда голосовали за
Зеленского. "Непластилиновые" - это те, кто не верят ни в какие идеологии.
Они-то и есть украинский (малороссийский) народ. С ними, я думаю, всё в
порядке. Их мало, а в фейсбуке так и вовсе они незаметны, но они-то по крайней
мере хоть что-то значат. Благодаря им Украина останется русской.

\end{cmtfront}

\ii{25_10_2021.fb.karevin_aleksandr.1.plastilin_ukraincy.cmt}
