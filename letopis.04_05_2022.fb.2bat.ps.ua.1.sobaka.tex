% vim: keymap=russian-jcukenwin
%%beginhead 
 
%%file 04_05_2022.fb.2bat.ps.ua.1.sobaka
%%parent 04_05_2022
 
%%url https://www.facebook.com/2batrightsector/posts/121729477175380
 
%%author_id 2bat.ps.ua
%%date 
 
%%tags 
%%title Хто знає що робити з собакою?
 
%%endhead 
 
\subsection{Хто знає що робити з собакою?}
\label{sec:04_05_2022.fb.2bat.ps.ua.1.sobaka}
 
\Purl{https://www.facebook.com/2batrightsector/posts/121729477175380}
\ifcmt
 author_begin
   author_id 2bat.ps.ua
 author_end
\fi

Вчора у фронтовий госпіталь солдати привезли пораненого хлопчика 9-ти років. На
щастя військові хірурги всі були вільні, та і поранення в ногу не загрожувало
життю та здоров’ю дитини, але в хол забіг схвильований солдат:

- Там собака ще? Хто знає що робити з собакою?

Забігаючи наперед скажу, що собацюрі в цей вечір пощастило тричі (якщо взагалі
тут можна говорити про везіння): 

Перше везіння в тому, що вона потрапила до рук українських солдатів  в яких
мама  виховала  співчуття до всього живого. 

Друге везіння - три осколки пробили м‘які тканини, не заділи кісток і не
залишились в тілі. Один осколок залишив круглу діру у вусі, інший пробив бік, а
третій вразив задню ногу. 

Третє, і головне, везіння звати доктор Чарльз - американський волонтер, який
допомагає нашим військовим хірургам в прифронтовій зоні. 

Собака лежала сумирно весь час доки її обстежували магнітом на предмет
осколків, доки голили лапу під крапельницю, доки знеболювали, зашивали і несли
на рентген. 

- я тебе люблю. Так люблю! Ти хороша дівчинка, - повторював Чарльз кожного разу
коли собака перелякано озиралась на купу людей що обступили її з усіх сторін, -
як її звати?

- Ада! Ада!

Ада, коли почула голос хазяйки,  спробувала зірватися на ноги, але ми її
притримали. Взагалі серед нас мінімум двоє постійно гладили її заспокоюючи тож
військові хірурги, що вже відправили хлопчика в головний госпіталь, жартували
що собака думає що вже потрапила в собачий рай - стільки уваги. 

Але блін...коли Чарльз закінчив роботу і погладив її морду - вона вдячно
лизнула його руку. Тільки його! Чарльз поцілував її в морду. 

\ii{04_05_2022.fb.2bat.ps.ua.1.sobaka.pic.1}

Коротше, доповідаю, силами медиків Другого окремого батальйону ДУК ПС та групи
анонімних американських волонтерів  хороша дівчинка Ада була врятована і зараз
відлежується і чекає приїзду ветеринарів. В той же час воєнні хірурги
попіклувалися про її хазяїна. 

Взагалі, армійські хірурги це окрема каста людей які заслуговують максимальної
поваги і я би не хотів, щоб історія про песика затьмарила їх роботу - нажаль
вони критично ставляться до медіа і не дають змоги прославити їх так, як вони
того заслуговують, але... нема слів. Вони тут людей збирають із таких агрегатних
станів... і жодного вихідного починаючи з першого дня. 

Мені здається що цих людей вже неможливо любити ще більше, але  щодня вони
доводять що люрба і повага річ безмежна. 

Фото: Мауглі і Француза

\ii{04_05_2022.fb.2bat.ps.ua.1.sobaka.cmt}
\ii{04_05_2022.fb.2bat.ps.ua.1.sobaka.cmtx}
