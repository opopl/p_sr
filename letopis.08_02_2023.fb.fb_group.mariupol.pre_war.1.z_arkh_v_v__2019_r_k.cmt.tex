% vim: keymap=russian-jcukenwin
%%beginhead 
 
%%file 08_02_2023.fb.fb_group.mariupol.pre_war.1.z_arkh_v_v__2019_r_k.cmt
%%parent 08_02_2023.fb.fb_group.mariupol.pre_war.1.z_arkh_v_v__2019_r_k
 
%%url 
 
%%author_id 
%%date 
 
%%tags 
%%title 
 
%%endhead 

\qqSecCmt

\iusr{Zhanna Savelyeva}

Ви викладаєте усі скарби разом. Розтягніть нам задоволення)

\begin{itemize} % {
\iusr{Ihor Yarmolenko}
\textbf{Zhanna Savelyeva} 

Я взагалі хотів викласти всі фото за 2019 рік в одному пості, але їх було надто
багато. По одному фото викладати фізично нема часу, але найголовніше - в
підбірці з кількох десятків світлин починаєш відчувати місто, його вайби. Це як
обходити скульптуру з різних боків і дивитись на неї з ріїної відстані. Коли
людина дивиться такі підбірки, вона переноситься в той час і в те місце. Тому я
обмежую їх за часом - 2015, 2017, 2019... Можна побачити, як мінялось місто, як
мінялись тренди, його дух. Або щось, навпаки, залишалось незмінним. Люди й їх
емоції на фото теж важливі, але це до метрів - я не фотограф, усе це я зберігав
для себе, не думав ніколи раніше, що буду викладати в такому форматі.

Щодо наступних випусків - є ще дуже багато, не хвилюйтесь 🙂

\iusr{Zhanna Savelyeva}
\textbf{Ihor Yarmolenko} бачу, що матеріал шикарний, треба смакувати.

\iusr{Галина Медведева}
\textbf{Ihor Yarmolenko} Дякую за фото, яка краса.
\end{itemize} % }

\iusr{Виктория Лененко}

👍

\iusr{Maryna Holovnova}

Чудові знімки! Ігорю, дякую, що ділитесь) Наче прогулянка містом справжня)

\iusr{Любов Каминская}

Дякую за фото.

\iusr{Наталья Удачина}

Спасибо

\iusr{Елена Сартания}

Дякую!

\iusr{Egita Zolota}

\ifcmt
  igc https://i.paste.pics/2135ee3e8e292a2b6bde1036c6c8402c.png
	@width 0.1
\fi
