% vim: keymap=russian-jcukenwin
%%beginhead 
 
%%file translations.rus2ukr.tkachev
%%parent translations
 
%%url 
 
%%author_id 
%%date 
 
%%tags 
%%title 
 
%%endhead 

Єдиний, Неповторний

Ми всі живемо в одному і тому ж Світі, який все більше набуває рис величезного
будинку або ж надто великої комунальної квартири.  Але в той же час, ми живемо
у досить обмежених, закритих Всесвітах, що є причиною того, що розмова двох
Землян тут і там ризикує бути схожою на розмову лунатика з марсіанином. Ці
відокремлені світи міцно зачинені в наших грудях і головах. Ці внутрішні світи
у двох випадково вибраних людей ніколи не співпадають, і хочеться сподіватись,
дай Боже, щоби хоча б іноді ці Світи мали бодай якусь точку дотику.

Чотири людини їдуть в одному купе до одної і тої ж станції. Їдуть фірмовим потягом з Києва до Львова. 

ЕДИНСТВЕННЫМ, НЕПОВТОРИМЫЙ

Мы все живем в одном и том же мире, который все больше приобретает черты огромного дома или даже непомерно разросшейся коммуналки. Но в то же время, мы живем в довольно обособленных, закрытых мирах, отчего разговор двух землян сплошь и рядом рискует быть похожим на разговор лунатика с марсианином. Эти обособленные миры заключены в наших грудных клетках и черепных коробках. Это внутренние миры, которые у двух наудачу выбранных людей никогда не совпадают. Дай Бог, чтоб хотя бы пересеклись или только соприкоснулись.

Четыре человека едут в одном купе до одной и той же станции. Едут на фирменном поезде из Киева во Львов. Пиджаки сняты, чемоданы водворены под сиденья, постели застелены, билеты проводнице отданы. Можно раскладывать на столике «что Бог послал», заказывать чай и коротать время за самыми приятными занятиями – беседой, едой, смотреньем в окно и мерным, усыпляющим покачиванием под стук колес.

Первая порция чая выпита, и проводницу попросили «повторить». Совершилось шапочное знакомство и нащупаны темы для разговора. Выборы, футбол, курс доллара, анекдоты, истории из жизни…

* * *

Один из попутчиков был в столице по делам министерства. Он ходил бесшумными шагами по мягким коврам, сидел с умным видом на убийственно длинном совещании, что-то доставал из папочки, зачитывал, выслушивал замечания, делал записи в блокнот. Он краснел, зевал, вытирал пот, смотрел на часы, считал оставшееся время до отъезда. Потом, за два часа до поезда он пообедал, нет, скорее – поужинал, наблюдая в окно за тем, как темнеет воздух на улице и зажигаются огни большого города. И вот теперь он здесь, в купе, слушает собеседников, вяло поддерживает разговор и мечтает вытянуть ноги на сырой простыне и такой же сырой простыней укрыться.

– Вот у нас был случай во дворе. Мужик имел любовницу в соседнем доме. Как-то раз говорит жене, что его отправляют в командировку. Собирает белье, бритву, тапочки в чемодан и «едет» в соседний дом на неделю. Ну а бабы – они везде бабы. Любовница его заставляет по вечерам, когда совсем темно, мусор выносить. Ну, он и задумался о чем-то своем, вынес мусор и на автопилоте пошел обратно. А ноги его привычно привели в собственный дом, задумчивого, в тапочках и с чужим мусорным ведром. Как он отверчивался, ума не приложу.

Последняя фраза была перекрыта взрывом дружного хохота.

* * *

Второй пассажир ездил к дочери. Помогал по хозяйству, нянчил внука. Уже третий раз он приезжает в Киев недели на две и сидит в четырех стенах на седьмом этаже, спускаясь вниз только вместе с коляской.


Он ни разу не был ни на Майдане, ни на Крещатике, не посетил ни одного театра или музея. Он не сделал ничего из того, что рисовало ему воображение, когда он гордо говорил соседям: «Еду в Киев к детям». По музеям он ходить, конечно, не любит и театр не понимает, но искренне считает, что житель столицы или гость ее обязан, так сказать, потребить некую порцию культурной пищи. В этот раз опять не заладилось. Но зато новости он слушал по телевизору регулярно и был способен поддержать любую болтовню на политическую тему.

– У моего батьки в колхозе был сумасшедший голова. Злой, как зверь. Мог людей избивать даже, и никто ему слова не говорил. Все боялись. Как-то раз одна баба пошла на колхозное поле кукурузу красть. Только приступила, видит – на дороге фары мелькнули. Она думает: «Это УАЗик головы!» – и со страху бежать. Через дорогу – кладбище старое. Она между могил села и сидит, как мышь, дрожит. Уже УАЗик проехал, а она не выходит. Вдруг смотрит – два пьяных мужика ведут по кладбищу бабу. Остановились недалеко от той, что спряталась, достают стакан и бутылку, наливают и говорят своей бабе: «Пей». Та отвечает: «Не буду». Они снова: «Пей!» Ясно, что хотят ее напоить и… того… Та баба, что спряталась, видит – надо спасать землячку. Ну и, пока те свое твердят: пей – не буду, пей – не буду, она вытягивает руку из венков и говорит: «Давай я выпью».

Поезд заезжает в тоннель, и взрыв хохота совпадает по времени с внезапным наступлением темноты. Кажется, что у всех потемнело в глазах именно от смеха.

– И что было дальше?

– А что было? Посадили ее – ту, что в венках спряталась.

– За что?

– За то, что один из тех мужиков умер на месте от разрыва сердца.

* * *

Третий путник был совсем еще молодым человеком, почти мальчиком. Он ездил сдавать документы в консерваторию и был плохо способен к поддержанию разговора. Из всего существующего на свете занимала его только музыка. В Киеве он был впервые и из всех впечатлений дня главными были испуг и усталость. Молодой человек был напуган многолюдством, суетой и расстояниями. Столица показалась ему муравейником, в котором все спешат и все друг другу безразличны. Уже к концу дня он смертельно устал от метро, от шума, от контраста между лицами, улыбающимися с рекламных плакатов, и угрюмо сосредоточенными лицами на улицах. Юноша привык слушать больше музыку, чем слова, и к концу этого дня звуки Киева измучили его слух.

Как тот набоковский шахматист, которому мир представлялся разбитым на клетки, а сама жизнь – похожей на хитрую партию с неизвестным соперником, этот молодой человек представлял мир зашифрованным нотными знаками. Он еще не успел испытать ни любви, ни ненависти, он еще даже не начал бриться, и не интересовало его покамест ничего, кроме специальных предметов, преподаваемых в только что оконченном музучилище. Ему было отчасти неловко, отчасти скучно. Но просто молчать и смотреть в окно он позволить себе не мог.

– У Мусоргского… Ну, знаете, есть такой композитор – Мусоргский. Он был очень талантлив и опередил свое время, только он пил очень много. Ну, короче, у него есть такое произведение – «Картинки с выставки». Это такие музыкальные пьесы: «Тюильрийский сад», «Два еврея, богатый и бедный», «Балет невылупившихся птенцов». И там есть такой фрагмент, который называется «Быдло». Нам преподаватель рассказал анекдот, как на концерте однажды выходит женщина-конферансье и объявляет: «Мусоргский. Падла.» Ей из оркестра шепчут: «Дура. Быдло. Быдло!». А она поворачивается к ним и говорит: «Сам ты быдло». Потом опять в зал громким голосом: «Мусоргский. Падла!»

Безразличные к Мусоргскому, мужики все же от души рассмеялись над историей, повторяя затем в уме «падла» и «быдло» и стараясь не перепутать, что здесь ошибка, а что – название произведения.

* * *

Четвертый попутчик ездил в Киев к товарищу. Они были в школьные годы «не разлей-вода» и продолжали дружить после армии. Потом дороги их разошлись, друг перебрался в Киев, и вот, лет двадцать спустя, они нашлись на сайте «Одноклассники». Стали переписываться в сети, потом друг пригласил его в гости. Теперь он возвращался домой, перебирая в памяти обрывки впечатлений. Главным впечатлением было посещение Лавры. Приглашавший и принимавший его друг искренно удивился, узнав, что школьный товарищ много раз бывал в Киеве, но ни разу не удосужился спуститься в пещеры и пройтись с молитвой по темным подземным коридорам. Б один из дней они и поехали в Лавру, спустились в Ближние пещеры и неторопливо обошли их. Возле каждого гроба останавливались, крестились и целовали стекло над мощами. Друг кратко рассказывал о каждом святом, и было видно, что с Лаврой и ее историей его связывает крепкая многолетняя любовь.

Это воспоминание и эти впечатления действительно сейчас казались возвращавшемуся мужчине главными во всей поездке. Все остальное отодвинулось на обочину сознания и представлялось маленьким и несущественным. Было действительно странно, почему до сих пор он ни разу не бывал в этом полумраке, где тьму слабо рассеивают свечи, где воздух пахнет особо и где люди молятся у гробов с мертвыми телами так, словно лично знакомы с усопшими и продолжают с ними общаться, несмотря на очевидно и давно наступившую для тех смерть.

Четвертый попутчик не позволял себе окунуться в мистические переживания и размышления полностью. Он чувствовал, что воспоминания эти остались на дне души, как не растворившийся сахар на дне выпитого стакана чая.


Он чувствовал, что воспоминания эти никуда не денутся, что они не раз еще воскреснут в душе и поведут ее, душу, за собой в какие-то пока не известные дали.

Он пил чай вместе со всеми, и смеялся над анекдотами и историями, и рассказывал их сам, когда подходила очередь. Только рассказывал что-то очень короткое, вроде: «Рабинович, вы устроились? – Нет. Работаю».

Утреннее пробуждение в поезде всегда хлопотно. Тот интимный момент, когда человек проснулся и хочет вылезти из-под одеяла, в поезде отягчен присутствием большого количества чужих людей.

Люди встают, одеваются, с помятыми лицами выходят в коридор, занимают очередь в уборную. Люди опять заказывают чай, смотрят на часы, спрашивают друг друга, сколько осталось до прибытия и без опоздания ли идем.

Когда поезд прибудет на перрон, пассажиры с чемоданами и сумками в руках будут, один за другим, покидать вагоны. Можно представить себе, как к каждому из них подходит некто и задает один и тот же вопрос: «Где вы были?». Люди, покидающие поезд «Киев – Львов», прибывший по назначению, будут отвечать: «Я был в Киеве».

«Я был в Киеве», – скажет чиновник министерства.

«Я был в Киеве», – скажет отец семейства, проведавший дочку с зятем и внука.

«Я был в Киеве», – скажет юноша, сдавший документы в консерваторию.

«Я был в Киеве», – скажет мужчина, гостивший у друга и впервые посетивший Лавру.

Никто из них не соврет, в случае если вопрос будет задан, но очевидно, что все они побывали в разных городах. И дело не в том, что Киев огромен и каждый находит там то, что его интересует. В этом смысле огромен всякий город и всякое место на земле.

Дело в том, что мы все живем в своем собственном, неповторимом мире, из которого изредка высовываемся, чтобы понять, нет ли какой-то опасности. Мы живем настолько обособленно, что ни совместное пребывание на одном кусочке земли, ни чтение одних и тех же книг, ни одинаковая пища, ни работа, ни учеба, ни война не делают нас одинаковыми. Каждый остается самим собой. Более того, и войну, и любовь, и работу каждый переживает по-своему, лишь приклеивая общеупотребительные слова к своему уникальному опыту.

Я думаю об этом часто. И, быть может, думают об этом проводники, провожающие взглядом людей, приехавших, казалось бы, из одного и того же места, но на самом деле побывавших в совершенно разных местах.
