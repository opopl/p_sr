% vim: keymap=russian-jcukenwin
%%beginhead 
 
%%file 11_10_2020.news.ua.strana_ua.1_dnr_trip
%%url https://strana.ua/articles/istorii/294118-kak-doekhat-iz-kieva-v-donetsk-cherez-rossiju.html
%%tags dnr,strana.ua
%%parent 11_10_2020
 
%%endhead 

\subsection{Город без масок. Путешествие из Киева в Донецк через Россию и обратно. Личный опыт }
\url{https://strana.ua/articles/istorii/294118-kak-doekhat-iz-kieva-v-donetsk-cherez-rossiju.html}

После того, как попасть в "ДНР" через украинскую границу стало проблематично,
те, кому все-таки нужно туда попасть, освоили обходной вариант - через
территорию России.

Это не одобряется украинскими властями, но, если ситуация безвыходная,
приходится идти на то, что не одобряется. Такая история случилась и с нами.
Скооперировавшись, мы отправились в Донецк по "российскому маршруту".

\subsubsection{Путь в Донецк}

Для того, чтобы найти перевозчика, вам не потребуется перевернуть мир. Их
контактами буквально усеяна сеть.

Вся задача - выбрать из них того, кто больше других покажется оптимальным по
соотношению "цена-качество". Есть такие, кто за одну поездку по маршруту
Киев-Донецк (или обратно) стараются содрать 5 тысяч гривен. Есть такие, манера
разговора которых сразу убеждает: лучше поискать другой вариант. Перебрав
несколько контактов, мы остановились на том, где с нами поговорили спокойно и
компетентно, а за дорогу попросили 2500 гривен с человека.

Условия вполне человеческие, с учетом состояния рынка…

Правда, накануне выезда нам позвонили и перенесли время старта с 16.00 на
03.00, что, согласитесь, не совсем одно и то же. Но деваться было некуда,
выезжать требовалось срочно - и мы смирились с неожиданным предстартовым
недосыпом.

Стартовали от киевского железнодорожного вокзала - это стандартная точка для
перевозчиков такого типа.

Нас вез белый "Опель Виваро" с полным комплектом пассажиров - 8 человек. Щедро
матерившийся, избыточно бодрый водитель уже под Харьковом столкнулся с
серьезной проблемой - одно из колес начало "гулять", его потребовалось срочно
заменить. На наше счастье, у него оказалась "запаска" и домкрат - за каких-то
полчаса он вернул машине боевое состояние, и мы завершили наш прыжок к
украинской границе.

Там, по схеме, украинский водитель нас оставлял, передавая в руки
"ДНР-овскому". Но непосредственной передачи не было. Нам сказали, что пересечь
"серую зону" придется пешком. Или, если мы хотим быстрее и комфортнее - на
такси, которые берут по 200 гривен с человека. Идти не хотелось никому, тем
более, что в нашем составе была пара пожилых тетенек. Поэтому ангажировали 2
такси, с которыми прошли украинскую таможню и добрались до российского
пропуска. Мы не стали разбираться, в доле ли водитель нашего "Опеля" с
таксистами - но сильно удивились бы, если бы оказалось, что нет. 

Украинский досмотр не был слишком обременительным, но вышел слегка
унизительным. Знаете, когда лопатят ваш бумажник и вашу дорожную сумочку,
заглядывая во все щели и вытаскивая всякие интимности - это не очень
вдохновляет.

Искали, в основном, какие-то "левые" кредитные и дебетные карты - их провоз
через границу категорически нервирует таможенников. Внимательно изучали
документы, с пониманием усмехались, видя донецкую прописку (а она была,
разумеется, у всех). Продолжалось это недолго, но впечатление оставило
неприятное - как будто в магазине тебя обматерили (но, слава Богу, не
обсчитали).

Следующий этап - российский пропускной пункт. Туда нас пустили не сразу.
Пришлось прождать более двух часов под жарким солнцем. Очередь просто не
двигалась, и пограничники никак не объясняли причину задержки - бесцветно
отвечали, что нет приказа пропускать, вот они и не пропускают. На толпу с
украинской стороны они смотрели совершенно безразличными глазами. Ну,
действительно, нет приказа - что тут поделаешь?

В конце концов, решетчатые ворота перед нами, наконец, распахнулись.

После этого все пошло легко - к дончанам, добирающимся через российскую
территорию, их таможенники относятся с пониманием и препятствий стараются не
чинить. Заполняешь миграционную карточку, где в графе "Цель визита" с невинным
видом пишешь "Транзит". Проходишь досмотр вещей (руками его не лапают -
просвечивают, как в аэропорту). Ну, в принципе, и все!

Ты - на российской стороне, в Белгородской области, пропускной пункт
"Нехотеевка" (нашим уставшим мозгам показалось, что в этом названии есть нечто
глубоко некрасовское).

Тут нас ждал водитель с "ДНР-овскими" номерами. Он сразу признался, что поспать
накануне ему не удалось, поэтому ему хорошо бы в дороге остановиться и
покимарить хотя бы полчаса. Раздраженные предыдущими задержками, мы выслушали
это без удовольствия, но вынуждены были проглотить. Было уже 14 с лишним часов,
всем хотелось, наконец, хотя бы удалиться от границы.

Забегая вперед, отметим, что в дороге водитель пытался поспать раза два, и
вроде ему это удавалось, но бодрее от этого он не становился.

В итоге, к нему подсел один из мужичков нашего состава и всю дорогу кормил его
байками сначала из своей жизни, потом - из жизни своих друзей, и наконец - из
жизни народов мира.

Благодаря этому мы добрались до границы с т.н. "ДНР" без приключений. А то ведь
в сентябре аналогичная машина с аналогичным маршрутом опрокинулась в кювет как
раз из-за заснувшего водителя. Итог - два трупа: сам водитель и один пассажир…

Но вообще, путешествие по России (мы пересекли три области - Белгородскую,
Воронежскую и Ростовскую) оказалось совсем не напряжным - благодаря прекрасным
дорогам и отлично развитому придорожному сервису. В итоге, к следующей границе
мы подъехали на позитиве. Было 2 часа ночи, звезды романтично подмигивали с
темного неба. Особо торопиться нам не приходилось - все равно, в т.н. "ДНР", на
ПП "Успеновка" раньше 5 утра не пустили бы, такие правила.

В России нас притормозили - опять без объяснения причин, спасибо, что всего на
час.

Немного пришлось подождать на "ДНР-овской" стороне - почему-то парни в хаки
бегали, как заводные, от одного строения к другому. То ли проверка, то ли
побудка… В конце концов, мы прорвались. Правда, чуть отъехав от границы, были
остановлены вновь - на выезде из поселка Успеновка установлен еще один пост,
дополнительный и окончательный.

Водителю пришлось выходить и решать вопрос, суть которого от нас ускользнула.
Но уже - действительно последний. Кстати, как он признался нам по большому
секрету, в результате всех подорожных расходов, с каждых двух с половиной тысяч
гривен, уплаченных клиентом, конторе остается всего тысяча. Остальное идет
добрым людям в разной форме. Сколько причитается водителю, он так и не сказал.
Его можно понять.

В центре Донецке мы высадились около 09.00. Таким образом, в дороге мы провели
почти 30 часов, с робкими попытками сна там и сям. Это тяжело, но оно того
стоило - ведь мы могли свободно передвигаться по городу, без всяких намеков на
обсервацию или самоизоляцию. Всех въезжающих из России на границе т.н. "ДНР"
пропускают без требования пройти вот это вот все.

Поэтому, кстати, водитель нас заранее предупредил, чтобы мы ни в коем случае не
признавались, что едем из Украины. В итоге, мы договорились, если нас спросят,
"косить" под группу курортников, возвращающихся в Донецк из Сочи. Но нас так и
не спросили.

\subsubsection{Что там в городе? }

Первое, что бросается в глаза, когда попадаешь в Донецк из Киева - практически
полное отсутствие людей в масках (мы были в городе в середине сентября). 

Предупреждающие надписи о том, что нужно пользоваться масками, есть везде - на
дверях магазинов, на входах в заведения, на окнах транспорта. Но им практически
никто не следует, кроме продавцов и кассиров, да и то не везде.

Говорят, иногда контролирующие органы устраивают проверки в супермаркетах,
изучая соблюдение карантинных норм. Нам рассказывали о том, как пытались
оштрафовать людей, оказавшихся там без масок. Возможно, даже оштрафовали. Мы ни
с чем подобным не сталкивались ни разу.

В нескольких донецких больницах организованы специальные отделения, где
работают медики по программе борьбы с пандемией. Там следят за местными, у
которых все-таки обнаружили коронавирус (такие есть, хотя официальная
статистика показывает крайне малое их число). Туда же попадают и приезжие из
Украины, обязанные пройти двухнедельную обсервацию. Иногда этот срок
сокращается - не во всех медицинских заведениях хватает персонала и ресурсов
для того, чтобы обеспечить полный срок пребывания пациентов. К тому же его
можно сократить, если знать, с кем и как договориться.

Донецк живет совершенно той же жизнью, что и обычно в последние годы. Есть,
правда, нюансы. Из-за карантина сильно затруднены поставки из России. Как
следствие, ассортимент товаров сузился. Правда, продуктов это касается лишь в
малой степени. Тот пищевой набор, к которому дончане привыкли после войны, в
основном, представлен на прилавках. Жить, в общем, можно, и голодать пока что
никто не собирается.

Хотя жить становится несколько труднее, и люди на это часто жалуются. Чаще
всего - на то, что потихоньку повышается стоимость "коммуналки". Это началось
прошлой осенью, и в ближайшие недели ждут очередных неприятных новостей.
Повышение не фатальное, но на фоне того, что тарифы оставались практически
неизменными с 2014 года, оно все равно выглядит удручающим симптомом. Для
дончан, совершенно не избалованных приятными новостями, это особенно
чувствительно.

Доминирующим настроением людей остается ожидание. Ожидание перемен к лучшему,
которые, правда, непонятно по какой причине случатся и из какого источника
проистекут.

Поскольку надежды слабо подпитываются хоть кем-то, наиболее активная часть
населения мечтает найти какую-то работу в России - и предпринимает шаги по ее
поиску. Больше стало людей, которые смогли найти себе какой-то вахтовый вариант
с выездом на несколько месяцев - потом возвращаются в Донецк и живут на
заработанное до следующей смены. Теперь, когда стало возможным получить
российский паспорт, такие варианты вполне реальны.

С получением российских паспортов особых проблем нет. Очереди, конечно,
существуют, но они продвигаются быстро и в течение пары месяцев вопрос может
решиться. А вот с паспортами т.н. "ДНР" сложнее, их получения приходится
ожидать существенно дольше. Говорят, до полутора лет. Эта тема постепенно
подогревается, все более обсуждаемой становится перспектива полного отказа от
украинских паспортов.

То есть, люди могут оставлять их себе, если хотят, но ни один правовой вопрос
без паспорта т.н. "ДНР" решить будет уже невозможно. Ни квартиру-машину
купить-продать, ни на работу устроиться, ни пенсию оформить. Ни даже на
кладбище место получить. Пока такие ожидания ничем официально не подкреплены,
но многие уверены, что в следующем году "это" обязательно случится.

Оценить уровень трат рядового жителя т.н. "ДНР" непросто. Цены в магазинах
остаются немаленькими, хотя опять-таки, по сравнению с украинскими, далеко не
на все товары они существенно выше. А кое на что даже ниже. А кое в каких
сферах дешевизна просто поражает. Например, такси. За 50 рублей (курс в "ДНР" к
гривне 2,5) легко довезут из центра на Калиновку (2-3 километра). Вечером нас
до железнодорожного вокзала (7 километров) доставили за 120 рублей. Могло
получиться и дешевле, но водитель объяснил, что, поскольку вокзал не
функционирует в связи с отсутствием железнодорожного сообщения, он сделал
наценку ("Для меня это ведь мертвый угол").

Кстати, любопытный нюанс. Как раз, когда мы были в Донецке, местные СМИ
распространили информацию о том, что скоро из Донецка пойдут поезда - на
Ростов, Москву, Питер. Люди этот прогноз много обсуждали.

Вообще, все, что обещает возвращение хоть к какой-то норме, вызывает отчаянную
радость, которую словами не передать - это можно понять, только почувствовав ее
на месте.

Есть ли жизнь в Донецке? На этот вопрос можно ответить, пожив, как мы, пару
недель на проблемной окраине (в нашем случае - на Петровке). Вот туда такси вас
вечером точно не довезет. Но большинство живущих там людей о такси вообще не
думает.

Многие годами не выезжают за пределы поселка. То, что нужно, там есть. Рядом с
нашим домом, например - рынок, обслуживающий соседние кварталы. По сравнению с
довоенным форматом, он ужался раз в пять, но все группы продуктов, бытовая
химия, хозтовары, туалетная бумага и прочее - на нем представлены.

Мы увидели те же лица продавцов, которые были нам близки и понятны в 2013-м и
ранее. Они продолжают делать свое дело. Рядом с рынком - пара магазинов,
работающих до 20.00. Тут же, правда, разбитая давними обстрелами остановка, но
на ее состояние уже никто не обращает внимание. 

\subsubsection{Обратная дорога в Киев}

Но вот две недели позади. Мы решили все свои донецкие вопросы и готовы
отправляться назад.

Нас подогревают распространяющиеся повсюду слухи о том, что Россия вот-вот
экстренно закроет границу с "ДНР" из-за ухудшения эпидемиологической
обстановки. Впрочем, такие слухи в течение сентября возникали как минимум
трижды. И каждый раз - со ссылками на верные источники (знакомый таможенник,
брат из министерства иностранных дел и т.д.). Тут же в пабликах раздувалась
необыкновенная истерия - похоже, что для этого слухи и запускались.

Обратную дорогу заказывали по тому же телефону, что и рейс в Донецк - благо,
"Водафон" на территории города ловится спокойно, причем все номера этого
оператора (равно как и местного "Феникса") опознаются сетью как украинские.
Заказ принимают легко, уверяют, что гарантируют нам выезд в любой день, даже
если мы обратимся накануне желаемой даты отъезда. Правда, с ценой ситуация уже
несколько иная. Оператор сначала говорит, что, наверное, проезд будет стоить
все те же две с половиной тысячи гривен, но обещает уточнить.

Уточняет и перезванивает - оказывается, уже три тысячи. Мы соглашаемся - все
равно, дешевле мы вряд ли найдем. В итоге, водитель взял с нас три с половиной.
Почему "наросли" еще 500 гривен, он объяснить не смог.

Стартуем в 14.00 от Крытого рынка. Выезд задерживается на час - водитель никого
об этом не предупреждает, но на звонки обеспокоенных клиентов охотно отвечает и
уверяет, что все будет нормально и он вот-вот приедет. В конце концов, вся
компания опять-таки из 8 человек на белом (фирменный цвет у них, что ли?)
"Фольксвагене Транспортере" отправляется в нелегкий путь (что легким он не
окажется, как-то сразу понятно).

Затык возникает уже на первом этапе - на "ДНР-овском" ПП "Успеновка".
Подъезжая, видим внушительную очередь. Аргументов проскочить каким-то особым
образом наш водитель не имеет - и мы покорно становимся в самый хвост.

Считаем - впереди около 50 машин. Пропускают группами по 5, с интервалом минут
в 15-20. Очередь движется медленно, между машинами снуют местные женщины с
кофе, семечками, пирожками и выражением покорной усталости на лице. За 3 часа,
которые проводим в очереди, мы успеваем сродниться и с ними, и друг с другом.
Успеновка делает из нашего экипажа крепко спаянную команду, готовую
противостоять ударам судьбы (которые, мы не сомневаемся, нам предстоят).

Самый яркий пассажир нашего "Фольксвагена" - дедушка лет 75, не меньше. Он
очень плохо слышит, чуть лучше соображает и чуть хуже двигается. Его нам
всунули какие-то родственники в Торезе. В Полтаве его должна принять дочка. У
дедушки - огромная клетчатая сумка, набитая вещами. Нас просят за ним
проследить, помочь во всех процедурах, и вообще... Мы киваем - конечно,
поможем, что ж мы, нехристи какие! Забегая вперед, скажем, что дедушка
благополучно пересек все границы и воссоединился с дочкой в Полтаве. Помогали
ему всем экипажем.

Наша очередь подходит, когда небо уже начинает серьезно мрачнеть - где-то около
19.00.

Из "ДНР" нас выпускают без проблем - мы вздыхаем с облегчением: значит, граница
все-таки не закрыта (опасения по этому поводу в салоне витали до самого конца,
вопреки всему происходившему на наших глазах). Российская сторона встречает нас
без особой ласки, но со всем вниманием: стандартные процедуры, опять досмотр
вещей через рентген. Все это мы уже видели. А потом наш водитель исчезает на
час. Мы сидим, не зная, куда он делся и что вообще происходит. На звонки он не
отвечает.

Над нами - тревожное беззвездное небо и почти пустая автомобильная площадка
российского ПП "Матвеев Курган".

Наконец, водитель появляется, шепотом посылая проклятия всему миру.
Оказывается, последний таможенник, который был должен был шлепнуть ему
окончательную печать, придрался к какому-то из его водительских документов, и
битый час мурыжил, пока, наконец, не дал добро на проезд.

"Сто раз по этим документам без проблем проходил - и на тебе. Вот уроды!" -
тихо негодует наш рулевой (громко негодовать опасается, пока мы еще на
территории ПП). Такие ситуации типичны для этого бизнеса: каждый раз может
случиться что угодно, и полная откатанность процедуры никак не гарантирует
плавности проезда.

Проведя ночь в российском пространстве, рано утром проскакиваем Белгород и
часов в 6 попадаем на границу, где пересаживаемся в микроавтобус с украинскими
номерами (белого цвета, чтоб вы не сомневались). Россияне пропускают нас, не
особо придираясь, хотя и просят всех открыть сумки - и внимательно смотрят на
содержимое, как будто ожидали увидеть там гранатомет (но вещи при этом не
перебирают).

С украинскими таможенниками получается сложнее. Нам устраивают тотальный
досмотр всего, и в результате один из нашей компании - мужчина под 60 с
неопределенной восточной внешностью, то ли осетин, то ли чеченец - привлекает
внимание проверяющих. Им кажется, что в его плечевой сумке - какой-то намек на
наркотики. С ним разбираются минут 20, в конце даже подводят к нему собаку, но
та не проявляет к нашему товарищу никакого интереса. В конце концов, на нас
машут рукой - и, преодолев последний пограничный пост, мы въезжаем на
территорию Украины.

Кстати, никто не потребовал от нас (как нам предрекали знающие люди) установить
на телефоны приложение "Дiя", чтобы с ее помощью контролировать нашу
самоизоляцию. Ни слова ни о какой самоизоляции сказано вообще не было. Кажется,
от этой затеи украинские власти сами немножечко устали…

Собственно, дальше неинтересно. Дальше была единственная задача - просто
доехать до Киева. Что мы и сделали, хотя опять-таки не без проблем (в машине
что-то случилось с переключением скоростей, и наш молодой водитель, советуясь
по телефону с более могущественным экспертом, несколько раз что-то там
налаживал). Но, в конце концов, мы оказались у киевского вокзала, полностью
закольцевав наш путь. Большие часы, господствующие над площадью, показывали
13.50. На сей раз, мы уложились в сутки.

Так окончилось наше киевско-донецкое путешествие, которое показало нам (как
будто мы этого не знали раньше), что не стоит верить никому. Ни прессе, ни
властям, ни знающим людям, ни очевидцам. Все равно на практике все выходит
иначе, а зависит ход событий зачастую от таких неуловимых мелочей, что и
описать невозможно. Все, что мы пережили в дороге туда и обратно, оказалось
очень мало похоже на то, что можно было себе представить до начала поездки,
опираясь на так называемые "открытые источники". И к счастью. и к несчастью.


