% vim: keymap=russian-jcukenwin
%%beginhead 
 
%%file 04_09_2023.stz.news.ua.mrpl.0629.1.pro_kozacke_korinnja_mariupolja_2
%%parent 04_09_2023
 
%%url https://www.0629.com.ua/news/3654781/pro-kozacke-korinna-mariupola-castina-2-novi-sprobi-privernuti-uvagu-do-dati-zasnuvanna-mariupola-ta-dii-komandi-bojcenka-na-comu-napramku
 
%%author_id korobka_julia.mariupol,korobka_vadim.mariupol,news.ua.mrpl.0629
%%date 
 
%%tags 
%%title Про козацьке коріння Маріуполя. Частина 2. Нові спроби привернути увагу до дати заснування Маріуполя та дії команди Бойченка на цьому напрямку
 
%%endhead 
 
\subsection{Про козацьке коріння Маріуполя. Частина 2. Нові спроби привернути увагу до дати заснування Маріуполя та дії команди Бойченка на цьому напрямку}
\label{sec:04_09_2023.stz.news.ua.mrpl.0629.1.pro_kozacke_korinnja_mariupolja_2}
 
\Purl{https://www.0629.com.ua/news/3654781/pro-kozacke-korinna-mariupola-castina-2-novi-sprobi-privernuti-uvagu-do-dati-zasnuvanna-mariupola-ta-dii-komandi-bojcenka-na-comu-napramku}
\ifcmt
 author_begin
   author_id korobka_julia.mariupol,korobka_vadim.mariupol,news.ua.mrpl.0629
 author_end
\fi

\begin{quote}
\em
Про козацьке коріння Маріуполя. Докази заснування Маріуполя саме козаками.
Друга частина статті доцентів кафедри історії та археології Маріупольського
державного університету Вадима Коробки та Юлії Коробки.

Перша частина тексту за \href{https://archive.org/details/01_09_2023.vadim_korobka.julia_korobka.mrpl.0629.pro_kozacke_korinnja_mariupolja_1}{посиланням}.
\end{quote}

Для пізнання козацької минувшини Маріупольщини, наперекір несприятливим
чинникам, накопичувались і позитивні передумови.

По-перше, непересічного значення  мав  початок наукового опрацювання
українськими істориками та видання окремими книжками унікального комплексу
джерел, утвореного в ході діяльності Коша Нової Запорозької Січі (1734 – 1775
рр.), який зберігається в Центральному державному історичному архіві Україні в
столичному Києві (ЦДІАК України).

\ii{04_09_2023.stz.news.ua.mrpl.0629.1.pro_kozacke_korinnja_mariupolja_2.pic.1}

Ці документи віддзеркалюють адміністративний та військовий устрій, основи
господарювання та церковне життя запорозького козацтва в 1734 – 1775 рр. Чимала
кількість документів стосуються козацького селища Кальміуська паланка
(Кальміус). До речі, на сьогодні світ побачили 8 томів цього унікального
зведення джерел.

\ii{04_09_2023.stz.news.ua.mrpl.0629.1.pro_kozacke_korinnja_mariupolja_2.pic.2}

\begin{leftbar}
	\begingroup
	\em
Географічний покажчик кожної з цих книг не містить жодної згадки про
козацьке селище Домаха, що дає підстави стверджувати: такого, саме
запорозького, укріплення на теренах нашого краю не існувало.
	\endgroup
\end{leftbar}

По-друге, в сучасних умовах завдяки інтернету та цифровим технологіям переважна
кількість виданих до нашого часу документів, що тим або іншим чином стосуються
Кальміуської паланки, доступні усім, хто цікавиться історією українського
козацтва на надазовських теренах.

По-третє, останнім часом з'явилось чимало наукових публікацій, що присвячені
Кальміуській паланці, на які не можна не звертати уваги при вивченні історичних
коренів нашого міста.

По-четверте, були відкриті та з'явились на просторах інтернету карти, а також
описи, створені імперськими службовцями в 60-х – 70-х рр. ХVIII cт., на яких
селище Кальміуська паланка локалізується, в основному, на території між правим
берегом гирла річки Кальміус, узбережжям Азовського моря та озером Домаха.
Паланочну територію майже з усіх боків було оточено водоймами. На підставі
вивчення описових та картографічних джерел можна дійти висновку, що жодних
фортифікаційних споруд вона не мала.

\ii{04_09_2023.stz.news.ua.mrpl.0629.1.pro_kozacke_korinnja_mariupolja_2.pic.3}

Громадський інтерес до історії присутності українського козацтва на теренах
Маріупольщини підштовхнув на початку 2010-х рр. керівництво Маріупольського
краєзнавчого музею до організації та проведення археологічних розкопок на
місцині, розташованій на схід від будівлі ТСО України (колишнього ДОСААФ), над
схилом річкової долини Кальміуса. Утім, збагачення місцевої козакознавчої
скарбниці не відбулося, бо паланка (козацьке селище) була в іншому місці. Але
керівники розкопок цього не знали, а сподівались, навіть були впевнені, що
досліджують культурний шар (ґрунт) \enquote{козацької фортеці Кальміус}.

% Лідія Пономаренко (1922 – 2013 рр.) – українська геодезистка, історико-картографка. Завдяки їй до наукового обігу було введено унікальні картографічні джерела, на яких позначено Кальміуську паланку Нової Запорозької Січі.
\ii{04_09_2023.stz.news.ua.mrpl.0629.1.pro_kozacke_korinnja_mariupolja_2.pic.4}

По-п'яте. На цьому тлі парадоксальною видається пошукова діяльність
маріупольця, що користується в пабліках псевдонімом LV, який не толерував й не
толерує українську державу. Утім, він цілеспрямовано накопичував та обнародував
відомості (джерела), що можуть сприяти виходу з імперської тіні козацької
минувшини нашого міста. На його сайті \enquote{Папакома} згруповано віднайдені в
попередні роки українськими дослідниками та їм самим картографічні матеріали,
які надають можливість встановити більш-менш точно місце розташування
Кальміуської паланки (селища Кальміус), та деякі інші джерела. Завдяки цим
матеріалам у подальшому можна буде вести науковий пошук на більш достовірній
джерельній базі та робити вірогідні узагальнення.

По-шосте. В нашому місті виросли нові покоління городян, що сформувалися як
особистості за української незалежності, кращі представники яких разом із
патріотами міста старшого віку виявили взірці любові до України, відданості
рідному місту в Силах Оборони, громадських та волонтерських організаціях. Їм не
байдужа історія рідного краю, не чуже прагнення зміцнення національної
української ідентичності на місцевому рівні шляхом залучення до маріупольського
історичного наративу, поки що малознаних сторінок козацької історії. А із
зростанням чисельності інтернет-користувачів до історичного краєзнавства
Маріупольщини долучилися десятки та, мабуть, сотні наших земляків. Здобутки
краєзнавства поширилися в тематичних групах Facebook, тлумачаться журналістами
та блогерами.

\ii{04_09_2023.stz.news.ua.mrpl.0629.1.pro_kozacke_korinnja_mariupolja_2.pic.5}

В умовах повномасштабної агресії РФ проти України,  зруйнування та окупації
нашого міста рашистами, у маріупольок та маріупольців, розкиданих по різних
куточках Батьківщини та світу, серце болить за нинішні дні Маріуполя. Багатьох
з них не полишають роздуми про його оновлення після визволення, хвилює, що
відроджене місто розповідатиме людям про свою історію, чи буде, з-поміж іншого,
подолано несправедливість, яка полягала в офіційному ігноруванні запорозького
поселення, що передувало Маріуполю. Вся ця сукупність чинників надає можливість
для формулювання нової оповіді (нового наративу) про історичне коріння
Маріуполя.

***

На цьому тлі мало б викликати повагу виявлене останнім часом прагнення міського
голови Маріуполя Вадима Бойченка та його команди відновити історичну
справедливість стосовно витоків (заснування) нашого міста, яке має козацьке
коріння. 27 січня 2023 р. відбулась  робоча зустріч Вадима Бойченка, його
заступника Дениса Кочубея та керівниці департаменту культурно-громадського
розвитку Маріупольської міської ради Діани Трими з в. о. директора
Науково-дослідного інституту українознавства Василем Чернецем.

\ii{04_09_2023.stz.news.ua.mrpl.0629.1.pro_kozacke_korinnja_mariupolja_2.pic.6}

На зустрічі йшлося про співробітництво  вищезгаданого інституту з
маріупольськими структурами самоврядування, які знаходяться на підконтрольній
українському уряду території. 

Все ніби гідне усіляких похвал. Проте є декілька \enquote{але}. Учасники зустрічі
заявили про прагнення встановлення історичної справедливості щодо заснування
міста Маріуполя у X – XII ст. Тут виявилось перше слизьке місце задекларованих
намірів. Спочатку складається враження, що на підставі серйозних наукових
досліджень якісь науковці розкрили зародження нашого міста на теренах
печенізької орди або половецьких кочовищ і треба обґрунтувати його генетичний
зв'язок з Маріуполем. Утім жодного такого відкриття не було. 

З тих пір минуло майже 7 місяців. Два тижні тому до нас дійшла від інсайдера
інформація, що спеціалісти Науково-дослідного інституту українознавства уклали
\enquote{Експертний висновок щодо встановлення українського коріння Маріуполя, його
історичних витоків, періоду заснування міста}. Це було друге слизьке місце, яке
виявилось у ході реалізації намірів \enquote{встановлення історичної справедливості}. У
документі містяться висновки про козацьке коріння Маріуполя.  Неправдоподібним
є визначення року заснування нашого міста. Логічний підсумок, зроблений членами
експертної комісії, не спирається на докладне вивчення історичних джерел, а
ґрунтується на гіпотезі, яку висунуто ще в 1971 р. та спростовано подальшим
вивченням документальної бази, розвитком знань про початкову історію нашого
міста.

Зазначимо, що в окупованому Маріуполі є краєзнавці, що стали на шлях
колабораціонізму з окупантами. Вони мають чимало аргументів для спростування
висновків експертної групи, коли їх буде оприлюднено. Можемо бути впевненими,
що з’являться контрекспертні висновки, море коментарів у пабліках, глузувань та
дошкульних кепкувань з  української історичної науки.

ДАЛІ БУДЕ

\textbf{ЧИТАЙТЕ ПОЧАТОК дослідження:} \href{https://www.0629.com.ua/news/3653256/pro-kozacke-korinna-mariupola-castina-1-davni-sprobi-osagnuti-cinnik-ukrainskogo-kozactva-v-istorii-mariupola}{%
Про козацьке коріння Маріуполя. Частина 1. Давні спроби осягнути чинник українського козацтва в історії Маріуполя, Вадим Коробка, Юлія Коробка, 0629.com.ua, 01.09.2023}
