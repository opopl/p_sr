% vim: keymap=russian-jcukenwin
%%beginhead 
 
%%file 05_07_2023.stz.news.ua.donbas24.1.u_kyevi_vidkryetsja_vystavka_teatr_zhyttja_demidko.txt
%%parent 05_07_2023.stz.news.ua.donbas24.1.u_kyevi_vidkryetsja_vystavka_teatr_zhyttja_demidko
 
%%url 
 
%%author_id 
%%date 
 
%%tags 
%%title 
 
%%endhead 

У Києві відкриється виставка, присвячена театральному життю Маріуполя (ФОТО)

На ній представлять унікальні світлини та документи, які розповідають про культуру міста

Жителі Києва та переселенці зможуть дізнатися більше про театральне життя
Маріуполя, побачити фотографії культурних подій, які відбувалися у місті та
переглянути документи, що зберігалися в архівах маріупольського драмтеатру. 7
липня в 13:00 у приміщенні Маріупольського державного університету відбудеться
відкриття фотовиставки «Літопис театрального життя Маріуполя».

Як розповідає координаторка проєкту, доцентка кафедри культурології, кандидатка
історичних наук Ольга Демідко, також заплановані творчі зустрічі з театральними
працівниками — акторами та режисерами Маріуполя. Крім того, відбудуться лекції
та круглі столі, присвячені історії театрального мистецтва Маріуполя.

Читайте також: Маріупольська актриса і режисерка очолила новий театр у
Німеччині (ФОТО) 

Виставка відкривається за адресою: Київ, вул. Преображенська, 6. Дата виставки
обрана невипадково — це день народження народної артистки України, почесної
громадянки Маріуполя Світлани Отченашенко.

На фотовиставці будуть представлені унікальні світлини, які зберігалися в
архівах Донецького академічного обласного драматичного театру (м.Маріуполь),
Народного театру «Азовсталь», особистому архіві Світлани Отченашенко, фондах
Маріупольського краєзнавчого музею. Також серед майбутніх експонатів —
регіональні періодичні видання Маріуполя, в яких друкували повідомлення про ті
чи інші вистави міських театрів та статті, що стосувалися діяльності
театральних закладів.

Всі світлини Ользі Демідко вдалося вивезти з блокадного Маріуполя. Крім того,
на відкритті обіцяють розвіяти міфи та стереотипи щодо розвитку театральної
справи Маріуполя і Приазов'я.

Читайте також: Унікальні факти про театральну культуру Приазов'я (ФОТО)

Фотовиставка «Літопис театрального життя Маріуполя» вже посіла перше місце в Конкурсі наукових проєктів за напрямом «Наука та культура: збереження і популяризація історико-культурної спадщини». Конкурс проводив Київський університет імені Бориса Грінченка в рамках Всеукраїнської науково-практичної конференції «Дослідження молодих вчених: від ідеї до реалізації».

Нагадаємо, Луганський обласний театр представив Україну на Міжнародному фестивалі «New Wave Theatre» в Румунії. Журі відзначило особливою нагородою луганських театралів.

Ще більше новин та найактуальніша інформація про Донецьку та Луганську області в нашому телеграм-каналі Донбас24

Фото: Ольги Демідко
