% vim: keymap=russian-jcukenwin
%%beginhead 
 
%%file 12_05_2018.stz.news.ua.mrpl_city.1.lavochka
%%parent 12_05_2018
 
%%url https://mrpl.city/blogs/view/lavochka
 
%%author_id burov_sergij.mariupol,news.ua.mrpl_city
%%date 
 
%%tags 
%%title Лавочка
 
%%endhead 
 
\subsection{Лавочка}
\label{sec:12_05_2018.stz.news.ua.mrpl_city.1.lavochka}
 
\Purl{https://mrpl.city/blogs/view/lavochka}
\ifcmt
 author_begin
   author_id burov_sergij.mariupol,news.ua.mrpl_city
 author_end
\fi

\ii{12_05_2018.stz.news.ua.mrpl_city.1.lavochka.pic.1}

На Торговой улице между двумя домами, стоявшими на ее нечетной стороне ближе к
Фонтанной, в незапамятные времена кто-то сгородил лавочку. Судя по всему, для
этого особые усилия не были приложены. С фасадной стороны пристроили дверь и
широкое окно, с тыльной – возвели стену в полкирпича, сверху сделали покатую
крышу из досок и толи, в глубине помещения сложили печурку. Кто располагался и
чем занимался в лавочке до революции и Отечественной войны? Кто теперь скажет?
А вот о ее обитателях послевоенных лет можно кое-что поведать. Это были
сапожники, сменявшие друг друга время от времени по разным причинам.

Первым из запомнившихся был \textbf{Сеня}. Его так называли и стар и млад. Ему было
хорошо за тридцать, был он сухопар, рыжеват и хром. Увечье, полученное в
детстве, позволяло ему, как сейчас принято говорить, легально заниматься
индивидуальной предпринимательской деятельностью, то есть сапожничать. В
разрушенной войной стране, потерявшей миллионы людей, все трудоспособное
население, за исключением инвалидов и древних стариков обязано было работать на
заводах и фабриках, на стройках и железных дорогах, в колхозах и совхозах, в
крайнем случае, в артелях кустарей. Такова была суровая необходимость. Сеня
отличался необыкновенной аккуратностью и чистоплотностью, часто свойственной
холостякам, - а он-то никогда и не был женат. Из-за фартука у него всегда
выглядывала застегнутая на все пуговицы, вплоть до подбородка, светлая рубаха,
чисто выстиранная, накрахмаленная и тщательнейшим образом отглаженная. На
верстачке, изготовленном из табуретки, в строгом, никогда не изменяемом порядке
располагались ножи, бруски для их правки, шилья, молотки, щипцы для затяжки
кожи на колодку, в жестяных баночках из-под ваксы лежали гвоздики разных
размеров и железные подковки. Такой же идеальный порядок царил и на полках, где
рядами покоились колодки отдельно, сданная в починку обувь отдельно,
отремонтированная – отдельно. И еще. Сеня, в противоречие с бытующим мнением о
представителях его профессии, был абсолютным трезвенником.

Работы у Сени было хоть отбавляй, как, впрочем, и у всех других его коллег того
времени. Народ за время войны обносился: обувь беспрестанно ремонтировали.
Мелкий ремонт делали сами, а что посложнее - несли специалистам. Мастера своего
дела ставили подметки из резины и кожемита, а людям более-менее состоятельным -
из воловьей кожи. Те же, кому приходилось едва сводить концы с концами,
довольствовались подметками из обрывков резинотканевых конвейерных лент. Если
верх ботинок или сапог еще имел мало-мальски пристойный вид, а состояние
подошвы доходило до ручки, сапожники прибегали к операции, которая называлась
перетяжкой. Вот тогда-то и находили применение колодки. Сеня делал свою работу
исключительно добросовестно и отдавал ее заказчику строго в срок им
назначенный. Но брал за ремонт больше, чем другие его собратья по профессии.
Однако люди дальновидные предпочитали услуги Сени, поскольку знали: обратись
они к кому-нибудь другому, себе же будет дороже. Сеня покинул лавочку по
причине более чем уважительной: он женился на вдове с малолетним мальчиком, у
которой муж погиб на фронте. И новая семья переехала в другой район города. Там
он и продолжил заниматься своим привычным делом.

\textbf{Читайте также:} 

\href{https://mrpl.city/blogs/view/holodnyj-sapozhnik-ili-pro-obuv-proshlogo-veka}{%
Холодный сапожник, или Про обувь прошлого века, Сергей Буров, mrpl.city, 21.10.2017}

Следующим обитателем лавочки стал \textbf{Иван Иванович}. Он, как и Сеня, был инвалидом,
но инвалидом Отечественной войны. В первый же ее месяц он был ранен в грудь -
осколок застрял в легком. Его поместили в медсанбат. Но вскоре армия, в которой
наш земляк воевал, была окружена гитлеровцами. Он оказался в лагере для
военнопленных. Пленных было там так много, немцы не могли на первых порах
организовать надлежащую охрану. Этим-то и воспользовался Иван Иванович. Улучшив
момент, он бежал из неволи. Ночами пробирался к своим, к фронту. Но фронт
продвигался на восток быстрее, чем изможденный от голода, к тому же раненый
красноармеец. Когда он достиг Мариуполя, город уже был оккупирован врагом.
Прятался, пока не пришло освобождение. И тут рана дала о себе знать: начался
туберкулез легкого. Ему повезло. Местные врачи его вылечили, но о том, чтобы
идти на \enquote{Азовсталь}, где он работал до войны, не могло быть и речи. Пришлось
заняться сапожным ремеслом.

Иван Иванович был высоким костлявым мужчиной, на лице его можно было заметить
следы перенесенной некогда оспы, у него были впалые щеки, обострившиеся скулы и
нос, болезненная желтоватая кожа. Все это придавало ему сердито-мрачноватый
вид. Но на самом деле он был добрейшим, даже кротким человеком. На него, не
стесняясь соседей, покрикивала жена – разбитная полная кубанская казачка. Он же
только конфузливо приговаривал: \enquote{Надя, Надя, хватит, я тебя прошу}, и тут же
покидал поле боя. Иногда в лавочку к Ивану Ивановичу захаживал кто-нибудь из
друзей. Тогда он прислонял к оконному стеклу замусоленную картонку с надписью
химическим карандашом: \enquote{Приема нет} и запирал свое заведение изнутри. Заглянув
через окно, можно было увидеть двух, реже трех человек, сидящих на низеньких
скамеечках за табуреткой, превращенной в столик, не прерывая друг друга, мирно
беседующих. Импровизированный стол украшала нехитрая снедь и бутылочка
\enquote{Московской}. 

Жизнь постепенно стала налаживаться, кое-какую обувь, правда, выстояв большую
очередь, можно было купить в магазинах. У Ивана Ивановича доходы стали падать.
И он рискнул заняться изготовлением \enquote{танкеток} - босоножек, скопированных
отечественными умельцами с трофейной немецкой обувки, завезенной из Германии.
Дело было явно прибыльным: многим мариупольским модницам хотелось пощеголять в
таких босоножках, но за подпольное производство обуви из кожи, которая заведомо
была украдена на государственном предприятии, можно было схлопотать срок, и
немалый. Заниматься таким делом в лавочке, в окно которой заглядывал каждый
второй прохожий, значило нарваться на большущие неприятности. Поэтому Иван
Иванович решил перенести свое рабочее место к себе на квартиру. Лавочка
опустела...

Вскоре после смерти Сталина в лавочке у окна над верстачком появился пожилой
человек с седой короткой прической и такими же усами и окладистой бородой. Его
называли не по имени, как Сеню, и не по имени-отчеству, как Ивана Ивановича, а
просто – \textbf{Голубенко}. Подобно Сене он был опрятен и аккуратен, что видно было по
его одежде, по начищенным до блеска ботинкам, а также по идеальному состоянию
рабочего места, и подобно ему же не брал в рот спиртного. Досужие аборигены из
близлежащих дворов на Торговой улице откуда-то прознались, что Голубенко \enquote{сидел
за веру}, что он то ли евагелист, то ли баптист. И попал он под ту же волну
репрессий, которая накрыла в тридцатые годы тысячи священнослужителей нашей
страны. Ему повезло: его не расстреляли, но зато лишили свободы на долгие годы. 

Голубенко был молчалив, одинаково вежлив, без заискивания, со всеми. Говорил
вполголоса, будто опасаясь, что его кто-то подслушивает. Работу выполнял
добросовестно, и плату за нее брал весьма умеренную. Заказчики им были вполне
довольны. Время от времени в лавочку к Голубенко приходил старовер Петров –
сутулый иссохший старик с редкой бороденкой и горящими глазами от сознания
своей правоты в истинности только его веры. Он сходу, едва переступив порог,
начинал излагать церковные постулаты, перемежая их проклятиями в адрес
патриарха Никона, приведшего русскую православную церковь к расколу, а вместе с
ним и Петра I, огнем и мечом расправлявшихся со старообрядцами. Петров явно
пытался вызвать Голубенко если не на спор, то хотя бы на разговор. Но тот
продолжал делать свое дело за верстаком, отвлекаясь лишь на то, чтобы взглянуть
на водяную баню, стоявшую на печке, в которой плавились плитки рыбьего клея.
Иногда, очень редко, он бесстрастно вставлял в пламенную речь Петрова цитату из
Ветхого завета или Евангелия, каждый раз снабжая ее соответствующим номером
главы и стиха. Удачей для Петрова было, если во время его визита приносил в
лавочку башмаки для ремонта скорняк Фридман. Петров надеялся получить у старика
поддержку в односторонней дискуссии: \enquote{Скажите, а как Талмуд трактует
первородный грех?} Скорняк только моргал слезящимися глазами. В детстве,
обучаясь в хедере, он задолбил тексты на древнееврейском языке, да так, что и
по прошествии многих лет мог бы, наверное, их произнести. Но вот незадача, он
намертво забыл их смысл. В обыденной жизни, если это не касалось веры, все трое
по-стариковски дружили: справлялись о здоровье, советовали, чем лечить ту или
иную болячку, занимали очередь друг для друга у магазинов – частью быта простых
людей той эпохи.

И Сеня, и Иван Иванович, и Голубенко позволяли соседским мальчишкам
присутствовать в своих апартаментах. А те с интересом смотрели, как ссучивают
суровые нитки, как их сначала натирают варом, а затем воском, чтобы дратва была
одновременно прочной, непромокаемой и при шитье хорошо скользила. Их восхищала
сноровистость рук сапожников, стежок за стежком сшивающих кожу. Они спрашивали,
почему подметки прибиваются шпильками из березы, тут же на их глазах наколотые
острым ножом, а не железными гвоздями. Их интересовала необычная форма
сапожного молотка с круглым и плоским, как пятак, бойком. Любопытство детей
всегда удовлетворялось.

Все это было очень давно. Сеня, Иван Иванович, Голубенко и его друзья
давным-давно взяты каждый в свой рай. Мальчики, завсегдатаи лавочки сами стали
стариками, а некоторые из них уже нашли вечное упокоение. А лавочку снесли. И
только отпечаток трубы ее печурки на стене соседнего дома напоминают, что она
здесь была.
