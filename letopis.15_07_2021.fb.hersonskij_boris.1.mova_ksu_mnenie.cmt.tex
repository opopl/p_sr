% vim: keymap=russian-jcukenwin
%%beginhead 
 
%%file 15_07_2021.fb.hersonskij_boris.1.mova_ksu_mnenie.cmt
%%parent 15_07_2021.fb.hersonskij_boris.1.mova_ksu_mnenie
 
%%url 
 
%%author 
%%author_id 
%%author_url 
 
%%tags 
%%title 
 
%%endhead 
\subsubsection{Коментарі}
\label{sec:15_07_2021.fb.hersonskij_boris.1.mova_ksu_mnenie.cmt}

\begin{itemize}
%%%fbauth
%%%fbauth_name
\iusr{Ирина Гудкова}
%%%fbauth_url
%%%fbauth_place
%%%fbauth_id
%%%fbauth_front
%%%fbauth_desc
%%%fbauth_www
%%%fbauth_pic
%%%fbauth_pic portrait
%%%fbauth_pic background
%%%fbauth_pic other
%%%fbauth_tags
%%%endfbauth
 

Горько...те же мысли. Чтобы вырастить, не нужно вырывать с корнем то, что
растёт т.к эти корни могут укреплять сыпучую почву.


%%%fbauth
%%%fbauth_name
\iusr{Ірина Михацька}
%%%fbauth_url
%%%fbauth_place
%%%fbauth_id
%%%fbauth_front
%%%fbauth_desc
%%%fbauth_www
%%%fbauth_pic
%%%fbauth_pic portrait
%%%fbauth_pic background
%%%fbauth_pic other
%%%fbauth_tags
%%%endfbauth
 

тут така ситуація, що хай криве, але щоб було, бо відступити і скасувати такий
закон - це ще гірше, ніж ухвалити його недосконалим. А про конкретику лекцій,
думаю, можна умовитися з керівництвом закладу. Наприклад, якусь частину Ви
читаєте українською (якою достатньо володієте), а коли втомлюєтеся - бо це
реальна інтелектуальна праця - нехай заклад поставить Вам перекладача. Зі
студентів. Це не так складно. І для всіх корисно. Більше завжди клопотів із
термінологією, ніж із поточною мовою викладання.

\begin{itemize}
%%%fbauth
%%%fbauth_name
\iusr{Елена Малеева}
%%%fbauth_url
%%%fbauth_place
%%%fbauth_id
%%%fbauth_front
%%%fbauth_desc
%%%fbauth_www
%%%fbauth_pic
%%%fbauth_pic portrait
%%%fbauth_pic background
%%%fbauth_pic other
%%%fbauth_tags
%%%endfbauth
 
\textbf{Ірина Михацька} 

Слухати Бориса Григоровича у перекладi, прекрасно разумiючи руську мову, - хiба
це нормально? Наша країна добровільно віддала свою другу мову, рiдну для
половини населення i добре розвинену. Ще б усю культурну спадщину Київської
Русі подарували Росії. Усi спiльнi досягення та iсторичнi перемоги...I землю на
додачу, якщо цього не вистаче, бо принизливо користуватися тим, що належало
імперії.


%%%fbauth
%%%fbauth_name
\iusr{Борис Херсонский}
%%%fbauth_url
%%%fbauth_place
%%%fbauth_id
%%%fbauth_front
%%%fbauth_desc
%%%fbauth_www
%%%fbauth_pic
%%%fbauth_pic portrait
%%%fbauth_pic background
%%%fbauth_pic other
%%%fbauth_tags
%%%endfbauth
 
\textbf{Ірина Михацька} 

Для чого зараз мені знижати рівень свого викладання? Для мене це неадеватне
приниження. Всі цитати, майже усі вірші, які я використовую написані або
перекладені російською. Не буду я перетворятися на мавпу, яка вміє розмовляти.
На побутовому рівні - ок. З доброзичливими друзями - будь ласка. В студентській
аудіторії - поки що (тобто до кінця життя) - ні. Я ж декілька разів вже читав
лекції державною. Й одержав негативну критику з боку мовознавців. Й навіть
післі того, як виголосив лекцию свободи у Львові були негативні коментарі в
сенсі що моя мова недосконала. Згоден. Але ж не у всіх україномовних вона
досконала. Вірші - ОК, їх можливо відредагувати, лекції - ні, в мене це -
імпровізація.

%%%fbauth
%%%fbauth_name
\iusr{Борис Херсонский}
%%%fbauth_url
%%%fbauth_place
%%%fbauth_id
%%%fbauth_front
%%%fbauth_desc
%%%fbauth_www
%%%fbauth_pic
%%%fbauth_pic portrait
%%%fbauth_pic background
%%%fbauth_pic other
%%%fbauth_tags
%%%endfbauth
 

Так що для мене персонально - це заборона на професію. Я вже проходив через це
у радянські часи з інших причин.

% -------------------------------------
\ii{fbauth.snigur_sergej.odessa.ukraina}
% -------------------------------------
 
\textbf{Борис Херсонский} 

>> Й навіть післі того, як виголосив лекцию свободи у Львові були негативні
коментарі в сенсі що моя мова недосконала.

Не варто реагувати на дурнів. Особливо на мовних дурнів. Собі дорожче виходить.


%%%fbauth
%%%fbauth_name
\iusr{Сергій Снігур}
%%%fbauth_url
%%%fbauth_place
%%%fbauth_id
%%%fbauth_front
%%%fbauth_desc
%%%fbauth_www
%%%fbauth_pic
%%%fbauth_pic portrait
%%%fbauth_pic background
%%%fbauth_pic other
%%%fbauth_tags
%%%endfbauth
 
\textbf{Борис Херсонский} 

>> Всі цитати, майже усі вірші, які я використовую написані або перекладені
російською.

Але ж ви чудово знаєте, що сучасні студенти просто не чули про більшість
російських поетів яких ви цитуєте.

(Ви писали про це, ще коли в Мєчнікова викладали.)

Дехто через брак загальної освіти, а більшість тому, що замість, Сєрєбряного
вєка (російської літератури) читали літературу українську того ж історичного
періоду, яка за совєтів була для україномовних недоступна, мала статус
"буржуазно-націоналістичної".

І саме через цей освітній розрив ваші (не сумніваюсь) блискучі імпровізації не
мають відповідного відгуку в аудиторії (за вашими ж багаторічними
спостереженнями).

%%%fbauth
%%%fbauth_name
\iusr{Борис Херсонский}
%%%fbauth_url
%%%fbauth_place
%%%fbauth_id
%%%fbauth_front
%%%fbauth_desc
%%%fbauth_www
%%%fbauth_pic
%%%fbauth_pic portrait
%%%fbauth_pic background
%%%fbauth_pic other
%%%fbauth_tags
%%%endfbauth
 
\textbf{Сергій Снігур} Більшість? Це ви жартуєте? Мені не смішно...

%%%fbauth
%%%fbauth_name
\iusr{Ірина Михацька}
%%%fbauth_url
%%%fbauth_place
%%%fbauth_id
%%%fbauth_front
%%%fbauth_desc
%%%fbauth_www
%%%fbauth_pic
%%%fbauth_pic portrait
%%%fbauth_pic background
%%%fbauth_pic other
%%%fbauth_tags
%%%endfbauth
 
\textbf{Борис Херсонский} 

зауваження зі Львова - можна ігнорувати. Бо лвіуска украінска то не та
українська, на яку варто орієнтуватися. А щодо самої ситуації, то ображеними
почуються багато хто - а що робити, якщо закон про свою мову у своїй країні -
це проста очевидна потреба? Суто логічно? Вас би допустили до викладання в
Польщі без знання польської? А як Ви би вчили польську, то розмовляли би
ідеально? (я себе згадую, як мене вчора вивернуло від свого акценту, бгг) А
тепер наступне запитання: коли є потрібний матеріал і викладач, то кого сильно
парить акцент? У мене практика слухання проповідей і конференцій - з 1993 року
конкретно. Що говорили і як говорили наші батюшки, які ледве щось вивчили -
поляки, німці, словаки, - це була люта мішанина мов. Але вони говорили про
важливе. І ми їх слухали. І розуміли. Ніхто їх не відсторонював від
проповідування (а хто би тоді це робив, муахаха), мовні помилки вони робили
такі, що тепер це анекдоти "впало з амвону", - але люди робили важливу роботу,
а решта їм була по барабану, в сумі.

Закон логічний і потрібний. Перебудовуватися під нього неприємно і тяжко. Коли
мій тато перейшов з військового училища в університет, йому довелося
перевчатися самому, бо треба було змінити мову викладання, а в училищі була
російська. Він читав свої лекції вдома мамі, практикувався. Але він це зробив.
Бо коли треба, то треба, куди дінешся. Одна моя знайома поїхала в Кармель у
Францію - вивчила французьку, щоб туди поїхати, друга пішла в монастир на
Лондонщину - і вивчила англійську, щоб туди їхати. Чому Україна має бути гірша?

%%%fbauth
%%%fbauth_name
\iusr{Сергій Снігур}
%%%fbauth_url
%%%fbauth_place
%%%fbauth_id
%%%fbauth_front
%%%fbauth_desc
%%%fbauth_www
%%%fbauth_pic
%%%fbauth_pic portrait
%%%fbauth_pic background
%%%fbauth_pic other
%%%fbauth_tags
%%%endfbauth
 

Борис Херсонский >> просто не чули про більшість російських поетів яких ви цитуєте.

Не жартую.

В українській школі останні 20-25 років нарешті вивчається весь той огром
української літератури, який від українців приховувався, історія викладається
(як і має бути) україноцентрично.

Російськомовні (так само, як і решта іншомовних) надбання вивчаються
факультативно. Ваші вірші і лекції (з того, що я чув) насичені різноманітними
мовно-змістовними асоціаціями з культурою, яка більшості випускників невідома,
а за останні сім років ще й стала символом ворога.

Хіба не так.

Тож мовний закон, який мали би ухвалити одразу після проголошення незалежності
в 1991-м, це не атака на російськомовних (як це трактують і в Кремлі), де б
вони не працювали, це останній захист народом самого себе.

Пу обіцяє і захищає усіма доступними засобами саме російськомовних в Україні,
тож хіба ми маємо чекати, коли він таки виконає свою обіцянку.

Якось так.

%%%fbauth
%%%fbauth_name
\iusr{Ірина Михацька}
%%%fbauth_url
%%%fbauth_place
%%%fbauth_id
%%%fbauth_front
%%%fbauth_desc
%%%fbauth_www
%%%fbauth_pic
%%%fbauth_pic portrait
%%%fbauth_pic background
%%%fbauth_pic other
%%%fbauth_tags
%%%endfbauth
 
\textbf{Сергій Снігур} 

відповідно, треба нарощувати базу цитат з української літератури, яку потроху
виводять із підпілля (що збереглося, слава Богу), і ніякої "заборони на
професію" - тільки наступний етап розвитку.


%%%fbauth
%%%fbauth_name
\iusr{Ірина Михацька}
%%%fbauth_url
%%%fbauth_place
%%%fbauth_id
%%%fbauth_front
%%%fbauth_desc
%%%fbauth_www
%%%fbauth_pic
%%%fbauth_pic portrait
%%%fbauth_pic background
%%%fbauth_pic other
%%%fbauth_tags
%%%endfbauth
 
\textbf{Елена Малеева} 

почнемо з того, що РУСЬКОЇ мови ви не знаєте. І навряд її хтось знає взагалі.
Мову Русі досліджують філологи-мовознавці відповідного напряму. Решта знає
радянську російську, бо вже навіть не дореволюційно-літературну, на яку часто
посилається поетика. А ваша логіка - помилкова. Російську не "віддали", її
намагаються позбутися - а це різні речі. Спадщину Київської Русі якраз і тяжко
повернути тому, що "ета всьо рускає", без розрізнення руського і російського,
що ви зараз прекрасно продемонстрували, дякую.

%%%fbauth
%%%fbauth_name
\iusr{Елена Малеева}
%%%fbauth_url
%%%fbauth_place
%%%fbauth_id
%%%fbauth_front
%%%fbauth_desc
%%%fbauth_www
%%%fbauth_pic
%%%fbauth_pic portrait
%%%fbauth_pic background
%%%fbauth_pic other
%%%fbauth_tags
%%%endfbauth
 

Языки развиваются, меняются. Разумеется, я сейчас пишу не на языке Киевской
Руси. Не хочется использовать слово «росiйська»: уж очень неприятные ассоциации
вызывает как «мова ворога» с 2014-го. Сам по себе русский язык не представляет
никакой угрозы. Путину не нужно повода для нападения на Украину, как показала
его (тех, кто для него пишет) недавняя статья: «один народ», а украинский язык
- «региональный диалект». Если бы переход на украинский всех граждан спас бы
нас от оккупации, я бы говорила в публичном пространстве и писала только на
единственном государственном языке. Но это не так.


%%%fbauth
%%%fbauth_name
\iusr{Ірина Михацька}
%%%fbauth_url
%%%fbauth_place
%%%fbauth_id
%%%fbauth_front
%%%fbauth_desc
%%%fbauth_www
%%%fbauth_pic
%%%fbauth_pic portrait
%%%fbauth_pic background
%%%fbauth_pic other
%%%fbauth_tags
%%%endfbauth
 
\textbf{Елена Малеева} 

если бы вы (и остальные 73\%) действительно говорили на языке этой страны как
единственном, было бы куда как меньше оснований у путина называть этот язык
региональным диалектом(((

%%%fbauth
%%%fbauth_name
\iusr{Остап Українець}
%%%fbauth_url
%%%fbauth_place
%%%fbauth_id
%%%fbauth_front
%%%fbauth_desc
%%%fbauth_www
%%%fbauth_pic
%%%fbauth_pic portrait
%%%fbauth_pic background
%%%fbauth_pic other
%%%fbauth_tags
%%%endfbauth
 

\textbf{Борис Херсонский} 

це доволі цікава позиція. Скажімо, якби я був вашим студентом - я безумовно
домагався б вашого звільнення, просто тому що ця позиція дискримінує будь-якого
студента, першою мовою котрого є не російська.

Але цікава тому що зазвичай трапляється якраз дзеркальна ситуація, коли
приватне лишається приватним.

Попри це, в мене були російськомовні викладачі в університеті. Хтось вивчив
українську краще, хтось гірше. Хтось, як і ви, знав цитати зі своїх
студентських років саме російською.

Попри це, всі все розуміли. В тому сенсі, що ніхто не шпетив викладачів за їхні
помилки, але й викладачі не змушували студентів слухати пари іноземною.


%%%fbauth
%%%fbauth_name
\iusr{Olena Guda}
%%%fbauth_url
%%%fbauth_place
%%%fbauth_id
%%%fbauth_front
%%%fbauth_desc
%%%fbauth_www
%%%fbauth_pic
%%%fbauth_pic portrait
%%%fbauth_pic background
%%%fbauth_pic other
%%%fbauth_tags
%%%endfbauth
 
\textbf{Сергій Снігур} а их много. тем больше, чем человек виднее (

%%%fbauth
%%%fbauth_name
\iusr{Елена Малеева}
%%%fbauth_url
%%%fbauth_place
%%%fbauth_id
%%%fbauth_front
%%%fbauth_desc
%%%fbauth_www
%%%fbauth_pic
%%%fbauth_pic portrait
%%%fbauth_pic background
%%%fbauth_pic other
%%%fbauth_tags
%%%endfbauth
 
\textbf{Ірина Михацька} 

Да ну! Неужели даже сейчас не очевидно, что Путину не нужны никакие основания и
оправдания? Вот он уже написал статью на «региональном диалекте»... Может и
местных чиновников обязать общаться с жителями Малороссийской губернии РФ на
«сельском говорке». 

В поле международных отношений у него есть важнейший аргумент - ядерное оружие
и один вспомогательный - энергоносители. 

Никого в мире не волнуют наши гуманитарные проблемы, а Путина и подавно. Нельзя
гражданам Украины авторитарными способами решать спорные вопросы. 

Мир внутри страны, свобода, уважение прав всех граждан гораздо важнее амбиций
украиноязычных и русскоязычных граждан. Организовать все разумно и максимально
справедливо - не такая уж сложная задача. Но политики предпочитают зарабатывать
капиталы на этой искусственной проблеме. 

Обычные люди говорят на двух языках, давая возможность собеседнику использовать
тот, на котором ему удобнее выражать мысли. Вот это и есть гуманный,
естественный путь для нашей страны. Наш русский уже отличается от российского
варианта, это отличие будет усиливаться и без всяких административных мер. 

Но заставить русскоязычных перейти на украинский - это посеять вражду и протест
на десятилетия. Казалось бы, западные украинцы должны были бы это понимать
лучше других. Да и южные сельские мигранты тоже.

Все они настрадались, можно только посочувствовать, но перетягивание одеяла на
себя, месть своим же согражданам (иногда под благовидным предлогом) - не выход.

Уверена в том, что если бы не наши продажные депутаты и президенты, даже сейчас
проблему можно было бы решить цивилизованным путём, западным по духу.

%%%fbauth
%%%fbauth_name
\iusr{Ірина Михацька}
%%%fbauth_url
%%%fbauth_place
%%%fbauth_id
%%%fbauth_front
%%%fbauth_desc
%%%fbauth_www
%%%fbauth_pic
%%%fbauth_pic portrait
%%%fbauth_pic background
%%%fbauth_pic other
%%%fbauth_tags
%%%endfbauth
 
\textbf{Елена Малеева} 

перепрошую, я не помиляюся - путєн \_написав статтю\_ цим "діалектом", себто
українською? Путін? Українською? Статтю?

Вибачте, розмову на цьому закінчую, бо з таким рівнем фактажу вона не має сенсу.


%%%fbauth
%%%fbauth_name
\iusr{Елена Малеева}
%%%fbauth_url
%%%fbauth_place
%%%fbauth_id
%%%fbauth_front
%%%fbauth_desc
%%%fbauth_www
%%%fbauth_pic
%%%fbauth_pic portrait
%%%fbauth_pic background
%%%fbauth_pic other
%%%fbauth_tags
%%%endfbauth
 
\textbf{Ірина Михацька} 

да, ошибаетесь.

Вы не знаете об этой статье? Она наделала много шуму. Пан Огрызько (думаю, он
для вас авторитет) ее комментировал - почитайте или послушайте. Путин формально
ее автор. Понятно, что сам он ничего не пишет, но за него помощники написали
то, что он хотел сказать. Она написана в двух вариантах - на русском и
украинском. Скучная статейка. Но впервые на двух языках, такого не бывало.

Путин назвал украинский язык региональным вариантом русского, а не я. Ему все
равно, на каком языке говорят украинцы, понимаете? «Один народ», «один язык» -
вот такой новый нюанс... Это показывает, что никакие причины для оккупации Путину
не нужны.

%%%fbauth
%%%fbauth_name
\iusr{Ірина Михацька}
%%%fbauth_url
%%%fbauth_place
%%%fbauth_id
%%%fbauth_front
%%%fbauth_desc
%%%fbauth_www
%%%fbauth_pic
%%%fbauth_pic portrait
%%%fbauth_pic background
%%%fbauth_pic other
%%%fbauth_tags
%%%endfbauth
 
\textbf{Елена Малеева} 

так, причини для окупації не потрібні. Я про рівень вашого представлення
фактів. Те, що стаття є українською, аж ніяк не означає, що пу її написав. Коли
говорите про факти, формулюйте їх, будь ласка, точно, бо так і виходить -
"какаяразніца" на всіх рівнях.


%%%fbauth
%%%fbauth_name
\iusr{Елена Малеева}
%%%fbauth_url
%%%fbauth_place
%%%fbauth_id
%%%fbauth_front
%%%fbauth_desc
%%%fbauth_www
%%%fbauth_pic
%%%fbauth_pic portrait
%%%fbauth_pic background
%%%fbauth_pic other
%%%fbauth_tags
%%%endfbauth
 
\textbf{Ірина Михацька} 

На мой взгляд, я сформулировала достаточно ясно для всех, кто знал о статье и
читал тот коммент внимательно.

%%%fbauth
%%%fbauth_name
\iusr{Sofia Sherengova}
%%%fbauth_url
%%%fbauth_place
%%%fbauth_id
%%%fbauth_front
%%%fbauth_desc
%%%fbauth_www
%%%fbauth_pic
%%%fbauth_pic portrait
%%%fbauth_pic background
%%%fbauth_pic other
%%%fbauth_tags
%%%endfbauth
 
\textbf{Елена Малеева}

\ifcmt
  ig https://scontent-lga3-1.xx.fbcdn.net/v/t1.6435-9/218857044_1823979577772865_6220851431863416250_n.jpg?_nc_cat=104&ccb=1-3&_nc_sid=dbeb18&_nc_ohc=i9d8yUKFvNUAX8-aq-h&_nc_ht=scontent-lga3-1.xx&oh=3270308f3198aa01d3f14809422bba3a&oe=61222AF3
  width 0.4
\fi

\end{itemize}

%%%fbauth
%%%fbauth_name
\iusr{Александр Кабанов}
%%%fbauth_url
%%%fbauth_place
%%%fbauth_id
%%%fbauth_front
%%%fbauth_desc
%%%fbauth_www
%%%fbauth_pic
%%%fbauth_pic portrait
%%%fbauth_pic background
%%%fbauth_pic other
%%%fbauth_tags
%%%endfbauth
 
Так!

%%%fbauth
%%%fbauth_name
\iusr{Олександр Михельсон}
%%%fbauth_url
%%%fbauth_place
%%%fbauth_id
%%%fbauth_front
%%%fbauth_desc
%%%fbauth_www
%%%fbauth_pic
%%%fbauth_pic portrait
%%%fbauth_pic background
%%%fbauth_pic other
%%%fbauth_tags
%%%endfbauth
 

адміністративні методи і не покликані посилювати любов до чого б там не було \Smiley[1.0][yellow]

\begin{itemize}

%%%fbauth
%%%fbauth_name
\iusr{Александр Кабанов}
%%%fbauth_url
%%%fbauth_place
%%%fbauth_id
%%%fbauth_front
%%%fbauth_desc
%%%fbauth_www
%%%fbauth_pic
%%%fbauth_pic portrait
%%%fbauth_pic background
%%%fbauth_pic other
%%%fbauth_tags
%%%endfbauth
 
\textbf{Олександр Михельсон} так, вони покликані посилювати інше.

%%%fbauth
%%%fbauth_name
\iusr{Борис Херсонский}
%%%fbauth_url
%%%fbauth_place
%%%fbauth_id
%%%fbauth_front
%%%fbauth_desc
%%%fbauth_www
%%%fbauth_pic
%%%fbauth_pic portrait
%%%fbauth_pic background
%%%fbauth_pic other
%%%fbauth_tags
%%%endfbauth
 
\textbf{Олександр Михельсон} 

згоден. мета тут інша - видалити з освіти (переважно - вищої) російськомовних, незалежно від їх впевнень й громадської позиції.

%%%fbauth
%%%fbauth_name
\iusr{Сергій Снігур}
%%%fbauth_url
%%%fbauth_place
%%%fbauth_id
%%%fbauth_front
%%%fbauth_desc
%%%fbauth_www
%%%fbauth_pic
%%%fbauth_pic portrait
%%%fbauth_pic background
%%%fbauth_pic other
%%%fbauth_tags
%%%endfbauth
 
\textbf{Борис Херсонский} >> видалити з освіти (переважно - вищої) російськомовних
мабуть, все ж: ввести в освіту (переважно - вищу) українську мову (НЕ українськомовних).

%%%fbauth
%%%fbauth_name
\iusr{Олександр Михельсон}
%%%fbauth_url
%%%fbauth_place
%%%fbauth_id
%%%fbauth_front
%%%fbauth_desc
%%%fbauth_www
%%%fbauth_pic
%%%fbauth_pic portrait
%%%fbauth_pic background
%%%fbauth_pic other
%%%fbauth_tags
%%%endfbauth
 

\textbf{Борис Херсонский} 

є, на жаль, такий побічний ефект. але насправді жодних інших способів
"коренізації" не існує. російська теж свого часу стала панівною зовсім не тому,
що нею писав пушкін, як і англійська в ірландії (наприклад) - не тому, що нею
писав шекспір. навпаки, спершу насаджується мова - саме насаджується - а вже
потім усе інше. ну, ми із запізненням на сто років проходимо шлях тих же чехів.

%%%fbauth
%%%fbauth_name
\iusr{Natalia Lazarević}
%%%fbauth_url
%%%fbauth_place
%%%fbauth_id
%%%fbauth_front
%%%fbauth_desc
%%%fbauth_www
%%%fbauth_pic
%%%fbauth_pic portrait
%%%fbauth_pic background
%%%fbauth_pic other
%%%fbauth_tags
%%%endfbauth
 
\textbf{Борис Херсонский}, е, ну більшість російськомовних викладачів у нашій країні до вашого рівня сильно не дотягують і російською читають лекції з причин, відмінних від ваших. Тому спроба на них натиснути чи витіснити - це не так уже й погано.

%%%fbauth
%%%fbauth_name
\iusr{Олександр Бродецький}
%%%fbauth_url
%%%fbauth_place
%%%fbauth_id
%%%fbauth_front
%%%fbauth_desc
%%%fbauth_www
%%%fbauth_pic
%%%fbauth_pic portrait
%%%fbauth_pic background
%%%fbauth_pic other
%%%fbauth_tags
%%%endfbauth
 

Панове, українська мова є де-юре єдиною державною в Україні з 1989 р. Ну, хай у
роки, поки ще існував СРСР, ця норма була умовною, бо діяло ще законодавство
СССР. Але з 1991 р. в Україні діє тільки українське законодавство. Це вже 30
років! Ті, кому зараз 70, живуть уже майже половину свого життя за чинності
цієї норми, вступила вона в офіційну дію, коли такі люди були доволі молодими.
Чи можна собі уявити іншу державу, де проблемою для висококваліфікованих
працівників - громадян цієї держави є здійснення своєї роботи державною мовою?
Мені важко уявити такі держави. А що санкціонує мовний закон? У закладах
освіти, при виконанні службових обов'язків - державна мова. Проте є можливості
за певних умов використання й інших мов: окремі курси можуть читатися
офіційними мовами Євросоюзу, мовами корінних народів України та нацменшин.

Звісно, викладання будь-якої мови - саме цією мовою. Плюс безперечно, цитувати
вірші, вкрапляти у свій виклад (з навчальною, ілюстративною метою) фрагменти
текстів будь-якою іншою мовою, зрозумілою аудиторії (зокрема, й російською) -
це цілковите право викладача. 

Немає в законодавстві щодо цього перешкод. Ну, і
звісно, якщо викладач чи студент застосує росіянізм, кальку або той чи інший не
рафіновано український зворот - це теж зовсім не біда, хоча, з іншого боку, -
це мотив вдосконалювати рівень знання української, що теж важливо й корисно.


%%%fbauth
%%%fbauth_name
\iusr{Борис Херсонский}
%%%fbauth_url
%%%fbauth_place
%%%fbauth_id
%%%fbauth_front
%%%fbauth_desc
%%%fbauth_www
%%%fbauth_pic
%%%fbauth_pic portrait
%%%fbauth_pic background
%%%fbauth_pic other
%%%fbauth_tags
%%%endfbauth
 
\textbf{Сергій Снігур} 

Де ви їх візьмете, пане Сергію? Принаймні у моїй галузі. Відповідати не
обов'язкого, питання ріторічне.

%%%fbauth
%%%fbauth_name
\iusr{Олександр Михельсон}
%%%fbauth_url
%%%fbauth_place
%%%fbauth_id
%%%fbauth_front
%%%fbauth_desc
%%%fbauth_www
%%%fbauth_pic
%%%fbauth_pic portrait
%%%fbauth_pic background
%%%fbauth_pic other
%%%fbauth_tags
%%%endfbauth
 

\textbf{Борис Херсонский} 

я - не науковець, у жодній сфері, навіть у гуманітарних (пардон за "навіть"). у
мене взагалі незавершена вища освіта. я, як говорив про себе рей бредбері,
"замість університету закінчив бібліотеку". що цікаво, в тій глибоко
провінційній бібліотеці станом на кінець 80-х я саме українською відкрив для
себе не лише ремарка чи хема, а й кіра буличова та братів стругацьких \Smiley[1.0][yellow]

пригадую, класі в третьому я був шокований, коли дізнався, що буличов насправді
писав російською \Smiley[1.0][yellow] моя мрія зараз - довести якість українських перекладів до
тодішнього рівня... але я відволікаюся. я от що лише хотів сказати: попри
відсутність у мене академічної освіти, а відтак і академічного масштабу
мислення, елементарна логіка підказує мені, що "взяти десь" національні кадри
неможливо. в жодній сфері. їх можна лише ВИХОВАТИ. а це моливо лише за
використання адміністративних - у тому числі - важелів. нормандська знать
англії почала вчити англійську (ну, хоч би "вернакуляр") лише після того, як
едуард третій подав приклад, від наслідування якого за едуарда третього важко
було відмовитися, бо едуард третій у плані культурної політики був той ще
сталін, хоч і без депортацій; просто от як за сталіна вся радянська знать
шанувала "хванчкару", так за едуарда всі норманни раптом возлюбили "мову
простолюду" \Smiley[1.0][yellow] а згадаймо, скільки класики російської літератури скаржаться на
засилля французької? від пушкінського "панталоны, фрак, жилет - этих слов на
русском нет" і до стьобу толстого над колом графині жюлі \Smiley[1.0][yellow] але ж не пушкін і
не толстой зробили російську мову поширеною і, це справді так, великою. це
зробила ВЛАДА ІМПЕРІЇ. ну, прикладів тут - легіон... я дуже перепрошую за такий
розлогий коментар.

%%%fbauth
%%%fbauth_name
\iusr{Natalia Lazarević}
%%%fbauth_url
%%%fbauth_place
%%%fbauth_id
%%%fbauth_front
%%%fbauth_desc
%%%fbauth_www
%%%fbauth_pic
%%%fbauth_pic portrait
%%%fbauth_pic background
%%%fbauth_pic other
%%%fbauth_tags
%%%endfbauth
 
\textbf{Борис Херсонский}, 

якщо чекати, поки "люди самі полюблять українську і захочуть перейти" - то ніде
і ніколи. В універі не вчать українською, людина випускається і сама не може
нею говорити на фахові теми, і ніби й не будеш її звинувачувати -- її викладачі
говорили з нею російською. Це замкнене коло, якщо його не розірвати, то так
буде завжди.
\end{itemize}

%%%fbauth
%%%fbauth_name
\iusr{Елена Малеева}
%%%fbauth_url
%%%fbauth_place
%%%fbauth_id
%%%fbauth_front
%%%fbauth_desc
%%%fbauth_www
%%%fbauth_pic
%%%fbauth_pic portrait
%%%fbauth_pic background
%%%fbauth_pic other
%%%fbauth_tags
%%%endfbauth
 
Як щиро, чесно, розумно...

%%%fbauth
%%%fbauth_name
\iusr{Alex Sitnitsky}
%%%fbauth_url
%%%fbauth_place
%%%fbauth_id
%%%fbauth_front
%%%fbauth_desc
%%%fbauth_www
%%%fbauth_pic
%%%fbauth_pic portrait
%%%fbauth_pic background
%%%fbauth_pic other
%%%fbauth_tags
%%%endfbauth
 

лекции можно читать хоть на английском ,это никак не американизирует Неньку ,
но миром это не решить, ибо Язык есть Бог


%%%fbauth
%%%fbauth_name
\iusr{Олександра Швайка}
%%%fbauth_url
%%%fbauth_place
%%%fbauth_id
%%%fbauth_front
%%%fbauth_desc
%%%fbauth_www
%%%fbauth_pic
%%%fbauth_pic portrait
%%%fbauth_pic background
%%%fbauth_pic other
%%%fbauth_tags
%%%endfbauth
 
Усе на добровільній основі дає добрий результат, як у Вас!

%%%fbauth
%%%fbauth_name
\iusr{Natalia Lazarević}
%%%fbauth_url
%%%fbauth_place
%%%fbauth_id
%%%fbauth_front
%%%fbauth_desc
%%%fbauth_www
%%%fbauth_pic
%%%fbauth_pic portrait
%%%fbauth_pic background
%%%fbauth_pic other
%%%fbauth_tags
%%%endfbauth
 
Не зрозуміла, чому відмовлятися від викладання.

%%%fbauth
%%%fbauth_name
\iusr{Natalia Lazarević}
%%%fbauth_url
%%%fbauth_place
%%%fbauth_id
%%%fbauth_front
%%%fbauth_desc
%%%fbauth_www
%%%fbauth_pic
%%%fbauth_pic portrait
%%%fbauth_pic background
%%%fbauth_pic other
%%%fbauth_tags
%%%endfbauth
 

І повторю те, що вже казала: нормальна у вас українська, краще, ніж у багатьох.
Якщо не акцентувати постійно увагу на тому, що це не рідна ваша мова, і десь
там є якісь трохи помилки (а в кого їх нема?), то ніхто і не помітить.

\begin{itemize}
%%%fbauth
%%%fbauth_name
\iusr{Борис Херсонский}
%%%fbauth_url
%%%fbauth_place
%%%fbauth_id
%%%fbauth_front
%%%fbauth_desc
%%%fbauth_www
%%%fbauth_pic
%%%fbauth_pic portrait
%%%fbauth_pic background
%%%fbauth_pic other
%%%fbauth_tags
%%%endfbauth
 
\textbf{Natalia Lazarević} 

я добое знаю, що моя українська "нормальна", але для лекцій цього не вистачає.
Як ти знаєш, я продовжію вивчати мову. Але де були перекладачі підручників усі
тридцять років нашого існування? Де зібрання творів Фройда, Фромма, Юнга, де
переклади кращих зразків античної філософії? Зараз читаю енциклопедію,
перекладену та видану Фоліо. Але в мене вже нема часу на таку мовну
перекваліфікацію. До речі, а хоч чимось допомогла мені Держава на цьому шляху?

%%%fbauth
%%%fbauth_name
\iusr{Natalia Lazarević}
%%%fbauth_url
%%%fbauth_place
%%%fbauth_id
%%%fbauth_front
%%%fbauth_desc
%%%fbauth_www
%%%fbauth_pic
%%%fbauth_pic portrait
%%%fbauth_pic background
%%%fbauth_pic other
%%%fbauth_tags
%%%endfbauth
 
\textbf{Борис Херсонский}, 

перекладачам треба платити. Перекладачам такої літератури треба платити багато,
бо там ще й певна обізнаність з предметом треба. На громадських засадах можна
перекласти щось десь колись, але це крапля в морі, проблему це не вирішує.

Особисто я не платила. Не заробила ще, щоб мати змогу власним коштом
фінансувати щось таке масштабне.

А Держава... А "Держава" - це у нас хто і що конкретно, взагалі?


%%%fbauth
%%%fbauth_name
\iusr{Natalia Lazarević}
%%%fbauth_url
%%%fbauth_place
%%%fbauth_id
%%%fbauth_front
%%%fbauth_desc
%%%fbauth_www
%%%fbauth_pic
%%%fbauth_pic portrait
%%%fbauth_pic background
%%%fbauth_pic other
%%%fbauth_tags
%%%endfbauth
 

Бо моя думка така, що Держава - це щось таке, що ми, громадяни, ще досі не
добудували, щоб воно було таке як має бути, ось в процесі...


%%%fbauth
%%%fbauth_name
\iusr{Борис Херсонский}
%%%fbauth_url
%%%fbauth_place
%%%fbauth_id
%%%fbauth_front
%%%fbauth_desc
%%%fbauth_www
%%%fbauth_pic
%%%fbauth_pic portrait
%%%fbauth_pic background
%%%fbauth_pic other
%%%fbauth_tags
%%%endfbauth
 
\textbf{Natalia Lazarević} Держава не робить нічого, але вимагає від нас неможливого.

%%%fbauth
%%%fbauth_name
\iusr{Борис Херсонский}
%%%fbauth_url
%%%fbauth_place
%%%fbauth_id
%%%fbauth_front
%%%fbauth_desc
%%%fbauth_www
%%%fbauth_pic
%%%fbauth_pic portrait
%%%fbauth_pic background
%%%fbauth_pic other
%%%fbauth_tags
%%%endfbauth
 
\textbf{Natalia Lazarević} згоден

%%%fbauth
%%%fbauth_name
\iusr{Natalia Lazarević}
%%%fbauth_url
%%%fbauth_place
%%%fbauth_id
%%%fbauth_front
%%%fbauth_desc
%%%fbauth_www
%%%fbauth_pic
%%%fbauth_pic portrait
%%%fbauth_pic background
%%%fbauth_pic other
%%%fbauth_tags
%%%endfbauth
 
\textbf{Борис Херсонский}, 

я ж не сперечаюся, що ситуація кепська. Але хто нам може її виправити? Хто
захоче робити так, щоб з бюджету фінансували мовні курси, переклади книжок
тощо? Ну не дивно, що комуняка Кравчук і червоний директор Кучма цього не
робили, чи не так? Лишаємось ми з вами і решта свідомих українців (різного
етнічного походження) і наша праця, наші спроби добитися від співгромадян
якогось більш адекватного голосування на виборах, а від уже обраних депутатів,
президента тощо - якоїсь конструктивної діяльності.


%%%fbauth
%%%fbauth_name
\iusr{Віталій Нечеса}
%%%fbauth_url
%%%fbauth_place
%%%fbauth_id
%%%fbauth_front
%%%fbauth_desc
%%%fbauth_www
%%%fbauth_pic
%%%fbauth_pic portrait
%%%fbauth_pic background
%%%fbauth_pic other
%%%fbauth_tags
%%%endfbauth
 

"де переклади кращих зразків античної філософії?" - пане Борисе Борис
Херсонский, а яких саме авторів вам найбільше бракує українською? Які автори чи
конкретні твори у вашому розумінні, у розумінні наведеної цитати є тими кращими
зразками?

%%%fbauth
%%%fbauth_name
\iusr{Остап Українець}
%%%fbauth_url
%%%fbauth_place
%%%fbauth_id
%%%fbauth_front
%%%fbauth_desc
%%%fbauth_www
%%%fbauth_pic
%%%fbauth_pic portrait
%%%fbauth_pic background
%%%fbauth_pic other
%%%fbauth_tags
%%%endfbauth
 

\textbf{Борис Херсонский} ну переклади антички є в "Апріорі", схоласти – в УКУ і не лише.
Фройд і Юнґ виходили, хоч і доволі несистемно.

%%%fbauth
%%%fbauth_name
\iusr{Борис Херсонский}
%%%fbauth_url
%%%fbauth_place
%%%fbauth_id
%%%fbauth_front
%%%fbauth_desc
%%%fbauth_www
%%%fbauth_pic
%%%fbauth_pic portrait
%%%fbauth_pic background
%%%fbauth_pic other
%%%fbauth_tags
%%%endfbauth
 
\textbf{Остап Українець} та знаю я

%%%fbauth
%%%fbauth_name
\iusr{Остап Українець}
%%%fbauth_url
%%%fbauth_place
%%%fbauth_id
%%%fbauth_front
%%%fbauth_desc
%%%fbauth_www
%%%fbauth_pic
%%%fbauth_pic portrait
%%%fbauth_pic background
%%%fbauth_pic other
%%%fbauth_tags
%%%endfbauth
 

\textbf{Борис Херсонский} я про те, що вже далеко не пустка.
Хоч і не з тридцятирічною перспективою, але тут привіт російському демпінґу.

%%%fbauth
%%%fbauth_name
\iusr{Natalia Lazarević}
%%%fbauth_url
%%%fbauth_place
%%%fbauth_id
%%%fbauth_front
%%%fbauth_desc
%%%fbauth_www
%%%fbauth_pic
%%%fbauth_pic portrait
%%%fbauth_pic background
%%%fbauth_pic other
%%%fbauth_tags
%%%endfbauth
 
\textbf{Остап Українець}, ще б достатнім накладом вони виходили і їх закупали для бібліотек. А то українською дві нещасні книжки на курс, а російською штук п'ятдесят.

%%%fbauth
%%%fbauth_name
\iusr{Natalia Lazarević}
%%%fbauth_url
%%%fbauth_place
%%%fbauth_id
%%%fbauth_front
%%%fbauth_desc
%%%fbauth_www
%%%fbauth_pic
%%%fbauth_pic portrait
%%%fbauth_pic background
%%%fbauth_pic other
%%%fbauth_tags
%%%endfbauth
 

Але це теж питання розподілу бюджетних коштів. Бо не студіки, які ледве до
стипендії перебиваються, будуть це все купувати по кілька творів на семінар.

%%%fbauth
%%%fbauth_name
\iusr{Natalia Lazarević}
%%%fbauth_url
%%%fbauth_place
%%%fbauth_id
%%%fbauth_front
%%%fbauth_desc
%%%fbauth_www
%%%fbauth_pic
%%%fbauth_pic portrait
%%%fbauth_pic background
%%%fbauth_pic other
%%%fbauth_tags
%%%endfbauth
 

Але це теж питання розподілу бюджетних коштів. Бо не студіки, які ледве до
стипендії перебиваються, будуть це все купувати по кілька творів на семінар.


%%%fbauth
%%%fbauth_name
\iusr{Віталій Нечеса}
%%%fbauth_url
%%%fbauth_place
%%%fbauth_id
%%%fbauth_front
%%%fbauth_desc
%%%fbauth_www
%%%fbauth_pic
%%%fbauth_pic portrait
%%%fbauth_pic background
%%%fbauth_pic other
%%%fbauth_tags
%%%endfbauth
 

Остап Українець про це саме намагаюся панові Борису натякнути. Античку свого
часу читав у основівських перекладах, а за читанку-хрестоматію нам якось уже
вистачало Рассела і тритомного Татаркевича.


%%%fbauth
%%%fbauth_name
\iusr{Борис Херсонский}
%%%fbauth_url
%%%fbauth_place
%%%fbauth_id
%%%fbauth_front
%%%fbauth_desc
%%%fbauth_www
%%%fbauth_pic
%%%fbauth_pic portrait
%%%fbauth_pic background
%%%fbauth_pic other
%%%fbauth_tags
%%%endfbauth
 

\textbf{Віталій Нечеса} я вивчав давю філософію в СПб та Одесі, в перекладах на
російську. Платона й Аристотеля прочитав усе. Навіть, якщо б на моїй полиці
стояли ці книжки українською я б не встиг прочитати й десяти відсотків.

\end{itemize}

%%%fbauth
%%%fbauth_name
\iusr{Borys Frankovych}
%%%fbauth_url
%%%fbauth_place
%%%fbauth_id
%%%fbauth_front
%%%fbauth_desc
%%%fbauth_www
%%%fbauth_pic
%%%fbauth_pic portrait
%%%fbauth_pic background
%%%fbauth_pic other
%%%fbauth_tags
%%%endfbauth
 

Бачте, Борисе Григоровичу, Ви вперше відчули "мовну дискримінацію" в поважному
віці. А українські діти з нею живуть змалечку і зрештою змушені русифікуватися.
Бо мої ровесники закінчивши українську школу не мали змоги одержати вищу освіту
рідною мовою. І ніхто ніяких перехідних періодів для нас не робив. В перші
місяці навчального року ми говорили на математиці "додАті" і "віднЯті" і нас
російськомовні принижували, і з нас насміхалися. Тож на 30-му році незалежності
вже можна дати людям право створити своє мовне середовище. Й не ображатися на
них і не дорікати. Ви чудово оволоділи мовою, творите українську поезію. Будьте
послідовним й надалі в своєму сприйняті України і її утвердження

\begin{itemize}
%%%fbauth
%%%fbauth_name
\iusr{Борис Херсонский}
%%%fbauth_url
%%%fbauth_place
%%%fbauth_id
%%%fbauth_front
%%%fbauth_desc
%%%fbauth_www
%%%fbauth_pic
%%%fbauth_pic portrait
%%%fbauth_pic background
%%%fbauth_pic other
%%%fbauth_tags
%%%endfbauth
 
\textbf{Borys Frankovych} А як бути з кваліфікованими викладачами які мають вирішити - чи шукати кожне десяте слово на лекції, что порушувати закон? Що є цінностю освіти? Може все ж таки інформація?

%%%fbauth
%%%fbauth_name
\iusr{Артур Групп}
%%%fbauth_url
%%%fbauth_place
%%%fbauth_id
%%%fbauth_front
%%%fbauth_desc
%%%fbauth_www
%%%fbauth_pic
%%%fbauth_pic portrait
%%%fbauth_pic background
%%%fbauth_pic other
%%%fbauth_tags
%%%endfbauth
 
\textbf{Борис Херсонский} а закон запрещает читать в ВУЗе курс на любом языке кроме украинского? Или только обязывает иметь каждый курс как минимум один на украинском?

%%%fbauth
%%%fbauth_name
\iusr{Борис Херсонский}
%%%fbauth_url
%%%fbauth_place
%%%fbauth_id
%%%fbauth_front
%%%fbauth_desc
%%%fbauth_www
%%%fbauth_pic
%%%fbauth_pic portrait
%%%fbauth_pic background
%%%fbauth_pic other
%%%fbauth_tags
%%%endfbauth
 
\textbf{Артур Групп} первый вариант

%%%fbauth
%%%fbauth_name
\iusr{Артур Групп}
%%%fbauth_url
%%%fbauth_place
%%%fbauth_id
%%%fbauth_front
%%%fbauth_desc
%%%fbauth_www
%%%fbauth_pic
%%%fbauth_pic portrait
%%%fbauth_pic background
%%%fbauth_pic other
%%%fbauth_tags
%%%endfbauth
 
\textbf{Борис Херсонский} хм.... А вы уверены в этом? Тогда это какой-то очень странный закон. А если это курс испанской филологии, какой смысл его вообще слушать не на испанском?

%%%fbauth
%%%fbauth_name
\iusr{Борис Херсонский}
%%%fbauth_url
%%%fbauth_place
%%%fbauth_id
%%%fbauth_front
%%%fbauth_desc
%%%fbauth_www
%%%fbauth_pic
%%%fbauth_pic portrait
%%%fbauth_pic background
%%%fbauth_pic other
%%%fbauth_tags
%%%endfbauth
 
\textbf{Артур Групп} Закон опубликован.

%%%fbauth
%%%fbauth_name
\iusr{Maria Shvedova}
%%%fbauth_url
%%%fbauth_place
%%%fbauth_id
%%%fbauth_front
%%%fbauth_desc
%%%fbauth_www
%%%fbauth_pic
%%%fbauth_pic portrait
%%%fbauth_pic background
%%%fbauth_pic other
%%%fbauth_tags
%%%endfbauth
 
\textbf{Arthur Groupp} Филологии это не касается.
++
7. Викладання іноземної мови в закладах освіти і на курсах з вивчення іноземних мов здійснюється відповідною іноземною або державною мовою.
++

%%%fbauth
%%%fbauth_name
\iusr{Артур Групп}
%%%fbauth_url
%%%fbauth_place
%%%fbauth_id
%%%fbauth_front
%%%fbauth_desc
%%%fbauth_www
%%%fbauth_pic
%%%fbauth_pic portrait
%%%fbauth_pic background
%%%fbauth_pic other
%%%fbauth_tags
%%%endfbauth
 
\textbf{Maria Shvedova} ну уже лучше. А если это курс связанный с хайтек? Очень многое в программировании почти невозможно преподавать не по-английски

%%%fbauth
%%%fbauth_name
\iusr{Maria Shvedova}
%%%fbauth_url
%%%fbauth_place
%%%fbauth_id
%%%fbauth_front
%%%fbauth_desc
%%%fbauth_www
%%%fbauth_pic
%%%fbauth_pic portrait
%%%fbauth_pic background
%%%fbauth_pic other
%%%fbauth_tags
%%%endfbauth
 
\textbf{Arthur Groupp} 

По-английски можно, если студенты выбирают английский и в вузе есть такая возможность (если я правильно понимаю этот пункт, он не вполне ясно сформулирован).
++
5. У закладах освіти відповідно до освітньої програми одна або декілька дисциплін можуть викладатися двома чи більше мовами - державною мовою, англійською мовою, іншими офіційними мовами Європейського Союзу.
++

%%%fbauth
%%%fbauth_name
\iusr{Артур Групп}
%%%fbauth_url
%%%fbauth_place
%%%fbauth_id
%%%fbauth_front
%%%fbauth_desc
%%%fbauth_www
%%%fbauth_pic
%%%fbauth_pic portrait
%%%fbauth_pic background
%%%fbauth_pic other
%%%fbauth_tags
%%%endfbauth
 
\textbf{Maria Shvedova} 

ну если так, то я трудностей не вижу. Если английский разрешен, то в принципе
на нем можно преподавать все что угодно \Smiley[1.0][yellow]


%%%fbauth
%%%fbauth_name
\iusr{Денис Голубицкий}
%%%fbauth_url
%%%fbauth_place
%%%fbauth_id
%%%fbauth_front
%%%fbauth_desc
%%%fbauth_www
%%%fbauth_pic
%%%fbauth_pic portrait
%%%fbauth_pic background
%%%fbauth_pic other
%%%fbauth_tags
%%%endfbauth
 
\textbf{Borys Frankovych} 

Право створити мовне середовище дають таким чином, за рахунок інших мовних
середовищ? Нас дискримінували колись, тепер ми будемо теж саме робити? Дивно
...

\end{itemize}

%%%fbauth
%%%fbauth_name
\iusr{Nelly Klos}
%%%fbauth_url
%%%fbauth_place
%%%fbauth_id
%%%fbauth_front
%%%fbauth_desc
%%%fbauth_www
%%%fbauth_pic
%%%fbauth_pic portrait
%%%fbauth_pic background
%%%fbauth_pic other
%%%fbauth_tags
%%%endfbauth
 

Я поважаю Ваше рішення перейти на українську і щиро захоплююсь. І розумію, чому
складно викладати українською, якщо довгі роки перед тим робили це іншою мовою.
Але якщо дивитись ширше. Коли мова йшла про обслуговування українською, все ж
було ок? Тобто працівники магазинів та кафе, до прикладу, мають перейти на
українську, а працівникам освіти це важко? Чи коли ми приходимо до жеку,
міськради чи суду - має бути державна мова, а в університеті - ні..?

\begin{itemize}
%%%fbauth
%%%fbauth_name
\iusr{Борис Херсонский}
%%%fbauth_url
%%%fbauth_place
%%%fbauth_id
%%%fbauth_front
%%%fbauth_desc
%%%fbauth_www
%%%fbauth_pic
%%%fbauth_pic portrait
%%%fbauth_pic background
%%%fbauth_pic other
%%%fbauth_tags
%%%endfbauth
 
\textbf{Nelly Klos} Словник, якого потребує читання лекції відрізняється від словника офіціанта.

%%%fbauth
%%%fbauth_name
\iusr{Nelly Klos}
%%%fbauth_url
%%%fbauth_place
%%%fbauth_id
%%%fbauth_front
%%%fbauth_desc
%%%fbauth_www
%%%fbauth_pic
%%%fbauth_pic portrait
%%%fbauth_pic background
%%%fbauth_pic other
%%%fbauth_tags
%%%endfbauth
 
\textbf{Борис Херсонский} 

я розумію. Але й професійна підготовка у них теж відрізняється.

Може й не в тему, але з власного досвіду. Останні 2 роки я займаюсь дизайном
серветок гачком. І я була здивована тим, що досі немає термінології
українською, її фактично доводиться створювати самостійно. При тому, що це
досить популярне хобі, ми за 30 років не маємо навіть назв для базових
елементів. Це звісно бентега, особливо коли ти плануєш колись таки видати свою
книгу візерунків. Але хтось же має почати.


%%%fbauth
%%%fbauth_name
\iusr{Борис Херсонский}
%%%fbauth_url
%%%fbauth_place
%%%fbauth_id
%%%fbauth_front
%%%fbauth_desc
%%%fbauth_www
%%%fbauth_pic
%%%fbauth_pic portrait
%%%fbauth_pic background
%%%fbauth_pic other
%%%fbauth_tags
%%%endfbauth
 
\textbf{Nelly Klos} саме теж з психологічною термінологією - навіть гірше, для одного терміна є безліч варіантів.
\end{itemize}

%%%fbauth
%%%fbauth_name
\iusr{Yegor Yukramov}
%%%fbauth_url
%%%fbauth_place
%%%fbauth_id
%%%fbauth_front
%%%fbauth_desc
%%%fbauth_www
%%%fbauth_pic
%%%fbauth_pic portrait
%%%fbauth_pic background
%%%fbauth_pic other
%%%fbauth_tags
%%%endfbauth
 

Вважаю, що будь-які докори людині, що добровільно спілкується українською
мовою, щодо якихось там похибок у мовленні чи чогось подібного — то є свідома
чи несвідома робота на ворога.

У мові нема і не може бути досконалості, вона ніколи не «застигає у мармурі».
Але, як казала колись моя вчителька біології, розвивається тільки то, що
упражняється. \Smiley[1.0][yellow]

Кваліфікована досвідчена людина, котра є природним білінгвом (чи навіть
більше), має повне право користуватися у своїй праці набутком усіх своїх мов.
Це й студентів, напевно, підштовхне до праці над собою.

\begin{itemize}
%%%fbauth
%%%fbauth_name
\iusr{Борис Херсонский}
%%%fbauth_url
%%%fbauth_place
%%%fbauth_id
%%%fbauth_front
%%%fbauth_desc
%%%fbauth_www
%%%fbauth_pic
%%%fbauth_pic portrait
%%%fbauth_pic background
%%%fbauth_pic other
%%%fbauth_tags
%%%endfbauth
 
\textbf{Yegor Yukramov} дуже образливо. А коли тобі пишуть, що ти "типовий російськомовний ж"док, гібрідний патріот", то це толерувати майже неможливо.

%%%fbauth
%%%fbauth_name
\iusr{Natalia Lazarević}
%%%fbauth_url
%%%fbauth_place
%%%fbauth_id
%%%fbauth_front
%%%fbauth_desc
%%%fbauth_www
%%%fbauth_pic
%%%fbauth_pic portrait
%%%fbauth_pic background
%%%fbauth_pic other
%%%fbauth_tags
%%%endfbauth
 
\textbf{Борис Херсонский}, 

так не треба це толерувати. Треба посилати всіх з такими претензіями нахуй.
Можете кликати мене - я на правах україномовної людини з дипломом філолога
пошлю їх за цією адресою, мені не тяжко.


%%%fbauth
%%%fbauth_name
\iusr{Yegor Yukramov}
%%%fbauth_url
%%%fbauth_place
%%%fbauth_id
%%%fbauth_front
%%%fbauth_desc
%%%fbauth_www
%%%fbauth_pic
%%%fbauth_pic portrait
%%%fbauth_pic background
%%%fbauth_pic other
%%%fbauth_tags
%%%endfbauth
 
\textbf{Борис Херсонский}, 

ну що тут зробиш... Недалекими бувають не тільки ватники, а й так звані
патріоти... Зрештою, ватники теж патріоти, тільки іншої держави. \Smiley[1.0][yellow] А
патріотизм — то не лише любов до свого народу і держави, а ще й, як відомо,
"останній прихисток мерзотника"...

Тільки що читав цікаве про окситоцин і швидкість реагування в системі «свій —
чужий». На жаль, свідоме вмикається набагато пізніше, аніж підсвідоме, та й то
не у всіх... Але про це Вам відомо значно більше, ніж мені. \Smiley[1.0][yellow]

Тому деяким людям, таким от, як Ви, лишається лише одне: перти свого воза на
самоті... Ну, майже на самоті, бо все-таки Ви не зовсім самотні у цьому світі.
Хоч і у меншості, на жаль, один проти скількох? проти ста?

%%%fbauth
%%%fbauth_name
\iusr{Борис Херсонский}
%%%fbauth_url
%%%fbauth_place
%%%fbauth_id
%%%fbauth_front
%%%fbauth_desc
%%%fbauth_www
%%%fbauth_pic
%%%fbauth_pic portrait
%%%fbauth_pic background
%%%fbauth_pic other
%%%fbauth_tags
%%%endfbauth
 
\textbf{Natalia Lazarević} я так і роблю. дуже поетично.

%%%fbauth
%%%fbauth_name
\iusr{Борис Херсонский}
%%%fbauth_url
%%%fbauth_place
%%%fbauth_id
%%%fbauth_front
%%%fbauth_desc
%%%fbauth_www
%%%fbauth_pic
%%%fbauth_pic portrait
%%%fbauth_pic background
%%%fbauth_pic other
%%%fbauth_tags
%%%endfbauth
 
\textbf{Yegor Yukramov} щиро дякую за розуміння

\end{itemize}

%%%fbauth
%%%fbauth_name
\iusr{Мусій Чуприна}
%%%fbauth_url
%%%fbauth_place
%%%fbauth_id
%%%fbauth_front
%%%fbauth_desc
%%%fbauth_www
%%%fbauth_pic
%%%fbauth_pic portrait
%%%fbauth_pic background
%%%fbauth_pic other
%%%fbauth_tags
%%%endfbauth
 

тридцятирічна лагідна українізація провалилася.

Це свідчення того, що треба вживати інших, більш дієвих заходів.

Коли сьогодні молода людина, яка тільки закінчила школу, заявляє про те що не
може і не хоче спілкуватися українською - то про яку дискримінацію ми говоримо,
якщо ми не зробили елементарного. Ми не змогли в школі(!) навчити людину самому
простому - володінню і знанню державної мови. І головне, ми не навчали людину
відноситись з повагою до країни, до мови і до людей(!) які її оточують.

\begin{itemize}
%%%fbauth
%%%fbauth_name
\iusr{Елена Малеева}
%%%fbauth_url
%%%fbauth_place
%%%fbauth_id
%%%fbauth_front
%%%fbauth_desc
%%%fbauth_www
%%%fbauth_pic
%%%fbauth_pic portrait
%%%fbauth_pic background
%%%fbauth_pic other
%%%fbauth_tags
%%%endfbauth
 
\textbf{Мусій Чуприна} 

И более действенные меры провалятся, только последствия станут уже
катастрофическими для страны. Могла бы написать этот коммент на украинском
языке, но не хочу. Любое насилие и даже призыв к его применению вызывают
естественное сопротивление.

Русскоязычные такие же граждане Украины. С уважением нужно относиться ко всем,
иначе страна перестанет существовать. Эта перспектива беспокоит и печалит.


%%%fbauth
%%%fbauth_name
\iusr{Мусій Чуприна}
%%%fbauth_url
%%%fbauth_place
%%%fbauth_id
%%%fbauth_front
%%%fbauth_desc
%%%fbauth_www
%%%fbauth_pic
%%%fbauth_pic portrait
%%%fbauth_pic background
%%%fbauth_pic other
%%%fbauth_tags
%%%endfbauth
 
\textbf{Елена Малеева} 

Можете, но не хотите? Это выбор. Типичный ватный выбор. Ничего нового, ничего
удивительного. Вы подтвердили только то, о чем я и написал.

И кстати, я не писал о насилии. О насилии пишите вы, но каждый мыслит в меру
своей распущенности или испорченности. Надеюсь, что назло кондукторам и во
избежание "насилия", вы также принципиально ходите пешком и не пользуетесь
общественным транспортом.

%%%fbauth
%%%fbauth_name
\iusr{Елена Малеева}
%%%fbauth_url
%%%fbauth_place
%%%fbauth_id
%%%fbauth_front
%%%fbauth_desc
%%%fbauth_www
%%%fbauth_pic
%%%fbauth_pic portrait
%%%fbauth_pic background
%%%fbauth_pic other
%%%fbauth_tags
%%%endfbauth
 
\textbf{Мусій Чуприна} 

«Типичный ватный выбор» - это как раз о вас: «Не хочешь - научим. Не умеешь -
заставим. Права человека, его жизнь - пустяк, главное - интересы страны
(власти).» Знакомо.

Вы же против мягкой и постепенной украинизации, «нужны другие меры» - разве не
так написали? Насилие - понятие широкое, оно не обязательно означает физические
действия. Вы именно сторонник насилия, судя по комментарию. Это не западный
цивилизационный путь, а вышиватный. «Другие меры» применяет Лукашенко и Путин.

При чем тут «назло кондуктору», если мне удобно и привычно писать на русском,
это мой родной язык? Живу там, где мои прадеды жили и где сама родилась 57 лет
назад. Никакого отношения к России не имею. Хватит уже совка! Не выношу
принуждения.

%%%fbauth
%%%fbauth_name
\iusr{Мусій Чуприна}
%%%fbauth_url
%%%fbauth_place
%%%fbauth_id
%%%fbauth_front
%%%fbauth_desc
%%%fbauth_www
%%%fbauth_pic
%%%fbauth_pic portrait
%%%fbauth_pic background
%%%fbauth_pic other
%%%fbauth_tags
%%%endfbauth
 
\textbf{Елена Малеева} 

блин, вы еще и ПСИХОЛОГ?

я в шоке - в двух предложениях, о школьном воспитании в детях чувства уважения
к другим, найти насилие и раздуть его до космических размеров - это надо уметь!

% -------------------------------------
\ii{fbauth.malejeva_elena.odessa.ukraina.psiholog}
% -------------------------------------
 
\textbf{Мусій Чуприна} 

так вы о детях, о воспитании уважения к согражданам и родной культуре! Ну да,
придавить хорошенько их родителей, научить их родину любить жёсткими
административными мерами - вот и вырастет поколение свободных, уважающих
других, толерантных людей...

Да, я психолог по образованию и сфере деятельности. А вы специалист по
«народному хозяйству», как видно из профиля. Ничуть не удивлена.

%%%fbauth
%%%fbauth_name
\iusr{Мусій Чуприна}
%%%fbauth_url
%%%fbauth_place
%%%fbauth_id
%%%fbauth_front
%%%fbauth_desc
%%%fbauth_www
%%%fbauth_pic
%%%fbauth_pic portrait
%%%fbauth_pic background
%%%fbauth_pic other
%%%fbauth_tags
%%%endfbauth
 
\textbf{Елена Малеева} а ну-ка, давайте протестируем вас на таком тексте:
«Он уважать себя заставил
И лучше выдумать не мог.
Его пример другим наука...»
Дайте свой анализ.

%%%fbauth
%%%fbauth_name
\iusr{Мидько Андрій}
%%%fbauth_url
%%%fbauth_place
%%%fbauth_id
%%%fbauth_front
%%%fbauth_desc
%%%fbauth_www
%%%fbauth_pic
%%%fbauth_pic portrait
%%%fbauth_pic background
%%%fbauth_pic other
%%%fbauth_tags
%%%endfbauth
 
\textbf{Елена Малеева} , 

держава, будь яка, західна чи східна, не має значення, це є інструмент примусу,
і у лінгвістичному сенсі - також. Спробуйте асимилюватися в Німеччині чи
Франції не володіючи відповідними мовами. То примусове та обов'язкове володіння
українською - симптом одужання української держави. Дивно інше: людей на 30
році незалежності такі речі дивують та "обурюють". Як діти.

% -------------------------------------
\ii{fbauth.malejeva_elena.odessa.ukraina.psiholog}
% -------------------------------------

\textbf{Мидько Андрій} 

Я не иммигрантка - нет задачи ассимилироваться в Украине. Может быть, вам стоит
подумать об адаптации к реальным условиям - к жизни в большой и неоднородной
стране (пока она не уменьшилась вдвое)? Это территория бывшей УССР, на которой
никто не выбирал родиться.

Не столько «обурюють» такие вещи (это ваше слово), сколько удивляют и огорчают.
Понятно, что будет с нашей страной, если так дело пойдёт. Неразумный закон
только ухудшил ситуацию.

Вы зрелый человек, не ребёнок - так работайте со своими давними обидами и
комплексами.

Украинский язык богат и красив - кто спорит? Украинцев незаслуженно притесняли
- да, очень плохо. Вы достойны уважения, это не подлежит сомнению. Всё,
ситуация кардинально изменилась. Развивайте свою культуру и язык. Живите
спокойно и давайте жить другим. Кстати, я наполовину украинка и владею
государственным языком в пределах необходимого для жизни. Для этого
недостаточно, но оплавляющее большинство студентов предпочитают русский. И они
тоже не мигранты. Их права и желания имеют значение.

Переходить на украинский стану только в том случае, если перееду в западную
часть страны. Но пока я у себя дома, на земле предков (прошу прощения за
пафос), не чувствую необходимости отказываться от родного языка. Никто вам
ничего не должен, если вы сами не готовы к компромиссам и уважению другой части
граждан.

Человек имеет право, живя на родине, пользоваться своим языком в полной мере.
Это должны знать люди, стремящиеся к европейским ценностям и интеграции с
западной цивилизацией. Время империй прошло. Поздно пытаться проделать с
русскоязычными то же, что некоторые из них проделывали с украиноязычными. Да и
неблагородно, рабское что-то. Неразумно и непрактично - это само собой,
последствия слишком очевидны, чтобы обсуждать их.


%%%fbauth
%%%fbauth_name
\iusr{Елена Малеева}
%%%fbauth_url
%%%fbauth_place
%%%fbauth_id
%%%fbauth_front
%%%fbauth_desc
%%%fbauth_www
%%%fbauth_pic
%%%fbauth_pic portrait
%%%fbauth_pic background
%%%fbauth_pic other
%%%fbauth_tags
%%%endfbauth
 
\textbf{Мусій Чуприна} 

Коммент с цитатами из Пушкина очень информативен. Диагност бы за него
ухватился. Но я не занимаюсь тестированием интеллекта и личностных черт. Другая
специализация.

%%%fbauth
%%%fbauth_name
\iusr{Мидько Андрій}
%%%fbauth_url
%%%fbauth_place
%%%fbauth_id
%%%fbauth_front
%%%fbauth_desc
%%%fbauth_www
%%%fbauth_pic
%%%fbauth_pic portrait
%%%fbauth_pic background
%%%fbauth_pic other
%%%fbauth_tags
%%%endfbauth
 
\textbf{Елена Малеева} , та давайте вже, залиште "половину України", усі зачекалися.

%%%fbauth
%%%fbauth_name
\iusr{Елена Малеева}
%%%fbauth_url
%%%fbauth_place
%%%fbauth_id
%%%fbauth_front
%%%fbauth_desc
%%%fbauth_www
%%%fbauth_pic
%%%fbauth_pic portrait
%%%fbauth_pic background
%%%fbauth_pic other
%%%fbauth_tags
%%%endfbauth
 
\textbf{Мидько Андрій} 

кто эти “усi” и почему русскоязычные должны соответствовать их ожиданиям?
Путин, российские имперцы заждались? Или вы, гепер-патриоты, свихнувшиеся на
мове? Не дождётесь! 😀 Мы живем в своей стране, Украине.

Вам ФСБ не платит?

%%%fbauth
%%%fbauth_name
\iusr{Мидько Андрій}
%%%fbauth_url
%%%fbauth_place
%%%fbauth_id
%%%fbauth_front
%%%fbauth_desc
%%%fbauth_www
%%%fbauth_pic
%%%fbauth_pic portrait
%%%fbauth_pic background
%%%fbauth_pic other
%%%fbauth_tags
%%%endfbauth
 
\textbf{Елена Малеева} , 

я мав на увазі Ваші погрози. Тому і пропоную Вам зробити так, щоб "осталась
половина Украины". Якщо в Вас вистачить сил, то українці Вам скажуть лише
"дякую". Україна - це територія де розмовляють українською, принаймні в шкалах,
крамницях, державних установах. "Варіанти" можливі були лише в УРСР. Схоже що
Ви все ще там і мешкаєте.


%%%fbauth
%%%fbauth_name
\iusr{Елена Малеева}
%%%fbauth_url
%%%fbauth_place
%%%fbauth_id
%%%fbauth_front
%%%fbauth_desc
%%%fbauth_www
%%%fbauth_pic
%%%fbauth_pic portrait
%%%fbauth_pic background
%%%fbauth_pic other
%%%fbauth_tags
%%%endfbauth
 
\textbf{Мидько Андрій} 

Якi погрози? Я ж пишу, що це було б трагичним наслiдком.. Для мене особисто. А
ви, «патрiоти», давно чекаєте відділення половини країни ... Цікаво. Напишіть
про це статтю, вона буде мати успіх. Добре, що в нашій країні існує свобода
слова... Тiльки це й тiше.

%%%fbauth
%%%fbauth_name
\iusr{Мидько Андрій}
%%%fbauth_url
%%%fbauth_place
%%%fbauth_id
%%%fbauth_front
%%%fbauth_desc
%%%fbauth_www
%%%fbauth_pic
%%%fbauth_pic portrait
%%%fbauth_pic background
%%%fbauth_pic other
%%%fbauth_tags
%%%endfbauth
 
\textbf{Елена Малеева} , до речі, я теж російськомовний. Але перш за все я українець.

%%%fbauth
%%%fbauth_name
\iusr{Мидько Андрій}
%%%fbauth_url
%%%fbauth_place
%%%fbauth_id
%%%fbauth_front
%%%fbauth_desc
%%%fbauth_www
%%%fbauth_pic
%%%fbauth_pic portrait
%%%fbauth_pic background
%%%fbauth_pic other
%%%fbauth_tags
%%%endfbauth
 
Елена Малеева , я нічого не чекаю крім перетворення УРСР на Україну.

%%%fbauth
%%%fbauth_name
\iusr{Елена Малеева}
%%%fbauth_url
%%%fbauth_place
%%%fbauth_id
%%%fbauth_front
%%%fbauth_desc
%%%fbauth_www
%%%fbauth_pic
%%%fbauth_pic portrait
%%%fbauth_pic background
%%%fbauth_pic other
%%%fbauth_tags
%%%endfbauth
 
\textbf{Мидько Андрій} 

Так в чем же проблема? Живем в русскоязычном городе. Уважаем украиноязычных, не
мешаем им говорить на родном языке. Все отлично понимают их в быту. Лекции
обычно читаются на том языке, на котором лектору удобнее. На украинском -
хорошо, наши слушатели немного напрягаются, но не считают себя вправе требовать
другого (могут мягко и осторожно попросить перейти на русский). 

Ещё ни один студент не потребовал от меня чтения лекций на государственном
языке, они сами не готовы, да и не очень хотят. Мы сосредоточены на смыслах,
содержании. Никому в этот момент нет дела до политики. Наши клиенты тоже
русскоязычные. Важно работать с ними на их родном языке. Украиноязычных
направляла бы к коллегам, для которых этот язык родной. Но пока таких случаев
не было. 

В моём окружении только одна русскоязычная семья с радостью встретила бы
российских зелёных человечков (они теперь живут в России). Остальные не просто
не хотят отделения от Украины, это было бы бедой для них. Русский нужно было
сразу сделать вторым государственным языком, заменив некоторые нормы
(фрикативное «г», например, сделать вариантом правильного произношения) - не
было бы такой вражды внутри страны. 

Постепенно южный вариант русского изменился бы ещё больше под влиянием
украинского - реформы закрепляли бы эти изменения. Пусть языки развиваются
естественным путём. Свои функции они выполняют и без вмешательства политиков.

Путину и всей этой шайке наплевать на то, как мы себя идентифицируем и на каких
языках говорим. Прилично выглядеть в глазах мировой общественности им не нужно.
Любят чувствовать силу и безнаказанность. Страх и подчинение их вполне бы
устроили. Остальное - фигуры речи, мы это слышали по ТВ ещё в СССР.

Ссорить людей из-за языков, предлагать недовольным отделиться - то, за что ФСБ
платит своим агентам, а вы им готовы бесплатно помогать - так выходит. Надеюсь,
непреднамеренно.

% -------------------------------------
\ii{fbauth.midjko_andrej.odessa.ukraina.psihiatria.doktor}
% -------------------------------------

\textbf{Елена Малеева} , Ви дивна людина: погрожуєте відділенням половини України, а потім звинувачуєте в цьому інших.

%%%fbauth
%%%fbauth_name
\iusr{Елена Малеева}
%%%fbauth_url
%%%fbauth_place
%%%fbauth_id
%%%fbauth_front
%%%fbauth_desc
%%%fbauth_www
%%%fbauth_pic
%%%fbauth_pic portrait
%%%fbauth_pic background
%%%fbauth_pic other
%%%fbauth_tags
%%%endfbauth
 
\textbf{Мидько Андрій} 

Это ваша неверная трактовка. Прочтите ещё раз мой комментарий, не додумывая
лишнего. Вот вы сами высказались совершенно прозрачно - разных трактовок быть
не может: «все давно ждут, отделяйтесь уже». Я вам ответила: «Не дождётесь!».


%%%fbauth
%%%fbauth_name
\iusr{Мидько Андрій}
%%%fbauth_url
%%%fbauth_place
%%%fbauth_id
%%%fbauth_front
%%%fbauth_desc
%%%fbauth_www
%%%fbauth_pic
%%%fbauth_pic portrait
%%%fbauth_pic background
%%%fbauth_pic other
%%%fbauth_tags
%%%endfbauth
 
\textbf{Елена Малеева} , 

тоді Вам доведеться звикнути до вживання української в позначених сферах. І
нема про що говорити. "Або/або", Ви доросла людина і самі це розумієте.


%%%fbauth
%%%fbauth_name
\iusr{Мидько Андрій}
%%%fbauth_url
%%%fbauth_place
%%%fbauth_id
%%%fbauth_front
%%%fbauth_desc
%%%fbauth_www
%%%fbauth_pic
%%%fbauth_pic portrait
%%%fbauth_pic background
%%%fbauth_pic other
%%%fbauth_tags
%%%endfbauth
 
\textbf{Елена Малеева} , 

ніякого "неравенства" не бачу: Ви маєте таке ж право розмовляти українською як
і усі інші. Саме тому, що це Украина. Хочете російської як державної? То берить
автомат чи кличте "головного руського". Вирішуйте вже щось:))) А ні, то в
російськомовних (не в "русских") своя держава вже є. Не бачу жодної
"несправедливості".

%%%fbauth
%%%fbauth_name
\iusr{Елена Малеева}
%%%fbauth_url
%%%fbauth_place
%%%fbauth_id
%%%fbauth_front
%%%fbauth_desc
%%%fbauth_www
%%%fbauth_pic
%%%fbauth_pic portrait
%%%fbauth_pic background
%%%fbauth_pic other
%%%fbauth_tags
%%%endfbauth
 
\textbf{Мидько Андрій} 

Вы не видите несправедливости в том, что люди, чьи предки поселились на этой
земле в девятнадцатом веке, кто никуда не переезжал, должны собрать чемоданы и
уехать в чужую им страну либо перейти на неродной язык? У вас своеобразное
чувство справедливости. Перемещать черт знает куда тех, кто не вписался в
политические планы руководства - сталинская идея. «Доведеться»... Увидим.

Именно потому, что я очень взрослая (спасибо за напоминание в каждом комменте),
доживу на своей земле как-нибудь. Заполнить документы на государственном языке
и т.п. - не проблема, нас в школе учили украинскому, это был обязательный
предмет. И лекцию могу прочитать на украинском криво-косо, если пожалует
языковой инспектор. Да и штраф могу выплатить. А если подобные вам придут к
власти и страна перестанет быть достаточно свободной, уеду без сожаления (в
одной стране дураков родилась и прожила полжизни - хватит). Не в Россию,
конечно.

Всё, мне разговор с вами наскучил. Вполне, возможно, российский
агент-провокатор или... Вариантов несколько. Травмированный несправедливостью,
настрадавшийся, мстительный сельский мигрант - наиболее вероятная версия. Если
так, сочувствую и понимаю. Но разговор наш закончен.

%%%fbauth
%%%fbauth_name
\iusr{Мидько Андрій}
%%%fbauth_url
%%%fbauth_place
%%%fbauth_id
%%%fbauth_front
%%%fbauth_desc
%%%fbauth_www
%%%fbauth_pic
%%%fbauth_pic portrait
%%%fbauth_pic background
%%%fbauth_pic other
%%%fbauth_tags
%%%endfbauth
 
\textbf{Елена Малеева} , 

Вам ніхто не заважає спілкуватися російською, так саме як ніхто не заважає
спілкуватся французською чи німецькою в Штатах, там є відповідні анклави, але
якщо комусь англійська "непереносима", то він поїде з англомовних Штатів на
історичну Батьківщину незважаючи на те, що його предки були засновниками
країни. І так, це є цілком справедливим. А взагалі, Ваші коментарі зайвий раз
підтверджують те, що російська мова - мова погроз та війни.


%%%fbauth
%%%fbauth_name
\iusr{Мидько Андрій}
%%%fbauth_url
%%%fbauth_place
%%%fbauth_id
%%%fbauth_front
%%%fbauth_desc
%%%fbauth_www
%%%fbauth_pic
%%%fbauth_pic portrait
%%%fbauth_pic background
%%%fbauth_pic other
%%%fbauth_tags
%%%endfbauth
 
\textbf{Елена Малеева} , 

зізнайтеся, що Вам неприємно опинитися частиною мовної меньшини. Так, бо
питання російської мови в Україні - це питання домінування, тут я Вас розумію і
повертаю Вам Ваші аргументи стосовно "сталінизму" та совка.

%%%fbauth
%%%fbauth_name
\iusr{Елена Малеева}
%%%fbauth_url
%%%fbauth_place
%%%fbauth_id
%%%fbauth_front
%%%fbauth_desc
%%%fbauth_www
%%%fbauth_pic
%%%fbauth_pic portrait
%%%fbauth_pic background
%%%fbauth_pic other
%%%fbauth_tags
%%%endfbauth
 
\textbf{Мидько Андрій} 

Не мешаете пользоваться родным языком тем, кто никуда не переезжал и живет на
земле предков?! ) Спасибо, но мне разрешение и согласие не нужно.

Чувствуете угрозу там, где ее нет (если действительно чувствуете, а не
подрабатываете провокациями). И с психиатром всякое может случиться, к
сожалению. Разубеждать вас бесполезно в любом случае.


%%%fbauth
%%%fbauth_name
\iusr{Мидько Андрій}
%%%fbauth_url
%%%fbauth_place
%%%fbauth_id
%%%fbauth_front
%%%fbauth_desc
%%%fbauth_www
%%%fbauth_pic
%%%fbauth_pic portrait
%%%fbauth_pic background
%%%fbauth_pic other
%%%fbauth_tags
%%%endfbauth
 
\textbf{Елена Малеева}, 

так, саме так, Вам ніхто не заважає і не треба вигадувати собі зайві проблеми

\end{itemize}

%%%fbauth
%%%fbauth_name
\iusr{Volodymyr Kulyk}
%%%fbauth_url
%%%fbauth_place
%%%fbauth_id
%%%fbauth_front
%%%fbauth_desc
%%%fbauth_www
%%%fbauth_pic
%%%fbauth_pic portrait
%%%fbauth_pic background
%%%fbauth_pic other
%%%fbauth_tags
%%%endfbauth
 

Прикро, що Ви теж виявилися нездатним підтримати запровадження української мови
саме в тій сфері, яка найбільше стосується Вас особисто. Бо легко підтримувати
зміни, які Вас не зачіпають, важче ті, які вимагають змінитися особисто Вас.

\begin{itemize}
%%%fbauth
%%%fbauth_name
\iusr{Борис Херсонский}
%%%fbauth_url
%%%fbauth_place
%%%fbauth_id
%%%fbauth_front
%%%fbauth_desc
%%%fbauth_www
%%%fbauth_pic
%%%fbauth_pic portrait
%%%fbauth_pic background
%%%fbauth_pic other
%%%fbauth_tags
%%%endfbauth
 
\textbf{Volodymyr Kulyk} 

Так, дуже прикро. Але я згоден припинити викладання тому, що визнаю свою
неспроможність викладати на тому рівні, якого потребує моя викладацька совість.
Що ще я можу віддати в жертву своєму патріотизму?


%%%fbauth
%%%fbauth_name
\iusr{Volodymyr Kulyk}
%%%fbauth_url
%%%fbauth_place
%%%fbauth_id
%%%fbauth_front
%%%fbauth_desc
%%%fbauth_www
%%%fbauth_pic
%%%fbauth_pic portrait
%%%fbauth_pic background
%%%fbauth_pic other
%%%fbauth_tags
%%%endfbauth
 
\textbf{Борис Херсонский} 

це принаймні буде чесно. Але якщо Вам доведеться через цю неспроможність піти
на пенсію, це зовсім не означає, що закон дискримінаційний. Тим паче, що у Вас
залишаються інші сфери діяльності, де Ви цілком спроможні навіть без доброї
української.

%%%fbauth
%%%fbauth_name
\iusr{Natalia Lazarević}
%%%fbauth_url
%%%fbauth_place
%%%fbauth_id
%%%fbauth_front
%%%fbauth_desc
%%%fbauth_www
%%%fbauth_pic
%%%fbauth_pic portrait
%%%fbauth_pic background
%%%fbauth_pic other
%%%fbauth_tags
%%%endfbauth
 
\textbf{Борис Херсонский}, 

я пропоную у цьому разі трошки прикрутити викладацьку совість. Бо вона
заганяється. У моєму житті було кілька чудових педагогів, які починали
викладати українською от саме на нашому класі/курсі, вони нас вчили предмету,
ми їм допомагали опановувати мову, гарно всі все вивчили, ніхто від того не
вмер.


%%%fbauth
%%%fbauth_name
\iusr{Борис Херсонский}
%%%fbauth_url
%%%fbauth_place
%%%fbauth_id
%%%fbauth_front
%%%fbauth_desc
%%%fbauth_www
%%%fbauth_pic
%%%fbauth_pic portrait
%%%fbauth_pic background
%%%fbauth_pic other
%%%fbauth_tags
%%%endfbauth
 
\textbf{Natalia Lazarević} не забувай про мій вік

%%%fbauth
%%%fbauth_name
\iusr{Natalia Lazarević}
%%%fbauth_url
%%%fbauth_place
%%%fbauth_id
%%%fbauth_front
%%%fbauth_desc
%%%fbauth_www
%%%fbauth_pic
%%%fbauth_pic portrait
%%%fbauth_pic background
%%%fbauth_pic other
%%%fbauth_tags
%%%endfbauth
 

Взагалі, це така дуже совєтська штука, оця ідея, ніби треба спочатку знати мову
досконало, а потім уже говорити. Ну не працює воно так, і не лише з
українською.

%%%fbauth
%%%fbauth_name
\iusr{Natalia Lazarević}
%%%fbauth_url
%%%fbauth_place
%%%fbauth_id
%%%fbauth_front
%%%fbauth_desc
%%%fbauth_www
%%%fbauth_pic
%%%fbauth_pic portrait
%%%fbauth_pic background
%%%fbauth_pic other
%%%fbauth_tags
%%%endfbauth
 
\textbf{Борис Херсонский}, а що вік? Я вам кажу, що \underline{вже} є цілком
пристойний рівень. Якщо десь якісь неточності - студенти доперекладають собі,
це ж виш, а не дитсадок. Ну і прогрес буде ще, якщо практикуватися. Але я так
розумію, що говорити українською для вас стрес, саме через переконання, що
помилки абсолютно неприпустимі.

%%%fbauth
%%%fbauth_name
\iusr{Борис Херсонский}
%%%fbauth_url
%%%fbauth_place
%%%fbauth_id
%%%fbauth_front
%%%fbauth_desc
%%%fbauth_www
%%%fbauth_pic
%%%fbauth_pic portrait
%%%fbauth_pic background
%%%fbauth_pic other
%%%fbauth_tags
%%%endfbauth
 
\textbf{Natalia Lazarević} завжди говорив що ти дуже, дуже розумна дівчина!

%%%fbauth
%%%fbauth_name
\iusr{Natalia Lazarević}
%%%fbauth_url
%%%fbauth_place
%%%fbauth_id
%%%fbauth_front
%%%fbauth_desc
%%%fbauth_www
%%%fbauth_pic
%%%fbauth_pic portrait
%%%fbauth_pic background
%%%fbauth_pic other
%%%fbauth_tags
%%%endfbauth
 
\textbf{Борис Херсонский}, спасибі на доброму слові. 😊

\end{itemize}

%%%fbauth
%%%fbauth_name
\iusr{Сергей Дибров}
%%%fbauth_url
%%%fbauth_place
%%%fbauth_id
%%%fbauth_front
%%%fbauth_desc
%%%fbauth_www
%%%fbauth_pic
%%%fbauth_pic portrait
%%%fbauth_pic background
%%%fbauth_pic other
%%%fbauth_tags
%%%endfbauth
 

Закон ухвалила Верховна Рада, і підписав Порошенко.

Закон може бути

\begin{itemize}
  \item — недосконалим,
  \item — невчасним,
  \item — недолугим,
  \item — таким, що не вілповілає Коеституції,
\end{itemize}

причому в будь-яких комбінаціях з цих ознак.

КСУ встановив, що цей закон не є таким, що не відповідає Конституції. Наразі це
вже є юридичним фактом.

\begin{itemize}
%%%fbauth
%%%fbauth_name
\iusr{Борис Херсонский}
%%%fbauth_url
%%%fbauth_place
%%%fbauth_id
%%%fbauth_front
%%%fbauth_desc
%%%fbauth_www
%%%fbauth_pic
%%%fbauth_pic portrait
%%%fbauth_pic background
%%%fbauth_pic other
%%%fbauth_tags
%%%endfbauth
 
\textbf{Сергей Дибров} 

Добре, Сергію! Але закон може бути й нереалістичним. Ви живете в Одесі. Ви
добре знаєе мовну сітуацію тут. Думаю, що знаєте мою позицію. Я не юрист. Але я
психотерапевт, яий має завдання повернути людину обличчям до реальності. До
речі - Конституція дає мені право висловити свою точку зору. Ось я використовую
своє громадянське право.

%%%fbauth
%%%fbauth_name
\iusr{Сергей Дибров}
%%%fbauth_url
%%%fbauth_place
%%%fbauth_id
%%%fbauth_front
%%%fbauth_desc
%%%fbauth_www
%%%fbauth_pic
%%%fbauth_pic portrait
%%%fbauth_pic background
%%%fbauth_pic other
%%%fbauth_tags
%%%endfbauth
 

\textbf{Борис Херсонский}
Якщо закон нереалістичний, проте не суперечить Конституції...

%%%fbauth
%%%fbauth_name
\iusr{Борис Херсонский}
%%%fbauth_url
%%%fbauth_place
%%%fbauth_id
%%%fbauth_front
%%%fbauth_desc
%%%fbauth_www
%%%fbauth_pic
%%%fbauth_pic portrait
%%%fbauth_pic background
%%%fbauth_pic other
%%%fbauth_tags
%%%endfbauth
 
\textbf{Сергей Дибров} то його не будуть виконувати.

%%%fbauth
%%%fbauth_name
\iusr{Mark Beigelzimer}
%%%fbauth_url
%%%fbauth_place
%%%fbauth_id
%%%fbauth_front
%%%fbauth_desc
%%%fbauth_www
%%%fbauth_pic
%%%fbauth_pic portrait
%%%fbauth_pic background
%%%fbauth_pic other
%%%fbauth_tags
%%%endfbauth
 
\textbf{Сергій Дібров} виконує, я виконую, у спілкуванні ми виконуємо...
\end{itemize}

%%%fbauth
%%%fbauth_name
\iusr{Iaroslava Strikha}
%%%fbauth_url
%%%fbauth_place
%%%fbauth_id
%%%fbauth_front
%%%fbauth_desc
%%%fbauth_www
%%%fbauth_pic
%%%fbauth_pic portrait
%%%fbauth_pic background
%%%fbauth_pic other
%%%fbauth_tags
%%%endfbauth
 

Як ви елеґантно зливаєтеся в риториці з російськими пропагандистами, які теж
вважають поширеня сфери побутування української мови "великою перемогою над
людським розумом". Мені шкода, що для вас людський розум - це лише російська, і
від невдоволення своїм рівнем української переходите до таких ганебних
узагальнень. Глибоко засмучена рівнем рефлексії, але кожен, звісно, має право
вивалити на загальний огляд всяке.

\begin{itemize}
%%%fbauth
%%%fbauth_name
\iusr{Борис Херсонский}
%%%fbauth_url
%%%fbauth_place
%%%fbauth_id
%%%fbauth_front
%%%fbauth_desc
%%%fbauth_www
%%%fbauth_pic
%%%fbauth_pic portrait
%%%fbauth_pic background
%%%fbauth_pic other
%%%fbauth_tags
%%%endfbauth
 
\textbf{Iaroslava Strikha} 

Ярослава, війна з людським розумом це неадекватні вимоги з ігноруванням
реальної стуації в Україні. Ви не обов'язані знати, який шлях я подолав, щоб
спілкуватися й писати державною мовою. Яку шалену критику я толерував на цьому
шляху. Не чекав від Вас того, що я зливаюся с російськими пропагандистами.


%%%fbauth
%%%fbauth_name
\iusr{Борис Херсонский}
%%%fbauth_url
%%%fbauth_place
%%%fbauth_id
%%%fbauth_front
%%%fbauth_desc
%%%fbauth_www
%%%fbauth_pic
%%%fbauth_pic portrait
%%%fbauth_pic background
%%%fbauth_pic other
%%%fbauth_tags
%%%endfbauth
 
\textbf{Iaroslava Strikha} О, Ви мене вже відфрендили. Добра справа.

%%%fbauth
%%%fbauth_name
\iusr{Остап Українець}
%%%fbauth_url
%%%fbauth_place
%%%fbauth_id
%%%fbauth_front
%%%fbauth_desc
%%%fbauth_www
%%%fbauth_pic
%%%fbauth_pic portrait
%%%fbauth_pic background
%%%fbauth_pic other
%%%fbauth_tags
%%%endfbauth
 

\textbf{Борис Херсонский} 

теза про те, що деколонізація є перемогою на людським розумом - направду дивна
і відгонить російським наративом про нормативність постколоніального становища,
але я таки припустив, що неправильно щось відчитав, бо й решті допису це не
дуже суголосне.

\end{itemize}

%%%fbauth
%%%fbauth_name
\iusr{Ольга Нагорнюк}
%%%fbauth_url
%%%fbauth_place
%%%fbauth_id
%%%fbauth_front
%%%fbauth_desc
%%%fbauth_www
%%%fbauth_pic
%%%fbauth_pic portrait
%%%fbauth_pic background
%%%fbauth_pic other
%%%fbauth_tags
%%%endfbauth
 

Борисе Григоровичу, Ваші сумніви щодо можливості досконалого викладання
українською, лише підкреслюють Вашу порядність, чесність, фаховість. Хто ж це
ще з викладачів так піклується, як буде з лекціями українською? При Вашому
гарному знанні мови Ви ще переймаєтеся. Не буде поки що ніякого реального
тотального переходу . Нічого не лишайте


%%%fbauth
%%%fbauth_name
\iusr{Михаил Шелудько}
%%%fbauth_url
%%%fbauth_place
%%%fbauth_id
%%%fbauth_front
%%%fbauth_desc
%%%fbauth_www
%%%fbauth_pic
%%%fbauth_pic portrait
%%%fbauth_pic background
%%%fbauth_pic other
%%%fbauth_tags
%%%endfbauth
 

Пане Борисе, а що саме в мовному законі ви вважаєте дискримінаційним? Я теж
деякі його положення вважаю зрадою українських національних інтересів, але мені
цікаве саме ваше сприйняття.

\begin{itemize}
%%%fbauth
%%%fbauth_name
\iusr{Борис Херсонский}
%%%fbauth_url
%%%fbauth_place
%%%fbauth_id
%%%fbauth_front
%%%fbauth_desc
%%%fbauth_www
%%%fbauth_pic
%%%fbauth_pic portrait
%%%fbauth_pic background
%%%fbauth_pic other
%%%fbauth_tags
%%%endfbauth
 
\textbf{Михаил Шелудько} 

поперед усього - закон не враховує регіональних відмінностей в культурно-мовній
ситуації й не передбачає системного влаштування освітніх програм для дорослого
населення, а поперед усього - викладачів вищих учбових закладів .
Дискримінаційним закон у цій галузі є відносто викладачів похилого віку, які
жили й живуть у російськомовному оточені (мій випадок) й студентів які на
знають державної мови - таких досить є багато. Якщо я читаю українською (а я це
роблю) рівень моїх лекцій страждає. Тут нічого не поробиш. Є багато чого
скзати, але це явно не ФБ-формат.


%%%fbauth
%%%fbauth_name
\iusr{Свящ. Олесь Август Чумаков}
%%%fbauth_url
%%%fbauth_place
%%%fbauth_id
%%%fbauth_front
%%%fbauth_desc
%%%fbauth_www
%%%fbauth_pic
%%%fbauth_pic portrait
%%%fbauth_pic background
%%%fbauth_pic other
%%%fbauth_tags
%%%endfbauth
 
\textbf{Борис Херсонский} 

студенти? Сьогоднішні першокурсники - це діти, що народилися в рік Помаранчевої
революції. Якщо серед них є такі, що НЕ ЗНАЮТЬ української мови, то лише для
того, що ці одиничні нехтували мовою своєї Вітчизни. Зрештою, я таких не
зустрічав. Але може я таких до себе не притягую? Курси - це тимчасова
технологія, яка швидко мине і потреби в ній в короткому часі вже не буде.

Впевнений, що впровадження відповідних курсів для тих, хто не володіє
українською мовою, є завданням губернатора і його структур. Бо сам пишеш, що
справа є "регіональною". Що до "недосконалості мови", благаю тебе: залиш цю
мороку перфекціоністам! 

Дорогой БГ! Я щоденно досліджую і Пушкіна, і Гоголя, і Булгакова, заглиблююсь
листування Волошина, щоденники Гіпіус - хто мені це заборонить? 

Але я щасливий тим, що українська мова від тепер є по справжньому захищеною і
обов'язковою в державних, офіційних і публічних стосунках: тепер не соромно
перед поляками, словаками і прибалтами, які не могли зрозуміти того, що
українці дозволяють нехтувати власною мовою. 

Не погоджуюсь з тобою, тим паче з Люсею. Але це ніяк не руйнує моєї любові до
вас!


%%%fbauth
%%%fbauth_name
\iusr{Michael Ivaniv}
%%%fbauth_url
%%%fbauth_place
%%%fbauth_id
%%%fbauth_front
%%%fbauth_desc
%%%fbauth_www
%%%fbauth_pic
%%%fbauth_pic portrait
%%%fbauth_pic background
%%%fbauth_pic other
%%%fbauth_tags
%%%endfbauth
 
\textbf{Борис Херсонский} Я Вас розумію, але так можна тягнути і вбивати українську мову вічно. Якщо , тільки, вічно...
\end{itemize}

%%%fbauth
%%%fbauth_name
\iusr{Микола Байдюк}
%%%fbauth_url
%%%fbauth_place
%%%fbauth_id
%%%fbauth_front
%%%fbauth_desc
%%%fbauth_www
%%%fbauth_pic
%%%fbauth_pic portrait
%%%fbauth_pic background
%%%fbauth_pic other
%%%fbauth_tags
%%%endfbauth
 
\textbf{Борисе Григоровичу}, Вам – щира підтримка. І респект. Думаю, Ви висловили думку багатьох, хто своєї позиції озвучити не наважується (з тих чи інших причин).

% -------------------------------------
\ii{fbauth.svjaschennik.oles_avgust_chumakov.odessa.ukraina}
% -------------------------------------


Дорогий БГ! Моя укоріненість, так само і задля здобутого фаху філолога,
викладача російської літератури, в мові мого знищеного після нещасної жовтневої
катастрофи народу. 

Так, російська мова є мовою всіх моїх праотців і моєю від них спадщиною. Але чи
це перешкоджає мені вести ефіри на "Радіо Марія"? 

Хоча я знаю, скільки помилок допускаю, але служу свому народові і своїй
Вітчизні з усіх моїх сил і можливостей. Чого і тобі щиро сердечно бажаю!

%%%fbauth
%%%fbauth_name
\iusr{Сергей Дибров}
%%%fbauth_url
%%%fbauth_place
%%%fbauth_id
%%%fbauth_front
%%%fbauth_desc
%%%fbauth_www
%%%fbauth_pic
%%%fbauth_pic portrait
%%%fbauth_pic background
%%%fbauth_pic other
%%%fbauth_tags
%%%fbauth_pubs
%%%endfbauth
 

\ifcmt
  ig https://scontent-lga3-1.xx.fbcdn.net/v/t39.30808-6/217622063_10224710634045816_31497547131298376_n.jpg?_nc_cat=109&ccb=1-3&_nc_sid=dbeb18&_nc_ohc=CJUHsxaC1lcAX_--u2W&_nc_ht=scontent-lga3-1.xx&oh=9af93bea33a2fa3503321be6fd5f7180&oe=6101646F
  width 0.4
\fi



%%%fbauth
%%%fbauth_name
\iusr{Michael Ivaniv}
%%%fbauth_url
%%%fbauth_place
%%%fbauth_id
%%%fbauth_front
%%%fbauth_desc
%%%fbauth_www
%%%fbauth_pic
%%%fbauth_pic portrait
%%%fbauth_pic background
%%%fbauth_pic other
%%%fbauth_tags
%%%fbauth_pubs
%%%endfbauth
 

Цей закон потрібно було приймати ще у 1992-93 рр. Тупо. Просто якнайтупіше
приймати. Раньше сядєш, раньше вийдєш ! Власне , оте гидотно -гниле "мова не на
часі ..." А що на часі , один з другим ? Що ? Твоє черево ? З мови все
починається. І ще про "тупість". Як говорили колись єврейські мудреці, "щоб
випрямити лінію, треба її викривити в протилежну сторону ".

%%%fbauth
%%%fbauth_name
\iusr{Valerii Zema}
%%%fbauth_url
%%%fbauth_place
%%%fbauth_id
%%%fbauth_front
%%%fbauth_desc
%%%fbauth_www
%%%fbauth_pic
%%%fbauth_pic portrait
%%%fbauth_pic background
%%%fbauth_pic other
%%%fbauth_tags
%%%fbauth_pubs
%%%endfbauth
 
бідні поети есперантисти. вони усюди відчувають мовний дискомфорт

%%%fbauth
%%%fbauth_name
\iusr{Татьяна Юркова}
%%%fbauth_url
%%%fbauth_place
%%%fbauth_id
%%%fbauth_front
%%%fbauth_desc
%%%fbauth_www
%%%fbauth_pic
%%%fbauth_pic portrait
%%%fbauth_pic background
%%%fbauth_pic other
%%%fbauth_tags
%%%fbauth_pubs
%%%endfbauth
 
❤️❤️❤️❤️❤️

%%%fbauth
%%%fbauth_name
\iusr{Алексей Панич}
%%%fbauth_url
%%%fbauth_place
%%%fbauth_id
%%%fbauth_front
%%%fbauth_desc
%%%fbauth_www
%%%fbauth_pic
%%%fbauth_pic portrait
%%%fbauth_pic background
%%%fbauth_pic other
%%%fbauth_tags
%%%fbauth_pubs
%%%endfbauth
 

У 1990-ті роки я не наважувався викладати українською з мотивів, дуже схожих на
мотиви топікстарера.

Але 1999 року виборов грант на річне стажування в США. Де в американському
університеті від третього місяця перебування там вже викладав англійською.

Повернувся - і сам себе присоромив: невже англійською викладати можу, а
українською ні?

І почав викладати українською.

Відтоді викладаю трьома мовами, залежно від аудиторії, матеріалу і контексту.

Тож особистому бар'єру можу поспівчувати, але жодної суспільної проблеми у
цьому разі не бачу.

\begin{itemize}
%%%fbauth
%%%fbauth_name
\iusr{Олександр Бродецький}
%%%fbauth_url
%%%fbauth_place
%%%fbauth_id
%%%fbauth_front
%%%fbauth_desc
%%%fbauth_www
%%%fbauth_pic
%%%fbauth_pic portrait
%%%fbauth_pic background
%%%fbauth_pic other
%%%fbauth_tags
%%%fbauth_pubs
%%%endfbauth
 
\textbf{Олексій Панич} 

До того ж Ви стали одним з найкращих перекладачів іншомовних філософських,
релігієзнавчих, богословських книжок на українську мову.

% -------------------------------------
\ii{fbauth.panich_aleksej.odessa.kiev.ukraina.perevodchik.professor.dum_vivo_ordino}
% -------------------------------------


\textbf{Олександр Бродецький} 

Дякую на доброму слові. \Smiley[1.0][yellow]

Заради справедливості, мушу додати, що українська мова в мене аж ніяк не
ідеальна, тож я досі вчуся в кожного літературного редактора, з яким маю
справу. Але наснага вчитися завжди зі мною.

\end{itemize}

%%%fbauth
%%%fbauth_name
\iusr{Сашко Павленко}
%%%fbauth_url
%%%fbauth_place
%%%fbauth_id
%%%fbauth_front
%%%fbauth_desc
%%%fbauth_www
%%%fbauth_pic
%%%fbauth_pic portrait
%%%fbauth_pic background
%%%fbauth_pic other
%%%fbauth_tags
%%%fbauth_pubs
%%%endfbauth
 

Рішення Конст.суду - це не заборона вносити до даного закону доповнення,
поправки. Сподіваюсь, що так і буде: приймуть зміни, а ще краще - Мовний
кодекс, де повинно бути і освіти, і офіціоз, і бізнес, і регіональні
особливості, і мова ЗМІ, і права нацменш тощо. Якщо закон прийматимуть
передвиборчому психозі, істериці, а спокійно і з компромісами, то все може
вийти. Щоб враховувалось якомога більше...  

%%%fbauth
%%%fbauth_name
\iusr{Марія Дмитрієва}
%%%fbauth_url
%%%fbauth_place
%%%fbauth_id
%%%fbauth_front
%%%fbauth_desc
%%%fbauth_www
%%%fbauth_pic
%%%fbauth_pic portrait
%%%fbauth_pic background
%%%fbauth_pic other
%%%fbauth_tags
%%%fbauth_pubs
%%%endfbauth
 

і як завжди, всім державам можна мати свої державні мови - і тільки українська
в Україні когось дискримінує. (висловлю крамолу - в мене складається враження,
що російські переклади вам ближчі не через кращу якість, а через звичність -
як у поціновувачів Толкіна, який переклад перший прочитали, той і люблять. про
якість російських перекладів я тут нічого не писатиму, бо матюкатися в
присутності чоловіків не вважаю за прийнятне, шкодуючи ваші вуха і тонку
душевну організацію)

\begin{itemize}
%%%fbauth
%%%fbauth_name
\iusr{Natalia Martin}
%%%fbauth_url
%%%fbauth_place
%%%fbauth_id
%%%fbauth_front
%%%fbauth_desc
%%%fbauth_www
%%%fbauth_pic
%%%fbauth_pic portrait
%%%fbauth_pic background
%%%fbauth_pic other
%%%fbauth_tags
%%%fbauth_pubs
%%%endfbauth
 
\textbf{Марія Дмитрієва} 

проф. Херсонський мав на увазі переклади професійної літератури за своїм фахом.
І він знає, про що говорить. Зверніть увагу, у далекому 2001 р. вийшла його
робота (у співавторстві), присвячена актуальній проблемі - українською мовою.

\href{https://www.livelib.ru/book/1003439771-zhinka-ta-nasilstvo-posibnik-po-konsultativnij-dopomozi-zhinkam-u-krizovih-stanah-boris-hersonskij}{%
Жiнка та насильство: посібник по консультативній допомозі жінкам у кризових станах,%
Борис Херсонский, А. Н. Моховиков, 2001%
}

%%%fbauth
%%%fbauth_name
\iusr{Борис Херсонский}
%%%fbauth_url
%%%fbauth_place
%%%fbauth_id
%%%fbauth_front
%%%fbauth_desc
%%%fbauth_www
%%%fbauth_pic
%%%fbauth_pic portrait
%%%fbauth_pic background
%%%fbauth_pic other
%%%fbauth_tags
%%%fbauth_pubs
%%%endfbauth
 
\textbf{Марія Дмитрієва} 

я справді мав на увазі переклади професійної літератури. Будемо також
пам'ятати, що в 70 Бог вже не оставляє в тебе часу перечитати книгу в іншому
перекладі. Мати державну мову - обов'язково. Поважати її - поважати себе й
співгромадян. Коли людина зневажає її - це прикро. Але адмінитративне
впровадження мови на мій погляд небезпечний шлях.


%%%fbauth
%%%fbauth_name
\iusr{Свящ. Олесь Август Чумаков}
%%%fbauth_url
%%%fbauth_place
%%%fbauth_id
%%%fbauth_front
%%%fbauth_desc
%%%fbauth_www
%%%fbauth_pic
%%%fbauth_pic portrait
%%%fbauth_pic background
%%%fbauth_pic other
%%%fbauth_tags
%%%fbauth_pubs
%%%endfbauth
 
\textbf{Natalia Martin} 2001 рік! Разом зі святої пам'яті Олександром Маховіковим! Начебто вчора було...

%%%fbauth
%%%fbauth_name
\iusr{Марія Дмитрієва}
%%%fbauth_url
%%%fbauth_place
%%%fbauth_id
%%%fbauth_front
%%%fbauth_desc
%%%fbauth_www
%%%fbauth_pic
%%%fbauth_pic portrait
%%%fbauth_pic background
%%%fbauth_pic other
%%%fbauth_tags
%%%fbauth_pubs
%%%endfbauth
 

\textbf{Борис Херсонский} 

це єдиний спосіб, який працює надійно. і я згідна з коментарорами вище - ми
мали прийняти такий закон в 1992 році. мовна політика - це частина прямих
обов"язків держави - і частина національної безпеки. і так, я розумію, що ви
говорили про фахову літературу - і як людина, яка багато років фахово
займалася перекладом саме наукової літератури, я маю свою обгрунтовану точку
зору на російські переклади фахових текстів. на чому і стою. втім, погоджуся -
ми нагально потребуємо якісних перекладів цих текстів українською, і наскільки
я можу судити, робота в цьому напрямку ведеться.

%%%fbauth
%%%fbauth_name
\iusr{Марія Дмитрієва}
%%%fbauth_url
%%%fbauth_place
%%%fbauth_id
%%%fbauth_front
%%%fbauth_desc
%%%fbauth_www
%%%fbauth_pic
%%%fbauth_pic portrait
%%%fbauth_pic background
%%%fbauth_pic other
%%%fbauth_tags
%%%fbauth_pubs
%%%endfbauth
 
\textbf{Natalia Martin} 

пані Наталю, я теж говорила про фахову літературу. вочевидь, я мала написати це
великими літерами по центру коментаря.


%%%fbauth
%%%fbauth_name
\iusr{Марія Дмитрієва}
%%%fbauth_url
%%%fbauth_place
%%%fbauth_id
%%%fbauth_front
%%%fbauth_desc
%%%fbauth_www
%%%fbauth_pic
%%%fbauth_pic portrait
%%%fbauth_pic background
%%%fbauth_pic other
%%%fbauth_tags
%%%fbauth_pubs
%%%endfbauth
 
\textbf{Natalia Martin} 

а за книжку дякую, пошукаю. (знову ж таки, я нічого не казала ані про
українську пана Бориса, ані про його творчий доробок). упд. знайшла, замовила.


%%%fbauth
%%%fbauth_name
\iusr{Natalia Martin}
%%%fbauth_url
%%%fbauth_place
%%%fbauth_id
%%%fbauth_front
%%%fbauth_desc
%%%fbauth_www
%%%fbauth_pic
%%%fbauth_pic portrait
%%%fbauth_pic background
%%%fbauth_pic other
%%%fbauth_tags
%%%fbauth_pubs
%%%endfbauth
 
\textbf{Olexandr Chumakov} 

так, тобто автори "не словом, а ділом" вже тоді поширювали українську мову у
своїй професійній сфері. Вчора ще згадували про термінологічний словник,
укладений проф. Херсонським, який було видано без вказівки його авторства. Ото
вже дійсно ситуація "за моє жито мене й бито".

%%%fbauth
%%%fbauth_name
\iusr{Марія Дмитрієва}
%%%fbauth_url
%%%fbauth_place
%%%fbauth_id
%%%fbauth_front
%%%fbauth_desc
%%%fbauth_www
%%%fbauth_pic
%%%fbauth_pic portrait
%%%fbauth_pic background
%%%fbauth_pic other
%%%fbauth_tags
%%%fbauth_pubs
%%%endfbauth
 
\textbf{Natalia Martin} у нас із розумінням інтелектуальної власності загалом складно - і навіть коли когось на цьому ловлять, це не має жодних наслідків - ані адміністративних, ані репутаційних 🙁

%%%fbauth
%%%fbauth_name
\iusr{Natalia Martin}
%%%fbauth_url
%%%fbauth_place
%%%fbauth_id
%%%fbauth_front
%%%fbauth_desc
%%%fbauth_www
%%%fbauth_pic
%%%fbauth_pic portrait
%%%fbauth_pic background
%%%fbauth_pic other
%%%fbauth_tags
%%%fbauth_pubs
%%%endfbauth
 
\textbf{Марія Дмитрієва} важко здогадатися, що мова про фахову літературу, якщо проводяться аналогії з поціновувачами Толкіна.

\end{itemize}

%%%fbauth
%%%fbauth_name
\iusr{Борис Вольфсон}
%%%fbauth_url
%%%fbauth_place
%%%fbauth_id
%%%fbauth_front
%%%fbauth_desc
%%%fbauth_www
%%%fbauth_pic
%%%fbauth_pic portrait
%%%fbauth_pic background
%%%fbauth_pic other
%%%fbauth_tags
%%%fbauth_pubs
%%%endfbauth
 

C'est pire qu'un crime, c'est une faute – Это хуже, чем преступление, это
ошибка (Антуан Жак Клод Жозеф Буле де ла Мёрт об убийстве герцога Энгиенского).
Вспомнилось почему-то.


%%%fbauth
%%%fbauth_name
\iusr{Lyudmyla Leiva Garsia}
%%%fbauth_url
%%%fbauth_place
%%%fbauth_id
%%%fbauth_front
%%%fbauth_desc
%%%fbauth_www
%%%fbauth_pic
%%%fbauth_pic portrait
%%%fbauth_pic background
%%%fbauth_pic other
%%%fbauth_tags
%%%fbauth_pubs
%%%endfbauth

Пане Борисе, у нас в сім’ї дуже поважають вашу творчість і вашу громадянську
позицію. Майже завжди ваші замітки на ФБ співзвучні з нашими думками. І
насправді це дуже тішить, адже в Одесі почуватись комфортно складно, якщо ти не
заражений імперськими міфами і любиш Україну. Тим більш прикро читати ваші
дописи подібні до цього.

Звісно, це ваша точка зору і я впевнена що вона вами ретельно обдумана як і все
що ви пишете.

Але дозвольте поділитись з вами історією.

Коли в 2005 році я починала навчання в маріупольському гуманітарному
університеті, з цілої групи студентів-міжнародників українською мовою говорила
1 студентка (не я, я перейшла на українську мову в 2010). Я маю на увазі - саме
говорила - це була її природна мова спілкування як на перервах, так і на
лекціях чи семінарах. Наприкінці навчального року 2005-2006 кількість
україномовних студентів складала 0. Не тому що когось відрахували, а тому що
навчання в українському ВНЗ відбило у студентки цю плебейську звичку -
«розмовляти по-сільському». Як на мене, це найкраща ілюстрація результатів
напів мір якими безкінечно намагаються улестити наших зросійщених громадян.
Подібних історій можу розказати вам ще дуже багато. Але наврядчи це якось
змінить вашу позицію.

Тим не меньш, хочу лише додати що була на зустрічах з вами і чула як ви
говорите і російською і українською і все вам добре вдавалось, тож бажаю вам не
бути надто критичним до свого володіння українською мовою.

\end{itemize}

