% vim: keymap=russian-jcukenwin
%%beginhead 
 
%%file 02_12_2020.news.ru.lenta_ru.1.karabah.opasnoje_sosedstvo
%%parent 02_12_2020.news.ru.lenta_ru.1.karabah
 
%%url 
 
%%author 
%%author_id 
%%author_url 
 
%%tags 
%%title 
 
%%endhead 
\subsubsection{Опасное соседство}
\label{sec:02_12_2020.news.ru.lenta_ru.1.karabah.opasnoje_sosedstvo}

Часть Нагорного Карабаха по-прежнему занимает непризнанный Арцах, куда,
наоборот, сейчас приезжают армяне, оставившие свои дома во время боевых
действий. По информации Минобороны России, при содействии миротворцев уже
вернулись 25 тысяч человек. Руководство НКР утверждает, что общее число
вернувшихся вдвое выше --- 55 тысяч, просто большинство добралось без помощи
военных.

\begin{leftbar}
	\bfseries
{\centering\color{orange}\Huge 90 000}\par
беженцев из Нагорного Карабаха насчитывалось в Армении к моменту подписания
договора о перемирии. Их количество на сегодняшний день неизвестно
\end{leftbar}

Ситуация в Карабахе в целом стабилизируется: перемирие соблюдается, беженцы
возвращаются, говорит старший научный сотрудник Центра постсоветских
исследований ИМЭМО РАН Станислав Притчин. Но вскоре жители региона могут
столкнуться с проблемой определения границы в районе территорий, которые
перешли под контроль Баку, полагает он.

Дело в том, что новая линия соприкосновения проходит буквально по отдельным
деревням и фактически чертится на ходу. Таким образом соседние поселения могут
оказаться по разные стороны «баррикад». Первые тревожные «звоночки» уже можно
было наблюдать в процессе перехода контроля над спорными районами. Например,
из-за отсутствия точных новых координат на золотом месторождении в районе
приграничной армянской деревни Сотк едва не вспыхнули межэтнические беспорядки:
азербайджанские военные, занявшие Кельбаджарский район согласно договору о
перемирии, снесли ворота, установленные армянами несколькими днями ранее. Это
породило слух, что Баку пытается незаконно захватить новые территории вдобавок
к уже оговоренным. Паника дошла даже до парламента Армении.

\ifcmt
pic https://icdn.lenta.ru/images/2020/12/01/17/20201201174844833/pic_9e802f3b9315572c5d3f1e1b4a88bbfa.jpg
caption Армянские беженцы в Степанакерте Фото: Sergei Grits / AP
\fi

Другие неприятные инциденты, только подогревающие и так накаленную обстановку,
связаны с недоработками местных властей. Например, жители города Бердзор
(Лачин) в Лачинском районе узнали о том, что их населенный пункт отойдет
азербайджанской стороне, только 26 ноября. На сборы у них оставалось всего
несколько дней. Хотя горожане озадачились этим вопросом сразу же, как узнали о
трехсторонних соглашениях, но в администрации им дали неправильный ответ.
Естественно, никакой организованной эвакуации из Арцаха не проводилось.

\begin{leftbar}
	\bfseries
Вопрос о границе не входит в соглашение о перемирии. Уже есть определенные
разночтения, где она должна проходить. Учитывая неприязнь [сторон], отсутствие
переговорных институтов, тут могут быть конфликтные точки\par
				\emph{Станислав Притчин}\par
				\emph{старший научный сотрудник Центра постсоветских исследований ИМЭМО}
\end{leftbar}

Кроме того, в скором времени в Карабах потянется новый поток беженцев --- теперь
уже с азербайджанской стороны. Для Баку принципиально важно, чтобы это
произошло, ведь это обеспечит моральную репатриацию территорий. «Беженцы в
Азербайджане, особенно первое время, жили в очень плохих условиях. Государство
сознательно не создавало каких-то новых условий для жизни, рассчитывая рано или
поздно обеспечить их возвращение», --- говорит Притчин.

Поэтому для Азербайджана в ближайшей перспективе ключевым вопросом будет
обеспечение безопасности, разминирование возвращенных регионов и строительство
там инфраструктуры. Скорее всего, будет принята долгосрочная системная
программа по восстановлению территорий и возвращению беженцев, которая
потребует колоссальных инвестиций, говорит собеседник «Ленты.ру».

\ifcmt
pic https://icdn.lenta.ru/images/2020/12/01/19/20201201190537766/pic_5489df69d04478f2586d71d10fb1f8fd.jpg
caption Фото: Vahram Baghdasaryan / Photolure / Reuters
\fi

Ситуация в Карабахе будет стабилизироваться неделями или даже месяцами.
Реальное число беженцев, которые вернулись или могут вернуться, неизвестно,
считает политолог Алексей Малашенко. Какие-то проблемы передела границ, которые
сейчас не очевидны, могут проявиться только к концу января, при этом существует
множество вариантов того, как будут развиваться события, пояснил он. Но уже
сейчас можно выделить два сценария развития конфликтов, связанных с тесным
соседством двух враждующих народов, и многое в их решении будет зависеть в
первую очередь от «тактичности миротворцев».

\begin{leftbar}
	\begingroup
	\color{orange}
	\bfseries
Если будет уличное противостояние между армянами и азербайджанцами в случае
возвращения --- это один вариант развития межэтнического конфликта. Если
политикам с двух сторон и миротворцам удастся это предотвратить —
другой\par
	\endgroup
\em
Алексей Малашенко\par
политолог\par
\end{leftbar}
