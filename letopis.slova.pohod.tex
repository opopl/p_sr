% vim: keymap=russian-jcukenwin
%%beginhead 
 
%%file slova.pohod
%%parent slova
 
%%url 
 
%%author_id 
%%date 
 
%%tags 
%%title 
 
%%endhead 
\chapter{Поход}

%%%cit
%%%cit_head
%%%cit_pic
%%%cit_text
В 1376 году князь Дмитрий Михайлович участвовал в важном \emph{походе} на Волжскую
Булгарию. Там он вместе с сыновьями суздальского князя Василием Кирдяпой и
Иваном 16 марта нанёс поражение правителям Булгара - эмиру Хасан-хану и
ордынскому ставленнику Мухаммад-Султану, несмотря на наличие у них артиллерии и
довольно сильного войска. В результате победы был получен огромный откуп в 5000
рублей, а Волжская Булгария стала на время вассальной по отношению к Москве.
9 декабря 1379 года Дмитрий Михайлович вместе с князьями Владимиром Андреевичем
и Андреем Ольгердовичем отправился в \emph{поход} в Брянское княжество. В итоге были
захвачены города Трубчевск и Стародуб, а также ряд других владений. Одним из
результатов этой военной кампании стало то, что правивший до этого в Трубчевске
князь Дмитрий Ольгердович перешёл со своим двором на московскую службу
%%%cit_comment
%%%cit_title
\citTitle{Военный гений Руси}, Илья Duke, zen.yandex.ru, 30.10.2021
%%%endcit
