% vim: keymap=russian-jcukenwin
%%beginhead 
 
%%file slova.gosudarstvo
%%parent slova
 
%%url 
 
%%author 
%%author_id 
%%author_url 
 
%%tags 
%%title 
 
%%endhead 
\chapter{Государство}
\label{sec:slova.gosudarstvo}

%%%cit
%%%cit_pic
%%%cit_text
И вот, в начале XX века, когда в результате Первой мировой войны, и
Австрийская, и Российская империи развалились, у западных русских, потерявших
свои корни и ставших «украинцами», появилась идея создания собственного
\emph{государства}. Ну, идеи создания собственных независимых \emph{государств} имеются не
только у национальных меньшинств, но даже и у однородных в этническом плане
социальных групп. Достаточно вспомнить Дальневосточную республику, образованную
в 1920 году и населенную русскими. Самый яркий пример – образование ряда
испаноязычных латиноамериканских \emph{государств}, образовавшихся на руинах Испанский
Америки. Можно вспомнить и прецедент образования США – ведь отделились от
Англии не индейцы и не негры, а те же самые англичане
%%%cit_title
\citTitle{Почему современную Украину назвали «окраиной», а не более престижно - Киевской Русью?}, 
Исторический Понедельник, zen.yandex.ru, 22.02.2021 
%%%endcit

%%%cit
%%%cit_head
%%%cit_pic
%%%cit_text
Українська нація на сучасному етапі своєї історії у здавалося б \enquote{незалежній}
\emph{державі} знаходиться у підневільному стані. Хоч ми вже чверть століття є начебто
\enquote{незалежними}, проте ще й досі не можемо цією незалежністю вдосталь
скористатися.  У наш час \emph{держава Україна} - неоколонія. Тобто, будучи \emph{державним
утворенням}, вона не є \emph{державою} для своєї титульної нації, а навпаки - всіма
можливими методами її поневолює та знищує, відстоюючи інтереси закордонних, як
відверто антиукраїнських сил, так і втілювачів ідеології ліберального
тоталітаризму у світовій політиці. А корінні українці тим часом і надалі
залишаються безправними на своїй землі. Жодні проблеми українців не вирішуються
на їх користь. Український народ загнали у ярмо, де його поступово і впевнено
знищують!
%%%cit_comment
%%%cit_title
\citTitle{Сучасний стан української нації та держави Україна}, 
, pravyysektor.info, 25.05.2017
%%%endcit

%%%cit
%%%cit_head
%%%cit_pic
%%%cit_text
Головна проблема українських бід полягає у \emph{бездержавності} української нації. У
тому, що у нас немає \emph{держави}, яка б забезпечувала українцям повноцінне
існування та розвиток, була центром об'єднання українців в Україні та за
рубежем. А те \emph{державне утворення}, що у нас наявне, українським по своїй суті не
є, оскільки в ньому забезпечуються інтереси будь-кого: чужоземців, педерастів,
сепаратистів, можновладців, олігархів, тобто всіх, окрім простих українців.
Представники титульної нації, яких у \emph{державі} 77\% (щонайменше), залишаються
безправними та беззахисними, неспроможними керувати своєю долею на своїй землі.
У цій \emph{державі} будь-яка українська націоналістична активність присікається, а
все, що є антиукраїнським, вітається чи мовчки допускається. Так було за кожної
влади в Україні, не був винятком навіть В. Ющенко, який вважався
\enquote{проукраїнським президентом} та на якого так покладали надії українці. Наразі
ми чітко бачимо, як лещата режиму внутрішньої окупації продовжують боротьбу на
знищення, допустивши війну з москалем, яку вигравати, очевидно, не збираються.
Створюється хіба вид лицемірної патріотичності та стурбованості, а паралельно
гине кращий генофонд нації у \enquote{перемир'ях} та у в'язницях
%%%cit_comment
%%%cit_title
\citTitle{Сучасний стан української нації та держави Україна}, 
, pravyysektor.info, 25.05.2017
%%%endcit

%%%cit
%%%cit_head
%%%cit_pic
%%%cit_text
Мрії українців про те, що європейці з американцями вирішать всі наші проблеми,
є ознакою їхньої меншовартості. Головна ж причина меншовартості українців – це
обірвана традиція \emph{державотворення}. Після входження України до складу
Московського царства, Речі Посполитої, Австро – Угорщини була перервана
\emph{державницька традиція}, яка підтримувалась руськими князями і королями Русі.
Перерваність традиції привчила українців до того, що ними можуть правити
іноземці, тому громадяни нашої країни так прихильно ставляться до членства в
НАТО та ЄС, не розуміючи того, що сильна Україна нікому з геополітичних гравців
не потрібна
%%%cit_comment
%%%cit_title
\citTitle{Що чекає на Україну?}, Стефан Закревський, 
analytics.hvylya.net, 19.06.2021
%%%endcit


%%%cit
%%%cit_head
%%%cit_pic
%%%cit_text
Для того чтобы изменить положение дел в Украине, достаточно привести жизнь
страны в соответствии с ее действующей Конституцией. В ней присутствуют все
демократические нормы, которые были де-факто отменены после победы Евромайдана.
Но будьте готовы, что защита Конституции может быть расценена как преступление
перед \emph{государством}, которое давно стало главным нарушителем основного закона
нашей страны
%%%cit_comment
%%%cit_title
\citTitle{Конституция выглядит сегодня самым радикальным и подрывным документом / Лента соцсетей / Страна}, 
Андрей Манчук, strana.ua, 28.06.2021
%%%endcit

%%%cit
%%%cit_head
%%%cit_pic
%%%cit_text
Третий, и важнейший изъян, на который указывает В. Путин: украинская
\emph{государственность} строится исходя не из прагматичных соображений, а из
романтических «хотелок». Президент В. Путин указывает, что в начале пути
Украина и Россия имели такую степень экономической интеграции, которая
Евросоюзу и не снилась. А ведь это и есть взаимные инвестиции, проникновение на
рынки, мировое распределение труда. Украина вечно выпускает синицу из рук, не
пытаясь поймать журавля в небе. Возможно, строить самолеты с ЕС куда выгоднее,
чем строить их с Россией. Только Украина и с ЕС их не начала строить, и с
Россией прекратила, и так с любой высокотехнологичной продукцией. Мы теряем, а
не получаем ничего, кроме иллюзий и несбывшихся надежд. То есть кто-то эти
иллюзии украинцам продает и неплохо зарабатывает. Но украинскому народу от
этого становится только хуже, и именно поэтому страна за последние годы
превратилась из бедной в нищую
%%%cit_comment
%%%cit_title
\citTitle{О будущем украинского и русского народов}, 
Виктор Медведчук, strana.ua, 15.07.2021
%%%endcit

%%%cit
%%%cit_head
%%%cit_pic
%%%cit_text
И нужно сначала прийти в состояние, когда \emph{государственные институты} могут
выполнять хотя бы базовые функции, и более менее, предсказуемо работать. И
держа в голове вот эту всю конструкцию по поводу необходимости сильного
общества и сильного \emph{государства}, дальше уже осознанно и рационально переходить
к тому, чтобы давать возможность развиваться сильному обществу, которое просто
необходимо переконструировать.
Потому что, если мы его не переконструируем, если у нас не появится
\emph{государство}, которое будет обеспечивать хотя бы базовые правила игры, мы
свалимся в состояние, как один из знакомых западных аналитиков мне сказал, что
«Украина все больше похожа на Йемен Восточной Европы».
И это очень опасное состояние. Потому что оно напоминает то, что было в
середине 17 века и его третьей четверти, когда Украина стала полем битвы для
многих \emph{государства} того времени: Оттоманская империя, Московия, Речь
Посполитая, Швеция, которая там с севера боролась с Польшей
%%%cit_comment
%%%cit_title
\citTitle{Государство в ХХI веке: неизбежна ли для Украины автократия?}, 
Михаил Минаков; Юрий Романенко, hvylya.net, 24.07.2021
%%%endcit

%%%cit
%%%cit_head
%%%cit_pic
%%%cit_text
Є головне загальнонаціональне питання – чому ми живемо в такій недолугій
\emph{державі}, адже після переїзду в інші держави, наші громадяни легко
адаптуються та займають гідне місце на новій батьківщині. Вочевидь, щось не так
з нашими соціальними практиками та й з самими громадянами.  Доводиться
констатувати, що в Україні створена \emph{кастово-феодальна держава}.
Населення, в повсякденному житті, намагається імітувати поведінку правлячої
касти. Бажає бути зрозумілою, правильною людиною феодальної системи та за
нагоди не проґавити свій шанс, взяти своє.  Головна біда правлячої касти –
власні діти. Як їм правильно передати гроші і майно, щоб вони не були забрані
новими феодалами, адже всі розуміють в який спосіб виникли ці надбання. Тому їх
нагальне завдання закріплення, на завжди, наявного стану речей в Україні, щоб
їх діти і внуки не тільки могли безперешкодно, довічно, володіти надбаннями
батьків, а й примножували їх надбання використовуючи свій привілейований стан
%%%cit_comment
%%%cit_title
\citTitle{Як змінити Україну на краще?}, 
Сергей Ищук, analytics.hvylya.net, 17.07.2021
%%%endcit

%%%cit
%%%cit_head
%%%cit_pic
%%%cit_text
І це в умовах, коли США, налякані розпадом могутньої \emph{ядерної держави},
намагалися максимально зменшити ризики для себе і всього світу. Ядерна зброя
могла бути чинником стримування в умовах біполярної системи, коли існували дві
столиці, які перебували в стані постійної комунікації. «Розпорошення»
потенціалу ядерної зброї не входило в плани американців. Західний світ не
хотів, образно кажучи, мати справу вже не з однією, а з двома «мавпами з
ядерною гранатою». Тому на Україну почався серйозний тиск, щоб вона перейшла до
без’ядерного статусу. Тим більше, що в Декларації про державний суверенітет
(1990) значилося, що Україна «не прийматиме, не вироблятиме і не набуватиме
ядерної зброї»
%%%cit_comment
%%%cit_title
\citTitle{Мавпа з ядерною гранатою}, 
Василь Расевич, zaxid.net, 23.07.2021
%%%endcit

