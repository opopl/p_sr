% vim: keymap=russian-jcukenwin
%%beginhead 
 
%%file 09_12_2020.news.ru.izvestia.litvinova_marta.1.uchitelja_udalenka
%%parent 09_12_2020
 
%%url https://iz.ru/1097419/marta-litvinova/derzhat-distantciiu-kak-uchitelia-spravliaiutsia-s-udalenkoi
 
%%author Литвинова, Марта
%%author_id litvinova_marta
%%author_url 
 
%%tags russia,covid,udalenka,teacher
%%title Держать дистанцию: как учителя справляются с удаленкой
 
%%endhead 
 
\subsection{Держать дистанцию: как учителя справляются с удаленкой}
\label{sec:09_12_2020.news.ru.izvestia.litvinova_marta.1.uchitelja_udalenka}
\Purl{https://iz.ru/1097419/marta-litvinova/derzhat-distantciiu-kak-uchitelia-spravliaiutsia-s-udalenkoi}
\ifcmt
	author_begin
   author_id litvinova_marta
	author_end
\fi

\begin{center}
\color{orange}\textbf{\em Кто помогает педагогам в организации уроков в период пандемии}
\end{center}

Сегодня учителям нередко самим приходится осваивать новые форматы работы, а
подготовка к урокам стала занимать в разы больше времени и усилий. Однако
помощь учителям начинают оказывать различные структуры повышения квалификации.
В Нижнем Новгороде открывается «Школа цифрового педагога», немногим ранее Центр
дистанционного образования начал работать и в Ульяновской области. Новые
платформы в первую очередь призваны помочь школьным работникам в организации и
сопровождении дистанционного обучения. Подробности — в материале «Известий».

\subsubsection{Обмен практикой}

Как сообщили в Нижегородском институте развития образования, планируется, что
«Школа цифрового педагога» будет представлять собой виртуальное пространство,
где специалисты смогут делиться опытом и практическими навыками использования
цифровых технологий в образовательной деятельности.

\ifcmt
pic https://cdn.iz.ru/sites/default/files/styles/900x506/public/photo_item-2020-12/539331.jpg?itok=qPS5yKKU
cpx  Фото: агентство городских новостей «Москва»/Денис Гришкин
\fi

Стараются помочь педагогам в освоении такого формата обучения и в других
регионах. Так, в ноябре в Ульяновской области начал работу Центр дистанционного
образования: новая площадка будет взаимодействовать со всеми муниципалитетами в
качестве методической базы.

— В ресурсном центре планируется создать интернет-платформу, где будут собраны
материалы уроков, тесты, презентации, а также возможность участия в
онлайн-уроках по отдельному расписанию, заранее согласованному как с
педагогами, так и со школьниками и их родителями. Кроме того, к функциям
данного центра будет относиться консультационная работа педагогов по
сопровождению дистанционного обучения, — рассказала министр просвещения и
воспитания Ульяновской области Наталья Семенова.

Сейчас в центре оказывается методическая помощь педагогам, обучающимся,
родителям при организации дистанционного обучения. Со следующего года
планируется приступить к организации в набранных классах (по желанию самих
школьников и их родителей) полноценного ведения образовательной деятельности в
дистанционной форме.

\subsubsection{Часы на подготовку}

Пока же в большинстве случаев справляться со всеми трудностями организации
урока в дистанционном формате приходится самим учителям, отметил в разговоре с
«Известиями» преподаватель русского языка школы «Интеллектуал», сопредседатель
профсоюза «Учитель» Всеволод Луховицкий. «В целом дистанционная работа намного
сложнее, в частности в плане времени, необходимого на подготовку к уроку. В
среднем у меня его уходит примерно в два раза больше», — пояснил он.

\ifcmt
pic http://cdn.iz.ru/sites/default/files/styles/900x506/public/photo_item-2020-12/539570.jpg?itok=BJdCPIm6
cpx Фото: агентство городских новостей «Москва»/Софья Сандурская
\fi

— Сложно и в смысле методическом: многие приемы, которые нормально использовать
на уроке, здесь не подходят. Например, чтобы сделать тест, обычно я его
придумывал, составлял, распечатывал на листочках два варианта и спокойно
раздавал в классе. А теперь я должен в «Google Классе» специально вносить
каждый вопрос, должен разделить детей на гораздо большее количество групп,
потому что если будет всего два варианта, то дети успеют друг с другом
посоветоваться через интернет. Поэтому я должен сделать пять-шесть вариантов, —
рассказывает Луховицкий. — Или, например, я не могу теперь давать никаких
заданий, связанных с исправлением ошибок, с расстановкой знаков препинания, то
есть никаких заданий на орфографию, пунктуацию, потому что дети, естественно,
тут же залезут в интернет и найдут эти предложения. Мне приходится работать
писателем, придумывая заново предложения, и они гораздо менее интересны, чем у
Толстого или Чехова, и качество урока от этого не улучшается.

Всем учителям чрезвычайно сложно готовиться к урокам в дистанционном формате,
считает преподаватель географии в физико-математическом лицее «Вторая школа»
Леонид Перлов. Особенно сложно разобраться в этом пожилым педагогам, которым в
большинстве случаев приходится постигать техническую сторону новой формы
обучения самостоятельно. «Проводятся обучающие вебинары на этот счет, но
опять-таки в удаленном формате. Их ведут люди, конечно, неплохо разбирающиеся в
технологиях, но не учитывающие, что их слушают люди с другой подготовкой,
поэтому учителя 60–65 лет и половины терминов не понимают», — отмечает он.

\ifcmt
pic http://cdn.iz.ru/sites/default/files/styles/900x506/public/photo_item-2020-12/RIAN_6344845.HR_.ru_.jpg?itok=O217pKU-
cpx Фото: РИА Новости/Алексей Майшев
\fi

— В общем-то сейчас наблюдается путаница понятий. То, что есть сейчас,
дистанционным образованием назвать пока нельзя, правильнее использовать термин
«удаленное обучение». В большинстве случаев берутся традиционные методики,
материалы, учебники, и просто все это подается не в классе напрямую ученикам, а
через монитор. Настоящее дистанционное образование — в процессе разработки,
исправления ошибок, — считает Перлов.

\subsubsection{Платформы для помощи}

Между тем существует ряд федеральных и региональных образовательных
онлайн-платформ, рекомендованных Министерством просвещения для организации
уроков вне стен школы. В частности, курс, помогающий преподавателям научиться
пользоваться дистанционными формами обучения, был создан союзом «Профессионалы
в сфере образовательных инноваций». Как рассказал «Известиям» руководитель
спецпроектов «ОбрСоюза» Евгений Антонов, весной ощущалась острая нехватка
знаний по выстраиванию эффективной схемы дистанционного обучения. Многие
учителя и родители отмечали, что детям сложно сидеть весь урок за компьютером,
а без живого контакта трудно обеспечить дисциплину и контролировать выполнение
заданий. Что касается самого рекомендованного министерством курса, то он
состоит из нескольких модулей, решающих разные задачи.

\ifcmt
pic http://cdn.iz.ru/sites/default/files/styles/900x506/public/photo_item-2020-12/KK302060.jpg?itok=VX3IWfXY
cpx Фото: ИЗВЕСТИЯ/Кристина Кормилицына
\fi

— В первом модуле показывается разнообразие образовательных ресурсов для
педагогов, второй модуль посвящен технологиям дистанционного образования, а
третий рассчитан на приобретения конкретной компетенции — от проведения
трансляции до основ создания онлайн-курсов. В целом наш ресурс позволяет
учителям освоить цифровые технологии и сделать дистанционное обучение
интересным и увлекательным. Обучение онлайн не должно состоять только из
просмотра видео, это очень разнообразный процесс, и педагог, владеющий
соответствующими навыками, может использовать яркие образы, интерактивные
моменты и в итоге сократить время нахождения у компьютера, — отмечает Антонов.
По данным «ОбрСоюза», прошли итоговое тестирование и получили сертификат на
сегодняшний день почти 32 тыс. человек.

\ifcmt
pic https://cdn.iz.ru/sites/default/files/styles/970x546/public/video_item-2020-10/%D0%A3%D1%87%D0%B8%D1%82%D0%B5%D0%BB%D0%B5%D0%B9%20%D0%B8%D0%B7%20%D0%B3%D1%80%D1%83%D0%BF%D0%BF%D1%8B%20%D1%80%D0%B8%D1%81%D0%BA%D0%B0%20%D0%BC%D0%BE%D0%B3%D1%83%D1%82%20%D0%B7%D0%B0%D0%BC%D0%B5%D0%BD%D0%B8%D1%82%D1%8C%20%D1%81%D1%82%D1%83%D0%B4%D0%B5%D0%BD%D1%82%D1%8B%20%2B0.jpg?itok=VOzACHM2
\fi

\subsubsection{Дело школы}

Председатель Ассоциация молодых педагогов России, директор петербургского лицея
№ 369 Константин Тхостов считает, что проблемы с организацией удаленки
необходимо решать как на низовом, так и на региональном уровне.

— В Санкт-Петербурге был создан отдельный ресурс совместно с Российским
государственным педагогическим университетом им. А.И. Герцена. В его рамках
ребятам предоставляется смешанная форма обучения, позволяющая освоить материал
из учебного плана, выполнить домашнее задание. И это задание потом проверяется
как студентами, так и учителями — так и происходит эта смешанная форма. Это
решение предложило правительство Санкт-Петербурга, но существуют решения и на
уровне образовательных организаций. Мы, например, три года работаем над тем,
чтобы сформировать облачный сервис, который позволяет внутри него вести общение
между учителем и учеником, обмениваться заданиями и заниматься индивидуальной
подготовкой, в том числе и в рамках олимпиадного движения, — отмечает
собеседник «Известий». — В целом любая образовательная организация, если она
стремится к современному качественному образованию, должна заботиться о том,
чтобы учитель имел возможность непрерывного общения как с детьми, так и с их
родителями.

\ifcmt
pic https://cdn.iz.ru/sites/default/files/styles/970x546/public/video_item-2020-09/%D1%83%D1%87%D0%B8%D1%82%D0%B5%D0%BB%D1%87.jpg?itok=HexfKvAs
\fi
