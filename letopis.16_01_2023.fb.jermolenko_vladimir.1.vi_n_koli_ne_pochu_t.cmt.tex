% vim: keymap=russian-jcukenwin
%%beginhead 
 
%%file 16_01_2023.fb.jermolenko_vladimir.1.vi_n_koli_ne_pochu_t.cmt
%%parent 16_01_2023.fb.jermolenko_vladimir.1.vi_n_koli_ne_pochu_t
 
%%url 
 
%%author_id 
%%date 
 
%%tags 
%%title 
 
%%endhead 

\qqSecCmt

\raggedcolumns
\begin{multicols}{2} % {
\setlength{\parindent}{0pt}

\begin{itemize} % {
\iusr{Hanna Chernenko}

Чотири дні вони бомбили місто. /Й не стало більше міста, бо містА - не люди й
не сховаються в під'їзді.

\iusr{Svitlana Druzenko}

Як ти добре пишеш! Дякую!

\iusr{Cynthia Krkoska-Nielsen}

Very powerful reflections, Volodymyr.

\iusr{Вікторія Шавурська}

Кожне слово прямісінько в душу. До сліз💔🙏🏽

\iusr{Olena Koshil}

Дуже сильно. І ще зробити б нам так, щоб міста не мовчали про свої трагедії, не
тримали в собі, а розказували про біль і побутовий героїзм, про стійкість і
незламність українців, про взаємодопомогу і перемогу. Щоб всі знали і не
забували, що таке російський фашизм. І пильно-пильно берегли нашу свободу.

\iusr{Tetiana Kovalenko}

...»Розрізаних ворожими ракетами і людською тишою». Це така рана. Усе і усі в цих
містах. Потужний текст.

\iusr{Olga Sereda}

Убивцям і катам,

призвичаєним (разом із своїми "совєстлівими"

і "так же нє-віноватими/нє-прічастнимі/нє-імєющімі..."

спів-"громадянами"/родичами)

до того, що

не менше 30 років від 1991

"мають належні і достатні

пом'якшувальні обставини" -

від

"провокативної поведінки" жертви

до

"важких обставин ..." у злочинця,

який "з огляду" на його належність

до

"в цілому Договороздатної автократії"

і "гуманізма ВєлікайКультури",

"має право"

розраховувати на

"щонайбільше - означення злочину конфліктом

й однакову вину ОбохСторін",

"амністію - неодмінно у формі

звільнення від відповідальності

(а не від відбування покарання)",

а також вимагати "докази"

в умовах,

коли "незалежні треті сторони"

можуть без втрати

не тільки статків і авторитету,

а й докорів сумління

"розформувати комісію" із збору даних

через "відсутність гарантій безпеки ОбохСторін".

І виникає альтернативне питання

(не самих по собі "не-пробачення/помсти"):

Як світ збирається існувати далі у "мирі",

в якому від 1945 "зникнення воєн відбулося"

шляхом:

- їх переозначення на "конфлікти";

- "відсутності доказів" через

"недотримання процедурних моментів"

навіть на етапі "ініціації збору фактів";

- банів за "чутливий контент" 🤔?

\iusr{Ольга Продан}

Дуже пронизливо читати це

\iusr{Romanna Yakivchyk}
Дякую. 💔

\iusr{Natalia Velbovets}
Скільки разів вони вже пробачали.

\iusr{Любовь Королькова}
Гениально и пронзительно...

\iusr{Antonina Vavrysh}
Contra spem spero!

\iusr{Євстахія Шимчук}
😔Моторошна Правда...

\iusr{Тамара Огданська}
Дякую, Володимире.

\iusr{Mila Kondrachuk}
Дякую

\iusr{Svitlana Feldman}
Дякую

\iusr{Sofia Tsysareva}
Дякую

\end{itemize} % }

\end{multicols} % }
