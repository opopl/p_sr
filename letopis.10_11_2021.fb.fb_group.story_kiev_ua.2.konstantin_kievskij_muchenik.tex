% vim: keymap=russian-jcukenwin
%%beginhead 
 
%%file 10_11_2021.fb.fb_group.story_kiev_ua.2.konstantin_kievskij_muchenik
%%parent 10_11_2021
 
%%url https://www.facebook.com/groups/story.kiev.ua/posts/1794425737420886/
 
%%author_id fb_group.story_kiev_ua,gorbov_vadim.kiev
%%date 
 
%%tags cerkov,istoria,nkvd,repressii,sssr
%%title Священномученик Константин Киевский
 
%%endhead 
 
\subsection{Священномученик Константин Киевский}
\label{sec:10_11_2021.fb.fb_group.story_kiev_ua.2.konstantin_kievskij_muchenik}
 
\Purl{https://www.facebook.com/groups/story.kiev.ua/posts/1794425737420886/}
\ifcmt
 author_begin
   author_id fb_group.story_kiev_ua,gorbov_vadim.kiev
 author_end
\fi

Уважаемые участники группы. Чтобы меня голословно не обвиняли в односторонней
прокоммунистической позиции в трактовке нашей общей и порою трагической
Истории, я нарушу джентльменскую договорённость в нашем Худсовете  не
публиковать больше одного поста одного автора в день и опубликую сегодня ещё
этот текст. 

Как объективный ни антисоветский и ни просоветский Летописец Эпохи. 

\ifcmt
  ig https://scontent-lga3-1.xx.fbcdn.net/v/t1.6435-9/255835688_4411690008959294_6001737053017784884_n.jpg?_nc_cat=107&ccb=1-5&_nc_sid=825194&_nc_ohc=q9htDtzzjVUAX_-GXS9&_nc_ht=scontent-lga3-1.xx&oh=c88487256cda564e99536078248fdd25&oe=61B176D5
  @width 0.5
  %@wrap \parpic[r]
  @wrap \InsertBoxR{0}
\fi

Советская власть и ее репрессивные органы ЧК-ГПУ-НКВД-КГБ унесла жизни
миллионов граждан нашей страны. Самых лучших, самых образованных, ученых и
медиков, философов и поэтов, творческую интеллигенцию, буржуазию,
крестьянство и священнослужителей. В годы Большого Террора были
репрессированы все епископы, муллы, ксендзы, раввины, абсолютное
большинство священников и монахов расстреляны или отправлены в лагеря,
храмы были или взорваны или переоборудованы в склады, монастыри - в колонии
и психиатрические и тубдиспансеры. 

Но трагическое событие, произошедшее 10 ноября 1937 года в Лукьяновской
тюрьме даже для того кровавого и жестокого времени Молоха было из ряда вон
выходящим. ГПУ даже проводило внутреннее расследование и наказали виновных,
следователя, дознавателя и тюремного врача понизили в должностях. 

Во время допроса с пристрастием скончался арестованный по делу о
«контрреволюционном заговоре церковников»  Митрополит Киевский и Галицкий
Константин (Дьяков). Высший церковный иерарх Украины в то время. 

Реабилитирован в 1956 за отсутствием состава преступления. 

Похоронен в общей могиле на Лукьяновском кладбище, куда по ночам  почти
ежедневно свозили тела расстрелянных в Лукьяновской тюрьме заключённых.  

Канонизирован Украинской Православной Церковью как Священномученик
Константин Киевский. Сегодня отмечается день его памяти.

\ii{10_11_2021.fb.fb_group.story_kiev_ua.2.konstantin_kievskij_muchenik.cmt}
