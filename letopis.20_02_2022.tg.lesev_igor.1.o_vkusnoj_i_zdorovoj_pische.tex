% vim: keymap=russian-jcukenwin
%%beginhead 
 
%%file 20_02_2022.tg.lesev_igor.1.o_vkusnoj_i_zdorovoj_pische
%%parent 20_02_2022
 
%%url https://t.me/Lesev_Igor/318
 
%%author_id lesev_igor
%%date 
 
%%tags ukraina,zelenskii_vladimir
%%title О вкусной и здоровой пище
 
%%endhead 
 
\subsection{О вкусной и здоровой пище}
\label{sec:20_02_2022.tg.lesev_igor.1.o_vkusnoj_i_zdorovoj_pische}
 
\Purl{https://t.me/Lesev_Igor/318}
\ifcmt
 author_begin
   author_id lesev_igor
 author_end
\fi

О вкусной и здоровой пище

Почитал вчера отзывы на «речь Зеленского в Мюнхене» в сегменте
авторов-лизоблюдов, которые сводились к тезисам, «ах, какое правильное
выступление» и «так держать». Отметил, что в Офисе подрядили поцеловать в попу
Блогера вполне респектабельных ребят, вроде Руслана Бортника.

Мотивы подобных попа-поцелуев разбирать не буду, думаю, они нам вполне
очевидны. И кидать камни в авторов тоже не призываю. Репутация создается
годами, а потерять ее можно буквально за два абзаца и штуку баксов. Но порой и
штука-другая денег тоже ведь очень и очень нужна, мы ж тут не сканируем чужие
потемки.

От себя скажу так. Вот если бы Чикатило не расстреляли, и он сейчас по-прежнему
бы чалился в тюряге, получил бы какое-то самообразование и раскрыл бы в себе,
скажем, таланты исследователя-рассказчика, ну и наваял нам статью об
Аустерлицком сражении. И почему бы и нет? У Андрея в загашнике было б 30 лет,
чтобы перелопатить всю теремную библиотеку, оформить новые подписки и,
действительно, стать знатоком в Наполеоновских войнах. И все же, вы готовы
вступать в дискуссию с автором, который пачками насиловал, убивал и расчленял
детей?

При Зеленском в стране приняты сегрегационные законы о языке и «коренных
народах». При Зеленском разгромлены все оппозиционные каналы. Зеленский лично
призывал русских убираться из Украины. Зеленский лично возглавил схемы по
откату на «большой стройке», в которой пилят сотни ярдов без тендеров. При
Зеленском в столице маршируют нацисты в память о дивизии СС «Галичина». Всё.
Этого достаточно. Список избыточен.

И вот имея подобный список к своей трехлетке, не пофиг ли, что он где-то
говорит? Как он это говорит? И кому он это говорит? Чикатило ведь тоже детям
мог говорить, чтобы те мыли руки перед едой. И Зеленский может рассказывать о
пользе мытья рук. Но если фигура из говна, пусть ее хоть Роден лепит, это всё
равно будет говно.

Куда интереснее сейчас движение на Донбассе. И я вам скажу, что русские очень
хорошие ученики у своих западных коллег. Особенно всего, что касается в
создании виртуальной реальности. Недаром там боятся тех же RT или «Спутник» (не
вакцина) не меньше, а то и больше, чем русских атомных подводных лодок.

Информацию и в России, и на Западе давно уже не анализируют по принципу
достоверности и фактов. Это идеалистические пережитки. Современная информация,
как и современные СМИ – это меню в ресторане. Сначала мы выбираем кухню,
которая нам по нутру – русскую или американскую. Затем достоверность заменяется
убедительностью, а факты – имитацией. И если вкусно подано – повар классный,
нет – дорабатывайте до мишленовской звездочки.

Вот тот же свежий пример с обстрелом детсада в Станице-Луганской. Ну
согласитесь, всем насрать, что там произошло на самом деле. Ты любитель русской
кухни – будешь искать там и неразбитые окна, и недопуск миссии ОБСЕ, и прочее.
Любишь американскую жратву – тебе уже очевиден звериный оскал адептов «русского
мира», которые ради провокации готовы убить детишек.

И вот смотрите, как работают шеф-повара на Западе и в России. Американские
поварята нам всё это время лепили из Украины жертву, на которую вот-вот нападет
Россия. 16 почти напали, но теперь должны 20-го. Не нападут 20-го, тогда верняк
уже 23-го. Все-таки день Советской армии и военно-морского флота, сам Ленин
велит Путину напасть в этот славный день. И пох что там на самом деле. Есть
кухня и есть ее любители, а в ресторане сейчас подают именно это блюдо.

В это же время большую жратву замутили русские поварята. У них началась
эвакуация и мобилизация в ДНР и ЛНР. И теперь уже Украина выступает не жертвой,
а агрессором. Так ли это на самом деле, насколько все это масштабно, насколько
вообще угроза большой войны реальна – это вообще вторично. Есть поклонники
русской кухни и они с удовольствием будут это потреблять.
