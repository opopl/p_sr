% vim: keymap=russian-jcukenwin
%%beginhead 
 
%%file 12_01_2019.stz.news.ua.mrpl_city.1.istoria_hatka_na_torgovoj_ulice_mariupolja
%%parent 12_01_2019
 
%%url https://mrpl.city/blogs/view/istoriya-hatka-na-torgovoj-ulitse-mariupolya
 
%%author_id burov_sergij.mariupol,news.ua.mrpl_city
%%date 
 
%%tags 
%%title История: Хатка на Торговой улице Мариуполя
 
%%endhead 
 
\subsection{История: Хатка на Торговой улице Мариуполя}
\label{sec:12_01_2019.stz.news.ua.mrpl_city.1.istoria_hatka_na_torgovoj_ulice_mariupolja}
 
\Purl{https://mrpl.city/blogs/view/istoriya-hatka-na-torgovoj-ulitse-mariupolya}
\ifcmt
 author_begin
   author_id burov_sergij.mariupol,news.ua.mrpl_city
 author_end
\fi

\ii{12_01_2019.stz.news.ua.mrpl_city.1.istoria_hatka_na_torgovoj_ulice_mariupolja.pic.1}

В глубине нашего двора на Торговой улице стояла хатка. За ней были только
дровяной сарай, отхожее место, сколоченное в незапамятные времена из досок,
почерневших от превратностей мариупольской погоды, да будка злющего Пирата –
пса с намеками на происхождение от немецкой овчарки. И хотя Пирата удерживала
прочная цепь, ходить в его владения было страшновато.

Хатка была приземистой, с крышей из черепицы-татарки. Над ней вились голуби.
Небольшой палисадник у ее входа был осенен вьющимся растением, которое взрослые
называли то \enquote{паутелью}, то \enquote{граммофончиком}. В наши дни, заглянув в \enquote{Яндекс},
можно найти и его научное название – ипомея, или фарбитис. Паутель была усеяна
нежно-розовыми и фиолетовыми цветами, напоминающими раструб граммофона. Вокруг
них вились стрекозы. Когда они садились на цветки, чтобы поживиться нектаром,
их ловила детвора за слюдяные крылышки. Развлечением для детворы были и цветки.
Их срывали, сжав пальцами за \enquote{раструб}, надували, а затем резко ими хлопали.
Этот звук тешил детей.

\textbf{Читайте также:} 

\href{https://mrpl.city/news/view/podshtanniki-dlya-papy-mira-ukraine-voennuyu-bazu-rebenku-chto-mariupoltsy-prosili-u-deda-moroza-foto}{%
Подштанники для папы, мира Украине, военную базу ребенку: что мариупольцы просили у Деда Мороза?, Яна Іванова, mrpl.city, 11.01.2019}

В хатке жила бабушкина сестра \textbf{Оришка} с \enquote{чоловиком} - дедом \textbf{Мироном} и
младшим сыном Ларкой – семнадцатилетним разбитным парнягой. Старшие дети уже
учились в вузах и наезжали в родные пенаты только на летние каникулы. Это было
интересное время. В палисадник выносили патефон, коленчатой ручкой заводили его
- на диск, обтянутый малиновой бархаткой, ставили пластинку, и окрест начинал
звучать сквозь скрип иглы голос Сергея Лемешева: \enquote{Дуня моя, Дуня,
Дуня-тонкопряха}. На одной стороне пластинки помещалась лишь половина этой
русской народной песни, поэтому пластинку переворачивали, чтобы дослушать
кумира женской половины Советского Союза.

Но самое интересное было внутри хатки. На стене висела большая картина без рамы
– предмет гордости всей семьи. На ней был изображен витязь, восседающий на
мощном белом коне. Перед всадником был камень с надписью, прочесть которую по
неграмотности малолетние зрители не могли. Копье всадника указывало на
человеческие и конские кости, разбросанные у подножия камня.

Много позже довелось узнать, что картина эта была копией известного полотна
Виктора Васнецова \enquote{Витязь на распутье}. Копию сделал старший в семье сын –
Виктор. Он занимался в кружке изобразительного искусства довоенного Дворца
пионеров у Григория Кузьмича Бендрика, талантливого художника и педагога,
последовательного приверженца реализма, воспринятого от своего учителя -
известного украинского живописца Федора Григорьевича Кричевского. Конечно, эти
подробности стали известны через много-много лет, когда Виктора уже не было в
живых. А в ту пору, когда летали стрекозы среди цветков паутели, когда играл
патефон, а верный страж Пират гремел цепью, было большим счастьем упросить
кого-нибудь из обитателей хатки показать домашнюю реликвию – аккуратный
фанерный ящичек дяди Вити. В нем помещались крохотные керамические ванночки с
акварельными красками. Видно было, что красками пользовались – в одних
ванночках он были почти не тронуты, в других – просвечивалось дно, некоторые
заполняли ванночки наполовину.

\vspace{0.5cm}
\begin{minipage}{0.9\textwidth}
\textbf{Читайте также:}

\href{https://mrpl.city/news/view/o-mariupolskih-detyah-popavshih-v-bedu-rasskazali-v-politsii-foto-plusvideo}{%
О мариупольских детях, попавших в беду, рассказали в полиции, mrpl.city, 10.01.2019}
\end{minipage}
\vspace{0.5cm}

Виктор всплывает перед внутренним взором, как смешение образа юноши с буйной
шевелюрой на пожелтевшей фотографии, прикрепленной к раме зеркала, побитой
шашелем, ящичка с красками и, конечно, картины с витязем. Иное дело Надежда.
Она отложилась в памяти вполне зримой. Суетливая, веселая, слегка сутулая – она
только что окончила французское отделение института иностранных языков и вышла
замуж за однокурсника – Исая. Голова Исая была увенчана копной иссиня-черных
жестких курчавых волос. Он был весел, как и его молодая жена. Время от времени
говорил взрослым что-то непонятное, отчего они хохотали, иногда до слез.
Малышню же щекотал, приговаривая: \enquote{Немец, перец – колбаса}. Это было последнее
мирное лето перед войной. Война коснулась своим черным крылом, оставив
трагические следы в судьбах каждого из четырех детей Оришки и Мирона…

\textbf{Виктор.} В 1940 году он был призван в Красную армию с третьего курса
Ленинградского института живописи, скульптуры и архитектуры. Часть, в которой
он служил, в первые же дни Отечественной войны вступила в тяжелые неравные бои
с вооруженным до зубов врагом. Потом было отступление, Виктор уцелел в
кровопролитных сражениях. В начале мая 1942 года войска Южного и Юго-Западного
фронтов предприняли попытку освободить Харьков. Харьковская наступательная
операция провалилась. Соединения и части Красной армии были окружены, только
малая их часть вырвалась из кольца. Около 240 тысяч солдат были пленены. Среди
них был и Виктор. Три года концентрационных лагерей, неимоверно тяжелый труд,
полуголодное существование, издевательства. Наконец в апреле 1945 года –
свобода. Это Виктор думал, что дожил до свободы. Советский фильтрационный
лагерь. Особист, проверявший бывшего рядового военно-топографического отряда,
усмотрел его вину в том, что он остался жив в провалившейся Харьковской
операции. Приговор трибунала был скор – десять лет лагерей... Колыма. Виктору
повезло – он хорошо чертил, рисовал. Через пять лет каторжных работ на золотых
приисках его перевели чертежником в строительную часть лагеря. Это-то и
сохранило ему жизнь. Он вышел на волю вскоре после кончины Сталина. Устроился
на работу в Воронежский заповедник. Изучал жизнь и поведение бобров. Написал
книгу воспоминаний о пережитом в лагерях. Безуспешно пытался ее опубликовать.
Виктор ушел из жизни, когда ему перевалило едва за шестьдесят...

\textbf{Читайте также:} 

\href{https://mrpl.city/news/view/v-mariupolskoj-shkole-poyavitsya-unikalnaya-uchebnaya-meteostantsiya}{%
В мариупольской школе появится уникальная учебная метеостанция, Олена Онєгіна, mrpl.city, 09.01.2019}

\textbf{Вера} училась в Сталино - так тогда назывался нам знакомый Донецк - на
предпоследнем курсе медицинского института. Когда началась война, студенты
прошли ускоренный курс обучения, получили дипломы и были немедленно направлены,
за малым исключением, на фронт. Всю войну она оказывала медицинскую помощь
бойцам и командирам в прифронтовых госпиталях. Кровь, смерть, крики раненых
сопровождали ее сны на протяжении всей оставшейся жизни. Где-то в круговерти
фронтов, отступлений и наступлений затерялся любимый ею человек. В сорок шестом
году старший лейтенант медицинской службы Вера было уволена в запас. Она
разыскала того, кому она была верна все эти грозовые годы. Но... он к тому
времени уже был женат, на другой...

Муж \textbf{Надежды} был мобилизован в армию на второй день после объявления войны.
Надежда вернулась домой к родителям. Еще до захвата Мариуполя врагом она успела
получить похоронку, где значилось: \enquote{Ваш муж, рядовой такой-то, пал смертью
храбрых в бою под городом Золотоноша}. В сорок втором Надежду угнали в
Германию. Она попала на завод, где делали снаряды. Ей пригодилось токарное
ремесло, приобретенное в довоенной школе фабрично-заводского обучения.
Двенадцатичасовой рабочий день... Скудный паек, состоящий из брюквы и ломтя
эрзац-хлеба... Ее спасли от дистрофии товарки-француженки. Они получали от
Красного Креста посылки с продуктами. Делились с девушкой из России, говорящей
по-французски. За годы неволи Надежда усовершенствовала язык так, как не смогла
бы, обучаясь ему еще пять лет в инязе. Она вернулась домой летом сорок пятого.
Ей нужно было каждый месяц отмечаться в комендатуре. В конце августа пошла
устраиваться на работу в школу. Ей вежливо сказали, что ее могут принять на
должность уборщицы, она согласилась. Только через год разрешили преподавать, но
только в младших классах. Прошло несколько лет, когда она вошла в класс и
поздоровалась со старшеклассниками по-французски...

\textbf{Читайте также:} 

\href{https://mrpl.city/blogs/view/okolo-3700-detej-prishli-k-nam-v-gosti-na-novogodnih-kanikulah}{%
Около 3700 детей пришли к нам в гости на новогодних каникулах, mrpl.city, 08.01.2019}

Ларка, \textbf{Илларион}, чудом избежал угона в Германию, прятался у родственников то на
Слободке, то в припортовом поселке, то в соседнем Володарске. Как только
Мариуполь был освобожден от гитлеровцев, его мобилизовали в Красную армию. Он
уцелел в кровавой мясорубке на реке Молочной, прошел со своей частью до
Кенигсберга. Но там сболтнул лишнего. Ему припомнили, что его брат – \enquote{изменник
Родины}, - всевидящее око ведомства маршала Берии знало все обо всех. Итог –
каторга на шахте в Ухте. Он бежал. Возвращался домой на крышах товарных
вагонов. На одной из станций был застрелен патрулем – его выдала черная
телогрейка, простроченная белыми нитками...

Те родители, чьи сыновья и дочери сложили головы на полях сражений, в
фашистской неволе, старикам завидовали: все четверо ваших остались живы. А
хатка? В ней побывало немало разных жильцов. Кто-то из них сломал каркас из
реек, на которых цвела паутель. Стрекозы исчезли. Сама хатка развалилась,
саман - материал недолговечный, особенно, если не вовремя штукатурить и белить
стены...

\vspace{0.5cm}
\begin{minipage}{0.9\textwidth}
\textbf{Читайте также:}

\href{https://archive.org/details/08_09_2018.sergij_burov.mrpl_city.k_240_letiu_mariupolja_perekrestok_na_torgovoj}{%
К 240-летию Мариуполя: перекресток на Торговой, Сергей Буров, mrpl.city, 08.09.2018}
\end{minipage}
\vspace{0.5cm}
