% vim: keymap=russian-jcukenwin
%%beginhead 
 
%%file 12_02_2022.tg.bilchenko_evgenia.2.poterjannaja_kniga
%%parent 12_02_2022
 
%%url https://t.me/bilchenkozhenya/6091
 
%%author_id bilchenko_evgenia
%%date 
 
%%tags bilchenko_evgenia,kniga,pamjatnik,st_peterburg
%%title БЖ. Потерянная книга - растерзанное сердце
 
%%endhead 
 
\subsection{БЖ. Потерянная книга - растерзанное сердце}
\label{sec:12_02_2022.tg.bilchenko_evgenia.2.poterjannaja_kniga}
 
\Purl{https://t.me/bilchenkozhenya/6091}
\ifcmt
 author_begin
   author_id bilchenko_evgenia
 author_end
\fi

БЖ. Потерянная книга - растерзанное сердце

В Петербурге есть памятник Потерянной Книге. Здесь, около типографии на
Литейном, в 1863 году поэт и издатель Н.А. Некрасов потерял рукопись романа
Н. Г. Чернышевского «Что делать?». Если бы рукопись не нашли, неизвестно, как бы
сложилась дальнейшая история нашей цивилизации.

\ii{12_02_2022.tg.bilchenko_evgenia.2.poterjannaja_kniga.pic.1}

Днями украинский учёный-классик мировой величины, друг Виктора Ароновича
Малахова, большой этик и педагог, Владимир Степанович Возняк писал докладную
начальству, что на его международной конференции делает \enquote{враг украинского
народа}, бывший профессор бывшей кафедры бывшего факультета НПУ Бильченко.
Зная, что так будет, я предпочла остаться по делам в федеральной миграционной
службе: там было более конструктивно. Но, несмотря на мое ФАКТИЧЕСКОЕ
отсутствие, Владимир Степанович, украинец, русин, умница, духовное существо,
все равно из-за меня пострадал. 1937? Или уже 1933? Доколе моя Украина будет
такое терпеть?

У Владимира Степановича - большое сердце. И мой доклад называется "Серце як
архетип культури\enquote{. На украинском. Это же мой родной язык, второй родной после
русского, и вы прекрасно знаете, что я изгнана из малой Родины не за язык. 

Эта потерянная книга, - как наше потерянное сердце. Мне очень стыдно от имени
петербургского певчего, русского и украинского учителя Григория Сковороды,
очень стыдно от имени Чернышевского и Некрасова, Ахматовой-Горенко и работавшей
здесь Марко-Вовчок за свою покинутую малую Родину. Доносы, объяснительные
записки, оправдания, поиск врагов народа, охота на сердце... 

Мне говорят, что я - \enquote{слишком резкая}. Да! Когда  терзают Сердце, в дело
включается Балабанов. \enquote{Хотят ли русские войны?} Нет, мы не начинаем ее первыми,
мы несём на ладони сердце. Несем и несём, а в него плюют и плюют, потом его на
части раздирают. А потом спрашивают, почему мы не \enquote{пацифисты}.  Мы - люди мира
и любви, пока издевательства над нами не превосходят мыслимую и немыслимую
черту. 

БЖ. История одного путешествия

\raggedcolumns
\begin{multicols}{2} % {
\setlength{\parindent}{0pt}

\obeycr
День был погожий – то есть, обычный, серый.
Из пустоты Вселенной планеты вынув,
Шёл человек и нёс на ладони Сердце:
Был он не то, что добрый, 
Но весь – навылет.
\smallskip
Так получилось, что пошутили боги
Вечным обманом зрения: век за вехой
Шло само Сердце, − 
Хоть и слыло безногим, –
Шло и несло в желудочке человека.
\smallskip
Переплавляясь в месиво цвета мокко,
К стокам текла брусчатка девятым валом.
Взрослые плыть ногами в грязи не могут:
Сердце смогло − ребёночком годовалым.
\smallskip
Сердце плыло вперёд, подминая клумбы;
Сердце пути разгадывало по рунам;
Именно Сердце в море вело Колумба,
В горы – гуцула, 
В пламя – Джордано Бруно,
\smallskip
В небо – пророка, 
Ангела – к аневризме,
Янку – на мост, 
СашБаша – на подоконник...
А по дороге сердце стучалось в избы
К тем, что лежали в койках, трясясь от колик.
\smallskip
Люди его, естественно, привечали:
Люди ему включали обогреватель,
Люди его поили горячим чаем,
Люди ему стелили свои кровати,
\smallskip
Люди его купали своей мочалкой;
Те, что богаче, даже давали мыло, −
Только одно условие:
Чтоб молчало,
Чтобы оно им правды не говорило.
\smallskip
Чтобы оно не выдало, как за краем
Мирной деревни жгут их рыбачьи лодки;
Как, на морозе брата их раздевая,
Новый нацист суёт ему пулю в глотку.
\smallskip
Сердце просили музыки, − а не ржавчин.
Если ж оно вело себя вдруг иначе,
Сердцу в лицо подчёркнуто громко ржали,
Чтобы не слышно было, как Сердце плачет:
\smallskip
«Лучшая в мире музыка – это шутка.
Лучшее в мире облако – это вата.
Лучшая в мире женщина – проститутка.
Лучшая в мире исповедь – это матом».
\smallskip
Сердце металось. 
Сердцу – хотелось в Мекку.
Сердце устало и, распростившись с фальшью,
Сплюнуло из желудочка человека −
Прямо им в борщ...
\smallskip
И чистым умчалось дальше.
\restorecr
\end{multicols} % }

PS. Поскольку судьба моего доклада в Дрогобыче под вопросом, а я уже написала,
с удовольствием вышлю его всем желающим просто так. Сразу говорю: он на мове, я
специально так написала, чтобы язык не забывать, и он не о политике, а о
нравственности. Поставьте плюс в комментариях. Живу сердцем. 

А пацифистом мне как-то стыдно именоваться. Ибо блаженны миротворцы, а не
пацифисты.
