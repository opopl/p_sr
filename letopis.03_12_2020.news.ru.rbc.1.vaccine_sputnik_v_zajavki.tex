% vim: keymap=russian-jcukenwin
%%beginhead 
 
%%file 03_12_2020.news.ru.rbc.1.vaccine_sputnik_v_zajavki
%%parent 03_12_2020
 
%%url https://www.rbc.ru/rbcfreenews/5fc9051c9a7947144b7afe87?
 
%%author 
%%author_id 
%%author_url 
 
%%tags 
%%title Более 50 стран отправили заявки на приобретение вакцины «Спутник V»
 
%%endhead 
 
\subsection{Более 50 стран отправили заявки на приобретение вакцины «Спутник V»}
\label{sec:03_12_2020.news.ru.rbc.1.vaccine_sputnik_v_zajavki}
\Purl{https://www.rbc.ru/rbcfreenews/5fc9051c9a7947144b7afe87?}

\index[rus]{Коронавирус!Вакцина!Спутник V (Россия), заявки, 03.12.2020}

\ifcmt
pic https://s0.rbk.ru/v6_top_pics/resized/1180xH/media/img/5/56/756070102043565.jpg
cpx Фото: Александр Земляниченко / AP
\fi

Российский фонд прямых инвестиций (РФПИ) получил заявки на приобретение 1,2
млрд доз вакцины от коронавируса «Спутник V» из более чем 50 стран. Об этом
сообщила на брифинге официальный представитель МИД России Мария Захарова,
трансляцию вел канал «Россия 24».

По словам Захаровой, РФПИ и разработчик вакцины, институт им. Н.Ф. Гамалеи,
занимаются не только расширением производства препарата для массовой вакцинации
в России.

Они также работают над «осуществлением технологического перехода к запуску
производства российского препарата на зарубежных площадках и его поставке на
международные рынки», сказала она.

Захарова напомнила, что накануне на полях 31-й специальной сессии Генассамблеи
ООН состоялась виртуальная презентация российской вакцины. По словам
представителя МИДа, препарат вызвал большой интерес.

Ранее глава РФПИ Кирилл Дмитриев
заявил,\Furl{https://www.rbc.ru/rbcfreenews/5fc7c0ef9a7947a786230b0f} что
заявки на одобрение «Спутника V» были поданы в регуляторы 40 стран.

В конце ноября Дмитриев
говорил,\Furl{https://www.rbc.ru/rbcfreenews/5fbd9c1f9a79475a1489348a?} что
фонд подал заявку на регистрацию препарата в Европейскую медицинскую
ассоциацию. Также РФПИ
договорился\Furl{https://www.rbc.ru/society/27/11/2020/5fc096fe9a794712503121db}
с ведущим индийским производителем препаратов Hetero о производстве в Индии
более 100 млн доз в год российской вакцины.


