% vim: keymap=russian-jcukenwin
%%beginhead 
 
%%file 28_08_2017.news.ua.strana.1_nasa_mars_dunes
%%parent 28_08_2017
%%url https://strana.ua/news/89382-foto-zasnezhennikh-djun-marsa-pokazali-v-nasa.html
 
%%endhead 

\subsection{В NASA показали фото заснеженных дюн Марса}

\url{https://strana.ua/news/89382-foto-zasnezhennikh-djun-marsa-pokazali-v-nasa.html}

NASA обнародовало фото с Марса, на котором запечатлены дюны "красной планеты",
покрытые снегом. Об этом сообщается на сайте космического агентства.

Фото было сделано 21 мая 2017. На нем - участок поверхности северного полушария
Марса, покрытый сухим льдом - замороженной двуокисью углерода.

\ifcmt
  img_begin 
    url https://strana.ua/img/forall/u/0/34/999.PNG
    tags science,nasa,strana.ua,mars
    width 0.5
    caption Фото: nasa.gov
  img_end
\fi

Рельеф, похожий на земные дюны, появляется вследствие того, что в северном
полушарии Марса в период аналогичный нашей весне, начинают раскалываться льды,
а пески, скрытые под ними, выходят на поверхность.

Напомним, что NASA не хватает денег для осуществления высадки человека на Марс.
Из-за этого пока невозможно назвать дату высадки американца на Красную планету.

Как сообщала "Страна", ученые-астрофизики Брауновского университета заявили о
том, что международный коллектив планетологов обнаружил воду внутри Луны.  (
\url{https://strana.ua/news/83463-planetologi-nashli-vodu-pod-poverhnostyu-luny.html})
