% vim: keymap=russian-jcukenwin
%%beginhead 
 
%%file 04_12_2020.news.ru.lenta_ru.3.vaccine_sputnik_v_stats
%%parent 04_12_2020
 
%%url https://lenta.ru/news/2020/12/04/procent/
 
%%author 
%%author_id 
%%author_url 
 
%%tags 
%%title Власти Москвы назвали процент заболевших COVID-19 после прививки «Спутником V»
 
%%endhead 
 
\subsection{Власти Москвы назвали процент заболевших COVID-19 после прививки «Спутником V»}
\label{sec:04_12_2020.news.ru.lenta_ru.3.vaccine_sputnik_v_stats}
\Purl{https://lenta.ru/news/2020/12/04/procent/}

\index[rus]{Коронавирус!Россия!Вакцина Спутник V}

\ifcmt
pic https://icdn.lenta.ru/images/2020/12/04/12/20201204124550422/pic_06cae4550dc4cc87ad1aeda173c25030.jpg
cpx Фото: Павел Львов / РИА Новости
\fi

В Москве COVID-19 заболели 1,5 процента участников пострегистрационных
исследований российской вакцины «Спутник V». Об этом сообщила вице-мэр столицы
Анастасия Ракова, передает «Интерфакс».

По словам Раковой, прививку получили более 20 тысяч человек. По данным властей,
среди них зафиксировано всего 273 заболевших.

Вице-мэр уточнила, что часть участников получила плацебо, но властям неизвестно
их точное число. По ее словам, данные указывают на эффективность препарата.

4 декабря в Москве на портале mos.ru открылась электронная запись на прививку
от коронавируса. В первую очередь вакцину получат основные группы риска —
врачи, учителя, сотрудники городских служб, социальные работники. В столице
будет действовать 70 пунктов на бесплатную прививку, они должны начать работу 5
декабря. Категории расширят после поступления новых партий вакцин.

Прививку смогут получить только те, кто прикреплен к поликлиникам Москвы. На
прививку надо будет прийти с паспортом, полисом ОМС и справкой с работы, где
указана должность. Вся процедура займет около часа: сначала пациенту проведут
10-минутный осмотр, затем около 15 минут будет размораживаться вакцина. а после
прививки еще 30 минут врачи будут наблюдать за состоянием человека.

2 декабря президент России Владимир Путин поручил начать на следующей неделе
масштабную вакцинацию от коронавируса. По его словам, это возможно благодаря
объемам произведенной в стране вакцины «Спутник V» — в ближайшее время
количество выпущенных доз достигнет двух миллионов.
