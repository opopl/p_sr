% vim: keymap=russian-jcukenwin
%%beginhead 
 
%%file 21_11_2020.fb.korchynskyi_dmytro.1.pljashku_benzini
%%parent 21_11_2020
 
%%url https://www.facebook.com/korchynskyi/posts/3502821526450021
 
%%author Корчинський, Дмитро
%%author_id korchynskyi_dmytro
%%author_url 
 
%%tags 
%%title Пляшку бензини і гноту, будьласка!
 
%%endhead 
 
\subsection{Пляшку бензини і гноту, будь-ласка!}
\label{sec:21_11_2020.fb.korchynskyi_dmytro.1.pljashku_benzini}
\Purl{https://www.facebook.com/korchynskyi/posts/3502821526450021}
\ifcmt
	author_begin
   author_id korchynskyi_dmytro
	author_end
\fi

\ifcmt
pic https://scontent.fiev6-1.fna.fbcdn.net/v/t1.0-9/126786193_3503194343079406_7210252313905021274_o.jpg?_nc_cat=104&ccb=2&_nc_sid=730e14&_nc_ohc=sj-tkXE6IKwAX-4quVo&_nc_ht=scontent.fiev6-1.fna&oh=57316323086e755f6f06deaca2da09ff&oe=5FEE8192
caption Неньку варто звикати любити!
\fi

\begin{multicols}{2}
	\obeycr
Ото ж бо й воно, а ще краще - авжеж!
Я викопував спогади з мулу
Я любив Україну МИНУЛУ,
Освітлену відблиском давніх пожеж,
Озвучену дзвону відлунням,
Освячену Тілом нетлінним,
Пропахлу болотом і полем
Відчуту коханням і болем.

Тісно в містах, але сумно у селах.
Я мріяв і пив безупину,
Я МАЙБУТНЮ любив Україну,
Збудовану з хмар і веселок, 
Яку не здолає пітьма,

Піднесену над усіма,
Прекрасну, мов Царство Небесне,
Безсмерту, відколи воскресне.
Пляшку бензини і гноту, будь-ласка!
Як ніколи, незвично, уперше

Я люблю Україну ТЕПЕРІШНЮ,
Українців з обличчями в масках,
Українок, що носять бруківку,
Київ, що дозріває узимку,
Вал вогняний на Грушевського,
Мента за щитом, а ти вріж йогО
Ланцюгом! Є петарди і бити.
Неньку варто звикати любити!
	\restorecr
\end{multicols}

\subsubsection{Commentaries}

\begin{itemize}
\item \textbf{German Glebovitsch}

\enquote{Ми маємо запалити чисте атомне сонце, над брудними болотами російського
мороку.  Провести процедуру морального очищення вогнем, чистою енергією зірок.
Звісно, що також важливо враховувати тип ґрунту, кліматичні умови, висоту
ініціювання і потужність заряду. (с)}

\item \textbf{Юрій Пендрак}
жорстко, провокативно, натхненно. ідеальна суміш. 

\item \textbf{Богдан Бановський}

Аби ж таке ж натхнення було зараз в людей! Бо всі в'ялі якісь стали як
вобла.Частину відколов нацкорпус, свободівці якісь ніякі,
пітьма з 1+1 залила голову і ветеранам і дурням, і звісно
кремляді радіють як шалені..

\item \textbf{Юрий Сычев}

Украину с праздником, седьмой годовщиной Майдана незалежности

Вот уже семь лет, как Украина стала жемчужиной демократической Европы.
Миллиарды евро инвестиций, хлынувшие в страну, преобразили ее.
Повсюду строятся новые заводы, счастливые работники которых
получают европейские зарплаты в десятки тысяч евро.
Развлекательные поездки в Европу на выходные стали нормой
жизни.

Обеспеченные пенсионеры планируют зарубежные вояжи, их больше не гнетут
копеечные пенсии и нищета, всё это осталось в тоталитарном
прошлом.

Вдоль всей Украины простираются современные автобаны и скоростные магистрали,
качеству которых позавидует и Европа. Хрустальные дворцы
современных аэропортов будто освещают изнутри улыбки
пассажиров. Уходящие ввысь города зеленеют парками,
поблескивают брусчаткой пешеходных зон, мягкой резиной
велодорожек.

Из-за атлантического пограничного вала, окружающего Украину, завистливо шипит
прозябающая Россия, регулярно получающая гуманитарную помощь от
человеколюбивой, всепрощающей Украины. Граждане Крыма кусают
локти и тайно надеются на прощение и возврат под крыло
украинской государственности.

С праздником, майдауны. Скачите выше.
\end{itemize}
