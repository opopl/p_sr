% vim: keymap=russian-jcukenwin
%%beginhead 
 
%%file 26_09_2021.fb.molchanov_jurij.1.usik_strana_pobeda
%%parent 26_09_2021
 
%%url https://www.facebook.com/george.molchanov.9/posts/4207704612675808
 
%%author_id molchanov_jurij
%%date 
 
%%tags __sep_2021.usik.pobeda.dzhoshua,bolelschik,politika,sport,strana,ukraina
%%title И все-таки замечательная у нас страна
 
%%endhead 
 
\subsection{И все-таки замечательная у нас страна}
\label{sec:26_09_2021.fb.molchanov_jurij.1.usik_strana_pobeda}
 
\Purl{https://www.facebook.com/george.molchanov.9/posts/4207704612675808}
\ifcmt
 author_begin
   author_id molchanov_jurij
 author_end
\fi

И все-таки замечательная у нас страна.

Прильнув к экранам запрещенного в Украине российского телеканала, миллионы из
нас буквально каменели в напряжении поединка, в перерывах между раундами
судорожно глотали воду/чай/виски/водку/корвалол и неистово взрывались салютом
восторга от победы.

Ночью вновь в воздухе воцарилась монополия адекватности: стар и млад, на
русском и на украинском, богатый и бедный, верующий и атеист, либерал и
коммунист - все болели за своего. За спортсмена. За победу в поединке МИРНОГО
состязания.

Как же мы отвыкли от этого.

Куда-то в далекую и тухлую помойку ушла вся дрянь пресс-конференций с
«чейкрымом», растворив в беспамятстве хайпивших ноунеймов с микрофонами.
Заглох, как старая колымага, весь скрежет зубов и про Церковь, и про язык, и
про политику.

Стыдливо заткнулись отдельные карикатуры на политиков, которые еще вчера
хайпили на церковной и «патриотической» теме. А сегодня очень смешно смотреть
как они, скорректировавшись, так натужно выдавливают из себя слова
поздравлений, будто пытаются разрешиться от трехнедельного запора. Раздраженно
прячась от собственной злобы, неактуальности и вчерашнести, доведенных до
уровня абсолют.

Понимают черти, что эта Победа настолько долгожданная и общая для всех, что
даже у самых профессиональных «патриотов» дар речи отнялся. Они своими глазами
увидели, что на фоне одного украинского флага, поднятого чемпионом мира на
такую высоту, весь гардероб их вышиванок и шароваров выглядит кучей
секонд-хендовского хлама.

Сегодня благодаря Александру Усику слово Ukraine снова пишется и произносится в
СМИ всего мира. Но НЕ в коннотации политических скандалов, разрухи, войны и
коррупции, а в контексте красивой борьбы, победы и МИРА в конце концов.

Удивительная у нас страна, в которой спортивные победы служат очистителем
международного имиджа государства, загаженного словно мухами, сонмом чиновников
и политиканов.

Страна, в которой в очередной раз миллионы людей послали в далекое пешее
путешествие своих политиков и совершенно наплевав на логотипы теле/медиаканалов
и язык комментаторов, мирно и с удовольствием смотрели праздник добра и спорта.
Еще раз: миллионы совершенно разных людей. Одновременно переживали об одном и
том же. 

Ведь можем. Ведь получается. Вопреки всему безумию последних лет, от которого
хоть на короткий миг поединка, но отвлеклось абсолютное и безоговорочное
большинство нашей страны.

А остальное…

А остальное в очередной раз подтвердило своё психиатрическое происхождение.
Тех, кого коробит гопак, флаг и украинский язык Усика – из одной палаты. Кого
крутит вера Саши, его русский язык и его отношение к политике и Церкви – из
другой. Ничего общего с жизнью в цивилизованном и развивающемся обществе это
самое остальное не имеет.

Возложив лавры своей победы на Бога, Александр стал триумфатором победы еще и
здравомыслия и элементарной человеческой любви, за которой так соскучились
абсолютно все.

\ii{26_09_2021.fb.molchanov_jurij.1.usik_strana_pobeda.cmt}
