% vim: keymap=russian-jcukenwin
%%beginhead 
 
%%file 09_05_2021.fb.buzhanskij_maksim.1.ss_galichina
%%parent 09_05_2021
 
%%url https://www.facebook.com/permalink.php?story_fbid=1941139116050558&id=100004634650264
 
%%author 
%%author_id 
%%author_url 
 
%%tags 
%%title 
 
%%endhead 

\subsection{Знакомьтесь, это Фриц, Фриц Фрейтаг. Бригадефюрер СС, генерал-майор войск СС}
\label{sec:09_05_2021.fb.buzhanskij_maksim.1.ss_galichina}
\Purl{https://www.facebook.com/permalink.php?story_fbid=1941139116050558&id=100004634650264}

\ifcmt
  pic https://scontent-frx5-1.xx.fbcdn.net/v/t1.6435-9/184665852_1941139092717227_4229942179548164136_n.jpg?_nc_cat=1&ccb=1-3&_nc_sid=8bfeb9&_nc_ohc=LckfcGQQKT0AX8kGihh&_nc_ht=scontent-frx5-1.xx&oh=5719fc9eba5f0623a815be100e5111a1&oe=60BEA6B8
\fi

Знакомьтесь, это Фриц, Фриц Фрейтаг.
Бригадефюрер СС, генерал-майор войск СС.
Именно ему выпало сомнительное счастье командовать 14 гренадерской дивизий ваффен-СС Галичина в тот единственный раз, когда её использовали не для карательных акций против партизан и мирного населения, а послали в бой.
Под Бродами.
Сказать, что местные эсэсовцы произвели на Фрица впечатление своей неистовой воинской лютью, это значит не сказать вообще ничего, и Фриц впал в полный ступор, в результате чего был отстранен от командования и отправлен прийти в себя в штаб одного из корпусов.
Пока его приводили в чувство, уцелевших эсэсовцев из "Галичины" кое как сколотили в дивизию заново, и тут выяснилось, что очередь желающих ими покомандовать, мягко говоря, не стоит.
И Фрица, не иначе как в наказание за тяжкие грехи перед Рейхом, снова прислали руководить уже знакомым ему коллективом.
Эсэсовцы жаловались, что командир им попался какой то грубый и холодный, а они видимо ждали каких то тёплых и приветливых, Фриц умолял Гиммлера выдать ему каких то других эсэсовцев, с немецкими паспортами, вместо этих, но получил Железный Крест и приказ не морочить голову.
Позднее, один из руководителей эсэсовцев рангом поменьше, Шандрук, припоминал, что якобы 24 апреля 1945 года в Галичине произошёл некоторый ребрендинг, и они решили считать себе не 14й дивизией ваффен-СС, а 1 дивизией украинской национальной армии.
Но об этом так никогда и не узнали ни в ОКВ, до 30 апреля малевавшем дивизию на картах как бесполезную, но эсэсовскую, ни сам Фриц ни сном, ни духом не заподозрил.
И продолжал командовать своими эсэсовцами из "Галичины" вплоть до 10 мая, когда предложение немедленно сдаться Союзникам стало практически девизом дивизии.
Зная немного больше о характере деятельности своих подопечных, чем рассказывают сейчас, Фриц никаких иллюзий не питал, поэтому благополучно застрелился 10 мая 1945 года, прямо в расположении дивизии СС Галичина, в деревне Санкт Андре под Тамсвегом в Австрии.
Продолжая категорически не подозревать, что его эсэсовцы якобы уже как то третью неделю не совсем эсэсовцы по их мнению, потом озвученном.
В связи с чем, просьба.
Вам когда начнут рассказывать сумасшедшие о том, что якобы дивизия ваффен-СС Галичина воевала за независимость Украины, вы сразу санитаров не зовите, а просто аккуратно поинтересуйтесь:
А командир 14й дивизии ваффен-СС Галичина, бригадефюрер СС, генерал-майор войск СС, кавалер Железного Креста Фриц Фрейтаг, он тоже воевал за независимость Украины, а?
Согласитесь, это интересный вопрос!))
