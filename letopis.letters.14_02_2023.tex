% vim: keymap=russian-jcukenwin
%%beginhead 
 
%%file letters.14_02_2023
%%parent letters
 
%%url 
 
%%author_id 
%%date 
 
%%tags 
%%title 
 
%%endhead 

Добрый день, Надежда! Слава Украине!

Читаю Ваши тексты! Знаете, читаю и думаю... читаю и думаю...  Много есть чего
сказать в ответ, тут даже всего и не уместишь за раз.  В этот раз я бы хотел
написать по поводу возвращения домой.  Надеюсь, вы не забросаете меня тапками.
Я не прошел через все эти ужасы, и поэтому смотрю на все это немножко со своей
колокольни. Как я уже писал ранее, я из Киева.

Скажем, с чего же начать... Я никогда не был в Мариуполе, даже друзей или
знакомых из Мариуполя у меня не было раньше. Поэтому все, что там происходило и
происходит, я черпаю из интернета.  Вот как то так получилось, что в Одессе
был, в Харькове был, во Львове был, в Ужгороде был, в Тернополе - был. В
Виннице был, в Николаеве был, в Херсоне - был. А вот в Мариуполе - нет. Как то
раз ехал в поезде Киев-Мариуполь, куда то там ехал, потом сошел на какой то
промежуточной станции, вот и все, что меня связывало с Мариуполем до начала
вторжения. Поэтому в смысле эмоционального воздействия всех этих поистине
ужасных и трагических событий мне безусловно легче сейчас читать и думать о
Мариуполе, - это понятно, - чем настоящим мариупольцам, которые спаслись и
выжили, пройдя через весь этот ужас и ад, люди потеряли родной дом, родной
Горол, семьи, детей, родителей, друзей. Поистине ужасная невообразимая
трагедия! И я вижу, что мариупольцам, всем вам, очень тяжело временами, если не
все время вообще, потому что тоска за утраченным домом, за разрушенным Городом,
очень сильна, память про пережитые ужасы и новости про то, что рашисты
продолжают творить, мучают и не отпускают.  Я не профессиональный психолог,
поэтому не очень знаю, как это все в научном смысле называется в психологии,
тем более, лечится это вообще или нет. Я здесь просто хочу поделиться немножко
своими мыслями, может быть, Вам это как-то пригодится, надеюсь.

Итак, насчет возвращения... Знаете, вы все очень хотите вернуться в тот
довоенный Мариуполь, до 24 февраля, хотите вообще отменить 24 февраля, как
будто его и не было. Физически хотите вернуться, в первую очередь. Физически
это сделать сейчас к сожалению невозможно.  Факты - самая упрямая вещь на
свете, и факты жестоко и неумолимо состоят в том, что ваш Город был варварски
разрушен, было убито множество невинных душ, а тем, кому посчастливилось
выжить, - вам пришлось бежать, испытав множество горя и ужасов. 24 февраля, как
бы всем нам - и мне тоже, ни хотелось, - уже невозможно отменить, потому что
это уже история, это уже прошлое, и это невозможно изменить... Время так
устроено, что есть прошлое, есть настоящее, и есть Будущее...  В настоящем
времени, а сейчас 14 февраля 2023 года от Рождества Христова, мы живем прямо
сейчас.. Вчера было 13 февраля, а завтра будет уже 15 февраля...  Зима
кончается, и скоро Весна... Время неумолимо бежит все вперед и вперед...
Прошлое уже случилось, и будет всегда таким, каким оно было, на то оно и
прошлое... А Будущее еще не наступило... На то оно и Будущее... А вообще
потрясающая штука Время, если задуматься! Стивен Кинг когда-то написал повесть
Лонгольеры, а потом был снят фильм по его повести, как раз про Время, его
природу, если смотрели... И еще... знаете, есть песня такая старая...
Призрачно всё в этом мире бушующем. Есть только миг - за него и держись.  Есть
только миг между прошлым и будущим.  Именно он называется жизнь...

Да, 24 февраля.  Эта дата уже навсегда вписана в историю Мариуполя, и Киева, и
вообще всей Украины... Это невозможно вытереть, искоренить из памяти, нельзя
вернуться назад в прошлое и переделать, как хочется, на то оно и прошлое...
Неумолимое, страшное прошлое... которое для многих, да, для всех нас,
украинцев, по-прежнему неумолимо вторгается в настоящее...  потому что война
все идет и идет, все идет и идет...  Но...  насчет возвращения...  знаете... я
уже говорил выше, что я никогда не был в Мариуполе. Поэтому у меня лично в душе
нет такого, что я мариуполец, и что у меня отняли мой дом...  Я же живу в
Киеве, и к счастью, никто у меня мой Киев пока что еще не отнял, да! Слава ЗСУ!
Но... у меня тоже есть своя позиция, позиция гостя, так сказать. И также я
украинец, понятное дело, а Мариуполь - это Украина!  это один из городов в моей
стране!  Вот в Киеве на Вокзале говорят по громкоговорителям периодически,
Кияни та Гості Столиці. Вот интересно, а есть ли такое в Мариуполе? Мариупольцы
и гости Мариуполя? Или может быть как то иначе? Ну ладно... вот... я думаю...
никогда я не был в Мариуполе, но мне вдруг раз и захотелось приехать - взять
билет, приехать на поезде, походить по улицам, сходить в ДрамТеатр на
представление, зайти в филармонию, послушать музыку, сходить на Море...
Посидеть на пляже, греясь на Солнышке, посмотреть на чаек на пирсе, покормить
голубей на площади возле ДрамТеатра. Пойти в какой нибудь музей, например в
музей Куинджи, а ведь он в свое время написал чудесную картину с луной и
Днепром! (Лунная Ночь на Днепре - очень даже киевская вещь!), даже зайти в
шахматный клуб (кроме того, что я программист, я еще играю в шахматы, и
довольно неплохо), поиграть с местными шахматистами. В общем, много чего можно
делать в Мариуполе, даже если ты не мариуполец, а всего лишь приехал в Город
Марии на поезде Киев-Мариуполь на пару дней!

Вы наверное спросите, а откуда я это все знаю, если я никогда не был в
Мариуполе?  Ну а как же, я залез в интернет, в фейсбук, например в группу
Мариуполь довоенный, и начал читать... Читаю комментарии, смотрю фото,
представляю... И не только читаю, а также записываю, сохраняю у себя... И
открываю для себя Город... Физически я далеко... Но я уже в Мариуполе...
Понимаете... Сидя в Киеве в своей квартире, я приехал в Мариуполь, хожу по его
улицам, туда-сюда... А рассказывают мне о Городе сами мариупольцы, и конечно,
сам Город... через то, как его видят мариупольцы... А никто лучше о Мариуполе
рассказать не может, чем сами мариупольцы... Понимаете...  Что я хочу сказать
этим... Город - это я уже написал под Вашим постом, в ответ тому злобному
несчастному зомби, который вообразил, что Мариуполь это якобы россия, это не
только физическое пространство само по себе, там, улицы такой то длины, или
дома такого-то размера и количества этажей... Город - это очень емкое, большое,
невообразимо большое понятие!  Город - это Мир, это Вселенная, это Книжка на
тысячи и миллионы страниц... Это живая история, это Люди! Связь поколений, труд
тысяч и миллионов людей! 

\ifcmt
  tab_begin cols=2,no_fig,center

		 pic https://scontent-ams4-1.xx.fbcdn.net/v/t39.30808-6/325736996_5755126211249847_2188810152860049525_n.jpg?_nc_cat=101&ccb=1-7&_nc_sid=5cd70e&_nc_ohc=bJsyPLtg-_gAX-kxG0F&_nc_ht=scontent-ams4-1.xx&oh=00_AfA0FF1TOKSsGwdAOubRbQGQ3lAWn3P86ykPZ4UeDAWg-w&oe=63CDC85D
		 @caption_here Дитячий майданчик бiля ЦНАПу. Липень 2020, Леся Власова, Маріуполь Довоєнний, фейсбук, 15.01.2023

		 pic https://scontent-ams4-1.xx.fbcdn.net/v/t39.30808-6/318737520_5612087545512886_3831833685038976649_n.jpg?_nc_cat=109&ccb=1-7&_nc_sid=5cd70e&_nc_ohc=8xx6j6QzX44AX9Qd8LH&_nc_ht=scontent-ams4-1.xx&oh=00_AfC0a-GE8sZsvo02Cpgu2BRcnLPFOjqiIlN9ARQv6qdnsw&oe=63D7E612
		 @caption_here Марiуполь - Мурал Мiлана - нiч, Леся Власова, Маріуполь Довоєнний, фейсбук, 10.12.2022

  tab_end
\fi

Да! Город - это также
духовное понятие!  И так получилось, что сейчас Мариуполь раздвоился...  Есть
Мариуполь в физическом пространстве - Мариуполь растерзанный, занятый мерзкими
чужаками без глаз, без песен, и без души и совести, с убитыми 100 000, с
изуверски разрушенными домами, не Мариуполь, а его тень... И есть Мариуполь
настоящий, живой, радостный, веселый!  Украинский Мариуполь!  Он есть,
поверьте, он никуда не пропал на самом деле!  Да, произошли ужасные вещи,
физически Мариуполь стал лишь тенью самого себя (насчет тени, вот вспомнился
сейчас фильм Рязанова про Андерсена), но духовно Мариуполь никуда не делся, и
он здесь, рядом, ваш любимый Город, стоит лишь только об этом задуматься и
осознать!  Он вылетел из своей физической оболочки, он в ваших душах, он в
ваших воспоминаниях, он в мариупольцях, да, живой, настоящий, украинский
Мариуполь!  и он в Интернете, а это тысячи, миллионы фотографий, текстов,
видео! Ваш Город жив, прямо сейчас и здесь, ваш родной Украинский Мариуполь!  И
он ждет тех, кто начнет открывать его секреты... Мариуполю уже более 200 лет,
конечно не возраст величавого и одновременно вечно юного Киева, но все же кое-что и немало!
Огромная Книжка на тысячи и миллионы страниц! Даже если взять последние 30 лет
независимости Украины, столько всего было в Мариуполе, стоит только сесть и
начинать читать эту Книгу под названием Город Мариуполь...

\ifcmt
  tab_begin cols=3,no_fig,center

     pic https://scontent-fra3-1.xx.fbcdn.net/v/t39.30808-6/323979606_717266183123607_604030733576487785_n.jpg?_nc_cat=104&ccb=1-7&_nc_sid=5cd70e&_nc_ohc=n926PVVK0HcAX_VdW0o&_nc_ht=scontent-fra3-1.xx&oh=00_AfAJWWohMPHjpFQ4aG6QxBLTJ4LOqKbcIPvRKod6U_Wo8w&oe=63ED11BB
		 @caption_here Прогулка по осеннему пляжу на Левом, Анатолiй Заболотнiй, Марiуполь довоєнний, фейсбук, 04.01.2023 

		 %pic https://scontent-ams4-1.xx.fbcdn.net/v/t39.30808-6/323452079_936511303926528_4810052904559851872_n.jpg?_nc_cat=110&ccb=1-7&_nc_sid=5cd70e&_nc_ohc=--jzBsjXP-MAX9Y7xJF&_nc_ht=scontent-ams4-1.xx&oh=00_AfBZVjmCjzSz3Md32BsOPwGMGgCcFTn-E8DuO17pINJiVg&oe=63CFC2E7
		 %@caption Мiсто, якого вже немає..., Володимир  Анiсiмов, Марiуполь довоєнний, фейсбук, 15.01.2023

     pic https://scontent-ams2-1.xx.fbcdn.net/v/t39.30808-6/325563751_492744552906614_1870820353248391888_n.jpg?_nc_cat=105&ccb=1-7&_nc_sid=5cd70e&_nc_ohc=Z9Ovw6mKWDIAX8wBj7_&_nc_ht=scontent-ams2-1.xx&oh=00_AfBCiivJJUO5zFI2W0ROuz_rKSJhkfVgR7A6l2605oXnGA&oe=63DA8FAA
		 @caption_here Нiч музеїв 2021, Леся Власова, Маріуполь Довоєнний, фейсбук, 14.01.2023

		 pic https://scontent-ams2-1.xx.fbcdn.net/v/t39.30808-6/325973274_858979121985225_3927845378278335084_n.jpg?_nc_cat=110&ccb=1-7&_nc_sid=5cd70e&_nc_ohc=Yd5bVV0nhc8AX_UFC3E&_nc_ht=scontent-ams2-1.xx&oh=00_AfC2FdyrSzpf-6NKXpvE1pDSzaNS21EP_jxpB0200wradQ&oe=63E41B25
		 @caption_here Сходи до Міського саду, Олег Присяжнюк, Марiуполь довоєнний, фейсбук, 20.01.2023

  tab_end
\fi

Ну что ж. А как то, что я написал, связано с желанием вернуться. А вот как...
Вы, мариупольцы, хотите вернуться физически... Все рассматриваете фото, видео,
причитаете, этого уже никогда не будет, это разрушено, то разрушено. но я думаю
так. Во-первых, никогда не говори никогда. Во-вторых, как поется в одном
известном старом фильме, говорят, - а ты не слушай, говорят, а ты не верь! Да,
физически вернуться в данный момент никак нельзя.  А вот духовно... вернуться в
свой родной Город... Прямо здесь и сейчас. Вы можете.  Это зависит только от
вас, от мариупольцев, хотите ли вы вернуться домой в первую очередь душой,
сердцем, - начинать открывать свой Город для себя заново, исследовать,
записывать, изучать. И также...  записывать свои истории...  рассказывать о
Мариуполе другим людям, всему миру... Именно о довоенном счастливом Мариуполе,
а не только об ужасах войны!  Чтобы вернуться физически в Мариуполь, чтобы
иметь силы - а сил понадобится много, очень много - духовных, дущевных сил, -
чтобы снова отстроить Мариуполь - чтобы он стал еще краше, еще лучше, - чем до
вторжения, - а этот момент обязательно наступит, поверьте - вы вернетесь домой
- я в этом совершенно уверен, - нужно сначала вернуться домой душевно, духовно.
Впустить в себя настоящий живой Мариуполь, целиком и полностью!  И как я
сказал, понадобятся силы, много сил, чтобы отстроить Город, когда придет время.
А силу, вдохновение для этого даст... Мариуполь. Да, настоящий Город Мариуполь, который сейчас
только на фото и видео и в ваших воспоминаниях.  И это мощная Сила, очень
мощная Сила, поверьте, стоит лишь только осознать это!

И знаете... вот Иван написал... а Вы поделились. Вот текст, а ниже мой комментарий...

\begingroup
\em
"
...
Але я повернусь додому.

До іншого дому. До нашого Моря-у-полі. Скільки має минути часу, щоб наші з ним
шляхи розійшлися, як розійшлися зі Сніжним? Що має статися в моєму житті, де я
маю осісти, щоб перехотіти повертатися? Сумніваюсь, що це можливо

Вкладаю собі цю просту думку, як персонажі Inception, щоб із неї проросло все
інше: Маріуполь буде українським, буде зовсім інакшим і сучасним – і я туди
повернуся
...
"
\endgroup

Вы знаете, тут мне немножко непонятно. Позиция автора непонятна немного. Я как
бы вернусь, когда Мариуполь будет украинским, будет другим, современным...
Автор как бы хочет на все готовенькое. Вот пускай ЗСУ отвоюют, - а это ведь чьи
то конкретные смерти, ранения, очень тяжелая ежедневная работа, - враг то наш
смертельный никуда не делся, и еще достаточно силен, жесток, коварен, - потом,
вот пускай там коммунальщики уберут весь мусор, потом, пускай полиция приедет,
переловит всех коллаборантов, посадит, потом, пускай это, пускай то. А потом уж
я вернусь. Вы мне пожалуйста все подготовьте, чтобы все было чики-пуки, и уж
тогда я вернусь. А так оно не работает, уж извините. Не работает. Сначала
вернуться нужно духовно, душевно, надышаться Азовским морем с фотографий, с
воспоминаний, как следует надышаться!  А потом - в свое время физически
вернешься, и по настоящему уже вдохнешь в легкие такой сладкий и приятный
соленоватый воздух моря. Я понимаю, что для тех, кто по настоящему пережил ужас
бомбардировок, кто вживую все это пережил, это очень тяжело, но это необходимо.
Надо найти в себе силы, и вернуться в живой настоящий Мариуполь, душой
вернуться, прямо здесь и сейчас, победить прежде всего в себе отчаяние и уныние
по поводу всех этих ужасов.

\ifcmt
  ig https://scontent-ams4-1.xx.fbcdn.net/v/t39.30808-6/326348585_899212244540911_6875119890412453022_n.jpg?_nc_cat=111&ccb=1-7&_nc_sid=5cd70e&_nc_ohc=W4XnIcUd25YAX9H2z14&_nc_ht=scontent-ams4-1.xx&oh=00_AfCl9ZLidWYJ1ClAQkAjoJw5kwXAgE6fYFdMvsy4EMYd9g&oe=63E4F18F
	@caption Проспект Мира, 107. Мурал "На зустріч до щастя", Олег Присяжнюк, Марiуполь довоєнний, фейсбук, 20.01.2023
  @width 0.4
  @minipage 0.4
  @wrap \parpic[r]
\fi

А иначе не будет. Вот евреи, знаете. У них
Храм разрушили в свое время. Они более 2000 лет мыкались по свету, и только в
прошлом столетии наконец-то вернулись домой. А вернулись они домой физически
только потому что Храм у них был в душе, их вера - Тора, обычаи, священные
книжки. Их родной дом у них всегда был в душе, и с Храмом в душе они пережили и
все столетия, в течение которых их притесняли и гоняли, и ужасы Холокоста, и
выиграли войны против арабов, и вообще, Израиль - это одна из мировых передовых
стран сейчас.

И далее... насчет записывания, исследования... Нужно научиться сохранять
память, как и о войне, так и о мирном времени. По-настоящему сохранять память,
серьезно, систематически, чтобы ничего не пропало...  Чтобы и через год, и
через десять лет можно было вернуться и открыть ту же страницу и прочитать тот
же текст или рассмотреть то же самое фото...  И в этом отношении, и фейсбук, и
инстаграм, и ютюб, и телеграм, и вообще интернет в целом, - - это не очень
надежные вещи для того, чтобы сохранять живую память об Мариуполе и войне.
Фейсбук периодически банит, и удаляет вообще аккаунты по своему усмотрению. А
Дуров, создатель Телеграма - тоже довольно эксцентричный человек, кто знает,
что там ему в голову может взбрести? Фейсбуку же в общем все равно, что там
именно люди пишут, ему важно, чтобы была монетизация и капала денежка с рекламы
и количества пользователей. А судьба Мариуполя, Украины в целом - это Фейсбуку
неинтересно.  Кроме того, с чисто технической стороны фейсбук может рухнуть,
обанкротиться, в целом. И все, что было там написано - просто исчезнет,
испарится.  То же самое касается телеграма, инстаграма, и вообще всей Сети в
целом, потому что Сеть, - это живой организм, и он тоже подвержен болезням,
например вирусам всяким. Хотя Сеть сама по себе довольно устойчивый механизм,
но все же никогда нельзя исключить такого сценария, когда какой-нибудь злой
гений придумает какой нибудь вирус, который заразит Сеть и начнет ее
методически разрушать, просто вытирая содержимое сайтов вчистую. Вот, например,
Наталья Дедова собирает истории мариупольцев, выкладывает их на сайте и на
фейсбуке. Фейсбук может в любую минуту забанить либо удалить ее аккаунт, а сайт
может тоже рухнуть, а все эти истории пропадут, если люди не будут их сохранять
у себя в надежном месте. Вот у Вас удалили аккаунт - Ваши тексты не пропали,
потому что Вы живы, и Ваши тексты у Вас в голове - а вот все комментарии
пропали. Я к сожалению успел только один Ваш пост из прошлого аккаунта
записать, поэтому все же кое что сохранилось.  Да, насчет глобального сбоя
Сети... Это довольно невероятный сценарий, но такое может быть, да. Был же уже
глобальный сбой фейсбука? Был. Может ли случиться глобальный сбой вообще всей
Сети? Конечно, может!  И в таком случае колоссальное количество важной нужной
правдивой информации может пропасть навсегда. Поэтому нужно уметь сохранять,
да, сохранять, записывать, чтобы память не была зависимой от таких факторов,
как политика модераторов фейсбука или же наличие вредоносных компьютерных
вирусов.  Записывать, записывать, сохранять, все это нужно делать!  Вот кто вам
мешает взять и записать себе на диски весь Мариуполь? Пускай 10 Гигабайт,
пускай 100, 1000 Гигабайт, - пускай в итоге будет 100 DVD-дисков.  Небольшой
рюкзачок в итоге с дисками.  Все рядом, все под рукой. И мирный Мариуполь, и
ужасы войны, и история прошлая, видео, тексты, все здесь - рядом.  Никакой
Цукерберг или вирус тогда не страшен, можешь делиться с кем захочешь.  Кто-то
другой тоже соберет - потом делитесь тем, у кого что есть. Весь Мариуполь - в кармане,
здесь, рядом, родной.

Да... написал я уже много, здесь наверное надо как-то уже заканчивать это письмо.
Вот недавно я ехал по Бульвару Леси Украинки, сейчас вот вспомнились такие ее строки...

\raggedcolumns
\begin{multicols}{2} % {
\setlength{\parindent}{0pt}
\begingroup
\bfseries\em\color{blue} Contra spem spero!\par
\endgroup
\smallskip
Гетьте, думи, ви хмари осінні!\par
То ж тепера весна золота!\par
Чи то так у жалю, в голосінні\par
Проминуть молодії літа?\par
\smallskip\par
Ні, я хочу крізь сльози сміятись,\par
Серед лиха співати пісні,\par
Без надії таки сподіватись,\par
Жити хочу! Геть, думи сумні!\par
\smallskip\par
Я на вбогім сумнім перелозі\par
Буду сіять барвисті квітки,\par
Буду сіять квітки на морозі,\par
Буду лить на них сльози гіркі.\par
\smallskip\par
І від сліз тих гарячих розтане\par
Та кора льодовая, міцна,\par
Може, квіти зійдуть — і настане\par
Ще й для мене весела весна.\par
\smallskip\par
Я на гору круту крем'яную\par
Буду камінь важкий підіймать\par
І, несучи вагу ту страшную,\par
Буду пісню веселу співать.\par
\smallskip\par
В довгу, темную нічку невидну\par
Не стулю ні на хвильку очей -\par
Все шукатиму зірку провідну,\par
Ясну владарку темних ночей.\par
\smallskip\par
Так! я буду крізь сльози сміятись,\par
Серед лиха співати пісні,\par
Без надії таки сподіватись,\par
Буду жити! Геть, думи сумні!\par
\end{multicols} % }

С уважением,

Иван
