% vim: keymap=russian-jcukenwin
%%beginhead 
 
%%file 02_09_2021.fb.dobrovolska_olena.1.odessa_den_goroda
%%parent 02_09_2021
 
%%url https://www.facebook.com/olena.dobrovolchik/posts/4472902762790533
 
%%author_id dobrovolska_olena
%%date 
 
%%tags den_goroda,gorod,odessa,ukraina
%%title Сьогодні фб друзі вітають Одесу з 227 річницею. Одесі 600+
 
%%endhead 
 
\subsection{Сьогодні фб друзі вітають Одесу з 227 річницею. Одесі 600+}
\label{sec:02_09_2021.fb.dobrovolska_olena.1.odessa_den_goroda}
 
\Purl{https://www.facebook.com/olena.dobrovolchik/posts/4472902762790533}
\ifcmt
 author_begin
   author_id dobrovolska_olena
 author_end
\fi

Бачу, що таки треба про це сказати.

Сьогодні фб друзі вітають Одесу з 227 річницею.

Одесі 600+.

Лінк на відео від Інституту Національної Пам'яті кину в коментарі.

Українські історики наполегливо працюють над тим, щоб спростувати міф про
"рскій город", що нав'язувався десятиліттями. Міф про те, що до наказу
російської імператриці Катерини тут була гола пустеля. Міф про те, що нічого
українського на цій землі не було, хоча мовчазні старовинні козацькі цвинтарі
гучно спростовують цю брехню.

\ifcmt
  ig https://scontent-frt3-2.xx.fbcdn.net/v/t1.6435-9/241003976_4472902606123882_4347477545200643179_n.jpg?_nc_cat=101&ccb=1-5&_nc_sid=8bfeb9&_nc_ohc=GvbKzgYlSBEAX8ceQqX&_nc_ht=scontent-frt3-2.xx&oh=fb175620500cb528c3213508e7842d72&oe=61649A88
  @width 0.4
  %@wrap \parpic[r]
  @wrap \InsertBoxR{0}
\fi

Я часто пишу про це в коментарях. Але сьогодні є привід винести це в окремий пост.

Існує ще один міф - що Одеса повністю ватна. І це теж брехня. 

Я не знаю нахіба люди так вчепились в цей ярлик, допомагаючи гебні, яка тут
варила воду, вливаючи своє бабло, що тхне трупами, в антиукраїнську пропаганду.

Одеса - вона різна.

Попри всі штампи, вона залишилась українським містом не лише територіально та
юридично. Я чую це на касах в супермаркетах у слові "дякую" у відповідь на
запитання українською. Я чую українську часто. Може тому, що я ретельно
прислухаюсь.

Тут є всі сегменти населення, які ви знайдете у всіх інших великих містах
України. Навіть у тих, де не очікуєте.

Тут є какаяразніца, яка лише хоче щоб їх не рухали і щоб мізки не боліли. З
кашею в черепних коробках. Вони йдуть за вітром і в той бік, куди сильніше дме.

Тут є старі малахольні поци, що виростили молодих малахольних поців, які носять
квіти на куликове поле і скриплять зубами у бік жовто-блакитних і ще сильніше у
бік червоно-чорних, навіть не замислюючись, від чого їх, тобто поців, вберегли
ці нарвані нацики.

Тут є саме ті нарвані люди, які люблять жартувати, посміхатись та смачно їсти.
Але так само жартома і з посмішкою цілком здатні міситись з ментами, штурмувати
мерію і ходити в суди по справах активістів. Кучка тих, хто не засцить. Як не
засцяли 3 березня 2014 року, коли орда і тітушки спробували змінити прапор у
будівлі ОДА на триколор.

Мене страшенно бісить, коли я бачу оце "ну це ж адєса", перекривлена назва мого
міста, будь-якою мовою - це миттєвий бан.

Одеса - така яка вона є, у неї дуже непростий характер, як і у кожного
українського міста. Бо Україна - це не просто, це ексклюзив, за словами
класика.
