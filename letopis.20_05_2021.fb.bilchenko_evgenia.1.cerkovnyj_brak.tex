% vim: keymap=russian-jcukenwin
%%beginhead 
 
%%file 20_05_2021.fb.bilchenko_evgenia.1.cerkovnyj_brak
%%parent 20_05_2021
 
%%url https://www.facebook.com/yevzhik/posts/3907567949278286
 
%%author 
%%author_id 
%%author_url 
 
%%tags 
%%title 
 
%%endhead 
\subsection{БЖ. Церковный брак}
\label{sec:20_05_2021.fb.bilchenko_evgenia.1.cerkovnyj_brak}
\Purl{https://www.facebook.com/yevzhik/posts/3907567949278286}

Мы умрём в один день: не спасут нас ни паспорт, ни пасма.
Мои волосы будут в твоём документе лежать.
Нерождённые дети, сложив наши тряпки в запасы,
Не пожнут наши книги: ведь мы не учили их жать.

Только сеять надежду внутри живота на рожденье.
Очень жаль, что родиться, продлив нас, не вышло у них.
Мы умрём в один день: я, не сладив с борщом, с рукодельем.
Ты, не сладив с мечом, им Георгий к победе приник.

Мы умрём по Петру: распинаемы вниз головою:
Нерадивая дщерь православная, полубуддист.
У Христа небеса - это луг с голубою травою.
Каждый миг я снимаю с креста позвоночный Сей диск.

Очень страшно, что Бог не успеет спасти нас из пекла.
Очень стыдно, что мы сомневались, Он - страсть или честь?
Мы умрём в один день: ты запомнишь, как я тебе пела...
Я запомню, как ты меня слушал... Мы живы. Мы есть.

20 мая 2021 г.

\ifcmt
  pic https://scontent-iad3-1.xx.fbcdn.net/v/t1.6435-9/187559937_3907567869278294_2927015570894383791_n.jpg?_nc_cat=106&ccb=1-3&_nc_sid=8bfeb9&_nc_ohc=j-CvqdumCxIAX-F8fDq&_nc_ht=scontent-iad3-1.xx&oh=39d753e554196bafdded9fa383dcba69&oe=60CAABA9
\fi

Anna Zhuravlova

Очень пронзительно! Только не допускайте, что Бог не успеет спасти нас из
пекла. Он все может, даже там, где человеку больше делать нечего.

Светлана Кллер

Бог - Жизнь... И осознаешь это лишь тогда, когда мимолётное, наносное теряет
свою ценность. Его невозможно ни постичь, ни обрести частично. Его не возможно
втиснуть в наши маленькие эгоистические мирки. Свет, приходя, пронзает темноту.
Старое ветшает и на смену ему приходит новое, живое, свободное. То, что у
человека уже никогда невозможно отнять. "Се творю ВСЕ новое". Жень, помнишь как
змея меняет кожу, протискиваясь между камней? Как то так и с человеками
видать))

Аркадий Веселов

Рано утром, на стержень хребта, верхний спонсор поставит пластинку, зазвучит
заводная лезгинка под которую сходят с креста )

Antonina Redkina

Потрясающе, но жизнь и вправду порой несправедлива и очень жестока.
