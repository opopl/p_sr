% vim: keymap=russian-jcukenwin
%%beginhead 
 
%%file 19_11_2021.fb.zadorozhnaja_natalia.2.skazka_sofia_2
%%parent 19_11_2021
 
%%url https://www.facebook.com/permalink.php?story_fbid=1284222425429455&id=100015251282349
 
%%author_id zadorozhnaja_natalia
%%date 
 
%%tags gorod,kiev,skazka,sofia_sobor
%%title 2-я часть вчерашней сказки
 
%%endhead 
 
\subsection{2-я часть вчерашней сказки}
\label{sec:19_11_2021.fb.zadorozhnaja_natalia.2.skazka_sofia_2}
 
\Purl{https://www.facebook.com/permalink.php?story_fbid=1284222425429455&id=100015251282349}
\ifcmt
 author_begin
   author_id zadorozhnaja_natalia
 author_end
\fi

2-я часть вчерашней сказки.

Киевские осенние вечера.

Это ли не сюжет для неё?!

\ii{19_11_2021.fb.zadorozhnaja_natalia.2.skazka_sofia_2.pic.1}

И здесь - малая толика того, что дарит этот город взглядам тех, кто любит его
особенное настроение, улочки во всякое время года и присущий только ему
особенный, немного лениво-снисходительный настрой.

А эта ось из двух невероятных соборов?!

А-а-ааа. Ну где ещё такое есть?!

Город как вдохновение, как праздник, который «всегда с нами». Как память наша.
Наша история.

Видевший многое и многих. 

Особенный!

Любимый!

От прогулок по его улочкам можно хмелеть без вина...
