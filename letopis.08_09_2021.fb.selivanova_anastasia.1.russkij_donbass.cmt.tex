% vim: keymap=russian-jcukenwin
%%beginhead 
 
%%file 08_09_2021.fb.selivanova_anastasia.1.russkij_donbass.cmt
%%parent 08_09_2021.fb.selivanova_anastasia.1.russkij_donbass
 
%%url 
 
%%author_id 
%%date 
 
%%tags 
%%title 
 
%%endhead 
\subsubsection{Коментарі}

\begin{itemize} % {
\iusr{Юлия Андриенко}
Да, красиво и концептуально. Особенно, как оформлено слово МЫ. На тот обветшалый баннер было уже больно смотреть.

\iusr{Владислав Русанов}
Старый снимали вчера утром. Я как раз шёл на лекцию.

\iusr{Алёна Касёнкина}

\ifcmt
  ig https://scontent-lga3-1.xx.fbcdn.net/v/t39.1997-6/s168x128/17633073_1652591054767295_6333333619058147328_n.png?_nc_cat=1&ccb=1-5&_nc_sid=ac3552&_nc_ohc=9l4KPN4DSSMAX8ShVEj&_nc_ht=scontent-lga3-1.xx&oh=d9d1bac685b0f4de800e9770d89dfdd5&oe=615478BC
  @width 0.1
\fi

\iusr{Андрей Соболев}

\obeycr
Нас когда-то уже разлучили,
Пуповину порвать не смогли.
Языку нас другому учили,
От своей отучали земли.
Мы не Киева против восстали,
Против киевской власти стоим.
Это наш край, угля край и стали,
Что поджечь так не терпится им.
Пуповину порвать их задача,
И залить кровью "русский вопрос"
А Донбасс превратить наш в край плача,
Не дождутся они наших слез.
Той вовек не порвать пуповины,
Власть какую там ни выбирай.
Мы не Юго-Восток Украины,
Юго-Запад России наш край!
08.09.18.
\restorecr

\iusr{Кристина Роянова}
А мне больше нравится- Россия - это Я! Россия - это Ты! Россия - это Мы!

\iusr{Артем Виницкий}
Только вряд ли нам эти надписи помогут  @igg{fbicon.smile} 

\begin{itemize} % {
\iusr{Анастасия Селиванова}
\textbf{Артем Виницкий} а какие надписи вам вообще по жизни хоть раз помогли?

\iusr{Артем Виницкий}
\textbf{Анастасия Селиванова} так в том то и дело , никакие , особенно в экономике .

\iusr{Артем Виницкий}
Я имею ввиду стать , частью РФ

\iusr{Анастасия Селиванова}

Вообще это прекрасный пост для выявления людей с негативными установками.
Приходить вот такое ко мне в позитивный пост и начинает свое внутреннее гавно
здесь разбрасывать... А я просто не готова тратить свое время и энергию на
таких токсичных... Вот и отправляю их в бан... Без сожаления

\iusr{Анастасия Селиванова}
\textbf{Артем Виницкий} вас отправить?

\iusr{Артем Виницкий}
\textbf{Анастасия Селиванова} ооо Анастасия , я реалист , и вижу к чему идёт , вы Приднестровью разюскажите о своих фантазиях  @igg{fbicon.smile} 

\iusr{Жанетта Каравацкая}
\textbf{Артем Виницкий} да оставьте вы этого угрюмого пессимиста.. Он не нашего поля ягода, это-сорняк. Известно, что написанная мечта имеет свойство осуществляться. Вспомните плакаты ВОВ. При всех недостатках-ПОБЕДИЛИ! Потому что хотели и верили.
\end{itemize} % }

\iusr{Андрей Северный}
Хорошо!

\iusr{Юлия Ялова}
Сегодня у меня спросили об этой надписи - Значит есть Донбасс не русский?))

\begin{itemize} % {
\iusr{Андрей Лимарев}
\textbf{Юлия Ялова} у кого это возник такой вопрос?  @igg{fbicon.face.eyebrow.raised} 

\iusr{Кристина Роянова}
\textbf{Юлия Ялова} Донбасс это не только ЛДНР. Донбасс это и подконтрольная часть .

\iusr{Юлия Ялова}
\textbf{Андрей Лимарев}, 

при чем здесь у кого? Я тоже подписываюсь под этим вопросом, если тебе так
важна персоналия))

\iusr{Андрей Лимарев}
\textbf{Юлия Ялова} я так и понял))

\iusr{Юлия Ялова}
\textbf{Андрей Лимарев}, ты очень "содержателен".

\iusr{Юлия Ялова}
\textbf{Кристина Роянова}, 

и? Каким образом подконтрольность территории какому-либо государству влияете на
самоопределение народа?.. если это народ, а не стадо овец. Кроме того надпись
говорит о русском, а не российском Донбассе, что не противоречит украинской
государственности как таковой и не является посягательством на нее.

Но я представила как бы выглядела надпись, например, "Русский Калининград". То
есть возникает вопрос - Кому и что пытается доказать Донбасс или кого и в чем
убедить? Экономикой нужно заниматься, социальной политикой, работой с
населением и на этих поприщах строить мотивационные плакаты и агитки, а не на
сомнительном патриотизме и противоречивых ура-лозунгах.

\end{itemize} % }

\iusr{Гайкарам Вартанян}

Национальный состав Донецкой области: 50,0\% - украинцы, 45,2\% - русские. С хера
Донбасс русский? Российский в перспективе? Да, но никак иначе. И написать бы
правду, что это всего лишь часть Донбасса. А так получается на заборе
написано... Предыдущий слоган был лучше, во всем. Идеологически правильнее. А
это пустые слова. Потраченное бабло в никуда. Даже ветхий баннер и убитый фасад
- не оправдывают его местонахождение. Лучше бы Портрет Ташкента повесили, а то
жители подзабыли героя ДНР.  @igg{fbicon.face.grinning.big.eyes}{repeat=3} 

\begin{itemize} % {
\iusr{Юлия Ялова}
\textbf{Гайкарам Вартанян}, а что там перед этим был за слоган?

\iusr{Гайкарам Вартанян}
Наш выбор - Россия!

\iusr{Юлия Ялова}
\textbf{Гайкарам Вартанян}, м-да... действительно было лучше.

\iusr{Виктория Павлова}
\textbf{Гайкарам Вартанян} 

А откуда эта статистика? Неужели из "Википедии"? Может, вспомните тех, кто в
СССР придумал давать национальность по папе? Это тоже правда. Я русская мозгом
и душой, а записана была всю жизнь украинкой, потому что папа украинец, а мама
русская.

Теперь историческая справедливость восстановлена. Какая я украинка, если толком
не знаю ни языка, ни литературы, ни истории Украины? Не потому, что не учила, а
потому что не хочу - душа не лежала и не лежит. Зато в русский язык,
литературу, культуру я влюблена. У меня русский менталитет. И таких, как я,
здесь сотни тысяч.


\iusr{Юлия Ялова}
\textbf{Виктория Павлова}, донецкие русские вдруг вспомнили, что они русские и при том большие русские, чем кто-либо другой. Бывает))

\iusr{Виктория Павлова}
\textbf{Юлия Ялова} если это камень в мой огород, то мимо. Я никогда не забывала о том, что русская.

\iusr{Юлия Ялова}
\textbf{Виктория Павлова}, это не камень и касается любого, кто говорит об угнетении его несчастного Украиной в период до 2014-го.

\iusr{Виктория Павлова}
\textbf{Юлия Ялова} свежо предание, но развивать тему желания нет. Я уж точно на угнетенную и несчастную не тяну.

\iusr{Юлия Ялова}
\textbf{Виктория Павлова}, но и комментирование было не о вас лично, а по ситуации в ДНР в отношении бездарной агитации.

\iusr{Виктория Павлова}
\textbf{Юлия Ялова}\textbf{Юлия Ялова} 

в таком случае нечего меня упоминать в комментариях. Я к вам не обращалась,
ваше мнение мне не особо интересно. Мне интересны цифры и обобщение в
комментарии товарища с армянской фамилией.


\iusr{Юлия Ялова}
\textbf{Виктория Павлова}, 

не комментируйте там, где комментирование общедоступно и никто к вам не будет
обращаться от слова вообще. Мне, например, неинтересно ваше мнение обо мне...
как и любого другого человека. Мы говорим на конкретную тему и в комментарии
Гайкарама приведены статистические цифры, нравятся они вам или нет. Но вы их
зачем-то решили разбавить своим единичным примером, на который резонно указать,
что никто вас не лишал права уточнить свою национальность и не позволять без
вашего ведома записывать вас украинкой/латышкой/еврейкой.

Почему, начиная с нулевых, Украина (кстати, как и Россия) обходит вопрос
национального состава страны - это уже другая тема. Но то, что банер этот
абсурден и его надпись весьма нелепа, это факт.

\iusr{Виктория Павлова}
\textbf{Юлия Ялова} 

серьезно? Детей до совершеннолетия спрашивали? Не смешите меня. И да, у любой
статистики есть официальный источник. Его я так и не увидела. А Украина не с
нулевых обходит национальность - вы, видно, слишком молоды. В 1994 году у меня
уже был паспорт без национальности. Дальше дискутировать с вами у меня желания
нет.


\iusr{Юлия Ялова}
\textbf{Виктория Павлова}, 

а при чем тут кем вас записали в документах школы? (с молчаливого согласия
ваших родителей). Став совершеннолетней вы уже сами заполняли документы и
идентифицировали себя самостоятельно, не так ли?

Официальный источник любой комментатор обязан предоставлять не по хотению
собеседника. Названные здесь цифры это данные переписи населения. Официальные.
Что же касается паспорта, то отсутствие там графы "национальность" тогда
явилось лишь копированием европейской практики и ее наличие считалось
пережитком советской эпохи... что в общем-то справедливо.

Вы уже дискутируете. Но стандартное "Дальше дискутировать с вами у меня желания
нет" неизбежно для тех, кому нечего сказать))

\end{itemize} % }

\end{itemize} % }
