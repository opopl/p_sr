% vim: keymap=russian-jcukenwin
%%beginhead 
 
%%file 16_01_2022.stz.news.ua.hvylya.1.anatomia_vraga
%%parent 16_01_2022
 
%%url https://hvylya.net/analytics/245483-anatomiya-vraga-chto-ukraina-nedoocenivaet-v-rossii
 
%%author_id savruckij_bogdan,news.ua.hvylya
%%date 
 
%%tags 
%%title Анатомия врага. Что Украина недооценивает в России
 
%%endhead 
\subsection{Анатомия врага. Что Украина недооценивает в России}
\label{sec:16_01_2022.stz.news.ua.hvylya.1.anatomia_vraga}

\Purl{https://hvylya.net/analytics/245483-anatomiya-vraga-chto-ukraina-nedoocenivaet-v-rossii}
\ifcmt
 author_begin
   author_id savruckij_bogdan,news.ua.hvylya
 author_end
\fi

\begin{zznagolos}
Россия является системным врагом именно украинской государственности. Причем,
врагом во всех предыдущих исторических периодах, включая современный.
\end{zznagolos}

\ii{16_01_2022.stz.news.ua.hvylya.1.anatomia_vraga.pic.1}

Современная Россия, наряду с нашей внутренней украинской, \enquote{аутентичной}
неконсолидированностью, общественным инфантилизмом, антисистемностью и
неорганизованностью, как государственная модель, безусловно, является нашим
настоящим внешним врагом. Прежде всего, Россия является системным врагом именно
украинской государственности. Причем, врагом во всех предыдущих исторических
периодах, включая современный.

Исходя из исторических фактов, имевших место в нашей истории, я считаю, что
важнейшим элементом в долгосрочной стратегии противостояния с РФ, является
правильное понимание мотивации и особенностей противника, с которым Украина
имеет дело. Кстати, в своей войне с Украиной, русские используют такое же
китайское правило - \enquote{для того, чтобы проблему понять, ее сперва нужно
правильно назвать}.

О очень упрощенном понимании характера нашей гуманитарной войны с РФ за
идентичность рекомендую прочесть в моей статье на \enquote{Хвиле}, ссылку на
которую я обязательно оставлю для Вас в конце этого текста.
