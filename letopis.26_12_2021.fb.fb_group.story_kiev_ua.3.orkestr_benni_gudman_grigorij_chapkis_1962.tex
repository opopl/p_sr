% vim: keymap=russian-jcukenwin
%%beginhead 
 
%%file 26_12_2021.fb.fb_group.story_kiev_ua.3.orkestr_benni_gudman_grigorij_chapkis_1962
%%parent 26_12_2021
 
%%url https://www.facebook.com/groups/story.kiev.ua/posts/1827145767482216
 
%%author_id fb_group.story_kiev_ua,atojev_konstantin.kiev
%%date 
 
%%tags 1962,chapkis_grigorij.horeograf.kiev,gudman_benni.muzykant.usa,kiev,kultura,muzyka,orkestr
%%title ОРКЕСТР БЕННИ ГУДМАНА В КИЕВЕ И ГРИГОРИЙ ЧАПКИС
 
%%endhead 
 
\subsection{ОРКЕСТР БЕННИ ГУДМАНА В КИЕВЕ И ГРИГОРИЙ ЧАПКИС}
\label{sec:26_12_2021.fb.fb_group.story_kiev_ua.3.orkestr_benni_gudman_grigorij_chapkis_1962}
 
\Purl{https://www.facebook.com/groups/story.kiev.ua/posts/1827145767482216}
\ifcmt
 author_begin
   author_id fb_group.story_kiev_ua,atojev_konstantin.kiev
 author_end
\fi

ОРКЕСТР БЕННИ ГУДМАНА В КИЕВЕ И ГРИГОРИЙ ЧАПКИС.

Летом 1962 г. в Киев приехал оркестр Бенни Гудмана - \enquote{короля свинга} и
легендарного кларнетиста. Джазовый тур проводился в рамках культурных обменов
между США и СССР. Американцы хотели послать оркестр Луи Армстронга, но
советским чиновникам он показался «слишком вульгарным» – «брал на трубе слишком
высокие ноты и старался рассмешить публику». В результате был выбран Бенни
Гудман, которого они сочли более серьезным музыкантом, имевшим музыкальное
образование и записавшим несколько произведений классики. Оркестр состоял из
цвета джаза тех лет. Он дал 32 концерта в Москве, Киеве, Сочи, Тбилиси,
Ташкенте и Ленинграде.

\ii{26_12_2021.fb.fb_group.story_kiev_ua.3.orkestr_benni_gudman_grigorij_chapkis_1962.pic.1}

Имя Бенни Гудмана широкой советской публике было не очень известно - во всяком
случае находилось в тени имени Гленна Миллера. Фильм \enquote{Серенада Солнечной
долины}, ставший культовым бенефисом миллеровского оркестра, был одним из самых
сильных музыкальных потрясений наших тогда еще очень молодых родителей. Тем не
менее, пишут, что уровень общественного интереса к выступлению Бенни Гудмана,
был сравним разве что с «вторжением» Битлз в США в 1964 г. Оркестр встречали и
провожали стоя, по десять раз вызывая музыкантов на бис. Люди не расходились,
пока в зале не выключали свет. Даже Н. Хрущев, не любивший джаз, посетил
выступление, но ушел после первого отделения, сославшись на то, что от этих
«джазов» у него болит голова. Не помогло даже исполнение Гудманом на кларнете
русской песни «Полюшко-поле» («Russian Patrol»). Однако 4 июля на приеме в
посольстве США по случаю Дня независимости Н. Хрущев пообщался с Бенни Гудманом
вполне дружелюбно: «Я в кармане ношу радиоприемник... и вот вдруг услышишь
джаз. Что это за музыка? Я думал, что это радиопомехи. Нет, говорят, это
музыка». 

\ii{26_12_2021.fb.fb_group.story_kiev_ua.3.orkestr_benni_gudman_grigorij_chapkis_1962.pic.2}

В Киеве музыканты выступали в недавно построенном Дворце спорта. Неслыханный
успех выступлений в Москве серьёзно насторожил власти, ведь слишком бурные
аплодисменты, трактовались идеологами как преклонение перед тлетворным Западом.
Поэтому основная часть билетов распространялась среди членов партии и
передовиков производства. Настоящие же любители джаза пробивались на концерты с
огромным трудом. Тем не менее, выступления Гудмана в Киеве прошли с грандиозным
успехом. Он хотел дать бесплатный концерт на родине отца в Белой Церкви, но
власти отказались, сославшись на условия его контракта. А он так готовился к
поездке на родину родителей, даже пригласил на последнюю репетицию оркестра
перед поездкой в СССР ансамбль танца УССР под управлением П. Вирского, весь
апрель, гастролировавший в США. Среди танцоров был и молодой Григорий Чапкис,
который станцевал для музыкантов Гудмана «ползунец».

\ii{26_12_2021.fb.fb_group.story_kiev_ua.3.orkestr_benni_gudman_grigorij_chapkis_1962.pic.3}

Для сувенирных подарков компания «Селмер» сделала для тура Бенни Гудмана в СССР
металлические значки на булавке, с изображением двух рук, играющих на кларнете
и надписи на русском языке: «БЕННИ ГУДМАН, США, 1962». Эти значки еще долго
были неким символом, объединявших любителей джаза тех лет, когда не было ни
социальных сетей, ни смартфонов, ни плейеров, а джаз звучал в основном в душах
его поклонников.

\ii{26_12_2021.fb.fb_group.story_kiev_ua.3.orkestr_benni_gudman_grigorij_chapkis_1962.pic.4}

\ii{26_12_2021.fb.fb_group.story_kiev_ua.3.orkestr_benni_gudman_grigorij_chapkis_1962.cmt}
