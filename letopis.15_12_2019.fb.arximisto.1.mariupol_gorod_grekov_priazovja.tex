%%beginhead 
 
%%file 15_12_2019.fb.arximisto.1.mariupol_gorod_grekov_priazovja
%%parent 15_12_2019
 
%%url https://www.facebook.com/arximisto/posts/pfbid02qmyPc9JyME4MWPbtaiqxz8kvQtnvq8m6FnpmDW43FzcxRq9KHSQRniXwdCNxBdBbl
 
%%author_id arximisto
%%date 15_12_2019
 
%%tags 
%%title Мариуполь - город греков Приазовья!
 
%%endhead 

\subsection{Мариуполь - город греков Приазовья!}
\label{sec:15_12_2019.fb.arximisto.1.mariupol_gorod_grekov_priazovja}

\Purl{https://www.facebook.com/arximisto/posts/pfbid02qmyPc9JyME4MWPbtaiqxz8kvQtnvq8m6FnpmDW43FzcxRq9KHSQRniXwdCNxBdBbl}
\ifcmt
 author_begin
   author_id arximisto
 author_end
\fi

Впервые публичное предложение о возвращении Мариуполю его исторического имени
прозвучало в книге отзывов на выставке архитекторов в 1987 году. А затем было
опубликовано в \enquote{Приазовском рабочем}, как следует из интервью
журналиста газеты Виктора Сухорукова на нашем youtube канале (см. здесь
\href{https://archive.org/details/video.29_11_2019.mariupol1989.viktor_suhorukov_on_mariupol_1989}{%
Viktor Sukhorukov on Mariupol 1989, Mariupol1989, youtube, 29.11.2019}).

Мы посмотрели в книгу отзывов нашей выставки в Центре А. Куинджи - что пишут
посетители 30 лет спустя? Три самых примечательных оценки ниже:

И. Балабанова благодарит за \enquote{выставку памяти} - \enquote{Мариуполь - город греков Приазовья!}

Перекличка поколений - в отзыве Софии Москвиной: \enquote{на одном из фотоэкспонатов}
она \enquote{увидела своего отца Москвина Е.А. - старшего пионерского вожатого средней
школы № 1 Жданова}! Жаль, что она не оставила своих контактов! Возможно, речь
идет о фотографиях с советского митинга на площади возле \enquote{1000 мелочей} - где
школьники стоят по лозунгом \enquote{Мы - за Мариуполь!}

Наконец, Владимир и Светлана Степановы благодарят самого Виктора Сухорукова -
\enquote{Ваш вклад в возвращение городу исторического имени останется в истории
Мариуполя} 🙂

%Виктора Сухорукова на нашем youtube канале (см. здесь \href{https://is.gd/pcAC2f}{}). 
