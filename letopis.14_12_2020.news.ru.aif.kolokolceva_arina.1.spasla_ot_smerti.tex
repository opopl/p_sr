% vim: keymap=russian-jcukenwin
%%beginhead 
 
%%file 14_12_2020.news.ru.aif.kolokolceva_arina.1.spasla_ot_smerti
%%parent 14_12_2020
 
%%url https://altai.aif.ru/incidents/spasla_ot_smerti_na_altae_sobaka_nashla_zabludivshegosya_v_sugrobah_starika
 
%%author Колокольцева, Арина
%%author_id kolokolceva_arina
%%author_url 
 
%%tags sobaka,smert,altai
%%title Спасла от смерти. На Алтае собака нашла заблудившегося в сугробах старика
 
%%endhead 
 
\subsection{Спасла от смерти. На Алтае собака нашла заблудившегося в сугробах старика}
\label{sec:14_12_2020.news.ru.aif.kolokolceva_arina.1.spasla_ot_smerti}
\Purl{https://altai.aif.ru/incidents/spasla_ot_smerti_na_altae_sobaka_nashla_zabludivshegosya_v_sugrobah_starika}
\ifcmt
	author_begin
   author_id kolokolceva_arina
	author_end
\fi

\ifcmt
pic https://aif-s3.aif.ru/images/022/210/4480cceaf16b31b17e1068cea05be1b7.jpg
\fi

\index[rus]{Алтайский край!Собака спасла от смерти, 14.12.2020}
\index[rus]{Собака!Спасла от смерти, Алтайский край, 14.12.2020}

В Каменском районе Алтайского края героиней декабря стала овчарка Ксюша.
Благодаря ей удалось спасти от верной гибели заблудившегося в сугробах старика.

\subsubsection{Прогулка на ночь глядя}

\textbf{84-летний Виктор Филипов} вечером 8 декабря отворил дверь и пошел по селу,
опираясь на палочку.

«У папы развивается деменция, он уже три раза уходил, но недалеко, соседи
возвращали», - рассказывает \textbf{дочь Любовь Романец}.

\ifcmt
pic https://static1-repo.aif.ru/1/e3/1677833/c/85689aefae0666dfec77b11b19e13517.jpg
caption Мужчина уже неоднократно уходил из дома. Фото: Пресс-служба ГУ МВД по Алтайскому краю 
\fi

60-летняя женщина специально приехала в глухое село Столбово, чтобы ухаживать
за стареющим отцом, и сняла дом рядом.

«В тот вечер я истопила печку, накормила его, пошла домой, думая, что отец
ляжет спать, но он решил отправиться гулять на ночь глядя… Утром прихожу, изба
пустая, я в панике - к соседям, в сельсовет», - вспоминает Любовь Романец.

Когда один из местных водителей объехал все село и не нашел пропавшего, были
вызваны полицейские, МЧС. На поиски прибыл и \textbf{кинолог Константин Байкин} с
овчаркой Ксюшей. У Кости в этот день был день рождения, и его умная собака
сделала самый лучший подарок - спасла человека.

\subsubsection{Первое серьезное дело}

\ifcmt
pic https://static1-repo.aif.ru/1/9c/1675843/c/b1e6b1cb451f7c475f8b3278a809cb25.jpg
caption Следы привели к речке Чертенюхе. Фото: Пресс-служба ГУ МВД по Алтайскому краю  
\fi

У Ксюши это было первое серьезное дело, но собака не растерялась, понюхала
рубаху пропавшего и быстро взяла след, дойдя до центра села. Там на снегу
лежала оброненная трость пенсионера. У дочери Виктора Филипова в этот момент
заколотилось сердце.

«Люди рассказали, что в центре села видели старика в десять вечера, я уже
приготовилась к самому худшему, - рассказывает женщина,  - столько времени
прошло, 18 часов …»

Собака сделала несколько кругов - видимо, так и плутал пропавший старик, а
потом в растерянности остановилась. После этого кинолог дал понюхать Ксюше
носок пожилого мужчины. И она тут же рванула на окраину села, а дальше в
сугробы, к речке Чертенюхе.

\ifcmt
pic https://static1-repo.aif.ru/1/a4/1677812/c/8909fe4891d0859d4b658512ec5d03a2.png 
caption Ксюша сделала кинологу самый лучший подарок на день рождения. Фото: личный архив/ Константин Байкин
\fi

«Когда я услышала крик: «Нашел!», побежала, думая, уж неживой он там, и вижу,
отец стоит задумчиво, усы обледенели. На просьбы идти не реагировал, пришлось
силой его с этого места уводить. Видимо, не зря так речка называется - будто
черти «путают»…», - эмоционально рассказывает Любовь Романец.

Старик, который, не обращая внимания на суету вокруг себя, еще и с расстегнутой
курткой, стоял среди сугробов. Его аккуратно завернули в одеяло и перенесли
через пеший мост, там уже дежурила «Скорая».

«Представьте себе, за 40 минут ожидания в приемном покое папины ноги, которые
до этого были белые, как мел, думала – все, отморозил, оттаяли. И врач не нашел
нужным оставлять его в больнице. Проведя столько времени на холоде, не простыл,
даже насморка не было», - с радостным  удивлением рассказывает дочь Виктора
Филипова.

Благодарная женщина не устает восхищаться людьми, которые помогали ей в поисках
родственника: соседями, полицейскими, спасателями, кинологом. Ну и, конечно,
талантливой и доброй овчаркой Ксюшей.
