% vim: keymap=russian-jcukenwin
%%beginhead 
 
%%file 24_07_2019.fb.lesev_igor.1.nacidea
%%parent 24_07_2019
 
%%url https://www.facebook.com/permalink.php?story_fbid=2572562629441491&id=100000633379839
 
%%author_id lesev_igor
%%date 
 
%%tags identichnost',nacia,nacidea,narod,obschestvo,strana,ukraina
%%title О национальной идее
 
%%endhead 
 
\subsection{О национальной идее}
\label{sec:24_07_2019.fb.lesev_igor.1.nacidea}
 
\Purl{https://www.facebook.com/permalink.php?story_fbid=2572562629441491&id=100000633379839}
\ifcmt
 author_begin
   author_id lesev_igor
 author_end
\fi

О национальной идее

Нацидея. Вопрос концептуальный для новоиспеченной партии-мейджора. А ведь
именно сейчас ложка поспевает к обеду.

Вот давайте честно. Появление Украины в таком виде в 1991 году – это была
случайность. Не результат борьбы храбрых петлюровцев. И не итог беготни по
лесам бандеровцев. Собрались москвичи на площади, разогнали ГКЧП и вдруг
появилась наша страна.

\ifcmt
  ig https://scontent-frt3-1.xx.fbcdn.net/v/t1.6435-9/67330583_2572562572774830_2684505515215552512_n.jpg?_nc_cat=108&ccb=1-5&_nc_sid=730e14&_nc_ohc=E1Zg9tyqmFcAX-IrWuy&_nc_ht=scontent-frt3-1.xx&oh=d6bd8effbe4f2dcfade357c577eee41b&oe=61B7F2B0
  @width 0.4
  %@wrap \parpic[r]
  @wrap \InsertBoxR{0}
\fi

Может кому обидно, но историческая хронология почти всегда обидна. Великие
битвы выигрывались на курьезах, а великие свершения появлялись по прихоти
случайных людей.

Вся современная история Украины – это борьба двух несовместимостей. Интеграции
в большой ру-проект. И бегство от большого ру-проекта. И выбор такой же, как в
сексуальной ориентации. Один раз, как Леша Гончаренко, еще можно переметнуться.
Но обобщенно, выбор не велик. Или ты трахаешь в жопу, или тебя. Абсолютно
гамлетовская дилемма.

И вот то, что сейчас произошло – это бунт. Бунт срединной Украины. Поверхностно
у нас видят отрыжку против старой элиты. Но опять же, это именно поверхностное
видение. Потому что когда мы подходим к персоналиям, то тут же начинается
сортировка. Восток- Запад. Восток – Запад. Вот их всех слили в унитаз.

По старинке живут только два региона. Донбасс, вернее, его юа-осколок. И
Галичина. Именно там остались наши местные Ланкастеры и Йорки. Наши Алая и
Белая розы. У местной пацанвы на дворе средневековье. С ними можно
разговаривать. Они водят машины. Пользуются банковскими картами. Вот все как у
шизофреников до приступа социальной дисфункции. Но только разговори, и
начинается иррациональный поток бреда.

Эти люди умеют говорить, убеждать. Но не умеют строить. Но на финише все равно
у них выходит кукушкино гнездо.

Формально более созидательны восточные. Они всегда срали на гуманитарку и
делали акцент на экономику. Заводы, экономические связи, добавленная
стоимость... Но по итогу у них все равно получается маленькая Россия, что
неприемлемо для сумасшедших с другой палаты.

Западные сильны в гуманитарке. Правильная история, правильная мова, уже и
правильная религия. Они всегда более дерзки и активны, потому что продают не
реальные ценности, а виртуальные. Но их конструкция отвратительна в деталях.
Они затрахают даже продавца маркета с окладом в 8 тысяч.

Это две сраные крайности. Одни строят НеУкраину. Другие Украину для
неукраинцев.

И вот теперь их обнулили. И этот протест народный. Вот это важно. Протест
самосохранения срединной Украины, которая расширилась до Харькова и Запорожья
на востоке и Волыни с Буковиной и Закарпатьем на западе.

А наверху теперь пацанва случайная. Повага и бабло этим ребятам не чужды. Ну
как и всем нам, давайте уж прямо. И чтобы чепчики летели. И чтобы софиты. Ну
весь набор.

Но в истории так бывает, что случайные люди создают интересные образования. Вот
как Бостонское чаепитие. Вопрос акциза на сраный чай, туда-сюда, чуть мордобоя
и нате – США.

Еще раз. У нас есть шанс построить Новую и я бы даже сказал, Первую полноценнуб
страну. Не АнтиРоссию и не МалоРоссию. При этом, временной отрезок очень
короткий. Нужны срочные институциональные изменения. Изменения, которые
вовлекут республиканские и региональные элиты в этот процесс. Которые покажут
им свой интерес. Бабловый для элит. И комфортный для масс.

А для этого нужно определить зону комфорта. А она не в плоскости налогов, дорог
или сервиса. Нет, это тоже очень важно, но это уже технические моменты. Как
ремонт в квартире. Согласитесь, между убитой квартирой и шикарной квартирой мы
выберем всегда свою. Украина для большинства ее граждан до конца так и не была
никогда своей.

Какой может быть национальной идеей Новой Украины? Короткий путь – бабы и
футбол. Путь чуть длиннее, но более продуманный – максимальное вовлечение
БОЛЬШИНСТВА граждан страны в управление/получение прибыли от общего пирога.
Украина для БОЛЬШИНСТВА – это шанс для субъективизации нашей территории. С
места жительства и места заработков, а также с места фантастического
представления о том, какой должна быть эта территория, на субъект
международного права, за который будут рвать жопу сограждане одного большого
общежития.

Получится – хорошо. Нет, в мире даже не десятки, а сотни потерянных и пропащих
образований.

\ii{24_07_2019.fb.lesev_igor.1.nacidea.cmt}
