% vim: keymap=russian-jcukenwin
%%beginhead 
 
%%file slova.vyshyvanka
%%parent slova
 
%%url 
 
%%author 
%%author_id 
%%author_url 
 
%%tags 
%%title 
 
%%endhead 
\chapter{Вышиванка}
\label{sec:slova.vyshyvanka}

%%%cit
%%%cit_pic
%%%cit_text
Трудно было не заметить, как старательно изображал из себя ультрапатриота глава
федерации футбола Павелко. И трудно было с этого не поржать, понимая, что
мутный постсоветский бизнесмен Павелко космически далек от всей этой
бандеровщины. Его настоящее отношение к ней – что-то между равнодушием и
презрением.  И он такой совершенно не один. Почему же все эти дорвавшиеся до
бюджета и схем бизнесюки так старательно строят из себя правоверных
националистов? А ничего личного – просто бизнес. Так сложилось, что «борцы с
коррупцией» и упоротые «патриоты» в Украине это одни и те же лица. При этом
именно эта узкая прослойка и формирует здесь «общественное мнение» - остальным
все пофиг.  Поэтому, сев на потоки, так важно одеть \emph{вышиванку}. Ведь когда
ворует Курченко – то это «нельзя терпеть, все на Майдан». А когда ворует
Порошенко – «це потрібно для захисту держави»
%%%cit_comment
%%%cit_title
\citTitle{Сев на украинские потоки, так важно одеть вышиванку},
Вячеслав Чечило, strana.ua, 13.06.2021
%%%endcit



%%%cit
%%%cit_head
%%%cit_pic
%%%cit_text
\emph{Вышиванки}. Футболки \verb|#FreeСавченко|. \emph{Вышиванки}. Футболки
\verb|#FreeСенцов|.  \emph{Вышиванки}. Футболки сборной Украины по футболу.  Таки
прогресс, конечно.  На заседании Кабмина.  А можно как-то без этих всех ряжений
просто взять и снизить тарифы, например?!  Я понимаю, что это скучно и не так
интересно, но за нашу сборную мы болеть можем и без Кабмина, а вот с тарифами
без них - никак
%%%cit_comment
%%%cit_title
\citTitle{За нашу футбольную сборную болеть мы можем и без ряженого Кабмина / Лента соцсетей / Страна}, 
Елена Лешенко, strana.ua, 03.07.2021
%%%endcit
