% vim: keymap=russian-jcukenwin
%%beginhead 
 
%%file 14_10_2021.fb.fb_group.respublika_lnr.3.lugansk_zahoronenie_2014
%%parent 14_10_2021
 
%%url https://www.facebook.com/groups/respublikalnr/posts/901354410500351
 
%%author_id fb_group.respublika_lnr
%%date 
 
%%tags 2014,donbass,lnr,lugansh,ukraina,vojna,zahoronenie
%%title ГАЗЕТА "РЕСПУБЛИКА" (№41, 2021г). НАША ОБЩАЯ БОЛЬ. НЕ ЗАБУДЕМ! НЕ ПРОСТИМ!
 
%%endhead 
 
\subsection{ГАЗЕТА \enquote{РЕСПУБЛИКА} (№41, 2021г). НАША ОБЩАЯ БОЛЬ. НЕ ЗАБУДЕМ! НЕ ПРОСТИМ!}
\label{sec:14_10_2021.fb.fb_group.respublika_lnr.3.lugansk_zahoronenie_2014}
 
\Purl{https://www.facebook.com/groups/respublikalnr/posts/901354410500351}
\ifcmt
 author_begin
   author_id fb_group.respublika_lnr
 author_end
\fi

ГАЗЕТА "РЕСПУБЛИКА" (№41, 2021г).

НАША ОБЩАЯ БОЛЬ. НЕ ЗАБУДЕМ! НЕ ПРОСТИМ!

Начались работы по вскрытию самого крупного стихийного захоронения в ЛНР

20 сентября 2016 года на окраине Луганска состоялась церемония открытия
мемориального комплекса, созданного на месте массового стихийного захоронения
луганчан, погибших и умерших летом 2014 года. Появление этого места на карте
Луганска – одна из драматичных страниц истории города. 

\ifcmt
  ig https://scontent-lga3-1.xx.fbcdn.net/v/t39.30808-6/245418196_238679624978811_3564244270801371157_n.jpg?_nc_cat=102&ccb=1-5&_nc_sid=825194&_nc_ohc=QRe87D6b6UgAX9UCeQR&_nc_ht=scontent-lga3-1.xx&oh=00b3834e16b5a7320948fd5b54f0d0f6&oe=6172252A
  @width 0.4
  %@wrap \parpic[r]
  @wrap \InsertBoxR{0}
\fi

Когда в августе 2014-го Луганск был полностью окружен украинскими карательными
войсками, началась настоящая гуманитарная катастрофа. Отсутствие воды и
электроэнергии, а соответственно и связи, привели к тому, что городской морг
был переполнен телами погибших под вражеским огнем горожан. Многие останки уже
невозможно было опознать. Тогда и было принято решение обустроить массовое
захоронение. 

Такая вынужденная мера была обусловлена еще и тем, что городские кладбища
постоянно обстреливались. В то время поле за ныне Луганской республиканской
клинической больницей было наиболее безопасным – снаряды просто перелетали
через него.

Когда бои за Луганск закончились и город шел по пути относительно мирного
становления, на массовое стихийное захоронение обратили внимание.

Началась работа по розыску родственников погибших. Общественникам удалось
установить 119 имен. Также территория была приведена в порядок: не по
славянским обычаям неуважительно относиться к умершим. Однако по сегодняшний
день остается неизвестным достоверное количество покоящихся там мирных
жителей. 

Спустя пять лет после открытия мемориального комплекса «Не забудем! Не
простим!», 8 октября специалисты Межведомственной рабочей группы по розыску
захоронений жертв украинской агрессии, их идентификации и увековечению памяти
приступили к вскрытию самого крупного массового стихийного захоронения в
Республике.

– Начат пятый этап работы. Работать здесь будем, по возможности, до конца,
чтобы поднять все останки, или до того момента, пока позволят, конечно,
погодные условия, – сказала руководитель рабочей группы, Первый заместитель
Министра иностранных дел ЛНР Анна Сорока. – Перед началом работ здесь
общественники провели большое исследование, чтобы иметь юридическое,
нравственное и моральное право вскрыть это захоронение, достойно поднять
останки погибших жителей города Луганска, совершить все необходимые
процессуальные мероприятия, собрать ДНК-материал, провести ДНК-исследования и
создать ДНК-паспорта. В конечном итоге нам нужно провести идентификацию,
обнаружить родных и близких всех захороненных, которые здесь лежат. 

Анна Сорока уточнила, что в дальнейшем на месте захоронения планируется создать
большой мемориальный комплекс в память обо всех жертвах агрессии Украины на
Луганщине.

Секретарь Межведомственной рабочей группы Сергей Белов отметил, что извлечение
останков проводится при участии следственной группы, судмедэкспертов МВД ЛНР, а
также представителей Генеральной прокуратуры Республики.

– В ходе работ по вскрытию массового захоронения жертв украинской агрессии,
расположенного на территории города Луганска, в районе поселка Видный извлечены
останки 17 мирных жителей, – сообщил Сергей Белов. – Осмотр судмедэкспертизы
показал, что останки принадлежат мирным жителям.

Оксана ЧИГРИНА, фото автора

ГАЗЕТА "РЕСПУБЛИКА" (№41, 2021г).

\begin{verbatim}
	#газета #республика #эксгумация_тел_жертв_киевской_агрессии #Луганск
\end{verbatim}
