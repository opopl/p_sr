% vim: keymap=russian-jcukenwin
%%beginhead 
 
%%file 24_10_2020.news.ua.gazeta.1.mike_jogansen_shoot
%%parent 24_10_2020
%%url https://gazeta.ua/articles/history/_poeta-zvinuvatili-u-pidgotovci-zamahiv-na-vladu/992724
 
%%endhead 

\subsection{Поета звинуватили у підготовці замахів на владу}

\url{https://gazeta.ua/articles/history/_poeta-zvinuvatili-u-pidgotovci-zamahiv-na-vladu/992724}

Українського поета 40-річного Майка Йогансена засудили до розстрілу на
закритому засіданні Військової колегії Верховного суду СРСР 26 жовтня 1937
року. Вирок виконали за добу у київській в'язниці НКВД.

Чоловік походив із Харкова. Складати вірші українською розпочав у період
радянсько-української війни 1918 року - був вражений жорстоким погромом Києва
військами Михайла Муравйова, а згодом політикою лідера Білого руху Антона
Денікіна, оголосив себе марксистом.

Після встановлення радянської влади працював у Всеукраїнському
фотокіноуправлінні. Разом із генералом УНР Юрком Тютюнником писав сценарій
фільму "Звенигора". У поезії цікавився революційною тематикою, згодом
проблемами буденного життя.

Товаришував із багатьма харківськими письменниками, з якими організував спілку
пролетарських письменників "Гарт", а згодом долучився до створення Вільної
академії пролетарської літератури - ВАПЛІТЕ, що з часом зазнала гострих
цькувань через творчість Миколи Хвильвого, а згодом була ліквідована.

ЧИТАЙТЕ ТАКОЖ: Микола Хвильовий застрелився у кабінеті

Майк Йогансен, разом із деякими колишніми членами ВПАЛІТЕ, організував
Техно-Мистецьку групу А. Організацію критикували за надмірне захоплення
техніцизмом - визнання науки і техніки, а не пролетаріату та селянства, рушієм
суспільного прогресу. 1934 року вступив до Спілки радянських письменників
України.

Його арештували влітку 1937 року у власній квартирі будинка "Слово",
побудованого на кошти кооперативу літераторів. НКВД закидали поетові розмови із
колегами на політичні теми під час виїздів на полювання. Підставою для
звинувачень стали доноси знайомих та друзів. На допитах сам нікого не викрив,
утім заявив, що масові арешти українських письменників є результатами безсилля
і розгубленості радянських керівників.

За це його звинуватили в участі у антирадянській націоналістичній організації,
що мала на меті повалення радянської влади. Наче з 1932 року сам вербував людей
і готувався виконувати теракти проти партійних діячів. Призначили вищу міру
покарання, а після розстрілу конфіскували усе особисте майно.

Реабілітації добилася дружина Валентина Ніколаєва ще за часів Радянського
Союзу.

20 жовтня 1930 року в Москві розстріляли генерал-хорунжого армії УНР та
командарма Другого зимового походу Юрія Тютюнника.

Ще 1923 року більшовицькі каральні органи обманом виманили його з Польщі в
Радянську Україну. Закликали очолити всеукраїнське повстання через підставну
організацію, на що не відмовив. Враховуючи його авторитет серед українців та
українських емігрантів, вирішили не вбивати, а використовувати.

