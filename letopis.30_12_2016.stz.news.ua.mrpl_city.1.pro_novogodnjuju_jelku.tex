% vim: keymap=russian-jcukenwin
%%beginhead 
 
%%file 30_12_2016.stz.news.ua.mrpl_city.1.pro_novogodnjuju_jelku
%%parent 30_12_2016
 
%%url https://mrpl.city/blogs/view/pro-novogodnyuyu-elku
 
%%author_id burov_sergij.mariupol,news.ua.mrpl_city
%%date 
 
%%tags 
%%title Про новогоднюю елку
 
%%endhead 
 
\subsection{Про новогоднюю елку}
\label{sec:30_12_2016.stz.news.ua.mrpl_city.1.pro_novogodnjuju_jelku}
 
\Purl{https://mrpl.city/blogs/view/pro-novogodnyuyu-elku}
\ifcmt
 author_begin
   author_id burov_sergij.mariupol,news.ua.mrpl_city
 author_end
\fi

\ii{30_12_2016.stz.news.ua.mrpl_city.1.pro_novogodnjuju_jelku.pic.1}

Всегда ли в Мариуполе устанавливали елки? Начнем, пожалуй, с запорожских
казаков, обитателей крепостного укрепления, находившегося  вблизи нынешней
площади Освобождения. Для них главным зимним праздником было  Рождество
Христово. А елка или сосна? Где ее возьмешь? До ближайшего соснового бора, а
тем более ельника, - сотни верст от устья Кальмиуса. Кроме того, они вряд ли
ведали, что на Рождество должно присутствовать хвойное дерево. И поэтому можно
твердо сказать, что казаки  елку как атрибут новогодних праздников не знали.

\ii{30_12_2016.stz.news.ua.mrpl_city.1.pro_novogodnjuju_jelku.pic.2}

Для греков, переселившихся из Крыма в Северное Приазовье, также было чуждо
использование хвойных деревьев для празднования чего-либо. Понадобилось немало
десятилетий, чтобы лесные красавицы, убранные шарами, флажками, серпантином,
\enquote{зашли} в жилища мариупольцев.

\ii{30_12_2016.stz.news.ua.mrpl_city.1.pro_novogodnjuju_jelku.pic.3}

Обратимся к документам, имеющим отношение к нашему городу. В Отчете
Мариупольской мужской гимназии за 1893 год, в разделе особо замечательных
событий, находим: \enquote{При участии почетного попечителя Хараджаева и почетной
попечительницы Чабаненко 4 января была устроена елка для учащихся мужской
гимназии}. Подшивки газеты \enquote{Мариупольская жизнь} начала ХХ века. Там немало
объявлений, касающихся нашей темы. Эти объявления печатались, как правило, на
первых полосах  и повторялись в нескольких номерах  подряд. \enquote{Получен большой
выбор елочных украшений. Наборы в 1 р., 1 р.50 к., 2, 3, 5 и дороже. Елки от 20
к. и дороже. Магазин П.Чабаненко. Торговая улица, угол Итальянской, собственный
дом}. \enquote{Елки получены в магазине бр. Мелековых. Георгиевская улица, собственный
дом}. \enquote{Елки разных размеров получены. Бондарная ул. №3 (где бондарная
мастерская Богачева). Цены умеренные}. \enquote{Елка и танцевальный вечер. В первый
день Рождества мариупольское общество попечения о детях устраивает в городской
управе елку для детей и танцевальный вечер для взрослых. Начало елки в 4 часа и
до 8 ч. вечера. После 8 часов вечера танцы для взрослых}. Шел второй год
мировой войны,  но в Мариуполе не забыли о новогоднем празднике. Вот пример.
\enquote{Мариупольская жизнь} за 26 декабря 1916 г. \enquote{В помещении Реального училища
родительский комитет при Мариупольской женской гимназии устраивает Елки. Начало
для младших детей в 5 час., для старших – в 8 1/2  час. вечера. Плата за вход –
75 коп.}. И еще одно приглашение. \enquote{Мариупольские известия} за 1 января 1919 г.:
\enquote{В среду, 1 января по новому стилю в помещении офицерского собрания 48 пешего
полка (Магазин братьев Адабашевых) будет устроен новогодний вечер с
благотворительной целью. После спектакля музыкально-вокальное отделение с
участием Ф. П. Городецкого и А. А. Терпиловского и учеников музыкального училища. В
заключение – танцы}.

\ii{30_12_2016.stz.news.ua.mrpl_city.1.pro_novogodnjuju_jelku.pic.4}

Новая власть в лице Совнаркома издала декрет от 24 сентября 1929 года, который
отменил все праздники, кроме 1 мая и 7 ноября, а заодно с религиозными и
абсолютно светский - Новый год.  Елку же обозвали \enquote{поповским пережитком}.
Новогодний праздник и ее символ – лесную красавицу - вернули детям накануне
1936 года. Это случилось после того, как было  опубликовано 28 декабря 1935
года в газете \enquote{Правда} письмо, в котором предлагалось вместо Рождества отмечать
Новый год с елкой. Вспомнился  старинный документ под  названием \enquote{Правила
внутреннего распорядка завода \enquote{Русский Провиданс}}, в котором перечислены, кроме
прочего, нерабочие праздничные дни в 1903 году. Таких было 25 религиозных
православных и один светский - 1 января.

\ii{30_12_2016.stz.news.ua.mrpl_city.1.pro_novogodnjuju_jelku.pic.5}

Когда шла кровопролитная война, когда Мариуполь был оккупирован гитлеровцами,
когда оккупанты сожгли полгорода, взорвали заводы и фабрики, водопровод,
электростанции, говорить о праздничных елках просто неприлично. Но уже накануне
1 января 1945 года в некоторых школах устраивались елки.  Из личных
воспоминаний запечатлелось новогоднее торжество, устроенное в преддверии года
1946-го в мариупольской мужской неполной средней школе №3. Тогда из самой
большой классной комнаты была вынесена школьная мебель. Правда, это было все
что угодно, только не парты, которых тогда не было. В середине класса
установили новогоднее дерево – редковатую сосенку чуть меньше  двух метров. Она
была  украшена несколькими  стеклянными  шарами и гирляндами  флажков из
почтовых открыток, квитанций на блеклой бумаге. В классе было холодно. Мальчики
одеты кто в чем. На ком-то пальтишки, из которых они выросли, кто-то был
облачен в стеганку не по росту, кому-то мама сшила курточку из грязно-зеленого
немецкого мундира. Обувь у большинства  ребят - матерчатые стеганые бурки.
Паренек из пятого класса бойко играл на аккордеоне, старшая пионервожатая Люба
Боецкая пыталась водить хороводы. Мальчишки стеснялись, теснились возле стен.

Пролетели десятилетия. Быстро возрождался город. Постепенно улучшалась жизнь.
Сначала ставили елку в сквере, потом и в Городском саду. Когда появился театр,
то перед его фасадом. Все красивее становилось их украшение. Все ярче  на них
иллюминация.  Но только та елка, что стояла в холодном классе, осталась в
памяти во всех деталях.
