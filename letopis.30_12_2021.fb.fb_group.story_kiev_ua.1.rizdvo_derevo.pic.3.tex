% vim: keymap=russian-jcukenwin
%%beginhead 
 
%%file 30_12_2021.fb.fb_group.story_kiev_ua.1.rizdvo_derevo.pic.3
%%parent 30_12_2021.fb.fb_group.story_kiev_ua.1.rizdvo_derevo
 
%%url 
 
%%author_id 
%%date 
 
%%tags 
%%title 
 
%%endhead 

\ifcmt
  tab_begin cols=4,no_fig,center

     pic https://scontent-frt3-2.xx.fbcdn.net/v/t39.30808-6/270380247_979103089346060_8160805467647041877_n.jpg?_nc_cat=103&ccb=1-5&_nc_sid=b9115d&_nc_ohc=Hq8RohDTqH0AX-lByKG&_nc_ht=scontent-frt3-2.xx&oh=00_AT-pTN1x0MBAUnsq_4_vF8CgHbKxUhb7xbDS30oQmDJu7Q&oe=61D2F801
     @caption_begin 
      1900-і рр. Купецьке зібрання, 
      в приміщенні якого ставили різдвяну ялинку і проходили святкові бали. 
      Зараз – Національна філармонія України.
     @caption_end
     %@width 0.1

     pic https://scontent-frx5-1.xx.fbcdn.net/v/t39.30808-6/270449755_979103169346052_2682833119973548132_n.jpg?_nc_cat=105&ccb=1-5&_nc_sid=b9115d&_nc_ohc=lSGjbE_4mCgAX-xNN2V&_nc_ht=scontent-frx5-1.xx&oh=00_AT_TfXQ79euStX7coZuhX2JdX3zaRA7EUGXGvsOfy3ytvg&oe=61D276B3
     @caption 1900-і рр. Газетна реклама різдвяних ялинкових прикрас і подарунків.

     pic https://scontent-frt3-1.xx.fbcdn.net/v/t39.30808-6/270382567_979103209346048_2925008716640747152_n.jpg?_nc_cat=108&ccb=1-5&_nc_sid=b9115d&_nc_ohc=13k-SsTcg7EAX-0w0-E&_nc_ht=scontent-frt3-1.xx&oh=00_AT-qYMQjPB7QOJ5LT0cmHI4Md7HktGpEx8txRnbKdQIYUw&oe=61D321B4
     @caption Дореволюційна різдвяна листівка. Ялинка – атрибут Різдва, а не Нового року.

  	 ig https://scontent-frx5-1.xx.fbcdn.net/v/t39.30808-6/270237050_979103246012711_4223995536347138866_n.jpg?_nc_cat=105&ccb=1-5&_nc_sid=b9115d&_nc_ohc=c9UVGr7mfSgAX_ntgIM&_nc_ht=scontent-frx5-1.xx&oh=00_AT8o6oTR60S_ZZXuc3AVLEAv7aWgUhzopZadJFkF46CFzg&oe=61D24842
  	 @caption Дореволюційна різдвяна листівка. Дата свята вказана – 25 грудня.

  tab_end
\fi
