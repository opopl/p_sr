% vim: keymap=russian-jcukenwin
%%beginhead 
 
%%file 17_11_2021.fb.razum_roman.1.opolchenochka
%%parent 17_11_2021
 
%%url https://www.facebook.com/rrazumr/posts/601566037725144
 
%%author_id razum_roman
%%date 
 
%%tags dnr,donbass,film,film.opolchenochka,lnr,lugansh,rossia,rusmir,vojna
%%title Художественный фильм "Ополченочка" - иммунный ответ Русской Идеи
 
%%endhead 
 
\subsection{Художественный фильм \enquote{Ополченочка} - иммунный ответ Русской Идеи}
\label{sec:17_11_2021.fb.razum_roman.1.opolchenochka}
 
\Purl{https://www.facebook.com/rrazumr/posts/601566037725144}
\ifcmt
 author_begin
   author_id razum_roman
 author_end
\fi

Спасибо большое за обзор фильма «Ополченочка»Сергею Моисееву, председателю
Харьковской областной общественной организации «Русь Триединая» 

Художественный фильм "Ополченочка" - иммунный ответ Русской Идеи.

В интернете появился фильм "ОПОЛЧЕНОЧКА". Первая картина киностудии «Лугафильм»
(2018 г.). 

Скажу сразу — первый блинчик удался! Фильм превзошел мои ожидания, неплохо
сделан и концептуально верен. 

\ifcmt
  ig https://scontent-frt3-2.xx.fbcdn.net/v/t39.30808-6/256485978_601565994391815_5837321562905919985_n.jpg?_nc_cat=103&ccb=1-5&_nc_sid=8bfeb9&_nc_ohc=wIKbYc_9ltIAX-FWP5U&_nc_ht=scontent-frt3-2.xx&oh=be762614a01c70cebe8e7a29b473692e&oe=61A32B1D
  @width 0.4
  %@wrap \parpic[r]
  @wrap \InsertBoxR{0}
\fi

Как ни боролся наш друг, продюсер фильма и соавтор сценария Роман Разум ленту
отказались презентовать в московском Центральном Доме кино, назвав ее жестокой
и политизированной и не предоставили прокатное разрешение на территории РФ. 

Да фильм изобилует жестокими сценами, а фильм Элема Климова «Иди и смотри» не
изобиловал жестокими сценами? Однако в прокате был. 

А фильм «ОПОЛЧЕНОЧКА» в России в прокат так и не вышел. Оно и понятно. 

Выход этого фильма на экраны явит россиянам, что там, в Луганске и Донецке,
люди во многом более русские, чем русские люди в самой России. Там люди не
заболели россиянством, и фильм это являет со всей очевидностью. 

Российские киностудии фильм с таким эмоциональным зарядом не могли снять в
принципе! Да была бы может более выигрышная картинка, но она была бы фальшивой!
Да конечно, может быть кинокритики найдут доработки в этой киноленте. В фильме
звучит мысль, что Харьков «слился под хунту». Но как харьковчанин отмечу –
Харьков не «слился», Харьков сосредотачивается! И главный бой еще впереди! 

Но в целом энергетика фильма потрясающая! Через годы этот фильм будет
документом сегодняшнего дня. Как в свое время документами, отразившими свои
эпохи, стали многие фильмы, в том числе и фильм «Брат». Ведь кроме
профессиональных актеров в фильме «Ополченочка», а по сути «Сестра-1»
задействованы ополченцы и жители Луганска. Это делает фильм практически
документальным. Через 20 лет режиссерам, которые решат снять фильм о
сегодняшних событиях на Донбассе трудно будет найти и аутентичную форму
ополченцев Новороссии и такие человеческие типажи. К тому-же фильм основан на
реальных событиях. Сама жизнь дает такие сюжеты, которые не придумает и опытный
сценарист. 

В персонаже фильма Кате Беловой режиссёре из Санкт-Петербурга, приехавшей в
поисках погибшего брата-добровольца и нашедшей здесь свою любовь — казачьего
атамана Егора, угадывается история одного из командиров ополчения ЛНР Павла
Дрёмова (позывной —«Батя»). Только в реальности невесту Дремова звали Татьяной,
и она действительно приехала на территорию ЛНР в поисках пропавшего брата. В
фильме по сути отражена трагическая гибель Дремова. 

Многие считают, что прототип главной героини фильма Светлана Дрюк, с позывным
«Ветерок», она в 2019-м году перешла на сторону Украины. 

По этому поводу пресс-служба Управления Народной Милиции ДНР заявила, что Дрюк
была уволена из Народной милиции из-за болезни, уехала для лечения в Молдавию и
там была похищена СБУ. 

Но главная героиня «Ополченочки» — собирательный образ. Экипаж Дрюк не был
единственным женским танковым экипажем на Донбассе. 

В очередной раз убедился в верности слов Достоевского – «ОКРАИНЫ СПАСУТ
РОССИЮ!» В наше время Россию спасает битва за НОВОРОССИЮ! У строителей
Глобального мирового порядка все пошло не по плану именно из-за действий
простых южнорусских людей, которые не покорились плану по созданию
однополярного мира. Воля жителей Донбасса, выраженная в слове НОВОРОССИЯ стала
осиновым колом для укро-нацистов, рабов мирового гегемона! 

Сейчас именно НОВОРОССИЯ является смысловым хребтом Русской Цивилизации! Битва
за НОВОРОССИЮ активировала здоровые иммунные силы общерусского организма. В
фильме слово Новороссия звучит неоднократно, и само это слово это уже является
оружием в современной войне смыслов! 

А фильм несмотря на все чинимые препоны найдет своего зрителя, и как когда-то
фильм «Александр Невский» будет воспитывать русских юношей являя им, что РУСЬ
ЖИВА! 

Фильм «Ополченочка» насыщен музыкальными произведениями звучит и Пркокофьев, и
русская народная песня, и русский рок. В саундтреке фильма отметились и широко
и малоизвестные группы, и исполнители: Юлия Чичерина «Смысловые галлюцинации»,
«Зверобой», «Куба», «7-Б» «Ломовой Бэнд», и группа Романа Разума «Новороссия». 

Задача творческих людей отобразить панораму событий 2014-15 годов так, чтобы
мусор плавающий на поверхности интернет-контента и набирающий миллионы
просмотров, внедряющий порочные модели поведения, не выдержал столкновения с
настоящим героическим началом, явленным в «РУССКОЙ ВЕСНЕ» и в по сей день
продолжающейся битве за НОВОРОССИЮ! И тогда сегодняшнее многолетнее стояние на
Угре закончится нашей победой! 

Сергей Моисеев, 

председатель Харьковской областной 

общественной организации «Русь Триединая» 

P.S. Меня лично особо порадовало, что в конце фильма в документальных кадрах
засветился с крестом и иконой мой друг, с которым мы сейчас периодически грузим
фуры с гуманитарной на Донбасс — Саша Кубанец!"

\ii{17_11_2021.fb.razum_roman.1.opolchenochka.cmt}
