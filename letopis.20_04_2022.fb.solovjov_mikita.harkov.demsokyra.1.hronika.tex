% vim: keymap=russian-jcukenwin
%%beginhead 
 
%%file 20_04_2022.fb.solovjov_mikita.harkov.demsokyra.1.hronika
%%parent 20_04_2022
 
%%url https://www.facebook.com/Mikita.Solovyov/posts/7421033277967021
 
%%author_id solovjov_mikita.harkov.demsokyra
%%date 
 
%%tags 
%%title Хроника Харькова, 20-е апреля
 
%%endhead 
 
\subsection{Хроника Харькова, 20-е апреля}
\label{sec:20_04_2022.fb.solovjov_mikita.harkov.demsokyra.1.hronika}
 
\Purl{https://www.facebook.com/Mikita.Solovyov/posts/7421033277967021}
\ifcmt
 author_begin
   author_id solovjov_mikita.harkov.demsokyra
 author_end
\fi

Хроника Харькова, 20-е апреля. 

Обстрелов в городе сегодня меньше, чем предыдущие два дня. \enquote{Меньше} не значит
\enquote{мало}. Но два предыдущих дня были кажется самыми тяжелыми с начала войны.
Трудно сказать, вернулось к уровню предыдущей недели, или более раннего
периода, но меньше этого пика однозначно. Поживем еще некоторое время, можно
будет говорить о некотором новом уровне обстрелов. 

Ну и все еще не расслабляемся. Идет предпасхальная неделя, ждем пакостей от
русских. Чистый четверг, страстная пятница, потом непосредственно Пасха. Не
знаю будут ли еще мощные обстрелы завтра и послезавтра. Но вот в мощном
обстреле непосредственно на Пасху я почему-то не сомневаюсь. Потому обращаюсь
ко всем с огромной просьбой воздержаться от посещения Всенощной и вообще в эти
дни по возможности обходить церкви по большому радиусу. В предпасхальные и
пасхальные дни это гарантированно будет самым опасным местом. Понимаю, что для
многих это традиция, для воцерковленных так вообще главный праздник в году. Но
очень прошу поберечься. Меня сложно назвать паникером и гиперозабоченным
безопасностью. Но здесь цель настолько очевидная и однозначная, что это уже
напоминает игру в русскую рулетку. 

По области у меня мало данных, чтобы говорить о тенденциях. Но по тем местам, о
которых я знаю, обстрелов тоже стало меньше. Сразу скажу, о изюмском
направлении информации мало. И там ситуация может заметно отличаться. Боевые
действия в области вроде бы тоже немного затихли кроме изюмского направления. 

Да, интересный момент. Хотя вчера и позавчера уровень обстрелов был заметно
выше среднего, никакого заметного роста эвакуации не наблюдалось. Сегодня не
знаю, а вчера уже единственный эвакуационный поезд ушел опять не полностью
заполненным. Это немного странно. Раньше каждый пик обстрелов давал через
день-два некоторый скачок количества эвакуировавшихся. Скорее всего, оставшиеся
в городе уже готовы к такой плотности обстрелов и не считают их поводом
уезжать. (Разумно это или нет, я сейчас не обсуждаю. Об этом я отдельные посты
писал.) Посмотрим как пройдет Пасха, и уже после нее можно будет делать
какие-то серьезные выводы. 

Если мои наблюдения верны. то можно считать население города во время войны
стабилизировавшимся. По моим оценкам это немного больше полумиллиона человек. И
это возвращает нас к вопросу о обустройстве жизни в городе на ближайшие месяцы.
Я по-прежнему считаю, что с одной стороны нужно отселять пять наиболее
пострадавших микрорайонов, а с другой в дневное время возвращать метро в режим
общественного транспорта. Оба эти решения мне кажутся повышающими устойчивость
Харькова как форпоста на случай еще одной попытки вторжения. И при этом
обеспечивающими совокупно большую безопасность и заметно более высокий уровень
функционирования города.

Харьков стоит!

Слава Украине!

Низкий поклон нашим защитникам!

\textbf{\#ХроникаХарькова}

\ii{20_04_2022.fb.solovjov_mikita.harkov.demsokyra.1.hronika.cmt}
\ii{20_04_2022.fb.solovjov_mikita.harkov.demsokyra.1.hronika.cmt.scr}
