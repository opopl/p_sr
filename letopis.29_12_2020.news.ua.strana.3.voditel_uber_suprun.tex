% vim: keymap=russian-jcukenwin
%%beginhead 
 
%%file 29_12_2020.news.ua.strana.3.voditel_uber_suprun
%%parent 29_12_2020
 
%%url https://strana.ua/news/309318-uljanu-suprun-taksist-uber-vysadil-iz-mashiny.html
 
%%author 
%%author_id 
%%author_url 
 
%%tags 
%%title Водитель Uber отказался везти Супрун с мужем и высадил их на полпути
 
%%endhead 
 
\subsection{Водитель Uber отказался везти Супрун с мужем и высадил их на полпути}
\label{sec:29_12_2020.news.ua.strana.3.voditel_uber_suprun}
\Purl{https://strana.ua/news/309318-uljanu-suprun-taksist-uber-vysadil-iz-mashiny.html}

\ifcmt
  pic https://strana.ua/img/article/3093/uljanu-suprun-taksist-18_main.jpeg
	caption Ульяна Супрун. Фото: "Страна" 
\fi

Бывшая и.о. министра здравоохранения Украины Ульяна Супрун рассказала, как их с
супругом на полпути высадил таксист Uber, потому что они попросили открыть
окно, чтобы снизить вероятность заражения коронавирусом в машине. 

Об этом Супрун рассказала на странице в Facebook.

"Все уже знают, что вирус SARS-CoV-2 передается воздушно-капельным путем.
Скорее всего, это может случиться при длительном контакте (более 15 минут) и на
близком расстоянии (менее 1,5 метра) от инфицированного человека. Еще
вероятнее, что это произойдет в небольшом и плохо проветриваемом помещении.
Например, в такси", - написала Супрун.

Бывшая руководительница Минздрава рассказала, что о необходимости обеспечивать
циркуляцию воздуха в машине, чтобы обезопасить себя от коронавируса, стало
известно после публикации в начале декабря результатов небольшого исследования. 

"Исследователи создали компьютерную модель, которая имитировала поток воздуха
внутри компактного автомобиля. На основе этой модели они имитировали различные
сценарии езды в машине: с открытыми окнами и закрытыми окнами. Целью было
определить, каким образом ездить будет безопаснее. Та модель, которая
демонстрировала поток воздуха при открытых окнах, показала низкую концентрацию
воздушных частиц, которые теоретически могут передавать вирус между людьми.
Зато наихудшим сценарием оказалась езда с закрытыми наглухо окнами и
кондиционером или отоплением", - сообщила Супрун. 

По ее словам, исходя из данных исследования, они с супругом Марком каждый раз
при поездке в такси просят водителей немного открыть хотя бы одно окно.

"Сегодня поступили так же - попросили опустить стекло на одном окне на 2
сантиметра ниже. Но водитель сказал, что не хочет этого делать из-за того, что
образуется сквозняк. А он боится заболеть. Тогда мы сказали ему, что от
сквозняка не передаются опасные возбудители болезней, в отличие от душного
воздуха. Пробовали рассказать об исследовании. Но он не захотел нас слышать.
Зато предложил выйти на полпути из авто. Водитель испугался мифического
сквозняка больше, чем реального Covid-19", - пожаловалась Супрун.

По ее словам, история закончилась тем, что они вызвали другое такси Uber, где
водитель открыл окно после соответствующей просьбы и довез их пункта
назначения.

\ifcmt
tab_begin cols=2
  pic https://strana.ua/img/forall/u/10/88/QIP_Shot_-_Screen_5793.png
	pic https://strana.ua/img/forall/u/10/88/QIP_Shot_-_Screen_5794.png
tab_end
\fi

