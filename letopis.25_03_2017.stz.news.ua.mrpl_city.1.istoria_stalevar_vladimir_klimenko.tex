% vim: keymap=russian-jcukenwin
%%beginhead 
 
%%file 25_03_2017.stz.news.ua.mrpl_city.1.istoria_stalevar_vladimir_klimenko
%%parent 25_03_2017
 
%%url https://mrpl.city/blogs/view/istoriya-stalevar-klimenko
 
%%author_id burov_sergij.mariupol,news.ua.mrpl_city
%%date 
 
%%tags 
%%title История: сталевар Владимир Клименко
 
%%endhead 
 
\subsection{История: сталевар Владимир Клименко}
\label{sec:25_03_2017.stz.news.ua.mrpl_city.1.istoria_stalevar_vladimir_klimenko}
 
\Purl{https://mrpl.city/blogs/view/istoriya-stalevar-klimenko}
\ifcmt
 author_begin
   author_id burov_sergij.mariupol,news.ua.mrpl_city
 author_end
\fi

Прежде сталевары были в большом почете. И такое внимание вполне оправдано.
Умелый сталевар – это сплав теоретических знаний, многолетнего
профессионального опыта, интуиции, организаторских способностей, выносливости,
наконец, недюжинной физической силы. Но и среди этих, далеко нерядовых людей,
были и есть настоящие асы. И один из них - ильичевец, Герой Социалистического
Труда Владимир Павлович Клименко.

\ii{25_03_2017.stz.news.ua.mrpl_city.1.istoria_stalevar_vladimir_klimenko.pic.1}

Встречу съемочной группы с легендарным сталеваром, которая произошла в октябре
2004 года, организовал тоже ильичевец Анатолий Иванович Томаш – человек
отзывчивый на чужие беды, чья рабкоровская деятельность постепенно переросла в
профессиональную журналистику,  а до того - помощник начальника мартеновского
цеха №2, где в свое время ставил рекорды легендарный Макар Мазай и работал
сталевар Клименко.

Время стерло из памяти и дом, и вид обстановки комнаты, где происходила беседа,
но сохранило настрой рассказа Владимира Павловича, неспешный, без малейшего
признака рисовки. Так - констатация фактов, как будто не стоял перед ним
оператор с телекамерой, фиксирующей каждое его слово. Он делился воспоминаниями
не с устройством с объективом и микрофоном, он рассказывал людям, слушающим его
с неподдельным интересом сейчас, совершенно не задумываясь о том, что слова его
могут стать достоянием истории. Запомнилось вот еще что. Пришлось приложить
немалые усилия, чтобы уговорить Владимира Павловича извлечь из шкафа и надеть
парадный пиджак с заслуженными им наградами.

Вот рассказ сталевара Клименко, практически не тронутый правкой:

- Родился я в крестьянской семье в селе Пески Николаевской области. Отец был из
крестьян, мама – тоже. Когда образовался колхоз, люди выбрали отца
председателем. Мама была домохозяйкой в основном. Нас было у родителей трое –
две сестры и я. В тридцатые годы пошел я в школу.

Потом нежданно- негаданно началась война. Отец ушел на фронт, в сорок первом
году погиб, мама умерла. Вся наша семья распалась, а мы остались втроем – кто
куда. Одну сестру забрала тетя, другую сестру дедушка забрал, а я остался не у
дел. Остался один в своем домике. Перед оккупацией взяли меня на мельницу
подручным кочегара. Потом пошел на курсы трактористов, работал на тракторе. Ну,
и учился. А потом уже, в 49 году был призван на флот.

Попал за границу. Был в составе советских войск в Австрии, на Дунае. Три года
там пробыл, а после, когда наши войска из Австрии были выведены, продолжал
службу на Черноморском флоте. В 55 году демобилизовался, и приехал в Мариуполь,
где жили мои два дяди. Началось мое трудоустройство на завод Ильича. Потянуло
то, что масштабы все-таки большие. Да еще во время службы на флоте прочитал
роман Попова \enquote{Сталь и шлак}, там о металлургах речь шла. Сосед, где я жил у
дяди на Новоселовке, был сталевар – Николай Ильич Рыбалко. Попросил я его,
чтобы познакомил меня с Яковом Григорьевичем Привезенцевым, начальником 1-го
мартеновского цеха. Это старый цех, по-моему, его закрыли в 64 году.

Приняли меня, и я там работал. Все прелести этой работы познал. Печи были не
ахти, какие большие, но работа на них была очень трудная. У нас работали
заключенные, так они говорили: \enquote{Лучше весь срок отсидеть на зоне, только не
работать тут в мартеновском цехе}. Были такие моменты, что страшно вспомнить.
Втянулся, привык, потом уже по-другому было. Пошло дело, когда уже освоился с
работой. Работал третьим подручным, затем – вторым, наконец,  первым подручным.
В 58 году начал подменять сталевара. После закрытия старого цеха перешел во 2-й
мартеновский цех. Тогда новый мартен с огромными печами 650-ти и 900-тонными
только-только начали осваивать. Много ребят от нас ушло на новое производство.
Ну, а я остался во втором мартене до конца.

Я никогда не забуду своих цеховых наставников, которых должен благодарить за
науку. Сталевар был Григорий Голощапов в подменной бригаде, мастером был
Василий Фотиевич Кравченко. Это люди, которые учили уму разуму и, конечно,
сталеварскому делу. Потом уже, когда я перешел в мартеновский цех №2, ближе
познакомился с Иваном Андреевичем Лутом. От него я тоже немало взял из
сталеварской науки. Между прочим, Иван Андреевич работал еще с нашим знаменитым
сталеваром Макаром Мазаем…

Когда перешел на пенсию, я по предложению руководителей по переводу перешел в
профтехучилище мастером производственного обучения. Десять лет проработал там.
Выпустил три группы сталеваров и четыре группы контролеров ОТК для
сталеплавильных цехов комбината. Мне как-то предложили группу электриков вести,
но я категорически отказался. Не мой это профиль. И вскоре я рассчитался.

Через какое-то, довольно продолжительное время после события, описанного выше,
встретился озабоченный Анатолий Иванович Томаш: \enquote{Володя Клименко тяжело
заболел, спешу в профком комбината выбивать деньги на операцию}. Операцию
сделали, но она не помогла. Герой Социалистического Труда, знатный сталевар
комбината имени Ильича Владимир Павлович Клименко скончался 7 марта 2007 года...

Поскольку Владимир Павлович в своих воспоминаниях не упомянул некоторые
подробности своей жизни, показалось необходимым привести некоторые
дополнительные сведения о нем. Он родился 3 января 1929 года,  пришел работать
на завод имени Ильича подручным сталевара в 1956 году. С 1961 по 1980 -
сталевар мартеновского цеха № 2. Значительную часть его производственной
деятельности прошла на 11-й мартеновской печи, в которой выплавлялись самые
сложные и ответственные марки стали, с том числе и оборонного назначения. В
1971 году был удостоен звания Героя Социалистического Труда, кавалер орденов
Ленина, Трудового Красного Знамени, награжден медалями, некоторое время
совмещал основную работу с руководством цеховой профсоюзной организацией. С
1981 по 1985 год по решению  дирекции комбината работал мастером
производственного обучения в ПТУ-99.
