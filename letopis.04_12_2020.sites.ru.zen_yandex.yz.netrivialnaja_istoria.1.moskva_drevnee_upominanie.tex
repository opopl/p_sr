% vim: keymap=russian-jcukenwin
%%beginhead 
 
%%file 04_12_2020.sites.ru.zen_yandex.yz.netrivialnaja_istoria.1.moskva_drevnee_upominanie
%%parent 04_12_2020
 
%%url https://zen.yandex.ru/media/id/5f6a1d4eaddbe610281941b4/drevneishee-pismo-s-upominaniem-moskvy-5fc9267feb95a53734b106bf
 
%%author 
%%author_id yz.netrivialnaja_istoria
%%author_url 
 
%%tags moskva,ancient,russia
%%title Древнейшее письмо с упоминанием Москвы
 
%%endhead 
 
\subsection{Древнейшее письмо с упоминанием Москвы}
\label{sec:04_12_2020.sites.ru.zen_yandex.yz.netrivialnaja_istoria.1.moskva_drevnee_upominanie}
\Purl{https://zen.yandex.ru/media/id/5f6a1d4eaddbe610281941b4/drevneishee-pismo-s-upominaniem-moskvy-5fc9267feb95a53734b106bf}
\ifcmt
  author_begin
   author_id yz.netrivialnaja_istoria
  author_end
\fi

\index[rus]{Русь!История!Древнейшее письмо с упоминанием Москвы, 04.12.2020}

\begin{multicols}{2}

Нет! Мы не будем много говорить про упомянутый во всех учебниках летописный
пассаж. Про то, как Юрий Долгорукий писал князю Святославу (отцу героя «Слова»)

\begin{leftbar}
  \begingroup
    \em\bfseries\color{blue}\Large Приди ко мне, брате, в Москов
  \endgroup
\end{leftbar}

и дал ему «обед силен» 4 апреля 1147 года.

Старейший список этой летописи, из костромского Ипатьевского монастыря, был
создан лишь в начале XV века, хотя и восходит к более древним, не дошедшим до
нас рукописям.

Другие летописи – старше: самый древний манускрипт с Лаврентьевской переписал
для Суздальского князя в 1377 году

\begin{leftbar}
  \begingroup
    \em\Large\bfseries\color{blue}
«худой, недостойный и многогрешный раб Божий Лаврентий монах» 
  \endgroup
\end{leftbar}



А самый старый (Синодальный) список Новгородской первой летописи создан еще в
конце XIII века. Конечно, и в Лаврентьевской, и в Новгородской летописях тоже
встречается имя «Москва».

Но копнем глубже! Существует ли рукопись XII века с упоминанием нашего города?

Манускриптов этого периода сохранилось очень мало (время, пожары,
татаро-монголы…), и среди них нет русских летописей – летописи дошли только в
позднейших списках.

И все-таки есть документ с упоминанием Москвы, написанный рукой современника
Юрия Долгорукого и Андрея Боголюбского.

В 1990 году в Великом Новгороде на Михаилоархангельском раскопе, квадрат 104,
археологи нашли неразорванную (большая удача!) берестяную грамоту №723.

\begin{leftbar}
\begingroup
\em\Large\bfseries\color{blue}
+ покланѧние ѿто душиль ко нѧсть шьль ти есьмъ кучькъву ажь ти
хътѧ жьдати ааи ти нь хътѧ жьдати а оу ѳьдокь обруць ее водадѧ а свое възьму
\endgroup
\end{leftbar}

Перевод:

\begin{leftbar}
  \begingroup
    \em\Large\bfseries\color{blue}
    «Поклон от Душилы Нясте (?). Я пошел в Кучков. Хотят ли ждать (не указано,
    кто) или не хотят (другой вариант: Захочу ли я ждать или нет), а я у Федки,
    отдав ей браслет, свое возьму».
  \endgroup
\end{leftbar}
    
\end{multicols}

\ifcmt
tab_begin cols=2
  caption Древняя Москва и Лаврентевская Летопись

  pic https://avatars.mds.yandex.net/get-zen_doc/3144955/pub_5fc9267feb95a53734b106bf_5fc9294a6bcad04cbc59248e/scale_1200
  caption Москва-городок и окрестности в XII веке. Рисунок Аполлинария Васнецова, 1929 год (Public Domain, Wikimedia)
  width 0.45

  pic https://avatars.mds.yandex.net/get-zen_doc/1675790/pub_5fc9267feb95a53734b106bf_5fc928bfaa85793798696410/scale_1200
  width 0.25
  caption Страница из Лаврентьевской летописи
tab_end
\fi

\begin{multicols}{2}
        

Письмо было нацарапано между 1160 и 1180 годами. Понять его (как и другие
древненовгородские грамоты) не очень просто. Дело не только в языковых отличиях
или в почтенном возрасте этой записки – 850 лет, не шутка. Проблема заключается
в самом литературном жанре!

Берестяные грамоты – это средневековые «эсемески» (наши бабушки б сказали –
«телеграммы»). В отличие от пергамена, береста ничего не стоила, но
процарапывать на ней буквы не очень удобно. Поэтому древние новгородцы
стремились к краткости и пропускали информацию, которая и так ясна для
адресата.

Лучший образец – это письмо конца XI века (№590), состоящее из 21 знака:

\begin{leftbar}
  \begingroup
    \em\Large\bfseries\color{blue}
    литва въстала на корѣлоу
  \endgroup
\end{leftbar}

(конечно, в подлиннике нет разделения на слова и знака переноса! иначе чтение
для нас превратилось бы в отгадку ребуса – а новгородцам было нормально)

Какой-то древнерусский «Штирлиц» послал в Новгород депешу в тот момент, когда
литовцы вышли в поход против карел – контекст был, очевидно, уже ясен, но было
важно, чтобы сообщение послали своевременно. Донесение разведчика – в 21 букве.
Все.

Но возвратимся к грамоте №723!

\end{multicols}

\ifcmt
pic https://avatars.mds.yandex.net/get-zen_doc/3380298/pub_5fc9267feb95a53734b106bf_5fc929b4eb95a53734b9c7f5/scale_2400
caption Берестяная грамота с упоминанием Москвы
\fi

Какой-то Душила пишет какому-то Нясте, что собирается «взять свое» у Федки и
вернуть ей ее обруч. Как полагают Зализняк и Янин, речь идет о расторжении
помолвки: «обруч» мог означать кольцо или браслет, а также обручальный залог.

Но интереснее всего «шьльтиесьмъкучькъву»: автор письма собирается поехать в
Кучков.

Топоним «Кучков» появляется в Ипатьевской летописи под 1176 годом: больного
князя Михалка Юрьевича пронесли,

\begin{leftbar}
  \begingroup
    \em\Large\bfseries\color{blue}
«идоша съ нимь до Кучкова, рекше до Москвы»
  \endgroup
\end{leftbar}

Ипатьевская летопись дошла до нас в рукописи XV века, а легенду о Кучке
записали только в XVII столетии.

Однако подлинная грамота XII века сообщает о Москве и подтверждает, что городок
в те годы действительно называли Кучковым.

Кстати, это косвенно подтверждает, что легенда о боярине Кучке имеет какую-то
историческую основу.

Вот каково самое ранее упоминание о Москве.
