% vim: keymap=russian-jcukenwin
%%beginhead 
 
%%file 11_11_2021.fb.fb_group.story_kiev_ua.2.pamjat_rahlina_eleonora
%%parent 11_11_2021
 
%%url https://www.facebook.com/groups/story.kiev.ua/posts/1795035910693202/
 
%%author_id fb_group.story_kiev_ua,majorenko_georgij.kiev
%%date 
 
%%tags gorod,istoria,kiev,kievljane,pamjat,rahlina_eleonora.kiev.ekskursovod,ukraina
%%title ПАМЯТИ СЛАВНОЙ КИЕВЛЯНКИ
 
%%endhead 
 
\subsection{ПАМЯТИ СЛАВНОЙ КИЕВЛЯНКИ}
\label{sec:11_11_2021.fb.fb_group.story_kiev_ua.2.pamjat_rahlina_eleonora}
 
\Purl{https://www.facebook.com/groups/story.kiev.ua/posts/1795035910693202/}
\ifcmt
 author_begin
   author_id fb_group.story_kiev_ua,majorenko_georgij.kiev
 author_end
\fi

ПАМЯТИ СЛАВНОЙ КИЕВЛЯНКИ

9 ноября был День рождения мамы моего школьного товарища скульптора Игоря
Лысенко - Элеоноры Натановны Рахлиной - известного киевоведа и экскурсовода. Я
с теплом вспоминаю эту добрую и замечательную женщину, бывал в гостях у
Рахлиных, Элеонора Натановна проводила экскурсии для нашего класса. Намедни
прочитал с ней интервью от замечательного киевского барда и журналиста Анатолия
Лемыша и хотел  бы эту киевскую историю представить вашему вниманию.

\ii{11_11_2021.fb.fb_group.story_kiev_ua.2.pamjat_rahlina_eleonora.pic.1}

"Среди любителей истории Киева время от времени пролетает весть: «Рахлина будет
водить!» Трезвонят телефоны, по цепочке передавая, где и когда состоится
экскурсия. Где-нибудь на Подоле собираются киевоманы, прошедшие вместе с
Леонорой Натановной по многим историческим улочкам Киева, приводят детей, а то
и внуков. 

Свыше тридцати лет Леонора Рахлина водит экскурсии. Причем, она рассказывает о
нашем городе так интересно, с таким заразительным энтузиазмом, что вокруг нее
образовался клуб почитателей истории Киева. В середине 70-х годов он назывался
«Летопись», затем «Клио». Часто Леонора Натановна приглашала своих учеников в
огромную квартиру на ул. Франко, оставшуюся от ее отца, выдающегося дирижера,
народного артиста СССР Натана Рахлина. В ней все дышало музыкой, историей, на
стенах висели антикварные полотна, возвышались грандиозные шкафы с книгами. Эта
квартира в свое время была одним из центров культурной жизни Киева. 

ВСПОМИНАЕТ ЭЛЕОНОРА НАТАНОВНА:

- В доме у нас всегда был культ отца.

Принято считать, что музыкант должен работать за роялем или другим
инструментом. Натану Григорьевичу это было совершенно не нужно. Он читал
партитуру, слышал ее! Воспринимал ноты с листа, читал их, как мы обычную книгу.
Улыбался, мог засмеяться или что-то подчеркнуть. Перед концертом ему надо было
обязательно отдохнуть. Это тяжелейший труд - быть дирижером, тем более таким
эмоциональным, как мой отец. Мама брала с собой две рубашки - они промокали
насквозь. После выступления к нему набивалась полная артистическая. Начиналась
ночная жизнь, в которой я, ребенок, не могла принимать участие. Конец 30-х
годов был для него удачным: он занял одно из ведущих мест среди музыкантов
Советского Союза, создал киевский симфонический оркестр и стал его главным
дирижером. 

В детстве у меня была война. В начале лета 1941 года оркестр был на гастролях,
а мама моя лежала в роддоме на сохранении - у нее очень тяжело протекала
беременность. И тут 22 июня, бомбежка. Вся медицина разбежалась, а у мамы
начались роды. Родился мальчик, но некому было его принять, и он погиб. У мамы
началась родовая горячка, заражение крови. Плюс 41-й год, эвакуация, отец на
гастролях, связи нет... 

Нам удалось выехать из Киева последним эшелоном, маму принесли к поезду на
носилках. Выжила она благодаря знакомому врачу, который взял ее в свой вагон. А
я на остановках бегала между теплушкой, где ехала бабушка, и медицинским
вагоном. Мне не было и пяти лет, я бегала-бегала, и в Пензе отстала от поезда.
Меня отдали в детдом, который был вывезен из Москвы. Его разместили в залах
пензенской художественной галереи. Там были высоченные потолки, непонятные
темные картины и огромные скульптуры. Мы, дети, их очень боялись и обходили
стороной. Но, может, с тех пор у меня такая тяга к живописи. Много лет спустя я
приехала в Пензу, нашла эту картинную галерею. Она оказалась очень скромным
двухэтажным зданием, с серийными гипсовыми копиями греческих скульптур. А в
детстве все виделось настолько иначе... 

Из Пензы нас эвакуировали во Фрунзе, в Среднюю Азию. Помню степь, а весной на
ней - сплошной ковер тюльпанов. Оказалось, что мама в госпитале в одном конце
страны, папа - в другом, я - в третьем. А я еще не умела говорить букву “р”, и
когда спрашивали, как моя фамилия, вместо «Рахлина» отвечала «Ляхина».
Говорила, что папа музыкант, но такого дирижера \enquote{Ляхина} никто, естественно, не
знал. Лишь спустя два года мама нашла меня, приехала во Фрунзе... Маме не
хотели меня отдавать, потому что все документы сгорели. И ей пришлось меня
удочерить. 

Окончилась война, и в 1945 году в Киеве Рахлин возобновил бесплатные концерты
на открытой сцене в Днепровском парке. Отец сочувствовал тем, кто был лишен
музыкального слуха, считал, что это такая же увечность, как быть без глаз или
без руки. Он говорил: когда я стою перед оркестром и у меня в руке палочка - я
всесилен! Взмахну рукой - и возникает лес, озеро, лебеди плывут. А еще в нем
горело огромное желание поделиться этим богатством. Это было время, когда на
его концерты в парке собирался весь город. Можно было остановить мальчишку на
улице, и он бы насвистел вальс из \enquote{Щелкунчика} или тему из \enquote{Полета шмеля}. А
его концерты в залах - это были каждый раз открытия новых произведений. 

- Леонора Натановна, с чего начались ваши исследования Киева, увлечение его
историей?

- Тоже от отца. Он водил меня по городу, много рассказывал. А потом получилось
так, что я много лет жила не в Киеве. Училась в Ленинграде, вышла замуж и
уехала на Крайний север. Когда вернулась, надо было думать о работе. А с этим
было туго. К этому времени Натан Григорьевич уже ничем мне помочь не мог. Его
тут \enquote{съели}, выжили из Киева. В 1962 году прекратились концерты в парке. Рахлин
вынужден был поехать в Саратов, руководил там оркестром. Это была практически
ссылка. 

Когда я стала искать работу, мне показалось заманчивым водить по Киеву
экскурсии. Но через несколько лет в рамках тем, которые мы на лекциях
рассказывали, мне стало тесно. Я попыталась в Бюро по путешествиям и экскурсиям
выделить тему о древнем Киеве. С огромным трудом мне удалось этого добиться.
Постепенно сложился круг людей, не безразличных к истории города. В 70-е годы
сносились целые улицы со своим неповторимым укладом. Получалось, что многие
страницы старого Киева уходят, так и не найдя своего историка. Официально клуб
\enquote{Летопись} возник в середине 70-х годов при библиотеке имени Гоголя на ул.
Горького. Сейчас многие дома уже снесены, выросли многоэтажки, и теперь мало
кто сможет восстановить, какой была улица, кто жил на ней. А она есть - в наших
фотографиях, в записанных беседах со старожилами. 

Могу с гордостью сказать, что многое в Киеве сохранилось благодаря клубу
\enquote{Летопись}. Нам удалось его отстоять обреченный на снос квартал Крещатика между
ул. Ленина (Б. Хмельницкого) и бульваром Шевченко. Уже был подписан приказ на
снос домов. Причем, учтите, это были не нынешние времена, когда с кем-то можно
спорить, выступать в прессе, поднимать народ. Тогда было сложней. Власти города
утверждали, что этот квартал Крещатика интереса в плане художественном и
историческом не представляет. Махали справкой в два листочка с подписью некоего
чина из ГлавАПУ Киева. А мы предъявили три тома документов, где по каждому дому
Крещатика было расписано, кто из замечательных жителей города в них жил, какие
события в них происходили. И – отстояли целый квартал в центре Киева! Благодаря
нашему вмешательству был сохранен дом Леси Украинки по ул. Стрелецкой, 15. И
таких домов десятки. 

Знаете, я считаю, что имею право проходить по нашему городу с высоко поднятой
головой. Потому что есть реально мною спасенные, реально сохраненные
архитектурные памятники. Это здания, каждое из которых для города, дорожащего
своей историей, может стать реликвией. К ним очереди экскурсантов должны
стоять! О них должны писать в книгах! 

Есть собранные архивы, огромный исторический материал. Но самое главное - это
целая когорта людей, которые пришли к нам в «Летопись», в «Клио», и стали
летописцами, исследователями, незаметными (а то и знаменитыми!) киевоведами.
Это Михаил Кальницкий, опубликовавший массу материалов по истории Киева. Это
Василий Галайба, заснявший в 70-80-х годах свыше тысячи домов, которые потом
были разрушены. Он выпустил замечательный альбом “Фотоспомин: Київ, якого
немає”. Это Михаил Рыбаков, Мария Кадомская, Юлия Смилянская, Елена Царовская.
Это десятки моих учеников, когда-то ходивших на заседания “Летописи”, и ставших
экскурсоводами. Я и сама провела не одну тысячу экскурсий. Мне умирать не
страшно: я знаю, скольких киевлян заразила я любовью к нашему городу. Тех, кто
ногами исходил со мной многие улочки Киева. 

- Леонора Натановна, а какие экскурсии вы водите сейчас?

- У меня 160 тем! Я могу провести по любой улице, рассказать о любом доме. Ну,
почти о любом: не все, конечно, этого заслуживают. Я сейчас готовлю книжку, она
так и называется: \enquote{История одного дома}. Надо, чтобы люди открывали глаза на
свой город, ходили по нему, глядя не на асфальт, а на фасады домов. Иногда
провожу тест: прошу человека рассказать о доме, в котором он живет. Ну, хотя
бы, какой у него фасад? Многие не могут вспомнить. 

- Помню историю, как вы заложили в банк свою квартиру, чтобы найти средства на
мемориальную доску отцу, Натану Рахлину. Потом была ваша голодовка у стен
Украинского дома, закончившаяся тем, что вас нашли на улице без сознания. Каков
финал у этого громкого скандала? 

— Перед 90-летием со дня рождения моего отца я много раз ходила в Министерство
культуры, в Главное управление культуры Киева, в Оперу, в филармонию, в
консерваторию — везде прекрасно знали Рахлина, но помочь увековечить его память
не могли. Но доску-то надо было открыть! 

Сначала мы у кого-то одалживали. Потом настало время отдавать деньги — и мне
пришлось заложить квартиру. Все финансовые гаранты обещали: деньги вот-вот
будут, потерпите. На долги наросли проценты, и возникла безвыходная ситуация —
надо было убираться на улицу. Да, я сделала ошибку, что не оговорила все
финансовые вопросы до последней запятой. Я просто хотела почтить память своего
отца, великого музыканта, столько сделавшего для Украины и для Киева... 

Мою квартиру продали на аукционе, но мне удалось зацепиться за юридические
тонкости и отсрочить ее освобождение. Тогда-то мне и пришлось прибегнуть к
голодовке, чтобы обратить внимание к этой вопиющей теме. Голодовка завершилась
больницей. А потом помог фонд Н. Рахлина, созданный в Америке музыкантами.
Кроме того, от отчаяния я пошла на телеигру в "Миллион", и выиграла довольно
крупную сумму. Это позволило выкупить квартиру обратно. Пришлось еще продать
часть уникальных книг из моей библиотеки... Но мемориальная доска моего отца
висит, хотя ни город, ни государство ничего не сделали для его памяти…

* Леонора Натановна Рахлина скоропостижно скончалась в апреле 2006 года."

АНАТОЛИЙ ЛЕМЫШ

\ii{11_11_2021.fb.fb_group.story_kiev_ua.2.pamjat_rahlina_eleonora.cmt}
