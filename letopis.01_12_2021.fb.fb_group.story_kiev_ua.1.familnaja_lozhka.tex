% vim: keymap=russian-jcukenwin
%%beginhead 
 
%%file 01_12_2021.fb.fb_group.story_kiev_ua.1.familnaja_lozhka
%%parent 01_12_2021
 
%%url https://www.facebook.com/groups/story.kiev.ua/posts/1809335035929956
 
%%author_id fb_group.story_kiev_ua,majorenko_georgij.kiev
%%date 
 
%%tags kiev
%%title ФАМИЛЬНАЯ ЛОЖКА, ЖЛОБЫ - БУБЛИКИ И ДНИ ШУЛЬГИНЫХ
 
%%endhead 
 
\subsection{ФАМИЛЬНАЯ ЛОЖКА, ЖЛОБЫ - БУБЛИКИ И ДНИ ШУЛЬГИНЫХ}
\label{sec:01_12_2021.fb.fb_group.story_kiev_ua.1.familnaja_lozhka}
 
\Purl{https://www.facebook.com/groups/story.kiev.ua/posts/1809335035929956}
\ifcmt
 author_begin
   author_id fb_group.story_kiev_ua,majorenko_georgij.kiev
 author_end
\fi


\headTwo{ФАМИЛЬНАЯ ЛОЖКА, ЖЛОБЫ - БУБЛИКИ И ДНИ ШУЛЬГИНЫХ}

Когда-то прадед Стефан Неводовский подарил мне серебряную ложку с дарственной
подписью и сея реликвия теперь вдохновляет меня в архивных поисках.

И чем глубже погружаешься в прошлое, тем больше интересного находишь и
открываются такие \enquote{пласты и залежи}, что просто диву даешься!

В этом сюжете хочу рассказать об одном старинном роде, с которым породнились
мои далёкие предки и у этого рода многое связано с Киевом.

\headTwo{БУБЛИКИ ИЗ ЖЛОБОВ}

Началось все в конце 18 века, когда Елена Неводовская (из наших) вышла замуж за
Якова Григорьевича Бублика - Погорельского. Тоже из дворян. И каково же было
мое удивление, когда я узнал, что в Малороссийском дворянстве Бублики-
Погорельские фигурируют, как Жлобы-Погорельские!

% Были и такие дворяне...
\ii{01_12_2021.fb.fb_group.story_kiev_ua.1.familnaja_lozhka.pic.2}

- Вот это номер! - подумал я.

Сразу вспомнились строки из песни "А без Подола Киев невозможен:

Верхний Вал и Нижний вал

сам Хмельницкий там бывал

и водил на водопой коня.

А где воду пил тот конь -

жлоб построил Оболонь

по проекту завтрашнего дня.

Самое интересное, что когда несколько человек из Жлобов\hyp Погорельских в 18
веке подали документы на внесение их в списки дворян, то один из них - Григорий
не захотел фигурировать в списке, как Жлоб, а записался Бубликом. А сын его -
Яков женился на нашей красавице Елене Неводовской. 

О Неводовских я уже писал немало. В роду были священники, ученые, генералы.. 

И по линии Бубликов-Погорельских подобрался народ колоритный!

Один из них - Григорий Иванович Бублик - Погорельский был надворным советником,
членом Киевского окружного суда.

Сын же его Михаил Бублик - Погорельский - пошел по купеческой линии. Держал в
Киеве кондитерские, бакалейные, чайные. А дочь Александру выдал замуж за
киевского богатея Ивана Степановича Слинько.

Знатные были люди!

\headTwo{НАСТОЯЩИЙ КАЗАК}

Хранителем казацких традиций в роду был Василий Дмитриевич Бублик-Погорельский
- носивший запорожские усы и напоминавший Тараса Бульбу с картины Репина. И
немудрено! По семейному преданию, один из Жлобов-Погорельских принимал участие
в написании знаменитого письма запорожцев Турецкому султану:

\enquote{...місяць у нeбі, год у книзі, а день такий у нас, як і у вас, поцілуй за те
ось куди нас!...}

В память о казацких предках Василий Дмитриевич соединил все старинные родовые
прозвища и получилась смешная фамилия - Тяпу-Тяпу Табунець - Жбан - Жлоба -
Бублик - Погорельский!

\headTwo{БУБЛИК С ДЫРОЧКОЙ}

Такая диковинная фамилия перешла по наследству дочери Василия Дмитриевича -
Лидии, что получила в Санкт- Петербургском Медицинском институте специальность
военного врача.

Как и отец, она вращалась в среде украинофилов, там и мужа нашла! Да не
простого, а самого Александра Яковлевича Шульгина из знаменитой киевской семьи! 

Александр окончил  историко-филологический факультет Санкт-Петербургского
университета. В Питере молодые люди и нашли свою судьбу.

Лидию и Александра друзья в шутку называли \enquote{Бубликом с дырочкой}.

\ii{01_12_2021.fb.fb_group.story_kiev_ua.1.familnaja_lozhka.pic.mlc.1}

\headTwo{В ЦЕНТРАЛЬНОЙ РАДЕ}

После февральской революции Александр и Лидия вернулись в Киев, и Александр
Яковлевич Шульгин, как человек авторитетный и уважаемый в среде украинофилов,
возглавлял в Центральной Раде министерство иностранных дел. Там у него возникли
разногласия с коллегами. 

Если руководство Рады ориентировалось на Германию, то Александр считал, что УНР
следует искать поддержки у Англии и Франции. После поражения Центральной Рады
Шульгин служил в Министерстве иностранных дел у гетмана Скоропадского, после
был послом в Болгарии. Оказавшись в эмиграции, Александр Яковлевич Шульгин
продолжал активно заниматься политикой, писал книги, вел научно -
исследовательскую работу.

\headTwo{\enquote{КИЕВЛЯНИНЪ} ВАСИЛИЯ ШУЛЬГИНА}

Следует заметить, что в семье Шульгиных достаточно близкие родственники имели
диаметрально противоположные политические взгляды и на этой почве даже прервали
общение.

Самым знаменитым из Шульгиных-русофилов был Василий Витальевич Шульгин -
дворянин, политик, предприниматель, редактор популярнейшей газеты \enquote{КиевлянинЪ}.
Шульгин неоднократно избирался во Всероссийскую Государственную Думу и жизнь
его была похожа на увлекательный приключенческий роман. 

Он уходил добровольцем на Русско-японскую войну,  был лично знаком с царем
Николаем Вторым и принимал отречение Государя от престола. Позже был идеологом
Белого движения и к нему с большим уважением относились Деникин и Врангель.
Находясь в эмиграции, Шульгин тайно посещал Советский Союз. Об этом даже был
снят фильм \enquote{Операция Трест}.

В 1944 году на территории Югославии Шульгин был арестован Советскими войсками и
находился в застенках более десяти лет.

Будучи лично знакомым с заместителем Андропова Филиппом Бобковым, в 1961 году,
в числе гостей, Василий Витальевич Шульгин присутствовал на XXII съезде КПСС. 

Невозможно в кратком сюжете описать все его приключения! 

\headTwo{ГИБЕЛЬ МОЛОДЫХ ШУЛЬГИНЫХ}

При похожих обстоятельствах (но за разные идеалы) в 1918 году погибли двое
Шульгиных из молодого поколения - Владимир (брат Александра Шульгина - министра
иностранных дел Центральной Рады) и Василид (так в семье называли сына Василия
Витальевича Шульгина - редактора газеты \enquote{КиевлянинЪ}).

\headTwo{ВЛАДИМИР}

Владимир Шульгин был лидером украинофильского студенчества и погиб под станцией
Круты, когда на Киев наступали красные войска Муравьева.

Про Круты многое написано, даже фильм снят. Там, по причине предательства
командования, погибли молодые парни из Студенческого куреня - воинство
Центральной Рады.

Владимиру на момент гибели было 23 года.

\headTwo{ВАСИЛИД}

Василид Шульгин погиб в декабре 1918, когда Директория подняла восстание против
гетмана Скоропадского и на Киев наступал Петлюра. Эти события описаны у
Булгакова в \enquote{Белой гвардии}. Тогда для обороны Города стали формироваться
боевые дружины, в одну из которых - \enquote{Орденскую} записался Василид. \enquote{Орденская
дружина} держала рубеж обороны у деревни Софиевская Борщаговка.

Командование бежало, оставив мальчишек на линии огня, и они все погибли от
петлюровских пуль.

Василиду было всего 19 лет.

Василид Шульгин похоронен на Байковом кладбище, Владимир Шульгин - на
Лукьяновском.

\ii{01_12_2021.fb.fb_group.story_kiev_ua.1.familnaja_lozhka.cmt}
