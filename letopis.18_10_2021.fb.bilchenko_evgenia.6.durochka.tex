% vim: keymap=russian-jcukenwin
%%beginhead 
 
%%file 18_10_2021.fb.bilchenko_evgenia.6.durochka
%%parent 18_10_2021
 
%%url https://www.facebook.com/yevzhik/posts/4356005064434570
 
%%author_id bilchenko_evgenia
%%date 
 
%%tags bilchenko_evgenia,poezia
%%title БЖ. Дурочка
 
%%endhead 
 
\subsection{БЖ. Дурочка}
\label{sec:18_10_2021.fb.bilchenko_evgenia.6.durochka}
 
\Purl{https://www.facebook.com/yevzhik/posts/4356005064434570}
\ifcmt
 author_begin
   author_id bilchenko_evgenia
 author_end
\fi

БЖ. Дурочка

\ifcmt
  ig https://scontent-lga3-1.xx.fbcdn.net/v/t1.6435-9/246820924_4356004957767914_3456400589004055091_n.jpg?_nc_cat=104&ccb=1-5&_nc_sid=8bfeb9&_nc_ohc=zhZzDxTfhtYAX92BIeo&_nc_ht=scontent-lga3-1.xx&oh=2749cdb97d5e9eb83afa622798c2caf2&oe=6193C835
  @width 0.4
  %@wrap \parpic[r]
  @wrap \InsertBoxR{0}
\fi

Сколько я себя помню, французские редакторА
(Мне нравится эта форма множественного числа)
Говорили мне, что поэзия -  не жизнь, а в неё игра,
Что действуют угнетающе мои русские свет и мгла.
Сколько я себя помню, английские вебофсайнс
(Мне нравится этот индекс слитным писать русизмом)
Говорили мне, что в науке - не до есенинского овса,
А Флоренского философия - вся расистская.
Сколько я себя помню, поклонники Бродского и Набокова,
Почему-то сейчас собирающиеся у Навального,
Говорили мне, что о военных писать - ремесло убогое,
Что нет в нем изящества слога, который лавирует, а не валит.
Но я видела столько жизни без сантиментов
И столько смерти, что не вмещается в форму метра -
Классическую, нормальную, - что даже малая толика
Того, что я видела, кое-как протискивается в тонику.
Сколько я себя помню, эмигранты, воспитанные на Галиче
И ныне живущие в Мюнхене, шарахались от плеера моего,
Когда на весь этот welfare рай я, противная, гадкая,
Врубала Егора Летова, оголяя свое ужасное естество.
Наверное, я - непутевая, и никогда ничего трендового,
Тонкого и пикантного не выйдет из стихотворной моей фигни.
Но я слышу, как плачут в земле кости, я слышу, как птицы требуют
Смастерить из костей им крылья, чтобы смогли они
Долететь туда, где грибные дожди идут в голове у Штирлица.
Туда, где безграмотный дед Мазай отпускает на небо зайцев.
И от всего вот этого вот, родного, мне дико штырится...
Не зовите на Запад. Я есть. И быть - мне сподручнее, чем казаться. 
18 марта 2021 г.
