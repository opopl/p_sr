% vim: keymap=russian-jcukenwin
%%beginhead 
 
%%file 28_09_2021.fb.pozyvnoj_adler.1.kniga_ekskavator
%%parent 28_09_2021
 
%%url https://www.facebook.com/permalink.php?story_fbid=4760364887308510&id=100000051389055
 
%%author_id pozyvnoj_adler
%%date 
 
%%tags biblia,chelovek,kniga
%%title Ещё удары, и снова книга...
 
%%endhead 
 
\subsection{Ещё удары, и снова книга...}
\label{sec:28_09_2021.fb.pozyvnoj_adler.1.kniga_ekskavator}
 
\Purl{https://www.facebook.com/permalink.php?story_fbid=4760364887308510&id=100000051389055}
\ifcmt
 author_begin
   author_id pozyvnoj_adler
 author_end
\fi

Оскаленный ковш огромного экскаватора крошит как скорлупу кирпичные стены
добротного дома...

Июль две тысячи двадцатого...

Уже не помню, какая волна вируса, следующего за ней карантина, следующего за
ним кризиса экономического, социального и т.д...

Огромная стальная челюсть в лучах беспощадного июльского солнца, с вынутыми на
всю клыками, не чувствуя никакого препятствия, разносит в пыль сложенные
прилежными мастерами более полувека назад безупречные стены дома и всё то, что
было внутри, укрыто этими прочными стенами...

Подобное зрелище гипнотизирует, как детей, так и взрослых...

Мне предстоит здесь прожить месяца четыре... В одиноком вагончике на пустыре...
Со стопкой книг... 

Увидеть дом и цветущий сад, увидеть руины дома и равнину пыли вместо цветущего
сада, увидеть огромный котлован на том месте, где некогда стоял дом и жил сад,
увидеть то, что едва начнёт возникать здесь после...

Пыль, стекло, разлетающиеся как спички, широкие тяжёлые балки, белоснежные
перья и пух подушек, перин, осколки посуды, фрагменты детских игрушек, обуви,
костюмов, платьев, в урагане взрывающихся кирпичных стен...

Я увидел книгу... Один из ударов её освободил, позволил встретиться с этим
небом, с солнцем, с моим взглядом... Чтобы через миг, удар следующий её
похоронил навсегда...

Времени для разгона машины перед следующим ударом было достаточно, чтобы успеть
шагнуть в руины и вернуться с книгой...

Том, страниц на восемьсот - девятьсот. "Лев Толстой". "Письма".

Ещё удары, и снова книга... 

"Повесть о настоящем человеке". 

Об известном лётчике, который был сбит, ранен, долго полз, потерял обе ноги, но
выжил...

Мои действия заставили людей нервничать... Подошёл прораб бригады, что после
должна была здесь работать... Сказал, что это очень небезопасно. Я ответил, что
всё нормально...

После очередного удара, в хаосе взрывающихся руин, в нескольких местах сразу,
мелькнули большие белые страницы. Страницы книги, которая была вскрыта прямым
ударом в неё.

Не знаю почему... Что-то сказало об их ценности... И не раздумывая, я шагнул в
руины... Скоро подбирая и идентифицируя найденное...

"От Марка"...

"Галатам"...

"Фессалоникийцам"...

И конечно, как в этих обстоятельствах без него: "Откровение Иоанна
Богослова"...

Не так уж много... В моих руках было страниц сорок фрагментов из этих глав...

Жара, пыль, эти пыльные обрывки...

Посетила вполне "разумная", "рациональная" мысль...

- "Если дома есть Библия, зачем нужны эти обрывки"...

Наличие их в руках, требовало добавлять к ним то, что возникало, обнажалось при
всех последующих ударах... Требовало автоматически и рефлекторно... 

Поэтому следующей "рациональной" мыслью, было вернуть уже собранное в эти
руины. Ведь - "Зачем они, если дома, в хорошем состоянии, есть вся эта
книга"... 

Чтобы освободиться и успокоиться, их нужно было вернуть в руины...

Вскоре, всё то, что подвергалось уничтожающим ударам, совсем потеряло характер
какой-то формы, став просто мусорной кучей... И тот самый экскаватор уже грузил
этим всем поочерёдно подъезжающие машины.

Было далеко до последней такой машины, когда я понял...

- "Ведь иногда, при ситуациях особых, именно эту книгу свойственно читать там,
где просто открыл"...

- "Когда нужна подсказка, в малом или в великом, многие считают нужным так
делать"...

- "Для тебя же, она была уже открыта, возможно, на нужных сегодня тебе
страницах"...

- "И ты ещё считал себя "иррациональным", "спонтанным", "неординарным"...

- "Тогда как, вот, экзамен на твою "иррациональность" и "поэтичность", и вот
как ты его сдал"...

- "Поступил как заурядный "рациональный" потребитель - "Зачем нужны эти
обрывки, если дома, в отличном состоянии, есть вся эта книга"...

- "Будь она целой, "закрытой", подобно другим, что подобрал, возможно именно в
этом случае её появление - ничего бы не значило"... "Подобной другим"... "Но
прямой удар пришёлся по ней одной"...

- "Да ведь этот, уничтожаемый на моих глазах дом, сегодня, в июле две тысячи
двадцатого, лучшая метафора на уничтожаемый сегодня привычный мир"... "Как эта
огромная, стальная уничтожающая машина, лучшая метафора на рок, губящий сегодня
привычный мир"... "Лучшая метафора на приговор могущественных, рациональных
сил, в отношении этого привычного мира"...

- "Три книги"... "Автор первой, известный мыслитель"... "Вторая о воине,
солдате, герое"... "Третью, открытую, вскрытую, принято считать божественной и
святой"...

- "То есть что"... "Крушение мира должно высвободить, призвать к жизни эти три
касты - Мыслителей, Героев, Святых"... 

- "Отмена приговора" и "Новое начало", требуют выхода из руин - Героев,
Мыслителей и Святых"...

- "Но то, что вскрыта была только одна книга, что удар пришёлся по ней одной,
приведя в хаос порядок её страниц, ведь это говорит о том, что самый важный
вопрос сегодня поставлен в мире - Духа, что главный поиск сегодня должен
вестись там"... "И если прямой удар пришёлся в это место, значит здесь
Сопротивление слабее, уязвимее, травматичнее всего"... "Значит всё ждёт
"Тайного Слова", всё ждёт "платформы Духа"...

- "И если бы я вовремя всё понял, собрал и прочитал, возможно, мой поиск был бы
вскоре справедливо завершён"... "Но то, что я ошибся, не заметил, вовремя не
понял, невозместимая утрата тех страниц, теперь потребует искать даже тогда,
когда казалось бы нашёл"... "Сиюминутная ошибка в перспективе даст мне больше,
чем своевременный ответ"...

- "Да и в конце концов, если всё это, всего лишь, ничего не значащая, банальная
случайность, даже в этом случае от неё есть благо - её последствия не банальны,
не случайны, и вероятно - значительны"...

........................................

Времени на телефон было более чем достаточно... Поэтому, благодаря пересказу
произошедшего, обвинениям в свой собственный адрес, перечню утраченного, игре
воображения и возникающей в пересказе образной трактовке произошедшего, я
хорошо запомнил главы, к которым принадлежали утраченные страницы, и в
наслаиваемых одно на другое объяснениях, я, возможно получил больше, чем
потерял.

Только освободившись от уже запланированного, расписанного наперёд, чтива,
решил перечитать не только эти главы, но произведение в целом...

Так совпало, что это произошло в последние дни перед Пасхой...

Сложное время требует сложных построений, очень сложное, требует очень
сложных... И возможно, всё дело в построении этой эффективной и актуальной
сложности... Всё дело в этом...

Но, какова бы ни была эта сложность, безусловно ясно одно...

Чтобы противостоять холодному, рептильному индивидуализму тех, техническое
оснащение которых позволяет казаться роком, индустриальное и постиндустриальное
могущество которых позволяет казаться приговором миру, приговором нам...

Отмена "окончательного приговора" и "новое начало"...

Требуют наличия под этими "эффективными" и "актуальными" построениями простого
и вечного -

"Имейте в себе соль, и мир имейте между собою"...

"Если бы вы были от мира, то мир любил бы своё; а как вы не от мира, ..........
потому ненавидит вас мир"...

"Нет больше той любви, как если кто положит душу свою за друзей своих"...

"Ибо кто не против вас, тот за вас"...

"И вот, есть последние, которые будут первыми, и есть первые, которые будут
последними"...

"и деньги у меновщиков рассыпал, а столы их опрокинул"...

"и выгнал всех продающих и покупающих в храме, и опрокинул столы меновщиков и
скамьи продающих"...

"Почитающиеся князьями народов господствуют над ними, и вельможи их властвуют
ими. Но между вами да не будет так: а кто хочет быть большим между вами, да
будет вам слугою"...

"но теперь, кто имеет мешок, тот возьми его, также и суму; 

а у кого нет, продай одежду свою и купи меч; 

ибо сказываю вам, что должно исполниться на Мне и сему написанному: "и к
злодеям причтён".
