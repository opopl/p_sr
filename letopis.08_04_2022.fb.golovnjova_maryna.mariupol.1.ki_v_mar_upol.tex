%%beginhead 
 
%%file 08_04_2022.fb.golovnjova_maryna.mariupol.1.ki_v_mar_upol
%%parent 08_04_2022
 
%%url https://www.facebook.com/m.holovnova/posts/pfbid035VMF8cYidA5LqpMG5PCb7dEh3A5wMpTjPLJxUqi1BvjKx3zo3ZB7Q4uZDpMf6atDl
 
%%author_id golovnjova_maryna.mariupol
%%date 08_04_2022
 
%%tags mariupol,mariupol.war,zheleznaja_doroga,poezd,kiev
%%title Київ-Маріуполь
 
%%endhead 

\subsection{Київ-Маріуполь}
\label{sec:08_04_2022.fb.golovnjova_maryna.mariupol.1.ki_v_mar_upol}

\Purl{https://www.facebook.com/m.holovnova/posts/pfbid035VMF8cYidA5LqpMG5PCb7dEh3A5wMpTjPLJxUqi1BvjKx3zo3ZB7Q4uZDpMf6atDl}
\ifcmt
 author_begin
   author_id golovnjova_maryna.mariupol
 author_end
\fi

Київ-Маріуполь

Увечері 23 лютого з Києва вирушив останній мирний потяг на Маріуполь. Я була в
ньому. Сусіди по плацкарту купили пиво, намагались пригостити і мене. Усі були
в доброму гуморі та пили за мир. Наступного ранку ми прокинулись о 5:30 від
тремтячого голосу провідниці. Вона швидко ходила по вагону і будила всіх,
плакала, говорила щось про обстріли під Волновахою, просила опустити шторки на
вікнах. 

Почалося.

Потяг простояв просто посеред поля 5 годин, але таки дістався місця
призначення. У цей самий час відправлялись евакуаційні рейси в напрямку Дніпра,
але на вокзалі було пусто. Потім виявиться, що всі чотири рятівні потяги, які
вивозили людей 24-25 лютого, вирушали практично порожніми. 

Тоді нам усім думалось, що Маріуполь, будучи форпостом сходу країни протягом
останніх 8 років, може виявитися найбезпечнішим і найзахищенішим містом серед
усіх. Тому з того київського потяга всі сходили на платформу з відносним
спокоєм на серці. Ми вдома, ми злі, ми нікуди не поїдемо. 

Але ми помилились. Коли на твоє місто скидають від 60 до 100 бомб щодня, тобі
не лишається нічого, крім як лежати на підлозі, ховатися в темному холодному
підвалі, відчувати приниження і безпорадність. 

Найкращою допомогою військовим з нашого боку було би покинути місто. У нас таку
можливість забрали. Зокрема мер, який від початку оптимістично радив
маріупольцям зберігати спокій і міцно спати замість того, аби їхати
якнайшвидше. Сам він, до речі, втік у супроводі озброєної охорони в перший же
день.

Частина міста, у якій я жила, першою опинилася під обстрілами російських
градів. Наш район чув вибухи регулярно з 2014 року, але вони здавались дуже
далекими і всі звикли. Тепер вони наближалися. 

На другий день війни російський снаряд уперше приземлився в нашому кварталі та влучив у мою школу. Потім ще три – в сусідні двори. Майже в усьому під'їзді вибило вікна. Світла та зв'язку не стало одразу. Без зволікань ми зібрали рюкзаки і всліпу вирушили на вокзал, сподіваючись, що там буде евакуація хоч кудись. На коліях стояли два київських потяги, які мали поїхати ще 5 годин тому, але під Волновахою знов велися бої. Пасажири залишилися спати у вагонах і сподіватися, що рейс відбудеться.

Ми провели ту ніч на вокзалі, сподіваючись, що його не обстріляють і що зранку таки вдасться виїхати. Спати від холоду було неможливо. Чергова Інна запросила погрітися до себе в кімнатку, пригостила чаєм. Вона з села під Маріуполем, на роботу діставалась електричкою. Того дня усі побоялися виходити на зміну через обстріли, а вона приїхала. В Інни четверо дітей, старший син на передовій. Вона обожнювала свою роботу, досягла уже певних успіхів і чекала на підвищення. 

Але це була її остання зміна, тому що тої ночі залізничний вузол у Волновасі було знищено росіянами, і єдиний залізничний шлях до Маріуполя перерізано назавжди. Ми дізнались про це зранку, поїхали назад у свою квартиру і провели там ще дві доби на підлозі у ванній без світла, зв'язку і опалення. Це були наші останні дні вдома. Місяць потому від нього нічого не залишиться.

На четвертий день розриви снарядів стали нестерпно близькими і частими, і було вирішено перебиратися в центр, який тоді ще здавався безпечним. Але через два дні російська ракета вперше влучить і в нього.

Саме відтоді Драмтеатр почав приймати тих, хто втрачав свої домівки. За кілька днів там уже був зайнятий кожен метр на усіх трьох поверхах і в підвалі. У театрі сформувалася активна група волонтерів на чолі з акторами, була навіть польова кухня. Гардеробна стала пунктом харчування, до якого мешканці годинами стояли в черзі, аби отримати стакан супу, гарячу воду та печиво. Якщо вистачить. 

Годували в основному тим, що військові ЗСУ та Азовці знаходили по магазинах і складах. Тримаючи лінію оборони, вони встигали подбати і про нас і майже щодня привозили їжу, ліки, теплі речі і, по можливості, добрі новини.

Одного дня хлопці приїхали з матрацами, ковдрами і подушками. Справжні скарби!
Ми придивились і зрозуміли, звідки їх взяли. З тих самих потягів на Київ, які
так і залишалися на коліях з 25 лютого. Вони все-таки врятували людей. Хоча би
від холоду.

p.s. дякую, \href{https://www.facebook.com/profile.php?id=100013159908154}{Кіндрат Дармограєнко}, за цінне пам'ятне фото.

%\ii{08_04_2022.fb.golovnjova_maryna.mariupol.1.ki_v_mar_upol.cmt}
