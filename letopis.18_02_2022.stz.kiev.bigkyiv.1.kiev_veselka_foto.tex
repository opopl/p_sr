% vim: keymap=russian-jcukenwin
%%beginhead 
 
%%file 18_02_2022.stz.kiev.bigkyiv.1.kiev_veselka_foto
%%parent 18_02_2022
 
%%url https://bigkyiv.com.ua/dyva-lyutogo-u-nebi-nad-kyyevom-syayala-veselka-foto
 
%%author_id lisichkіna_ljubava
%%date 
 
%%tags kiev,krasota,raduga
%%title Дива лютого. У небі над Києвом сяяла веселка (ФОТО)
 
%%endhead 
 
\subsection{Дива лютого. У небі над Києвом сяяла веселка (ФОТО)}
\label{sec:18_02_2022.stz.kiev.bigkyiv.1.kiev_veselka_foto}
 
\Purl{https://bigkyiv.com.ua/dyva-lyutogo-u-nebi-nad-kyyevom-syayala-veselka-foto}
\ifcmt
 author_begin
   author_id lisichkіna_ljubava
 author_end
\fi

Веселку сьогодні бачили і на Хрещатику, і на Нивках, і в Бучі.

Фотографії незвичного для зими природнього явища містяни з радістю публікують у
соцмережах. 

\ii{18_02_2022.stz.kiev.bigkyiv.1.kiev_veselka_foto.pic.1}

У дописах до фото кияни пишуть, що  побачити веселку – то є добрий знак.
Прикмети про веселку різні, але їх об’єднує одне — побачити веселку до добра,
до щастя, до везіння.

За прикметами, зимова веселка — одна з найщасливіших прикмет. Якщо людина
побачила взимку веселку,  сміливо можна починати задумане, удача буде
супроводжувати у всьому.  Зимова веселка ніби  дає  знак, що все обов’язково
вийде.

На фото видно, що веселка простягається над центром міста, поки надворі вже
починає сутеніти.

Щасливими свідками природного дива стали мешканці Нивок:

\ii{18_02_2022.stz.kiev.bigkyiv.1.kiev_veselka_foto.pic.2}

\ii{18_02_2022.stz.kiev.bigkyiv.1.kiev_veselka_foto.pic.3}
