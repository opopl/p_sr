% vim: keymap=russian-jcukenwin
%%beginhead 
 
%%file 05_08_2021.fb.bandrivskij_nikolaj.1.galickaja_rus_otcy_dominikancy
%%parent 05_08_2021
 
%%url https://www.facebook.com/permalink.php?story_fbid=5834129713325306&id=100001850064547
 
%%author 
%%author_id bandrivskij_nikolaj
%%author_url 
 
%%tags istoria,kievrus,rusj.galickaja,ukraina
%%title Галицька Русь - отці-домініканці
 
%%endhead 
 
\subsection{Галицька Русь - отці-домініканці}
\label{sec:05_08_2021.fb.bandrivskij_nikolaj.1.galickaja_rus_otcy_dominikancy}
 
\Purl{https://www.facebook.com/permalink.php?story_fbid=5834129713325306&id=100001850064547}
\ifcmt
 author_begin
   author_id bandrivskij_nikolaj
 author_end
\fi

Чому на Галицькій Русі нашими князями була заблокована діяльність отців-домініканців, які були першокласними фахівцями у боротьбі з проявами сатанізму..?
.
Всі ми знаємо, що отці-домініканці вже через сімнадцять років після заснування свого ордену, рушили на Русь. Наприклад, у 1233 році в Києві отці-домініканці вже мали свій монастир, а у Львові, за повідомленням одного з хроністів ордену, ченець-домініканець Яцек Одровонж у 1234 році керував тут  будівництвом окремого храму.
.
Звичайно ж, без запрошення чи спеціального дозволу князя і за погодженням з Римом, ніхто б не дозволив отцям-домініканцям навіть переступити межу Галицької держави, чи, Русі, в цілому. Значить, мусіла бути першопричина або ж проблема, причому - поважна і мабуть доволі пекуча, якщо для її вирішення вирішили запросити саме отців-домініканців (їх інша, офіційна назва: брати-проповідники).
.
А, при чому тут цей жебручий орден братів-проповідників  до Галицької Русі? - спитаєте ви. 
.
Почати слід з того, що затверджуючи Статут ордену домініканців, папа Григорій ІІІ спеціально доручив їм боротися з єретичними вченнями і всім тим, що називаємо сатанинськими проявами. Саме із членів цього ордену Ватикан, в усі наступні століття, призначав Головних інквізиторів (на той момент брати-проповідники вже заповзялися викорінювати антикатолицьку єресь альбігойців на півдні Франції). Одним словом - там, де з'являлися домініканці, невдовзі з"являлися й інквізиторські суди.
.
Висилаючи отців-домініканців у спеціальну місію на Русь, папа Григорій IX, одночасно, доручив їм з’ясувати можливості утворення римо-католицького єпископату на руських землях. Для цього ще в Римі папа затвердив кількох «єпископів без постійної кафедри». Однак, не всюди на Русі, домініканцям були раді. Так, один з хроністів ордену свідчив, що у середині XIII століття в Києві було вбито 92 домініканських ченці, у Галичі — 27, у Житомирі — 4, у Коломиї — 4, у Теребовлі — 8, Новогрудку — 6, Любартові — 7, а в околицях Жовкви біля Львова, був повішений пріор з усім конвентом...
.
А, отут, власне, й виникає питання: що ж такого робили, чи говорили, на Русі ці, вбрані у білі плащі, отці-домініканці, якщо їх присутність викликала такий, немислимий за масштабами, спротив..?
.
Відповідь може бути лише одна: домініканці в Галичині лише сумлінно виконували свою справу, якої вони навчилися у своїх західноєвропейських колегіумах і університетах, а саме: відстежувати і викорінювати сатанинські прояви та різну антирелігійну єресь.
.
Наші ж галицькі князі, почавши будувати з початку ХІІ століття  білокам"яні величні і гарно оздоблені храми (причому - десятками), разом з тим зіткнулися зі сильною поганською ортодоксією: галицькі язичники й надалі приносили у своїх капищах людські жертвоприношення, а у своїх загумінкових святилищах жерці відмовляли людей від Христового вчення. Своїх же університетів, Галицька держава в той час, не мала і як побороти те мракобісся, яке дозволяло спалювати людей живцем і приносити жертви кам'яним божищам, не знала...
.
А, якраз, саме на той момент, отці-домініканці на півдні Франції доводили на практиці ефективність своїх методик, зачистивши "під нуль" єресь катарів у Лангедоку... Отож, вибір - хто б це мав щось подібне робити на Русі, був невеликий.
.
Тому й вирішили наші князі за краще звернутися до фахівців, які б допомогли навести лад  у духовно-обрядовому житті нашого краю. Мабуть, згодом не все пішло так, як задумувалося на початку. Тому, подекуди, серед частини галицьких русинів й виявилася та ненависть до отців-домініканців, яка вилилася у їх масові знищення (див. вище).
.
Судячи з усього, вінценосні правителі тогочасної Галичини, спостерігши, який спротив викликає діяльність запрошених домініканців, почали пригальмовувати отой процес "очищення краю". До чого це призвело - неважко здогадатися: писемні джерела наступних століть переповнені свідченнями про різні марновірства і нехристиянську обрядовість по галицьким селам.
.
Ось, наприклад, що писав один з найблискучіших краєзнавців Галичини ХІХ століття - львів'янин Антоні Шнайдер, коли він у 1840-их роках переглядав в економічному архіві Собеських у Жовкві інструкції краківського каштеляна Якуба Собеського до порядників та економів ключа Золочівського й Озерянського. Між різними вказівками, він знайшов спеціальне нагадування  «...щоби забороняли підданим перегринації на Бабі біля божища». Ця інструкція середини XVII століття доказує, писав Шнейдер, що «...в давні часи нарід з тої околиці був проникнутий великою шанобою до того божества». Тобто, навіть вже у модернову епоху наші селяни так і не позбулися тих відьмацьких звичаїв. 
.
Біля села Паньківці біля Бродів на Львівщині, - продовжував Антоні Шнайдер, довший час стояло кам'яне божище, яке в народі називали «Бабою»: "...Воно нагадувало ніби дівчину в кожушку, з виступаючими персами, в ярмулці чи чепцю на голові, з опущеними руками. Ноги були відбиті. У персах жіночого божества були дірки. Переказували, що давніше молодь, аби розбудити між собою кохання, виколуплювала звідти порох. Те, що зішкробували з внутрішньої поверхні, вживали за різноманітні ліки. Про відсутність ніг у баби селяни казали, що їх «чорт»» бабі відібрав через те, що часто йому докучала, і закинув десь далеко аж до Чистопад" (сусіднього села - М.Б.).
.
Усі ці речі діялися з нашим людом не десь у глибокому Середньовіччі, а, ще, якихось, сто-двісті років тому... 
.
Тому й подумалося: якби наші князі, у свій час, не здрейфили і таки дали можливість отцям-домініканцям завершити їх розпочату працю в Галицькій Русі, то й розвивались би ми, в наступні століття, значно швидше і цивілізованіше, аніж інші наші сусіди.
.
Чого ж тоді дивуватися, що останню "відьму" на нашому Підгір"ї, галицькі селяни спалили у середині ХІХ століття, коли Європою і Америкою вже їздили залізничні потяги, робила свої перші кроки фотографія, а наш славний  Іван Франко вже  починав свій гіркий життєвий шлях у нашій, благословенній Богом, Галичині...
.
.
.
.
.
.
.
.
.
.
.
.
.
.
.
.
P.S.
Трапилась, на очі, цитата з листа папи римського Григорія ІХ від 18 липня 1231 року, в якому той, в черговий раз закликає князя Данила опам'ятатися, вийти зі схизми (тобто, з Православ'я) і навернутися до католицької Церкви:
.
"...Як дізнались ми з повідомлення шановного брата нашого єпископа пруського, ти — християнський князь, але дотримуєшся грецьких і руських звичаїв та обрядів і вимагаєш цього від тих, що живуть у твоєму королівстві, але тепер, натхнений Божою милістю, бажаєш перейти до поклоніння й покірності апостольському престолові і нам. Ми ж, з найглибшою любов’ю турбуючись про спасіння твоєї душі та твою найбільшу вигоду і славу, вмовляємо і закликаємо твою світлість в ім’я Господа, щоб ти не відкидав правильного вчення, але, пройнявшись благочестям, побожно прийняв обряд і звичаї християн-латинян і дотримувався їх, віддавши себе і своє королівство, заради любові до Христа, під солодку владу Римської церкви... Дано в Рієті 18 липня у  п’ятий рік нашого понтифікату». 
.
Ще одному листі від 3 травня 1246 році вже іншого папи - Інокентія, той просить галицького князя Данила надати допомогу особам, яким папа доручив зібрати інформацію про монголів. В одному з послань, який аналізує С.Большакова, папа Інокентій брав державу князя Данила під свою опіку.
Тоді-ж, 1246 році папа Інокентій направив одному з прибалтійських єпископів листа, в якому надає йому право призначати (!) єпископів та священиків на Русь, тобто, на думку О.Головко, фактично папа прагнув розпочати процес остаточного «навернення» Галицької Русі до латинства.
.
А навів вище я цю цитату тому, що саме цього 2021 року виповнюється 790 літ відтоді, як з канцелярії Ватикану вийшов лист за підписом і печаткою тогочасного понтифіка, який мав на меті радикально - раз і назавжди - змінити вектор духовного розвитку Галицького-Волинського князівства.
.
Як знаємо - не сталося... 
.
Обман з боку Риму (обіцяв нашому князеві підмогу проти татаро-монголів, але обдурив), міжусобні чвари князів, врешті, Батиєва навала 1241 року, швидко звели нанівець усі плани того, хто в одному зі своїх титулів, гордовито величав себе "намісником Бога на землі"...
.
Використана література:
.
 Головко О. Держава Романовичів у східноєвропейській політиці римської курії 40-50-х років ХІІІ ст.// Княжа доба.-Львів, 2008.-вип.2.-с.74-75;
.
Дашкевич Н.П. Переговори пап с Даниилом Галицким об унии юго-западной Руси с католичеством// Киевские университетские известия.-Киев,1884.- №8.-с.136-181;
.
Бандрівський М. З Рима на руські землі // Ратуша.-Львів, 1999,-20.ХІ.- №34.-С.9; він же: Від Мегалі Екклісія до церкви Успіння Пресвятої Богородиці на вул. Руській у Львові.-Львів, 2014.-с.17-18
.

Використано матеріали з ресурсів:
.
https://kzd.org.ua/men.../orden-propovidnykiv-dominikanci
.
\url{https://risu.ua/slidami-psiv-gospodnih-na-ukrajini_n40199}
.
http://www.vox-populi.com.ua/.../molitvidokamanihbogivavt...
.
https://www.pravmir.ru/inkvizitsiya-prosto-statistika/
.
https://history.wikireading.ru/hYu5PH7GGl
.
.
