% vim: keymap=russian-jcukenwin
%%beginhead 
 
%%file 27_11_2021.stz.news.ua.misto_marii.1.velomarshrut_step_bike_step
%%parent 27_11_2021
 
%%url https://mistomariupol.com.ua/uk/vid-mariupolya-do-arabatskoyi-strilky-novyj-velomarshrut-stepamy-pryazovya
 
%%author_id news.ua.misto_marii
%%date 
 
%%tags 
%%title Веломаршрут степами Приазов'я "Степ Bike Step"
 
%%endhead 
 
\subsection{Веломаршрут степами Приазов'я \enquote{Степ Bike Step}}
\label{sec:27_11_2021.stz.news.ua.misto_marii.1.velomarshrut_step_bike_step}
 
\Purl{https://mistomariupol.com.ua/uk/vid-mariupolya-do-arabatskoyi-strilky-novyj-velomarshrut-stepamy-pryazovya}
\ifcmt
 author_begin
   author_id news.ua.misto_marii
 author_end
\fi

\ifcmt
  ig https://i2.paste.pics/PI0SW.png?trs=1142e84a8812893e619f828af22a1d084584f26ffb97dd2bb11c85495ee994c5
  @wrap center
  @width 0.7
\fi

\begin{quote}
\em
Днями відбулася презентація довгоочікуваного веломаршруту степами Приазов'я під
назвою \enquote{Степ Bike Step}, розробленого за підтримки організацій \enquote{Спільно HUB} та
USAID.	
\end{quote}

\ii{27_11_2021.stz.news.ua.misto_marii.1.velomarshrut_step_bike_step.pic.1}

Робота над проектом тривала понад півроку. У розробці взяли участь Запорізький
велоклуб, Halabuda (Маріуполь), MeloVeloCity (Мелітополь), Бердянський велоклуб
та Імпульс Херсонщини. Головна ідея цього проекту – повернути відчуття єдності
та цілісності країни, а згодом і об'єднати веломаршрутом тимчасово окуповані
наразі території Донецька і Криму.

Сьогодні маршрут починається з виїзду з Маріуполя у напрямку Белосарайської
коси і завершується озером Сиваш, біля Генічеська.

Маршрут розтягнувся на 377 км, розрахований приблизно на тиждень, але можна
долати його і частинами. \enquote{Степ Bike Step} проходить територією трьох
областях – Донецької, Запорізької та Херсонської.

\ii{27_11_2021.stz.news.ua.misto_marii.1.velomarshrut_step_bike_step.pic.2}

\enquote{Створенням веломаршруту ми також наголошуємо, що велосипед – це не тільки вид
транспорту для пересування у містах, але й комфортний та екологічний засіб для
подорожей на дальні відстані}, – зазначають автори ідеї.

У назві \enquote{Степ Bike Step} кожне слово має значення: Степ – це, вочевидь, степова
територія, Bike – велосипед, і Step – крок до об'єднання. Організатори мріють
про те, що колись маршрут можна буде продовжити: на південь від озера Сиваш до
самого Криму та від Маріуполя – на північ до Донецька.

\ii{27_11_2021.stz.news.ua.misto_marii.1.velomarshrut_step_bike_step.pic.3}

Аби забезпечити подорожуючим належний комфорт, на відстані кожних 40-50
кілометрів було встановлено \enquote{байкпоінти} – зручні велосипедні альтанки з
навісами, велопарковками, лавками та столиками. Там можна перепочити, розпалити
багаття та встановити намети на ніч.

\ii{27_11_2021.stz.news.ua.misto_marii.1.velomarshrut_step_bike_step.pic.4.spilnohub}

Веломандрівники матимуть змогу проїхати неймовірно живописними дорогами
степного ландшафту, мальовничим узбережжям Азовського моря, курортними містами
Приазов'я, повз національні парки та лимани. Аби охопити більше красивих
природних місць, розробники маршруту намагалися прокладати його в об'їзд
завантажених трас твердими ґрунтовими дорогами.

\ii{27_11_2021.stz.news.ua.misto_marii.1.velomarshrut_step_bike_step.pic.5}

Із мапою маршруту можна ознайомитися на сайті проекту і там же завантажити трек
і координати всіх байкпоінтів на телефон.

