% vim: keymap=russian-jcukenwin
%%beginhead 
 
%%file 27_10_2022.stz.news.ua.donbas24.1.chy_ljubyv_marko_kropyvnyckyj_priazovja_stvorennja_ukr_realistic_teatr.1
%%parent 27_10_2022.stz.news.ua.donbas24.1.chy_ljubyv_marko_kropyvnyckyj_priazovja_stvorennja_ukr_realistic_teatr
 
%%url 
 
%%author_id 
%%date 
 
%%tags 
%%title 
 
%%endhead 

\subsubsection{Коли Марко Кропивницький почав приїжджати до Приазов'я?}

Наприкінці XIX століття особливе місце в театральному житті Маріуполя належало
саме українському музично-драматичному театру. Гастролі великих майстрів
української сцени \textbf{Марка Кропивницького, Івана Карпенка-Карого, Панаса
Саксаганського, Марії Садовської-Барілотті} дали можливість містянам
ознайомитися з національною театральною драматургією. Цікаво, що \enquote{Товариство
російсько-малоросійських артистів} під керівництвом П. Саксаганського першим
містом для своїх виступів обрало Катеринослав, а другим — Маріуполь. Це
обумовлювалося високим рівнем театральної культури Маріуполя та вдячністю
глядачів. Найбільшу популярність серед маріупольців мали вистави Марка
Кропивницького. У Музеї театрального, музичного та кіномистецтва України
зберігається годинник із лаконічним написом: \emph{\enquote{Кропивницькому від маріупольців
29.09.1889 р.}}. Цей подарунок є матеріальним свідченням того, що в 1889 р. М.
Кропивницький вперше виступав у Маріуполі.

\ii{27_10_2022.stz.news.ua.donbas24.1.chy_ljubyv_marko_kropyvnyckyj_priazovja_stvorennja_ukr_realistic_teatr.pic.1}

\ii{insert.read_also.demidko.donbas24.u_kyevi_pokazhut_vystavu_pro_mariupol}

\ii{27_10_2022.stz.news.ua.donbas24.1.chy_ljubyv_marko_kropyvnyckyj_priazovja_stvorennja_ukr_realistic_teatr.pic.2}

Також того ж 1889 року Марко Лукич взяв до своєї трупи \textbf{Любов Ліницьку}, яка була
в маріупольській трупі \textbf{Василя Шаповалова} на перших ролях і мала бенефіс.
Найбільший успіх вона мала у п'єсах: \enquote{Наталка Полтавка}, \enquote{Назар Стодоля},
\enquote{Невільник}, \enquote{Шельменко-денщик}, \enquote{Сватання на Гончарівці}. У неї навіть був
власний сценічний гардероб. До трупи одного із засновників професійного театру
приєднався і чоловік Ліницької \textbf{Іван Загорський}, який теж починав свій
акторський шлях у Маріуполі. При цьому, якщо його Кропивницький взяв до своєї
трупи одразу, то Любов Ліницьку — після випробувального терміну. Отже, деякі
маріупольські актори були запрошені в трупу М. Кропивницького, творця
реалістичного українського театру, що слугувало \emph{найкращою атестацією
маріупольському антрепренеру Василю Шаповалову як режисеру і організатору
театральної справи.}

Відомо, що у Маріуполі та його околицях трупа М. Кропивницького гастролювала й
у 1891 році впродовж декількох днів — з 9 по 14 серпня. Збереглися відомості
лише про одну виставу \textbf{\enquote{Зайдиголова}}, в якій М. Кропивницький — автор п'єси —
виконав роль Захарки Лободи. Разом з ним виступали такі видатні актори, як
\textbf{Ганна Петрівна Затиркевич-Карпинська і Леонід Якович Манько}.

\ii{insert.read_also.demidko.donbas24.jaki_teatr_proekty_mrpl_zakordon_mytci}

\ii{27_10_2022.stz.news.ua.donbas24.1.chy_ljubyv_marko_kropyvnyckyj_priazovja_stvorennja_ukr_realistic_teatr.pic.3}

16 липня 1892 року до Маріуполя знову приїхала трупа М. Кропивницького, яка
представила концерт-виставу за участю Л. Ліницької на користь чоловічого і
жіночого хору та танцюристів. Автор кореспонденції не випадково наголосив на
участі у концерті-виставі Л. Ліницької: \emph{маріупольці пам'ятали Любов Павлівну і
пишалися тим, що свій блискучий шлях актриси вона розпочала саме в їхньому
місті.}

\ii{27_10_2022.stz.news.ua.donbas24.1.chy_ljubyv_marko_kropyvnyckyj_priazovja_stvorennja_ukr_realistic_teatr.pic.4}
