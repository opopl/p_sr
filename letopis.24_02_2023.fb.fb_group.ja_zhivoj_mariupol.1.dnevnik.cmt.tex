% vim: keymap=russian-jcukenwin
%%beginhead 
 
%%file 24_02_2023.fb.fb_group.ja_zhivoj_mariupol.1.dnevnik.cmt
%%parent 24_02_2023.fb.fb_group.ja_zhivoj_mariupol.1.dnevnik
 
%%url 
 
%%author_id 
%%date 
 
%%tags 
%%title 
 
%%endhead 

\qqSecCmt

\iusr{Янбаева Янбаева}

Мой дневник остался в подвале...

\begin{itemize} % {
\iusr{Михаил Мирошниченко}
\textbf{Янбаева Янбаева} Главное что Вы там не остались. Мирного неба.
\end{itemize} % }

\iusr{Видимо Невидимо}

Виїжджав 19.03.22р. п. Парковий.

\iusr{Валентина Блакитная}

ВИВОЗИЛИ З ПОРАНЕННЯМ 16,02,22Р,

\iusr{Наталья Бандурко}

Мы выехали 18.03.

\iusr{Ирина Сариджа}

Выезжали 16.03 Парковый....

\iusr{Nana Lomsadze}

ჩვენ გამოვედიყ 24. 03

\iusr{Виктория Ветрова}

Виїхали 15 березня, вранці. Не можу згадувати без сліз. Вибачте! Сподіваюсь, що
дочекаємось мира та покарання винних. А ще маю мрію, що наша рідна країна
подолає, вижене всіх тих, хто вдягає то вишиванку, то триколор клятий.

\iusr{Elena Sharipova}

А мы выходили пешком из убежища школы 1 колонной под канонады всех видов
оружия. Тянули маму которой 72 года, которая уже не могла идти мы её за руки
вели. Через город шли два дня с остановками и в Мангуше были 22. 03.

\iusr{Лена Задоева}

Мы вышли из дома 22 марта, маме на тот момент было 81год, везли на инвалидной
каляски, до горбольницы 2, а там на автобус до Володарска

\iusr{Елена Якунина}

Выезжали на 2-х бусах 30.03.22 после загорания дома на Строителей. Вывезли всех
жильцов подьезда в село Урзуф. Вырвались чудом. А так ждали зеленый
коридор...😥

\iusr{Виктория Ветрова}

Мы выехали, слава богу, машиной: 92 года, 67, 38, 37 и 15. Три
собаки, дрова, одеяла. До сих пор не могу слышать гул самолёта и внезапные громкие
звуки. Никакого зелёного коридора не было. Была колона машин с белыми тряпками и
надписью \enquote{дети}, что не помешало обстреливать нас. Это было 15. 03. 22. 2500 машин
выбралось в тот день, точнее ночь из Мариуполя и доехало до Запорожья. Навсегда в
памяти останется тот чай, ужин, горячая вода и сон в тишине на детской кроватке в
детском садике. Огромное спасибо всем запорожцам за организацию приема нас,
беженцев из Мариуполя. И пусть сгинут наши вороженьки! Пусть их детям и старикам
будет то, что их отцы и сыновья принесли на нашу землю. Должно быть справедливое
возмездие. Ведь в жизни все 50 на 50. Мира и тишины всем мариупольцам!

\iusr{Вікторія Горбаченко}

Мы вышли 3.04. 2022. Перекрестив свой дом. Наш город казнила проклятая орда.

Я тоже, как и все мариупольцы, которые выжили и вышли из того ада, вздрагиваю
от громких звуков. А когда пролетает самолёт, всегда смотрю в небо. Это стало
привычкой.
