% vim: keymap=russian-jcukenwin
%%beginhead 
 
%%file 30_09_2017.stz.news.ua.mrpl_city.1.iz_istorii_mariupolskoj_biblioteki_im_korolenko
%%parent 30_09_2017
 
%%url https://mrpl.city/blogs/view/iz-istorii-mariupolskoj-biblioteki-im-korolenko
 
%%author_id burov_sergij.mariupol,news.ua.mrpl_city
%%date 
 
%%tags 
%%title Из истории Мариупольской библиотеки им. Короленко
 
%%endhead 
 
\subsection{Из истории Мариупольской библиотеки им. Короленко}
\label{sec:30_09_2017.stz.news.ua.mrpl_city.1.iz_istorii_mariupolskoj_biblioteki_im_korolenko}
 
\Purl{https://mrpl.city/blogs/view/iz-istorii-mariupolskoj-biblioteki-im-korolenko}
\ifcmt
 author_begin
   author_id burov_sergij.mariupol,news.ua.mrpl_city
 author_end
\fi

\ii{30_09_2017.stz.news.ua.mrpl_city.1.iz_istorii_mariupolskoj_biblioteki_im_korolenko.pic.1}

История старейшей в нашем городе общедоступной библиотеки имени Короленко (она
была открыта  в 1904 году) полна драматичных, а порой и трагических страниц.
Попытаемся воссоздать те из них, которые относятся к послевоенному периоду. И
сделаем это с помощью воспоминаний библиотекарей - людей, посвятивших свою
жизнь благородному делу просвещения народа.

\ii{30_09_2017.stz.news.ua.mrpl_city.1.iz_istorii_mariupolskoj_biblioteki_im_korolenko.pic.2}

Лидия Михайловна Дарда: "Старожилы города помнят  небольшой особняк, который
находился по ул. Греческой, 13. В этом особняке и была городская библиотека
имени Короленко. Она обслуживала в основном читателей центра нашего города и
имела два отдела: абонемент и читальный зал. Читальный зал занимал небольшую
комнату, примерно 20 квадратных метров. Посредине стоял большой обеденный стол,
вокруг стулья, два дивана – вот и вся мебель. Читатели занимали очередь, чтобы
попасть в читальный зал, приходили гораздо раньше, чем открывалась библиотека.
Я пришла в библиотеку в 1951 году, после окончания Донецкого
культпросветучилища. Встретили меня три очаровательные женщины. Это были
заведующая библиотекой Людмила Михайловна Самойлович и непосредственная моя
начальница, мой учитель, мой руководитель - Наталия Николаевна Кузнецова.

Библиотека имела очень маленький фонд, и в основном он состоял из тех книг,
которые дарили читатели, ведь здание довоенной библиотеки им. Короленко вместе
с книжным фондом было сожжено гитлеровцами. Как рассказывали ветераны
библиотеки, после войны, в 1947 году, в библиотеке был еще один пожар - пропало
много хо­роших книг. И пришлось  опять начинать сначала. Опять читатели
приносили книги, чтобы записаться в библиотеку, а также листочек бумажки для
формуляра.  Настолько была бедна библиотека, что вынуждены были обращаться к
читателям за помощью. Уже позже начали комплектование библиотеки в Донецком
библиотечном коллекторе и в книжных магазинах. И когда мы переехали в 1959 году
в новое помещение, где находилась раньше библиотека горкома партии, мы уже
имели 28 тысяч книг. Там у нас уже был и хороший читальный зал, и большая
комната для абонемента.

Самые теплые воспоминания у меня о Наталии Николаевне Кузнецовой. Она была
очень обязательна, очень внимательна к каждому человеку. Всегда со вкусом
одета, аккуратно причесана, всегда с улыбкой. Жизнь не очень баловала ее, но
когда она приходила на работу, обо всем забывала. Наблюдая за ней, у меня
сложилось такое впечатление, что она знает судьбу и интересы каждого читателя".

Людмила Яковлевна Скорцеско: \enquote{Кто дал мне путевку в жизнь, научил любить свою
профессию? Прежде всего Людмила Михайловна Самойлович, в то время заведующая
библиотекой. Настоящий служитель книги, чело­век большой и светлой души. Все
свои знания, силы и умение отдала для развития родной библиотеки. При ней, в
1961 году, библиотека стала центральной, ей было присвоено звание \enquote{Библиотека
отличной работы}. Екатерина Васильевна Омилаева возглавила читальный зал
библиотеки с 40-х годов до ухода на пенсию. Это был эталон профессионализма,
человек энциклопедических знаний, окончила два института: библиотечный и
иностранных языков. Очень скрупулезная в работе, несмотря на свою внешнюю
закрытость и даже суховатость, была очень отзывчивым и добрым человеком}.

Людмила Ильинична Белау: \enquote{В библиотеку Короленко я пришла в 1971 году, когда
заведовала библиотекой молодая, энергичная, с такой комсомольской искоркой в
глазах Галина Михайловна Захарова. В это время в библиотеке был небольшой
коллектив – всего девять человек. Начинала я работать на абонементе рядышком с
Лидией Михайловной Дардой, но небольшое время. Потом меня перевели в читальный
зал. Анастасию Александровну Проскурину, Галину Михайловну Захарову и Людмилу
Яковлевну Скорцеско я считаю своими учителями. С ними я пережила и \enquote{переучеты}
и тяжелей­шие ремонты, и основное - переезд в новое здание. Массовые
мероприятия проходили не только в стенах библиотеки, но и в Городском  саду, в
кинотеатре \enquote{Победа}, в ДК \enquote{Азовсталь}. В кинотеатре \enquote{Победа} перед сеансами
библиотечные работники шли с обзорами новых поступлений книг. Добрая память
осталась об  Ульяне Алексеевне Журавлевой, которая долгие годы работала у нас
техничкой. Она пришла в библиотеку совсем молодой девочкой, еще до войны, и
перенесла все тяготы, можно так сказать, библиотечного работника. И таскала
книги, и перевязывала их, и убирала, и была для молодых сотрудниц, и для меня в
том числе, второй мамой}.
