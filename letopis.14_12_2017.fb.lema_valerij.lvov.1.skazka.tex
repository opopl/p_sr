% vim: keymap=russian-jcukenwin
%%beginhead 
 
%%file 14_12_2017.fb.lema_valerij.lvov.1.skazka
%%parent 14_12_2017
 
%%url https://www.facebook.com/permalink.php?story_fbid=946445258837024&id=100004146750457
 
%%author_id lema_valerij.lvov
%%date 
 
%%tags literatura,skazka
%%title Сказка
 
%%endhead 
 
\subsection{Сказка}
\label{sec:14_12_2017.fb.lema_valerij.lvov.1.skazka}
 
\Purl{https://www.facebook.com/permalink.php?story_fbid=946445258837024&id=100004146750457}
\ifcmt
 author_begin
   author_id lema_valerij.lvov
 author_end
\fi

Сказка 

Ветра не было. Полупрозрачный дым из печной трубы поднимался прямо над красной
черепичной крышей и тут же исчезал, разбиваясь в клочья о низкое небо.
Появляясь словно ниоткуда редкие снежинки, плавно кружась в рассеянном свете
серого зимнего дня падали на землю, припорашивая все вокруг маленького
кирпичного домика, медленно, но уверено скрывая под белым пушистым одеялом уже
чуть заметные волчьи следы. След тянулся от изгороди со стороны леса, огибал
рубленную баню в глубине двора и пропадал под навесом крыльца, куда вели,
вприпрыжку от калитки чуть более свежие детские следы. Впрочем, может это были
и не волчьи следы, да и откуда в наше время взяться волкам в пригороде.

- Бабушка, у Снегурочки родители есть, или она сирота?

- Есть, конечно.

- Почему о них в сказках никогда ничего не рассказывается?

- Долго рассказывать.

Бабушка отвечала по военному четко, не задумываясь, строго взирая поверх
очков на внучку, совершенно очаровательную девочку в красной дедморозовой
шапке нахлобученной слегка набекрень.

- А ты расскажи сколько успеешь,- попросила девочка задвигая подальше под елку
фигурки сказочных персонажей и устраиваясь поудобней в ногах у бабушки.

- Дело не во времени. Просто родители у Снегурочки люди самые обыкновенные,-
пожала бабушка плечами,- ничем не примечательные. Также, как и все, они едят,
спят, ходят на работу и в гости. Когда день ото дня ничем не отличается, о
таком не расскажешь, такое только можно прожить: день в день, день в день,-
прожить и не заметить.

- Как это не заметить,- встрепенулась девочка, распахнув и без того огромные, в
пол лица глаза,- вокруг ведь столько интересного?!

- Это у тебя постоянно что-то случается.

- А у тебя?

- И у меня тоже,- легко согласилась бабушка,- дети и старики похожи. Дети еще
ничего не ждут от жизни, а старики уже ничего от нее не ждут, и потому для них
все, чтобы не случилось, всегда неожиданно и интересно, из-за чего они и сами
выглядят необычно для обычных людей.

- Мы с тобой сказочные, как Снегурочка?

- Можно и так сказать.

- Тогда и родители у Снегурочки тоже сказочные, просто они заколдованные,-
решила для себя девочка и расстроилась. - Получается, что взрослые
заколдованные?- не поверила она сама себе.

- Да.

- Все?

- Нет, некоторые из них очарованы,- улыбнулась, не сдержавшись бабушка,
невольно залюбовавшись ладной фигуркой на полу.

Рядом с разряженной елкой девочка и сама выглядела подарком. Не девочка -
сказка. - Лучшее, что на сегодня получилось у Сказочника,- подумала бабушка, с
горьким каким-то удовольствием разглядывая внучку. Те же, что и у сына, крупные
черты лица в обрамлении больших ушей делали ее личико маленьким, а головку
изящной, и эта утонченность в ясном свете голубых глаз была в ней главным,
скрывая за собой все то, что у сына торчало напоказ.

- А кто их раззочаровал? Снежная королева? - не унималась девочка.

- Снежная королева? - переспросила бабушка.

Ну да, все наверное с нее и началось,- вспомнила она некстати невестку. -
Маленькая оборванка! Если бы не она, все могло бы для сына сложиться иначе.
Мальчика ждала блестящая карьера. Кадровый военный, благодаря постоянным
войнам в стране он быстро поднялся по служебной лестнице и был уже майoром,
когда в его жизни появилась эта неприметная, обыкновенная во всех отношениях
девчонка, жившая недалеко, через лесок от бабушкиного домика. После
знакомства с ней, сын сделался сам не свой. Уволился в запас и принялся
писать книги, на все лады описывая армейские будни. С  ее, матери точки
зрения, занятие совершенно бесполезное, людям гражданским солдатчина всегда
претила, а военные, как известно, книг не читают. Совсем скоро быт заел
парня. Некогда решительный взгляд его потух. Волчьи, близко посаженные над
большим носом глаза потускнели. Характер испортился. Упорство сменило
упрямство, веселую злость - беспричинная раздражительность. И все это
началось с неприступной нынче девицы, в упор не желающую разделить с матерью
заботу о сыне.

- Ну да, все наверное с нее и началось,- повторила бабушка в слух.

- Нет, все началось с тролля,- спокойно и уверенно заявила девочка,- без тролля
снежная королева не смогла бы околдовать мальчика.

- С тролля?- переспросила бабушка,- Нет, никакого тролля не было,- думая о
своем рассеяно проговорила она.

Последние несколько лет тролль жил в девочке. С незапамятных времен тролль
жил и во многих других людях, и тех, кто был, и в тех, кто стал,- во всех
сразу, и всякий раз его звали по разному. У тролля было множество имен. Как
только люди обнаружили на самом глазном дне его в девочке присутствие, они
стали звать его Кох, откармливать и поить парным молоком.

Девочке такое соседство нисколько не вредило. Кох следил, чтобы она не
поправлялась от избытка пищи, а взамен, как только кто-то из ее ближайшего
окружения оказывался с ней на одной волне, тролль, поглощая все внимание
очарованного девочкой человека, использовал его, рассказывая девочке разные
истории, которые знал не понаслышке, и которые девочка оживляла богатым
своим воображением. Вместе они были одно целое, вместе им было интересно.
Поглощенные друг другом они могли часами проводить в воображаемом ими мире,
пока не появлялись охотники послушать эти истории, и не отпускали бабушку и
внучку обратно в серые будни из разноцветной сказочной вечности.

Это тролль несколько раз кряду, воспользовавшись расчувствовавшимся отцом
девочки заставил думать того о себе как о Сказочнике, а чуть раньше
живописать своей будущей супруге, с которой тролль также был близок, всю
подноготную неласковой армейской жизни. Тролль мог бы это делать и дальше, и
даже очень этого хотел. Человек тот был не без способностей, каждый раз по
настоящему переживая свои воспоминания, но парень стал много думать уже о
себе, и как о писателе, и как о мыслителе, отдалился от людей, отдалился от
всех, кто его любил, а без общения тролль был бессилен, без общения его
попросту не было. К тому же человек этот начал пить и нюхать всякую дрянь,
от чего настоящее не исчезало, конечно, но приобретало болезненные какие-то
очертания, от чего у Коха нередко болела голова.

- Тролль был, ты просто не помнишь,- терпеливо и несколько снисходительно
повторила девочка,- это тролль придумал зеркало в котором доброе казалось
смешным и глупым, и когда зеркало лопнуло от смеха, его осколки в глазах людей
сделали людей злыми.

-  Ах вот ты о чем, - умилилась бабушка характеру девочки, так похожему на ее
собственный,- только почему же злыми, люди сделались целеустремленными.

- Как это?

- Они узнали, что надо делать, а что не надо, чтобы занять в обществе достойное
место, и общество дало бы им все, что нужно.

- А как они узнали?

- Сравнили себя с другими, и увидели, что место важнее всего.

- Место?- удивилась девочка.

- Место в обществе дает уважение и доход.

- Всем-всем?

- Достойным, всем тем, кто сможет его занять.

- Оно что, одно?

- Одно.

- А как же быть остальным, они же все разные?

- Главное идти, все дороги ведут к нему, и тогда человек делает то, что должен.

- А почему человек не может делать то, что хочет,- не унималась девочка,-
почему человек не может делать то, что ему больше всего нравится и что у него
лучше всего получается?

- Потому, что человек рождается глупым и делает все на зло,- тролль спокойно
смотрел на девочку бабушкиными глазами, но та уже ничего не желала слушать.

- Общество - это место, где пропадают заколдованные взрослые, сказала она в
сердцах, делая вывод сразу и бесповоротно, как всегда в таких случаях поражая
тролля и развеселяя бабушку, прерывая ход, тягостных для бабушки мыслей,
девочка своей решительностью рушила связь времен, тем самым возвращая бабушку
из прошлого в настоящее.

- Да и Бог с ним, с троллем этим,- вздохнула бабушка с облегчением.

В последнее время с детьми стало особенно трудно разговаривать,- думал
предоставленный самому себе тролль. Оно и раньше было не просто отвечать на
вопросы без ответов. Пришлось выдумать массу авторитетных источников, чтобы
логично объяснить, почему Земля тяжелая, и как это она висит ни на чем, вот
только детей логика перестала интересовать. Они вообще перестали слушать
кого бы то ни было, а не только общепризнанных авторитетов, предпочитая
делать выводы сами для себя после общения друг с другом в социальных сетях.
Сами себе они верили, доверяя своему мнению больше, чем им говорили, другое
их попросту не интересовало и тролль все чаще оказывался невостребованным.
Не то, чтобы вера его убивала, воспринимая все его несомненные достижения
как должное, еще и снисходительно, мол, а для чего ты в таком случае вообще
нужен, раз не для этого? 

Если бы тролль мог чувствовать, он бы наверное расстроился. Чувства тролль не
знал, никакого. По человеческим меркам его бы считали  отмороженным, но не в
том смысле, что без царя в голове, когда ум подчинен необузданным желаниям. 

Никаких желаний у тролля не просто не было,- он был сам само обуздание. Но и в
смысле чистого разума, на который часто ссылаются как на нечто послужившее
причиной всего,- тролль тоже не был.  Тролль был царем всего безо всего,
королем Лир и Санта Клаусом в одном проявлении. Не голым королем и не царскими
одеждами,- он был чистой идей, идеей права: тролль в этом мире был по праву, а
не по рождению. Можно родиться господином, а можно стать таковым в борьбе, тем
самым создавая в сутолоке место. Не тронное место, и не пьедестал, хотя и не
без этого, конечно, но - святое. 

Святое место, место смывающее грехи таким образом, что на нем мог оказаться кто
угодно, но не сам Santo, которому вся эта суета была ни к чему, который был
таковым, и без труда, а как это без труда?  Тролль такого не понимал.
Чувственное знание, такое, чтобы сразу и навсегда,- оно было троллю недоступно,
в его понимании совершенно невероятно. 

Поэтому он не добро выставлял в дурном свете, когда, как в кривом зеркале,
демонстрировал людям все их достоинства, он заставлял людей смеяться над верой,
над верой в их собственную непогрешимость, и во всю прочую безосновательность.
