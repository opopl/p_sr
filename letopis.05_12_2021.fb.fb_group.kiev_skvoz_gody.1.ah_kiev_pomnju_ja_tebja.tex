% vim: keymap=russian-jcukenwin
%%beginhead 
 
%%file 05_12_2021.fb.fb_group.kiev_skvoz_gody.1.ah_kiev_pomnju_ja_tebja
%%parent 05_12_2021
 
%%url https://www.facebook.com/groups/1088107311247388/posts/4984467968277950
 
%%author_id fb_group.kiev_skvoz_gody,mike_mikes.kiev
%%date 
 
%%tags gorod,kiev,poezia
%%title Ах, Киев, помню я тебя!
 
%%endhead 
 
\subsection{Ах, Киев, помню я тебя!}
\label{sec:05_12_2021.fb.fb_group.kiev_skvoz_gody.1.ah_kiev_pomnju_ja_tebja}
 
\Purl{https://www.facebook.com/groups/1088107311247388/posts/4984467968277950}
\ifcmt
 author_begin
   author_id fb_group.kiev_skvoz_gody,mike_mikes.kiev
 author_end
\fi

\begin{multicols}{2} % {
\ifcmt
  ig https://scontent-frt3-1.xx.fbcdn.net/v/t39.30808-6/264627524_4616157965164877_3303866670641015206_n.jpg?_nc_cat=104&ccb=1-5&_nc_sid=825194&_nc_ohc=53cnaA4WKuoAX98CdO3&_nc_ht=scontent-frt3-1.xx&oh=5faa82ef73ec5c1f91c755a74065953d&oe=61BC099B
  @width 0.4
\fi

\obeycr
Ах, Киев, помню я тебя.
Тогда мне было лет пятнадцать:
Каштаны, розы, тополя
И дым дурманящих акаций.
Патона мост и Гидропарк,
Фуникулёр, высокий берег...
Чудесный город, что не так
Ты сделал? Не могу поверить!
Ах, Киев, помню я тебя!
Пусть миновало четверть века,
Но есть по-прежнему земля,
Куда язык вёл человека.
А человек тот был поэт?
Нет, он был просто человеком.
Вот в этом-то и весь секрет,
Открывшийся с двадцатым веком...
Терентий Травник.
\restorecr
\end{multicols} % }
