% vim: keymap=russian-jcukenwin
%%beginhead 
 
%%file slova.oksjumoron
%%parent slova
 
%%url 
 
%%author 
%%author_id 
%%author_url 
 
%%tags 
%%title 
 
%%endhead 
\chapter{Оксюморон}
\label{sec:slova.oksjumoron}

%%%cit
%%%cit_head
%%%cit_pic
%%%cit_text
В литературе есть такой термин – \emph{оксюморон}. Он означает «сочетание слов с
противоположным значением, то есть сочетание несочетаемого». В качестве примера
\emph{оксюморона} часто приводят название одного из рассказов Л.Н. Толстого – «Живой
труп». Вот такой \emph{«оксюморон»} сегодня и представляют собой слова «украинская» и
«дипломатия». Чтобы это заявление не было голословным, приведу несколько
примеров, которые прокомментировал эфире украинского телеканала «Эспрессо»
бывший посол Киева в Минске Роман Безсмертный
%%%cit_comment
%%%cit_title
\citTitle{Украинская дипломатия: сочетание несочетаемого}, 
Мое Мнение, zen.yandex.ru, 12.06.2021
%%%endcit

