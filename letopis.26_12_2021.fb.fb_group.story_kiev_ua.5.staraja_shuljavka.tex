% vim: keymap=russian-jcukenwin
%%beginhead 
 
%%file 26_12_2021.fb.fb_group.story_kiev_ua.5.staraja_shuljavka
%%parent 26_12_2021
 
%%url https://www.facebook.com/groups/story.kiev.ua/posts/1826878294175630
 
%%author_id fb_group.story_kiev_ua,majorenko_georgij.kiev
%%date 
 
%%tags gorod,kiev,shuljavka.kiev
%%title СТАРАЯ ШУЛЯВКА
 
%%endhead 
 
\subsection{СТАРАЯ ШУЛЯВКА}
\label{sec:26_12_2021.fb.fb_group.story_kiev_ua.5.staraja_shuljavka}
 
\Purl{https://www.facebook.com/groups/story.kiev.ua/posts/1826878294175630}
\ifcmt
 author_begin
   author_id fb_group.story_kiev_ua,majorenko_georgij.kiev
 author_end
\fi

\headCenter{СТАРАЯ ШУЛЯВКА}

\begin{zznagolos}
\obeycr
Есть улицы центральные
высокие и важные
с витринами зеркальными,
с гирляндами огней
А мне милей нешумные,
милей - одноэтажные...
\smallskip
Ю. Антонов
\restorecr
\end{zznagolos}

\ii{26_12_2021.fb.fb_group.story_kiev_ua.5.staraja_shuljavka.pic.1}

В сюжете речь пойдет о районе с одноэтажными улицами. О старой Шулявке, где
появился на свет я, отец мой, бабушка с прабабушкой и другие далекие предки. 

%% to remove
%\ii{26_12_2021.fb.fb_group.story_kiev_ua.5.staraja_shuljavka.pic.1.cmt}

\headTwo{ЛЕСОК С ЛУЖАЙКАМИ}

В древние времена эта местность именовалась «Шелвова борка» т.е. лесок с
большим количеством лужаек на берегу реки Лыбедь, где возникло поселение под
названием Шелвово сельцо.

Позже оно называлось Шулявщиной, затем - Шулявкой. Когда-то в этой местности
находилась летняя резиденция Митрополита.

\headTwo{ДОБРЫЙ ЦАРЬ}

Шулявку в народе называли западными воротами Киева. 

В 1857 году к приезду в город царя Александра Второго была построена деревянная
триумфальная арка (где сейчас Воздухофлотский мост). Там встречали царский
поезд. Планировали сделать арку добротной и капитальной, но царь сказал:
\enquote{- Лучше потратьте деньги на благотворительность!}


\headTwo{ГОРОДСКАЯ И ДЕРЕВЕНСКАЯ}

На старых картах Киева мы видим, что Шулявка была поделена на две части.
Шулявка - городская в составе Лукьяновского полицейского участка и Шулявка
деревнская, относящаяся к Киевской губернии. А рассекал эти части надвое
Брест-Литовский проспект.  

\headTwo{СТУДЕНЧЕСКАЯ И ПРОЛЕТАРСКАЯ}


Когда на Шулявке был прстроен Политехнический институт, этот район ожививили
задорные и веселые студенты.

А после открытия завода Гретера и Криванека - Шулявка стала пролетарской.

\headTwo{ШУЛЯВКА РЕВОЛЮЦИОННАЯ}

В период революции 1905 года Совет рабочих депутатов Киева решил начать
массовую забастовку и 12 декабря остановили работу все крупные предприятия
города: заводы Гретера и Криванека, \enquote{Южнорусский машиностроительный} (Ленкузня),
«Арсенал», Железнодорожные мастерские и др.

%Марие-Магдалиновская церковь.
\ii{26_12_2021.fb.fb_group.story_kiev_ua.5.staraja_shuljavka.pic.2}

Революционный штаб разместился в помещении главного корпуса Киевского
политехнического института и Шулявка была провозглашена «рабочей республикой».
Вооруженные рабочие дружины сами стали патрулировать территорию, на которой
была обьявлена власть Революционного комитета. Но 16 декабря 1905 года
территория Шулявской республики была окружена  жандармерией и казацкой конницей
и жестоко разгромлена. 

\headTwo{ЗООСАД}

В 1909 году на Шулявку из Ботанического сада, что возле красного корпуса
Универа, был перенесен киевский зоосад. 

Тоже шулявская достопримечательность!

\headTwo{ВОТ ТРАМВАЙ НА РЕЛЬСЫ СТАЛ}

Когда-то Шулявку связывал с центром трамвай, что ходил по Брест-Литовскому
проспекту от Киевского политехнического института до Бессарабки. Я этот трамвай
помню. При мне под него однажды мужик попал.

\headTwo{КРИМИНАЛ}

А ещё за Шулявкой тянулся шлейф криминальных легенд и дурная слава босяцкого района.

Не зря пел Розенбаум:

А я родился на Шулявке

на блатной...

Из песни слов не выкинешь! Шулявка слыла бандитской окраиной. Там в частном
секторе ворам было удобно скрываться и прятать краденные вещи.

\headTwo{МАРИЕ - МАГДАЛИНОВСКАЯ}

Центром духовной и административной жизни Шулявки в дореволюционнные времена
была церковь св. Марии Магдалины. Она находилась рядом с тем местом, где сейчас
метро КПИ.

В этой церкви жители района венчались, крестили детей и отпевали покойников.

\raggedcolumns
\begin{multicols}{3} % {
\setlength{\parindent}{0pt}

% Свидетельство тети Ани.
\ii{26_12_2021.fb.fb_group.story_kiev_ua.5.staraja_shuljavka.pic.3}

% Шулявская триумфальная арка.
\ii{26_12_2021.fb.fb_group.story_kiev_ua.5.staraja_shuljavka.pic.4}

% Батька с друзьями в Пушкинском парке.
\ii{26_12_2021.fb.fb_group.story_kiev_ua.5.staraja_shuljavka.pic.6}
\ii{26_12_2021.fb.fb_group.story_kiev_ua.5.staraja_shuljavka.pic.6.cmt}

% Отец - парень с Борщаговской!
\ii{26_12_2021.fb.fb_group.story_kiev_ua.5.staraja_shuljavka.pic.5}
\ii{26_12_2021.fb.fb_group.story_kiev_ua.5.staraja_shuljavka.pic.5.cmt}


% Новый год в нашей старой квартире. Конец 50-х. Родня и соседи. Шулявка - одна семья!
\ii{26_12_2021.fb.fb_group.story_kiev_ua.5.staraja_shuljavka.pic.7}
\ii{26_12_2021.fb.fb_group.story_kiev_ua.5.staraja_shuljavka.pic.7.cmt}

%Борщаговская в начале 60-х. Мама справа, тетя слева. Это недалеко от того
%места, где потом был магазин "Военторг". Где Борщаговская соприкасалась с
%Брест-Литовским.
\ii{26_12_2021.fb.fb_group.story_kiev_ua.5.staraja_shuljavka.pic.8}

\end{multicols} % }

\headTwo{ВРЕМЕНА СЕДОЙ СТАРИНЫ}

Мои предки с Шулявки носили фамилию Куличенко и с \enquote{времен царя-гороха} обитали
в усадьбе по адресу Борщаговская 130. Это недалеко от теперешней остановки
скоростного трамвая \enquote{Политехнический институт}.

А по соседству проживала семья нашего родственника Никиты Тарасовича Рутенко.
Был он человеком известным и уважаемым. Из киевских мещан. Какое-то время
служил в церкви дьяконом и многие семьи на Шулявке почтитали за честь, если он
был крестным у их детей. Упоминаниями о Никите Тарасовиче Рутенко пестрят
документы старого Киева.

А ещё семьи Рутенко и Куличенко имели усадьбы на улице Землянской в районе
Зверинца.

В группе есть правнучка Никиты Тарасовича Рутенко - экскурсовод, писательница,
консультант и звезда КИЕВСКИХ ИСТОРИЙ - Татьяна Гурьева. Киевлянка, в более чем
семи поколениях.

\headTwo{ЕЩЁ ОДИН АДРЕС}

С конца 19 века мои предки проживали в старинном двухэтажном доме с длинными
коридорами и однокомнатными квартирами. Домовладельцем был житель Шулявки по
фамилии Сказин. До революции почти все квартиры на втором этаже здания занимали
прадед, прабабушка и их пятеро дочерей. Адрес - Борщаговская 30. И я родился в
этом доме.

Квартирка была маленькая, удобства на улице.

\headTwo{ЛИЧНАЯ ПОЧЕТНАЯ ГРАЖДАНКА}

На одной из мной опубликованных фотографий запечатлена жительница Шулявки -
личная почетная гражданка Параскева Никифоровна Павловская (в девичестве
Куличенко, родная сестра моей прабабушки ). Параскева была женой почетного
гражданина Киева унтер-офицера Николая Ивановича Павловского и \enquote{на районе} была
очень уважаемой. Скольким детям на Шулявке была крестной, не перечесть!

\headTwo{ПАМЯТЬ ОБ АННЕ}

И второй документ мне особенно дорог.

Храню его, как реликвию. Это свидетельство об окончании Шулявской
церковно-приходской школы Анны Владимировны Балтушевич - родной сестры моей
бабушки.

В семье был культ тети Ани. Когда родители умерли, на плечи Анны легла забота о
своих четырех  младших сестрах.

А время было тяжелейшее! Революция, гражданская война, разруха. И если  до
революции Анна была добропорядочной киевской мещанкой, оконичившей
церковно-приходскую школу, то в суровые времена она - комсомолка,
общественница, позже вступила в партию. В годы Великой Отечественной войны Анна
была в киевском подполье, арестована Гестапо и расстреляна в Бабьем Яру. 

Полагаю, Анна шла на казнь с высоко поднятой головой. 

Сильная была женщина!

Остальные фото из личного архива 

отца и фрагменты карт старинного Киева из моей коллекции.
