% vim: keymap=russian-jcukenwin
%%beginhead 
 
%%file slova.diskriminacia
%%parent slova
 
%%url 
 
%%author 
%%author_id 
%%author_url 
 
%%tags 
%%title 
 
%%endhead 
\chapter{Дискриминация}

%%%cit
%%%cit_head
%%%cit_pic
%%%cit_text
А ведь это еще не все. Оказывается, в понятие \enquote{патриотического воспитания}
входят еще и формирование поддержки вступления в НАТО, и уважительное отношение
к \enquote{героям борьбы украинского народа за обретение независимости}.  Но самый
главный пункт в этом перечне – последний: Кабмин предлагает привлекать на
работу в органы государственной власти и местного самоуправления только граждан
со \enquote{сформированной национальной идентичностью}. Проще говоря, если у тебя
украинский не родной, если ты не поддерживаешь вступление в НАТО и не
относишься уважительно к Бандере и Шухевичу, дорога в органы власти тебе
заказана.  То есть, по сути, речь идет о \emph{дискриминации} сразу и по национальному
признаку, и по политическим взглядам
%%%cit_comment
%%%cit_title
\citTitle{Половину граждан Украины заклеймили в языковой неполноценности / Лента соцсетей / Страна}, 
Андрей Манчук, strana.ua, 02.07.2021
%%%endcit

