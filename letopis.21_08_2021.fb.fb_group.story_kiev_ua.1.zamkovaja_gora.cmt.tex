% vim: keymap=russian-jcukenwin
%%beginhead 
 
%%file 21_08_2021.fb.fb_group.story_kiev_ua.1.zamkovaja_gora.cmt
%%parent 21_08_2021.fb.fb_group.story_kiev_ua.1.zamkovaja_gora
 
%%url 
 
%%author_id 
%%date 
 
%%tags 
%%title 
 
%%endhead 
\subsubsection{Коментарі}

\begin{itemize} % {
\iusr{Анна Анюшка}

А слабо помимо нацепить табличку что гора охраняется, привести её в нормальный
вид. Загородить старое кладбище, отреставрировать старый склеп, и написать кому
был построен. А то только таблички научились тыкать везде

\begin{itemize} % {
\iusr{Алла Ильенко}
\textbf{Анна Анюшка} А слабо не писать, а самой внести посильный вклад ?

\begin{itemize} % {
\iusr{Анна Анюшка}
\textbf{Алла Ильенко} Алла если вы такая деловая колбаса давайте подтягивайтесь будем вместе дерзать и облагораживать гору. А то только на слабо брать. Деловуха этакая

\iusr{Алла Ильенко}
\textbf{Анна Анюшка} А при чем тут Я? ВЫ продиктовали свои требования. Простите, но я всю жизнь избегала начальников женского пола.

\iusr{Анна Анюшка}
\textbf{Алла Ильенко} Словесный понос вы несёте. Ничего я никому не диктовала. Что-то у вас с головушкой милая, идите проветриться на свежий воздух. Несёте черти шо

\iusr{Ира Титова}
\textbf{Анна Анюшка} Не нужно хамить! Тут люди спокойно обсуждают прекрасную прогулку по старинным заброшенным местам!
\end{itemize} % }

\iusr{Татьяна Сирота}
\textbf{Анна Анюшка} Очень надеюсь, что и Замковую приведут в порядок.

\iusr{Alex Maler}
Таблічко на фотографиях не видели?!?

\begin{itemize} % {
\iusr{Анна Анюшка}
\textbf{Alex Maler} что за каламбур, смысла не улавливаю. Что вы хотели этим сказать @igg{fbicon.face.grinning.smiling.eyes} 

\iusr{Alex Maler}
\textbf{Анна Анюшка} видеоряд фотографий включает фото охранной таблички.
\end{itemize} % }

\end{itemize} % }

\iusr{Анна Невская}
Спасибо большое за познавательную и содержательную экскурсию. Фотографии великолепны!

\iusr{Наталия Привалко}
Спасибо за интересную экскурсию по знаковым местам нашего города

\ifcmt
  ig https://i2.paste.pics/fd8cbde09f4734da4b6757902496ed1b.png
  @width 0.3
\fi

\iusr{Андрей Надиевец}
Считайте, что взошли на Джомолунгму, та всего немного выше

\begin{itemize} % {
\iusr{Татьяна Сирота}
\textbf{Андрей Надиевец} Ну...для кого-то и Замковая - Джомолунгма. @igg{fbicon.face.upside.down} 

\iusr{Андрей Надиевец}
\textbf{Татьяна Сирота} Ведь у каждого свой Эверест, и в прямом смысле и в переносном.

\iusr{Ирина Тывус}
\textbf{Татьяна Сирота} С моим коленом - уж точно!!!

\iusr{Татьяна Сирота}
\textbf{Ирина Тывус} . @igg{fbicon.face.pensive}{repeat=3} 
\end{itemize} % }

\iusr{Ирина Тывус}
Спасибо! Никогда там не была. Как всегда, очень интересно и поучительно!
@igg{fbicon.thumb.up.yellow} 

\iusr{Ирина Иванченко}
Спасибо, Таня!

\iusr{Татьяна Соловьева}

Вы вдвоём просто герои и замечательные альпинисты) Всегда хотела взойти по этой
лестнице @igg{fbicon.heart.red}, но нет  @igg{fbicon.face.disappointed} .
Благодаря вам, как будто бы и совершила восхождение)

\begin{itemize} % {
\iusr{Татьяна Сирота}
\textbf{Татьяна Соловьева} Танечка, не всё так страшно с восхождением)))
Главное, на него настроиться.

\iusr{Татьяна Соловьева}
\textbf{Татьяна Сирота} Теперь уже точно @igg{fbicon.heart.red} @igg{fbicon.face.blowing.kiss} )
\end{itemize} % }

\iusr{Елена Гурьянова}
Вчера гуляла там) получила массу удовольствия

\ifcmt
tab_begin cols=2,no_fig,center

  ig https://scontent-mxp1-1.xx.fbcdn.net/v/t39.30808-6/240258938_4333484806713062_6347162407184171276_n.jpg?_nc_cat=111&ccb=1-5&_nc_sid=dbeb18&_nc_ohc=xZQAA8Gn-dQAX_aoxCJ&_nc_ht=scontent-mxp1-1.xx&oh=714a8b633b3b7fda1df641e6a8a50023&oe=61B9AAC7

	ig https://scontent-mxp1-1.xx.fbcdn.net/v/t39.30808-6/240399675_4333485176713025_8555747386952491427_n.jpg?_nc_cat=108&ccb=1-5&_nc_sid=dbeb18&_nc_ohc=jNSFe2hc6nEAX_g7RjR&_nc_ht=scontent-mxp1-1.xx&oh=590f09fa30ec354d90d002553286540e&oe=61BADCA1

	ig https://scontent-mxp1-1.xx.fbcdn.net/v/t39.30808-6/240385216_4333485436712999_8456276838580132421_n.jpg?_nc_cat=105&ccb=1-5&_nc_sid=dbeb18&_nc_ohc=ojzP9uLVq14AX80vV_e&_nc_ht=scontent-mxp1-1.xx&oh=852509f45175deaa98f93c24691ef708&oe=61B965A9

	ig https://scontent-mxp1-1.xx.fbcdn.net/v/t39.30808-6/240393422_4333485736712969_8907024407448528052_n.jpg?_nc_cat=111&ccb=1-5&_nc_sid=dbeb18&_nc_ohc=0ocLCLdiEU0AX-VQFdk&_nc_ht=scontent-mxp1-1.xx&oh=00_AT-KuRTZbdenVykeJ_XDjrUtVmJs4wBUEYMKt9fBDvoLGA&oe=61BAB050

tab_end
\fi

\iusr{Елена Гурьянова}
Подъем на Замковую через старое кладбище - это нечто!!!

\begin{itemize} % {
\iusr{Татьяна Сирота}
\textbf{Елена Гурьянова} А мы через него спускались. @igg{fbicon.face.upside.down} 

\iusr{Елена Гурьянова}
\textbf{Татьяна Сирота}  @igg{fbicon.hands.applause.yellow}  наверное спускаться проще)))

\iusr{Татьяна Сирота}
\textbf{Елена Гурьянова} Конечно. Но заброшенность и безлюдность чуть-чуть настораживают...

\iusr{Елена Гурьянова}
\textbf{Татьяна Сирота} ну это да. Местность там такая черная, кстати вчера рассказывали что эта улица так и называлась Черная Грязь.

\iusr{Alex Maler}
\textbf{Елена Гурьянова} раньше вдоль той лестницы стояли дома и была мощеная дорога для машин. Было проведено электричество и водопровод на самый верх...
Наверху же была закрытая и огороженная зона с военным караулом и радиоглушительная станция с вышкой.
\end{itemize} % }

\iusr{Alex Maler}
Ещё в впродовж 80-х собирал осенью на ней по два ведра опенок...

\iusr{Татьяна Сирота}
\textbf{Alex Maler} Класс!!! @igg{fbicon.face.smiling.eyes.smiling} 

\iusr{Елена Супрун}

Спасибо, Танечка! Как всегда интересно, впечатляюще и захватывающе! Бродили там
же лет 10 назад и все было также запущено. А жаль.

\ifcmt
  ig https://i2.paste.pics/647e65a37807f9a2dc9dbaf945cd2ff4.png
  @width 0.3
\fi

\begin{itemize} % {
\iusr{Татьяна Сирота}
\textbf{Елена Супрун} Да, очень жаль, что Замковая в запущенном состоянии. @igg{fbicon.face.pensive}{repeat=3} 

\iusr{Елена Супрун}
А ведь это такой лакомый кусочек Киева!
\end{itemize} % }

\iusr{Нелли Стельмах}
Сьогодні також там гуляла)

\ifcmt
  ig https://scontent-mxp1-1.xx.fbcdn.net/v/t39.30808-6/239888345_1748932178625766_4777782981358363884_n.jpg?_nc_cat=109&ccb=1-5&_nc_sid=dbeb18&_nc_ohc=5j-SYCtb3KQAX_3qESo&_nc_ht=scontent-mxp1-1.xx&oh=ecfcf4981b1ffbc5f0047bfc6872cd9c&oe=61B93856
  @width 0.4
\fi

\iusr{Alex Maler}
На Замковой захоронены кадеты - герои Крут...
Памятник бы...

\begin{itemize} % {
\iusr{Елена Гурьянова}
\textbf{Alex Maler} вчера была на экскурсии, такого не слышала

\iusr{Татьяна Сирота}
\textbf{Alex Maler} Не знала об этом. Про воинскую часть с \enquote{глушилками} знала, а про кадетов нет.

\iusr{Alex Maler}
\textbf{Татьяна Сирота} надо нашего кладбищенского поводыря из КИ, Диму Горбатюка (чы Вадика Горбова, не помню) припахать... Нехай шукает!

\iusr{Любов Огородня}
\textbf{Alex Maler} а де саме вони там поховані?

\iusr{Alex Maler}
\textbf{Любов Огородня} не знаю.

\iusr{Любов Огородня}

А звідки інформація, що саме там поховані? Я народилася і виросла на Гончарній,
але вперше чую про це @igg{fbicon.face.astonished} 

\iusr{Alex Maler}
\textbf{Любов Огородня} та нещодавно хтось писав...

\iusr{Игорь Васильевич Переводов}
Герої Крут захоронені на Аскольдовій могилі. І Лиса гора поряд, але не на Замковій.

\iusr{Alex Maler}
\textbf{Игорь Васильевич} Переводов \enquote{поряд} - за Лаврою та Видубичами?!?

\iusr{Ольга Волынец}
\textbf{Alex Maler} . точных данных нет. В братскую могилу на Аскольдовом
кладбище, по данным тогдашней прессы, было опущено 18 гробов.


\iusr{Григорий Гришин}
Это миф.

\iusr{Alex Maler}
\textbf{Grigory Grishin} вы проктолог-материалист?

\iusr{Вал Пал Гол}
\textbf{Alex Maler} 

Похороны погибших под Крутами

Так, на заседании Малой Рады глава УНР Михаил Грушевский предложил почтить
память погибших под Крутами и перезахоронить их на Аскольдовой могиле в Киеве.
Многолюдные похороны состоялись 19 марта 1918 года. На заупокойную службу
собрались их родные, студенты, гимназисты, воины, духовенство, хор под
руководством А. Кошица, множество киевлян. С траурно-торжественной речью
обратился к собранию Михаил Грушевский.

В братскую могилу на Аскольдовом кладбище, по данным тогдашней прессы, было
опущено 18 гробов. (Википедия)

\iusr{Alex Maler}
\textbf{Grigory Grishin} 

примите извинения за неудачную шутку.

Совсем недавно, в каком-то \enquote{киевском} сообществе прочел... в каком контексте - забыл.

Сам удивлен, ведь Фроловскую горку с 1961 года облазил всю...

А в киевских сообществах, как правило, народ ответственный.

\begin{itemize} % {
\iusr{Григорий Гришин}
\textbf{Alex Maler} Вся история пронизана мифами, так что все может быть... или не быть.

\iusr{Киевские истории}
Григорий, ссылки на сторонние ресурсы запрещены правилами группы!

\iusr{Григорий Гришин}
\textbf{Киевские истории} Извините, учту, что и в комментариях, тоже.

\iusr{Киевские истории}
\textbf{Григорий Гришин}, 

Напоминаю правила нашей группы. Оскорбления, ненормативная лексика, ссылки в
постах и комментариях на сторонние ресурсы категорически запрещены.

Нарушителей администрация будет переводить в \enquote{немой режим}, особо злостных
блокировать.

Правила обсуждению не подлежат.

p.s. Тем, кто оставлял ссылки в комментариях, рекомендую удалить
самостоятельно, в противном случае их удалят вместе с последующей блокировкой
участника...

\iusr{Григорий Гришин}

Правила не обсуждаются. Я извинился, Вы уже удалили сами мой комментарий.
Вопрос исчерпан? Или в \enquote{немой} режим переведете или заблокируете?

\iusr{Олег Коваль}
Григорий, Вам напомнили правила.

\iusr{Alex Maler}
\textbf{Grigory Grishin} 

кстати - в этой ветке снова появился тот Oleksandr V Didichenko, и опять
упомянул о захоронении на Замковой горе, о которой идёт речь в статье.

Обратитесь прямо к нему.

Немного ниже.

\iusr{Григорий Гришин}
\textbf{Alex Maler} 

Спасибо, но он не очевидец @igg{fbicon.face.upside.down}  а сылки на источники
в сети в группе запрещены ее правилами.

\end{itemize} % }

\iusr{Игорь Васильевич Переводов}
\textbf{Alex Maler} Набагато ближче. В цьому пості навіть є фото - шукайте  @igg{fbicon.smile} .

\iusr{Балу Балу}
\textbf{Alex Maler} я только написать хотел про крутян

\begin{itemize} % {
\iusr{Балу Балу}

Всего во время того боя с украинской стороны полегли около 500 бойцов. До сих
пор известно официальное захоронение на Аскольдовой могиле и Замковой гор где
остальные - неизвестно по сей день. Если бы позиционная к УНР прессы тогда не
подняла скандал, то и этих захоронений бы мы не знали. А кроме крутян были
юнкера Константиновского училища, которых тоже полегло немало в том бою.

\iusr{Alex Maler}
\textbf{Oleksandp V Didichenko} ...и всех разом именуют \enquote{Героями Крут}.

\iusr{Балу Балу}
\textbf{Alex Maler} гимназистов крутян погибших в том бою. Шутки у вас

\iusr{Балу Балу}
\textbf{Alex Maler} Кто как именует. Дело вкуса

\iusr{Alex Maler}
\textbf{Oleksandp V Didichenko} согласен, неудачно... Удаляю
\end{itemize} % }


\end{itemize} % }

\iusr{Наталия Платонова}

Спасибо! Сколько здесь исхожено... помню ещё домики, стоявшие со времён Свирида
Петровича и Прони Прокоповны, белёные, с синими наличниками, с ватой и
настриженной цветной бумагой между рамами, утопающие в сугробах...

\begin{itemize} % {
\iusr{Alex Maler}
\textbf{Наталия Платонова} ага!
...и краном водопровода, вечно текущим во дворах...
И козочьки пасуцца... и кнурык рохкает с сарайи!
Это был участок моей мамы, човгала по вызовам.
\end{itemize} % }

\iusr{Нара Ахмедова}

Поднималась, но там всё запущено и неухожено. А виды действительно красивые

\begin{itemize} % {
\iusr{Татьяна Сирота}
\textbf{Нара Ахмедова} Полностью согласна с Вами.

\iusr{Ира Титова}
\textbf{Нара Ахмедова} Там всегда было,,все запущено и неухожено,,!Лет 40 точно! @igg{fbicon.face.screaming.in.fear} 

\iusr{Нара Ахмедова}

При этом это место очень посещаемо - там и художники сидят пейзажи рисуют, и
школьные экскурсии проводят, и просто люди полюбоваться Киевом поднимаются. А
впечатление, что хозяина у этой территории нет. А место прям вот пуп Киева, там
и Андреевский спуск, и церковь, и тд

\end{itemize} % }

\iusr{Мария Константиновская}

Спасибо Вам огромное! И не устаю повторять: здоровья Вашим ножкам! Счастья и
радости от внуков!!! @igg{fbicon.hand.victory} @igg{fbicon.sunflower} 


\iusr{Валентина Товстенко}
Прогулялась вместе с Вами с більшим удовольствием, спасибо. Очень интересно!

\iusr{Larisa Olesyuk}
Спасибо @igg{fbicon.hands.pray}  @igg{fbicon.hearts.two}. 
Очень красивые виды. Жаль, что так все запущено. Хорошо бы парк разбить.

\begin{itemize} % {
\iusr{Татьяна Сирота}
\textbf{Larisa Olesyuk} Ну... парк, как по мне, здесь не нужен.
А вот привести это место в порядок очень даже ПРОСИТСЯ.

\iusr{Larisa Olesyuk}
\textbf{Татьяна Сирота} ну, я не имела ввиду классический парк, а вот именно
облагородить территорию.
\end{itemize} % }

\iusr{Larisa Olesyuk}
Я бы с вами пошла на гору @igg{fbicon.exclamation.mark.double} @igg{fbicon.hands.applause.yellow} 

\iusr{Валентина Ивановна}
Отличная экскурсия. Интересно читать

\iusr{Людмила Краснюк}

Спасибо за очень интересное восхождение на Замковую гору и великолепные фото-
картины!  @igg{fbicon.heart.sparkling} Вы с внуком - просто герои ! @igg{fbicon.thumb.up.yellow} Восхищаюсь Вами! @igg{fbicon.heart.red}

\iusr{Татьяна Сирота}
\textbf{Людмила Краснюк} СПАСИБО! @igg{fbicon.hands.pray}{repeat=3} 

\iusr{Svitlana Levkovets}
Спасибо большое очень интересно рассказываете и показываете

\iusr{Людмила Терещенко}
Спасибо, отличная прогулка @igg{fbicon.laugh.rolling.floor}  @igg{fbicon.face.grinning.squinting}  @igg{fbicon.laugh.rolling.floor} . Здоровья вам. @igg{fbicon.heart.eyes}{repeat=3} 

\iusr{Татьяна Сирота}
\textbf{Людмила Терещенко} Спасибо. ☺ ️  ☺ ️  ☺ ️ 

\iusr{Татьяна Трофимова}

Прочитала и подумалось очередной раз: интересно почему именами Кий, Щек и Хорив
не называют современных мальчиков, ведь другие старые имена возрождаются?

\begin{itemize} % {
\iusr{Татьяна Сирота}
\textbf{Татьяна Трофимова} . @igg{fbicon.thinking.face}{repeat=3} 

\iusr{Ира Титова}
\textbf{Татьяна Трофимова} 

Потому что молодежь адекватная и представляет, как ребёнку потом жить с такими
именами! @igg{fbicon.face.screaming.in.fear} 

\iusr{Татьяна Трофимова}
\textbf{Ира Титова} , 

а чем имя Хорив или Щек хуже Архипа, Назара, Тараса, Ратибора, Добрыни, Платона
и т.д.? ( Кий ладно, с бильярдом ассоциируется)

\iusr{Ира Титова}
\textbf{Татьяна Трофимова} Хотя бы будущими кличками, типа ,,щека,, или ,,хоря,,! У детей фантазия уникальная! @igg{fbicon.wink} 
\end{itemize} % }

\iusr{Liubov Abukhendi}
Спасибо за рассказ и фото. Очень интересно.

\iusr{Olena Savych}
Щиро дякую! Цікава розповідь про унікальне місце!!!

\iusr{Лариса Манзюк}

Красивые виды. Повезло бабушке с внуком или внуку с бабушкой? Смелые и умелые.
И нам повезло с вами путешествовать.  Спасибо @igg{fbicon.heart.eyes}
@igg{fbicon.heart.red}

\iusr{Tatyana Gusach}
Комусь завадило гусине перо в руці Гоголя вже немає!!! От вандали!!!

\begin{itemize} % {
\iusr{Татьяна Сирота}
\textbf{Tatyana Gusach} Ой-ой-ой!
А я и внимания не обратила... @igg{fbicon.face.downcast.sweat} 

\iusr{Татьяна Митрофанова}
\textbf{Tatyana Gusach} у нашего Херувима, стрелы, все время «улетают».
Пришлось придумать легенду ))

\iusr{Костянтин Черній}
\textbf{Tatyana Gusach} гусиНЄ?, чи - гусяЧЕ?....Гоголь - російський, чи, український письменник..?

\iusr{Альбина Орел}
\textbf{Tatyana Gusach} в вихідні багато люду, дивляться, фотографуються, і чомусь вважають, що найкраще фото вийде, якщо залізти Гоголю на коліна або прилягти на Булгакова .. Коли я бачу здоровезну тітку на колінах у Гоголя, сприймаю, як напис \enquote{тут був Вася}. Тобто відмітилася...
\end{itemize} % }

\end{itemize} % }
