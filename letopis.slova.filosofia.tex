% vim: keymap=russian-jcukenwin
%%beginhead 
 
%%file slova.filosofia
%%parent slova
 
%%url 
 
%%author 
%%author_id 
%%author_url 
 
%%tags 
%%title 
 
%%endhead 
\chapter{Философия}
\label{sec:slova.filosofia}

%%%cit
%%%cit_head
%%%cit_pic
%%%cit_text
Народу нет места ни в одном, ни в другом формате.  О новом формате экономики,
которая вам не понравится.  Все что сейчас происходит в Украине многие
воспринимают искаженно, не хватает дальности обзора.  За деревьями не видят
леса.  А ключевой вопрос - это формат применяемой политической системы, но не
во внешних атрибутах, а в ее глубинной сути.  А глубинную суть раскрыл Карл
Шмитт - немецкий \emph{философ} и теоретик политических систем.  В его
представлении, универсальная борьба происходит между двумя концепциями:
государство-imperium (примат национальных интересов над глобальными) и
государство-dominium (подчинение государства глобальным интересам
транснациональных компаний.  Истоки его теории изложены в книге \enquote{Номос
Земли в праве народов}. Напомню, Номос - это древнегреческий бог законов и
указов
%%%cit_comment
%%%cit_title
\citTitle{Украина мучительно перерождается из страны олигархов в страну транснационалов}, 
Алексей Кущ, strana.ua, 17.06.2021
%%%endcit

%%%cit
%%%cit_head
%%%cit_pic
%%%cit_text
Но Сторож остановил его: — Не вы одни задумывались, кто я такой, и я отвечу
всем вам, но в должное время. Я сказал вам, что я Сторож, и так оно и есть. Но
вообще меня можно назвать \emph{философом}, хотя это слово не совсем точно подходит ко
мне. В вашем мире нет слова, которое точно бы соответствовало моему занятию.
Может быть лучше всего подойдет определение «\emph{философский} инженер». Но если вы,
мистер Джоунз, и вы, сэр Марк, хотите обсудить этот вопрос, я попрошу вас
немного подождать.  — Мы подождем с вопросом, — сказал Корнуэлл, — но одно я
все же хотел узнать.  Вы знаете наши имена, хотя мы вам их не называли
%%%cit_comment
%%%cit_title
\citTitle{Зачарованное паломничество}, Клиффорд Саймак
%%%endcit
