% vim: keymap=russian-jcukenwin
%%beginhead 
 
%%file 16_05_2021.fb.bilchenko_evgenia.2.knigi_filmy_osvobozhdenie
%%parent 16_05_2021
 
%%url https://www.facebook.com/yevzhik/posts/3896312180403863
 
%%author 
%%author_id 
%%author_url 
 
%%tags 
%%title 
 
%%endhead 
\subsection{БЖ. Как нас освобождают любимые книги и фильмы}
\Purl{https://www.facebook.com/yevzhik/posts/3896312180403863}

В общем, в этот критический период из дома практически исчез весь алкоголь
(странно, да?) Я поняла, что черные полосы заливать не могу и не хочу. Вместо
него появился (только не смеяться!) спазмомен (на нервной почве у меня - жуткие
боли в животе, о но-шпа можно забыть) и тот самый гидазепам, который выписал
гастроэнтеролог строго для ЖКТ, но я перебираю дозу, балансируя в рамках
границы высшей нормы в инструкции (я его пить уже тоже не хочу, к нему
привыкаешь нещадно, но мне необходим хотя бы один дневной транквилизатор
нормального действия: об адаптоле и прочих бифренах для легонько депрессирующих
дамочках можно забыть). Иначе - боли в душе (в животе) такие, что я ползаю по
хате под углом, не существующем в природе. 

\ifcmt
  pic https://scontent-bos3-1.xx.fbcdn.net/v/t1.6435-9/s720x720/186494978_3896219987079749_7166451059412465405_n.jpg?_nc_cat=107&ccb=1-3&_nc_sid=730e14&_nc_ohc=5qJNVzb1XjcAX9K00OP&_nc_ht=scontent-bos3-1.xx&tp=7&oh=55b9b8c61927b751c55aa4bce9930046&oe=60C6BB05
\fi

К чему был весь этот медицинский абзац-прелюдия? Не к битью на жалость. Боже
упаси. Просто после 18 января ничего не "сгладилось", ибо, чем дальше, тем
жестче я иду по пути радикального разрыва отношений с людьми, которые хотя бы
как-то толеризируют ужас, происходящий в моей стране (потому что это - уже не
аффект, а решение: мне в бою не нужны личности типа "и не друг и не враг, а
так"). При этом я полностью открыта к диалогу с людьми любых взглядов, мне не
важно: "правый" или "левый", традиционалист или марксист, классик или
постмодернист. Критерий - это то, что появляется на лице человека или
проскальзывает в его неуверенной речи при слове "нацизм". 

А к чему этот политический абзац-подводка? К тому, что, когда на лице у
человека, которого я считала другом, в очередной, сотый, раз при этом слове
появляется нечто размытое, а в речи его - нечто бормочущее, и я теряю человека
(причем, если раньше человек это ощущал, то теперь многие и не догадываются,
что они для меня потеряны: я научилась той самой, циничной, улыбке), я впадаю в
состояние, которое не снимается даже гидазепамом, но снимается четырьмя
любимыми книгами этого периода. Они даже не прячутся. Они стоят на полке в
положении и на расстоянии, удобном для вытянутой руки. Выбор - странный, сразу
предупреждаю. 

- Лев Толстой. "Анна Каренина".
- Анна Долгарева "Уезжают навсегда". 
- Славой Жижек "Как наслаждаться посредством другого".
- Аленка Зупанчич "Любовь как комедия".

Из последних потерь: из дома пропала Каренина. Это значит, что в очередном
припадке боли, когда я перечитывала то, что соответствует моему нынешнему
состоянию (сейчас - это история Кити и Левина, потому что, да, такая любовь
существует, это не сказки, раньше, конечно, я перечитывала историю Анны и
Вронского), моим локтем  драгоценный том восьмой был выпихнут за один из
диванов. Стыдно будет при количестве у меня полных собраний сочинений русских
классиков обретать сие в "Азбуке-классике" или у букинистов, но придется,
видно. С книгой Долгаревой вышло вообще так: я зачитала ее до дыр, до следов
пепла и пота на страницах, и мне ничего не оставалось, как на ее примере (одной
этой книги) показать сложнейшую проблему, которую взяли в РУДН (Москва).

Вообще - мистика отношений с РУДН - это отдельная история о добре, разуме и
любви. И еще история о том, что мне стало увлекательно соединять философию с
психоанализом, кино и поэзией. Это вообще интереснее нынешней украинской
культурологии арт-психоза и травма-памяти для меня сейчас в сто крат. И это - и
есть правдивая академическая культурология, из которой сейчас в нашей стране, в
принципе, официально убираются диалектика и психоанализ социума. Сейчас я не
пошутила: вчера мы это обсуждали. Следующей стала диалектика после Франкфурта и
русской семиотики. 

Из последних состояний: бывают такие, что не помогают вообще никакие книги,
даже эти, и не помогает гидазепам ни в какой дозировке. Есть последнее
средство. Я пересматриваю фильм (в какой раз, не знаю, счет идет на сотни или
тысячи) - "Служебный роман" Эльдара Рязанова. Почему именно он, я не знаю. Нет,
есть еще один, но здесь писать не буду, - начнется опять мое выпадение из
информполя. Я вас много раз засылала ко мне в Телегу и в ВК. Там более полные
варианты постов и другие репортажи.

И вот, сижу я, значит, и смотрю "Служебный роман" (то есть, реанимация по Егору
Летову идет по полной), и мне приходит мысль написать о роли поэзии в русском
кино. Не о "поэтическом кино", а именно о поэзии в кино, в русском кино, и
как-то отделить эти понятия. Получилось: и отделить, и сблизить. А сблизил их
тот, кто и придумал слово "поэтическое кино", которое почему-то украинские
искусствоведы пытаются монополизировать, - Пьер Паоло Пазолини. Есть его
теоретические семинары, очень советую. Их-то я и соединила со "Служебным
романом" и Беллой Ахмадулиной. 

Получилось нечто, что я писала, испытывая неимоверное чувство свободы. Делюсь с
вами копенгагенским сертификатом и ссылкой на свою статью. Считайте, что это
пост: как рвать с тем, что делало тебя зависимым, и наконец-то дышать воздухом
свободы (не путать с либерализмом, пожалуйста). 

Ссылка на: Бильченко Е.В. Поэзия в семиотике и психоанализе кино // Global And
Regional Aspects Of Sustainable Development Copenhagen, Denmark 4-5.05.2021. –
С. 281-290:

\url{https://www.interconf.top/archive.html}
