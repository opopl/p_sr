% vim: keymap=russian-jcukenwin
%%beginhead 
 
%%file 20_01_2022.fb.baumejster_andrej.kiev.filosof.1.vtorzhenie_analiz_bajden
%%parent 20_01_2022
 
%%url https://www.facebook.com/andriibaumeister/posts/4672231192898403
 
%%author_id baumejster_andrej.kiev.filosof
%%date 
 
%%tags analiz,biden_joe,rossia,ugroza,ukraina,usa,vtorzhenie
%%title Оказывается вторжения могут быть большими и небольшими!
 
%%endhead 
 
\subsection{Оказывается вторжения могут быть большими и небольшими!}
\label{sec:20_01_2022.fb.baumejster_andrej.kiev.filosof.1.vtorzhenie_analiz_bajden}
 
\Purl{https://www.facebook.com/andriibaumeister/posts/4672231192898403}
\ifcmt
 author_begin
   author_id baumejster_andrej.kiev.filosof
 author_end
\fi

Оказывается вторжения могут быть большими и небольшими! Появилась новая
формулировка... Внимательно слежу за событиями вокруг Украины. Обратите
внимание на выступление Джозефа Байдена в Белом доме 19.01. (годовщина
вступления в должность). Об Украине он говорит уже отвечая на вопросы. 

Вначале - что и как сказано. Вот краткое изложение от \enquote{Голоса Америки}: \enquote{Он
(Байден - А.Б.) предупредил президента РФ Владимира Путина, что Россия
столкнется «с очень серьезными последствиями» и будет «привлечена к
ответственности», если нападет на Украину. Он также предположил, что вторжение
станет «катастрофой» для России. По словам Байдена, союзники и партнеры США
готовы сделать так, чтобы российская экономика «заплатила серьезную цену».
Однако он допустил, что «небольшое вторжение» вызовет меньшую реакцию}.

Что здесь в первую очередь бросилось мне в глаза? Упоминание о разногласиях
НАТО по поведу реакции на \enquote{небольшое вторжение} России (minor incursion)

Это уже какое-то измерение \enquote{большого} и \enquote{небольшого} вторжения. За большое
вторжение вы ответите! Это будет для вас полной катастрофой! А если небольшое?
Нуууу..... 

Но ради всего святого, объясните мне, что такое \enquote{небольшое вторжение}!
Такое бывает? 

Оригинал: And so, I think what you’re going to see is that Russia will be held
accountable if it invades.  And it depends on what it does.  It’s one thing if
it’s a minor incursion and then we end up having a fight about what to do and
not do, et cetera. 

Цитирую дальше. 

\enquote{Байден рассказал, что, несмотря на достаточно откровенные дискуссии с
президентом России, он не вполне понимает Путина, а Путин не вполне понимает
его. По словам Байдена, в личной беседе он напомнил Путину, что Россия уже
оккупировала территории других стран с сомнительными для себя выгодами и что
оккупация Украины обойдется Москве очень дорого}. 

Прозвучала  цифра военной помощи САША Украине - 600 млн. долларов (I’ve already
shipped over \$600 million worth of sophisticated equipment, defensive equipment
to the Ukrainians). 

Там, где \enquote{Голос Америки} реферирует слова Байдена о том, что он не вполне
понимает Путина, в оригинале так: My conversation with Putin — and we’ve been —
how can we say it?  We have no problem understanding one another.  He has no
problem understanding me, nor me him.  And the direct conversations where I
pointed out — I said, \enquote{You know, you’ve occupied, before, other countries.  But
the price has been extremely high.  How long?  You can go in and, over time, at
great loss and economic loss, go in and occupy Ukraine.  But how many years?
One?  Three?  Five?  Ten?  What is that going to take?  What toll does that
take?}  It’s real.  It’s consequential. 

Мои вопросы:

- Так все-таки \enquote{не вполне понимает} или \enquote{нет проблем с пониманием}? 

- Какие \enquote{другие страны} (во множественном числе) имеются в виду? Да еще и
\enquote{прежде} (before).  

Кстати, мне было очень интересно читать изложение версии Байдена о разговорах с
Путиным. До этого мы имели скупые пресс-релизы. А теперь от первого лица
получили версию разговора двух президентов. Замечайте, фиксируйте. это очень
важно. 

Вопросов к Байдену по Украине было несколько. В ответе на второй вопрос я бы
отметил несколько сбивчивую речь. Слишком много неопределенных местоимений: I
think he will.  But I think he’ll pay a serious and dear price for it that he
doesn’t think now will cost him what it’s going to cost him.  And I think he
will regret having done it. 

Требование России в изложении Байдена: 

- Украина никогда не станет членом НАТО;

- НАТО не размещает стратегические вооружения в Украине. 

I’ve indicated to him — the two things he said to me that he wants guarantees
of it: One is, Ukraine will never be part of NATO.  And two, that NATO, or the
— there will not be strategic weapons stationed in Ukraine.  Well, we could
work out something on the second piece (inaudible) what he does along the
Russian line as well — or the Russian border, in the European area of Russia.

Что говорит Байден в возможном членстве Украины? Здесь словесная
эквилибристика: But the likelihood that Ukraine is going to join NATO in the
near term is not very likely... Да, четкость и определенность речи - не самая
сильная сторона американского президента. Прочтите ответы полностью. 

Еще меня заинтересовало в ответах Байдена, как он излагал мотивы Путина и давал
мастер-класс по политической истории и геополитике. Обратите внимание. Ссылки -
в комментариях. 

Продолжаю внимательно следить за политическими, дипломатическими и
семантическими играми в контексте эскалации международного конфликта. 

Почему это важно? Важно учиться внимательно следить за смещением акцентов, за
словами и фразами политических лидеров. В какой форме и с в какой тональности
высказывается то или иное утверждение. Когда большая часть политических и
дипломатических ходов скрыта от глаз и умов граждан, это иногда единственный
доступ к реальности.
