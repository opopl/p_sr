% vim: keymap=russian-jcukenwin
%%beginhead 
 
%%file people.gogol_nikolaj
%%parent people
 
%%url 
 
%%author_id 
%%date 
 
%%tags 
%%title 
 
%%endhead 
\chapter{Николай Гоголь}
\label{sec:people.gogol_nikolaj}

% Кто же вы, Николай Васильевич?
% https://zen.yandex.ru/media/id/5c6e907b6ecf9600c0144a52/kto-je-vy-nikolai-vasilevich-61eed54d80135218642853aa

%%%cit
%%%cit_head
%%%cit_pic
\ifcmt
  pic https://avatars.mds.yandex.net/get-zen_doc/5248867/pub_617a5f15aa83b66f09805d56_617a61066a98e21cd2c5eba4/scale_1200
  @width 0.4
\fi
%%%cit_text
А легендарный «борец с коррупцией» Сергей Лещенко, член наблюдательного совета
государственного акционерного общества «Укрзалізниця»? Не стесняясь, получает
зарплату в 400 тыс. гривен. Ничего не зная про эффективность каждой вверенной
его организации колёсной пары.  Но твёрдо уверенный — необходимо срочно
перевести движущий состав Украины... на немецкие рельсы. Когда неосмотрительный
журналист в прямом эфире старается узнать: какова сравнительная ширина
украинской и немецкой железнодорожной колеи, великий специалист... не знает,
что прогудеть в ответ.  Вот почему столь востребован \emph{Гоголь}, чтобы не
впадать в отчаяние украинскому народу.  Наблюдая каждый день стройные ряды
смешных образов, уже давно точно схваченных пером великого мастера. Всё те же
взяточники, казнокрады, невежды, глупцы, лжецы и недалекие врунишки
%%%cit_comment
%%%cit_title
\citTitle{Такие смешные: украинская политика... через призму творчества Гоголя}, 
Исторические напёрстки, zen.yandex.ru, 28.10.2021
%%%endcit

%%%cit
%%%cit_head
%%%cit_pic
%%%cit_text
Вирушаючи після навчання в Ніжині до Петербурга, \emph{Гоголь} був далекий від
містичних покликань, він взагалі не мав наміру ставати літератором чи
художником, він їхав "служить отечеству", а це, як відомо, найкраще робити в
столиці. Не в Миргороді ж, де в прославленій ним згодом калюжі брьохаються
свині, і не в провінційній Полтаві, де поліцмейстер боїться губернатора, а
губернатор остерігається поліцмейстера і тому не наважується звернути його
увагу на курей, що кубляться просто перед губернаторським особняком. Столиця —
чарівний магніт для всіх молодих обдарувань, притягальна сила для надій,
зухвальств і навіть для нахабств.  Але \emph{Гоголь} їде з найчеснішими
намірами. Про літературне майбуття, про славу він і не мріє. Він хоче "робити
благо". Як, де, яким чином? Це покаже майбуття. А тим часом він на зібрані
матір'ю гроші наймає на Гороховій (цьому Невському проспекті бідноти)
пристанище і замість високого неба і ясного місяця сцоєї Полтавщини вимушений
впиратися поглядом у брудно помальовані глухі стіни сусідніх будинків, смітники
й помийниці царської столиці
%%%cit_comment
%%%cit_title
\citTitle{Три долі. Гоголь, Шевченко, Чехов}, Павло Загребельний
%%%endcit
