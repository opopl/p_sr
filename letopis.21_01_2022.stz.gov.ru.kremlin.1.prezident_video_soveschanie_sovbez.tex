% vim: keymap=russian-jcukenwin
%%beginhead 
 
%%file 21_01_2022.stz.gov.ru.kremlin.1.prezident_video_soveschanie_sovbez
%%parent 21_01_2022
 
%%url http://www.kremlin.ru/events/president/news/67618
 
%%author_id gov.ru.kremlin
%%date 
 
%%tags putin_vladimir,rossia
%%title Президент в режиме видеоконференции провёл оперативное совещание с постоянными членами Совета Безопасности
 
%%endhead 
\subsection{Президент в режиме видеоконференции провёл оперативное совещание с постоянными членами Совета Безопасности}
\label{sec:21_01_2022.stz.gov.ru.kremlin.1.prezident_video_soveschanie_sovbez}
 
\Purl{http://www.kremlin.ru/events/president/news/67618}
\ifcmt
 author_begin
   author_id gov.ru.kremlin
 author_end
\fi

В совещании приняли участие Председатель Правительства Михаил Мишустин,
Председатель Совета Федерации Валентина Матвиенко, Председатель Государственной
Думы Вячеслав Володин, Заместитель Председателя Совета Безопасности Дмитрий
Медведев, Руководитель Администрации Президента Антон Вайно, Секретарь Совета
Безопасности Николай Патрушев, Министр обороны Сергей Шойгу, Министр внутренних
дел Владимир Колокольцев, директор Службы внешней разведки Сергей Нарышкин,
специальный представитель Президента по вопросам природоохранной деятельности,
экологии и транспорта Сергей Иванов, а также временно исполняющий обязанности
Министра по делам гражданской обороны, чрезвычайным ситуациям и ликвидации
последствий стихийных бедствий Александр Чуприян.

\ii{21_01_2022.stz.gov.ru.kremlin.1.prezident_video_soveschanie_sovbez.pic.1}

В.Путин: Уважаемые коллеги, добрый день!

У нас на сегодня – вопрос о дополнительных мерах по повышению пожарной
безопасности и снижению рисков возникновения чрезвычайных ситуаций в паводковый
период. Всё нужно делать своевременно, и обсудим сегодня вопросы, которые
предложены в качестве основных в повестке дня.

Но, перед тем как мы начнём работу, хотел бы отдельно остановиться на очень
важной теме, важной для миллионов наших граждан. Имеются в виду вопросы
пенсионного обеспечения. Совсем недавно мы приняли решение о том, чтобы поднять
пенсии, проиндексировать пенсии гражданским пенсионерам. И она [эта тема]
поднималась в ходе съезда партии «Единая Россия», также на недавней
пресс-конференции моей. Речь идёт о поддержке благополучия и сохранении доходов
людей старшего поколения в целом.

Как я уже сказал, мы приняли решение в отношении гражданских пенсионеров: с 1
января [пенсии] должны вырасти на 8,6 процента, то есть выше инфляции за
прошлый год.

Но, разумеется, мы не должны забывать и про военных пенсионеров. Считаю
необходимым применить такой же подход и в отношении военных пенсионеров и
приравненных к ним лиц. Имею в виду вышедших на пенсию сотрудников МВД,
Росгвардии, МЧС, Генеральной прокуратуры, Следственного комитета и других
структур. В целом это порядка двух миллионов 600 тысяч человек.

У нас, как вы знаете, ранее была запланирована индексация военных пенсий с 1
октября наступившего года и на четыре процента. Очевидно, это ниже фактической
инфляции, и ждать индексации людям, кроме того, пришлось бы ещё довольно долго. 

Вместо этого предлагаю здесь следующее решение, а именно: проиндексировать
пенсии военных пенсионеров и приравненных к ним лиц уже с 1 января текущего
года, и не на четыре процента, как было предусмотрено, а более высоким темпом –
на 8,6 процента, как и для всех остальных пенсионеров. Повторю, это выше
фактической инфляции за прошедший год. И прошу Правительство в самое ближайшее
время реализовать это решение, в том числе пересчитать уже выплаченные пенсии
за январь, чтобы положенные средства граждане получили вместе с ближайшей
пенсией.

Давайте перейдём к сегодняшней повестке дня.

<...>
