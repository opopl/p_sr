% vim: keymap=russian-jcukenwin
%%beginhead 
 
%%file 16_02_2022.stz.news.ua.strana.4.putin_ne_napal.3.bajden_spas_ukrainu
%%parent 16_02_2022.stz.news.ua.strana.4.putin_ne_napal
 
%%url 
 
%%author_id 
%%date 
 
%%tags 
%%title 
 
%%endhead 

\subsubsection{\enquote{Байден спас Украину}}
\label{sec:16_02_2022.stz.news.ua.strana.4.putin_ne_napal.3.bajden_spas_ukrainu}

Если для одних волна, раздутая западными медиа, стала уже очевидно фейковой, то
для сторонников США в Украине ситуация обстоит сложнее. 

Прямо опровергать вероятность \enquote{вторжения} они не могут. Как и заявлять,
что англо-американские власти вкупе со СМИ организовали поток ложных прогнозов
и утверждений. Чем нанесли серьезный урон украинской экономике. 

Поэтому они изобрели свое объяснение, почему Путин в очередной раз не напал.

Оно заключается в том, что Байден своим вчерашним выступлением за пять часов до
\enquote{вторжения} \enquote{спас Украину} и \enquote{напугал Путина}. 

О том, что это будет воспринято именно так, \enquote{Страна} говорила еще
вчера, после того как президент США обратился к нации. Ничего такого Байден по
сути не сказал, но в случае чего можно было бы потом говорить, что это
\enquote{внушение} вынудило Москву отменить операцию. 

Впрочем, в этой версии есть два уязвимых места.

Первое - об отсутствии угрозы вторжения все последние месяцы говорила
украинская власть. Второе - Запад и после несостоявшегося вторжения продолжает
раздувать панику вокруг нападения России, которое (по этой версии) Байден вроде
как уже предотвратил.

Тем не менее, \enquote{перемога} несется по украинской Сети на всех парах. 

\enquote{Джозеф Байден и Борис Джонсон, заняли принципиальную позицию по
недопущению вторжения РФ в отличие от Барака Обамы и Девида Кэмерона которые в
2014 не захотели проявить лидерство и жёсткими и решительными действиями не
заставили Путина отступить из Крыма и Донбасса. А наоборот потакали агрессору
своей мягкостью}, - написал в Facebook советник главы МВД Антон Геращенко.

Примерно о том же пишет нардеп Виктория Сюмар.

\ifcmt
  ig https://strana.news/img/forall/u/0/92/%D1%81%D1%8E%D0%BC%D0%B0%D1%80(5).png
  @wrap center
  @width 0.8
\fi

Главред \enquote{Цензора} Юрий Бутусов назвал речь Байдена \enquote{исторической}.

\enquote{Успешная стратегия применения мягкой силы Байдена-Джонсона, которую можно
назвать стратегия 16 февраля, поставила Путина на грань столкновения с США и
всеми странами НАТО. Байден выстраивает дипломатическую защиту Украины системно
и эффективно}, - заявил Бутусов.

\ifcmt
  ig https://strana.news/img/forall/u/0/92/%D0%B1%D1%83%D1%82%D1%83%D1%81%D0%BE%D0%B2(23).png
  @wrap center
  @width 0.5
\fi
