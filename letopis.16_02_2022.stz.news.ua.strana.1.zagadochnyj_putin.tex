% vim: keymap=russian-jcukenwin
%%beginhead 
 
%%file 16_02_2022.stz.news.ua.strana.1.zagadochnyj_putin
%%parent 16_02_2022
 
%%url https://strana.news/news/377077-the-sun-otmenil-vtorzhenie-putina-v-chem-prichina.html
 
%%author_id minin_maksim
%%date 
 
%%tags napadenie,putin_vladimir,rossia,ugroza,ukraina
%%title "Загадочный Путин". The Sun объяснил, почему "вторжения" России этой ночью не произошло
 
%%endhead 
 
\subsection{\enquote{Загадочный Путин}. The Sun объяснил, почему \enquote{вторжения} России этой ночью не произошло}
\label{sec:16_02_2022.stz.news.ua.strana.1.zagadochnyj_putin}
 
\Purl{https://strana.news/news/377077-the-sun-otmenil-vtorzhenie-putina-v-chem-prichina.html}
\ifcmt
 author_begin
   author_id minin_maksim
 author_end
\fi

Британский таблоид The Sun, который вчера заявил, что \enquote{вторжение} России в
Украину состоится в три часа ночи 16 февраля, отредактировал свою статью.

Теперь там говорится, что ночь, когда должна была состояться атака России,
прошла без происшествий.

\ii{16_02_2022.stz.news.ua.strana.1.zagadochnyj_putin.pic.1}

\enquote{Холодное ясное небо над Киевом, где люди приготовились к воздушному налету,
оставалось безмолвным}, - констатирует газета.

Отсутствие атаки The Sun объяснил так: \enquote{Путин и дальше заставляет Запад
теряться в догадках} (Putin continued to keep The West guessing).

\enquote{Но напряжение остается высоким}, - заключает издание.

Напомним, что сообщая о непременном нападении на Украину сегодня ночью,
британская газета рисовала настоящий апокалипсис: ракетные обстрелы городов и
атаку со всех направлений, и даже из Беларуси.

С аналогичной статьей вчера вышел британский таблоид Mirror, но он пока никак
не объяснил, почему война не началась. 

Впрочем, \enquote{Страна} вчера уже объясняла, что целью этих публикации скорее
всего было \enquote{опровергнуть} заявления Кремля о частичном отводе войск от
границ Украины. 

Кстати, на самой украино-российской границе в три утра было тоже тихо.

\url{https://t.me/stranaua/24929}

Вот, как выглядит украино-российская граница в три утра по Киеву (именно на это
время британские СМИ приурочили \enquote{вторжение}).

\ifcmt
  ig https://i2.paste.pics/cff423cfd5668bd91741ff162040ae93.png
  @width 0.4
	@wrap center
\fi

Пограничники, с которыми пообщалась \enquote{Страна}, подтвердили, что обстановка
спокойная.

По их словам, контрольно-пропускной пункт в \enquote{ночь вторжения} работает в штатном
режиме без какого-либо усиления.

Правда, по данным пограничников, пассажиропоток из РФ за последние дни упали.

Пока мы были здесь, на территорию России заехало четыре украинских автобуса, а
выехало обратно - три.

А вот самолетов и танков не видно.
