% vim: keymap=russian-jcukenwin
%%beginhead 
 
%%file 15_04_2023.stz.news.ua.donbas24.1.velykden_golovni_tradycii_zaborony_svjata
%%parent 15_04_2023
 
%%url https://donbas24.news/news/velikden-golovni-tradiciyi-i-prikmeti-svyata
 
%%author_id demidko_olga.mariupol,news.ua.donbas24
%%date 
 
%%tags 
%%title Великдень — головні традиції і заборони свята
 
%%endhead 
 
\subsection{Великдень — головні традиції і заборони свята}
\label{sec:15_04_2023.stz.news.ua.donbas24.1.velykden_golovni_tradycii_zaborony_svjata}
 
\Purl{https://donbas24.news/news/velikden-golovni-tradiciyi-i-prikmeti-svyata}
\ifcmt
 author_begin
   author_id demidko_olga.mariupol,news.ua.donbas24
 author_end
\fi

\ii{15_04_2023.stz.news.ua.donbas24.1.velykden_golovni_tradycii_zaborony_svjata.pic.front}
\begin{center}
  \em\color{blue}\bfseries\Large
Великдень — головні традиції і заборони свята
\end{center}

\begin{center}
\textbf{\color{blue} На Великдень не можна сумувати та скупитися}
\end{center}

Великдень вважається найдавнішим і найважливішим \href{https://donbas24.news/news/obmezennya-na-velikden-ta-pominalni-dni-v-doneckii-oda-zvernulis-do-gromadyan}{\emph{християнським святом}},%
\footnote{Обмеження на Великдень та поминальні дні: в Донецькій ОДА звернулись до громадян, Артем Батечко, donbas24.news, 13.04.2023, \par\url{https://donbas24.news/news/obmezennya-na-velikden-ta-pominalni-dni-v-doneckii-oda-zvernulis-do-gromadyan}}
що встановлено на честь Воскресіння Ісуса Христа. Дата Великодня кожного року
обчислюється за місячно-соняч\hyp{}ним календарем, що робить це свято переходящим.
Цьогоріч православні віряни і греко-католики святкуватимуть Великдень 16
квітня. Багато хто вірить, що на Великдень відкриваються небеса. З огляду на це
народні цілителі вважають, що період великодніх свят — найкращий момент для
зцілення від хвороб.

\textbf{Читайте також:} \href{https://donbas24.news/news/shho-ne-mozna-klasti-do-velikodnyogo-kosika-zaboroneni-produkti}{\emph{Що не можна класти до великоднього кошика — заборонені продукти}}%
\footnote{Що не можна класти до великоднього кошика — заборонені продукти, Еліна Прокопчук, donbas24.news, 12.04.2023, \par%
\url{https://donbas24.news/news/shho-ne-mozna-klasti-do-velikodnyogo-kosika-zaboroneni-produkti}%
}

\subsubsection{Історія і традиції свята}

Після того, як Ісус прийшов до Єрусалима, був зраджений і страчений на хресті,
а потім похований у печері відданими учнями — Він воскрес. За Євангеліями,
недільного дня, прийшовши до Його гробу, жінки-мироносиці виявили, що місце
поховання порожнє. Натомість побачили ангела, який сповістив їм, що Ісус
воскрес. Невдовзі одній із них явився Христос і повідомив про Своє воскресіння.
Радісна звістка надзвичайно швидко поширилася Єрусалимом, а потім і всією
Іудеєю. Саме тому в ніч проти неділі в усіх церквах правиться святкова
всенощна, кульмінацією якої є урочиста хода вірян навколо храму з запаленими
свічками, великодній подзвін і сповіщення священиком двохтисячолітньої звістки:
\enquote{Христос воскрес!}, на що паства радісно відповідає: \enquote{Воістину воскрес!}.

\ii{15_04_2023.stz.news.ua.donbas24.1.velykden_golovni_tradycii_zaborony_svjata.pic.1}

\textbf{Читайте також:} \href{https://donbas24.news/news/velikodnii-kosik-2023-skilki-dovedetsya-zaplatiti-za-svyatkovii-nabir-produktiv}{\emph{Великодній кошик-2023: скільки доведеться заплатити за святковий набір продуктів}}%
\footnote{Великодній кошик-2023: скільки доведеться заплатити за святковий набір продуктів, Тетяна Веремєєва, donbas24.news, 05.04.2023, \par%
\url{https://donbas24.news/news/velikodnii-kosik-2023-skilki-dovedetsya-zaplatiti-za-svyatkovii-nabir-produktiv}%
}

За традиціями на Великдень віряни відвідують церкву, освячують паски, воду,
продукти та крашанки. Освятити паски і крашанки потрібно до сходу сонця. У
церквах освячується і артос — квасний хліб. Його роздають вірянам для
зберігання вдома протягом року. Артос приймають натщесерце при хворобах. Також
є традиція цього дня стукатися крашанками. У деяких сім'ях на Великдень
обмінюються подарунками. У свято до батьків часто приїжджають діти, сім'ї
збираються великими компаніями за столом, спілкуються, відпочивають. Ще один з
атрибутів свята — квіти. Люди святять на службах живі або штучні букети. Потім
приносять їх додому, ставлять перед іконами, прикрашають ними святковий стіл.
Традиційними стравами на Великдень є холодець, домашні ковбаси, сало, запечене
молоде порося, гусак, фарширований яблуками, пироги з м’ясною і сирною
начинками. На святковій трапезі центральне місце займають паски, сирні паски та
фарбовані яйця. Починати трапезу потрібно саме з яєць. У деяких українських
сім'ях існує давня традиція вимірювати силу шляхом розбивання пасхальних яєць.
Вважається, що той член сім'ї, який розіб'є всі яйця і залишиться з цілим,
вважається найсильнішим.

\textbf{Читайте також:} \href{https://donbas24.news/news/shhiri-i-dusevni-privitannya-z-velikodnem-u-listivkax-prozi-ta-virsax}{\emph{Красиві привітання з Великоднем — у листівках, прозі та віршах}}%
\footnote{Красиві привітання з Великоднем — у листівках, прозі та віршах, Ольга Демідко, donbas24.news, 03.04.2023, \par%
\url{https://donbas24.news/news/shhiri-i-dusevni-privitannya-z-velikodnem-u-listivkax-prozi-ta-virsax}%
}

\ii{15_04_2023.stz.news.ua.donbas24.1.velykden_golovni_tradycii_zaborony_svjata.pic.2}

\subsubsection{Головні заборони свята}

прибирати, прати. Саме тому господині намагаються напередодні завершити всі
приготування на кухні, щоб в день свята бути вільними від домашніх турбот.

\begin{itemize}
  \item Забороняється займатися важкою фізичною працею. Не можна будувати, лагодити,
  \item Не рекомендується сумувати, ходити похмурим, грубити, лаятися з близькими людьми, скупитися.
  \item Краще не відмовляти в милостині або допомозі нужденним.
  \item Крім того, на Великдень не можна вінчатися та відвідувати кладовище.
  \item Не слід викидати освячену їжу — її радять віддати тваринам чи спалити.
  \item На Великдень не можна святити:
    \begin{itemize}
      \item алкоголь;
      \item страви з крові тварин;
      \item фрукти та овочі;
      \item матеріальні цінності.
    \end{itemize}
\end{itemize}

\ii{15_04_2023.stz.news.ua.donbas24.1.velykden_golovni_tradycii_zaborony_svjata.pic.3}

\textbf{Читайте також:} \href{https://donbas24.news/news/shho-poklasti-u-velikodnii-kosik-spisok-produktiv}{\emph{Що покласти у великодній кошик — список продуктів}}%
\footnote{Що покласти у великодній кошик — список продуктів, Еліна Прокопчук, donbas24.news, 13.04.2023, \par%
\url{https://donbas24.news/news/shho-poklasti-u-velikodnii-kosik-spisok-produktiv}%
}

\subsubsection{Основні народні прикмети}

\begin{itemize}
  \item На Великдень мороз — варто чекати на щедрий врожай.
  \item Якщо вдень на Пасху тепло, літо буде сонячним.
  \item Після освячення великодніх страв, потрібно роздати милостиню нужденним. За віруваннями, це допомагає залучити в дім достаток.
  \item Дощ на Пасху — до дощової весни.
  \item Якщо вдасться побачити великодній світанок, то слід очікувати удачу в справах.
  \item Нагодувати вуличних птахів хлібними крихтами — весь рік буде супроводжувати удача і багатство.
  \item Якщо на Великдень прохолодно, буде холодне літо.
  \item До Великодня розтане весь сніг — буде гарний урожай.
\end{itemize}

Текст підготовлено за наступними джерелами:\par\noindent\href{https://glavcom.ua/country/society/sogodni-velikden-tradiciji-zaboroni-ta-zvichaji-svyatkuvannya-840431.html}{glavcom.ua},%
\footnote{Сьогодні – Великдень: традиції, заборони та звичаї святкування, Дзвенислава Карплюк, glavcom.ua, 24.04.2022, \par\url{https://glavcom.ua/country/society/sogodni-velikden-tradiciji-zaboroni-ta-zvichaji-svyatkuvannya-840431.html}}
\href{https://chas.news/news/koli-vidznachayut-velikden-u-2023-rotsi-traditsii-i-zaboroni-svyata}{chas.news}%
\footnote{Коли відзначають Великдень у 2023 році: традиції і заборони свята, Chas News, 03.04.2023, \par\url{https://chas.news/news/koli-vidznachayut-velikden-u-2023-rotsi-traditsii-i-zaboroni-svyata}}

Раніше Донбас24 розповідав, які сімейні ігри можна \href{https://donbas24.news/news/yaki-simeini-igri-mozna-provesti-na-velikden-korisni-poradi}{\emph{провести на Великдень.}}%
\footnote{Які сімейні ігри можна провести на Великдень — корисні поради, Ольга Демідко, donbas24.news, 03.04.2023, \par\url{https://donbas24.news/news/yaki-simeini-igri-mozna-provesti-na-velikden-korisni-poradi}}

Ще більше новин та найактуальніша інформація про Донецьку та Луганську області
в нашому телеграм-каналі Донбас24

ФОТО: з відкритих джерел

\ii{insert.author.demidko_olga}
%\ii{15_04_2023.stz.news.ua.donbas24.1.velykden_golovni_tradycii_zaborony_svjata.txt}
