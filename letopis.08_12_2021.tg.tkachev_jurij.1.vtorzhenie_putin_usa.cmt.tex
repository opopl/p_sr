% vim: keymap=russian-jcukenwin
%%beginhead 
 
%%file 08_12_2021.tg.tkachev_jurij.1.vtorzhenie_putin_usa.cmt
%%parent 08_12_2021.tg.tkachev_jurij.1.vtorzhenie_putin_usa
 
%%url 
 
%%author_id 
%%date 
 
%%tags 
%%title 
 
%%endhead 
\subsubsection{Коментарі}

\ifcmt
  ig https://i2.paste.pics/2156ebd5fe0b7d3b676930279b757062.png
  @width 0.4
\fi

\begin{itemize} % {
\iusr{Ruslan}

Допустим, не только Путина. Но да, Путин один из главных защитников Украины.
Без Путина Украины давно уже не было бы.

\iusr{Вячеслав Шмарин}

Так наоборот, весь запад всеми доступными способами подталкивает Путина - ну
давай уже нападай, давай захвати ... Это и есть основная идея. Потому что
противостояние уже идет , но оно идет на политическом, экономическом  и
идеологическом фронте. И Штатам очень надо завязать нас на конкретике, чтобы
все остальные фронты были оголены.

\iusr{Kіевлянинъ}

Они этого не понимают. Политические украинцы до сих пор считают США нашими
партнёрами и братьями. У этих людей мозг испарился и пустая черепная коробка
которая рефлексирует на клич Слава Украини. За эти 8 лет народ деградировал и
оскотинился в край. Плюс пропаганда по ТВ и интернет СМИ помогает им верить в
параллельную реальность где Украина воюет с РФ и ее уважают в мировом масштабе.
Мои родители живо тому пример.

\iusr{Виталий Куртов}

Я вообще не понимаю в какой стране \enquote{Слуги}  живут) Но явно мы с ними в разных
Украинах живём) У них украинцев всё ещё 40 миллионов и против РФ партизнанить
будут старики, младенцы, жители Крыма и Донбасса)

\iusr{Вячеслав Шмарин}
\textbf{Kіевлянинъ}

Не так все просто. В 90-е Россию тоже захватили штаты. Их люди сидели во всех
кабинетах. И чтобы выкинуть их оттуда надо было очень кропотливо их
выдавливать. Потому что открыто им противостоять было невозможно. Сейчас такая
же ситуация на Украине.

\iusr{Vladique86}

Худшее, чем он бы столкнулся в этом случае - \enquote{экономические санкции}.

Эх, Юра-Юра! Как ты мог, о циник, забыть о КОПРОТИВЛЕНИИ, которое окажет в
едином порыве население чудо-государства? Как будут отчаянно бросаться на
\enquote{вторженцев} (правда, тех ещё уговорить надо вторгнуться, лол) наши земляки -
после многократно возросших тарифов, отката \enquote{активистам} с бизнеса  и битв за
урожай, украинизации, блокпостов \enquote{лиц без опознавательных знаков}, замерзающих
квартир и мега-пенсий! 

Как ты мог, о маловерный, усомниться в отчаянной борьбе за идеалы Майдана - то
есть за сайт \enquote{Миротворец}, пятилетние сроки за герб СССР на Facebook, Роттердам
Плюс и сгоревшие в банкопаде сбережения?

\iusr{Yury M}

Та ладно!!!? Как так то?? мы ж самую сильную армию в Европе создали! Мы же.. Мы
же сдерживаем российские войска, защищаем там всех в Европе от злого Путина!
вот.

\iusr{Александра Кормалева}

У меня складывается такое впечатление, что это игра штатов! Они своей игрой
толкают Украину на войну! И об этом знает Зеленский! Чем это может закончится
страшно и подумать! Но и бесконечно жить в подвешенном состоянии долго
продолжаться не может! Только один путь будет верным, это диалог Киева и
Донбасса!  Все посредники хотят что-то своего! Так что САМИ!

\iusr{Roma}

Нет, Западу нужно, чтобы Путин не мешал строить Антироссию. И это она нападет,
когда удобный момент возникнет. А пока её к этому готовят.

\iusr{motor unknown}

да всё будет хорошо, спите  спокойно)

\iusr{Вячеслав Шмарин}

Это вы оцениваете по западным меркам - выгодно/невыгодно. А Россия и Путин
живут по меркам справедливости. А если с этой точки зрения смотреть, то все
несколько сложнее. С одной стороны конечно и народ несет ответственность за то
что творится. Но далеко не полную.

\iusr{Вячеслав Шмарин}

Чтобы строить Антироссию - ее надо содержать, кормить и обогревать. А у запада
уже на себя денег не хватает. Поэтому он хочет чтобы тут все подрались, а запад
подливал бензинчику в огонь.

\iusr{MORGENSHTERN}

Вряд ли политические украинцы читают ваш блог.

\iusr{Александр}

\url{https://t.me/Shulyavskiy/190}

\obeycr
Не помню, но кажись, в тринадцатом году,
Транзитом Зрада Киев проезжала,
Дай, думает, по городу пройду,
Бывала раньше, но конкретно не гуляла,
Короче, ходит, смотрит, тратит наличман,
Красивый город, но видала интересней,
И Зрада выбрела, случайно, на Майдан,
А тот встречал ее лампадками и песней,
Что с нею сделалось, пером не описать,
Когда увидела на площади всю шоблу,
Бомжи, жлобы, на сцене мразей рать,
И все пропитано печалью, грустью, болью,
Пришлось уехать, много дел, и в том беда,
Но не прикажешь - не имеет сердце меры,
Вернулась Зрада, чтоб остаться навсегда,
И, кстати, сделала отменную карьеру...
\restorecr

\iusr{Telema}

0.01\% бандеровцев которые будут бухать в крыивках за деньги Запада. Вот какая
партизанская война ждет Путина. Но у Путина нет возможности. Тянуть всех это
придётся в социализм уйти.

\iusr{Roma}

Они строят. Каждый день это наблюдаю. И готовят её к войне.

\iusr{Telema}
И это плохо. Потому что я за дружбу, равенство, братство!

\iusr{Roma}
А кто будет разбираться, когда все ресурсы в руках нацистского государства?

\iusr{Вячеслав Шмарин}

Если мы считаем что народ братский, значит надо помогать. Но чтобы помогать
народ сам должен об этом заявить.

\iusr{Roma}

Да как бы Янукович в феврале 2014 года заявил. С нулевым результатом.

\iusr{Сергей Ившин}

\ifcmt
  ig https://i2.paste.pics/bab21fcf95151198379d570ca9588426.png
  @width 0.4
\fi

\iusr{Вячеслав Шмарин}

Идея неверна, долги надо набирать. И потом уже послать всех подальше. ПОтому
что прецеденты есть, например Великобритания подрезала деньги Венесуэлы. Так
что набирайте долгов и кредитов  - это пойдет западу на пользу, в другой раз не
будут лезть.

\iusr{Roma}

Народ? Как вы себе это представляете, всеукраинское Вече? Народ свою власть
через представительные органы власти осуществляет. Президент им и является.

\iusr{Telema}

1. Кому вам? А вам не надоело олигархов кормить?

2. То есть вы не считаете украинцев с русскими единым народом? Окей, критерий
различия?

\end{itemize} % }
