% vim: keymap=russian-jcukenwin
%%beginhead 
 
%%file 05_12_2021.fb.baumejster_andrej.kiev.filosof.1.skazka_andersen_snezhnaja_koroleva
%%parent 05_12_2021
 
%%url https://www.facebook.com/andriibaumeister/posts/4504764732978384
 
%%author_id baumejster_andrej.kiev.filosof
%%date 
 
%%tags andersen_gans_kristian,rozhdestvo,skazka
%%title Сказка Андерсена "Снежная королева"
 
%%endhead 
 
\subsection{Сказка Андерсена \enquote{Снежная королева}}
\label{sec:05_12_2021.fb.baumejster_andrej.kiev.filosof.1.skazka_andersen_snezhnaja_koroleva}
 
\Purl{https://www.facebook.com/andriibaumeister/posts/4504764732978384}
\ifcmt
 author_begin
   author_id baumejster_andrej.kiev.filosof
 author_end
\fi

Сегодня второе воскресение Адвента. В это время я стараюсь часть времени
посвящать медитациям и рассуждениям о событии Рождества. В это вечер я бы хотел
поделиться несколькими мыслями о сказке Андерсена "Снежная королева". 

\ii{05_12_2021.fb.baumejster_andrej.kiev.filosof.1.skazka_andersen_snezhnaja_koroleva.pic.1}

Эта сказка была опубликована 21 декабря 1844 года, год спустя после публикации
"Рождественской песни" Диккенса (о ней - в следующий раз). "Снежная королева"
содержит много загадок и тайн и может быть прочитана как философское, точнее,
теологическое послание. 

1. Сад и комната бабушки - это первозданный рай. Дети окружены розами, они
живут в своем безмятежном мире, исполненные любви друг к другу. В самый светлый
период их райской жизни, летом, когда Герда выучила Рождественский псалом
("Розы цветут... Красота! Красота // Скоро узрим мы младенца Христа") и когда
дети, взявшись за руки, глядя на солнце, чувствуют присутствие самого Младенца
Христа, сердце и зрение Кая поражают осколки зеркала Тролля. 

\ii{05_12_2021.fb.baumejster_andrej.kiev.filosof.1.skazka_andersen_snezhnaja_koroleva.pic.2}

Это момент отпадения от рая. Кай меняется, он видит мир в искаженном свете. И
следующей зимой его похищает Снежная королева. Герда пускается в опасное
путешествие. Преодолев множество опасностей, она находит Кая. Она исцеляет его
молитвами и горячими слезами любви. Перед нами сцена воскрешение Кая (ведь у
кого ледяное сердце, тот уже не человек, как говорила колдунья-финка). 

Когда дети возвращаются домой, в свой рай, они вдруг замечают, что стали
взрослыми. Бабушка сидит и читает Евангелие: "Если не будете как дети, не
войдете в царствие небесное". И снова звучит Рождественский псалом (третий раз
в этой сказке). Кай и Герда сохраняют свою детскую веру и любовь, но теперь они
возвращаются в рай уже зрелыми людьми. 

Этот мотив падение и восстановления, ухода и возврата, смерти и исцеления
(силой любви, силой молитв и, конечно, силой Христа) образует главную нить
повествования. 

2. Сюжет с зеркалом Тролля тоже является переосмыслением теологического сюжета.
Тролль (диавол) - завистник Бога. В его зеркале мир отражается в негативном
свете. Все великое и достойное в этом зеркале умаляется и высмеивается
(становится карикатурой), а все злое раздувается до максимальных размеров. Не
правда ли, очень современно! Зеркало разбивается на мелкие кусочки, но в каждом
осколке заключена вся сила дьявольского зерцала, сила искажать и умертвлять,
лишать веры и любви. 

3. И сама Снежная королева необычный персонаж. Когда королева второй раз
поцеловала Кая, он взглянул на нее: "Более умного, прелестного лица он не мог
себе и представить. Теперь она казалась ему совершенством". Между ними особая
связь. Это можно назвать подобием любви, скорее, анти-любви. Снежная королева -
надменная и холодная интеллектуалка, владычица ледяной и математически
выверенной вечности. Ее королевство - это своеобразная версия ада. 

У Андерсена здесь все время проступает образ холодного интеллектуализма,
воплощенного в математике. Кай увлечен точностью и строгостью фигур. После
своего "падения" он испытывает отвращение к розам. Но зато его восхищает
точность и искусность снежинок. "Какая точность! Ни единой неправильной линии"! 

Вместо "Отче наш" он, в минуту опасности, вспоминает лишь таблицу умножения. А
в самих чертогах Снежной королевы он увлечен "ледяными играми разума". Сам трон
Снежной королевы - это зеркало разума ("по ее мнению единственное и лучшее
зеркало в мире"). По заданию Снежной королевы Кай составляет из льдинок слово
"вечность". Если он сложит это слово, то станет "сам себе господином" и получит
в подарок весь мир и пару новых коньков. Как это прекрасно! "Я подарю тебе весь
свет и пару новых коньков". Детские коньки как бы оттеняют и немного понижают
пафос философского посыла для взрослых. 

Как удивительно! Андерсен обыгрывает этот древний философский идеал владения
своими страстями, идеал самопознания. Но для него - это холодный идеал.
Холодная философия, строгая математика без веры и любви, без жара сердца и без
Бога. 

4. В тексте противопоставление холодного разума и горячей веры - один из
главных мотивов. Воплощение холодного разума - математика. Образ живой веры -
Христос-Младенец. И детские сердца влюбленных. Живым розам противопоставляются
мертвые цветы (морозные узоры на окнах). Свет солнца противопоставляется тьме
ночи. Горячее любящее сердце - ледяному сердцу. А Божий мир, исполненный блага
- злым карикатурам Тролля.  

5. В сказке много совсем недетского, загадочного и трагического. Чего стоят
расспросы Герды, когда цветы ей рассказывают свои истории. Там и восходящая на
костер индийская вдова, любящая другого, но вынужденная принять смерть в
послушании бездушной традиции. И гробы трех сестер, плывущие по ночному озеру.
И нарцисс, грезящий о далекой изящной танцовщице ("полуодетой"). Наконец,
одуванчик, грезящий о девичьем поцелуе. 

Есть еще много интересный и важных деталей. О них я подробнее рассказываю в
своей беседе на Patreon-канале. 

\url{https://www.patreon.com/andriibaumeister} 

Если у вас появится свободная минутка - перечитайте внимательно эту сказку. Мне
она представляется одной из версий библейского сюжета. Это теология в форме
сказочного повествования. Это метафизические медитации данные в форме волшебной
истории. И эта история - не только для детей. Более того, для детей главные
смыслы "Снежной королевы" закрыты, еще не понятны. Дети могут воспринимать
только внешний сюжет, общую канву. 

Это сказка - для взрослых. Для тех, кто стал зрелым, но сохранил детскую
простоту сердца. "Если не будете как дети, не войдете в царствие небесное".
Ведь только чистые сердцем узрят Бога...

\ii{05_12_2021.fb.baumejster_andrej.kiev.filosof.1.skazka_andersen_snezhnaja_koroleva.cmt}
