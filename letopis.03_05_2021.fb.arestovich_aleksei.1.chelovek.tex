% vim: keymap=russian-jcukenwin
%%beginhead 
 
%%file 03_05_2021.fb.arestovich_aleksei.1.chelovek
%%parent 03_05_2021
 
%%url https://www.facebook.com/alexey.arestovich/posts/4291267337603980
 
%%author 
%%author_id 
%%author_url 
 
%%tags 
%%title 
 
%%endhead 
\subsection{Есть у человека идеал. Есть набор ценностей, порождаемый им: ценностей и принципов}
\label{sec:03_05_2021.fb.arestovich_aleksei.1.chelovek}
\Purl{https://www.facebook.com/alexey.arestovich/posts/4291267337603980}

- Есть у человека идеал. Есть набор ценностей, порождаемый им: ценностей и принципов.

\ifcmt
  pic https://scontent-bos3-1.xx.fbcdn.net/v/t1.6435-9/180450493_4291237350940312_2827454028160978794_n.jpg?_nc_cat=101&ccb=1-3&_nc_sid=8bfeb9&_nc_ohc=vU8v2K3KA_oAX9fi87v&_nc_ht=scontent-bos3-1.xx&oh=5384b391de81b062218c3cc2abef1cd8&oe=60B7044A
\fi

Есть жизненный поток, в котором человек плывёт. 

Есть разум, отвечающий за понимание, планирование, ориентацию, действия,
рефлексию.

Казалось бы, нет ничего проще, чем планировать и совершать действия,
ориентированные на достижение своих целей, воплощение идеала, поддерживать и
развивать свои ценности своими поступками, словом - жить, выполняя своё
предназначение.

На деле же данная задача представляет собой грандиозную проблему, едва ли не
главную для современного человека.

Во-первых, мало кто мыслит ситуацию потребным образом (задача, предназначение),
обычно это привилегия людей, нагруженных большими талантами, или результат
верной картины мира, которую при сегодняшних трендах в воспитании, взять
практически неоткуда: пусть предназначением мучаются всякие там Эйнштейны и
Стравинские, а меня мать-природа создала свободным!..

Поддержка и развитие своими поступками своих ценностей - жанр, практически
отсутствующий в нашей практике. 

Следование своему предназначению - жанр, практически отсутствующий в нашей
культуре.

Между тем, нет практически ни одной (!) жизненной проблемы, которая не была бы
связана с отсутствием вышеуказанной культуры, но близок локоть - да не укусишь!

Культура, в которой мы воспитаны, вставляет между человеческим разумением и
человеческим предназначением непробиваемую блокирующую систему (БС).

Состоит эта система из воспитания, образования, опыта и окружения, и носит
дисциплинарный характер: человека воспитывают не Личностью, но удобством, одним
из многих других удобств вокруг.

В результате, первичная моральная-волевая структура, лежащая в глубине души
человека, импульсы добра и красоты, своего (!) понимания мира и своей (!)
принципиальности просто не могут дойти до поверхности, реализоваться в
жизненном потоке в виде его конкретных поступков и целей.

Не мы живем нашу жизнь: ее проживает наша блокирующая система.

Чтобы стать собой и не придти в свящённый, но слишком поздний ужас на смертном
одре за свою, спущенную в отвал жизнь, нужно суметь стать выше своей
блокирующий системы: выше и сильнее своего воспитания, образования, опыта и
окружения.

Человечество стремительно несётся в новую эпоху, и тот, кто не умеет прыгать выше собственной головы, отстанет.

Нужно уметь соответствовать темпам и вызовам. А лучше - уметь опережать ограничивающие системы. 

Однако, даже для того, чтобы помыслить данную проблему, нужно очень много
внутренней честности.

Данный семинар является продолжением семинара «Внутренняя честность -1», и в то
же время является совершенно самостоятельным.

Нужно очень много внутренней честности, чтобы признать в себе ее недостаток. 

Речь идёт о хорошем контакте с самим собой: умением почувствовать первичные
импульсы своей морально-волевой системы и воплотить их в конкретные поступки;
обойти блокирующую систему.

Подробности - на семинаре 15 мая, 17-20.00.
