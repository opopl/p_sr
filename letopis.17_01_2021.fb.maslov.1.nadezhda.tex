% vim: keymap=russian-jcukenwin
%%beginhead 
 
%%file 17_01_2021.fb.maslov.1.nadezhda
%%parent 17_01_2021
 
%%url https://www.facebook.com/maslovevgeniy14/posts/1396483880693316
 
%%author 
%%author_id 
%%author_url 
 
%%tags 
%%title 
 
%%endhead 
\subsection{Крещенский Сочельник}
\Purl{https://www.facebook.com/maslovevgeniy14/posts/1396483880693316}
\ifcmt
  author_begin
   author_id maslov_evgenij
  author_end
\fi

В своих воспоминаниях митрополит Вениамин Федченков (1880-1961) обращает
внимание на паремию из Книги пророка Исайи в службе Крещенского сочельника. В
этой паремии мы слышим слова надежды:

«Возвеселится пустыня и сухая земля, и возрадуется страна необитаемая, и
расцветет как нарцисс! Великолепно будет цвести и радоваться... 

Укрепите ослабевшие руки и утвердите колена дрожащие! Скажите робким душою:
будьте тверды, не бойтесь: вот – Бог ваш; придет отмщение, воздаяние Божие! Он
придет и спасет вас. Тогда откроются глаза слепых, и уши глухих отверзутся.
Тогда хромой вскочит, как олень; и язык немого будет петь... 

И превратится... жаждущая земля в источники вод... 

И будет там большая дорога; и путь по ней назовется путем святым... 

Льва не будет там, и хищный зверь не взойдет на него... а будут ходить
искупленные... 

И радость вечная будет над головою их! Они найдут радость и веселие, а печаль и
воздыхание удалятся!» (Ис. 35).

Как нам нужна сейчас эта НАДЕЖДА!
