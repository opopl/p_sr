% vim: keymap=russian-jcukenwin
%%beginhead 
 
%%file 15_12_2020.news.lnr.lug_info.1.dostojanie_respubliki
%%parent 15_12_2020
 
%%url http://lug-info.com/news/one/molodezhnyi-lider-iz-krasnodona-stal-pobeditelem-konkursa-dostoyanie-respubliki-2020-foto-62824
 
%%author 
%%author_id 
%%author_url 
 
%%tags 
%%title Молодежный лидер из Краснодона стал победителем конкурса "Достояние Республики-2020"
 
%%endhead 
 
\subsection{Молодежный лидер из Краснодона стал победителем конкурса \enquote{Достояние Республики-2020}}
\label{sec:15_12_2020.news.lnr.lug_info.1.dostojanie_respubliki}
\Purl{http://lug-info.com/news/one/molodezhnyi-lider-iz-krasnodona-stal-pobeditelem-konkursa-dostoyanie-respubliki-2020-foto-62824}

\ifcmt
tab_begin cols=3
	pic http://img.lug-info.com/cache/6/e/(9)_IMG_0684.JPG/1000wm.jpg
	pic http://img.lug-info.com/cache/5/e/(3)_IMG_0667.JPG/1000wm.jpg
	pic http://img.lug-info.com/cache/9/6/(7)_IMG_0585.JPG/1000wm.jpg
	pic http://img.lug-info.com/cache/4/9/(4)_IMG_0578.JPG/1000wm.jpg
	pic http://img.lug-info.com/cache/0/8/(8)_IMG_0530.JPG/1000wm.jpg
	pic http://img.lug-info.com/cache/3/1/(6)_IMG_0525.JPG/1000wm.jpg
tab_end
\fi



\index[rus]{ЛНР!Достояние Республики-2020, 15.12.2020}
\index[names.rus]{Орлов, Олег!Победитель конкурса Достояние Республики-2020, ЛНР, 15.12.2020}

Молодежный лидер из Краснодона Олег Орлов стал победителем в главной номинации
официального республиканского конкурса "Достояние Республики-2020", подведение
итогов которого состоялось в Луганской государственной академии культуры и
искусств (ЛГАКИ) имени Михаила Матусовского. Об этом сообщила пресс-служба
Министерства культуры, спорта и молодежи (МКСМ) ЛНР.  

Шестой официальный республиканский молодежный конкурс "Достояние Республики"
стартовал в Республике 1 октября и проходил в 15 номинациях. Конкурс
организован с целью поддержки молодежных инициатив, консолидации усилий по
реализации в ЛНР государственной политики, направленной на работу с молодежью,
обмена опытом и инновационными технологиями.

"В конкурсе приняли участие более 250 молодых людей из всех регионов ЛНР.
Победитель в главной номинации – "Достояние Луганской Народной Республики" –
был избран из числа призеров в остальных номинациях. Им стал победитель в
номинации "Молодежный лидер года", главный специалист сектора профилактики и
социальной защиты семьи отдела по делам семьи и детей администрации Краснодона
и Краснодонского района, член Молодежного парламента ЛНР, координатор проекта
"Молодая гвардия" общественного движения "Мир Луганщине" в Краснодоне и
Краснодонском районе Олег Орлов", – говорится в сообщении.

Орлов отметил, что к участию в конкурсе его подтолкнуло "желание получить
оценку проделанной работы".

"В рамках реализации молодежной политики мы стараемся поддерживать различные
инициативы – мероприятия, образовательную деятельность. Сегодня чувствую
безумную радость. Я работал не ради конкурса, а ради поддержки молодежи. Мне
нравится работать с молодежью, и меня ни в коем случае не остановит победа в
конкурсе. Наоборот, появится стимул привлекать других людей к участию", –
рассказал он.  

\ifcmt
tab_begin cols=3
	pic http://img.lug-info.com/cache/c/0/(9)_IMG_0732.JPG/1000wm.jpg
	pic http://img.lug-info.com/cache/b/b/(7)_IMG_0725.JPG/1000wm.jpg
	pic http://img.lug-info.com/cache/4/4/(5)_IMG_0718.JPG/1000wm.jpg
tab_end
\fi

По итогам конкурса в номинации "Молодой медицинский работник года" победила
старшая медицинская сестра подземного пункта охраны здоровья шахты
"Самсоновская-Западная" Краснодонского лечебно-диагностического центра
"Ультрамед" Республиканской топливной компании "Востокуголь" Олеся Стадничук.

"Детским лидером года" стал житель Лутугино, заместитель председателя детской
организации "Юная гвардия" Семен Аблимов.

Лидером в номинации "Лучшая молодежная инициатива года" была признана
заместитель директора Института государственной службы и управления Луганского
государственного университета имени Владимира Даля Екатерина Гутько.

"Молодым работником профсоюзной организации" стала луганчанка Дарья Набока, с
2015 года являющаяся работником профессионального союза работников культуры
ЛНР.

В номинации "Молодой педагог года" лучшей была признана учитель химии
Стахановского учебно-воспитательного комплекса № 29, руководитель Стахановского
отделения Малой академии наук "Интеллект будущего" РФ Екатерина Глущук.

Специалист по социальной работе филиала № 2 Республиканского центра социальной
поддержки семей, детей и молодежи Кировска Екатерина Буравкова одержала победу
в номинации "Молодой работник социальной сферы".

"Молодым государственным служащим года" жюри назвало начальника отдела
молодежи, спорта и туризма администрации Лутугинского района Елену Сухих.

Победа в номинации "Молодой журналист года" досталась координатору проекта
\verb|#MediaГвардия| Анастасии Иванюк.

"Открытием года в области спорта" стал студент третьего курса отделения легкой
атлетики Луганского высшего училища физической культуры, мастер спорта ЛНР по
легкой атлетике Данил Павлов.

"Я занимаюсь легкой атлетикой уже 11 лет, добился довольно-таки неплохих
успехов и собираюсь, если позволит ситуация, выполнить норматив мастера спорта
международного класса. Поучаствовать в конкурсе не было моей инициативой, меня
уговорили старшие наставники. Конечно, я рад, что выиграл, но вообще все
участники – молодцы", – отметил спортсмен.

В номинации "Молодой предприниматель года" победил житель Алчевска,
предприниматель в сфере торговли, заместитель председателя Молодежного
парламента ЛНР, финалист конкурса "Лидеры Луганщины" Виктор Михайленко.

Звания "Молодой библиотекарь года" удостоилась сотрудник библиотеки-филиала № 1
для детей Алчевска Ирина Воробьева.

В номинации "Открытие года в области науки" лидером стал врач-дерматовенеролог
отделения экстренной и неотложной медицинской помощи Луганской республиканской
клинической больницы, ассистент кафедры фундаментальной и клинической
фармакологии Луганского государственного медицинского университета имени
Святителя Луки Богдан Кривоколыско.

"Открытием года в области культуры и искусств" жюри определило ассистента
балетмейстера образцового коллектива современного танца "Стар-данс"
стахановского городского Дворца культуры имени Горького Анну Скопич.

"В коллективе танцую пять лет, являюсь его выпускницей и пять лет работаю как
хореограф, так что мы вместе уже 10 лет. До боевых действий 2014 года мы были
многократными чемпионами мира и Украины, а сейчас являемся чемпионами ЛНР и
международных фестивалей как на территории Республики, так и в России. В
конкурсе "Достояние Республики" я принимала участие и в прошлом году. Тогда, к
сожалению, не получилось победить, и я решила попробовать еще раз. Я рада, что
мои достижения оценили по достоинству, это была одна из моих целей, и она
достигнута", – рассказала она.

Победители в каждой номинации получили статуэтки, дипломы и призы.

"Сегодня мы с огромным удовольствием в шестой раз подвели итоги
республиканского молодежного конкурса "Достояние Республики". В ЛНР более 395
тысяч молодых людей, которые ежегодно делают много полезных дел, каждый год
развивают все новые инициативы. В этот сложный для всей Республики год особый
подъем вызвали волонтерские организации, которые по зову сердца помогали
гражданам Республики", – отметил министр культуры, спорта и молодежи ЛНР
Дмитрий Сидоров.

Напомним, организатором конкурса выступает МКСМ ЛНР при поддержке структурных
подразделений администраций городов и районов ЛНР, осуществляющих деятельность
в сфере реализации государственной молодежной политики, а также органов
государственной власти ЛНР и профсоюзных организаций.

Ранее в ведомстве сообщали, что более 100 представителей городов и районов ЛНР
стали победителями первого, регионального, этапа республиканского молодежного
конкурса "Достояние Республики".

Ранее администрация Луганска наградила победителей первого (городского) этапа
республиканского молодежного конкурса "Достояние Республики".

Впервые молодежный конкурс "Достояние Республики" прошел в ЛНР в 2015 году.
Тогда в конкурсе приняли участие 124 человека. По результатам пяти этапов
состязания жюри из 10 независимых экспертов определило победителей в 24
номинациях, представлявших различные сферы жизни Республики.

В 2019 году на участие в конкурсе было подано 388 заявок от представителей всех
городов и районов Республики, проявивших себя в различных сферах.

ЛуганскИнформЦентр — 15 декабря — Луганск

