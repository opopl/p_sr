%%beginhead 
 
%%file 02_11_2020.fb.fb_group.mariupol.nekropol.1.otchet_volonter_rabota_31_oktjabrja_1_nojabrja_2020
%%parent 02_11_2020
 
%%url https://www.facebook.com/groups/278185963354519/posts/402169747622806
 
%%author_id fb_group.mariupol.nekropol,arximisto
%%date 02_11_2020
 
%%tags 
%%title Отчет о волонтерской работе в Некрополе 31 октября – 1 ноября 2020
 
%%endhead 

\subsection{Отчет о волонтерской работе в Некрополе 31 октября – 1 ноября 2020}
\label{sec:02_11_2020.fb.fb_group.mariupol.nekropol.1.otchet_volonter_rabota_31_oktjabrja_1_nojabrja_2020}
 
\Purl{https://www.facebook.com/groups/278185963354519/posts/402169747622806}
\ifcmt
 author_begin
   author_id fb_group.mariupol.nekropol,arximisto
 author_end
\fi

\textbf{Отчет о волонтерской работе в Некрополе 31 октября – 1 ноября 2020}

\textbf{Открытия и находки}

В субботу \href{https://www.facebook.com/profile.php?id=100004930356528}{Илья Луковенко}%
\footnote{\url{https://www.facebook.com/profile.php?id=100004930356528}}
и \href{https://www.facebook.com/andriynykyforenko}{Андрій Никифоренко}%
\footnote{\url{https://www.facebook.com/andriynykyforenko}}
почти полностью расчистили
неизвестный доселе склеп в ограде усыпальницы Спиридона Гофа. Закончить работу
помешал ливень...

Волонтеры обнаружили могилу архитектора Георгия Антоновича Крамаренко
(1897-1959). Кем он был? Об этом сделаем отдельный пост.

\textbf{Благоустройство}

В воскресенье \href{https://www.facebook.com/m.holovnova}{Maryna Holovnova}%
\footnote{\url{https://www.facebook.com/m.holovnova}}
и Андрей Марусов начали разметку пространства
вокруг склепов Гофов и Хараджаева. Где находятся заброшенные могилы? Где
прокладывать дорожки? Где садить цветы или деревья? Какие старинные плиты
следует поставить на прочное основание, чтобы их не занесло землей?

Если вы разбираетесь в ландшафтном дизайне – будем благодарны за помощь и
консультации.

Одновременно с разметкой пришлось выкапывать вездесущую дикую сирень – иначе в
следующем году всё снова превратится в беспорядочные заросли...

Огромная благодарность всем волонтерам за самоотверженную работу!

\textbf{Планы на выходные 7-8 ноября}

Мы изменили время сбора участников – теперь стартуем в 10 утра и в субботу, и в
воскресенье.

Место сбора неизменно – у белого Памятного креста в центре Некрополя. Приходите
и после старта - контактный телефон 096 463 69 88. Надеемся, что погода будет
более благоприятной...

Подробный анонс об очередной волонтерской экспедиции сделаем позднее...

Присоединяйтесь, друзья!

\#mariupol\_necropolis\_report
