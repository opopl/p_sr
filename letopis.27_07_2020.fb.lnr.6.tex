% vim: keymap=russian-jcukenwin
%%beginhead 
 
%%file 27_07_2020.fb.lnr.6
%%parent 27_07_2020
 
%%endhead 
  
\clearpage
\subsection{6 --- КОГДА ДАЖЕ ГИБЕЛЬ РЕБЁНКА --- ПОВОД ДЛЯ ХАЙПА}
\label{sec:27_07_2020.fb.lnr.6}
\url{https://www.facebook.com/groups/LNRGUMO/permalink/2881508928627329/}

Юлия Витязева:
КОГДА ДАЖЕ ГИБЕЛЬ РЕБЁНКА --- ПОВОД ДЛЯ ХАЙПА

Трагедия в Крыму.  Погиб 3-летний малыш.  Гулял во дворе. А потом исчез.  Два
дня его искали сотрудники полиции, кинологи, представители управления МЧС
России по Крыму. На поиски приехали несколько тысяч людей со всего Крыма.
Подключились и волонтеры из «Лизы Арлет».  26 июля управление МЧС России по
Крыму сообщило, что тело ребенка было обнаружено в выгребной яме возле дома.

Прокуратура возбудила дело по п. «в» ч. 2 ст. 105 УК России --- убийство
малолетнего лица.+

Самое страшное, что может случиться в жизни женщины — это потеря ребёнка.

Но кошмар этой истории — не только в трагедии одной отдельно взятой семьи. Дело
в том, что малыш был сыном фигуранта второго симферопольского «дела Хизб
ут-Тахрир*» Руслана Сулейманова.  То, что сейчас происходит в украинских
соцсетях можно охарактеризовать одним словом.  Это ад.  Все известные и
неизвестные вам «патриоты Украины», начиная с Березы и Муждабаева и заканчивая
агрессивными женщинами, страдающими крайней стадией русофобии, обвиняют в
случившемся Россию.  А точнее --- ФСБ.  Дескать, именно сотрудники этой
организации убили малыша.  С целью давления на его отца.  И взывать к логике,
доказывая абсурдность такого обвинения --- тщетно.  Они уже понесли свои домыслы
в форме утверждения по всему информационному пространству.

Особо ретивые добрались и до британских СМИ, так что в ближайшее время нас
наверняка ждёт череда статей о том, как «российские оккупанты убивают детей
крымско-татарских активистов».

И тут я вынуждена сказать чудовищную вещь.  О том, что если кому и была выгодна
гибель малыша, то именно тем, кто сейчас выжимает из этой истории максимум
политической составляющей и использует смерть ребёнка в качестве
«доказательства репрессий», которым, якобы, подвергаются на полуострове
крымские татары.  Так что дело по статье «убийство» возбуждено не зря.

Не знаю, как далеко зайдёт и чем закончится эта история.  Но вчера Зеленский в
телефонном разговоре с Путиным затронул вопрос освобождения крымского татарина
Руслана Сулейманова в связи с трагической гибелью его трехлетнего сына.

Что касается реакции украинского общества, то она отвратительна. Не только тем,
что эти люди, выключив режим здравомыслия, шлют проклятия и угрозы в адрес тех,
кого они назначили виновными.

Ужасно то, что, обливаясь слезами по-своему, они напрочь забывают о детях
Донбасса, которых за эти шесть лет убила и покалечила их страна и их армия.

Сегодня — шестая годовщина массированного удара по Горловке, в результате
которого погибло 27 человек.  В том числе — Горловская Мадонна Кристина Жук и
ее 10-месячная дочь Кира.

Думаете, на Украине сегодня вспомнят о них?  Говорю сразу — нет.  Потому что на
этом, в отличие от гибели мальчика в Крыму, не хайпануть.

Да-да! Я не оговорилась.  Смерть малыша для них — исключительно повод для
хайпа, сознательного, скоординированного и целенаправленного.

Потому что соболезнования матери мальчика принесли единицы. Зато выдвинули
обвинения и прокляли Россию уже тысячи.  Плевать им на мальчика.  И на татар
плевать.

Для них они всего лишь инструмент.  С помощью которого они всеми способами
пытаются сделать гадость России.  И даже если нынешняя трагедия произошла
случайно --- они будут использовать чужое для них горе в своих корыстных,
откровенно людоедских интересах.

И сделают со своей стороны все, чтобы смерть несчастного ребёнка принесла им
максимум дивидендов.

Потому что нелюди.

*организация запрещена в РФ Юлия Витязева, специально для News Front
  
