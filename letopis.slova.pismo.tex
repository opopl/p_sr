% vim: keymap=russian-jcukenwin
%%beginhead 
 
%%file slova.pismo
%%parent slova
 
%%url 
 
%%author 
%%author_id 
%%author_url 
 
%%tags 
%%title 
 
%%endhead 
\chapter{Письмо}

%%%cit
%%%cit_pic
\ifcmt
  pic https://avatars.mds.yandex.net/get-zen_doc/4612968/pub_604f437b126a3d455a2e3a4b_605230e13eb67941681f6780/scale_1200
	caption Граффити. Собор Св. Софии в Киеве.

	pic https://avatars.mds.yandex.net/get-zen_doc/3518430/pub_604f437b126a3d455a2e3a4b_60523275a779365202de90a2/scale_1200

	pic https://avatars.mds.yandex.net/get-zen_doc/2945823/pub_604f437b126a3d455a2e3a4b_605231e33eb6794168218ff8/scale_1200
	caption «Кузьма тать украл еси масо аселонь твои кокошь аминь». Буквально переводится «Кузьма – вор, украл мясо, чтобы тебе ноги спутало. Аминь», что является ничем иным, как пожеланием смерти. Граффити Софийского собора в Киеве.

\fi
%%%cit_text
Только по старинным \emph{письменным источникам} мы можем судить о том, как выглядел
этот храм в период расцвета Киевской Руси. Предполагается, что его окружали
широкие и просторные галереи-гульбища, из которых «вырастали» неокрашенные
кирпичные стены основного массива здания. Кровлю же собора венчали 13 глав,
динамично сужающимися кверху уступами
%%%cit_comment
%%%cit_title
\citTitle{Древнерусское зодчество. Собор Святой Софии в Киеве}, 
Искусство С Ириной Дружининой, zen.yandex.ru, 15.05.2021
%%%endcit
