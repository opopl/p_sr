% vim: keymap=russian-jcukenwin
%%beginhead 
 
%%file 13_06_2021.fb.bilchenko_evgenia.1.foto_moim_gostjam_kvartirnik
%%parent 13_06_2021
 
%%url https://www.facebook.com/yevzhik/posts/3976115772423503
 
%%author Бильченко, Евгения
%%author_id bilchenko_evgenia
%%author_url 
 
%%tags bilchenko_evgenia,kiev,kultura,kvartira,kvartirnik,literatura,poezia,ukraina
%%title Моим гостям - первые фото: как я вляпывалась
 
%%endhead 
 
\subsection{Моим гостям - первые фото: как я вляпывалась}
\label{sec:13_06_2021.fb.bilchenko_evgenia.1.foto_moim_gostjam_kvartirnik}
\Purl{https://www.facebook.com/yevzhik/posts/3976115772423503}
\ifcmt
 author_begin
   author_id bilchenko_evgenia
 author_end
\fi

Моим гостям - первые фото: как я вляпывалась.

Спасибо за добро, родня. Перед вами - фрагменты подготовки к квартирнику.
Постфактум: я так волновалась, так устала, что чуть ли не полпальца ножом
оттяпала, делая себе бутер после своего выступления и вашего ухода...

Оказалось, что вида крови я таки не боюсь. Вообще. Просто последний раз я
столько видела на войне. В мякоть ножиком попала. Теперь могу уверенно сказать:
стихи буквально даются кровью. Оказалось, что она перестала сворачиваться: а
врачи говорили, что она у меня запекается быстрее, чем у солдат. В квартире муж
уже отмыл следы \enquote{алых парусов} с новой жёлтой юбки (она, фу, стала оранжевой),
раковины, пола, дивана. Осталась лишь одежда одного из студентов. Если б я
сфоткала все это, а не цветы, вы бы расстроились, - и был бы шкандаль: \enquote{Типа БЖ
- cutter, ее наци довели до попытки суициднуться}. Прикиньте, был бы эффект? 

А так  - усталость и нежность. И ваши розовые к новому платью букеты. Такие
красивые... Николай Топало, спасибо за то, что ты один понял: я другая после
выступлений. Никто не понял ещё. Так четко. Vika Bondar, я не могла выйти тебя
встретить: шел международный стрим, но твоя роза ароматом уравновесила мою
лилию: а то я уже с ума сходила от запаха последней. 


\ifcmt
  tab_begin cols=2

     pic https://scontent-lga3-2.xx.fbcdn.net/v/t1.6435-9/199230775_3976115402423540_3932602975277304419_n.jpg?_nc_cat=108&ccb=1-3&_nc_sid=8bfeb9&_nc_ohc=sEvrkfXEywIAX8Zgin8&tn=ntrKbsW_7ChXu3v-&_nc_ht=scontent-lga3-2.xx&oh=f644297bdc9cab9339f1aeec8505a53e&oe=60CBC6C2

     pic https://scontent-lga3-2.xx.fbcdn.net/v/t1.6435-9/200009789_3976115522423528_5652728205540046903_n.jpg?_nc_cat=101&ccb=1-3&_nc_sid=8bfeb9&_nc_ohc=pjEi-KSYhsIAX_5-ySY&tn=ntrKbsW_7ChXu3v-&_nc_ht=scontent-lga3-2.xx&oh=f1dd48b801910b0fd45c03149f81561f&oe=60CA6495

  tab_end
\fi


Дарите нам цветы, любимые. \enquote{Пьета} ещё осталась. Новые книги - уже у
дизайнеров. Я вам готовлю бомбезность. Спасибо моим героям книг: поэзия о Зорге
и его жене Зорьке сделана по истории моих  дедушки с бабушкой (частично и
моей), а проза сочинена по великому Виктору Малахову, который делает меня
счастливой своей лаской, как и твой брат, \verb|#СергейВозняк|: не смогла сегодня на
его античную конференцию выйти, людей встречала, попроси прощения у моего
золотца. Кроме того, мы возобновили \enquote{Сентиментальное насилие либерализма}.
Скоро в продаже.

\ifcmt
  tab_begin cols=2

     pic https://scontent-lga3-2.xx.fbcdn.net/v/t1.6435-0/p600x600/199716233_3976115049090242_3414635975016762051_n.jpg?_nc_cat=102&ccb=1-3&_nc_sid=8bfeb9&_nc_ohc=NHXPB1wSuU4AX_d3nGf&_nc_ht=scontent-lga3-2.xx&tp=6&oh=d78756c69086feaebe24fe9fd91c29b6&oe=60CB9683

     pic https://scontent-lga3-2.xx.fbcdn.net/v/t1.6435-9/199463344_3976115239090223_6417219669728471032_n.jpg?_nc_cat=100&ccb=1-3&_nc_sid=8bfeb9&_nc_ohc=VXDbEYS8q4oAX-Pq0-V&_nc_ht=scontent-lga3-2.xx&oh=41931fa7ac1f3188cace67db02ce9b80&oe=60CAC2B5

  tab_end
\fi


Приятно начинать первое утро субботы после бессонной ночи с письма Юнны Мориц,
второе утро (час поспав) - с поддержки лучшего этика мира Виктора Ароновича,
день - с новой черной кепки, вечер - с вас, милые мои. Ночь же  - просто с
истерики наци. Обострение у больных пришло после вручения мне премиума-пакета
на Гугль-академии после резкого скачка международного индекса цитирования,
успеха на луцкой конференции и квартирника, где люди помогают мне публиковаться
и читают меня.


\ifcmt
  tab_begin cols=2

     pic https://scontent-lga3-2.xx.fbcdn.net/v/t1.6435-9/200009789_3976115522423528_5652728205540046903_n.jpg?_nc_cat=101&ccb=1-3&_nc_sid=8bfeb9&_nc_ohc=pjEi-KSYhsIAX_5-ySY&tn=ntrKbsW_7ChXu3v-&_nc_ht=scontent-lga3-2.xx&oh=f1dd48b801910b0fd45c03149f81561f&oe=60CA6495

     pic https://scontent-lga3-2.xx.fbcdn.net/v/t1.6435-9/199342945_3976115712423509_1240361857033169657_n.jpg?_nc_cat=111&ccb=1-3&_nc_sid=8bfeb9&_nc_ohc=wQPuqlrmFUgAX-4Us0o&_nc_ht=scontent-lga3-2.xx&oh=3f1c269b7d653037d6a82b6d3a7f4ec9&oe=60CBB311

  tab_end
\fi

Я так опустошена сейчас покоем, грустью, отрешенностью, порезом и концертом,
что не могу писать ночные статьи: обычно я спасаюсь письмом, так легче пережить
дурь вокруг себя. Я пишу, как дикая. Андрей Чупахин, Алексей Костромин, Настя
Бузиашвили, Надежда Сточко-Бабий, Настенька Иванютенко, чьи щелчки фото я
слышала: тэгните. Как я там сижу... Ещё не в крови. Люблю, целую, БЖ.

\emph{Алексей Костромин}

Фото не делал, но вот тебе скриншот)

\ifcmt
  pic https://scontent-lga3-2.xx.fbcdn.net/v/t1.6435-0/s261x260/199518863_10223417808866560_3747637404643574888_n.jpg?_nc_cat=105&ccb=1-3&_nc_sid=dbeb18&_nc_ohc=G7g7LnBZyvMAX9_TdNE&_nc_ht=scontent-lga3-2.xx&tp=7&oh=3a17d4ec6d5e1bfc8396dfef7fb4d9d7&oe=60CA5B45
	width 0.3
\fi

\emph{Евгения Бильченко}
\textbf{Алексей Костромин} Смехотули) здорово!

\emph{Tanya Ponomareva}

Меня, жалко, не было... Я б пощёлкала. Ну, ты в курсе, что я люблю это дело...
Цветы фееричные! Желтый наряд тоже прекрасен! Стихи мне понравились очень! С
голоса они смотрятся еще лучше, чем с листа. (Я просто а последнее время
торможу на незнакомый ритм. Деменция, видимо, на всю голову). Спасибо, было
прекрасно! Ночь моя была светла примерно до 2 часов, пока не села батарейка :)

\emph{Настя Бузиашвили}

Женя, спасибо Вам огромное за приглашение! Я чувствовала Ваше волнение, и как
Вы постепенно погружали нас в невероятную атмосферу Вашего богатейшего
творчества. Такое богатство образов и смыслов, чтобы понять такую поэзию, нужно
иметь большое сердце и большую эрудицию, широкий кругозор. Поэзия такого уровня
и правда дается кровью. Спасибо Вам огромное! Это был один из приятнейших
вечеров)

Опустошенность - приятное ощущение. На её место приходит гармония, спокойствие
и постепенно возвращается наполненность нашим вниманием, нашей благодарностью!

\emph{Евгения Бильченко}

Настя Бузиашвили низкий поклон)

\emph{Настя Бузиашвили}

Кстати, маникюр - бомба) как и вообще Ваш стиль

