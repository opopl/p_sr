% vim: keymap=russian-jcukenwin
%%beginhead 
 
%%file 29_11_2020.sites.ru.zen_yandex.yz.valerongrach.1.pop_14_vek
%%parent 29_11_2020
 
%%url https://zen.yandex.ru/media/valerongrach/obojaiu-etogo-bezymiannogo-popa-xiv-veka-vam-on-toje-mojet-ponravitsia-5fc34b744c127965dbe22b19
 
%%author Блог Valerongrach-а (Yandex Zen)
%%author_id yz.valerongrach
%%author_url 
 
%%tags istoria,14_vek,russia
%%title Обожаю этого безымянного попа XIV века. Вам он тоже может понравиться
 
%%endhead 
 
\subsection{Обожаю этого безымянного попа XIV века. Вам он тоже может понравиться}
\label{sec:29_11_2020.sites.ru.zen_yandex.yz.valerongrach.1.pop_14_vek}
\Purl{https://zen.yandex.ru/media/valerongrach/obojaiu-etogo-bezymiannogo-popa-xiv-veka-vam-on-toje-mojet-ponravitsia-5fc34b744c127965dbe22b19}
\ifcmt
	author_begin
   author_id yz.valerongrach
	author_end
\fi

\index[rus]{Книги!Пролог, сборник сокращенных житий святых и поучений на каждый день, 14 век}
\index[rus]{Русь!История!Записи попа на полях Пролога, 14 век, 29.11.2020}

В Российском архиве древних актов есть одна интересная старая книга. Очень
старый «Пролог» - сборник сокращенных житий святых и поучений на каждый день.
«Пролог» этот был создан в первой половине XIV века, скорее всего, где-то во
Пскове, потому что считается, что именно оттуда рукопись привезли в 1679 году
на Печатный двор.

Сам по себе «Пролог» вроде бы обычная рукописная книга. В ней нет красивых
иллюстраций. В лучшем случае буквицы выписаны. Вроде вот этой.

\ifcmt
pic https://avatars.mds.yandex.net/get-zen_doc/2783222/pub_5fc34b744c127965dbe22b19_5fc34bf34c127965dbe2cb94/scale_2400
\fi

Но в ней есть другое. Еще более интересное.

Листаешь себе его страницы. И вдруг на одной из них, на поле приписка. Внизу:

\begin{leftbar}
	\begingroup
		\em «Похмельне»
	\endgroup
\end{leftbar}

\ifcmt
pic https://avatars.mds.yandex.net/get-zen_doc/1897860/pub_5fc34b744c127965dbe22b19_5fc34c07b545e63488549759/scale_2400
\fi

Дальше листаешь, еще на одном поле, внизу:

\begin{leftbar}
	\begingroup
		\em «Плечо болить, а попадья пошла в гости».
	\endgroup
\end{leftbar}

\ifcmt
pic https://avatars.mds.yandex.net/get-zen_doc/1687249/pub_5fc34b744c127965dbe22b19_5fc34c284c127965dbe30c46/scale_2400
\fi

В голове тут же возникает анекдот

\begin{leftbar}
	\begingroup
		\em «Дорогой, я пошла к подруге Маше на пять минут, а ты каждые полчаса помешивай суп на плите»
	\endgroup
\end{leftbar}

Дальше еще прекраснее:

\begin{leftbar}
	\begingroup
		\em «Покуша»
	\endgroup
\end{leftbar}

\ifcmt
pic https://avatars.mds.yandex.net/get-zen_doc/1852544/pub_5fc34b744c127965dbe22b19_5fc34c4563d574041517ec71/scale_2400
\fi

И мы теперь знаем, что когда-то давно, неизвестный нам поп, как и мы в наше
время страдал от похмелья. Видно в один день так хорошо поел, что посчитал
нужным про это записать.

Он был не слишком богатым, этот поп, про которого мы никогда не узнаем, как его
звали. Он один из многих, оставшихся там, во глубине веков. Но из этой глубины,
из этих сотен лет, он говорит с нами о простых своих бедах и горестях.

И я читаю, и умиляюсь вот от этого:

\begin{leftbar}
	\begingroup
		\em «О горе мрут попы»
	\endgroup
\end{leftbar}

\ifcmt
pic https://avatars.mds.yandex.net/get-zen_doc/3397137/pub_5fc34b744c127965dbe22b19_5fc34c70d57ee92752aa6b8b/scale_2400
\fi

\textbf{А самое мое любимое:}

\begin{leftbar}
	\begingroup
		\em «Пойти в торг а кун нет лучше не иду»
	\endgroup
\end{leftbar}

\ifcmt
pic https://avatars.mds.yandex.net/get-zen_doc/3380298/pub_5fc34b744c127965dbe22b19_5fc34c8339eab6574dfc9ea7/scale_2400
\fi

Или просто ясно и четко

\begin{leftbar}
	\begingroup
		\em «Не могу» 
	\endgroup
\end{leftbar}

\ifcmt
pic https://avatars.mds.yandex.net/get-zen_doc/1589949/pub_5fc34b744c127965dbe22b19_5fc34c95d57ee92752aa91d1/scale_2400
\fi

У этого попа, похоже регулярно не бывало денег, но зато болело плечо, потому
что упоминания об этом попадаются не один раз. Вот еще:

\begin{leftbar}
	\begingroup
		\em «Плечо болит, не могу писать» 
	\endgroup
\end{leftbar}

\ifcmt
pic https://avatars.mds.yandex.net/get-zen_doc/1706869/pub_5fc34b744c127965dbe22b19_5fc34caad57ee92752aaae68/scale_2400
width 0.3
\fi

Похоже, что был он любитель выпить, потому что про похмелье упоминается
несколько раз. Но как же это, черт побери прекрасно, когда на старых страницах
ты видишь записи человека, которого давно нет. Это точно также как с
берестяными грамотами из Новгорода.

\textbf{Я обожаю тебя, неизвестный поп}. И надеюсь, что ты все-таки раздобыл
кун до следующего торга и твоя попадья не дала их тебе пропить. Жалко, что
записи об этом ты сделал не в этом "Прологе", если сделал их вообще.

Ведь время было непростое. XIV век. Москва режется с Тверью, ордынцы регулярно
налетают на Русь. Злое, жестокое, совсем нетолерантное и неплюшевое время. И
где-то там, в этом времени жил этот поп со своим больным плечом и попадьей...

Увидеть и полистать этот «Пролог» можно прямо в Интернете. Он оцифрован и
выложен на сайте РГАДА. Это ф. 381. Оп. 1. Ед. хр. 174.

\ii{29_11_2020.sites.ru.zen_yandex.yz.valerongrach.1.pop_14_vek.comments}


