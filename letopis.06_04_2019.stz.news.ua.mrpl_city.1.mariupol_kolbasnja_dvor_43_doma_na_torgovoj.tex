% vim: keymap=russian-jcukenwin
%%beginhead 
 
%%file 06_04_2019.stz.news.ua.mrpl_city.1.mariupol_kolbasnja_dvor_43_doma_na_torgovoj
%%parent 06_04_2019
 
%%url https://mrpl.city/blogs/view/mariupolskaya-kolbasnyao-dvore-43-doma-na-torgovoj-ulitse
 
%%author_id burov_sergij.mariupol,news.ua.mrpl_city
%%date 
 
%%tags 
%%title Мариупольская "колбасня" - о дворе 43 дома на Торговой улице
 
%%endhead 
 
\subsection{Мариупольская \enquote{колбасня} - о дворе 43 дома на Торговой улице}
\label{sec:06_04_2019.stz.news.ua.mrpl_city.1.mariupol_kolbasnja_dvor_43_doma_na_torgovoj}
 
\Purl{https://mrpl.city/blogs/view/mariupolskaya-kolbasnyao-dvore-43-doma-na-torgovoj-ulitse}
\ifcmt
 author_begin
   author_id burov_sergij.mariupol,news.ua.mrpl_city
 author_end
\fi

% 1 - Ул. Торговая, д. 43
\ii{06_04_2019.stz.news.ua.mrpl_city.1.mariupol_kolbasnja_dvor_43_doma_na_torgovoj.pic.1}

Лет сорок-пятьдесят назад этот двор на Торговой улице жители близлежащей округи
редко называли по номеру. Чаще его именовали \enquote{колбасней}. Люди пожилые, быть
может, и знали о происхождении названия, молодежь же вряд ли задумывалась о
таких \enquote{пустяках}. Также как не задумывается, наверное, большинство молодых
москвичей, каково происхождение слова \enquote{Арбат} или слова \enquote{Бессарабка} у киевлян.
Пожилые люди давно покинули наш бренный мир, а бывшая молодежь сама давно
переступила пенсионный рубеж. Казалось, история двора под номером 43 канула в
Лету.

\textbf{Читайте также:} 

\href{https://mrpl.city/news/view/oranzhevoe-nastroenie-chto-prigotovit-mariupoltsam-vo-vsemirnyj-den-morkovi-foto}{%
Оранжевое настроение: что приготовить мариупольцам во Всемирный день моркови?, Анастасія Папуш, mrpl.city, 04.04.2019}

\ii{06_04_2019.stz.news.ua.mrpl_city.1.mariupol_kolbasnja_dvor_43_doma_na_torgovoj.pic.2.kargina}

В школьные годы мы с \textbf{Эллой Антоновной Каргиной} были почти соседями. Однажды при
встрече стали вспоминать то, уже давно прошедшее время: перебирать имена общих
знакомых, дворы, в которых те обитали. Вдруг наш разговор споткнулся о слово
\enquote{колбасня}. \enquote{\emph{А знаешь, когда я была маленькой, моя мама - Людмила Ивановна -
водила меня в этот двор}, - сказала Элла, - \emph{она рассказывала, что в двухэтажном
доме, который выходит на улицу, прошло ее детство вместе с мамой, моей бабушкой
Альбиной, папой – это мой дедушка Иоган, - тетей и дядей, родными и двоюродными
братьями. Это здание принадлежало моему дедушке, а также его свояку и
компаньону. Их фабрика по изготовлению колбас и коптильня находились во дворе,
за домом}}...

На столе разложены документы. Одни - ветхие от времени листки бумаги, другие -
в виде свежих ксерокопий. Тут же старинные фотографии прошлого века и новенькие
снимки, сделанные на цифровую камеру. Элла Антоновна, обращаясь то к одной
бумаге, то к другой, излагает страницы истории своих предков. Вот что довелось
услышать от нее. В 1900 или 1901 году в Мариуполь приехали два чеха – \textbf{Войтех
Карасек и Иоган Вайц}. Они были женаты на родных сестрах \textbf{Брунсликовых}. Войтех -
на \textbf{Екатерине}, Иоган - на \textbf{Альбине}. Войтех занялся изготовлением колбас, а Иоган
обеспечивал это производство мясом: купил в Ейске скотобойню, и, кроме того,
доставлял в Мариуполь скот живым. Дела, судя по всему, шли весьма успешно, раз
уже через пару лет, а может быть и раньше, удалось своякам общими усилиями
построить и дом, и колбасную фабрику при нем, о которых здесь идет речь.

\textbf{Читайте также:} 

\href{https://mrpl.city/news/view/sejchas-budu-shhupat-nechto-mariupolskie-futbolisty-ispytali-chudoboks-video}{%
\enquote{Сейчас буду щупать нечто}: мариупольские футболисты испытали \enquote{Чудо-бокс}, Анастасія Селітріннікова, mrpl.city, 01.04.2019}

У Альбины Вайц было два магазина, в которых продавались колбасы: копченые и
вареные, сервелат и салями, нежнейшие франкфуртские сосиски и настоящие
пражские шпикачки. Один из магазинов был на Торговой улице ближе к
Итальянской, второй – на Екатерининской. В том из них, что на Екатерининской,
торговала сама Альбина. Впрочем, понятие \enquote{торговала} было весьма условно,
поскольку она появлялась у прилавка только в тех случаях, когда в заведение
захаживал кто-нибудь из Бахаловых, Арихбаевых, Найденовых или других уважаемых
в городе семей, менее значимыми покупателями занимались приказчики. Колбасными
изделиями Карасека лакомились не только мариупольцы. Значительное их
количество отправлялось в Москву, в знаменитый елисеевский магазин на
Тверской. Там должным образом оценили безупречное качество колбасной продукции
чехов из Мариуполя и посоветовали им заняться еще и изготовлением твердых
сортов сыров. Более того, порекомендовали пригласить известного сыровара из
подмосковных Мытищ, Василия Александровича Быкова. Ему предложили очень
приличное жалование, и он переехал в наш приморский город, чтобы заняться
своим ремеслом.

Жизнь в двухэтажном доме и колбасной фабрике на Торговой улице шла размеренно и
спокойно, и вдруг пошла череда несчастий. Осенью 1915 года здоровяк и силач
Иоганн Вайц переправлял скот на барже - она принадлежала фирме - из Ейска в
Мариуполь. Неожиданно бык-вожак выскочил за борт, а за ним двинуло все стадо.
Иоганн прыгнул в холодную морскую воду, стараясь подогнать животных к берегу,
на мелководье. Спасение скота обернулось для Вайца сильной простудой,
скоротечной чахоткой, забравшей его жизнь за каких-то два или три месяца.
Альбина стала вдовой с тремя сыновьями и дочкой. К тому же предприятие
требовало крепкой мужской руки. И тут пришел на помощь сыровар Василий
Александрович, ставший на долгие годы опорой семьи. В 1919 году на глазах у
всей родни скоропостижно скончалась Екатерина Карасек.

\textbf{Читайте также:} 

\href{https://mrpl.city/news/view/v-donetskoj-oblasti-vareniki-dorozhe-chem-v-stolitse}{%
В Донецкой области вареники дороже, чем в столице, Олена Онєгіна, mrpl.city, 29.03.2019}

Стоит ли удивляться тому, что за годы, последовавшие после Октябрьской
революции, дело предприимчивых и трудолюбивых чехов рухнуло? Дом и фабрику у
них забрали. Сыновья Вайца - \textbf{Петр и Павел} - вместе с двоюродными братьями
\textbf{Ростиславом и Виктором Карасеками} присоединились к отступающей армии. Для них
это был единственный способ бежать из России, охваченной Гражданской войной, и
попытаться вернуться на родину. Удалось это только Ростиславу. Судьба остальных
братьев неизвестна. Быть может, они погибли в боях, или кого-то из них сразил
сыпной тиф, они могли умереть от испанки, и не исключено, что их просто
расстреляли. Войтех Карасек также решил немедленно покинуть страну, где ему
стало страшно жить, где он лишился всего своего состояния, нажитого ценою
упорного труда. Альбине пообещал, что как только он с семьей обустроится в
Чехии, то заберет ее и детей - \textbf{Людмилу и Ивана}. Предваряя последующие события,
скажем: этот план оказался нереализованным. Советская власть укрепилась, двери
за границу плотно закрылись, и Альбина с дочерью и сыном навсегда остались в
Мариуполе.

Войтех с женой Ростислава \textbf{Верой}, урожденной \textbf{Запорожцевой}, с двумя ее детьми
\textbf{Виктором и Евгением}, старшему тогда было десять лет, младшему – девять, с
большими трудностями и лишениями через Турцию добрались до Чехии. В конце
января 1921 года они уже были в Праге. А без малого через десять месяцев ушла
из жизни его невестка Вера, ей было всего двадцать девять лет. Связь между
потомками сестер Екатерины и Альбины прервалась на многие десятилетия, а точнее
– навсегда.

\textbf{Читайте также:} 

\href{https://mrpl.city/news/view/mrpl-gastrofest-mariupoletsev-ozhidaet-gastro-tur-po-raznym-stranam-mira-foto}{%
\enquote{MRPL. ГастроФест}: мариупольцев ожидает \enquote{гастро-тур} по разным странам мира, Олена Онєгіна, mrpl.city, 28.03.2019}

В 2003 году Элла Антоновна отправилась в Чехию, чтобы отыскать корни своих
предков. Встретили там гостью из Украины более чем доброжелательно, подсказали
с каких архивов нужно начинать поиски. Следует отметить, что в чешских архивах
хранение документов находится в идеальнейшем порядке. Основываясь на
воспоминаниях Эллы Антоновны, почерпнутых в свое время от мамы, архивариусы,
перерыв горы папок, нашли документы, касающиеся Ростислава, его жены Веры и
детей, с которыми они бежали в двадцатых годах из России. Разыскали они и
бумаги, из которых следовало, что Войтех Карасек родился 27 ноября 1858 года в
городе Ичине, который находится на северо-востоке Чехии на реке Цидлине,
притоке Эльбы, что умер он 28 ноября 1923 года в Праге. Они даже указали точный
адрес дома Карасеков в Ичине. И вот Элла Антоновна прибыла в этот чистенький
чешский городок. Она идет по указанному адресу и вдруг останавливается. Перед
ней строение по своей планировке и фасаду в точности похожее на дом на
мариупольской Торговой улице - он тоже двухэтажный, разве что этажи чуть ниже.
Она подходит ближе - номер строения тот же, что указали в пражском архиве. Да,
это и есть дом Карасеков. Стало быть, попав на чужбину, Войтех постарался
построить дом в Мариуполе таким же, каким он был у него на родине. Только
пожить в нем довелось ему недолго.
