% vim: keymap=russian-jcukenwin
%%beginhead 
 
%%file 18_04_2021.fb.zharkih_denis.1.ukraina_rossia_kapitalizm_bratstvo
%%parent 18_04_2021
 
%%url https://www.facebook.com/permalink.php?story_fbid=2968992876647443&id=100006102787780
 
%%author 
%%author_id 
%%author_url 
 
%%tags 
%%title 
 
%%endhead 

\subsection{Сложности войны}
\label{sec:18_04_2021.fb.zharkih_denis.1.ukraina_rossia_kapitalizm_bratstvo}
\Purl{https://www.facebook.com/permalink.php?story_fbid=2968992876647443&id=100006102787780}

В ленте идут сражения начнется ли атака России на Украину (в Украину?) или это
все чушь? Проблема в том, что ситуация в регионе имеет противоречивый характер.

\subsubsection{Современная Украина не нужна РФ от слова \enquote{совсем}}

Экономика современной РФ формировалась при Ельцине, как сырьевой придаток
Европе. Поэтому Украину отпустили просто так, как бедных родственников. Помню,
как Маргарет Тетчер сказала: "СССР - это не новые рынки, это потребители без
денег". Ну вот, примерно так современная Украина просто экономический груз для
РФ. 

Кто сомневается - пусть посмотрит на Беларусь. Вот уж кто прислонялся к России,
да так до конца и не прислонился. И дело не в общем желании. Экономическая
система России не позволяет такой интеграции. То есть, Россию теоретически
интересует и общая денежная система с соседями, и рынок для своих товаров (кто
еще вазы купит?), и рабочие руки, и кооперация. Но экономической системы входа,
кроме как переломить через колено (не Беларусь, а собственные экономические
интересы) - нет. В 2014 году Украина убежала от экономического союза с Россией.
Много кто в руководстве России порвал три баяна на радостях. 

Сегодня, когда Запад отрывает от России Беларусь, то либеральные российские
экономисты, многие из которых до сих пор управляют экономикой, держат кулачки.
А вдруг оторвут пиявку? А тут Украина, Украина... 

\subsubsection{Повторить не можем}

Дело в том, что только социалистический СССР мог дать отпор фашизму. Это только
в советской культуре Плохиш отрицательный персонаж, в постсоветской его долго и
упорно навязывали, как положительный. 

Советская система строилась на коллективизме, который нынче оплеван. При этом
культурные деятели РФ явно переплюнули украинских коллег. Европейцы не могли
справиться с фашизмом, поскольку это не было для них вопросом жизни, они
старались с ними договориться, чтобы не делать лишних жертв. 

Современная Россия пытается договориться с Украиной, а смысл этого в пункте
первом: завоевать можно, но кормить придется. Поэтому даже если Россия и
решиться на удар, то, скорей всего, опять будут возобновлены переговоры,
которые результат этого удара уничтожат.  

И дело не в том, что россияне глупые. Капиталистическая система, особенно та,
что в России и Украине, не желает никого освобождать, а, тем более, кормить.
Украина тоже не разбежалась освобождать Донбасс и Крым, кормить их,
восстанавливать, спасать от тех, кого они объявили захватчиками и оккупантами.
Ведь если россияне мучают честных украинцев на Донбассе, то их надо спасать. Но
нет, Донбасс идут не спасать, а завоевывать, наказывать за сепаратизм и измены. 

С Россией зеркальная ситуация - согласно пропаганде миллионы соотечественников
в руках фашистов. По всем канонам надо идти и спасать братьев. Но проще
объявить всех граждан Украины фашистами и их прислужниками, чем следовать своей
же легенде. Украинскими фашистами пугают самих россиян.   Да и откуда взять
массовый героизм советских людей?

\subsubsection{Методы устарели}

Помню, как Стрелков требовал Украину разбить военным путем, а потом с ней
разбираться. Ну вот Россия разбивает ВСУ (там полно Плохишей) и...? Понимать,
что Москва хочет от Украины нужно до, а не после разгрома. А Москва ничего
особо не хочет, смотри пункт первый. 

А вот США понимают, что хотят, и планомерно это делают. Но США не торопиться
вводить войска в Украину, ее завоевали информационным путем. Современные
бандеровцы не выползли из схронов, а большей частью руководятся либеральной
прослойкой управленцев, созданной США. Внизу это работяги, лишенные работы
деиндустриализацией,  вверху это комсомольско-олигархическая верхушка,
контролируемая США (кто не знает, Бандера давно погиб, да и тот...). 

То, что Россия не добилась восстановления четырех каналов, напрямую говорит,
что эти каналы не российские. Да и Медведчук, несмотря на беспредел, не уехал в
Россию. Если бы он был агентом России, ему бы угрожала, как минимум, тюрьма. Но
нет. Получается, что для Украины российская угроза только повод к внутренним
разборкам, там Россию особо не боятся и ее интересы принято не учитывать.
Захват территорий тоже не беда, а повод у Запада денег попросить. Украине
большинство депрессивных регионов просто не нужны, как Украина России. 

Это капитализм, господа! Никто никому не нужен. Какое там, на фиг, братство,
справедливость и честность? Бизнес и ничего личного. Важно правильно надуть
ближнего. Американцы пока рекордсмены в этом отношении, куда нам.
