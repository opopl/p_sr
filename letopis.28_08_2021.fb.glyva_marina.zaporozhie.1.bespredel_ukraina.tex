% vim: keymap=russian-jcukenwin
%%beginhead 
 
%%file 28_08_2021.fb.glyva_marina.zaporozhie.1.bespredel_ukraina
%%parent 28_08_2021
 
%%url https://www.facebook.com/permalink.php?story_fbid=2980256102233943&id=100007487479006
 
%%author_id glyva_marina.zaporozhie
%%date 
 
%%tags bespredel,konstitucia,obschestvo,prava_cheloveka,strana,ukraina
%%title Беспредел в Украине
 
%%endhead 
 
\subsection{Беспредел в Украине}
\label{sec:28_08_2021.fb.glyva_marina.zaporozhie.1.bespredel_ukraina}
 
\Purl{https://www.facebook.com/permalink.php?story_fbid=2980256102233943&id=100007487479006}
\ifcmt
 author_begin
   author_id glyva_marina.zaporozhie
 author_end
\fi

ХОЧУ, ЧТОБЫ ОБЩЕСТВО  ЗНАЛО, КАКОЙ БЕСПРЕДЕЛ ПРОИСХОДИТ В УКРАИНЕ, КАК ЗДЕСЬ
ИЗДЕВАЮТСЯ НАД ЖЕНЩИНАМИ  И НАРУШАЮТ ПРАВА ЧЕЛОВЕКА И КОНСТИТУЦИЮ

Можно ли в Украине добиться правды и справедливости? Когда нарушаются
Конституция, права человека и полный беспредел становится нормой и нет такой
инстанции, которая может наказать виновных. В детстве я читала много сказок, в
них добро всегда было сильнее и побеждало зло, но вот теперь сама жизнь мне
показала, что зло сильнее и наглее,  а добро бесправно и беззащитно. 

Второй раз жизнь меня ставит в ситуацию, когда приходится отстаивать свои права
в высших инстанциях, и хотя закон и правда на моей стороне, беспредел
чиновников оказывается выше законов. Уже несколько месяцев меня терроризирует
Запорожский Департамент социальной защиты, который незаконно требует с меня
вернуть социальное пособие матери-одиночки, которое они  мне выплатили год
назад в начале июля 2020 года в соответствии с Постановлением Кабинета
Министров от 8 апреля 2020 года о том, что в период карантина выплата
продляется в июле автоматически. 

Потом 22 июля Кабинет Министров принял новый закон, в соответствии с которым
выплата безработным матерям-одиночкам отменяется.  Согласно 94 статье
Конституции Украины все законы вступают в силу через 10 дней после их
официальной публикации, но не раньше принятия. А социальная служба утверждает,
что принятый 22 июля 2020 года закон об отмене пособий вступает в силу
прошедшим временем 1 июля 2020 года. Два раза оставляла жалобу на урядову
горячу линию, жалобу передают не в вышестоящий орган, а в Департамент, на
который я жалуюсь. 

Вчера  они прислали очередное письмо, в котором очередная ложь: пишут, что 28
октября 2020 года Кабинет Министров принял закон, который обязывает вернуть
социальную выплату за июль 2020 года.  То есть через семь месяцев одним законом
отменяют другой и требуют возврата.  Позвонила снова на горячую линию
государственной юридической консультации, юрист сказала, что законы обратной
силы не имеют, не могут вступать в силу задним числом и что в законе за октябрь
2020 года нет требования возвращать социальную выплату, юрист посоветовала
обратиться Министерство социальной политики. 

Я написала   сегодня в Министерство социальной политики, они отослали меня в
Национальную сервисную службу. Хотьба по замкнутому кругу.  Нервы и здоровье
мне все эти департаменты и министерства не вернут.  Я не чувствую себя
человеком в этой стране, нигде не могу добиться правды и справедливости @igg{fbicon.cry}{repeat=3} 

\ii{28_08_2021.fb.glyva_marina.zaporozhie.1.bespredel_ukraina.cmt}
