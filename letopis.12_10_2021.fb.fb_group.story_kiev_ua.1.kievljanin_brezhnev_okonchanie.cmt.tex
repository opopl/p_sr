% vim: keymap=russian-jcukenwin
%%beginhead 
 
%%file 12_10_2021.fb.fb_group.story_kiev_ua.1.kievljanin_brezhnev_okonchanie.cmt
%%parent 12_10_2021.fb.fb_group.story_kiev_ua.1.kievljanin_brezhnev_okonchanie
 
%%url 
 
%%author_id 
%%date 
 
%%tags 
%%title 
 
%%endhead 
\subsubsection{Коментарі}

\begin{itemize} % {
\iusr{Александр Митряев}
Прям таки, кардинал. Я жил в то время и кардиналом был некто Суслов, начетчик.

\begin{itemize} % {
\iusr{Георгий Майоренко}
\textbf{Александр Митряев} Суслов был идеологом.

\iusr{Станіслав Теліцький}
\textbf{Aleksandr Mitryaev}, 

Суслова називали "сірим кардиналом". Ця алегорія застосовувалась для означення
осіб, котрі не маючи прямих офіційних повноважень, чинили значний вплив на
ухвалення рішень, формування політики тощо. Сутана кардинала має червоний
колір. Сірий кардинал - несправжній кардинал у сірій сутані, впливова особа,
котра не посідає високе положення в церковній ієрархії. Суслов був секрктарем з
ідеології ЦК КПРС.

\begin{itemize} % {
\iusr{Георгий Майоренко}
\textbf{Станіслав Теліцький} К тому же, Суслов был на виду. Его нельзя отнести к закулисным политикам. Он из так называемого малого Политбюро - совета старейшин, куда входили Брежнев, Андропов, Устинов и Суслов.
\end{itemize} % }

\end{itemize} % }

\iusr{Dmytro Vovk}

У меня в музыкальной школе был учитель по флейте Григорий Эммануилович Поляков.
Прям напомнило детство.

\begin{itemize} % {
\iusr{Георгий Майоренко}
\textbf{Dmytro Vovk} Да, сочетание похожее! Поляков тоже киевлянин?

\begin{itemize} % {
\iusr{Dmytro Vovk}
\textbf{Георгий Майоренко} да. Работал в симфоническом оркестре Гостелерадио и музыкальной школе \#6 на БВС

\iusr{Георгий Майоренко}
\textbf{Dmytro Vovk} Так вы ему расскажите про Георгия Эммануиловича. Или ссылку пришлите на сюжет. Человеку будет приятно.

\iusr{Dmytro Vovk}
\textbf{Георгий Майоренко} к сожалению, он уже не с нами.

\iusr{Георгий Майоренко}
\textbf{Dmytro Vovk} Светлая память творческому человеку!
\end{itemize} % }

\end{itemize} % }

\iusr{Protsenko Sergii}
Что с мемуарами? Вышли?

\begin{itemize} % {
\iusr{Георгий Майоренко}
\textbf{Protsenko Sergii} Веду трудные переговоры с родственниками. Есть надежда, что удастся выпустить.
\end{itemize} % }

\iusr{Sam Zilman}
Очень интересный рассказ.
Я лично знал многих из руководителей Украины

\begin{itemize} % {
\iusr{Георгий Майоренко}
\textbf{Sam Zilman} Здорово! Если лично знали, то расскажите. Это же история!

\begin{itemize} % {
\iusr{Sam Zilman}
\textbf{Георгий Майоренко} Есть интересные истории с Леонидом Макаровичем Кравчуком

\iusr{Георгий Майоренко}
\textbf{Sam Zilman} Кравчук -яркий исторический персонаж! Думаю, это интересно.

\iusr{Sam Zilman}
\textbf{Георгий Майоренко} 

Леонид Макарович слава Богу жив и здравствует, а просить сейчас у него
разрешения на раскрытие некоторых деталей из его личной жизни мне как-то
неудобно.

В действительности,, многое из его жизни уже написано и есть интересные детали...

Вот можете послушать из его собственных уст.

\url{https://youtu.be/qlOUmwGdMKs}

\iusr{Георгий Майоренко}
\textbf{Sam Zilman} Насколько я знаю, по правилам группы запрещены ссылки. Вас могут за ссылку забанить.

\iusr{Sam Zilman}
\textbf{Георгий Майоренко} Ну так я уйду отдыхать.
Если админы группы посчитают не нужным и не важным послушать речь живого 1-го Президента их страны, то это только будет минус им...
Справка. Я уже 32 года живу в Америке и общаюсь с людьми со всего Мира, в том числе, с сотнями киевлян, потому что я сам блоггер
\end{itemize} % }

\iusr{Tatyana Mazerati}
rasskajite, pojaluysta!

\end{itemize} % }

\iusr{Елена Полякова}
Интересно, спасибо

\begin{itemize} % {
\iusr{Георгий Майоренко}
\textbf{Елена Полякова} Рад, что понравился рассказ. Удачи и добра!
\end{itemize} % }

\iusr{Алексей Вознюк}
А точно Цуканов, а не Пономарев был завотделом?

\begin{itemize} % {
\iusr{Георгий Майоренко}
\textbf{Алексей Вознюк} У Понамарева была своя группа консультантов. Как и у Александрова-Агентова, Голикова и др. Но у Георгия Эммануиловича была именно та самая "ударная" группа (Бовин, Арбатов, Черняев), которая потом плавно перетекла к Яковлеву.

\begin{itemize} % {
\iusr{Алексей Вознюк}
\textbf{Георгий Майоренко} Я дневники Черняева перечитал. Они в сети есть. Изданы ккгигой. И завотделом был Б.Н.

\iusr{Георгий Майоренко}

Шикарные дневники у Черняева!!! Черняев - международник. Я тоже удивляюсь, что
он постоянно в дневниках: "Сидим у Цуканова, пишем..." Видимо, его привлекали,
как специалиста?

\end{itemize} % }

\end{itemize} % }

\iusr{Владимир Киевский}

Так вроде на место Брежнева было несколько кандидатов помимо Щербицкого -
Романов, Машеров...

\begin{itemize} % {
\iusr{Георгий Майоренко}
\textbf{Владимир Киевский} Там борьба шла жестокая. Но Щербицкий был из земляков. Ведь он, как и Брежнев работал на Днепродзержинском металлургическом заводе. Кстати, на этом же заводе трудились брат Щербицкого и брат Брежнева. Такая заводская династия. И Цуканов оттуда. Брежнев его забрал в Москву с должности главного инженера завода.

\iusr{Ольга Гураль}
Машеров был 1м секретарем ЦК в Беларуссии, и ему КГБ устроило автокатастрофу в Минске еще в 70-х, задолго до смерти Брежнева! Уж больно популярным и самостоятельным стал.

\begin{itemize} % {
\iusr{Георгий Майоренко}
\textbf{Ольга Гураль} Да, безумно жаль Машерова! Был одним из самых толковых руководителей той поры!
\end{itemize} % }

\end{itemize} % }

\iusr{Olena Klymenko}
Спасибо, за очень интересный рассказ.

\begin{itemize} % {
\iusr{Георгий Майоренко}
\textbf{Olena Klymenko} Благодарю за добрые слова. Удачи и благополучия!
\end{itemize} % }

\iusr{Ирина Косянчук}

Удивлена Киевской родословной Цуканова. На моей родине в г. Днепродзержинске на
ул. Институтской висит мемориальная доска Цуканову. Там же долго проживала и его
мама.

\iusr{Георгий Майоренко}

Как здорово! Я про мемориальную доску даже не знал. Маму его Антонину
Владимировну хорошо помню. Тетя Тоня! Все никак не выберусь в Днепродзержинск.


\iusr{Михайло Наместник}
Цікаво було б почитати його спогади, бо й про його особистість майже нічого немає. Не дарма ж його звали "сірий кардинал".

\begin{itemize} % {
\iusr{Георгий Майоренко}
\textbf{Михайло Наместник} Будем надеяться, что смогу "выцыганить" у родственников рукописи воспоминаний Георгия Эммануиловича. Ему было что рассказать.
\end{itemize} % }

\iusr{Микола Веселий}
Спасибо, узнал много нового. А подковерная борьба что при социализме, что при капитализме неотъемлемый атрибут власти...

\begin{itemize} % {
\iusr{Георгий Майоренко}
\textbf{Микола Веселий} Вечная тема борьбы за власть. Получается, человечество без этого не может.

\iusr{Ольга Писанко}
\textbf{Микола Веселий} огромное спасибо за такой обстоятельный рассказ ! Много интересного «из первых уст»!

\begin{itemize} % {
\iusr{Ольга Писанко}
Простите, Георгий, мой коммент предназначался для Вас!

\iusr{Георгий Майоренко}
\textbf{Ольга Писанко} Благодарю. Хочется, конечно, знать об этих людях больше. Живу надеждой, что удастся каким-то образом добраться до его мемуаров. Но родственники, у которых они хранятся - сложные в общении люди.

\iusr{Ольга Писанко}
\textbf{Георгий Майоренко} думаю, они просто боятся возможной волны негатива ...

\iusr{Георгий Майоренко}
\textbf{Ольга Писанко} По поводу коммента, я так и понял.

\iusr{Георгий Майоренко}
\textbf{Ольга Писанко} Возможно и негатива не хотят. А может, планируют их продать за хорошую сумму. Но одно дело, если тексты попадут к историкам, которые их опубликуют. Но могут же попасть к случайным людям и будут где-то пылиться. Одним словом, есть проблема.
\end{itemize} % }

\iusr{Анатолий Крупин}
\textbf{Микола Веселий} 

Возможно так оно так и есть. Не исключено, что здесь тоже срабатывают свои
законы. Но при социализме продвижение возможно было если есть мозги. Оно так и
было. Почему-то вспомнилось выражение Щербицкого: "Нехай краще працюють
білоголові, ніж тупоголові".

\begin{itemize} % {
\iusr{Микола Веселий}
\textbf{Анатолий Крупин} Да, если мажоры рождались тогда тупыми, их старались не светить во власти. Где-то там они числились, может и какими-то начальниками, пользовались всеми благами папаш, но были далеко не на первых ролях
\end{itemize} % }

\end{itemize} % }

\iusr{Natasha Levitskaya}

Спасибо, Георгий. Интересно было прочитать все истории! Какая яркая биография у
вашего знаменитого родственника, и вы очень хорошо описали политические интриги
того времени, тем более мы жили в те времена и все имена нам известны. Может
быть у вас получиться опубликовать мемуары Георгия Эммануиловича! Успехов вам!

\begin{itemize} % {
\iusr{Георгий Майоренко}
\textbf{Natasha Levitskaya} Спасибо, Наталья, за добрые слова. Даже если я и не смогу уговорить родственников скопировать для меня воспоминания Георгия Эммануиловича, все равно постараюсь выпустить книгу по тем материалам, которыми располагаю. Удачи и добра!

\iusr{Natasha Levitskaya}
\textbf{Георгий Майоренко}

Браво! Удачи и успехов, Георгий!
У вас всё получится! Читая ваши публикация о предках, понимаешь, как это всё непросто восстановить, разыскать и фото, и историю жизни этих людей. @igg{fbicon.hands.applause.yellow}  @igg{fbicon.thumb.up.yellow} 
\end{itemize} % }

\iusr{Татьяна Ховрич}
Спасибо! Интересно.

\begin{itemize} % {
\iusr{Георгий Майоренко}
\textbf{Татьяна Ховрич} Рад, что понравился пост! Удачи и хорошего дня!

\iusr{Татьяна Ховрич}
\textbf{Георгий Майоренко} , взаимно! Пишите ещё! Здорово!

\iusr{Александр Шукевич}
\textbf{Георгий Майоренко}, Благодарю, за Очень Интересную публикацию !

\iusr{Георгий Майоренко}
\textbf{Татьяна Ховрич} Спасибо на добром слове! Успехов и благополучия!
\end{itemize} % }

\iusr{Валентин Тим}
От якби вдалось видати "непідредаговані" спогади Цуканова? А

\begin{itemize} % {
\iusr{Георгий Майоренко}
\textbf{Валентин Тим} Было бы круто издать! Но не от меня это зависит. Они находятся у его внучки. Она пока не хочет ими делиться с остальными родственниками. Но, надеюсь, совесть у нее взыграет...

\iusr{Валентин Тим}
\textbf{Георгий Майоренко} А може порадьте їй запропонувати їх Дмитру Гордону. Так і книжка може реально з'явиться і внучка залишиться задоволеною? @igg{fbicon.wink} 

\iusr{Георгий Майоренко}
\textbf{Валентин Тим} Издать не проблема. Важно, чтобы у внучки было желание этим заниматься. Пока на этом направлении - глухо!
\end{itemize} % }

\iusr{Ольга Волынец}
\textbf{Георгий Майоренко} , с открытым ртом прочла Вашу публикацию, увы первую пропустила. Потрясающе, спасибо. @igg{fbicon.hands.applause.yellow} 

\begin{itemize} % {
\iusr{Георгий Майоренко}
\textbf{Ольга Волынец} Благодарю. А вы зайдите на мою страницу в группе "Киевские истории". Там четыре публикации, посвященные Георгию Эммануиловичу Цуканову. Эта последняя - завершающая.

\iusr{Ольга Волынец}
\textbf{Георгий Майоренко} ,благодарю Вас. @igg{fbicon.hands.applause.yellow} 

\iusr{Ольга Волынец}
\textbf{Георгий Майоренко} , наша и скопировала, спасибо, еще раз.

\iusr{Георгий Майоренко}
Удачи и добра!
\end{itemize} % }

\iusr{Тамара Ар}
Интересно!...Но! Сейчас кланов хватает, в том числе и днепропетровских......

\begin{itemize} % {
\iusr{Георгий Майоренко}
\textbf{Тамара Ар} 

Горбачев ликвидировал днепропетровцев-брежневцев. Кадрами у него заведовал
Лигачев. Он и "зачистил" старую гвардию. А в период перестройки на политическом
небосводе взошли другие Днепропетровские звёзды - Кучма, Лазаренко, Тимошенко,
Турчинов. Это уже другое поколение и другая эпоха.

\iusr{Тамара Ар}
\textbf{Георгий Майоренко} ну да, и так далее
\end{itemize} % }

\iusr{Александр Шелест}
Мой дядя самых честных правил... ушла воровская банда...

\begin{itemize} % {
\iusr{Сергей Оборин}
\textbf{Александр Шелест} А, Вы, по случаю, не родственник ли, Петра Ефимовича Шелеста, 1-го Секретаря ЦК КП Украины? Так, по идее, должны были быть однодумцами. Или, подковерная борьба не дала ходу карьере?

\iusr{Александр Шелест}
я не родственник и не однодумец, никогда не был членом кпсс, а Вы не родственник ли ГБшников, что подобные вопросы задаете?.. или просто - пробел в воспитании?.. А с карьерой и благосостоянием у меня все ОК...

\iusr{Георгий Майоренко}
\textbf{Александр Шелест} А вы считаете, что ваш столь грубый и бестактный комментарий - свидетельство высокого уровня культуры? Всё-таки общаемся в уважаемой киевской группе. Не в генделыке.

\iusr{Александр Шелест}
и тут коммуняки достали...

\iusr{Георгий Майоренко}
В том, что у вас отсутствует культура общения виноваты коммуняки?
\end{itemize} % }

\iusr{Людмила Старовойтенко}
Спа ибо за интересный рассказ многого не знали

\begin{itemize} % {
\iusr{Георгий Майоренко}
\textbf{Людмила Старовойтенко} Благодарю за добрые слова! Удачи!
\end{itemize} % }

\iusr{Владимир Картавенко}
Дід Путіна годував царську, потім сімю Леніна. Пропонував перенести столицю імперіі в Україну...

\begin{itemize} % {
\iusr{Георгий Майоренко}
\textbf{Владимир Картавенко} А это к чему? История о киевлянах.

\iusr{Владимир Картавенко}
\textbf{Георгий Майоренко} В гавани Почайна киевсовет планирует за подарком Путина В В построить музей ПОБЕДА

\iusr{Георгий Майоренко}
\textbf{Владимир Картавенко} Не вижу взаимосвязи с постом. Рассказ в моем сюжете о делах давно минувших дней.
\end{itemize} % }

\iusr{Ирина Середа}

Спасибо за приоткрытые створки событий брежневских времён, прочла с интересом
все ваши публикации

\begin{itemize} % {
\iusr{Георгий Майоренко}
\textbf{Ирина Середа} Благодарю. Конечно же, хотелось бы знать об этих людях гораздо больше. Удачи вам, и добра!
\end{itemize} % }

\iusr{Электрозвук Плюс}
Старая Борщаговская, новая Шулявка, Квадрат потом.

\begin{itemize} % {
\iusr{Георгий Майоренко}
\textbf{Электрозвук Плюс} Да, Квадрат это уже с 1966 года. А до этого - старинная Шулявка! Борщаговская 30. Позже 24.
\end{itemize} % }

\end{itemize} % }

