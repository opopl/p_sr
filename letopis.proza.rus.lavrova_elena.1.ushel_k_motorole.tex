% vim: keymap=russian-jcukenwin
%%beginhead 
 
%%file proza.rus.lavrova_elena.1.ushel_k_motorole
%%parent proza
 
%%url https://proza.ru/2016/10/18/750?fbclid=IwAR1PvvhZrUxqpiK2qbA5oOR0BK_VB4I-HBlbm9UXgNeCSlhzKiOgY8CgPB4
 
%%author 
%%author_id 
%%author_url 
 
%%tags 
%%title 
 
%%endhead 

\section{Ушёл к Мотороле}
\label{sec:proza.rus.lavrova_elena.1.ushel_k_motorole}

\Purl{https://proza.ru/2016/10/18/750?fbclid=IwAR1PvvhZrUxqpiK2qbA5oOR0BK_VB4I-HBlbm9UXgNeCSlhzKiOgY8CgPB4}
\index[writers.rus]{Лаврова, Елена!Ушёл к Мотороле}

Василий Петрович вышел из дому, постоял на крылечке, с наслаждением вдыхая
тёплый сентябрьский воздух, и раздумывал: покурить сейчас или погодить и
покурить после работы. Пролетела, каркая, чёрная крупная ворона. Василий
Петрович проводил её взглядом. Ишь, раздобрела, подумал он. Нынче много вам
пищи по посадкам. Потом покурю, решил он. Сначала работу сделаю. Надо было
сколотить дополнительные полки в подвале дома, и приделать несколько крючков
для одежды. Надо было хорошо обустроить подвал к зиме.  Там было безопасно, но
это при условии, что не накроет прямой наводкой крупным снарядом. В таком
случае, дом бы разрушился, мог заняться пожар, и было бы трудно выбраться
наружу: вход в подвал был в передней. Василий Петрович, предвидя, что война
может затянуться надолго, запланировал сделать второй, запасной выход из
подвала. Каждый день Василий Петрович помаленьку копал этот самый запасной
выход, таскал вёдра вынутой земли в огород, где Галина Ивановна разравнивала
эту землю граблями.

Подвал теперь был их спасением, и его следовало сделать уютным --- по мере сил,
конечно. Ещё в начале войны, когда стало прилетать с украинской стороны,
Василий Петрович сколотил в подвале топчаны из горбыля. Жена, Галина Ивановна,
принесла с чердака старые пуховые перины, подушки  и ватные одеяла и постелила
на топчаны, радуясь, что в своё время не дала детям сжечь «это старьё», как они
выражались, на костре. Не дам, сказала им тогда Галина Ивановна и перетащила
всё это добро, доставшееся ей от бабушки, на чердак. Она оставила без внимания
смех и издёвки детей, мол, не современная ты, мама, теперь это не в моде,
теперь нейлон, лавсан и что там ещё --- химическое, а ты со своими перьями и
пухом возишься. К тому же бабушка умерла на одной из этих перин, а теперь, чёрт
знает, на которой из них. И подушки тоже! На которой из них лежала её мёртвая
голова, поди, знай. Неприятно же!

- Пусть будут! --- коротко сказала Галина Ивановна и спрятала подальше ключ на чердак, чтобы дети в её отсутствие не привели свои угрозы в исполнение и не сожгли пух и перья в угоду лавсану и нейлону.

И вот теперь, через много лет, всё это добро пригодилось. В подвале было
прохладно в летнюю жару и сыровато, так что пух и перья замечательно грели,
потому как буржуйка, сваренная из железяк Василием Петровичем, к утру остывала.
Железяки Василий Петрович тоже не позволял сыну выбросить. Пусть лежат, говорил
он. Места не пролежат! И ведь не пролежали. Пригодились!

В подвале были полки, но все они были заняты банками с солениями и компотами.
Дополнительные полки нужны были для всяких мелочей, всякой всячины, которую
приходилось класть прямо на топчаны. Это было неудобно. Книгу некуда положить,
жаловалась Галина Ивановна. У Василия Петровича вертелось на языке, так, пусть
Миша возьмёт и соорудит полочку, но он языку воли не давал, потому что сын был
занят весь день и приходил по вечерам поздно. А то и вовсе не приходил. Поэтому
Василий Петрович и решил сделать дополнительные полки, потому что они и в самом
деле были необходимы. Чем занимался Михаил последние несколько месяцев, Василий
Петрович не знал. И Галина Ивановна не знала. Сын приходил домой сильно
уставшим, поспешно ел, и падал в постель. И Василий Петрович приметил, но
никому не говорил, что сын стал носить странную обувь. Название ботинок
походило на «бутсы», но это были не бутсы, хотя тоже на «б». Василий Петрович
всё не мог запомнить название этих ботинок, пока не догадался сравнить их
название с названием кости --- «берцовая». Ботинки были крепкие, высокие, с
толстой рифлёной подошвой. Василий Петрович был бы рад носить такие
замечательные ботинки. Наверное, на несколько сезонов хватит. А если аккуратно
носить, то лет на десять. Почитай, до конца жизни. Ещё десять лет Василий
Петрович рассчитывал прожить. Он спросил сына, где он их взял и сколько за них
заплатил. Сын отвечал, что достанет ему такие ботинки бесплатно, но позже. Кое
о чём Василий Петрович догадывался, но догадки свои держал при себе. Все
молчали. И он молчал. И ни о чём никого не спрашивал. Надо будет --- скажут.

Василий Петрович поискал глазами жену. Во дворе её не было. Значит, она на
огороде за домом. Василий Петрович спустился с крыльца, обошёл дом и увидел
жену. Она и в самом деле была на огороде, и сгребала в кучки ботву. Сегодня
выдался день затишья, и можно было устроить уборку.  Василий Петрович смотрел
на жену, а она его не видела. Василий Петрович смотрел и гордился, какая Галя у
него статная, даже в шестьдесят два года. Он же был на десять лет старше.
Работа у неё в руках спорилась. И крепко и уверенно стояли на земле её ноги.
Четверых родила, а не раздобрела, как некоторые бабы. Соблюдает себя. Не
сдаётся.

Василий Петрович вздохнул, и отправился в сарайчик на краю огорода мастерить
полки. За ним трусил старый спаниэль Тимоша, спутник по рыбалке и охоте.
Проходя мимо жены, Василий Петрович хотел сказать ей, какая она у него ладная
да ловкая, но не сказал. Стругая доски на верстаке в сарайчике, прочищая иногда
рубанок от стружки, Василий Петрович думал о том, что всё так хорошо сложилось
в его жизни. Сорок пять лет назад, отслужив в армии, приехал он на Донбасс из
посёлка Знаменка Орловской области в поисках работы. Из Донбасса родом был его
армейский друг Толян. Он-то и посоветовал Василию Петровичу, что лучше всего
приехать на Донбасс и стать шахтёром. И зарплата хорошая и уважение. И девчата
красивые в Горловке. Василий Иванович собрался и приехал. Он думал, что
Горловка это поселок вроде его родной Знаменки на Орловщине, а оказалось, что
это довольно-таки большой город. И Василий Петрович стал шахтёром. А вскоре
увидел он свою Галю. А увидел он её в парикмахерской, где она работала. Пришёл
подстричься, а ушёл с сердцем, полным радости. Начали с Галей встречаться.
Василий Петрович высокий, русоволосый, с голубыми глазами. Руки, как лопаты. А
Галя черноволосая, чернобровая, кареглазая дивчина. Василий Петрович ласково
звал жену: моя хохлушечка. А его хохлушечка по-украински ни бум-бум. Родители
её ещё кое-как могли объясняться на этом языке, но всегда говорили на русском,
потому, что на Донбассе все говорили на русском, и только песни в праздники
пели не только русские, но и украинские, и непременно на украинском языке. Даже
Василий Петрович со временем научился петь их на украинском.

Галина Иванова, слыша от мужа ласковое «хохлушечка», в пику называла его «москалём». Василий Петрович улыбался в свои сталинские усы и говорил, что от этого слова веет родиной, Москвой.

Василий Петрович закончил выстругивать первую доску и подумал, что теперь может
позволить себе маленький перекур. Он вышел из сарайчика, сел на лавочку,
сколоченную им же лет тридцать назад, и закурил. Пуская сизый дымок, он следил
за женой, как она сгребает маленькие кучки ботвы в одну общую кучу. Издалека
донеслось кряканье уток. Много их развелось на местном ставке. Надо бы пойти,
пострелять завтра, подумал Василий Петрович. Будет хороший обед, а то всё
курятина, да курятина. Курочки у них были свои. Больше никакой живности не
было, если не считать Тимошу. Но Тимоша не был живностью. Он был почти человек.

Василий Петрович нередко беседовал с ним, а пёс слушал и кивал головой.
Понимал. Поддакивал.

Были у Василия Петровича и Галины Ивановны дети: три сына и дочь. Дочь Ирина
выучилась на врача, вышла замуж и жила в Ростове.  Старший сын Иван был
электриком и жил со своей семьёй в Краснодаре.

Средний сын Андрей после школы пошёл, как отец, в шахту, но в шахте ему не
понравилось. Через два месяца он уволился и стал водителем фургона при каком-то
складе. Он всё ещё не был женат и жил с отцом и матерью в их просторном доме.

Младший сын Михаил был шахтёр, как и отец, не женат и тоже жил в отцовском
доме. Конечно, хорошо было бы, если бы все дети жили рядом, но, что поделаешь!
Так получилось, что старшие живут отдельно и надо смириться. Детьми Василий
Петрович и Галина Ивановна были довольны, гордились ими, хотя немного
расстраивал Андрей, гуляка и большой любитель пива. Андрей словно родился с
бутылкой пива в руке.


- Тебе нельзя, - строго говорил ему отец. --- Ты же водитель!

Андрей смеялся: - Я не людей развожу по магазинам, а овощи да селедку.

Василий Петрович закончил курить и побрёл строгать вторую доску --- для боковин
полки. Дочь его внешне была похожа на него, на отца: светленькая с голубыми
глазами. Андрей --- тоже. А вот Иван и Михаил были в мать: чернобровую,
черноокую, черноволосую хохотушку-хохлушку Галю. Россия и Украина перемешалась
в их семье прочно - не растащить.  Но перемешались они всё-таки на русской
основе, думал Василий Петрович, принимаясь за вторую доску. Впрочем, с Андреем
вышла осечка. Непонятная, странная осечка.

Жизнь Андрей вёл скрытную. Никогда ни о чём о своей жизни не рассказывал. С кем
дружил? Кого любил? О чём думал? Если друзья и были, то в дом их Андрей не
приводил. Если девушка у него была, то где он с ней встречался, родители не
знали.

Из деликатности не спрашивали, когда женится.  Надеялись, что всё будет путём.
Но всё пошло наперекосяк в конце 2013 года. То есть, для них, родителей, вышло
наперекосяк, а для Андрея, может быть; всё шло, как надо. Но сначала что-то
пошло не так в Киеве.

Василий Петрович следил за событиями «по ящику» и переживал, что Янукович хочет
подписать документ об ассоциации Украины с ЕС.

- С Россией надо бумажки подписывать! С Россией-матушкой, а не с европами
этими. Чужие они нам! А матушка --- не выдаст. Ну, куда этот шапошник лезет!

Шапошником Василий Петрович звал Януковича за старые грехи его.

Когда стало известно, что Президент отложил подписание документа, Василий Петрович радовался:

- Понял, дурак, что не надо подписывать. Так и страну недолго загубить, что попало подписывая.

А 21 ноября начался майдан.

- Ничего! --- рассуждал Василий Петрович. --- Посидят-посидят, жрать захочется --- уйдут.

Но протестующие не только не уходили, но их становилось всё больше и больше. И
жрать им кто-то давал, и пить давал, и, наконец, они зажгли первые шины. Смрад
горящих покрышек накрыл улицы Киева.

- Чего же Янукович их не разгонит? --- недоумевал Василий Петрович. --- Вот, Путин Болотную разогнал, и все дела! Тихо и порядок! Чего же он ждёт? Шото в последнее время всё Бога поминал к месту и не к месту. На Бога надейся, а меры принимать надо.

В начале декабря пришёл Андрей с работы, весёлый, и непривычно разговорчивый. За ужином внезапно объявил:

- Я уволился.

Все умолкли и смотрели на него, ожидая продолжения. Продолжение всех ошеломило:

- Мы с друзьями уезжаем в Киев!
- Зачем? --- спросил Василий Петрович, но он уже понял --- зачем.
- Жизнь будем менять! Надоели воры у власти. Мы к власти придём.
- Кто --- мы? --- испуганно спросила Галина Ивановна.
- Мы, молодые!
- Какие, такие молодые! --- не выдержал Василий Петрович. --- Тебе уже тридцать пять стукнуло.
- Это ещё молодость, - снисходительно улыбался Андрей. --- Всех из власти повыгоним и молодых посадим. Прогрессивных. В ЕС войдём. Не жизнь будет, а малина!
- Как бы той малины не объесться, - съехидничал Василий Петрович. --- Так вот, та Европа сидит и ждёт вас. А ты читал, во что та ЕС некоторые восточные страны превратила? Ну, Болгарию, например. Читал?
- Это всё пропаганда! --- заявил Андрей. --- Меньше свои совковые передачи слушай! Ты меня слушай! Вступим в ЕС, знаешь, какие зарплаты и пенсии будут?! В евро! А режим - безвизовый. Берёшь маман под ручку и катишься в Германию. Или там --- в Италию.
- Зачем мне катиться в Германию или в Италию, - начал закипать Василий Петрович. --- Мне и здесь хорошо! А начальник нашей шахты чуть не каждый месяц за границу ездит. Оформил визу, и едет. Какая разница!
- Тебе хорошо, - начал тоже закипать Андрей, - а мне - здесь --- плохо! Я хочу, как человек, пожить. Понимаешь, как человек!
- А чего тебе дома-то не хватает? --- встряла Галина Ивановна в мужской разговор. --- Работа --- есть! Дом --- есть! Мебель --- есть. Если тебе машину хочется, то мы с отцом накопим, и будет тебе машина.
И тут встрял Михаил, до того молчавший:
- Так ведь ему не «Фольксваген» подержанный хочется, так ведь? Ему что покруче подавай, да новенькое! Там, в Европе, между прочим, своих таксистов хватает. Унитазы захотелось европейские чистить? Или санитаром в хосписе горшки выносить?
Андрей не сдавался:
- Мир хочу посмотреть.
- Ага, - подначивал Михаил. --- Много ты мира увидишь в чужом унитазе.
Андрей встал и с шумом отодвинул стул:
- А что хотите, то и говорите! А я --- еду! --- Прозябайте тут, если нравится!

Наутро Андрей укатил в Киев.

Василий Петрович ходил мрачный и всё спрашивал себя, чего это Андрею, паршивцу, не хватало? Живи, да радуйся! И не находил ответа.
А майдан в Киеве всё длился и длился. И когда пролилась первая кровь, Василий Петрович не мог без боли сердечной смотреть на жену: так она переживала. А поделать ничего нельзя было. Андрей не давал о себе знать: не звонил. Где он был? С кем? Что ел? Что пил? Где спал? Был ли жив? По вечерам, включив телевизор, Василий Петрович внимательно слушал новости из Киева и вглядывался в мелькавшие на экране картинки с майдана: нет ли там сына. Но сына там не было.
23 февраля   узнали, что Президент Янукович бежал из страны.

- Всё! --- сказал Василий Петрович Тимоше. --- Просрали страну!

Тимоша огорчённо постучал хвостом по полу.

В середине марта пришла неожиданная весть: Путин подписал указ о присоединении Крыма к Российской Федерации.
Василий Петрович, слушая по телевидению новости, качал головой в недоумении и приговаривал:
- Ай, да Путин! Ай, молодца! Р-р-аз, и готово! За это дело, мать, надо бы выпить.
Вечером за ужином, пропустив рюмку водки за Путина, рюмку за воссоединение Крыма с Россией, Василий Петрович внезапно спросил жену:
- А ты --- не против? Может, ты --- против? Может, тебе обидно, ты же украинка.
- Да, какая я украинка! --- махнула рукой Галина Ивановна и засмеялась. --- Скажешь, тоже! У меня только паспорт - украинского-то, как и в тебе!
И вместе они снова выпили за Путина.
В конце апреля 2014 года Михаил принёс Василию Петровичу гостинец. Василий Петрович и Галина Ивановна обедали. Михаил выпростал гостинец из полиэтиленового мешка и поставил его перед отцом на стол.
- Мне? --- спросил Василий Петрович, вытирая полотенцем руки.
- Тебе, - отвечал Михаил. --- Тебе же их хотелось. Ну, вот!
- Где взял? --- спросил Василий Петрович, лаская корявыми пальцами берцы.
- Там уже больше нет, - лаконично ответил сын. --- Ты примерь.
Василий Петрович примерил. Берцы были чуть-чуть великоваты. С шерстяным носком хорошо будет, подумал Василий Петрович. Крепкая обувь! Аккуратно носить, так до смерти хватит.
- А себе? --- спросил Василий Петрович сына.
- У меня есть, - ответил сын и мигнул отцу, чтобы тот вышел на крылечко покурить.
Сели покурить. Василий Петрович прихватил с собой берцы и поставил рядом на крылечке. Любовался, пока курил. Сын молчал, но Василий Петрович чувствовал, что он хочет что-то сказать. Василий Петрович ждал.
- Ты, это, батя, - начал сын, закуривая вторую сигарету, -  матери скажи, что я уехал в командировку. Далеко. Чтобы не волновалась. Я в дом заходить не буду. Просто, я уйду сейчас. То есть, за мной сейчас заедут. Не прощаясь. Сам понимаешь, слёзы и всё такое … А ты ей потом скажешь, хорошо? Будто сам не знал, а потом узнал.
Василий Петрович, молча, кивнул головой. Докурил сигарету, поднялся:
- Ты погоди-ко. Я щас!
Василий Петрович ушёл в дом, а через несколько минут вернулся со свёртком. Отдал сыну.
- Шо там?
- Посмотри.
Михаил развернул полотно. В свёртке лежала армейская фляга времён Отечественной.
- Деда?
- Деда! Ты это, не потеряй в командировке-то. Домой принеси!
- Так точно! --- тихо сказал сын и поднялся. Обнялись. К воротам тихо подкатила обшарпанная голубая семёрка.
Михаил быстро пошёл к воротам и Василий Петрович смотрел ему вслед. И, хотя верующим он не был, он быстро три раза перекрестил удаляющегося сына. Хлопнула дверца семёрки.
Василий Петрович переобулся в берцы, прямо на крылечке и сидел, вытянув ноги, чтобы как следует их рассмотреть. На крылечко вышла Галина Ивановна, вытирая руки фартуком.
- А Миша, где?
- Так это, - начал Василий Иванович, продолжая рассматривать берцы, - он тебе шо, не рассказал? В командировку он уехал. Длительную. Приехали за ним, он и попрощаться не успел. Побежал, крикнул, чтобы ты не волновалась.
_ Понятно! --- сказала Галина Ивановна и села на ступеньку рядом с мужем. --- Так я и знала!
Она утёрла фартуком глаза.
- Шо ты знала? --- притворялся Василий Иванович, полируя берцы носовым платком. Галина Ивановна вынула его из пальцев мужа.
- Ну, не обувь же им вытирать! Стирать-то, мне! --- упрекнула она. --- А то и знала, что в Славянск он поедет! И про того ирода заранее знала, что в Киев поедет. Разные они.
Под «иродом» Галина Ивановна имела в виду Андрея.
Василий Иванович покосился на жену,
- Прям-таки, знала! Я вот не знал.
- Ты не знал, а я догадывалась. Только не по-людски это, втихаря уезжать. Надо было посидеть, Проводить. Хоть бы живыми вернулись!
- Лучше встретим. Тогда посидим, --- подвёл итог Василий Петрович.
Больше они о сыновьях не разговаривали. Вместе ждали вестей. А вести приходили не часто. Чаще узнавали о новостях по радио и по телевидению. А вести были тревожные. И чем ближе было лето, тем тревожнее они становились. Василий Петрович сделался мрачен и молчалив, а Галина Ивановна всё чаще пила валерьянку и корвалол.
Василий Петрович пристрастился ходить на митинги. Ему нравилось находиться в гуще народа, нравилось слушать перед митингом торжественное «Вставай, страна огромная …», нравились охрипшие ораторы-женщины и веяние русских триколоров над головой. Ему тогда, в марте-апреле, как и многим другим жителям Донбасса, казалось, что всё будет хорошо, несмотря ни на что. Ему казалось, что русские войска вот-вот придут на помощь, как в Крыму, произойдёт воссоединение Донбасса с Россией, и всё будет прекрасно!
Но что-то пошло не так. 3-го мая все узнали о трагедии в Одессе, случившейся накануне.
Галина Ивановна плакала и всё повторяла: - Ты ходишь на митинги! А что, если и вас сожгут заживо!
- В Горловке нет нациков, - убеждал Василий Петрович. --- В Горловке такое не произойдёт.
Но Галина Ивановна плакала и просила его на митинги не ходить. Боялась за него. Но Василий Петрович всё равно ходил и чувствовал себя причастным историческим событиям, совершающимся на его глазах.
А потом подлетело 11 мая. Утро выдалось солнечное, ясное, тёплое. Листья на деревьях развернулись в полную силу и цвели каштаны.
Василий Петрович и Галина Ивановна нарядились, как на праздник. Василий Петрович надел серый костюм, который он надевал вот уже двадцать лет только по большим праздникам, и поэтому костюм выглядел, как будто его вчера купили в магазине. К костюму он надел белую рубашку. И, конечно, на ногах Василия Петрович красовались берцы, хотя, правду сказать, было в них немного жарковато. Но, надевая берцы, Василий Петрович чувствовал себя немножко причастным ополченцам, и это было ему приятно.
Галина Ивановна надела новое весеннее платье, синее в белый горошек, с белым кружевным воротником и синие туфельки. На её правом запястье красовался золотой браслет, подаренный ко дню рождения старшим сыном Иваном. А на левом запястье скромно пристроился магнитный браслет, который Галина Ивановна не снимала ни днём, ни ночью вот уже двадцать лет, веря в его целебную силу.
В кармане пиджака Василия Петровича лежали два паспорта, его и жены. А нарядились муж и жена, как на праздник, потому что 11 мая на Донбассе проводился референдум и каждого жителя спросили, хочет ли он независимости от Украины? А потом обещали сразу же второй референдум, на котором каждого жителя должны были спросить: хочет ли он, чтобы Донбасс присоединился к России?
Правда, второй референдум зажилили, и потом о нём никто не вспоминал. А почему зажилили, Василий Петрович не знал и строил потом разные предположения. И сильно обижался на новую донецкую власть, что второго референдума она не провела. И немножко сердился на Путина, потому что в одном из его предположений, Путин тайно просил донецкую власть второй референдум не проводить. Вот, и не провели.
Василий Петрович и Галина Ивановна хотели, чтобы Донбасс стал независимым от Украины. И ещё больше они хотели, чтобы Донбасс присоединился к России. Им было невыносимо думать, что в Одессе так деловито были заживо сожжены люди, ходившие на митинги за федерализацию Украины, и многие украинцы и правительство это массовое убийство одобрили. В ТАКОЙ Украине, где заживо сжигают граждан за убеждения, они жить не хотели.
Пока Василий Петрович с супругой шли на избирательный участок, нарядные, в приподнятом настроении, им навстречу шли толпы уже проголосовавших людей. И эти люди, как и Василий Петрович с супругой, были нарядны, и улыбались.
- Как хорошо! --- сказала Галина Ивановна. --- Впервые в жизни иду на избирательный участок с радостью. Одно только заботит: почему Путин просил референдум отложить?
- Кто его знает! - отвечал Василий Петрович. --- Если просил, значит, были причины. Только после Одессы не стоит ждать. Если будем ждать, то и референдума не видать, и нацики здесь появятся, как пить дать. Нет, нельзя откладывать. Думаю, Путин всех здешних тонкостей не ведает. Что советники доложат, то для него - правда. А советники тоже, всё ли знают?
- Смотри, смотри! --- зашептала Галина Ивановна. --- Это же ополченцы! У входа в школу, где был размещён избирательный участок, стояли четверо крепких рослых ребят в камуфляжной форме с автоматами. Оружие ребята держали, как младенцев, на обеих руках, прижав к груди.
Проходя мимо ребят, Василий Петрович уважительно поклонился и приветственно приподнял над седой головой соломенную шляпу. Парни вежливо улыбнулись в ответ, но глаза их оставались серьёзны и внимательны.
На избирательном участке в местной школе толпился народ, к столам  и урнам стояли длинные очереди, играла музыка, да не просто, абы какая, а патриотические песни военных лет, что взбадривало избирателей также же, как сознание, что сейчас они выбирают судьбу своего края --- Донбасса. Повсюду  слышались возбуждённые голоса, у людей сияли глаза, и никто ни о чём не спорил --- такое было единодушие.
- Праздник! Чисто, праздник! --- удовлетворённо повторял Василий Петрович и ласково сжимал руку супруги.
Когда через пару дней они узнали о результатах голосования, Василий Петрович сказал:
- Ну, мать, новая жизнь начинается, без этой хунтовой Украины. Сами себе хозяева теперь. Только всё-таки, где второй референдум?

В десятых числах июля пришёл Михаил. Он уже не скрывал, где был. Пришёл в военной полевой форме.
- Что, не отстояли? --- спросил Василий Петрович за обедом. --- Как же так? Почему? Они же теперь на Горловку и Донецк попрут.
- Так получилось, - коротко отвечал Михаил. --- К сожалению, так получилось. Попрут!
- Ой, божечки! --- запричитала Галина Ивановна. --- Да что же это такое творится?!
- Батя, мама, лучше бы вам в Краснодар к Ивану или к Ирине в Ростов? Скоро здесь будет жарко.
Василий Петрович положил ложку и, постукивая пальцем в столешницу, отчеканил:
- Никуда я не поеду! Последнее дело --- тикать, портки роняя. Здесь - мой дом. Здесь - моя земля. Здесь всё --- моё! Я сказал! А мать сама за себя скажет. Может, Галя, тебе и правда уехать? Ты --- женщина. Какой с тебя спрос!
Галина Ивановна глядела с упрёком на мужа:
- Чего жеж я из своего дому и от своего мужа тикать буду? Выдумаешь, тоже! Вместе беду переживём.
- Мне было бы спокойнее … - начал, было, Михаил, но Василий Петрович оборвал его.
- Может, тебе было бы спокойнее, только, зная, что мы дома, ты злее будешь. Я не знаю, почему вы ушли из Славянска. На войне всякое бывает. И отступления бывают. Вспомни ту войну. И дед отступал. Но потом --- стали наступать и победили. Вот так и должно быть! Наступать надо и победить! Всё!
Вышли на крылечко покурить. Вечерело, но воздух был всё ещё раскалён.
- От Андрея вести есть? --- спросил Михаил.
- Нет.
- От Андрея вести --- есть, - сказал Михаил и вынул из кармана свой смартфон. Щас, покажу. Только --- спокойно.
Василий Петрович внутренне напрягся. Михаил нашёл нужную фотографию и дал телефон отцу. Сначала Василий Петрович не понял и даже обрадовался. На фотографии был живой, здоровый и улыбчивый Андрей.
- О, чертяка! --- вымолвил Василий Петрович. --- Похудел, что ли? А что это за форма на нём?
Михаил постучал пальцем по монитору, изображение увеличилось, и Василий Петрович увидел детали: красно-чёрный шеврон на рукаве с надписью «АЗОВ». В руках у Андрея был автомат.
- Бля-я-я-я! --- выдохнул Василий Петрович, разглядывая чёрного паука свастики на жёлто-голубом фоне. --- Твою мать! Вобля-я-я-я-я! Да это ж вервольф! У партии молодого Гитлера была такая штука! Оборотень! Мой сын! Это ж кого я родил?!
- Матери показать?
- Спрячь! Спрячь и не показывай! Пусть не знает об этом позорище.  Пусть не знает!
Михаил спрятал телефон.
- Так получается, вы, братья, по разные стороны?
- Получается, батя.
- Так, получается, это --- гражданская?
- Получается, гражданская.
- Те, значит, за Европу, а мы --- за Россию, - подвёл итог Василий Петрович. --- А шо та Европа житья не давала России, то шведы, то поляки, то французы, то немцы, это, значит, им наплевать?
- Значит, наплевать!
- Ну, дела-а-а-а!
- А ну, как он домой заявится?! Шо нам с ним делать?!
- Не заявится, батя. Его здесь в плен возьмут. Исключено!
- Так, что же он теперь, пропащая душа?
Михаил не ответил, но по его молчанию Василий Петрович понял, что дело зашло так далеко, что ничего уже не вернёшь. Назад пути нет.
- Ну, и что теперь? --- спросил Василий Петрович.
- Иду к Мотороле, - улыбнулся Михаил.
Брови Василия Петровича поползли вверх от удивления.
- К командиру Павлову, - пояснил Михаил. --- Моторола --- его позывной. У меня тоже позывной есть. Будем оборонять Горловку. Моторола - достойный мужик! Твёрдый! С принципами и ценностями, как теперь выражаются.
- А твой позывной, какой?
Михаил наклонился, потрепал Тимошу, лежащего на ступеньке, по голове, и засмеялся:
- Тимоха мой позывной.
Василий Петрович усмехнулся.
- А чего жеж, как у пса?
- О доме напоминает.
Василий Петрович кивнул.
- Как услышишь «Тимоха», знай, это я.
- А звание у тебя, какое?
- Самое высокое, батя --- рядовой.
- А род войск, - продолжал любопытствовать Василий Петрович.
- Пулемётчик. Щас крупнокалиберный на полигоне осваиваю. Здорово наяривает! Знатная штука! Готовимся принимать «гостей».
«Гости» не заставили себя ждать.
Утро  27 июля выдалось жарким и солнечным. Василий Петрович и Галина Ивановна занимались хозяйственными делами. Покончив с ними, они собрались было отобедать, но в 13.00 началось. Над городом загремела канонада. Такое Василий Петрович и Галина Ивановна слышали только в фильмах про Великую Отечественную войну. Ужас объял их. Это был первый в их жизни артиллерийский обстрел города.
Василий Петрович сидел в зале. Услышав орудийный гром, он страшно закричал жене, находящейся на кухне: - В подвал! Немедленно в подвал!
Галина Ивановна так испугалась и ослабела от страха, что плюхнулась на табуретку и не могла двинуться с места. Василий Иванович вбежал на кухню, подставил жене плечо, обхватил рукой за талию и потащил вниз по лестнице в подвал. В подвале он зажёг свет и посадил жену на топчан. Галина Ивановна прижала ладони к щекам, качалась из стороны в сторону и повторяла одну и ту же фразу: - Что же это такое?! Что же это такое?! Что же это такое?!
Мощные удары раздавались всё ближе и ближе. Дом вздрагивал и начал мигать свет. Василий Иванович положил поближе фонарь.
- Успокойся, - сказал он жене. --- Здесь не опасно.  Всё хорошо!
Но он знал, что совсем всё не так хорошо. Он боялся прямого попадания в дом. Если снаряд попадёт в крышу, дом не выдержит. Он старый. И кто будет доставать их с женой из-под завалов? Соседи ведь в таком же положении. Именно тогда, Василий Иванович задумал копать второй выход из подвала наружу в огород.
Тот первый обстрел города унёс жизни примерно двух десятков человек, и множество людей были ранены. Украинские орудия били по улицам города среди бела дня и люди были застигнуты врасплох. Некоторые из тех, кто пережил этот ужас, признавались, что даже не подумали упасть на землю при первых выстрелах. Не знали, что нужно падать. Среди погибших в тот день горожан была та самая молодая мать с десятимесячной дочкой на руках, которую впоследствии назвали Горловской Мадонной.
Каждый день были обстрелы. Каждый день звонил старший сын Иван, ругался и требовал, чтобы родители немедленно приехали к нему в Краснодар. Звонила Ирина, тоже ругалась и требовала, чтобы родители немедленно приехали к ней в Ростов. Василий Петрович научился различать выстрелы гаубиц и Градов, миномётов, танков  и САУ. А когда стучал крупнокалиберный пулемёт, Василий Петрович разглаживал указательным пальцем свои сталинские усы и шептал: - Давай, сынок, давай! Может, это был не пулемёт его сына, но ему хотелось думать, что это непременно его пулемёт.
Галина Ивановна тоже помаленьку привыкла к ежедневным обстрелам. Постепенно так привыкли, что днём не всегда и прятались в подвал. Но спали этим летом только в подвале, чтобы ночью не быть застигнутыми врасплох. Привыкнуть к обстрелам не мог только Тимоша. Заслышав свист снарядов, он стремглав мчался в подвал, забивался под топчан и дрожал всем телом. А после обстрела долго не вылезал из-под топчана.
- Тимоха, ты же охотник. Ты же ружейных выстрелов не боишься, - шутливо упрекал его хозяин. Тимоша виновато стучал хвостом по полу, мол, это, дорогой друг, не ружейные выстрелы, мол, понимаем-с разницу.
Василий Петрович привинтил саморезами боковины, вбил сверху гвозди и загнул их, чтобы на них подвесить полку. Полюбовался, и решил спуститься в подвал и примерить, где её подвесить. Подвешу, думал он, шагая к дому, и пару-другую вёдер земли выну. Тимоша трусил рядом. Надо побыстрее сделать второй выход из подвала. Галина Ивановна уже сгребла ботву и стояла, опершись на грабли --- отдыхала. Проходя мимо, Василий Петрович показал ей полку.
- Нормально? --- спросил он.
- Нормально, - ответила жена. --- Щас обедать будем.
Василий Петрович спустился в подвал и начал прилаживать полку. Измерил расстояние между загнутыми гвоздями, начертил крестики на стене подвала и взял дрель. Сверло легко вошло в кирпич. Василий Петрович вбил молотком деревянный чопик, а в чопик вогнал гвоздь. Вдруг заскулил Тимоша.
- Ты шо? --- спросил Василий Петрович и в это время раздался страшный удар. Дом содрогнулся и в подвале погас свет. Василий Петрович бросил дрель и в темноте лихорадочно шарил по столу, чтобы найти фонарь. Он точно помнил, что оставил фонарь лежать на столе. В это время раздался второй удар. Василий Иванович пошатнулся, зацепил рукавом фонарь и уронил его на пол. Но искать фонарь он не стал, и ощупью пошёл к выходу. По лестнице он взлетел, сам не помня как, и помчался через двор на огород. Там, где десять минут назад стояла Галина Ивановна, зияла воронка, и от воронки поднимался к небу лёгкий сизый дымок.
Нет, сказал Василий Петрович. Нет! Она, наверное, в доме! Да, она в доме, раз её не было в подвале. И Василий Петрович помчался обратно в дом. Он вбежал в переднюю, потом на кухню, в комнаты, но Галины Ивановны нигде не было. Василий Петрович бросился к входу в подвал и крикнул вниз --- в темноту: - Галя! Галочка!
Всё молчало. Весь дом молчал. На улице тоже было тихо. Буханье слышалось теперь вдали. Василий Петрович снова побежал в огород и остановился у калитки, оглядывая пустой огород во всех направлениях. Сизый дымок над воронкой исчез. И Галина Ивановна исчезла бесследно. Василий Петрович побрёл к воронке, ещё не понимая, вернее, ещё не желая понимать, что случилось страшное. Метрах в пяти от воронки он увидел … да, он увидел браслет. Магнитный браслет, который Галина Ивановна носила вот уже много-много лет от давления. Помогал этот браслет или нет, но Галина Ивановна верила, что помогает. Василий Петрович не раз подшучивал над верой жены в чудодейственную силу браслета. Но от его шуток вера жены не поколебалась. И вот этот браслет лежал среди комьев сырой земли и Василий Петрович поднял его. Это было всё, что осталось от Галины Ивановны, его жены. Василий Петрович стал бродить вокруг воронки, сужая круги, но больше ничего не нашёл. Он держал браслет перед собой, как некий талисман, при помощи которого ему удастся найти жену. Но жены не было. И Василий Петрович остановился у края воронки и стал смотреть в её глубину. Но и там он ничего не увидел, кроме комьев земли. И Василий Петрович сел на корточки у воронки и замер, держа магнитный браслет в вытянутой руке на ладони.
Между тем, около Василия Петровича потихоньку стали собираться соседи. Они сразу всё поняли и не отрывали взглядов от браслета на ладони Василия Петровича и от дыры в земле, навеки поглотившую Галину Ивановну. Женщины тихо плакали. Мужчины молчали. Через некоторое время соседи по очереди стали подходить к Василию Петровичу и каждый говорил ему что-нибудь, что могло бы его, по их мнению, утешить. В ответ на их корявые, но искренние фразы Василий Петрович слегка поднимал и снова опускал руку с магнитным браслетом.
Наконец, соседи ушли, и Василий Петрович остался один. Он сидел у воронки до сумерек. И когда пришла ночь, и засветились первые звёзды, Василий Петрович встал, положил магнитный браслет в нагрудный карман рубашки и побрёл к дому. Он лёг на кровать, не раздеваясь, и долго лежал, глядя в темноту. Ему казалось, что он засыпан в шахте, и выхода нет. Ему было трудно дышать, словно на его груди лежала тяжёлая, сырая земля.
Здесь, рядом с ним сейчас должна была лежать его Галя, с которой они прожили бок о бок, несколько десятков лет. Василий Петрович протянул руку, но рядом никого не было. Это было так странно, так несправедливо, так ужасно, что не поддавалось уразумению, и он всё ещё не верил в случившееся. Наверное, надо было настоять, чтобы Галя уехала к старшему сыну или к дочери. Там она была бы в безопасности и жива. Василий Петрович терзался этой мыслью всю ночь. Пришёл Тимоша, запрыгнул на кровать и прижался к боку хозяина.
Так лежали они долго, пока не начал светлеть проём окна. Вдруг Василий Петрович резко встал и стремительно вышел из дома. Тимоша бежал за ним. Василий Петрович шёл к воронке, словно надеялся, что никакой воронки нет, и сейчас он увидит Галю. Но воронка была, а Гали --- не было. И тогда Василий Петрович понял, наконец, что всё это правда, реальность, от которой никуда не денешься и не убежишь. И Василий Петрович в отчаянии завыл в голос, как старый пёс, и вслед за ним завыл Тимоша. Этот вой разбудил соседей. Соседка Татьяна, испуганно вцепившись в руку мужа, спросила:
- Шо это?! Шо это такое?! Кто это?!
Муж Татьяны, Павел Игнатьевич, встал с постели, выглянул в окно, и ответил:
- Должно быть, собака Петровича по хозяйке воет.
- Нет, - встревожено отвечала Татьяна, - это не собака. Это человек!
Павел Игнатьевич закурил.
- Надо пойти к нему. Как бы он чего не сотворил над собой, - сказала Татьяна и спустила полные ноги с кровати, нашаривая тапочки.
- Лежи! --- приказал Павел Игнатьевич. --- Здесь ничем не поможешь. Если он до утра доживёт, то ничего. Его отпустит.
- А если не отпустит? - настаивала Татьяна.
- Лежи! Не мешай человеку переживать. За него ты переживать не сможешь. И я не смогу. И никто не сможет.

Когда на востоке показался золотой краешек молодого солнца, Василий Петрович был уже в хлопотах. Он нашёл в чулане старый рюкзак, положил в него брезентовую куртку, которую всегда надевал на рыбалку, отцовский котелок, прошедший с отцом Василия Петровича всю войну, хохломскую обгрызенную по краям ложку, помятую временем алюминиевую кружку, добрый запас дроби, кусок сала в холстинке и начатую буханку хлеба. Потом он перешнуровал берцы, надел рюкзак, снял  в прихожей с гвоздя дробовик, повесил его на левое плечо и свистнул Тимоше.
Василий Петрович не запер дверь дома, и оставил её приоткрытой, чтобы тот, кто придёт, свободно вошёл бы в дом.
И они пошли с Тимошей сначала по улице, потом вышли на шоссе, и двинулись вдоль посадки на запад, туда,  где уже гремела канонада.

В кухне на обеденном столе была оставлена Василием Петровичем записка, которую немного позже прочтут соседи, пришедшие навестить его. В записке была только одна фраза:
«Ушёл к Мотороле».

17 октября 2016,
Горловка
