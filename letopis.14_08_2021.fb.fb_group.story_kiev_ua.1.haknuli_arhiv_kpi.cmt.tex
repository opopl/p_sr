% vim: keymap=russian-jcukenwin
%%beginhead 
 
%%file 14_08_2021.fb.fb_group.story_kiev_ua.1.haknuli_arhiv_kpi.cmt
%%parent 14_08_2021.fb.fb_group.story_kiev_ua.1.haknuli_arhiv_kpi
 
%%url 
 
%%author_id 
%%date 
 
%%tags 
%%title 
 
%%endhead 
\zzSecCmt

\begin{itemize} % {
\iusr{Даниэль Шапшович}
Вот кто подал Моссаду идею похищения иранского атомного архива из Тегерана)))

\iusr{Дмитрий Мойсеенко}
огонь @igg{fbicon.flame} 

\iusr{Елена Гродзицкая}

\ifcmt
  ig https://i2.paste.pics/d4dc7c62e6fde714c060ccf1e59b8d76.png
  @width 0.2
\fi

\iusr{Андрей Зражва}

\ifcmt
  ig https://scontent-frx5-2.xx.fbcdn.net/v/t39.1997-6/s168x128/17633073_1652591054767295_6333333619058147328_n.png?_nc_cat=1&ccb=1-5&_nc_sid=ac3552&_nc_ohc=8ZKhRg5cTeUAX_s-srr&_nc_ht=scontent-frx5-2.xx&oh=00_AT9D_HZzt9CW7GhPCaYynf97UTbZWIT6j_RC6yCM1GuJUg&oe=61C13C7C
  @width 0.1
\fi

\iusr{Виталии Рыков}

Блиииин!!!! Я как раз в это время учился в корпусе 20. Правда козы не делал.
Очень любил чертить. Да и получалось у меня это.

\begin{itemize} % {
\iusr{Геннадий Хоролец}
\textbf{Виталии Рыков} 20й Корп, этф 81-87

\begin{itemize} % {
\iusr{Yuri Cherniavsky}
\textbf{Геннадий Хоролец} , ЭТФ 1976-1982, Электрические машины

\iusr{Геннадий Хоролец}
\textbf{Yuri Cherniavsky} 81-87эпа, я-первокурсник, а вы-дипломник, но Попович, Гурин, Пыжов-знакомые фамилии

\iusr{Yuri Cherniavsky}
\textbf{Геннадий Хоролец} , Пыжова не помню, но Давыдова, куратора группы, запомнил на всю жизнь @igg{fbicon.face.symbols.mouth} 

\iusr{Геннадий Хоролец}
\textbf{Yuri Cherniavsky} 

Давыдов-очки и рыжие усы? А у нас \enquote{машины} читал Вакуленко. На потоке в
ЭМ-12 был Чернявский Александр Вильевич, колоритная личность...


\iusr{Геннадий Хоролец}
\textbf{Yuri Cherniavsky} вспомнил: Давыдов- с палочкой

\iusr{Yuri Cherniavsky}
\textbf{Геннадий Хоролец} , Вакуленко и у нас читал. Усы и очки правильные приметы, палочку Д. очевидно приобрёл позднее, неужели в ваше студенческое время он уже ковылял или вы его уже позже в 90е видали?

\iusr{Yuri Cherniavsky}
\textbf{Геннадий Хоролец} , я с 1976 был в группе ЭМ 33А, потом вроде после слияния факультетов группы переименовали, так что, номер той , что я заканчивал не помню, что означает группа ЭМ12? Какой год формирования? Однофамильца такого не знаю, но фамилия не редкая, даже здесь в Австралии встретил бывшего киевлянина-однофамильца.  @igg{fbicon.wink} 

\iusr{Геннадий Хоролец}
\textbf{Yuri Cherniavsky} не, в 80х в20ке ковылял, лабы по машинам вел Тимченко. Общага 18ка в ваши годы запускалась?

\iusr{Yuri Cherniavsky}
\textbf{Геннадий Хоролец} , 

Тимошенко звучит знакомо, но лабы у нас вёл Д. и ещё кто-то , может быть и Т.

18 общага!!!! Да была при нас, более того нас зимой 1980 сняли с занятий на
несколько месяцев на \enquote{ремонт} общедитий, их готовили для туристов к олимпиаде
1980. Как известно большинство туристов так и не приехало. Именно с этих
\enquote{ремонтных} работ на 18 общежитии и начался мой конфликт с Д.

\iusr{Геннадий Хоролец}
\textbf{Yuri Cherniavsky} 

ЭМ-12:электрические машины, год поступл-81, группа-2я. Из ваших годов знаю Ваню
Левченко(с его группы Креп-факультетский комсомол). Мы в 81м учились во вторую
смену, а потом запустили 7й корпус. Куда было распределение?

\iusr{Геннадий Хоролец}
\textbf{Yuri Cherniavsky} 

в 81 вселялись в18ку-конфета! До сих пор существует формулировка \enquote{олимпийские
матрасы}. Да, могли встречаться на перекурах на мостике 20го корпуса


\iusr{Yuri Cherniavsky}
\textbf{Геннадий Хоролец} , 

я защищался осенью 1981, потом военные сборы. Зима-Весна 1982. Красный диплом,
выбрал КБ Института Электросварки. По специальности не работал ни дня.

Впрочем ТОЭ пригодилось ещё на многие годы: делал контрольные работы заочникам
разных киевских техникумов.


\iusr{Геннадий Хоролец}
\textbf{Yuri Cherniavsky} 

О, ТОЭ! Федора(Федоренко)! Первую часть знает студент,
вторую-преподователь,третью- не знает НИКТО!(студенч. фольклор) Должны помнить
Залевского-факультетского \enquote{черчение} и \enquote{начертала}. Баба Вера-матан и точно
была баба Клара-\enquote{тряпки} (изоляц материалы)

\iusr{Yuri Cherniavsky}
\textbf{Геннадий Хоролец} , 

фольклор помню, к сожалению других преподавателей, что вы упомянули нет. Из
всех преподов не относящихся к факультету помню Ситенко ( 1-2 курс) физика.
Помню потому, что она одна носила короткую стрижку разных цветов от зелёного до
фиолетового. Курила импортные сигареты, пила кофе с коньяком на экзаменах.
Согласитесь- редкость в советское время. По-моему её муж был директором кафедры
\enquote{твердого тела}. Возможно Залевский вел начерталку на 1м курсе, этого препода
я помню по факту, но не по имени. Помню еще одну приятную даму по теормеханике,
к сожалению не по имени, единственный предмет, который я пересдавал, чтобы не
получить 3-ку.

\iusr{Геннадий Хоролец}
\textbf{Yuri Cherniavsky} 

теормех-скажу завтра. Неужели не помните Федору-бич
факультета, отсев по армии?! Да, а военной кафедры что, не было?

\iusr{Yuri Cherniavsky}
\textbf{Геннадий Хоролец} , 

Федоренко помню смутно, я вообще плохо помню тех преподавателей, с которыми у
меня не было проблем  @igg{fbicon.wink} . Военная кафедра была, я же писал про сборы зимой
весной 1982. ПВО СВ СССР, ЗРК Круг, помню как сейчас  @igg{fbicon.laugh.rolling.floor} 


\iusr{Виталии Рыков}
\textbf{Геннадий Хоролец} 

ЭТФ, группа ЭМ-11. 81-83(декабрь), ТОЭ, Федоренко. Его \enquote{Москвич} грязно -жёлтого
цвета. Армия. Куратор группы Гурин. Как же это было давно! Осенью прошлого года
заходил в 20й корпус, сложилось впечатление какого-то запустения. Особенно
поразила входная дверная группа с приколхожеными деревянными дверьми.

\iusr{Геннадий Хоролец}
\textbf{Виталии Рыков} Шурик Марченко, Эльденант, Мазур, подольцев, Кармазин, староста-Циля. Ваши?

\iusr{Геннадий Хоролец}
\textbf{Виталии Рыков} 

я-ЭП-12 (Тер-Геворгян-друг Марченко). Привет от Шурика (только что перезвонил). До
сих пор в кошмарных снах вижу Линиченко

\iusr{Геннадий Хоролец}
\textbf{Yuri Cherniavsky} теормех-Ильичишина. Круговская батарея, станция наведения ракет.

\iusr{Виталии Рыков}
\textbf{Геннадий Хоролец} Все!!!

\iusr{Виталии Рыков}
\textbf{Геннадий Хоролец} К сожалению многих фамилий я не помню!!! Но напрягая память вспоминаю одногруппников: Потиха Лена, Грудева Таня, Скрипка Вовка

\iusr{Геннадий Хоролец}
\textbf{Виталии Рыков} а красавица Жанна! Я в общаге жил с Фориным(староста эм-12) иМамедовым. То знаю больше их группу. О вспомнил-Ванька Слободянюк! Из потерь (нет в живых) : Юра Ковалев(12я), Мариуш(наш поляк)-женат на Оле с 12й, Гурин в прошлом году...

\iusr{Виталии Рыков}
\textbf{Геннадий Хоролец} 

Мариуша думаю что помню, хотя возможно путаю с нашим Сашей Матяшем, Жанна была
нашей третьей девушкой в группе, а в ЭМ 12 по-моему было две девушки, одна из них
Оля, даже где-то есть с ней фото из колхоза.


\iusr{Геннадий Хоролец}
\textbf{Виталии Рыков} 

посл инфа о Матче: возглавлял какую-то сельхоз газету, а с Толстым (Марченко) все
эти годы ходим в байдарочных походы (Урал, Карелия), приходил на наши встречи
выпускников (немцы, поляки приезжали), Пыжов был

\iusr{Виталии Рыков}
\textbf{Геннадий Хоролец} 

Пыжов запомнился своей борьбой с джинсами,чесно говоря, не помню чем
спровоцированной. А у меня тогда появились первые \enquote{формовые}, и мне пришлось
доказывать стоя на \enquote{ковре}, что то одежда американских пролетариев. Даже
какие-то вырезки из журналов предоставлял! Лето 83го.

\iusr{Геннадий Хоролец}
\textbf{Виталии Рыков} 

тоже под роздачу попадал: пропуски, отпуск для поездки домой, в общаге что-то. Но
больше всего изощрялся ваш Гурин. Посасывая леденец...

\end{itemize} % }

\iusr{Yuri Cherniavsky}
\textbf{Геннадий Хоролец}, 

да теормех у меня вела она. Спасибо за напоминание. Защиту диплома у меня
принимал Новиков.

\begin{itemize} % {
\iusr{Yuri Cherniavsky}
\textbf{Геннадий Хоролец} 

введение в специальность вёл Постников Иван Матвеевич, совсем уже старый
дедушка. У меня с ним ( его именем) уже в Австралии была любопытная встреча, но
об этом в другой раз. У нас уже ночь.


\iusr{Yuri Cherniavsky}
Статья о моей группе в институтской многотиражке За Радянського Iнженера

\ifcmt
  ig https://scontent-frx5-1.xx.fbcdn.net/v/t39.30808-6/237359218_10158105120966674_7279767062476425378_n.jpg?_nc_cat=105&ccb=1-5&_nc_sid=dbeb18&_nc_ohc=BItwaSASEkkAX9g7lfF&_nc_ht=scontent-frx5-1.xx&oh=00_AT_4chAouQ1TYJGgl88N3j2dFzlp0pB70JAlNF-SxMHpcQ&oe=61C09496
  @width 0.3
\fi

\iusr{Геннадий Хоролец}

По-моему-это зав. кафедрой. О себе: учеба 81-87, распредел - Жулянский машиностр
завод, 1990-1997 зам нач снабжения КПИ ( в 90е это величина! ) - поэтому знаю о КПИ
много...


\iusr{Yuri Cherniavsky}
\textbf{Геннадий Хоролец} , понятно откуда такая память!  @igg{fbicon.beaming.face.smiling.eyes} 

\iusr{Yuri Cherniavsky}
Лето 1981, \enquote{последний звонок}. Прощальный поход нашей группы, сбор у 18 общежития. Я во втором ряду, второй слева.

\ifcmt
  ig https://scontent-frt3-1.xx.fbcdn.net/v/t39.30808-6/235229718_10158105129036674_1558502248398385111_n.jpg?_nc_cat=104&ccb=1-5&_nc_sid=dbeb18&_nc_ohc=knfBi94RM9QAX-bo0aZ&_nc_ht=scontent-frt3-1.xx&oh=00_AT-nw92ob3PyMVYfdPyvSMaatcV7_Ad76vH_FfMPYIhW6A&oe=61C19AB9
  @width 0.3
\fi

\iusr{Геннадий Хоролец}
\textbf{Yuri Cherniavsky} 

Разминулись... Я в конце августа только вселялся в 18ку. Газету переименовали в
\enquote{Политехник}. На фото чудак с рюкзаком-только выселился?

\end{itemize} % }

\end{itemize} % }

\iusr{Александр Прыжиков}

Вечно эти политехники ищут приключения. У меня ПТМ, на 4-м курсе, замдекана
факультета, вел. Станевский Валерий Петрович. Автор всех учебников и методичек
по кранам. На пары даже опаздывать не смели, не говоря о прогулах. Предмет был
профильным, подавался легко и очень познавательно. Потому и слушался с открытым
ртом. Да и сам преподаватель, был очень высокого уровня специалист. И никогда
не \enquote{зверствовал}. Хороший дядька был.

\begin{itemize} % {
\iusr{Irina Zhogina}
\textbf{Aleksandr Pryzhikov} сосед мой площадке)

\iusr{Александр Прыжиков}
\textbf{Irina Zhogina} Да, хорошее было семейство...
\end{itemize} % }

\iusr{Олена Медведева- Прицкер}

Спасибо, напомнили старый студенческий термин \enquote{коза}, передавался от
поколения к поколению во всех Технических Вузах.

\iusr{Vladimir Popoff}
Козлотрон) у нас был термин - дралоскоп.

\begin{itemize} % {
\iusr{Валентина Валентина}
\textbf{Vladimir Popoff} первый раз слышу термин \enquote{коза} в тех. Вузе: драласкоп.

\iusr{Александр Прыжиков}
\textbf{Валентина Валентина} хм... Коза и в Африке - коза. Но прибор назывался \enquote{козлоскопом}. @igg{fbicon.face.hand.over.mouth} 

\iusr{Олександр Сєдих}
\textbf{Vladimir Popoff} так, це термін харківських політехніків, а ось аналогу \enquote{козі} не пригадую.

\iusr{Татьяна Волынец}
\textbf{Vladimir Popoff} окончила КАДИ в 1976г. Козлотрон студентов представлял собой перевернутую табуретку, на которую устанавливалось стекло, а под ним была настольная лампа. И этот термин использовался до компьютеризации

\iusr{Наталия Педос}
\textbf{Vladimir Popoff} И у нас на ТЭФе (1969-1975гг.)!
\end{itemize} % }

\iusr{Валентина Козачук}

Я, когда заканчивала пищевой институт, то на дипломный проект покупала себе
козу, по ней было очень легко писать свою тему.. Чертить любила, все чертежи
делала сама без козы.. Любовь к черчение привил в школе учитель математики и
черчения Нименький Владимир Михайлович, был у нас классным руководителем...

\iusr{Maxim Zuk}
Мой родной корпус... ПТМ сдавали примерно так же...

\end{itemize} % }
