% vim: keymap=russian-jcukenwin
%%beginhead 
 
%%file poetry.rus.evgenija_bilchenko.ballada_o_svobode
%%parent poetry.rus.evgenija_bilchenko
 
%%url https://t.me/bilchenko_z/142
%%author 
%%tags 
%%title 
 
%%endhead 

\subsubsection{БЖ. Баллада о свободе}
\label{sec:poetry.rus.evgenija_bilchenko.ballada_o_svobode}

\Purl{https://t.me/bilchenko_z/142}
\Pauthor{Бильченко, Евгения}

У меня в кабинете - пусто. Смешно и пусто.
Я бываю там крайне редко. Такое чувство,
Что начальником быть - совсем не моё искусство.
От большого стола без кукол мне крайне грустно.
От решёток на стенах кости скрипят до хруста.
И, вообще, Диоклетиан мой, моя капуста - 

Это вовсе не овощ. Ясно, что и не доллар.
Это - тексты: мои леса, и моря, и долы.
Это - Киев, Москва и Питер, Арбат с Подолом.
Я ношу их пешком от шапочки до подола
По огромному миру тела, дороги, дома.

То легко их носить: пуховый платок. То миги
Отягчаются веком: пух, обращась в вериги,
Давит рифмами так, что страшно смотреть на книги.
Вот в такие моменты пачками куришь сиги.
Пастернак для Марины образ лесной шишиги

На газетном листе ушедшей эпохи пишет.
Змий боится, что вдруг в пещеру нагрянет Гриша.
Сквозь бескрайнее небо в космос смелей и выше
Поднимаются ели, птицы и те парниши,
У которых под камуфляжем живёт излишек

Позабытой Отчизны с именем "Лёня Быков".
В кабинетах таких шептал на меня до рыка
Тот, кто "шил" мне статью, допрашивая, от пика
Незаслуженной славы скинув меня до писка
Маргиналий культуры. Яму я вижу ликом

Николая Святого, Муромца, Змееборца.
Посему в кабинете этом грешно - бороться.
Лучше делать свою работу без сна, без роста,
Без карьерных неврозов: всякий из них погостом
Завершается... Бог задумал нас очень просто.

1 ноября 2020 г.
