% vim: keymap=russian-jcukenwin
%%beginhead 
 
%%file 25_08_2021.fb.moskovskij_jurij.1.belyje_gribu_fotja
%%parent 25_08_2021
 
%%url https://www.facebook.com/yuriy.moskovskiy/posts/10220265601857952
 
%%author Московский, Юрий
%%author_id moskovskij_jurij
%%author_url 
 
%%tags griby,prazdnik,priroda
%%title На Фотю теплая и ясная погода предвещала множество белых грибов в лесах
 
%%endhead 
 
\subsection{На Фотю теплая и ясная погода предвещала множество белых грибов в лесах}
\label{sec:25_08_2021.fb.moskovskij_jurij.1.belyje_gribu_fotja}
 
\Purl{https://www.facebook.com/yuriy.moskovskiy/posts/10220265601857952}
\ifcmt
 author_begin
   author_id moskovskij_jurij
 author_end
\fi

Сегодня был Фотя Поветенный.

Теплая и ясная погода на Фотю предвещала множество белых грибов.

25 августа А по старому стилю: 12 августа.

Праздник установлен в честь мучеников Аникиты и Фотия, которые приходились друг
другу дядей и племянником. Аникита был военным сановником в Никомидии во
времена императора Диоклетиана (конец 3 – начало 4 веков), но однажды выступил
против своего правителя — когда тот выставил на городской площади орудия казни
для устрашения христиан. Разгневанный император приказал подвергнуть сановника
пыткам, а затем бросить на съедение диким зверям; но выпущенный лев, вместо
того чтобы растерзать мученика, стал ласкаться к нему.

\ifcmt
  pic https://scontent-cdt1-1.xx.fbcdn.net/v/t1.6435-9/240760180_10220265636098808_7407066432847229616_n.jpg?_nc_cat=101&ccb=1-5&_nc_sid=8bfeb9&_nc_ohc=oQB8keSUi2MAX_P2sOj&_nc_ht=scontent-cdt1-1.xx&oh=0a9874b5c56ca35cffb608c797e962bb&oe=614CE542
  width 0.4
\fi

Аникиту пытались казнить разными способами, но каждый раз Господь спасал его. В
итоге император приказал разжечь огромную печь и казнить в ней мучеников.
Множество христиан, вдохновленные подвигами святых родственников, сами шагнули
в жерло и погибли с молитвой на устах. Аникита и Фокий тоже умерли, но тела их
не пострадали от огня. Увидев это чудо, многие язычники уверовали во Христа.

На Руси этот день прозвали Фотей Поветенным, потому что было принято
прибираться на поветях, где хранились лошадиные упряжи, бороны и сохи. «Фотя
Поветенный хозяину покоя не дает — на поветь зовет», — говорили мужики. По их
убеждению, нельзя было допустить, чтобы «на поветях черт ногу переломил».

На Фотю с утра примечали: если выпал иней, значит можно ждать хорошего урожая
озимых на следующий год. Если в этот день было дождливо — не стоило надеяться
на длинное бабье лето. А теплая и ясная погода предвещала множество белых
грибов в лесах.

Именины в этот день:

Александр, Алексей, Антон, Аркадий, Василий, Виссарион, Вячеслав, Герман,
Дмитрий, Ефим, Иван, Илья, Леонид, Матвей, Михаил, Николай, Памфил, Петр,
Савва, Сергей, Степан, Федор, Яков
