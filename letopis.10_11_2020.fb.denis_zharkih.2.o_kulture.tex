% vim: keymap=russian-jcukenwin
%%beginhead 
 
%%file 10_11_2020.fb.denis_zharkih.2.o_kulture
%%parent 10_11_2020
 
%%url https://www.facebook.com/permalink.php?story_fbid=2851931538353578&id=100006102787780
%%author 
%%tags 
%%title 
 
%%endhead 

\subsection{О культуре}
\label{sec:10_11_2020.fb.denis_zharkih.2.o_kulture}
\Purl{https://www.facebook.com/permalink.php?story_fbid=2851931538353578&id=100006102787780}
\Pauthor{Жарких, Денис}

Деградация управления и профессионализма началось не вчера, это проблема
мировая. Застал СССР весьма молодым человеком, но что-то помню. Уже тогда в
столицу приезжало огромное количество людей из провинции в поисках лучшей
жизни. 

То, что они приезжали и приезжают, ничего плохого в себе не несет, но проблема
в том, что многие из них хотят получить столичные блага и удобства, особо не
напрягаясь, не платя настоящую цену, растолкав по дороге всех прочих. Понимают
они или нет, но они эти блага хотят отобрать, а не заработать. 

Естественно, это не распространяется на всех приезжих, да и не все коренные
киевляне ангелы.  Забрать и заработать не только разные психологии, но и
противоположные, совершенно не совместимые. В СССР существовали квоты на
колхозников и правильные национальности (были национальности и неправильные).
Украинец, да еще и колхозник в СССР была правильная национальность и занятие в
Киеве, а потому масса народу, по сути не являвшимися ни украинцами, ни
колхозниками, в таковые записывались. 

Отличие профессиональной среды СССР от нынешней была в том, что в большинстве
социальных сред колхозник, реальный и фальшивый, должен был отойти от
периферийных замашек и показать свою культуру и интеллект. Для многих
провинциалов это была непосильная мука, о чем в каждом номере свежего "Перца"
была карикатура. 

Современная жизнь не только не ставит ограничения на дикость, хамство и
невоспитанность, но всячески их поощряет. Американская мечта, по крайней мере,
как ее нам подавали, это и есть мечта жлоба в чистом виде. Напрямую никто не
требует хамить и наступать на ноги, но активно учат не замечать других, жить
своими интересами, а это, по сути, тоже самое. 

В идеале советская система пыталась навязать чувство прекрасного, здоровый
образ жизни и творческое раскрытие, но ключевое слово тут навязать. Многие
навязывающие этих ценностей не разделяли. А тем, кто разделял, навязывать
ничего не нужно было. А потом пошла другая крайность - людей освободили от
условностей и ограничений и многие показали свое истинное лицо. 

Если бы не события последних лет, я бы не понял, кто живет возле меня и что из
себя представляет. При этом картина совершенно не однородная - одни люди
оказались чудовищами, а другие имели в себе огромные достоинства, мудрость и
героизм. В спокойное время никогда бы не различил одних и других. Тогда были
все одинаково более-менее приятные. 

Мой современник живет в культурной среде, когда нужно быть жлобом, просто
необходимо. Но жлобами становятся далеко не все. И это подогревает мою веру в
человечество. Через всю мою жизнь идет одна картина, как жлобство уничтожает, и
это история современной Украины. Вот раньше нужно было прятать свое жлобство, а
теперь выпячивать, даже если его нет. И эта среда сама не изменится. Но важно
увидеть, поддержать профессиионализм в других людях, поощрить их, помочь им. То
же и с порядочностью и честностью. И все это называется культура, а не
этнические скачки и неродной для многих язык.
