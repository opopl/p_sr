%%beginhead 
 
%%file 14_08_2023.fb.vobornikova_golovko_julia.1.uragan_proty_koloradiv_bytva_persha
%%parent 14_08_2023
 
%%url https://www.facebook.com/100001565218169/posts/pfbid0sbMJApu2q346bvokVmghbGkWRmgvu1n7Nn38sozKN3j46PWPak7HPxt5YfANZRkgl
 
%%author_id vobornikova_golovko_julia
%%date 14_08_2023
 
%%tags 
%%title Історії з ненаписаної книги життя. Ураган проти колорадів))) Битва Перша))))
 
%%endhead 

\subsection{Історії з ненаписаної книги життя. Ураган проти колорадів))) Битва Перша))))}
\label{sec:14_08_2023.fb.vobornikova_golovko_julia.1.uragan_proty_koloradiv_bytva_persha}

\Purl{https://www.facebook.com/100001565218169/posts/pfbid0sbMJApu2q346bvokVmghbGkWRmgvu1n7Nn38sozKN3j46PWPak7HPxt5YfANZRkgl}
\ifcmt
 author_begin
   author_id vobornikova_golovko_julia
 author_end
\fi

\#дитинство\_цинічного\_снайпера)).

Історії з ненаписаної книги життя.

Ураган проти колорадів))) Битва Перша)))).

Якщо ви думаєте що підготовка бійців ССО починається з учебки 142го НЦ, то
глибоко помиляєтесь))). Мене до  війни з сепарсько-колорадсткою пошестю почало
життя готувати ще з дитинства))). І перший мій інструктор - то бабчин город)).

Як знають усі, хто родився ще за часів пізнього совка, єдиним нормальним
бажанням наших батьків, було збагрити своїх чад на канікули в село до дідусів і
бабусь. А в селі принцип один - пахають там усі, від мала до велика))). Село
моєї бабці євпорейськими цінностями не відрізнялося, тому дітей, привезених з
міста, старалися припахати до посильного для них заняття))). Бабка моя не була
виключенням, і як тільки сходила картопля, мені була поставлена задача із
категорії \enquote{міссія нєвилолніма} - збирати на городі, з тої ж таки вічно
потірпевшої картоплі, колорадських жуків(((.

Колорадський жук то ще та падла. Під листям ховається, в землю ничкується, і
жере наче американський морський піхотинець Фарва, з яким напару мені довелося
навалювати сепарам в 17му) - тоєсть жере постійно і багато))). 

Розуміючи всю безнадьогу мого херового положення собіратіля орогодних
колорадів, я хоч і було гімно мале, але ідея оптимізації процессу утилізації
колорадів, в меня зародилася вже тоді))). Сиділа я значить на криші свининця і
думу думала, яж же мені цю падлу полосату з городу звести))). Варіант випустити
курей щоби вони їх зклювали відпав майже зразу. По-перше я потім цю діч хер
соберу обратно в курятник, а по-други ці падли настільки харчами перебирають,
що фіг тих короладів їсти будуть(((. 

Варіант втопити в воді теж відпав, бо такої довгої шланги в дворі небуло.
Спалити теж не катіть, бо згорять разом з картоплею(((. А не буде картоплі -
значить не буде бабчиних дерунів, а їх я наминати дуже любила з домашньою
сметаною))).

Кароче - тупік(((.

І тут на сусідньому городі я взріла сусідку, тітку Вальку Тхорівську, яка
чимчикувала з відром і облізлим віником в строну своєї картоплі)))). Я не
поняла смисла проісходящого. Невже тітка Валька підмітати город буде?

Рішила я значить розвідать тактічеську обстановку, і підійти взнать, що за
шабаш такий вона задумала)))

Воняти почало ще на підході. Не сильно, але неприємно. На моє питання що вона
буде робити, я отримала гучну відповідь : колорадів кропити!!!)))

Мені ясен пень стало цікаво, на який ляд їх кропити і шо це дасть? Тітка як
людина опитна, бо у самої неї була пара внуків, постаралась мені як дитині
пояснити, що оцією смердючою водою вона кропить кущі. Жукам то не нравицця, і
вони пиздують з її городу))). Я уточнила чи вони пиздують того що воно воняє, і
якшо так, то куди саме пиздують? На перший вопрос тітка відповіла ствердно, а
от на другий промовчала)))). Але в мені вже тоді жив геній військової
стратєгії, і я поняла, шо жуки з її городу пиздують на наш!!!)))

Але ж в мені ще жив і розвідник, який і підказав, що боротися з ворогом тре
його методами))). Тому я ще раз уточнила у тітки: чи ідуть жуки з городу коли
їм смердить? На що вона ствердно кивкула, і сказала не морочить їй голову))).

Отже інформація була прийнята, діло оставалося за малим - найти те що смердить
як дурне, і покропити ним кущі по периметру города))).

В моєму босоногому дитинстві найгірше що воняло, це були бабчині духи " Красная
Москва"), які ій дід регулярно привозив зі своїх відряджень зі стольного граду
Києва))). З них я рішила і почати))). Надибала я їх в бабчиному триляжі, в
коробочці. Коробочка була оставлена, а духи забрані на город))). Розвела я їх в
своєму пласмасовому маленькому відерці з водою, витряхуючи із флакону як дурна,
і поняла що концентрація воні замала, тре ше шось добавить))). 

Набіг на хату повторився))). В миснику були найдені духи маман, в такому
синенькому флакончику ( як я потім дізналася то були духи Пані Валєвкая, і
стояли вони дурних грошей по блату). Ці духи маман необережно забула під час
свого приїзду в гості на тій неділі))). 

А ще с того ж таки мисника я дістала здорову непрозрачну бутилку, жидкость в
якій трохи воняла самогоном і якимись лікарствами ( Рижський Бальзам. Його
дядько, моряк загранплавання привіз з рейсу діду на подарок). І бутилка та була
почті повна))).

Отже, вооружившись цим боєкомплектом, я почапала на город))). Злила я ті духи і
ту бутилку в відеречко)). Сморід від тої суміші мене вполнє удовлітворив))).

Метнулась я до кролятнику, і найшла там обшарпаного малого віника, яким бабка у
кролів чистила, і була цілком довольна))). І так, во всеоружії я героїчно
вступила в свою першу битву з колорадами)))

Смерючої суміші на весь периметр не хватило, але опасні зони, тоєсть межу з
сусідами я таки покропила))). Шмон я скажу  стояв ще той))).

Першим на нього зреагірував сусід дядя толя, що якраз ішов у уборну за
сараєм))). Його тонкий нюх старого алкаша, мабуть став би предметом для
заздрощів любого самельє))). Бо зі слів його жінки, тітки гальки, самогон він
як собака знаходив по запаху))). 

Швиденько до мене підбігши, цей раб бога Діонісв спитав,  а чим це у нас
покроплено? Мене переповнила гордість, і я мала, взявши його за руку, повела на
місце де я ту суміш бодяжила, попутно розповідаючи, що то для зничтоженя
колорадів я сама рецепт придумала))).

Коли сусід побачив пусті пляшечки і велику коричневу пляшку, то опустився на
коліна, по ходу від болі і розпачу))). На його лиці в той момент читалася вся
одвічна скорбота єврейського народу... Він ледь не плакав)))).

В цей момент з магазину у хвіртку, на велосипеді Аїст зарулила бабка))).
Глянувши на мене нещасним поглядом дядя толя лише тихо прохрипів: біжи дитино,
біжи. Бо приб'є тебе щас бабка))).

Два рази мені повторять не тре було. Бо у бабки рука тяжола, а лозина
крепка))).

І поки бабка ставила покупки в холодільнік, я уже сиділа на горіхові)))).
Наблюдаючи з безпечного місця за подіями на городі)))).

В цей час підтягнулася і тітка валька, яка йшла доїть свою корову Зорьку, і уже
у всю горлянку звала мою бабцю на город....)))

Далі були три години проклять бабки під горіхом, з погрозами всіма карами
небесними, вплоть до того, що мене вернуть в місто, якщо я зараз не злізу))). Я
на бабчині обіцянки не велась років з двох, тому всі її тіради були без
толку))).

Ввечері повернувся дідо))). Як завжди вислухавши крики бабці, він виліз до мене
на горіх.

Я сиділа обіжена. Бо щож то таке, хотіла помогти, а мене за це ще й бить
хотять.

Довго дідові пояснювала, що жуки не люблють коли смердить, а тітка валька
падлюка, їх на нашу картоплю вигнати хотіла))). Дід так сміявся, що ледь з
горіха не впав))).

Далі бабку успокоїв що духи їй він нові купить, а самогон у неї кращий за любий
бальзам))). А мене сказав не чіпать строго-настрого))). Бо чортиня я хоч і
вредне, але по натурі добре, і до захисту свого городу схильне))). І ба на той
захист нічого не жаліє)))).

Іноді мені здається що дядя толя, після того як усі пішли з городу, намагався
облизувати кущі)) А може то моя мала дитяча фантазія увірвалася в пам'ять))).
Не знаю. Але те що сусід ще довго ходив з тугою в очах, то це факт))). 

А суть цієї історії проста - з самого дитинства я нічого не шкодувала для
боротьби із ворогом.

Як не шкодую і понині. Ні тисячі доларів вкладених у власну зброю і
спорядження, ні свого здоров'я, ні  життя.

Пройшли роки, і на мою землю прийшли інші колоради. Тільки тепер це двоногі
тварини, що вбивають мою країну. І я як завжди, чи то по одному як в дитинстві
в баночку, чи по багато як тітка валька за допомогою отрути, буду їх
відстрілювати і вбивати, поки або геть не винищу, або поки вони у свій
москальський город самі не повернуться.!

Написано 2020го року.

В деяких реченнях сміливо міняйте колорадів на орків))). 

Вибачаюся за мат і суржик)). Але так воно написалося.
