% vim: keymap=russian-jcukenwin
%%beginhead 
 
%%file 27_11_2021.fb.iljenko_andrej.1.golodomor
%%parent 27_11_2021
 
%%url https://www.facebook.com/andriy.illenko/posts/7334270559947004
 
%%author_id iljenko_andrej
%%date 
 
%%tags golodomor,ukraina
%%title Голодомор
 
%%endhead 
 
\subsection{Голодомор}
\label{sec:27_11_2021.fb.iljenko_andrej.1.golodomor}
 
\Purl{https://www.facebook.com/andriy.illenko/posts/7334270559947004}
\ifcmt
 author_begin
   author_id iljenko_andrej
 author_end
\fi

Причиною Голодоморів була поразка у війні за Незалежність 1918-1921 років.
Геноцид можливий лише проти бездержавної нації. Українці свою державу втратили.
Чому так сталося?

\ifcmt
  ig https://scontent-frt3-1.xx.fbcdn.net/v/t39.30808-6/261997211_7334270206613706_9126139359221592610_n.jpg?_nc_cat=104&ccb=1-5&_nc_sid=8bfeb9&_nc_ohc=NXGetSnWSaEAX-t7VbB&_nc_ht=scontent-frt3-1.xx&oh=c7e1792e842a3720968e42fd69600a4c&oe=61AC9604
  @width 0.4
  %@wrap \parpic[r]
  @wrap \InsertBoxR{0}
\fi

«Головна причина нещастя нашої нації – брак націоналізму серед її загалу». Так
сказав колись Микола Міхновський, один з ключових українських політиків тих
часів. Сто років тому мільйон естонців змогли зберегти свою державу, а сорок
мільйонів українців — ні. Чому? Бо в естонців було багато націоналізму, а в
українців недостатньо. Була фанатична меншість, яка боролася за Україну до
кінця, але більшості це не стосувалося. 

Нема націоналізму — нема своєї держави. Нема своєї держави — є геноцид. І
геноцид стосується всіх, у тому числі тиж, кому було байдуже. Окупантам не
байдуже, їм треба було знищувати українців як таких, без різниці ідейних чи ні.

Якщо хтось вам говорить, що нам не потрібна держава і армія, що анархія це
добре і т.д. — перед вами небезпечний ідіот чи відвертий ворог, що в принципі
те саме. І його маразматичні ідеї призводять до того, що мільйони загинуть у
муках. 

Україна сто років тому була охоплена лівацькою ідеологією. Правда в тому, що не
лише Україна, а вся Європа. Але, наприклад, польські соціалісти змогли стати
державниками (а по суті націоналістами) і втримати незалежність, а українські
соціалісти лишилися в полоні утопічних ілюзій і Україна програла. Винниченко,
Махно та їм подібні несуть свою частину відповідальності за українську
катастрофу у ХХ столітті. 

Москва хоче повторити геноцид, якщо в неї буде така можливість. Єдине, що її
зупиняє — українська армія. І національна свідомість українців, тільки завдяки
якій є українська держава. 

Згадаймо сьогодні наших мертвих. Заради живих і наступних поколінь.

\ii{27_11_2021.fb.iljenko_andrej.1.golodomor.cmt}
