% vim: keymap=russian-jcukenwin
%%beginhead 
 
%%file 30_04_2020.fb.fb_group.story_kiev_ua.2.kievskie_mozaiki.pic.17
%%parent 30_04_2020.fb.fb_group.story_kiev_ua.2.kievskie_mozaiki
 
%%url 
 
%%author_id 
%%date 
 
%%tags 
%%title 
 
%%endhead 

\ifcmt
  ig https://scontent-frt3-1.xx.fbcdn.net/v/t1.6435-9/95283806_3174548275912091_3069542465649246208_n.jpg?_nc_cat=104&ccb=1-5&_nc_sid=b9115d&_nc_ohc=dA14h9xjVJgAX-epHKQ&_nc_ht=scontent-frt3-1.xx&oh=d3315156fe3ab57a6ab531f41db37d42&oe=61B4F774
  @width 0.4
\fi

\iusr{Ирина Петрова}
Морська тема

\iusr{Nina NinaNina}

довга багатофігурна мозаїка, багато кадрів, бул Лесі Українки, 24, автори
мозаїки: Анатолій Гайдамака, Лариса Міщенко,

\iusr{Ирина Петрова}

скажіть, удь ласка, Ви якось причетні до творців мозаїк? Дуже хвилює одне
питання. Чи це просто вивчення питання по гугупошуку? Дякую за коментарі та
роз'яснення, смае на таке обговорювання сподівалась при створенні теми! Дякую!

\iusr{Nina NinaNina}

багато років фотографую мозаїки, і не тільки київські, але і в інших містах, і
за кордоном, бо часто подорожую, та й вже деякий час розказую, як екскурсовод
про місто, і не тільки в Києві - як гід у подорожах вихідного дня в автобусних
або індивідуальних маршрутах. Колись мріяла стати мистецтвознавцем, але не
склалось. Можливо, це не зовсім те, що ви очікували, але використовую мотиви
мозаїчних панно та вітражів для створення робіт з тканин - печворку та розпису
скла, бо закінчила університет Шевченка за технічною спеціальністью.

\iusr{Ирина Петрова}
\textbf{Nina NinaNina} дуже дякую! Саме таких коментів очікувала. Пишіть ще, дякую!

\iusr{Елена Галузевская}
Школа 82, вул. Шпака 4, з боку пр. Перемоги було панно, чи збереглось?
