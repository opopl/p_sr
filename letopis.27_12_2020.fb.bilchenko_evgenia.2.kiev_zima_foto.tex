% vim: keymap=russian-jcukenwin
%%beginhead 
 
%%file 27_12_2020.fb.bilchenko_evgenia.2.kiev_zima_foto
%%parent 27_12_2020
 
%%url https://www.facebook.com/yevzhik/posts/3510914628943622
 
%%author Бильченко, Евгения
%%author_id bilchenko_evgenia
%%author_url 
 
%%tags bilchenko_evgenia,foto,gorod,hristos,kiev,ukraina
%%title Импрессионизм Царя Небесного. Киев сегодня. Зимняя весна.
 
%%endhead 
 
\subsection{Импрессионизм Царя Небесного. Киев сегодня. Зимняя весна.}
\label{sec:27_12_2020.fb.bilchenko_evgenia.2.kiev_zima_foto}
\Purl{https://www.facebook.com/yevzhik/posts/3510914628943622}
\ifcmt
 author_begin
   author_id bilchenko_evgenia
 author_end
\fi

\ifcmt
  tab_begin cols=2

     pic https://scontent-lga3-2.xx.fbcdn.net/v/t1.6435-9/132443885_3510913478943737_923474578559131984_n.jpg?_nc_cat=105&ccb=1-3&_nc_sid=8bfeb9&_nc_ohc=n4DnAVBWmKQAX8d-_4N&_nc_ht=scontent-lga3-2.xx&oh=54291d76728786a5b5052450c378f79a&oe=60CB95CE

     pic https://scontent-lga3-2.xx.fbcdn.net/v/t1.6435-9/133562302_3510913658943719_4266291644933343263_n.jpg?_nc_cat=111&ccb=1-3&_nc_sid=8bfeb9&_nc_ohc=cB2LDNMoz6AAX_vRdcJ&_nc_ht=scontent-lga3-2.xx&oh=f91afe78ce38621c1fdb1581ec8940cc&oe=60CB1F67

		 pic https://scontent-lga3-2.xx.fbcdn.net/v/t1.6435-9/133357836_3510913838943701_5340610521987301936_n.jpg?_nc_cat=101&ccb=1-3&_nc_sid=8bfeb9&_nc_ohc=ySEa5T-eF1UAX9PKi4-&tn=ntrKbsW_7ChXu3v-&_nc_ht=scontent-lga3-2.xx&oh=f94b9e1f3260d0aa9f97433d96af8c8c&oe=60CB7A8B

		 pic https://scontent-lga3-2.xx.fbcdn.net/v/t1.6435-9/132870932_3510914065610345_5567249868932363607_n.jpg?_nc_cat=104&ccb=1-3&_nc_sid=8bfeb9&_nc_ohc=ltzIk-RzTAsAX-vdRBS&_nc_ht=scontent-lga3-2.xx&oh=41b40d4438c4a5d131af49eb0a0fae96&oe=60CA6A7D

		 pic https://scontent-lga3-2.xx.fbcdn.net/v/t1.6435-9/132947954_3510914198943665_7284408313414416844_n.jpg?_nc_cat=105&ccb=1-3&_nc_sid=8bfeb9&_nc_ohc=iqZM-9hD9SoAX_0B1NT&tn=ntrKbsW_7ChXu3v-&_nc_ht=scontent-lga3-2.xx&oh=3579f4678fcf9e0222fc7ec8bf60fadb&oe=60CB7722

  tab_end
\fi

\emph{Мирослава Александровна Бердник}
Алаверды

\ifcmt
  pic https://scontent-lga3-2.xx.fbcdn.net/v/t1.6435-0/p235x165/133357834_1119203521865230_2103934228158326655_n.jpg?_nc_cat=107&ccb=1-3&_nc_sid=dbeb18&_nc_ohc=pDU9-Yh6Yb4AX9CTgNS&_nc_ht=scontent-lga3-2.xx&tp=6&oh=9af23daf1df1f57ef00614ffa23c75a5&oe=60CBC933
	width 0.2
\fi

\emph{Мирослава Александровна Бердник}

\ifcmt
  pic https://scontent-lga3-2.xx.fbcdn.net/v/t1.6435-0/s526x296/133404443_1119203605198555_803062240499612140_n.jpg?_nc_cat=109&ccb=1-3&_nc_sid=dbeb18&_nc_ohc=Y0IpmNjBYsMAX_anhJl&_nc_ht=scontent-lga3-2.xx&tp=7&oh=0b5a9c4e1077feef77f69c5d1cd8b81b&oe=60CAC5A7
	width 0.2
\fi

\emph{Мирослава Александровна Бердник}

\ifcmt
  pic https://scontent-lga3-2.xx.fbcdn.net/v/t1.6435-0/p526x296/133343471_1119203685198547_1303325568289189878_n.jpg?_nc_cat=102&ccb=1-3&_nc_sid=dbeb18&_nc_ohc=TtF-q0yIewAAX_8N10I&_nc_ht=scontent-lga3-2.xx&tp=6&oh=529fa128e77aaa3d59ef83093eb837c7&oe=60CBD67A
	width 0.2
\fi

\emph{Tanya Ponomareva}
Да, красиво... На Москву похоже..

\emph{Евгения Бильченко}
\textbf{Таня Пономарева} Киев и Москва вообще очень похожи.

\emph{Tanya Ponomareva}
\textbf{Евгения Бильченко} Мне тоже так кажется. Да и люди очень похожи.

\emph{Мария Исаева}
Евгения Бильченко не похожи...

\emph{Мария Исаева}

Ну конкретно тут похоже

\emph{Евгения Бильченко}

А мне - похожи: по южному женскому духу: и в этом плане, как говорил В.
Топоров, они предстоят северному Питеру. Но лучше всего - семиотическая
гармония двух полюсов цивилизации.

\emph{Мария Исаева}
Возможно... Я по-другому воспринимаю. У них разные характеры, как по мне.

\emph{Евгения Бильченко}
\textbf{Мария Исаева} Я вижу общность в доминации Anima. Нежность, ведьмовство,
чувственность, кабацкость, карнавальность, скоморошечность, гигантскость и
крошечность, златоглавая купольность...

\emph{Kevork K Hayrabedian}
NKR

\ifcmt
  pic https://scontent-lga3-2.xx.fbcdn.net/v/t1.6435-9/133881431_10164753774670083_8285893051998714316_n.jpg?_nc_cat=107&ccb=1-3&_nc_sid=dbeb18&_nc_ohc=LUpysF-B01sAX9bsYxN&_nc_ht=scontent-lga3-2.xx&oh=9d442f37ca6fed96cd2477dabdacbcbf&oe=60CB50EF
	width 0.2
\fi

\emph{Tanya Ponomareva}

Я вижу общность ощущения ДОМА. Этот город так же странноприимен, так же добр,
несмотря ни на что. Я чувствую родство. Хотя, конечно, стиль одежды москвичей -
это уже притча во языцех. Киев наряден, разноцветен, индивидуален и свободен в
самовыражении. Люди в Москве все еще предпочитают черное, монашеское. И если
попадаются в метро люди, одетые в красное, розовое и всячески яркое - это
скорее всего туристы. Но отдельное чудо - это питерский стиль. Это вообще
другая планета. На это я бы смотрела и смотрела. Как на огонь и воду. Мне этого
не повторить, как невозможно повторить и этот город.

\emph{Евгения Бильченко}
\textbf{Таня Пономарева} Киев разрушаем и несвободен. Но подлинный еще заявит о себе.

\emph{Sergey Kutuzov}

В Москве в чёрном ходят преимущественно кавказцы и приезжие не русские. Просто
их много и кажется, что всё серое и чёрное. Эти люди одеваются как в 90-е годы,
когда всё грошовое китайское прибывало на рынок в Лужники. Есть и яркие
разноцветные южные люди, но их мало.

\emph{Tanya Ponomareva}

\textbf{Сергей Кутузов} Я бы спорила, я сама хожу в черном. Но есть в Ваших
словах опрелеленная правота. Приезжих с Кавказа очень много, и да, у них своя
субкультура, которая тоже привносит свою краску в цвет толпы. Но мне кажется,
что есть еще и психологический момент - желание стать невидимым, не привлекать
внимание, определенный камуфляж. В переполненности города мир кажется
агрессивным, в нем хочется затаиться. И люди неосознанно выбирают темные
краски. Город отяжелел от людей, их много, им тесно, они невольно вторгаются в
личное пространство друг друга.

\emph{Sergey Kutuzov}

\textbf{Таня Пономарева} Одеть цветное - надо подумать сперва над подбором
цветов и понять себя. Непривычное занятие для большинства людей из
родоплеменных отношений. Иерархия, внутренний зажим, агрессия и неуверенность в
себе - причины чёрных тонов. Второе - хороших вещей в продаже не много, надо
искать и опять думать. Может советоваться с кем-то. Люди пока на том уровне,
где они есть - ничего не поделаешь.

\emph{Tanya Ponomareva}

Про родоплеменные отношения, конечно, интересно...

\emph{Евгения Бильченко}

А я и то, и то ношу. Причем я никогда не думала о тренде, когда я ношу. Просто
когда мне хорошо, у меня в голове звучит Есенин и хочется яркого. А когда плохо
- в голове Бродский и хочется черного. А когда совсем плохо - в голове Высоцкий
- и я надеваю предельно яркое. Мне кажется, что не стоит впадать в эту риторику
отличий субкультур, свойственную для американской постструктуралистской
антропологии, к которой постсоветское население приучили для разделения людей
на некие референтеные группы по внешним идентификационным маркерам.

\emph{Tanya Ponomareva}
\textbf{Евгения Бильченко} Правильно!

\emph{Виталий Бильченко}

Дитя моё... может в период пандемии подумать и о карьере фотохудожника?
Мне нравится..)
Целую
