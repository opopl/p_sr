% vim: keymap=russian-jcukenwin
%%beginhead 
 
%%file 05_11_2021.fb.fb_group.story_kiev_ua.1.rasskaz_borsch
%%parent 05_11_2021
 
%%url https://www.facebook.com/groups/story.kiev.ua/posts/1790765497786910
 
%%author_id fb_group.story_kiev_ua,koval_oleg
%%date 
 
%%tags borsch,rasskaz
%%title Повтор шуточного рассказа Борщ. Все персонажи и исторические факты вымышлены!
 
%%endhead 
 
\subsection{Повтор шуточного рассказа Борщ. Все персонажи и исторические факты вымышлены!}
\label{sec:05_11_2021.fb.fb_group.story_kiev_ua.1.rasskaz_borsch}
 
\Purl{https://www.facebook.com/groups/story.kiev.ua/posts/1790765497786910}
\ifcmt
 author_begin
   author_id fb_group.story_kiev_ua,koval_oleg
 author_end
\fi

Повтор шуточного рассказа Борщ. Все персонажи и исторические факты вымышлены!
 @igg{fbicon.smile} 

— Заходь и сидай сынку, — подтягивая деревянной клюкой стул к столу у окна,
хриплым, едва слышным и немного дрожащим, но полным надежды голосом, проскрипел
дед Коля.

— Сейчас я расскажу тебе все, шо знаю о борще и Борщаговке, как обещал.
Расскажу как на духу, ни чего не тая о славных людях, населявших эти земли, и о
том, как ходили они в походы, добывая заморские трофеи. Живу я здесь с самого
своего рождения, так же как и мой отец, отец моего отца и отец его отца, вплоть
до самого Царя Гороха. О своей прапрабабке, котороя была из анатолийцев, тоже
упомяну, потому как именно из-за её кровей в нашем роду у всех темные волосы,
смуглая кожа и чёрные как смоль глаза. Да ты сидай, не стой у порога, у нас
невеста в доме. Ну шо, принёс?

\ifcmt
  ig https://scontent-frt3-1.xx.fbcdn.net/v/t1.6435-9/252494937_4551254698268983_5034608744036752304_n.jpg?_nc_cat=107&ccb=1-5&_nc_sid=825194&_nc_ohc=-EkhqzoRpoEAX_YzNCG&_nc_ht=scontent-frt3-1.xx&oh=a5547319060eb35f5ef7aa4fc1bdc1c6&oe=61AC87F8
  @width 0.4
  @wrap \parpic[r]
\fi

— Да, принёс, — ответил я.

— Ну, дык сидай, не томи, — проглотив слюну, будто в каком-то предвкушении,
потирая скрюченные ревматизмом ладони, молвил старик.

Я присел за грубый деревянный стол, стоявший в светлой, хорошо освещённой
дневным светом, но немного захламлённой старыми вещами, комнате, с большой,
белой, расписанной яркими красными цветами сельской печью.

— Наливай, — поставив алюминиевую кружку перед собой, приказным тоном, скрипнув
словно амбарная дверь, отрезал дед, — да только по троху, и шоб Катька не
видела, она шибко не любит, когда я подшофе.

Я достал бутылку и налил грамм 50 в слегка помятую посудину.

Дед Коля, забыв о лёгкой дрожи в руках, мигом подхватил кружку. Сосредоточив
все своё внимание на этом действии, быстрым движением поднес её ко рту, после
чего с той же, непонятно откуда взявшейся удалью, ни капельки не морщась,
залпом осушил содержимое.

— Ух, — занюхав огромной желтой грушей, сиротливо лежавшей на массивном
деревянном столе, старик продолжил, — ну теперь слухай, дело было так, заметно
оживившись, продолжал свою речь дед Коля.

— История эта берет своё начало с тех незапамятных времён, когда Киевские
Князья поклонялись Перуну. А в памяти киевлян еще свежи были воспоминания о
славных походах Игоря на Византию. Южные окраины Киева в то время населяла
воинственная община русичей, одним из промыслов которой были набеги на
заморских соседей. Местность эта именовалась Нивой из-за необъятных глазу
полей, на которых жители выращивали хлеб. Мужчины, жившие в наших краях, слыли
богатырской силой и отличным знанием военного дела, также были хорошими
плотниками и неплохо ориентировались на местности по звёздам. А их женщины
славились своей неповторимой природной красотой и незаурядными кулинарными
способностями.

— Да не седи ты как истукан, подлей, — хитро поглядывая то в окно, то на
входную дверь, немного раздраженно просил не на шутку взбодрившийся старик.

Подлив ещё 50, я чуть было не поставил бутылку на стол, но дед меня одернул.

— Ну, что ж ты творишь! Спрячь, я же просил. Катька, шельма, ругать будет.

Я спрятал бутыль в рюкзак. Дед тяпнул и не спеша продолжил.

— С приходом весны, когда вся Русь праздновала Перунову неделю, мужчины
начинали приготовления к походу на Анатолию. По всей округе начиналась вырубка
леса для строительства лодок. В дело шли самые ровные и высокие сосны. К
моменту, когда начинались весенние паводки и речка Нивка полноценно наполняла
свои берега, чайки, а именно так назывались лодки, сделанные умелыми руками
русичей, были готовы к спуску на воду. С собой в поход брали только самое
необходимое. В качестве провианта сушёные фрукты и питьевую воду. Да, точно,
воду, — старик остановил рассказ, задумчиво потёр рукой огромные козачие усы,
затем продолжил.

— Мёд, как тогда называли напиток, содержащий алкоголь, брать категорически
запрещалось, — немного прищурившись, уводя взгляд куда-то в сторону, затем
сосредотачивая на одиноко лежащей груше, вроде как припоминая что-то, дед
продолжал.

— Дисциплина, значит, у них поддерживалась суровыми наказаниями, пьянство
считалось самым тяжким проступком, ну прям, как моя Катька, трясця ейной
матери, чертова девка, — смачно выругавшись, выпалил старик.

— Да... До и после пей, хоть сколько душе угодно, но только не во время похода.
Железные нравы, знаешь ли. Ну, а если кто и был замечен в пьянстве, того в тот
час же и без разбирательств, по негласному уставу, выбрасывали за борт с
привязанным грузом к ноге. О, значит как! Но как правило, этого не случалось,
потому как желающих пойти на корм рыбам было мало, хотя, кто нас остановит.

— Правда?! — дед пристально посмотрел на меня.

Я подлил 50 и замер в ожидании продолжения.

Опрокинув очередную порцию и закусив грушей, рассказчик продолжил.

— Вооружение, значит, в те времена было лёгким, с собой брали только мечи, щиты
и копья. Ремонтным комплектом для устранения течей служила сосновая смола и
льняная ветошь. В набегах брали участие только женатые мужчины, достигшие
сорокалетнего возраста, у которых к тому времени было как минимум пятеро детей.
Молодые же должны были оставаться для защиты поселения и работы на нивах. В
последний день Перуновой недели на воду спускались от десяти до двадцати таких
лодок по десять человек в каждой. Чайка представляла собой длинное низкобортное
судно пятнадцати метров в длину, с высокой мачтой для квадратного паруса,
который мог ловить только попутный ветер. Во время штиля или лобового ветра
лодка передвигалась с помощью весел или попросту шла по течению. После спуска
на воду Нивка подхватывала своим бурным потоком судна и несла их к реке Ирпень,
которая, в свою очередь, впадала в Борисфен, как тогда называли Днепр. Пройдя
Вышгород, в котором в те времена располагался гарем князя Владимира, чайки
ловили попутный ветер и, подняв паруса, шли к Чёрному морю в сторону берегов
Анатолии.

Дед возмущённо остановился, в недоумения глянул на меня.

— Ну, и шо сидим, заслухался? Да штоб тебя! Наливай! — в приказном тоне,
украдкой поглядывая в окно, продолжал слегка потдатый старик.

Но вдруг дверь скрипнула, в дом вошла молодая девушка, темноволосая красавица
лет двадцати пяти, аппетитные формы которой напоминали изящный плод спелой
груши.

— А вот и моя голубка, внучка Катенька. Хозяюшка, а до чего ж заботливая, а как
готовит! М-м-м, язык проглотишь, — дед быстрым взглядом указал мне на кружку,
мол, спрячь.

— Деда, ты снова за старое взялся, опять свои байки людям травишь? Ну, не
дурачь людей!

Затем, направив на меня свой смолистый взгляд, добавила, — Не слушайте его, наш
дедушка большой выдумщик.

— Цыц мне! Это из райсобеса пришли, интересуются, как мы тут живы-здоровы, —
хитро подмигнув мне, быстро объяснился с внучкой старик.

Покраснев, девушка юркнула в угол комнаты, сделав вид будто что-то потеряла.

— Ну, ты иди Катенька, мы тут ещё потолкуем,  — мягко и по-отцовски заботливо
просил внучку дед.

Поправив в печи чугунный глечик, девушка покинула комнату также быстро, как и
вошла. Дед Коля снова тяпнул и продолжил.

—  Так на чем я остановился?

— Сплав чаек по Днепру.

— Ах, да... Весь сплав занимал около десяти дней, самым опасным участком были,
так называемые, пороги в районе острова Хортицы, после сухого преодоления
которых чайки по берегу перекатывались на брёвнах, чтоб не разбиться о
скалистое дно, — дед вкратце объяснил способ преодоление припятствий.

— Затем дружинам оставалось всего-то ничего, пройти незамеченными мимо
византийских галер, дежуривших на Чёрном море в районе устья Борисфена. Галера
представляла собой большое военное, парусно-гребное судно от десяти метров в
ширину до пятидесяти метров в длину, с двадцатью шестью вёслами по каждому
борту и тремя треугольными парусами, имеющая на вооружении абордажные гарпунны
и около 500 человек на борту. Как правило, бросок мимо византийских
сторожевиков происходил ночью, потому как за счет своих низких бортов чайки
сливались с темным горизонтом и были не заметны глазу наблюдателей. Именно
малый размер чаек и их узкий силуэт помогали пройти незамеченными мимо
сторожевых кораблей. Выйдя в открытое море и поймав попутный ветер, русичи
направлялись к противоположному берегу. Спустя неделю они достигали берегов
Анатолии, причаливая, разбивали лагерь и начинали тщательную подготовку. По
периметру стоянки выставлялся пост из десяти человек, остальные расходились по
прибрежной территории, углубляясь до пятидесяти километров, а то и более, от
береговой линии. Целью набегов были нападение на поселения Анатолийцев, которые
в те времена славились как искусные ремесленники. Добыча была разнообразной от
семян сельскохозяйственных культур и предметов быта, до молодых девушек,
которых в дальнейшем можно было выгодно продать как наложниц. Собрав добычу,
хлопцы возвращались в лагерь, дожидались побратимов и отходили восвояси к устью
Боресфена, к тому месту где дежурили византийские суда. Бывали случаи, когда с
наступлением темноты чайки незаметно подплывали к одной из галер и по команде
брали её на абордаж. Решающим фактором в захвате военного судна была
неожиданность. Галеру грабили, команду вырезали и выбрасывали за борт. Трофеи
были разнообразные от навигационных приборов и карт, до абордажных орудий. В
некоторых случаях попадалось золото. Ограбленную галеру поджигали, после чего
она необратимо шла ко дну. Путь к дому становился открыт. Шото запершило в
горле, — покашливая, пожаловался старик.

Я снова подлил... Дед выпил и немедля ни секунды продолжил.

— Спустя пару месяцев чайки подходили к Киеву. Причалив на Подоле, так до сих
пор называется один из самых старых районов города, промысловики сбывали часть
награбленного. В основном товары ремесленного характера. Дальше путь лежал в
Вышгород, к летней резиденции Владимира, где устраивались торги наложницами.
После чего князь давал трёхдневный пир, на котором раздавал титулы и земли
самым щедрым мореплавателям. Отдохнувши в княжеских палатах, промысловики
оставляли в дар князю свои чайки, нанимали волов и, перегрузив остатки
награбленного, шли дальше через девственный лес Берковца к своим Нивам. Дома их
ждали жены и дети.

Задумавшись и немного переведя дыхание, старик продолжил.

— Добравшись, значит, до своих поселений, русичей ждал тёплый приём. Остатки
награбленного, корнеплоды овощей и семена заморских злаков, орудия земледелия,
военные трофеи отдавались в хранилище общины. Девушкам, которые не были проданы
на торгах, даровалась свобода. Празднование продолжалось около семи дней.
Гуляния сопровождались пьяными плясками вокруг костров и дикими ночными
оргиями.

На следующий год семена овощей и злаков, раздобытых в Анатолии,
высаживали в плодородный чернозём. А через пару месяцев собирался богатый
урожай. Так впервые на Руси был выращен буряк, который в своё время
древние греки использовали исключительно для жертвоприношений Богу
Аполлону, морковь, история которой своими корнями уходит далеко в Азию,
где она использовалась как целебное растение, и томат, до этого
считавшийся ядовитым плодом, но выращиваемый европейцами как декоративное
растение.

Девушки — анатолийки, одна из которых моя прародительница, научили
местных умелиц готовить из помидоров томатную пасту, которой впоследствии
заправлялась пассировка к первому киевскому борщу. Позже приготовление
этого блюда превратилось в целое искусство. Любая, уважающяя себя хозяйка
умела наколдовать из диковинных заморских овощей это божественное блюдо.
В каждой хате по всем Нивкам ощущался завораживающий запах борща. Ходят
слухи о том, как южными ветрами ароматы овощного отвара доносились до
самих Золотых Ворот. Владимир так впечатлился вкусом борща, что издал
указ о присвоении особого статуса этой местности. Позже, после Крещения
Руси, нивы были переданы монастырям, с тех пор их называют Борщаговка, и,
кстати, речка Нивка тоже по подобию тех мест стала называться
«Борщаговка».

Спустя века Киевский борщ пополнился новым ингредиентом, всеми нами любимой
картошкой, настоящий рецепт которого теперь хранится в архивах города, как
наследие стародавней гастрономической культуры Киевской Руси. Во, как!

Дверь скрипнула, в дом снова вошла смуглянка.

— Ну, шо, Катюша, поспел наш борщ? Давай-ка угостим гостя миской фирменного со
сметанкой да с тоненько нарезанным сальцом, так, как ты умеешь. И бабушкины
пампушки с чесночком тоже неси. Ну-ка, давай. А?

А ты сынку, шо замечтался?  Наливай!

Ближе к вечеру, дед провожал меня до калитки.

— Пойдём сынку в сад, я тебе ещё не такое покажу, смори какие у меня на вербе
груши растут...

Поддакивая ему я вышел в сад у небольшого водоема. Пройдя немного к ставку,
образованному речкой Нивка, дед указал рукой на огромную иву, в свисающих
ветвях которой росли большие желтые груши...

\ii{05_11_2021.fb.fb_group.story_kiev_ua.1.rasskaz_borsch.cmt}
