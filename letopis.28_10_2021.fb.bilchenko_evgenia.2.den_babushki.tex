% vim: keymap=russian-jcukenwin
%%beginhead 
 
%%file 28_10_2021.fb.bilchenko_evgenia.2.den_babushki
%%parent 28_10_2021
 
%%url https://www.facebook.com/yevzhik/posts/4386005384767871
 
%%author_id bilchenko_evgenia
%%date 
 
%%tags babushka,bilchenko_evgenia,pamjat,semja
%%title БЖ. День бабушки
 
%%endhead 
 
\subsection{БЖ. День бабушки}
\label{sec:28_10_2021.fb.bilchenko_evgenia.2.den_babushki}
 
\Purl{https://www.facebook.com/yevzhik/posts/4386005384767871}
\ifcmt
 author_begin
   author_id bilchenko_evgenia
 author_end
\fi

БЖ. День бабушки

Она была такая городская, что аж жуть...Она была такая худая, что ее страшно
было обнять, чтобы не сломать. Она сидела на балкончике, увитом девичьим
виноградом, и читала любимых писателей: Дюма, Пушкин, Тютчев, Драйзер, Чехов.
Она танцевала в балете, и ее отчим считал, что хватит уже ей смотреть в то
зеркало. Она была дочерью русского белого и красного потом (до 1917, не
приспособился, избрал) офицера, дворянина, "врага народа" родом из Москвы и
частично из Ленинграда, и все говорили: на отца похожа, слишком тонкие черты,
слишком язвительный юмор для здешних мест. Она училась на врача, и четверок у
нее не было. И отец ее реабилитирован ПОЛНОСТЬЮ. Без всяких там.

\ifcmt
  tab_begin cols=3

     pic https://scontent-lhr8-2.xx.fbcdn.net/v/t1.6435-9/250055838_4386005094767900_7235891892081645914_n.jpg?_nc_cat=105&ccb=1-5&_nc_sid=8bfeb9&_nc_ohc=EoBJPD66CD8AX_vbHtS&_nc_ht=scontent-lhr8-2.xx&oh=61eee5c722a66c67d908294eb41acf13&oe=61A2654E

     pic https://scontent-lhr8-1.xx.fbcdn.net/v/t1.6435-9/250143425_4386005174767892_4985200452719205746_n.jpg?_nc_cat=107&ccb=1-5&_nc_sid=8bfeb9&_nc_ohc=dVbbuEX4A3wAX94pv9U&_nc_ht=scontent-lhr8-1.xx&oh=9853eb328b79a18238e18aca44877a49&oe=61A07D15

     pic https://scontent-lhr8-1.xx.fbcdn.net/v/t1.6435-9/250198476_4386005251434551_7329366507928684784_n.jpg?_nc_cat=108&ccb=1-5&_nc_sid=8bfeb9&_nc_ohc=kK7PQ04tx2IAX-lU9Ns&_nc_oc=AQlQ0kAokNXTfLATg4rA9-m9eOhJkW9EP3vf22ZbA95exkV1zV5qQmA4T9ZbJJIVPdI&tn=lCYVFeHcTIAFcAzi&_nc_ht=scontent-lhr8-1.xx&oh=3be87ecdc8a26c24d8b5aeb1007e0853&oe=61A2BF33

  tab_end
\fi

А потом в дом к падчерице главного архитектора города, тоже бывшего дворянина,
и дочери партизанки, кавалера ордена мужества, к этой девочке, всей такой, в
шелковых платьицах, в квартиру, куда захаживало на ликер партийное руководство,
пришел... украинский пацан Иван. Пришел, немного поломал будущему тестю дорогие
готовальни и камеры (мишка пробежал, лапкой махнул), повредил кохиноры, коими
была завалена вся коммуналка вкупе с книгами, циркулями и ватманами... и...
Забрал ее на 50 лет.

\ifcmt
  ig https://scontent-lhr8-2.xx.fbcdn.net/v/t39.30808-6/249757671_4386005551434521_2546465299052576206_n.jpg?_nc_cat=102&ccb=1-5&_nc_sid=8bfeb9&_nc_ohc=jcNm4C_g_FcAX9ik8du&_nc_ht=scontent-lhr8-2.xx&oh=c45fc0b7ec0f1e7f843bb2680ba1c25f&oe=618125F5
  @width 0.5
  @wrap \InsertBoxR{0}
  %@wrap \parpic[r]
\fi

И сказала она, улыбаясь, как Наташа Ростова перед первым балом: "Ванечка, котик
мой, стань военным. Тебе форма идёт больше цивильного тряпья. Ты совсем не
умеешь носить изящные костюмы, а сельские рубахи навыпуск - совсем не мое, ты
их вечно заляпываешь".

И Ваня такой: куда служить едем? Холодная война, все дела. В уютное родное
село, где она задружилась с его родителями, хотя они не умели почти писать, и
сменила шляпу на грабли и косынку, - или на русский Север? 

"На Север, в гарнизон, блат моего отчима и мамы-героя войны не берём, это
стыдно", - непререкаемо и командно изрекла она, ибо из Наташи Ростовой за миг
умела рождаться Хозяйка Медной Горы. Я это помню: моим главным врачом была
именно она. Когда не было опасности, не назначалось ничего. Когда дело было
швах, приказы отдавались быстрее деда самому деду. 

Ну, не Север, а граница с Японией, но все равно: поселок Сокол - это не балкон
с виноградом и Дюма. Она отвоевала себе шляпу, брюки и двадцать вызовов в день
рядового врача по сугробам Сахалина. Она встретила маршала Советского Союза и
сказала: "Я тут командую, на кардиограмму, срочно, вы мне не маршал, а больной,
вас лечить надо". Маршал обнял ее и сказал остальным врачам: "Вот такие нам в
медицине нужны, а не вы, лизоблюды". Вылечила. И было ей 25 тогда.

Где такие люди сейчас, или мы реально совсем подофигели?
