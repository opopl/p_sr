% vim: keymap=russian-jcukenwin
%%beginhead 
 
%%file 27_01_2022.stz.news.lnr.lug_info.1.lardt_premjera_mjuzikl
%%parent 27_01_2022
 
%%url https://lug-info.com/news/luganskij-teatr-na-oboronnoj-priglashaet-na-prem-eru-myuzikla
 
%%author_id news.lnr.lug_info
%%date 
 
%%tags donbass,kultura,lardt,lnr,lugansk,muzyka,teatr
%%title Луганский театр на Оборонной приглашает на премьеру мюзикла
 
%%endhead 
 
\subsection{Луганский театр на Оборонной приглашает на премьеру мюзикла}
\label{sec:27_01_2022.stz.news.lnr.lug_info.1.lardt_premjera_mjuzikl}
 
\Purl{https://lug-info.com/news/luganskij-teatr-na-oboronnoj-priglashaet-na-prem-eru-myuzikla}
\ifcmt
 author_begin
   author_id news.lnr.lug_info
 author_end
\fi

Луганский академический украинский музыкально-драматический театр (ЛАУМДТ) на
Оборонной 29 и 30 января представит премьеру восстановленного спектакля \enquote{Моя
прекрасная леди} по пьесе Бернарда Шоу \enquote{Пигмалион}. Об этом сообщила
пресс-служба Министерства культуры, спорта и молодежи ЛНР со ссылкой на
учреждение культуры.

\ii{27_01_2022.stz.news.lnr.lug_info.1.lardt_premjera_mjuzikl.pic.1}

Премьера спектакля \enquote{Моя прекрасная леди} состоялась в ЛАУМДТ в 2009 году, и он
стал одним из самых популярных у зрителей. В связи с боевыми действиями в 2014
году и последующими изменениями в составе труппы показы мюзикла были
приостановлены.

\enquote{В восстановленном спектакле театра – прекрасные мелодии, живое исполнение
музыкальных номеров, яркая хореография, костюмы, декорации, в которых оживет
лондонский мир – от знаменитых королевских скачек в Аскоте до рыночной площади,
– рассказали в театре. - По сюжету, профессор фонетики Генри Хиггинс и
ученый-индолог, полковник Пикеринг на площади Ковент-Гарденского рынка
задумывают поучаствовать в эксперименте – превращении простой цветочницы Элизы
Дулиттл в настоящую леди. Профессор Хиггинс приводит Элизу в свой дом и
ручается за шесть месяцев сделать из нее леди}.

Режиссером-постановщиком восстановленного спектакля выступил заслуженный артист
ЛНР Виталий Лясников, художник – заслуженный деятель искусств ЛНР Виктор
Горбулин, балетмейстер-постановщик – Владимир Замлынный, хормейстер – Ирина
Юсупова. Дирижером-постановщиком спектакля был недавно ушедший из жизни
заслуженный работник культуры УССР Георгий Снаговский.

Начало спектаклей – в 15:00.

ЛАУМДТ находится по адресу: Луганск, улица Оборонная, 11. Дополнительную
информацию можно получить по телефонам (0642) 92 11 77, (0642) 92 11 81, (0642)
92 11 83, на сайте театра или его странице в социальной сети \enquote{ВКонтакте}.
