% vim: keymap=russian-jcukenwin
%%beginhead 
 
%%file 09_12_2021.fb.fb_group.fajna.1.bander_chupakabra
%%parent 09_12_2021
 
%%url https://www.facebook.com/groups/202660744246697/posts/647173359795431
 
%%author_id fb_group.fajna,bova_tatjana
%%date 
 
%%tags banderovec,nacionalizm,nenavist,pravyj_sektor,ukraina
%%title Бандеровская чупакабра
 
%%endhead 
 
\subsection{Бандеровская чупакабра}
\label{sec:09_12_2021.fb.fb_group.fajna.1.bander_chupakabra}
 
\Purl{https://www.facebook.com/groups/202660744246697/posts/647173359795431}
\ifcmt
 author_begin
   author_id fb_group.fajna,bova_tatjana
 author_end
\fi

Бандеровская чупакабра.

\ifcmt
  ig https://scontent-frx5-1.xx.fbcdn.net/v/t39.30808-6/264878163_650733922972626_8274274797953346651_n.jpg?_nc_cat=110&ccb=1-5&_nc_sid=825194&_nc_ohc=FD5UPV-RtjsAX8U81nS&_nc_ht=scontent-frx5-1.xx&oh=00_AT8Y1ew7E6nUvpqLC0yToiQoLhDKevtYfT2YNsGtD3U4mg&oe=61C11669
  @width 0.4
  %@wrap \parpic[r]
  @wrap \InsertBoxR{0}
\fi

\enquote{Сільпо}, каса, черга. Переді мною викладає покупки бабуся. Старенька така, в
хустині пуховій закашлатаній, пальто з вилинялим песцем, 1983 року народження,
на комірі. Під пахвою затиснуті паличка і ручної в'язки рукавички. Кладе на
транспортер вагову вермішель, два курячих крильця, півхлібини і жменьку
карамельок. Розраховується, бере решту. 5грн. і копійки. Спаковує продукти до
матерчатої сумки. Звертається до касирки:

- А это коробки стоят для сборов на Армию или на инвалидов? А то я очки
забила...

- Та яка вам різниця, - каже касирка.- Не затримуйте очерідь.

У мене вже під кутнім зубом щось залоскотало, серпентивне таке і жутко
педагогічне. По методиці Макаренка. Але бабуся спокійно так каже :

- Позовите администратора.

Касирка починає пропікувати мої покупки:

- Зара! Робити нема шо. Йдіть собі.

Але адміністратор вже тут. Ввічлива, но нервова. Слухає, типу важно, бейдж
теребить.

- То, бабо, для бандерівців з Правого Сектора, - каже.- Йдіть, бо то вам не
треба, бережіть пенсію, а то на аптеку не хватить.

Бабуся забирає решту, кладе її до потертого гаманця. Виймає звідти 100 грн і
запихає в скриньку.

- Эти Герои всё преодолеют! И жить будут по-человечески. А вы, две дури выйдете
замуж за имбецилов и будете битые пьяными мужьями, как драные козы. Потому что
дуры - они всегда битые...

©Дзвінка Торохтушко
