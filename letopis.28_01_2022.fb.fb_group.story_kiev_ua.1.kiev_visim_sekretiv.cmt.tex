% vim: keymap=russian-jcukenwin
%%beginhead 
 
%%file 28_01_2022.fb.fb_group.story_kiev_ua.1.kiev_visim_sekretiv.cmt
%%parent 28_01_2022.fb.fb_group.story_kiev_ua.1.kiev_visim_sekretiv
 
%%url 
 
%%author_id 
%%date 
 
%%tags 
%%title 
 
%%endhead 
\zzSecCmt

\begin{itemize} % {
\iusr{Ірина Дебкалюк}
Дякую!

\iusr{Инна Ревуцкая}
Благодарю, за такую интересную экскурсию в историю родного города!!! @igg{fbicon.hands.applause.yellow}{repeat=3} 

\iusr{Олена Гороховець}
Дякую!

\iusr{Rita Ritulya}
Спасибо за замечательную экскурсию.

\iusr{Ирына Лыпченко}
Дуже дякую!

\iusr{Inna Latysheva}

Неймовірно цікаво і несподівано. Зізнаюсь, ніколи не чула про князівські
поховання. Чи в музеях про які Ви згадуєте є таблички і пояснення англійської
мовою?

\iusr{Igor Poluektov}
\textbf{Inna Latysheva} 

в \textbf{Музей Становлення української нації} точно є кілька мов (англійською і жива екскурсія).
В інших переважно оригінали. Тобто, латинь переважно.

\iusr{Валентина Світельська}
Щиро дякую.

\iusr{Люда Турчина}
Здорово, написано с любовью до Киева!

\iusr{Наталья Снигирь}
Дякую!

\iusr{Sofiya Kazakova}
Дякую за допис!

\iusr{Людмила Пащинська}

коли литово-руські !!!!!!!!!!!!! війська перемогли Ординців і звільнили більшу
частину Руси-України. В то время о Украине никто не знал - что она когда-то
будет !


\iusr{Людмила Пащинська}
Если пишите о истории - надо быть точными.

\iusr{Людмила Пащинська}

А Белая Русь на карте есть ? Куда ее дели ? В Великом княжестве Литовськом и
руськом - будущие украинцы и белорусы - назывались именно - руським народом.

\begin{itemize} % {
\iusr{Людмила Пащинська}

А слышали о \enquote{Ливонской хронике} – документальном произведении 13 века,
рассказывающем о войнах немецких крестоносцев в Прибалтике. Воевать им пришлось
в основном с Польшей, Литвой и Русью. Но до Киева (да что там до Киева, даже до
Смоленска!) они не доходили.

Так что же тогда считал автор хроники Русью? Посудите сами. В одном месте он
пишет: \enquote{Псков находится в Русской земле}. Еще в одном сообщает: \enquote{В Русской
земле есть город Новгород}. Наконец, имеется у него фраза, которая повергает
любого сторонника теории об \enquote{украденном} имени в шок: \enquote{В Русской земле есть
город Суздаль}. А Суздаль – это в двух шагах от Москвы. Собственно Москва и
была тогда маленьким городишком во владениях суздальского князя. Он туда своих
сыновей назначал.

\iusr{Людмила Пащинська}

В походе на Киев приняли участие 11 князей: младшие братья Андрея Боголюбского:
Глеб Юрьевич Переяславский и Всеволод Юрьевич (будущий Всеволод Большое
Гнездо), их племянник Мстислав Ростиславич; князья - Роман Ростиславич
Смоленский, Рюрик Ростиславич Овручский,!!!!!!!!!!!!!!!! Давыд Ростиславич
Вышгородский !!!!!!!!!!!!!!!и Мстислав Ростиславич (место княжения не названо);
Ольговичи: Олег Святославич Новгород-Северский, Игорь Святославич Курский и
Путивльский (будущий участник похода на половцев и герой «Слова о полку
Игореве»); Владимир Андреевич Дорогобужский; командование осуществляли сын
Андрея Боголюбского Мстислав Андреевич.

\end{itemize} % }

\iusr{Людмила Пащинська}

Юрія Довгорукого, якого зовсім юним (як найменшого в родині) відправили в
далекі провінційні землі Залісся, які залежали від Руси, але нею не були, а
саме в Ростово-Суздальську землю. І там, на землях угро-фінського населення,
під боярським наставництвом, він виріс зовсім відірваним від традицій Києва.

\iusr{Людмила Пащинська}

Даже читать не хочется - так все перекручено........... Почему такие ...типа
историки даже своих не читают - например Антоновича - он то уж точно описывал
МАГДЕБУРЗЬКОЕ ПРАВО.

\iusr{Olga Lys}
Щиро дякую!

\iusr{Natali Mel-Nik}
Дякую!

\iusr{Александра Лобынцева}
Большое спасибо!

\iusr{Sergyi Grabar}

Дякую, шановний пане Ігорю, за Вашу послідовну, стійку позицію щодо відродження
правдивої історії нашого міста і усієї України. Хотілося б зустрітися і
поділитися своїми враженнями і знаннями. Надзвичайно цікаво і надзвичайно
корисно те, що Ви робите. Мій телефон: 0975129948

\iusr{Igor Poluektov}
\textbf{Sergyi Grabar} 

дякую за добрі слова, пане Сергію!

Знаю, що і Ви вірно захищаєте інтереси України багато років.

Буду радий поспілкуатись на теми історії Києва і України.

Напишу Вам.

\iusr{Вікторія Молявко}
Цікаво, дізналась про дві нові для мене цікавинки

\iusr{Олег Синельников}
Чудово! Щиро дякую! Багато нового. Виникло бажання пройтися тими маршрутами.

\iusr{Victoriya Yarovaya}
Дякую

\iusr{Natalia Bastun}
Chydovo!!!! Dyakyu za taky zikavy informaziy!!!!

\iusr{Ольга Іванюк}
Дякуємо.

\end{itemize} % }
