% vim: keymap=russian-jcukenwin
%%beginhead 
 
%%file 10_09_2021.fb.gorovyj_ruslan.1.2091_kiev
%%parent 10_09_2021
 
%%url https://www.facebook.com/gorovyi.ruslan/posts/6308091809201751
 
%%author_id gorovyj_ruslan
%%date 
 
%%tags 2091,fantastika,future,kiev,nezalezhnist,ukraina
%%title 2091 рік. Київ
 
%%endhead 
 
\subsection{2091 рік. Київ}
\label{sec:10_09_2021.fb.gorovyj_ruslan.1.2091_kiev}
 
\Purl{https://www.facebook.com/gorovyi.ruslan/posts/6308091809201751}
\ifcmt
 author_begin
   author_id gorovyj_ruslan
 author_end
\fi

\obeycr
2091 рік. Київ.
Над Дніпром повільно рухається прогулянковий міськоплан, на якому стоять хлопець і дівчина. 
- Який же він прекрасний, - сплескує в долоні дівча. 
- Так, відтоді як винайшли телепортацію, переселили людей в Ніжин та Бахмач і знесли всі хмарочоси на схилах, він неймовірний, - погоджується хлопець.
- Важко повірити голографічний проекції, яка показує, шо раніше було тут набудоване.
- Так. Раніше було стрьомно. Будували хто шо хотів, без генплана. Добре, що Київ відновили як було в часи УНР, в двадцятому сторіччі.
- Зато тепер лише історичні пам’ятки чи їхні візуалізації. Ну і туристи.
- Так і є. Ну і є метро. Його лишили для екстремалів. Хочеш покататися на метро?
- А це безпечно?
- Зі мною, мала, все безперечно, - хлопець натискає кнопку на пульті, - метро Університет. 
- Який шлях обрати? - питається металевий голос.
- Найкрасивіший.
- Виконую.
Міськоплан повільно рухається від Дніпра в бік схилу і далі до червоного корпусу університета Свтого Володимира.
\#100роківНезалежності.
\restorecr

\ii{10_09_2021.fb.gorovyj_ruslan.1.2091_kiev.cmt}
