% vim: keymap=russian-jcukenwin
%%beginhead 
 
%%file 08_10_2021.fb.bryhar_sergej.1.deti_progulka_mova_jazyk
%%parent 08_10_2021
 
%%url https://www.facebook.com/serhiibryhar/posts/1805519569647987
 
%%author_id bryhar_sergej
%%date 
 
%%tags deti,jazyk,kultura,mova,park,rossia,ukraina
%%title Гуляю з малими в "Парку перемоги"
 
%%endhead 
 
\subsection{Гуляю з малими в \enquote{Парку перемоги}}
\label{sec:08_10_2021.fb.bryhar_sergej.1.deti_progulka_mova_jazyk}
 
\Purl{https://www.facebook.com/serhiibryhar/posts/1805519569647987}
\ifcmt
 author_begin
   author_id bryhar_sergej
 author_end
\fi

Гуляю з малими в "Парку перемоги". Доходимо до мосту через став, і я чую якесь
добре знайоме примітивне "бум-бурум". 

Підходжу ближче, і ось вонооо: "забірай мєня скорєй, увозі за сто морєй". 

На мосту діти, як правило залипають надовго - там штурвал, ринда, - і мої - не
виключення. Стою. Слухаю. Далі: "ти прості мама, что я бил такой упрямий, но я
служить должен так же, как всє...". Потім знову щось із "рукі вверх", потім ще,
здається, з "отпєтих мошенніков". І все це супроводжується дурними в криками
"пєвца" на зразок "юхххууу, Одєсса, зажигаєм под наши родниє мотівиии".

\ifcmt
  ig https://scontent-frt3-2.xx.fbcdn.net/v/t39.30808-6/244997856_1805519536314657_880488193970680027_n.jpg?_nc_cat=101&ccb=1-5&_nc_sid=8bfeb9&_nc_ohc=bVIWBuqg4cAAX-CoVCS&_nc_ht=scontent-frt3-2.xx&oh=17d11a7d9cf087cf5620611a16b170ab&oe=61672D68
  @width 0.4
  %@wrap \parpic[r]
  @wrap \InsertBoxR{0}
\fi

А найнеприємніше те, що під це моск***ське лайно танцюють малюки (дякувати
Богові - не мої... поки їх вдається втримувати поза межами
"загальноприйнятого", поза межами тутешньої "нормальності", але зрозуміло, що
так не може тривати вічно)...

І знову згадується, як я за тижні перебування у Львові, так ніде й не почув
"русскіє пєсні" (ну хіба що з машин окремих довбнів, але не можу сказати, що це
було часто). У Винниках, в кафе, пам'ятаю, крутили "Холодно" "Мертвого півня",
"Хата моя" "Плачу Єремії" та інші якісь в буквальному сенсі "раритетні" речі...
І тут в мене виникає закономірне питання: "може ось так і має бути?"

Але ні. В нас тут люба успєнская, іріна круг, мєладзе та інша московська
шолупонь не просто звучать у публічних місцях, а ще й виступають з концертами.

Боротися з Росією, безумовно, потрібно не лише на фронті, на Донбасі. Біда в
тому, що вона тут, у наших, українських містах, у наших парках і оселях наших
сусідів, мешканці наших міст вважають її культуру своєю і привчають до неї
своїх дітей.

Може я особливо уважний до фонових композицій, написів, вивісок, але от
повертаючись з Галичини додому, маю відчуття, ніби приїжджаю в іншу країну.

\ii{08_10_2021.fb.bryhar_sergej.1.deti_progulka_mova_jazyk.cmt}
