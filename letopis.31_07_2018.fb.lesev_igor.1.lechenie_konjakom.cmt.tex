% vim: keymap=russian-jcukenwin
%%beginhead 
 
%%file 31_07_2018.fb.lesev_igor.1.lechenie_konjakom.cmt
%%parent 31_07_2018.fb.lesev_igor.1.lechenie_konjakom
 
%%url 
 
%%author_id 
%%date 
 
%%tags 
%%title 
 
%%endhead 
\subsubsection{Коментарі}

\begin{itemize} % {
\iusr{Сергей Головачёв}
Представляю, как радуется таким постам Доктор Смерть. Свою задачу она выполнила.

\begin{itemize} % {
\iusr{Игорь Лесев}

Верняк. До нее было все платное, но хотя бы не было проблем найти терапевта и
сделать рентген. Теперь вообще нихера нельзя получить, начиная с диагноза.

\end{itemize} % }

\iusr{German Gorozhanski}
А шо то за история, когда у мужика при каком-то слове рука тянулась к пистолету? Уж не реформа-ль медицины в стране подопытных мышей?

\iusr{Анддрей Михайлов}
Осторожней будьте, я в прошлом году с воспалением легких в больницу и попал, 21 день там пролежал.

\begin{itemize} % {
\iusr{Игорь Лесев}
вот в нашу больницу меньше всего хочется, это уже если совсем жопа-жопа, но спасибо за предостережение
\end{itemize} % }

\iusr{Юрий Воронежцев}

Я, к сожалению, вынужден регулярно проходить в нашей гомельской клинике курсы
химиотерапии. И почти всегда среди коллег по несчастью - люди из Украины. Для
них у нас все платно, но они говорят, что все равно обходится дешевле, чем
дома, да и получается эффективнее. Хотя у нас медицина тоже "не фонтан".

\begin{itemize} % {
\iusr{Игорь Лесев}

у меня френд-знакомый как раз из Украины в Гомель перебрался. очччень
нахваливает местных врачей. Для белорусов привычно, а у нас уже взгляд на все,
как у ангольцев или танзанийцев

\end{itemize} % }

\iusr{Марина Прохорова}

У меня на прошлой неделе из счётчика искры летели. В ОСББ электрика нет, в
Киевэнерго по всем телефонам автоответчик предлагал занести показания,
диспетчерская послала туда, "куда вы деньги платите", аварийку вызвать
невозможно. И это Киев

\begin{itemize} % {
\iusr{German Gorozhanski}
Потому шо "спасение утопающих дело рук самих утопающих"

\iusr{Игорь Лесев}

у меня на прошлой неделе все вместе полетело - сначала появился кашель, потом
наипнулось что-то в машине, и я музыку даже не слышу, а потом как раз стала
плавиться розетка... но с электриком мне повезло. Это как качественный
стоматолог - таких людей нужно сразу же записывать в телефонную книжку


\iusr{Марина Прохорова}
Это на самом деле распад. Такого не было ещё пару дет назад. Вы можете гореть, тонуть, умирать - система больше не реагирует ни на что ((

\iusr{Марина Прохорова}
\textbf{Игорь Лесев} Нас спас одноклассник сына - электрик, случайно встреченный, когда мы метались по парадному)

\iusr{Игорь Лесев}

Верно. Есть какая-то самореакция в обществе, когда на крики девушки за окном
начинают свистеть/кричать люди, но все это в режиме "как повезет".

\iusr{German Gorozhanski}
Система замкнулась в себе. Всякие попытки проникновения в неё мочатся мощным антивирусником Системы.

\iusr{Игорь Лесев}
\textbf{Марина Прохорова} да, ща на электрике можно жениться) кругом одни балбесы - юристы, экономисты и политологи, а у кого есть электрик - у того в доме свет и гарантия того, что все не сгорит к ипеням, пока возвращаешься домой... хотя нет гарантий, что квартиру не очистят, но электриков-ментов - таких я еще не встречал

\iusr{Ирина Кузнецова}

Я тоже недавно об этом думала . Все возвращается на круги своя. В старых
советских фильмах прославляли рабочие профессии. Тогда монтажники, плотники,
трактористы были в почёте. Потом пришло время 'интеллигентов', и озвучивать,
что твоё чадо учится в ПТУ стало стыдом стыдным. А сейчас вижу, что найти
хорошего плиточника, сантехника - за счастье. И зарабатывает хороший плиточник
больше хорошего экономиста или учителя.

\iusr{Natalie Kon}

А частной конторы, в которой есть мастера на все руки, в Киеве нет? В Донецке
такая фирма была - типа скорой помощи по хозяйству, где можно вызвать любого
специалиста - электрика, сантехника, прибивателя гвоздей и навешивателя
карнизов.

\iusr{Марина Прохорова}
\textbf{Natalie Kon} Может и есть. Но в ситуации, когда всё вот-вот вспыхнет синим пламенем, нужна аварийная служба.

\iusr{Natalie Kon}
\textbf{Марина Прохорова} Да, тут вы правы. Значит. такие люди должны быть в телефоне в группе "уmergency":))

\iusr{Ксения Гурбич}
\textbf{Марина Прохорова}, нужно поднять вопрос на собрании и заключить договор на обслуживание с какой-нибудь фирмой, занимающейся подобными вопросами. У нас ОСББ и мы заключили подобный договор, и у каждого квартиросъемщика есть телефоны авариек.
\end{itemize} % }

\iusr{Светлана Соколова}

Игорь, ты в следующий раз позвони, я тебе дам телефон врача. Ну их те
добробуты-борисы. Назначат кучу ненужных анализов и обдерут, как липку.

\begin{itemize} % {
\iusr{Игорь Лесев}
Я в добробуте только рентген сделал. 30 минут консультации терапевта у них 540 гривен. А я ведь и дня еще не был народным депутатом. Рано еще по добробутам консультироваться)

\iusr{Светлана Соколова}
Там рентген бесплатно) Правда, от тебя далеко. Но врач супер.

\iusr{Игорь Лесев}
\textbf{Светлана Соколова} ну, буду знать, спс

\iusr{Ирина Кузнецова}
\textbf{Светлана Соколова}, напишите уж всем, хоть больницу, а лучше фамилию и координаты хорошего терапевта. Сейчас эти данные будут передаваться тщательнее, чем когда то запрещённый Булгаков
\end{itemize} % }

\iusr{Anna Vladimirovna}
Игорь, ищо раз скажешь "врачиха"...))

\begin{itemize} % {
\iusr{Игорь Лесев}
я перепишу весь текст, пусть будет виртуальный врач из личного кабинета

\iusr{Anna Vladimirovna}
"врачиха" - это я, так понимаю, самка врача?)) давай я буду врач, просто врач.

\iusr{Игорь Лесев}
сестричка - это девушка врача, а врачиха - это девушка, но не мужчинка с предположительно медицинским образованием

\iusr{Anna Vladimirovna}
ну хоть предположил образование и на том спасибо))

\iusr{Игорь Лесев}
\textbf{Anna Vladimirovna} это обобщенно, красавица

\iusr{Игорь Лесев}
о, мое любимое у Тарантино, мне такая сеня в попу колола, только чуть толще и без повязки

\iusr{Anna Vladimirovna}
ну, все. тогда я за тебя спокойна.

\iusr{Ирина Кузнецова}
Врачиня. Думаю, скоро введут в сучасный украинский.))

\iusr{Василий Стоякин}
А я сразу узнал  @igg{fbicon.face.savoring.food} 

\iusr{Александр Вееруга}
аж слышу мелодию))

\end{itemize} % }

\iusr{Sveta Suhova}
искренне сопереживаю.

\iusr{Таня Кушнир}

Все так и есть. Прямо один в один мой прошлогодний опыт. Договор 40 минут,
бабушка терапевт, "рентген сами где-то сделайте и принесите". Анализы лучше
тоже не у нас - у нас "УСИ позвильнялыся" и талончики на две недели вперёд.
Антибиотики по рецепту, который уже заранее напечатан. Потом в аптеке
оказывается, что есть только аналоги и они в четыре раза дороже.. На дом не
приходят - перезванивают или "лучше вы к нам". Хочешь уколы - берешь денюжку и
мимо регистратуры отлавливаешь медсестру и договариваешься. В платных можно
остаться не просто больным как был, но больным и нищим. Потому что там никто
тебе просто так не скажет диагноз, даже если вообще очевидно. Там надо
минимальный чек выполнять:((( к рентгену 20 анализов, две консультации, потом
ещё на мрт для гарантии. Потом только тот диагноз скажут, с каким ты сам
изначально и пришёл:((( ужас конечно:(

\iusr{Евгений Отовчиц}

Ну, собственно, кругом так. Вот, к примеру, написано заявление в полицию, о
возможной угрозе похищения ребенка. С пятницы оно летает из отделения в
отделение, назначаются номера, оно теряется, находится, снова теряется, дважды
проходит через одно отделение и путешествует дальше.

Пять дней оно не может найти ответственное лицо, которое возьмется хоть что-то
делать. За это время можно уже выкрасть, вывезти заграницу, попилить на органы
и привезти обратно. А у вас как дела? А опять на Герцена поехало...

Хорошие реформы. Пономаря мне в бок.

Будьте здоровы, Игорь.

\begin{itemize} % {
\iusr{Игорь Лесев}
Спасибо. О походе в полицию я где-то год назад писал. Чудесные воспоминания.
\end{itemize} % }

\iusr{Алла Фидельская}

Сегодня на 112, слушала Богомолец, которая рассказала, что Супрун работает на
кредиторов страны и её оснавная задача, сократить расходы на медицину. Экономия
и развал- вот наша рэхворма, вот почему Вова не убирает Улю, он в курсе её
делишек, но обязательства, как цепи, а он всё брэшэ и брэшэ! Обидно, что народ
поддерживает этих упырей и не исключено- приведёт их к власти снова, по крайней
мере, они на это расчитывают, и верят:-(

\begin{itemize} % {
\iusr{Игорь Лесев}

лет через 10-15 они будут голосовать сами за себя, вместе со своими женами,
любовницами, охранниками, водителями, судьями, прокурорами, ментами и
журналистами. Остальных не останется


\iusr{Алла Фидельская}

Они свалят, для того и грабят страну. Посему" спасение утопающих- дело самих
утопающих"! Страной правят запродавци, у ник индульгенция на грабёж и
беззакония, ещё и приплачивают кураторы. "Мы лишние, на этом празднике жизни",
Зимбабве маячит на горизонте.

\iusr{Игорь Лесев}
Все так. Но в Зимбабве все равно кто-то живет. Плохо, но не долго(с)))
\end{itemize} % }

\iusr{Сергей Коломиец}

В Феофании сестры не знают, как пользоваться банальным катетером. Если бы был
порт, они б в обморок упали, наверное...

Мне, конечно, грех жаловаться, в мое положение они вошли. И катетер был
мелочью. Но факт.

И там ничего не могут, но жалуются, что до них обращались в Лисод, блин.

\begin{itemize} % {
\iusr{Игорь Лесев}
шож вы иллюзии так безжалостно разбиваете?

\iusr{Сергей Коломиец}
\textbf{Игорь Лесев} дык. Реальная жизнь она такая, все разбивает.
Или вы всерьез считаете, что верхушка лечится в Феофании?)

\iusr{Ксения Гурбич}
Так, а если они с ними не работают, то откуда же им знать как ними пользоваться.

\iusr{Сергей Коломиец}
\textbf{Ксения Гурбич} учиться, не?

\iusr{Егор Верюжский}
\textbf{Сергей Коломиец} учиться? тю!
\end{itemize} % }

\iusr{Иван Славинский}
Поправляйтесь скорее, Игорь!

\iusr{Игорь Лесев}
спасибо

\iusr{Таня Ткаченко}
А чем отличается антибиотик от трахеита и от бронхита?

\begin{itemize} % {
\iusr{Игорь Лесев}
я на философском факультете учился, а не на медицинском... как мне объяснили, силой воздействия

\iusr{Таня Ткаченко}
\textbf{Игорь Лесев} так не бывает. Выздоравливайте.

\iusr{Игорь Лесев}
не готов с вами спорить, потому что не в теме. Спасибо

\iusr{Anna Vladimirovna}
\textbf{Таня Ткаченко}, ради общего развития гляньте классификацию антибиотиков, назначения, применение и все такое. Антибиотики, назначаемые при трахеите или бронхите легкой степени отличаются от тех, что назначают при бронхите тяжелой степени, который грозил перейти в пневмонию.

\iusr{Sveta Suhova}
\textbf{Игорь Лесев} "не тяну". но на тёлочку повёлся. Вы вы мужики - козлы. Человек, который зарабатывает ртом, обычно в постели = лох

\iusr{Игорь Лесев}
\textbf{Sveta Suhova} куй пойми о чем коммент, но, сука, смешно

\iusr{Sveta Suhova}
\textbf{Игорь Лесев} проще быть говном, чем попытаться мозг включить

\iusr{Игорь Лесев}
\textbf{Sveta Suhova} Света, давай еще шота напиши))

\iusr{Таня Ткаченко}
\textbf{Sveta Suhova} 

извините за поздний ответ. Товарищ уже наверное и забыл, что болел. А мы тут
пишем. У меня с развитием нормально. Это Вы изучите. Я еще раз говорю не бывает
сильных и слабых антибиотиков. Например представитель макролидов суммамед
пьется 3 дня, продолжая циркулировать в крови еще10 дней. И назначают его и при
трахеите и при бронхите и при пневмонии.


\iusr{Таня Ткаченко}
Ой Света это не Вам, а Аннн Владимировне ответ.

\iusr{Anna Vladimirovna}

ну ок, я упростила, скажу "сложнее" - разница есть в поколениях, спектре
действия, восприимчивости, есть антибиотики "стартовые" и есть антибиотики
"замены". а есть еще большущая разница в фирмах-производителях.

\iusr{Anna Vladimirovna}
и макролид в таблетках от цефалоспоринов в уколах при лечении пневмонии сильно отличается)).

\iusr{Игорь Лесев}
\textbf{Anna Vladimirovna} и шо, ни слова о говне и "в постеле лох"? такие антибиотики нам не нужны

\iusr{Anna Vladimirovna}
\textbf{Игорь Лесев} да, там они нам точно не нужны.

\end{itemize} % }

\iusr{Ирина Касьянова}
жЭсть(((

\end{itemize} % }
