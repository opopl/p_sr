% vim: keymap=russian-jcukenwin
%%beginhead 
 
%%file 04_09_2021.fb.karevin_aleksandr.1.kuleba_moskovia
%%parent 04_09_2021
 
%%url https://www.facebook.com/permalink.php?story_fbid=2999326116973355&id=100006879888184
 
%%author_id karevin_aleksandr
%%date 
 
%%tags istoria,kuleba_dmitrii,moskovia,rossia,rusmir
%%title НЕ БУДЬТЕ КУЛЕБАМИ!
 
%%endhead 
 
\subsection{НЕ БУДЬТЕ КУЛЕБАМИ!}
\label{sec:04_09_2021.fb.karevin_aleksandr.1.kuleba_moskovia}
 
\Purl{https://www.facebook.com/permalink.php?story_fbid=2999326116973355&id=100006879888184}
\ifcmt
 author_begin
   author_id karevin_aleksandr
 author_end
\fi

НЕ БУДЬТЕ КУЛЕБАМИ!

\index{Московия}
\index{Кулеба, Дмитрий! Украина! Министр иностранных дел}

Неоднократно уже писал о названии «Московия», но, видимо, стоит повторить,
поскольку высокопоставленные (не по уму) украинские должностные лица никак не
оставляют эту тему. Последним по времени отличился глава укроМИД  Дмытро
Кулеба. Пару дней назад он заявил, что до 18-го века Россия «имела одно
название – Московия». Ну и дальше пошло заурядное бла-бла-бла об «украденной» у
Украины истории.

\ifcmt
  pic https://scontent-frt3-1.xx.fbcdn.net/v/t1.6435-9/238718464_2999323476973619_6699503548196118897_n.jpg?_nc_cat=108&_nc_rgb565=1&ccb=1-5&_nc_sid=730e14&_nc_ohc=QkzQTC0Sa_0AX-sDGak&_nc_ht=scontent-frt3-1.xx&oh=c58ad04c92393531e304f6e441424600&oe=615AB14B
  width 0.4
\fi

В том, что это чушь, может убедиться любой, кто возьмёт на себя труд
ознакомиться с первоисточниками. Но я сейчас о другом.

Обзывая Россию «Московией», Кулеба просто хотел её унизить. Вот и напомнил об
этом названии, которое, наряду с названием Русь или Руссия, и в самом деле
часто встречалось раньше в иностранных источниках. Проблема, однако, в том,
что Московией в прошлом называли не только территорию нынешней РФ, но и
территорию теперешней Украины.

Например, турецкий путешественник Эвлия Челеби (17 век) сообщает о «пятидесяти
белоликих московских красавицах», захваченных крымскими татарами у гор
«крепости Киев в Московской земле» и подаренных ханом турецкому сановнику
Мелеку Ахмед-паше. Также, рассказывая о Киеве, Челеби называет его древней
крепостью и говорит, что «со времени своего основания и по сей день эта
крепость находится в руках московитов».

Антонио Поссевино (16 век), дипломат, находившийся на службе у римского папы,
пишет, что Руссия приняла христианскую веру «500 лет назад, при московитском
князе Владимире».

На распространённость «московского языка» во владениях великого князя
Литовского (нынешние Украина и Белоруссия) указывает литовский дипломат и
писатель 16-го века Михалон Литвин.

Журнал «Голландский Меркурий» в марте 1656 года опубликовал статью о городе
Львове (Лемберге), где утверждается, что живут в этом городе поляки, евреи,
армяне и московиты.

Как Московия обозначена территория Левобережной Украины на изданной в Голландии
в 1593 году карте  Герарда де Иоде. А в атласе фламандского картографа Герарда
Меркатора (год издания 1628) Московией названа территория к северу от Чёрного
моря (см. фото).

Конечно, некорректно называть Владимира Крестителя московитским князем, а
жителей Киева или Львова – московитами. Но ошибка тогдашних авторов вполне
объяснима. Они воспринимали население Северо-Восточной (Московской) и
Юго-Западной (Польско-Литовской) Руси как один народ.

А это и есть один народ.

Ну и кого унизил укроминистр?

Не будьте кулебами! Знайте историю своей страны.

\ii{04_09_2021.fb.karevin_aleksandr.1.kuleba_moskovia.cmt}
