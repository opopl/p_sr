% vim: keymap=russian-jcukenwin
%%beginhead 
 
%%file 25_10_2021.fb.bilchenko_evgenia.1.russkij_podvig_kanyshev_aleksej.cmt
%%parent 25_10_2021.fb.bilchenko_evgenia.1.russkij_podvig_kanyshev_aleksej
 
%%url 
 
%%author_id 
%%date 
 
%%tags 
%%title 
 
%%endhead 
\subsubsection{Коментарі}

\begin{itemize} % {
\iusr{Eu May}
до мурашек...

\iusr{Елена Швец}
Проникновенная в самое сердце и душу история...

\iusr{Сергей Долинский}
Духовный подвиг.....

\begin{itemize} % {
\iusr{Евгения Бильченко}
\textbf{Сергей Долинский} Да, это высшая метафизика свободы как всеединства. Наш ответ их либерализму.

\iusr{Сергей Долинский}
\textbf{Евгения Бильченко} то, что Вы рискнули своей жизнью и благополучием, отстаивая свои убеждения и духовную позицию, не меньший подвиг. Не все выдерживают такие испытания... Удачи Вам!
\end{itemize} % }

\iusr{Тетьяна Стадниченко}
А я верю в искренность Его.

\iusr{Михаил Иванович Вэй}

И что? Кому-то стало легче? От такого чудачества чудес не случается. Нам уже с
колен пора подняться, а мы снова и снова встаём на колени и цепляемся за
средневековые догмы

\begin{itemize} % {
\iusr{Евгения Бильченко}
\textbf{Михаил Иванович Вэй} мне стало. Ибо поднявшиеся с колен перед Богом встали на колени перед дьяволом, имя коему легион.

\iusr{Сергей Долинский}
\textbf{Михаил Иванович Вэй} это не "средневековые догмы"...это истины...вне времени...
\end{itemize} % }

\iusr{Игорь Красиков}

\obeycr
Увы, так и вспоминается...
-Вот,- сказал демиург Мазукта.- Это твоя доля в будущем мире.
Человек повертел свою долю так и сяк, покачал на ладони и поднял на демиурга недоумевающий взгляд.
-А почему так мало?
-Это я у тебя должен спросить, почему так мало,- парировал демиург.- Плохо старался, наверное.
-Я страдал!- с достоинством заявил человек.
-А с каких это пор,- удивился Мазукта,- страдания стали считаться заслугой?
-Я носил власяницу и вервие,- упрямо нахмурился человек.- Вкушал отруби и сухой горох, не пил ничего, кроме воды, не притрагивался к женщинам и мальчикам - хотя, сам знаешь, иногда очень хотелось. Я изнурял своё тело постом и молитвами...
-Ну и что?- перебил Мазукта.- Я понимаю, что ты страдал - но за что именно ты страдал?
-Во славу твою,- не раздумывая ответил человек.
-Хорошенькая же у меня получается слава!- возмутился Мазукта.- Я, значит, морю людей голодом, заставляю носить всякую рвань и лишаю радостей секса?
-Вообще-то, да,- тихо заметил сидящий в сторонке демиург Шамбамбукли.
-Не мешай,- отмахнулся Мазукта.- Тут другая ситуация, в этом мире я играю Доброго Дядю Небесного.
-А, понятно,- кивнул Шамбамбукли и замолчал.
-Так что с моей долей?- напомнил о себе человек.
Мазукта задумчиво почесал за ухом.
-Да как бы тебе объяснить, чтобы понял... Вот, например, плотник. Он строит дом, и тоже иногда попадает себе по пальцам, через это страдает. Но он всё-таки строит дом. И потом получает свою честно заработанную плату. Ты же всю жизнь только и делал, что долбил себе молотком по пальцу. А где же дом? Дом где, я спрашиваю?
\restorecr

\url{https://bormor.livejournal.com/455722.html}

\iusr{Serge Staricoff}

Эта история жизни перекликается с историей героя поэмы Николая Мельникова
"Русский крест", автора из деревни моего дедушки, села Лысые, Брянской области.

\end{itemize} % }
