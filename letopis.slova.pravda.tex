% vim: keymap=russian-jcukenwin
%%beginhead 
 
%%file slova.pravda
%%parent slova
 
%%url 
 
%%author 
%%author_id 
%%author_url 
 
%%tags 
%%title 
 
%%endhead 
\chapter{Правда}
\label{sec:slova.pravda}

Який тільки народ не адаптував сюжет стародавньої казки про відомих українцям
братів \emph{Правду} і Кривду. Проте найкраще, здається, це вдалося зробити в
англосаксонському світі, де цю моралізаторську казку назвали «Засліплення
правди фальшивістю». І тут йдеться не тільки про маніпуляції та перебріхування.
Мова йде також про тих, хто облудою і фальшем забирає в інших здатність бачити,
чути і розуміти. Хоча у відомій казці є безліч незрозумілих для сучасної людини
поведінкових практик, а сюжет містить багато спрощень, основна мораль зрозуміла
– \emph{неправдою} щасливим не станеш. Але цього разу нам важливо зосередитися на
тому, що одному з братів за допомогою цинічних маніпуляцій вдалося виманити у
брата його коржа. Ще цікавішим є те, чому брат \emph{Правда} погодився на те, щоб його
скалічили до напівживого стану,
\textbf{Правда і Кривда}, Василь Расевич, zaxid.net, 04.06.2021

Может, хватить улучшать - и будем просто жить?  Я вот удивляюсь - почему, когда
власть берется что-то улучшить, граждан охватывает смятение? Почему все
улучшения, по итогу заканчиваются Ухудшением? Вспомните недавние «реформы». От
медицинской до судебной!  Может, прав индейский вождь: «Когда белый человек
нашёл нашу землю, на ней правили индейцы. Не было налогов, не было долгов,
буйволы и бобры были в изобилии, вода была чиста. Женщины делали домашнюю
работу, лекарь не брал денег. Мужчины тратили весь день на охоту, а всю ночь на
секс». Потом вождь улыбнулся и добавил: «Только белый человек может быть
настолько глупым, чтобы думать, что такую систему можно улучшить».  А и
\emph{правда}, может хватит «улучшать жизнь и преодолевать эпоху бедности»?,
\citTitle{Почему все наши реформы, от медицинской до судебной, заканчиваются ухудшением?}, Константин Стогний, strana.ua, 10.06.2021

%%%cit
%%%cit_head
Россия - Нация-Девчонка
%%%cit_pic
\ifcmt
	pic https://avatars.mds.yandex.net/get-zen_doc/3989805/pub_60c48c27b582164697cb2369_60c49e6e72119574b73e6391/scale_1200
  caption Девчонки с флагами. Россия вперёд!
\fi
%%%cit_text
Как сказал кто-то из западных правителей (Бисмарк, что ли, не важно):
\enquote{Россия - это нация ртуть. Её рвёшь на куски, а она, временно
рассыпавшись, снова неумолимо собирается в одно целое, и никто, и ничто этому
помешать не сможет никогда}. Да, это \emph{Правда}. У России множество
феноменов. Главный - это язык.  Все россияне (все и везде) говорят на
одинаковом языке, без всяких каких-то серьёзных отступлений, как в других
европейских и иных странах - все россияне говорят на одном чистом,
литературном русском языке. Это феномен
%%%cit_comment
%%%cit_title
\citTitle{Слава России! Мы на вершине могущества и спускаться более не намерены}, 
Алексей Наст, zen.yandex.ru, 12.06.2021
%%%endcit

%%%cit
%%%cit_head
%%%cit_pic
%%%cit_text
Но, как бы россияне и русские не пытались приобрести европейские черты лица, и
стать чуть-чуть исторически старше, \emph{правда} все равно лезет наружу, разрывая
наспех сметанные исторические симулякры.  Кто же, они, русские?  Знаете, что в
СССР была милая шовинистическая поговорка про русских: \enquote{Ты почему русский?
Потому, что у тебя глаз узкий!}. Но именно она открывала всю \emph{правду} \enquote{русских}
%%%cit_comment
%%%cit_title
\citTitle{Страна без исторических корней всегда будет оккупантом}, 
Олена Степова, news.obozrevatel.com, 17.06.2021
%%%endcit


%%%cit
%%%cit_head
%%%cit_pic
%%%cit_text
Десь після 1016 року той же Ярослав (тоді ще геть не мудрий, а радше
братовбивця) видав свій судебник, \enquote{найдавнішу \emph{правду}} – щось на зразок
найпростішого кримінального кодексу зі збереженням кровної помсти та рабства.
Років за 50 вона була доповнена \enquote{\emph{правдою} Ярославичів}, яка закріпила поділ
еліти і плебсу. Разом із \enquote{Поконом вирним} і \enquote{Уроком мостникам} все це склало
коротку редакцію \enquote{\emph{Руської Правди}} – найважливішого документа тих часів. Пізніше
до розширеної редакції увійшов \enquote{Закон судний людям}.  Але порівняно з присягою
громадян Херсонеса, державотворчий процес на Русі – це все ж крок назад
%%%cit_comment
%%%cit_title
\citTitle{Українській Конституції – 2300 років}, 
Сергій Громенко, gazeta.ua, 26.06.2021
%%%endcit

%%%cit
%%%cit_head
%%%cit_pic
%%%cit_text
При этом, когда в страну завезли новую партию \enquote{АстраЗенеки}, МОЗ
внезапно запретило ее колоть людям младше 60 лет. То есть сделало то же, что и
в Европе, где сочли эту вакцину опасной для более молодых людей.  Это говорит о
том, что с одной стороны дыма без огня нет. С другой - вряд ли власти скажут
\emph{правду}, если даже установят связь между вакциной и летальным исходом.  И
тут можно лишь повторить уже звучавшую выше рекомендацию: людям с
сердечно-сосудистыми заболеваниями или аллергией (еще один фактор риска) лучше
провериться у семейного врача накануне вакцинации и уже тогда принимать решение
- прививаться или нет
%%%cit_comment
%%%cit_title
\citTitle{Pfizer - первая смерть в Украине после вакцинации. Что об этом известно}, 
Максим Минин, strana.ua, 03.07.2021
%%%endcit

%%%cit
%%%cit_head
%%%cit_pic
%%%cit_text
У нашому тривалому протистоянні радикальних патріотів і прибічників ідей СРСР,
що відчувають чи то страх, чи то відразу до легендарного Степана Бандери,
\emph{правда} і сила будуть на боці того, хто зможе наважитися на щирість
Німеччини – винести минуле на поверхню і \enquote{звільнити пацієнта, щоб він
міг жити новим успішним життям} 
%%%cit_comment
%%%cit_title
\citTitle{Українці не розуміють одне одного не через мову, а через небажання слухати, чути і сприймати}, 
Юлія Мендель, www.pravda.com.ua, 07.07.2021
%%%endcit

%%%cit
%%%cit_head
%%%cit_pic
%%%cit_text
В то же время данное дело имеет исключительно важное значение. Без
\emph{правды} о событиях последней декады февраля 2014 года мы будем обречены
жить в условиях войны, в состоянии глубокого раскола общества, в ситуации,
когда каждый следующий день будут приносить новости все хуже и хуже. Так длится
уже много лет и будет продолжаться до тех пор, пока понятия «право» и «правда»
не станут в Украине тождественными.  Все эти годы власти прилагали усилия,
чтобы в сердце каждого украинца поселились два мифа о Президенте Януковиче.
Первое, что якобы он отдал приказ о расстреле майдановцев. На эту тему уже
опубликовано множество расследований журналистов и профессиональных экспертов
из США, Италии, Израиля и Канады. Указанные расследования, а также сами
пострадавшие свидетели-майдановцы говорят в один голос: и протестующих и
правоохранителей расстреливали из здания подконтрольной оппозиции гостиницы
«Украина». На это намекал Парасюк, об этом прямо рассказывал Бубенчик.    Миф о
«преступном приказе» давно опровергнут, поэтому остановимся на втором, согласно
которому якобы из-за обращения Президента Януковича к Российской Федерации
Украина потеряла Крым.  Убедиться в нежизнеспособности этого мифа украинцы
могли во время просмотра многолетнего тв-шоу под названием «процесс в
Оболонском суде». Я же как непосредственная участница событий февраля 2014 года
хочу рассказать, что произошло на самом деле
%%%cit_comment
%%%cit_title
\citTitle{Неудобная правда о событиях 2014 года / Лента соцсетей / Страна}, 
Анна Герман, strana.news, 05.11.2021
%%%endcit
