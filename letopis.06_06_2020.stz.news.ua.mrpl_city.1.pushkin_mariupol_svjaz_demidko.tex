% vim: keymap=russian-jcukenwin
%%beginhead 
 
%%file 06_06_2020.stz.news.ua.mrpl_city.1.pushkin_mariupol_svjaz_demidko
%%parent 06_06_2020
 
%%url https://mrpl.city/news/view/svyaz-velikogo-poe-ta-s-mariupolem
 
%%author_id news.ua.mrpl_city,onegina_olena.mariupol
%%date 
 
%%tags 
%%title Ко дню рождения великого писателя Александра Пушкина: его связь с Мариуполем (ФОТО)
 
%%endhead 
 
\subsection{Ко дню рождения великого писателя Александра Пушкина: его связь с Мариуполем (ФОТО)}
\label{sec:06_06_2020.stz.news.ua.mrpl_city.1.pushkin_mariupol_svjaz_demidko}
 
\Purl{https://mrpl.city/news/view/svyaz-velikogo-poe-ta-s-mariupolem}
\ifcmt
 author_begin
   author_id news.ua.mrpl_city,onegina_olena.mariupol
 author_end
\fi

\ifcmt
  ig https://i2.paste.pics/be9b30a0b3147b3f9fa836edeec52354.png
  @wrap center
  @width 0.9
\fi

Традиционно с 1997 года, 6 июня отмечается Пушкинский день. Исследователи
творчества Александра Сергеевича Пушкина называют сотни достоверных адресов
пребывания его в Украине. Историк \href{\urlDemidkoIA}{Ольга Демидко} обнаружила уникальную связь
великого поэта с Мариуполем.

Поэт побывал в 124 населенных пунктах. Это и города, села, хутора, дома на
переправах через реки. Все адреса связанные с пребыванием поэта на службе
(неофициальной ссылке). Имя великого гения, бесспорно, имеет прочную духовную
связь и с Мариуполем.

Южная ссылка Александра Сергеевича до сих пор вызывает массу вопросов и
дискуссий. Мариупольские краеведы и историки разошлись во мнении, был ли поэт в
Мариуполе, видел ли Азовское море... В частности, краеведы Л. Яруцкий, А.
Проценко, А. Гроссов считали, что поэт в Мариуполе все же останавливался.
Сегодня, в День рождения одного из самых авторитетных литературных деятелей
первой трети XIX века, основоположника современного русского языка, предлагаю
вернуться в Мариуполь на 200 лет назад, в 1820 год, когда Александр Сергеевич
мог проезжать или останавливаться в нашем городе. Тогда на Базарной  площади
(сейчас – площадь Освобождения) кипела жизнь. Толпы мариупольских купцов и
мещан не подозревали, что летним вечером на их почтовой станции мог
остановиться величаший поэт и писатель... 

\ii{06_06_2020.stz.news.ua.mrpl_city.1.pushkin_mariupol_svjaz_demidko.pic.1}

Базарная площадь на тот момент была главным административным и торговым центром
города. С гостинного двора на площади открывался великолепный вид на устье реки
Кальмиус, море и рейд для стоянки судов.  

Начнем с начала. В известном письме  генерала Раевского к дочери Е. Н. Раевской
(которое неоднократно приводилось в публикациях Аркадия Дмитриевича Проценко)
от 13 июня – 6 июля 1820 года, где он подробно описывает свое путешествие с
семьей на Кавказ (а с ними вместе ехал и Пушкин), указаны только две даты:
выезд из Киева и прибытие в Смелу. Таким образом, в этом письме даты выезда из
Екатеринослава (сейчас – Днепр) нет. Известный пушкинист М. Цявловский считает
датой выезда из Екатеринослава – 28 мая (по старому стилю). Но мариупольский
кравед А. Проценко  опровергает эту версию. Сложно с ним не согласиться. Ведь
тогда путешественники должны были проехать 311 верст от Екатеринослава до
\enquote{первой почты за Мариуполем} почти за одни сутки. Приведем отрывок из книги М.
Цявловского \enquote{Летопись жизни и творчества А. С. Пушкина} \enquote{Май. 28. С разрешения
Инзова отъезд утром Пушкина с Раевскими из Екатеринослава}. \enquote{Май. 29. Пушкин с
Раевскими проезжает через Мариуполь. На \emph{почтовой станции Безыменная}, \enquote{на
первой почте за Мариуполем} их угощает завтраком жена управляющего феодосийской
в Широкингоскладочной таможни П. В. Гаевского}. Если предположить, что были
периоды, когда лошади шли шагом, ведь Мариупольский тракт был нелегким для
лошадей, что в свою очередь, тоже снижало скорость, можно утверждать, что к
первой почте за Мариуполем путешественники доехали позже.

В своем письме генерал Раевский достаточно подробно описывает путешествие на
Кавказ, но, к сожалению, в нем нет сведений о том, где они ночевали перед
приездом на \enquote{первую почту за Мариуполем}. И все же, как отмечает, А. Проценко,
в письме есть любопытные данные, которые косвенно подтверждают предположение о
том, что Александр Сергеевич не просто проезжал мимо, а останавливался вместе
со своими спутниками в Мариуполе на ночлег. Интересно, что генерал подробно
описывает город у моря. Но, как допускали многие исследователи, эти данные
Раевский мог получить от третьих лиц. Самым убедительным доказательством в его
письме может служить цена на хлеб на мариупольском базаре: \enquote{... а хлеб, то есть
пшеница, и в теперешнее дешевое время продается до 16 рублей}. Зачем человеку,
просто проезжавшему через город, описывать такие подробности? Думаю, это было
связано с непосредственными впечатленими генерала  о городе.

\ii{06_06_2020.stz.news.ua.mrpl_city.1.pushkin_mariupol_svjaz_demidko.pic.2}

Осталось разобраться с дальнейшим маршрутом великого поэта. Приведя данные из
многих архивных документов, Лев Яруцкий проследив южный маршрут Пушкина, связал
с приазовскими берегами некоторые его произведения, в частности строфу ХХХІІІ
первой главы \enquote{Евгения Онегина} – \enquote{Я помню море пред грозою...}, \enquote{У лукоморья дуб
зеленый...} из \enquote{Руслана и Людмилы}... Но известный краевед тоже мог ошибаться. У
некоторых исследователей можно найти сведения, что наш герой проезжал Мариуполь
– Безыменное – Кирпичево (Широкино) – Новоазовское Лукоморье... Так ли это?
Несколько месяцев назад хорошие друзья предложили мне попытаться разобраться в
маршруте Пушкина, приэтом помогли найти очень редкие карты. Особенно благодарна
мариупольскому активисту Леониду Рыдкину. У него есть своя версия, исходя из
которой Пушкин не был ни в Широкино, ни в Безымянном. И море он видеть не
мог... Только степь. Приазовская легенда о происхождении названия села Безыменное
общеизвестна. Она повествует о том, что А. С. Пушкин, проезжая через это село в
конце мая 1820 года, якобы, в разговоре со старожилами села, экспромтом
придумал ему название. Многолетние споры среди краеведов по поводу легенды не
иссякают. 

\textbf{Читайте также:} \href{https://mrpl.city/news/view/pandemiya-v-istorii-chem-boleli-mariupoltsy-bolee-sta-let-nazadputi-zarazheniya-i-sposoby-vyzdorovleniya}{%
пандемия в истории: чем болели мариупольцы более ста лет назад – пути заражения и способы выздоровления, %
Олександра Невська, mrpl.city, 22.05.2020%
}

Сторонники с. Безыменное пытаются подтвердить свою версию о том, что Пушкин
проезжал через это село, ссылкой на легенду. Но \enquote{Генеральные карты
Екатеринославской губернии с показанием почтовых и больших проезжих дорог,
станций и расстояний между оными верст за 1818 и 1821 годы} подтверждают, что
Таганрогский тракт не проходил на близком расстоянии от берега моря и поэтому
он не мог проезжать через это село. Споры не утихают и по сей день, но какая
красивая легенда... Первой станцией за Мариуполем нужно назвать почту Широкую,
затем следовали мелкие станции Грузко-Еланчинская, Мокро-Еланчинская, Носова.
Следующая остановка Коровьебродская, дальше – Таганрог. Вот отрезок пути, по
которому двигался Пушкин с семьей генерала Раевского в конце мая 1820 года. Но
это означает, что село Широкино и Безыменное находились в стороне от его пути,
и, следовательно, поэт не проезжал через эти села. Возможно, легенда ведет
рассказ о \enquote{первой почтовой станции за Мариуполем}. Но со временем почтовая
станция потеряла свое значение, а село с одноименным названием продолжало жить.
Благодаря картам 1818 и 1821 можно понять, где же находилась первая почтовая
станция за Мариуполем.  Станция Широкая находилась  в 17 верстах от предыдущей,
верховьях Широкой балки, а это в районе чуть дальше Пикуз (Коминтерново),
точнее, между Пикузами и Заиченко. Это видно из карт. 17 верст, это немного
больше  18 км. Если от Широкой балки отложить 18 км, попадаем не в центр
Мариуполя, а именно на переправу через Кальмиус. От центра к переправе около 6
км. Если откладывать 17 верст от центра, почта должна была бы располагаться
между поворотом на Талаковку и Водино, а там никакой Широкой балки нет. Почты
устанавливались в балках из-за воды, необходимой лошадям. Все предположения
исследователей о размещении почтовой станции в Широкино и Безыменном
безосновательны. Первая почтовая станци за Мариуполем находилась в пределах
современного Мариуполя – в районе современного Коммунальника, подтверждением
чего явлются карты. Да, море Пушкин у нас не видел, так как после завтрака на
Коммунальнике он мог лицезреть лишь бескранюю приазовскую степь.

И несмотря на то, что Александр Сергеевич не видел Азовского моря и долго не
задерживался в нашем городе, его связь с историей Мариуполя неразрывна. В нашем
городе есть улица, названная в его честь, памятник ему посвященный. В
центральной библиотеке хранится томик стихов поэта

\ii{06_06_2020.stz.news.ua.mrpl_city.1.pushkin_mariupol_svjaz_demidko.pic.3}

выпуска 1887 года. Заслуженный художник Украины, житель Мариуполя Виктор
Пономарев написал картину с изображением Пушкина \enquote{Я помню море перед грозою}, а
мариупольские поэты: Анатолий Стрегло, Сергей Алымов, Николай Новоселов, Иван
Битюков посвятили свои строки великому гению. Интересно, что в 1989 году
Мариуполь посетил 75-летний правнук поэта \textbf{\emph{Григорий Горигорьевич
Пушкин}}. \emph{И сколько бы не прошло дестилетий, уверенна, что мариупольцы
разных возрастов будут чтить память великого поэта в своих сердцах...}
