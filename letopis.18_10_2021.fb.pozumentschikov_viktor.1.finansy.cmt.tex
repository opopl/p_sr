% vim: keymap=russian-jcukenwin
%%beginhead 
 
%%file 18_10_2021.fb.pozumentschikov_viktor.1.finansy.cmt
%%parent 18_10_2021.fb.pozumentschikov_viktor.1.finansy
 
%%url 
 
%%author_id 
%%date 
 
%%tags 
%%title 
 
%%endhead 
\subsubsection{Коментарі}

\begin{itemize} % {
\iusr{Игаль Иерусалимский}
Согласен. Все вокруг печатают, аж гай шумит!

\iusr{Alexandr Poslavski}
Только для этого нужно быть "субъектом" !

\iusr{Sergey Stegno}
ВК, я рад что мы вместе, в мыслях и духе!  @igg{fbicon.grin} 

\begin{itemize} % {
\iusr{Виктор Позументщиков}
\textbf{Sergey Stegno} это пока первый кирпичик @igg{fbicon.face.smiling.sunglasses}  Начнём новую тему - новый финансовый порядок Украины @igg{fbicon.face.smiling.sunglasses} 

\iusr{Leonid Tieriekhov}

Никогда не думал, что печатание новых денег могут приносить пользу экономике, а
вот инфляцию это подстёгивает. Инфляцию мы уже не раз переживали - и что - это
как то решило наши проблемы в экономике? Кроме обесценивания вкладов граждан в
нац. валюте и роста цен - это ни к чему не приводило. Люди и так не сильно
доверяют банковской системе- а после подобных эксцессов у многих пропадает
мотивация зарабатывать деньги в национальной валюте- и тем более её
капитализировать. Другое дело получать иностранные инвестиции- и под них уже
печатать новые деньги- то тогда может будет в этом какой то смысл. Благо, что
есть МВФ -который даёт умные рекомендации в подобных вопросах- а иначе -вот
такой вот "треш" как напечатать, а затем изъять "лишнюю" наличность из оборота
- кто то бы применил из новой власти - что усугубило бы и так низкий уровень
доходов. А чо - есть дефицит бюджета - то давайте напечатаем новых денег и
раздадим людям "фантики" в виде зарплат и пенсий. Это моё вот такое мнение как
обывателя и неспециалиста в области экономики и финансов.

\iusr{Sergey Stegno}
\textbf{Leonid Tieriekhov} 

инфляция - суть есть надувание. Надувание экономики денежной массой, и как
следствие надувание самой экономики, увеличение количества экономической
деятельности как следствие через финансовое стимулирование и профицит денег.
Надувание в хорошем смысле. Другое дело, когда появляется кризис смыслов и
технологий и реальный сектор не может догнать денежную массу. Но у нас такого
нет, т.к. мы догоняем, а не лидируем в общемировом зачете и копировать чужой
опыт можно ещё долго, как ресурс развития нашего сообщества.

Более того, есть способы лишнюю денежную массу изъять из оборота и сжечь после
достижения поставленных целей.

\iusr{Leonid Tieriekhov}
\textbf{Сергей Стегно} - какие это механизмы изъятия лишних денег? Если появится определённая масса денег для функционала системы - то что будет причиной снова уменьшить наличность в стране? Если к примеру - сегодня доллар стоит около 30грн.- то печатание новы... Ещё

\iusr{Игорь Нагорный}
Чем меньше денег у народа, тем выше замки правителей.

\iusr{Лилия Дрыкина}

У нас история с деньгами, как у Винни Пуха с медом:как только они есть, так сразу
их нет! Подозреваю, что при этой власти нас не спасет и эмиссия. Слишком большие
аппетиты у некоторых. Дай бог, чтобы я ошибалась!

\end{itemize} % }

\end{itemize} % }
