% vim: keymap=russian-jcukenwin
%%beginhead 
 
%%file 12_01_2022.fb.fb_group.story_kiev_ua.1.kiev_restorany_sssr.pic.4.cmt
%%parent 12_01_2022.fb.fb_group.story_kiev_ua.1.kiev_restorany_sssr
 
%%url 
 
%%author_id 
%%date 
 
%%tags 
%%title 
 
%%endhead 

\iusr{Людмила Романова}
Вот это да! Такое сохранить! @igg{fbicon.thumb.up.yellow} 

\iusr{Людмила Билык}

Вот бы нынешняя молодёжь удивилась бы: цены то - в копейках!!!
@igg{fbicon.thumb.up.yellow}  @igg{fbicon.grin} 

\iusr{Ilya Gutnik}
\textbf{Людмила Билык} 

А как это относительно зарплат в то время? Особенно для низко оплачиваемых с
детьми? Многие могли позволить себе рестораны? Я не говорю о разных событиях,
отмечаемых в ресторане.

\iusr{Natali Pritula}
\textbf{Ilya Gutnik} мы студентами при стипендии в 40 руб. частенько зависали в рестиках, чего не скажешь о нынешних временах((((

\iusr{Ilya Gutnik}
\textbf{Natali Pritula} Вам родители помогали материально, поэтому вы могли себе это позволить.

\iusr{Людмила Билык}
\textbf{Ilya Gutnik} 

Так мы ж и не ходили каждый день...  @igg{fbicon.grin} ... Первый раз на выпускной - мне было 17
лет, а потом уже, когда работала с зарплатой 90 рублей, могла позволить себе на
10 рублей погулять в праздник...  @igg{fbicon.wink}  

\iusr{Людмила Билык}
\textbf{Ilya Gutnik} 

Илья, а Вы вспомните, сколько стоила коммуналка?.. А каждый год \enquote{домики} и
пионерский лагеря?.. Папа и мама работали на заводах и путёвки им давали для
отдыха с семьёй за символическую плату... Например: пионер лагерь, кстати с
прекрасным питанием, смена (28 дней) - 10 рублей, а база отдыха: домик на 4х
человек с трехразовым питанием на 14 дней - 25-30 рублей... Так что, не всё так
уж и плохо было...
