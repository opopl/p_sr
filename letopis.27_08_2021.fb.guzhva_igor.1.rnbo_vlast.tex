% vim: keymap=russian-jcukenwin
%%beginhead 
 
%%file 27_08_2021.fb.guzhva_igor.1.rnbo_vlast
%%parent 27_08_2021
 
%%url https://www.facebook.com/veprwork/posts/4377168322342011
 
%%author Гужва, Игорь
%%author_id guzhva_igor
%%author_url 
 
%%tags guzhva_igor,rnbo,sankcii,ukraina,vlast
%%title С февраля этого года СНБО стал ключевым органом власти в Украине
 
%%endhead 
 
\subsection{С февраля этого года СНБО стал ключевым органом власти в Украине}
\label{sec:27_08_2021.fb.guzhva_igor.1.rnbo_vlast}
 
\Purl{https://www.facebook.com/veprwork/posts/4377168322342011}
\ifcmt
 author_begin
   author_id guzhva_igor
 author_end
\fi

С февраля этого года СНБО стал ключевым органом власти в Украине.

Почти в еженедельном режиме он принимает решения, которые ломают судьбы людей,
вводит санкции против граждан Украины, лишая их конституционных прав, без суда
и следствия закрывает СМИ (в том числе и "Страну"), отбирает предприятия.

Причем все эти решения принимаются в закрытом режиме, группой из 23 человек.
Подавляющее большинство из которых - это назначенцы или самого президента, или
подконтрольного ему парламента.

Грубо говоря - это для президента абсолютно ручной орган, который принимает
любые решения, нужные Зеленскому.

При этом общественность не имеет никакого представления - как готовятся и
принимаются эти решения, как проходит обсуждение (и проходит ли оно вообще),
какая представляется доказательная база под введение санкций?

Svetlana Kryukova поговорила с рядом украинских топ-чиновников и выяснила
механизм введения санкций СНБО.

Самые интересные цитаты:

"Внешне все выглядит иногда просто как эмоциональные припадки президента
Зеленского, но это не совсем так. Фактор личного отношения президента к тем или
иным вопросам или личностям имеет, безусловно, значение. Но главное в другом.
Главное - стратегия. Санкции СНБО - это инструмент создания режима личной
власти президента и концентрации в руках его окружения контроля над потоками.
Плюс - расправа с теми, кто этому мешает. Отдельный вопрос - санкции против
СМИ. Здесь задача совершенно очевидная - зачистка информационного поля от
оппозиционной повестки и запугивание этим примером других медиа, чтоб они себя
вели смирно. Ну и конечно же пиар на том, как президент круто работает и одним
махом решает разные проблемы.

Также, большое влияние имеет коррупционный фактор.

"Типичный пример - санкции по контрабандистам. На Банковой начали качать эту
тему заранее, провели совещание по теме борьбы с контрабандой, на котором
приняли решение вводить санкции. С одной стороны это соответствовало стратегии
взятия под контроль окружения президента контрабандных потоков и отсечения от
них конкурентов. С другой стороны - это пиар. Но была и третья сторона -
вымогательство. Изначально в список внесли более 30 человек. Но санкции по
итогу были введены против 10. И потом еще против нескольких. Остальные -
порешали. И работают в тесной связке с теми, на кого им указали из Киева".

"Папку с документами члены СНБО видят в последний момент. Им раздают макулатуру
уже на заседании. Они лежат на рабочем столе. Члены СНБО могут полистать,
почитать, но при выходе папки они оставляют на том же месте. Сами бумаги - это
даже не фабула каких-то уголовных дел, как их обычно представляют
правоохранительные органы. Это больше напоминает статьи с сайтов-сливных бочков
по компромату против тех или иных лиц. Никакой доказательной базы не
приводится. Просто предлагается поверить на слово". 

При этом даже таких сомнительных бумаг по многим участникам санкционного списка
нет.

Например, никаких документов и материалов не было представлено по санкциям
против главного редактора "Страны" Игоря Гужвы и нашего издания.

"Это было почти в самом конце заседания, перечислили какие-то компании, потом
назвали фамилию Игоря Гужвы, назвали невнятную мотивацию что-то вроде
"информационная безопасность" и вынесли на голосование. Многие так и не поняли,
за что проголосовали. Аргументов или объяснений, почему было принято такое
решение, никто не услышал, как и не увидел этого в раздаточных материалах. Там
в этих документах вообще не было ни одного листа бумаги с аргументами, почему
вопрос Гужвы вынесли на СНБО. Разумков возразил, что-то в своём стиле, мол, а
какая причина такого решения, насколько оно законно, но по сути ответа не
получил. Баканов лишь сказал, что по законности всех решений дают заключения
специалисты Офиса президента. Что само по себе интересно. Хотя какая может быть
законность, если санкции применяют к гражданину Украины Игорю Гужве,
признанному в Европе политическим беженцем?".

Полностью читайте статью по ссылке в первом комментарии.
