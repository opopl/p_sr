% vim: keymap=russian-jcukenwin
%%beginhead 
 
%%file 16_12_2022.fb.rojz_svitlana.kyiv.2.dity_krasyvi
%%parent 16_12_2022
 
%%url https://www.facebook.com/svetlanaroyz/posts/pfbid02YoWytTBN9CZ9rUg3DNmF9jtN2CtBmMopsm2b4JgaVwrnBEHdpQuFqH7GEf7Y8ArXl
 
%%author_id rojz_svitlana.kyiv
%%date 
 
%%tags 
%%title Сьогодні весь день звертаю увагу, які красиві у нас діти!
 
%%endhead 
 
\subsection{Сьогодні весь день звертаю увагу, які красиві у нас діти!}
\label{sec:16_12_2022.fb.rojz_svitlana.kyiv.2.dity_krasyvi}
 
\Purl{https://www.facebook.com/svetlanaroyz/posts/pfbid02YoWytTBN9CZ9rUg3DNmF9jtN2CtBmMopsm2b4JgaVwrnBEHdpQuFqH7GEf7Y8ArXl}
\ifcmt
 author_begin
   author_id rojz_svitlana.kyiv
 author_end
\fi

Сьогодні відвели доньку в школу, при ній я трималась, а коли залишилась сама -
вже трясло трохи. Сьогодні було гучніше і масованіше, ніж зазвичай. 

Після мобілізації в стресі іде відкат. В такому стані відчуваєш себе дуже
вразливо. І на будь-яке недоречне чи грубе втручання вже може не вистачити
почуття гумору. Вони відчуваються, як загроза, а важливо ж відновити відчуття
безпеки. Найкраще, що можна зробити,коли хтось поруч переживає стрес - нічого
не радити, не навантажувати, не шукати винних - а, можливо, сказати - я з
тобою. Ми разом, ми впораємося. 

Після стресу - нам потрібно чи в кубельце, чи відчути чиїсь обійми - реальні та
віртуальні. Чи розповісти комусь про себе, поплакати, чи в фізичну активність,
чи в роботі зробити щось, що б не занадто навантажувало. 

Мені потрібно було в магазин - нагадую собі, що дитині ж важливо їсти. Знайома
касирка (я про неї розповідала, вона схожа на фею чи ельфа) запитала: а де ваші
вуха? (пам'ятаєте, я ділилася якось, що ходила в магазин в ельфійських вухах).
А я ж гальмую. Почала серйозно торкатись своїх вух - а що з ними? 🙂 Вона з
розумінням  сказала - складний ранок. сьогодні збираю дитину в школу, сирена. А
мені на роботу. Прийшлось залишити саму вдома... Йшла від каси, думала, як їй
вистачає сил посміхатись покупцям і жартувати,  поки її дитина сама вдома. 

І тут - написала мама однокласника доньки - вони зараз в іншій країні - діти
підготували нашій малій листа та подарунки. І я розплакалася. Уявила, як зрадіє
донька. І якою це буде для неї важливою підтримкою. 

А в НП на мене чекав подарунок від колеги - і я знов попливла - від розчулення.
Напевно, цей турботливий "окситоцин" - як обійми - дає можливість
розморозитись, відігрітись. 

А поруч з НП велика гірка. Йшов дивний зимовий дощ. А на гірці  10-15 дітей в
різнокольорових куртках та комбінезонах, з санчатами та підпопниками (як
назвати ті круглі штуки, на яких з'їзжають з горок?) сміються, кричать, та як у
всі часи - радіють зимі. 

Уявіть. Пройшла година після обстрілів. Гірка. Дитячій сміх. 

Сьогодні весь день звертаю увагу, які красиві у нас діти! 

Обіймаю, Родино ❤️ так сильно, як вірю у всіх нас, в силу наших дітей,  в ЗСУ і
Перемогу.

\ii{16_12_2022.fb.rojz_svitlana.kyiv.2.dity_krasyvi.orig}
\ii{16_12_2022.fb.rojz_svitlana.kyiv.2.dity_krasyvi.cmtx}
