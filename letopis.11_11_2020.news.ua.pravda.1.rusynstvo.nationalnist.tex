% vim: keymap=russian-jcukenwin
%%beginhead 
 
%%file 11_11_2020.news.ua.pravda.1.rusynstvo.nationalnist
%%parent 11_11_2020.news.ua.pravda.1.rusynstvo
 
%%url 
%%author 
%%tags 
%%title 
 
%%endhead 

\subsubsection{\enquote{З національністю не народжуються\ \textemdash\ з національністю лише
помирають}}

Ми п’ємо каву неподалік колишньої синагоги, побудованою у мавританському стилі \dshM у 1947 році її перетворили на філармонію.

У сквері поруч\ \textemdash\ погруддя Євгена Фенцика, русина-москвофіла, засновника
\enquote{Товариства імені Духновича}.

\ifcmt
img_begin 
        url https://img.pravda.com/images/doc/b/7/b73f2b2-dscf1063.jpg
        caption Погруддя Євгена Фенцика, засновника \enquote{Товариства імені Духновича}. Саме завдяки Фенцику сьогоднішні закарпатські русини так люблять російський триколор.
        width 0.7
img_end
\fi

Це погруддя, встановлене тут ще у 1926 році, за останні роки неодноразово було
облито чорною або червоною фарбами. Зараз його в черговий раз відчистили.

Біля Фенцика голосно розмовляють літні чоловіки: \enquote{Так просрати гарне місто
Ужгород можуть тільки галичани та наші місцеві бандюки, що називають себе
бізнесменами, прокурорами, суддями та взагалі міською владою}.

Згадувати \enquote{незлим тихим словом} галичан\ \textemdash\ то давня закарпатська традиція.

Мармуровий Фенцик робить вигляд, що нічого не чує. Саме ця здібність вдавати з
себе нечуючого допомогла погруддю пережити всі режими, які пройшли Ужгородом.

На завзятого українофоба, а саме в цьому звинувачували Фенича одразу після
виходу його книги, пан Володимир, не схожий\ \textemdash\ занадто елегантний та привітний.

Але хто знає, як має виглядати справжній українофоб?

До 1993 року Володимир Фенич, як й усі народжені у СРСР, не надто переймався
власною етнічною ідентичністю. Навіть його мати-угорка була записана у
радянському паспорті як українка. 

Але у 1993-му Фенич знайомиться з книгою Пола Магочія \enquote{Формування національної
самосвідомості: Підкарпатська Русь}. Саме з цієї книги він дізнається, що на
Закарпатті до 1946 року жили русини.

Після цього Володимир Фенич повернувся до ідентичності своїх предків по
батьківській лінії. Що не заважає йому бути патріотом України. Але після виходу
книги \enquote{Угорська Русь} його наукове бачення історії зненацька виявилось
\enquote{сепаратистським}.

Про що була ця книга? Про те, що Михайло Драгоманов був першим, хто вважав, що
русини Угорщини\ \textemdash\ то братерський народ, і цей народ обов’язково треба залучати
до українського руху.

Начебто нічого крамольного. Але деякі українські історики були настільки не
згодні з окремими тезами цієї книги, що замість того, щоб відкрити наукову
дискусію, писали доноси в СБУ з проханням перевірити книгу на прояви
екстремізму та сепаратизму.

Звісно, ніхто нічого не знайшов, але після цього інциденту Володимир Фенич
змушений був покинути університет.

\ifcmt
img_begin 
	url https://img.pravda.com/images/doc/4/6/46af240-dscf1163.jpg
	caption Книга Володимира Фенича \enquote{Угорська Русь} пройшла перевірку щодо проявів екстремізму та сепаратизму. А скоро в нього вийде нова книга\ \textemdash\ біографія Пола Магочія.
    width 0.7
img_end
\fi

Я питаю його, як взагалі сталося, що русинський дискурс у нашій країні почав
сприйматись як геополітична спекуляція Кремля?

І чи можна інакше сприймати етнічну громаду, чиї лідери їздять до окупованого
Криму та супроводжують Земана під час його візиту до Москви? Або закликають
Путіна \enquote{активно освоювати Закарпаття} та проголошують віртуальну республіку
\enquote{Підкарпатська Русь}? Або хваляться тим, що формують у так званій \enquote{ЛНР}
русинські бойові загони, які воюють проти української армії (Петро Гецко)?

Адже з такими лідерами русинам не потрібні вороги\ \textemdash\ вони самі собі найгірший
ворог. \dshM Навколо русинського питання завжди було багато спекуляцій та маніпуляцій, \dshM каже Володимир Фенич.\ \textemdash\ Чому, наприклад, ви відразу прирівнюєте русинськи
товариства, що намагаються зберегти мовне різноманіття та традиції, і всіляких
Джуганів та Тяско? В кожного народу є такі Джугани.

В нас ще є й Світова Рада Русинів, почесним головою якого залишається Павло
Роберт Магочій, завідувач кафедри україністики університету Торонто. В 2008
році він зробив заяву про відмежування цієї інституції від будь-яких форм
екстремізму в русинському русі.

Але ви, говорячи про русинів, згадуєте не Магочія, а Сидора та Гецка, які
ніякими лідерами тутешніх русинів ніколи не були! Це афільовані з Росієї особи,
не треба забувати, що саме комуністична Москва у 1946 році в адміністративному
порядку замінила підкарпатським русинам національність у паспортах на
українську.

В словах пана Фенича є сенс. Коли мова йдеться про русинів, останнім часом
згадують саме тих, \enquote{афільованих з Росією}. \dshM Чомусь українські націоналісти або ангажовані \enquote{експерти з історії} вважають,
що від русинів може надходити якась геополітична загроза або національна
небезпека,\ \textemdash\ знизує плечима Володимир Фенич.

Згідно з останніми переписами населення, русинів зараз 106 тисяч у 12 країнах
світу. Як вони можуть бути загрозою для когось? Але коли в українських медіа
заходить мова про русинів, чомусь завжди у контексті політичних фанаберій
Росії, Угорщини або Чехії.

Я не кажу, що ці держави не намагаються використати русинську карту у своїх
інтересах\ \textemdash\ це безумовно так. Але чому тоді Україні не зіграти на випередження,
вирішити русинське питання самим і у такий спосіб вибити цей \enquote{козир} з рук
усіх, хто грає на цьому питанні?

Визнавши русинів, Україна одразу нівелює усі дії Москви на Закарпатті, забуває
про усіх Тяско та Джуганів та спостерігає, як ОПЗЖ поступово втрачає будь-який
інтерес до отця Сидора. Логічно?

До кав’ярні, де ми спілкуємося, заходять хлопець з дівчиною у яскраво-рожевому
пуховику. Напружено прислухаюся до їхньої розмови. Чи не прибічники отця
Сидора?

\enquote{Як ти йому сказав?}\ \textemdash\ питає вона хлопця. \enquote{Фрас боо тя узяв!}\ \textemdash\ відповідає він.
\enquote{А що це означає?}. \enquote{Те ж саме, що й у вас у Львові, коли ви кажете }шляк би
тебе трафив\enquote{\ \textemdash\ каже він і замовляє два тістечка.

Здається, на сепаратистів не схожі. \dshM Людина з національністю не народжується\ \textemdash\ продовжує Володимир Фенич після
паузи.\ \textemdash\ Вона народжується з громадянством. Національність вона може тільки
набути. З національністю можна лише померти. Тому що це\ \textemdash\ свідомі, набуті у
зрілому віці, речі.

Українське громадянство не означає, що я українець за етнічним походженням. На
Закарпатті окрім українців та русинів проживає багато угорців, румунів,
словаків, ромів, поляків... І у демократичній державі усі ці національності
повинні мати такі самі права, що і представники більшого етносу.

А коли держава не опікується своїми етносами, чому вона потім дивується, що
ними зацікавились сусіди і використовують їх заради політичних спекуляцій?

