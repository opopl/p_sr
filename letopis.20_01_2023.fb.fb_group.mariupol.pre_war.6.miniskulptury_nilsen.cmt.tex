% vim: keymap=russian-jcukenwin
%%beginhead 
 
%%file 20_01_2023.fb.fb_group.mariupol.pre_war.6.miniskulptury_nilsen.cmt
%%parent 20_01_2023.fb.fb_group.mariupol.pre_war.6.miniskulptury_nilsen
 
%%url 
 
%%author_id 
%%date 
 
%%tags 
%%title 
 
%%endhead 

\qqSecCmt

\iusr{Олеся Зыкова}

Он был возле башни красиво было!

\begin{itemize} % {
\iusr{Natalia Tkachenko}
\textbf{Олеся Зыкова} и сейчас там стоит...
\end{itemize} % }

\iusr{Людмила Глушенкова}

Сколько же я времени потратила, пока искала все эти скульптурки. Сейчас это
приятные воспоминания! 🤗

Все нашла. Одного Нильсена, самого загадочного, помогли найти. Он был во
внутреннем дворике ДК \enquote{Молодежный}. Ворота были закрыты и увидеть его
получилось только издалека. Но я пыталась...

Даже рассмешила ребят работавших рядом в \enquote{ритуальные услуги}, предложив им
разрешить мне вылезть через их окно, чтобы его увидеть поближе ! 😀

Ребята юмор оценили, но между двумя зданиями свой забор.

Но какое чудесное время было!

\iusr{Людмила Глушенкова}

А остальные 6 скульптур : пеликан, парусник, мамонт, кузнец, козачка и осётр -
это дубль два в поисках сокровищ! 😍🤩

Скульптурки были расставлены в разных районах города. Это был квест! Чтобы их
найти нужно было поискать и даже поехать в дальний район, в парк \enquote{Веселка} и
напротив новый ЦНАП. Дааа, хорошее время.

Огромное спасибо тем,кто придумал эту затею. Я одна из тех,кто в свои выходные носился по городу и искал их.

Последнюю статуэтку Осётр нашла за месяц до войны. 🫣

Зашла в Вежа и говорю ребятам : \enquote{понимаете, у меня мечта, я хочу их всех найти и остался один...}

Ребята вынесли \enquote{Осетра}, он должен был стоять на новом пирсе.

Ну какое же счастье было у меня!!!

Я их всех нашла! 🥰

\ifcmt
  igc https://scontent-frt3-2.xx.fbcdn.net/v/t39.30808-6/326462723_1120663721937034_6410350465757616611_n.jpg?_nc_cat=110&ccb=1-7&_nc_sid=dbeb18&_nc_ohc=vVQ-AtUOeOAAX8En7Yh&_nc_ht=scontent-frt3-2.xx&oh=00_AfDh9A38sT4qNLhvCy4NmUAQ6eWvgdIwKIXu2THdPUhNDQ&oe=643D265F
	@width 0.4
\fi

\iusr{Ирина Павленко}

Пам'ятаю, ми із моїми учнями їх відшукували по місту!
