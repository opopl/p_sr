% vim: keymap=russian-jcukenwin
%%beginhead 
 
%%file 12_08_2022.stz.news.ua.donbas24.1.azovstal_fortecja_mrpl_narys_metalurg_kombinat.3
%%parent 12_08_2022.stz.news.ua.donbas24.1.azovstal_fortecja_mrpl_narys_metalurg_kombinat
 
%%url 
 
%%author_id 
%%date 
 
%%tags 
%%title 
 
%%endhead 

\subsubsection{\enquote{Азовсталь} і Маріуполь}

Розбудова відомого заводу безпосередньо відображалася в істо\hyp{}рії Маріуполя та
зведенні в місті спортивних, культурних, житлових об'єктів.

\textbf{\color{blue} Яхт-клуб \enquote{Азовсталь}.} Він з'явився на мапі міста у 1959 році за ініціативою
тодішнього директора \enquote{Азовсталі} Володимира Лепорського. Згодом яхт-клуб став
міським водноспортивним комплексом, де до початку повномасштабного вторгнення
розвивався парусний спорт, веслування на ялах, вейкбордінг, а територія
колишнього яхт-клубу стала майданчиком для проведення міських заходів. І це ще
не все, адже найпопулярніший пірс східного узбережжя знаходився саме в цій
локації. До тимчасової окупації міста російськими військами це було справжнє
місце сили для маріупольців і гостей міста!

\textbf{\color{blue} Центр сучасного мистецтва \enquote{Готель Континенталь}.} 

Саме тут з 1933 року розміщувався Клуб металургів заводу \enquote{Азовсталь}. У роки
окупації Маріуполя німцями будівля була зруйнована, після відновлення в ній
почав працювати Палац культури і техніки металургійного комбінату \enquote{Азовсталь},
який із 2010 року став Палацем культури \enquote{Молодіжний}. На жаль, циклічність
історії призвела до руйнування визначної будівлі знову — рашисти зруйнували
\enquote{Готель Континенталь} у березні 2022 року. Зараз архітектурна пам’ятка
Маріуполя перетворена на руїни.

% 13 - Фото: Анатолій Кінащук.
\ii{12_08_2022.stz.news.ua.donbas24.1.azovstal_fortecja_mrpl_narys_metalurg_kombinat.pic.13}
% 14 - Фото: Анатолій Кінащук.
\ii{12_08_2022.stz.news.ua.donbas24.1.azovstal_fortecja_mrpl_narys_metalurg_kombinat.pic.14}
% 15 - Фото: Анатолій Кінащук.
\ii{12_08_2022.stz.news.ua.donbas24.1.azovstal_fortecja_mrpl_narys_metalurg_kombinat.pic.15}

\textbf{\color{blue} Маріупольська камерна філармонія.} Будівля була зведена у 1958 році і мала назву
Палац культури коксохімічного заводу. Для дітей працівників комбінату
\enquote{Азовсталь} це місце свого часу було пов'язане зі спогадами про новорічні свята
і хороводи. Продовжуючи аналогію з сьогоденням, варто підкреслити, що під час
обстрілів російською армією Маріуполя в філармонії знаходили прихисток тисячі
містян — культурний заклад працював як бомбосховище. А після захоплення міста
рашисти \href{https://donbas24.news/news/u-mariupolskii-filarmoniyi-vstanovili-klitki-rosiyani-planuyut-samosud-nad-ukrayinskimi-viiskovimi-foto}{подейкують},%
\footnote{У маріупольській філармонії встановили клітки: росіяни планують самосуд над українськими військовими, Олена Онєгіна, donbas24.news, 06.08.2022, \par\url{https://donbas24.news/news/u-mariupolskii-filarmoniyi-vstanovili-klitki-rosiyani-planuyut-samosud-nad-ukrayinskimi-viiskovimi-foto}} що саме в цій будівлі вони проводитимуть показові судилища над українськими захисниками.

\textbf{\color{blue} Кооператив \enquote{Азовсталець}.} У 80-х роках на Лівобережжі\par\noindent з'явився 
\href{https://archive.org/details/video.12_06_2019.alevtina_shvecova.kooperativ_azovstalec_mrpl}{гаражний кооператив}%
\footnote{Відео: Кооператив \enquote{Азовсталець} МАРІУПОЛЬ, Алевтина Швецова, 12.06.2019, %
\par\url{https://www.youtube.com/watch?v=-Q4mskcIgYE}, \par%
Internet Archive: \url{https://archive.org/details/video.12_06_2019.alevtina_shvecova.kooperativ_azovstalec_mrpl}%
}
біля моря. Працівники заводу отримували тут ділянки, де здебільшого
розташовували гаражі не з метою зберігання автівок, а для відпочинку —
риболовля, човники, дозвілля на Азовському узбережжі.

\ifcmt
  ig https://i2.paste.pics/PT1MH.png?trs=1142e84a8812893e619f828af22a1d084584f26ffb97dd2bb11c85495ee994c5
  @wrap center
  @width 0.8
\fi

\ifcmt
  ig https://i2.paste.pics/PT1MQ.png?trs=1142e84a8812893e619f828af22a1d084584f26ffb97dd2bb11c85495ee994c5
  @wrap center
  @width 0.8
\fi

\ifcmt
  ig https://i2.paste.pics/PT1N8.png?trs=1142e84a8812893e619f828af22a1d084584f26ffb97dd2bb11c85495ee994c5
  @wrap center
  @width 0.8
\fi

Одною з головних міських артерій Лівобережжя є \textbf{\color{blue} вулиця Азовстальська.} Які
об'єкти можна тут побачити? Починається Азовстальська вулиця зі Східних
прохідних комбінату \enquote{Азовсталь}, не\hyp{}подалік розташований стадіон \enquote{Азовсталь}, ще
тут є Центр культури \enquote{Лівобережний}, навчальні заклади (школа № 10 і
школа-інтернат № 2), ранковий ринок \enquote{Привоз} і пам'ятник будівельникам. До 24
лютого 2022 року це був звичайний житловий район, після — суцільний жах і
руйнування. Мешканців на цій вулиці майже не лишилося.

% 16 - Вулиця Азовстальська.
\ii{12_08_2022.stz.news.ua.donbas24.1.azovstal_fortecja_mrpl_narys_metalurg_kombinat.pic.16}
% 17 - Вулиця Азовстальська.
\ii{12_08_2022.stz.news.ua.donbas24.1.azovstal_fortecja_mrpl_narys_metalurg_kombinat.pic.17}

