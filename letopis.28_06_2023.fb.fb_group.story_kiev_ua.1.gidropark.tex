%%beginhead 
 
%%file 28_06_2023.fb.fb_group.story_kiev_ua.1.gidropark
%%parent 28_06_2023
 
%%url https://www.facebook.com/groups/story.kiev.ua/posts/2225410137655775
 
%%author_id fb_group.story_kiev_ua,sirota_tatjana.kiev
%%date 28_06_2023
 
%%tags 
%%title Гидропарк
 
%%endhead 

\subsection{Гидропарк}
\label{sec:28_06_2023.fb.fb_group.story_kiev_ua.1.gidropark}
 
\Purl{https://www.facebook.com/groups/story.kiev.ua/posts/2225410137655775}
\ifcmt
 author_begin
   author_id fb_group.story_kiev_ua,sirota_tatjana.kiev
 author_end
\fi

\enquote{А почему бы нам не поехать в Гидропарк, ведь мы так давно там не были ?}-
спросила я младшенького.

В ответ прозвучало громкое \enquote{УРА!}

Это означало, что моё предложение получило полное одобрение.

И вот наша дружная компания, состоящая из двух Людмил и меня с Владом,
отправляется на целый день наслаждаться красотой любимого многими киевлянами
парка.

Погода, конечно, не пляжная, но для длительной прогулки на природе, как раз то, что
надо.

Приехав в Гидропарк, мы сначала отправились в парк-музей под открытым
небом, который называется \enquote{Україна в мініатюрі}.

Небольшой дождик, который решил нас чуть-чуть попугать, совсем не уменьшил
удовольствия от посещения этого музея. Здесь выставлены макеты исторических и
архитектурных памятников Киева, а с 2012 года - всей Украины.

Потом были Долобецкий остров, где мы отдохнули на берегу Десенки (Чертороя), и
пляж \enquote{Венеция} с самой крутой в мире \enquote{качалкой} и живописное озеро
\enquote{Березка}, что на Венецианском острове, и прогулка по лесным дорожкам к пляжу
\enquote{Золотой} и обратно, но уже вдоль берега Днепра.

Находили около десяти километров. Но... конечно, с перерывами на отдых и
\enquote{перекус}.

Надышались речным воздухом, смешанным с запахом леса. Устали...

Завтра будем отдыхать.

Гидропарк в понедельник.

26 июня 2023 года.
