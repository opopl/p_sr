% vim: keymap=russian-jcukenwin
%%beginhead 
 
%%file 18_07_2020.fb.lnr.10
%%parent 18_07_2020
 
%%endhead 
\subsection{После испытаний американской вакцины от COVID-19 умерли четыре бойца ВСУ  }
\label{sec:18_07_2020.fb.lnr.10}
\url{https://www.facebook.com/groups/LNRGUMO/permalink/2856095471168675/}
  
\vspace{0.5cm}
{\small\LaTeX~section: \verb|18_07_2020.fb.lnr.10| project: \verb|letopis| rootid: \verb|p_saintrussia|}
\vspace{0.5cm}

\index{Коронавирус!Вакцина}
\index{ВСУ}
\index{Смерть}

Трагедией завершились испытания вакцины от коронавируса, которые проводят в
Харьковской области американские вирусологи. С таким заявлением выступил в
пятницу на брифинге представитель пресс-службы управления Народной милиции
самопровозаглашенной ЛНР Александр Мазейкин.

По его словам, для испытаний инновационного лекарства и вакцины были отобраны
15 кандидатов, в том числе 10 бойцов ВСУ, заболевших коронавирусом. В ходе
исследований, у восьми пациентов резуко ухудшилось состояние здоровье и их
перевели в реанимацию. Трех человек подключили к аппаратам ИВЛ, передает ЛИЦ.

По словам Мазейкина, врачи неотложной помощи не знали о составе американского
препарата, поэтому не могли оказать пациентам необходимой помощи.

"В результате пятеро пациентов, в том числе четверо военнослужащих скончались
от осложнений, вызванных инфекцией", - утверждает представитель ЛНР.

Он также отметил, что по неподтвержденным данным экспериментальные вакцины
якобы использовали в медучреждениях Харьковской области, что вызвало осложнения
у пациентов со слабым иммунитетом. Также он привел данные без ссылок на
источник о том, что якобы свыше 110 привитых жителей региона были вынуждены
обратиться в медучреждения.
