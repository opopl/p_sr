% vim: keymap=russian-jcukenwin
%%beginhead 
 
%%file 26_06_2020.stz.news.ua.mrpl_city.1.5_maljovnychyh_misc_priazovja_vlitku
%%parent 26_06_2020
 
%%url https://mrpl.city/blogs/view/5-malovnichih-mists-priazovya-dlya-vidpochinku-vlitku
 
%%author_id demidko_olga.mariupol,news.ua.mrpl_city
%%date 
 
%%tags 
%%title 5 мальовничих місць Приазов'я для відпочинку влітку
 
%%endhead 
 
\subsection{5 мальовничих місць Приазов'я для відпочинку влітку}
\label{sec:26_06_2020.stz.news.ua.mrpl_city.1.5_maljovnychyh_misc_priazovja_vlitku}
 
\Purl{https://mrpl.city/blogs/view/5-malovnichih-mists-priazovya-dlya-vidpochinku-vlitku}
\ifcmt
 author_begin
   author_id demidko_olga.mariupol,news.ua.mrpl_city
 author_end
\fi

Є в українській мові одне унікальне слово \textbf{\enquote{галасвіта}}, що означає невідомо
куди, світ за очі. Після довгого карантину дуже хочеться вирушити галасвіта.
Хоча можна обрати більш легкий шлях – насолодитися краєвидами рідного краю,
який всі ми часто недооцінюємо. Я вирішила  скласти невеличкий список
приазовських місць, які дозволять сповна насолодитися літом, отримати заряд
здоров'я і незабутніх вражень. Водночас поїздка до цих місць не забере багато
часу і, на мій погляд, може заощадити кошти.

\ii{26_06_2020.stz.news.ua.mrpl_city.1.5_maljovnychyh_misc_priazovja_vlitku.pic.1}

Перше місце, куди раджу всім відправитися здивує багатьох, адже більшість
маріупольців про нього навіть не чула. Але мальовничі краєвиди цього місця не
можуть залишити когось байдужим. Це \textbf{Знаменівське водосховище}, що розташоване
біля села Знаменівка Волноваського району. Це місце ідеально підійде для
любителів кемпінгу та прогулянок, як піших, так і на велосипедах. Особисто мене
приємно вразила чистота водосховища. До речі, інфраструктура теж налагоджена.
Встановлені альтанки, дерев'яні столики, мангали, сміттєві баки, що можуть
зробити відпочинок більш комфортним. Але найбільше вражають величезні і
пухнасті хмари, які відображаються на поверхні річки. Родзинкою цього місця є
старенький міст, який приваблює своїм колоритом. Виникає враження, що міст
тримається тільки завдяки воді. Таких мальовничих пейзажів, як в цьому
водосховищі, я давно не бачила. Хоча, як виявилося в селі давно приймають
туристів і для багатьох  водосховище є улюбленим  місцем для відпочинку,
рибальства, купання та відзначення родинних свят.

\ii{26_06_2020.stz.news.ua.mrpl_city.1.5_maljovnychyh_misc_priazovja_vlitku.pic.2}

Другим місцем в Приазов'ї, куди слід відправитися влітку, може стати \textbf{Мангушська
конеферма}. Зручне розташування дозволить зробити її першим пунктом перед
продовженням невеличкої подорожі. Переваг в цьому місці для відпочинку безліч.
Це і чисте повітря, і неповторна атмосфера, і потужна енергетика. Перебування
на конефермі не тільки піднімає настрій, а й відновлює здоров'я. Нині крім
виконання основних функцій (племінне розведення і тренінг скакових коней) на
території Мангушської конеферми проводять іпотерапію, лікувальну верхову їзду,
спортивні свята та змагання, а також екскурсії для всіх охочих. Це місце
відрізняється від інших можливістю побачити неймовірно розумних і сильних
істот, спілкування з якими приносить колосальне задоволення.

\ii{26_06_2020.stz.news.ua.mrpl_city.1.5_maljovnychyh_misc_priazovja_vlitku.pic.3}

Після Мангушської конеферми можна відправитися до \textbf{Федо\hyp{}рівського лісництва}, де
найкращим відпочинком можуть стати як велосипедна прогулянка, так і пікнік з
рідними та друзями. В цьому місці унікальним є поєднання лісових насаджень та
гірських рельєфів, а наявність річки робить відпочинок ще більш яскравим. В
лісництві можна знайти зручні альтанки, столи і лавочки, місця для кемпінгу та
приготування шашлику.

\ii{26_06_2020.stz.news.ua.mrpl_city.1.5_maljovnychyh_misc_priazovja_vlitku.pic.4}

Думаю, приазовський регіон міг би вразити багатьох туристів завдяки
\textbf{Старокримському водосховищу}. Найбільше вражають незабутні краєвиди і яскраві
панорами, що відкриваються перед очима. Живописна річка і зручні місця для
кемпінгу давно приваблюють багатьох відпочивальників. Але купа сміття, яку
можна побачити поруч з яскравими пейзажами дуже засмучує і водночас обурює. Це
може свідчити не тільки  про популярність місця а й про неготовність деяких
туристів відпочивати цивілізовано... 

\ii{26_06_2020.stz.news.ua.mrpl_city.1.5_maljovnychyh_misc_priazovja_vlitku.pic.5}

Наостанок хочу запропонувати відправитися до унікального  пам'ятника природи,
який залишається досі не дуже популярним. Це \textbf{\enquote{Соснові культури}} в Мангушському
районі. Соснова перлина Приазовського урочища розташована за кілька метрів до
берегової лінії, перемежовується з ділянками і окремими представниками листяних
порід, наймасовіші з яких представлені вербами і тополями уздовж пішохідних
доріжок, ясенями, білими акаціями, кленами, шипшиною, колосняком чорноморським,
безсмертником піщаним, катраном татарським та іншими видами рослин,
пристосованими виростати в умовах сухих пісків та  в близькості солоних вод. На
галявинах серед сосен  зростає цінна лікарська рослина солодка гола. Повітря в
цьому місці неймовірне чисте і м'яке. У 2009 році \enquote{Соснові культури} в складі
парку \enquote{Меотида} були зараховані до парків національного значення і увійшли до
складу природно-заповідного фонду України, який охороняється законом як
національна спадщина і є частиною світової системи природних територій та
об'єктів під особливим захистом.

Звісно, цими місцями багате Приазов'я не обмежується. Відправитися можна і до
села Кременівка чи Білосарайської коси, де багата природа і незабутні краєвиди
теж нікого не залишать байдужими. Думаю, що у кожного буде свій власний список,
куди поїхати влітку але сподіваюся, що мальовничі місця рідного краю
маріупольці теж встигнуть відвідати.
