% vim: keymap=russian-jcukenwin
%%beginhead 
 
%%file 17_11_2021.fb.fb_group.story_kiev_ua.1.kino
%%parent 17_11_2021
 
%%url https://www.facebook.com/groups/story.kiev.ua/posts/1799966806866779
 
%%author_id fb_group.story_kiev_ua,fedjko_vladimir.kiev
%%date 
 
%%tags 1974,kiev,kino
%%title Ну, хто не мріяв знятися в кіно!? Ну, хоча б у маленькому епізоді!
 
%%endhead 
 
\subsection{Ну, хто не мріяв знятися в кіно!? Ну, хоча б у маленькому епізоді!}
\label{sec:17_11_2021.fb.fb_group.story_kiev_ua.1.kino}
 
\Purl{https://www.facebook.com/groups/story.kiev.ua/posts/1799966806866779}
\ifcmt
 author_begin
   author_id fb_group.story_kiev_ua,fedjko_vladimir.kiev
 author_end
\fi

Ну, хто не мріяв знятися в кіно!? Ну, хоча б у маленькому епізоді! 

Ця історія трапилася зі мною в травні 1974 року...

В один із робочих днів, перед святом – Днем Перемоги, я поїхав на кіностудію
Довженка, щоб придбати рулон 35-мм кіноплівки (оператори продавали її по 10
коп. за метр).

Чудовий весняний день! Іду і насолоджуюся сонечком, повітрям, отримую позитивні
емоції від життя. До прохідної кіностудії залишається метрів п’ятдесят... Раптом
чую голосний жіночий голос: «Оце типаж! Ти мені потрібний!» І з цими словами до
мене кидається якась жінка років під п’ятдесят... Підбігає і хапає мене під руку!

«Психопатка! – думаю я, – мабуть переплутала мене з якимось актором!»

А вона, не даючи мені оговтатися, представляється заступником режисера
багатосерійного фільму «Генерал міліції розповідає» і каже, що у мене гарний
типаж – якраз грати бандита 1920-х років! Виявляється, що її підвели троє
людей, яких вона підібрала для зйомок в епізодах, і вона змушена терміново
ловити біля студії підходящих персонажів. Я виявився третім... Кіношниця діловито
пообіцяла, що одразу після зйомок видасть нам по 10 рублів гонорару!

Ну, хто не мріяв знятися в кіно!? Ну, хоча б у маленькому епізоді! Та ще й
отримати гонорар за роботу!

Помреж веде нас у кінознімальний павільйон, в костюмерну. Там нам підбирають
типажний одяг і ми переодягаємося. Потім вона веде нас на знімальну площадку і
пояснює нашу роль... Ми сидимо втрьох за столиком і пиячимо, імітуючи разом з
іншими персонажами, що сидять за сусідніми столиками, атмосферу
ресторану-притону... Бандити, сутенери, проститутки... По її команді, коли камера
буде дивитися на нас, один з нас розливає «самогон» по гранованих стаканах, ми
випиваємо, імітуючи, що п’ємо міцний самогон-первак, і закусюємо. Все.

А в цей час дівчатка-асистенти накривають столики. Кіношниця каже, що всі
продукти, окрім «самогону», справжні: дунайський оселедець з маринованою
цибулею, квашені огірки, свіжий хліб, свіжозварена картопля... На нашому столі
стоїть сулія літра на півтора, до половини наповнена якоюсь мутною рідиною.
Кіношниця говорить, щоб ми не хвилювалися – в сулії дистильована вода,
підфарбована декількома краплями молока. 

- Все зрозуміло!?

- Так! А через скільки зйомка?

- Через 20-25 хвилин.

Помреж іде інструктувати чоловіків і жінок за сусідніми столиками. 

Один з моїх товаришів по чарці, пропонує скинутися по два рублі і він швидко,
через паркан, збігає в «Гастроном» за справжньою пляшкою, оскільки «гріх
марнувати таку шикарну закуску»! Крім того, вираз обличчя повинен бути
натуральним, а з водою в склянці, добитися такого ефекту неможливо! Ми ж не
професійні актори! Даємо йому гроші і він мчить в лавку. Через 15 хвилин
повертається. По його вигляду розуміємо, що все в порядку! Він бере сулію і йде
в туалет. Через декілька хвилин повертається, підходить до дівчини-асистента і
щось їй говорить... Бачимо, як вона бере піпетку, набирає з пляшки трохи молока і
капає в нашу сулію. Антураж відновлено. Але в нашій сулії 750 грам справжньої
горілки!

[Пізніше, вже під час спілкування, я дізнався, що він дуже інтелігентна людина,
кандидат технічних наук, працює в одному з науково-дослідних інститутів.]

У мене з собою був фотоапарат. Пропоную сфотографуватися «на пам’ять», поки
тверезі, і прошу дівчину-асистента натиснути на спуск...  

Буквально через декілька хвилин помреж через мегафон просить всіх учасників
зйомки зайняти свої місця за столиками. Сідаємо... По сигналу помрежа накладаємо
на тарілки закуску... По другому сигналу наш тамада розливає по повному стакану
горілки (200 грам), ми цокаємося, випиваємо і енергійно закусюємо! Закуски
багато... Тамада розливає залишки горілки, ми тихенько допиваємо і доїдаємо
харчі. Все дійсно було свіже і смачне! 

Помреж оголошує, що на сьогодні зйомка закінчена, запрошує отримати «червонець»
і розписатися у відомості. Ми з задоволенням отримуємо гонорар і розписуємося. 

Помреж записує наші телефони і каже, що введе наші дані в свою картотеку, щоб
запрошувати на зйомки. Ми їй сподобалися!

Обмінюємося телефонами з товаришами по чарці... Я обіцяю подарувати наше фото і
прощаємося. 

***

Коли я повернувся до інституту, в якому працював, то до кінця робочого дня
залишалося ще з півгодини. Я вирішив, що на роботу мені іти немає сенсу – по
моєму обличчю і по запаху можна було одразу дізнатися, що я випивши. А за появу
на роботі у нетверезому вигляді можна нарватися на серйозні неприємності! 

Знайшов зручне місце неподалік від входу, щоб бачити, коли вийде Надійка, моя
наречена... Вона йде, задумавшись про щось своє... Я метрів п’ятдесят йду за нею,
милуючись її стрункою фігурою і красивими ніжками. Потім доганяю... Йдемо разом і
я розповідаю їй про пригоду з кіно. 

Ми приїжджаємо на площу Шевченка і я пропоную попрощатися біля будинку, не
піднімаючись на нашу площадку – не варто, щоб мене могли побачити у нетверезому
стані її батьки.

Підпис до фото:

На фото я з лівого краю. Наш тамада - в центрі.

На землі, біля знімального павільйону, імітація снігу. Там теж знімалися якісь
епізоди серій цього фільму (всього у фільмі було 10 серій).

В прокаті фільм отримав назву «Рождённая революцией. Комиссар милиции
рассказывает».

\ii{17_11_2021.fb.fb_group.story_kiev_ua.1.kino.cmt}
