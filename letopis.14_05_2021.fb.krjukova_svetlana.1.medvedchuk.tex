% vim: keymap=russian-jcukenwin
%%beginhead 
 
%%file 14_05_2021.fb.krjukova_svetlana.1.medvedchuk
%%parent 14_05_2021
 
%%url https://www.facebook.com/kryukova/posts/10159393960908064
 
%%author 
%%author_id 
%%author_url 
 
%%tags 
%%title 
 
%%endhead 
\subsection{Вчерашняя неудавшаяся попытка власти закрыть Медведчука наталкивает на три вывода}
\Purl{https://www.facebook.com/kryukova/posts/10159393960908064}

Вчерашняя неудавшаяся попытка власти закрыть Медведчука наталкивает на три вывода.

\begin{itemize}
	
\item 1. Своими дурными действиями в надежде загнать под себя суды власть Зеленского
сама себя сделала врагом всей судебной системы. Хотя ранее отношения между
президентской и судебной властью, да и при Зеленском при его на то желании –
были достаточно конструктивными (чего стоит только позиция КС, когда Зе решил
досрочно прекратить полномочия Рады). Ныне же, после попытки «перезагрузки»
Конституционного суда, преследования ряда «неправильных» его судей,
законопроекта о ликвидации ОАСКа, демонстративной отдачи на откуп иностранным
партнерам системы отбора судей – надежды на конструктив похоронены раз и
навсегда. И теперь любой уважающий себя судья будет вести себя
показательно-принципиально, используя любые юридические конструкции и
возможности, чтобы пресекать на корню любые попытки Банковой протянуть через
суд свою линию и вынудить принять выгодное ей политическое решение.

\item 2. Систему прокуратуры с СБУ можно считать полностью разваленной. Если даже по
главному на нынешний момент и крайне принципиальному для Зеленского «делу
Медведчука» усилиями ГПУ и СБУ удалось собрать настолько ничтожную и крайне
неубедительную доказательную базу с совершенно смешными претензиями к
подследственному – то на что же они способны в более сложных делах? Вопрос
риторический. Реформа Луценко-Рябошапки-Венедиктовой «удалась». Привела к
полнейшей деградации прокуратуры, когда посланные на суд обвинители были не в
состоянии дать судье ответы на элементарнейшие вопросы. Если даже избрание меры
пресечения вылилось в трагикомический фарс, то какой же спектакль мы получим,
когда дело будет рассматриваться по сути… 

\item 3. Ввязавшись в из рук вон плохо подготовленный процесс и ставя задачу
«отминусовать Медведчука», Зеленские и его команда нарвались на обратное – дали
шикарный пас своим оппонентам, в том числе медийный. Если после закрытия трех
телеканалов «опзжисты» всех мастей стали растворяться в информационном
пространстве и теряться в нем, не имея площадок для своей пропаганды, то «кейс
Медведчука» в контексте очередной неудачи власти теперь обязательно там будет
крутиться достаточно продолжительное время. Ну и включите русское ТВ, куда
перебежала значительная  часть украинских пенсионеров после закрытия тройки
каналов - насладитесь эффектом.
\end{itemize}

