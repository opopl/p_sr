% vim: keymap=russian-jcukenwin
%%beginhead 
 
%%file 25_07_2021.fb.zharkih_denis.1.faine_savenkovoj
%%parent 25_07_2021
 
%%url https://www.facebook.com/permalink.php?story_fbid=3040403686173028&id=100006102787780
 
%%author 
%%author_id zharkih_denis
%%author_url 
 
%%tags dnr,donbass,literatura,lnr,savenkova_faina,ukraina,vojna
%%title Моему другу Фаине Савенковой
 
%%endhead 
 
\subsection{Моему другу Фаине Савенковой}
\label{sec:25_07_2021.fb.zharkih_denis.1.faine_savenkovoj}
 
\Purl{https://www.facebook.com/permalink.php?story_fbid=3040403686173028&id=100006102787780}
\ifcmt
 author_begin
   author_id zharkih_denis
 author_end
\fi

Моему другу Фаине Савенковой.

\obeycr
В своих недавних эссе я писала, что половина моей жизни — это война.
И самое страшное — привыкнуть к смерти.
Это неправильно, так не должно быть.
И хотелось бы, чтобы взрослые это поняли. 
\restorecr

Ф.В.Савенкова

Для меня до сих пор не покидает мысль, почему и как мы допустили войну на
Донбассе. Мы, мирный, образованный и хлебосольный народ и тут такое.
Официальные ответы Порошенко и Зеленского меня не устраивают. Я же видел, как
это все готовилось, как раскачивалось, как внедрялось. Первый Майдан 2004 года
только чудом избежал столкновений и крови. Тогда еще были культурные тормоза,
тогда еще не принято было звереть и хвататься за оружие. Политика знала
подкуп, ложь, предательство, но не массовое убийство людей и обстрелы городов
сел. Теперь все иначе.

\ifcmt
  pic https://scontent-lga3-1.xx.fbcdn.net/v/t1.6435-9/222941855_3040403652839698_4519750148863880454_n.jpg?_nc_cat=110&ccb=1-3&_nc_sid=730e14&_nc_ohc=D3TkbLuxD9MAX_CCmvZ&_nc_ht=scontent-lga3-1.xx&oh=c88fef9e68a133e4d96f8fe18e00dbfb&oe=61261AC2
  width 0.4
\fi

К убийствам на Донбассе украинцы привыкли сразу. Не то, чтобы совсем не больно,
больно когда убивают своих, но чужие украинцы не вызывают никакой жалости, даже
дети. На гражданской войне убивают неправильных граждан. Но в нашем случае
сложно понять в чем они неправильные. Впрочем, команды сочувствовать от
кураторов не поступало, значит, не сочувствуем. 

А когда гибель чужих детей не вызывает эмоций, когда страдания и боль
соотечественников, скажем так, чужих соотечественников не вызывает желание
прекратить это немедленно, то это признаки гражданской войны. 

При этом же война не тотальна – тут воюем, тут торгуем. Тем, кто воюет, да,
плохо, но тем, кто торгует, очень хорошо.  Ну, стреляют иногда на Донбассе, но
это же в другом конце страны, а мы живём и решаем свои вопросы по мере их
поступления, "идём в Европу", стараемся соответствовать, а по факту
мимикрировать этическим, моральных, экологическим эталонам западных партнёров,
словом, не до "сепаров" теперь. Украина становится всё более глобализированным
государством – беды Судана для нас страшнее слёз соотечественников. Там нефть,
интересы США, а тут какие-то неправильные граждане, понять которых нет никакой
возможности. 

А это всё боль десятков, если не сотен тысяч людей, вынужденных покинуть свои
дома и родные города, сёла в поисках не счастья, а куска хлеба и хлипенькой, но
крыши, которая бы не ходила по вечерам ходуном от обстрелов. Уже на глазах
наших соотечественников из оккупированных территорий растут их дети, для
которых война – это что-то привычное, чей взгляд под стать тридцатилетним,
прошедшим через огонь, воду и медные трубы и уж не как не под стать детям,
которые растут в тепле, заботе, покое и достатке. Политики нам упрямо
объясняют, что это не те дети, и это вообще другое. Масса украинцев верят. Так
удобнее. 

И вот во всей этой вакханалии лжи, глупости откровенного паразитизма и
политической пошлости появляется ребенок с чистой, как слеза, душой, и пишет
сказки, книги, обращения. Она не жалуется, не обличает, не требует, как
правильные политические девочки в западной политике. Она творит культуру.
Именно то, что нам сейчас не хватает. Двенадцатилетняя писательница Фаина
Савенкова, уроженка города Луганска несет свет, несмотря на обстрелы и
проклятия с той стороны. Вот, к примеру, что говорит Фаина, когда у неё
спрашивают о том, отняли ли у детей Донбасса детство (она же говорит в первую
очередь за себя в силу своего возраста): 

"Я не думаю, что у нас отняли детство. Просто оно у нас немного другое. Ну а
для того, чтобы ребенок вырос хорошим, надо чтобы родители его любили и учили
самостоятельно различать зло и добро. Мне кажется, война тут практически ни при
чем. Разве что на войне чуть быстрее понимаешь, что человеческую жизнь нужно
беречь, а не уничтожать.”

Так почему же маленькая Фаина понимает, чего взрослые дяди не понимают? Не тому
учились? И не то, чтобы Фаину уж совсем не замечали.

В свои 12 лет Фаина Савенкова является членом Союза писателей ЛНР, а также
Интернационального Союза писателей.

Её рассказ «Глициния и кошка» победил на конкурсе «Мир глазами ребенка 2019» в
Чехии, эссе «Детский смех Победы» опубликовано в литературном журнале «Юность»,
переведено на английский, сербский, итальянский, болгарский, чешский, арабский,
итальянский, французский, немецкий языки.

Эссе «Взрослое молчание» опубликовано во французском англоязычном блоге
Stalker, сербских газетах «Исток» и «Факти», во французской Donbass Insider,
kantasuomalainen.net (Финляндия), AgoraVox (Франция), переведено на английский,
сербский, итальянский, болгарский, чешский, финский языки.

А в этом году она получила премию Мира Интернационального союза писателей и
начала вести колонку на английском в исландском англоязычном портале The Saker.
Вышел ее совместный с российским писателем Александром Конторовичем роман
«Стоящие за твоим плечом».

А кроме того, Фаина хорошо учится в школе увлекается тхэквондо, и уже стала
двукратным чемпионом ЛНР по этому виду восточных единоборств. 

Все это хорошо, но на Донбассе не прекращают стрелять. 

И маленькая Фаина встала против этих выстрелов своим словом, проснувшимся
талантом, открытой душой. Начинающийся жизненный путь Фаины Савенковой можно
назвать путём мирного воина, который пытается достучаться до власть имущих и
борется с несправедливостью с глупой и бесчеловечной политической ситуацией на
востоке Украины..

1 июня этого года юная писательница записала видеообращение членам Совета
Безопасности ООН и поделилась своими детскими размышлениями о человеческих
добродетелях, таких как добро, миролюбие, ответственность, любовь к миру, о
том, что невзирая на то, что жизнь таких детей войны как она сама не даёт
ответа на вопрос \enquote{что будет завтра и будет ли оно вообще?}: \enquote{Я хочу, чтобы вы
вспомнили улыбки ваших детей. Хочу сказать, что мы тоже хотим улыбаться, хотим
быть счастливыми, хотим выбирать свое будущее и хотим просто жить. А для этого
война должна закончиться. Я хочу, чтобы в Организации Объединенных Наций не
забывали про то, что мы, дети Донбасса, тоже имеем право на детство и мирную
жизнь.}

И вот маленький воин Фаина встала за справедливость, за жизнь, за человечность.
Ведь достаточно вспомнить старые сказки, которые не рассказывают нынешним
детям, их просто нет в смартфонах, что злоба лечится добром, ненависть любовью,
жадность щедростью, а глупость мудростью. И маленькая Фаина несет в мир любовь
и мудрость для того, чтобы мир стал лучше. И только очень сильным и талантливым
людям такое по плечу. И это говорит о том, что раз столько силы и любви в
маленькой девочке, то нам взрослым, как-то стыдно отставать,, если мы, конечно
не забыли, что такое стыд.
\ii{25_07_2021.fb.zharkih_denis.1.faine_savenkovoj.cmt}
