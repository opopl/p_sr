% vim: keymap=russian-jcukenwin
%%beginhead 
 
%%file 27_01_2022.stz.news.ru.eho_moskvy.1.kiev_stolica_shtatov_evropy
%%parent 27_01_2022
 
%%url https://echo.msk.ru/blog/mikhail_freidlin/2971494-echo
 
%%author_id news.ru.eho_moskvy,frejdlin_mihail
%%date 
 
%%tags evropa,future,geopolitika,kiev,rossia,ukraina,zapad
%%title Киев — столица Соединенных Штатов Европы. Сказка, которую следует сделать былью
 
%%endhead 
 
\subsection{Киев — столица Соединенных Штатов Европы. Сказка, которую следует сделать былью}
\label{sec:27_01_2022.stz.news.ru.eho_moskvy.1.kiev_stolica_shtatov_evropy}
 
\Purl{https://echo.msk.ru/blog/mikhail_freidlin/2971494-echo}
\ifcmt
 author_begin
   author_id news.ru.eho_moskvy,frejdlin_mihail
 author_end
\fi

\begin{zznagolos}
Стремительно развивающиеся геополитические события дают возможность украинскому
руководству (нынешнему либо следующему), как бы это ни выглядело фантастично,
принять лучшее в своей истории стратегическое решение. Именно сейчас у Украины
появился редкий шанс избежать участи страны, чья судьба решается другими
государствами. Стать, наконец, субъектом самостоятельной внешней политики,
установить прочный мир в стране, ускорить подъем национальной экономики, а
также заметно влиять на развитие экономик других стран на евроазиатском
пространстве. Попробуем пояснить.
\end{zznagolos}

\textbf{1. Известно, что ради обеспечения своей безопасности Россия пошла ва-банк.}
Обладая мощнейшим военным и союзническим потенциалом, она серьезно озадачила за
«покерным столом» других геополитических игроков, главным образом США. А что,
если, размышляют они, Путин действительно не блефует, укоротит НАТО, лишит
государственности Украину, перекроит европейскую карту? Сохранит ли тогда США
роль мирового гегемона, рассуждают американские стратеги? Но и с этим
сценарием, думают американцы, мы не особенно-то теряем, рискуем, поскольку
делаем ставки в игре за европейский счет и находимся далеко за океаном. А
потому, будем дальше честно блефовать за чужой счет и вести свою линию до
конца.

Повышение ставок между двумя основными игроками может закончиться
осуществлением военно-технических и военных решений со стороны России. А что
тогда делать «за покером» странам Западной Европы, чьи карты заметно слабее
(нарастающий дефицит природных ресурсов, снижение экономики, социальных
стандартов населения, невысокий военный потенциал). И что делать Украине,
которая все поставила на кон и при этом уже не участвует в игре? Надо признать,
что и Россия в процессе обеспечения своей безопасности, экономически ослабнет.
В общем, всей Европе будет несладко с таким сценарием развития событий.

А вот американцы будут потирать руки, как это было в I и II мировых войнах
(если только военное противостояние не перерастет в ядерную войну). Больше всех
пострадает население Европы. США, разумеется, затем милостиво восстановят
Западную Европу, а заодно поднимут свою экономику. И опять будут вести себя
вальяжно в мировой политике.

Таковы ожидаемые результаты предложенной Россией парадигмы обеспечения своей и
общеевропейской безопасности. Прямо сказать, «коэффициент полезного действия»
ее полноценной реализации невысок.

\textbf{2. Но есть радикально иная парадигма решения всех выдвинутых Россией требований
по обеспечению своей безопасности.} И не только своей, но и в качестве бонуса —
общеевропейской безопасности. Более того, в условиях усиления
межконтинентальной экономической конкуренции за рынки сбыта и ресурсы,
европейские страны заметно поднимут свое благоденствие. Речь идет об изящной по
своей простоте, и известной не одно столетие концепции мирного решения
перманентных геополитических потрясений в Европе.

И вот здесь, как бы это ни выглядело парадоксальным, фраза Владимира
Зеленского, что «Украина нужна Европе, а не наоборот», над которой многие, по
инерции, подшучивают, на самом же деле, содержит в себе мощный смысл. Украина
действительно нужна Европе (а не наоборот), если в озвученной им фразе Украина
будет видеться не «баррикадой», защищающей Западную Европу от России, а,
напротив, крепким мостом, соединяющим их под девизом «Европейцы всех стран,
соединяйтесь!». Пока весь мир вникает в документ, который Москва предъявила США
и всему коллективному Западу в виде табу на расширение НАТО и военное освоение
этой организацией постсоветского пространства, Киев мог бы выдвинуть свое
пацифистское встречное предложение. А именно – проект воплощения в жизнь давно
предсказанной судьбы европейских государств о создании Соединенных Штатов
Европы (СШЕ).

Об этом мега проекте говорили и писали Наполеон I, Виктор Гюго, Тадеуш
Костюшко, Карл Каутский, Владимир Ленин, Уинстон Черчилль, Конрад Аденауэр,
Шарль де Голль, Жан Моне, Владимир Путин, Ангела Меркель, Эммануэль Макрон и
еще много других выдающихся людей. Так, в начале своего правления, 23 ноября
1959 года, де Голль выступил со знаменитой речью о «Европе от Атлантики до
Урала». Речь, по существу, шла о создании Соединенных Штатов Европы(Более
детально об этом — в статье «Соединенные Штаты Европы — столица Киев», вышедшей
в 2011 году в «Еженедельнике «2000»», № 29–30). Збигнев Бжезинский в своей
книге «Стратегическое предвидение» предложил даже перенести штаб-квартиру
Совета Европы из французского Страсбурга в «древнюю столицу Киевской Руси».
Бжезинский здесь рассуждает о «расширении конфигурации Запада» за счет
присоединения к ней России, Турции и Украины. Исходной точкой его концепции
служит признание «угасания Запада» и «передвижения центра тяжести в мире с
Запада на Восток». Владимир Путин также говорил, что проблемы Европы не будут
решены до тех пор, «пока не будет достигнута ее идентичность, органическое
единство всех ее интегральных частей, включая Российскую Федерацию».

У Украины, являющейся интегральным ядром всей Евразии, есть геополитические
предпосылки и уникальный, можно сказать, последний шанс катализировать процесс
такой интеграции во благо западноевропейским государствам, России, другим
странам Евразийского пространства, а главное — во благо своему народу. В этом
проекте больше всех выигрывает Украина, поскольку находясь в центре Европы она
обладает естественной геоэкономической рентой, позволяющей извлекать
сверхприбыль благодаря оптимальной логистики размещения производительных сил на
ее территории и транзита товаров по ней.

Для этого Украина должна стать, наконец, внеблоковым государством.

Чтобы сказку сделать былью — превратить Киев в cтолицу Соединенных Штатов Европы — следует реализовать следующую программу действий (тезисы):

Украина становится ВНЕБЛОКОВЫМ государством и всячески поддерживает идею
интеграции «от Лиссабона до Владивостока».

Принимается общеевропейская Декларация о начале процесса интеграции «от
Лиссабона до Владивостока», с перспективой создания на этом пространстве
международной свободной экономической зоны (МСЭЗ).

Платформой интеграции и формирования МСЭЗ становятся базовые правовые
стандарты, деловые обычаи, требования к экологической безопасности. Возможны
договоренности о ЕДИНОЙ рыночной инфраструктуре, согласованной ценовой политике
в энергетической, транспортно-логистической, продовольственной и иных сферах
экономического взаимодействия. Тогда на евразийском пространстве возникнет
практически неограниченный обмен ресурсами, капиталами, высокими технологиями,
будут созданы мощные региональные финансовые центры, в корзине которых заметно
увеличится доля евро, рубля, гривны, других национальных валют стран МСЭЗ.

Все страны, разумеется, и Россия, на самом высоком уровне (Совбез ООН)
гарантируют территориальную целостность Украины (Крым в процессе евразийской
интеграции де-факто станет территорией-кондоминиумом, «общим ребенком» двух
стран).

Что особенно важно — на уровне Совбеза ООН должны быть достигнуты
договоренности, в которых пространство МСЭЗ от Атлантического до Тихого океана
провозглашается территорией неиспользования вооруженных сил в каких бы то ни
было конфликтах. (По сути, это и есть альтернатива упомянутому выше российскому
предложению по обеспечению безопасности).

Первым проектом развития МСЭЗ становится Украина — прежде всего, восстановление
Донбасса.

Понятно, если интеграция и создание МСЭЗ — задача на годы, то прекращение
вооруженного противостояния — задача самая, что ни на есть, краткосрочная,
первейшая. Есть ли у Украины другая, более достойная альтернатива решить свои
внутри— и внешнеполитические проблемы, остановить кровопролитие на Востоке?
Конечно же, нет! Именно сейчас у нее появился реальный шанс не только
остановить военные действия на Донбассе, но с достоинством, сохранением лица и
национальных интересов всех стран — прямо или косвенно имеющих отношение к этой
острой проблеме — выйти из геополитического тупика.

Может, мы излишне оптимистичны, но нам представляется, что именно сейчас у
Украины есть шанс стать не только одной из передовых стран мира, но и придать
импульс созидательным процессам в других странах. Для этого следует начать
самостоятельно осуществлять свою внешнюю политику с опорой на гениальную
сентенцию Платона, который вещал, что «политика — это способность мирно жить с
соседями». Жизнь уже доказала Украине закономерность этой сентенции: не следует
поворачиваться «одним местом» ни к западным, ни к восточным соседям, а стать
внеблоковым государством и развивать добрососедское сотрудничество по всем
азимутам.

И тогда у Президента Украины Владимира Зеленского, наконец, появится реальная
тема для встречи с Президентом Российской Федерации, Владимиром Путиным, а
также для разговора в Нормандском формате с первыми лицами крупнейших
государств Европы. Думается, такой разворот не может не понравиться его визави.
Может «лед тронется», и тогда на нашем континенте, в конце концов, появятся
напророченные Соединенные Штаты Европы. Возможно, со столицей в Киеве.

P.S. Кратко о геополитических интересах «фигурантов» создания СШЕ

Думается, что представленное Россией предложение американцами,
персонифицирующими интересы всего НАТО, будет реализовано с весьма низким КПД.
Это связано с объективными обстоятельствами. В условиях накала
межконтинентальной экономической конкуренции, американцы не заинтересованы в
интеграции стран на евроазиатском материке, экономическом росте своих партнеров
по НАТО в Европе, что уж тогда говорить о России. Они, также как и любая другая
страна, будь она на месте США, «кровно» заинтересованы в монополизации своего
влияния на политико-экономические процессы в мире. Американцы стараются не
только сохранить, но и умножить свою геополитическую власть, обеспечивая всеми
доступными ей способами высокую монетизируемость своего ВПК,
конкурентоспособность транснациональных корпораций, всей экономики, поддерживая
на высоком уровне социальные стандарты своих граждан.

Но на «Великой шахматной доске», в отличие от обычной шахматной игры и покера,
не должно быть победителя. Здесь все должно заканчиваться миром, в выигрыше
должны оказаться все игроки. Поэтому решение Киева о своей внеблоковости и
готовности стать ядром интеграции от «Лиссабона до Владивостока», надеемся,
будет поддержано всеми странами Европы.

США же, после общеевропейской интеграции, вернется в рамки доктрины Монро и
станет крепкой региональной державой в Западном полушарии. Все к этому идет,
поскольку это объективно обусловленный, целесообразный геополитический процесс,
ведущий, в конечном счете, к равноправному, сбалансированному развитию всего
мирового сообщества.

А государствам Западной Европы предстоит, наконец, твердо осознать, что без
опоры на ресурсный и интеллектуальный потенциал постсоветских стран, прежде
всего, России, Украины и Казахстана (почти 40\% всех природных ресурсов
планеты), они не смогут противостоять нарастающей межконтинентальной
конкуренции — и не только со стороны США, но и мощной китайской экономики,
Японии, других крупных стран.

\textbf{У Киева есть реальный шанс присоединить к званию «матери городов русских»
статус «столицы Соединенных Штатов Европы». Дело за малым — Киеву следует
заявить о своем стремлении стать такой столицей!}

%...Из Киева полетят сигналы на Марс, Юпитер и Нептун. Сообщение с Венерой
%сделается таким же легким, как переезд из Рыбинска в Ярославль. А там, как
%знать, может быть, лет через восемь в Киеве состоится первый в истории
%мироздания междупланетный шахматный турнир! Остап вытер свой благородный лоб.
%Ему хотелось есть до такой степени, что он охотно съел бы зажаренного
%шахматного коня...
