% vim: keymap=russian-jcukenwin
%%beginhead 
 
%%file 14_04_2021.fb.respublikalnr.2.kto_slil_ukrainu
%%parent 14_04_2021
 
%%url https://www.facebook.com/groups/respublikalnr/permalink/796157774353349/
 
%%author 
%%author_id 
%%author_url 
 
%%tags 
%%title 
 
%%endhead 
\subsection{ПОЛИТ-БЛОГ. ЮЛИЯ ВИТЯЗЕВА:  Кто слил Украину?}
\label{sec:14_04_2021.fb.respublikalnr.2.kto_slil_ukrainu}
\Purl{https://www.facebook.com/groups/respublikalnr/permalink/796157774353349/}

ПОЛИТ-БЛОГ. ЮЛИЯ ВИТЯЗЕВА:  Кто слил Украину?

Чтобы понять нынешнее положение дел на Украине — достаточно посмотреть реакцию
тамошней патриотической тусовки на вчерашний звонок Байдена Путину.


\ifcmt
  pic https://scontent-bos3-1.xx.fbcdn.net/v/t1.6435-9/173574690_122130683300373_2690442159820976044_n.jpg?_nc_cat=110&ccb=1-3&_nc_sid=825194&_nc_ohc=XtVDR34gEGwAX-b9gZ-&_nc_ht=scontent-bos3-1.xx&oh=75947260d2393f3af093ec66101ebfae&oe=609BC309
\fi


Условно она разделилась на два лагеря.

Одни аплодируют «хитрому плану» Байдена, согласно которому подготовка к
возможной встрече двух президентов отвлечёт Путина от Украины, усыпит его
внимание и даст фору во времени для вступления в НАТО и размещения на
территории Украины всех вожделенных ништяков, призванных умерить аппетиты
«аХрессора».

Вторые буквально рыдают и пьют валерьянку, донося до читателей печальную
новость о том, как Байден слил Украину и теперь дело только в сроках, в течении
которых Киев вынужден будет принять все условия Москвы, навсегда забыв про Крым
и Донбасс.

Разумеется, диванная аналитика далека от реалий. Но оба лагеря, сами того не
подозревая, констатируют один печальный, но непреложный факт.

А именно — Украина на данный момент не субъект, а объект международной
политики. И больная для многих тема «про Украину без Украины» — это следствие
той политики, которую семь лет проводили украинские власти, обменивая
какой-никакой суверенитет и нацинтересы на личные ништяки и сомнительные
сиюминутные перемоги.

Так что, это не Байден слил, а слили сами украинцы из числа тех, кому любой
ценой хотелось если не быть, то хотя бы казаться цеЕвропой и равноправным
игроком.

Но они этого до сих пор не поняли.

И, похоже, ещё долго не поймут.

Просто потому, что на уровне сливного бачка (а именно там в моральном плане
сейчас находится Украина) такие познания не приходят.

Юлия Витязева, специально для News Front
