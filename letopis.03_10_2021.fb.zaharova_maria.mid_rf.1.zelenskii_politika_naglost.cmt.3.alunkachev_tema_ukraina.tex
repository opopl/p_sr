% vim: keymap=russian-jcukenwin
%%beginhead 
 
%%file 03_10_2021.fb.zaharova_maria.mid_rf.1.zelenskii_politika_naglost.cmt.3.alunkachev_tema_ukraina
%%parent 03_10_2021.fb.zaharova_maria.mid_rf.1.zelenskii_politika_naglost.cmt
 
%%url 
 
%%author_id 
%%date 
 
%%tags 
%%title 
 
%%endhead 
\paragraph{Абдурахман Алункачев - Любимая тема Украина ваша}

\begin{itemize} % {
\iusr{Абдурахман Алункачев}
Любимая тема Украина ваша внутри страны нечего хорошего нету независимые Сми не осталось

\begin{itemize} % {
\iusr{Александр Софинский}
\textbf{Абдурахман Алункачев} , а где три телеканала закрыли? Неужто в России? @igg{fbicon.wink}  @igg{fbicon.face.smiling.sunglasses} 
\iusr{Василий Шабунин}
\textbf{Абдурахман Алункачев} Захарова пресс секретарь МИДа, занимается своей работой, причём здесь внутренние проблемы страны?

\iusr{Iakov Le Ffoleg}
\textbf{Абдурахман Алункачев} ЗАТКНИСЬ!

\iusr{Лариса Михайлова}
\textbf{Абдурахман Алункачев} вы по стране часто путешествуете? Не надо таки за всю страну вешать из своего региона)))
\end{itemize} % }

\iusr{Jurij Olechnovic}
Это как с 277 миллиардами запрошеными Зеленским на трансформацию Украины.

- На что будут потрачены эти деньги?

- Мы пока не знаем. Пока мы просто цифру придумали. Специально не круглую,
чтобы сильно в глаза не кидалось то, что она взята просто "с потолка".   @igg{fbicon.face.smiling.eyes.smiling} 

\iusr{Алесь Карніенка}
ваша кремлёвская - сама взвешенность и миролюбие!

\begin{itemize} % {
\iusr{Iakov Le Ffoleg}
\textbf{Алесь Карніенка} ВОТ!

\ifcmt
  ig https://scontent-mia3-2.xx.fbcdn.net/v/t1.6435-9/243952098_255831713052434_7172874655766684472_n.jpg?_nc_cat=110&ccb=1-5&_nc_sid=dbeb18&_nc_ohc=ZEfEGsgE59UAX_CSmZz&_nc_ht=scontent-mia3-2.xx&oh=159c0c24a056ab1c743888714261b247&oe=6186DBBE
  @width 0.4
\fi

\end{itemize} % }

\iusr{Mikael Multipolar}
Не могу дождаться этого дня

\ifcmt
  ig https://scontent-mia3-1.xx.fbcdn.net/v/t1.6435-9/244317036_167324928891795_7948194380767472576_n.jpg?_nc_cat=111&ccb=1-5&_nc_sid=dbeb18&_nc_ohc=DDwrmdHiXfwAX9U3uni&_nc_ht=scontent-mia3-1.xx&oh=4f0fe7b304ff0e6cafef38df32dbf60d&oe=6186B2EF
  @width 0.4
\fi

\begin{itemize} % {
\iusr{Валера Короткий}
\textbf{Mikael Multipolar} Миша не поможет?)))..

\iusr{Николай Буренков}

Путин включился в процесс интеграции России и Белоруссии. Вся эта ду * * цкая
комедия с " объединением " тянется уже пятый год, если не больше, и
замечательно иллюстрирует способности наших чиновников к поганой волоките с
одновременным симилурованием кипучей деятельности. За это надлежит вешать на
вонючих верёвках, как рекомендовал В. И. Ленин. Но не за шею.

\iusr{Кирилл Жирихин}
\textbf{Mikael Multipolar} обнулённого своего раньше в гробу увидишь.

\iusr{Irina Neverec}
\textbf{Кирилл Жирихин} а давайте навального в президенты Украины

\iusr{Adam Shohat}
\textbf{Mikael Multipolar}

\ifcmt
  ig https://scontent-mia3-1.xx.fbcdn.net/v/t39.30808-6/243780901_134217755611250_113132306755928647_n.jpg?_nc_cat=100&ccb=1-5&_nc_sid=dbeb18&_nc_ohc=afCD1FJ9dYIAX_aCRFx&_nc_ht=scontent-mia3-1.xx&oh=cb937e593bd5008d91f761837aff9926&oe=6165C8D9
  @width 0.4
\fi

\iusr{Yu Riy}
\textbf{Mikael Multipolar} кто же его посадит, он же памятник. А вот Владимира Ильича пора подменить в мавзолее на Владимира Владимировича, считаю что он заслужил. И желательно не затягивать с этим делом.

\iusr{Emil V Beck}
Передвижной зоопарк?

\iusr{юрий логвинов}
\textbf{Mikael Multipolar} не дождетесь @igg{fbicon.beaming.face.smiling.eyes} 

\iusr{Алекс Кубичек}
\textbf{Mikael Multipolar} и не дождешься

\iusr{Сергей Богданов}
\textbf{Mikael Multipolar} рядом с ним, надо посадить порошенко и остальную братву, которая нажилась и наживается на Украине, грабя свой народ.

\end{itemize} % }

\iusr{Ilya Schu}
А у России какое развитие? ВВП 0\% роста за 10 лет….

\begin{itemize} % {
\iusr{Игорь Рубцов}
\textbf{Ilya Schu}
А вы умеете считать ввп?

\iusr{Елена Шабанова}
\textbf{Ilya Schu} Сказал человек без лица .Как вы достали работники. Ничего в жизни не делающие. Вот это работа. И блин ,общество такого кормит

\iusr{Башкатов Владимир}
\textbf{Ilya Schu} тролль, как говорила Верка Сердючка, сама придумала, сама поверила

\iusr{Natasha Cannon}
\textbf{Ilya Schu} ну, если учесть, как почти весь капитал ваших исключмтельных хозяев из демократий и их еврподсывал все эти годы гробится на изоляцию нашей "бензоколонки", то даже просто ни шагу вниз для нашего валовой продукта - это уже победа над вами.

\iusr{Ilya Schu}
\textbf{Башкатов Владимир} а с чем вы не согласны? Где рост?

\iusr{Ilya Schu}
\textbf{Игорь Рубцов} а зачем мне его «уметь» считать? Его считает ВБ и ЦБ.

\iusr{Ilya Schu}
\textbf{Елена Шабанова} вы в Киеве не согласны с тем, что ВВП России не растёт 10 лет? А вот фразу «ничего в жизни не делающие» применительно ко мне оставьте себе, так как вы, мягко говоря, ошибаетесь.

\iusr{Елена Шабанова}
\textbf{Ilya Schu} Смотрим на аватарку. И, что же там? @igg{fbicon.face.tears.of.joy}{repeat=3} 

\iusr{Ilya Schu}
\textbf{Елена Шабанова} там всего лишь машина, которая мне понравилась. И вы не ответили на вопрос

\iusr{Игорь Рубцов}
\textbf{Ilya Schu}
ВБ и ЦБ никакого отношения к реальной экономике не имеют.
Если исходить из их методик, то при нынешнем уровне цен на нефть и газ, ВВП РФ за этот год вырос на 30%.

\iusr{Ilya Schu}
\textbf{Игорь Рубцов} но он не вырос. 1.6 трлн много лет подряд.

\iusr{Игорь Рубцов}
\textbf{Ilya Schu}

Меня абсолютно не интересует, какие цифирьки рисуют западные банкиры, к реальной экономике это не имеет никакого отношения.

Ещё по ППС как-то можно сравнивать, да и то ...

\iusr{Ilya Schu}
\textbf{Natasha Cannon} 

у меня к вам два вопроса - почему вы поддерживаете режим и живете в оплоте
демократии штате вашингтон? Вы не пробовали любить бензоколонку, находясь на ее
территории, а не в прекрасном тихом месте на берегу океана ? Вы правда
считаете, что при столь высоких ценах на нефть последнее десятилетие в
отсутствии роста экономики виноват коллективный запад, а не тотальная коррупция
и несменяемость власти? И кстати кто виноват в убыли населения последние годы?
Кто виноват в высоких налогах? Низкой пенсии, плохих дорогах? А кто сократил
количество больниц вдвое за 20 лет? А кто увеличил количество чиновников в три
раза? Запад?


\iusr{Елена Шабанова}
\textbf{Ilya Schu} Таких людей в соц. сетях очень много. Вас ведь видно за версту.))) А вы не знаете почему такие проблемы в стране? Вы посчитали бы ВВП Украины. Или вы начитались и наслушались украинской токсичной пропаганды?

\iusr{Ilya Schu}
\textbf{Игорь Рубцов} 

если вас данный вопрос не интересует, зачем вы лезете в дискуссию? А самое
главное, что вы хотите мне доказать? Сформулируйте мысль? Или ответьте на
простые вопросы- Доходы граждан всей страны растут? Пенсии в реальном выражении
растут? Бензин дешевеет? ЖКХ дешевеет? Пром производство может растёт? Может
цены на продукты падают? Я понимаю, что ввп вас не волнует, но может простые
вещи вас волнуют, которые можно потрогать?


\iusr{Ilya Schu}
\textbf{Елена Шабанова} начнём с первого - меня Украина не интересует. Давайте вы там как-то без нас разбирайтесь. Какое-то неведомое государство аннексировало исконно русские территории и переделывает русских в неведомую нацию. А второе - каких «таких» людей?

\iusr{Игорь Рубцов}
\textbf{Ilya Schu}
Я у вас спросил. Вы считали ВВП или оперируете цифирьками банкиров?
Во сколько они оценивают наличие комплексов Авангард, Посейдон, Кинжал, С-500? Как это соотносится с бюджетом к примеру пентагона у которого и близко ничего подобного нет?
Если вы человек в здравом уме, то никогда не будет все эту ахинею писать и ссылаться на неё.
По всем остальным вопросам, все эти показатели в реальном выражении падают по ВСЕМУ МИРУ С 2008 года.
И связано это с мировой долларовой системой, а не с конкретной страной.
А если брать чистыми, то внешний долг США во сколько раз вырос за 10 лет?
И наоборот сколько РФ накопила резервов за тот же срок?

\iusr{Елена Шабанова}
\textbf{Ilya Schu} 

Где там? Да кто же спорит? Что Украина это на самом деле территория России .Кроме
возможно наших пришибленных нациков гуцулов. Вы же понимаете ,что страну душат
санкциями и внутри ,что творят либералы. А ведь много бед именно из за Украины.


\iusr{Ilya Schu}
\textbf{Игорь Рубцов} 

я оперирую цифрами банкиров, потому что это позволяет в одних цифрах сравнивать
экономики, в неработающих с500, как и арматах, как и пак-5, которые все на
картинках и в мультиках все ещё, сложновато. Так вот в одинаковых цифрах наша
экономика 10 лет 1.6трлн долларов, наши звр как были 500 с небольшим млрд, так
и остались, наш госдолг при этом вырос с 50 до 300 млрд. ВВП США 10 лет назад
был 16трлн, сейчас 21трлн, госдолг изменился точно так же. Так что наш госдолг
вырос в разы, американский на проценты. Средний доход американских семей растёт
все последнее десятилетие и превысил 60 тысяч. так что падает только у нас.

\iusr{Ilya Schu}
\textbf{Елена Шабанова} 

либералы сидят по тюрьмам. Все что могут творят чиновники и казнокрады. Санкции
небольшое влияние оказывают. А вообще Либерализм провозглашает права и свободу
каждого человека высшей ценностью и устанавливает их основой общественного и
экономического порядка. Не пойму, чего плохого вы вкладываете в это слово.


\iusr{Natasha Cannon}
\textbf{Ilya Schu} 

о ты, завистливый, ленивый и бестолковый любитель замочных скважин, подальше
полистай и увидишь, что я много где живу - и вовнаукре, и на Урале, и в Баку, и
в Лондоне, и в Стамбуле, в Дохе даже, в Колорадо, и в Москве, и на Каме. Ты не
пробовал изучать мир в его реальности, а не по указивкам наглецов-"демократов"?

Ты никогда не задумывался - как это к нас не было даже слова коррупция до
самого прихода либеральных ценностей 30 лет назад, а то место откуда оно пришло
- его родина, соответственно, демокоатия с либералом - его родители, это их
дитя и менталитет. Про больницы - размеры больниц считаешь или на 5 коек и на
400 - тебе либеральной пофиг?

Кого ты считаешь чиновником - перечисли чины по порядку.

Несмегяемость власти - в сша, где всего две партии почти 300 лет, а байдены с
пелосями и маккейнами по полвека, сидят в капитолиях и прочих сенатах.

\iusr{Елена Шабанова}
\textbf{Ilya Schu} 

В том то и дело ,что люди не имеющие никакого отношения к либерализму так
очернили само это понятие. Что слышать уже невозможно. Да кого же вы считаете
либералами седящими по тюрьмам? У вас чиновников хорошо трусят. И сажают их. В
России всегда все жалуются. Никто и не говорит, что нет проблем. Я сама русская. И
поверьте мне есть с чем сравнивать. Не так все у вас и плохо.

\iusr{Ilya Schu}
\textbf{Natasha Cannon} 

Начнём с того, что как только собеседник переходит на личности - это означает,
что разумные аргументы кончились. И кстати, без ошибок можете писать, а то не
понятно, что такое вовнаукре? Несменяемость власти это Путин с 1999 года, а
Клинтон, Буш, Обама, Трамп и Байден за тот же период - это не похоже на
несменяемость. Либерализмом в России не пахнет, в РФ попахивает ЦК и застоем.
Что касается отсутствия иностранного слова в лексиконе не отменяло
существование казнокрадства, которое при совке было и обуславливало прекрасную
жизнь близких к «телу» поваров, завхозов, начальников распределителей и тп, я
уж молчу про отдельную жизнь номенклатуры с «березками», ресторанами и прочими
благами.

\iusr{Ilya Schu}
\textbf{Елена Шабанова} если сравнивать с зимбабве, то в России все прекрасно. Если сравнивать в целом с приличными экономиками, то мы много потеряли за последнее десятилетие. И станет хуже. Потому что ничего не происходит кроме воровства и повышения поборов.

\iusr{Natasha Cannon}
\textbf{Ilya Schu} 

первым ты перешёл на мою личность, когда стал шариться по моему аккаунту, это
для меня характерный признак определённых персон из интернета.

Вовнаукра это то несчастье, что нынче выделывается на остатках территории
бывшей усср.

Про что такое несменяемость власти не греми погремушкой, не убедительно.
Перечисленные тобой клинтоны и обамы ничто, так как власть не у них, а у
партии, для которой они служат петрушками, которых никто всерьёз не принимает.
Повдыхать вонючий либерализм России хватило 8-10 лет, до сих пор острая
аллергия на такой. У нас было другое понимание о том, что значит, быть
либеральным, что такое демократия, это не то, что несут ястребы и гиены в
шкурах ослов и слонов.

Если слова в лексиконе не было, то, значит, не было и того, что им описывалось.
А вот отдельные блага - это как раз из разделов западных демократий с их
благами исключительно для элит.


\iusr{Елена Шабанова}
\textbf{Ilya Schu} 

Ну зачем же Зимбабве. Но зря вы так. Много в России делается Да, бедность это бичь
страны. Не забывайте, что КОВИД то никто не отменял. И в странах которыми мы
считаем благополучными ничуть дела обстаят не лучше.

\iusr{Игорь Рубцов}
\textbf{Ilya Schu}

Ещё раз, это не одинаковые цифры, от слова совсем. У нас экономика рублевая.
Если хотите сравнивать, то хотя бы нужно применять индекс бигмака, согласно ему
наша уже повышается в три раза.

Далее, львиная часть ввп сша в услугах, которые отношения к реальной экономике
не имеют. Один и тот же консалтинг может стоить 1000\$, а может в стони раз
больше. Поэтому показатель экономики США завышен, как минимум в 2 раза.

Про мультики Турции, Индии, Китаю расскажите, а то стоят в очередь за покупкой
мультиков.

А уж скромное долг сша увеличился пропорционально экономике повеселило.

Можете впаривать это все школьникам, но они на других ресурсах и ветках сидят,
а здесь лучше не позориться.

\iusr{Emil V Beck}
За последние 10 лет ВВП России вырос почти в два раза. Это легко проверить.

\iusr{Ilya Schu}
\textbf{Игорь Рубцов} 

госдолг был в 2013 17млрд, сейчас 26. Где тут разы, о которых вы говорили?
Львиная доля в сфере услуг? Открыл данные, 29\%. Это львиная? дальше
дискутировать о ввп не вижу смысла. Вы сначала с цифрами в своей голове
разберитесь. Теперь что касается мультиков. Сколько С500 на вооружении стоит?
Ноль? Сколько Армат на вооружении? Я знаю про пару. Сколько истребителей пятого
поколения? Ноль. И очереди за этим не видно. Особенно со стороны китайцев.
Можете впаривать школьникам, а здесь лучше не позориться.

\iusr{Ilya Schu}
\textbf{Natasha Cannon} 

я не помню, чтобы мы с вами переходили на ты... но уточните, признак каких же
персон из интернете является то, что мне стало интересно, откуда пишет
собеседник, если вас так, конечно, можно назвать? Я с вашего позволения, заменю
слово либерализм на «свободное общество» что собственно это слово и обозначает.
Так вот времена настоящего свободного общества пришлись на нефть по 7\$ и на
руины советской экономики, которая никогда не могла конкурировать на мировой
арене, а дальше власть взяли чекисты, нефть стала расти и идеи свободы стали
душиться ради того, чтобы все блага от высоких цен доставались узкому числу
близких людей, которые по факту контролируют большую часть экономики и не дают
ей развиваться, все больше втравливая страну в репрессии ради удержания себя у
этой жирной кормушки. По итогу мы имеем нулевой рост экономики десятилетие,
отсутствие дорог, аэропортов, образования, больниц и единственным экспортным
продуктом страны являются природные ресурсы, нищенские пенсии и средние
зарплаты, при этом имеем платные дороги, шведские налоги в размере почти 50\% на
физических лиц и почти 40\% на юридических, отсутствие гарантии частной
собственности и справедливого суда. Хотя, понимаю, резидентов штата вашингтон
это не сильно волнует.


\iusr{Ilya Schu}
\textbf{Emil V Beck} 

В 2013 ввп был 2.2трлн \$. Сейчас 1.6. В рублях можно что угодно рисовать.
Десять лет назад он был 70трлн рублей, сейчас 100. Завтра рубль будет 200 и ввп
станет 200. Кстати двумя разами не пахнет. Напомню рубль с тех пор упал с 24
рублей до 36 и потом до 75.

\iusr{Andrey Vardugin}

Это какие данные вы открыли? По разным оценкам (включающими или нет те или иные
отрасли в услуги, например, платное образование ), доля услуг колеблется от 65
до 78 \% ВВП. Если у населения нет денег или желания их тратить на услуги, эта
отрасль быстро схлопывается. Можно напечатать и раздать денег населению, как в
пандемию, но инфляция это неминуемая плата.

А за наш ВВП вы не переживайте, растёт, как бы его не считали. Строительный бум
и дефицит автомобилей о чём-то должны говорить.

\iusr{Natasha Cannon}
\textbf{Ilya Schu} 

Есть персоны в интернетах - типа тебя, которые лазят по аккаунтам оппонентов,
вот таким я сразу начинаю тыкать, так как они мне становятся этакими знакомыми
из породы сплетниц.

Словосочетание "свободное общество" так же требует пояснений - от чего оно
свободно, для кого оно свободно, за счёт чего и кого оно свободно, ну и тд.
Советская экономика была мощнейшей в мире в 20м веке, особенно если учесть, чтО
против неё вытворяли "свободные".

Нулевой рост экономики - это точно не про Россию, не болтай своей ерундой
зазря)) Шведские налоги и нравы нас не колеблют, пусть живут как хотят. Про
резидента вашингтона не понятно - это ты про себя или про своего хозяина?

\emph{Ilya Schu}
\textbf{Andrey Vardugin} например вот 

https://www.statista.com/.../percentage-added-to-the-us.../ 

где же 70 процентов?


\end{itemize} % }


\end{itemize} % }
