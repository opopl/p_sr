% vim: keymap=russian-jcukenwin
%%beginhead 
 
%%file 01_12_2021.fb.titarenko_lubov.1.babushka_galja
%%parent 01_12_2021
 
%%url https://www.facebook.com/permalink.php?story_fbid=3064164530508840&id=100007459690216
 
%%author_id titarenko_lubov
%%date 
 
%%tags ukraina
%%title Не раздражайтесь на нас гневом и недоверием
 
%%endhead 
 
\subsection{Не раздражайтесь на нас гневом и недоверием}
\label{sec:01_12_2021.fb.titarenko_lubov.1.babushka_galja}
 
\Purl{https://www.facebook.com/permalink.php?story_fbid=3064164530508840&id=100007459690216}
\ifcmt
 author_begin
   author_id titarenko_lubov
 author_end
\fi

\ifcmt
  pic https://i2.paste.pics/530bfad44279defe65527c3def2de3a8.png
  @width 0.8
  %@wrap \parpic[r]
\fi

\href{https://www.youtube.com/watch?app=desktop&v=ZRyTcvr_oAs}{%
Маруся жива, пока есть кому стать рядом со мной, Галиной, %
Любовь Титаренко Театр Политика Общество, youtube, 01.12.2021%
}

\begin{multicols}{2}
Не раздражайтесь на нас гневом и недоверием.
Друзья, даже не знаю, как описать встречу с Бабушкой Галей.
Что такое переворот  в натуре и сознании я увидела и осознала, 
глядя на ее условия жизни и быта. Холод, нет газа и электричества. 
Оставленная ей в наследство, после смерти сёстры Маруси
 - больная дочь, лет 50.
Много вещей, сложённых в большие груды, 
и каждая дорога, как память.

\ifcmt
  ig https://i2.paste.pics/68807e178a843469a46030c43144e328.png
  @width 0.4
\fi

С недоверием смотрит на гостей. Говорит, - много раз ее обманули.  Тема ЖУКОВА
сразу нас сблизила. Она тут же принесла его портрет. Решила переодеться, делала
это долго скрупулёзно.  Показала больные ноги,  рассказывая, как от боли горят,
при этом даже сменила обувь.  Подарила мне платок своей  Маруси, сдерживая
слезы.  Достала ее костюм с наградами и сказала: 

«Ты, как Маруся. Стань по другую сторону портрета.  Какой трепет и дрожь я
испытала, прижав к груди  пиджак и ее награды.»

Мне было и больно и стыдно, мы предали старшее поколение, оставили их, - сам на
сам.  Да, впрочем мы предали и самих себя. 

У неё трезвый ум и надежда - крик души  для нас, что, если не струсим, то
победим.  Для неё сегодня, происходящее,- это война с коричневой чумой.
Главные и лучшие дни  ее воспоминаний - это конец войны и ЖУКОВ, пришедший к
ним в монастырь с подарками- чемоданами. 

\ifcmt
  ig https://i2.paste.pics/5404886e17b3124916a44e8cd249cd15.png
  @width 0.4
\fi

«Он взял нас на руки и долго не отпускал:
Мы дети войны- до того знали насилие, 
в прямом смысле этого слова в перемешку с голодом и холодом. 
Но никогда не могли подумать, что старость будет ещё страшнее. Жестокость и смерть от своих» 
-Наша команда привезла  ей  строителя. Зовут его Виталий. Представили.  Решили.
Они договорились, если не наступит холод приступает к забору, 
ибо вокруг развалины.  Пригласила меня кресты на могилке красить.
Очень трудно было с ней общаться, это, - как по краю пропасти. 
Где потеряно доверие и обманута совесть, честь и память. 
То, чем она дорожит особенно в свои годы.
И поэтому сегодняшнее выступление нашего гаранта звучит особенно нагло и с плевком.  Оказывается, что мы живем в сказке. 
Вот только сказка это о том, что Король не любит свой народ!
Как написала Лидия Дельфин:
Надо же, оказывается, что мы живём в раю, который создал Зеленский, но только рай Зеленского, - это ад а сам Зеленский, - это дьвол.»
 —
Господин Президент, идите .... В отставку. Иначе,- сметут!
Ибо Ваш рай, с войной на Донбассе, ссорами с Россией, Беларусью, Крымом и собственным народом- это АД. Дно.
/И да, если что- не боюсь ваших вассалов , и угроз./
Александр Потемкин
Денис Жарких
Татьяна Мазурова
Владимир Быстряков
Екатерина Жарких
Руслан Коцаба, и много много  неравнодушных, к происходящему в стране - друзей. Помните, плюющий в прошлое,- теряет родство. 
Забыть и не помнить тех, кто победил и подарил нам жизнь- грех.
Как хорошо сказал \href{https://www.facebook.com/profile.php?id=100008532614385}{Александр Лазарев}
\href{https://www.facebook.com/100008532614385/posts/2735352603425800/}{%
Переворот давно уже произошёл, переворот сознания, логики и морали, Александр Лазарев, facebook, %
01.12.2021%
}
\end{multicols}
