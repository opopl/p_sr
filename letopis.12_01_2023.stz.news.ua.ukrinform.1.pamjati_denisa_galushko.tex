% vim: keymap=russian-jcukenwin
%%beginhead 
 
%%file 12_01_2023.stz.news.ua.ukrinform.1.pamjati_denisa_galushko
%%parent 12_01_2023
 
%%url https://www.ukrinform.ua/rubric-ato/3661882-pamati-vijskovosluzbovca-ta-volontera-denisa-galuska.html
 
%%author_id news.ua.ukrinform
%%date 
 
%%tags 
%%title Пам'яті військовослужбовця та волонтера Дениса Галушка
 
%%endhead 
 
\subsection{Пам'яті військовослужбовця та волонтера Дениса Галушка}
\label{sec:12_01_2023.stz.news.ua.ukrinform.1.pamjati_denisa_galushko}
 
\Purl{https://www.ukrinform.ua/rubric-ato/3661882-pamati-vijskovosluzbovca-ta-volontera-denisa-galuska.html}
\ifcmt
 author_begin
   author_id news.ua.ukrinform
 author_end
\fi

\textbf{Хвилина мовчання 31.01.2023 09:00}

\ifcmt
  ig https://static.ukrinform.com/photos/2023_01/thumb_files/630_360_1675065806-580.jpg
  @wrap center
  @width 0.7
\fi

12 січня відбулося прощання з українським військовим та відомим волонтером
Денисом Галушком, який загинув у боях під Бахмутом у Донецькій області.

До повномасштабного вторгнення рф чоловік займався ресторанним бізнесом,
цікавився історією, читав багато книжок, мріяв написати й свою. Дуже цінував
родину. Він міг би прожити повне та яскраве життя, якби не росія...

\enquote{У Дениса була можливість не йти на війну. По-перше, він батько трьох дітей.
По-друге, у нього не було однієї нирки, тому він міг би не піти на цю війну.
Він пішов. Денис любив Україну, але не кричав про це на кожному кроці. Він не
здавався патріотом, а саме був ним}, - розповів Укрінформу однокласник Дениса
Галушка адвокат Масі Найєм.

\ii{12_01_2023.stz.news.ua.ukrinform.1.pamjati_denisa_galushko.pic.1}

За його словами, Галушко із дитинства цінував Батьківщину. \enquote{Ми із ним вчилися
із п'ятого класу разом. І вже тоді Денис не любив р..ню. Він любив Україну.
Історія про те, що сталося після 24 лютого, - це вже продовження його
ментальної поведінки}, - сказав Найєм і додав: \enquote{У Дениса вся родина
проукраїнська. Тато у нього є віруючою людиною. Все, що я знаю про батька,
крутилося навколо української ідентичності}.

У військовослужбовця із позивним \enquote{Брат} були дуже теплі стосунки із молодшим
братом Максимом, який продовжує захищати Україну від російської навали. \enquote{Максим
залишається на війні. Він хоче помсти. І я розумію, що росіяни цю помсту
отримають}, - переконаний Масі Найєм.

За словами адвоката, нині дружина Дениса Катерина важко переживає втрату
чоловіка, адже їхня родина була щасливою. Денис пішов на війну в тому числі
заради того, щоб захищати свою сім'ю.

\ii{12_01_2023.stz.news.ua.ukrinform.1.pamjati_denisa_galushko.pic.2}

Мінометник 58 омпбр (група Грека) Юрій Лук'янчук розповів, що загиблий захисник
був людиною дуже цілеспрямованою, тим, хто вмів працювати із людьми, знаходив
контакт навіть з власниками складного характеру. Денис, за його словами, умів
морально налаштувати на досягнення результату, всім цікавився щиро, а не «для
галочки». Воїн спробував себе у декількох військово-облікових спеціальностях,
попри те, що це була його перша війна. До лютого минулого року Галушко займався
волонтерством. \enquote{Ми були в одному підрозділі, але виконували різні завдання.
Навесні я познайомився спочатку з його братом Максимом, а потім із Денисом.
Наприкінці березня – на початку квітня разом воювали під Макаровим. А потім,
коли звільнили Київщину, з Денисом, його братом та із саперами з
президентського полку збирали трофеї під Києвом}, - сказав Юрій Лук'янчук.

Подруга родини Галушків Ольга Тулімова зазначила, що з Денисом працювала в
одній команді багато років. \enquote{Денис - первісток родини Галушків. Його дитяча
фотографія \enquote{маленького козачка} стала частиною бренду та логотипу їхньої мережі
етноресторанів \enquote{Козачок}. Батько та мама Дениса Олександр та Світлана Галушки
свого часу відкрили перший ресторан мережі, і цей факт виділяв їхній бренд
особливою теплотою, родинним затишком. Усе це про цінності та традиції, волю та
славу української родини, в якій виріс Денис}, - сказала вона та додала, що для
себе назавжди запам'ятає полеглого воїна самодостатнім, цілісним, глибоким,
добрим і навіть впертим.

\enquote{Він був щирою людиною та розумним хлопцем. Коли у нього був поганий настрій,
він обирав лишатися наодинці. Але Денис був суцільний оптимізм та позитив. Був
амбіційним та невгамовним. Мав легке ставлення до життя, і радив ближнім, що
треба простіше дивитися на різні речі}, - поділився друг та кум полеглого воїна
Олег Прокопець.

\ii{12_01_2023.stz.news.ua.ukrinform.1.pamjati_denisa_galushko.pic.3}

Батько загиблого захисника Олександр Галушко пишається сином. Діди та прадіди
Дениса були козаками, тому без любові до України ніяк. Ще у дитинстві тато із
сином разом цікавилися усім українським, побували у багатьох куточках нашої
держави, побачили її красу. \enquote{Тому Денис був таким. Він завжди, якщо
щось робив, то доводив справу до завершення. Якщо не знав, то шукав інформацію,
радився зі мною. Побратими Дениса привезли мені його речі, а там, дивлюся,
книги з тактичної війни, довідники. Він сам читав і, кажуть хлопці, їх навчав.
Це в поміч їм було}, - розповів Олександр Галушко.

За його словами, син Дениса після загибелі батька підтримує матір.
\enquote{Каті зараз дуже-дуже погано, бо вони щасливі були. Мають трьох дітей.
Онуку моєму 13 років, онучка Соня на рік молодша, а Матильді 4 роки виповнилося
у листопаді.  На другий день після того, як не стало Дениса, його син Семен
підійшов до мами та сказав, що тепер стане батьком для Матільди. Наскільки він
мужній, хоча ж іще малий}, - зазначив батько полеглого захисника.

Останню свою публікацію у Фейсбуці Денис Галушко оприлюднив за декілька годин
до загибелі. Вона була про Бахмут. {\bfseries\enquote{Прийшов мінус і тепер всі хто під
Бахмутом в буквальному сенсі розуміють фразу \enquote{вгризатися в землю}.
Боже, дай мені душевний спокій прийняти те, що не можу змінити, відвагу змінити
те, що можу. І мудрість відрізнити одне від другого. Жити нинішнім днем, радіти
кожній миті, Вважати незгоди... Гори у пеклі вся @бана наволоч, що прийшла до
нас з мечем, від нього і загинете, безславно, на Богом нам даній землі}}, -
написав український воїн.

Денис Галушко народився 27 серпня 1984 року. Він міг жити, виховувати дітей, а
згодом й онуків, планувати, працювати, радіти, сумувати, падати та підніматися,
але зранку 8 січня 2023 року ворожа міна змінила все. Син та дві донечки не
дочекалися повернення рідного та найкращого тата з жорстокої війни.

%\ii{12_01_2023.stz.news.ua.ukrinform.1.pamjati_denisa_galushko.pic.4}
\ii{12_01_2023.stz.news.ua.ukrinform.1.pamjati_denisa_galushko.pic.4.font_17pt}

Відспівали Героя у Михайлівському золотоверхому соборі 12 січня. Вічний
спочинок воїн знайшов на Совському цвинтарі.
