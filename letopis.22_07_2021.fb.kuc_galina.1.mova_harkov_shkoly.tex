% vim: keymap=russian-jcukenwin
%%beginhead 
 
%%file 22_07_2021.fb.kuc_galina.1.mova_harkov_shkoly
%%parent 22_07_2021
 
%%url https://www.facebook.com/permalink.php?story_fbid=1544598909221453&id=100010140758684
 
%%author 
%%author_id kuc_galina
%%author_url 
 
%%tags jazyk,mova,obrazovanie,ukraina,ukrainizacia
%%title Розмова про мову. Про російську мову в УКРАЇНСЬКИХ школах Харкова.
 
%%endhead 
 
\subsection{Розмова про мову. Про російську мову в УКРАЇНСЬКИХ школах Харкова.}
\label{sec:22_07_2021.fb.kuc_galina.1.mova_harkov_shkoly}
 
\Purl{https://www.facebook.com/permalink.php?story_fbid=1544598909221453&id=100010140758684}
\ifcmt
 author_begin
   author_id kuc_galina
 author_end
\fi

Розмова про мову.

Про російську мову в УКРАЇНСЬКИХ школах Харкова.

Капслок у моєму дописі я використовую лише для представників Департаменту
освіти Харківської міської ради.

На депутатські прийоми до мене, як депутата Харківської обласної ради,
приходять батьки, діти яких навчаються в українських школах Харкова. Часто ці
батьки, свідомо обираючи українську школу для навчання дитини, розуміють, що ця
школа може бути віддалена від місця проживання. Але для них важливо, щоб дитина
здобувала освіту державною мовою.

Проблеми та скарги, з якими приходять батьки на прийоми, доволі однотипні: в
деяких українських школах Харкова чомусь значна частина часу виділяється на
вивчення російської мови та літератури. І в таких школах батькам нібито
пропонується самим обрати цю дисципліну для вибору. Батьки запитують дирекцію
школи: чи можна, щоб замість російської мови дітям більше годин було виділено
на вивчення іншої іноземної мови. Але такі ситуації переважно завершуються
цькуванням цих батьків та їхніх дітей. Адже знаходиться лояльна більшість серед
інших батьків, які обирають те, що просить дирекція. А права дітей, які в
меншості, і які б хотіли вивчати іншу іноземну мову, врешті, нікого не
цікавлять. 

Аргументи батьків, які приходять на прийом, наступні: ми й так обрали школу, в
яку дитину далеко возити, але ми готові йти на такі жертви, щоб дитина чудово
знала державну мову. Ці батьки вважають, що їхні діти й так чудово володіють
російською мовою, оскільки мешкають у Харкові. Тому пріоритетнішим для них є
вивчення іншої іноземної мови. Також ці батьки відкидають закиди щодо того, що
їхні діти зможуть краще оволодіти граматикою російської мови, якщо будуть
вивчати її. Вони аргументують, що для того, щоб діти грамотно писали, вони
повинні читати книги. Грамотність серед дітей не залежить від кількості годин
вивчення мови, лише від кількості прочитаних книг. Ось такі аргументи.

Тепер щодо законодавчих підстав введення російської мови в УКРАЇНСЬКИХ школах у
навчальний план. Існує документ під скороченою назвою «Типова освітня
програма»: 

\href{https://mon.gov.ua/ua/npa/pro-zatverdzhennya-tipovoyi-osvitnoyi-programi-zakladiv-zagalnoyi-serednoyi-osviti-ii-stupenya}{
ПРО ЗАТВЕРДЖЕННЯ ТИПОВОЇ ОСВІТНЬОЇ ПРОГРАМИ ЗАКЛАДІВ ЗАГАЛЬНОЇ СЕРЕДНЬОЇ ОСВІТИ II СТУПЕНЯ, mon.gov.ua,%
20.04.2018%
}

Цей документ регламентує формування навчальних планів для шкіл різного типу. Так ось! Для шкіл з УКРАЇНСЬКОЮ мовою навчання навчальний план має формуватися згідно з Таблицею 1. 

Але!!! Окремі батьки скаржилися, що їм показують Таблицю 12, де написано, що
має вивчатися «мова корінного народу, національної меншини». І на тій підставі
вводять російську мову в навчальні плани в УКРАЇНСЬКИХ школах. Але Таблиця 12
призначена лише для шкіл «з українською мовою навчання і вивченням мови
корінного народу, національної меншини».

Ще раз доступно пояснюю: якщо, згідно зі Статутом школи, школа УКРАЇНСЬКА – це
Таблиця 1. Якщо ж школа українська+мови корінного народу – це Таблиця 12.
Статус школи зафіксований у її Статуті. Виходить, зі слів батьків, що фактично
відбувається ігнорування Статутних документів навчального закладу, коли
«просовуються» такі рішення.

Звісно, що звинувачувати школи у фальсифікаціях, базуючись на усних скаргах
батьків, я не можу. Тому я написала депутатське звернення на ім’я Директора
Департаменту освіти Харківської міської ради. Я попросила надати перелік
УКРАЇНСЬКИХ шкіл Харкова, де вивчається російська мова та російська література.
Також попросила, щоб було вказано, на якій підставі ці дисципліни включено в
навчальні плани шкіл з УКРАЇНСЬКОЮ мовою навчання.


\ifcmt
  tab_begin cols=4

     pic https://scontent-lga3-1.xx.fbcdn.net/v/t1.6435-9/220221009_1544598585888152_4793242407477734293_n.jpg?_nc_cat=109&ccb=1-3&_nc_sid=730e14&_nc_ohc=zyt45EDYH1wAX8hegrj&_nc_oc=AQnvgwwDtlT2n5xPDylRovH0G1q083-XcTF8QV9IWw8LKmqSoTI565yyUFpSnvXt_V4&_nc_ht=scontent-lga3-1.xx&oh=6dc0c382db5d321d33161da8366713d8&oe=61268C9D

     pic https://scontent-lga3-1.xx.fbcdn.net/v/t1.6435-9/218987637_1544598635888147_1816580626570640113_n.jpg?_nc_cat=105&ccb=1-3&_nc_sid=730e14&_nc_ohc=xfcmhhlRYA4AX8PiN9q&_nc_ht=scontent-lga3-1.xx&oh=3b47d42c1573d08308884b1a555d33ce&oe=61240C9A

     pic https://scontent-lga3-1.xx.fbcdn.net/v/t1.6435-9/220668074_1544598599221484_2166785224152008192_n.jpg?_nc_cat=109&ccb=1-3&_nc_sid=730e14&_nc_ohc=TJWA4wciyQAAX_tI8kz&tn=lowUrFCbCbt-jOWu&_nc_ht=scontent-lga3-1.xx&oh=8ff4e0bffa3874f331be5661706e1058&oe=612523CD

     pic https://scontent-lga3-1.xx.fbcdn.net/v/t1.6435-9/221282574_1544598682554809_8019658630966260717_n.jpg?_nc_cat=102&ccb=1-3&_nc_sid=730e14&_nc_ohc=7V6I6uUF1vIAX-_tsXY&_nc_ht=scontent-lga3-1.xx&oh=a8b437e8bf48779ab23843cf689f8d3c&oe=6126CF4F

  tab_end
\fi


Тобто, мене зацікавили суто фактичні речі: перелік шкіл Харкова з УКРАЇНСЬКОЮ
мовою навчання та номери Таблиць Типової освітньої програми, на підставі яких
введено російську мову у навчальні плани цих шкіл.

Окрім того, я ще декілька важливих питань задала, зокрема, щодо забезпечення прав дітей.

І ось я отримала відповідь від Департаменту освіти. Типова відписка ні про що…
Ніякого переліку шкіл мені надано. Більше того, взагалі проігноровано, що моє
звернення стосувалося лише шкіл з УКРАЇНСЬКОЮ мовою навчання.

Ще одне. Ця відповідь-відписка своєю некомпетентністю вражає з першої фрази.
Вони мені пишуть, що дають відповідь на мій ЗАПИТ. Але я ЗАПИТІВ не подавала, я
подавала ЗВЕРНЕННЯ ! Це, згідно з Законом, зовсім різні документи! І люди, які
отримують зарплати з бюджету та причетні до сфери формування освітньої політики
в регіоні, мали б знати законодавство.

Перші два скріни – це відповідь від Департаменту освіти Харківської міської ради на моє звернення.

Другі два скріни – це, власне, моє депутатське звернення.

P.S. Що ж, відповіді на ЗВЕРНЕННЯ я так і не отримала. Напишу нове депутатське звернення. 

Надіюся, що дочекаюся фактажу: 

\begin{itemize}
  \item 1) перелік шкіл Харкова з УКРАЇНСЬКОЮ мовою навчання, де включено російську мову у навчальний план; 
  \item 2) підстави включення цього предмету у навчальні плани шкіл Харкова з УКРАЇНСЬКОЮ мовою навчання.
\end{itemize}


\ii{22_07_2021.fb.kuc_galina.1.mova_harkov_shkoly.cmt}
