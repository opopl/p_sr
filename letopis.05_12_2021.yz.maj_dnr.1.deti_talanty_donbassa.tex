% vim: keymap=russian-jcukenwin
%%beginhead 
 
%%file 05_12_2021.yz.maj_dnr.1.deti_talanty_donbassa
%%parent 05_12_2021
 
%%url https://zen.yandex.ru/media/id/5f8f226b1fe36c1d9e02a36b/malenkie-bolshie-talanty-donbassa-naperekor-voine-61ad1403d087c16fc38147f3
 
%%author_id yz.maj_dnr
%%date 05_12_2021
 
%%tags donbass,vojna,deti,tvorchestvo,talant,doneck
%%title Маленькие Большие Таланты Донбасса: наперекор войне
 
%%endhead 
\subsection{Маленькие Большие Таланты Донбасса: наперекор войне}
\label{sec:05_12_2021.yz.maj_dnr.1.deti_talanty_donbassa}

\Purl{https://zen.yandex.ru/media/id/5f8f226b1fe36c1d9e02a36b/malenkie-bolshie-talanty-donbassa-naperekor-voine-61ad1403d087c16fc38147f3}

\ifcmt
 author_begin
   author_id yz.maj_dnr
 author_end
\fi

\ii{05_12_2021.yz.maj_dnr.1.deti_talanty_donbassa.pic.1}

Однажды в декабре 2015 года в Донецк приехали матери попавших в плен украинских
вояк. Они собрались у нас в переговорной, но не одни – к ним присоединились
мамочки наших ополченцев, попавших в плен к украинцам. Бесконечно жаль, что я
так и не смогла найти видео того разговора, но хорошо запомнила мысль, которая
тогда меня пронзила насквозь – это хорошо, что этим украинским матерям не
довелось познакомиться с нашими мамочками, потерявшими деток из-за того, что
вот эти самые пленные украинские вояки, за которыми приехали их мамаши,
стреляли по нашим мирным городам и поселкам, прекрасно понимая и зная, что в
результате погибнут наши дети Донбасса.

Эта мысль не дает мне покоя до сих пор. И позвольте мне не поверить в то, что
кто-то из родственников украинских военных не знает, по ком именно стреляют их
сыновья. 

\subsubsection{А знаете, какие у нас замечательные дети?}

Они разбираются в звуках войны – они точно знают, что, когда свистит, значит,
снаряд пролетит мимо, а когда шуршит – лучше поскорее спрятаться.

Они умеют не плакать, если украинский снаряд прилетел в дом, но произошло Чудо
и все остались живы – они знают цену вещим и тем более, знают цену человеческой
жизни.

Они умеют радоваться мелочам и даже научились жить так, как будто за окном не
таится гибель.

Они загадывают Деду Морозу главнее желание: «Хочу мира». Они приходят в школу и
звонят мамам с сообщением, что добрались нормально, имея в виду не что иное как
возможные обстрелы по пути.

Они учатся, ходят в спортивные секции, музыкальные школы и всякие кружки,
играют на разных инструментах, рисуют, лепят и вообще, живут насыщенной жизнью.
Как будто и нет войны.

Недавно я побывала в двух Школах искусств – в Макеевке и в Донецке – в самом
обстреливаемом районе, на Петровке. И я снова плакала от счастья. От того, что
у нас удивительные дети.

Я думаю, что во всех, до единого, малышах есть какой-нибудь талант. Главное –
его увидеть, понять и раскрыть. Наверное, не у всех родителей это получается
из-за бытовых хлопот, работы и прочих жизненных проблем. Но в последние годы я
наблюдаю колоссально увеличившееся количество внешкольных заведений – искусства
или спорта - которые стали для детей едва ли не вторым домом. И я не буду вам
рассказывать, насколько полезнее учиться играть на скрипке или гонять в футбол,
чем зависать в гаджете.

Но позвольте я вас познакомлю со Уколами искусств, где мне повезло побывать.

\subsubsection{1. Макеевская Школа искусств №4}

Макеевка теперь в тылу, слава Богу. Возможно поэтому на конкурс «Донбасс –
любовь моя» дети не прислали картин с военной тематикой. Зато пейзажи были
великолепны. Но лучше я вам просто их покажу:

\ii{link.05_12_2021.youtube.maj_dnr.1.vse_deti_talantlivy}

Но это еще не все! После самого мероприятия я решила поподробнее поговорить с
директором школы, и она мне показала невероятные работы из архивов. Более всего
меня поразили, и я таки засняла на камеру именно эти работы, афиши, которые
дети рисовали к фильмам о войне. То есть, школа организовала показы военных
фильмов, а уж потом ребята выбирали какой из фильмов попадет на их кисти и
карандаши. Вот это видео:

\ii{link.28_11_2021.youtube.maj_dnr.1.patrioty_vospitanie}

Правда здорово? Я уже не говорю о том, насколько важны для нормального
воспитания настоящего человека и патриота своей наши военные картины. И
огромное спасибо педагогам школы за их работу!

\subsubsection{2. Школа искусств №2 г Донецка. Петровский район.}

Я специально упомянула о районе нашего города, куда украинские снаряды
прилетают чаще всего. Но оказывается, она не прост работает, а собирает детей
из самых отдаленных прифронтовых уголков Петровки. Так вот, на базе этой школы
организовали конкурс «Волшебный мир искусства», в котором приняли участие более
200 маленьких музыкантов Республики. А я попала на подведение итогов конкурса и
была просто поражена таланту выступающих детей. Жаль, что их было не много, но
тогда мероприятие бы точно затянулось надолго. Однако, некоторые все же попали
на мою камеру. И прошу вас обратить внимание на девчушку с гитарой и парнишку,
сыгравшего «Болеро» на скрипке. Да! И еще на малышню из ансамбля народных
инструментов – это было круто!

