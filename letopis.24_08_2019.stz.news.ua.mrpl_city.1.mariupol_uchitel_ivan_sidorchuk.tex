% vim: keymap=russian-jcukenwin
%%beginhead 
 
%%file 24_08_2019.stz.news.ua.mrpl_city.1.mariupol_uchitel_ivan_sidorchuk
%%parent 24_08_2019
 
%%url https://mrpl.city/blogs/view/ko-dnyu-nezavisimosti-ukrainy-mariupolskij-uchitel-ivan-sidorchuk
 
%%author_id burov_sergij.mariupol,news.ua.mrpl_city
%%date 
 
%%tags 
%%title Ко Дню Независимости Украины: мариупольский учитель Иван Сидорчук
 
%%endhead 
 
\subsection{Ко Дню Независимости Украины: мариупольский учитель Иван Сидорчук}
\label{sec:24_08_2019.stz.news.ua.mrpl_city.1.mariupol_uchitel_ivan_sidorchuk}
 
\Purl{https://mrpl.city/blogs/view/ko-dnyu-nezavisimosti-ukrainy-mariupolskij-uchitel-ivan-sidorchuk}
\ifcmt
 author_begin
   author_id burov_sergij.mariupol,news.ua.mrpl_city
 author_end
\fi

\ii{24_08_2019.stz.news.ua.mrpl_city.1.mariupol_uchitel_ivan_sidorchuk.pic.1}

Малолетние дети не могут обобщать. И слова запоминаются, если они связаны с
конкретными предметами, конкретными людьми, конкретными погодными явлениями.
Так, у автора этих строк в \enquote{нежном} возрасте слова \enquote{украинец} и
\enquote{украинка} были связаны только с дедушкой Петей и бабушкой Таней.
Потому что они говорили только по-украински. Был еще дед Мирон, который также
говорил, как бабушка и дедушка, но за глаза его называли \enquote{казак},
отсюда для ребенка следовало, что он не украинец. С годами стало понятно, что
не всякий говорящий по-украински - украинец. И не всякий говорящий не
по-украински, но преданный Украине, ее традициям, ее истории, работящий, гордый
за свой свободолюбивый народ, и есть украинцем. Таким был \textbf{Иван
Терентьевич Сидорчук.}

В двухэтажном угловом здании по адресу Евпаторийская, 56  с 1944 по 1955 год
находилась Мужская средняя школа № 4 имени И. А. Крылова. Впрочем, это учебное
заведение имело номер 4 еще с довоенных времен и после 1955 года также, но
мужской, т.е. такой, где учились одни мальчики, она существовала только в
упомянутый выше период. В то же самое время директором ее был Иван Терентьевич.

\ii{24_08_2019.stz.news.ua.mrpl_city.1.mariupol_uchitel_ivan_sidorchuk.pic.2}

Среднего роста, легкая сутуловатость, прожилки кровеносных сосудов на скулах,
непокорный чуб, зачесанный назад, волосы, присыпанные кое-где сединой,
просторный пиджак, подчеркивающий сухопарость торса, левый рукав пиджака
засунут в карман. Таким, вероятно, остался в памяти его учеников внешний облик
Ивана Терентьевича. Он запомнился своим воспитанникам и поступками, глубокий
воспитательный смысл которых был осознан ими через много-много лет, уже в
зрелом возрасте. Да, он был требователен, иногда грубоват. Сорванцы его
побаивались, но вместе с тем и любили. Эта любовь выражалась не во внешних
проявлениях и, боже упаси, не в словах, а в том, что ему без обид прощали
\enquote{непедагогические} деяния...

\textbf{Читайте также:} 

\href{https://mrpl.city/news/view/mariupoltsy-v-den-nezavisimosti-prinesli-tsvety-velikomu-kobzaryu-fotofakt-1}{%
Мариупольцы в День Независимости принесли цветы Великому Кобзарю, Богдан Коваленко, mrpl.city, 24.08.2019}

%\ii{24_08_2019.stz.news.ua.mrpl_city.1.mariupol_uchitel_ivan_sidorchuk.pic.2}

\textbf{История с галошами, обувью, давно вышедшей из моды.}

Родительское собрание. Одна из мамаш, осмотревшись, с ужасом воскликнула:
\emph{\enquote{Неужели это малолетнее хулиганье научилось бегать по стенам и потолку?}}.
Действительно, на побелке класса тут и там красовались отпечатки галошных
подошв. Конечно, подопечные Ивана Терентьевича не могли преодолевать земное
притяжение, а отпечатки – следы баталий на переменах, в которых оружием были
галоши. С этими \enquote{шалостями} Иван Терентьевич боролся способом своеобразным.
Приоткрыв дверь, он некоторое время наблюдал за действом, а когда оно достигало
кульминации, то поднимал ближайшую к нему галошу и подзывал кого-нибудь из
бойцов. Тотчас вся орава шалопаев устремлялась к выходу, тут-то директор и
\enquote{угощал} галошей каждого второго или третьего по месту, на котором более всего
протирались штаны школяров. Но никто никогда на директора не жаловался, более
того, не держал на него зла.

\textbf{Свидетелем разделения в 1944 году школ на мужские и женские был Юрий Дзюман-Грек, выпускник 1946 года.}

– Здание школы было сожжено немцами при отступлении. Занятия проводились в
полуподвальных помещениях и на первом этаже. На полу второго этажа была
устроена крыша. Старшеклассники помогали рабочим эту крышу обустраивать. Иван
Терентьевич жил в маленьком домике в углу школьного двора. Запомнилась такая
картина: класс, в котором я учился, занимает небольшое помещение против
лестницы, и вот в этой небольшой комнате наш директор ведет урок украинской
литературы. Он говорит тихо, но так образно и увлекательно, что теряется
ощущение реальности: стены раздвигаются, ты уже на открытом пространстве с
героями повествования. В конце лета 1945 года я и Миша Шкредов - он был двумя
классами моложе меня - помогали Ивану Терентьевичу копать картошку на огороде,
который находился где-то в направлении Агробазы. Самому эту работу ему было
сделать не под силу - у него не было левой руки. Глядя на собранную кучу
картофеля и благодаря нас за помощь, он расчувствовался так, что у него
увлажнились глаза.

\textbf{Игорь Клычев, учился в 4-й школе с 1944 по 1954 год.}

- К нам, мальчишкам послевоенной поры, где было много переростков, Иван
Терентьевич относился с большой чуткостью и заботой. В то же время был строг с
нами. Ребятам, которые были старше основной массы учеников и в учебе слабее, он
помогал в устройстве в ремесленные училища, где они оканчивали седьмой класс и
одновременно получали рабочую специальность. Летом 1951 года Иван Терентьевич
устроил нам поход на Белосарайскую Косу. Трамваем № 4 доехали до конечной
остановки в порту, а затем двинулись пешком по берегу моря, который за поселком
Песчаный в то время был пустынным. Минуя Мелекино, мы пришли в поселок
Белосарайку. Шли мы с восьми часов утра до пяти вечера, делали привалы на
берегу моря. Вдоль берега возвышались глинистые обрывы. Мы, как знаменитые
путешественники, называли понравившиеся нам места своими именами: пещера
Ружинского, пик Бондаренко, гора Щербака. Было нас в походе вместе с нашим
наставником человек восемь. Иван Терентьевич был заядлым охотником. К слову,
руку он потерял на охоте – разорвался ствол его ружья. Разместились мы на Косе
в домике его друга, тоже охотника. Нас накормили ухой, рыбой, помидорами,
арбузами. На следующий день утром мы отправились в обратный путь. Через
некоторое время директор организовал поездку в заповедник Аскания-Нова. Туда мы
уже ехали на грузовике, в кузове которого были устроены скамейки. От этой
поездки осталось много впечатлений. И ночевка в степи в скирде соломы, и
футбольный матч с ребятами из заповедника.

\textbf{Константин Фисун, выпускник 1955 года.}

– Иван Терентьевич организовал курсы автодела. Купил учебные плакаты, на
которых были нарисованы узлы автомобиля, коробка передач, двигатель и т.п. Он
пригласил для учеников преподавателя из автошколы. Говорили, что доплачивал ему
своими деньгами. Иногда озорники просили его рассказать, как будет при
коммунизме. Запомнился рассказ о тракторах. Мол, тракторами и комбайнами будут
управлять со стороны специальными пультами. А эти машины сами будут пахать, а
затем и убирать урожай. И сделано это будет для того, чтобы люди не дышали
пылью.

\textbf{Павел Левин, выпускник 1955 года.}

- В школьные годы мне поручили заведовать школьным радиоузлом. Однажды я
поставил кем-то переданную пластинку с фокстротом или танго, запрещенными
тогда. Дело происходило в 1953 или 1954 году. После того, как прозвучали первые
такты этой музыки, в радиорубку зашел Иван Терентьевич. Он попросил показать
злополучную пластинку. Посмотрел на нее и разломал об колено. Затем сказал:
\emph{\enquote{Никогда не ставь такую музыку. Тебя выгонят из школы, а меня посадят}}. По тем
временам в это можно было поверить. Я увлекался электроникой, об этом узнал наш
директор. После этого стал настойчиво советовать идти в военное
радиотехническое училище: \emph{\enquote{Ты там что-нибудь усовершенствуешь, станешь
знаменитым}}. Но мне не хотелось посвятить жизнь армии. Запомнилось еще, как
Иван Терентьевич стрелял из пневматического ружья по спичечному коробку,
поставленному к нему ребром, и попадал в него при каждом выстреле...

Нужно отдать должное Ивану Терентьевичу – он сумел собрать в школе очень
хороших педагогов. И, несмотря на проказы мальчишек из четвертой школы,
большинство из них получали добротную подготовку. По\hyp{}тому-то из питомцев его
выросло множество достойных людей и граждан. Вот кем стали ученики, чьи
воспоминания приведены здесь. \textbf{Юрий Дзюман-Грек} почти три десятилетия работал
главным конструктором отдела подвижных средств заправки и транспортирования
авиационной и космической техники, \textbf{Игорь Клычев, Павел Левин, Константин Фисун}
стали инженерами, специалистами высочайшего класса. Но кроме них эту школу
окончили \textbf{Константин Девин} – доктор наук, директор научно-исследо\hyp{}вательского
института животноводства, \textbf{Георгий Остапенко} и кандидат технических наук \textbf{Кирилл
Брызгунов,} занимавшие высокие руководящие посты на \enquote{Азовстали}, главный инженер
\enquote{Азовмаша} кандидат технических наук \textbf{Павел Нефедов,} член-корреспондент АН СССР
доктор технических наук \textbf{Анатолий Манохин,} заместитель генерального директора ПО
\enquote{Южмаш} \textbf{Станислав Тосхопаран,} аспирант нобелевского лауреата Льва Ландау,
кандидат физико-математических наук \textbf{Анатолий Москаленко,} почетный профессор
ПГТУ, кандидат технических наук \textbf{Михаил Человань,} заслуженный художник Украины
\textbf{Олег Ковалев,} кинооператор ЦТ и журналист \textbf{Юрий Коваленко,} снимавший своей
кинокамерой и в Кремле, и на Северном полюсе, побывавший на всех континентах
Земли, включая и Антарктиду, а это лишь малая часть успешных воспитанников
легендарного директора...

\textbf{Читайте также:} 

\href{https://mrpl.city/news/view/kak-google-pozdravlyaet-ukrainu-vse-dudly-ko-dnyu-nezavisimosti-za-devyat-let-foto}{%
Как Google поздравляет Украину: все дудлы ко Дню Независимости за девять лет, Богдан Коваленко, mrpl.city, 24.08.2019}

Пришлось приложить немалые усилия, чтобы собрать хоть какие-ни\hyp{}будь
биографические данные об учителе, посвятившем десятилетия воспитанию и обучению
детей. Личное дело директора школы не удалось разыскать. И тем не менее кое-что
узнать удалось. Иван Терентьевич Сидорчук родился 23 сентября 1903 года в
городе Ольгополь Подольской губернии, по современному административному делению
этот населенный пункт сейчас расположен в Винницкой области. У Ивана
Терентьевича было высшее образование, но в каком учебном заведении он его
получил, узнать не удалось. Послужной список выглядит таким образом. С сентября
1923 года по май 1927 года - заведующий Воловянской сельской семилетней школой
Ольгопольского района, с 1927 по 17 августа 1931 года - преподаватель
украинского языка в Горловской школе ФЗО, а через три дня - завуч школы № 5
Ильичевского района Мариуполя. С этого и начинается мариупольский период И. Т.
Сидорчука. В октябре 1932 года его назначают старшим преподавателем рабфака. Но
в 1940 году рабфак ликвидировали, и Ивана Терентьевича переводят в
Мариупольский металлургический техникум, где он преподает украинский язык и
литературу. А в 1941 году из учебного плана техникумов были изъяты украинский
язык и литература, а преподаватели этого предмета - уволены, в том числе и Иван
Терентьевич. Но он не остался без работы. С 1 сентября 1941 года он - завуч
школы № 4. Но на этом посту он побыл лишь месяц с небольшим. 8 октября
Мариуполь заняли немцы, он остался в оккупированном городе. 20 октября 1943
года И. Т. Сидорчук назначен директором средней школы № 10. А через год его
переводят на должность директора школы № 4. 15 августа 1955 года в его трудовой
книжке появилась запись: \emph{\enquote{Освобожден от занимаемой должности директора школы по
собственному желанию и оставлен преподавателем украинского языка и литературы
СШ № 4}}...

Иван Терентьевич Сидорчук ушел из жизни 25 сентября 1968 года. Его хоронили из
Дома учителя, который находился тогда на Митрополитской улице. Траурная
процессия растянулась на несколько кварталов, проводить его в последний путь
пришли ученики. Сразу же после похорон они решили установить достойный
надгробный памятник учителю. Вскоре художник Олег Ковалев, в свое время
учившийся в 4-й школе, выполнил гравюру и размножил ее. На ней была сделана
надпись: \emph{\bfseries\enquote{Памятник учителю И. Т. Сидорчуку}.} Каждому, кто сдавал деньги,
вручалась на память гравюра. Необходимые средства были собраны, памятник из
черного гранита был изготовлен и водружен на могиле на Новоселовском кладбище.
