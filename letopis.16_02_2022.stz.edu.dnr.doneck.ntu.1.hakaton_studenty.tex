% vim: keymap=russian-jcukenwin
%%beginhead 
 
%%file 16_02_2022.stz.edu.dnr.doneck.ntu.1.hakaton_studenty
%%parent 16_02_2022
 
%%url http://donntu.org/news/id202202161325
 
%%author_id edu.dnr.doneck.ntu
%%date 
 
%%tags donbass,hakaton,obrazovanie,programmirovanie,studenty
%%title Студенты ДонНТУ – финалисты хакатона «КиберДонбасс»
 
%%endhead 
 
\subsection{Студенты ДонНТУ – финалисты хакатона «КиберДонбасс»}
\label{sec:16_02_2022.stz.edu.dnr.doneck.ntu.1.hakaton_studenty}
 
\Purl{http://donntu.org/news/id202202161325}
\ifcmt
 author_begin
   author_id edu.dnr.doneck.ntu
 author_end
\fi

Общественное движение Народная Дружина 6 февраля подвело итоги первого в
Донецкой Народной Республике хакатона «КиберДонбасс». Конкурсантам было
предложено техническое задание на разработку программной системы «Народный
контроль» для обеспечения контроля состояния сфер жизнедеятельности города и
Республики в целом.

\ii{16_02_2022.stz.edu.dnr.doneck.ntu.1.hakaton_studenty.pic.1}

В хакатоне приняли участие студенты факультета интеллектуальных систем и
программирования, группы ПИ-19а, в составе двух команд:

- «Мастера подземелий»: Вадим Олейник, Олег Саевский;

- Hectic: Даниил Перлик, Максим Никитин, Алексей Истягин.

\ii{16_02_2022.stz.edu.dnr.doneck.ntu.1.hakaton_studenty.pic.2}

На разработку большой системы, включающей множество подсистем, было выделено
две недели. За это время команда «Мастера подземелий» заметно продвинулась в
разработках на .NET и Python. К их числу относятся:

- база данных для хранения всех новостей города, заявок на публикацию и
информацию о пользователях;

- API - специальный слой для обеспечения безопасного и надежного соединения
различных приложений и сайтов с базой данных;

- VK-бот и Telegram-бот как внешние интерфейсы пользователей, через которые
предоставляется доступ к информационной системе «Народный контроль»;

- административная панель для управления системой.

%\ii{16_02_2022.stz.edu.dnr.doneck.ntu.1.hakaton_studenty.pic.3}

Разработки команды Hectic имеют некоторые общие черты с конкурентами (возможно,
именно это означает правильность выбора пути). Для реализации следующих
подсистем были использованы .NET, Android Kotlin и React Native:

- большая база данных с возможностью дальнейшего расширения в любых
направлениях;

- обширный API, имеющий множество слоев абстракции для обеспечения большей
надежности и безопасности системы;

- мобильное приложение на Android, раскрывающее все возможности системы в
приятном и удобном графическом интерфейсе;

- административная панель для  администрирования и модерирования
пользовательских заявок.

Из двадцати команд в финал прошли лишь восемь, и в их числе обе команды
факультета интеллектуальных систем и программирования.

Студенты ДонНТУ получили практический опыт работы с новыми технологиями,
познали сложности командного взаимодействия и ощутили себя
экстрим-разработчиками.
