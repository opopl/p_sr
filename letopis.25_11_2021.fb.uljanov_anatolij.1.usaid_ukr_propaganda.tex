% vim: keymap=russian-jcukenwin
%%beginhead 
 
%%file 25_11_2021.fb.uljanov_anatolij.1.usaid_ukr_propaganda
%%parent 25_11_2021
 
%%url https://www.facebook.com/dadakinder/posts/4956612307691226
 
%%author_id uljanov_anatolij
%%date 
 
%%tags propaganda,radio_svoboda,ukraina,usa,usaid
%%title USAID и украинская пропаганда
 
%%endhead 
 
\subsection{USAID и украинская пропаганда}
\label{sec:25_11_2021.fb.uljanov_anatolij.1.usaid_ukr_propaganda}
 
\Purl{https://www.facebook.com/dadakinder/posts/4956612307691226}
\ifcmt
 author_begin
   author_id uljanov_anatolij
 author_end
\fi

\ii{25_11_2021.fb.uljanov_anatolij.1.usaid_ukr_propaganda.pic.1}

Ахаха, в рамках программы американского правительства USAID, украинский
Институт массовой информации признал официальный орган государственной
пропаганды США «Радио свобода» одним из самых честных СМИ ))) Это как если бы
Кремль дал денег на исследование, в результате которого выяснилось, что Россия
1 – светоч журналистских стандартов.

USAID прямым текстом обозначает свою функцию у себя на сайте: «продвижение
внешнеполитических интересов Америки». «Радио Свобода» – ЦРУшная креатура
времён Холодной войны, буквально на днях подстрекавшая к уголовной расправе над
украинскими студентом за «неуважительные высказывания про события в Украине». 

Обе организации являются официальными политическими инструментами империи, на
которые она имеет полное право, но информирование и «продвижение интересов» –
это, всё-таки, разные задачи. При чем тут журналистика?

В «белом списке» (списке белого господина?) указан и ряд других медиа, что
очень удобно – сразу понятно ху из ху на медиарынке. Стоит ли удивляться, что
большинство этих USAID-approved медиа выражают одну на них всех точку зрения
под одним углом?

\ii{25_11_2021.fb.uljanov_anatolij.1.usaid_ukr_propaganda.cmt}
