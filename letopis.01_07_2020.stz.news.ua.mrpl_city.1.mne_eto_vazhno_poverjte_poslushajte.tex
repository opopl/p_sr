% vim: keymap=russian-jcukenwin

%%beginhead 
 
%%file 01_07_2020.stz.news.ua.mrpl_city.1.mne_eto_vazhno_poverjte_poslushajte
%%parent 01_07_2020
 
%%url https://mrpl.city/blogs/view/mne-e-to-vazhno-poverte-poslushajte
 
%%author_id demidko_olga.mariupol,news.ua.mrpl_city
%%date 
 
%%tags 
%%title "Мне это важно! Поверьте, послушайте"
 
%%endhead 
 
\subsection{\enquote{Мне это важно! Поверьте, послушайте}}
\label{sec:01_07_2020.stz.news.ua.mrpl_city.1.mne_eto_vazhno_poverjte_poslushajte}
 
\Purl{https://mrpl.city/blogs/view/mne-e-to-vazhno-poverte-poslushajte}
\ifcmt
 author_begin
   author_id demidko_olga.mariupol,news.ua.mrpl_city
 author_end
\fi

7 липня єдина народна артистка України в Маріуполі, лауреат премії імені
Заньковецької Світлана Іванівна Отченашенко святкуватиме своє 75-річчя. У
Донецькому академічному обласному драматичному театрі (м. Маріуполь) Світлана
Іванівна працює з 1966 року, де за півстоліття творчої праці зіграла більше 200
ролей.

\ii{01_07_2020.stz.news.ua.mrpl_city.1.mne_eto_vazhno_poverjte_poslushajte.pic.1}

Життєвий і творчий шлях нашої героїні не може не вражати. Вона народилася в
Полтаві, але своїм найріднішим містом вважає Маріуполь. Саме тут вона знайшла
своє покликання, зустріла кохану людину та народила сина. Вже у 27 років
отримала звання заслуженої артистки. І це не дивно! Неповторні образи актриси
– викликали захоплення не тільки у глядачів, але й в інших акторів, режисерів
та відомих театральних критиків. Я і раніше писала про життя і творчість
видатної маріупольчанки, проте цього разу хочу розповісти про ще один, не всім
відомий, талант актриси.

З дитинства Отченашенко дуже любить читати. Вона неодноразово відзначала своє
особливе ставлення до поезії. Втім Світлана Іванівна і сама писала вірші, хоча
й не довгий період, оскільки вирішила, що вона не має великого таланту... Однак,
читаючи вірші  почесної громадянки Маріуполя, розумієш, що талановита людина,
дійсно, талановита в усьому! Поезія актриси дозволяє зрозуміти наскільки
багатогранний внутрішній світ Отченашенко. Лише один вірш Світлани Іванівни був
опублікований в газеті \enquote{Приазовський робочий}, яким актриса залюбки поділилася
і з нами:

\begin{quote}
\em
Мне это важно! Поверьте, послушайте\par
Вдруг на асфальте мокром,\par
На танцплощадке пустынной, далекой\par
Листья-веснушки кружатся, кружатся в вальсе осеннем, \par
Утром и вечером четкие такты\par
Очень беспомощно, очень доверчиво жмутся к асфальту.\par
Вы это видели? Вы это помните?\par
Кто на бульварах рассыпал золото,\par
Развешал на ветках капли громадные...\par
Вдруг сорвется и прямо на голову\par
Солнце на небе давно не видели,\par
Надо грустить, да кто это выдумал?\par
Ничто не уходит, ничто не кончается\par
Листьями снизу земля освещается.\par
Нет, не сидится в уютной комнате,\par
Мне хорошо, улыбаюсь прохожим.\par
Вы ее помните, вы ее помните? \par
Осень свою на весну похожую... \par
\end{quote}

\ifcmt
  ig https://i2.paste.pics/PO1WW.png?trs=1142e84a8812893e619f828af22a1d084584f26ffb97dd2bb11c85495ee994c5
	@caption_here Відео: Светлана Отченашенко читает свои стихи. Мариупольский драмтеатр, Маріупольське телебачення, 01.07.2020 
  @wrap center
  @width 0.9
\fi

Як наголошує актриса, їй пощастило зробити в своєму житті два правильних
вибори: обрати свою професію та зустріти надійного супутника життя. Чоловік
Світлани Іванівни – Харабет Юхим Вікторович, скульптор і медальєр, заслужений
діяч мистецтв України, – завжди надихав і підтримував у всьому кохану. В першу
розлуку, коли Отченашенко поїхала на гастролі, а мобільних телефонів тоді не
було, вона написала своєму чоловікові цей вірш:

\begin{quote}
\em
Оденусь на улицу, выскочу и побегу к телефону\par
Краткое, очень привычное в трубку скажу: \enquote{привет}!\par
Что же тут необычного, но в этом мире сонном,\par
Нет твоего телефона, и дома нашего нет.\par
Мне бы привыкнуть к этому, все далеко-далече:\par
И счастье, и расставанье, и боль тупая в груди.\par
Встряхнуть головой незаметно, бодро расправить плечи\par
И дожидаться встречи, которая впереди...\par
Но не могу привыкнуть, но не хочу поверить\par
И если в трубке молчание, тебя во всем обвиню.\par
Сейчас дождевик накину, сейчас я открою двери,\par
Сейчас на улицу выскочу и все-таки позвоню...\par
\end{quote}

Всього три вірші прочитала Світлана Іванівна під час останнього інтерв'ю зі
мною. І все ж доторкнутися до яскравого світу художнього слова видатної
маріупольчанки, на мою думку, –  унікальна можливість. До речі, всі вірші
актриса знає напам'ять, а записів з ними ніде не зберегла, тому вони
публікуються вперше. А цим віршем, як підкреслила актриса, закінчилася її
\enquote{поетична кар'єра}:

\begin{quote}
\em
Я не люблю читать свои стихи,\par
Строка, как призрак, память, ускользая, \par
Разбудит мертвый уголок души \par
И требует: пиши, пиши пиши! \par
И начинается игра немая \par
С собой на смерть жестокая игра\par
И капля крови на конце пера,\par
И сердце, задыхаясь, умирая,\par
А тело разом оторвется вдруг и взмоет \par
В высоту полетом птичьим\par
Со мной сидящий рядом чуткий друг зевнет и скажет:\par
\enquote{Ничего, прилично!} \par
Я не люблю читать свои стихи...\par
\end{quote}

І незважаючи на те, що Світлана Іванівна \enquote{не любит читать свои стихи},
впевнена, що вірші справжньої легенди Маріуполя обов'язково знайдуть відлуння у
душі читача. 

\ii{01_07_2020.stz.news.ua.mrpl_city.1.mne_eto_vazhno_poverjte_poslushajte.pic.2}
