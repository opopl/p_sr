% vim: keymap=russian-jcukenwin
%%beginhead 
 
%%file 24_12_2021.stz.news.lnr.lug_info.3.podvig_donbassa
%%parent 24_12_2021
 
%%url https://lug-info.com/news/novogodnij-blic-literator-andrej-chernov-podvig-donbassa-budet-otrazhen-v-knigah-i-fil-mah
 
%%author_id 
%%date 
 
%%tags 
%%title Новогодний блиц. Литератор Андрей Чернов: "Подвиг Донбасса будет отражен в книгах и фильмах"
 
%%endhead 
\subsection{Новогодний блиц. Литератор Андрей Чернов: \enquote{Подвиг Донбасса будет отражен в книгах и фильмах}}
\label{sec:24_12_2021.stz.news.lnr.lug_info.3.podvig_donbassa}

\Purl{https://lug-info.com/news/novogodnij-blic-literator-andrej-chernov-podvig-donbassa-budet-otrazhen-v-knigah-i-fil-mah}

Секретарь правления Союза писателей ЛНР, литератор Андрей Чернов в традиционном
новогоднем блиц-интервью ЛИЦ делится своим видением итогов уходящего года и
прогнозом на год грядущий.

\ifcmt
  ig https://storage.lug-info.com/cache/8/c/04a7b94c-238d-455b-9c91-bbb9420e6f47.jpg/w1000h616%7Cwm
  @width 0.4
  %@wrap \parpic[r]
  @wrap \InsertBoxR{0}
\fi

- Как вы оцениваете положение и ситуацию в Республиках Донбасса к началу 2022
года?

- Уходящий год был непростой. Верилось, что будут в 2021 году подвижки в мирном
процессе, что откроют новые пункты пропуска в Счастье и Золотом, что старики
Донбасса получат доступ к украинским пенсиям и документам, что Киев перестанет
продвигать свою людоедскую политику. Но гуманизм вычеркнут из лексикона
украинских политиканов. В итоге прошел год – и урегулирование конфликта в
Донбассе не продвинулось ни на йоту. Наоборот, критически запахло войной,
большой, затяжной. Видно, \enquote{Байрактары} (беспилотные летательные аппараты
Bayraktar турецкого производства) вскружили Киеву голову.

И, конечно же, огромной проблемой для Республик Донбасса является пандемия
коронавируса. Это большое испытание – и для системы здравоохранения, и для
наших граждан. Думаю, из ситуации все вынесли правильные выводы: власти – о
том, что нельзя экономить на здравоохранении, что нужно развивать медицину,
поддерживать медработников, а граждане – что общественное здоровье напрямую
зависит от каждого, что вакцинирование – оптимальный путь, который позволит нам
преодолеть эту инфекцию.

- Что вы считаете наиболее значимыми достижениями Луганской и Донецкой Народных
Республик и каковы были самые серьезные проблемы в уходящем году?

- Вопреки всему Республики Донбасса живут и укрепляют свою государственность. А
ведь Республики Донбасса – организм с простреленной грудью, истекающий кровью,
душимый жгутом блокады. Из года в год, с 2014-го. Но вопреки всему Республики
Донбасса живут. И не только выживают, но и развиваются. Учитывая их состояние –
незаживающую рану войны и блокаду – это уже подвиг. Когда-нибудь человечество
это оценит, уверен. Подвиг свободных людей Донбасса будет отражен и в книгах, и
в фильмах. И даже Запад признает в нас не парий, а истинных борцов за свою
свободу, честь, достоинство.

Ну, а пока мы живем, мы дышим, мы строим нашу страну. Особенно значительны в
этом году подвижки в сферах образования и культуры. В ЛНР полнокровно проведен
Год Владимира Ивановича Даля, нашего знаменитого земляка. Министерством
образования и науки намечены большие планы по реформированию образования,
необходимость в этом назрела давно.

- Ваш прогноз развития ситуации вокруг Донбасса в 2022 году?

- Пока нет оснований считать, что Киев изменит свою политику в будущем году. Не
следует забывать, что Киев полностью утратил даже намеки на суверенитет, он
давно уже послушно ретранслирует то, что ему прикажут в Вашингтоне. А там
заказывают все ту же пластинку: опять на Украине нагнетается русофобия, военная
истерия набирает обороты. Опять будут военные провокации, опять будет
тормозиться мирный процесс. Вашингтону выгодно обострение с Россией, и Донбасс
– одна из болевых точек Русского мира. Конечно, и в США, и на Украине наплевать
на наших стариков, на наших детей. Какие еще дети, когда они вершат
геополитику!
