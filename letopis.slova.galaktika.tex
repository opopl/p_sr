% vim: keymap=russian-jcukenwin
%%beginhead 
 
%%file slova.galaktika
%%parent slova
 
%%url 
 
%%author_id 
%%date 
 
%%tags 
%%title 
 
%%endhead 
\chapter{Галактика}
\label{sec:slova.galaktika}

%%%cit
%%%cit_head
%%%cit_pic
%%%cit_text
\emph{Галактические} зонды \enquote{Вояджер-1} и \enquote{Вояджер-2}, запущенные ещё в 1977 году,
вышли за пределы Солнечной системы, и присылают нам оттуда странные сигналы...
Учёным понадобится несколько лет и мощности самых больших суперкомпьютеров,
чтобы попытаться расшифровать, и хоть как-то интерпретировать получаемую от
Вояджеров информацию. Одно уже понятно - вырвавшись за пределы солнечного
защитного \enquote{пузыря}, зонды услышали и увидели совершенно другую Вселенную,
нежели та, что доступна наблюдению с Земли.  Оказалось, что множество
загадочных сигналов, похожих на разумные, пронизают вакуум со скоростью, в
тысячи раз превышающей скорость света. Вся современная физика летит к чертям. А
мы, в лучшем случае, похожи на современников Галлилея и Джордано Бруно, и с
религиозным восторгом подбрасываем в костёр средневековой темноты свои
убожество и глупость...  Мы боремся за чистоту нации и святость олигархических
идей, устраиваем погромы и указываем кому, кого и как правильно трахать.
Договариваемся с врагами, пасуем перед сильными, унижаем слабых и заискиваем
перед богатыми... Мы готовы уничтожить весь свой маленький, хрупкий мир из-за
жадности, подлости, и во имя призрачных химер...  ...А в это время в далёкой
\emph{Галактике} смотрят на нас с любопытством и отвращением... и может, к счастью для
нас, с состраданием... существа, живущие разумом, покончившие с болезнями,
преодолевшие смерть, покорившие пространство и время.  Они как боги. Они и есть
- боги. Какое дело им до нас?.. Какое дело нам до них, пока их не покажут в
топовой телепередаче?..
%%%cit_comment
%%%cit_title
\citTitle{Мы боремся за чистоту нации и святость идей / Лента соцсетей / Страна}, 
Юрий Касьянов, strana.news, 14.12.2021
%%%cit_url
\href{https://strana.news/opinions/367041-my-boremsja-za-chistotu-natsii-i-svjatost-oliharkhicheskikh-idej.html}{link}
%%%endcit
