% vim: keymap=russian-jcukenwin
%%beginhead 
 
%%file 06_08_2021.fb.kushnarjev_roman.1.mova
%%parent 06_08_2021
 
%%url https://www.facebook.com/roman.kushnaryov/posts/4101046229949098
 
%%author Кушнарев, Роман
%%author_id kushnarjev_roman
%%author_url 
 
%%tags mova,ukraina,ukrainizacia
%%title Вона - Мова - для всіх громадян України будь-якої національності
 
%%endhead 
 
\subsection{Вона - Мова - для всіх громадян України будь-якої національності}
\label{sec:06_08_2021.fb.kushnarjev_roman.1.mova}
 
\Purl{https://www.facebook.com/roman.kushnaryov/posts/4101046229949098}
\ifcmt
 author_begin
   author_id kushnarjev_roman
 author_end
\fi

У коментарях під моїм запитанням до адмінів сторінки новоствореного
харківського ФК "Металіст" про те, коли ж вони нарешті розчохляться і почнуть
вести офіційну сторінку українського клубу українською мовою, мені очікувано
прилетіло "дасталі вашей мовай". Я відразу автоматично відповів, що українська
мова не лише моя, але і, власне, отої коментаторки, яка теж є громадянкою
України.

А оце сьогодні ввечері коли бігав, то якось не міг викинути цей момент з
голови.

Але ж це дійсно так!

Українська мова не є екслюзивною лише для частини населення України. Вона є
інклюзивною! Вона - для всіх громадян України будь-якої національності. І це
мало б звучати з кожного холодильника й у різних формах. На мовах всіх цих
національностей. В тому числі і - найбільше! - російською. 

Всі оці от російськомовні блогери по ФБ та інших майданчиках мали б це постійно
вбивати в голови своїх читачів: що українська мова - вона не лише для заходу чи
центру, не лише для дикторів на ТБ чи радіо, не лише для шкіл і садочків, не
лише для сфери обслуговування. Вона - для всіх.

Так, я розумію, що це - копітка робота. Так, я розумію, що треба буде витримати
тиск від російських ботоферм. Але на те ж вони і ЛСД, щоб задавати напрямок, а
не просто бабратися в болоті, яке ми маємо зараз.

А то я читаю і чую багато високопарних фраз російською про патріотизм і любов
до України, її мови і культури, то хотілося б, щоб слова не розходилися з
ділом.

Нам всім треба пам'ятати, що українська мова є СВОЄЮ для всіх тих, хто називає
себе громадянами України.

\ii{06_08_2021.fb.kushnarjev_roman.1.mova.cmt}
