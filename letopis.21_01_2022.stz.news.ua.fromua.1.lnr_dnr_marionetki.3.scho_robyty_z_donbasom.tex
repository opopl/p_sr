% vim: keymap=russian-jcukenwin
%%beginhead 
 
%%file 21_01_2022.stz.news.ua.fromua.1.lnr_dnr_marionetki.3.scho_robyty_z_donbasom
%%parent 21_01_2022.stz.news.ua.fromua.1.lnr_dnr_marionetki
 
%%url 
 
%%author_id 
%%date 
 
%%tags 
%%title 
 
%%endhead 

\subsubsection{Що Україні робити з Донбасом?}
\label{sec:21_01_2022.stz.news.ua.fromua.1.lnr_dnr_marionetki.3.scho_robyty_z_donbasom}

У серпні минулого року президент Володимир Зеленський дав слушну пораду жителям
Донбасу, які відчувають себе росіянами. Він закликав їх їхати у Росію та не
мучити себе. Але навряд чи всі бажаючі бути частиною «руского міра» послухають
українського президента. Треба чесно визнати: у найближчий час Україні не
повернути окупованих територій ДНР/ЛНР. Головне завдання зараз – зупинити
ворога на нинішніх кордонах. Не дати йому поширити свій вплив на інші
українські землі.

Російська пропаганда продовжує кидати у інформаційний простір відверті фейки.
То вона лякає своє суспільство тим, що Україна от-от розпочне військову
операцію за звільнення Донбасу. То стверджує, що в нас розпочнеться
громадянська війна через те, що США введуть санкції проти російських банків.
Насправді, якщо спалахне велика війна, то розв’яже її Кремль і його сателіти на
Донбасі. Зараз в України немає ні достатніх ресурсів, ні політичної волі
керівництва, щоб звільняти Донбас.

\ii{21_01_2022.stz.news.ua.fromua.1.lnr_dnr_marionetki.3.scho_robyty_z_donbasom.pic.1}

Навіть якщо станеться диво і окуповані території ДНР і ЛНР увійдуть до складу
України, буде потрібно багато часу, щоб позбутися наслідків промивання масової
свідомості на цих теренах. Інакше Донбас перетвориться на порохову бочку у
складі України. Якщо строго слідувати букві та духу Мінських угод, сумнівно, що
нам вдасться здійснити успішну реінтеграцію частини Донбасу. Потрібно шукати
інші варіанти.

Світ намагається стримати Росію від ескалації дипломатичними методами. Також
Москву постійно попереджають: агресія матиме дуже погані наслідки. Та поки що
неясно, чи вдасться напоумити Путіна. Але Україна могла б скористатися
моментом, щоб позбутися невигідних для себе договорів на зразок Мінських
домовленостей. Визнання незалежності ДНР і ЛНР Росією – абсолютна ознака, що
минулі угоди зазнали повного фіаско. Продовжувати наполягати на виконанні
Мінська в цих умовах – це як перебувати у паралельній реальності.

З іншого боку, якщо ескалація відбудеться, то ДНР/ЛНР майже точно братимуть у
ній участь. Це ще один привід відмовитися від угод, які загрожують цілісності
України і нереальні для виконання. Бо про який особливий статус ДНР і ЛНР в
складі України може йти мова? Чи про яку повну амністію та місцеві органи
влади, якщо в них працюють сепаратисти?

Україні потрібен Донбас. Потрібні всі території, які перебувають під окупацією
російських сил та їх пособників. Але нам точно не потрібні російські бойовики
та громадяни, які люто ненавидять усе українське. Саме з цієї формули потрібно
виходити, коли настане час інтеграції ДНР та ЛНР у склад України.
