% vim: keymap=russian-jcukenwin
%%beginhead 
 
%%file 06_01_2022.stz.news.lnr.mirlug.1.tamozhenniki_jolka_zhelanij
%%parent 06_01_2022
 
%%url https://mir-lug.info/novosti-proektov/tamozhenniki-lnr-ispolnili-tri-novogodnih-zhelaniya-detej-iz-mnogodetnyh-semej
 
%%author_id news.lnr.mirlug
%%date 
 
%%tags 
%%title Таможенники ЛНР исполнили три новогодних желания детей из многодетных семей
 
%%endhead 
\subsection{Таможенники ЛНР исполнили три новогодних желания детей из многодетных семей}
\label{sec:06_01_2022.stz.news.lnr.mirlug.1.tamozhenniki_jolka_zhelanij}

\Purl{https://mir-lug.info/novosti-proektov/tamozhenniki-lnr-ispolnili-tri-novogodnih-zhelaniya-detej-iz-mnogodetnyh-semej}
\ifcmt
 author_begin
   author_id news.lnr.mirlug
 author_end
\fi

Шары с новогодними желаниями ребят снял с «Ёлки желаний» Председатель
Государственного таможенного комитета ЛНР Юрий Афанасьевский. В этот же день, 6
января, желания детей были исполнены.

\ii{06_01_2022.stz.news.lnr.mirlug.1.tamozhenniki_jolka_zhelanij.pic.1}

В семье Щербаковых четверо детей. Восьмилетний Владик загадал желание получить
в подарок радиоуправляемого робота, а для своей одиннадцатилетней сестрёнки
Виктории он попросил у Деда Мороза большую мягкую игрушку. Эти два желания
исполнили таможенники Республики. Также не смогли оставить без гостинца и
маленького Ванечку, которому всего полтора годика. Старший брат Даня загадывал
своё отдельное желание, которое уже исполнили представители другого ведомства.

\ii{06_01_2022.stz.news.lnr.mirlug.1.tamozhenniki_jolka_zhelanij.pic.2}


О геймерских наушниках с микрофоном двенадцатилетний Егор из поселка Челюскинец
Лутугинского района мечтал уже давно. И вот под Рождество желание мальчика
исполнилось. Председатель ГТК ЛНР Юрий Афанасьевский вручил Егору долгожданный
подарок.

– Новый год и Рождество Христово время настоящих чудес, в которые верят не
только дети, но и взрослые. Это особое ощущение праздника. И мне очень приятно
погрузиться в эту чудесную атмосферу. Исполняя детские желания, мы укрепляем
веру в добро, в хорошие начинания, своим примером показываем, что благие
начинания делают людей лучше, человечнее. Это то, чего, к сожалению, в нашем
мире многим не хватает, – отметил Юрий Афанасьевский.

\ii{06_01_2022.stz.news.lnr.mirlug.1.tamozhenniki_jolka_zhelanij.pic.3}

Бабушка Щербаковых Людмила Борисовна поблагодарила инициаторов прекрасной
новогодней акции «Ёлка желаний» и таможенный комитет ЛНР за подарки внукам.
Когда писали письма с желаниями, мы и подумать не могли, что всё исполнится.
Это же ведь настоящее чудо! Дети очень довольны. Спасибо всем вам! – поделилась
эмоциями Людмила Борисовна.

Напоминаем, период исполнения желаний в акции «Ёлка желаний» продлится по 28
февраля 2022 года. Все желающие смогут снять шарик с ёлки, которая находится по
адресу: г. Луганск, ул. Карла Маркса, 7, и выполнить желание ребёнка. Время
работы ёлки желаний по будням, с 9.00 до 18.00.
