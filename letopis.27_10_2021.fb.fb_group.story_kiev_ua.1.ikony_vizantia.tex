% vim: keymap=russian-jcukenwin
%%beginhead 
 
%%file 27_10_2021.fb.fb_group.story_kiev_ua.1.ikony_vizantia
%%parent 27_10_2021
 
%%url https://www.facebook.com/groups/story.kiev.ua/posts/1784380221758771
 
%%author_id fb_group.story_kiev_ua,denisova_oksana.kiev.ukraina.gid
%%date 
 
%%tags gorod,ikona,istoria,kiev,vizantia
%%title Киевская история в духе Дэна Брауна...
 
%%endhead 
 
\subsection{Киевская история в духе Дэна Брауна...}
\label{sec:27_10_2021.fb.fb_group.story_kiev_ua.1.ikony_vizantia}
 
\Purl{https://www.facebook.com/groups/story.kiev.ua/posts/1784380221758771}
\ifcmt
 author_begin
   author_id fb_group.story_kiev_ua,denisova_oksana.kiev.ukraina.gid
 author_end
\fi

Киевская история в духе Дэна Брауна... 

Когда-то в далеких 6-7 веках в Византии писали удивительные иконы, писали их на
простых досках и расплавленным воском наносили краски, которые поражали
яркостью и красотой! Техника эта называлась энкаустика. Через 200 лет после их
создания Византию охватило иконоборчество. Ссылаясь на Ветхий завет, императоры
признали поклонение иконам идолопоклонством. Иконы, мозаики, фрески,
христианские скульптуры уничтожались по всей стране, за их хранение казнили.
Уничтожены были и энкаустические иконы... почти все, осталось всего 12 икон и
уцелели  они только потому, что хранились в Египте. 8 икон по-прежнему хранятся
в монастыре на Синае, а вот 4 из этих 12 икон хранятся в Киеве и их можно
увидеть!

\ii{27_10_2021.fb.fb_group.story_kiev_ua.1.ikony_vizantia.pic.1}

В середине 19 века епископ Порфирий Успенский, ученый и знаток  иконописи,
будущий настоятель киевского Михайловского Златоверхого монастыря, путешествует
по Палестине и Египту. Во время путешествия он  попадает в Синайскую обитель
Святой Екатерины и в старинной башне находит 4 древние иконы. Он описывает их,
и синайцы дарят эти иконы Порфирию, а он привозит их в Киев и передает в дар
Киевскому братскому монастырю. 

\begin{multicols}{2} % {
\setlength{\parindent}{0pt}
\ii{27_10_2021.fb.fb_group.story_kiev_ua.1.ikony_vizantia.pic.2}
\end{multicols} % }

Представьте, из 12 сохранившихся в мире старинных икон  4 в конце 19 века
попадают в Киев, а впереди войны и революции. Вы мне не поверите, но они
сохранились и чтобы их посмотреть, не нужно ехать в Синай или лезть на
старинную башню, нужно просто прийти в Музей Ханенко на улице Терещенковской,
зайти в зал и увидеть эти  необыкновенные иконы 6-7 веков. Поверьте, они просто
завораживают...

\ii{27_10_2021.fb.fb_group.story_kiev_ua.1.ikony_vizantia.cmt}


