% vim: keymap=russian-jcukenwin
%%beginhead 
 
%%file 23_12_2020.stz.news.ua.mrpl_city.1.ljudmyla_kolosovych
%%parent 23_12_2020
 
%%url https://mrpl.city/blogs/view/lyudmila-kolosovich-vidchuvayu-shho-povinna-buti-tut
 
%%author_id demidko_olga.mariupol,news.ua.mrpl_city
%%date 
 
%%tags 
%%title Людмила Колосович: "Відчуваю, що повинна бути тут!"
 
%%endhead 
 
\subsection{Людмила Колосович: \enquote{Відчуваю, що повинна бути тут!}}
\label{sec:23_12_2020.stz.news.ua.mrpl_city.1.ljudmyla_kolosovych}
 
\Purl{https://mrpl.city/blogs/view/lyudmila-kolosovich-vidchuvayu-shho-povinna-buti-tut}
\ifcmt
 author_begin
   author_id demidko_olga.mariupol,news.ua.mrpl_city
 author_end
\fi

\ii{23_12_2020.stz.news.ua.mrpl_city.1.ljudmyla_kolosovych.pic.1_2}

2020 рік став для Маріуполя багато в чому переломним, приголомшливим,
стресовим... Втім в цей непростий час відбувалися і позитивні зміни. Зокрема, до
нашого міста приїхала талановита режисерка, заслужена артистка України \emph{\textbf{Людмила
Леонідівна Колосович}}, чиї новаторські вистави стали справжньою знахідкою для
маріупольського глядача. Режисерка наголошує, що відчуває, що на сьогодні
повинна бути саме тут, на Донеччині, в маріупольському драматичному театрі...

\ii{23_12_2020.stz.news.ua.mrpl_city.1.ljudmyla_kolosovych.pic.3}

Народилася Людмила у селищі Билбасівка (Слов'янський район Донецької області).
Ходила в українську школу. Маленька Люда дуже любила співати. До речі, чудовий
голос дівчинка успадкувала від мами та бабусі, які були дуже співочими.
Незважаючи на те, що театральний світ відкрився Люді не одразу, її бажання
співати і розкутість на сцені, допомогли батькам обрати спеціальність для
улюбленої доньки. У 15 років дівчина вирішила вступати до Дніпропетровського
державного театрального училища. Однак з першого разу вступити не вдалося. Слід
враховувати, що на той час на вступних іспитах до театральних навчальних
закладів студентів відбирали дуже прискіпливо, був великий конкурс. Провалитися
на вступі було дуже болісно для Людмили. Ще й у селищі насміхалися, чутки
швидко розповзалися про \enquote{артистку}. Втім завзята Люда не опустила руки - і вже
через рік була першою у списку зарахованих. 

\ii{23_12_2020.stz.news.ua.mrpl_city.1.ljudmyla_kolosovych.pic.4}

Загалом наша героїня завжди
навчалася \enquote{на відмінно}, що приносило їй неабияке задоволення, адже навчатися
вона дуже любила. Хоча і зараз мисткиня продовжує з радістю вивчати щось нове.
Зокрема, Людмила Леонідівна не полишає думки вивчити англійську, французьку,
іспанську мови, тому у вільний час із задоволенням перечитує конспекти, вчить
мови за допомогою ютубу. Полюбляє відкривати для себе нові культури та звичаї
інших народів. Мріє поставити виставу за кордоном, тому знання мов вважає
необхідним.

Після закінчення училища понад двадцять років пропрацювала у Львівському театрі
юного глядача. Цікаво, що Люда була однією з трьох студентів, яка отримала
направлення до найпрестижнішого Санкт-Петербурзького (тоді – Ленінградського)
державного інституту кіно і телебачення. Після закінчення театрального училища
треба було зачекати 4 місяці, аж поки в Ленінграді почнеться вступна компанія.
Але дівчина не хотіла чекати, вона хотіла працювати в театрі, тому поїхала до
Львова за розподілом. Це місто для Людмили на багато років стало найріднішим. З
ним пов'язана і велика трагедія, що загартувала Люду та зробила її сильнішою. І
позбавила найбажанішої мрії у житті – навчатися у ЛГІТМІКу. На першій виставі у
Львівському театрі юного глядача вона виконувала складний трюк на цирковій
трапеції. Під час репетиції 19-ти річна дівчина мало не загинула. Чотири рази
вона виконувала трюк ідеально, але на п'ятий – машиністи сцени не догледіли за
трапецією, Люда впала з 5-метрової висоти... та зламала хребет. Складно уявити,
що тоді пережила актриса.  Адже, щоб видихнути і вдихнути повітря після
падіння, їй знадобилося багато сил. Люда отримала велику дозу опромінення під
час встановлення діагнозу. Лікарі стверджували, що вона не зможе мати дітей.
Пів року витяжки і необхідність постійно лежати рівно, не перевертаючись ні
ліворуч, ні праворуч – виробили у дівчини шалену силу волі. А потім ще пів року
реабілітації, спроби навчитися самостійно ходити. Мабуть, ця сторінка біографії
надихнула режисерку на створення вистави \enquote{Фріда}, адже Людмила Колосович дуже
добре розуміла, що довелося пережити художниці. На той час батьки дівчини були
далеко і Людмилі допомагали лише колеги по театру. Попри важкий стан актриси в
цей час до неї ходило багато залицяльників. Отже, зламаний хребет не зламав
Люду. Від інвалідності вона відмовилася, бо була гордою. Ця травма 15 років не
полишала режисерку. Але все  ж таки у 20 років вона виконала трюк, який забрав
у Колосович рік життя та попросила зняти її з ролі. Незважаючи на невтішні
прогнози лікарів Людмила Леонідівна незабаром стала мамою і народила синочка
Станіслава, який, до речі, теж став актором. А через 17 років у неї народилася
донечка Юлія, яка наразі вчиться на психолога.

\ii{23_12_2020.stz.news.ua.mrpl_city.1.ljudmyla_kolosovych.pic.5}

У Львівському театрі юного глядача актриса зіграла всі головні ролі. Згодом
вирішила навчатися далі. На той час була вже заслуженою артисткою України.
Вступила на художньо-педагогічний факультет Рівненського державного
гуманітарного університету, кафедру театральної режисури. Тоді для Людмили
Леонідівни це була закрита професія, вона нічого в ній не розуміла. Хоча від
самого початку, коли працювала над своїми головними ролями, то практично сама
собі їх вибудовувала, без режисера. А коли стала готувати свої перші
студентські режисерські роботи, то зрозуміла, що це надзвичайно цікаво. Для неї
це було щось нове: занурюватися у вивчення психології, історії, музики,
образотворчого мистецтва, а потім складати своєрідні пазли з набутої
інформації.

\ii{23_12_2020.stz.news.ua.mrpl_city.1.ljudmyla_kolosovych.pic.6}

Дипломною роботою Людмили Колосович стала вистава за\par\noindent п'єсою сучасного
білоруського драматурга \emph{Діани Балико \enquote{Білий ангел з чорними крилами}}, яку вона
поставила у Львівському театрі Західного оперативного командування (тепер
Львівський драматичний театр імені Лесі Українки). Це була вистава про дівчину,
яка захворіла на СНІД. Постановка була складна, ймовірно не все вдалося, проте
вистава мала шалений глядацький успіх, завжди збирала великі зали. Особливо
подобалася молоді. 32 роки свого життя режисерка присвятила львівським театрам.
Останніх 5 років була художнім керівником Львівського драматичного театру імені
Лесі Українки. Поставила там 15 вистав. Це був один з найцікавіших періодів її
життя. Отримала запрошення від директора театру М. М. Лисюка стати художнім
керівником театру, який майже не функціонував. Це був ліквідований
Міністерством оборони колишній Львівський драматичний театр ЗахОК. Трупа
розбалансована, практично не було молоді і, відповідно, не було репертуару. До
того ж, працівникам театру потрібно було зробити перехід з російської на
українську мову. Одночасно з художнім керівництвом театру вона почала викладати
в ЛНУ ім. Івана Франка, випустила акторський курс, який став основою трупи
цього театру.

\ii{23_12_2020.stz.news.ua.mrpl_city.1.ljudmyla_kolosovych.pic.7}

Вона створила репертуарний театр, єдиний на той час україномовний театр ім.
Лесі Українки. Часто їздила на міжнародні театральні фестивалі, побувала за
кордоном. До театру пішов глядач. Це було п'ять років успішного розвитку. За
зароблені кошти зробили ремонт в театрі. Але змінилося керівництво театру, а
згодом закінчився і контракт Людмили Леонідівни. Контракт у театрі – цікава
штука. Він не дає змоги розслабитися ні акторам, ні режисерам. Наша героїня
вважає, що необхідно завжди залишатися конкурентоспроможною і не зупинятися на
досягнутому.

Після цього працювала у Луцьку, Дніпрі, Хмельницькому, Черкасах,
Кропивницькому, Києві. Досить плідним був період роботи в Дніпрі, де вона
поставила шість вистав, побувала зі своїми виставами на міжнародних театральних
фестивалях у Кракові, Херсоні, Києві. Ніколи не планувала залишатися в якомусь
театрі надовго, адже не могла забути львівську історію. Настав час переїхати до
Києва.

\ii{23_12_2020.stz.news.ua.mrpl_city.1.ljudmyla_kolosovych.pic.8}

Сьогодні у режисерки є власний \emph{театр \enquote{Solo}}, який вона створила для душі. Цей
театр не має постійного місця розташування, він подорожує разом із Людмилою
Леонідівною та завжди готовий до нових експериментів. Жінка розуміє, що тільки
в театрі вона може бути по-справжньому щасливою. Найбільше Людмилі Леонідівні
подобається ставити вистави авторів різних національностей, таким чином
відкриваючи для себе цікавий світ національних культур. Режисерським роботам
притаманне вдале поєднання авангардизму із реалістичністю. Зараз у Донецькому
академічному обласному драматичному театрі (м. Маріуполь) мисткиня ставить
театральне дійство \emph{\enquote{Різдвяна коляда}}, яке складається з колоритних автентичних
колядок та щедрівок Донеччини. З цією постановкою маріупольські актори будуть
подорожувати населеними пунктами прифронтової зони. Сподіваюся, що і
маріупольці її побачать.

\ii{23_12_2020.stz.news.ua.mrpl_city.1.ljudmyla_kolosovych.pic.9}

Нещодавно Людмила Колосович перехворіла на COVID-19. Захворіла ще в Києві.
Хвороба протікала важко, частими були панічні атаки. \emph{Наша героїня радить всім у
разі захворювання зберігати спокій та дотримуватися режиму, прописаного
лікарями. Головне пам'ятати, що в цій ситуації порятунок залежить від нас
самих, тому потрібно взяти себе в руки  і не впадати у відчай.}

\ii{23_12_2020.stz.news.ua.mrpl_city.1.ljudmyla_kolosovych.pic.10}

\emph{\textbf{Улюблене місце в Маріуполі:}} Міський парк, пляж Піщанка.

\emph{\textbf{Улюблена книга:}}

\begin{quote}
Майже не читає, береже зір. Натомість, режисерка слухає аудіокниги. Любить
слухати лекції Паоли Волкової. З останніх прослуханих книг – Карел Чапек, а
також Антон Павлович Чехов. Вважає  їх дуже сучасними письменниками. Любить
Чапека за гротесковість (улюблений жанр Колосович), а Чехова – за душевність та
гумор.
\end{quote}

\ii{23_12_2020.stz.news.ua.mrpl_city.1.ljudmyla_kolosovych.pic.11}

\emph{\textbf{Улюблений фільм:}}

Обожнює фільмографію видатного іспанського кінорежисера Педро Альмодовара.
Улюблений фільм \enquote{Поговори з нею}, \enquote{Матадор}. Наразі переглядає фільмографію
італійського кінорежисера Паоло Соррентіно.

\emph{\textbf{Улюблена акторка:}}

\begin{quote}
американська кіноактриса Меріл Стріп. Переглянула всі її фільми. На думку
Людмили Леонідівни, \emph{\enquote{передивлятися роботи видатних акторів – дуже
корисно. Особливо для акторів. Таким чином можна підвищити свій
професійний рівень, навчитися діяти в кадрі чи на сцені, а не просто
говорити текст}.}
\end{quote}

\emph{\textbf{Хобі:}} велоспорт. Обожнює подорожувати. Мріє об'їздити весь світ.

\emph{\textbf{Порада маріпольцям:}}

\begin{quote}
\em\enquote{Не припиняйте вчитися і відкривати для себе новий світ, нові культури.
Розвивайтеся та самовдосконалюйтеся!}.
\end{quote}
