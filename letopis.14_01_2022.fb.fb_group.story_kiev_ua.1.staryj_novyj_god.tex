% vim: keymap=russian-jcukenwin
%%beginhead 
 
%%file 14_01_2022.fb.fb_group.story_kiev_ua.1.staryj_novyj_god
%%parent 14_01_2022
 
%%url https://www.facebook.com/groups/story.kiev.ua/posts/1839891796207613
 
%%author_id fb_group.story_kiev_ua,goncharova_tatjana
%%date 
 
%%tags kiev,kievljane,novyj_god,novyj_god.staryj,pozdravlenie,prazdnik
%%title С наступающим Cтарым Новым годом!
 
%%endhead 
 
\subsection{С наступающим Cтарым Новым годом!}
\label{sec:14_01_2022.fb.fb_group.story_kiev_ua.1.staryj_novyj_god}
 
\Purl{https://www.facebook.com/groups/story.kiev.ua/posts/1839891796207613}
\ifcmt
 author_begin
   author_id fb_group.story_kiev_ua,goncharova_tatjana
 author_end
\fi

С наступающим Cтарым Новым годом!

В Старый Новый год хочется пожелать как можно больше исполнившихся старых
желаний. 

\ii{14_01_2022.fb.fb_group.story_kiev_ua.1.staryj_novyj_god.pic.1}

Пусть по-новому блестят счастьем глаза, по-новому крепнет здоровье, появляются
новые желания и цели. 

И пусть друзья будут по-старому преданными, верность любимых — так же
безгранична, а жизнь — прекрасна! 

И мира! Самого обычного, настоящего мира!

И открою всего несколько стареньких коробочек с моими детскими новогодними
стеклянными игрушками, которые напоминают мне праздники у нас дома, в квартире
моей бабушки Гончаровой Марии Андреевны, киевский Подол.

Подол нашего детства. Как сегодня уничтожают и рушат его историю...

Но пусть всё светлое и доброе живет, радует новые поколения!

История, древняя история Киевской Руси и следующих поколений на этой земле,
побольше ей сил!


Рядом с домом грохочет трамвай.

Под землей уже слышен шум вагонов нового метро, напротив дома станция Красная
площадь.

Дома нашего двора ещё живы, часть отреставрированы и коммунальные квартиры
перестроены в отдельные. В наших квартирах печки не топят дровами, теперь у нас
газ. В отреставрированных уже горячая вода и отопление.

Наша квартира на втором этаже.

В общем коридоре лестница на чердак, на крышу. 

\ii{14_01_2022.fb.fb_group.story_kiev_ua.1.staryj_novyj_god.pic.2}

В квартире мебель, которую ещё до войны успел для нас сделать мой дедушка.
Сервант, комод, круглый стол и стулья, шкаф для книг, тумбочки.

Каждый год моя настоящая ёлка. Она такая большая, нарядная, в папиных и моих
игрушках.

Белые стены, высокие потолки и горячая газовая печка до потолка.

На наших окнах ленинградская хлопковая тюль в красивые цветочки, шторы и ещё
снежинки на стекле, которые мы с мамой вырезали из салфеток.

А ещё запах мандарин, тогда мы делились дольками...

Счастливые лица родных и соседей, мы поздравляем друг друга!

С Новым годом, окончательно и бесповоротно наступившим! 

Щедруем, а утром засеваем!

Пусть живут наши киевские истории, наш родной Город!
