% vim: keymap=russian-jcukenwin
%%beginhead 
 
%%file 14_01_2022.fb.sebastjanovich_pavel.1.sporidnenist
%%parent 14_01_2022
 
%%url https://www.facebook.com/p.sebastianovich/posts/5107698669264745
 
%%author_id sebastjanovich_pavel
%%date 
 
%%tags nacia,obschestvo,ukraina
%%title Есть хорошее украинское слово спорідненість
 
%%endhead 
 
\subsection{Есть хорошее украинское слово спорідненість}
\label{sec:14_01_2022.fb.sebastjanovich_pavel.1.sporidnenist}
 
\Purl{https://www.facebook.com/p.sebastianovich/posts/5107698669264745}
\ifcmt
 author_begin
   author_id sebastjanovich_pavel
 author_end
\fi

Есть хорошее украинское слово спорідненість. Это то, чего нам сегодня не
хватает в проявлениях, в общении. Мы это чувствуем, тянемся друг к другу,
чувствуем взаимозависимость, но не выражаем. 

Русскоязычные и украиномовные, богатые и бедные, чиновники и граждане, взрослые
и дети. Мы в большинстве своём колючие. Колем друг друга, даже если хотим
обняться. Нет в обществе камертона, который задаст нужный тон. Камертона любви,
сострадания, уважения.

Органично было бы, чтобы такой тон задали первые лица страны. Но кроме
нагнетания страха и возбуждения ненависти керманычи ни на что не способны.
Тогда источником этого гармоничного тона станет кто?

Воля, как личное устремление, вроде есть. Воля, как пространство для выражения,
тоже есть. Воля, как свобода самовыражения, тоже никем не зажата. Так почему не
выражаем этого единения? Почему не проявляемся?

\ii{14_01_2022.fb.sebastjanovich_pavel.1.sporidnenist.cmt}
