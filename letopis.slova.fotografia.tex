% vim: keymap=russian-jcukenwin
%%beginhead 
 
%%file slova.fotografia
%%parent slova
 
%%url 
 
%%author 
%%author_id 
%%author_url 
 
%%tags 
%%title 
 
%%endhead 
\chapter{Фотография}

%%%cit
%%%cit_head
%%%cit_pic
\ifcmt
  tab_begin cols=3
		width 0.33

     pic https://strana.ua/img/forall/u/0/36/233941014_100725945642855_4626010813299551360_n(1).jpg

     pic https://strana.ua/img/forall/u/0/36/2021-08-09_11h21_23.png

		 pic https://strana.ua/img/forall/u/0/36/2021-08-09_10h46_07.png

  tab_end
\fi
%%%cit_text
Что известно про совместное \emph{фото}.  7 августа легкоатлетка Ярослава
Магучих завоевала бронзовую медаль по прыжкам в высоту, показав результат в 2
метра.  Более успешным стало выступление австралийской спортсменки Николы
Макдермотт, которая выдала результат в 2,02 м. И россиянки Марии Ласицкене,
которая прыгнула на 2 метра 4 сантиметра и заняла "золото".  Сразу после
исполнения прыжков легкоатлетки не сдерживали своих эмоций, взяли флаги своих
стран и сделали совместные \emph{дружеские фотографии}, которые опубликовали
мировые \emph{фотоагентства}
%%%cit_comment
%%%cit_title
\citTitle{\enquote{Поддерживают дискурс вражды}. Как власти кошмарят олимпийскую призерку Магучих за фото с россиянкой}, 
Анна Копытько, strana.ua, 09.08.2021
%%%endcit

%%%cit
%%%cit_head
%%%cit_pic
\ifcmt
  pic https://strana.ua/img/forall/u/0/36/rue96bf96f967.jpg
  width 0.4
\fi
%%%cit_text
Тогда сеть заполонили не только \emph{фото}, но видео и гиф двукратного чемпиона
Европейских игр Верняева с российским спортсменом.
Белявский тогда сказал, что Верняев "победил на все сто процентов" и рассказал
о том, что они дружат с украинскими спортсменами. "Мы действительно с ними
хорошо общаемся, дружим, нам не обязательно смотреть друг на друга косо.
Показушничать для СМИ нам смысла не было", - заявил Давид Белявский
%%%cit_comment
%%%cit_title
\citTitle{\enquote{Поддерживают дискурс вражды}. Как власти кошмарят олимпийскую призерку Магучих за фото с россиянкой}, 
Анна Копытько, strana.ua, 09.08.2021
%%%endcit
