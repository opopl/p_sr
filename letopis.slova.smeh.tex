% vim: keymap=russian-jcukenwin
%%beginhead 
 
%%file slova.smeh
%%parent slova
 
%%url 
 
%%author 
%%author_id 
%%author_url 
 
%%tags 
%%title 
 
%%endhead 
\chapter{Смех}
\label{sec:slova.smeh}

%%%cit
%%%cit_head
%%%cit_pic
%%%cit_text
Крымский полуостров, над которым \emph{посмеялся} на днях секретарь СНБО
Алексей Данилов, продолжает бороться с последствиями наводнения.  Дожди пока
что перестали заливать Южный берег Крыма и пик наводнения вроде как пройден. Но
спасательные работы еще в самом разгаре.  Местное население, коммунальщики,
военные и все неравнодушные бросились откачивать воду из затопленных домов,
убирать мусор с улиц и перегораживать горные реки, которые вышли из берегов и
продолжают извергать воду на города и поселки ЮБК.  Черное море на побережье в
одночасье замусорилось, купаться во многих местах нельзя. Но запреты некоторые
туристы игнорируют даже в Ялте - не пропадать же отпуску.  Зато теперь
полуостров гарантированно обеспечен водой до конца года, а то и гораздо дольше.
И это уже не шутка.  \enquote{Страна} разобралась, что сейчас происходит на
южном берегу и будут ли еще дожди
%%%cit_comment
%%%cit_title
\citTitle{Ялта - наводнение в Крыму. Что происходит 21 июня и как ликвидируют катастрофу}, 
Максим Минин, strana.ua, 21.06.2021
%%%endcit

%%%cit
%%%cit_head
%%%cit_pic
%%%cit_text
КИРКОРОВ. Угрожал национальной безопасности с 25 июня по 27 июня 2021 года.
Два дня. Внесение артиста в список «угрожающий» это \emph{смешно}. Но ещё
\emph{смешнее} как в ручном режиме СБУ исключает его из списка.  После видео
самого артиста «это не меня, это всю Россию в список внесли».  Получается,
одной волне первого можно как внести человека, так и исключить.  Волей
августейшего.  И, что важно, Зеленский и дальше меряет себя личными
отношениями, а не государством. Его советники пустые.  Ну предложите в суд
подать, а в суде пускай он выиграет допустим. И прецедент, и не вы, а суд
виноват.  Но нет
%%%cit_comment
%%%cit_title
\citTitle{Внесение Киркорова в угрожающий список - это уже смешно / Лента соцсетей / Страна}, 
Дмитрий Раимов, strana.ua, 28.06.2021
%%%endcit

%%%cit
%%%cit_head
%%%cit_pic
%%%cit_text
Та чи програвали вони, чи вигравали, малому було однаковісінько. Бо їхати верхи
на таткові йому було так само весело, як і в трамваї, а то й ще веселіше.
Головне — що вони вдвох, що неділя, що Ервін із мачухою десь далеко-далеко,
мовби їх і зовсім нема.  Але, на жаль, решту шість днів тижня вони були... І
Тімові тоді жилося достоту як тим дітям у казці, що мали лиху мачуху. Тільки
Тімові було навіть гірше, бо ж казка — то казка та й годі, вона починається на
першій сторінці й закінчується щонайпізніш на дванадцятій. А терпіти отаку
халепу день у день, рік у рік — це вам неабищиця. І якби не оті неділі, Тім,
певно, просто навсупереч усім зробився б справжнім зухвалим халамидником. А так
він усе ж лишався життєрадісним хлопчиком і не забував свого \emph{сміху},
дзвінкого заливистого \emph{сміху}, що починався десь глибоко в животі й
кінчався кумедним "ік!"
%%%cit_comment
%%%cit_title
\citTitle{Тім Талер, або Проданий сміх}, Джеймс Крюс
%%%endcit
