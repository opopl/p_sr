% vim: keymap=russian-jcukenwin
%%beginhead 
 
%%file 11_09_2021.stz.news.ua.tyzhden.1.kraina_balzak_vozrast
%%parent 11_09_2021
 
%%url https://tyzhden.ua/Columns/50/252991
 
%%author_id nestelejev_maksim
%%date 
 
%%tags ukraina,strana
%%title Країна бальзаківського віку
 
%%endhead 
\subsection{Країна бальзаківського віку}
\label{sec:11_09_2021.stz.news.ua.tyzhden.1.kraina_balzak_vozrast}

\Purl{https://tyzhden.ua/Columns/50/252991}

\ifcmt
 author_begin
   author_id nestelejev_maksim
 author_end
\fi

Матеріал друкованого видання \textSelect{№ 36 (719) від 8 вересня}

\textSelect{Відгриміли паради, відлунали привітання, минули святкування і настав час
визнати сувору правду: нам уже тридцятка. Тож виправдовуватися, що ми молода
держава, вже пізно. Бо настав наш бальзаківський вік, як заведено називати таку
кількість років після виходу роману Оноре де Бальзака «Тридцятирічна жінка»}

Три декади незалежності України — радісна віха для нас і прикре непорозуміння
для наших супостатів, які впродовж віків «неситим оком зазирали» на нашу
«оновлену землю», відмовляючись бачити тут окрему державу і націю. Крім того,
нашу боротьбу за незалежність завжди ускладнювали свої ж таки краяни:
запроданці, корисні ідіоти, зрадники та загалом «несвідомі людські маси». Й у
першій половині ХХ століття це різностороннє протистояння мало безліч визначних
битв і вирішальних моментів, які найкраще змогли зафіксувати не підручники з
історії, а мистецтво.

У 1907 році Винниченко в назві оповідання «Уміркований і щирий» спохопив ті два
стародавні психологічні типи, які, живучи в Україні, по-різному бачать роль
українця. Знаково, що двоє персонажів потрапили до в’язниці зовсім не через
національне питання, а через те, що надмірна балакучість одного та мовчазна
згода другого роблять із них соціалістів, ворогів, яких тоді боялися більше за
всяких там українців. Щирий «надміру любить Україну», а «уміркований»
переконаний, що «нам, українцям, воно якраз і не до лиця. Нам більше треба
політикою брать: там промовчати, там ухилитися, там потанцювати під дудку
дужчих». І радує хіба те, що друзі трималися разом до кінця, а щирому ще й
попротестував у буцегарні: «Геть, чортова кацапня, з наших українських тюрмів!
Чого поналазили сюди?!» Уже за тридцять років таке покричати він не зміг би, бо
його змусили б замовкнути кулею в потилицю.
