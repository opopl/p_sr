% vim: keymap=russian-jcukenwin
%%beginhead 
 
%%file 04_01_2019.stz.news.ua.lb.1.arhitekturnyj_atlas_dorevoljucijnogo_mariupolja.10.duhovnoje_uchilische
%%parent 04_01_2019.stz.news.ua.lb.1.arhitekturnyj_atlas_dorevoljucijnogo_mariupolja
 
%%url 
 
%%author_id 
%%date 
 
%%tags 
%%title 
 
%%endhead 

\subsubsection{Духовное училище}

Мало кто знает, что строительство здания по адресу ул. Митрополитская 37/52
окончилось почти полтора века назад. 11 сентября 1880 года епископ
Екатеринославский и Таганрогский преосвященный Феодосий провел торжественное
освящение постройки, предназначенной для духовного училища. На сегодняшний день
это самый старый дом в Мариуполе, имеющий подтвержденную дату строительства. С
установлением в Мариуполе советской власти, духовное училище прекратило свое
существование. Помещения в разное время занимали: полковой комитет 24-го
запасного полка, правление профсоюза работников-металлистов, 238-й стрелковый
полк, вечерний рабфак, средняя школа № 10 и даже контора Ждановтеплосети. В
годы войны здесь лечили немецких солдат, а после отступления вермахта здание
сожгли. Второй раз оно загорелось уже в восьмидесятых годах. Сегодня бывшее
духовное училище пустует, а само здание принадлежит частному лицу. Не очень
привлекательный район препятствует коммерческому использованию, а нахождение в
частной собственности практически обрекает дом на дальнейшее прозябание. Что
может спасти духовное училище? Разве что воскрешение проявлений духовности в
нашем обществе.

\ii{04_01_2019.stz.news.ua.lb.1.arhitekturnyj_atlas_dorevoljucijnogo_mariupolja.10.duhovnoje_uchilische.pic.1}
\ii{04_01_2019.stz.news.ua.lb.1.arhitekturnyj_atlas_dorevoljucijnogo_mariupolja.10.duhovnoje_uchilische.pic.2}
\ii{04_01_2019.stz.news.ua.lb.1.arhitekturnyj_atlas_dorevoljucijnogo_mariupolja.10.duhovnoje_uchilische.pic.3}
\ii{04_01_2019.stz.news.ua.lb.1.arhitekturnyj_atlas_dorevoljucijnogo_mariupolja.10.duhovnoje_uchilische.pic.4}
\ii{04_01_2019.stz.news.ua.lb.1.arhitekturnyj_atlas_dorevoljucijnogo_mariupolja.10.duhovnoje_uchilische.pic.5}


