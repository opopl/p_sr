% vim: keymap=russian-jcukenwin
%%beginhead 
 
%%file slova.shkola
%%parent slova
 
%%url 
 
%%author 
%%author_id 
%%author_url 
 
%%tags 
%%title 
 
%%endhead 
\chapter{Школа}
\label{sec:slova.shkola}

%%%cit
%%%cit_head
%%%cit_pic
%%%cit_text
Теперь местным детям придется учиться в куртках и пальто. В Купянске люди
перекрыли дорогу, после того как власти декоммунизировали местную \emph{школу}
- в рамках антисоциальной реформы, которую проводят в жизнь грантоеды:
\url{https://liva.com.ua/ukraine-schools.html} Детей переводят в помещение
старой неотремонтированной \emph{школы}, где зимой учатся в куртках и пальто -
потому что там холодно. Участников акции разогнала полиция, а честные
журналисты не обратили на нее большого внимания
%%%cit_comment
%%%cit_title
\citTitle{В Купянске декоммунизировали очередную среднюю школу}, 
Андрей Манчук, strana.ua, 20.06.2021
%%%endcit

%%%cit
%%%cit_head
%%%cit_pic
%%%cit_text
За Гаалом пришли. Это был самый настоящий арест. Вопросы задавали чрезвычайно
вежливо. Все выглядело очень пристойно. Гаал объяснил, что он всего-навсего
провинциал из Синнакса, учился в таких-то и таких-то \emph{школах} и
институтах, тогда-то и тогда-то получил степень доктора математики, его
пригласили работать в группу Хари Сэлдона, и он согласился. Вновь и вновь Гаал
повторял все сначала, но допрашивающие неустанно возвращались к вопросу о его
присоединении к группе Сэлдона: как он узнал об этой группе, каковы его
обязанности, какие тайные инструкции он получил, в чем заключается сэлдоновский
проект?  Гаал ответил, что ничего не знает, что не получал никаких тайных
инструкций. Он ученый, математик, и совершенно не интересуется политикой
%%%cit_comment
%%%cit_title
\citTitle{Основание}, Айзек Азимов
%%%endcit
