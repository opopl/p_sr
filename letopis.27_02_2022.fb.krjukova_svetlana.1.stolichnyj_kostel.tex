% vim: keymap=russian-jcukenwin
%%beginhead 
 
%%file 27_02_2022.fb.krjukova_svetlana.1.stolichnyj_kostel
%%parent 27_02_2022
 
%%url https://www.facebook.com/kryukova/posts/10159971853343064
 
%%author_id krjukova_svetlana
%%date 
 
%%tags __feb_2022.vtorzhenie,kiev,kostel.kiev.nikolaevskij
%%title Столичный Костёл гордо замер в кромешной тьме киевской темноте
 
%%endhead 
 
\subsection{Столичный Костёл гордо замер в кромешной тьме киевской темноте}
\label{sec:27_02_2022.fb.krjukova_svetlana.1.stolichnyj_kostel}
 
\Purl{https://www.facebook.com/kryukova/posts/10159971853343064}
\ifcmt
 author_begin
   author_id krjukova_svetlana
 author_end
\fi

Столичный Костёл гордо замер в кромешной тьме киевской темноте. Тишину можно
резать ножом как масло. 

Притихли сказочные герои, воины и чудища с хвостами. Они стоят на страже своего
дома, как делали это в 1941-ом, во время пожаров, во все века. Все держат
оборону, прислушиваются. 

На улицах ни людей, ни машин. Центр замер как исторический музей, где объявлен
общегосударственный выходной и вход - воспрещён. Час комендантский.

Холодные фонари, софиты витрин закрытых магазинов, кафе, аптек и банков. Многие
потушили вывески и заклеили окна, чтобы не дразнить мародеров, чтобы не стать
мишенью. 

Город замер, как стоп-кадр, как два года назад, когда в страну пришла первая,
иного рода новая чума непрошеная гостья - Пандемия. Страх замораживает время,
планы, мысли, фокусируя человеческую энергию на переваривание новостей и
спасение. Тогда, в прошлые зимы люди читали сводки заражённых и умерших от
короны, сегодня - сводки с полей. 

Русские танки прут со всех сторон. Бегство по территории бессмысленно и опасно.
Стреляют везде. Люди уехали из городов в пригород и попали под обстрел. Выехали
из восточных городов в Киев, и теперь прячутся в бомбоубежищах. Сегодня под
угрозой Киев, Харьков. 

В бомбоубежищах родилось 6 детей. 

Под пулями погибло 16. 

По Раша-ТВ - кладбище сдохшей журналистики, показывают, что русские пришли
освобождать украинцев от нацистов, что они близки к победе, вот почти захватили
Родину Мать и у них незначительные потери. Между тем реальность иная. 

Вооруженные силы сообщают: более 4 тыс положенного состава российской армии.
146 танков, 26 самолётов, сложённая техника. Миллион видео по сети с короткими
интервью российских солдатов, которые все как один говорят: мы не хотим этой
войны, мы не знали, куда нас ведут, нам стыдно. 

Звук сирен становится обычным городским фоном. Это низкий протяжный тон, как
брошенная бродячая собака, сидящая в водосточной трубе. Протяжный звук, от
которого бегут по телу мурашки и мутит в животе. 

- Мама, Сиреноголовый приближается! - кричат дети по дороге в укрытие. Это,
оказывается, такое чудище-гуманоид из интернета, типа Момо. Костлявый, высокий,
с кожей цвета ржавого метала. Столб вместо головы и шеи, на котором сирены.
Короче, страшный как война. 

Война…

У нас война.

- Как дела? Ты где? 

- Я в Киеве, тут - война. 

Русские воют с украинцами. 

Русские убивают украинских детей. 

Русские стреляют ракетами и танками по нашим домам. А по Раша-ТВ рассказывают
про освобождение от нацистов. 

И вот опять орет Сиреноголовый. 

Ночью Киев ждёт обстрел. 

- С детьми тревожно, за них взрослые волнуются больше всего, но с ними легче.
Они смеются, даже когда страшно. Им не впадло бежать в укрытие, они мастерски
лепят холобуды в любых условиях и четко прислушиваются ко всем звукам по
воздуху. 

Опять сирена. Стрельба. Взрывы. Сиреноголовый. Страх. Меня дед учил с детства
не любить людей, которые принуждают тебя бояться. Они парализуют волю, разум и
будущее. 

Сегодня в Киевской подземке показывали кукольный театр. Дети смеялись. Смех
лечит. Красивые дети, здравые, умные, все понимающие, не все говорящие, с
горящими глазами, уже со своей маленькой историей детства, которые они пронесут
через всю жизнь. Когда-то они приведут сюда своих детей и промолвят: «Я ночевал
тут со всеми. И мы смотрели кукольный театр. А на улице была война. Россия
напала на Украину. С танками, ракетами и бомбами. Но мы выстояли. Мы
победили!».

Впереди ночь. И утро.

Сил нашим военным, терпения тем, кто не бросил страну в трудное время. 

Зло будет наказано.

Киев, 27.02.2022
