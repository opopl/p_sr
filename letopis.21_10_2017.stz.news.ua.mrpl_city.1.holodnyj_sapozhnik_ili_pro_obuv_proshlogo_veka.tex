% vim: keymap=russian-jcukenwin
%%beginhead 
 
%%file 21_10_2017.stz.news.ua.mrpl_city.1.holodnyj_sapozhnik_ili_pro_obuv_proshlogo_veka
%%parent 21_10_2017
 
%%url https://mrpl.city/blogs/view/holodnyj-sapozhnik-ili-pro-obuv-proshlogo-veka
 
%%author_id burov_sergij.mariupol,news.ua.mrpl_city
%%date 
 
%%tags 
%%title Холодный сапожник, или Про обувь прошлого века
 
%%endhead 
 
\subsection{Холодный сапожник, или Про обувь прошлого века}
\label{sec:21_10_2017.stz.news.ua.mrpl_city.1.holodnyj_sapozhnik_ili_pro_obuv_proshlogo_veka}
 
\Purl{https://mrpl.city/blogs/view/holodnyj-sapozhnik-ili-pro-obuv-proshlogo-veka}
\ifcmt
 author_begin
   author_id burov_sergij.mariupol,news.ua.mrpl_city
 author_end
\fi

Поговорим об обуви того периода  существования нашего города, когда в стране
уже отменили карточную систему, но еще здравствовал великий вождь и учитель,
организатор всех побед, гениальный продолжатель дела Ленина, генералиссимус...
Читатели, представители старшего поколения, вероятно, догадались о ком и о
каком времени идет речь. Кто-то скажет: \enquote{Какая проза! Кругом вершились
исторические события, а тут – о каких-то  приземленных, в прямом и переносном
смысле, вещах}. Не скажите, вещи эти, как не верти – часть материальной
культуры. Ею, между прочим, занимаются историки. Но здесь, понятное дело, не
историческое исследование, а воспоминания из детства и юности.

Как было с обувью? Конечно, донашивали то, что осталась еще с \enquote{до войны}. Ее
подшивали, подклеивали, подбивали, перетягивали. Кто умел - ремонтировал сам.
Но большинство все-таки обращалось к профессионалам. Мелкий ремонт – там
приторочить подметки или сделать набойки на каблуки, положить заплату на
прореху -  могли \enquote{холодные} сапожники. Стихотворное определение дал им когда-то
Сергей Михалков: 

Холодный сапожник\par
У наших ворот\par
Дырявую обувь\par
В починку берет. \par
\medskip
Весь день\par
Продувает его ветерком.\par
Он песни поет\par
И стучит молотком.\par

Холодными сапожниками были или инвалиды, или очень пожилые люди. Всё их
оборудование состояло из обыкновенной табуретки.  Верх ее  с четырех сторон был
ограничен бортиками. Это - чтобы не могли свалиться шилья, крючки, клещи, ножи,
заточенные до остроты бритвы, оселок для  правки ножей, мотки дратвы, жестяные
коробочки с гвоздиками разных размеров и подковками, да банка с клеем с
воткнутой в него кистью; внизу табуретки была пристроена полочка, на которой
были сложены куски кожи и ее заменители. Еще присутствовали \enquote{лапа} между колен
собрата михалковского героя и молоток с расплющенным бойком в его руке. Мастер
восседал на маленькой скамеечке. Таких умельцев  можно было увидеть и на старом
базаре, и под каким-нибудь домом, или во дворе старой части города, где мастер
проживал. И все в округе знали этот двор, и несли в починку прохудившиеся
башмаки, дамские туфли со стоптанными каблуками, детские сандалии и тому
подобное. 

Кроме холодных, были сапожники, которые делали свое нужное людям дело в
собственных или арендуемых будках и лавочках. \enquote{Инструментарий} у них был такой
же, как и у уличных коллег, разве что вместо табуретки-верстака был низенький
столик, да на полках стояли рядами колодки разных размеров. Их применяли, когда
нужно было \enquote{раздать} тесноватую обувь или сделать перетяжку. Что это за
операция? Допустим, у вас есть сапоги, у которых подошвы изношены до крайности
и не подлежат ремонту, а голенища еще хороши. И тогда за дело брался мастер –
он срезал ошметки низа сапог, а к верхам пришивались стельки, подошвы, к
которым присоединялись новые каблуки - вот тогда-то использовались колодки.

Трудились еще в Мариуполе кустари, которые тачали летнюю обувь на заказ, но
больше - на продажу. Ходовым товаром были, так называемые, лосевки – легкие
черевички, верх которых выкраивался из тонкой черной, реже – коричневой, кожи,
а подошва вырезалась будто бы из кожи лося. Впрочем, была ли это лосевая кожа
на самом деле или – нет, сейчас спрашивать уже не у кого. Украшение лосевок и
их главный отличительный признак – три короткие полоски красной кожи,
размещенных одна над другой на каждой туфле. 

Люди, которым были не по карману лосевки, покупали на базаре на лето тапочки из
грубого брезента с подошвой из обрезков прорезиненных транспортерных лент,
умыкнутых на заводах. Кстати, коль скоро вспомнили брезент, стоит, наверное,
рассказать о специфической обуви той эпохи – \enquote{колодки}. Каждая из них
представляла собой след ноги,  выпиленный примерно из двадцатимиллиметровой
доски, к которому гвоздями было прибито подобие верха ботинка из брезента.
Колодки выдавали рабочим строительных управлений, завербованным в селах юношам
и девушкам.  Стук сотен ног по брусчатке той же  Торговой улицы возвещал, что
солнце должно скоро взойти. Этот стук, гудки заводов, сообщавших о начале
рабочей смены, чихание  раздрыганных моторов полуторок, уверенное урчание
двигателей американских студебеккеров, щелканье кнутов возниц, поощряющих
лошадей к движению, составляли звуковую картину тех лет...

Вместе с тем, были и мастера, которые изготавливали модельную дамскую обувь.
Правда, не в одиночку, а кооперируясь. Один кроил и сшивал на специальной
швейной машине заготовку верха, другой вырезал основу каблука из древесины липы
или березы, третий – это как раз был сапожник –  доводил работу до завершения –
обклеивал кожей каблук, пришивал подошву, ну, и так далее. Деятельность эта
преследовалась государственными органами, за это можно было схлопотать срок, и
немалый. Поэтому ремесленникам приходилось работать подпольно. Вот один из
примеров конспирации. В каждый будний день, около восьми часов утра на Торговой
улице, там, где она пересекается с Фонтанной, появлялся средних лет
благообразный мужчина. Он мерно шагал то в сторону проспекта Республики, то –
наоборот, по направлению пивзавода, то сворачивал на Фонтанную. Зимой он был
одет в добротное пальто с каракулевым воротником, на голове шапка-ушанка из
того же меха. В теплое время на нем был тщательно отутюженный двубортный
коверкотовый костюм благородного серого цвета, белоснежная тугонакрахмаленная
сорочка, застегнутая на все пуговки. При нем всегда был среднего размера
фибровый чемодан. Если бы облачение этого человека дополнить галстуком, то
можно было подумать, что какой-то доцент, а может быть и профессор,
отправляется в служебную  командировку. На самом деле это был один из
изготовителей модельной дамской обуви на финальной стадии ее производства. В
чемодане у него находились все необходимое для сапожного ремесла. Чтобы не
попасть в поле зрения властей за незаконным занятием, он снимал комнаты на
один-два дня, то на Шишманке, то на Слободке, то где-нибудь на Марьинске.
Этим-то и  объяснялась смена направления его утренних походов...

\enquote{Танкетки - это обувь на подошве, которая одновременно выполняет функции
каблука. Эта подошва совсем тонкая в области носка, но постепенно утолщается к
пятке}.  Такое определение дает им Яндекс. Но нужно ли объяснять современным
модницам, что это такое? А в описываемый здесь период времени танкетки были
невидалью. Их завезли из побежденной Германии, и они сразу же стали наивысшим
шиком, - и потому что были редки, и потому что были удобны в носке. К началу
50-х годов прошлого столетия \enquote{трофеи} поизносились, а местные умельцы научились
делать сами экзотическую по тем временам обувь. Правда, пробковую часть каблука
оригинала пришлось им  заменить деталями из липы. 

Зимняя обувь. Хорошо было демобилизованным офицерам – они донашивали фронтовые
сапоги, остальная часть населения в холодное время года щеголяла в бурках –
стеганных матерчатых сапогах, как определяют энциклопедии. Голенища мужских
бурок были до колен,  женских -  чуть ниже середины голени, кроме того они
имели опушку из меха, подвернувшегося под руку. Непременным атрибутом этой
обуви были галоши – фабричные или чуни, склеенные из кусков камер автомобильных
скатов. Старики поговаривали, что \enquote{матерчатые сапоги} были позаимствованы в
Маньчжурии у китайцев во время русско-японской войны 1904 -1905 годов. Будто бы
один из участников этой войны, в мирной жизни портной,  подсмотрел, как жители
поднебесной  делают бурки, а когда возникли трудности с обувью, применил свои
знания на практике. Полезное дело получило очень скоро широкое распространение.
Правда, насколько верна эта версия, едва ли можно проверить.

Гораздо меньшее распространение получили бурки из белого фетра с кожаными,
коричневого цвета носками и задниками, швы были закрыты  полосками из того
материала. Признаком особой роскоши было украшение края союзки узором из
дырочек и зубчиками. Как правило, верх голенищ отворачивался. Такие бурки в
сочетании с синими галифе и диагоналевые полувоенные френчи - сталинки
темно-защитного цвета свидетельствовали о том, что их обладатель является
ответственным работником. 

Наряду с кустарной, ближе к пятидесятым годам, мариупольцы стали носить и новую
фабричную обувь. Ее можно было купить в магазинах, правда, постояв в очереди.
Это были, прежде всего, \enquote{прорезинки} - комбинированные спортивного типа
тапочки, у которых верх был синий, а низ – из черной лакированной резины,
единственным украшением их было три пары дырочек для шнурка. Позже появился,
так сказать, их девичий вариант – сплошь белый. Детская обувь была представлена
сандалиями всех размеров из свиной кожи. Иногда \enquote{выбрасывали} в продажу ботинки
ленинградской обувной фабрики \enquote{Скороход}, парусиновые туфли коричневого и
белого цвета. Последние были выходными. Для пущей белизны их красили зубным
порошком. Когда местные франты отплясывали  кадриль или какой-нибудь другой
зажигательный танец на танцплощадке в Городском саду, то от их парадных штиблет
поднимались фонтанчики белой пыли...

Все это уже давно ушло в прошлое.
