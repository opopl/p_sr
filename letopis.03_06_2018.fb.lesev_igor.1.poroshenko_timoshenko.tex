% vim: keymap=russian-jcukenwin
%%beginhead 
 
%%file 03_06_2018.fb.lesev_igor.1.poroshenko_timoshenko
%%parent 03_06_2018
 
%%url https://www.facebook.com/permalink.php?story_fbid=1938549826176111&id=100000633379839
 
%%author_id lesev_igor
%%date 
 
%%tags politika,poroshenko_petr,timoshenko_julia,ukraina
%%title Почему у Порошенко все так, а у Тимошенко совершенно иначе?
 
%%endhead 
 
\subsection{Почему у Порошенко все так, а у Тимошенко совершенно иначе?}
\label{sec:03_06_2018.fb.lesev_igor.1.poroshenko_timoshenko}
 
\Purl{https://www.facebook.com/permalink.php?story_fbid=1938549826176111&id=100000633379839}
\ifcmt
 author_begin
   author_id lesev_igor
 author_end
\fi

Почему у Порошенко все так, а у Тимошенко совершенно иначе?

Как мне кажется, из-за избранной избирательной стратегии, уровня обслуживающего
персонажа и личных качеств. Юлия Владимировна умудрилась вступить уже в
полноценную избирательную кампанию, не используя свой главный козырь прошлых
лет – агрессивные и эмоциональные атаки на конкурента(ов). Но тут есть свои
объяснения. Первый и основной – власть сейчас держит не бесхребетный, но, сука,
невероятно кровавый Янукович, а вполне себе беспредельный Порошенко, который
по-настоящему обижается за личные выпады.

\ifcmt
  ig https://scontent-frx5-1.xx.fbcdn.net/v/t1.6435-9/34190671_1938549719509455_5365385652831518720_n.jpg?_nc_cat=110&ccb=1-5&_nc_sid=730e14&_nc_ohc=XzMpkTP4SkcAX_WT9t_&_nc_ht=scontent-frx5-1.xx&oh=60568129ed380163f5c15eadf5694e75&oe=61BABEF0
  @width 0.4
  %@wrap \parpic[r]
  @wrap \InsertBoxR{0}
\fi

Другие причины формально личной вялости Тимошенко тоже объяснимы. Она пытается
мягко выйти на восточное электоральное поле, но сделать это желает так, чтобы
не спугнуть своих базовых бабулек из центральной Украины. И делает это, кстати,
вполне успешно. В современных истерично-властных условиях «борьбы с
агрессором», достаточно просто промолчать там, где вся «патриотическая
спильнота» дружно высрется предсказуемыми и уже такими утомительными пассажами.

Есть и третья причина сравнительно пассивного поведения Юли – нежелание
переусердствовать с мочиловом Порошенко, который и так уже хромает на обе ноги.
Текущая избирательная кампания Тимошенко заточена на победу именно над
Порошенко, а не над каким-то другим кандидатом. Но как раз сейчас все
складывается таким образом, что Тимошенко во втором туре будет воевать с кем-то
другим. С кем – вопрос решительно открытый. Отсюда и пассивный старт рысцой.

Но в любом случае, информационная кампания Тимошенко выделяется не
агрессивностью и встроенностью под самые разные целевые аудитории. Пафосного
бреда в стиле «вона працюе» уже нет. Политической йопли мозга, тоже.
Категоричность вообще отсутствует. Тезисы в основном экономического характера.
И пока шансы Порошенко на попадание во второй тур сохраняются, нет оснований
полагать, что информационная повестка Тимошенко, дающая ей без видимых
напряжений профиты, как-то изменится.

А вот у Петра Алексеевича все на порядок грустнее. Вся его избирательная
кампания идет наперекосяк во многом из-за самого Порошенко. Президент не
позволяет никому сказать в лицо, что это и вот это – полное говно и так делать
не надо. Сразу же будет истерика, оскорбления и удаление от священной тушки. В
результате, Порошенко живет в выдуманном мире вещей, в котором его все любят,
ценят, восхищаются и гордятся. Проводить избирательную кампанию в таких
условиях в принципе невозможно. К тому же, следует учитывать, что у Порошенко
практически всегда работают несколько дублирующих друг друга штабных структур,
которые ведут срач за бюджеты и за улыбку Первого. Т.е., вся структура в
принципе не работающая.

Вторая беда Порошенко – это «креаторы» в его окружении. Кроме желания
понравиться шефу, там есть люди, уверенные, что линейный ход формирования
общественного мнения можно резко переиначить какой-то «выдающейся»
многоходовочкой. К чему все это приводит по итогу, мы все – ходящие по земле –
знаем. С Порошенко смеется ВСЯ Украина. Ну за вычетом тех, кто у него на зп.
Хотя нет, те тоже смеются, только в кулачок.

Наконец, третья особенность избирательной кампании Порошенко – это дикое
попугайничанье с Путина. Дичайшее, неуместное и нелепое. Путин открыл мост –
Порошенко открыл мостИк. Путин проехал в кабине грузовика – Порошенко проехал в
кабине машиниста. Путин фотосессит с рабочими – Порошенко фотосессит с
рабочими. Разница между событиями – неделя/две. Качество в картинках
разительное. И ведь кто-то ему говорит, что это работает. И в результате там,
где можно было бы ИНОГДА получить положительный пиар, Порошенко теряет больше,
чем приобретает.

Да, и последнее, что отличает очень здорово Порошенко от Тимошенко – это
абсолютное неумение разговаривать с целевой аудиторией. Юлия Владимировна та
еще «пафосная кокетка», актриса больших и малых театров, кудесница
заготовленных мизансцен и постановочных актов. Но пафос от Тимошенко
воспринимается именно как игра. А вот то, что несет каждый день Порошенко – это
какой-то отдельный вид искусства, который еще следует изучать. Все что он
говорит вне зависимости от содержания и истинности, вызывает автоматическое
отторжение. Все его речевые обороты не воспринимаются. Соответственно, он и
продать себя не может. Порошенко разговаривает на публике не живым, не
человеческим языком.

В итоге, пока что на горизонте следующим Первым маячит Юля. Варианты есть, но
чаще всего самое очевидное и становится настоящим. Но обольщаться этим не
стоит. Юля – это «равно» Петя. По крайней мере, все, кто вокруг Порошенко – это
в большинстве бывшие юльки. И со сменой Пети на Юлю нет никаких оснований на
100\% менять действующих. Просто, кто-то переедет на заслуженный отдых в
Андалусию, а Вона продолжит працюваты.

\ii{03_06_2018.fb.lesev_igor.1.poroshenko_timoshenko.cmt}
