%%beginhead 
 
%%file 12_02_2022.fb.mariupol.shlub_za_dobu.1.vesilna_tradicia_nastupannja_na_rushnyk
%%parent 12_02_2022
 
%%url https://www.facebook.com/BrakZaSutki/posts/pfbid02ZQjgYksuRgLKJSwLbtkgZ2jh9m26KxGdxcZbwHKdL9zZBcZkuPCfyDR3NJHjQ9Mtl
 
%%author_id mariupol.shlub_za_dobu
%%date 12_02_2022
 
%%tags 
%%title Весільна традиція наступання на рушник
 
%%endhead 

\subsection{Весільна традиція наступання на рушник}
\label{sec:12_02_2022.fb.mariupol.shlub_za_dobu.1.vesilna_tradicia_nastupannja_na_rushnyk}

\Purl{https://www.facebook.com/BrakZaSutki/posts/pfbid02ZQjgYksuRgLKJSwLbtkgZ2jh9m26KxGdxcZbwHKdL9zZBcZkuPCfyDR3NJHjQ9Mtl}
\ifcmt
 author_begin
   author_id mariupol.shlub_za_dobu
 author_end
\fi

А ви знали, що для традиційного українського весілля було потрібно близько
сорока спеціальних рушників? У сучасному весільному обряді  їх залишилося
всього п'ять.

🤍Багато молодят почали уникати  весільної традиції наступання на рушник під
час церемонії. 

Але це досить зворушливий та цікавий обряд, як і обмін обручками, який робить
церемонію більш затишною та колоритною. Коли рідні матусі або близькі друзі
вистилають рушник біля молодят та запрошують вступити нібито проголошують, що
саме с цієї митті починається сімейне життя.

🤗 Вважається, що той хто першим з молодої пари стане на рушник, того
визначають як майбутнього главу сім'ї.

Разом з нами Ви можете обрати урочисту церемонію за власним сценарієм.

За докладною інформацією про проєкт \enquote{Шлюб за добу} звертайтесь у Direct
або за контактними номерами телефону Лівобережний район, вул. Українського
козацтва, 51 (Мультицентр)

+38 068 9 269 543\par
+38 099 0 654 989\par
Центральний район, пр. Металлургів 52 (Маріупольська камерна філармонія, 1 поверх)\par
+38 067 830 27 79 \par
+38 050 505 52 82\par

\#шлюб\_за\_добу \#брак\_за\_сутки

\#ТутВарто\_бути\_щасливим 

\#24hour\_wedding \#шлюб24 \#Мариуполь\_брак
