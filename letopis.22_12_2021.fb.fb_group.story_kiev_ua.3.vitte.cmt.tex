% vim: keymap=russian-jcukenwin
%%beginhead 
 
%%file 22_12_2021.fb.fb_group.story_kiev_ua.3.vitte.cmt
%%parent 22_12_2021.fb.fb_group.story_kiev_ua.3.vitte
 
%%url 
 
%%author_id 
%%date 
 
%%tags 
%%title 
 
%%endhead 
\zzSecCmt

\begin{itemize} % {
\iusr{Людмила Кузьменко}
Вот таких бы людей нам в президенты!

\iusr{Регина Кучеренко}
Ох, как нам в Киеве не хватает Витте!

\iusr{Yuriy Pokrass - Krichevsky}
Да, в Киеве можно родится, но реализовать себя невозможно! Сотни примеров.

\begin{itemize} % {
\iusr{Mykhaylo Losytskyy}
Якщо столиця в Москві/СПб, тоді так. Та й то, Ханенко з Терещенком посперечалися б.

\iusr{Lena Pavlikova}
\textbf{Yuriy Pokrass - Krichevsky} И сотни тысяч наоборот. Вам просто не повезло или не хватило чего-то...

\iusr{Татьяна Долгополова}
\textbf{Yuriy Pokrass - Krichevsky} реализовать себя можно везде зависит очеловечат а не от места а уж в Киеве -тем более действительно возможно не повезло но и это может произойти в любом месте
\end{itemize} % }

\iusr{Елена Дамаскина}
Удивительный человек!

\iusr{Людмила Задерей}

Сорри за занудство, но с вашего позволения - небольшое уточнение ) два )

Барельеф в память о Сергее Юльевиче установлен на фасаде управления
Юго-Западной железной дороги. Министерства ж.д. (с) в Украине нет. Есть АТ
\enquote{Українська залізниця} и у него шесть региональных филиалов (ЮЗЖД один из них)

А находится управление ЮЗЖД на улице Лысенко ) Леонтовича - это совсем рядом,
но по другую сторону улицы Хмельницкого )

\begin{itemize} % {
\iusr{Галина Полякова}
\textbf{Людмила Задерей} Вы правы! Спасибо за замечание.
\end{itemize} % }

\iusr{Ольга Хоботня}
Истинно государственники, а-у-у!!!

\iusr{Mykhaylo Losytskyy}

Наскільки я знаю, там же цар не просто почув з-за спини. Вітте написав царю
доповідну про можливу небезпеку, цар відмахнувся, а потім якраз і потрапив у
аварію. І от після цього і згадав того, хто попереджав.

\begin{itemize} % {
\iusr{Галина Полякова}
\textbf{Mykhaylo Losytskyy} 

Цілком вірогідно. Мій батько збирав колекцію свідчень самовидців процессу
взяття в полон Паулюса. Усі були поруч, брали участь, але бачили геть інше.
Епізод \enquote{цар- Вітте - швидкість потягу} також описаний у кількох джерелах. Але
сутність одна.

\end{itemize} % }

\iusr{Марат Баратов}

Сколько людей, столько и мнений. Читал когда-то книгу в которой автор называл
Витте чуть ли не главным виновником краха Российской империи. И обосновывал.

\begin{itemize} % {
\iusr{Галина Полякова}

Марат Баратов \enquote{На перекрестках путей и мнений
Рождались звезды и вдохновенье.}

Но есть факты - великолепная денежная реформа, рост промышленности,
строительство дорог, конвертируемость рубля. Уровень 1913 года Россия так и не
достигла. Однако, это благополучие основывалось на работе Витте. Столыпин
пришел позже. Я не умаляю его заслуг, однако, давайте признаем, что результаты
реформ не сразу заметны.

\end{itemize} % }

\iusr{Олена Шелест}

Благодарю. Прочитала с удовольствием, восхитилась масштабом Личности. Ваш пост
заставляет задуматься, проводить исторические параллели.


\iusr{Vadym Makhnytskyy}

И с какого такого перепугу Сергей Юльевич Витте, родившийся в Тифлисе стал
киевлянином? Только потому, что несколько лет жил и работал в Киеве? Но тогда с
таким же успехом он может быть кишеневцем, где учился в гимназии, и одесситом,
учась на физико-математическом факультете в одесском Новороссийском
университете, а затем служа в службе эксплуатации Одесской железной дороги,
когда в Одессу после смерти отца перебралась вся его семья, и в большей степени
петербуржцем, где начал службу, получив должность начальника эксплуатационного
отдела при правлении Общества Юго-Западных железных дорог и затем продолжил на
высших государственных должностях...

\begin{itemize} % {
\iusr{Галина Полякова}
\textbf{Vadym Makhnytskyy} Какое-то время он жил и работал в Киеве. И это дает ему право считаться киевлянином. К тому же он любил Киев и многое для города сделал.
\end{itemize} % }

\iusr{Tatyana Smirnova}

Спасибо, Галина, за вашу серию миниатюр и в частности за эту. Незаслуженно
забыт многими киевлянами. Как- то походя мимо здания министерства ж/б услышала,
как- то спросил, а кому это памятник. Никто не знал. В двух словах рассказала,
кто это был, но многое из его биографии узнала сейчас, например, о том, что был
преподавателем у Николая II, что участвовал в переговорах с Японией. Словом,
достойный и умнейший киевлянин


\iusr{Анатолій Зборовський}

А у нас дехто захоплюється Столипіним, який перервав реформи, започатковані
Вітте, і в кінцевому підсумку довів Російську імперію до краху.

\begin{itemize} % {
\iusr{Галина Полякова}
\textbf{Анатолій Зборовський} Абсолютно з Вами згодна! Реформи Столипіна грунтувалися на наробках Вітте.
\end{itemize} % }

\iusr{Серёж Лихей}
Вот это да... я в восторге...

\iusr{Oleg Gerassimenko}
Выдающийся человек. Написал прекрасные воспоминания.

\iusr{Валентина Щербинина}
Жаль, что умные не могут реализовать себя в нашей стране, нам реформаторы не нужны.

\iusr{Вадим Горбов}
Нет у нас давно такого министерства Ж/Д.

\begin{itemize} % {
\iusr{Людмила Задерей}
\textbf{Вадим Горбов} и не было никогда

\iusr{Вадим Горбов}
\textbf{Людмила Задерей} может в УНР и было такое в доме Мишеля Терещенко

\iusr{Людмила Задерей}
\textbf{Вадим Горбов} 

честно, не подумала об этом ))) когда оставляла комментарий - имела в виду
\enquote{пост-МПСовский} период ) о периоде УНР - подучу матчасть, спасибо за идею )


\iusr{Галина Полякова}
\textbf{Вадим Горбов} Виновата.

\iusr{Инна Сергеева}
\textbf{Вадим Горбов} та там всё сложно. Пише шо попало.

\iusr{Вадим Горбов}
\textbf{Инна Сергеева} Я здесь вроде как литредактор. В последнем абзаце поста неправильные падежи. ))

\iusr{Elen Tsarovska}
\textbf{Вадим Горбов} 

Ну конечно не совсем тот масштаб, да и до конца развернуться, увы, не
получилось, но не следует забывать такого руководителя ЮЗЖД как Кирпа
(масштабная реконструкция ЖД вокзалов, ст- во Южного вокзала в Киеве и многое
другое)


\iusr{Вадим Горбов}
\textbf{Elen Tsarovska} и Кривонос

\iusr{Elen Tsarovska}
\textbf{Вадим Горбов} 

Да, согласна Но у Кривоноса за спиной была советская плановая система... Кирпа
оказался в другой \enquote{системе координат}

\end{itemize} % }

\iusr{Игорь П.}
Нам бы сейчас Витте, а вместо него только Найомы(((

\begin{itemize} % {
\iusr{Инна Сергеева}
\textbf{Игорь П.} Так вы ж за ними шли на майдане, а теперь не нравятся?

\iusr{Галина Полякова}
\textbf{Инна Сергеева} Вы заблуждаетесь. Шли не за ним. Мы шли за Украину.

\iusr{Игорь П.}
\textbf{Инна Сергеева} пардон, с меня хватило и 2004. С тех пор я поумнел. Так что пуштун мне не брат.

\iusr{Инна Сергеева}
\textbf{Галина Полякова} шоб вы впалы колы йшлы.

\iusr{Галина Полякова}
\textbf{Инна Сергеева} Впечатлившись Вашими затейлевыми комментариями, я навестила Вашу страничку. Впечатлена.

\iusr{Галина Полякова}
\textbf{Инна Сергеева} Не упали и не упадем. НЕ дождетесь.

\iusr{Инна Сергеева}
\textbf{Галина Полякова} упали и не заметили.

\iusr{Инна Сергеева}
\textbf{Галина Полякова} и я на вас глянула. Лучше б не смотрела.

\iusr{Оксана Оболенська}
\textbf{Инна Сергеева} , то й не треба шастать куди не запрошують
\end{itemize} % }

\iusr{Мария Иванова}
Спасибо

\iusr{Петр Сазонов}

Должен заметить, что после Высочайшего пожалования Сергею Юльевичу графского
Российской империи достоинства - приставки Сахалинский он конечно не получил,
это в юмористических журналах его прозвали « графом Полусахалинским «, за то
что по результатам Портсмутского мира Японии отошла Южная часть Сахалина.

\begin{itemize} % {
\iusr{Галина Полякова}
\textbf{Петр Сазонов} 

Это точно. Но недоброжелателей и злопыхателей и просто завистников всегда
хватает с избытком. Россия потерпела в войне сокрушительное поражение. Но Витте
на переговорах добился очень большого успеха. Половину Сахалина Россия
потеряла, но могла потерять значительно больше. А позлословить - медом не
корми.

\end{itemize} % }

\iusr{Петр Сазонов}

Портрет графа Витте кисти Репина. Эскиз к картине Торжественное заседание
Госсовета. ( Русский музей в СПб)

\ifcmt
  ig https://scontent-frt3-1.xx.fbcdn.net/v/t39.30808-6/269747438_452318809941636_6879178142499642580_n.jpg?_nc_cat=104&ccb=1-5&_nc_sid=dbeb18&_nc_ohc=dmHkT1zZM_EAX9NF2RR&_nc_ht=scontent-frt3-1.xx&oh=00_AT8JotWv-1EhABWOT2mIgeBvIVdz4ksorwq5MXD0rned-A&oe=61D0D113
  @width 0.3
\fi

\iusr{Петр Сазонов}
Александро невская лавра

\ifcmt
  ig https://scontent-frt3-1.xx.fbcdn.net/v/t39.30808-6/269746648_452319129941604_7211180321610739785_n.jpg?_nc_cat=108&ccb=1-5&_nc_sid=dbeb18&_nc_ohc=kNdP0ltQK1MAX9xhOno&_nc_ht=scontent-frt3-1.xx&oh=00_AT95NprK-25H18zk454Vs4Ziy0oplzLT8dadFXQl6y-wgg&oe=61D11C4B
  @width 0.3
\fi

\iusr{Jerzy Szalacki}

... это от его надгробия на Байковом, в левой части, в виде готического пинакля
с шатром остался только каркас ??? Кажется, это по проекту Городецкого, как
часть Николаевского костела ?!

\begin{itemize} % {
\iusr{Галина Полякова}
\textbf{Jerzy Szalacki} Витте умер в Петербурге.

\iusr{Ксения Погорлецкая}
\textbf{Галина Полякова} 

Не помню где умер Витте но могила его была, изначально, на одной из террас
некрополя возле Аскольдовой могилы и маму рассказывала (она 26 года рождения)
что часто видела жену Витте сидящую возле могилы. Кажется , в конце 30-
некрополь уничтожили....


\iusr{Галина Полякова}
\textbf{Ксения Погорлецкая} 

Витте умер в 1915 году в Петербурге и был погребен в Свято-Троицкой
Александро-Невской Лавре. Я не нашла сведений о судьбе Матильды Витте. После
смерти мужа она занялась изданием его мемуаров. Их искали по поручению царя, но
не нашли - рукопись хранилась в банке за границей. Я не думаю, что вдова Витте
осталась в стране. У них ведь была вилла в Биаррице. Вероятно, Ваша мама имела
в виду какую-то другую женщину. Нашла только такую фразу: \enquote{Витте женился на
Марии Ивановне Лисаневич (1863—после 1924), урождённой Матильде Исааковне
Нурок}.

\iusr{Ксения Погорлецкая}
\textbf{Галина Полякова} 

Ошибалась мама или нет, а говорила о вдове и могиле Витте не только она; нужно
спросить у Михаила Кальницкого, он киевовед, автор не одной книги и, уверена,
точно ответит на этот вопрос. Есть еще книга Людмила Проценко ( у меня ее нет к
сожалению) \enquote{Некрополи Киева} там должны быть и поименные захоронения наиболее
известных людей на Аскольдовой могиле. Упоминаний об умерших детях Витте Вы
нигде не находили?

\iusr{Галина Полякова}
\textbf{Ксения Погорлецкая} Своих детей не было. Были дочки от первых браков обеих жен. \enquote{Некрополи Киева} у меня есть.
\end{itemize} % }

\iusr{Инна Сергеева}
Повествование так себе.
Для начала - Витте родился в Тифлисе, украинцем не был.
Работу на железной дороге начал в Одессе. Принимал участие в организации работ по строительству Транссиба, КВЖД и Юго-Западных железных дорог. Приветствовал инвестиции в эти проекты.
На Леонтовича никогда не было министерства, тем более железных дорог (можно было спросить у гугля).
Барельеф Витте установлен на здании управления Юго-Западной железной дороги.
Ну и так далее...

\begin{itemize} % {
\iusr{Галина Полякова}
\textbf{Инна Сергеева} Вы абсолютно правы. Но Витте также жил и работал в Киеве, любил этот город и много для него сделал. Я также приветствую Ваше замечание по поводу несуществующего министерства.

\begin{itemize} % {
\iusr{Инна Сергеева}
\textbf{Галина Полякова} не знаю любил или не любил. Работал, да. А про любофф нигде ни слова.

\iusr{Галина Полякова}
\textbf{Инна Сергеева} По своей недалекости я привыкла считать, что любовь не в словах, а в делах.

\iusr{Оксана Оболенська}
\textbf{Инна Сергеева} , фи, добрее надо быть сударыня...

\iusr{Инна Сергеева}
\textbf{Оксана Оболенська} 

если вам в кайф фигню читать, то пжлста. Можете ещё хромадское смотреть, до полного сч

\iusr{Оксана Оболенська}
\textbf{Инна Сергеева} , 

хамское поведение ещё никого не красило. Не вам давать советы, судя по тому,
что вы пишите. А вообще, на месте модераторов сайта, надо бы подобных вам
отсекать. Здесь ведь приличные люди собрались.

\iusr{Инна Сергеева}
\textbf{Оксана Оболенська} вы с авторкой приличные люди? @igg{fbicon.laugh.rolling.floor}{repeat=3} 

\iusr{Оксана Оболенська}
\textbf{Инна Сергеева}, 

не судите о людях по себе. Приличие вам не присуще. Да и вряд ли вам ведомо
само понятие слова. Что вы вообще потеряли на \enquote{Киевских историях} ?
Понятно же, что это не ваша тема.

\end{itemize} % }

\end{itemize} % }

\iusr{Mykhaylo Losytskyy}
За очі його прозвали \enquote{Полусахалинский}, бо пів Сахаліна РІ японцям відступила.

\begin{itemize} % {
\iusr{Галина Полякова}

І це був дуже непоганий результат мирних переговорів. Навіть успішний. Адже
Росія війну вщент програла.

\begin{itemize} % {
\iusr{Олег Курилов}
\textbf{Галина Полякова} 

немного не согласен с вами. Во первых при переговорах японцы не надеялись даже
на пол Сахалина. Но из за утечки информации получили таки часть сахалина. Ну и
во вторых, если на море р.И. потерпела сокрушительное поражение то на суше было
примерное равенство. проблема России это революция и восстания 1905года.
Проблема Японии - у них не было сил и средств продолжать войну и Япония первая
стремилась заключить мир. Короче всё непросто и неоднозначно...


\iusr{Галина Полякова}
\textbf{Олег Курилов} 

А Порт Артур ведь сдали, не так ли? А Мукден? Я Ляоян? В какой битве Россия
победила Японию? Япония выдохлась, но мы судим по результату.

\iusr{Олег Курилов}
\textbf{Галина Полякова} 

Вот так оценивает японский историк Окамото Сюмпэй итоги Мукденского сражения:
"Битва была жестокой, она окончилась 10 марта победой Японии. Но это была
крайне неуверенная победа, так как потери Японии достигли 72 008 человек.
Российские войска отступили на север, «сохраняя порядок», и начали готовиться к
наступлению, в то время как подкрепления к ним все прибывали. В императорском
штабе становилось ясно, что военная мощь России была сильно недооценена и что в
Северной Маньчжурии могут оказаться до миллиона русских солдат. Финансовые
возможности России также далеко превосходили подсчёты Японии... После
«просчитанного отступления» российские силы восполнили свою военную мощь на
маньчжурской границе". Русская армия вышла к Телину, а затем отступила ещё
дальше на север к подготовленным сыпингайским позициям (в 175 км от Мукдена).
Там она и оставалась до конца войны, пополняя силы и готовясь к переходу в
новое наступление, которое так и не состоялось в связи с заключением мира.
Потрясенные огромными потерями в Мукденской операции японцы прекратили всякие
активные действия. Пополнять армию было некем и нечем. Ограниченные людские и
материальные ресурсы Японии закончились. Маршал И. Ояма доносил в Токио, что
его войска не имеют ни людей, ни боеприпасов, чтобы дальше противостоять
могущественному противнику, силы которого увеличивались с каждым днем. Под
Мукденом уже были сконцентрированы все сухопутные силы страны, её военная мощь
была напряжена до предела. Даже японским генералам было вполне очевидно, что к
ещё одному генеральному сражению Япония толком подготовиться уже не может.
Начальник Генерального штаба маршал Ямагата Аритомо отмечал, что опасно
продолжать войну с упорным противником, большая часть войск которого ещё
находится дома, тогда как японцы «свои силы уже истощили».

\iusr{Галина Полякова}
\textbf{Олег Курилов} 

Замечательно интересно. Спасибо. Но и Россия не могла продолжать военные
действия. Вероятно, все же сказывались огромные расстояния. Россия нахватала
себе огромные пространства, которые не могла освоить. Эта хворь у них по сей
день не проходит.

\end{itemize} % }

\end{itemize} % }

\iusr{Mykhaylo Losytskyy}

З \enquote{Історії дипломатії} (книжка радянська, що надало викладу певного відтінку)
про переговори по Сахаліну: "Витте отказался и от уступки Сахалина. Японское
правительство стало перед вопросом, продолжать ли войну ради захвата этого
острова. Кабинет и Совет генро собрались на совместное заседание. Оно длилось
целый день и всю ночь. Было решено, что Япония так истощена, что больше воевать
не может. В присутствии императора было вынесено решение отказаться от
Сахалина. Это произошло 27 августа 1905 г.

Между тем за несколько дней до этого, стремясь скорее покончить с войной,
Рузвельт послал царю телеграмму, в которой советовал уступить Сахалин Японии.
23 августа царь принял американского посланника и заявил ему, что в крайнем
случае согласен отдать южную половину острова. Царь готов был на любой мир,
лишь бы развязать себе руки для подавления надвинувшейся революции.

Случайно заявление царя стало известно японцам. Они узнали о нём тотчас же по
окончании упомянутого заседания 27 августа. Японское правительство изменило
своё решение. Правда, морской министр заявил, что если информация о согласии
царя не верна, передавшему её чиновнику придётся произвести себе харакири.
Однако, сокрушался министр, это не вернёт Японии возможности заключить столь
необходимый мир. Главе японской делегации в Портсмут была послана инструкция
требовать южной части Сахалина. Витте уступил, следуя велению царя: японцы
получили часть острова к югу от 50-й параллели северной широты. Этот эпизод
свидетельствует, до какой степени Япония была истощена войной."

\begin{itemize} % {
\iusr{Галина Полякова}
\textbf{Mykhaylo Losytskyy} 

Але й Росію та війна виснажила. Було втрачено дві ескадри, немало живої сили,
Порт-Артур... А престиж? у яких одиницях його вимірювати?

\begin{itemize} % {
\iusr{Mykhaylo Losytskyy}
\textbf{Galina Poliakova} 

- так, звичайно. Просто цікаво розписано, як приймалися рішення і японцями і
росіянами. Обидвом країнам був потрібен мир, але кожна хотіла вийти з війни з
максимальним прибутком / мінімальними втратами. Тому кожна сторона мала
вимагати максимум з можливого, не перетнувши ту червону лінію, зайшовши за яку,
противник вирішить, що ну його такий мир, краще далі війна. І як знати, де та
лінія?

\iusr{Галина Полякова}
\textbf{Mykhaylo Losytskyy} Мабуть саме тому видатних дипломатів згадують з такою повагою?

\iusr{Олег Курилов}
\textbf{Галина Полякова} 

НЕт вы не правы. Россию эта война никак не истощила! Потеря флота это очень
болезненный удар. но совсем не смертельный. Так как флот всё же играл
второстепенную роль для Росии учитывая театр военных действий - Манджурию.
Потеря престижа - да, Согласен. Но его можно было вернуть победой в сухопутной
битве. Учитывая истощение сил Японии, это было вполне возможно. Главная
проблема России внутри неё! Восстания 1905года!

\iusr{Галина Полякова}
\textbf{Олег Курилов} 

Не буду спорить. Но факт есть факт: Россия капитулировала. Одной причины не
бывает, их целый комплекс. Но результат все равно один.

\iusr{Олег Курилов}
\textbf{Mykhaylo Losytskyy} 

\enquote{Історії дипломатії} мне вообще очень нравиться этот трёхтомник. И
дипломатия 19 и начала 20 века там отлично описана.

\iusr{Олег Курилов}
\textbf{Галина Полякова} 

Извините, термин капитуляция здесь не уместен.  @igg{fbicon.smile} . Капитуляция это полное
поражение и сдача на милость победителя. Этот как Германия и Япония во второй
мировой или допустим Франция потерявшая всё в 1870 и в 1940. Тут этот термин
совсем не подходит  @igg{fbicon.smile} . Кроме того. япония первая запросила мира.  @igg{fbicon.smile} 

\iusr{Галина Полякова}
\textbf{Олег Курилов} 

Вы просто умничка! Я этот трехтомный труд только частично прочла. Очень уж
замысловато. Мне ближе и понятнее мемуары Бьюкенена и Палеолога. А Вы с
\textbf{Mykhaylo Losytskyy} такую махину одолели! Даже завидно.

\iusr{Mykhaylo Losytskyy}
\textbf{Олег Курилов} 

- у мене лише перших два. Мій дід, мабуть тому що вважав, що про часи СРСР в
СРСР правди не напишуть, третього не купив  @igg{fbicon.smile} 

\iusr{Олег Курилов}
\textbf{Mykhaylo Losytskyy} \textbf{Галина Полякова} 

Извините. Но я сам третий том не читал  @igg{fbicon.frown}. Не интересно. У
меня вполне спокойное отношение к СССР. Просто не интересно
@igg{fbicon.smile}. Кроме того История второй мировой немного знаю, А там
будет крайне предвзятое отношение учитывая время создание книги.


\iusr{Олег Курилов}
\textbf{Галина Полякова} А за статью о Витте Огромное Спасибо!!!! @igg{fbicon.bouquet} 

\end{itemize} % }

\end{itemize} % }

\iusr{Татьяна Шиверская}
Жаль, что нет сейчас такого Витте в Украине!

\iusr{Volodymyr Nekrasov}
Дякую! Цікаво

\iusr{Тамара Корженко}
У наших бьіл шанс, они начали с низов, но не доросли...

\iusr{Сергей Хромешкин}

\ifcmt
  ig@ name=scr.hands.applause.bravo
  @width 0.2
\fi

\iusr{Игорь Кокарев}
Каким слогом написано и как занимательно! Шедевр репортажной журналистики. Спасибо!

\iusr{Галина Полякова}
\textbf{Игорь Кокарев} Спасибо.

\iusr{Світлана Куликова}

\ifcmt
  ig@ name=scr.hands.applause
  @width 0.2
\fi

\iusr{Константин Кравченко}

...у Сергея Юльевича был - замечательный племянник. Вошедший в историю как -
...лейтенант Шмидт... Которого он всю жизнь - вытаскивал из всех его неурядиц.
Также и более близкий дядя - адмирал... Кроме правда последнего - неудавшегося
восстания в Севастополе. Где он командовал - крейсером \enquote{Очаков}... Но с него
сталось, в следствии тех причин, что он дваждый находился на обследовании в
психиатрической клинике в Петербурге. С диагнозом - ...шизофрения... Осталась -
история болезни... Что до дядюшки, то я годами практикую ритуал, проходя мимо
бюста - поглаживаю по мудрой голове... После чего - получаю откат - в
деньгах... И всем -+ рекомендую... Да-с...

\begin{itemize} % {
\iusr{Галина Полякова}
\textbf{Константин Кравченко} 

Как интересно. Для меня это новость. Знала, что у Петра Шмидта был дядя
-адмирал и сенатор Владимир Петрович Шмидт, который а много с ним возился. А
вот про родство с Витте ... Подскажите, пожалуйста, источник! Это очень
интересно. Как же замысловато переплетены родственные связи: тут и Елена
Блаватская, и Шмидт. Спасибо за коммент.

\iusr{Виктория Шевелёва}
\textbf{Константин Кравченко} беру на вооружение)) а где бюст-то?

\begin{itemize} % {
\iusr{Константин Кравченко}
\textbf{Виктория Шевелёва} ...стыдно - матушка - должно быть. Сами найдите и обрящете...

\iusr{Виктория Шевелёва}
\textbf{Константин Кравченко} спасибо, уже нашла)

\iusr{Виктория Шевелёва}
Вообще не стыдно, кстати
\end{itemize} % }

\iusr{Галина Полякова}
\textbf{Константин Кравченко} 

хорошо порылась, но о родственной связи Шмидта и Витте не нашла ничего. Не
обессудьте. Пришлите хоть намек на источник. Спасибо!

\end{itemize} % }

\iusr{Лариса Мысник}
Дуже цікаво. Спасибі.

\iusr{лариса погосова}
Вот кого нам нынче не хватает в государстве!!!

\iusr{Евгений Маринич}
Таких можно пересчитать по пальцам.

\iusr{Лариса Лымарева}
Честь - не просто слово. Это образ жизни.

\iusr{Олег Курилов}
Очень легко читается и замечательно написано. Спасибо!

\iusr{Людмила Скомаровська}
Как всегда интересно и оптимально читабельно! Спасибо.@igg{fbicon.heart.exclamation}

\iusr{Наталія Соколик}
Змістовно, цікаво, доступно!
Дякую. @igg{fbicon.face.happy.two.hands} 

\iusr{Anatoliye Anatoliy}
Маленькая ошибка. Этот памятник находится недалеко от улицы Леонтовича, на улице Лысенко

\begin{itemize} % {
\iusr{Галина Полякова}
\textbf{Anatoliye Anatoliy} Совершенно верно. Я запуталась а композиторах. Уже и позором умылась!

\iusr{Anatoliye Anatoliy}
\textbf{Galina Poliakova} не надо посыпать голову пеплом. Всё остальное в заметке хорошо
\end{itemize} % }

\iusr{Ирина Лещенко}
на улице Лысенка

\iusr{Галина Полякова}
\textbf{Ирина Лещенко} Вы абсолютно правы.

\iusr{Tatiana Volkova}
Спасибо. Конкректно кино.

\iusr{Нина Алексеева}
Благодарю вас.

\iusr{Дмитрий Фоменко}
Блестящий менеджер! Но как вам Николай 1? - ...Поскольку мала была зарплата
чиновника, Государь решил доплачивать из своего кармана!!!

\iusr{Нина Гордийчук}
Спасибо за интересную информацию о графе Витте.

\iusr{Алла Парадня}
Спасибо.

\iusr{Natalia Marevic}
Спасибо, очень интересно.

\iusr{Татьяна Горовенко}

Россия во все времена высасывала всё лучшее из Украины - богатства недр,
культурное наследие, талантливых людей, оставляя её на переферии и унижая.

\begin{itemize} % {
\iusr{Yevgeniy Goldshtein}
\textbf{Татьяна Горовенко} вот Брежнева с его днепропетровскими друзьями  @igg{fbicon.wink} 

\iusr{Валентина Прибутько}
\textbf{Татьяна Горовенко} комент совсем не по существу

\iusr{Юрий Стебельский}
\textbf{Татьяна Горовенко} 

но уже 30 лет Россия ничего не \enquote{высасывает}. Где культура? Где таланты? Где
небывалый расцвет когда-то пятой экономики Европы?

\begin{itemize} % {
\iusr{Татьяна Горовенко}
\textbf{Юрий Стебельский} выхолостили....ждём, когда новые ростки появятся.

\iusr{Галина Полякова}
\textbf{Юрий Стебельский} А СЕМЬ лет войны со счетов сбросим? А до войны разве Россия не паслась у нас?

\iusr{Юрий Стебельский}
\textbf{Татьяна Горовенко} ага, и ростки новые появятся, и сотни не Россией попиленных индустриальных гигантов заново отстроятся. Ждём...

\iusr{Юрий Стебельский}
\textbf{Галина Полякова} 

семь лет войны? А разве в 2015-м году минские соглашения Украиной не подписаны?

С тех пор активных боевых действий на территории Украины не ведётся.
Бомбардировок и артобстрелов, танковых клиньев и котлов нет. Шесть лет как нет.


\iusr{Галина Полякова}
\textbf{Юрий Стебельский} 

И тем не менее у нас около 1 400 000 внутренне перемещенных лиц, у на каждый
день, повторяю: каждый день, есть убитые или раненые. У нас тысячи семей,
потерявших своих сыновей, мужей, отцов. У нас тысячи молодых парней, ставших
инвалидами. Вам мало?

\iusr{Юрий Стебельский}
\textbf{Галина Полякова} 

может, это вам мало?

Выполняйте подписанные Украиной шесть лет назад минские соглашения и тем самым
возвращайте ордло в состав Украины. Это надо делать как можно скорее.

\iusr{Галина Полякова}
\textbf{Юрий Стебельский} 

Это долгий и сложный процесс. И следует помнить, что должны учитываться не
только пожелания России, но и пожелания Украины. Россия не должна нам
диктовать. Минские соглашения нуждаются в кардинальной доработке. Мне кажется,
что обсуждения этого вопроса не соответствует теме этой группы. Кроме того,
наши взгляды очень расходятся, и спор ни к чему не приведет. Я слишком хорошо
помню, как в Слвянске в феврале 2015 принимала беженцев из Дебальцево.

\iusr{Юрий Стебельский}
\textbf{Галина Полякова} 

гарантами выполнения подписанных Украиной минских соглашений являются Германия
и Франция. Они тоже диктуют Украине, что и как выполнять согласно
международного договора, подписанного Украиной в 2015-м и согласованного
Зеленским в 2019-м.

\iusr{Галина Полякова}
\textbf{Юрий Стебельский} Угу. И Будапештского меморандума тоже?

\iusr{Юрий Стебельский}
\textbf{Галина Полякова} 

и его тоже, который Украина первой и нарушила похерив внеблоковость -
устремившись в НАТО, продав США новую военную технику, стоящую на вооружении
РФ, участвуя с НАТО в оккупации Ирака и послав в Грузию буки с укрэкипажами.

\end{itemize} % }

\iusr{Галина Полякова}
\textbf{Татьяна Горовенко} К сожалению, все именно так. Но мы пробьемся!

\iusr{Всеволод Файнберг}
\textbf{Татьяна Горовенко} Это так. К сожалению. При совке было то же самое.

\begin{itemize} % {
\iusr{Татьяна Горовенко}
\textbf{Всеволод Файнберг} мы возродимся. От Мойсея до государства Израиль сколько прошло тысячелетий? Главное помнить. А украинцы помнят.

\iusr{Всеволод Файнберг}
\textbf{Татьяна Горовенко} Очень надеюсь...
\end{itemize} % }

\iusr{Татьяна Горовенко}
\textbf{Юрий Стебельский} всё так. Воспитаем новых Витте

\iusr{Александр Сорокин}
\textbf{Татьяна Горовенко} все лучшие люди стремились послужить царю и Отечеству.

\begin{itemize} % {
\iusr{Татьяна Горовенко}
\textbf{Александр Сорокин} 

хотелось бы, чтобы и сейчас талантливые люди стремились послужить прежде
вітчизні и закону, а уже после \enquote{царю-олигарху}.

\end{itemize} % }

\end{itemize} % }

\iusr{Нина Бондаренко}
Цікаво, прочитала із задоволенням.

\ifcmt
  ig@ name=scr.hands.applause
  @width 0.2
\fi

\iusr{Yevgeniy Goldshtein}
Спасибо!

\iusr{Ольга Черниченко}
\textbf{Татьяна Горовенко} Так Киев и был Россией!

\begin{itemize} % {
\iusr{Галина Полякова}
\textbf{Ольга Черниченко} НИКОГДА

\iusr{Valeriya Dorofeyeva}
\textbf{Olga Chernichenko} школьный курс истории @igg{fbicon.thinking.face} 

\begin{itemize} % {
\iusr{Олег Курилов}
\textbf{Valeriya Dorofeyeva} 

именно шольный курс истории. Русское царство официально купило у поляков Киев
грубо говоря в середине 17 века и с тех пор по 1917 он был в составе Р.И.
@igg{fbicon.smile}.  Или тут речь идёт про Древнюю Русь? Киев был центром
древней Руси ( по гречески Росси )  @igg{fbicon.smile} 


\iusr{Галина Полякова}
\textbf{Олег Курилов} 

Я не согласна с такой трактовкой. Какие документы Вы можете назвать в
подтверждение такой сделки купли-продажи?

\iusr{Олег Курилов}
\textbf{Галина Полякова} 

Мирный договор о \enquote{Вечном мире}. Изучайте  @igg{fbicon.smile} . А зачем вам подтверждение? Это
общеизвестный, просто не популярный факт. Там даже документы сохранились о
процессе торга с Польшей за город  @igg{fbicon.smile} . Всё задокументировано. (\enquote{Торг за Київ
йшов кілька місяців. Початково польські посли назвали суму 4 млн злотих (800
тисяч рублів), у відповідь московські дали свою ціну, у 26 разів меншу, — 30
тисяч рублів. Переговори ретельно протоколювалися сторонами.}украинская Вики
одобренная Вятровичем  @igg{fbicon.smile}  ) Официальное присоединение Левобережной Украины и
Киева отторгнутые у Речи Посполитой произошло по Андрусовскому мирному договору
и в дальнейшем закреплённому в \enquote{Вечном мире}. Вот. к сожалению сюда нельзя
ссылки бросать: Покупка Киева — официальное приобретение Киева Русским царством
в результате соглашения с Речью Посполитой в рамках «Вечного мира» 1686 года и
уплаты ей 146 тысяч рублей. Даже в украинском варианте Вики есть  @igg{fbicon.smile} 

\end{itemize} % }

\end{itemize} % }

\iusr{Татьяна Оржеховская}
Вот бы нам таких министров. Благодарю за столь содержательную историю

\iusr{Люба Кобцева}
Вот с кого надо брать пример нашим правителям-бездарям!

\iusr{Павел Хорт}
А нынче всё наоборот........, увы

\iusr{Людмила Кушнаренко}

\ifcmt
  ig https://scontent-frx5-2.xx.fbcdn.net/v/t39.1997-6/s480x480/47472862_1846683478763869_5271603212267290624_n.png?_nc_cat=1&ccb=1-5&_nc_sid=0572db&_nc_ohc=SXe04OJrfkEAX-UBYtP&_nc_ht=scontent-frx5-2.xx&oh=00_AT8BAU20IYp_4eM1eUxB5uFcs13Ylcj30BfcyG8oQW953A&oe=61D04A27
  @width 0.2
\fi

\iusr{Константин Бебко}
улица не Леонтовича, а Лысенка

\begin{itemize} % {
\iusr{Галина Полякова}
\textbf{Константин Бебко} Вы абсолютно правы. Спасибо за уточнение. И я уже трижды повинилась в том, что спутала имена двух выдающихся композиторов.

\iusr{Константин Бебко}
\textbf{Галина Полякова} Бывает...ничего страшного  @igg{fbicon.face.smiling.halo} 
\end{itemize} % }

\iusr{Елена Тарасова}
Спасибо!!!

\iusr{Нора Устименко}

Спасибо за ценное и интересное сообщение. Все ошибочки и неточности ерунда, не
это было главное, а главное -раскрыть личность, с чем Полякова достойно
справилась. Понятно на каких примерах можно воспитывать детей. Сейчас таких
людей - днём с огнём не найдёшь. Какая широта личности!!! Сколько самоуважения-
отказать самому царю! Он то понимал масштаб своей личности и царь понял.
Блеск!!

\iusr{Галина Полякова}
\textbf{Нора Устименко} Спасибо на добром слове!

\iusr{Arcadia Olimi}

 @igg{fbicon.hands.applause.yellow}  @igg{fbicon.heart.beating} 
Прекрасно!

\iusr{Ирина Бузунова}
Спасибо за такие истории

\iusr{Olena Klymenko}
Спасибо большое, побольше бы таких людей, как Витте С. Ю.

\iusr{Сергей Шевченко}
 @igg{fbicon.hands.applause.yellow} 

\iusr{Людмила Ива}
Спасибо!) @igg{fbicon.rose} 


\end{itemize} % }
