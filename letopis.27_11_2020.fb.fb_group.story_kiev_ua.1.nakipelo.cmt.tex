% vim: keymap=russian-jcukenwin
%%beginhead 
 
%%file 27_11_2020.fb.fb_group.story_kiev_ua.1.nakipelo.cmt
%%parent 27_11_2020.fb.fb_group.story_kiev_ua.1.nakipelo
 
%%url 
 
%%author_id 
%%date 
 
%%tags 
%%title 
 
%%endhead 
\zzSecCmt

\begin{itemize} % {
\iusr{Тома Храповицкая}
Браво Виктория!!!!!!!!!

\begin{itemize} % {
\iusr{Виктория Угрюмова}
\textbf{Тома Захарова} спасибо)))))

\iusr{Тома Храповицкая}
\textbf{Виктория Угрюмова} реально, читая Вас, не могу ничего добавить, только подтвердить, всё читаю!!!!!!!!

\iusr{Александр Бергельсон}
Чёртовы глисты...

\iusr{Тома Храповицкая}
\textbf{Александр Бергельсон} согласна с Вами!!!!!!

\iusr{Галина Гурьева}
\textbf{Александр Бергельсон},Тома Захарова, поддеживаю 100\%!!! Как же они надоели ( мягко сказано (((

\iusr{Тома Храповицкая}
\textbf{Галина Гурьева} да, надоели это ещё мягко сказано! Они мутируют, плодятся как кубло, ненавидя всё и всех, это страшные творения живущие и паразитирующие на всём чем можно поживиться, оставляя после себя только разруху, ненависть и заражая слабых духом!!!

\iusr{Олексій Мачинський}
\textbf{Тома Захарова} все буде добре, гарантую!

\iusr{Тома Храповицкая}
\textbf{Олексій Мачинський} дякую Вам за гарантію, а як же інакше? Головне залишатися людиною з палким серцем і ладом в голові!!!

\iusr{Галина Гурьева}

\ifcmt
  ig https://scontent-frx5-2.xx.fbcdn.net/v/t39.1997-6/p480x480/105941685_953860581742966_1572841152382279834_n.png?_nc_cat=1&ccb=1-5&_nc_sid=0572db&_nc_ohc=EsITRl0vkWIAX_ZscYV&_nc_ht=scontent-frx5-2.xx&oh=00_AT-XMGTI0xGKW37JdMZNxAs23FYgWlcLjHaw72XcKPYT1w&oe=61C1E08B
  @width 0.2
\fi

\end{itemize} % }

\iusr{Леся Фандралюк}
Я с Вами, дорогая Виктория!))

\iusr{Ирина Гукасян}
Метко, «не в бровь, а в глаз».... поддерживаю!!!

\iusr{Тетяна Кутишенко}
Благодарю Вас

\iusr{Svetlana Naulko}
Невероятная Виктория! С благодарностью  @igg{fbicon.hands.raising}  @igg{fbicon.thumb.up.yellow} @igg{fbicon.heart.red}{repeat=3}

\iusr{Марина Прокопенко}
Виктория, полностью Вас поддерживаю!

\iusr{Ольга Сімонова}
Спасибо! Дякую! @igg{fbicon.heart.red} @igg{fbicon.rose}  @igg{fbicon.heart.eyes} 

\iusr{Andriy Demydenko}

Моих предков расстреливали просто посмотрев на руки, но это не повод всех
ненавидеть и усугублять разделение на горожан и приехавших, крестьян и
ремесленников. Разделять надо только профессионалов и бездельников

\begin{itemize} % {
\iusr{Lola Madino}
\textbf{Andriy Demydenko} одного мого прадіда - навіть в цивільному одязі - за виправку...

\iusr{Олексій Мачинський}
\textbf{Lola Madino} а розстрілювали в обличчя, бо такі часи були

\iusr{Lola Madino}
\textbf{Олексій Мачинський} дівчата з гарних родин вчились носити червону косинку та сидіти на безкінечних зборах з відкритим ротом - щоб обличчя не було занадто розумним...
\end{itemize} % }

\iusr{Анна Сидоренко}
Лести не переношу, но с вами согласна.

\iusr{Валентина Луконина}
Я в восторге от Вашего комментария. Я с Вами, Виктория!

\iusr{Наталія Топоровська}
Дякую щиро!

\iusr{Iryna Busol}

Пані Вікторіє, не витрачайте свої сили та нерви на таких людей, прошу Вас! Вони
були, є і будуть. Хай собі проживають у своїй реальності.

Читати Вас - насолода. Для мене честь бути знайомою з Вашими батьками, родиною,
неймовірним дядьком). Про запахи Ви написали так, що я просто зупинилася в
сторонці на тротуарі і не зрушила з місця, доки не дочитала.

Щиро дякую Вам за спогади, у які ніби переносишся і переживаєш сам  @igg{fbicon.hands.applause.yellow}  ☀ ️ 

\begin{itemize} % {
\iusr{Виктория Угрюмова}
\textbf{Iryna Busol} 

я дякую всім моїм читачам, але написала як модератор - бо знали б ви, скільки
такі люди додають адміністративного клопоту, і скільки часу життя вони
намагаються відібрати, точачи скарги - ви пам'ятаєте - пан Олег нещодавно
вивішував комент однієї пані. Справа не в емоціях, я просто озвучую такий
момент - наша група засновувалась саме для спогадів, а з них нічого не
викинеш)) дґякую вам за теплі слова

\end{itemize} % }

\iusr{Valentyna Duhnovska}
Пані Вікторіє! Пишіть. Дуже цікаво. Дякую ВАМ

\iusr{Олег Коваль}

В любом случае, нужно следить за гигиеной общения и гнать от себя таких индивидуумов.

\begin{itemize} % {
\iusr{Ольга Кирьянцева}

Олег, каким образом мы можем их гнать от себя? Пока все эти \enquote{глисты} в группе,
нам остаётся только молча возмущаться, или поддержать, когда кто-то открыто не
начнет протестовать, как в данном случае Виктория или раньше Александр Б.


\iusr{Олег Коваль}
Ольга, в группе существует функция, \enquote{пожаловаться на комментарий}. Действуйте.

\iusr{Анна Солдатова}
\textbf{Олег Коваль} вы модератор. Баньте ботов. Их видно

\iusr{Анна Солдатова}
Я не знала о функции \enquote{пожаловаться на комментарий}. Группа загадка троллями

\iusr{Олег Коваль}
\textbf{Анна Солдатова}, повторяю, я не в силах читать комментарии, делайте, как я просил.
\end{itemize} % }

\iusr{Светлана Мед}
Поддерживаю !!!

\iusr{Мария Андрушко}
Підтримую, дякую, а спогади тільки на краще.

\iusr{Владимир Новицкий}

Дуже гарно сказано, скiльки ще людей якi залишилися там в старому i не
розумiють, що то був страшний час, який калiчив людей не тiльки фiзично, а як
видно i духовно, якщо навiть серед молодих людей ми помiчаемо ознаки пристрастi
до того часу.

\begin{itemize} % {
\iusr{Andriy Demydenko}
\textbf{Vladimir Novitsky} все они понимают, им просто там нравится, как 70 миллионам, проголосовавших за Трампа, как 73\% наших соотечественников  @igg{fbicon.frown} 

\iusr{Олег Коваль}
\textbf{Andriy}, вот, давайте не будем сейчас делить людей на проценты по политическим убеждениям!

\iusr{Andriy Demydenko}
\textbf{Hélg Smith} это не убеждения, у Трампа нет убеждений, да и у нас
\enquote{какая разница} вместо убеждений

\iusr{Олег Коваль}
\textbf{Andriy}, в нашей группе нет места политике.

\iusr{Andriy Demydenko}
\textbf{Hélg Smith} не я начал про политические убеждения

\iusr{Олег Коваль}
\textbf{Andriy}, я сейчас закончу.

\iusr{Анна Сидоренко}
Олег, я вам сочувствую, очень трудно сдержать людей, когда они входят в раж. Я
тоже была заблокирована, но не в Киевские истории.
\end{itemize} % }

\iusr{Натали Сайкевич}
 @igg{fbicon.hands.applause.yellow}{repeat=3}  Браво!
Неперевершено!
Влучно!

\iusr{Svetlana Batrak}

Правильные слова. Особенно сейчас. Вот странно.. в сложный период наоборот
нужно поддерживать друг друга и быть добрее. Но озлобленных становится все
больше((((

\iusr{Irena Visochan}
\textbf{Виктория Угрюмова} Браво-Брависсимо! Полностью поддерживаю, с
благодарностью! @igg{fbicon.heart.suit}

\iusr{Ольга Вакуленко}
100\% правда!

\iusr{Михайло Наместник}

Влучно сказано, пані Валерія. Совок багатьом роз'їв мозок та їхніми руками і
нас відпускати не хоче. Але будьмо доброї думки - на них чекає доля динозаврів.

\iusr{Ольга Юровских}

Благодарю! Меня тоже злит, что детские и юношеские воспоминания, превращают в
политические бессмысленные дебаты. Да мы все родом из СССР, но это не повод
пинать нас, за то что мы помним себя, своих друзей и мороженое по 19 копеек...

Моя бабушка тоже помнила цены до революции, по мы, внучки, относились к этому с
пониманием. Бабушка не любила советскую власть, а дети были коммунистами...

\begin{itemize} % {
\iusr{Irena Visochan}
\textbf{Ольга Юровских} 

Олечка! Браво! Точно так и моя бабушка рассказывала о жизни до революции, всегда
говорила, что при Царе жизнь была лучше, всякий раз просила об этом молчать. И мы
относились с пониманием и почтением! И проценты, и про Голодомор, и про многое
другое, но всегда предупреждала, что нужно молчать, Слава богу, что мы не забываем
наших предков и большинство в нашей группе КИ-единомышленники. @igg{fbicon.hands.pray} 

\iusr{Борис Лушельский}
\textbf{Ольга Юровских}

Оля, Вы абсолютно правы. Мы, наше поколение, родом из СССР. Как бы ни было хорошо
или плохо, но мы выучились, выросли и, теперь, можем критиковать. Нужно помнить и
уважать то, что было.

\iusr{Люда Бабич}
\textbf{Ольга Юровских} 

согласна с Вами на все 100! Мы не тоскуем за колбасой и мороженным, не нужно
упрекать нас \enquote{соком}. Мы выросли в Нашем Городе и просто очень любим тот
старый Киев котрого больше нет. С красавцами каштанами, огромными клубами с
красными каннами и розами, тенистыми дворами в центре и чистыми окраинами
.Очень многие наши оппоненты напоминают мне шариковых и швондеров - всё
разрушить до основания а затем. .....

\iusr{Виктория Угрюмова}
\textbf{Люда Бабич} и главное - мы не можем поменять время и место рождения)

\iusr{Наталя Кудря}

тим більше що вся наша влада корінням прямо в той совок і найгірше з нього на
жаль все залишилося, а от те, що створене працею наших батьків і дідів-медицина
.освіта. підприємства- все знищено- тому й провокують-щоб не сильно питали де
поділося!! словеса вигадають - не буду цитувати - вивертає ...

\end{itemize} % }

\iusr{Наталия Марченкова}
Спасибо, Виктория. Не обращайте внимания на дураков

\iusr{Любов Білоножко}
 @igg{fbicon.hands.applause.yellow}{repeat=3} !

\iusr{Tamara Ivanova}
Написано на одному подиху

\iusr{Владимир Картавенко}

Батьки мого батька були вчителями. Їх знищили безслідно на Біломор каналі бо в
них була садиба в передмісті. Моя дружина писала текстові нормативно правові
документи про Пам"ятки історії Києво-Подолу. Її вбили в день народження нашого
сина, коли син, солдат УБД АТО був на військових зборах. Документи на квартиру
в будинку Пам"ятка історії де була явочна квартира групи Кушніров на Подолі
Юрківька-Фроліківська було підроблено. Я вже не пишу про інше, бо Кирілівка
навпроти мого дому. А питання полягає лише в тому щоби покласти на стіл ДВА
НАЯВНІ рішення Київради що до ДВОХ парків на Рибальському. 5.32 Гектари.


\iusr{Марина Прокопенко}
И ещё раз БРАВО!!!
Прочла с таким удовольствием...
 @igg{fbicon.face.happy.two.hands}{repeat=3} 

\iusr{Олег Коваль}
Сергей Скот заблокирован! Жду новых претендентов.

\iusr{Татьяна Милана}

Это специфика социальных сетей - поиск врага и загон инакомыслящих. Тупости
человеческой предела нет. Либо тут же банить, либо использовать обсценную
лексику, по выбору.

\begin{itemize} % {
\iusr{Andriy Demydenko}
\textbf{Тетяна Свербілова} divide and rule было задолго до соцсетей

\iusr{Олексій Мачинський}
\textbf{Тетяна Свербілова} добривечір, сусідко!
Як на мене, ліпше брати на глузи!

\iusr{Татьяна Милана}
\textbf{Олексій Мачинський} 

Доброго ранку, сусіде  @igg{fbicon.smile}  Кому що більше подобається
@igg{fbicon.smile}  В мене вже немає сил читати все те лайно, що вважають за
необхідне публікувати тут користувачі Фейсбук. Перетворили укрнет на суцільну
війну з уявним ворогом.

\end{itemize} % }

\iusr{Светлана Семенец}
Виктория, спасибо, полностью поддерживаем.

\iusr{Таня Васильева}

\ifcmt
  ig https://scontent-frx5-2.xx.fbcdn.net/v/t39.1997-6/s168x128/16781161_1341101952618574_7704631035023065088_n.png?_nc_cat=1&ccb=1-5&_nc_sid=ac3552&_nc_ohc=tv2YBZ97BCkAX9owrAM&_nc_ht=scontent-frx5-2.xx&oh=00_AT9dSuEQ7aqL9DCWzcotuclbkJx5-6WL59oeeXuu1UIdbQ&oe=61C1E82B
  @width 0.1
\fi

\iusr{Евгения Бочковская}
КРИК ДУШИ СТРАНЫ... голосом и душой Виктории !
Полностью согласна...
Самое легкое, ...облив грязью, уничтожить...
Просто удивительно..., как же не любят
умных людей !
А ОНИ, к сожалению, предпочитают не пачкаться...

\iusr{Tatyana Shpak}
Згодна на всі  @igg{fbicon.100.percent} 

\iusr{Людмила Каштан}

Я согласна. Практически все комментарии в конце концов сводятся к политике.
Забывают, что мы все киевляне и просто любим город, в котором родились и где
прошла жизнь.


\iusr{Микола Амфіногенов}
Мой любимый фонтан в Мариинском парке. Его всегда чистили осенью.

\iusr{Борис Лушельский}

Это часть нашей с вами истории и забывать ее нельзя. К тому же ,наша группа не
политизирована и давайте, дамы и господа не забывать этого.

У каждого из нас и наших предков было своё отношение к тому или другому
событию. Если начать винить, то можно винить всех без исключения: и
коммунистов, и националистов и колобарационистов. Все они погрязли в крови.

Не надо высказывать всё, что думается. Нужно уважать и других людей.

Кто не согласен, может меня забанить.

\begin{itemize} % {
\iusr{Олег Аптекарь}
\textbf{Борис Лушельский} 

это значит, борюсик, что общество больно ненавистью в очень тяжелой форме.. и
врядли при нашей жизни что-то изменится к лучшему... почему немцы не бросаются
друг на друга при виде старых фотографий?.. или французы не обвиняют друг друга
при упоминании маршала петена?... а некоторая часть украинцев, безусловно
меньшая, но жутко агрессивная до умопомешательства, считает себя точкой
отсчета правды, судьей и палачем в одном лице.. это хвороба, которая не
лечится..

\begin{itemize} % {
\iusr{Борис Лушельский}
\textbf{Олег Аптекарь}
Нет, землячок. Тут дело в другом.
Я не говорю о хозяйке этого поста. Может у нее были причины написать это, а в
общем, очень много людей, поменяв строй, тут же меняет свою ариентацию.

\iusr{Олег Аптекарь}
\textbf{Борис Лушельский} не понял, если честно..)) и я автора поста вообще не имел в виду..

\iusr{Борис Лушельский}
\textbf{Олег Аптекарь}
Ну мы же пишем под этим постом.

\iusr{Олег Аптекарь}
\textbf{Борис Лушельский} ну так я с ней согласен..))

\iusr{Борис Лушельский}
\textbf{Олег Аптекарь}
Я, наверное, тоже.

\iusr{Виктория Угрюмова}
\textbf{Борис Лушельский} 

Я написала этот пост не о своих предках - не знаю, читали лли вы мои тексты в
этой групе - я пишу о другом всегда. И я могу не обращать внимания и не обращаю
внимания на всю эту дичь, как автор. Но модераторам все равно приходится
разгребать кучу жалоб и претензий по каждому такому комменту. Я вообще в
наименьшей степени от этого страдаю, но у всех есть жизнь, ничуть не легче и не
проще, чем у других - и получается, что группа, созданная для объединения,
радости и интересных историй, приносит администраторам массу искусственно
созданных доп проблем, начиная с того, что личное время они тратят на всю эту
\enquote{прелесть}. Потом пишут жалобы на группу. Потом снова пишут. И администраторы
уже в ауте. Так и выходит. Мы молчим - значит, все правильно?

\iusr{Борис Лушельский}
\textbf{Виктория Угрюмова}

Да, Виктория. Я понял то, о чем Вы писали. Простите меня за то, что я раскретиковал Вас. Вы правильно написали.
Прошу, ещё раз, прощения у Вас.

\end{itemize} % }

\iusr{Виктория Угрюмова}
\textbf{Галина Баленко} 

спасибо обоим, что согласны. Я о том, что молчание большинства делает историю
такой, какой мы ее потом, вздыхая, принимаем

\end{itemize} % }

\iusr{Yuriy Korogodskyy}

Как-то у себя на странице запостил фото моста Патона в 1980 году. Тут же
получил упрек, что тоскую о временах брежневизма. Так и не понял какая тут
связь.


\iusr{Леонид Яковенко}
Браво, Вiкторiя, браво!!!!!!!

\iusr{Петр Кузьменко}

Цілком вірно сказано, влучно. Хочу додати, що такий порядок речей не зміниться
доти, доки більшість у нашій Величній країні не буде складати народ, а не
\enquote{населення}, як зараз... @igg{fbicon.face.sad.but.relieved} 

\begin{itemize} % {
\iusr{Fabrizio Conti}
Всех уничтожить предлагаете? В зеркало не смотрели?

\iusr{Петр Кузьменко}
\textbf{Fabrizio Conti} 

ни на что не претендую. Уничтожать ни кого не нужно. Это те же большевитские
методы, о которых пишет автор. Просто нужно развиваться, меняться, расти, даже
над собой прежним. Отбросить заблуждения, раньше служившие убеждениями.
Что-что, а пропаганда в СССР, как в любом тоталитарном государстве, была на
высоте.


\iusr{Любовь Линник}
\textbf{Петр Кузьменко}

\ifcmt
  ig https://scontent-frx5-2.xx.fbcdn.net/v/t39.1997-6/s168x128/93118771_222645645734606_1705715084438798336_n.png?_nc_cat=1&ccb=1-5&_nc_sid=ac3552&_nc_ohc=r5H2JVtNyrYAX_WTNLl&tn=lCYVFeHcTIAFcAzi&_nc_ht=scontent-frx5-2.xx&oh=00_AT9YncOXVwULdFsur7BzvazUxsoWWE4MtO0ttDev82h8xg&oe=61C2CB7F
  @width 0.1
\fi

\end{itemize} % }

\iusr{Nina Lanka}

Как бы избавить народ от политического налета в коммуникации о нашей прекрасной
поре жизни - молодости!?

\iusr{Наташа Гордикова-Жарикова}

Набагато більше! Цілком вірно!

\iusr{Наталя Кудря}

найсмішніше те-що людці яке постійно таке пишуть - явно професійні провокатори
- Вас же спровокували чи не так?? а де в нас працювали такі проффі??? так ото
ж!!! найголосніше хто кричить \enquote{лови злодія}?? @igg{fbicon.wink} не
стримуйтесь - пишіть правду у відповідь

\begin{itemize} % {
\iusr{Юлія Качура}

Одного разу зачепила таке чмо в якійсь київській групі. Судячи з того, що з
нього точилося-професіонал. Просто треба стримуватись, а в мене самої так часом
руки зудять...

\iusr{Наталя Кудря}
\textbf{Юлія Качура} 

ну! так я теж поспілкувалась пару разів - але одразу натякнула що провокації -
це патентований засіб бувших служб- і якось устаканилось @igg{fbicon.wink} 

\iusr{Виктория Угрюмова}
\textbf{Наталя Кудря} 

мене не спровокували - і це не крик душі. Це моя загальна відповідь - де
ключове слово - нудно, нудно та дуже знайомо. Але ця група була створена задля
інших цілей, і стояти та мовчати, дивлячись, як під виглядом - я митець, я так
бачу))))) - роблять зовсім інші речі, розпалюючи пристрасті - це ж не під моїм
єдиним постом, я до речі щаслива на мізерну кількість таких коментів -
неприпустимо

\end{itemize} % }

\iusr{Maryna Shavrova}

АБСОЛЮТНО ЗГОДНА З ВАШИМ КРИКОМ ДУШІ!!! ПІДТРИМУЮ ОБОМА РУКАМИ

\begin{itemize} % {
\iusr{Виктория Угрюмова}
\textbf{Maryna Shavrova} 

Дякую за підтримку. Але це не є емоції. Це моя відповідь всім загалом - бо
писати кожному окремо не стану, але й ігнорувати мовчки не буду. Бо це група,
яка створена для радості. І радість тут і буде присутня. Це- так би мовити - не
допис автора, а маніфест модератора.

\iusr{Maryna Shavrova}
\textbf{Виктория Угрюмова} підтримую все: і емоції, і маніфест! )) Хай живе Радість!

\iusr{Maryna Shavrova}
\textbf{Виктория Угрюмова}  @igg{fbicon.rose}  @igg{fbicon.heart.growing} 
\end{itemize} % }

\iusr{Олексій Мачинський}
Вікуся, доню, не переймайся - не хворів Тулуз сифілісом!!!
То таваріщь Лєнін на майовках під Парижом нагуляв...

\begin{itemize} % {
\iusr{Ольга Морозова}
\textbf{Олексій Мачинський} Вам би тільки про сифіліс. Зміст поста, ви не зрозуміли. На жаль.

\iusr{Олексій Мачинський}
\textbf{Ольга Морозова} 

я просто жартую, бо лозунг \enquote{давайтє жіть дружна} у країні з війною, епідемією
та недолугим керівництвом, провокують лишень розпач.

\iusr{Ольга Морозова}
\textbf{Олексій Мачинський} Пост Вікторії не про дружбу. Ваш жарт мені не сподобався, вибачте.
\end{itemize} % }

\iusr{Екатерина маковецкая}

текст 5+, а фото... фото это моё детство в Мариининском парке, фонтан, в котором
мы пускали кораблики... благодарю за фото!!!


\iusr{Людмила Козлова}
Згодна повністю!  @igg{fbicon.thumb.up.yellow} 

\iusr{Алла Квасницкая}

Как же я с Вами согласна! И я уже стараюсь и не спорить с такими. Без толку.
Только нервы себе трепать. А в этой группе мне просто приятно встречать
единомышленников, трепетно относящихся к нашему Киеву и его истории.
Остальное - мимо.

\iusr{Людмила Мойсеєнко}

Активні піонерки й досі нашіптують та задзьобують тих, хто не погоджується з їх
думкою, навіть абсурдною. Колишня вседозволеність дається взнаки.

\iusr{Larissa Lomonosova}

Шановна пані Вікторія! Не зважайте на таких людей, їм бог суддя. Але я розумію,
інколи це важко. Але думайте про добрих і щирих, їх точно більше, і ви будете
зустрічати їх набагато частіше, коли будете вірити і сподіватися. Хай щастить
@igg{fbicon.heart.red}

\begin{itemize} % {
\iusr{Виктория Угрюмова}
\textbf{Larissa Lomonosova} дякую, я не переймаюся особисто, але як модератор груапи я бачу. скільки клопоту це викликає саме адміністративно - ці жлюди пишуть скарги, у групи, як у утворення - проблеми. Це таки віднімає час)
\end{itemize} % }

\iusr{Ирина Халед Эльсаед}

Очень точно !!! Кто то видит грязные лужи, а кто- то, отражнние неба и кресты
на куполах. Кто-то упивается фактами психического расстройства Достоевского,
проблемами с морфием Чехова, несовершенством Булгакова... А кто то замирает от
восторга перечитывая безсмертные строчки... Каждому свое... Спасибо за
откровение ! @igg{fbicon.thumb.up.yellow}  @igg{fbicon.face.blowing.kiss} 


\iusr{Ольга Морозова}
Спасибо за пост, Виктория.

\iusr{Неля Архипова}

Ви чудовi, ви так яскраво та точно описуете те, що вiдчувае тiльки справжня,
порядна, та чесна людина, але не може це сформувати в такiй точнiй та
довершенiй формi. Дякую вам за це та низькiй уклiн за однодумнiсть та
талановитiсть. Пишiть, будь ласка, слухати вашi думки-справжня насолода, яка
надихае та окриляе тим, що такi люди, як ви, iснують.

\iusr{Ирина Вержбицкая}
Дякую за чудові проникненні дописи, пишіть будь ласка ще

\iusr{Герман Кудинов}
Сколько пафоса. Смысла ноль. И Достоевский кстати, нудное чтиво.

\begin{itemize} % {
\iusr{Мария Иванова}
\textbf{Герман Кудинов}, извините, вы не подскажите, а кто такой Герман Кудинов? Кто такая Виктория Угрюмова мы знаем, тем более знаем кто такой Достоевский, а это что за личность? В белом пальто?

\begin{itemize} % {
\iusr{Неля Архипова}
\textbf{Мария Иванова} Герман Кудинов добровольно взял на себя роль наглядного примера авторского поста.

\iusr{Герман Кудинов}
\textbf{Мария Иванова} Всем спасибо. Поизощрялись в остроумии? Наглядный пример-это зашоренность советской школьной программой. Вот тут комментируют прекрасные образчики жертв типичного советского мышления. Или думай как все-шаблонно, или на тебя выльют тонны помоев. Отлично! Еще раз всем спасибо!
\end{itemize} % }

\iusr{Лариса Петрова}
Герман, очччень может быть, классика вообще штука для шевеления не только
мозга, но и еще много чего малопонятного в нас самих

\iusr{Лариса Петрова}
Мы поняли, новое мышление - Достоевского ГЕТЬ, он нудный

\begin{itemize} % {
\iusr{Герман Кудинов}
\textbf{Лариса Петрова} Не просто нудный. Пустой. И очень сильно переоцененный. И вряд ли Вы это поняли.
\end{itemize} % }

\iusr{Лариса Петрова}

Братья Карамазовы - пустое произведение?? Ну, тогда и Тургенев, к примеру,
тоже нудный? А кто у вас суперписатель?

\begin{itemize} % {
\iusr{Герман Кудинов}
\textbf{Лариса Петрова} Только не рассказывайте мне, что Вы регулярно перечитываете \enquote{Вешние воды} и \enquote{Братьев Карамазовых}. Хороших и отличных писателей очень много. Это те к которым хочется возвращатся и перечитывать раз за разом. Хватит уже советской школьной программой забивать мозги.
\end{itemize} % }

\iusr{Лариса Петрова}
Правельно. Декомунизировать.

\iusr{Герман Кудинов}
\textbf{Лариса Петрова} У вас мозг скоммуниздили, уж извините.

\iusr{Лариса Петрова}
Желаю вам нескучного чтива и прекращаю эту полемику. Вы мне не интересны.

\iusr{Герман Кудинов}
\textbf{Лариса Петрова} Можно подумать я рвался буквально с вами общаться @igg{fbicon.face.tears.of.joy} 

\end{itemize} % }

\iusr{Галина Буркацкая}

Дякую за ваші дописи. Вони завжди як повітря, сповнене приємним ароматом
дитинства, де Київ з його красою і складним життям. Головне - вони дають
більшості людей добро і позитивні відчуття. Даріть і далі добро. Дякую. На
тролів не зважайте.


\iusr{Виктория Угрюмова}
\textbf{Галина Буркацкая} дякую)))

\iusr{Тамара Кулиева}
Вика! Спасибо тебе за возможность погружаться в прекрасные воспоминания

\ifcmt
  ig https://i2.paste.pics/3a0814a8ffc79aa626e30c7c6f17703f.png
  @width 0.2
\fi

\iusr{Ирина Горбенко}
Хорошо-складно пишет УгрюмовА  @igg{fbicon.face.grinning.sweat} 

\iusr{Людмила Новичкова}
Дорога Вiкторiя !
Менi б дуже хотiлось вас зараз просто обiйняти. Все розумiю i cприймаю як свою бiлю.
Можливо колись свiт стане чеснiшим i благороднiшим i те погане лишиться в минулому.

\begin{itemize} % {
\iusr{Виктория Угрюмова}
\textbf{Людмила Новичкова} 

Дорога моя, дякую. Але мені не боляче, це просто загальна відповідь на такі
коментарі - аби не витрачати час. Це, скоріше, заклик та маніфест - знаєте, не
хочеться мовчати, коли читаєш такі коменти, але й сперечатися марно - бо вони
народжені для цих суперечок. Нудно))) Нудно та марно. А я пишу багато років і
знаю, що таких прекрасних, чудових, ніжних людей, як ви - їх більше. І світ
стоїть на них))) обіймаю вас. І мені накипіло за всіх)))


\iusr{Людмила Новичкова}
\textbf{Виктория Угрюмова}
Дякую ! Дякую за все до Ви робите ! Бажаю успiхiв та натхнення !
\end{itemize} % }

\iusr{Валентина Белозуб}

Дякую за чудовий емоційний пост! На жаль, не всі розуміють Ваші емоції ! Але
більшість Ваших прихильників вдячні Вам! @igg{fbicon.person.raising.hand} 

\begin{itemize} % {
\iusr{Виктория Угрюмова}
\textbf{Валентина Белозуб} 

Дякую Це не є емоції. Це моя відповідь всім загалом - бо писати кожному окремо
не стану, але й ігнорувати мовчки не буду. Бо це група, яка створена для
радості. І радість тут і буде присутня. Це- так би мовити - не допис автора, а
маніфест модератора

\end{itemize} % }


\iusr{Юлия Семеренко}

Виктория, такие люди были во все времена. Их не переделаешь, да и напрасный это
труд.их удел-это зависть, желчь и прочее, которые их же и съедают. Радостно
одно, что их все равно меньше. Наболело, а им радостно, если на их комментарии
кто-то откликается, возможно, в их окружении уже никто на них не обращает
внимания, поэтому и ищут новые \enquote{чистые реки, куда можно слить помои}. Будьте
здоровы и берегите себя. Улыбок Вам и радости. Даже в плохом можно найти
хорошее. Сколько людей из группы Вас поддержало, значит хорошего больше. Ура.

\begin{itemize} % {
\iusr{Виктория Угрюмова}
\textbf{Юлия Семеренко} 

спасибо - но я вот всем разъясняю свою позицию)) я могу не обращать внимания и
не обращаю, как автор. Но модераторам все равно приходится разгребать кучу
жалоб и претензий по каждому такому комменту. Я вообще в наименьшей степени от
этого страдаю, но у всех есть жизнь,ничуть не легче и не проще, чемудругих - и
получается, что группа, созданная для объединения, радости и интересных
историй, приносит администраторам массу искусственно созданных доп проблем,
начиная с того, что личное время они тратят на всю эту \enquote{прелесть}. Так и
выходит. Мы молчим - значит, все правильно?

\end{itemize} % }

\iusr{Юлия Семеренко}

\ifcmt
  ig https://scontent-frx5-2.xx.fbcdn.net/v/t39.1997-6/s168x128/16781161_1341101952618574_7704631035023065088_n.png?_nc_cat=1&ccb=1-5&_nc_sid=ac3552&_nc_ohc=tv2YBZ97BCkAX9owrAM&_nc_ht=scontent-frx5-2.xx&oh=00_AT9dSuEQ7aqL9DCWzcotuclbkJx5-6WL59oeeXuu1UIdbQ&oe=61C1E82B
  @width 0.1
\fi

\iusr{Инна Иванова}
Очень верно! Спасибо!

\iusr{Мария Богданова}

Виктория, вы пишете прекрасно! Кроме хорошей ,правильной речи, (боже мой, это
сейчас такая редкость...), ...кроме таких близких и понятных для каждого
киевлянина историй-воспоминаний,... в ваших рассказах всегда присутствует
доброжелательность и интеллигентность...

Вы одна из тех, кто создал дух этой страницы...

Зачем вы обращаете внимание на дураков ? ...Ну, не стоит, пожалуйста... Посмотрите
скольким людям вы приносите радость своими чудесными историями!...

Пишите,... киевский дух порядочности, неконфликтности, умение замечать и ценить
красоту и неповторимость нашего Города... это так ценно в наши дни... Спасибо
вам.

\begin{itemize} % {
\iusr{Юлия Семеренко}
\textbf{Мария Богданова} Вы написали не менее красиво @igg{fbicon.thumb.up.yellow}  @igg{fbicon.rose} 
\end{itemize} % }

\iusr{Татьяна Грицай}
Виктория, Вы меня очаровываете своими воспоминаниями. Бесконечно благодарна за
теплоту текстов и образность. @igg{fbicon.heart.with.ribbon}  @igg{fbicon.rose}{repeat=5} 

\iusr{Andrei Dvoynos}

Одна невеличка помилка... там по контексту треба було написати \enquote{... вчені які
були проти повертання річок у зворотному напрямку}.

Хоча може то я такий нудний?  @igg{fbicon.smile} 

А допис про салон квітів прекрасний! Читаючи Вас кусав лікті бо ніколи не
зайшов у цей салон ні в кулінарію поруч із «Столичним»!  @igg{fbicon.smile} 


\iusr{Марина Парфентиева}

Громадное спасибо за тёплые воспоминания. Не жившие в то время в Киеве -не
поймут очарования «Зеленого театра», Симфонических концертов в ракушке
Мариинского. Как после катка «Динамо» идёшь в открытый бассейн »Динамо» А потом
пирожки с мясом-4коп томатный сок. в «Соки воды пирожки !!» Аж дух
захватывает! Спасибо Вам!!!

\begin{itemize} % {
\iusr{Edita K. Ishchuk}
\textbf{Марина Парфентиева} 

а горячее молоко у метро Крещатик, помните? А кофе на Заньковецкой у Пассажа?
Всегда на улице стояли с чашечкой кофе и разговоры, разговоры.... так мило

\iusr{Yuri Arkadyev}
\textbf{Марина Парфентиева} причём томатный сок в гАСТРОноме (на разлив в гранёный стаканчик) был в 2х вариантах: уже с солью или соль отдельно рядом в баночке с ложечкой  @igg{fbicon.face.smiling.sunglasses}  @igg{fbicon.hand.ok}  @igg{fbicon.thumb.up.yellow} 

\iusr{Olena Klymenko}
\textbf{Марина Парфентиева} 

Помните надпись на огромной рекламе, висевшей на доме, справа от стадиона
\enquote{Динамо}, которая гласила: \enquote{Пийте друзі вітаміни натуральні, свіжі соки, і
рум'янцем неодмінно запалають ваші щоки}. Внизу под бигбордом, возле дома,
стояла будочка сапожника. В ней работал папа моей одноклассницы Тони. Вот такие
странички возникают в памяти...

\begin{itemize} % {
\iusr{Марина Парфентиева}
\textbf{Olen A Klymenko} 

до сих пор при случае само цитирую. похоже мы сВами. из одной эпохи. Ходили в
зелёный театр в первый день показа. яМожно было увидеть всех знакомых, которых
давно не видела.

\end{itemize} % }

\end{itemize} % }

\iusr{Георгий Майоренко}

Уважаемая Виктория, а можно узнать, в чем конкретно претензии недоброжелателей?
Пишете вы интересно, размещаете замечательные фото. Что же им не нравится?

\begin{itemize} % {
\iusr{Edita K. Ishchuk}
\textbf{Георгий Майоренко} 

этим людям не чем в жизни заняться. Сидят в Фейсбуке и нудят... всё не так и не
эдак. Специально вызывают людей на споры и подпитываются энергией. Вы заметьте,
они же прямо входят в раж при спорах, заводят в такие исторические дебри, что
люди уже и не помнят, о чем пост и какая тема. Просто не надо вступать в
полемику с этого рода людьми и они сами уйдут, им будет не интересно. Ведь они
никого не задели, не обидели, не довели до истерики. Цель не достигнута. И вообще
это люди, обиженные на жизнь.

\begin{itemize} % {
\iusr{Георгий Майоренко}

Знаете, всегда интересны аргументы оппонентов. Мне кажется, у Виктории столь
безупречная репутация, что я даже не представляю, за что ее можно \enquote{зацепить}.


\iusr{Наталия Новикова}
\textbf{Георгий Майоренко}
Жора, они просто энергетические вампиры.

\iusr{Георгий Майоренко}
\textbf{Наталия Новикова} Это ясно, что вампиры. Просто интересно, к чему конкретно цепляются?

\iusr{Виктория Угрюмова}
\textbf{Георгий Майоренко} 

ну, зацепить меня лично трудно - и сразу - благодарю за столь высокую оценку.
Но можно написать под фотографией моего прапрадеда, героя русско-турецкой -
надеюсь, на этого оккупанта нашлась у турок пуля. Самое замечательное в этой
истории то, что другую мою прапрабаку другой мой прапрадед как раз с этой войны
и привез - и на многих друзей семьи пуля нашлась. И я как раз и пишу все время
- что мои предки воевали друг против друга - еще при Азенкуре, и при
Грюнвальде, и где только не воевали, потом их дети женились, а их внуки опять
воевали. Так что дело в желании зацепить - по работе ли, по- как верно замечено
- нехватке энергетической. Реагировать душой на это не стоит, но и молчать
постоянно тоже нерилично. Потому что когда передергивают факты, случается
однажды не то, что хотелось бы

\iusr{Георгий Майоренко}
\textbf{Виктория Угрюмова} 

Спасибо, что ответили. Думаю, такой бред под фотографией мог написать либо
глупец, либо неадекват. А с такими как спорить? Писать на глупости глупости?
Тот ли уровень?

\iusr{Виктория Угрюмова}
\textbf{Георгий Майоренко} 

дело не в глупости и не в неадеквате. я могу не обращать внимания и не обращаю
на это внимания, как автор. Но модераторам все равно приходится разгребать кучу
жалоб и претензий по каждому такому комменту. Я вообще в наименьшей степени от
этого страдаю, но у всех есть жизнь,ничуть не легче и не проще, чем у других -
и получается, что группа, созданная для объединения, радости и интересных
историй, приносит администраторам массу искусственно созданных доп проблем,
начиная с того, что личное время они тратят на всю эту \enquote{прелесть}. Так и
выходит. Мы молчим - значит, все правильно?

\iusr{Георгий Майоренко}
\textbf{Виктория Угрюмова} 

А в чем проблемы модераторов? Если коммент не соответствует правилам группы -
удалять! Допустим, вы привели пример оскорбительного комментария и он явно
нарушает правила, так в корзину его мусорную!

\iusr{Виктория Угрюмова}
\textbf{Георгий Майоренко} 

у ФБ свои правила - и некоторые эти товарищи потом пишут жалобу уже на группу,
что их личное мнение и тыпы, обижают, наших бьют))) это уже скучно - да и
просто вычищать вручную все эти комменты - если бы вы знали, сколько их на
самом деле. Опять-таки возвращаемся к вопросу времени. Я - будучи модератором -
могу сама разобраться под своим постом, например. Но я не так часто захожу в ФБ
- вот сейчас конец год, я сдаю 4 книги в печать. У меня реально отпадают руки
писать))) и вы не поверите - уже есть пробема с сухожилием, и просто даже
щелкать лишний раз мышкой - правда, не так легко. А те авторы, которые не
являются модераторами, те пишут жалобы администраторам. Онидолжна а) почитать.
Б) отыкрыть страницу, в) наити и выбросить коммент. Теперьберите число
публикаций, число комментов - до 600-800. Уффффф))))

\iusr{Георгий Майоренко}
\textbf{Виктория Угрюмова} 

Ясно. Целая история. Тогда надо просто игнорировать оскорбления. В группе
большинство умных людей. Они поймут. А по таким комментам, кстати, \enquote{дурь
каждого видна}. Тот пример, что вы привели, показали, что комментатор -хам
неадекват из каменного века.

\end{itemize} % }

\end{itemize} % }

\iusr{Yuri Sheluxa}
Очень правильные слова! Согласен.

\iusr{Yan Nudelman}

Вау, как точно Вы описали этих людей! Не обращайте внимания. Они не понимают, о
чем Вы говорите. Возможно эти люди до сих пор чувствуют запах прогнившей КПСС.
Несчастные, они неспособны радоваться воспоминаниям. Нет, не ностальгировать,
но просто радоваться. То, что Вы пишите, не для них , случайных попутчиков.

\begin{itemize} % {
\iusr{Виктория Угрюмова}
\textbf{Yan Nudelman} 

спасибо - я не обращаю, но я полагаю, что порой следует напоминать, что выходцы
из того, что они называют \enquote{совком} - именно они. И молчать не следует, а
переживать тоже нет смысла - вон сколько великолепных, радостных и ярких людей
в нашей компани

\end{itemize} % }

\iusr{Tetiana Morozova}
КИЕВ—ЧУДО расчудесное и этим сказано всё!

\iusr{Тетяна Слись}
Один з дописувачів ФБ порівняв подібних описаних осіб з глистами. Як на мене, дуже влучно.

\iusr{Zoya Tsarenok}

Брльшое спасибо Вам за ваши прогулки по родному Киеву! Я 20 лет уже живу в
Израиле, и Ваши репортажики, как глоток воздуха. Очень скучаю, но приехать не
представляется возможным. Пока... Так что очень благодарна Вам за возвращение,
хотя бы в виде рассказов.

Будте здоровы, и не берите в голову! @igg{fbicon.face.grinning.smiling.eyes}
Дебилоидов хватает, их не переделать. Берегите себя.
@igg{fbicon.rose}{repeat=3} 

За \enquote{Лилею}, кулинарию отдельное спасибо! Вернули в детство. Я училась в муз.
школе при консерватории, и бегали туда с друзьями каждый раз, как
предоставлялась возможность.

\iusr{Светлана Дубински}

Все это так! Но теребить раны это не путь к исцелению.
Совершенства нет, везде нужно не придавать значению тому или другому.

Но без свободы, я не могу. @igg{fbicon.face.grinning.big.eyes} 	@igg{fbicon.heart.red}

\iusr{Юлиана Бронникова}

Виктория, спасибо огромное за, то что я все это читаю, и жду новенького! Никого
не слушайте на никому не отвечайте! Берегите свой ум, свое сердце, свою душу! И
самое главное, продолжайте писать! Жду, с нетерпением!

\begin{itemize} % {
\iusr{Виктория Угрюмова}
\textbf{Юлиана Бронникова} 

не могу не отвечать. Сейчас поясню. я могу не обращать внимания и не обращаю,
как автор. Но модераторам все равно приходится разгребать кучу жалоб и
претензий по каждому такому комменту. Я вообще в наименьшей степени от этого
страдаю, но у всех есть жизнь,ничуть не легче и не проще, чемудругих - и
получается, что группа, созданная для объединения, радости и интересных
историй, приносит администраторам массу искусственно созданных доп проблем,
начиная с того, что личное время они тратят на всю эту \enquote{прелесть}. Так и
выходит. Мы молчим - значит, все правильно?

\end{itemize} % }

\iusr{Марина Перепелиця}
Браво, Виктория!

\iusr{Оксана Дубинина}

Виктория, вижу, как Вас \enquote{задело}, хочу поддержать - я с Вами @igg{fbicon.heart.red}  А еще не на
каждую \enquote{моську} стоит обращать внимание. Успехов Вам и вдохновения!!

\begin{itemize} % {
\iusr{Виктория Угрюмова}
\textbf{Оксана Дубинина} С

пасибо, но меня не задело, как автора. Меня это задело, как модератора - и меня
в наименьшей степени. Коллеги модераторы устали писать, что у нас группа без
политики. Сколько времени уходит на разгребание жалоб - потому что на каждый
такой коммент идет масса жалоб - то ли закрывать комментарии, то ли удалять
вручную. И я написала о главном - это съедает массу времени. Потом они пишут
жалобы на группу. Это тоже съедает время. И я не думаю, что молчать - это
отличная политика. Наша история кишит молчаливыми людьми, которые молча
смотрели, как с другими что-то делали

\end{itemize} % }

\iusr{Галина Менжинская}
Вікторія! Велике спасибі Вам за ці рядки. Цілком згодна з Вами. Здоров'я.

\iusr{Олена Чепега}

Как откликнулось @igg{fbicon.face.smiling.eyes.smiling} !!! Надо держаться подальше от \enquote{токсичных} людей - категории,
которым всегда все плохо и видят только одну сторону жизни.


\iusr{Галина никифорова}
Виктория! Безумно рада за ваши прекрасные воспоминания давно минувших дней!

\iusr{Ирина Власенко}
Спасибо за память и историю воспоминаний  @igg{fbicon.bouquet} 

\iusr{Лариса Лобановська}

Абсолютно правильно Вы написали, но не обращайте внимания на такие каменты! Тем
более что, если кто-то выставляет прекрасные фотки современного Киева, то точно
так же налетают любители колбасы по 2. 20 и начинается точно такой же шабаш на
тему, как сейчас плохо! Это просто диагноз, это просто такие люди, которым
всегда все плохо и которые никогда ничем не довольны!

\begin{itemize} % {
\iusr{Виктория Угрюмова}
\textbf{Лариса Лобановська} 

я могу не обращать внимания и не обращаю, как автор. Но модераторам все равно
приходится разгребать кучу жалоб и претензий по каждому такому комменту. Я
вообще в наименьшей степени от этого страдаю, но у всех есть жизнь, ничуть не
легче и не проще, чемудругих - и получается, что группа, созданная для
объединения, радости и интересных историй, приносит администраторам массу
искусственно созданных доп проблем, начиная с того, что личное время они тратят
на всю эту \enquote{прелесть}. Так и выходит. Мы молчим - значит, все правильно?

\iusr{Лариса Лобановська}
\textbf{Виктория Угрюмова} 

Понимаете, начинаешь таким людям что-то объяснять, и, во-первых, получаешь ушат
помоев, во-вторых, постепенно втягиваешься в дискуссию с ними и, получается,
опускаешься до их уровня.. Так что, это очень обидно, конечно, получать такие
отклики, но ориентируйтесь лучше на позитивные отклики, тем более их гораздо
больше!!! Пишите, пожалуйста, у Вас очень хорошо получается!

\end{itemize} % }

\iusr{Ирине Вильчинская}

Вікторіє, Ви, як завжди, дуже тонко вловлюєте тенденцію і гарно її
висвітлюєте... А мене ще дуже тішить те, що до групи приєднуються не тільки
кияни (і сучасні, і колишні -зі своїми спогадами, розмірковуваннями), але й не
кияни, яким до душі припадає атмосфера цієї групи та її толерантний настрій... І
це також добра традиція Києва - ми завжди раді добрим гостям!

\begin{itemize} % {
\iusr{Iryna Busol}
\textbf{Ирине Вильчинская} Щодо атмосфери групи - Ви абсолютно маєте рацію! Словами складно описати. Тут так затишно, цікаво, смішно, пізнавально, \enquote{вживотілоскотно}, коли з'являється наступний шедевр. Які талановиті тут люди! Читаєш і насолоджуєшся!

\iusr{Ирине Вильчинская}
\textbf{Iryna Busol} дякую за доброзичливу реакцію!
\end{itemize} % }

\iusr{Ігор Михайленко}
Чудово сказано!

\iusr{Юлия Чернышева}
А я помню Динамо - было счастье, когда разрешали в мороз прыгнуть с вышки  @igg{fbicon.face.upside.down} .
И вы же понимаете, что люди делятся тем, что у них в избытке- а у многих в избытке только глупость и ограниченность.

\begin{itemize} % {
\iusr{Lola Madino}
\textbf{Юлия Чернышева} і ще злоба, заздрість та чорнота...
\end{itemize} % }

\iusr{Наталия Ковалева}

Виктория, мы любим Киев и вспоминаем его! Зеленым, с чистыми умытыми улицами,
не разрисованного и обплеваного, ходили на концерты и в театры, на выставки и
музеи! На работу приезжали артисты, были клубы выходного дня и доступные
экскурсии. И все было доступно. Официальный отпуск и больничный. 

В 84 году у меня была травма. Так операция была бесплатной, еще выплачивали
больничный, с работы девочки приходили на работу, а потом домой. После выхода,
я еще пол года была на легком труде, меня не отправляли в командировки и, на
час позже шла на работу, чтобы меньше было людей в транспорте. Летом дали
отпуск. 

Затем через год курсовка в Сочи, частично оплачивал профсоюз. Был какой-то
конкурс самодеятельности. Я заняла призовое место, так мы поехали на неделю в
поезде кататься! Львов, Минск, Вильнюс, Каунас. Каждый день другой город.
Обедали и ужинали в вагоне-ресторане, а обед в том городе, где остановились,
ночь в поезде. Заплатили что-то 60 рублей. Такое сейчас даже во сне не
приснится! 

Во Львове были в Костеле Петра и Павла, старинной аптеке, музее старинной
мебели. Затем дали свободное время, купила себе осенний костюм из пальтовой
ткани в магазине от Дома моделей, в кондитерском конфеты, затем были на
Лычаковском кладбище. Прекрасные скульптуры с ангелами, склепы. 

В Минске была экскурсия по городу, прекрасный парк, проспект Машерова, ездили в
Хатынь. Очень жуткое впечатление, особенно, когда посмотришь нс таблички
сколько погибло людей! 

Вильнюс прекрасен, экскурсия пт городу, музей, костелы. Больше всего меня
поразил Каунас, небольшой город, центральная площадь, загс в маленьком костеле.
Игрушечные Андерсеновские домики. Была в музее Черлениса. Этот замечательный
художник был еще и музыкант, каждая картина расшифровывает музыкальное
произведение. В Прибалтике я увидела пиекрасные детские кафе. Мы пили ароматный
кофе в кафе \enquote{Тулпа}, тюльпан. Очередь небольшая была, но все вели себя
достойно. И на остановке, когда ехали в кафе на обед, садились по очереди,
никто не напирал.

Да, работали мы, на заводах в НИИ, иногда работы не было, но все равно
приходилось пииходить вовремя. Когда работа появлялась, приходилось оставаться,
выходить в выходные, да, иногда приходилось ездить в колхозы, но ведь и
привозили овощи и на консервацию, и на засолку. Быди дружные, общались.
Выписывали прессу. Если попадались дефицытные журналы, \enquote{Новый мир} или
\enquote{Иностранная литература}, так выписывали на отдел и читали по очереди,
были культорги и заказывали билеты в театр или на фнстивальные фильмы. Да,
может тяжело было достать хорошие сапоги или кожаную куртку, дубленку, но если
доставали, то все было отличного качества. Многие женщины сами вязали, шили,
особенно увлекались макроме. Вязали пояса.

\begin{itemize} % {
\iusr{Юлия Семеренко}
\textbf{Наталия Ковалева} 

согласна @igg{fbicon.bouquet}  @igg{fbicon.rose}{repeat=3}  @igg{fbicon.flag.triangular} давайте и дальше
жить на позитиве. Возможно, для кого-то крамольно мое выражение, но мы рождены
и воспитаны в СССР, без фанатизма.

\ifcmt
  ig https://i2.paste.pics/1f52a3c16d9d67944eb91a66097f8818.png
  @width 0.2
\fi

\end{itemize} % }

\iusr{Александр Ройтман}
\textbf{Виктория Угрюмова}

\obeycr
Якось прошло повз ! Наздогоняю.
Накипіло- це, мабуть, не тільки про ці відгуки та привласнення ярликів. Це ще й настрій такий!
Ви же така терпляча до всякого дилетантизму та відвертого дебілізму. Один я чого стою!
У російській мові є прислів'я:
Час терпеть - век жить.
Чем больше терпения, тем умнее человек.
А ще я згадаю К. Чуковського. Вислів дівчинки:
\enquote{Терпи, коза, а то мамой станешь!}.
Головные пам'ятайте- ми Вас любимо ( а дехто насіть і кохає)! Нас в рази більше!!!!!
А є ще й запитання.
Соромлюся спитати, а в чому перекладі, за винятком Л. Пастернака, Вам до вподоби Фауст фон Гете?
Дякую!
Терпіть!
а помните замечательный фильм Быкова \enquote{айболит-66}.
там Бармалей говорит такие слова: \enquote{докторишка, Ты ещё не знаешь какой я подлый}.
Я Вам ещё и о вкусах и запахах напишу)))
\restorecr

\begin{itemize} % {
\iusr{Ольга Морозова}
\textbf{Александр Ройтман} коли чоловіки не захищають жінок, то пропонують терпіти.

\begin{itemize} % {
\iusr{Александр Ройтман}
\textbf{Ольга Морозова} 

який дивний підхід до коментаря. Чоловіки - то всі чоловіки в світі? Можу
тільки припустити скільки ж страждань доставляли вам чоловіки.

Розтлумачу свою думку: чоловік пропонує жінці набратися терпіння й не реагувати
на лайки, тому що це буде тільки нервування і більш нічого. А захистити в цьому
випадку від кого, на Вашу думку, повинен чоловік? Від віртуального паґанця? А
як?

Гадаю, що це зрозуміло. Пробачте, якщо вам здається, що образив - і думки не
мав.

\iusr{Ольга Морозова}
\textbf{Александр Ройтман} 

Підтримати, висловивши свою позицію. Терпеливе мовчання - знак згоди. Свою
позицію треба захищати від віртуальних поґанців. Пробачте, якось так.

\end{itemize} % }

\iusr{Виктория Угрюмова}
\textbf{Александр Ройтман} 

це не є мій погляд як автора, це мій погляд, як модератора та людини, що знає,
скільки часу життя відбирає в колег адмінів та модераторів оці всі скарги,
коменти, скарги всередині групи, скарги та кляузи на групу. Для мене
інтелігентність та толерантність не є синонімом безпорадності. Меня здається,
якщо ми мовчимо, мовчимо, мовчимо - то це теж не є вірний підхід. Я наисала
єдину важливу річ - коли пересмикують будь-яку думку аби зробити чвару - і мені
байдуже - чи це робота людини, чи це енергетичний вампір - він посягає на час
мій та вельмишановних колег, які, на відміну від мене кожного дня працюють з
цими штучно створеними проблемами, відбираючи час та сили від власних сімей,
власної роботи, проблем та важливих, дійсно важливих справ. І, на жаль, автори
можуть ігнорувати такі коменти, а колеги модератори - вони роблять цю справу,
аби група існувала. І дуже легко пояснювати настроєм озвучування проблеми - але
це не настрій, це я проголошую факт. Голосно кажу, что в цьому місці,
створеному задля внших цілей, таким фактам не місце

\end{itemize} % }

\iusr{Владислав Поляковский}
Когда-то близ этого фонтанчика я делал утренние пробежки...

\iusr{Sergey A. Mozgovoy}

Прекрасно написано, хорошо сказано! А в \enquote{Динамо} мною много наплавано!
Спасибо Вам!

\end{itemize} % }
