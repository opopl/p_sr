% vim: keymap=russian-jcukenwin
%%beginhead 
 
%%file slova.semja
%%parent slova
 
%%url 
 
%%author 
%%author_id 
%%author_url 
 
%%tags 
%%title 
 
%%endhead 
Народ Бутана сплотился как одна \emph{семья}. Король постоянно ездил во все районы,
чтобы лично оценить ситуацию и оказать моральную поддержку и успокоить воинов
на ковидной передовой, \textbf{Ковиду не победить сплоченный и счастливый
народ. Даже если денег не так много}, Владимир Спиваковский, strana.ua,
01.06.2021

%%%cit
%%%cit_pic
\ifcmt
  pic https://strana.ua/img/forall/u/0/34/photo_2021-06-18_12-08-14(1).jpg
	caption Влад Яма рассказал \enquote{Стране}, что будет помнить о великом танцоре Григорие Чапкисе.
\fi
%%%cit_text
Григорий Чапкис родился в Румынии в 1930 году. Он рассказывал, что фамилия
\emph{семьи} была Чаопеску, но после эвакуации во время Второй мировой войны,
когда \emph{семья} осела в Киеве, они стали Чапкисами.  В столице Чапкисы осели
случайно.  Возвращались из эвакуации в Казахстане в Кишинев, но по дороге их
высадили, поскольку \emph{вся большая семья} заболела брюшным тифом. Это
произошло спустя всего два месяца после того, как из Киева выбили немцев. В
интервью Чапкис рассказывал, что видел, как вешали недобитых фашистов,
принимавших участие в убийствах. Он участвовал в восстановлении Крещатика,
который стоял в руинах.  Артист рассказывал, что танцевал с трех лет. В
\emph{семье} было одиннадцать детей и, чтобы помочь родителям прокормить всех,
он выходил на улицу, клал шапку и танцевал для прохожих. Это стало хорошей
закалкой.  Позже он связал свою жизнь с танцами. Рассказывал, что в 15 лет
танцевал гопак в кремлевском театре, когда его подхватил и посадил себе на
колени Сталин.  В 16 лет был зачислен в знаменитый ансамбль имени Вирского, где
танцевал двадцать шесть лет
%%%cit_comment
%%%cit_title
\citTitle{Чапкис умер - причина смерти, биография, жена}, 
Оксана Малахова, strana.ua, 13.06.2021
%%%endcit

