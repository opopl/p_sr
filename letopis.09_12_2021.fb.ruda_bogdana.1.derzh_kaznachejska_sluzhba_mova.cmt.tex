% vim: keymap=russian-jcukenwin
%%beginhead 
 
%%file 09_12_2021.fb.ruda_bogdana.1.derzh_kaznachejska_sluzhba_mova.cmt
%%parent 09_12_2021.fb.ruda_bogdana.1.derzh_kaznachejska_sluzhba_mova
 
%%url 
 
%%author_id 
%%date 
 
%%tags 
%%title 
 
%%endhead 
\subsubsection{Коментарі}

\begin{itemize} % {
\iusr{Вася Заровний}

В Українській мові це колись було нормально. В багатьох письменників і поетів є
така форма. Правда сучасні правила написання кажуть, що держави з чіткими
кордонами і статусом мають писатися з \enquote{в}, а з \enquote{на} - території з умовними
межами. Тут складна справа... В писанні про свою землю у нас часто вживалось
\enquote{на} (на Русі, на Україні). До того ж сама назва Україна/Вкраїна має в собі
\enquote{в}, тобто назва означає всередині своєї землі. Можливо тому раніше вживали
\enquote{на}, бо вроді як \enquote{в} двічі якось не звучить.

Тепер же дійсно оце \enquote{на} ріже слух і аж око сіпається і мимоволі стискаються
пальці щоб задушити @igg{fbicon.face.angry} 

\begin{itemize} % {
\iusr{Богдана Руда}
\textbf{Вася Заровний} 

історія історією. Так було колись, справді. Проте нині це як образа вже. Та й
стаття ж не про історію України, де б то могло бути доречним, а просто стаття
із цікавими фактами. І написана вона ось, у 2015 році, а не сотні і навіть
десятки років тому @igg{fbicon.face.smirking} 

% -------------------------------------
\ii{fbauth.teslenko_roman}
% -------------------------------------

\textbf{Vasyl Zarovnyi} 

За часів Русі мову українською ніхто не називав. І я вважаю правильним називати
мову руською. Україна - це назва територіальної одиниця Русі, як зараз
\enquote{область} чи \enquote{район}, вживати котру в якості етноніму не бачу доцільним.
Нововведення харківської інтелігенції, що почала називати народ українцями
через свої малоросійські комплекси та сепаратистські настрої, накоїло надто
багато путанини. А радянська влада цю назву закріпила. І тепер уже виглядає
так, наче Росія прямий спадкоємець Русі, а українці і не русь зовсім. Окремі
персонажі навіть русофобами вважаються. Іронія долі.

\iusr{Богдана Руда}
\textbf{Roman Teslenko} ну то я ж про ситуацію вже в теперішніх реаліях і у тому всьому що наплутали/наміняли тощо

\iusr{Roman Teslenko}
\textbf{Богдана Руда} 

та чого воно Вас так заділо? Ви що: вчителька української мови? Навіть
вчителька української мови в нас казала \enquote{да} і \enquote{а іменно}. А слухаючи як
більшість пересічних смертних в Україні розмовляє, як Ви живете, взагалі, без
депресії?

\iusr{Богдана Руда}
\textbf{Roman Teslenko} складно @igg{fbicon.face.grinning.squinting} 
Виправляючи «на Україні» в україномовних та «укрАинский» у російськомовних)

\iusr{Roman Teslenko}
\textbf{Богдана Руда} то перестаньте так серйозно запарюватися. Всіх людей не перевчите. Любіть свою сім'ю та радійте життю.


\iusr{Богдана Руда}
\textbf{Roman Teslenko} дякую)
\end{itemize} % }

\iusr{Pavlo Zubyuk}

До початку 90-х це було норма. Зараз \enquote{на Україні} нормально вживається в
російській, польській, чеській, словацькій. При чому словаки кажуть про свою
країну \enquote{на Словаччині}.

\begin{itemize} % {
\iusr{Богдана Руда}
\textbf{Pavlo Zubyuk} до початку 90х, отож
А у 2015 (стаття того року) на державному сайті - зовсім не норма, як на мене

\iusr{Pavlo Zubyuk}
\textbf{Bogdana Ruda} 

це тому, що українці люблять символи. Пам'ятаю в кінці 90-х криком волали: всі
проблеми від того, що гімн \enquote{Ще не вмерла Україна}. Типу, ще не вмерла, а от-от
вмре! Треба поміняти і все буде в шоколаді.


\iusr{Богдана Руда}
\textbf{Pavlo Zubyuk} от і все вияснили) гімн то не змінили, от і срань всяка досі відбувається @igg{fbicon.face.grinning.smiling.eyes} 

\iusr{Pavlo Zubyuk}
\textbf{Bogdana Ruda} 

змінили. Зробили \enquote{Ще не вмерла України ні слава, ні воля}. Типу,
держава не збирається помирати, от тільки зі славою і волею деякі нюанси.

\iusr{Богдана Руда}
\textbf{Pavlo Zubyuk} суть до якої мало претензію не змінили, я про це

\iusr{Pavlo Zubyuk}
\textbf{Bogdana Ruda} 

Якраз до того і мали. Суть була в тому, що не можна в гімні співати про смерть
держави, цю суть було змінено чітко згідно з вимогами ображених.

\iusr{Богдана Руда}
\textbf{Pavlo Zubyuk} аааа
Зрозуміла. Дякую за роз’яснення

\iusr{Roman Teslenko}
\textbf{Pavlo Zubyuk} 

Але гімн і справді унилий, наче здихає щось. Ті слова були актуальні в часи
свого написання в Російській Імперії, коли малороси заявляли про свою
особливість. А тепер, коли є незалежність, жалібний спів про те, що ще не
померли, не доречний. Я за радісний мотив ще й в такій важливій пісні, як
національний гімн. Але співати треба не про Україну, а про Русь - так
правильно. Землі ж не називають \enquote{область}, а називають \enquote{Галичина},
\enquote{Наддніпрянщина}, \enquote{Поділ}. Так і тут.

\end{itemize} % }

\iusr{Roman Teslenko}

Немає ніякої української мови. Це руська мова. Якщо людина хоч трохи знає
історію, вона розуміє. Щодо \enquote{на Україні}, то поясніть Тарасу Шевченку, що він
не знає українську мову. Та йому й байдуже, бо він руською писав, коли про
українську мову ще й не чули.

\begin{itemize} % {
\iusr{Pavlo Zubyuk}
\textbf{Roman Teslenko} в Русі не було єдиної мови, а тим більше нею не був полтавський діалект 19-го століття.

\iusr{Roman Teslenko}
\textbf{Pavlo Zubyuk} так, вірно. Русь надто велику територію займала. Зате мова сторіччями називалася руською.

\iusr{Pavlo Zubyuk}
\textbf{Roman Teslenko} 

сучасна літературна українська мова так не називалася. А \enquote{руською}
називалися всі мови і діалекти слов'ян Русі: і у Новгороді, і у Полоцьку, і у
Галичі.

\iusr{Roman Teslenko}
\textbf{Pavlo Zubyuk} 

сучасна так. А мова, якою розмовляли та писали ще сто років тому на території
сучасної України, називалася руською. І лист турецькому султану козаки писали
не українською, а руською мовою. Навіщо тисячорічну назву знадобилося
перейменовувати в українську через придумки окремих горе-від-ума, не зрозуміло.

\end{itemize} % }

\end{itemize} % }
