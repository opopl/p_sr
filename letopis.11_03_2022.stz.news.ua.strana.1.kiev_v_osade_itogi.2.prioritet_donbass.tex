% vim: keymap=russian-jcukenwin
%%beginhead 
 
%%file 11_03_2022.stz.news.ua.strana.1.kiev_v_osade_itogi.2.prioritet_donbass
%%parent 11_03_2022.stz.news.ua.strana.1.kiev_v_osade_itogi
 
%%url 
 
%%author_id 
%%date 
 
%%tags 
%%title 
 
%%endhead 

\subsubsection{Приоритет – Донбасс?}

Что касается оперативных планов, то, судя по информации из различных
источников, россияне на данный момент сосредоточены, в первую очередь, на
Донбассе.

Российский Минобороны заявил, что \href{https://strana.news/news/381243-volnovakha-minoborony-rossii-zajavilo-chto-vojska-dnr-polnostju-kontrolirujut-horod.html}{\enquote{ДНР} полностью взяли под контроль Волноваху},
однако украинский Генштаб этого не подтверждает, хотя и не опровергает. По
некоторым данным часть украинских войск из города вышло в направлении
Запорожской области.

В свою очередь, военное руководство Украины сообщает о продолжающихся жестоких
боях за Изюм (россияне пытаются занять его южную часть), Северодонецк и
Мариуполь. Ни один из этих городов полностью взять еще не удалось, однако
интенсивность боевых действий в этих точках является доказательством того, что
попытка отрезать подразделения зоны ООС от остальной страны остается
приоритетной задачей для российских войск и \enquote{ЛДНР}.

Однако, как мы уже много раз писали, тут главный вопрос – это наличие резервов
у российских войск для такой масштабной операции. Без них, максимум, что смогут
сделать армия РФ и \enquote{ЛДНР} - взять Мариуполь и еще несколько городов около
которых уже идут бои (Изюм и Северодонецк).  

О том, что вопросом резервов российские власти активно занимаются говорит
сегодняшние заявления Путина и Шойгу о том, что могут быть привлечены для
участия в войне в Украине 16 тысяч сирийских \enquote{добровольцев}. Также
сегодня вновь активизировались слухи, что в ближайшее время на стороне России
\href{https://strana.news/news/381326-belorusskaja-armija-mozhet-vstupit-v-vojnu-v-ukraine-11-marta-v-2100.html}{вступит
в войну Беларусь} (хотя ранее Лукашенко такую возможность опровергал).

Но так или иначе, резервы Россия накапливает, а значит готовится к новым
наступлениям. И не только на Донбассе.
