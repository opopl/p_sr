% vim: keymap=russian-jcukenwin
%%beginhead 
 
%%file 10_02_2022.stz.news.lnr.lug_info.1.sojuzy_hudozhnikov_dnr_lnr
%%parent 10_02_2022
 
%%url https://lug-info.com/news/souzy-hudoznikov-lnr-i-dnr-zaklucili-dogovor-o-sotrudnicestve
 
%%author_id news.lnr.lug_info
%%date 
 
%%tags dnr,dogovor,donbass,isskustvo,kultura,lnr,sotrudnichestvo,zhizn
%%title Союзы художников ЛНР и ДНР заключили договор о сотрудничестве
 
%%endhead 
 
\subsection{Союзы художников ЛНР и ДНР заключили договор о сотрудничестве}
\label{sec:10_02_2022.stz.news.lnr.lug_info.1.sojuzy_hudozhnikov_dnr_lnr}
 
\Purl{https://lug-info.com/news/souzy-hudoznikov-lnr-i-dnr-zaklucili-dogovor-o-sotrudnicestve}
\ifcmt
 author_begin
   author_id news.lnr.lug_info
 author_end
\fi

Общественная организация \enquote{Союз художников ЛНР} и творческий союз \enquote{Союз
художников ДНР} заключили договор о сотрудничестве. Об этом сообщила
пресс-служба Министерства культуры, спорта и молодежи ЛНР со ссылкой на
председателя правления Союза художников ЛНР Артема Фесенко.

\ii{10_02_2022.stz.news.lnr.lug_info.1.sojuzy_hudozhnikov_dnr_lnr.pic.1}

Церемония подписания договора состоялась 9 февраля в Донецком республиканском
художественном музее. В мероприятии приняли участие Фесенко, председатель Союза
художников (СХ) ДНР Мелик Агабекян, члены СХ ЛНР и ДНР, представители
Министерства культуры ДНР, сотрудники Донецкого республиканского
художественного музея.

\ii{10_02_2022.stz.news.lnr.lug_info.1.sojuzy_hudozhnikov_dnr_lnr.pic.2}

\enquote{Договор мы заключили с целью более масштабного вовлечения наших организаций в
культурную жизнь общества. Объединив силы, мы получим гораздо больше
возможностей для реализации разнообразных проектов. Этот путь тернист, но он
плодотворный и, пожалуй, единственно возможный}, – сказал Фесенко.

Он отметил, что творческие организации уже запланировали ряд совместных
проектов.

\enquote{В первую очередь, это издание книги \enquote{Художники Донбасса},
посвященной 65-летию Союза художников Луганска, правопреемником которого
является Союз художников ЛНР. Также мы пригласили на празднование юбилея
художников из Донецка, планируем провести масштабную выставку. Обсуждали и
возможность проведения весной совместных пленэров, будем вместе писать этюды на
природе, это хорошая практика для совершенствования навыков мастеров}, –
рассказал руководитель Союза художников ЛНР.

В завершение церемонии гости из Луганска получили подарочные каталоги с
произведениями донецких живописцев.
