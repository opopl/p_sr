% vim: keymap=russian-jcukenwin
%%beginhead 
 
%%file 20_05_2021.fb.promovugroup.1.jazyk_mova
%%parent 20_05_2021
 
%%url https://www.facebook.com/groups/promovugroup/permalink/974267699813758/
 
%%author 
%%author_id 
%%author_url 
 
%%tags 
%%title 
 
%%endhead 
\subsection{Російська мова все більше втрачає статус так званої \enquote{мови міжнаціонального спілкування}}
\label{sec:20_05_2021.fb.promovugroup.1.jazyk_mova}
\Purl{https://www.facebook.com/groups/promovugroup/permalink/974267699813758/}

\textbf{Марусик, Тарас}

Російська мова все більше втрачає статус так званої \enquote{мови міжнаціонального
спілкування}. Об'єктивно. Але ця істина - не для президента Зеленського, не для
його важливого секретаря з титулом \enquote{глава Офісу Президента}, не для міністра
культури Ткаченка і ще для численних інших (хотів написати, рушіїв життя
України, але дисонанс) \enquote{рєшал} нашого буття.

А от для Прем'єр-міністра Естонської Республіки Каї Каллас цілком ясно, що
порозумітися з Україною можна і треба українською й англійською. Практика, коли
іноземний урядовець починає свій виступ українською мовою (хай навіть слова
привітання) стає звичною. І немає значення, що для естонки українська
надзвичайно важка мова - як і для українців естонська. Але це - необхідний
елемент поваги й ознака тектонічних геополітичних змін, на початковій чи
середній стадії яких ми перебуваємо.

(З 9:40)

\ifcmt
  pic https://external-iad3-1.xx.fbcdn.net/safe_image.php?d=AQGvzcnQ09JttbOr&w=500&h=261&url=https%3A%2F%2Fi.ytimg.com%2Fvi%2FZz7SYG7DSDM%2Fmaxresdefault_live.jpg&cfs=1&ext=jpg&ccb=3-5&_nc_hash=AQE1cfZ3WXb-QSHw
\fi

\href{https://www.youtube.com/watch?v=Zz7SYG7DSDM}{youtube - 17.05.2021 Онлайн-брифінг Дениса Шмигаля і Каї Каллас}

