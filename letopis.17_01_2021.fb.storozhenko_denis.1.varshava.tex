% vim: keymap=russian-jcukenwin
%%beginhead 
 
%%file 17_01_2021.fb.storozhenko_denis.1.varshava
%%parent 17_01_2021
 
%%url https://www.facebook.com/denis.storozhenko.332/posts/405556490712762
 
%%author 
%%author_id 
%%author_url 
 
%%tags 
%%title 
 
%%endhead 
\subsection{76 лет назад, 17 января 1945 года - освобождение Варшавы}

76 лет назад, 17 января 1945 года, советские войска освободили важнейший
стратегический узел обороны немцев на реке Висла, столицу Польши город \verb|#Варшава| 

В ходе начавшейся 14 января 1945 года Варшавско-Познанской операции (составной
части стратегической Висло-Одерской операции) 1-го Белорусского фронта начала
наступать 61-я армия генерал-полковника П.А. Белова. Она наносила удар южнее
Варшавы. На следующий день, охватывая город с севера, пошла в наступление 47-я
армия генерал-майора Ф.И. Перхоровича. За день она продвинулась на глубину до
12 км и вышла к р. Висла. В 8 часов утра 16 января с плацдарма на левом берегу
р. Пилица в прорыв была введена 2-я гвардейская танковая армия
генерал-полковника С.И. Богданова, которая начала развивать наступление в
направлении Сохачева, преследуя разгромленные в предыдущих боях части
противника и охватывая правый фланг 46-го танкового корпуса фашистов. Вражеское
командование, опасаясь окружения своих войск в районе Варшавы, начало поспешно
отводить их в северо-западном направлении. 

В сложившейся обстановке 61-я армия увеличила темпы преследования отходящего
противника до 16-24 км в сутки, а 1-я армия Войска Польского под командованием
генерала дивизии С.Г. Поплавского форсировала Вислу и начала наступать на
Варшаву. 

Командующий польской армией Поплавский вспоминал: "Издали донёсся рокот
моторов. Затем в полумраке обрисовались контуры машин. Обыкновенные советские
танки Т-34, каких много уже прошло через этот мост. Только на броне их
красовался белый польский пястовский орел. 

Из люков выглядывали танкисты. Для такого торжественного случая они надели вместо кожаных шлемов конфедератки. 

"Да здравствуют польские танкисты!", "Да здравствует народная Польша!" - звучало на русском языке. 

"Нех жие братерство брони!", "Нех жие незвыценжона Армия Радценска!" - неслось
в ответ по-польски. Переправа танков по мосту прошла благополучно". 

Утром 16 января действующая на правом фланге 1-й армии Войска Польского 2-я
пехотная дивизия, используя успех 47-й армии, начала переправу через Вислу в
районе Кэмпы Келпинской и захватила плацдарм на западном берегу. Её командир,
Ян Роткевич, быстро перебросил на западный берег главные силы дивизии. 

На левом фланге армии активные действия начались во второй половине дня 16
января с разведки боем, которую провела кавалерийская бригада (из-за отсутствия
лошадей солдаты бригады действовали как обычные пехотинцы). 

Разведывательным группам 2-го и 3-го уланских полков удалось зацепиться за
противоположный берег и, тесня немцев, захватить плацдарм. Командир
кавалерийской бригады полковник Владзимеж Радзиванович сразу же переправил туда
свои главные силы. Действуя энергично и напористо, кавалерийская бригада к
концу дня освободила пригородные посёлки Оборки, Опач, Пяски, что позволило
польской 4-й пехотной дивизии выдвинуться на исходные позиции в районе Гура
Кальвария. 

В центре оперативного построения польской армии на столицу наступала 6-я
пехотная дивизия Войска Польского. Здесь гитлеровцы сопротивлялись особенно
упорно. Первую попытку форсировать Вислу по льду полковник Г. Шейпак предпринял
ещё днём 16 января. Противник встретил наступавших сильным артиллерийским
огнём. Снаряды и мины рвались, образуя большие полыньи и преграждая солдатам
путь. Но едва те залегали, как на них обрушивался шквал пулемётного огня.
Пришлось приостановить наступление и возобновить его лишь в темноте. 

Наступление 47-й и 61-й советских армий развивалось весьма успешно. Были
освобождены Гура Кальвария и Пясечно. Население Пясечно от мала до велика
высыпало на улицы, ликующими возгласами встречая советские и польские части.
Стремительно шли вперёд главные силы 2-й гвардейской танковой армии.
Требовалось ускорить продвижение передовых частей 1-й армии Войска Польского. 

В Пясечно прошёл летучий митинг. Вот как вспоминает об этом С. Поплавский: 

"Через город проходил один из полков 3-й пехотной дивизии - два остальных полка
уже сражались в предполье Варшавы. На площади остановились три танка с группой
десантников-автоматчиков на броне. Когда мы с Ярошевичем подошли к ним, то
увидели офицера, которого окружили жители окрестных улиц. 

- Пане, скажите, откуда и каким чудом взялись польские воины? - спросил его
старичок с бородкой клинышком, в пенсне. 

- На танках белый орёл… Неужели они польские? - Худая как скелет женщина, не
отрываясь, смотрела большими увлажненными глазами на эмблему, украшавшую броню. 

- Немцы днём и ночью кричали по радио, что польской армии вовсе нет, а советским войскам никогда не взять Варшавы, - добавил парнишка лет пятнадцати с рукой на грязной перевязи. 

Офицер терпеливо отвечал на вопросы, поясняя, что и грозные боевые машины с
белым орлом на броне, и русоволосые парни в танкошлемах, и автоматчики в касках
- всё это частица новой народной армии - Войска Польского, пришедшего вызволить
родную землю из-под фашистского ига". 


\ifcmt
tab_begin cols=3
  pic https://scontent-mad1-1.xx.fbcdn.net/v/t1.0-9/139460767_405556287379449_6592634878807073292_n.jpg?_nc_cat=110&ccb=2&_nc_sid=8bfeb9&_nc_ohc=3OW6GSakgzQAX-UeK1a&_nc_ht=scontent-mad1-1.xx&oh=cc15f0f2b3a1af05bbd48ecccba3bbbb&oe=60284E99

	pic https://scontent-mad1-1.xx.fbcdn.net/v/t1.0-9/139807516_405556317379446_2379805867104412895_n.jpg?_nc_cat=110&ccb=2&_nc_sid=8bfeb9&_nc_ohc=NMR73dWQqXIAX8-DIRp&_nc_ht=scontent-mad1-1.xx&oh=17e61652471b0dca18cd3b7075a8131b&oe=602AF3BF

	pic https://scontent-mad1-1.xx.fbcdn.net/v/t1.0-9/139840940_405556350712776_8408381530722736988_n.jpg?_nc_cat=109&ccb=2&_nc_sid=8bfeb9&_nc_ohc=kY7kKJdXCPAAX-Ddomw&_nc_ht=scontent-mad1-1.xx&oh=650f41b522366d4966b6a438fa8d78ef&oe=6029A4DA

	pic https://scontent-mad1-1.xx.fbcdn.net/v/t1.0-9/140251708_405556377379440_6049832902831611975_n.jpg?_nc_cat=105&ccb=2&_nc_sid=8bfeb9&_nc_ohc=ELtiVyXLKiEAX_Be6oS&_nc_ht=scontent-mad1-1.xx&oh=6eba77d7adee8541bdf25d0b7fbeea40&oe=602B04B7

	pic https://scontent-mad1-1.xx.fbcdn.net/v/t1.0-9/140015608_405556397379438_890892044931415054_n.jpg?_nc_cat=108&ccb=2&_nc_sid=8bfeb9&_nc_ohc=DNjE6XQDwacAX89u0lJ&_nc_ht=scontent-mad1-1.xx&oh=43be39bb89a0bfeb2bf5297064229421&oe=60281684

	pic https://scontent-mad1-1.xx.fbcdn.net/v/t1.0-9/140046694_405556420712769_673561425323559502_n.jpg?_nc_cat=103&ccb=2&_nc_sid=8bfeb9&_nc_ohc=Hq5ApvYIgpUAX-J5p8-&_nc_ht=scontent-mad1-1.xx&oh=9f7d124691dac6a11df0afc9572d8413&oe=6029309F

	pic https://scontent-mad1-1.xx.fbcdn.net/v/t1.0-9/139751812_405556454046099_4551762994792504338_n.jpg?_nc_cat=110&ccb=2&_nc_sid=8bfeb9&_nc_ohc=l5ytIi8Xb_oAX_zqonu&_nc_ht=scontent-mad1-1.xx&oh=1af61750abea419e940948e09da5bf64&oe=6028C342
tab_end
\fi


В 8 часов утра 17 января 4-й пехотный полк 2-й дивизии Яна Роткевича первым
ворвался на улицы Варшавы. Уже через 2 часа он продвинулся до самой большой и
популярной варшавской улицы - Маршалковской. Тяжелее пришлось 6-му пехотному
полку, наступавшему на левом фланге дивизии: на площади Инвалидов он встретил
яростное сопротивление гитлеровцев, засевших в старой цитадели. Только
благодаря героизму солдат и офицеров удалось овладеть этим важным опорным
пунктом. Затем 6-й полк продвинулся к площади Тжеха Кжижи. Впереди наступал
батальон под командованием советского офицера Александра Афанасьева. В ходе
ожесточённой схватки удалось уничтожить целое подразделение противника,
засевшее в развалинах углового здания, захватив при этом исправные орудие,
пулемёты и боеприпасы. Взаимодействуя, полки 6-й и 2-й дивизий разгромили
противника в Саксонском парке, а один из батальонов 16-го пехотного полка
неудержимой атакой выбил фашистов с Дворцовой площади. 

Очень тяжёлыми были бои за важный опорный пункт - Главный вокзал. Враг цеплялся
за каждое крыло здания, за каждый угол. Стрельба в этой части города постепенно
затихала - противник отступал. Но группы немецких снайперов и автоматчиков ещё
вели огонь из полуразрушенных зданий, из развалин и баррикад. 

В это время 1-я кавбригада через Повсин и Служивец уже ворвалась в городской
район Мокотув, 1-я пехотная дивизия, наступавшая через Грабице и Чарны Ляс,
вышла в район Окенце, а 4-я дивизия, обогнув город с юга, заняла Кренчки,
Петрувек. 

Сражение за столицу Польши близилось к концу. Обойденная с двух сторон
советскими войсками, сомкнувшими кольцо окружения в Сохачеве, расчленённая
затем ударами польских частей, нацистская группировка в Варшаве терпела
поражение в уличных боях. Многие гитлеровцы, видя безнадёжность сопротивления,
бежали из города, другие продолжали драться с отчаянием обречённых, некоторые
сдавались в плен. 

В 3 часа дня Варшава была освобождена. 

Вслед за 1-й армией Войска Польского в Варшаву вошли части 47-й и 61-й армий
советских войск.  "Фашистские варвары уничтожили столицу Польши - Варшаву", -
докладывал военный совет фронта Верховному Главнокомандующему. 

Г.К. Жуков вспоминал: "С ожесточённостью изощрённых садистов гитлеровцы
разрушали квартал за кварталом. Крупнейшие промышленные предприятия стёрты с
лица земли. Жилые дома взорваны или сожжены. Городское хозяйство разрушено.
Десятки тысяч жителей уничтожены, остальные были изгнаны. Город мёртв. Слушая
рассказы жителей Варшавы о зверствах, которые творили немецкие фашисты во время
оккупации и особенно перед отступлением, трудно было даже понять психологию и
моральный облик вражеских войск". 

Начальник штаба 1-го Белорусского фронта генерал-полковник М.С. Малинин доложил
начальнику Генштаба генералу армии А.И. Антонову, что противник оставил Варшаву
заминированной. "В ходе разминирования было снято, собрано и подорвано 5412
противотанковых мин, 17 227 противопехотных, 46 фугасов, 232 "сюрприза", свыше
14 тонн взрывчатых веществ, около 14 тысяч снарядов, авиабомб, мин и гранат". 

Освобождение Варшавы позволило Красной Армии существенно продвинуться к границе
Германии и сыграло важную роль в послевоенных отношениях СССР с Польшей. 

Советские войска разгромили главные силы 9-й армии противника, осуществили
прорыв её обороны на всю оперативную глубину, продвинувшись на 100-130 км. 

За освобождение Варшавы была учреждена награда - медаль "За освобождение
Варшавы". Ею награждались военнослужащие Красной армии, Военно-Морского Флота и
войск НКВД - непосредственные участники боёв 14-17 января 1945 года, а также
организаторы и руководители боевых операций при освобождении этого города.
Медаль "За освобождение Варшавы" получили более 690 тысяч человек. 

Войскам, участвовавшим в боях за освобождение Варшавы, приказом Верховного
Главнокомандования от 17 января 1945 года была объявлена благодарность и в
Москве дан салют 24 артиллерийскими залпами из 324 орудий.
