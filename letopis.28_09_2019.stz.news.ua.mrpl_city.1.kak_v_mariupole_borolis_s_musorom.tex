% vim: keymap=russian-jcukenwin
%%beginhead 
 
%%file 28_09_2019.stz.news.ua.mrpl_city.1.kak_v_mariupole_borolis_s_musorom
%%parent 28_09_2019
 
%%url https://mrpl.city/blogs/view/kak-v-mariupole-borolis-s-musorom-v-proshlom
 
%%author_id burov_sergij.mariupol,news.ua.mrpl_city
%%date 
 
%%tags 
%%title Как в Мариуполе боролись с мусором в прошлом
 
%%endhead 
 
\subsection{Как в Мариуполе боролись с мусором в прошлом}
\label{sec:28_09_2019.stz.news.ua.mrpl_city.1.kak_v_mariupole_borolis_s_musorom}
 
\Purl{https://mrpl.city/blogs/view/kak-v-mariupole-borolis-s-musorom-v-proshlom}
\ifcmt
 author_begin
   author_id burov_sergij.mariupol,news.ua.mrpl_city
 author_end
\fi

Караул! Мир задыхается от мусора. Пластиковые \enquote{поля} покрывают все большие
площади мирового океана. Погибает все живое. Кто виновник? Говорят,
полиэтиленовая упаковка. На поверхности Земли все меньше свободных территорий
для свалок. Проблема мусора во Львове была предметом телевизионных передач и
острых газетных публикаций с поиском виновных. \enquote{Диверсиями} был вывоз тайком
львовского мусора в соседние области. Спасайтесь, кто может! Но оставим
обсуждение глобальных проблем экологам, журналистам и активистам.

\ii{28_09_2019.stz.news.ua.mrpl_city.1.kak_v_mariupole_borolis_s_musorom.pic.1.dvor_na_torgovoj}

\textbf{Читайте также:} 

\href{https://mrpl.city/blogs/view/kak-v-mariupole-borolis-s-musorom-v-proshlom}{%
Історії zerowaste XX століття, Маріанна Бойко, mrpl.city, 05.02.2019}

Когда дряхлые старики говорят, что мариупольские улицы в 40-50-е годы были
чище, молодые только посмеиваются. Надеясь, что глуховатые дедушки и бабушки их
не слышат, в полголоса обмениваются репликами: \enquote{и вода была мокрее}, \enquote{и конфеты
были слаще}. Сразу нужно отметить, что до середины 50-х лишь относительно
небольшая часть домов имела центральное отопление и газовые печки. Большинство
же жилищ в плане отопления оставалось на уровне XIX века. Для примера возьмем
отдельно взятый двор на Торговой улице. Где дома лепились друг к дружке. Где
жилища обогревались углем при помощи плиты с грубкой. А летом на печке,
сооруженной почти посреди двора, варили, жарили, пекли, делали заготовки на
зиму: домашние овощные консервы и томатную заправку для борщей. И в зимних, и в
летних очагах сгорало все, что могло гореть. Донельзя изношенные, много раз
чиненные-перечиненные детские сандалии, тряпицы, из которых уже ничего нельзя
было сшить, шелуха семечек, стебли засохших растений. Делалось это не столько
для того, чтобы сэкономить дрова и уголь, сколько чтобы избавиться от мусора. А
дрова и уголь все равно оставались основным топливом. Извлеченную из остывшей
печи золу тщательно перебирали, стараясь извлечь несгоревшие кусочки угля.

Некоторые рачительные хозяйки держали кур или гусей. Птицы эти склевывали не
только приобретаемый для них корм, но и очистки овощей, в бахчевой сезон
измельченные корки арбузов и дынь, отходы от приготовления томатных морсов и
тому подобное. Изношенная одежда тоже шла в дело. Из нее выкраивали заготовки
торбочек, мешочков для круп, сахара, муки и других, говоря казенным языком,
сыпучих продуктов. Полиэтиленовые пакеты появились в наших краях едва ли не в
начале 70-х годов. В наши дни, если и забудешь пакет – не беда. В кассе
супермаркета обязательно предложат пакет за смешную цену. Да и на рынке без
упаковки не останешься. А во время, отдаленное от эпохи пластиковых пакетов и
бутылок, в быту широко использовались авоськи – сетчатые сумочки. В юные годы
это слово воспринималось как имя собственное. Дитя тотального дефицита, это
нехитрое изделие было в кармане каждого маломальски хозяйственного человека.
При возвращении с работы в подсознании таких людей вертелось – \enquote{\em авось
какой-нибудь дефицит попадется}. Правда, когда попадался, а когда и нет. Да,
пожалуй, что чаще – нет.

\textbf{Читайте также:} 

\href{https://mrpl.city/news/view/mariupolskij-kommunalnik-izmenit-nomera-bankovskih-schetov}{%
Мариупольский \enquote{Коммунальник} изменит номера банковских счетов, Богдан Коваленко, mrpl.city, 09.08.2019}

В нынешних продовольственных магазинах - от самых крошечных до супермаркетов -
редко встретишь нефасованные сахар, соль, крупы и т.п. Помнится, что в годы
50-е и даже 80-е все было наоборот. Фасованного товара было мало. Разве что
соль, сода да маргарин. Все остальное - в ящиках насыпью. В качестве упаковки
использовали частенько плотную бумагу, чуть ли не тонкий картон. Это был
неучтенный доход магазину. Конечно, дальновидные покупательницы приносили свои
тряпичные сумочки для макарон или там гороха, например. Для молока,
подсолнечного масла или других жидкостей почти в каждой семье были
эмалированные или алюминиевые бидончики разной емкости. Когда в городских
магазинах появились пол-литровые бутылки с широким горлышком с молоком, кефиром
и ряженкой, изготовленные Мариупольским молокозаводом, настала \enquote{новая эра} в
обслуживании покупателей. Идешь в магазин, сдаешь тщательно вымытую бутылку, а
взамен получаешь наполненную молочным продуктом с зеленой, золотистой или
малиновой крышечкой. За тару не платишь. Тару от горячительных напитков
принимали, насколько помнится, в специальных пунктах.

Вот и получается, что раньше бытовых отходов было меньше по причинам, изложенным выше.

\textbf{Читайте также:} 

\href{https://mrpl.city/blogs/view/motoroshne-vidovishhe-podorozh-do-smittezvalishha}{%
Моторошне видовище. Подорож до сміттєзвалища, Маріанна Бойко, mrpl.city, 12.10.2018}

%Джерело: \url{https://mrpl.city/}
