% vim: keymap=russian-jcukenwin
%%beginhead 
 
%%file 06_07_2020.fb.fb_group.story_kiev_ua.1.fotografia_vospominania
%%parent 06_07_2020
 
%%url https://www.facebook.com/groups/story.kiev.ua/posts/1395646023965528
 
%%author_id fb_group.story_kiev_ua,petrova_irina.kiev
%%date 
 
%%tags fotografia,kiev,pamjat
%%title Иногда волну воспоминаний рождает одна всего лишь фотография...
 
%%endhead 
 
\subsection{Иногда волну воспоминаний рождает одна всего лишь фотография...}
\label{sec:06_07_2020.fb.fb_group.story_kiev_ua.1.fotografia_vospominania}
 
\Purl{https://www.facebook.com/groups/story.kiev.ua/posts/1395646023965528}
\ifcmt
 author_begin
   author_id fb_group.story_kiev_ua,petrova_irina.kiev
 author_end
\fi

Иногда волну воспоминаний рождает одна всего лишь фотография...

Я не могу понять, отчего мы испытывали такую "личную неприязнь" (c)...

Я уже рассказывала, что наш садик располагался на площади Франко. На площадку у
памятника нас водили гулять наши любимые воспитательницы Ольга Ульяновна, Майя
Антоновна и Клавдия Викторовна. Приходили, приносили огромный полотняный мешок
игрушек. Становились в хоровод и репетировали песенки:

"Солнышко  на дворе,

а в саду тропинка.

 Сладкая ты моя,

ягодка-малинка!"

"Пусть всегда будет солнце,

 Пусть всегда будет небо,

 Пусть всегда будет мама,

 Пусть всегда буду я!"

После совместного исполнения обязательного репертуара, все расходились по своим
делам. Мальчишки, отпросившись, якобы "по делам, за лесенку",  у подпорной
стены искали червяков или жуков для подкладывания нам на стул в группе или в
суп. Украдкой жевали за обе щеки запретные плоды груши-дички, твердые, как
гранит и кислющие, как уксус!

\begin{multicols}{2}
\ii{06_07_2020.fb.fb_group.story_kiev_ua.1.fotografia_vospominania.pic.1}
\ii{06_07_2020.fb.fb_group.story_kiev_ua.1.fotografia_vospominania.pic.2}
\ii{06_07_2020.fb.fb_group.story_kiev_ua.1.fotografia_vospominania.pic.3}
\ii{06_07_2020.fb.fb_group.story_kiev_ua.1.fotografia_vospominania.pic.4}
\ii{06_07_2020.fb.fb_group.story_kiev_ua.1.fotografia_vospominania.pic.5}
\ii{06_07_2020.fb.fb_group.story_kiev_ua.1.fotografia_vospominania.pic.6}
\ii{06_07_2020.fb.fb_group.story_kiev_ua.1.fotografia_vospominania.pic.7}
\ii{06_07_2020.fb.fb_group.story_kiev_ua.1.fotografia_vospominania.pic.8}
\ii{06_07_2020.fb.fb_group.story_kiev_ua.1.fotografia_vospominania.pic.9}
\end{multicols}

Малышня копалась со своими пасочками и совочками в песочнице. А мы ! С
девочками мы рассаживались в качели-лодочки, раскачивали их до невероятно
опасной амплитуды и во все разогретые горла орали, ой, сорри ПЕЛИ :

"Дунай, Дунай,

 А ну, узнай,

 Где чей подарок!

 К цветку цветок

 Сплетай венок,

 Пусть будет красив он и ярок."

"Городок наш ничего,

 Населенье таково -

 Незамужние ткачихи

 Составляют большинство"

"Ландыши, ландыши

 Светлого мая привет

 Ландыши, ландыши

 Белый букет."

В умопомрачительных бусах из веток плакучей ивы и с потрясающе красивыми
ногтями из цветов сальвии мы ощущали себя не звёздами, нееет... Мы были
ЗВЕЗДИЩАМИ !!!

Как мы не переворачивались в этих "лодочках" - непонятно, как выдерживали уши
окружающих - еще непонятнее. Мы знали наизусть ВСЕ песни тогдашней эстрады!

Иногда наш творческий полёт обрывала суровая проза жизни ! На площадку
привозили песок для песочницы. Прямо на площадку машина заехать не могла,
сбрасывали около заборчика, и все мы - актрисы, певицы, копатели червей и
прочая мелюзга в детских ведерочках муравьиной шеренгой переносили его в
песочницу. Воспитательницы носили большими ведрами. Как он пах! Рекой, влагой,
вкууусно!

К сожалению,  фотографий этого процесса не сохранилось. Гаджеты еще не
родились, а папы с фотоаппаратами были на работе.

Песок перенесен, мы возвращались к творчеству. Мы были Майей Кристалинской,
Эдитой Пьехой и Муслимом Магомаевым вместе взятыми.

НО! Вдруг случалось нечто ужасное! К нашей площадке приближались ОНИ, ЧУЖИЕ!!!

В обычной жизни это была группка детишек нашего возраста, с которыми гуляла
бонна (их еще называли фребелички). Человек десять. Почему мы испытывали к ним
такую "личную неприязнь"? Может, обществом пролетарских деток мы представляли
их "аристократами",  "эй, интеллигент, шляпу надел!". Хотя такого не должно
было быть, наши родители были людьми воспитанными и такие мысли в наши
непутевые бошки заронить не могли. То ли понимание их отличия от нас...не знаю,
объяснить не могу.

Они медленно приближались к забору, беседуя на французском. Мы опрометью
кидались к калитке, становились, что те скалы из песни Муслима, и во все наши
разогретые глотки орали, что было мочи : "НЕЛЬЗЯ!!! У НАС КАРАНТИН!!!"
Сраженный неприятель отступал к площадке у фонтана, мы победно вопили им вслед,
потом продолжали вокал.

Был момент, когда я некоторое время была непринимаема в компанию звезд из-за
совершенно несценической внешности! Тогда я совсем не понимала, почему вдруг
каждый вечер бабуня чешет меня отвратительным, очень частым гребешком над
листом белой бумаги, а потом меня из "баядерочки" простригут в "под
котовского". Очевидно, слово "вши" старались не упоминать при живом,
сообразительном ребенке.

да...Потом судьба отомстит мне по-взрослому...

Один мальчик из той группы в школе стал моей совершенно безответной, горькой,
трагической, в своей безысходности, любовью. Как могла подушка вмещать столько
литров ночных слёз... Другой не захотел встать со мной в пару на фигурном
катании... бумеранг ВСЕГДА возвращается... а вот сейчас КАРАНТИН - это вовсе не
выдумка глупых жадных ребятишек, а суровая действительность

"Через годы, через расстояния" догнала судьба...

Если бы!! Если бы мы тогда их пустили, подружились еще тогда, то...но, мадам
история не любит сослагательного наклонения...

Вот такие воспоминания навеяла мне фотография, присланная одним из тех детишек,
воспитанников бонны. Она сделана зимой, автор фото, к сожалению, неизвестен.
Фото предоставил Виктор Бухтияров (стоит второй ряд слева, с деревянным
мечом)."

Моя горькая любовь сидит на санках с пистолетом, уже тогда нацеленным в моё
сердце.

Остальные фотографии из нашего семейного архива. Они сделаны тоже не только
летом, хочется показать наш уютный сквер, наши качели.

И еще - с девочкой в белой шубке мы стоим под окнами нашего садика. Может,
кто-то, совершенно случайно, был или знает выпускников садика 139. Не могу
никого найти...
