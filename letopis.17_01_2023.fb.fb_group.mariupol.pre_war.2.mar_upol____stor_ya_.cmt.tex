% vim: keymap=russian-jcukenwin
%%beginhead 
 
%%file 17_01_2023.fb.fb_group.mariupol.pre_war.2.mar_upol____stor_ya_.cmt
%%parent 17_01_2023.fb.fb_group.mariupol.pre_war.2.mar_upol____stor_ya_
 
%%url 
 
%%author_id 
%%date 
 
%%tags 
%%title 
 
%%endhead 
\qqSecCmt

\begin{itemize} % {
\iusr{Олеся Зыкова}

Какой красивый и счастливый был наш Мариуполь

\begin{itemize} % {
\iusr{Оксана Аксютина}
\textbf{Олеся Зыкова} да.. мы этого порой не замечали... теперь осталась только Память..
\end{itemize} % }

\iusr{Іванова Яна}

Щодо фонтану на МДУ - існувала така прикмета: якщо з дитиною у возику обійти
фонтан, малеча в майбутньому стане студентом вишу))) Це мені розповідав один
місцевий історик-краєзнавець)

Які прикольні іноді бували маріупольці у своїх вигадках..

\begin{itemize} % {
\iusr{Елена Мариупольская}
\textbf{Іванова Яна} не знала! Але збулося...

\iusr{Іванова Яна}
\textbf{Елена Сугак} працювала прикмета ))))

\iusr{Елена Мариупольская}
\textbf{Іванова Яна} ага! Доцю маленьку завжди там вигулювала... вступила і закінчила...

\iusr{Lesya Vlasova}
\textbf{Іванова Яна} 

не збулося. Навіть поступати до вишу не хоче ((

Одного разу зі мною іспит з грецької мови здавав. Йому тоді було 2 місяця 😊 А
в 2022 склав іспит за брата в ПДТУ. На дістанційному навчанні мабуть викладачі
облич не запам'ятовують ☺️

\end{itemize} % }

\end{itemize} % }
