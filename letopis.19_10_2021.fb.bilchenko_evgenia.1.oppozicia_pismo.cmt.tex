% vim: keymap=russian-jcukenwin
%%beginhead 
 
%%file 19_10_2021.fb.bilchenko_evgenia.1.oppozicia_pismo.cmt
%%parent 19_10_2021.fb.bilchenko_evgenia.1.oppozicia_pismo
 
%%url 
 
%%author_id 
%%date 
 
%%tags 
%%title 
 
%%endhead 
\subsubsection{Коментарі}

\begin{itemize} % {
\iusr{Александр Непомнящих}
За смелость - моë уважение, Евгения.

\begin{itemize} % {
\iusr{Евгения Бильченко}
\textbf{Александр Непомнящих} взаимно спасибо, мне просто нужна какая-то ноосфера моральной поддержки, иногда по утрам без гидозепама не проснуться
\end{itemize} % }

\emph{Алексей Бажан}

Сейчас этот человек известен как пропагандист, но "не на камеру", так сказать,
бывал очень откровенен. "нужно понимать, что в Украине никогда не было
пророссийских элит. Все, кто вам скажет, что в Украине есть пророссийские силы,
пророссийские элиты, вводят вас в заблуждение. ...Донецкая и Днепропетровская
области в Советском Союзе снабжались по первой категории, т.е. так же, как и
Москва. Поэтому они нисколько не чувствовали себя какой-то провинцией, к
чему-то присоединенной. Более того, они чувствовали себя достаточно
квалифицированными людьми, и коль выдался такой удивительный шанс, то можно,
действительно, попробовать построить свою страну, организовать в ней свой
бизнес, свой уклад и много чего, что этому слову соответствует. Но при этом
элита, которая занялась этим делом с приходом Кучмы, никогда ни на минуту не
была пророссийской, так же, как группа, в которой лидером сейчас является
Янукович. В следующем тезисе я еще буду этого касаться. Просто хотел обратить
ваше внимание на то, что две элиты – и ни одна из них не является
пророссийской. При этом одна пыталась и, может быть, до сих пор пытается
строить свою государственность, а другая на словах и на деле провозглашает
самоопределение по простой схеме смены хозяина."

\href{https://polit.ru/article/2006/09/28/kulikov/}{%
Россия без Украины, Украина без России, Дмитрий Куликов, polit.ru, 28.09.2006
}

\begin{itemize} % {
\iusr{Евгения Бильченко}
\textbf{Алексей Бажан} мнение принимается
\end{itemize} % }

\iusr{Алексей Бажан}

И далее не менее важная мысль. "Но также на Украине нет и пророссийского
населения. Если вы посмотрите, как они тогда голосовали, то все без исключения
проголосовали за отделение... Поэтому если где-то в Украине и можно найти
пророссийское население, то, может быть, только в одном городе – в Севастополе.
Но я бы там поменял название: оно не пророссийское, оно просто российское. В
городе Севастополе живет российское население, а не население Украины, которое
является пророссийским. А так в Украине, в общем-то, пророссийского населения
нет."С тех пор, конечно, кое-что изменилось, да и тогда можно было заметить
нюансы, но в первом приближении картина верная и полезная.

\begin{itemize} % {
\iusr{Евгения Бильченко}
\textbf{Алексей Бажан} 

ну, а моя аудитория? Я ж все же хоть бомж, но публичный. Любезный Алексей, вы
доводить меня до волокардина: это один Логос, одна цивилизация, один народ.
Зачем вы делите, как американцы, чтобы изолировать? Ну, они зачем - понятно. А
русские зачем? Устали? Махнули рукой? Я тута одна дура? А статья ВВП? Просто
так такие вещи не появляются. Даже у меня. Я, никто, каждый пост продумываю. А
Толочко? А мой Киев, мать городов? Все сольете Бжезинскому? Из-за мазепинских
элит. Фу, из-за вас вообще никакой настрой.

\iusr{Алексей Бажан}
\textbf{Евгения Бильченко} 

2003-ий, коса Тузла, "пророссийский" Кучма стягивает армию в Крым и всерьез
готовится воевать с Россией. Ну, элиты-то понятно (Медведчук тоже на Тузле
отметился, кстати, как и "украинофоб" Дмитрий Кисилев с канала Россия - да, и
он тоже защитник Тузлы от российской агрессии), но где тогда было пророссийское
население? Где акции протеста, перекрытие дорог, препятствование движению
военной техники? Куликов утрирует, конечно, тем не менее, вот Вам реальный
уровень пророссийских настроений за год до первого майдана.

\end{itemize} % }

\iusr{Руслан Павлов}

Они все благополучны и работают(с) Да эта тусовка Моисеев собрана в гопкомпанию
(ОПЗЖ), чтобы водить русских Украины по пустыне 400 лет пока они не кончатся.
Поэтому и благополучны. И не факт, что останутся украинцы которые смогут это
увидеть. Молох глобализации и трансгуманизма сожрет всех не разбираясь. Холопов
же виртуозно отвлекают неправильной формой чубов у соседей. Принципиальность
вышибла Вас из этого каравана на кладбище народов. Так что какая может быть
депрессия по этому поводу? Пойдите, лучше, рогаликов купите)

\begin{itemize} % {
\iusr{Евгения Бильченко}
\textbf{Руслан Павлов} 

Если вынести за скобки Ваш оцененный мной (я серьезно) белый (рогалики) и
черный (кладбище народов) юмор, весьма необходимый нам юмор, скажите мне как
москвич русскому поэту: я - не сторонник хуторского вождения русских по пустыне
до последнего уничтоженного здесь носителя цивилизации, кстати, нашей общей
(простите, я, анархист, но согласна с ВВП, знаю и предчувствую деконструкцию
мозга), это не мои методы. Я также не упоротый антиглобалист настолько, чтобы
надевать лапти, мне приятнее русские ракеты в космосе, хотя в лаптях потрясно
пишется рэп. В общем, я очень жёсткий современный русский, и все эти
перформансы поднадоели. Есть нормальный рецепт?

\iusr{Руслан Павлов}
\textbf{Евгения Бильченко} 

Как москвич не могу) Скажу как мигрант из советского Запорожья) Все наши общие,
русские и украинские беды вызваны проигрышем в третьей мировой войне гибридного
типа и последующей колониальной зависимостью. Народы СССР отодвинули от
источников ресурсов и подтолкнули к корыту у телевизора. Корма отсыпают ровно
столько чтобы не воспроизводились, но и не сорвались с голодухи. Держат
внимание, рассказывают про величие, борьбу с врагами, танцы со звездами и
скидки на новые телевизоры. Ну и про добрых и заботливых мясников и
скотопромышленников. Для неравнодушных устраивают регулярные политигрища между
своими марионетками. Возмутителей спокойствия отправляют на бойню. Анархистам
это должно быть заметнее, они всегда с прищуром на любую систему смотрели.

\iusr{Евгения Бильченко}
\textbf{Руслан Павлов} Это понятно. Об этом и мои книги. Выход неясен

\iusr{Руслан Павлов}
\textbf{Евгения Бильченко} 

Рецепт тут такой. Сначала привести в порядок голову. Посредством здорового
скепсиса, подвергая сомнению все и вся, провести разбор навязанных мифов и
ложных представлений. Выпрыгнуть из чуждой информационной матрицы. Находить
настоящее и делиться с ищущими. Строить свое и искать своих. На роевых
принципах. Не привлекая излишнего внимания пожарных бригад. Тех, которые готовы
жечь книги вместе с людьми.

\iusr{Евгения Бильченко}
\textbf{Руслан Павлов} это проделано - результат не айс)
\end{itemize} % }

\iusr{Natalia Rezanova}
Заслуживаете уважения, бесспорно! Удачи!

\iusr{Рей Віктор}
І вірші-паршиві..

\end{itemize} % }
