% vim: keymap=russian-jcukenwin
%%beginhead 
 
%%file 08_10_2021.fb.bryhar_sergej.1.deti_progulka_mova_jazyk.cmt
%%parent 08_10_2021.fb.bryhar_sergej.1.deti_progulka_mova_jazyk
 
%%url 
 
%%author_id 
%%date 
 
%%tags 
%%title 
 
%%endhead 
\subsubsection{Коментарі}

\begin{itemize} % {
\iusr{Наразі На Часі}
У Львові, москволайнові пісні в машинах швидко виключають, якщо паркуються під будинком, бо може пляшка на капот прилетіти з якогось поверху

\begin{itemize} % {
\iusr{Svitlana Rudych}

\ifcmt
  ig https://scontent-frx5-2.xx.fbcdn.net/v/t39.1997-6/s168x128/851575_126361970881933_2050936102_n.png?_nc_cat=1&ccb=1-5&_nc_sid=ac3552&_nc_ohc=wP9ouyGTFWQAX-HCsxZ&_nc_ht=scontent-frx5-2.xx&oh=f449999632c1af57d623636d1a498a81&oe=6166DB33
  @width 0.1
\fi

\iusr{Назарій Біленький}
\textbf{Наразі На Часі} а їх треба відслідковувати і пляшки кидати
\end{itemize} % }

\iusr{Галина Слободянюк}

Там взагалі не можна гуляти. Не знаю, як витримують мами з малятами у візках.
Вони, бідні сплять під це лайно. Щиро співчуваю. Але це ж можна припинити, а
вони чомусь мовчать

\begin{itemize} % {
\iusr{Serhii Bryhar}
\textbf{Галина Слободянюк} Я сьогодні не побачив людей, яким щось не подобається. Навпаки - це були усміхнені обличчя, діти, що танцюють, кілька не дуже тверезих чоловіків, що намагалися робити як діти). Так, я думаю, помітній частині місцевих це не подобається, але явно не більшості. Більшість, здається, вважає, що це абсолютно нормально.
\end{itemize} % }

\iusr{Yaroslav Lazor}
Від цього не можливо боротися демократичними методами.
А не демократичними ніхто не буде.
Одеса на 99.9\% місто, населення якого сприймає себе виключно в російському культурному просторі. Отже, й музика в кафе, на вулиці, концерти будуть російськомовні.

\iusr{Olga Tumenko}

Особливо дуже це відчувається в Одесі. Ми коли переїхали після окупації Криму до
Одеси, думали що попадемо в середовище українців. Але то якийсь жах був, не
навчила одеситів ні окупація Криму, ні війна на Донбасі. Таке враження, що
більшість населення цього міста живуть в другій реальності.


\iusr{Вікторія Валевська}

Тільки що бачила по Еспресо опитування львів'ян на вулиці, чи хоче зеленський
узурпувати владу. Це, вибачте, пздц. Президент хоче для нас найкраще, ми за
нього, всяка влада хоче узурпувати владу та інша дичина.

\begin{itemize} % {
\iusr{Serhii Bryhar}
\textbf{Вікторія Валевська} 

Схід і Захід разом, "Львові ґаваріт по-русскі", альфа-джаз - це все даром не
минається. Мав одного, скажімо так, партнера львівського... Бере він якось, і
каже: я вважаю, що немає ніякої різниці - українська, російська, - то все одно,
то політики видумали якісь там проблеми, щоб людей сварити... Довелося провести
"мотиваційну частину". І через 5-7 хвилин він почав виправдовуватися. Почав,
чомусь, зі слів: ну то мій дідо в УПА служив, то ми ж пам'ятаємо, то ми ж...
Усе вони знають і розуміють, просто інтерни, г+г та інше повторне лайно не
виносить мозок лише стійким, підготовленим бійцям.. На жаль, маємо справу з
таким лайном, що... Власне, визвольна боротьба триває.

\iusr{Олена Сушко}
\textbf{Вікторія Валевська} Так ,як з'ясувалося ,це такий Львів ,росій.мільяр.Фрідман там рулить , а мер його "слуга".

\iusr{Сергій Лащенко}

Одеса специфічна. На масових заходах, гуляннях треба вимагати хоча б половину
музичного компоненту українською мовою. І обговорювати такий варіант усім
містом. Може хоч такий підхід приведе до зміни настроїв. Сам Ківалов пропагував
двомовність! От і ми за те ж... Якось так )))

\iusr{Svitlana Chub-Krywuzka}
\textbf{Serhii Bryhar} дідо в УПА, а онукові какаяразніца? Оце так хрунь ваш львівський партнер...
Повно таких, на жаль. Чиї діди упівці й січові стрільці в гробах перевертаються.

\iusr{Леся Менкуш-Мальків}
\textbf{Svitlana Chub-Krywuzka} щоб Ви знали, що якраз дуже багато внуків і правнуків упівців виросли манкуртами... Маю таких прикладів багато навіть в своїй родині.

\iusr{Леся Менкуш-Мальків}
\textbf{Вікторія Валевська} а можете знайти посилання на це опитування? Бо я тут під одним постом написала коментар, а люди не вірять

\iusr{Вікторія Валевська}
\textbf{Леся Менкуш-Мальків} навіть забула в якій це програмі. Бо просто шматок подивилась у мами. Пам'ятаю тільки, що Тарас Стецьків був в студії

\iusr{Леся Менкуш-Мальків}
\textbf{Вікторія Валевська} спробую знайти. Дякую!

\end{itemize} % }

\iusr{Anatoliy Lustyk}

Ці придурки так нічого й не навчились співати, крім цих двох дебільних пісень.
Але у Києві ще гірше. На День Незалежності на Хрещатику бачив дві такі групки,
обидві співали "Вьіхода нєт"


\iusr{Dina Wild}

Моя донька наразі у вашому місті. Каже: "Мамо, я питаю, як пройти кудись, а
вони мене сахаються, як чорти ладану". Зайшла в кав'ярню - просто ігнорують. Як
вам там живеться, не уявляю ((

\begin{itemize} % {
\iusr{Леся Рой}
\textbf{Dina Wild} я сама з-під Одеси, щоліта там у родичів, з початку війни говорю укр - на вулицях, в магазинах - то хтось чемно відповідаж російською, а хтось на українську переходить, раз якась бабця втекла, але то виняток..
правда, я страшна як кара божа - мо, тому )

\iusr{Dina Wild}
\textbf{Lesya Roy} вона мала, їй 16. Ще не сміливо відстоює свої кордони. А мене теж бояться, но не всі ))

\iusr{Olga Tumenko}
\textbf{Dina Wild} 

А я після приїзду з Криму розмовляю тільки українською, і мені вони до одного
місця зі своїм язІком. Мене вже захистили в Криму, я не хочу, щоб таке повторилось
в Одесі. Нехай не тягнуть до України рашку, а відвалять до неї самі.

\iusr{Тетяна Навроцька}
\textbf{Dina Wild} Нехай приходить у Книгарню-Кав'ярню, де люблять українців, та спілкуються тільки державною)))
\end{itemize} % }

\iusr{Ivan Gonta}
Увазі мєня скарєй,
Забірай меня
В маскву!

\begin{itemize} % {
\iusr{Ольга Брох}
\textbf{Ivan Gonta} та нехай ідуть н.....й!

\iusr{Елена Пака}
\textbf{Ivan Gonta}
Беру собі на цитати)
\end{itemize} % }

\iusr{Леся Рой}

мене дуже вибісило, коли у 16му виступали по ато, і от ми всі співаємо наживо
українською, а масовики, що з нами ж приїхали, крутять рос попсу і волають
"налєтай, бєсплатная сладкая вата!.." (

\begin{itemize} % {
\iusr{Анна Саутина}
\textbf{Lesya Roy} оця любов до халяви - в піснях про бєсплатноє. Совок

\iusr{Леся Рой}
\textbf{Anna Sautina} зате в 14му на ній попалось кілька бойовиків - полізли без черги за безплатними сосисками, а місцеві їх здали )

\iusr{Анна Саутина}
\textbf{Lesya Roy} життя буремне
\end{itemize} % }

\iusr{Maria Nakonecna}

Бо у Львові боком люди не пройдуть, а заставлять себе поважати. Шпанюки .Яке воно
виросте, а піде голосувати, то жах.  Та вся шушара має свою родіну мать їхню, там
хай собі слухають-пригають

\begin{itemize} % {
\iusr{Людмила Демянчук}
\textbf{Maria Nakonecna} це віднедавна, бо ще місяць - два назад у Львові завивали рускоязикі, але
УНА- УНСО провели виховну роботу , і вуличні трубадури плавно перейшли на солов'їну.
Хоча російська у Львові на кожному кроці, особливо в центрі, бо багато рускоязиких туристів, а місцеві заробітчани - продавці швидко підстосовуються, щоб заробити лишню копійчину, для них Бог - гешефт !
\end{itemize} % }

\iusr{Ivan Melobensky}
Дуже дратує тут ця музика, коли гуляю там...

\iusr{Олександр Кондратюк}

7-8 років тому подібне відчуття я мав, повертаючись до Києва з Волині та
Галичини. Зараз Київ, попри те, що ситуація далека від ідеалу, все ж не
викликає подібного враження. Повільно, малими кроками, але тут ситуація
покращилася

\begin{itemize} % {
\iusr{Nataliia Purzhash}
\textbf{Олександр Кондратюк} і це правда..
\end{itemize} % }

\iusr{Олена Матвійчук}

Росія в кожному домі, в основному, через ютюб лізе. Дивишся українське і тут
вискакує реклама російською. От скажіть на що ці рекламодавці надіються?! Вони
просто дратують!.

\begin{itemize} % {
\iusr{Світлана Стрижак}
\textbf{Олена Матвійчук} та чому тільки через ютуб? Ви 5 хв. гляньте рекламу на інтері. переключаючи канали, в мене від 1 хв. волосся дибки стає

\iusr{Лариса Гошкодеря}
\textbf{Олена Матвійчук} А діти ж сучасні з інтернету не вилазять, от вам і тотальна русифікація...
\end{itemize} % }

\iusr{Ігор Мазепа}
Це можна назвати суспільною ентропією - тяга примітиву до чогось дурнуватого.

\iusr{Оксана Мойсеенко}

Останнім часом частенько чутно рашеннські пісні на вулиці. І що цікаво, дуже
голосно!!! І лунає вона з динаміків, які летять на самокатах або
мотоциклах, іноді з машин. Таке враження, що це все навмисно робиться...

\begin{itemize} % {
\iusr{Олександр Журавель}
\textbf{Oxana Moyseenko}
Ой, не враження. Так воно і є - навмисно. Пліснява, вона завжди навмисно пролазить і розповзається.

\iusr{Оксана Ковалишин}
\textbf{Oxana Moyseenko} Ні, вони такі і є.
\end{itemize} % }

\iusr{Dmytro Dzyuba}
На Запорожжі і в Дніпрі те саме...
Ці вже нафталінові "рукі ввєрх", отой увесь "русскій шансон і блатняк" з нафталіновими круґамі-крічєвскімі та "апґрейжені" під "русскій блатной репчік", а ше з ганделиків можна почути половину отого нафталінового "сборніка саюз" від ґубіна до апіной:
"он уєхал прочь на начьной елєктрічкє..."
("а патамушта в двєрі зажало яїчкі!") — зазвичай додаю подумки...
Совки як застигле гівно мамонта застрягли у своїх "ліхіх 90-х"!
"...і снова трєтьє сєнтябля..."

\begin{itemize} % {
\iusr{Елена Пака}
\textbf{Dmytro Dzyuba}
Підтримую.
Про "електрички-яїчки" беру собі на цитати.
І одразу покаюся-признаюся: "Трєтьє сєнтября" мене лікувало після невдалого роману  @igg{fbicon.face.tears.of.joy} 

\iusr{Dmytro Dzyuba}
Та будь ласка, беріть на цитати  @igg{fbicon.smile} 
Коли мені було 17 і влаштувався на першу "роботу" продавати касети з музикою на базарі (ох вже ці 90-і), то якраз Україну почав бомбардувати "руссцкій шоу-бізнєс" і полізло оце все лайно, знищуючи на своєму шляху паростки української музики й пагони нормальної західньої музики...
Тому я вигадував ось такі "продовження" їхніх "пєсєн" і розважав друзів, котрі зазвичай кучкувались поруч.
"І там шальная Імпєратріца
всё над конямі і потёмкіньім ґлумітца да ґлумітца..."
"Зіма-халада, наступаєт всєм πzda..."
Люто ненавиджу будь-яку "руссцкую музьіку"!

\iusr{Антоніна Грицан}
\textbf{Dmytro Dzyuba} такe скрізь, і трапляється також у Львові, і Карпатах, музика, навіть продавці, мeнe взагалі цe засмутило, що відпочиваючих ,,грeблю гати,, що й продавці, сумно, дно, мабуть сама найгірша нація, нічого свого нe вивчає, нe любить, ганьбить, всe дно

\iusr{Dmytro Dzyuba}
Пані \textbf{Антоніна Грицан}, ні, ми не "найгірша нація".
Українці — найкраща Нація!
Просто у нас забагато манкуртів, покручів і понавезених окупантів. Адже криваве ХХ століття війнами, геноцидами, трьома хвилями Голодомору суттєво підкосило нашу Націю.
Українців, справжніх свідомих Українців лишилось небагато, але достатньо для того, щоби відродити Націю!
І ми відродимо нашу Українську Націю!
Якщо будемо до цього всі прагнути!

\iusr{Антоніна Грицан}
\textbf{Dmytro Dzyuba} 

я в розпачі, повністю, мeнe ,,добило,, пeрeймeнування, нашого сeлища Більмак,
попeршe нeзаконно, згідна багатьом нe подобається, по-пeршe зручно на Новій
пошті, нe завeзуть, по -другe, ми вжe сeло можна гарну назву дати, просто
Кам, янок навіть у Пологівському районі дві, а по Украіні, і люди просто нe
розуміють, і нe хочуть розуміти, спочатку я також думала щоб Кам, янка, так на
мeнe стільки виллилося бруду, і,, душа кам, яна,, і іншe, а тeпeр з, ясувалося що
супeр, мeні цe нагпдало лиш би нe Порошeнко,, от маємо e що маємо а можна було
гарну назву дати, ну вжe нічого нe змінити, так в мeнe є сумніви вeликі що до
Украіни

\iusr{Dmytro Dzyuba}
\textbf{Пані Антоніна} 

Грицан, я один з тих, хто сприяв захисту топоніма Більмак спільно з нині
покійним паном Віктором Вернигором. Саме серія публікацій на моєму майданчику
«Порохівниця» стали вагомим арґументом у Дніпровському апеляційному суді під
час спроб коллаборантів скасувати цілком законне рішення Держави Україна у
перейменування в рамках декомунізації райцентру "куйбишево" на Більмак.

Нагадаю, згідно Закону про декомунізацію, перейменування декомунізованих назв
оскарженню не підлягає і змінювати чи повертати назву буде порушенням цього
Закону, адже це поставить під сумнів рішення Держави Україна.

Назва Більмак, як і назва Токмак — це стародавні тисячолітні топоніми
тюркського походження, це назви двох найвищих гір Лукоморської височини
(відомої нині як "Приазовська") — найвищі колись гори "між луками й морем" —
між Великим Лугом Запорожським і Озівським морем.

Запорожські козаки не цурались цих тюркських назв, і згідно запорожської
січової леґенди, саме на горі Більмак була одна зі спостережних чот, з якої
подавались сигнальні вогні в разі наступу ворога зі сходу. А під горою Більмак
жив за леґендою козак, який і взяв собі таке ж прізвисько.

Тож вам, земляки з Більмака, варто пишатись такою кольоритною назвою.

Більмак не гірше аніж Токмак.

\ifcmt
  ig https://scontent-frt3-1.xx.fbcdn.net/v/t1.6435-9/244792148_1584751858526983_7705880002539674851_n.jpg?_nc_cat=102&ccb=1-5&_nc_sid=dbeb18&_nc_ohc=LTYrIVmjtFUAX_PG_ey&_nc_ht=scontent-frt3-1.xx&oh=47e0b81567055a2b20183e8e2563e612&oe=61868255
  @width 0.4
\fi

\iusr{Антоніна Грицан}
\textbf{Dmytro Dzyuba} та я цe всe знаю, алe ж ми вжe Кам,янка

\iusr{Антоніна Грицан}
ВР проголосувала 7жовтня

\iusr{Dmytro Dzyuba}
Це пряме порушення Закону про декомунізацію!!!  @igg{fbicon.anger} 
ЧОРТ! Так ці коллаборанти таки домоглися свого? Виродки!
Вже й вікістаттю переписали, з маячнею про "більмо на оці"!

\url{https://uk.wikipedia.org/wiki/Кам%27янка_(смт)}

\iusr{Антоніна Грицан}
що тeпeр?

\iusr{Антоніна Грицан}
я хотіла вам ранішe написати, думала ви знаєтe

\iusr{Dmytro Dzyuba}
Ну, вже всьо...
Ні, я не знав...
Ну, тепер чекати, доки режим коллаборантів буде знесено й повертати назад все те що ці тварюки сєпарські зараз витворяють...

\iusr{Dmytro Dzyuba}
Дякую, пані \textbf{Антоніна Грицан}, що повідомили...

\iusr{Антоніна Грицан}
\textbf{Dmytro Dzyuba} вжe ж нe змінити чи є щe надія?

\iusr{Антоніна Грицан}
подали скаргу до ВС,

\iusr{Антоніна Грицан}
мабуть вжe нічого нe змінити

\iusr{Dmytro Dzyuba}
Так, при нинішньому режимі коллаборантів, сєпарів і українофобів нічого не змінити...
Хто там нинішній "ґлава" - сєпар боріс ковальчук?
Що ж, сєпари свого добились...
На жаль, більшість населення вашого селища цьому активно посприяли...
Ворогам України чудово вдалось заліпити "більмом" очі мешканцям колишнього "куйбишево"...
Прикро...
 @igg{fbicon.frown} 

\iusr{Dmytro Dzyuba}

...найбільше в цій ситуації жаль нині покійного пана Віктора Вернигора... Ваші
"хуйбишевські ватніки" йому життя вкоротили нервомоткою з судами...

\url{https://porokhivnytsya.com.ua/2020/11/02/kozak-vernygor-bilmak}

\iusr{Dmytro Dzyuba}
Пане \textbf{Serhii Bryhar}, погляньте, про який реванш коллаборантів стало відомо...
Друже \textbf{Іван Мелобенський}, не лише у вас в Одесі вата "торжествуєт", у нас ось теж...
Зрештою, це було очікувано...
Адже Державу руйнують...
"Епоха негідників"...
 @igg{fbicon.frown} 

\end{itemize} % }

\iusr{Юлия Кривенко}
Мене вранці шансон з гучномовця дратує.....

\iusr{Микола Орел}
Прикро!

\iusr{Роксолана Роксолана}

Живу у Львові. На жаль масковську попсу чую дуже часто з колонок, які носять з
собою підлітки. Слухають русскій реп... А я в свої ...надцять за часів СРСР
захоплювалась сестричкою Вікою та Братами Гадюкіними.

\begin{itemize} % {
\iusr{Оксана Ковалишин}
\textbf{Роксолана Роксолана} Ті колонки - взагалі дно. Що б звідти не виповзало.
\end{itemize} % }

\iusr{Приступа Володимир}

Чому це відбувається? А от чому бо все закінчується ось тут в ФБ. Ось написали
про Львів, про пляшки , про капоти ... Це все залякування. Жодного такого
випадку не чув. У нас все тут. У ФБ лайки про мову, поширення коментарі, а ще в
тролейбусі чи трамваі поговорити про мову, оглядаючись при цьому щоб не дай
Боже не почув якийсь російськомовний- може образитися. Ось так в є. Там війна а
тут все як було так і є. Один купує товари паРаші в магазині і кричить - а що
тут такого? Яке це відношення має до війни. Другий стоїть на АЗС і заправляє
авто з номерами RUS, третій приймає туристів з паРаші. Ті йдуть на концерти, ті
дивляться фільми, ті ідуть в паРашу на спортивні змагання, ті приймають у себе
вчених з паРаші на симпозіум чи семінар...Там сьогодні вбили хлопця на фронті а
у нас теплі посиденьки за смачною кавою з друзями з паРаші. Ми війну ще
програли в 2014 році коли не назвали окупанта окупантом, агресора агресором. Ми
воюємо з уявним х*йлом- бо це він поганий і при цьому дружимо і співпрацюємо з
паРашею - бо вони ні в чому не винні. Це не солдат з паРаші вбив нашого хлопця
- це все поганий дядя х*йло змусив силою його вистрелити

\begin{itemize} % {
\iusr{Диновський Євген}
\textbf{Володимир Приступа} 1000\%
\end{itemize} % }

\iusr{Олена Сушко}

Та влітку в Гідропарку не можна було пройти спокійно , російські місні "на всю
мощь" лунали з усіх генделиків .Зрозуміло, що для виборців зелі ,але ж противно
....

\begin{itemize} % {
\iusr{Войткова Юлія}
\textbf{Олена Сушко} 

я не прихильниця "зєл", але давайте будемо відвертими. Хіба не можна було на
хвилі масового патріотизму з 2014 заборонити все запоребрикове лайно? Чому
цього не зробив порох? Не вистачило сили волі? А чи тому, що сам -
російськомовний,?

А мав можливість! Скільки тоді нових україномовних виконавців з'явилось. А які
патріотичні пісні для юного покоління! Бери - й неси в маси. А не було кому!
Тільки й хватило, що одноглазніки і контакти закрити

\iusr{Валентина Яковенко}
\textbf{Войткова Юлія} а скільки було вою за закриття цих ОК і К, а як Сватів
заборонили, як Вова Зєлєнский гнівно обурювався з цього приводу(напевно і
мститься за це Порошенкові),а як почали непускати російських співаків до нас -
зною вой на весь світ був.  Так що не кажіть, для того періоду немало було
зроблено. Багато почало виходити україномовних фільмів, засвітилися українські
гурти і співаки, їх більше можна було почути по радіо, все потихеньку двигалося
на укранізацію і якби було так і дальше ,то років через десять вся б Україна
заговорила б українською. Але наш "мудрий нарід" повернув вектор розвитку назад.
Дай Бог Україні хоча б залишитися Україною.

\iusr{Олена Сушко}
\textbf{Войткова Юлія} Знов "порох" .Слухайте ,ну це вже зовсім ....Кругом повинен був він . Навіть зелю він привів ,а що ,це ж він і кончених сватів ,і для дурбецалів 95 робив .Ну так надоїло вже одне і теж.

\iusr{Олена Сушко}
\textbf{Валентина Яковенко} 

Дякую ,дійсно така вонища була проти ПОП і на раші і тут. Зато тепер все добре,
більшості все подобається. А за Гідропарк ,ну це вже місцева влада повинна
дбати, та в ті генделики видно одна сепарня гопна ходить, а їм рашиське лайно
музичне довподоби.

\end{itemize} % }

\iusr{Анна Саутина}
Це все від непокараності

\begin{itemize} % {
\iusr{Dmytro Dzyuba}
Саме так... Безкарність породжує вседозволеність...
\end{itemize} % }

\iusr{Олександр Калюжний}
Так воно і є. Нашу країну наші відверті внутрішні вороги, роблять нас чужою, а нас в ній - чужаками

\iusr{Олександр Руденко}
Заради нашого існування - вкрай необхідно працювати з дітьми, починаючи це робити з дошкільного віку, бо за ними майбутнє України.

\iusr{Maryna Byshenko}
Це щось дивне. Дітки зараз на моргенштерна і славу марлов підсіли, а не цей олдскул

\begin{itemize} % {
\iusr{Tetiana Brygar}
\textbf{Maryna Byshenko} та в таких місцях діти по 2-5 рочків застряють. Старші, зазвичай, вже не танцюють біля вуличних музикантів.

\iusr{Serhii Bryhar}
\textbf{Tetiana Brygar} Є таке. Старші дійсно з колонок і телефонів злухають москворепчік. Ну а 30-річним, тобто, батьками отих, кому 2-5, ця х***я заходить аж бігом(.
\end{itemize} % }

\iusr{Єлизавета Бойко}
Саме гірше те що багато українців розмовляють суржиком і не задумуються,
переконати їх неможливо.

\begin{itemize} % {
\iusr{Serhii Bryhar}
\textbf{Єлизавета Бойко} Та краще вже так, ніж взагалі ніяк. Оті ж, що "хочуть, але не переходять, бо не хочуть псувати мову", теж пояснюють свою поведінку негативним ставленням до суржику. От і виходить: "краще я вже була російською"...

\iusr{Анна Бутова}
\textbf{Єлизавета Бойко} суржик є нормою для тих, хто переходить на українську.

\iusr{Vlodko Gray}
\textbf{Єлизавета Бойко}
Чушь

\iusr{Olexandr Svirinenko}
\textbf{Єлизавета Бойко} неосвідчені.
Починаеться усе з народження.
Які батьки, такі й діти.
\end{itemize} % }

\iusr{Олечка Катрич}

Дуже прикро таке читати ... І часто з таким теж стикаюсь за межами Львівщини.
Відпочивали цього року в Закарпатті ( Солотвино ) , то на моє прохання вимкнути
російську музику - дивились на мене як на інопланетянку. Повертайтесь до нас,
до Львова !

\begin{itemize} % {
\iusr{Serhii Bryhar}
\textbf{Олечка Катрич} Скоро буду  @igg{fbicon.smile} 
\end{itemize} % }

\iusr{Володимир Катькало}

В кожному краї, свої звичаї, свій колорит, але вичавить совка складно.... може
б в ставок його, так мовчки.....і все!!! А ви написав мені, щоби я це
зробив.....одні балачки, робить справу.... ні ні, хай хтось....В тім і біда
наша.....

\begin{itemize} % {
\iusr{Serhii Bryhar}
\textbf{Володимир Катькало} Не може одна людина, ще й змалюками, перти проти двох чи трьох десятків, як м усе подобається. Не кажучи вже про те, що формально вони нічого не порушують... Колись я пробував говорити, але це має сенс там, на заході, а тут потрібно якось інакше...

\iusr{Serhii Bryhar}
І до речі, цікаві думки, поради, прогнози ось тут є вже.
\end{itemize} % }

\iusr{Надія Зволінська}
Я з Запоріжжя, це жахіття усюди.

\iusr{Володимир Шашкевич}

Моя думка така що росія готує плацдарми для нових захоплення, для нових
вторжень на наші території, чим більше народу буде русифіковано, тим більше
нарід буде вважати какая разніца, ми адін народ, тим більше росія порятує
українських територій від злісних фашистів, Бандерівців...

\iusr{Ntina Ntoubrova}

Просто 1) У Львові місцева рада ЗАБОРОНИЛА у публічних місцях грати московську
музику. 2) вуличним музикантам дали по пиці місцеві АТОшники.

От тільки через те у Львові немає російської музики. Бо якби не ця
принциповість - будьте певні, місцевих хробаків всеядних тут не менше, ніж в
Одесі. Особливо якщо зауважити, скільки переселенців із Донбасу живе. У і
кожному в голову не залізеш.

А щодо Одеси. Пан мер створив міську варту і керує тітушнею. Тому навіть якщо
розпочинати такі рейди із програвання музики - то тітушні буде більше, яка
почне поножовщину. Тому така ситуація. А було б жорсткіше - я думаю більшість
би гуділа, але потім сприйняла.

\iusr{Ярослава Рысцова}
Треба це припинити на законодавчьому рiвнi.

\iusr{Наталія Бонь}

Я недавно на заході в освітньому закладі почула "до болю рідну" мелодію "
Піонерського маршу" Отого що "Блізітся ера свєтлих годов Кліч піонєра- всегда
будь готов!" Тільки музика, без слів.  @igg{fbicon.cry}{repeat=3} Але...

\begin{itemize} % {
\iusr{Anna Shepko}
\textbf{Наталія Бонь} "Я уб'ю сасєда для тєбя, для тєбя" почуте в 90-их роках у львівській маршрутці виявилось пророчим!

\iusr{Наталія Бонь}
\textbf{Anna Shepko} Зі словами треба бути обережними...
\end{itemize} % }

\iusr{Artem Artem}

Який сенс писати це, якщо нічого не робити? Я колись у парку в Борисполі зробив
зауваження щодо московської музики, яка лунала зранку 14 жовтня. З тих пір, не
знаю, чи пов'язано зі мною, чи ні, але там такої ахінеї вже немає і близько

\begin{itemize} % {
\iusr{Liudmyla Chayka}
\textbf{Artem Artem} І я роблю зауваження, не лінуюся. Іноді працює

\iusr{Artem Artem}
\textbf{Liudmyla Chayka} я до речі згадав ще декілька ситуацій у нашому місті - всі вони були на дитячих заходах і щоразу зауваження приносили результат
\end{itemize} % }

\iusr{Тетяна Гривнак}

У Львові в центрі співаки часто співають російські пісні. Дивно, що Ви не чули.
І повно переселенців, бо в спальних районах їх чути. І поведінка така ж, як і
їх язик. Чому русскоязичні така худоба?.. Чому вони так говорять, ніби глухі,
ніби окрім них, нікого більше на світі нема....

\iusr{Vlodko Gray}
Є набагато краща іноземна музика
А відчуття що ми в Тундрі ?

\iusr{Петро Паславський}

Мова УКРАЇНСЬКА має бути всюди премійована(в навчанні і роботі)!!!
РОБОТУ КРАЩУ І В ПЕРШУ ЧЕРГУ МАЮТЬ ОТРИМУВАТИ УКРАЇНОМОВНІ УКРАЇНЦІ!
І ГОЛОВНЕ!!!- особливо впертим і тупим ВИДАВАТИ ПАСПОРТ НЕГРОМАДЯНИНА УКРАЇНИ (з усіма наслідками)!
ВОРОГІВ(а їх ще в нас є достатньо-ВАЛІЗА, ВОКЗАЛ, МАЦКОВІЯ)-ГОНИТИ ГОЛИМИ ,ЯК ВОНИ СЮДИ ПРИЙШЛИ!

\iusr{Маряна Лесик}

В Брюховичах на озері в кафе крутили якийсь шансон. Слава Богу, що я не
розбираюсь в цьому лайні. Але люди були в захваті від музичного супроводу. Я
запитала чи є українська музика? ''Так є''. Питаю ''ііііі де вона?'' -''Пізніше
включимо''........ завіса.

Було багато людей, які підспівували і косо на мене дивились. От чесно страшно
будо щось доводити в місці де повно шакалів, а я одна з двома дітьми.....

\iusr{Оля Сорока}
Так і є... Поки нам пістоль в дупу не встрілить, мозок не працює. А потім незавісімі республіки... Ітд

\iusr{Liudmyla Chayka}
Я набираюся духу, підходжу і прошу російське вимкнути. Іноді допомагає

\iusr{Olexandr Svirinenko}

Ця псевдо субкультура, лізе як скажена, з усіх дірок.
Бо вони розуміють, що втрачають Україну, тому хочуть зупинити, або загальмувати цей процес.

\iusr{Віра Ткачук}
Має кругом бути Українська мова і пісні насамперед бо вони найкращі, це Україна вже 30 років відновлена!

\iusr{Лариса Бондаренко}

Минулої суботи ілу повз ТЦ, там відкрився новий магазин дитячих іграшок і як
реклама звучать дитячи пісні. Ну, " от улибкі.." ще так - сяк" , але тут " ех,
харашо і странє савєцкай жить.." Мене аж підкинуло. Пішла, питаю , чому такий
репертуар, - " а что, совецкие песни ето плохо?". Не стала їй пояснювати чому
це " плохо" , просто попросила :" поставте, будь-ласка українські". На диво,
адміністраторка не стала сперечатися, хоча була, як мені здалося налаштована
войовниче, відповіла:" добре, поставимо". Не знаю , чи поставила...

\begin{itemize} % {
\iusr{Svitlychnyi Olexandr}
\textbf{Лариса Бондаренко} за таку пісню ще й можна строк отримати
\end{itemize} % }

\iusr{Natalja Melnyk}
Львів не знав 400р рашки

\iusr{Юлия Горда}
Вона не тільки у парках і маршрутках, вона й в садочках і школах, на спортивних руханках, на святах..

\iusr{Любовь Войтович}

ВИКОРЕНИТИ З СЕБЕ РАБА НЕ ВСІ МОЖУТЬ. НАЛЯКАНІ СТАЛІНИМ, ГУЛАГОМ. І БАГАТО
ПІДЛАБУЗНИЦТВА- ОСОБЛИВО ПО СЕЛАХ БУЛО. ЯК УЖЕ ВОНО ДЕПУТАТ СІЛЬРАДИ ЧИ
БРИГАДИР-ВСЕ. ВЕЛИКИЙ ПЕРЕЦЬ.

В МІСТІ ПЛЮНУВ-ТА ПІШОВ ПРАЦЮВАТИ НА ДРУГИЙ ЗАВОД ЧИ ФАБРИКУ.

ДАЙТЕ ЛЮДЯМ СВОБОДУ ПО ЗЕМЛІ-ЧИ ПРОДАДУТЬ , ЧИ ЗАЛИШАТЬ. ФЕРМЕРСТВО РОЗВИВАТИ.

\iusr{Людмила Даценко}

А молодь якась інертна стала, пливе, як г*мно за течієї.А проти течії, проти цієї сміттєвих музики не йдуть....
От вам і виховання....і дитсадочку... школа.... І вузи.... Або батьки проросійські клоуни...

\begin{itemize} % {
\iusr{Тетяна Плачинда}
\textbf{Lyudmila Datsenko} тут я з Вами не можу погодитися. Не можна про всю молодь так думати. Мої діти, багато хто з моїх студентів слухають і просувають саме український контент. Не все втрачено панове! Потроху рухаємося до свого, до рідного  @igg{fbicon.heart.blue}  @igg{fbicon.heart.yellow} 
\end{itemize} % }

\iusr{Анна Федорів}
Треба штрафувати, тих хто крутить російську попсу!!!

\iusr{Ярослав Шевчук}
ти мій пейсдруг, спостерігаю за тобою у пейсбуці, ось зараз виникло А ЩО ДАЛІ, бо не видно розвитку, поступу нема

\iusr{Микола Орел}
Привіт Всім. ШАНУЙМОСЯ!!!

\iusr{Ирина Васьковская}
А ще, вона у більшості в головах засіла і ніяк вони не хочуть з цим боротись і це розуміти! Прикро дужжжжжеее прикро....

\iusr{Alla Uhta}
7.10.2021:Киїа, Хрещатик, ""...я солдат, нєдоношений рєбьонок войни..". - і глядачі довольні, танцюють, кайфують. Що з нами не так? Чи може я чогось не розумію?

\iusr{Елена Хотимченко}
Треба заборонити це лайно. грати і слухати в. публічних місцях!

\iusr{Alyona Dovgan}

А Ви вмикайте ту музику, що Вам подобається і слухайте теж на всю гучність. Як
це роблять школярі з колонками.

\begin{itemize} % {
\iusr{Федір Опанасенко}
\textbf{Alyona Dovgan} та не хочеться й ганьбитися, тому що скажімо діти з
портативною колонкою — то таке, а доросла людина з портативною колонкою, з якої
волає музика на всю — то всім здаватиметься, що з тою людиною щось не теє...)))
тому навушники — то є найкращий варіянт)

\iusr{Alyona Dovgan}
\textbf{Федір Опанасенко} 

у нас на Харківщині на місцевій водоймі на пляжі літом відпочіває купа
компаній. І от здебільшого у них лунає расєянська музяка типу "Доктор, пропиши
мне траву. Я ее курить не буду, я ее заварю". Тільки з однієї машини чула один
раз щось типу ОЕ.

\end{itemize} % }

\iusr{Олена Тарасова}

Це дійсно варто уваги. Моя доня колись вчилася у школі для музичнообдарованих
дітей, і вона першою за вернула увагу на те, що співають її однолітки. Це
печалька. У Ха теж такого, нажаль, багато, цього рузького співу, яке називають
красивим французьким словом шансон


\iusr{Konstanty Panov}

Раджу пану автору втрапити до Львова у переддень восьмого березня на якусь
закриту карпаруху, особливо десь на окраїні!

\iusr{Ольга Фарина}
\textbf{Konstanty Panov} а що у Львові мало понаєхавших? Багато.. Але потрохи іх привчають...

\iusr{Федір Опанасенко}
Закрити радіо «Шансон»! Таку туфту крутять, що хай Бог милує...

\iusr{Раїса Марковець}

Пам"ятаю в срср також забороняли слухати американських співаків, носити
американські джинси і т. д. Причина: пропаганда капіталістичного світу. Чим
більше забороняли, тим більше молодь слухала зарубіжну музику. Тепер те ж саме.
Не потрібно перегинати палку. Зламається.

\begin{itemize} % {
\iusr{Lubov Mastepanova}
\textbf{Раїса Марковець} Порівняння некоректне. Америка не прийшла з війною до СРСР. Не забрала Крим, Донбас. Нам торочили про " загнівающій запад" і впарювали про світле комуністичне майбутнє. Але багато людей і тоді мали свої мізки і не хотіли жити по вказівках компартії.

\iusr{Олена Вілівчук}
\textbf{Раїса Марковець} ви є просто скарб для кремлівського маньяка  @igg{fbicon.thumb.down.yellow}{repeat=3} 
\end{itemize} % }

\iusr{Ольга Смеловская}

Кілька років тому вже після 14 року у нас на якомусь міському святі теж таке
було. Хто телефонував на гарячу лінію міськради, хтось на сайті залишив
відгуки, але більше такого не повторювалося. Все в ваших руках

\iusr{Viktor Petrenko}

\ifcmt
  @width 0.2

  ig https://scontent-frx5-1.xx.fbcdn.net/v/t1.6435-9/244573316_252916150043695_7569005247188139861_n.jpg?_nc_cat=110&ccb=1-5&_nc_sid=dbeb18&_nc_ohc=eb3Fhu-j0OQAX-_UZ-B&_nc_ht=scontent-frx5-1.xx&oh=d5d972e6e2813b79e3a5c6439a211379&oe=618937E3

	ig https://scontent-frt3-1.xx.fbcdn.net/v/t1.6435-9/244915486_252916233377020_664807489630276857_n.jpg?_nc_cat=102&ccb=1-5&_nc_sid=dbeb18&_nc_ohc=HAgBM4evFTgAX-MuC_z&_nc_ht=scontent-frt3-1.xx&oh=749479ba258bc13da5d0b98f2620c71f&oe=618802F2

	ig https://scontent-frx5-1.xx.fbcdn.net/v/t1.6435-9/244964550_252916396710337_1414638927854629769_n.jpg?_nc_cat=110&ccb=1-5&_nc_sid=dbeb18&_nc_ohc=vJuqt5HTV-gAX_ilid6&_nc_ht=scontent-frx5-1.xx&oh=2905c87de50ffdd00b5eeb9797c66639&oe=6187EE75
\fi

\end{itemize} % }
