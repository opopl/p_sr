% vim: keymap=russian-jcukenwin
%%beginhead 
 
%%file 25_05_2021.fb.fb_group.story_kiev_ua.2.na_zare_tumannoj_junosti
%%parent 25_05_2021
 
%%url https://www.facebook.com/groups/story.kiev.ua/permalink/1670146889848772/
 
%%author Киевские Истории
%%author_id fb_group.story_kiev_ua
%%author_url 
 
%%tags 
%%title На заре туманной юности 1944-1954
 
%%endhead 
 
\subsection{На заре туманной юности 1944-1954}
\label{sec:25_05_2021.fb.fb_group.story_kiev_ua.2.na_zare_tumannoj_junosti}
\Purl{https://www.facebook.com/groups/story.kiev.ua/permalink/1670146889848772/}
\ifcmt
 author_begin
   author_id fb_group.story_kiev_ua
 author_end
\fi

Leonid Shleifman
\url{https://www.facebook.com/groups/736908309839306/user/100001115657853/}

На заре туманной юности 1944-1954.

– В сентябре 1944 г. мы вернулись из Казахстана в освобожденный от немецкой оккупации,  родной город Киев.  

Почти весь центр Киева был разрушен до основания. На Крещатике – одни
развалины. Кроме центрального Универмага и ещё нескольких домов, всё было
разрушено. Война ещё продолжалась, но она переместилась на территорию других
западных стран. 

– Первое время, мы жили прямо на улице, во дворе нашего довоенного дома. Папа с
мамой оформляли какие-то документы, а я сторожил наши личные  вещи, хотя у нас
ничего, кроме одежды не было. Наша довоенная квартира была занята семьей
управдомши. Отец потребовал, чтобы она освободила нашу  квартиру, но она
вызвала своего сына, - милиционера. Он забрал отца в милицию и продержал его
там 2-е суток в карцере, после чего заболел. Мама тоже кашляла и чихала. Мы
продолжали жить на улице во дворе. Кто-то дал нам раскладушку на которой я
спал. Прошел месяц и  управдомша предложила моим родителям временно пожить в
полуподвале на соседней улице Чапаева 9  (сейчас ул. Липинского  9) на фото
показано стрелкой красного цвета. Отец согласился, т.к. мы все были больны и
оставаться на улице больше не могли.  Уже был октябрь месяц.

– В этом полуподвале мы прожили «временно» -  19 лет.

Во время войны мой отец стал инвалидом и еле-еле мог передвигать ноги. Отец
оформил все документы на прописку в Киеве, и его тут же взяли на работу в трест
“Укрфото”, куда входили все фотографии гор. Киева. Этому способствовал его
Правительственный вызов за подписью Бажана.

Справка: - Микола Бажан – український поет-футурист, письменник, сценарист,
громадський та політичний діяч, перекладач, постійний автор і фактичний
редактор журналу «Кіно».

Как политический деятель, он в 1944 году  работал Заместителем Председателя
Совета Министров Украины.

 Он начал работать в фотографии №1 по ул. Ленина 24 (сейчас ул. Богдана
 Хмельницкого) в качестве художника-портретиста и ретушера. Это было в получасе
 ходьбы от нашего дома и это вполне  устраивало его. Перед войной он окончил
 Киевский художественный институт и успел немного проработать там же. 

– Чем же он занимался на работе? Во время войны погибло очень много людей. А их
близкие хотели иметь на память портреты своих погибших родственников. У людей
оставались лишь небольшие потертые и мятые старые фотокарточки.  В «Укрфото»
делали репродукцию этих маленьких фотокарточек и печатали на большую
фотобумагу, обычно размером 24х30 или 30х40см. только силуэт нужного лица. А
мой отец с помощью обычной сажи, которую он сам  делал дома, дорисовывал все
детали лица. Когда портрет был готов, он закреплял его, чтобы сажа не
стиралась. Он разводил водой одеколон и пульверизатором брызгал на портрет.
Жидкость высыхала и сажа прилипала к фотобумаге.

– Мы жили в нашем полуподвале в одной комнате 18 кв.м. В комнате был кран с
холодной водой и раковина. 2 окна выходили на улицу, но упирались в стену
огромной ямы перед окнами. Они были ниже уровня земли, где-то на 1.5 м. С
потолка в комнате нависала лестница. Это был вход для соседей с 1-го этажа в
подъезде дома. Одна из стенок комнаты являлась стеной нашего подъезда и в
зимнее время она всегда была сырой  и  мокрой.  Газа и  электричества не было,
а туалет находился во дворе на свежем воздухе. Мама готовила пищу на примусе.

– Я должен был пойти в школу, но не смог. У меня не было никакой одежды, ни
обуви, ни пальто, ни шапки. Просидел всю зиму дома. Из консервной банки  сделал
коптилку и  вечерами — это был наш единственный источник света.  Нужно было
наливать керосин и поджигать фитилек.

Мне исполнилось 7 лет и я понял как работает примус и какие у него бывают
неисправности. Когда понимаешь, как устроен какой-то прибор, то проще его
привести в идеальный порядок. Наш примус всегда был в отличном состоянии. А
когда в нашем дворе у кого-то из соседей появлялись проблемы с примусом, то
меня , как «профессора» приглашали соседи для ремонта. У меня были примусные
иголки, которыми я чистил форсунки, ремонтировал насосы и делал профилактику
примусов. После моего ремонта, примус блестел так, как будто он позолоченный.
За ремонт примуса мне давали кусок хлеба со свиным смальцем и я был счастлив
таким угощением. А один раз меня пригласила жена учителя и за мою работу в знак
благодарности, - намазала  мне кусок хлеба сливочным маслом, а сверху еще и
мёдом. В тот период нашей  жизни, была карточная система  и кусок хлеба с
маслом и с медом были для меня большим удовольствием, чем «Киевский Торт» за 3
рубля до распада СССР.

– Наступила весна 1945 года. 8 мая 1945 года Главный диктор Всесоюзного радио,
Юрий Левитан объявил по радио об окончании войны и о Победе над фашистской
Германией. Ю. Левитан всю войну читал по Всесоюзному радио сводки
Совинформбюро. Его мощный и красивый голос был знаком каждому Советскому
человеку.  Гитлер обещал повесить Левитана, после взятия Москвы.. Когда мы
услышали  о том, что Германия подписала Договор о Капитуляции, - все ликовали
от радости. И мы – мальчишки, целыми днями гоняли по улицам и кричали: “Ура-а-
а!”

– Никаких детских игр, как сейчас, тогда не было! Мы играли или в  "жмурки" или
в “цурки-палки” или в подкидного дурака с картами. Мы прятались по дворам во
все дыры, а кто-нибудь из детей нас искал. А когда он заходил за угол, мы
летели со всех ног, не видя перед собой никого и ничего, и орали во всё горло:
“Тра-та-та за себя!” Соседи ругались и требовали от нас, чтобы мы не кричали,
но нам было не до них. Ну а цурки-палки требовали от нас большой сноровки и
выносливости.

Один мальчик выкапывал небольшую ямку и ложил поперёк неё палочку длиной 15-20
см. Затем  палкой, длиной 60 см. он упирался в землю и сильно запускал
маленькую палку в воздух, чем подальше вперед. А другой мальчик должен был
после пуска,  бежать вслед за летящей палочкой и кричать не передыхая:

– “Цурки-палки на кувалки на кули-и-и-и-и-и-и!”

– И не передыхая, добежать до упавшей палочки, поднять её и так же бегом, пока есть воздух, вернуться на место. 

– У меня было любимое занятие катать колесо по улицам города. Я где-то нашёл
металлическое колесо, сделал себе из проволоки правилку и бегал босиком по
всему городу, толкая перед собой это колесо. Это была моя машина и я был
счастлив управлять ею. 

– Я, как и другие дети, избегал дружбы с хулиганами, хотя был со многими
знаком. Помню, как родители советовали своим детям: «Дружи только с Леней!» Ни
мама, ни папа не знали, где я пропадал целыми днями. Я не был беспризорным, но
я был совершенно свободен и делал то, что хотел. Ни отец, ни мать — никогда не
делали мне замечаний и они были уверены в том, что я не попаду в плохую
компанию.

По всему Киеву работали пленные немцы. Они были одеты в зелёную немецкую
военную форму, а мы – дети их ненавидели. Я до сих пор не могу терпеть заленый
цвет одежды. Иногда пленные немцы подходили к нам и предлагали поменять
что-нибудь на соль. (Предлагали немецкие перочинные ножички, карманные
фонарики с 3-х цветными светофильтрами и пр.) Я с ними никогда не имел никаких
дел, но некоторые ребята менялись. 

– Наступила осень 1945 года и меня отвели в школу. Я пришел в школу с
перебинтованной головой, потому что незадолго до этого нечаянно стукнулся об
пожарную лестницу в нашем дворе и пробил себе лоб. Меня определили в 1-й класс
91-й мужской средней школы гор.Киева. Она находилась на Чеховском переулке и от
нашего дома нужно было идти пешком до неё 15 мин мимо огромного и красивого
здания штаба военно-воздушных сил Киевского военного округа. Там всегда
дежурили солдаты и они не разрешали вблизи Штаба даже останавливаться, а на
круглом куполе крыши висел красный флаг.

– Я проучился в 1-м классе всего один день,  на следующий день меня перевели
сразу во 2-й класс. Я был очень маленький и поэтому меня посадили на первую
парту вместе с Мариком Островским и с этого момента, мы с ним дружим до сих
пор.  Марик был очень весёлый и смешной. Он такое вытворял в классе, что вся
школа ходила ходуном. Нет, он никогда не хулиганил, но очень любил всякие
клоунские выходки.

– У нас в школе в то время не было никаких столовых,  но в буфет привозили
какие-то бутерброды. Мы их не покупали. Не было денег. Машин тоже не было, и
продукты привозили на телеге, запряженной старой лошадью. Однажды, Марик сказал
мне, что будет показывать цирковой номер. Он подошел во время переменки к этой
лошади, разжал руками ей пасть и стал всовывать свою голову ей в рот. Не знаю,
чем бы это закончилось, но извозчик схватил кнут и заорал на нас так, что мы
испугались и убежали.

– Прошел год, я перешел в 3-й класс. На лето меня отвезли в пионерский лагерь.
Он находился под Киевом в пригородном живописнейшем лесу в Клавдиево. Мы ездили
туда пригородным поездом на паровозной тяге. Я помню как у меня радовалась
душа, когда я смотрел в окно поезда на проплывающие мимо меня поля и думал:
“Какая у нас бескрайняя и красивая земля!”

– Мы жили в военных палатках по 25 человек в каждой. Каждый день у нас в
пионерлагере была утренняя и вечерняя линейка, под музыку поднимали и опускали
лагерный флаг. У нас проводили военные игры, мы бродили по лесам, разыскивали
секретные записки, стрелки на песке и прочее. Нас хорошо и сытно кормили, а наш
повар каждый раз придумывал нам какие-то сюрпризы. Например, в день, когда
давали на обед вареники с мясом, кому-то обязательно попадался вареник с перцем
или с солёным огурцом. Дети смеялись и кричали от восторга. Мне в лагере было
очень хорошо.

– Многие дети в те послевоенные годы были истощены и родители старались их
как-то укрепить. Поэтому, договаривались в Клавдиево с местными жителями, чтобы
дети приходили вечером к ним домой пить “парное” молоко. Оно было тёплое,
только что из-под коровы и многие дети его не переносили. Иногда, мои друзья
брали меня с собой и я тоже пил такое молоко за кого-то. Мы жили очень бедно и
у нас не было денег на такие мероприятия. Я отдыхал в этом лагере 6 лет подряд.
И я считал этот пионерский лагерь своим вторым домом.

– Пионервожатые мне доверяли и однажды послали меня на молочную ферму привезти
творог и сметану. Мне дали двухколёсную тачку, два пустых бидона, я запряг себя
вместо лошади и с удовольствием поехал выполнять поручение. На обратном пути, я
решил попробовать сметану. Остановился посреди просёлочной дороги, поставил
оглобли на землю и стал двигать по площадке деревянной тачки бидон со сметаной,
открыл крышку и стал думать, как же мне её попробовать? Тачка была на двух
колёсах и я доигрался  до того, что равновесие нарушилось и бидон с творогом
упал на землю. Творог высыпался на песок и я очень испугался. Я понимал, что
назад его не положить, потому что он в песке. А того количества, которое
осталось в бидоне, могло не хватить на всех. Как-то обошлось без наказания, но
случай этот я запомнил на всю жизнь и сделал для себя вывод:

– “Не тронь чужое — вылезет боком!” 

– В лагере у меня было много друзей с которыми я иногда встречался и после
закрытия сезона пионерских лагерей. Осенью, когда дети начинали учиться, мы
встречались в Киеве по выходным дням на площади Ленинского Комсомола возле
здания Киевской филармонии у входа в Первомайский парк у фонтана. Там всегда
было много детей со всех Киевских пионерских лагерей. Там всегда стоял шум от
детских голосов, машин, трамваев и потоков падающей воды.

– До 7-го класса, я учился в 91-й Киевской школе, которая находилась на
Чеховском переулке. Наша школа была огромная. Она находилась в старом довоенном
5-ти этажном здании, с высокими потолками. На каждом из 4-х этажей был огромный
вестибюль, а по периметру располагались классы.

– Почти все наши учителя в то время были мужчины – фронтовики. И только
«англичанка» и «русачка» – были женщины. В нашей школе было паровое отопление.
Топили углём в подвале. И каждый год, на зиму завозили целую гору
угля-антрацита. Гора была высотой до 2-го этажа и находилась прямо под окнами
нашего класса.

– Нашу учительницу русского языка звали Евгения Маркияновна. Она была очень
худая и старая женщина. Дети дали ей прозвище “Кляча”. Марик её не любил, и
поэтому делал ей всякие пакости.

– Однажды, он положил на пол под ногу граненный карандаш и стал катать его
ногой по полу. Полы были деревянные и издавали противный звук. Марик сидел на
парте, стоящей у окна, расположенного как раз под кучей угля. Он умышленно
действовал на нервы нашей "Кляче" до тех пор, пока она не обнаружила кто это
делает и не крикнула ему:

– “Островский! Вон из класса!”.

– Марик вышел в дверь, оттуда выбежал во двор, оббежал вокруг дома, влез на
кучу угля и когда учительница отвернулась к доске, быстро влез через окно в
класс и сел на своё место. Когда Евгения Маркияновна снова повернулась, к нам
лицом и увидела его,  она закричала на него:

– “Островский! Почему ты опять здесь? Я же тебя выгнала!  Вон из класса!”

– Он опять вышел в дверь, но через 5 мин. снова сидел за своей партой.

После 7-го класса нас перевели в новую отремонтированную школу №54, которая
находилась рядом с велотреком и в нее можно было попасть или с улицы Липинского
через забор во дворе дома №11, или с ул. Богдана Хмельницкого, через проходной
двор. Окна школы выходили прямо на велотрек и поэтому  на асфальте трека можно
было увидеть разбитые чернильницы, которые швыряли дети через  школьные . Тогда
все писали в школах только перьевыми ручками. А наш двор находился в двух шагах
от велотрека и мы там летом играли в футбол, зимой там заливали каток, а весной
все таяло и получалось целое море воды.  Мы находили выброшенные на мусор
большие миски или корыта, садились в них и плавали на воде, как в Венеции. Вход
на велотрек был свободный.

– Все наши учителя были уникальными людьми. Географию – преподавал учитель,
которого звали Сергей Лукич. Это был человек огромного почти 2-х метрового
роста, который всегда ходил в военной форме, в брюках-галифэ, сталинском
пиджаке-френче и хромовых сапогах. Он прошёл всю войну и был контужен. Его
нельзя было нервировать – он мог и ударить. Сергей Лукич заставлял нас учить
географию на-зубок. У него была палка-указка 3-х метровой длины. Он ставил
ученика в 4-х метрах от географической карты Мира, тыкал указкой в какую-
нибудь точку и требовал точно ответить, что это такое. Кто не слушался и плохо
успевал по географии имел большие неприятности. С таким учителем, мы знали
географию лучше старшеклассников. 

– В нашем классе учился один мальчик по фамилии Брыскин. Однажды, он своим
поведением довёл Сергея Лукича до бешенства. Учитель  рассвирепел, схватил
Брыскина рукой за шиворот, подтащил к двери и дал ему сапогом такого пинка
под-зад, что тот пулей вылетел из класса. В полёте, своим телом он открыл дверь
и пролетев еще 5-7 метров рухнул на пол.

– А другой наш ученик – Лёвка Казанджи, однажды написал себе шпаргалку на
ладони.  Сергей Лукич вызвал его к доске и стал спрашивать, а когда  увидал,
как Лёвка что-то подсматривает, схватил его за эту руку и поднял его так
высоко, что ноги повисли в воздухе. Он держал его в таком положении до тех пор,
пока не прочёл ему мораль и пока Лёвка не поклялся, что останется после уроков
и будет учить все что задано.

– Наш учитель истории – Зиновий Александрович Полонский, бывший боевой
лётчик-истребитель также ходил по школе в военной форме. Когда он в своих
скрипучих и всегда блестящих сапогах двигался по коридору школы, дети бежали
впереди него и кричали “Воздух !”.

На фронте, это означало, что начался налёт вражеской авиации. По этой команде
нужно было падать на землю и прятаться, но в школе – это означало, что нужно
пулей лететь в класс и садиться за парты. Зиновий Александрович был очень
строгий, но очень хороший умный человек.

У него всегда и во всём был порядок. И в его внешнем виде и в его отношении к
нам. Кроме классного журнала, он вёл специальный блокнот, в котором отмечал
когда и по какой теме он спрашивал ученика. Я историю не любил, но его блокнот
помогал мне в нужное время хорошо и своевременно подготовиться к ответу. 

Был у нас и физик  Сметанин Дмитрий Арсентьевич. Благодаря ему, я очень люблю
физику всю свою жизнь. Он тоже был требовательным учителем. Ходил по классу с
линейкой длиной в 1 метр и если кто-то клал руки на парту — бил по рукам.

И наконец, наш математик — фронтовик-артиллерист. Тоже прекрасный человек и
мне перед ним иногда даже было стыдно за то, что я  срывал его уроки ради
своего престижа в классе. В те годы дома у нас не было никаких развлечений.
Поэтому, мы записывались в различные кружки. Мне нравилась техника и я ходил
на Станцию юных техников в Первомайский парк. Посещал кружки по
авиа-моделированию и  фото-кружок. А в ДОСААФЕ учился азбуке МОРЗЕ. Заодно я
очень любил посещать читальный зал библиотеки КПСС. Тогда это был аналог
Интернета. Там можно было посмотреть и почитать любые журналы и книги. И меня
очень заинтересовали математические несуразицы.

Например, доказательство того, что прямой угол, равен острому. Я тщательно
изучил этот абсурд и когда мои одноклассники попросили меня сорвать урок по
математике, я предложил учителю рассмотреть это на уроке. Он разрешил мне
выйти к доске, где я представил доказательство о том, что прямой угол равен
острому и мы два урока искали ошибку в доказательстве.  

В 1954 году мы закончили 10 классов, получили Аттестаты Зрелости и  практически
расстались с нашим детством навсегда!  Хорошее было время... Как интересно было
жить...

\ifcmt
tab_begin cols=3

  pic https://scontent-bos3-1.xx.fbcdn.net/v/t1.6435-9/191189417_3923583354355481_5939010068840798230_n.jpg?_nc_cat=100&ccb=1-3&_nc_sid=b9115d&_nc_ohc=zwMhiavZiL4AX_gVxFr&_nc_oc=AQl2mAI5zcWz23UE9UzcRaE9ifRAg8ESDhiKF4kx-UrKZuWiWt13-qDwHi0BK3F0LKI&_nc_ht=scontent-bos3-1.xx&oh=06bcc99f09d93dec15b6f6153896bf85&oe=60D3FD5A
  width 0.2

  pic https://scontent-bos3-1.xx.fbcdn.net/v/t1.6435-9/190772804_3923584601022023_7042507841366314781_n.jpg?_nc_cat=110&ccb=1-3&_nc_sid=b9115d&_nc_ohc=fMkRFnwtID0AX_Quplq&_nc_ht=scontent-bos3-1.xx&oh=2cc8ba7c2c0554ab481aa9e7eba91dfb&oe=60D2826F

  pic https://scontent-bos3-1.xx.fbcdn.net/v/t1.6435-9/191762343_3923589184354898_8160315977957062527_n.jpg?_nc_cat=101&ccb=1-3&_nc_sid=b9115d&_nc_ohc=6e2nwH9C6i0AX8Rdp0d&_nc_ht=scontent-bos3-1.xx&oh=388041ee23e8170f8027c4a0c0f9d491&oe=60D3CE5E

  pic https://scontent-bos3-1.xx.fbcdn.net/v/t1.6435-9/192306752_3923589754354841_8808956094869205562_n.jpg?_nc_cat=106&ccb=1-3&_nc_sid=b9115d&_nc_ohc=00klNHUpVG0AX-muRZT&_nc_ht=scontent-bos3-1.xx&oh=aaf1bff57db7a22d78a23ac69f489b34&oe=60D0ED73

  pic https://scontent-bos3-1.xx.fbcdn.net/v/t1.6435-9/192176562_3923591367688013_5907645955406709877_n.jpg?_nc_cat=104&ccb=1-3&_nc_sid=b9115d&_nc_ohc=ZlJaPd1dUd8AX_uDoMk&_nc_ht=scontent-bos3-1.xx&oh=3484d26b632d4cae81c98f3fcafd3332&oe=60D4A7C6

  pic https://scontent-bos3-1.xx.fbcdn.net/v/t1.6435-9/191027585_3923594621021021_7952614282030170009_n.jpg?_nc_cat=105&ccb=1-3&_nc_sid=b9115d&_nc_ohc=SsGtjLFkUmQAX_YwdQ_&_nc_ht=scontent-bos3-1.xx&oh=de16ea6a9167fe84ef3cfe0fd99c7b80&oe=60D16D4A

tab_end
\fi

