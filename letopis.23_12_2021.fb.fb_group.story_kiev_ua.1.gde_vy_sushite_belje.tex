% vim: keymap=russian-jcukenwin
%%beginhead 
 
%%file 23_12_2021.fb.fb_group.story_kiev_ua.1.gde_vy_sushite_belje
%%parent 23_12_2021
 
%%url https://www.facebook.com/groups/story.kiev.ua/posts/1824968094366650
 
%%author_id fb_group.story_kiev_ua,sobolevskaja_marina.kiev
%%date 
 
%%tags kiev,kievljane
%%title ГДЕ ВЫ СУШИТЕ БЕЛЬЕ?
 
%%endhead 
 
\subsection{ГДЕ ВЫ СУШИТЕ БЕЛЬЕ?}
\label{sec:23_12_2021.fb.fb_group.story_kiev_ua.1.gde_vy_sushite_belje}
 
\Purl{https://www.facebook.com/groups/story.kiev.ua/posts/1824968094366650}
\ifcmt
 author_begin
   author_id fb_group.story_kiev_ua,sobolevskaja_marina.kiev
 author_end
\fi

\begin{multicols}{2} % {
\setlength{\parindent}{0pt}

ГДЕ ВЫ СУШИТЕ БЕЛЬЕ?

Или, как говаривал мой папа:

- «Где вы сохнете белье?

- В духовке на веровке, чтоб не свиснули воровки.»

Вот интересно! Действительно ли в старину за бельем на свежем воздухе надо было
присматривать? Тырили чистенькие панталоны?

.

Как же сушили белье в образцово – показательной столице? Лице и эмблеме нашей
республики. 

В нашем новом девятиэтажном доме цивилизованная сушка белья предусматривалась
во дворе. На квадратной асфальтированной площадке, укрепленные в бетоне,
торчали четыре железные штанги в форме удлиненных грабель, выстроенных попарно.
На каждый зубец грабель жительницы дома наматывали свои бельевые веревки. С
веревками приходили, и с веревками уходили. 

Такие облагороженные дворовые сушки встречались по всему Киеву. Остальные
жители сушили белье, растягивая веревки прямо под окнами, между стволов
деревьев или кустов.

Пару раз сушили на нашей площадке, на солнце, и мы с мамой. Прищепки собираешь
на веревочку на шее. Свежее белье, пахнущее ветром, складываешь в тазик.
Полюбуешься на лето вокруг, вдохнешь запахи листвы и травы, послушаешь птичьи
чириканья, и домой. 

\ii{23_12_2021.fb.fb_group.story_kiev_ua.1.gde_vy_sushite_belje.pic.1}

\ii{23_12_2021.fb.fb_group.story_kiev_ua.1.gde_vy_sushite_belje.pic.2}
\ii{23_12_2021.fb.fb_group.story_kiev_ua.1.gde_vy_sushite_belje.pic.2.cmt}

\end{multicols} % }

.

В 80-ые годы все меньше и меньше встречалось белье во дворах. В продаже
появились стальные сушки – раскладушки. Они прикручивались на стену в ванной,
выдвигались вперед по необходимости, и для хранения задвигались к стене.

Особенно хозяйственные жильцы прибивали над ванной самодельные сушки -
алюминиевые уголки с отверстиями для веревки. И наслаждались собственной
сушилкой в отдельной квартире.

.

Была и у нас такая сушка- гармошка, которую папа прикрутил самостоятельно над
ванной. Однажды в снежном феврале к нам в гости приехал издалека один
высокопоставленный мужчина, настолько высоко поставленный, что его мама летала
в Париж к парикмахеру. Сейчас может это не диво, а в советское время это было
редкостью. Мне представлялось, что у них дома техника – фантастическая, сама
стирает, сушит, гладит, раскладывает вещи по цветам, чтоб уголок к уголку.
Глянул он на нашу гордость, новенькую желто – голубую чудо сушку и говорит мне: 

- А почему ты не сушишь белье на открытом балконе? На морозце! Попробуй! Как
оно вкусно будет пахнуть, когда ты станешь его надевать! Такой свежестью!
Никакие духи не сравнятся!

.

С сушкой белья на советских балконах были и курьезы. Во время государственных
парадов и шествий вывешивать свои богатства на публичный обзор строго
запрещалось. А если, не дай Бог, живешь на стратегической улице, и рискнешь на
личном балконе прищепнуть на собственную веревку персональные выстиранные трусы
– тянуло на ЧП районного масштаба. По квартирам города ходили дружинники и
заставляли хозяек снимать все белье с балконов, что виднелось выше ватерлинии.

.

Прошло еще несколько десятилетий. И сушить белье в Киеве на балконе стало
небезопасно. Особенно в местах активного движения городского и частного
транспорта. То есть сушить его можно было. Повесите его чистым, простиранным
лучшим порошком, отбеленный помощницей Белизной. После сушки его ожидает своя
полка, на которой в собственноручно вышитом мешочке пахнет сухая лаванда. Но
нет. Снимете его вы почерневшим от слоя сажи. Так однажды жаловалась старинная
подруга моей свекрови, живущая на площади Победы.

.

Осенью, в воскресный день решили мы с семьей прогуляться в Мамаевой слободе.
Обошли все знакомые тропки, покормили толпу скачущих рыб в заросшем пруду,
подкрепились горячим сытным кулешом. Выходим по пути к машине на городскую
улочку. Гляжу – не может быть! Как много лет назад! Давно знакомые веревки
натянуты между деревьями. А с них свисают уютными флагами чей - то
пододеяльник, умудренная опытом простынка, пара цветных кофточек, любимые в
семействе полотенца, суровые мужские брюки. И так пахнуло детством! Живы еще
давние киевские традиции!  

.

А у папиного народного стишка, оказалось, есть продолжение:

«Здрасьте Вам через окно,

Где вы сохните белье?

На вировке, у духовки,

Шоб не слямзили воровки...

И чуть - чуть на чегдаке,

Было шоб на ветерке.»

На ветерке – до сиз пор это самая вкусная сушка!

\ii{23_12_2021.fb.fb_group.story_kiev_ua.1.gde_vy_sushite_belje.cmt}
