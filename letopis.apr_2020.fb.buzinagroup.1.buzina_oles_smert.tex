% vim: keymap=russian-jcukenwin
%%beginhead 
 
%%file apr_2020.fb.buzinagroup.1.buzina_oles_smert
%%parent apr_2020
 
%%url 
 
%%author 
%%author_id 
%%author_url 
 
%%tags 
%%title 
 
%%endhead 
\subsection{16 апреля 2015 г. оборвалась жизнь украинского писателя, историка, журналиста, телеведущего Олеся Бузины}
\label{sec:apr_2020.fb.buzinagroup.1.buzina_oles_smert}
\url{https://www.facebook.com/groups/buzinagroup/permalink/441318376683278/}

\ifcmt
  pic https://scontent-amt2-1.xx.fbcdn.net/v/t1.6435-9/93193544_228946668192983_3384402154432233472_n.jpg?_nc_cat=105&ccb=1-3&_nc_sid=825194&_nc_ohc=cJtZhmy3G0AAX-70Wkt&_nc_ht=scontent-amt2-1.xx&oh=276e1ab55284a28d0943a3d20b02551f&oe=60914F32
  width 0.4
\fi

Он, без преувеличения, был замечательным человеком. Великий интеллектуал
Украины, Олесь Бузина, благодаря своему креативному уму, масштабности,
проницательности и глубине своего мышления, мог, не взирая на авторитеты,
высказать свои суждения и дать, часто кардинально противоположную (официозу),
оценку и свое видение тех или иных проблем. Он смог переоценить часто
десятилетиями устоявшиеся исторические и литературные штампы в отношении как
значимых исторических событий, так и известных персоналий. Олесь Бузина мог
интуитивно прочувствовать тенденции развития и предсказать последствия тех или
иных, происходящих в стране, событий. Это был величайший дар, основанный на
глубоком знании и понимании жизни. Торжество этих высказываний подтверждает
сегодня сама жизнь. Отсюда проистекает широкая амплитуда в подходах к его
личности и неоднозначность оценок его исторического и литературного наследия.
Многие его оппоненты просто интеллектуально не могли в его время( и сегодня!)
приблизиться к нему, а, следовательно, понять. Иные, видя свою полную не
способность и бездарность осилить формат его личности, не тривиальность и
глубину суждений, изложенных в произведениях, просто дико его боялись, а
поэтому, с высоты своего интеллекта, пытались критиковать. Их ограниченность и
зашоренность в оценках, по сравнению с Олесем, скудоумие, рождали зависть,
которая перерастала в агрессивность, а порой в осуждение. Известно, что серость
всегда агрессивна! Это аксиома! Вот и с ним поступили очень подло и, до сих
пор, безнаказанно! Чтобы очиститься от этого позора и предупредить повторение
подобного в будущем, новая власть должна, восстановить справедливость, а суд
должен наказать виновных. Это же нонсенс. В 21 веке в центре Европы произошло
такое преступление, а дело не доведено до приговора. К сожалению, так всегда
бывает с Великими людьми в истории, пока народ осознает, полностью поймет и
оценит их роль в жизни общества, проходит определенное, порой длительное,
время! Причина тому - величие и значимость этих людей для своего Отечества.
Именно этот процесс полностью применим к творчеству Олеся Бузины. Суть в том,
что чем больше времени проходит со дня написания его произведений, тем более
понятными, привлекательными, а во многом пророческими, становятся высказанные
им идеи и суждения. Кажется, как это возможно? В этом и состоит величие и
значимость гениальных личностей в истории. Они способны предсказать ход и
содержание предстоящих событий. На примере жизни Олеся Бузины надо учить детей
в школах. Душа радуется за то, что народ Украины с каждым днем все больше и
больше осознает величие и значимость подвига этого бесстрашного и
замечательного Человека: Писателя, Историка, Патриота своей страны. Безусловно,
его произведения, со временем, будут включены в школьные программы.
Справедливость будет восстановлена.Народ никто не сможет постоянно держать в
системе ложных координат и иллюзорных схем. Именно граждане Украины воздадут
должное Олесю Бузине, определив его достойное место в истории нашей страны.
