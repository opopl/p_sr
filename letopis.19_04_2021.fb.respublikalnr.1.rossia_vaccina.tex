% vim: keymap=russian-jcukenwin
%%beginhead 
 
%%file 19_04_2021.fb.respublikalnr.1.rossia_vaccina
%%parent 19_04_2021
 
%%url https://www.facebook.com/groups/respublikalnr/permalink/798773817425078/
 
%%author 
%%author_id 
%%author_url 
 
%%tags 
%%title 
 
%%endhead 

\subsection{Газета \enquote{Республика} (№15, 2021г).  Караул.  О чем молчат украинские СМИ}
\Purl{https://www.facebook.com/groups/respublikalnr/permalink/798773817425078/}


\ifcmt
  pic https://scontent-mad1-1.xx.fbcdn.net/v/t1.6435-9/175969822_124057766440998_354973179941880560_n.jpg?_nc_cat=105&ccb=1-3&_nc_sid=825194&_nc_ohc=0f_ZmJw5usQAX8qEZk0&_nc_ht=scontent-mad1-1.xx&oh=55c3b77c9f9c9a0ba49ab1b48e07aa0e&oe=60A13D18
\fi


Заседание СНБО, национальный план вакцинации, изучение международного опыта и
поиски вакцины. Нет, это было не в прошлом году и даже не в начале 2021 года.
Украина спохватилась и стала искать способы остановить эпидемию COVID-19 только
сейчас. 

Между тем, суточный прирост заболевших превысил отметку в 20 тысяч граждан.
Гарантируем, что после пасхальных каникул, когда заробитчане вернутся на свою
вторую родину, поляков ждут неприятные сюрпризы.

На сегодняшний момент на Украине привито немногим более 100 тысяч граждан. Если
темпы вакцинирования не увеличить, то коллективный иммунитет выработается не
ранее 2027 года. Именно поэтому министра здравоохранения всё ещё держат на
своём посту: весь негатив по коронавирусу «повесят на него», но уволят не
ранее, чем через 3–4 месяца. Весной у людей ослаблена иммунная система, что
гарантированно обеспечит резкий всплеск заболевания. А если учесть любовь
украинцев к пикникам на природе – то заболеет немало и молодых, социально
активных людей. 

Тема заболеваемости коронавирусной инфекцией и провала кейса вакцинирования
сейчас находится под строгим запретом во всех украинских медиа. Кстати,
любители почитать украинские телеграммчики, обратите внимание – вопросы
COVID-19 поднимаются у якобы «оппозиции» минимально. Указание о запрете было
спущено из офиса президента и удивительно, как «ответственно» его бросились
выполнять даже «инакомыслящие». 

Министр здравоохранения Степанов недавно отрапортовал о якобы 10 миллионах доз
вакцины «Pfaizer», которая в скором времени поступит на Украину. Однако пока
подписано только рамочное соглашение о намерениях, и даже речи нет о покупке
столь масштабной партии. Да и денег на дорогую вакцину в бюджете нет – все были
украдены через коррупционную схему «Большая стройка». 

«Страна-бензоколонка» – как привыкли называть Россию либералы и бандеровцы –
сегодня является мировым лидером по производству вакцин. В Центре им. Гамалеи
российскими учёными создано мощное оружие против микобактерий – возбудителей
туберкулеза. Новая вакцина – не замена созданной 100 лет назад вакцины БЦЖ, а
дополнение к ней. Теперь эффективность БЦЖ будет усилена. 

Эта по-настоящему сенсационная новость не появилась на первых полосах мировых
СМИ. Хотя вполне объяснимо: Европа ещё не пережила шок от информации, что
российская вакцина «Спутник V» – лучшая и при этом абсолютно безопасная. Как бы
ни сопротивлялись власти стран Евросоюза вакцинации именно российским
продуктом, давление общественности рано или поздно, но заставит прекратить
обслуживание интересов «большой фармы».

Игорь МИРТОВСКИЙ, ГАЗЕТА "РЕСПУБЛИКА" (№15, 2021г).
\verb|#газета #республика #политика_Украины #вакцинация|
