%%beginhead 
 
%%file 20_11_2021.fb.fb_group.mariupol.nekropol.1.otchet_volonter_rabota_13_14_nojabrja
%%parent 20_11_2021
 
%%url https://www.facebook.com/groups/278185963354519/posts/641477320358713
 
%%author_id fb_group.mariupol.nekropol,marusov_andrij.mariupol
%%date 20_11_2021
 
%%tags 
%%title Отчет о волонтерской работе в Некрополе 13-14 ноября 2021 г.
 
%%endhead 

\subsection{Отчет о волонтерской работе в Некрополе 13-14 ноября 2021 г.}
\label{sec:20_11_2021.fb.fb_group.mariupol.nekropol.1.otchet_volonter_rabota_13_14_nojabrja}
 
\Purl{https://www.facebook.com/groups/278185963354519/posts/641477320358713}
\ifcmt
 author_begin
   author_id fb_group.mariupol.nekropol,marusov_andrij.mariupol
 author_end
\fi

\textbf{Отчет о волонтерской работе в Некрополе 13-14 ноября 2021 г.}

Прошедшие выходные оказались плодотворными: мы вернули из забвения нескольких
мариупольцев, обнаружили старинную плиту и продвинулись в очистке от зарослей
Платановой аллеи...

**Открытия и находки **

*Никита Васильевич Ковалевский *(1851-1910) был гласным Мариупольской городской
думы в 1906-10 годах, заведовал Земской почтой, был бессменным членом
Попечительского совета Мариупольской Мариинской гимназии...

В зарослях мы обнаружили могильную плиту его четырех детей, умерших в 1885-1888
годах (подробнее о находке см. https://cutt.ly/pTUqggt). А вот где находится
само захоронение – неизвестно, ибо плиту явно передвигали. Мы также не знаем,
где находится могила самого Никиты Ковалевского...

На участке перед усыпальницей Пиличева-Оксюзовой волонтеры вернули из забвения
священника Василия Петровича Ножникова (Пичахчи) (1849-1908). Он служил в
соборной Харлампиевской церкви, Георгиевской церкви Сартаны. Его братья были
известными мариупольскими врачами (подробнее см. https://cutt.ly/7TUy1DL).

Неожиданной находкой стала старинная плита возле усыпальницы купцов Егоровых.
Дело в том, что \enquote{купеческие кварталы Некрополя} (у стен нынешнего
\enquote{Электробытприбора}) стали формироваться на рубеже XIX-XX вв. А плита, похоже,
относится к более раннему времени. К сожалению, на ней отсутствуют надписи...

**Уборка и благоустройство **

Мы значительно продвинулись в очистке Платановой аллеи. Наша задача –
освободить обе стороны аллеи от хаотических зарослей хотя бы на том участке,
где мы посадили двадцать саженцев платанов. Надеемся ее выполнить в течение
ближайших двух недель, если позволит погода.

\textbf{Благодарность благотворителям и волонтерам}

За этими несколькими абзацами отчета – бесценная помощь благотворителей и самоотверженный труд волонтеров.

Der Fliegender Holländer привез редуктор – и мы сумели поднять и исследовать плиту детей Ковалевского за полчаса вместо обычных двух-трех.

Slava Yaroslavskiy подарил волонтерской команде канистру бензина – и уборка
Платановой аллеи состоялась! Кстати, Слава был участником самой первой
волонтерской экспедиции в Некрополь – весной 2020 г. 🙂

Спасибо, друзья!

Наконец, все достижения стали возможными благодаря труду Юра Таран, Сергея Катрича, Славы Ярославского, Андрея Марусова.

Не менее важной была и исследовательская работа – Сергей Катрич \enquote{откопал}
информацию о Никите Ковалевском и его детях, Андрей Марусов – об иерее Василии
Ножникове...

Приглашаем в команду всех желающих!

И - до встречи в ближайшие выходные, друзья!

\#mariupol\_necropolis\_report
