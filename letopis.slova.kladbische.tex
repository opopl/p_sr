% vim: keymap=russian-jcukenwin
%%beginhead 
 
%%file slova.kladbische
%%parent slova
 
%%url 
 
%%author 
%%author_id 
%%author_url 
 
%%tags 
%%title 
 
%%endhead 
\chapter{Кладбище}

%%%cit
%%%cit_head
%%%cit_pic
%%%cit_text
Мій досвід показує, що найважливішими точками в містах є не центральна площа,
замок чи пам'ятник, а базар і \emph{цвинтар}. Саме вони можуть найбільше розказати нам
про історію й повсякденне життя міста. Не завжди в цих локаціях ви зможете
зробити красиві фото, але саме там отримаєте шанс не тільки побувати, а й
зрозуміти місто, відчути його приховані пульси і норов. Адже базар і \emph{цвинтар} –
це як ерос і танатос населеного пункту, десь саме між ними й треба шукати його
серце, - пише Андрій Любка для видання "День".
\emph{Цвинтар} – найспокійніше місце, навіть місцеві намагаються обходити його
стороною, призначаючи в році всього кілька днів для обов'язкових відвідин. Але
саме там розташоване найточніше управління статистики, в якому ви зможете
довідатися про майновий стан жителів, про структуру населення, наявність
національних меншин і різноманіття релігій, про суспільно-політичні настрої –
кого тут поважають і ховають на центральній алеї, а кому відведено місце на
периферії. Саме \emph{цвинтар} є розкритою перед вами історичною хронікою міста,
найточнішим літописом
%%%cit_comment
%%%cit_title
\citTitle{Базар і цвинтар – найважливіші точки в місті. Між ними й треба шукати його серце}, 
Андрій Любка, gazeta.ua, 26.07.2021
%%%endcit
