% vim: keymap=russian-jcukenwin
%%beginhead 
 
%%file 31_12_2020.fb.bilchenko_evgenia.1.nauchnyj_otchet_za_god
%%parent 31_12_2020
 
%%url https://www.facebook.com/yevzhik/posts/3518721644829587
 
%%author Бильченко, Евгения
%%author_id bilchenko_evgenia
%%author_url 
 
%%tags bilchenko_evgenia,chelovek,god,nauka,novyj_god,otchet
%%title БЖ. Научный отчет за год
 
%%endhead 
 
\subsection{БЖ. Научный отчет за год}
\label{sec:31_12_2020.fb.bilchenko_evgenia.1.nauchnyj_otchet_za_god}
\Purl{https://www.facebook.com/yevzhik/posts/3518721644829587}
\ifcmt
 author_begin
   author_id bilchenko_evgenia
 author_end
\fi

БЖ. Научный отчет за год.

Я очень долго молчала в тряпочку, но почему я должна терпеть? Я же - не
терпила, да и пассивную толерантность с христианским ненасилием не путаю. Кроме
того, мне кажется, что люди должны знать в Украине, в какой стране они живут, а
в России - относительно какой страны они еще питают иллюзии, полагаясь на
эфемерные перспективы диалога культур - в науке, поэзии, человеческих
отношениях. В общем, меня как научной единицы в этом году... не существует.

\ifcmt
  pic https://scontent-lga3-2.xx.fbcdn.net/v/t1.6435-9/134088617_3518723798162705_1100223897275847047_n.jpg?_nc_cat=104&ccb=1-3&_nc_sid=730e14&_nc_ohc=ffftwcB3ltMAX-AOKim&tn=ntrKbsW_7ChXu3v-&_nc_ht=scontent-lga3-2.xx&oh=f0da05080ec7ce584e320a5ce266c0ab&oe=60CB8EC4
\fi

Ниже на украинском прилагается отчет о моих научных достижениях за год, который
вычистили в институте до минимальной информации, превратив меня из мыслителя в
серого рядового околонаучного сотрудника. Вычистили за публикации в российских
научных международных журналах. Но и этого им показалось мало. Второй месяц я
нахожусь под ежедневным политическим прессингом, на меня затеваются какие-то
\enquote{разоблачающие} статьи националистами. Мстят мне за то, что я НЕ ПРИЗНАЮ, что
великий древнегреческий историк Геродот называл \enquote{людоедами}... предков
современных россиян. Это всё делается тайно, в частных письмах, которые я
больше не могу читать, отвечая на \enquote{маркерные} вопросы политоты. Стримы и
международные лекции, как вы уже убедились на примере Одессы, тоже начали
срывать. Предпринимаются попытки запретить даже... мою европейскую марксистскую
школу и психоаналитическую критику идеологии в cultural studies вообще как
якобы \enquote{непрофильную}, При этом писать про ОУН-УПА и Бжезинского - это вполне
соответствует \enquote{профилю}. А теперь представьте, что это делается, пока мы с вами
еле-еле помогаем мне выпутаться из полного физического и морального угасания...
Да, и я ни капли не жалею за все, что написала в период заточения, я безумно
благодарна всем российским научным коллегам: эти статьи были истинным событием
в моей философской жизни в это нелегкое время, никакой цензуры и никакого
сковывания полета моей мысли НЕ БЫЛО. Наоборот, русские философы старой
академической школы только поощряли артикуляцию собственной позиции. Я горжусь
этим. Моим \enquote{оппонентам} будет интересно узнать, что за месяц травли, вышла моя
последняя статья о необходимости усиления возрождения памяти Победы в
Волгограде в очень уважаемом журнале РАН: вышла под свежим лакановским
ракурсом, как я хотела. Я шла к ней два года. Дальше - сухой отчет. Это просто
должен увидеть мир - это мое дитя после аборта. Оно никому не нужно здесь. Оно
хотя бы будет публично похоронено в сети.

БІЛЬЧЕНКО ЄВГЕНІЯ ВІТАЛІЇВНА.

1. Вийшли друком:
1.1. Монографії: 
1.1.1. Більченко Є. Суб’єкт. Символічне. Розрив: Монографія / Більченко Євгенія Віталіївна. – К.: Ін-т культурології НАМ України, 2020. – 304 с.
1.1.2. Бильченко Е. Сентиментальное насилие либерализма: от шока к китчу / Евгения Бильченко. – К.: Друкарский двор Олега Федорова, 2020. – 464 с. 
1.2. Статті Web of Science:
1.2.1. Данилова С. А., Бильченко Е. В. Имя литературного журнала как коммуникативный маркер // Вестник НГУ. Серия: История, филология. 2020. Т. 19, № 6: Журналистика. С. 205–215. DOI 10.25205/1818-7919-2020-19-6-205-215.
1.2.2. Бильченко Е.В., Данилова С.А. Поэт как субъект языка: онтологические основания знаков и трендовые значения социума / Евгения Витальевна Бильченко, Стефания Анатольевна Данилова // Филологические науки. Научные доклады высшей школы. – 2020. - № 6 (2). – С. 260-286.
1.2.3. Бильченко Э.В. Философия как модель диалога: концепт «смерти Отца» в коннотациях современной гуманитаристики // Вестник Томского государственного университета. - 2020. - № 459. – С.  68-79.
1.3. Міжнародні статті:
1.3.1. БИЛЬЧЕНКО Е.В.РЕКЛАМНАЯ ПАМЯТЬ КАК СМЕРТЬ
ПАМЯТИ: РЕКОНСТРУКЦИЯ ВЕЛИКОЙ ПОБЕДЫ КАК ИСТИНЫ-СОБЫТИЯ // История и современность. – 2020. - № 3. – С.3-22. 
1.4. Статті ВАК:
1.4.1. Більченко Є.В., Калита Т.В.  Структура сучасної культурології: семіотична тріада  // Культурологічна думка. - 2020. - № 18. – с. 55-64.
1.5. Збірники наукових праць:
1.5.1. Більченко Є. Суб’єкт, пам’ять, Символічне: екзистенційні суперечності // Людина. Екзистенція. Культура. Підхід філософської антропології як метаантропології: збірник наукових праць / За редакцією Н. Хамітова і С. Крилової. – К.: КНТ, 2020. – С. 220-226.
1.5.2. Данилова С.А., Бильченко Е.В. Феномен креатива в медиастратегиях поэтов XXI века // Полилингвиальность и транскультурные практики. 2020. Т. 17. № 2. С. 221—230. DOI 10.22363/2618-897X-2020-17-2-221-230
1.5.3. Більченко Є. Онтологічна природа сміху в дискурсі критичної теорії / Євгенія Більченко / Докса. Збірник наукових праць з філософії та філології. Вип. 1 (33). Сміх як інтелектуальна розвідка: семантика та герменевтика смішного – 1. – Одеса: Акваторія, 2020. – С. 8-17.
1.5.4. Більченко Є. Штучний інтелект і пустота: дигітальна культура очима скептика Людина і штучний інтелект: виміри філософської антропології, психоаналізу, арт-терапії та філософської публіцистики. Підхід філософської антропології як метаантропології: збірник наукових праць / За ред. Н. Хамітова і С. Крилової. – К.: КНТ, 2020. – С. 87-92. 
1.6. Тези доповідей:
1.6.1. Більченко Є. «Смерть батька» як філософсько-культурологічний концепт: психоаналітичний вимір медіа-паніки // «Київські філософські студії-2020». Матеріали Всеукраїнської наукової конференції з міжнародною участю 19 травня 2020 року. Тези доповідей, м. Київ, 2020. - С. 251-253.
1.6.2. Більченко Є. Поліщук Ю. Пастка підтексту: подвійне кодування динаміки і статики в інформаційному світі// «Київські філософські студії-2020». Матеріали Всеукраїнської наукової конференції з міжнародною участю 19 травня 2020 року. Тези доповідей, м. Київ, 2020. - С. 222-225.
1.6.3. Більченко Є. В.  Культурологічний аналіз гегемонії як символічної структури / Євгенія Більченко // Єдність навчання і наукових досліджень – головний принцип університету. Матеріали звітної науково-практичної конференції викладачів, докторантів та аспірантів факультету філософії та суспільствознавства (20-21 травня 2020 року) / ред. рада І. І. Дробот (голова) та ін. – Київ : Вид-во НПУ імені М. П. Драгоманова, 2020. – с. 27-29. 
1.6.4. Більченко Є.В. Багатоманітність творчих діалогів у дискурсі глобальної культури: екран і пам’ять // Діяльність продюсера в культурно-мистецькому просторі ХХІ століття: творчі діалоги. Зб. наукових праць / Наук. ред., упор. : С. Садовенко. Київ : НАКККіМ, 2020. – С. 26-36. 
2. Взято участь у роботі конференцій: 
2.1. Науковий круглий стіл «Діалог культур у процесі гармонізації постсучасного світу» на базі Інституту культурології Національної академії мистецтв України 28 квітня 2020 р. Тема доповіді Більченко Є.В.: «Діалог як розрив і повернення: від культури до суб’єкта». 
2.2. Фестиваль науки – 2020: «Всеукраїнська наукова онлайн-конференція з міжнародною участю «Київські філософські студії – 2020», 19 травня 2020 року Київський університет імені Бориса Грінченка». Тема доповіді Більченко Є.В.: «Смерть Батька» як філософсько-культурологічний концепт: психоаналітичний вимір медіа-паніки». 
2.3. Науково-практична конференція викладачів, докторантів та аспірантів факультету філософії та суспільствознавства (20-21 травня 2020 року). Подано до друку доповідь «Культурологічний аналіз гегемонії як символічної структури».
2.4. VII Міжнародна науково-практична конференція «Людина і штучний інтелект: виміри філософської антропології, психоаналізу, арт-терапії  та філософської публіцистики. Підхід філософської антропології як метаантропології», Київ, 21-22 травня 2020 року. Тема доповіді Більченко Є.В.: «Штучний інтелект і пустота: дигітальну культура очима скептика». 
2.5. Дистанційна Міжнародна науково-теоретична конференція «Суб’єкт. Пам’ять. Символічне», проведена Національною академією мистецтв України спільно з Інститутом культурології НАМ України та Інститутом проблем сучасного мистецтва НАМ України, Київ, 28 травня 2020 року. Тема доповіді Більченко Є.В.: «Спротив події щодо історії: психоаналіз пам’яті, личина і терор». 
2.6. ХIV Міжнародна конференція «Про природу сміху» на базі Одеського національного університету імені І.І. Мєчнікова, Одеса, 12-13 червня 2020 р. 
Тема доповіді Більченко Є.В.: «Онтологічна природа сміху в дискурсі критичної теорії ». 
2.7. Історико-філософські читання пам’яті Марата Вернікова на базі факультету історії і філософії Одеського національного університету імені І.І. Мєчнікова «Історія філософії як історія проблем», Одеса, 18-19 вересня 2020 р. Тема доповіді Більченко Є.В.: «Істина як філософська проблема: деструктивність та конструктивність номінації вакууму».
2.8. ХХХ-ті Міжнародні людинознавчі філософські читання «ГУМАНІЗМ. ЛЮДИНА. КАМО ГРЯДЕШИ?», 16-17 жовтня 2020  р. (м. Дрогобич, Львівська обл., Україна). Підготовлено і виголошено доповідь на тему: «Постмодерний переклад Ніцше: що б сказав Заратустра?»
2.9. Міжнародна наукова конференція «Куль¬ту¬ра. Мистецтво. Освіта. Актуальний полілог» 01 листопада 2020 р. Підготовлено і виголошено доповідь на тему: «Сучасний зсув освіти: від діалогічної етики В.С. Біблера  до етики розриву А. Бадью»
2.10. Другий Російський Естетичний Конгрес, Єкатеринбург, 22-24 жовтня 2020 р.  Тема доповіді Більченко Є.В.: «Постструктуралистская эстетика как альтернатива постмодерну: универсализм и традиция». 
2.11. Науково-теоретичний семінар на тему: «Георг Вільгельм Фрідріх Гегель: особистість в схрещених променях історії, філософії, ідеології, життєвого досвіду» 23 – 24 жовтня 2020 р. Підготовлено і виголошено доповідь на тему: «Знайомий незнайомий Гегель: страх постмодернізму і страх перед постмодернізмом»
2.12. Наукова конференція «Проблеми методології сучасного мистецтвознавства та культурології» 11-12 листопада 2020 року. Підготовлено доповідь на тему: «Два крила культурології: деконструкція полярності через діалог»
2.13. ХVІ Міжнародна науково-практична конференція Україна першого двадцятиліття  ХХІ   століття: культурно-мистецький вимір» 17-18 листопада 2020 року. Підготовлено статтю на тему «Ціннісні орієнтації сучасної молоді у світлі психоаналізу культури:відчуження та культурологічні механізми його подолання».
2.14. Тринадцатый ежегодный теоретический семинар  «Поиск истины и правда жизни в пространстве современной культуры» 17-18 ноября 2020 г. Підготовлено і виголошено доповідь на тему: «Универсальная сингулярность истины субъекта в этическом пространстве: от поэтики до политики»  (у співавторстві з С. Даніловою)
2.15. VIII Международная научная конференция «Космизм и органицизм: эволюция и актуальность» 19–20 ноября 2020 года. Підготовлено і виголошено доповідь на тему: «Потенциал космизма  в творчестве русских религиозных философов: диалог сквозь хронотопы», виступала спів-ведучою секції «Классика философии космизма и ее современное прочтение».
2.16. VI Международная научно-практическая конференция «Би-, поли-, транслингвизм и языковое образование» під егідою МАПРЯЛ 4-5 грудня 2020 року РУДН.  
3. Проведено 3 спеціалізовані міжнародні лекції на теми:
1. «Гуманітарій у колі природознавців» за запрошенням науково-просвітницького проекту Intellect Networking (Харків).
2. «Рекламна пам’ять як смерть пам’яті» за запрошенням науково-просвітницького проекту Intellect Networking (Харків). Перегляд: https://www.youtube.com/watch?v=zDnQI8LEPNs. Країни-учасники: Україна, Росія, Білорусь, Чехія, Німеччина, США.
3. «Традиция против тренда, традиция как тренд» за запрошенням Відкритої філософської школи (ВФШ, Одеса) на базі Кафедри філософії ОНУ імені І. І. Мечникова. Перегляд: https://www.youtube.com/watch?v=YaEa_XN0NmQ...
4. Підготовлено до друку:
4.1. Авторську монографію «Символічні діалекти діалогу: мова ненависті, мова толерантності, мова любові» у межах ІК НАМ. На розгляді у рецензентів. 
4.2. Авторський розділ «Альтернативна антропологія майбутнього: від постмодерного екрану до неомодерної завіси, від Символічного до Реального» у колективній монографії «Людина в сучасному світі: соціально-філософський та культурно-антропологічний виміри» у межах ІК НАМ. Здійснено редагування розділів монографії, укомплектування плану, написання вступу і висновків до монографії. У стані друку.
4.3.Авторський розділ «Чужий в слов’янській культурі» для колективної монографії Зеленогурського університету (Польща). Подано до друку.
4.4. 13 тез доповідей конференцій (зазначені вище), 2 статті в журнали ВАК, 1 Міжнародна стаття Web of Science, друк яких планується в 2021 році.  Зокрема, міжнародна стаття Web of Science («Вопросы философии») – на розгляді. Статті ВАК: «Культурологічна думка» ІК НАМУ (грудневий номер) та 20-й випуск Альманаху «Актуальні питання культурології» та фахових наукових збірників «Українська культура: минуле, сучасне, шляхи розвитку»: Вип. 34-37 («Культурологія» і «Мистецтвознавство») (група «Б») – у друці.
5.Авторські наукові сторінки на електронних ресурсах: 
Лекції: https://www.youtube.com/channel/UCZfjnx3evDAIxKL3ZQF2f1A
(80 лекцій). 

Проведення конференцій, професійний пакет (ліміт 24 години):
\url{https://zoom.us/profile}, Евгения Бильченко, № учетной записи 124602271

