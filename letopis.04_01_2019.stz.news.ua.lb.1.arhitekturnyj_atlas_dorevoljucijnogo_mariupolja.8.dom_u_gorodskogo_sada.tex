% vim: keymap=russian-jcukenwin
%%beginhead 
 
%%file 04_01_2019.stz.news.ua.lb.1.arhitekturnyj_atlas_dorevoljucijnogo_mariupolja.8.dom_u_gorodskogo_sada
%%parent 04_01_2019.stz.news.ua.lb.1.arhitekturnyj_atlas_dorevoljucijnogo_mariupolja
 
%%url 
 
%%author_id 
%%date 
 
%%tags 
%%title 
 
%%endhead 

\subsubsection{Дом у Городского сада}

Этот дом находится в очень живописном и уютном месте - недалеко от главного
входа в городской сад. Виктор Александрович Нильсен возвёл его в начале ХХ века
в модном тогда стиле модерн. В этом доме городской архитектор прожил со своей
семьёй до самой революции, пока недвижимость не реквизировали большевики.
Парадокс эпохи – человек, построивший половину города, в советское время своего
угла не имел. В восьмидесятых годах помещения занимало детское дошкольное
учреждение, позже - инспекция экологической безопасности, но вот уже несколько
лет дом пустует. Кроме самого строения, украшенного барельефами и декоративной
башенкой, интерес представляет и окружающая территория. Вернее представляла.
Старинный декор палисадников и металлические ограждения растащены в утиль.
Здание у городского сада находится в полушаге от катастрофы, и даже
прославленное имя создателя не может ему помочь.

\ii{04_01_2019.stz.news.ua.lb.1.arhitekturnyj_atlas_dorevoljucijnogo_mariupolja.8.dom_u_gorodskogo_sada.pic.1}
