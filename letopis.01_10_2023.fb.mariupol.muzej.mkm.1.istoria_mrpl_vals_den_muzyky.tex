%%beginhead 
 
%%file 01_10_2023.fb.mariupol.muzej.mkm.1.istoria_mrpl_vals_den_muzyky
%%parent 01_10_2023
 
%%url https://www.facebook.com/100093184796939/posts/pfbid0juomhqyGcK4nf42MmMW4WphqM71e86hP9uk8jRmQGi5ntri9jfzJUU5siuxTxnfTl
 
%%author_id mariupol.muzej.mkm
%%date 01_10_2023
 
%%tags 
%%title Історія Маріупольського вальсу до Міжнародного дня музики
 
%%endhead 

\subsection{Історія Маріупольського вальсу до Міжнародного дня музики}
\label{sec:01_10_2023.fb.mariupol.muzej.mkm.1.istoria_mrpl_vals_den_muzyky}

\Purl{https://www.facebook.com/100093184796939/posts/pfbid0juomhqyGcK4nf42MmMW4WphqM71e86hP9uk8jRmQGi5ntri9jfzJUU5siuxTxnfTl}
\ifcmt
 author_begin
   author_id mariupol.muzej.mkm
 author_end
\fi

\begin{center}
\textbf{Історія Маріупольського вальсу до Міжнародного дня музики}
\end{center}

Музичне мистецтво живе з людиною споконвіку, звуки та мелодії дають можливість
ділитись почуттями, враженнями, радістю та смутком. Кожен музичний витвір несе
в собі і частинку душі свого автора, і атмосферу своєї епохи. Талант
композитора криється в здатності передати музикою величний світ душі, природи і
навіть подій, а талант слухача – відчути та зрозуміти це дивовижне повне
переживань послання.

В музичному світі Маріуполя багато історій вартих уваги. Але сьогодні хочемо
розповісти про одного з маріупольських музикантів, автора композиції, що
багатьом вже стала відома як маріупольський вальс \enquote{Біла квітка}.

Мелодія в ритмі вальсу під авторською назвою \enquote{Белый цветок} була написана в
Маріуполі у 1913 році молодим талановитим музикантом Яковом Сименовським. І
хоча на першій сторінці нотного твору зазначено: \enquote{Посвящаю моей сестре Наде},
ймовірно, на створення композиції мали вплив і події в Маріуполі того періоду.

На початку ХХ століття по всій країні було поширене проведення благодійних свят
– \enquote{День білої квітки}. Ці заходи спрямовувались на збір коштів для боротьби з
туберкульозом, який помітно поширювався, особливо після подій революції
1905–1907 рр. З 1912 року цей урочистий захід проводився і в Маріуполі.
Символом свята була штучна квітка білого кольору, схожа на ромашку. Такими
квітами прикрашали кіоски для збору пожертв, автомобілі, учасники заходів
кріпили їх на одяг. На благодійні внески, зібрані під час Днів білої квітки, в
Маріуполі було відкрито перший літній санаторій на узбережжі моря для
реабілітації хворих на туберкульоз. Благодійні акції в місті не залишались без
уваги громадян, тож, можливо, і Якова Сименовського на написання вальсу
надихнули саме ці урочистості.

Фотокартка молодого композитора зберігалась в колекції Маріупольського
краєзнавчого музею з 1988 року. Але в той час про це навіть не замислювались.
Цінність світлини бачили в образі зображеного разом з ним друга – Віктора
Голіцинського, відомого в Маріуполі дослідника-натураліста, що зробив значний
внесок у розвиток природоохоронного руху краю. Про Сименовського повідомляв
тільки напис на звороті знімку – автор музики вальсу \enquote{Білі троянди}. Лише
ретельне дослідження музейників дозволило дізнатись більше про долю цього
стрункого чоловіка у шкіряній куртці та кавалерійських чоботях, з шашкою,
портупеєю та козацькою папахою на голові.

У 2010 році в мережі Інтернет з'явився відеоролик з видами старого Маріуполя,
накладеними на звучання вальсу \enquote{Біла квітка} у гітарному виконанні
Олексія Симоновського. Виконавець зазначав, що оригінальне нотне видання вальсу
\enquote{Біла квітка}, написаного у Маріуполі, вже майже 100 років є родинною
реліквією їх великої сім'ї спадкових музикантів. Публікація привернула увагу
маріупольських музейників, адже спостерігалась ціла низка співпадінь: і в
прізвищі композитора, і в назві твору, і в періоді написання, і в причетності
до Маріуполя. Так розпочалось спілкування, яке дало можливість розкрити
біографію автора вальсу.

Яків Веніамінович Сименовський народився 6 грудня 1890 року в Маріуполі в
родині міщанина Зельмана Сименовського та його дружини Клори. До речі, в
записах метричної книги Маріупольської синагоги прізвище родини зафіксоване
саме в такій початковій формі. Надалі прізвище трансформувалось у
\enquote{Семеновський}, \enquote{Симоновський}, \enquote{Симановський}.

Освіту майбутній музикант отримав в Маріупольській Олександрівській гімназії,
де, ймовірно, й познайомився з В. Голіцинським. Вивчаючи музику з юних років,
Яків Веніамінович спеціалізувався на ударних інструментах. Після початку Першої
світової війни вступив на службу вільнонайманим диригентом військового
оркестру. Фото з музейної колекції зроблено як раз в цей період – 1916 рік, і
деякі елементи в зовнішності Я. В. Сименовського, а саме погони, обшиті
трибарвним кантом, підтверджують статус добровільно найманого на військову
службу. Під час громадянської війни частина, в якій служив Яків Веніамінович
опинилась спочатку в складі Білої армії, пізніше – Червоної.

У міжвоєнні часи він керував двома оркестрами – народних інструментів і
духовим. Відома афіша з написом \enquote{Яків Сименовський. Віртуозний виконавець на
ксилофонах}. У 1930-ті роки через переслідування у зв'язку з \enquote{білогвардійським
минулим}, був вимушений покинути музику та переїхати до Москви. На новому місці
він освоїв бухгалтерію і працював заступником головного бухгалтера в медичній
артілі \enquote{Санмедгал}. На цій посаді його й було заарештовано у 1938 році. Яків
Сименовський був етапований до Тбілісі, розстріляний 30 квітня 1938 року за
\enquote{шпигунську роботу}. Реабілітований у 1954 році.

В наш час вальс Якова Сименовського \enquote{Біла квітка} став одним із артефактів
історії Маріуполя. Приємна мелодія дозволяє доторкнутися до минулого, уявити,
чим жило місто більше століття тому та відчути його автентичну культуру.
