% vim: keymap=russian-jcukenwin
%%beginhead 
 
%%file 27_12_2021.fb.fb_group.story_kiev_ua.1.odnoj_krovi_ty_i_ja
%%parent 27_12_2021
 
%%url https://www.facebook.com/groups/story.kiev.ua/posts/1827571490772977
 
%%author_id fb_group.story_kiev_ua,poljakova_galina.kiev
%%date 
 
%%tags bazar,chelovechnost,chelovek,kiev,kievljane,sobaka,sochuvstvie
%%title «МЫ С ТОБОЙ ОДНОЙ КРОВИ, ТЫ И Я!»
 
%%endhead 
 
\subsection{«МЫ С ТОБОЙ ОДНОЙ КРОВИ, ТЫ И Я!»}
\label{sec:27_12_2021.fb.fb_group.story_kiev_ua.1.odnoj_krovi_ty_i_ja}
 
\Purl{https://www.facebook.com/groups/story.kiev.ua/posts/1827571490772977}
\ifcmt
 author_begin
   author_id fb_group.story_kiev_ua,poljakova_galina.kiev
 author_end
\fi

Мы, киевляне, в общем-то, большие пофигисты. Нет, это не означает равнодушие,
пресловутое \enquote{моя хата скраю}. Это нечто иное. Мы склонны отмахнуться от очень
многого. НО! Когда ситуация достигает критичного уровня, мы быстро включаемся.
Об этом и не только эта миниатюра. 

«МЫ С ТОБОЙ ОДНОЙ КРОВИ, ТЫ И Я!»

Киевские базары – это неотъемлемый элемент не только нашей городской экономики,
но и культуры. Или, как теперь говорят, нашей идентичности. Продукты
предпочитаю покупать на базаре. Супермаркет, конечно, штука хорошая,
современная, обширная и благоустроенная. Но на базаре я могу товар и понюхать,
и попробовать. Мне отрежут кусок, который мне больше понравился. Могу
перекинуться шуткой с продавщицей. Могу получить у нее консультацию и рецепт. И
вообще, они славные тетки. 

Клише «торговка базарная» давно устарело и неприменимо к продавцам на базаре.
Изрядная их часть имеет высшее образование. Рука нужды вытянула их в павильоны,
«роллеты», клетушки, палатки, которые оледеневают зимой и чуть ли не докрасна
раскаляются летом. Я не имею в виду Бессарабку. Бессарабка это вообще не базар,
а выставочный павильон. Базары, настоящие базары, разбросаны по всему Киеву с
привязкой к станциям метро и автостанциям. Мой базар на Березняках. Многих
продавщиц я знаю в лицо, как и они меня. Возвращаясь из командировки, слышу:

- Что-то Вас не видно было. Не болели?

Болезнь для них сродни стихийному бедствию. Это невыход на работу и,
следовательно, потеря в зарплате. Порядки и будни на базаре суровые. 

На каждом базаре есть целые своры собак и кошек. Их любят и балуют. Коты
дрыхнут где хотят, и угощаются чем хотят. Им особый респект и привилегии. Они
охотятся. От собак, пожалуй, проку никакого. Мышей не ловят, защитно-караульной
службе не обучены. Потявкать, конечно, могут. Но не больше. Однако их кормят,
им устраивают лежанки из картонных коробок, приносят старую одежду и коврики на
подстилки. Интересно, что собаки на базарах в основном крупные, способные
внушить страх. Но не внушают. Они тоже граждане этого государства в
государстве. 

В течение нескольких лет я встречала на Березняковском базаре крупного пса без
задней правой лапы. Он был упитан, как и все прочие базарные собаки. Лихо
скакал на трех лапах и чувствовал себя вполне уверенно. Продавщица яиц
рассказала мне его историю. 

К базару он прибился, будучи еще на четырех лапах. Молодой и резвый, он высоко
ценил заботу и кормежку, а потому старался проявлять бдительность и облаивал
машины, привозившие товар. Гонялся за ними, всячески демонстрируя, как и его
собратья, что он не зря хлеб ест, что работает и охраняет порядок. То ли он
увлекся и замешкался, то ли водитель недоглядел, сдавая назад, но случилось
так, что псу заднюю лапу колесом раздробило. 

Окровавленного зверя замотали в халаты и на руках оттащили в ближайшую
ветеринарную клинику «Лесси». Она, кстати, находится метрах в 500 от базара.
Врачи пса осмотрели и ничего кроме ампутации предложить не могли. Заикнулись
было айболиты об эвтаназии, дескать, лечение в копеечку влетит, а инвалиду на
улице все равно не выжить. Эвтаназию с негодование отвергли. Весь базар собрал
необходимую сумму на операцию, препараты и даже на оплату стационара. Зная,
тамошние цены, я склоняю голову перед базарной «республикой». Думаю, что не
менее дневной выручки выложила каждая продавщица за спасение лохматого
пустобреха. 

Пса выходили. Лет пять он еще носился по базару, тявкая на тех, кто почему-то
ему не нравился. Ему все прощалось.

\ii{27_12_2021.fb.fb_group.story_kiev_ua.1.odnoj_krovi_ty_i_ja.cmt}
