% vim: keymap=russian-jcukenwin
%%beginhead 
 
%%file 19_11_2021.fb.zadorozhnaja_natalia.1.skazka_sofia_1
%%parent 19_11_2021
 
%%url https://www.facebook.com/permalink.php?story_fbid=1284163448768686&id=100015251282349
 
%%author_id zadorozhnaja_natalia
%%date 
 
%%tags kiev,sofia_sobor,ukraina
%%title «Сказка» в двух частях - София Киевская - часть 1
 
%%endhead 
 
\subsection{«Сказка» в двух частях - София Киевская - часть 1}
\label{sec:19_11_2021.fb.zadorozhnaja_natalia.1.skazka_sofia_1}
 
\Purl{https://www.facebook.com/permalink.php?story_fbid=1284163448768686&id=100015251282349}
\ifcmt
 author_begin
   author_id zadorozhnaja_natalia
 author_end
\fi

«Сказка» в двух частях. 

Случившаяся не только со мной вчера, в особенный ноябрьский вечер…

Часть первая.

Честь быть приглашённой Нелей Михайловной Куковальской на презентацию
уникального проекта  в Софию Киевскую обернулась радостью и вдохновением. 

\ii{19_11_2021.fb.zadorozhnaja_natalia.1.skazka_sofia_1.pic.1}

Такие состояния я называю «потрясением красотой»! После будто немного хмельной
от радости и эмоций, которые не вмещаются в границы собственного мира.

Когда смысл камерного события, его атмосфера, лица участников, повод,
окружение, великолепные стены собора Софии Киевской, вобравшие в себя историю
нашей страны и веры, наполняются энергией созидания; когда оживает история,
время меняет свой ход и из его глубин выплывают образы удивительных женщин,
несших свет веры, любви, просвещения. 

\ii{19_11_2021.fb.zadorozhnaja_natalia.1.skazka_sofia_1.pic.2}

Презентация проекта «Три поколения благоверности», который вобрал в себя
научные и исторические исследования их судеб: Ингигерта (у нас её называли
Ириной) - супруга Ярослава Мудрого,
Анна, их дочь, ставшая королевой Франции, дочь Анны, внучка Ярослава и
Ингигерты, - Эдигна, вошедшая в историю как Блаженная (Святая) Эдигна
Баварская.

Судьба каждой - невероятна. Вместе - пример того, как «сшивались» этими
судьбами страны. И это заслуживает отдельного рассказа подробного. Знанием
истории судеб удивительных героинь, размахом их созидательных поступков можно
только восхищаться! 

\ii{19_11_2021.fb.zadorozhnaja_natalia.1.skazka_sofia_1.pic.3}

Представлены были факсимильные издания Молитвословов, с молитвами, написанными
одной героинь. Рассказы о том, как это происходило из уст участников и
организаторов. 

Теплые слова поздравлений от Шведского посольства и от супруги Посла Франции.

И в завершение - особенное событие - открытие киота с мозаичными образами трёх
героинь!

\ii{19_11_2021.fb.zadorozhnaja_natalia.1.skazka_sofia_1.pic.4}

И не просто открытие - а с рассказами живыми тех молодых ребят, которые
воплотили это всё в удивительных образах, воссоздав со скрупулёзной точностью
детали орнаментов и символику тех лет; с добрыми словами организаторов и
меценатов этого волшебства - воплощения связи поколений через тысячелетие без
малого!!!

\ii{19_11_2021.fb.zadorozhnaja_natalia.1.skazka_sofia_1.pic.5}

Ведь оснащение этого собора, его открытие, удивительные фрески его - это всё
происходило под непосредственным руководстом Ярослава Мудрого и супруги его -
Ингигерты...

Замыкаются круги! 

Это был волшебный, удивительный вечер. В сопровождении прекрасного хора!!!

\ii{19_11_2021.fb.zadorozhnaja_natalia.1.skazka_sofia_1.pic.6}
