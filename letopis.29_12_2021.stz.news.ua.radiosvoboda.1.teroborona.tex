% vim: keymap=russian-jcukenwin
%%beginhead 
 
%%file 29_12_2021.stz.news.ua.radiosvoboda.1.teroborona
%%parent 29_12_2021
 
%%url https://www.radiosvoboda.org/a/foto-terytorialna-oborona-ukrainy/31631686.html
 
%%author_id nuzhnenko_sergіj,kushnir_marjan
%%date 
 
%%tags teroborona,oborona,kiev,ukraina,rossia,napadenie,ugroza
%%title Україна чинитиме опір. Чи готова тероборона? (фоторепортаж)
 
%%endhead 
\subsection{Україна чинитиме опір. Чи готова тероборона? (фоторепортаж)}
\label{sec:29_12_2021.stz.news.ua.radiosvoboda.1.teroborona}

\Purl{https://www.radiosvoboda.org/a/foto-terytorialna-oborona-ukrainy/31631686.html}
\ifcmt
 author_begin
   author_id nuzhnenko_sergіj,kushnir_marjan
 author_end
\fi

Відповідно до Закону України «Про основи національного спротиву» в Україні,
формують бригади територіальної оборони (ТрО) в усіх регіонах. Разом з тим
аналітики зазначають, що нині існує чимало проблем із формуванням підрозділів
тероборони: фінансування, матеріально-технічна база і неврегульований час на
підготовку. Чим живуть столичні тероборонівці і як проходить їхнє тренування. І
що потрібно зробити, щоб ТрО стала важливою складовою оборони країни в разі
повномасштабної російської агресії.

\href{https://www.youtube.com/watch?v=JUxKN_znObA}{%
Територіальна оборона зупинить російські війська?, Радіо Свобода Україна, youtube, 27.12.2021%
}

Відповідно до закону «Про основи національного спротиву» в Україні розпочали
формувати територіальну оборону. Наразі у Києві уже є такий підрозділ, який
регулярно тренується щосуботи. Утім, чи вдасться зібрати необхідну кількість
цивільних, які готуватимуться чинити опір? Зрештою, чого бракує на сьогоднішній
день та чи зможе зупинити територіальна оборона російські війська? Деталі у
відео Радіо Свобода.


\ifcmt
  tab_begin cols=3,no_fig,center

     pic https://i2.paste.pics/c877f21eb70ba097fcb4d029cbb326f8.png
		 pic https://i2.paste.pics/6b5ed3540fe134b62496649687460f93.png
		 pic https://i2.paste.pics/1df5933ca2dd04a4192ff41cdc9ad696.png

  tab_end

  tab_begin cols=4,no_fig,center

		 pic https://i2.paste.pics/4d5752c7b19f8ce1dcf913c1d6961e41.png
		 pic https://i2.paste.pics/804f596372994fdfb70f6e6a684dac53.png
		 pic https://i2.paste.pics/fed091c5fd3bf893b09924b49d4276c8.png
		 pic https://i2.paste.pics/3964ed86e9786384b91d50229301f584.png

  tab_end

\fi

«Якщо ми допустимо те, що нам треба саме місто боронити, то це означає, що
ствольна та реактивна артилерія буде стояти під Броварами», – розповідає нам
Денис Семирог-Орлик. Він очолює громадську організацію «Територіальна оборона
столиці», разом з тим є молодшим сержантом резерву. Раніше у війську не служив.
Від початку російської агресії в Україні волонтерив, каже, що продовжує це
робити по цей день.

\ii{29_12_2021.stz.news.ua.radiosvoboda.1.teroborona.pic.1}

Серед більшості тих, хто в суботу прийшов на тактичні тренування територіальної
оборони – добровольці, які раніше не служили в армії і не мають бойового
досвіду. В руках у них бутафорна дерев’яна зброя, хтось брав свою страйкбольну
зброю, а у когось справжні мисливські рушниці.

Свою мотивацію пояснюють просто – готові захищати своє місто.

\subsubsection{В чому полягає роль територіальної оборони?}

Відповідно до закону «Про основи національного спротиву», кожна область на
початку 2022 року має сформувати підрозділ територіальної оборони. В задуманому
сценарії бійці цих формувань у разі необхідності мають прийти на підсилення
регулярних підрозділів Збройних сил України. Що входитиме до їхніх обов’язків?

Передбачається, що територіальна оборона має охороняти об’єкти критичної
інфраструктури: теплові електростанції, атомні електростанції, урядові будівлі,
штаби військових й інше. Разом з тим, тероборонівці можуть долучитися до
охорони громадського порядку чи супроводження військових колон.

\ii{29_12_2021.stz.news.ua.radiosvoboda.1.teroborona.pic.2}

Одне з таких тренувань розігрують на імпровізованому полігоні під час тренувань
бійці тероборони. За легендою, вони пересуваються у двох автомобілях,
супроводжуючи цінний вантаж. У різних місцях на колону здійснюють напад
диверсійно-розвідувальні групи противника. Основне завдання – відбити атаку та
відкинути противника подалі від автівок.

«Контакт на 6 годин», – голосно кричить один із бійців, які попереду колони.
Він помічає озброєних людей, які готують засідку. Бійці вибігають із умовного
автомобіля, займають лінію оборони та, застосовуючи димову завісу, чинять опір
нападникам.

Такі тренування, звісно, далекі від реальних бойових дій: умовні автомобілі,
бутафорна зброя і крики «пам-пам» замість реальних пострілів.

\ii{29_12_2021.stz.news.ua.radiosvoboda.1.teroborona.pic.3}

\subsubsection{Чи зможуть столичні тероборонівці впоратися із своїм завданням?}

«Ви ж розумієте, що таке територіальна оборона? Це не регулярна частина. Для
того, щоб з ними проводити заняття, у нас для цього є тільки вихідні. Й те, ці
вихідні, для розуміння – це ініціатива самих людей. Ті збори, де залучають на
10 діб для того, аби підготувати людей – це не те, що смішно. Це не серйозно.
Для початку потрібна логіка в заняттях. Має бути, бодай, як в інших країн, де
функціонує тероборона. Можливо, теоретичні заняття після роботи, втілення цього
на практичних заняттях. Однак за 10 діб підготувати підрозділ, де більша
частина людей не служила в армії – неможливо», – зазначає Сергій, інструктор,
військовий із понад 20-річним досвідом служби в армії.

Він, як і бійці тероборони, добровільно взявся за підготовку охочих. Каже, нині
у цього підрозділу непоганий рівень підготовки, втім, вони ще тренуються і
покращують свої навички.

Територіальна оборона має бути підготовлена не лише для підсилення регулярних
військ ЗСУ чи виступати на допомогу охорони громадського порядку. Ці бійці
мають стати фундаментом для ймовірного партизанського руху – руху опору у тій
ситуації, коли та чи інша територія може буде окупованою іноземними військами.

«Цільова модель територіальної оборони України» – таку доповідь наприкінці року
презентували Центр оборонних стратегій спільно із Українським інститутом
майбутнього. В першу чергу експерти визначають виклики, які можуть стояти перед
Україною в оборонній сфері. Найважливіші ризики наразі – нарощування військової
потужності Росії вздовж українських кордонів, ризики масштабного війкового
вторгнення російської армії на територію України і, зрештою, стратегія
контрольованою Росією ескалації на території України.

\ii{29_12_2021.stz.news.ua.radiosvoboda.1.teroborona.pic.4}

Всеохопна оборона в рамках стратегії воєнної безпеки передбачає залучення
цивільного населення до оборони, – вважає міністр оборони (2019–2020 років)
Андрій Загороднюк.

«Україна не досягне військового паритету з Російською Федерацією в осяжній
перспективі. Мало того, усвідомлення цього факту закладено в стратегію воєнної
безпеки. При математичному підрахуванні спроможності Росії завжди більші, ніж
спроможності України. Це базується на їхньому бюджеті та стані економіки
(йдеться про Росію, – ред.), – каже Загороднюк. – У всіх країнах, без винятку,
які мають потенційного чи реального ворога близько до їхніх кордонів,
запроваджена концепція суцільної оборони. Всеосяжної. До якої залучена більша
кількість громадян».

На сьогодні в рамках закону «Про основи національного спротиву» формується
територіальна оборона і цей підхід на перший погляд нагадує більше примусовий
збір підрозділів у відповідних регіонах. Однак на цей момент відсутня як і
матеріальна база для таких підрозділів, так і окреме фінансування під
розгортання на навчання ТрО.

«На сьогодні ми бачимо певні проблемні точки, намагалися запропонувати рішення
для розв’язання проблем. Перше – примусовий принцип комплектування сил
територіальної оборони не спрацює. Друге – залишковий принцип фінансування...
Тобто виокремлення тероборони в окремий вид діяльності. Також відсутність
доктрини територіальної оборони», – такі ризики бачить нині експерт Центру
оборонних стратегій Віктор Кевлюк.

\ii{29_12_2021.stz.news.ua.radiosvoboda.1.teroborona.pic.5}

У разі реалізації задумів, як зазначають аналітики, у підсумку можна зібрати до
400 тисяч людей, які можуть стати на оборону країни. Варто діяти за принципом
оборони свого населеного пункту: чітка ієрархія від створення командного штабу
до формування ТрО в дрібних населених пунктах:

\begin{itemize}
  \item Штаб сил ТрО;
  \item Штаб ТрО військово-сухопутної зони;
  \item Штаб бригади ТрО;
  \item Штаб батальйону ТрО.
\end{itemize}

Загалом бригади мали би створити в 25 областях України, також у Києві та
Севастополі як окремих адміністративних одиниць.

Загалом підрозділи мали б охопити 1720 населених пунктів, що б дало нагоду в
рамках оперативного реагування зібрати близько 400 тисяч осіб. Разом з тим
важливою складовою є добровільне формування. Утім, це лише аналітичні
дослідження експертів.

Чи встигне Україна сформувати такі підрозділи на тлі скупчення російських сил
вздовж кордону України – не відомо.

Волонтер, який з початку агресії Росії тісно співпрацює з військовими і не з
розмов розуміє, як можуть розвиватися події в Україні, Георгій Тука вважає, що
нині Україна не готова до можливої повномасштабної операції України.

«Стратегічна мета створення сил територіальної оборони переслідує дві цілі.
Перше, в разі необхідності термінова підтримка особовим складом сил
територіальної оборони Збройних сил України. І друга складова – використання
територіальної оборони як фундаменту для організації партизанського руху. Як
колишній бюрократ можу зазначати, що будь-яка ініціатива не варта нічого, якщо
вона не забезпечена фінансами. Розробка нормативної бази – це дуже важливо. Але
це виключно одна з необхідних складових. Чи достатньо розробки нормативної
бази, аби організм запрацював? Ні», – каже Тука.

Він додає, що нині Міноборони заявляє, що недофінансовується на 7 мільярдів
гривень і, навіть, з’являється брак дизельного пального, – навіть цивільна
людина чудово розуміє, що ані танки, ані бронетранспортери, ані вантажівка не
здатна рушити з місця, якщо в неї немає пального, то про яку готовність може
йти мова?

\ii{29_12_2021.stz.news.ua.radiosvoboda.1.teroborona.pic.6}

Попри матеріальні проблеми в Мінобороні, сили тероборони таки створюються. І
головна цінність, як зазначає співзасновник Українського інституту майбутнього
і радник міністра внутрішніх справ Антон Геращенко – можливість залучити
добровольців до сил оборони України.

«Найголовніше, що дає цей закон, це можливість державі залучати добровольців,
витрачаючи кошти на їхнє навчання, дисциплінування, щоб вони були разом у
відділеннях. У Києві 4 мільйони населення. Теоретично, якщо б в нас була така
ситуація, то у нас тільки в Києві було б 60 чи 70 тисяч добровольців зі зброєю
в руках. Наприклад, ви чи я, ми разом могли б бути в одному відділенні. Ви
журналіст, а я радник міністра. І у випадку тривоги, ми би могли разом стати з
вами і захищати залізничний вузол, трансформаторну підстанцію, і таке інше».

\ii{29_12_2021.stz.news.ua.radiosvoboda.1.teroborona.pic.7}

Денис Семирог-Орлик один із таких добровольців, які при першій тривозі готові
зі своєю зброєю долучитися до борони столиці і виконувати поставлені перед ним
завдання.

«За себе можу сказати, що якщо щось таке відбудеться, то я вас запевняю, що за
викликом я прийду зі своєю зброєю. Ми не маємо допустити того, що було те, що в
Луганську, Донецьку й інших містах, коли їх захопили, а патріоти ходили з
прапорами. Я вже буду в формі. Виконувати поставлене завдання», – зазначає
Семирог-Орлик.
