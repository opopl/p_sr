% vim: keymap=russian-jcukenwin
%%beginhead 
 
%%file 27_12_2018.fb.bilchenko_evgenia.1.otvet_chitatelju
%%parent 27_12_2018
 
%%url https://www.facebook.com/yevzhik/posts/1958541384180962
 
%%author Бильченко, Евгения
%%author_id bilchenko_evgenia
%%author_url 
 
%%tags bilchenko_evgenia
%%title БЖ. Ответ читателю.
 
%%endhead 
 
\subsection{БЖ. Ответ читателю.}
\label{sec:27_12_2018.fb.bilchenko_evgenia.1.otvet_chitatelju}
 
\Purl{https://www.facebook.com/yevzhik/posts/1958541384180962}
\ifcmt
 author_begin
   author_id bilchenko_evgenia
 author_end
\fi

БЖ. Ответ читателю.

Позавчера Елена Курилова написала обо мне трогательное эссе, в котором меня
тэгнула. Я читала, и у меня ком в горле стоял. Наверное, это был первый раз,
когда меня так громко пожалели как девочку. Но всё-таки, милая Леночка, не
пишите мне больше такого, потому что, во-первых, я расслабляюсь, а, во-вторых,
я начинаю культивировать в себе неправославный эгоизм. Я расту, ориентируясь на
Летова, ярость иногда лечит (не путать с агрессией). Обида же только
увеличивает боль. С другой стороны, я знаю, что когда моя боль достигает
порога, я перестаю ее лелеять и играть с ней. Оскар Уайльд говорил, что есть
боль такая острая, что организм, подчиняясь законам природы, ее сбрасывает,
чтобы выжить, или трансформирует в светлую печаль.

\ifcmt
  pic https://scontent-cdt1-1.xx.fbcdn.net/v/t1.6435-9/48416687_1958541340847633_4820552038968459264_n.jpg?_nc_cat=101&ccb=1-5&_nc_sid=8bfeb9&_nc_ohc=kpg4K3nlj0AAX8iXWpo&_nc_ht=scontent-cdt1-1.xx&oh=00f9ff16e69fd3363c9cf2dd58701c38&oe=6150091A
  width 0.4
	fig_env wrapfigure
\fi

Лена пишет о том, что меня разрывают пополам поэт (мужское) с его мессианским
предназначением да странной жизнью и женщина с ее житейским желанием простого
семейного счастья. То же самое мне сказал мой священник. Это абсолютная правда.
Лена считает, что женщину-поэта надо как-то особенно любить. Так вот, не надо
особенно. В общем, уже и никак не надо. Вторая женская ипостась моей жизни не
состоялась и вряд ли когда-то состоится: все, кого я любила, не любили меня
так, как я их, точнее - никак не любили (секс и нежности не в счет), потому что
осетринка бывает только первой свежести, а ответственность - это и есть
осетринка. Не ходить со мной развлекаться, а принять взаимную ответственность в
горе и радости. Ответственность за меня никто никогда не нес, но, я полагаю,
раз не нес, значит, это я не заслужила ответственности и заботы, доверия и
брака. Я не люблю в Цветаевой ее игр со смертью, но одна фраза у нее
гениальная: "Не любит - не надо мне женских бус, не любит, так я на коня
вздымусь". Я утомляла близких людей спектром своих ожиданий. Я постоянно
чего-то жду и клянчу нежность. Это не только унизительно, но и вредно для нас
всех. Насильно никого не заставишь заботиться о себе или принять заботу. Но я
еще не готова к православной роли смиренной домостроевской жены, которая все
ждет да ждет. Я могу любить односторонне (нет ничего страшного в игре в одни
ворота, кроме боли), но при этом уйти в себя как в пацана. Мне суждено
исдохнуть пацаном. Потому я слушаю каждый день всего три песни: "Пацан"
Бранимира/Прилепина, "Я живой" РИЧа, "Русский подорожник" 25/17. Послушайте их,
Леночка: это секрет моего выживания как поэта и забвения себя как женщины. Ну,
не дадено. Не с собой же кончать. Скорая вчера советовала транки, но я
отказалась. 

Знаю также, что меня почитывают мои бывшие атошники. Про русскую церковь
лайкают. Хорош тихариться, пацаны, из нескурвившихся, сцуко, эти ополченские
песни - про вас тоже. И я верю, что ценой жизни двум или трем из вас смогу
налить водку и посадить за один стол. Никаких диванных кровососов я не боюсь,
братишки. Плюс у меня есть уже Церковь. Ничего, выживу. Китчевых суицидов а ля
Полозкова не будет.

Слово читателю (реакция на текст "Молитва о любви моей"):

"Женя, как напишет, так я весь день думаю. Я всё время думаю, потому что она
взрывает и вспахивает. И вот, прочитав это стихотворение, я подумала, что
Бильченко ведь покруче Летова будет. С Башлачёвым, наверное, на одном уровне, а
Летова превосходит. Если бы Егор прочитал этот стих, то его глаза бы широко
раскрылись и рот, и он так бы сидел какое-то время. Он бы затих, замер. А потом
бы, может, орал, ревел. Нервно курил. Девочка не виновата. Вряд ли она просила
нахлобучить её Талантом. Чтобы она могла выразить то, что другие чувствуют, но
передать не могут. Они немы, а она нет. Она наказана Гениальностью, которую под
силу тащить и выдержать только здоровому мужику. Мужик он же может бухать,
драться, его могут избить и облёванный он уснёт на полу. Он может не есть и не
спать, мерзнуть, таскаться по вокзалам, рисковать. Или его могут в дурке качать
наркотой, и он выдержит, выживет, он дух и только станет упорней, закалённей и
злее. А с девочками так нельзя. Девочек нужно беречь. Их нужно защищать. О них
нужно заботиться. Девушка не может жить как былинка на ветру. Она не может быть
одна во Вселенной, в холодном аду мира. Она вообще не должна воевать и
защищаться. Это её ломает. То, что мужчину делает сильнее, женщину уничтожает.
В фильмах всё правильно: героиню защищает герой. Или два. И они не просто
соперничают и борются за её внимание, они её берегут. В жизни всё не так.
Особенно с женщинами, не обладающими особой манкой женственностью, а имеющие -
к тому же - огромный Талант. Понятно же, что он терзает её. Что она не спит, не
ест, пока не произведёт на свет стихотворение. Что ей покоя нет, и это
изматывает. Ко всем её проблемам со здоровьем, политикой и умищем (умище и
талант это не одно и то же), еще суровая, безжалостная, чудовищная
Гениальность. А девочка не виновата. Может, ей хочется прижаться и плакать... И
чтобы жалели её, к себе притянув. И эти чувства направляли ей, окутывали её
заботой. Евгения Бильченко, я тебя вижу, я с тобой. Ты не одна. Ты не одна с
ЭТИМ. С этим всем, о чём ты рассказываешь. Не думай, что ты шепчешь, уже едва
шепчешь - в пустоту. Тебя слышат и видят. Посмотри внутрь, сколько сияющих
тёплых линий увидишь. От людей, которые думают о тебе" (Елена Курилова).
