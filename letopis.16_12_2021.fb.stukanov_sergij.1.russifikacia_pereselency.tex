% vim: keymap=russian-jcukenwin
%%beginhead 
 
%%file 16_12_2021.fb.stukanov_sergij.1.russifikacia_pereselency
%%parent 16_12_2021
 
%%url https://www.facebook.com/serhii.stukanov/posts/4884948291536536
 
%%author_id stukanov_sergij
%%date 
 
%%tags donbass,jazyk,mova,russifikacia,ukraina,zakon
%%title Русифікація під прикриттям дотримання прав переселенців
 
%%endhead 
 
\subsection{Русифікація під прикриттям дотримання прав переселенців}
\label{sec:16_12_2021.fb.stukanov_sergij.1.russifikacia_pereselency}
 
\Purl{https://www.facebook.com/serhii.stukanov/posts/4884948291536536}
\ifcmt
 author_begin
   author_id stukanov_sergij
 author_end
\fi

Русифікація під прикриттям дотримання прав переселенців 

Вчора, 15 грудня, комітет ВРУ з прав людини ухвалив рішення включити в
законопроєкт 4487 "Про внесення змін до Закону України «Про забезпечення прав
та свобод внутрішньо переміщених осіб»" норму, яка запускає мовлення на всю
Україну державного російськомовного телеканалу. Відповідно до внесених
комітетом поправок, російськомовний канал «Дом», створений для мовлення на
окуповані території, має поширити мовлення на всю Україну, у складі так званої
універсальної програмної послуги.

Як пише Детектор медіа, "комітет виходить з того, що внутрішньо переміщені
особи мешкають на всій території вільної України, отже, призначені для них
телепрограми мають поширюватись на всю державу".

Друзі! Я не хочу зараз говорити про те, що пропонована норма порушує закон Про
мову та закон Про телебачення і радіомовлення, якими встановлено чіткі квоти
для трансляції продуктів недержавними мовами. Це очевидно. 

Я не хочу також говорити про те, що це є сумнівна ідея - створювати державний
телеканал в країні, яка кілька років тому вдало здійснила роздержавлення НСТУ й
розбудовує суспільне мовлення. Це також очевидно.

Я, як колишній донеччанин, хочу висловитися в контексті теми переселенців.
Вважаю здійснення русифікації під прикриттям дотримання прав переселенців
цинізмом та блюзнірством найвищого ґатунку. 

Річ у тому, що це не має абсолютно нічого спільного з дотриманням прав
переселенців. Держава, будьмо відвертими, не надто допомогла переселенцям в
2014 році, як і в подальші роки. Й якщо держава хоче допомогти сьогодні, в 2021
році, то є низка можливостей і актуальних проблем - забезпечити житлом,
створити робочі місця тощо. Непочатий край роботи. Російськомовний канал аж
ніяк не входить навіть в топ-10 потреб переселенців. 

Сказати відверто, то навіть якщо брати окуповані території, то суцільно
російськомовний телеканал виглядає мені дуже сумнівною ідеєю. Доносити
інформацію до мешканців окупованих територій потрібно, водночас слід розуміти,
що війна з Росією точиться передовсім на полях ідентичності, а мова є важливим
маркером ідентичності. Тому я переконаний, що просувати на окуповані території
треба не тільки меседжі (російською мовою), але й українську мову. Це
стратегічна мета. 

Що ж стосується переселенців, які переселилися на вільні терени (більшість -
вже багато років тому), то при чому тут взагалі російська мова? Ми, вихідці з
Донецька й Луганська, - хіба не громадяни України? Чи ми не вчили українську
мову? Ми що - з Ростова чи Рязані переселилися? Ми що - не такі ж громадяни, як
всі? Для нас - що, треба створити окрему мовну сегрегацію? Окремий телеканал,
окремі школи, садочки тощо? 

Навпаки. Інтеграція переселенців має включати долучення до української мови й
культури, до українських контекстів, яких чимало з нас були позбавлені на
Донбасі. Мешкаючи на Донбасі, спраглі української культури мусили докладати
зусиль, щоб її віднайти. А в 2014 році нас за українську мову, прапори, погляди
просто вбивали. І переїхали ми на вільні терени в тому числі, щоб мати
можливість спілкуватися українською й навчати українською наших дітей. 

Відтак, саме створення умов для переселенців щодо входження в українську
культуру є завданням великої державної ваги. Особливо це стосується
підконтрольних Україні теренів Донбасу. Створіть нарешті умови, щоб Донбас
повертався до української мови. Як це було ще в 30-40-ві, в часи перед
тотальною русифікації краю. 

Тому, виходячи з цього, я вважаю, що всеукраїнський російськомовний канал
нібито \enquote{для переселенців} - це образа мене, як колишнього донеччанина та
українця. 

Закликаю всіх свідомих громадян виступити проти цього блюзнірства. А всіх
депутатів не голосувати за таке неподобство.
\ii{16_12_2021.fb.stukanov_sergij.1.russifikacia_pereselency.cmt}
