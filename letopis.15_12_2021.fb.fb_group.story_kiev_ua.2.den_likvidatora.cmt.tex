% vim: keymap=russian-jcukenwin
%%beginhead 
 
%%file 15_12_2021.fb.fb_group.story_kiev_ua.2.den_likvidatora.cmt
%%parent 15_12_2021.fb.fb_group.story_kiev_ua.2.den_likvidatora
 
%%url 
 
%%author_id 
%%date 
 
%%tags 
%%title 
 
%%endhead 
\zzSecCmt

\begin{itemize} % {
\iusr{Елена Семененко}

Ми живём в Киеве и уже больше тридцати лет после радиации первое время старались виезжать хотяби на лето а теперь безвиездно
Сидим по нас уже можно изучать влияние радиации на популяцию и индивидуально

\iusr{Егор Красава Тебякин Егор}
А как наши государственные мужи периодически снимают с ликвидаторов льготы, а потом возвращают?
А то что некоторые из них не могут до сих пор доказать свое присутствие и участие в ликвидации аварии на ЧАЭС из-за наших бюрократических проволочек.
Ничего так.
Я пару таких людей знаю лично.

\iusr{Татьяна Оржеховская}
Трагедия века, когда на первомайский парад вывели школьников, а в это время тушили пожар на АЭС

\begin{itemize} % {
\iusr{Люда Турчина}
\textbf{Татьяна Оржеховская} 

еще повезло, что ветер долго дул не на Киев, и еще 1-го был всесоюзный или
всеукраинский веломарафон, точго не помню. Только 5 мая оповестили про
частичную правду, что не надо гулять, окна не открывать и т.д., потому что в
Швеции замеряли радиацию на машинах наших дальнобойщиков. Пришлось на весь мир
открыть правду о взрыве на Чернобыльской АС.

\begin{itemize} % {
\iusr{Татьяна Оржеховская}
\textbf{Люда Турчина} И никто за это не ответил

\iusr{Victoria Ivaniy Stein}

Люда в советское время разве были наши дальнобойщики в Швеции? На сколько я
знаю, то в Скандинавию ветром и осадками занесло радиацию. Знаю в Норвегии, так
как здесь живу и из рассказов местного населения, то в некоторых районах
северной части Норвегии тоже замеры были высокие и пришлось им на некоторых
фермах и в некоторых хозяйствах уничтожить скот, урожай и ограничить территории
пастбищ для серерных оленей в тот год и в некоторых случаях в последующий/е
год/ы.

\iusr{Дмитро Левицький}
\textbf{Люда Турчина} Это была международная «Велогонка мира». Много команд не приехали. В Скандинавию и Германию радиоактивные осадки занёс ветер начиная с самого первого выброса в момент аварии.
\end{itemize} % }

\iusr{Гала Алданькова}
\textbf{Татьяна Оржеховская} 

а нас студентов-первокурсников загнали в стройотряд летом строить Троещину. Со
всего потока наскребли 30 девушек, кому уже исполнилось 18 и 2х парней. До сих
пор помню отведенные в сторону глаза главного комсомольца факультета Радика
Потирайло.. Сам он на стройку почему то не записался


\iusr{Надежда Шитюк}
Татьяна Оржеховская да, так и было... шли, одуванчики срывали, веночки из них
плели... 30 км от Киева

\end{itemize} % }

\iusr{Ирина Левина}

я работала в помещении МСЧ №126 в городе Чернобыль в 1991 году в составе
бригады исследователей от ННЦРМ (Національний Науковий Центр Радіаційної
Медицини).


\iusr{Валерий Стричко}

Были знакомые ликвидаторы. Теперь нет. Кум Иван умер ещё в 1995г., а сосед Николай
в 2003г. Очень тяжело уходили ребята. Надо помнить о тех, кто спас людей от
страшной беды.

\begin{itemize} % {
\iusr{Егор Красава Тебякин Егор}
\textbf{Валерий Стричко} Этих людей надо приравнивать к статусу героев Украины.
Они спасли страну и Европу также.
Светлая и вечная память героям-ликвидаторам!
\end{itemize} % }

\iusr{Надежда Лабик}

Мы 10лет подряд отдыхали в Оташево, на Припяти. Пили воду из реки, на этой воде
варили еду, ели рыбу и грибы. А радиация там уже была. В 1986 первым ушел мой
муж, за ним с небольшим перерывом ещё 4 человека... Это тоже последствие мирного
атома.

\iusr{Антоніна Переверзєва}

Земний уклін усім ліквідатора м.

\iusr{Генадий Пинский}

Вечная память ликвидаторам, которых уже нет и которым повезло остаться в живых.
@igg{fbicon.thumb.up.yellow}  @igg{fbicon.hands.pray}  @igg{fbicon.face.anxious.sweat} 

\end{itemize} % }

