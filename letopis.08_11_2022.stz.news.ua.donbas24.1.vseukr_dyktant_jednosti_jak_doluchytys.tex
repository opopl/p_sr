% vim: keymap=russian-jcukenwin
%%beginhead 
 
%%file 08_11_2022.stz.news.ua.donbas24.1.vseukr_dyktant_jednosti_jak_doluchytys
%%parent 08_11_2022
 
%%url https://donbas24.news/news/vseukrayinskii-diktant-nacionalnoyi-jednosti-yak-dolucitisya
 
%%author_id demidko_olga.mariupol,news.ua.donbas24
%%date 
 
%%tags 
%%title Всеукраїнський диктант національної єдності — як долучитися
 
%%endhead 
 
\subsection{Всеукраїнський диктант національної єдності — як долучитися}
\label{sec:08_11_2022.stz.news.ua.donbas24.1.vseukr_dyktant_jednosti_jak_doluchytys}
 
\Purl{https://donbas24.news/news/vseukrayinskii-diktant-nacionalnoyi-jednosti-yak-dolucitisya}
\ifcmt
 author_begin
   author_id demidko_olga.mariupol,news.ua.donbas24
 author_end
\fi

\ii{08_11_2022.stz.news.ua.donbas24.1.vseukr_dyktant_jednosti_jak_doluchytys.pic.front}
\begin{center}
  \em\color{blue}\bfseries\Large
9 листопада всі охочі українці зможуть взяти участь у традиційному всеукраїнському Радіодиктанті
\end{center}

Цієї середи, 9 листопада, у День \href{https://archive.org/details/28_07_2022.olga_demidko.donbas24.jak_shvydko_perejty_na_movu_porady}{\emph{української писемності та мови}},
\footnote{Як швидко перейти на українську мову: корисні поради, Ольга Демідко, donbas24.news, 28.07.2022, \par%
\url{https://donbas24.news/news/yak-svidko-pereiti-na-ukrayinsku-movu-korisni-poradi}, \par%
Internet Archive: \url{https://archive.org/details/28_07_2022.olga_demidko.donbas24.jak_shvydko_perejty_na_movu_porady}%
} в Україні відбудеться традиційний Радіодиктант національної єдності-2022. Свято розвитку
державної мови було започатковане 9 листопада 1997 року Указом Президента
України Леоніда Кучми. Дата обрана не просто так — 9 листопада українці також
відзначають День вшанування Нестора Літописця — послідовника творців
слов'янської писемності Кирила і Мефодія. Про те, як приєднатися до
всеукраїнського диктанту та перевірити його, \href{https://www.kmu.gov.ua/news/radiodiktant-nacionalnoyi-yednosti-2022-yak-vzyati-uchast}{розповіло}%
\footnote{Радіодиктант національної єдності-2022: як взяти участь, Міністерство культури та інформаційної політики України, kmu.gov.ua, 07.11.2022, \par\url{https://www.kmu.gov.ua/news/radiodiktant-nacionalnoyi-yednosti-2022-yak-vzyati-uchast}}
Міністертво культури та інформаційної політики України.

\ifcmt
  ig https://i2.paste.pics/PSWRI.png?trs=1142e84a8812893e619f828af22a1d084584f26ffb97dd2bb11c85495ee994c5
  @wrap center
  @width 0.9
\fi

\begin{leftbar}
\emph{\enquote{Мета радіодиктанту — об'єднати людей навколо України в усьому світі, а не
перевіряти грамотність учасників всеукраїнського флешмобу}}, — наголосили в
Міністерстві. 
\end{leftbar}

\textbf{Читайте також:} \href{https://donbas24.news/news/movni-normi-vstupayut-v-silu-yakix-pravil-treba-dotrimuvatisya}{\emph{Мовні норми вступають в силу — яких правил треба дотримуватися}}%
\footnote{Мовні норми вступають в силу — яких правил треба дотримуватися, Еліна Прокопчук, donbas24.news, 15.07.2022, \par%
\url{https://donbas24.news/news/movni-normi-vstupayut-v-silu-yakix-pravil-treba-dotrimuvatisya}%
}

\ii{08_11_2022.stz.news.ua.donbas24.1.vseukr_dyktant_jednosti_jak_doluchytys.pic.1}

Авторка цьогорічного тексту — українська письменниця і режисерка, членкиня
Українського ПЕНу \emph{\textbf{Ірина Цілик.}} Читатиме\par\noindent текст народна артистка, Герой України
\emph{\textbf{Ада Роговцева.}}

\subsubsection{Як взяти участь?}

Долучитися до написання всеукраїнського диктанту можуть усі охочі. Для цього
потрібно 9 листопада об 11:00 за київським часом увімкнути:

\begin{itemize}
  \item Українське Радіо або Радіо Культура;
  \item телеканал Суспільне Культура;
  \item діджитал-платформи Суспільного: фейсбук-сторінку Суспільного та ютуб-канали Українського Радіо та Суспільне Новини;
  \item додаток suspilne.radio;
  \item додаток \enquote{Дія}.
\end{itemize}

\begin{quote}
\emph{\enquote{Написання Радіодиктанту-2022 адаптовано до умов воєнного часу. Також у різних
країнах світу будуть облаштовані студії для написання диктанту для вимушених
переселенців та українців, які живуть за кордоном}}, — додали у Верховній Раді.
\end{quote}

\textbf{Читайте також:} \href{https://donbas24.news/news/skoli-visi-ta-sadocki-ukrayini-perexodyat-na-distanciiku-detali}{\emph{Школи, виші та садочки України переходять на дистанційку: деталі}}%
\footnote{Школи, виші та садочки України переходять на дистанційку: деталі, Яна Іванова, donbas24.news, 10.10.2022, \par%
\url{https://donbas24.news/news/skoli-visi-ta-sadocki-ukrayini-perexodyat-na-distanciiku-detali}%
}

\subsubsection{Як можна перевірити Радіодиктант національної єдності-2022?}

\ii{08_11_2022.stz.news.ua.donbas24.1.vseukr_dyktant_jednosti_jak_doluchytys.pic.2}

\textbf{10 листопада} буде оприлюднено текст диктанту, тому якщо ви хочете, щоб ваш
диктант перевірили фахівці, ви можете:

\begin{itemize} % {
\item Надіслати паперового листа за адресою: 1 001, м. Київ, вул. Хрещатик,
26.  Важливо, щоб гриф дати надсилання був не пізніше ніж 10 листопада.

\item Сфотографувати/відсканувати написаний текст (у форматі .jpg, .png, .jpeg,
.tiff, .pdf) та надіслати його до 11:00 10 листопада на адресу:
rd@suspilne.media або через електронну форму для отримання робіт. Її
оприлюднять на сайті ukr. radio 9 листопада.
\end{itemize} % }

Раніше Донбас24 розповідав, що в Україні \href{https://donbas24.news/news/v-ukrayini-zapuskayut-google-znannya-comu-mozna-navcitisya-na-osvitnyomu-xabi}{\emph{запускають Google Знання}}.%
\footnote{В Україні запускають Google Знання: чому можна навчитися на освітньому хабі, Наталія Сорокіна, donbas24.news, 02.11.2022, \par\url{https://donbas24.news/news/v-ukrayini-zapuskayut-google-znannya-comu-mozna-navcitisya-na-osvitnyomu-xabi}}

Ще більше новин та найактуальніша інформація про Донецьку та Луганську області
в нашому телеграм-каналі Донбас24.

ФОТО: з відкритих джерел.

\ii{insert.author.demidko_olga}
%\ii{08_11_2022.stz.news.ua.donbas24.1.vseukr_dyktant_jednosti_jak_doluchytys.txt}
