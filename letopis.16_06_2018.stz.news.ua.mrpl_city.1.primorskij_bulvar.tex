% vim: keymap=russian-jcukenwin
%%beginhead 
 
%%file 16_06_2018.stz.news.ua.mrpl_city.1.primorskij_bulvar
%%parent 16_06_2018
 
%%url https://mrpl.city/blogs/view/primorskij-bulvar
 
%%author_id burov_sergij.mariupol,news.ua.mrpl_city
%%date 
 
%%tags 
%%title Приморский бульвар
 
%%endhead 
 
\subsection{Приморский бульвар}
\label{sec:16_06_2018.stz.news.ua.mrpl_city.1.primorskij_bulvar}
 
\Purl{https://mrpl.city/blogs/view/primorskij-bulvar}
\ifcmt
 author_begin
   author_id burov_sergij.mariupol,news.ua.mrpl_city
 author_end
\fi

Местность, простирающуюся вдоль моря от Слободки до судоремонтного завода,
коренные мариупольцы с незапамятных времен называ\hyp{}ли \textbf{\em\enquote{пляжем}}. В это понятие
включали и собственно пляж – относительно неширокую полосу песка, омываемую
мутноватой, - особенно в ветреную погоду, - азовской морской водой, и так
называемое Нижнее портовское шоссе, и проложенную вдоль него железную дорогу,
связывающую станцию Мариуполь с портом, и склоны обрыва, засаженные
виноградниками, - позже их место заняли декоративные деревья и кустарники.

Кажется, еще в предвоенные годы шоссе с расположившимися вдоль него
санаториями, домами отдыха и дачами обрело официальное название; на указателях
появилось – \textbf{\emph{\enquote{Санаторный проспект}}}. В 1949 году, когда вождю и учителю всех
народов, верному ленинцу исполнилось семьдесят лет, Санаторный проспект получил
имя \emph{\textbf{Сталина}}. Об этом переименовании довелось слышать такую историю от
старожила, занимавшего в городе в свое время весьма ответственный пост.
Переименовать-то переименовали, но потом спохватились: проспект, носящий столь
\enquote{высокое} название, имел уж очень неприглядный вид. Дорога вся в рытвинах и
ухабах, тротуаров практически нет, покосившиеся в разные стороны фонарные
столбы дополняли унылую картину. И тогда бывший на посту председателя
Ждановского горсовета Назар Львович Кудрявцев на свой страх и риск направил
письмо в Москву. Мол, в городе Жданове назвали один из проспектов в честь
товарища Сталина. Хотелось бы сделать его самым красивым и нарядным, а денег на
это нет. Письмо попало на стол самому Берии. На удивление никого не посадили,
более того были выделены средства из союзного бюджета, которые и использовали
на приведение в божеский вид прибрежной магистрали. Но таблички с надписью \enquote{Пр.
имени И. В. Сталина} продержались не так уж долго: в феврале 1956 года на ХХ
съезде КПСС Н. С. Хрущев осудил культ личности – их сняли, и проспект снова стал
\textbf{\emph{Санаторным}}.

Более позднее преображение санаторного проспекта связывают с именем первого
секретаря горкома партии В. М. Цыбулько. За город была вынесена нефтебаза,
правда, случилось это уже после того, как Владимир Михайлович был переведен на
работу в Киев. Нефтебаза когда-то занимала площадь от пешеходного моста,
переброшенного через железнодорожное полотно, до места, облюбованного сейчас
рощицей из деревьев разных пород. Снесли частные домишки, построенные едва ли
не с основания Слободки. Мостовую из гранитной брусчатки заменили асфальтовым
покрытием, отороченным аккуратными бордюрами и газонами, среди которых то там,
то здесь были высажены кусты роз. Разобрали заборы, в том числе и довольно
красивые и изящные, а их место заняла подпорная стенка из местного бутового
камня с расшивкой швов цементным раствором. Построили круглой формы ресторан из
стекла и бетона, отличавшийся тем, что летом в нем было нестерпимо жарко, а
зимой – неимоверно холодно. Позже убрали трамвайный путь и пустили троллейбусы.
Строй новых осветительных столбов довершил картину преобразования по существу
нового проспекта. Оставалось только переименовать его. Что и было сделано; не
мудрствуя лукаво, назвали его... \textbf{\em Приморским бульваром}.

Покончив с \enquote{новейшей} историей Приморского бульвара, стоит, наверное,
обратиться к эпизодам его более ранних эпох. На карте Мариуполя 1888 года
обозначены \textbf{купальни}, располагались они примерно там, где сейчас находится
яхт-клуб металлургического комбината \enquote{Азовсталь}. Заметим, что купальни были
для мужчин - свои, для женщин - свои, поэтому пляжники и пляжницы погружались в
животворные воды Азовского моря, не обременяя себя купальниками или иной
одеждой.

В начале нынешнего века среди виноградников появились дачи именитых
мариупольских граждан, Де Полоне, Михайловского и других. Примерно к тому
периоду относится и сооружение в приморской зоне известной Белой дачи. После
революции ее передали дому отдыха для рабочих, потом ее занял начальник
строительства и первый директор завода \enquote{Азовсталь} \textbf{Яков Семенович Гугель}. Будто
бы здесь его и арестовали в 1937 году как \enquote{врага народа}. 

%Санаторий "Металлург"
\ii{16_06_2018.stz.news.ua.mrpl_city.1.primorskij_bulvar.pic.1}

Из статьи заведующей отдела досоветского периода Мариупольского краеведческого
музея Раисы Петровны Божко \enquote{Отдых у моря в старом Мариуполе} ( статья
опубликована в сборнике \enquote{Лукоморье} в 2005 году) узнаём: \emph{\enquote{На бывшей даче
Гринберга в 1912 году был устроен первый в Мариуполе санаторий или, как тогда
было принято говорить, \enquote{санатория}. Это лечебное заведение было создано на
средства Мариупольского уездного земства и городского отдела Всероссийской лиги
по борьбе с туберкулезом. По нашим нынешним представлениям санаторий был
небольшим. В первый сезон он приютил только сорок неимущих жителей города и
уезда, больных чахоткой. Действовал санаторий только четыре теплых месяца,
когда наиболее полно можно было использовать целебные свойства морской воды и
воздуха}}. Своеобразным продолжателем дела борьбы с коварной болезнью стал
\textbf{детский костно-туберкулезный санаторий имени Н. К. Крупской}. Добрые руки врачей,
медицинских сестер, нянечек в буквальном смысле этого слова поставили тысячи
детей на ноги. В санатории есть своя школа, и дети, пока проходят курс лечения
- а он иногда длится месяцами, - не прерывают учебу. В этой школе педагоги
приходят в палаты к лежачим больным ребятам и проводят с ними занятия.

% 2 - вид на террасы
\ii{16_06_2018.stz.news.ua.mrpl_city.1.primorskij_bulvar.pic.2}

В 1938 году построены корпуса санатория центрального комитета профсоюза
угольщиков. Здесь отдыхали в основном горняки Донбасса, но бывало, что
приезжали сюда оздоровиться шахтеры и из других угольных бассейнов страны. Уже
в послевоенные годы он получил название - \textbf{\enquote{Металлург}}. Между прочим, в нем
любил совмещать отдых с литературным трудом писатель Владимир Федорович Попов,
автор романов \enquote{Сталь и шлак}, отмеченного Сталинской премией, \enquote{Разорванный
круг}, \enquote{Обретешь в бою}, прототипом главного героя этого произведения послужил
профессор Мариупольского металлургического института Мирон Яковлевич
Меджибожский.

%Санаторий "Металлург" - single
%\ii{16_06_2018.stz.news.ua.mrpl_city.1.primorskij_bulvar.pic.1}

В первые же дни войны в санатории был устроен эвакогоспиталь. В нем лечили
раненых и больных бойцов и командиров перед отправкой в медицинские учреждения
глубокого тыла. Утром 8 октября 1941 года, когда передовые части гитлеровцев
уже были в Ильичевском районе, на платформе (она существует и сейчас и
располагается между магазином \enquote{Волна} и яхт-клубом комбината \enquote{Азовсталь})
спешно шла погрузка пациентов, персонала и имущества госпиталя. Позже, уже
после войны, его главный врач Александр Миронович Фролов вспоминал, что в
середине дня, когда их эшелон с огромными красными крестами на стенах и крышах
вагонов проходил мимо переезда у гавани Шмидта, там стояли немецкие
автоматчики. Нет, они не стреляли, лишь жестами показывали в сторону Сартаны,
давая понять, мол, там уж точно вас разнесут в щепы.

Оккупанты в тех же палатах поместили свой госпиталь, который позже, при подходе
Красной Армии к Мариуполю, так быстро эвакуировался, что оставил не только
прекрасную мебель, вероятно, награбленную в одной из захваченных Германией
стран, но, как говорят завхозы, и мягкий инвентарь: подушки, матрасы, простыни
и т.п. До сегодняшнего дня в кабинете директора здравницы стоят старинные часы,
циферблат которых обрамлен кусками оленьих рогов. Часы эти остались в
наследство от немецкого лазарета.

% 2 - вид на террасы
%\ii{16_06_2018.stz.news.ua.mrpl_city.1.primorskij_bulvar.pic.2}

% Водная станция завода "Азовсталь"
\ii{16_06_2018.stz.news.ua.mrpl_city.1.primorskij_bulvar.pic.3}

Приморский бульвар даже в будние погожие весенние или осенние дни был заполнен
людьми. Прогуливались пенсионеры, наслаждаясь свежим морским воздухом, гуляли
отдыхающие из здравниц: кто в одиночку, кто парами, а кто целой гурьбой. В
пляжный сезон курортники спускались по широкой лестнице к морю, другие
поднимались им навстречу после купания с опаленными на солнце носами, с
прилипшими песчинками к ступням ног. Сейчас людей стало меньше.

Увы, сейчас часть оздоровительных учреждений находится в заброшенном состоянии.
Главное объяснение их владельцев: тяжелые времена, отсутствие средств.
Насколько помнится, сразу после войны времена были куда сложнее, однако как-то
умудрялись поддерживать чистоту и порядок в местах отдыха трудящихся.
Свидетельства тому - многочисленные фотографии \enquote{на память}, которые можно
встретить в старых альбомах тех, кому довелось отдыхать в домах отдыха \textbf{\enquote{Мир}},
\textbf{\enquote{Металлург}} или иных.

27 мая 1951 года на берегу моря была открыта водная станция коксохимического
%завода, ее воздушное здание с круглыми окнами-иллюминатора\hyp{}ми, которая стала
завода, ее воздушное здание с круглыми окнами-иллюминаторами, которая стала
украшением теперешнего Приморского бульвара. В 1961 году завод \enquote{Азовсталь}
построил свою водную станцию, теперь – яхт-клуб этого предприятия. Впрочем, и
сейчас есть чем полюбоваться на Приморском бульваре. Хотя бы высокого уровня
гостиницами \enquote{Чайка} и \enquote{Европейская}. Но первым здесь учреждением подобного
типа, - конечно, далеко не таким комфортабельным, как вновь построенные, -
является гостиница \enquote{Турист}.

% Водная станция завода "Азовсталь"
%\ii{16_06_2018.stz.news.ua.mrpl_city.1.primorskij_bulvar.pic.3}

\ii{16_06_2018.stz.news.ua.mrpl_city.1.primorskij_bulvar.pic.4}

Можно было бы еще рассказать об украшавших любимую мариупольцами зону отдыха
скульптурах: двух львах, \enquote{охранявших} лестницу, ведущую вверх по склону еще с
дореволюционных времен, и тех изваяниях, что были установлены в советское
время. О пионерских лагерях, устроенных на побережье предприятиями города в
первые послевоенные годы. О ресторанах \enquote{Золотая рыбка} и \enquote{Юг}, примостившемся
на крутом склоне, местом развлечений и знакомств молодежи середины 50-х годов
теперь уже прошлого ΧΧ века. Да мало ли что еще можно вспомнить?

