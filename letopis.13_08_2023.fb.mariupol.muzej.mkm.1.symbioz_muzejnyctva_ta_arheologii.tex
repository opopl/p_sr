%%beginhead 
 
%%file 13_08_2023.fb.mariupol.muzej.mkm.1.symbioz_muzejnyctva_ta_arheologii
%%parent 13_08_2023
 
%%url https://www.facebook.com/100093184796939/posts/pfbid02isHAMcdTobfTkehzCJmXXKxWMHQYjHjaGWzUVtSe2NzipB1RKVtakT9d4fbmhdZ2l
 
%%author_id mariupol.muzej.mkm
%%date 13_08_2023
 
%%tags 
%%title Симбіоз музейництва та археології
 
%%endhead 

\subsection{Симбіоз музейництва та археології}
\label{sec:13_08_2023.fb.mariupol.muzej.mkm.1.symbioz_muzejnyctva_ta_arheologii}

\Purl{https://www.facebook.com/100093184796939/posts/pfbid02isHAMcdTobfTkehzCJmXXKxWMHQYjHjaGWzUVtSe2NzipB1RKVtakT9d4fbmhdZ2l}
\ifcmt
 author_begin
   author_id mariupol.muzej.mkm
 author_end
\fi

📍 Користуючись нагодою святкування Дня археолога, Маріупольський краєзнавчий
музей пропонує разом пригадати давню історію нашого краю, відому нам завдяки
археологічним дослідженням, розглянути найбільші та найцікавіші пам'ятки,
вшанувати працю археологів та їх внесок у вивчення історії Північного
Приазов'я. Тож, пропонуємо вашій увазі серію публікацій про симбіоз музейництва
та археології.

✅️ Розпочнемо з невеличкої розповіді про поховальний комплекс шостого
тисячоліття до нашої ери, який увійшов в історію під назвою Маріупольський
неолітичний могильник, та є однією з найбільших неолітичних пам'яток Східної
Європи. Він був виявлений під час закладки доменної печі №1 на заводі
\enquote{Азовсталь} у 1930 році.

Дослідження проводилися під керівництвом професора М. О. Макаренка. До складу
експедиції також входили співробітники Маріупольського краєзнавчого музею І. П.
Коваленко (директор музею), Н. П. Єгорова, Н. С. Коваленко.

🏕Розкопки почались 10 серпня 1930 року і тривали протягом 65 днів, напружена
праця без вихідних по 12 годин на добу, була обумовлена жорсткими термінами
будівництва заводу. Але попри численні труднощі, пам'ят\hyp{}ник був врятований для
науки, а в Маріупольському краєзнавчому музеї з'яви\hyp{}лися унікальні експонати:
кам'яні знаряддя праці та булави, прикраси з іклів вепра, намиста з кістки та
мушлі, кераміка тощо. Тоді ж були зроблені чотири монолітні вирізки поховань:
дві – доби неоліту, дві – бронзового віку. Багатотонні ящики на возах привезли
до музею. Щоб доставити їх у будівлю, довелося розбирати віконний отвір на
другому поверсі. Відтоді три моноліти стали предметом гордості та перлинами
археологічної колекції музею, що займали почесне місце в його експозиції, до
весни 2022 року, коли вони, як і більша частина музею, були практично знищені,
внаслідок російської агресії проти України.

За результатами розкопок у 1933 році було опубліковано звіт Миколи Макаренка –
\enquote{Маріюпільський могильник} (з текстом та матеріалами звіту можна ознайомитися
за посиланням: \url{https://elib.nlu.org.ua/view.html?id=1044}

Могильник мав розмір завдовжки 28 метрів та завширшки близько 2 метрів. 

❗️Всього було знайдено 124 поховання. Приблизно половина поховань мали
похоронний інвентар. Більша половина поховань (83) була вкрита червоним
порошком – вохрою (символом крові та вогню) вони знаходилися в індивідуальних
ямах. Поховання відбувались орієнтовно протягом 200 років.

📖 Завдяки матеріалам отриманим під час дослідження могильника можна скласти
доволі повне уявлення про господарську діяльність мешканців Приазов'я в період
VI-V тис. до н.е. Люди, що мешкали поблизу річки та Азовського моря, займалися
риболовлею (на вирозуба та інших риб), збиранням (намистинки зроблені з
перламутрових мушель молюсків), полювали на оленя, вовка, лисицю (намиста з
зубів), борсука, дикого кабана, на птахів. Судячи з фігурки бика, вони мали і
велику рогату худобу. Вміли розщеплювати кремінь, шліфувати кам'яні знаряддя,
обробляти дерево кам'яними сокирами; можливо, вони вже вміли робити човни.
Знайдені фрагменти посуду свідчать, що люди освоїли перший штучний матеріал –
кераміку. А знахідка пряслиця свідчить про винахід ткацтва. Володіли художнім
смаком та майстерністю у виготовленні різних прикрас.

⏳️Перші десятиліття після відкриття могильнику не було близьких аналогій, але
незабаром археологи побачили його подібність до інших могильників, таким чином
Маріупольський могильник став проявом широкої культурної спільності.

Маріупольський неолітичний могильник дав назву області поширення кількох
подібних між собою археологічних культур доби неоліту — початку енеоліту —
Маріупольська культурно-історична область, що включала культури:
азово-дніпровську, нижньодонську та інші. Маріупольська культура вважається
частиною великої культурної спільності, до складу якої входили також волинська,
верхньодніпровська, києво-черкаська, надпорізька, донецька та нижньодонська
культури. Так, для відомого археолога Н. С. Котової, \enquote{існування
азово-дніпровської культури охоплює близько 1200 років}.

\#mkmmariupol \#mariupol \#маріупольськиймузей \#маріуполь \#відродимомаріупольськиймузей \#археологія
