% vim: keymap=russian-jcukenwin
%%beginhead 
 
%%file 13_10_2022.stz.news.ua.donbas24.1.trymajemos_razom_mrpl_teatr_conception_nova_vystava.txt
%%parent 13_10_2022.stz.news.ua.donbas24.1.trymajemos_razom_mrpl_teatr_conception_nova_vystava
 
%%url 
 
%%author_id 
%%date 
 
%%tags 
%%title 
 
%%endhead 

«Тримаємось разом» — маріупольський театр «Conception» представить нову виставу (ФОТО)

Актори театру авторської п'єси «Conception» підготували нову музично-поетичну виставу

Маріупольський театр авторської п'єси «Conception» готовий представити нову
музично-поетичну виставу «Тримаємось разом». Актори зачитають вірші Ліни
Костенко, Сергія Жадана, Петра Маги, а також твори власного авторського
написання.

«Поєднані однією історією, одним болем, ми тримаємось усі разом, хто постраждав
від рук війни…», — наголосили актори театру.

Читайте також: Як в Маріуполі проходив фестиваль «Театральна брама» — деталі

Як виникла ідея створити нову виставу? 

Режисер театру авторської п'єси «Conception» Олексій Гнатюк розповів, що
спочатку актори театру виступили на літературному вечорі з віршами Ліни
Костенко та Сергія Жадана, який було підготовлено до Дня Незалежності України.
Творчий вечір маріупольців підтримало київське видавництво «Саміт Книга»,
запросивши відомих літераторів Києва. Так колектив познайомився з українським
актором, поетом-піснярем та телеведучим, який наразі служить в ЗСУ Петром
Магою. Автор подарував акторам свої брошури з віршами, які надихнули весь
колектив. Саме вірші Петра Маги лягли в основу вистави, адже вони дуже
актуальні та злободенні. У цьому спектаклі задіяні всі актори театру. Також
завдяки грі на фортепіано Людмили Гричаненко вистава матиме живий музичний
супровід.

«Актори читатимуть вірші наших сучасників, але будуть присутні і елементи
театралізації. Нашому колективу тема війни і втрати власної домівки дуже
близька, тому актори не зможуть просто читати вірші. Звісно, вони будуть і
грати», — підкреслив режисер театру Олексій Гнатюк.

Читайте також: «Театроманія» продовжує розвивати та підтримувати культуру
Маріуполя

Які ще плани має колектив?

Маріупольський колектив театру авторської п'єси «Conception» нещодавно
повернувся з турне українськими містами. Театр представив свою виставу «Обличчя
кольору війна» за документальними подіями. Режисер і актори майстерно
висвітлили тему війни, виживання і людяності. У кожному місті вистава проходила
при переповнених залах та бурхливих оваціях захоплених та вдячних глядачів.

«Найбільше нас вразила реакція глядачів у Вінниці. Глядачі аплодували 15 хвилин
стоячи. Це нас дуже надихнуло на нові звершення», — розповів Олексій Гнатюк.

Читайте також: Драматичний театр Маріуполя виступив у Польщі — актори взяли участь у міжнародному фестивалі

Наразі театр поповнився новими акторами. До нього приєдналися маріупольські
актриси, яким вдалося виїхати з окупованого Маріуполя, актор з «Молодого
театру» та актриса з Івано-Франківська. Репетиції проходять у приміщенні
Київнаукфільму. Колектив театру вже відновлює виставу «Другий шанс» і готує
відеопривітання до Дня захисників України.

«Головним нашим завданням зараз є інтегруватися в київську культуру і знайти
спонсорів, адже ще не вистачає реквізиту та потрібного обладнання», — поділився
Олексій Гнатюк.

Читайте також: У Києві покажуть виставу про Маріуполь

Ще одним проєктом, який незабаром побачить глядач, стане телевізійне ток-шоу
«Мистецький максимум», телеведучим якого стане актор театру Дмитро Гриценко.
Проєкт буде присвячено співпраці київської та маріупольської культур. Один раз
на тиждень Дмитро проводитиме інтерв'ю з діячами культури Києва, які
розповідатимуть про свою готовність допомагати розвитку культури Маріуполя.

А музично-поетичну виставу «Тримаємось разом» можна буде побачити 17 жовтня о
17:00 в Київському академічному театрі українського фольклору «Берегиня» (вул.
Івана Миколайчука, 3а), яка відбудеться в рамках співпраці з видавництвом
«Саміт-Книга», Міжнародним благодійним фондом «Твоя Доброта», за підтримки DDI
Croup. Вхід вільний. 

Раніше Донбас24 розповідав, які театральні проєкти та культурні заходи,
присвячені Маріуполю, реалізуються закордонними митцями.

Ще більше новин та найактуальніша інформація про Донецьку та Луганську області
в нашому телеграм-каналі Донбас24.

ФОТО: з відкритих джерел.
