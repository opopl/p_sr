% vim: keymap=russian-jcukenwin
%%beginhead 
 
%%file slova.krasota
%%parent slova
 
%%url 
 
%%author 
%%author_id 
%%author_url 
 
%%tags 
%%title 
 
%%endhead 
\chapter{Красота}
\label{sec:slova.krasota}

%%%cit
%%%cit_pic
%%%cit_text
Четвертое место - песня \enquote{Несе Галя воду}. В этой песне сошлось все и
сразу: певучесть украинского языка, невероятная мелодичность, доходчивость,
нежность и проницательность. А как она звучит на несколько голосов, невероятно
красиво. Впервые на телевидении эту песню исполнило трио Мариничей. Она сразу
полюбилась. Удивительно, что как только начинаешь ее петь, то сразу возникает
образ села, невероятная природа и ситуация, где девушка несет воду, а парень
пытается комплиментами привлечь к себе внимание. Вроде все просто, но как
\emph{красиво} и выразительно. Вспомним эту песню вместе в исполнении Людмилы
Сенчиной,
%%%cit_title
\citTitle{5 задушевных украинских песен, которые пели наши родители, а теперь поем мы}, 
Кино Вояж И Не Только, zen.yandex.ru, 07.06.2021
%%%endcit

