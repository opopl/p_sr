% vim: keymap=russian-jcukenwin
%%beginhead 
 
%%file 17_10_2020.fb.oleg_orljanskii.halatnikov
%%parent 17_10_2020
%%url https://www.facebook.com/oleg.orlyansky/posts/3265838753533136
%%tags isaak halatnikov,physics,lev landau
%%author oleg_orljanskii
 
%%endhead 

\subsection{Халатнікова Ісааку Марковичу - 101 рік}

\url{https://www.facebook.com/oleg.orlyansky/posts/3265838753533136}

\index[names.rus]{Халатников!Исаак Маркович, физик-теоретик}

Сьогодні, 17 жовтня 2020 р., виповнився 101 рік випускнику кафедри теоретичної
фізики ДНУ видатному фізику-теоретику, Халатникову Ісааку Марковичу.

З ДНЕМ НАРОДЖЕННЯ! ЗДОРОВ’Я І БЛАГОПОЛУЧЧЯ! 

У 2008 р. від імені кафедри я запросив Ісаака Марковича на наш ювілей. На жаль,
обставини так склалися, що він не зміг приїхати. Але під час телефонної
розмови, коли мова зайшла про Alma mater, він, звертаючись до когось у своїй
приймальні (в заснованому ним Інституті теоретичної фізики у Чорноголовці), як
до свідків, що чули це не один раз, урочисто промовив:

- Передайте всем привет и скажите, что я, как говорится, горжусь, что я окончил
Днепропетровский университет и всегда хвалюсь этим. Свидетели подтвердят, что
это был лучший университет в стране.

І потім, звертаючись, до наших студентів:

- Чтоб они гордились своим университетом и кафедрой, чтоб они свысока смотрели
даже на Московский университет… один студент уехал доучиваться в Москву после
третьего курса мы его жалели, что меняет хороший университет на плохой.
Расскажите им историю. Его фамилия была Мицкевич. И мы очень жалели, когда он с
третьего курса... и даже не могли объяснить, понять как же так из такого
университета уезжает в московский.

Чтобы они гордились своим университетом, гордились своими профессорами.

Поздравьте своих студентов!

Минулорічний пост, у якому  про наукові досягнення Ісаака Марковича:

\url{https://www.facebook.com/oleg.orlyansky/posts/2378533082263712?__cft__[0]=AZXvmLTdMfT26pHBrpeVmciiME-FX9q0h3meTltbt-Az_LfIMNtx_W1YLMLARAdaFfD5ZkgCHeFbdKPZrN42KvSYDY3FlCmjkrBtuN-uc27HBWuzDlI01KoY47lr_W4unjo&__tn__=%2CO%2CP-R}

Из книги «Дау, Кентавр и другие»:

Судя по данным моего паспорта, я родился 17 октября 1919 г., но у меня есть
ощущение, что у меня по ошибке отняли тринадцать дней, то есть на самом деле
это произошло 4 октября. А путаница могла выйти из-за перехода на новый,
григорианский календарь… 

В день моего рождения в город Екатеринослав входили банды Махно. Моя мама,
схватив новорожденного меня, побежала прятаться... И в суматохе она несла меня
головой вниз. Возможно, именно эта встряска сыграла роль в дальнейшем развитии
моих умственных способностей…

В школе я был шустрым, подвижным мальчиком. Не хулиганом, конечно, но
подвижным. На уроках мне было, как я себе представляю, немножко скучно, и я вел
себя, наверное, не самым лучшим образом. Моя мама… рассказывала потом
следующее: учитель сказал ей, что мальчик я хоть и хороший, но «шкідник». А
«шкідник» по-украински значит «вредитель», и в то время – шел уже 1929 год –
это слово было не совсем безобидным, так как процессы по делу вредителей
набирали силу. Так что я с самых юных лет был причислен к отряду этих самых
вредителей. После этого мама в школу больше не ходила…

Учеба в школе давалась мне очень легко. Ближе к окончанию десятилетки
выяснилось, что ученики, окончившие школу на «отлично», могут поступать в
институт без экзаменов. И тут у меня образовалась проблема. Так как я учился в
украинской школе, у меня было не очень хорошо с русским языком. Но я за полгода
с помощью учебника русских диктантов выучил все правила так, как учат точные
науки, написал все диктанты и получил нужную мне пятерку по русскому языку. И
это обеспечило мне свободное, без экзамена, поступление в университет.

Окончив школу, я получил аттестат «з відзнакою» и поступил на физическое
отделение университета. Тут у меня, конечно, был вопрос – я был уже связан с
математикой, но, подумав, поступил все-таки на физическое отделение. Это был
1936 год. В это время в Днепропетровске был организован филиал Ленинградского
физтеха. А.Ф.Иоффе как раз в то время организовывал эти филиалы – Харьковский,
Днепропетровский, Уральский. В университет приехала большая группа
ленинградских профессоров. Среди них были Б.Н.Финкельштейн, приятель Ландау,
Г.В.Курдюмов, который организовал потом в Москве Институт физики твердого тела,
В.И.Данилов – все первоклассные физики Ленинградского ФТИ, ученики А.Ф.Иоффе.
Все они преподавали в Днепропетровском университете. Б.Н.Финкельштейн стал
потом моим научным руководителем. В этом смысле мне очень повезло.

Кроме того, на физическом факультете уже вовсю ходили рассказы о Ландау,
который преподавал тогда в Харькове. И хотя учебников Ландау-Лифшица тогда еще
не было, рукописи его лекций циркулировали среди студентов, и мы изучали
теорфизику по конспектам лекций Ландау. Их красота произвела на меня глубокое
впечатление, и я очень быстро решил, что буду физиком-теоретиком.

…Но свою первую любовь, математику, я тоже не забывал. Теорию функций
комплексного переменного у нас тогда читал С.М.Никольский... Математическая
наука в университете тоже была очень сильной. В этом большая заслуга Сергея
Михайловича. У него были тесные контакты с московскими светилами математики –
А.Н.Колмогоровым, П.С.Александровым, В.Ф.Каганом. Они регулярно приезжали к нам
с лекциями, и в университете была атмосфера высокой математики.

Последний выпускной спецэкзамен по теорфизике в Днепропетровском университете я
сдал в субботу, в июне. Помню, мы сидели с моим доцентом на лавочке на бульваре
проспекта Карла Маркса, и я сдавал ему этот экзамен. Это было двадцать первого
июня 1941 г. А утром по черному радиорупору, которые, совершенно одинаковые,
висели у всех на кухне, я услышал, что началась война.

Вскоре после начала войны я получил из военкомата повестку. В принципе с
письмом Ландау я мог бы, наверное, уехать в Москву, но тогда это было
как-то…неприлично. Дело даже не в патриотизме, а, наверное, в той примитивной
приверженности дисциплине, которая отличала то время в целом, и к которой мы
все были приучены.
