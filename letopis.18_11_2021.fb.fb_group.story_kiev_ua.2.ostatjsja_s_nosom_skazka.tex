% vim: keymap=russian-jcukenwin
%%beginhead 
 
%%file 18_11_2021.fb.fb_group.story_kiev_ua.2.ostatjsja_s_nosom_skazka
%%parent 18_11_2021
 
%%url https://www.facebook.com/groups/story.kiev.ua/posts/1800571116806348
 
%%author_id fb_group.story_kiev_ua,jeremenko_evgenia
%%date 
 
%%tags kiev,skazka
%%title СКАЗКА О НЕСБЫВШЕМСЯ ЖЕЛАНИИ или ОСТАТЬСЯ БЫ С НОСОМ!
 
%%endhead 
 
\subsection{СКАЗКА О НЕСБЫВШЕМСЯ ЖЕЛАНИИ или ОСТАТЬСЯ БЫ С НОСОМ!}
\label{sec:18_11_2021.fb.fb_group.story_kiev_ua.2.ostatjsja_s_nosom_skazka}
 
\Purl{https://www.facebook.com/groups/story.kiev.ua/posts/1800571116806348}
\ifcmt
 author_begin
   author_id fb_group.story_kiev_ua,jeremenko_evgenia
 author_end
\fi

Добрый поздне-осенний день всем!
Есть у меня очень хорошая приятельница, обладающая удивительным даром сатирически-обличительного по отношению к себе повествования. И как-то рассказала она мне свою историю, очень вкусно рассказала: просто бери и записывай. Ан нет: "Только после моей смерти",- говорит. Но история не давала мне покоя. Пришлось изменить жанр, вздохнуть об ушедшем колорите - и всё-таки запечатлеть её. Приятельница, прочитавши, поворчала, что, дескать, нос свой именно совала куда и не надо постоянно. Но я отрезала: "Это сказка, персонаж вымышленный!" К тому же нос свой она и сейчас суёт, но очень с пользой.
СКАЗКА О НЕСБЫВШЕМСЯ ЖЕЛАНИИ или ОСТАТЬСЯ БЫ С НОСОМ! 
В некотором царстве, в некотором государстве жила-была девица. Не совсем принцесса, но собой видная, роскошная, с огромными серо-зелёно-голубыми глазами и изысканно вылепленным лицом. И всё в ней было замечательно гармонично: фигура, движения, голос, решительность, сердечная доброта, мудрое понимание жизни. 
Но вот только самой девице изо всего, что ей Господь щедро подарил, почему-то больше всего ‒ ну прямо совсем! ‒ не нравился… нос. Настолько он ей досаждал своим изгибом, формой, длиной (совершенно, между прочим, соответствующими всем чертам лица), что вознамерилась наша несчастливица от него избавиться. То есть, не совсем, понятное дело, носа лишиться, но ‒ как бы так сказать? ‒ облагородить его, что ли. 
Работала непонятая собой красавица, прошедшая к тому времени серьёзную школу жизни и уже молодая мама, в медицине. А тогда, в те не очень и давние времена, не было такой повадки, чтобы всем подряд по их капризному желанию-хотению что-нибудь открамсывать или, наоборот, пришивать. И пустилась девица искать входы-выходы, чтобы согласился бы кто приличный без серьёзных на то причин укоротить ей нос. К слову сказать да и к чести этой девушки, нос она, несмотря на кажущуюся его длину, в чужие дела не совала, а, значит, сор из избы не выносила и за глаза никого не злословила. В общем, никаких убедительных доводов о неполезности своего носа привести не могла.  Но видно так уж опротивел ей этот нос, что поднатужилась ‒ поднакопила деньжат для неё немалых, отыскала через знакомых хирурга ооочень высокого уровня и отправилась в назначенный день на операцию, после которой ну всё буквально в этом мире должно было стать для неё распрекрасным, расчудесным, необыкновенно другим и заиграть-запереливаться всеми цветами радуги. 
Однако приключилось нежданно-негаданно невообразимое препятствие. Тот самый доктор ‒ ну, что ооочень высокого уровня, глянув на нашу отважную мечтательницу, решительно отказался ей нос коротить. "Не буду, ‒ говорит, ‒ и вся недолга! И уходи со своим носом от меня, не отвлекай от других носов и ушей, вон их у меня целая очередь!" Так и выпроводил несостоявшуюся счастливицу из кабинета и клиники. 
Вот уж горе так горе! Приехала носа своего лишиться, а какой-то профессоришка не захотел из неё вторую Елену Проклову сделать! Идёт наша девица, слёзы горькие льёт ‒ «Нос не обрезали!!!» ‒ и даже на почти весь мир обижается: и на друга закадычного, что к профессору спровадил за новым носом, и на самого светилу, который ну ничего в женской красоте, как оказалось, не понимает, обещаний не выполняет, и на вообще. Садится обманутая красавица в общественный транспорт ‒ в маршрутку, и, не видя никого вокруг, рыдает, слёз горючих унять не в силах. Но вдруг, почему-то перестав всхлипывать, замечает, что все люди смотрят на неё с огромным сочувствием. Молча сочувствующе смотрят. Да и то сказать! Остановка маршрутки возле серьёзной клиники. Каждый входит в неё со своими недугами, а то и с настоящим горем. Кто-то болен сам, у кого-то не здоровы близкие. Кому-то сегодня сообщили горькую новость о родном человеке…  
И так нашу умницу эти сочувствующие глаза поразили, что залились её щёчки алой краской, а в голове помчались замечательные мысли. И очень ясно поняла наша настоящая красавица, что чего стоит. И стала благодарить Бога за всё, что он ей посылал и посылает. За ясный день и дождливый, за прибыль и безденежье, за работу и безработицу, за здоровье и болезни… Поверила Господу, себя и чадо своё ему вверила и пошла туда, куда Он её повёл. 
Сказка ложь да в ней намёк, добрым девицам урок!
И всем со своими носами оставаться!
Вот такие сказки сказываются и случаи случаются в стольном граде Киеве 🙂
