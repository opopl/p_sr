% vim: keymap=russian-jcukenwin
%%beginhead 
 
%%file slova.dnestr
%%parent slova
 
%%url 
 
%%author 
%%author_id 
%%author_url 
 
%%tags 
%%title 
 
%%endhead 
\chapter{Днестр}
\label{sec:slova.dnestr}

%%%cit
%%%cit_head
%%%cit_pic
\ifcmt
  pic https://img.pravda.com/images/doc/3/7/37c5aab-4-1-litohraphia.jpg
	caption У стен Белгород-Днестровской крепости раскопали остатки жилых кварталов древнегреческой Тиры
\fi
%%%cit_text
На берегу \emph{Днестровского} лимана, всего в 540 километрах от украинской
столицы, покоятся руины древнего города Тира с историей больше двух с половиной
тысяч лет. На его месте в Средневековье появилась крепость, которая сейчас
является одной из главных достопримечательностей независимой Украины. И одной
из крупнейших крепостей в Восточной Европе. Почему одну из башен называют
\enquote{пушкинской}, какие артефакты находят археологи, при чем здесь Овидий и
что мешает крепости попасть в список мирового наследия – в репортаже УП из
\emph{Белгород-Днестровского}
%%%cit_comment
%%%cit_title
\citTitle{Білгород-Дністровська фортеця: історія і легенди для туристів}, 
Євген Руденко, www.pravda.com.ua, 23.08.2019
%%%endcit

