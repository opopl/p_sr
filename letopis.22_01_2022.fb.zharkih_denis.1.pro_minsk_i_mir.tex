% vim: keymap=russian-jcukenwin
%%beginhead 
 
%%file 22_01_2022.fb.zharkih_denis.1.pro_minsk_i_mir
%%parent 22_01_2022
 
%%url https://www.facebook.com/permalink.php?story_fbid=3172041056342623&id=100006102787780
 
%%author_id zharkih_denis
%%date 
 
%%tags donbass,minsk_dogovor,ukraina,vojna
%%title Про Минск и мир
 
%%endhead 
 
\subsection{Про Минск и мир}
\label{sec:22_01_2022.fb.zharkih_denis.1.pro_minsk_i_mir}
 
\Purl{https://www.facebook.com/permalink.php?story_fbid=3172041056342623&id=100006102787780}
\ifcmt
 author_begin
   author_id zharkih_denis
 author_end
\fi

Про Минск и мир

Вся дипломатия, включая американскую, твердит о том, что выход в Минских
соглашениях. Но раз в них выход, то почему не работает? Давайте разберем три
уровня - украинский, украино-российский и российско-американский. 

Современной политической системе Киева Донбасс не нужен даром, и даже с
доплатой. 

Современной украинской политической тусовке в страшном сне не снилось
возвращение Донбасса (да и Крыма тоже). В многомиллионном притоке избирателей,
которые не будут голосовать ни за Порошенко, ни за Зеленского, ни за любого
другого, кто будет продолжать строить политику Анти-России, украинская
политическая система не заинтересована. 

Жители Донбасса им нужны лишенные прав, покаявшиеся, пристыженные и покорные во
всем. Впрочем, почему только жители Донбасса? Все без исключения граждане,
их-то, то есть тех, что на подконтрольной территории, выгодно стыдить, лишать
прав и свобод, заставлять отдавать имущество, как раз под шарманку агрессии
России. У нас война? Так не вякайте, а делайте, что вам говорит великая власть,
и радуйтесь.  

И тут эта лафа кончается, и оказывается, что не народ должен этой власти, как
земля колхозу, а власть что-то там должна народу. Ужас! Средний украинский
политик уже совершенно разучился что-то народу давать, он всецело поглощен
войной с Путиным, поэтому вернуть его во времена Януковича, когда электорат
обычно покупали за гречку, не представляется никакой возможности.  

Военная угроза России сформировала полную безответственность в управлении
страной. Ситуация окончательно затуманилась и потеряла всякие остатки логики.
Украина, вроде, как ведет войну за Европу и остановила российские полчища, но
вторжение в Украину только предстоит. А что ранее было? А с Крымом и Донбассом
не вторжение? А что тогда? А если тогда было вторжение, то почему опять? 

Украинский политик и активист не задают подобных вопросов, поскольку научились
так жонглировать словами и лозунгами, что задурили голову не только доверчивым
украинским гражданам, но и самим себе. Порядка в голове массы управленцев
совершенно не наблюдается, а Минские соглашения это как раз попытка наведения
порядка в Украине, установления некоторых правил. Но в современной Украине нет
правил, а есть только настроения. И под эти настроения Минские соглашения не
подходят от слова совсем. Ведь придется что-то выполнять, и наделять правами и
властью граждан, которые массе нынешних политиков не нравятся. Нет, это
решительно невозможно, а потому большинство украинских политиков и экспертов
считают выполнение Минска позорным поражением. 

России срыв соглашений, в целом, выгоден. 

Украина пошла на нарушение всех договоренностей с Россией с 2014 года, а
Минские соглашения появились, когда Киев получил силовой отпор (явно не без
участия России). Получается, что заставить Украину что-то выполнять можно
только силой, но тогда из-за хрупкого плеча Украины вылазит НАТО. 

Россия долго чесала затылок, и поперла на НАТО. НАТО, лишившись прикрытия
Украины, оказалось в замешательстве. Ведь претензии предъявили не Украине, а
лично Альянсу. Если дать реальную, а не формальную команду Украине выполнять
соглашения, то тогда придется признать, что НАТО и являлось причиной
конфронтации России и Украины, а это никак нельзя. 

Россия, которая заявляет, что НАТО просто натравил на нее Украину, находится в
уязвимой позиции. Ей говорят, что она сама напала на Украину, а НАТО только
защищает украинцев от российской агрессии. Эта версия позволяет НАТО и США
сохранить лицо. Но и Россия не в накладе, она превратила Украину в антирекламу
западных реформ, что дает Москве козыри в разговоре не только с регионами, но и
с Беларусью и Казахстаном, например. Да, Украину потеряли, но зато у себя
порядок навели. 

Хорошо, Украина выполнит Минские соглашения, получит Донбасс обратно и...
начнет поднимать вопрос Крыма. А вот невыполнение соглашений приведет уже
Донбасс "в родную гавань", и вообще развяжет руки в переделе границ
окончательно.  Поэтому Россия много лет терпела клоунаду украинской власти по
поводу переговоров. Это был простой расчет, это было выгодно. 

Истинным автором русофобской истерии в Украине является США. 

Минские соглашения предполагают затухание русофобской истерии в Украине. Но тут
вопрос, а кто, собственно, эту истерию Украине навязал? Ведь антироссийские
настроения стимулировались не только в Украине, но в Беларуси, Казахстане, не
говоря уже о Прибалтике. И дело тут не в русских. 

Были и антисербские настроения, и антиарабские, и антииракские/антииранские.
Тут схема одна - какой-либо народ объявляется агрессором и завоевателем, а США
представляет свои услуги миротворца, несут демократию. А на народ-изгой
демократия не распространяется, задача этот \enquote{агрессивный} народ загнать в
резервацию. Эта схема хорошо читается в речах украинских националистов. Кто бы
мог их к этому надоумить?

И вот Украина выполняет Минские соглашения и спрашивает, а кто собственно,
довел страну до ксенофобской истерии? Хватают всяческих шестерок, а они
показывают на посольство США... Невдобно получится.  

Короче, сегодня Минск особо не нужен никому, поскольку всем от него больше
убытка, чем пользы. Ну, не совсем всем. Минские соглашения при всех своих
нюансах выгодны украинскому народу, как система наведения порядка и надежда на
мир. Но это если в Украине власть народная, если она ответственная, если она не
обезумила от безнаказанности. Если... как много этих \enquote{если}...
