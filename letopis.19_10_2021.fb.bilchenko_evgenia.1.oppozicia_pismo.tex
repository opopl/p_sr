% vim: keymap=russian-jcukenwin
%%beginhead 
 
%%file 19_10_2021.fb.bilchenko_evgenia.1.oppozicia_pismo
%%parent 19_10_2021
 
%%url https://www.facebook.com/yevzhik/posts/4356512174383859
 
%%author_id bilchenko_evgenia
%%date 
 
%%tags bilchenko_evgenia,narod,obschestvo,oppozicia,pismo,politika,rossia,rusmir,ukraina
%%title БЖ. Много праведников - одна грешница: нежное, но итоговое письмо украинской оппозиции...
 
%%endhead 
 
\subsection{БЖ. Много праведников - одна грешница: нежное, но итоговое письмо украинской оппозиции...}
\label{sec:19_10_2021.fb.bilchenko_evgenia.1.oppozicia_pismo}
 
\Purl{https://www.facebook.com/yevzhik/posts/4356512174383859}
\ifcmt
 author_begin
   author_id bilchenko_evgenia
 author_end
\fi

БЖ. Много праведников - одна грешница: нежное, но итоговое письмо украинской
оппозиции...

Наверное, я должна сказать читателям. Ведь, если кто-то на кого-то обижен,
высказанная честно обида нейтрализует яд ярости. Хочу очиститься. Хочу покоя и
смирения навсегда.

\ifcmt
  ig https://scontent-lga3-2.xx.fbcdn.net/v/t39.30808-6/247201425_4356512497717160_6756728757087930600_n.jpg?_nc_cat=106&ccb=1-5&_nc_sid=8bfeb9&_nc_ohc=vkYr4duoH8UAX_kXO6N&_nc_ht=scontent-lga3-2.xx&oh=fd514f6ab0a93d6c2a926aae298b8d4b&oe=6174C031
  @width 0.4
  %@wrap \parpic[r]
  @wrap \InsertBoxR{0}
\fi

Я очень обижена на нашу оппозицию. Все эти годы я боготворила этих людей и
считала их небожителями... искренно любящими Россию. Вот такое я вбила в
голову. Наивно, да? Оказалось, что многие из них трусят и боятся в этом
признаться, конформистски изображая центристов-русофобов, менее русофобских,
чем правые, но в меру "сознательных" по курсу. Эта полуправда называется
"адекватной позицией". То есть, я - не адекватная. Тяжело разочаровываться в
кумирах. Не сотвори их себе. Я очень пыталась им понравиться, доказать свою
преданность. И доказала! А они все не прощали мне и не прощали. А потом я
поняла, что за это время я свою душу русскому миру отдала целиком: и как поэт,
и как человек. А они не смогли. Они все благополучны и работают. Исключение -
Лесь. Могу назвать ещё пару людей. И всё. То есть, я не заметила, что, догоняя,
перегнала их в дерзости. Я поняла это, когда оппозиция стала тоже считать меня
агентом Кремля. Это не было сказано прямо (впрочем, звучало из уст Друга слово
vatnitsa), но я девочка умная, я поняла, почему меня слили. А другие за те же
слова получают эфиры и дотации. Слова надо уметь по чейкрымам комбинировать.

Когда я потеряла всё, мне пару раз позвонили бывшие друзья из ОПЗЖ и левых и
посочувствовали. Все те, на чьи суды я ходила, чьи студфорумы воспламеняла
своими лекциями, на чьих социалистов ставила всю свою судьбу. Просто раз-другой
позвонили и сказали, что "им очень жаль, но я им не подхожу". Когда-то я была
слишком недокаявшаяся, а теперь, как мне сказали, - слишком пророссийская, мол,
меня качает.

Вы знаете, когда я смотрю, как украинская оппозиция осторожно и каверзно
говорит о восточном соседе, какие гранты она получает, у кого какая крыша, на
каких курортах она отдыхает... Мне СЕЙЧАС РАДОСТНО, что меня просто выбросили в
бомжи. Мне было бы стыдно из просто глупой майдановки, реально глупой, хотя
искренной в этой глупости, стать русскоязычным русофобом-центристом. На 180
градусов, так на 180. Назад пути нет. Я обрываю этот цирк полумер.

А ещё они все просят меня убирать отовсюду мое имя, чтоб не потерять
символический капитал. Я, наверное, - дура, но я так не умею. Просто убираю
имя, чтобы не вредить людям.

Ещё я хочу сказать спасибо простому народу, не политикам: от учёных до
продавцов, украинцам и русским, которые помогали мне издалека, в то время, как
друзья по антифашистским митингам оставили в беде. Спасибо, Россия и Украина,
низовая, горняя, колокольная, звонкая. Не ожидала, ожидала, служу братскому
народу, ибо вы - мега. Альфа и Омега Цивилизации, мой последний Сталинград, как
пел Егор. ЛЮДИ ИЗ НАРОДА, ВЫ ЧУДО. НЕ ОНИ.

Я - русская. И я горжусь тем, что я русская поэтесса. Потому что я - Поэт. Не
политик "вашим/нашим". А тупо поэт. Цой в кочегарке работал. Эдичка... Лучше -
в кочегарке, чем на центристских карачках и оппозиционером в кавычках. Сказала,
- и мне полегчало. Фамилий не тэгаю: я этих людей из оппозиции все равно ЛЮБЛЮ
и буду любить, несмотря на их кидалово. Я их страх понимаю. Кому нужен
отчаянный доктор наук, свободно общающийся с "той стороной" (все это знают),
если все роли расписаны на полутонах, и нельзя - за флажки?

Депрессию я преодолею. Я прощаю их всех: и травящих врагов справа и бросивших
меня друзей слева. Россию умом не понять. И вам меня - тоже. Никогда. Я и есть
То, что в вашем воображении ест снегирей. Присмотритесь. Вот это и есть мир,
который вас пугает. Видите, в этом мире - тридцать кило тела и море пушкинской
любви... Простите и прощайте. Теперь я - просто копирайтер.  \#накипело

PS. Я все равно вспоминаю о каждом с теплом. Наши пять лет сопротивления были
прекрасны, хотя и бессмысленны, потому что я, дурочка, ждала радикальной
мужественности там, где нельзя, пока не вспоролась полностью. Целую, русский
поэт БЖ.

\ii{19_10_2021.fb.bilchenko_evgenia.1.oppozicia_pismo.cmt}
