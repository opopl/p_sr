% vim: keymap=russian-jcukenwin
%%beginhead 
 
%%file 21_05_2018.fb.lesev_igor.1.obostrenie_na_donbasse.cmt
%%parent 21_05_2018.fb.lesev_igor.1.obostrenie_na_donbasse
 
%%url 
 
%%author_id 
%%date 
 
%%tags 
%%title 
 
%%endhead 
\subsubsection{Коментарі}
\label{sec:21_05_2018.fb.lesev_igor.1.obostrenie_na_donbasse.cmt}

\begin{itemize} % {
\iusr{Марина Прохорова}

20.05.18. Стихотворение горловчанки Л. Гальцовой, написанное во время обстрела.

\obeycr
Горловчане, тише погибайте.
Не мешайте миру мчаться в лето,
Ехать к морю по дорогам новым,
Пиво заготавливать к футболу.
Зайцево и Докучаевск,
Дела нет до ваших ран и душ убитых...
\restorecr


\iusr{Стас Косаренко}
Собственно, оно уже началось. Сейчас сообщают о наступлении ВСУ на Горловку.

\iusr{Матвей Кублицкий}

Никто не сделал столько для отсоединения Донбасса, сколько Киев. Одна
железнодорожная блокада чего только стоит. перекрыли всё что можно - тем самым
вытолкнув Донбасс окончательно в зону экономики России. И никто в Киеве даже не
обсуждает возможность переговоров с Донецком....

\begin{itemize} % {
\iusr{Игорь Лесев}

все так и есть. Но в Киеве своя логика по этому поводу - Донбасс оккупирован
РФ, а значит вести переговоры там не с кем... Но что интересно, Киев и с
Москвой переговоров не ведет))

\iusr{Матвей Кублицкий}

Донецкие в итоге выборы проведут и полностью легимитизируются. паралельно может
кто помудрее в Киеве к власти придет. И тогда твой сценарий на счет союза меча
и орала. Москва точно донбасс не хочет прихватизировать: дорого и безсмысленно

\iusr{Игорь Лесев}

Выборы в Донбассе могут быть только с одобрения Москвы. Это раз. Ну а содержать
часть Донбасса России все равно приходится, точно также, как и Приднестровью и
закавказские республики. Это два. Но самое главное, что с каждым новым годом
жизни отдельно, там создается своя местная идентичность. Плюс по всей стране
куча кровников - в войну втянуто сотни тысяч людей. Жить в единой стране "по
взаимному согласию" при таких вариантов не реально. Может быть или победа одной
стороны, или фактическое разделение.

\iusr{Матвей Кублицкий}
\textbf{Игорь Лесев} значит будет фактическое разделение. пройдут выборы, потом референдум о независимости. Естессно, с согласия Москвы. И этот сценарий проталкивает Киев, а не Москва.

\iusr{Матвей Кублицкий}
\textbf{Игорь Лесев} смотря что понимать под серой зоной. у донбасса теперь основной партнер - Россия. а России без разницы серая или белая. небольшие проблемы с поездками в другие страны - но это решаемо

\iusr{Матвей Кублицкий}
\textbf{Игорь Лесев} прочитал. массовое бегство людей. только это проблема общая и Донбасса и Украины. Это общая тенденция последних 30 лет для многих стран.
\end{itemize} % }

\iusr{Елена Максюченко}

в ваших словах много разумного, но есть одно НО, такое впечатление, что вопрос
отделения на большом СТОПе, практически ничего не работает, если и запускают
громко, то только для снятия социальной напряженности, чтобы не было
стремительного оттока. Очень много людей выехало и продолжает выезжать, хотя
многие возвращаются, преимущественно пенсионеры и люди, которые не могут себя
реализовать, а ноша сьема квартиры неподьемна. Для меня все очень
неопределенно, люди обозлены на обе стороны, факт. а глобольно, все верно

\begin{itemize} % {
\iusr{Игорь Лесев}
именно неопределенность больше всего и напрягает людей

\iusr{Елена Максюченко}

еще добавлю, еще пару лет такой неопределенности и тотальной нехватки средств,
начнет валится инфраструктура и будет тотальный кадровый голод, и это реально,
я это так вижу, поскольку знаю регион и промышленность


\iusr{Елена Максюченко}
\textbf{Игорь Лесев} 

конечно, но жизнь не останавливается и люди предпринимают шаги для улучшения
своей жизни. Врачи, контингент, пенсионеры, молодняка, практически нет.
Металлурги, весь топменеджмент, пенсионеры или люди предпенсионного возраста,
список можно продолжать дальше

\iusr{Елена Максюченко}
\textbf{Игорь Лесев} 

так это уже есть, только много людей оттуда вернулось, поскольку, особенно в
Ростовской области, сильны настроения "понаехали", с работой то тоже не очень,
а тут еще конкуренция. вижу жизнь только в Донецке, достаточно нормальную, во
всех остальных городах, нет, также и в Луганске, в Алчевске завод запускают,
чтобы сделать пару заказов, у топменеджмента очень плохой настрой, многие
уезжают. и еще, не забывайте о количестве населения, даже если говорить о
Луганске и Алчевске, суммарно, плюс близлежайшие города, это, как минимум
600тысяч людей, а в Донецкой области, как минимум суммарно 1,5 миллиона. Здесь
процессы происходят быстрее, поскольку очень большой пром потенциал был и
высокая плотность населения. С Осетией сравнивать не получится, там и людей
мало и предприятий с гулькин нос. В интеграцию с Россией верю слабо, уже
показало, что это только создает напряженность, спецы у нас отличные, а это
ведет к уменьшению рабочих мест. Многие мотаются туда сюда, например, грузовые
перевозки, практически сведены к нулю, нет экономики, нерентабельно

\iusr{Елена Максюченко}
\textbf{Игорь Лесев} 

не Игорь, не складывается, зона неопределенности без правового статуса. У меня
напрашивается вывод, что будут изменения, возможно, как вы написали, кроме
противоречий, как я отметила(имхо). Но долго так продолжаться не может, мож
год, максимум два. Не забывайте у нас ведь тоже все не статично

\end{itemize} % }

\end{itemize} % }
