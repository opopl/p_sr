% vim: keymap=russian-jcukenwin
%%beginhead 
 
%%file 25_08_2021.fb.peseckij_valerij.1.nezavisimost_prodazha_lzhi
%%parent 25_08_2021
 
%%url https://www.facebook.com/permalink.php?story_fbid=5987301921342174&id=100001872762859
 
%%author 
%%author_id peseckij_valerij
%%author_url 
 
%%tags nezalezhnist,ukraina,vranje
%%title Тридцать лет продажи лжи
 
%%endhead 
 
\subsection{Тридцать лет продажи лжи}
\label{sec:25_08_2021.fb.peseckij_valerij.1.nezavisimost_prodazha_lzhi}
 
\Purl{https://www.facebook.com/permalink.php?story_fbid=5987301921342174&id=100001872762859}
\ifcmt
 author_begin
   author_id peseckij_valerij
 author_end
\fi

Тридцать лет продажи лжи.

Если подводить главный ментальный итог независимости Украины, то успешным в
стране было только одно: продажа лжи украинцам. Причем, чем наглее и масштабнее
была ложь, тем успешнее становились те, кто ее нам втюхивал! Вспомните конец
80-х начало 90-х, первый подъем национальной свидомости. На чем прорастала? На
ожиданиях, которым не было места в реальности. "Мы кормим Союз, отделимся - все
будет наше. От металла до сахара. За пять лет догоним по уровню жизни Францию".
И что? Вышли из СССР. Союз кормили, а себя не смогли. Голые прилавки 90-х,
гиперинфляция, тысячи закрытых предприятий, миллионы безработных, вся страна -
большой блошиный рынок и доктора наук - челноками в Турцию. И так не пять лет,
которые лжецы с Печерских холмов отвели для рывка на уровень Франции, а все -
тридцать. С некоторыми перерывами на время правления Кучмы и Януковича. Потом
майдан 2004, рубивший окно в Европу. Хорошо помню, как Ющенко то через месяц,
то через полгода после своего незаконного прихода во власть рассказывал, что
вот-вот и Киев примут в ЕС. И миллионы этой лжи верили. 

Даже я сомневался в своем скепсисе. Таков был гипноз обмана Ющенко и Нашей
Украины....

Про минимальные зарплаты в 1000 евро и пенсии в 500 кто только в нулевые и
позже не трындел. И верили ведь. Голосовали за обещалкиных.

В 2014 Порошенко гарантировал мир на Донбассе. Через месяц после прихода к
власти. На этом обещании выиграл президентские. И что? Через пять лет верхом на
этом же обмане въехал на Банковую Зеленский. Теперь ВСУ регулярно готовятся к
наступательно-освободительным операциям на Донбассе. С возможной
полномасштабной  войной с ВС РФ и утратой остатков государственности.

Про "мелочь" вранья о последнем и окончательном повышении цен на коммуналку за
последние семь лет я молчу. Это уже традиция у нас такая.
Президентско-премьерская. Правда в декабре 19-го Зеленский во вранье решил
переплюнуть всех разом и пообещал через год, то есть в декабре 20-го, понизить
тарифы на 40\%. И где? И что?

Если поднять и проанализировать  историю отношения власти и избирателя за
последние тридцать лет, то вся она написана языком лжи и обмана. Но самое
печальное, что видимо иного языка украинцы и не хотят знать. И пока это так, к
власти будут приходить те, кто наиболее умел в "разводе лохов" на неисполнимые
обещания.
