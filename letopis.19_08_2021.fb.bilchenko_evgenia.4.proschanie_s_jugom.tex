% vim: keymap=russian-jcukenwin
%%beginhead 
 
%%file 19_08_2021.fb.bilchenko_evgenia.4.proschanie_s_jugom
%%parent 19_08_2021
 
%%url https://www.facebook.com/yevzhik/posts/4168167853218293
 
%%author Бильченко, Евгения
%%author_id bilchenko_evgenia
%%author_url 
 
%%tags bilchenko_evgenia,herson
%%title БЖ. Прощание с Югом.
 
%%endhead 
 
\subsection{БЖ. Прощание с Югом.}
\label{sec:19_08_2021.fb.bilchenko_evgenia.4.proschanie_s_jugom}
 
\Purl{https://www.facebook.com/yevzhik/posts/4168167853218293}
\ifcmt
 author_begin
   author_id bilchenko_evgenia
 author_end
\fi

БЖ. Прощание с Югом. 

До свидания, маленький южный город. В последний вечер ты подарил мне салют на
набережной. Но, увы. Рабам пора на галеры. Мы провели здесь сказочные три дня:
это были три дня света, солнца, моря, медуз (их можно гладить, они не кусь),
теплых рук друзей, стихотворений и Спаса. Пора возвращаться в рабочие будни.
Да, бывают люди вроде Бильченко, которые морской отпуск позволяют себе всего
лишь на три дня за два года. Но это - волшебные и фатальные в своей конечности
дни. На эти три дня мне почти удалось забыть о своей болезни и о своих потерях.
Но, чем ближе отъезд, тем сильнее болят душа и горло. И погода испортилась:
предгрозье. Мой организм и природа чувствуют: пора, пора, привычно мерцает
монитор, требует, мучит. Господи, сделай так, чтобы хорошие люди дольше
оздоравливались, меньше болели, чтобы они отдыхали, как я с бабушкой в
санатории: 24 дня в год. У моря. Только у моря. Чёрное, Балтийское, Средиземное
- три моря я видала за жизнь: все прекрасные. Спасибо, Скадовск.

\ifcmt
  pic https://scontent-cdt1-1.xx.fbcdn.net/v/t1.6435-9/232736336_4168167679884977_7092580389848966211_n.jpg?_nc_cat=101&ccb=1-5&_nc_sid=8bfeb9&_nc_ohc=eNb58mZKENoAX8Oimur&_nc_ht=scontent-cdt1-1.xx&oh=8e3d604d306be0064c3f7612a525ad1b&oe=6145C415
  width 0.4
\fi

\ii{19_08_2021.fb.bilchenko_evgenia.4.proschanie_s_jugom.cmt}
