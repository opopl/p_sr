% vim: keymap=russian-jcukenwin
%%beginhead 
 
%%file 30_06_2021.fb.haperskaja_marina.1.mova_futbol
%%parent 30_06_2021
 
%%url https://www.facebook.com/khaperska/posts/3931433813646232
 
%%author Хаперская, Марина
%%author_id haperskaja_marina
%%author_url 
 
%%tags futbol,jazyk,mova,obschestvo,sport,ukraina,ukrainizacia
%%title Ви што і дома гаварітє на украінскам
 
%%endhead 
 
\subsection{Ви што і дома гаварітє на украінскам}
\label{sec:30_06_2021.fb.haperskaja_marina.1.mova_futbol}
 
\Purl{https://www.facebook.com/khaperska/posts/3931433813646232}
\ifcmt
 author_begin
   author_id haperskaja_marina
 author_end
\fi

Моя френдострічка мовно орієнтовано, тому не дивно, що "мій фейсбук" наповнений
дописами, що висловлюють жаль,  обуренням та розчарування  прес-конференцією
футболістів російською мовою.

"Як так сталося, що чудовий козак,  народжений в незалежній Україні,  не може
пару речень сформулювати українською?" - звучить справедливе запитання.

Мій син-підліток, потрапляючи в новий колектив в Харкові, щоразу чує: "Впервиє
віжу, что кто-то в жизні гаваріт па-украінскі", "Ви што і дома гаварітє на
украінскам?", "Тебя заставляют?" тощо.

Прихильники лагідної українізації переконували,  що треба почекати, мовляв,  ми
не зможемо перейти на мову,  от виросте нове покоління..

Ось виросло, ну і?

\ii{30_06_2021.fb.haperskaja_marina.1.mova_futbol.cmt}
