% vim: keymap=russian-jcukenwin
%%beginhead 
 
%%file 19_11_2020.fb.volga_vasilii.1.gorod_pripjat
%%parent 19_11_2020
 
%%url https://www.facebook.com/Vasiliy.volga/posts/2756459824671432
 
%%author Волга, Василий Александрович
%%author_id volga_vasilii
%%author_url 
 
%%tags 
%%title ГОРОД КОТОРОГО БОЛЬШЕ НЕТ
 
%%endhead 
 
\subsection{Город которого больше нет}
\label{sec:19_11_2020.fb.volga_vasilii.1.gorod_pripjat}
\Purl{https://www.facebook.com/Vasiliy.volga/posts/2756459824671432}
\ifcmt
	author_begin
   author_id volga_vasilii
	author_end
\fi

\index[cities.rus]{Припять!Город, которого больше нет}

Припять. 

Мои дети съездили в этот город-призрак на экскурсию. Я же после аварии на ЧАЭС
не был там никогда. Не могу даже до конца объяснить почему, но мне, с одной
стороны, очень хочется съездить туда и одновременно что-то меня не пускает. Я
могу даже собраться, все спланировать, но когда подойдет день отъезда, то
обязательно накануне этого дня я вспомню о каком-то «важном деле», которое вот
никак невозможно отложить, и откладывать приходится поездку. Так было несколько
раз. Больше я не планирую. 

Дети приехали полные впечатлений. Сделали много фотографий. Одна из них --- этот
спортзал, когда-то полный радости, напряжения, смеха, спорта… 

Но более всего их поразило то, что город этот был заложен в 1970 году, а уже в
1974 его начали заселять люди, и в 1979 году он получил статус города, т.е.
более тридцати тысяч человек получили в нем бесплатное жилье. 

За восемь лет, с нуля, с дикого леса, с болота был построен просторный
город-сад. Были построены школы, больницы, парк развлечений, спортзалы,
огромный бассейн. Весь молодой красивый город, полный молодых красивых людей,
полный детского смеха и надежд на будущее был обсажен розами и разукрашен
цветниками. Всего же в Припяти на момент аварии на ЧАЭС жило пятьдесят тысяч
человек. 

Пятьдесят тысяч надежд на будущее. Руководство города приняло решение посадить
пятьдесят тысяч кустов роз. По одному кусту на каждого жителя. К моменту аварии
было посажено уже более тридцати тысяч кустов. Каждый человек шёл утром на
работу и смотрел на кусты роз и думал: это мой!

К моменту аварии… 

Без объявления войны в один только день было принято решение о тотальной
эвакуации города. Людям говорили, чтобы они ничего не брали с собой, и что
через два дня они все вернутся по своим домам, чтобы с собой люди брали только
документы и никакой одежды, кроме той, что была на них по сезону. Был почти уже
май. Было тепло. Чемоданы брать с собой не разрешили. 1200 автобусов
бесконечной змеёй вывезли всех до единого жителей Припяти. 

Я постоянно пытаюсь представить себе психологическое состояние этих людей,
которые еще только вчера строили планы на выходные, собирались на рыбалку и
шашлык, делали карьеру, а сегодня их вместе с детьми усаживают в автобусы и
развозят за сотню километров от дома, временно расселяя по базам отдыха и
дешевым гостиницам. И я не могу себе этого представить. Мне говорят --- не надо,
не представляй, беду накликаешь. Но как не представлять, если беженцы в нашей
стране стали обыденным постоянным явлением. Не десятки, а сотни тысяч людей,
наших сограждан, так же вынуждены были оставить свои дома и бежать с линии
соприкосновения на Донбассе от наших же танков, от нашей же армии, от наших же
добробатов. 

Как здесь не думать? Есть ли гарантия того, что это не повторится уже здесь --- в
Киевской области, или в Харьковской, или в Одесской? Ведь и здесь многие думают
так же, как думает Донбасс, так же, как думаю я, моя жена, мои дети, а не так,
как та наша делегация в ООН, которая проголосовала против осуждения нацизма,
которая проголосовала за нацизм. 

Есть гарантия того, что они не придут в наш дом? Нет. 

О многом мы говорили с детьми, которые приехали с экскурсии по городу-призраку. 

\ifcmt
tab_begin cols=2

	pic https://scontent.fiev6-1.fna.fbcdn.net/v/t1.0-9/126298129_2756460234671391_564762743080495980_n.jpg?_nc_cat=109&ccb=2&_nc_sid=730e14&_nc_ohc=nzOZX8MLuggAX_CTbSb&_nc_ht=scontent.fiev6-1.fna&oh=54f8109e4c83da590ada4b6b1b65046c&oe=5FEBFDE1

	pic https://scontent.fiev6-1.fna.fbcdn.net/v/t1.0-9/126346123_2756460224671392_8114553818780917467_n.jpg?_nc_cat=109&ccb=2&_nc_sid=730e14&_nc_ohc=nmMIPbc-qg8AX-XPfyh&_nc_ht=scontent.fiev6-1.fna&oh=0e642f9678a3ae2839092547aeb006d3&oe=5FEC0318

tab_end
\fi

Во время экскурсии им рассказали о том масштабе экстренных аварийно-технических
работ, которые были проведены по ликвидации аварии и её последствий. Рассказали
о том, каким невероятным и героическим напряжением всего Советского народа
удалось залатать ту страшную дыру в ад, которая зияла из разорванного в клочья
энергоблока, выбрасывая, подобно вулкану, смертельные дозы радиации и заражая
этой дрянью целые страны. И задались мои дети одним вопросом: «Папа, а что
будет, если не дай Бог (даже говорить об этом страшно), нечто подобное
произойдет сегодня? Ведь если мы не умеем так строить города, как умели это
делать те удивительные люди в той удивительной Стране, то сможем ли справиться
с такой бедой, с которой смогли справиться они?».

Вот такой ещё один опасный вопрос поставили мне мои дети. 

Я ответ, конечно, знаю. Писать не буду. Чтобы не накликать.
