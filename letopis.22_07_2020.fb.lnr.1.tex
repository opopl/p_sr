% vim: keymap=russian-jcukenwin
%%beginhead 
 
%%file 22_07_2020.fb.lnr.1
%%parent 22_07_2020
 
%%endhead 

\subsection{Вандалы забросали камнями Икону Блаженной Ксении на Храме в Золотом Городе Киеве}
\label{sec:22_07_2020.fb.lnr.1}
\url{https://www.facebook.com/groups/LNRGUMO/permalink/2866225903488965/}

\vspace{0.5cm}
{\small\LaTeX section: \verb|22_07_2020.fb.lnr.1| project: \verb|letopis| rootid: \verb|p_saintrussia|}
\vspace{0.5cm}

Необходима помощь в восстановлении святыни.

Вандалы забросали камнями икону на храме блаженной Ксении Петербургской на
Святошинском кладбище г. Киева.  Об этом сообщил настоятель храма протоиерей
Глеб Помелов 19 июля 2020 года.

«За 12 лет жизни храм блж. Ксении Петербургской на Святошинском кладбище
перенёс немало испытаний...

Были неоднократные попытки взлома.  Были удавшиеся ограбления. Был пожар в
соседнем здании, на расстоянии менее метра от храма, от которого вокруг пылали
сосны, и плавился металл, но церкви огонь даже не коснулся.  Многое было, но
такое, как в этот раз --- впервые.

Вандалы посягнули на саму блаженную Ксению, камнями бросали в её образ и
повредили стекло и икону», --- рассказал протоиерей Глеб Помелов.

На восстановление святыни понадобится около 6 тысяч гривен. Настоятель храма
просит верующих помочь в сборе средств.  «Обращаемся ко всем неравнодушным к
этой великой святой: помогите восстановить икону Ксенюшки.

Нам надо собрать 5-6 тысяч гривен. Очень хочется её восстановить к 29 июля –
дню памяти Христа ради юродивых схимонахинь Серафимы и Марии.  Также это день
рождения нашего храма, в этот день в 2008 году состоялась первая Литургия», –
сообщил священник.  Перечислить пожертвование можно на карту ПриватБанка: 5168
7422 2838 1459.  Получатель: Помелов Глеб Викторович.
  
