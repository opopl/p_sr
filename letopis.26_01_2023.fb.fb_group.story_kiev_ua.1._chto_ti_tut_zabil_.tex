%%beginhead 
 
%%file 26_01_2023.fb.fb_group.story_kiev_ua.1._chto_ti_tut_zabil_
%%parent 26_01_2023
 
%%url https://www.facebook.com/groups/story.kiev.ua/posts/2124393444424112
 
%%author_id fb_group.story_kiev_ua,chudnovskij_roman.kiev
%%date 26_01_2023
 
%%tags kiev,gorod
%%title Что ты тут забыл? - спросил кто-то осторожный внутри меня, - когда нужно быть в укрытии!
 
%%endhead 

\subsection{Что ты тут забыл? - спросил кто-то осторожный внутри меня, - когда нужно быть в укрытии!}
\label{sec:26_01_2023.fb.fb_group.story_kiev_ua.1._chto_ti_tut_zabil_}
 
\Purl{https://www.facebook.com/groups/story.kiev.ua/posts/2124393444424112}
\ifcmt
 author_begin
   author_id fb_group.story_kiev_ua,chudnovskij_roman.kiev
 author_end
\fi

\enquote{Что ты тут забыл?} - спросил кто-то осторожный внутри меня, - \enquote{когда нужно быть
в укрытии!} На Подоле вокруг было пустынно - единичные фигурки людей. С утра -
воздушная тревога, станции метро переполнены. Только что на Контрактовой среди
толпы укрывшихся от ракет людей кто-то пел, и ему аплодировали...

\enquote{Но ведь наконец пошёл снег! - шепнула незримая, тоже живущая в моей душе и
склонная к авантюрам творческая личность. - Подол под снегом - это же что-то!}

Одним словом, художник во мне победил, и я решил быстро пробежаться в сторону
Почтовой площади.

Зима на Подоле - это всегда зрелище для эстета, порой пейзаж вне времени: то ли
19-й век, то ли 21-й... Но тут мессенджер сообщает о прилетах и чьей-то
гибели...

Идешь - с тревогой в  сердце - и быстро щёлкаешь мобильным уголки старого
Киева. Медленно и красиво падает снег.

У Житнего рынка тихо, кафешки украшены, но - никого внутри... Заворачиваю к
Флоровскому монастырю. Там голуби у кормушки, да монашка вышла на крылечко...
Вдруг - чудесный довольно упитанный котяра вальяжно выбрался на свежий снежок и
пошел осматривать свои владенья... 

А я бегу дальше. Снова заглядываю на Андреевский. Недавно - в густом тумане -
он был похож на акварель в манере \enquote{по мокрому}, сейчас - на гуашь: то там, то
здесь тронут он белыми мазками небесного художника... Несмотря на опасность,
Андреевский живёт своей жизнью: вон кто-то, несмотря ни на что, делает ремонт,
а там парочка изучает афишу театра \enquote{Колесо}, у Андреевской церкви продавец на
лотке разложил свой товар, а девушка с рюкзачком фотографирует замок Ричарда...

А снег идёт - и кажется это не снежинки слетают к нам, а ангелочки: словно
целое небесное воинство ниспослано нашему любимому городу, дабы защитить Киев и
всех-всех людей от всего страшного, что случилось с ними наяву...
