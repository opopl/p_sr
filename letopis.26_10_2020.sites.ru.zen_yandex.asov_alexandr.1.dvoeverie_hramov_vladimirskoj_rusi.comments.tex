% vim: keymap=russian-jcukenwin
%%beginhead 
 
%%file 26_10_2020.sites.ru.zen_yandex.asov_alexandr.1.dvoeverie_hramov_vladimirskoj_rusi.comments
%%parent 26_10_2020.sites.ru.zen_yandex.asov_alexandr.1.dvoeverie_hramov_vladimirskoj_rusi
 
%%url 
 
%%author 
%%author_id 
%%author_url 
 
%%tags 
%%title 
 
%%endhead 

\subsubsection{Комментарии}
\label{sec:26_10_2020.sites.ru.zen_yandex.asov_alexandr.1.dvoeverie_hramov_vladimirskoj_rusi.comments}

\begin{itemize}

\iusr{Альберт Денисов}

Небольшое дополнение к теме. Храм Покрова на Нерли. В своё время я обнаружил,
что цокольная часть или просто нижняя часть храма выполнена из более старого
белого камня. Отличался камень и по цвету и по степени разрушения. Уже позже
удалось найти причину этого. Недалеко от храма ближе к берегу когда-то в
древние времена стоял древний каменный храм. Его впоследствии разобрали и
пустили на строительство храма Покрова на Нерли. А верхнюю часть храма была
выполнена из нового строительного материала. Фундаментная часть древнего храма
сохранилась, находится под землёй. И что самое интересное в месте нахождения
древнего храма находится аномальная зона с периодическими выбросами энергии. Я
называю такие зоны "энергозонами". Энергетические зоны такого типа полезны для
здоровья человека. Похоже, что древний храм поставлен специально над
энергозоной. А древние сунгирцы и владимирцы часто посещали это место.

\iusr{Николай Олегович}

Альберт, и для чего это делали?

\iusr{Альберт Денисов}

Николай Олегович, я не понял вопроса. Кто, чего делал? Предполагаю, что этот
вопрос относится с старому храму, установленному над аномальной энергозоной.
Этот способ известен с очень давних пор. Храм имел, вероятно форму пирамиды.
Пирамида - это конденсатор-усилитель выбрасываемой энергии сложной формы. Как
пример, пирамида Хеопса. Стояла раньше над геопатогенной сильной зоной. Служила
чёрным атлантическим жрецам для тёмных дел. Есть более простой вариант. На
таких зонах устанавливали плоские каменные валуны определённой породы. Это
чистые усилители. Пример, Синь-камень в Переславле-Залесском, Девий камень и
Гусь-камень в Коломенском парке. Энергозоны-искусственно созданные Высшим
Космосом зоны, для получения информации о состоянии глубинных недр Земли. Как
всё это устроено - неизвестно. Зоны все имеют круговую форму. Есть одиночные
зоны, есть спаренные, а есть целые поля. Воздействие этих зон фиксируются в
разных точках планеты. Пример, перевал Дятлова, урочище Белые Боги, Урочище
Шушмор, энергополе в Аркаиме, энергополе в Кабардинке-Цемесская бухта,
Бермудский треугольник, поле рядом с Кайласом...... в одной Москве имеется
несколько таких зон, но они относительно слабые. Некоторые зоны в России
исследовал лично. Вот и всё.

\iusr{Валерий Баженов}

Альберт, благодарствую за пояснения. Мало пока нас, интересующихся и убеждённых
в древних корнях нашей Родины.

\iusr{Алексей В.}

Альберт, древние сунгирцы и владимирцы не могли посещать древний храм, т.к. до
храма Покрова -на Нерли там не было и не могло быть никакого храма. Это место-
заливная пойма рек Клязьма и Нерль, причем вода каждую весну поднимается на
несколько метров. Андрей Боголюбский построил храм в память об умершем от ран
сыне после его похода на Волжскую Булгарию. Причем храм (согласно летописям был
построен в одно лето, а для этого, после спада половодья был насыпан
рукотворный холм высотой не менее 10 м. К зиме строительство было завершено.
Подумайте сами, мог ли там находиться древний храм, который бы затопляло каждую
весну на 5 метров. А темные нижние камни известняка могут свидетельствовать о
притоплении нижнего пояса храма аномальными разливами рек, хотя при моей
жизни(67 лет) и жизни моего отца и деда такого ни разу не случалось. Живу в 5
км от храма. Все вами написанное о "патогенных зонах древнего храма" и т.д.
будем считать вашей выдумкой.

\iusr{Альберт Денисов}

Алексей В., насмешил! Древний храм, также как и Покров на Нерли находятся на
высоком месте. Насыпали ли под храмы грунт или это высокое место природное-не
знаю. Но в весеннее половодье храмы как бы оказывались на острове. Кругом была
вода. Я прожил во Владимире много лет, да и дача находится в 8 км от Покрова на
Нерли. Поэтому многократно видел разлив Нерли и Клязьмы, и не разу не видел,
чтобы вода подтопило это место, даже близко. А что касается Андрея
Боголюбского, то он никакого отношения к храму не имел. И князем он не был. Его
же свои люди убили за жестокий характер и унижения. И похоронен он не в
Успенском храме Владимира, а в пойме Клязьмы. Там его могила и его семейных. А
в Успенский храм в раку положили какого-то азиата. Читайте меньше извращённую
историю России, написанная неграмотными монахами.

\iusr{Алексей В.}

Альберт, это ты насмешил своими "бреднями" знатно. Я же написал, что живу почти
70 ле у храма и все мои предки здесь жили. Конечно, храм притапливало не в те
несколько лет, когда ты ездил на дачу(в Лемешки, Оргтруд или Грезино). Это
могло быть в 14-ом 15-ом, или начале 19-го века. Но, интересно другое. Откуда
вы черпаете всю эту чушь, о том что Андрей Боголюбский князем не был? А кто же
по-твоему построил Успенский собор, Боголюбово, Золотые ворота? Князь Андрей по
праву считается первым великороссом и основателем России здесь на
Владимиро-Суздальской земле. Неужели, живя здесь, несовестно так пакостить
своих славных предков, прославивших эту землю? Да князь Андрей был
"самовластцем", но только так в те времена и можно было объединить русские
земли, что было поперек горла суздальским , ростовским и части владимирских
бояр, устроивших заговор и убийство. Список убийц(так и написано убийц, а не
обиженных жестокостью людей) не давно был открыт под более поздними фресками
древнего храма в Переславль-Залесском. Откуда вы такие, как ты, только беретесь
со своей"новой" извращенной историей? Вы бы лучше настоящую историю своего края
изучили!! И никакого азиата в его раке нет, это подтасовка Герасимова. Недавнее
изучение черепа князя показало, что внешность его была европеоидной.

\iusr{Валерий}

Альберт, вы правы, но одно уточнение: так как у наших предков до
насильственного крещения не было религии, то храмы выполняли совсем другую
роль. А именно, служили местом обучения детей, то есть были буквально храмами
науки и действительно строились на особых местах выхода потоков энергии(эфира,
по Менделееву), что способствовало процессу обучения. Форма же практически
сохранилась без изменений(только добавили иконостас), так как
форма(архитектура) здания способствовала теперь уже оболваниванию верующих.

\iusr{noalive noalive}

Алексей В., жил там же. Полностью согласен.

Кстати, во времена Андрея Боголюбского Нерль была куда полноводнее, даже её
приток - река Каменка в Суздале была судоходной. Во времена сунгирцев Клязьма
из-за обильного питания ледниками и промерзания грунтов была шириной несколько
километров.

\iusr{Алексей В.}

noalive, даже во Владимире в устье Лыбеди(которая сейчас в трубе) и Рпени
заходили на стоянку тогдашние, пусть и небольшие по нынешним меркам, суда. А по
поводу искусственности насыпанного холма под храмом - это легко определяется по
структуре почвы. Почва на холме совершенно другой структуры, чем в пойме, т.е.
привозная. Это было определено еще сто лет назад. Только такие вещи невдомек
сегодняшним "писателям". Рад был услышать ваше мнение разумного человека.

\iusr{Отличные друзья}

Альберт, вы хрень говорите, про разные места добычи камня слышали?, Даже в
Корраре, нет двух одинаковых блоков !

\iusr{lnetsha}
отредактировано

Алексей В., Вы пишите, что живёте в 5 км от Покрова на Нерли, хорошо знаете эти
места. Я из тех, у кого дача 50 лет в Грезино, но не в этом дело:). В бывшем
посёлке Оргтруд (ныне р-н Владимира) вдоль Клязьмы тянутся огороды и часть
дороги между ними, параллельно берегу, выложена "белым камнем", известняком.
Месторождений вблизи нет, а камушки крупные. Дальше есть небольшой кусок
похожих камней вмонтированных слоями"в берег" напротив друг друга через реку.
Может быть Вам что-нибудь известно?

\iusr{Алексей В.}

lnetsha, У меня, как раз, в Оргтруде на берегу Клязьмы бала дача. Знаю эти
места не понаслышке. А камень этот -известняк. Владимирская область очень
богата им. Самые крупные карьеры: Мелехово(Ковровский р-н и
Ликино(Судогодский). Неслучайно все белокаменные соборы во Владимире,
Боголюбово, Покрова-на-Нерли, Золотые ворота построены из него. А камень вдоль
берега: скорее всего им укрепляли дорожки, берег, подходы к воде. Это остатки
камня от строительства дорог. Раньше все дороги мостили известняком: крупным, а
потом мелким щебнем. Когда-то вся дорога от Лемешков до Оргтруда была из белого
камня(асфальта не было), поинтересуйтесь у старожилов. Вот оттуда и камень.

\iusr{Сергей Ясев}
отредактировано

Альберт, так Асов же пишет , что Боголюбский был двоеверцем, как впрочем и
практически все восточные христиане того времени, носившие кресты с двойной
символикой, таки и христианский храм в пойме он стало быть построил на месте
какого- то высокого старого кургана( холм ) - могильника. Противоречий нет.
Христианские фигуры были на эти храмы дополнены к "языческим сюжетам"....на
других подобных храмах " языческая" резьба была стёсана и зачищена, для
камнетёсов это не проблема, проблема для них была вырезать христианские
сюжеты...поэтому стены так и остались "голыми".

\iusr{lnetsha}
отредактировано

Алексей В., Спасибо :) Когда нам дали эти участки, то мы ходили от электрички
до Грезино пешком по каменной дороге (остатки её ещё недавно были между
кладбищем и церковью в Лемешках, как сейчас не знаю, езжу на автобусе)), вся
дорога была выложена разнокалиберным булыжником, потом, уже на моей памяти,
положили асфальт, но там известняка не было, камни с кулак и больше,
разноцветные после дождя :). Только на том участке в Оргтруде был этот кусочек
белокаменной дороги, а учитывая, что Оргтруд - бывшая Лемешинская мануфактура,
вроде бы появилась в начале ХХ века, просто интересно откуда и куда этот,
несколько не рентабельный. путь. За последние лет десять этот известняк
капитально ухайдокали машинами. :)

\iusr{Николай}
отредактировано

Интересная статья. Почему то так и думал, что должно было быть как то так.
Думаю, что христианство тяжело вливалось к славянам и должно быть куча
подтверждений этому!!! Не может народ (любой!!!) Просто так взять и поменять
веру в одночасье! Это какое же должен иметь был влияние Владимир, чтоб вот так
все как в истории! Учитывая и то, что староверы до сих пор ещё не признают
современную церковь.

Автор, коли уж занялись этим вопросом интересно было бы чтоб вы углубляться в
этот вопрос. Я по поводу существования церквей, на купола которых вместо
крестов были полмесяца, причём с разным расположением полмесяца (вроде как
разные школы). Интересно будет почитать

\iusr{Звездалет Сигареты Вега}

Николай, на верхушке русских храмов это не месяцы. И вообще не крест, те крест
не в современном понимании. Это изображение корабля с мачтами. Корабль церковь,
Христос кормчий, ловец душ в море грехов, и тд.

\iusr{Андрей Новоженин}

Николай, напомню: староверы - это христиане, а не язычники. Для язычников они и
были современной церковью. Статья, на мой взгляд, свидетельствует не о тяжёлом,
а об органическом переходе от одной системы ценностей к другой.

\iusr{Тамара Копейкина}

Наконец-то! Огромное спасибо автору за поданный материал. Всё верно здесь
сказано. Ведические храмы были на Руси, это потом на них крестов наставили.
Русь никогда не была религиозной настолько, что считать себя рабой Божьей.
"Бог-то, Бог, да будь сам не плох", "Богу молись, а к берегу гребись" - русские
пословицы, их не счесть в русском языке. У нас свои были Боги и мы их славили,
а не вымаливали себе что-то. Потому что славяне знали о ЗАКОНЕ СОХРАНЕНИЯ ВСЕГО
на земле: сколько от чего-то отымется, столько же к чему-то прибавится. Даже
здоровья своим детям они желали так: "У сороки заболи, у вороны заболи, а
дитятки заживи". Поэтому ходить в церковь и молить о здоровье не стоит, хотя
Бог даст здоровья, но он же и отнимет его у кого-то из близких, например, у
детей.

\iusr{Николай Олегович}

Тамара Копейкина, а кресты у нас были и они везде и с самого начала! Это надо
знать - почему, зачем и для чего! От Солнца!

\iusr{Лилия Матвеенко}

Тамара Копейкина, еще говорят, когда например ребёнок поранился, ударился
(ушибся) : У кошки боли, у собачки боли, а у Поленьки (Санечки, Ванечки)
заживи!

\iusr{Оксана *}

В киево-печёрском патерике есть сюжет где ветхий завет объявляется дьявольской
книгой. В русском православии ветхий завет считался противоположным по смыслу к
евангелию, и раньше не издавались одной книгой. Священным писанием считалось
только евангелие. Сейчас всё переделали, и раньше наверное много было таких
тихих реформаций с целью скрыть и уничтожить ведическое наследие.Белокаменные
храмы тоже перестраивали, но даже сейчас видна ведическая основа. Благодарю,
очень интересная статья!

\iusr{Сергей Очкивский}

Существующую русскую историографию с треском разрушает ОДИН только артефакт -
«Иерусалимский крест», хранящийся в ризнице собора в Гильдесгейме (Хильдесхайм,
Германия, Нижняя Саксония). ИЕРУСАЛИМСКИЙ ПАТРИАРХ ИОАНН ДАРИТ его КАРЛУ
ВЕЛИКОМУ (742-814 гг. н.э.). КРЕСТ, ИСПИСАННЫЙ РУССКИМИ НАДПИСЯМИ: «СЕ КР СТО
ЧЬСТЬНУ, ГРОБ СТ ГО ДАНИЛА, ГРОБ СТ ОЕ ПЕЛАГИЕ И СТ ГО САВЫ, ГРОБЪ ЛАЗОРЕВ,
ОДРЪ СТОЕ БЦ Е, АН ГЛОВА СТОПА, ПОСТЬНИЦА ГН Я, ГРОБЪ КОСТЯНТИНА И ЕЛНИ, СЕ
ГРОБЪ ГН Ь И ГРОБЪ СТЛЕ БЦ И, ГРОБЪ ИОАНА КУЩНИКА».

Этот АРТЕФАКТ свидетельствует о том, что русские имели письменность на полвека,
как минимум (863 г., как считается), РАНЬШЕ. А так же о том, что Русь приняла
христианство ЗАДОЛГО до ЗАКОНОДАТЕЛЬНО утверждённой даты этого ПРОЦЕССА, в 988
г. князем Владимиром. Нынешняя концепция истории России полностью ИГНОРИРУЕТ
нижеследующие свидетельства многочисленных древних исторических источников,
которые :

- Подтверждают историческую достоверность крещения Руси Андреем Первозванным
(одним из 12 апостолов времён ИХ) ;

- Андрей Первозванный не только крестил древнюю Русь, но и правил там, т. е.
его можно с полным основанием назвать царём Руси, либо её части;

- Во времена Андрея Первозванного Рим находился на СЕВЕРЕ Руси.

А вот уже более поздние западноевропейские историки, в том числе и написавшие
историю до романовской России, стали подвергать эти сведения, сомнениям, не
утруждая себя аргументацией.

Христианство ЗАРОДИЛОСЬ СРЕДИ РАБОВ. В романо-германской языковой группе
НАПИСАНИЯ СЛОВ РАБ И СЛАВЯНИН, ФАКТИЧЕСКИ ТОЖДЕСТВЕННЫ! Поэтому зададим вопрос,
который НЕ ПОДНИМАЕТСЯ в нынешней истории, почему не рассматривается гипотеза
зарождения христианства ОТ СЛАВЯН?В Многочисленных средневековые документы
повествуют о существовании на Востоке, огромного и сильного христианского
царства, управлявшееся могущественным монархом, Пресвитером (глава,
одновременно, религиозной и государственной власти) Иоаном. До сих пор вызывает
недоумение и даже раздражение тот факт, что православие является фактически
государственной религией. Титул восточного верховного правителя, соединяющего
духовную и светскую власть – Калиф или в форме Калифа. В русских документах
даже XVII в. встречаются фразы: «Того они Папу чтут, как мы Калифа». Есть и
реальный персонаж в нашей истории - Иван Калита. Второе слово после имени
почему-то трактуют, как прозвище (мошна, кошелёк) . А ведь в старом написании
буквы Т и Ф – очень похожи. Так не логичней ли просто считать его
правителем-священником?

\iusr{борис корин}

Сергей, Текст на древне славянском. И что? Почему это не может быть болгарский
язык? Он до сих пор сохранил архаичность, законсервированный турецким
нашествием. Зачем сочинять что-то.....

\iusr{Михаил Дрюков}

Прочитал статью, стал смотреть свои старые фото и вот 2015 год:

\ifcmt
  ig https://avatars.mds.yandex.net/get-zen_pictures/3435364/2135289-1604916052125/orig
  width 0.2
\fi

\iusr{Сергей Сычев}

Собор о котором идёт речь жители города Владимира называют не иначе как
"Дмитриевский", поэтому название которое применяет автор, а именно
"Дмитровский", как говорят для местных "режет слух". Можно отметить, что и сам
Великий князь Владимирский Всеволод Большое гнездо вместе с детьми изображён в
камне на стенах собора, среди всего того о чём говорит автор. Великое творение
"домонгольской Руси", потом так не делали. Где-то читал, что кто-то из учёных
мужей делал соскобы со стен собора. Было установлен, что собор возможно
когда-то был разноцветным, интересно было бы взглянуть хотя бы на
реконструкцию.

\iusr{Elena Kanivec}

Сергей, ничего себе соскобЫ, почти от крыши и до земли. Сохранилась лишь
верхняя окантовка, местные гиды уверяют, что раньше резной камень был до земли,
зачистили стены к приезду кого то из Романовых, совсем недавно 18 -19 веках.

\iusr{Валерий Баженов}

У нас до сих пор сохранилось множество храмов, эксплуатируемых ныне
попами-язычниками, не относящимися изначально к христианству.

В одном только бывшем Ленинграде столько ведических символов!

\iusr{Людмила Лагунова}

Валерий Баженов,"... до раскола на религии жили совсем с другим смыслом жизни,
парируя удары природных апокалипсисов.

На самом деле вера Руси была астрофизической и атеистической.Вера-ведание РА.
РА - это кольцо 13 эклиптических созвездий,

по которым при помощи креста рей апостола Андрея определялись координаты".
Вячеслав Куланов" АЗ БУКА ИЗТИНЫ"

\iusr{Elena Kanivec}

Валерий Баженов, ничего себе "бывший Ленинград", ну и залепуха. Ленинград
никогда не будет бывшим, пока город стоит, поаккуратнее с такими вещами.

Вы часом ведические символы с массонскими не перепутали?

\iusr{Анатолий}

Христиансто на Руси появилось только в 19 веке, и насаждалось как говориться
огнём и мечом, даже установили в уголовном кодексе уголовное наказание за
непосещение церквей. А Храм Покрова на Нерли, всегда был Ведической
Православной Веры, и только в 19 веке попы захватили его и оборудовали под
христианскую религию, заложив одну из дверей. Изначально в русских храмах
всегда было четыре двери по четырём сторонам света, люди приходили в храмы
разговаривать с богами, а теперь приходят грехи замаливать.
\url{https://clck.ru/FDzbX}

\iusr{Виктория *}
отредактировано

Да понятно, что всегда на Руси строили храмы и иконы рисовали, это красивейшая
ветвь словенского искусства, на которую просто наложили христианство и даже
название менять не стали- ПравоСлавие - так и оставили, как было. Иконы
рисовали всем "языческим" Богам - Сварогу, Даждь Богу, Марене, Коляде, Макоши и
т.д.

\iusr{В.А.У}

Русское двоеверие и сейчас существует.Преподобному Сергию Радонежского низкий
поклон за все его труды.Даже католик-Никон не стер из памяти народа любовь к
своим Богам,ибо она в сердце и в крови нашей.

\iusr{Ирина Федотова}

Какое "русское" язычество? Славянское, варяжское, балтское, угро-финское?
Русские появились через сто примерно лет после принятия православия и при
помощи этого принятия. А до этого были русы-варяги и другие отдельные языческие
племена, каждое со своей верой.

\iusr{Лев Давыдов}

На кресте собора Святой Софии в Великом Новгороде "сидит" голубь, но такой
развитой мифологемы вокруг него нет. Есть только старинное поверие, что сидит
голубь до тех пор, пока стоит сам город.

\iusr{ТИМУР ИВАНОВti}

не поиму одного,зачем велосипед придумывать,автор! кто Христианскую веру принес
на Русь? кто строил и учил строить из камня такие храмы? не проще ли у них
узнать,что за узоры и тп .далеко ходить не надо! и хоть один храм, построенныи
ими,унесло водои или еще чем?что что,но в строительстве из камня им равных
нет,это сугубо личное мнение.самые старинные храмы в россии из камня-они на
Кавказе, 9 век,Сентинскии.у него ни однои трещинки нет.там же Шаонинскии и
Архызскии.

\iusr{Андрей Илларионов}

В Африке на христианских храмах на иконах изображали негров и слонов, а в Сиаме
- тигров. Адаптация к местным условиям, так сказать. То же было и на
Владимирщине. Строили христианские храмы с учетом местного колорита. Не
исключено, что для популяризации христианства были к храму приделали и
языческие символы, т.к. тогдашние служители в этом не видели крамолы, лишь бы
народ в церковь ходил А вот то, что язычники каменные храмы не строили - это
факт. Первые храмы на Владимирщине стали строить со времен Андрея Боголюбского,
после того как тот завел дружбу с Фридрихом Барбароссой. Ну просто не знали
наши предки до общения с европейцами каменного строительства! Кругом было много
дерева, и стоить камнем было очень непрактично и нецелосообразно.

\iusr{Елена Веда}
отредактировано

Очень интересная статья. А что означает полумесяц на кресте, поясните
пожалуйста.

\iusr{Славян Демонов}

Елена Веда, полумесяц на христианском кресте и перекрестья на верхней и боковых
перекладинах, это часть круга... От древнего символа - равносторонний крест в
круге... Но мы теперь видим только полумесяц)...

\iusr{Елена Веда}

Славян, если вы имеете ввиду колесо Сансары или знак Перуна, то этот полумесяц
мало чем его напоминает. Он больше похож на перевёрнутую Лунницу. Хотя кто
знает.

\iusr{Степан Семенов}

на стенде в Дмитровском соборе-перевод со старорусского-,, али не найдем среди
мастеров русских толковых кто многие к пьянству и воровству склонны,то искать
требуем среди немчин мастеровых...итд. НИ ХРЕНА НЕ ИЗМЕНИЛОСЬ!

\iusr{Александр Рыбаков}

Ведь и храмы строились на месте языческих капишь, что из этого? Всё, что здесь
описано - это только внешний орнамент храма, т.е. украшение. Наверно, для
вчерашних язычников важно было показать хоть такую преемственность.

Всё, что осталось от языческой традиции (наряды, песни) надо бережно хранить.
Но играть в язычество, отказываясь от Христианства - очень глупо. В язычестве
мы были кривичами, радимичами, полянами, древлянами и т.д., в Христианстве мы
стали Великим русским народом.

\iusr{Илья Иванов}

Даждьбог, Велес, Перун... Что за фантазии? Откуда они могли бы быть во
Владимире? Это древний город марийскоязычных "меря"- велет (то есть областной
центр) древней Волжской Булгарии, до завоевания крестоносцами носил другое
название. Эти древние храмы - булгарские храмы, они приспособлены под церкви
после захвата "могол-татарского" славяно-тюркскими крестоносцами. Даждьбог,
Велес, Перун - это боги славян, то есть украинцев и поляков, "меря" не являются
славянами, это волжские финны.

\iusr{Ольга Першина}

Илья, да, это были мерянские земли и города,культ Велеса тоже был. Раскопки это
подтверждают. Травкин П.А. "Язычество древнерусской провинции"

\iusr{Николай Никишин}

Илья, а булгары кто?

\iusr{Elena Kanivec}

Илья, ну и каша, конкистадоров забыл, а так все о кей, не порти людям приздник,
вали отсюда.

\iusr{Илья Иванов}

Николай, "бул" - из языка мари и мордвы означает "сторона, край, страна", а
"ар", "арий" означает "человек". Большинство финно-угров считают себя "ариями".
Это ма-арийцы (марийцы), мордва, и удмуртов тоже в старину называли просто
"арами". Т.о., "булгары" означает всего лишь гражданство страны Булгария (то
есть "страны ариев"). Такой национальности "булгар" не было и нет.

\iusr{Николай Никишин}

Илья, насчёт булгар у меня у меня другая версия, более правдоподобная, а не
притянутая за уши. Между прочим финно-угры, появились на нынешних территориях
после славян на много позднее.

\iusr{Sychev Vitali}

Илья, есть еще город Арск.
\end{itemize}
