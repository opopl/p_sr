% vim: keymap=russian-jcukenwin
%%beginhead 
 
%%file 18_02_2020.fb.fb_group.story_kiev_ua.1.okonchanie_elegii_detstva.cmt
%%parent 18_02_2020.fb.fb_group.story_kiev_ua.1.okonchanie_elegii_detstva
 
%%url 
 
%%author_id 
%%date 
 
%%tags 
%%title 
 
%%endhead 
\subsubsection{Коментарі}

\begin{itemize} % {
\iusr{Ирина Архипович}

Чудесные элегии!! Прочитала с огромным удовольствием!!! Спасибо!!
@igg{fbicon.hands.applause.yellow}  @igg{fbicon.100.percent}
@igg{fbicon.heart.eyes} 

\begin{itemize} % {
\iusr{Ирина Петрова}
\textbf{Irina Arhipovich} спасибо! Так вылились на бумагу милые сердцу воспоминания)
\end{itemize} % }

\iusr{Ирина Петрова}

У меня есть традиция - я ставлю цветы летом, вербовки весной, пихтовые лапы
зимой к мемориальной доске Анны Ахматовой, которая висит как раз под моими
бывшими окнами и балконом второго этажа на Заньковецкой. Это память о нашей
семье, о моем родном доме. Анне Ахматовой тоже дань.

\ifcmt
  ig https://scontent-frt3-1.xx.fbcdn.net/v/t1.6435-9/86970315_3004350916265162_2586253495459905536_n.jpg?_nc_cat=107&ccb=1-5&_nc_sid=dbeb18&_nc_ohc=JrCKGmR_w1EAX_XrZkz&_nc_ht=scontent-frt3-1.xx&oh=54e7ddac3521c0eeda7389de0b658336&oe=61B23A6E
  @width 0.4
\fi

\begin{itemize} % {
\iusr{Татьяна Гурьева}
\textbf{Ирина Петрова} я бы с удовольствием присоединилась к Вам

\iusr{Ирина Петрова}
\textbf{Татьяна Гурьева} организуем!

\iusr{Ольга Шаленко}
Огромное спасибо за Ваши воспоминания! С этим домой у меня очень многое связано....

\iusr{Татьяна Гурьева}
\textbf{Ольга Шаленко} и с этим пора что-то делать)

\iusr{Ольга Шаленко}
Обязательно что-нибудь придумаем!  @igg{fbicon.smile} 

\iusr{Ирина Селезнёва}
\textbf{Ирина Петрова} Вы переехали в другой район?

\iusr{Ирина Петрова}
\textbf{Ирина Витальевна} Селезнёва увы... теперь приют убого чухонца Левобережка)))
\end{itemize} % }

\iusr{Ирина Петрова}

\ifcmt
  ig https://scontent-frt3-1.xx.fbcdn.net/v/t1.6435-9/86730593_3004352732931647_1175060175527084032_n.jpg?_nc_cat=107&ccb=1-5&_nc_sid=dbeb18&_nc_ohc=yvCTaS2oRSwAX8p7DHl&_nc_ht=scontent-frt3-1.xx&oh=31cf26ab940811c829df42b7ecf0254e&oe=61B4C5CF
  @width 0.4
\fi

\iusr{Галина Гурьева Николенко}

На этих фото, конечно, нет моих родителей и их друзей, но как приятно
рассматривать каждое. Как будто одна большая семья! Добрые светлые лица! Именно
такие хлебосольные столы мне запомнились из детства. Спасибо!

\begin{itemize} % {
\iusr{Ирина Петрова}
Спасибо! Хорошо, что есть теперь возможность поделиться памятью не только с друзьями.
\end{itemize} % }

\iusr{Светлана Солдатенко}

Прекрасное эссе! Все очень точно подмечено, "дух" той жизни, хоть я тогда была
ребенком, но это все впиталось в сознание. Спасибо.

\ifcmt
  ig https://scontent-frx5-1.xx.fbcdn.net/v/t39.1997-6/p720x720/22471081_2046746945653102_2957250678839312384_n.png?_nc_cat=110&ccb=1-5&_nc_sid=0572db&_nc_ohc=16qsd84MiFcAX98Lxga&_nc_oc=AQmikQuCOjiXMlRgska-tNE0YxTBm6Sy_y-kTkIfM9m-x5R3qV5dKY_96QL4eKwZ5Z8&_nc_ht=scontent-frx5-1.xx&oh=7c54890bc04e1554fb07e2d51c995f84&oe=6192CF84
  @width 0.2
\fi

\begin{itemize} % {
\iusr{Ирина Петрова}
\textbf{Svetlana Soldatenko} детская память цепкая, чистая, впитывает больше, чем память взрослого. Наверное, поэтому и запомнилось все это, потому что тоже была ребенком. Правда, потом ещё из маминых рассказов
\end{itemize} % }

\iusr{Леся Фандралюк}
Спасибо огромное, прочла на одном дыхании))

\begin{itemize} % {
\iusr{Ирина Петрова}
\textbf{Леся Фандралюк} спасибо! Приятна автору похвала читателя!)
\end{itemize} % }

\iusr{Виктория Русакова}
Все тоже самое, только по адресу пер. Дзержинского.

\begin{itemize} % {
\iusr{Ирина Петрова}
\textbf{Виктория Русакова} да, тогда это было , наверное, у всех, или у подавляющего большинства)

\iusr{Виктория Русакова}
\textbf{Ирина Петрова} 

И любимая мелодия моего папы- танго "Рио-Рита". У нас было очень много
пластинок с музыкой 30-х годов. Я на ней выросла. Когда бываю в Киеве,
обязательно прихожу к нашему дому. И до сих пор снится огромная, красивая
лестница нашего подъезда.

\iusr{Ирина Петрова}
\textbf{Виктория Русакова} 

мы детском садике знали наузусть почти все песни Эдиты Пьехи)))) раскачивались
на качелях-лодочках и орали погромче "Дунай-Дунай, а ну, узнай..." А папа с
мамой под неё вальс танцевали)

\iusr{Виктория Русакова}
\textbf{Ирина Петрова} Замечательное было детство, мягкое, счастливое!

\iusr{Ирина Петрова}
\textbf{Виктория Русакова} 

наверное, исключая военные годы, детство всегда такое. Мой сын, например,
детство которого пришлось на начало 90-х!!!! с удовольствием вспоминает нашу
квартиру, площадку на Франко, квартиру вторых бабушки и дедушки на Головка))) в
моих воспоминаниях это капец какой-то, а не годы...

\iusr{Виктория Русакова}
\textbf{Ирина Петрова} 

Я в 90-е жила на Сахалине. В магазинах не было НИЧЕГО. Рыбу вилами в горных
речках ловили. Сейчас смешно вспоминать, тогда было на выживание. Благо все это
происходило без глобальных политических потрясений.

\iusr{Ирина Петрова}
\textbf{Виктория Русакова} ого! Охотники!)))

\end{itemize} % }

\iusr{Татьяна Гурьева}
И мне вспомнилось привет от Печерска

\begin{itemize} % {
\iusr{Ирина Петрова}
\textbf{Татьяна Гурьева} привет, поклон и наилучшие пожелания!

\iusr{Татьяна Гурьева}
\textbf{Ирина Петрова} спасибо воспоминания объединяют, к счастью

\iusr{Ирина Петрова}
\textbf{Татьяна Гурьева} супер!

\iusr{Светлана Манилова}
\textbf{Татьяна}, " Есть память сердца, память детства..."

\iusr{Татьяна Гурьева}
\textbf{Светлана Манилова} да, наверное, единственное, что осталось

\iusr{Ирина Петрова}
\textbf{Татьяна Гурьева} 

Танюша, не соглашусь. Осталось ещё воспитанное теми годами чувство радостного
восприятия жизни) трудности поменялись, конечно, но, куда ж без них...

\iusr{Татьяна Гурьева}
\textbf{Ирина Петрова}  @igg{fbicon.rose} 

\iusr{Ирина Петрова}
\textbf{Татьяна Гурьева}  @igg{fbicon.bouquet}  @igg{fbicon.face.happy.two.hands} 
\end{itemize} % }

\iusr{Валентина Кряжева}

\ifcmt
  ig https://scontent-frx5-2.xx.fbcdn.net/v/t39.1997-6/s480x480/14050144_1775288802711824_1454378351_n.png?_nc_cat=1&ccb=1-5&_nc_sid=0572db&_nc_ohc=pclXKHkjMbgAX_0Nsqw&_nc_ht=scontent-frx5-2.xx&oh=9f8a71d0b026150c23cd79e63b7db35c&oe=61930095
  @width 0.2
\fi

\iusr{Николай Гайдай}
Чудово!

\begin{itemize} % {
\iusr{Ирина Петрова}
\textbf{Mykola Gaid} дякую! Приємно!)

\iusr{Николай Гайдай}

У Вашому есе "про голубів" -:) розведіть окремо історію про голубів і про
докторську ковбасу, а то вони якось злилися. А ще цікаві подробиці: де ці
голуби брались, скільки їх тримали на балконі і т. і. Дійсно, дуже цікаво.
Пишіть подробиці. Дякую!  @igg{fbicon.rose} 

\iusr{Ирина Петрова}
\textbf{Mykola Gaid} 

Справа в тому, що коли були ці події, я була зовсім маленька, Наумчик пішов у
засвіти, коли мені було рочків шість. Тому досконально тему розведення нещасних
пташенят я не знаю. Звісно, що одразу після відходу його, голуби були випущені
на волю. Мабуть, вінкупував їх у тих, хто тримав голубів, тоді таких було
багато. Навіть, на нашій горочці у дворі була голуб'ятня, пам'ятаю її. Ковбаска
та птички бідолашні зливались на столі Наумчика, відокремлювати їх не можу @igg{fbicon.laugh.rolling.floor} 
хай вже так і йдуть у вічність разом.

Дякую за квіточку  @igg{fbicon.face.happy.two.hands} 

\iusr{Николай Гайдай}
\textbf{Ирина Петрова} 

дякую. Пам'ятаю голубятні були майже в кожному дворі, але щоб їли голубів у
Києві, ніколи не чув. Зараз їх можна скуштувати в ресторанах Єгипта...

\iusr{Ирина Петрова}
\textbf{Mykola Gaid} я сама таке не чула, окрім нашого сусіда. ))))
\end{itemize} % }

\iusr{Надежда Маслак}
Это была жизнь наших Родителей, Бабушек и Дедушек, а потом и наше детство, юность ..

\iusr{Oksana Svechnikova}

 @igg{fbicon.hands.applause.yellow}  Дякую за навіяні спогади... Саме ці теплі відчуття спілкування у колі рідних
та друзів батьків - найбільш яскраве відчуття дитинства...

\begin{itemize} % {
\iusr{Ирина Петрова}
\textbf{Oksana Svechnikova} 

так. Нещодавно, 14 лютого ми відзначали 95- річчя Ніни Вікторівни Пименової. На
моїх фото до розповіді, де за святковим столом мої батьки та родина Піменових.
Її чоловік Міша та мій батько були однокласниками. Батьків мої вже пішли у
засвіти, Міхаіл Павлович також, Нінуся нас радує і тішить. Я дзвоню до неї, ми
згадуємо багато історій. Одна з них є у моєму оповіданні. Якось викладу. Для
Нінусі я ще дитинка, це дуже приємне відчуття!)

\end{itemize} % }

\iusr{Ирина Петрова}
Ось доказ моїх слів, висить картиночка, тішить @igg{fbicon.heart.red}

\ifcmt
  ig https://scontent-frt3-1.xx.fbcdn.net/v/t1.6435-9/87028620_3006565262710394_7766464309089009664_n.jpg?_nc_cat=108&ccb=1-5&_nc_sid=dbeb18&_nc_ohc=budj2GoxwfYAX8ggCb5&_nc_ht=scontent-frt3-1.xx&oh=147c459e32e7c1de8d74df5d59b7a285&oe=61B5714F
  @width 0.3
\fi

\begin{itemize} % {
\iusr{Татьяна Гурьева}
\textbf{Ирина Петрова} насколько оно приятно для восприятия
\end{itemize} % }

\iusr{Алена шевченко}
 @igg{fbicon.hands.applause.yellow}{repeat=3}  @igg{fbicon.hibiscus}{repeat=3}  @igg{fbicon.thumb.up.yellow} 

\iusr{Ирина Петрова}

тут немного получилось так, что после некоторого редактирования, тема всплыла,
как новая публикация. Это окончание историй. Первые две части можно найти через
мой профиль в группе.

\iusr{Олег Коваль}

часть 1-я: \url{https://www.facebook.com/groups/story.kiev.ua/posts/1245492932314172/} 

часть 2-я: \url{https://www.facebook.com/groups/story.kiev.ua/posts/1254759088054223/} 

часть 3-я: \url{https://www.facebook.com/groups/story.kiev.ua/posts/1276874229176042/}

\begin{itemize} % {
\iusr{Ирина Петрова}
\textbf{Олег Коваль} спасибо большое за помощь! @igg{fbicon.heart.eyes} 
\end{itemize} % }

\iusr{Анна Сидоренко}
Спасибо.

\iusr{Valentina Urban}

Огромное спасибо за рассказ! Все так знакомо и близко, мы дети 50- х являемся
очевидцами вышеупомянутых незабываемых событий на Печерске.

\begin{itemize} % {
\iusr{Ирина Петрова}
\textbf{Valentina Urban} память детства всегда греет сердце @igg{fbicon.hearts.two} 
\end{itemize} % }

\iusr{Наталия Ковалева}
Коммунальные квартиры, это особый мир!

\begin{itemize} % {
\iusr{Ирина Петрова}
\textbf{Наталия Ковалева} о, да!!! Я прожила почти полвека в коммунальной  @igg{fbicon.wink} 
\end{itemize} % }

\iusr{Наталия Ковалева}

\ifcmt
  ig https://scontent-frx5-2.xx.fbcdn.net/v/t39.1997-6/s168x128/118211260_659965514618313_1559572347261946985_n.png?_nc_cat=1&ccb=1-5&_nc_sid=ac3552&_nc_ohc=_cOn3UKNgMsAX91v6Dv&_nc_ht=scontent-frx5-2.xx&oh=b36a3a73c348ee9d982b158700da5c6d&oe=61926BF7
  @width 0.15
\fi

\iusr{Yevgeniy Goldshtein}

Спасибо. Вспомнил, как ходил к соседям по коммуналке смотреть телевизор, когда
показывали убийство президента Кеннеди.

Тот знаменитый коротенький черно-белый любительский фильм, снятый случайным
свидетелем из толпы в Далласе.

Потом снились кошмары. Мне было 6 лет.

Я уже знал про войну Вьетнаме, события на Кубе и не понимал, почему родители и
все соседи притихли и жалеют президента США.

У меня хранятся трофейные готовальня и логарифмическая линейка, привезенные
отцу его дядей. Пригодились и отцу, и мне при оформлении 10 листов диплома.

\begin{itemize} % {
\iusr{Ирина Петрова}
\textbf{Yevgeniy Goldshtein} да, я тоже в КИСИ чертила рейсфедером из папиной готовальни.)
Кеннеди мне всегда нравился, такой симпатичный. И я очень расстроилась, когда Жаклин вышла за Онасиса, мне было так обидно @igg{fbicon.grin} 
\end{itemize} % }

\iusr{Ольга Dzhun}
Спасибо за рассказ

\begin{itemize} % {
\iusr{Ирина Петрова}
\textbf{Ольга Dzhun} мне приятно, что читателям нравится
\end{itemize} % }

\iusr{Наталія Крюкова}
Петрова...
не жила ты в полуподвале....
почти рядом, на Терещенковской(Репина), в 12 ти метровой комнатке, где я родилась и прожила до 8ми лет...
С туалетом во дворе и одним краном там же.
Но....
Папа, известнейший на весь Киев художник-оформитель, получив очень солидный гонорар, не донес его до мамы, а купил в 58 году КВН...
Даже в страшном сне маме не пришло бы в голову брать деньги за просмотр телевидения...
А соседей набивалось под потолок.
Отец войну закончил в Праге, если это что-то говорит...
"Срібний дзвін- передзвін весело лунає..."
Помнишь песенку, которой начинались трансляции?

\begin{itemize} % {
\iusr{Люся Киевская}
\textbf{Наталія Крюкова} 

а я прожила с родителями - в коммуналке на 5 семей, без удобств , с печным
отоплением ( дрова и уголь в сарае во дворе) , как и все остальное, 20 лет ( с
Рождения), и лишь на 21 году моем - получили в Старой Дарнице - 2 х комн . «
хрущевку» ( комн. смежные) - потому что у нас в комнате упала часть потолка., а
весь дом признали аварийным и соседей расселили позже ...

\iusr{Ирина Петрова}
\textbf{Наталія Крюкова} 

да, Наташ, Бог миловал, прожила полжизни на втором этаже, а теперь аж на 15-м
@igg{fbicon.wink} в моем доме дружила с девочкой, они жили в нашем полуподвале,
мама работала прачкой в прачечной, которая была в нашем доме, папа был
сапожник, в будке на Бессарабке. И можно не поверить, но, я завидовала!!!!
Потому что из их комнаты, из окна можно было выйти в приямок и пройти по нему
вдоль окон из квартиры)))) а у меня был только балкон
@igg{fbicon.face.rolling.eyes}  и я ей с балкона на ниточке в этот приямок
опускала карандаши и маленьких куколок. Звали её Таня, а фамилии их не помню (
потом они уехали куда-то...

Ну, а по поводу платного просмотра. Да, это, и конечно, была редкость, но, это
было. Из песни слов не выкинешь. Все люди разные)

Песенку не помню, потому что меня не всегда мама брала с собой, наверное, мы не
с начала трансляции приходили. У нас телевизор появился только в 1966, "Вечер".

Папа был абсолютно уверен (и современная история это доказывает), что телевизор
вреден для здоровья. Особенно мне, очкарику с детства. Мы как-то больше читали,
гуляли вечерами на Франко.

\iusr{Наталія Крюкова}
\textbf{Люся Киевская} Старая Дарница - хуже, чем наш полуподвал на Терещенковской....
Была, видела этот ужас...

\iusr{Наталія Крюкова}
\textbf{Ирина Петрова} Мой отец был книжником сумасшедшим, в хорошем смысле.
И читать я начала только благодаря ему сразу на ТРЕХ языках- украинском, русском и польском.
В 4 года шпарила Матка Боска Ченстоховска и Еще Польска не сгинела! только так!
Отец, главный оформитель Большевика, выписывал любые газеты и журналы на Красный уголок.
Фронтовик, прошел всю войну и закончил в Праге, позже всех...
А вместо взорванного дома дали полуподвал
12 м с окном в подъезд.
И тем не менее.
Польска гоноровость...
О, это отдельный рассказ...

\end{itemize} % }

\iusr{Татьяна Гурьева}

Очень интересно, переживательно. В том смысле, что пережила часть эпохи, можно
сказать во многом счастливой. Очень хорошо Вы описали, жизненно

\begin{itemize} % {
\iusr{Ирина Петрова}
\textbf{Татьяна Гурьева} спасибо за теплые слова, Танечка
\end{itemize} % }

\iusr{Татьяна Сидорук}
Спасибо, интересно.

\ifcmt
  ig https://scontent-frt3-1.xx.fbcdn.net/v/t39.30808-6/248583477_359767839280018_5061391857029612886_n.jpg?_nc_cat=104&ccb=1-5&_nc_sid=dbeb18&_nc_ohc=ZOIerG1r-okAX8tMTFG&_nc_ht=scontent-frt3-1.xx&oh=a30f4a1310bb1d275fe52adff025e38f&oe=619326E1
  @width 0.2
\fi

\iusr{Лариса Гаркун}
Я теж пам'ятаю, у сусідів був телевізор, і вони нас кликали до себе дивитися.

\iusr{Сергей Оборин}

"Всё было интересно, обсудить увиденное Зое надо было с кем-нибудь немедленно,
во время просмотра. А Ваня, после третьей ежевечерней рюмочки, начинал сладко
дремать. Сыновья, немного послушав новости, отвлекались игрой с деревянными
солдатиками, и Зое становилось скучно. Вот тогда она и приняла решение
приглашать соседей на просмотр вечерних телепередач." - цитата, взятая из
текста, теперь пояснение. В начале 60-х мне было 10-11 лет. Жили на общей
кухне. Соседи купили телевизор (уже без линзы) "Рекорд", по-моему. Тогда же
начались передачи КВН и Голубой огонек. После работы, ежедневно, все соседи,
включая тех, что жили в квартире напротив, приходили и усаживались на свободных
местах, кто-то даже приносил свои стулья и табуретки. На столе расстилался лист
газеты ("Правда", либо "Известия", кстати, тогда появилась поговорка, что в
"Известиях" нет правды, а в "Правде", нет известий), и на него высыпалась
огромная гора поджаренных семечек. За время передачи вся эта гора семечек
съедалась присутствующими. Выпивки на моей памяти не было, ибо вся мужская
часть квартир-друзей, по будням была в отлучке. Люди, в основном военные, были,
либо на работе допоздна, либо в отъезде. Как же "вкусны" были эти вечера перед
телевизором. И, никто не собирал гривенников, ибо все были дружны, воодушевлены
хрущевскими переменами в жизни, и, каждый старался принести с собой что-нибудь
из угощения для всех.

\begin{itemize} % {
\iusr{Ирина Петрова}
\textbf{Сергей Оборин} да, коммерческая жилка тогда была у единиц)

\iusr{Yulianna Portnoy}
\textbf{Ирина Петрова} Вот вот, жаль что соседи не брали с нее же гривенник за выслушивание мнения, а
маленькая девочка пятачек
\end{itemize} % }

\iusr{Ирина Берлянд Зельманова}

Спасибо за прекрасный рассказ из детства.
Всё точно, один в один.
У моей мамы 1 Мая был день рождения.
Представляете, какой двойной праздник был в нашем доме!
Самые светлые и радостные воспоминания детства 60-х!

\iusr{Люся Киевская}

Два мелких замечания :- фильм назывался не « Старые ... «, а « Верные друзья « .,
и в телевизоре, наверное , были не « лапочки», а лаМпочки ?! ...

\begin{itemize} % {
\iusr{Ирина Петрова}
\textbf{Люся Киевская} да, спасибо огромное!!! Конечно, "Верные" и лампочки) Сейчас исправила бы, но, тогда опять премодерация...
\end{itemize} % }

\iusr{Марина Соболевская}

Прочла с огромным удовольствием! Очень хорошо, что тема всплыла. Так бы не
увидела.) Вы замечательно пишите воспоминания. Легко, с юмором.)

\iusr{Янина Ромова}
К сожалению, коммунальные квартиры существуют и поныне.

\iusr{Ирина Ящук-Лантушенко}

Старшая сестра рассказывала. Когда отец купил телевизор, для всех соседей был
праздник. Жили на Горького 62, кв 7 в полуподвальном помещении и телевизор
выставлялся в окно, все соседи собирались во дворе. Даже когда дом расселили
ездили к друг другу в гости и я уже помню как вспоминали эти посиделки перед
телевизором.



\end{itemize} % }
