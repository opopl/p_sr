% vim: keymap=russian-jcukenwin
%%beginhead 
 
%%file 31_10_2020.sites.ru.zen_yandex.yz.mir_istorii.1.moskovia_rossia_li_eto
%%parent 31_10_2020
 
%%url https://zen.yandex.ru/media/history_world/moskoviia--rossiia-li-eto-5f9d63269ac0705ae4842a0c
 
%%author Мир Истории
%%author_id yz.mir_istorii
%%author_url 
 
%%tags moskovia,russia
%%title Московия – Россия ли это?
 
%%endhead 
 
\subsection{Московия – Россия ли это?}
\label{sec:31_10_2020.sites.ru.zen_yandex.yz.mir_istorii.1.moskovia_rossia_li_eto}
\Purl{https://zen.yandex.ru/media/history_world/moskoviia--rossiia-li-eto-5f9d63269ac0705ae4842a0c}
\ifcmt
	author_begin
   author_id yz.mir_istorii
	author_end
\fi

\index[rus]{Русь!История!Московия, наследие Руси, 31.10.2020}

Сегодня в публицистике, а еще чаще среди политиков можно услышать весьма
странные комментарии относительно наследия Руси. В России говорят о том, что
Русь – исключительно «российское наследие», украинские политики заявляют, что
существовала только «Русь-Украина», а Россия – это «Московия».

В самой России не любят, а иногда даже избегают термина «Московия», хотя это не
решает исторический (и политический) спор, а наоборот создает еще большую
путаницу. Давайте попробуем разобраться, откуда появился термин «Московия»,
какое он имел значение, а главное: Московия – это Россия?

\ifcmt
  pic https://avatars.mds.yandex.net/get-zen_doc/230233/pub_5f9d63269ac0705ae4842a0c_5f9d69059ac0705ae48e0efa/scale_1200
  caption Карта Russia фламандского картографа Герарда Меркатора, 1595 г. Московия (указана стрелкой) обозначена как одна из её местностей.
\fi

\subsubsection{Вопрос исторического наследия}

Русь – это вообще чье наследие? Это вопрос уже больше века волнует историков,
журналистов и политиков. И вот тут кроется самая большая сложность: если мы
хотим получить простой ответ – то нам нужно слушать политиков, причем
желательно «своих», они нам точно правду скажут. То, что звучит в газетных
статьях и высказываниях политиков, часто около научные разговоры. Приведу
примеры нескольких таких полемик:

\begin{itemize}

\item 1. Украинский историк М. Грушевский вводит термин «Украина-Русь», чтобы подчеркнуть факт того, что украинцы формируются в период Руси, а само государство – и их наследие. Историк писал:

		«Мы знаем, что Киевское государство, право и культура, являются творением
		одной народности, а Владимиро-Суздальская держава – другой». 

Другие журналисты (и к сожалению историки) продолжают эту мысль, говоря, что
наследник должен быть один: если это Украина, тогда Россия должна искать другие
корни своей истории.

\item 2. В России выводят начало Руси с 862 года, с Новгорода, тем самым
				заявляя, что это «раньше прихода Олега в Киев». Да и сам Олег (как и
				Рюрик, родоначальник династии) был с Новгорода.

Затем идут дальше, находят в истории киевских князей этапы биографии, когда они
начинали свою «политическую карьеру» с нынешних российских земель: Святослав,
Владимир Великий и Ярослав Мудрый. Чаще всего среди россиян звучат
династические (Рюриковичи правили в Москве до конца 16 века) и церковные
(киевский митрополит перебрался в Москву) аргументы в пользу «монополии на
наследство».

\ifcmt
  pic https://avatars.mds.yandex.net/get-zen_doc/1328418/pub_5f9d63269ac0705ae4842a0c_5f9d6d0a9ac0705ae494cf70/scale_1200
  caption Русь при Ярославичах / Koryakov Yuri / ru.wikipedia.org
\fi

\item 3. В украинском парламенте звучат мысли о том, что «Россия – это
Московия, она украла украинское прошлое». Российские политики
отказываются принять факт того, что украинский этнос
формировался в период Руси (а некоторые и вовсе, что он
существует). Все это примеры шовинизма с обеих сторон.

Украинский историк Ярослав Грицак проводит параллели между Западной и Восточной
Европой: 

\begin{leftbar}
	\begingroup
		\em
			«На Западе похожий спор вызывало наследие Карла Великого и империя
				Франков. Но постепенно сложился компромисс, что итальянцы, немцы,
				французы, а также швейцарцы причастны к истории империи».
	\endgroup
\end{leftbar}

Однако не тех, не тех в тот период не существовало, они лишь формировались.
Похожий компромисс в отношении Руси использовали в советское
время: Древняя Русь считалась «колыбелью трех
восточнославянских» народов.

\ifcmt
  pic https://avatars.mds.yandex.net/get-zen_doc/4004066/pub_5f9d63269ac0705ae4842a0c_5f9d7075b09e797ceb179e13/scale_1200
  caption Русь, Польша и Литва в 1139 году
\fi

\end{itemize}

Если историку отвечать на вопрос, чья же Русь, то ответ будет весьма
противоречивый на первый взгляд: ничья! Попробую объяснить свое
мнение.

Князья Олег, Святослав и другие не были ни украинцами, ни россиянами. Нации –
это часть мировоззрения модерного мира, начиная с 19 века. Само
слово «Русь» пришло из Скандинавии. Государство не было
этническим, это был конгломерат разных племен (славянских,
тюркских, балтийских и финно-угорских) во главе с варяжскими
правителями. Это было типичное государственное образование
раннего средневековья: «мы вас защищаем от окружающих врагов,
вы нам дань, мы ее на продажу».

Этничность вообще не имела значение. Австрийский историк Андреас Каппелер в
книге «Малая история Украины» также затрагивают тему современного спора России
и Украины о наследии: 

\begin{leftbar}
	\begingroup
		\em
			«Он [спор], если подходить трезво, абсолютно лишний, потому что в нем оперируют современными национальными категориями мышления, далекими от Средневековья».
	\endgroup
\end{leftbar}

Именно в период Руси сформировались элементы идентичности, общие для восточных
славян: православное христианство, язык и письменность на основе кириллицы.
Кроме того, именно в Руси зародились названия, которые активно использовались
после ее распада, порождая путаницу.


\ifcmt
  pic https://avatars.mds.yandex.net/get-zen_doc/1537151/pub_5f9d63269ac0705ae4842a0c_5f9d6db23910530e0dea39ae/scale_2400
  caption Пример совместного использования: Russia, в просторечии Moscovia, Космография Блау, Голландия, 1645 г. 
\fi

\subsubsection{История названий}
\index[rus]{Русь!История!Малая Русь, Великая Русь}

После начала феодальной раздробленности каждая из частей Руси использовала
название «Русь»: от Галиции до Новгорода и Рязани. Затем князь Галицкого
княжества Юрий Львович добился для своего государства права на митрополита,
которых теперь в восточной Европе стало два: в Москве и Галиче. Чтобы не
путать, патриарх Константинополя придумал названия: «Малая Русь» и «Великая
Русь».

В западной Европе появлялся термин «Россия» - та же «Русь», но написано на
латинский лад. Потом Польша захватила Галицию и назвала ее «Русское
воеводство», а в Великом княжестве Литовском началась гражданская война: князь
Свидригайло хотел создать свое государство – «Великое княжество Русское», с
центром в Киеве. На востоке усиливается Владимиро-Суздальское княжество,
которое становится Московским, а его правители начинают называть себя «русскими
князьями» и «собирателями земель русских». Все это вносит ужасную путаницу в
работу картографов 15-16 века.

Путешественник и картограф Фра Мауро на своей карте 5 раз указывает название
«Россия». На других картах восточной Европы эту путаницу пытались предотвратить
используя название «Московия». Тогда Московское княжество, а затем царство
стремительно расширялось, победило Орду, вступило в войны с Литвой, Ливонским
орденом, затем Польшей. 

\ifcmt
  pic https://avatars.mds.yandex.net/get-zen_doc/1362253/pub_5f9d63269ac0705ae4842a0c_5f9d6a5349505f6811f23cdf/scale_1200
  caption Фрагмент карты мира Фра Мауро, 1459 год
\fi

\index[writers.rus]{Маржерет, Жак!17 век!Состояние Российской
				державы и Великого княжества Московского, 1607, Париж}

Московское царство было тогда главным «ньюсмейкером» в восточной Европе,
отсюда и такой интерес к его границам. Поэтому постепенно картографы и
путешественники задумались об уместности использования термина
«Московия».

Француз \textbf{Жак Маржерет} в начале 17 века написал книгу \emph{«Состояние Российской
державы и Великого княжества Московского»}, которая в 1607 году была
напечатана в Париже. В ней он отмечал:

\begin{leftbar}
	\begingroup
		\em
«Считают ошибкой именовать жителей московитами, а не русскими, но так делаем не только мы, далекие жители, но и их соседи. Сами люди, на вопрос, какой они нации, отвечают: Russac, что означает – русский». 
 
	\endgroup
\end{leftbar}

Для автора говорить «Московия» - это так же, как называть французов
«парижанами». С начала 18 века Московское царство начинают называть «Россией»,
то есть «Русью» на латинский лад. Окончательно Петр I в 1721 году меняет
официальное название: Московское царство становится Российской империей.


\ifcmt
  pic https://avatars.mds.yandex.net/get-zen_doc/118779/pub_5f9d63269ac0705ae4842a0c_5f9d6f7b9ac0705ae49912b8/scale_1200
  caption Российская империя (1745 год)
\fi

Так мы плавно подобрались к вопросу о том, является ли Московия Россией. Мне
кажет, что и да, и нет. Несомненно, этап Московского княжества и царства –
важная часть истории России. Это период, когда одно княжество захватило все
остальные и на карте Восточной Европы возникло единое государство.

С другой стороны, Россия – это не только Москва и ее история. Сегодня в России
проживают десятки народов, которые имеют свою историю, а сводить все к истории
Московского царства – это не современный путь формирования политической нации.
Россия – это единство в многообразии.

\index[writers.rus]{Паули, Густав-Теодор!«Этнографическое описание народов России», СПб. 1862}

\ifcmt
  pic https://avatars.mds.yandex.net/get-zen_doc/1703615/pub_5f9d63269ac0705ae4842a0c_5f9d6af51f9f737992305ea9/scale_2400
  caption Великороссияне разных губерний (Густав-Теодор Паули. «Этнографическое описание народов России», СПб. 1862)
\fi

Таким образом, можно сказать, что «Московия» - это не Россия, это ее часть, а
также отдельный этап истории формирования российской (а не «русской»)
идентичности. Вместе с тем, не стоит забывать, что «Московия», как и «Россия»,
- это изначально латинские термины, а не самоназвания. В конечном итоге
правители выбрали термин «Россия», так и начался новый этап ее истории.

\begin{itemize}
\item \cusr{Одереж}

Глупо искать. Поясню. Мы все россияне русский татарин калмык чуваш сахалинец.
Наши родные всех кровей. Я гуран жена моя белорусска у дочки
муж украинец а у сына татарка все мы любим друг друга ....

\item \curs{Олег Волков посмотреть можно!}

Одереж, Всё правильно говоришь,только нациков пора проучить!Время пришло!Нужно их на колени поставить перед матерями ДОНБАССА и пусть они их судят!А оставшихся в живых, выдворить в 3 области,где их историческое место,а не засовывать голову в песок и говорить ничего не можем сделать!Можем и ещё как - можем!Сколько ещё детей должно погибнуть?Сколько матерей и не в чём не виновных мирных жителей должно погибнуть из - за бездействия?Нам нужно вводить войска и плевать на крики! Мы быстро помойку захлопнем,если кто - то её откроет!У нас есть чем ответить! В АМЕРИКЕ - АЛЯСКА + 3 штата!У АНГЛИИ часть РОССИЙСКИХ земель! Германии можем напомнить,что,ГДР принадлежала и принадлежит - РОССИИ! Почему мы с такой быстротой и лихостью освободили РЕЙСТАГ?Не потому ли,что,рейхстаг строили наши предки - РУСИЧИ(по теперешнему - РОССИЯНЕ! И мы, его, восстанавливали! А посмотрите на всё остальное!Там прошли раскопки,даже ГИТЛЕР не тронул и не стёр всё,потому,что,он думал,что,всё строили АРИИ,а раз так,то,нужно сохранить(ведь они считали себя - АРИЙЦАМИ!)А строили - предки РОССИЯН,то есть - РУССКИХ!Вы почитайте эти названия и сами всё поймёте! А теперешний РЮГЕН,он во все не РЮГЕН,его названия - БУЯН!И столица у него называлась - АРКОНА!Пора всё вернуть!ВРЕМЯ ПРИШЛО! И справедливость ТРЕБУЕТ!

\item \cusr{Сибиряк Петров}

Одереж, что за национальность сахалинец!?)) Тогда и уралец, и сибиряк
национальности?)) Смешал всё в кучу,"россиянин". Русские -
основа, стержень и создатели России и без них "россияне"
разбегутся по своим чувашиям и татариям.

\item \cusr{елена матвеева}

Сибиряк Петров, В Москву в 30 годы 20 века очень много мордвы переселилось и других финно-угров,а улица АРБАТ татарское название,после войны 1612 года там были татарские жилые кварталы и торговые ряды.Мы русские россияне нация собранная из наций и народностей,в отличии от всяких там хохлов и бульбашей,что от жадности и зависти к нам подселились и понаехали,а теперь стучат себе в грудь что они люсски.

\item \cusr{елена матвеева}

Сибиряк Петров, В Сибирь при советах хохлов много набежало и этнически себя чистыми великороссами считают.

\item \cusr{Граф Галас}

Олег Волков посмотреть можно!, ну какой ты русич? Ты москаль, к Руси никакого отношения не имеешь.

\item \cusr{Матвей Рустов}

Сибиряк Петров, правильно, еще Бисмарк сказал:

\enquote{ Даже самый благополучный исход войны никогда не приведёт к распаду России, которая держится на миллионах верующих русских греческой конфессии. Эти последние, даже если они впоследствии международных договоров будут разъединены, так же быстро вновь соединятся друг с другом, как находят этот путь друг к другу разъединённые капельки ртути.}

И для Украины:

\enquote{ Не надейтесь, что единожды воспользовавшись слабостью России, вы будете получать дивиденды вечно. Русские всегда приходят за своим. И когда они придут — не надейтесь на подписанные вами иезуитские соглашения, якобы вас оправдывающие. Они не стоят той бумаги, на которой написаны. Поэтому с русскими стоит или играть честно, или вообще не играть}

\item \cusr{Георгий}

ИСТОРИЯ РОССИИ- ВОЙНА- ПРИРОСТ ТЕРРИТОРИЙ. ЕСЛИ ЕЕ УРЕЗАЮТ-ЕЩЕ БОЛЬШИЙ ПРИРОСТ.
РОССИЯ- ЗЕМЛЯ МАГНИТ- ХРАНИМЫЙ БОГОМ. ДЕЛО ВРЕМЕНИ.......

\item \cusr{Павел Ряряев}

				"...С начала 18 века Московское царство начинают называть «Россией», то есть «Русью» на латинский лад..."

Поскольку написано несколько невнятно и двусмысленно (кто начал называть страну Россией), уточню:

Как раз на "латинский лад" это и будет в разных вариантах именно "Русь" (Russia, Russie, Russland итд...), а "Россия" - это уже на греческий лад со времён Никона...
\item \cusr{DurDom}

Все разговоры о том, что на самом деле называется Русью будут
бессмысленными до тех пор, пока эти псевдо- историки, которые
слепили свои докторские и кандидатские диссертации на основе
сфальсифицированной версии Истории государства Российского,
написанной придворным литератором Н.Карамзиным по заказу немки
Екатерины2., будут руководствоваться фальшивой летописью
"Повесть временных лет", которая якобы была написана каким-то
безвестным киевским летописцем Нестором, а на самом деле эта
Повесть была написана немецким историком Миллером в середине 18
века, который при этом заявил, что он списал свою копию с
подлинной летописи, но при этом не представил подлинника и не
сказал где его взял. А никакого подлинника просто не было!!
Кроме Миллера сочиняли "древние" русские летописи ещё два немца
- Байер и Шлёцер, которых тоже пригласили в Россию для
сочинения фальшивой типа "древние" русские летописи. А потом
все фальшивки передали Карамзину, который и написал фальшивую
Историю Руси- России., и которой до сих пор засирают мозги
российским школьникам. Московия - это безусловно русская
идентичность. Московское княжество разросталось постепенно в
результате завоеваний московскими князьями соседних удельных
русских княжеств. Главную роль сыграл Иван3 , который за годы
своего правления увеличил территорию Московского княжества в 5
раз , покорив Великий Новгород, и включил его в состав своего
княжества. А если иностранцы называли Московское княжество
Русью, то это их право, Как хотели так и назвали, и на своих
топографических картах нарисовали. Мы же называем немецкий
Дойчлянд Германией, и при этом немцы на нас не обижаются.

\item \cusr{Андрей Гуркин}

Была огромная империя Русов, много было провинций

\item \cusr{Qainar Moldash}

Как логично все расписано: московия и есть. москву, как кормили, так и кормят

\item \cusr{Юрий}

Россия искаженное слово от Русь. Этимология слова Русь раскрыта в
толковом словаре русского языка выдающимся этнографом В.Даль, означает
светлое место. Русь это место размещения деревни (деревянных жилищ )
поселения родовой общины Людей на берегу реки. Согласно выводам В.Даля
население восточной Европы до крещения называли свою этническую
принадлежность Люди Мужской род Люд женский род Людь. Славянами людей
окрестила православная церковь, ибо сказано Все рабы божьи.

\item \cusr{petrovich}

Наше государство и учёные историки не решают вопрос исторического
именования территорий и народностей. Хотя бы конференцию историков
-представителей бывших республик созвали бы. Да принародно поговорили.
Вроде хотели с западными славянами пообщаться. Это хорошо. Время то
идёт. Надо же предлагать "умным" людям способы и формы единения ,жить
то надо. Всё равно объединимся. Или в жизни или в погибели. Нам
выбирать. Амбиции историков, закрепленные в разных вариантах учебников
только прибавляет хаоса в голову школяра. Это плохо. Люди сами
занимаются стихийным и исследованиями . Это хорошо. И судят историю
народа и государства - это плохо. "Академики" Идеологии боятся, а как
без неё, если государство существует. И не одно.

\item \cusr{Serge Mark}

Я живу не в Москве, но естественно русский, но называться "московитом"
("москалем") и пр. не считаю для себя зазорным. Москва создала империю. Петр
столицу в Петербург, но и только. Московский диалект лег в основу русского
разговорного языка ("аканье" ), Москва, как бы ее не хаяли в провинциии ,дорога
сердцу каждого настоящего русского человека... И лишь горечь возникает от
ощущения, что великий русский город Киев, пока находится во власти народа даже
не имеющего по большому счету самоназвания,не имеющего общего менталитета и
общей исторической гордости, народа, продававшего себя всегда и везде по сходной
цене. Конечно это не относится к многочисленным людям, говорящим и думающим на
Украине по-русски и не понимающим в какую западню они попали, согласившись с
антирусской политикой руководителей псевдостраны и ее религиозных деятелей. Если
они думают,что это берет начало после Донбасса и Крыма, то глубоко
ошибаются, Это началось даже гораздо раньше, чем правление подонка Кравчука.

\item \cusr{Аляксандр Гурыновіч}

Ак-Урусь, или Белая Русь – Московия

\url{https://nicolaev.livejournal.com/1499995.html}

"Поскольку монгольский Белый Хан был верховным владыкой всех русских княжеств,
то и после распада Золотой Орды его власть, по московской интерпретации,
перешла к московскому Великому князю, который стал Белым Ханом. Или в переводе
на адаптированный греческий - Царем.

Термин «самодержец Белой Руси» исчезает только у Петра I, и вместо него
появляется другое - «самодержец России». Таким образом, благодаря усилиям Петра
Московия - Белая Русь впервые в истории меняет свое название - на имя
«Россия»."

\item \cusr{Светлана Инусова}

"Украинский историк Ярослав Грицак проводит параллели между Западной и
Восточной Европой: «На Западе похожий спор вызывало наследие Карла Великого и
империя Франков. Но постепенно сложился компромисс, что итальянцы, немцы,
французы, а также швейцарцы причастны к истории империи»."

Вранье украинского националистов. Они всегда приводят этот пример, отрицая
множество фактов. Грицаку надо бы проводить параллель не с империей франков,
так как она распалась в начале 9 века. Когда Русь только зарождалась. А с
феодально раздробленными Польшей, Италией, Германией, Францией. Вот такое
сравнение будет корректным. Древнюю Русь населяли в основном русские. Древняя
Русь была главным образом государством одной русской народности. А империю
Карла Великого населяли галло-римляне, италийцы, франки, германцы, южные
славяне и другие совершенно разные народы. Не случайно Карл Великий принял
титул императора в Риме как наследник прежних римских императоров. Римская
империя была многоэтничным государством. Этнографические же различия
представителей древнерусской народности 12 века не превышали такие у
великорусов 19 века. Россия-Русь и Древняя Русь одна и та же страна и
государство в разное время. У Франции тоже более половины территории веками
входило в состав Англии. Это не повод считать что Франция и французы появились
в 15-16 веках. И не повод считать что немцы появились в 19 веке. Немецкая
народность сложилась в 10 веке. Хотя теперь австрийцы отдельная нация, никто не
говорит что Германии и немцев до 19 века не было.

Век они разбираются. Вы лучше скажите, кто теперь вам светлое будущее будет строить? 

\end{itemize}
