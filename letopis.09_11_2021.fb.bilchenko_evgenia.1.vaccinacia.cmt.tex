% vim: keymap=russian-jcukenwin
%%beginhead 
 
%%file 09_11_2021.fb.bilchenko_evgenia.1.vaccinacia.cmt
%%parent 09_11_2021.fb.bilchenko_evgenia.1.vaccinacia
 
%%url 
 
%%author_id 
%%date 
 
%%tags 
%%title 
 
%%endhead 
\subsubsection{Коментарі}
\label{sec:09_11_2021.fb.bilchenko_evgenia.1.vaccinacia.cmt}

\begin{itemize} % {
\iusr{Евгения Бильченко}
Родня, вы самые лучшие, такие разные, но реально классные.

\iusr{Мария Луценко}
Вот в чём ты права, это в том что тельцУ не прикажешь. Вот это ТОЧНО!

\begin{itemize} % {
\iusr{Евгения Бильченко}
\textbf{Мария Луценко} Я вчера набирала, испугалась.
\end{itemize} % }

\iusr{Евгений Дорошкевич}
Аферист Маск уважаемый? Ну-ну. Забыл, у вас же гиперлуп построили.

\begin{itemize} % {
\iusr{Евгения Бильченко}
\textbf{Евгений Дорошкевич} 

это же стёб, скоммунизженный мной у Ани Долгаревой. Так наши страницы
сохраняются. Иначе мы (не вы, следят за нами, вы им не опасны) окажемся там,
где Пелевин (навсИгда) и Захар в инсте (навсИгда), и будете тут реально читать
не мой тонкий лимоновский стёб, а только уважаемых господ.

\iusr{Евгения Бильченко}
\textbf{Евгений Дорошкевич} 

Или вы считаете, что БЖ за эти пять лет боица СБУ? Вы реально полагаете, что я
чего-то боюсь после отказа извиняться? Я год прожила в аду. Мой дом внесён в
миротворец. Жить можно. Трудно, но можно. "а вы ноктюрн сыграть смогли бы?"
Если да, за мной, воин.

\iusr{Евгений Дорошкевич}
\textbf{Евгения Бильченко} 

я офицер, бойцы идут за мной, извините. Маск и сбу пока вместе никак. Я в курсе
ваших страданий, сочувствую.

\iusr{Евгения Бильченко}
\textbf{Евгений Дорошкевич} Ладно. Я внучка полковника СА. Служу советскому народу. Все знаю. Все осознаю. Работаю.

\iusr{Евгения Бильченко}
\textbf{Евгений Дорошкевич} 

Жень... Вы меня так тронули. За моим дедушкой шли бойцы. А дедушка шел за
бабушкой. Вы - за женой, я верю в любовь. А когда признаете во мне бойца, я
могу и за вами, у меня, кроме тельца, все железное остальное.

\end{itemize} % }

\iusr{Любовь Томская}

Женя, голосование не решает проблем. Каждый принимает решение сам. И меня
мучает вопрос прививаться или нет. За столько же сколько против. Если есть
противопоказания и все же мучает страх, то можно короноваком привиться. Мой
семейный говорит, что это как водичка, с ним и заболеть можно. И все же это
лучше, чем ничего и безопаснее, чем RNK вакцины

\begin{itemize} % {
\iusr{Евгения Бильченко}
\textbf{Любовь Томская} What is RNK?))

\iusr{Ид Рис}
\textbf{Евгения Бильченко} Такое жульничество от транснациональных фармацевтических компашек. Говоря научным языком, мертвому припарка с непредсказуемыми последствиями. Мертвый, конечно, не воскреснет, но перспектива превращения в зомби не исключена.
\end{itemize} % }

\iusr{Олександр Дідух}

Серьезные издания на английском сообщают, что иммунитет после болезни, особенно
тяжелая - среднетяжелая формы, сохраняется в течении 17 месяцев, а может и
дольше. времени не хватило на дольше узнать. Я переболел, и верю в свой
иммунитет. Поэтому не пойду колоться.

\begin{itemize} % {
\iusr{Евгения Бильченко}
\textbf{Олександр Дідух} У вас пока антитела есть.

\iusr{Олександр Дідух}
\textbf{Евгения Бильченко} 

Женя - это зависит от очень многих составляющих. Если человек с неплохим
иммунитетом и здоровьем тяжело переболел, то есть шанс, что у него иммунитет
останется на всю жизнь. А у кого-то через месяц снова тяжелая форма, встречал
таких. В любом случае вакцинация переболевшим может совсем сломать иммунку.

\iusr{Алена Морозкина}
\textbf{Олександр Дідух} 

У моей племянницы иммунитет ниже плинтуса. Лишний вес, сердце, ангины
бесконечные, хронический тонзилит и проч. В марте болела ковидом. Ну... средне
болела, но тяжелее всех в семье. В октябре моя дочь принесла ковид из школы.
Все переболевшие ранее, даже не заметили повторный ковид. Включая племянницу с
никакой иммункой. Поэтому необязательно болеть тяжело для устойчивого
иммунитета к ковиду. И необязательно быть при этом до болезни м неплохим
иммунитетом))) Тут, имхо, рулетка.

Но вот да, с чем согласна, так с тем что переболевшим прививаться может выйти
боком. И неплохо сдать перед вакцинацией антитела. Что бы при этом не говорили
эскулапы...

\iusr{Ид Рис}
\textbf{Олександр Дідух} 

Подскажете, что за издания такие?... Я тут, грешным делом, зарабатываю на жизнь
редактурой профильных статеек в некоторых бульварных псевдонаучных изданиях,
AJMS там, ARM и все в таком роде... Так вот, в них какие-то якобы ученые
утверждают, что титры антител начинают стремительно снижаться на 90е-100е сутки
после клинического выздоровления и критически снижаются ( это когда вероятность
того, что организм даст инфекции адекватный иммунный ответ менее 30 процентов)
к 210-м- 230-м суткам... Не, я понимаю, неучи... Но вот что смущает: где ж
такую выборку-то 17-тимесячную набрали?... Неувязочка...

\iusr{Олександр Дідух}
\textbf{Id Ris} 

\href{https://www.nature.com/articles/s41586-021-03647-4}{%
SARS-CoV-2 infection induces long-lived bone marrow plasma cells in humans, Jackson S. Turner et al., nature.com, %
24.05.2021%
}

\iusr{Олександр Дідух}

\href{https://www.biorxiv.org/content/10.1101/2020.12.08.416636v1}{%
Persistent Cellular Immunity to SARS-CoV-2 Infection, Gaëlle Breton et al., %
www.biorxiv.org, 09.12.2020%
}

\iusr{Олександр Дідух}
\textbf{Id Ris} 17 месяцев забыл где читал, может вспомню, американцы точно

\iusr{Олександр Дідух}

Вот здесь упоминают 11 месяцев, и эти работы, приведенные выше 

\href{https://www.bbc.com/russian/news-57270589}{%
Ученые: иммунитет к коронавирусу у переболевших Covid-19 может быть пожизненным, Николай Воронин, bbc.com, 27.05.2021%
}

\iusr{Олександр Дідух}

\href{https://rg.ru/2021/10/13/uchenye-vyiasnili-kak-dolgo-sohraniaetsia-immunitet-u-perebolevshih-covid-19.html}{%
Ученые выяснили, как долго сохраняется иммунитет у переболевших COVID-19, Евгения Думанская, rg.ru, 13.10.2021%
}

\end{itemize} % }

\iusr{Мария Исаева}

Я б тебе не советовала, с твоим здоровьем... Каждый решает для себя сам. Вот
только решая не прививаться, многие лишаются работы, проезда в метро, походов в
магазин и тд... "У нас есть права" - не работает. "У тебя есть ты" - работает.

\begin{itemize} % {
\iusr{Евгения Бильченко}
\textbf{Мария Исаева} 

и "я хочу кушать/на "Живаго"/на концерт (незакрытого типа)/на кафедру/в НИИ. Я
же очень системная, Маняся. Я была самой системной из наших хиппи и являюсь
самой хипповой из своих системщиков. А все потому что я те самые Весы. Ужасный
знак. То их качает от храма до кабака, то вдруг в них реально просыпается
жёсткий и четкий аналитик усредненного толка. Это идеально понимаемые мной
люди. Их мало кто понимает.

\iusr{Мария Исаева}
\textbf{Евгения Бильченко} А, ну это все объясняет)

\iusr{Мария Исаева}
\textbf{Евгения Бильченко} Ну, ты повремени немного... Подлечись вначале, если реально решишь колоться.

\iusr{Евгения Бильченко}
\textbf{Мария Исаева} Вот это и хочу. Немного выравнять цитологию (не айс) и анализ крови (мега не айс).

\iusr{Мария Исаева}
\textbf{Евгения Бильченко} Правильно!

\iusr{Евгения Бильченко}
\textbf{Мария Исаева} и ещё: ты полностью совпала с врачом из Нижнего. Таки антибиотики. Помнишь? Над нами все смеялись...

\iusr{Мария Исаева}
\textbf{Евгения Бильченко} Жень, а такая возможность быть реальным неформалом!)) Как в "старые добрые" советские времена... Концерты закрытого типа, преследования, люди закладывают друг друга... Не разберёшься где "наши", где "не наши", и кто вообще наши?...)

\iusr{Мария Исаева}
\textbf{Евгения Бильченко} Ну, лично мне они помогли... Пусть смеются!

\iusr{Мария Исаева}
А знаешь какое противопоказание прививке? - Анафилактический шок!)) Вот такое впечатление, когда он случиться, тогда и освободят...
\end{itemize} % }

\iusr{Руслан Павлов}

Как вариант, купите сертификат и не парьтесь. Антиваксеры утверждают, что
регулярные, раз в полгода, инъекции снесут годика за два любой естественный
иммунитет и люди будут дохнуть от любой ерунды. Тут страшно что-либо
советовать, все могут заблуждаться. Возможно любой выбор фатален) Сам склонен
скорее бояться данайцев ибо разнообразием даров их уже пресыщен.

\begin{itemize} % {
\iusr{Olya Kharchenko}
\textbf{Руслан Павлов} раньше говорили, что через 6-9 месяцев, теперь уже 2 года?
Потом буде 10 лет? 40?

\iusr{Руслан Павлов}
\textbf{Olya Kharchenko} Ольга, указанный срок озвучил американский доктор Владимир Зеленко при выступлении на раввинском суде. Видео широко разошлось, найти не трудно.

\iusr{Руслан Павлов}
\textbf{Olya Kharchenko} 

Цитирую с купюрами слова Зеленко: Доктор Кейел из Аирлэнд считает, что в
течение 2 лет 90\% вакцинированных будут ***. Я не знаю, может быть это будет не
90\%. Может быть не за 2 года, а за 3 года?(с)

\iusr{Olya Kharchenko}
\textbf{Руслан Павлов} 

Владимир Зелегко на раввинском суде... ага. Не смотрела никогда в ту сторону. А
вы уверены, что это авторитет, которому можно доверять? Даже если он называет
сложные термины и проведённые опыты, то вы уверены, что это правда? Вы сами
специалист?

\iusr{Руслан Павлов}
\textbf{Olya Kharchenko} 

Я стараюсь смотреть по сторонам. Ни разу не специалист и ни в чем не уверен. И
ничего никому не советую. А Вы какую сторону смотрите? На доброго царя,
бояр-радетелей? Бигфарму? ВОЗ? В зомбоящике мудрость черпаете? В мировое
сообщество веруете?

\iusr{Olya Kharchenko}
\textbf{Руслан Павлов} я верю в попытку выжить.

\iusr{Руслан Павлов}
\textbf{Olya Kharchenko} Извините, но это еще никому не удалось)

\iusr{Olya Kharchenko}
\textbf{Руслан Павлов} из моих известных привиты никто не заболел. Так что пока по моему плану.

\iusr{Taty Vin}
\textbf{Olya Kharchenko} мой коллега был привит файзером, через несколько месяцев заболел, тяжело перенёс.

\iusr{Olya Kharchenko}
\textbf{Taty Vin} ну у меня не было такого плохого опыта, никто не болел. Хотя у меня не очень большой круг общения.
Сама привилась немецким Биотех

\iusr{Евгения Бильченко}
\textbf{Руслан Павлов} 

У меня тяжёлый случай. У меня лет 20 нет зомбоящика, я в сети. По идее должна
быть хипстером-глобалистом, дитям постмодерна, но, блин, не верую (это
тактично) в мировое сообщество. Включая ВОЗ. Бояр считаю его частью. Как ни
странно, но ура-патриархалы для меня - это искусственный продукт постмодерна и
глобализма, его карманный сервис, бунт для вида. Верую ли в царя? Да. В Христа.
И ещё в то, что в нас внутри от Него есть как Логос. Иногда я ощущаю этот Логос
в отдельных сильных личностях. К ним я привязываюсь. Я бываю не просто
традиционной, я бываю жестче Елизарова. Но я никогда не унижу Иное, я буду
слушать Елизарова, троллить им свое украинское начальство и своих российских
хейтеров, я иногда ставлю "Сталинский костюм" и, смеясь, плачу,думая, как бы
ржал дед. Но я не могу никого обидеть. Даже врагов. Я их обижаю тем, что жива.
И мертвой я их буду обижать ещё больше.

\end{itemize} % }

\iusr{Владимир Морозов}

Как жеж у Вас всегда много текста)) Текст конечно очень умный, правильно
построенный, в этом Вы гений. Но зачем его так много. Я сам такой поговорить
люблю долго, за это меня тоже иногда ненавидят. Но когда сам читаю чей то
текст, то понимаю, что был бы покороче, дочитал бы до конца. Этот текст я
дочитал. Прививаться надо тем кто ещё не переболел, но опять же если нет
противопоказаний, добровольно рисковать не стоит.(лучше уж надеяться на удачу
что не заболеешь, чем на удачу что не будет последствий от прививки).

Лично я переболел двумя штаммами, Альфой и Дельтой. Альфой переболел до начала
пандемии, в октябре 2019 после приезда из Польши (видимо там вирус появился
раньше чем в Ухани). Болел тяжело, но дома. Дельтой переболел легко, но тоже
дома.. Поэтому не вижу смысла прививаться для себя, разве что только для Ковид
сертификата(для галочки и удовлетворения постановление правительства)

P.S. в том что этим переболеют если не все то большинство, тут уже и к бабке не
ходи

\begin{itemize} % {
\iusr{Olya Kharchenko}
\textbf{Владимир Морозов} 

а вы точно знаете, что это был тот самый ковид в 2019? Спрашиваю потому, что
сама болела на Новый год и в Германии больницы были переполнены и болели тяжело
и только потом мы узнали об Ухани... потом я заподозревала таки Корону в
Германии, но как то без подтверждения сомнительно...

\iusr{Владимир Морозов}
\textbf{Olya Kharchenko} 

по описанным симптомам и по тому , что я ранее так не болел, по долго не
восстанавливающемуся здоровью, жуткой депрессии с мыслями о конце я понял, что
это он. И кстати, до второго раза, до Дельты я не болел простудными
заболеваниями вообще. А раньше несколько раз в год...

\iusr{Olya Kharchenko}
\textbf{Владимир Морозов} 

понятно. У меня депрессии вроде не было, хотя точно не помню. А жуткий кашель
частенько, но это что то от бронхит, который часто плавно перетекает из орви.
Кстати, как маску одела с началом эпидемии, так больше ничего и не было.

Тьфу тьфу тьфу, чтоб не сглазить.

\iusr{Евгения Бильченко}
\textbf{Владимир Морозов} Мое состояние этот год... Что за... Это можно объяснить? Рационально?

\iusr{Владимир Морозов}
\textbf{Евгения Бильченко} ну по идее же все поддается какой то логике? Или нет?
\end{itemize} % }

\iusr{Mary Koval}

Молитвой и химиотерапией рак лечится намного эффективней, чем просто молитвой.
Так и с короной. Можно молиться, можно молиться и сделать вакцину.

\begin{itemize} % {
\iusr{Виталий Бильченко}
\textbf{Mary Koval} доброе слово и револьвер...(с)

\iusr{Mary Koval}
\textbf{Виталий Бильченко} угу

\iusr{Виталий Бильченко}
\textbf{Mary Koval} ага...)


\iusr{Евгения Бильченко}
\textbf{Mary Koval} никто не против: ни я, ни папа мой, Мари, я вполне в себе позитивизм и православие сочетаю. Вопрос решается изнутри позитивистской парадигмы: показания к вакцинации.

\iusr{Виталий Бильченко}
Вумна, аж страшно... гены...)

\iusr{Евгения Бильченко}
\textbf{Виталий Бильченко} Папа, я тебя умоляю как внучка Веры Исааковны и русская в дупель дура, тут вопрос на уровне фобий решается
\end{itemize} % }

\iusr{Настя Бузиашвили}

Вакцинироваться надо. Иммунитет вырабатывается и от новых штаммов в том числе,
происходит аттенюация вируса, это научно доказано

\begin{itemize} % {
\iusr{Евгения Бильченко}
\textbf{Настя Бузиашвили} Поделись опытом.

\iusr{Настя Бузиашвили}
\textbf{Евгения Бильченко} болезнь легче переносится, потому что будут выработанные антитела. Они "склеивают" вирусные частицы, и иммунные клетки их легче находят и уничтожают
\end{itemize} % }

\iusr{Борис Никифоров}

Смотря для кого вакцинироваться. Лично я, крепко поразмыслив, будучи летом в Петербурге, дважды укололся.

Был я после недельного запоя (давно дома не был), чувствовал себя отвратительно
и "лёгкое недомогание" на следующий день приписал естественному ходу событий. Я
не верю ни "белым" ни "красным", а верю только Йоке Оно и мужу её Джону, ныне
покойному. Сказки что прививка доконает меня - слишком уж самонадеянны. Я эти
занимался много лет и уж коли пишу своими руками - значит смог. Уверовать что
"преступная власть" собралась меня укокошить... ну... это опять таки гордыня.
Обокрасть - да, но укокошить... Я вас умоляю. Так что однозначно делать!
Исключительно ДЛЯ СЕБЯ.

\begin{itemize} % {
\iusr{Евгения Бильченко}
\textbf{Борис Никифоров} Вот мое сокровище олдскульное, это вы. Вы окна малиновые поняли. Вы один поняли. Это был наш диалог. Думаю, скоро общение станет личным и "ты - ты".

\iusr{Борис Никифоров}
\textbf{Евгения Бильченко} уже давно))
\end{itemize} % }

\iusr{Victoria Popsueva}
Иногда прививка - это просто прививка  @igg{fbicon.wink} 

\begin{itemize} % {
\iusr{Евгения Бильченко}
\textbf{Victoria Popsueva} Фрейд))
\end{itemize} % }

\iusr{Olga Smolina}
Коронавак лучше. Легче и проверен на большом количестве китайцев.

\iusr{Ирина Сербина}

У меня Спутник, зашёл нормально, только плечо пару дней поболело и чуток ломало
кости и тянуло мышцы также пару дней, но мне кажется что в Вашем случае нужно
наверное ещё раз подумать, ну или как минимум на момент вакцинации не должно
быть обострения

\iusr{Elena Solobaenko}

ЖЕНЯ ВЕРУЮЩИХ ПРИВИВАЮТ ДУХОВНО, ПРИЧАСТИЕ ТА ЖЕ ВАКЦИНА ЗОЛОТКО И НАДО ЖЕ КОМУ
ТО В ПОПУЛЯЦИИ ОБЕСПЕЧИТЬ КОЛЛЕКТИВНЫЙ ИММУНИТЕТ

\iusr{Olya Kharchenko}

Самое умное решение привиться тем, что вы хотите и купить сертификат той, что
надо. Но, если не будет расследований, потому как слышала, что за это
наказывать сильно будут.

\begin{itemize} % {
\iusr{Евгения Бильченко}
\textbf{Olya Kharchenko} 

Ну, я не хочу украинскую вакцину, я ей не верю. Вера - это
залог мобилизации сил. Никакой политики. Личное переживание.

\iusr{Тим Печеніг}

\obeycr
Волає еліта країни:
– Народе! Ми вкажемо путь!
От тільки російську вакцину
Навіки, рідненький, забудь.
Як вколеш її, небораче,
Прийде тобі повна хана,
Бо вскочить страшенна болячка,
До того ж, іще не одна.
Затрусить тебе лихоманка,
В могилу потрапиш умить...
– Даруйте, пани та панянки,
Щось важко мені зрозуміть.
Як кажуть в Росії, «однако»
Я бачив – в елітній юрбі
Шкідливу вакцину у...попу
Ви нишком кололи собі.
– Ти правий, наш синку хороший,
Відомо, і це не секрет:
На кляту вакцину та гроші
Ми маємо імунітет.
2020.ян Таксюр
\restorecr

\end{itemize} % }

\iusr{Taty Vin}

Я была против вакцинации, то есть против мандата/насильного вакцинированная
(здесь в Америке это уже распространятся на всех федеральных работников, мед
персонала и работников учебных заведений, а на очереди все дети Калифорнии и
там дальше по списку). Мой семилетний ребёнок перенос ковид на ногах - дети
легко и безболезненно ее переносят, но я заразилась от своего сына, и то, что
сейчас испытываю некому не пожелаю. Это какой-то нескончаемый ужас. Так что
берегите себя и по возможности вакцинируетесь. Если это облегчает болезнь, то я
только за.


\iusr{Алла Каргина}

Любая прививка человеку с ослабленным иммунитетом не показана. Это аксиома. Тем
более в период если не эпидемии или пандемии, так в период повышенной
заболеваемости. Прислушайтесь к комментарию светлой головы нашей медицины,
человека, не предавшего своего честного имени, - Александра Алексеевича Редько.

\begin{itemize} % {
\iusr{Евгения Бильченко}
\textbf{Алла Каргина} знаю такого, да, уважаю. Мой дедушка был таким.
\end{itemize} % }

\iusr{Natalie Gadzhi}

Уважаемая Евгения, акцина не пускается по венам. Она остаётся в мышце, где и
формируется иммунный ответ - как-то так говорят иммунологи.

Я кололась Лайтом, но у нас в Донецке выбора особо не было. Может, вам уколоть
что хотите, а сертификат купить? К плохому подбиваю, каюсь. Но какой-то выход
ведь должен быть.

\begin{itemize} % {
\iusr{Olya Kharchenko}
\textbf{Natalie Gadzhi} самое умное решение

\iusr{Евгения Бильченко}
\textbf{Natalie Gadzhi} Хорошее советуете. Я знаю, что не по венам. Это метафора одного товарища питерского.
\end{itemize} % }

\iusr{Elena Solobaenko}
ГЛАВНОЕ НЕ БОЯТЬСЯ НА ДРУГИХ ВИБРАЦИЯХ СМЕХА РАДОСТИ СЧАСТЬЯ ВОЛИ ТЕБЯ ПРОСТО НЕ ПОЙМАЮТ НЕ ДОСТАНУТ

\iusr{Ирина Новикова}
Сделайте тест на антитела в Синево или ещё где. Может, у вас там их столько, что можно вашей кровью лечить @igg{fbicon.face.nerd} 

\begin{itemize} % {
\iusr{Евгения Бильченко}
\textbf{Ирина Новикова} 

Я сдавала антитела, когда дед умирал, в клинике "Лисод": там же без этого не
пускают, там люди умирают от своего. И, дивно, но они тогда были. Но антитела
держатся недолго, я с тех пор болела столько, что, да, реально могу раздавать.
Может, и так.

\iusr{Ирина Новикова}
\textbf{Евгения Бильченко} 

я болела в середине декабря 2020г, сдавала кровь на антитела в апреле и недавно
,в начале ноября 2021 , так мои антитела увеличились в три раза в сравнении с
апрельскими, Андрей Пальчевский вот тоже говорил , что у него антитела
повысились в сравнении с теми, которые были после болезни... послушайте ещё
иммунолога Надежду Желобак , которая говорит, что при наличии антител вакцина
не нужна

\iusr{Алена Морозкина}
\textbf{Евгения Бильченко} антитела вырастают при новом столкновении с вирусом. Профессор Зверев говорит об этом и у сестры моей такое же случилось...
\end{itemize} % }

\iusr{Натали Пилат}

Если Вы понимаете, что у Вас слабый организм и Вам ранее не делали прививки
именно по этой причине, то какой смысл сделать эту?????

ПЦР ничего не показывает. Не выбрасывайте деньги на ветер. Есть лечение. 5
дней. В Украине ещё есть адекватные врачи, которые не в системе заготовки
орг@@@в из выпотрошеных короновых трупов. Если и одобрят прививку, то по
крайней мере самую щадящую, а не убойный ф-зер!!! Если доктор начинает Вас
убеждать, что это единственный выход, вставайте и уходите. Ищите адекватного.

В любом случае процедура вакцинации должна быть проведена не в торговом центре,
не на вокзале, не в каком-то замухрыженом магазинчик, любезно предоставившем
своё помещение. А в медицинском учреждении, где доктор НЕСЁТ ОТВЕТСТВЕННОСТЬ за
последствия. Где Вас не станут заставлять подписывать добровольное согласие на
участие в эксперименте.

\iusr{Владимир Павлович}
Болеем, последнее время, не знаю тех ,кого пронесло...

\iusr{Наталия Медведева}
Будьте здоровы и живите богато.
Решили-делайте. зачем пост?
Для смелости? @igg{fbicon.face.smiling.eyes.smiling} 

\iusr{Alex Melnik}
Сегодня пойду к Буквоеду за "Китчем" БЖ!
А вы, Женя, здравствуйте!
Я резистентный антиваксер с отрицательным резусом

\iusr{Alex Kot}
Я не привьюсь, и точка. Слишком важна связь с Б-гом. Хочется быть егошним творением, а не Васьки Тряпкина ошибкой.

\iusr{Тим Печеніг}
Вирішувати вам , Євгенія Віталіївна , але моє власне Спостереження: треба робити щеплення

\iusr{Тим Печеніг}

Я в Ізраїлі мешкаю , й можу вас запевнити що саме через завчасні заходи по
щепленню наша країна оговталась від наслідків ковід обмеженнь та пандемії

\iusr{Дмитрий Соколовский}

Дело в том, что нет никакой вакцинации от ковида, это прикрытие для
эксперимента над людьми.

Все имеющиеся вакцины это экспериментальные препараты, не прошедшие полный цикл
испытаний, с неизвестно какими последствиями.

Участвовать в этом эксперименте или нет — каждый решает сам.

\begin{itemize} % {
\iusr{Olya Kharchenko}
\textbf{Дмитрий Соколовский} 

в начале всякой пандемии прививки не до конца испытанные. Вот те самые, что нам
короли в детстве тоже когда то были первый год.

\iusr{Евгения Бильченко}
\textbf{Дмитрий Соколовский} Да, это начало новой эры. Мы живём в ней.

\iusr{Гамельнский Крысолов}
\textbf{Olya Kharchenko} 

Тогда государство думало о здоровье граждан. А ныне, когда медицина
"оптимизировалась" и коммерциализировалась, когда много "лишних "людей... и,
ВДРУГ, такая "забота" властей о здоровье граждан? Как говорил Станиславский...

\iusr{Olya Kharchenko}
\textbf{Гамельнский Крысолов} 

все может быть. Сам вирус может быть для регулировки населения. Так что не
спасём я по любому, если захотят

\iusr{Тим Печеніг}
\textbf{Дмитрий Соколовский} , є. ,й приклад Ізраїлю це яскраво доводить

\iusr{Дмитрий Соколовский}
\textbf{Тим Печеніг} 

кто-то, не помню кто, писал: есть большая ложь, есть маленькая, а есть
статистика.

По другой информации, там сейчас большинство больных это вакцинированные.

\iusr{Тим Печеніг}
\textbf{Дмитрий Соколовский} ,

уважаемый , я просто не только здесь живу ,но и постоянно интерсуюсь ,не по
статистикам новостных источников , а реально от людей там работающих и людей
кому повезло из короновирусных отделений больниц выйти на своих ногах ,пусть и
отчасти подорванным ,но все же сохраненным здоровьем , так вот, большинство
тяжело заболевших непривиты

\end{itemize} % }

\iusr{Татьяна Стаховская}

Вот я тоже так дергалась между двумя огнями. Врачи опасались давать добро на
прививку, потому что были проблемы с картиной крови. Приняла решение, что
прививке таки быть и в итоге не успела. Чуть не сдохла от короны. Пневмония,
плеврит, капельницы, уколы в живот и кислород. Спасибо врачам - откачали,
подлечили. Именно подлечили, потому что до выздоровления очень далеко -
постковид во всей красе. Через положенное время таки привьюсь, пока что есть
время, чтобы подумать чем именно.

\begin{itemize} % {
\iusr{Евгения Бильченко}
\textbf{Татьяна Стаховская} Помню. Та же картина: кровь не в звезду. И это сказал врач из Нижнего, я ему верю. Он меня летом ставил.
\end{itemize} % }

\iusr{Тим Печеніг}

Непроста ділема , багато шляхів дії ,придбання підробного сертифікату ,як
запропонував тут дехто вище варто відразу відкинути , це

\ifcmt
  ig https://scontent-lga3-2.xx.fbcdn.net/v/t39.30808-6/254129793_10220239277081144_703392052823240177_n.jpg?_nc_cat=103&ccb=1-5&_nc_sid=dbeb18&_nc_ohc=scOfNIdCuiIAX_GgNeK&_nc_ht=scontent-lga3-2.xx&oh=01aa0aaf4ef988036f28e13183e037fe&oe=6191AD2F
  @width 0.4
\fi

\end{itemize} % }
