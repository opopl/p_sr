% vim: keymap=russian-jcukenwin
%%beginhead 
 
%%file 10_06_2021.fb.krjukova_svetlana.1.vopros_ljubov_deti
%%parent 10_06_2021
 
%%url https://www.facebook.com/kryukova/posts/10159459498718064
 
%%author Крюкова, Светлана
%%author_id krjukova_svetlana
%%author_url 
 
%%tags chelovechnost,chelovek,deti,ljubov,psihologia,vopros
%%title Дети - что такое Любовь - Вопрос
 
%%endhead 
 
\subsection{Дети - что такое Любовь - Вопрос}
\label{sec:10_06_2021.fb.krjukova_svetlana.1.vopros_ljubov_deti}
\Purl{https://www.facebook.com/kryukova/posts/10159459498718064}
\ifcmt
 author_begin
   author_id krjukova_svetlana
 author_end
\fi

Группа исследователей задавала детям от 4 до 8 лет один и тот же вопрос: «Что значит любовь?»

Ответы оказались намного более глубокими и обширными, чем кто-либо вообще мог себе представить...

И это расширяет наше понимание любви, выводит её за пределы любви-секса между мужчиной и женщиной.
Итак,

«Когда моя бабушка заболела артритом, она больше не могла нагибаться и красить
ногти на ногах. И мой дедушка постоянно делал это для неё, даже тогда, когда у
него самого руки заболели артритом. Это любовь» 6 лет.

«Если кто-то любит тебя, он по-особенному произносит твоё имя. И ты знаешь, что
твоё имя находится в безопасности, когда оно в его рту» 4 года.

«Любовь - это когда ты идёшь куда-то поесть и отдаёшь кому-нибудь бо́льшую часть
своей жареной картошки, не заставляя его давать тебе что-то взамен» 6 лет

«Любовь - это то, что заставляет тебя улыбаться, когда ты устал» 4 года.
«Любовь - это когда моя мама делает кофе папе, и отхлёбывает глоток перед тем, как отдать ему чашку, чтобы убедиться, что он вкусный» 7 лет.
«Любовь - это когда ты говоришь мальчику, что тебе нравится его рубашка, и он носит её потом каждый день» 7 лет.
«Любовь - это когда твой щенок лижет тебе лицо, даже после того, как ты оставила его в одиночестве на весь день» 4 года.
«Когда ты любишь кого-нибудь, твои ресницы всё время взлетают и опускаются вверх-вниз, а из-под них сыплются звёздочки» 7 лет.
«Любовь - это когда мама видит папу в туалете и не думает, что это противно» 6 лет.
«Если ты не любишь, ты ни в коем случае не должен говорить «я люблю тебя». Но если любишь, то должен говорить это постоянно. А люди забывают» 8 лет.

Автор исследования Лео Баскаглиа однажды объяснил смысл этого опроса. Целью было найти самого заботливого ребёнка. 

Так вот, победителем стал четырёхлетний малыш, чей старенький сосед недавно потерял жену.

Увидев, что мужчина плачет, ребёнок зашёл к нему во двор, залез к нему на
колени и просто сидел там. Когда его мама позже спросила, что же такого он
сказал соседу, мальчик ответил:

«Ничего. Я просто помог ему плакать».
