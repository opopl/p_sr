% vim: keymap=russian-jcukenwin
%%beginhead 
 
%%file 18_01_2022.stz.news.ua.hvylya.1.katarsis
%%parent 18_01_2022
 
%%url https://hvylya.net/analytics/245614-ukrajina-koli-nastane-katarsis
 
%%author_id news.ua.hvylya,romanchuk_oleg
%%date 
 
%%tags 
%%title Україна: коли настане катарсис?
 
%%endhead 
\subsection{Україна: коли настане катарсис?}
\label{sec:18_01_2022.stz.news.ua.hvylya.1.katarsis}

\Purl{https://hvylya.net/analytics/245614-ukrajina-koli-nastane-katarsis}
\ifcmt
 author_begin
   author_id news.ua.hvylya,romanchuk_oleg
 author_end
\fi

\begin{zznagolos}
Основними факторами, які спричиняли упадок нації і упадок держави – були: зовнішня агресія і внутрішня зрада	
\end{zznagolos}

«Якщо подивитися на історію української нації та Української держави в глибину
віків і у недавнє минуле, без великих зусиль можна встановити, що основними
факторами, які спричиняли їхній упадок – упадок нації і упадок держави – були:
зовнішня агресія і внутрішня зрада», – це непроминальне спостереження знаного
українського історика Ярослава Дашкевича, на жаль, постійно знаходить своє
підтвердження.

\ii{18_01_2022.stz.news.ua.hvylya.1.katarsis.pic.1}

У минулорічному популістському посланні до Верховної Ради Зеленський озвучив
вкрай сумнівну з позиції здорового глузду ідею щодо прямих переговорів з
російською стороною, читай: російським президентом: «І ми повинні сказати
правду, що ми не зможемо зупинити війну без прямих перемовин з Росією».

Незабаром заступник голови офісу президента України Андрій Сибіга підтвердив:
«Президент України готовий до прямих переговорів з президентом Путіним.
Українська сторона готова до цих переговорів, має відповідну позицію».

Невдовзі під час спілкування з представниками мас-медій у Брюсселі Зеленський
повідомив, що «Україна готова до будь-якого формату перемовин з Російською
Федерацією щодо врегулювання цього процесу (ситуації на Донбасі. – О. Р.). Ми
про це говоримо відкрито і вже не вперше. Потрібне таке саме бажання зупинити
війну і з боку Російської Федерації».

Наступного дня речник президента РФ Дмитро Пєсков заявив, що Володимир Путін
готовий готовий зустрітися з очільником України Володимиром Зеленським. За
словами Пєскова, Росія зацікавлена у діалозі і, мовляв, Путін відкритий для
діалогу.

Щодо такої більш ніж сумнівної ініціативи Семен Глузман, відомий український
лікар-психіатр, правозахисник має власну думку: «Чи може Зеленський сам по собі
тягатися с президентом Росії Владіміром Путіним? Ні, звичайно. Зеленський –
ніхто… У нього козир один – за його спиною стоїть цивілізований світ».

Наразі цивілізований світ стоїть. Але цього «козиря» замало. Допомагають
мужнім, сильним, розумним, цілеспрямованим.

«Перемовини з Путіним, так само, як і з Гітлером, як засвідчив досвід Другої
світової війни, можливі лише з питання його капітуляції. Саме на таку позицію
свого часу стала Велика Британія, залишившись узагалі сам на сам із
гітлерівською Німеччиною. Стала й перемогла! Утім, про це треба періодично
нагадувати і українському керівництву» (Олександр Самарський, експерт центру
дослідження Росії, екс-посол України в Ірані).

У новорічному зверненні Володимир Зеленський пояснив посполитим, перейшовши на
російську, коли на українській мапі зникне лінія розмежування з окупованим
Кримом і Донбасом: «И как только мы уберем линию разграничения в голове, она
исчезнет и на карте».

Це виразно проросійське послання по суті є наративом Путіна в його несамовитому
прагненні реанімувати СССР. На думку колишнього генерального прокурора України
Юрія Луценка, президент України Володимир Зеленський хоче зустрітись з
президентом РФ Володимиром Путіним через те, що той має на нього компромат:
«Мені здається, що така поїздка означає тільки одне - бажання закрити компромат
на себе особисто. Той компромат, який має Путін і російські спецслужби проти
нього. Не хочу говорити, який він, хоча дещо знаю».

Раніше Володимир Зеленський планував винести на референдум питання про
олігархів, нині готовий ініціювати інший захід: «Я не виключаю референдуму щодо
Донбасу в цілому».

Про який референдум може йти мова? Питання суверенітету, територіальної
цілісності, державного устрою не можуть бути винесені на референдум. Це замах
на Констутуцію. Тим паче, що українське суспільство незріле. Це підтверджують
наслідки голосування 2019 року. А ще низький рівень відповідальності та
освіченості. Скільки громадян тримали в руках Конституцію? Скільки відсотків
українців розуміють, що таке держава? Який алгоритм її функціонування. Якщо я
нічого в цьому не тямлю, то навіщо маю голосувати?

Ідея референдуму цілком могла народитися в кремлівському «захристії», як і
ймовірне формулювання питання для винесення на всеукраїнське обговорення: «Чи
підтримуєте ви, що заради миру, добробуту і процвітання України треба надати
особливий статус Донбасу?»

Хто ж не проголосує за таке? За мир, за дружбу, жуйку…Ось і секретар Ради
національної безпеки і оборони Олексій Данілов допускає можливість якогось
компромісу з Російською Федерацією: «Чи можемо піти на якісь компроміси? Так,
можемо. Але ці компроміси повинен сприйняти наш народ. Ми не можемо
безкомпромісно щось вирішувати і йти на поступки в тому, що на сьогоднішній
день пропонує Російська Федерація».

Про які компроміси йдеться? Про яке спілкування з ворогом може йти мова? Які
компроміси повинні/можуть сприйняти українці?

Такі гучні заяви лише породжують і посилюють в суспільстві стан невизначеності.
Бо найменші спроби втілення цих заяв можуть призвести в до дестабілізації
соціально-політичної ситуації в державі.

Оточення Володимира Зеленського, так чи інакше пов'язане з російськими
політичними інтересами, цілком може наполягати на укладенні сумнівних
домовленостей з Московією. І це насторожує. Чи не про сепаратні переговори з
Кремлем йдеться? Бо навіть держсекретар США Ентоні Блінкін наголошує, що Росія
не виконала жодного положення Мінського процесу. Замість того, щоб вирішувати
якісь питання і сприяти просуванню Мінського процесу, вона намагається
представити себе не агресором, а стороннім спостерігачем.

У квітні минулого року Сєргєй Лавров в інтерв'ю російським ЗМІ погрозливо
заявив: «Вважаю, що ми не повинні «спускати з гачка» (йдеться про Мінські
домовленості. – О. Р.) пана Зеленського і всю його команду, а вони звиваються,
як можуть». Ще раніше Лавров висловив думку, що в разі зміни Мінських угод на
Донбасі «почнеться різанина».

Так чи інакше, однак всі політичні кульбіти Кремля замикаються на непогамовному
бажанні Путіна підкорити Україну та повернути її під контроль Москви. Для
Путіна це modus operandi. Як реагує Київ?

21 грудня минулого року під час загальної наради послів у державній резиденції
«Синьогора» Володимир Зеленський у спілкуванні з журналістами несподівано
заявив, що передав нові пропозиції усім, хто може брати участь у майбутніх
переговорах з РФ: «Ми написали чіткий, зрозумілий план з 10 кроків, як виконати
мінські угоди. Це 10 кроків із залученням Нормандії та США, і передали ці кроки
президенту Байдену, і російській стороні, і Франції, і Німеччині. Ми готуємо
різні пропозиції, щоби нам не закидували, що ми не хочемо (знайти шлях
завершити війну)».

Щоправда жодних подробиць про зміст документа українська сторона не розкрила.
Натомість впливова російська газета «Коммерсантъ» 24 грудня опублікувала текст
згаданого українським президентом документа з так званими кроками назустріч
мирові на Донбасі у всіх існуючих форматах. Привертає увагу «Крок 4». Цитуємо
«…українська сторона буде готова на засіданні ТКГ повернутися до розгляду
проекту закону України «Про особливий порядок місцевого самоврядування в
окремих районах Донецької та Луганської областей України» з імплементованою в
ньому «формулою Штайнмайєра».

Тепер зрозуміло, чому Зеленський не розкрив для громадян України зміст цього
покрокового плану?

Є й інші «кроки». Більш ніж сумнівні. Які викликають чимало запитань:

«Крок 7. Підготовка та погодження порядку денного та підсумкових документів у
рамках підготовки зустрічі президента України та президента Російської
Федерації.

Крок 8 . Проведення зустрічі президента Російської Федерації та президента
України.

Крок 9. Проведення саміту лідерів Нормандського формату.

Крок 10. Погодження в рамках ТКГ наступних законів із подальшим внесенням їх до
Верховної Ради України: про особливості місцевого самоврядування ОРДЛО з
імплементованою в нього «формулою Штайнмайєра»; про амністію; про
децентралізацію (з урахуванням особливостей ОРДЛО); про спеціальну (вільну)
економічну зону; про особливості проведення місцевих виборів до ОРДЛО».

Залишається сподіватися, що Верховна Рада не затвердить ці сумнівні
«компроміси», інакше вони знищать Українську державу.

Коментуючи «десять кроків» Володимира Зеленського, російський представник у ТКГ
Борис Гризлов зазначив, що вони були передані Росії на початку грудня в
неофіційному порядку, «але ніколи не передавались Києвом на розгляд до
контактної групи».

Треба думати, що напередодні Нового року українська сторона не хотіла нового
скандалу – оприлюднення чергової ідеї щодо легалізації терористичних республік
через референдум або безпосередньо через контрольований «слугами народу»
парламент.

Відомий російський публіцист і політолог Андрєй Піонтковський не без сарказму
та іронії змоделював монолог Байдена під час його телефонної розмови з
українським президентом: «Я зробив для Вас велику справу, Володимире. Здається,
мені вдалося запобігти масштабному вторгненню Путіна в Україну. Але й Ви
повинні зробити частину своєї роботи. Поцілуйте лиходію ручку і погодьтеся на
його інтерпретацію мінських угод. Так, так. І на особливий статус ОРДЛО, і на
«формулу Штайнмайєра», і на вибори під контролем окупантів. Зрештою, йдеться
лише про degree of autonomy однієї з українських провінцій. Ми з Вами не можемо
наражати людство на ризик світової війни з такого незначного приводу.
Американський виборець нас не зрозуміє».

Безумовно, це лише дотепна імітація поради Байдена Зеленському. Але. Посол США
при НАТО Джуліан Сміт 11 січня зробила заяву: «Наше послання залишається
незмінним: ми хочемо бачити повну імплементацію Мінських угод із відведенням
Росією її сил. Але якщо Росія продовжить конфронтацію – будуть серйозні
наслідки».

Хай там як, але йдеться про «повну імплементацію Мінських угод». Передусім. Тож
коментуючи візит до Києва радників із зовнішньої політики канцлера Німеччини та
президента Франції екс-посол України в США Валерій Чалий в ефірі програми
«Україна Сьогодні» на телеканалі «Україна 24» висловив занепокоєння, що
Володимир Зеленський готує ґрунт для виконання Мінських домовленостей в
інтерпретації Росії і незабаром від української влади очікуються компроміси
щодо Донбасу, які раніше вважалися «червоними лініями».

За словами дипломата, Росія нав’язує українцям обговорення членства в НАТО,
проте це «прикриття абсолютно інших питань, які стануть очевидними за
тиждень-два-три». Чалий вважає: «Президент Зеленський сказав дуже коротко – «ми
готові до необхідних рішень щодо мінського процесу». Це, я думаю, ключове
питання, яке нам доведеться «вирулювати», зокрема дипломатично. Це і «формула
Штайнмайєра», і вільна економічна зона на Сході, і повна амністія – це все те,
що після грудня 2019 року не наважилася зробити українська влада, а зараз вони
готують ґрунт – очевидно для того, щоб це все спробувати зробити. Таким чином,
виконати умови, які висуває Росія, і встановити мир. Такий приблизно план».

Дипломат зазначив, що «для європейських партнерів це прийнятна історія, для США
в більшості прийнятна, щоб вирішилося питання за рахунок України, а не за
рахунок ширшого контексту, де Україна має отримати ПДЧ уже цього року».

Складається враження, що Банкова готова на втілення вкрай ризикованого для
нашої державності політичного сценарію. Його автори використовують Зеленського
з його комплексами і некомпетентністю для розвалу Української держави. Діють,
ймовірно, за лекалами трагічного «Мюнхена-38». І можна уявити, як в
міжнародному аеропорту «Бориспіль» з трапу літака, що прибув з Москви,
прозвучить: «Я привіз вам мир». Прозвучить бадьоро, цілком у стилі 95-го
Кварталу.

«Зараз найголовніше – щоб оточення Зеленського не здало нас Росії. США
стурбовані щодо закулісних переговорів за їхньою спиною. Основна тема
переговорів – компроміси. Вони хочуть миру за всяку ціну. Імплементація формули
Штайнмаєра, амністія та все решта. Це принесе мир на Донбас, але якою ціною? Це
може розколоти Україну та викликати соціальний вибух усередині країни, тобто ту
саму громадянську війну, про яку люблять говорити в Кремлі» (Олег Жданов,
військовоий експерт).

Інакше кажучи, може спрацювати ефект доміно. Згадаймо недавню історію. 31
серпня 2015 року на позачерговому засіданні Верховної Ради народні депутати
мали ухвалити в першому читанні законопроект про зміни до Конституції у частині
децентралізації, який розширював повноваження місцевих громад та визначав
«особливості місцевого самоврядування» на тимчасово окупованих територіях
Донецької та Луганської областей, тим самим надаючи їм ширші повноваження, ніж
мають інші регіони України. До тексту Основного Закону планувалося включити
доволі суперечливий пункт Перехідних положень. Так, у фразі «Особливості
здійснення місцевого самоврядування в окремих районах Донецької і Луганської
областей визначаються окремим законом» чимало політиків вбачали надання
непідконтрольній Києву частині Донбасу «особливого статусу», розцінюючи це як
«капітуляцію» України у протистоянні з Росією.

Однак Верховна Рада законопроєкт у першому читанні таки ухвалила. У висліді
біля стін парламенту відбулися жорсткі сутички представників різних політичних
сил, які протестували проти внесення змін до Конституції, та міліцією. У
результаті зіткнень загинули четверо бійців Нацгвардії, постраждала 141 людина
(зокрема 131 міліціонер і боєць Нацгвардії). Тоді ефект доміно, на щастя, не
спрацював…

7 січня під час брифінгу державний секретар США Ентоні Блінкен зазначив, що
було б не дивно, якби Росія влаштувала який-небудь інцидент або провокацію аби
використати її для обґрунтування військової інтервенції в Україні.

Американський високопосадовець має рацію. Щоправда повторення провокацій типу
захоплення радіостанції в Ґляйвіці чи Майнільського інциденту не буде. Існують
інші випробувані Московією приводи до розв’язання війни – зокрема так званий
«захист співвітчизників» за допомогою «миротворців» або «ввічливих чоловічків».
Про це в інтерв’ю «РИА Новости» 31 грудня 2021 року відверто заявив міністр
закордонних справ Росії Сергій Лавров: «Что касается жителей Донбасса, где
проживают сотни тысяч граждан нашей страны, то Россия предпримет все
необходимые меры для их защиты».

\ii{18_01_2022.stz.news.ua.hvylya.1.katarsis.pic.2}

Відповідь на питання, звідки взялися на Донбасі «сотни тысяч граждан» Росії
лежить, як мовиться, на поверхні. Про це добре знає і Москва, і Київ, і
Вашингтон… «Тотальна російська паспортизація… на території України є формою
агресії. І вона у фіналі завжди закінчується воєнним вторгненням. На прикладі
Донбасу і Криму ми це побачили. Але це не тільки історія Росії та Путіна. Це
історія Німеччини. Як вона загарбала Судети, а потім Австрію. Путін йде тією ж
стежкою» (Григорій Перепелиця, директор Інституту зовнішньополітичних
досліджень, доктор політичних наук).

Продовжує надходити й інша тривожна інформація. Так, 14 січня Головне
управління розвідки Міністерства оборони України повідомило у Facebook, що
російські спецслужби готують провокації проти військовослужбовців збройних сил
РФ, щоб звинуватити в цьому Україну. Зокрема, у повідомленні сказано: «13 січня
у 3-й окремій мотострілецькій бригаді Збройних сил Придністровського регіону
Республіки Молдова було проведено службову нараду під час якої особовому складу
доведено, що очікуються провокації зі сторони України в районі н. п. Ковбасна
(артсклади оперативної групи військ РФ)».

14 січня Патріоти України

(\href{https://patrioty.org.ua/politic/rosiia-hotuie-provokatsiiu-dlia-vtorhnennia-do-ukrainy-bilyi-dim-hotovyi-opryliudnyty-dokazy-406573.html}{
\enquote{Росія готує провокацію для вторгнення до України}: Білий дім готовий оприлюднити докази, patrioty.org.ua, 14.01.2022%
})

з посиланням на CNN повідомили, що Адміністрація президента США має докази, що
Росія під чужим прапором на Донбасі підготувала бойовиків для проведення
операції, яка має створити привід для російського вторгнення в Україну.

11 січня заступниця державного секретаря США Вікторія Нуланд, виступаючи перед
ЗМІ, заявила: «Саме Росія приготувала внутрішній саботаж, дестабілізацію, та
варіанти [дій] під чужим прапором (false flag – англ.) для України». Саме Росія
вивергнула дезінформацію та брехню про Україну, США та НАТО, щоб виправдати її
власні дії».

Думка слушна. Однак спровокувати масові заворушення своїми нерозважливими
(продуманими?) діями цілком може й українська влада, підштовхуючи громадян до
нового Майдану.

У розмові з виданням «Лєнта.ру» генеральний директор Російської ради з
міжнародних справ Андрій Кортунов озвучив один із сценаріїв вторгнення РФ в
Україну – дочекавшись «буквального розвалу України», Москва зможе почати
«збирати уламки» у вигляді регіонів, близьких до російського культурного
простору...

\ii{18_01_2022.stz.news.ua.hvylya.1.katarsis.pic.3}

Історія з арештом активів Порошенка не просто скандальна. «Це вже близько до
божевілля Зеленського-Єрмака-Венедіктової – переслідувати колишнього президента
Порошенка, коли російські війська скупчуються на кордоні. Чи думають ці люди
про щось, крім себе?» – дивується старший науковий співробітник «Atlantic
Council» Андерс Аслунд, який, до речі, про Порошенка-президента писав вельми
критичні статті.

17 січня в інтерв’ю головному редакторові «Української правди» Севгіль Мусаєвій
Петро Порошенко до евфемізмів не вдавався: «Зеленський – людина неадекватна і
це небезпечно для країни». Псевдоюридична вистава, спартолена того ж дня
сценаристами зеленого «кварталу» в Печерському суді Києва 17 січня, – це не
драма і не фарс, не буфонада і не фантасмагорія. Це брутальне судилище.

Питання по очільнику «Європейської солідарності» суто політичне. Йдеться про
політично вмотивоване переслідування очільника політичної партії. Вести мову
про верховенство права через численні процесуальні порушення, через відверту
заангажованість процесуальних керівників, через брак належним чином оформлених
документів про підозру у державній зраді за відсутності самого факту злочину не
доводиться. Це знущання над здоровим глуздом. І це переконливо демонстрували
адвокати п’ятого президента України і сам Петро Порошенко.

У цьому контексті дивною виглядає історія Віктора Медведчука, запідозреного у
державній зраді – ст. 111 Кримінального кодексу України і сприяння діяльності
терористичної організації – ч.1 ст. 258-3. Цей нардеп від ОПЗЖ перебуває під
домашнім арештом. Уже не один місяць. Однак його активи ніхто не конфісковує...

Sic! У жовтні минулого року в інтерв'ю ICTV Володимир Зеленський несподівано
зазначив: «Віктор Медведчук – громадянин України і до нього застосовується
українське законодавство. З іншого боку, якби Росія, наприклад, підтвердила, що
пан Медведчук має російське громадянство, то після цього ми з радістю
розглянули б можливість обміну Медведчука на героїв України, громадян, тих, хто
перебуває не в Україні».

\ii{18_01_2022.stz.news.ua.hvylya.1.katarsis.pic.4}

Як мовиться, без коментарів.

Не тільки Петра Порошенка переслідує нинішня влада, але й інших активних
лідерів опозиції. Такі дії української Феміди, безумовно, роз’єднують
суспільство і грають на руку країні-агресору.

Існують ймовірні сценарії, здатні розхитати суспільство, викликати соціальну
напругу. Так, в Україні бракує вугілля та газу – його запаси в підземних
сховищах майже вдвічі менші, ніж роком раніше. Шалена продовольча інфляція. За
даними Державної служби статистики, за минулий рік ціни на харчові продукти
зросли майже на 12\%. Бракує якісного зерна для хліба, призупиняють виробництво
маслозаводи та хлібозаводи, змушені призупиняти виробництво підприємства
кондитерської галузі. Маємо рекордне за 25 років зростання промислових цін. У
2020 році ціни виробників промислової продукції зросли на 14,5\%, а торік –
одразу на 62,2\%. Можливе запровадження карток на продукти – «слуги народу»
пропонують запровадити електронні продуктові картки для бідних…

Верховний представник ЄС із закордонних справ та політики безпеки Жозеп Боррель
після відвідування лінії розмежування в Луганській області України 5 січня дав
зрозуміти, що Євросоюз визнає Росію стороною збройного конфлікту на сході
України: «Немає сумніву, що Росія є стороною цього конфлікту, а не
посередником, як вона часто стверджує». Справдедливі слова, але Сенат США
провалив законопроєкт від республіканців про санкції проти «Північного
потоку–2»...

Путін уже довів, що Росія дипломатію не сприймає. Висновок для України:
найвагомішим ґрантом успішних дипломатичних перемовин і надійною запорукою миру
є сильна армія. Але. «Національна зрада, безкарна, нахабна, безсоромна, зухвало
захоплює чимраз більше ділянок нашого політичного, військового, економічного,
громадського життя при повному потуранні, бездіяльності, а в багатьох випадках
– при сприянні та підтримці українських власть-імущих найвищих, середніх та
низьких рангів» (Ярослав Дашкевич, історик).

Про невдачі української розвідки через засилля «кротів» не писав хіба що
лінивий. Та є інші драстичні речі. В умовах війни Міністерство оборони для
потреб Збройних сил України закупило лише третину від необхідної кількості
дизельного пального. На 14 грудня було закуплено 29\% від потреби. Реального
пального є лише 17\%. І це під час реально існуючої загрози вторгнення...

Головнокомандувач Збройних Сил України Валерій Залужний в інтерв’ю
новинній агенції «Укрінформ» заявив: «Давайте говорити прямо й відкрито: ми
зараз маємо справу з армією Російської Федерації, яка на сьогодні налічує майже
мільйон військовослужбовців... я доводив до наших партнерів по НАТО, що це не
ілюзія, це реальна загроза. Варто враховувати, що нам протистоїть
держава-агресор, у якої є майже мільйонне військо».

Натомість Верховний Головнокомандувач святкує ювілей голови свого офісу,
розважається на концерті свого «Кварталу».

Новорічне поздоровлення Володимира Зеленського записувалось три дні. 4 січня
стало відомо, що український президент відпочиває у Буковелі. Стоячи на лижах,
він привітав українців із Різдвом: «Наша Україна. Наші гори. Бачу наші Карпати.
Наші люди. Усім миру. Усім добра. Усіх із Різдвом!»

5 січня Зеленського помітили під час застілля у шинку з Андрієм Єрмаком і
начебто Русланом Демченком. А 10 січня 2022 року в Instagram-акаунті президента
з’явилася світлина з підписом: «Понеділок. Ранок. Банкова». Щоправда фото це
старе – з минулорічної селекторної наради 29 листопада.

Колишній народний депутат України Ігор Мосійчук повідомив, що глава української
держави начебто разом з дружиною та керівником Офісу президента подалися на
один із курортів в Індійському океані. Два роки тому Зеленський вже потрапляв у
скандал через відпочинок на узбережжі Оманської затоки...

У квітні 2019 року напередодні виборів Президента українські офіцери звернулись
до співвітчизників із закликом не допустити російського реваншу. Вони нагадали,
що відомий російський пропагандист, протоірей РПЦ Всеволод Чаплін заявив, що
кандидат в президент України Володимир Зеленський – саме той президент, який
потрібний Росії. У своєму Twitter-аккаунті Чаплін написав, что Україна стане
Малоросією: «Поражение Порошенко будет означать скорый разворот Украины в
сторону России и Русского Мира. Украинство как идеология обанкротилась».
Українські офіцери нагадали, що в ефірі телеканалу 1+1 (ТСН) Володимир
Зеленський заявив: «Если на Востоке и в Крыму люди хотят говорить по-русски,
отцепитесь, отстаньте от них, дайте им возможность говорить по-русски. У меня
єврейская кровь, я говорю по-русски, но я гражданин Украины».

Чомусь згадалося 31 грудня 2019 року, коли Зеленський і Путін привітали один
одного з Новим роком та взаємним звільненням утримуваних осіб. Три дні
перегодом Володимир Пилипчук, політик, народний депутат України кількох
скликань, науковець, громадський діяч емоційно прокоментував телефонну розмову
президентів: «Таке вітання, в таких умовах може здійснити лише жалюгідна
бутафорія на президента»...

Вельми симптоматично, що з посади позаштатного радника голови Офісу президента
Андрія Єрмака та з посади спікера делегації України в ТКГ щодо Донбасу
звільнився Олексій Арестович. Перед тим він висловився, що в нас держави немає,
що у нас феодальна система. Більш ніж промовиста оцінка дій влади Арестовичем:
«Коли начальник, великий начальник, президент, віддає розпорядження, то воно
буде виконано лише за умови чи суми умов, що це підлеглий, який отримав
розпорядження любить президента, що це нікому не зашкодить, що не цьому він
може заробити, що це схвалюють його коханка, собака, кінь, бізнес-партнери з-за
кордон. І коли йому зовсім вже нічого робити, тоді він може виконати завдання.
І за це нікого не карають». Таке враження, що Арестович нарешті зрозумів, що
коїться на Банковій...

Що робити українцям? Чинити згідно стародавнього девізу лицарів: роби, що
повинен, і будь що буде. Треба готуватися до найгіршого сценарію.

Водночас наполягати, аби українська влада взяла до уваги оприлюднені в
«Дзеркалі тижня» слушні рекомендації та поради Олександра Самарського,
екс-посла України в Ірані, експерта Центру дослідження Росії: «Припинення
будь-яких перемовин у мінському, а можливо — й нормандському форматах. Підстава
– демонстративний де-факто вихід із мінських домовленостей Росії. Як відомо,
всупереч цим домовленостям, РФ відмовляється визнати себе стороною конфлікту,
наполягаючи на ролі посередника у перемовинах між Києвом та ОРДЛО. Таку позицію
РФ, як відомо, заперечують навіть Франція та ФРН. Слід також категорично
припинити будь-які спроби «розморозити» конфлікт на Донбасі. Навпаки, ступінь
«замороження» треба посилювати. <...> Врегулювання здійснюватиметься виключно за
рахунок інтересів України. Не можна міняти Україну на Донбас! А інші варіанти
не проглядаються. Для інших варіантів Україна просто не має необхідного
економічного, військового та політичного потенціалу.<...> Тривале «замороження»
конфлікту — найбезпечніший і найдешевший для України варіант. І не треба
гратися чи вестися на всілякі «гуманістичні» гасла, бо йдеться не багато не
мало — про виживання всієї країни. Відповідальною за ситуацію на окупованих
територіях, згідно з міжнародним правом, є країна-окупант. Усе! І якщо Франція,
Німеччина чи США будуть незадоволені такою нашою позицією, то Україна абсолютно
не заперечувала б проти фінансової підтримки ними необхідного рівня життя
населення окупованих територій. <...> У принципі, якщо говорити дуже відверто, то
доцільно було б узагалі раз і назавжди вийти з мінських домовленостей. Їх
реалізація для України значно шкідливіша, ніж зняття Заходом із Росії своїх
багато в чому недолугих санкцій. Біда Заходу, якщо він не розуміє, що цими
санкціями захищає насамперед сам себе.

Порушення питання про те, що Росія незаконно зайняла місце постійного члена РБ
ООН. РФ не значиться у Статуті ООН серед постійних членів. Ним є СССР, а Росія
– далеко не єдиний правонаступник колишнього Совєтського Союзу.

Розрив дипломатичних відносин із Росією. Наявність таких відносин, із погляду
міжнародно-правової практики, спростовує всі звинувачення Росії в агресії проти
України та окупації українських територій. Вони є ознакою визнання Україною
дружнього характеру відносин із РФ. Тому збереження дипвідносин між нашими
державами вигідне лише Росії...».

І насамкінець. 10 січня 2022 року Вільям Тейлор, виконавчий віцепрезидент
Інституту миру США, колишній посол США в Україні (2006-2009 рр.), колишній
тимчасовий повірений у справах США в Україні з літа 2019 по січень 2020 року в
інтерв’ю «Главкому» вкотре наголосив: «Зробіть перший крок – отримайте статус
головного союзника США поза НАТО. За цим першим кроком можуть йти певні більш
формальні домовленості, наприклад, такі, які має Японія, Південна Корея,
Австралія. Всі ці країни є головними союзниками США поза НАТО і мають додаткові
оборонні угоди зі Сполученими Штатами, у тому числі гарантії безпеки»

Я про це писав неодноразово. Зокрема: «Як Україні отримати статус основного
союзника США поза межами НАТО?», "Безпека України понад усе!"
