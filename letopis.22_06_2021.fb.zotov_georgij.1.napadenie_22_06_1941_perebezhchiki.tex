% vim: keymap=russian-jcukenwin
%%beginhead 
 
%%file 22_06_2021.fb.zotov_georgij.1.napadenie_22_06_1941_perebezhchiki
%%parent 22_06_2021
 
%%url https://www.facebook.com/george.zotov.5/posts/4114881371931280
 
%%author Зотов, Георгий
%%author_id zotov_georgij
%%author_url 
 
%%tags germania,istoria,nacizm,napadenie,preduprezhdenie,sssr,vov,vov.22.06.1941
%%title «Нападут на вас утром 22 июня». Кто пытался предупредить СССР о немецком вторжении
 
%%endhead 
 
\subsection{«Нападут на вас утром 22 июня». Кто пытался предупредить СССР о немецком вторжении}
\label{sec:22_06_2021.fb.zotov_georgij.1.napadenie_22_06_1941_perebezhchiki}
\Purl{https://www.facebook.com/george.zotov.5/posts/4114881371931280}
\ifcmt
 author_begin
   author_id zotov_georgij
 author_end
\fi

«Нападут на вас утром 22 июня». Кто пытался предупредить СССР о немецком вторжении.

...Одежда гостя была мокрой насквозь. Из рукавов рубахи на пол капала вода.
Вздрагивая, он стучал зубами. Мешая русские и польские слова, со слезами на
глазах человек повторял одно: «Они уже наводят переправу. У шоссе, где был
паром. Другую - у большого камня. Я старый солдат русской армии, воевал за
Россию ещё в 1914 году... Хочу помочь. Их офицеры открыто говорят: нападут на
вас утром, в четыре часа».

«Это мельник из Польши. Только что переплыл реку и потребовал доставить его к
вам, - хмурясь, доложил один из бойцов командиру 2-й заставы 17-го погранотряда
Василию Горбунову. - Утверждает, что завтра немцы начнут наступать на Брест...»

Как вспоминал позднее в своих мемуарах младший лейтенант Горбунов, ему и
верилось, и не верилось: «Факты говорят: это война. А разум - нет, это абсурд».
Он сделал попытку связаться с Брестской крепостью, но телефон молчал - провода
к тому времени уже перерезали немецкие диверсанты. Поразмыслив, Горбунов поднял
заставу в ружьё и послал на границу дополнительные наряды с ручными пулемётами.
Именно они и расстреляли в упор надувные лодки с солдатами вермахта, когда те
начали переправляться через Западный Буг. Благодаря предупреждению вторую
заставу не застали врасплох: 22 июня 1941 года, оказывая упорное сопротивление,
она продержалась до 18.00, пока её защитникам не пришлось отступить. По сути
перебежчик спас жизнь множеству людей.

\ifcmt
  pic https://scontent-mxp1-2.xx.fbcdn.net/v/t1.6435-9/202514259_4114871318598952_2936445886497823815_n.jpg?_nc_cat=102&ccb=1-3&_nc_sid=730e14&_nc_ohc=EPcb1uIg3RcAX-vPGdC&tn=ntrKbsW_7ChXu3v-&_nc_ht=scontent-mxp1-2.xx&oh=cd85f2c16c7613b2674ef8cfdfe595aa&oe=60D67F7D
\fi

- Я согласна с мнением, что перебежчиков было несколько, - рассказывает Тамара
Юрчик, сотрудница мемориального комплекса «Брестская крепость-герой». - На это
указывают как архивные источники, так и свидетельства очевидцев. Польского
мельника звали Иосиф Бодзинский. Этот человек, переплыв Буг, прибыл на
территорию второй погранзаставы в 22 часа 21 июня 1941 года, а уже в три часа
ночи пограничники были готовы к бою. Самого Бодзинского отправили в комендатуру
в 8 км от границы и поместили под охрану. Что случилось дальше, непонятно,
однако Иосиф остался жив. В нашем досье одна строчка: «Умер в городе Быдгощ в
конце 60-х гг.». Судя по фотографии, ему было лет 80. Наградили ли его? Я не
знаю...

\ifcmt
  tab_begin cols=3

     pic https://scontent-mxp1-2.xx.fbcdn.net/v/t1.6435-9/202371142_4114871571932260_1115416554059961922_n.jpg?_nc_cat=100&ccb=1-3&_nc_sid=730e14&_nc_ohc=dJElyP-ucBQAX_R9XVe&tn=ntrKbsW_7ChXu3v-&_nc_ht=scontent-mxp1-2.xx&oh=6795f6e69b92df555fe092f3fe88920b&oe=60D6F38A

     pic https://scontent-mxp1-2.xx.fbcdn.net/v/t1.6435-9/202882689_4114871748598909_2348992891996201374_n.jpg?_nc_cat=1&ccb=1-3&_nc_sid=730e14&_nc_ohc=pt9a27iUPKkAX-nDSkS&_nc_ht=scontent-mxp1-2.xx&oh=9d7c864faa5725ca616bda7f75c2fb34&oe=60D5D444

		 pic https://scontent-mxp1-2.xx.fbcdn.net/v/t1.6435-9/203063487_4114871848598899_2808907362429003216_n.jpg?_nc_cat=1&ccb=1-3&_nc_sid=730e14&_nc_ohc=nNpRseqNKPoAX8eyP4p&_nc_ht=scontent-mxp1-2.xx&oh=ff243525853bf2a9c1abf5155adad004&oe=60D7A16E

  tab_end
\fi

Другим человеком, предупредившим о вторжении, оказался немецкий коммунист,
обер-ефрейтор вермахта Альфред Лисков. 21 июня 1941 года в 21.00 он переплыл
Буг и сдался бойцам 90-го пограничного отряда, заявив - в 4 часа утра
нацистская Германия нападёт на СССР. О появлении перебежчика доложили по
спецсвязи в Киев, начали допрос. Немцу откровенно не верили, но пришлось
поверить - когда под утро начался обстрел границы. Лискова переправили дальше в
СССР, он выезжал на фронт, агитируя немецких солдат сдаваться. В январе 1942
года советские власти его арестовали (по другим данным - направили на
психиатрическое лечение) из-за трений с руководством Коминтерна, включая
Георгия Димитрова. В июле 1942 года Лисков был освобожден и направлен в
Новосибирск, но после его следы теряются: дальнейшая судьба перебежчика неясна.

Третьим перебежчиком, как предполагают, была женщина - машинистка немецкой
комендатуры в Тересполе, по национальности полька. Она тоже переплыла реку (по
некоторым данным, аж 20 июня 1941 года), и на ломаном русском сообщила
советским пограничникам о готовящемся вторжении. Но вот насчёт неё мне
рассказали только устно, никаких документов о подвиге этой женщины в архивах
нет. 

Могло ли 22 июня 1941 года сложиться по-другому? Сложно сказать. Германия
сосредоточила против СССР отборные части с лучшей техникой - так или иначе, в
первое время пришлось бы отступать. «Как профессиональный военный я считаю:
поступок Бодзинского не мог изменить ход войны, - говорит бывший директор
мемориального комплекса «Брестская крепость-герой» генерал-майор Валерий
Губаренко. - Однако этот человек явно заслуживает награды за личное мужество». 

Если бы перебежчикам поверили везде, а не только на второй заставе Брестского
погранотряда, возможно, удар немцев оказался бы не столь внезапным. Жаль, что
этих людей не услышали... Спасибо им за то, что они пытались сделать.

* На фото - реакция советских людей на начало войны 22 июня 1941 года, мельник Бодзинский и обер-ефрейтор Лисков.

(с) Zотов
