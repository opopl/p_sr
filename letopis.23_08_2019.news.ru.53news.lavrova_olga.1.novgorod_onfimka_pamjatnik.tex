% vim: keymap=russian-jcukenwin
%%beginhead 
 
%%file 23_08_2019.news.ru.53news.lavrova_olga.1.novgorod_onfimka_pamjatnik
%%parent 23_08_2019
 
%%url https://53news.ru/novosti/50216-na-novgorodskoj-ulitse-velikoj-poyavilsya-zamechatelnyj-pamyatnik-zemlyaki-beregite-nashego-onfimku.html
 
%%author Лаврова, Ольга
%%author_id lavrova_olga
%%author_url 
 
%%tags novgorod,russia,beresta,onfim,pamjatnik
%%title На новгородской улице Великой появился замечательный памятник. Земляки, берегите нашего Онфимку!
 
%%endhead 
 
\subsection{На новгородской улице Великой появился замечательный памятник. Земляки, берегите нашего Онфимку!}
\label{sec:23_08_2019.news.ru.53news.lavrova_olga.1.novgorod_onfimka_pamjatnik}
\Purl{https://53news.ru/novosti/50216-na-novgorodskoj-ulitse-velikoj-poyavilsya-zamechatelnyj-pamyatnik-zemlyaki-beregite-nashego-onfimku.html}
\ifcmt
	author_begin
   author_id lavrova_olga
	author_end
\fi

\index[cities.rus]{Новгород Великий!Россия!Памятник Онфимке, 23.08.2019}
\index[rus]{Русь!История!Великий Новгород, Памятник Онфимке, 23.08.2019}

\ifcmt
pic https://53news.ru/images/wsscontent/articles/2019/08/na-novgorodskoj-ulitse-velikoj-poyavilsya-zamechatelnyj-pamyatnik-zemlyaki-beregite-nashego-onfimku.jpg
caption Памятник Онфиму, Новгород Великий
width 1.0
\fi

\textbf{Сегодня в полдень в Великом Новгороде произошло очень красивое и важное событие
– на месте находки первой берестяной грамоты появился необыкновенный, добрый
памятник.} 

«Онфимка», «Онфимчик» - так ласково называет свое творение скульптор Сергей
Гаев. С такой же теплотой говорят о новом памятнике новгородцы.

Мы обязательно расскажем об этом событии подробнее, но сейчас хочется, в первую
очередь, поделится с вами словами, которые произнес археолог \textbf{Пётр Гайдуков},
\footnote{заместитель директора Института археологии РАН, начальник Новгородской археологической экспедиции Института археологии.}
потому что он говорил очень важные вещи. 

\ifcmt
tab_begin cols=2
  width 0.45

  pic https://53news.ru/images/images/2019/avgust/23/800.jpg
  pic https://53news.ru/images/images/2019/avgust/23/IMG_0140.jpg
tab_end
\fi

Он напомнил о том, что вскоре после находки первой берестяной грамоты
руководитель Новгородской археологической экспедиции \textbf{Артемий Владимирович
Арциховский} написал своему другу, известному советскому археологу \textbf{Сергею
Владимировичу Киселеву}: «Надеюсь, что такие находки продолжатся, и что для
Новгорода береста будет иметь такое же значение, как для эллинистического
Египта папирус. Древние новгородцы заговорят с нами сотнями живых человеческих
голосов». 

«Артемий Владимирович не ошибся в своих ожиданиях, - сказал Пётр Григорьевич. -
К сегодняшнему дню в Новгороде найдено 1122 берестяные грамоты, а изучение этих
текстов превратилось в самостоятельную, очень сложную научную дисциплину.

Через пять лет после описанных мной событий, 63 года назад, были найдены
грамоты мальчика Онфима. Это произошло вот здесь, где мы стоим, в нескольких
метрах от этого мальчика, воплощенного в металле. Это были двенадцать обрывков
бересты и четыре рисунка, это были донца туесков, которые мама, видимо, давала
этому мальчику, чтобы он упражнялся в написании азбуки, в написании первых
слогов… Когда он уставал, он рисовал (всадник на коне и написано «Онфиме», то
есть он). Здесь же он писал «Поклон от Онфима к Даниле»…

Это были такие ученические заметки, листочки из школьной тетради. Изученные
Артемием Владимировичем Арциховским и воспетые Владимиром Лаврентьевич Яниным в
его замечательной книге «Я послал тебе бересту», они превратили Онфима в очень
выдающуюся личность. Он разошелся по десяткам, сотням публикаций, по тысячам
учебников. Его рисунки можно встретить во многих учебниках стран мира, не
только России. И Онфим, этот  неведомый мальчик превратился в персонажа русской
истории подобного Ярославу Мудрому, Марфе Посаднице, Ивану Грозному… Он занял
свое место в истории благодаря тому, что эти невзрачные листочки мама не
сожгла, может быть, вымела вместе с мусором куда-то на задворки усадьбы. Они
затоптались в грязь и сохранились.

Конечно, подобных рисунков были сотни.

Что я хочу этим сказать? То, что берестяные грамоты очень уязвимы. Вот эта
грамота в руках мальчика – ее трудно сломать, трудно разорвать, она из металла,
а эти листочки бересты, пережившие века под нашими ногами, они очень хрупки и
уязвимы. Их легко разрушить, легко не заметить, их легко потерять и по
неопытности уничтожить.

\ifcmt
  pic https://53news.ru/images/images/2019/avgust/23/Gaidukov.jpg
  width 0.4
  fig_env wrapfigure
\fi

Но еще трагичней то, что они могут находиться в культурном слое, который в
Великом Новгороде иногда нарушается, уничтожается строительными и другими
работами. Вот те триста гектаров русской земли, которые заключены в валах
Окольного города XIV века, средневековый Новгород с его культурным слоем – это
некое чудо света.

Но мы должны также понимать, что примерно около половины этого чуда света уже
не существует. Люди всегда здесь жили. C XVIII века здесь строили здания с
глубокими подвалами. И часть этого слоя неизбежно уничтожалась. Значительная
часть культурного слоя была уничтожена после войны, и это было оправдано –
город лежал в руинах. Было не до археологии, строились дома… Очень много домов
повредили культурный слой.

Но с тех пор прошло несколько десятилетий, и есть законодательство, которое
сохраняет культурный слой, поэтому, открывая этот замечательный памятник, мы
должны помнить о том, что у нас под ногами. Мы должны беречь это наше
национальное достояние… Этот мальчик всегда будет напоминать о том, что здесь,
в этом квартале, фактически начиналась современная научная археология Великого
Новгорода, которая дала огромный толчок к развитию изучению древнерусских
городов у нас в стране».

Добавим, что беречь всё же надо и металлическую грамоту в руках Онфимки, и
самого Онфимку, и остальные части памятника. Хоть Пётр Григорьевич и сказал,
что их не сломаешь... И речь сейчас не о вандалах, а о самых мирных гражданах,
которые любят придумывать всякие приметы и натирать ту или иную часть
скульптуры на счастье. Специально для них Сергей Гаев предусмотрел специальную
деталь – натирайте, сколько угодно! Это кошель:

\begin{center}
\ifcmt
ig https://53news.ru/images/images/2019/avgust/23/IMG_0139.jpg
width 0.3
\fi
\end{center}

Предалагем вашему вниманию фоторепортаж с открытия памятника. 

\ifcmt
tab_begin cols=3
	width 0.3

pic https://53news.ru/images/wsscontent/gallery/deb5cd73ea8d8dc980351791def49c4a.jpg
pic https://53news.ru/images/wsscontent/gallery/fe8bf96baf5fb75e87b691c08408ab87.jpg
pic https://53news.ru/images/wsscontent/gallery/e1756a15b9faa1711d7166e3c7656576.jpg

pic https://53news.ru/images/wsscontent/gallery/6daedd3cda07a70be6dee58fc28eca60.jpg
pic https://53news.ru/images/wsscontent/gallery/57e2e62c0d8f1f7a8bdb12657d4ab6ee.jpg
pic https://53news.ru/images/wsscontent/gallery/8a288238e873a1670c4b81f878c9d83c.jpg

pic https://53news.ru/images/wsscontent/gallery/55480296e56b30bbbde56607573d43d9.jpg
pic https://53news.ru/images/wsscontent/gallery/97033211ce5ea7c6c25196ac00579498.jpg
pic https://53news.ru/images/wsscontent/gallery/5839d0a9d5852376d2f543fe066a19ba.jpg

pic https://53news.ru/images/wsscontent/gallery/54f22b3cd58d36fba406a431e7115413.jpg
pic https://53news.ru/images/wsscontent/gallery/99f5ee2e2a5a4e89bec207d4919b3748.jpg
pic https://53news.ru/images/wsscontent/gallery/7e574425103b0ddc36d4172cb2c73f07.jpg

tab_end
\fi


\begin{itemize}
\iusr{пурген}
24.08.2019
00:48

В городе нет другого памятника, который бы так мощно передавал его дух.
Имя этого мальчика звучало из уст учителя так:

Ανθέμιος

А когда он подрастет, он станет скорее всего служить в храме Спаса на Ильине
улице, подобно своему святому покровителю "Во священницех благочестно пожив", в
учении же книжном наверняка будет с особым тщанием изучать точные науки и
усердствовать в освоении Софии Премудрости Божией чтобы превзойти в рукомесле
своего тезку - Анфимия из Тралл, с которым в Премудрости сравниться разве что
сам Соломон.

А сейчас он освоил уже буквы и может написать не только свое святое имя, но и
текст молитвы.

Мне он тоже таким и представлялся.

\iusr{У.с.}
23.08.2019
15:37

Теперь мэрии надо бы позаботиться об окружающем Онфимку пространстве. Так, на
фасаде дома хотелось бы видеть историческое граффити. Ну, а
газон, естественно, нуждается в хорошем ландшафтном дизайне.

\end{itemize}
