% vim: keymap=russian-jcukenwin
%%beginhead 
 
%%file 17_01_2022.stz.news.ua.radiosvoboda.1.ukraina_rik_foto.pic.1
%%parent 17_01_2022.stz.news.ua.radiosvoboda.1.ukraina_rik_foto
 
%%url 
 
%%author_id 
%%date 
 
%%tags 
%%title 
 
%%endhead 


\ifcmt
  tab_begin cols=2,no_fig,center
     @captionsetup font=Huge

     pic https://gdb.rferl.org/084a0000-0aff-0242-3d86-08d9d90e8d8c_w1023_s.jpeg
     @caption_begin
14 січня. У Чернівецькій області, в мальовничому селищі Красноїльськ, що
неподалік кордону з Румунією, на старий Новий рік вже століттями
відбувається одне з найяскравіших карнавальних дійств в Україні –
святкування Маланки. Більше фото та відео зі свята можна побачити 
\href{https://www.radiosvoboda.org/a/31048285.html}{тут} 
     @caption_end

     pic https://gdb.rferl.org/084a0000-0aff-0242-943e-08d9d90e8bde_w1023_s.jpeg
     @caption_begin
18 лютого. Вшанування пам'яті
\href{https://www.radiosvoboda.org/a/photo-nebesna-sotnya-vshanuvannya-kyiv/31109687.html}{Небесної
сотні} в Києві. Більше про розстріли на Майдані – у \href{https://www.radiosvoboda.org/a/dbr-i-henprokuror-pro-zavershennya-rozsliduvannya-rozstriliv-na-maydani/31564810.html}{цій статті}
     @caption_end

  tab_end
\fi
