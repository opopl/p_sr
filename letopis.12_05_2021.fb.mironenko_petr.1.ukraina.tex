% vim: keymap=russian-jcukenwin
%%beginhead 
 
%%file 12_05_2021.fb.mironenko_petr.1.ukraina
%%parent 12_05_2021
 
%%url https://www.facebook.com/petro.myronenko/posts/3128231570737816
 
%%author 
%%author_id 
%%author_url 
 
%%tags 
%%title 
 
%%endhead 
\subsection{Опозиція до Держави}
\Purl{https://www.facebook.com/petro.myronenko/posts/3128231570737816}

Поки ми \enquote{боремося} з російським агресором, в Україні виникла і вільно діє
опозиція до держави, що є суто українським феноменом.

Важко собі уявити іншу країну, де діють політичні сили проти власної держави.

Опозицію до держави треба ліквідувати, бо це зрадники України.

В пострадянській Україні, яка ще не стала правовою, демократичною, соціальною
державою, Конституція є декларативним документом, а політична "еліта" діє у
власних корпоративних інтересах на фоні відсутності класичних політичних
партій, що замінені політичними бізнес-клубами.

Іншими словами, Україна знаходиться у затяжному стані переходного періоду від
тоталітаризму до демократії.

Що робити?

Це питання стоїть на порядку денному вже 30 років.

Відповім словами мого друга японця Акіри Ідеміцу.

"Вам треба почати будувати свою державу на кращому досвіду передових країн
світу з урахуванням ваших традицій і ментальності. У постійному процесі
реформування ви втрачаєте і час, і державу. Будуйте державу і громадянське
проукраїнське суспільство, бо інакше за вас це будуть робити росіяни".

Від себе додам: кругле колесо вже давно котиться по Землі, а ми все намагаємось
зробити крадратне, а тому займаємось вічним процесом реформування того, що вже
давно вмерло.

А починати все треба з молодого покоління. Науку і освіту, де мох і нафталіт є
основними складовими, треба будувати на чистому полі, а старе відгородити
високим парканом.

Будуймо Україну в собі і навколо себе.

З повагою П.В. Мироненко
