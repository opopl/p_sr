% vim: keymap=russian-jcukenwin
%%beginhead 
 
%%file 12_02_2022.stz.news.ua.strana.1.vojna_nachnetsja_15_fevralja.5.two
%%parent 12_02_2022.stz.news.ua.strana.1.vojna_nachnetsja_15_fevralja
 
%%url 
 
%%author_id 
%%date 
 
%%tags 
%%title 
 
%%endhead 

\subsubsection{2. Подготовка к реальной войне}

Другой вариант - негативный. Он подразумевает, что проводится информационная
\enquote{артподготовка} к реальным военным действиям на территории Украины, в которые
будет вовлечена Россия. 

Далее США говорят \enquote{мы были правы насчет вторжения}, и включается тот или иной
режим санкций.

Санкции наверняка будут серьезными и вызовут большой раскол между Москвой и
Евросоюзом, что в целом отвечает геополитическим интересам США и Британии. Да и
экономическим тоже. В частности, может быть надолго похоронен \enquote{Северный
поток-2}, а это усилит зависимость Европы от сжиженного газа из Америки и
монархий Ближнего Востока, союзных англо-американцам. 

Вне зависимости от того, какую часть Украины охватят военные действия, Россия
понесет потери - и людские, и экономические. То есть ее в данном случае ждет
\enquote{двойной удар} от санкций и затрат, связанных с воной в Украине. 

Впрочем, нельзя исключать, что этот вариант США развивают пока только в
поддержку первого сценария - как метод заставить Москву быть более уступчивой
на ведущихся переговорах. Например, как мессидж России, что если она не примет
условия американцев (например, пойти на уступки Киеву по Минским соглашениям в
плане очередности выполнения их норм либо что-то еще более глобальное,
например, в плане отношений с Китаем), то ее ждет война, изоляция и удары по
экономике.

Но есть вопросы - а как, собственно, эта война начнется? И кто именно ее
начнет?

\textbf{Россия}? 

По причинам описанным выше, война в любой форме ей не выгодна. А главное - на
Украине не происходит никаких процессов, которые бы несли смертельную угрозу
для России (пробуксовка Минских соглашений и санкции из-за продолжения
конфликта на Донбассе к смертельным угрозам явно не относятся, так как РФ живет
с ними уже много лет). И потому крайне маловероятно, что Москва решится на
войну по собственной инициативе.

Более того, что важно: в информационном плане никакой подготовки к началу
масштабной войны с Украиной в России не проводится. Наоборот - каждый день идут
заверения официальных лиц, что планов нападения нет. Чтоб в таких условиях
вдруг резко \enquote{поменять концепцию} и отправить страну на большую войну, чреватую
большими жертвами нужен какой-то очень сильный повод, который даст Украина.

\textbf{Украина}? 

То, что Киев даст такой повод также выглядит не очень реальным.

Хоть в \enquote{ЛДНР} и в России и обвиняют практически каждый Украину в подготовке
наступления на Донбассе, но пока все говорит о нежелании Киева провоцировать
Москву на прямой военный ответ, который чреват коллапсом всей украинской
государственности (как неоднократно угрожали российские представители) и
катастрофическими последствиями лично для представителей украинской власти.

Одно дело воевать с вооруженными формированиями \enquote{ЛДНР} на Донбассе, от которых
Москва официально открещивается. Другое - в открытую с российской армией и всем
ее потенциалом.

Есть, правда, теория, что Запад может принудить украинские власти
спровоцировать конфликт. Однако риторика Зеленского и Ко по отрицанию угрозы
вторжения, показывает, что Киев на такой вариант не готов идти. 

Тем более, что и с Западом все не просто.

\textbf{Запад}? 

Тут одна из главных проблем - неготовность Евросоюза отказываться от российских
энергоносителей и опасения ценового коллапса на рынках газа и нефти. Слишком
большую долю там занимает Россия. Поэтому из ЕС пока идут противоречивые
сигналы о \enquote{вторжении} и санкциях за него. 

Так, например, немцы до сих пор официально не заявили, что остановят \enquote{Северный
поток-2} в случае атаки России на Украину. И если после \enquote{вторжения} и захвата,
к примеру, Киева, Европа продолжит покупать у Москвы газ - для США не будет
особого смысла вообще все это начинать. Кроме того, непонятно, насколько сам
Запад вообще готов жертвовать Украиной.

Если ограниченный конфликт на Донбассе выйдет из-под контроля, то американцы
влияние на всю страну или ее значительную часть скорее всего утратят.

А это будет их крупным геополитическим проигрышем. Что не замедлят использовать
оппоненты Байдена, которые начнут обвинять его в поражении Путину и \enquote{втором
Афганистане}. Вряд ли это нужно демократам перед выборами в Конгресс, которые
состоятся осенью. 

Россия же как раз своими действиями постоянно дает понять, что Донбассом война,
если она вообще начнется с подачи Украины, не ограничится. Видимо, цель у этих
заявлений \enquote{охладить пыл} тех, кто рассчитывает на локальный конфликт -
\enquote{незначительное вторжение} в терминологии Байдена - за которое можно
будет ввести \enquote{значительные} санкции. 

Кроме того, санкционный \enquote{занавес} между Россией и Западом объективно
усилит Китай, спаяв Москву и Пекин в более тесный континентальный союз,
которому будет сложнее противостоять. И это станет огромной геополитической
проблемой для США. 

Поэтому пока что \enquote{жесткий} военный сценарий по Украине менее вероятен,
чем первый упомянутый нами вариант - нагнетание угрозы войны, чтобы потом
героически ее снять, подписав мировую с Россией. 
