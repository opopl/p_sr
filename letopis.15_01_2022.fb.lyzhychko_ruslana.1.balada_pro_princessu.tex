% vim: keymap=russian-jcukenwin
%%beginhead 
 
%%file 15_01_2022.fb.lyzhychko_ruslana.1.balada_pro_princessu
%%parent 15_01_2022
 
%%url https://www.facebook.com/ruslana.lyzhychko.5/posts/4895630353832469
 
%%author_id lyzhychko_ruslana
%%date 
 
%%tags kultura,muzyka,ukraina
%%title Балада про Принцесу
 
%%endhead 
 
\subsection{Балада про Принцесу}
\label{sec:15_01_2022.fb.lyzhychko_ruslana.1.balada_pro_princessu}
 
\Purl{https://www.facebook.com/ruslana.lyzhychko.5/posts/4895630353832469}
\ifcmt
 author_begin
   author_id lyzhychko_ruslana
 author_end
\fi

Сьогодні «Баладу про Принцесу» розповідають зі мною діти, які дізналися про цю
пісню від своїх батьків. Тобто це вже через покоління справжній мультфільм,
озвучують майбутні українські артисти, мрійники, що хочуть стати зірками.
@igg{fbicon.star}

\ii{15_01_2022.fb.lyzhychko_ruslana.1.balada_pro_princessu.pic.1}

Історія «Балади про Принцесу» - це для них і про них. Вони взяли участь у
проекті «Співаю із зірками» і стали в ньому найкращими. 

Перша моя зустріч з ними відбулася влітку і подарувала мені ще одну мрію. Мрію
– знайти якомога більше таких друзів.

\ii{15_01_2022.fb.lyzhychko_ruslana.1.balada_pro_princessu.pic.2}

Дякую всім, хто допоміг мені ще раз відчути в моєму серці віру і безмежну любов
до казки. @igg{fbicon.heart.red}

Я ж пропоную відчути цю казку і вам, переглянувши наше відео - Балада про
Принцесу. 25 років потому ця пісня зазвучала голосами юності сьогоднішньої
України. 

(Відео в попередньому дописі або за посиланням в першому коментарі)
