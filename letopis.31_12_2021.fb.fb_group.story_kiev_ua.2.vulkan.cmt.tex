% vim: keymap=russian-jcukenwin
%%beginhead 
 
%%file 31_12_2021.fb.fb_group.story_kiev_ua.2.vulkan.cmt
%%parent 31_12_2021.fb.fb_group.story_kiev_ua.2.vulkan
 
%%url 
 
%%author_id 
%%date 
 
%%tags 
%%title 
 
%%endhead 
\zzSecCmt

\begin{itemize} % {
\iusr{Vadim Vadim}
Креативный парень.))

\iusr{Олена Шелест}

Благодарю. Я вспомнила, как посещала химический кружок в школе. Мы,
старшеклассники, показывали малышам всякие фокусы, вулкан в том числе,
рассказывали популярно о природных явлениях. Было интересно, весело. Вспоминаю
свою учительницу по химии и эти опыты (иногда растворившиеся от кислоты
колготки) с любовью и теплотой.


\iusr{Ольга Хоботня}
Так это был гениальный ход!

\iusr{Ксения Гвоздева}
Сопру

\iusr{Vyacheslav Sliva}

\ifcmt
  ig https://scontent-frt3-2.xx.fbcdn.net/v/t39.30808-6/270165832_3234864073453335_4087879948887277685_n.jpg?_nc_cat=101&ccb=1-5&_nc_sid=dbeb18&_nc_ohc=_6JncXBAWFEAX-HQWYP&_nc_ht=scontent-frt3-2.xx&oh=00_AT9VmEAGPFTUxzcrCVU5n5YW8ZjYG0CnlMlwhIblWfsDDA&oe=61D762DE
  @width 0.3
\fi


\iusr{Владимир Баваровский}
А мне бы такой вулкан понравился!  @igg{fbicon.face.grinning.smiling.eyes} 

\iusr{Виктор Задворнов}

Да, химия - наука чудес, кузница народного богатства. Пейте херес! В нем
русское \enquote{нет} и английское \enquote{да}. Сплошное противоречие

\iusr{Ирина Ящук-Лантушенко}

Не так, як у всіх, тому поведінка 3. Всі повинні були бути однаково вдягнені і
з однаковими зачісками. Креатив не заохочувався.


\iusr{Helen Shkabar}
Спілберг відпочиває

\iusr{Дмитрий Крыськов}

Хорошо излагаешь Братан. Вспоминается мне не только Женя баловался подобными
экспериментами. За что регулярно получал от родителей и мне за компанию
прилетало

\begin{itemize} % {
\iusr{Сергій Криськов}

Да, отрочество и пиромания, зачастую шагают рядом... Меня тоже сие не минуло тогда.
\end{itemize} % }

\iusr{Ирина Бондаренко}
До сих пор помню на уроке химии такой вулкан нам показывали! Как давно...

\iusr{Виктор Марченко}

Марганцовокислый калий плюс чистый глицерин. Выделяется атомарный кислород и
происходит возгорание. @igg{fbicon.face.wink.tongue} 

\begin{itemize} % {
\iusr{Сергій Криськов}
\textbf{Виктор Марченко} Да, такое тоже делали в школе.
Но данный \enquote{вулкан} был приведён в действие электрическим запалом.
\end{itemize} % }

\iusr{Kristina Vlasova}
Настоящее искусство! Круть)

\iusr{Oleksa Vovk}
Ми таке робили. Селітра змішана з цукром.

\begin{itemize} % {
\iusr{Сергій Криськов}
\textbf{Oleksa Vovk} Мы тоже. А ещё лучше - селитра, сплавленная с сахаром, называемая \enquote{карамелью}.

\iusr{Oleksa Vovk}
\textbf{Сергій Криськов} То ваще.
\end{itemize} % }

\iusr{Антонина Сорочинская}

А дома какие эксперименты проводили и получали \enquote{выговор} от соседей и
родителей!

\iusr{Алина Би}
Это Евгений Хачериди коренной киевлянин?

\iusr{Сергій Криськов}

Не знаю, кто такой Хачериди.

\iusr{Volodymyr Nekrasov}

Мені колись старша сестра подарувала набор «Юний Хімік». На якийсь час я був
повністю занурений у дослідження хімічних процесів. Потерпали від того всі
рідні. Завершальним акордом став дослід з самозапалюючими газами. Карбіт і
марганцовка виділили стільки газу, що на кухні запалало все навколо. А самі
реактиви вивергнулись мов гейзери і розмалювали щойно зроблений ремонт на стелі
і стінах у стилі абстракціонізму. На щастя вогонь зник так само швидко як і
з‘явився, спаливши хіба що мої вії. Тоді я зрозумів що хіміком мені стати не
судилось.

\end{itemize} % }
