% vim: keymap=russian-jcukenwin
%%beginhead 
 
%%file 20_01_2022.fb.uljanov_anatolij.1.psiho_terror.cmt
%%parent 20_01_2022.fb.uljanov_anatolij.1.psiho_terror
 
%%url 
 
%%author_id 
%%date 
 
%%tags 
%%title 
 
%%endhead 
\zzSecCmt

\begin{itemize} % {
\iusr{Юрий Франк}

Чувак, ты извини, конечно, но очень легко \enquote{не паниковать}, сидя в США. Не на
твою голову потенциально может прилететь. Особенно легко \enquote{не паниковать} тем,
кому уже прилетало и кто были вынуждены бежать из донецкой и луганской области.

\begin{itemize} % {
\iusr{Анатолий Ульянов}
\textbf{Yuri Frank} 

Это понятно, но панику нагнетают. Не паниковать из США действительно легче.
Именно поэтому Америка не переживает, разгоняя волну про \enquote{вот-вот, с минуты на
минуту}. Это жестокая хуйня.

\iusr{Юрий Франк}
\textbf{Anatoli Ulyanov} 

нагнетают, да. это само собой и в этом нет ничего нового. проблема в том, что
даже если абстрагироваться от нагнетания и того, чо там несёт дедушка Байден,
мы все равно, исходя из собственного печального опыта, отлично знаем, что от РФ
можно в любой момент ожидать говна любой степени невменяемости. особенно, когда
она в очередной раз начинает размахивать ружжом поблизости от украинских
границ. если бы мне в 2012 сказали, что будет конфликт с РФ, аннексия Крыма и
кровавая баня на Донбассе, я бы повертел пальцем у виска и послал этого
человека полечить паранойю. что было дальше мы все знаем. проблема в этом.
можно сколько угодно критиковать хуйню, которую несут и западные, и наши СМИ и
политики, но если будет очередная эскалация, провокация, похуй как называть, в
общем — если в нас опять начнут стрелять и нас бомбить, то будут трупы мирных
жителей, и разрушенные дома, и беженцы, и страшное горе. и мы это знаем
независимо от того, кто там что нагнетает. это уже было, и это более чем
реальные основания для страха. особенно для людей, у которых нет возможности
отсюда бежать. такая хуйня.

\iusr{Алексей Баскаков}
\textbf{Юрий Франк} 

вы один из немногих украинцев, кто понимает опасность ситуации. Большинство
настроены очень воинственно, но при этом ни разу не допускают, что ракеты прямо
завтра полетят к ним в Киев и Львов. Путин предлагает весьма умеренную плату за
мир: Минские соглашения и нейтральный статус. Завтра цена будет выше.

\end{itemize} % }

\iusr{Ilko Gladshtein}
Толик, позволю себе один отвлеченный вопрос: кто сбил малазийский боинг по твоему мнению?

\begin{itemize} % {
\iusr{Анатолий Ульянов}
\textbf{Ilko Gladshtein} 

Допускаю ли я, что его сбил российский \enquote{Бук}? Да, допускаю. С выводами доклада
JIT знаком. Склоняюсь к тому, что эта трагедия – collateral damage, но могу
только спекулировать на этот счёт.

\iusr{Ilko Gladshtein}
\textbf{Anatoli Ulyanov} а что там делал российский бук?

\iusr{Анатолий Ульянов}
\textbf{Ilko Gladshtein} 

господи, ну ты как маленький – ты не знаешь, что делает Россия в Украине? То
же, что делает в Украине Америка. Или Америка в Ираке. Или Россия в Сирии.
Защищает свои интересы. Или ты думаешь если завтра какая-нибудь Мексика
додумается объявить о намерении вступить в военный блок с Китаем, к ней
американский \enquote{Бук} не приедет? Get real! This is what imperialism is all about:
большие ребза делят мир на выдой, не спрашивая у украин, можно ли им \enquote{Буки}
свои выкатить или нет. Никто в реальном мире не пускает к своим границам
конкурентов. И уж тем более не интересуется мнением слабых стран-должников,
которые принадлежат тем, кто им кредиты даёт.

\iusr{Ilko Gladshtein}
\textbf{Anatoli Ulyanov}, 

российский бук на территории Украины, управлялся российскими солдатами, которые
приехали окупировать часть наших территорий. Они сделали это один раз и могут
сделать это еще раз - потому, истерика вполне оправдана. (но мне нравится, как
ты оправдываешь агрессора)


\iusr{Константин Станиславский}
\textbf{Ilko Gladshtein} 

ну хватит кривляться то. Как будто не понятно о чем речь. Когда то американцы
поставили ракеты в Турции, а Советы отправили свои на Кубу. Чуть апокалипсис не
сделали. И наши и янки ракеты убрали. Забавно что в американских учебниках
объявляется о чистой победе Проекции Рая на Земле над bolshevik. О том что им
тоже пришлось убрать свои ракетки из Турции - ни слова.

Забавляют визги упоротых на УкрТВ - Россия не может решать быть нам в НАТО или
нет! И это говорят те же люди которые считают что США очень правильно делают
запрещая СП-2. Хотя казалось бы какое им дело? Два государства заключили
договор о поставке газа, а третье государство, расположенное за многие тысячи
километров от них говорит нет, не нужен газопровод. И всячески ему
препятствует.

\iusr{Ilko Gladshtein}
\textbf{Константин Станиславский} вы сильны в геополитике!

\iusr{Константин Станиславский}
\textbf{Ilko Gladshtein} 

нет, я излагаю вам элементарные факты, от того что вы зоологический русофоб,
вам кажется что нагибание Америкой всей планеты это хорошо, а попытка дать
ответ свидетельствует о патологической преступности России.


\iusr{Ilko Gladshtein}

Да мне насрать на Россию. Пускай дадут нам жить

\iusr{Анатолий Ульянов}
\textbf{Ilko Gladshtein} 

Оправдываю? Я просто понимаю какие-то элементарные факты реальности, и то, как
эта реальность работает.

У каждой стороны есть свои интересы, которые она пытается реализовать в меру
своих сил и возможностей. Любая империя развивается путём расширения.
Расширение дает ей ресурсы и открывает новые возможности. Не расширяющихся
империй не бывает. Это всё азбучные истины. Не понимая элементарной
политической логики и экономики действий, остается только наивно ждать, что
кто-то кому-то даст жить, и позволит это делать в сфере своих интересов как ему
вздумается. Но так не бывает. Ни Россия, ни Америка не позволит украинам
свободно плавать. Такое право не дают, а берут, но для того, чтобы взять, надо
быть умной и сильной страной, а не нищим сырьевым придатком с большими
мечтательными глазами.

Выход есть. Всегда. Но для начала нужно понять, как устроена игра, чтобы либо
её изменить, либо выкусить себе кое-что получше в рамках её правил.

\end{itemize} % }


\iusr{Ilko Gladshtein}
так я не понял, кто готовит к провокации?

\begin{itemize} % {
\iusr{Анатолий Ульянов}
\textbf{Ilko Gladshtein} 

те, кто гонит эту волну, и из всех розеток запугивает граждан. это же классика
политики ястребов. вся биография блинкена об этом.

я могу понять, зачем это америке, но не вижу выгоды в большой военной кампании
для россии, которой выгоднее спокойно запустить северный поток-2, и держать
соседей в амбивалентном напряжении, снизить которое может только гарантия
внеблокового статуса, который, с моей точки зрения, является безальтернативным
путём для украины.

как показывает пример Казахстана, если надо, Россия оказывается где ей надо в
течении суток, и времени порассуждать \enquote{нападёт или нет} не оставляет.
надо будет – нападет. но по моменту это \enquote{надо} не очевидно. по крайней
мере для меня. что там из кабинетов видно, и какие карты на столе лежат –
другой вопрос.

смысл нынешней информационной кампании в том, что бы мы обсуждали, сомневались
и нервничали. к этому нужно относиться как к нарративной атаке, и не вестись,
безотносительно того, кто и в ком видит сволочь.

\iusr{Ilko Gladshtein}
\textbf{Anatoli Ulyanov} 

опуская момент с тонной видео перемещающейся к нашим границам российской
военной техники, и применив твою теорию злокозненных сша, раздувающих истерию,
есть более дешевые методы ведения нарративных атак, чем десятками бортов
перекидывать сюда оборонительное вооружение.

\iusr{Владимир Могилевский}
\enquote{оборонительное}

\iusr{Анатолий Ульянов}
\textbf{Ilko Gladshtein} 

Борты перемещаются по российской территории, и это как раз make sense ввиду той
риторики, которая звучит из вашингтона, а также в контексте обострившихся
отношений, и российских интересов относительно продвижения НАТО. Войска должны
быть под рукой. Их концентрация сообщает конкурентам красную линию, и то, что
последует, если она будет нарушена. Но сама по себе эта игра мускулами ещё не
даёт оснований говорить о вторжении с той помпой, с которой это делается.
Потому что вторжение предполагает не просто перемещение бортов, а
полномасштабную логистическую операцию, организацию тыловой машины снабжения,
разворачивание целого ряда технических и медицинских структур – и вот это уже
заметили бы все, потому что это втихую не сделать.

У меня нет иллюзий в отношении способности России к военной операции, которая
при любых раскладах ничего хорошего для Украины не означает, но, повторяю, – за
каждым действием имеется экономическая логика, это действие мотивирующая.
Выгоды устраивать кровавую кашу раньше, чем это нужно, я не вижу.

У нас ограниченная информация. Но мы можем фиксировать, что с нами делают – нас
последовательно запугивают, держа в состоянии неопределенности, и провоцируя
спекулятивные споры. Всё это за версту разит манипуляцией, и потому требует
бдительного спокойствия )

\iusr{Владимир Могилевский}
\textbf{Anatoli Ulyanov} 

Я предполагаю, что под \enquote{бортами} имеются недавние интенсивные поставки оружия
из Англии. Оборонительного, разумеется.

\iusr{Ilko Gladshtein}
\textbf{Anatoli Ulyanov} 

ты перепутал (или \enquote{перепутал}): я говорю о самолетах, которые нам везут
противотанковые установки из нескольких европейских стран. Это все очень
дорого. Нагнетать информационно можно значительно дешевле. Про право
перемещаться российским войскам по своей территории - я уже читал у Путина в
речи. Кстати, Крым - украинская территория, превращенная в российский военный
плацдарм. Там они этого права делать не имеют


\iusr{Alexander Belyaev}
\textbf{Ilko Gladshtein} 

\enquote{Это все очень дорого}. Погодите, так за чей счёт банкет? Вы думаете, США и UK
от сердца отрывают? Джавелины - были не помочью, а продажей, причём выше рынка.
Точно так же c британскими NLAW.

\iusr{Ilko Gladshtein}

Это не так важно в данном случае. Моя мысль: истерию нагнетать можно более
дешевыми способами. И, кстати, формирование 100+тысячной армии на границе тоже
очень, очень дорогое удовольствие

\iusr{Константин Станиславский}
А кто напал на Гренаду, Панаму и многая и многая, несть им числа?
Кто устраивал госперевороты в Чили, Иране и проч.?
Кто жег напалмом Вьетнам? Страна замечу расположенная за десятки тысяч километров от территории США. Применял там запрещенные фосфорные бомбы. Вы кстати в курсе что США одна из немногих стран отказавшихся присоединится к конвенции о полном запрете фосфорных бомб?
Вот если вы ответите на эти вопросы, сразу же вам станет ясно кому нужны провокации.
Россия конечно же не ангел. Но, с волками жить - по волчьи выть.

\iusr{Ilko Gladshtein}
\textbf{Константин Станиславский} дада, все правильно Россия делает, пускай вторгается в Украину, покажем этим американцам!

\iusr{Константин Станиславский}
\textbf{Ilko Gladshtein} 

не дождетесь, оккупацию Россией нужно еще заслужить

Учитывая состояние украинской инфраструктуры, затея эта абсолютно
бессмысленная. Вы знаете сколько Россия вложила только в Крым, после 23 лет
пребывания его в составе незалежной Украины?. Я там был в 2016-м, катастрофа
полная. Сейчас сильно лучше.

Все критически важные технологии мы уже освоили, импортозамещение сработало
неплохо. Зачем нам Украина? Кормить 30 миллионов людей, со своеобразным
менталитетом, такое себе удовольствие.

Судьба Украины стать аграрной сверхдержавой. А такие страны, по определению не
могут быть сильными и богатыми.

Так что и не мечтайте об оккупации, Путин - гений, он ваших Байденов с
Макронами на завтрак кушает, про зеленое недоразумение и не говорю.

\iusr{Ilko Gladshtein}
\textbf{Константин Станиславский} зачем кормить, можно же просто расхуячить, как Сирию

\iusr{Magrit Sibiryakov}
\textbf{Ilko Gladshtein} 

насколько я учил историю - вся холодная война состояла из перекидывания
вооружения. Ну типа - почему бы не повторить? Это работает. На что они так уж
потратились - на горючку? Ясно, что ни мне ни вам это не нравится.

\iusr{Magrit Sibiryakov}
\textbf{Ilko Gladshtein} 

ну в смысле не имеют и Украинская территория? Чьи там войска? Украинская
территория там где правительство украинское и войска. Я не понимаю зачем
самообманом заниматься. Крым не украинский. Был, возможно будет. Но сейчас -
нет. От того что я скажу - \enquote{Крым это Украина} - российские войска разплачутся и
домой уйдут в стыде или как?

\end{itemize} % }

\iusr{Илья Чучуев}
Напал на полшишечки это не считается

\iusr{Алексей Блюминов}
\textbf{Анатолий Ульянов} 

Все верно. Но я вижу в этой избыточной бомбардировке психотерором и плюс. Вне
зависимости от целей тех кто таким макаром отвлекает нас либо от закулисного
торга либо от подготовки к провокациям.

На самом деле вот эта долбежка вызывает обратный эффект. Эффект ребенка
перекормленного манной кашей. И чем интенсивнее долбежка тем сильнее
эмоциональное отгораживание от нее людей, для которых она становится не фактом
жизни, а скорее назойливым бубнящим фоном к содержнию которого давно перестали
прислушиваться.

Значит - пропаганда перестает работать. Брехунцы извергают потоки слов но эти
слова проходят сквозь сознание озабоченных своей реальной жизнью людей как вода
сквозь сито. Люди это все просто перестают как то рефлексировть и эмоционально
закрываются.

И индоктринации токсичной пропагандой не происходит.


\iusr{Вячеслав Жидков}
Дык, а кто нервничает? Большинство с нетерпением ждёт когда эту кодлу сбросят и вздёрнут.

\iusr{Zakharchenko Dmytro}
Эталонный

\iusr{Илья Медведев}

Анатолий, привет. Я тебя читаю с 2006, часто жестко критикую, но ведь и ты сам
много критики выдаешь... Короче говоря, я прошу тебя зайти на мою страницу и
внимательно изучить мой верхний пост и все мои же коменты под ним... потому
как, если выражаться языком героч Лео Дикаприо в недавнем кинохите, кое-кто из
власть имущих окончательно просрал свои мозги...

это ОЧЕНЬ ВАЖНО! ты уже один бустер вроде бы колол...


\iusr{Михайло Лебедь}

Великий военно-политический эксперт опять слово молвил. Значит, Путин не
нападал, а США готовит провокацию?  @igg{fbicon.smile} 

\iusr{Алексей Орлов}

Просто цифра дня. Телеграм канал «Рессенимент», ведомый украинским левым
еврейского происхождения, за сегодня выдал 22 сообщения посвящённых тому, как
Путин собирается напасть на Украину, как Великобритания готова поставить
Украине оружие, как США готовы поставить Украине оружие, как Чехия готова
поставить Украине оружие (ака. «накачка вооружением»... за деньги, между прочим,
не за бесплатно... при этом «воевать» на стороне Украины никто из них не
собирается)... Как Германия НЕ собирается поставлять Украине оружие (но там
поганые социалисты пришли к власти! УЪУ!! 1) Вообще, кроме закономерного
вопроса: «а откуда у мужика столько времени на вот это все», понимаешь, что
ЖИВЯ ВНУТРИ вот этого вот информационного пузыря, в котором ежесекундно, из
каждого буквально утюга на тебя вываливаются тысячи сообщений о готовящемся
нападении... трудно сохранить твердость ума и не поддаться всеобщей панике и
массовому гипнозу.

\iusr{Anton Gruba}

Ну так то ж он гибридно напал, а щяс негибридно.

\end{itemize} % }
