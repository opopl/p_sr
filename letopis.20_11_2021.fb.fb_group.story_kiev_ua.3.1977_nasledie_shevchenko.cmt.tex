% vim: keymap=russian-jcukenwin
%%beginhead 
 
%%file 20_11_2021.fb.fb_group.story_kiev_ua.3.1977_nasledie_shevchenko.cmt
%%parent 20_11_2021.fb.fb_group.story_kiev_ua.3.1977_nasledie_shevchenko
 
%%url 
 
%%author_id 
%%date 
 
%%tags 
%%title 
 
%%endhead 
\zzSecCmt

\begin{itemize} % {
\iusr{Клим Форманчук}

Не думаю, что отказался. \enquote{Там} умели склонять к сотрудничеству. Так, что
попросту не признался. Может и до сих пор \enquote{в работе}.

\iusr{Zoya Sharykova}

Десь майже у ті ж роки я була студенткою факультету журналістики КДУ. Нам
переповідали історію, як двох студентів за два роки до нас виключили з нашого
ідеологічного ф-ту (ну й, зрозуміло, університету) за те, що вони 22 травня
поклали квіти до пам'ятника Шевченку. От як цей маразм пояснити нинішній
молоді? А на нашому курсі через рік від нас вимагали проголосувати за
виключення з комсомолу однокурсника, який у Чернігові влітку під час практики
спілкувався українською мовою і так привітався з кимось із партфункціонерів.
Намагались пришити йому націоналізм... До честі мого курсу - ми не
проголосували! Довго нашому комсомольському активу це згадував викладацький
партактив... А ще на філфаці чи на романо-германському - теж на бульварі
Шевченка, 14 - у цей період навчалась Леся Маланчук. І навіть викладачі
вимовляли це ім'я з пересторогою. Були часи...

\begin{itemize} % {
\iusr{Надежда Владимир Федько}
\textbf{Zoya Sharykova} Про цей випадок у мене десь на компі є спогади одного мого товариша, який в той час вчився в КДУ. Обов'язково пошукаю.

\iusr{Тарас Єрмашов}
\textbf{Zoya Sharykova} 

Та нинішній молоді ще можна пояснити - на відміну від нинішніх-таки упоротих
совкофілів-рашистів, котрі за всієї наявної інформації нахабно розпатякують про
\enquote{счастливую жизнь в сєрєсєрє}...

\begin{itemize} % {
\iusr{Oleg Chorny}
\textbf{Тарас Єрмашов} 

Моя мама працювала викладачем на факультеті мехмату університету. Тоді
факультет знаходився в червоному корпусі напроти пам'ятника Шевченку.
Напередодні 22 травня викладачів завжди на кафедрі \enquote{інструктували}, що ні в
якому разі ніхто з них не міг з'являтися цього дня в парку, бо це
\enquote{націоналістичне} свято. Один раз хтось з її колег цього дня таки пройшов через
парк, бо спершу мав читати пари в червоному корпусі, а потім в жовтому. От він
і пішов в жовтий, вирішив зрізати дорогу. Викликали в перший відділ наступного
дня, писав доповідну чому в цей день був в парку. Зрештою аргументи щодо
розкладу в різних корпусах таки взяли до уваги і якихось наслідків не було. Але
сам факт...

\iusr{Оля Лемишко}
\textbf{Тарас Єрмашов} 

зате ковбаса і морозиво! Про це верещать, аж уші закладає. Я навчалася в КПІ в
середині 70, викладання, спілкування і література виключно російською. Одно
разу я заговорила українською, то побачили б ви вираз обличчя моїх
одногрупників! Ти, українською!!!

\iusr{Сергей Юрченко}

Ну навіщо ви весь сир в один вареник валите? Теж мені, дисиденти і жертви
репресій! Саме в той час моя мама навчалась на журнфаці КДУ їм. Шевченка на
вечірньому відділенні, на одному курсі з Коротичем, і деякими, вже потім моїми
викладачами наприклад, Василенка. На заняття майже завжди мама мене брала з
собою, тож я таким чином і курс лекцій в універі прослухав 2 рази і в 70-х і
80-х. Також був присутнім і на дійсно націоналістичних зібраннях, і коли потім,
і моїм рідним, і їхнім друзям доводилось відволікати увагу кдбшників
респектабельними сімейними прогулянками з дитиною під час справді шпигунських
пристрастей при передачах \enquote{туди} і \enquote{звідти} літератури та
самвидаву, а також і таких фейкових \enquote{подвигів}, як пройти по скверику
чи квіти покласти у знаменну дату. У підсумку про таких \enquote{героїв} -
жертв кдбшних дебілів-звітників дуже точно говорить відоме українське
прислів'я: \enquote{Дурних і в церкві б'ють}.


\iusr{Ірина Оснач}
\textbf{Oleg Chorny} 

На 1 курсі нам одразу повідомили про заборону з"являтися у цей день в парку. Це
був 1972 рік, філфак, російське відділення. Пізніше просто відміняли заняття в
цей день.

Я одного разу (вже під кінець навчання) все ж ризикнула пройтися з червоного
корпусу (вірніш з бібліотеки) до жовтого через парк в цей день. Нікого там не
було: ні закоханих парочок, ні мамочок чи бабусь з дітками, ні шахістів, жодних
націоналістів, якими нас лякали, - лише кілька непримітних сірих чоловіків.
Мене, красиву дівчину, ніхто не затримав, хоча всередині все бриніло, а йшла я
впевнена в собі і гордовита. )) Що то молодість! ))

А ще я бесшабашно ходила (в той день принципово та в інші) в угорських
босоніжках синьо-оранжевого кольору, прекрасно знаючи, що синій і жовтий у парі
заборонені!  @igg{fbicon.wink} 

\end{itemize} % }

\end{itemize} % }

\iusr{Дмитро Коваль}

В 70-80 роках минулого століття відбувся апофеоз КГБшного ренесансу на наших
землях та на наших людях. Кількість штатних агентів зашкалювала, вони стучали
вже одно на одного не тямлячи що це так. Це не я придумав, це нам пізніше
розповіли \enquote{зрадники від КГБ}, що повтікали до країн НАТО. В 90-роках
вони вже почали \enquote{зачистку}, серед свого \enquote{корпусу}, головним
чином отруюючи один одного.  І це не був спонтанний процесс, там були власні
\enquote{планіровщики} та \enquote{майстри}. Ще багато залишилось в живих і
нині, до біса їх і в цій вашій групі \enquote{Киевскіє історії} Ви бачите, їм
\enquote{шило в одному місці шкрябає}. Тусуються і далі. Це покидьки, а не
люди, не ведіться на їхні провокації.

\begin{itemize} % {
\iusr{Надежда Владимир Федько}
\textbf{Дмитро Коваль} До Ваших слів хочу додати один документ з розсекречених архівів КДБ...

\ifcmt
  ig https://scontent-frx5-2.xx.fbcdn.net/v/t39.30808-6/260077366_4722538304472257_5351386075143372153_n.jpg?_nc_cat=109&ccb=1-5&_nc_sid=dbeb18&_nc_ohc=8O2jZ22whKkAX-kIbuY&_nc_ht=scontent-frx5-2.xx&oh=00_AT-ooeFYi69dHKiPKjaeNLCT0uP1Ka3TzLVvK1nDC6cypw&oe=61C9F73B
  @width 0.4
\fi

\iusr{Дмитро Коваль}
Таких \enquote{документів} ця контора продукувала тонами.

\iusr{Tetiana Samoilovych}
\textbf{Дмитро Коваль} 

А пiсля так званоi \enquote{революцii гiдностi} деякi з них вiдвiдали краiну, щоб
познущатися над дiтьми тих, на кого вони ранiше писали своi \enquote{генiальнi}
повiдомлення

\end{itemize} % }

\iusr{Петр Парвус}
Может все таки за фарцовку?

\iusr{Ирина Петрова}

У 1976 році було наше весілля. Тоді була традиція нести квіти до Пам'ятника
Невідомому Солдату. Там фотографії, потім на \enquote{Чайці} по Києву... перед весіллям
я сказала, що ще ми заїдемо неодмінно до пам'ятника Шевченку. Свекр був тоді
парторгом цеху на заводі ім. Петровського та звісно, різко був проти . Що,
казав, за витівки націоналістичні? Що за дурь? Будуть неприємності і у мене (
яким боком то могло його чіпляти???), і у вас в інституті @igg{fbicon.anger}  в мене вже були
трішки ...ммм... нууу ..своєрідні відносини з майбутніми свекрами, його
заперечення тільки масла @igg{fbicon.grin}  у вогонь додало

\ifcmt
  ig https://scontent-frt3-1.xx.fbcdn.net/v/t39.30808-6/260641647_4894820690551499_1268775638948075398_n.jpg?_nc_cat=107&ccb=1-5&_nc_sid=dbeb18&_nc_ohc=TibXJMbQG7EAX9Zkfmn&_nc_ht=scontent-frt3-1.xx&oh=00_AT9HtIoCD2ysCPViGaECnypEx8By_LWaTe7t8UXUzeOq-w&oe=61C8CB8D
  @width 0.4
\fi

\begin{itemize} % {
\iusr{Надежда Владимир Федько}
\textbf{Ирина Петрова} Мужній вчинок! Перш за все для відносин з майбутніми родичами.

\iusr{Ирина Петрова}
\textbf{Надежда Владимир Федько}  @igg{fbicon.laugh.rolling.floor} 
\end{itemize} % }

\iusr{Leonid Dukhovny}

Весь разгул КГБУ в 80-х годах я ощутил на собственной шкуре: от \enquote{критических}
статей в прессе до регулярных вызовов в районое отделение КГБ, а то и в само
главное управление. Среди прочего мне в вину вменялись сионизм и украинский
национализм, якобы просматривающийся в моих песнях и в самом духе Слётов КСП
\enquote{Костёр}, которые я со товарищи устраивал ежегодно, в середине июля в
прикиевских лесах. \enquote{Бурильщики} тайно вербовали осведомителей в нашей среде и,
надо сказать, не безуспешно. Некоторые из сексотов перебрались в Штаты или в
другую страну и затихли. Другие остались в Киеве и тихонько погавкивают из
подворотни, надеясь, что их заметят новые хозяева. Один из тогдашних
руководителей ГБ говаривал, что из всех видов диалога он предпочитает
допрос!... Дошло до того, что я оказался под гласным надзором \enquote{товарищей}

Мою переписку перлюстрировали, телефонные разговоры прослушивались, за моими
перемещениями за пределы Киева следили и проверяли на месте ли я. В течении
почти года каждое божье утро, часов, эдак, в 7.30 утра звонил домашний телефон
и раздавался голос моего персонального куратора: \enquote{Доброе утро, Леонид
Самойлович. Вы меня узнаёте?} И я приловчился радостно отвечать: \enquote{К сожалению,
да}.... \enquote{Были времена, ох было времечко} - как пел гениальный Александр
Дольский

\begin{itemize} % {
\iusr{Надежда Владимир Федько}
\textbf{Leonid Dukhovny} Цікаво було б прочитати про КСП \enquote{Костер}...

\iusr{Leonid Dukhovny}
\textbf{Nadegda Volodymyr Fedko} 

- Вот была такая пресса

Авторская песня в постсоветской Украине: вопросы двуязычия. 

Часть I Поэтическая публицистика Автор: Александр Резник Дата: 17-10-2004 |
01:34:39

О некоторых нынешних \enquote{бардах}....

.....Умолкла муза Семена Каца. Больше не штурмует новые творческие вершины и
экс-киевлянин Леонид Духовный, в прошлом крупнейший организатор бардовской
жизни в Украине (к слову, с его отъездом в 1992 г. практически прекратил
существование КСП «Костер», вошедший в историю не только как бардовский клуб,
но и как политически знаковое явление — единственный из украинских КСП,
снискавший в свое время репутацию отчетливо диссидентского сообщества).

\iusr{Надежда Владимир Федько}
\textbf{Leonid Dukhovny} Дякую!
\end{itemize} % }


\end{itemize} % }
