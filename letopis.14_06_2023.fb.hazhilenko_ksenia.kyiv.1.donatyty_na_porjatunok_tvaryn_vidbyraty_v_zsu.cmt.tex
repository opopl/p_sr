% vim: keymap=russian-jcukenwin
%%beginhead 
 
%%file 14_06_2023.fb.hazhilenko_ksenia.kyiv.1.donatyty_na_porjatunok_tvaryn_vidbyraty_v_zsu.cmt
%%parent 14_06_2023.fb.hazhilenko_ksenia.kyiv.1.donatyty_na_porjatunok_tvaryn_vidbyraty_v_zsu
 
%%url 
 
%%author_id 
%%date 
 
%%tags 
%%title 
 
%%endhead 

\qqSecCmt

\iusr{Олена Шевченко}

Найважче в житті то пошук розумного балансу. Наприклад, іноді тішити себе, але
не розкошувати. Любити дітей, але не розбалувати. Підтримувати життєдіяльність
і функціональність свого дому, вулиці, громади, країни, але не \enquote{закопати} ті
гроші дурно. Особливо зараз, коли наша піраміда Маслоу вивела на найвищий
щабель підтримку сил опору. Якщо ти мисляча і свідома істота то ти постійно
будеш шукати цей баланс)

\begin{itemize} % {
\iusr{Ксения Хажиленко}
\textbf{Олена Шевченко} отож, баланс, а не крайності, в яких або все або нічого 😞

\iusr{Олена Шевченко}
\textbf{Ксения Хажиленко} я бухгалтер, мені професійно легше з тим справлятися 😉 А ще я хитрий бухгалтер і іноді вмію підкорегувати в ту сторону в яку мені треба 😉 Хоча манікюр я так і не роблю, напевне то мені зараз не треба)))
\end{itemize} % }

\iusr{Alexandra Gromova}

Ох той баланс! І наші рефлексії. Я пока без манікюру, але з походом до
косметолога, бо безсоння намалювало на обличчі якийсь суцільний жах ( Аби ще
влада так переймалась цими рефлексіями і пошуком балансу. Як ті шалені гроші
відкопати? Пан Ошовський в своєму пості сьогодні просто підняв з душі
найтемніше, що кипить кожного разу як бачиш \enquote{овочерізки і барабани в укриття}
та розкопана Севастопільську площу

\begin{itemize} % {
\iusr{Ксения Хажиленко}
\textbf{Alexandra Gromova} 

щодня їжджу через ту Севастопольську із прокльонами і думками, що \enquote{баланс}
утворюється десь між тими, хто все віддає і тими, хто все краде. Але я не знаю
від того лікувальних засобів, нажаль

\iusr{Dan Tyalin}
\textbf{Ксения Хажиленко} галоперидол.
\end{itemize} % }

\emph{Оксана Урбан}

100\% з Вами згідна. А оті, що стверджують, що ми обділяємо ЗСУ, ще невідомо, чи взагалі самі донатять на ЗСУ.

А за себе треба дбати. Бо нікому ти не потрібен крім себе самого. Розумна,
співчутлива людина знайде баланс між власними потребами і війною. А немудра, то
так, щоб потявкати, толку всерівно ніякого.

Tetiana Stepanova
Дякую!

    Reply
    See translation
    2w

Іванна Бондаренко
Як завжди влучно. Я тішуся з того що ви почали виділяти час на себе. Не мені вам розповідати про здоров'я , а воно нам дуже потрібне. Нам треба мати сили і здоровий розум щоб зберегти і відновити Україну, щоб у доброму гуморі зустріти наших захисників і допомогти їм адаптуватися у мирному житті. Тому дбаємо про ЗСУ і про себе.✌️🙏💙💛
