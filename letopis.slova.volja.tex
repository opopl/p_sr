% vim: keymap=russian-jcukenwin
%%beginhead 
 
%%file slova.volja
%%parent slova
 
%%url 
 
%%author 
%%author_id 
%%author_url 
 
%%tags 
%%title 
 
%%endhead 
\chapter{Воля}

Воля. Это слово состоит из двух понятий: стремления (\enquote{сила воли},
\enquote{воля к победе}) и свободы (\enquote{вольный ветер}, \enquote{воля
вольная}). В отличие от более социальной \enquote{свободы}, \enquote{воля} это
природная категория (можно сравнить выражения \enquote{свобода слова} и
\enquote{дать волю словам}). Свобода - это право, но оно ограничивается правами
других людей. Воля же не имеет никаких ограничений и не связана с законом. О
том, кто покинул тюрьму по окончании срока, мы говорим \enquote{вышел на
свободу}, а о том, кто сбежал - \enquote{вырвался на волю}. Русский менталитет
определённо предпочитает волю как ничем не сдерживаемую силу. Воля - это
русские бескрайние просторы, «где гуляем лишь ветер... да я!»,
\citTitle{Русский языковой менталитет: 10 основных понятий}, Языковедьма, zen.yandex.ru, 08.06.2021

