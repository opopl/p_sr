% vim: keymap=russian-jcukenwin
%%beginhead 
 
%%file 05_08_2021.fb.bilchenko_evgenia.1.medicina_literatura
%%parent 05_08_2021
 
%%url https://www.facebook.com/yevzhik/posts/4123424067692672
 
%%author Бильченко, Евгения
%%author_id bilchenko_evgenia
%%author_url 
 
%%tags bilchenko_evgenia,literatura,medicina
%%title БЖ. Попытка медицины и литературы
 
%%endhead 
 
\subsection{БЖ. Попытка медицины и литературы}
\label{sec:05_08_2021.fb.bilchenko_evgenia.1.medicina_literatura}
 
\Purl{https://www.facebook.com/yevzhik/posts/4123424067692672}
\ifcmt
 author_begin
   author_id bilchenko_evgenia
 author_end
\fi

БЖ. Попытка медицины и литературы.

Курс уколов цефтриаксона - done.

По 4 таблетки метронидазола в день - done.

Упрямая жажда жизни - done.

Волосы цвета "Тициан" - done.

Дурацкое слово - "дан", типа ты крутой.

Рассказываю по просьбам читателей, с похожими проблемами. Таня Пономарева ,
слушай сюда, Москва, Киев на связи. Не подходит мне глобальная постмодерная
медицина - я месяцами кормлю фармобизнес. 

\ifcmt
  tab_begin cols=2
		width 0.3

     pic https://scontent-cdt1-1.xx.fbcdn.net/v/t1.6435-9/228729152_4123423767692702_428210749826919943_n.jpg?_nc_cat=103&ccb=1-4&_nc_sid=8bfeb9&_nc_ohc=7VIFxn_E1KIAX8ppPq4&_nc_ht=scontent-cdt1-1.xx&oh=625d2e177aff0ad9f157f7103374f4d1&oe=6131E284

     pic https://scontent-cdg2-1.xx.fbcdn.net/v/t1.6435-9/231660527_4123423991026013_8598074924230156332_n.jpg?_nc_cat=108&ccb=1-4&_nc_sid=8bfeb9&_nc_ohc=iDqejPCv5M8AX-tdvI_&_nc_ht=scontent-cdg2-1.xx&oh=74e5af1daefaddadeb9447193cba6fa8&oe=6134A7E6
		width 0.55

  tab_end
\fi

Никаких курсов иммуностимуляторов, модных препаратов по пятьсот гривен с
длинными названиями... Ничего мою болезнь не берет. 

Можно забыть о модных словах типа "психосоматика" и "ВСД": это просто способ не
лечить. Моя болезнь - системное хроническое воспаление. Предварительные очаги:
носоглотка, ЖКТ и мочеполовая система. 

Если у вас любое ОРВИ или простуда длятся по два месяца, если вы не можете
оторвать от подушки голову и еле передвигаетесь по хате, если горло и живот
болят через день, а слабость длится месяцами со свечами небольших температур...
Это может быть рак или хрониосепсис. Домашние животные внутри организма дичают
на фоне нулевого иммунитета, который не подхлестнешь стимулятором. Аутовакцина?
Нет, не верю. 

Как проверить, что с тобой? Способ, которым в новом дивном мире принято пугать
людей, чтобы они вываливали бешеные деньги на гомеопатию, психиатрию и
околонаучную квазиметафизику с длинными перечнями процедур и препаратов, поз и
полезных веществ за сотни купюр. Этот способ - антибиотик. Внутривенно.
Внутримышечно. Часто. Долго. Да, больно. Да, жёстко. Да, побочки. Но помогает
же ж, етить, же ж. Значит, воспаление. Не помогает - копаем дальше. Мне
помогает.

На третий день я начала - не то, что шастать, но подшастывать. Препарат
сработал. В подшастывание вошли небольшие физнагрузки, вроде безуспешной
попытки дерева и не менее безуспешной попытки турника. Парк обещал жизнь и
хамил мне светозеленью прямо в лицо. На шестой я слегла: обострение. На седьмой
я кончила курс. Локальные симптомы ушли. Практически все. Но обещали вернуться.
Общие остались: примерно врач объяснил, как бы такое в его больнице
обследовали. Нужна более четкая диагностика. Мне оказывали первую помощь, для
лечения нужно класть в больницу.

Да, я нашла врача в сети. Сама. Он - сторонник старой военной советской
медицины. Олдскульный. Донецкая стратегия. Он рекомендует то, что я все время
от деда с бабушкой слышала, но не слушала. Они помогали мне. Всегда. И он
первый, кто помог. Но он далеко, в Нижнем Новгороде, большего он не может.
Больше: это ложиться, он с таким кладет. Все равно спасибо. Его зовут Владимир,
пусть Бог пошлет ему здоровья. 

А мне надо думать: рядом нет никого. 

Вот так, родня. Я желаю вам не болеть. Это очень мерзко. То, что случилось со
мной. От бессилия я так зарылась в работу за компом, что света белого не вижу.
Или сплю от слабости. Эх, прощай, лето, здравствуй, тельце немощное и негодное. 

Немного страшно, что меня ждёт, но я не боюсь уколов и никогда не боялась.
Никаких. И вида крови, которая у меня сейчас неважно свёртывается (пшл нх,
аспирин) тоже не боюсь, Николай Топало , ты знаешь: кровь - ручьем, а БЖ -
отмороженная (резала палец, а вытек литр). Гемоглобин низкий, но это понятно. 

Ну, вот. Я не сплю: мы делаем новую книгу для вас с великим рыцарем эпохи
\verb|#АндрейКостинский|. На обложке: я и книга моего дизайна. Очень хочется
поделиться, она ещё на станке. Спасибо, что не забываете.

\ii{05_08_2021.fb.bilchenko_evgenia.1.medicina_literatura.cmt}
