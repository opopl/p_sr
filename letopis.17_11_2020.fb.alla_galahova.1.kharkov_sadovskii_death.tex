% vim: keymap=russian-jcukenwin
%%beginhead 
 
%%file 17_11_2020.fb.alla_galahova.1.kharkov_sadovskii_death
%%parent 17_11_2020
 
%%url https://www.facebook.com/permalink.php?story_fbid=760585078206025&id=100027636202546
%%author 
%%tags 
%%title 
 
%%endhead 

\subsection{В Харькове забили насмерть известного режиссера-постановщика, уроженца Донбасса Леонида Садовского}
\Purl{https://www.facebook.com/permalink.php?story_fbid=760585078206025&id=100027636202546}

Четырнадцатого ноября в Харькове было совершено нападение на Леонида Садовского, который являлся завкафедрой Харьковского национального университета, передаёт корреспондент «Донбасс Сегодня» со ссылкой на коллег педагога.

Сообщается, что на преподавателя напали неизвестные и ударили тупым предметом по голове. Пострадавший был срочно госпитализирован в Харьковскую больницу неотложной хирургии.

Вчера от полученных травм, профессор Леонид Садовский скончался в больнице.

\obeycr
«Леонид Викторович Садовский... Светлая память... 
Мой любимый Учитель, человек, первый в меня поверивший в профессии, давший так много знаний, умений, любви, поддержки, тепла... 
Нет слов... 
Жалею, что так и не успела к вам съездить повидаться в последнее время.... 
Пусть Вам там будет тихо, радостно и спокойно... 
Или как Вы захотите... 
Я буду очень скучать!!! 
И, надеюсь, убийца будет наказан!», - написала в Facebook его студентка.
\restorecr

Леонид Садовский был уроженцем Донбасса, родился первого августа 1949 года в
Енакиево, был заслуженным мастером искусств Украины.  На его счету было много
постановок для детей и юношей. 

Свои программы он представлял по всей Украине и России: «Валентин и Валентина»
М. Рощин, «Февраль» М. Шатров, «Последняя женщина сеньора Хуана» Л. Жуховицкий,
и многие другие.
