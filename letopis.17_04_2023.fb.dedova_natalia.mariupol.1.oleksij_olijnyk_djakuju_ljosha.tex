%%beginhead 
 
%%file 17_04_2023.fb.dedova_natalia.mariupol.1.oleksij_olijnyk_djakuju_ljosha
%%parent 17_04_2023
 
%%url https://www.facebook.com/permalink.php?story_fbid=pfbid02F71YGrbYKVA7kPFyx3eM3VaAk5MSTMovcjJqgjRvjLT7WFczwWMLjN7vAg3tyarcl&id=100007662284921
 
%%author_id dedova_natalia.mariupol
%%date 17_04_2023
 
%%tags mariupol,mariupol.war
%%title З Олексій Олійник ми знайомі з 1 класу. І саме його листівки до 8 березня я зберігала в батьківській квартирі
 
%%endhead 

\subsection{З Олексій Олійник ми знайомі з 1 класу. І саме його листівки до 8 березня я зберігала в батьківській квартирі}
\label{sec:17_04_2023.fb.dedova_natalia.mariupol.1.oleksij_olijnyk_djakuju_ljosha}

\Purl{https://www.facebook.com/permalink.php?story_fbid=pfbid02F71YGrbYKVA7kPFyx3eM3VaAk5MSTMovcjJqgjRvjLT7WFczwWMLjN7vAg3tyarcl&id=100007662284921}
\ifcmt
 author_begin
   author_id dedova_natalia.mariupol
 author_end
\fi

\#міймаріуполь

\#голосимирних

З Олексій Олійник ми знайомі з 1 класу. І саме його листівки до 8 березня я
зберігала в батьківській квартирі. 

Але в березні 2022 листівки згоріли. Разом з квартирою. 

І саме в березні загинули наші дві однокласниці Людмила Нестерчук та Галина
Евдокимова . 

І саме тоді світ розірвався. Навпіл. 

Дякую, Льоша, що вижив! І що поруч! 

"Люди ходили, как муравьи. Искали убежище". 

Сирены, очереди в банкоматы. Люди покупали все, что видели. Снял деньги. Пошел
в магазин. Но купил не все. Не было соли, муки. Ребенок в школу не пошел. 

Пытались анализировать ситуацию. Но в происходящее не верилось. Когда город
оказался в блокаде, стало страшно и жутко. 

2 марта подготовили бомбоубежище на "Азовстали". Ожидали людей. Были запасы
воды, сухпайки. 

2 марта отключили связь. 3 марта пришел домой. Хаос. Люди искали продукты.
Магазины почти не работали. Хлеба, муки, соли не было. 

Потом отключили газ. Скооперировались с соседями, построили жаровню. Собирали
дрова. 

7 марта. Запомнил этот день. Двор обстреляли градами. 7 прилётов. 2 погибли.
Было много раненых. Парень, лет 20, сосед, работал в "Коммунальнике". Свозил
трупы за Центральный рынок. Двое погибли. Вынесли их к дороге. Чтобы дети не
видели. 

С братом пошли к родителям. С Кирова на Черёмушки. Шли 6 часов. Город был
разбит. Люди ходили, как муравьи. Искали убежище. 

Парень сказал, не ходите на Черёмушки. Там много трупов. 

Мужчина поймал украинское радио. Вокруг него много людей собралось. Послушать. 

Вокруг криницы много людей. Кто-то не возвращался. 

Жил на 9 этаже. Заделывал окна. Окна выходили на криницу. В этот момент упала
мина. Люди замерли в ступоре. Все остались живы. 

В подъезде жила врач. Она помогала, чем могла. Но достать лекарства было
невозможно. Мародёры волокли два мешка лекарств. Отбили. Отнесли врачам. 

Самое страшное - это авиаудары. Предугадать, где упадет, было невозможно. Ночью
прилетел самолёт. Проснулись от того, что дом затрясся. Думал, что дом
сложится. После этого переселились в подвал. Оборудовали его. 

Когда закипела вода, чайник начал свистеть. Люди испугались, подумали, что
самолёт летит. 

Когда выбрасывал мусор, на бак сбросили бомбу. Электроподстанция приняла на
себя удар. На меня посыпались камни. Было страшно. 

По ночам слышали вой волков. В зоопарке. Или от голода, или от холода. Это
жутко было слушать.  

🇺🇦 Друзі, якщо Ви готові розповісти свою історію, пишіть у приват або в
коментарях. Світ має почути кожного! 
