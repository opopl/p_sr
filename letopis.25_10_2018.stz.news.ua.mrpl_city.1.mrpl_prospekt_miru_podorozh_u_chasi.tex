% vim: keymap=russian-jcukenwin
%%beginhead 
 
%%file 25_10_2018.stz.news.ua.mrpl_city.1.mrpl_prospekt_miru_podorozh_u_chasi
%%parent 25_10_2018
 
%%url https://mrpl.city/blogs/view/prospekt-miru-podorozh-u-chasi
 
%%author_id demidko_olga.mariupol,news.ua.mrpl_city
%%date 
 
%%tags 
%%title Маріупольський проспект Миру: подорож у часі
 
%%endhead 
 
\subsection{Маріупольський проспект Миру: подорож у часі}
\label{sec:25_10_2018.stz.news.ua.mrpl_city.1.mrpl_prospekt_miru_podorozh_u_chasi}
 
\Purl{https://mrpl.city/blogs/view/prospekt-miru-podorozh-u-chasi}
\ifcmt
 author_begin
   author_id demidko_olga.mariupol,news.ua.mrpl_city
 author_end
\fi

\begin{quote}
\em Пропоную здійснити невелику подорож у часі. Сьогодні ми спробуємо повернутися в
Маріуполь XIX ст., відчути шум його центральної вулиці 100 років тому, в епоху,
коли ще не було автомобілів і комп'ютерів. Ми познайомимося з історією
найстарішої та найголовнішої вулиці міста – проспектом Миру. Дізнаємося, які
історії та таємниці зберігають її будівлі.
\end{quote}

\ii{25_10_2018.stz.news.ua.mrpl_city.1.mrpl_prospekt_miru_podorozh_u_chasi.pic.1}

\textbf{Проспект Миру, він же пр. Республіки, раніше - Катерининська вулиця, ще раніше
- вулиця \enquote{Велика}} – найголовніша, стратегічно важлива вулиця як старого, так і
сучасного Маріуполя, магістраль з минулого в сучасне. До 1876 р. вулиця в
ужитку називалась \emph{\enquote{Великою}}, а з вересня 1876 р. – \emph{Катерининською}. Головна
вулиця змінила свою назву на згадку про те, що Маріуполь виник з благословення
імператриці Катерини II. Після Жовтневого перевороту вулиця стала офіційно
називатись \emph{проспектом Республіки}, проте в ужитку слово \enquote{Республіка} опускалося.
В часи \enquote{хрущовської відлиги} молодь дала свою неофіційну назву вулиці – \emph{Брод},
скорочено від \enquote{бродвей}. У 1960 р. на честь 90-річчя з дня народження В. І.
Леніна проспект отримав ім'я \emph{Леніна}. У 2016 році в рамках політики
декомунізації вулицю було названо \emph{проспектом Миру}.

Центральна вулиця Маріуполя пам'ятала засновника міста - преосвященного
Ігнатія, по ній двічі проїжджав імператор Олександр I, гуляв великий пейзажист
А. І. Куїнджі.

На честь 240-річчя з дня народження Маріуполя була проведена масштабна
реконструкція Театрального скверу, завдяки якій центр Маріуполя став більш
сучасним та європейським.

\textbf{Читайте також:} \emph{Мариупольский сквер}%
\footnote{Мариупольский сквер, Сергей Буров, mrpl.city, 14.10.2018, \url{https://mrpl.city/blogs/view/mariupolskij-skver}} %
\footnote{Internet Archive: \url{https://archive.org/details/14_10_2018.sergij_burov.mrpl_city.mariupolskij_skver}}

\ii{25_10_2018.stz.news.ua.mrpl_city.1.mrpl_prospekt_miru_podorozh_u_chasi.pic.2}

Про деякі архітектурні пам'ятки, розташовані на проспекті Миру, я вже
розповідала раніше. Проте давайте згадаємо найголовніші з них. Все ж таки
особливу увагу слід звернути на стару частину міста. Традиційно почнемо з
будівлі драматичного театру в Маріуполі. Приміщення драмтеатру будувалося
протягом 1956–1960-х рр. Будівля маріупольського Драматичного театру побудована
в стилі радянського монументального класицизму з великою кількістю ліпних
декорованих елементів. Варто зазначити, що будівля драматичного театру – одна з
6 памяток архітектури місцевого значення. Цікаво, що протягом кількох років
площу перед приміщенням драмтеатру прикрашала монументальна арка, яку було
зруйновано напередодні відкриття театру. Нині вона збереглася тільки на
світлинах.

\ii{25_10_2018.stz.news.ua.mrpl_city.1.mrpl_prospekt_miru_podorozh_u_chasi.pic.3}

А якщо ми повернемося в минуле, то на місці театру побачимо чарівну \emph{церкву
святої Марії Магдалини}, фундамент якої був закладений ще в 1862 р. Сьогодні в
Театральному сквері можна побачити скляний саркофаг, в якому зберігається
фундамент церкви, а поруч незабаром буде встановлений під скляним куполом
бронзовий макет церкви святої Марії Магдалини.%
\footnote{В Мариуполе установят бронзовый макет храма Марии Магдалины (ФОТО), Роман Катріч, mrpl.city, 08.10.2018, \url{https://mrpl.city/news/view/v-mariupole-ustanovyat-bronzovyj-maket-hrama-marii-magdaliny-foto}}

\ii{25_10_2018.stz.news.ua.mrpl_city.1.mrpl_prospekt_miru_podorozh_u_chasi.pic.4}

Гуляючи проспектом Миру, можна побачити унікальні \emph{будинки зі шпилями}. Вони
почали будуватися в 1953 р. за проектом Харківського інституту
\enquote{Міськбудпроект}, архітектором було призначено Л. Яновицького. Шпилі
будинків видно з усіх точок міста, за своєю архітектурою будинки виділяються з
навколишньої забудови і є головними орієнтирами старої частини міста. Два
будинки розділяє проїжджа частина і тротуари вулиці Куїнджі. У 2000-х роках
західний будинок зі шпилем був пофарбований у білий колір (східний залишився
природного цегельного кольору). Будинки створені у традиціях класицизму
середини ХХ століття: масивний рустований цоколь, арочні отвори в еркерах,
колони та пілони іонічного ордера, ліпнина на стінах, кутові частини будинків
вінчаються шпилями та фігурними парапетами. У центрі – 7-поверхова частина, до
якої примикають 4- і 5-поверхові крила. Завдяки шпилям будівлі є архітектурним
акцентом перетину проспекту Миру і вулиці Куїнджі. Обидва будинки зі шпилями є
пам'ятками архітектури місцевого значення.

\textbf{Читайте також:} \emph{Мариуполь: каким он был, какой он есть}%
\footnote{Мариуполь: каким он был, какой он есть, Сергей Буров, mrpl.city, 01.07.2017 } %
\footnote{Internet Archive: \url{https://archive.org/details/01_07_2017.sergij_burov.mrpl_city.mariupol_kakim_on_byl_kakoj_on_est}}

\ii{25_10_2018.stz.news.ua.mrpl_city.1.mrpl_prospekt_miru_podorozh_u_chasi.pic.5}

Загадковим є будинок на \emph{проспекті Миру, 40}. Для мене особисто це маріупольський
будинок-хамелеон. Сьогодні тут розташована кав'ярня \enquote{IZBA-читальня}. Однак до
революції ця будівля належала адвокату Юр'єву, потім тут знаходилася редакція
\enquote{Мариупольского справочного листка} (однієї з перших міських газет), був
розміщений один з перших кінотеатрів. Тут же збиралися діти з першого
піонерського загону. Але незабаром кіно, як кажуть, закінчилося, і в цьому
особняку розмістилася сувора служба – НКВС. Характерно, що нацисти, захопивши
місто восени 1941 р., відразу ж розмістили тут... гестапо. Напевно, тут
перетнулися енергетичні потоки, як сказали б езотерики. А сам по собі будинок
приємний поєднанням пілястр з фронтонами. Тут зустрічаються бароко та ампір.

\textbf{Читайте також:} \emph{Стерлись из памяти, но не из истории: потерянные храмы Мариуполя в видеотуре 360˚ (ВИДЕО 360°)}%
\footnote{Стерлись из памяти, но не из истории: потерянные храмы Мариуполя в видеотуре 360˚ (ВИДЕО 360°), Анастасія Папуш, mrpl.city, 24.10.2018, \url{https://mrpl.city/news/view/sterlis-iz-pamyati-no-ne-iz-istorii-poteryannye-hramy-mariupolya-v-videoture-360-video-360}}

\ii{25_10_2018.stz.news.ua.mrpl_city.1.mrpl_prospekt_miru_podorozh_u_chasi.pic.6}

З 1905 р. чудово збереглось \emph{колишнє відділення державного банку} (або Державна
банківська контора). Цікаво, що і наразі тут розташована фінансова установа –
філія \enquote{Укрсоцбанку}. Після чималих організаційних клопотів в 1867 році був
створений Маріупольський громадський банк. Однак через кілька років він був
закритий, швидше за все, з причини появи в першій половині 70-х років нових
банківських установ. На замовлення Маріупольської міської думи в 1897 р. в
Санкт-Петербурзі був складений проект представницької двоповерхової будівлі для
Маріупольського відділення Державного банку. У 1905 р. Маріупольське відділення
перейшло в нову будівлю, виконану в класичних архітектурних традиціях. Це
найстаріша банківська будівля в Приазов'ї і на південному сході України. Саме
через цей банк проходило фінансове забезпечення будівництва маріупольських
металургійних заводів \enquote{Нікополь} і \enquote{Провіданс}, зведення міських приватних і
громадських будівель. Багатопромислові, сільськогосподарські і культурні
об'єкти по всьому Маріупольському повіту виникли завдяки кредитним можливостям
місцевого відділення Державного банку. Саме це відділення Державного банку СРСР
забезпечувало надходження платежів на будівництво житлового фонду,
індустріальних і соціальних об'єктів міста в 50-80-ті рр. минулого століття.
Той факт, що до кінця шістдесятих років біля стін цього банку на першотравневі
і листопадові свята встановлювалися трибуни, з яких партійні і радянські
керівники, так би мовити, батьки міста вітали святкові колони демонстрантів,
свідчить про центральну роль цього місця і банківської будівлі протягом всієї
історії міста. Проте сьогодні будівля Маріупольської філії АКБ \enquote{Укрсоцбанк}
виставлена на продаж.

Саме на центральній вулиці міста зберігся найстаріший кінотеатр у місті –
\emph{\enquote{Победа}}, тут маріупольці насолоджувалися театральним мистецтвом, відвідуючи
Зимовий театр (розташовувався за адресою: пр. Миру, 24).

\ii{25_10_2018.stz.news.ua.mrpl_city.1.mrpl_prospekt_miru_podorozh_u_chasi.pic.7}

\emph{У будинку № 37} була розташована біржа, а у сусідньому будинку – кондитерський і
булочний магазин. Нині будинок № 37 прикрашає меморіальна дошка, присвячена
Георгію Антоновичу Костоправу – відомому перекладачу, румейському і радянському
поету, класику літератури приазовських греків. На табличці вказано, що він у
1932–1935 рр. працював у редакції газети \enquote{Коллективістіс}. Зазначимо, що
редакція неодноразово змінювала своє місцезнаходження. І цей будинок – один з
багатьох.

\textbf{Читайте також:} \emph{В Театральном сквере Мариуполя снова изменился ландшафт, выросли зеленые холмы}%
\footnote{В Театральном сквере Мариуполя снова изменился ландшафт, выросли зеленые холмы, mrpl.city, 25.10.2018, \url{https://mrpl.city/news/view/v-teatralnom-skvere-mariupolya-snova-izmenilsya-landshaft-vyrosli-zelenye-holmy}}

Повертаючись у минуле, на вулиці Катерининській можна було побачити особняк Г.
Гофа, сьогодні ж тут розташувалася редакція газети \enquote{Приазовский рабочий} і
будівля залишається яскравою прикрасою головної вулиці міста.

\ii{25_10_2018.stz.news.ua.mrpl_city.1.mrpl_prospekt_miru_podorozh_u_chasi.pic.8}

Багато будівель на проспекті Миру побудовані в післявоєнний час на місці
зруйнованих одноповерхових будівель довоєнної споруди. Ці будинки побудовані в
стилі \emph{\enquote{сталінського ампіру}} (інша назва \enquote{сталінський неоренесанс}, характерні
риси: стилізація під початок XIX ст. епоху Наполеона, пізній класицизм або
ампір (фр. Ампір – \enquote{імперія}). Зовні використовувалися жовта штукатурка, білі
колони, біла ліпнина з великою кількістю радянської символіки (п'ятикутні
зірки, серп і молот, барельєфи робітників). В інтер'єрі для прикраси
використовувалися мармурові плити, бронзові лаврові вінки з радянською
символікою, світильники, стилізовані під факели. Також в оформленні
використовувалися елементи бароко. Придивимося? Бачимо безліч рельєфів з
вищезгаданої символікою. У Маріуполі цього стилю дотримувались всього 5 років
(з 1950 р.). Саме в стилі \enquote{сталінського ампіру} побудовані вже знайомі нам
\enquote{будинки зі шпилями} та будинок № 24.

\ii{25_10_2018.stz.news.ua.mrpl_city.1.mrpl_prospekt_miru_podorozh_u_chasi.pic.9}

Крім інших важливих і унікальних споруд, все ж найбільше вражав своєю величчю і
масштабами \emph{Собор святого Харлампія} (побудований на місці ДТСААФ). Це було
найбільше приміщення в історії Маріуполя, адже Харлампієвський собор вміщував
п'ять тисяч осіб одночасно.

Проспект Миру сьогодні, як і колись - центральна вулиця міста...  Розвинута,
велика, затишна та історично важлива... \emph{Головне – завжди пам'ятати, що зв'язок
часів не розпадається. Те, що було вчора, існує і сьогодні – в пам'яті й
слідах, які залишає людина. Ці сліди – вчинки, мистецтво і все, що побудоване...}

\textbf{Читайте також:} \emph{В Мариуполе открылся креативный центр для творческих детей}%
\footnote{В Мариуполе открылся креативный центр для творческих детей, Анастасія Папуш, mrpl.city, 23.10.2018, \url{https://mrpl.city/news/view/v-mariupole-otkrylsya-kreativnyj-tsentr-dlya-tvorcheskih-detej-foto}}

\clearpage
