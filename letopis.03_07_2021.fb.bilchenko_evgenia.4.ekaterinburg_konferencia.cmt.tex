% vim: keymap=russian-jcukenwin
%%beginhead 
 
%%file 03_07_2021.fb.bilchenko_evgenia.4.ekaterinburg_konferencia.cmt
%%parent 03_07_2021.fb.bilchenko_evgenia.4.ekaterinburg_konferencia
 
%%url 
 
%%author 
%%author_id 
%%author_url 
 
%%tags 
%%title 
 
%%endhead 
\subsubsection{Коментарі}

\begin{itemize}
%%%fbauth
%%%fbauth_name
\iusr{Alex Melnik}
%%%fbauth_url
%%%fbauth_pic
%%%endfbauth
Так тигру Сурю отстранили только от преподавания студентам Украины?

А в соседних свкрхдержавах она представляет Науку Укрвины, пока штатные учёные
НПУ получают потом и кровью честно заработанную получку?

Суря же не работает больше!

\begin{itemize}
%%%fbauth
%%%fbauth_name
\iusr{Евгения Бильченко}
%%%fbauth_url
%%%fbauth_pic
%%%endfbauth
\textbf{Alex Melnik} Суря - это Стефания Данилова . А. моя тигра названа в ее честь. Суря печатает монографии в: Петербурге, Вашингтоне, Варшаве и статьи в Мюнхене, Мадриде, Торонто, Бостоне и... Забыла где: за полгода отстранения около 40 международных проектов . Тем временем, дражайший вуз не собрал студентов на магистратуру: после случая с БЖ две группы распались, третья не собралась.

%%%fbauth
%%%fbauth_name
\iusr{Евгения Бильченко}
%%%fbauth_url
%%%fbauth_pic
%%%endfbauth
Суря работает без преподавания: она очень опасна для студентов своим свободомыслием, поэтому студенты собираются у нее дома и слушают раз в сто больше. А уволить Сурю по профнепригодности не вышло. Суря первая по научному рейтингу кафедры в 16 тысяч (с отрывом от второго места в 3 тысячи).

%%%fbauth
%%%fbauth_name
\iusr{Alex Melnik}
%%%fbauth_url
%%%fbauth_pic
%%%endfbauth
Об этой ситуации необходимо трубить громко и ярко!
ПреЗе знает?
СМИ знают?
Молчат или где?

%%%fbauth
%%%fbauth_name
\iusr{Евгения Бильченко}
%%%fbauth_url
%%%fbauth_pic
%%%endfbauth
\textbf{Alex Melnik} Ты знаешь, я так счастлива, что мне все равно. Мой научный рейтинг:

%%%fbauth
%%%fbauth_name
\iusr{Евгения Бильченко}
%%%fbauth_url
%%%fbauth_pic
%%%endfbauth
\textbf{Alex Melnik} пруфы через скрины. С вордовскими переписками, как они, я не работаю.

\ifcmt
  ig https://scontent-mia3-1.xx.fbcdn.net/v/t1.6435-9/211621938_4031394673562279_2384170832265773440_n.jpg?_nc_cat=108&ccb=1-3&_nc_sid=dbeb18&_nc_ohc=4Ag7iw3PnAEAX_vM0jp&_nc_ht=scontent-mia3-1.xx&oh=dd30ed2fe2412e95ec29b59396bad354&oe=60E69ADE
  width 0.4
\fi

%%%fbauth
%%%fbauth_name
\iusr{Alex Melnik}
%%%fbauth_url
%%%fbauth_pic
%%%endfbauth
Плохо, что все равно!
Это же война.
Война не на живот...

%%%fbauth
%%%fbauth_name
\iusr{Евгения Бильченко}
%%%fbauth_url
%%%fbauth_pic
%%%endfbauth
\textbf{Alex Melnik} война не там, Лех. Хотя и тут тоже, да. Ладно. Когда ты прав, я не спорю. Видишь? Я хорошая .

%%%fbauth
%%%fbauth_name
\iusr{Alex Melnik}
%%%fbauth_url
%%%fbauth_pic
%%%endfbauth
Желаю победить!
Готов внести посильную (и не только) лепту в Победу добра над злом.

\end{itemize}

%%%fbauth
%%%fbauth_name
\iusr{Dimitry Nemtsov}
%%%fbauth_url
%%%fbauth_pic
%%%endfbauth
Dixi...

\ifcmt
  ig https://scontent-mia3-1.xx.fbcdn.net/v/t1.6435-9/200546552_10225788509401032_9147834633450845178_n.jpg?_nc_cat=106&ccb=1-3&_nc_sid=dbeb18&_nc_ohc=YdTWmMrg7bAAX-uODFW&_nc_ht=scontent-mia3-1.xx&oh=629ec05ad42173d3f22982635080f1d7&oe=60E78237
  width 0.4
\fi

\begin{itemize}
%%%fbauth
%%%fbauth_name
\iusr{Евгения Бильченко}
%%%fbauth_url
%%%fbauth_pic
%%%endfbauth
\textbf{Dimitry Nemtsov} Да, старая информация обо мне до перемены взглядов из
сайта \enquote{Перемоги} от одержимых бесами непрощения грешников, кстати, без
русофобии, но вы ее лет пять мне припоминаете и множите, видно, заело
пластинку. Что удивительно, книга таки вышла в федеральном издательстве и
получила федеральную издательскую премию РФ за гуманизм в освещении страданий
людей. А министр культуры вручил мне за нее планшет \enquote{Престижиум} и диплом с
гербом России). Было там в зале много правительственных лиц и русских
писателей. Все хлопали тексту \enquote{Кто я}. Был то 2014 или 2015 год. Обязательно
прочтите. Книга хорошая, стихи я всегда с Божьей помощью слышала духовные.
Удивительно, но именно позолоченный официальный диплом у меня только за эту
книгу и от России. Вы можете написать Путину, ему будет очень интересно, это
государственное мероприятие. Но почему-то пока ресурс, хранящий эти данные,
интересует только русскую разведку и ФСБ. Странное совпадение, да?

%%%fbauth
%%%fbauth_name
\iusr{Евгения Бильченко}
%%%fbauth_url
%%%fbauth_pic
%%%endfbauth

Кстати, я не знаю, что такое dixi: меня кураторы из США не учили. Ни методички
писать, ни сленгу, ни пустым страницам, ни прокси-платформам, ни русских
ссорить по злопамятству. Вы на Донбассе добровольцем не были, не хотели русский
мир защитить? Мы вот его в Киеве защищаем. И ничего. Но информация о ваших
\enquote{товарищах} от российских военных - все интереснее и интереснее. Я так понимаю,
тема метафизики русской души, прощения и приложения хоть каких-то усилий, чтобы
у вас Ксения Собчак то же самое сейчас не орала с центральных каналов и не
собирала толпы на Невском вас не беспокоит. И это учтем. Вам же так надо: убить
здесь антифашистское сопротивление, отколоть от России Новороссию и устроить
цветную революцию посреди Дворцовой. И все под предлогом \verb|#непростимпоэтуБЖ|. Уже
после всего, что я тут пережила, за такое даже Сорос не заплатит.

%%%fbauth
%%%fbauth_name
\iusr{Dimitry Nemtsov}
%%%fbauth_url
%%%fbauth_pic
%%%endfbauth
dixi с латыни значит \enquote{у меня все}.

%%%fbauth
%%%fbauth_name
\iusr{Евгения Бильченко}
%%%fbauth_url
%%%fbauth_pic
%%%endfbauth
\textbf{Dimitry Nemtsov} ясно, спасибо, латинянин.

%%%fbauth
%%%fbauth_name
\iusr{Dimitry Nemtsov}
%%%fbauth_url
%%%fbauth_pic
%%%endfbauth
Мы пскопские...

%%%fbauth
%%%fbauth_name
\iusr{Евгения Бильченко}
%%%fbauth_url
%%%fbauth_pic
%%%endfbauth
\textbf{Dimitry Nemtsov} то есть, перекрашенные в красных белые? Потомки БЖ? А мы из Кронштадта, так что посмотрим, кто кого. Внучки боевых советских офицеров с пониманием Церкви глаз перед ц вилами не опускают. А то, что вы - цивил, тут и к гадалке не ходи.

\end{itemize}

%%%fbauth
%%%fbauth_name
\iusr{Владислав Сушков}
%%%fbauth_url
%%%fbauth_pic
%%%endfbauth

отлично вы с ним сработались )))

\begin{itemize}
%%%fbauth
%%%fbauth_name
\iusr{Евгения Бильченко}
%%%fbauth_url
%%%fbauth_pic
%%%endfbauth
\textbf{Владислав Сушко}

в С кем, с Екатеринбургом или с анонимами из Киева-Москвы-Киева? Я вроде
отлично с обеими сработалась: в первом случае научными кейсами, во втором-
несколько иными.



%%%fbauth
%%%fbauth_name
\iusr{Владислав Сушков}
%%%fbauth_url
%%%fbauth_pic
%%%endfbauth
\textbf{Евгения Бильченко} с тигрой ))))



%%%fbauth
%%%fbauth_name
\iusr{Евгения Бильченко}
%%%fbauth_url
%%%fbauth_pic
%%%endfbauth
\textbf{Владислав Сушков} 

Тигры - моя дрессировочная стихия). Но этот - друг. Жаль, я ее красноту
предала: закрасили седину бальзамом, и теперь хожу какая-то русо-медная.



%%%fbauth
%%%fbauth_name
\iusr{Алексей Бажан}
%%%fbauth_url
%%%fbauth_pic
%%%endfbauth
\textbf{Евгения Бильченко} Нет, потому что не понимаете, с чем имеете дело.
Читать надо современных авторов, а не спекуляции о "метафизике русской души"
столетней давности. Не то чтоб в России все поголовно читали Галковского и
Крылова, но некий сдвиг в умонастроении они зафиксировали. "Традиционализм —
это, скажем так, нечто противоположное такому вот «без памяти прощенью».
Традиция — это Память, и — прежде всего — память о содеянном против нас зле.

Это, впрочем, тавтология. Память — это всегда память о зле, совершенном над
нами, память о нашем поражении, о потерях и утратах. Иначе она не нужна и
бессмысленна. Благо, которое есть, присутствует. Его не вспоминают, им
обладают. Если его нет, оно было утрачено, и память — это память о том, что
было утрачено и как оно было утрачено. Но Благо — это не событие. Память о
самом Благе почти невозможна. Оно или есть, или нет. Нас соединяет с ним только
память о его утрате, и отказ от этой памяти есть отказ от нашей единственной
связи с ним, связи, которая — одна! — поддерживает в нас дух. Эта связь слаба,
а желание забыть почти всесильно, на это есть тысячи причин, — ну хотя бы
потому, что воспоминания о гибели Блага невыносимы. Но если их не останется, не
останется даже тени надежды. Забвение Зла — это смерть сущности. Тот, кто забыл
о причиненном ему зле, тем самым согласился с ним, одобрил и принял Зло, то
есть сам его совершил и стал его частью.

Добро нужно помнить и чтить, как велят нам чувства чести и благодарность. Зло
нужно помнить вечно, всегда, даже если все наши чувства и порывы противятся
этому. Ибо только память о совершенном Зле может остановить его бесконечное
повторение. Не нужно помнить о добре, чтобы творить добро. Но необходимо
помнить о Зле, чтобы противостоять ему. На этом простом соображении основаны
онтология, гносеология и этика традиции." 

\url{https://traditio.wiki/Текст:Константин_Крылов:Традиция}

%%%fbauth
%%%fbauth_name
\iusr{Евгения Бильченко}
%%%fbauth_url
%%%fbauth_pic
%%%endfbauth
\textbf{Алексей Бажан} Вы излагает не Дугина и не \enquote{бытие-к-смерти} Хайдеггера.
Вы излагаете - чьими устами мне все равно - американскую теорию травма-памяти,
которую поддерживает Вятрович. Слово в слово. Это либеральная политика
искусственно производимой памяти жертвы, коей русский народ никогда не был.
Память - всегда жива, она в настоящем переживании события сорадования, она - 9
мая, праздник со слезами на глазах. Никто не призывает к забвению зла, я резко
критикую тиранию всеобщего покаяния и репрессивную толерантность, - речь идёт о
том, чтобы не превращать себя из победителя зла в его вечную жертву,
беспомощного страдальца вместо богатыря. Именно этого от нас ждут поляки и
американцы. Не будет этого. Слова типа \enquote{вы не понимаете, с чем имеете дело} -
это слабость оппонента, решившего начать с личного трюизма.В Традиции нет
парадигмы времени, лишь синтагма архетипа: живы все: от Данилевского до Дугина,
от Баратынского до Ермолаева. Применять капиталистическую скорость к традиции
суть психологии потребления: чем моднее, тем лучше. Метафизика не является
спекуляцией: иначе вам придется отринуть Библию и Пушкина. Прощается человек:
грех является вечным предметом нравственного покаяния и искоренения: я ильинец,
а не толстовец в этом. Память же лишь о вечном поражении - это начало нацизма
Веймарской республики. Я понимаю, что вы начитались ерунды и вас прет. Слава
Бессмертному полку. Память - жива. И это память Победы. И я не потомок жертвы,
я потомок победителя. Слишком много в вас постмарксистской постмодернистской
зауми с двойными стандартами культа садомазохистской виктимности. Извините. Тем
более, что вы оставили комментарий под конференцией на другую тему, что вообще
удивляет. Волновались? Или у вас в семье никто не воевал, не побеждал?
\verb|#бессмертныйполк|

\ifcmt
  ig https://scontent-mia3-2.xx.fbcdn.net/v/t1.6435-9/209933919_4032984210069992_5119909071037491600_n.jpg?_nc_cat=102&ccb=1-3&_nc_sid=dbeb18&_nc_ohc=Y73MziusONIAX9KVWQx&_nc_ht=scontent-mia3-2.xx&oh=3e86a37381d9a95dbae7f32ce9a0b4b8&oe=60E7BD99
  width 0.4
\fi

%%%fbauth
%%%fbauth_name
\iusr{Алексей Бажан}
%%%fbauth_url
%%%fbauth_pic
%%%endfbauth
\textbf{Евгения Бильченко} Я Вам, собственно, написал, что в России заметно
упали акции \enquote{метафизики русской души} а ля Соловьев-Бердяев в пользу
институционального понимания нации/народа (что на нерефлексивном уровне
проявляется в размышлениях \enquote{а что мы с этого получим?}, в том числе в денежном
эквиваленте). Чисто институциональная логика свободна, между прочим, от
рессентимента, тогда как представления о коллективных телах и симфонических
личностях для него комфортная почва. Что до русофобов из институтов
национальных памятей, то, увы, их катехизис это Истоки и смысл русского
коммунизма Бердяева.

\end{itemize}


\end{itemize}

