% vim: keymap=russian-jcukenwin
%%beginhead 
 
%%file 18_11_2020.fb.tanya_naumchuk.1.ja_dushu_shtopala
%%parent 18_11_2020
 
%%url https://www.facebook.com/groups/dmitrydonskoyfamily/permalink/4772038909503787/
%%author Наумчук, Таня
%%author_id tanya_naumchuk
%%tags poetry,ukraine
%%title Я душу штопала...
 
%%endhead 

\subsection{Я душу штопала...}
\label{sec:18_11_2020.fb.tanya_naumchuk.1.ja_dushu_shtopala}
\Purl{https://www.facebook.com/groups/dmitrydonskoyfamily/permalink/4772038909503787/}
\Pauthor{Наумчук, Таня}
\index[writers.rus]{Наумчук, Таня!Я душу штопала...}

\ifcmt
pic https://scontent-waw1-1.xx.fbcdn.net/v/t1.0-9/126156824_790036455187911_6770255381941909212_n.jpg?_nc_cat=107&ccb=2&_nc_sid=825194&_nc_ohc=Ac3ftch1hZEAX8voTOa&_nc_ht=scontent-waw1-1.xx&oh=c63bcaec5e867f801f1674659ef92d78&oe=5FDBFA3B
\fi

\begin{multicols}{2}

\obeycr
\lettrine[lines=4]{Я}{ душу штопала и шила...}
Она от боли вся рвалась на части.
Быть может я себе внушила,
Что горя нет, есть только счастье.

Я белой ниткой душу зашивала,
Она терпела все мои старания.
Но, только снова кровоточить стала,
То ли от боли, то ли от страдания.

Она просила: \enquote{Это не поможет!}
А, я -- упрямая, все шила и латала.
Таким вот способом, быть может,
Свою я душу  так спасала.

Она все кровоточила и ныла,
Просила изменить мои попытки:
-- Ты лучше бы меня вновь полюбила,
Зачем такие наносить мне пытки!?

Наверное была сама я виновата,
Когда уныние я в душу допустила.
И не спасла пришитая заплата,
Ее унынием разворотило.

Просьбам души своей внимая,
Один здесь выход может быть:
Ее всю нежность и ранимость понимая,
Жалеть нам надо свою душу и любить!

18.11.2020. г. (Наумчук Татьяна) 
(Фото из интернета)
\restorecr
\end{multicols}
