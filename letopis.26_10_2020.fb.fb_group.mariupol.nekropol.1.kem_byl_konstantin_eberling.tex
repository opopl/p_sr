%%beginhead 
 
%%file 26_10_2020.fb.fb_group.mariupol.nekropol.1.kem_byl_konstantin_eberling
%%parent 26_10_2020
 
%%url https://www.facebook.com/groups/278185963354519/posts/396501234856324
 
%%author_id fb_group.mariupol.nekropol,marusov_andrij.mariupol
%%date 26_10_2020
 
%%tags 
%%title А кем был Константин Эберлинг (Konstantin Eberling)?
 
%%endhead 

\subsection{А кем был Константин Эберлинг (Konstantin Eberling)?}
\label{sec:26_10_2020.fb.fb_group.mariupol.nekropol.1.kem_byl_konstantin_eberling}
 
\Purl{https://www.facebook.com/groups/278185963354519/posts/396501234856324}
\ifcmt
 author_begin
   author_id fb_group.mariupol.nekropol,marusov_andrij.mariupol
 author_end
\fi

А кем был Константин Эберлинг (Konstantin Eberling)? В субботу мы обнаружили
его могилу. Вроде бы из немецких колонистов...

\ii{26_10_2020.fb.fb_group.mariupol.nekropol.1.kem_byl_konstantin_eberling.pic.1}

В 1905 году в Мариуполе жил Эберлинг Адам Иванович. Он значится в списке
избирателей выборщиков в Екатеринославское губернское избирательное собрание.
Он попал в список, поскольку владел неким торгово-промышленным предприятием.
Имеет ли он отношение к Константину – непонятно...

Helga Buzlami, эксперт по генеалогии, встречала данные о фамилии Эберлинг в
колонии Грюнау (современная Розовка)...

Это все, что удалось узнать... Нужна помощь клуба! 🙂

Это уже третий немецкий колонист, которого мы нашли в Некрополе. Другие – это
\emph{Нарцисс Якобсон} и \emph{Эльза Курц}. Есть еще \emph{Валентин Фердинандович Мантель}, но его
отец был шлиссельбургским мещанином...

... а всего-то - Maryna Holovnova cтала расчищать заросли в древней части рядом с
Гофами и - вернула человека из небытия!

\href{https://www.facebook.com/shtambur}{Сергей Штамбур},\footnote{\url{https://www.facebook.com/shtambur}} может, Вы что-то встречали об Эберлингах в Мариуполе?

\#eberling \#эберлинг

%\ii{26_10_2020.fb.fb_group.mariupol.nekropol.1.kem_byl_konstantin_eberling.cmt}
