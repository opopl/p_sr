% vim: keymap=russian-jcukenwin
%%beginhead 
 
%%file 11_11_2021.fb.baumejster_andrej.kiev.filosof.1.200_let_dostoevskij
%%parent 11_11_2021
 
%%url https://www.facebook.com/andriibaumeister/posts/4426362324151959
 
%%author_id baumejster_andrej.kiev.filosof
%%date 
 
%%tags 200_let,chelovek,dostojevskii_fedor,dusha,literatura,rusmir
%%title 200 лет со дня рождения Достоевского
 
%%endhead 
 
\subsection{200 лет со дня рождения Достоевского}
\label{sec:11_11_2021.fb.baumejster_andrej.kiev.filosof.1.200_let_dostoevskij}
 
\Purl{https://www.facebook.com/andriibaumeister/posts/4426362324151959}
\ifcmt
 author_begin
   author_id baumejster_andrej.kiev.filosof
 author_end
\fi

200 лет со дня рождения Достоевского. 200 лет духовного влияния на умы ведущих
интеллектуалов и художников. И простых читателей, таких как мы с вами. 

Для Ницше, Эйнштейна, Хайдеггера, Гадамера, Бенедикта XVI, Кафки, Марселя
Пруста и многих других властителей дум ХХ века, Достоевский был одним из
главных учителей и вдохновителей. И, конечно, одним из главных оппонентов. 

\ifcmt
  tab_begin cols=3

     pic https://scontent-lga3-2.xx.fbcdn.net/v/t39.30808-6/255773199_4426251197496405_6661239033634282888_n.jpg?_nc_cat=103&ccb=1-5&_nc_sid=730e14&_nc_ohc=wgpqa71KWNgAX8aVb3k&_nc_ht=scontent-lga3-2.xx&oh=7e6f1b9be8c1604150a27fbca303cf58&oe=6192FD70

     pic https://scontent-lga3-2.xx.fbcdn.net/v/t39.30808-6/256167961_4426251710829687_2578236817146296715_n.jpg?_nc_cat=108&ccb=1-5&_nc_sid=730e14&_nc_ohc=hYhuBCLMvn4AX8_NHfz&_nc_ht=scontent-lga3-2.xx&oh=12ed711514e7fa0c952e4cf8e57481ce&oe=61918708

		 pic https://scontent-lga3-2.xx.fbcdn.net/v/t39.30808-6/255985461_4426252514162940_1718278462648625205_n.jpg?_nc_cat=111&ccb=1-5&_nc_sid=730e14&_nc_ohc=fHdMVRb8FiUAX_tfEt5&tn=lCYVFeHcTIAFcAzi&_nc_ht=scontent-lga3-2.xx&oh=b9b974a8c0134f00977b2a7e0903d04f&oe=61915495

  tab_end
\fi

Вот что говорил Гадамер (уже на исходе прошлого века): ""Карамазовы" были для
нас в 20-годы (а по свидетельству многих - и в последующие десятилетия - А.Б.)
важнейшей книгой после Библии". "Интерес к православию у всех нас возник через
чтение Достоевского. В "Карамазовых", особенно в черновых набросках к ним, мы
видели новую форму теологического письма, новую попытку апологии христианства". 

Эйнштейн говорил, что "Братья Карамазовы" - "Это самая поразительная книга из
всех, которые попадали мне в руки". 

\ifcmt
  tab_begin cols=3

     pic https://scontent-lga3-2.xx.fbcdn.net/v/t39.30808-6/256065314_4426253230829535_3866634954766790524_n.jpg?_nc_cat=103&ccb=1-5&_nc_sid=730e14&_nc_ohc=MBpWxavqByYAX8Rk2Vp&_nc_ht=scontent-lga3-2.xx&oh=593a2f28e4af5f1e68a139c64cf52eb1&oe=6192FFF2

     pic https://scontent-lga3-2.xx.fbcdn.net/v/t39.30808-6/255827230_4426254200829438_808496822362392554_n.jpg?_nc_cat=101&ccb=1-5&_nc_sid=730e14&_nc_ohc=Bk86qsQozUUAX8pnGiu&_nc_ht=scontent-lga3-2.xx&oh=c9abaa1f73118fb8f654cafb67686cbd&oe=61923E47

		 pic https://scontent-lga3-2.xx.fbcdn.net/v/t39.30808-6/247852085_4426254530829405_9014268942036801834_n.jpg?_nc_cat=108&ccb=1-5&_nc_sid=730e14&_nc_ohc=KgseJFjuUj8AX96X5Qa&_nc_ht=scontent-lga3-2.xx&oh=99ff29b8427e25b1536e22c2ec7b489d&oe=61931315

  tab_end
\fi

Ницше проводил аналогии между атмосферой евангельских событий и атмосферой
романов Достоевского. 

Мне кажется, что чтение "Братьев Карамазовых" и "Преступления и наказания" -
важнейшее событие в жизни каждого мыслящего человека. И тем более удивительно,
какое расстояние между Достоевским-писателем, Достоевским-мыслителем и
Достоевским-человеком. Расстояние, преодолеваемое в неизвестных и невидимых для
внешнего взгляда измерениях и глубинах... 

\ifcmt
  tab_begin cols=2

     pic https://scontent-lga3-2.xx.fbcdn.net/v/t39.30808-6/256344457_4426254744162717_1463255169652566504_n.jpg?_nc_cat=107&ccb=1-5&_nc_sid=730e14&_nc_ohc=GL7MTM7CLRAAX-md4cG&_nc_ht=scontent-lga3-2.xx&oh=1df543d36e80da571dce18c03cf07328&oe=61920E77

     pic https://scontent-lga3-2.xx.fbcdn.net/v/t39.30808-6/255766431_4426255307495994_2089665774546114624_n.jpg?_nc_cat=106&ccb=1-5&_nc_sid=730e14&_nc_ohc=Fk56BC359cMAX93MaGC&_nc_oc=AQn3-lnpPlA1dToEqbOjcEgG7-6F6kBR8wLgCuektux94uaURkyEsOlBtpmbSVki8kk&tn=lCYVFeHcTIAFcAzi&_nc_ht=scontent-lga3-2.xx&oh=acfb3692f5a871040cf8e5598e69948f&oe=6192DBCD

  tab_end
\fi

Гениальные романы и глубочайшие идеи создавались человеком болезненным,
мнительным, инфантильным, в чем-то даже мелочным. Человеком со всеми слабостями
и пороками. Почитайте воспоминания о Достоевском. Почитайте его переписку с
женой, Анной Григорьевной. 

Вот Достоевский приезжает в Москву к издателю Каткову, просить деньги на новый
роман ("Братья Карамазовы"). После первого посещения не спит всю ночь, думает -
даст или не даст Катков просимую сумму. Катков деньги дает. Но на этом мучения
и страхи не заканчиваются. Потом Достоевский размышляет, ехать домой к Каткову,
чтобы поздравить того с днем рождения или не ехать (Катков его на обед не
пригласил и Достоевский боится показаться заискивающим и выпрашивающим
милостыни). Затем решается ехать, прекрасно принят в кабинете хозяина дома
(прежде побывав в комнате жены Каткова), восторженно пишет своей жене, что с
ним там же, у Каткова, любезно разговаривает московский генерал-губернатор
князь Долгорукий (тот старичок, которого играл Табаков в "Статском советнике")
и даже ему первому (!!!) подает руку, поблескивая четырьмя звездами и алмазом
Андрея Первозванного. Потом Федор Михайлович еще раз (ему сказали накануне) с
облегчением отмечает, что званый обед у Каткова только для родственников и
поэтому Катков его на обед не пригласил не потому, что не уважает, а потому
что... (и т.д. и т.п.). 

Наконец, выходя из кабинета Каткова и видя накрытый праздничный стол он еще раз
"проверяет", что обед только для родных: "К Софье Петровне (жене Каткова) я уже
не заходил, а прошел другим ходом, между прочим (ну да, так мы и поверили -
А.Б.) через столовую, и заметил (так для того и прошел "между прочим" - А.Б.),
что стол накрыт не более как на 20 или даже 18 кувертов (столовые приборы -
А.Б.). А так как только Каткова семейство садится за стол не менее 12 человек
(настоящий Шерлок - А.Б.), то я и заключил, что званого обеда никакого нет, а
обедают лишь ближайшие родственники" (письмо А.Г.Достоевской от 9.11.1878). 

Таков Достоевский, вынашивающий идеи гениального романа и ревностно считающий
количество приборов на столе в доме Каткова... Дорогой Федор Михайлович,
спасибо Вам за все! Без Ваших романов и Ваших слабостей я бы не стал тем, кем я
есть на данный момент. Спасибо...
