% vim: keymap=russian-jcukenwin
%%beginhead 
 
%%file 27_12_2021.fb.fb_group.story_kiev_ua.2.novyj_god_malchik.pic.1.cmt
%%parent 27_12_2021.fb.fb_group.story_kiev_ua.2.novyj_god_malchik
 
%%url 
 
%%author_id 
%%date 
 
%%tags 
%%title 
 
%%endhead 

\ifcmt
  ig https://scontent-frx5-1.xx.fbcdn.net/v/t39.30808-6/270119408_977521969504172_1393287763250938996_n.jpg?_nc_cat=100&ccb=1-5&_nc_sid=dbeb18&_nc_ohc=_i3rVD8tWHkAX_dLx0Z&_nc_ht=scontent-frx5-1.xx&oh=00_AT9BRdHgSSN5wm3xuTEeZFOgffqr1FVX5JKWgQ00rRx9qA&oe=61D4EEA0
  @width 0.3
	@caption_begin
		Листівка, видана у 1957 році. Цікавий сюжет: зустріч двох років – старого і
		молодого. Постарівший 1957-й передає як новорічну естафетну паличку юному
		1958-у загальновідомий у той час символ миру – голубку Пікассо.
	@caption_end
\fi

\iusr{Олексій Горєлов}

Хлопчик несе мир до Будапешту в складі 38ї армії Прикарпатьского ВО та інших
частин і з'єднань
