% vim: keymap=russian-jcukenwin
%%beginhead 
 
%%file 03_10_2021.fb.medvedeva_olesja.1.uchitel_chelovek_podnevolnyj.cmt
%%parent 03_10_2021.fb.medvedeva_olesja.1.uchitel_chelovek_podnevolnyj
 
%%url 
 
%%author_id 
%%date 
 
%%tags 
%%title 
 
%%endhead 
\subsubsection{Коментарі}

\begin{itemize} % {
\iusr{Андрей Мошненко-Высоцкий}
Я учился по учебникам начала 2000. Это была жесть особенно про ВОВ и УПА (УПА герои, советские солдаты отребье и захватчики).

\begin{itemize} % {
\iusr{Олеся Медведева}
\textbf{Андрей Мошненко-Высоцкий} я тоже училась по таким, но в меня там был баланс

\iusr{Natali Romanovsky}
\textbf{Андрей Мошненко-Высоцкий} Я училась в 1993 году и там уже это было! За эту четверть у меня неуд.

\iusr{Тарас Албуров}
Держитесь, скоро этот весь бред закончится.

\iusr{Алексей Вахненко}
\textbf{Тарас Албуров} это закончится только для тех, кто покинет страну, для остальных это навсегда.

\iusr{Тарас Албуров}
\textbf{Aleksey Vakhnenko} , та нет покинут ее те кто против нас а мы останемся

\iusr{Dima Yu}
\textbf{Тарас Албуров} к сожалению пока усугубляется. 2 года назад надеялись, что закончится и выбрали хер знает кого...и все несется еще ниже дна

\iusr{Тарас Албуров}
\textbf{Dima Yu} , не много терпения и все будет, хорошо

\iusr{Марк Зоряний}
\textbf{Natali Romanovsky} к тебя вся жизнь неуд

\iusr{Natali Romanovsky}
\textbf{Марк Зоряний} как скажешь

\iusr{Марк Зоряний}
\textbf{Тарас Албуров} тебя кремлебот не будет и твой бред закончиться

\iusr{Alexandr Razumey}
\textbf{Алексей Вахненко} -"кто покинет страну,"
- все умные уже уехали, остались самые умные...  @igg{fbicon.smile} 
\end{itemize} % }

\iusr{Виталий Голубушкин}

Вы абсолютно правы. Я постоянно интересуюсь подачей истории в школе и
подправляю своей внучке знания, а иногда и исправляю. Благо у меня есть
собрание сочинений, точнее лекций по истории Ключевского, История дипломатии.
Иначе у детей будут полностью перевёрнуты понятия о прошлом.

\begin{itemize} % {
\iusr{Олеся Медведева}
\textbf{Виталий Голубушкин} школьная оценка - ничто. Формирование мировоззрения - все

\iusr{Виталий Голубушкин}
\textbf{Олеся Медведева} Мировозрение формируется в первую очередь на фактах. Вопрос кто и как их преподносит, а также раскрывает или наоборот скрывает истинные причины присходящего.


\iusr{Олеся Медведева}
\textbf{Виталий Голубушкин} это хотелось, чтобы так было, но на самом деле мировоззрение формируется окружением, авторитетами, эмоциями. Факты дети воспринимают только через обложку: игровую форму, эмоции и впечатления. Мировоззрение формируется в детстве

\iusr{Виталий Голубушкин}
\textbf{Олеся Медведева} Если рассматривать мировозрение как восприятие окружающего мира - вы правы. Однако, согласитесь, что оно меняется в течении всей жизни под влиянием многих факторов. Пониманием ошибочности собственных взглядов и приоритетов, а также теорий, усвоенных образованием.


\iusr{Олеся Медведева}
\textbf{Виталий Голубушкин} меняется, но на это способны, к сожалению, не все

\iusr{Виталий Голубушкин}
\textbf{Олеся Медведева} Про таких говорят:"С кастрюлькой на голове". Плохо когда такие приходят во власть. Они не видят своих ошибок и не в состоянии реально оценить ситуацию.

\iusr{Марк Зоряний}
\textbf{Виталий Голубушкин} перевернутые понятия у тебя совковые и русскомирские. Они детям не нужны

\iusr{Маргарита Яровая}
\textbf{Марк Зоряний}, ну а xyле нам, славаyкpаинцам, да, Маркуша? @igg{fbicon.face.tears.of.joy} 

\ifcmt
  ig https://scontent-mia3-2.xx.fbcdn.net/v/t1.6435-9/244358627_288035932996394_2934424646871539357_n.jpg?_nc_cat=105&_nc_rgb565=1&ccb=1-5&_nc_sid=dbeb18&_nc_ohc=VyLpuIUSgCIAX-bixiJ&_nc_ht=scontent-mia3-2.xx&oh=a765902ce8ed97c67c0fe2e6f5df78d7&oe=6183B1B0
  @width 0.4
\fi

\iusr{Марк Зоряний}
\textbf{Виталий Голубушкин} 

и Ключевский это имперский историк не может быть даже на 1\% объектным и тем
более быть историком Украины. И при чем здесь история дипломатиии @igg{fbicon.face.tears.of.joy}{repeat=4} 

\iusr{Марк Зоряний}
\textbf{Маргарита Яровая} 

может ты и холопка, а свободный человек в свободной стране. Перепортила бред из
методички для кремлеботов

\iusr{Виталий Голубушкин}
\textbf{Марк Зоряний} 

В истории дипломатии международные договора, которые двигали политику
государств. К примеру Москва в 1686 году выкупила Киев у поляков за 7 тонн
серебра. ("Вечный мир» — договор России с Речью Посполитой, заключенный 26
апреля (6 мая) 1686 г. в Москве по инициативе польского правительства). В
истории дипломатии много интересных исторических фактов, о которых не говорят
учебники.

\end{itemize} % }

\iusr{Nikolay Lanovoy}

Ну, судя из определения термина "коррупция", коррупционерами являються
абсолютно все представители власти, даже те, кто теоретически, не ворует, не
требует взяток и не берет их ))) Замечательное, определение!

Интересны также причины майдана. Как бы это коррупция. Но из текста можно
сделать вывод, что коррупцию можно было бы и дальше терпеть, но вот влияние
России и неподписание Ассоциации терпеть было нельзя )))

Я бы не только эту тему разрешил не учить ребенку; всю историю, преподаваемую в
школах, можно разрешать не учить...

\iusr{Ксения Павлушко}

а я вообще покинула ряды гуманитариев, ушла в народ, причём порою на самое дно
- по той же причине. А ведь могла бы карьеру провластной журналистки забацать!
Зато душа спокойна и совесть не мучает...

\iusr{David Aprasidze}
Противно читать эту гнусную ложь.

\iusr{Алексей Иванов}
Прочитал учебник, брехня написана. Мешанина из лжи, правды, фактов и неверных выводов. Надеяться на адекватность учителей? Возможно.

\iusr{Anton Koval}
Олеся, сыну такое не советуйте, пожалейте. Научите его лучше держать дулю в кармане до лучших времен, до участия в выборах хотябы...
\end{itemize} % }
