% vim: keymap=russian-jcukenwin
%%beginhead 
 
%%file 24_02_2023.fb.fb_group.ja_zhivoj_mariupol.1.dnevnik
%%parent 24_02_2023
 
%%url https://www.facebook.com/groups/2850411311922693/posts/3103950029902152
 
%%author_id fb_group.ja_zhivoj_mariupol
%%date 
 
%%tags 24.02.2022,dnevnik,mariupol,mariupol.war
%%title Мой дневник, который я начал вести 24 февраля 2022 года
 
%%endhead 
 
\subsection{Мой дневник, который я начал вести 24 февраля 2022 года}
\label{sec:24_02_2023.fb.fb_group.ja_zhivoj_mariupol.1.dnevnik}
 
\Purl{https://www.facebook.com/groups/2850411311922693/posts/3103950029902152}
\ifcmt
 author_begin
   author_id fb_group.ja_zhivoj_mariupol
 author_end
\fi

Мой дневник, который я начал вести 24 февраля 2022 года.

Я так и не смог его прочесть до конца, слезы и ком в горле мешают мне до сих
пор.

Я живой, это первые слова которые разрывали меня когда Я 16.03.22 выезжал из
Мариуполя.

👉Но я решил выложить свои воспоминания здесь.

▶️Предисловие.

Казалось ли вам когда-нибудь, что Господь вас испытывает с целью подготовить 
к следующему жизненному этапу? 

Эти испытания порой бывают трудными, однако они просто необходимы для духовного роста.

☝️ Даже апостол Павел рассказывает, что 

его испытывал Бог. Причём пишет об этом апостол такими словами, которые чётко
дают 

понять: некоторые из испытаний были поистине огненными. Так, в Первом послании
к Фессалоникийцам 2:4 Павел говорит: «Но, как Бог удостоил нас того, чтобы вверить
[нам] благовестие, так мы и говорим, угождая не человекам, но Богу, испытующему сердца
наши». 

Мариуполь, город герой!

Глава первая.

⬇️

Утро 24 февраля, четверг, 5 часов утра, началось не так как все предыдущие
утра.

Вместо прогулки с собакой,  тревога, Россия нанесла удар по украинским, военным
объектам на всей территории Украины от запада до востока, от севера к югу.
Вторглась на территорию Украины захватив Чернобыльскую, Запорожскую АЭС и
пустили воду в Крым через канал.

Так началась военная операция или просто закончилась спокойная и размеренная
жизни и началась война. По новостям сообщили, что захватчики начали
обстреливать Киев, Харьков, Мелитополь и подошли к  Бердянску и Запорожью.

В магазинах начались очереди за продовольствием и водой, на заправках собрались
очереди в  десятки и сотни машин, на некоторых заправках ограничили продажу
топлива до 20 литров в сутки в одни руки.

Эвакуироваться из Мариуполя мы не успели, поэтому всей семьёй поехали к
сестре по линии жены, она жила не далеко от нас в своем доме, где мы могли
переждать некоторое время до того как откроют зелёный коридор в сторону
Запорожья.
