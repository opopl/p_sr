% vim: keymap=russian-jcukenwin
%%beginhead 
 
%%file 08_12_2021.tg.tkachev_jurij.1.vtorzhenie_putin_usa.cmt
%%parent 08_12_2021.tg.tkachev_jurij.1.vtorzhenie_putin_usa
 
%%url 
 
%%author_id 
%%date 
 
%%tags 
%%title 
 
%%endhead 
\subsubsection{Коментарі}

\ifcmt
  ig https://i2.paste.pics/2156ebd5fe0b7d3b676930279b757062.png
  @width 0.4
\fi

\begin{itemize} % {
\iusr{Ruslan}

Допустим, не только Путина. Но да, Путин один из главных защитников Украины.
Без Путина Украины давно уже не было бы.

\iusr{Вячеслав Шмарин}

Так наоборот, весь запад всеми доступными способами подталкивает Путина - ну
давай уже нападай, давай захвати ... Это и есть основная идея. Потому что
противостояние уже идет , но оно идет на политическом, экономическом  и
идеологическом фронте. И Штатам очень надо завязать нас на конкретике, чтобы
все остальные фронты были оголены.

\iusr{Kіевлянинъ}

Они этого не понимают. Политические украинцы до сих пор считают США нашими
партнёрами и братьями. У этих людей мозг испарился и пустая черепная коробка
которая рефлексирует на клич Слава Украини. За эти 8 лет народ деградировал и
оскотинился в край. Плюс пропаганда по ТВ и интернет СМИ помогает им верить в
параллельную реальность где Украина воюет с РФ и ее уважают в мировом масштабе.
Мои родители живо тому пример.

\iusr{Виталий Куртов}

Я вообще не понимаю в какой стране \enquote{Слуги}  живут) Но явно мы с ними в разных
Украинах живём) У них украинцев всё ещё 40 миллионов и против РФ партизнанить
будут старики, младенцы, жители Крыма и Донбасса)

\iusr{Вячеслав Шмарин}
\textbf{Kіевлянинъ}

Не так все просто. В 90-е Россию тоже захватили штаты. Их люди сидели во всех
кабинетах. И чтобы выкинуть их оттуда надо было очень кропотливо их
выдавливать. Потому что открыто им противостоять было невозможно. Сейчас такая
же ситуация на Украине.

\iusr{Vladique86}

Худшее, чем он бы столкнулся в этом случае - \enquote{экономические санкции}.

Эх, Юра-Юра! Как ты мог, о циник, забыть о КОПРОТИВЛЕНИИ, которое окажет в
едином порыве население чудо-государства? Как будут отчаянно бросаться на
\enquote{вторженцев} (правда, тех ещё уговорить надо вторгнуться, лол) наши земляки -
после многократно возросших тарифов, отката \enquote{активистам} с бизнеса  и битв за
урожай, украинизации, блокпостов \enquote{лиц без опознавательных знаков}, замерзающих
квартир и мега-пенсий! 

Как ты мог, о маловерный, усомниться в отчаянной борьбе за идеалы Майдана - то
есть за сайт \enquote{Миротворец}, пятилетние сроки за герб СССР на Facebook, Роттердам
Плюс и сгоревшие в банкопаде сбережения?

\iusr{Yury M}

Та ладно!!!? Как так то?? мы ж самую сильную армию в Европе создали! Мы же.. Мы
же сдерживаем российские войска, защищаем там всех в Европе от злого Путина!
вот.

\iusr{Александра Кормалева}

У меня складывается такое впечатление, что это игра штатов! Они своей игрой
толкают Украину на войну! И об этом знает Зеленский! Чем это может закончится
страшно и подумать! Но и бесконечно жить в подвешенном состоянии долго
продолжаться не может! Только один путь будет верным, это диалог Киева и
Донбасса!  Все посредники хотят что-то своего! Так что САМИ!

\iusr{Roma}

Нет, Западу нужно, чтобы Путин не мешал строить Антироссию. И это она нападет,
когда удобный момент возникнет. А пока её к этому готовят.

\iusr{motor unknown}

да всё будет хорошо, спите  спокойно)

\iusr{Вячеслав Шмарин}

Это вы оцениваете по западным меркам - выгодно/невыгодно. А Россия и Путин
живут по меркам справедливости. А если с этой точки зрения смотреть, то все
несколько сложнее. С одной стороны конечно и народ несет ответственность за то
что творится. Но далеко не полную.

\iusr{Вячеслав Шмарин}

Чтобы строить Антироссию - ее надо содержать, кормить и обогревать. А у запада
уже на себя денег не хватает. Поэтому он хочет чтобы тут все подрались, а запад
подливал бензинчику в огонь.

\iusr{MORGENSHTERN}

Вряд ли политические украинцы читают ваш блог.

\iusr{Александр}

\url{https://t.me/Shulyavskiy/190}

\obeycr
Не помню, но кажись, в тринадцатом году,
Транзитом Зрада Киев проезжала,
Дай, думает, по городу пройду,
Бывала раньше, но конкретно не гуляла,
Короче, ходит, смотрит, тратит наличман,
Красивый город, но видала интересней,
И Зрада выбрела, случайно, на Майдан,
А тот встречал ее лампадками и песней,
Что с нею сделалось, пером не описать,
Когда увидела на площади всю шоблу,
Бомжи, жлобы, на сцене мразей рать,
И все пропитано печалью, грустью, болью,
Пришлось уехать, много дел, и в том беда,
Но не прикажешь - не имеет сердце меры,
Вернулась Зрада, чтоб остаться навсегда,
И, кстати, сделала отменную карьеру...
\restorecr

\iusr{Telema}

0.01\% бандеровцев которые будут бухать в крыивках за деньги Запада. Вот какая
партизанская война ждет Путина. Но у Путина нет возможности. Тянуть всех это
придётся в социализм уйти.

\iusr{Roma}

Они строят. Каждый день это наблюдаю. И готовят её к войне.

\iusr{Telema}
И это плохо. Потому что я за дружбу, равенство, братство!

\iusr{Roma}
А кто будет разбираться, когда все ресурсы в руках нацистского государства?

\iusr{Вячеслав Шмарин}

Если мы считаем что народ братский, значит надо помогать. Но чтобы помогать
народ сам должен об этом заявить.

\iusr{Roma}

Да как бы Янукович в феврале 2014 года заявил. С нулевым результатом.

\iusr{Сергей Ившин}

\ifcmt
  ig https://i2.paste.pics/bab21fcf95151198379d570ca9588426.png
  @width 0.4
\fi

\iusr{Вячеслав Шмарин}

Идея неверна, долги надо набирать. И потом уже послать всех подальше. ПОтому
что прецеденты есть, например Великобритания подрезала деньги Венесуэлы. Так
что набирайте долгов и кредитов  - это пойдет западу на пользу, в другой раз не
будут лезть.

\iusr{Roma}

Народ? Как вы себе это представляете, всеукраинское Вече? Народ свою власть
через представительные органы власти осуществляет. Президент им и является.

\iusr{Telema}

1. Кому вам? А вам не надоело олигархов кормить?

2. То есть вы не считаете украинцев с русскими единым народом? Окей, критерий
различия?

\iusr{Юри Ткачёв}

куда писать? на чьё имя и адрес? от каждого представителя народа надо или хотя
бы от большинства представителей?

\iusr{Вячеслав Шмарин}

Нет ни разу не является. Все знают что реальной демократии нет и выбирают того
кого проплатили и подрекламировали. А вот когда народ заявит своим права и свое
желание мы обязательно увидим.

\iusr{Esve}

Каким образом заявить?

\iusr{Esve}
))))

\iusr{Кот в пальто}
Слава Путину и Украине тоже @igg{fbicon.face.smiling.eyes.smiling} 
Или так, Слава Путину, а потом уже Украине @igg{fbicon.beaming.face.smiling.eyes} 

\iusr{Roma}
И Путина в том числе?  @igg{fbicon.face.flushed}  А я догадывался.

\iusr{Александра Кормалева}

Юра, не дай Бог произнести имя Путина! Да еще в контексте спасителя! А агрессию
куда деть? Будут же друг на друга бросаться! Свидомые они такие!

\iusr{Telema}

Уже заявляет, все чаще. Но без поддержки народ задавят. Да, только начинается
процесс профсоюзов, их мало, но они уже есть! Но без поддержки нереально
сбросить ярмо Запада.

\iusr{Виктор}
\textbf{Kіевлянинъ}

Не согласен. Старшее поколение 60+ может ещё и верит в эту галиматью, но очень
многие из тех кто верил в Майдан, сейчас открыто его называют переворотом, и
перестали верить в искренность Запада

\iusr{Вячеслав Шмарин}

Старая испытанная схема. Подполье, группы, консолидация, и уже когда
действительно будет массовость - выступление. Сейчас власть слаба как никогда,
любой сильный толчок и посыпется. И дальше будет только слабеть, потому что
держится за счет штатов, а они сами слабеют. Поэтому и войну не надо начинать,
власть сама осыпется как в Афганистане.

\iusr{Вячеслав Шмарин}

да все верно. своих сбросить сможете, а вот запад отогнать только с помощью
России. Но Россия поддержит, не сомневайтесь

\iusr{Юри Ткачёв}
\textbf{Вячеслав Шмарин}
и всё-таки вы не уточнили, как именно должно выглядеть обращение? Расскажите всё-таки

\iusr{Ivan Petrov}
\textbf{Виктор}

\enquote{Искренность Запада} - это оксюморон. Запад столетиями весь мир грабит под
сказки о приносимой им цивилизации, свободе и демократии.

\iusr{Roma}
\textbf{Вячеслав Шмарин}
Да как-то сомневаемся после 2014 года.

\iusr{Artyom}
\textbf{Юрий Ткачёв}

Само собой. Как однажды сказал Президент РФ: \enquote{Проснулся утром - подумай, что ты
сделал для Украины}

\iusr{Artyom}

Обращение Виктора Владмировича Медведчука к Владимиру Владимировичу Путину :)

\iusr{Telema}

Ну значит вы тоже в фашизм удариться хотите, как наши политические украинцы?
Так запросто! Только, сдаётся мне, вы просто бредите

\iusr{Юри Ткачёв}
\textbf{Вячеслав Шмарин}
именно деревни нужно? Города не годится?

\iusr{Artyom}
\textbf{Вячеслав Шмарин}

Юрий как-то описывал обращение в консульство РФ с просьбой поддержать
прорусское движение в Одессе, там ответили, что все подобные вопросы должны
решаться через Партию Регионов

\iusr{Юри Ткачёв}
\textbf{Artyom}

ну это давно было, 2011 год ещё, если не раньше

\iusr{Вячеслав Шмарин}
\textbf{Юри Ткачёв}

не именно, это же просто пример. Город тоже подойдет, только по началу в городе
будет опасно.

\iusr{Юри Ткачёв}
\textbf{Вячеслав Шмарин}

ОК, в Харькове сдлелали, вернули проспект Маршала Жукова. И где реакция?

\iusr{Artyom}
\textbf{Юри Ткачёв}
Теперь вместо Януковича Медведчук)

\iusr{Вячеслав Шмарин}
\textbf{Юри Ткачёв}

Пункт 2. Набрать статистику. Должно быть понятно что это мнение народа.

\iusr{Юри Ткачёв}
\textbf{Вячеслав Шмарин}

какую статистику? какой она должна быть?

\iusr{Artyom}
\textbf{Юри Ткачёв}

Дать власти набить фрагов. От нескольких сотен до нескольких тысяч

\ifcmt
  ig https://i2.paste.pics/b522e0c099b0dff7dd1901e4ace42284.png
  @width 0.4
\fi

\iusr{Вячеслав Шмарин}
\textbf{Юри Ткачёв}

Один эпизод показали. Покажите 100 эпизодов. ЭТо статистика

\iusr{Юри Ткачёв}
\textbf{Вячеслав Шмарин}

Ага, т.е. нужно 100 эпизодов. ОК. А не переименования обратно, а отказы переименовывать сойдут?

\iusr{Ivan Petrov}
\textbf{Вячеслав Шмарин}

Wishful thinking. Там у большинства населения в головах такого насрано, что ни
приведи Господи. Сперва бандеровское говно должно перебродить, а потом уже
видно будет.

\iusr{Вячеслав Шмарин}
\textbf{Юри Ткачёв}
Это уже сложно. ПОтому что тут неясно, может это бумага в канцелярии застряла

\iusr{Евгений}
\textbf{Юри Ткачёв}

Такая забавная агитка. Путин придёт, всех победит.

\iusr{Вячеслав Шмарин}
\textbf{Ivan Petrov}

Вот в том то и дело. Что раньше времени влезать это только лишний народ
положить.

\iusr{Telema}
\textbf{Вячеслав Шмарин}

Ясненько, москвич интересующийся криптонае.. вом решает как будут дружить
русские с украинцами, кормит он, вся Россия зае.. лась кормить вашу Москву.

\iusr{Ivan Petrov}
\textbf{Вячеслав Шмарин}
Именно.

\iusr{Esve}
\textbf{Юри Ткачёв}

Ну если \enquote{а почему русский Ваня должен воевать}, то и Джон тоже не должен.
Кредиты, оружие, инструкторы не считаются. Так что, конечно, не пришлют. 

Чтобы заслужить амерские войска (как и оккупацию)), нефть хотя бы нужна в
достойных размерах.

\iusr{Вячеслав Шмарин}
\textbf{Esve}

Между Ваней и Джоном большая разница. Джон воюет только за бабло, а Иван только
за идею.

\iusr{Esve}

Джон тоже за идею - демократию наносит

\iusr{Вячеслав Шмарин}
\textbf{Esve}

Надо понять что такое свобода и демократия. Чтобы много тут не писать глянь как
я это вижу \href{https://youtu.be/hn3p9s9dRkE}{%
природа русофобии, Шмарин Вячеслав, youtube, 31.08.2017%
}


\ifcmt
  tab_begin cols=2,no_fig,center
     pic https://i2.paste.pics/33099c4ffba8798fc7f1a9620fb0fc07.png
		 pic https://i2.paste.pics/9d95e0bd02b061f8460b314bfb584a42.png
  tab_end
\fi

Рассмотрение природы русофобии. В догонку 3-й части программы Владимира Соловьева от 31.08.2017 

\href{https://youtu.be/Cr2GYByLcPk}{%
Вечер с Владимиром Соловьевым от 31.08.17, youtube, 31.08.2017%
}

\ifcmt
  tab_begin cols=3,no_fig,center
		pic https://i2.paste.pics/d28a3741c69383f30696cd3c0c38e01c.png
		pic https://i2.paste.pics/b9632cc46a79cdc748695dcfdcb47e5d.png
		pic https://i2.paste.pics/4388dc6f4f9d3669af644b128cc6fb61.png
  tab_end
\fi

Почему нас ненавидит запад ?

Почему мы в ответ хотим дружить?

Почему нельзя сдаваться ?

В чем разница природы русского мира и западного?

\iusr{Ivan Petrov}
\textbf{Юрий Ткачёв}

Можно подумать, американские войска сильно помогли бы. С кем они когда воевали,
кроме папуасов? Одно дело каких-то полудиких вьетнамцев напалмом жечь, совсем
другое - иметь реальную возможность огрести по самое не балуйся.

\iusr{Anatolii Romanov}
\textbf{Юри Ткачёв}
С тех пор политика РФ сильно изменилась?

\iusr{Вячеслав Шмарин}
\textbf{Anatolii Romanov}

Конечно изменилась. Мы только с конца 13-го года отважились сами деньги
печатать. До этого только с разрешения США.  А сейчас мы уже доллар собрались
отменять. Есть разница ?

\iusr{Kіевлянинъ}
\textbf{Виктор}

Хотелось бы в это верить. Но увы, в Киеве у нас ещё огромное количество людей
верит, что Россия хочет воевать и надо воевать с Донбассом до конца. Среди
таких и мои родители, некоторые коллеги на работе (молодые люди до 35 лет). Я с
ними на такие темы не говорю ибо когда майдан переворотом назвал раз, то они
стали невероятно агрессивными.

\ifcmt
  ig https://i2.paste.pics/b559ae95433c3ef30d3e579fd2a5187f.png
  @width 0.4
\fi

\iusr{Вячеслав Шмарин}
\textbf{Kіевлянинъ}

Не надо идти на конфликт и спор. Для разрешения проблемы надо найти третью
независимую сторону и свалить все на нее. Тогда можно спокойно и без эмоций
попробовать все расставить по полочкам.

\iusr{Anatolii Romanov}
\textbf{Вячеслав Шмарин}

Неточно сформулировал вопрос.  Попробую по-другому.  Если бы сейчас
(теоретически)было бы обращение за поддержкой в консульство РФ от оппозиционных
нынешнему украинскому режиму, пророссийских движений,, была бы такая поддержка
оказана?  Или бы снова послали бы в...Партию Регионов, или что там сейчас
вместо неё?

\iusr{Вячеслав Шмарин}
\textbf{Anatolii Romanov}

Если маленькая группа людей напишет обращение. Надо ли лезть в другую страну и
там что то менять  ?  Нет конечно. Нужно точное понимание что это именно
массовое явление. Например почему мы ввели войска и поддержали Крым? Потому
что и без референдума было ясно что население за Россию. Если бы были сомнения
, точно не пошли бы.

\iusr{Constantine Y}
\textbf{Юрий Ткачёв}

украинцы чинять муть

\iusr{Юри Ткачёв}
\textbf{Вячеслав Шмарин}
нене, там чёткие решения с отказами от переименований

\iusr{Вячеслав Шмарин}
\textbf{Юри Ткачёв}
может и четкие, но непонятно сколько людей за этим стоит? может один чиновник?

\iusr{Вячеслав Шмарин}
\textbf{Юрий Ткачёв}

Т.е. событие должно быть публичным, с четкой ориентацией но без опасности для
участников и с четким пониманием что участников много

\iusr{Nikirurk}
\textbf{Юрий Ткачёв}

Да не нападет Путин на Украину. Зачем ему это надо? Ему Крым был нужен. Там
база ВМФ была и вообще кто владеет Крымом, тот владеет черным морем. Максимум
заберет ЛДНР и всё!!! А из остальной Украины они с Западом сделают нейтральное
государство и бросят на произвол судьбы. Мол сами зарабатывайте и кормитесь
сами, а мы типа не приделах! Будут время от времени какие-нибудь подачки кидать
и не более того!

\iusr{Петро Вибоширов}
\textbf{Nikirurk}

Я вам напомню, что вы общаетесь с людьми, которых немало расстраиваете утверждением, что Путин не нападет...

Каждый раз, когда я слышу от демократических" политиков о нападении Путина, я
надеюсь на чудо, но, к сожалению, закончится все тем, что я опять скажу: от,
бля, пиздаболы!

\iusr{Вячеслав Шмарин}
\textbf{Nikirurk}

Конечно не нападет. Но и бросать тоже нельзя. Во-первых потому, что там все
таки наши люди - славяне. Во-вторых потому, что если появляется территория без
нормальной адекватной власти, она быстро превращается в проблему для всех
соседей. И пример с Игил я думаю уже всему миру понятен. Поэтому придется
подождать пока США откажутся тащить и оплачивать это государство и потом
придется помогать как минимум восстановить там порядок, чтобы население смогло
само себя обеспечивать.

\iusr{Roma}
\textbf{Вячеслав Шмарин}
Так потом дороже будет.

\iusr{Вячеслав Шмарин}
\textbf{Roma}

Дороже/дешевле, это пусть запад считает. У нас другие мерки. Важно чтобы
общество уже переболело, чтобы не надо было бороться с внутренним
сопротивлением. А сообща мы очень быстро все поднимем. Если глянуть в историю,
сколько раз нам приходилось отталкиваться от дна ? от полной разрухи?  И чем
сложнее было, тем быстрее развивались. В этом наша суть. Простые задачи нам не
интересны.

\iusr{Roma}
\textbf{Вячеслав Шмарин}
Так платить РФ будет.

\iusr{Вячеслав Шмарин}
\textbf{Юрий Ткачёв}

Это очень относительно. Вот сейчас деньги дают США, а кто платит ? США их
только печатают, значит платят другие ? А РФ благодаря Украине сейчас получает
оплату за газ в три раза выше чем раньше. Кто платит?  Так что когда будет
ближе к началу, тогда и разберемся кто будет платить.

\iusr{Roma}
\textbf{Вячеслав Шмарин}

Общество само без доктора не переболеет. Число русофобов будет только
множиться. А потом они пойдут Кубань возвращать. Вы знаете, что в Киеве теперь
есть улица Кубанской Украины? Приучают и готовят.

\iusr{Вячеслав Шмарин}
\textbf{Roma}

Ничего, лечебное голодание и охлаждение должно взбодрить тело и голову. 

А оплачивать будут те кто сейчас разваливает, можно переписать представителей
стран которые сидят в наблюдательных советах. \enquote{А если не будут ? А если
не будут отключим газ!} (Бриллиантовая рука)

\iusr{Roma}
\textbf{Вячеслав Шмарин}

Опять Украина зимой замерзнет? Ну сколько можно, восьмой год уже!

\iusr{Вячеслав Шмарин}
\textbf{Roma}

Ну в этот раз по-моему как следует запланировали замерзнуть. Такого еще не было.

\iusr{Roma}
\textbf{Вячеслав Шмарин}

Читали про медианную зарплату в РФ и Украине? На Украине всего на \$100  меньше,
чем в РФ. При том, что душевой ВВП в два раза ниже. Где-то у вас пробоина.

\iusr{Roma}
\textbf{Вячеслав Шмарин}

Батареи теплые. Горячая вода в кране тоже есть. Да и погода пока плюсовая. Не
Воркута, слава богу.

\iusr{Вячеслав Шмарин}
\textbf{Roma}

Ну и хорошо. По крайней мере мы вам не желаем никаких катастроф. А уж если
случится холод так чтобы пошел на пользу. У нас также было в 90-е говорили: все
равно голодаем так хоть давайте за что-нибудь, выдвинем требования от нашей
голодовки .

\iusr{Constantine Y}
\textbf{Roma}

они Донецк вернуть не могут, какая Кубань

\iusr{Roma}
\textbf{Вячеслав Шмарин}

Катастрофа давно уже нас всех накрыла. Как сказал Михал Иваныч (владелец дворца
в Геленджике) - величайшая катастрофа ХХ века.

\iusr{Вячеслав Шмарин}
\textbf{Constantine Y}

Тут не в Кубани дело, а в голове. Рома говорит что болеть будут долго.

\iusr{Roma}
\textbf{Constantine Y}

Есть мнение что вернут зимой. Главное чтобы РФ не вмешивалась во внутренние
дела соседнего государства. Бидон сейчас над этим работает.

\end{itemize} % }
