% vim: keymap=russian-jcukenwin
%%beginhead 
 
%%file 21_11_2020.fb.volga_vasilii.1.maidan_godovschina
%%parent 21_11_2020
 
%%url https://www.facebook.com/Vasiliy.volga/posts/2758043724513042
 
%%author Волга, Василий Александрович
%%author_id volga_vasilii
%%author_url 
 
%%tags 
%%title МАЙДАН. ГОДОВЩИНА
 
%%endhead 
 
\subsection{Майдан. Годовщина}
\label{sec:21_11_2020.fb.volga_vasilii.1.maidan_godovschina}
\Purl{https://www.facebook.com/Vasiliy.volga/posts/2758043724513042}
\ifcmt
	author_begin
   author_id volga_vasilii
	author_end
\fi

По какой причине Порошенко сегодня будет праздновать седьмую годовщину
«Революции достоинства», я понимаю. Это его праздник. Каждый убитый на Майдане
человек приближал его к президентскому креслу, и после того, как людей было
убито достаточно, он президентом стал.

По какой причине лещенки, найемы, кивы и прочие мелкие политические насекомые
будут праздновать седьмую годовщину, я тоже понимаю. Семь лет назад у них не
было ничего, а сегодня у них есть ФСЁ! 

Но что будем праздновать мы, украинцы?

Спланированные убийства на Майдане, в Одессе, В Мариуполе, в Харькове?
Гражданское, религиозное, национальное, языковое и культурное противостояние?
Потерю части своей суверенной территории, как следствие этого противостояния?
Оккупацию Киева Галичиной? Трагедию Донбасса?

Или может быть эта их «революция» привела к тому, что лучшие сыны и дочери
украинского народа придя во власть стали служить украинскому народу, а не своей
мошне? Так нет же. Откровенные воры, бандиты, люди без малейшего образования и
просто психически больные люди сегодня занимают самые высокие государственные
должности. 

Именно они – Майданные лидеры – погрузили Украину в пучину коррупции, ранее
никогда и нигде еще в мире не виданную. Именно благодаря их «рвению»
остановлены фабрики и заводы и пять миллионов украинцев потеряли работу. Именно
благодаря их «умению» пенсии, зарплаты, социальные выплаты в долларовом
эквиваленте уменьшились в четыре раза, а коммунальные платежи возросли в десять
раз. 

Что же мы будем праздновать? Радость в чем? 

Мне говорят: «Благодаря «революции» мы стали бедные, но свободные».

Вона как.

\ifcmt
pic https://scontent.fiev6-1.fna.fbcdn.net/v/t1.0-9/125969181_2758045201179561_2506123313988078870_n.jpg?_nc_cat=102&ccb=2&_nc_sid=8bfeb9&_nc_ohc=nYRTNTAoCL4AX-JhGLS&_nc_ht=scontent.fiev6-1.fna&oh=853a746e676154df17f158625bcc6297&oe=5FECB19C
\fi

А чем сегодняшняя «бедная свобода» отличается от вчерашней «несвободы»? Тем,
что сегодня рабочий человек свободен от своего рабочего места? Или тем, что
пенсионеров освободили от необходимости носить тяжелые сумки из магазинов?

Сегодня им вообще в магазин ходить не обязательно. После уплаты коммунальных
платежей ни на продукты, ни на лекарства денег у них уже не остается. 

Так в чем же «свобода»? 

Не в том ли, что за несогласие с этим "майданным счастьем", сегодня можно
угодить в тюрьму или быть забитым насмерть "активистами"?

Не «годовщина» это. Поминки. Пошёл восьмой год, как  мы будем поминать некогда
успешное государство Украина, в котором  жили счастливые люди и строили планы
на жизнь. Семь лет назад мы бы ни за что не поверили, что мы способны вот так
убивать друг друга. С азартом и ненавистью. 

Семь лет прошло, как мы перестали быть людьми. Помянем.

\subsubsection{Комментарии}

\paragraph{Леонид Зарапа}
Кто же эту Украинскую власть себе выбрал? Кто за них голосовал? Где оно это
нормальное Украинское большинство?

\paragraph{Владимир Сохарев}

Одно могу сказать - все это результат отсутствия идеологической доминанты,
которая могла бы направлять развитие общества в конструктивное русло. И вопрос
обретения идеологии Россией, на деле, касается не одной только России,
поскольку вопрос этот, по сути своей, является вопросом выживания всего нашего
человеческого сообщества. А для России вопрос идеологии - это вопрос
осознанного обретения Бога, через понимание логики развития, в основе которой
лежит неизменное и всегдашнее стремление общества к Справедливости. Подчеркну
отдельно - Осознанного Обретения Бога. О чем и не устаю повторять раз от разу.
... .
