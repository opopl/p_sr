% vim: keymap=russian-jcukenwin
%%beginhead 
 
%%file 09_01_2022.stz.news.ua.depo.1.zaporozhie
%%parent 09_01_2022
 
%%url https://zp.depo.ua/ukr/zp/banka-gutalinu-shayba-lodoviy-maydanchik-u-kozhnomu-dvori-de-retro-zaporizhzhya-ganyalo-na-kovzanakh-201912071075663
 
%%author_id nedelina_olesja.zaporozhie
%%date 
 
%%tags 
%%title Банка гуталіну – шайба, льодовий майданчик – у кожному дворі: Як Запоріжжя ганяло на ковзанах
 
%%endhead 
\subsection{Банка гуталіну – шайба, льодовий майданчик – у кожному дворі: Як Запоріжжя ганяло на ковзанах}
\label{sec:09_01_2022.stz.news.ua.depo.1.zaporozhie}

\Purl{https://zp.depo.ua/ukr/zp/banka-gutalinu-shayba-lodoviy-maydanchik-u-kozhnomu-dvori-de-retro-zaporizhzhya-ganyalo-na-kovzanakh-201912071075663}
\ifcmt
 author_begin
   author_id nedelina_olesja.zaporozhie
 author_end
\fi

\begin{zznagolos}
Де запоріжці пів століття тому ставали на ковзани та чи справді зими тоді були
більш сніжними й холодними	
\end{zznagolos}

Цьогоріч \href{https://zp.depo.ua/ukr/zp/budinochok-svyatogo-mikolaya-ta-kovzanka-u-zaporizhzhi-na-rayduzi-vidkrilosya-novorichne-mistechko-fotoreportazh-202112191403954}{ковзанки просто неба} працюють у Запоріжжі на двох локаціях – на площі
Фестивальній, що перед будівлею обладміністрації, та на каскаді фонтанів
\enquote{Райдуга}. Портал Depo.Запоріжжя з’ясував, де \enquote{кувалися традиції}, тобто де
запоріжці каталися на ковзанах, коли за кригу відповідали не спеціальні
агрегати для охолодження, а природа.  

\ii{09_01_2022.stz.news.ua.depo.1.zaporozhie.pic.1}

Запорізькі зими й у XX сторіччі не завжди та недовго тішили містян
по-справжньому зимовою погодою: з морозом та гарним снігом. \enquote{Різні зими були,
бувалі зовсім теплі. Справжня зима, як і зараз, починалася у січні-лютому, у
грудні зазвичай снігу не було}, – згадує зими п’ятдесятирічної давнини місцевий
журналіст Олег Петренко. \enquote{Ну, може трошки холодніше було і, здається, вітри не
такі сильні}, – додає водій Віталій Вовченко.

Так чи інакше, а гарною зимовою погодою поспішали насолодитися усі містяни,
особливо, діти. Одною з найголовніших зимових розваг ставали ковзанки.

У 1950-х роках найпопулярніша запорізька ковзанка була на стадіоні \enquote{Металург},
нині на його місці \enquote{Славутич-Арена}. \enquote{Чув, що заливали там усе: і бігову
доріжку, і поле. Там же був пункт прокату ковзанів, буфет, \enquote{обігрівочна},
музика грала. Сам величезний каток умовно можна було розділити на дві частини:
бігові доріжки для спринтерів-ковзанярів і майданчик для вправ у фігурному і не
дуже фігурному катанні}, – розповідає запорізький краєзнавець Роман Акбаш.


Певний час існувала ковзанка й на нинішній площі Фестивальній. Саме на цій
ковзанці здобував у 1970-х перші дитячі перемоги та зазнавав перших поразок
нинішній заступник директора запорізького музею техніки Олександр Бобров. \enquote{Тоді
це ще була площа Жовтнева. Там заливали величезну ковзанку, встановлювали
кілька, три-чотири, хокейних майданчики й проводили змагання між школами. Я теж
брав участь. Іноді перемагали, іноді програвали. Завжди поверталися із синцями,
бо ж захисту ніякого не було}, – згадує Бобров.

Попитом у хокеїстів-аматорів користувався і майданчик за палацом спорту
\enquote{Юність}. \enquote{За \enquote{Юністю} був відкритий хокейний майданчик. Іноді на ковзанах
можна було покататися в самій \enquote{Юності}, коли там заливали лід на майданчику під
дахом}, – розповідає Акбаш.

В \enquote{Юності} тренувалися та грали професійні хокеїсти, саме вони \enquote{екіпірували}
для гри запорізьких хлоп’ят. \enquote{Бігали на змагання, пробивалися у перший ряд,
коли у гравців ламалися ключки, вони кидали їх нам. Ми брали, за допомогою
бинтів, епоксидної смоли лагодили їх та грали}, – розповідає заступник
директора музею техніки.

\ii{09_01_2022.stz.news.ua.depo.1.zaporozhie.pic.2}

Замість шайби хлопці 1960-1970-х використовували банки з гуталіном, які тягли з
дому. Гуталін був недешевим, як згадує Бобров, батьки хлопців сварили. Тож за
неписаним правилом, банку для гри приносили по черзі. Іноді брали порожні, в
які для ваги засипали вологий пісок. Запоріжець Олександр Кравець згадав також,
що замість шайби іноді, після певних маніпуляцій, використовували колеса від
дитячих машинок.

Великі ковзанки влаштовували у старій частині міста, нинішньому
Олександрівському районі. \enquote{У старій частині міста такі ковзанки заливали,
наприклад, на стадіоні \enquote{Локомотив} і в парку Франка – зараз це сквер навпроти
\enquote{машинки} (колишнього машінституту, нині університету \enquote{Запорізька політехніка},
– ред.). Щось на зразок ковзанки взимку робили на ставку в парку \enquote{Дубовий гай}.
Там діяв і пункт прокату ковзанів, лиж, санок}, – розповідає Акбаш.

На ковзанку \enquote{Локомотиву} ходив хлопцем кататися оператор запорізького
телеканалу \enquote{Алекс} Володимир Отрішко. \enquote{Біля \enquote{Локомотиву} була велика ковзанка.
На прокат можна було брати ковзани й за невеликі гроші кататися. Здається, там
навіть музика грала. Людей приходило чимало}, – згадує Отрішко.

У Шевченківському районі найпопулярніша ковзанка була на стадіоні \enquote{Стріла}
заводу \enquote{Мотор Січ}. Про неї згадує співробітниця підприємства Надія Щербань,
яка ганяла тут на ковзанах на початку 1970-х. \enquote{Заливали її, коли морози були.
Ми після школи з подружками туди бігли, кидали портфелі й каталися. Людей
каталося багато. Музики там чи чогось такого не було, але нам подобалося, було
дуже весело}, – каже Надія.

У Комунарському районі мала попит ковзанка біля басейну \enquote{Байкал}. \enquote{Там поруч
встановлювали ще й дитяче містечко на час зимових канікул}, – розповідає Акбаш.

Для жителів Правого берега, принаймні тих, хто жив ближче до Дніпра, кращою
ковзанкою ставало водосховище перед греблею. \enquote{Від Правобережного пляжу
завжди добре замерзало. Крига була гладенька, рівненька. Там каталися й на
ковзанах, й просто на своїх двох, я, наприклад. Батьки мені ковзани купили, але
вони були на кілька розмірів більші, ніж потрібно, тому мені не подобалися}, –
розповідає журналіст Петренко.

Загалом же ковзанок не бракувало, бо їх заливали чи не у кожному запорізькому
дворі. \enquote{У 1970-х-1980-х це робили робітники \enquote{жеків}. Ковзанки
заливали й там, де п’ятиповерхівки, й новому районі. У Хортицькому житломасиві
ковзанки були з огородженням}, – розповідає запоріжець Вовченко. Нерідко
ставали ковзанками та хокейними майданчиками для дітлахів й дороги. \enquote{На
вулиці Сталеварів машин тоді майже не було, ми ганяли там}, – каже Бобров.
\enquote{У нас у дворі була власна ковзанка. Це – вулиця Гоголя, будинок 126.
Двір тоді був закритий, непрохідний, самі заливали й каталися. Місця було
небагато, але нам вистачало}, – згадує Отрішко.
