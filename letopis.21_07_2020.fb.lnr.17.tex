% vim: keymap=russian-jcukenwin
%%beginhead 
 
%%file 21_07_2020.fb.lnr.17
%%parent 21_07_2020
 
%%endhead 
\subsection{Киев выходит из Минских соглашений: России все-таки придется делать выбор}
\label{sec:21_07_2020.fb.lnr.17}
\url{https://www.facebook.com/groups/LNRGUMO/permalink/2863805737064315/}
  
\vspace{0.5cm}
{\small\LaTeX~section: \verb|21_07_2020.fb.lnr.17| project: \verb|letopis| rootid: \verb|p_saintrussia|}
\vspace{0.5cm}

\index{Минские Соглашения}

Прошлая неделя стала знаковой в долгом, многотрудном и, стоит признать
очевидное, абсолютно бесперспективном процессе «мирного урегулирования на
Донбассе», проводимого в рамках «Минских соглашений». Официальный Киев, по сути
дела, решился на полную денонсацию этих соглашений --- причем сделано это было не
в виде очередных провокационных заявлений дипломатических представителей
«нэзалэжной» или иных тамошних высокопоставленных лиц, а на законодательном
уровне.

Это, впрочем, прекрасно совмещалось с рядом новых русофобских демаршей в
исполнении украинских политиков, вплоть до главы государства… Что на самом деле
означают принятые в Киеве не допускающие двойного толкования решения для
жителей Донбасса и для России?

Никаких выборов, никакого мира, никакого «Минска»…

Прежде всего, речь в данном случае идет о принятии Верховной Радой Украины 15
июля постановления №3809, касавшегося проведения в «нэзалэжной» местных
выборов. Этот «праздник демократии» состоится 25 октября нынешнего года --- но
только не в Республиках Донбасса, именуемых в Киеве мерзкой аббревиатурой
«ОРДЛО». Казалось бы --- имей украинская сторона хоть малейшие намерения
следовать собственным обещаниям, подписывавшимся в свое время в столице
Белоруссии и неоднократно подтверждавшимся на встречах лидеров «Нормандского
формата» --- вот он, прекрасный случай сделать хоть шаг к их выполнению. В
соответствующих соглашениях черным по белому написано: сперва внесение в
Конституцию особого статуса для Донбасса, затем проведение там местных выборов
новых органов власти. Только после этого можно будет вести речь о «передаче
границы» и тому подобных вещах, к которым так стремится Киев.

Тем не менее украинские парламентарии, похоже, совершенно сознательно пошли на
то, чтобы разорвать эти договоренности, внеся в собственное постановление пункт
относительно того, что на земле Донбасса никаких выборов не будет ровно до тех
пор, пока «Россия не выполнит пять обязательных условий». Фактически –
ультиматум, предельно наглый и, к тому же, составленный явно впопыхах и
совершенно безграмотно. Так, пункт о выводе с неподконтрольной Киеву территории
привидевшихся ему там «наемников, представителей незаконных вооруженных
формирований» и, что самое главное, «российских оккупационных войск со всей
техникой», отчего-то повторяется в тексте дважды. Очевидно, тугодумы мерят всех
по себе… Еще речь идет о «восстановлении правопорядка и конституционного
строя». И «порядок», и «строй», естественно, подразумеваются исключительно
украинские.

При этом наименее бредовым (но при этом, опять-таки, совершенно невыполнимым) остается пункт о «полном контроле над государственной границей», который должен быть передан украинской стороне. На этом фоне не так уж дико смотрятся даже слова о «временной невозможности провести выборы в оккупированном Крыму» и требования «обеспечить безопасность населяющих его граждан Украины». Тут все просто и ясно --- это диагноз и его уточнение мы оставим медикам. Но вот любой юрист (особенно --- в области международного права) однозначно скажет: подобного рода выверты, совершаемые высшим органом законодательной власти страны, имеют четкое и однозначное определение. Это фактически полная денонсация «Минских соглашений», теряющих в той системе координат, которую пытается задать парламент «нэзалэжной», и суть, и смысл, и цель.

Собственно говоря, ничего неожиданного не происходит. В последнее время Киев чуть ли не криком кричит о невозможности выполнения спасшего его в свое время от окончательного военного разгрома «Минска», поскольку тот-де «не соответствует нынешним реалиям». Именно такое заявление прозвучало дней десять назад из уст вице-премьер-министра Украины, являющегося также вторым человеком в украинской делегации на Минских переговорах Алексея Резникова. При этом он нес какую-то, простите, галиматью о якобы «захваченных Россией» у «нэзалэжной» «тысячах квадратных миль земли» и призывал Запад всемерно посодействовать «пересмотру и адаптации Минского процесса к новым реалиям». Ну, о настоящей реакции Запада на такие «заманчивые» предложения мы еще поговорим, а пока вернемся к Украине. Происходящие там сегодня процессы более чем красноречиво свидетельствуют о намерениях ее нынешних руководителей и их истинных хозяев.

Трибуналы для шахтеров

Буквально на следующий день после того, как Верховная Рада Украины отметилась
упоминаемым выше «историческим» нормативным актом, там состоялось торжественное
заседание, посвященное какой-то по счету годовщине другого «эпохального»
события --- принятия Декларации о государственном суверенитете. Выступивший на
этом заседании президент страны Владимир Зеленский заявил с трибуны буквально
следующее: «Шестой год подряд мы защищаем свой суверенитет от российской
агрессии и платим за это высокую цену --- жизни наших граждан…» По сути дела,
забравшийся в президентское кресло комик впервые за время своего «правления»
открыто и публично назвал «агрессором» нашу страну. Тем самым он окончательно
перешел «красную линию», а вернее сказать, поставил жирный крест на имевшихся,
похоже, кое у кого наивных чаяниях увидеть в нем «президента мира», который
прекратит братоубийственную войну на Востоке страны и начнет хотя бы делать
попытки нормализовать отношения с Россией. Подобные надежды изначально не
стоили выеденного яйца, но подпитывались в глазах особо упорных оптимистов
некоторой толикой снижения градуса русофобии в высказываниях и делах нового
лидера. Что ж, период иллюзий закончился.

В частности, об этом свидетельствует резкое оживление деятельности ищеек из
Службы безопасности Украины, после выборов несколько поумеривших свой пыл (а
вдруг и вправду «смена курса» грядет?), но буквально в последнее время с
утроенной энергией принявшихся «отлавливать» в массовых количествах
«сепаратистов», «шпионов ДНР» и «пророссийских интернет-агитаторов». Ушлые
ребятки четко поняли: курс остается прежним --- на конфронтацию с Россией и
силовой захват Донбасса. Судя по всему, в этом они нисколько не ошиблись…

То, что бесславный конец «Минским соглашениям» уготован не вынужденно
подписавшем и ненавидевшем их всей душой Порошенко, а «мирным» Зеленским, ясно
было давно. Одно только назначение «вице-премьером по Донбассу» Резникова, не
далее как месяц назад в Лондоне вещавшего о том, что главной своей задачей он
видит «создание особой модели правосудия» на «деоккупированных территориях»,
необходимой для того, чтобы учинить судилище над, как он выразился, «шахтерами
на танках», является более красноречивым, чем тысячи слов. Глядя на этого
господина, даже внешне чертовски смахивающего на Генриха Гиммлера, при его
словах об «особой модели» немедленно вспоминаешь о военно-полевых судах и
«особых тройках».

Кстати, и о полной неприемлемости для Киева идеи об «особом статусе» Донбасса в любом виде (а не то, что закрепленном конституционно) Резников также неоднократно высказывался. Примерно в таком же роде звучит риторика и главы МИД «нэзалэжной» Владимира Кулебы, совершенно не краснея заявляющего, что Киев совершенно не обязан выполнять какие-то там «Минские соглашения». Настоящие хозяева своего слова: захотели --- дали, захотели --- взяли обратно. Совершенно очевидно, что Украина этих договоренностей придерживаться более не будет даже на словах. И вовсе не потому, что, как утверждали некоторые, «их выполнение станет политическим крахом нынешней Украины». О чем вы, господа? То, что мертво, умереть не может… Окончательное крушение «нэзалэжной» --- лишь вопрос времени, с Донбассом или без него. Никакому «Минску» в ее политике не может быть места, прежде всего, оттого, что даже эти крайне противоречивые, половинчатые и, признаем откровенно, изначально невыполнимые пункты на протяжении нескольких лет удерживали ситуацию на самой грани масштабного вооруженного противостояния, не давали тлеющему локальному конфликту разгореться до большой войны.

Решать необходимо, пока не решили за нас

Нынче же тех, кто решил до конца отыграть «украинскую партию», подобное
вялотекущее развитие событий уже не устраивает. Понятно, что речь идет об
истинных правителях Украины, находящихся по другую сторону океана и
осуществляющих свою волю через посольство США в Киеве. Сразу же после принятия
ставящего вне закона основные положения «Минских соглашений» постановления
Верховной Рады, между президентом России Владимиром Путиным и канцлером
Германии Ангелой Меркель состоялся очень показательный телефонный разговор. В
его ходе лидеры обоих государств назвали позицию, занятую в последнее время
Киевом, «контрпродуктивной» и «заводящей в тупик переговорный процесс». В то же
самое время снова прозвучали ритуальные заявления о «безальтернативности
минских договоренностей», в которые, уверен, ни на грош не верят ни президент,
ни канцлер. Может ли Берлин принудить Украину «сбавить обороты» и хотя бы
внешне демонстрировать «приверженность мирному процессу»? Теоретически, шанс
есть.

Для этого «нэзалэжной» должны быть приняты конкретные меры: жестко и немедленно
перекрыты все подпитывающие ее каналы финансовой и прочей помощи,
приостановлено любое сотрудничество с ней, а граница Евросоюза наглухо
закрыться как для официальных представителей Украины, так и для всех ее
граждан. Вопрос в том, что ни Германия, ни ЕС в целом на такое вряд ли
осмелятся. Мало того, что выглядеть они при этом будут предельно глупо --- после
собственной многолетней поддержки «евромайдана» и порожденной таковым власти.
Подобный поворот будет воспринят в качестве предельно недружественного демарша
Вашингтоном, с которым Берлин сегодня и так, что называется, «на ножах».

Пойдут ли там на новый серьезный виток обострения? Крайне маловероятно. Скорее,
предпочтут и далее играть в бессмысленные «нормандские» игры», тем самым
укрепляя уверенность Киева в собственной безнаказанности. А вот Москва
придерживаться подобной политики вряд ли сможет. Рассмотрим в самом примитивном
приближении возможные варианты дальнейших действий Кремля. Реальными среди них
можно назвать три сценария. Соглашаться на «пересмотр» и без того не слишком
выигрышных «Минских соглашений»? Это будет означать полную сдачу Донбасса и его
жителей и, соответственно, окончательный крах всех внешнеполитических амбиций
России на «постсоветском пространстве», где прямо сейчас и без того мы имеем
обострение на белорусском и армянском направлениях. Далее продолжать делать
«грозные» заявления и выносить «последние предупреждения», выглядя при этом все
смешнее и нелепее?

Ну, заявил Сергей Лавров о том, что воинственный треп украинской власти «от
лукавого» и повторил мантру о «необходимости строгого выполнения договоров»…
Ну, потребовал Дмитрий Козак от Киева «опровергнуть» заявления Резника о
«необязательности «Минска»… Жарко стало на Украине кому-то от этого? Холодно
ли? Да им от этих все более выглядящих пустыми словес --- никак. Никто ничего
опровергать или дезавуировать даже не подумал. И не подумает. Единственная
достойная нашей страны линия --- это готовиться к признанию Донецкой и Луганской
республик хотя бы на уровне Южной Осетии. Да, со всеми вытекающими отсюда
возможными последствиями. В противном случае, скорее всего, в Кремле за
считанные часы или даже минуты придется принимать решение о введении на эту
территорию регулярных войск для спасения сотен тысяч жителей Донбасса от
уничтожения в процессе начатой Киевом карательной операции и вступлении в
прямое боестолкновение с украинскими войсками. И хорошо, если только с ними, а
не с «внезапно» появившимися там «миротворцами» из НАТО. Логика развития
ситуации подсказывает, что все может прийти именно к этому.

Александр Неукропный
  
