% vim: keymap=russian-jcukenwin
%%beginhead 
 
%%file 25_10_2019.news.pravda_com_ua.SPARTA
%%parent 25_10_2019
 
%%endhead 
\subsection{Республіка Щастя. Як в комуні під Харковом третій десяток років будують утопію}
\label{sec:25_10_2019.news.pravda_com_ua.SPARTA}
\Purl{https://www.pravda.com.ua/articles/2019/10/25/7230003/}
\Pauthor{Кузубов, Дмитро}

\ifcmt
pic https://img.pravda.com/images/doc/f/1/f1655bc-12sparta1.jpg
\fi

\ii{25_10_2019.news.pravda_com_ua.SPARTA.abstract}
\ii{25_10_2019.news.pravda_com_ua.SPARTA.doroga_k_SPARTE}
\ii{25_10_2019.news.pravda_com_ua.SPARTA.psih_bolnica_KGB_bratstvo}
\ii{25_10_2019.news.pravda_com_ua.SPARTA.starshii_gvardeec_TAMARA}
\ii{25_10_2019.news.pravda_com_ua.SPARTA.PORTOS_gorbachev_i_elcin}



\subsubsection{Высоцкий, Макаревич и ``Рідна мати моя''}

Внезапно из дома под названием Терем раздается задорный звук баяна. Разговор к
тому моменту продолжается уже три часа, поэтому решаем сделать музыкальную
паузу.

Главный зал Терема напоминает кабинет безумного ученого. Стены и потолок вдоль
и поперек увешаны бумагами с многочисленными формулами, графиками, диаграммами,
плакатными слоганами и стихотворениями. 

У входа в глаза бросается карта трезвых и пьяных стран по состоянию на 2006
год. Напротив двери стоит телевизор и большой стеллаж с книгами и
видеокассетами. На шкафу --- знамя "С.П.А.Р.Т.Ы.", на соседней стене –
перевернутый флаг Украины. С потолка свисают желтые липкие ленты с трупиками
мух. Воздух затхлый и отдает старостью.

Один из аксакалов коммуны Евгений Привалов играет на баяне "Рідна мати моя".
Остальные участники сидят за столом с песенниками под кодовым названием
"Э.Х.О". ("Эволюционное Хоровое Обеспечение"), и поют в унисон.

{\em
В перерывах между работой участники коммуны поют под баян
\/}

В песеннике "С.П.А.Р.Т.А.нцев" можно найти "Песню о Ленине" и другие
коммунистические шлягеры

Репертуар песенника своеобразен и разнообразен: от "Гимна СССР" и
"Интернационала" до песен Высоцкого и Макаревича.

"С.П.А.Р.Т.А.нцы" заканчивают импровизированный концерт своей коронной "Мы
желаем счастья вам". Музыкальная пауза переходит в обед. Кухарка Люба подает к
столу окрошку, пельмени, котлеты, оладьи, фрукты и компот.

– Мы еще называемся "П.О.Р.Т.О.С.", потому что любим покушать. Люба прошла
тест, чтобы нас не отравить. Это, конечно, шутка, но в каждой шутке, как
известно… --- смеется Тамара и предлагает нам чай каркаде.

Ланч весьма уместен перед началом разговора о деле всей их жизни --- вычислении
уровня счастья.

\subsubsection{Формула счастья}

Перед приездом в "С.П.А.Р.Т.У." Ольга Широкая попросила нас заполнить
"откровенник" из 50 вопросов, который в коммуне обычно выдают "перворазникам".
"Это из нашего начального теста по феличологии. Паспортные данные можно не
заполнять", --- написала Ольга.

"Откровенник", помимо прочего, содержит цепочку вопросов "Вы пьете?" –
"Почему?" --- "Вы намерены пить в дальнейшем?" --- "Почему?" --- "Сколько держится в
крови алкоголь?" --- "Откуда вы это узнали?", а также спрашивает "Чего Вам не
хватает для полного Счастья?". 

О счастье в "С.П.А.Р.Т.Е." готовы говорить бесконечно.

"Феличология --- это 77 наук, --- заверяет Ольга. --- Чтобы это все понять, нужна
недюжинная оперативка души. Учитывая, что у некоторых пропускная способность у
души маленькая, то у них просто процессор подвисает".

{\em
Уровень счастья в "С.П.А.Р.Т.Е." вычисляют по причудливым формулам
\/}

F(EST)O --- суммарный показатель счастья человека —"С.П.А.Р.Т.А.нцы" вычисляют по
специальной формуле.

"Лемма счастья Пифагора-Давыдова называется F(EST)O, в переводе с эсперанто  –
"праздник", --- просвещает нас Тамара. --- Счастье состоит из произведения Energio
– Энергии, Sociala --- Интеллекта, Tempo --- Времени и Organizeco --- количества
друзей. EST --- у всех условно одинаково, а О --- разное. Энергия измеряется в
Джоулях или Килокалориях. Интеллект вычисляем в ДАВах --- в честь нашего
основателя Давыдова. 

ДАВ расшифровывается как Дезинтеграл Активных Выживаний. Можно вычислить его
через электроэнцефалограф, а можно --- через написание поэзии. 

Считаем количество зарифмованных строк и строф. Рифмование обязательно для
входа в парламент. Норма --- условно 10 куплетов в неделю, если норму не
выполняешь, твой рейтинг понижается".

Уровень счастья и другие показатели в "С.П.А.Р.Т.Е." считают раз в три месяца
или в полгода. Свои подсчеты в коммуне называют "Н.О.В.А.Я. М.О.С.К.В.А." –
"Нахождение и Обучение Великих Альтруистов Явных --- Матрица Общего Соревнования
Коллективного Выполнения Amikidji". Людей с низким уровнем просят покинуть
организацию.

"Когда попытки поднять уровень счастья превращаются в пытки, не пытаемся, –
ухмыляется Тамара. --- Если индивид свыше 25 лет с нами и настойчиво нарушает
закон, напоминаю раз в три месяца или раз в полгода. Если "заблудился",
попытаюсь чаще. Но если сильно упирается --- отпускаю: куда праотцы позовут, туда
и отправляйся. Мы все Дети Галактики, куда пошлют --- туда и пойдешь".

\subsubsection{С.П.А.Р.Т.А. вне закона}

В конце 90-х --- начале 2000-х уровень счастья участников "С.П.А.Р.Т.Ы." вряд ли
был удовлетворительным --- у организации начались серьезные проблемы с законом. К
тому времени у "Б.К.Н.Л. --- П.О.Р.Т.О.С." под Люберцами в Московской области
появился "Городок Солнце" на 200 человек. 

Организация запустила бизнес --- 40 грузовиков занималось доставкой продуктов
питания, а в 2000-м выкупила территорию бывшего военного завода "Салют" в
Подмосковье.

Российским властям такие маневры не понравились. Генпрокуратура РФ обвинила 14
участников коммуны --- 10 граждан России и 4 украинцев --- в создании незаконного
вооруженного формирования. 5 из них, в том числе Юрий Давыдов, оказались за
решеткой, остальных объявили в межгосударственный розыск.

"Мы с Тамарой находились в межгосударственном розыске с 2000 года, --- утверждает
Ольга. --- Юру 7 декабря 2000-го посадили в тюрьму. У него было пять охотничьих
ружей. Еще в 90-е у некоторых наших были газовые пистолеты, в то время приняли
решение, что инструкторы их будут носить.

Когда Юру пытали, ему говорили: "Мы тебя посадим, посмотришь, теперь все твои
разбегутся". Он расписал это во втором томе "Теории Счастья" "Бутырская
баллада".

\subsubsection{История "Б.К.Н.Л. --- П.О.Р.Т.О.С." продолжается уже больше 30 лет}

После объявления в розыск Тамара написала письмо Путину, в котором разъяснила
суть деятельности организации и просила отнестись к ним справедливо. Не
помогло. Юрий Давыдов вышел на свободу только в мае 2006-го, а спустя три года
его не стало. 

Ольгу и Тамару сняли с розыска, закрыв дело за отсутствием состава
преступления, лишь в 2011-м. За это время объединение снова сменило название и
превратилось в "Ф.А.К.Э.Л. --- П.О.Р.Т.О.С." --- "Формирование Альтруистов
Кандидатов в Эволюционирующие Люди --- Поэтизированного Объединения Разработки
Теории Общенародного Счастья".

"В 2000-м у нас было 100 участников, --- вспоминает Тамара. --- Когда в Москве
понаехал РУБОП и СОБР, пришлось разбираться с решетками, с мусорской системой,
осталась четверть --- 25. К 2007-му мы по-новому выстроили экономику и снова
выросли: в Подмосковье было 42 человека, в "Подхарьковье" --- 37, в
"С.П.А.Р.Т.Е." --- 12. И еще 30 сменных. Итого 121 человек. 

Сегодня в Караване осталось пять матерых участников, которые тиражируют идеи
движения. В РФ наша организация сейчас находится в Одинцово под Москвой. Ждем
оттепель в России. Обычно раз в 20 лет бывает. Наше время пришло --- есть у нас
такой слоган и песня "Русь вперед". Каждому свое время, но наше время пришло".

\subsubsection{От оператора ПК к доярке}

Спрятав копну кислотных волос под косынкой, Алина идет доить коров. Путь в
коровник Ликург лежит через море грязи. "Слава Героям Труда!" --- гласит покрытая
слоем многолетнего налета надпись на входе.

В помещении царит зловоние. При этом даже коровы в "С.П.А.Р.Т.Е." названы
поэтично --- Ариадна, Красота, Поэма, Пальмира, Аура, Теза, Мирина и другими
изысканными именами.

– Маму Мира звали, бабушку --- Мойра, а она --- Мирина, --- объясняет Алина,
присоединяя к вымени коровы доильный аппарат.

\emph{Алина пришла в "С.П.А.Р.Т.У." как "оператор ПК", но стала дояркой}

Тем временем, свободолюбивая Аура пытается выбраться из стойла.

– Сука, Аура, на место! --- негодует девушка и вытирает пот со лба.

Через полчаса, после завершения процесса, Алина закуривает сигарету.

– А ты разделяешь их идеи? --- интересуемся у нее.

– Я не знаю, это их идеи, --- пожимает плечами Алина. --- Пить, курить нельзя, но я
курю, на праздники могу выпить, для них это неправильно. У них штрафы: за
курение --- 333 грн, за выпивку --- 999 грн. Требовали, чтобы я курить бросила, но
я не хочу. Если я курю, то не при них. 

Вообще курят все здесь, но чтоб не видели. И я не считаю неправильным на
праздник выпить шампанского, любой человек, даже президент делает это. У них
нельзя жениться, они незамужние и неженатые. После вступления в организацию
детей нет. Им нельзя иметь интимные отношения. Нам тоже, если мы тут работаем и
живем".

Алина тушит сигарету.

\emph{Вместе с двухлетней дочерью Алина снимает часть дома по соседству от "С.П.А.Р.Т.Ы."}

– Как давно ты в "С.П.А.Р.Т.Е." и как здесь оказалась? --- спрашиваем Алину.

– Я сама с Лозовой, с ними знакома четыре года. Пришла сюда по объявлению в
интернете. Было красиво написано: "Требуется оператор ПК". А какой тут оператор
ПК? Стала научным технологом, потом пастухом, последний год работаю дояркой,
потому что не было людей --- мало кто согласен с тем, что нельзя пить-курить.

– Ты ведь не за еду здесь работаешь?

– Зарплату платят, не за идею здесь. Хотелось бы больше, конечно. Но я понимаю,
шо це ферма. Самое хуже --- 150 грн в день. Бывало и 450 грн. Но если посчитать
по часам --- начинаешь с 4, заканчиваешь в 9-10, а бывает и позже. Когда
переработка молока, целую ночь сидишь. Работу хотела поменять, но жалко, если
распадется все. Они собираются продавать коров. Невыгодно, убыточные.

Последний год Алина с двухлетней дочкой снимают часть дома через несколько
дворов от "С.П.А.Р.Т.Ы." 

"Ей 23, Катька --- это у нее второй ребенок от второго мужа, --- характеризует
девушку Тамара. --- Одну девочку или мальчика у нее забрали уже. А муж такой был
– объелся груш. Делал у нас грядки, как-то она его намотала и он с переляку
переспал с ней и смотался, ребенка бросил. 

Научили ее коров доить. Она то в абрикосовый покрасится, то в еще какой-то.
Дочке два года, а мать такой пример подает. Чем она хороша? Мы абы кого не
берем. Если не умеет на бумаге писать --- не берем. А эта умеет. Но просто нужно
бегать за ней, чтобы заставить".

– Тут еще будет "сотка", интересное зрелище, --- улыбается Алина. --- Я не бегаю,
смысла не пойму. Захотел --- бегаешь каждый день по чуть-чуть. Но за раз
пробежать 100 километров --- смысл? Пробегают, то в обморок падают, то еще
что-то. Хоть бы призы были какие-то... А чисто за идею, за грамотку
нарисованную и медальку --- типа тех шоколадных, что в детстве дарили…

Алина приглашает нас на экскурсию в Крупскую --- женское общежитие. Интерьер
напоминает бедные постперестроечные квартиры 90-х в какой-нибудь глухой
провинции. Здесь как будто бы вечный Новый год: на выцветших стенах растянулись
старые гирлянды, а на полке пылится советский Дед Мороз. Под потолком висит
икона. Готовые декорации для неснятого фильма Балабанова.

\emph{В женском общежитии Крупской, как и всей коммуне, время как будто бы остановилось}

– Когда я рано встаю, здесь остаюсь, --- объясняет Алина. --- Лежишь ночью, не
спится --- на потолок смотришь: "О Боже!" А за закрытой дверью комната Ленинской
называется. 

Когда Томкин батя покойный закладывал, его закрывали в той комнате с Лениным,
отдавали поварихе ключ, чтоб его кормила. Он дядя такой был, шибанок. Каждой
дамочке, которая здесь находилась, предлагал интим за деньги. Короче,
интересный был мужчина. В прошлом году зимой помер --- замерз на остановке. 

\subsubsection{Молочник, тракторист, пастух и "девятка" Фидель}

Старожил "С.П.А.Р.Т.Ы." Евгений Привалов грузит банки с молоком в багажник
потрепанной временем "девятки", прозванной Фиделем в честь знаменитого
кубинского революционера. На стекле машины --- трещины, из приборной панели
торчат провода, салон набит всяким хламом.

– Молоко везу в Люботин на "большой рынок" --- по дачным домам развозить, –
объясняет Привалов с характерным русским акцентом.

Евгений родился в Димитровграде Ульяновской области РФ, долгое время жил в
Калуге. Один из близких товарищей Давыдова, в организации он состоит с начала
90-х. 

Вместе с другими участниками движения Привалов защищал Белый дом в Москве,
проходил по делу о создании незаконного вооруженного формирования и отбывал
срок в тюрьме.

В "С.П.А.Р.Т.Е." Евгений управляет хозяйством и играет на баяне. "И тракторист,
и дежурный, бывает и пастух", --- описывает функционал Привалова Тамара.

\emph{Евгений Привалов защищал вместе с основателем коммуны Белый дом и сидел в
тюрьме}

В 90-х и 2000-х Евгений занимался транспортными перевозками --- возил по России и
Украине корма. В Харьков Привалов передислоцировался 10 лет назад из-за
стечения обстоятельств.

– Я заехал сюда на неделю подменить Андрея Петрова, который здесь управлял
хозяйством, --- рассказывает Евгений. --- Неделя еще не закончилась, а 10 лет
прошло. Андрей спасал рабочего, который в колодец провалился, и сам погиб. Тот
с бодуна был --- руки тряслись, залез туда на метр, чтобы кран починить, не
удержался и упал. Неблагополучный колодец, там скопились какие-то яды --- дыхнул
один раз и готов.

– В какой вы партии? --- интересуемся у Привалова.

– Переходил несколько раз из партии в партию, уже и забыл. Скорей всего я в
партии дженералов стажер. Настроение зависит от количества сделанных добрых
дел. Успел сделать --- ты в тонусе, не успел --- настроение стало хуже. А звание
только отражает твои наработки. Заработал --- получил.

– Изначально чем вас привлекла организация Давыдова?

– Я товарища в третьем классе попросил: "Дай попробовать пиво". Попробовал –
дерьмо. Хуже, чем навозная жижа. С тех пор я не пью. Так и здесь --- если счастье
вещь нужная, то почему бы его не приобрести? 

– Мне кажется, у вашего движения много общего с хиппи --- они тоже
пропагандировали мир, счастье, любовь...

– Хиппи считать не умели. Счастье, которое не подсчитано, высоким не может
быть. Это как прибыли неподсчитанные. 

И потом, хиппи что, не курят и не пьют? И курят, и пьют. Живут ради
удовольствия. А как отказаться от такого удовольствия, как балдеж? Похмелье,
рак --- это потом. Поэтому нет у них ни мира, ни любви.

Пока мы разговариваем с Евгением, рабочие ремонтируют крышу коровника Кобзарь,
а Алина купает дочку в тазике.

\emph{Большинство зданий "С.П.А.Р.Т.Ы." нуждаются в капитальном ремонте}

– Следите ли вы за новостями? --- продолжаем расспрашивать Привалова.

– За политической ситуацией слежу, --- утвердительно кивает он. --- Сейчас, слава
Богу, новый президент. Надеемся, что воровство прекратится, сколько ж можно
воровать! Крадут вон через забор. Только отвернулся, смотришь --- аккумулятора
нет. Вроде и народ тут был, и сторож был --- никто ничего не видит.

– Вы считаете себя коммунистом?

– Мы и так при коммунизме живем давно. От каждого по способностям, каждому по
потребностям. Мне сколько надо, у меня все есть. Что такое коммунист? Если
дадите точное определение, тогда надо подумать. Иисус был первым коммунистом.
Можно назвать верующих, которые ходят в церковь, коммунистами?

– В какой-то степени да.

– Если в какой-то степени --- то и я в такой степени коммунист. Правда, им [в
церкви] до коммунизма далеко, они только амбары свои набивают и пузо. Почему их
попами называют? Потому что попы такие здоровые. А есть священники --- но их
мало. Большинство --- попы.

– Что думаете о конфликте России и Украины?

– Нет никакого конфликта между Россией и Украиной. Деньги отмывают умники.

– А гражданство у вас русское или украинское?

– У меня русский паспорт. Я написал заявление в Администрацию президента на
получение украинского гражданства, я 10 лет здесь как украинец в сердце, да и
раньше, как говорится… Мне уже можно в выборах участвовать. Проголосовал бы,
например, за партию "Слуга народа", так я не могу, у меня нет паспорта. А я бы
хотел поддержать умных ребят. 

– У вас есть семья, дети?

– Все, что есть у нас --- это и есть семья. Чужих детей не бывает. Наш главный
постулат --- это "П.О.Б.Е.Д.А." --- "Повсеместное Обеспечение Бесплатной Едой Детей
Альтруистов". Если обеспечить детей в школах бесплатным питанием, независимо от
того, родители зарабатывают или пьянствуют, у них не будет белкового голодания,
и они начнут соображать головой. 

Еще обязательно должно быть трудовое воспитание, чтобы могли заработать сами.
Не знаниям нужно учить, а навыкам --- чтобы умел, например, хоть дрова нарубить
или костер поджечь. В школах навыков практических не дают, там дают знания.
Пока новый президент, надо перестроить срочно образование. Хай Зеленский это
почитает.

В отличие от Евгения, который верит в конспирологию и винит в развязывании
войны на Донбассе украинские власти, Ольга прямо называет виновником Россию. 

Несмотря на то, что "С.П.А.Р.Т.А." позиционирует себя как пацифистская
организация, в условиях российской агрессии она готова поступиться своими
принципами --- по крайней мере, до тех пор, пока война не закончится.

"Я-то могу сказать, что я пацифист, но в данной ситуации я понимаю, что это
[война России и Украины] --- дело рук ФСБшников, --- объясняет Ольга. --- Я против
любого разделения территорий. И естественно, я против того, что отшманывают
Крым, пытаются хитрыми манипуляциями оттяпать часть Донбасса и т.д. К России я
всегда хорошо отношусь, к нерусям --- отвратительно. Русский --- значит "светлый",
а неруси --- это скоты. 

Лично я против развала Союза, потому что они по сути что сделали --- не стали
реформировать, а просто обворовали всех. Юра был в большей степени за то, чтобы
объединить славянские государства --- Украину, Россию и Белоруссию".

\subsubsection{Эсперанто говорили...}

Любые обвинения по поводу построения сектантской организации в "С.П.А.Р.Т.Е."
категорически отрицают. Старший гвардеец коммуны сравнивает деятельность учхоза
с инновациями Ford и Apple.

"Люди, которые говорят, что мы секта --- малограмотные дураки, --- заверяет Тамара.
– Это признак того, что они несовершенны и ищут причину вне себя. Я отношусь к
ним с сочувствием.

Учитывая, что единственной статьей доходов "С.П.А.Р.Т.Ы." сейчас является
продажа молока, возникает логичный вопрос, каким образом организации удается
оставаться на плаву. 

В коммуне объясняют, что помимо "С.П.А.Р.Т.Ы." участники задействованы "в
нескольких десятках проектов" в рамках "Ф.А.К.Э.Л. --- П.О.Р.Т.О.С." --- в
частности, поставляют продукты питания фермерам и конным клубам. Кроме того,
занимаются различными индивидуальными активностями на стороне.

– Мы 20 лет занимаемся доставкой продуктов питания и кормов, обеспечиваем
радиус 50-100 километров вокруг Харькова и по Харькову, --- рассказывает Ольга
Широкая. --- Нашу организацию мы между собой называем "Честность". У нас
несколько тысяч покупателей. Еще я работаю с трезвенниками как областной
координатор, в школах лекции по Теории Счастья и трезвости проводила в этом
году.

Тамара Костюк, между делом упомянувшая о своих путешествиях в Лондон и
Копенгаген, говорит, что работает "тайм-менеджером, мотиватором и коучем". При
этом в своих методиках также использует Теорию Счастья. Своими клиентами Костюк
называет айтишников, менеджеров по продажам и других.

– У меня всегда нестабильный доход --- много и очень много, но всегда много,
потому что я все время работаю, --- улыбается Тамара.

В этот момент из Терема выходит кухарка Люба с пакетами.

– Ну теперь вже все, не знаю, коли ми побачимся, завтра мені на работу, –
прощается со всеми Люба.



В лучшие времена организация Давыдова насчитывала сотню человек. Сегодня в коммуне остались единицы

– Спасибо, Люба! --- благодарит ее Тамара. --- Я тебе стишок написала неделю назад, пока ты будешь идти, успею прочитать. "Жили-были, ели, пили"…

– …эсперанто говорили… --- предлагает свой вариант Ольга.

Тамара на секунду запинается, но, собравшись с мыслями, начинает декламировать снова.

– Жили-были, ели, пили.

Пели, плакали, любили,

Счастье строили, творили,

Для друзей друзьями были.

По поэме написали,

Для кого легендой стали,

На вопросы отвечали,

По уму людей встречали.

Вы себе теперь ответьте:

Для чего вам жить на свете?

Польза в чем и в чем растрата,

За чей счет идет оплата?

Есть ли будущего шанс,

Кто ответит, кто за вас?

– Молодец, Томочка, мы будем за тебе голосовать в президенты! --- смеется Люба. –
Ты у нас будешь президентом, я тобі говорю! Ты всіх алкашів постреляешь! Их
надо стрелять!

– Может, одного мафиозу надо слегка припугнуть, а остальные перепугаются и
пойдут работать, --- предлагает компромисс Тамара, прищуривая глаза в лучах
заходящего солнца.

Перед тем, как уйти в закат, рискуем спросить ее о главном.

– И все же, как найти счастье?

– Первое --- узнать о нас, второе --- взять ручку в руки и начать учиться. Мы
сделали свою технологию, в которой чувствуем себя, как рыбы в воде. Мы за любые
технологии, которые помогают человечеству продуплицироваться (от лат. duplicate
– удвоение --- прим. УП). 

Некоторые приписывают Людовику XIV фразу "После меня хоть потоп". А я хочу,
чтобы после меня все продолжилось. Мы как 300 спартанцев --- будем бороться до
последнего.

Дмитрий Кузубов, фото --- Константин Буновский, для УП
