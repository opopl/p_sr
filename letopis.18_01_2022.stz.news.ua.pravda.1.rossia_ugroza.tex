% vim: keymap=russian-jcukenwin
%%beginhead 
 
%%file 18_01_2022.stz.news.ua.pravda.1.rossia_ugroza
%%parent 18_01_2022
 
%%url https://www.pravda.com.ua/news/2022/01/18/7320883
 
%%author_id 
%%date 
 
%%tags napadenie,psaki_dzhen,rossia,ugroza,ukraina,usa
%%title Росія ладна почати вторгнення в Україну будь-якої миті – Білий дім
 
%%endhead 
\subsection{Росія ладна почати вторгнення в Україну будь-якої миті – Білий дім}
\label{sec:18_01_2022.stz.news.ua.pravda.1.rossia_ugroza}

\Purl{https://www.pravda.com.ua/news/2022/01/18/7320883}

У Білому домі заявили, що Росія готова розпочати повномасштабне вторгнення до
України будь-якої миті.

Джерело: брифінг речниці Білого дому Джен Псакі

Пряма мова Псакі: "Президент Путін створив цю кризу, зібравши 100 тисяч
російських солдат уздовж українських кордонів. Включно з російськими
військовими корпусами, які прибули до Білорусі для спільних навчань.

\ii{18_01_2022.stz.news.ua.pravda.1.rossia_ugroza.pic.1}

Тож, наша позиція – це надзвичайно небезпечна ситуація. Ми досягли такого рівня
напруги, що Росія ладна будь-якої миті розпочати повномасштабне вторгнення до
України.

Державний секретар Блінкен ще раз наголосить, що США готові до дипломатичного
розв’язання ситуації, а інакше Росії доведеться вирішувати, чи хочуть вони
зустрітися з серйозними економічними наслідками (з боку США – ред.)".

Передісторія: 

Вранці 18 січня Блінкен зателефонував міністру закордонних справ Росії
Сєрґєю Лаврову для обговрення безпекової ситуації з російськими вйськами
вздовж українських кордонів. Блінкен повторив, що США пропонують розв’язати
ситуацію через дипломатичні перемовини. Крім того, Блінкен наголосив, що
будь-які рішення про європейську безпеку не можливо ухвалювати без участі
представників України та країн Європи.

19 січня Блінкен відвідає Київ для особистої зустрічі з президентом
Володимиром Зеленським, на якій обговорюватимуть ситуацію з російськими
військами на українських кордонах. Також держсекретар планує зустрітися з
американськими дипломатами та їхніми родинами в Києві, щоби обговорити
плани евакуації американських громадян в разі нападу Росії. Перед цим Київ
таємно відвідував голова ЦРУ, який зустрічався з Зеленським та іншими
українськими високопосадовцями.

Генеральний секретар НАТО Єнс Столтенберг у Берліні заявив, що НАТО готові
продовжувати спілкуватися з Росією щодо безпекових питань, зокрема
обопільного зменшення балістичних ракет та військових навчань.

Увечері 17 січня Велика Британія повідомила, що надасть Україні
протитанкові гранатомети новітньої розробки. Вранці 18 січня британські
військові літаки доставили перші партії озброєння до Києва. Росія
відреагувала, що це є \enquote{надзвичайно небезпечними кроками, які не сприяють
зниженню напруги в регіоні}.

Згідно з останніми дислокаціями, Росія сконцентрувала близько 100 тисяч
солдатів у прикордонних з Україною областях та тимчасово окупованому Криму.
Журналісти-розслідувачі повідомляли, що Росія перекидає з Далекого Сходу
танки та бойові установки \enquote{Град} та \enquote{Іскандер}.


