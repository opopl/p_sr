% vim: keymap=russian-jcukenwin
%%beginhead 
 
%%file 04_10_2021.fb.gorovyj_ruslan.1.jazyk_posty_facebook
%%parent 04_10_2021
 
%%url https://www.facebook.com/gorovyi.ruslan/posts/6443768775634053
 
%%author_id gorovyj_ruslan
%%date 
 
%%tags facebook,internet,jazyk,mova,ukraina,ukrainizacia
%%title Настав час задати це запитання ( Російська Мова Дописів )
 
%%endhead 
 
\subsection{Настав час задати це запитання ( Російська Мова Дописів )}
\label{sec:04_10_2021.fb.gorovyj_ruslan.1.jazyk_posty_facebook}
 
\Purl{https://www.facebook.com/gorovyi.ruslan/posts/6443768775634053}
\ifcmt
 author_begin
   author_id gorovyj_ruslan
 author_end
\fi

Настав час задати це запитання.

Якщо ви живете в Україні, бачите її виключно як незалежну країну, хочете щоб
ваші майбутні погєкоління теж тут жили, що вас заставляє писати пости
російською? Не говорити нею, бо часом хтось переживає за темп чи вимову, а саме
писати? 

Навряд ви не знаєте, що в Україні немає людей які не розуміють українською.
Тоді чому? Навіщо? 

Звісно, це запитання стосується лише тих, хто так робить. І ще. Я попрошу не
влаштовувати в коментах братовбивчу війну. Мені дійсно це цікаво. Якщо в кого є
що сказати - скажіть. Вважайте це дослідженням. Ще раз підкреслюю, мова не про
ваші коменти в постах у інших, можливо не українців, а саме про власні пости.

\ii{04_10_2021.fb.gorovyj_ruslan.1.jazyk_posty_facebook.cmt}


