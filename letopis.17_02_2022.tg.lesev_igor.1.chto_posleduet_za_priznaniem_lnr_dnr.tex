% vim: keymap=russian-jcukenwin
%%beginhead 
 
%%file 17_02_2022.tg.lesev_igor.1.chto_posleduet_za_priznaniem_lnr_dnr
%%parent 17_02_2022
 
%%url https://t.me/Lesev_Igor/314
 
%%author_id lesev_igor
%%date 
 
%%tags dnr,donbass,lnr,ukraina,vojna
%%title Что последует за возможным признанием ДНР и ЛНР
 
%%endhead 
 
\subsection{Что последует за возможным признанием ДНР и ЛНР}
\label{sec:17_02_2022.tg.lesev_igor.1.chto_posleduet_za_priznaniem_lnr_dnr}
 
\Purl{https://t.me/Lesev_Igor/314}
\ifcmt
 author_begin
   author_id lesev_igor
 author_end
\fi

Что последует за возможным признанием ДНР и ЛНР.

16 февраля на Украине случился новый большой праздник – День отражения
российской агрессии. Миллионы патриотов по призыву Зеленского вывесили
национальные флаги, надели вышиванки и в 10 утра спели гимн. И тут уже шансов у
Путина не было. Ровно в 3 часа ночи он дрогнул, вторжение было отменено, и
украинцы одержали легендарную перемогу, которую теперь потомки будут воспевать
веками.

И это ведь не выдержка из психиатрического диагноза пациента. Это абсолютно
официальная позиция украинских властей, где нет даже толики гиперболы. Чтобы
понимать и разделять в России подобные чувства, нужно быть немножко Лией
Ахеджаковой, т.е., патентованным русофобом.

Но между очередной грандиозной победой Украины, случилось событие не из мира
хоббитов и эльфов, а из мира реальной политики. Госдума проголосовала за
обращение к Путину признать независимость ДНР и ЛНР.

Российская политика достаточно закрытая и специфическая. Важные решения,
особенно стратегического направления во внешней политике, здесь не принимаются
эмоционально и с кондачка. Не нужно быть каким-то заслуженным специалистом по
ру-политике, чтобы понимать, что такое голосование Госдумы могло быть в чем-то
неожиданным для Владимира Путина. А это ключевой момент. Это означает, что
решение по будущему ЛДНР, да и в целом, будущему Украины полностью моделируется
Кремлем. С соответствующими опциями.

Причина, почему появилось подобное решение тоже вполне очевидно. Минские в их
политической части полностью саботируются украинской стороной, и в Кремле
максимально поднимают ставки. Правда, в России существует иллюзия, что Минские
соглашения возможно выполнить, была бы только политическая воля. Но это именно
иллюзия, и я коротко поясню почему.

Существуют две Украины – ленинская и бандеровская. Вернее, одна проходит
трансформация в другую. Современная Украина – это наследница УССР, а
субъектность последней создали большевики. Они начертили границы Украины,
проводили политику коренизации, при них же УССР стала членом-основателем ООН,
при Союзе там появился свой МИД, свои конституции и все прочие атрибуты
государственности, которые финализировались в 1991 году в сами знаете что.

Но ленинскую Украину основали два ключевых народа – русские и украинцы.
Соответственно, и республика была русских и украинцев. Были там, конечно же, и
десятки других народов, но мы говорим о базовых величинах. А базовые – это
русские и украинцы.

С 1991 года мультикультурный проект стал сворачиваться, и появилась концепция
бандеровской Украины. Украины для украиноязычных украинцев. Сначала появилась
конституция, где государственным языком был вписан только украинский. А в итоге
все докатилось до сегрегационного закона против русского языка и закона о
коренных народах. Это реалии современной Украины, в которой русским и русскому
нет места.

И вот Минские соглашения. Проблема их реализации для Киева ведь не в амнистии,
не в проведении выборов или контроле над границей. ЛДНР – это сохранившийся
осколок ленинской Украины, при том той ее части, где исторически доминирует
русское. А Минские предполагают этот осколок законсервировать в виде
автономного статуса с внутренним самоуправлением. И, естественно, со свободным
функционированием русского языка, что для бандеровской Украины в принципе
неприемлемо. Как можно будет пояснить, что в 100\%-ом русскоязычном Мариуполе
запрещены все русскоязычные школы, а в соседнем Донецке детям разрешено
образование на русском? Это несовместимые дефиниции, как наркомания и здоровый
образ жизни.

Именно в этом, а ни в каком бы то ни было еще плане, Минские не принимаются и
физически не могут быть приняты в бандеровской Украине. В Украине для
украиноязычных украинцев не может быть русскоязычных анклавов. Иначе вся
конструкция посыплется, а она для адептов этнической модели страны имеет
экзистенциальное значение.
