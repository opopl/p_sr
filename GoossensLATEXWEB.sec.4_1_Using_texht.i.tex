
\isubsec{4_1_Using_texht}{Using \texht}

To have a source \ \LaTeX\  file translated into HTML, many users will
find it sufficient just to load the \texht package \verb|tex4ht.sty| (Figure~4.1). 
The translation is activated with a command of the form 

\begin{verbatim}
htlatex filename, 
\end{verbatim}

or a similar system-dependent command. The output is stored in a set of
files with the root file named \hbox{\emph{filename}\verb|.html|}.

\isubsubsec{4_1_1_Package_options}{Package options}

The outcome of the translation of the \ \LaTeX\  source files into HTML
can be varied using package options (for example, see \reffig{4-2}).
There are a number of package options available. Some are of general
interest; most are not at all essential. Some options are provided as
shortcut mechanisms to offer convenient ways of activating useful
features without needing to learn their details. 

\iii{vb_1}

For instance, the options 1, 2, 3, and 4 are simply shortcuts for a
single \verb|\tableofcontents| and a few \verb|\CutAt| and
\verb|\TocAt*| commands. Users looking for a more refined outcome than
that offered by these options will need to invest a little effort into
studying these commands (see \refsec{4_5_2_Tables_of_contents} on page 172 and \refsec{4_5_3_Parts_chapters_sections_and_so_on}
on page 175) and using them directly. 

The first package option must be the name of a configuration file (see
\refsec{4_5_1_Configuration_files} on page 170) or the option html. Otherwise, the first
option is ignored. 

The order in which the nonleading options are listed is not significant.
As its name suggests, the html option asks for HTML output. Without that
option, the output should be a standard DVI file. The following are some
of the other options that are available: 

\pindx{package options}

\begin{itemize}
  \item \textbf{1, 2, 3}, or \textbf{4} These options request a breakup of the documents
    into hierarchies of hypertext pages. The hierarchies reflect the
    logical organization of the content, as specified by the sectioning
    commands \verb|\part, \chapter, \section|, and so on. The values of
    the option determine the desired depth of the hierarchy of hypertext pages. 
  \item \textbf{sections+} The entries in tables of contents provide
    hypertext links to the sections to which they refer. This option
    asks for extra backward hypertext links from the titles of the
    sections to the tables of contents. 
  \item \textbf{next} When partitioning a document into hypertext pages
    along sections, navigation links to establish paths between the
    pages are introduced. In the default setting, some paths capture
    previous-next relationships that reflect the logical tree-structured
    organization of the sections, providing connections between
    immediate siblings. This option asks for alternative previous-next
    navigation links, reflecting the linear succession of all the
    sections in the document. 
  \item \textbf{pic-array, pic-displaylines, pic-eqnarray, pic-tabbing, pic-tabular} 
    These options request pictorial representations for the named environments 
    instead of representations employing HTML tables. 
  \item \verb|_13, ^13, no_, no^| \texht modifies the native definitions
    of the special characters \verb|_| and \verb|^| in order to
    enable the inclusion of HTML tags for subscripts and superscripts in
    mathematical formulae. The first two options request alternative
    definitions (with category codes of 13 instead of 12 for the
    characters). The second pair of options disables that behavior. 
  \item \verb|refcaption| References to figure and table environments inserted with the \verb|\ref|
    command are assigned hypertext links pointing to the entry points of
    these environments. This option makes hypertext links that point to
    the captions within these environments. 
  \item \verb|3.2| This option associates a request for HTML version 3.2 with the html option, 
    instead of the default association with the Transitional 4.0 version of HTML. 
    The latter version is considerably closer in its nature to BTEX than 3.2 is. It 
    differentiates issues of style from content and logical structures, and we rec    ommend its use wherever possible. 
  \item \verb|fonts+| The normal setting expects the browsers to use their own default fonts for 
    the default font of the document. This option suggests a specific font for the 
    browsers to use. 
  \item \verb|no_|\emph{style} This option turns off the loading of HTML features that are designed 
    specifically for the specified \emph{style file}. For instance, the option
    \verb|no_amsart| asks \texht not to load the code it holds for the \verb|amsart|
    package and the option \verb|no_array| makes a similar request for the code
    that \texht has to offer especially for the package array. In such
    cases, the user might need to tailor private contributions for the
    packages. 
  \item \verb|info| This option requests clues about the features of the package to be written 
    into the log file. 
  \item \verb|htm| This option is a variant of the html option in which filenames are given main 
    names of, at most, eight characters, and the HTML filenames are given the 
    extension of htm. 
\end{itemize}
 
\isubsubsec{4_1_2_Picture_representation_of_special_content}{Picture representation of special content}

The ultimate challenge is to get all the translated objects expressed in
terms of hypertext tags capturing both the structure and the semantics
of the content. When such goals cannot be met, alternative
representations have to be used. 

\paragraph{Mathematics and special characters}

The inline math environments \verb|\(|\emph{formula}\verb|\)| and the
display math environments \verb|\[|\emph{formula}\verb|\]| request
pictorial representations for their content. 

On the other hand, plain TEX inline math environments
\verb|$|\emph{formula}\verb|$| and
display math environments \verb|$$|\emph{formula}\verb|$$| produce a mixture of text output
for simpler subformulae and pictures for the other parts. 

In addition to mathematics, some accented characters, some other special
characters, and \ \LaTeX\  picture environments are automatically
translated into pictures in the default setup. Other entities require
explicit requests to be translated to bitmap pictures. 

\paragraph{Linking to, and making, pictures}

The bitmap pictures rendered by browsers can be specified in
stand-alone files referenced from the HTML code. The command 

\begin{verbatim}
\Picture [alt] {filename attributes} 
\end{verbatim}

creates such a reference to an existing file and specifies attributes and alternative 
text representation for the pictures. The [alt] component is optional; when it is 
not provided, a default text representation is provided by the system. If attributes 
are given, they must be separated from the filename by a space. 
For example, the command 

\begin{verbatim}
\Picture[TUG logo]{httpt//www.tug.org/logo.gif ID=``tuglogo''} 
\end{verbatim}

references a GIF file containing the logo of the TEX Users Group (TUG). This 
command produces the HTML code 

\begin{verbatim}
<IMG SRC="http://www.tug.org/logo.gif" ALT=``TUG logo'' ID=``tuglogo''> 
\end{verbatim}

The following is a variant with a pair of commands: 

\begin{verbatim}
\Picture+ [alt] {filename attributes} 
\EndPicture 
\end{verbatim}

This requests a bitmap picture to be \emph{created} for what is placed between them. Here, 
however, a missing \verb|[alt]| parameter is taken to be a request to produce an alternative 
content-related textual representation for the picture. 
 
%%page page_181

\begin{verbatim}
                              \Picture+{}% 
                                \begin{tabular}{lcr} 
                                    1 & 2 & 3\\ 
<IMG SRC="try0x.gif"                xxx & xxx & xxx\\ 
ALT="1 2 3                          1 & 2 & 3 
xxx xxx xxx                     \end{tabular}% 
1 2 3"> 
                              \EndPicture 
\end{verbatim}

The plus \verb|(+)| can be replaced by a star \verb|(*)|. In this case the content is typeset 
within a vertical box. The filename is optional within the command, as is the file 
extension. If either is omitted, a system-created entry is provided. 

The filename of the last bitmap picture created is recorded in 

\begin{verbatim}
\PictureFile 
\end{verbatim}

The command 

\begin{verbatim}
\NextPictureFile{filename} 
\end{verbatim}

provides a filename for the \emph{next} bitmap picture, if no filename is provided in the 
\verb|\Picture| command. The \verb|\NextPictureFile| command can come in handy for 
reaching concealed \verb|\Picture| commands. 

\begin{verbatim}
<IMG SRC="mypic.gif" ALT=``ab''> and        \[\alpha^\beta\] and 
<IMG SRC="mypic.gif" ALT="[Picture]">       \Picture{\PictureFile} use 
use the same bitmap file.                   the same bitmap file. 
\end{verbatim}

The default setting assumes the extension gif for the names that \texht
assigns to the bitmap files. The extensions \verb|jpg| and
\verb|png| can be requested
as an alternative with, respectively, the package options
\verb|jpg| and \verb|png|. On the other hand, the 
command 

\begin{verbatim}
\ConfigurePictureFormat{_extension_} 
\end{verbatim}

can be used dynamically to change the setting. 
