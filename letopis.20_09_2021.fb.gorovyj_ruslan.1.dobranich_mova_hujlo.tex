% vim: keymap=russian-jcukenwin
%%beginhead 
 
%%file 20_09_2021.fb.gorovyj_ruslan.1.dobranich_mova_hujlo
%%parent 20_09_2021
 
%%url https://www.facebook.com/gorovyi.ruslan/posts/6366799506664314
 
%%author_id gorovyj_ruslan
%%date 
 
%%tags identichnost',jazyk,mova,ukraina
%%title Добраніч. Бережіть одне одного. І не забувайте хто @@@ло
 
%%endhead 
 
\subsection{Добраніч. Бережіть одне одного. І не забувайте хто @@@ло}
\label{sec:20_09_2021.fb.gorovyj_ruslan.1.dobranich_mova_hujlo}
 
\Purl{https://www.facebook.com/gorovyi.ruslan/posts/6366799506664314}
\ifcmt
 author_begin
   author_id gorovyj_ruslan
 author_end
\fi

Жодна боротьба не буває без втрат.

Є такі і в моїй боротьбі за ідентичність і мову. Десь рік ту о сталася одна з
найнеочикуваніших. 

Моя добра знайома під одним з моїх постів несподівано написала «ну вот і ти…».

Далі йшов буквальний потік болю від того, що я нібито буду її вбивать за те, що
«.. мой радной язик ета язик пушкіна в блока, булгакова і салжиніцина». Потім
йшлося про «ні я ні мая дочь нє вінавати, что карєнниє кієвлянє…» і далі в
такому ключі. Допоки я побачив комент, то їй встигли добряче насувати тут, у
фб. Однак оскільки вона, в кращих традиціях срібного віку, не написала в
особичку, а влупила прилюдно «я нє могу бить в друзьях с собствєнним убійцей,
пращайтє…» то ні на що інше як на серйозний піздюль з боку інших коментаторів,
сподіватися було марно.

Звісно, побачивши, я написав їй в особисті все, що вважав за потрібне і
пояснив, що відправляю в бан більше для її ж блага, аби вона не обпікалася
надалі.

Наступного дня отримав в особичку купу болю від її чоловіка, прекрасного
чоловіка і батька, якого також знаю багато років. «Вона, звісно, зря так
написала, однак ти її не захистив від купи хейту, бувай…»

Тут взаглі не було про що говорити. Попрощалися й усе.

Отже. Двоє гарних людей яких я знав багато років раптом включили щодо мене
режим «убійца русскоязичних». Повторюся, це інтелігентні розумінні люди які, як
мені здавалося, досить добре розуміють, що відбувається в країні. Тож чому з
україномовного друга я раптом став вбивцею? Оце і є гибридна війна і вплив з
боку московії. 

Те що я сам не споживаю нічого російськомовного, це не означає, що я кидаюся на
російськомовних в побуті і ріжу їх. Побут це вибір кожного. Навіть якщо мені це
не подобається.

Однак я пропагую все українське. Двадцять чотири години на добу, сім днів на
тиждень. Бо чітко розумію свою приналежність. І так, мене не цікавить ані
московитська література, ані культура в принципі. Ані класична, якої я наївся
до війни, ані тим більше сучасна. Я навіть нашу російськомовну не споживаю, а
чужу то й взагалі. А ще я првсякчас це декларую. І оця декларація рано чи пізно
спрацьовує і в очах розумних порядних людей я стаю «катом русскіх людей» хоча
лише захищаю право на власну ідентичність і незалежність від імперії.

Я пишу це щоб не складалося враження, що боротьба за ідентичність це «ги-ги» в
фейсбуці. Часом ментальний водорозпподіл проходить не лише між друзями, а й між
батьками і дітьми і подружжями. Мої пости про мову не треба любить. Я просто
хотів би, щоб ви розуміли суть того, що відбувається з мовою і її носіями.

Добраніч. Бережіть одне одного. І не забувайте хто хуйло.
