% vim: keymap=russian-jcukenwin
%%beginhead 
 
%%file 09_03_2019.stz.news.ua.mrpl_city.1.rena_sajenko
%%parent 09_03_2019
 
%%url https://mrpl.city/blogs/view/rena-saenko-zhizn-posvyashhennaya-istorii-mariupolya
 
%%author_id burov_sergij.mariupol,news.ua.mrpl_city
%%date 
 
%%tags 
%%title Рена Саенко: жизнь, посвященная истории Мариуполя
 
%%endhead 
 
\subsection{Рена Саенко: жизнь, посвященная истории Мариуполя}
\label{sec:09_03_2019.stz.news.ua.mrpl_city.1.rena_sajenko}
 
\Purl{https://mrpl.city/blogs/view/rena-saenko-zhizn-posvyashhennaya-istorii-mariupolya}
\ifcmt
 author_begin
   author_id burov_sergij.mariupol,news.ua.mrpl_city
 author_end
\fi

\ii{09_03_2019.stz.news.ua.mrpl_city.1.rena_sajenko.pic.1}

Когда журналистам или научным сотрудникам, аспирантам или студентам, школьникам
или учителям, составителям истории предприятий, а то и просто жителям города,
интересующимся событиями старины, нужно было узнать, кто раскопал и описал
мариупольский неолитический могильник, в каком году прокатали первую трубу на
заводе \enquote{Никополь}, где находится дом, в котором родился знаменитый певец Михаил
Гришко, и многое другое, связанное с историей Мариуполя и Мариупольщины - они
обращались к \textbf{Рене Ильиничне Саенко}. И каждый получал исчерпывающий ответ.
Полвека своей жизни она отдала старейшему музею Донбасса, изучению родного
края, пропаганде его истории. За этим стоит несколько тысяч проведенных
экскурсий, тысячи описанных экспонатов, сотни подготовленных выставок,
неисчислимое количество лекций на предприятиях, в школах, в окрестных селах,
статьи в академических сборниках, в научных журналах, газетах. Ее публикации
многократно цитировались солидными исследователями.

\textbf{Читайте также:} 

\href{https://mrpl.city/blogs/view/mariupolchanka-olena-osipenko-zakohana-i-viddana-profesii}{%
Маріупольчанка Олена Осипенко: закохана і віддана професії, Ольга Демідко, mrpl.city, 08.03.2019}

Рена Ильинична охотно и бескорыстно участвовала в создании кино- и
телевизионных фильмов, посвященных истории города, никогда не отказывала
журналистам дать интервью, не задумываясь о том, сошлются они на нее или нет.
Но никогда не оставалась равнодушной, когда кто-то вольно или невольно искажал
факты истории города. Звонила авторам и редакторам, возмущалась, доказывала
свою правоту, опираясь на документы. Она, не задумываясь, раздавала свое
драгоценное время. Хотя могла бы использовать его, чтобы написать и с
легкостью защитить кандидатскую, а может быть и докторскую диссертацию...

Почти на всех детских фотографиях Рена Ильинична запечатлена улыбающейся. Но
это вовсе не означает, что детство ее было безоблачным. Война, оккупация,
ужасающая нужда послевоенных лет. Вспоминая те годы, она всегда с
благодарностью говорила о своей бабушке - это она спасла семью от голодной
смерти, перешивала старые вещи, меняла на продукты. И все-таки в той жизни были
свои радостные события. Блузка, сделанная из маминого платья, занятия в
танцевальном, а затем в драматическом кружке, школьные годы. Училась она
увлеченно, всегда с нежной благодарностью вспоминала своих наставников.
Директора школы и преподавателя истории Анну Федоровну Токарь, которая
специально для нее писала поурочные планы, чтобы та в отсутствие преподавателя
могла проводить уроки с одноклассниками. Ее любимой учительницей была Ирина
Ивановна Ширяева, Рена Ильинична считала, что именно она научила излагать мысли
на бумаге, правильно писать, правильно воспринимать прочитанное.

\textbf{Читайте также:} 

\href{https://mrpl.city/blogs/view/dvorets-kultury-kotoryj-ne-perezhil-perestrojku}{%
Дворец культуры, который не пережил перестройку, Анатолий Ломакин, mrpl.city, 05.08.2018}

Именно пример ее наставниц предопределил выбор профессии, она решила стать
учительницей. Молоденькая выпускница средней школы едет поступать в
Ростов-на-Дону. Успешно сдает вступительные экзамены в педагогический
институт, но одного единственного балла не хватает, чтобы стать студенткой
дневного отделения. Пришлось учиться на историческом факультете заочно.
Однажды ей нужно было написать контрольную работу по археологии. Пошла в
краеведческий музей, подобрать необходимые материалы. Заодно прошлась по
залам. Экспозиция ей понравилось. Зашла к директору – тогда этот пост занимал
Максим Семенович Клименко – узнать насчет устройства на работу. Через неделю
ее приняли на временную работу. С 26 апреля 1955 года Рена Ильинична стала
работником музея.

С той поры вся ее жизнь, все важнейшие события связаны с музеем. Даже ее
свадьба начиналась в его стенах. За праздничным столом сидела она с молодым
мужем Иваном Петровичем, а также немногочисленные ближайшие родственники и все
сотрудники музея.

Рена Ильинична прошла почти все должности музейного работника. Начинала с
экскурсовода, затем стала научным сотрудником, заведующей отделом, пять лет
была директором музея, затем снова заведовала отделом, а когда достигла
пенсионного возраста, передала свой пост любимой и преданной ученице Раисе
Петровне Божко, а сама продолжала работать старшим научным сотрудником музея,
да еще как работать!

\textbf{Читайте также:} 

\href{https://archive.org/details/14_04_2018.sergij_burov.mrpl_city.kogda_byl_osnovan_mariupol}{%
Когда был основан Мариуполь?, Сергей Буров, mrpl.city, 14.04.2018}

Приведем отрывок из одного официального документа: 

\begin{quote}
\em\enquote{За годы деятельности в
Мариупольском краеведческом музее Р. И. Саенко собрала около 4 000 экспонатов,
написала и опубликовала более 500 научных статей на краеведческие темы,
выполнила исследование \enquote{Из истории основания города Мариуполя}. Ей принадлежит
основной текст по истории нашего города в академическом издании \enquote{История
городов и сел Украинской ССР} (1976 г.), а также в книге \enquote{Весь Мариуполь} (2004
г.). Р. И. Саенко на основании документов установила точную дату основания г.
Мариуполя и генеалогическое древо нашего великого земляка – художника А. И.
Куинджи. С ее участием проводились раскопки многих курганов и поселений
Приазовья, материалы которых стали экспонатами археологической коллекции музея.
Она активно участвовала в подготовке к изданию уникального в своем роде
сборника архивных документов Мариупольского греческого суда 1780 – 1869 гг.
Является соавтором фундаментального издания \enquote{Мариуполь и его окрестности.
Взгляд из ХХI века} (2006 г.). Рена Ильинична хорошо известна в научных кругах
далеко за пределами Мариуполя. Она является членом Всеукраинского
геральдического общества, активно сотрудничает с институтом истории Украины
Национальной академии наук, переписывается с научными учреждениями Российской
Федерации}.
\end{quote}

Рена Ильинична в работе никогда не ограничивала себя только прямыми
обязанностями, предусмотренными должностными инструкциями. Лекции по линии
общества \enquote{Знание}, подготовка методических пособий, работа в бесчисленных
юбилейных комиссиях, в комиссии по наименованию и переименованию улиц, собрания
по подготовке научных конференций, а еще на ней было руководство практикой
студентов. Но, кроме того, у нее была еще одна должность. Быть может, самая
главная – \enquote{должность} жены, матери, бабушки. Ее работу ценили, награждали
грамотами, почетными знаками, ей было присвоено звание \enquote{Мариупольчанка года}.
Но самыми ценными наградами для себя она считала любовь и уважение своих
сотрудников.

\vspace{0.5cm}
\begin{minipage}{0.9\textwidth}
\textbf{Читайте также:} 

\href{https://archive.org/details/08_03_2019.mrpl_city.vystavka_muzej_kuindzhi}{%
В Мариуполе открылась уникальная выставка из прошлого века, mrpl.city, 08.03.2019}
\end{minipage}
\vspace{0.5cm}

На большинстве фотографий Рена Ильинична Саенко улыбается. Открытой улыбкой
человека верного себе, своим близким, делу, которому посвятила свою жизнь.
Странно и тяжело писать об этой неунывающей, жизнерадостной, полной оптимизма
женщине в прошедшем времени. Увы, непоправимое случилось. Она ушла из жизни 13
декабря 2006 года, неожиданно для всех. Ушла, и вместе с тем осталась в истории
Мариуполя, которой посвятила всю свою жизнь.

20 февраля 2007 года Рене Ильиничне Саенко за весомый вклад в изучение и
популяризацию истории Мариуполя, значимую краеведческую и просветительскую
деятельность было присвоено звание почетной гражданки города Мариуполя
(посмертно).
