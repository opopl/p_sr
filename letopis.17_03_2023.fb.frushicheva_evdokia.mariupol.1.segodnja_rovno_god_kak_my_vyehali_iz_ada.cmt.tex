% vim: keymap=russian-jcukenwin
%%beginhead 
 
%%file 17_03_2023.fb.frushicheva_evdokia.mariupol.1.segodnja_rovno_god_kak_my_vyehali_iz_ada.cmt
%%parent 17_03_2023.fb.frushicheva_evdokia.mariupol.1.segodnja_rovno_god_kak_my_vyehali_iz_ada
 
%%url 
 
%%author_id 
%%date 
 
%%tags 
%%title 
 
%%endhead 

\qqSecCmt

\iusr{Сергей Тертышник}

Да, не верится уже, что всё это мы переживали. Полмиллиона человек, 21 век.
Каждый день хоть щипай себя. С нами ли всё это происходит? И самое страшное -
это прогрессирующая деградация. Сегодня отключили свет и включили. Завтра свет
и связь погасли полностью. Ничего, есть пауэрбанк. Потом потух и пауер банк и
Айкосы и прочее. Вода, отопление, потом газ, готовим на кострах, ничего, зато
есть дом, потом выбиваются окна и двери, потом пожары. Авиация, артиллерия,
танк, стрелковое. Не верится, что мы это пережили и нужно жить дальше

\begin{itemize} % {
\iusr{Євдокія Фрушичева}
\textbf{Сергей Тертышник} мы когда-то проснёмся и примем реальность происходящего.

\iusr{Сергей Тертышник}
\textbf{Евдокия Фрушичева} И начнём жить дальше без огладки. Дай бог
\end{itemize} % }

\iusr{Татьяна Киор}

Тогда не верилось, что это с нами происходит, а сейчас не вереться, что это
было...

Каждый раз думаю, все забуду, не буду никому рассказывать, всё в прошлом....

А потом, нет.... вспоминаю и не забуду никогда, и буду помнить, чтобы жить так,
как не жила....

\begin{itemize} % {
\iusr{Євдокія Фрушичева}
\textbf{Татьяна Киор} я тоже считаю, что нужно помнить все, чтобы ценить то, что есть.

\iusr{Yuliya Savenkova}
\textbf{Евдокия Фрушичева} наша жизнь поделена на \enquote{до} и \enquote{после}, перевернулась на 180 градусов.... как можно забыть?
\end{itemize} % }
