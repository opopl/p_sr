% vim: keymap=russian-jcukenwin
%%beginhead 
 
%%file 29_05_2019.stz.news.ua.mrpl_city.1.ganna_zajceva_ne_bojatys_dijaty
%%parent 29_05_2019
 
%%url https://mrpl.city/blogs/view/mariupolchanka-ganna-zajtseva-golovnene-boyatisya-diyati
 
%%author_id demidko_olga.mariupol,news.ua.mrpl_city
%%date 
 
%%tags 
%%title Маріупольчанка Ганна Зайцева: «Головне – не боятися діяти!»
 
%%endhead 
 
\subsection{Маріупольчанка Ганна Зайцева: \enquote{Головне – не боятися діяти!}}
\label{sec:29_05_2019.stz.news.ua.mrpl_city.1.ganna_zajceva_ne_bojatys_dijaty}
 
\Purl{https://mrpl.city/blogs/view/mariupolchanka-ganna-zajtseva-golovnene-boyatisya-diyati}
\ifcmt
 author_begin
   author_id demidko_olga.mariupol,news.ua.mrpl_city
 author_end
\fi

\ii{29_05_2019.stz.news.ua.mrpl_city.1.ganna_zajceva_ne_bojatys_dijaty.pic.1}

Наша наступна героїня відрізняється особливою чуйністю і не\hyp{}байдужістю,
відповідальністю і сміливістю. Завдяки її енергійності та неабиякій завзятості
одна з болючих і актуальних проблем – захист безпритульних тварин Маріуполя –
була вперше виведена на загальноміський рівень. І дійсно, турботі про братів
наших менших не завжди приділяли належну увагу, а справжніх ентузіастів і
активістів у цьому питанні було зовсім небагато. Між іншим, Україна входить в
десятку країн, які лідирують за кількістю бездомних тварин. Одна з головних
причин того, чому на вулицях так багато бездомних тварин – відсутність
законодавства, яке могло б регулювати ці процеси і не дозволяло б людям робити
аморальні вчинки по відношенню до тварин та стимулювало відповідальне ставлення
до домашніх улюбленців. Водночас стерилізація тварин, попри те, що залишається
важливим і необхідним заходом, досі лягає на плечі найбільш свідомих громадян.

\ii{insert.read_also.demidko.zhyvko}

Ганна народилася в Маріуполі, який любить з дитинства і, незважаючи на всі
проблеми, вважає його найкращим містом України. Батьки походять з Запорізької
та Січеславської областей, але зустрілись та прожили все життя в Маріуполі. У
сім'ї часто говорили українською мовою. Сьогодні Ганна мешкає з батьками,
чоловіком і донькою, які її в усьому підтримують. Для жінки важливо, що вона
може покластися на рідних людей, без підтримки і розуміння яких не змогла б
реалізувати все задумане.

У 1994 році Ганна закінчила Мелітопольський університет і отримала унікальну
спеціальність: \enquote{вчитель біології і англійської мови}. Виявляється, що настільки
дивне поєднання предметів може існувати. Наша героїня змалку полюбляла природу,
але й мала хист до мов, тому ця спеціальність їй підходила ідеально. Однак
життєвий шлях складався дуже цікаво і незвично, адже Ганні після інституту
довелося навіть попрацювати круп'є в казіно. Проте найбільшу частину свого
життя вона присвятила педагогічній діяльності. Вже 25 років Ганна працює
вчителем англійської мови в ЗОШ № 15. Вчителька пишається, що деякі її учні
вирішили стати саме викладачами англійської мови, вважає це своїм маленьким
досягненням.

\ii{29_05_2019.stz.news.ua.mrpl_city.1.ganna_zajceva_ne_bojatys_dijaty.pic.2}

Тварин Ганна любила завжди. П'ять років до активної громадської діяльності
займалася стерилізацією тварин самотужки, адже ця проблема була завжди для неї
близькою. Наприкінці 2015 року Ганна стала депутатом Маріупольської міської
ради. Саме ця подія посприяла на законодавчому рівні вирішувати наболілу
проблему. У 2017 році команда, до якої входить і наша героїня, створила
громадську організацію \enquote{Волонтерське об'єднання
\enquote{Зооконтроль-Маріуполь}}.

\ii{29_05_2019.stz.news.ua.mrpl_city.1.ganna_zajceva_ne_bojatys_dijaty.pic.3}

Найбільше Ганну надихає голова організації \textbf{Галина Лєкунова}, яка працює
ветеринарним лікарем і невтомно доводить, що безпритульні тварини мають право
на життя і захист. До команди входить не так багато маріупольців, але всі хто є
– справжні ентузіасти своєї справи. Головні напрями організації: стерилізація
та популяризація бездомних тварин серед маріупольців. Всі заходи члени
організації проводили за власні кошти та кошти небайдужих громадян, лише цього
року місто виділило матеріальну підтримку на проведення вже щорічного \textbf{\enquote{Кубку
дворНяшки}}. Також організація співпрацює з всеукраїнським Благодійним фондом
\enquote{Щаслива лапа}. Разом вони забезпечують маріупольські школи матеріалами, які
дозволять вчителям краще познайомити дітей з проблемами бездомних тварин.
Загалом Ганна Зайцева стала організатором багатьох акцій і заходів, присвячених
проблемам тварин Маріуполя.

\ii{29_05_2019.stz.news.ua.mrpl_city.1.ganna_zajceva_ne_bojatys_dijaty.pic.4}

Зокрема, важливою стала акція \textbf{\enquote{Березневий кіт}} (2018 р.) – всім бажаючим
проводили безкоштовну стерилізацію тварин на кошти, вилучені від продажу
календарів. Також були проведені акції (мітинги і марші) проти тварин в цирку,
проти полювання на тварин та хутряної індустрії – Всеукраїнські заходи,
запропоновані Зоозахисною ініціативою UAnimals. Ганну підтримують колеги і учні
школи. Так, вчителька праці \textbf{Гридасова Світлана} провела корисну акцію з
виготовлення разом з дітьми прикрас на новорічну ялинку, дозволивши учням
проявити ініціативу і заробити кошти для підтримки менших братів міста. Також
силами волонтерів був проведений підрахунок безпритульних собак в Маріуполі за
допомогою програми, розробленої ГО ID animals.

\textbf{Читайте також:} \emph{В Мариуполе спасенные от смерти племенные шиншиллы обрели новые дома}%
\footnote{В Мариуполе спасенные от смерти племенные шиншиллы обрели новые дома, Анастасія Папуш, mrpl.city, 21.05.2019, \par%
\url{https://mrpl.city/news/view/v-mariupole-spasennye-ot-smerti-plemennye-shinshilly-obreli-novye-doma}%
}

\ii{29_05_2019.stz.news.ua.mrpl_city.1.ganna_zajceva_ne_bojatys_dijaty.pic.5}

У вересні 2018 року вдалося відкрити \enquote{Ветеринарний центр \enquote{Західний}}, в якому
теж проводять безкоштовну стерилізацію тварин за пожертви громадян. У
майбутньому Ганна хотіла б вийти на співпрацю з міжнародними організаціями,
які, як показує досвід, надають суттєву допомогу безпритульним тваринам в інших
містах України.

В Маріуполі вже створено комунальне підприємство \enquote{Центр сучасного поводження з
тваринами \enquote{Щасливі тварини}}. Водночас громадська діячка вважає, що необхідно в
майбутньому розширити і більш якісно організувати роботу в цьому напрямі.

Найбільше громадська діячка полюбляє проводити час на морі, особливо там, де
можна побути наодинці і насолодитися чудовими краєвидами і такою потрібною
тишею.


\textbf{Улюблена книга:} науково-фантастичний роман Рея Бредбері \enquote{Марсіанські хроніки}.

\textbf{Хобі:} \enquote{Колекціоную бездомних тварин (сміється). Наразі вже щаслива власниця 4 чарівних кішок та 4 чудових собак. А взагалі я люблю подорожувати}.

\textbf{Улюблений фільм:} \enquote{Через тернії до зірок} (1981 р.)

\ii{29_05_2019.stz.news.ua.mrpl_city.1.ganna_zajceva_ne_bojatys_dijaty.pic.6}

\textbf{Курйозний випадок:} 

\begin{quote}
\em\enquote{Близько 10 років тому я була класним керівником у 9 класі. Оголосили карантин
і мені довелося ввечері всім телефонувати і повідомляти цю приємну для учнів
новину. Розмову я починала так: \enquote{Вітаю. У нас карантин!}. Проте, випадково
набравши одну іншу цифру в кінці, замість учня я зателефонувала заступнику мера
Львова Андрія Івановича Садового. Він на мої слова відповів: \enquote{Вітаю! А у нас
карантину нема!}. Після цього представився і ми довго сміялися}.
\end{quote}

\textbf{Порада маріупольцям:} \enquote{Не проходьте повз проблеми, не бійтеся взяти на себе відповідальність. Якщо не ми, то хто!}.

\textbf{Читайте також:} \emph{В Мариуполе более 50 мартовских котов больше никогда не обзаведутся потомством}%
\footnote{В Мариуполе более 50 мартовских котов больше никогда не обзаведутся потомством, Ярослав Герасименко, mrpl.city, 07.03.2018, \par%
\url{https://mrpl.city/news/view/v-mariupole-bolee-50-martovskih-kotov-bolshe-nikogda-ne-obzavedutsya-potomstvom-foto}}

\ii{29_05_2019.stz.news.ua.mrpl_city.1.ganna_zajceva_ne_bojatys_dijaty.pic.7}
