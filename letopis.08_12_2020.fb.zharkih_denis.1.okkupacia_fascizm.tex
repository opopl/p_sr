% vim: keymap=russian-jcukenwin
%%beginhead 
 
%%file 08_12_2020.fb.zharkih_denis.1.okkupacia_fascizm
%%parent 08_12_2020
 
%%url https://www.facebook.com/permalink.php?story_fbid=2874817242731674&id=100006102787780
 
%%author Жарких, Денис
%%author_id zharkih_denis
%%author_url 
 
%%tags 
%%title Оккупация, интеграция и бандеризация
 
%%endhead 
 
\subsection{Оккупация, интеграция и бандеризация}
\label{sec:08_12_2020.fb.zharkih_denis.1.okkupacia_fascizm}
\Purl{https://www.facebook.com/permalink.php?story_fbid=2874817242731674&id=100006102787780}
\ifcmt
	author_begin
   author_id zharkih_denis
	author_end
\fi

Мне тут российские френды пишут что-то типа да где вы видели имперские
настроения в России? Ответить можно только аналогичным - а где вы видели на
Украине фашистов? Тут та же ситуация: и того, и другого официально нет. Но вот
даже на официальном уровне такое отморозят, что держись. А потому я хочу
рассмотреть перспективы имперского расширения России в современном мире. 

Вот раньше с империями было все просто. Основной вопрос был не в завоевании
территорий, а в том, как до них добраться.  Доплыть до Индии и Америки и не
утонуть, добраться до сибирских ханств и не погибнуть в тайге, дойти до Средней
Азии и не умереть от жажды в пустыне.  А дальше все просто местным аборигенам
наплевать, кто с них будет драть налоги, лишь бы поменьше, потому
сопротивляться будет только немногочисленная знать. 

Российская Империя легко росла во все стороны, поскольку, куда не глянь, все
была кормовая база. Основу империи составлял крестьянин, который сам себя
кормил за счет земли. Земля была основным ресурсом, которого у Империи было
навалом, а вот людишек первое время не хватало. Пока работающих на земле не
хватало, все было хорошо. К концу империи ситуация измениться - людишек полно,
кормить нечем, имперский аппарат с управлением не справляется. 

Тогда пришли большевики. Они раздали землю крестьянам, которым опять все равно,
кто ими там управляет, а потом  начали индустриализацию. Вот эта самая
индустриализация и забрала из деревни излишек рабочей силы и создала новый
жизненный уклад, в том числе и в колхозной деревне. Пока людей на производстве
не хватало все было хорошо. Но потом город тоже разросся и его сложно стало
прокормить, а ел он все больше. Партийный аппарат с управлением не справился,
объявил капитализм и развалил СССР. 

А основной вопрос развала был в том, кто кого и сколько будет кормить. Раньше
элита считала, что окончательно всех накормит коммунизм (нужно только
подождать), то теперь она отчетливо поняла, что всех накормит рынок (нужно
только подождать). Плохо то, что рынок, как ранее коммунизм, всех кормить
отказался, о чем традиционно сытое начальство догадывается последним. Поэтому
вопрос о новом укладе стоит на всем постсоветском пространстве во весь рост. 

В современном постсоветском укладе никакое выгодное расширение империи
невозможно. Это раньше можно было наделять крестьянам землю на границах и они
себя кормили (казачество), или построить завод, а он что-то нужное выпускает
(создание рабочего класса). Уже с середины двадцатого века практикуют
неоколониализм. Вы там сами себя кормите, а все интересное мы у вас купим. 

В этом плане Украина (да и другие советские республики) стала интересна России
от слова НИКАК, и с каждым годом это никак росло все больше. Украина не
производила ничего такого, что было интересно российской элите, а традиционную
продукцию легко было заменить западной или из третьих стран. Поэтому сегодня
экономической базы для интеграции/аннексии/оккупации просто нет. 

Оторвав у Украины дотационный регион Россия добилась немногого. России Крым
обходится еще дороже, чем Украине, и он требует ресурсов, которых мифический
рынок не дает. Ведь вопрос расширения сегодня не состоит в том сколько
убыточных регионов еще прибавить к империи. Вопрос в том, чтобы свои убыточные
регионы сделать прибыльными. А с этим не только Россия, США не справляются, а
они, между прочим, деньги на весь мир печатают. Но даже с печатным станком у
США не очень выходит. Кто-то печатает, кто-то колотит либеральные понты, кто-то
несет демократию в танках и бомбах, прочим делать нечего, а кормить их надо.
Явное избыточное количество штатов.


Явление знакомое. Товарищ Ельцин насчитал лишними в СССР ровно четырнадцать
республик. А до этого Горбачев посчитал нерентабельными все страны
социалистического блока. Германию продали по цене гамбургера (украинцы Крым
отдали за так, и это патриотично - ничего не берем у врагов!). Короче, для
роста империи нужен новый уклад, а старом-то так хорошо живется начальству. И
тут на помощь приходят бандеровцы. Раз есть бандеровцы, то расширятся не надо,
зачем нам эти бандеровцы? Лишние они! Вот круг и замкнулся. Расти, конечно,
как-то надо, но нельзя - бандеровцы. Вот она боль имперски мыслящих людей. Что
сказать - сочувствую, Рим тоже погубили варвары. Что с них с варваров взять.

Правда, упадок был не у варваров, а у Рима, но это не считается, это вообще
неприлично даже говорить в обществе. Так и живем.
