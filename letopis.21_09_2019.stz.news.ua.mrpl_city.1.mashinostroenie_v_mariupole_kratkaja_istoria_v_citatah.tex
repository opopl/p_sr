% vim: keymap=russian-jcukenwin
%%beginhead 
 
%%file 21_09_2019.stz.news.ua.mrpl_city.1.mashinostroenie_v_mariupole_kratkaja_istoria_v_citatah
%%parent 21_09_2019
 
%%url https://mrpl.city/blogs/view/mashinostroenie-v-mariupole-kratkaya-istoriya-v-tsitatah
 
%%author_id burov_sergij.mariupol,news.ua.mrpl_city
%%date 
 
%%tags 
%%title Машиностроение в Мариуполе: краткая история в цитатах
 
%%endhead 
 
\subsection{Машиностроение в Мариуполе: краткая история в цитатах}
\label{sec:21_09_2019.stz.news.ua.mrpl_city.1.mashinostroenie_v_mariupole_kratkaja_istoria_v_citatah}
 
\Purl{https://mrpl.city/blogs/view/mashinostroenie-v-mariupole-kratkaya-istoriya-v-tsitatah}
\ifcmt
 author_begin
   author_id burov_sergij.mariupol,news.ua.mrpl_city
 author_end
\fi

\ii{21_09_2019.stz.news.ua.mrpl_city.1.mashinostroenie_v_mariupole_kratkaja_istoria_v_citatah.pic.1}

\textbf{Сборник \enquote{Мариуполь и его окрестности}, 1892 год. Статья преподавателя
Мариупольской Александровской мужской гимназии М. И. Кустовского \enquote{Фабричная и
заводская деятельность г. Мариуполя}:} 

\begin{quote}
\enquote{В 1886 году во II части города основан г. Уваровым чугунолитейный механический
завод; на нем выделываются всевозможные сельскохозяйственные орудия. Кроме
этого завода плуги разных видов выделываются в мастерских Ясина на Торговой,
Бабекова, Шевцова и других у фонтана под горой}.
\end{quote}

\textbf{\enquote{Справочная книга. Фабрики, заводы, рудники Донецкого бассейна},} 1915 год.
Владимиро-Львовский машиностроительный завод. Арендатор - Мошкевич. Ул.
Таганрогская. 200 рабочих.

\textbf{\enquote{Адрес-календарь Екатеринославской губернии на 1916 год}.} Мариуполь.
Машиностроительный завод В. С. Сойфера. Завод земледельческих машин Г. И. Балжи.
Слесарная мастерская Вердникова.

\textbf{\enquote{Мариуполь и его окрестности: взгляд из XXI века}. Заведующая отделом
Мариупольского краеведческого музея Р. П. Божко:} 

\begin{quote}
\enquote{В 1902 году на заводе В. С.
Сойфера работало 65 человек, через два года – 176. На заводе действовал паровой
котел, использовалась электроэнергия. В 1904 году предприятие выпустило
рудничное оборудование, чугунное литье на 100 тыс. рублей, заняв шестое место
по объему товарного производства среди шестнадцати подобных заводов Южной
России}. 
\end{quote}

\textbf{\enquote{Приазовский рабочий} 10.11.2011. А. Попов, зам. директора Мариупольского завода технологического оборудования:} 

\begin{quote}
\enquote{В 1918 году после национализации завод Сойфера влился в металлургический завод
\enquote{Никополь}. После окончания гражданской войны на предприятии изготавливали
сельхозорудия, детали для мельниц, запчасти на суда АМП. Вскоре завод
становится базовым по производству дезинфекционного оборудования. В 1926-1927
гг. специально был построен корпус по выпуску вошебоек, камер Сакса, ранее
покупаемых за границей. Развитию завода помешала начавшаяся Великая
Отечественная война. Но уже в первом полугодии 1944 года он был восстановлен, в
1946 году достиг довоенного объема производства, а в 1947 году превысил его в
1,4 раза. Впечатляющие успехи были достигнуты в 60-80-е годы, когда
Мариупольский завод технологического оборудования стал ведущим в стране по
выпуску специального технологического оборудования для производства
отечественных лекарственных средств. Многие изделия создавались впервые и не
имели аналогов в отечественной, а порой и в мировой практике}.
\end{quote}

\textbf{Читайте также:} 

\vspace{0.5cm}
\begin{minipage}{0.9\textwidth}
\href{https://mrpl.city/news/view/sudba-mariupolskogo-azovmasha-ostaetsya-pod-voprosom}{%
Судьба мариупольского \enquote{Азовмаша} остается под вопросом, Олена Онєгіна, mrpl.city, 18.09.2018}
\end{minipage}
\vspace{0.5cm}

\textbf{Ю. Некрасовский \enquote{Огненное столетие} (1897 – 1997):} 

\begin{quote}
\enquote{11 марта 1945 года Государственный комитет обороны принял решение об
организации на заводе имени Ильича производства железнодорожных цистерн.
Трудности организации такого производства заключались не только в том, что
нужно было строить новые цехи и переориентировать работу многих действующих
цехов, но и в том, что подобная продукция на заводе никогда не выпускалась и
нужно было осваивать технологию ее производства. На заводе был организован
отдел машиностроения. (Нужно напомнить читателям, что в 1946 году завод
назывался \enquote{Мариупольский машиностроительный и металлургический завод им. Ильича
Министерства транспортного машиностроения СССР})}. 
\end{quote}


Пресс-центр ОАО \enquote{Азовмаш} сообщает: 

\begin{quote}
	
Постановлением № 325 Совета министров УССР от 26 марта 1958 года и
распоряжением № 314-Р Сталинского совнархоза от 28 апреля 1958 года на
базе машиностроительного и металлургического производств завода имени
Ильича (по состоянию цехов на 1 мая 1958 года) образован Ждановский
завод тяжелого машиностроения (ЖЗТМ). Общая площадь завода составила
165,3 тыс.кв. метров, в том числе производственные площади — 145,5
тыс.кв. метров. Численность трудового коллектива - 11 627 человек.
Номенклатура выпускаемой продукции: железнодорожные цистерны,
металлургическое оборудование, топливозаправщики. В состав завода вошли
цеха, расположенные на площадке \enquote{Б}, территории бывшего завода \enquote{Русский
Провиданс}. 

Создаются новые цеха: цех металлоконструкций и инструмен\hyp{}тально-штамповочный. В
области цистерностроения создаются образцы новых цистерн. Выпущена первая
партия алюминиевых цистерн для перевозки молока. Впервые в СССР разрабатывается
проект железнодорожной цистерны грузоподъемностью 120 тонн. Сталеразливочные
ковши клепаной конструкции заменены ковшами сварной конструкции. Впервые в СССР
на заводе изготовлены сталеразливочные ковши емкостью 330 и 350 тонн.

\end{quote}

\textbf{Читайте также:} 

\href{https://mrpl.city/news/view/mariupolskij-azovmash-vtorichno-popal-v-obnovlennyj-spisok-obektov-bolshoj-privatizatsii}{%
Мариупольский \enquote{Азовмаш} вторично попал в обновленный список объектов большой приватизации, Анастасія Папуш, mrpl.city, 16.01.2019}

\begin{quote}
В 1963 году начато производство топливозаправщика ТЗ-22 — самого массового в
истории мировой техники.

В 1984 году впервые в мире введен в эксплуатацию вскрышной конвейерно-отвальный
комплекс (производительностью 4000 кубометров в час) для крупнокусковых
скальных пород золотодобывающего карьера \enquote{Мурунтау} (Узбекистан). 2002 год.
18-20 марта в Абу-Даби состоялась выставка вооружений, где гвоздем программы
стал украинский бронетранспортер. В его создании принимал участие и \enquote{Азовмаш}.
\end{quote}

Эту краткую историю мариупольского машиностроения, наверное, у\hyp{}местно завершить
словами выдающегося инженера-машиностроителя, воина и руководителя \textbf{Владимира
Федоровича Карпова}, произнесенными в фильме \enquote{Командармы индустрии} в 1980 году:

\begin{quote}
\em\enquote{Если отбросить всё то, что говорят, какой я хороший, то о главном можно
сказать так: практически завода не было, а была группа разрозненных цехов.
Сейчас это крупнейший завод, тридцать девять тысяч народа, семь с половиной
тысяч инженерно-технических работников, выпуск продукции перевалил за триста
миллионов рублей. Значит, если я при этом участвовал вместе с коллективом
завода, это действительно стало делом моей жизни}.
\end{quote}

Именно при В. Ф. Карпове была начата масштабная реконструкция ЖЗТМ.

\textbf{Читайте также:} 

\href{https://archive.org/details/22_09_2018.sergij_burov.mrpl_city.ko_dnju_mashinostroenia_vladimir_karpov}{%
Ко Дню машиностроителя: Владимир Карпов, Сергей Буров, mrpl.city, 22.09.2018}

\clearpage
