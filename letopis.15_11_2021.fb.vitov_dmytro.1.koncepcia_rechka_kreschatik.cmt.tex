% vim: keymap=russian-jcukenwin
%%beginhead 
 
%%file 15_11_2021.fb.vitov_dmytro.1.koncepcia_rechka_kreschatik.cmt
%%parent 15_11_2021.fb.vitov_dmytro.1.koncepcia_rechka_kreschatik
 
%%url 
 
%%author_id 
%%date 
 
%%tags 
%%title 
 
%%endhead 
\subsubsection{Коментарі}

\begin{itemize} % {
\iusr{Роман Коляда}
Вже давно називаю крайній від проїжджої частини тротуар «набережною», хз чому)

\iusr{Елена Яцкевич}
А далі що? Більше схоже на утопію!

\begin{itemize} % {
\iusr{Дмитро Вітов}
Поки це утопія, але може стати реальністю))

\iusr{Елена Яцкевич}
\textbf{Дмитро Вітов} А що з річкою Либідь, Почайна? Дніпро теж потребує захисту!

\iusr{Татьяна Литвиненко}
\textbf{Дмитро Вітов} Когда городской транспорт научится летать

\iusr{Kostiantyn V Bespavlov}
\textbf{Helena Jackiewicz}

\href{https://bigkyiv.com.ua/richka-pochajna-otrymaye-pryrodni-beregy-z-dnyshha-znyato-betonni-plyty-foto/}{%
Річка Почайна отримає природні береги: з днища знято бетонні плити (ФОТО), bigkyiv.com.ua, 02.11.2021%
}

\end{itemize} % }

\iusr{Viacheslav Kharytonov}
А потім відновимо піски на місці Троєщини

\begin{itemize} % {
\iusr{Дмитро Вітов}
\textbf{Viacheslav Kharytonov} на місці Троєщини були залівні луги

\iusr{Viacheslav Kharytonov}
\textbf{Дмитро Вітов} ото до піску знімемо, розберемо дамбу у Вишгороді і за пару років намиє нормально.
\end{itemize} % }

\iusr{Ahava Teslenko}
Это было бы супер!

\iusr{Кириченко Олег}
\textbf{Ahava Teslenko} і щоб підводні човни могли заходити

\iusr{Дмитро Вітов}
\textbf{Кириченко Олег} а чому у Льві авіаносці можуть заходити, а у Київ ні?)))

\iusr{Настасия Бо}
Там навіть вже є готове русло частково, у вигляді метрогограду

\iusr{Дмитрий Бирко}
Гарна утопія. Але - утопія. Але гарна. Я за!

\iusr{Людмила Коваленко}
Хороша новина!  @igg{fbicon.thumb.up.yellow} 

\begin{itemize} % {
\iusr{Дмитро Вітов}
\textbf{Lyudmila Kovalenko} це тільки мрії. Щоб воно стало новиною довгий- довгий шлях.

\iusr{Людмила Коваленко}
\textbf{Дмитро Вітов} але ж є початок. Добре вже те, що про це думають, розмовляють, мріють. Я навіть на це не сподівалась. Головне почати лупати цю скалу. Яким би привабливим став Київ.
\end{itemize} % }

\iusr{Daniel Tchikin}
По-моєму, там більшою проблемою буде не відновити річку, власне, а зберегти її більш-менш чистою.
Ну і, власне, не впевнений, як вона "уживется" з метро.
А от створення публічного пішоходного простору – хоча б від Прорізної до Європейської – по-моєму, давно назріло

\iusr{Катя Жук}

а что делать с речками, которые уже снаружи? видел что с речкой Сырей и около?
у меня ощущение, что надземные никого не волнуют, потому что стоимость
восстановления подземных сильно выше и стырить можно больше.

\begin{itemize} % {
\iusr{Дмитро Вітов}
\textbf{Kate Zhook} пока это только мечты энтузиастов.

\iusr{Катя Жук}
Почайну ж восстанавливают. че ж мечты?

\iusr{Дмитро Вітов}
Там неи ничего, намного легче восстановить.

\iusr{Катя Жук}
видел что с Сырцом?

\iusr{Дмитро Вітов}
Я даже часто там хожу

\iusr{Катя Жук}
ну вот... но вообще никому не интересно. включая наших местных депутатов
\end{itemize} % }

\iusr{Дима Макагон}
Все звичайно гарно, але як я на машині тут проїду?

\begin{itemize} % {
\iusr{Дмитро Вітов}
\textbf{Діма Макагон} по Володимирський

\iusr{Дима Макагон}
\textbf{Дмитро Вітов} це ж пішки доведеться йти на Хрещатик)


\iusr{Дмитро Вітов}
 @igg{fbicon.laugh.rolling.floor}{repeat=3} 

\iusr{Григорий Мельничук}
\textbf{Діма Макагон} По Окружній @igg{fbicon.face.smiling.sunglasses} 
\end{itemize} % }

\iusr{Дима Макагон}
А взагалі цікаво, є більше інфо? Тут же Кличко вирішив 1 млрд на перекладку плитки на Майдані витратити.

\iusr{Ігор Чичкань}
там ще колись й кіз пасли, чому в концепції нема нічо про повернення кіз?? @igg{fbicon.face.partying} 

\iusr{Olexandr Chornogorodskij}
Мєчтать нє врєдно

\iusr{Lesya Rubashova}
Але ідея дуже гарна!

\iusr{Алена Балаба}

В Тайвані так і зробили, а там населення в рази більше. Я особисто мрію, що так
звільнять з-під землі Глибочицю в Києві. Було б чудово. І Дніпро стане
повноцінною суднохідною рікою.

\iusr{Анатолий Ман}
Хм.. А Хрещатик, бо там хрестили? Оце тільки зараз подумалось ...

\iusr{Інна Грищенко}
Було б круто, аби здійснилося!!!

\iusr{Ирина Белоцерковская}
Ох, бідне моє Місто(

\iusr{Alex Surkov}
Не додержали вброс до 1 апреля?))))

\iusr{Дмитро Вітов}
\textbf{Alex Surkov} це ж не вброс. Дизайнери за власною ініціативою помріяли.

\end{itemize} % }
