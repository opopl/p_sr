% vim: keymap=russian-jcukenwin
%%beginhead 
 
%%file 05_08_2021.fb.hristenko_oleg.1.lbgt_jazyk_post
%%parent 05_08_2021
 
%%url https://www.facebook.com/KhrystenkoOleh/posts/1011359989612082
 
%%author Христенко, Олег
%%author_id hristenko_oleg
%%author_url 
 
%%tags jazyk,lbgt,mova,obschestvo,ukraina
%%title Анатолий Ульянов - пост - язык - ЛБГТ - ультраправые
 
%%endhead 
 
\subsection{Анатолий Ульянов - пост - язык - ЛБГТ - ультраправые}
\label{sec:05_08_2021.fb.hristenko_oleg.1.lbgt_jazyk_post}
 
\Purl{https://www.facebook.com/KhrystenkoOleh/posts/1011359989612082}
\ifcmt
 author_begin
   author_id hristenko_oleg
 author_end
\fi

Анатолий Ульянов написал хороший пост. Но я задаюсь вопросом - связанно ли это
полностью с реверансами в сторону ультраправых? Я думаю, что лишь отчасти. 

\ifcmt
  pic https://scontent-cdt1-1.xx.fbcdn.net/v/t1.6435-9/232596847_4595302243822236_3764565289722081914_n.jpg?_nc_cat=105&ccb=1-4&_nc_sid=8bfeb9&_nc_ohc=_llVoD4Ww08AX9eId9m&_nc_ht=scontent-cdt1-1.xx&oh=7a8f4d285e4f91e0fc2d8acd0b2549af&oe=6139ABE9
  width 0.4
\fi

\href{https://www.facebook.com/dadakinder/posts/4595411350477992}{%
Вот это трэш. В посте «Страна для всех» на инсте Киевпрайда ..., Анатолий Ульянов, facebook, 05.08.2021%
}

Бион выделил три базовых эмоциональных групповых допущений, которые определяют
групповые реакции: 

1) допущение о зависимости - группа идеализирующая лидера, которая
сопровождается чувством всемогущества, отрицанием, завистью и жадностью; 

2) допущение о борьбе и бегстве - группы объединённые восприятием внешнего
врага; 


3) группа базового допущения об образовании пар - группа живущая надеждой
своего продолжения в творческой активности некой пары, эти группы подобны стаям
бонобо и воспринимают близость как защиту от агрессии.  Какое базовое допущение
украинцев? Вероятно, что второе - группа с базовым допущением о борьбе-бегстве.
Это допущение помогает мобилизировать агрессию группы и направить ее на врага,
но враг этой группой воспринимается расплывчато, поэтому любая оппозиция и
альтернатива воспринимается угрозой и вражеской, поэтому такие группы
распадаются на подгруппы, которые сражаются друг с другом. Это группы
невозможной инклюзии и толерантности, поэтому попытки ЛГБТ побрататься с
хранителями «культурных кодов» будут безуспешны, так как ЛГБТ несут на себе
печать отличия. Но ЛГБТ - часть общества и поэтому отображают и живут внутри
общей групповой динамики, поэтому тяжело обосновывать промахи нашего движения
интеллектуальной провинциальностью и скудоумием отдельных участников.
