%%beginhead 
 
%%file 31_03_2022.fb.lafazan_natalia.mariupol.2.moja_vtoraja_isterika
%%parent 31_03_2022
 
%%url https://www.facebook.com/permalink.php?story_fbid=pfbid028meH41yXBxo7mJi9RpskQCUbriqppSg595uKJyC1uzN9mdf2FunG4pUMzBf7k1XBl&id=100030592628843
 
%%author_id lafazan_natalia.mariupol
%%date 31_03_2022
 
%%tags mariupol,mariupol.war
%%title Моя вторая истерика меня догнала, когда я пошла на девятиэтажку отзвониться людям, которые за меня переживали
 
%%endhead 

\subsection{Моя вторая истерика меня догнала, когда я пошла на девятиэтажку отзвониться людям, которые за меня переживали}
\label{sec:31_03_2022.fb.lafazan_natalia.mariupol.2.moja_vtoraja_isterika}

\Purl{https://www.facebook.com/permalink.php?story_fbid=pfbid028meH41yXBxo7mJi9RpskQCUbriqppSg595uKJyC1uzN9mdf2FunG4pUMzBf7k1XBl&id=100030592628843}
\ifcmt
 author_begin
   author_id lafazan_natalia.mariupol
 author_end
\fi

Моя вторая истерика меня догнала, когда я пошла на девятиэтажку отзвониться
людям, которые за меня переживали. Для понимания - дорога до девятиэтажки это
пять частных домов. Я шла по разбитой дороге и видела толпу людей. Они стояли в
очереди за водой. Я писала ранее, что воду возил сосед на своей машине в
больших бочках. Людей было сотни две. Я не встретила знакомых лиц. Но я
услышала как меня окликнули по имени. Я развернулась и глянула на незнакомую
женщину в платке. Человек из 20 века. Я ее не знала. Мне так казалось. Но она
спросила как я . По голосу я узнала, что это Аня, цветущий человек в мирной
жизни. Сейчас блеклая и серая, старше своих годов. Я думаю, я выглядела не
лучше. Я перекинулась парой фраз и побежала дальше. И это напоминало палату в
дурке. Меня окликали по имени, я узнавала по голосу, но лица были мне не
знакомы. Я поняла, что я не тяну воспринимать такую реальность. Я подошла к
подъезду моей родной девятиэтажки и, набирая номер, окинула взглядом двор. Нет.
Мне не показалось. Не только люди стали мрачными, но и наш двор был засыпан
черти чем. Не было клумб и асфальта. Не было даже бордюров, которые раньше это
все разделяли. Я смогла дозвониться двоим. Потом меня закинул незнакомый мне
мужчина в подъезд со словами "падай". Я не упала. В этом не было смысла. Я
ждала затишья. Мои не знали где я. Я побоялась сказать, но не вернувшись
быстро, я могла подвергнуть их опасности. Как только взрывы стали звучать
дальше, я побежала не видя ничего. Только очередь за водой исчезла. Я добежал
до дома и села на веранде, пытаясь делать вид, что тут и была. Но те, кто
заметили мое отсутствие мне высказали все. Пульс шкалил. Я спряталась от всех и
разрыдалась. Мой город. Мой любимый город. Мои соседи. Милые и родные лица. За
что все это? Это ломает мышление. Я собрала себя в комок и продолжила сохранять
покой в нашем доме. Но так я никогда не рыдала, как рвюыдала после этого выхода
позвонить. Больше я не предпринимала таких попыток. Слишком сложно это
воспринимать

%\ii{31_03_2022.fb.lafazan_natalia.mariupol.2.moja_vtoraja_isterika.cmt}
