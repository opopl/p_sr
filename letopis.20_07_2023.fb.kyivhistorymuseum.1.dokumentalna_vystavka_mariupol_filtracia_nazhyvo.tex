%%beginhead 
 
%%file 20_07_2023.fb.kyivhistorymuseum.1.dokumentalna_vystavka_mariupol_filtracia_nazhyvo
%%parent 20_07_2023
 
%%url https://www.facebook.com/kyivhistorymuseum/posts/pfbid0LLb6yErgmdV6Qpk6RfugF6DnWeFKswRm6EKQuLB5C5sgeoJVdZtXhG6Q826ZWcwJl
 
%%author_id kyivhistorymuseum
%%date 20_07_2023
 
%%tags 
%%title Документальна виставка "Маріуполь. Фільтрація #НаЖиво"
 
%%endhead 

\subsection{Документальна виставка \enquote{Маріуполь. Фільтрація \#НаЖиво}}
\label{sec:20_07_2023.fb.kyivhistorymuseum.1.dokumentalna_vystavka_mariupol_filtracia_nazhyvo}

\Purl{https://www.facebook.com/kyivhistorymuseum/posts/pfbid0LLb6yErgmdV6Qpk6RfugF6DnWeFKswRm6EKQuLB5C5sgeoJVdZtXhG6Q826ZWcwJl}
\ifcmt
 author_begin
   author_id kyivhistorymuseum
 author_end
\fi

У Музеї історії міста Києва представили документальну виставку \enquote{Маріуполь.
Фільтрація \#НаЖиво} Маріупольська міська рада, ЯКультура Маріуполь, M EHUB  

Багато маріупольців пройшли через принизливий і важкий процес фільтрації.
Допити, побиття, роздягання, знущання, збір особистих даних, зникнення людей...

Ціна, яку люди платять при проходженні  процедури фільтрації в окупації. 

Окупанти називають фільтрацією примусові перевірки громадян України на вулицях,
у житлових приміщеннях і на блокпостах. Маріупольці в обов'язковому порядку з
весни 2022 року проходять реєстрацію у спеціальних фільтраційних пунктах. 

На рівні міжнародних організацій вже офіційно визнано, що росіяни шукають не
лише людей, пов'язаних із військовою сферою – 10 000 цивільних маріупольців не
пройшли фільтрацію, а їхні родичі та друзі не мають актуальної інформації про
місцеперебування та стан здоров'я своїх близьких. Велика кількість зруйнованих
сімей,  дітей, яких вивезено на територію росії і їх подальша доля невідома.

\enquote{Забракне усіх слів світу, щоб зрозуміти людей, які пройшли фільтрацію,
та цей проєкт наблизить нас до усвідомлення який важкий шлях пройшли громадяни
України. Неможливо переоцінити значення цього соціального проєкту, адже
\enquote{Маріуполь. Фільтрація\#НаЖиво} лише привідкриває частину злочинів, про
які ми ще дізнаємось. Це біль усієї України й саме тому ми говоримо про це в
столиці у просторі Музею історії міста Києва}, – наголошує генеральна
директорка Diana Popova

Окупанти намагаються знищити тих, хто у своїй країні вважає себе громадянином
України та її патріотом. Країна-агресор не хоче розлучатися зі своїм
тоталітарним минулим, і такі порушення  прав людини необхідно вивчати та
зробити все можливе для припинення цих злочинів.

Ганебний папірець з написом \enquote{дактилоскоповано}, який люди отримали після
проходження фільтрації, став документом, що дає право на існування в
окупованому місті. 

Втрачені долі, зруйновані життя... Поверніть волю, поверніть наших людей!

Виставка триватиме до 6 серпня.

Графік роботи виставки згідно графіку роботи музею: 

Ср.-Нд.: 12:00-19:00, Пн-Вт.: вихідні

Каса працює з 12:00 до 18:30

Вартість вхідного квитка: 80/40 ₴ (повний/пільговий)
