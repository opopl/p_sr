% vim: keymap=russian-jcukenwin
%%beginhead 
 
%%file 27_12_2015.fb.zharkih_denis.1.sssr_stalin.cmt
%%parent 27_12_2015.fb.zharkih_denis.1.sssr_stalin
 
%%url 
 
%%author_id 
%%date 
 
%%tags 
%%title 
 
%%endhead 
\subsubsection{Коментарі}
\label{sec:27_12_2015.fb.zharkih_denis.1.sssr_stalin.cmt}

\begin{itemize} % {
\iusr{Александр Искандеров}
А на каких других основаниях? Чем вам всем коммунизм не угодил? Может вы просто не умеет его готовить? )

\begin{itemize} % {
\iusr{Денис Жарких}
Я ведь не про коммунизм писал, а про коммунистов, просравших СССР.
\end{itemize} % }

\iusr{Денис Жарких}
Очень может быть. В любом случае с рыночным обществом пока не складывается.

\iusr{Светлана Шматок}

Молодец, Денис! Как когда-то говорил мой отец, что не те люди возглавили ГКЧП.
Я тогда очень молодая была, не понимала его боли от развала СССР. А вот он не
пережил...вскоре ушел.


\iusr{Елена Коцарь}
Хорошая аналогия про живую и мертвую воду. Именно что сначала мертвая, потом живая.

\iusr{Евгения Касьяненко}
Нам еще геноцид в Украине пережить надо. До возвращения Союза.

\begin{itemize} % {
\iusr{Денис Жарких}
Закончить
\end{itemize} % }

\iusr{Елена Коцарь}

Увы, судя по всему Украине придется пережить очень трудные времена пока Россия
укрепится настолько, что сможет ее присоединить без риска для себя.


\iusr{Таня Земцова}

Когда сейчас со всех экранов несётся просто вой - коммунист!, ату его! Это
ответственность власти. И пусть Порошенко не косит на Яценюка по любому поводу-
СЯДУТ УСЕ, как говорил известный герой. Моя бабушка была коммунисткой. Студенткой
рыла котлован для ГАЗа, занималась беспризорниками (тогда не было
волонтёрства), мы её помним когда она была преподователем ВУЗа. Честнейший
человек, скромная и немногословная (жили мы в коммунальной квартире). И я горжусь
моей бабушкой-коммунисткой.

\iusr{Андрей Юхарев}

Исторический опыт показывает что Россия либо существует как сильное государство
, либо исчезнет совсем . Без союзных республик сильная Россия вряд ли возможна
, так что хош как хош , или Союз или писец , третьего не дано

\begin{itemize} % {
\iusr{Таня Земцова}

Нет вечных союзов, и ЕС не вечно, понятно уже. Возможно создание нового
союза-каолиции Украина+Турция+Болгария, да мало ли, если умные подумают)+Россия,
как долевой партнёр. Это если успокоиться и подумать о самосохранении а не о
самоуничтожении.

\end{itemize} % }

\iusr{Сергей Курбанов}
мёртвой воды напились уже по гланды
когда будет живая?

\iusr{Nadia Litvin}

Более того, повторюсь: вернётся не только союз, не только коммунизм, но и
монархия. Просто, история повторяется. Относиться к этому нужно спокойно и
мудро), никому ничего не объясняя и не извиняясь. События этих лет можно
спокойно игнорировать, они не переломные)), а просто мутные. Рекомендую даже не
следить за ними)


\iusr{Юрий Николаевич Тявин}
Всё утонуло в фарисействе. Судьба Израиля. \url{https://www.youtube.com/watch?v=28A2yNqAMXE}

\iusr{Юрий Николаевич Тявин}
Денис. Мы с вами живы. а значит и Жив Союз. Умерла теократия.

\iusr{Svetlana Teplaia}

Все было не совсем так: людей, несогласных с режимом, Сталин заставил работать
на благо своей страны. Так что мощь России и репрессии это 2 очень
взаимосвязанные вещи. Без их труда ничего бы не было. Тем более обидно,_ что
это их обокрали и оболгали, забрав и растащив все ими созданное по карманам
олигархов.


\iusr{Павел Никитин}

Великую страну нельзя развалить. А СССР развалился сам. Величие страны
определяется тем, как страна относится к своим гражданам и как эти граждане
живут. Если и тогда и сейчас Россия ничего миру не может предложить кроме газа
и нефти и при этом 99 процентов людей живут неважно это не величие. Это
убогость

\iusr{Александр Костенков}

" Советский Союз дал миру очень много - прежде социальные достижения и
гражданские права, все то чем сейчас бравирует Запад, технологические
достижения в самых передовых отраслях- ракетостроение, космос, атомная
промышленность (вот первое что приходит в голову)." Развал СССР был
закономерен, так же как и его создание. Эксперимент был признан удачным, но
угрожал создателям. И потому был свернут. Про 99\% это просто передергивание
фактов, для тех кто помнит ту жизнь.Жили все гораздо хуже , чем 1\% населения
золотого миллиарда, но гораздо лучше чем процентов 80.

\iusr{Игорь Николаенко}

"Когда его развалили никто не пикнул даже" - а что можно было сделать простым
людям - в знак протеста развалить собственное государство, приказом сверху
закрывшее коммунистический проект? Сравнение с немцами предельно некорректно:
поступил приказ о капитуляции - немецкая армия одномоментно потарахтела на
запад сдаваться англо-американцам. Вот и всё их "до последнего".

\begin{itemize} % {
\iusr{Денис Жарких}
По поводу немцев согласен.
\end{itemize} % }

\iusr{Гринь Сергей}

Не согласен, что закономерно. У СССР была сильная экономика. И доля ВПК и
экспорта нефти не были решающими факторами. К закату СССР пропаганда уже
работала не в том направлении. И никто не пикнул, потому что сначала хорошо
почву подготовили - подорвали экономику, создали дефицит, промыли хорошенько
головы населению - и, вуаля, никто и не рыпнулся. Союз предали свои же элиты,
хотя, какие они элиты, к моменту распада они уже стали псевдоэлитами.

\begin{itemize} % {
\iusr{Денис Жарких}
а с чем тогда несогласен? Я об этом и пишу  @igg{fbicon.wink} 

\iusr{Гринь Сергей}
Наверное, неверно интерпретировал вашу мысль. Закономерен именно результат. )
\end{itemize} % }

\iusr{Татьяна Личенко}

Когда Украину разваливали, Янукович и Ко тоже обделались, как и большинство
граждан, которые "против". Как и в СССР, когда проголосовали за сохранение
страны. Увы((

\iusr{Ирина Нарижная}
"Недаром же в русских сказках оживление делалось не только живой водой, но и мертвой." - мне понравилось. Запомню.

\iusr{Тамара Журавлева}
Путин сказал: *Кто не жалеет о распаде СССР у того нет сердца, а кто мечтает о его восстановлении, у того нет ума*

\iusr{Лада Куликовская}

Сейчас мы можем отлично сравнить - при Сталине победили неграмотность и
отстроили страну, причем рабы бы страну не отстроили - многое можно у нас
сейчас отстроить даже погоняя палками? НИЧЕГО - рабы только разваливать умеют.
Даже один невинно убиенный - смертный грех, убитых было много - только не надо
Сталина винить - все должны оглянуться на себя - кто делал революцию 17-го,
если в 60-х я слышала шепотом, что все мои знакомые из дворян, кто писал доносы
и расстреливал, охранял лагеря и тюрьмы, участвовали в собраниях и пр и
конвоировал- если все кроме Сталина праведники мира?...

\begin{itemize} % {
\iusr{Денис Жарких}
Вроде, Сталин эту самую революцию и делал. Гражданскую так точно. Все неоднозначно.
\end{itemize} % }

\iusr{Николай Лысенков}
Сначала мертвая вода, потом живая.

\iusr{Tatiana Shevchuk}
Ключевое слово СОЮЗ! и во имя чего!

\end{itemize} % }
