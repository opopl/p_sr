% vim: keymap=russian-jcukenwin
%%beginhead 
 
%%file 28_11_2021.fb.bogatyrev_daniil.1.obschestvo_amoralnost.cmt
%%parent 28_11_2021.fb.bogatyrev_daniil.1.obschestvo_amoralnost
 
%%url 
 
%%author_id 
%%date 
 
%%tags 
%%title 
 
%%endhead 
\subsubsection{Коментарі}

\begin{itemize} % {
\iusr{Андрей Рубрук}
Уроды нардепы не показатель.

\iusr{German Viktorov}
\textbf{Андрей Рубрук} ) Наоборот - они ведь представители этого народа, им в свои представители и выбранные.

\iusr{Юрий Шаргородский}
Бытие определяет сознание.

\begin{itemize} % {
\iusr{Даниил Богатырёв}
\textbf{Юрий Шаргородский} Этот тезис ошибочен. Обратитесь к моим прошлым публикациям.

\iusr{Всеволод Степанюк}
\textbf{Даниил Богатырёв} Вы опять же не понимаете диалектику. Вы действительно считаете, что образование, культура и образ жизни не влияет на формирование сознания? По моему отрицать влияние материального окружения на формирование сознания человека глупо. Для справки, формирование сознания и создание сознания это вообще из разных разделов науки.

\iusr{Даниил Богатырёв}
\textbf{Всеволод Степанюк} "Влиять" и "определять" - это разные определения. Влияет ли? - Да, несомненно, влияет. В ряду других факторов. Но не более того. Определяет ли? - Нет, не определяет, так как не является определяющим фактором при формировании сознания отдельного индивида и/или общества.

\iusr{Всеволод Степанюк}

Вопрос конечно интересный грехопадение Адама и Евы повлияло на их судьбу или
было так определено с самого начала? Факторы всегда влияют, а среди них есть
определяющие и второстепенные. Например, определяющие это где и в какой семье
родился, какие базовые способности душа при рожджении имеет. А вот как всё это
человек использует зависит исключительно от его воли и окружающего мира. Вы
наверное сторонних фатализма, но это достаточно узкое течение. Но если почитать
духовные практики и святое письмо, то именно человек своими поступками
определяет своё будущее. Христос учил, что если поверишь в Его отца и Его
царствие, то будешь спасён, если нет то будешь сам отвечать за грехи свои. "Я
стою у двери каждого и стучу, кто услышет, тот откроет" Евангелие. Если брать
Индуизм и колесо Сансары, то именно закон кармы (поступки) влияют на дальнейшую
судьбу человека. Так что именно от человека и его мировозрения и духовности,
формируется его сознание. Если не будет материального мира, то человек
останется тёмным, он даже Евангелие не прочитает и не узнает о Боге. Так что
бытиё (в широком смысле не только в материальном), определяет сознание.

\iusr{Даниил Богатырёв}
\textbf{Всеволод Степанюк} 

Я - не сторонник фатализма. То, "какие базовые способности душа при рожджении
имеет" - это, как раз, сознание, а не бытие. Бытие, как его понимали марксисты
(ведь сам этот ложный лозунг из их концепции) - это совокупность социальной
среды, ролей в ней и материального окружения.

"Если не будет материального мира, то человек останется тёмным" - странное
утверждение. По этой логике получается, что "темны" ангелы и сам Бог, так как
они не имеют физического ("материального") тела. Но это, разумеется, не так.

Материальный мир или среда обитания - это всего лишь то, что оформляется и
преобразовывается божественным и человеческим сознанием. Не более. Глупо
говорить о том, что бытие, понимаемое как место в этом мире в конкретный момент
времени, всецело определяет сознание индивида или социальной группы. В
противном случае, все люди, формировавшиеся как личности в определённых
условиях (например, нищеты, богатства и т. п.) были бы наделены одинаковым
сознанием, одинаковыми способностями, задатками и, в конечном итоге,
мировосприятием.

\iusr{Всеволод Степанюк}
\textbf{Даниил Богатырёв} 

Возражу Вам с Ваших же позиций. Во-первых бытиё это не только место конкретного
человека в мире, во вторых. Бог вначале создал материальный мир, а только затем
человека, фактически материальный мир создан для человека. Кришна учил Арджуну,
что мир создан для человека и не имеет другого смысла, как быть использованным
человеком. (да, зелёным это не понравится). Именно свободная воля, которую дал
Бог, посредством бытия и формирует сознание человека, ибо познаёт человек, в
том числе и Бога, через созданый Им, материальный мир. Материя это первая
ступень познания, но не пройдя её, человек никогда не познгает мир духовный.
АНгелы и Бог не живут в 4-х мерном мире. Мир N-мерен.

\iusr{Даниил Богатырёв}
\textbf{Всеволод Степанюк} 

Всё то, что Вы только что написали, не имеет отношения к дурацкому
марксистскому лозунгу о том, что "Бытие определяет сознание", так как он
вульгарно-материалистичен. Я согласен с Вами, что бытие - это не то, что
понимают под ним марксисты. Мы можем здесь долго говорить, например, об
описанной Хайдеггером разнице между бытием и сущим, и т. п. Но к
вульгарно-материалистическому контексту, в котором употребляется лозунг "Бытие
определяет сознание", это не имеет никакого отношения. Данный лозунг ложен
именно по причине контекста. Контекст определяющ.

Ну, а если обсуждать проблему вне контекста, то даже приведённые Вами тезисы
говорят о том, что материальный мир был создан для ИСПОЛЬЗОВАНИЯ человеком.
Следовательно, он имеет служебную, утилитарную функцию. Сознание человека
призвано его использовать, преобразовывать, и отчасти - преодолевать. Но не
бывает такого, чтобы человек родился безо всякого сознания и сформировался
исключительно под воздействием социальной среды или материальных обстоятельств.


\iusr{Всеволод Степанюк}
\textbf{Даниил Богатырёв} Маркс исследовал только одну сторону бытия - социально-экономическую и в рамках этой науки он был совершенно прав. Это Вы почему то не правильно его обобщаете, Маркс не обобщал.

\iusr{Всеволод Степанюк}
\textbf{Даниил Богатырёв} Кстати, бывает рождение человека без сознания, как и потеря его в процессе жизни, можете у врачей профильных спросить.

\iusr{Юрий Шаргородский}
\textbf{Даниил Богатырёв} Это не тезис. Это - аксиома. А Ваши публикации в ТЕОРЕТИЧЕСКОМ плане весьма беспомощны.

\iusr{Даниил Богатырёв}
\textbf{Всеволод Степанюк} 

Это не я обобщаю, это сами марксисты обобщают, выстраивая на его учении теорию
"диалектического материализма", который они, почему-то, называют наукой. Что же
до медицинских патологий, некие формы сознания есть даже у врождённых кретинов
и микроцефалов. Впрочем, это - явное отклонение и тут я с Вами не спорю.

\iusr{Всеволод Степанюк}
\textbf{Даниил Богатырёв} 

Правильно, диалектический материализм есть частью философии, изучающий
диалектические процессы материального мира, именно по отношению к этой науке и
было применено правило "бытиё формирует сознание" и оно абсолютно верно для
материальных отношений. Маркс никогда не претендовал на абсолютность своего
утверждения, как и Дарвин, никогда не говорил и не писал об эволюционном
происхождении человека, он лишь писал об естественном отборе в процессе
межвидовой конкуренции. Остальное додумали люди в политических целях. Так и с
учением Маркса.

\end{itemize} % }

\iusr{Полина Дьяченко}

Читая новости отовсюду кажется что по всему миру такое происходит. Чего стоит
насильственное навязывание лгбт. А погромы н###ров в Америке. Украинцы в тренде
если можно так сказать.

Человечество глобально сбивают с пути и ведут на другой путь, с извращенным
пониманием добра и зла. На какой, зачем и кому это будет выгодно? Далее
размышления приводят к конспирологии и церковному(

\iusr{Якубовский Юрий Георгиевич}

Все это есть в России. Кучма нагло врал

\begin{itemize} % {
\iusr{German Viktorov}
\textbf{Якубовский Юрий Георгиевич} Нет, такого массового лицемерия в России нет.
\end{itemize} % }

\iusr{Владислав Антипов}

\obeycr
В Раде форменный бардак,
Слуги гуглят лядей,
Гео Лерос кажет фак
Хулиганства ради!
\restorecr

\iusr{Александр Карпец}
Во-во! Обильный переизбыток массового недоумка, порождающего соответствующую элиту.

\iusr{German Viktorov}
Просто украинская культура избавилась от принуждения угнетателей и с каждым годом заявляет о себе всё уверенней и свободней...

\begin{itemize} % {
\iusr{Александр Карпец}
\textbf{German Viktorov} Это мировая тенденция

\iusr{German Viktorov}
\textbf{Александр Карпец} Нельзя смотреть на мир через украинскую призму - изображение будет искажено до неузнаваемости ) В Украине вполне конкретные тенденции, собственного разлива, в том числе и с культурой. Каков бы не был этот СССР, но оставшись без него, мы встали на четвереньки и стали пятиться назад, к истокам, так сказать и для нас это вполне гармонично и естественно.

\iusr{Александр Карпец}
\textbf{German Viktorov} Повторю то, что только что написал у себя на странице в комментах в дискуссии с Сергеем Удовиком:
"сейчас трэш - это мировой тренд, как модно говорить на современной придурочном новоязе" ))

\iusr{German Viktorov}
\textbf{Александр Карпец} По моим личным наблюдениям, мы, в массе, очень примитивны, относительно других народов, по крайней мере, тех стран, которые мы называем развитыми, поэтому и выхватываем у них выборочно только то, что можем себе как-то объяснить.

\iusr{Александр Карпец}
\textbf{German Viktorov} По моим наблюдениям, в так называемых развитых странах в мессе примитива бывает поболее нашего. Это правда...
Не низкопоклонствуйте перед Западом, как правильно говорили при товарище Сталине, уважаемый ))

\iusr{German Viktorov}
\textbf{Александр Карпец} Примитив бывает везде, но у нас он преобладает и является нормой. В общем, тут у нас с Вами разное видение...

\iusr{Александр Карпец}
\textbf{German Viktorov} Опыт, сын ошибок трудных... ))
\end{itemize} % }

\iusr{Андрей Бока}
Украинское общество скатилось глубоко в пучину аморального поведения
==================
Это кремлёвские нарративы.  У нас население предельно морально-духовное, ибо
церквей масса, население молится, крестится и чтит все религиозные праздники.

\begin{itemize} % {
\iusr{German Viktorov}
\textbf{Андрей Бока} Мы очень лицемерны, в массе. Вряд ли ещё кто-то может в этом с нами соперничать )

\iusr{Андрей Бока}
\textbf{German Viktorov} Ложь и лицемерие - визитная карточка капитализма.
Святые европейцы такие же внутри гнилые.

\iusr{Natalya Varetskaya}
\textbf{Андрей Бока} То, что европейцы такие - нас оправдывает?

\iusr{German Viktorov}
\textbf{Андрей Бока} ) До нас им в этом далеко, очень.

\iusr{Андрей Бока}
\textbf{Natalya Varetskaya} Конечно же не оправдывает.
Просто констатирую, что мы не хуже и не лучше других.

\iusr{Андрей Бока}
\textbf{German Viktorov} До нас им в этом далеко, очень.
===================
Это Вам только так кажется.
Они научились маскировать лицемерие за эвфемизмами.

\iusr{German Viktorov}
\textbf{Андрей Бока} ) Ну конечно, мы такие же, только ценности у нас разные, поэтому и наша жизнедеятельность явно отличается ) Да и с нами же и избираемыми Лидерами нам просто никак не везёт, почему-то ) Возможно, те, кто распределяет этих Лидеров по разным странам, нас недолюбливают, по каким-то причинам ))

\iusr{Андрей Бока}
\textbf{German Viktorov} Возможно, те, кто распределяет этих Лидеров по разным странам, нас недолюбливают, по каким-то причинам ))
==================
В метрополиях нужен один тип говорящих голов.
В колониях нужен другой тип говорящих голов.
Поэтому и разница.
 @igg{fbicon.smile} 

\iusr{German Viktorov}
\textbf{Андрей Бока} 

А самим украинцам что-нибудь вообще нужно, чем они сами тут постоянно
занимаются, в собственной-то стране ? Ждут, когда им кто-то и почему-то,
каким-то образом наладит жизнь, к их удовольствию... С таким собственным
подходом, ничего принципиально другого, кроме того, что тут и есть, получится
никак не может...


\iusr{Андрей Бока}
\textbf{German Viktorov} А самим украинцам что-нибудь вообще нужно
=============
А население нигде, никогда не решает НИЧЕГО.

Вся прелесть буржуазной представительной демократии в том, что выборов-то, как
таковых, и нет: население выбирает из заранее заботливо определённого списка
"кандидатов".

Поэтому лично я с 1999 года не хожу ни на какие выборы.

\iusr{German Viktorov}
\textbf{Андрей Бока} 

Выбор из кандидатов всегда был и есть и он вполне однозначно сказывается на
условиях жизни населения. Например, в зависимости от делегирования во власть
ЮВТ, Зе или Медведчука (ОПЗЖ), с соответствующей поддержкой в ВР, условия жизни
населения вполне ощутимо были бы разные, в частности и в связи с разной
тарифной политикой, в каждом случае. Можно же ведь избирателям влиять на свою
жизнь, вполне прагматично занимаясь своими непосредственными и вполне
конкретными интересами, поддерживая политиков, наиболее близко эти интересы
выражающими и имеющими реальные возможности претворять их в жизнь, а не ждать
откуда-то манны небесной, то есть Чуда...

\iusr{Андрей Бока}
\textbf{German Viktorov} Можно же ведь избирателям влиять <...> поддерживая политиков, наиболее близко эти интересы выражающими
==================
Наивность несусветная.
ЛЮБОЙ политик реализовывает только интересы того, кто водрузил этого политика в кресло.

\iusr{German Viktorov}
\textbf{Андрей Бока} ) Это просто невозможно ) Ну а "водружает" то кто ?) Не хочет "водрузить", а "водружает", своей поддержкой, как и делает политиком, также своей поддержкой ?! )

\end{itemize} % }

\iusr{Анна Акулова}
Вы правы. Деградация.

\end{itemize} % }
