%%beginhead 
 
%%file 19_03_2018.fb.fb_group.750_hudozhnykiv.1.ja_vzhe_tyzhden_ne_splju
%%parent 19_03_2018
 
%%url https://www.facebook.com/groups/1013559558679300/posts/1613861635315753
 
%%author_id fb_group.750_hudozhnykiv,violett_obryvko
%%date 19_03_2018
 
%%tags fb.hashtag.#750художників
%%title Я вже тиждень не сплю... Признавайтесь: кому дістався зайцемонстрик??
 
%%endhead 

\subsection{Я вже тиждень не сплю... Признавайтесь: кому дістався зайцемонстрик??}
\label{sec:19_03_2018.fb.fb_group.750_hudozhnykiv.1.ja_vzhe_tyzhden_ne_splju}
 
\Purl{https://www.facebook.com/groups/1013559558679300/posts/1613861635315753}
\ifcmt
 author_begin
   author_id fb_group.750_hudozhnykiv,violett_obryvko
 author_end
\fi

Я вже тиждень не сплю... Признавайтесь: кому дістався зайцемонстрик?? 🤔

%\ii{19_03_2018.fb.fb_group.750_hudozhnykiv.1.ja_vzhe_tyzhden_ne_splju.cmt}
