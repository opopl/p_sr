% vim: keymap=russian-jcukenwin
%%beginhead 
 
%%file 12_12_2015.fb.bilchenko_evgenia.1.patrioty_gorlovka_uchitel_vojna
%%parent 12_12_2015
 
%%url https://www.facebook.com/yevzhik/posts/919515381416906
 
%%author Бильченко, Евгения
%%author_id bilchenko_evgenia
%%author_url 
 
%%tags bilchenko_evgenia,donbass,gorlovka,patriot,ukraina,vojna
%%title БЖ. «Патриоты», вот с таким врагом вы боретесь?
 
%%endhead 
 
\subsection{БЖ. «Патриоты», вот с таким врагом вы боретесь?}
\label{sec:12_12_2015.fb.bilchenko_evgenia.1.patrioty_gorlovka_uchitel_vojna}
\Purl{https://www.facebook.com/yevzhik/posts/919515381416906}
\ifcmt
 author_begin
   author_id bilchenko_evgenia
 author_end
\fi

БЖ. «Патриоты», вот с таким врагом вы боретесь?

Учительница Татьяна из Горловки: 

Им все равно, люди там или деревья. И одним и другим.

Я понимаю, что вам теперь можно всё. Потому что вы – победители. Потому, что вы
пришли вторыми после Небесной Сотни. Потому что, когда мой студент перевязывал
рану, которую он сам же нанёс беркутовцу, вы сидели дома, за компьютерами, в
тюрьмах, в научных заведениях, по заграницам, еще хрен знает где. 

Потому что, когда надо было ездить читать в окопы нашим пацанам, вы все были
очень заняты делом восславления Украины, а те, кто ездил, – а я их по пальцам
могу пересчитать, и не себя я имею в виду, заткнитесь и успокойтесь, – те
скромно молчат. 

А теперь у вас, оказывается, появилось право на Родину! Да еще какое право!
Большое, как член. Еще бы: за вами – власть. За вами – доносы, наводки, поиски
врагов народа, мастурбация на лживые СМИ. Кто-то из вас всю жизнь ходил в
церковь Московского патриархата, − но отрекся от своей конфессии, когда она
стала не модной. Я не кричу «Ура», что Лавру забирает Киевский патриархат, хотя
крестилась я в Автокефальной церкви в Западной Украине. Патриоты! Кто-то
по-русски на ушко мне хвалит Советский Союз и презентует свои графоманские
вирши в сине-желтом Украинском доме. Я считаюсь «поэтом Майдана», но ни в какие
«дома» со своими, наверное, не худшими стихами не лезу. 

Патриоты! Кто-то в ответ на пост мирной женщины из Горловки о гибели под
градами детей, пишет: \enquote{Чем докажешь?} Или, что еще хуже: «Так вам и надо!»

Патриоты! Мне тошнит от вас, «патриоты». Мне тошнит от вас потому, что вы
Россию ненавидите больше, чем любите Украину. Мне тошнит от вас потому, что
ваши дети не похоронены в огородах, а учатся за ваши деньги в столичных вузах.

Мне тошнит от вас, когда вы орете об украинском языке – единственном языке, на
котором я плачу, когда такие, как вы, меня достают. Мне тошнит от вас, потому
что за вас на Донбасс убивать идут мальчики из Тернополя, которых мы одеваем и
обуваем. 

Мне тошнит от вас потому, что вы взяли на вооружение наше знамя Свободы,
сделали из него кровавую тряпку, во имя которой можно мочить всех, кто не похож
на нас. Вчера я взяла интервью у человека, который живет в Горловке (ДНР) и
ненавидит Майдан. Так вот, «патриоты», она - мой народ. Потому что в ней,
несмотря на то, что она ненавидит Майдан, больше человеческого, чем в вас. А
Майдан мы задумывали как Человечность. Пока не пришли вы, «патриоты».

Интервью с Татьяной, полная версия в ЖЖ: \url{http://bilchenko-e.livejournal.com/2082.html}

\begin{itemize}

\item Татьяна: Здравствуйте, Женя! Прочитала Ваш последний пост. Тронул. Увидела в
нем себя. Спасибо. П.С. Майдан - ненавижу, что бы не возникало иллюзий.

\item БЖ: 

А зачем вы как будто оправдываетесь? Я оцениваю людей не по тому, любят или
ненавидят они Майдан. Ценю в людях сердечность. Таня, если вас не затруднит,
узнайте, кому нужна помощь, из мирных жителей, и скажите (может, я много прошу)
мне правду, пожалуйста. В вас стреляют наши? Вы не бойтесь, я не выдам вашего
имени: мне надо знать как писателю и человеку, который еще честь не потерял в
этом свинюшнике.

\item Татьяна С.: 

Я не боюсь. Я в открытую говорю, военные играют в пинг-понг, с
одной стороны ВСУ, с другой ВСН. Мы - декорации - куда попадет, туда и попадет.
Так как ВСН находятся с нашей стороны, а ВСУ с другой, то в нас стреляют ВСУ. Я
в этом абсолютно уверена. ВСН тоже стреляет, но в обратную сторону, в сторону
Дзержинска, Артемовска, только там жертв и разрушений в разы меньше. Горловка −
рекордсмен по погибшим детям, их 17. ВСУ стреляет по инфраструктуре города, а
так как котельные, трансформаторы и т.д. находятся в жилых дворах, то попадают
очень часто и по жилым кварталам.

\item БЖ: Дети среди погибших от выстрелов? Научите меня, как разбираться, кто стреляет: ВСН или ВСУ. Как вы поняли, кто это? 

\item Татьяна: Да, 17 детей, погибших в обстрелы. Вы не сможете разбираться, если Вы живете не здесь.

\item БЖ: 17 детей за какой срок?

\item Татьяна: 17 детей с 27 июля 2014 года, то есть за горячий период

\item БЖ: И еще: если вы не против, я могу использовать вашу информацию, чтобы
приводить в чувство здесь людей? Спасибо, что не послали.

\item Татьяна: Та ну, чего я посылать буду. Ну, вот пример первый: моя подруга сидит
в Артемовске, я в Горловке. Она мне пишет в ФБ - залп, я через 6-8 секунд пишу
- принял.  И так час, два, три. Дело не в логике, что ВСН по себе стрелять не
может [ответ на мою реплику – БЖ]. Если будет нужно, и ВСН будет стрелять по
городу, только это физически почти невозможно. Я показывала В. [наш волонтер
−БЖ], специально возила и показывала на месте - как определяем откуда стреляют.
Вот моя подруга живет на 9 этаже в крайнем доме. С одной стороны поле, с другой
микрорайон. В ясную погоду она смотрит вдаль и видит позиции ВСУ. До них
несколько км, и вечером оттуда начинают лететь снаряды. Когда стреляют –
вспышка, и ночью летящий снаряд видно, он светится. И днем она видит позиции
ВСУ, а ночью оттуда идут вспышки и летят снаряды. Далее. У артиллерии есть не
только максимальная дальность, но и минимальная. И ближе, чем например 5 км -
не выстрелишь в цель. А ВСУ находятся в некоторых районах ближе, чем 5 км. И
ВСН даже если выедут в поле и развернутся - не могут выстрелить на 2 км. Потому
что должно быть минимум 5. А ВСУ стоит через 4 км, например. И стрелять с 5 км
могут только они. Или еще пример. В мой огород 8 августа 2014 г. прилетел
снаряд града. В этот день было 2 залпа, только два. И в 11.00 в мой огород
прилетел снаряд града. Через полчаса позвонил мой знакомый, коллега, офицер в
отставке, он в этот момент был под Дзержинском и ЛИЧНО видел как стрелял Град в
11.00. И он у меня спрашивает, куда попали из Града в 11.00. Они попали в мой
огород в том числе. А стреляли полным пакетом по подстанции, возле которой мы
живем. В 11.00 других залпов не было. И в тот день был только еще в 9.00, в
другую часть города. Как я могу сомневаться, если в 11.00 выстрелили в
Дзержинске и человек лично видел, а через 6-8 секунд эти снаряды упали в моем
огороде и на моей улице? И ГРАДы были ВСУ. Других в Дзержинске не было. А когда
идет арт-дуэль, тем более если идет часами, то ты уже не определяешь, откуда
стреляют. Ты только взрывы от залпов отличаешь.

\item БЖ: Вы считаете, что ВСН и ВСУ одинаково могут стрелять по людям?

\item Татьяна: Конечно. Они военные, у них задачи, мы - декорации, которые им мешают.
Они стреляют не по людям, а по объектам. И если там оказываются люди, то это их
проблема, а не ВСУ или ВСН. Им все равно, люди там или деревья. И одним и
другим.

\item БЖ: Военные из ВСН - россияне или местные? Вы россиян видели?
\item Татьяна: Регулярных российских войск я не видела ни разу. Добровольцев полно.
\item БЖ: Больше, чем местных ополченцев?
\item Татьяна: Не могу сказать, больше или нет. Но они отличаются от местных.
\item БЖ: А чем?
\item Татьяна: На наших одежда из магазина – «Все для рыбалки», и брюшко висит, и оружие не знает куда девать, и все разномастное. Это как меня одеть в камуфляж, но я же не буду от этого Рембо. Не местные - они выглядят как наемники, на них одежда сидит как вторая кожа, одежда хорошая, оружие - как вторая рука. И они могут быть местные или не очень местные, но они профессиональные воины, это видно.
\item БЖ: А кто лучше всех к людям относится: украинцы, россияне или местные ополченцы?
Татьяна С.: К каким людям?
\item БЖ: К вам. Или вы с ними ни с кем не общались? Они же должны понимать, что стреляют по людям. Или вообще не понимают?
\item Татьяна: К нам? Украинцев здесь нет, с россиянами мы не контактируем, ополченцы – ну, люди и люди. Мы с ними вообще никак не контактируем. Ну, ходят по улицам, ну, на рынке, с семьей, скупляются. Местных много, моих бывших учеников до 10 в ополчении. Родители детей - знаю тоже таких, кто в ополчении. Ну, люди и люди.
\item БЖ: А сейчас стреляют?
\item Татьяна: Прямо сейчас идет перестрелка на окраинах города. Мне не слышно, я на другом конце. Но в местных сетях пишут, что идет стрельба из мелочи.
\item БЖ: ВСУ понимают, что стреляют в женщин и детей из Градов? Потому что многие наши люди здесь просто не понимают. Как забрало на глазах.
\item Татьяна: Ну, наверное, же понимают. Если в 16.00 произведен единственный выстрел по Горловке, и в 16.00 убивает отца и дочь, а матери отрывает руку. И других выстрелов больше не было. И сети, СМИ шумят и обсуждают. Тот, кто стрелял, понимает, что это он убил ребенка и ее отца? Думаю, что понимает. Как он к этому относится? Не знаю, я не убивала людей. Если признаться в том, что понимаешь, что ты, или твой сын, или муж стреляет по живым мирным людям, то значит взять на себя ответственность за то, что они являются убийцами. Не каждый на это согласиться. Поэтому проще сделать вид, что ты не знал, что не понимаешь, что мы все врем или сами виноваты - так намного проще оправдать убийство детей.
\item БЖ: Можно предположить, что ВСУ не видят - если расстояние 5 км - они же стреляют в «объект»?
\item Татьяна: Артиллерия - очень точная наука. Я знаю, что тот, кто стреляет, знает, что стреляет в спальный район. Во всяком случае, они так говорят.
\item БЖ: Зачем они стреляют в спальный район?
\item Татьяна: Причины три. 1. Разрушение инфраструктуры города для того, чтобы выжать отсюда людей и создать трудности местным властям. 2. Ответ на выстрелы со стороны ВСН, которые бывает, что стреляют из жилых районов. Только град ВСН отстрелялся во вторник и через 5 минут уехал, а ВСУ ответили градом в четверг по тому же месту. Зачем? Не знаю. 3. Кошмарят население просто так, для подрыва морального духа. Бывало такое, что ведут обстрел поселка до тех пор, пока не разнесут в щепки трансформаторную будку, сразу после этого обстрел поселка прекращается надолго.
\item БЖ: Спасибо - это очень системно. Вы учитель?
\item Татьяна: Я − учитель истории.
\item БЖ: Это заметно. Я - философии.
\item БЖ: Таня, это будет ужасно, если я попрошу у вас разрешения опубликовать этот разговор у себя в блогах как интервью? Я ничего менять не буду: фамилию скрою, вашу, конечно, мне влетит, но мне все равно уже.
\item Татьяна: Да можно, конечно. Спрашивайте. Согласна. Подумайте что Вы хотите спросить, я отвечу максимально честно.
\item БЖ: Почему вы не любите Майдан?
\item Татьяна: Это очень глубокий вопрос. Я не отвечу в одном - двух предложениях. И болезненный. Вообще я предпочитаю о Майдане не говорить, и если обсуждать, то отдельной темой.
\item БЖ: Без вопросов.
\item Татьяна: Я могу рассказать об обстрелах много и по-разному. Я дважды была
под обстрелом непосредственно, когда летело над головой. Пережила месяц летом и
месяц зимой, когда из дома не могла выйти вообще. Две недели сидела в погребе.
В августе прошлым летом. Попала под обстрел, лежала головой в асфальт, лицо
обожгла об асфальт, так в него вжималась. Это издалека градами стреляли. В тот
день у меня умер папа, от инфаркта. Он ребенок войны. И мы думали, что будем
хоронить его в огороде. Потому что стреляли не прекращая. И когда все-таки
похоронили, то за 5-7 минут. И я не могла плакать полгода. Летом в городе
оставалось 10\% населения. И мы сидели дома с 7 собаками: 4 наши и 3 соседских
ротвейлера, и самым большим ужасом было то, что завтра их нечем будет кормить.
На весь город (270 тыс населения до войны) работало 5-6 магазинов полчаса в
день. И еды реально было негде взять. И муж под обстрелами бегал, искал еду
собакам, так было 6 недель. Соседи уехали тогда в Бердянск, по моему, и не
могли вернуться. Их не пускали из-за обстрелов. Было такое, что муж для компота
за фруктами в нашем же дворе выходил 4-5 раз, с первого раза не получалось
дойти до яблони, потому что стреляли. Выйдет, назад заскочил, выйдет, присел.
Было такое, что завтракали в 16.00, до этого нам не давали поднять головы.
Потом отключили свет и воду. Света не было 11 дней, воды 14 дней. Перебили
водоканал (у нас вода по трубам в город поступает по большим) и повредили
подстанции в обстрелы. И мы просто сидели и тупо смотрели в стену и слушали
обстрелы. Нет мобильной связи вообще. И было жутко, что город вымер. Казалось,
что люди больше не вернутся. И ты один на улице, нет света, нет воды, нет еды,
нет людей, ничего не работает вообще. И казалось, что так наступает конец
света.
\item БЖ: Я могу себе представить...
\item Татьяна: Нет, не можете. Зимой обстрелы были серьезнее 21 января, накануне Дня соборности обстрел длился 18 часов, не прекращаясь. И так продолжалось месяц. Но был свет, газ и вода. И интернет Весной были дни, когда гуманитарный обед в школе был единственныс разом, когда я кушала за весь день. В школе
\item БЖ: Не было еды?
\item Татьяна: Да, не было денег, что бы ее купить. Яблоко хоть какое-нибудь было бы за счастье.
\item БЖ: это уже при ополченцах было или при украинской власти?
\item Татьяна: При ополченцах. Это весной 2015 года Сейчас нам платят зарплату и у нас есть еда. В магазинах почти все есть.
\item БЖ: Это сейчас же ополченцы платят?
\item Татьяна: С сентября 2014 были нерегулярные выплаты. С мая – ополченцы начали выплачивать зарплату в рублях регулярно, месяц в месяц. Украина нас всех уволила июлем 2014. Я в августе сидела и ждала ВСУ, а мое государство меня уволило еще в июле.
\item БЖ: Надеялись на его помощь, ВСУ?
\item Татьяна: Я в сентябре ездила под обстрелами прятала Конституцию Украины и учебники по истории Украины, а МОН уже 1 сентября издало указ, по которому я считалась пособником оккупантов. В августе я знала, что стреляют ВСУ, и все равно их ждала. Но после череды предательств своим государством, ожидание сходило на нет.
\item БЖ: А почему тогда ждали, ополченцы не нравились?
\item Татьяна: Ополчение здесь появилось в неоформленном виде с марта - апреля, это были местные с дубинками. В июле, когда из Славянска выпустили Гиркина и он 8 часов шел по полям и их не тронули - вся эта колонна, несколько километровая ввалилась в город - с этого момента ополчение было с оружием. Мне ДНР не нравилось раньше активно. От рос. флагов тошнило.
\item БЖ: понимаю. Но получается, что украинские силы подвели? Обстрелами?
\item Татьяна: Нет. Началось не с обстрелов. Началось еще в феврале с отмены Закона о языках
\item БЖ: Я так и думала.
\item Татьяна: Это уже было предательство. Предательство меня моим государством было тогда, когда не защитили меня как гражданина от майдана, от титушек, от российских понаехавших, от ДНР
\item БЖ: от всех то есть?
\item Татьяна: Потом все эти высказывания в адрес Донбасса, в адрес дончан, разжигание розни.
\item БЖ: Это да, от этого тошнит
\item Татьяна: Если я выполняю свои обязанности перед государством, почему государство этого не делает?
\item БЖ: Общественный договор Вольтера не работает
\item Татьяна: Потом чем дальше, тем больше. Я виновата в том, что граница оказалась не на замке?
\item БЖ: Нет, конечно.
\item Татьяна: Я виновата в том, что все областное и городское начальство спонсировало, руководило, прикрывало ДНР? Я виновата в том, что Гиркин и Бес ворвался в мой город?
\item БЖ: Нет! Это (не хотела прибегать к медиа-жаргону) зомби-вышиватники так считают, а ведь учили их свободе думать: как в сито. 
\item Татьяна: И все равно я терпела и ждала ВСУ, даже зная, что они в меня стреляют. Зная, что мой отец умер, во время обстрела ВСУ. Но начиная с сентября, когда от нас начали отказываться официально - я не могла больше дальше прощать.
\item БЖ: Когда уволили? Когда лишили статуса, да?
\item Татьяна: Процесс завершился в январе-феврале 2015 г, когда просто часами молотили по городу из всех видом оружия.
\item БЖ: Понятно.
\item Татьяна: Да, когда лишили статуса, когда объявили меня пособником агрессора. И все это на фоне комментов патриотов Украины. Женя, у ополченцев оружие, очень много оружия. Если ВСУ и спецслужбы не могут противостоять, что могу сделать я, безоружная женщина. Я и не должна это делать. Мое дело - учить детей, я это делаю хорошо. За то, чтобы меня защищали - я платила всю жизнь налоги. Украинские патриоты сами своими комментами лишили Украины поддержки здесь. Обстрелы только завершили этот процесс.
\item БЖ: Вас сильно комментарии задевают?

\item Татьяна: Меня да, иногда выводят из жизни. Был момент, когда убили
родственников моей ученицы. И мама ее, родная тетя убитой девочки и сестра
убитого мужчины - звонит мне и в онлайн, в тот момент, когда их откапывали из
под обломков, когда думали, что погибло трое детей - она мне это все в шоковом
состоянии рассказывала. И я это пишу онлайн в сетях, а патриоты начинают
издеваться - вы все врете, а чем докажете, а фамилии подайте нам.

\end{itemize}

\ifcmt
  pic https://scontent-mia3-1.xx.fbcdn.net/v/t1.18169-9/12369152_919515288083582_5151234514994831896_n.jpg?_nc_cat=111&ccb=1-3&_nc_sid=730e14&_nc_ohc=e2YqONch2LkAX-FvCsG&_nc_ht=scontent-mia3-1.xx&oh=180882979d3840afdb8f4652ec7dcc29&oe=60E6F62F
  width 0.4
\fi
