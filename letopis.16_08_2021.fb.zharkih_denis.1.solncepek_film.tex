% vim: keymap=russian-jcukenwin
%%beginhead 
 
%%file 16_08_2021.fb.zharkih_denis.1.solncepek_film
%%parent 16_08_2021
 
%%url https://www.facebook.com/permalink.php?story_fbid=3056037771276286&id=100006102787780
 
%%author 
%%author_id zharkih_denis
%%author_url 
 
%%tags donbass,film,film.solncepek.donbass,kultura,vojna
%%title Посмотрел фильм "Солнецепек", спасибо френдам скинули
 
%%endhead 
 
\subsection{Посмотрел фильм \enquote{Солнецепек}, спасибо френдам скинули}
\label{sec:16_08_2021.fb.zharkih_denis.1.solncepek_film}
 
\Purl{https://www.facebook.com/permalink.php?story_fbid=3056037771276286&id=100006102787780}
\ifcmt
 author_begin
   author_id zharkih_denis
 author_end
\fi

Посмотрел фильм "Солнецепек", спасибо френдам скинули. Фильм основывается на
факте, который невозможно опровергнуть - обстрел ВСУ и добробатами мирных
жителей Донбасса, который привел к многочисленным жертвам. Тут украинской
власти крыть нечем, я до сих пор не понимаю, кто и зачем отдавал такие приказы,
и кто эти приказы исполнял. 

\ifcmt
  pic https://scontent-cdg2-1.xx.fbcdn.net/v/t1.6435-9/238829610_3056037754609621_6351784724273376431_n.jpg?_nc_cat=108&ccb=1-5&_nc_sid=730e14&_nc_ohc=6oih99hvFNQAX8HvFBT&_nc_ht=scontent-cdg2-1.xx&oh=39434830f9b3f6bc2bfdbb94b589a74d&oe=61410AD0
  width 0.4
\fi

В фильме дается разгадка - это американцы надоумили. Ну ладно, допустим
американцы, но кто-то исполнял, кто-то этим очень гордился. Сам механизм
дегуманизации жителей Донбасса намного сложнее, в фильме все упрощено, что
может быть допустимо, поскольку главную задачу, а именно показать
несправедливость обстрелов, фильм выполняет. 

В фильме американцы однозначно плохие, а украинцы, в целом, хорошие, даже те,
кто стрелял по мирным жителям. Последние или осознают свою вину и отправляются
домой, либо пускают заряд в голову. Даже заправских фашистов по законам
индийского кино убивают не ополченцы, а уголовники. Да и уголовника, на  руках
которого кровь целой деревни, ополченцы еще некоторое время испытывают, думают,
что с ним делать. Короче, в ополченцах есть что-то от гимназисток, хотя война к
подобным пассажам явно не располагает.  Да и пленных они держат в больнице, а
не на подвале или в тюрьме. 

Линия бандитов вообще интересна, именно с нее начинается фильм. Вот они убивают
мирных жителей, не щадят ни женщин, ни детей. Вопрос - а зачем это им? Что
возьмешь с небогатых жителей, а убийство в уголовном мире дело серьезное, точно
так же, как изнасилование женщины и убийство ребенка. Ну, ладно, допустим
жителей казнят по политическим мотивам, тогда зачем нападать на пост добробата,
то есть на своих единомышленников? А потом добробат героически расправляется с
этими бандитами. Возможно, так бывшим членам добробатов программируют
дальнейшее поведение, но с точки зрения сюжета поведение бандитов крайне
нелогично. 

Отдельной строкой идет линия российской ЧВК. Они появляются в конце фильма и
отражают атаку ВСУ. Тут тоже интересно. С одной стороны, они, вроде как,
защитили ополченцев, не дали ВСУ их раздавить. С другой стороны именно в это же
время гибнут два лирических героя, старик учитель и подросток, оставшийся
сиротой после обстрелов. То есть получается именно по фильму, что ЧВК защитили
только руководство, а простые ополченцы погибли, и будут погибать в дальнейшем,
поскольку ЧВК, оставив трофейную технику в распоряжении обороняющихся, уходят.
Идет героическая музыка... но убийство людей продолжается. 

Фильм, безусловно, ценен тем, что показывает ужасы войны, делает определенный
ее образ. И тут правы те, кто сравнивает этот фильм со знаменитым "Иди и
смотри", где мотивы героев совершенно не раскрыты, где сама картина войны
приводит в ужас, и именно это и есть цель авторов фильма. Как по мне
"Солнцепек" этой цели достиг. Война тут не романтика, а общенародное бедствие.
И, значит, этому бедствию нужно положить конец, другого пути нет.

\ii{16_08_2021.fb.zharkih_denis.1.solncepek_film.cmt}
