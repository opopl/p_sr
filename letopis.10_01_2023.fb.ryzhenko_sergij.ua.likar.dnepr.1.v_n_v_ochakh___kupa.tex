%%beginhead 
 
%%file 10_01_2023.fb.ryzhenko_sergij.ua.likar.dnepr.1.v_n_v_ochakh___kupa
%%parent 10_01_2023
 
%%url https://www.facebook.com/rsa010963/posts/pfbid02BfgN65juAvSbtaDhUqcx2HKSoZvX7rLYzX1tVVPywyg2eW6aydUAYcYZCsgmuUFl
 
%%author_id ryzhenko_sergij.ua.likar.dnepr
%%date 10_01_2023
 
%%tags 
%%title Він в очах її купався, гладячи волосся рукою…
 
%%endhead 

\subsection{Він в очах її купався, гладячи волосся рукою...}
\label{sec:10_01_2023.fb.ryzhenko_sergij.ua.likar.dnepr.1.v_n_v_ochakh___kupa}

\Purl{https://www.facebook.com/rsa010963/posts/pfbid02BfgN65juAvSbtaDhUqcx2HKSoZvX7rLYzX1tVVPywyg2eW6aydUAYcYZCsgmuUFl}
\ifcmt
 author_begin
   author_id ryzhenko_sergij.ua.likar.dnepr
 author_end
\fi

Він в очах її купався, гладячи волосся рукою...

Струна життя дуже тонка.

Я завжди вірила, що ми будемо разом назавжди, каже Наташа.

Вона не знала, як його знайти, не знала, як з ним зв'язатися.

Усі мрії, всі ночі лише про нього.

І вірила, що він буде тільки мій, незважаючи на те, що з ним стало.

Івану 35.

Під час мінометного обстрілу за долю секунди побачив, як відірвало ноги.

Тоді подумав: чи повернусь я додому живим?

Постійне відчуття близької смерті.

Більше шести літрів перелитої крові.

Опритомнів після операції в реанімації Мечникова.

Приходячи до тями, називав ім'я дружини та доньки.

Сьогодні вже без апарату ШВЛ.

У реанімації посміхається, жартує із рідними.

Впевнений, що й надалі все налагодиться.

Приховуючи внутрішній біль, навчився посміхатися, зчепивши зуби.

Жартує: мені судилося вижити і тут, і на нулі. Я сьогодні щасливий. І я не
герой, я такий, як всі українці. 

Бо ми всі разом.

P.S.: Валентина Лисничая, завідувачка реанімації сепсису Мечникова, день за
днем витягує поранених з того світу, бачачи ріки сліз та усміхнені обличчя
близьких при одужанні їх синів та чоловіків.

He fell into her eyes, stroking her hair... He fell into her eyes

The string of life is very thin.

- I always believed that we would be together forever, says Natasha.

She did not know how to find him, how to contact him.

All dreams, all nights only about him.

- And I believed that he would be only mine, despite what happened to him.

Ivan is 35 years old.

During the mortar shelling, in a split second he saw how his legs were severed. 

Then he thought: \enquote{Will I return home alive?}

Constant feeling of imminent death. 

More than six liters of transfused blood.

He regained consciousness after the operation in Mechnikov’s intensive care
unit and called the names of his wife and daughter.

Today he is already without artificial lung ventilation apparatus.

In the unit of intensive care he smiles and jokes with his relatives.

He is sure that in future everything will be fine. 

Hiding the inner pain, he learned to smile, gritting his teeth.

He's joking: \enquote{I'm destined to survive both here and at null. I am happy today.
And I am not a hero, I am like all Ukrainians}.

Because we are all together.

P.S.: Valentina Lisnycha, head of Mechnikov’s sepsis intensive care unit, day
after day pulls the wounded out of the other world, seeing rivers of tears and
smiling faces of loved ones as their sons and husbands recover
