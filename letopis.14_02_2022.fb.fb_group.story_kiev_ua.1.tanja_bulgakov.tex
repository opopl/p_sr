% vim: keymap=russian-jcukenwin
%%beginhead 
 
%%file 14_02_2022.fb.fb_group.story_kiev_ua.1.tanja_bulgakov
%%parent 14_02_2022
 
%%url https://www.facebook.com/groups/story.kiev.ua/posts/1861526270710832
 
%%author_id fb_group.story_kiev_ua,denisova_oksana.kiev.ukraina.gid
%%date 
 
%%tags bulgakov_mihail,kiev
%%title Вообще-то ее звали Таня
 
%%endhead 
 
\subsection{Вообще-то ее звали Таня}
\label{sec:14_02_2022.fb.fb_group.story_kiev_ua.1.tanja_bulgakov}
 
\Purl{https://www.facebook.com/groups/story.kiev.ua/posts/1861526270710832}
\ifcmt
 author_begin
   author_id fb_group.story_kiev_ua,denisova_oksana.kiev.ukraina.gid
 author_end
\fi

Вообще-то ее звали Таня. Тасей ее, гимназистку, назвал при первой встрече
гимназист Миша. Она, 15-летняя, приехала из Саратова в Киев к тетке на летние
каникулы. Тетя дружила с мамой Булгакова, и 17-летнего Мишу попросили показать
девочке город. Он и показал...

\ii{14_02_2022.fb.fb_group.story_kiev_ua.1.tanja_bulgakov.pic.1}

Любовь вспыхнула сразу, они бродили по Киеву, по Владимирской горке, катались
на лодке, вечером ходили в театр. Договорились встретиться на Рождество, но
Тасин отец сказал, что рано ей думать о любви и в Киев ее не пустил. В Саратов
пришла телеграмма от Мишиного друга: \enquote{Телеграфируйте... приезд. Миша
стреляется}. Отец Таси перехватил телеграмму и посмеялся, а Тасю запер на ключ.
Но обошлось...

После гимназии ей предлагали ехать учиться в Париж, но она хотела только в Киев
и приехала. Миша уже был студентом университета, она поступила на Высшие
женские курсы, родители смирились и 26 апреля 1913 года они обвенчались в
небольшой церкви Николы Доброго на Подоле. Жили славно и весело, хотя денег
часто не было. Когда ее  отец присылал 50 рублей, они шиковали, ходили в кафе и
рестораны, ездили на извозчике, а потом сидели впроголодь.

А потом началась первая мировая война и в 1916 году Михаил, который был еще
студентом-медиком. уезжает на фронт, работает в прифронтовом госпитале в
Черновцах, а Тася едет за ним и становится сестрой-милосердия. Она, выросшая в
очень богатой семье, в роскоши, попала туда, где кровь, грязь, она помогала ему
при ампутациях, она помогала ему во всем. Когда он заразился дифтерией и для
лечения стал принимать морфий, она по сути спасла его – она просто втайне от
него уменьшала дозу морфия и добавляла воду: все меньше морфия, все больше
воды, а он не знал, верил в силу морфия и незаметно излечился. В голодном и
холодном 1918 году они вернулись в Киев, чуть позже им пришлось продать
свадебные кольца с гравировкой – у нее на кольце было написано «Михаил», у него
– «Татьяна», плохая была примета...

В 1922 году они расстались. Почему, знали только они... У него потом было еще
две жены, у нее - два мужа. Но, умирая в 1940 году, по свидетельству его сестры
Елены, совершенно ослепнувший Булгаков в бреду звал только Тасю. А она пережила
его на 42 года и в конце жизни написала воспоминания, в которых сказала, что
любила всю жизнь только его...

Вот такая история любви и тайну этой любви они унесли с собой.
