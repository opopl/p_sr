% vim: keymap=russian-jcukenwin
%%beginhead 
 
%%file 19_11_2013.news.ua.pravda.1.farion_kpss
%%parent 19_11_2013
 
%%url https://www.pravda.com.ua/news/2013/11/19/7002519/
 
%%author 
%%author_id 
%%author_url 
 
%%tags 
%%title "Свободівка" Фаріон таки була членом КПРС
 
%%endhead 
 
\subsection{\enquote{Свободівка} Фаріон таки була членом КПРС}
\label{sec:19_11_2013.news.ua.pravda.1.farion_kpss}
\Purl{https://www.pravda.com.ua/news/2013/11/19/7002519/}

Вівторок, 19 листопада 2013, 23:38
\index[names.rus]{Фаріон, Ірина!Член КПРС}

Протоколи засідань парткому Львівського державного університету ім. І. Франка,
а також засідань комітету комсомолу цього вишу підтверджують членство депутата
\enquote{Свободи} Ірини Фаріон у Компартії Радянського Союзу (КПРС).

Відповідні документи отримало видання Zaxid.net в Державному архіві Львівської
області. Вони містяться в трьох справах партійного архіву Львівського обкому
Компартії України.\Furl{https://zaxid.net/chlenstvo_irini_farion_v_kprs_pidtverdzheno_ofitsiyno_n1297534}

\ifcmt
pic https://img.pravda.com/images/doc/d/a/da483e5-1.jpg
\fi

Справа №356, опис №1, фонд №3567, містить протоколи засідань комітету комсомолу
ЛДУ ім. І.Франка, серед яких --- протокол №10 засідання 10 березня 1987 року.
Третім питанням порядку денного цього засідання було \enquote{Про надання
характеристики-рекомендації для вступу кандидатом в члени КПРС тов. Фаріон
І.Д.}.

Надання відповідної характеристики-рекомендації, як свідчить протокол, учасники
засідання підтримали одноголосно. Поміж іншим, в протоколі вказано, що Ірина
Фаріон була членом ВЛКСМ з вересня 1978 року.

Також справа містить і саму характеристику-рекомендацію, підписану тодішнім
секретарем комсомолу комітету ЛДУ Степаном Кубівим (народний депутат від
\enquote{Батьківщини}, пройшов за партійним списком).

Помітним в цій характеристиці є місце, в якому йдеться про те, що Фаріон
заохочувала вивчення російської мови іноземними студентами.

\ifcmt
pic http://img.pravda.com.ua/files/a/c/acaefd2-2-2.jpg
caption Ірина Фаріон, характеристика КПРС
\fi

Ця характеристика-рекомендація стала потім підставою для того, аби Ірина Фаріон
стала кандидатом в члени КПРС. 

Справа №248, опис №2, фонд №92 містить, зокрема, протокол засідання парткому
ЛДУ ім. І. Франка від 3 квітня 1987 року. У порядку денному цього засідання
стоїть питання про прийом  Ірини Фаріон кандидатом в члени КПРС. Позитивне
рішення прийняли одноголосно.

Протокол засідання, на якому нинішнього народного депутата від \enquote{Свободи}
приймали безпосередньо в члени КПРС, міститься в справі №258, опис №2, фонд
№92. Йдеться про протокол №45 від 15 квітня 1988 року. Тоді в порядку денному
розглядали питання про прийом до КПРС 15 кандидатів. Рішення ухвалили
одноголосно.

\ifcmt
tab_begin cols=2
  pic http://img.pravda.com.ua/files/e/4/e41946c-------.jpg
  pic http://img.pravda.com.ua/files/c/7/c74acff-88909459--1--f.jpg
tab_end
\fi

Відтак ці документи можна вважати офіційним підтвердженням членства Ірини
Фаріон у КПРС, яке вона сама категорично заперечує.

Документів, які б свідчили про вихід чи виключення Ірини Фаріон з КПРС до 30
серпня 1991 року, коли діяльність компартії на території України була
заборонена офіційно, виявити не вдалося.

За словами начальника відділу забезпечення зберігання документів № 2 Державного
архіву Львівської області Лариси Полякової, подібних документів в архіві немає.

Як відомо, сама Фаріон неодноразово заперечувала своє членство в КПРС.

До ВО \enquote{Свобода} вона вступила у 2005 році, наступного року стала депутатом
Львівської облради від цієї партії. У 2006 і 2007 роках балотувалася до
Верховної Ради за списком ВО "Свобода", однак тоді партія не подолала виборчого
бар’єру.

У 2012 році Фаріон виграла вибори до Верховної Ради у мажоритарному окрузі №116
(Львів), набравши понад 68\% голосів виборців.
