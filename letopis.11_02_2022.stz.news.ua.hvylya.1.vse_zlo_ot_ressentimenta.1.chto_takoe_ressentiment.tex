% vim: keymap=russian-jcukenwin
%%beginhead 
 
%%file 11_02_2022.stz.news.ua.hvylya.1.vse_zlo_ot_ressentimenta.1.chto_takoe_ressentiment
%%parent 11_02_2022.stz.news.ua.hvylya.1.vse_zlo_ot_ressentimenta
 
%%url 
 
%%author_id 
%%date 
 
%%tags 
%%title 
 
%%endhead 

\subsubsection{Что такое ресентимент?}
\label{sec:11_02_2022.stz.news.ua.hvylya.1.vse_zlo_ot_ressentimenta.1.chto_takoe_ressentiment}

По определению Википедии Ресентимент это чувство враждебности к тому, что
субъект считает причиной своих неудач («врага»), бессильная зависть, «тягостное
сознание тщетности попыток повысить свой статус в жизни или в обществе».

Ресентимент это установка на отказ от положительной активности в пользу
негативной активности: подозрений, ревности, обвинений, мести, контроля,
принуждения, агрессии в отношении других. Ресентимент это установка «если бы не
кто-то или что-то, то я бы...» в контексте переживания унижения или возвышения,
которые время от времени меняются местами. Ресентимент суть отказ от
собственной ответственности за судьбу. Ресентимент перенаправляет
индивидуальную, групповую и общественную энергию из созидания сложного на
воспроизведение простого и упрощенного и разрушения сложного.

Ресентимент возникает в индивидуальной психике и в массовом сознании как
упрощенное переживание разрыва между простым и сложным, что не может быть
напрямую преодолено и поэтому требует опосредованного компенсаторного
замещения. Ресентимент — это не разрыв, не замещение, а само упрощенное
переживание разрыва, который может быть как преемственным, так и созданным
собственной деградацией. Ресентимент возникает из-за сознательного отказа от
самопреобразования-усложнения, из-за сознательного продолжения деградации,
из-за сознательного желания упростить сложную среду или персонифицированных
носителей сложности до уровня своего упрощения.

Впервые термин «Ресентимент» (Ressentiment) употребляет Серен Кьеркегор в
работе «Два века: литературное обозрение» (1846).

Его употребляет Фридрих Ницше в работе «О генеалогии морали» (1887), описывая
как соотношение морали владельца и раба. Рессентимент, по Ницше, деятельно
проявляет себя в «восстании рабов»: «Восстание рабов в морали начинается с
того, что ressentiment сам становится творческим и порождает ценности…»

Затем его подробно описывает Макс Шелер в книге «Ресентимент в структуре
нравов» (1913).

Макс Вебер в работах по социологии религии (1920-1921) связывает Ресентимент с
иудаизмом, нравственной религией спасения «людей-изгоев», где он выступает как
теодицея страдания (несчастья). Из русских переводов трудно догадаться, что
«теодицея несчастья», хотя точнее было бы переводить «теодицея страдания», это
фактически Ресентимент, по поводу которого была дискуссия между Вебером и
Шелером.

Уже тогда на Западе сложилась определенная традиция употреблять это слово на
французском «Ressentiment», как оно было употреблено Кьеркегором, чтобы
сохранить широкий контекст дискурсивного значения и применения. Поэтому любые
его переводы как «оскорбление», «злопамятность», «ожесточение»,
«неприемлемость», «страдание», «несчастье» некорректны.

Жиль Делез в книге «Ницше и философия» (1962) посвящает Ресентименту (переведен
на русский как «злопамятность») главу четвертую «От злопамятности к нечистой
совести», где осуществляет психоаналитическую и культурологическую
интерпретацию идей Ницше. И снова в таком переводе трудно догадаться, что речь
идет о Ресентименте.

Фрэнсис Фукуяма в своей книге «Идентичность: запрос на достоинство и политика
ресентимента» (2018) дает очень ограниченный и внутренне цензурированный обзор
углубления Ресентимента в современном мире. И опять же сравните переводы: на
русском — «Идентичность. Стремление к признанию и политика неприятия», на
украинском – «Ідентичність. Потреба в гідності й політика скривдженості».

То есть в российских и украинских источниках переводы названий и текстов книг о
Ресентименте осуществлены таким образом, чтобы скрыть и замолчать проблему
Ресентимента из-за нежелания его понимать и публично обсуждать или речь идет о
вопиющем невежестве. Выхолащивать смыслы идей — это тоже способ цензуры. При
нацистской диктатуре книги Макса Шелера просто запрещали или сжигали, и это
было в каком-то смысле честнее.

Также Ресентимент исследовали: Рене Жерар, Марк Анжено, Стефано Томеллери,
Мишень Онфре, Синтия Флери.

Понимание Ресентимента имело определенный путь в истории идей. Серен Кьеркегор
сформулировал его, чтобы показать современность как «нынешний век, высасывающий
смысл из этических понятий бесстрастной ленью. Понятия все еще используются, но
лишены всякого смысла из-за своей оторванности от взгляда на жизнь, рожденного
страстью и порождающего последовательность действий».

Фридрих Ницше показал его как нравственное отношение этики рабовладельца и
этики раба. Макс Шелер показал его как психо-социальное явление, изменяющее все
социальные отношения, а также индивидуальное осознание себя в группе и обществе
в целом. Вебер показал роль Ресентимента в религиозной этике. Делез попытался
произвести психоаналитическую и культурологическую интерпретацию Ресентимента.
Фукуяма вынужден был уже анализировать Ресентимент как идентичность и политику.

Итак, мы столкнулись с постоянным разрастанием масштаба Ресентимента в истории
человечества. Следовательно, мы должны вернуться к пониманию Кьеркегора и
Шелера, но уже на новом уровне рефлексии.
