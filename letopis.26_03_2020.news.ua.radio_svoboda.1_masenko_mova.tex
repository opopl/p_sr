% vim: keymap=russian-jcukenwin
%%beginhead 
 
%%file 26_03_2020.news.ua.radio_svoboda.1_masenko_mova
%%parent 26_03_2020
%%url https://www.radiosvoboda.org/a/30509991.html
 
%%endhead 

\subsection{«Українська мова – головний ворог Росії» – професор Лариса Масенко}

\url{https://www.radiosvoboda.org/a/30509991.html}

\subsubsection{Пострадянська демократія. Провали і сподівання}

Поняття єдності соціалістичної батьківщини для всіх народів, що її населяли,
належали до головних гасел радянської пропаганди. Інструментами безпрецедентної
ідеологічної та культурної уніфікації «необъятной родины» були російська мова,
комуністична ідеологія та однопартійна система.

Для приховування реальної політики «злиття націй» в єдиний радянський народ
пропагандивний апарат оперував такими поширеними штампами, як «дружба народів»,
«сім’я народів-братів», «братерська єдність народів», «інтернаціональна
єдність» тощо.

Ця лагідна миролюбна риторика була лише зітканою з фальшивих слів завісою, за
якою ховалась хижацька колонізаторська сутність Росії у ставленні до
неросійських народів імперії. Її мета полягала у послабленні в масовій
свідомості почуття належності до певної нації й заміна його загальноімперською
ідентичністю.

У статті «Пороги і Запоріжжя» Юрій Шевельов, полемізуючи з тими істориками, які
припускали можливість демократичного розвитку Російської імперії перед
більшовицьким переворотом, зазначав, що демократизація Росії була неможливою,
поки вона лишалась колоніальною країною.

На підтвердження Шевельов наводив заповіт найліберальнішого з російських царів
Олекандра ІІ: «Потужність Росії заснована на єдності держави, а тому все, що
може вести до потрясення цієї єдності і до окремішнього розвитку різних
народностей, для неї згубне і не повинно припускатися» (1876 рік).

«Мати демократичний режим у метрополії, а тоталітарний у колоніях – важко. Коли
колонії не за морем, а безпосередньо прилягають до метрополії, – неможливо», –
прокоментував наведену цитату російського царя Шевельов.

\subsubsection{Українська мова – головний ворог Росії}

На жаль, і сьогодні немає жодних підстав сподіватися на демократизацію Росії,
на визнання нею права колишніх колоній на відокремлення й побудову власних
суверенних держав.

В геополітичних планах відновлення Російської імперії в її колишніх межах
Україна посідає особливе місце. З нашою державою Путін та його оточення ведуть
запеклу війну на знищення.

Варто ще раз наголосити, що потужною зброєю в цій війні лишається російська
мова. Збереження російськомовних середовищ в Україні належить до стратегічних
завдань Росії.

Закономірно, що інтеграційним амбіціям Кремля найбільше заважає українська
мова, розширення сфер її вживання, активізація її культуротворчого потенціалу,
що зміцнює національну ідентичність української людності.

Як не без іронії зауважив на своїй сторінці у фейсбуці Анатолій Стрєляний, у
Росії «є тільки один ворог, при думці про якого вона впадає в тугу воістину
смертельну. Це не США, не Захід в цілому, не Китай. Це українська мова».

Ненависть до української мови спонукає окупантів негайно вилучати її з освіти й
знищувати всі пам’ятки української культури й історії в окупованих анклавах. На
окупованих територіях Луганської й Донецької областей з 2016 року майже
повністю витіснили викладання української мови. Діти вчаться за завезеними з
Росії підручниками.

6 березня окупаційна адміністрація Росії позбавила українську мову статусу
державної на підконтрольній їй території.

Під час однієї з акцій на підтримку української мови (архівне фото)

В окупованих анклавах Криму й Донбасу Росія, таким чином, русифікує населення,
позбавляючи його історичної пам’яті й виховуючи молодь в атмосфері ненависті до
народу, якому донедавна ці землі належали.

11 березня Російська Рада Федерації ухвалила закон про визнання громадян
Білорусі та України, які вільно володіють російською мовою, її носіями без
проходження співбесіди, що дає їм можливість отримати російське громадянство в
спрощеному порядку.

Цілком очевидно, що через надання російських паспортів громадянам
пострадянських держав Російська Федерація створює потенційний привід для
майбутнього вторгнення на їхню територію. Деякі оглядачі називають таку
практику «реокупацією в спосіб паспортизації».

Під час пікету Конституційного суду України, який тоді розглядав справу щодо
конституційності «мовного закону Ківалова-Колесніченка». Київ, 17 листопада
2016 року. (В кінцевому результаті КСУ в лютому 2018 року скасував «закон
Ківалова-Колесніченка»)

\subsubsection{Хвороба совєтськості, її симптоми та загрози}

З огляду на те, що на суміжних з окупованими землями більшість населення
проголосувало за проросійську партію ОПЗЖ, найгіршим сценарієм для України може
стати поширення вірусу «русского мира» на цих територіях.

Але і в загальноукраїнському вимірі минулорічні вибори президента й Верховної
Ради показали, що значна частина нашого населення досі не позбулася пов’язаної
із зросійщенням хвороби совєтськості. Ця хвороба вразила передусім старше
покоління, але частково її успадкувала і молодь.

До її симптомів належать: соціальний інфантилізм, нездатність до критичного
мислення, відсутність імунітету до політичної демагогії і шахрайства,
зосередженість на приватних інтересах й ігнорування загальнонаціональних
вартостей.

Подібна ментальність властива й обраному шостому президентові України. Це
засвідчує його соціальна незрілість, впевненість в тому, що на найвищу державну
посаду можна претендувати, не маючи відповідної компетентності й досвіду
політичної діяльності.

До радянського минулого апелює вже назва серіалу й партії Зеленського «Слуга
народу», яка реанімує відоме ще зі сталінських часів гасло «Депутат – слуга
народу».


Радянський плакат «Депутат – слуга народу», створений у 1954 році московським
художником Борисом Зеленським до виборів до Верховної Ради СРСР. У «ролі»
депутата на плакаті зображення радянського актора Володимира Гусєва. А вислів
«Депутат – слуга народу» став крилатим після того, як цю фразу вжив у своїй
промові 11 грудня 1937 року Йосип Сталін

Як і всі інші радянські лозунги, воно було складене зі слів з порожньою
семантикою і слугувало лише для імітації особливої «соціалістичної демократії»,
за фасадом якої народне волевиявлення було цілковитою фікцією, оскільки
виборчий бюлетень містив лише одне прізвище призначеного партійним керівництвом
депутата до Верховної Ради СРСР.

Все чіткіше вимальовується неспроможність нинішнього президента дотримуватися
демократичних принципів управління державою, його бажання підпорядкувати собі
всі гілки влади, самочинно призначати й звільняти уряд, прокурора та інших
посадовців, просувати на відповідальні посади колишніх прибічників Януковича.

Ментально Зеленський досі, очевидно, не позбувся впливу колишніх фальшивих
декларацій про «священну дружбу братніх російського й українського народів». Це
зраджує той факт, що він майже ніколи не називає Росію ворогом. Про це свідчить
і його заява на Мюнхенській безпековій конференції, де він сказав: «…у моїй
ментальності, у моїй особистій культурі, у моєму мозку я закінчив війну». З
боку особи, яка мусить виконувати обов’язки Верховного Головнокомандувача
Збройних сил України, та ще й в час війни, це звучить, м’яко кажучи, дивно.

Присутність людей з такою ментальністю у владного керма в ситуації війни з
Росією й економічної кризи, що насувається, вкупі з пандемією коронавірусу,
нічого втішного нашій країні не обіцяє.

І все ж, попри невдачі й провали, маємо один незаперечний здобуток демократії.
Завдяки їй в Україні сформувалось потужне громадянське суспільство, яке вже
двічі рятувало країну від сповзання до проросійської диктатури.

Саме з ним і з об’єднанням всіх конструктивних сил української спільноти можна
пов’язувати надії на подолання тих великих загроз, що нині постали перед
державою.

Лариса Масенко – доктор філологічних наук, професор Національного університету
«Києво-Могилянська академія»

Думки, висловлені в рубриці «Точка зору», передають погляди самих авторів і не
конче відображають позицію Радіо Свобода



