% vim: keymap=russian-jcukenwin
%%beginhead 
 
%%file 08_10_2021.fb.fb_group.protydia_zrosijschennju_volynj.1.magazin_salut_mova_luck.cmt
%%parent 08_10_2021.fb.fb_group.protydia_zrosijschennju_volynj.1.magazin_salut_mova_luck
 
%%url 
 
%%author_id 
%%date 
 
%%tags 
%%title 
 
%%endhead 
\subsubsection{Коментарі}
\label{sec:08_10_2021.fb.fb_group.protydia_zrosijschennju_volynj.1.magazin_salut_mova_luck.cmt}

\begin{itemize} % {
\iusr{Богдан Жишко}

\ifcmt
  ig https://scontent-frt3-2.xx.fbcdn.net/v/t1.6435-9/245138675_6154454091296296_4881515950607542795_n.jpg?_nc_cat=103&ccb=1-5&_nc_sid=dbeb18&_nc_ohc=SQ9jK5VbQT0AX9JujRf&_nc_ht=scontent-frt3-2.xx&oh=4a8f7b72d6f7bdae933748f749c79ea4&oe=618AA29C
  @width 0.2
\fi

\iusr{Богдан Жишко}
Текст українською. Називається: "їжте - не обляпайтеся!"

\ifcmt
  ig https://scontent-frt3-1.xx.fbcdn.net/v/t1.6435-9/245037093_6154455037962868_8338718423574688402_n.jpg?_nc_cat=102&ccb=1-5&_nc_sid=dbeb18&_nc_ohc=atRrNWTAh4UAX-wZFgn&_nc_oc=AQlFGJ0pE3q6abVa2BpAjkpJkO17RoB6dOmaIb-hW2YMQELxrS1IP4OnwqTFQet2J0U&_nc_ht=scontent-frt3-1.xx&oh=24db8b5cb0701e9b7437980c9027247e&oe=6188AE01
  @width 0.2
\fi

\iusr{Лариса Бондаренко}
Як ви взагалі могли купити товар під такою назвою -"гаврюша"))

\begin{itemize} % {
\iusr{Вадим Шмирьов}
\textbf{Лариса Бондаренко} Блокувати таке треба!... За порєбрік... Москальські прихвостні...

\iusr{Богдан Жишко}
\textbf{Лариса Бондаренко} побачив лише вдома. Йшов по сир.

\iusr{Лариса Бондаренко}
\textbf{Богдан Жишко} А ну, признавайтеся, з'їли чи викинули?)))

\iusr{Богдан Жишко}
\textbf{Лариса Бондаренко}  @igg{fbicon.monkey.speak.no.evil} 


\end{itemize} % }

\iusr{Олена Дубень}
Дуже шкода що так сталося.
Проте, я дякую працівникам маркетів, що вони в пандемію працювали для нас!

\iusr{Валя Данко}
А уявіть що робиться у нас на півдні! Майже всюди кацапська  @igg{fbicon.anger} .

\begin{itemize} % {
\iusr{Богдан Жишко}
\textbf{Valentina Chayka} гуртуйтеся в товариства. Раз в місяць збирайтеся, щоб обговорити ситуацію й вирішити, що можна зробити.
Зараз приходять хороші звістки з Дніпра, багато чого роблять.
Треба бути готовим до того, що це відчутно для нервів. Але інакше змін не буває.

\iusr{Валя Данко}
\textbf{Богдан Жишко} ви праві. Але у нас дуже, дуже багато "какаяразніца"! Одиниці переходять на українську мову, нажаль.
\end{itemize} % }

\iusr{Олена Юхимчук}
А як вам назва сирка "Гаврюша"? Ні чим не попахіваєт? @igg{fbicon.face.grinning.squinting} 

\begin{itemize} % {
\iusr{Богдан Жишко}
\textbf{Олена Юхимчук} та я вдома побачив, коли читав чек.

\iusr{Бажан Козаченко}
\textbf{Олена Юхимчук}: ви лихо намагаєтесь знайти з нічого коли навколо сила-силенна негараздів. Насамперед мусимо з себе вичавлювати оте все чуже - як московитське, там і з іншого неслов'янського боку. Звинувачувати когось є дуже легкою справою.

\iusr{Богдан Жишко}
Це Харків. А гаврюшу вони зробили невидимим у порівнянні із "сирок".

\ifcmt
  ig https://scontent-frx5-1.xx.fbcdn.net/v/t1.6435-9/245137882_6159549920786713_441314226840401887_n.jpg?_nc_cat=100&ccb=1-5&_nc_sid=dbeb18&_nc_ohc=z0eti438Q9AAX8c9B18&_nc_ht=scontent-frx5-1.xx&oh=9663a3847be98c70809eafcb2b0d87a1&oe=6189A9B0
  @width 0.3
\fi

\iusr{Бажан Козаченко}
\textbf{Богдан Жишко}: який негаразд ви побачили у "сирку" ? Хіба українська мова НЕ має слова "сир" ? "Сирко" точно НЕ є воно, бо "-ко" позначає зменшену тяму щодо живої істоти, наприклад псів на Україні часто називали "Сірко".
Синок, відпочинок, барвінок, пасок, ґанок, молоток, будинок тощо.

\iusr{Богдан Жишко}
\textbf{Бажан Козаченко} а ви контекст гілки аналізуєте? Йдеться про московське "Гаврюша" і розміри шрифтів.
\end{itemize} % }

\iusr{Бажан Козаченко}
Чи маєте ви бажання позбутись тих "миючих", "керуючих" у вашому мовленні ?

\iusr{Богдан Вівчарик}
молодець... борись до кінця, не здавайся друже...

\end{itemize} % }
