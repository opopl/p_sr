% vim: keymap=russian-jcukenwin
%%beginhead 
 
%%file 28_01_2022.fb.fb_group.story_kiev_ua.1.kiev_visim_sekretiv.2.knjazi_ostrozki
%%parent 28_01_2022.fb.fb_group.story_kiev_ua.1.kiev_visim_sekretiv
 
%%url 
 
%%author_id 
%%date 
 
%%tags 
%%title 
 
%%endhead 
\subsubsection{2. КНЯЗІ ОСТРОЗЬКІ: ЄВРОПЕЙСЬКИЙ ВИМІР УКРАЇНСЬКОЇ ІСТОРІЇ}
\label{sec:28_01_2022.fb.fb_group.story_kiev_ua.1.kiev_visim_sekretiv.2.knjazi_ostrozki}

2. На території Печерської лаври обов’язково необхідно відвідати виставку:

\enquote{КНЯЗІ ОСТРОЗЬКІ: ЄВРОПЕЙСЬКИЙ ВИМІР УКРАЇНСЬКОЇ ІСТОРІЇ}.

Саме з цим князівським родом пов’язана золота доба української історії. 

Навіть більше, Острозькі відігравали значну роль в історичних процесах, що
відбувались в Європі в XV-XVII ст. 

Найвідомішими представниками українського шляхетського роду були:

Данило Острозький, який приймав участь в переможній битві 1362 року на Синіх
водах, коли литово-руські війська перемогли Ординців і звільнили більшу частину
Руси-України.

Федір Острозький, що став пострижеником Києво-Печерського монастиря і похований
тут як один із найшанованіших святих - преп. Феодосій.

\ii{28_01_2022.fb.fb_group.story_kiev_ua.1.kiev_visim_sekretiv.pic.2}

Костянтин Іванович Острозький - один з головних героїв нашої історії,
багаторічний гетьман, найкращий полководець Європи XVI ст., некоронований
король Руси-України.

Костянтин-Василь Острозький - київський воєвода, засновник Острозької академії
та друкарні, \enquote{Перше вогнище освіти, яке як факел освітило всю східну
Європу новими знаннями} - за словами М. Грушевського.  

Про родовід Острозький я писав в пості, коли подорожував в Дубно.

На виставці також можна побачити одну з моїх улюблених мап 1613 року, яка
містить назву «Україна» (Vkraina; позначаючи так Київщину і землі Подніпров‘я),
що була розроблена за участі Острозьких і кращих інтелектуалів Острозької
школи.

А в відновленому Успенському соборі, на місці де був похований князь, можна і
потрібно побачили відтворений надгробок Костянтина Івановича Острозького
(1460-1530). Символ великого європейського минулого України.

Майже 500 років тому тут з великими почестями був похований наш видатний
гетьман, славетний переможець Московського князівства у війні 1514 року на
Дніпрі, а також переможець інших численних набігів орд на землі Руси.

Чому так важливо побувати тут і віддати шану славетного українському герою, я
писав у пості в цій групі. 

