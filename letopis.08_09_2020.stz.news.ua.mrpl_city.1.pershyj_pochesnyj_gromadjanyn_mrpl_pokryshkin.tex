% vim: keymap=russian-jcukenwin
%%beginhead 
 
%%file 08_09_2020.stz.news.ua.mrpl_city.1.pershyj_pochesnyj_gromadjanyn_mrpl_pokryshkin
%%parent 08_09_2020
 
%%url https://mrpl.city/blogs/view/pershij-pochesnij-gromadyanin-mariupolyado-77-richnitsi-zvilnennya-mariupolya
 
%%author_id demidko_olga.mariupol,news.ua.mrpl_city
%%date 
 
%%tags 
%%title Перший почесний громадянин Маріуполя (до 77 річниці звільнення Маріуполя)
 
%%endhead 
 
\subsection{Перший почесний громадянин Маріуполя (до 77 річниці звільнення Маріуполя)}
\label{sec:08_09_2020.stz.news.ua.mrpl_city.1.pershyj_pochesnyj_gromadjanyn_mrpl_pokryshkin}
 
\Purl{https://mrpl.city/blogs/view/pershij-pochesnij-gromadyanin-mariupolyado-77-richnitsi-zvilnennya-mariupolya}
\ifcmt
 author_begin
   author_id demidko_olga.mariupol,news.ua.mrpl_city
 author_end
\fi

\ii{08_09_2020.stz.news.ua.mrpl_city.1.pershyj_pochesnyj_gromadjanyn_mrpl_pokryshkin.pic.1}

Перемога над нацистською Німеччиною – визначна подія світової історії. А день
звільнення від окупантів у багатьох містах колишнього СНД святкується як \emph{День
міста}. Маріуполь теж не став виключенням. Одна з найбільш значущих і важливих
подій для наших містян сталася 10 вересня – \emph{77 років} тому – Маріуполь був
звільнений від німецько-нацистських  загарбників. Ця дата – добра нагода
вшанувати доблесних оборонців і визволителів у роки Другої світової війни, тих,
хто виявляв небувалий героїзм на бойових рубежах, хто відбудовував місто в
повоєнні часи...

Не всім маріупольцям відомо, що \emph{першим почесним громадянином нашого міста} став
чоловік, який не народився у Маріуполі, але завдяки подвигу якого, наше місто
було звільнено. Мова піде про \emph{\textbf{Олександра Івановича Покришкіна}} – маршала
авіації, Тричі Героя Радянського Союзу.

Народився Олександр у 1913 році у місті Новомиколаївську (нині Новосибірськ) у
родині робітників-переселенців. Скрутне матеріальне становище змусило хлопця
почати працювати з 10 років покрівельником в артілях будівельників. Проте
робота не завадила продовжувати навчання. У 1928 році закінчив семирічну школу.
З 1929 року працював покрівельником, слюсарем-лекальни\hyp{}ком заводу \enquote{Сибкомбайн}.
У 1930 році Олександр вступив до місцевого технічного училища, де провчився 18
місяців. Потім добровільно пішов в армію, був направлений в авіаційну школу.
Цікаво, що Олександра Івановича завжди відрізняла неабияка наполегливість в
досягненні цілей та  гострий розум.

Після закінчення авіаційної школи, в грудні 1934 року Покришкін став старшим
авіаційним техніком авіаланки 74-ї стрілецької дивізії. Але і на цьому амбітний
хлопець не зупинився – взимку 1938 року під час відпустки він потай від
начальства пройшов річну програму цивільного пілота за 17 днів, що автоматично
робило його придатним до вступу до Качинської льотної школи. Випустився з
найвищими оцінками і в 1939 році, в званні лейтенанта був розподілений у 55-й
винищувальний авіаційний полк.

У перший день Другої світової війни заступник командира ес\hyp{}кадрильї 55-го
авіаполку Олександр Покришкін підбив... радянський бомбардувальник Су-2.
Звичайно, помилково, в метушні і плутанині початку бойових дій. Підбитий
льотчик приземлився на фюзеляж в поле, а винуватець не отримав суворого
покарання. А вже наступного дня при розвідці пара Покришкіна зіткнулася з
п'ятіркою \enquote{Мессершмітів}, і він зміг не тільки відвести екіпажі від нападників
ворогів, але й підпалив один \enquote{Мессер}. Ворожі бійці добре запам'ятали його
бортовий номер – \enquote{5}.

У березні 1943 року полк Покришкіна отримав американські \enquote{Аерокобри}  і почав
літати на них з Краснодарського аеродрому. 5 травня 1943 року льотчик вперше
вилетів з бортовим номером \enquote{100} на своїй новій \enquote{Аерокобрі}. Три місяці
кубанської служби прославили знаменитого комеск на весь світ.

Указом Президії Верховної Ради СРСР за 354 бойових вильоти, 54 повітряні бої,
13 особистих і 6 в групі збитих літаків противника йому було присвоєно звання
Героя Радянського Союзу з врученням ордена Леніна і медалі \enquote{Золота Зірка}.

Другої медалі \enquote{Золота Зірка}  Покришкін був удостоєний в 1943-му році за 455
бойових вильотів і 30 особисто збитих літаків супротивника, а третю отримав в
1944 році за зразкове виконання бойових завдань командування і \emph{геройські
подвиги} на фронті боротьби з німецько-нацистськими загарбниками.

\ii{08_09_2020.stz.news.ua.mrpl_city.1.pershyj_pochesnyj_gromadjanyn_mrpl_pokryshkin.pic.2}

Один з таких подвигів Олександр Іванович здійснив на перегоні Маріуполь –
Волноваха. Винищувачі під його командуванням, виконуючи розвідувальний політ,
виявили багато товарних вагонів, з вікон яких подавалися якісь знаки.
Знизившись, льотчики побачили десятки рук, простягнутих до їхніх червонозоряних
літаків, волаючи про допомогу. Група винищувачів завдала удар, вивівши з ладу
паровози передніх ешелонів. З вагонів висипали люди, яких нацисти мали намір
відвезти в рабство.

Покришкін брав безпосередню участь у звільненні Маріуполя. Зокрема, у взаємодії
з частинами 130-ї та 221-ю стрілецькими дивізіями, а також з кораблями
Азовської військової флотилії хоробро билися в небі над Маріуполем льотчики 9-ї
гвардійської винищувальної авіаційної дивізії (командувач І. М. Дзусов, в
складі якого і  воював двічі Герой Радянського Союзу підполковник Покришкін).
Тоді, в далекому 1943 році над штормовим Азовським морем він застосував новий
тактичний прийом ведення бою \emph{\enquote{Вільне полювання}}. Це була найвища форма бойової
діяльності винищувачів, якими керував Олександр Покришкін під час війни. 10
вересня місто і морський порт Маріуполь були звільнені.

Популярність Олександр Покришкін отримав не тільки завдяки своїм подвигам, але
також і тому, що розробив нові елементи повітряного бою – наприклад, вихід
з-під удару на віражі низхідній \enquote{бочкою} з втратою швидкості (тоді противник
проскакував повз ціль, а сам опинявся в прицілі). Ну і, звичайно,
найзнаменитіша \enquote{Кубанська етажерка}  – її ввели в усі підрозділи радянської
винищувальної авіації. Кубанський бій став найбільш інтенсивним за всю Другу
Світову війну –  за два місяці там збили не менше 800 німецьких літаків.

У списку бойових нагород нашого героя 17 орденів СРСР: шість орденів Леніна,
орден Жовтневої революції, чотири ордени Червоного Прапора, два –  Суворова 2
го ступеня, орден Вітчизняної війни 1 ступеня, два ордени Червоної Зірки, орден
\enquote{За службу Батьківщині в ЗС} 3 ступеня, медалі. Особливе місце серед багатьох
іноземних нагород Покришкіна займає золота медаль \enquote{За бойові заслуги} США від
імені президента Франкліна Рузвельта. І це при тому, що багато хто з очевидців
стверджують – у героя Покришкіна багато не зарахованих перемог. Його земляки,
новосибірці, вивчили щоденники самого маршала, бойові документи полку,
відомості однополчан і нарахували 94 збитих Покришкіним німецьких літаків, 19
підбитих і 3 спалених на землі.

До кінця Другої Світової війни Покришкін став найвідомішим у світі льотчиком.
Німецькі пости оповіщення попереджали своїх вояків: російський ас в повітрі!
Радянські пропагандисти допомагали розпалювати страх в душах німецьких
льотчиків: \emph{\enquote{Ахтунг! Ахтунг! Покришкін у повітрі!}}. За знищення
цього російського аса призначали високі нагороди.

І все ж крім льотного таланту наш герой мав чудове тактичне мислення.  Крім
теорії польотів він самостійно вивчав фізику, математику, накреслювальну
геометрію, фізіологію. Фізичні заняття теж були підпорядковані ідеї
вдосконалення себе як льотчика –  гімнастика, батут, рейнське колесо і вправи
для тренування вестибулярного апарату

Олександр Іванович також є автором книг \enquote{Крила винищувача},
\emph{\enquote{Твій почесний обов'язок}}, \emph{\enquote{Небо війни}, \enquote{Пізнати себе
в бою}}.

Першим почесним громадянином нашого міста став маршал авіації у 1968 році. Це
звання присвоїла йому міська рада за видатні заслуги під час визволення міста
від німецько-нацистських загарбників. До речі,через 7 років після присвоєння
звання Олександр Іванович приїхав до Маріуполя, де для легендарного льотчика
влаштували справжнє свято і безліч теплих зустрічей з юними маріупольцями.

Помер Покришкін 13 листопада 1985 року. Похований на Новодівочому кладовищі в
Москві.

У Маріуполі в його честь названо вулицю в Кальміуському районі і встановлено
меморіальну дошку на будинку № 172 на проспекті Металургів.
