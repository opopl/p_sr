% vim: keymap=russian-jcukenwin
%%beginhead 
 
%%file 29_03_2022.stz.pc.ua.dou.1.kapellan.3.chto_delaet_kapellan
%%parent 29_03_2022.stz.pc.ua.dou.1.kapellan
 
%%url 
 
%%author_id 
%%date 
 
%%tags 
%%title 
 
%%endhead 

\subsubsection{Что же делает Военный капеллан?}

Если быть дословным, наши обязанности — это:

\begin{zznagolos}
Духовна, психологічна, гуманітарна та освітня допомога військовослужбовцям і
цивільному населенню.
\end{zznagolos}

Из всего этого я не занимаюсь только образованием. Остальное происходит каждый
день.

\ii{29_03_2022.stz.pc.ua.dou.1.kapellan.3.chto_delaet_kapellan.pic.1}

За первый месяц войны мы спасли и эвакуировали из Киева 75 людей. Завезли в
Украину тонны гуманитарных грузов. Также я помог одной американской семье
организовать передачу 32 тонн медикаментов в Минобороны. Я стараюсь помогать,
чем могу, подразделению одного из отрядов специального назначения ВСУ в Киеве.

13 Марта 2022 года был знаменательный день. Это было воскресенье. Я послушал в
церкви, что нужно благословлять своих врагов и поехал на Новоирпенскую трассу к
взорванному мосту для эвакуации людей.

На мосту стоит колонна обстрелянных брошенных гражданских автомобилей. Я не
стал это фотографировать, но сейчас то место похоже на сцену из фильма «Я —
Легенда».

Во время того, как я ждал людей, чтобы посадить к себе в автобус, орки начали
бомбить нас. Мы скрылись с солдатами в окоп. Примерно 10 минут мы находились
под миномётным обстрелом. Вот на этом фото, которое я нашел в интернете, рядом
с мостом, я лежал в окопе с солдатами.

\ii{29_03_2022.stz.pc.ua.dou.1.kapellan.3.chto_delaet_kapellan.pic.2}

Помолившись коротко за наши жизни, я увидел пятерых гражданских, которые в
последствии пообещали, что назовут своего сына Виктором. Я выскочил из окопа, и
прямо во время падающих мин схватил их вещи и загрузил всех в наш автобус.
Параллельно с ними, я эвакуировал двух журналистов.

Вот так прошёл тот день, я понял, что в ближайшее время я могу молиться только
за раскаяние всего оркостана и не могу сказать и слова благословения в их
сторону...
