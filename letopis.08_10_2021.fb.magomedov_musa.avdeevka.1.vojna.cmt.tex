% vim: keymap=russian-jcukenwin
%%beginhead 
 
%%file 08_10_2021.fb.magomedov_musa.avdeevka.1.vojna.cmt
%%parent 08_10_2021.fb.magomedov_musa.avdeevka.1.vojna
 
%%url 
 
%%author_id 
%%date 
 
%%tags 
%%title 
 
%%endhead 
\subsubsection{Коментарі}

\begin{itemize} % {
\iusr{Александр Барамиков}

Печально и грустно читать. В войне страдают мирные люди и люди, которые
повелись на речи "вождей". Очень бы хотелось узнать как беспрепятственно зашли
в Донецк колонна Гиркина.

\begin{itemize} % {
\iusr{Муса Магомедов}
\textbf{Александр Барамиков} это один из многих вопросов...

\iusr{Olena Demydiuk}
\textbf{Муса Магомедов} лично для меня это - единственный вопрос. И я хочу знать на него ответ!

\iusr{Александр Павлов}
\textbf{Olena Demydiuk} Я знаю, как он зашел в Донецк. Мы были 5 июля 2014 года с финским журналистом в Славянске и Краматорске и видели, как колонна из 150 машин зашла в Донецк. Никто, еще раз говорю - НИКТО, даже не пытался этому воспрепятствовать. Они шли, не парясь, со скоростью 60-70 км в час. По открытой степи. Остальное - думайте сами...

\iusr{Игорь Шендер}
\textbf{Александр Павлов} Так с журналистом - фотографий небось наделали и видео сняли? Показывайте, всем интересно.
\end{itemize} % }

\iusr{Елена Лимарева}

На два месяца мира дольше ! Можно позавидовать даже! У меня война началась 26
мая, когда украинская военная авиация бомбила Донецк. Первые мирные жертвы на
Привокзальной площади – женщина со смертельными ранениями в голову, её 8-летний
сын ранен... И убитый парень- парковщик.

\begin{itemize} % {
\iusr{Муса Магомедов}
\textbf{Елена Лимарева} я в это время жил в Донецке и боевые вертолёты над аэропортом видел, и дым над ним. И в дом, котором я жил прилетело, но я был на работе в это время. Но мне казалось всё это ещё можно остановить, что это не по-настоящему и вот-вот найдут решение, договорятся... Не сложилось.

\iusr{Валентина Щербак}
Да, до слёз...
мне стало страшно и непонятно , когда начался обстрел 26, 27 июля ,, и вырубился свет и на заводе перебили два ввода.. сказали , если за сутки не сделают - все конец всему. И между обстрелами - эта звовещая тишина...соседи - заводские, сутками были на работе, и говорили - Муса с нами..
Спасибо вам, Муса Сергоевич, на вас была единственная Надежда- будет жить завод , будет жить Авдеевка.!
\end{itemize} % }

\iusr{Владимир Вередин}

Можно сколько угодно сокрушаться, надеяться и верить, но я при памяти и буду
всегда помнить как ,,мирное население донбасса,, при молчаливом выжидании
самого богатого украинца-дончанина слили донбасс вместе с ,,мирным населением,,
...

\begin{itemize} % {
\iusr{Муса Магомедов}
\textbf{Владимир Вередин} про Крым ничего не хотите сказать?

\iusr{Владимир Вередин}
\textbf{Муса Магомедов} , Крым сдал Янукович при ,,харківських угодах,, ... как бы тоже донецкий персонаж ...

\iusr{Татьяна Данильченко}
\textbf{Владимир Вередин} а Турчинов с Яценбком совсем ни причём???


\iusr{Муса Магомедов}
\textbf{Владимир Иванович}, очень удобно смотреть на историю, забывая половину фактов, ту что не нравится и не ложится в Вашу картину мира. Продолжайте в том же духе, оно так проще.

\iusr{Ольга Алексеевна Хвощинская}
\textbf{Муса Магомедов} для некоторых история - это события произошедшие, а для других это трактуется как байки.
\end{itemize} % }

\iusr{Александр Мелешко}

Двадцатые числа июля дали нам всем понять, что это не просто какие то сводки в
новостях. Что вот это все где то там в Славянске, и что нас это не касается и
снова скоро будет мир и спокойствие  @igg{fbicon.face.unamused} . Первые прилеты на территорию АКХЗ, весь
город в коксовом дыму и отсутствие электроснабжения - как вчера помню свое
беспомощное состояние, то что мы все стали заложниками ситуации и абсолютно не
знаем что происходит, что нас ждёт завтра. Но при этом от каждого из нас
зависит что-то, и мы каждый можем повлиять, пусть даже случайно оказаться в
нужном времени и месте и помочь аварийной бригаде добраться на место для
устранения аварии. И каждый тот день это теперь история 

\href{https://youtu.be/a6HcNbYIeYE}{%

}



\end{itemize} % }
