% vim: keymap=russian-jcukenwin
%%beginhead 
 
%%file 30_04_2020.fb.fb_group.story_kiev_ua.2.kievskie_mozaiki
%%parent 30_04_2020
 
%%url https://www.facebook.com/groups/story.kiev.ua/posts/1338751132988351/
 
%%author_id fb_group.story_kiev_ua,petrova_irina.kiev
%%date 
 
%%tags gorod,kiev,mozaika
%%title Про київські мозаїки
 
%%endhead 
 
\subsection{Про київські мозаїки}
\label{sec:30_04_2020.fb.fb_group.story_kiev_ua.2.kievskie_mozaiki}
 
\Purl{https://www.facebook.com/groups/story.kiev.ua/posts/1338751132988351/}
\ifcmt
 author_begin
   author_id fb_group.story_kiev_ua,petrova_irina.kiev
 author_end
\fi

Про київські мозаїки. Почну з кінця. Дуже приємно було, коли в розповіді
згадали про участь нашої групи у відновлені однієї із київських мозаїк. Це фото
відкриває ряд илюстрацій. Буде їх ще багато. Всі мозаїки, про які піде розмова
- надбання 70-х - 80-х років, тоді це стало дуже поширеним явищем. Від
"сталінського  вампіра" перейшли до "хрущовського барако" і сірі стіни будинків
без прикрас стали "підживляти" різними красками мозаічних панно. Інститути
Академії наук  прикрашені різними сюжетами - інститут ядерної фізики,
онкології, кібернетики, фізики. В останньому дуже цікава історія - як у ті часи
автор використав тему купола храму, та ще назвав своє полотно "Ставропігія"?
Інститут гігієни.

\begin{multicols}{2}
\ii{30_04_2020.fb.fb_group.story_kiev_ua.2.kievskie_mozaiki.pic.1}
\ii{30_04_2020.fb.fb_group.story_kiev_ua.2.kievskie_mozaiki.pic.2}
\ii{30_04_2020.fb.fb_group.story_kiev_ua.2.kievskie_mozaiki.pic.3}
\ii{30_04_2020.fb.fb_group.story_kiev_ua.2.kievskie_mozaiki.pic.4}
\ii{30_04_2020.fb.fb_group.story_kiev_ua.2.kievskie_mozaiki.pic.5}
\ii{30_04_2020.fb.fb_group.story_kiev_ua.2.kievskie_mozaiki.pic.6}
\ii{30_04_2020.fb.fb_group.story_kiev_ua.2.kievskie_mozaiki.pic.7}
\ii{30_04_2020.fb.fb_group.story_kiev_ua.2.kievskie_mozaiki.pic.8}
\ii{30_04_2020.fb.fb_group.story_kiev_ua.2.kievskie_mozaiki.pic.9}
\ii{30_04_2020.fb.fb_group.story_kiev_ua.2.kievskie_mozaiki.pic.10}
\ii{30_04_2020.fb.fb_group.story_kiev_ua.2.kievskie_mozaiki.pic.11}
\ii{30_04_2020.fb.fb_group.story_kiev_ua.2.kievskie_mozaiki.pic.12}
\ii{30_04_2020.fb.fb_group.story_kiev_ua.2.kievskie_mozaiki.pic.13}
\ii{30_04_2020.fb.fb_group.story_kiev_ua.2.kievskie_mozaiki.pic.14}
\ii{30_04_2020.fb.fb_group.story_kiev_ua.2.kievskie_mozaiki.pic.15}
\ii{30_04_2020.fb.fb_group.story_kiev_ua.2.kievskie_mozaiki.pic.16}
\ii{30_04_2020.fb.fb_group.story_kiev_ua.2.kievskie_mozaiki.pic.17}
\ii{30_04_2020.fb.fb_group.story_kiev_ua.2.kievskie_mozaiki.pic.18}
\ii{30_04_2020.fb.fb_group.story_kiev_ua.2.kievskie_mozaiki.pic.19}
\ii{30_04_2020.fb.fb_group.story_kiev_ua.2.kievskie_mozaiki.pic.20}
\ii{30_04_2020.fb.fb_group.story_kiev_ua.2.kievskie_mozaiki.pic.21}
\ii{30_04_2020.fb.fb_group.story_kiev_ua.2.kievskie_mozaiki.pic.22}
\ii{30_04_2020.fb.fb_group.story_kiev_ua.2.kievskie_mozaiki.pic.23}
\ii{30_04_2020.fb.fb_group.story_kiev_ua.2.kievskie_mozaiki.pic.24}
\ii{30_04_2020.fb.fb_group.story_kiev_ua.2.kievskie_mozaiki.pic.25}
\ii{30_04_2020.fb.fb_group.story_kiev_ua.2.kievskie_mozaiki.pic.26}
\ii{30_04_2020.fb.fb_group.story_kiev_ua.2.kievskie_mozaiki.pic.27}
\ii{30_04_2020.fb.fb_group.story_kiev_ua.2.kievskie_mozaiki.pic.28}
\end{multicols}

Для багатьох учасників нашої групи Поділ - найрідніше та найкрасивіше місце на
землі. Та так воно і є. А Річковий вокзал - одна з перлин Подолу.  Тут все для
відпочинку та спорту!

Відомі будинки-килими на проспекті Перемоги. Одна з композицій під назвою "На
защиту мира" носила в народі назву "Вася з ножичком". Цікавий елемент -
трипільські визерунки.

Бульвар Лесі Укоаїнки. Мозаїчні панно з видами України Карпати, море, а ось
пінгвіни - це вже від тодішнього захоплення подорожами до Антарктиди.

Бульвар Шевченка. Це робота Ольги Рапай, дочки відомого єврейського поета
Переца Маркеша. Вона була репресована, повернулась у Київ. Всю плитку і все
панно в цілому зроблено вручну!!!

Єдина із збережених панно славетної Алли Горської "Вітер" на ресторані
"Вітряк". Про долю цієї дивовижної жінки та її чоловіка писати дуже довго.
Інформацію можна знайти. Знайдіть, почитайте! Це варто знати!

Майстри Ада Рибачук та Володимир Мельнічук, на центральному автовокзалі. Там є
дуже цікаве панно нашої центральної площі!

Дворець піонерів тодішній, зараз творчості молоді - мозаїк безліч, казки,
птиці, хлопчик скрипалик. І неймовірний фонтан "Зірки та сузір'я".

Доповнення, розлогіші описання вітаються. Ця тема оголошується автором "free of
politics". Прохання до адміну та модераторів викидати політичні інсінуації. Ми
ПРОСТО згадуємо красу нашого Міста! Дякую за розуміння.

\ii{30_04_2020.fb.fb_group.story_kiev_ua.2.kievskie_mozaiki.cmt}
