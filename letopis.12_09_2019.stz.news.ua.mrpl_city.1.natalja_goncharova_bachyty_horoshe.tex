% vim: keymap=russian-jcukenwin
%%beginhead 
 
%%file 12_09_2019.stz.news.ua.mrpl_city.1.natalja_goncharova_bachyty_horoshe
%%parent 12_09_2019
 
%%url https://mrpl.city/blogs/view/mariupolchanka-natalya-goncharova-golovnenavchitisya-bachiti-horoshe
 
%%author_id demidko_olga.mariupol,news.ua.mrpl_city
%%date 
 
%%tags 
%%title Маріупольчанка Наталя Гончарова: "Головне – навчитися бачити хороше!"
 
%%endhead 
 
\subsection{Маріупольчанка Наталя Гончарова: \enquote{Головне – навчитися бачити хороше!}}
\label{sec:12_09_2019.stz.news.ua.mrpl_city.1.natalja_goncharova_bachyty_horoshe}
 
\Purl{https://mrpl.city/blogs/view/mariupolchanka-natalya-goncharova-golovnenavchitisya-bachiti-horoshe}
\ifcmt
 author_begin
   author_id demidko_olga.mariupol,news.ua.mrpl_city
 author_end
\fi

Працюючи у двох театрах – професійному та народному – вона віддається на всі
100\%. Можливо, саме завдяки такій самовідданості, наполегливості і старанній
роботі її вистави у народному театрі \textbf{\enquote{Театроманія}} мають великий успіх. Мова
піде про талановиту та енергійну \textbf{Наталю Олександрівну Гончарову} – корінну
маріупольчанку, яка вміло пробуджує маріупольських глядачів, змушує їх щиро
сміятися і плакати, співпереживаючи героям вистав.

\ii{12_09_2019.stz.news.ua.mrpl_city.1.natalja_goncharova_bachyty_horoshe.pic.1}

Наталя народилась і виросла в Маріуполі, в сім'ї справжніх інтелектуалів. За
освітою батьки – інженери. Мати – програміст, тато – конструктор. Саме мама
часто водила доньку в театр. З дитинства батьки багато читали Наталці, вони
зуміли прищепити їй любов не тільки до літератури, а й до самовдосконалення і
роботи над собою. Загалом вдома було дуже багато книг, адже батьки регулярно
поповнювали домашню бібліотеку.

Спочатку дівчина навчалася в Харкові і намагалася освоїти акторську професію,
але, зрозумівши, що це не зовсім її, університет покинула. Вже пізніше, коли
народився перший син, вона вступила до Київського національного університету
культури і мистецтв та отримала спеціальність режисера театру. Дванадцять років
Наталя працювала в Дніпродзержинському академічному музично-драматичному
театрі.

\textbf{Читайте також:} \emph{Антон Тельбизов: человек за занавесом}%
\footnote{Антон Тельбизов: человек за занавесом, ДК Молодежный, mrpl.city, 20.02.2019, \par%
\url{https://mrpl.city/blogs/view/anton-telbizov-chelovek-za-zanavesom}
}
\ii{12_09_2019.stz.news.ua.mrpl_city.1.natalja_goncharova_bachyty_horoshe.pic.2}

До Маріуполя Наталя повернулася в серпні 2014 року. Жінка усвідомлювала, що в
настільки складні для всієї країни часи її сім'я повинна бути саме тут – у
рідному Маріуполі. У 2015 році Наталія почала працювати в \textbf{Народному театрі
\enquote{Театроманія}}. З 2016 року працює репетитором з техніки мови в \textbf{Донецькому
академічному обласному драматичному театрі (м. Маріуполь)}. Справа в тому, що
жоден ВНЗ не готує фахівців з мови, тому актори та режисери можуть викладати
цей предмет. Для Наталії Олександрівни і викладацька діяльність (працює і в
Музичному училищі на спеціальності \enquote{Акторська майстерність}), і робота з
акторами, і постановка вистав є тим рушієм, який потрібен кожній творчій і
талановитій людині. Робота для неї – це справжнє хобі, сам процес може її
надихати. Наталя підкреслює, що підготовка вистав з артистами \enquote{Театроманії}
приносить їй величезне задоволення і задоволення не менше, аніж робота з
професіоналами. Сьогодні у режисерки вже є своя аудиторія шанувальників, адже
її вистави – \enquote{Готель двох світів}, \enquote{Королівські ігри} – користуються неабиякою
популярністю серед маріупольців. До речі, вже \textbf{21 вересня} відбудеться нова
прем'єра Наталії Олександрівни \textbf{\enquote{Мій великий кенгуру}} – лірична комедія про
сімейні цінності з несподіваним поворотом. На майбутнє мисткиня планує
співпрацю з філармонією. Це буде вражаючий і незабутній творчий проект для
маріупольців, в якому запланована і участь \textbf{Василя Крячка}. Кожна вистава для
режисерки – це справжній Еверест. Проте абсолютно недосяжною висотою
залишається А. П. Чехов, робота над п'єсами якого поки що є найбільш складною
місією. 

Сьогодні у Наталі вже є розроблені методики з техніки мовлення і постановки
голосу. Вона дуже хоче проводити майстер-класи і серед звичайних маріупольців.
Вже на початку жовтня у кожного, хто бажає покращити власну вимову, з'явиться
така можливість, адже Наталія Олександрівна проведе великий безкоштовний
майстер-клас (пізніше з'являться оголошення, в яких будуть зазначені і точна
дата, і час). Завдяки майстер-класам, що проводить професійний репетитор з
техніки мови, можна навчитися робити правильні наголоси, почати працювати з
власним диханням, зрозуміти, які існують проблеми у мовленні, спробувати їх
подолати і, найголовніше, набути впевненості в собі.

\ii{insert.read_also.demidko.krjachok}
\ii{12_09_2019.stz.news.ua.mrpl_city.1.natalja_goncharova_bachyty_horoshe.pic.3}

Самовіддано працюючи, жінка успішно виконує і головну роль в своєму житті –
виховує двох синів – Дениса (20 років) та Всеволода (6 років). Старший вже
працює, а молодший тільки починає відкривати власні таланти (нещодавно почав
займатися танцями). Чоловік Наталі працює теж в маріупольському театрі. Інколи
подружжя має різні погляди на вистави, чи гру інших акторів, але найголовніше
те, що вони є справжньою опорою один одному. Водночас Наталія Олександрівна
підкреслила, що, працюючи з чоловіком, вона зрозуміла, що він дійсно
професіонал, прискіпливий і вимогливий до себе, тому працювати з ним легко і
приємно. 

\ii{12_09_2019.stz.news.ua.mrpl_city.1.natalja_goncharova_bachyty_horoshe.pic.4}

Наша героїня завжди дуже любила Маріуполь, який є для неї \emph{\enquote{справжньою
підзарядкою}}. Згадує, що коли навчалася в Харкові, часто звертала увагу на
великий білборд з написом: \emph{\enquote{Харків – наше рідне місто}}. Цей напис трохи
дратував маріупольчанку, яка змалечку безмежно любила свою малу Батьківщину.
Найбільше Наталці подобається сквер біля театру. Водночас полюбляє просто
гуляти вулицями міста. Вона наголошує, що краще не загострювати увагу на
негативних рисах міста, як це полюбляють робити. Завжди легше бачити тільки
погане, але це не зовсім правильна і об'єктивна позиція, яка до того ж є
руйнівною. Тому своє місто необхідно не просто любити, а знати його сильні
сторони та у разі потреби вміти захистити. 

\textbf{Улюблена книга:} \enquote{Лисиці у винограднику}, \enquote{Мудрість дивака, або Смерть і перетворення Жан Жака Руссо} та інші твори Ліона Фейхтвангера.

\textbf{Улюблений фільм:} \enquote{Краса по-американськи} (1999).

\textbf{Хобі:} у дитинстві колекціонувала марки. Сьогодні справжнім хобі є робота.

\textbf{Порада маріупольцям:} 

\begin{quote}
\em\enquote{Любіть своє місто і пишайтеся тим, що є, шукайте в ньому хороше. Погане легко
знайти, складніше навчитися бачити хороше. Це більше стосується внутрішнього
погляду людини. Необхідно щось робити для свого міста. Важливо робити те, що
можете. При цьому не обов'язково бути депутатом, можна просто докладати якихось
зусиль для змін в кращу сторону!}. 
\end{quote}

\emph{Фото з архіву героїні. Групове фото - Євген Сосновський.}

\textbf{Читайте також:} \emph{Станислав Боклан приглашает мариупольцев на спектакль - показ можно посетить бесплатно}%
\footnote{Станислав Боклан приглашает мариупольцев на спектакль - показ можно посетить бесплатно, Анастасія Селітріннікова, %
mrpl.city, 11.09.2019, \par%
\url{https://mrpl.city/news/view/stanislav-boklan-priglashaet-mariupoltsev-na-spektaklpokaz-mozhno-posetit-besplatno-video}
}
