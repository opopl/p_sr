%%beginhead 
 
%%file 20_08_2023.fb.mariupol_neskorenyj.1.rasim_sejdimov
%%parent 20_08_2023
 
%%url https://www.facebook.com/100066312837201/posts/pfbid02s8Gq2oyg7h17LVoKsfJe4AtDWPaaBx4FXcbQAYzohhrCJHkby8XDfjs975V1RSxsl
 
%%author_id mariupol_neskorenyj
%%date 20_08_2023
 
%%tags 
%%title Расім Сейдімов
 
%%endhead 

\subsection{Расім Сейдімов}
\label{sec:20_08_2023.fb.mariupol_neskorenyj.1.rasim_sejdimov}

\Purl{https://www.facebook.com/100066312837201/posts/pfbid02s8Gq2oyg7h17LVoKsfJe4AtDWPaaBx4FXcbQAYzohhrCJHkby8XDfjs975V1RSxsl}
\ifcmt
 author_begin
   author_id mariupol_neskorenyj
 author_end
\fi

📍Ще один учасник виставкового проєкту \enquote{Маріуполь нескорений} - Rasim Seidimov.
Художник живе та працює у Києві. Член Спілки художників України з 2008 року.

✅️Є учасником групових виставок, зокрема \enquote{Бітломанія}, галерея \enquote{Да Вінчі},
Київ, 2008; Гоголь-фест, Мистецький арсенал, Київ, 2009; \enquote{Ті, хто прийшли у
2000-ні}, ЦСМ М17, Київ, 2010. Персональні виставки: ЦСМ \enquote{Совіарт}, 2008-2009;
Міжнародний фестиваль сучасного мистецтва, Ба\hyp{}ден-Баден, 2016, 2017 року.

🖼 Творчий метод Расіма Сейдімова полягає в осмисленні східної архаїки та
викликів західного сучасного мистецтва. Художник працює у стилі  етно-модерн,
який поєднує у собі східні архаїчні мотиви та дух сучасних інновацій. Головною
особливістю його творчості є синтез візуального та музичного засад. Картини
Расіма Сейдімова – це поєднання внутрішнього світу людини та зовнішнього
простору глобалізованого світу, що розмиває кордони між Сходом та Заходом.

✨️Використовуючи різні матеріальні предмети, художник створює духовну
архітектуру свого інтуїтивного відчуття. 

\enquote{Я є Міст, Ми є Міст, що з'єднує Все, і ми є Дзеркало, в якому  відбивається
все створене нами. Я торкаюся простих предметів і  створюю між ними відносини і
ритміку, образи і сенс, інтуїтивно відчуваючи предмети, я усвідомлюю гармонію
Природи і Всесвіту, ту  гармонію, що відкриває шлях до справжніх почуттів і
їхнього втілення}.🎨

Роботи художника зберігаються у Музеї сучасного мистецтва України  (Київ),
Криворізькому краєзнавчому музеї, Володимир-Волинському історичному музеї, а
також у приватних колекціях в Україні, Азербайджані, США, Німеччині та Великій
Британії. 

Нагадуємо, що вже завтра о 21:00 відбудеться відкриття виставка \enquote{Ма\hyp{}ріуполь
нескорений}, де можна буде побачити і картини Расіма Сейдімова.

Департамент культурно-громадського розвитку Маріупольської міської ради \par
Diana Tryma \par
Костянтин Чернявський \par
Група \enquote{Національна Спілка Художників України} \par
Галерея мистецтв \enquote{Лавра} \par
КУ \enquote{Маріупольський краєзнавчий музей} \par
Маріупольська міська рада \par
Місто Марії \par
ЯМаріуполь \par
\#Маріупольнескорений \#виставка \#культурнадеокупація \#Маріуполь \#Київ \#художники\par
