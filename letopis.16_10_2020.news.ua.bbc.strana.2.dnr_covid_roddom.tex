% vim: keymap=russian-jcukenwin
%%beginhead 
 
%%file 16_10_2020.news.ua.bbc.strana.2.dnr_covid_roddom
%%parent 16_10_2020
%%url https://strana.ua/news/294840-koronavirus-v-dnr-statistika-karantin-distatsionnoe-obuchenie.html
%%tags bunt,roddom,dnr,covid
%%author strana.ua

%%endhead 

\subsection{Бунт в бывшем роддоме и очереди в аптеках. Что происходит с коронавирусом в "ДНР"}
\label{sec:16_10_2020.news.ua.bbc.strana.2.dnr_covid_roddom}

\index[rus]{ДНР}
\index[rus]{коронавирус}

\url{https://strana.ua/news/294840-koronavirus-v-dnr-statistika-karantin-distatsionnoe-obuchenie.html}

\ifcmt
img_begin 
	url https://strana.ua/img/article/2948/40_main.jpeg
	caption Камуфляжные маски, которые шьют в "ДНР". Фото: ТАСС 
	width 0.7
img_end
\fi

В "ДНР", как и на подконтрольной территории Украины, резко пошла в рост
заболеваемость коронавирусом.

В Донецке с начала октября вновь вернули ряд санитарных ограничений. Школьников
отправили на каникулы, студентов - на дистанционку. 

В остальном все карантинные меры в "республике" вводят с оглядкой на Москву -
они схожи с теми, что в столице России. Работодателям рекомендовали перевести
сотрудников на удаленку, жителей в возрасте 65+ попросили воздержаться от
выхода из дома. 

"Страна" рассказывает, как обстоят дела в "ДНР" с коронавирусом. Также мы
опросили местных жителей, которые высказали свое мнение по поводу действующего
карантина в "республике". 

\subsubsection{Статистика коронавируса в "ДНР"}

В "ДНР", как в Украине и России, с последних чисел сентября резко увеличилось
число новых случаев коронавируса. 9 октября был установлен рекорд - плюс 200
инфицированных.

Вчера, 15 октября, тоже были высокие показатели - 153 заболевших. Но даже это
цифра большая по сравнению с весенним пиком. Для сравнения, 13 мая там выявили
лишь 9 новых инфицированных. 

Утверждать, что сейчас в "ДНР" кратно больше больных, чем весной, нельзя. Дело
в том, что тогда тесты в "республике" были дефицитными и фиксировали лишь
откровенно больных ковидом - с двухсторонней пневмонией и нуждающихся в
аппарате ИВЛ. 

Общее число заболевших ковидом с начала пандемии - 4905. Скончались 324
человек.

Эти цифры опубликованы на сайте "Минздрава ДНР". Киев статистику о коронавирусе
на неподконтрольных территориях не собирает. 

Из-за карантина по решению Киева до 31 октября прекратил работу единственный
действовавший пункт пропуска на Донбассе - Станица Луганская.

\subsubsection{Какой карантин ввели в "ДНР" }

Впервые с лета в "ДНР" ввели карантин в начале октября. Меры схожи с теми, что
в российских городах и Москве в частности. А именно: 

\begin{itemize}
\item 1. Работодателем рекомендовано перевести сотрудников на удаленку. 

\item 2. В школах с 5 по 25 октября включительно объявлены досрочные и
расширенные осенние каникулы. В техникумах и вузах обучение
организовано в режиме онлайн. 

\item 3. Пенсионерам и людям с хроническими заболеваниями рекомендовано не
				покидать жилье без острой необходимости. 
\end{itemize}

Также останавливают работу различные ведомства: о временном прекращении приема
заявил "Центральный республиканский банк", также за последние дни перестали
работать городские театры и филармония, которые сейчас являются средоточием
культурной жизни Донецка и переживают настоящий ренессанс. 

Жители Донецка, с которыми поговорила "Страна", рассказали об обстановке в
городе.

"У нас, в отличие от Украины и России, никогда не было жестких ограничений на
передвижение по городу. Общественный транспорт ходил даже весной, маршрутки
были забиты битком. Сейчас точно так же", - говорит нам дончанка Татьяна. 

По словам женщины, все больше людей начинают надевать маски. Но в магазинах
покупателей без них все равно продолжают обслуживать. "Весной в АТБ раньше
стоял охранник на входе. Без масок просто разворачивал, измеряли температуру.
Сейчас такого пока нет", - рассказывает Татьяна.

Дончанка недовольна досрочными каникулами, которые ввели в школе, и объяснила
причину: "В школе, помимо учебы, детей кормят и они находятся там под
присмотром. Сейчас они сидят дома весь день в планшетах. Но гарантии, что они
не заболеют, все равно нет, потому что мы с мужем ходим на работу. А вот то,
что мой первоклассник не научится нормально читать, писать и считать - в этом я
уверена". 

Несмотря на большие каникулы в школах, ограничение на передвижение детей по
городу (как, например, в Москве) нет.

"Конечно, зачем детям запрещать перемещение? Вот приехал российский цирк на
льду. Нас туда агитируют ходить. Зал под завязку, большинство детей без масок.
Очень хороший карантин получается - со своими клоунами, которые его вводили", -
говорит женщина.

Фото циркового представления в Донецке постят в соцсетях. После этого в городе
заговорили о закрытии цирка на карантин

\ifcmt
pic https://strana.ua/img/forall/u/10/91/%D0%A1%D0%BD%D0%B8%D0%BC%D0%BE%D0%BA_%D1%8D%D0%BA%D1%80%D0%B0%D0%BD%D0%B0_2020-10-15_%D0%B2_17.20_.10_.png
\fi

По словам жительницы Донецка, на днях педагог в родительском чате задавал
вопрос: у всех ли есть дома компьютер с подключением Интернету. Из чего женщина
сделала вывод, что после трехнедельных каникул их переведут на дистанционное
обучение. 

\subsubsection{Очереди в аптеки, дефицит антибиотиков}

Другая дончанка - Ирина - не верит в официальную статистику. По ее мнению, она
занижена в разы.

"Сто пятьдесят человек заболевших в сутки - это, наверное, в одном микрорайоне
Донецка. Зайдите в любую аптеку: противовирусные, антибиотики, жаропонижающие -
все разметают сразу после завоза", - говорит Ирина.

В местных телеграм-каналах действительно много фотографий очередей в аптеках. В
них не помещаются все желающие купить лекарства. Многие стоят на улице или же
хвост очереди виден в холле торгового центра.

\ifcmt
pic https://strana.ua/img/forall/u/0/92/%D0%BE%D1%87%D0%B5%D1%80%D0%B5%D0%B4%D1%8C_%D0%B2_%D0%B0%D0%BF%D1%82%D0%B5%D0%BA%D0%B5.jpg
pic https://strana.ua/img/forall/u/0/92/%D0%BE%D1%87%D0%B5%D1%80%D0%B5%D0%B4%D1%8C-2(1).jpg
\fi

Горожане рассказывают "Стране", что популярное лекарство от ковида -
азитромицин - только недавно начал попадаться в аптеках. За упаковку просят 120
рублей - это чуть больше 40 гривен. Точно такая же цена у этих капсул и Киеве. 

Также в донецких аптеках рекламируют российский препарат "Коронавир" за 15 000
рублей, который в РФ стоит порядка 11-12 тысяч. 

\ifcmt
img_begin 
	url https://strana.ua/img/forall/u/0/92/%D0%B0%D0%BF%D1%82%D0%B5%D0%BA%D0%B0_%D0%BD%D0%B0%D1%80%D0%BE%D0%B4%D0%BD%D0%B0%D1%8F.jpg
	caption Фото паблика "Типичный Донецк" 
	width 0.7
img_end
\fi

Коронавир - это российское название препарата фавипиравира, который был
разработан еще несколько лет назад в Японии для лечения гриппа, но позже
показал свою эффективность и против других вирусов, например Эбола.

После начала пандемии фавипиравир начали испытывать против коронавируса в
Китае, Италии и самой Японии. Там обнаружили, что он помогает. Летом в России
зарегистрировали свой препарат на его основе - "Коронавир".

В рекламных проспектах заявляется, что коронавирус отступает после приема двух
упаковок препарата. Правда, недавнее исследование показало, что фавипиравира в
целом нашли ряд побочных эффектов для моторных функций человека: синдромов
ночного хождения и быстрого бега, а также повышенного риска падений.

Пока в Минздраве РФ изучают это исследование и при необходимости внесут
изменения в спецификацию препарата.

При этом с масками в Донецке проблем нет. Сейчас они продаются примерно по 15
рублей за штуку (5,5 гривен). В Киеве за одну маску берут 11 гривен - то есть
вдвое дороже.

\subsubsection{Что происходит в больницах}

По данным "Страны", большинство людей лечатся от коронавируса в "ДНР" на дому -
как и в Киеве. Но все-таки не все. Под коронавирусные госпитали
перепрофилировали два роддома - №6 в районе площади Ткаченко и №3 в Калининском
районе. 

С последним возник серьезный скандал. В соцсетях распространили видео, где
медики отказались выходить на работу. Они мотивировали это тем, что у них не
хватает ни халатов, ни масок, ни средств дезинфекции.

"Никто не имеет права меня заставить. Нормально: чего ты вчера не вышла? Ну,
берите и выходите. У вас же все готово было. Вы же готовились", - возмущались
медработники.

"Должны были принимать с понедельника, а нас тут в субботу тупо кинули, когда
еще мамы с детьми тут были" - рассказали те, кто отказался выходить на работу.

Также собравшиеся рассказали, что покупают все за свои деньги, причем это не
спецкостюмы, а защитные костюмы для покраски с авторынка.

"Это вчера ездила на автомобильный рынок и там где-то покупала. Для покраски.
На себя, на меня. Это не тот костюм! Защитите людей сначала, потом
предлагайте!"

\url{https://youtu.be/x8MMNHCjs3I}

Кроме этого сотрудники больницы обсуждают ситуацию с трупами умерших от
коронавируса, которые пролежали несколько дней. В соцсетях пишут, что в субботу
в больнице умер человек и до понедельника его родным не сообщали об этом.

\ifcmt
pic https://strana.ua/img/forall/u/0/34/%D0%A1%D0%BD%D0%B8%D0%BC%D0%BE%D0%BA(232).JPG
\fi

Основные претензии персонала были в том, что им не обеспечили безопасные
условия работы.

\url{https://youtu.be/QQXulqA8rX8}

Руководство больницы подтвердило, что около 30 человек отказались проходить на
рабочие места.

В "Минздраве ДНР" заявили, что после перепрофилирования родильных отделений № 1
и № 2 ЦГКБ № 3 под госпитальную базу для пациентов с Covid-19 "среди
сотрудников возникла социальная напряженность" и что для разрешения ситуации
предпринимаются меры. Также там отметили, что третья больница обеспечена
одноразовыми средствами защиты (халаты, маски, респираторы, дезинфицирующие
средства и др.), решается вопрос поступления многоразовых средств защиты.

Тесты на ковид в Донецке делают бесплатно в случае, если человек с подозрением
на него вызовет скорую. Тогда потенциального больного госпитализируют, всю
семью отправляют на карантин.

"На такой расклад сами люди идут в крайних случаях. Потому что даже если вдруг
коронавируса нет, то в стационаре он точно появится", - рассказывает нам
дончанка Ирина. 

Проблема бесплатных тестов - как и в Киеве, приходится долго ждать результатов.
Подчас более недели. 

В частном порядке делают за 2500 рублей - на украинские деньги это 900 гривен.
Примерно та же цена, что и в Киеве. Тесты на антитела можно сдать за 1000
рублей. 

Жители Донецка, с которыми поговорила "Страна",  жесткого локдауна в "ДНР" не
ждут.

"Даже если в России будет тотальный карантин, в "ДНР" его не введут. Иначе мы
умрем не от коронавируса, а от отсутствия денег", - заявил нам дончанин Данил. 
