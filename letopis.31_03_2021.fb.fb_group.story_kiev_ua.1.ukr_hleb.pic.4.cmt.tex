% vim: keymap=russian-jcukenwin
%%beginhead 
 
%%file 31_03_2021.fb.fb_group.story_kiev_ua.1.ukr_hleb.pic.4.cmt
%%parent 31_03_2021.fb.fb_group.story_kiev_ua.1.ukr_hleb
 
%%url 
 
%%author_id 
%%date 
 
%%tags 
%%title 
 
%%endhead 

\iusr{Semyon Belenkiy}
Жданова и снова Подол!

\iusr{Олег Гринюк}

Хлебный на Жданова ( Сагайдачного ) родной Подол, на противоположной стороне в 60-70
был хлебный где мы пацанами покупали Жулики

\iusr{Simon Bricklin}
\textbf{Олег Гринюк} Жулики, что-то очень знакомое?

\iusr{Олег Гринюк}
\textbf{Simon Bricklin} небольшие продолговатые житние булочки с изюмом, сейчас
изредка покупаю на продуктовой ярмарке на Печерске

\iusr{Simon Bricklin}
\textbf{Олег Гринюк} точно, спасибо, вернулся в детство

\iusr{Юрий Панчук}
Напротив помню Бахуса

\iusr{Людмила Задерей}
\textbf{Yury Panchuk} и я  @igg{fbicon.wink} 

\iusr{Volodymyr Nekrasov}
\textbf{Юрий Панчук} ой, було, було! І раки, і пиво місцеве, нефільтроване.

\iusr{Юрий Панчук}
\textbf{Volodymyr Nekrasov} ми з другом портвейн або інше вино купляли а потім можна було перекусити на другому поверсі в кафетерії в Золотому колосі

\iusr{Ганна Путова}
Так, у цьому магазині ми ще у встигли у студентські роки купувати булочки.

\iusr{Ирина Шусть}
А сейчас на этом месте впиндудили, какую-то одоробину!
