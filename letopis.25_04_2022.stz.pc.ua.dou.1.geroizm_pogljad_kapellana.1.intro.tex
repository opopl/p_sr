% vim: keymap=russian-jcukenwin
%%beginhead 
 
%%file 25_04_2022.stz.pc.ua.dou.1.geroizm_pogljad_kapellana.1.intro
%%parent 25_04_2022.stz.pc.ua.dou.1.geroizm_pogljad_kapellana
 
%%url 
 
%%author_id 
%%date 
 
%%tags 
%%title 
 
%%endhead 

Доброго дня, ком'юніті. Мене звати Віктор і я військовий капелан. Працюю
програмістом на американську продуктову компанію. Це моя перша стаття написана
українською, тому вибачте за помилки та русизми.

Підійшов до кінця вже другий місяц опору українського народу і дуже багато чого
трапилось за квітень саме в моєму житті. Сьогодні я хотів би поділитись з вами
звітом про зроблену роботу за цей місяць, обстріл балістичними ракетами в моєму
місті, зустріч з сином, котрого я не бачив 50 днів та декількома історіями від
людей, волонтерів та військових, яких ми разом з вами підтримуємо!

Мені навіть вдається поєднувати програмування та капеланську службу! Зараз я
співпрацюю з Eastern Europe Reformation Foundation, Об'єднанням Церков
Спасіння, Війcьковим Капеланським Корпусом, церквою Skeemans, ТрО та ЗСУ.

У квітні більшу частину моєї роботи складала підтримка ЗСУ, ТрО та фінансування
евакуації з Чернігівщини та Донбасу. Наразі ми евакуювали 45 людей з Києва, 70
людей з Чернігівщини, та вже майже 100 з Донбасу — загалом майже 220 людей. І в
цьому всьому є частинка кожного з вас.

Від редакції:

\begin{itemize} % {
\item стаття Віктора \href{https://dou.ua/forums/topic/37433/}{про перший місяць на війні} у ролі капелана, 
\item його розповідь \href{https://dou.ua/forums/topic/34022/}{про полон}, який довелося пережити у 2014 році в Луганську.
\end{itemize} % }
