% vim: keymap=russian-jcukenwin
%%beginhead 
 
%%file 18_07_2018.stz.news.ua.mrpl_city.1.robota_v_arhivah_mify_ta_realnist
%%parent 18_07_2018
 
%%url https://mrpl.city/blogs/view/robota-v-arhivah-mifi-ta-realnist
 
%%author_id demidko_olga.mariupol,news.ua.mrpl_city
%%date 
 
%%tags 
%%title Робота в архівах: міфи та реальність
 
%%endhead 
 
\subsection{Робота в архівах: міфи та реальність}
\label{sec:18_07_2018.stz.news.ua.mrpl_city.1.robota_v_arhivah_mify_ta_realnist}
 
\Purl{https://mrpl.city/blogs/view/robota-v-arhivah-mifi-ta-realnist}
\ifcmt
 author_begin
   author_id demidko_olga.mariupol,news.ua.mrpl_city
 author_end
\fi

\ii{18_07_2018.stz.news.ua.mrpl_city.1.robota_v_arhivah_mify_ta_realnist.pic.1}

Кожного разу я пишу про Маріуполь, його архітектурну і культурну спадщину. Але
цього разу пропоную поміркувати над іншою темою, яка теж має прямий зв'язок з
історією. Проте вона може розкрити очі на, здавалося б, прості речі.

Справа в тому, що вже не перший рік, працюючи над науковим дослідженням,
пов'язаним з театральним життям Північного Приазов'я, мені доводиться працювати
у фондах архівів та музеїв. Пам'ятаю себе в студентські роки, свій ентузіазм і
рішучість на шляху встановлення нових історичних фактів чи пошуку потрібної
інформації. Я собі придумала \textbf{\em міф}, що робота в музеях та архівах наблизить мене
до минулого, надихне і надасть сил. Але весь мій ентузіазм зникав у ту хвилину,
коли мені повідомляли, що ціна однієї світлини з фонду музею складає 50 грн, чи
та інформація, яку я знайшла в архіві є настільки цінною, що треба заплатити за
роботу з нею ще 100 грн. Оскільки я досліджую театральне мистецтво, то
важливими є фото програмок чи афіш. У моєму дослідженні вперше введено в
науковий обіг більше 286 світлин, серед яких програмки, афіші, світлини з
вистав. На жаль, майже всі фото зберігалися у фондах музеїв. Було б набагато
легше, якщо вони знаходилися у фондах архівів, адже там часто дозволяють
фотографувати - інколи безкоштовно, а інколи за невеликі кошти (до 5 грн за
одну світлину). Отже, якщо вартість однієї світлини складає \emph{50 грн}, то за всі
286 мені треба було б заплатити майже \emph{15 000 грн}. Непогано, як для науковця,
який все ж таки налаштований \underline{оптимістично і рішуче}.

Я працювала в різних музеях: у Маріуполі, Бердянську, Запоріжжі, Києві. Не
завжди одразу вдавалося отримати всю інформацію, оскільки не вистачало коштів,
але потім мені щастило отримати її завдяки особистим чи поточним архівам. Так,
особистий архів народної артистки Світлани Івановни Отченашенко допоміг мені
отримати ту інформацію, яку я не стала фотографувати в музеї. А поточний архів
Донецького академічного обласного драматичного театру (м. Маріуполь) став моїм
улюбленим архівом України. Незважаючи на те, що він несистематизований і там
потрібно провести досить масштабну роботу, але стільки матеріалів я не
отримувала \emph{БЕЗКОШТОВНО} в жодній архівній установі України.

\textbf{Читайте також:} \href{https://archive.org/details/09_10_2017.demidko_olga.mrpl_city.serce_kulturnogo_zhyttja_mariupolja}{%
Серце культурного життя Маріуполя, Ольга Демідко, mrpl.city, 09.10.2017}

Хоча все ж таки мені пощастило попрацювати ще в одному архіві, який залишив
приємні враження. Так, у Центральному державному архіві вищих органів влади та
управління України достатньо показати паспорт, щоб отримати безкоштовне
посвідчення і мати дозвіл працювати зі справами фондів, фотографувати,
сканувати на власній апаратурі. Робота в цьому архіві мене приємно вразила.
Особливо після роботи в інших архівах, де часто доводилося все писати, адже
фотографувати на свою апаратуру було заборонено, тільки робити копію за 50 – 60
грн.

Що дивує? Це те, що наука продовжує розвиватися, що попри брак коштів і
\emph{незрозумілі внутрішні статути} музеїв чи архівних установ, вчені знаходять в
собі сили йти на компроміс, чи все ж таки витрачати шалені гроші. Але так не
повинно бути!

Розуміючи, що я так просто не здамся, і буду й надалі продовжувати наукові
дослідження, я вирішила не втрачати свій оптимізм, який все ж таки небезмежний.
Саме тому я виробила власний механізм роботи в архівних установах і музеях
України. Одразу скажу, що він діє і допомагає попри всі внутрішні інституції і
статути. 

По-перше, важливо завжди мати з собою \emph{паспорт}. Краще підготувати й листа від
установи, де ви працюєте, якщо це ВНЗ, але і паспорта вистачить. Правові
проблеми доступності архівів в Україні вже давно врегульовано національним
законодавством відповідно до рекомендацій Комітету міністрів Європейського
Співтовариства стосовно європейської політики з цього питання. Зокрема, згідно
із Законом України \enquote{Про внесення змін до Закону України \emph{\enquote{Про
Національний архівний фонд і архівні установи}}}, документи НАФ і довідковий
апарат до них надаються у користування в архівних установах з часу їх
надходження на зберігання. Обмеження доступу до документів застосовується у
разі, якщо вони містять конфіденційну інформацію чи таку, що віднесена до
державної або іншої передбаченої законом таємниці. Іншими словами, якщо
установа забороняє працювати безкоштовно, тоді звертаємося до закону, згідно з
яким достатньо показати документ, що посвідчує особу (паспорт) і працювати над
описами фондів як архівів, так і музеїв. Ніяких коштів, окрім як за ксерокопії,
чи підготовку довідок витрачати непотрібно. Більше того, необхідно зберігати
спокій і бути наполегливим, намагатися ввічливо пояснити, що часи і так
складні, що коштів на опрацювання великого масиву першоджерел не вистачає,
треба ж і жити за щось.  Інколи це діє, але незавжди! Головне - \emph{закон і
паспорт}. 

Якщо ж вам безпідставно відмовили в доступі до потрібних документів або ж
надали їх не в повному обсязі, не опускайте руки. Закон дозволяє оскаржити
незаконні дії чи рішення посадових чи службових осіб архівів до їхнього
керівництва, до органу вищого рівня, якому архів підпорядковується, або відразу
до суду.

Відкриті архіви – це гарна нагода плідно попрацювати над темою свого наукового
дослідження, вивчати маловідомі сторінки сімейної історії, дізнаватись більше
про минуле, намагатись відновити історичну справедливість, а можливо, й зробити
власне відкриття. Пам'ятайте, що державні архіви дійсно є відкритими для всіх
громадян України. І працювати можна як у фондах музеїв, так і архівних
установах. Головне – вміти захищати свої права і відстоювати власну позицію,
адже в іншому випадку можна понести втрати і не такі вже й малі... 

\emph{Джерело: \url{http://mrpl.city/}}

\ii{insert.author.demidko_olga}
