% vim: keymap=russian-jcukenwin
%%beginhead 
 
%%file slova.aktivist
%%parent slova
 
%%url 
 
%%author 
%%author_id 
%%author_url 
 
%%tags 
%%title 
 
%%endhead 
\chapter{Активист}
\label{sec:slova.aktivist}

%%%cit
%%%cit_head
Националисты начали травлю Девушки
%%%cit_pic
%%%cit_text
Отметим, что в Украине нет возможности получать образование на русском языке. В
рамках политики украинизации все ВУЗы переведены на украинский язык. В том
числе и в русскоязычных регионах. В ответ на видео девушки националисты начали
травлю девушки. \enquote{От имени нашего сообщества призываем все компетентные
органы, СМИ, \emph{активистов} и неравнодушных граждан немедленно отреагировать
на эти антиукраинские заявления Маргариты}, - сказано в сообщении одной из
групп \enquote{мовных \emph{активистов}}
%%%cit_comment
%%%cit_title
\citTitle{Студентка негативно высказалась об украинском языке}, , kharkov.strana.ua, 16.06.2021
%%%endcit

%%%cit
%%%cit_head
Активисты не любят, когда Чернобыль называют Чернобыль
%%%cit_pic
%%%cit_text
Недавно украинская студия-разработчик компьютерных игр GSC Game World
представила новый трейлер популярной игры S.T.A.L.K.E.R. 2 с подзаголовком
Heart of Chernobyl (\enquote{Сердце Чернобыля}).  Видеоигра выйдет только в 2022 году.
Но националисты уже пытаются сорвать запущенную рекламную кампанию, обвиняя
разработчиков в недостаточной любви к украинскому языку.  Ряд радикально
настроенных \emph{активистов} в соцсетях возмущаются официальному трейлеру игры на
русском языке и транслитерацией на английский. Они запустили в Twitter свой
флешмоб с хэштегом \verb|#ChornobylNotChernobyl| и предлагают запретить выпуск
франшизы.  \enquote{Страна} разбиралась в тонкостях очередной \enquote{зрады} на языковой
почве
%%%cit_comment
%%%cit_title
\citTitle{Сталкер 2 Сердце Чернобыля - почему хейтят видеоигру украинских разработчиков}, 
Екатерина Терехова, strana.ua, 16.06.2021
%%%endcit


%%%cit
%%%cit_head
%%%cit_pic
\ifcmt
  pic https://strana.ua/img/forall/u/11/2/%D0%BA%D0%BE%D1%80%D0%B0%D0%B1%D0%BB%D1%8C_%D1%81%D0%B0%D0%BF%D0%B0%D1%82%D0%B8%D1%81%D1%82%D0%BE%D0%B2.jpg
	caption Судно, на котором приплыли сапатисты в Европу. Фото Amehd Coca Castillo, eulixe.com
\fi
%%%cit_text
Лозунг \enquote{Вторжение началось} \emph{активисты} опубликовали на 41 языке, включая
русский, украинский и азбуку Морзе.  \enquote{Мы скажем народу Испании две простые
вещи. Первое - они нас не покорили. Мы продолжаем сопротивляться и бунтовать.
Второе - они не должны просить прощения}, - говорится в заявлении сапатистов.
Среди бунтарей, которые пересекли Атлантический океан, семь человек в возрасте
от 19 до 57 лет: четыре женщины, двое мужчин и один трансгендер по имени
Марихосе. Как пошутил один из сапатистов, их девиз: \enquote{Сдавайтесь,
гетеро-патриархальные бледнолицые!} (хотя никто такого по прибытии не
восклицал, как ошибочно написали некоторые СМИ).  Остальные прилетят самолётами
и пробудут в Европе несколько месяцев
%%%cit_comment
%%%cit_title
\citTitle{Зачем в Испанию приплыли революционеры из Мексики}, 
Александра Харченко, strana.ua, 27.06.2021
%%%endcit
