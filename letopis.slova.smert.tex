% vim: keymap=russian-jcukenwin
%%beginhead 
 
%%file slova.smert
%%parent slova
 
%%url 
 
%%author 
%%author_id 
%%author_url 
 
%%tags 
%%title 
 
%%endhead 
\chapter{Смерть}
\label{sec:slova.smert}

%%%cit
%%%cit_head
%%%cit_pic
%%%cit_text
\enquote{В субботу, 19 июня, около 4 часов утра в Теребовлянский отдел полиции
поступило сообщение от местной жительницы о том, что в доме своего брата она
обнаружила \emph{мертвой} его 28-летнюю жену. На место происшествия выезжала
следственно-оперативная группа, руководство Главного управления Национальной
полиции в области}, - говорится в сообщении.  На теле женщины имелись признаки
насильственной \emph{смерти}. Обследовав территорию хозяйства, полицейские обнаружили
на приусадебном участке \emph{мертвым} 39-летнего мужа \emph{погибшей}. По предварительным
данным, он совершил самоубийство, взорвав себя взрывным устройством.
%%%cit_comment
%%%cit_title
\citTitle{В тернопольской области мужчина убил жену утюгом и взорвал себя бомбой}, Полина Пронина, strana.ua, 22.06.2021
%%%endcit

%%%cit
%%%cit_head
%%%cit_pic
%%%cit_text
Еще раз. Вся наша жизнь состоит из череды бессмысленных и роковых \emph{смертей}.
Вчера я узнал, что мой знакомый с гайсинского рынка \emph{умер} от короны. Молодой
пацан. И на этой же неделе дальний родственник в Закарпатье \emph{умер} от рака. А
пару лет назад мамина подруга \emph{утонула} на море. Так бывает. Не все доживают до
пенсии.  И военные тоже в особой группе риска. Они \emph{гибнут} на учениях, \emph{гибнут} по
синьке, \emph{гибнут} при неосторожном обращении с оружием. И, естественно, \emph{гибнут} на
войне
%%%cit_comment
%%%cit_title
\citTitle{Наши ребята гибнут там, где у власти нет ни стратегической, ни тактической цели}, 
Игорь Лесев, strana.ua, 08.07.2021
%%%endcit

%%%cit
%%%cit_head
%%%cit_pic
%%%cit_text
Муки, страждання, сльози, \emph{смерть}, стихії, лиха, голод, багато дітей, багато
роботи, мало землі, ще менше щастя і ніяких радощів, а я живу, радію, я
щасливий, як ніхто на світі.  Але ж брат приїхав недаремно. Не для щастя він
приїхав. Сувора й холодна сила за ним, та він не бачить її і не знає. Тільки я
повинен відкрити йому очі, хоч для цього й треба мені прокричатися крізь тисячу
років.  Як це неймовірно тяжко!
%%%cit_comment
%%%cit_title
\citTitle{Тисячолітній Миколай}, Павло Загребельний 
%%%endcit

%%%cit
%%%cit_head
%%%cit_pic
%%%cit_text
У моєму організмі розпочалися трансформації. Я поставив себе на грань \emph{смерті}. І
вступив у дію мільйонолітній еволюційний рефлекс, древня материнська здатність
жити просто так, мати життя в собі, а не від поглинання енергії чи плоті інших
істот. Ця здатність не діяла досі, бо хворобливе буття, умови покаліченої
біосфери закривали її, блокували, гальмували її прояви. Людина живе в теплиці,
в умовах виродженого, дисгармонійного енергоциклу. А коли я почав пробуджувати
первісну здатність самобуття, коли усвідомив її — вона почала пробуджуватися до
дії...
%%%cit_comment
%%%cit_title
\citTitle{Вогнесміх}, Олесь Бердник
%%%endcit

%%%cit
%%%cit_head
%%%cit_pic
%%%cit_text
На завершення варто сказати, що технологія а ля «історичні герої» своє
відживає. Бандера і Сталін \emph{мертві}. Ніхто з них не встане з могили і не наведе
«порядок». Сподіваємося, що українці нарешті перехворіли історією і тепер
візьмуться за облаштування власного життя й добробуту. Змогли ж вони це зробити
в США, Канаді, Італії, Іспанії і навіть у Польщі
%%%cit_comment
%%%cit_title
\citTitle{Мертві герої проти живих людей}, 
Василь Расевич, zaxid.net, 29.10.2021
%%%endcit

%%%cit
%%%cit_head
%%%cit_pic
%%%cit_text
Жак де Молэ с удивлением думал о том, как он  смог  пережить  все  это.
Несомненно, лишь потому, что истязатели действовали с  расчетом,  и  пытки
никогда не доходили до того предела, за которым  должна  была  последовать
\emph{смерть}, и еще потому, что организм престарелого рыцаря, закаленного в боях и
походах, оказался куда более живучим, чем сам он мог предположить.  Узник упал
на колени, обратив  взор  к  бледному  лучу,  пробивавшемуся сквозь оконце
%%%cit_comment
%%%cit_title
\citTitle{Железный король}, Морис Дрюон
%%%endcit

%%%cit
%%%cit_head
%%%cit_pic
%%%cit_text
Анатолій Якименко поднял тему о том, что элита мол сама не вакцинируется и
покупает сертификаты. Я считаю, что это прекрасно в свете законов, которые они
приняли и ещё примут.  Ибо за фальшивую вакцинацию можно будет и срок поймать.
И НАЗК с прочими правоохранительными органами получат мотив проверять элиту на
предмет наличия легальной вакцинации. Проверять на детекторе лжи, проверять
докторов на декторе лжи и процессуально.  То есть, сейчас появился мощный
крючок на который можно будет подцепить многих. При желании. А такое желание
рано или поздно появится.  Вот почему я рукоплещу всему происходящему. Ковид -
могильщик социальных отношений, построенных на лжи. И социальных,
государственных институтов, которые эту ложь порождают, поддерживают и
провоцируют. Ибо себе и окружающим врать можно сколько угодно, а \emph{смерть}
не обманешь
%%%cit_comment
%%%cit_title
\citTitle{Ковид - могильщик социальных отношений, построенных на лжи / Лента соцсетей / Страна}, 
Юрий Романенко, strana.news, 07.11.2021
%%%endcit

%%%cit
%%%cit_head
%%%cit_pic
%%%cit_text
"Страна" спросила у тех, кто участвовал в тех событиях - каковы для них итоги
Евромайдана?  Вдова убитого "беркутовца" Татьяна Булитко говорит, что сейчас
вспоминают лишь о погибших протестующих. "Мой муж погиб, когда был там на
работе. Но другой человек погиб - он был там по собственному желанию. Он же не
меня там в этот момент защищал? Так почему мы говорим, что тот герой, а тот
нет? Погибли все в одном месте", - считает она.  "К чему привели эти
\emph{смерти}? Что мы получили безвиз? Сошли с русской газовой иглы? Это все
относительно. Среди моих знакомых есть люди, которые говорят, что эти
\emph{смерти} - "нормальное явление". Я этого не понимаю".  По ее мнению, в
целом для страны ситуация ухудшилась. "Получая платежки, ты пьешь корвалол. В
холодильнике я не заметила, чтобы стало повеселее. То, что дали безвиз? Так у
кого были деньги, и так ездили в Европу. А у кого их не было, деньги и не
появились", - говорит Татьяна.  По ее словам, она голосовала за Зеленского в
надежде, что какая-то правда о тех событиях всплывет. "Но потом, день за днем,
я поняла, что это все те же люди".  Майдановец Андрей Кульчицкий говорит, что
не испытывает ненависти к "беркутовцам". "Если из тысячи людей есть один, кто
убил моего отца, почему я должен относиться плохо ко всей тысяче?" 
%%%cit_comment
%%%cit_title
\citTitle{Как убивали на Майдане и что он дал стране. Воспоминания участников событий 8 лет спустя}, 
Юлия Колтак, strana.news, 21.11.2021
%%%endcit
