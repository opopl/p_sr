%%beginhead 
 
%%file 13_10_2022.fb.galushko_denis.1.koroche__mav_chest_p
%%parent 13_10_2022
 
%%url https://www.facebook.com/HalushkoDenis/posts/pfbid02YWAcWX1KZGsqgNuQJMgV749jkm3XitPxXUTqm86zakhKDj9pk5qSvgTbtW3pT5Tjl
 
%%author_id galushko_denis
%%date 13_10_2022
 
%%tags 
%%title Короче, мав честь працювати з 80-ткою в Пісках
 
%%endhead 

\subsection{Короче, мав честь працювати з 80-ткою в Пісках}
\label{sec:13_10_2022.fb.galushko_denis.1.koroche__mav_chest_p}

\Purl{https://www.facebook.com/HalushkoDenis/posts/pfbid02YWAcWX1KZGsqgNuQJMgV749jkm3XitPxXUTqm86zakhKDj9pk5qSvgTbtW3pT5Tjl}
\ifcmt
 author_begin
   author_id galushko_denis
 author_end
\fi

Верховному Головнокомандувачу Збройних Сил України, 

Президенту України 

\#ВолодимируЗеленському!

Я діючий військовослужбовець ЗСУ, що боронить та воює на східному напрямку
нашої країни та мої побратими і посестри по зброї, висловлюємо  пропозицію до
дня захисника і захисниць України!

Присвоїти 80-тій десантно штурмовій бригаді почесне звання генерала Романа
Шухевича.

Присвоїти сотенному Української Повстанської Армії Симчичу Мирославу звання
героя України, присвоїти 45-тій артилерійській бригаді звання імені полковника
Євгена Коновальця. 

Далі можете не читати, Володимире Олександровичу, то не для вас😅 то мені на
спогади

Короче, мав честь працювати з 80-ткою в Пісках.

В той день, ми вночі сиділи в засідці, чатували наших жертв, у полі, класика
підйом о 4 ранку, добу на ногах, чергування нічне по черзі люлі, ранок обстріл
танком о 7:00  різкій підйом, зміщення в бліндаж, це було в районі Марїнки.

Думали після добового полювання  потрапим в распалагу, ага, потрапили в той же
день в Піски.

Дали мені дані, сказали приїдеш туди, там тебе зустрінуть хлопці по місцю все
пояснють, розкажуть, приймуть. 

В нас як ви розумієте не завжди є часу новини передивлятися але вже чули, що
під Пісками гаряче від наших, там і наших трохи налякали, танчик хату розібрав.
То якраз був Липень – Серпень.

Ми приїхали, зустрівся я з таким гоноровим командиром з 80-тки.

Я йому:

- Машина тут буде, ми на вашій одній зїздемо, там бк просто, обладнання.
Місцевість поки не знаємо, орієнтування, мітки покажеш і на ксп назад, добре?

Він:

- Бля ми так не працюємо, нам координати дали ми їбошим, я що з вами за ручку
тут ходити буду!??

Мені одразу сподобався цей командувач, адже в нього був точно такий же патч як
і у мене, тільки в нього він був на фронтальній стороні броніка, а в мене на
спині.

З написом:

«Тому що іди на хуй! Ось чому!»

- Нє, за ручку не треба, але ризикувати машиною з добром я не буду на
незвіданих преріях. Так що працюємо так…

Він вже погодився адже цей діалог відбувся майже по дорозі до їх машини. Їх
двоє, нас двоє, вони везли нас на зустріч з іншими спеціалістами. 

На блокпості нас зупинили, щоб попередити, що далі тільки що було касетне
мінування і машина замкомбріга вже підірвалася, без жертв, та машині пізда,
щоб були обережні.

Я був правий і на двох машинах, натовпом ,нічого було туди їхати. Це таке
знаєте тепло на серці, коли вояк правий за старшого, я думаю, що навіть
посміхнувся в цей момент, але на втомленій морді то було не видно. 

Нас познайомили з ким треба, але інформації, було замало.

На наступний день  по нашим позиціям відпрацював рсзв, що чотирма пострілами
спалив нашу хату і осколками над нашими головами дав зрозуміти, що ми не туди
приїхали і не там стали де треба.  

Я дилетант, не знав, що 80-тка, це Львів, доречі згодом цей же командир,
зважаючи на мою вимову, питав в мене чи часом я не зі Львова, місто Лева і тут
мене не залишає. 

Десант, це сила, згодом, козаки виконали поставлену перед ними бойову задачу і
вирушили далі в найгарячіші точки нашого двобою з нечистю.

Подякували і нам за роботу, потисли руки. Вони лишили там полеглими чимало
своїх козаків, героїв.

Піски для всіх хто там був стали легендою, фортецею та памятником незламності
українського духу. А для ворога це чергова могила.

Честь!

Ваш  Позивний Брат
