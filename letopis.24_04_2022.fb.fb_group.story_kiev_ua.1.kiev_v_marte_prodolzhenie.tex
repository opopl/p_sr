% vim: keymap=russian-jcukenwin
%%beginhead 
 
%%file 24_04_2022.fb.fb_group.story_kiev_ua.1.kiev_v_marte_prodolzhenie
%%parent 24_04_2022
 
%%url https://www.facebook.com/groups/story.kiev.ua/posts/1909661372563988
 
%%author_id fb_group.story_kiev_ua,aleksejenko_aleksandr.kiev
%%date 
 
%%tags 
%%title Киев в марте. Продолжение
 
%%endhead 
 
\subsection{Киев в марте. Продолжение}
\label{sec:24_04_2022.fb.fb_group.story_kiev_ua.1.kiev_v_marte_prodolzhenie}
 
\Purl{https://www.facebook.com/groups/story.kiev.ua/posts/1909661372563988}
\ifcmt
 author_begin
   author_id fb_group.story_kiev_ua,aleksejenko_aleksandr.kiev
 author_end
\fi

Киев в марте.

Продолжение.

К нам уже долетали слухи о том, с каким трудом выбирались из города люди в
первый день войны. Во Львов некоторые ехали более суток. Это при том, что туда
на автомобиле в мирное время можно было добраться часов за семь-восемь. Поэтому
мы не находили себе места, когда представляли, как в этот самый момент по
забитым автомобилями дорогам где-то плетутся наши родные.

Остаток дня мы только тем и занимались, что периодически созванивались или
переписывались с ними:

- Как вы? Где вы уже?..

Ответы были неутешительны:

- Еще в пути и непонятно, когда доберемся...

Так и вышло, что они ехали более четырнадцати часов, хотя расстояние до места
назначения было чуть за триста километров. Автомобили шли сплошным потоком, а
ко всему прочему, еще хватало остановок для проверок. Одно хорошо: пусть
глубокой ночью, но они все-таки добрались до места, разместились и теперь
приходят в себя после нервной поездки.

Выдохнули и мы, радуясь, что все малыши и из одной семьи, и из другой находятся
вдалеке от взрывов и стрельбы. А у нас с каждым днем становилось все громче и
громче. В нашу сторону еще не прилетало, а в городе уже началось...

Одним из первых случаев стало попадание ракеты в жилую высотку неподалеку от
дома, в котором проживал старший сын. Будто знак свыше, потому что сразу пришло
в голову – хорошо, что они уехали...

Это место нам было очень хорошо знакомо. Рядом, буквально под этим домом,
находится торговый центр, а в нем ресторан, в котором мы часто собирались на
всевозможные семейные праздники, где принимали друзей, куда мы могли просто
заехать, возвращаясь из каких-нибудь прогулок или вылазок по городу. А теперь
наблюдаем леденящую сердце картину – ракета вырывает кусок дома... Как он еще
устоял и не сложился?..

Потом пошли новые прилеты, в городе вой сирен стал обычным делом. В нашем
жилищном комплексе народ и без сирен бегал в якобы бомбоубежище, а порой
прятался в обыкновенных подвалах. После посещения укрытия вместе с внуками в
первый день войны, мы туда больше ни разу не ходили: отнеслись к своему
спасению истинно с фаталистических позиций.

Покоя нам в целом это не добавило, но один раз приняв такое решение, мы ни разу
ему не изменили, как бы ни грозно звучали вокруг нас выстрелы, и как бы ни
жутко чертили в ярко-синем небе свои следы пролетавшие ракеты...

Гораздо тяжелее нам пришлось перенести беспрерывный прессинг от родных и
друзей:

- Срочно выбирайтесь из Киева!.. Будет мясорубка... Умоляем вас...

Иногда наши родные переходили от простых уговоров к отчаянной мольбе,
подключали друзей, которые добавляли свой голос в унисон всем, кто настаивал на
эвакуации. Случались такие эмоциональные, граничащие со слезами просьбы, что
этот напор выдержать становилось невозможным, чтобы в ответ не сорваться на
крик:

«Мы никуда не уедем!.. Отстаньте от нас... С нами всё будет хорошо!..»

Но я ни разу не позволил себе подобного кощунства, понимая, как переживают за
нас родные, друзья и добрые знакомые, сами находившиеся в безопасном месте, и
чувствуя ответственность и, возможно, даже невольный грех за оставленных не
юных ближайших родственников. Однако это давление очень влияло на наше душевное
равновесие.

А еще каждый день наблюдая, как отъезжают от подъездов перегруженные
автомобили, и как наш жилищный комплекс вымирает на глазах, да под всё
возрастающее и опоясывающее наши дома грохотание орудий и ракетных установок,
то становилось невыносимо сохранить свою психику на достойном уровне. Особенно
мне... Я и сам раздирал себе сердце:

«Верно ли мое решение?.. Может, все-таки уехать?»

К этим вопросам добавлялись постоянные мысли о том, почему и из-за чьей прихоти
сейчас убивают людей, уничтожают мирную жизнь, ломают тысячи человеческих
судеб, разрушают города и села, и в целом страну? Кто вам дал такое право,
твари?!

Отсюда и краткий сон, больше смахивающий на забытье, отсутствие аппетита,
взвинченность... В общем, какие бы мысли не крутились в голове, нужно было
защищать свое душевное спокойствие.

Я перестал брать трубку, когда прогнозировал, что сейчас абонент будет вести
весьма предсказуемый разговор: или предлагать уехать, или обсуждать вопросы
геополитики, или банальное – «ну, что там у вас»? Ни на один вариант беседы я
не был согласен. Я создавал вокруг себя защитный кокон. Перед очередной
попыткой заснуть мы глотали таблетки валерьянки, а я иногда мог принять и
снотворное. И мы приспособились.

Со временем и наши доброжелатели поняли, что нам спокойнее жить с тем решением,
которое мы приняли. До них, наконец-то, дошло, что от их вопросов попахивает
обыкновенной жаждой новостей, граничащих с циничным любопытством:

«Мол, как там у вас, на сковородке? Подгорает?..»

И темы разговоров стали более прозаичными: чем занимаетесь, как здоровье,
работают ли магазины?.. Им только оставалось ахать от неожиданности, когда во
время разговора услышат, как рявкнет очередной залп из чего-нибудь
огнестрельного...

Впрочем, привыкнуть к неожиданным залпам было сложно и нам. Одно дело –
постоянная канонада, к ней мы уже притерпелись. А вот если после затишья
раздавался первый грохот, то вздрагиваешь, как будто услышал в первый раз. Но
постепенно стрельба превращалась в осадные будни.

Защищать свою психику пришлось и от друзей и родственников из-за «поребрика».
Так теперь называют границу с восточным соседом. Казалось бы, столько лет жили
вместе, дружили, общались и вдруг в одночасье люди оттуда превратились в
кровожадных одурманенных монстров.

Поначалу даже не верилось, что они со всей серьезностью своими устами могли
транслировать российский телевизор и делали это с полной уверенностью в своей
правоте!

Когда два властных негодяя после первых пусков ракет, убивших мирных жителей
Украины, начинают вести разговор о том, что «если бы мы не начали, то через
шесть часов они ударили бы первыми…» и «поэтому это они начали, а не мы…», то
думаешь: разве нормальный человек может воспринимать подобные речи без оторопи?

И не появится ли у разумного человека уверенность, что их произносит идиот?

Если бы после этих слов всё последующее не было трагично и не касалось обычной
жизни, то можно даже улыбнуться: похоже, дурачок научился шутить... Но когда
эти слова, сидя перед наставленными на них камерами, произносят двое нелюдей,
один из которых представляет собой людоеда, а другой записного облысевшего шута
и лизоблюда, то твоя душа просто обрывается... Как же так можно?!

Когда-то в своей квартире я установил тарелку с российским телевидением, и до
2014 года все было относительно терпимо. А потом я стал замечать, как начинают
врать и перекручивать любые новости из Украины и из Киева в частности. Поначалу
я удивлялся, потому что я живу здесь и знаю реальность:

«Ну, это же неправда... Это же подлог, зачем вы так?..»

Причем, новости из Украины корреспонденты произносили с брезгливым и
презрительным выражением лица, отчего лица становились неприятными. Но это дело
вкуса: возможно, кому-то лица корреспондентов неприятными не казались. Может,
это я такой ранимый и привередливый...

С каждым последующим годом накал неприязни к Украине возрастал. Я начал
ограничивать свое присутствие у экрана просмотром только спортивных событий.
Благо, что российские спортивные комментаторы никогда не рассказывали о наших
спортсменах в уничижительном тоне. Их репортажи отличались дружелюбием и
объективностью. Но смотреть «соловьиный помет» или «истеричную сатанееву» – это
себя не уважать.

А потом для себя решил: продукт их пропагандистов рассчитан на внутреннего
потребителя, пусть радуются и распаляют себя. Это их дело. Поэтому я вообще
перестал беспокоиться по поводу словесного поноса, лившегося с «их» экрана в
«их» головы. Пусть себе купаются в этом...

Вероятно, не только я так думал, но это, надо честно признать, было ошибкой.
Российский телевизор достиг своей цели. Перегрев «внутреннего потребителя»
налицо, если столько народу поддержало людоедскую риторику – распни Украину…

И все доводы, как для уравновешенных и думающих людей, за гранью логики, а
посему кровожадны и бесчеловечны. Думал, насколько же соседний народ наивен,
как же он в своей массе недалек, если может верить, и верить с удовольствием
всему, что им предлагают...

Украина хотела напасть на Россию... Вы на карту смотрели?! Сколько там России и
сколько Украины? Даже нашпигованная оружием и людьми со всей Европы фашистская
Германия ничего сделать не смогла, а тут практически безоружная Украина...

Украина собиралась производить ядерное оружие и – как же без этого – ударить по
соседям... Ага, атомную бомбу собрать еще проще, чем кубик Рубика. Тяп-ляп и
бомба готова. Вы в своем уме?!

Да, еще и про гусей-диверсантов расскажите...

Или про «нациков» и ущемление русского языка...

Когда мне об этом агрессивно заверещала моя бывшая одноклассница, которую
теперь называю не иначе, как просто существо из соседней страны, я, человек
пишущий и думающий на русском языке, ответил:

- Теперь и меня в тот список включи! И забудь...

После чего вычеркнул ее номер отовсюду: коммуникации больше не будет, ибо
бесполезно лечить больных и одурманенных. Свои нервы нужно сберечь, а не
тратить их на умалишенных.

Продолжение следует.
