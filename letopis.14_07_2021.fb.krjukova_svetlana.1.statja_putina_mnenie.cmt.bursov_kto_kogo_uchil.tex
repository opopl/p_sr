% vim: keymap=russian-jcukenwin
%%beginhead 
 
%%file 14_07_2021.fb.krjukova_svetlana.1.statja_putina_mnenie.cmt.bursov_kto_kogo_uchil
%%parent 14_07_2021.fb.krjukova_svetlana.1.statja_putina_mnenie.cmt
 
%%url 
 
%%author 
%%author_id 
%%author_url 
 
%%tags 
%%title 
 
%%endhead 
\paragraph{Бурсов - КТО КОГО УЧИЛ? ЗА СЧЕТ КОГО СОЗДАВАЛАСЬ "ВЕЛИКАЯ РОССИЯ"?}

\begin{itemize}
%%%fbauth
%%%fbauth_id
%%%fbauth_tags
%%%fbauth_place
%%%fbauth_name
\iusr{Георгій Бурсов}
%%%fbauth_front
%%%fbauth_desc
%%%fbauth_url
%%%fbauth_pic
%%%fbauth_pic portrait
%%%fbauth_pic background
%%%fbauth_pic other
%%%endfbauth
 
КТО КОГО УЧИЛ? ЗА СЧЕТ КОГО СОЗДАВАЛАСЬ "ВЕЛИКАЯ РОССИЯ"? 

Один факт из многих. Первое высшее учебное заведение в Восточной Европе
Киево-Могилянская Академия. Училась вся Восточная Европа. Все научные
дисциплины, тогда существовашие. Екатерина посылает Ломоносова в Киев, работать
над созданием литературного языка "великоросов" (до этого - московиты).
Ломоносов полтора года штудирует изданную в Могилянке "Грамматику украинского
языка Смотрицкого", уже столетия существующего языка коренных жителей страны.
По ней Ломоносов создает грамматику для великороссов - вчера еще и надолго
потом разноязыких, разноэтнических племен, где количество славян было мизерным
(уничтоженный славянский Новгород, с расселением оставшихся в живых по
закоулкам). 

Имперское объединение этнического конгломерата происходит молитвами нового
вероисповедания на староболгарском (македонском) языке. Теперь он,
подправленный помором Ломоносовым, усовершенствованный Пушкиным, с корнями из
Эфиопии - называется "русским языком". Через 100 (С Т О) лет после
Киево-Могилянской Академии создается первое высшее учебное заведение
Великороссии Московская Академия. Первые 90 лет ее существования ректорами были
воспитанники киевской Могилянки. Далее - преподавателей из Могилянки
рекрутируют в Москву. 

А первый вуз Восточной Европы превращают в поповское заведение, для кадров так
называемой русской православной церкви. 

РАЗВИТИЕ НАУЧНО-ДУХОВНОГО ЕВРОПЕЙСКОГО ЦЕНТРА в Киеве прекращено.... В 11-12
в.в. В Киеве жило от 50 до 150 тысяч жителей. В это время Лондон -15 тысяч,
Париж - 20 тысяч. Сколько лягушек квакало на месте будущей Москвы - не
подсчитывалось.

Первый удар, замедливший развитие Руси-Украины - татаро-монгольское нашествие.

Второй удар, делающий невозможным, тормозящий развитие страны и ее народа -
ползучая, гибридная оккупация после так называемой Переяславской Рады,
документы которой отсутствуют (сварганить фальшивки в свое время не
догадались). Много чего следует уточнить из баек про "строительство в Украине"
за счет кого-то там. 

Советская империя - наследница царской - готовилась к походу на весь мир. Для
этого рабским трудом украинцев строились в Украине заводы тяжелой
промышленности (танки, самолеты...) Три голодомора. Репрессии украинцев
постоянно, более 200 лет акты и постановления, запрещающие использования в
печати украинского языка.... Чья б корова мычала, таварисчь Путин, а ваша бы
молчала.

\end{itemize}

