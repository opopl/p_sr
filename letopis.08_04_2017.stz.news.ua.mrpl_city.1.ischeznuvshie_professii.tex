% vim: keymap=russian-jcukenwin
%%beginhead 
 
%%file 08_04_2017.stz.news.ua.mrpl_city.1.ischeznuvshie_professii
%%parent 08_04_2017
 
%%url https://mrpl.city/blogs/view/ischeznuvshie-professii
 
%%author_id burov_sergij.mariupol,news.ua.mrpl_city
%%date 
 
%%tags 
%%title Исчезнувшие профессии
 
%%endhead 
 
\subsection{Исчезнувшие профессии}
\label{sec:08_04_2017.stz.news.ua.mrpl_city.1.ischeznuvshie_professii}
 
\Purl{https://mrpl.city/blogs/view/ischeznuvshie-professii}
\ifcmt
 author_begin
   author_id burov_sergij.mariupol,news.ua.mrpl_city
 author_end
\fi

\ii{08_04_2017.stz.news.ua.mrpl_city.1.ischeznuvshie_professii.pic.1.gazirovshica}

В Мариуполе на проспекте Республики у подножья здания, которое последовательно
меняло названия, начиная с гостиницы \enquote{Континенталь}, затем \enquote{Дворец труда}, Клуб
металлургов завода \enquote{Азовсталь}, Дворец культуры того же предприятия и, наконец,
Дворец молодежи в послевоенное время на низкой табуреточке сидел старик с
восточными чертами лица. Сидел, но это вовсе не значит, что он ничего не делал.
Перед ним стояла подставка с приколоченным деревянным следом подошвы. По
соседству на небольшом ящике находился арсенал его ремесла: несколько щеток с
длинным ворсом, жестяные баночки с ваксой разных цветов.

Время от времени к старику подходили местные франты и просто люди, любящие
чистоту и порядок, ставили ногу на подставку. И тут начиналось действо. Старик
живо вставлял две картонки в пространство между краем обуви и носком, легкими
движениями щетки смахивал пыль с обуви, а затем двумя другими щетками с
ошеломительной быстротой, вроде бы и не свойственной его почтенному возрасту,
начинал обрабатывать ботинок ли, сапог ли или туфлю. Щетки так и мелькали в его
сухих загоревших руках с натруженными жилами. Финалом процедуры было доведение
поверхности обуви до зеркального блеска. Для этого старик использовал бархотку
малинового цвета. Покончив с обувкой на одной ноге, старик легким щелчком щетки
о задник ботинка давал знак, что нужно сменить ногу. У старика можно было не
только почистить обувь, но и купить ваксу черную, коричневую разных оттенков
или бесцветную, а также шнурки разной длины и цветов.

Он был молчалив и только изредка мог произнести одну - две фразы, и всегда
только по делу. При этом чувствовался сильный акцент неведомого языка.
Некоторые говорили, что он - армянин. Но на самом деле старик был
представителем древнего ассирийского народа, трагическая история которого
изобиловала гонениями, уничтожением тысяч ни в чем не повинных людей, жизнью в
изгнании, рассеянию по всему белому свету. Судьба занесла этого человека и его
семью в Мариуполь. А профессия его называлась – чистильщик сапог. Профессия,
которая в нашем городе исчезла...

Летом в Мариуполе у кинотеатров, дворцов культуры, в скверах и других людных
местах можно было увидеть окрашенные в голубой цвет тележки для продажи
газированной воды, которую в народе для краткости называли газводой. Рядом с
каждой из тележек стоял баллон с углекислым газом, накрытым чехлом из белой
ткани. Нехитрое оборудование тележки состояло из приспособления для мытья
стаканов. Чтобы вымыть стакан, нужно было поставить его на круг, поворот ручки
— и струи воды омывали его изнутри и снаружи. Еще был кран, источающий газводу,
носик которого обертывался в несколько слоев марли, чтобы вода, насыщенная
газом, не разбрызгивалась по сторонам. Венчала тележку стойка с двумя
стеклянными мерными сосудами - трубками, оснащенными со дна краниками.

Один из сосудов заполнялся мандариновым или грушевым сиропом, содержимое
другого — тоже сироп, но малиновый или, скажем, вишневый. Уровень сиропа в
сосудах, на которые были нанесены горизонтальные риски, - расстояние между ними
обозначало норму сиропа на стакан - постепенно снижался, но всегда оставался
прикрытым от взоров жаждущих хлебнуть стаканчик-другой газводы
наклейкой-ценником. На ценнике была обозначена стоимость содержимого стакана
воды без сиропа — ребятня называет ее \enquote{чистой}, - с одинарной порцией сиропа и
порцией двойной. Когда-то стакан наполнялся до краев, позже — не более чем на
три четверти. Но и за то спасибо. В конце концов, можно было купить два иди
даже три стакана. Цена-то — плевая. Профессия женщины, которая обслуживала
живительный источник, называлась газировщицей. Газировщиц вытеснили автоматы
для продажи газированной воды. А потом и автоматы исчезли...

Война лишила жизни многих мариупольцев. Одни сложили головы на полях сражений,
иные скончались от ран в госпиталях, других лишили жизни гитлеровцы во время
оккупации нашего города, многие из тех, кто попал в плен, умерли от голода,
болезней и зверств фашистов. И не было ни одного дома, ни одной семьи, где бы
не было близких родственников, унесенных войной в небытие. И остались на память
о них лишь фотографии, часто любительские, пожелтевшие от времени, на которых
можно было лишь с трудом разглядеть черты лица сына, мужа, брата, сестры, отца,
матери, словом - близкого человека, потерянного навсегда.

Тогда-то и пришли на помощь, так называемые, портретисты. Как правило, они
работали в паре – фотограф и ретушер. Фотограф снимал фотографию, с негатива
делал увеличение до размера фотобумаги 24х36 сантиметров – это был самый
ходовой размер, затем шли привычные операции: проявление, фиксирование,
промывка, сушка. Изображение на фотобумаге, - кстати, нужно сказать, что она
всегда была матовой, -  иногда бывало едва различимо. Тут вступал в работу
ретушер. Он втирал в снимок порошок черного пигмента, чтобы \enquote{поднять}
изображение. Результат напрямую зависел от мастерства, вкуса и прилежности
ретушера. И сейчас можно увидеть, правда редко, в некоторых семьях портреты
близких людей, сделанных по этой технологии. До наших дней из таких
портретистов, пожалуй, никто не дожил. С ними ушла и их профессия. Между
прочим, с внедрением в быт цифровых технологий в фотографию сравнительно
недавно исчезли профессии ретушеров, лаборантов, которые в растворах
обрабатывали пленку и фотобумагу. Компьютерная технология вытеснила из
типографий специалистов по ручному набору текстов, линотипистов, цинкографов, а
в конторах - машинисток и их инструмент – пишущие машинки. Но это тема для
других воспоминаний.
