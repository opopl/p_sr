% vim: keymap=russian-jcukenwin
%%beginhead 
 
%%file 20_10_2020.fb.fb_group.vzaimovyruchka_doneck_dnr.1.pandemia_vrachi
%%parent 20_10_2020
 
%%url https://www.facebook.com/groups/vzaimoviruchka.donetsk.dnr/posts/1074432426340695
 
%%author_id fb_group.vzaimovyruchka_doneck_dnr
%%date 
 
%%tags covid,dnr,donbass,doneck,medicina,pandemia,vrach
%%title Пандемия - врачи - просьба о помощи
 
%%endhead 
 
\subsection{Пандемия - врачи - просьба о помощи}
\label{sec:20_10_2020.fb.fb_group.vzaimovyruchka_doneck_dnr.1.pandemia_vrachi}
 
\Purl{https://www.facebook.com/groups/vzaimoviruchka.donetsk.dnr/posts/1074432426340695}
\ifcmt
 author_begin
   author_id fb_group.vzaimovyruchka_doneck_dnr
 author_end
\fi

@igg{fbicon.exclamation.mark}	 @igg{fbicon.megaphone}  Инна Сасим, 19 октября:

«Пандемия не приговор!!!!

Я думала, долго думала, как правильно написать пост, чтобы быть услышанной.
Решила писать от души, от сердца, так, как вижу.

В пятницу я получила обращение от ЦГКБ №3 с просьбой выделить медицинские
препараты и средства защиты, список публикую ниже.

Просят растерянно. Я слышу в голосе врачей страх и панику. Они сейчас на
передовой, они не готовы к этому. К этому не готов целый город, весь мир. Но по
нам ударит сильнее всех, каждый здравомыслящий человек это понимает.
Непризнанная республика, практически с закрытыми границами, и шесть лет войны.
Для нас это будет еще один бой, тяжелый бой.

Я призываю обратить максимальное внимание на обращение больницы и думаю, оно
будет не единичным. Это обращение не только к фонду, это обращение ко всем нам.
Врачам и медсестрам нужна помощь: средствами защиты, лекарствами. Поймите меня
правильно, врачи должны знать, что они не одни. А мы, общественность, как
можем, помогаем врачам: церкви – молитвами, предприниматели – деньгами или
продукцией, дети – рисунками. Все детские рисунки отвезу и передам, лучшие
опубликую. Я так это вижу. В первую очередь нужно обеспечить всех врачей и
медсестер средствами защиты, иначе нас некому будет лечить!

Наши врачи были лучшими всегда, к нам приезжали люди на лечение со всей
Украины! Сейчас важно их поддержать – и они справятся!

Кто-то спросит, возможно, ну а власти, что же делают власти? Отвечаю – не знаю.
Думаю, что не сидят сложа руки – и  партия медикаментов уже на подходе, как
говорилось на брифинге. Но уверена, что пока трудно прогнозировать даже
властям, достаточно ли будет средств защиты и медикаментов. Ну нет у нас опыта
в пандемии. Нет – к сожалению или к счастью. Я считаю, мы должны сплотиться,
чтобы выжить. Пока только паника и скандалы. Возмущенные родственники, ломающие
двери больницы, и перепуганные врачи – это неправильная тактика, и мы в силах
ее изменить!

Телефон сотрудника больницы – отправлю в личку, каждому, кто пожелает лично
связаться с больницей.

Мой номер 071-338-91-82, звоните в любое время.

Список:

Памперсы для взрослых – 10 упаковок.
Пеленки одноразовые – 200 шт.
Бумажные полотенца – 500 рулонов.
Лекарственные препараты:

\begin{itemize}
  \item - антибиотики:
  \item - Цефтриаксон 1г – 6000 фл.
  \item - Цефепим 1г – 6000 фл.
  \item - Лазолван 100,0 – 200 фл.
  \item - Лазолван 2,0 – 500 амп.
  \item - Гепарин №10 – 300 уп.
  \item - Пентоксифилин (таблетки) – 500 уп.
  \item - Дексамитахон – 320 амп.
  \item - Витамин D 500 МЕ №50- 30 фл.
  \item - Витамин С (драже)-15 уп.
  \item - лорогексал- 30 уп,
  \item - энтерол-90 уп, 
  \item - Халаты влагонепроницаемые».
\end{itemize}

* Все комментарии не по существу обращения медиков ЗА ПОМОЩЬЮ -  будут
удаляться.

С конкретными предложениями обращаться к Инне, за уточнениями - это с больницей
(номер тф сотрудника - у Инны).

\begin{itemize} % {
\iusr{Инна Сасим}
Спасибо Вам Светлана Саф!!!!
\end{itemize} % }
