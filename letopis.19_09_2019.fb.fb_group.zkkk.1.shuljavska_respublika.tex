% vim: keymap=russian-jcukenwin
%%beginhead 
 
%%file 19_09_2019.fb.fb_group.zkkk.1.shuljavska_respublika
%%parent 19_09_2019
 
%%url https://www.facebook.com/groups/2430033423886749/posts/2463144183909006/
 
%%author_id fb_group.zkkk,murashova_bogdana.kiev
%%date 
 
%%tags gorod,istoria,kiev
%%title ШУЛЯВСЬКА РЕСПУБЛІКА
 
%%endhead 
 
\subsection{ШУЛЯВСЬКА РЕСПУБЛІКА}
\label{sec:19_09_2019.fb.fb_group.zkkk.1.shuljavska_respublika}
 
\Purl{https://www.facebook.com/groups/2430033423886749/posts/2463144183909006/}
\ifcmt
 author_begin
   author_id fb_group.zkkk,murashova_bogdana.kiev
 author_end
\fi

ШУЛЯВСЬКА РЕСПУБЛІКА.

На першій світлині – дідусь моєї бабусі, полковник Іван Семенович Сосновський.
Іван Семенович був героєм російсько-турецької війни. Близько 1893 року у чині
полковника він вийшов у відставку. Разом з дружиною Адель Антонівною, вони
придбали будиночок на Шулявці. Треба зазначити, що в той час Шулявка була
околицею Києва. Ця місцевість називалась Казенними дачами. Адреса будиночку
звучала як 2-га Дачна лінія, № 6.

\begin{multicols}{2}
\ii{19_09_2019.fb.fb_group.zkkk.1.shuljavska_respublika.pic.1}
\ii{19_09_2019.fb.fb_group.zkkk.1.shuljavska_respublika.pic.2}
\ii{19_09_2019.fb.fb_group.zkkk.1.shuljavska_respublika.pic.3}
\ii{19_09_2019.fb.fb_group.zkkk.1.shuljavska_respublika.pic.4}
\end{multicols}

В далекому 1905 році будиночок в якому мешкала родина Сосновських опинився в
епіцентрі революційних подій. 12 грудня 1905 року, повсталі робітники Шулявки
оголосили квартали між заводом Ґретера і Криванека (згодом завод «Більшовик»)
та Політехнічним інститутом «Робітничою республікою». Штаб бунтарів
розташувався у приміщенні 1-го корпусу Київського політехнічного інституту,
який було проголошено «столицею» республіки. Там само містилася і Рада
робітничих депутатів, якій робітники надали повноваження єдиного органу влади у
місті Києві. «Президентом» Шулявської республіки та Головою Ради депутатів був
обраний пролетар Федір Алєксєєв. 

Та вже через чотири дні — у ніч на 16 грудня 1905 року — Шулявська республіка
та повсталі робітники були розгромлені урядовими військами.

Рятуючись від жандармів, «президент» Шулявської республіки Федір Алєксєєв
забіг до нашого будиночку та попросив про прихисток. Не те, щоб члени нашої
родини підтримували революціонерів, але надати допомогу тому, хто її просить
було справою честі. Втікача заховали. Коли в двері постукали жандарми, до них
вийшов полковник Іван Семенович в офіцерському мундирі. Один з жандармів почав
щось говорити про те, що в нашому будинку міг заховитись бунтівник і, що вони
хочуть перевірити. «Как ты смееш, скотина. Я офіцер...», - командним тоном
сказав їм господар будинку. Жандарми віддали честь, вибачились та пішли до
генерал-губернатора за дозволом на обшук, а біля будинку з'явився шпік... 

«Президента» помили, поголили, одягнули в пристойний костюм і він покинув
будинок під руку з донькою Івана Семеновича, моєю прабабусею 23-х річною
Клотильдою Іванівною. Таким чином Ф. Алєксєєву вдалось утекти, проте вже
незабаром його арештували в Саратові...

Цю історію я завжди вважала легендою. Але декілька місяців тому, переглядаючи
альбоми з сімейними світлинами знайшла документ, який офіційно засвідчує даний
факт. Тому це вже не легенда, а частинка історії нашого Міста. Факт, який
можливо бути корисним професійним історикам.

\ii{19_09_2019.fb.fb_group.zkkk.1.shuljavska_respublika.cmt}
