% vim: keymap=russian-jcukenwin
%%beginhead 
 
%%file 08_04_2022.fb.golovnjova_maryna.mariupol.1.ki_v_mar_upol.cmt
%%parent 08_04_2022.fb.golovnjova_maryna.mariupol.1.ki_v_mar_upol
 
%%url 
 
%%author_id 
%%date 
 
%%tags 
%%title 
 
%%endhead 

\qqSecCmt

\iusr{Дарья Ткаченко}

Марино, я не можу навіть на якусь частину уявити, шо пережила ти і твої рідні.
Мені дуже, дуже жаль (хоча це не те слово, немає таких слів, щоб виразити наші
почуття). По можливості, пиши про все, що сталось з вами. Це надважливо. Бережи
себе 🙏🏼

\begin{itemize} % {
\iusr{Maryna Holovnova}
\textbf{Дарья Ткаченко} дякую, що читаєш🙏
\end{itemize} % }

\iusr{Tetiana Slobodianiuk}

Інтуїція і зі мною зіграла злий жарт. 24-го лютого, коли ми прокинулись о 6-й і
зрозуміли, що почалась війна, ми поїхали під Гостомель до друзів, щоб
переночувати, зібратись з думками і вирішити, що робити завтра. У них приватний
будинок у селищі посеред лісу, здавалось, що там безпечніше, ніж у Києві.. Але
вже за дві години нам разом з цими друзями довелося швидко їхати далі, тому що
ми зрозуміли, що не втікли, а, навпаки, приблизились до епіцентру подій. На
щастя, ми ще встигли. Зараз дивлюсь на ці вулиці Ірпеня, Бучі, Гостомеля, якими
ми часто гуляли, на цю дорогу, якою виїзжали, на ці чорні пакети… хочеться
ревіти від люті. За Маріуполь я взагалі промовчу, тому що немає таких слів,
якими я б змогла передати, що відчуваю 😭 Я дуже щаслива за тебе і твоїх
рідних, що змогли виїхати з цього пекла. Бережіть себе. Ми переможемо, але
ціна, яку ми за цю перемогу заплатимо, надто висока.

\begin{itemize} % {
\iusr{Maryna Holovnova}
\textbf{Tatyana Slobodyanyuk} у перші дні було неможливо вірно оцінити, де буде безпечніше і куди втікати. Моторошно від того, як близько ви були до небезпеки, і велика удача, що зуміли вчасно виїхати🙏
\end{itemize} % }

\iusr{Alina Komarova}

Де \href{https://www.facebook.com/margarita.sherbak}{Маргарита Щербак} з донечкою?!

\begin{itemize} % {
\iusr{Andrei Marusov}
\textbf{Аліна Комарова} точно - ничего о ней не слышал 🙁

\iusr{Maryna Holovnova}
\textbf{Alina Komarova} вона була у списку тих, хто отримував гуманітарку 4 квітня.... більше нічого.

\iusr{Andrei Marusov}
\textbf{Maryna Holovnova} да, сейчас посмотрел - Щербак Маргарита Александровна, 4 апреля... если она Александровна... А дочку как зовут?

\iusr{Maryna Holovnova}
\textbf{Andrei Marusov} да, Александровна. Дочка Мия. Еще мама Маргариты, но ее имя не помню, к сожалению 🙁

\iusr{Andrei Marusov}
\textbf{Maryna Holovnova} не, Мию не вижу, только взрослых регистрируют, видать

\iusr{Maryna Holovnova}
\textbf{Andrei Marusov} да, скорее всего.
\end{itemize} % }


\iusr{Іван Станіславський}

Нам сказали що все буде добре і ми повірили...

\begin{itemize} % {
\iusr{Maryna Holovnova}
\textbf{Іван Станіславський} саме так.... 😔
\end{itemize} % }

\iusr{Valentina Bero}

Марина, очень страшно читать, как быстро вы остались без возможности
выехать!...(((

Про этот поезд.. я так поняла, что железная дорога уже не могла работать,
правильно?

Потом уже не было эвакуации даже автобусами?

Читать страшно, но очень важно, что Ты это описываешь! Спасибо, пиши, обнимаю.

\begin{itemize} % {
\iusr{Maryna Holovnova}
\textbf{Valentina Bero} 

щодо залізниці - все так, розбомбивши дорогу у Волновасі, росіяни перекрили
єдиний залізничний виїзд з міста.

Ніякої офіційної евакуації автобусами не було. Були приватні ініціативи, про
які мало хто знав, і ті перестали вивозити людей уже 27 лютого. Крім того, з
перших днів звідкісь пішла інформація (як на мене - диверсія) про те, що
автодороги перекриті і сильно небезпечні. Насправді тоді ще можна було виїхати
і нормально дістатися Запоріжжя годин за 5, а не за кілька діб. Але багато з
тих, хто мав свій транспорт, вирішили не ризикували і перечекати.

Уже в перших числах березня, коли Маріуполь був давно оточений, міська рада
підготувала аж 22 автобуси (на місто з населенням в 500 тисяч) на випадок
узгодження зеленого коридору, але ними так ніхто ніколи і не виїхав.

Дякую, що читаєш!🙏

\iusr{Valentina Bero}

А чим виїхала Ти? Чим виїхали різні наші друзі з Маріуполя? Був період, коли
почали публікувати, що повиїзжали.. Був зелений коридор?

Вибач, я не встигаю слідкувати за новинами, насправді місяць можна сказати не
заходила в фб, постійно писала на меседжері, а тепер зрідка заходжу по справам
і якось попадаю на твої чудові пости, пиши, будь ласочка, ще.

\iusr{Maryna Holovnova}
\textbf{Valentina Bero} 

нам з батьками пощастило, ми зустріли на Драмтеатрі чоловіка, який був сам у
машині і він вивіз нас і ще одну дівчину з дочкою. Якби не він, то виходили би
пішки, як робили наші друзі. Зеленого коридора всі чекали щодня, із Запоріжжя в
Марік мали приїхати десятки автобусів з гуманітаркою, вони мали би забирати
людей, але росіяни не пускали їх. Щойно коридор нібито узгоджувався - росіяни
порушували режим тиші і фактично півмільйонне місто опинилося в заручниках.
15го березня відкрили коридор для приватного транспорту, але навіть при цьому
режиму тиші росіяни не дотримувалися і деякі машини на шляху до Запоріжжя
обстрілювали...

\iusr{Maryna Holovnova}

Наші друзі пішки виходили з міста, йшли близько 25 км до Мангуша, а звідти вже
можна було їхати автобусами далі.

\iusr{Valentina Bero}

йой..

Може потім запитаю як виїхали три родини наших друзів з пленера, десь в один
час..

\end{itemize} % }

\iusr{Alina Matiushenko}

Читаю як страшну книгу спогадів.... Навіть не уявляю як ти з родиною це
пережила. ( Розумію що це не все, що тобі є про що написати далі.... Добре що
кінець історіі буде щасливим, вірю 🙏🏻

\begin{itemize} % {
\iusr{Maryna Holovnova}
\textbf{Alina Matiushenko} і я вірю❤ і ще писатиму!
Дякую тобі!!!
\end{itemize} % }


\iusr{Oksana Daniluk}

Маринко, ти не лише чудовий фотограф. Чи чудовий документаліст жахливоі
реальноі сучасності 😔

\iusr{Іванова Яна}

Господи, то ви також із Лівого берега?

\begin{itemize} % {
\iusr{Maryna Holovnova}
\textbf{Яна Иванова} так так. Проспект Перемоги, біля Сонячного....

\iusr{Іванова Яна}
\textbf{Maryna Holovnova} 

я жила здебільшого у приватному секторі, депо-2. Був час, що й на Східному.
Так. Ми там звикли, що стріляють у полях та селах за кілометри. Але вранішні
пробудження перших днів повномаштабної війни не забуду. В перший же ранок
влучило у будинок в 10 хвил. від моєї хати.

Що сказати: було ясно, що дуже близько, але не вірилося.... Зараз там
жахливо... По моїй вулиці катаються танки... За все місто дуже боляче.... Центр
міста справді врятував багатьох. Читаю і дивуюся скільки знаймих людей були
поруч, біля драму! Але ми їх не бачили, бо всі ховалися.

\iusr{Maryna Holovnova}
\textbf{Яна Иванова}, 

з Драмом у мене було і таке навіть, що зустрічала там людей, яких не бачила взагалі років 10!

А про місто так... Дивлюся на фото і відео, які зараз викладають, і не можу
повірити, що все перетворили на руїни і братські могили. Боляче.

\iusr{Іванова Яна}
\textbf{Maryna Holovnova} 

здивована, що потяги були порожні... це були ті вагони, на які моя родина не
встигла сісти - просто довго розкачувалися в рішенні

\iusr{Maryna Holovnova}
\textbf{Яна Иванова} 

про потяги на Львів чула це в чергах за водою від людей, чиї родичі та знайомі
встигли виїхати. Казали, що з Маріуполя рушали напівпорожні вагони, а вже у
Запоріжжі та Дніпрі людей заходило стільки, що їхали навіть стоячи.

\end{itemize} % }

\iusr{Vasylyna Buryanyk}

Це так важливо, що ти вціліла, дуже 🙏💛

\iusr{Марійка Андрієвська}

За Маріуполь, за тих, хто лишилися, серце кров'ю обливається, важко щось
писати, добре, що ти та твої близькі живі!!! Бережіть себе!!!))

Ми обов'язково переможемо!🇺🇦
