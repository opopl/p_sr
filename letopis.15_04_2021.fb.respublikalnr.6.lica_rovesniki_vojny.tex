% vim: keymap=russian-jcukenwin
%%beginhead 
 
%%file 15_04_2021.fb.respublikalnr.6.lica_rovesniki_vojny
%%parent 15_04_2021
 
%%url https://www.facebook.com/groups/respublikalnr/permalink/796915530944240/
 
%%author 
%%author_id 
%%author_url 
 
%%tags 
%%title 
 
%%endhead 
\subsection{Газета \enquote{Республика} (№15, 2021г).  Лица. Ровесники войны}
\label{sec:15_04_2021.fb.respublikalnr.6.lica_rovesniki_vojny}
\Purl{https://www.facebook.com/groups/respublikalnr/permalink/796915530944240/}

\ifcmt
  pic https://scontent-mxp1-2.xx.fbcdn.net/v/t1.6435-9/174174458_122682686578506_570935665836843885_n.jpg?_nc_cat=104&ccb=1-3&_nc_sid=b9115d&_nc_ohc=GyLshjRLO-4AX9ANf1U&_nc_ht=scontent-mxp1-2.xx&oh=34e52f0879c69970f9495c206f4e26db&oe=609F5C70

	pic https://scontent-mxp1-2.xx.fbcdn.net/v/t1.6435-9/174593219_122682713245170_40692783897029773_n.jpg?_nc_cat=102&ccb=1-3&_nc_sid=b9115d&_nc_ohc=9KVP6PLpynUAX-LvXlq&_nc_ht=scontent-mxp1-2.xx&oh=cf0a7ea5d0d31f267a0a6db1e84f5f0e&oe=60A00CB1

	pic https://scontent-mxp1-2.xx.fbcdn.net/v/t1.6435-9/174614144_122682676578507_7499383644820899953_n.jpg?_nc_cat=105&ccb=1-3&_nc_sid=b9115d&_nc_ohc=HirAMrwrjLIAX8Pfug4&_nc_ht=scontent-mxp1-2.xx&oh=f62e918fe69f63a0fbd70ede76aa9a63&oe=609CB4EC
\fi

Семь лет назад, 14 апреля 2014 года, вступил в силу указ самозваного президента
Украины Александра Турчинова о начале так называемой антитеррористической
операции, а по факту войны против жителей непокорившегося Донбасса. Миллионы
людей были шокированы, дезориентированы, растерянны от этой новости. Но были
семьи, которые словно и не заметили ее. Их жизни были озарены появлением новой
жизни.

Редакция нашего издания разыскала детей, родившихся семь лет назад – в день объявления войны Донбассу. И сегодня мы расскажем о них.

– Она дает мне силы идти вперед, – говорит жительница Родаково Юлия Славская, мама Киры – ребенка безудержной энергии, которая используется для ежедневного исследования окружающего мира.
– Валерьевна, – зовет Юлия дочку, – встречай гостей!

Девочка вприпрыжку вбежала в комнату, а через пару минут уже вовсю исследовала
до того неизвестное ей чудо техники – вспышку на фотоаппарате нашего
фотокорреспондента. Затем Кира с энтузиазмом принялась разрисовывать в яркие
цвета единорогов в раскраске.

– Мой любимый цвет розовый, – прощебетала девочка, – а еще желтый и сиреневый.
А недавно мы были цирке. И там клоун спросил, кто хочет ему помочь? И я подняла
руку и вышла. А еще там один клоун подшутил над другим. И питона давали
погладить!

Как и любой ребенок, Кира своими маленькими ручками готова объять весь огромный
мир. Сегодня у нее свои переживания, которыми она делится с мамой, ведь 1
сентября ей предстоит начать путешествие в школьный мир.

Поселок городского типа Родаково Славяносербского района известен крупным
железнодорожным узлом, за что и стал объектом нападений украинских боевиков.
Первый массированный обстрел пришелся на здание школы. 30 августа 2014 года
ракеты реактивных систем залпового огня с кассетными боеприпасами ковром
накрыли здание учебного учреждения. Во время обстрела попавших снарядов в
здании школы случился пожар. Сгорело целое крыло здания, были выбиты стекла. В
обугленном, покрытом копотью здании отдельные возгорания тушили несколько дней.
Придя на следующий день на работу, сотрудники школы обомлели, а из глаз хлынули
слезы. Но взяв себя в руки, начали стирать войну с лица школы, из сердца, из
душ. Всего через месяц школа была готова принять ребят. А на ее стенах
родители, словно назло войне, нарисовали сказочный лес, цветы и бабочек.

– До сих пор, хоть и семь лет прошло, осознание того, что в день, когда родилась дочка, Украина объявила нам войну, тяготит душу, – делится Юлия Славская. – Сначала такое счастье – появление ребенка… А потом было очень тяжело. Но все же хорошо, что она родилась именно тогда. Кира давала надежду на жизнь, силы не останавливаться, прорываться, не опускаться. Ради нее жили и живем.
Родители предприняли две попытки спасти малышку от войны. Первая была летом 2014 года.
– Кирин папа работал на луганской шахте, но ее бомбили, и работа остановилась, – вспоминает наша собеседница. – Решили уезжать, потому что бои подбирались все ближе и ближе. Страшно было, но больше не за себя, а за ребенка.
Первым пунктом назначения стало небольшое село на Украине, где жили родственники Юлии. Но отсутствие работы вынудило ехать дальше. В Артемовске, оккупированном вооруженными формированиями Украины городе Донецкой Народной Республики, молодая семья тоже не нашла себе места – мужчину с родаковской пропиской нигде не брали на работу. Плюс постоянные переживания из-за негативного отношения оккупантов к местным жителям и гостям города.
– С младенцем бывало и по трое суток в поезде. Было очень тяжело, и мы решили вернуться домой.
__________
Людоедскую политику Турчинова эстафетой перенял Порошенко, слова которого, сказанные всего за пару месяцев до Дебальцевского котла, болезненными рубцами остались в душах каждого жителя Донбасса:
– У нас работа будет, у них – нет. У нас пенсии будут, у них – нет. У нас поддержка под пенсионеров и детей будет, у них – нет. У нас дети пойдут в школу и в детские сады, а у них они будут сидеть в подвалах. Потому что они ничего не умеют делать. Так, и именно так мы выиграем эту войну.
Прихвостни Порошенко щепетильно воплощали в жизнь его угрозы. Да, дети сидели в подвалах. Да и сейчас им приходится туда спускаться. Но в школы и детские садики наши дети ходят, моделируют и изобретают, гоняют мяч и играют в дочки-матери! В одном Порошенко ошибся – мы умеем все. И самое главное – умеем преодолевать трудности.
__________
Вторая попытка спасения от войны уже 10-месячной Киры была предпринята после шквального обстрела железнодорожной станции Родаково 10 февраля 2015 года. Но пробыв всего месяц вдали от дома, семья вернулась.
– Возможно, мы и расслабились, но тревожный чемоданчик у двери у нас не стоит, – сегодня уже с улыбкой говорит Юлия Славская. – Все же живем с верой, что все будет хорошо.
__________
Украину долго готовили к переориентации в сторону Запада. Оголтелая русофобия, подмена нравственных ориентиров и моральной оценки происходящих событий настоящего сделали свое дело – герои стали злодеями, а абсолютное зло героизируется. Те, кто бросали молодогвардейцев в шурфы шахт, стали героями. Имена людей, которые в XXI веке бомбят мирное население, пишут на мемориальных табличках в украинских школах. Всю жизнь украинцев перевернули – добро стало злом, а то, что ранее считалось злом, представляется как благо. Черное стало белым, а белое – черным. Дальше – хуже. Пресмыкаясь перед Западом, украинское правительство навязывает гражданам страны чуждые славянским ценности. Что было неприемлемо для русского Донбасса.
__________
С Кирой Громик и ее папой Андреем мы встретились в парке города Артемовска, что под Перевальском. Остановились возле детской площадки, где мимо нас прошла стайка старшеклассников во главе с учителем, что-то живо обсуждающих. Почему-то именно в этот момент промелькнула мысль – ведь по всей нашей Республике одногодки Киры не знают жизни без войны, а старшие дети уже с трудом могут вспомнить, как это – жить и не бояться.
В этом году семье Громик исполняется 12 лет. Одиннадцать лет назад у воспитательницы детского сада Натальи и сварщика, а сегодня главного инженера Алчевского межрайонного эксплуатационного газового хозяйства Андрея родился сын Данил. Но молодые родители решили сразу, что без дочки их семья будет неполной. И вот 14 апреля 2014 года они подарили миру маленькую девочку Киру.
Кира Громик, в отличие от тезки из Родаково, излучает серьезность и сосредоточенность, и тому есть весомая причина – ведь она уже заканчивает первый класс.
– Первое УЗИ показало нам, что будет девочка. Через время, скрестив пальцы, отправились с супругой на второе, которое подтвердило, что у нас будет дочка, – вспоминает Андрей Громик. – Тут же бабушки и дедушки начали готовить подарки для внучки. 2014 год, конечно, был тяжелым, но появление Киры было радостью. Да, трудности начались скоро – это и продукты питания, которые были не в достаточном количестве, особенно это касалось детского питания, перекрытие финансирования. Но все это было ничтожно в сравнении с рождением дочки.
__________
Следующим после Порошенко эстафету ненависти принял Зеленский. Вторя предшественникам о желании мира, он продолжил уничтожать города, поселки, села. Кроме того, избрание нового украинского президента не изменило и отношения к жителям Донбасса украинских чиновников. Ярким подтверждением этому стал спич министра социальной политики Украины Андрея Ревы, который в интервью Би-би-си так охарактеризовал жителей ЛНР и ДНР: «Все, кто был проукраинский, те выехали. А те, кто хочет получать две пенсии – там и тут, пусть потерпят. Мне их не жаль абсолютно. Мне жаль тех солдат и офицеров с семьями, которые убиты там за этих мразей».
И все семь лет ознаменованы не только высказываниями деградирующих личностей, неожиданно ставших политиками, варварскими наступлениями украинской армии и прочих вооруженных бандформирований, но и их сокрушительными поражениями. Жители Луганской Народной Республики, спасая, в первую очередь, жизни детей, оставляли свои родные и уезжали в безопасные места. Кто-то уехал навсегда, а кто-то с нетерпением ждал возвращения домой. И не просто ждал, а прилагал к этому максимум усилий. Можно бесконечно долго искать причины этому, но самая объективная – это человечность, которую в нас не удалось задушить.
__________
– Летом 2014-го мы были в Крыму, а когда пришло время уезжать по обратным билетам, железная дорога прекратила работу. Пытались различными способами добраться домой. В итоге частным транспортом удалось доехать. Как раз застал момент восстановления поселков Центральный и Чернухино, где и удалось воочию увидеть масштабы трагедии, – рассказывает Андрей Громик.
 – Сейчас мы живем, трудимся, воспитываем детей. Стараемся быть оптимистами, ведь дети должны видеть только светлое будущее!
__________
За время вооруженной агрессии Украины против населения Луганской Народной Республики погибли 35 детей, младшему из которых было 28 дней, разрушения различной степени получили 113 школ и 78 детских садов. Ряд учебных учреждений по сей день находятся в зоне обстрелов украинской армии. 
__________
В то время как свежий номер нашей газеты готовился к сдаче в печать, девочки получали поздравления и подарки от самых близких людей – бабушек, дедушек и друзей. Но первыми их, конечно, поздравили родители, которые не только подарили жизнь, но и уберегли ее в войну. И теперь этой жизни, рожденной в середине весны, нестись вперед вприпрыжку. Вопреки всему!
Оксана ЧИГРИНА, 
фото Ксении ЩЕРБИНЫ, Николая СИДОРОВА, Ксении БОЖКОВОЙ
ГАЗЕТА "РЕСПУБЛИКА" (№15, 2021г).
#газета #республика #дети_войны_ЛНР #судьбы
