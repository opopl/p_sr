% vim: keymap=russian-jcukenwin
%%beginhead 
 
%%file 30_07_2021.fb.fb_group.story_kiev_ua.1.velosiped_shuljavka_i_vojna.cmt
%%parent 30_07_2021.fb.fb_group.story_kiev_ua.1.velosiped_shuljavka_i_vojna
 
%%url 
 
%%author_id 
%%date 
 
%%tags 
%%title 
 
%%endhead 
\zzSecCmt

\begin{itemize} % {
\iusr{Алла Гершкович}
Какая страшная судьба...

\iusr{Георгий Майоренко}
\textbf{Алла Гершкович} Да, это большая боль семьи. До конца войны надеялись, что он вернётся. Думали, где-то в партизанах...

\iusr{Irina Bentaleb}

\ifcmt
  ig https://i2.paste.pics/d036862164dc7b75adefa725383e69a4.png
  @width 0.1
	@name scr.eye.cry.yellow
\fi

\iusr{Елена Немерицкая}
Печально... Спасибо, что поделились снимками и за рассказанную историю.

\begin{itemize} % {
\iusr{Георгий Майоренко}
\textbf{Елена Немерицкая} Просто такие судьбы это в какой-то мере боль всех киевлян....

\iusr{Елена Немерицкая}
\textbf{Георгий Майоренко} И не только киевлян... Спасибо, Георгий, что храните память.
\end{itemize} % }

\iusr{Вахтанг Кварелашвили}

Мои прабабушка и прадедушка ( по материнской линии, естественно) жили до войны
на Шулявке, именно на Борщаговской, там где сейчас 19-й корпус КПИ.

\begin{itemize} % {
\iusr{Георгий Майоренко}
\textbf{Вахтанг Кварелашвили} Наши жили на Борщаговской 30. После того, как снесли старую Шулявку, там сейчас Борщаговская 10.

\iusr{Ирина Панченко}
\textbf{Вахтанг Кварелашвили} Мы жили на Борщаговской 56А, там построены корпуса КПИ.Наши дома снесли последними.

\begin{itemize} % {
\iusr{Георгий Майоренко}
\textbf{Ирина Панченко} О, наши дореволюционные соседи! Янковские - фамилия на Шулявке уважаемая!

\iusr{Ирина Панченко}
\textbf{Георгий Майоренко} 

Но, Янковские жили не там, сейчас, даже не представляю где, а спросить негде. А это
была квартира маминого папы Фартушного Иллариона Митрофановича и его мамы
Дарьи.

\end{itemize} % }

\end{itemize} % }

\iusr{Оксана Тернавская}

Георгий Цуканов, помощник Л. И. Брежнева до самой его смерти - это и есть брат
Жорж? Называли его «серым кардиналом». Очень интересно! Расскажите обязательно.

\begin{itemize} % {
\iusr{Георгий Майоренко}
\textbf{Оксана Тернавская} 

Да, это брат Володи - Георгий Эммануилович Цуканов. Я сейчас веду переговоры с
родственниками, чтобы не пропали его мемуары. Он не давал интервью, а внуку
своему говорил: \enquote{-Потом прочитаешь мои воспоминания.}

Георгий был старше Володи на 3 года. Накануне войны был инженером на
Металлургическом заводе в Днепродзержинске. Завод эвакуировали в Челябинск, и
Георгий руководил цехом, \enquote{ковал} победу. Вот фото Георгия Цуканова. Ему здесь
17 лет. Конечно же, сделаю о нем отдельный сюжет.

\ifcmt
  ig https://scontent-mxp2-1.xx.fbcdn.net/v/t39.30808-6/223819524_2698055967160294_8920150328446410144_n.jpg?_nc_cat=100&ccb=1-5&_nc_sid=dbeb18&_nc_ohc=UijpQIt7CSMAX9RExAy&_nc_ht=scontent-mxp2-1.xx&oh=00_AT_TyAFv9Vz91nT39yC49oTtPU-z-nqnJdZGeG2NrO7Dkw&oe=61DB20CF
  @width 0.3
\fi

\begin{itemize} % {
\iusr{Оксана Тернавская}
\textbf{Георгий Майоренко} спасибо! Ждём!

\iusr{Irena Visochan}
\textbf{Оксана Тернавская} Да, очень интересно. @igg{fbicon.hearts.two} 

\iusr{Olena Klymenko}
\textbf{Георгий Майоренко} Спасибо, ждем.

\iusr{Георгий Майоренко}
Стараемся хранить память
\end{itemize} % }

\iusr{Ирина Нищимная}
Война самое страшное,,, спасибо за воспоминания,,,,

\iusr{Валентина Ободянская}
Как больно и страшно. Какие судьбы!

\end{itemize} % }

\iusr{Valentina Okladnaya}

И как бы ни было понятно, что этот мальчик всего лишь один из тысяч с подобной
судьбой, как больно читать и осознавать страшную трагедию жестоко загубленной
жизни...

\begin{itemize} % {
\iusr{Георгий Майоренко}
\textbf{Valentina Okladnaya} Да, подобные истории были во многих семьях. Светлая память всем киевлянам, не вернувшимся с той страшной войны...
\end{itemize} % }

\iusr{Всеволод Шевчук}
Цікаво прочитати оповідь про Жоржа сподіваюсь вона не буде такою сумною?

\begin{itemize} % {
\iusr{Георгий Майоренко}
\textbf{Всеволод Шевчук} У Жоржа, кроме потери брата (они с Володей были почти ровесники), была ещё одна трагедия. У него в автокатастрофе погиб сын Михаил.

\iusr{Всеволод Шевчук}
\textbf{Георгий Майоренко} сумно, нажаль в житті бувають неприємності я в 16 років втратив старшого брата!
\end{itemize} % }

\iusr{Oksana Romanova}
Светлая память.

\iusr{Георгий Майоренко}
\textbf{Oksana Romanova} Светлая память всем не вернувшимся...

\iusr{Виктор Михайлюк}
Тихо, стоп!!! Как говорят в. Адессе и на Шулявке, это две разные улицы!!! Адесс, это одно, а Шулява - это совсем другое!!!
\end{itemize} % }
