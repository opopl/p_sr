% vim: keymap=russian-jcukenwin
%%beginhead 
 
%%file 22_12_2020.news.ua.strana.sibircev_aleksandr.1.ukraina_respublika.ludi_vne_gosudarstva
%%parent 22_12_2020.news.ua.strana.sibircev_aleksandr.1.ukraina_respublika
 
%%url 
 
%%author 
%%author_id 
%%author_url 
 
%%tags 
%%title 
 
%%endhead 
\subsubsection{\enquote{Люди вне государства}}
\label{sec:22_12_2020.news.ua.strana.sibircev_aleksandr.1.ukraina_respublika.ludi_vne_gosudarstva}

"Страна" писала о подобном движении еще в 2016 году.

Тогда единомышленники из нескольких областей Украины на основании своего
видения 5 статьи Конституции и нескольких юридических коллизий украинского
законодательства, организовала своего рода систему самоуправления и
параллельной экономики.

Так, они ссылались на статью 5 Конституции Украины ("носителем
суверенитета и единственным источником власти в Украине является народ.
Народ осуществляет власть непосредственно и через органы государственной
власти и органы местного самоуправления"). А также на статью 2 закона о
местном самоуправлении ("местное самоуправление осуществляется
территориальными громадами… как непосредственно, так и через сельские,
городские советы и их исполнительные органы"). 

"Люди вне государства" не платили налогов, выдавали разрешения на занятия
бизнесом и даже ввели аналог своих денег в виде "трудозатратных" расписок,
которые имели хождение внутри нескольких предприятий, ушедших под крыло
"громад самоуправления людей".

Тогда правоохранители предприняли масштабную атаку на активистов
самопровозглашенных органов "народных территориальных громад". В офисах
движения людей вне государства в нескольких городах произошли обыски,
документы движения изъяли.

Однако идея отрицания легитимности существующей власти и государства
никуда не делась.

Тем более что после атаки полиции и СБУ на это движение, никаких
последствий для активистов так и не наступило. По словам представителей
движения, правоохранители не нашли законных оснований для доведения до
суда дел против них. 

