% vim: keymap=russian-jcukenwin
%%beginhead 
 
%%file 24_01_2022.stz.news.ua.zbruc.1.esej_pro_rozum
%%parent 24_01_2022
 
%%url https://zbruc.eu/node/110121
 
%%author_id jabchanka_oleksandr
%%date 
 
%%tags chelovek,obschestvo,razum,ukraina,psihologia
%%title Есей про розум
 
%%endhead 
 
\subsection{Есей про розум}
\label{sec:24_01_2022.stz.news.ua.zbruc.1.esej_pro_rozum}
 
\Purl{https://zbruc.eu/node/110121}
\ifcmt
 author_begin
   author_id jabchanka_oleksandr
 author_end
\fi

\begin{zznagolos}
На фоні розмов про енергетичну кризу, з невтішними прогнозами щодо відключень
електроенергії і виправданнями влади щодо нашої залежності від енергоресурсів,
я оце спробував на ресурси поглянути трішки з іншого боку.  	
\end{zznagolos}

Минулий рік у мене великою мірою пов’язаний із наукою. Я відмовився від
наукового ступеня кандидата наук з медицини, але здобув ступінь магістра з
публічної політики і врядування. Зрештою, наукові ступені не так важливі, як
знання, тому по закінченню Київської школи економіки періодично шукаю, чого б
розумного повчитись. Благо, в Україні є освітні платформи, а щоб знання
структурувались у голові, намагаюсь підсумовувати їх написанням есе. Так, за
грудень я мав приємність побувати на чотирьох курсах: щодо історії України з
Ярославом Грицаком, щодо соціальної нерівності і сучасних політичних процесів з
Іваном Гомзою, щодо свідомості з Андрієм Баумейстером, і щодо міст і
інструментів роботи з майбутнім з Юрієм Чудновським.

\ii{24_01_2022.stz.news.ua.zbruc.1.esej_pro_rozum.pic.1}

Мотивований обнадійливими висновками курсу \enquote{Подолати минуле: глобальна
історія України}, наступне есе я вирішив присвятити майбутньому, озброївшись
знаннями, отриманими на вищеописаних курсах.

Серед трьох інструментів роботи з майбутнім (планування, проєктування і
програмування) я б зупинився на третьому. Бо програмування – це зв’язок
декількох процесів: і проєктування, і планування, та ще й дослідження з
аналітикою. Якщо спрощено, то у проєктах зв’язуються інтереси (проєкт \enquote{Україна}
зв’язує інтереси великої кількості стейкхолдерів пропорційно до їхніх
можливостей), а у програмах – процеси. Програми контекстні, вони не можуть
працювати поза контекстом. От мені і хотілось би подумати над програмою
розвитку держави Україна. У цьому есе я спробую спочатку описати процеси, котрі
відбуваються у світі і у котрі залучена держава Україна, а далі намагатимусь
подивитись на нашу країну як на спільноту мільйонів громадян, кожен з яких має
унікальну свідомість.

Отже, почнемо з процесів. На сьогодні я би виділив такі: науково-технічна
революція, глобалізація, зростання нерівності. З курсу Івана Гомзи я зробив
наступні висновки. Нині ми перебуваємо у черговій науково-технічній революції.
Наслідки таких революцій загалом для людства дають непогані результати. Нас,
людей, стало кількісно більше, а тривалості і комфорту життя сучасної людини
середнього достатку могли б позаздрити найбагатші людини минулого. Звісно,
подібні зіставлення є грубим узагальненням і є доволі умовними, хоча б тому, що
немає, та й не було якогось єдиного визначення хорошого чи поганого, як ото
тривалість чи комфорт життя. Проте сам факт зростання середньої тривалості
життя людини сучасної сумніву не підлягає, як і те, що сьогодні мені
сконтактувати з друзями у Києві займає секунду, тоді як ще сотню років тому це
би була значно складніша історія. Проте кожна така технічна революція, окрім
позитивів, має і наслідки, зокрема зростання загальної нерівності між людьми. І
біс би з нею, з тією нерівністю, мене особисто геть не напружує факт, що мої
статки незіставні зі статками якоїсь частини співгромадян. Проте, як
неодноразово показувала історія, коли нерівність стає критичною, справа
закінчується апокаліпсисом, чотири вершники якого: війна, епідемія, революція,
крах держави. Іншими словами, коли нерівність заводить суспільство у глухий
кут, виходи з цього кута, жоден нічого хорошого нам не пророкує. Причиною
глухого кута може бути \enquote{ресурсна спіраль}, коли на виробництво благ потрібні
людські руки, а людські руки хочуть їсти, чим більше рук потрібно на
виробництво, тим більше рук потрібно прогодувати. І тут би саме час радіти,
адже з новітніми технологіями є усі шанси, що за якийсь час усе людство буде
нагодоване. Проте навіть за умови перерозподілу благ таким чином, аби всі були
ситі, виникає набагато складніша проблема – проблема справедливості. Бо якщо
нерівність – категорія емпірична (вимірювана), то справедливість – категорія
моральна. Спроби забезпечити тотальну справедливість були неодноразово і
закінчувались вони не меншим апокаліпсисом, аніж критична нерівність. Зрештою,
поняття нерівності і несправедливості, як уже зазначалось вище, то не одне і те
ж, проте ідуть вони по історії поруч. Не хотілось би вірити, що і цього разу
нас очікує щось подібне, бо, як каже історик Ярослав Грицак, \enquote{історія – не
в’язниця}, від себе ж додам, що \enquote{історія – не в’язниця, але нерідко глухий
кут}. Отож, вартувало би пошукати вихід з цього глухого кута. Нерівність і
супроводжуюча її несправедливість стосувались якогось скінченного ресурсу (їжі,
грошей), але на бесіді з Юрієм Чудновським я почув цікаву тезу, що ресурси
нескінченні. До прикладу, ще яких 150 років тому уявити собі, що частоти будуть
ресурсом, було неможливо, а 500 років тому складно було собі уявити, що базовим
ресурсом людства стане нафта. До використання нафти, чи частот, чи ядерної
енергії як ресурсу, людство дійшло своїм розумом. Тобто наш розум – то
нескінченний ресурс, котрий нескінченно може придумувати ресурси з того, що
раніше ресурсом не було. То, може, спробувати детальніше придивитись до цього
ресурсу, бо якщо він нескінченний, то як можна розділити нескінченність
несправедливо, а головне, навіщо – нескінченності ж на всіх вистачить.
 
Отже, маємо в Україні мільйони нескінченних ресурсів, котрі кожен із нас носить
на своїх плечах. Далі спробую пошукати шляхи, як ті ресурси примножувати, бо
сьогодні, за моєю теорією, ми нескінченно багаті, а практично маємо купу
ресурсних проблем.
