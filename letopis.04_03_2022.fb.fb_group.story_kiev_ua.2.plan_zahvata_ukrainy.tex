% vim: keymap=russian-jcukenwin
%%beginhead 
 
%%file 04_03_2022.fb.fb_group.story_kiev_ua.2.plan_zahvata_ukrainy
%%parent 04_03_2022
 
%%url https://www.facebook.com/groups/story.kiev.ua/posts/1874114199452039
 
%%author_id fb_group.story_kiev_ua,atojev_konstantin.kiev
%%date 
 
%%tags 
%%title ВВЕРХ ПО ЛЕСТНИЦЕ, ВЕДУЩЕЙ В АД: КАК РОЖДАЛСЯ ПЛАН ЗАХВАТА УКРАИНЫ
 
%%endhead 
 
\subsection{ВВЕРХ ПО ЛЕСТНИЦЕ, ВЕДУЩЕЙ В АД: КАК РОЖДАЛСЯ ПЛАН ЗАХВАТА УКРАИНЫ}
\label{sec:04_03_2022.fb.fb_group.story_kiev_ua.2.plan_zahvata_ukrainy}
 
\Purl{https://www.facebook.com/groups/story.kiev.ua/posts/1874114199452039}
\ifcmt
 author_begin
   author_id fb_group.story_kiev_ua,atojev_konstantin.kiev
 author_end
\fi

ВВЕРХ ПО ЛЕСТНИЦЕ, ВЕДУЩЕЙ В АД: КАК РОЖДАЛСЯ ПЛАН ЗАХВАТА УКРАИНЫ.

Украина призывает НАТО закрыть небо, чтобы от российских обстрелов не гибли
мирные граждане. Но альянс пока на это не идет, опасаясь вступить в прямое
столкновение с Путиным. Эти страхи понятны, но закрыть небо могли бы
пилоты-легионеры из частных военных компаний на самолетах, переданных нам НАТО.
Ведь если падет Украина, следующими на очереди будет Польша и страны Балтии, и
НАТО все равно придется воевать с РФ, еще более закусившей удила после расправы
над Украиной. Будем надеяться, что понимание этого придет к нашим партнерам -
все-таки не бином Ньютона. РФ технологически все более отставая и превращаясь в
сырьевой придаток ведущих экономик мира, только в войнах видит свой \enquote{особый
путь}. И не стоит Западу себя тешить, что речь идет лишь о воссоздании СССР.
Россия замахивается на мировое господство, основанное на ядерном шантаже.  Еще
в далеком 2006 г. была издана книга Михаила Юрьева \enquote{Третья империя. Россия,
которая должна быть}, которая во многом задала траекторию дальнейшего развития
России. Ассоциации с \enquote{Третьим рейхом} тут не случайны. Юрьев  – вице-спикер
Думы от «Яблока» (1996-2000), был сторонником изоляционизма, именно в нем он
видел причину расцвета промышленности довоенной Германии. Он стал тем самым
Франкенштейном, загрузившим в голову Путина иллюзорные геополитические
устремления.  Из созданной им утопии обитатели Кремля до сих пор черпают многие
свои смысловые конструкции.

\ii{04_03_2022.fb.fb_group.story_kiev_ua.2.plan_zahvata_ukrainy.pic.1}

Действие происходит в 2054 году. К этому времени Россия оккупирует Европу и
захватывает США, одержав победу в ядерной войне. Это можно было бы назвать
горячечным бредом - только все, что написано в книге о конфликте России и
Украины, было принято Кремлем как руководство к действию почти на 100\%. Юрьев
писал о  Донецко-Черноморской республике в составе Новороссии. Якобы на
Юго-востоке Украины вспыхивает восстание против вступления в НАТО и
евроинтеграции. За присоединение к РФ голосует 82\%. Кремль вводит войска, Запад
«сливает» Украину, так как Путин готов идти до конца и не поддается на
давление. Он сам вводит санкции: отключает Европе газ, выходит из ООН. В РФ
вводят самодержавие, так как, только ни от кого не зависящий лидер может
защитить Россию. Упомянута и гибридная война, когда люди гибнут, но торговля
идет полным ходом, а границы по-прежнему открыты.

Как же выглядит «русский мир» после победы над США? Это нечто! Бред «пацана» из
подворотни, мечтающего рассчитаться с взрослым дядькой, давшим ему подзатыльник
за издевательства над котенком. Особенно много об инфантильных мечтаниях
кремлевских «подростков» говорит сцена Парада победы над Америкой. В отличие от
шествия немецких военнопленных в Москве по Красной площади в наручниках, с
табличками на шее проходит вся элита США: президенты, члены Конгресса и сената,
промышленники, судьи, адвокаты, финансисты, звезды Голливуда, представители
шоу-бизнеса и СМИ. Так миру демонстрируют, что Россия победила американскую
цивилизацию. Затем начинается захват Европы - она присоединяется к РФ. Империя
уничтожает более 600 тысяч поляков и украинцев, из них более двух третей
представляет гражданское население. Запрещаются языки стран Балтии, так же, как
и сами слова «Латвия», «Литва» и «Эстония». Литовцы сохраняют право жить на
своей территории, хотя все имущество у них отбирается. Латыши и эстонцы
выселяются из Прибалтики. Русскоязычные жители этих стран на пять лет
поражаются в правах: «нечего было терпеть унижения прибалтов».

Эта книга, конечно, содержит элементы запугивания соседей. В ней умело спрятаны
от глаз наиболее уязвимые точки путинской России. Однако одну из них скрыть не
удалось. На поверхность проступает вся ущербность кремлевских политиков и
убогость их устремлений – провести по Красной Площади в наручниках, поставить
на колени, запретить языки, лишить имущества, вывезти трофеи. Разве могли такие
люди вести себя иначе на заседании Совета безопасности РФ, с трясущимися
руками, заиканиями, полной прострацией после окрика их властителя. Разве могут
они как-то иначе поступать с украинскими городами, которые теперь подвергаются
обстрелам, нарастающим день ото дня, не смотря на гибель мирного населения.
Итак, еще в 2006 году элита России считала своими смертельными врагами
Прибалтику, Польшу и Украину, разрабатывала планы изменения мирового порядка и
не скрывала этого. Так что мысль о том, что агрессия России является
импульсивной реакцией сошедшего с ума старика, вряд ли соответствует
действительности. РФ внедряла своих агентов в высшие эшелоны власти, разлагала
внутреннюю структуру Украины и Европы, раскручивала истерию с помощью своей
пятой колонны, подтягивала войска к границе Украины, готовя нападение. Она
никогда не скрывала свои планы, начиная с Мюнхенской речи Путина 2007 г.,
просто тогда Запад, не дав вовремя отпор, сыграл роль незадачливого профессора,
выпустившего злой дух из бутылки.

Почему же описанное в книге русское «завтра» не наступит никогда? По причине
убогости его строителей. В том, что касается подчинения СМИ, разжигания
низменных инстинктов толпы, создания сословного самодержавного государства они,
конечно, преуспели. Но вот с созданием конкурентоспособной экономики, новых
высокотехнологических производств, передовой науки, эффективного управления,
как-то не сложилось. Да и коррупция при самодержце всея Руси достигла не
виданных размеров. И все это ей теперь аукнется – без инвестиций, без
современных технологий, с эмбарго на экспорт энергоносителей, с коллапсом
финансово-экономической системы, с массовой безработицей, так как крупнейшие
иностранные компании, дававшие миллионы рабочих мест уходят из РФ, и многим
таким, что и не снилось кремлевским «мудрецам».
