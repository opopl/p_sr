% vim: keymap=russian-jcukenwin
%%beginhead 
 
%%file 19_08_2021.fb.krutojarov_aleksandr.kiev.1.esche_ne_vse_poterjano
%%parent 19_08_2021
 
%%url https://www.facebook.com/permalink.php?story_fbid=1241187726319641&id=100012852751261
 
%%author_id krutojarov_aleksandr.kiev
%%date 
 
%%tags chelovek,dusha,psihologia
%%title Еще не всё потеряно...
 
%%endhead 
 
\subsection{Еще не всё потеряно...}
\label{sec:19_08_2021.fb.krutojarov_aleksandr.kiev.1.esche_ne_vse_poterjano}
 
\Purl{https://www.facebook.com/permalink.php?story_fbid=1241187726319641&id=100012852751261}
\ifcmt
 author_begin
   author_id krutojarov_aleksandr.kiev
 author_end
\fi

Еще не всё потеряно...

Наша главная проблема заключается в том, что значительная часть нынешней элиты
охвачена сетью культа \enquote{золотого тельца} (личного обогащения), которая сцепляет,
замыкает и определяет все её ценности. И по этой причине она не только не имеет
собственного мнения по тем или иным событиям, затрагивающим интересы нашего
общества, но и пытается навязать нам в качестве нормы поведения нравы "дикого
Запада", делая тем самым каждого из нас заложником этой полюбившейся в верхах
формы взаимоотношений.  

В результате, многие из граждан не знают, как правильно поступать в подобной
ситуации, хотя интуитивно понимают, что помещены в систему, запрограммированную
и заполненную посылами, которые изначально не имеют к ним отношения. Но самое
печальное, что с этим ничего не поделаешь – само по себе это не исчезнет, так
как противодействие желанию вырваться из этих тисков тоже заложено внутри
системы. А это значит, что, во-первых, в наших людях скопились огромные
напластования чуждых им мыслей и желаний, а во-вторых, они сами от них
избавиться не в состоянии. Что, в свою очередь, наводит на недостаточно
оптимистические размышления.

И все же... 

И все же не все потеряно. Живущая во мне «обезьянка», способна принять иную
форму – форму Человека, существующего внутри каждого из нас. Для этого нужно,
чтобы во мне проступило намерение к построению собственными усилиями особого
окружения, благодаря чему я смогу уже в этом мире выстраивать в себе частицу
единой, общей души. И таким образом, создавать внутри себя некий «виртуальный
объем», в котором буду свободно развиваться – если, конечно, полностью реализую
свою свободу воли.

Причем это не помешает мне подчиняться внешнему воздействию и дальше, но
несколько иначе - теперь я подчиняюсь уже сам, по желанию, по собственным
решениям, склоняясь перед своим новым окружением с его особенными ценностями. В
остальном же у меня всё остается как и прежде -  нет ничего своего.

Если же говорить в целом, то всё это (обобщая сказанное выше) означает
следующее. С одной стороны, подчиняясь воздействию внешнего окружения, я
нахожусь в своеобразном «изгнании». И наоборот, ставя себя под воздействие
нового окружения (выбранного мной), я стараюсь выйти из изгнания и в итоге
освобождаюсь от внешнего влияния, эгоистического по своей сути. 

То есть, «изгнание» здесь означает мое пребывание под давлением внешнего,
широкого мира и усвоение внешних нравов, внешних ценностей, подменяющих помощь
и поддержку других, как самого себя. Стараясь же выйти из «изгнания», мы
выстраиваем такое окружение, которое будет проникнуто ценностями милосердия и
поручительства друг перед другом. И потому конечный результат зависит только от
меня, от выбора конкретного средства – окружения, которое ставит единство
превыше всего.
