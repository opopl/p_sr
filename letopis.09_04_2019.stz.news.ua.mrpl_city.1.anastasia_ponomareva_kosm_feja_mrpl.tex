% vim: keymap=russian-jcukenwin
%%beginhead 
 
%%file 09_04_2019.stz.news.ua.mrpl_city.1.anastasia_ponomareva_kosm_feja_mrpl
%%parent 09_04_2019
 
%%url https://mrpl.city/blogs/view/anastasiya-ponomarovakosmichna-feya-mariupolya
 
%%author_id demidko_olga.mariupol,news.ua.mrpl_city
%%date 
 
%%tags 
%%title Анастасія Пономарьова – космічна фея Маріуполя
 
%%endhead 
 
\subsection{Анастасія Пономарьова – космічна фея Маріуполя}
\label{sec:09_04_2019.stz.news.ua.mrpl_city.1.anastasia_ponomareva_kosm_feja_mrpl}
 
\Purl{https://mrpl.city/blogs/view/anastasiya-ponomarovakosmichna-feya-mariupolya}
\ifcmt
 author_begin
   author_id demidko_olga.mariupol,news.ua.mrpl_city
 author_end
\fi

\ii{09_04_2019.stz.news.ua.mrpl_city.1.anastasia_ponomareva_kosm_feja_mrpl.pic.1}

Дорогі читачі, сподіваюся, що ви готові продовжити відкривати для себе яскраві
постаті Маріуполя, які завдяки нестандартному мисленню і креативному підходу
створюють нові місця дозвілля та своєю енергійною діяльністю і щирому серцю
надихають маріупольців на нові звершення. Саме такою яскравою і самобутньою є
наша наступна героїня – художниця, письменниця і громадська діячка \textbf{Анастасія
Пономарьова}. Вона має свій унікальний світогляд і є справжнім генератором ідей.
За її ініціативи було започатковано ряд важливих проектів. Маленьким
маріупольцям вона нагадує справжню фею, яка з легкістю перетворює будні на
свята, а для дорослих її фонтануючі ідеї та швидкість їхнього втілення в життя
є проявом чогось космічного і неординарного.

\ii{insert.read_also.demidko.rudenko}

\ii{09_04_2019.stz.news.ua.mrpl_city.1.anastasia_ponomareva_kosm_feja_mrpl.pic.2}

Народилася Анастасія в Маріуполі у великій і дружній родині. Вона завжди мала
особливу любов до сім'ї та свого будинку. Художниця підкреслює, що все в її
домі завжди створювалося з особливим почуттям любові. Згадує, як її прадід
власними руками з тим же почуттям побудував їхню оселю. Зараз вона мешкає тут з
коханим чоловіком та сином. Сім'я завжди в усьому підтримує Настю, їй ніколи не
казали, що вона щось не може створити чи зробити, навпаки, завжди говорили:
\emph{\enquote{Спробуй}}. Тому вона відчуває у всьому підтримку батьків, чоловіка, сина.
Анастасія підкреслює, що, якби не її син, багатьох дитячих проектів могло б не
бути. У Маріуполі найбільше любить море. Однак вона наголошує, що \emph{\enquote{любов до
міста це не любов до стін, це люди. Емоції, почуття, які ти тут переживав}}. Її
надихає життя, вона з легкістю знаходить щось гарне і особливе у звичних речах.

\ii{09_04_2019.stz.news.ua.mrpl_city.1.anastasia_ponomareva_kosm_feja_mrpl.pic.3}

Настя отримала вищу педагогічну освіту і сьогодні працює за спеціальністю в
школі НВК № 69 вчителем географії. Але водночас їй вдається поєднувати свою
роботу з активною громадською діяльністю. Вона вже давно реалізувала себе як
унікальний маріупольський ілюстратор, проектний менеджер і керівник освітнього
центру \enquote{Місто}. Анастасія займається тим, що їй подобається, що вона всім
серцем любить. Вона намагається робити все якісно, але зізнається, що інколи
може й помилятися, тому не припиняє вчитися і працювати над самовдосконаленням.

\textbf{Читайте також:} \emph{Мариуполь присоединился к борьбе всего человечества с \enquote{пластиковой оболочкой} Земли}%
\footnote{Мариуполь присоединился к борьбе всего человечества с \enquote{пластиковой оболочкой} Земли, Анна Хижникова, %
mrpl.city, 01.04.2019, \par%
\url{https://mrpl.city/blogs/view/mariupol-prisoedinilsya-k-borbe-vsego-chelovechestva-s-plastikovoj-obolochkoj-zemli} }

Наша героїня є співзасновницею ІГ \enquote{Місто щасливих людей} та освітнього центру \enquote{Місто}.

\ii{09_04_2019.stz.news.ua.mrpl_city.1.anastasia_ponomareva_kosm_feja_mrpl.pic.4}

\textbf{\enquote{Місто щасливих людей}} почалось, коли вона разом з \textbf{Олександрою Ваксман} вирішила
почитати дітям, а закріпилося малюнком щасливої дитини на флешмобі \textbf{\enquote{Маріуполь
чекає на океан}}. Саме тоді вони й почали свій довгий і яскравий творчий шлях.

Одним з найбільш унікальних для маріупольських дітлахів став проект \textbf{\enquote{Маріуполь
читає дітям}}. Головна його мета – зробити серед діточок міста українську мову
цікавою та важливою. Організатори проекту читали маленьким маріупольцям відомі
оповідання, книжки, казки українською, при цьому перевтілювалися в образи
героїв. Кожне читання відбувалося на новому місці, яке було пов'язане з
тематикою обраного оповідання. Крім маріупольських громадських діячів, читали й
відомі українські поети, художники, співаки. Серед них: Сергій Жадан, Вікторія
Ковальчук, В'ячеслав Купрієнко. Кожна дитина отримувала значок \enquote{Я люблю
читати}, щоденник читача, багато смаколиків та незабутніх вражень.

\ii{09_04_2019.stz.news.ua.mrpl_city.1.anastasia_ponomareva_kosm_feja_mrpl.pic.5}

\textbf{Читайте також:} \emph{Мариупольчанка Даша Романова может отправляться на лечение в Италию}%
\footnote{Мариупольчанка Даша Романова может отправляться на лечение в Италию, Олена Онєгіна, mrpl.city, 08.04.2019, \par%
\url{https://mrpl.city/news/view/mariupolchanka-dasha-romanova-mozhet-otpravlyatsya-na-lechenie-v-italiyu}
}

Наступним проектом став невеличкий і перший фестиваль, організований Анастасією
та Олександрою Ваксман \textbf{\enquote{Арт і Шок}}, на якому вони презентували історичну
розмальовку з видами Маріуполя. Загалом розмальовки Анастасії \textbf{\enquote{Розфарбуй
місто}}, в яких маріупольська архітектура представлена в антистресовому
варіанті, посідають своє особливе місце. Головна мета видання – зацікавити
маріупольців історією, архітектурою рідного міста, виховати любов та повагу до
архітектурної спадщини Маріуполя. У результаті юні маріупольці, розмальовуючи
архітектуру власного міста, дізнавалися про неї більше, починали розуміти, що і
в Маріуполі є чим пишатися.

\ii{09_04_2019.stz.news.ua.mrpl_city.1.anastasia_ponomareva_kosm_feja_mrpl.pic.6}

Щоб допомогти дітлахам подолати свої страхи, бути дружелюбними та відкритими до
навколишнього світу, Анастасія видала книгу \textbf{\enquote{Руді історії}}. Авторка вважає, що
завдяки книзі діти можуть навчитися пізнавати світ і паралельно розуміти
чуттєву сторону цього світу.

Енергії Насті можна лише дивуватися, адже її проекти продовжували з'являтися
один за одним. Незабаром була створена \enquote{Наукова не-школа}. Організатори проекту
завдяки власній креативності навчали дітей цікавій науці. Вони дивилися у
телескоп на плями на Сонці, з поліцейськими вивчали дактилоскопію, влаштовували
батли вчених. Юні науковці навіть пройшлися по червоній доріжці, коли їм
вручали сертифікати науковців.

\textbf{Читайте також:} \emph{В Мариуполе классик фотографии представил уникальные работы на своей первой выставке}%
\footnote{В Мариуполе классик фотографии представил уникальные работы на своей первой выставке, mrpl.city, 05.04.2019, \par%
\url{https://mrpl.city/news/view/v-mariupole-klassik-fotografii-predstavil-unikalnye-raboty-na-svoej-pervoj-vystavke-foto}
}

Також громадська діячка стала одним з організаторів дитячих локацій на
фестивалях \textbf{\enquote{З країни в Україну}} та \href{https://archive.org/details/18_04_2018.roksana_kuranova.mrpl_city.knizhna_toloka_vernutsja_2019}{\enquote{Маріупольська книжкова толока}}.\footnote{%
Почему \enquote{Мариупольская книжная толока} должна вернуться в 2019 году, и как один фестиваль может объединить всю Украину, %
Роксана Куранова, mrpl.city, 18.04.2018, \par%
\url{https://mrpl.city/blogs/view/pochemu-mariupolskaya-knizhnaya-toloka-dolzhna-vernutsya-v-2019-godu-i-kak-odin-festival-mozhet-obedinit-vsyu-ukrainu}, \par%
Internet Archive: \url{https://archive.org/details/18_04_2018.roksana_kuranova.mrpl_city.knizhna_toloka_vernutsja_2019}
} Провела
декілька заходів з поліцейськими, зокрема, в Маріуполі з'явилися \enquote{мафія з
копами} та \enquote{кава з поліцейським}. Ще встигла випустити капсульну колекцію
суконь ручної роботи.

\ii{09_04_2019.stz.news.ua.mrpl_city.1.anastasia_ponomareva_kosm_feja_mrpl.pic.7}

Освітній центр \textbf{\enquote{Місто}} починався з вінницького активіста \textbf{Івана Шихат-Саркісова},
який допоміг зібрати кошти по всій Україні для відкриття робоклубу для
маріупольських дітей. Приблизно рік Анастасія і Олександра Ваксман абсолютно
безкоштовно вчили дітей робототехніці на основі LEGO Education. Базувалися вони
у вільному просторі \enquote{Халабуда}. Та, на жаль, тоді активно розвивати цю корисну
ініціативу вони не змогли. Коли ж захотіли відправити дітей на фестиваль
робототехніки у Києві, їх підтримала компанія TRINITY, разом з якою на базі
їхнього офісу відкрили освітній центр \enquote{Місто}. Знайшли однодумців серед
представників бізнесу і разом впровадили доступну технічну освіту для дітей.

\ii{09_04_2019.stz.news.ua.mrpl_city.1.anastasia_ponomareva_kosm_feja_mrpl.pic.8}

Наразі у робоклубі вже постійно діють курси з робототехніки на основі LEGO
Education. Нещодавно освітній центр став представником мережі cod clab –
ініціативи, яка дозволяє вчити дітей програмуванню і давати їм IT-освіту за
програмами Массачусетського університету абсолютно безкоштовно. Головна мета
організаторів, щоб за цією програмою працювали школи міста. Найближчим часом
Анастасія з партнерами планує відкривати курси з ЗD-моделювання, урбаністики,
наукової школи та електроніки. Вона підкреслює, що учасники центру регулярно
їздять на фестивалі та олімпіади з робототехніки по всій Україні.

\textbf{Читайте також:} \emph{Mariupol FM. Нова українська музика: ВІДВЕРТО}%
\footnote{Mariupol FM. Нова українська музика: ВІДВЕРТО, Маріуполь FM, mrpl.city, 03.04.2019, \par%
\url{https://mrpl.city/blogs/view/mariupol-fm-nova-ukrainska-muzika-vidverto}
}

\ii{09_04_2019.stz.news.ua.mrpl_city.1.anastasia_ponomareva_kosm_feja_mrpl.pic.9}

Ідеї Анастасії підтримувало і підтримує дуже багато громадських організацій,
ініціатив, активістів, волонтерів і просто небайдужих людей. Вона безмежно
вдячна всім за час, сили, гроші на те, щоб зробити своє місто кращим. Саме
завдяки таким людям вона любить Маріуполь і вірить у його світле майбутнє.
Маріупольчанка планує продовжити свою діяльність як ілюстратора, розвивати
освітній центр \enquote{Місто} та доступну якісну освіту для дітей, а також вона хоче
відродити проект \enquote{Маріуполь читає дітям}. З творчістю унікальної і самобутньої
космічної феї Маріуполя можна познайомитися на сторінці у FB \enquote{Місто щасливих
людей}.

\textbf{Улюблена книга Анастасії Пономарьової:} \enquote{Маленький принц} Антуана де Сент-Екзюпері.

\textbf{Хобі:} велосипед, колекціонує черепашок, любить фотографувати і те, чим займається.

\textbf{Порада маріупольцям:} 

\begin{quote}
\em\enquote{Не боятися, що не вийде, а просто починати рухатися до мети. Ресурсів завжди
не вистачає, але якщо є бажання змінювати світ, треба спробувати. Завжди варто
спробувати!}
\end{quote}

\ii{insert.read_also.demidko.stomina}

\ii{09_04_2019.stz.news.ua.mrpl_city.1.anastasia_ponomareva_kosm_feja_mrpl.pic.10}
\ii{09_04_2019.stz.news.ua.mrpl_city.1.anastasia_ponomareva_kosm_feja_mrpl.pic.11}
\ii{09_04_2019.stz.news.ua.mrpl_city.1.anastasia_ponomareva_kosm_feja_mrpl.pic.12}
