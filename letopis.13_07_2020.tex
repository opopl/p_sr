% vim: keymap=russian-jcukenwin
%%beginhead 
 
%%file 13_07_2020
%%parent articles
 
%%endhead 
\subsection{Русским не за что вставать перед неграми на колени}
\url{https://www.facebook.com/groups/LNRGUMO/permalink/2841452399299649/}
  
\vspace{0.5cm}
{\small\LaTeX~section: \verb|13_07_2020| project: \verb|letopis| rootid: \verb|p_saintrussia|}
\vspace{0.5cm}
  
Российский автогонщик Даниил Квят во второй раз отказался встать на колено ради
Black lives matter перед гонкой "Формулы-1"

В этот раз его поддержали не пятеро, как в прошлый раз, а всего три гонщика из
пелетона. Остались стоять пилоты Леклер, Ферстаппен и Райкконен. "Откололись"
Сайнс и Джовинации.

Всё верно. Русские к проблемам англосаксов, которые любят проецировать свои
проблемы на весь остальной мир, отношения не имеют. А то на следующей гонке
дело может дойти до целования ботинок Хэмилтона, нам это зачем? Это все
разборки между бывшими рабами и их бывшими хозяевами, в которых русским
принимать участие не за чем.

Пусть встают на колено и каются потомки семей рабовладельцев, королевские семьи
Европы. Для Британской короны встать на колено, малая толика извинения. Штаты
эксплуатировали рабских труд негров, Англия убила в 50-60 годы 20 века в Африке
огромное количество населения, занималась геноцидом. Королю Бельгии тоже стоит
повиниться.

Если пилоты Формулы раком в поддержку геев вдруг начнут вставать, то это их
право, но наши не будут этого делать.

\ii{13_07_2020.fb.lnr.2}
