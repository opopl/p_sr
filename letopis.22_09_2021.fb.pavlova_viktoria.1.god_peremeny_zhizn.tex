% vim: keymap=russian-jcukenwin
%%beginhead 
 
%%file 22_09_2021.fb.pavlova_viktoria.1.god_peremeny_zhizn
%%parent 22_09_2021
 
%%url https://www.facebook.com/pavlova1975/posts/835883463797700
 
%%author_id pavlova_viktoria
%%date 
 
%%tags chelovek,dnr,donbass,doneck,god,zhizn
%%title Всего-то год прошел - Facebook напоминает, а сколько всего переменилось
 
%%endhead 
 
\subsection{Всего-то год прошел - Facebook напоминает, а сколько всего переменилось}
\label{sec:22_09_2021.fb.pavlova_viktoria.1.god_peremeny_zhizn}
 
\Purl{https://www.facebook.com/pavlova1975/posts/835883463797700}
\ifcmt
 author_begin
   author_id pavlova_viktoria
 author_end
\fi

Всего-то год прошел - Facebook напоминает, а сколько всего переменилось. Год
назад в моей жизни было все беспросветно, как мне казалось. 

Тогда все навалилось сразу. Блокировка банковских карт, закрытые из-за
карантина границы, потеря работы, просроченный паспорт. Я лишилась работы,
денег, надежды. 

\ifcmt
  ig https://scontent-lga3-1.xx.fbcdn.net/v/t1.6435-9/242620781_835883423797704_437405533456417788_n.jpg?_nc_cat=106&ccb=1-5&_nc_sid=730e14&_nc_ohc=y_STt-PBt0YAX-uKPnu&_nc_oc=AQkXbzOmgHfIA7PpkuBTQE0OhO4TporMrzICJtBKtl6nYbVShb79FoAuX67V0UGgjso&_nc_ht=scontent-lga3-1.xx&oh=f6b51447923757b42449726fff0a6dcd&oe=617448F9
  @width 0.4
  %@wrap \parpic[r]
  @wrap \InsertBoxR{0}
\fi

Я, активный человек по жизни, безвылазно сидела в Донецке, перебиваясь
случайными заказами, медленно сходя с ума от отчаяния и доводя до исступления
своими истериками близких. Бросила зал, бассейн, закрылась от всех. Да, и у
эмоционально сильных людей бывают срывы.

Тогда я сравнивала себя с кавказской овчаркой - свободолюбивым животным,
запертым и тоскующим в тесном  вольере. Изредка муж брал меня с собой за город
- в Иловайск, Углегорск, Шахтерск, Енакиево, Торез, чтобы я развеялась,
отвлеклась. 

Я отвлекалась - на несколько часов, а потом отчаяние и безысходность замкнутого
периметра накрывали меня с новой силой. Да и, чего греха таить, города, в
которых я побывала, не добавляли оптимизма. Наоборот, видя разруху и безнадегу,
я с новой силой погружалась в нее сама.

Не заметила, когда начала привыкать или приспосабливаться к обстоятельствам.
Наверное, было обычное утро. Наверное, я проснулась и сказала себе: "На все
плевать, будет - как будет". Наверное, после этого взяла книгу или включила
сериал, чтобы переключиться.

Прошел год. Сейчас, оглядываясь в прошлое, понимаю, как круто все изменилось.
За год я дважды сменила работу. Восемь месяцев проработала на новостях в
уфимском информагентстве, но, устав от рутины, вернулась к корпоративной
журналистике и копирайтингу. Довольна? Да, довольна. Паспортный, банковский и
другие рабочие вопросы тоже постепенно решились.

Я снова свободна - как только чувствую, что давит обстановка, обстоятельства,
погрязла в быту, уезжаю. Пусть на день, на два - помогает. Но это так,
частности. Хотя, вся наша жизнь - это цепочка частностей. 

Безнадега больше не накрывает. Наверное, потому что  научилась жить здесь и
сейчас, одним днем. Не строить далеко идущих планов, радоваться каждой мелочи,
баловать себя, не расходовать энергию и время на нытиков, циников и утопистов,
а по возможности дистанцироваться от них.

А еще я усвоила главный урок. Иногда правильнее не изводить себя, а
остановиться, замедлиться и довериться судьбе. Особенно, если ты фаталист.
Сегодня кажется, что черная полоса затянулась, что ты задыхаешься от отчаяния,
что это никогда не закончится, но пройдет время, и забрезжит просвет. Главное,
что близкие живы и здоровы. 

Безнадега, отчаяние, ощущение тупика - все это уже было, когда в критический
момент, в момент жизни и смерти уходил любимый человек, подставляли коллеги,
лучшая подруга плела интриги за твоей спиной. Когда, когда, когда...

Но ведь ты выстоялась. Ты жива и здорова - это главное. Значит, все у тебя
будет. И будет вовремя. Просто живи.
