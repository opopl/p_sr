% vim: keymap=russian-jcukenwin
%%beginhead 
 
%%file 21_11_2014.fb.fb_group.patrioty_ukrainy.1.godovschina_evromaidan.cmt
%%parent 21_11_2014.fb.fb_group.patrioty_ukrainy.1.godovschina_evromaidan
 
%%url 
 
%%author_id 
%%date 
 
%%tags 
%%title 
 
%%endhead 
\subsubsection{Коментарі}

\begin{itemize} % {
\iusr{Анатолій Козирєв}

Ну раз розчарування, то почухайте потилицю! Хочете сказати ,що нічого було
рипатись? Чи чекаєте, що на блюді рай принесуть прямо у ліжко? Згадайте, що
зробили особисто, відмітьте особисті досягнення і помилки і далі до роботи.
Зерно і то не зразу проростає. Суспільство вимагає куди більших зусиль і
кваліфікованої праці для свого зростання. Праці всіх, без винятків.

\iusr{Юрий Вакуленко}
\textbf{Анатолій Козирєв}, а хто працює, окрім суспільства? влада? політики?

\iusr{Natalia Polinnyk}
Юрій, тобі вже писали не раз-ПНХ.

\iusr{Юрий Вакуленко}
\textbf{Natalia Polinnyk}, вибачте, але я не розумію..

\iusr{Юрий Вакуленко}
\textbf{Natalia Polinnyk}, 

я проти влади і проти тих людей, що зараз керують країною. яне бачу позитивних
зрушень за рік з початку формування нового суспільства. кожного дня все тільки
гірше. і чому я не можу висловити свою думку з цього приводу!? я впевнений, що
багато людей вважають так само, як і я. і що, їх всіх посилати? ви не праві. а
якщо я не правий, доведіть це аргументовано, а не висловлюваннями
школярів-дегенератів.

\iusr{Natalia Polinnyk}

А я бачу, мало того, розумію, наскільки важко владі. За мільйони я б не хотіла бути
у їх шкірі. Нам потрібно підтримати владу, але вказувати на помилки. Я відчуваю
зміни, до речі, чиновники з якими стикаюсь різко поміняли тон.

\iusr{Юрий Вакуленко}
\textbf{Natalia Polinnyk}, 

а я не бачу змін. чиновники може і стали ввічливіше, але не це головне. Владі
важко?! де ви бачити щоб їм важко було. коалцію майже місяць обговорювали. а
люди чекають, коли почнуться реформи...

\iusr{Анатолій Козирєв}

Пане Юрію. Коли ві міняєте запобіжник Ви не лаєте Фарадея. Влад це також і Ви.
Кожен робить що може. Виказ невдоволення може зробити і мала дитина. Наприклад,
що льодяника не дали. Управління державою відбувається великими масами людей,
відповідно до розпідулу праці. 

Отже берете певне управлінське питання і
визначаєте, який алгоритм його вирішення. Порівнюєте з фактом. Зназодте
помилковий етап. Аналізуєте умови, кваліфікацію виконавців, динаміку зміни
обставин, зннаходте правильне рішення, оцінюєте його вартість, проектуєте
алгоритм виходу із наслідків управлінського збою, доповідаєте особі, що приймає
рішення, берете відповідальність ( якщо потягнете), працюєьте. Тоак просто. Ну
як у Леніна, про управління держави кухарками. Але керувати ті хочуть, тільки
вчитися, працювати і відповідать чомусь зась. Тут дописують для громади, для
створення думки. Бо конкретну критику пишуть тим, кого критикують.

\iusr{Анатолій Козирєв}

Даруйте за помилки- погано бачу.

\iusr{Юрий Вакуленко}
\textbf{Анатолій Козирєв}, 

алгорітм повинна знати кожен керівник, але в нащій країні неможливо потрапити
до влади чесній людині. Ви багато викладачів, професорів, лікарів, вчених,
бічити у Раді?! самі бізнесмени, їх маріонетки та співаки-співачки, блазені, а
не депутати. і Ви хочете, щоб вони працювали на користь звичайних українців,
які отримують зарплатню 2000 грн, коли ці виродки куштують у сніданок за таку
ціну в ресторанах. чи Ви бачили депутата, який їздить у метро? чи Ви бачили
депутата, який живе у звичайній квартирі?!

\iusr{Юрий Вакуленко}

я відповім за Вас - ні, таких немає, бо чесні люди не можуть вплинути на
політику. навіть Майдан привів до влади злочинців та зрадників. не даврма люди
сьогодні освистали Порошенко. Тому що люди знають, що окрім обіцянок, Пєтя
нічого не зробив для країни.

\iusr{Анатолій Козирєв}

Даруйте. поки обрали не законодавців, а політичних лобістів. Чи є людина, що
знає ВСЕ законодавство України? Потрібно системне чищення всьго закондавства,
побудованого під клас "неодворян" - "державних службовців. Боркратисна систма
управління з формалізованими стандартними рішеннями ефективна тільки при
мінімумі управлінців. 

Хороший чиновник подібний регулювальнику на дорозі, його
минають не зупиняючись. В спадок від СРСР нам дісталась система управлінцями
ставали віддані люди, основними діловими характеристиками були " може випити
відро горілки" та добре співає, ну і може зробити особисту послугу. Європейці
не можуть вкласти такого в голову.... Для України зараз норма - йти в чиновники,
щоб мати додаткові винагороди. То хто ж прості люди тоді? Чи ті ж професори не
беруть хабарів з незнаючих? Хворе все суспільство. Знаючи це воно починає
очищатись. Через те Майдан, звідси волонтери.. Інтеллігенція першою, ще з 2004.

Але велика маса людей хоче жити по старому, щоб за неї подумали, вирішили,
дали. Зверніть увагу на дрібниці коло кіосків з кавою маса викинутих посудин.
Бо навіщо іти до урни - приберуть. Це таке собі панство.. Його зміст - для мене
мусять. Замість того щоб було гасло Я роблю і Ти мусиш робити. Тільки тоді не
заженуть в куток. Управління - як шахи. Може грати і дитина і великий майстер.

Різниця в кваліфікації. Для обрання квалііфікованого менеджера теж потрібні
знання, що можна зробити і при яких умовах. Поки для реалізації вимог майдану
умови були відсутні. 

Є сподівання, що зараз може щось зрушитись, Як що обрані за
гречку не організують ігри з постійною зміною керівництва. В такому разі будем
по колу міняти, виганяти, реорганізувати, критикувати і проклинати... Поки не
зрозуміємо, що потрібно знати, що і з кого і коли вимагати та на яких
підставах.

\iusr{Сергей Величко}
Наші політики забули, завдяки кому прийшли до влади! Ганьба!

\iusr{Ульяна Соломей}
Можливо потрібно дати ще трошки часу можливо покращення всеж буде?

\iusr{Виктория Котенко}

дуже багато людей загинуло на Майдані для нашого світлого майбутнього! згадую
ті жахливі кадри, коли Беркут розганяв Майдан-студентів... ніхто навіть подумати
не міг, що все зайде так далеко, а сталося все через те, що влада боялась того, що
люди зможуть зкинути систему, що була... у людей вийшло, але все сталося не так
як хотілося... українці стали жити ще гірше+ війна+розкол країни+люди
продовжують гинути

\iusr{Кира Рібалко}

На даний момент зовсім не згодна з діями влади, але все одно я сподіваюся, що все
владнається і ми заживемо щасливо! Обранi нами полiтики не виправдали сподiвань
до сих пiр, але ж може ще замало часу минуло

\iusr{Юрий Терещенко}

Це вже може бути вже третій Майдан! Підсумки показують, що краще жити не стало!
всі обіцянки так і залишилися обіцянками

\iusr{Андрей Яровой}
І нащо то був потрібен майдан. який тільки горе приніс?

\iusr{Ирина Савчук}
Обіцянки треба виконувати, а вони їх позабували...

\iusr{Данил Фролов}
\textbf{Владислава Крамаренко} майдан був потрібен! І ще потрібен!

\iusr{Данил Фролов}
Зрадники у владі

\ifcmt
	ig https://scontent-frt3-1.xx.fbcdn.net/v/t1.18169-9/10300010_1503364286583059_2527902924866137280_n.jpg?_nc_cat=107&ccb=1-5&_nc_sid=dbeb18&_nc_ohc=AWVMFuZBd0cAX8ObtWV&_nc_ht=scontent-frt3-1.xx&oh=b23a0ca934168eae336eab8c324536ab&oe=61BA600A
	@width 0.3
\fi

\iusr{Константин Дорошенко}
подскажите зачем нужен был майдан ящитаю это и зАвела этот безжалостный механизм

\iusr{Sergey Ruzhinsky}

\ifcmt
	ig https://scontent-frt3-1.xx.fbcdn.net/v/t1.18169-9/10547595_1507060739555160_1337323483375147974_n.jpg?_nc_cat=108&ccb=1-5&_nc_sid=dbeb18&_nc_ohc=02bgYeGvZT4AX9yiyZr&_nc_ht=scontent-frt3-1.xx&oh=f21f767b8cfb20b1fc8d25a78bb6233e&oe=61BCAB62
	@width 0.4
\fi

\iusr{Василь Середа}

А якщо без зайвих емоцій а тільки по суті, то ось, що я скажу. Янукович (мені
не подобався) брав і у Росії, брав у Европи, брав у Америці, але він не вступав
ні до ЕС, НАТО та усеякі союзи. Трусив морквою(обіцянки) перез
кроликами(іноземці), але ми не влазили ні в яку халепу. 

У мене особисто була стабільність. Я знав чітко свою заробітну плату, знав на
що її витрачу, знав коли та як відпочину, скільки проплачу за комунальні
послуги та їжу та інше. Я патріот своєї Країни, але пару років потому хтось
мені сказав би, що я буду допомагати пораненим, своїм братам, на своєї землі у
війні проти своїх же громадян, то я плюнув би у пику, або навіть по щелепі
врізав би. Але ми зараз маємо, те що маємо. 

Особисто у мене зараз не вистачає грошей на все. Справа не у відпочинку, у мене
зараз навіть додаткова робота є, але це теж копійки.  Уявіть скільки зараз
коштує 1 знаряд для гармати? Що можна купити на ці гроші?

Зараз амеріка продає нам зброю на 115 або 116 мільонів баксів, а віддавати
будемо ми! Навіщо це? Я особисто не розумію. Провели майдан, для того щоб
викинути бізнесменів та олігархів. Чого ми добилися? Змінили Пупкіна на
Васечкіна.

\iusr{Екатерина Подопригора}

новий Майдан Україні не потрібен. Треба змінювати політичну систему країни, щоб
люди вирішували, у якому напрямку має рухатись політика.

\iusr{Екатерина Подопригора}
а у нас як влада сказала - так і буде. і як правило, рішення влади не робить життя кращим...

\iusr{Надежда Юрченко}
Катерина, в точку Вы сказали. Мало того у них слова с делом-то расходятся.

\iusr{Надежда Юрченко}

Касательно Майдана, то год назад народу предоставили право выбора и кого народ
выбрал? Необразованных, неуравновешенных и кровожадных маньяков. которые больше
всего орали на Майдане. Из всех мне более менее нравился Ярош, но я понимаю, что
его к власти нельзя допускать, так как получится государство с военной
диктатурой.

\iusr{Ульяна Соломей}
Інна, тепер потрібно почекати все налагодиться

\iusr{Юрий Вакуленко}
Ми вже на цьому рівні.

\end{itemize} % }
