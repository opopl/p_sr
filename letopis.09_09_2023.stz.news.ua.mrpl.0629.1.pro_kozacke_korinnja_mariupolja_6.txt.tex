% vim: keymap=russian-jcukenwin
%%beginhead 
 
%%file 09_09_2023.stz.news.ua.mrpl.0629.1.pro_kozacke_korinnja_mariupolja_6.txt
%%parent 09_09_2023.stz.news.ua.mrpl.0629.1.pro_kozacke_korinnja_mariupolja_6
 
%%url 
 
%%author_id 
%%date 
 
%%tags 
%%title 
 
%%endhead 

Вадим Коробка (Маріуполь)
Юлія Коробка (Маріуполь)
09_09_2023.mrpl.0629.vadim_korobka.julia_korobka.pro_kozacke_korinnja_mariupolja_6
Маріуполь,Україна,Мариуполь,Украина,Mariupol,Ukraine,date.09_09_2023

Про козацьке коріння Маріуполя. Частина 6. Яку ж спадщину від селища Кальміуська паланка (Кальміус) отримав Маріуполь

Про козацьке коріння Маріуполя. Шоста частина статті доцентів кафедри історії
та археології Маріупольського державного університету Вадима Коробки та Юлії
Коробки, які наводять доводи та факти щодо справжньої дати заснування
Маріуполя. 

Перша частина тексту за посиланням. Друга частина тексту тут. Третя частина -
тут. Четверта - за цим посиланням. П'ята частина дослідження тут.

Історичні джерела свідчать, що у 1740-х рр. запорізькі козаки почали системне,
осіле господарське освоєння узбережжя Азовського моря. Завдяки цьому
встановлювалися господарські (торговельні) зв’язки Надазов’я з українськими
землями, імперським простором та закордоном. Селище Кальміуська паланка була
зародком торговельно-транспортного хабу (вузла), повноцінне становлення якого
відбулося вже в межах Маріуполя протягом майже всього ХІХ ст. 

Наступність із козацьким періодом (особливо на початку опанування краєм
переселенцями з Кримського ханату) полягала в безперервній наявності
українського етнічного компонента у розвитку продуктивних сил краю. Так,
внаслідок неприхильності «маріупольських греків» до рибальства «одними только
малоросиянами лов производится» .

Українське козацтво започаткувало те, що у нашому краї триває до сьогодні –
безперервну традицію розвитку продуктивних сил. У той час,  насамперед, до них
відносились знаряддя для рибальського промислу – судна, човни, риболовецькі
снасті, а також люди, здатні до праці, які мали певні навички й знання
судноплавства, рибальства, обробки та зберігання риби, ікри та інших
рибопродуктів.

Місто Маріуполь влаштовувалося не на порожньому місці. Від  історичних
попередників залишилась матеріальна спадщина, ймовірно, до 9\% початкової
потреби в житлових приміщеннях. У тогочасному звіті зазначалось: «…Куплено  у
прежде живших там малороссиян пятьдесят пять [55] домиков». 

Українська складова Маріуполя (етнічна, мовна, побутової культури та
інтелігентське середовище від 1870-х років) є неодмінним компонентом його
міської історії від Кальміуської паланки до сьогодення.

Імперське законодавство вимагало створення креслень планів міст, що відображали
б реальні та запроектовані об’єкти міської забудови. На всіх зараз відомих,
опублікованих у мережі, планах Маріуполя 80 – 90-х рр. XVIII cт. зображено
зовсім порожньою ту місцину, де колись розташовувалось селище Кальміуська
паланка (Кальміус) – територія між правим берегом гирла річки Кальміус,
узбережжям Азовського моря та озером Домаха. 

Зображення пустиря на місці, де було селище Кальміус на проектах міської забудови Маріуполя 80-х – 90-х рр., XVIII ст., звичайно, можна пояснити тим, що його залишили жителі – колишні козаки, сімейні та одинаки. Дійсно, таке було. Дехто в краєзнавчих пабліках використовує ці факти для заперечення наступності між селищем Кальміуська паланка (Кальміус), з одного боку, а з іншого – Маріуполем. 

Утім,  такі умовиводи спростовуються низкою міркувань та фактів.  По-перше,
картографічні матеріали не можуть слугувати вичерпним джерелом відомостей.
Вочевидь, вони укладались в губернській канцелярії, у відриві від місцевих
реалій. 

По-друге, недосконалість креслень міської забудови Маріуполя 80-х – 90 х рр.
XVIII ст. підтверджується великою хибою – неправильним позначенням гирла р.
Кальміус – того місця, де він впадає в Азовське море. Ця помилка крокувала з
одного плану Маріуполя в інший, від 1782 р. до 1799 р., більше 10 разів.
Зрозуміло, що перше креслення слугувало зразком другому й так далі. Ясно те, що
робилось воно за сотні верст від натури, як і наступні. Справа в тім, що в
уряду імперії руки не доходили до впорядкування судноплавства у цій місцевості,
а місцевого самоврядування фактично не існувало, тож не брались до уваги реалії
та приватні ініціативи.

По-третє, саме приватна ініціатива розквітала тут із часів заснування селища
Кальміуська паланка (Кальміус). Гирло річки Кальміус було своєрідною
мілководною затокою Азовського моря. На цьому місці була база для стоянки
(гавань) козацької видобувної «флотилії». Тут безперервно, саме з часів
паланки, до нашого часу функціонували пункти для причалювання й розвантаження з
суден (здебільшого риби). Тут же відбувалось навантаження на засоби тогочасного
водного транспорту рибальського знаряддя, ймовірно, солі, харчу та горілки для
рибарів. Є поки що поодиноке свідчення вивезення звідси за кордон імперії зерна
ще в паланочні часи. 

У часи паланки на узбережжі гирла р. Кальміус не були обладнані спеціальні
причальні споруди для маленьких козацьких суден. Зараз немає відповіді на
питання про наявність протягом останньої чверті XVIII cт., у постпаланочні,
маріупольські часи, спеціально обладнаного причалу, складських приміщень. Але
судна  не припиняли відчалювати з метою рибного промислу та його матеріального
забезпечення, не зупинялось, а збільшувалось завезення сюди видобутої риби.  

«Главный внутренний торг в Мариуполе есть рыбной», – стверджував Таганрозький
градоначальник барон Балтазар Кампенгаузен у звітності за 1807 р. в
міністерство внутрішніх справ (з цього року наше місто потрапило в
адміністративне підпорядкування Таганрозького градоначальства).

За відомостями іншого таганрозького градоначальника, генерал-майора Петра
Папкова, які були опубліковані Оленою Дружиніною, визнаним фахівцем з історії
Південної України, риба та рибні продукти у великій кількості доставлялися з
різних місць азовського узбережжя до Маріупольської пристані. З цією метою
використовувалися спеціальні човни (мовою документа – «лотки»), що піднімали по
3–5 тис. пудів вантажу (1810 р. таких човнів було до 70, причому кожен робив по
3–4 ходки). Суходолом до Маріуполя приходило за рибою та рибними продуктами до
10 тис. фур. На нашу думку, це були чумацькі мажі. Є й інші  подібні відомості,
вочевидь, за інший рік. У них йдеться, зокрема, про приїзд у Маріуполь до 13
тис. чумаків (мовою документів – «фурщиков»), які вивозили виловлену рибу на
продаж у малоросійські губернії.

Кількість суден, маж (фур) та чумаків взята не зі стелі. У місті вівся їх
облік, стягувались відповідні міські збори: з приходящих суден по 2 рублі, з
маж (фур) по 10 копійок. За нашими підрахунками в Маріуполі, як центрі рибної
торгівлі, в першому десятилітті ХІХ ст. щороку відбувалась продаж-купівля  від
6,4 до 22,4 тис. т риби. Надходження від риботоргівлі, зборів з рибних ловів на
теренах міської громади могли складати 80 і більше відсотків доходів дуже
скромного тоді маріупольського бюджету. У такий спосіб відгукувався паланочний
спадок.

Результати ділової активності та перспективи розвитку були помічені імперською
адміністрацією. 1799 р. світ  побачив законодавчий акт, що декларував створення
в Маріуполі відпускної митної застави. Це знаменувало початок вивозу за кордон
хліба. Кальміуська гавань перетворилась на порт, який відігравав чималу роль у
експорті пшениці в ХІХ ст. 

Непересічне значення на теренах Вольностей Війська Запорозького Низового мали
справи душпастирської опіки про козаків та наймитів паланки, її околиць й
задоволення їх релігійних потреб.  Неодноразові згадки про наявність у
Кальміусі православного храму свідчать, що тут священик та миряни складали
церковну громаду – парафію. Беззаперечним фактом є й те, що вона перебувала в
складі єпархії Київського митрополита. Чи не першим православним храмом на
тернах, де  виріс Маріуполь, була ця козацька Свято-Миколаївська церква?
Думаємо це питання риторичне, тобто містить у собі ствердну відповідь. Це факт
цивілізаційного значення. Переселенні з Кримського ханату православні християни
потрапили у споріднене у цивілізаційному плані людське оточення, що полегшувало
їх адаптацію. Певне значення мало те, що вони користувались спочатку храмом
цілком, а після вивезення церковного начиння його приміщенням.

Сміємо припустити, що листи з Кальміусу, адресовані в Кіш (уряд Січі), були
першими текстами, написаними  в нашій місцевості із застосуванням кириличної
абетки. Із Кальміуською паланкою щонайменше можна пов’язувати перше службове
листування, яке велось на теренах маріупольського Надазов’я, що репрезентує
Кальміуську паланку (Кальміус) як частину налагодженого адміністративного
організму. 

Враховуючи те, що наприкінці ХVII cт. Південь України як частина Османської
імперії був у сфері ісламскої цивілізації, просування українського козацтва на
терени гирла Кальміуса та узбережжя Азовського моря вперше презентувало тут
сталий осілій та християнський світ (цивілізацію).

ЧИТАЙТЕ ПОЧАТОК дослідження: 

Про козацьке коріння Маріуполя. Частина 1. Давні спроби осягнути чинник
українського козацтва в історії Маріуполя

Про козацьке коріння Маріуполя. Частина 2. Нові спроби привернути увагу до дати
заснування Маріуполя та дії команди Бойченка на цьому напрямку

Про козацьке коріння Маріуполя. Частина 3. Яким було селище Кальміуська паланка

Про козацьке коріння Маріуполя. Частина 4. Початок імперського упорядкування
теренів селища Кальміус та його околиць. Кальміуська слобода. Павловськ

Про козацьке коріння Маріуполя. Частина 5. Зустріч православних християн з
Кримського ханату та українців на теренах, де виріс Маріуполь
