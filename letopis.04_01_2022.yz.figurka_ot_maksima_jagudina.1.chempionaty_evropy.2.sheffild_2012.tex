% vim: keymap=russian-jcukenwin
%%beginhead 
 
%%file 04_01_2022.yz.figurka_ot_maksima_jagudina.1.chempionaty_evropy.2.sheffild_2012
%%parent 04_01_2022.yz.figurka_ot_maksima_jagudina.1.chempionaty_evropy
 
%%url 
 
%%author_id 
%%date 
 
%%tags 
%%title 
 
%%endhead 

\subsubsection{Чемпионат Европы 2012 (Шеффилд)}

Свою 4ю победу на Чемпионатах Европы одержала 24-летняя \textbf{Каролина Костнер},
которая в том сезоне находилась в великолепной форме и выиграла в итоге все
главные старты - кроме ЧЕ еще финал Гран-при и Чемпионат Мира, причем первый и
последний раз в своей карьере. На Чемпионате Европы как видим ее преимущество
было огромным, при этом в произвольной программе Каролина откатала чисто с 5
тройными прыжками и без каскадов 3+3, такой тогда был уровень. К слову для
того, чтобы стать затем чемпионкой Мира 2012 Каролине хватило точно такого же
контента, уровень конкуренции был не высок - \textbf{Андо} к тому моменту карьеру
закончила, а \textbf{Юна Ким} взяла перерыв на год.

\ii{04_01_2022.yz.figurka_ot_maksima_jagudina.1.chempionaty_evropy.pic.3}

Лучшей из россиянок стала дебютантка взрослых стартов 15-летняя Полина
Коробейникова, которая прошла через полуфинал, т.е. откатала за турнир дважды
свою произвольную программу. Полина показала на турнире самый дорогой каскад из
всех фигуристок тройной флип + тройной тулуп, редкое зрелище по тем временам
для женских стартов. Оказаться в числе трех лучших ей помешали крайне низкие
компоненты при лучшей технике в произвольной программе. Появление
Коробейниковой стало предвестником предстоящего нашествия молодых российских
фигуристок с каскадами 3+3. Между тем тот сезон 11/12 станет для Полины первым
и последним взрослым сезоном с международными стартами.

\ii{04_01_2022.yz.figurka_ot_maksima_jagudina.1.chempionaty_evropy.pic.4}

Лишь седьмой в Шеффилде стала Алена Леонова, но буквально через несколько
месяцев она достигнет пика в своей карьере и завоюет серебро на чемпионате Мира
в Ницце.
