% vim: keymap=russian-jcukenwin
%%beginhead 
 
%%file 25_10_2019.news.pravda_com_ua.SPARTA.starshii_gvardeec_TAMARA
%%parent 25_10_2019.news.pravda_com_ua.SPARTA
 
%%url 
%%author 
%%tags 
%%title 
 
%%endhead 

\subsubsection{Старший гвардеец Тамара}

Нынешний руководитель, или, как говорят в коммуне, старший гвардеец
"С.П.А.Р.Т.Ы." Тамара Костюк, родилась в 1970-м в Люботине "в роддоме, где
раньше было поместье люботинской дамочки, которая вышла замуж за философа
Бердяева". В школе она была комсоргом, параллельно с этим пела и пыталась
писать стихи.

В 1987-м Тамара в ХГУ им. Горького (сейчас --- Харьковский Национальный
Университет им. Каразина) познакомилась со сторонницами идей Давыдова, вступила
в "Б.К.Л." и начала участвовать в разработке Теории Счастья.

"Наш инструктор все время стремился к совершенству, занимался всю жизнь
исследованиями --- как устроены коллективы и мышление окружающих, что такое
дружба, любовь, верность, ревность, производительность труда, --- рассказывает
Тамара.

Нам тоже это было интересно, а рыбак рыбака видит издалека. Когда я после КПСС
и ВЛКСМ начала изучать феличологию (местная наука о счастье --- УП), это было как
переход в новый мир".

Вскоре после вступления Тамары в организацию у "Б.К.Л." появилась первая
штаб-квартира на улице Чигрина в Харькове. Давыдов и другие участники движения
стали жить в ней коммуной. На следующий год, сразу же после своего 18-го дня
рождения, к ним присоединилась и Тамара.

"Дамочки и ребята решили съехаться и жить в одной квартире, --- вспоминает
Тамара. --- Их было четверо --- это не было деление по парам, мы просто так
коллектив строили. 

Мама была в большом огорчении, падала на колени, спрашивала, на кого я ее
оставляю. Когда она закрыла хату, я пролезла в форточку, забрала свой паспорт и
побежала за Юрой. Я была экстремальной радикалкой или радикальной
экстремалкой".

\ifcmt
img_begin 
	url https://img.pravda.com/images/doc/3/6/364e254-3sparta.jpg
	caption ЗОЖ --- один из главных принципов организации Давыдова
	width 0.7
img_end
\fi

В 1988-м в Одессе Тамара пробежала свои первые 100 километров, затем начала
вести свою группу. К тому моменту "Б.К.Л." насчитывало уже 40 участников. После
разочарования руководства коммуны в некоторых из участников организацию решили
переименовать в "Б.К.Н.Л." --- Братство Кандидатов в Настоящие Люди". На
следующий год коммуна перебралась в квартиру на Пушкинском въезде в центре
Харькова.

"Мы жили вшестером в одной комнате, --- утверждает Тамара. --- Хозяйке квартиры не
нравилось, что мы туда-сюда ходим и однажды она вызвала милицию. 

Юра был дома, его менты схватили, бросили в тюрьму на 15 суток, вменяли
хулиганство. Мы устроили на "политпяточке" (площади --- УП) пикет, собирали
подписи для его освобождения. Тогда Перестройка была на улицах, и решение суда
отменили через девять суток".

В 1989-м году во время традиционной "сотки" в Одессе у "Б.К.Н.Л." появилась
собственная символика --- четырехцветное знамя.

Среди множества аляповатых элементов на флаге в глаза бросается золотой силуэт
Ленина. К вождю мирового пролетариата в коммуне относятся с пиететом. Бюст
Владимира Ильича также покрывается пылью в одной из комнат, прозванной
Ленинской.

"Ленин на знамени, потому что всем знаком --- и правым, и левым, и красным, и
коричневым, и жовто-блакитным, --- пространно объясняет Тамара. --- Это один из
примеров того, что когда индивид много делает в масштабах нации и
межнациональной интеграции, след становится заметным, и когда он переходит в
мир иной, этот след исследуется. 

Мы поддерживаем две стороны: приходят демократы --- поддерживаем коммунистов.
Приходят коммунисты --- поддерживаем демократов. Мы считаем, что нужно помочь
остаться в живых и тем, и другим. Иначе одни других заколошматят. Наша задача –
сбалансированное счастливое общество".

\ifcmt
img_begin 
	url https://img.pravda.com/images/doc/5/6/56c60e5-4sparta.jpg
	caption В Ленинской комнате в С.П.А.Р.Т.Е. раньше проживал топ-менеджмент
	width 0.7
img_end
\fi
\ifcmt
img_begin 
	url https://img.pravda.com/images/doc/7/2/7209090-8sparta.jpg
	caption На знамени С.П.А.Р.Т.Ы. изображены вождь мирового пролетариата и символика
эсперанто
	width 0.7
img_end
\fi

В 1990-м коммуна в очередной раз переехала --- из центра Харькова в спальный
район ХТЗ. На следующий год развалился СССР, но на участниках "Б.К.Н.Л."
падение империи сильно не отразилось.

"СССР --- это и Советский Союз, и совдепия. Есть борцы, а есть палачи, стукачи и
ГУЛАГ, --- подчеркивает Тамара. --- С одной стороны Союз однозначно надо было
обновлять, Михаил Сергеевич (Горбачев) был за обновление, и мы тоже. Но то, что
развал так подействует --- до такой степени пойдет обнищание народа и падение
экономики... ".

Одновременно с распадом СССР "С.П.А.Р.Т.А.нцы" изобрели собственную иерархию
управления. С 1991-го в организации действует "двухпартийный парламент", в
состав которого входят "партии" дженералов и универсалов. 

Универсалами сейчас руководит Ольга Широкая, еще одна участница движения Юлия
Приведенная, с которой Тамара знакома уже 28 лет, курирует дженералов из
Москвы. Сама Тамара при этом координирует обоих.

"Дженералы и универсалы соревнуются на семи уровнях управления, --- объясняет
руководитель "С.П.А.Р.Т.Ы." Тамара. --- Они устроены как конструктивная
организация --- суть в том, что одна партия помогает другой выкарабкиваться. Мы
называем это "К.О.Р.О.Л". --- "Коллективный Объективный Разум Объединения
Любского (от слова Любовь)". 

Поскольку мы сделали парламент, бросились все в процесс Amikidji (в переводе с
эсперанто --- "подружиться"). У тебя команды не будет, если ты ее не соберешь.
Выбрали самый простой способ --- пошли по селу и говорили: "Люди, люди, хто зі
мною?" Amway отдыхает по сравнению с нами".

В начале 90-х Давыдов запустил свое движение в ряде городов России --- в Москве,
в Калуге и в Перми. Вскоре участники квазипарламента "Б.К.Н.Л." внезапно
оказались в эпицентре противостояния у парламента настоящего.

