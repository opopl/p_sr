% vim: keymap=russian-jcukenwin
%%beginhead 
 
%%file 23_08_2021.fb.bacman_alesja.1.krym_platforma
%%parent 23_08_2021
 
%%url https://www.facebook.com/alesia.batsman/posts/10224195408864517
 
%%author_id bacman_alesja
%%date 
 
%%tags krym,krymskaja_platforma,rossia,ukraina
%%title Крымская платформа
 
%%endhead 
 
\subsection{Крымская платформа}
\label{sec:23_08_2021.fb.bacman_alesja.1.krym_platforma}
 
\Purl{https://www.facebook.com/alesia.batsman/posts/10224195408864517}
\ifcmt
 author_begin
   author_id bacman_alesja
 author_end
\fi

Крымская платформа.

Опытные дипломаты называют такие инициативы созданием новых козырей, которых в
колоде изначально нет. До сегодняшнего дня тему Крыма поднимали только самые
упёртые и романтичные, с сегодняшнего - она опять остро встала на повестке дня
46 основных государств мира. 

Крымская платформа - однозначно главная геополитическая победа Владимира
Зеленского за два года президентства. Это не просто классный пиар-ход (хотя он
реально классный, чего только стоит армада флагов стран Евросоюза, США,
Великобритании и других наших друзей на фоне флага Украины в сердце Украины),
это тот фундамент, на котором предстоит построить проект возвращения Крыма
домой. 

Лень много писать о том, что эту историю должен был сделать президент Порошенко
по свежим следам в 2014-2015-ом. Но подписав Минские соглашения, в которых нет
ни слова о Крыме, он успокоился до конца своего срока. Теперь нелепо пытается
догонять этот Поезд деоккупации имени Зеленского, потому что его собственный
безвозвратно ушел.  

России очень больно, их тв и их политики в реальном припадке. А это всегда
лакмусовая бумажка. Оказалось, обещание Путина  о том, что «вопрос Крыма закрыт
навсегда» - фальшивка.

Что нужно делать дальше по Крыму, когда есть основа в виде Крымской платформы:

- Юридическая составляющая. Очень важно четко и грамотно документировать все
преступления и все, что происходит на оккупированном полуострове. Очень важно
иметь работающую систему правосудия переходного периода. Кстати, это
направление, идеологом которого выступил и блестяще вёл Гюндуз Мамедов. 

- Информационная составляющая. Продолжаю повторять, что украинской власти
крайне важно создать сильный канал коммуникации и с населением оккупированного
Крыма, и Донбасса, и в принципе со всей русскоязычной аудиторией в мире.
География и история не оставили нам выбора: Украина обречена стать форпостом в
борьбе с российской пропагандой и российскими фейками. Мы просто обязаны
создать канал, который будет на русском языке доносить правду и об Украине, и о
роли России и затмит любой раша-тв, потому что будет интересным, честным и
объективным. Это наше важнейшее, но пока не используемое оружие на
государственном уровне. 

- Экономика в Украине. Чем богаче будут жить украинцы, тем больше жители
оккупированных Крыма и Донбасса, будут скучать за Украиной. Жалеть и
ностальгировать. СМИ и сарафанное радио сделают своё дело и во время «Ч» это
может стать решающим фактором. 

- Беженцы из Крыма и Донбасса. Политкорректно называемые внутреннеперемещенными
лицами, но по факту беженцы, которым пришлось бежать от своего дома, своей
работы, от своей жизни. А это страшно в любом возрасте. И если Украина покажет,
что она дорожит каждым своим гражданином, наконец-то создаст для таких людей
возможности высокооплачиваемого трудоустройства, социализации, льготного
обучения, возможности иметь достойное жилье и достойно реализовываться, это
будет важнейшим знаком, посланными всем украинцам, оставшимся под оккупацией.
Украина своих не бросает. Каждый украинец важен. 

- Свобода и демократия в Украине. Население оккупированных Крыма и Донбасса нам
завидует, потому что мы свободные. Те из них, кто склонен к простейшему
анализу, уже могут сравнивать жизнь в Украине, и жизнь под оккупационной
властью, когда за анекдот, рассказанный на кухне, люди попадают в подвалы.
Свобода и демократия это то, что нам нужно холить и лелеять. Это наша главная
ценность, которую нужно укреплять.
