% vim: keymap=russian-jcukenwin
%%beginhead 
 
%%file 09_06_2018.stz.news.ua.mrpl_city.1.park_kultury_i_otdyha_sezon_1958_goda
%%parent 09_06_2018
 
%%url https://mrpl.city/blogs/view/park-kultury-i-otdyha-sezon-1958-goda
 
%%author_id burov_sergij.mariupol,news.ua.mrpl_city
%%date 
 
%%tags 
%%title Парк культуры и отдыха: сезон 1958 года
 
%%endhead 
 
\subsection{Парк культуры и отдыха: сезон 1958 года}
\label{sec:09_06_2018.stz.news.ua.mrpl_city.1.park_kultury_i_otdyha_sezon_1958_goda}
 
\Purl{https://mrpl.city/blogs/view/park-kultury-i-otdyha-sezon-1958-goda}
\ifcmt
 author_begin
   author_id burov_sergij.mariupol,news.ua.mrpl_city
 author_end
\fi

\ii{09_06_2018.stz.news.ua.mrpl_city.1.park_kultury_i_otdyha_sezon_1958_goda.pic.1.v_gorsadu}

Эта небольшая книжица о двадцати восьми страницах и размером в стандартную
почтовую открытку попалась на глаза случайно. Она обнаружила себя, когда была
раскрыта давно нечитанная книга. Произведение полиграфического искусства,
отпечатанное блеклой синей краской в местной типографии тиражом 1000
экземпляров, пряталось между пожелтевшими страницами. Когда-то книжонка была
обычной рекламой, быть может, по современным понятиям не очень умелой. Но
теперь, по прошествии шестидесяти лет она превратилась в исторический документ,
поскольку дает молодому поколению некоторое представление о том, как проводили
досуг мариупольцы в старейшем парке города, а людям, чья молодость совпала с
тем временем, - повод для ностальгических воспоминаний.

\ii{09_06_2018.stz.news.ua.mrpl_city.1.park_kultury_i_otdyha_sezon_1958_goda.pic.2}

На ее мягкой обложке значится: \textbf{\enquote{Парк культуры и отдыха им. Жданова}}, под этой
надписью - неважное клише со скульптурным изображением А. А. Жданова,
облаченного в военную форму, с книжкой в левой руке. Этот монумент стоял
когда-то на одной из смотровых площадок парка. После странички с фотографией
центрального входа в парк следует набранная крупно надпись: \emph{\enquote{В сезоне 1958 года
в Парке культуры и отдыха будут проведены...}}.

\textbf{Читайте также:} 

\href{https://mrpl.city/news/view/interesnye-fakty-skolko-parkov-i-skverov-v-mariupole-foto}{%
Интересные факты: cколько парков и скверов в Мариуполе?, Яна Іванова, mrpl.city, 25.05.2018}

\ii{09_06_2018.stz.news.ua.mrpl_city.1.park_kultury_i_otdyha_sezon_1958_goda.pic.3}

Далее сообщалось, что к открытию сезона уже оборудованы стенды передовиков
производства заводов \enquote{Азовсталь} и коксохимического, треста \enquote{Азовстальстрой},
Жовтневого района, а также стенд героев-комсо\hyp{}мольцев Великой Отечественной
войны. Доводилось до сведения будущих посетителей парка, что в
библиотеке-читальне (жаль, что эта легкая веранда с решетчатыми стенами теперь
не существует) будут \emph{\enquote{прочитаны лекции на общественно-политические,
международные, естественнонаучные, технические, медицинские, литературные и
другие темы}}. Следует заметить, что эти лекции охотно посещались горожанами.

\ii{09_06_2018.stz.news.ua.mrpl_city.1.park_kultury_i_otdyha_sezon_1958_goda.pic.4.vhod_50_gody}

А еще библиотека-читальня предлагала: выдачу книг, журналов, газет,
литературные вечера и викторины, библиографические обзоры, читательские
конференции, книжные и фотовыставки, настольные игры с карандашом. Да, это
было. Вспоминаются седовласые головы, склоненные летними вечерами над
шахматными досками, лекции сотрудников библиотеки имени Короленко, обзоры
печати лекторов Общества по распространению научных и политических знаний.

Четверги отводились молодежи. В книжке напечатаны призывы: \enquote{Юноши и девушки!
Приходите в парк с музыкальными инструментами! Пойте песни, веселитесь!
Участвуйте в конкурсах на лучшего чтеца, певца, танцора!} И, можете себе
представить, конкурсы такие устраивались. Под баяны пели и плясали молодые, и
не только молодые, люди. Время от времени в летнем театре показывали
документальные и научно-популярные фильмы бесплатно. Благо, контора кинопроката
была рядом с парком.

Среда была посвящена вечерам под названием \enquote{За здоровый быт}, которые
проводились дирекцией парка совместно с городским домом санитарного
просвещения. В программе таких вечеров предусматривались вопросы и ответы на
медицинские темы, консультации врачей, лекции, выставки медицинской литературы.
Особой популярностью такие мероприятия пользовались, конечно, у людей
преклонного возраста.

Анонсировались в сезоне 1958 года гулянья, посвященные историческим датам,
тематические вечера по выбору профессии, окончанию учебного года в школах
города, молодой матери, Дням Советской печати, Военно-морского флота, шахтера,
молодежи. В том году впервые был отмечен День металлургов. Празднества эти,
кроме одного, как-то плохо отложились в памяти, но вот День Военно-морского
флота в Парке культуры и отдыха оставили яркий след. Его организовывали в
последнее воскресенье июля.

Позвольте привести отрывок из очерка, написанного в 1997 году:

\begin{quote}
\color{blue}
\enquote{Еще с утра в Городской сад из морского клуба ДОСААФ привозили экспонаты
выставки: водолазные костюмы, водолазные помпы, щиты с образцами морских узлов,
шарообразные мины, многочисленные модели боевых кораблей, нечто длинное,
сигарообразное - то ли торпеду, то ли элемент заградительного устройства,
обязательно присутствовали также штурвал, небольшой якорь, несколько
бело-красных пробковых спасательных кругов.

Поглядывая на занятых расстановкой по своим местам морских диковинок
устроителей выставки - морских офицеров в белоснежных кителях с надраенными до
солнечного блеска пуговицами, в фуражках в белых чехлах с \enquote{крабами}, мальчишки
норовили все потрогать руками, а что могло вращаться, то и повращать. \enquote{Морские
волки} делали вид, что ничего не замечают. Через главную площадь в нескольких
направлениях развешивали флотские флаги расцвечивания, прикрепляли к фонарным
столбам портреты прославленных флотоводцев. На смотровой площадке над обрывом
пиротехник и его помощники, предварительно расставив деревянные рамы,
прилаживали к ним длинные картонные трубки, хлипкие конструкции из тоненьких
жердочек, хитроумно соединяли все это хозяйство шнурами.

Ближе к вечеру центральная часть Городского сада постепенно начинала
заполняться празднично одетой публикой. То там, то здесь встречались на головах
мужчин морские фуражки. Заметим, что не все их обладатели действительно имели
отношение к флоту. В послевоенное время среди молодежи особым шиком считалось
носить невесть откуда добытые форменные морские головные уборы. Кто помоложе,
толпились у выставки, где бывалые моряки с ловкостью фокусников с невероятной
быстротой завязывали и развязывали морские узлы, ученики школы ДОСФЛОТа
передавали сообщения друг другу флажным семафором. Наконец объявлялось
соревнование по перетягиванию каната между командой ребят, одетых в полосатые
фуфайки, и добровольцев из публики. Можете поверить, что волонтеров всегда было
с избытком, и не всегда победа была за \enquote{морячками}.

Пока зрители громкими выкриками и свистом подбадривали соревнующихся, на
эстраде рассаживался духовой оркестр. Его концерт начинался с попурри на темы
морских песен. Солнце постепенно скрывалось за горизонтом. Морская выставка
сворачивалась. Темнело. На фонарных столбах загорались лампы. Оркестр замолкал.
Теперь все внимание присутствующих было обращено на площадку, где во тьме лишь
угадывалось замысловатое произведение пиротехника. Внезапно электричество
выключалось. Раздавался оглушительный хлопок - и в небо винтом взлетала, шипя и
разбрасывая по сторонам искры, ракета-шутиха. Сначала одна, затем другая,
третья. Каждая своего цвета: красная, зеленая, лимонно-желтая. Толпа, умолкшая
от неожиданности на секунду, издавала восторженный вздох и разражалась
аплодисментами. С пуском каждой ракеты окрест разносился визг девчонок и
пронзительный посвист их кавалеров. Огненное действо набирало силу.

Маленькие огоньки быстро продвигались по путям, проложенным шнурами, запуская
то три ракеты сразу, то пять. Ракеты со свистом прочерчивали темноту неба,
разрывались в вышине на огромные цветные шары из разбегавшихся от центра
цветных стрел, разбрасывали в вышине мириады сверкающих блесток. У земли же
крутились мельницы, приводимые в движение огнем, все больше раскручиваясь, они
образовывали ослепительные разноцветные круги. Всего несколько минут длился
фейерверк, затем запасы пороха иссякали и лишь догорающие угольки в темноте
ночи напоминали о только что завершившемся зрелище. Вновь включали
электрическое освещение, духовой оркестр играл прощальную мелодию, люди
начинали расходиться, обменивались впечатлениями о пережитом празднике}... 
	
\end{quote}

Главным концертными площадками парка были летний театр и эстрада, так
называемая раковина. В издании, о котором идет здесь речь, сказано, что
любителей музыки и драматического искусства ждут в 1958 году эстрадные
концерты, спектакли профессиональных театров, спектакли и концерты
художественной самодеятельности, духового и симфонического оркестров. С кем же
встречалась мариупольская публика в Парке культуры и отдыха в сезоне 1958 года?
С 31 мая по 29 июня в Летнем театре проходили гастроли областного
драматического театра имени А.С. Пушкина, со 2 июня шли спектакли Крымского
драматического театра, с 5 по 15 июля перед зрителями выступали артисты
Шахтерского ансамбля песни и пляски Донбасса \enquote{Молодая гвардия}. С 16 июля
начались гастроли областного театра оперы и балета, который предложил местным
поклонникам высокого искусства оперы \enquote{Бал-маскарад}, \enquote{Трубадур}, \enquote{Запорожец за
Дунаем}, \enquote{Травиата}, \enquote{Кармен}, \enquote{Чио-Чио-Сан}, \enquote{Риголетто}, балеты \enquote{Лебединое
озеро} и \enquote{Большой вальс}. Среди молодых исполнителей был Юрий Гуляев,
впоследствии известный на всю страну народный артист. А на летней эстраде с 4
июля начал концерты симфонической оркестр областной филармонии.

Что еще обещала дирекция парка в сезоне 1958 года? На спортивной площадке
ежедневные игры и соревнования в волейбол, баскетбол, настольный и большой
теннис, а в шахматном павильоне конкурсы на лучшего игрока в шахматы. В детском
городке намечались занятия кружков Дворца пионеров, встречи со знатными людьми,
дни пионерских лагерей городского типа...

На последней странице книжечки значилось: \emph{\enquote{Вход в парк бесплатный}}.
