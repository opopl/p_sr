% vim: keymap=russian-jcukenwin
%%beginhead 
 
%%file slova.sport
%%parent slova
 
%%url 
 
%%author 
%%author_id 
%%author_url 
 
%%tags 
%%title 
 
%%endhead 
\chapter{Спорт}
\label{sec:slova.sport}

%%%cit
%%%cit_pic
\ifcmt
  pic http://img.lug-info.com/cache/e/4/1623679443_779.jpg/1000wm.jpg
  caption Юные спортсмены из Республики завоевали 20 медалей в первом этапе открытых соревнований по плаванию \enquote{Звездная Лига Эволюции} в Крыму
\fi
%%%cit_text
Юные \emph{спортсмены} из Республики завоевали 20 медалей в первом этапе открытых
соревнований по плаванию \enquote{Звездная Лига Эволюции} в Крыму. Об этом сообщил
директор государственного образовательного учреждения (ГОУ) ЛНР "Центр
Олимпийской подготовки \emph{спортсменов} (ЦОПС) \enquote{Олимп-спорт} Сергей Ломакин.
Он рассказал, что в соревнованиях, которые прошли в Евпатории 13 июня, приняли
участие пловцы 2006-2012 годов рождения.
"В городе Евпатория на открытых соревнованиях по плаванию 
\enquote{Звездная Лига
Эволюции}, первый этап, \emph{спортсмены} ГОУ ЛНР \enquote{Центр Олимпийской подготовки
спортсменов \enquote{Олимп-спорт}} завоевали 20 медалей, из них 14 золотых, 4
серебряных, 2 бронзовых и командный кубок", - сообщил Ломакин
%%%cit_comment
%%%cit_title
\citTitle{Луганский Информационный Центр — Юные пловцы из ЛНР завоевали 20 медалей на открытых соревнованиях в Крыму}, 
, lug-info.com, 14.06.2021
%%%endcit

%%%cit
%%%cit_head
%%%cit_pic
%%%cit_text
Вот куда бы за умом и прагматизмом поехать киевским мудрецам и правителям. Но
не поедут. Потому что погоня за миражами и видимостью продолжается. Отправляя
\emph{украинских спортсменов} в Токио, президент Украины Владимир Зеленский
призывал их не завоевывать олимпийские медали в честной и справедливой борьбе,
а утверждать Украину на японской земле: «Дайте нам то, что дороже золота,
серебра и бронзы, — это эмоции от победы, гордость за то, что мы украинцы!
Выше, быстрее, сильнее! И никаких травм! Слава Украине!» И передал олимпийцам
флаг государства, подписанный украинскими военными, участвовавшими в
гражданской войне в Донбассе. То есть воевавшими с другими украинцами, у
которых просто другой взгляд на развитие страны
%%%cit_comment
%%%cit_title
\citTitle{«Синица в журавле». Вокабулярная война Украины за виртуальные границы}, 
Владимир Скачко, ukraina.ru, 26.07.2021
%%%endcit

%%%cit
%%%cit_head
%%%cit_pic
\ifcmt
  pic https://img.strana.ua/img/article/3459/darja-beloded-rasskazala-40_main.jpeg
  width 0.4
  caption Украинка Дарья Белодед вернулась в Украину после Олимпиады в Токио и рассказала о том, почему расстроена из-за бронзовой медали. Фото: "Страна"
\fi
%%%cit_text
С олимпийской призёршей Дарьей Белодед, которая принесла Украине первую медаль
на Олимпийских играх-2020, мы, как ни странно, встречаемся в McDonald's в день
ее возвращения на родину. Место, к слову, предложил отец Дарьи и по
совместительству ее тренер, Геннадий Белодед.  Выбор места для интервью
нетипичный как для \emph{спортсменки}, которая последние несколько лет вынуждена была
практически голодать, чтобы держать форму для весовой категории 48 кг. Что при
росте 1,72 м давалось 20-летней дзюдоистке непросто.  Но Дарья сходу дает
понять, что даже после долгих лет "воздержания" фаст-фуд ее не соблазняет.
После перелета из Токио в Киев, который занял почти сутки, \emph{спортсменка}
признается, что больше хочет выспаться, чем поесть. И хотя ближайшие месяцы она
планирует отдохнуть после трудных соревнований, \emph{спортивная дисциплина} все еще
не позволяет ей принять в угощение бургер
%%%cit_comment
%%%cit_title
\citTitle{Дарья Белодед: \enquote{После медали плакала, потому что я максималистка}}, 
Анастасия Товт; Полина Пронина, strana.ua, 28.07.2021
%%%endcit

%%%cit
%%%cit_head
%%%cit_pic
\ifcmt
  pic https://img.strana.ua/img/article/3484/mahuchikh-zajavila-chto-32_main.jpeg
  width 0.4
  caption Ярослава Магучих
\fi
%%%cit_text
Нож в спину - это всегда больнее, чем в грудь. Разочарование пришло оттуда,
откуда его не ждали. Наша прекрасная солнечная олимпийка прогнулась под всех и
вся и уже рассуждает нарративами Запада, используя сплошь и рядом
пропагандистские шаблоны майдановской власти. Довольно неприятно и даже
противно, ведь мы все за неё так болели, но прогиб одного не сломит волю
миллионов, предательства были, будут и есть, а нам нужно идти вперёд, на
сближение двух братских народов! Они могут запугать одну девочку, но они не
смогут устрашить целый народ, слишком уж они слабы и примитивны для этого.  Я
был готов услышать такую агрессивную риторику на "Прямом" из уст очередного
политика - "патриота", но читать такие слова на странице профессиональной
\emph{спортсменки}, человека, на которого мы все возлагали большие надежды, это ещё
раз признать, что гигантский государственный маховик никого не щадит и находит
для себя всё новые жертвы. А эта девочка, конечно, была настоящей тогда, когда
обнималась с российской \emph{спортсменкой}, а не сейчас, извергая из себя слова злобы
и ненависти по отношению к соседней державе. Прорвёмся", - написал нардеп
Харьковского горсовета от ОПЗЖ Андрей Лесик
%%%cit_comment
%%%cit_title
\citTitle{\enquote{Многие вас поддерживали и теперь разочарованы}. Что пишут в сети о покаянном заявлении спортсменки Магучих}, 
Оксана Малахова, strana.ua, 12.08.2021
%%%endcit

%%%cit
%%%cit_head
%%%cit_pic
\ifcmt
  tab_begin cols=3

     pic https://storage.lug-info.com/cache/a/f/af4be1d8-0a8b-48c2-a0ac-e8a1df343603.jpg/w700h474

     pic https://storage.lug-info.com/cache/0/b/a2ca4dd3-afb3-447f-9753-9ad8b9c7670f.jpg/w1000h616

     pic https://storage.lug-info.com/cache/a/c/0571a91e-8962-472e-90c4-5774da23506f.jpg/w1000h616

  tab_end
\fi
%%%cit_text
\emph{Спортсмены} из ЛНР завоевали 14 медалей на всероссийских соревнованиях
"Кубок России по каратэ-до фудокан", которые прошли в Иваново. Об этом сообщила
пресс-служба Министерства культуры, \emph{спорта} и молодежи (МКСМ) ЛНР со
ссылкой на общественную организацию "Союз каратэ Донбасса".  "В масштабном
\emph{спортивном} мероприятии, проходившем 16-17 октября в Иваново, выступили
каратисты со всех уголков России. ЛНР на соревнованиях представили спортсмены
клуба боевых искусств "Эдельвейс" Союза каратэ Донбасса, лутугинской
детско-юношеской спортивной школы (ДЮСШ) и молодогвардейской ДЮСШ № 2,
динамовцы Мария Ткаченко, Дмитрий Шумских, Данил Нагорный и Алексей Мельников.
По итогам соревнований наша команда завоевала 14 медалей", – говорится в
сообщении
%%%cit_comment
%%%cit_title
\citTitle{Каратисты из ЛНР завоевали 14 медалей на всероссийских соревнованиях в Иваново}, 
, lug-info.com, 18.10.2021
%%%endcit
