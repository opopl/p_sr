% vim: keymap=russian-jcukenwin
%%beginhead 
 
%%file 08_06_2020.fb.zharkih_denis.1.vremja_klounov
%%parent 08_06_2020
 
%%url 
 
%%author 
%%author_id 
%%author_url 
 
%%tags 
%%title 
 
%%endhead 

\subsection{Время клоунов}
\url{https://www.facebook.com/permalink.php?story_fbid=2745003812379685&id=100006102787780}

Говорил сегодня с моим другом известным украинским издателем Иваном Степуриным.
После разговора пришел к выводу, что профессия писателя, в смысле инженера
человеческих душ на Украине умерла безвозвратно. Уже невозможно просто написать
книгу и продать ее читателям. И читателей меньше и производство книг дороже.
Поэтому писатель должен еще сам и продавать книгу, организовывать встречи,
толкать свое детище в соцсети. Проще сказать, стать клоуном, привлекать к себе
внимание. Это значит, что рынок мыслей, образов и душевной работы на Украине
умер. Неудивительно, что в такой обстановке власть захватили злые клоуны и
требуют от людей убивать друг друга ради страшной забавы. Мы стали меньше
людьми, но как-то этого не заметили. Просто появилось масса новых профессий, но
исчезли профессии писателя, мыслителя и создателя глубоких образов, а это ничем
уже не восполнить.  

2015
