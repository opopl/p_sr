% vim: keymap=russian-jcukenwin
%%beginhead 
 
%%file 25_01_2019.stz.news.ua.mrpl_city.1.tradycii_prykmety_zabobony_studenty
%%parent 25_01_2019
 
%%url https://mrpl.city/blogs/view/traditsii-prikmeti-i-zaboboni-mariupolskih-studentiv
 
%%author_id demidko_olga.mariupol,news.ua.mrpl_city
%%date 
 
%%tags 
%%title Традиції, прикмети і забобони маріупольських студентів
 
%%endhead 
 
\subsection{Традиції, прикмети і забобони маріупольських студентів}
\label{sec:25_01_2019.stz.news.ua.mrpl_city.1.tradycii_prykmety_zabobony_studenty}
 
\Purl{https://mrpl.city/blogs/view/traditsii-prikmeti-i-zaboboni-mariupolskih-studentiv}
\ifcmt
 author_begin
   author_id demidko_olga.mariupol,news.ua.mrpl_city
 author_end
\fi

\ii{25_01_2019.stz.news.ua.mrpl_city.1.tradycii_prykmety_zabobony_studenty.pic.1}

В Україні і, зокрема Маріуполі, звикли святкувати День студента двічі: 17
листопада та 25 січня. Студентські роки ніколи не зникають з пам'яті, адже вони
наповнені не тільки радістю і безтурботністю, але й бажанням опанувати ази
майбутньої професії. Нові знайомства, друзі, переїзди, яскраві враження,
стипендія, життя в гуртожитку, часте недосипання, лекції, сесії, згуртований
відпочинок, смішні життєві історії, незабутні викладачі, пропущені пари – все
це є невід'ємною складовою студентського життя.

\textbf{Читайте також:} \emph{Призыв халявы и домик в зачетке: мариупольцы празднуют День студента}%
\footnote{Призыв халявы и домик в зачетке: мариупольцы празднуют День студента, Анастасія Папуш, mrpl.city, 25.01.2019, %
\url{https://mrpl.city/news/view/prizyv-halyavy-i-domik-v-zachetke-mariupoltsy-prazdnuyut-den-studenta}%
}

Напередодні Дня студента я згадала всі прикмети, яких дотримувалася моя група 5
років тому, і попитала своїх студентів, чи є у них свої традиції, забобони та
прикмети. Виявилося, що таких прикмет вдосталь. Пропоную ознайомитися з
найбільш яскравими.

Так, вважається, що якщо потрапиш під дощ в перший навчальний день – рік буде
вдалим і закінчиться з відмінними оцінками. Ще для успішного року
рекомендується 1 вересня у лівий черевик покласти п'ятак, тоді точно весь рік
будуть п'ятірки. Існує прикмета, що ввечері 1 вересня, якщо дивитися на зоряне
небо, збуваються бажання студентів, які не прогуляли навчання. Проте загадати
можна тільки одне, пов'язане з університетом.

\ii{25_01_2019.stz.news.ua.mrpl_city.1.tradycii_prykmety_zabobony_studenty.pic.2}

Найбільше прикмет пов'язують з підготовкою до іспитів. Так, деякі студенти
перед сесією не миють голову, не стрижуться, вважаючи, що волосся чіпати не
можна, адже в ньому зберігається вся мудрість. Ще, готуючись до іспиту, існує
добра традиція – посидіти на підручнику після прочитання. Перед сном забобонні
студенти кладуть конспекти та підручники під подушку. Також не можна змінювати
одяг, особливо якщо отримав на першому іспиті п'ятірку. На всі інші іспити слід
ходити в тому ж одязі, прати його категорично забороняється. В день екзамену
треба вставати з ліжка, виходити з дому, заходити в транспорт слід з лівої
ноги. Перед тим як відпустити товариша в аудиторію складати іспит, інші
студенти можуть цілувати його в ніс \enquote{на щастя}.

\textbf{Читайте також:} \emph{Студентам Мариуполя вернули право полноценно питаться в вузах}%
\footnote{Студентам Мариуполя вернули право полноценно питаться в вузах, Яна Іванова, mrpl.city, 26.12.2018, %
\url{https://mrpl.city/news/view/studentam-mariupolya-vernuli-pravo-polnotsenno-pitatsya-v-vuzah}%
}

Я у студентські роки теж вірила в одну прикмету. Так, рівно опівночі відчиняла
вікно і дуже голосно кричала тричі: \enquote{Шара, прийди}! Всі сусіди, що жили поруч,
завдяки цьому ритуалу знали про мою підготовку до іспитів. Проблем з оцінками
не було, завжди все складала на \enquote{відмінно}, хоча і готувалася теж ретельно, але
вважала, що \enquote{шара} все одно не завадить.

Цікавими є прикмети, що розповіли мені мої студенти. Так, дехто з них перед
іспитом зовсім не їсть, каже, що, коли голодний, складати легше. Більшість
перед іспитом не миють голову. Є й ті, хто вважає, що готуватися напередодні не
можна, слід користуватися знаннями, здобутими під час навчання. Цікаво, що й у
деяких викладачів є свої традиції. Так, моя колега на іспит першому курсу
завжди готує щасливий білет № 30. Він має 3 традиційні питання, хоча це більше
нагадує привітання. Текст цього білета наступний:

\enquote{1. Шановний студенте!

2. Вітаємо! Ваш білет щасливий. Ваша оцінка – А90.

3. Сподіваємося, перша сесія проходить для вас вдало}.

Правда, цей білет так ніхто й не витягнув. Але традиція дуже гарна, підкреслює
доброзичливе ставлення викладача до студентів.

\ii{25_01_2019.stz.news.ua.mrpl_city.1.tradycii_prykmety_zabobony_studenty.pic.3}

Повір'я про іспити користувалися популярністю кілька століть назад. Судячи з
відгуків, це працює. Але актуальною завжди є одна порада: до всіх цих прикмет
та забобонів треба ставитися зі здоровим почуттям гумору і готуватися до
іспитів серйозно. Ось тоді точно все буде \enquote{на п'ять}.

\textbf{Читайте також:} С\emph{туденты мариупольского вуза смогут стажироваться в университетах Эстонии, Швеции, Литвы и Польши}%
\footnote{Студенты мариупольского вуза смогут стажироваться в университетах Эстонии, Швеции, Литвы и Польши, Анастасія Папуш, %
mrpl.city, 30.10.2018, %
\url{https://mrpl.city/news/view/studenty-mariupolskogo-mgu-smogut-stazhirovatsya-v-universitetah-e-stonii-shvetsii-litvy-i-polshi-foto}}

\ii{25_01_2019.stz.news.ua.mrpl_city.1.tradycii_prykmety_zabobony_studenty.pic.4}

До речі, щодо 25 січня, то існує цілий перелік заборон і традицій. Оскільки це
свято все ж таки пов'язують зі святою Тетяною, то традиції стосуються не тільки
студентів, але й всіх вірян. Так, цього дня заведено ходити до церкви, не можна
ворожити, сваритися і дозволяти собі поганих думок. Свято потрібно відзначати
виключно миролюбно і доброзичливо. У жодному разі не можна цього дня відмовляти
в допомозі тим, хто потребує її, інакше ви й самі можете потрапити у велику
біду. Щодо студентів, то тут ціла купа традицій, адже свята Тетяна вже давно
вважається покровителькою студентів. По-перше, студенти в цей день можуть
загадати бажання, пов'язані з навчанням, в народі вірять, що якщо зробити все
за певними правилами, то воно точно збудеться. А ось і самі правила: потрібно
знайти найвищу точку у вашій місцевості, почекати, поки її освітить сонце, і
чітко вимовити своє бажання. Можливо, це допоможе закрити сесію, або ж отримати
червоний диплом. Головною прикметою цього дня є повна відмова від читання
конспектів. Особливо перед іспитом.

Загалом студентських традицій і прикмет дуже багато. Здається, що нові
виникають кожного року, адже це найбільш креативний і творчий народ, якому не
вірити не можна. Дорогі студенти, вітаю всіх вас з вашим днем і бажаю, щоб
навчання давалося легко, іспити складалися швидко, а вільного часу було все
більше. Бажаю, щоб кожен день був сповнений нових цілей, впевнених прагнень,
чудових ідей та особистих перемог. Нехай студентські роки стануть яскравими і
незабутніми, а нові знання та невтомний ентузіазм приведуть до улюбленої й
престижної професії!

\textbf{Читайте також:} \emph{Кем мариупольские студенты могут поработать во время каникул?}%
\footnote{Кем мариупольские студенты могут поработать во время каникул?, mrpl.city, 26.06.2018, %
\url{https://mrpl.city/news/view/kem-mariupolskie-studenty-mogut-porabotat-vo-vremya-letnih-kanikul}%
}
