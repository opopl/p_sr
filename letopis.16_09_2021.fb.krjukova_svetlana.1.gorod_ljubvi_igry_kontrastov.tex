% vim: keymap=russian-jcukenwin
%%beginhead 
 
%%file 16_09_2021.fb.krjukova_svetlana.1.gorod_ljubvi_igry_kontrastov
%%parent 16_09_2021
 
%%url https://www.facebook.com/kryukova/posts/10159664604763064
 
%%author_id krjukova_svetlana
%%date 
 
%%tags chelovek,gorod,kiev,rasskaz,zarisovka
%%title Город любви, большой игры и контрастов
 
%%endhead 
 
\subsection{Город любви, большой игры и контрастов}
\label{sec:16_09_2021.fb.krjukova_svetlana.1.gorod_ljubvi_igry_kontrastov}
 
\Purl{https://www.facebook.com/kryukova/posts/10159664604763064}
\ifcmt
 author_begin
   author_id krjukova_svetlana
 author_end
\fi

Дворник так неистово заметал жесткой метлой, что измельчил листву в ржавую
дорожную крошку. Опавшие терракотовые листья каштана, скрученные и хрустящие,
как японские чипсы, рассыпаются в пепел как только на них ступает крепкая нога
настойчивой киевлянки. Рыжая девушка, такая же осенняя, как и грядущая пора
года, быстро смекнула эту игру, и начала прыгать по листьям, как по игре в
классики, щёлкая с таким наслаждением, вроде она целый год ждала когда листья
из каштанов вылупятся, вызреют, протомятся на солнце и упадут прямо под ноги. 

Красивая пара, мужчина и женщина на летней террасе наблюдают за девчонкой, не
открываясь от разговора, как за белкой, за птицами и за естественным порядком
вещей, происходящем в большом зелёном городе. 

Ресторан только открыли, парковка забита элитными авто, как у казино Monte
Carlo в Монако. Шум машин создаёт иллюзию океана. 

Пара в полном восторге от вечера и друг друга. Он жадно разглядывает ее в тот
момент, когда она роется в телефоне, а она слишком часто поправляет укладку,
выдавая глубокое небезразличие. 

Официант, так не во-время, как родители вернувшиеся из дачи, кладёт на стол
меню и терпеливо ждёт, когда на него обратят внимание.

- А что бы вы заказали своей девушке, если бы пригласили ее в этот ресторан? -
кокетливо спрашивает незнакомка у официанта. 

- А я бы ее сюда не пригласил. - холодно отвечает официант. - Нам запрещено
посещать заведения, в которых мы обслуживаем. Да и не пошёл бы я. Я б тут не
расслабился. 

Они заказывают десерт и продолжают свой разговор, который очевидно ее увлекает
больше, чем его. А если быть точным: он увлечён ею, она разговором с ним. 

- ... Ведь известная фраза «Никогда ни о чем ни у кого не проси...», - гораздо
глубже чем мы все думаем. - философствует незнакомец. - Она о том, что нельзя
позволять другим втягивать тебя в игру, в результате которой ты погибнешь. 

- Или проиграешь? - спрашивает девушка.

- В твоём случае проигрыш не всегда поражение. Ты - женщина, ты можешь
проиграть мужчине. А я не могу. Поэтому когда я что-то прошу, я позволяю брать
себя в заложники, если я понятно выражаюсь.

- А если оба игрока ничего не просят?

- Делают вид. А значит, возьмут друг от друга слихвой. И в этом случае звучит
заявочка на интересную игру. 

Их диалог прерывает скрипящий звук торзмозных колодок троллейбуса номер 38 с
уродливой неоновой табличкой "Видубичі". И в этот момент авиалинии реальности
возвращают пару из виртуального Сен-Тропе в жёсткий, индустриальный Киев.

Город любви, большой игры и контрастов.

\ii{16_09_2021.fb.krjukova_svetlana.1.gorod_ljubvi_igry_kontrastov.cmt}
