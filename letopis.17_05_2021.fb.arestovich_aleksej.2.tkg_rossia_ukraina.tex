% vim: keymap=russian-jcukenwin
%%beginhead 
 
%%file 17_05_2021.fb.arestovich_aleksej.2.tkg_rossia_ukraina
%%parent 17_05_2021
 
%%url https://www.facebook.com/alexey.arestovich/posts/4334144873316226
 
%%author 
%%author_id 
%%author_url 
 
%%tags 
%%title 
 
%%endhead 
\subsection{Завтра и послезавтра - очередные заседания Трехсторонней контактной группы}
\label{sec:17_05_2021.fb.arestovich_aleksej.2.tkg_rossia_ukraina}
\Purl{https://www.facebook.com/alexey.arestovich/posts/4334144873316226}

- Завтра и послезавтра - очередные заседания Трехсторонней контактной группы.

Нормандский формат является старшим для Минского и последний реализует решения первого. 
Политический советник Украины в Н4 Андрей Ермак на этой неделе направил письма своим визави из Германии, РФ и Франции, с предложением провести внеочередную встречу.

\ifcmt
  pic https://scontent-iad3-2.xx.fbcdn.net/v/t1.6435-9/187038765_4334144616649585_1157208273377259151_n.jpg?_nc_cat=107&ccb=1-3&_nc_sid=8bfeb9&_nc_ohc=8RBKboz15ZIAX8wXzgi&_nc_ht=scontent-iad3-2.xx&oh=26cf3401c87dabb2872da78cc933d0e5&oe=60C6952D
\fi

Главные темы:
- возвращение к строгому соблюдению режима полного прекращения огня на фронте (потому, что его нарушения практически ежедневные),
- разблокирование мирного процесса (потому, что он в стагнации),
- организации новой встречи лидеров Нормандского формата (потому, что не реализуются многие решения Парижского саммита 2019 года и с того времени, слишком много всего изменилось в мире, начиная с усиления группировки российских войск на наших границах). 
До этого украинская дипломатия несколько раз обращалась с предложением провести внеочередное заседание либо Подгруппы по безопасности либо общего заседания ТКГ, учитывая обстановку на фронте и количество смертей, как военных так и гражданских, но российское руководство не пошло на это, демонстрируя, кто в действительности заинтересован в прекращении огня, а кто - нет. 
Политика Кремля ясна: они вложили миллиарды долларов, тысячи своих и чужих жизней, нарвались на санкции и Ледяную войну с Западом для того, чтобы легализовать т.н. «молодые республики», и пока не собираются отказываться от этого направления, потому, по по их замыслу признание «…республик» и наделение из «особым статусом» - это один из ведущих механизмов разрушения украинской государственности и постановки под контроль украинской внутренней и внешней политики.
Мы никогда не пойдём на такое признание, потому, что это означало бы:
- вывод РФ из статуса государства-агрессора и участника конфликта, а значит поощрения ее (и других подобных ей режимов) к дальнейшей безнаказанности, разрушению существующей системы мировой безопасности,
- начало медленного крушения украинской государственности, добровольное ограничение собственного суверенитета на своей же территории.
Стороной переговоров в ТКГ является Российская Федерация.
Ее же руководство предпринимает титанические усилия, для того, чтобы изобразить из себя «…посредников во  внутриукраинском конфликте».
Для этого, они весь комплекс вопросов (гуманитарных, безопасности, экономических, политических) последовательно привязывают к вопросу признания Украиной т.н. «ЛДНР».
Приходится вновь и вновь напоминать им истинное положение вещей.
Из-за данной позиции РФ, процесс консультаций находится в стагнации. 
Не решаются не только стратегические вопросы мирного урегулирования, как такового, но и тактические - открытие КПВВ, возобновления социально-экономических связей с нашими гражданами на оккупированных территориях, выплаты пенсий, обмена пленными, но и, в первую очередь, обеспечение полной безопасности: прекращения огня, отвода тяжёлых вооружений, вывода российских войск, только после выполнения каковой и можно всерьёз говорить о дальнейших этапах урегулирования.
Заставить кремлевское руководство на данном этапе изменить такую свою позицию не удаётся пока даже коллективному давлению Запада. 
Политика, которую мы проводим на данном этапе:
- удержать РФ от открытого применения силы, показав бесперспективность данного пути,
- сдвинуть с места процесс мирного урегулирования.
Наши базовые условия:
- полное прекращение огня,
- партнёром по переговорам является РФ и никаких иных вариантов быть не может.
Эта позиция Украины находит понимание и поддержку со стороны наших международных партнёров.
Многое будет зависеть от встречи Байдена и Путина, «украинская часть» повестки которой согласуется администрациями президентов США и Украины.
Прорывов пока ожидать не приходится.
Наивно ждать от Украинской делегации для участия ТКГ прорывов там, где пока не справляется коллективный Запад.
Тем не менее, мы работаем: с малым КПД, в периоде стагнации мирного процесса, отыскивая пути и способы даже в этих условиях добиться решения конкретных вопросов - от прекращения огня до открытия КПВВ, возвращения пленных и выплаты пенсий.
