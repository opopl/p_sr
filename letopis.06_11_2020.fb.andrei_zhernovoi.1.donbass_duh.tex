% vim: keymap=russian-jcukenwin
%%beginhead 
 
%%file 06_11_2020.fb.andrei_zhernovoi.1.donbass_duh
%%parent 06_11_2020
 
%%url https://www.facebook.com/groups/LNRGUMO/permalink/3180368978741321/
%%author 
%%tags 
%%title 
 
%%endhead 

\subsection{Донбасс показал России, что такое истинный русский дух - Сергей Маховиков}
\label{sec:06_11_2020.fb.andrei_zhernovoi.1.donbass_duh}
\Purl{https://www.facebook.com/groups/LNRGUMO/permalink/3180368978741321/}
\Pauthor{Маховиков, Сергей}

Когда началась война, я приехал сюда одним из первых, а возможно, вообще
первым. И то, что здесь происходило, что я видел, было страшно. Даже погода
была под стать обстановке --- было очень хмуро. Тогда только недавно попали
снаряды в Донецкий краеведческий музей, мы туда ездили…

Я сегодня утром въезжал в Донецк --- какой же красивый город! Солнце восходит,
подсвечивая «Донбасс Арену» --- мы даже остановились и смотрели на это, и слёзы
выступили у нас на глазах. Как можно такую красоту бомбить?! Я уже не говорю о
том, что здесь дети, старики…

ЛЮДИ!

Я всегда с восторгом сюда приезжаю. Оттого, что Донбасс --- непокорённый. Он не
выдумывает себе ни новую национальность, ни новый язык; не предаёт своих
стариков, свою историю --- это самое главное. Я обожаю Донбасс за это. Потому что
он всему миру показал, и даже России, что такое истинный русский дух…

Стилистика и пунктуация автора сохранена

Сергей Маховиков актёр, режиссёр, исполнитель

В проекте «Как я встретил начало войны» каждый житель Донбасса может
рассказать, как именно изменила война его жизнь, что произошло в его судьбе с
началом боевых действий в Донбассе. Необходимо, чтобы весь мир узнал о тех
тревожных днях 2014 года, когда началась гражданская война.

Вы можете отправить свою историю нам на почту: pismo@donbasstoday.ru
