% vim: keymap=russian-jcukenwin
%%beginhead 
 
%%file 27_10_2022.stz.news.ua.donbas24.1.chy_ljubyv_marko_kropyvnyckyj_priazovja_stvorennja_ukr_realistic_teatr.txt
%%parent 27_10_2022.stz.news.ua.donbas24.1.chy_ljubyv_marko_kropyvnyckyj_priazovja_stvorennja_ukr_realistic_teatr
 
%%url 
 
%%author_id 
%%date 
 
%%tags 
%%title 
 
%%endhead 

Чи любив Марко Кропивницький Приазов'я: до дня створення українського
реалістичного театру

Трупа Марка Кропивницького досить часто виступала як в Маріуполі, так і в
навколишніх селищах

27 жовтня 1882 року в Єлисаветградському театрі (нині Кропивницький) відбулась
перша вистава українського театру з тріумфальної постановки «Наталка Полтавка»
з Марією Заньковецькою у головній ролі. 27 жовтня й досі вважається Днем
народження першого професійного національного реалістичного театру на
українських землях. Згодом театральна трупа і її філії об'їздили майже всю
територією Південної України з виставами. У 1885 р. трупа розпалася на дві
частини: однією керував Марко Кропивницький, другою — Михайло Старицький. За
їхнім прикладом пішли інші і вже через кілька років на Україні діяло понад 30
мандрівних театрів. Виявляється, видатний український письменник, драматург,
театральний режисер і актор Марко Кропивницький полюбляв приїжджати до
Маріуполя та його околиць, адже приазовські глядачі обожнювали вистави
українського реалістичного театру.

Читайте також: «Театроманія» продовжує розвивати та підтримувати культуру
Маріуполя

Коли Марко Кропивницький почав приїжджати до Приазов'я?

Наприкінці XIX століття особливе місце в театральному житті Маріуполя належало
саме українському музично-драматичному театру. Гастролі великих майстрів
української сцени Марка Кропивницького, Івана Карпенка-Карого, Панаса
Саксаганського, Марії Садовської-Барілотті дали можливість містянам
ознайомитися з національною театральною драматургією. Цікаво, що «Товариство
російсько-малоросійських артистів» під керівництвом П. Саксаганського першим
містом для своїх виступів обрало Катеринослав, а другим — Маріуполь. Це
обумовлювалося високим рівнем театральної культури Маріуполя та вдячністю
глядачів. Найбільшу популярність серед маріупольців мали вистави Марка
Кропивницького. У Музеї театрального, музичного та кіномистецтва України
зберігається годинник із лаконічним написом: «Кропивницькому від маріупольців
29.09.1889 р.». Цей подарунок є матеріальним свідченням того, що в 1889 р. М.
Кропивницький вперше виступав у Маріуполі.

Читайте також: У Києві покажуть виставу про Маріуполь

Також того ж 1889 року Марко Лукич взяв до своєї трупи Любов Ліницьку, яка була
в маріупольській трупі Василя Шаповалова на перших ролях і мала бенефіс.
Найбільший успіх вона мала у п'єсах: «Наталка Полтавка», «Назар Стодоля»,
«Невільник», «Шельменко-денщик», «Сватання на Гончарівці». У неї навіть був
власний сценічний гардероб. До трупи одного із засновників професійного театру
приєднався і чоловік Ліницької Іван Загорський, який теж починав свій
акторський шлях у Маріуполі. При цьому, якщо його Кропивницький взяв до своєї
трупи одразу, то Любов Ліницьку — після випробувального терміну. Отже, деякі
маріупольські актори були запрошені в трупу М. Кропивницького, творця
реалістичного українського театру, що слугувало найкращою атестацією
маріупольському антрепренеру Василю Шаповалову як режисеру і організатору
театральної справи.

Відомо, що у Маріуполі та його околицях трупа М. Кропивницького гастролювала й
у 1891 році впродовж декількох днів — з 9 по 14 серпня. Збереглися відомості
лише про одну виставу «Зайдиголова», в якій М. Кропивницький — автор п'єси —
виконав роль Захарки Лободи. Разом з ним виступали такі видатні актори, як
Ганна Петрівна Затиркевич-Карпинська і Леонід Якович Манько.

Читайте також: Які театральні проєкти та культурні заходи, присвячені
Маріуполю, реалізуються закордонними митцями

16 липня 1892 року до Маріуполя знову приїхала трупа М. Кропивницького, яка
представила концерт-виставу за участю Л. Ліницької на користь чоловічого і
жіночого хору та танцюристів. Автор кореспонденції не випадково наголосив на
участі у концерті-виставі Л. Ліницької: маріупольці пам'ятали Любов Павлівну і
пишалися тим, що свій блискучий шлях актриси вона розпочала саме в їхньому
місті.

Чому вистави Марка Кропивницького викликали справжній фурор?

Влітку 1908 року антрепренер М. Кононенко запросив М. Кропивницького приїхати в
Маріуполь. У міській газеті «Маріупольське життя» повідомлялося, що Марко
Кропивницький приїздить на гастролі і що він виступить у ролі Виборного
(«Наталка Полтавка»), Карася («Запорожець за Дунаєм»), а у виставі «Доки сонце
зійде — роса очі виїсть» зіграє одразу дві ролі — колишнього селянина-кріпака
Максима Хвортуни й поміщика Воронова. Вистави мали проходити у приміщенні
театру братів Яковенків і всі квитки були негайно розпродані. Неперевершений
майстер сцени цілковито виправдав очікування публіки, яка щиро винагороджувала
гру Марка Кропивницького бурхливими оплесками, що переходили в овації. Вистави
відбувалися при переповнених залах. Через репертуар «батько української сцени»
показував глядачеві реалістичну правду дійсності, розкривав соціальні вади і
протиріччя тогочасного життя, чим завоював у Маріуполі та загалом Приазов'ї
безліч прихильників. Крім запланованих трьох виступів, актор погодився взяти
участь ще в двох виставах. Він дуже любив Маріуполь та особливу атмосферу, яка
панувала в залі, коли він виступав, адже глядачі його обожнювали.

Читайте також: «Тримаємось разом» — маріупольський театр «Conception»
представить нову виставу (ФОТО)

На закінчення гастролей 19 серпня 1908 року пройшов блискучий бенефіс Марка
Кропивницького у виставі «Глитай, або ж Павук».

Не минуло й двох років після цієї визначної події в культурному житті міста, як
у Маріуполь прийшла трагічна звістка про смерть корифея української сцени. У
той час у місті працювала трупа Т. Левченка та А. Матусина. Вона організувала в
театрі братів Яковенків «Вечір пам'яті Марка Кропивницького». По закінченню
програми була показана вистава за п'єсою М. Кропивницького «Дай серцю волю,
заведе в неволю». Це свідчить про особливе ставлення не тільки глядачів, але й
театральної трупи міста до Марка Кропивницького та високу оцінку його внеску в
розвиток театрального мистецтва Приазов'я.

Раніше Донбас24 розповідав, як в Маріуполі проходив фестиваль «Театральна
брама».

Ще більше новин та найактуальніша інформація про Донецьку та Луганську області
в нашому телеграм-каналі Донбас24.

ФОТО: з відкритих джерел.
