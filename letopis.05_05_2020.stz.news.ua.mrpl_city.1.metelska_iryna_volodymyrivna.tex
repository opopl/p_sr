% vim: keymap=russian-jcukenwin
%%beginhead 
 
%%file 05_05_2020.stz.news.ua.mrpl_city.1.metelska_iryna_volodymyrivna
%%parent 05_05_2020
 
%%url https://mrpl.city/blogs/view/metelska-irina-volodimirivna-tsinujte-svoe-zhittya-do-mizhnarodnogo-dnya-akusherki
 
%%author_id demidko_olga.mariupol,news.ua.mrpl_city
%%date 
 
%%tags 
%%title Метельська Ірина Володимирівна: "Цінуйте своє життя!" (До Міжнародного дня акушерки)
 
%%endhead 
 
\subsection{Метельська Ірина Володимирівна: \enquote{Цінуйте своє життя!} (До Міжнародного дня акушерки)}
\label{sec:05_05_2020.stz.news.ua.mrpl_city.1.metelska_iryna_volodymyrivna}
 
\Purl{https://mrpl.city/blogs/view/metelska-irina-volodimirivna-tsinujte-svoe-zhittya-do-mizhnarodnogo-dnya-akusherki}
\ifcmt
 author_begin
   author_id demidko_olga.mariupol,news.ua.mrpl_city
 author_end
\fi

Сьогодні, 5 травня, більше 50 країн світу відзначають Міжнародний день
акушерки. Свято веде свою історію ще з 1987 року, коли в Нідерландах на
конференції Міжнародної конфедерації акушерок виникла ідея його проведення.
Втім регулярно відзначати дату стали тільки з 1992 року. Саме від
професіоналізму акушерок, їх знань і досвіду часто залежить здоров'я кожної
породіллі-матері і кожної дитини, а іноді навіть їхнє життя. З нагоди цього
важливого свята я вирішила познайомити вас, дорогі читачі, з головною акушеркою
Перинатального центру № 1 – \emph{\textbf{Метельською Іриною Володимирівною}}, яка є прикладом
справжньої відданості своїй професії.

\ii{05_05_2020.stz.news.ua.mrpl_city.1.metelska_iryna_volodymyrivna.pic.1}

Ірина Володимирівна народилася 12 травня (до речі, в \emph{Міжнародний день медичної
сестри}) в Маріуполі в сім'ї водія та торговельного робітника. З дитинства Іра
росла вихованою і старанною дівчинкою. Після закінчення школи одразу ж вирішила
вступати до медичного інституту, але з першого разу не вийшло, тому пішла
працювати до маріупольського пологового будинку № 1 (зараз – Перинатальний
центр № 1). Спочатку її взяли молодшою медичною сестрою. Ще тоді вона почала
захоплюватися роботою акушерок, їхньою самовідданістю та сумлінним виконанням
обов'язків. Багато нюансів професії для дівчини відкрила \emph{Шиліна Світлана
Василівна}, яка працювала у пологовому будинку з його відкриття. Однак
справжньою вчителькою і своєю наставницею Ірина вважає акушерку \emph{Гурьєву Асю
Вікторівну}.

\ii{05_05_2020.stz.news.ua.mrpl_city.1.metelska_iryna_volodymyrivna.pic.2}

У 1989 році Іра вступила до Бердянського медичного училища, після закінчення
якого (у 1991 році) повернулася до пологового будинку № 1, де почала працювати
акушеркою. З грудня 2018 року Ірина Володимиріна займає посаду головної
акушерки. Вже довгих 29 років Ірина Володимирівна працює в Перинатальному
центрі, який став для неї другим домом. В професії її надихає щоденна атмосфера
свята. Вона наголошує, що в цій надзвичайній енергії не втомлюєшся купатися.
Інколи навіть не віриться, що через декілька хвилин на одну людину стає більше.
І щоразу це неймовірне дійство захоплює і вражає. Головна акушерка підкреслює,
що з кожними пологами хочеться ще більше допомагати і підтримувати жінок,
робити пологи легшими і швидшими, адже в ці миті відчуваєш себе корисними та
потрібними. В пологовому будинку випадкових людей не буває. Ніколи не знаєш, що
відбудеться через наступні декілька хвилин, тому повинен бути готовий до
будь-якої ситуації. І тільки злагоджена робота команди може посприяти успішному
результату. Ірина Володимирівна живе своєю роботою, адже інакше просто
неможливо. Або жити цією професією, або краще зовсім не обирати її. Іншого бути
не може, вважає наша героїня. Також вона зазначає, що всі у Перинатальному
центрі звикли бути однією родиною: підтримувати у всьому одне одного,
допомагати. Разом співробітники утворили таку міцну і унікальну силу, яка
здатна витримати і важкий графік, і будь-які стреси. Ірина Володимирівна радіє,
що їй пощастило працювати в такому колективі, який складається з дійсно хороших
Людей, що люблять свою роботу.

%\ii{05_05_2020.stz.news.ua.mrpl_city.1.metelska_iryna_volodymyrivna.pic.3}

Своє життя жінка посвятила не тільки роботі, але й донечці. І хоча вона не
стала медиком, як мама, але у них дуже теплі і близькі стосунки. На випускному,
коли доньці вручали золоту медаль, вона повісила медаль мамі, сказавши, що це і
її заслуга. А на своєму весіллі заспівала для мами пісню, підкресливши, що у
них одне серцебиття на двох. Маріуполь Ірина Володимирівна дуже любить,
найбільше її захоплюють прогулянки до моря. Загалом жінка веде дуже активний
образ життя. Любить свою кішку, а найбільшим хобі вважає вирощування і догляд
за квітами, які її і заспокоюють, і просто радують око.

\ii{05_05_2020.stz.news.ua.mrpl_city.1.metelska_iryna_volodymyrivna.pic.3_4}

Цього року Ірина Володимирівна (вже через декілька днів) святкуватиме 50-ти
річчя. Оскільки вітати з Днем Народження заздалегідь не можна, хочеться
привітати з Міжнародним днем акушерки і висловити свою повагу, захоплення і
щиру вдячність за невтомну і щоденну благородну працю, яка допомагає з’явитися
новій людині на світ.

\begingroup
\em
\textbf{Улюблені книги:} \enquote{Люди в білих халатах} В Пузікова та Д. Сергєєва, \enquote{Одіссея капітана Блада} Рафаеля Сабатіні.

\textbf{Улюблений фільм:} фільми Леоніда Гайдая.

\textbf{Курйозний випадок:}

\begin{quote}
Запам'яталися пологи, коли дуже швидко змінювалося серцебиття дитини, команда
вже не знала, як бути. А коли народилося немовля, то всі побачили, що дитина
просто гралася з власною пуповиною, оскільки вона була намотана у нього на
пальчиках.
\end{quote}

\textbf{Порада маріупольцям:}

\begin{quote}
\enquote{Цінуйте своє життя – це найважливіше. Все інше – дрібниці. Всі вони зникнуть.
Головне – цінувати себе, своє здоров'я та взаємовідносини з близькими і
дорогими людьми. Потрібно дорожити одне одним, вміти слухати і чути, любити і
цінувати тих хто поруч з нами...}.
\end{quote}
	
\endgroup
