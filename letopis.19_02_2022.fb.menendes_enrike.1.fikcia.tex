% vim: keymap=russian-jcukenwin
%%beginhead 
 
%%file 19_02_2022.fb.menendes_enrike.1.fikcia
%%parent 19_02_2022
 
%%url https://www.facebook.com/e.menendes/posts/7052387408136958
 
%%author_id menendes_enrike
%%date 
 
%%tags 2022,rossia,ugroza,ukraina,vojna
%%title Когда будут выбирать ключевое слово 2022 года, я предложу слово «фикция»
 
%%endhead 
 
\subsection{Когда будут выбирать ключевое слово 2022 года, я предложу слово «фикция»}
\label{sec:19_02_2022.fb.menendes_enrike.1.fikcia}
 
\Purl{https://www.facebook.com/e.menendes/posts/7052387408136958}
\ifcmt
 author_begin
   author_id menendes_enrike
 author_end
\fi

Когда будут выбирать ключевое слово 2022 года, я предложу слово «фикция».

Сначала фиктивное вторжение России. Вторжение, которое так ждали, громко
анонсировали, несколько раз переносили, всех вогнало в стресс, а оно не
состоялось.

Потом фиктивная война. Нет обострение на фронте реальное. Реальные обстрелы,
пострадавшие люди и здания. Но ожидания большой войны значительно выше, чем то,
что происходит на деле. А на деле сезонное обострение – сколько мы их уже
видели? Как мы от них устали! Но что делать?

Потом фиктивные переговоры. Серьёзные люди надувают щёки, делают вид что решают
проблему, а на самом деле просто её бесконечно мусолят без какого-либо
продвижения вперёд. Без какой-либо ответственности перед реальными людьми,
которые являются заложниками ситуации. Вечное "это не мы, это они". Тычут друг
в друга пальцами, как дети.

И фоном постоянно идёт бла-бла-бла, просто бесконечно. На ток-шоу и
пресс-конференциях, на брифингах и круглых столах, в теле-обращениях и
публикациях в СМИ. От него уже пухнет голова, сил нет терпеть этот поток бреда.
Бестолковая болтовня, которая не несёт никакого смысла. Просто набор слов от
людей, которые обязаны наполнять слова смыслом, а потом и действиями. Это их
работа, которую они не выполняют.

И в центре этого симулякра мы с вами. Обычные люди, которые бояться, паникуют,
ждут новостей и надеются на лучшее, а кто-то может и потерял надежду, отчаялся.
Но в любом случае именно мы страдаем от фикций, которыми занимаются наши
политики.
