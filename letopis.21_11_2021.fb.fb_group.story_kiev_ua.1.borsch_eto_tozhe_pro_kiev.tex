% vim: keymap=russian-jcukenwin
%%beginhead 
 
%%file 21_11_2021.fb.fb_group.story_kiev_ua.1.borsch_eto_tozhe_pro_kiev
%%parent 21_11_2021
 
%%url https://www.facebook.com/groups/story.kiev.ua/posts/1802223643307762
 
%%author_id fb_group.story_kiev_ua,zolotushkin_anatolij.hajfa
%%date 
 
%%tags borsch,eda,kiev,ukraina
%%title Борщ - это тоже про Киев
 
%%endhead 
 
\subsection{Борщ - это тоже про Киев}
\label{sec:21_11_2021.fb.fb_group.story_kiev_ua.1.borsch_eto_tozhe_pro_kiev}
 
\Purl{https://www.facebook.com/groups/story.kiev.ua/posts/1802223643307762/}
\ifcmt
 author_begin
   author_id fb_group.story_kiev_ua,zolotushkin_anatolij.hajfa
 author_end
\fi

Борщ - это тоже про Киев.

Недавно все прогрессивное и не очень человечество отмечало день Борща, а я, как
всегда, проспал. Судорожно догоняю.

\ii{21_11_2021.fb.fb_group.story_kiev_ua.1.borsch_eto_tozhe_pro_kiev.pic.1}

Как поступает дилетант?

Он открывает интернет на букву Б и идёт закупать всякие там петрушки-морковки и
загадочную сахарную косточку. (Ни разу не видел костей с сахаром)

Опытный борщеед самодостаточен - он знает, что всё необходимое таится в недрах
холодильника и с нетерпением ждёт своего часа.

Найденное сортируется, моется и подвергается обрезанию.

А затем немедленно  жарить, парить, мешать,  нюхать, солить, перчить по вкусу и
варить до готовности.

Потом пару дней настаивать, подогреть, добавить сметаны и поедать большой
ложкой, желательно деревянной.

P.S. Не забудьте посыпать зеленью и натереть хлеб чесноком, чтобы окружающие
начинали завидовать от одного исходящего от вас запаха.

P. P. S. По этому рецепту можно приготовить практически все, от супа харчо до
невиданного в наших краях буйабеса.

Картинка из интернета исключительно для выделения слюны.

\ii{21_11_2021.fb.fb_group.story_kiev_ua.1.borsch_eto_tozhe_pro_kiev.cmt}
