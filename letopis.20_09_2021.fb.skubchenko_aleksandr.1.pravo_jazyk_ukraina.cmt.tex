% vim: keymap=russian-jcukenwin
%%beginhead 
 
%%file 20_09_2021.fb.skubchenko_aleksandr.1.pravo_jazyk_ukraina.cmt
%%parent 20_09_2021.fb.skubchenko_aleksandr.1.pravo_jazyk_ukraina
 
%%url 
 
%%author_id 
%%date 
 
%%tags 
%%title 
 
%%endhead 
\subsubsection{Коментарі}

\begin{itemize} % {
\iusr{Dima Privalov}

Я недавно разговаривал с каталонцами. Никто в мире не подвергает сомнению, что
каталанский язык запрещался в Испании, во времена Франко. Это общеизвестный
исторический факт. Так вот на вопрос, что включали в себя эти запреты, они
отвечали: запрет на образование, и запрет на делопроизводство. И все! На улице
никто никому не запрещал говорить на каталан. И понятно почему, просто
запретить именно это невозможно. Наверняка хотелось бы, как Франко, так и
украинским националистам и украинскому государству. Но такой запрет им не под
силу, во всяком случае пока.

Также я спросил у каталонцев про то был ли какой-то запрет, или какие-то
ограничения для официантов, продавцев и других работников сферы услуг, говорить
на каталан? Нет, такого запрета там не было! Они были очень удивлены, что такой
запрет есть у нас.

Следующим я хотел бы украинизаторам напомнить, что по их же словам, валуевский
указ был одним из самых страшных ударов по украинскому языку. Так вот в нем не
было пункта, о запрете говорить на украинском.

\begin{itemize} % {
\iusr{Александр Скубченко}
\textbf{Dima Privalov} так Зеленский по сути — последователь нацистов, раз идёт их методами.
\end{itemize} % }

\iusr{Александр Денисов}

Зеленский беспринципный человек без каких-либо моральных устоев. Слабо
образованный и достаточно недалекий. Предвыборные обязательства ему писали
другие. Они не являются его убеждениями. Он дал - он и взял.

\begin{itemize} % {
\iusr{Александр Скубченко}
\textbf{Александр Денисов} да, мир, тарифы, язык. Только на словах герой
\end{itemize} % }

\iusr{Тая Дунаева}

Вы не слушали внимательно его(на нашем?)
После слов о конституции и тд ,он добавил дома в частном порядке )
Спасибо благодетель - разрешил дома под одеялом)

\begin{itemize} % {
\iusr{Александр Скубченко}
\textbf{Тая Дунаева} ааа. Я не смотрю, время жалко на тв. Я цитату увидел.

\iusr{Тая Дунаева}
\textbf{Александр Скубченко} причем сам переходил время от времени на русский и суржик.даже спрашивал как слово на украинском)

\iusr{Александр Скубченко}
\textbf{Тая Дунаева} страна дураков. Страна для нацистов

\iusr{Тая Дунаева}
\textbf{Александр Скубченко} будет видео - посмотрите.
Лучше б он молчал.
Я не я и хата не моя

\iusr{Александр Скубченко}
\textbf{Тая Дунаева} политика Зеленского такая тупая, что её невозможно иначе оправдать. Только вот так, криво

\iusr{Тая Дунаева}
\textbf{Александр Скубченко} нет никакой политики.
Ситуативно.вбрасывают тему - если не пошла- то не мы
\end{itemize} % }

\iusr{Вячеслав Беленький}

вы чуть ошибаетесь.

В сфере услуг именно Запрещено встречать клиента на русском языке, т.е.
"говорить" до персонального "разрешения" от клиента и только по его инициативе.

А Арестович - банальный языковой коллаборант.

Ему бы зарплату, а платят только за работу на русофобскую власть. Про честь
речи нет, когда кушать хочется. Есть такие люди.

\begin{itemize} % {
\iusr{Александр Скубченко}
\textbf{Вячеслав Беленький} та ладно. Травля сферы обслуживания за русский язык как бы и нет, да? И увольнений за русский язык тоже нет. Это всё Путин придумал

\iusr{Вячеслав Беленький}
\textbf{Александр Скубченко} так а я о чем?
Вы написали "никто не в состоянии запретить "говорить" на родном языке", вот и отвечаю - в состоянии. Уже запретили.

\iusr{Marina Glyva}
\textbf{Вячеслав Беленький} Встречать клиента? Да продавцам и кассирам вообще запрещено по-русски с клиентами говорить, сама видела в АТБ, как менеджер орала на кассира и угрожала уволить за то, что она по-русски с покупателем говорила. Но вот парадокс, оказало... Ещё

\iusr{Тая Дунаева}
\textbf{Александр Скубченко} нене.
Скабеева ж теперь все придумывает)

\iusr{Вячеслав Беленький}
\textbf{Marina Glyva} 

те, кто реально стоит за проводимой языковой сегрегацией, а это не дурачок
зеленский и не порошенко - хорошо владеют психологией.

Система требует только задать направление страха, остальное сами холопы сделают с десятикратным запасом.


\iusr{Vadim Slim}
\textbf{Александр Скубченко} полным ходом идет

\iusr{Marina Glyva}
\textbf{Вячеслав Беленький} 

Скорее наоборот, пофигизм будет, а не страх, людям надоело терпеть принуждения,
они будут говорить на родном русском языке. Украинизация ни у кого радости не
вызвала в русскоязычном регионе, поэтому все продолжают говорить на родном
языке, а только документы заполнять на украинском.

\iusr{Вячеслав Беленький}
\textbf{Marina Glyva} 

не будет пофигизма, когда от твоего подчинения зависит хлеб на столе семьи
завтра.

Перфоменсы с мелочью в кассира - это элемент программы по коллективному
запугиванию. Дальше начинает работать средний менеджмент "абы чего не вышло"

Авторы знают, с каким материалом работают.

Печаль в том, что нет мотивированного ядра сопротивления процессу.

Тренд легко ломается десятком ярких флешмобов или прямых встречных атак.

\iusr{Marina Glyva}
\textbf{Вячеслав Беленький} Я знаю, о чем пишу, вижу, как украинизация проходит в русскоязычном городе. По улице идешь, едешь в транспорте, везде все по-русски говорят.

\iusr{Вячеслав Беленький}
\textbf{Marina Glyva} 

зайдите в школы, посмотрите меседжеры школьников - прозреете. Медленно, но
деградация прогрессирует.

Раньше хоть начальная школа оставалась или предметы языка или литературы были.

Самое подлое кроется именно в обработке детей, работа на перспективу, ведущая к
тотальному незнанию русского.

Причем знание самого украинского даже не ставится в основу - пофиг. Главное -
выбить русский. даже вместе с общим уровнем развития поколения.

За что лично в реальности автору по физии бы зарядил, знай кто и встретив.


\iusr{Igor Markov}
\textbf{Marina Glyva} странно, в Славянске не было с этим проблем.

\iusr{Marina Glyva}
\textbf{Вячеслав Беленький} 

Вы понимаете, когда все говорят вокруг по-русски и против насильственной
украинизации, никто никого не принуждает говорить на украинском, переступая
порог школы, все уже по-русски разговаривают. Я даже когда в Киев в официальные
инстанции звоню и говорю, что мне по-русски легче общаться, со мной тоже
говорят по-русски. Украина - страна, где законы не работают, Конституция не
работает, но и закон об украинизации тоже не работает - все законы! Ну и что
что в школе обучение на украинском, все равно дети говорят на русском языке.
Украинский - просто рабочий язык, вот так. Все равно будет двуязычие.

\end{itemize} % }

\iusr{Дана Крон}
Смешной этот Карамелька.Интересно-он умеет шевелить ушами?

\begin{itemize} % {
\iusr{Александр Скубченко}
\textbf{Дана Крон} я ж не знаю
\end{itemize} % }

\iusr{Ира Гаврилова}
А что Даша Заривная? Молчит?

\iusr{Marina Glyva}

Да вот интересно, кто внушил русофобам, что в стране не могут сосуществовать
несколько языков, а только один и принудительно. Самое печальное, когда друзья
оказываются русофобами и перестают общаться @igg{fbicon.cry} всю жизнь по-русски говорили и
вдруг поняли, как сильно ошибались, и перешли только на украинский. Вот это
удивляет. Этот псевдопатриотизм, когда люди предают свое родное.

\begin{itemize} % {
\iusr{Александр Скубченко}
\textbf{Marina Glyva} печаль, да

\iusr{Yuriy Storchak}
Русофобы делают хуже сами себе.
\end{itemize} % }

\iusr{Елена Елена}

Я жду, когда президенту на прессухе зададут вопрос на крымско-татарском. Жду
когда их всех заставят его выучить. Для разнообразия

\begin{itemize} % {
\iusr{Александр Скубченко}
\textbf{Елена Елена} )))
\end{itemize} % }

\iusr{Игорь Гомольский}
Кто-то просто обязан сказать этому клоуну, что у нас есть доступ к украинским СМИ)

\begin{itemize} % {
\iusr{Александр Скубченко}
\textbf{Игорь Гомольский}, только ДОМ, а там врать не будут))

\iusr{Тая Дунаева}
\textbf{Игорь Гомольский} да все они знают.
\end{itemize} % }

\iusr{Сергей Васильефф}
Даа..жаль я проголосовал за этого зелёного ублюдка... Гавно гавном....

\iusr{Эмилия Махсма}
Брешет как пес Сирко этот арестович

\iusr{Анатолий Лернер}
Кого вообще волнует, кто что говорит? В Конституции есть упоминание о равноправии языков? Нет. А если так, то и говорить не о чем.

\iusr{Владимир Быстряков}

Вам ли не понять, что целью этих господ-грантоедов была не поддержка "укр.мовы,
а СТРАВЛИВАНИЕ ЛЮДЕЙ в нашем обществе, с последующим распадом государства?!

\begin{itemize} % {
\iusr{Александр Скубченко}
\textbf{Владимир Быстряков} так и есть
\end{itemize} % }

\iusr{Екатерина Меренцова}

Да все они понимают. Мы все ещё думаем, что можно убедить фактами, примерами
стран Европы, вот же у них как, а у нас... Им плевать. Мир Оруэлла

\end{itemize} % }
