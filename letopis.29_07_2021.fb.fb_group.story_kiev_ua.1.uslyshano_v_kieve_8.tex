% vim: keymap=russian-jcukenwin
%%beginhead 
 
%%file 29_07_2021.fb.fb_group.story_kiev_ua.1.uslyshano_v_kieve_8
%%parent 29_07_2021
 
%%url https://www.facebook.com/groups/story.kiev.ua/posts/1717747081755419/
 
%%author Киевские Истории
%%author_id fb_group.story_kiev_ua
%%author_url 
 
%%tags facebook,gorod,istoria,kiev,ukraina
%%title Услышано в Киеве - ЧАСТЬ ВОСЬМАЯ – итоговая.
 
%%endhead 
 
\subsection{Услышано в Киеве - ЧАСТЬ ВОСЬМАЯ – итоговая.}
\label{sec:29_07_2021.fb.fb_group.story_kiev_ua.1.uslyshano_v_kieve_8}
 
\Purl{https://www.facebook.com/groups/story.kiev.ua/posts/1717747081755419/}
\ifcmt
 author_begin
   author_id fb_group.story_kiev_ua
 author_end
\fi

ЧАСТЬ ВОСЬМАЯ – итоговая.

Комменты для предыдущих частей поста будут закрыты.

За 10 дней обсуждений, тема «Услышано в Киеве» получила три с половиной тысячи
комментов и собрала 230 фраз и выражений.

Всем спасибо! Несомненный лидер по количеству поданных фраз – Анна Шустерман.

Комменты этой части не закрываю. Жалко закрывать такую поляну для воспоминаний,
но и сил окучивать ее нет. Публикую последние вставки і скажу, що так і було… 

Друзья, я серьезно предлагаю продолжить тему, но уже другому автору. У кого
есть время и желание - только скажите.

А темы есть "жирные" - особенно суржики. И украинский и идиш и польский в
проекции на великоимперский. 

Ну, кто возьмется одеть курточку и пойти в Дарницу?

А оттаке ми з вами наколядували:

\begin{itemize}
\item  «А гройсе пуриц» (идиш) – дословно большой помещик, о человеке с высоким самомнением.
\item  «Аз ох-н-вей» (идиш/иврит) - когда (хочется сказать) «ох!» и «вей!». Первое междометие употребляется не от хорошей жизни, второе же слово означает «горе». 
\item  «А Менделе герой мит бритва» (идиш/рус.) - о человеке, который храбрый только с ножом в руке. Евбазовское.
\item  «А он крутой шнарант» – от фамилии хозяина дома №63 на углу Межигорской и Еленовской. Он был знаменит тем, что сдавал временную жилплощадь торгашам. Там заключались сделки насчет купли-продажи оптовых партий товаров и останавливались разношерстные аферисты. Именно поэтому впоследствии слово «шнарант» стало олицетворять авантюриста, прохиндея, нечистого на руку дельца. Когда дом Шнаранта сносили, в стенах нашли тайники с деньгами, расписками, а рабочие обнаружили в подвале шкатулку с золотыми монетами.
\item  «А пошёл ты в Печерскую Лавру пешком - туда и обратно» - ласковый посыл.
\item  «Артисты с погорелого театра» - неудачники.
\item  «А Сара хочет негра» – в ответ на «Я хочу…».
\item  «А там вся аристократия липовая» — ироническое наи¬менование владельцев усадеб на Липках. В стари¬ну это выражение не имело того саркастическо¬го смысла, который вкладывается в него теперь.
\item  «Ах ты старая кошолка» (от кошеля) – Евбазовское приветствие.
\item  «А шейнер пунэм» (идиш) – красивое личико.
\item  «Бачили очі що купували їжте хоч повилазьте» - без комментариев.
\item  «Бекицер, бекицер» (иврит-бекицур) – ускорься, быстрее.
\item  «Біля Лева печінка дешева» - рядом с Самсоном был базарчик.
\item  «Брешет, как дышит, а дышит часто» - без комментариев.
\item  «Брешет как сивый мерин» - без комментариев.
\item  «Будете в Киеве- заходите в Лавру» – так киевляне приглашали в гости, но своего адреса не давали.
\item  «Бывает хуже, но реже» - без комментариев.
\item  «Была команда не пыхтеть» - не ныть.
\item  «Ваш борщ такий смачний, що у цицьках пече. Так виньте цицьки из борщу» – без комментариев.
\item  «Вей из мир» (идиш) – горе мне.
\item  «Вилазь з ванни, а то верба в сраці виросте» - хватит сидеть в воде.
\item  «Він мов гівно з просом» – він гливкий як в’ялений інжир. 
\item  «В инвалидку за бананами» – пойти в магазин для инвалидов.
\item  «В Киеве и кирпич не краснеет» – поговорка богомольцев. 
\item  «В Киеве одни стены святы, а люди поганы» - поговорка богомольцев.
\item  «В левом кармане в старом пальте» - в ответ на вопрос «Где?»
\item  «Возьми с полки пирожок, сдуй пыль и положи на место» - без комментариев.
\item  «Вона так кричить, аж ВИРЯЧИЛА (выпучила) очі» - без комментариев.
\item  «Вот такие пироги с котятами» - вот так-то.
\item  «В свинячий голос» - когда что-то делается не вовремя или с опозданием.
\item  «Все, Бобик сдох» - без комментариев.
\item  «В собачий брех» - сказать или сделать что-то в догонку, когда уже поздно.
\item  «Встретимся на шариках» - на улице Мечникова у дома с шарами на парапете. 
\item  «Встретимся у пеликанов» - у фонтана с пеликанами  на Лютеранской.
\item  «Встретимся у Черного моря» - в парке Шевченко у фонтанчика.
\item  «Встречаемся у лаптей» - у Сковороды.
\item  «Вус эрцих? Вус ретих?» (идиш) - что слышно? что говорят?
\item  «Вус трапилось» (идиш/укр.) – что случилось?
\item  «В холодильнике мышь повесилась» - пусто.
\item  «Выпить лампочку коньяку» - выпить из плошки для иллюминации.
\item  «Вы посмотрите на эту дарницкую дворянку» – возглас возмущения евбазовских торговок.
\item  «Гейн какен вайтер» (идиш – иди какать подальше) – ласковый посыл.
\item  «Гиб а кик, шо робыться» (идиш/укр.) – Взгляни, что творится.
\item  «Глухих повезли» - если человек не расслышал и переспросил.
\item  «Граф Алексей Бобринский - «А кто это будет делать - Бобринский?», "Теж мені, графиня Бобринська!"», и пр. - на месте Щорса, стоял памятник графу Бобринскому, который много в жизни успел сделать. Вот оттуда и пошло.
\item  «Грузи на Бессарабку» – динамовско-стадионное.
\item  «Гуляй отсюда... в Дарницу» - отстань.
\item  «Дай мне кецик» - кусочек чего-либо съестного. Евбазовское.
\item  «Двойная половинка» - первые кофейни на Крещатике и на Заньковецкой, в которых готовили аппаратный кофе, в 80-е назывались кафетериями и были оснащены автоматами, настроенными на стандартную порцию кофе на стакан воды. Напиток получался так себе. Для кофеманов работница кафетерия вручную отмеряла две порции кофейного порошка и половинку порции воды. Получалось полстакана (разливали в граненые стаканы) кофе, носившего название «двойная половинка».
\item  «Девушка ты где была, когда Бердичев горел?» - к опоздавшей и с издевкой.
\item  «Дела идут - контора пишет» - без комментариев.
\item  «Делать шалохмонес» (идиш) - делать подарки.
\item  «Доведешь до цугундера» - до тюрьмы.
\item  «Дрек мит фефер» (идиш– гавно с перцем) – крученый пройдоха. 
\item  «Дурень думкой багатіє» - без комментариев.
\item  «Дурне, як сало без хліба» - без комментариев.
\item  «Дурное дело не хитрое» - без комментариев.
\item  «Душить мерзавчика» – пить водку тайком в общественных местах.
\item  «Дякую, аж підскакую» - без комментариев.
\item  «Езжай на Бесарабку, купи себе петуха и морочь ему голову» - без комментариев.
\item  «Ееееперный театр» - крайняя степень удивления, возмущения или безнадеги.
\item  «Если кому хрен, то мне всегда два» - подольское…
\item  «Если хочешь поработать ляжь поспи и все пройдет» - без комментариев.
\item  «Ест, как не в себя» - обжора и не поправляется.
\item  «Ешь пока рот свеж» - без комментариев.
\item  «Жизнь дала трещину еду на Троещину» - без комментариев.
\item  «Забегалась, как Ицикова сучка» - без комментариев.
\item  «Забрать в торбу» -  посадить в тюрьму.
\item  «Закрой душу» - одень шарф.
\item  «Замнем, для ясности» - без комментариев.
\item  «Зарікалася свиня гівна не їсти…» - на заведомо невыполнимые обещания.
\item  «Зачем мне такие ландыши?» - без комментариев.
\item  «Зачем мне этот гембель?» - …эти крупные неприятности.
\item  «Згадала бабця як дівкою була» - без коментариев.
\item  «З'їв три і морду втри» - про большие вареники.
\item  «Знайшов сокирку пiд лавкою» - если нашёл предмет, который лежал у всех на виду.
\item  «Їхати на ковбасі» - У старих трамваях був ззаду маленький майданчик, на який ставали чоловіки, а тримались за драбину, яка вела на дах. Отак їхали від зупинки до зупинки.
\item  «Иди на Евбаз курочек ловить» – ласковый посыл.
\item  «Из под пятницы – суббота» - если из-под рубашки видно было майку или из-под пиджака - подол рубашки - это считалось очень неряшливо.
\item  «Их вейс?» (идиш) – откуда я знаю?
\item  «Как два пальца об асфальт» - сделать нечто очень простое и лёгкое.
\item  «Как зайцу стоп-сигнал» - о ненужном.
\item  «Как корове пятая нога» - о ненужном.
\item  «Как рыбке зонтик» - о ненужном.
\item  «Как ты мене дорог» - говорилось если треплют  нервы.
\item  «Кино и немцы» - без комментариев.
\item  «Летел, как сраный веник» - без комментариев.
\item  «Лопай, лопай, ровняй морду с жопой» - без комментариев.
\item  «Мишигене копф» (идиш) - дурная голова, чокнутый.
\item  «На городі бузина, а в Києві дідько» - що ти дурню верзеш.
\item  «Наздогад буряків дайте капусти» - говорить об одном, имея в виду другое.
\item  «На ней бэрэт, мышиный цвет и юпочка в полосочку» – модница.
\item  «На "нет" - и суда нет! Но есть... Особое совещание» -поговорка сталинских времен.
\item  «Наплюй и забудь» - не переживай.
\item  «Наружное образование» – умение одеваться и держать себя на людях.
\item  «Насипь мені миску борщу» - без комментариев.
\item  «Наче здрасьте среди ночи» - сюрприз.
\item  «Не вари воду» - не капризничай.
\item  «Не делай мне вырванные годы» - без комментариев. 
\item  «Не делай мне нервы» - не нервируй.
\item  «Не дрек мне копф» (идиш/рус.) - не гамни мне голову. 
\item  «Не звони, метличка, метлик фартовый» – сленг проституток про богатого клиента.
\item  «Не компостируй мне мОзги» - не надоедай.
\item  «Не люблю, коли вмер і дивиться» – про ненадежного человека, который может изменить свое решение.
\item  «Нема лаве - нема кохання» – без денег – никуда.
\item  «Не питай кума, чому заплакані очі» - без комментариев.
\item  «Не разводи понты» - не хвастайся.
\item  «Не стой над душой» - отстань.
\item  «Не пришей кобыле хвост» - чужой, посторонний.
\item  «Нет ума-считай калека» - неодобрительно о чем-то.
\item  «Неудобно спать на потолке - одеяло спадает» - без комментариев.
\item  «Не хочешь нормального - будешь есть дордочки на постном масле» - будешь есть всякую дрянь.
\item  «Нечего считать зубы в чужом рту» - не завидуй.
\item  «Не швендяй» (укр.) – не слоняйся без цели.
\item  «Не швыцай» (идиш – швыц-потеть) – не фраерись.
\item  «Нi з глови мови - нi з дупи перду»  (польск.) – не с головы, не с жопы.
\item  «Ні кує, ні меле» – ни то ни се. 
\item  «Ни рыба ни мясо» - ни то ни се.
\item  «Ну ты и босяк» – так звались голоногие изгои, артельная подольская «босая команда» временные грузчики на Притыке (пристани) еще при царе. Они спали там босиком и на подошвах у них мелом были написаны цены, за которые они нанимались.
\item  «Одевай курточку и иди в Дарницу» –  посыл из «Адъютанта...»
\item  «Она как форц ин росл» (идиш) – она, как пук в рассоле.
\item  «Он из бичевых (аристократов)» – из польских помещиков, которые ездили с бичами (кнутами). 
\item  «Опять за рыбу гроши» – ну сколько можно?
\item  «Пана видно по халяві» - по сапогам.
\item  «Пани мої дрібнесенькі , а воші як біб» - без комментариев.
\item  «Партач (a parte) подольский» – кустарь-одиночка.  
\item  «Пенициллинка (кирпичка, дрожжевой, ДШК) задула» – потянуло запашком.
\item  «Перемелется - мука будет» - со временем все пройдет.
\item  «Підожды» - без комментариев.
\item  «Після хупИ цілують у дупу» - поздно извиняться.
\item  «Плевать с лаврской колокольни» - без комментариев.
\item  «Позовите городового - пусть он отведёт меня в сумасшедший дом» – в конфликтных ситуациях местного значения.
\item  «Поедем на эмочке?» – на такси.
\item  «Позвать оркестр милиционеров?» - в ответ на «мне скучно…»
\item  «Полный швах» - ужасно.
\item  «Пора на Фрунзе 103/в Кириловку/в дурку» - без комментариев.
\item  «Поц аид, хуже гоя стахановца» - без комментариев.
\item  «Поц Додя в вашем взводе?» - не неси ахинею.
\item  «Пойти в Яму» - пойти за продуктами в низину в магазинный ряд между ул. Бастионной и Струтинского (Болсуновской). 
\item  «Прекращай свои выбрыки» - не последовательные поступки.
\item  «Приравняли мого Льоню до свого Льоньки» - без комментариев.
\item  «Прошу, пані, до мешкані на скляночку хербати» (польс.) - приходите в гости на чай.
\item  «Прямо барыня сраной масти» - высокомерная.
\item  «Пся крев, а скільки гонОру» - пренебрежительно о заносчивом человеке.
\item  «Пьет, как слепая лошадь» - пьет в больших объемах.
\item  «Расскажи это своей бабушке» - не бреши.
\item  «Расти большой, не будь лапшой» - напутствие.
\item  «Ривка така медікована» – такая хитрая соседка.
\item  «Розумному лихо, дурному радість» - без комментариев.
\item  «Рупь или в морду?» – предложение босяка к кавалеру гуляющей парочки.
\item  «Рубай компот - он жирный» - без комментариев.
\item  «Самый цимес» – великолепно.
\item  «С длинными руками под монастырь» - без комментариев.
\item  «Се тит зих хойшех» (идиш - прямо дым идет) – подготовка к какому-то 
событию.
\item  «Сиди жди, поки, жаба цицки дасть» – жди, мечтай - может дождёшься.
\item  «Сказка про белого бычка»  - ложь.
\item  «С Короленка видна Ленка» – с Короленка, 15 (с милиции) видна Сибирь…
\item  «Сорок бочек арестантов» - явная ложь.
\item  «Срали-мазали» - когда кто-то говорил ерунду.
\item  «Срать захочешь - штаны спустишь» - беда научит.
\item  «Стелися» - стели постель.
\item  «С тобой гамно хорошо есть, прямо из рта выхватываешь» - перебивающему разговор.
\item  «Судьбу конём не объедешь» - без комментариев.
\item  «Так далеко, як до Києва рачки» – о чем-то далеком.
\item  «Та, ладно…» - не верю, обманываете.
\item  «Танцуем от печки» - из лексикона знаменитого киевлянина Соломона Шкляра, героя известной песни, который, вообще-то, был парикмахером и работал на Бибиковском бульваре. Однако решил, что сможет преуспеть и совершенно неожиданно открыл в доме Пфалера на большой Васильковской, 10 школу танцев. Нанятые учителя танцев преподавали, а Шкляр, как сказали бы сейчас, занимался конферансом. Он объяснял фигуры танца, беспрестанно шутил, много комментировал, давал рекомендации, короче, создавал настроение. 
\item  «Танька (или любое другое имя) на базаре семечками торгует» - в ответ на уничижительное обращение.
\item  «Та тож кожемяцкие аристократы» - без комментариев.
\item  «Твое сейчас, як еврейське зараз... - без комментариев.
\item  «Твои штаны меня в Киеве держат» - без комментариев.
\item  «Твою дивизию…» - без комментариев.
\item  «Тебе на Берковцах прогулы ставят» – тебя заждались на кладбище.
\item  «Ти, як шкварка на сковороде, когда водой брызнули» - к очень злой.
\item  «Тиць-гриць» - что-то, сказанное не в тему.
\item  «Ти як сорока на лозині» - болтаешся туда-сюда.
\item  «Тож мені бариня з Борбулаївки» - про даму с Борщаговки.
\item  «То цирульник из-за канавы» – без комментариев.
\item  «Трясця твоїй матері» - без комментариев.
\item  «Тю…» – без комментариев.
\item  «У всех дети, как дети, а это какой-то выродок» - без комментариев.
\item  «Ума нет- на Бесарабке не купишь» - без комментариев.
\item  «Умный, как моя бабушка потом» - без комментариев.
\item  «Устроить пасовку» – прогулять уроки.
\item  «Усюсяный-ухвысяный» (совр. уси-пуси) — Это означало, что младенец "любленный-голубленный".
\item  «У тебя ум в голове или пепел» - без комментариев.
\item  «У тебя что? Шило в заднице?» - к непоседе.
\item  «Химині кури, а Мотрині яйця» – спорный вопрос.
\item  «Хороший тухес - тоже нахес» (идиш/рус.) - тухес-попа, нахес-счастье. 
\item  «Хочу на копки-баранки» – посади меня на плечи.
\item  «Храбрый  Янкель-держит дулю в кармане» - без комментариев.
\item  «Цирк на дроті» — о нелепой ситуации.
\item  «Цьом, цьом, Пацю в срацю» - о ком-то избалованном.
\item  «Через духовку на Куреневку» – не в том направлении.
\item  «Чорти шо и с боку бантик» - на какую-то нелепость.
\item  «Что на Шулявке конь - на Липках кошка» – без комментариев.
\item  «Что ты на горло берешь» - не требуй.
\item  «Что ты понимаешь в колбасных обрезках» - снисходительное.
\item  «Ша, селяне, земля ваша» – не шумите за столом.
\item  «Шоб Вы мне были здоровы» - без комментариев.
\item  «Шоб ты жил напротив своего дома» – лукьяновское выражение про Деда Лукьяна (тюрьму).
\item  «Шо Вам сказать…» – нет слов.
\item  «Шо за фойле штык» (укр/идиш) – что за дурная шутка?
\item  «Шо он корчит из себя большого балабуста» - большого хазяина.
\item  «Шо ты, как здрасте из-под дивана» - что за неожиданность?
\item  «Шо ты носишься с ним, как дурень с писаной торбой» - без комментариев.
\item  «Шо ты смотришь, как Ленин на буржуазию» - без комментариев.
\item  «Шо ты крутися, як гімно в ополонке» - непоседа.
\item  «Шрэк мир готыню, но штруф мир ништ» (идиш) - пугай меня, Господи, но не наказывай.
\item  «Штил зол зайн» (идиш) – пусть будет тихо.
\item  «Щас, возьму разбег с Бессарабки» – выражение торговок Сенного рынка на выпады насчет цены товара.
\item  «Щоб тебе підняло та три рази гепнуло» - без комментариев.
\item  «Що занадто, то не здраво" - без комментариев.
\item  «Що з возу впало, то пропало» – тысячелетняя киевская поговорка с Андреевского спуска, где была таможня.
\item  «Що ти швендяєш?» – чего гуляешь?
\item  «Эсен тренем дафн кенем» (идиш) - кушать и заниматься любовью, надо уметь.
\item  «Эта цаца, полосатее матраца» – крутая дамочка.
\item  «Эта шапка/штаны/платье/т.п. меня держит в Киеве» - без комментариев.
\item  «Это форменное надувательство» – киевские мясники «надували» мясо для улучшения внешнего вида;
\item  «Это что за шарашкина контора» – контора строительного подрядчика Шарашкина, неофициальным девизом которого было: «Главное подрядиться», т.е. взять аванс. Называлась в противовес конторе Льва Гинзбурга.
\item  «Эх, ты шляпа без полей" – растяпа.
\item  «Я Вас умоляю» - без комментариев.
\item  «Языком трепать - не мешки ворочать» - без комментариев.
\item  «Яке корiння - таке i насiння» - без комментариев.
\item  «Як справи? Як i зліва» - без комментариев.
\item  «Яхна ты такая» – грубая, не образованная еврейка. Так ул. Ярославскую называли Яхнославской.
\end{itemize}
