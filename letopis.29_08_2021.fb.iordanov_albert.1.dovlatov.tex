% vim: keymap=russian-jcukenwin
%%beginhead 
 
%%file 29_08_2021.fb.iordanov_albert.1.dovlatov
%%parent 29_08_2021
 
%%url https://www.facebook.com/permalink.php?story_fbid=832742217439520&id=100021112487542
 
%%author Иорданов, Альберт
%%author_id iordanov_albert
%%author_url 
 
%%tags 1990,dovlatov_sergej.pisatel,kultura,literatura,rusmir
%%title Александр Мелихов - 24 августа 1990 года ушел из жизни Сергей Довлатов
 
%%endhead 
 
\subsection{Александр Мелихов - 24 августа 1990 года ушел из жизни Сергей Довлатов}
\label{sec:29_08_2021.fb.iordanov_albert.1.dovlatov}
 
\Purl{https://www.facebook.com/permalink.php?story_fbid=832742217439520&id=100021112487542}
\ifcmt
 author_begin
   author_id iordanov_albert
 author_end
\fi

Александр Мелихов

24 августа 1990 года ушел из жизни Сергей Довлатов.

Сергей Довлатов писатель культовый: его поклонникам он дарит не просто
эстетические переживания, до которых есть дело лишь сравнительно узкому кругу
ценителей изящного, но — уроки жизни. Он создал обаятельнейший образ беспечного
странника, который, словно сказочный Иванушка, и в огне не горит, и в воде не
тонет. Ведь самое тягостное в реальном мире это вечная необходимость быть
серьезным и предусмотрительным, а Довлатов создал мир, в котором легкомыслие не
карается так жестоко, как в реальности. Еще бы нам не любить этот мир и его
творца!

\ifcmt
  pic https://scontent-cdt1-1.xx.fbcdn.net/v/t39.30808-6/240966536_832742194106189_7367129696738430034_n.jpg?_nc_cat=106&ccb=1-5&_nc_sid=730e14&_nc_ohc=bMHXoMP22n0AX85GRlN&_nc_oc=AQmI9OXbnZDxxxtJBaw8U0AHMB5oL1mEREOUKXnEVPmrB2lpRxuUPbuBkfYxW111Coo&_nc_ht=scontent-cdt1-1.xx&oh=625d1eb6f38a92d809222735d4eee887&oe=613671F9
  width 0.4
\fi

Однако люди пишущие прекрасно знают, что из реальности в литературу без
целенаправленного, сознательного или бессознательного, художественного
преображения не попадает ничто, ни личность, ни творчество. Не только стихи, но
и проза растет иногда из удивительного сора, и заслуга художника тем выше, чем
более неузнаваемым и совершенным в его творениях предстает этот сор. Из
«Эпистолярного романа» — переписки Сергея Довлатова с тоже недавно ушедшим из
жизни  Игорем Ефимовым (М., 2001) художественная лаборатория Довлатова
предстает совершенно неожиданной.

В ленинградские годы Игорь Ефимов, по тогдашним меркам, вполне успешный
писатель, поддерживал абсолютно не признанного официально Довлатова, и тот,
естественно, был ему благодарен. Затем в разное время они оба оказались в
Америке, и об эмигрантских тяготах и унижениях Ефимов очень откровенно
рассказал в своих воспоминаниях «В Новом Свете» (М., 2012). У Довлатова же все,
как обычно, выглядело увлекательно и забавно, и эта манера наконец принесла ему
заслуженное признание, а сложившаяся иерархия отношений с коллегой начала его
тяготить, в результате чего стал нарастать ком взаимных обид. И вот 13 января
1989 года Ефимов пишет Довлатову длиннейшее письмо, в котором помимо чисто
бытовых упреков уличает Довлатова в том, что он недолюбливает людей, которые в
ладах с собой, с жизнью, друг с другом, и сочиняет о них «сплетни-самоходки»,
настолько забавные, что их начинают распространять даже те, кто в них не верит.

Но, прежде чем расстаться окончательно, Ефимов великодушно признает и большие
достоинства обидчика: «Кроме обаяния и одаренности, в Вас привлекает еще одна
черта, одна страсть — благородная и бескорыстная, — за которую люди Вам многое
прощают: страсть к литературе. И, к Вашей чести, страсть эта остается
неудовлетворенной. Ни успехи, ни популярность не ослепляют Вас (хотя и тешат,
что вполне простительно). Вы хотели бы писать лучше». 

И тут Ефимов предлагает Довлатову рецепт нового творческого рывка: хватит
писать про других — напишите, наконец, про себя. Он предвидит и ответ: как это
«не про себя»? — да почти все мои вещи написаны от первого лица, почти все —
про себя! И разъясняет свою мысль примерно так: типичные довлатовские персонажи
всегда смешные и нелепые, зато живые, а герой-рассказчик не смешон, зато
схематичен.

Что же, по мнению Ефимова, сковывает Довлатова? Некий джентльменский кодекс.

«Ваши отношения с людьми полны бурных и неподдельных чувств, но чувства эти
гораздо ярче и шире исповедуемого Вами кодекса «джентльменского поведения»,
гораздо многообразнее и неуправляемее того, что Вы считали бы достойным,
нестыдным, — и Вы отказываете в них своему литературному alter ego. Ваш «я»,
Ваш Алиханов, в общем-то, всегда сохраняет достоинство — даже в бедности, в
неудачах, в поражениях, в пьянстве он не делает — а главное, не испытывает —
ничего, за что человеку могло бы быть по-настоящему стыдно. И поэтому остается
скучноватым (даже если попадает в смешное положение), беспорочным (даже если
совершает что-то неблаговидное) и неубедительным». 

Один из источников обаяния довлатовского alter ego — добродушие,
снисходительность. Но Ефимов советует Довлатову разрабатывать ровно
противоположную черту — раздражительность.

«Раздражение может дорасти до уровня великой страсти, и оно так же достойно
описания, как и всякая другая страсть. Спрашивается: почему же в устном
рассказике Вы не боитесь нарисовать себя обуянным этой страстью, а в письменном
— не решаетесь? Не потому ли, что в дружеском расположении своих слушателей Bы,
как правило, уверены, а невидимым читателям все еще до сих пор, по привычке
стараетесь понравиться? Неужели долгий опыт не показал Вам, что понравиться им
можно, только вывернув душу наизнанку, а наскучить — очень легко? 

Перечтите «Записки из подполья». Перечтите «Падение» Камю. Перечтите исповеди
Толстого, Руссо, Блаженного Августина. Не для того, чтобы научиться у них
каяться (в этом нет никакой нужды), а для того, чтобы дать душе некую раскачку,
необходимую для такого резкого и жутковатого оборота».

Переписку мало кто прочтет, поэтому не поскуплюсь на цитаты. 

«Вы часто жаловались, что не знаете, о чем писать. Вы устраивали себе игру с
разнобуквенными началами слов в одной фразе. Вы не даете жизненным впечатлениям
отстояться и спешите их запихнуть в какую-нибудь «Иностранку», хотя должны были
бы уже заметить, что все Ваши лучшие вещи написаны о событиях, отошедших хотя
бы лет на пять, на десять назад. Тут же Вы сможете соединить два дела, которые
любите больше всего на свете: говорить о себе и создавать хорошую литературу.
Только говорить уж придется полную правду. До конца. Придется расстаться с
образом «симпатичного и непутевого малого» — самообольстительная
характеристика, включенная в Ваше письмо, — которым Вы тешили себя так долго.
Поверьте, никто не видит Вас таким. Все равно спрятаться за этим муляжом так же
невозможно, как успеть прикрывать ладошкой испорченный зуб — Ваш трогательный
жест, — когда на Вас внезапно нападает приступ смеха». 

«Всю жизнь Вы использовали литературу как ширму, как способ казаться. Вы
преуспели в этом. Вы достигли уровня Чехонте, Саши Черного, Тэффи, Пантелеймона
Романова. Но я чувствую, что Вам этого мало. Вас не устраивает остаться до
конца дней «верным литературным Русланом», который гонит и гонит колонну одних
и тех же персонажей по разным строительно-мемориальным (то есть
вспоминательным) объектам. Вам хочется большего. И если это так, я не вижу
другого способа, как превратить ширму в экран — экран, на который будут
спроецированы Ваши настроения. Ваши сильнейшие чувства, какими бы
неблаговидными они Вам ни казались. 

Олеша прославился повестью «Зависть». У Вас есть все данные, чтобы написать на
том же уровне повесть «Раздражение». Сюжет даже неважен. Это может быть просто
серия портретов людей, сильно задевших вас в жизни. Но не умелые зарисовки с
натуры (в этом-то Вы набили руку), а портреты — в буре тех чувств, которые эти
люди в Вас вызвали. Мне кажется, можно взять любой персонаж, проходящий
сквозной линией через Ваши писания — жену, возлюбленную, сослуживца,
начальника, Веру Панову, Грубина, брата и десятки других — и написать про них
совершенно по-новому: хронику (до мельчайших деталей) моих (автора) чувств по
отношению к ним». 

«Я понимаю, что то, к чему я призываю Вас, — не шутка. Я понимаю, что нет
ничего проще, как отмахнуться от моих затянувшихся разглагольствований,
привычным ходом найти все слабые и смешноватые места в них и быстренько
сочинить какую-то отбрехаловку. (Например: «Если ты такой умный, почему же сам
не напишешь повесть на уровне «Записок из подполья»?». Отвечаю: потому что у
меня в душе нет — признаю это с завистью и некоторым почтением — таких бурных
страстей, и мне приходится подолгу высматривать и выведывать их в других
людях). Но мне сдается, что Вам сейчас не до таких простых уловок. 50 лет —
какой-то рубеж, на котором многих бросало в нежданный водоворот, переворачивало
всю жизнь». 

Довлатов сегодня так любим и популярен, что для многих молодых литераторов он
являет собою не просто венец, но и чуть ли не конец литературы: будь ироничным,
наблюдательным, избегай пафоса, и больше ничего не требуется. Поэтому я хотел
бы, чтобы именно молодые прочли и обличения Ефимова, и ответную исповедь
Довлатова, отправленную примерно через неделю.

«Я уже писал Вам, что в Вашем письме много справедливого, увы, и даже есть
неожиданная для меня (по степени) проницательность, и сначала я решил, что не
буду оправдываться, потому что не в состоянии обсуждать собственную личность,
да и не так уж я люблю говорить о себе, как Вы считаете, не больше, чем о
других, во всяком случае. Но это неважно. Я думал, что могу, то есть, в
состоянии не оправдываться, но выяснилось, что по истечении двух суток не
проходит тоска, в которую меня повергло Ваше письмо, и, в общем, я решил Вам
написать. Не потому, повторяю, что хочу оставить за собой последнее слово, а
потому, что надеюсь смягчить некоторые Ваши представления, а некоторые просто
уточнить или даже опровергнуть. 

…Единственное, с чем я могу согласиться — это моя страшная раздражительность и
невоздержанность, которые не полностью, но хоть процентов на 50 связаны с
насильственной трезвостью, но с этим я согласен, хотя все-таки надеюсь, что
между раздражительностью и ненавистью есть большая разница. 

…Что касается деловой части Вашего письма, то если не считать, что Вы написали
все это в шутку, то тогда Вы обратились не по адресу: я не обладаю талантом,
говорю это, поверьте, без кокетства, чтобы написать психологическую драму и
вообще — книгу о внутреннем мире, у меня это не получится, и я даже не возьмусь
никогда. Я знаю предел своих способностей, и думаю, что уже сейчас получил за
свою литературу больше, чем заслуживаю. Вот уже два года я ничего такого, что
увлекло бы меня, не пишу, а до этого два года писал то, что меня в результате
не устраивало. К сожалению, я знаю, что способности — это физиология, они могут
иссякнуть, отказать, это произошло с неизмеримо более талантливыми, чем я,
людьми. Во всяком случае, сейчас я уверен, что ничего хорошего больше не
напишу. 

Если же напишу то, что меня бы самого устраивало, то, как я понял, могу Вам это показать. 

Теперь я перехожу к тому, в чем Вы, к сожалению, правы, но письмо уже сейчас
такое длинное, а смысл всего этого мне как-то все менее ясен, так что
постараюсь, чтоб было не очень длинно. 

… Правы Вы и в том, что я не люблю людей, которые «в ладах с собой, с жизнью,
друг с другом», вернее — не «не люблю», а просто я завидую им, потому что сам я
никогда ни с чем в ладах не был, но при этом хотел бы быть и веселым, и
успешным, и вообще, быть похожим на Аксенова. Зависть, как известно, не
очень-то побуждает к добру. Вы правы, что я неудачник, и это даже не связано с
конкретными обстоятельствами, не всегда плохими, потому что «неудачник» — это
такое же врожденное качество, как рост или цвет волос — кому надо, тот всегда и
во всем неудачник. 

Вы правы и в том, что моими друзьями всегда в конце концов становились люди
слабые и неудачливые, и хотя в этом смысле у меня есть знаменитый
предшественник, «тот самый малый из Назарета», но я, конечно, шучу, потому что
всю жизнь мне, действительно, естественно жилось лишь в атмосфере неудачи, что
и подтвердилось в результате конкретными обстоятельствами — разочарование в
своих творческих возможностях, проблемы со здоровьем (увидите меня на радио —
все поймете) и довольно-таки мрачный, боюсь, остаток жизни впереди. Никогда мне
не дано было ощутить довольства собой или жизнью, никогда я не мог произвести
впечатление человека, у которого все хорошо, к которому стоит тянуться, который
располагает к себе именно своей успешностью, для простоты — тот же Аксенов. Все
это может быть связано у меня с какими-то детскими душевными травмами — Ася,
плюс мечты о героизме при полном расхождении с возможностями по этой части, и
так далее. Есть и много такого, чего Вы, при всей Вашей проницательности,
просто не знаете, и это «много такого» — не в мою пользу. 

Справедливо и то, что по натуре я очернитель, как бы я ни старался представить
этот порок — творческим занятием, но это — правда. 

Правы Вы и в том пункте, в котором проявили наибольшую степень
проницательности. Вы пишете, и это, может быть, гораздо умнее, чем Вы думаете,
и имеет отношение не только ко мне, но и к литературе вообще, и даже во многих
случаях объясняет эту литературу, потому что очень часто, чаще, чем кажется,
писатель старается не раскрыть, а скрыть, я говорю о Вашей фразе: 

«Всю жизнь Вы использовали литературу как ширму, как способ казаться». Это
правда. Все мое существование сопровождается проблемой «быть-казаться», и Вы
даже не можете себе представить, до каких пошлых и невероятных вещей я доходил
в этом смысле. Суть в том, что мне не дано быть таким, как я хочу, выглядеть
так, как я хочу, и вообще, соответствовать тем представлениям о человеке
достойном, которые у меня выработались под влиянием литературы Чехова и
Зощенко. Я никогда не буду таким, как люди, нравящиеся мне, а притвориться
таким человеком нельзя, я это знаю, что не является гарантией того, что я не
буду притворяться всю свою жизнь. 

Вот так. Не уверен, что все это писание мое имеет какой-то смысл, но отправить
его я все же намерен. Не удивляйтесь, если к концу чтения перестанете понимать,
зачем все это написано. Я сам перестал понимать. Написано от тоски». 

Скажите, эта исповедь возвысила или уронила Довлатова в ваших глазах? В моих —
чрезвычайно возвысила, превратив непутевого, но симпатичного малого в глубокую
трагическую личность. И как жаль, что он не успел или не сумел раскрыть ее в
прозе.

Да, дело искусства не только раскрывать, но и скрывать от нас ужас и безобразие
реальности, — не пряча, но преображая страшное в красивое, а противное в
забавное. Красота это жемчужина, которой душа укрывает раненое место, и если
под ней нет подлинной раны, не будет и подлинной красоты.
