% vim: keymap=russian-jcukenwin
%%beginhead 
 
%%file 18_11_2021.fb.bilchenko_evgenia.1.gorod.cmt
%%parent 18_11_2021.fb.bilchenko_evgenia.1.gorod
 
%%url 
 
%%author_id 
%%date 
 
%%tags 
%%title 
 
%%endhead 
\subsubsection{Коментарі}

\begin{itemize} % {
\iusr{Евгения Бильченко}
\textbf{Андрей Гнездилов} 

дорогой, человеку, который вместо слов Россия и Русь использует слово
"Moskovia", я благодарна за осуждение. Это значит, "хорошие сапоги, надо
брать". Больше либералам, которые уничтожают мою Украину и мою Россию, мне
сказать нечего, кроме:

\iusr{Евгения Бильченко}

\obeycr
Кроме сего: Евгений Евтушенко
Хотят ли русские войны?..
Хотят ли русские войны?
Спросите вы у тишины
над ширью пашен и полей
и у берез и тополей.
Спросите вы у тех солдат,
что под березами лежат,
и пусть вам скажут их сыны,
хотят ли русские войны.
Не только за свою страну
солдаты гибли в ту войну,
а чтобы люди всей земли
спокойно видеть сны могли.
Под шелест листьев и афиш
ты спишь, Нью-Йорк, ты спишь, Париж.
Пусть вам ответят ваши сны,
хотят ли русские войны.
Да, мы умеем воевать,
но не хотим, чтобы опять
солдаты падали в бою
на землю грустную свою.
Спросите вы у матерей,
спросите у жены моей,
и вы тогда понять должны,
хотят ли русские войны.
\restorecr

\begin{itemize} % {
\iusr{Тим Печеніг}
\textbf{Евгения Бильченко} Жданов , в честь которого был целый город назван не русские по вашему ? Тот же Щербаков .это все крайне известные лица

\iusr{Александр Мурашов}
\textbf{Тим Печеніг} 

СССР вообще проект ненациональный, а как раз таки принципиально
антинациональный, основанный на различных утопических доктринах. Русские как раз
были тормозом для коммунистов (в таком радикальном формате как большевики).


\iusr{Тим Печеніг}
\textbf{Alexandr Murashov} началось ... А ваш проект то какой , батенька ? Царская Россия ?

\iusr{Александр Мурашов}
\textbf{Тим Печеніг} у меня нет проектов, слава Б-гу))

\iusr{Тим Печеніг}
\textbf{Alexandr Murashov} ну и славненько , на те вам котика БЕГЕМОТика

\iusr{Евгения Бильченко}
\textbf{Тим Печеніг} я вас убедительно просила не загружать страницу большим количеством видео? Могу повторить просьбу, спасибо за понимание

\iusr{Александр Мурашов}
\textbf{Тим Печеніг} очаровательный котейка..


\iusr{Евгения Бильченко}
\textbf{Тим Печеніг} а у вас те, кто победил фашизм, и те, кто фашист, - в одном ряду стоят? Тогда у меня нет к вам вопросов. Я не коммунист, но вопросов к вам нет.

\end{itemize} % }

\iusr{Андрей Гнездилов}

\obeycr
Над ней смеялись все почти в России,
И даже упражняясь в матерке,
Но всё-таки трёхцветный флаг впервые
Я видел в её слабенькой руке.
Поэт, воспевший паспорт молоткастый,
Ты слышал там, на Маяковке, смех
Над женщиной очкастой и щекастой
И древка хруст на обозренье всех?
Флаг вырывали с наслажденьем, хряском.
Надеюсь я, что ни один мой сын
Не будет белым и не будет красным,
А просто человек и гражданин.
Евгений Евтушенко о Валерии Наводворской.
\restorecr

В жизни есть всего два пути один правильный второй неправильный. Вы Женя выбрали второе.

\begin{itemize} % {
\iusr{Александр Мурашов}
\textbf{Андрей Гнездилов} 

это Евтушенко написал про тетку, поехавшую крышей на русофобии, т. е. про бабу
Леру. У неё русские, больше всех пострадавшие от большевицкого
эксперимента, отождествлялись с этим экспериментом. Все таки СССР создавался
подпольщиками и эмигрантами, почти не жившими в России и русских воспринимали
как тормоз коммунизма, как реакционеров. Ленин, Троцкий, Сталин, Дзержинский и
почти все большевики были естественные русофобы, как Карл Маркс. Для них Россия
была чуждой страной, просто плацдарм для мировой революции. А уж русский
сталинист это такой же дикий нонсенс, как еврей гитлерист(возможно, такие евреи и
есть, но только как нонсенс)

\iusr{Евгения Бильченко}
\textbf{Александр Мурашов} да и вообще, приплетать не самые лучшие тексты обожаемого мной поэта к своим фантазмам - не этично.

\end{itemize} % }

\end{itemize} % }
