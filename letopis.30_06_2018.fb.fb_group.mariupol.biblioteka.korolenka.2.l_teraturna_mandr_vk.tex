%%beginhead 
 
%%file 30_06_2018.fb.fb_group.mariupol.biblioteka.korolenka.2.l_teraturna_mandr_vk
%%parent 30_06_2018
 
%%url https://www.facebook.com/groups/1476321979131170/posts/1734220420007990
 
%%author_id fb_group.mariupol.biblioteka.korolenka,lunina_tetjana.mariupol
%%date 30_06_2018
 
%%tags mariupol,biblioteka,literatura,japonia
%%title Літературна мандрівка до Японії
 
%%endhead 

\subsection{Літературна мандрівка до Японії}
\label{sec:30_06_2018.fb.fb_group.mariupol.biblioteka.korolenka.2.l_teraturna_mandr_vk}
 
\Purl{https://www.facebook.com/groups/1476321979131170/posts/1734220420007990}
\ifcmt
 author_begin
   author_id fb_group.mariupol.biblioteka.korolenka,lunina_tetjana.mariupol
 author_end
\fi

Літературна мандрівка до Японії

Користувачі Центральної бібліотеки ім. В.Г. Короленко продовжують мандрувати
літературною картою світу. В цьому місяці бібліотекарі пропонують познайомитися
з дивовижною, неповторною та такою загадковою літературою країни висхідного
сонця – Японії.

Японія та її народ славиться своєю багатошаровою культурою, яку потрібно
відчути та пропустити крізь себе, щоб краще зрозуміти їх ставлення до багатьох
речей.  І як найкраще для цього підійдуть твори видатних представників цієї
таємничої країни. Тож саме час дізнатися, які літературні шедеври обрали наші
читачі на цей раз.

1. Кавабата Ясунарі «Сплячі красуні»

Серед запропонованих книжок японських письменників, найбільшу кількість голосів
назбирала книга з прекрасною назвою «Сплячі красуні». 

Автором цієї книги є Кавабата Ясунарі – перший японець що став лауреатом
Нобелівської премії з літератури. Визначною темою його творів є – краса у всіх
своїх проявах. У книзі, що потрапила до Топ-5 читацьких уподобань підіймається
проблема старості, та глибина почуттів літніх людей, яка пронизає до глибини
душі, завдяки неймовірному таланту та смаку автора. 

2. Харукі Муракамі «Післямрак»

Якщо Ви досі не знайомі з жодною книгою автора, чий твір зайняв другу ланку
рейтингу, радимо уважніше до нього придивитися, бо саме він відрізняється
особливою легкістю і глибиною водночас. 

Відкриваючи роман Харукі Муракамі  «Післямрак», ми немов розкриваємо людську
душу. Ця книга береже в собі сотні життєвих таємниць, трагедій, або навпаки
радощів і чекає на читача, якому вона їх розкаже. 

Куди поспішають люди, чим вони займаються у вільний час? Відповіді на ці
запитання Ви знайдете під обкладинкою  «Післямрака».

3. Акутаґава Рюноске «Страждання пекла»

Чи міркували Ви колись про те, що відбувається з тими хто після смерті потрапив
у найжахливіше місце – пекло. Вже багато хто з письменників та художників
намагались створити свій образ цього місця. Але тільки японський письменник
Акутаґава Рюноске написав книгу в якій він описав погляд на цей світ очима
художника.

«Страждання пекла» - це просто неймовірний твір, який поставить перед Вами
багато запитань. В ньому позначені важливі соціально-філософські повчання.

Для того щоб зробити свій особистий висновок Вам  варто поринути у розповідь з
запахом сливи і шелестом ширм, отримавши на додачу забобонні страхи, що
оточують все навколо. Будьте готові пережити найсильніші емоції разом з
головними героями. 

4. «Кодекс Бусідо»

Таємнича та непізнана душа Японії захоплює та кличе до занурення в світ її
надзвичайної культури. Складно навіть знайти ще одну країну, яка змогла б
позмагатися з нею глибиною традиції та особливістю менталітету. 

Чим же зумовлений таких особливий характер цієї дивовижної країни? Саме про це
розповість Вам книга, яка не могла оминути рейтинг ТОП-5 читацьких уподобань -
«Кодекс Бусідо».

Вона являє собою збірку мудрих думок японських філософів та воїнів, а її дещо
жартівливий стиль не дасть Вам змоги засумувати над нею. Відкривши її, Ви
начебто відкриєте саму душу Країни висхідного сонця.

5. Такасі Мацуока «Стріли на повітрі»

«Стріли на повітрі» - це перша книга з дилогії, Такасі Мацуока, яка читається
на одному подиху. Вона розповідає про великого князя Гендзі, який відрізнявся
особливою військовою хитрістю та вмінням воювати чесно. А головною темою є
зіткнення двох абсолютно різних культур. 

Якщо Вас цікавить японська література, культура та погляди на життя славетних
воїнів – самураїв, тоді ця книга для Вас.
