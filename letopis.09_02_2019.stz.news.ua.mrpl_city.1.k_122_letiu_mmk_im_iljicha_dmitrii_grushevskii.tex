% vim: keymap=russian-jcukenwin
%%beginhead 
 
%%file 09_02_2019.stz.news.ua.mrpl_city.1.k_122_letiu_mmk_im_iljicha_dmitrii_grushevskii
%%parent 09_02_2019
 
%%url https://mrpl.city/blogs/view/k-122-letiyu-mmk-im-ilicha-dmitrij-grushevskij
 
%%author_id burov_sergij.mariupol,news.ua.mrpl_city
%%date 
 
%%tags 
%%title К 122-летию ММК им. Ильича: Дмитрий Грушевский
 
%%endhead 
 
\subsection{К 122-летию ММК им. Ильича: Дмитрий Грушевский}
\label{sec:09_02_2019.stz.news.ua.mrpl_city.1.k_122_letiu_mmk_im_iljicha_dmitrii_grushevskii}
 
\Purl{https://mrpl.city/blogs/view/k-122-letiyu-mmk-im-ilicha-dmitrij-grushevskij}
\ifcmt
 author_begin
   author_id burov_sergij.mariupol,news.ua.mrpl_city
 author_end
\fi

В 1966 году в издательстве \enquote{Донбасс} вышла в свет тиражом 11 500 экземпляров
книга преподавателя Ждановского металлургического института Дмитрия Николаевича
Грушевского \enquote{Имени Ильича}. Это было первое опубликованное научное исследование
истории прославленного предприятия – металлургического завода имени Ильича. В
основу этого труда были положены материалы кандидатской диссертации ученого.
Чтобы написать диссертацию, а затем книгу Дмитрию Николаевичу пришлось
проштудировать неопубликованные на то время документы в исторических архивах
Москвы, Ленинграда, Киева, Харькова, Донецка, а также труды авторов, как
дореволюционных, так и советских, по истории черной металлургии России, Юга
России, Украины и СССР, публикации в периодической печати, воспоминания
ветеранов предприятия. Только в одной из своих статей - она называется
\enquote{Развитие черной металлургии в Мариуполе в дооктябрьский период} - Дмитрий
Николаевич ссылается на 132 источника.

\vspace{0.5cm}
\begin{minipage}{0.9\textwidth}
\textbf{Читайте также:} 

\href{https://archive.org/details/02_02_2019.sergij_burov.mrpl_city.k_122_letiu_mkk_im_iljicha_vladimir_bojko}{%
К 122-летию ММК им. Ильича: Владимир Бойко, Сергей Буров, mrpl.city, 02.02.2019}
\end{minipage}
\vspace{0.5cm}

\ii{09_02_2019.stz.news.ua.mrpl_city.1.k_122_letiu_mmk_im_iljicha_dmitrii_grushevskii.pic.1}

Объем книги \enquote{Имени Ильича} должен был быть куда большим. Сокращение его
произошло по той причине, что \enquote{по совету} рецензентов (а это были
идеологические работники обкома партии) автору пришлось убрать из рукописи
многие страницы, касающиеся положительных сторон труда и быта дореволюционных
металлургов Мариуполя. Были исключены имена руководителей завода, подвергшихся
в тридцатые годы ХХ столетия репрессиям. Поэтому, вероятно, на титульном листе
книги значится не \enquote{история ордена Ленина Ждановского завода имени Ильича}, а
\enquote{из истории...}. Тем самым Дмитрий Николаевич как бы намекает – не все им
написанное попало в книжку. Тем не менее, она до сих пор остается важным
источником для краеведов, часто цитируется, к сожалению, не всегда со ссылкой
на автора.

* * *

А теперь о самом Дмитрие Николаевиче Грушевском. Кого бы из его учеников ни
приходилось встречать, неизменно они с теплотой вспоминали о своем учителе. Это
были разные люди. Одним Дмитрий Николаевич преподавал историю в 17-й школе,
другим – в вечерней школе рабочей молодежи, третьим – политэкономию в
металлургическом институте. Оставил он о себе добрую память и у слушателей
университета марксизма-ленинизма. Было в свое время и такое учебное заведение...

Митя Грушевский стал педагогом в восемнадцать лет, когда его – выпускника
учительских курсов – направили преподавать историю в Верхнюю Тавду – городок на
Урале. Но учительствовать пришлось недолго – всего несколько месяцев. Уже в
начале декабря сорок первого года молоденький учитель попал на фронт. Он
испытал все тяготы солдата на передовой: и изнурительные оборонительные бои на
Калининском фронте, и минометные обстрелы врага, и многосуточные переходы по
топким болотам, и гибель однополчан, а главное, постоянное ощущение присутствия
смерти. Военная карьера командира минометного расчета Грушевского прервалась 13
июля сорок третьего года на Курской дуге. Именно в этот недобрый день его
тяжело ранило осколком снаряда. После девяти месяцев лечения в госпиталях
решение медицинской комиссии было однозначным: сержант Грушевский, 1923 года
рождения, для службы в армии не годен, определить ему вторую группу
инвалидности.

\vspace{0.5cm}
\begin{minipage}{0.9\textwidth}
\textbf{Читайте также:} 

\href{https://archive.org/details/25_03_2017.sergij_burov.mrpl_city.istoria_stalevar_vladimir_klimenko}{%
История: сталевар Владимир Клименко, Сергей Буров, mrpl.city, 25.03.2017}
\end{minipage}
\vspace{0.5cm}

В мае сорок четвертого года Дмитрий приезжает в Мариуполь. Находит себе работу
по силам, время от времени лечится в госпиталях и заочно учится в Сталино (так
назывался известный нам Донецк). В сорок девятом году с дипломом учительского
института Дмитрий Николаевич начинает работать преподавателем истории в 17-й
школе и продолжает учиться, теперь уже в педагогическом институте. К тому
времени он был отцом семейства, у него и его жены Клавы рос годовалый сын. Ни
напряженная работа в школе, ни заботы о семье не помешали Дмитрию Николаевичу
окончить институт с отличием.

Не будем останавливаться на деталях послужного списка героя этого очерка,
скажем лишь то, что в пятьдесят шестом году его пригласили в металлургический
институт преподавать политическую экономию. Именно с этого времени Дмитрий
Николаевич углубился в изучение истории Мариуполя, в частности,
металлургического завода имени Ильича. О том, чем завершился многолетний
напряженный труд ученого сказано выше. Доцент, кандидат экономических наук Д.Н.
Грушевский в 1971 году издал историко-экономический очерк о нашем городе, а
через семь лет – его дополненное издание в соавторстве с журналистом Николаем
Григорьевичем Руденко.

\textbf{Читайте также:} 

\href{https://archive.org/details/22_04_2017.sergij_burov.mrpl_city.istoria_stalevar_ivan_katrich}{%
История: сталевар Иван Катрич, Сергей Буров, mrpl.city, 22.04.2017}

Преподавательскую деятельность, исследовательскую работу Дмитрий Николаевич
всегда сочетал с множеством общественных нагрузок: внештатный корреспондент
\enquote{Приазовского рабочего}, руководитель городской организации Общества охраны
памятников истории и культуры Украины, многолетний председатель профсоюзного
комитета металлургического института, популярный у жителей города лектор
общества \enquote{Знание}. Да разве можно перечислить все общественные обязанности
этого неуемного человека!

При всей своей загруженности доцент Грушевский никогда не жалел времени для
семьи. Семья была главным его оплотом в жизни: жена Клавдия Ивановна, его три
сына всегда оставались предметом радости и гордости. И нужно сказать: не без
основания. Дмитрий Николаевич был человеком добрым и жизнелюбивым. Уже тяжело
больной, все говорил своей Клаве: \emph{\enquote{Вот не смогли мы с тобой сыграть свадьбу
тогда, в сорок пятом, доживу до полувека нашей семейной жизни, да отгрохаем
\enquote{золотую} свадьбу, как это полагается}}. Но не сбылась заветная мечта. 28 апреля
1993 года Дмитрия Николаевича не стало.
