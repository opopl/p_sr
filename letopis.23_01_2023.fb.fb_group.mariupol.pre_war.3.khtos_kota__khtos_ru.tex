%%beginhead 
 
%%file 23_01_2023.fb.fb_group.mariupol.pre_war.3.khtos_kota__khtos_ru
%%parent 23_01_2023
 
%%url https://www.facebook.com/groups/1233789547361300/posts/1395349687871951
 
%%author_id fb_group.mariupol.pre_war,partsej_roman.pardubice.mariupol
%%date 23_01_2023
 
%%tags mariupol,pushkin_aleksandr,foto,poezia
%%title Хтось кота, хтось русалку запостив. А у мене Кощей безсмертний
 
%%endhead 

\subsection{Хтось кота, хтось русалку запостив. А у мене Кощей безсмертний}
\label{sec:23_01_2023.fb.fb_group.mariupol.pre_war.3.khtos_kota__khtos_ru}
 
\Purl{https://www.facebook.com/groups/1233789547361300/posts/1395349687871951}
\ifcmt
 author_begin
   author_id fb_group.mariupol.pre_war,partsej_roman.pardubice.mariupol
 author_end
\fi

Хтось кота, хтось русалку запостив. А у мене Кощей безсмертний. \enquote{Руслан та
Людмила} майже в зборі 😊

Це найкращій переклад, який мені доводилось бачити, автора нажаль не знаю:

\obeycr
Край лукомор'я дуб зелений,
Ланцюг на ньому золотий
Щодень круг дуба Кіт учений
Ступає, наче вартовий.
Праворуч йде-пісні співає,
Ліворуч- казку муркотить
Там, в хащах лісовик блукає
І мавка на гіллі сидить.
На переплутаних стежинках
Сліди небачених страхіть.
На лапках курячих хатинка
Без вікон, без дверей стоїть.
Опівночі, як місяць сяє,
Раптово море відступає:
На мокрий берег із глибин
Рушає легенів загін.
І до ранкової зорі
Там тридцять три богатирі
Несуть довічний свій дозор,
А з ними дядько Чорномор.
\restorecr
