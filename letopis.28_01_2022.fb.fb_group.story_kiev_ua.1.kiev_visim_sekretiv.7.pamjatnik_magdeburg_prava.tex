% vim: keymap=russian-jcukenwin
%%beginhead 
 
%%file 28_01_2022.fb.fb_group.story_kiev_ua.1.kiev_visim_sekretiv.7.pamjatnik_magdeburg_prava
%%parent 28_01_2022.fb.fb_group.story_kiev_ua.1.kiev_visim_sekretiv
 
%%url 
 
%%author_id 
%%date 
 
%%tags 
%%title 
 
%%endhead 

\subsubsection{7. НАЙДАВНІШИЙ ПАМ'ЯТНИК КИЄВА - МАГДЕБУРЗЬКОМУ ПРАВУ}

7. НАЙДАВНІШИЙ ПАМ'ЯТНИК КИЄВА - МАГДЕБУРЗЬКОМУ ПРАВУ. 

Важлива сторінка життя Києва і України, яка майже чотири сотні років нерозривно
пов’язувала нас з Європою і цивілізацією. І яку після окупації наших земель
імперією намагались відмінити (двічі) і забути (багаторазово). 

Це найстарший досі збережений пам’ятник столиці - і вже лише у цьому його
виключна цінність. А ще тут і місце, де річка Почайна сходилась із Дніпром і
могло відбуватись хрещення Руси. Насправді, ні, це було північніше, але
пам’ятні знаки встановлені саме тут  @igg{fbicon.smile}   

Пам’ятник присвячений такій важливій сторінці нашої історії, коли самоврядні
громади з автономними судово-адміністративними інституціями почали
засновуватись у галицьких і волинських містах наприкінці XIII ст., з середини
XIV ст. і до кінця XVIII ст., одне за іншим, українські міста, від Львова до
Полтави, долучаються до прогресивного явища, яке сприяло не лише розвитку
місцевого самоврядування, але і розквіту ремесел, торгівлі, ініціативи.
Магдебурзьке право встановлювало виборну систему органів міського
самоврядування та суду, визначало їх функції, регламентувало діяльність
купецьких об'єднань та цехів, регулювало питання торгівлі, опіки, спадкування,
визначало покарання за злочини тощо. Його поширення на території України
сприяло формуванню нових рис ментальності місцевого населення, залученню до
європейської родини народів, демократизму, меншій орієнтації на центральну
владу, появі схильності будувати суспільне життя на основі правових норм тощо.
Магдебурзьке право сприяло формуванню в Україні засад громадянського
суспільства. Саме ця ознака і досі вирізняє українців. Що ми всі могли бачити
під час Революції гідності та на фронті.

І саме тому, ми два роки тому винесли десятки мішків з брудом звідти,
заштукатурили втрати і пофарбували нижній рівень, щоб він був чистеньким і
яскраво-білим. Від технагляду Охорони пам'яток були поставлені жорсткі умови по
можливим матеріалам, а також і переконували мене, що цього не треба робити, бо
вже в 2021 році буде зроблено комплексний ремонт (показували кошторис на 5,5
млн.грн.). Але його не виконано досі. А так ми хоча б на якийсь час зробили
пам'ятник охайним і доглянутим. 

Проте, вже зараз він знову і терміново потребує таки ремонту. Можновладці не
можуть розібратись у кого на балансі він перебуває, тому це все затягується.
Якщо весною не почнуть ремонт, то доведеться збирати громадськість і зробимо
знову своїми силами (Мінкульт, вважай це погрозою). 
