% vim: keymap=russian-jcukenwin
%%beginhead 
 
%%file 14_07_2018.stz.news.ua.mrpl_city.1.futbol_nashego_detstva
%%parent 14_07_2018
 
%%url https://mrpl.city/blogs/view/futbol-nashego-detstva
 
%%author_id burov_sergij.mariupol,news.ua.mrpl_city
%%date 
 
%%tags 
%%title Футбол нашего детства
 
%%endhead 
 
\subsection{Футбол нашего детства}
\label{sec:14_07_2018.stz.news.ua.mrpl_city.1.futbol_nashego_detstva}
 
\Purl{https://mrpl.city/blogs/view/futbol-nashego-detstva}
\ifcmt
 author_begin
   author_id burov_sergij.mariupol,news.ua.mrpl_city
 author_end
\fi

\ii{14_07_2018.stz.news.ua.mrpl_city.1.futbol_nashego_detstva.pic.1}

У детей, вступивших в сознательный возраст в последние два года Отечественной
войны, было много развлечений. Девчонки расчерчивали мелом на тротуарных плитах
классики и прыгали, сменяя одна другую. Скакалка имела не последнее место в их
играх. Тем более, что особого оборудования не требовалось – выпросить у мамы
или бабушки клочок бельевой веревки – скачи, сколько хочешь. Устав от игр
подвижных, в пору цветения акации девочки собирали ее цветы, запихивали в чудом
сохранившиеся довоенные, а порой и дореволюционные флакончики от духов и
лекарств, добавляли воду, затыкали их пробкой - вот вам и духи. Устав от
прыганья и беготни, девочки прятались от солнца где-нибудь под козырьком двери,
играли в \enquote{кремешки}, проявляя при этом чудеса жонглирования. Тряпичные куклы
тоже присутствовали в девичьем обиходе.

\textbf{Читайте также:} \href{https://archive.org/details/22_06_2018.sergij_burov.mrpl_city.mariupol_22_iunja_1941_goda}{%
Мариуполь: 22 июня 1941 года, Сергей Буров, mrpl.city, 22.06.2018}

А мальчишки? У них - самодельные самокаты, состоящие их двух дощечек и двух
подшипников. А еще -  рогатки с резиной от противогазов, охота с этим оружием
за воробьями. Кто постарше - у тех самопалы. Удочки из длинной кленовой прямой
ветки, суровой нитки, бутылочной пробки с воткнутым перышком, свинцовым
грузиком и.... и крючка. Можно идти на рыбалку. Ну, еще были игры: лапта,
\enquote{казаки-разбойники}, \enquote{цурка}, городки и другие не олимпийские виды спорта.

\textbf{Но самая главная игра - футбол}. Двор на Торговой улице был интересен тем, что
значительная часть его была пуста и относительно ровна. Она-то и стала
\enquote{стадионом}. Разметка поля состояла из одной поперечной линии, отмеренной
шагами и начерченной палкой. Эта линия обозначала середину поля. Ширину ворот
определяли шагами и отмечали половинками кирпичей. За этой процедурой ревностно
следили обе команды в полном составе, чтобы все было честно.

Долгое время для игры служил тряпичный мяч. Изготовление его было чрезвычайно
просто: выпросить у бабушки старый чулок, набить его тряпьем, зашить ниткой
покрепче.

Роли нападающих исполняли быстрые и юркие, медлительные - в защиту. Вратарями
ставили самых высоких из них, хотя смысла в этом не было – тряпичный мяч высоко
не подбросишь. Просто старшие ребята, бывавшие на стадионе рядом с Городским
садом, рассказывали, что вратари в командах были выше всех игроков. Крики
азартных игроков, топот их босых ног и пыль столбом взрослые до поры до времени
терпели, мол - пусть шумят, зато - на глазах.

Но все изменилось, после того, как дядя \textbf{Пава Лосев}, военный летчик, а после
демобилизации - сотрудник спортобщества, подарил ребятам мяч и насос к нему.
Пусть кирза его покрышки была слегка потерта, но это был \textbf{\em Мяч}. Все было
прекрасно, можно было взять мяч \enquote{на кумпол} и забить гол головой, его можно
было \enquote{подкатить}, передать пас своему игроку. Но это счастье превратилось в
прах за несколько секунд. Однажды кто-то из увлекшихся игроков угодил в окно
квартиры одной из жительниц двора, Удар был сильный, и стекло разбилось в
вдребезги...

С той поры обе команды были изгнаны со двора. Но к тому времени футболисты
повзрослели, и стали играть на \enquote{поляне} - пространстве за железной дорогой
между Пост-мостом и трубой газопровода. Поверхность там была идеально ровной,
ее намыли земснарядом еще в первые месяцы строительства \enquote{Азовстали}. Теперь
дворовая ребятня не делилась на две команды, а сражалась с соперниками с улицы
Фонтанной. Но и на поляне вместо стоек ворот были обломки кирпичей, а середина
поля процарапывалась палкой...

Выпросив \enquote{трешку} у мам или бабушек, юные футболисты с Торговой улицы ходили,
именно ходили к Городскому саду на стадион \enquote{Строитель}. Такое название это
спортивное сооружение получило по той причине, что капитальный ремонт ему
сделал и взял на свой баланс трест \enquote{Азовстальстрой}. \enquote{Болели} кто за фаворитов
местного футбола - {\em\bfseries азовстальскую \enquote{Сталь}}, кто за {\em\bfseries ильичевский \enquote{Судостроитель}}.
Были фанаты {\em\bfseries\enquote{Пищевика}} и {\em\bfseries азовстальстроевского \enquote{Строителя}}. На матчах кричали:
\enquote{Судью на мыло!}, когда усматривали предвзятость арбитра. Громкими криками
приветствовали каждый гол. С придыханием произносили фамилии мастеров мяча
{\em\bfseries Ливенцева, Каракаша, Парахина, Балабанова, Котлубея.}

\textbf{Читайте также:} \href{https://mrpl.city/blogs/view/pochemu-ya-vybirayu-futbolrustam-hudzhamov}{%
\enquote{Почему я выбираю... футбол}, - Рустам Худжамов, Георгий Федоренко, mrpl.city, 10.12.2017}

В спорах ребят то и дело звучали футбольные термины \emph{\enquote{корнер}}, \emph{\enquote{аут}}, \emph{\enquote{бек}}, то
что они слышали от взрослых болельщиков. Но в 1950 году в газете \enquote{Правда} была
опубликована работа вождя \enquote{Марксизм и языкознание}. Где в частности, осуждалось
\enquote{засорение} русского языка иностранными словами. И теперь вместо \enquote{корнер} нужно
было говорить - \enquote{угловой}, вместо \enquote{аут} - \enquote{вне поля}. вместо \enquote{бек} - \enquote{защитник}
и т.д.

Мальчишки внимательно следили за событиями первенства СССР по футболу. Знали
наперечет всех игроков ведущих команд того времени. Кто-то болел за московское
\enquote{Динамо}, кто-то за \enquote{Динамо} киевское, но больше всего болельщиков - подростков
было у \enquote{ЦДКА} - команды Центрального дома Красной Армии. Кумирами молодых и
старых футбольных фанатов были {\em\bfseries Лев Яшин и Всеволод Бобров, Никита Симонян и
Григорий Федотов, Константин Бесков и Игорь Нетто, Николай Дементьев и Анатолий
Башашкин}. И вот, что интересно, ведь тогда не было еще телевидения, редкая
семья получала газеты, но зато были яркие, эмоциональные репортажи Вадима
Синявского, которые транслировались по Всесоюзному радио, а значит заходили в
каждый дом. Ведь в описываемый период истории нашего города радиоточки с
военного времени никогда не выключались в домах.

Пролетели, прошмыгнули годы и десятилетия. Но проезжая по Торговой улице мимо
старинного двора, откуда-то из подсознания всплывают образы игроков команды, у
которой не было имени, ни покровителей, ни даже, какое-то время, мяча. Но было
жгучее, все другое отвергающее желание играть в футбол. Вот \emph{\textbf{Жора}}, скрывавший
свое \enquote{девчоночье} имя \emph{\textbf{Валентин}}, наш центр нападения; \emph{\textbf{Вова}} – бек, ушедший на
заслуженный отдых с должности начальника цеха; \emph{\textbf{Толя}} – инженер, преподаватель
техникума; \emph{\textbf{Люда и Иза}} - инженеры-конструкторы, они иногда тоже играли с
мальчишками. А еще раньше был \textbf{\emph{Казя}}, судьба которого неизвестна и \emph{\textbf{Слава}}, ставший
пианистом-виртуозом, заслуженным деятелем искусств. Был и \textbf{\emph{Шура}}, награжденный
орденом за разработку систем космической тематики. Но почему-то никто из нашей
команды не стал профессиональным футболистом. Странно...
