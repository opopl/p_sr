% vim: keymap=russian-jcukenwin
%%beginhead 
 
%%file 26_11_2020.fb.semesjuk_ivan.1.rich_kosmopolyta_2020
%%parent 26_11_2020
 
%%url https://www.facebook.com/ivan.semesyuk/posts/3846425288741343
 
%%author Семесюк,Іван
%%author_id semesjuk_ivan
%%author_url 
 
%%tags 
%%title Нічні маніпулятивні діалоги в башті Національного Порозуміння
 
%%endhead 
 
\subsection{Нічні маніпулятивні діалоги в башті Національного Порозуміння}
\label{sec:26_11_2020.fb.semesjuk_ivan.1.rich_kosmopolyta_2020}
\Purl{https://www.facebook.com/ivan.semesyuk/posts/3846425288741343}
\ifcmt
	begin_author
   author_id semesjuk_ivan
	end_author
\fi

\obeycr
Нічні маніпулятивні діалоги в башті Національного Порозуміння. 
Річ Космополита, 2020 рік вашої ери.
Latynka
Я: Лук ет зіс. Здається, на часі кодифікація національного варіанту і паралельне з кирилицею, строго упорядковане використання в деяких придатних для цього сферах. Чому ні?
Хлодвіг: Задовбали! Ви уперто хочете повністю перевести українську мову на цю чужинську абетку! Вона нам не підходить взагалі. Латинка підходить усім крім нас. Це неможливо в принципі, ми особливі бо таких кириличних звуків як у нас більше ніде нема. ЇҐ на вас!
Я: Ні, ні, не перевести. Паралельне використання. Паралельне. 
In parallel. Παράλληλα. Parallellt . 並行して. Як би це ще сказати, щоби ти почув. Не заміна, а одночасне використання там де воно і так зараз використовується. Очевидно же, що не завадить повноцінна кодифікація, строго заради ладу в хаотизованій сфері реального її застосування. Нормальна ж тема!
Хлодвіг: Ясно. Ви просто хочете віддати наш спадок Росії! Відпрацьовуєте. Крім того ж є нормальний трасліт.
Я: Блін. Формально він ніби є, а по факту його нема, бо моє варварське прізвище має два паспортних написання і кілька побутових, бо всім насрати на той транслит і усі пишуть як хочуть. Людям все ж таки варто дати офіційну національну версію, так би мовити залізну. А кирилиця звичайно ж форева, бікоз ві воз, лише за, куди ж без неї? Вона ахуєнна, слався кирилице наша во віки!
Хлодвіг: Я тебе почув. Все ж таки ти вирішив повністю заборонити рідну кирилицю на радість Москві. Тепер ми не зможемо прочитати Руську Правду в оригіналі на ніч. Скільки рублєй тобі дав Кремль?
Я: Ти і так не можеш ніхріна внятно прочитати в оригіналі на ніч, бо старі кириличні тексти і так графічно адаптуються. Тільки слово "сорося" ти й можеш прочитати на паркані Софії Київської.
Хлодвіг: Зрадник! Дай жерти!
Я: Боже, не дай мені прийти до влади!
Шаровари
Я: Прікінь, Хлодвічек, цікава деталь. Існують сучасні наукові розвідки які підтверджують, що козаки часів Богдана Хмельницького не носили шароварів. А з'явилися вони пізніше, вже у 18 столітті. Інтересно, як з часом мінялася мода, тепер ми краще уявляємо їхній зовнішній вигляд. Це міняє стереотипне уявлення, і це корисно. А крім того, уяви, що крій історичних шароварів не має нічого спільного зі штанями з ансамблю Вірського. Тобто вони у них не українські. Народний стрій не знав таких штанів, в яких вони танцюють. Те саме стосується й шапок. Як тобі таке?
Хлодвіг: На поталу віддаєш національні символи, відмовляєшся від святого! Козаки носили виключно шаровари Вірського, завжди! І шапки Вірського теж. Хто стверджує інакше --- працює на Москву і сіє розбрат. Спочатку ти заперечуєш шаровари, а потім ти заперечиш Україну. Це очевидно, ахаха!
Я: Звичайно ж їх носили, просто саме тоді ще не носили, і точно геть не такі. Шо тут такого? Хіба це погано?
Хлодвіг: І знову прямо в очі брешете, що шароварів ніколи не було. Працюєте на Москву за чіткими методичками з центру. Це ясно як день.
Я: Доїв?
Хлодвіг: Насип ще, зраднику.
Шляхта
Я: Ось тобі дещо важливе для розуміння історичних процесів. Руська, читай українська, шляхта ВКЛ і Речі Посполитої....
Хлодвіг: Зрада! Україножер!
Я: А йди но мабуть нахуй, їбучий народник.
Хлодвіг: Сам іди нахуй, папіст! 
Комарь зі стелі: Панове, майте Бога в серці. Zima nadchodzi!
Усі разом: Dieu le veut!
\restorecr
