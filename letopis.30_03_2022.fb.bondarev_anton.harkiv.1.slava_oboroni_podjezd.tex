% vim: keymap=russian-jcukenwin
%%beginhead 
 
%%file 30_03_2022.fb.bondarev_anton.harkiv.1.slava_oboroni_podjezd
%%parent 30_03_2022
 
%%url https://www.facebook.com/anton.bondarev.7/posts/10209429231044854
 
%%author_id bondarev_anton.harkiv
%%date 
 
%%tags 
%%title Слава Під'їздній Обороні
 
%%endhead 
 
\subsection{Слава Під'їздній Обороні}
\label{sec:30_03_2022.fb.bondarev_anton.harkiv.1.slava_oboroni_podjezd}
 
\Purl{https://www.facebook.com/anton.bondarev.7/posts/10209429231044854}
\ifcmt
 author_begin
   author_id bondarev_anton.harkiv
 author_end
\fi

Їде хуй який зна день війни У місто-герої Харків, повертаються потроху містяни
та  Волонтерки-Кохання. Сусідки сваряться на тему які квіти саджати на клюмбі.
Мешканці дому жарять жашлики на подвір'ї. Курс Закарпатської Паленки залишается
незмінним. А пильність самих містян тим часом  стрімко зростає.

\ii{30_03_2022.fb.bondarev_anton.harkiv.1.slava_oboroni_podjezd.pic.1}

Тож, нещодавно, 91 річна мешканка 5 поверху дому, не зважаючи на вади слуху,
клялась та Божилась.. Що глибоко вночі чула як по даху будинку ходять якісь
люди...Після цього, у ганебних шкідників Української держави, увірували
мешканці інших під'їздів (навіть не глухі) і пів дня були міркування та
філософічні роздуми, яким чином падлюки дісталися на дах та як ім там зробити
засідку..

Інши не меншь пильні містяни, бачили зі своїх вікон вночі якісь ну дуужжее
підозрілі  вогники у вікнах 2-3 поверху школи що навпроти. І дуже бажали
знешкодити ворога. Але нажаль будівля школи була зачинена тож зачистка
невдалася..

А ще, одна мешканка дуже похилого віку, страшенно от полюбляла проводити весь
вільний час, біля глазка своє двері і спостерігати за тим що ж коїться у
тамбурі. Так трапилося, що нещодавно, лампа там була  перегоріла. Але пильна
громодянка, одразу ж  зрозуміла що це не просто так а справжня диверсія, яку
зробила її підступна клята сусідка щоб приховати свої ворожі диверсійні дії у
пітьмі.

Також, вчора ввечері, близько опівночі... У дверь  під'їзду який зачиняється
під час коменндантської години, раптом почали потужно стукати. Пильні містяни
одразу зрозуміли шо це може бути тільки ворожа диверсійна группа. Тож половина
з них одразу заховалися (у засідку звісно ж) а інша миттєво озброїлася
усілякими туристичними ножами та бейсбольними битами і мовчки героїчно зайняли
позицію біля вхідної двері. Після невдалих стуків, біля під'їзду неочікувано
зявилися вогники невідкладної допомоги...І саме завдяки ним боїовою радою
під'їзду було ухваленно рішення що \enquote{Це не ворог а наші приїхали
диверсантів ловити. Тож треба двері відчиняти} Колиж врешті решт, двері у
непохитну цитадель були відчинені, оборонці побачили офігевшего сусіда з 3
поверху у якого сів телефон, який затримався бо сам автомеханік і ремонтує
автівки наших і якого підвезла вночі швидка...

От так і живемо у Харкові, вперто тримаючись до Перемоги
 @igg{fbicon.beaming.face.smiling.eyes} @igg{fbicon.flag.ukraina} @igg{fbicon.beaming.face.smiling.eyes} 

P. S

Слава Під'їздній Обороні @igg{fbicon.face.tears.of.joy}{repeat=3} 

\ii{30_03_2022.fb.bondarev_anton.harkiv.1.slava_oboroni_podjezd.cmt}
\ii{30_03_2022.fb.bondarev_anton.harkiv.1.slava_oboroni_podjezd.cmtx}
