% vim: keymap=russian-jcukenwin
%%beginhead 
 
%%file 20_03_2021.fb.fb_group.story_kiev_ua.1.chess
%%parent 20_03_2021
 
%%url https://www.facebook.com/groups/story.kiev.ua/posts/1622279764635485
 
%%author_id fb_group.story_kiev_ua,ugrjumova_viktoria.kiev.pisatel
%%date 
 
%%tags chess,kiev
%%title Шахматисты–любители традиционно собираются в Киеве в трех местах
 
%%endhead 
 
\subsection{Шахматисты–любители традиционно собираются в Киеве в трех местах}
\label{sec:20_03_2021.fb.fb_group.story_kiev_ua.1.chess}
 
\Purl{https://www.facebook.com/groups/story.kiev.ua/posts/1622279764635485}
\ifcmt
 author_begin
   author_id fb_group.story_kiev_ua,ugrjumova_viktoria.kiev.pisatel
 author_end
\fi

Эти «гроссмейстеры» родились и выросли в то время, когда не Пушкин, а шахматы
были «наше все». Имена Таля, Ботвинника, Майи Чебурданидзе и Нонны
Гаприндашвили не сходили с газетных полос. 

\ii{20_03_2021.fb.fb_group.story_kiev_ua.1.chess.pic.1}

В 1978 году вся страна, без преувеличения, следит за чемпионатом мира по
шахматам. Предыдущий матч за шахматную корону не состоялся – тогдашний чемпион
мира Бобби Фишер отказался играть с претендентом Анатолием Карповым. ФИДЕ
присудила победу советскому гроссмейстеру, однако это был первый в истории
случай, когда претендент вообще не играл с действующим чемпионом.

Теперь же, в филиппинском курортном городе Багио Анатолию Карпову предстоит
матч с политической подоплекой. Нынешний претендент – бывший советский
гражданин Виктор Корчной остался на западе во время турнира в Амстердаме. В
матче претендентов он обыгрывает всех своих бывших коллег и соратников: Тиграна
Петросяна, Бориса Спасского и Льва Полугаевского. Теперь страсти накалены до
предела. Турнир проходит в «наэлектризованной» обстановке; сначала выигрывает
Карпов, затем вперед вырывается Корчной; и все должен решить финальный
поединок. Карпов выигрывает, и в первый и последний раз в истории СССР
программа «Время» открывается сообщением о результатах шахматного турнира.
Генсек лично вручает победителю Орден Трудового Красного Знамени со словами:
«Взял корону – держи, не отдавай никому!».  

И из уст в уста передавали знаменитую фразу Брежнева, который, вручая Карпову
очередную награду за очередную победу на чемпионате мира, смачно расцеловал его
и изрек: «Ну, Анатолий, теперь спокойно можешь себе это самое».

Хорошее было время, вздыхал потом Вяч. Курицын, можно было спокойно себе это
самое...

Но мы о шахматах. Анатолий Карпов и новый претендент, а впоследствии чемпион
Гарри Каспаров были знамениты, как космонавты, и всенародно любимы, как
голливудские звезды. До такой степени, что на первый матч между Карповым и
Каспаровым «Литературная газета» откликнулась серией пародий о Крабове и
Кальмарове. 

Коробка шахматной доски распахивалась как дверь в прекрасное будущее, и в
Советском Союзе несть числа было шахматным кружкам и секциям.

Шахматисты–любители традиционно собираются в Киеве в трех местах: в Мариинском
парке, парке Шевченко (там находится их «штаб–квартира») и в Гидропарке. Многие
из них играют весьма недурно, имеют разряд, в молодости участвовали в серьезных
соревнованиях или просто талантливы. С чужаками они чаще всего играют на
деньги, и в ожидании таких соперников постоянно совершенствуют свое мастерство
в дружеских партиях.

\ii{20_03_2021.fb.fb_group.story_kiev_ua.1.chess.cmt}
