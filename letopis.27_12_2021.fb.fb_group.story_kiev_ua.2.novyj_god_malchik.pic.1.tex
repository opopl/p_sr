% vim: keymap=russian-jcukenwin
%%beginhead 
 
%%file 27_12_2021.fb.fb_group.story_kiev_ua.2.novyj_god_malchik.pic.1
%%parent 27_12_2021.fb.fb_group.story_kiev_ua.2.novyj_god_malchik
 
%%url 
 
%%author_id 
%%date 
 
%%tags 
%%title 
 
%%endhead 

\ifcmt
  ig https://scontent-frt3-1.xx.fbcdn.net/v/t39.30808-6/270207908_977521572837545_706704844038594152_n.jpg?_nc_cat=106&ccb=1-5&_nc_sid=b9115d&_nc_ohc=tbgDC_o2ZKIAX9WHTCI&_nc_ht=scontent-frt3-1.xx&oh=00_AT8iyh27qQFhS_O50VeH3zya1OdHiNJkmjppQHd6O0EhGA&oe=61D448A2
  @caption_begin
    Радянська новорічна поштова листівка.

    Хлопчик-Новий рік був популярним у ті часи новорічним персонажем, що брав майже
    обов’язкову участь в дитячих «ялинках» і новорічних виставах. Звичайно він був
    у образі хлопчика, якому приблизно до 10 років, у лижному костюмі і в’язаній
    шапочці або ушанці. У 60-х і подальших роках його часто зображали в костюмі,
    стилізованому під космічний скафандр. На грудях у такого хлопчика стояли цифри
    того року, який він символізував.
  @caption_end

  @width 0.4
\fi
