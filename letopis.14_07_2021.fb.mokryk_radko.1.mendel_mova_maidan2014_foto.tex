% vim: keymap=russian-jcukenwin
%%beginhead 
 
%%file 14_07_2021.fb.mokryk_radko.1.mendel_mova_maidan2014_foto
%%parent 14_07_2021
 
%%url https://www.facebook.com/radko.mokryk/posts/10219872957037640
 
%%author Mokryk, Radko
%%author_id mokryk_radko
%%author_url 
 
%%tags 2014,foto,identichnost',ideologia,maidan2,mendel_julia,mova,narod,obschestvo,ukraina
%%title Ок, я мовчав, коли пані Мендель писала про мову
 
%%endhead 
 
\subsection{Ок, я мовчав, коли пані Мендель писала про мову}
\label{sec:14_07_2021.fb.mokryk_radko.1.mendel_mova_maidan2014_foto}
 
\Purl{https://www.facebook.com/radko.mokryk/posts/10219872957037640}
\ifcmt
 author_begin
   author_id mokryk_radko
 author_end
\fi

Ок, я мовчав, коли пані Мендель писала про мову. Бо, в принципі, не дуже є
сенс, мабуть, дискутувати. Тепер пішли розмисли про ідентичність. 

Добре, кожен має право на власні фантазії. Але в стрічці почав з’являтися новий
шматочок з опусу наближеної до президента, який, як на мене, симптоматичний. 

Ось:  

«Розгромні, рекордні 73 \% у другому турі теж у суті своїй були революцією.
Лишень прийнятною для більшості населення: без брудних наметів, розтрощеного
Майдану, негігієнічних і холодних ночей з ризиком бути побитим, знесеним
водометами чи обприсканим газом, без усієї цієї революційної «романтики»
минулих століть, але з чітким, жорстким і емоційним вироком тодішній владі.»

Цей «брезгливий» тон – дуже показовий.

Не знаю, як пані Мендель, а особисто я під час «негігієнічних і холодних ночей»
на Майдані зустрів найкращих людей своєї країни. Оті дискусії «на бочках» на
морозі в мінус 20 градусів, з чорними руками і запахом палених шин, коли
справді дрижаки хапали від холоду – це крутіше і важливіше, ніж академічні
дискусії в найкращих університетах світу.

В «брудних наметах» я протягом місяців зустрічав людей, які кинули все і
приїхали рятувати країну від диктатури. Кидали все, використовували свої
невеликі запаси грошей, закривали бізнес і їхали до Києва, «бо треба щось
робити».

В ночі, «з ризиком бути побитим» я стояв пліч о пліч з незнайомими людьми, яких
і після цього ніколи не зустрічав. Це неймовірне відчуття  - «свої» - було
настільки гострим і прекрасним, що його неможливо описати рядками. 

Довіра і солідарність. Не знаю, чи доведеться ще колись в житті відчути
настільки сильне відчуття спільноти, як тоді на Майдані. І я страшенно радий і
вдячний, що міг бути краплинкою цієї спільноти. З брудними палатками,
негігієнічними ночами і ризиком бути побитим. Бо я не знаю, чи є щось
прекрасніше за відчуття тієї справедливої напруги і відчуття солідарності на
межі екзистенційної небезпеки між людьми, об’єднаними спільною і чесною метою.

А це фото я зробив зранку 19 лютого 2014 приїхавши на Майдан. Я не знаю, хто ця
жінка, ми не спілкувались. Я з якимись чуваками був надто зайнятий розбиранням
бруківки біля пошти. І переносячи мішки з бруківкою  від пошти до профспілок
(де стояли беркути), побачив цю жінку і нашвидкоруч зазнимкував. І як на мене –
вона прекрасна. Ось це і є – Майдан.

\ifcmt
  pic https://scontent-lga3-1.xx.fbcdn.net/v/t1.6435-9/218164877_10219872956717632_3681399799819141772_n.jpg?_nc_cat=107&ccb=1-3&_nc_sid=730e14&_nc_ohc=j3heXsjpqVcAX8-dlU3&_nc_ht=scontent-lga3-1.xx&oh=d0969033062610c1f1ccf71cbc02327f&oe=6123C65C
  width 0.3
	fig_env wrapfigure
\fi
