% vim: keymap=russian-jcukenwin
%%beginhead 
 
%%file 06_05_2022.stz.pc.ua.dou.1.litachok.3.wiki
%%parent 06_05_2022.stz.pc.ua.dou.1.litachok
 
%%url 
 
%%author_id 
%%date 
 
%%tags 
%%title 
 
%%endhead 

\subsubsection{Додаток - стаття з Вікіпедії, версія 16.05.2022}

\url{https://uk.wikipedia.org/wiki/PD-2}

\ii{06_05_2022.stz.pc.ua.dou.1.litachok.pic.3.wiki}

PD-2

Матеріал з Вікіпедії — вільної енциклопедії.

\textbf{UKRSPECSYSTEMS PD-2} — український багатоцільовий БпЛА. Призначений для
здійснення повітряної розвідки та бойового застосування як носій бомбового
навантаження з бойовою частиною вагою до 3 кг[1]. Може злітати зі злітної смуги
(«по-літаковому»), також може оснащуватися модулями системи вертикального
зльоту та посадки, що робить його апаратом вертикального злету та приземлення
(англ. VTOL — \emph{Vertical Take-Off and Landing}).

\ii{06_05_2022.stz.pc.ua.dou.1.litachok.pic.4.wiki}

\paragraph{Історія розробки}

У 2014 році почалася російська збройна агресія проти України. Бойові дії
показали потребу в безпілотних апаратах, перш за все для розвідки. Волонтери
закупляли цивільні апарати, які або використовували як є, або адаптували до
потреб війни[2].

\ii{06_05_2022.stz.pc.ua.dou.1.litachok.pic.5.wiki}

Волонтерська організація «Народний проект» брала участь у допомозі українським
військовим на сході України. У спробах імпорту сучасних БпЛА, організація
зіткнулася з небажанням деяких іноземних компаній продавати свої апарати. Тоді
у співпраці з компанією «UkrSpecSystems» було розроблено безпілотний авіаційний
комплекс PD-1, який станом 2020 рік випускався для потреб Збройних сил
України[джерело?].

У 2020 році компанія «UkrSpecSystems» розробила безпілотник PD-2, який є
глибокою модернізацією PD-1 зі збільшенням корисного навантаження, дальності
зв'язку та іншими вдосконаленнями[1].

У листопаді 2020 року новий літальний апарат проходив визначальні
випробування[3]. 

\paragraph{Опис комплексу}

Наземна станція управління, що входить до складу комплексу, може забезпечувати
одночасне управління двома БПЛА. Літаки комплексу оснащені оптичною станцією з
гіростабілізованою платформою, розробленою та створеною в Україні. Підвіс
оснащений кількома камерами та тепловізором, що дозволяє проводити
спостереження на висоті до кількох тисяч метрів у будь який час доби. Вбудовані
датчики положення та спеціальне програмне забезпечення дозволяють забезпечити
чіткість та якість зображення[1]. 
