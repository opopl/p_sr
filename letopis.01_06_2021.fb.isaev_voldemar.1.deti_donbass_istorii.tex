% vim: keymap=russian-jcukenwin
%%beginhead 
 
%%file 01_06_2021.fb.isaev_voldemar.1.deti_donbass_istorii
%%parent 01_06_2021
 
%%url https://www.facebook.com/permalink.php?story_fbid=511912726895424&id=100042301026222
 
%%author 
%%author_id isaev_voldemar
%%author_url 
 
%%tags deti,donbass,pamjat,smert,vojna
%%title В День Защиты Детей, вспомним!
 
%%endhead 
 
\subsection{В День Защиты Детей, вспомним!}
\label{sec:01_06_2021.fb.isaev_voldemar.1.deti_donbass_istorii}
 
\Purl{https://www.facebook.com/permalink.php?story_fbid=511912726895424&id=100042301026222}
\ifcmt
 author_begin
   author_id isaev_voldemar
 author_end
\fi

В День Защиты Детей, вспомним!

"Я ещё чуть-чуть полежу, а потом мы пойдём домой "– 25 историй о погибших  детях Донбасса.

С 2014 года, за время войны в Донбассе, по данным мониторинговой миссии ООН
погибли боле150 детей.  Были ранены более 400 ребёнка. Общественно-политический
деятель, блогер Артём Кондратенко в своём телеграм-канале публикует истории
детей, погибших в результате вооружённой агрессии Украины. Корреспондент МИА
«Исток» собрал эти истории к Международному дню защиты детей. Здесь нет «самых»
душещипательных историй – каждая из них вызывает море слёз по погибшим ангелам,
ведь защитить их не смог никто.

\ifcmt
  tab_begin cols=4

     pic https://scontent-cdg2-1.xx.fbcdn.net/v/t1.6435-9/193197645_511911976895499_2244389142571982977_n.jpg?_nc_cat=102&ccb=1-5&_nc_sid=8bfeb9&_nc_ohc=YNqkF9ORrHYAX_65uB-&_nc_ht=scontent-cdg2-1.xx&oh=b781cf7a80085c9beb084144d0ab0db4&oe=614C70F0

     pic https://scontent-cdt1-1.xx.fbcdn.net/v/t1.6435-9/193647286_511912010228829_860063674705901645_n.jpg?_nc_cat=103&ccb=1-5&_nc_sid=8bfeb9&_nc_ohc=bv4Qc8yuQY8AX8A1ucE&_nc_ht=scontent-cdt1-1.xx&oh=6c9187afc6e639479edf2182736ee9c7&oe=614BB259

		 pic https://scontent-cdg2-1.xx.fbcdn.net/v/t1.6435-9/193335045_511912043562159_2723292204570273182_n.jpg?_nc_cat=107&ccb=1-5&_nc_sid=8bfeb9&_nc_ohc=lV7fMHbxp_UAX_5IHiv&_nc_ht=scontent-cdg2-1.xx&oh=8a24c65b63ae25fe87529c195b9a0e8f&oe=614CF049

		 pic https://scontent-cdg2-1.xx.fbcdn.net/v/t1.6435-9/193510638_511912093562154_7566444950317918120_n.jpg?_nc_cat=107&ccb=1-5&_nc_sid=8bfeb9&_nc_ohc=XtQaUIcW2aEAX-avnL7&_nc_ht=scontent-cdg2-1.xx&oh=f9ffd16343a31826bcb24c65fb70d99b&oe=614E2200

  tab_end
\fi

Дарья Барилова (11 октября 2006 – 15 августа 2014 (7 лет)

В ночь на 15 августа 2014 года во время обстрела Донецка со стороны ВСУ в дом,
в котором жила семья Бариловых, попал снаряд. Начался пожар, дом полностью
сгорел. Семилетняя Дарья Барилова погибла вместе со своими родителями.

Виктория Береговая (дата гибели – 7 августа 2014 года (14 лет)

В тот день украинские военные начали артиллерийский обстрел Горловского
водохранилища. Виктория Береговая в это время была в дачном посёлке, вместе с
бабушкой поливала огород. В результате подрыва боеприпаса четырнадцатилетняя
девочка получила осколочное ранение головы, от которого погибла на месте.

\ifcmt
  tab_begin cols=4

     pic https://scontent-cdg2-1.xx.fbcdn.net/v/t1.6435-9/193335048_511912133562150_3203151855581916503_n.jpg?_nc_cat=104&ccb=1-5&_nc_sid=8bfeb9&_nc_ohc=FakYvfb4GMgAX_BF-60&_nc_ht=scontent-cdg2-1.xx&oh=50d81ec9d8b055f02d95465af6a06c5e&oe=614BDD60

     pic https://scontent-cdt1-1.xx.fbcdn.net/v/t1.6435-9/193442542_511912163562147_7443883491985426153_n.jpg?_nc_cat=106&ccb=1-5&_nc_sid=8bfeb9&_nc_ohc=juVroIdWqjIAX-PE95n&_nc_ht=scontent-cdt1-1.xx&oh=fc2d1a8e38151afd3f96c5aa1cbfcaa3&oe=614F778C

		 pic https://scontent-cdg2-1.xx.fbcdn.net/v/t1.6435-9/194126426_511912200228810_829005494583974534_n.jpg?_nc_cat=102&ccb=1-5&_nc_sid=8bfeb9&_nc_ohc=peVs3EmLVhIAX-X36af&_nc_ht=scontent-cdg2-1.xx&oh=2a9cd41ed6fc85f053ccb69ca2e8eaf1&oe=614CBD53

		 pic https://scontent-cdg2-1.xx.fbcdn.net/v/t1.6435-9/194003031_511912236895473_4742905914089559857_n.jpg?_nc_cat=100&ccb=1-5&_nc_sid=8bfeb9&_nc_ohc=fNzNnXpVVVoAX-LR-rm&_nc_ht=scontent-cdg2-1.xx&oh=07e8501df61f45f75967b05ac5352065&oe=614E1E8D

  tab_end
\fi

Анастасия Подлипская (7 августа 2013 – 8 августа 2014 (11 месяцев)

Из-за обстрелов Горловки семья Подлипских решила перебраться в дачный посёлок.
Семья – это бабушка Татьяна Степановна, 26-летняя Валерия, молодая инженер, и
её дочь Настя. На даче был надёжный подвал, который обустроили под временное
убежище. Думали, что теперь-то бояться нечего. Татьяна Степановна ушла на
работу, Валерия возилась с помидорами, Настя играла рядом. Снаряд приземлился в
паре метров от них. Обе погибли на месте. Настя не дожила один день до своего
дня рождения. Ей должен был исполниться год.

\ifcmt
  tab_begin cols=4

     pic https://scontent-cdt1-1.xx.fbcdn.net/v/t1.6435-9/193576177_511912273562136_712513212092498789_n.jpg?_nc_cat=106&ccb=1-5&_nc_sid=8bfeb9&_nc_ohc=4tFaTvP4yqgAX_PuUo_&_nc_ht=scontent-cdt1-1.xx&oh=bfbbcb7b5d55d30fcc0b472f546362e5&oe=614E11B8

     pic https://scontent-cdt1-1.xx.fbcdn.net/v/t1.6435-9/193509164_511912313562132_1671626012847130588_n.jpg?_nc_cat=110&ccb=1-5&_nc_sid=8bfeb9&_nc_ohc=4vyQ45v4fPQAX-vBCK-&_nc_ht=scontent-cdt1-1.xx&oh=ac7200d81b7fd05eb4c6f044204be1d8&oe=614F67B2

		 pic https://scontent-cdt1-1.xx.fbcdn.net/v/t1.6435-9/193525495_511912336895463_7823213353165636427_n.jpg?_nc_cat=110&ccb=1-5&_nc_sid=8bfeb9&_nc_ohc=HB3cs18zbmgAX-XaC0P&tn=lCYVFeHcTIAFcAzi&_nc_ht=scontent-cdt1-1.xx&oh=f9a1bb6b67d7d81b137501e2819183aa&oe=614C1FD2

		 pic https://scontent-cdg2-1.xx.fbcdn.net/v/t1.6435-9/193651673_511912363562127_1730011424342473189_n.jpg?_nc_cat=107&ccb=1-5&_nc_sid=8bfeb9&_nc_ohc=6np0iHI-IccAX_rxXRC&tn=lCYVFeHcTIAFcAzi&_nc_ht=scontent-cdg2-1.xx&oh=e72bf32a449c5b105e28243b293895d5&oe=614E79D2

  tab_end
\fi

Никита Руссов (1 августа 2002 – 27 ноября 2014 (12 лет)

Микрорайон Азотный города Донецка военнослужащие украины обстреляли из
установок «Град». Никита Руссов был убит по пути на спортивную тренировку. По
словам родственников, ребёнок пытался добежать до убежища после начала
бомбёжки, но не успел – снаряд упал слишком близко. Опознать его тело удалось с
трудом, ранения были очень серьёзные.

\ifcmt
  tab_begin cols=4

     pic https://scontent-cdt1-1.xx.fbcdn.net/v/t1.6435-9/193737738_511912390228791_6199777089069839239_n.jpg?_nc_cat=106&ccb=1-5&_nc_sid=8bfeb9&_nc_ohc=PbGeZS5NhCIAX9wEZJw&_nc_ht=scontent-cdt1-1.xx&oh=6686e74563909fd0cae6af5b132ca18a&oe=614BED1E

     pic https://scontent-cdg2-1.xx.fbcdn.net/v/t1.6435-9/193510638_511912423562121_4650337864756834873_n.jpg?_nc_cat=111&ccb=1-5&_nc_sid=8bfeb9&_nc_ohc=Z790M4BUHqYAX8n31bY&_nc_ht=scontent-cdg2-1.xx&oh=1331310723eafd17934f4c26ebaae896&oe=614ED319

		 pic https://scontent-cdt1-1.xx.fbcdn.net/v/t1.6435-9/193479310_511912453562118_8363022801115665475_n.jpg?_nc_cat=101&ccb=1-5&_nc_sid=8bfeb9&_nc_ohc=OWkmzuiO264AX_eaA_5&_nc_ht=scontent-cdt1-1.xx&oh=a9b3c6192d8391cebce1e3f3c4d5a747&oe=614F6180

		 pic https://scontent-cdg2-1.xx.fbcdn.net/v/t1.6435-9/193261111_511912490228781_2787869994297924547_n.jpg?_nc_cat=104&ccb=1-5&_nc_sid=8bfeb9&_nc_ohc=k1RVZT3kFqUAX_lghPm&_nc_ht=scontent-cdg2-1.xx&oh=301fc04a5e522cd50adccddbabda0907&oe=614C8CB2

  tab_end
\fi

Валерия Суглобова (20 ноября 2001 – 18 августа 2014 (12 лет)

Валерия жила в посёлке Хрящеватое (ЛНР). Погибла 18 августа 2014 года при
попытке выезда из села, расстреливаемого украинскими войсками. Вместе с ней
погибли мама Людмила, бабушка Любовь Степановна и дедушка Валерий Михайлович
Суглобовы.

Дмитрий Орлов (13 января 1998 – 13 августа 2014 (16 лет)

Дима договорился с друзьями встретиться на школьном дворе, чтобы поиграть в
теннис. Ребята часто проводили там время. Во время школьных каникул теннисный
стол стоял рядом со зданием, и все желающие могли там поиграть. Созвонившись с
ребятами, сказав матери, что ненадолго ушёл.


\ifcmt
  tab_begin cols=4

     pic https://scontent-cdg2-1.xx.fbcdn.net/v/t1.6435-9/192552589_511912516895445_2090580223970807629_n.jpg?_nc_cat=108&ccb=1-5&_nc_sid=8bfeb9&_nc_ohc=o5i6KmQiItIAX8KraZR&_nc_ht=scontent-cdg2-1.xx&oh=306e6ac1ce6d726836dbe92621d90f07&oe=614BF9DE

     pic https://scontent-cdt1-1.xx.fbcdn.net/v/t1.6435-9/193888868_511912550228775_5179044301850261793_n.jpg?_nc_cat=109&ccb=1-5&_nc_sid=8bfeb9&_nc_ohc=7-HRiEyet7EAX_Je9ds&_nc_ht=scontent-cdt1-1.xx&oh=9144ab531d9f0c43741766ea28cf5825&oe=614D74B1

		 pic https://scontent-cdg2-1.xx.fbcdn.net/v/t1.6435-9/193384006_511912573562106_7557007857368190037_n.jpg?_nc_cat=111&ccb=1-5&_nc_sid=8bfeb9&_nc_ohc=qLRsQ3ngDjcAX_L1_5h&_nc_ht=scontent-cdg2-1.xx&oh=df516d5d90cf50efda5867bca7b2e72f&oe=614EE537

		 pic https://scontent-cdt1-1.xx.fbcdn.net/v/t1.6435-9/193315875_511912603562103_4483284927977988917_n.jpg?_nc_cat=103&ccb=1-5&_nc_sid=8bfeb9&_nc_ohc=_Xmgji45-poAX9pP0_M&_nc_ht=scontent-cdt1-1.xx&oh=b225083efdeb204a1709af55206fda05&oe=614BD092

  tab_end
\fi


Начался обстрел, снаряды ВСУ полетели в сторону школы. Растерянные ребята
носились по двору в поисках убежища. Рванув ручку двери, поняли, что школа
закрыта. Тогда побежали к выходу из заграждённой территории. Причиной
мгновенной смерти стало осколочное ранение в голову.

\ifcmt
  tab_begin cols=3

     pic https://scontent-cdt1-1.xx.fbcdn.net/v/t1.6435-9/193718784_511912630228767_2087912904085059996_n.jpg?_nc_cat=103&ccb=1-5&_nc_sid=8bfeb9&_nc_ohc=A4p53Il9bIYAX_Ch4mR&_nc_ht=scontent-cdt1-1.xx&oh=8b84bcf6a9a9f1c9340f613b77d0a9d3&oe=614BBB79

     pic https://scontent-cdg2-1.xx.fbcdn.net/v/t1.6435-9/193348322_511912663562097_2339842241811790435_n.jpg?_nc_cat=104&ccb=1-5&_nc_sid=8bfeb9&_nc_ohc=oks2mDCct8kAX_nrJXE&_nc_oc=AQnps2ObjHhazXPjLg3C7lLXjl4hc6igGdHCdGhLwDZkHqU2oaZL_QajhZVbUrisckM&tn=lCYVFeHcTIAFcAzi&_nc_ht=scontent-cdg2-1.xx&oh=38589e18a4bfef89949df71652e06081&oe=614D6D6D

		 pic https://scontent-cdg2-1.xx.fbcdn.net/v/t1.6435-9/192882034_511912700228760_497222833122975709_n.jpg?_nc_cat=104&ccb=1-5&_nc_sid=8bfeb9&_nc_ohc=Gmks7oovA28AX92lIFW&_nc_ht=scontent-cdg2-1.xx&oh=5f9ee0d38cb711dbb89a4865a26b06b7&oe=614BCDA8

  tab_end
\fi

Анна Костенко (13 февраля 2012 – 13 августа 2014 (2 года)

Аня Костенко была на пляже вместе с родителями. Ей было два годика. Папа учил
ребёнка не бояться воды. Ей нравилось кататься у него на спине. Девочка
смеялась и радовалась новым ощущениям. Всё произошло неожиданно – Анечка вместе
со своим отцом погибли при обстреле украинскими военными пляжа в городе
Зугрэсе. Папа с ребёнком так и остались лежать на берегу водоёма.

Виктория Мирошниченко (день гибели – 27 июля 2014)

Вика была доброй, весёлой, и любила животных. Её мечты теперь никогда не
сбудутся, потому что снаряд ВСУ лишил её жизни, как и многих других в тот же
день в Горловке.

В тот день дедушка с внучкой решили между обстрелами сходить в ближайший
магазин – дома закончились продукты. Налёт начался неожиданно. Горожане
попрятались в укрытия и долгое время не могли выбраться наружу. Только к вечеру
семья Мирошниченко смогла оставить укрытие, ждали дедушку с внучкой. Они были
уверены, что родные успели где-то спрятаться.

– Отца я узнал сразу, а моя девочка была похожа на новорождённого ребёнка –
такая маленькая была, – вспоминает отец Виктории. – Это была зажигательная
бомба. Вика сгорела.

Арина Гусак (4 сентября 2010 – 21 января 2015 (4 года)

Утром 21 января 2015 года украинские войска открыли массированный обстрел жилых
кварталов города Стаханова. Четырёхлетняя Арина с мамой погибли по дороге в
детский сад. В тот день Стаханов дважды подвергся шквальному огню из реактивной
артиллерии ВСУ.

Кирилл Сидорюк (21 июня 2001 – 29 августа 2014 (13 лет)

Живший в посёлке Буткевич под Петровcким (ЛНР) Киpилл в тот день c родителями и
девятилетней сеcтрой Тaтьянoй пошёл за xлебoм. Bo вpемя нaчaвшегося oбстрелa
подpoстoк нaкрыл cобой cеcтpу. Kиpилл пoгиб нa месте от paнений, не сoвмеcтимыx
c жизнью, Тaня получила ocкoлочные pанения и ушибы, но осталась жива.

Марина Леднева (23 июля 2008 – 7 августа 2014 (6 лет)

Шестилетняя Марина Леднева погибла возле своего дома, получив тяжёлые
осколочные ранения во время одного из многочисленных артобстрелов Луганска.

Евгений Орехов (15 сентября 2006 – 20 августа 2014 (7 лет)

Ученик луганской школы №17 Женя Орехов закончил в 2014 году первый класс. Он
вместе с родителями возвращался от бабушки с дедушкой, и недалеко от дома, в
районе поликлиники №11, семья попала под миномётный обстрел. Мама Жени погибла
на месте, Женя скончался на операционном столе в больнице.

Георгий Комаровский (21 июля 2007 – 18 декабря 2014 (7 лет)

Украинские военные в тот день начали очередной артиллерийский обстрел города
Углегорска. Семилетний Георгий Комаровский вместе со своим старшим братом
находился дома, когда один из снарядов попал в их дом, полностью его разрушив.
Братья погибли на месте.

Семья Булаевых: родители Олег и Татьяна Булаевы и их дети, Даниил и София

Во время артобстрела города Горловка, в дом, где жила семья Булаевых, попало
несколько снарядов. В этой квартире находилась молодая семья: папа, мама, сын и
дочь. В детской погибли восьмилетний Даниил и пятилетняя София. В большой
комнате – их отец Олег. Татьяна вскоре скончалась в больнице.

Александра и Кристина Коваленко (26 сентября 2008 – 23 августа 2014 (5 лет)

Во время обстрела ВСУ окраин города Донецка, снаряд попал в большую комнату,
где находилась вся семья. Огня не было, но вокруг всё рушилось. Ирина,
Александр и Кристина погибли на месте. Пятилетняя Сашенька умерла от
осколочного ранения в больнице.

Анастасия, Дарья и Кирилл Коноплёвы (дата гибели – 12 февраля 2015)

За три дня до вступления в силу второго «Минска», в Горловке случилась
трагедия. Во время обстрела ВСУ один из снарядов попал в кровлю дома, где
проживала семья Коноплёвых. Украинские военные отобрали у семьи Коноплёвых
бесценное – троих детей.

Прямое попадание украинского снаряда в жилой дом оборвало жизни
четырнадцатилетней Насти, семилетней Даши и полуторагодовалого Кирилла.
Вследствие детонации снаряда произошло обрушение стены и кровли в ванной
комнате, где в тот момент находилось трое детей семьи, они все погибли от
полученных травм.

Артём Логинов (15 мая 2009 – 8 сентября 2014 (5 лет)

Пятилетний Артём Логинов погиб от осколочного ранения в голову при обстреле
украинскими военными посёлка городского типа Нижняя Крынка, ДНР. Маленький
Артёмка не успел добежать до дома, когда начался обстрел.

Александр Смирнов (17 августа 2011 – 25 января 2015 (3 года)

24 января 2015 года Петровский район города Донецка в очередной раз подвергся
массированному артиллерийскому обстрелу со стороны украинских карателей. В дом,
в котором жила семья Смирновых попал снаряд: последовал сильный взрыв, кровля
крыши обрушилась, под завалами оказались отец и маленький Саша.

Мать пыталась вытащить своих родных из под обломков: спасатели и скорая помощь
в связи с продолжающимися обстрелами не смогли прибыть вовремя. Трёхлетний Саша
вместе с отцом погибли под завалами обрушившегося дома.

Дмитрий Поляк (20 апреля 2007 – 30 января 2015 (7 лет)

Углегорск обстреливали несколько месяцев подряд. Долгое время людям приходилось
прятаться в подвалах, лишь на короткое время выбираясь оттуда, чтобы взять еду
и тёплые вещи. В тот зимний январский день семья Поляк только на пять минут
заскочила в дом, и сразу же их всех накрыл снаряд, влетевший в крышу их дома.
Сильной ударной волной Диму отбросило на пол. Ребёнок погиб от травм,
несовместимых с жизнью.

Роман Мисечко (29 июня 1998 – 18 декабря 2014 (16 лет)

Рома Мисечко вместе со своим семилетним сводным братом Гришей сидели перед
телевизором, а их родители хлопотали во дворе частного дома. В тот момент,
когда в гости к мальчикам зашла восьмилетняя соседка Таня Чернобай, вокруг дома
с детьми начали рваться украинские снаряды. Один из них попал прямиком в дом.

Роман и Георгий получили ранения, несовместимые с жизнью. Таня Чернобай выжила,
но была тяжело ранена и получила 30\% ожогов кожи. Чуть позже ей пришлось
ампутировать ногу.

Иван Нестерук (20 августа 2010 – 4 июня 2015 (4 года)

В посёлке Тельманово (ДНР) была солнечная погода и дети играли в песочнице.
Ванина мать с маленькой дочерью сидели совсем рядом. Обстрел ВСУ посёлка из
«Града» начался внезапно. Никто не успел ничего сделать. Разрывы были
неподалеку. Дети разбежались, а четырёхлетний Ваня упал. Осколок пробил ему
лёгкое и застрял в позвоночнике.

«Я еще чуть-чуть полежу, а потом мы пойдём домой», – сказал он подбежавшему
отчиму. Это были последние слова Ванечки. Больше он никогда ничего не сказал.

Екатерина Тув (21 мая 2004 – 26 мая 2015 (11 лет)

Во время миномётного обстрела окраин Горловки с территории Дзержинска,
подконтрольного ВСУ, один из снарядов попал в дом семьи Тув. Под руинами дома
погибли отец семейства Юрий и старшая дочь Катя. Двухлетний Захар получил
контузию и ожог роговицы, осколок попал в спину, после трагедии он несколько
дней не мог открыть глаза и ещё долго кричал во сне. Новорождённая Милана была
исцарапана. Мать Анна лишилась руки, получила множественные переломы и разрыв
барабанных перепонок.

На похороны родных её привезли в инвалидной коляске, женщина не хотела верить в
то, что её любимых больше нет. Проститься с двумя жертвами украинской агрессии
пришли тысячи жителей Горловки.

Сергей Пивень (14 апреля 2008 – 16 августа 2014 (6 лет)

Серёжа Пивень вместе с отчимом, Сергеем Булыгиным погиб в результате
артиллерийского обстрела украинскими военными города Ждановка (ДНР). В
результате обстрела были ранены мать, Светлана Козырь (лишилась руки) и сестра,
Александра Пивень, 3 октября 2004 года рождения (ранение бедра).

Кира Жук (9 сентября 2013 – 27 июля 2014 (10 месяцев)

27 июля 2014 года город Горловка подвергся массированному обстрелу из РЗСО
«Град». В результате детонации боеприпасов были ранены и убиты мирные жители. В
числе жертв была и десятимесячная Кира Жук, которая вместе со своей мамой
Кристиной в тот момент находилась в сквере Коммунаров. Смерть застала женщину с
ребёнком на руках.

Случай приобрёл мировую огласку под названием «Горловская Мадонна».

Даниил Белых (28 октября 1999 – 24 августа 2014 (14 лет)

Даниил с друзьями пошёл на родник, чтобы набрать воды. Украинские военные
целенаправленно вели обстрел местности, где находились подростки. В результате
миномётного огня Даниил и другие дети погибли на месте. Тот летний воскресный
день стал последним днём жизни подростка

