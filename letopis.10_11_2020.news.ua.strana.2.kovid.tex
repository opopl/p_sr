% vim: keymap=russian-jcukenwin
%%beginhead 
 
%%file 10_11_2020.news.ua.strana.2.kovid
%%parent 10_11_2020
 
%%url https://strana.ua/articles/analysis/300198-kak-v-ukraine-zapretili-lechit-vse-krome-kovida-i-chto-teper-delat-bolnym.html
%%author 
%%tags 
%%title 
 
%%endhead 

\subsection{Платить, судиться или умереть. Как в Украине запретили лечить все, кроме ковида, и что теперь делать больным}
\label{sec:10_11_2020.news.ua.strana.2.kovid}
\Purl{https://strana.ua/articles/analysis/300198-kak-v-ukraine-zapretili-lechit-vse-krome-kovida-i-chto-teper-delat-bolnym.html}

\Pauthor{Ксенз, Людмила}

\ifcmt
img_begin 
  url https://strana.ua/img/article/3001/98_main.jpeg
  caption В Минздраве считают, что украинцы могут полечиться и после эпидемии. Фото из открытых источников 
  width 0.7
img_end
\fi

В Украине из-за коронавируса ставят на стоп плановые госпитализации и операции.
Как сообщил главный санитарный врач Виктор Ляшко, теперь больницы будут
принимать только ургентных пациентов (то есть по скорой помощи). 

В Минздраве пояснили: это является "вторым этапом реагирования на
распространение коронавируса".

Главных целей две. Во-первых, не допустить вспышек коронавируса в самих
больницах (как показывает зарубежная статистика, там "корону" подхватывает до
30-40\% пациентов), во-вторых, освободить медучреждения не инфекционного
профиля под прием ковидных больных, ведь в отведенных под коронавирус больницах
свободные койки во многих регионах уже фактически закончились.

Стоит отметить, что подобные запреты уже были во время весеннего карантина. Но
действовали они недолго. Кроме того, тогда врачи быстро нашли схему их обхода:
плановых пациентов просто записывали как "срочных", понятно, за дополнительный
гонорар врачу.

Как считает глава ассоциации "Медицинское право в Украине" Виктор Сердюк, в
этот раз такая схема (и соответствующее удорожание операций) тоже будет. Но
есть одно но: многие больницы и врачей забирают под лечение коронавируса. Это
значит, что лечить "обычные" болезни и проводить операции будет попросту негде
и некому.

Что может стать серьезной проблемой для украинцев. Ведь пока люди будут ждать
окончания эпидемии, их болезни могут перейти в действительно "неотложные
состояния".

Виктор Сердюк говорит, что граждане имеют все основания идти в суд - ведь
нарушается их конституционное право на получение медпомощи.

Впрочем, как говорит глава адвокатского объединения "Кравец и партнеры"
Ростислав Кравец, доказать в суде вину Минздрава или Кабмина будет непросто.
"Особенно, если речь идет о такой тонкой материи, как ухудшение здоровья. Не
так просто связать его с отказом, скажем, в плановой госпитализации или
операции", --- пояснил юрист.

"Страна" разбиралась, почему из-за коронавируса украинцев прекращают лечить от
других болезней.

\subsubsection{Запретили лечить}

Минздрав, поясняя запрет на плановые госпитализации и операции, ссылается на
постановление Кабмина № 641 от 22 июля 2020 года об установлении карантинных
ограничений. В частности, для оранжевой зоны (а сейчас это, по-сути, вся
территория Украины) там прописано, что лечебным учреждениям запрещено принимать
плановых больных за исключением ряда случаев. В частности, ограничения не
распространяются:

\begin{itemize}
\item на медпомощь при родах и их последствиях
\item медпомощь новорожденным
\item медпомощь с специализированных онкологических отделениях
\item паллиативную медпомощь
\item другие неотложные состояния, если вследствие переноса сроков госпитализаций или
операций существуют значительные риски для жизни или здоровья
людей.
\end{itemize}

То есть, как видно, существует ряд исключений.

Стоит отметить, что весной, когда на волне жесткого карантина уже были похожие
ограничения, врачи научились их обходить. Многих плановых пациентов просто
оформляли "по скорой", правда, за дополнительное вознаграждение врачам. Как
говорят сами пациенты, операции обходились им на 20-30\% дороже, чем было бы в
"мирное время". Но так как выхода не было, люди соглашались платить.

Но в этот раз ситуация более серьезная. Весной все ограничительные мероприятия
носили скорее превентивный характер, ведь тогда реальные показатели
коронавирусной статистики были более-менее благополучными. Сейчас же ситуация
кардинально изменилась.

Как рассказали "Стране" главврачи нескольких больниц, из местных управлений
здравоохранения поступают "устные рекомендации" не брать "лишних" пациентов.
Ведь койки под коронавирусных пациентов в стране реально заканчиваются (а в
ряде регионов их уже и вовсе нет), поэтому все лечебные учреждения находятся в
«повышенной боевой готовности» - их в любой момент могут перевести в разряд
"коронавирусных", а некоторые уже переводят. К примеру, под ковид закрывают
один из роддомов в Чернигове.

Поэтому в этот раз запрет на плановые госпитализации будет более жестким, чем
весной.

\subsubsection{Ждать ухудшения или платить "сверху" до 15 тысяч}

Медицинское сообщество уже активно обсуждает новые запреты.

"Отменены плановые операции. Конечно, это неоказание необходимой медицинской
помощи и дискриминация больных. Раз уж вы (Минздрав - Ред.) смогли насчитать
250 тысяч коек в стране и 50 тысяч отдали под ковид то хотелось бы понять
предназначение еще 200 тысяч", --- написал на своей странице в Фейсбуке известный
врач-инфекционист Андрей Волянский.

"Большинство плановых операций назначается по жизненным показаниям. И при
многих составляющих ждать неделю - это уже очень долго. Человек, может, и не
умрет, но болезнь будет запущена, а здоровье явно ухудшится. Непонятно, почему
вдруг решили ограничить конституционное право украинцев на медпомощь", —
удивляется глава Национальной медицинской палаты Сергей Кравченко. 

По словам Виктора Сердюка, в новых ограничения есть как плюсы, так и минусы.

Положительный момент - за счет снижения числа госпитализаций удастся не
допустить вспышек коронавируса в больницах. "Есть европейская статистика,
которая показывает, что из-за внутрибольничного распространения коронавируса
может заразиться 30-40\% пациентов. По Украине статистики нет, так как не было
соответствующих эпидемиологических расследований. Но стоит отметить, что такие
цифры заражений были раньше, когда еще массово не применяли "быстрые" тесты на
ковид-антиген, позволяющие определить коронавирус буквально с первых дней
заражения. То есть, в теории, если тестировать пациентов при госпитализации и
дальше 1-2 раза в неделю, то риски можно значительно снизить", --- говорит
Сердюк.

Минусов в запретах на госпитализацию и операции, по его словам, значительно больше, чем плюсов.

"Многие состояния здоровья, которые сейчас выглядят как "несрочные", можно
запустить. Скажем, с грыжей человек может ходить и год. Но если произойдет ее
защемление, есть риск перитонита, а там смертность 25\%. Точно так же по
сердечным и многим другим болезням. Неужели человеку с язвой, чтобы получить
медпомощь, нужно ждать желудочного кровотечения", --- возмущается эксперт.

Он отмечает, что, как и в прошлый раз, врачи будут обходить запреты. "Доктора,
конечно же, будут входить в положение, и к тому же они не против получить
благодарность от пациента. Поэтому станут оформлять плановое лечение как
экстренное", --- говорит Сердюк.

Киевлянка Светлана весной, когда  действовали ограничения на плановые
госпитализации, чтобы оформиться в кардиологическое отделение, заплатила врачу
сверху озвученной ранее суммы еще 2 тысячи гривен. "Доктор сказал: вы же
понимаете, я рискую. Если будет проверка и обнаружится, что вы не "срочная",
накажут меня. Пришлось заплатить, так как чувствовала себя плохо, а домашнее
лечение не помогало. Но это я еще дешево отделалась. За то, чтобы врачи взялись
за плановую операцию, люди платили сверху по 10-15 тысяч гривен и больше", -
вспоминает Светлана. 

\subsubsection{\enquote{Время потеряно, остается констатировать коллапс медицины}}

Впрочем, в этот раз не все можно будет решить деньгами.

На самом деле власти не столько беспокоятся о том, чтобы уберечь украинцев от
рисков заражения коронавирусом в больницах, сколько о том, что заканчиваются
койки в ковидных отделениях. Это значит, что принимать коронавирусных больных
будут и обычные больницы.

На сегодняшний день под коронавирус в украинских больницах отведено порядка 53
тысяч коек. Как заявил недавно министр здравоохранения Максим Степанов, всего
может быть задействовано 83 тысячи коек. 

И не исключено, что, если эпидемия будет набирать обороты, эту цифру придется
увеличивать. Всего в отечественных медучреждениях порядка 270 тысяч коек. То
есть получается, остальные места оказались в своего рода "резерве". Плановые
госпитализации и операции им проводить нельзя, но и коронавирусных пациентов
они пока тоже принимать не будут. Следовательно, станут просто простаивать и
ждать времени "Ч".

Во врачебном сообществе обсуждается версия, что таким образом власти не только
перестраховываются, но еще и хотят сэкономить. Ведь за время вынужденного
простоя больницам будут платить по-минимуму - деньги от НСЗУ за количество
пролеченных случаев им не положены (так как нет "случаев"), а по ковидным
пакетам оплаты также не будет, поскольку пока нет пациентов и специальных
медицинских бригад для их лечения.

Но при этом, как отмечает Сергей Кравченко, вовсе не факт, что освобожденные за
счет отказов от плановых больных койки подойдут для лечения коронавирусных
больных.

"Там главное условия - кислород (централизованная подача или хотя бы
концентратор). Его нет, и быстро решить этот вопрос уже не получится даже при
наличии денег. Госказначество закрывает финансирование по бюджетным программам
в конце декабря. Но еще за две недели до этого начинается аудит использованных
средств, то есть фактически получить деньги уже невозможно. Остается всего
месяц, и за это время больницы должны разработать и согласовать документацию,
выбрать подрядчиков, провести работы. Обычно на это уходит не менее трех
месяцев. Я уже не говорю о том, что, скажем, купить кислородный концентратор
сейчас - большая проблема, так как они попали в разряд дефицитных. Время,
которое у нас было для подготовки к пику эпидемии, а это 8 месяцев,
безвозвратно потеряно. Теперь остается разве что констатировать медицинскую
катастрофу", --- говорит Кравченко.

Отдельный вопрос - по врачам. Минздрав уже собирает данные о медперсонале,
включая врачей-интернов - у больниц затребовали личные данные медиков, что, как
поясняют в лечебных учреждениях, связано с готовящейся мобилизацией докторов
под лечение ковида. 

"Но при этом не было массового обучения врачей неинфекционных специальностей.
Как, скажем, проктолог, должен лечить ковид? Это, конечно, лучше, чем если бы у
койки корнавирусного больного поставили уборщицу или официанта, но проблему не
решает", --- добавил Виктор Сердюк.

Эксперты говорят, что во врачебном сообществе зреют массовые протесты - медики,
которых, по сути, оставляют без денег и собираются "мобилизовать" на борьбу с
эпидемией, все чаще подумывают об увольнениях.

\subsubsection{Как засудить Минздрав}

Виктор Сердюк говорит, что пациенты, которым откажут в госпитализации или
операции, вполне могут судиться. "Нужно предоставить заключение врача о
необходимости стационарного лечения и официальный отказ от госпитализации,
связанный с распоряжением Минздрава. Это может стать достаточным основанием для
подачи иска", - уверен Сердюк.

Кравченко также считает, что больницы могут отказаться выполнять распоряжения
Кабмина и Минздрава. "У нас прошла автономизация лечебных учреждений. Это
значит, что формально все решения Кабмина носят для них лишь рекомендательный
характер. Чтобы ограничения вступили в силу, должно быть еще решение
собственника больницы, к примеру, местного органа власти. Но в этом случае уже
он, а не Минздрав, будет нести ответственность за непредоставление медпомощи.
Возможно, поэтому местные власти пока ограничиваются в основном устными
указаниями на этот счет", --- отмечает Сергей Кравченко.

Как пояснил нам Ростислав Кравец, в своем прежнем решении относительно
законности жесткого карантина, Конституционный суд уже признал, что ограничение
на лечение, образование, свободу передвижения и т.п могут вводиться только при
объявлении в стране чрезвычайного или военного положения.

Формально это значит, что люди могут подавать в суд на Кабмин и Минздрав за
запрет на госпитализации. Но выиграть его будет нелегко, - предупреждает юрист.

"В случае, если пациент из-за несвоевременно проведенной операции умрет, и это
будет подтверждено соответствующим медицинским заключением, суд, возможно, и
станет на стену истца. Но если речь будет идти, скажем, об ухудшении состояния
здоровья, то тут доказать вину будет сложнее, так как сложно сказать, как бы
себя чувствовал больной, если бы операцию ему провели вовремя. То есть, нюансов
очень много", --- говорит юрист. Другими словами, чтобы добиться правды в суде,
человек должен разве что умереть. Понятно, что ждать такого исхода никто не
будет, и люди предпочтут "искать выходы" или переплачивать врачам, чтобы решить
проблему со здоровьем.
