% vim: keymap=russian-jcukenwin
%%beginhead 
 
%%file 10_12_2021.fb.gorovyj_ruslan.1.vna_ukraine.cmt
%%parent 10_12_2021.fb.gorovyj_ruslan.1.vna_ukraine
 
%%url 
 
%%author_id 
%%date 
 
%%tags 
%%title 
 
%%endhead 
\subsubsection{Коментарі}
\label{sec:10_12_2021.fb.gorovyj_ruslan.1.vna_ukraine.cmt}

\begin{itemize} % {
\iusr{Тетяна Самборська}

ну не можна пережити це \enquote{на}. євроньз російськомовне - на Украине, але
в Донбассе. ага, розкажіть, що це норма русскага язика, ага-ага

\begin{itemize} % {
\iusr{Тетяна Самборська}

нє, це подвійні стандарти. на украине - тому що це просто територія, край,
невизнання державного утворення, наприклад, на алтае, на колимі, але в донбассе
- бо так мовленнєво надається суб'єктності тому, що там робиться.

я не знаю, як в росії вживають, в укр мові виключно в Узбекістані

\iusr{Світлана Проценко}

нмд, левова частка росіян просто не знає російську.

точніше, знає не літературну, а таке собі російське \enquote{кокні}.

бо, згідно їх літературних правил

"Предлог \enquote{в} употребляется с административно-географическими названиями (в городе, в России).

Предлог \enquote{на} употребляется

-с названиями горных областей местности без точно очерченных границ (на Кавказе, на Урале);
-с названиями островов и островных государств"

і усьо.
\end{itemize} % }

\iusr{Максим Мошковский}

Кстаті, в поляків та ж проблема з цим «на». І теж скільки не виправляй, все
одно так кажуть(Імперське минуле воно таке

\begin{itemize} % {
\iusr{Тетяна Носир}
\textbf{Max Moshkovsky} не чула, щоб десь казали " на Польщі"!

\iusr{Максим Мошковский}
\textbf{Тетяна Носир} так поляки кажуть щодо нас «на Україні». Я їх не виправдовую, а кажу шо в них така ж хвороба як і в мсклв.

\iusr{Тетяна Носир}
\textbf{Max Moshkovsky} щодо нас, то це не дивно! Вони Україну теж вважають колонією.

\iusr{Halyna Chumak}
\textbf{Max Moshkovsky} Поляци ще кажуть - на східних кресах, отут мають лунати постріли!
\end{itemize} % }

\iusr{Elmira Novosolova}
Або ось ще

\ifcmt
  ig https://scontent-mxp1-1.xx.fbcdn.net/v/t39.30808-6/266072386_4717326364994036_5996361768814814955_n.jpg?_nc_cat=107&ccb=1-5&_nc_sid=dbeb18&_nc_ohc=2FvD_6YbXoQAX90m3nf&_nc_ht=scontent-mxp1-1.xx&oh=d3a38305ebe474a3ea338c5855c2b6b1&oe=61B75D7A
  @width 0.4
\fi

\begin{itemize} % {
\iusr{Олександр Пшенік}
\textbf{Elmira Novosolova}

\ifcmt
  ig https://scontent-mxp1-1.xx.fbcdn.net/v/t39.30808-6/266050670_709871956656097_5550904604684092128_n.jpg?_nc_cat=103&ccb=1-5&_nc_sid=dbeb18&_nc_ohc=-CLZ2do822wAX8i-GXd&_nc_ht=scontent-mxp1-1.xx&oh=65200703cadb46e02b8d999a04ca6e0b&oe=61B85BDD
  @width 0.4
\fi

\end{itemize} % }

\iusr{Любов Ломакіна}
Норма для них - неповага до усіх і всього і до себе самих в першу чергу.

\iusr{Олена Добровольська}
Хуйорма

\iusr{Сергій Марченко}
Оце ти задав квест, Горовий!! минати довбойобів не так просто
@igg{fbicon.face.tears.of.joy}{repeat=4} 

\begin{itemize} % {
\iusr{Ruslan Gorovyi}
нада пробувать
\end{itemize} % }

\iusr{Ірина Курбатова}
Діствітєльна.. от яка мсклю різниця де він буде лежать - \enquote{на} землі чи \enquote{в} землі

\iusr{Тетяна Носир}

Ніхто не каже на Франції чи на Іспанії! Це спосіб не визнавати Україну, як
державу, причому це свідомо кажуть ведучі телеканалів Росії, політики, а не
тільки пересічні росіяни. На Росії вважають Україну колонією, як і колись!

\begin{itemize} % {
\iusr{Ruslan Gorovyi}
\textbf{Тетяна Носир} та я ж все розумію

\iusr{Олександр Пшенік}
\textbf{Тетяна Носир} це єдина причина впертого порушення правил власної мови тупими московитами

\iusr{Тетяна Носир}
\textbf{Олександр Пшенік} не тільки московитами, а і тими, кому \enquote{какая разніца}!
\end{itemize} % }

\iusr{Вероніка Някшу}

Тобто в руском язикє є норми для всіх загалом, а для України - винятки? Це ж як
треба сцятися, щоб вигадувати винятки @igg{fbicon.thinking.face}{repeat=3} 

\iusr{Helgis Gerus}
Канєшно це норма для них - ніхуя не знати їх ізиг..

\ifcmt
  ig https://scontent-mxp1-1.xx.fbcdn.net/v/t39.30808-6/266069840_867930497221287_1365534711406210897_n.jpg?_nc_cat=111&ccb=1-5&_nc_sid=dbeb18&_nc_ohc=SAI3HpqxY34AX9PC3CZ&_nc_ht=scontent-mxp1-1.xx&oh=4b660ef6915db98a6eada8b4d820f1cc&oe=61B7A8D3
  @width 0.4
\fi

\begin{itemize} % {
\iusr{Мария Скакунова}
\textbf{Helgis Gerus} зберегла собі) дивно, але якщо «с названиями горных областей употребляется предлог на: на Алтае..» то мало би бути рос На Карпатах?

\iusr{Helgis Gerus}
\textbf{Мария}

Норми з часом змінювалися.

У Шевченка є \enquote{на Україну}.

Але тут йдеться про те, що у період підготовки інтервенції хрюкогавських
запорєбріків з 2008 року ( нагугліть це інтерв'ю) Пуйло і його поплічники в
інфопросторі в усіх усюдах вбивають у свідомість пересічного, що \enquote{Україна не
відбулася, як держава}, що вона лише ареал, територія, НА якій щось
відбувається.

Якийсь легінь навіть зробив частотний аналіз - у період підготовки до
вторгнення \enquote{В Україну} в текстах запорєбриків змінилося \enquote{НА Україну} і
слугувало тій меті.

А скрін вище це доводить - для АДМІНІСТРАТИВНИХ одиниць вживається і за
хрюкогавськими нормативами \enquote{В}

перехід на \enquote{НА} стосовно Украіни слугував лише тому, щоб підкреслити доктрину
про те, що Украіна - лише територія з втраченою державністю.

\iusr{Катерина Нігай}
\textbf{Mariya Denyshchenko} на Альпах)))
на Андах)))

\iusr{Helgis Gerus}
\textbf{Мария}
на Закарпатті, але в Карпатах, чи не так?

\iusr{Мария Скакунова}
\textbf{Helgis Gerus} це я чудово розумію. Мене наступне речення про «гірські» здивувало. Але, якщо мова про адміністративні гірські області, то ясно.

Мария Скакунова
Helgis Gerus це Українською. Правопис Розенталя - це рос мова ж?

\iusr{Мария Скакунова}
\textbf{Helgis Gerus} я просто не так добре знаю російську. Але, власне, прагнення вивчити досконало їхню граматику в мене теж нема))

\iusr{Helgis Gerus}
\textbf{Мария}
Ну, я колись задля перевірки знайшов той текст в оригіналі - там багато чого дивного, але саме "в Україні" навіть у них усталена норма
\end{itemize} % }

\iusr{Ковалевський Андрій}

Завжди питаю в тих зайвохромасомних якщо їхати \enquote{на Україну} то як буде
вертатися? Входять у ступор а потім говорять \enquote{еду из України}, бо щось заважає
\enquote{язику} казати \enquote{с України}! Але потім продовжують своє- це невиліковно!

\begin{itemize} % {
\iusr{Роман Дондоха}
\textbf{Ковалевський Андрій} це просто природня) на жаль не виліковна ЗВЕРХНІСТЬ ! Але культурою .....,та потужною армією ,це лікується ))
\end{itemize} % }

\iusr{Max Pobokin}
Пойти в органы или пойти на органы)))

\iusr{Дудковський Микола Іванович}

\ifcmt
  ig https://scontent-mxp1-1.xx.fbcdn.net/v/t39.30808-6/266335730_1870814813125465_6079447099834549370_n.jpg?_nc_cat=108&ccb=1-5&_nc_sid=dbeb18&_nc_ohc=ZC4Oi0H8xYEAX99Z4gV&_nc_ht=scontent-mxp1-1.xx&oh=00_AT8jKsvqMUL4PBRTfr7HUN0ix8NO7MgvuaknxH_UtD3GIw&oe=61B8F123
  @width 0.4
\fi

\iusr{Любов Крижановська}
Ти знаєш мене вже ніц не дивує.

Взагалі!

Наша Галька ( жінка брата мого Романа) яка є українкою, львів'янкою хоч і живе
20 років на раші. Цього року літом їздила в Крим до падружкі.

А падружці, москвичці дали якимось дивним чином дім в Криму. Зауваж,дали. Це
напевно в когось його забрали. Бо фото я бачила в інстаграм то там типовий
татарський дім.


\iusr{Abram Kats}
в мене вже давнненько всі на расейе живуть

\iusr{В'ячеслав Маноха}
Граматика - останній притулок російського ліберала.

\iusr{Ярослав Матюшин}

Так склалось, що я носій російської мови та лінгвист.

Це все повна хуйня, стосовно норми язика.

В кожній мові є денотат (пряма номінативна функція) та коннотат
(супроводжувальний зміст). Ну штибу кіт та котєнька. Котєнька - це вже
прагматичне відношення суб‘єкта сінтагми, яке ілюструє любов до рижої жопи.

Та ось це «на Україну» - це підсвідомий коннотат, який передбачає, що на нашу
країну можно нападати.

В Україні живуть. На Україну нападають.

\begin{itemize} % {
\iusr{Ruslan Gorovyi}
я підозрював, шо з тобою шось не так, бро... а лінгвіст якось виводиться?

\iusr{Ярослав Матюшин}
\textbf{Ruslan Gorovyi} айрішем.

\iusr{Ruslan Gorovyi}
\textbf{Ярослав Матюшин} фух... пийте більше...нуй нах

\iusr{Margareta Poustovit}
\textbf{Ярослав Матюшин} о. Поширю собі. Дякую
\end{itemize} % }

\iusr{Тетяна Шкуренко}
Це норма імперського рузького, а не руского

\iusr{Олег Вовкодав}

Та прикидаються просто ті московити і ховаються за нормами так званого рускаго
язика. Чомусь ті самі норми не заважають їм частенько казати \enquote{в (!) Данбассє}

\iusr{Богдана Руда}

Від росіян то відносно (!)очікувано, проте у нас виявляється навіть
Держказначейство вважає, що «на Україні» то норм @igg{fbicon.face.smirking} 

\href{https://www.facebook.com/bogdana.ruda/posts/4653629018082354}{%
Державна казначейська служба України, ви чого там подуріли?!, Богдана Руда, facebook, 09.12.2021%
}

\iusr{Сергій Мельковський}
Чуєте \enquote{на}, посилайте туди ж.

\iusr{Людмила Лещук}

Їхня єзикавая норма не з повітря взялася, а з імперського історичного
контексту. Тому - хто шо більше любить, те тому і норма.

\iusr{Oleksiy Zubenko}

\enquote{The Ukraine} теж довгий час було нормою в англійській мові. Поміняли на наше
прохання, бо \enquote{the} ставиться перед власними назвами територій, але не
незалежних країн (за винятками \enquote{The United States}, \enquote{The United Kingdom} та
іншими, де в назві є \enquote{штати} \enquote{королівство}, \enquote{союз} чи щось інше). Відтак \enquote{The
Ukraine} з'явилося через сприйняття України як союзної республіки, для окремої
держави це некоректно. Попросили - і не одразу, але таки поміняли.

А от справжнім братам-слов'янам не до того, щоб заморочуватися такими речами -
у них ґєнацид русскава єзика і нєонацистскій шабаш в русскам горадє Кієвє!

\iusr{Микола Гаврилко}
А \enquote{в Донбассє} - це теж норма язика?

\begin{itemize} % {
\iusr{Serhiy Butkov}
\textbf{Микола Гаврилко} тут потрібен контекст. Якщо рашиста вже закопали, то в
Донбасі, якщо він його ще топче своїми копитами, то на Донбасі.
\end{itemize} % }

\iusr{Акинари Симамура}

Еге ж, тільки \enquote{в} було рекомендованою нормою до 2008 року, аж поки \enquote{коєкто} не
виступив у Бухаресті і не розкрив очі всьому світу на \enquote{історічєскоє
нєдорозумєніє}. І тоді лінгвісти єво царсково вєлічєства кинулися переписувати
довідники, а в імперську Вікіпедію додали бота, який автоматично виправляє \enquote{в}
на \enquote{так сложилось історічєскі} (с).


\iusr{Алла Лебедєва}
Тоді хай кажуть: на Франції, на Німеччині, на Штатах.

\iusr{Valerya Boeva}
Та в них норма чуже пиZдити, вони собі які хотять норми, такі і пишуть. Тільки от вони з нормами цивілізованого світу не співпадають.

\iusr{Світлана Соляник}
я завше картинкою відповідаю - ображаються чогось

\ifcmt
  ig https://scontent-mxp1-1.xx.fbcdn.net/v/t39.30808-6/265598835_4696721353726719_8309002166858037172_n.jpg?_nc_cat=105&ccb=1-5&_nc_sid=dbeb18&_nc_ohc=2oRW3PJWzEgAX8ijNnm&_nc_ht=scontent-mxp1-1.xx&oh=00_AT--2W27fF_RevRAW6zQJxbUPO8aJFYirrbA6ULh9ljobg&oe=61B8A360
  @width 0.4
\fi

\iusr{Olga Vladimirovna}

- Какая тебе разница \enquote{в Украине} или \enquote{на Украине}?

- А тобі є різниця - лежати На землі чи В землі?

\iusr{Николай Петрович}
С последним пунктом есть частичные проблемы)

\iusr{Лариса Владимирова}

людина, яка стверджує, що «на Україні» - це норма російської мови, насправді
російської мови не знає. Нагадайте йому, що за радянських часів звикли казати
«на Україні» бо мали на увазі територію. Це як на Кубані, на Донбасі. Але
казали «в УРСР» якщо мали на увазі республіку. Коли кажуть «на Україні»,
свідомо чи не свідомо визнають Україну не державою, а так собі - територією. До
речі, ніхто не каже «на Білорусі» чи «на Казахстані», нікоди не чула «на
Молдові» чи «на Грузії». Ці «норми» російської мови чомусь стосуються лише
України.


\iusr{Oleg Dergatch}

Намагаюсь в таких випадках бути толерантним і зваженим: на кожне 'на Україні'
відповідаю рівно одним 'на росії'. Воно працює!  @igg{fbicon.smile} 

\iusr{Ольга Хотячук}
що це за \enquote{норма} якась особлива лише для України?

\iusr{Олександр Пшенік}
Є дуже чітке роз'яснення (на скріні)

Від себе ж додам, що в московській мові НА використовується, коли йдеться про
території - на Уралі, на Далекому сході - а В використовується з країнами - В
Україні, В росії, в Америці. Поки Україна була окупована Московією, казати \enquote{на
Україні} було правильно згідно правил їхньої мови. З моменту набуття Україною
незалежності так казати безграмотно. Нам потрібно розуміти, що кожен московит,
що досі каже \enquote{на Україні}, є звичайним імперцем та шовіністом, якому наша
незалежність надто муляє.

*в мовному питанні я експерт, бо закінчив російськомовну школу з медаллю і
потім вивчав мовознавство в інституті іноземних мов.

\ifcmt
  ig https://scontent-mxp1-1.xx.fbcdn.net/v/t39.30808-6/266020611_709860366657256_3445226023283646911_n.jpg?_nc_cat=102&ccb=1-5&_nc_sid=dbeb18&_nc_ohc=TrPheVIPbPoAX93ZZiZ&_nc_ht=scontent-mxp1-1.xx&oh=1b35ab4cbc3691c412b069fc97fafe65&oe=61B912CD
  @width 0.4
\fi

\iusr{Алексей Меркулов}
А коли наші місцеві кажуть \enquote{укрАїнська}??

\iusr{Ruslan Gorovyi}
це і є прояв впливу імперії

\iusr{Петро Москалець}

Навіть по їхніх правилах правопису треба казати \enquote{В} в даному випадку.
Але вони тупі...

\begin{itemize} % {
\iusr{Максим Руженський}
\textbf{Петро Москалець}

\href{https://news.obozrevatel.com/politics/18654-pelevin-obyasnil-pochemu-govorit-na-ukraine-nepravilno.htm}{%
Пелевин объяснил, почему в российских словарях осталось "на Украине", news.obozrevatel.com, 20.03.2014%
}

\end{itemize} % }

\iusr{Катерина Нігай}
Я за останні роки так звикла казати «на Росії», що вже «в Росії» мені ріже слух)))

\iusr{Irena Khodurska}
Какая разница (С).... \enquote{НА} земле или \enquote{В} земле (сарказм)

\iusr{Оксана Городецкая}
Тобто НА росіі це норма?))))

\iusr{Тетяна Чеховська-Косцова}

\enquote{на} - це наступ і агресія. Ніби показує напрямок удару.  \enquote{На
Берлін}- це про наступ, \enquote{в Берлін} - щось можливо туристичне.  Так і
сприймаю те \enquote{на}.  В останні роки говорю і пишу на росію, на московію.
І з маленької. Коли закінчиться війна - перегляну норми граматики. Поки так.

\iusr{Olga Parkhomenko}
на Росії, вже по-іншому складно говорити)

\begin{itemize} % {
\iusr{Ruslan Gorovyi}
\textbf{Olga Parkhomenko} в правилах українського правопису з діда прадіду Московія... нема россіі
\end{itemize} % }

\iusr{Vitaliy Kolomiets}
На болотах

\iusr{Ruslan Gorovyi}
\textbf{Vitaliy Kolomiets} наголос на останній склад причому

\iusr{Александр Курбатов}

За поребриком навіть своєї мови не знають. Правильно \enquote{в Україні} і
\enquote{на мошканському болоті} - мова дружби!

\iusr{Богдан Савко}
Це аби повийобуватися.

\iusr{Yuriy Dyachenko}

\enquote{на} - це навіть російською неправильно, якщо, звісно, норм мови притримуватися, а не політичної заангажованості.
Ця \enquote{псевдонорма} з'явилася у 2013 з подачі Путіна, коли він вирішив захоплювати Україну.
А за правилами російської мови: \enquote{в Україні}

\end{itemize} % }
