%%beginhead 
 
%%file 31_03_2023.fb.korovanenkova_natalia.mariupol.1.29_30_marta_2022
%%parent 31_03_2023
 
%%url https://www.facebook.com/natali.korovanenkova/posts/pfbid033cdQwmjEbhuh3r44Pa1CjyDJJKzqiw7dnMPitRGN5KB77bV5pmyCcvBkKhwi5Uxul
 
%%author_id korovanenkova_natalia.mariupol
%%date 31_03_2023
 
%%tags mariupol,mariupol.war
%%title 29-30 марта 2022
 
%%endhead 

\subsection{29-30 марта 2022}
\label{sec:31_03_2023.fb.korovanenkova_natalia.mariupol.1.29_30_marta_2022}

\Purl{https://www.facebook.com/natali.korovanenkova/posts/pfbid033cdQwmjEbhuh3r44Pa1CjyDJJKzqiw7dnMPitRGN5KB77bV5pmyCcvBkKhwi5Uxul}
\ifcmt
 author_begin
   author_id korovanenkova_natalia.mariupol
 author_end
\fi

\#маріуполь\_2022 \#як\_це\_було

29-30 марта 2022

Мост по пр Металлургов разминировали, поставили блок пост,чтобы пройти  через
мост и спустится к речке набрать воды , надо показать документы ,сказать куда и
за чем ты идешь. Спустившись на речку ,набрав воды, на обратном пути снова
показать документы .

Сарафаное радио работает быстро , на Метро дают гуманитарку 

Некоторые пошли на Метро ,путь от нас не близкий, на выезде из города.

За гуманитаркой  очередь, огромная толпа грязных ,испуганых, голодных людей.

Можно что то купить  и за наличку,  сигареты 100гр пачка , нам приносили по
200-300 гр, накрутка за доставку и риск, пройти через весь город был большой
риск

На этом доставка "продуктов" заканчивалась.

Я в первый день ходила четыре или пять раз, через мост и блок пост, мне нужна
была не так вода , как информация , как выбраться из города. 

На блок посту трое  , один из них молоденький совсем  военный и два  днр\_овца,
один злой с пустыми глазами ,другой лет за пятьдесят, вид у него как у
партизана вышедшего из леса , который затерялся в сороковых годах. Вот с ним
можно общаться, он отвечал , что знал  , говорил на суржике.

Я примелькалась за день своей красной шапочкой, документы у меня уже не
спрашивали , зато узнала,  что от Метро ходят атобусы  и вывозят в разных
направлениях, в каких и куда никто не знал.

Еще мне очень нужно было попасть на Кировский узнать живы ли ...,  если
получится попытаться дойти до сестры  это  ближе к Драму. 

.

а еще... мне очень нужна была расческа , вот прямо край, как нужна))))мои
длинные волосы превратились в гнездо из штукатурки с приросшей  к ним шапкой.

На утро следующего дня я уговорила  Ирину и  мы двинулись в путь, Прошли мост,
вдоль него  лежали мины ,их еще не убрали.

На блок посту спросили разрешения пройти дальше,  перешли на другую сторону
проспекта , там стоял молоденький военный.Подходим к нему , спрашиваю, как
безопастнее  нам пройти,через дворы или по проспекту, где меньше стреляют?

На что он отвечает:"Я не местный, я не знаю."

У меня не произвольно вырывается:"а какого х\_@\_я ты здесь делаешь неместный((("

Ира хватает меня за руку ,тянет и повторяет :" быстрее, уходим сейчас
стрельнет"

Идем на Кировский ,заходим во двор , подымаемся на нужный мне этаж, узнаю, что
все живы ,  

Мне дают расческу и банку варенья , за что  , огромное спасибо)

Двигаемся дальше по разбитому проспекту , пустые окна домов ,оборванные траллеи
, провода , танки на всречу на бешенной скорости, надо быстро отбежать подальше
,чтобы тебя не задело и не перезало проводом .

Добрались до центрального РОВД переходим на Куинджи,  все разбито. лежат люди .

Ире становится плохо. Днр\_овцы разгружают машину с продуктами, прошу воды...
швыряют в нас бутылкой с водой ..

Заходим во двор к сестре , одни обломки , ее подезд вроде не сильно пострадал ,
возле входа сгоревшие машины,вместо первого этажа  дыра , но лестничный пролет
цел.

В подезде следы костера , лежат подушки, одеяла , остатки чего то или кого
то.Становится жутко.

Продолжаю подымаюсь на этаж ,дверь цела и закрыта ,нет никого.

Все .Возвращаемся.

Выходим на проспект,мужик на тачке катит огромный стелаж с пепсиколой, Прошу
бутылку.дает нам две , я ничего вкуснее оказывается не пила..

На Кировском  толпа ,стоят две машины, дают ХЛЕБ,мы становимся в очередь и
получаем два заветных батона, у кого есть дети , дают еще и бублики , но что бы
их получить ребенку надо выйти с подвала/укрытия ..

При этом все снимают на камеру.

Еще одна машина с газетами за 23 февраля, днр новостями.  

Я к машине: "Дайте мне! А можно несколько???"

Военный радостно кивает:"Конечно, читайте новости"

Ирина мне шипит на ухо :"Только  молчи!!!!"

Она знает для чего мне газеты,.  Постелить собакам для туалета.

И тут я вижу , его! А в т о б у с! 

Который забирает людей на выезд

Автобус старенький ,люди его облепили, открыта только передняя дверь в нее
пытаются вниснуться с детьми ,сумками...

Подхожу к водителю, начинаю распрашивать. Автобус приезжает каждое утро ,везет
на Метро и выезд из города на блок пост , а там  много автобусов , которые
вывозят из города.

Прошу водителя , что мне надо с собаками, через такую толпу не влезу,я заплачу)

На что он отвечает:"Подойдешь завтра , когда я подеду и открою переднюю дверь
ты иди к задней,  как подойдешь я тебе ее открою,ступеньки высокие , влазь если
сможешь."

Смогу!еще как смогу!

Это будет завтра,а сегодня хлеб и пепси.

Фото с инета ,но мост наш , и место где мы брали воду

%\ii{31_03_2023.fb.korovanenkova_natalia.mariupol.1.29_30_marta_2022.cmt}
