% vim: keymap=russian-jcukenwin
%%beginhead 
 
%%file slova.slovo
%%parent slova
 
%%url 
 
%%author 
%%author_id 
%%author_url 
 
%%tags 
%%title 
 
%%endhead 
\chapter{Слово}

Рада после \emph{слов} \enquote{свободовки} о \enquote{героях-бойцах СС} провалила
проект о запрете героизации нацизма,
Эллина Либцис, strana.ua, 03.06.2021

И вначале было \emph{Слово} и \emph{Слово} было у Бога и \emph{Слово} было Бог,
от Иоанна, Евангелие (Благая Весть)

Во-первых, Зеленский совершенно не умеет давать интервью, что странно для
человека, много лет говорящего со сцены, и исходя из этого, должного владеть
\emph{словом}. У него совершенно путанные мысли, отсутствие логических связок и
переходов при формировании ответов, фразы не выверенные, а
импровазиционно-эмоциональные, противоречивые по содержанию и сути. Такой себе
Голобородько-стайл, хотя за два года уже можно было бы освоить совершенно
другой уровень общения со СМИ – президент-стайл,
\textbf{Это странно, но Зеленский совершенно не умеет давать интервью}, Светлана Крюкова, 
strana.ua, 03.06.2021

Друзі! Тижневик \enquote{\emph{Слово Просвіти}} — один із небагатьох нині
існуючих островів незалежного українського ­ \emph{слова} в інформаційному морі
— за певного \enquote{сприяння} нашої влади опинився на межі припинення свого
існування. І тільки ми самі — ті, хто любить, шанує і цінує українське
правдиве і об'єктивне \emph{слово}, можемо допомогти тижневику. Будь-яка
фінансова підтримка видання — це вагомий внесок в українську справу. Тож
подаємо Вам наші банківські реквізити: АТ \enquote{Альфа-Банк}, р/р UA 30
300346 0000026002016994001; ­ отримувач: ГО \enquote{Всеукраїнське товариство
\enquote{Просвіта} ім. Тараса Шевченка}; код ЄДРПОУ/ІНН: 00031756.  Призначення
платежу ­заповнюється платником так, щоб надати повну інформацію про платіж.
На наше прохання про допомогу для газети \enquote{\emph{Слово Просвіти}}
відгукнулися: Кропивницька Мар'яна Андріївна – 800 грн.  Стебловська Леонілла
Петрівна – 100 грн.  Лизанчук Василь Васильович – 500 грн.  Стасик Юлія
Іванівна – 100 грн.  Мельниченко Володимир Юхимович – 1000 грн,
\textbf{Звернення до наших читачів, передплатників, просвітян}, slovoprosvity.org, 17.05.2021

%%%cit
%%%cit_pic
%%%cit_text
У Зеленского были все возможности выполнить обещание - и снизить тарифы.  Но,
видимо, ему скучно держать данное \emph{Слово} ...  Впрочем, Зеленскому давно
привычно жить по принципу одного героя известного советского кинофильма: я -
хозяин своего \emph{Слова}. Я его дал, я его обратно взял. Тем более, что еще
столько можно хорошего наобещать избирателю, что к чему исполнять? Скучно ведь
станет жить. Буднично. Серо
%%%cit_title
\citTitle{У Зеленского были все возможности выполнить обещание - и снизить тарифы}, Валерий Песецкий, strana.ua, 11.06.2021
%%%endcit

%%%cit
%%%cit_pic
%%%cit_text
Есть здесь и откровенная дичь, которую сами составители «Стратегии» в силу
своих особых способностей наверняка даже не заметили. В одном месте они
перечисляют страны, территориальную целостность и суверенитет которых
поддерживает Украина – Азербайджан, Молдавия, Грузия. Сербии в этом списке как
бы «случайно» нет. Ну нет и нет. Как говорит Блогер, \emph{Cлова} не нужны.
Главное – это действия.  И вот вам действия. В другом месте они же пишут –
Украина продолжает брать активное участие в миссиях под руководством НАТО в
Афганистане и... Косово. В Косово, блд. Но вы же дебилы сами на официальном
уровне Косово продолжаете считать территорией Сербии. И одновременно без
согласия Белграда принимаете участие в миссиях НАТО на ее территории в Косово.
И тут же на каждой страничке отражаете раз за разом ру-агрессию и декларируете
приверженность международному праву
%%%cit_title
\citTitle{Украинская власть разродилась стратегией внешнеполитической деятельности}, Игорь Лесев, strana.ua, 11.06.2021 
%%%endcit
