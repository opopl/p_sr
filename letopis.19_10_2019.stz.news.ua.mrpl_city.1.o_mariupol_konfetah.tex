% vim: keymap=russian-jcukenwin
%%beginhead 
 
%%file 19_10_2019.stz.news.ua.mrpl_city.1.o_mariupol_konfetah
%%parent 19_10_2019
 
%%url https://mrpl.city/blogs/view/o-mariupolskih-konfetahko-dnyu-rabotnikov-pishhevoj-promyshlennosti
 
%%author_id burov_sergij.mariupol,news.ua.mrpl_city
%%date 
 
%%tags 
%%title О мариупольских конфетах - ко дню работников пищевой промышленности
 
%%endhead 
 
\subsection{О мариупольских конфетах - ко дню работников пищевой промышленности}
\label{sec:19_10_2019.stz.news.ua.mrpl_city.1.o_mariupol_konfetah}
 
\Purl{https://mrpl.city/blogs/view/o-mariupolskih-konfetahko-dnyu-rabotnikov-pishhevoj-promyshlennosti}
\ifcmt
 author_begin
   author_id burov_sergij.mariupol,news.ua.mrpl_city
 author_end
\fi

\ii{19_10_2019.stz.news.ua.mrpl_city.1.o_mariupol_konfetah.pic.1}

\textbf{\enquote{Адрес-календарь \enquote{Мариуполь и его уезд на 1910 г.}}}: 
\enquote{Конфетные фабрики. Горелов И.Х., Георгиевская ул., собственный дом, Эйдинов Ш.М., Митрополитская
ул., собственный дом, Литвинов, Торговая ул., дом Шарапея, Фукс И.Б., Торговая
ул., дом Арабаджи}.

Конечно, \enquote{фабрики} - громко сказано. Это были крохотные предприятия с небольшим
количеством работников. Бывало, что сам хозяин заведения принимал
непосредственное участие в изготовлении конфет. Несмотря на то, что названы
улицы, на которых находились кондитерские фабрики Горелова, Литвинова и Фукса,
конкретное расположение их определить невозможно. Кто теперь знает, по какому
адресу находились те или иные домовладения? Ведь номера домовладений не
указаны. А вот адрес эйдиновской фабрики известен. Улица Николаевская, 89. Но
номер этот узнан, конечно, не из вышеупомянутого издания...

Трудно сказать, когда отобрали фабрику у Эйдинова - с установлением Советской
власти в Мариуполе или во время НЭПа? Пока не удалось узнать. Остались
материальные свидетельства о ее существовании в виде конфетных оберток. К
слову, теперь эти небольшие листики бумаги стали предметом торгов на
интернет-аукционах. Надписи на обертках: \enquote{Конфетная фабрика Ш.М. Эйдинова},
\enquote{Кондитерская фабрика \enquote{Просвет}}, \enquote{Маріуп. Окрспілки Спож. Т-в. 
\enquote{МАРСПО}, м. Маріуполь}, \enquote{Цукер. пряничн. фабр. Маріуп. районової спілки спожив. Товариств
\enquote{МАРСПО}, м. Маріуполь}.

Фабрике много раз изменяли название. Но это на качество продукции совершенно не
влияло. Традиции, установленные еще при Эйдинове, передавались у создателей
сладостей из поколения в поколение.

\textbf{Читайте также:} 

\href{https://mrpl.city/news/view/mariupolskaya-konditerskaya-fabrika-prevrashhaetsya-v-razvaliny-fotofakt}{%
Мариупольская кондитерская фабрика превращается в развалины, Олена Онєгіна, mrpl.city, 16.10.2018}

22 июня 1941 года гитлеровская Германия вероломно напала на Советский Союз. Все
предприятия города, в том числе и кондитерская фабрика, перешли на режим работы
по плану \enquote{на особый период}. Наверное, поэтому так мало известно о работе
фабрики от начала войны и до оккупации Мариуполя. Ее военнообязанные работники
были мобилизованы в армию. Остались в цехах женщины да пожилые мужчины. На
второй день оккупации, 9 октября 1941 года, по городу были развешены листки
бумаги с текстом: \emph{\enquote{Объявление главнокомандующего германскими войсками о мерах
наказания за нарушение населением приказов оккупационных властей}}. В нем, в
частности, было сказано, что все трудоспособные жители должны явиться на
рабочие места в течение трех дней, за неявку на работу - расстрел. И ведь
расстреливали. Пришлось кондитерам, рецептурщикам, конфетчикам и работникам
других профессий идти на свои рабочие места. Да, кондитерская фабрика во время
оккупации работала, но ее продукцией пользовались только оккупанты и их
приспешники.

В газете \enquote{Приазовский рабочий} в номере за 19 октября 2010 года была напечатана
статья журналистки Людмилы Потоцкой к 100-летию кондитерской фабрики. Вот
цитата из статьи: {\color{blue}\em\enquote{Фабрика в 70-80-е годы прошлого столетия выпускала 200
наименований продукции и 24 тысячи тонн в год. Старожилы Мариуполя помнят
любимые шоколадные конфеты: \enquote{Мишка на Севере}, \enquote{Кара-кум}, \enquote{Птичье молоко} и,
конечно, бренды Мариупольской кондитерской фабрики – это \enquote{Стрела} и \enquote{Курочка
Ряба}, радовали ребят, как игрушки для детей. Продукцию нашей фабрики можно
было встретить в Прибалтике и на Дальнем Востоке, в Сибири и на Урале – в
магазинах всего Советского Союза, а также Польши и Афганистана... Теперь о
брендах остались только воспоминания... С 1997 года фабрика, войдя в состав
корпорации ROSHEN, специализируется на производстве бисквитных изделий.
Ассортимент включает более 50 наименований сладостей, среди которых печенье,
вафли, конфеты, карамель и торты}}. 

200 наименований кондитерских изделий до ROSHEN и 50 наименований при нем. Есть
разница? Печальная хроника событий, почерпнутая из интернета: 31 декабря 2014
года на кондитерской фабрике остановили технологическое оборудование, 3 февраля
2015 года начался демонтаж для отправки на Винницкую фабрику \enquote{Рошен}. Здание
Мариупольской кондитерской фабрики опустело. 700 человек остались без работы.
На этом завершилась история предприятия, которое работало без остановки 114
лет. Даже если отсчитывать его возраст от первого упоминания в известном
историческом источнике \enquote{Адрес-календарь \enquote{Мариуполь и его уезд на 1910 г.}}.

\textbf{Читайте также:} \href{https://mrpl.city/news/view/mariupolskaya-konditerka-stanet-biznes-tsentrom-foto-plusvideo}{%
Мариупольская кондитерка станет бизнес-центром?, Яна Іванова, mrpl.city, 23.01.2019}

А есть ли сейчас в Мариуполе пищевая промышленность вообще? Прекратили свое
существование ликеро-водочный завод, рыбоконсервный комбинат, пищевкусовая
фабрика, маслоцеха, пивоваренный завод, мясокомбинат, три хлебозавода. От того,
что находилось на проспекте Металлургов, осталась только труба. Но остались
бывшие работники мариупольских предприятий пищевой промышленности. Вот их-то мы
и поздравляем с их профессиональным праздником!
