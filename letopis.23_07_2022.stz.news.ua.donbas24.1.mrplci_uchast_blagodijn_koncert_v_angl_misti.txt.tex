% vim: keymap=russian-jcukenwin
%%beginhead 
 
%%file 23_07_2022.stz.news.ua.donbas24.1.mrplci_uchast_blagodijn_koncert_v_angl_misti.txt
%%parent 23_07_2022.stz.news.ua.donbas24.1.mrplci_uchast_blagodijn_koncert_v_angl_misti
 
%%url 
 
%%author_id 
%%date 
 
%%tags 
%%title 
 
%%endhead 

Маріупольці взяли участь в благодійному концерті в англійському місті

22 липня в англійському місті Вітчерч відбувся благодійний концерт на підтримку
біженців України, в організації та проведенні якого взяли участь маріупольці

Українці, які виїхали закордон, намагаються не сидіти склавши руки, а
допомагати своїй країні та співгромадянам, які опинилися у важкому
матеріальному становищі. Зокрема, маріупольська хореографиня і танцівниця,
керівниця дитячої танцювальної студії «Імпульс» ПК «Молодіжний» Анна Паніотова
та викладачка англійської мови і громадська діячка Маріуполя Олена Попова
долучилися до організації та проведення благодійного концерту на підтримку
біженців України, який пройшов 22 липня в англійському місті Вітчерч. Анна та
Олена — дві маріупольчанки, які опинилися разом в Англії — познайомилися з
Оленою Батовською, докторкою мистецтвознавства, професоркою Харківського
національного університету мистецтв імені І. П. Котляревського. Саме вона
створила хор «Калина». Олена з Анною в Маріуполі займалися танцями і виступали
в студії танців фламенко «ROSAS PARA MARIA». Жінки вирішили об'єднати зусилля і
створити спільний колектив. Українки почали активну діяльність у Вінчестері, де
всі вони і проживали. Там вони змогли заявити про себе і взяти участь в
декількох концертах.

Концерт, що відбувся у Вітчерчі має досить високий рвіень. Він був
організований спільно з Британським фольклорним музичним клубом (місто Вітчерч)
та колективом українок, який було створено у Вінчестері.

«Witchurch folk club — це такий англійський клуб, який ґрунтується на
англійській фольклорній музиці.Ось організатори цього клубу запропонували мені
станцювати український народний танець. Я запропонувала дівчатам, які співають
взяти участь в цьому благодійному заході і вони з радістю погодилися», —
розповіла Анна Паніотова.

Танцівницею та ведучою на концерті була Олена Попова, якій вдалося весь вечір
тримати увагу багаточисельної публіки. Жінки підготували один спільний номер,
який отримав назву «Віночок». Спочатку його готували до Івана Купала. Але саме
в цьому номері вдалося дуже вдало поєднати хореографію та вокал.

«Ми не хочемо називати українців, які опинилися в інших країнах, біженцями. Ми
називаємо себе гостями з України. Дуже радіємо, що за кордоном можемо приносити
користь власній державі та відчувати себе корисними завдяки таким заходам», —
наголосила Олена Попова. 

На концерті були представлені українські та британські музиканти і танцюристи.
Саме завдяки цьому вдалося представити насичену програму та різнопланову
музику. Адже зі сцени лунала і народна, і класична, і сучасна музика. Проте
найголовнішим на цьому заході все ж стала підтримка і популяризації української
культури. На концерті були представлені:

— Гімн України (виконував Хор «Калина»);

— Танок з прапором (виконували Анна та Олена);

— Ой у лузі червона калина (Хор «Калина»);

— Чарівна скрипка (Ноелла);

— Ой у вишневому садку (Лала);

— Несе Галя воду (Олена та Ноелла);

— Танок Анни;

— Купальські пісні (Хор «Калина»).

"Дуже важливо, що в цьому концерті брали участь дві сторони: і британська, і
українська. Це був спільний, платний концерт. Білет коштував 15 фунтів. Було
продано близько 150 квитків. Зала була переповнена. Це дуже велике досягнення і
дуже гарний результат, адже гроші від цього заходу підуть на підтримку
українців. Після концерту люди підходили, дякували, фотографувалися. Це був
справжній фурор", — поділилася Олена Попова.

«Захід вийшов інтернаціональний. Публіка дуже підтримувала, зустріла тепло,
було багато оплесків. Ми й досі знаходимося під сильними враженнями» — додала
Анна Паніотова.

Концерт проходив у старовинній церкві у Вінчерчі, що додавало заходу особливої
атмосфери. У Анни та Олени ще багато нових ідей та планів. Можливо, наступний
захід буде проведено у Лондоні і стане ще більш масштабним.

Нагадаємо, раніше Донбас24 розповідав, як морпіхи захищали Маріуполь.

ФОТО: з особистих архівів Анни Паніотової та Олени Попової.
