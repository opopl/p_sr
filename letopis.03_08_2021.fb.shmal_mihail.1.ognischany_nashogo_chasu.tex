% vim: keymap=russian-jcukenwin
%%beginhead 
 
%%file 03_08_2021.fb.shmal_mihail.1.ognischany_nashogo_chasu
%%parent 03_08_2021
 
%%url https://www.facebook.com/permalink.php?story_fbid=2712427865714232&id=100008411352474
 
%%author_id shmal_mihail
%%date 
 
%%tags poezia,ukraina
%%title ОГНИЩАНИ НАШОГО ЧАСУ
 
%%endhead 
 
\subsection{ОГНИЩАНИ НАШОГО ЧАСУ}
\label{sec:03_08_2021.fb.shmal_mihail.1.ognischany_nashogo_chasu}
 
\Purl{https://www.facebook.com/permalink.php?story_fbid=2712427865714232&id=100008411352474}
\ifcmt
 author_begin
   author_id shmal_mihail
 author_end
\fi

\headCenter{ОГНИЩАНИ НАШОГО ЧАСУ}

Шановні друзі! 

Продовжуємо публікувати вибрані поезії із літературно-мистецького
альманаху «Огнищани» вихованців гуртка юних шанувальників художнього
слова «Горислово» з нагоди його 35-річної діяльності. Минулої суботи ви
мали нагоду ознайомитись з першою частиною книги. 

Альманах видало Броварське видавництво «АНФ ГРУП, ТОВ» – якісно і
оперативно, за що ми складаємо йому сердечну подяку. Книга увібрала в
себе масу матеріалів, які засвідчили багатогранну діяльність творчого
колективу. Як ми уже зауважували, вихованці гуртка писали не тільки
вірші, а й художню прозу, наукові праці у системі МАН України, та чи не
найбільшої популярності зазнав колектив, коли долучився до мистецтва
декламації.

\ifcmt
  tab_begin cols=3

     pic https://scontent-yyz1-1.xx.fbcdn.net/v/t1.6435-9/228586563_2712427602380925_6485801567120565393_n.jpg?_nc_cat=101&ccb=1-5&_nc_sid=730e14&_nc_ohc=UGxWeSv9JrQAX90q7r7&_nc_ht=scontent-yyz1-1.xx&oh=2a478d990cfa5f87f015c031b3080879&oe=616E27CF

     pic https://scontent-yyz1-1.xx.fbcdn.net/v/t1.6435-9/230525453_2712427535714265_2110184326358634283_n.jpg?_nc_cat=100&ccb=1-5&_nc_sid=730e14&_nc_ohc=i8bI08j5HeYAX9HH65B&_nc_ht=scontent-yyz1-1.xx&oh=7aa9d9c7507ddf95f9d189b4c57be379&oe=616F7978

		 pic https://scontent-yyz1-1.xx.fbcdn.net/v/t1.6435-9/230114902_2712427552380930_3435828195578893430_n.jpg?_nc_cat=106&ccb=1-5&_nc_sid=730e14&_nc_ohc=hY3D2BPqv0EAX8gzc-V&tn=lCYVFeHcTIAFcAzi&_nc_ht=scontent-yyz1-1.xx&oh=2a9bce0c5b5c5e05294f5afe521a0e2f&oe=616FF246

  tab_end
\fi

Особливо вихованцям полюбилось читати гумористичні твори. Це мистецтво
відкрило не одного юного талановитого митця. Ми свідомо не називаємо
жодної дитини, не виділяємо тих, що були більш успішними під час
навчання, аби не образити інших. Попереду у кожного з них ще  велике
життя, аби проявити себе і вийти на свою творчу доріжку.                                         

Загалом же вихованці гуртка «Горислово» взяли участь у 45 творчих
форумах від регіонального до міжнародного рівня, де отримали понад 150
відзнак різного ґатунку, третина з яких - дипломи І ступеню та Гран-прі.

Насамкінець хочеться попросити вибачення у тих дітей, які не побачать
себе на знімках у цьому матеріалі. Не вдалось сьогодні підкорити
Faceebook, та віриться, що світ не без добрих людей, і нам вдасться
видати ще один тираж книги «Огнищани», яка буде суттєво доповнена
досягненнями вихованців за час, який минув після її виходу у світ. 

Дякую усім вихованцям різних років, які будь-якої пори року, за будь-
якої погоди, ранками і вечорами поспішали на заняття, працювали над
Словом, перемагали на творчих змаганнях. Навчання не буває одностороннім
– росли вихованці, а разом з ними зростав у творчості і їхній педагог.
Цей колектив – одне з найцінніших надбань у моєму житті. Складаю Вам,
мої рідні, дорогі молоді побратими сердечну подяку за роки спільної
праці і зичу зоряних успіхів! Слава Вам і честь!

З безмежною вдячністю – Ваш Михайло Михайлович.

\ifcmt
  tab_begin cols=3

     pic https://scontent-yyz1-1.xx.fbcdn.net/v/t1.6435-9/230871648_2712427705714248_3174600914092629135_n.jpg?_nc_cat=111&ccb=1-5&_nc_sid=730e14&_nc_ohc=wfiwBRZTD1cAX9EnGwB&_nc_ht=scontent-yyz1-1.xx&oh=41e553931026a473bb84a05d6c86596c&oe=616E6E92

     pic https://scontent-yyz1-1.xx.fbcdn.net/v/t1.6435-9/228918845_2712427672380918_3705682776552669764_n.jpg?_nc_cat=106&ccb=1-5&_nc_sid=730e14&_nc_ohc=bdeiJF7Ty9gAX9uhESE&_nc_ht=scontent-yyz1-1.xx&oh=09402d21049f543df9b2050e3b0e84c4&oe=616F6515

		 pic https://scontent-yyz1-1.xx.fbcdn.net/v/t1.6435-9/229034128_2712427659047586_2948215746316599338_n.jpg?_nc_cat=109&ccb=1-5&_nc_sid=730e14&_nc_ohc=eXqjrtQvassAX8Bq1d8&tn=lCYVFeHcTIAFcAzi&_nc_ht=scontent-yyz1-1.xx&oh=fa71f5fc63608f69d5acc5a44c613cd4&oe=616CCFF2

  tab_end
\fi

ЮЛІЯ ВОЙЛЕНКО

ДОРІГ НА СВІТІ Є БАГАТО

Зима холодним поглядом у світ наш зазирнула,
Стає слизькішим для людей важкий життєвий шлях,
Вже душі багатьох вона на кригу обернула,
Я ж задихаюся, як зло свій підіймає стяг.
Як важко жити там, де править люд, гріхом умитий,
Ще важче новачку знайти в тім світі вірну путь.
Залишилось лише у Бога долі попросити,
Бо тут йому в завдаток й крихти навіть не дадуть.
Де впевненість, що завтра юнь оця не посковзнеться,
Що не розвіє вітер злий троянди пелюстки,
Що бідам всім на зло вона із гідністю всміхнеться
І пригадають посмішку цю майбуття віки.
Я хочу зберегти своєї юності троянду
І в душу не пущу я паперовий ідеал,
Хай навіть горем змучену, та непокірну знайдуть
І скаже хтось: «Це саме та, що я завжди шукав».

\ifcmt
  tab_begin cols=2

     pic https://scontent-yyz1-1.xx.fbcdn.net/v/t1.6435-9/229382678_2712427759047576_6906848389016401761_n.jpg?_nc_cat=100&ccb=1-5&_nc_sid=730e14&_nc_ohc=wXxui3jnj3kAX-iNM9T&tn=lCYVFeHcTIAFcAzi&_nc_ht=scontent-yyz1-1.xx&oh=35e15c93382dff67f51dd613b1dddd65&oe=616CDC5A

     pic https://scontent-yyz1-1.xx.fbcdn.net/v/t1.6435-9/230171888_2712427782380907_8539150483728664876_n.jpg?_nc_cat=103&ccb=1-5&_nc_sid=730e14&_nc_ohc=oa49FInPtuAAX-mIVcn&_nc_ht=scontent-yyz1-1.xx&oh=f616771f6659f59c1338264cb060f137&oe=616E5B08

  tab_end
\fi


ЄВГЕН ЧЕРНЯК
ВКЛОНЯЮСЬ ПУШКІНУ
Яскрава свічка породила тіні,
Дивами-дивними розмалювала стіни…
Перо поета розбудило тишу
І сіє за собою зерна віршів.
Бездонна ніч дріма в притихлих звуках,
Не спить поет — ще невідомий Пушкін,
Та перший вірш вже паростком із серця
За словом слово у майбутнє рветься.
Ще казка лиш вві сні йому приходить:
І цар Салтан, і кіт по ланцях ходить,
І Лебідь-Діва, і Чаклун горбатий,
І чарами збудовані палати.
Життя ще там — нескорена вершина,
Ще там — кар’єра, дім, родина,
Слухняне Слово й з ним — могутні крила,
Що їх натхнення горде породило.
Там — осінь в Болдіно й в Михайловськім зима,
І слово, як натягнута струна, —
Дзвенить у щасті і тремтить від болю —
Таку вже хтось наврочив йому долю.
Струна порветься на найвищій ноті,
Та найманцям який до того клопіт?..
Ганьба Дантесу і анафема — по праву.
«Погиб поет…» Не вмре повіки його слава...
Вже вік двадцятий добігає краю,
Нічну пітьму настільна лампа крає…
Твій внук з далекої у часі України —
Вклоняюся тобі, великий Слов’янине!
ОЛЬГА КРИВОРУЧКО
ВОРОБЕЙ
Вот воробышек на ветке, —
Как он мал и как смешон!
У воробышка есть детки,
Все такие же, как он.
Быстро вырастут птенцы,
Скоро станут молодцы,
Все такие же, как папа, —
Мухоловы-удальцы!
КАТЕРИНА ПРОМСЬКА
РІДНА ПІСНЯ
Рідна пісня знову лине
Над степами України,
В ній лани, Дніпро могутній,
Наша доля незабутня.
Кий, брати його й сестра
Йдуть на берег із човна.
Златоглавий Київ сяє,
І, неначе, оживає
У віках минула слава:
І Софії древня брама,
Й запорізькі козаки
Шлють привіт нам крізь віки…
Їх веде Богдан Хмельницький —
Славний гетьман український.
МАРІЯ ЗАРІНОВА
ПОДАРУЙ МЕНІ…
Подаруй мені сонечко ясне,
Подаруй мені вірші й пісні,
Подаруй мені зорі прекрасні,
І слова подаруй чарівні.
Подаруй мені, любий, обійми,
Поцілунки солодкі свої,
Подивися очами надії,
Що я бачу у снах їх вночі.
Подаруй мені голос свій ніжний,
Що пробуджує серце моє…
Подаруй мені білий підсніжник,
А із ним і кохання своє.
АНАСТАСІЯ ПАНЧЕНКО
ПЕРЕД БЕЗОДНЕЮ
Навіщо людям слова?
Вони лише кажуть неправду.
Навіщо людям почуття?
Вони тільки додають страждань.
Навіщо людям життя?
Це лише мить перед безоднею...
* * *
Ти мрієш про життя —
Очисти природу,
Ти думаєш про гроші —
Іди зароби,
Ти замислився про кохання —
Ну так полюби!
Немає такої ситуації,
З якої немає виходу.
* * *
Надворі дощ,
І я не чую твоїх кроків,
Може, вони зливаються з музикою дощу,
Чи їх зовсім немає?
Мабуть, твої кроки
Пішли іншою стороною,
Бо я тебе не чую.
Пішов — іди,
Назад дороги не буде.
* * *
Я закрила двері свого дому,
Я зачинила кватирку,
Щоб ти не зайшов.
Я зачинила ...
Але не змогла зробити одного —
Зачинити серце для тебе.
АНАСТАСІЯ ФЕДОРЧУК
ІДЕ ВЕСНА
Замела хуртовина
Всі дороги і доли,
І засохлу билину
Студить вітер край поля.
Ще далеко до літа
І лютують морози,
Та весна вже привітна
Мчить до нас у дорозі.
Вже ласкавіше сонце,
І пташки веселіші,
І під нашим віконцем
Вітерець теплий дише.
ОЛЬГА ПОЖАР
МОЯ МАТУСЯ 
У коси, що барвінком в’ються, 
Вже доля сивину впліта. 
Для мене ж ти, моя матусе, 
Повіки будеш молода... 
Світанок обійма калину, 
Пшеничні золотить моря, 
Поспи ще, лагідна пташино, 
Допоки світиться зоря. 
Від праці хай спочинуть руки, —   
Бо ти ще зовсім  молода... 
Ще рано, спи, моя голубко, —  
Благословенна і свята.
АНТОН САХНО
РІЧЕЧКА
Серед вербиць та очеретів,
В пахучих бур’янах
Струмочок річечці дорогу
Крізь зарослі проклав.
Водою щедро напувалась,
Дощами й росами вмивалась,
Росла, мужніла, розливалась,
Та річечка мала.
І виросла вона велика,
Широка та швидка!
Лиш сни солодкі бачить
Про братика струмка.
Іскриться, світиться, хлюпоче,
На сонці виграє,
Свою красу і чисту воду
Людині віддає.
ЯКІВ ЯКОВЦЕВ
* * *
Колесо весни невпинно котиться вгору,
Запалюючи вечірні вогні по садах,
Зазираючи прозирливим оком
У крони дерев, у людські серця,
В глибини Всесвіту.
* * *
Матуся розпочинає день,
Розвіюючи нічну пітьму
Вогником газової плити,
Подихом і дотиком
Розкриваючи мені світ.
* * *
Сонце — іскринка душі
Величезного Всесвіту —
Запалює пломінь тепла
В наших душах.
КАТЕРИНА ЄГОРУШКІНА
ЗАКАТ НА ЛИМАНАХ
Закат мелькает алой тростью
Последней искорки луча.
Ушло неспешно солнце в гости
К другому дню, к иным мечтам.
Ушло куда-то за лиманы,
Махая красным рукавом,
И кружат сонные туманы
Над желтой плесенью холмов.
НАТАЛІЯ САМЧИК
КАРТИНА
Картина на шафі —
І гори і ліс,
І вечір там пахне,
Що вітер приніс.
Там хата маленька,
Хто йде — не мине,
Там ненька старенька
Чекає мене.
У тій милій хаті
Зросла я колись,
А потім, як птаха,
Злетіла увись.
Та скільки б крилато
Мені не літать,
До матері й хати
Я буду вертать.
ІРИНА КОВАЛЬОВА 
ЗАБЛУКАВ
Як мчить замріяний метелик
У вихрі вітру степового!
І барв на крилах п’янкий келих,
І мрія в серденьку малому.
Чи заблукав він, бідолашний?.. 
Листки кружляють і збивають
Його з дороги. Холод, страшно…
Та він надії — не втрачає.
Летить чимдуж і все шукає
Те розмаїття квіток літніх,
Малює літечко в уяві
І неба синь в волошках  житніх.
Узрів останні чорнобривці —  
Скоріш на жар їх присідає
І, ледь тремкі згорнувши крильця,
В солодких мріях засинає.
АНАСТАСІЯ КИРИЧЕНКО
ТАРАСОВІ СНИ
Нагородила Україну
Лицарем безстрашним доля,
Став він їй найкращим сином,
Бо вказав їй шлях до  волі.
Довгий час він жив далеко —
Був солдатом на чужині,
Лише в снах йому лелеки
Слали вісті з Батьківщини.
В снах він бачив Україну,
І Дніпра-Славути хвилі,
І заквітчана калина
У тих снах його світилась.
Материнські ніжні руки,
Батька сильні й мужні плечі,
І сестер кріпацькі муки
Слались в тих вістях лелечих.
В снах світилось рідне сонце,
Й неба дзеркало прозоре,
Й білих хат ясні віконця,
Як маленькі чисті зорі.
В них літала диво-пташка —
Ластівка із Батьківщини,
І хліба хилились важко
В незалежній неньці-Україні.
ОЛЬГА ЛЕЩИК
СВІЧКА
До тебе прилину
Я тихо, як нічка.
Запалим з тобою
Кохання ми свічку.
Вона буде сонцем
Всю ніч нам палати
Й своєю красою
Обох чарувати.
Вона буде символом,
Світлом кохання,
Й навколо літатимуть
Наші зізнання…
Вона буде тихо
В пітьмі нам світити,
А ми з нею разом
Коханням горіти…
ІРИНА ЛЕЩИК
МУЗИКА ДОЩУ
Впевнено й весело
Дощик накрапає,
Наче десь піднесено
Музика лунає.
Бють краплинки по листках,
По залізній стрісі,
Спіють яблука в садках
І суниці в лісі.
Не змовка десь птаха спів,
Що співа про весну…
Під його гучний мотив
Жити всім чудесно.
ЮЛІЯ ЛИСЕНКО
СИНІЙ ВЕЧІР
Синя квітка під вікном
Задрімала синім сном,
Бо вже лине у кареті
Синій, синій тихий вечір.
ТЕТЯНА СОКОЛ
КОЛИСАНКА ДЛЯ ІВАНКА
Мати колисала братика Іванка, 
І співала ніжно: «Спи дитя до ранку».
Місяць посміхався мерехтливим зорям —
Керівник оркестру в мовчазному хорі.
Морозець у вікна заглядав прозорі,
Цілу ніч без втоми малював узори.
А на ранок хуга плакала, зітхала,
Бо з тієї пісні нот і слів не знала.
ОЛЕКСІЙ БОЙКО
* * *
«Ще не вмерла України…»
З Гімну України
Ще жива Україна — 
Наша матір одна,
Як у лузі  калина,
Заквітчалась вона.
Мов свіча непогасна,  
Вона в наших серцях — 
Її мова прекрасна
І натхнення борця.
Ще не вмерло козацтво,
Бо не можна згубить 
Віковічне те братство,
Що за волю стоїть.
Ще живе Україна 
Й кличе нас до звитяг —  
Наша матір єдина,
Що дала нам життя.
ЕЛІНА ІВАЩЕНКО
БАБУСЕНЬКА МОЯ
Рік за роком йдуть поволеньки літа,
Тільки спогади лишають в думці…
Зостається все  в старій шкатулці,
Лиш моя бабуся — вічно молода!
Все пройшла вона в житті — куди не кинь,
Та хоча нестримно час  минає,
Ще її десь зіронька сіяє —
Ще не владні над бабусею роки!
АННА ТУГАЙ
ЛІТО
Рано-вранці, на світанку,
Ясне сонечко зійшло,
Золотисту вишиванку
Красне літо одягло.
Загорілись в росах трави,
І в долині, край ріки,
Грає коник величавий
Свої вранішні пісні.
Літо, літо — сонця свято 
Між зелених пишних нив…
Гарно дітям там гуляти
Разом з ранком голосним.
