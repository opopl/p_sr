%%beginhead 
 
%%file 28_03_2022.fb.tymchenko_mykola.kyiv.fotograf.1.malii_b_znes_u_ki_v_
%%parent 28_03_2022
 
%%url https://www.facebook.com/nick.tymchenko1/posts/pfbid07fnv87EDqwBheP57kMF7kGBpV8nXjWRkot1RUYpvJavxY23JBMiWDoHPoKhjZLS3l
 
%%author_id tymchenko_mykola.kyiv.fotograf
%%date 28_03_2022
 
%%tags kiev
%%title Малий бізнес у Києві потроху оживає. Історія Валентини – господині кав'ярні у віддаленому районі
 
%%endhead 

\subsection{Малий бізнес у Києві потроху оживає. Історія Валентини – господині кав'ярні у віддаленому районі}
\label{sec:28_03_2022.fb.tymchenko_mykola.kyiv.fotograf.1.malii_b_znes_u_ki_v_}

\Purl{https://www.facebook.com/nick.tymchenko1/posts/pfbid07fnv87EDqwBheP57kMF7kGBpV8nXjWRkot1RUYpvJavxY23JBMiWDoHPoKhjZLS3l}
\ifcmt
 author_begin
   author_id tymchenko_mykola.kyiv.fotograf
 author_end
\fi

Малий бізнес у Києві потроху оживає. Історія Валентини – господині кав'ярні у віддаленому районі.

Зараз у Києві потроху відновлює роботу малий бізнес, починають працювати
кав'ярні, перукарні, невеликі магазини з промисловими товарами. Минулого тижня
завітав у трохи віддалений район ДВРЗ, який знаходиться на околиці Києва. Ще
років 15-20 тому він мав досить сумну славу, через свою замкненість та те, що
переважна частина місцевих працювала тут, на вагоноремонтному заводі й досить
непривітно ставилась до приїжджих. Часи змінилися, зараз це хороший район
відносно «неподалік» від метро, в якому збереглась цікава оригінальна
архітектурна забудова 1930-50х років. Зараз район так само залишається
непривітним, але тепер тільки для загарбників. Але з привітними людьми, які в
такі скрутні часи працюють та надихають інших. Саме такою є господиня
невеликого кафе Валентина. Першими днями вона, як і всі інші в Києві, не
працювала, але за декілька днів не витримала, відкрила свою кав'ярню. Випивши
кави, я скористався нагодою записати історію  цієї надзвичайно приємної та
оптимістичної жінки.

Ось що розповіла пані Валентина:

«Мені  53 роки, народилась у Хмельницькому. Тут працюю вже третій рік, і
живу теж неподалік. 24го лютого мене розбудив дзвінок сина. Каже: мам,
почалася війна, бомблять аеродроми. Дуже було страшно. Все зруйнувалося в
один день, життя стало зовсім іншим. Почалися такі хвилювання, не знала що
робити, люди починають їхати, хтось кудись збирається, на вокзали, все
незрозуміле. А ці смерті, смерті дітей.... Як знищують наші міста, красиві
міста.... Це дуже велика біда....

Але за деякий час я відкрилася, навіть почала готувати їжу, волонтери
приїжджали, забирали, готувала, поки у мене були продукти. Потім почала
приходити та просто відкривати, щоб люди могли зайти. Це дуже важко,
коли людина не знає, куди себе подіти. А так прийшла, попила кави,
зігрілася. Це не стільки бізнес зараз, як бути просто потрібною, щоб
просто хтось прийшов, зігрівся, сказати комусь хороше слово і самому
втриматися, тому що дуже болить. Люди приходять, плачуть, розказують про
свою біду. Були у мене з Миколаєва, розказували про жахи. Сьогодні
прихистили сім'ю, вони поверталися з Донецької області, дали їм поки
квартиру, щоб відпочили, їм їхати далі. Машина обстріляна, в осколках.
Страшно, що це сталося в нашій прекрасній країні. Мирні, розумні
трудолюбиві люди, щирі, і така страшна біда сталась. Але ми повинні всі
триматися разом, ми вистоїмо, ми красива, розумна інтелігента нація, яка
має право жити на своїй землі і забрати кожен метр своєї землі. Ніякого
Криму, ніякого Донецька, нічого їм не бачити! Все буде наше. Ми сильні і
ми повинні жити в Україні!

Зараз пригощаю кавою, булочками. Булочки випікаю сама. Є з яблучком, з
капустою, сосиски є. І піца теж є. З чого виходить. Я не знаю, чи
працюють у районі ще кав'ярні, думаю, що ні. Я розумію тих людей, які
поїхали з України: треба рятувати дітей. А я буду до кінця на своїй землі
і буду молитися, просити бога, щоб скоріше цих паразитів з нашої землі
вигнати. І щоб їх не було, ніколи. Я за період війни сказала дуже багато
негарних слів, я матюкалася, я посилала той самий корабель туди, коли
посилають всі. Я прокляла путіна, мабуть, з ніг до голови. 

А за наших військових молюся – вони гарні, сильні, тримаються, молодці,
справжні чоловіки і у нас справжня країна. І президент наш, я за нього
голосувала, він виправдав мої надії. Україна вже не буде такою, як була.
Вона має стати єдиною, сильною, демократичною.

У перший день після перемоги, напевно, буду цілувати землю, дякувати
богу, дякувати військовим, що ми перемогли і що ми можемо ходити по
своїй землі, радіти сонцю, тим людям, які приходять, бачити своїх
рідних. Це найголовніше. І помолитися за тих, кого вже нема з нами. А
далі буду будувати, щось робити, планувати бізнес. Найголовніше –
вижити, вистояти і перемогти».

Вже на виході з кав'ярні пані Валентина у дуже наказовому тоні
пригостила мене теплою випічкою з яблуками. Ледь стримався, щоб не
з'їсти дорогою додому, запах свіжої здоби проходить легко через будь-які
захисні пакети. Очікування виправдались: просто неймовірно смачно! Тепер
знаю, куди їхати за свіжою випічкою, як би це не було далеко.

