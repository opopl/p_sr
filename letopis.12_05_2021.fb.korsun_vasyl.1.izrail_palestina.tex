% vim: keymap=russian-jcukenwin
%%beginhead 
 
%%file 12_05_2021.fb.korsun_vasyl.1.izrail_palestina
%%parent 12_05_2021
 
%%url https://www.facebook.com/korsunvasyl/posts/1098774363939584
 
%%author 
%%author_id 
%%author_url 
 
%%tags 
%%title 
 
%%endhead 
\subsection{Палестинцы - наследники древних евреев?}
\Purl{https://www.facebook.com/korsunvasyl/posts/1098774363939584}

Ще цікавий момент. Споживати росйський контент – то недобре. Бо таким чином ми
сприяємо поширенню російської мови. То ясно.  А перекладати? Перекладати з
російської, то норм? Нє?

Ок, а перекладати з іноземного перекладу з російської?  Добре, ось переклад з
чудового телеграм-каналу «Галєєв», російськомовного. Це який раніше був «Висока
Порта».

—
У зв'язку з крайніми новинами з Єрусалиму мені згадалася одна книга Давида
Бен-Гуріона і Іцхака Бен-Цві – майбутніх прем'єр-міністра і президента Ізраїлю,
що була написана ними під час їх перебування в Нью-Йорку в 1918 році. Книга
називалася «Країна Ізраїлю в минулому і в сьогоденні» і містила в собі кілька
важливих тез, які в наш час виглядали б як повна єресь. Наприклад, Бен-Гуріон і
Бен-Цві вважали, що вигнання євреїв з Ерец-Ісраеля ніколи не було. Святу Землю
покинула лише мала частина спільноти, більша ж частина єврейського народу
залишилася на місці і з часом прийняла Іслам. Їх нащадками є палестинці:

«Фелахи [= осілі палестинці] не є нащадками арабських завойовників, які
захопили Ерец-Ісраель і Сирію в VII столітті н.е. Араби-переможці не знищили
хліборобне населення країни. Вони лише вигнали іноплеменних візантійських
володарів, не заподіявши при цьому ані найменшої шкоди місцевим мешканцям. Крім
того, араби не ставали осілими жителями. У своїх колишніх місцях проживання
вони теж не займалися землеробством... Вони не прагнули передати родючі землі
завойованих країн своїм селянам, оскільки таких майже не було...

Стверджувати, що після завоювання Єрусалиму Тітом і поразки повстання Бар-Кохби
євреї повністю перестали обробляти землю Ерец-Ісраель, означало б виказати
абсолютне незнання ізраїльських історії та літератури того періоду...
Єврейський селянин, втім, як і будь-який інший, дуже важко розлучається з
землею, що просочена його власним потом і потом його предків... Сільське
населення залишилося на своєму місці».

Сучасний історик Шломо Занд підсилює тезу Бен-Гуріона і Бен-Цві. Якщо
тотального вигнання євреїв зі Святої Землі не відбувалося, то звідки ж взялася
чисельна єврейська громада по всьому Середземномор'ю? Як вважає Занд,
здебільшого – це все нащадки новонавернених. Зараз про це забувають, але ще в
ранні середні віки євреї активно займалися прозелітизмом і навертали в іудаїзм
масу іновірців. Наприклад, єврейська громада Іспанії у значній мірі складалася
з нащадків берберських племен, які приняли іудаїзм. Прозелітизм припинився
тільки в високе Середньовіччя: до цього кордони громади залишалися відкритими.

Якщо скласти пазл докупи, перед нами постає дивовижно красива картина. Сучасні
ізраїльтяни, які в масі своїй не є біологічними нащадками древніх євреїв,
воюють з палестинцями, які як раз мають генетичний континуітет з древнім
населенням Ерец-Ісраеля.

гени != меми
