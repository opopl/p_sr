% vim: keymap=russian-jcukenwin
%%beginhead 
 
%%file 05_04_2021.fb.georgii_protoierej.1.shestoj_vnuk
%%parent 05_04_2021
 
%%url https://www.facebook.com/permalink.php?story_fbid=291036425770399&id=100045921323126
 
%%author 
%%author_id 
%%author_url 
 
%%tags 
%%title 
 
%%endhead 

\subsection{Шестой внук}
\label{sec:05_04_2021.fb.georgii_protoierej.1.shestoj_vnuk}
\Purl{https://www.facebook.com/permalink.php?story_fbid=291036425770399&id=100045921323126}

\ifcmt
  pic https://scontent-ams4-1.xx.fbcdn.net/v/t1.6435-0/s600x600/169192229_291036409103734_1015873283252944010_n.jpg?_nc_cat=100&ccb=1-3&_nc_sid=730e14&_nc_ohc=G9wErglZsMYAX8U7Hpe&_nc_ht=scontent-ams4-1.xx&tp=7&oh=4681be6f6cd707e39398fd88c5d6fe57&oe=6091EE69
  width 0.4
\fi

Были у Михайловны внуки, и немало, целых пятеро. И этот, неродной, которого притащила с собой Сарка, третья жена сына Ивана, такая же непутящая и пьющая, как и Иван. Раньше и не знамо было, что у Сарки сын есть. Да позвонили из опекунского совета: «Или забирай ребенка, или отвезем в детдом». Как Михайловна такое-то услыхала, сразу в крик:
- В нашем роду такого еще не было – от детей не отказывались, в войну всех сберегли, а это в мирное-то времечко, да дите бросить? Нет!
И Кольку привезли, этакого затравленного звереныша. Видать, хватил пацан лиха, несмотря на свои семь годков. И Михайловна прикипела к мальчонке остатками, что ли, материнского инстинкта. Когда Иван с Саркой запивали, Колька прибегал к ней. Бабушка ни о чем не спрашивала, кормила, поила, мыла в бане, зимой на печи стелила, летом в зале, около телевизора. Мальчишка боялся темноты, да много чего малой боялся. Вот Михайловна водила Колю по бабкам, те отливали его, и чего только не выливалось, собаки, люди с кулаками. Верь - не верь, но ведь лучше мальчонке стало. Мальчишки-друзья завелись. Как-то щенка принес, а сам ревмя ревет. Розка, собака дворняга, щенков принесла, шесть штук, мать всех потопила, одного только Коля и спас.
- Пойдем-ко, - говорит бабушка, - и Розку к нам заберем. Я что ли щенка-то сиськой кормить буду?
Коля по полдня возле собачьего семейства ворковал. А уж как для бабули старался: воды ей натаскает, двор выметет, огород вместе с ней полет.
- Не велика польза от мальца, дорого старание! Добрый парень растет! – хвалилась Михайловна вечером соседкам.
- Чего ж ты так с чужим внуком возгаешься, свои ведь есть, - подкалывали соседи.
- А что свои, у них отцы-матери хорошие, бабушки другие есть, я им не больно и нужна. Ну, ходят иногда, вместе с родителями, я им денег суну, подарки-то не даю, боюсь не угодить. Так ни один воды не принесет. А этот лоб разобьет, а все помочь пытается, - им Михайловна отвечает.
Сарку с Иваном посадили, бычка в соседней деревне украли, да и грехов накопилось до кучи. А Колю Михайловна не отдала, упросила дочь Анну под опеку его взять, ей-то самой не дают, возраст, да и не родная она бабушка. И зажили старый и малый вдвоем. Михайловна во второй класс парня проводила. Ничего, башковитый парнишка, хорошо учится, учителя хвалят. Картоху вдвоем вырыли, солений, варений, грибов наготовили, зима не страшна. Ездит Михайловна передачки возит, когда на кратковременную с Сарой приезжает, Колю с собой берет. Сара-то что-то больно плоха, все кашляет, это ведь курево поганое свое дело делает, пагуба такая. А весной померла Сара, Коля не плакал, видно не понимал, а, может, и не привык еще к матери. Пенсию мальчишке выхлопотали.
Совсем жить веселей стало. Через два года Иван вернулся. Тихий, пить не стал, а вскоре к нему женщина с девочкой приехала и зажили. То ли жена такая попалась, то ли понял чего мужик-то. Дом отремонтировал, жена скот завезла и зажили не хуже людей. Михайловна аж сглазить боится. Иван машины на дому ремонтирует, руки-то золотые, мастерство не пропил. Вот и Коля все около крутится, помогает да учится, а Иван с ним как с товарищем обращается. Деньги за работу прикопил, два велосипеда купил, один Жене – падчерице, второй Коле. Радости у парня было!
Однажды Михайловна заболела, в больницу увезли, она не о здоровье своем думала, а о Коле беспокоилась. А он и прикатил на велосипеде, аж 15 км. Молоко топленое привез, варенье, меду и картохи вареной, сам варил.
- Бабуля, - плачет Коля, - ты только не умри, как мамка, я ведь тогда тоже помру.
- Да что ты, боговый, разве ты дашь помереть-то, как я тебя оставлю?!
Однопалатницы завидовали:
- Как это, Михайловна, ты парня такого золотого вырастила? – спрашивают они ее.
- А любовь и ласка все сделают, - отмахивается Михайловна.
Так, по-доброму, и еще шесть годков минуло. Проводили Колю в армию.
- Ты, бабуленька, жди меня, я вернусь, и мы опять вместе жить будем!
Год пролетел незаметно. Вот и внук дорогой на пороге.
- Все, и умереть не страшно, дождалась я тебя, - плачет Михайловна.
- Нет уж, живи, ты еще внуков, моих детей, понянчишь, тебе же только шестьдесят. Живи всем на радость! – хохочет внук.
«Вот встала бы Сара да поглядела бы, какого я красавца из ее замухрышки вырастила. Конечно, не без помощи Ивана, тот не бросил пасынка, во всем помог», - думает старая.
Вот оно – родство, родство не по крови, по духу, а оно дорогого стоит!
Елена Дуденкова
