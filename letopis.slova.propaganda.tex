% vim: keymap=russian-jcukenwin
%%beginhead 
 
%%file slova.propaganda
%%parent slova
 
%%url 
 
%%author 
%%author_id 
%%author_url 
 
%%tags 
%%title 
 
%%endhead 
\chapter{Пропаганда}
\label{sec:slova.propaganda}

%%%cit
%%%cit_pic
%%%cit_text
С Украины может начаться обнуление суверенитетов, полученных в 1991 году. И это
процесс экономический в первую очередь. Но скажи это \emph{украинским пропагандистам},
они уши заткнут. Они не хотят, не желают видеть объективные процессы
экономического поражения Украины. Главная опасность суверенитету Украины вовсе
не Россия, а глупое руководство, которое свою страну грабит, и управлять ей не
умеет. Но \emph{пропаганда} трубит, что это руководство очень умное. Наш президент
строит страну дураков, которых объявил умными. Не смешно, если честно
%%%cit_title
\citTitle{Зеленский продолжает раскалывать Украину по целому ряду признаков}, 
Денис Жарких, strana.ua, 13.06.2021
%%%endcit

%%%cit
%%%cit_head
%%%cit_pic
%%%cit_text
Так, рассуждая о путях \enquote{подрыва биологической силы народа} и уменьшения
рождаемости славян, чиновник из ведомства Розенберга пишет, что в первую
очередь эта политика будет направлена на русских. Но в будущем такая практика
будет применяться \enquote{частично и к Украине}.  \enquote{Пока мы заинтересованы в том, чтобы
увеличить численность украинского населения в противовес русским. Но это не
должно привести к тому, что место русских займут со временем украинцы}, - пишет
Ветцель.  Далее он перечисляет весь инструментарий, который планируется для
этого применять: пропаганда абортов и \emph{пропаганда} \enquote{счастья} бездетной жизни
(мол, дети - это расходы и хлопоты), а также содействие добровольной
стерилизации. Должен быть обеспечен полный доступ населения и к
противозачаточным средствам
%%%cit_comment
%%%cit_title
  \citTitle{22 июня - 80 лет нападения на СССР. Что немцы готовили для украинцев}, Максим Минин, strana.ua, 22.06.2021
%%%endcit

%%%cit
%%%cit_head
%%%cit_pic
%%%cit_text
– Жертвы, потери, разрушение – не то, чем можно в сознании широких народных
масс перечеркнуть слово \enquote{великая}. Особенно если в ваших руках все механизмы
\emph{пропаганды}.  Финляндия в 1939-1944 годах три раза по-крупному воевала против
СССР. Теряла в масштабах этой малочисленной страны большое количество людей и
значительную часть территорий, причем наиболее обжитых и промышленно значимых.
И тем не менее для финнов – это великая победа. Они сохранили свою страну и не
превратились в российское Нечерноземье
%%%cit_comment
%%%cit_title
  \citTitle{Мовчання Сталіна, помилки Гітлера, окупація Києва і провалений бліцкриг – Марк Солонін про війну СРСР і Німеччини}, 
	Євген Руденко, www.pravda.com.ua, 22.06.2021
%%%endcit

