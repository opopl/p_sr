% vim: keymap=russian-jcukenwin
%%beginhead 
 
%%file 16_12_2021.fb.pozdnij_taras.1.kazbek
%%parent 16_12_2021
 
%%url https://www.facebook.com/taras.pozdniy/posts/4549999355096332
 
%%author_id pozdnij_taras
%%date 
 
%%tags alpinizm,gora.kazbek,gory,ukraina
%%title Казбек 2022 - что мы изменили и как работаем над качеством
 
%%endhead 
 
\subsection{Казбек 2022 - что мы изменили и как работаем над качеством}
\label{sec:16_12_2021.fb.pozdnij_taras.1.kazbek}
 
\Purl{https://www.facebook.com/taras.pozdniy/posts/4549999355096332}
\ifcmt
 author_begin
   author_id pozdnij_taras
 author_end
\fi

Казбек 2022 - что мы изменили и как работаем над качеством

В 2021-м году мы решили перенести  опыт с базовых лагерей на Эвересте и 8000 на
вершины пониже. Попробовать на Казбеке. Мы поставили большую палатку \#kailas,
Дмитрий Гуляев  с Андрей Ненченко туда занесли стулья, столы и наводили
порядки. 

\ii{16_12_2021.fb.pozdnij_taras.1.kazbek.pic.1}

Жить на метео было бы даже дешевле, но не нравится там нашим участникам -
воняет и хаос. А тут у группы есть возможность комфортно разместиться своей
компанией, очень круто. 

Группы были большие, в одной 10 человек, две группы на одну дату - это уже 20. 

Местные гиды как увидели, такой поток - сразу стал вопрос, чтобы пользовались и
их услугами. Это нормально, но если бы обсудить это все к началу сезона. А так
для нас это был большой удар, ведь стоимость для участников мы уже не могли
менять и в итоге теряли весомую часть прибыли. 

\ii{16_12_2021.fb.pozdnij_taras.1.kazbek.pic.2}

В целом, на Казбеке уже в 2021 году сервис был одним из лучших, по крайней
мере, так говорили те, кто ходил с другими компаниями. НО! Из-за больших групп,
сложностей с местными, плохой погоды (все личные палатки порвало в начале
сезона и нам пришлось их заменить на новые) - было и достаточно много негатива.

А любой негатив - это возможность стать лучше. 

Мы собрались с тимлидерами, которые работали на Казбеке и совместно придумали,
как сделать программу оптимальной. 

Не стараясь сделать максимально дешевой, но и без лишних изысков. Именно
оптимальной. 

Вот список доработок: 

1. 1 гид на максимум 3-х участников. Одна группа - до 6 человек. На одну дату -
не больше двух групп. Кроме наших гидов в день штурма работают и местные (тут
без вариантов). И того - максимум 12 участников. 

2. Включили проживание в Альтихат в стоимость, чтобы акклиматизация была более
плавной. Хотя это весомая прибавка к цене. 

3. Базовый лагерь на метеостанции будет еще комфортнее. Личные палатки лучше,
столовую палатку, возможно, тоже заменим, чтобы был отдельный отсек и для
кухни. И есть еще ряд мелких плюшек по БЛ, которые надеемся реализовать. 

4. Все переезды на заказных трансферах, в общем как и было. 

5. Заброска всего общественного груза на лошадях, а личное - по желанию. Но
напомню, что теперь мы ночуем в Альтихат и за один день на Метео не нужно идти. 

6. Каждые две программы - больше перерыв в цепочке , чтобы тимлидеры лучше
отдыхали

7. Питание - никаких сублиматов, нормальная вкусная еда

8. Включили отель в Степанцминде после спуска с горы

9. Саму программу укоротили, сделав график более оптимальным

10. Берем участников только с опытом походов, хотя бы летних Карпат. Без опыта
- нет.

11. Ну и бонусом - красивая медаль о восхождении, которую можно будет повесить
на нашу же медальницу)) 

Еще мы убрали базовый пакет - мы старались не отличаться в цене от самых
дешевых предложений. Но давать хороший цервис и стараться сделать самую дешевую
цену - это нереально. Решили уходить от данной схемы. Давать оптимальный сервис
по адекватной цене. Без излишеств, но с должным сервисом и комфортом. 

В общем, пока много дат и мест - выбирайте удобные. Я уверен, места будут
заканчиваться быстро)) 

Мы работаем над улучшением и стараемся сделать лучшие предложения. 

\#kuluarpohod \#kuluarclimbinf \#kazbek

фото: Nata Mostova

\ii{16_12_2021.fb.pozdnij_taras.1.kazbek.cmt}
