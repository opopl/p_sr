% vim: keymap=russian-jcukenwin
%%beginhead 
 
%%file 17_04_2022.stz.news.ua.pravda.2.budynok_volontery_istorii_harkov
%%parent 17_04_2022
 
%%url https://life.pravda.com.ua/authors/6244d751ecd43
 
%%author_id lapshin_oleksandr,news.ua.pravda
%%date 
 
%%tags 
%%title Як мешканці одного будинку стали волонтерити разом. Історії з Харкова
 
%%endhead 
 
\subsection{Як мешканці одного будинку стали волонтерити разом. Історії з Харкова}
\label{sec:17_04_2022.stz.news.ua.pravda.2.budynok_volontery_istorii_harkov}
 
\Purl{https://life.pravda.com.ua/authors/6244d751ecd43}
\ifcmt
 author_begin
   author_id lapshin_oleksandr,news.ua.pravda
 author_end
\fi

\ii{17_04_2022.stz.news.ua.pravda.2.budynok_volontery_istorii_harkov.pic.1}

\begin{zznagolos}
У Харкові активно працюють волонтерські організації. Всі вони об'єднані
спільною метою допомоги, але по-різному її втілюють.

Одна з історій волонтерства – про співмешканців одного житлового будинку, які
згуртувалися та стали єдиним організмом заради допомоги іншим.

Розповідаємо, як сусіди, які до війни активно не спілкувалися, стали опорою
один для одного та для інших жителів міста.
\end{zznagolos}

\ii{17_04_2022.stz.news.ua.pravda.2.budynok_volontery_istorii_harkov.pic.2}

\textbf{Від автора:}

Ми познайомилися з Льохою через біг. Спочатку бачили та коментували тренування
один одного у Strava (мобільний застосунок для спортсменів), а потім домовилися
про спільний довгий, недільний крос. Це було на початку 2021 року.

За приблизно дві години бігу харківськими вулицями та пагорбами ми багато про
що встигли поговорити з Олексієм.

Зараз я вже не пригадаю усіх деталей, але в моїй пам’яті добре закарбувалося
загальне враження про міцну духом, загартовану фізично, вольову людину. І ще
одна деталь – у переважно російськомовному Харкові, Олексій спілкувався
українською. І це викликало повагу.

Після цього ми довгий час з ним не бачилися. Лише продовжували
\enquote{лайкати} нові тренування.

Так тривало аж до початку березня цього року, коли ми випадково зустрілися з
Льохою біля одного з харківських перехресть. Ця подія в той день мене дуже
вразила, і я написав на своїй сторінці у Facebook.

\ii{17_04_2022.stz.news.ua.pravda.2.budynok_volontery_istorii_harkov.pic.3}

Після цього я був певен, що Льоха зараз в ТРО. І через власне фотографічне
бажання побачити, як там у них все облаштовано, чим саме живуть хлопці та як
несуть службу, писав та пропонував йому зустріч. І за деякий час вона таки
відбулася!

Але те що я побачив, та про що дізнався з багатьох годин розмов, вразило мене
більше? аніж очікуваний військовий побут. Я дізнався про стихійно створену
волонтерську організацію.

Її засновниками, учасниками, натхненниками та постійним працівниками є звичайні
люди – мешканці однієї багатоповерхівки. Саме такі, які кожного мирного ранку
їхали поруч з вами в вагоні метро на роботу. А зараз залишились в Харкові та,
ризикуючи власним життям, допомагають іншим.

Далі – пряма мова мешканців будинку.

\raggedcolumns
\begin{multicols}{2} % {
\setlength{\parindent}{0pt}

\textbf{Аня Верба}

\emph{фармацевт-косметолог}

\end{multicols} % }
