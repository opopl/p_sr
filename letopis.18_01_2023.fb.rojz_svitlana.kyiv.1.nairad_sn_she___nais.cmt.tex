% vim: keymap=russian-jcukenwin
%%beginhead 
 
%%file 18_01_2023.fb.rojz_svitlana.kyiv.1.nairad_sn_she___nais.cmt
%%parent 18_01_2023.fb.rojz_svitlana.kyiv.1.nairad_sn_she___nais
 
%%url 
 
%%author_id 
%%date 
 
%%tags 
%%title 
 
%%endhead 

\qqSecCmt

\begin{itemize} % {
\iusr{Іванна Остапчук}

Якщо не аналізувати причини, а це все одно хтось щось зробив не потрібне, або
не зробив потрібне, то який сенс рухатися далі? Буде повторення без аналізу

\begin{itemize} % {
\iusr{Світлана Ройз}
\textbf{Іванна Остапчук} у мене немає слів про те, що не потрібно аналізувати. Є слова про те, що потрібно дочекатись розслідування

\iusr{Іванна Остапчук}
\textbf{Світлана Ройз} сьогодні син в мене спитав у чому причини усіх трагедій, я, щоб відповісти, сіла і почала виписувати усі 10 заповідей, щоб окремо по кожній роз'яснити на життєвих прикладах. І ось приклад не забарився
\url{https://www.facebook.com/100003543583440/posts/5544952482299483/}

\iusr{Татьяна Тарасюк}
\textbf{Іванна Остапчук} у Вас достатньо інформації для аналізу?
\end{itemize} % }

\iusr{Олечка Оса}

У мене вже повний ступор... після Дніпра, і сьогодні ранок... я не знаю як брати себе в руки, я здалась(((

\begin{itemize} % {
\iusr{Світлана Ройз}
\textbf{Олечка Оса} 

давайте по крокам. почнемо з того, що ви зробите собі чай солодкий, якщо
можливо, з лимоном. Руки під воду поставте, трохи потримайте і вмийтеся
холодною водою. Поставте, якщо можливо музику - щось, не пов'язане із
вербальною інформацією. Розітріть тіло. почніть чи прибирати, чи пригадайте, що
мали зробити. Якщо є можливість йти на вулицю, до людей, щоб бачити людей - це
потрібно. Якщо у вас є контакт із психологом - будь ласка, зверніться

\iusr{Alisa Klymenko}
\textbf{Олечка Оса} 

мені дуже допомагає така вправа: відчути, як торкається ваш одяг вашого тіла, в
якому воно положенні, де твердо, де м'яко. Коли після цього сам відбудеться
глибокий видих - значить ви тут.

Як тільки подумками ви \enquote{відлітаєте у непродуктивні роздуми} - робіть цю
вправу. Щоденно. Через місяць ви помітите збільшення свого ресурсу.

\iusr{Олечка Оса}
\textbf{Світлана Ройз} дякую 🙏

\iusr{Олечка Оса}
\textbf{Alisa Klymenko} дякую, спробую 🙏

\iusr{Veronika Volovnykova}

Я так само... навіть не почула як волає моє маленьке дитя ...😓

\iusr{Татьяна Крупина}
\textbf{Світлана Ройз} Дуже Дякую Вам за поради!

\iusr{Елена Селезнева}
\textbf{Світлана Ройз} Жах Біль. Смуток. Серце рве на куски. Нічого не спасає.
Господи зупини. Земля здригається від болю🙏🏻🙏🏻🙏🏻

\iusr{Тетяна Ільніцька}
\textbf{Світлана Ройз} я перемила весь можливий і неможливий посуд. Але від болю мені хотілося його розколошматити

\iusr{Viktor Vinnik}
\textbf{Олечка Оса} не здаватися, на те проклятий ворог і надіється. Тримаймося, як потрібно сильній нації і ми переможемо...

\iusr{Светлана Михайловна}
\textbf{Світлана Ройз} все бісить!!! Плакати вже немає сил

\iusr{Алла Куценко}
\textbf{Олечка Оса}!
Так може бути з тiлом...
Душа мiцнiша. Прошу, дайте собi час i момент тепла.
Хоча б помислити тепло про загиблих... Ви можете!

\iusr{Halyna Voloshyn}
\textbf{Світлана Ройз} дякую Вам за поради і підтримку!❤️ Мені після ступору тільки прибирання, фізична праця і допомогла.

\iusr{Елена Гриценко}
\textbf{Світлана Ройз} дякую! Спасибо!

\iusr{Галина Пасічник}
\textbf{Олечка Оса} я Вас розумів бо зі мною коється це саме


\end{itemize} % }

\iusr{Elvira Golik}

Для мене є справжнім болем, коли лунає сирена, і я розумію, що мій тепленький
комочок будять під час денного сну в садку і ведуть через вулицю в підвал. Я
навіть забирала раніше, аби дитина поспала, по суті, нехтуючи безпекой. Але
останнім часом стала радіти, що мала там в безпеці. Тобто садочок почав
асоціюватися з безпекою. Але я неодмінно повинна бути поряд, працювати поряд. А
зараз це відчуття суцільної небезпеки завжди, кожну хвилину - воно не
відпускає. Ті кляті ракети, що не можливо виявити, падаючі на дитячі садки
гвинтокрили - як взагалі жити? Не допомагає ані прості заспокійливі, ані
гідозепам. Я допомогти собі? Як зібрати в купу? Як скинути з себе цей страх
неконтрольованого зла, що приносить секундну смерть, та, навіть страшніше - не
секундну....

\begin{itemize} % {
\iusr{Світлана Ройз}
\textbf{Elvira Golik} 

так, я сьогодні теж думаю, про те, що нам залишилось в \enquote{контурах безпеки}. Але
саме зараз важливо спробувати повернути контроль в тому, над чим він точно є.
Будь ласка, подумайте, що ви можете зробити саме зараз для себе. Коли дитина
повернеться з садочку - ми маємо залишатись для неї \enquote{простором безпеки}. Саме
тому, будь ласка, зробіть для себе те, що можливо.

\iusr{Alisa Klymenko}
\textbf{Elvira Golik} 

мені дуже допомагає одна вправа: відчути, як торкається ваш одяг вашого тіла, в
якому воно положенні, де твердо, де м'яко. Коли після цього сам відбудеться
глибокий видих - значить ви тут.

Як тільки подумками ви \enquote{відлітаєте у непродуктивні роздуми} - робіть цю
вправу. Щоденно. Через місяць ви помітите збільшення свого ресурсу.

\iusr{Galyna Kutsan}
\textbf{Elvira Golik} саме такі думки в мене!
\end{itemize} % }

\iusr{Tatyana Solomka}

Кілька місяців, починаючи з жовтня, з початку регулярних обстрілів, вдалося
вибудували якийсь алгоритм дій, щоб якось вберегтися. Виє сирена - готуємось -
вильоти - їдемо з дітьми до батьків (там трохи безпечніше) - донька в школі
(там дорослі і є укриття).... Тепер все ... Нема більше алгоритму... Сирени не
завжди попереджають... будинки розлітаються від ракет... В будь-який момент може
прилетіли, звідки не чекав....останні трагедії показують такі звичайні
ситуації, в яких ми постійно буваємо... Хтось просто йде повз будинок...в нього
летить ракета; батько трохи затримався вдома, коли діти і дружина вже вийшли...
І він загинув...

Такі буденні ситуації, а закінчуються таким горем... Куди від цього сховатися?

\begin{itemize} % {
\iusr{Світлана Ройз}
\textbf{Tatyana Solomka} 

теракти на це і розраховані. Щоб ми втратили взагалі відчуття безпеки. Щоб були
в паніці, а потім в апатії. Щоб припинили супротив. Я ще, насправді, не знаю,
які нові алгоритми нам потрібні. Точно покаи важливо триматись старих. І
зробити все можливе, щоб не піддатись на провокації терористів

\iusr{Tatyana Solomka}
\textbf{Світлана Ройз} дякую 🙏
\end{itemize} % }

\iusr{Svetlana Kozlova}

Я сегодня опять подумала, не буду ли жалеть, что не уехали, и что не увезла.
Как я буду потом жить с этим, если. Светлана, а вы думаете об этом?

\begin{itemize} % {
\iusr{Svetlana Kozlova}

Мне нравится, как вы пишете. У вас теплая атмосфера. С вами хорошо, спокойно.

\iusr{Світлана Ройз}
\textbf{Svetlana Kozlova} я теж цілий день думаю про це. Але трагедії і теракти розраховані на те, щоб вибити опору. Я сама собі казала - якщо є сумнів - їдь.

\iusr{Ірена Смирнова}
\textbf{Svetlana Kozlova}, я повернулася, бо там, куди ми поїхали, ті люди, що приймали нас, довели до того, що моїй дитині довелося викликати швидку та везти в лікарню... Підняли все місто, мер виділив нам тимчасово свою квартиру для проживання (Італія). Немає безпечних місць навіть в Європі. Так що безпека в нас. Ми самі собі БЕЗПЕКА.

\iusr{Svetlana Kozlova}
\textbf{Ірена Смирнова} Наверное, и я так примерно себе отвечаю .

\iusr{Svetlana Kozlova}
\textbf{Світлана Ройз} Сумнівів немає?

\iusr{Світлана Ройз}
\textbf{Svetlana Kozlova} зараз - ні.

\iusr{Svetlana Kozlova}
\textbf{Світлана Ройз} дякую
\end{itemize} % }

\iusr{Елена Дюльгер}

Так, ми зцеплені 5аразі один з одним міцно. В цьому сила і небезпека одночасно.
Коли кожен раз дивлячись на рік народження загиблого Воїна порівнюєш його з
роком народження твоєї дорослої але дитини, це страшно. Наші діти малі, дорослі
вони наші. Спільні. І дуже болить за кожного! І потрібна сила і мудрість
роз'єднувати їх долі. Інакше ми самі робимо недобре для них. Це м'яко кажучи. Я
усвідомлюю це, я трохи знаю як це працює завдяки багаторічній роботі в
растановках. І знаю також як це складно. По собі. Дякую Світлано, що нагадуєте.
\enquote{Я поважаю ваш вибір. В вас своя доля, а в мене своя. Я знаю, що ви впораєтесь}

\iusr{Mike Kaufman-Portnikov}

Колись дон Карлеоне сказав, що буде звинувачувати кожного з \enquote{партнерів}, якщо щось станеться в його родині

Ми завжди будемо розуміти, хто винен, замовник

\iusr{Kseniya Rudenchenko}

Дякую! І буду вдячна все життя за всі швидкі практики, які я з дня в день
повторяю🙏

Зараз розумію, що Ваш курс дав тоді набагто більше!

Перше з чим я прокидаюсь - благословіння на Життя! Перше, що я роблю - розділяю
долі - це моє життя, а це твоє🙏

З чим засинаю - слова вдячності за день!

\iusr{Svetlana Kresina}

Саме пологи мне сьогодні на цілий день закрили від цього жаху. Тобто народження
життя накрило и захистило. Молюся благословіння Життям!

\iusr{Victoria Zhyr}

Дякую. Дуже вчасно...❤️🩹

\iusr{Nataliya Prokopchuk}

дякую! дуже цінно!

\iusr{Tetiana Zavhorodnia}

Дякую за вчасний допис. Я одна з тих багатьох мам, яка після страшної звістки
мала непереборне бажання бігти через Київ і забрати молодшого із садочка.
Благословляю своїх дітей на життя. Молюся за невинних янголят, яких вже не
заберуть сьогодні ввечері додому...

\iusr{Veronika Volovnykova}

Знаєте, подвійно серце країть новина. Я особисто знаю одного з загиблих
посадовців, ми сім,ями спілкувалися, чоловіки приятелювали. Це в першу чергу
оглушило і невимовне співчуття до дружини, дітей, і як це страшно..., знов
побігли думки що я хочу пошвидше повернутися до чоловіка, життя
коротке...приміряю на себе ситуацію....тут же і думки за діток із садочка, про
яких будуть менше говорити ніж про міністрів, але кожна мама буде приміряти на
себе це горе і страх за своїх дітей...і ось думаєш, як жити в серії цих новин,
як сподіватися на краще...чому ми мусимо з цього навчитися- прийняти
невідворотність долі, швидкоплинність життя?? Готувати себе до життя вічного,
більше дбати про душу і жити кожну секунду і радіти...це з одного боку
очевидно, а з іншого...горе від втрат завжди буде горем, гірким, важким...

\iusr{Yuliya Miroshnichenko}

Дякую Вам! Хтось повинен був мені зараз це сказати, бо......
Дякую! Благослови, Господи, усіх нас на життя!

\iusr{Юля Нікітіна}

Бровари пов'язані з моїм життям там чоловік працює, батьки живуть. Поборола
свій страх і повела свою дитину в садок після тривалої перерви. І так плакала
за тих діток, за батків, які відвели своїх діток і більше ніколи їх не
побачуть. Весь жах в тому, що на їхньому місті міг бути кожний (як і у Дніпрі).
Йшла забирати дитину перед сном, зайшла в церкву, помолилась з батюшкою,
поставила свічку. Залишається тільки молитись(

\iusr{Maria Nadra}

Як же то болить... після Дніпра, тої дівчинки Марійки зловила себе що то ж така
як моя дочка(((...але я кожного разу як тільки мій підліток йде гуляти
проводжаю словами \enquote{з Богом!} Часом так щоб і не чула... але як важливо

\iusr{Valentyna Kropyvko}

Дяка за Благословення) лиш усвідомила, як нам важливі благі слова та новини. Тут. Зараз. Завжди

\iusr{Olena Varchenko}

Годину не можу нічого робити, руки наче не моі((( дякую за підтримку.


\iusr{Nataliya Gerovska}

Як же цей Ваш пост був потрібний!! Дякую, що Ви завжди поруч❤️❤️❤️

\iusr{Olga Manko}

Світлано, те, що ви робите, дуже цінно!!! Ваш вчорашній ефір так допоміг мені
зібратися сьогодні. Відірвалася від новин і налила собі теплого чаю (ще до
цього посту). Повторила: \enquote{Я обираю життя}. Дякую вам!!!

\iusr{Maria Koval}

У мене немає дітей але і у мене проскочила думка що ніяких дитсадків. Потім
наздогнало, ой, мені ж нікого туди водити.

Зараз дуже сильне спільне поле і так, треба якось відділяти себе, змінювати
фокус уваги...хоч це і не просто... дуже.. Бо так, наша сила та енергія нам
знадобиться.

Дуже дякую за допис

\iusr{Oksana Levchenko}

Це жахлива наочна ілюстраціяз моїх найяскравіших страхів, що я відведу дитину в
садок... і потім ніколи її не побачу.

Співчуття всім хто зазнав страшної втрати сьогодні.


\end{itemize} % }
