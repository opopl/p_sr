% vim: keymap=russian-jcukenwin
%%beginhead 
 
%%file 06_09_2021.fb.fb_group.story_kiev_ua.1.dom_lapinskogo.cmt
%%parent 06_09_2021.fb.fb_group.story_kiev_ua.1.dom_lapinskogo
 
%%url 
 
%%author_id 
%%date 
 
%%tags 
%%title 
 
%%endhead 
\zzSecCmt

\begin{itemize} % {
\iusr{Олена Цюкевич}

Ой, і наш дворик! яке цінне фото!

\begin{itemize} % {
\iusr{Ольга Чекрыгина}
\textbf{Олена Цюкевич} а дворик дома 57? Я там часто бывала. Подруга жила в двухэтажном домике, стоявшем вдоль улицы

\iusr{Олена Цюкевич}
\textbf{Olga Chekrygina} ,58
\end{itemize} % }

\iusr{Vladimir Aniskov}
Превосходно

\iusr{Оксана Денисова}
\textbf{Vladimir Aniskov} Спасибо большое!

\iusr{Раиса Карчевская}
Оксана!
Большое спасибо за пост, очень познавательно и интересно

\iusr{Оксана Денисова}
\textbf{Раиса Карчевская} Спасибо огромное!

\iusr{Вячеслав Стрелец}

В детстве лазили по отселеным домам, их активно сносили в конце 70х, этот дом
был между ними.

Там под домом огромная каретная в подвале, тогда была заброшена, забита хламом.

\begin{itemize} % {
\iusr{Оксана Денисова}
\textbf{Вячеслав Стрелец} Как интересно! Здорово!

\iusr{Сергій Бєлах}
\textbf{Вячеслав Стрелец} 

а от уявіть, як би зараз так активно зносили, як в кінці 70-х! Скільки б було
крику. А тоді, нічого, норм. І зараз ніхто не згадує, скільки позносили старого
Київа при савєтах. Але все одно Клічко знищує Київ.

\begin{itemize} % {
\iusr{Владимир Писаренко}
\textbf{Сергій Бєлах} , тоді зносили старий фонд - який не можна було відремонтувати. Новобудови не порушували норм містобудування (зараз цим грішить кожна споруда!).
Впрочем, \enquote{какая разница} (с)

\iusr{Вячеслав Стрелец}
\textbf{Владимир Писаренко} на Ленина 41 снесли дом в котором жила оперная певица Мирошниченко, дом был 2 этажный но кирпичный, добротный, дружил с её сыновьями , живу через дорогу.
Не все дома были деревянными.
По сей день во дворах 70-76 по Богдана есть двух этажки

\iusr{Сергій Бєлах}
\textbf{Владимир Писаренко} 

да, да, не порушували норм містобудування... Панельні дев'ятиповерхівки в
історичному центрі на Шовковичній та Басейній зовсім не порушують норм
містобудування. Весь Поділ затиканий хрущовками ніразу не порушує норм
містобудування. Ага. Та, авжеж, \enquote{какая разніца} (с). Зате пломбир який смачний
був.


\iusr{Владимир Писаренко}
\textbf{Сергій Бєлах} , 

то тепер давайте все запаскудимо, якщо раніше, виявляється, було щось подібне?!
Мова про інше - висотність та відсутність належної інфраструктури. Знести
хрущовку - плове діло. Ви \enquote{Halmolnd}, \enquote{Oazis} або \enquote{монстра на Подолі} знесіть!

\end{itemize} % }

\iusr{Ольга Чекрыгина}
\textbf{Вячеслав Стрелец} мы тоже там лазили в конце 70-х. Дорога из 91 школы занимала гораздо больше времени, чем в школу

\iusr{Вячеслав Стрелец}
\textbf{Ольга Чекрыгина} 

я, в 58 ходил по Ленина, где и живу, там точно такие двух этажки, но иногда
ехал на 71 от шапито до Чкаловского  @igg{fbicon.laugh.rolling.floor} ,

\end{itemize} % }

\iusr{Татьяна Максимец}
Улица моего детства - каждый день ходила мимо этого прекрасного дома в школу
@igg{fbicon.heart.sparkling} 

\begin{itemize} % {
\iusr{Оксана Денисова}
\textbf{Татьяна Максимец} По- хорошему завидую Вам, люблю эту улицу!

\iusr{Игорь Запольский}
\textbf{Татьяна Максимец} А я напротив него работаю )

\iusr{Диана Ишунина}
\textbf{Игорь Запольский} Здание напротив тоже требует ремонта

\ifcmt
  ig https://scontent-frx5-1.xx.fbcdn.net/v/t39.30808-6/241627143_1744214125768742_7555882249011876194_n.jpg?_nc_cat=100&ccb=1-5&_nc_sid=dbeb18&_nc_ohc=O2mQZc8WQXMAX-3ac4j&_nc_ht=scontent-frx5-1.xx&oh=00_AT_WUFOF0cL56f1s4LOacwrM6Eo0afHnGRyGzFEs_VV0Eg&oe=62020CB7
  @width 0.3
\fi

\iusr{Valerii Gontarenko}
\textbf{Татьяна Максимец} Та й 91 школа була непогана. Шкода, що парк за будинком ортопедія знищила

\iusr{Ольга Чекрыгина}
\textbf{Татьяна Максимец} любимая улица, тот же путь в 91 школу мимо этого дома

\end{itemize} % }

\iusr{Anna Kievatz}
Как же так, что архитектор неизвестен?

\begin{itemize} % {
\iusr{Оксана Денисова}
\textbf{Anna Kievatz} Так бывает... в Киеве это не единственный красивый дом, архитектор которого неизвестен.

\iusr{Anna Kievatz}
\textbf{Оксана Денисова} Увы...
\end{itemize} % }

\iusr{Alya Mykhailova}
Оксаночка, вчера были на Вашей замечательной экскурсии! Благодарим за чудесные 3 часа! До встречи!@igg{fbicon.heart.red} @igg{fbicon.hands.applause.yellow}  @igg{fbicon.face.happy.two.hands} 

\iusr{Оксана Денисова}
\textbf{Alya Mykhailova} Спасибо Вам огромное за отзыв! И буду ждать Вас на других моих экскурсиях!!!

\iusr{Таня Сидорова}
Дякую Оксана, як завжди цікаво. Плануєте екскурсію?

\begin{itemize} % {
\iusr{Таня Сидорова}
\textbf{Оксана Денисова} дякую

\iusr{Наташа Сазонова}
\textbf{Оксана Денисова} спасибо!!!

\iusr{Ksenia Sereda}
\textbf{Оксана Денисова}, спасибо

\iusr{Таня Сидорова}
\textbf{Оксана Денисова} дякую

\iusr{Alla Manzhelii}
\textbf{Оксана Денисова} спасибо, тогда до встречи, зашла в Ваши публикации, оторваться не могу.

\iusr{Оксана Денисова}
\textbf{Alla Manzhelii} Спасибо Вам большое!
\end{itemize} % }

\iusr{Наталія Крюкова}
Спасибо за публикацию!

\iusr{Татьяна Оксаненко}

Очень интересно и познавательно. Помню, что родители так и называли этот дом,
домом Лапинского, чуть дальше, ближе к площади Победы, жил друг детства моего
отца, все мы часто его навещали. Как же так случилось, что неизвестно имя
архитектора такой роскошной усадьбы Спасибо, Вам огромное! ВЫ КАК ВСЕГДА на
высоте!

\begin{itemize} % {
\iusr{Оксана Денисова}
\textbf{Татьяна Оксаненко} Спасибо Вам! Вот такой загадочный дом, не сохранилось имя архитектора в архивах.

\iusr{Татьяна Оксаненко}
\textbf{Оксана Денисова} очень жаль..
\end{itemize} % }

\iusr{Елена Шевченко}
 @igg{fbicon.hands.pray} @igg{fbicon.heart.red}{repeat=3}

\iusr{Татьяна Абрамчук}
Спасибо за публикацию

\iusr{Nina Danilko}
Работаю рядом и всегда восхищаюсь этим домом сейчас такого не строят, а старое ломают

\iusr{Оксана Денисова}
\textbf{Nina Danilko} Будем надеяться, что такие красивые дома сохранятся!

\iusr{Ирина Иванченко}

Как вы умеете так \enquote{вкусно} преподать киевскую старину, Оксана, просто
прелесть. Каждый ваш пост читаю с удовольствием! Каюсь, обещала быть на
экскурсию, но всё никак не соберусь \enquote{с мыслями}... Но не теряю оптимизЬму всё же
быть... Благодарю вас, очень интересно.

\iusr{Оксана Денисова}
\textbf{Ирина Иванченко} Спасибо Вам! И приходите на экскурсии, на них интересно @igg{fbicon.grin} 

\iusr{Петр Кузьменко}

Благодарю! Очень интересно и познавательно! Это нужно знать киевлянам и всем, кто любит Город. @igg{fbicon.hands.applause.yellow} 

\iusr{Оксана Денисова}
\textbf{Петр Кузьменко} Спасибо Вам!

\iusr{Ольга Кожедуб}
Спасибо, Оксана! Очень красиво написано и, как всегда познавательно. Как попасть к Вам на \enquote{Тайны Золотоворотского квартала}?

\iusr{Оксана Денисова}
\textbf{Светлана Шварц} Напишу обязательно!

\iusr{Светлана Шварц}
\textbf{Оксана Денисова} спасибо @igg{fbicon.heart.red}

\iusr{Savelyeva Olena}
Очень интересно. Спасибо.

\iusr{Оксана Денисова}
\textbf{Savelyeva Olena} Спасибо!

\iusr{Людмила Прищепа}
Дякую за ваш труд ! Пізнавально і цікаво !@igg{fbicon.heart.red}

\iusr{Оксана Денисова}
\textbf{Людмила Прищепа} Спасибо!

\iusr{Roman Sytnyk}
А чому нема згадки, що в цьому домі мешкав Симон Петлюра?

\iusr{Тома Храповицкая}

Спасибо Вам огромное Оксана! Вы как всегда великолепны! Так преподнести,
преподать историю, красоту и величество Дома и его жителей! Восхитительно!!!

\iusr{Оксана Денисова}
\textbf{Тома Храповицкая} Спасибо Вам огромное!

\iusr{Людмила Сидоренко}

Я була в цьому будинку в комунальній квартирі. Запам'яталися високі стелі, десь
біля 4 метрів. Дуже дивно виглядали сучасні меблі в кімнаті 36 кв.м. До стелі
було ще 2м.


\iusr{Оксана Денисова}
\textbf{Людмила Сидоренко} да, там внутри очень интересно!

\iusr{Дмитрий Сливный}
Часто рядом прохожу. В маленьком флигеле сейчас костюмерная

\iusr{Эллина Перельман}
\textbf{Дмитрий Сливный} раньше флигель занимали скульпторы

\iusr{Juri Stefan Dubrovski}

в этом доме жила близкая подруга моей матери и в моем далеком детстве мы с
мамой часто бывали у нее... я хорошо помню фойе лестничной группы с лифтами,
если не изменяет память оставшиеся еще со времени постройки, квартиры в этом
доме конечно были перестроены властью рабочих и крестьян, но выветрить дух
романтики, таинственности, дух другой эпохи никакая власть не в состоянии ... и
конечно это архитектурная удача, взять детали дома, поэтажное разнообразие
оконных проемов и сандриков над ними в 3-х стилях: готики, ренессанса и
рационального модерна,... могу сказать, что архитектор был крепкий профессионал
будучи на ты с разными стилями архитектуры, но четко подчиняя разнообразие форм
единству замысла, а еще там потрясающая лестница подиума ведущая ко входу в дом

\begin{itemize} % {
\iusr{Оксана Денисова}
\textbf{Juri Stefan Dubrovski} спасибо Вам за такие интересные воспоминания! Да, архитектор был явно очень талантлив! И я надеюсь, что может быть все- таки эта тайна - кто строил этот дом- будет разгадана!

\iusr{Juri Stefan Dubrovski}
\textbf{Оксана Денисова} в архивах Софии должно быть

\iusr{Оксана Денисова}
\textbf{Juri Stefan Dubrovski} Когда делали «Свод памятников архитектуры» собирали данные из всех архивов, но так фамилии архитектора и не нашли.

\iusr{Juri Stefan Dubrovski}
\textbf{Оксана Денисова} жаль, а есть там имя арх. автора проекта дома Ричарда на Андреевском?

\iusr{Оксана Денисова}
\textbf{Juri Stefan Dubrovski} Проект дома был украден у петербургского архитектора Роберта Марфельда, кто- то из Киевских архитекторов его переделал, но кто поставит свою подпись под украденным проектом?

\iusr{Juri Stefan Dubrovski}
логишь, а архитектор Марфельд, вы знаете о нем больше, он не фортификационный проектировщик?

\iusr{Оксана Денисова}
\textbf{Владимир Федорович} Не знаю...

\iusr{Оксана Денисова}
\textbf{Juri Stefan Dubrovski} Мало что о нем знаю, только то, что написано в Википедии
\end{itemize} % }

\iusr{Maksim Pestun}

Окна папиной квартиры, где он вырос, выходят прямо на этот дом.

\begin{itemize} % {
\iusr{Оксана Денисова}
\textbf{Maksim Pestun} Завидую! По- хорошему @igg{fbicon.grin} 

\iusr{Олена Цюкевич}
\textbf{Maksim Pestun}, а в каком доме он жил? Можете это наш сосед?

\iusr{Maksim Pestun}
\textbf{Олена Цюкевич} Чеховский 3. Квартира до сих пор сохранилась за нами

\iusr{Олена Цюкевич}
Нет...

\iusr{Monya Feldman}
\textbf{Maksim Pestun} а я смотрел с окна на квартиру твоего папы.. Давно это было...
\end{itemize} % }

\iusr{Евгения Верченко}
Дякую дуже!
Цікаво дуже, про братів Булгакових не знала...

\iusr{Оксана Денисова}
\textbf{Евгения Верченко} Дякую!

\iusr{Elena Dubovenko}

Спасибо за подробности(!), о Булгаковых, связанных с Лапинским совсем не знала.
И фотография прекрасна! Парадные лестницы в доме мраморные, в комнатах под
высокими потолками изящная лепка, в квартирах были печки, потом их убрали и
появилось отопление, а на месте печей были хозяйственные ниши. Благодаря
высоким потолкам, из кухни лестница ведёт на довольно объемные антресоли,
используемые как дополнительное хоз. место.

\iusr{Оксана Денисова}
\textbf{Elena Dubovenko} Спасибо Вам за такие чудесные подробности о доме!

\iusr{Калерия Ходос}
Красивый дом ,,самое главное его сохранить и не дать снести и поставить на этом месте новый монстр стеклянную многоэтажку.

\iusr{Оксана Денисова}
\textbf{Калерия Ходос} Будем надеяться, что этот чудесный дом сохранится!

\iusr{Оксана Велкова}
Мой муж вырос в доме напротив. Загадочное место

\iusr{Ирина Нецора}
Спасибо, очень интересно.

\iusr{Оксана Денисова}
\textbf{Ирина Нецора} Спасибо!

\iusr{Лариса Артемчук}
Ул. Гончара 33?
Там ведь и Столыпин жил? Так?

\iusr{Оксана Денисова}
\textbf{Лариса Артемчук} 

Нет, это Гончара 60. А в доме 33 была клиника Маковского, где Столыпин умер
после покушения на него в Городском театре.

\iusr{Евгения Ерёменко}
Спасибо, Оксана  @igg{fbicon.face.smiling.eyes.smiling} 

\iusr{Оксана Денисова}
\textbf{Евгения Ерёменко} Спасибо Вам!

\iusr{Татьяна Сирота}

Прекрасная публикация @igg{fbicon.hearts.two} 
Спасибо огромное!

\iusr{Оксана Денисова}
\textbf{Татьяна Сирота} Спасибо Вам!

\iusr{Lyudmila Kravcova}
Удивительный дом!.. спасибо!

\iusr{Георгий Готесман}

Во! А я всё ждал - когда же в жтой рубрике (\enquote{КИ}) он возникнет! Наконец-то... В
возрасте лет 12, будучи в гостях у двоюродного дедушки Наума (бывшего военного
врача, а в мирное время - имевшего лицензию для приема на дому больных с
кожно-венерическими заболеваниями) заприметил на тогдашней улице Чкалова (дед
жил на Гоголевской) это чудо. Проник в него и даже прокатился на крохотном
антикварном лифте. Фасад также был светло-жёлтого цвета. Это здание врезалось в
память навсегда!

\begin{itemize} % {
\iusr{Оксана Денисова}
\textbf{Георгий Готесман} Рада, что оправдала Ваши ожидания и дом возник @igg{fbicon.grin}  и как Вам повезло, что Вы были внутри!

\iusr{Георгий Готесман}
\textbf{Оксана Денисова} Спасибо, Оксана! Но тогда это было совсем не сложно: год стоял 1966, либо 1967-й.

\iusr{Оксана Денисова}
\textbf{Георгий Готесман} Да, сейчас внутрь дома попасть нелегко  @igg{fbicon.face.pensive} 

\iusr{Александра Лобынцева}
\textbf{Оксана Денисова} А что там сейчас?

\iusr{Оксана Денисова}
\textbf{Александра Лобынцева} Просто жилой дом, дверь закрыта.
\end{itemize} % }

\iusr{Liudmila Boiko}

Дякую за прекрасну публікацію. Мов близькі родичі прожили в цьому домі все
життя, а ж по не почали продавати квартири які перетворили на комунальні, щоб
\enquote{роз'іхатись}. Мені теж пощастило на протязі року спостерігати
мармурові білі сходи в парадному, скляну стелю, до якої підіймався найстаріший
в Києві ліфт, привезений з Америки. Квартири по 250 кВ. Метрів із стінами по
півметра товщиною. Каміни оздоблені чеськими кахлями, але більшість була
замурована. Я жила в кімнаті, яка колись належала прислузі. З неі був
додатковий вихід в коридор, що вів до роздільних санвузлів і кухні, з якої був
вихід на \enquote{чорний} хід , до якого за будинком доставляли дрова і продукти. Перед
містком був \enquote{паліладнік} - місце відпочинку, садок. Під мостиком
окремий вхід в квартиру двірника. Все продумано, комфортно. Всього 12 квартир.

\begin{itemize} % {
\iusr{Оксана Денисова}
\textbf{Liudmila Boiko} Какие интересные подробности! Спасибо Вам огромное!!!

\iusr{Elena Dubovenko}
\textbf{Liudmila Boiko} щиро дякую, все так!

\iusr{Эллина Перельман}
В каждой из квартир 7 комнат.
Каждая комната( если повезло, 2) принадлежали разным семьям.
\end{itemize} % }

\iusr{Александр Муратов}

Спасибо! С 1952 года, с первог курса мединмтитута этот дом был для меня
загадкой, Я тогда ходли в дом напротив, где изучал \enquote{Латынь} и \enquote{Микробы}. . Эот
дом удивлял могие годы. Кому ни задавал вопросы, никто ничего вразумительног
сазвть не мог. Только став нерабтающим песионром, овладев компом за бгром,
начал встречать в паутине сообщения о братьях Штенгелях, об их домах. Узнал ,
что в молодости изучал ортопедию и травматологию в дном из их домов на
Бульв.-Кудрявской, примыкавшем \enquote{с тыла} к этому загдочному дому.

\begin{itemize} % {
\iusr{Оксана Денисова}
\textbf{Александр Муратов} Да, Вы изучали ортопедию в доме Барона Рудольфа Штейнгеля!

\iusr{Олег Верхоградов}
У Булгакова в \enquote{Мастер и Маргарита} был персонаж барон Майгель. Возможно, прообраз взят отсюда.

\iusr{Оксана Денисова}
\textbf{Олег Верхоградов} Может быть...
\end{itemize} % }

\iusr{Олена Кириченко}
В этом доме рос мой папа!

\iusr{Элина Юрчак}

Присоединяюсь к многочисленным благодарностям за познавательный пост автору, а
также спасибо за не менее интересные комментарии.

\iusr{Оксана Денисова}
\textbf{Элина Юрчак} Спасибо Вам!

\iusr{Людмила Петровна}

Я возле этого дома жила всю жизнь. И дом этот мне нравился своей загадочностью.
Спасибо вам за такую историю. Вы молодец.

\iusr{Оксана Денисова}
\textbf{Людмила Петровна} Спасибо огромное!

\iusr{Olena Klymenko}
Очень интересный пост!!! Спасибо большое, Оксана! Как всегда, очень познавательно!!!!

\iusr{Оксана Денисова}
\textbf{Olena Klymenko} Спасибо большое!

\iusr{Наталия Лебедь}

Давно в журнале Юность был напечатан очерк известного киевлянина Виктора
Некрасова \enquote{Городские прогулки}, где он описывал время учебы на архитектурном
факультете, который тогда находился на Большой Владимирской недалеко от этого
дома и каждый раз, проходя мимо, конечно же любовался им. Так вот он в этом
очерке написал, что автор толи сам Городецкий, толи кто-то из его учеников.
Много раз я пыталась найти этот очерк и перечитать, но не смогла найти, а в
книге с почти одноименным названием об этом ничего не сказано. Действительно
ЗАГАДКА.

\begin{itemize} % {
\iusr{Оксана Денисова}
\textbf{Наталия Лебедь} Да, версия о Городецком была, но она не подтвердилась.

\iusr{Juri Stefan Dubrovski}
\textbf{Наталия Лебедь} 

стиль действительно похож на Городецкого и его творчество, по годам и положению
заказчика проекта это мог сделать только он, как выдающийся архитектор того
времени, которому было интересно и у него были полностью развязаны руки, не
стоит забывать про большой земельный участок под проект, а такие возможности
можно доверить человеку с большим именем, результатами и известными объектами

\end{itemize} % }

\iusr{Олена Кириченко}

Моего папу в этот дом принесли с роддома в 1949 году и жил он там по 1962 год в
квартире номер 3

\begin{itemize} % {
\iusr{Оксана Денисова}
\textbf{Олена Кириченко} Как Вашему папе повезло жить в таком красивом доме!

\iusr{Олена Кириченко}
\textbf{Оксана Денисова} это да, спасибо Оксана

\iusr{Олена Кириченко}
Столько интересных и теплых воспоминаний
\end{itemize} % }

\iusr{Александра Лагно}
Спасибо. Интересно, как сейчас выглядит этот дом?

\begin{itemize} % {
\iusr{Оксана Денисова}
\textbf{Александра Лагно}

\ifcmt
  ig https://scontent-frx5-1.xx.fbcdn.net/v/t39.30808-6/241427030_10158671290227198_6124940526365505591_n.jpg?_nc_cat=100&ccb=1-5&_nc_sid=dbeb18&_nc_ohc=db606VCRi4IAX8XQr35&_nc_ht=scontent-frx5-1.xx&oh=00_AT_9qVm9eMXXktdXd9KxnqfhEH2GAzYBIQAkDlw5uLJZSQ&oe=620241AA
  @width 0.3
\fi

\iusr{Александра Лагно}
\textbf{Оксана Денисова}
Спасибо большое.

\iusr{Alya Mykhailova}
\textbf{Александра Лагно} дом выглядит как и прежде, а вот \enquote{дворницкая} удивляет своей похабной нарядностью! @igg{fbicon.face.rolling.eyes} 

\iusr{Наталия Журавская}
\textbf{Александра Лагно}
Почти так же. Только кое-где окна пластиковые...

\iusr{Оксана Денисова}
\textbf{Наталия Журавская} Этот дом как раз неплохо сохранился!

\end{itemize} % }

\iusr{Неда Окопова}

Большое спасибо за превосходную публикацию! В школьные годы очень много ходила
этими улицами, дом остался в памяти навсегда, хотя после окончания школы мы
переехали в другой район.

Вы описали все с великолепной глубиной!

\iusr{Оксана Денисова}
\textbf{Неда Окопова} Спасибо Вам большое!

\iusr{Татьяна Чута}
Очень интересная история, шикарная архитектура....

\iusr{Татьяна Чувилева}
Спасибо!!))

\iusr{Тати Берг}
Работала там, в физиотерапевтической поликлинике, неск. лет назад.

\iusr{Любовь Сыроватка}

Как странно жизнь поворачивает судьбами людей. Как она проста для одних и как
сложна для других. Как мы все перед ней, судьбой, голые. Что хочет, то
пишет на наших телах и судьбах. Удивительно.


\iusr{Георгий Горбенко}
Напротив него был филиал нашего института ( КИА ).

\begin{itemize} % {
\iusr{Оксана Денисова}
\textbf{Георгий Горбенко} Точно, был!

\iusr{Георгий Горбенко}
\textbf{Оксана Денисова} Частенько там бывал. У нас было 5 филиалов ( один в Черновцах, там тоже очень красивое старинное здание, около Университета - там тоже часто бывал.)
\end{itemize} % }

\iusr{Маргарита Марченко}
Это один из моих любимых уголков/домов Киева) спасибо!

\iusr{Оксана Денисова}
\textbf{Маргарита Марченко} Спасибо тебе!

\iusr{тетяна огнєва}
А що зараз із цим будинком?
Він житловий чи там знаходиться якась організація?

\begin{itemize} % {
\iusr{Оксана Денисова}
\textbf{тетяна огнєва} Дом жилой, на первом этаже офисы.

\iusr{тетяна огнєва}
\textbf{Оксана Денисова} Щиро дякую!

\iusr{Алла Суббатовская}
\textbf{Оксана Денисова} А какой адрес? Я, хоть и киевлянка, но не была возле этого дома. Но вот все планирую пройтись по улицам Старого Киева!

\iusr{Оксана Тукалевская}
\textbf{Алла Суббатовская} Гончара, напротив почто что сквера Чкалова. Не пропустите

\iusr{Оксана Денисова}
\textbf{Алла Суббатовская} Улица Гончара 60

\iusr{Алла Суббатовская}
\textbf{Оксана Денисова} Спасибо!
\end{itemize} % }

\iusr{Marina Lentz}
Интерсно, спасибо.

\iusr{Ирина Архипович}
Много интересных подробностей!!! Спасибо!!  @igg{fbicon.hand.ok}  @igg{fbicon.thumb.up.yellow}  @igg{fbicon.heart.eyes} 

\iusr{Ирина Карманова}

Всем нравится это здание (нас много), зданию требуется ремонт и реставрация,
или... начнем собирать деньги, когда оно начнет гореть или разрушаться?

\begin{itemize} % {
\iusr{Оксана Денисова}
\textbf{Ирина Карманова} Вы знаете, в здании несколько лет назад был капремонт и оно в очень неплохом состоянии.

\iusr{Ирина Карманова}
Спасибо за информацию, приятно удивили. Сами жильцы сделали ремонт?

\iusr{Оксана Денисова}
\textbf{Ирина Карманова} Честно, не знаю...
\end{itemize} % }

\iusr{Tamara Khmara}
Я забыла, этот дом находится недалеко от бывшего Сенного рынка?

\iusr{Людмила Петровна}
\textbf{Tamara Khmara} Да это находиться возле Сеного рынка. Сеного рынка лестница спускается на улицу Гончара (Чкалова). Вы праведно сориентировались.

\iusr{Татьяна Гурьева}
Какое величественные здание, наверное, населено призраками, полное очарования и мистики

\begin{itemize} % {
\iusr{Оксана Денисова}
\textbf{Татьяна Гурьева} Мне оно тоже кажется немного мистическим...

\iusr{Татьяна Гурьева}
\textbf{Оксана Денисова} прекрасно Вами описано)

\iusr{Оксана Денисова}
\textbf{Татьяна Гурьева} Спасибо большое!
\end{itemize} % }

\iusr{Вадим Менжулин}
С домом профессор Лапинский перещеголял своего учителя - профессора Сикорского.

\begin{itemize} % {
\iusr{Оксана Денисова}
\textbf{Вадим Менжулин} Дом Сикорского тоже был красивым, может отреставрируют все- таки!

\iusr{Вадим Менжулин}
\textbf{Оксана Денисова} 

мне кажется, восстановить уже не удастся. Максимум - построят что-то новое с
какой-то репликой старого, как в посольстве Японии на Пейзажке

\iusr{Оксана Денисова}
\textbf{Вадим Менжулин} 

Ой, не расстраивайте меня. Я каждый раз на экскурсии захожу в этот двор, чтобы
показать дом и рассказать об Игоре Сикорском, и боюсь, что дом развалился.


\iusr{Вадим Менжулин}
\textbf{Оксана Денисова} Будем надеяться!

\iusr{Вита Вовченко}
\textbf{Оксана Денисова} Уже вряд ли. Его сознательно доводят до состояния \enquote{под снос}.

\iusr{Maksim Pestun}

и даже без клада!..
\end{itemize} % }

\iusr{Oleksandr Tarashchuk}
В прошлом году посетил этот дом. Мостик в плачевном состоянии ((

\iusr{Оксана Денисова}
\textbf{Oleksandr Tarashchuk} Нет, его уже укрепили.

\iusr{Инна Иванова}

Люблю Киевские истории за такие посты. Не всегда есть возможность попасть в
такие шедевры архитектуры, и всегда интересно читать комментарии. Прочитываешь
все до единого. Спасибо за пост и комменты.


\iusr{Оксана Денисова}
\textbf{Инна Иванова} Спасибо!

\iusr{Людмила Гриценко}
Дякую за цікавий допис!

\iusr{Оксана Денисова}
\textbf{Людмила Гриценко} Дякую!

\iusr{Svetlana Turenko}
Спасибо автору публикации. Очень интересно и познавательно.

\iusr{Оксана Денисова}
\textbf{Svetlana Turenko} Спасибо Вам!

\iusr{Афанасенко Тетяна}
Захотелось посетить @igg{fbicon.flame} 

\iusr{Victoriya Yarovaya}
Дякую, цікаво

\iusr{Олег Верхоградов}
Фото, видимо, 30-х г.г.

\begin{itemize} % {
\iusr{Оксана Денисова}
\textbf{Олег Верхоградов} 20-е - 30-е, где- то там

\iusr{Олег Верхоградов}
\textbf{Оксана Денисова} 20-е - нет. Автомобиль на фото выпускался с 1932-го по 1936-й год.

\iusr{Оксана Денисова}
\textbf{Олег Верхоградов} Спасибо Вам, как интересно! Теперь буду знать точнее дату фотографии!

\iusr{Олег Верхоградов}
\textbf{Оксана Денисова} Но в 40-х они могли быть еще в обиходе.
\end{itemize} % }

\iusr{Ludmila Kool}
Очень интересная история.

\iusr{Оксана Денисова}
\textbf{Ludmila Kool} Спасибо!

\iusr{Марьяна Винер}
А мы жили чуть повыше в 46 доме. Он не сохранился.

\begin{itemize} % {
\iusr{Оксана Денисова}
\textbf{Марьяна Винер} Это на углу с Чеховским переулком, жалко, что не сохранился @igg{fbicon.face.pensive} 

\iusr{Vladimir Davidenko}
\textbf{Mariana Wiener} я жил в 45-В

\iusr{Марьяна Винер}
Вид из окна 46 дома.

\ifcmt
  ig https://scontent-frt3-1.xx.fbcdn.net/v/t39.30808-6/241162702_2894731437444352_4402993308141148260_n.jpg?_nc_cat=102&ccb=1-5&_nc_sid=dbeb18&_nc_ohc=ehQoQnnGolwAX-DXHYF&_nc_ht=scontent-frt3-1.xx&oh=00_AT86fwqeHEg9IorqaDPsOMR71IklZxliMV-FgGHSfyqcGQ&oe=62018946
  @width 0.2
\fi

\iusr{Марьяна Винер}
Я рисовала с этюда моей мамы.

\iusr{Марьяна Винер}
\textbf{Vladimir Davidenko} от твоего дома остался фасад.

\end{itemize} % }

\iusr{Любовь Сыроватка}

Для меня конечно Киев город Герой. Но я маленькая сибирячка его не знала, но
очень любила, но больше любила город герой Днепропетровск, там мой дед был
героем. Он толком даже не поучаствовал в войне. Его убили практически сразу,
сначало контузило, а потом его дострелила зондеркоманда. Прям возле села бой
был. К стати все бойцы были вооружены от силы вилами и изредка ружьями. Дед,
молодой мужик на то время с палкой пошол. Их там просто переехали. Вот и все
участие в боях моего деда. Сначало взрывом контузило, а потом дострельная
немецкая команда в упор его расстреляла. А он, Леонтий Жуков жил, дышал,
пока жена его Феня не пошла поискать на поле боя возле дома, хотя бы тело мужа
своего ночью. И нашла таки и привезла расстрелянного молодого мужика своего с
соседнего поля боя. Выходила. А дальше если захотите дальше расскажу.

\begin{itemize} % {
\iusr{Ольга Кузнецова}
\textbf{Любовь Сыроватка} Расскажите, как жизнь сложилась у них? Герой! Ужас какой пережили...

\iusr{Любовь Сыроватка}
\textbf{Ольга Кузнецова}, 

я Вам расскажу за сестру Фени, бабушки моей. Звали её Аннушка. Я была тогда
детей пятилетним, а тут бабушки мои. И причём не родные, мама и тетя отчима
моего. Ну и соотвественно дед. Герой наш контуженный. Аннушка была злючкой.
Не замужней бабушкой, но со страшной судьбой. В первые дни войны она со своим
любимым человеком расставаться не хотела, ездила с ним на бензовозе до точек
дислокации. Пока в их машину снаряд вражеский не попал. Парень загорел заживо
, а Аннушка обгорела, но жива осталась. Потом так одна и прожила, злючкой
была, меня любила, хоть и не родня я ей была. И к стати и бабушка Феня меня
любила и дедушка Лёня. Я их очень люблю и помню, как самое яркое событие
детства. Желаю им быть под Божьим благословением.

\iusr{Наталія Кравчук}
\textbf{Любовь Сыроватка} И какое отношение Ваш рассказ имеет к этой публикации ? Или Вы чего- то нам не договорили?

\iusr{Валентина Саєцька Пащенко}
\textbf{Любовь Сыроватка} расскажите, очень интересно.

\iusr{Людмила Петровна}
\textbf{Любовь Сыроватка} 

Да расскажите. Мне это интересно. Напишите мне пожалуйста. Мой папа тоже
воевал. Правда папе было 15 лет. Буду ждать вашего ответа. Это хорошо что вы
знаете эти истории семьи.

\iusr{Людмила Петровна}
\textbf{Любовь Сыроватка} Напишите мне на страничку про вашего дела. Я буду ждать у себя на страничке.

\iusr{Ольга Денисова}
\textbf{Любовь Сыроватка} Пожалуйста расскажите что было дальше

\iusr{Любовь Сыроватка}
\textbf{Ольга Денисова} , напишу .

\end{itemize} % }

\iusr{Светлана Блаус}

Отличная публикация! О связи Булгакова с хозяином дома прочитала впервые. Мой
любимый институт геологии ( об этом здании тоже есть, что написать) находится
на противоположной стороне улицы. Он и сегодня выглядит величественно!

\begin{itemize} % {
\iusr{Оксана Денисова}
\textbf{Светлана Блаус} Да, прекрасное здание, строил ведь Александр Кобелев, прекрасный был архитектор!

\iusr{Светлана Блаус}
\textbf{Оксана Денисова} 

Так все таки известно, кто архитектор! Я жила в 30 номере, чуть выше по улице.
Интересно, кто был архитектором нашего дома? Да, там сплошные шедевры
архитектуры

\iusr{Оксана Денисова}
\textbf{Светлана Блаус} Нет, это я про здание напротив, его строил Александр Кобелев.

\iusr{Вячеслав Стрелец}
\textbf{Оксана Денисова} 

институт геологии в убитом состоянии с тех времён когда я в \enquote{рябинку} ходил,
детсадик с тыльной стороны, а это был 70-73, ремонта небыло и внутри и снаружи

\end{itemize} % }

\iusr{Lidiya Novynska}
А що, ще не розвалився? А то років 30 тому там хтось намагався заселитись, а затим \enquote{добудувати}

\iusr{Мария Дроботько}
Большое спасибо, очень интересно!

\iusr{Оксана Денисова}
\textbf{Мария Дроботько} Спасибо!

\iusr{Panavas Pranas}
RA VEDA Vilnius BlagoDAriu @igg{fbicon.sun.with.face}  @igg{fbicon.flame}  @igg{fbicon.sun.with.face} 

\iusr{Вита Вовченко}

Спасибо огромное автору публикации! Уж сколько про этот дом читалось, а всё
равно что-то новое(про Булгаковых). Я живу в доме справа, и ежедневно вижу этот
необыкновенный замок. Даже с тыльной стороны))) И комментарии все адекватные,
приятно было прочесть все, без исключения.

\begin{itemize} % {
\iusr{Оксана Денисова}
\textbf{Вита Вовченко} Спасибо большое!

\iusr{Олексій Хорунжий}
\textbf{Вита Вовченко} Только, наверное, не в том 2-х этажном доме, который на фото, а 13-этажном, построенным в 1982 году?

\iusr{Вита Вовченко}
\textbf{Олексій Хорунжий} Разумеется)))
\end{itemize} % }

\iusr{Marina Alekseeva}
Шикарный

\iusr{Наталия Платонова}

Как я люблю этот волшебный дом! Киев - самый невероятный ГОРОД на Земле, и его
населяли ( и населяют) невероятные ЛЮДИ. Сколько волшебных птиц, вылетело из
этого ГНЕЗДА...

\begin{itemize} % {
\iusr{Оксана Денисова}
\textbf{Наталия Платонова} Спасибо Вам за Вашу любовь к Городу!

\iusr{Наталия Платонова}
\textbf{Оксана Денисова} ГОРОДУ срасибо!
\end{itemize} % }

\iusr{Татьяна Данилова}

Огромное спасибо, очень интересная информация!
@igg{fbicon.hands.applause.yellow}  @igg{fbicon.thumb.up.yellow}
@igg{fbicon.hands.pray} 

\iusr{Оксана Денисова}
\textbf{Татьяна Данилова} Спасибо!

\iusr{Татьяна Зубко Маркина}
Спасибо за прекрасный рассказ

\iusr{Людмила Ива}
Спасибо за интереснейшую историю!))

\iusr{Елена Острат}
Уникальное здание! А в архиве разве нельзя посмотреть и узнать кто архитектор ?

\begin{itemize} % {
\iusr{Оксана Денисова}
\textbf{Елена Острат} Есть \enquote{Свод памятников архитектуры}, когда его готовили, работали со всеми архивами, но сведения об архитекторе этого дома утеряны.
\end{itemize} % }

\iusr{Александр Атавин}

\obeycr
Тікали....з свого дому за Світи
Від кривди, злоби та пороцтва
Ріднеє добро й свої любії хати
Лишали, аби подалі....скотства.
Нездійсненність і смерть мети
Хто зовсім мер, а хто лишився,
На чужині, не свій чужий де ти,
Рід наш Українським залишився.
\restorecr

\iusr{Іллона Зейкан}

Так архітектори Ніколаєв і Сичугов. В цьому будинку жила бабуся моєї подружки, була там один раз

\begin{itemize} % {
\iusr{Оксана Денисова}
\textbf{Іллона Зейкан} 

Николаев и Сычугов строили дом Барона Штейнгеля, где была водолечебница и от
которого остался только фрагмент на Бульварно-Кудрявской.

\iusr{Іллона Зейкан}
\textbf{Оксана Денисова} інформацію 'перенесли' на весь будинок )
\end{itemize} % }

\iusr{Юлія Малиміна}

Хотілося б звернутися до тих, хто, можливо знає щось про Людмилу Деркач. Вона
якось пов'язана з цим будинком. Хоча і значно пізніше, ніж описані події. Перед
війною 1941 р. вступила до медичного інституту.

Можливо, хтось поділиться інформацією про цю людину.

\iusr{Ірина Юшко}
Дякую за просвіту.

\iusr{Natalia Pigulevska}
Сейчас он в ужасном состоянии, практически развалился...

\begin{itemize} % {
\iusr{Вита Вовченко}
\textbf{Natalia Pigulevska} 

Нет. Разваливается ограда с улицы, ступеньки. Там нюансы насчёт ремонта есть. А
квартиры сдаются в аренду. Со стороны института ортопедии на нижних этажах с
двух сторон жильцы сделали себе что-то наподобие мини частных двориков. И
что-то вроде художественной студии на втором этаже справа.

\begin{itemize} % {
\iusr{Natalia Pigulevska}
\textbf{Вита Вовченко} я видела только со стороны Гончара. Впечатление, что ещё немного и восстановить это будет уже невозможно...

\iusr{Светлана Чепкевич}
\textbf{Вита Вовченко} Интересно увидеть фото, одно из последних.

\iusr{Вита Вовченко}
\textbf{Natalia Pigulevska} 

Они хотели вообще развалить самую крайнюю ограду, но мы вызвали комиссию. Там
оказались строгие нормы по ремонту, так как дом - памятник архитектуры. Поэтому
закрыли зелёным профнастилом, и он разрушается теперь уже за ним...


\iusr{Вита Вовченко}
\textbf{Светлана Чепкевич} Сейчас не могу: болею. Когда смогу, сфотографирую и с тыльной стороны.

\iusr{Natalia Pigulevska}
\textbf{Вита Вовченко} я давно там была, но вид был ужасный... Почему-то мне кажется что сейчас только хуже может быть.

\iusr{Lyudmila Beregova}
\textbf{Вита Вовченко} Оооо. такую студию и я бы не против иметь!

\iusr{Диана Ишунина}
\textbf{Светлана Чепкевич} Сам дом еще ничего, а вот заборы и ступеньки просто рассыпаются. Это две недели назад.

\ifcmt
  ig https://scontent-frt3-2.xx.fbcdn.net/v/t39.30808-6/241351724_1744228405767314_8091889834613471779_n.jpg?_nc_cat=103&ccb=1-5&_nc_sid=dbeb18&_nc_ohc=yr2ifeuhyFoAX8zwMYY&_nc_oc=AQnjGuFEb9nnG6_XUWKwkTInv5vVu0__uidO-A0kOai24HOg_HROM1-06fEVFo0w0X0&_nc_ht=scontent-frt3-2.xx&oh=00_AT-alawD6_PPyBL1WX7zeqJxcOT8KVFFWnemDlCO1EiHSg&oe=62029C4E
  @width 0.2
\fi

\iusr{Вита Вовченко}
\textbf{Natalia Pigulevska} 

Разумеется, за такими жемчужинами нужно ухаживать! А у нас нет такой тенденции
с 1922года. Посмотрите, что сотворили с замком Штейнгеля! Его просто разрезали
пополам, и одну часть, с башенкой, предоставили саморазрушению. Обнесли
профнастилом и повесили таблички \enquote{Будьте осторожны! Здание разрушается, падают
фрагменты!} И я уж совсем промолчу о том, как можно было делать коммуналки
внутри дома Лапинского(а это полная перепланировка, вынос стен и
соответственный стресс для здания...

\iusr{Вита Вовченко}
\textbf{Lyudmila Beregova} 

А то! Довольно симпатично смотрится, а с улицы и не просматривается, лишь со
стороны института. Спрятан от людского ока, так сказать)))

\iusr{Диана Ишунина}
\textbf{Вита Вовченко} Фото из палаты. Печаль печальная(

\ifcmt
  ig https://scontent-frx5-1.xx.fbcdn.net/v/t39.30808-6/241626582_1744309865759168_6705745344513334277_n.jpg?_nc_cat=110&ccb=1-5&_nc_sid=dbeb18&_nc_ohc=2XUV9ibXMOcAX8O2PAw&_nc_ht=scontent-frx5-1.xx&oh=00_AT-7Wj8wTt5yfNkJsCYR9vVWbNjSlwoRjC7-OOXy5kZiWQ&oe=6201BAD8
  @width 0.2
\fi

\iusr{Вита Вовченко}
\textbf{Diana Ishunina} 

Именно эту часть здания я имела ввиду. А там такая арка необыкновенная внизу!
Одно из самых богатых строений в дореволюционном Киеве! А другая часть, со
стороны Б.-Кудрявской, имеет вроде пристойный вид, но мне рассказывали
старожилы, что там было уникальнейшее зеркало, привезенное из Венеции. Ещё в
Сов. Союзе оно мистически исчезло. Никто не знает, где оно: можно лишь
догадываться)))

\end{itemize} % }

\end{itemize} % }

\iusr{Lyudmila Beregova}

Благодарю Вас за уникальную информацию!
Радует, что дом еще жив. И здесь отозвались даже соседи)

\iusr{Александр Григорьев}
Красота тогда и красота сейчас

\iusr{Татьяна Даниленко}
Моя мама училась в 91 школе...

\iusr{Gennadiy Smertenko}

В детстве он казался старинной крепостью, с рвами и стенами, пожалуй этот
комплекс зданий один из самых интересных в старом центре.


\iusr{Оксана Денисова}
\textbf{Gennadiy Smertenko} Да, мне и сейчас этот дом кажется похожим на крепость, со рвами и башнями !

\iusr{Зоя Юліївна}
Котрого Гончара?

\iusr{Tatiana Thoene}
Неповторимый дом. Но выживет ли он, не уступит место ТРЦ или офисному монстру?

\iusr{Оксана Денисова}
\textbf{Tatiana Thoene} Будем верить, что устоит!

\iusr{Ірина Оснач}

Дуже дякую автору за розповідь про цей незвичайний будинок. Для мене це спогад
про дитинство. Колись (це був кінець 50-х років) мама взяла мене з собою, ми
поїхали в гості до її подруги дитинства. Льоля мала в цьому будинку кімнату в
одній з комунальних квартир, жила там з донькою. Мене вразили будинок, двір,
місток - наче я потрапила в казку! Треба обов'язково зберегти цю красу!

\iusr{Оксана Денисова}
\textbf{Ірина Оснач} Дякую!

\iusr{Розалия Гольдецки}

Сто раз я проходила и проезжала мимо и всегда восхищалась странной непохожестью
на другие дома.. всегда хотелось окунуться в тайну этого архитектурного
творения.. но я ничего не знала ни о бывшем владельце... ни о истории этого
удивительного дома... мне кажется, что там были ворота, закрытые, и что просто
так вокруг дома невозможно совершить экскурсию.. но всегда восхищалась этим
старинным волшебным замком...


\iusr{Оксана Денисова}
\textbf{Розалия Гольдецки} Ворота открыты и можно по ступеням подняться к дому!

\iusr{Розалия Гольдецки}
Браво архитектору

\iusr{Ольга Берг}
Емко и информативно, переплетение эпох и судеб- сериалы отдыхают, спасибо

\iusr{Оксана Денисова}
\textbf{Ольга Берг} Спасибо большое!

\iusr{Жаник Асатрян}

Это 🏠 дом Лапиского Михаила. Жил он в этом 🏠 доме до 1918 года. Профессор
психиатр купил этот 🏠 дом в 1908 году Михаил Лапинский у барона Рудольфа
Штейнгеля на улице., Бульварно-Кудрявская. Открыл там водолечебницу с модными
водными массажа и лечебными ванными. Лечебницу так и назвал
Бульварно-Кудрявскский санаторий.

\end{itemize} % }
