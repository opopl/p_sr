% vim: keymap=russian-jcukenwin
%%beginhead 
 
%%file 19_11_2020.news.ua.strana.1.nenuzhnaja_babushka
%%parent 19_11_2020
 
%%url https://strana.ua/news/301935-v-kieve-politsija-s-trudom-vernula-domoj-poterjavshujusja-babushku-ot-kotoroj-otkazalis-rodnye.html
%%author 
%%author_id 
%%tags 
%%title 
 
%%endhead 

\subsection{\enquote{Она нам не нужна}. В Киеве нашли потерявшуюся старушку, которуе родные отказывались забирать домой}
\label{sec:19_11_2020.news.ua.strana.1.nenuzhnaja_babushka}
\Purl{https://strana.ua/news/301935-v-kieve-politsija-s-trudom-vernula-domoj-poterjavshujusja-babushku-ot-kotoroj-otkazalis-rodnye.html}

21:51, сегодня

\ifcmt
pic https://strana.ua/img/article/3019/35_main-v1605815371.jpeg
caption Две дочери этой женщины заявили, что она им не нужна. Фото: Патрульная полиция
\fi

Семья из Житомирской области отказалась забирать домой свою бабушку,
которую полицейские обнаружили потерявшейся в Киеве.

\enquote{Бабушка нам не нужна}, - сообщили патрульным родные пожилой женщины,
сообщает Патрульная полиция Киева.

Уточняется, что 18 ноября на службу 102 поступило сообщение об обнаружении
потерявшегося человека. Охранник торгового центра рассказал полицейским,
что заметил бабушку, которая долгое время бродила, была растеряна,
выглядела очень устало. Мужчина обратился за помощью в полицию.

Пообщавшись с женщиной, патрульные узнали, что она родом из Житомирской области
и не понимает, как оказалась в Киеве. Полицейские позвонили к коллегам из
соседней области и выяснили номера телефонов дочерей потерявшейся женщины,
которая, как оказалось уже находилась в розыске как пропавшая без вести.

Радость инспекторов была не долгой, поскольку родные женщины заявили, что
забирать ее не будут, потому что она им не нужна.

Пожилую женщину привезли в Шевченковское отделение полиции, угостили чаем и
едой, оказали необходимую помощь.

В телефонном режиме инспекторы провели беседу с дочерьми, одну из них все же
убедили приехать в полицию и забрать свою мать домой. Данная ситуация находится
на контроле местного отделения полиции.

В полиции напомнили, что согласно ст. 51 Конституции Украины совершеннолетние
дети обязаны заботиться о своих нетрудоспособных родителях.

\ifcmt
tab_begin cols=2
	caption Полиция нашла в Киеве потерявшуюся бабушку, которая оказалась не нужна детям
	pic https://strana.ua/img/forall/u/10/79/%D0%B1%D0%B0%D0%B1(3).jpg
	width 0.3
	
	pic https://strana.ua/img/forall/u/10/79/%D0%B1%D0%B0%D0%B12(1).png
	width 0.5
tab_end
\fi

Ранее \enquote{Страна} писала о том, что пенсионерка из Полтавы собиралась пешком
дошагать до Киева, патрульные подобрали старушку на трассе.

Также мы сообщали, что в Нидерландах после заражения коронавирусом
выздоровела 107-летняя старушка.

