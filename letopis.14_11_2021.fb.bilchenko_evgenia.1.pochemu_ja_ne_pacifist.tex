% vim: keymap=russian-jcukenwin
%%beginhead 
 
%%file 14_11_2021.fb.bilchenko_evgenia.1.pochemu_ja_ne_pacifist
%%parent 14_11_2021
 
%%url https://www.facebook.com/yevzhik/posts/4440946225940453
 
%%author_id bilchenko_evgenia
%%date 
 
%%tags bilchenko_evgenia,pacifizm
%%title БЖ. Почему я не пацифист
 
%%endhead 
 
\subsection{БЖ. Почему я не пацифист}
\label{sec:14_11_2021.fb.bilchenko_evgenia.1.pochemu_ja_ne_pacifist}
 
\Purl{https://www.facebook.com/yevzhik/posts/4440946225940453}
\ifcmt
 author_begin
   author_id bilchenko_evgenia
 author_end
\fi

БЖ. Почему я не пацифист

Просто есть такой Логос - русский мир. Это совсем не страшно, это нежно и
круто. Это не значит, что других Логосов нет, настал "шовинизм" и капец-капец,
нет: есть романо-германский и буддийско-индийский миры, и Гёте прекрасен, и
Радхакришнан, и многие другие миры - восхитительны, кроме преступных, конечно.
И диалог между ними - цветущее поле, если это диалог, а не коварная игра в
него. И есть - маленький уязвимый наш мир. Это не значит, что нет личности,
настал "тоталитаризм" и капец-капец: это значит, что личность, если этот мир
находится у нее в кишках, сердце и мозгах по всем чакрам и во всю длину,
расширяется до границ космоса. Просто есть такой Логос - русский мир. Это не
значит, что нет социального равенства, настал "олигархизм" и капец-капец, ибо
русский мир - это космическое "чувствилище", это обострение чувства
справедливости, иногда до безумия. Потому в нем были белые и красные, Блок, 12.

\ifcmt
  ig https://scontent-frx5-1.xx.fbcdn.net/v/t39.30808-6/257549532_4440946425940433_470123707547795757_n.jpg?_nc_cat=110&ccb=1-5&_nc_sid=8bfeb9&_nc_ohc=sC0x5TgIdJcAX-nCwGc&_nc_ht=scontent-frx5-1.xx&oh=e6a7c7050c58cfa9f1b17dde105ab67f&oe=619BA541
  @width 0.4
  %@wrap \parpic[r]
  @wrap \InsertBoxR{0}
\fi

Потому я лично конфликта между здоровым классовым интернационализмом, здоровым
персональным индивидуализмом и здоровым классическим традиционализмом не вижу,
пока болезнь не начинает точить трёх атлантов (экономического, политического и
культурного) одного красивого здания: равенство, личность и память. Личность
становится либеральным истероидом, равенство - ностальгическим партийным
психозом, память - "чудовищем консерватизма" (выражение Дугина).

Особенно я печалюсь, когда вместе сталкиваются пламенные борцы за непонятно
что, часть из которых носит костюмы православных консервативных патриотов, а
вторая часть - православных либеральных пацифистов. Причем, если патриотов я
ещё могу понять: они обороняются от глобального мира в той детской форме, на
какую способны, они не понимают, что я им не враг, а друг, просто у меня
больший опыт борьбы с глобализмом изнутри системы и скиллов по этой части в
силу прошлого. Но их пластинка заела на ностальгическом языческом (на языке
церкви - этнофилитическом) фантазме этого прошлого, включая моё
(\#плохокаяласьru), и она не проворачивается. Но в перспективе сие исправимо. У
них только одна проблема: они мыслят верно, но не делают ничего и плохо
смотрятся в спорах с нашими оппонентами. Если быть честными: они заваливают
русскому миру нафиг информвойну.  То есть, вредят искренно, но подсознательно.
Советов слушать не хотят, ибо дети они все - вредные и упрямые.

Но ещё больше я печалюсь от православных "пацифистов" в стиле Льва Толстого, уж
простите, с самим Львом не путать, я сказала: "в стиле". Они все делают не так,
ещё и меня подшить хотят под свое "не так". Они берут нормальный тезис про
"единый братский народ" и начинают его релятивировать. То есть, всё зло
зарождается в цветастом постмодерне в попытке "релятивировать" дуализм. Это
подаётся как "свобода" и становится источником ужасной несвободы. "А давайте
посмотрим иначе...." При этой фразе у меня начинается мороз по коже, как будто
мне говорят: "Дурень, смотри на это, как я!"  Помните, как в разбираемом С..
Жижеком в "Киногиде извращенца" фильме "Чужие среди нас": герой не снимает
очки ч/б, чтобы видеть цветной мир, а, наоборот, надевает, чтобы понять: за
показной цветастостью  мнений скрывается ещё большая диктатура, чем в классике
ч/б. Она - полностью черна. 

Нам якобы даётся возможность быть "относительными" и "гуманными" и посмотреть
на нацизм и антинацизм как на "равно пострадавшие" стороны. Под этим предлогом
антинацизм умаляется, занижаясь от победителя к жертве, а от жертвы - к
агрессору. Соответственно, нацизм, превращаясь в жертву, косвенно оправдывается
всё больше и больше. В постмодерне работает закон story telling, или, в
простонародье, "окно Овертона": сначала нечто ужасное становится предметом
невинной дискуссии, затем - переосмысления и затем - принятия. Так в маске рая
приходит дьявол. "Дьявол - в деталях": это средневековое выражение повторил
один из военных генералов США о России: "We overcome them in details". Но и
молчать об этом, набрав в рот воды, как предлагают патриоты-изоляторы, - вообще
неверно, ибо "вытесненное возвращается" (старик Фрейд). Из чего следует, что я
не за изоляцию и не за коллаборацию, а за освобождение и последующую соборность
земли русской.

В силу раскрутки технологии дискурса о... поражает чисто постмодерное желание
православных пацифистов установить "братство" между русским народом и...
украинским националистическим режимом. Не между русскими и украинцами, ещё не
отравленными пропагандой, например, на Донбассе, а между русскими и теми
милитарными и хуторскими силами, которые уничтожают всех, кто любит русскую
культуру и пытается мне втереть, что русофобское нечто/ничто на месте моей
Украины и есть Украина.

Уважаемые господа православные пацифисты! Не следует подшивать меня под борьбу,
в которой я ощущаю дикую фарисейскую ложь. Я скорее - из сторонников жесткости
Ильина в культуре и нежности Ильенкова в социуме, нежели толстовец-хипстер. Эта
группа - не для поэта и, тем более, не для христианина, ибо имеет мало общего с
православным миротворчеством, но много общего с глобализмом и американизмом.
Запомните: я не пацифист. Не толстовец. Не ми-ми-ми. Можете считать меня
"радикалом, шовинистом, консерватором, перебежчиком", мне все равно. 

"Пацифизм" - это источник вечной войны, это облик, в котором нечистый искушает
Христа в пустыне, ибо нечистый никогда не является напрямую.

PS. На фото - кафе. Не устраивайте вой).

\ii{14_11_2021.fb.bilchenko_evgenia.1.pochemu_ja_ne_pacifist.cmt}
