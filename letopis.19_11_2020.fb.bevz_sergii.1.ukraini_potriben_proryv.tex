% vim: keymap=russian-jcukenwin
%%beginhead 
 
%%file 19_11_2020.fb.bevz_sergii.1.ukraini_potriben_proryv
%%parent 19_11_2020
 
%%url https://www.facebook.com/spollloh/posts/207961504113868
 
%%author Бевз, Сергій
%%author_id bevz_sergii
%%author_url 
 
%%tags 
%%title Сьогодні Україні потрібен справжній прорив
 
%%endhead 
 
\subsection{Сьогодні Україні потрібен справжній прорив}
\label{sec:19_11_2020.fb.bevz_sergii.1.ukraini_potriben_proryv}
\Purl{https://www.facebook.com/spollloh/posts/207961504113868}
\ifcmt
	begin_author
   author_id bevz_sergii
	end_author
\fi

\ifcmt
pic https://scontent.fiev6-1.fna.fbcdn.net/v/t1.0-9/126198025_207961167447235_753746908997673287_o.png?_nc_cat=105&ccb=2&_nc_sid=730e14&_nc_ohc=tRATo9mKKQcAX8sHECV&_nc_ht=scontent.fiev6-1.fna&oh=4d1e668289e1758d434bf7bcf3f9dbbd&oe=5FE60FE8
caption Нереалізований потенціал України
\fi

\begin{itemize}
\item Україна зайняла 42 місце в світовому рейтингу конкурентоспроможності
				талантів. За результатами дослідження наша держава суттєво випередила
				не лише Московію, а й більш розвинуті європейські країни. Зокрема,
				Словаччину, Угорщину та Хорватію.  

\item Результат, звісно, залишає бажати кращого. Проте 42 позиція в глобальному
				рейтингу конкурентоспроможності свідчить, що ситуація з людським
				капіталом в Україні далеко не катастрофічна. Тож, попри песимістичні
				оцінки наша держава зберігає неабиякий інтелектуальний потенціал.
				Незважаючи на безнадійно хвору політичну еліту, сумнівну якість
				державної освіти та деградацію суспільного загалу. Ми маємо безцінний
				скарб у вигляді тисяч обдарованих особистостей. Очевидно, що при
				належній політиці держави Україна  увійде принаймні у першу двадцятку
				за рівнем конкурентоспроможності талантів.

\item Ключова проблема полягає в тому, що в реаліях кланово-олігархічної
				держави Україна не може використати свій інтелектуальний потенціал.
				Тому багато перспективних  українських науковців змушені шукати
				працевлаштування за кордоном. Інші ж, не маючи змоги реалізувати себе,
				приречені на злидні або зміну фаху на Батьківщині. Керована
				олігархічними кланами та корупціонерами держава не просто не забезпечує
				належних можливостей для молодих талантів. Клептократична система
				виштовхує та деморалізує їх.

\item Саме тому попри велику кількість талановитих особистостей Україна
				залишається відсталою та неконкурентоспроможною. В опублікованому днями
				світовому рейтингу добробуту наша держава посіла лише 92 сходинку. Це
				найгірший результат серед країн Європи. Жителі держав з набагато меншим
				інтелектуальним потенціалом на кшталт Молдови чи Боснії випереджають
				українців за рівнем добробуту. Така картина цілком закономірна,
				зважаючи на пріоритети керманичів України з 1991. Це промовистий
				результат 29 з гаком років правління державою меркантильних істот, для
				яких національні інтереси --- це порожній звук. 

\item Сьогодні Україні потрібен справжній прорив. Для подолання відсталості й
				реалізації свого неоціненного потенціалу. Тільки розвиток людського
				капіталу дозволить нам це зробити. Ми маємо великий інтелектуальний
				потенціал, який при грамотній політиці відкриє широкі перспективи. Але
				поки при владі блазні та злодії, Україна приречена бути країною
				втрачених талантів і можливостей.
\end{itemize}
