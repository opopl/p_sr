% vim: keymap=russian-jcukenwin
%%beginhead 
 
%%file 05_08_2020.news.ru.53news.rogalev_mihail.1.staraja_russa_arheologia
%%parent 05_08_2020
 
%%url https://53news.ru/novosti/59624-arkheolog-elena-toropova-o-letnej-shkole-i-pervykh-nakhodkakh-v-staroj-russe.html
 
%%author Рогалев, Михаил
%%author_id rogalev_mihail
%%author_url 
 
%%tags staraja_russa,arheologia,russia
%%title Археолог Елена Торопова рассказала о новых находках в Старой Руссе
 
%%endhead 
 
\subsection{Археолог Елена Торопова рассказала о новых находках в Старой Руссе}
\label{sec:05_08_2020.news.ru.53news.rogalev_mihail.1.staraja_russa_arheologia}
\Purl{https://53news.ru/novosti/59624-arkheolog-elena-toropova-o-letnej-shkole-i-pervykh-nakhodkakh-v-staroj-russe.html}
\ifcmt
	author_begin
   author_id rogalev_mihail
	author_end
\fi

\index[cities.rus]{Старая Русса, Россия!Новые находки, археология, 05.08.2020}
\index[names.rus]{Торопова, Елена!Руководитель археологической экспедиции Новгородского
университета}

\ifcmt
pic https://53news.ru/images/wsscontent/articles/2020/08/arkheolog-elena-toropova-o-letnej-shkole-i-pervykh-nakhodkakh-v-staroj-russe.jpg
\fi

\textbf{На протяжении десятка лет в работе археологической экспедиции Новгородского
университета принимают участие волонтёры. О том, как всё начиналось и каких
результатов удалось достичь сегодня, корреспонденту «53 новостей» рассказала
исполняющая обязанности заведующей кафедры истории России и археологии НовГУ и
руководитель экспедиции Елена Торопова. Сразу отметим, в интервью не обошлось и
без рассказа о новых находках на раскопках, которые уже не первый год идут в
Старой Руссе.}

- Первые наши волонтёры увлеклись наукой, будучи участниками археологического
кружка, который в своё время создал \textbf{Павел Колосницын}, в прошлом — сам волонтёр,
- рассказывает Елена Торопова. - Для некоторых археология стала делом всей
жизни. К примеру, принимавшие на протяжении нескольких лет участие в раскопках
\textbf{Валерий Сюборов} и \textbf{Арина Юсифова} настолько прониклись духом этой увлекательной
профессии, что стали постоянными научными сотрудниками. Для этого они получили
профильное историческое образование, хотя ранее основной профессией первого
была механизация сельского хозяйства, а второй — филология.

Сравнительно недавно, в 2017 году, появилась идея пойти дальше — создать при
археологической экспедиции Новгородского университета летнюю археологическую
школу.\Furl{https://53news.ru/novosti/59189-novgorodskie-arkheologi-prodolzhat-izuchat-pyatnitskij-raskop-v-staroj-russe.html} Базой для неё стал Пятницкий раскоп в Старой Руссе, исследования на
котором ведутся при поддержке Российского фонда фундаментальных исследований,
по проекту «Городская усадьба средневековой Руссы». Как раз в то время фонд
«История Отечества» Российского исторического общества объявил конкурс на
участие молодёжи в археологических раскопках. Естественно, мы не могли пройти
мимо такой заманчивой перспективы. Уже в 2018 году при активной поддержке фонда
школа начала свою деятельность. Конечно же, дело не ограничилось одними лишь
раскопками — специалисты читали молодым людям лекции, проводили для них
экскурсии.

\ifcmt
pic https://53news.ru/images/images/2019/%D1%84%D0%B5%D0%B2%D1%80%D0%B0%D0%BB%D1%8C/19/f5FEv1JT8eo_1_1.jpg
\fi

К сожалению, в 2019 году в силу ряда обстоятельств нашу заявку фонд не
поддержал, но мы решили, несмотря ни на что, продолжить зародившуюся традицию.
Через краудфандинговую платформу объявили сбор средств на проведение
археологической летней школы. К счастью, всё получилось очень удачно — нужную
сумму собрать удалось, и очередная смена состоялась.

В нынешнем году нас вновь поддержал фонд «История Отечества» — благодаря его
гранту 27 июля школа открыла свой третий сезон, за что мы ему очень благодарны.
Нынче в неё зачислены 25 слушателей, пожелавших приобщиться к миру науки. В
основном это студенты-историки НовГУ, в том числе — первокурсники. В силу
известных всем и каждому обстоятельств они лишились летней археологической
практики, но археологическая школа помогла некоторым из них исполнить давно
лелеемые мечты. Кроме того, из Москвы и Санкт-Петербурга прибыли уже хорошо нам
знакомые волонтёры, в прошлом неоднократно работавшие на раскопках в Старой
Руссе. Школа будет действовать до 16 августа.

\textbf{- Успел ли вас порадовать нынешний сезон какими-либо находками?}

- Первая неделя работы Пятницкого раскопа выдалась очень непростой — необходимо
было убрать мусор, к тому же постоянно шли дожди. Тем не менее, на нашем счету
— несколько интересных находок. В первую очередь это створчатый браслет с
интересным изображением, по всей видимости, детский. А также кожаные ножны
длиной около полуметра. Есть предположение, что в них когда-то хранился меч или
иное холодное оружие, но эта версия нуждается в подтверждении.

\ifcmt
tab_begin cols=2
pic https://53news.ru/images/18/nkE1lnteuhs_1_1.jpg
pic https://53news.ru/images/images/2019/%D1%84%D0%B5%D0%B2%D1%80%D0%B0%D0%BB%D1%8C/19/Hw_LmVsRdyQ_1.jpg
tab_end
\fi

Последняя находка интересна тем, что подобные ей встречаются крайне редко. А
потому обнаруженная в раскопе вещь иногда служит своего рода спусковым
механизмом для длительного и увлекательного исследования, порой напоминающего
детектив. Будем изучать ножны и пытаться понять их истинное предназначение.

Буквально вчера была сделана ещё одна неординарная находка — детский янтарный
крестик.

А вообще основные работы ещё впереди, остаётся лишь трудиться и надеяться на
удачу и хорошую погоду.

Кстати, информацию об этих и других находках Старорусской археологической
экспедиции НовГУ имени Ярослава Мудрого можно почерпнуть в электронной базе
данных «Древности Новгородской земли».\Furl{https://www.novsu.ru/archeology/db/}

\emph{Фото: Павел Колосницын.}
