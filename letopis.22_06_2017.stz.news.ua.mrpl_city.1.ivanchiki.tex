% vim: keymap=russian-jcukenwin
%%beginhead 
 
%%file 22_06_2017.stz.news.ua.mrpl_city.1.ivanchiki
%%parent 22_06_2017
 
%%url https://mrpl.city/blogs/view/ivanchiki
 
%%author_id burov_sergij.mariupol,news.ua.mrpl_city
%%date 
 
%%tags 
%%title Иванчики
 
%%endhead 
 
\subsection{Иванчики}
\label{sec:22_06_2017.stz.news.ua.mrpl_city.1.ivanchiki}
 
\Purl{https://mrpl.city/blogs/view/ivanchiki}
\ifcmt
 author_begin
   author_id burov_sergij.mariupol,news.ua.mrpl_city
 author_end
\fi

\ii{22_06_2017.stz.news.ua.mrpl_city.1.ivanchiki.pic.1}

В то воскресенье - 22 июня сорок первого года - в Мариуполе с самого утра
установилась солнечная, теплая погода. Мариупольские хозяйки спозаранку
отправились кто в магазин, а кто на базар. Молодежь и курортники поспешили на
пляж, они располагались на песке, пробовали, не холодна ли вода, готовились к
завтраку на свежем воздухе, малыши ожидали разрешения старших окунуться в воду,
а ребята постарше очертя голову бросались в море, слегка покрытое утренней
рябью. Трамваи, весело звеня, подвозили все новых пляжников. День обещал быть
очень хорошим... Мариупольцы и их гости не знали, что уже несколько часов
идет война.

В полдень черные тарелки репродукторов в домах и квадратные раструбы
громкоговорителей, установленных в городе в людных местах, слегка заикающимся
голосом заместителя Председателя Совнаркома Молотова сообщили: \enquote{Фашистская
Германия напала на Советский Союз...} Пока еще до конца не осознанная тревога
вошла в мариупольские дома. Люди еще не знали, что каждому из них и каждому из
их родных и близких война уже отмерила свою чашу горя. Только одни изопьют ее
до дна в ближайшие дни, а другим предстоит пить ее долгие-долгие годы.

\ii{22_06_2017.stz.news.ua.mrpl_city.1.ivanchiki.pic.2.ivan_saverskij_kursant}

И в домике на третьей улице Слободки, красовавшимся узорчатой кладкой
наличников, известковой побелкой стен, аккуратной покраской оконных рам,
чистотой идеально промытых стекол, с замиранием сердца слушали Молотова. Здесь
жили две семьи: фотографа и сапожника. Жена фотографа Матрена была родной
сестрой сапожника Елисея Григорьевича. В каждой семье было по единственному
сыну - позднему ребенку, а потому особенно дорогому. У фотографа Григория
Кузьмича Саверского - чуть постарше, у сапожника Елисея Григорьевича Гмыри -
помоложе. Но разница была небольшая - какие-то месяцы. В тот роковой год Ваня
Саверский еще учился в Ейском училище морской авиации. А Ваня Гмыря только что
окончил восемь классов железнодорожной школы. Впрочем, может, и не этой?
Спросить-то теперь не у кого. Родители называли своих сыновей любовно
Иванчиками: Саверского - Иванчиком Черным, Гмырю - Иванчиком Белым.

В те мгновения, когда Саверские и Гмыри вслушивались в слова Молотова,
фашистские полчища уже топтали поля Украины и Белоруссии, шли ожесточенные бои
у Брестской крепости. Рушились под бомбовыми ударами дома, полыхали деревни,
падали на землю охваченные пламенем самолеты. Кровавая война началась. С ее
первых дней от Вани Саверского не было ни слуху ни духу. Это потом стало
известно, что с приближением врага училище перебазировали сначала в Моздок, в
сорок втором - в село Борское Куйбышевской области.  Наскоро сдавших выпускные
экзамены пилотов из Ейского училища морских летчиков отправили служить на
Дальний Восток: тогда еще угроза Японии стояла во весь рост. Это после
Сталинграда стало ясно, что японцы оставили в сторону планы нападения на
Советский Союз.

Потом в Мариуполь пришли немцы. Ваня Гмыря прятался, чтобы не угнали в
Германию на рабский труд. Сразу же одряхлевшие Григорий и Елисей перебивались
случайными заработка­ми, а Матрена Григорьевна молила Господа, чтобы он уберег
ее единственного сына. На короткое время успокаи­валась, когда у
соседки-гадалки выпадали карты, что Иванчик ее жив и здоров.

10 сентября сорок третьего Мариуполь освободили от фашистов. Подоспел срок
призыва в армию для Вани Гмыри. Он пошел вместе с мариупольскими хлопцами в
одной из маршевых рот на фронт. Теперь уже и Катерина Ивановна с Елисеем
Григорьевичем стали ждать письма от сына. И вскоре дождались. Пришла похоронка:
\enquote{Ваш сын, Иван Елисеевич Гмыря, скончался от ран, полученных в боях на реке
Молочной под Мелитополем...}

Наконец и Саверские получили весточку от своего Иванчика. Ее Матрена
Григорьевна достала из почтового ящика перед Новым, сорок четвертым годом.
Иванчик поздравлял родителей с освобождением Мари­уполя, сообщал, что едет на
Балтику бить фашистов. Три с половиной месяца плутало письмо по полевым почтам,
прежде чем достигло родительского дома. Это была последняя весточка от сына,
которую прочла Матрена Григорьевна. Второе - страшное - письмо получил
Григорий Кузьмич. Он не показал жене похоронку. Так до последнего своего часа
Матрена думала, что Иванчик ее жив. Нет, ни Саверские, ни Гмыри не искали могил
своих сыновей. Найти могилы значило для них навсегда потерять надежду увидеть
их. А они все-таки надеялись. Надеялись, что они живы...

Ушли один за другим в мир иной родители Иванчиков. Сегодня домишко на третьей
Слободке пуст. Обвалившая­ся штукатурка, глубокие щели между кирпичами стен,
хилый сеянец вишни с единственной ягодой, выросший перед дверью, покосившаяся
калитка, вросшая в зем­лю, - все говорит о том, что подворье давно опустело. И
здесь чаша горя, принесенного войной, была испита до дна...

22 июня сорок первого года в Мариуполе установилась солнечная погода. День
обещал быть очень хорошим...

**

P. S. Саверский Иван Григорьевич, 1923 года рождения, младший лейтенант 3-го
штурмового авиаполка Балтийского флота, погиб 25 марта 1944 года.

Гмыря Иван Елисеевич, 1923 года рождения, рядовой 899-го стрелкового полка,
умер от ран 19 ноября 1943 года. Похоронен в селе Гирсовка Приазовского района
Запорожской области.
