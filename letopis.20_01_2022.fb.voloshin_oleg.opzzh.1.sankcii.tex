% vim: keymap=russian-jcukenwin
%%beginhead 
 
%%file 20_01_2022.fb.voloshin_oleg.opzzh.1.sankcii
%%parent 20_01_2022
 
%%url https://www.facebook.com/oleg.voloshin.7165/posts/5119805168052503
 
%%author_id voloshin_oleg.opzzh
%%date 
 
%%tags politika,sankcii,ukraina,usa,voloshin_oleg.opzzh.deputat
%%title Приятно получать слова поддержки от столь широкого числа людей
 
%%endhead 
 
\subsection{Приятно получать слова поддержки от столь широкого числа людей}
\label{sec:20_01_2022.fb.voloshin_oleg.opzzh.1.sankcii}
 
\Purl{https://www.facebook.com/oleg.voloshin.7165/posts/5119805168052503}
\ifcmt
 author_begin
   author_id voloshin_oleg.opzzh
 author_end
\fi

Приятно получать слова поддержки от столь широкого числа людей. Приятно, что
даже оппоненты в большинстве признают, что причиной мои убеждения. Но особенной
изюминкой стали звонки от нескольких очень авторитетных западных журналистов,
экспертов по России и авторов множества статей и книг жёсткой антикремлевской
направленности, каждый из которых сказал: «Какого черта! Почему ты! Мы знаем
твои взгляды, не всегда их разделяем, но причём тут ФСБ?!»

Да и несколько влиятельных людей из американских христианских кругов не
поленились перезвонить и выразить свою солидарность. 

Так что обиды на какую-то страну нет и быть не может. Жаль только, что этот
некогда «град на холме» оказался под властью такой коррумпированной верхушки,
что для решения о санкциях им достаточно (как сказал знающий человек)
развёрнутого мониторинга моих высказываний с критикой Байдена, его сына,
посольства США и «агентов прозападных реформ». Ещё в июле, когда в аэропорту
Далласа со мной, приехавшим на международный христианский форум, три с лишним
часа «по-дружески» беседовали ФБРовцы, меня шокировал вопрос: «Что Вас
мотивировало публично требовать расследования деятельности младшего Байдена?»
Думал: эксцесс исполнителя. Оказалось: новая норма.

\ii{20_01_2022.fb.voloshin_oleg.opzzh.1.sankcii.cmt}
