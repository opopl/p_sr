% vim: keymap=russian-jcukenwin
%%beginhead 
 
%%file 05_06_2021.fb.nicoj_larisa.1.tualet_kiev_ekskursia
%%parent 05_06_2021
 
%%url https://www.facebook.com/nitsoi.larysa/posts/928323187734054
 
%%author Ницой, Лариса
%%author_id nicoj_larisa
%%author_url 
 
%%tags 
%%title Як гадаєте, що то за тумба? Правильно, туалет
 
%%endhead 
 
\subsection{Як гадаєте, що то за тумба? Правильно, туалет}
\label{sec:05_06_2021.fb.nicoj_larisa.1.tualet_kiev_ekskursia}
\Purl{https://www.facebook.com/nitsoi.larysa/posts/928323187734054}
\ifcmt
 author_begin
   author_id nicoj_larisa
 author_end
\fi

Як гадаєте, що то за тумба? Правильно, туалет )) Вона стоїть на нашому
екскурсійному маршруті. Не історична пам'ятка, але цікава. Хто не знає,
запрошую в Київ на екскурсії з поселенням і харчуванням.

Повернулися з чоловіком з Хмельницького до Києва пішли ще раз перевірити
екскурсійну дорогу для груп дітей.  

Ідемо по маршруту, звертаємо увагу на різні дрібнички, на які раніше не
звертали. Хоп, перед нами тумба. І одразу думка. Про обіди продумала, про
\enquote{привали} щоб посидіти, на траві полежати - подумала. Де морозивко/водичку по
дорозі купити - подумала. А про туалети забула. Де в нас ще є туалети на шляху?
Ага, отам і отам. Дякую тобі, тумбо )))

Ви вже надумали їхати до мене на екскурсії? У нас робота кипить, щоб вас зустрічати. 0680028890

\ifcmt
  pic https://scontent-cdg2-1.xx.fbcdn.net/v/t1.6435-9/197040476_928323164400723_2463996821002480912_n.jpg?_nc_cat=111&ccb=1-3&_nc_sid=8bfeb9&_nc_ohc=wjTtE9nJzM8AX-VlvV6&tn=ntrKbsW_7ChXu3v-&_nc_ht=scontent-cdg2-1.xx&oh=e92ea73f37f4695859f4ec987a78d558&oe=60E2DBBF
\fi
