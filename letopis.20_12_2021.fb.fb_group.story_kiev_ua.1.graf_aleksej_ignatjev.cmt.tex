% vim: keymap=russian-jcukenwin
%%beginhead 
 
%%file 20_12_2021.fb.fb_group.story_kiev_ua.1.graf_aleksej_ignatjev.cmt
%%parent 20_12_2021.fb.fb_group.story_kiev_ua.1.graf_aleksej_ignatjev
 
%%url 
 
%%author_id 
%%date 
 
%%tags 
%%title 
 
%%endhead 
\zzSecCmt

\begin{itemize} % {
\iusr{Елена Фросталь}
Давным давно читала « 50 лет в строю», тогда было интересно, а сейчас мне почему-то жаль графа

\iusr{Mykhaylo Losytskyy}

Білі офіцери, які були смертельно ображені на графа за перехід на радянський
бік (разом з грошима, що лежали на його ім'я), називали його книжку \enquote{50 лет в
строю, ни одного дня в бою}. А книжка цікава, так.

\begin{itemize} % {
\iusr{Кретов Андрей}
\textbf{Mykhaylo Losytskyy} 

Якби не так звані \enquote{военспецы} з царських офіцерів, невідомо, чи вдалося
б більшовикам перемогти. Серед білих офіцерів було набагато більше, але в
політиці вони були дилетантами. І в національному питанні особливо.

\end{itemize} % }

\iusr{Victoria Novikov}
Спасибо . Очень интересно.

\iusr{Vadym Makhnytskyy}
Красивая сказка о воре, казнокраде и иуде графе Игнатьеве  @igg{fbicon.face.tears.of.joy} 

\begin{itemize} % {
\iusr{Галина Полякова}
\textbf{Vadym Makhnytskyy} 

Зачем же так? Он ничего не украл. Денежки передал стране, которую считал своей
родиной. Просто полагал, что она попала в беду. Он присягал народу. Ему и
служил так, как умел.

\begin{itemize} % {
\iusr{Олег Курилов}
\textbf{Галина Полякова} он присягал в первую очередь царю  @igg{fbicon.wink} 

\iusr{Станислав Кшановський}
\textbf{Олег Курилов} царь отрёкся от престола без помощи Игнатьева

\iusr{лариса погосова}
\textbf{Станислав Кшановський}
Ленину зачем российские деньги отдал?
Они что, народу пошли или на благие дела?
Пайки получали пока народ сдыхал от голода.
Не его эти деньги были и не он должен был ими распоряжаться.

\iusr{Анна Муратова}
\textbf{лариса погосова} А кому их надо было отдать? Если уж так сложилось, что на его имя записаны, то только он и мог распорядиться. Великодушный и достойнейший порыв - отдать голодающей родине.

\iusr{лариса погосова}
\textbf{Анна Муратова}
Голодающим отдал?
Если бы...
Это не его деньги были!
Не его личные средства!
Не имел права единолично распоряжаться!!!

\iusr{Анна Муратова}
\textbf{Галина Полякова} повторюсь: а кому по вашему мнению их надо было оставить? Франции?

\iusr{Станислав Кшановський}
\textbf{лариса погосова} должен был раздать бедным царским чиновникам и беглым кааиталистам?

\iusr{Станислав Кшановський}
или пропить и прогулять?

\iusr{Галина Полякова}
\textbf{Анна Муратова} 

Вопрос не ко мне. Я не знаю, что там происходило на самом деле. Князья Юсупові
все, что смогли вівезти из России, потратили на бедствующих эмигрантов.
Игнатьев, вероятно, мыслил иначе. Считал, что эти деньги принадлежали его
родине. Безусловно, высшее дворянство было в значительной степени от реальной
жизни оторвано, они были невероятными идеалистами и романтиками. Имеем ли мы
право их в этом упрекать? Себе-то он этих денег не взял.


\iusr{Станислав Кшановський}
\textbf{лариса погосова} он за них отвечал и подписывал документы.

\iusr{Олег Курилов}
\textbf{Станислав Кшановський} И? Причём здесь это? Ась? Он давал присягу народу? Большевикам? Нет! Вопрос был про присягу! Исключительно про присягу!

\iusr{Олег Курилов}
\textbf{лариса погосова} Как вы видите, что бы он отдал именно голодающим? Тем более в 1925 году, как написал автор, когда он их отдавал, никакого голода уже не было. Хотя.. можно было как то купить зерно и поставить зерно. Но это очень, очень геморно. Но на это тоже нужно идти на полное сотрудничество с Большевиками, иначе это зерно пойдёт не туда.

\iusr{Станислав Кшановський}
\textbf{Олег Курилов} присягал царю, царь отрёкся, присягал Временному правительству - оно профукало власть. Страна не исчезла.

\iusr{Олег Курилов}
\textbf{Станислав Кшановський} Поэтому новому правительству он ничего не должен! И сделал на своё усмотрение как считал нужным.
\end{itemize} % }

\iusr{лариса погосова}
Он должен был оставить их в банке до лучших времён.
Хотя какие там лучшие времена в России?
Воры, казнокрады и пьяницы.!
Как говаривал Салтыков-Щедрин: Разбудите меня в России через 100 лет..,

\end{itemize} % }

\iusr{Валерий Елисеев}

Обязательно ещё нужно добавить, что от этого персонажа даже его родной брат
Павел отказался, хотя они в тот период времени проживали в одном городе Париже,
не говоря уже о товариществе выпускников Пажеского корпуса и офицеров
Кавалергардского полка, которые сочли необходимым навсегда вычеркнуть его из
своих списков.

\begin{itemize} % {
\iusr{Галина Полякова}
\textbf{Валерий Елисеев} 

И это было. Революция и гражданская война, которая за ней последовала,
разрушили многие семьи, развели многих когда-то близких людей.

\iusr{Ludmila Vakulenko}
\textbf{Валерий Елисеев} 

что же ими владело и руководило? Зависть ? Ведь читаем и узнаем,- везде был
первым И за что бы не взялся все исполнял блестяще

Женщину в спутницы и ту, выбрал неординарную натуру !

Или другая была причина ?

\begin{itemize} % {
\iusr{Галина Полякова}
\textbf{Ludmila Vakulenko} 

Не совсем Вас понимаю. О какой зависти речь? Денег он не крал. Кормился тем,
что выращивал шампиньоны вдвоем с женой и продавал их на базаре \enquote{Чрево Парижа}.
Если бы он денежки крал, то какие там шампиньоны!!

\iusr{Ludmila Vakulenko}
\textbf{Галина Полякова} Стоп  @igg{fbicon.stop.sign} 
Я сказала что он денежки крал ?

Как по мне так в повествовании о нем наоборот, он идеализированный персонаж

А что до предположения о зависти, то это всего лишь предположение. Потому , что
один из коментов как раз говорил о том что от него все отвернулись Об этом
герое ранее и не знала

А вот что пишут в коментах читатели полюбопытствуйте Мнения очевидно разнятся с
Вашими

\iusr{Галина Полякова}
\textbf{Ludmila Vakulenko} 

\enquote{Все отвернулись}. Это кто? Кто-то отвернулся, а кто-то нет. Если взять
информацию из одного источника, то мнение будет неполным. Audiatur et altera
pars.

\iusr{Ludmila Vakulenko}
\textbf{Галина Полякова} 

\obeycr
не ведите со мной дискуссию на эту тему
Это мнение не мое
Мои только вопросы
И сводятся к тому ; имел ли человек достоинства, воспитание, образование?
Занимал ли посты, служил Отечеству, семье, друзьям ? Достойным ли восхищения
был человек ?
Что говорит о нем история ?
Все это вопросы для семейных и музейных архивов
Мы же потомки ,- принимаем не всегда объективные мнения, а именно те, что нам навязывают
По сему вопросы- ответы точно не ко мне
А информация ?А информация доставлена нам людьми, которые сами не без пороков
И вот по этой причине не сужу о ЧЕЛОВЕКЕ никак
\restorecr

\iusr{Валерий Елисеев}
\textbf{Galina Poliakova} 

именно поэтому наверное и принято говорить, что de gustibus et coloribus non
est disputandum. Но если иметь ввиду сотрудничество бывшего графа Игнатьева с
большевиками, то тут, на мой взгяд, будет более уместно сказать о нем, что
errare humanum est. Но ведь цена этой роковой ошибки Игнатьева для участников
как Белого движения, так и последовавшей за тем эмиграции, была слишком высока.
Не на ту Россию сделал свою политическую ставку бывший граф - пусть и через
много лет, уже в 1991 году, но Красный режим на территории бывшей Российской
империи все же пал.

\end{itemize} % }

\iusr{Ludmila Vakulenko}
\textbf{Валерий Елисеев} почему?! Или за что?

\begin{itemize} % {
\iusr{Валерий Елисеев}
\textbf{Ludmila Vakulenko} 

потому, что по их мнению он предал их Россию, свою семью, своих однокашников по
корпусу, своих однополчан. В конце-концов за то, что благодаря своим подлым
действиям по отношению к Белому движению он получил от большевиков
«индульгенцию» и право возвратиться и умереть на свою Родину в то самое время,
когда многих тысяч представителей его социального класса красные вели на
эшафот.


\iusr{Ludmila Vakulenko}
Понятно
\end{itemize} % }

\iusr{Татьяна Аверкиева}
\textbf{Валерий Елисеев} смутные времена. Сейчас тоже рушатся семьи и дружеские связи. Так было всегда.
\end{itemize} % }

\iusr{лариса погосова}
Во многих городах Болгарии есть улица Графа Игнатьева.
Наверное другого?

\begin{itemize} % {
\iusr{Галина Полякова}
\textbf{лариса погосова} Его дядя, родной брат отца.

\iusr{лариса погосова}
\textbf{Галина Полякова}
А я удивилась.!!!
За что такая честь?
Улица Царя-освободителя в каждом городе понятна, генерала Гурко, контр-адмирала Крейга...

\iusr{лариса погосова}
\textbf{Галина Полякова}
Улица генерала Скобелева.
Но это все заслуженные люди.

\iusr{Lyubov Pakholchenko}
\textbf{лариса погосова} Других Игнатьевых не бывает.

\iusr{Alexander Motytskiy}

Граф Игнатьев до революции был очень влиятельным человеком, будучи дипломатом
лично приложил руку к важнейшим переговорам и подписанию документов по которым
Болгария получила независимость как государство, в винницкой области, в селе
круподеренци он похоронен в склепе под церковью которая точная копия только
меньше главного одного из главных храмов Болгарии, на церкви табличка от посла
Болгарии, от благодарного болгарского народа, там же единственный в мире памятник
погибшим при сражении в цусиме российским морякам, он в Болгарии национальный
герой)

\begin{itemize} % {
\iusr{лариса погосова}
\textbf{Alexander Motytskiy}
Это Николай Павлович Игнатьев.
А к большевикам перешёл его племянник.

\iusr{Галина Полякова}
\textbf{лариса погосова} Я не назвала бы его большевиком.
\end{itemize} % }

\iusr{Alexander Motytskiy}

Его называли Красный Граф, ходила непроверенная информация что он передал
советам солидную часть золотых миллионов, и Сталин приказал его не трогать, но
это слухи, возможно родственники по всему миру в то время, что бы не было
международного резонанса, имели влияние на международные отношения, в наше время
к примеру Майкл Игнатьев лидер либеральной партии Канады

\end{itemize} % }

\iusr{Vadym Makhnytskyy}

Вор, укравший на военных поставках и снабжении русского экспедиционного корпуса
во Франции как минимум 80 миллионов франков.

\begin{itemize} % {
\iusr{Галина Полякова}
\textbf{Vadym Makhnytskyy} Есть документы, которые бы подтвердили факт хищения?

\iusr{Vadym Makhnytskyy}
\textbf{Галина Полякова} 

Поищите в военно-исторической периодике, где давно и не раз публиковались и
документы специальной следственной комиссии 1918 года и донесения военных
агентов, и расшифровывались схемы каким образом на откатах и банальном
мошенничестве «красный граф» воровал казённые деньги на содержание своей
любовницы Трухановой. А затем, когда запахло жаренным, передал большевикам всю
агентуру русской разведки в Бельгии и Швейцарии, и эту информацию красные сразу
передали своим хозяевам в Германии погубив десятки истинных патриотов и
офицеров разведки.


\iusr{Галина Полякова}
\textbf{Vadym Makhnytskyy} Если Вы с таковой знакомы - поделитесь. Следственная комиссия 1918 года? Простите, но не звучит убедительно...
\end{itemize} % }

\iusr{Hardashnyk Irene}

А в Винницкой области в селе Круподеринцы сохранилось имение графов Игнатьевых
(таперь там небольшой музей), церковь- уменьшеная копия одного из главных
болгарских соборов.ю а на её территории - огромный морской якорь, как память о
погибших при Цусиме детях Игнатьевых. В подвальных помещениях церкви находится
семейный склеп Игнатьевых. И каждый год на храмовый праздник приезжают потомки
Игнатьевых из Болгарии.


\iusr{Петр Сазонов}

Могила отца б.графа Игнатьева на тихвинском кладбище Александро-Невской лавры в
СПБ. Сыну пришлось указать все свои звания на памятнике, чтоб уберечь его от
сноса.

\ifcmt
  ig https://scontent-frt3-1.xx.fbcdn.net/v/t39.30808-6/269625788_451220510051466_7295245239120760098_n.jpg?_nc_cat=108&ccb=1-5&_nc_sid=dbeb18&_nc_ohc=U1dpQsoIMYgAX-YmgoY&_nc_ht=scontent-frt3-1.xx&oh=00_AT9llpoE493q0OWvINSGiLIzIywJvjVWojoWetR_4quFNg&oe=61D1D476
  @width 0.3
\fi

\iusr{Галина Полякова}
\textbf{Петр Сазонов} Большое спасибо! Очень интересно!

\iusr{Мария Константиновская}
очень познавательные комментарии!!

\iusr{Татьяна Аверкиева}
Сравнивая с теперешними понимаешь, что такое отрицательный отбор.

\begin{itemize} % {
\iusr{Владимир Шипов}
\textbf{Татьяна Аверкиева} 

Как вы правы! В 17 году к власти пришли сатанисты, деграданты и садисты! Это
еще мягко про них сказано. И давно замечено-мерзость не может существовать
рядом с человеком чести, порядочности, образованности и ума! И к сожалению
\enquote{сорняки} победили \enquote{благородные культуры}...

\iusr{Yuriy Austin}
\textbf{Владимир Каледин} Так никого ведь и не осталось, чтобы к власти могли прийти приличные люди. Их всех истребили большевики ленинцы и развратили до глубин души коммунисты брежневцы.

\iusr{Yuri Ray}
\textbf{Владимир Шипов} и продолжают побеждать к всеобщему стыду
\end{itemize} % }

\iusr{Наталья Толстова}

В переломные годы многие службы ломались не только по границам государств, но и
по семейным отношениям. Сложно судить с позиций нынешнего времени. Но нужно
всё-таки отдавать должное поступкам людей, жившим по законам совести и чести.
Служил Родине и не его вина, что к власти пришли бандиты. Ещё десять лет тому
никому и в голову не могло прийти, что Россия нарушит все законы, забудет все
обещания и начнет братоубийственную войну. Так что снова время разделило многое
и многих.


\iusr{Константин Кострыкин}
\textbf{Наталья Толстова} голодранцы изгнали.

\iusr{Наталья Эгатова}

Хорошо написана историческая миниатюра о киевлянине, имя которого мне не было
известно! То, что он был человеком умным, образованным, неоднозначным - это
ясно! А судить его, мне кажется, мы н имеем права, так как много неизвестных,
неподтвержденных фактов. А в то мутное время вообще трудно было разобраться!

\iusr{Оксана Денисова}

Большое Вам спасибо за такой интересный и честный рассказ. Многое неоднозначно
в его судьбе, но ведь и время тогда было неоднозначное. И он, и его жена были
очень красивыми и очень талантливыми людьми. А судить их с позиции нашего
времени мы не имеем права. И еще раз спасибо Вам!

\begin{itemize} % {
\iusr{Татьяна Ткаченко}
\textbf{Оксана Денисова} Присоединяюсь к Вашему комментарию, благодарю и поддерживаю Вас, о таких людях хочется знать больше.
\end{itemize} % }

\iusr{Валентина Бодина}
Спасибо.

\iusr{Людмила Скомаровська}

\ifcmt
  ig@ name=scr.hands.applause
  @width 0.2
\fi

\iusr{Александр Мищенко}
Россия агрессор!

\iusr{Lyubov Pakholchenko}
Очень многим его не понять. Не дано.

\iusr{лариса погосова}

Безусловно имелся в виду граф Николай Павлович
Игнатьев., участвовавший в освобождении Болгарии от османского ига. Его не
забывают болгары и в каждом городе называют его именем улицы.
Похоронен в с.Круподеринцы.

\begin{itemize} % {
\iusr{Инна Валентиновна}
\textbf{лариса погосова} в селе этом было их имение.

\iusr{лариса погосова}
\textbf{Инна Валентиновна}
Да,я знаю
\end{itemize} % }

\iusr{Ludmila Vakulenko}
Превосходное повествование Благодарю !
Всегда восторгают подобные герои былых времен
Действительно оправдана чеканная фраза СЛУЖУ ОТЕЧЕСТВУ !- высказанная такими незаурядными людьми

\iusr{Andrey Smotritsky}
Среди военных библиографов его книгу называют: \enquote{50 лет в строю и ни одного дня в бою}

\begin{itemize} % {
\iusr{Галина Полякова}
\textbf{Andrey Smotritsky} Вполне возможно. Он не был боевым офицером. Он был дипломатом. И не претендовал на боевую славу.

\iusr{Станислав Кшановський}
\textbf{Andrey Smotritsky} Да ладно, а Манчжурия?! 1904-1905. Ордена боевые за русско- японскую .

\begin{itemize} % {
\iusr{Andrey Smotritsky}
\textbf{Станислав Кшановський} 

во все времена больше всех награждали именно \enquote{штабных}. Почитайте его
биографию. Он был помощником адъютанта. Не личным (как предостерегал его отец),
но всё же.


\iusr{Станислав Кшановський}
\textbf{Andrey Smotritsky} но в боевых действиях участвовал в т.ч. непосредственно в боях.

\iusr{Andrey Smotritsky}
\textbf{Станислав Кшановський} откуда Вам это известно?

\iusr{Станислав Кшановський}
\textbf{Andrey Smotritsky} Публиковались его наградные документы, лет 10 назад. Точно сейчас не скажу. Т.е. не он их автор.
\end{itemize} % }

\end{itemize} % }

\iusr{Ольга Хоботня}
Вот и генофонд нации..
А мы удивляемся сегодня невежеству населения...

\iusr{Надежда Дунаева}
Замечательные люди, сформированные в прекрасном Киеве!

\iusr{Андрей Кужеев}
\enquote{Да, были люди...}

\iusr{Vladimir Feldman}

Вижу подмену понятий. Произносится красивое слово «Родина», а деньги передаются
конкретному правительству со вполне конкретной репутацией.

Не, ну можно, конечно, искренне полагать то правительство законным
представителем воли народа... Но тогда, как говорится, поезд проследует со всеми
остановками...

\begin{itemize} % {
\iusr{Татьяна Аверкиева}
\textbf{Vladimir Feldman} 

очень многие бывшие тогда заблуждались по поводу тогдашнего правительства. Мы и
сейчас это видим. Наши соотечественники, живущие в Канаде, Америке и т. д. видят
нашу жизнь совсем по другому.


\iusr{Vladimir Feldman}
\textbf{Татьяна Аверкиева} 

Это давно уехавшие. А у кого синяк от сапога на заду ещё не зажил как следует,
те, как правило, воспринимают реальность вполне адекватно.

\iusr{Татьяна Аверкиева}
\textbf{Vladimir Feldman} у меня друзья уехали лет пятнадцать тому назад. Никакой объективности.

\begin{itemize} % {
\iusr{Vladimir Feldman}
\textbf{Татьяна Аверкиева} 

Ну, во-первых, наш герой начал попытки своей транзакции намного раньше, чем
через 15 лет. И во-вторых, я не уверен, что тогдашняя эмиграция подвергалась
активному информационно-пропагандистскому давлению со стороны Москвы. Во всяком
случае, не элитная верхушка и не в 21 году.

\end{itemize} % }

\iusr{Татьяна Аверкиева}
\textbf{Vladimir Feldman} в 29году многие возвращались по идейным соображениям.

\begin{itemize} % {
\iusr{Vladimir Feldman}
\textbf{Татьяна Аверкиева} 

1929? «Год великого перелома»? Не слыхал. Были массовые возвращения в 21, 23
годах, были амнистии в 21-24-м, но (порылся малость) именно о 29-м ничего не
вижу.

В любом случае, Игнатьев - «штучный товар», его история нетипичная, статистикой
не описывается.

\iusr{Татьяна Аверкиева}
\textbf{Vladimir Feldman} именно в Киев много вернулось в 29. Там были и мои родственники и как рассказывала бабушка много их друзей.
\end{itemize} % }

\iusr{Олег Курилов}
\textbf{Vladimir Feldman} 

очень спорно всё... Сложная и не однозначная тема... Ведь тогда было не
понятно, что из себя предстваляют новые власти. Очень много было идиалистов,
которые совсем не знали все тонкости и тезисы В.И. Ленина. Хотя... как бы там
не было, они уж получше предыдущих прогнивших старых были, в том числе и в
глазах современников. кстати, как бы и естественный отбор показывает силу
нового режима победой в гражданской войне.

\begin{itemize} % {
\iusr{Vladimir Feldman}
\textbf{Олег Курилов} 

Да, казалось, что может быть хуже царского режима?

А тут ещё этот тренд первой половины ХХ века - вера в то, что проблемы общества
можно решить, если убить правильно выбранных людей... Это много кому планку
дозволенного снизило.

\iusr{Олег Курилов}
\textbf{Vladimir Feldman} 

ну этот тренд начался ещё в последней четверти 19в.  @igg{fbicon.smile} .
Просто гражданская война, одна из моих любимых тем в 20 веке, и когда я в
последнее время углубился в то, что происходило в тылу у белых.... ну шансов у
них не было, если бы даже у них и войск было бы больше и всё удачно сладывалось
бы для них.... всё равно шансов не было с тем бардаком и огромными хищениями и
откоровенной бездарностью управления и воровством. , ну и главное. Они ничего
не смогли предложить селянам. А это главная сила в то время. Врангель очень
поздно это понял, когда ужё всё заканчивалось.

\iusr{Vladimir Feldman}
\textbf{Олег Курилов} 

Очень обнадёживающие наблюдения... вот придёт дядя с горячей кочергой, пошурует
где надо, и наступит благорастворение воздухов... А без дяди так и сидели бы на
деревьях. Что ж, вполне возможно, что и граф так размышлял...

\end{itemize} % }

\iusr{Станислав Кшановський}
\textbf{Vladimir Feldman} 

Какому правительству он должен был передать денежные средства? Деникинскому,
врангелевскому, колчаковскому, меркуловскому. семеновскому, Юденича или
Керенскому? Или поделить между ними по потребностям?

\end{itemize} % }

\end{itemize} % }
