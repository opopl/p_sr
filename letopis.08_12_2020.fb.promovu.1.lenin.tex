% vim: keymap=russian-jcukenwin
%%beginhead 
 
%%file 08_11_2020.fb.promovu.1.lenin
%%parent 08_12_2020
 
%%url https://www.facebook.com/groups/promovugroup/permalink/876805266226669/
 
%%author 
%%author_id 
%%author_url 
 
%%tags 
%%title 8 грудня 2013 року в Києві повалили пам’ятник Леніну
 
%%endhead 
 
\subsection{8 грудня 2013 року в Києві повалили пам’ятник Леніну}
\label{sec:08_12_2020.fb.promovu.1.lenin}
\Purl{https://www.facebook.com/groups/promovugroup/permalink/876805266226669/}

За ним, наче фішки доміно, попадали тисячі інших пам’ятників \enquote{вождю}. Ну як
\enquote{попадали} — їм наполегливо допомогли впасти.  Завдяки потужній енергії, яку
дала Революція Гідності, українці нарешті розпочали масовий процес ліквідації
совєцьких комуністичних ідолів зі своїх вулиць, дворів та будівель.

Спочатку все відбувалося стихійно, зусиллями небайдужих людей. Пізніше, у 2015
році, Верховна Рада ухвалила Закон \enquote{Про засудження комуністичного та
націонал-соціалістичного (нацистського) тоталітарних режимів в Україні та
заборону пропаганди їхньої символіки}. \url{https://bit.ly/3glM14u} Це допомогло
системно розпочати процес визволення від пропаганди.

Слідом за Леніним, звідусіль поприбирали й інших \enquote{вождів} і об’єкти совєцької
спадщини. А ще — перейменували історичні назви міст, сіл та вулиць. Вдалося
відновити багато назв, які були стерті совєцьким режимом. Одним словом,
розпочалася декомунізація — процес, який був потрібен Україні вже дуже давно.

На жаль, українцям не вдалося розпочати декомунізацію одразу після відновлення
незалежності. Як, наприклад, громадянам Литви, де швидко позбулися залишків
тоталітарного минулого і почали будувати вільну державу. \url{https://bit.ly/3fLOQe1}  

В Україні десятиріччями зберігали монументи диктаторів чи назви вулиць з їхніми
іменами. І це одна з причин, чому багато українців досі відчувають
ностальгічний зв’язок з СССР чи просто прагнуть зберігати якусь повагу до цього
режиму і його діячів. 

В Україні досі немає злагодженої підтримки необхідності декомунізації. Згідно з
останнім опитуванням, лише 32\% українців підтримують заборону комуністичної
символіки: \url{https://bit.ly/2WAOW0m}

Втім, вже сьогодні на вулицях просто неможливо уявити жодного бетонного
совєцького бовдура. Пройшло лише 7 років, але таке враження, що всі ці \enquote{вожді}
— не більше ніж відгомони далекого і несправжнього минулого. За цей час
народились та виросли мільйони українців, які ніколи навіть не згадають
пам’ятник Леніну у своєму парку чи біля школи. Незворотні зміни відбулися.

Все більше українців знають свою правдиву історію про режим, який тероризував
їхню державу, їхні сім’ї та друзів. Геноцид Голодомором, воєнні злочини,
нескінченні репресії, депортації, тортури й знущання в ГУЛАГу — весь
комуністичний режим був суцільним злочином проти людства. Неможливо порахувати,
скільки людей попрощалися з життям в час совєцької окупації України. 70 років
червоний терор перетворював вільних людей на рабів. Переслідував, катував і
вбивав усіх, хто намагався хоч якось виявляти непокору чи намагатися зберегти й
врятувати українську самобутність та шанс колись розпрощатися з тоталітарним
режимом.

\ifcmt
pic https://scontent.fiev6-1.fna.fbcdn.net/v/t1.0-9/130822806_2828003720817510_8384384935621897204_n.png?_nc_cat=102&ccb=2&_nc_sid=730e14&_nc_ohc=IEuLNxRuoy4AX-w0rM6&_nc_ht=scontent.fiev6-1.fna&oh=e57370ea12ae5f2eb12c21f73aec7946&oe=5FF420A5
\fi

Правильне розуміння історія формує наше ставлення до сьогодення. Тому пам’ятати
й відновлювати історичну справедливість — завжди на часі. І кому це робити, як
не українцям — нації, котра зазнала  стільки болю й втрат через совєцьку
окупацію. Тож не зупиняємося у тому, що почали 7 років тому — борімося за
гідність, виганяємо нечисть та мочимо мракобісся!
