% vim: keymap=russian-jcukenwin
%%beginhead 
 
%%file 10_12_2021.fb.fb_group.story_kiev_ua.2.samyj_samyj_podol.cmt
%%parent 10_12_2021.fb.fb_group.story_kiev_ua.2.samyj_samyj_podol
 
%%url 
 
%%author_id 
%%date 
 
%%tags 
%%title 
 
%%endhead 

\zzSecCmt

\begin{itemize} % {
\iusr{Tatiana V Nskaia}
 @igg{fbicon.heart.growing}  я ходила в 100 школу в шахматную секцию...

\iusr{Таня Солодкая}
всегда любила Подол, и даже сейчас, с Оболони возим детей в 100 лицей

\iusr{Sophie Tiupyushev}
Cпасибо!

\iusr{Павел Мар}
Хаотичность впечатляет! Суперское фото!

\iusr{Ella Zalavsky}

Спасибо за экскурсию в прошлое, я тоже родилась на Подоле, улица Мирная. Скучаю
по родному городу и очень благодарно за возможность пройтись по знакомым местам

\iusr{Alina Monroz}
Живу на Подолі і на фото побачила свій будинок...

\iusr{Iryna Tymoshok}

Спасибо, получилось очень душевно... и моя улица Юрковская, дальше набережная,
где прошли мои детство и юность, очень переплетаются с Вашими воспоминаниями. И
гимназия 19, которую закончила наша доченька, и Ильинская церковь, где они с
мужем венчались... теплом накрыло...

\iusr{Ира Цирульницкая}
Я тоже кореная подолянка, но все очень изменилось, узнать тяжело, сердце болить
все это видеть где прошла молодость.

\iusr{Елена Мирошниченко}
Спасибо!

\iusr{Виктор Задворнов}
Чудесная экскурсия, благодарю!

\iusr{Татьяна Иванова}
Спасибо!

\iusr{Виктория Луцкая}

Я с вами, я с Боричева тока 28, наш дом ( коммуналки) расселили - ощущения те
же, а в 100 школе моя сестра отучилассь

\begin{itemize} % {
\iusr{Vitali Andrievski}
\textbf{Виктория Луцкая} 

я тоже родился и жил в этом доме. Но почему то вас не помню. Напомните
пожалуйста. Или мы с вами жили там в разные времена.

\begin{itemize} % {
\iusr{Виктория Луцкая}
\textbf{Vitali Andrievski} 

моя бабушка Куканова, квартира 32, я у нее была до 4лет, а потом наездами, все
выходные, а мои двоюродные там жили всю дорогу Игорь Кордин и Оксана, помните
их? Это Оксана отучилась в 100 школе, закончила в 91

\end{itemize} % }

\iusr{Vitali Andrievski}

Сапсибо. Вспомнил. Наша квартира была 29 а над нами жила семья отца Глаголева.
Его дочка Мария с мужем Володей и детьми.

\begin{itemize} % {
\iusr{Виктория Луцкая}
\textbf{Vitali Andrievski} 

это вы на 1этаже жили? Потому что священник со второго мою сестру на дому
крестил, тогда нельзя было, папа коммунист, а мама с бабушкой покрестили 65год

\iusr{Виктория Луцкая}
\textbf{Vitali Andrievski} А я с первого этажа никого тоже не помню, вам сколько лет?

\iusr{Vitali Andrievski}
\textbf{Виктория Луцкая} уже 67. Вы конечно моложе. Может быть я видел вас еще совсем ребенком
\end{itemize} % }

\end{itemize} % }

\iusr{Елена Чурюмова}

Спасибо, за теплые слова о Подоле. Люблю приезжать сюда ранним утром и бродить
среди домиков, сохранивших воспоминания, деревьев осталось очень мало. Утреннее
солнце ярко заливает улицы, брызжет и сбивает с ног.

\iusr{марина зак}
Душевно. Спасибо

\iusr{Маргарита Дуда}

Навещая Подол, я как будто в гостях у своїй бабушки. Тепло, пахне детством.
Безусловно, 'модные' тенденции портят сей патриархальный уклад... но флюиды
Подола, это нечто.

\iusr{Зинченко Виктор}
Боже, какой слог - замечательно! Спасибо

\iusr{Света Буренок}
В 100школе учился мой муж Подгурский Сергей а жил он на Дехтярной 13

\iusr{Лідія Ковмір}

Дякую! Наш дім там де наша душа.. мені пощастило залишитись жити там де моя
душа. А ваш допис зачепив до сліз...

\iusr{Ольга Яворовская}

Спасибо!!! Вспомнила. какие чудесные новогодние елки были в \enquote{Пищевике}
во время каникул.

\iusr{Елена Харченко}
Да, не одни зимние каникулы не проходили без посещения \enquote{Елки} в \enquote{Пищевике}.

\iusr{София Бойко}
Спасибо. Родилась на ПОДОЛЕ. Но судьба распорядилась и \enquote{переселила на берег
ЛЕВЫЙ}. Душа осталась на правом берегу @igg{fbicon.heart.growing}  @igg{fbicon.hearts.revolving}  @igg{fbicon.heart.beating} 

\ifcmt
  ig https://scontent-lhr8-1.xx.fbcdn.net/v/t39.1997-6/s168x128/240595295_280280763862242_7752062239402781192_n.png?_nc_cat=1&ccb=1-5&_nc_sid=ac3552&_nc_ohc=kLYYHfRHdI0AX_VjWLe&tn=lCYVFeHcTIAFcAzi&_nc_ht=scontent-lhr8-1.xx&oh=00_AT8Zg5cd6hGW97_8CT9shG_2Ccz64wN9abgkN0sLr8Vl8A&oe=61F1861D
  @width 0.1
\fi

\iusr{Татьяна Гордиенко}
Как я вас понимаю! Подол в моем сердце навсегда.

\iusr{Stella Stavitsky}
Спасибо за экскурс, увы, ничего уже не узнаю .....

\ifcmt
  ig https://scontent-lhr8-1.xx.fbcdn.net/v/t39.1997-6/s168x128/197820122_1581651598693950_778064872271246176_n.png?_nc_cat=1&ccb=1-5&_nc_sid=ac3552&_nc_ohc=rMWx7qMwrsIAX8Nk7WF&_nc_ht=scontent-lhr8-1.xx&oh=00_AT_UtYmxyRz7rAxOIZOzmrxINtjxRJHZ4rTxhMUkVzO-Ow&oe=61F13969
  @width 0.1
\fi

\iusr{Оксана Билык}
Чудесный кадр!!

\iusr{Natalie Lordkipanidze}
 @igg{fbicon.hearts.revolving}  @igg{fbicon.flame}  @igg{fbicon.hearts.revolving} 

\iusr{Владимир Картавенко}
Место Крещения Руси!

\iusr{Нина Кубанова}
А я буду покороче.
Помнится, взобравшись на Цыганку (кажется, она тогда была выше), мы ЭТО всё
видели. И пытались узнать в зелени деревьев свой дом.

\iusr{Нина Бондаренко}
Чудовий нарис і прекрасна світлина. Трошки сумні, але дорогі спогади.

\iusr{Ирина Громовик}
Дякую за допис.

\iusr{Егор Красава Тебякин Егор}
Спасибо!

\iusr{Галя Александрова}
20 и 9 школ, 17 возле Жовтня

\iusr{Петр Кузьменко}
\textbf{Галя Александрова}, я уже признал ошибку. Перепутал. Я писал и о Константиновской.

\iusr{Lena Zabelina}
\textbf{Галя Александрова} молодееец @igg{fbicon.face.tears.of.joy} 

\iusr{Калерия Ходос}

Жили на Подоле на ул, Мирной. К сожалению на месте нашего дома стоит монстр - дом
который построила Билозир просто ужас. К сожалению очень много снесли хороших
домов хоть и старинных. А ведь Подол это сердце Киева и хотелось бы чтоб его
сохранили, а не строили монстрв, жаль...

\iusr{Людмила Бабич}
Благодарю! Мило! Трогательно и очень тепло... это о нашем родном Киеве, хотя я
родом не с Подола, а с ул. Чкалова( с Сенного...!)

\iusr{Анна Загорулько}

Чудный вид на Подол! Контрактовал площадь теперь угадывается по колесу
обозрения. А вот что за красное пятно на переднем плане, понять не могу, это
крыша такого цвета?

\iusr{Алла Власова}
Все мы родом из детства и родные места самые лучшие. А Подол особенный район Киева.

\iusr{Yurii Kadochnikov}
За клубом КВВМПУ видніється церквушка. Там дочку хрестили

\iusr{Татьяна Гурьева}
Спасибо, увидела все любимое Вашими глазами подолянина. В груди защемило...

\iusr{Galina Golubitskaya}

\ifcmt
  ig https://scontent-lhr8-1.xx.fbcdn.net/v/t39.1997-6/s168x128/46949788_1074619346032831_6110986696902311936_n.png?_nc_cat=1&ccb=1-5&_nc_sid=ac3552&_nc_ohc=7FTt2PqfqqcAX9F-k29&_nc_ht=scontent-lhr8-1.xx&oh=00_AT_y6k1cJs-NM2fEvXaSu1Z9RR-bUziAb4gKlyqJdW9tGw&oe=61F169B7
  @width 0.1
\fi

\iusr{Татьяна Домаскина}

Спасибо! Люблю гулять на Подоле, рассматривать старинные дома! Сама родилась на
Зоологической и местом наших игр был зоопарк. @igg{fbicon.face.smiling.eyes.smiling}  Лазили через дырку в заборе!
Счастливое детство!!!


\iusr{Vitali Andrievski}
Петр спасибо. С такой любовью, точностью и легкой ностальгией описан почти каждый дом нашего Подола.

\iusr{Natalija Grouzman}

Вам всё удалось! Прекрасное фото и очень интересное, трогательное до слез
описание Подола. Очень Вам признательна за рассказ.


\iusr{Ирина Касьянова}
 @igg{fbicon.hands.applause.yellow}  Прекрасная ремарка получилась @igg{fbicon.face.smiling.eyes.smiling} 

\iusr{Татьяна Сирота}

Хорошо написано.
Тепло,с любовью!
Спасибо,Петр,за прекрасный пост!
И фото замечательное,его автор большой молодец!

\iusr{Людмила Пятыгина}

Как прекрасно написано! Как вроде мы в один класс в сотой школе ходили. Все как
моими глазами, спасибо! Очень тронуло! @igg{fbicon.heart.red}

\iusr{Светлана Сокур}
Огромное спасибо!!!

\iusr{Oksana Shostal}
Дякую @igg{fbicon.heart.suit}

\iusr{Александра Пихур}
СПАСИБО за фото, можно взять в свою ленту?

\iusr{Петр Кузьменко}
\textbf{Александра Пихур}, нужно! @igg{fbicon.wink} 

\iusr{Светлана Александренко}
Спасибо за прекрасный пост, за красоту и любовь к нашему Киеву!

\iusr{Инна Валентиновна}

Ну, что такое Киев без Подола? Как святой Владимир без креста.. Нас все меньше
, кто так трепетно и душевно ценит Подол и старый Киев. Носителям памяти
старого города, низкий поклон.

\iusr{Инна Шполянская-Кисиленко}

Очень хорошая панорама. Фото отличное. Да взгляд на наше детство и молодость.
Когда приезжаю в Киев всегда гуляю здесь. Ноги сами знают дорогу.  @igg{fbicon.100.percent}{repeat=4} 

\iusr{Людмила Старовойтенко}
Спасибо за рассказ и такие искренние воспоминания классссс

\iusr{Лилия Клименко}
красивый вид

\iusr{Ирина Медына}

Подол нельзя не любить...

\iusr{Михайло Присенко}

«..остался подолянином в душе...»! Так! Це на все життя і в мене, і в друзів
мого зростання! Дякую Вам!


\iusr{Светлана Никитаева}

Когда смотришь на это великолепие - душа радуется. Наш красавец Киев Ведь это
наше родное Спасибо огромное


\end{itemize} % }
