% vim: keymap=russian-jcukenwin
%%beginhead 
 
%%file 04_01_2022.fb.fb_group.story_kiev_ua.1.istoria_odnoj_kievljanki
%%parent 04_01_2022
 
%%url https://www.facebook.com/groups/story.kiev.ua/posts/1833287633534696
 
%%author_id fb_group.story_kiev_ua,kirkevich_viktor.kiev
%%date 
 
%%tags istoria,kiev,kievljane
%%title Захватывающая история одной киевлянки
 
%%endhead 
 
\subsection{Захватывающая история одной киевлянки}
\label{sec:04_01_2022.fb.fb_group.story_kiev_ua.1.istoria_odnoj_kievljanki}
 
\Purl{https://www.facebook.com/groups/story.kiev.ua/posts/1833287633534696}
\ifcmt
 author_begin
   author_id fb_group.story_kiev_ua,kirkevich_viktor.kiev
 author_end
\fi

Захватывающая история одной киевлянки.

Велик Киев не только своей древностью и красотой, но удивительными людьми,
которые в нем появились на свет. В парижском музее «Гемэ» в редкой брошюре о
Викторе Голубеве – фундаторе этого собрания, я прочитал о его супруге Натали
Кросс. В книге было сказано, что она родилась в Киеве в семье «руководителя
Киевского края». Эту неординарную личность ваял, влюблённый в неё Огюст Роден,
рисовал Валентин Серов. Она героиня пьесы «Сумасшедшая любовь Донателлы Кросс и
Габриэля Д'Аннунцио», десятилетиями шедшая на Бродвее. С ней связана одна из
самых ярких авантюр начала ХХ века - республика Фиуме. Стоит о ней рассказать.

\ii{04_01_2022.fb.fb_group.story_kiev_ua.1.istoria_odnoj_kievljanki.pic.1}

В 1919 году итальянский поэт и писатель Габриеле д’Аннунцио возглавляет отряд
из 3200 итальянских националистов и захватывает спорный город Фиуме (ныне
хорватский город Риека). Из города изгоняются американо-британо-французские
оккупационные силы. Д’Аннунцио провозглашает Фиуме независимым государством и
называет его «Республика Красоты». Пост министра культуры в государстве
соглашается занять знаменитый дирижер Артуро Тосканини.

\ii{04_01_2022.fb.fb_group.story_kiev_ua.1.istoria_odnoj_kievljanki.pic.2}

Эстетика абсурда и гротеска становятся ориентиром поведения для всех. Вводится
мода на нудизм среди солдат, хождение без одежды не возбраняется и не мешает
армейской дисциплине. Наркотики, в основном кокаин, общедоступны и дешевы.
Местные женщины симпатизируют освободителям. Каждый третий день в атмосфере
свободной любви проходит парад цветов. Его участники, в основном «легионеры» и
их возлюбленные, зачастую облачаются в одежду противоположного пола. 

Внешний мир объявляет блокаду свободного города. Прекращение поставок продуктов
вынуждает революционеров ввести карточную систему распределения продовольствия,
построив хозяйство вне системы денежного обращения. Налоги с населения решено
не взимать. Порядка восьмисот детей приходится отправить к родственникам в
сопредельные округа Италии, чтобы не подвергать их возможным тяготам длительной
блокады. 

Для решения проблемы снабжения Д'Аннунцио реквизирует местный морской флот и
принимает к себе на службу несколько десятков военных летчиков, которые
самовольно слетаются в Фиуме со всех концов Италии. Он организует эти силы в
мобильные пиратские отряды, которые захватывают проплывавшие мимо корабли или
обворовывают крупных землевладельцев тех мест Италии, до которых легко можно
было долететь. Флибустьерская рента позволяет революционному обществу безбедно
существовать.

Согласно Конституции, гражданам гарантируется: личная свобода; бесплатное
начальное образование; оплата труда, обеспечивающая достойную жизнь;
гражданские права в полном объеме вне зависимости от пола, расы и религиозной
принадлежности; прожиточный минимум для безработных. Конституционно
закрепляется своеобразная концепция прав собственности: отныне никто не может
претендовать на имущество, если оно не было приобретено непосредственно за счет
личных трудовых усилий. Д'Аннунцио выдвигает лозунг «труд без утомления».

Фундаментальным принципом организации государства объявляется музыка, поэтому
карнавальные шествия не прекращаются ни днем, ни ночью. Постулируется приоритет
возможно более полной свободы гражданина. Права любых меньшинств гарантируются
в полном объеме. Вводится абсолютный запрет на насилие. И действительно,
никаких репрессий не было. Новые порядки принимаются на ура. Именно Д’Аннунцио
вводит массовые шествия в черных рубашках, воинственные песни, древнеримское
приветствие поднятой рукой и эмоциональные диалоги толпы с вождем. 

Великие державы не перестают спорить по поводу статуса города. В конце концов,
решать этот вопрос оставляют непосредственно Италии с Югославией, усилиями
которых компромисс был достигнут. Фиуме должен был стать вольным городом под
итальянской протекцией. Итальянское правительство отправляет ультиматум войскам
поэта, потребовав от них покинуть город и вернуться в свои части. 

В ответ Д'Аннунцио объявляет Италии войну и несколько недель ведет тяжелые бои
с итальянской армией.

Республика, продержавшаяся почти шестнадцать месяцев, перестает существовать.

Все эти 16 месяцев возле Д'Аннунцио была его гражданская жена Донателла Кросс.
Я о ней неоднократно писал. Она венчалась во Владимирском соборе с Виктором
Голубевым. Да, да, именно тем, кто был заказчиком Покровской церкви в с.
Пархомовка и дружил с Н. К. Рерихом. В этом храме в запрестольном фундаторском
кресте высечены имена её и двух её сынов. Об этом храме у меня вышло две книги
и несколько десятков статей. О Натальи Кросс глава в моей книге \enquote{Киев и
\enquote{Кіевлянинъ}}

\ii{04_01_2022.fb.fb_group.story_kiev_ua.1.istoria_odnoj_kievljanki.cmt}
