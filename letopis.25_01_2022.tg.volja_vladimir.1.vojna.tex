% vim: keymap=russian-jcukenwin
%%beginhead 
 
%%file 25_01_2022.tg.volja_vladimir.1.vojna
%%parent 25_01_2022
 
%%url https://t.me/Volodymyr_Volya/886
 
%%author_id volja_vladimir
%%date 
 
%%tags rossia,ugroza,ukraina,vojna,vtorzhenie
%%title Якщо сьогодні війна... Які уроки? Кому і навіщо потрібна Україна?
 
%%endhead 
 
\subsection{Якщо сьогодні війна... Які уроки? Кому і навіщо потрібна Україна?}
\label{sec:25_01_2022.tg.volja_vladimir.1.vojna}
 
\Purl{https://t.me/Volodymyr_Volya/886}
\ifcmt
 author_begin
   author_id volja_vladimir
 author_end
\fi

Якщо сьогодні війна... Які уроки? Кому і навіщо потрібна Україна?

Бо настала широко анонсована дата повномасштабного нападу на Україну. І що?

1) Найбільш серйозно загрозу нападу Росії на Україну сприйняли США, Велика
Британія та Німеччина. Настільки серйозно, що навіть дозволили представникам
своїх установ повернутися до свої країн.  Наші чиновники раптом злякалися
такого повороту в подіях, бо почала розкручуватися паніка. І раптом стали
казати, що все нормально, нападу не буде. Але ж тепер виглядає так, що всі
розумні навколо, а ось уряди США, Великої Британії та Німеччини виявилися
полохливими боягузами. Розказувати зараз усім, що вони на так все зрозуміли, це
наче казати: «Не лякайтесь, ми пожартували з нашою розвідувальною інформацією».
Нагадую, що роздмухувати істерію в західних державах допомагали не лише
представники чинної влади, але й різноманітні рятувальники з опозиції. З їхніми
експертами.  

2) В той час, коли західні держави готуються до евакуації дипломатичних
працівників, неспішно, неквапно збирається Верховна Рада.  Нардепи були на
найдовших за 10 років зимових «канікулах».  І ніхто так і не наполіг, щоб через
широко анонсовану загрозу Рада припинила свій відпочинок хоч на тиждень раніше.
Хтось з вас бачив хоч одну вуличну масштабну акцію з вимогою скликати Раду
достроково? Я не бачив. Була акція зустріч та підтримки Порошенка. Була
контракція. А про дострокове засідання у зв’язку із загрозою нападу –
нічого!!!!!  Інші політичні лідери теж не організовували акцій з вимогою
дострокового скликання Ради.  Красномовно, правда?

3) Вражаюча подія 25 січня: по графіку – вже війна, а міністр Резніков
повідомляє, що міноборони неспішно готує і збирається обговорювати пакет
законопроєктів щодо невідкладних потреб ЗСУ та можливого додаткового
фінансування армії.  «Ми провели зустріч з ініціативи голови Верховної Ради
Руслана Стефанчука з головами фракцій та груп, де ми обговорювали якраз
невідкладні потреби, оперативні потреби, стратегічні потреби Збройних сил,
Міністерства оборони... Ми зараз готуємо пакет можливих невідкладних потреб, і
обговорюватимемо спочатку з прем'єр-міністром, міністром фінансів».

Це дурні США, Велика Британія та Німеччина смикаються, переважна більшість
громадян України нервує, а наші депутати, чиновники, лідери опозиції поводяться
якось дивно... Кожен займається своїми справами і відпочинками.   

Після цієї ситуації на Заході просто менше довірятимуть «розвідданим» та
усіляким данним і словам від української влади і опозиції.
