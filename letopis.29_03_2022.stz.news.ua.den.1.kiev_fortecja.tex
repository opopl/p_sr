% vim: keymap=russian-jcukenwin
%%beginhead 
 
%%file 29_03_2022.stz.news.ua.den.1.kiev_fortecja
%%parent 29_03_2022
 
%%url https://day.kyiv.ua/uk/photo/kyyiv-velyka-y-micna-fortecya
 
%%author_id kapsamun_іvan,news.ua.den
%%date 
 
%%tags 
%%title «Київ – велика й міцна фортеця»
 
%%endhead 
 
\subsection{«Київ – велика й міцна фортеця»}
\label{sec:29_03_2022.stz.news.ua.den.1.kiev_fortecja}
 
\Purl{https://day.kyiv.ua/uk/photo/kyyiv-velyka-y-micna-fortecya}
\ifcmt
 author_begin
   author_id kapsamun_іvan,news.ua.den
 author_end
\fi

\begin{zznagolos}
«День» побував на одному зі столичних блокпостів 112-ї бригади Територіальної
оборони Збройних сил України	
\end{zznagolos}

У цьому районі Києва я бував дуже рідко, але так трапилося, що якраз незадовго
до початку війни довелося пішки пройтися його вулицями. А зараз, після місяця
повномасштабної агресії московії, після досить тривалої оборони Києва, у складі
групи журналістів, фотографів та операторів я потрапив на ту саму вулицю.
Впізнати можна, але вигляд вона має вже зовсім інший. Це не просто бетонні
плити на дорозі та кілька військових з автоматами, яких дуже багато по всьому
місту, це грамотно організований і структурований блокпост, який займає велику
територію з відповідним ешелонованим захистом і перешкодами: бетонні плити,
їжаки для колісної техніки, рови, укріплення, вогневі точки для особового
складу та військової техніки, заміновані ділянки, захисні споруди для перших
поверхів житлових будинків, звісно, військові з різноманітною зброєю. Навіть
під час нашого перебування на місці, відповідна техніка продовжувала працювати
– укріплювати об'єкт.

\ii{29_03_2022.stz.news.ua.den.1.kiev_fortecja.pic.1}

«Наш підрозділ взаємодіє з підрозділами Сухопутних військ ЗСУ й виконує
завдання як на лінії зіткнення з ворогом у передмісті, так і у самому місті, -
пояснює присутнім Олександр Павлій, полковник, командир 112-ї бригади
Територіальної оборони Збройних сил України. - Завдання у місті – захист,
охорона порядку, зразкова служба на блокпостах. Крім того, ми знаходимо час для
тренувань з тактичної медицини, готуємо особовий склад протитанкового резерву,
проводимо тактичний вишкіл. Як командира мене вражає ставлення й поведінка
самих киян у цей складний час. Усі ставляться з розумінням, прагнуть вступити
до лав ТрО, ЗСУ, добровольчі підрозділи, займаються волонтерством, намагаються
допомогти всім чим можуть, щоб захистити наше місто, нашу країну. Ми готові до
будь-якого розвитку подій».

\ii{29_03_2022.stz.news.ua.den.1.kiev_fortecja.pic.2}

Зважаючи на цей настрій та організацію оборони, переконуєшся, що ворога кияни
точно готові зустріти не з квітами, а як слід. Хоча весь час, поки ми
перебували на об'єкті, було чутно віддалений бій та вибухи на околицях столиці.
Там українська армія тримає оборону й не пускає агресора у саме місто. А на
певних напрямках ЗСУ проводить контратаки та поступово відкидає супротивника
подалі від столиці. Всі прекрасно розуміють, що найкращий варіант – не
допустити московітів у саме місто. Але бути готовими зустріти їх на вулицях
Києва, безумовно, потрібно. Це розуміють усі – і військові, і містяни.

\ii{29_03_2022.stz.news.ua.den.1.kiev_fortecja.pic.3}

«Київ нас надзвичайно порадував – охочих потрапити до лав ТрО було дуже багато,
і люди продовжують приходити до центрів комплектації, - розповідає журналістам
Андрій Ковальов, прессекретар 112-ї бригади Територіальної оборони Збройних сил
України. - Ми створили електронну базу всіх тих, хто бажає приєднатися до
тероборони, звідти потім мобілізуємо людей – з військовими знаннями,
військовими спеціальностями, з військовим досвідом. Звідти ми комплектуємо не
тільки нашу бригаду, а й інші підрозділи ЗСУ. Також у Києві та інших містах
існують добровольчі територіальні формування, які передбачені Законом «Про
основи національного спротиву». Вони входять до підпорядкування сил ТрО. Наша
112-та бригада вже сформована з профіцитом, усі люди озброєні та мають усе
необхідне».

\ii{29_03_2022.stz.news.ua.den.1.kiev_fortecja.pic.4}

Багато хлопців мають бойовий досвід, тому що брали участь у війні ще з 2014
року, - такі підрозділи вже на передовій, або там, де визначає їм виконувати
завдання командування. Інші, хто записався в ТрО впродовж останнього місяця,
проходять навчання та необхідну підготовку. «Територіальна оборона – це і є
Збройні сили, всі наші бійці підписали контракти: рядові – на три роки, офіцери
– на п'ять, - пояснює Андрій Ковальов. - Люди, які ще вчора були вчителями,
бізнесменами, айтівцями, автомайстрами, куховарками... 24 лютого покинули все та
стали до лав ЗСУ. Сьогодні це тисячі й тисячі високомотивованих і патріотично
налаштованих киян, які готові боронити своє місто. Підрозділами нашої бригади
вже знищено 25 одиниць російської техніки, 3 одиниці техніки було захоплено й
передано до лав ЗСУ. Єдине про що шкодують наші військовослужбовці, що москалів
може на всіх не вистачити».

\ii{29_03_2022.stz.news.ua.den.1.kiev_fortecja.pic.5}

Власне, зараз ворог робить усе, щоб їх вистачило. Перша хвиля атаки на різних
напрямках, зокрема київському, захлинулася, інтенсивність бойових дій уже не
та, але агресор усе одно намагається йти вперед, завдаючи шкоду нам і кладучи
тисячі своїх (бабы еще нарожают). Цю оперативну павзу, як її називають
військові, путін намагається використати для того, щоб наскребти ще «пушечного
мяса» й кинути його у пекло війни в Україні. Збирають де тільки можуть – у всіх
регіонах рфії, оголяючи свої фланги, з окупованих територій Абхазії, Південної
Осетії, Донбасу, з так званого миротворчого контингенту Карабаху, військових
баз Кавказу та Центральної Азії, з Сирії та Лівії, вербуючи найманців, з
Білорусі, намагаючись остаточно нагнути васала Лукашенка. Чи допоможе це
кремлю?

\ii{29_03_2022.stz.news.ua.den.1.kiev_fortecja.pic.6}

«Головна наша відмінність від ворога в тому, що він не розуміє навіщо прийшов
на українську землю, за що воює, за що має помирати, - каже Андрій Ковальов. -
Українські ж військові – чоловіки, жінки, мають величезну мотивацію захистити
свою землю та вигнати звідси ворога. Наші люди перебувають на десятках
блокпостів, на шляхах сполучення, на вулицях, у полях, у лісах. Столиця зараз
нагадує одну велику й міцну фортецю. Наші підрозділи з кожним днем окопуються,
будують захисні споруди. Київ готовий до оборони. І це російські окупанти вже
остаточно відчули. Ми розуміємо підступність ворога, але готові його зустріти
на тисячу відсотків. Навіть якщо є якась умовна спроба прорватися до Києва,
вони все одно будуть зупинені. Люди чергують на позиціях цілодобово. Ми не
можемо розкривати ворогу наших планів, але повірте, це не просто блокпости, за
цим усім стоїть дуже складна система оборони, тисячі людей. Довіряйте й вірте
Збройним силам України!».

\ii{29_03_2022.stz.news.ua.den.1.kiev_fortecja.pic.7}

У повітрі роздається гул сирени – тривога, яка оповіщає про повітряну небезпеку
й необхідність перейти в укриття. Військові кажуть нам, що перебувати на вулиці
небезпечно, треба покинути об'єкт. Перед від'їздом ще встигаємо перекинутися
кількома словами з перехожими, які вийшли за продуктами: «Ми віримо нашій
армії, ми бачимо як нас захищають, тому нікуди не збираємося їхати. Це наша
земля – хай ці покидьки забираються з України, рашисти грьобані. А наші Збройні
сили – найсильніші у світі!».  

Іван КАПСАМУН; фото Миколи ТИМЧЕНКА, «День»

\ii{29_03_2022.stz.news.ua.den.1.kiev_fortecja.pic.8}
