% vim: keymap=russian-jcukenwin
%%beginhead 
 
%%file 27_05_2021.fb.bilchenko_evgenia.1.vremja_pered_rassvetom
%%parent 27_05_2021
 
%%url https://www.facebook.com/yevzhik/posts/3930387486996332
 
%%author Бильченко, Евгения
%%author_id bilchenko_evgenia
%%author_url 
 
%%tags 
%%title БЖ. Время перед рассветом
 
%%endhead 
 
\subsection{БЖ. Время перед рассветом}
\label{sec:27_05_2021.fb.bilchenko_evgenia.1.vremja_pered_rassvetom}
\Purl{https://www.facebook.com/yevzhik/posts/3930387486996332}
\ifcmt
 author_begin
   author_id bilchenko_evgenia
 author_end
\fi

БЖ. Время перед рассветом
... А ночью они меня окружают - инфернальные, злые сущности:
Те самые, из глубины Хичкока, из ницшеанской бездны.
Я отгоняю их Волгоградами мысленными, на Сумщине
Мною поставленными, чьи звезды - кресты против всяких бесов.
Я - человек в погонах, да. От напористой моей ясности
Зарывалось в землю и не такое, колом до мезозоя.
Они бы меня убили, да. Но Зоя Космодемьянская 
В информвойне постмодерна - ноль, тем более, я - не Зоя.
Нынче в моде - не пытки с виселицами, а ирония да обструкция:
Она ломает без явной травмы, но чётко до позвоночника.
А впрочем, забвенье мне стали в кайф и травля: ведь что-то русское
Вырастает именно в ночь, когда не дожить до конца сей ноченьки.
Вот ведь, не думала, а предстала - такой, будто твердь, разреженной:
То ли камень, а то ли воздух. То ли воздух, а то ли камень.
Господи Иисусе Христе, помилуй мя многогрешную.
Я вижу утро. И крест. И свет. И звёзды кую руками.
Кузьмодемьяновское дитя, христианское и Сварожье:
Был кузнец такой на Руси, а, может, их было двое -
Кузьма и Демьян... И каждый из них - несомненное чадо Божье.
А в земле мезозойской - мой дед и всё. Всё кончено. Всё - живое.
27 мая 2021 г.
Авторское фото: Елена Николаева/АиФ

\ifcmt
  pic https://scontent-bos3-1.xx.fbcdn.net/v/t1.6435-9/192061740_3930387403663007_1681626291689811295_n.jpg?_nc_cat=104&ccb=1-3&_nc_sid=8bfeb9&_nc_ohc=1Sf9FQ8EfCUAX-kuEqN&_nc_ht=scontent-bos3-1.xx&oh=00726d2f32e615e5d3441773cda73055&oe=60D72535
\fi

