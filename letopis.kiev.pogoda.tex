% vim: keymap=russian-jcukenwin
%%beginhead 
 
%%file kiev.pogoda
%%parent kiev
 
%%url 
 
%%author 
%%author_id 
%%author_url 
 
%%tags 
%%title 
 
%%endhead 
\chapter{Погода}

\section{01-06-21}

Солнце ярко светит; но довольно прохладно, когда утром идешь за пачкой сигарет
и кофе; вместо долгожданного лета - по ощущениям весна; когда же станет жарко
уже?

Заходжу на сыноптик-точка-уа... - Схід 4:49, Захід 21:02, 12$^{\circ}$ по
Цельсию, максимальная температура была в 2011 году - +31.7$^{\circ}$, а
минимальная +4.1$^{\circ}$, по Цельсию - ого! Так холодно было, невероятно! Это
ж что получается, наши киевские девушки вместо нарядных воздушных платьиц
должны были укутываться в теплые осенние пальто - куда ж это годится? Это ж так
ничего и не рассмотреть толком, а ведь лето же!

И что ж это за года такие, 2011 и 1904? 

В 2011 году в России 1 января случилась авиакатастрофа Ту-154 в Сургуте;
печально тогда год начался...  Еще в 2011 году Нобелевскую Премию по медицине и
физиологии получили \emph{Брюс Бетлер, Ральф Стейнман та Жуль Гоффман за
роботи, пов'язані з імунною системою}, это украинская википедия мне сообщает...
А я... а я где ж тогда был? В консерватории учился, Мусоргского на пару с
Римским-Корсаковым осваивал...

Ну а в 1904 году меня не было еще даже в самых фантастических планах моих
прапрадедушек и прапрабабушек; что же тогда было?... о! 3 августа 1904 года
родился Клиффорд Саймак, американский писатель-фантаст. Он много всего
понаписывал, мне вот больше всего из его творчества нравится повесть о Гоблине
со Cтропил, и о путешествии к Древним... Там есть вечно голодный Сплетник, и
Зверь Хаоса, и ученый-монах, ищущий таинственный университет... Правда,
названия повести не помню, надо будет посмотреть...  Еще у него есть повесть о
Городе, правда, это не о Киеве... Саймак в Киеве вроде не бывал, а жаль! многое
ты, Клиффорд, упустил в жизни, если ни разу не был в Киеве, пойми...

А насчет погоды, так сыноптики еще (Сыны Оптики?) пишут такое: \emph{У Києві
протягом усього дня небо буде вкрите хмарами.  Без опадів. Народний прогноз
погоди: У перший день літа вшановується пам'ять благословенного князя Івана
Углицького і великого князя Дмитра Донського. У давні часи в цей день ще бували
заморозки на ґрунті.  Наші пращури помітили, якщо перші два дні червня йде дощ
- весь місяць сухим, а літо теплим буде. А якщо літо розпочинається з холодів,
то наступні 40 днів будуть холодними.  Спостерігали також за пташиним царством:
якщо ворони сідають дзьобами в одну сторону - до сильного вітру.}

Хм... князья Иван Углицкий и Дмитрий Донской... Надо бы почитать про них. Еще
бы посмотреть за вороньем - куда клювы будут свои вострить? - будет ли ветер и
нужно ли брать ветровку?  А пока что собираю рюкзачок, кушаю овсянку, наливаю
себе кофей, закрываю квартиру на ключ (да, да!  та самая квартира, где деньги
лежат! - пару сотен гривен, не больше; - тарифы ж у нас европейские, а замки за
высокими пятиметровыми заборами в Конча-Заспе покруче будут Версаля и
Тадж-Махала ) - и выхожу навстречу Солнышку, навстречу Городу... 
