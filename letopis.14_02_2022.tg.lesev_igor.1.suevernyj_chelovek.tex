% vim: keymap=russian-jcukenwin
%%beginhead 
 
%%file 14_02_2022.tg.lesev_igor.1.suevernyj_chelovek
%%parent 14_02_2022
 
%%url https://t.me/Lesev_Igor/311
 
%%author_id lesev_igor
%%date 
 
%%tags napadenie,rossia,ugroza,ukraina,zelenskii_vladimir
%%title Знаете, я очень суеверный человек
 
%%endhead 
 
\subsection{Знаете, я очень суеверный человек}
\label{sec:14_02_2022.tg.lesev_igor.1.suevernyj_chelovek}
 
\Purl{https://t.me/Lesev_Igor/311}
\ifcmt
 author_begin
   author_id lesev_igor
 author_end
\fi

Знаете, я очень суеверный человек. Если все свои бытовые суеверия собрать в
один список, то слово «суеверный» как-то непроизвольно заменяется на эпитет
«ебанутый». Навскидку сейчас накидаю свой «рабочий день».

Итак, с дивана утром я ВСЕГДА встаю с левой ноги. Лет эдак с 14-15, когда об
этом мне как-то сказала подруга мамы. Не помню уже, почему так, но с левой, так
с левой.

При выходе из дома хорошая примета – первым увидеть мужика. Тут 50 на 50, но
как вижу именно мужика, сразу поднимается настроение.

О черных котах и хождениями под лестницами, думаю, и сами знаете. Правда, с
котами уже чуть попустило и ограничиваюсь трехкратным плевком через левое
плечо. А были времена, когда я по газонам бегал, пытаясь опередить котяру
наперерез.

Баба с пустым ведром дорогу не должна переходить. Это тоже знаете. Хотя женский
пол у меня особенно демонический. Иногда кривой взгляд женщины (по моей версии
параноика) тут же приводит к плевкам через плечо. За кривые взгляды от мужиков
плеваться не приходилось.

Мусор после захода солнца не выношу. Даже если он воняет.

Никаких пустых бутылок от алкоголя на столе! Это к ссоре. С застольными ссорами
и так полный порядок, но кто знает, что б произошло, будь на столе в тот
скандальный момент еще и пустая бутылка.

Вернулся – посмотри в зеркало. Но я перестраховываюсь, и если кто дома,
ору/звоню, чтобы забытое скинули с балкона. Иногда летят очень хрупкие вещи,
вроде мобильного, но лучше ведь телефон разбить, нежели какое несчастье
схватить, так ведь?

Еду куда-то с ночевкой – присядь на дорожу. Новое начинание – «ни пуха, ни пера
– к черту». Ложусь спать – трижды крещусь. Через порог ничего не беру. Так, что
еще? Наверняка еще какое-то дерьмо присутствует, потому что оно копится и
обогащается само по себе.

Да, и самое главное. Все эти атрибуты/приметы – мракобесие. Да, я это соблюдаю
ради душевного успокоения, потому что всё это не затратно и ненапряжно. В
спортзал сложнее заставить себя сходить. Но сами по себе эти дурацкие
предостережения не работают, если нет к какому-либо начинанию должной
подготовки. Вот когда я был студентом, то перед каждый экзаменом ровно в
полночь у нас орали из окна «Шара, приди!». И я орал. Но еще на протяжении
сессии пытался получить по экзаменам автоматом пятерку, выступая на семинарах.
Это было гораздо легче, если сравнивать с нервотрепкой и фортуной на самом
экзамене. А где не ставили автоматом пятерки, «банально» готовился к экзаменам,
и уже там добирал свои «видминно» или «добре». И, конечно же, кричал «Шара,
приди!». Но это все-таки была не совсем шара.

И вот наш обдолбанный Блогер предлагает 16 февраля – напомню, на всех
сумасшедших Украины в этот день нападает Путин – вывесить в каждом окне по
флагу и в 10 утра спеть гимн.

В очередной раз разбирать шизофреническую историю о ру-нападении не буду, уже
утомился. Но для справки еще раз уточню. Если вдруг когда-то начнется настоящий
замес, это зеленый обдолбыш предложит вывесить флаг и спеть гимн. И ничего
другого у него в загашнике для нас больше не будет.

\begin{itemize} % {
\iusr{Лебедев Андрей}

Похоже предлагается молебен за упокой

\iusr{Виктор}

С датой ошибочка

\ifcmt
  ig https://i2.paste.pics/c1ac4b848e48ba2c75671f3c369921c6.png
  @width 0.3
\fi

\iusr{Алена Барвинская}

А может быть лучше устроить флешмоб, по стрельбе из рогаток? Так больше шансов,
Путина напугать! Обдолбанное недоразумение, плюс ко всем уклонист,
говнокомандующий Зеленский!

\iusr{Вася Пупкин}

Ну я поборол в себе такое же, у меня черный кот, на даче соседки с ведрами
постоянно шмыгают, с памятью тоже. И вообще, Игорь, с вашим то интелектом
,аналитическим складом ума и юмором несколько странно

\iusr{Звягін}

Ага, разбежались. Два года нас разъединял на титульных и некоренных, а теперь
хочет одним Указом, когда жареный петух клюнул, продемонстрировать  
единение - прапоры, гимн.

Вот хрен теперь

\iusr{Olga Schon}

\enquote{Так грустно, что хочется курить...} (с)

\iusr{Александр Александр}

Так Крым так-же сдавали, маршировали под гимн перед "вежливыми людьми",
накануне отдав им ключи от оружеек. Вот они ржали..... Печально, что кто громче
перед ними пел - получили украинские награды.

\iusr{Димитрий Соловьёв}

Здравствуйте, Игорь! Всегда встаю с правой. По мусору и пустым бутылкам 100\%
совпадение. Принципиальное разногласие: для вас \enquote{суеверие} аналог \enquote{ебанутости},
а для мене \enquote{суверенности}. Вот вы суеверный, поэтому и суверенный. Приятно
иметь дело со свободным (независимым 41 год) человеком.

\iusr{Olga}

Мне его предложение напоминает капитуляцию Германии в 45 году.  Там из каждого
окна белые флаги вывешивали.  Ну, вот такие это вызывает ассоциации.
Господи... ему больше сорока... он что, до сих пор не повзрослел ?

\iusr{Olga}

@igg{fbicon.thumb.up.yellow}.  Некоторые даже депутатами стали.  Ну а почему
бы и нет ?   Страна абсурдов...

\iusr{Александр Александр}

А самый главный солист - Мамчур, то мурло, которое должно было умереть за
Родину, построил всех офицеров части ( лётчики) и маршируя впереди строя,
сделал два круга ( не на самолёте) под гимн с флагом части(!!!!) перед
москалями. В итоге ему дали героя Украины. Голосистое опущенное чмо.

\end{itemize} % }
