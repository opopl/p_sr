% vim: keymap=russian-jcukenwin
%%beginhead 
 
%%file 28_07_2021.fb.uhman_mihail.1.ukraina_1139_rokiv
%%parent 28_07_2021
 
%%url https://www.facebook.com/permalink.php?story_fbid=2885488198381887&id=100007622059423
 
%%author 
%%author_id uhman_mihail
%%author_url 
 
%%tags istoria,kievrus,ukraina
%%title Україні - 1139 років
 
%%endhead 
 
\subsection{Україні - 1139 років}
\label{sec:28_07_2021.fb.uhman_mihail.1.ukraina_1139_rokiv}
 
\Purl{https://www.facebook.com/permalink.php?story_fbid=2885488198381887&id=100007622059423}
\ifcmt
 author_begin
   author_id uhman_mihail
 author_end
\fi

Україні - 1139 років.

На світлинах - пам'ятник князю Володимиру   Великому в Лондоні, на якому
написано - Правитель України.

\ifcmt
  tab_begin cols=2

     pic https://scontent-cdt1-1.xx.fbcdn.net/v/t1.6435-9/225184554_2885488118381895_1154315583322646418_n.jpg?_nc_cat=106&ccb=1-3&_nc_sid=8bfeb9&_nc_ohc=5TS6avWEpRIAX8NxaWQ&_nc_ht=scontent-cdt1-1.xx&oh=a10f633d2596142a5994b8a821d51372&oe=61271EEF

     pic https://scontent-cdg2-1.xx.fbcdn.net/v/t1.6435-9/220558129_2885488165048557_6496897345227057727_n.jpg?_nc_cat=108&ccb=1-3&_nc_sid=8bfeb9&_nc_ohc=JDZ7XpYRm88AX-FdJlU&_nc_ht=scontent-cdg2-1.xx&oh=76a49720020c68f5297834af19bfa3ac&oe=6128614C

  tab_end
\fi

Мудрі люди, все розуміють. Українська держава сягає роками далеко в глибину
віків, тисячоліть.

882 рік - це перша згадка про могутню Київську державу, яку поважають і
бояться. Могутні київські князі не дивляться ворогам в очі, беруть меч і йдуть
на ворогів, кажучи: "Іду на ви"! 

957 рік - Київська княгиня Ольга їде в Константинополь і веде переговори з
імператором. Приймає християнство, чим завойовує повагу, отримує силу, більшу,
ніж від завоювань мечем.

988 рік - київський князь Володимир Хрестить Київську державу. 

Перемігши ромеїв під містечком Херсонес, він бере в дружини сестру
Константинопольського імператора. І сам стає імператором.

Київ бояться, його поважають, ненавидять, заздрять. Через прийняття
християнства - офіційно визнають.

1051 рік - дочка Київського князя Ярослава Мудрого стає королевою Франції!!!

Київська княгиня - Королева Франції...!!!

Світ штормить, імперії розпадаються і виникають. Не проходить безслідно життя і
в Київській державі.

1253 - один з її повелителів, князь Данило Галицький стає королем. Нашу державу
визнає Папа Римський, з нею намагаються подружитися всі європейські монархи.

Тут в мене виникає запитання, а де там була Москва з кремлем?  Якого кольору
жаби співали на їхньому місці!???

1620  рік - у скромному храмі на київському Подолі жовтневої ночі 1620 року в
стані суворої секретності відбувалося висвячення православних ієрархів. Вікна
церкви затулили дошками і завісили цупкими шторами, щоб жоден промінчик світла
не пробився назовні й не привернув сторонньої уваги.

Літургію служили ледь не пошепки, а замість хору був єдиний півчий з почту
патріарха Теофана. 

Тієї ночі ігумена київського Братського монастиря Ісайю Копинського висвятили
на єпископа Перемишльського, Йова Борецького – на митрополита Київського,
Мелетія Смотрицького – на архієпископа Полоцького.

Православна церква України - відродилася з допомогою Київського гетьмана
Сагайдачного! 

Який громив ворогів на ліво і на право, і саме головне - перед тим, славетний
гетьман взяв в облогу Москву, знищивши чимало московитів, нагадавши їм, що вони
ніхто...

Далі московитів бив  гетьман Виговський, Українські Січові стрільці, війська
гетьмана Скоропадського, воїни Директорії, воїни УПА і ми продовжуємо славетний
шлях визначних лицарів.

Хоча за ці століття, завжди знаходилося чимало лайна, яке не давало Київській
державі розвиватися, ми вистояли і стали сильнішими.

Тому що - сам апостол Андрій благословив українську державу.

На берегах Дніпра у Києві - хрестили Київську державу.

Від Константинополя нам була дана влада перемагати бісів в Ім'я Отця і Сина, і
Святого Духа!!!

Два роки тому, ми отримали Томос! Підтвердження відновлення Православної Церкви
України!

Ще раз запитую, де там була Московія, московський патріархат? Хто вони, що за
плем'я???

Будьмо сильні, пишаймося своїм корінням. Тільки таких бояться! 

Слава Україні!

\ii{28_07_2021.fb.uhman_mihail.1.ukraina_1139_rokiv.cmt}
