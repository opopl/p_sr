% vim: keymap=russian-jcukenwin
%%beginhead 
 
%%file slova.plan
%%parent slova
 
%%url 
 
%%author 
%%author_id 
%%author_url 
 
%%tags 
%%title 
 
%%endhead 
\chapter{План}
\label{sec:slova.plan}

%%%cit
%%%cit_head
%%%cit_pic
%%%cit_text
По его ведомству разрабатывались и даже отчасти воплощались \emph{планы} колонизации
Крыма немцами, который должен был сменить название на \enquote{Готенланд} или
\enquote{Готенгау}. Симферополь предлагалось переименовать в Готенбург, Севастополь - в
Теодорихсхафен.  Немцы настолько ценили Крым, что даже собирались, не дожидаясь
конца войны, выселить оттуда практически все население - включая крымских татар
- заменив его выходцами из Южного Тироля. Однако стало непонятно, кто будет
выполнять черные работы, и этот вопрос отложили на \enquote{после войны}.  Также с
подачи Розенберга гитлеровское руководство даже после погрома ОУН активно
использовало украинских коллаборационистов, давая им ряд культурных и
экономических преференций
%%%cit_comment
%%%cit_title
  \citTitle{22 июня - 80 лет нападения на СССР. Что немцы готовили для украинцев}, Максим Минин, strana.ua, 22.06.2021
%%%endcit
