% vim: keymap=russian-jcukenwin
%%beginhead 
 
%%file 03_12_2021.fb.dudkin_jurij.1.kosmos.cmt
%%parent 03_12_2021.fb.dudkin_jurij.1.kosmos
 
%%url 
 
%%author_id 
%%date 
 
%%tags 
%%title 
 
%%endhead 
\subsubsection{Коментарі}

\begin{itemize} % {
\iusr{Регина Ханивецкая}

Ну, вообще-то, пару лет назад писали, что она сделала это в отместку за то, что
её не пускали в туалет в российском сегменте МКС. В американском забился
туалет, и она рвалась к соседям. А ей объяснили, что это невозможно. Почему
невозможно, не помню. Но вместо того, чтоб решить эту проблему в своём модуле,
она отомстила. Теперь придумывается история про несчастную любовь. Просто
дегенераты встречаются везде, и в космосе тоже.

\iusr{Николай Лятошинский}

Собственно говоря, наблюдая за всем безумием и паранойей, котороя творится на
Западе, не удивительно, что у них даже астронавты шизанутые.

\iusr{Natalia Vitalievna}
Логично. Отбор хромает.

\iusr{Андрей Черевута}
Не женское это дело... @igg{fbicon.face.confused} 

\iusr{Юрий Дудкин}
\textbf{Андрей Черевута} . Женское, мужское.... не для идиотов!)

\iusr{Борис Савин}
Не Серина, а серун.

\iusr{Рустам Таран}
Как они туда еще грету не запустили

\iusr{Рустам Таран}
Ну так и продрявила бы свой модуль вместе со своим бойфрендом че на российский модуль лезть

\iusr{Natalia Vitalievna}
Да и бой френд вдруг откуда взялся? Или их специально парочкой заслали?

С этим тоже внимательно - зачем там любовь морковь? А если он устал, то она, значит, скандалы будет закатывать 2 раза в день?
Странное допущение таких бойфрендских отношений....
Пусть сами без моторов катаются в космос.
Их космос такой совсем короткий - 80 км. Пусть тем и довольствуются. Ауньоны.

\begin{itemize} % {
\iusr{Вадим Аксенов}
Есть такая штвка как интернет. В интернете есть скайп. Бойфренду вовсе не обязательно присутствовать на борту физически.)))))
\end{itemize} % }

\iusr{Natalia Vitalievna}
На лице написано. Да и имя говорящее: серина. Всем хотела насеринить. Насеринила.
Короче, полный ахуньон.
Роскосмос сам должен проверять гастронавтоф на вменяемость. И с такими говорящими именами лучше оставлять за океаном. Мировой опыт. Как назовёшь аhуньон, так он и поплывёт.

\iusr{Михаил Ключник}
Где же NASA находит такие мерзкие рожи....

\iusr{Людмила Роговик-Никулина}
\textbf{Михаил Ключник} ну так выбора нет, там все мерзкие

\iusr{Татьяна Сидоренко}
Дурья башка, она домой могла бы пораньше вернуться в виде горстки пепла .

\iusr{Natalia Vitalievna}
Да кто бы её собирал? Молекулы кто ловить будет в межгалактике?


\iusr{Игорь Фадеев}

" НАУКА И ВЛАСТЬ

19.04.2021 17:57

Источник: NASA

Ищите женщину: космонавты рассказали, кто мог просверлить дыру на МКС

Космонавт Кондакова допустила, что дыру на МКС просверлила американка

Алексей Ленский, Павел Котляр

Летчик-космонавт РФ Елена Кондакова в беседе с «Газетой.Ru» допустила, что дыру
в МКС проделала астронавтка Серина Ауньон-Ченселлор, у которой незадолго до
этого обнаружили тромб. По ее словам, «американцы — люди специфические». Таким
образом она отреагировала на заявление космонавта Олега Атькова о том, что
отверстие на МКС просверлил психически неустойчивый человек, у которого в школе
не было уроков труда.

Летчик-космонавт Российской Федерации, Герой России Елена Кондакова
прокомментировала «Газете.Ru» намек космонавта Олега Атькова на то, что дыру в
МКС, возможно, просверлила женщина.

«Не пойман — не вор, кто конкретно это сделал, не доказано, но предположения о
том, что с большой вероятностью это было сделано ею (астронавткой Сериной
Ауньон-Ченселлор. — «Газета.Ru»), были. Я ее не знаю, но в отечественной
космонавтике тоже была ситуация с одной из девушек. Ее пытались тянуть [в
космос], но медики дали заключение, что человек психологически неуравновешен,
непонятно, как он будет себя вести в полете в невесомости. Думаю, у американцев
это далеко не первый случай, думаю, вы помните, как одна из астронавток
помчалась за своим возлюбленным в другой город, — сказала она «Газете.Ru». —

Американцы вообще люди специфические, мы за коллективизм, а они все
индивидуалисты. Ну, может, человеку захотелось раньше домой вернуться...».

Космонавт Олег Атьков, выступая на общем собрании Отделения медицинских наук
РАН, заявил, что дыру на МКС просверлил психически неуравновешенный человек, у
которого «не было уроков труда». Прямую трансляцию мероприятия провел портал
«Научная Россия».

Говоря о развитии космической медицины, Атьков уделил внимание гетерогенным
экипажам. Он продемонстрировал фото экипажа советской станции «Салют-7»,
обратив внимание, что в нем присутствовала женщина — вторая в мире
женщина-космонавт Светлана Савицкая, и отметил, что показывает это фото не
просто так.

Проблемы гетерогенности экипажа связаны с культурными и языковыми различиями,
которые могут повлиять на общение между членами экипажа, пояснил он, и такие
проблемы уже возникали. В связи с этим он упомянул случай с просверленным в
корабле «Союз МС-09» отверстием.

«Если мы хотим избежать ситуации, когда неумелой рукой, сверлом сверлится
отверстие, а сверло соскальзывает, потому что человек на уроке труда... а там
не было уроков труда, и он не знает как сверлить кривизну.

Потом все это заклеивается, чтобы потом была обнаружена или не обнаружена
утечка кислорода и падение давления... Нужно очень хорошо отбирать людей и
обеспечивать психологическую поддержку на всем протяжении полета», — сказал
Атьков.

Незадолго до обнаружения дыры у одной из членов экипажа, американской
астронавтки Серины Ауньон-Ченселлор, был найден тромб в сосуде шеи. Хотя
Роскосмос так до сих пор и не назвал виновника ЧП на станции, высказывались
версии, что из-за стресса, вызванного беспокойством за свою жизнь, она могла
просверлить отверстие, чтобы поскорее вернуться домой.

В Роскосмосе прокомментировать слова космонавта Атькова отказались.

Дыру в обшивке корабля «Союз-МС-09» экипаж МКС обнаружил еще 30 августа 2018
года. В тот день в ЦУПе заметили, что на МКС происходит медленная утечка
воздуха. Скорость утечки не представляла угрозу — давление на станции падало
примерно на 0,8 миллиметра ртутного столба в час. С такой скоростью станция
полностью лишилась бы воздуха за 18 дней, подсчитали эксперты.

Вскоре причина утечки была локализована — ею оказалась дыра диаметром около
двух миллиметров в обшивке российского корабля «Союз МС-09».

После заделки дыры специальным герметиком встал вопрос о ее происхождении.
Версия о зацепившем обшивку маленьком метеорите отпала сразу после того, как
космонавты увидели дыру — она была явно рукотворной. Это оказался след от
дрели, с несколькими бороздами рядом, словно у сверлившего соскакивали
руки...."


\iusr{Игорь Фадеев}

Скажем провидению спасибо за то что Клинтонше не повезло стать астронавткой,
ведь она тоже мечтала покорять звёзды, неизвестно чем бы закончился её полёт
для мировой космонавтики ....

\iusr{Nadiya Shelest}
Wow...

\iusr{Наталия Вертинская}
Истеричка неадекватная она.

\iusr{Кон Здравбудь}
Понабирають янки астронавтів по оголошенню, а потім списують проблеми на інших. Та щей дивуються якже воно таке трапилось.

\iusr{Оксана Яровенко}
Кого-то она мне напомнила, прям вчерашняя истерика  @igg{fbicon.man.facepalming} 

\iusr{Галина Скрипник}
Безумная дамочка

\iusr{Марк Запорожье}
Уж лучше бы вместо этой ненормальной американской бабы в космос собаку запустили. Собаки точно не способны на такую подлость.

\iusr{Василий Табуреткин}
Тут в стране дыр наделали...

\iusr{Ivan Ilin}
Пришло же в голову, блять.

\iusr{Леонид Любимов}
На хрена нам эти соседи там ???

\iusr{Сергей Пересунько}
Нет ничего удивительного. Это их норм.

\iusr{Serge Leroi}
Там же на литсе написано, что она психически нездорова.
 @igg{fbicon.face.rolling.eyes} 

\iusr{Константин Жеренков}
Есть и другое мнение, что это брак с земли.

\begin{itemize} % {
\iusr{Юрий Дудкин}
\textbf{Константин Жеренков} . Роскосмос опроверг это мнение. При старте контроль давления это бы выявил сразу.

\iusr{Константин Жеренков}
\textbf{Юрий Дудкин} тем не менее есть ныне удаленное интервью инженера, где он говорит, что это брак с земли.


\iusr{Юрий Дудкин}
\textbf{Константин Жеренков} . Вы полагаете, что там плохие эксперты и рукотворную дыру не могут отличить от брака, при чем в совсем обычном месте?)

\iusr{Константин Жеренков}
\textbf{Юрий Дудкин} 

я полагаю, что этот вопрос политический, и одни будут говорить одно, а
другие-другое. А как на самом деле было? А как угодно, это бесполезное
распространение пропаганды.

\iusr{Юрий Дудкин}
\textbf{Константин Жеренков} . Только пропаганду нужно разоблачать фактами, а не штампами!

\iusr{Константин Жеренков}
\textbf{Юрий Дудкин} и где тут факт? Одни говорят одно, другие -другое.

\iusr{Valerian Malyschko}

Чего вы напали на женщину. Может там не было электродрели и даже отверстие 2-3
мил. было бы за считаные часы смертельны для экипажа. Почему молчат физики? В
космосе летят частицы со скоростью пробивающие землю

\iusr{Natalia Vitalievna}

Ну да, с земли. То есть утверждающий это, думает, что хлеб на дереве растёт. А
корабли нежно на облачках подымаются в космос ))

\iusr{Игорь Фадеев}

Утечка появилась после того как весь экипаж проводил профилактические работы с
наружи корабля а эта дама оставалась внутри, типа на дежурстве, перед этим она
просила досрочного возвращения на Землю для прохождения обследования по поводу
тромба, который вроде бы как обнаружилася во время полёта.

Получив отказ она видимо решила ускорить завершение миссии и скорую эвакуацию
..

\iusr{Константин Жеренков}
\textbf{Игорь Фадеев} 

вам именно в такую информацию нравится верить, это называется самообман, вам
было бы не приятно, если это заводской брак. Именно поэтому вы выбираете то,
что Вам нравится, а не то, что было на самом деле. Мало того, Вы про это ничего
не можете знать. Вы как и остальные получаете всю инфу из СМИ)))


\iusr{Игорь Фадеев}
\textbf{Константин Жеренков} 

включите логику, отверстие оставленное во время работ на Земле давало бы утечку
сразу, со времени старта и этого не возможно было не заметить ...

\iusr{Константин Жеренков}
\textbf{Игорь Фадеев} нет не сразу, есть всякие космические герметики, клеи и т п. Могли заклеить, но продержалось вот ровно столько.

\iusr{Игорь Фадеев}
\textbf{Константин Жеренков} 

может вы и правы, по крайней мере представители Роскосмоса раньше не
акцентировали внимание на этом инциденте

"... Согласно сообщениям ряда российских специалистов, Серину Ауньон-Чанселлор
подозревают в создании внештатной ситуации на борту «Союза», приведшей к
падению давления воздуха внутри МКС, — просверливанию отверстия в обшивке.
Причиной этому называется нервный срыв, из-за обнаружения у неё
тромбообразования возникшего на орбите, что вынудило её разными способами
ускорить возвращение на планету. Диагноз она узнала во время планового
медицинского осмотра на МКС. После полёта она стала соавтором научной статьи
2019 года о влиянии невесомости на тромбообразования астронавтов МКС. Глава
«Роскосмоса» Дмитрий Рогозин в августе 2021 года заявил, что Роскосмос не
обвиняет американского астронавта Серину Ауньон-Чэнселлор и что организация не
хочет препятствий для дальнейшего сотрудничества с НАСА. А представитель НАСА,
Кэти Ледерс сообщила, что они отвергают предположения о любых психологических и
медицинских проблемах у астронавтов на МКС. "

\iusr{Константин Жеренков}
\textbf{Игорь Фадеев} 

вот пипец, комонавт вообще-то понимает, что сверление отверстий -очень
рискованное занятие, никто, тем более космонавт, на такую фигню не пойдет. Если
уж Вы хотите приколов, спросите, почему у нашего космонавта Сергея Рязанского
книга называется "как забить гвоздь в космосе"


\iusr{Yanina Garbovskaya}

Эта женщина летела в космос по собственному желанию и думаю она была
нормальная. Но, как обнаружилось у нее заболевание тромбоза она запаниковала и
у нее получилось нервное расстройство. Не каждый туда может лететь, а тем
более, нежная женщина. Ее осуждать нельзя. С каждым человеком может такое
случится. Она таким способом, как просверлить дырочку, решила спасти свою
жизнь. Не приятная конечно история но, мы люди и должны простить эту милую
дамочку, тем более, что все закончилось хорошо.

\end{itemize} % }

\iusr{Вадим Пинаев}
Правильнее писать не Чанселлор, а Чэнселлор. Да и на бирке у нее так.

\begin{itemize} % {
\iusr{Юрий Дудкин}
\textbf{Вадим Пинаев} . Очень "существенное" к этому посту замечание. Но, я цитирую агенство ТАСС.

\iusr{Natalia Vitalievna}
То же самое. Произношение и то, и то допустимо. Тем более транслитерация.

\iusr{A'lder Crimea}
\textbf{Вадим Пинаев}
Член-сел-в-хлор  @igg{fbicon.face.grinning.squinting}{repeat=2}  @igg{fbicon.thumb.up.yellow} 
\end{itemize} % }

\iusr{Валентина Ккк}
Вот дрянь... Это мягко сказано...

\iusr{Sergey Uvarov}

К сожалению в этой популяции, подавляющее большинство неадекваты, поэтому и
происходят все эти кризисы, войны, перевороты и даже эпидемии.


\iusr{Алекс Алекс}
Соединёнки в космос стали психов посылать. У неё на роже написано - палата № 6.

\iusr{Александр Минин}
Сучка...


\end{itemize} % }
