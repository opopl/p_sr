% vim: keymap=russian-jcukenwin
%%beginhead 
 
%%file slova.cennosti
%%parent slova
 
%%url 
 
%%author 
%%author_id 
%%author_url 
 
%%tags 
%%title 
 
%%endhead 
\chapter{Ценности}
\label{sec:slova.cennosti}

Что делать с \emph{ценностями}? Смотрим. ...  Как же их объединить в одной душе
и в одном мозге?  Печаль заключается в том, что обретая \emph{европейские
ценности}, придется чуть-чуть пожертвовать своими, \emph{национальными}, так
как они не всегда совпадают.  Если руководствоваться лишь своими \emph{личными
ценностями}, то можно войти в клинч с общечеловеческими.  Отсюда вопрос: в
какой пропорции должны быть эти \emph{ценности}? (А-Б-С-Д) Одни пишут так
(10-20-30-40), другие наоборот 40-30-20-10. Третьи – все поровну, по 25\%.  А у
Вас какая \emph{пропорция ценностей}?,
\textbf{Где баланс между европейскими, национальными и личными ценностями?},
Владимир Спиваковский, strana.ua, 06.05.2021

%%%cit
%%%cit_head
%%%cit_pic
%%%cit_text
«Сегодня мы живём в мире, где зачастую под светлыми идеями скрываются тёмные
намерения извратить базовые человеческие потребности и \emph{ценности}. Чуждая
традиционной украинской культуре западная идеология под предлогом борьбы с
гендерным неравенством и любой дискриминацией навязывает нам приятие
нетрадиционной сексуальной ориентации, однополых браков и тотальной половой
распущенности. Мы подробно и понятно будем объяснять, как это делается и как от
этого защитить себя и близких, но главное — мы дадим информационную
альтернативу своими образовательными и творческими инициативами», — говорится в
публикации на странице «Миряне» в Facebook
%%%cit_comment
%%%cit_title
\citTitle{«Невидимое большинство». Православных на Украине призывают проснуться}, 
Андрей Лубенский, ukraina.ru, 09.06.2021
%%%endcit

%%%cit
%%%cit_head
%%%cit_pic
%%%cit_text
Чи винна держава в тому, що її представляють люди, котрі просто не здатні
нормально її представити? Так. З огляду на 30 років обережної незалежности, яка
в нас відбувалася помежи російські інтереси – так, винна. Але від того, що ми
будемо медитувати над історичною виною України у транслюванні власних
\emph{цінностей}, сьогодні ліпше не стане.  Так само, як підтримка фанатів не
зробить численних спортивних представників України усвідомлено українцями, а
обурення соцмереж не додасть Усику розуму.  Це, коли що, до питання про те, на
якому етапі можна припиняти вважати \enquote{наших} \enquote{нашими}. Я спостерігаю збірну
Малоросії плюс Яремчука. Якщо вас цікавить не техніка гри, а саме участь
\enquote{наших} – поділіться, як ви з ними ідентифікуєтесь
%%%cit_comment
%%%cit_title
\citTitle{Що, крім футболок, робить збірну України з футболу \enquote{нашою}?}, 
Остап Українець, gazeta.ua, 15.06.2021
%%%endcit

%%%cit
%%%cit_head
%%%cit_pic
%%%cit_text
Эпилог или цена выживания. Разворот российского общества от движения к слиянию во всемирном, вернее,
западном гомосексуальном интернационале к традиционной модели, спасающей от
наблюдаемой и неминуемой депопуляции, выдвижение примата неких этических и
духовных \emph{ценностей} над стремлением к материальному обогащению, отказ от
главенства прав человека в пользу приоритета интересов общества и популяции
потребует колоссальных усилий. Контрреволюции и слома сознания по масштабам
равной русской революции столетней давности. О том, что теоретически такой слом
общества возможен, свидетельствует опыт многочисленных социалистических
революций, рушивших традиционные общества, и пример духовной исламской
революции в Иране
%%%cit_comment
%%%cit_title
\citTitle{Революция Духа – единственный путь спасения России}, 
Юрий Барбашов, voskhodinfo.su, 30.06.2021
%%%endcit

%%%cit
%%%cit_head
%%%cit_pic
%%%cit_text
І справді, в Україні за останні сім років у багатьох суспільних ланках
відбувається розмивання майданних \emph{цінностей} – як-от, на жаль, у системі освіти.
Старий «суспільний договір» не мириться з європейськими новаціями і, як
мовиться у порівняно недавньому зверненні УКУ, прагне відтворити «милу серцю
реальність українських не-стандартів ручного управління, кумівства,
патерналізму і, звісно, цинізму».  Сказати, щó насправді приживеться, а щó буде
відторгнуто – ніхто не може наперед. Відповідь народ знайде в результаті
реального експерименту. Бо ніхто з нас до пуття не розуміє, які риси в нинішній
українській ментальності є лише наслідком цивілізаційної відсталості,
спричиненої комуністичним «ув’язненням», а тому піддатні для коригування, а
які, навпаки, належать до нашого архетипу, змінити який практично неможливо.
Скажімо, одіозна українська отаманщина й міжгрупова ворожнеча відтворюються в
нашому житті за різних суспільно-політичних обставин
%%%cit_comment
%%%cit_title
\citTitle{Чи зміниться чинний суспільний договір без наших зусиль?}, 
Мирослав Маринович, zbruc.eu, 06.11.2021
%%%endcit
