%%beginhead 
 
%%file 26_01_2023.fb.demidko_olga.mariupol.1.foto_aktory_teatr_xix_stolittja
%%parent 26_01_2023
 
%%url https://www.facebook.com/permalink.php?story_fbid=pfbid02bB6fryBX5DLzXBckHYHCUeak7uFJUqMDr3zpKNCRTwX91nsBCWxTGMTwTUcqaU9gl&id=100009080371413
 
%%author_id demidko_olga.mariupol
%%date 26_01_2023
 
%%tags mariupol.pre_war,mariupol,teatr,kultura,mariupol.kultura
%%title Сьогодні вкотре переглядала світлини акторів, які з XIX століття грали на різних сценах Маріуполя
 
%%endhead 

\subsection{Сьогодні вкотре переглядала світлини акторів, які з XIX століття грали на різних сценах Маріуполя}
\label{sec:26_01_2023.fb.demidko_olga.mariupol.1.foto_aktory_teatr_xix_stolittja}

\Purl{https://www.facebook.com/permalink.php?story_fbid=pfbid02bB6fryBX5DLzXBckHYHCUeak7uFJUqMDr3zpKNCRTwX91nsBCWxTGMTwTUcqaU9gl&id=100009080371413}
\ifcmt
 author_begin
   author_id demidko_olga.mariupol
 author_end
\fi

%Yevheniia Buzuk
%https://www.facebook.com/groups/1654154218147103/user/100003167260627/
%Ирина Руденко
%https://www.facebook.com/irina.rudenko111

Сьогодні вкотре переглядала світлини акторів, які з XIX століття грали на
різних сценах Маріуполя. Мені вдалося вивезти на флешці величезний архів, який
я збирала, досліджуючи тему своєї дисертації: це понад 1000 світлин акторів та
актрис, зокрема і з Державного грецького театру, Зимового міського театру
Маріуполя, Театру братів Яковенків, Театру "Солей",  Маріупольського
музично-драматичного театру ім. Т.Шевченка,  Народного драматичного театру ПК
"Азовсталь" тощо.  Близько 500 фото програмок і афіш. Це і професійні театри, і
самодіяльні. Всі ці фото я почала оцифровувати з 2014 року, працюючи у фондах
Маріупольського краєзнавчого музею, Донецького академічного обласного
драматичного театру (м. Маріуполь), Палацу культури "Молодіжний", Лялькового
театру. Також багато матеріалів зібрала з  Особистих архівів народної артистки
України Світлани Отченашенко, краєзнавця Сергія Бурова, режисерки самодіяльного
театру "Діалог"Лідії Хаджинової та засновниці Маріупольського театру ляльок і
заслуженої діячки естрадних мистецтв Ірини Руденко. Є і матеріали з архівів
Запорізької області, Одеси, Києва. Але саме фотодокументи з Маріуполя набувають
наразі особливої значущості, адже всі ці важливі історичні джерела знищені
окупантами. Мрію продовжити популяризувати історію театрального мистецтва
Маріуполя і загалом Приазов'я та сподіваюся, що завдяки матеріалам, які вдалося
вивезти, вдастся зберегти театральну історію цілого регіону.

%\ii{26_01_2023.fb.demidko_olga.mariupol.1.foto_aktory_teatr_xix_stolittja.cmt}
