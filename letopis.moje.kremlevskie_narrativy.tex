% vim: keymap=russian-jcukenwin
%%beginhead 
 
%%file moje.kremlevskie_narrativy
%%parent moje
 
%%url 
 
%%author_id 
%%date 
 
%%tags 
%%title 
 
%%endhead 

\subsection{Кремлевские нарративы vs нарративы Киева}

Февраль 2022 года, город Киев. Этот текст был начат до вторжения Российской
Федерации в Украину, а закончен - уже после. Поэтому в основном этот текст
отображает наше восприятие (настроение) до начала настоящей
российско-украинской войны, с позднейшими вставками уже по ходу войны, которая
стала настоящим водоразделом в массовом сознании миллионов простых людей,
прежде всего граждан Украины, Российской Федерации, и Республики Беларусь, как
наиболее близко родственных, и ментально, и исторически.

О чем пойдет речь в этом тексте? Не о том, как мы именно относимся к Российской
Федерации, государству-агрессору, вторгнувшемся в Украину в феврале 2022 года,
здесь наше отношение как граждан Украины совершенно однозначно, а то, как мы
видим мир как Киевляне. Вам, наверное, если Вы читаете эти строки, больно
видеть само это слово Россия, очень больно... вы наверное уверены на сто
процентов, что Россия - теперь наш навечно враг, и никогда уже не будет иначе.
И поэтому зачем что-то еще читать, тут и все так понятно без лишних слов. Зачем
что-то еще читать, если на третий день войны счет погибших исчисляется уже
несколько тысячами, включая гражданских и детей, нанесен колоссальный вред
инфраструктуре по всей Украине, и что будет дальше, вообще неизвестно.  А
сколько было пролито слез, сколько людей оказались в шоковом, тяжелом
психологическом состоянии, этого вообще наверное никто и не подсчитает, разве
что сам Бог знает.

Тем не менее, надеемся, Вы найдете здесь что-то для себя полезное и важное.
Потому что речь, в итоге, все-таки, в первую очередь пойдет о Киеве, а значит -
об Украине, поскольку Киев является нервным узлом всей Украины, ее настоящим
сердцем, и все, что происходит в Киеве, имеет воздействие в итоге на всю
Украину. К сожалению, в современном мире политики и культуры, в средствах
массовой информации и в интернете, Киев как город и как отдельная
самостоятельная сущность, а также Киевляне, как самостоятельная общность/группа
людей, до сих пор занимал незаслуженно мало места, и мы бы хотели по возможности
восполнить этот пробел. И как известно, Киев имеет также особое, духовное
значение вообще для всего славянского мира, как место, где более чем тысячу лет
назад произошло Крещение Руси князем Владимиром, но это тема вообще для
другого, отдельного исследования.

\ifcmt
  tab_begin cols=2,no_fig,center
     pic https://ru.oddviser.com/photo/place/1600/963.jpg
		 pic https://pbs.twimg.com/media/EePq2EiWAAEbcmQ.jpg
  tab_end
\fi

Значит так... мы бы хотели здесь обсудить кремлевские нарративы, которые всем
уже известны, а наше обсуждение их собственно и будет нашим Киевским
нарративом.  И насколько мы знаем, никто еще не говорил о Киевских нарративах,
а зря! Чем Киев хуже Кремля в качестве источника нарративов? Чем мы, простые
Киевляне, хуже Путина с Сурковым и Скабеевой? Их же всего несколько человек, а
нас - несколько миллионов.

Итак, прежде всего, что такое есть нарратив. Согласно Википедии, нарратив
(англ. и фр. narrative от лат. narrare «рассказывать, повествовать») —
\enquote{самостоятельно созданное повествование о некотором множестве
взаимосвязанных событий, представленное читателю или слушателю в виде
последовательности слов или образов}. А слова будут здесь наши собственные, ну
а образы - (а) фото нашего прекрасного Киева, (б) также рисунки/картинки/фото
из различных исторических эпох Киева, и (в) просто красивые картинки. В общем,
мы просто любим Киев, и поэтому не удивляйтесь, что Киева здесь будет много. И
мы не претендуем на какую-то высокую научность наших мыслей, но будем по
возможности стараться мыслить четко и последовательно.

Далее, как вы может быть и сами знаете, очень часто, когда идет обсуждение
каких-то проблем в Украине, идет в ход аргумент относительно того, что то или
иное мнение является так называемым кремлевским нарративом, то есть как бы
исходящим напрямую от Кремля. Также говорят о так называемых агентах Кремля,
или же агентах Путина.

\ifcmt
  tab_begin cols=2,no_fig,center
     pic https://avatars.mds.yandex.net/i?id=2e06b1c83ed05948efde9bcaffca748f-5223638-images-thumbs&n=13
		 pic https://avatars.mds.yandex.net/i?id=b95ffc618d1cb237a4569b1ab99c40ae-4092070-images-thumbs&n=13
  tab_end
\fi

Кремлевские нарративы, кремлевскую пропаганду нынче постоянно упоминают в
разговорах политиков и простых людей... Здесь - кремлевский нарратив, там -
кремлевская пропаганда.  А вот тут - кремлевский подлиза Шарий, а вот там - так
вообще полная скабеевщина. Вот там - пророссийское пропагандистское издание
Страна, а тут - снова вездесущий киевлянин в изгнании Шарий с его сайтами
sharij.net и Новым Изданием. Тут пророссийский канал Наш, а там - сам кум
Путина, всем известный зловещий Медведчук. Далее, ущемление прав русскоязычных
- Кремль. Нейтральный (внеблоковый) статус Украины - конечно, Кремль. Профессор
Евгения Бильченко и Светлана Пикта, многодетная мать, изгнанная за свою
пацифистскую позицию из Киева - тоже все агенты Кремля. Олеся Медведева и
Антонина Белоглазова - тоже агенты Кремля. Русско-украинский писатель Шевченко
- тоже Кремль. Русский мир - конечно Кремль. Триединый русский народ и
древнерусский язык - это все мифы Кремля. Геноцид на Донбассе - кремлевский
нарратив.  Обнищание населения и гражданская война - Кремль. Высокие тарифы -
Кремль снова. Бунтуют против вакцинации - снова Кремль. Идет беспощадная
информационная война Кремля против Украины, и конца и края ей не видно...
Русский борщ, русский писатель Гоголь, акция бессмертный полк, русский мир,
Великая Отечественная Война, а также исконно украинские колядки, перепетые
обычными детьми на русском языке, у России ж априори не может быть своих
колядок, все это ж личная интеллектуальная собственность украинцев - все это
информационные спецоперации Кремля по планомерному уничтожению Украины, ее
собственной уникальной идентичности и культуры.

И понимаете... уж не обессудьте за простой язык, но знаете, как говорят у нас
на районе...  крыша едет неспеша, тихо шифером шурша... информационное обилие
на протяжении многих лет всех этих Кремлей и Путиных, которые все время хотят
уничтожить Украину, и непременно восстановить СССР или Российскую империю, -
все эти сотни и тысячи Кремлей, Путиных, путинистов и рашистов, русни и гебни,
- на украинском ТВ, и в украинском интернете, приводит к тому, что... начинает
казаться поневоле, что... все люди вокруг вообще - агенты Кремля. Вот этот -
агент. И этот тоже - агент. И вот эта милая светловолосая девушка, что сейчас в
наушниках у себя слушает Газманова, уже пол-часа тщетно ожидая маршрутку на
Троещину - конечно, тоже, агент Кремля. А водитель маршрутки, включивший
Владимирский Централ, - или, боже упаси, - самого Иосифа Кобзона, вообще полный
зашквар, - это точно личное доверенное лицо Путина. И даже более того, как
сказала где-то Лариса Ницой, в Киеве Украины нет, тут вообще повсюду Путин. Вы
думали, Путин в Кремле? Да нет, он давно уже в Киеве, ведь Лариса Ницой
ошибаться не может.  Правда, Лариса Ницой? ( Учитывая, что вторжение уже таки
состоялось, Вам же нравится, то, что Вы когда-то ляпнули по поводу Путина в
Киеве, не так ли? )

И какой вывод получается? Украина просто кишит агентами Кремля, это ж просто
ужас какой, - поскольку смотрите, сайт Миротворец - более чем 100 000  человек
вроде уже записали в агенты Кремля, а ведь сайт Миротворец, конечно, ошибаться
не может!  Кто ж там на этом сайте есть? Мы можем тут вспомнить... подло
убитого Олеся Бузину, светлая ему память, - неутомимую Олесю Медведеву с ее
блогами Ясно-Понятно, спевшую также крутую песню про Киев, талантливую
профессора-поэтессу Евгению Бильченко, обаятельную Антонину Белоглазову с ее
опросами простых Киевлян и неизменным \enquote{Поехали!}, умнейшего академика
Петра Толочко, одного из патриархов нашей исторической науки, написавшего кучу
классных книг по Киеву, Любовь Титаренко, художественного руководителя
Киевского театра Браво, конечно, Анатолия Шария (Киевлянина в изгнании) и также
Фаину Савенкову, девочку писателя-фантаста из Луганска, и даже Президента
Хорватии. И много-много других интересных людей.  Просто поразительно!  И
кстати, столько Киевлян на Миротворце! Скоро наверное и вообще весь Киев
целиком запишут в Миротворец, как самый главный оплот Русского Мира, Киев же
как-никак Матерь Городам Русским.

\ifcmt
  tab_begin cols=2,no_fig,center
     pic https://avatars.mds.yandex.net/i?id=31ae2249ba1e06400bddba166a204810-5874640-images-thumbs&n=13
		 pic https://s00.yaplakal.com/pics/pics_original/0/7/4/4208470.jpg
  tab_end
\fi

Слушайте. Ну надоело уже. Все время несется изо всех утюгов... Кремль, Кремль,
Кремль. Кремль сделал то, Кремль сделал это. Путин мечтает о том, Путин хочет
это. У Путина есть душа? Нет у него души, он вообще убийца, так Байден
сказал... И тому подобное. Ну окей...  вы вот все такие умные и красивые, а
скажите вот по-честному, Вы правда влюблены в Кремль, в Путина, или что? Вы ж
помешаны на Путине, разве не так? В каких-то жалких кремлевских рабов
превратились на ментальном уровне. Ни дня не можете прожить без Кремля и
Путина, Лаврова и Шойгу. Путин - просто какой-то бог у вас, причем бог
противоречивый. С одной стороны, это кремлевский карлик, какой-то просто
сумасшедший, а с другой стороны, вы все его всегда внимательно слушаете и
боитесь, и даже песни про него постоянно поете!

\ifcmt
  tab_begin cols=3,no_fig,center
     pic http://photos.wikimapia.org/p/00/03/86/83/55_full.jpg
		 pic https://upload.wikimedia.org/wikipedia/commons/5/5e/%D0%A2%D1%80%D1%91%D1%85%D1%81%D0%B2%D1%8F%D1%82%D0%B8%D1%82%D0%B5%D0%BB%D1%8C%D1%81%D0%BA%D0%B0%D1%8F_%D1%83%D0%BB%D0%B8%D1%86%D0%B0_%D0%9A%D0%B8%D0%B5%D0%B2_2012_02.JPG
		 pic https://avatars.mds.yandex.net/i?id=3e72a0b484b3f8631780ad2d1e399331-4077532-images-thumbs&n=13 
  tab_end
\fi

А как же Святая София, Киев, Днепр, университет Шевченко и институт
философии имени Сковороды? Где ваши собственные Киевские нарративы? Где ваше
Киевское слово, твердое, уверенное и решительное, основанное на тысячелетней
мудрости и истории Стольного Града Киева? 

\ifcmt
  tab_begin cols=2,no_fig,center
     pic https://avatars.mds.yandex.net/i?id=4f5bfd24d995c3a5f31ee9005e89b8de-4322265-images-thumbs&n=13
		 pic https://avatars.mds.yandex.net/i?id=2c2f0377c8a3c42d841b62fdee445adb-5468554-images-thumbs&n=13
  tab_end
\fi

Разве ж Киев не старше Москвы, не мудрее, не умнее Москвы?  Почему так много
Москвы в информационном пространства и так удручающе мало Киева? Везде Москва,
Москва... также везде Украина, Украина и еще раз Украина... НАТО, ЕС, лучшие в
мире западные партнеры...  а бедный Киев... Спрятался где-то в красивых
альбомах и фейсбук-группах по истории Киева... Как же так можно, елки-палки,
обидно же! Такой же удивительный и прекрасный Город, куда там Москве до него,
хоть и Москва тоже красивый город! И вы же каждый день ездите не в Московском
метрополитене имени В. И. Ленина, а в Киевском метро - просто метро и безо
всяких \enquote{имени}, садитесь на центральном Киевском Вокзале на поезд
Киев-Львов, или же  Киев-Ужгород, а не на поезд Москва-Екатеринбург; едете на
Троещину, а не на Химки; и гуляете с детьми не в парке Горького, а по
Голосеевскому лесу, где находятся Главная Астрономическая Обсерватория Академии
Наук (ГАО НАНУ) и также музей Пирогово (Национальный музей народной архитектуры
и быта Украины).

\ifcmt
  tab_begin cols=2,no_fig,center
		 pic https://ic.pics.livejournal.com/tov_tob/21010671/2766179/2766179_original.jpg
		 pic https://avatars.mds.yandex.net/i?id=39c3b676331a509accd7cdfa13574214-5870057-images-thumbs&n=13
  tab_end
\fi

... В красивейшем киевском метро вы
едете от Арсенальной до Гидропарка, а может быть - от Золотых Ворот до
Выдубичей. Берете Киевское такси, например, Уклон или Болт, если надо быстро
куда-то доехать, а не московское такси - как оно называется, извините, - мы не
в курсе, наверное, надо позвонить в Кремль и узнать у Пескова; кладете ребенка
в Киевскую больницу, например, Охматдет, а не в институт Склифосовского (или
какие там еще есть больницы?). Почему так мало гордости за Киев, за Столицу,
почему такое безумное преклонение перед Москвой.  Киев же уже как тридцать лет
имеет собственное политическое значение, времена, когда весь центр принятия
важных решений по Украинской ССР находился в Москве, давно уже в прошлом, так
почему же вы не пускаете в ход наш родной Киев? Слово Киева, ценности Киева,
мудрость Киева? Почему же все время говорят проукраинские политики (партии,
газеты, сми); проукраинские активисты (блоггеры) - и это считается чем-то
хорошим и правильным, хотя то, что человек себя просто назвал проукраинским
патриотом, еще ни о чем не говорит, собственно говоря, поскольку украинцы как
раз известны своим умением быстро перекрашиваться в нужные цвета... Сегодня
повесил один флажок, а завтра пришли другие - быстро повесил другой...
Вчерашние бравые комсомольские лидеры сегодня бодро машут флажками и увлеченно
поют про Украину, Бандеру, даже про Путина поют песенки... Что же вы будете петь завтра, интересно... 

\ifcmt
  tab_begin cols=3,no_fig,center
     pic https://pbs.twimg.com/media/Cd6YwEHUAAAUk79.jpg
		 pic https://s.44.ua/s/22/section/newsInText/upload/images/news/intext/000/054/450/7_6214feee4bd38.jpeg
		 pic https://sharij.net/wp-content/uploads/2019/12/photo_2019-12-19_13-13-46.jpg
  tab_end
\fi

С другой
стороны, говорят, пророссийский - и сразу как бы чувствуешь что-то нехорошее,
неправильное. Все, что носит оттенок или корень -рус-, это все плохо,
некачественно.  Украина - это априори хорошо, а вот Россия - это конечно плохо.
Хотя... если вспомнить, что Украина - это вообще говоря и есть Россия тоже,
поскольку Россия - это греческий вариант слова Русь, только с двумя буквами С,
то есть, Киевская Россия, Киевская Русь, Kiev Russia, то возникает когнитивный
диссонанс, поскольку получается, что Украина - это одновременно и хорошо, и
плохо. 

Ну что ж! У нас есть люди, которые топят за Украину, есть за Россию, есть за движение на Запад,
есть на Восток, а вот тех, кто прежде всего за Киев, а значит, за самое Сердце
Украины, за самый исток Украины, а значит, за Украину втройне, - за
удивительный, многоликий, неповторимый Город на Днепре, - не слышно совершенно,
не видно совсем в этом многоголосии. И здесь мы бы хотели в общем немножко
улучшить эту ситуацию, и постараться сформулировать свою, собственную, Киевскую
позицию по животрепешущим проблемам современной Украины. Что такое кремлевские
нарративы, все уже и так знают, а какие могли бы быть наши собственные,
Киевские нарративы?

\ifcmt
  tab_begin cols=2,no_fig,center

		pic https://scontent-lhr8-1.xx.fbcdn.net/v/t39.30808-6/274300563_256216460015591_6695519426888780594_n.jpg?_nc_cat=111&ccb=1-5&_nc_sid=5cd70e&_nc_ohc=Wo7vYQe7uW4AX_ICmC6&_nc_oc=AQlWGUWb-PbCysQ1xmXHcJNwmCqULxMp_8W9Y9IE9bDPLvK4-FEWqMDwHJi4-jltaRE&_nc_ht=scontent-lhr8-1.xx&oh=00_AT_xld6qA-3uWtk5ko1TPU-TBbElZ9Lyn5UeCi5330yuSg&oe=621BEAB6

		pic https://i2.paste.pics/c8cbbd49e226f77d37eca320b6448a75.png

  tab_end
\fi

\ii{moje.kremlevskie_narrativy.jazyk}
