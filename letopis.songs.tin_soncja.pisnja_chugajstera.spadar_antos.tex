% vim: keymap=russian-jcukenwin
%%beginhead 
 
%%file songs.tin_soncja.pisnja_chugajstera.spadar_antos
%%parent songs.tin_soncja.pisnja_chugajstera
 
%%endhead 

\subsubsection{Песни чугайстра - Спадар Антось}

\url{https://fantlab.ru/blogarticle36101}

Чугайстер — фантастичний образ української демонології; у фольклорі — не зовсім
виразний, постає як високий «лісовий чоловік», зодягнутий у білий одяг; нагадує
лісовика, полісуна; за іншими версіями, заклятий чаклунами чоловік, якому
«пороблено»; навіяний віруваннями в чародійство і перевертнів; найпоширенішим
цей образ був на Гуцульщині, де він нібито ловить лісових нявок і поїдає їх,
проте до людей ставиться доброзичливо...

Жайворонок В. В. «Знаки української етнокультури: Словник-довідник»

В 1996 году гомельская группа GODS TOWER выпустила альбом «The Turns», который
враз стал культовым на андеграундной сцене и прославил коллектив по всей
Европе. Отличительными чертами диска были его ярко выраженное языческое
настроение и гитара, что имитировала звучание беларуской дуды. И была на том
альбоме апокалиптическая песня «Twilight Sun» про ожидание скорого конца,
пропитанная вязким ощущением обречённости и сумасшествия. «Когда солнце
танцует, / закатываясь за пустынные холмы, / бабочки лениво машут крыльями. / Я
открываю безумие, / слышу далёкие звуки странных троп — / это значит, конец
солнечным денькам, / кто-то знает моё имя. / Вселенная разупорядоченна».

When the Sun is dancing
Rolling over Hills of Noone
Butterflies are lazy
Madness is what I discover
Distant sounds of strangeways
It means the sunny days are over
Someone knows of my name
Universe of disorder

And I don't pine for
If I go insane
And I don't pine for
If I were afraid
And I don't pine for
If you say hello
I'm only dreaming
I shouldn't care at all

Curtain disappears
Doomsday show is going down
Something in my ears
Music of my narrow home
Twilight Sun is actor
Really great eternal hero
Happy end of this world
The end of show is near

Broken hearts are bleeding
They feel sorry for this story
Noone can't stop heeding
Hailing to the actor's glory
Twilight Sun is dancing
Rolling over Hills of Noone
Warriors of wasteland
Singing «world is over»

There is a home I live alone
The storming waters in my pond
Unbeaten tracks and endless roads
Since the creation of the world
And I don't pine for if I die
And I don't pine for if I lie
No way to see the crown of skies
Because that fire in my eyes

\url{https://youtu.be/vJ4lUjhoUWg}

В 2005 году свет увидел двухдисковый трибьют GODS TOWER под названием «Варта
Вежы Багоў», содержавший две дюжины кавер-версий на песни коллектива от
различных исполнителей. И была на том диске «Пісня Чугайстра» киевской группы
ТІНЬ СОНЦЯ — творчески переосмысленная «Twilight Sun». Украинские музыканты
переписали текст, посвятив его Чернобыльской трагедии: в их версии
лесовик-чугайстер живёт один в глухих чащах Полесья, покинутых всеми и
пропитанных смертоносной радиацией, и ждёт возвращения людей, мало-помалу сходя
с ума. В 2007 году «Пісня Чугайстра» вошла в номерной альбом ТІНЬ СОНЦЯ
«Полум’яна Рута».

Интересно, что изначально чугайстер встречался только в фольклоре Гуцульщины и
обитал в Карпатах, а никак не на берегах Припяти. Помните, как в «Тінях забутих
предків» Коцюбинского Иван танцевал с чугайстром?

\url{https://youtu.be/IjhBE69wSGg}

В том же 2007 году на «Полум’яну Руту» в качестве бонус-трека вошла
беларускоязычная «Песьня Чугайстра», несказанно порадовав тем самым беларуских
поклонников группы. Так персонаж украинской демонологии запел по-беларуски.

В интернете я нашёл перевод этой песни на русский язык, автор мне неизвестен:
