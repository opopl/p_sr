% vim: keymap=russian-jcukenwin
%%beginhead 
 
%%file 28_01_2022.yz.jazykovedjma.1.o_kotah_i_kitah
%%parent 28_01_2022
 
%%url https://zen.yandex.ru/media/feel_fuck_online/o-kotah-i-kitah-61f3a308c0e80846a09276fc
 
%%author_id yz.jazykovedjma
%%date 
 
%%tags bulgakov_mihail,filologia,jazyk,mova
%%title О котах и китах
 
%%endhead 
 
\subsection{О котах и китах}
\label{sec:28_01_2022.yz.jazykovedjma.1.o_kotah_i_kitah}
 
\Purl{https://zen.yandex.ru/media/feel_fuck_online/o-kotah-i-kitah-61f3a308c0e80846a09276fc}
\ifcmt
 author_begin
   author_id yz.jazykovedjma
 author_end
\fi

\begin{zznagolos}
\enquote{— Сволочь он, — с ненавистью продолжал Турбин, — ведь он же сам не говорит на
этом языке! А? Я позавчера спрашиваю этого каналью, доктора Курицького, он,
извольте ли видеть, разучился говорить по-русски с ноября прошлого года. Был
Курицкий, а стал Курицький... Так вот спрашиваю: как по-украински «кот»? Он
отвечает «кит». Спрашиваю: «А как кит?» А он остановился, вытаращил глаза и
молчит. И теперь не кланяется.}	
\end{zznagolos}

Этим отрывком из \enquote{Белой гвардии} Булгакова я хочу, во-первых, с прискорбием
показать, что спустя уже целый век подобные конфликты не теряют актуальности, а
во-вторых, эффектно начать рассказ о данном конкретном фонетическом различии в
наших языках.

\ii{28_01_2022.yz.jazykovedjma.1.o_kotah_i_kitah.pic.1}

Начнём с того, что \enquote{кот и кiт} - не единственный такой случай. Как и любое
фонетическое соответствие двух родственных языков, соответствие русского \enquote{о}
украинскому \enquote{i} является регулярным:

\begin{itemize}
  \item сок (рус.) - сiк (укр.)
  \item брод (рус.) - брiд (укр.)
  \item конь (рус.) - кiнь (укр.)
\end{itemize}

Однако смотрите, есть и вот такое:

\begin{itemize}
  \item сон (рус.) - сон (укр.)
  \item мох (рус.) - мох (укр.)
\end{itemize}

В чём же тут закономерность? В русском языке в обоих случаях \enquote{о}, а в
украинском - то \enquote{о}, то \enquote{i}.

Всё дело в том, что в древнерусском и праславянском языках было два разных
звука, которые соответствуют \enquote{о} в современном русском.

\begin{itemize}
  \item просто \enquote{о}
  \item ер - \enquote{ъ}, один из упавших в XII веке редуцированных. Вот только после падения он и превратился в \enquote{о} (в тех местах, где был под ударением)
\end{itemize}

Например, слово \enquote{ръвъ} превратилось в \enquote{ров}. Однако память о
том, что раньше в нём был редуцированный сохраняется в его беглости в других
формах (рвы, рва, рвов...).

Там, где и был изначально \enquote{о}, беглости нет. Например, \enquote{дом} -
раньше выглядел как \enquote{домъ}, и в других формах этот гласный не выпадает
(дома, домов, дому...).

Теперь мы можем проверить слова, которые мы сравнивали с украинским:

\enquote{Кот - кота, коты...}, \enquote{конь - коня, кони...}, \enquote{брод - брода, броды...}, \enquote{сок -
сока, соки...} - значит здесь всегда был \enquote{о}, и именно он в украинском перешёл
в \enquote{i}

Но:

\enquote{Сон - сна, сны...}, \enquote{мох - мха, мхи...} - значит здесь раньше
был ер, и он в украинском перешёл в \enquote{о}, как и в русском. А вот в
чешском тут будет \enquote{е}: \enquote{sen}, \enquote{mech}. Ну это так,
просто любопытный факт, про то, что закономерности будут везде, но разные.

Мы уже определили, что \enquote{о} в украинском перешёл в \enquote{i}, а
\enquote{ъ} в \enquote{о}. Теперь давайте разберёмся, почему.

Тот \enquote{о}, что изначально был редуцированным, то есть кратким и слабым, таковым и
остался, а тот, что сразу был \enquote{о}, по контрасту оказался более долгим. То есть
какое-то время существовало два звука \enquote{о}, отличавшихся по долготе.

Однако различие по долготе не самое удобное (вспомните, как неохота
заморачиваться в английском на разницу \enquote{sheep - ship} или \enquote{beach -
сами-знаете-что}).

Поскольку в долготе гласного довольно просто ошибиться, некоторые диалекты
и/или языки отказывались от этого различия в пользу различия качественного, то
есть по характеру звука.

Поэтому будущий украинский язык, а в то время ещё западный диалект
древнерусского, постепенно преобразовал долгий \enquote{о} в закрытый, а
краткий - в открытый.

Чтобы понять разницу, рекомендую следующее. Вытяните немного губы, как будто
готовясь сказать \enquote{у}, но скажите \enquote{о} - вот это закрытый
\enquote{о}. А теперь откройте рот, как будто готовясь сказать \enquote{а}, но
скажите \enquote{о} - а вот это открытый \enquote{о}. 

Так что долгий \enquote{о} в \enquote{коте} стал закрытым. И в \enquote{коне}.
А во \enquote{сне} - открытым.

Но если вы очень долго в каком-то слове произносите \enquote{о}, делая рот как
будто для \enquote{у}, то этот \enquote{о} рано или поздно и перейдёт в этот
\enquote{у}. Что и произошло.  \enquote{Кот} стал звучать как \enquote{кут}.

Но эта ниша уже была занята словом \enquote{кут}, которое значит
\enquote{угол}.

Пришлось украинскому котяре гулять дальше.

Звук \enquote{у} стал переходить в более тонкий, как [ю] в слове \enquote{пюре}. Но тут уж
совсем стало неудобно говорить.

Попробуйте произнести \enquote{кют}, и почувствуйте, как напрягаются все мышцы от губ
аж до шеи.

И тогда звук \enquote{ю} стал переходить в звук \enquote{и}. Произнесите теперь \enquote{кит} и
почувствуйте, что теперь напрягается только основание языка. Лень - двигатель
прогресса!

И фонетических изменений тоже.
