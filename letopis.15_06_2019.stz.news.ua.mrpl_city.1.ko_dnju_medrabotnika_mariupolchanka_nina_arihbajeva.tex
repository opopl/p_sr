% vim: keymap=russian-jcukenwin
%%beginhead 
 
%%file 15_06_2019.stz.news.ua.mrpl_city.1.ko_dnju_medrabotnika_mariupolchanka_nina_arihbajeva
%%parent 15_06_2019
 
%%url https://mrpl.city/blogs/view/ko-dnyu-medrabotnika-mariupolchanka-nina-arihbaeva
 
%%author_id burov_sergij.mariupol,news.ua.mrpl_city
%%date 
 
%%tags 
%%title Ко Дню медработника: мариупольчанка Нина Арихбаева
 
%%endhead 
 
\subsection{Ко Дню медработника: мариупольчанка Нина А\hyp{}рихбаева}
\label{sec:15_06_2019.stz.news.ua.mrpl_city.1.ko_dnju_medrabotnika_mariupolchanka_nina_arihbajeva}
 
\Purl{https://mrpl.city/blogs/view/ko-dnyu-medrabotnika-mariupolchanka-nina-arihbaeva}
\ifcmt
 author_begin
   author_id burov_sergij.mariupol,news.ua.mrpl_city
 author_end
\fi

\ii{15_06_2019.stz.news.ua.mrpl_city.1.ko_dnju_medrabotnika_mariupolchanka_nina_arihbajeva.pic.1}

У врача Арихбаевой была благороднейшая медицинская профессия. Она помогала
детям появиться на свет Божий и заботилась о здоровье их матерей. Нина
Васильевна родилась 7 ноября 1918 года в селе Платоновка под Волновахой, а вот
родители ее были коренными мариупольцами. Отец – Василий Павлович – после
окончания Мариупольского технического училища поступил в Варшавский медицинский
институт, стал ветеринарным врачом. Мать – Мария Федоровна, урожденная
Шевченко, окончившая с серебряной медалью Мариупольскую Мариинскую женскую
гимназию, - некоторое время была классной дамой в этой же гимназии. Уже в
зрелом возрасте она заочно получила высшее образование в Луганском
педагогическом институте и на протяжении пятидесяти лет преподавала в школах
математику.

Детство и юность Нины прошло в Сталино, так тогда назывался Донецк. В этом
городе она с отличием окончила среднюю школу, оттуда поехала в Москву, где
поступила в знаменитый Московский институт истории, философии и литературы,
который был сформирован в ходе реорганизации гуманитарных факультетов МГУ. К
слову скажем, что в этом вузе в разное время учились А. Т. Твардовский, А. И.
Солженицын, К. М. Симонов и другие, не менее известные люди. А вот Нине
Арихбаевой довелось быть студенткой этого учебного заведения всего полгода. Она
успешно сдала все экзамены за первый семестр и ... тяжело заболела.

Нина вернулась в Сталино, к родителям, и тут семью постигло тяжелое горе.
Трагически погиб ее младший брат Георгий – одаренный и всеми любимый юноша.
Чтобы хоть как-то отвлечься от тягостного душевного состояния, она стала
посещать лекции в медицинском институте, где учились ее школьные друзья. Дошло
дело до сессии. Нина сдала все экзамены на \enquote{отлично}. Ее уговорили остаться в
мединституте. Постепенно медицина увлекла, она начала заниматься научной
работой, изучала на животных влияние канцерогенных веществ на возникновение
рака. На пятом курсе на студенческой конференции за эти исследования она
получила первую премию, разделив ее с Ростиславом Богославским, который стал
впоследствии доктором медицинских наук и профессором.

Впереди был еще год учебы, но началась Великая Отечественная война.
Пятикурсников срочно вызвали в институт, там им ускоренными темпами были
прочитаны лекции за шестой курс, а уже ранней осенью 1941 года у студентов были
приняты экзамены и выданы дипломы врачей. Большинство из них тут же были
мобилизованы в действующую армию. Нину в армию не взяли из-за болезни. 26
октября гитлеровцы захватили Сталино, началась оккупация. Оставшихся в городе
медработников заставили выйти на работу. Молодой врач Арихбаева работала
терапевтом в гражданской поликлинике.

Вскоре после освобождения Донбасса от немецко-фашистских оккупантов Нина
Васильевна познакомилась с Федором Романовичем Волковым, тяжело раненным в бою
под Запорожьем. Они создали семью. В 1944 году переехали в Мариуполь. В 1948
году Н. В. Арихбаева приступила к работе цехового врача медпункта, а затем
участкового терапевта поликлиники на Левом берегу. Работать было очень тяжело,
из города на Левый берег ходил только рабочий поезд, участок Нины Васильевны
включал в себя несколько поселков, удаленных друг от друга на большие
расстояния, да и работа терапевта казалась мало привлекательной. Тогда-то она и
присмотрелась к акушерству и гинекологии. Ей пришлось приложить немалые усилия,
чтобы добиться направления на шестимесячные курсы переквалификации в Харьков.
Нина Васильевна вернулась оттуда акушером-гинекологом и начала работать в
родильном отделении рядовым врачом, а затем, когда заведующая отделением ушла
на пенсию, заняла ее место.

\textbf{Читайте также:} 

\href{https://mrpl.city/news/view/teper-ya-slyshu-malenkomu-nikite-iz-mariupolya-podarili-sluhovye-apparaty-kak-poluchit-pomoshh}{%
\enquote{Теперь я слышу}: маленькому Никите из Мариуполя подарили слуховые аппараты. Как получить помощь?, Роман Катріч, mrpl.city, 07.06.2019}

В 1953 году Нина Васильевна понесла невосполнимую утрату. После тяжелой
болезни, вызванной фронтовым ранением, ушел из жизни ее муж, Федор Романович. В
том же году в Орджоникидзевском районе был открыт первый в нашем городе
родильный дом. Его главным врачом была назначена Арихбаева. Ей пришлось
преодолеть много трудностей, приложить невероятные усилия, чтобы это
медицинское учреждение стало образцовым. Дочь Нины Васильевны - Татьяна
Федоровна, тоже врач, рассказывала: \emph{\enquote{На одном из совещаний, где мама
докладывала, как построена работа в этом родильном доме, один из присутствующих
высокопоставленных чиновников сказал: \enquote{Вы все неправду говорите, я приеду и все
сам проверю}. И, действительно, была организована комиссия в составе маститых
специалистов, которые проверяли в роддоме все - от дворового туалета до
операционных кроватей. И все пришли к выводу, что то, о чем говорила Нина
Васильевна на совещании, - правда. По результатам проверки родильному дому было
присвоено звание школы передового опыта, куда направляли врачей для проведения
занятий}}. Все лучшее, что было внедрено Ниной Васильевной и ее коллегами за
годы работы родильного дома, она изложила в книге, которая была удостоена
серебряной медали ВДНХ.

В 1979 году рядом с родильным домом был построен типовой акушерский корпус.
Строительство здания шло очень тяжело, несмотря на то, что должностные лица
высокого уровня практически ежедневно проводили оперативные совещания, здание
было сдано в эксплуатацию с большими недоделками. Нина Васильевна долго
сопротивлялась и не подписывала акт приемки, но, в конце концов, ее заставили
сделать это. Ей понадобилось много времени и сил, чтобы промахи строителей были
устранены.

Вот что довелось однажды услышать от Татьяны Федоровны о том, как относилась
Нина Васильевна к своей работе: 

\begin{quote}
\emph{\enquote{Мама была человеком самоотверженным. Она не
ложилась спать до тех пор, пока не позвонит в роддом дежурному врачу. И это
каждый день, каждый вечер. Куда бы она ни уходила: в кино, театр или в гости,
то всегда звонила к себе на работу и говорила, где ее можно найти. В доме
телефон был, как в пожарной команде или милиции. Ее могли вызвать в любое время
суток. То срочно нужна была кровь для переливания, то требовался вызов
санавиации, то тяжелые роды с осложнениями, где нужен был ее опыт, ее участие.
Она жила в постоянном напряжении. Мало того, что были врачебные всякие нюансы,
так еще одолевали всякие хозяйственные неприятности. Однажды среди зимы слесари
перекрыли задвижку на системе отопления. В палатах холодно, а там новорожденные
дети, их мамы. Было воскресенье, она помчалась к знакомому сантехнику.
Уговорила его поехать на Левый берег, отопление было восстановлено. Нина
Васильевна не просто присутствовала на обходах. Она смотрела, как лечат. Порой
она советовала молодым врачам, да и со своими коллегами могла поспорить}}. 
\end{quote}

Удивительно, но при огромной загруженности главного врача, Н. В. Арихбаева
находила время для организации в роддоме университета культуры и библиотеки,
для проведения различных мероприятий, например, прием молодых специалистов в
семью медработников. Она выполняла множество общественных поручений за
пределами больницы. Участвовала в деятельности городского женского и
опекунского советов, исполняла депутатские обязанности. Ее избирали в горсовет
пяти созывов. А еще она любила путешествовать, это были и поездки на
научно-практические конференции врачей, и по линии женсовета, а иногда и по
туристским путевкам. При этом старалась прочесть новинку литературы, попасть на
концерт симфонической музыки. И вместе с тем, Нина Васильевна оставалась
заботливой матерью и бабушкой...

Самоотверженный труд и многогранная общественная деятельность Нины Васильевны
Арихбаевой были отмечены многочисленными наградами и поощрениями. Ее имя,
несомненно, останется в истории медицины Мариуполя, как пример беззаветного
служения здоровью матери и ребенка.

\textbf{Читайте также:} 

\href{https://mrpl.city/news/view/devyateryh-ranenyh-azovtsev-mariupolskogo-polka-e-vakuirovali-voennym-samoletom-foto-plusvideo}{%
Девятерых раненых \enquote{азовцев} мариупольского полка эвакуировали военным самолетом, Анастасія Папуш, mrpl.city, 13.06.2019}
