% vim: keymap=russian-jcukenwin
%%beginhead 
 
%%file 30_06_2018.stz.news.ua.mrpl_city.1.ijul_zhara_gastroli
%%parent 30_06_2018
 
%%url https://mrpl.city/blogs/view/iyul-zhara-gastroli
 
%%author_id burov_sergij.mariupol,news.ua.mrpl_city
%%date 
 
%%tags 
%%title Июль, жара, гастроли
 
%%endhead 
 
\subsection{Июль, жара, гастроли}
\label{sec:30_06_2018.stz.news.ua.mrpl_city.1.ijul_zhara_gastroli}
 
\Purl{https://mrpl.city/blogs/view/iyul-zhara-gastroli}
\ifcmt
 author_begin
   author_id burov_sergij.mariupol,news.ua.mrpl_city
 author_end
\fi

Издавна мариупольцы были заядлыми театралами и меломанами. Не случайно же в
нашем городе возник один из старейших театров на Юге Российской империи.
Пожалуй, самим старым из дошедших до нас свидетельством о гастролях в нашем
городе является анонс из газеты \textbf{\em\enquote{Мариупольский справочный листок}} за 11 августа
1900 г. Вот его содержание: \emph{\enquote{Александровский парк, пятница 11 августа 1900 г.
даны будут концертные вечера симфонического оркестра, состоящего из 30 человек,
под управлением дирижера М. Я. Свердлова}}.

Папа рассказывал, что в конце 20-х годов прошлого века ему довелось слушать в
Зимнем театре выдающихся скрипачей своего времени - Михаила Эрденко, Мирона
Полякина и совсем юного Давида Ойстраха, мама же вспоминала спектакли
навещавшего наш город Одесского украинского театра — Госдрамы, где искрился
юмор и звучали украинские мелодии.

Выписав наугад из подшивок \enquote{Приазовского рабочего} анонсы и отчеты о гастролях,
получил такую вот картину:

\textbf{1955 г. 1 июля} в Летнем театре ПКиО (Городского сада) начались гастроли
Днепропетровского русского драматического театра им. М. Горького.

\textbf{17 июля.} В День физкультурника на летней эстраде ПКиО состоялся концерт
уроженца Мариуполя, народного артиста СССР, лауреата Сталинской премии Михаила
Степановича Гришко. Он пел в сопровождении симфонического оркестра Сталинской
областной филармонии. Им были исполнены арии из опер \enquote{Князь Игорь} А. Бородина
и \enquote{Богдан Хмельницкий} К. Данкевича, \enquote{Алеко} С. Рахманинова, украинские
народные песни и песни советских композиторов.

\textbf{23–31 июля} в Летнем театре ПКиО после семнадцатилетнего перерыва состоялись
Гастроли Ленинградского государственного большого драматического театра им. М.
Горького. Зрителям были показаны пьесы Г. Филдинга \enquote{Разоблаченный чудотворец},
В. Гюго \enquote{Рюи Блаз}, Лопе де Вега \enquote{Девушка с кувшином}, Г. Гауптмана \enquote{Перед
заходом солнца}, А. Арбузова \enquote{Домик на окраине} и др.. В спектаклях были заняты
Виталий Полицеймако, Валентина Кибардина, Александр Арди, Владислав Стржельчик,
Нина Ольхина, Ефим Копелян. Все спектакли прошли с аншлагом.

\ii{30_06_2018.stz.news.ua.mrpl_city.1.ijul_zhara_gastroli.pic.1}

\textbf{1958 г. 2 июля} в в Летнем театре ПКиО начались гастроли Крымского
драматического театра.

\textbf{4 июля} - начало гастролей симфонического оркестра Сталинской областной
филармонии, оркестр играл до \textbf{3 августа}.

\textbf{С 5 по 15 июля} проходили гастроли шахтерского ансамбля \enquote{Молодая гвардия}.

\textbf{16 июля} - начало гастролей Сталинского областного театра оперы и балета,
главный режиссер театра народный артист УССР Александр Здиховский. На сцене
летнего театра ПКиО были представлены оперы \enquote{Бал-маскарад}, \enquote{Травиата},
\enquote{Риголетто}, \enquote{Трубадур} Дж. Верди, \enquote{Запорожец за Дунаем} С.
Гулака-Артемовского, \enquote{Кармен} Ж. Бизе, \enquote{Чио-Чио-Сан} Дж. Пуччини, а также
балеты \enquote{Лебединое озеро} П. Чайковского и \enquote{Большой вальс} на музыку И. Штрауса.
Среди молодых исполнителей выделялся будущий народный артист СССР Юрий Гуляев.

\textbf{1961 г. 15 июля} начались гастроли цирка-шапито. Его установили на
бывшей базарной площади, теперь на этом месте высится Дом военно-технической
учебы бывшего ДОСААФ. В цирковой программе участвовала Тереза Дурова с группой
дрессированных животных.

\textbf{1965 г. С 4 июля} симфонический оркестр Донецкой областной филармонии в
составе 65 музыкантов дал 20 концертов, оркестром дирижировали Гаспар Торикян и
Павел Кравченко.

\textbf{С 12 июля по 4 августа} - гастроли Днепропетровского русского
драматического театра им. М. Горького

\textbf{1967 г. 8 июля} начались гастроли симфонического оркестра Донецкой
областной филармонии.

\textbf{10 июля} на стадионе \enquote{Металлург} состоялся большой концерт с участием популярных
артистов: певцов Гюлли Чохели, Валерия Ободзинского, Иосифа Кобзона, Ларисы
Мондрус, Владимира Макарова, пародиста Сергея Чистякова, танцора Владимира
Шубарина. Вел концерт знаменитый конферансье Эмиль Радов.

27 июля на сцене Дворца культуры им. Карла Маркса состоялся большой вечер
балета. Перед ильичевскими металлургами выступили артисты Государственного
Академического Большого Театра СССР. В программе были сцены и отрывки из
классических и советских балетов.

\textbf{1973 г. 2 – 29 июля} в помещении Драмтеатра прошли гастроли Ставропольского
краевого театра музыкальной комедии (г. Пятигорск).

\textbf{1975 г. 1 – 31 июля.} Состоялись гастроли Ленинградского театра драмы и комедии
в нашем драматическом театре. Репертуар состоял из следующих спектаклей по
пьесам С. Ясных \enquote{Любимая}, М. Мэйо, М. Эннекена. \enquote{Моя жена - лгунья!}, А.
Кристи \enquote{Мышеловка}, А. Вампилова \enquote{Прощание в июне}, Т. Уильямса. \enquote{Трамвай
\enquote{Желание}}, А. Толстого \enquote{Касатка} и др.

\textbf{12 июля} - начало гастролей чехословацкого цирка \enquote{Прага}, в программе
дрессированные слоны, воздушные акробаты под куполом цирка.

\textbf{1976 г. 22 июля.} Во Дворце культуры \enquote{Искра} артисты Государственного
академического Малого театра СССР Роман Филиппов, Ольга Чуваева, Галина Демина,
Виталий Коняев, Валерий Носик и их коллеги представили на суд зрителей
спектакль по пьесе Бранислава Нушича \enquote{Доктор философии} в постановке народного
артиста СССР Константина Зубова.

\textbf{1978 г. 3 июля.} На сцене Дворца культуры завода \enquote{Азовсталь} выступали артисты
Московского театра им. Ленинского Комсомола. Был показан спектакль по пьесе Б.
Рацер и В. Константинова \enquote{Вечером после работы}.

\textbf{5 июля} начались гастроли Астраханского областного театра драмы им. С. М. Кирова

\textbf{1980 г. 3–31 июля} - гастроли Гомельского русского драматического театра.

\textbf{1981 г. 2–31} - июля гастроли Липецкого областного театра драмы.

\textbf{1982 г. 2–31} - июля гастроли Могилевского областного драматического театра.

\textbf{1987 г. 30 июля.} В помещении драматического театра состоялся концерт
популярных артистов театра и кино Александра Абдулова, Олега Янковского и
Леонида Ярмольника.

Гастролеры, вероятно, любили навещать наш город летом, не только зная о
благодарной понимающей публике, но и потому, что можно вдоволь накупаться в
теплых водах Азовского моря.
