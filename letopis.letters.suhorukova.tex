% vim: keymap=russian-jcukenwin
%%beginhead 
 
%%file letters.suhorukova
%%parent letters
 
%%url 
 
%%author_id 
%%date 
 
%%tags 
%%title 
 
%%endhead 

Ну так Вы же меня на фб забанили, удалили. Значит, пытаетесь заткнуть мне рот.
Я же то вижу )). И чем Вам Шевченко не понравился - я просто не пойму, ну
ладно... Если бы я умел обижаться, я бы сказал, что вы просто плюнули в душу,
смачно так. А поскольку я не умею обижаться, то в принципе, считайте, я этого
не писал... Ваша жизнь не изменится. Хм. Ну да. Совершенно верно. Будете
торчать в Германии, проливать слезы, вариться в этом всем дерьме... Насылать
депресуху на украинцев, ляляля, нас никто не понимает, мы особенные. мы такие
бесконечно особенные. Бесконечно особенные плачущие мариупольцы, которые бесконечное число раз 
все время жалуются, но у которых жопа не поднимется даже книжку свою собственную переиздать, а не то чтобы целый 
Город отстроить... ляляля... 
Мы так хотим вернуться, но при этом нам в общем то наплевать на всех, кто хочет нам по
настоящему помочь... Ваша Воля. В Мариуполь Вы больше никогда в своей жизни не
вернетесь с таким подходом... Понимаете. Мариуполю нужны сильные Люди, сильные
Духом люди, которые отстоят Город заново, и которые освободят Город, многие
ценой своих жизней... Вот как например те Люди, - Люди с большой буквы - Полк Азов я имею в виду -
которые душу и тело отдали за Ваш город (их фото в Историческом Музее сейчас на
четвертом этаже). Понимаете. Молодые парни... Целый зал в их фото. Они 
все погибли... а вы то живете... Дерьмовенько так живете...

%13:05:09 11-03-23
Доброе утро. Вы открывали и занимались сбором - Вы и пишите. Мое дело было
скинуть - я скинул.  И дело не в деньгах, а в доверии, и в помощи неравнодушных
людей, в которых вы просто плюнули, вот и все... Даже вот сейчас, можно было бы
сказать, "спасибо, что помогли". Нет, не сказали. И Вы думаете, с таким
скотским и жлобским отношением по жизни Вы чего-нибудь добьетесь. Я очень
сомневаюсь. Вы вот катите бочку на бойченко. Наверное, так оно и было, я не
знаю. Но по сути - Вы с Надей такая же как и бойченко. Каков поп - таков и
приход. Если Вам жаль даже открыть рот и сказать просто "спасибо, что помогли
пацану на ноут"... короче, х@й с вами

%13:26:17 11-03-23
Знаю я уже поболее вашего о Мариуполе. Это раз. Я пишу книжку сейчас о
Мариуполе. А два. Мариуполь - не Ваша личная собственность или вотчина, так
что, нечего тут распускать пальцы веером, дескать, я из Мариуполя, я такая
уникальная, руки прочь от моего Мариуполя...  Мариуполь же - это Достояние
Украины и всего Человечества, и до него есть дело всем. Буквально всем. Вот и
все.

%13:42:50 11-03-23
знаєте, я колись був у Єревані. Там взагалі нввіть краще з водою ніж у Києві...
тому що... у них прямо на вулицях усюди стоять такі собі міні-фонтанчики, з
яких можна пити воду. Неймовірно класно!

%14:40:59 11-03-23
%Натали Михайловская
%reply to: https://www.facebook.com/groups/1233789547361300/user/100018467570074/

добрый день. Я из Киева, я программист, вот как раз таким занимаюсь. Издать
альбом физически я вряд ли Вам помогу, это вопрос уже конкретного договора с
издательством, либо заказа на печать в какой то конкретной конторе, а вот с
электронной версией, которую можно будет потом распечатать в конторе или
разослать друзьям и по всему свету - нет проблем вообще. Я уже кучу постов
записал для себя про Мариуполь, и для меня не будет проблемы технической
составить электронный альбом с какими угодно фото Мариуполя, хоть на десять,
хоть на сто страниц. А насчет увидет весь мир - нужно размещать такие альбомы в
таких местах, где их никто не удалит, и все смогут увидеть, например на ресурсе
Машина Времени - Интернет Архив - https://archive.org/details/@kyiv_chronicler
(пока что там только копии некоторых постов про Мариуполь - и фотоальбомы в
постах других людей - в том числе)

спасибо, без проблем. Моя здесь роль только в технических навыках, оформлении и
т.д. В принципе, это может делать любой человек при желании, тут нет никаких
особых тайн. Технология, которую я использую, называется LaTeX. И... Просто я
уже набил на этом руку, и поэтому делаю такие штуки очень быстро. А примеры
моей работы уже выложены по ссылке выше (конечно, это не единственный вариант
оформления, Латех позволяет делать оформление какой угодно степени сложности).
А насчет фото Виктора Дедова. (1) на самом деле фото Виктора не принадлежат
лично Наталии, эти фотографии абсолютно бесценны и принадлежат Городу
Мариуполю, мариупольцам и вообще всей Украине, при всем уважении и сочувствии к
Наталии, Виктору и ее личной трагедии, как и трагедии всех вас. Потому что...
эти фото не лично-семейные, это фото прекрасного Города, в создание и
построение которого вложило сил и души огромное число людей в разные времена.
Это моя личная позиция насчет авторских прав вообще любых фото Города Мариуполя
(2) лично у меня с Дедовой в последнее время возникли некоторые разногласия, в
общем, комментировать не буду здесь, поэтому, возможно, если Вы будете
обсуждать этот вопрос с ней, она Вам может наговорить всяких разных вещей
насчет меня... но, здесь вопрос не в каких то личных разногласиях, а в том,
чтобы память о Мариуполе, как мирном, так и блокадном, стала доступной как
можно большему числу людей, да, вопрос состоит именно об этом, а так же о
будущем Мариуполя, о возрождении Мариуполя... Это необычайно нужные вопросы
глобальной важности для мариупольцев и вообще для всей Украины, и я буду
счастлив помочь Вам в этом с техническо-информационной стороны 
