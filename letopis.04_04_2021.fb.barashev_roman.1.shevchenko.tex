% vim: keymap=russian-jcukenwin
%%beginhead 
 
%%file 04_04_2021.fb.barashev_roman.1.shevchenko
%%parent 04_04_2021
 
%%url https://www.facebook.com/roman.barashev/posts/2288987997900333
 
%%author 
%%author_id 
%%author_url 
 
%%tags 
%%title 
 
%%endhead 

\subsection{Всё дело – в дефисе. Вот в этой короткой черточке между российско-украинским}
\url{https://www.facebook.com/roman.barashev/posts/2288987997900333}

Экий гнев и репрессивный синдром вызвало у правителей-министров словосочетание
«российско-украинский» по отношению к Кобзарю. Сказали: не позволим
приватизировать. И по последним объектам, позволяющим себе этакую вольность,
как поддержка культурных связей, врезать. Из всех стволов политической цензуры! 

А если бы украинско-российским Тараса Шевченко назвали бы – может, не столь
жесткой была б реакция. Или столь же?

Но всё дело не в перемене «слагаемых», конечно, и не в большой любви к
народному поэту.

Всё дело – в дефисе.

Вот в этой короткой черточке между российско-украинским.

Черточку, в которой осталось так мало. И так много.

В этой черточке – и слабая надежда разумных людей, что не все еще потеряно, и
сильная вера, что враждебность недолговечна.

В ней – последние культурные связи родных по крови и родных душей людей, в ней
– то доброе и вечное, когда знаешь, что за совместное чтение Пушкина или
Шевченко нам, виноватым без вины, и нашим детям ничего не будет. В нем, этом
дефисике – и кровь моя, одна – русская отцовская, вторая – украинская
материнская, и никакие снаряды не перечеркнут во мне этот дефис.

В этом дефисе – миллионы российско-украинских судеб живых и почивших людей,
только из фамилий известных деятелей и гениев можно солидный
украинско-российский справочник составить.

Но есть, конечно, и те, на кого этот дефис действует, как ладан на черта. Ибо
они полагают, что время писать иные дефисы. К примеру, такой "2016-2021" и ему
подобные.

