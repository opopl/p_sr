% vim: keymap=russian-jcukenwin
%%beginhead 
 
%%file 30_10_2019.fb.makarevich_ekaterina.1.ulybka_iskrennjaja
%%parent 30_10_2019
 
%%url https://www.facebook.com/ekmakarevich/posts/2725701187494333
 
%%author_id makarevich_ekaterina
%%date 
 
%%tags chelovek,dusha,metro,psihologia,ulybka
%%title Иногда достаточно искренней улыбки
 
%%endhead 
 
\subsection{Иногда достаточно искренней улыбки}
\label{sec:30_10_2019.fb.makarevich_ekaterina.1.ulybka_iskrennjaja}
 
\Purl{https://www.facebook.com/ekmakarevich/posts/2725701187494333}
\ifcmt
 author_begin
   author_id makarevich_ekaterina
 author_end
\fi

Еду вчера в метро. Сижу, читаю. В вагон заходит большой такой парень, которого
под руку держит женщина. Парень крутит головой и все пытается вырваться. Ну
мало ли, думаю я, и дальше читаю книжку про силу момента сейчас и о том, как
вырваться из-под гнета интеллектуальных построений, которые формируют наши
иллюзорные представления о реальности. 

В этот момент рядом освобождаются два места. Парочка движется в мою сторону. Я
поднимаю глаза и встречаюсь взглядом с парнем. Он смотрит на меня с такой
детской непосредственностью и прямо в глаза, что я невольно улыбаюсь. Он
расцветает. 

И только тогда я понимаю, что этот парень, как это правильно сказать, с ДЦП,
или с ментальными нарушениями. Я к таким людям отношусь положительно, поскольку
однажды для себя поняла, что они просто чувствуют иначе, чем мы. Гораздо
сильнее. 

В общем, он садится рядом со мной. И тянет палец к моему телефону. Женщина его
одергивает. Он успокаивается. 

Едем дальше. Я с интересом читаю, потому что как раз в книжке описываются
рекомендации, как научиться взгляду «быть здесь и сейчас». Парень периодически
ко мне поворачивается, смотрит ко мне в телефон. Но я стараюсь на всякий случай
не реагировать. 

Рядом со мной, с другой стороны, освобождается место. У меня мелькает мысль, -
может, передвинуться. Но, - думаю я, - мало ли как это будет воспринято этим
парнем, да и какой пример я подам другим людям, показав, что от таких людей
надо пересаживаться. Я остаюсь на месте. Рядом место пустует. 

И тут парень ко мне поворачивается. Молча смотрит на меня. А в этот момент в
книге как раз описывается сам взгляд «сейчас», и я, как обычно, когда читаю
книгу, мысленно примеряю написанное на себя, то есть как будто включаю этот
взгляд в себе. 

Поднимаю глаза и во мне возникает знакомое ощущение, словно ты смотришь на все
другим взглядом, свежим, и мир вокруг внезапно расцветает, предстает
удивительным, и люди тоже вдруг вокруг оживают, вернее, под всеми масками,
проблемами и думами, вдруг видишь настоящие лики людей, их чистоту, такую
трепетную и нежную, что хочется улыбаться. 

Уж не знаю, как этот парень почувствовал во мне изменения, но он повернулся ко
мне и очень нежно, будто боясь уронить звуком хрупкую вазу, произнес «мя-мя». 

Я не знаю, что означали эти звуки, да это было и неважно. Важно было не «что»,
а «как» они были произнесены. В них было столько детской любви...а я же ведь
была как раз «в моменте».

И я улыбнулась ему снова. 

Видели бы вы его лицо. Представьте лицо ребенка, который еще не научился
говорить, но что-то хочет и не может объяснить. И его лицо, когда взрослый дает
ему это. Вот примерно такое же лицо было в этот момент у парня - чистая, не
замутненная радость. Ему просто нужна была улыбка...

Поезд остановился. Женщина потащила его к выходу налево, а я пошла в правую
дверь. Я слышала, как он еще пару раз сказал: «мя-мя»...

А я уже стояла у дверей. Рядом стояли люди, обычные взрослые, и смотрели на
меня серьезным, изучающим взглядом, видимо, пытаясь понять происходящие с нами
метаморфозы. 

Иногда достаточно искренней улыбки, - подумала я. И попробовала взглянуть на
этих сурьезных людей с тем же открытым взглядом, что посмотрел вначале на меня
тот парень. Но они не выдержали взгляд и отвернулись...

Но не все сразу. Нужно много искренних улыбок, чтобы даже человек, не привыкший
улыбаться незнакомцам, однажды улыбнулся в ответ. 

Просто так. 

Может, в этом и есть сила момента сейчас  @igg{fbicon.smile} 
