% vim: keymap=russian-jcukenwin
%%beginhead 
 
%%file 17_11_2020.sites.ru.zen_yandex.yz.putj_i_tekst.1.vladimir_vtoraja_stolica_rusi
%%parent 17_11_2020
 
%%url https://zen.yandex.ru/media/maximus101/kogda-byl-osnovan-vladimir-vtoraia-stolica-rusi-5fb29e9a7eb1fe4ba089160f
 
%%author 
%%author_id yz.putj_i_tekst
%%author_url 
 
%%tags 
%%title Когда был основан Владимир - вторая столица Руси?
 
%%endhead 
 
\subsubsection{Когда был основан Владимир - вторая столица Руси?}
\label{sec:17_11_2020.sites.ru.zen_yandex.yz.putj_i_tekst.1.vladimir_vtoraja_stolica_rusi}
\Purl{https://zen.yandex.ru/media/maximus101/kogda-byl-osnovan-vladimir-vtoraia-stolica-rusi-5fb29e9a7eb1fe4ba089160f}
\ifcmt
	author_begin
   author_id yz.putj_i_tekst
	author_end
\fi

17 ноября 2020

Владимир и сейчас считается столицей, только туристический, это наверно
самый посещаемый город на Золотом кольце. Но при такой туристической
популярности город, как это ни странно, очень мало исследован. До сих пор
нет устоявшегося мнения ни о месте его древнейшей части, ни даже о дате
его основания. Во многом, это связано с его очень короткой "столичной"
историей - несмотря на то, что Владимир обустраивался как"второй Киев",
расцвет города был недолгим. В 12 веке он начался, а в 14 в. уже
закончился, столичный статус перешел к Москве. После чего он стал простым
провинциальным городком. Поэтому во Владимире осталось мало свидетельств
его былого расцвета. Археологические слои города не велики и не очень
богаты на находки.

\ifcmt
pic https://avatars.mds.yandex.net/get-zen_doc/1567436/pub_5fb29e9a7eb1fe4ba089160f_5fb29e9f9bb3e623748f60f3/scale_1200
caption Вид Владимира со стороны реки Клязьмы.  Любой, кто бывал в Киеве, сразу заметит схожесть ландшафтов - во Владимире те же высокие холмы над речной гладью.
\fi

\subsubsection{Когда был основан Владимир - вторая столица Руси?}

Вокруг Владимира сохранилось много "киевских" топонимов - это реки Рпень
(Ирпень), Почайна и даже есть своя речка Лыбедь, ныне текущая в трубе вдоль
северной гряды владимирских холмов.

Безусловно, Владимир основали выходцы из южной Руси, но вот вопрос когда?  До
недавнего времени официальной точкой зрения считалось, что основание города -
это заслуга Владимира Мономаха, от его имени и произошло название.

Об этом говорится в древнейшей Новгородской первой летописи: 

\begin{leftbar}
  \begingroup
    \em\Large\bfseries\color{blue}
Сын Володимеров Мономах, правнук великого князя Владимира. Сии поставил град
Володимерь Залешьскый в Суждальской земле и осыпа его спом, и созда первую
церковь святаго Спаса за 50 лет до Богороднчина ставления. Потом приде из Киева
в Володимерь сын Мономашь Юрьи Долгая Рука и постави другую церковь, каменну,
святаго Георгиа, за 30 лет до Богородичина ставления
  \endgroup
\end{leftbar}

Причем, приводятся весьма точные даты возведения владимирских церквей.
Время становления Владимира здесь - 1108 г.

Когда был основан Владимир - вторая столица Руси?

Но в нынешние времена все чаще проталкивается мнение, что город основал
другой Владимир - Креститель Руси. Якобы русский каган Владимир Святой
основал город еще в 990 г. Это точка зрения базируется на данных из
поздних летописных сводов, где многократно повторяется, что заложил город
именно он.

Например, можно привести пример Львовской летописи - под 6616 (1108) г.
говорится: “Того же лета свершен бысть град Владимер Залешьский
Володимером Мономахом, и созда в нем церковь камену святаго Спаса, а
заложи его бе прежде Володимер Киевский”.

Надо признать, что киевский князь Владимир как основатель
Владимира-на-Клязьме упоминается в очень многих летописных сводах.
Объяснен этот феномен уже давно. Дело в том, что со времен владимирского
князя Андрея Боголюбского в хрониках велась сознательная идеологическая
пропаганда призванная удревнить возраст столицы северо-восточной Руси. Для
этого специально вставлялись в летописи упоминания про князя Владимира
Святославича - Святого крестителя, эта пропаганда должна была поднять
статус города как второй столицы Руси.

\ifcmt
  pic https://avatars.mds.yandex.net/get-zen_doc/3397137/pub_5fb29e9a7eb1fe4ba089160f_5fb2bc35268198734d253dbc/scale_1200
  width 0.6
\fi

\subsubsection{Когда был основан Владимир - вторая столица Руси?}

Столь древний возраст Владимира не подтверждается археологией, на данный
момент нет никаких доказательств существования города Владимира Святого.
Но тут нужно признать, что и от времен Владимира Мономаха свидетельств
тоже почти нет. Неизвестно даже какой район старого города считать
древнейшим.

До сих пор общепризнанной считается версия основания города выдвинутая
известным исследователем Н. Н. Ворониным. Древний Владимир, как известно,
располагался на высоком плато в междуречье рек Клязмы и Лыбеди. Город был
вытянут с запада на восток и занимал ряд отдельных холмов. Н.Н.Воронин
считал, что древнейший город появился на среднем самом высоком холме, он
называл его городом Владимира Мономаха, западнее лежал Новый город,
основанный Андреем Боголюбским, а на востоке, на узкой полосе холмистой
гряды находился Ветчаный город, возможно тоже созданный Боголюбским.

В музее г. Владимира можно увидеть макет старого города, созданный на
основе версии Н. Н. Воронина. На переднем плане здесь Новый город со
знаменитыми Золотыми воротами, далее идет город Мономаха, его еще называют
Печерним городом, и узким хвостом тянется Ветчаный (Ветхий) город.

\ifcmt
  pic https://avatars.mds.yandex.net/get-zen_doc/1550999/pub_5fb29e9a7eb1fe4ba089160f_5fb29ea39bb3e623748f69cf/scale_1200
  width 0.5
	caption Макет города Владимира времен его расцвета в 12 веке.
\fi

Надо признать, что версия Воронина до сих пор не потеряла своей
актуальности, хоть знаменитый исследователь древнерусской культуры и
допустил определенные ошибки. Поэтому в своих постах о Владимире я буду
опираться прежде всего на нее.

Читайте далее:

Владимир - вторая столица Руси. Древнейший город
