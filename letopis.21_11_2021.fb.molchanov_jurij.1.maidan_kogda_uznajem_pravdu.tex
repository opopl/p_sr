% vim: keymap=russian-jcukenwin
%%beginhead 
 
%%file 21_11_2021.fb.molchanov_jurij.1.maidan_kogda_uznajem_pravdu
%%parent 21_11_2021
 
%%url https://www.facebook.com/george.molchanov.9/posts/4381603951952539
 
%%author_id molchanov_jurij
%%date 
 
%%tags 1917,maidan2,obschestvo,pravda,revolucia,revolucia.oktober,ukraina
%%title Майдан - когда узнаем правду
 
%%endhead 
 
\subsection{Майдан - когда узнаем правду}
\label{sec:21_11_2021.fb.molchanov_jurij.1.maidan_kogda_uznajem_pravdu}
 
\Purl{https://www.facebook.com/george.molchanov.9/posts/4381603951952539}
\ifcmt
 author_begin
   author_id molchanov_jurij
 author_end
\fi

Как известно, на завтрак революции любят лакомиться своими врагами, обедают
попутчиками, а на ужин предпочитают своих создателей. С разницей лишь в
длительности перерыва между трапезой.

История до безобразия циклична и цинична. Сейчас, как и восемь лет назад
понятно, что никто никому свою правоту не докажет. Такие события возводятся в
культ и навязывается соответствующее поклонение – памятники, почитание,
единственно правильная точка зрения и проекция рассмотрения. А вздумаешь не
верить и критиковать - (н̶а̶ ̶к̶о̶с̶т̶е̶р) в тюрьму.

Относительно полную историю революции 1917 года мы узнали почти 80 лет спустя.
Через сколько мы узнаем все подробности произошедшего в Киеве 8 лет назад –
покажет время. Когда основной массив фактажа начнет освобождаться от груза оков
своих держателей и охранников – главных действующих персонажей того времени.
Ведь скелеты в шкафах могут жить долго, но не вечно. Особенно в современном
информационном мире. 

И ростовские беглецы, и киевские подлецы, рано или поздно утратят контроль над
правдой. Все прослушки, слежки, документы, встречи, условия, кэш, игры большие
и маленькие – вся эта детализация, как правило, бесследно не исчезает. Пока
компроматы используются отчасти как щит, чтобы защищать бизнес и
неприкосновенность. Но и это пройдет. Наружу выйдет очень многое. И тогда,
вероятнее всего, уверовавших в догмы революционной непогрешимости простых
смертных ждет та же участь, что и сегодняшних несчастных стариков, искренне
веровавших в идеалы ленинизма десятки лет тому назад и до слез непонимающих, за
что так обошлись с их памятью и ценностями.

Настоящая инвентаризация событий восьмилетней давности начнется тогда, когда
мы, поляризованное общество, окажемся к этому готовы хотя бы эмоционально.
Воспринимать факты без потерь человеческого облика. Когда максима Достоевского
«не злой вы человек, а исковерканный» будет понятна без дополнительного
разъяснения.

\ii{21_11_2021.fb.molchanov_jurij.1.maidan_kogda_uznajem_pravdu.cmt}
