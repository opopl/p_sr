% vim: keymap=russian-jcukenwin
%%beginhead 
 
%%file 03_12_2021.fb.zaborin_dmitrij.1.den_ukr_borscha
%%parent 03_12_2021
 
%%url https://www.facebook.com/d.zaborin/posts/4857655564255068
 
%%author_id zaborin_dmitrij
%%date 
 
%%tags 1584,borsch,poturaev_nikita,prazdnik,ukraina,vjatrovich_vladimir
%%title День украинского борща на самом деле - День русского сорняка
 
%%endhead 
 
\subsection{День украинского борща на самом деле - День русского сорняка}
\label{sec:03_12_2021.fb.zaborin_dmitrij.1.den_ukr_borscha}
 
\Purl{https://www.facebook.com/d.zaborin/posts/4857655564255068}
\ifcmt
 author_begin
   author_id zaborin_dmitrij
 author_end
\fi

\begin{multicols}{2}
\ifcmt
  ig https://scontent-mxp1-1.xx.fbcdn.net/v/t39.30808-6/263692506_4857653780921913_1562853123106055837_n.jpg?_nc_cat=108&ccb=1-5&_nc_sid=730e14&_nc_ohc=5NOhZgjCiGsAX9c7WjQ&_nc_ht=scontent-mxp1-1.xx&oh=57b5596596ee0ffd2395b2e0d4b58229&oe=61B72A78
  @width 0.45
  %@wrap \parpic[r]
  %@wrap \InsertBoxR{0}
\fi
%\columnbreak

\setlength{\parindent}{0pt}
\Large

Группа патриотов во главе с Потураевым и Вятровичем предложила новый
государственный праздник - День украинского борща. Авторы указывают, что первое
его упоминание приходится на октябрь 1584 года в путевых записках немецкого
торговца Мартина Грюневега.

И это действительно правда. 

Только Грюневег писал про дикорастущее растение и его употребление в пищу
жителями Киева в том числе такими словами: 

«Русские (die Reussenn) покупают борщ редко или никогда, потому что каждый
готовит его сам у себя дома, поскольку это их повседневная еда и питье».

То есть, День украинского борща на самом деле - День русского сорняка. 

Хорошие сапоги, надо брать (аплодирует стоя).

\#историческое \#ДБ

\end{multicols}

\ii{03_12_2021.fb.zaborin_dmitrij.1.den_ukr_borscha.cmt}
