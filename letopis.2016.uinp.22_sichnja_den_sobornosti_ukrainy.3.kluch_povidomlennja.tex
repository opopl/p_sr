% vim: keymap=russian-jcukenwin
%%beginhead 
 
%%file 2016.uinp.22_sichnja_den_sobornosti_ukrainy.3.kluch_povidomlennja
%%parent 2016.uinp.22_sichnja_den_sobornosti_ukrainy
 
%%url 
 
%%author_id 
%%date 
 
%%tags 
%%title 
 
%%endhead 

\subsubsection{Ключові повідомлення}

1. 22 січня ми згадуємо дві рівнозначні за вагою події української історії:
проголошення незалежності Української Народної Республіки 1918 року та Акт
Злуки українських земель рівно через рік – 1919-го.

2. Українська Народна Республіка – перша українська держава у ХХ ст., а 24
серпня 1991 року відбулося фактично не здобуття, а відновлення незалежності.

3. Українська Народна Республіка першою серед нових держав у Східній Європі
проголосила незалежність – раніше, ніж три країни Балтії, Польща та Чехія.

4. Проголошення Соборності УНР та ЗУНР 22 січня 1919 року є історичним актом
об'єднання українських земель в єдину державу. Саме ці події, а не приєднання
Західної України до СРСР 1939 року, є підставовими для історії новітнього
українського державотворення.

5. Українська незалежність була повалена більшовиками внаслідок \enquote{гібридної
війни}: невизнання наявності своїх військ на території УНР, створення
маріонеткових проросійських \enquote{республік} та підтримка антиукраїнських
повстанських рухів.

6. Національна єдність є не тільки базовою цінністю, а й обов'яз\hyp{}ковою
передумовою успішного спротиву зовнішній агресії. Для того, щоб відстояти УНР,
українцям забракло єдності та національного усвідомлення.

7. Акт Злуки був не випадковим явищем, а наслідком і вершиною об'єднавчого
руху, що тривав від середини ХІХ ст. на українських землях, що були в складі
різних держав.

8. Незалежність і соборність є запорукою виживання Української державності.
Втрата Україною незалежності у результаті більшовицької окупації у
довготерміновій перспективі призвела до міль\hyp{}йонних втрат від Голодомору,
репресій та війн.

9. Сьогодні бійці в зоні АТО так само відстоюють не лише незалежність, а й
соборність України, як і їхні попередники майже 100 років тому. Важливо
проаналізувати і врахувати помилки минулого: брак національної єдності на
початку ХХ ст., брак досвіду і містечковість інтересів тодішніх українських
керманичів призвели до втрати української державності.

10. Сьогодні, коли маємо українські території, непідконтрольні українській
владі, День Соборності – це привід нагадати, що Крим та Донеччина – це Україна.
