% vim: keymap=russian-jcukenwin
%%beginhead 
 
%%file 13_12_2020.news.ua.strana.tovt_anastasia.1.masha_buduscii_mer_kieva
%%parent 13_12_2020
 
%%url https://strana.ua/articles/interview/306431-masha-moiseenko-stala-zvezdoj-interneta-karabkajas-po-hololedu-na-andreevskom-spuske.html
 
%%author Товт, Анастасия
%%author_id tovt_anastasia
%%author_url 
 
%%tags kiev
%%title "Маша – будущий мэр Киева!". "Страна" нашла школьницу, которая после видео с гололедом стала звездой интернета
 
%%endhead 
 
\subsection{\enquote{Маша – будущий мэр Киева!}. \enquote{Страна} нашла школьницу, которая после видео с гололедом стала звездой интернета}
\label{sec:13_12_2020.news.ua.strana.tovt_anastasia.1.masha_buduscii_mer_kieva}
\Purl{https://strana.ua/articles/interview/306431-masha-moiseenko-stala-zvezdoj-interneta-karabkajas-po-hololedu-na-andreevskom-spuske.html}
\ifcmt
	author_begin
   author_id tovt_anastasia
	author_end
\fi

\index[cities.rus]{Киев!Маша Моисеенко, вверх по льду по Андреевскому, 13.12.2020}

\ifcmt
pic https://strana.ua/img/article/3064/masha-moiseenko-stala-31_main.jpeg
caption Девятилетняя Маша Моисеенко стала знаменитой благодаря видео, на котором она карабкается по гололеду на Андреевском спуске. Фото предоставлено мамой Маши "Стране" 
\fi

Самое популярное видео трех последних "гололедных" дней в Киеве - эпизод, где
девочка с большим портфелем на плечах пытается взобраться вверх по Подолу, но
безуспешно - школьница падает снова и снова, множество раз.

Героиней видео оказалась 9-летняя киевская школьница Маша Моисеенко. Девочка на
этой неделе буквально проснулась знаменитой. Видео, на котором она безуспешно
карабкается по гололеду на Андреевском спуске, облетело все украинские и
мировые СМИ.
 
"Страна" разыскала "покорительницу Андреевского спуска". Нашему изданию Маша
сообщила, что ее уже стали узнавать на улицах – по той самой шапке, в которой
она пыталась преодолеть стихию.
 
– Я не считала, сколько раз упала, но, судя по видео, раз 40 точно, –
признается Маша.

Чувствует себя девочка хорошо – обошлось только синяками.

– Я обрадовалась, когда увидела, что все постят видео со мной. Не ожидала, что
мой поступок попадет в интернет. Многие люди пишут, звонят – спрашивают, как я
себя чувствую. Я ощущаю популярность. Даже на улицах меня узнают – по шапке, в
которой я шла в том видео.
 
Семья девочки говорит, что из-за неожиданной славы стараются уже не надевать на
Машу ту самую шапку.

\video{https://youtu.be/YzWgsUw5--E}
\url{https://t.me/kryuchoktv/1825}

"Страна" пообщалась с мамой девочки, Инной Моисеенко, о том, как ее дочь стала
интернет-звездой и как сама Маша воспринимает случайную популярность.
 
\textbf{– Что вы почувствовали, когда увидели, что видео с вашей дочкой растиражировали
сотни украинских и зарубежных СМИ?}
 
– Я сразу узнала на видео Машу. Мне ее было очень жалко. Первые полдня, когда я
думала об этом видео, мне хотелось плакать. А потом мы уже свыклись с мыслью,
что Маша – интернет-звезда.
 
\textbf{– Маша ушиблась при падении на льду?}

– Нет, слава богу, у нее только пару синяков на коленке и ноге. Сама Маша
пришла в тот день домой жизнерадостная, абсолютно не испуганная. Правда, потом
признавалась, что, когда она карабкалась по Андреевскому спуску, ей было
страшно – она впервые столкнулась с таким гололедом. Ей казалось, что она
пройдет этот участок, и дальше льда не будет. Это же все случилось неожиданно.
Буквально за час до этого дороги были сухие, не было и намека на лед. Потом
резко похолодало, пошел сильный дождь, и за 15 минут уже все улицы были в
гололеде. Маша тогда шла с занятий, ее всегда встречает старший брат, они идут
друг другу навстречу: он спускается по Андреевскому спуску, а Маша –
поднимается. Ей учительница сказала – мол, там гололед. Но Маша была уверена,
что дойдет. Когда она пришла на место встречи с братом, удивилась, что его нет.
 
\textbf{– Он тоже не мог пройти из-за гололеда?}
 
– Да, ему нужно было с другой улицы вскарабкаться наверх, а потом спуститься по
Андреевскому. Там никто не мог пройти. Там же не просто лед был, а еще и резкий
подъем – даже руками не зацепиться! Именно на этом месте был просто каток.
Брату идти за Машей ровно 10 минут – он и вышел за 10 минут. А шел 40 минут.

\ifcmt
pic https://strana.ua/img/forall/u/0/25/WhatsApp_Image_2020-12-13_at_17.06_.01_.jpeg
caption Незнакомые люди в чате жителей Подола пророчат школьнице Маше мэрскую карьеру после покорения оледеневшего Андреевского спуска
\fi

\textbf{– Вы читали комментарии под видео с Машей? Как реагируете на критику в свой адрес – мол, ребенок не так одет, не так обут, зачем туда лезть и прочее?}
 
– Вы не представляете – мне пишут тысячи людей! И по-японски, и по-китайски, и
по-итальянски, и по-английски. Вся Америка смотрела видео с Машей, его
показывали в Индии и Пакистане. Я всем не могу отвечать на комментарии – сын
помогает. На самом деле, в основном все восторгаются упорством Маши, ее
характером. В комментариях под видео пишут, что моя Маша – наш будущий мэр! Но
есть и 1\% тех, кто критикует. "Надо ж было мозгами подумать и не лезть на эту
горку", или "Почему она не поехала на метро?". А зачем ей метро, если она была
почти возле дома? Или "Почему у нее такие сапоги?". У Маши – новые, хорошие
сапоги, не "угги", с толстой качественной подошвой. Я так понимаю, все это
пишут люди, у которых нет детей.

\textbf{– Вы не пытались из любопытства подсчитать, как много СМИ опубликовало видео с
вашей дочкой?}

– Для нас вообще полное удивление, что это видео получилось таким «вирусным». К
нам обращались из разных СМИ, из многих стран. Только на одном из зарубежных
медиа видео с Машей набрало 7,5 млн  просмотров за сутки. Даже для редакции это
был полный шок. Как нам сказал один журналист, даже их известные спортсмены
после победы не набирают столько просмотров. Никто такого не ожидал. Мы сами
вообще к раскрутке этого видео никакого отношения не имеем. Мы не знаем, кто
записал это видео и выложил его в интернет. Но, с другой стороны, поведение
Маши на записи меня абсолютно не удивило. 

\textbf{– Что вы имеете в виду?}
 
– У нашей Маши – сильный, спортивный характер. Она маленькая, но очень
целеустремленная. Если решила что-то сделать, ее не надо заставлять – все сама
доведет до конца. Она занимается воздушной акробатикой и очень это любит. Еще
она поет, занимается рукоделием, шьет мягкие игрушки и играет на фортепиано. И
вы знаете, эта случайная популярность, которая на Машу свалилась, все-таки
неслучайна. Наверное, даже в ролике видно, что Маша – жизнерадостная и упорная.
Это людям передается.

\ifcmt
pic https://strana.ua/img/forall/u/0/25/WhatsApp_Image_2020-12-13_at_16.44_.41_(1).jpeg
caption Девятилетняя Маша Моисеенко занимается воздушной акробатикой и отличается упорством во всем, включая карабкание по гололеду на Подоле, говорят родители девочки. Фото предоставлено Инной Моисеенко "Стране"
\fi

\textbf{– Как сама Маша восприняла случившееся?}
 
– Для нее это не было стрессом. Скорее, она восприняла это как приключение. Мы
же все в детстве падали, катались с горок. И вообще, Маша очень положительный
ребенок: она встает и ложится с улыбкой. Когда брат ее забрал с Андреевского
спуска, они так и не смогли подняться по горке – пошли в обход по Валам,
погрелись в зоомагазине. Обычно дети домой идут 10 минут – в тот день
добирались целый час. Но Маша пришла домой в хорошем настроении. Она просто не
понимала, почему не может подняться. Наверное, если бы она осознавала, что там
вся горка во льду, она бы вернулась на занятия. Но она ребенок. И раньше
никогда не видела такого гололеда.
