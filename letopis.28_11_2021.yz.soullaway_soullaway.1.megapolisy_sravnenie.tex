% vim: keymap=russian-jcukenwin
%%beginhead 
 
%%file 28_11_2021.yz.soullaway_soullaway.1.megapolisy_sravnenie
%%parent 28_11_2021
 
%%url https://zen.yandex.ru/media/id/5dbc544f2f1e4400aff32830/sravnenie-megapolisov-rossii-i-ukrainy-61a2507297769e40fd7bd28a
 
%%author_id yz.soullaway_soullaway
%%date 
 
%%tags dnepropetrovsk,ekaterinburg,gorod,kiev,lvov,megapolis,moskva,odessa,rossia,sravnenie,st_peterburg,ukraina
%%title Сравнение мегаполисов России и Украины
 
%%endhead 
 
\subsection{Сравнение мегаполисов России и Украины}
\label{sec:28_11_2021.yz.soullaway_soullaway.1.megapolisy_sravnenie}
 
\Purl{https://zen.yandex.ru/media/id/5dbc544f2f1e4400aff32830/sravnenie-megapolisov-rossii-i-ukrainy-61a2507297769e40fd7bd28a}
\ifcmt
 author_begin
   author_id yz.soullaway_soullaway
 author_end
\fi

Мне тут иной раз попадаются размышления о том, что на Украине всё же есть
какое-то сейчас развитие и перспектив там, в ряде городов великое множество.
Заявления доходят даже до такого абсурда, что жизнь в некоторых мегаполисах
Украины лучше, чем в мегаполисах России. Хорошо, давайте попробуем сравнить
несколько городов. Само собой Москву мы тут брать не будем. Это город из
мирового списка наряду с Парижем или Нью-Йорком. Его влияние огромно и не менее
велико население. Было бы странно пытаться сравнивать Москву с Киевом. Киев это
локальный город, который не может влиять на мировые процессы.

\ii{28_11_2021.yz.soullaway_soullaway.1.megapolisy_sravnenie.pic.1}

Такая же история и с Санкт-Петербургом. Ничего аналогичного на Украине нет. Но
справедливости ради отмечу, что и в России нет ничего подобного Киеву. Нам вот
вообще очень не хватает такой точки притяжения с населением миллиона в три.
Хорошо если когда-нибудь на Урале и в Сибири появятся подобные города, но
сейчас их нет, и пока что не предвидится. Таким образом, можно попробовать
сравнить Харьков и, допустим что? Ну, пусть будет Новосибирск. Они, в общем-то,
схожи по населению и влиянию внутри страны.

Если рассматривать эти мегаполисы с точки зрения влияния на свои страны, то они
идентичны. Правда, Новосибирск немного крупнее. Там проживает 1 миллион 600
тысяч, а в Харькове 1 миллион 400 тысяч человек. Если взять метрополитены, то
Харьков обгоняет нашу столицу Сибири. Три ветки метро и 30 станций против двух
веток и 13 станций. То есть тут Россия, как ни странно отстаёт и отстаёт
серьёзно. В плане же притока населения, учебных заведений и предприятий оба
города почти идентичны. Распространяется это и на инфраструктуру и на
привлекательность города для жителей. Фотографии соответственно верху Харькова,
а ниже Новосибирска.

В России почему-то принято считать, что Украина ограничивается одним Киевом.
Особенно с учётом накачивания пропагандой о других мегаполисах Украины вообще
мы стали забывать. Но ведь это одна из крупнейших европейских стран. Когда-то
там хватало и промышленности. И совсем неудивительно, что среди десятков
городов Украины есть мегаполисы.

\ii{28_11_2021.yz.soullaway_soullaway.1.megapolisy_sravnenie.pic.2}

Вот, например Днепропетровск, ныне переименованный в Днепр. На мой взгляд, один
из самых интересных городов там. Но много ли с ним ассоциаций у русского
человека? Да вряд ли. Хотя там проживает около миллиона человек.

Аналогом в России я, пожалуй, назвал бы Екатеринбург. Даже не знаю, почему у
меня эти города вызывают одинаковый интерес. Возможно, из-за попыток строить
небоскребы. Возможно, из-за тяги как-то перейти к новой архитектуре. Ведь и
там, и там хватает новых интересных зданий. Хотя конечно в столице Урала их
больше. Но там и население больше. В Екатеринбурге сейчас уже почти полтора
миллиона человек живет, а в Днепропетровске 980 тысяч. Недавно он выбыл из
числа городов с миллионом населения. Видимо сказывается, что с работой на
Украине дела несколько хуже, чем в России и крупные мегаполисы там почти не
растут.

Говоря же о метрополитене, а это, на мой взгляд, очень важная штука для
понимания города то не могу не отметить, что ситуации схожи. В Днепропетровске
6 станций, в Екатеринбурге 9. Интересно, что оба города обзавелись метро уже в
90-е годы. Конечно же, это всё советский задел, но тот же Челябинск так и не
осилил метро. Хотя и там осталось советское наследие.

Выше я указал, что мы стали забывать о мегаполисах Украины, но я упустил такое
место как Одесса. Это больше чем просто город для нашей культуры. С ним связано
действие, как исторических событий, так и литературных. Огромный и бесспорно
красивый город известный с начала 20 века как один из крупнейших портов. А
большой порт это всегда отдельная история. Тем более в случае с Одессой тут
накладывается ещё и местный колорит. И, наверное, для современной Украины
Одесса это почти то же самое, что для России Питер. Конечно же, эти города
нельзя сравнивать, но я веду речь именно о культурном влиянии сейчас.

\ii{28_11_2021.yz.soullaway_soullaway.1.megapolisy_sravnenie.pic.3}

Однако поскольку в самом начале я сказал, что Питер, Москва и Киев города,
которые мы сегодня не трогаем из-за слишком уж большого размаха и столичного
статуса, то Одессу логичнее всего сравнивать с нашим Ростовом на Дону. Тем
более есть же выражение, что Одесса мама, Ростов отец.

Сравнивать население на сегодняшней Украине дело сложное. По некоторым оценкам
в Одессе сейчас живёт меньше миллиона человек, по некоторым сведениям больше.
Пусть будет нечто среднее – миллион. Ростов же уверенно растёт. Он давно уже
перевалил отметку в миллион сто тысяч и продолжает наращивать жителей. Порт
там, кстати, тоже есть и довольно немалый. С Одессой сравнивать нельзя, но
Ростов и расположен не на берегу моря. Но важно, что он привлекателен для
многих как крупный научный центр. Между прочим одни из лучших специалистов в IT
сфере у нас готовятся там. Хотя, казалось бы, юг предполагает что-то другое, но
вот удалось местным ВУЗам наладить качественную подготовку кадров.

Ну и из больших городов Украины было бы нечестно забыть о таком месте как
Львов. Так сложились обстоятельства, что именно Львов как столица запада страны
оказался наиболее влиятельным и привлекательным в 21 веке для украинцев.

Конечно, тут стоит учитывать, что украинским городом он стал не так уж и давно.
Но именно это и делает его привлекательным. Видимо глядя на европейские улочки
у некоторых граждан закрались подозрения, что как только они сметут какую-то
власть, так тут же внешне изменятся и Кривой Рог и Мариуполь. Не изменятся.
Какую власть не ставь. Слишком разная история и слишком разное прошлое. Если
Одессу или Запорожье строили, в том числе и русские люди, то история Львова
вообще другая. Об этом почему-то украинцы забывают.

\ii{28_11_2021.yz.soullaway_soullaway.1.megapolisy_sravnenie.pic.4}

Ближайшим аналогом с Львовом у нас мог бы быть Калининград. Но дело в том, что
Калининград смели во время Великой Отечественной войны. При штурме старый город
был практически полностью уничтожен. И сравнивать тут теперь нечего. Если
исходить из количества населения, то во Львове живёт примерно 700 тысяч
человек. Городов с таким населением у нас нет. Можно было бы взять Тюмень, но
она уж слишком бурно развивается, плюс там рядом есть нефть. Можно было бросить
взор на Ярославль или Иркутск. Огромный исторический пласт сопоставимый со
Львовом, но я считаю, что честнее всего будет взять Владивосток. Тем более это
компенсирует предыдущую Одессу.

Так вот Владивосток сейчас явно развивается быстрее украинского Львова. И дело
не только в мостах и саммитах. Дело-то в том, что туда активно вливают деньги,
а это правильно.

Подводя итог можно сказать, что с одной стороны по подборке может показаться,
что на Украине дела-то идут неплохо. Там есть сопоставимые мегаполисы. Но в
подборке же не учтены Нижний Новгород и Казань. Есть у нас Сочи и Самара. А уж
про Псков или Тулу я и вовсе молчу.

Очевидно, что Россия масштабнее и перспективнее. Ведь сюда едут работать
мигранты, а не на Украину. Но главное, что я хотел до вас донести это простую
мысль о том, что и на Украине есть мегаполисы. И не так уж там всё плохо,
несмотря на все прогнозы 2014 года. Эта страна существует и там продолжается
развитие, она не замерла и не остановилась. Вот об этом стоит помнить, когда
нам в очередной раз будут что-то рассказывать по телевизору.

У меня же на сегодня всё, спасибо вам за внимание и до новых встреч.
