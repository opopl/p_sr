% vim: keymap=russian-jcukenwin
%%beginhead 
 
%%file 06_04_2021.fb.marusyk_taras.1.jazyk_mendel
%%parent 06_04_2021
 
%%url https://www.facebook.com/groups/promovugroup/permalink/947850539122141/
 
%%author 
%%author_id 
%%author_url 
 
%%tags 
%%title 
 
%%endhead 

\subsection{\enquote{Культуру не насаджувати, а бачити у форматі космополітизму}}
\url{https://www.facebook.com/groups/promovugroup/permalink/947850539122141/}


\ifcmt
  pic https://external-yyz1-1.xx.fbcdn.net/safe_image.php?d=AQGMCLFG4rHxMcfA&w=500&h=261&url=https%3A%2F%2Fkanaldom.tv%2Fwp-content%2Fuploads%2F2021%2F04%2F03-5.jpg&cfs=1&ext=jpg&ccb=3-4&_nc_hash=AQECYKHXDhl9RDX7
\fi


Такого ще ніхто з Банкової не озвучував. Чергові "наДомні менделізми"
засвідчили не лише облуду з боку Банкової і телеканалу "Дом", а й те, що в
Зеленського вирішили не відступати від ідеї російської мови як спадщини
України.

Нагадаю, що після першого етеру, який викликав обурення, в тому числі і в мене,
Телеканал Дом вибачився за введення в оману "щодо участі пані Мендель в проєкті
в якості ведучої на телеканалі. Просимо вибачення за те, що одразу не внесли
ясності щодо цього питання".

\url{https://www.facebook.com/MarusykT/posts/2916717171986148}

Тепер вже ясність є: "Взгляд с Банковой" той самий, тільки ще з однією
русскоязичною, яка ставить заготовлені питання.

Аргументуючи тезу про те, що "українська російська мова - це частина
культурного розмаїття нашої країни", Юлія Мендель вдалася до виразу одного
філософа про "ангелів з піною на губах", які породжують дух ненависті, "завдяки
якому зло на Землі не має кінця". І сказала, що українці перетворилися в таких
ангелів.

У цитованого філософа Григорія Соломоновича Померанца є й таке: "Мамо, не
говори голосно, від цього засихають дерева". Згадуючи, як Юлія Мендель фактично
програла суд професору Юрієві Коваліву за наклеп про хабар (не згадую інші
"зашквари", зокрема, й на міжнародному рівні), можна перефразувати Померанца:
"Юлю, не бреши, від цього засихають дерева".

Юрій Ковалів
