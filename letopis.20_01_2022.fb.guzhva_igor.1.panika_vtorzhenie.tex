% vim: keymap=russian-jcukenwin
%%beginhead 
 
%%file 20_01_2022.fb.guzhva_igor.1.panika_vtorzhenie
%%parent 20_01_2022
 
%%url https://www.facebook.com/veprwork/posts/4870487336343438
 
%%author_id guzhva_igor
%%date 
 
%%tags isteria,panika,rossia,ugroza,ukraina,vojna,vtorzhenie
%%title Паника относительно российского вторжения в Украину
 
%%endhead 
 
\subsection{Паника относительно российского вторжения в Украину}
\label{sec:20_01_2022.fb.guzhva_igor.1.panika_vtorzhenie}
 
\Purl{https://www.facebook.com/veprwork/posts/4870487336343438}
\ifcmt
 author_begin
   author_id guzhva_igor
 author_end
\fi

Ситуация вокруг раздуваемой в западных СМИ паники относительно «российского
вторжения в Украину» становится все более мутной.

Зеленский вчера выступил с обращением, в котором заявил, что паника эта
раздувается с целью давления на украинскую власть, чтоб вынудить ее к уступкам.

Кто давит и к каким уступкам вынуждает он не уточнил. Но это и так понятно.
Заявления о скором вторжении России в Украину идут исключительно с Запада
(Россия такие намерения отрицает). 

А что касается уступок, то о них вчера говорил Блинкен во время визита в Киев.

Это выполнение Минских соглашений.

По ним госсекретарь сделал два ключевых заявления.

Первое - Минские соглашения не нужно пересматривать и это единственный путь
урегулирования на Донбассе.

Второе - в Украине сейчас "решаются" некие действия по той части соглашений,
которые Киев еще не выполнил. Под этим, очевидно, следует понимать политическую
часть Минских соглашений (выборы до передачи контроля над границей, амнистия,
особый статус и закрепление его в Конституции). 

При том, что украинские власти официально заявляют о нежелании выполнять
политическую часть соглашений и настаивают на ее изменении.

То есть госсекретарь своими заявлениями фактически посоветовал Киеву изменить
свою позицию. 

После этого и появилось обращение Зеленского, из которого следует, что он
совету Блинкена не внял, а заявления с Запада о скором вторжении воспринимает
как попытку принудить Киев к уступкам, напугав угрозой войны и ослабив
экономически (из-за паники инвесторы выводят деньги из Украины и гривня
падает).

Тут, конечно, многое зависит от степени давления, которое могут оказать США на
Зеленского, чтоб тот передумал (то есть будет ли Вашингтон давить ещё как-то
кроме как через публикацию «карт вторжения России» в западных газетах).

Однако объективно говоря вероятность того, что Зеленский и его власть пойдут на
реализацию Минских соглашений близка к нулю.

А значит выполнение Минска и окончание войны на Донбассе – это вопрос смены
власти.

Готовы ли США подключиться к этому вопросу, либо все, как это уже не раз
бывало, закончится  уговорами и разговорами – увидим в ближайшие месяцы.

Теперь что касается большой войны, которой пугают западные СМИ.

На самом деле ответ на вопрос будет она или нет лишь очень косвенно связан с
переговорами по безопасности между Россией и США и с реакцией Запада на
требования РФ по нерасширению НАТО.

Россия, даже если переговоры провалятся, вряд ли пойдет танками на Украину. 

Это не решит ни одной из ее задач, но создаст много проблем. Эскалация на
Донбассе (по типу января-февраля 2015 года) не приведет ни к Минску-3, ни к
краху Украины как государства (даже в случае потери новых территорий), а лишь
задаст еще более антироссийский курс Киева (не исключено, что уже с новым
президентом на Банковой), создаст угрозу новых санкций и «заморозки» Северного
потока-2.

Масштабное открытое вторжение по всей протяженности границы, даже в случае
быстрого военного успеха, однозначно приведет к масштабным санкциям с огромными
потерями для российской экономики. Да и вопрос, что потом делать с
подконтрольной Украиной, которую придется брать на полное обеспечение (а
главное – зачем).

Поэтому, кстати, обсуждаемые сейчас (даже на уровне слухов) варианты реакции
России на провал переговоров не включают войну в Украине. Речь идет лишь о
каких-то военно-технических мерах вроде размещения ракет в Латинской Америке.

И, наконец, еще один момент. Февраль 2022 года (на который большинство
«прогнозистов» в западных СМИ назначают вторжение России) – это месяц зимней
олимпиады в Пекине. Для КНР она очень важна с имиджевой точки зрения, чтоб
перебить разгоняемую на Западе антикитайскую волну. И трудно предположить, что
Россия решит крупно подставить своего ближайшего союзника, начав войну в
Украине, чем полностью затмит Олимпиаду.

Означает ли крайне малая вероятность начала войны по инициативе России то, что
войны точно не будет?

К сожалению нет.

Инициаторы могут быть с другой стороны.

На Западе (не говоря уже об Украине) есть немало сил, которые крайне
заинтересованы в срыве диалога США и России с последующим скатыванием их
отношений к масштабной конфронтации. Ну и чтоб параллельно обнулить имиджевые
усилия Китая по Олимпиаде (см выше).

И война в Украине с открытым участием России – самый очевидный способ это
сделать.

Поводом может стать наступление Украины на Донбассе. Правда, чтобы Россия
вмешалась это не должны быть провокации, обстрелы или диверсии каких-то
анонимных групп. Это должно быть открытое наступление официальных силовых
структур Украины с официальным подтверждением. Как это сделал Саакашвили в
августе 2008 года в Цхинвале.

Готов ли на это пойти Зеленский? Я слышал версии, что это было бы ему выгодно,
так как позволило бы ввести в стране военное положение и решить одним махом все
проблемы с оппозицией и СМИ, фактически установив режим авторитарного
правления. При этом в масштабное вторжение России даже в таком случае в Киеве
не верят, а локальные поражения и потери новых территорий на Донбассе не
считают критической проблемой. Удержался же ведь Пашинян после поражения в
Карабахе.  

Но Украина не Армения. Украинский госаппарат сейчас находится в крайне
разваленном состоянии (в чем Зеленский мог уже неоднократно убедиться) и
военная авантюра может привести к очень тяжелым последствиям лично для Зе.

Поэтому вероятность того, что Зеленский решится на войну, небольшая. Да и его
риторика последнего времени говорит, что он на резкие шаги на фронте идти не
готов.

Однако в Украине есть силовые структуры, которые Зеленский не контролирует. Это
полк Азов Нацгвардии, который по-прежнему остался под влиянием Билецкого. А сам
Билецкий сейчас ориентируется на противников Зе – Авакова и Турчинова. Азов –
это вполне официальная силовая структура с тяжелым вооружением.

Кроме того, по всей стране сейчас формируются подразделения территориальной
обороны. В Минобороне им уже пообещали раздать оружие (причем как раз тем, кто
находится в приграничных областях). Что там за люди, кто их контролирует –
варианты могут разные. 

А атмосфера напряженная. И чтоб ее взорвать может быть достаточно и «перехода в
наступление на агрессора» одного батальона – под камеры, с развернутыми
знаменами и с обильным освещением в «патриотических» СМИ.

Именно такие провокации и являются сейчас главной угрозой начала войны.

Нужно быть очень бдительными.

P.S. В комментариях - пару ссылок по теме.
