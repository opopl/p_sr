% vim: keymap=russian-jcukenwin
%%beginhead 
 
%%file slova.politika
%%parent slova
 
%%url 
 
%%author 
%%author_id 
%%author_url 
 
%%tags 
%%title 
 
%%endhead 
\chapter{Политика}
\label{sec:slova.politika}

%%%cit
%%%cit_head
%%%cit_pic
%%%cit_text
Но, в системе, где даже Главный закон страны в лучшем случае \enquote{как дышло}, в
худшем - просто слова, нет смысла просто его переписывать - не растут груши на
Марсе. Нужно прежде всего кардинально менять практику его правоприменениния и
наказания чиновников за игры с Конституцией, при этом создавая условия для
будущего реформирования Конституции с цель перехода на чистую модель управления
- парламентскую (предпочтительнее) или президентскую - только в таких условиях
возможно обеспечить хоть какую-то ответственность \emph{политиков}
%%%cit_comment
%%%cit_title
\citTitle{Из 161 статьи Конституции Украины системно нарушаются 125 / Лента соцсетей / Страна}, 
Руслан Бортник, strana.ua, 28.06.2021
%%%endcit

%%%cit
%%%cit_head
%%%cit_pic
\ifcmt
  pic https://avatars.mds.yandex.net/get-zen_doc/5226854/pub_60e1daa72dcce634f5902d51_60e200dd2dcce634f5f8eeda/scale_1200
  caption это юмор из Украины. На русском, кстати, языке!!!
\fi
%%%cit_text
Это даже не глупость, это \emph{ПОЛИТИКА}, \emph{политические игры} и хайп.
Сами же украинские политики и активисты в быту по-русски разговаривают, и еще
не известно на каком языке Ницой в поисковике запросы забивает.
Теперь немного о русско-украинских отношениях через призму ЧЕ по футболу:
%%%cit_comment
%%%cit_title
\citTitle{Мова, политика и сборная Украины по футболу}, Мак Сим,
zen.yandex.ru, 04.07.2021
%%%endcit

%%%cit
%%%cit_head
%%%cit_pic
%%%cit_text
«Бо історія перетворилася на зброю, зброю проти України. Її легко можуть
звинуватити у колаборації з нацистами в період Другої світової війни, при тому
що не було окупованої території, де б колаборації не було, але Норвегії, чи
Франції ніхто цих претензій не виставляє» – говорить Цибульська.
Так само як і праворадикали існують в кожній країні Європи, при чому на рівні
\emph{топ-політиків}, але тільки в Україні, де вони не представлені ані впливовими
\emph{політичними} силами в парламенті, ані хразиматичними \emph{політичними фігурами}, вони
викликають занепокоєння
%%%cit_comment
%%%cit_title
\citTitle{Спецоперація під назвою «стаття Путіна»}, 
Марія Щур; Сашко Шевченко, www.radiosvoboda.org, 13.07.2021
%%%endcit

%%%cit
%%%cit_head
%%%cit_pic
%%%cit_text
Це правда, так. За фактом кастова система.
Да. И, по сути, это привело к тому, что формируется такой
сословно-кастово-клановый дух, в котором позицию человека определяет лояльность
главе клана. Главой клана может быть человек, который… Допустим, Аваков, глава
клана ментовского сейчас. Или сбушные генералы, которых никто не знает, но они
десятилетиями формируют \emph{внутреннюю политику} в спецслужбах, и очень влиятельны,
люди типа Наумова какого-нибудь. И условно говоря, рамка целостная, она никогда
не формировалась. То есть, она формировалась, наверное, только Кучмой.
Вот Кучма единственный, кто пытался, исходя из своих представлений о том, как
должно быть устроено государство, потому что он работал в конструкторском бюро,
и он подходил как инженер к конструированию Второй Украинской республики. И он
ее сконструировал, по сути. И он же в нее заложил те пороки, которые сегодня
являются причиной наших серьезных проблем
%%%cit_comment
%%%cit_title
\citTitle{Украина в плену консервативного мышления: почему 30 лет шли не туда и что делать}, 
Сергей Иванов; Юрий Романенко, hvylya.net, 15.07.2021
%%%endcit

%%%cit
%%%cit_head
%%%cit_pic
%%%cit_text
Против поглощения Украины другими странами нам необходимо запустить
\emph{политический} процесс поиска места Украины в мировых экономических и
\emph{политических процессах}. А это невозможно с сегодняшними примитивными подходами
к \emph{политике}, экономике, управлению. Все нужно менять, ставить на
профессиональные рельсы. И никакая помощь со стороны не поможет нам, поскольку
надо учиться жить своим умом, а не следовать инструкциями со стороны. Найдем
свое место в современной истории – будет государство Украина, заиграемся в
популизм, пойдем по легкой, но неверной дороге ‒ погубим свое государство
%%%cit_comment
%%%cit_title
\citTitle{О будущем украинского и русского народов}, 
Виктор Медведчук, strana.ua, 15.07.2021
%%%endcit

%%%cit
%%%cit_head
%%%cit_pic
%%%cit_text
В «історичній» дискусії двох президентів, поза сумнівом, слушною є думка
Президента України. Але її справедливість доводять не історичні дослідження, не
пам’ятники державним діячам минулого, а сьогоднішній стан розуму громадян трьох
країн, передусім України. Абсолютна більшість із них сьогодні вважають себе
частиною українського народу, а росіянами бути не хочуть. І в цій рішучості є
деяка «заслуга» Володимира Путіна, оскільки після його воєнної агресії процес
формування української \emph{політичної} нації вочевидь прискорився
%%%cit_comment
%%%cit_title
\citTitle{Який народ хрестив князь Володимир?}, 
Ігор Яковенко, day.kyiv.ua, 29.07.2021
%%%endcit

%%%cit
%%%cit_head
%%%cit_pic
\ifcmt
  pic https://strana.news/img/forall/u/0/0/9653746-57713010ad196-1418052720_1200.jpg
  @width 0.4
\fi
%%%cit_text
В международной \emph{политике} вчера поводом для скандала стало дебютное
использование украинской армией на Донбассе беспилотника "Байрактар". После
того, как Минобороны Украины признал факт такого дебюта, МИД Германии выступил
с заявлением, в котором призвал к "сдержанности" в использовании беспилотных
летательных аппаратов, которые имеет право использовать только миссия ОБСЕ.
Подытожили свое заявление немцы призывом к отказу от эскалации, которая чревата
переходом к полномасштабной войне.  Вчера в эту дискуссию вступил скандально
известный посол Украины в Германии Андрей Мельник. Тот самый, который уже
удостоился резкой критики со стороны президента ФРГ Франка-Вальтера Штайнмайера
за свои антигерманские заявления. В этот раз Мельник продемонстрировал, что ему
плевать на мнение руководства страны, в которой он аккредитован: в своем
заявлении посол предложил немцам не совать нос не в свои дела, а вместо этого
"удвоить давление" на Москву
%%%cit_comment
%%%cit_title
\citTitle{Киевский полулокдаун, "анонимные" освободители Украины, украино-немецкий скандал. Итоги "Страны"}, 
, strana.news, 29.10.2021
%%%endcit

%%%cit
%%%cit_head
%%%cit_pic
%%%cit_text
В Российской империи Тарас Шевченко не был популярен. К его столетию
представители украинской интеллигенции решили провести сбор средств на
памятник, но обнаружили, что поэт неизвестен в народной массе. Только после
революции 1917 года, в связи с директивным созданием Украинской ССР и
«украинского народа» (миллионы русских людей просто записали в «украинцы»),
\emph{политикой} коренизации (масштабного поощрения нацменьшинств в ущерб русскому
народу), началась массовая пропаганда образа «великого кобзаря». Так
малороссийского художника и поэта превратили в культовую фигуру украинской
интеллигенции
%%%cit_comment
%%%cit_title
\citTitle{Шевченко без украинизма}, , topwar.ru, 09.03.2019
%%%endcit

%%%cit
%%%cit_head
%%%cit_pic
%%%cit_text
Беда не только во Владимире Зеленском, в котором быстро очаровались от
безысходности в 2019, а к 2021 всеми этими 73\%-ми дружно грохнулись в еще
большее разочарование, чем в Петре Порошенко. Беда в том, что так называемые
альтернативы, которых регулярно показывают, например, у Савика Шустера, —
навевают такую же кромешную безысходность, как и власть. И точно не желание
голосовать ни за действующего, ни за остальных.  Да и этим полузабытым
полубывшим крайне трудно теперь вызывать новое желание у избирателей — то есть
у тех, чьи интересы они никак не могут представлять, потому что финансируют их
\emph{политическую} карьеру совсем другие люди, и электорат уже начал о чем-то
догадываться и не верить в цирк для бедных
%%%cit_comment
%%%cit_title
\citTitle{Альтернативы власти - такая же кромешная безысходность, как и она сама / Лента соцсетей / Страна}, 
Елена Дьяченко, strana.news, 04.12.2021
%%%cit_url
\href{https://strana.news/opinions/365443-alternativy-vlasti-takaja-zhe-kromeshnaja-bezyskhodnost-kak-i-ona-sama.html}{link}
%%%endcit
