% vim: keymap=russian-jcukenwin
%%beginhead 
 
%%file 22_07_2020.news.dnr.1
%%parent 22_07_2020
 
%%endhead 

\subsection{ДНР оказалась на грани гуманитарной катастрофы из-за расстрела ВСУ трубопровода}
\label{sec:22_07_2020.news.dnr.1}
\url{https://donbasstoday.ru/dnr-okazalas-na-grani-gumanitarnoj-katastrofy-iz-za-rasstrela-vsu-truboprovoda/?fbclid=IwAR1LYvheVWbgh0yDfIC0qtbshnV6tBibaPhD5z85kWJlF873iCqWewKsbsM}
  
\vspace{0.5cm}
{\small\LaTeX section: \verb|22_07_2020.news.dnr.1| project: \verb|letopis| rootid: \verb|p_saintrussia|}
\vspace{0.5cm}

Из-за повреждения участков водонапорного трубопровода д.2300 мм. и д.2100 мм.
от насосной станции третьего подъема канала «Северский Донец --- Донбасс» может
произойти более серьёзное ЧП, передаёт корреспондент «Донбасс Сегодня» со
ссылкой на СЦКК ДНР.

«Происходит колоссальная утечка питьевой воды --- 65 000 м³ в час, — подчеркнули
в СЦКК. — На данный момент насосная станция 3-го подъема остановлена,
водоснабжение части территории Республики осуществляется через резервные линии
Верхнекальмиусского водохранилища в объеме 15 000 м³, подача воды сокращена
более чем на 50\%.

Представительством ДНР в СЦКК срочно запросило гарантии безопасности для устранения порыва.

«В ближайшее время будет остановлена ДФС, подача воды сократится по всей
территории Республики. Возобновление работы ДФС станет возможным после
восстановления поврежденного водовода», — говорят в СЦКК.
  
