% vim: keymap=russian-jcukenwin
%%beginhead 
 
%%file 17_07_2020.fb.lnr.3
%%parent 17_07_2020
 
%%endhead 

\subsection{КОРОНАВИРУС ПРОДОЛЖАЕТ ВЫКАШИВАТЬ СЕМЬИ}
\label{sec:17_07_2020.fb.lnr.3}
\url{https://www.facebook.com/groups/LNRGUMO/permalink/2854503531327869/}
  
\vspace{0.5cm}
{\ifDEBUG\small\LaTeX~section: \verb|17_07_2020.fb.lnr.3| project: \verb|letopis| rootid: \verb|p_saintrussia|\fi}
\vspace{0.5cm}

«Фото на опознание прислали в WhatsApp».  Монолог сибирячки, похоронившей из-за
ковида бабушку и дедушку

Пока одни толпятся на пляжах и отказываются надевать маски в метро, другие
ХОРОНЯТ РОДНЫХ

За ежедневными сухими цифрами оперштаба в 2-4 смерти скрываются настоящие
трагедии.  Ещё больше — тех, кто не попадает в официальную статистику.

Среди таких — бабушка и дедушка сибирячки Марины.  Оба заразились
коронавирусной инфекцией, попали в больницы и умерли с интервалом в несколько
дней.

Родственникам не дали попрощаться с ними, опознание проходило по WhatsApp, а
похороны — в замотанных полиэтиленом закрытых гробах.

Публикуем монолог девушки, которая потеряла двух самых близких людей.

ЧИТАТЬ ДАЛЕЕ ПО ССЫЛКЕ: 
