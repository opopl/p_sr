% vim: keymap=russian-jcukenwin
%%beginhead 
 
%%file 19_04_2021.fb.respublikalnr.3.poezia
%%parent 19_04_2021
 
%%url https://www.facebook.com/groups/respublikalnr/permalink/798978540737939/
 
%%author 
%%author_id 
%%author_url 
 
%%tags 
%%title 
 
%%endhead 

\subsection{Газета \enquote{Республика} (№15, 2021г).  Мизансцена.  Антология современной поэзии}
\Purl{https://www.facebook.com/groups/respublikalnr/permalink/798978540737939/}

\ifcmt
  pic https://scontent-frt3-1.xx.fbcdn.net/v/t1.6435-9/175947461_124214133092028_5572733441939371620_n.jpg?_nc_cat=106&ccb=1-3&_nc_sid=825194&_nc_ohc=zshtmfpTtyoAX_4YiW9&_nc_ht=scontent-frt3-1.xx&oh=e7507a9984bd29e9e367703bfb64eac3&oe=60A4B10C
\fi

Артисты Луганского академического русского драматического театра имени Павла
Луспекаева в рамках проекта «Культурная среда» познакомили театралов с
творчеством поэтов-современников.

Концепция апрельской встречи основана на тезисе: для поэта слово – глоток
свежего воздуха. Опираясь на это утверждение, появилось название программы
«Воздуха, воздуха…».

– Артистам было предоставлено право выбирать любой материал самим. В принципе,
вся «Культурная среда» соткана из предложений артистов, – рассказал организатор
и постановщик программы, актёр театра Владислав Кузьминых.

Эта встреча стала своеобразным ликбезом, знакомством с современными поэтами,
которые зачастую мало известны широкой общественности. Кто-то читал стихи,
которые не требуют лишних нагромождений, кто-то усиливал эффект рифмованных
строк небольшими театральными этюдами.

– Стихи разные. Какие-то кажутся слишком поверхностными, не имеющими глубокой
проблематики, но они подкупают своей простотой, доступностью. Есть
произведения, в которые заложена глубокая мысль. Написаны они витиеватым
поэтическим стилем и нам – читателям, зрителям, слушателям – нужно домысливать,
что хотел сказать автор, – объяснил Владислав Кузьминых.

Очередной эксперимент артистов театра имени Луспекаева завершился положительно,
и луганчане узнали, что в современной поэзии есть такие талантливые авторы, как
Игорь Мазунин, Валентина Беляева, Анна Тукина, Андрей Добрынин и Игорь Царёв.

Надежда ПЕРЕСВЕТ, Фото Ксении ЩЕРБИНЫ, ГАЗЕТА "РЕСПУБЛИКА" (№15, 2021г).

\verb|#газета #республика #театр_ЛНР #культурная_среда|
