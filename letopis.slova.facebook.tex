% vim: keymap=russian-jcukenwin
%%beginhead 
 
%%file slova.facebook
%%parent slova
 
%%url 
 
%%author_id 
%%date 
 
%%tags 
%%title 
 
%%endhead 
\chapter{Фейсбук}

%%%cit
%%%cit_head
%%%cit_pic
%%%cit_text
The Washington Post прошелся по событиям 6 января 2021 года, когда трамписты
штурмовали Капитолий. По их мнению, \emph{Facebook} подстрекал их к этому -
хотя после и забанил аккаунт действующего президента.  Согласно внутреннему
отчету за 6 января, резко усилились жалобы на фейковые новости, их число
достигало почти 40 000 в час. В Instagram чаще всего сообщалось о
подстрекательстве к насилию под официальным аккаунтом президента
@realdonaldtrump.  При этом указывается, что компания отклонила рекомендацию
своего собственного Наблюдательного совета изучить, как ее политика
способствовала насилию.  CNN также пишет о штурме Капитолия и считает, что
\emph{Facebook} то ли намеренно, то ли случайно проигнорировала призывы к
штурму Капитолия
%%%cit_comment
%%%cit_title
\citTitle{Разжигает ненависть от Азии до Африки. Почему Facebook одновременно атаковали крупнейшие мировые СМИ}, 
Максим Минин; Екатерина Терехова, strana.news, 26.10.2021
%%%endcit

%%%cit
%%%cit_head
%%%cit_pic
\ifcmt
  pic https://icdn.lenta.ru/images/2021/08/17/11/20210817112805807/preview_3ffd3b8c4c20d094d2a737df0607fcb1.png
  @width 0.4
\fi
%%%cit_text
Любая популярная соцсеть управляется алгоритмами — автоматическими системами,
которые постоянно ищут для каждого пользователя тот контент, который точно ему
понравится. Их работа настолько сложна, что сами разработчики вряд ли в
состоянии точно определить, как в ленте пользователя появляется та или иная
публикация. Как алгоритмы влияют на людей и могут ли люди влиять на работу
алгоритмов? Об этом «Ленте.ру» в рамках масштабного спецпроекта «Алгоритм. Кто
тобой управляет?» рассказал исследователь искусственного интеллекта в
Калифорнийском университете в Беркли, бывший сотрудник подразделения \emph{Facebook}
по изучению искусственного интеллекта Марк Фаддул
%%%cit_comment
%%%cit_title
\citTitle{«У людей должен быть выбор» Как соцсети взламывают мозг человека и кто на самом деле их контролирует?: Coцсети: Интернет и СМИ: Lenta.ru}, 
Федор Тимофеев, lenta.ru, 28.10.2021
%%%endcit
