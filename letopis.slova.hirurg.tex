% vim: keymap=russian-jcukenwin
%%beginhead 
 
%%file slova.hirurg
%%parent slova
 
%%url 
 
%%author 
%%author_id 
%%author_url 
 
%%tags 
%%title 
 
%%endhead 
\chapter{Хирург}
\label{sec:slova.hirurg}

%%%cit
%%%cit_head
%%%cit_pic
%%%cit_text
У Вінниці є держаний медичний університет, який носить ім’я Миколи Пирогова –
видатного російського \emph{хірурга}. Микола Пирогов народився у Москві, а закінчив
своє життя у селі Вишня, яке нині входить в межі міста Вінниці.
Вінницький медичний університет почав свою історію з інституту у 1934 році.
1960-го вищому навчальному закладу присвоєно ім’я М.І.Пирогова. Наукова та
лікарська діяльність відомого росіянина Пирогова у Вінниці, сакралізація імені
вченого радянською владою дало потужний поштовх розвитку медицини провінційному
містечку, яке набуло статусу центру медичної освіти союзного значення. Інерцією
тих часів Вінниця користується по сьогодні. Ім’я росіянина Пирогова донині
служить прикладом та професійним орієнтиром для українських студентів та
молодих лікарів
%%%cit_comment
%%%cit_title
\citTitle{Здатність бачити велике в чужому та ворожому як ознака дорослості}, 
Анатолій Якименко, analytics.hvylya.net, 04.08.2021
%%%endcit
