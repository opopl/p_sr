% vim: keymap=russian-jcukenwin
%%beginhead 
 
%%file 17_02_2022.fb.fb_group.story_kiev_ua.1.dzvonar
%%parent 17_02_2022
 
%%url https://www.facebook.com/groups/story.kiev.ua/posts/1863353030528156
 
%%author_id fb_group.story_kiev_ua,grabar_sergij.kiev
%%date 
 
%%tags 1240,kiev
%%title Дзвонар
 
%%endhead 
 
\subsection{Дзвонар}
\label{sec:17_02_2022.fb.fb_group.story_kiev_ua.1.dzvonar}
 
\Purl{https://www.facebook.com/groups/story.kiev.ua/posts/1863353030528156}
\ifcmt
 author_begin
   author_id fb_group.story_kiev_ua,grabar_sergij.kiev
 author_end
\fi

Дзвонар

...імені його ніхто не знав. Між собою називали Дзвонарем.

Коли прийшов до монастиря, мабуть, і сам не пам'ятав. Життя це до душі припало.
Відчував себе затишно за товстезними мурами.

\ii{17_02_2022.fb.fb_group.story_kiev_ua.1.dzvonar.pic.1}

Скільки йому років? Він і сам інколи замислювався. Начебто і не молодий, але, з
іншого боку, відчуття молодості не минало. А у монастирі відлік життя взагалі
зупинився.

Ні, він не збирався у ченці, навпаки, з його характером відчайдушним, веселим
чернече життя було не для нього.

Що ж його штовхнуло до обителі? Він не розповідав. Звичайно, у кожного
свійпривід. У нього — одному йому відома таємниця.

Цього року було неспокійно. І у попередні роки на місто чинилися набіги, та
цього року було якось інакше. У повітрі стояло передчуття лиха. І воно не
забарилося.

Лихо з обличчям багатотисячного війська, що прийшло зі Сходу, обличчям
косооким, різким, напівдиким. Це лихо все навкруги нищило, жерло, ґвалтувало,
калічило.

Спочатку сподівалися - храми врятують. За стінами їхніми боронилися. Та де там.
Не встояли твердині. І загинули всі. Хто хотів зберегти родину — її втратив,
хто прагнув зберегти себе — був закатований. Лихо прийшло і до монастиря. Ніщо
не зупинило дикунів — ні мури, ні хрести на соборах, ні шалене калатання
дзвонів. Монастир був фортецею, захисником, тане встояв. І ченці — хто куди.

Монастир-пустка...

Він любив дзвони. Тож і у монастирі допомагав дзвонарям сповіщати про свято і
тугу, про трапезу і Всеношну. Він першим і побачив це величезне дике лихо, що
сунуло зі Сходу. Він першим і сповістив про нього. Але що з того? Лихові
косоокому, зі злостивою посмішкою і солодко-хтивим поглядом, кінця-краю не було
видно.

Як він тоді калатав у дзвони, як попереджав і братію, і паству — всіх, але
проти такої сили не встояти. Тепер пустка — і у місті, і у обителі.

До монастиря він, ховаючись, наче звір йдекожного дня. Хоч,які хитрі ці
косоокі, але він хитріший. Він завжди тут, на Святій землі. Він бачив, як вони,
маленькі, коротконогі, стягали хреста з Головної церкви, як хапали все, що
могли з монастирських келій та ризниці, як тягнули, невідомо на що їм потрібні
дзвони, як по'їдали все живе, що бігало, літало у місцевих хащах, плавало у
Великій річці або росло на крутих пагорбах. Братія подейкувала, що навіть
сарана, яка інколи винищувала ці землі, і та зникла і більше не поверталася
після нашестя косооких. Вони були жахливішими за сарану.

Братія... Вона ховалася в навколишніх нетрях — хто де міг. А він кожного дня
після заходу сонця скликав їх на Вечірню. І вони сходилися — спочатку один,
двоє, а дедалі більше й більше.

У напівзруйнованому соборі вони молилися за спасіння свого краю, за спасіння
душ вірних і тих, що відійшли, і тих, що ще живі. Дзвонів у нього не було.
Старовинне било, яке зберіглося у монастирі ще з часів його заснування, знову
стало в пригоді. І щоразу він скликав ним братію тавірних — усіх, хто не
втратив духу.

А потім він ховав свою святиню, яка була дорожча від усіх скарбів світу, і
ховався сам з тим, щоби наступного дня знову зібрати всіх разом.

Імені його ніхто не знав. Між собою називали Дзвонарем.

У літо 6748 року, від Створення Світу, 1240 року від Різдва Христового до Києва
прийшло лихо. Місто перетворилося на пустку. Лише у Печерському монастирі
невідомий Дзвонар день за днем скликав людей до молитви.
