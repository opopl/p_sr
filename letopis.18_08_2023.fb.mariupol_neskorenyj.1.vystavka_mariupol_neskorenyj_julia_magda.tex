%%beginhead 
 
%%file 18_08_2023.fb.mariupol_neskorenyj.1.vystavka_mariupol_neskorenyj_julia_magda
%%parent 18_08_2023
 
%%url https://www.facebook.com/100066312837201/posts/pfbid03rYqyEackgB7FDvVwy47Vp2Aixtp18Fmz5W8WrndVkEix5iPrPhxoBrx6x2BF37wl
 
%%author_id mariupol_neskorenyj
%%date 18_08_2023
 
%%tags 
%%title Виставка "Маріуполь нескорений" - Юлія Магда
 
%%endhead 

\subsection{Виставка \enquote{Маріуполь нескорений} - Юлія Магда}
\label{sec:18_08_2023.fb.mariupol_neskorenyj.1.vystavka_mariupol_neskorenyj_julia_magda}

\Purl{https://www.facebook.com/100066312837201/posts/pfbid03rYqyEackgB7FDvVwy47Vp2Aixtp18Fmz5W8WrndVkEix5iPrPhxoBrx6x2BF37wl}
\ifcmt
 author_begin
   author_id mariupol_neskorenyj
 author_end
\fi

✨️ У виставковому проєкті \enquote{Маріуполь нескорений} бере участь і талановита
художниця та поетка Yuliia Mahda. 

Вона народилася у м. Барвінкове, Харківська область.  Прожила у Маріуполі 29
років.

🎨 Юлія Магда працює у змішаній техніці, пише картини олією, акрилом. 

✅️ До війни працювала в університеті зі студентами у м.Маріуполі, була
волонтеркою, співорганізаторкою культурних та освітніх заходів.

У 2022 провела більше місяця у блокадному Маріуполі,  вийшла з міста пішки
разом з родиною.

24 лютого 2022 року частина її робіт з пластикових кришечок знаходилась на
виставці у маріупольському культурному центрі \enquote{Вежа}, які згодом порізані людьми
на шматки, щоб закрити вікна \enquote{Вежі}, рятуючись від обстрілів. Інша частина
робіт попала під бомби. Картини олією зникли.😪

🖇\enquote{Для мене честь брати участь  у виставці \enquote{Маріуполь Нескорений}. Вважаю, що
не можна забувати про подвиг нашого блокадного міста,  який прийняв на себе
величезний удар на початку повномасштабного вторгнення, про мирних мешканців,
волонтерів, медиків, наших Захисників, про оборону Азовсталі. Не можна також
забувати, у яких умовах знаходяться наші люди в окупації зараз, не можна
вибачати загарбників та їх злочини та треба постійно нагадувати усьому світу
про пережите нами-мешканцями Маріуполя та про те, що наше місто чекає на
визволення, деокупацію та відбудову. І це нагадування можна зробити і мовою
мистецтва, через картини}, - наголосила художниця.

 🖼❤️ Роботи Юлії Магди вражають не стільки майстерністю та яскравістю, як
 глибоким розумінням трагедії міста, що стало символом нескореності та
 боротьби.  Водночас полотна художниці – вияв великої любові до рідного міста й
 того, де вона живе зараз.
