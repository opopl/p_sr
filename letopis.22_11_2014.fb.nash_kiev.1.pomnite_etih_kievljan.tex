% vim: keymap=russian-jcukenwin
%%beginhead 
 
%%file 22_11_2014.fb.nash_kiev.1.pomnite_etih_kievljan
%%parent 22_11_2014
 
%%url https://www.facebook.com/nashkiev/posts/10152359911116012
 
%%author_id nash_kiev
%%date 
 
%%tags izbienie,kiev,kievljane,maidan2,nasilie,revgidnosti,ukraina
%%title Помните этих киевлян?
 
%%endhead 
 
\subsection{Помните этих киевлян?}
\label{sec:22_11_2014.fb.nash_kiev.1.pomnite_etih_kievljan}
 
\Purl{https://www.facebook.com/nashkiev/posts/10152359911116012}
\ifcmt
 author_begin
   author_id nash_kiev
 author_end
\fi

Помните этих киевлян?

На Майдан Кузнецовы ходили с ноября всей семьей.

— Когда обстановка стала накаляться, я попыталась отговорить сына с мужем от
походов на Майдан, — признается жена Николая Юрьевича 56-летняя Алла Шумская,
доцент Киевского политехнического института. — Но супруг сказал:
«Двадцатилетние дети противостоят тоталитарному режиму, как же я смогу в это
время спокойно сидеть дома».

18 февраля Николай Кузнецов с сыном, как обычно, пошли в институт, а после
обеда поспешили на Майдан.

\begin{multicols}{2}
\ii{22_11_2014.fb.nash_kiev.1.pomnite_etih_kievljan.pic.1}
\ii{22_11_2014.fb.nash_kiev.1.pomnite_etih_kievljan.pic.1.cmt}
\end{multicols}

— На улицу Институтскую мы пришли за несколько минут до начала атаки, —
вспоминает 59-летний Николай Юрьевич. — Неожиданно силовики стали наступать,
майдановцы бросились назад. Люди спотыкались и падали. Спецназ напирал
стремительно, мы с Игорем оказались зажатыми со всех сторон. Каждый из
спецназовцев, пробегая мимо, пытался несколько раз ударить дубинкой. Это были
не резиновые дубинки, а пластиковые, у которых более мощная сила удара. Нас
били по плечам, по почкам, но преимущественно целились в голову.

Позже окажется, что демонстрантов не только избивали, их убивали.

— Я боялся потерять в толпе сына и крепко держал его за руку,— продолжает
мужчина. — Мы так и бежали, схватившись за руки, а в это время на нас сыпались
удары. Из ран хлестала кровь, но это, похоже, только подзадоривало
«беркутовцев». Метров пятьдесят мы преодолели, продержавшись на ногах. Но возле
какой-то тумбы нас повалили на землю, били уже не только дубинками, но и
ногами.

И все-таки мы упали очень удачно. Хотя глаза были залиты кровью, я разглядел в
двух метрах от себя неглубокий подвальчик. И дал знак сыну отползать туда.
«Беркутовцы», пробегавшие мимо, заглядывали в подвальчик, но, чтобы ударить
нас, им нужно было приноровиться.

Когда наступило затишье, мы вышли из своего убежища. Нас заметил «беркутовец» —
он велел показать содержимое наших карманов, удосужившись объяснить, что его
интересует оружие. У меня, кроме пары жетонов на метро, больше ничего не
оказалось. Игорь достал из кармана часы, однако положить их обратно у него уже
не было сил. Пройдя двориками, мы хотели покинуть опасное место. По дороге
встретили медиков-волонтеров, которые оказали нам первую помощь, перевязав
головы. Сил идти дальше ни у меня, ни у сына не было, мы еле двигались.
Наконец, добрались до одного из подъездов, чтобы отдышаться и передохнуть.

Когда на улице стало более-менее спокойно, Николай Кузнецов позвонил жене.

— Коля сказал, что с ними все в порядке, они прячутся в одном из домов, скоро
приедут, — говорит Алла Антоновна. — Но я чувствовала: что-то случилось.
Странно вела себя наша собака, она жалобно скулила и ходила из комнаты в
комнату. Я включила телевизор и увидела, как «беркутовцы» избивают
демонстрантов. Стала вызывать такси, чтобы привезти мужа с сыном домой. Как
только диспетчеры узнавали адрес, напрочь отказывались принимать заказ.
«Знаете, что там сейчас творится? — интересовались у меня. — Вы бы послали туда
своего сына?» В отчаянии я вышла на улицу и, не зная, что делать, стала бродить
в поисках машины. Наконец водитель частного такси согласился ехать. Из-за
огромных пробок мы с трудом добрались к тому месту, где меня ждали муж и сын.
На улице уже стемнело, поэтому я не сразу увидела, в каком состоянии были оба.
От радости, что они живы, я прижала их к себе и… не смогла оторваться —
приклеилась. Липкой оказалась кровь, которой была залита вся одежда сына и
мужа. Только тогда увидела, что у них перевязаны головы, а у мужа болтается
рука. Оба еле стояли на ногах.

Дома оказалось, что у Коли и Игоря болит абсолютно все — они не могли ни
сидеть, ни стоять, ни лежать. Все тело ныло, гудела голова. Вызывать «скорую» я
побоялась, решила, что утром отвезу их к знакомому врачу. И стала сама
обрабатывать раны.

Николаю Кузнецову пробили голову в двух местах, медикам пришлось наложить
десять швов. Кроме того, «беркутовцы» сломали мужчине лопатку, ему долго
пришлось ходить в гипсе. Игорю тоже досталось — парню латали голову в трех
местах. От побоев у мужчин были синие плечи и спина, остались следы от
резиновых пуль.
