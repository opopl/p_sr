% vim: keymap=russian-jcukenwin
%%beginhead 
 
%%file 29_03_2022.fb.solovjov_mikita.harkov.demsokyra.2.o_metro_ubezhischah
%%parent 29_03_2022
 
%%url https://www.facebook.com/Mikita.Solovyov/posts/7323317367738613
 
%%author_id solovjov_mikita.harkov.demsokyra
%%date 
 
%%tags 
%%title О метро-убежищах
 
%%endhead 
 
\subsection{О метро-убежищах}
\label{sec:29_03_2022.fb.solovjov_mikita.harkov.demsokyra.2.o_metro_ubezhischah}
 
\Purl{https://www.facebook.com/Mikita.Solovyov/posts/7323317367738613}
\ifcmt
 author_begin
   author_id solovjov_mikita.harkov.demsokyra
 author_end
\fi

О метро-убежищах. 

Еще одна тема, по которой мнения будут не просто разные, а строго
противоположные, мне кажется. Но вопрос уже назрел и перезрел. И что-то с ним
нужно уже решать. 

Когда в самом начале войны метро было остановлено, а станции начали работать в
режиме убежищ, это было абсолютно правильно и необходимо. Потому что не было
понимания что происходит, на какой это все срок, как и от чего нужно спасаться.
Кроме того, первые две недели ситуация с эвакуацией была жуткая. И для многих
убежища в метро было единственным способом в безопасности переждать налеты. 

Но сейчас ситуация изменилась довольно кардинально. 

1) С эвакуацией сейчас никаких серьезных проблем нет. Поездами, автобусами
можно уехать на запад Украины или в страны ЕС достаточно спокойно. Места есть.
Волонтеры помогают с выездом на вокзал и к местам сборов автобусов. 

2) Ситуация с обстрелами уже стала понятной. Сотни тысяч людей живут в своих
домах в Харькове. И если не считать нескольких микрорайонов города, то все мы
живем в относительной безопасности. 

3) Уже стало понятно, что обстрелы и бомбежки это наша реальность не на день
или два, а в лучшем случае на недели. И длятся они уже месяц. То есть уже
очевидно, что речь не о \enquote{переждать}, а о \enquote{жить в этих
условиях}. 

При этом станции метро как были, так и остались абсолютно неприспособленными
для длительного пребывания людей помещениями. Условно говоря, в них можно
спокойно пересидеть один налет пару часов. Можно относительно спокойно
переночевать одну-две ночи. Но никак не оставаться там неделями и месяцами. 

Постоянное пребывание людей в метро неделями не решает ни одной проблемы, при
этом создает много новых. Больше находящихся там плюс-минус постоянно не имеют
возможности даже просто регулярно мыться. Скопление людей уже по словам
некоторых работающих там волонтеров приводит к росту респираторных заболеваний.
И на мой абсолютно непрофессиональный взгляд, там уже существует и постоянно
растет риск вспышки какой-нибудь инфекционной хрени.

Точно также, все кто регулярно общается с этими \enquote{детьми подземелья} говорят,
что у них уже буйным цветом идут нездоровые процессы с психикой. И в этих
условиях было бы удивительно, если бы они не начались. У людей нет никакого
осмысленного занятия. Люди практически без движения и любого вида активности.
На большинстве станций довольно хреново со связью и люди себя чувствуют
отрезанными от мира. 

О происходящем наверху судят по пабликам и СМИ, в которых только фото
разрушений. И у них складывается картина, что здесь уже постапокалипсис.
Грозный. Алеппо и Мариуполь. Причем еще и работает положительная обратная
связь. Люди в таком закрытом комьюнити и настолько нестобильном состоянии
психики накручивают себя и друг друга, пересказывая друг другу и самим себе по
кругу все ужасы. Панические слухи распространяются там с дикой скоростью одни
фантастичнее других. 

Я позавчера и вчера много говорил по телефону с знакомой, которая там сидит с
семьей уже больше трех недель. И она отказывалась мне верить, когда я объяснял,
что у меня в квартире есть свет, газ, вода и интернет и все стекла в квартире
целые. По ее мнению, таких квартир в Харькове не осталось. Убедить ее в том,
что дорога домой в четыре квартала. а потом с волонтерами до вокзала не
представляет серьезной опасности для жизни мне было безумно сложно. И я не
уверен, что мне это до конца удалось. 

И я понимаю, что ситуация будет дальше становиться только хуже, а не лучше. И
со здоровьем сидящих там. И с состоянием их психики. 

Что я предлагаю? 

1) Принять волевое решение и объявить, что станции метро в течение скажем
недели перестанут работать в режиме круглосуточного убежища. Или вообще в
режиме убежища. 

2) Всем сидящим там предложить системную помощь в эвакуации. Вплоть до
сопровождения домой, помощи в сборе вещей и сопровождении на вокзал и посадки в
поезд. 

3) Каждого под любым предлогом за это время хотя бы по разу вытащить на
поверхность не просто на подышать возле метро, а вынудив отойти хотя бы на
500-600 метров. На всех станциях метро кроме Гертруды и, возможно, Студенческой
с ХТЗ, картина окружающей действительности у людей должна сильно измениться. 

4) Тем, кто оттуда выберется, обеспечить по их запросу на 2-3 дня временное
жилье. Это может быть необходимо в ситуации, если дом сильно пострадал от
обстрелов и бомбежек. 

5) Разъяснить людям порядок и возможности по эвакуации на запад Украины или в
страны ЕС. Порядок приема там беженцев и т.д. 

6) Закрыть таки метро в качестве убежищ. 

Я прекрасно понимаю, что большинство сидящих там сейчас в первую очередь просто
не могут принять какое-то решение. И продолжают сидеть просто из страха и
пребывая в некотором ступоре. Я уверен, что большая часть из них при подобных
действиях примет решение об эвакуации. И я уверен, что условия в эвакуации даже
в самых переполненных местах приема ВПЛ на западе Украины окажутся значительно
лучше, чем сейчас у них в метро. Многие же останутся в Харькове у себя дома. И
вернутся за какое-то разумное время к если не совсем нормальной, то вполе
понятной жизни, которой сейчас в городе живем мы все. 

Но я убежден, что дальнейшее сидение в метро только ухудшает ситуацию по всем
параметрам. 

Буду благодарен за содержательную полемику. Желающих рассказать мне о
смертельной опасности жизни в Харькове просьба не беспокоиться и проходить
мимо.

\ii{29_03_2022.fb.solovjov_mikita.harkov.demsokyra.2.o_metro_ubezhischah.cmt}
\ii{29_03_2022.fb.solovjov_mikita.harkov.demsokyra.2.o_metro_ubezhischah.cmtx}
