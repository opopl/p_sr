% vim: keymap=russian-jcukenwin
%%beginhead 
 
%%file 12_12_2021.fb.fb_group.story_kiev_ua.3.zhmurki
%%parent 12_12_2021
 
%%url https://www.facebook.com/groups/story.kiev.ua/posts/1817450055118454
 
%%author_id fb_group.story_kiev_ua,bergelson_aleksandr.kiev
%%date 
 
%%tags gorod,kiev
%%title СКАЗОЧНЫЕ НОЧНЫЕ ЖМУРКИ...
 
%%endhead 
 
\subsection{СКАЗОЧНЫЕ НОЧНЫЕ ЖМУРКИ...}
\label{sec:12_12_2021.fb.fb_group.story_kiev_ua.3.zhmurki}
 
\Purl{https://www.facebook.com/groups/story.kiev.ua/posts/1817450055118454}
\ifcmt
 author_begin
   author_id fb_group.story_kiev_ua,bergelson_aleksandr.kiev
 author_end
\fi

СКАЗОЧНЫЕ НОЧНЫЕ ЖМУРКИ...

Когда там далеко – за горизонтом – мутный, размытый и распухший от влаги
солнечный диск уже втянула в себя земная твердь...

Когда он, уже провалившийся в земные тар-тарары, перестал сопротивляться и
хвататься последними лучами-ладонями за тяжелые и плотно-беспросветные тучи,
сжимая их до красного каления...

Когда Город устал нервничать и бросаться в суетливо снующие и вяло ползущие
разноцветно-серые бесконечные ленты из машин грязной водой и мокрой грязью
со всех возможных сторон и направлений...

Когда усталые и хмурые, закутанные и закапюшоненные люди с оттягивающими
руки и плечи раздутыми к вечеру пакетами и рюкзаками наконец-то расползлись
по ячейкам своих человейников...

Когда уже догорели почти все свечи окон домов, кое-где дотлевая синевой
экранов телевизоров сквозь плотно задернутые шторы, выпускающие наружу
только световые контуры и свято храня накопленное домашнее тепло и уют...

Вот тогда... Только тогда... Город нажимает на крышку своей гигантской
бутыли-баллона... И наносит на улицы и дома толстый и ровный слой тумана... Как
очистительной пены, которая клубится свежими волнами, наползая и покрывая
все вокруг... А к утру осядет и смоет всю накопившуюся за день грязь, суету и
усталость быстрым, упругим и крепким утренним дождем... Подсушив жирненькую
влажность асфальта легким морозным предрассветным ветерком...

И вот тогда... Только тогда... Я выйду в ночь и проедусь по пустынным улицам...
Рассекая ледоколом туман и отбрасывая его лучами фар... Не торопясь и не
спеша... И у нас с Городом наконец-то будет немножко времени на спокойно
поболтать... На поделиться тем, что уже ушло... И поговорить за то, что еще
впереди... И чего от него ждать...

И немножко пошалить... И поиграть пространством... И немножко временем... Как
взрослые на какой-то момент решившие окунуться в сказочное и беззаботное
детство... Этакие волшебные ночные жмурки... 

Когда Он показывает и открывает мне порталы-коридоры в другие миры, чуть
подсвечивая расступившийся по взмаху Его руки-луча фонарного света  туман... С
мерцающей неизвестностью там вдали... А я волен выбирать в какой из миров я
сегодня окунусь... И где вынырну, вернувшись обратно... 

И как иногда нужно Его, расшалившегося, одергивать... А то и правда –
разыграется и умыкнет куда-то в свои недра телебашню, позабыв вернуть к утру
на ее законное место... Или бросит под ноги каштанчик, неизвестно откуда
взявшийся посреди декабря и свеже-блестящий, как кошачьи ушки... И тогда наша
с Ним дружеская тайна перестанет быть только нашей... А нам это обоим ни к
чему...

\ii{12_12_2021.fb.fb_group.story_kiev_ua.3.zhmurki.cmt}
