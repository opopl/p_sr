% vim: keymap=russian-jcukenwin
%%beginhead 
 
%%file 07_10_2021.fb.nikonov_sergej.1.pikta_donbass_2_trudovskie_petrovka
%%parent 07_10_2021
 
%%url https://www.facebook.com/alexelsevier/posts/1597508787261097
 
%%author_id nikonov_sergej,pikta_svetlana
%%date 
 
%%tags donbass,pikta_svetlana,poezdka,reportazh,vojna
%%title Светлана Пикта - поездка на Донбасс - 2. ТРУДОВСКИЕ. ПЕТРОВКА
 
%%endhead 
 
\subsection{Светлана Пикта - поездка на Донбасс - 2. ТРУДОВСКИЕ. ПЕТРОВКА}
\label{sec:07_10_2021.fb.nikonov_sergej.1.pikta_donbass_2_trudovskie_petrovka}
 
\Purl{https://www.facebook.com/alexelsevier/posts/1597508787261097}
\ifcmt
 author_begin
   author_id nikonov_sergej,pikta_svetlana
 author_end
\fi

Поездка журналистки Светланы Пикты на Донбасс. Подаю как репортаж и точку
зрения.  В оригинале сделаны отдельные видео. Я объединил, чтобы разместить
здесь. 

2. ТРУДОВСКИЕ. ПЕТРОВКА

- Вот, Света, мы заехали туда где волки с\#\#ать боятся. Здесь 300 метров до
позиций Ukr, можешь выйти помахать им ручкой, - смеётся Андрей, однако машину
не останавливает и выйти мне не даёт.

\ifcmt
  ig https://scontent-mia3-1.xx.fbcdn.net/v/t1.6435-9/244656811_1597567853921857_318278480242544497_n.jpg?_nc_cat=104&_nc_rgb565=1&ccb=1-5&_nc_sid=730e14&_nc_ohc=otws4woS_4YAX8uhg1m&_nc_oc=AQm6noCUAz6AFkiRxgPEylVWaCfK7CV3ZE4R-7S-35Qt1pqUV_XjhbuJrIEbcR_RfGQ&_nc_ht=scontent-mia3-1.xx&oh=fb6547023cc79ac577c743972f141b45&oe=6185B5E3
  @width 0.4
  %@wrap \parpic[r]
  @wrap \InsertBoxR{0}
\fi

- Помахал тут один корреспондент, так они потом такую охоту устроили за нами.
Они нас прекрасно сейчас видят через снайперскую винтовку.

\ifcmt
  ig https://scontent-mia3-2.xx.fbcdn.net/v/t1.6435-9/244631359_1597568150588494_5321046433502212187_n.jpg?_nc_cat=102&_nc_rgb565=1&ccb=1-5&_nc_sid=730e14&_nc_ohc=G-2zJy55b3EAX9AQTcU&_nc_ht=scontent-mia3-2.xx&oh=2b92b98499c7a3cbcccb72b6dcca0f1d&oe=6185A81E
  @width 0.4
  %@wrap \parpic[r]
  @wrap \InsertBoxR{0}
\fi

Мы видим позиции Ukr отчётливо, как друг друга. Вот балка, вот террикон, вот
позиции. Наверно это самое неприятное чувство, которое я испытала там -
отчётливое ощущение, что за тобой наблюдают через прицел. Захотят - убьют,
захотят - испугают.

- Сколько раз попадал под обстрел? - интересуюсь.
- Бесчисленное количество раз.
- А расписание у них есть?

Смеётся. Он постоянно смеётся, этот удивительный человек, волонтёр Андрей ,
посвятивший свою жизнь ежедневным поездкам по линии разграничения, чтобы люди,
которые здесь живут, не чувствовали себя брошенными.

\ifcmt
  ig https://scontent-mia3-2.xx.fbcdn.net/v/t1.6435-9/244612149_1597568090588500_6712957479916036685_n.jpg?_nc_cat=109&_nc_rgb565=1&ccb=1-5&_nc_sid=730e14&_nc_ohc=0MSqB5riPP8AX9QpP1O&tn=lCYVFeHcTIAFcAzi&_nc_ht=scontent-mia3-2.xx&oh=5f85ffa2f421c46ec8726c019abe83a1&oe=6182D0B2
  @width 0.4
  %@wrap \parpic[r]
  @wrap \InsertBoxR{0}
\fi

- Нет, Света, расписания у них нет. Конечно, в основном обстреливают утром и
вечером, в тёмное время суток. Но и днём бывает. Вот сейчас время
послеобеденное, может начаться в любой момент.

- А как себя вести при обстреле?
- Если они периодические, то постараться смотаться побыстрее, если постоянно кроют, то только лечь и лежать. Большинство людей погибает при перемещении, пытаясь убежать. Нужно лечь и лежать, накрыв голову.
- Что-то мы тогда не слишком лежали, когда нас накрыли - смеётся наш проводник Алёна.
- Вот, видишь магазин? Когда здесь по нам жахнули, мы давай сматываться, я ключ не могу вставить, колотит. Едем - а они за нами: накрывают. Мы еле выехали, доехали до этого магазина. Заходим, нас трясёт, а там такая спокойная продавщица:
- Что, водочки?
- Откуда вы знаете??
- Да слышали, как вас там накрыли. Шо ж вы туда попёрлись?
...
Подъезжаем к бабушке Лере. Те же "пейзажи": иссечённые ворота, сгоревшие во
время обстрела дома. Ворота закрыты изнутри. Опытный Андрей чудным образом их
открывает. Мы с Алёной смеёмся:

- Кажется, Андрей, ты приобрёл здесь ещё одну профессию.

Заходим. Знакомые с детства интерьеры: рукомойник, трюмо, буфеты, Николай
Угодник, фотографии, коврик с лебедями. Обои с надписями Paris, Italy...
Бабушка Лера лежачая, к ней как раз пришла в гости её подруга.

- Бабушка, вы с какого года?
- С 39-го.
- Так вы помните ещё ту войну?
- Конечно! Конец её вообще хорошо помню. Вот бы конец ЭТОЙ застать!
- А перемирия бывали? Вот СМИ говорили, мол, "объявлено перемирие" - время от времени...
- Перемирие? Как же! Бывало! Помню целых пять дней не стреляли.

Оставляем продукты и деньги здесь. Едем к следующей лежачей. Там совсем
"жесть": в доме холодно, стены в трещинах, окна дырявые. Раиса, так зовут её,
79 лет, совсем не встаёт. Родственников нет. Приходят только соседи иногда.
Печальные блуждающие глаза, которые не забыть. Итог жизни: одиночество и
беспомощность под обстрелами. На столике, заваленном лекарствами бросаются в
глаза бусы из искусственного жемчуга, а над столом с чёрно-белой фотографии
смотрит на нас женщина редкой красоты... Она в таком беспросветном унынии , что
мне хочется обнять её. Но я не решаюсь, я не знаю, какая будет реакция, не
будет ли ей хуже. Ведь я уеду, а она останется.

Третий адрес - бабушка Люба, 92 года, на ногах, весёлая, пишет стихи.

Выходит к нам на улицу , по улице бегают дети, трое красивых мальчишек.

- А ну оденься!! - кричит одному из них наша проводница.
- Это её внук, - поясняет Андрей.

Присматриваюсь к одному из мальчишек, белобрысый, голубоглазый, как из описания
Малороссии Павла Алеппского 17 века, "старчики с белыми волосами"

- Ох, а что у тебя в руках?? - пригляделась и ахнула: в руках у мальчика снаряд... Present Ukr
- Да, такие у нас тут "игрушки".
Бабушка Люба глухая, "разрывы слышу, только когда близко гэпають, а дальние и не слышу" , она протягивает листочек Алёне.
- Вот, это мой новый стих, прочитай. Ты по-украински читаешь?
Мы читаем по очереди, пытаясь разобрать каракули бабы Любы:
Vovaputin, поможи,
Ze накажи,
Вверх ногами до дерева
За \#\#\# привя
15:59
жи
...
Пан Z-y будет знать
Как с народом воевать
Странное дело: тогда я смеялась вместе со всеми. А пересматривая это видео уже дома - наревелась вволю.

По дороге назад проехали магазин, где погибло 12 человек, которых обстреляли в
день привоза хлеба, святой день для прифронтовой зоны, нелюди об этом знали.

Знали, что люди прийдут за хлебом все, чтобы не погибнуть от голода. И всё же
выпустили по ним снаряд.

Проехали ядерное бомбоубежище, в котором до сих пор живут 12 человек, лишенные крова.

- Ну вот, выезжаем, слава Богу.
- Завтра я тоже с тобой проеду, у меня завтра целый день есть, - решаю я.

Я вдруг отчётливо понимаю, что это был один из главных дней в моей жизни. Как
будто с меня стёрли пыль и затем содрали кожу. Каждый русский человек должен
быть здесь, чтобы понять, какой ценой достаётся желание оставаться русскими,
чтобы увидеть всё своими глазами.
