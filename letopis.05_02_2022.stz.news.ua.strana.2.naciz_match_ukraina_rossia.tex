% vim: keymap=russian-jcukenwin
%%beginhead 
 
%%file 05_02_2022.stz.news.ua.strana.2.naciz_match_ukraina_rossia
%%parent 05_02_2022
 
%%url https://strana.news/news/375153-match-ukraina-rossija-po-futzalu-i-krichalki-o-moskaljakh.html
 
%%author_id malahova_oksana
%%date 
 
%%tags futbol,nacionalist,nacionalizm,oskorblenie,rossia,sport,ukraina
%%title "Бий москаля, складайте трупи". Как националисты устроили скандал на матче по футзалу Украины с Россией
 
%%endhead 
 
\subsection{\enquote{Бий москаля, складайте трупи}. Как националисты устроили скандал на матче по футзалу Украины с Россией}
\label{sec:05_02_2022.stz.news.ua.strana.2.naciz_match_ukraina_rossia}
 
\Purl{https://strana.news/news/375153-match-ukraina-rossija-po-futzalu-i-krichalki-o-moskaljakh.html}
\ifcmt
 author_begin
   author_id malahova_oksana
 author_end
\fi

В Амстердаме впервые за 14 лет сборные Украины и России сыграли в футбол. В
полуфинале чемпионата Европы по футзалу Украина уступила РФ со счетом 2:3.

От матча ждали не только спортивного, но и политического накала. И проявили
его, как и ожидалось, болельщики с украинской стороны. Они оглашали трибуны
песнями о \enquote{трупах москалей}. 

В какой-то момент диктор на арене даже попросил фанатов прекратить песни.

Разбирались в скандале с кричалками, а также собрали реакцию украинских
пользователей в соцсетях.

\subsubsection{Как прошел матч}

Этот поединок привлек внимание к себе не только составом соперников. Еще раньше
\enquote{зраду} вызвало, что капитан Петр Шотурма и один из футболистов Владимир
Разуванов играли в России, да еще и за клуб коммунистической партии, который
так и называется - КПРФ.

Уже на стадии подготовки матча этот момент начали обсуждать в
националистическом лагере. Но за команду все равно решили болеть. 

Тем не менее, Украина проиграла сборной России со счетом 2:3. Матч проходил
очень напряженно, с большим количеством нереализованных моментов у обеих
сборных.

За минуту до конца матча украинцы не смогли забить пенальти. Кроме того, у них
было два верных момента в самом конце встречи, но сравнять счет наши футзалисты
так и не смогли. Тем не менее, играли они активно и вышли по итогу в \enquote{бронзовый
финал}. То есть, теперь им предстоит сыграть с аутсайдром второй полуфинальной
пары Испания - Португалия.

У украинцев забитыми голами отметились Серый и Абакшин. У российской сборной
забили Соколов, Афанасьев и Ниязов.

\subsubsection{Путин и \enquote{москали}. Кричалки на трибунах}

Если сборная Украины играла достойно, то ее фанаты (среди них преобладали
националисты) вели себя мягко говоря неадекватно.

Во время матча украинские болельщики выкрикивали националистические речевки,
включая матерную о Владимире Путине. Во время трансляции было слышно, как
болельщики на трибунах кричат \enquote{Украина превыше всего}, \enquote{Кто не скачет, тот
москаль}.

\ii{05_02_2022.stz.news.ua.strana.2.naciz_match_ukraina_rossia.pic.1}

Комментатор на \enquote{UA: Першем} даже порадовался такому поведению. Но
диктор на арене на украинском языке попросил фанатов прекратить песни
неспортивного характера. Тем не менее, болельщики продолжили петь лозунги.

Во втором тайме кричалки стали еще суровее. Фанаты спели песню \enquote{Бей
москаля}:

Бий москаля, складайте трупи.

Ще й кулемет беріть у руки.

І нову ленту заряджай.

\ii{05_02_2022.stz.news.ua.strana.2.naciz_match_ukraina_rossia.pic.2}


Как прошел матч Украина - Россия полуфинала Евро 2022 по футзалу
Как прошел матч Украина - Россия полуфинала Евро 2022 по футзалу

В Амстердаме впервые за 14 лет сборные Украины и России сыграли в футбол. В полуфинале чемпионата Европы по футзалу Украина уступила РФ со счетом 2:3.

От матча ждали не только спортивного, но и политического накала. И проявили его, как и ожидалось, болельщики с украинской стороны. Они оглашали трибуны песнями о "трупах москалей". 

В какой-то момент диктор на арене даже попросил фанатов прекратить песни.

Разбирались в скандале с кричалками, а также собрали реакцию украинских пользователей в соцсетях.
Как прошел матч

Этот поединок привлек внимание к себе не только составом соперников. Еще раньше "зраду" вызвало, что капитан Петр Шотурма и один из футболистов Владимир Разуванов играли в России, да еще и за клуб коммунистической партии, который так и называется - КПРФ.

Уже на стадии подготовки матча этот момент начали обсуждать в националистическом лагере. Но за команду все равно решили болеть. 

Тем не менее, Украина проиграла сборной России со счетом 2:3. Матч проходил очень напряженно, с большим количеством нереализованных моментов у обеих сборных.

За минуту до конца матча украинцы не смогли забить пенальти. Кроме того, у них было два верных момента в самом конце встречи, но сравнять счет наши футзалисты так и не смогли. Тем не менее, играли они активно и вышли по итогу в "бронзовый финал". То есть, теперь им предстоит сыграть с аутсайдром второй полуфинальной пары Испания - Португалия.

У украинцев забитыми голами отметились Серый и Абакшин. У российской сборной забили Соколов, Афанасьев и Ниязов.

Первый гол сборной Украины

Второй гол сборной Украины
Путин и "москали". Кричалки на трибунах

Если сборная Украины играла достойно, то ее фанаты (среди них преобладали националисты) вели себя мягко говоря неадекватно.

Во время матча украинские болельщики выкрикивали националистические речевки, включая матерную о Владимире Путине. Во время трансляции было слышно, как болельщики на трибунах кричат "Украина превыше всего", "Кто не скачет, тот москаль".

Комментатор на "UA: Першем" даже порадовался такому поведению. Но диктор на арене на украинском языке попросил фанатов прекратить песни неспортивного характера. Тем не менее, болельщики продолжили петь лозунги.

Во втором тайме кричалки стали еще суровее. Фанаты спели песню "Бей москаля":

Бий москаля, складайте трупи.

Ще й кулемет беріть у руки.

І нову ленту заряджай.

После матча председатель комитета Госдумы по физкультуре и спорту Дмитрий
Свищев заявил, что Ассоциация мини-футбола России должна подать протест на
поведение украинских фанов. 

\enquote{Это, однозначно, хорошо спланированная провокация против России. На нее
необходимо жестко отреагировать. УЕФА обязан запретить вход украинским
болельщикам на следующие матчи Евро, оштрафовать ФФУ, провести расследование и
наказать организаторов акции по полной строгости} - заявил представитель
Госдумы.

\subsubsection{\enquote{День национального позора}. Что пишут в соцсетях}

Украинцы в соцсетях комментируют матч и кричалки на трибунах.

"День национального позора Не потому, что сборная Украины по футзалу проиграла.
Играла сборная пассионарно и достойно. Но побеждает сильнейший. Это спорт.

Но матч Украина – Россия превратили из спорта в пропагандистский навоз.
Стараниями \enquote{патриотов}, должностных лиц и правых политиков.

Благодаря им и массированной кампании в стиле \enquote{победим москалей!}, матч
по футзалу, ранее не вызывавший большого интереса у публики, смотрело много
украинцев. И они увидели: нашим спортсменам запретили пожимать руки своим
спортивным соперникам. Во время исполнения гимна РФ трансляцию прервали.

Мы помним времена, когда поведение советских спортсменов регламентировалось
чиновниками от спорта, но такого, чтобы свободным людям запрещали
фотографироваться со спортсменами из США и говорить по-английски, не было.

Далее в трансляции неслись спортивные возгласы \enquote{фанатов}:
\enquote{...Смерть врагам...  уйло... убей... трупы...}. С трибун футзала в
Амстердаме на нас смотрело перекошенное от злобы лицо, почему-то обернувшееся
украинским флагом.

\enquote{Вы слышите эти возгласы? Их в прямой трансляции слышат зрители из
многих стран мира! Вот какая поддержка нашей сборной!}, - радостно сообщал
комментатор телеканала UA Перший. Комментатор, говоривший не \enquote{сборная
России}, а \enquote{команда в красной форме}. Не называя при этом фамилии
российских игроков. Комментатор телеканала, финансируемый из украинского
бюджета.

Зрители из многих стран, слава богу, не поняли, что вопили на трибунах. Иначе
они могли бы перенести впечатление от людей, которые потеряли человеческий
облик, на всех украинцев. Но весь этот перформанс видели многие украинцы.

Украинцев, эти перекошенные лица постоянно видят на улицах украинских городов.
Несмотря на непрерывную мантру власти и \enquote{соросят}, что \enquote{фашизма
у нас нет}.  Украинцев, которые в абсолютном большинстве хотят мира и свободы
говорить о том, что они думают.

И мы, украинцы, знаем, что украинцы не такие. И навязать нам такое поведение
даже через телевизор не удастся. Но есть еще одна причина называть сегодняшний
день днем национального позора. YouTube заблокировал аккаунты Первого
Независимого и UkrLive. \enquote{Патриоты} и \enquote{соросята} бросились
радоваться цензуре и уничтожению свободы слова в Украине, уподобившись стадам
на трибунах амстердамского стадиона...", - написал политический эксперт
Владимир Чемерис.

\ii{05_02_2022.stz.news.ua.strana.2.naciz_match_ukraina_rossia.pic.3}

\enquote{Футбол - это часть моей жизни. Несбывшаяся мечта детства, в силу отсутствия
способностей. Судьба потом подарила возможность побыть в футболе, не играя.
Футбол полон суеверий, их множество. Как-то можем обсудить. Пост не об этом. В
1999-м в Украине ещё спокойно показывали ОРТ. Очень хорошо помню как напыщенно
тогда там анонсировали матч Россия-Украина в группе за выход на ЕВРО 2000.
Заставкой о том, как рыдают в Украине после матча. Все помнят результат и
легендарный гол Шевченко. Я прыгал до потолка от радости. Радуясь за результат,
за свою страну и футбольной справедливости. Мне стыдно за боление украинских
болельщиков. Это дно. Вот если есть футбольный сглаз- это он. У меня чувство,
что мы болеем за разные команды. Короче, разные мы. А матч хороший. И ребята
молодцы}, - прокомментировал нардеп от ОПЗЖ Николай Скорик.

\ii{05_02_2022.stz.news.ua.strana.2.naciz_match_ukraina_rossia.pic.4}
