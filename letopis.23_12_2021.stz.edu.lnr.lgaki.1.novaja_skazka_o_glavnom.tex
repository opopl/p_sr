% vim: keymap=russian-jcukenwin
%%beginhead 
 
%%file 23_12_2021.stz.edu.lnr.lgaki.1.novaja_skazka_o_glavnom
%%parent 23_12_2021
 
%%url https://lgaki.info/novosti/novuyu-skazku-o-glavnom-o-novogodnem-chude-predstavili-v-akademii-matusovskogo
 
%%author_id lgaki
%%date 
 
%%tags 
%%title Новую сказку о главном – о новогоднем чуде – представили в Академии Матусовского
 
%%endhead 
\subsection{Новую сказку о главном – о новогоднем чуде – представили в Академии Матусовского}
\label{sec:23_12_2021.stz.edu.lnr.lgaki.1.novaja_skazka_o_glavnom}

\Purl{https://lgaki.info/novosti/novuyu-skazku-o-glavnom-o-novogodnem-chude-predstavili-v-akademii-matusovskogo}
\ifcmt
 author_begin
   author_id lgaki
 author_end
\fi

В эти дни у всех – «ёлки». Но в главном творческом вузе Донбасса и ёлка –
особенная. Не просто праздник, а можно сказать, комплексный экзамен по
режиссуре, сценарному и актерскому мастерству для студентов третьего курса
кафедры театрального искусства, будущих режиссеров театрализованных
представлений и праздников. В этом году это курс народной артистки ЛНР,
профессора кафедры Веры Дмитриевны Евдокимовой.

\ii{23_12_2021.stz.edu.lnr.lgaki.1.novaja_skazka_o_glavnom.pic.1}

— Для сказки, которую показали сегодня на сцене Творческого центра «Красная
площадь, 7», студенты сами написали сценарий. Он абсолютно оригинальный. Сами
выступили в роли актеров и режиссеров – у каждого был свой эпизод, — рассказала
педагог кафедры и куратор курса Надежда Рубель. — Преподаватели оценят, как
ребята продумали и представили своих персонажей, как у них получилось
организовать взаимодействие со зрителями, особенно – с маленькими зрителями.
Для этого в действо вплели игровые моменты...

\ii{23_12_2021.stz.edu.lnr.lgaki.1.novaja_skazka_o_glavnom.pic.2}

Кроме Деда Мороза, Снегурочки и работников Фабрики подарков – трудолюбивых
гномов, в сказке участвуют добрейший Снеговик Снегослав, а еще Бяка, Бука и
Ледяная Борода – эти трое пытаются воспрепятствовать празднику. Но, к слову,
даже отрицательные персонажи этой новогодней истории знают, что зимой нужно
носить шапку и шарфик, а в наши дни еще и маску и пользоваться антисептиком. И
учат этому маленьких зрителей.

\ii{23_12_2021.stz.edu.lnr.lgaki.1.novaja_skazka_o_glavnom.pic.3}

— Наша сказка называется «Нужно верить в чудо!». И название говорит само за
себя: мы призываем всех наших маленьких и взрослых зрителей верить в чудо, в
Новый год, в Деда Мороза. И обязательно читать сказки, — рассказала
исполнительница роли Бяки Алина Горшкова. — Самое главное в любой сказке –
мораль, воспитательный момент... У нашей он тоже есть. В конце мы – Бука, Бяка,
Ледяная Борода – исправляемся, начинаем говорить хорошие слова, совершать
хорошие поступки. Учимся верить в чудо!

\ii{23_12_2021.stz.edu.lnr.lgaki.1.novaja_skazka_o_glavnom.pic.4}

Премьера позади, но сказка продолжается. До наступления нового года «Нужно
верить в чудо!» студенты Академии Матусовского покажут еще не в одной детской
аудитории. И в том, что получат заслуженные «пятерки» на экзамене, не
сомневаемся. Восторга маленьких зрителей от соприкосновения с придуманным ими
чудом третьекурсникам-режиссерам добиться уже удалось.

А это дороже любых оценок.

Фото – Марина Машевски.

\ii{23_12_2021.stz.edu.lnr.lgaki.1.novaja_skazka_o_glavnom.pic.5}
