% vim: keymap=russian-jcukenwin
%%beginhead 
 
%%file poetry.rus.mihailo_staryckii.poklyk_do_brativ_slovjan
%%parent poetry.rus.mihailo_staryckii
 
%%url https://www.myslenedrevo.com.ua/uk/Lit/S/Starycky/Poetry/PoklykDoBrativSlovjan.html
%%author 
%%tags 
%%title 
 
%%endhead 

\subsubsection{Поклик до братів слов’ян}
\label{sec:poetry.rus.mihailo_staryckii.poklyk_do_brativ_slovjan}

\Purl{https://www.myslenedrevo.com.ua/uk/Lit/S/Starycky/Poetry/PoklykDoBrativSlovjan.html}

Чи довго ще кривді і гвалту тлумити
Святе наше право розвою?
Невже не одної ми матері діти,
Невже не брати між собою?
За віщо ж зневага? Для чого не милі
Братам наші щирі жадання?
Кого ми чіпали, кому ми вчинили
Чи кривду, чи зле руйнування?
Ми тільки боролись за власную хату,
За те, що нам дорого й нині;
Бажаємо ми і тепер небагато:
Рідного розвою родині, –
Щоб дума славутня і мова співоча
Ширіли й пишались в народі…
Ми всіх пригорнули б до серця охоче,
Якби нам хоч трохи свободи!
Чому не ймеш віри ти, брате москалю?
Невже ти лякаєшся зради?
З тобою недолі нас кревно з’єднали, –
Не буде між нами розради.
З тобою давно ми працюєм на полі
Розросту науки й освіти,
З тобою давно ми шукаємо волі –
Її нам з тобою ділити.
А ти, брате ляше, --- невже пак до суду
Між нас буде нелад невдячний?
Ой скинь-бо з очей ти сю давню полуду
Та праведно глянь, необачний!
Ми тільки за наше лягали кістками,
Коли нам чинилися шкоди;
Ніколи не гралися ми кайданами,
Чужої не гнули свободи.
Хай чех оповіда, чи ми не стояли
За волю братерню горою?
Та з Жижкою ж вкупі діди наші дбали
Й здобули козацького строю!
Хай сербин розкаже, чи ми не братались
Оружно з ним в щасті і в горі?
За віру, за волю чи ж ми не рубались
На горах, на полі, на морі?
Спитайсь у словака --- і той не згадає
Від нас коли-небудь образи:
Злоби не було в нашім серці й немає –
Ми хтіли б загоїти врази.
До згоди ж, до гурту, до купи, слов’яне!
Забудьмо днедавнії свари,
Вгамуймо, братове, розладдя погане:
Встають-бо на заході хмари!..
Спізнаймося сами, вважаймо на брата –
Тоді нас ніхто не здолає;
Тоді не дамо ми німоті урвати
Ні цяти слов’янського краю!
Подаймо ж ми руки на вічне кохання
І крикнім на бенкеті згоди:
«Ми цілому світу бажаєм братання,
Поради, освіти й свободи!»
1872 
