% vim: keymap=russian-jcukenwin
%%beginhead 
 
%%file 04_08_2018.stz.news.ua.mrpl_city.1.v_chest_kogo_nazvan_nash_gorod
%%parent 04_08_2018
 
%%url https://mrpl.city/blogs/view/v-chest-kogo-nazvan-nash-gorod
 
%%author_id burov_sergij.mariupol,news.ua.mrpl_city
%%date 
 
%%tags 
%%title В честь кого назван наш город?
 
%%endhead 
 
\subsection{В честь кого назван наш город?}
\label{sec:04_08_2018.stz.news.ua.mrpl_city.1.v_chest_kogo_nazvan_nash_gorod}
 
\Purl{https://mrpl.city/blogs/view/v-chest-kogo-nazvan-nash-gorod}
\ifcmt
 author_begin
   author_id burov_sergij.mariupol,news.ua.mrpl_city
 author_end
\fi

\ii{04_08_2018.stz.news.ua.mrpl_city.1.v_chest_kogo_nazvan_nash_gorod.pic.1}

У дореволюционных жителей Мариуполя сомнений относительно происхождения
названия нашего города не было. Конечно же, в честь великой княгини, а затем и
императрицы Марии Федоровны, жены Павла I. Не случайно в книге \enquote{Мариуполь и его
окрестности: отчет об учебных экскурсиях Мариупольской Александровской
гимназии}, изданной в 1892 году, отведено место \enquote{Краткому биографическому
очерку императрицы Марии Федоровны}. Мария Фёдоровна (Феодоровна), до перехода
в православие — София Мария Доротея Августа Луиза Вюртембергская, вторая
супруга российского императора Павла I, мать двенадцати детей, из которых двое
стали российскими императорами -  Александр I и Николай I.

Отбросив конъюнктурные соображения сегодняшнего дня и ради объективности, стоит
подробнее обратиться к биографии этой женщины. По отзывам многочисленных
современников, в том числе и ее свекрови – Екатерины II, София Мария отличалась
кротостью характера, с самого детства отзывчивой к человеческим страданиям. В
основу ее воспитания были положены идеи Жан-Жака Руссо – близость к природе,
призыв к добродетели. В детстве она часто отказывалась от какой-нибудь прихоти,
чтобы помочь нуждающемуся семейству. Родители принцессы позаботились дать ей
наилучшее по тем временам образование. Обязательными предметами были Закон
Божий, история, география, французский и итальянский языки и, сверх того,
геометрия и мифология. Бытописатели того времени подчеркивали, что невеста
наследника российского престола получила образование, которое чрезвычайно редко
встречалось среди женщин того времени.

\textbf{Читайте также:} 

\href{https://archive.org/details/14_04_2018.sergij_burov.mrpl_city.kogda_byl_osnovan_mariupol}{%
Когда был основан Мариуполь?, Сергей Буров, mrpl.city, 14.04.2018}

Прибыв в Россию, принцесса Вюртембергская была крещена и стала Марией
Федоровной. Ее небесной покровительницей стала святая равноапостольная \textbf{Мария
Магдалина}. Ей суждено было испытать беспричинную ревность супруга Павла
Петровича, будущего императора, она была отстранена от своих детей, великих
князей Александра и Константина, сразу же по их рождению. Их забрала к себе
Екатерина II и сама руководила их воспитанием. Ее уделом стало изучение
русского языка и русских законов да изредка присутствие при дворцовых
церемониях. С восшествием Павла I на престол Мария Федоровна стала руководить
воспитательным обществом благородных девиц. Она с большим рвением отнеслась к
этой деятельности, привлекла много пожертвований в пользу общества. 

И в царствование своего мужа Павла I, и в царствование ее сыновей Александра I
и Николая I участие императрицы Марии Федоровны в государственной жизни было
ограничено попечением о женском образовании. И на этом поприще она сделала
немало. Благодаря ее заботам было учреждено несколько женских учебных заведений
не только в Санкт-Петербурге и Москве, но и в Харькове, Симбирске и других
крупных городах того времени.

\textbf{Читайте по теме:} 

\href{https://archive.org/details/07_04_2018.sergij_burov.mrpl_city.cerkov_sv_marii_magdaliny}{%
Церковь св. Марии Магдалины, Сергей Буров, mrpl.city, 07.04.2018}

После Октябрьской революции и в последующие годы о \enquote{царском}
происхождении названия нашего города помалкивали. Мягко говоря, не
\enquote{очень модной} была это тема. Тем более, что с 22 октября 1948 года
город получил имя А.А. Жданова, крупного партийного функционера. Интерес к
названию возник в годы перестройки, особенно в тот период, когда неравнодушные
и отважные люди боролись за возвращение городу имени \textbf{Мариуполь}. Борьба
завершилась победой – 13 января 1989 года наш город вновь стал Мариуполем. 13
сентября 2003 года в газете \enquote{Приазовский рабочий} была опубликована
статья авторитетнейшего краеведа Р. И.  Саенко \enquote{Имя нашего города}, в
которой {\em\bfseries на основе документов и исторических фактов обосновано,
что Мариуполь назван в честь императрицы Марии Федоровны} – супруги Павла I. И
тем не менее, находились и находятся люди, которые утверждают, будто имя городу
дано в честь \textbf{Пресвятой Девы Марии}. Иногда это делается без всяких
аргументов, иногда с критикой исследований и выводов ученых-историков,
выпускаются сборники стихов под названием \enquote{Город Марии}, создаются
телевизионные фильмы, говорят об этом на встречах с читателями.

\textbf{Читайте также:} 

\href{https://mrpl.city/news/view/klaster-marii-chem-v-mariupole-udivyat-k-yubileyu-goroda-foto}{%
\enquote{Кластер Марии}: чем в Мариуполе удивят к юбилею города, Яна Іванова, mrpl.city, 21.06.2018}

Однако стоит, наверное, посмотреть на проблему шире. Сколько было городов в
Российской империи, в названии которых было имя Мария? Ответ – три. Наш
Мариуполь. Мариинск в Кузбассе, бывшее село Кийское, получившее статус города,
переименованное в 1857 году в честь императрицы Марии Александровны, жены
Александра II, до принятия православия - принцесса Максимилиана Вильгельмина
Августа София Мария Гессенская и Прирейнская. Ее имя носит также город
Мариехамн – столица Аландских островов. Острова эти - часть Великого княжества
Финляндского, которое входило в состав Российской империи вплоть до ее падения
в 1917 году.

При рассмотрении городов стран с преобладанием православных верующих - это
Грузия, Румыния, Болгария, Македония, Сербия, Черногория, Греция и Кипр - ни в
одной из этих стран не обнаружено города, в наименовании которого было имя
Мария, тем более имя Пресвятой Девы Марии.

Случайность это или закономерность?
