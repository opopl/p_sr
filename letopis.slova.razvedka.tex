% vim: keymap=russian-jcukenwin
%%beginhead 
 
%%file slova.razvedka
%%parent slova
 
%%url 
 
%%author 
%%author_id 
%%author_url 
 
%%tags 
%%title 
 
%%endhead 
\chapter{Разведка}
\label{sec:slova.razvedka}

%%%cit
%%%cit_head
%%%cit_pic
\ifcmt
  pic https://img.strana.ua/img/article/3402/kak-sledjat-za-64_main.jpeg
	caption Полицейский Артем Белевский занимается изучением соцсетей в \enquote{ДНР} и \enquote{ЛНР}. Фото с сайта npu.gov.ua 
\fi
%%%cit_text
На сайте Нацполиции сегодня, 24 июня, появилось довольно откровенное интервью -
о том, как ведомство проводит \emph{разведку} по социальным сетям.  Детали своей
работы поведал начальник управления уголовного анализа Донецкой области Артем
Белевский.  Он занимается в этом ведомстве особым видом \emph{разведки} - на основе
открытых источников.  И следит не только за жителями Украины, но и за
населением \enquote{ДНР/ЛНР}.  Этот метод более известен сегодня как OSINT - Open
source \emph{intelligence}. И широко применяется в \emph{разведках} государств НАТО и
Европейского союза.  \enquote{Страна} проанализировала, какие тайны приоткрыл
правоохранитель. И можно ли уберечься от \enquote{всевидящего ока}  OSINT
%%%cit_comment
%%%cit_title
\citTitle{Как следят за жителями Донецкой области по интернету. Что такое OSINT}, 
Екатерина Терехова; Максим Минин, strana.ua, 24.06.2021
%%%endcit
