% vim: keymap=russian-jcukenwin
%%beginhead 
 
%%file 02_04_2021.fb.fb_group.story_kiev_ua.1.kreschatik.cmt
%%parent 02_04_2021.fb.fb_group.story_kiev_ua.1.kreschatik
 
%%url 
 
%%author_id 
%%date 
 
%%tags 
%%title 
 
%%endhead 
\subsubsection{Коментарі}

\begin{itemize} % {
\iusr{Vladimir Glikin}

\ifcmt
  ig https://i2.paste.pics/c8614490ab23e57428977ecf0aa332b9.png
  @width 0.2
\fi

\iusr{Игорь Бойко}
Музыка кошмарная

\begin{itemize} % {
\iusr{Мила Подлесная}
\textbf{Игорь Бойко} Отчего же? Типичная для того времени.

\iusr{Игорь Бойко}
\textbf{Мила Подлесная} Для того времени много чего было типично. Это не меняет ее кошмарности.

\iusr{Виктор Петровский}
\textbf{Игорь Бойко} сейчас музыка для души? Что аж срать хочетса извините за грубое слово. Просто душа сейчас поёт и радуетса.

\begin{itemize} % {
\iusr{Игорь Бойко}
\textbf{Виктор Петровский} Я что-то говорил про сейчас? У автора был выбор. Автор выбрал хрень.

\iusr{Виктор Петровский}
\textbf{Игорь Бойко} я понял, хрени сейчас помоему выше крышы

\iusr{Игорь Бойко}
\textbf{Виктор Петровский} Говорите о современной хрени в другом месте, я тут при чем?

\iusr{Елена Червоная}
\textbf{Виктор Петровский} хорошего воспитания Вам музыка не дала(

\iusr{Виктор Петровский}
\textbf{Игорь Бойко} я что к вам с претензиями? Мы наверно не так поняли друг друга. Извините если что.

\iusr{Игорь Бойко}
\textbf{Виктор Петровский} Всего доброго, тоже не имел намерения вас задеть.

\iusr{Виктор Петровский}
\textbf{Елена Червоная} За грубость я извинился.

\iusr{Елена Червоная}
\textbf{Виктор Петровский} ну и хорошо. @igg{fbicon.smile}  @igg{fbicon.hot.beverage} 
\end{itemize} % }

\end{itemize} % }

\iusr{Ната Бузань}

\ifcmt
  ig https://i2.paste.pics/3ea38b7a28ccdc0ed454a0787afa9227.png
  @width 0.2
\fi

\iusr{Мила Подлесная}

Глядя на такие видео, осознаешь, что прожил не одну, а несколько жизней, причем
очень разных.

\begin{itemize} % {
\iusr{наталья колесник}
\textbf{Мила Подлесная} а сколько еще проживем @igg{fbicon.face.grinning.smiling.eyes}  каждый день, как на войне

\iusr{Вася Лейкин}
\textbf{наталья колесник} Теперь и мы можем представить, что творилось в сердцах людей в далёком 1917....

\ifcmt
  ig https://scontent-frx5-2.xx.fbcdn.net/v/t39.1997-6/s168x128/10333122_298592793654253_1393149385_n.png?_nc_cat=1&ccb=1-5&_nc_sid=ac3552&_nc_ohc=mveKXQAb0-EAX9AMwCW&_nc_oc=AQn-l89ij5ICcVqBORP4agU_gNQDcP9wxSAwcAYRkoTSjHFJh6rGL0LHYnWJD4h6Ij4&tn=lCYVFeHcTIAFcAzi&_nc_ht=scontent-frx5-2.xx&oh=00_AT8IQpm553Bc_ETdrztVv9k-OYk8O9ziLln9cQ2UPuRqgQ&oe=61BCE147
  @width 0.1
\fi

\end{itemize} % }

\iusr{Алла Поповская}
Замирает сердце,как хочется вернутся...... @igg{fbicon.heart.red}{repeat=2}

\iusr{Марина Лабунец}
Каштаны!!!!!!!!!!!!

\iusr{Татьяна Задорожная}

Ого немногочисленных, людей дуже багато, немногочисленно авто це да, и ще
невелик1 каштан, бо коли 82 роц1 це вже велетн1

\begin{itemize} % {
\iusr{Леонид Красиловский}
\textbf{Татьяна Задорожная} зачекайте ще пару років і по Хрещатику тільки собаки бездомні будуть бігати((((
\end{itemize} % }

\iusr{Надежда Баторевич}

\ifcmt
  ig https://i2.paste.pics/9b53d25f3d4f3631c17230458a0210ad.png
  @width 0.2
\fi

\iusr{Анатолий Лернер}
Да, это был ещё Киев...

\iusr{Наталья Ершова}
Да это был Киев наш!!! Киевляне меня поймут!!!

\iusr{Людмила Кралина}
Грустно.

\iusr{Полина Микульская}
Ностальгия

\iusr{Юрий Панчук}

У меня тоже есть раннее воспоминание, как мы с бабушкой переходили Проспект
Победы поверху и не по переходу примерно напротив ЗАГСа. Помню один грузовичок
ГАЗ 51 даже остановился, хотя мы могли подождать и бабушка сказала \enquote{Вот хороший
водитель}. Тогда было по две полосы в каждую сторону и подземных переходов еще
не построили, ходил трамвай.


\iusr{Катя Теркина}
Какая зелёная улица

\iusr{Надежда Василенко}
Это город, который имеет лицо. А не лоскутное одеяло, утопленное в бетоне

\iusr{Леонид Красиловский}
А когда мне было два годика то по Крещатика конки запряженные бегали.....)))))
и Сталину ещё год жизни был отведён ....

\iusr{Людмила Иванова}
А музыка из Золотой мины!

\iusr{Людмила Иванова}
Еще были каштаны на Кресте!просто тогда не было Кличко с елками

\begin{itemize} % {
\iusr{Светлана Сергеева}
\textbf{Людмила Иванова} Ели в Киеве тоже были помимо каштанов. Голубые ели тоже были прекрасным украшением города.
\end{itemize} % }

\iusr{Тамара Мельничук}
Це прекрасно але в той час транспорту було на процентів 90 менше

\begin{itemize} % {
\iusr{Ольга Поканевич}
\textbf{Тамара Мельничук} Ага, зігнали з усього Києва. І людей теж.  @igg{fbicon.face.woozy} 

\iusr{Виктор Петровский}
\textbf{Тамара Мельничук} потому что работа была у людей на местах. Извините за грубое слово, какого хрена человеку ехать в Киев, мучитса, снимать квартиру, койко место. Всю страну уничтожили, ветер гуляет по полям и селам. И что делать людям? Все в Киев тут хоть не здохнешь. Так и живём 30 лет, а что дальше?
\end{itemize} % }

\iusr{Ирина Архипович}
Очень жаль, что нет!!! Ах, как хочется вернуться!!!!.....

\iusr{Aleksy Ivanov}
Moy Kiev... noviy ne znayu... chuvstvuyu sebya chujakom... za 30 let vse izmenilos

\iusr{Людмила Киенко}
Теперь,,каштаны Киева,, только в песне остались.

\iusr{Сергей Хромешкин}

\ifcmt
  ig https://scontent-frx5-2.xx.fbcdn.net/v/t39.1997-6/s168x128/70049209_2494354163990780_8707503201399603200_n.png?_nc_cat=1&ccb=1-5&_nc_sid=ac3552&_nc_ohc=nFi3Z53RGOoAX-bijcH&_nc_ht=scontent-frx5-2.xx&oh=00_AT8zNBg6RbZpThzDvC9CUDGwP_nEDxx3s5p-ln3KulRdVA&oe=61BE37E6
  @width 0.1
\fi

\iusr{Сергей Хромешкин}
Да был...

\iusr{Владимир Каледин}
Не вернуть уже того счастливого времени! На превеликий жаль!!!

\begin{itemize} % {
\iusr{Виктор Петровский}
\textbf{Владимир Каледин} истинные слова
\end{itemize} % }

\iusr{Виктор Петровский}

Подумаешь, Барыги начали в конце 80 красть, воруйте чтобы вас удавило. Но город
оставте в покое, зелень вырубали, всё уничтожили. Я уверен что половина этих
Барыг, киевляне родившиеся в нашем родном Киеве. Ну что нельзя было уничтожать?


\iusr{Юра Савчак}
Магія часу

\begin{itemize} % {
\iusr{Yuri Arkadyev}
\textbf{Юра Савчак} - машина времени, тёзка  @igg{fbicon.hand.ok}  @igg{fbicon.thumb.up.yellow}  @igg{fbicon.hand.waving} 
Как раз в прошлом, ХХм веке её и изобрели...
Называется интернет...
\end{itemize} % }

\iusr{Андрей Долохов}
Помню, весь Крест был зеленый. Кому мешали каштаны?

\begin{itemize} % {
\iusr{Наталья Ершова}
\textbf{Андрей Долохов} нелюди деньги отмывали, одно вырубали, другое сажали и так по кругу!!!

\iusr{Natalya Tarasenko}
\textbf{Андрей Долохов} Чужакам. @igg{fbicon.face.disappointed} 
\end{itemize} % }

\iusr{Irina Gavrilenko}
Мені тоді було два роки. Цікаво дивитися. Київ, мій Київ... Машини неймовірні. Всі ж такі правопорушники дорожнього руху.

\iusr{Олена Угнівенко}
Можно было Крещатик перейти не на светофоре)

\iusr{Елена Остапенко}

\ifcmt
  ig https://scontent-frx5-2.xx.fbcdn.net/v/t39.1997-6/s168x128/47270791_937342239796388_4222599360510164992_n.png?_nc_cat=1&ccb=1-5&_nc_sid=ac3552&_nc_ohc=vYZtHV62k4wAX_xwE90&_nc_ht=scontent-frx5-2.xx&oh=00_AT_zFyP5suouMvQGkPRs1JJE_z0hPrAaadhnA4wkg6XjAA&oe=61BD1176
  @width 0.1
\fi

\iusr{Лина Агаркова}
Спасибо. Счастливое время в любимом Городе

\iusr{Алла Кириленко}
Класс! мне тоже было 2 годика! @igg{fbicon.face.grinning.smiling.eyes} 

\begin{itemize} % {
\iusr{Arkadi Romansky}
\textbf{Алла Кириленко} а я родился. И счастлив, что родился в таком замечательном городе. И глубоко несчастлив видя, как его изуродовали.

\iusr{Светлана Недайбида-Бучко}
\textbf{Алла Кириленко} И мне!

\iusr{Алла Кириленко}
\textbf{Светлана Недайбида} я 18 июля

\iusr{Светлана Недайбида-Бучко}
\textbf{Алла Кириленко} А я - 13 сентября


\iusr{Алла Кириленко}

\ifcmt
  ig https://i2.paste.pics/82286c31a0e048d47ddc830086af9b9c.png
  @width 0.2
\fi

\end{itemize} % }

\iusr{Polina Potakh}
а мне было 12 лет и я обожала и обожаю Крещатик хотя живу уже в Сиднее с 1989 года..

\iusr{Лариса Гаврилюк}
Де наші дерева каштани???

\begin{itemize} % {
\iusr{Maria Dakicat}
\textbf{Лариса Гаврилюк} Та ось же!

\ifcmt
  ig https://scontent-frt3-1.xx.fbcdn.net/v/t1.6435-9/168034464_2895297067421695_7830589022707284700_n.jpg?_nc_cat=108&ccb=1-5&_nc_sid=dbeb18&_nc_ohc=svbyy63ofiQAX_qWTIK&_nc_ht=scontent-frt3-1.xx&oh=00_AT9BUApWz8W42P8sxmjUbom9KLY2FYk1Eo9T_hcS6S3eKQ&oe=61DF40DC
  @width 0.4
\fi

\iusr{Maria Dakicat}
А ще ось.

\ifcmt
  ig https://scontent-frx5-1.xx.fbcdn.net/v/t1.6435-9/168502599_2895298034088265_8160538025604346181_n.jpg?_nc_cat=111&ccb=1-5&_nc_sid=dbeb18&_nc_ohc=_K6LdWLoSOwAX-mYybw&_nc_ht=scontent-frx5-1.xx&oh=00_AT-FfScb0rAOS3vLQgP8iuwluRdv0yp0Jar3jeKBwmx6JQ&oe=61DE1027
  @width 0.4
\fi

\iusr{Лариса Гаврилюк}

\ifcmt
  ig https://scontent-frx5-2.xx.fbcdn.net/v/t39.1997-6/p480x480/106011002_953858235076534_2503726066745003202_n.png?_nc_cat=1&ccb=1-5&_nc_sid=0572db&_nc_ohc=aeGvVQ9s4YgAX9I1CCm&tn=lCYVFeHcTIAFcAzi&_nc_ht=scontent-frx5-2.xx&oh=00_AT_cqPOv5FHVRsVcgkYtQDp3ijhlk6Mv6voNntzcTQ_hnQ&oe=61BD6090
  @width 0.2
\fi

\end{itemize} % }

\iusr{Надежда Лабик}

Это и мой Киев. В том году у меня родился сын. Так ничего не стоило прогуляться
с коляской с пл. Победы до памятника Ленину /там рядышком жили мои
родственники/

\iusr{Yuri Arkadyev}

...Киев-67  @igg{fbicon.thinking.face} ...ну привет, родной город... через 4е
дня мне \enquote{...адын год...}  @igg{fbicon.wink}
@igg{fbicon.face.smiling.sunglasses} 

\ifcmt
  ig https://scontent-frx5-2.xx.fbcdn.net/v/t1.6435-9/168557338_10226644783172779_3030679425313623338_n.jpg?_nc_cat=109&ccb=1-5&_nc_sid=dbeb18&_nc_ohc=QFAElrX0MUsAX8nTh6D&_nc_ht=scontent-frx5-2.xx&oh=00_AT9jCJF4O7lfybZbJx_rSnynpH8uV9HCuTMMLCkT6_WEag&oe=61DCC72C
  @width 0.4
\fi

\iusr{Евгений Степаненко}
Чистый город счастливых людей

\iusr{Раиса Карчевская}
Прекрасное видео. Огромное спасибо

\iusr{Лена Смовженко}
Как красиво, уютно и зелено! Но смотришь-и почему-то становится грустно...

\begin{itemize} % {
\iusr{Бабич Лариса}
\textbf{Лена Смовженко} разделяю и поддерживаю. Тоже такие слова пришли, но вы успели первая.

\iusr{Владимир Степаныч}
 @igg{fbicon.100.percent} \%. ФИЛЬМ УЖАСОВ @igg{fbicon.ogre} 

\iusr{Валерий Маслов}
\textbf{Лена Смовженко}: и времена молодости, и времена спокойствия, развития, созидания и надежд...
\end{itemize} % }

\iusr{Наталья Конончук-Шахрур}
Супер .Скучаю .

\iusr{Oksana Kirichenko}

Я в тот году только родилась, но ностальгирую по тому, старому и зеленому,
Киеву! Всегда любила в детстве с родителями гулять по Крещатику! Сейчас уже
совсем не тот Крещатик...

\iusr{Светлана Макаренко}
Ностальгия до слез. @igg{fbicon.face.sleepy} 

\ifcmt
  ig https://i2.paste.pics/37fd67c3d833d36f51603bab86335872.png
  @width 0.2
\fi

\iusr{Настасья Старинец}
Как стильно  @igg{fbicon.cat.heart.eyes}  @igg{fbicon.face.smiling.hearts} 

\iusr{Ольга Волынец}

А мне в 1967 уже 15 лет))) Вот такие свадебные машины были возле фотографии на
бывшй Ленина, бегала с подружками на них смотреть))) А каштаны давали тень
летом, красота была. Сейчас видимо экология и \enquote{выхлопники} убили каштаны. @igg{fbicon.cry} 

\iusr{Владимир Степаныч}

ЧТО ЗА ЛЮДИ ЕДУТ В ЭТИХ ОГРОМНЫХ СУПЕРКАРАХ ПО ПРОЕЗЖЕЙ ЧАСТИ?
И КАКАЯ ЖЕ ПРОПАСТЬ МЕЖДУ НИМИ И ТЕМИ, КТО ТОЛПОЙ ГРЕБЁТ ПО ТРОТУАРУ!!!

\iusr{Юра Ключник}
Як було чудово! Після цього року, ще цілих 20 років вбивали, катували і гноїли
по тюрмах українську інтелігенцію!

\begin{itemize} % {
\iusr{Виктор Петровский}
\textbf{Юра Ключник} это звери. Но здесь о Киеве речь, о его красе, зелени. О том что был цветущий наш Киев. О ГОРОДЕ. А убивали Украинскую Нацию. Демоны Ада.
\end{itemize} % }

\iusr{Алла Канаева}
Тоска по прошлому @igg{fbicon.face.pensive} 

\iusr{Галя Твердохлебова}
\textbf{Алла Канаева} эх .. @igg{fbicon.cry} 

\iusr{Светлана Дубински}
Таким и запомнился Крещатик!@igg{fbicon.heart.red}

\iusr{Natalya Tarasenko}
Какая красивая музыка...

\begin{itemize} % {
\iusr{Iryna Dashkovska}
\textbf{Natalya Tarasenko} Музыка из фильма «Золотая мина», композитора Исака Шварца

\iusr{Natalya Tarasenko}
\textbf{Iryna Dashkovska} Спасибо, Ирина. (Уже и фильм захотелось посмотреть с такой музыкой).  @igg{fbicon.heart.growing} 

\iusr{Iryna Dashkovska}
\textbf{Natalya Tarasenko} кстати фильм 77 года, с прекрасными актерами - Даль, Киндинов, Глузский, Удовиченко.

\iusr{Natalya Tarasenko}
\textbf{Iryna Dashkovska} Что-то выплывает из памяти... Даль, Глузский - теперь точно посмотрю! Спасибо, Ирина. @igg{fbicon.bouquet} 
\end{itemize} % }

\iusr{Владимир Крылов}
Мой. Киев. Город который вспоминаю. Родился и живу

\iusr{Nataliya Borodina}
Как дядька не спеша перешел дорогу прям перед Волгой газ 21

\iusr{Наталья Швед}
Какой же он был красивый!!! Особенно когда цвели каштаны на Крещатике!!!

\ifcmt
  ig https://i2.paste.pics/b083b1aa43168f94e84254728a97f2f7.png
  @width 0.2
\fi

\iusr{Дмитрий Пушкарев}
еще нет этого пресного навязчивого красного цвета

\iusr{Ирина Соколова}
Время летит быстро, изменяются города, люди, мода... Останется история

\iusr{Галина Святненко}
И нет 1967 года...

\iusr{Maryna Sadchenko}
Как раз по нему ... стою, трудно движение со скоростью 1 км/ч назвать движением  @igg{fbicon.wink} 

\iusr{Dimitri Statnikov}

Кому нужна была эта зелень если город в то время был рассадником паскудников из
КГБ и прочих мохровых Активистов кот повально подозревали людей, арестовывали и
следили.. Мою бедную маму кот в то время работала гидом и переводчиком в
Интуристе постоянно подозревали и таскали к кураторам на Владимирскую. Она
чудом не села в тюрьму ни за что, а другим девушкам работавшим с ней повезло
меньше..

\begin{itemize} % {
\iusr{Арт Юрковская}
Всем остальным 2 миллионам киевлян.

\iusr{Oleg Kukshyn}
Только как тут взаимосвязаны зелень и КГБисты - совсем непонятно.
Киев тогда был городом-парком, сейчас такого уже нет.

\iusr{Светлана Токарева}
\textbf{Dimitri Statnikov} А причём тут политика?
\end{itemize} % }

\iusr{Властелин Кота Мяу}
И какой то Вася переходит дорогу где приспичило

\iusr{Tatiana Loukianova}
Я тоже так любила делать....

\iusr{Коневцева Натали}
Мой родной и любимый город @igg{fbicon.heart.red} но почти без каштанов на Крещатике  @igg{fbicon.cry} 

\iusr{Nadejda Popova}
Совершенно другой темп жизни

\iusr{Semyon Belenkiy}
«Як тебе не любити,
Києве Наш...»

\begin{itemize} % {
\iusr{Михаил Чартин}
\textbf{Semyon Belenkiy} К большому сожалению, эта любовь в прошлом.

\iusr{Semyon Belenkiy}
\textbf{Mikhail Chartin} Да уж, ты прав Мишенька
\end{itemize} % }

\iusr{Алексей Ткаченко}

Интересно что доброго он нашёл в те времена. Разве что был ребёнком и всё
казалось в розовом цвете. Кстати тогда водители никогда не пропускали пешеходов
даже на зебре.

\begin{itemize} % {
\iusr{Дмитрий Бартюк}
\textbf{Алексей Ткаченко} да, сбивали на хрен и ехали дальше. Да и самих зебр вообще не было, если честно. И электричества. Я помню. Это был ужас.

\iusr{Eva Evidze}
\textbf{Алексей Ткаченко} 

Але які були каштани! Вони створювали густу крону, і по Хрещатику можна було
пройти повністю в тіні дерев. А ще були густі-густі кущі навколо газонів і
клумб. Там ще стояли дуже зручні лавки. Але якийсь ідіот першими викорчував ці
гарні кущі. Потім інші ідіоти прибрали каштани із їхніми чудовими кованими
решітками з візерунками, які захищали корені дерев. А ще Хрещатик весь час
поливали. Особливо пам'ятаю, як вранці мама вела мене з метро Хрещатик до
зупинки 18-го тролейбуса, а асфальт був весь мокрий але чистий. И повітря було
чистим.

\begin{itemize} % {
\iusr{Алексей Ткаченко}
\textbf{Eva Evidze} и колбаса по 2.20

\iusr{Лариса Олейникова}
\textbf{Алексей Ткаченко} И не только 2.20, была и 1.60. Завидно?

\iusr{Alexander Fefelov}
\textbf{Лариса Олейникова} бред. Чему завидовать?

\iusr{Sofia Shutaya}
\textbf{Eva Evidze} глрод был добрым, без спешки, чистым и безопасным.

\iusr{Света Медецкая}
\textbf{Алексей Ткаченко} зато вкусная была!!!)))
\end{itemize} % }

\iusr{Elena Garam}
\textbf{Алексей Ткаченко} наверно Вам не понять

\end{itemize} % }

\iusr{Надежда Смаглюк}
Какой зеленый, уютный, теплый город. В то же время такой монументальный !

\iusr{Таня Гур}
Как жаль, что нет каштанов...

\iusr{Maria Dakicat}

От тоді були часи! Не те, що зараз! )))) Тоді трава була зеленішою, і каштани
були кращі, а машини, а люди )))) Пенсіонери з групи так люблять ностальгувати
по часам совка, я помітила.)

\begin{itemize} % {
\iusr{Петр Кузьменко}
\textbf{Maria Dakicat} і Ви, колись, повірте будете ностальгувати за теперешнім часом, коли ще можна ходити лише в масці, та дихати повітрям без ізолюючого протигазу, або балонів з повітрям за спиною скафандру. Все пізнається у проівнянні...

\begin{itemize} % {
\iusr{Maria Dakicat}
\textbf{Петр Кузьменко} Я й зараз по вулицях в масці не ходжу. Навіщо зомбі-апокаліпсіс уявляти? Чи вам так приємніше, коли думаєте,що ви пожили в супер-часи,а ми будемо мучитись? Київ зовні зараз став набагато краще, ніж був навіть 30 років тому.

\iusr{Петр Кузьменко}
\textbf{Maria Dakicat} 

саме так. Хрещатик зеленіше, автомобілей меньше, Місто \enquote{прикрашене}
баготоповерховими монстрами на колись зелених схилах тоді ще чистого та
повноводного Дніпра. І на моєму Андріївському узвозі будівля \enquote{красеня} - театру
та зовсім новий пам'ятник Гоголю, схожий на голанського неформала. Може
продовжити перелік досягнень які зробили Київ \enquote{набагато кращим}?

\iusr{Maria Dakicat}
\textbf{Петр Кузьменко} Діалог не має сенсу. Мені, доречі, подобається будівля театру на Андрієвському. Трабл в тому, що пенсіонери хочуть залишити місто таким, яким воно було в їхньому дитинстві. Тоді давайте усі разом поностальгуємо за Києвом геть без машин часів Київської Русі )

\iusr{Петр Кузьменко}
\textbf{Maria Dakicat} нажаль, діалог не має сенсу. Особливо після Вашого допису про захват будівлею Театру на Подолі. Років через 40 Ви будете на моєму боці...

\iusr{Светлана Манилова}
\textbf{Maria Dakicat}, в \enquote{Киевских историях} нет такой возрастной группы, которая бы вспоминала времена Киевской Руси... @igg{fbicon.smile} 

\iusr{Maria Dakicat}
\textbf{Светлана Манилова} Это запрещено правилами группы?  @igg{fbicon.thinking.face} А то я бы поностальгировала . Одни церкви да дома одноэтажные. Лепотааа  @igg{fbicon.face.grinning.big.eyes}  Не то, что нынешнее племя

\iusr{Светлана Манилова}
\textbf{Maria}, нет, конечно не запрещено. Просто, как я уже написала выше, вспоминать этот период (период Киевской Руси) некому... @igg{fbicon.smile} А вот свидетелей того времени, о котором вспоминает автор, в группе много.

\iusr{Петр Кузьменко}
\textbf{Maria Dakicat} чтобы предметно ностальгировать о древних временах нашего Города внимательно прочтите посты участника и автора нашей группы Юрия Никитина. Полагаю, будет не лишним.

\iusr{Alexander Fefelov}
\textbf{Петр Кузьменко} місто не може не розвиватись. Невже це так важко зрозуміти. Хоча каштанів жаль.

\iusr{Петр Кузьменко}
\textbf{Alexander Fefelov} Ви бували у Празі, у Відні? Чому так важко зрозуміти, що місто може розвиватися цивілізовано?

\iusr{Света Медецкая}
\textbf{Maria Dakicat} ну у вас и вкус!!!)))

\iusr{Alexander Fefelov}
\textbf{Петр Кузьменко} 

Бував скрізь. І Ваше порівняння досить дивне. Особливості розташування,
кількість населення, а головне, що місто після війни треба було оновлювати
треба ж приймати до уваги. Я жив на Смирнова-Ласточкина 10, практично 19
сторіччя, поряд Киянівський провулок, - це було майже село у центрі міста. На
Гончарах- Кожем'яках жив товариш фактично в 19 сторіччі, топили дровами, туалет
і вода на вулиці. У нас на Львівський теж були дома без води та каналізації, з
сараями з дровами. Так у столиці в 21 ст. не повинно бути і майже не буває в
пристойних країнах. Важко уявити велике місто з шматками сел прямо у центрі. А
таких місць було дуже багато. Я часто проїзжаю Софію та Бухарест, порівняно з
Києвом, повірте, це глибока переферія. Розумію Вас, хотілося б порівнювати Київ
з Лондоном або Парижем. Але... завдяки комунякам і іншим реальним причинам не
Гонконг і не Брюгге. Наше місто, якщо порівнювати з менш крутими, навіть
європейськими, столицями, - супер, зовсім не соромно. Якби ще не пострадянських
мери, що народились не в Києві.. от про що треба думати... як можна оцінити
роботу Омельченко, при якому спотворили Майдан та знищили Сінний ринок,
будівництво деяких хмарочосів, що закривають Лавру, при Космосі, будівництво
острівків \enquote{смерті} на дорогах, замість ремонту, при Віталіку? Якби тільки
загублені каштани були головною проблемою. Мабуть при Ярославі чи Воломирі
зелені та річок-озер в Києві було набагато більше. Але зараз 21 сторіччя і 5
млн мешканців. Будьте реалістом.

\iusr{Петр Кузьменко}
\textbf{Alexander Fefelov} 

тут я цілком згоден з Вами у більшості питань! Я сам народився та виріс у
комунальній квартирі на Андріївському узвозі 2. Повністю підтримую, що мерію та
ключові посади міского землелерозподілу та містобудування повинні посідати
небайдужі та обізнані кияни. Реконструкція та забудова сучасного Міста не
повинна перетворюватися на знищення його пам'яток та зелених насаджень.


\iusr{Master Valerii Boiko}
\textbf{Alexander Fefelov} А чого в Гончари Кожумяки заглядати, Пашка з нашого класу жив в провулку Десятинному в дерев'яному будинку, туалет на подвір'ї)))

\iusr{Alexander Fefelov}
\textbf{Valerii Boiko} 

да, Валера, точно. Я и забыл. А ведь Паша жил ну прям в самой центральной точке
Киева, причем ещё и древнего. Во дворе раскопки все время вели. И что, надо
было оставлять эту хату там? Может и не самый красивый в мире дом построили на
ее месте, но, в принципе, неплохой. Город, безусловно, должен развиваться. Хотя
немного ностальгии все же внутри присутствует  @igg{fbicon.smile}  блин

\iusr{Master Valerii Boiko}
\textbf{Alexander Fefelov} 

Ещё и Слава с первого класса жил на Андреевском в таком же деревянном доме,
помню потолки низкие были и все деревянное красного цвета. 2-3 мин ходьбы от
школы нашей. Эх Ностальжи....)


\iusr{Maria Dakicat}
\textbf{Петр Кузьменко} 

Ось вам Відень. Серед старовинних будівель нормально себе почуває новодєл. І
ніхто не охає: Пропала Мальвина, невеста моя .)

\ifcmt
  ig https://scontent-frt3-2.xx.fbcdn.net/v/t1.6435-9/168398163_2895944747356927_1970153126459682874_n.jpg?_nc_cat=103&ccb=1-5&_nc_sid=dbeb18&_nc_ohc=8CuVTXu-mwoAX9shunQ&_nc_ht=scontent-frt3-2.xx&oh=00_AT_7ctCb8qytUVimaf4gJW4-Rp6mt2t9y3iR17z554Tvuw&oe=61DE5B67
  @width 0.4
\fi

\end{itemize} % }

\end{itemize} % }

\end{itemize} % }
