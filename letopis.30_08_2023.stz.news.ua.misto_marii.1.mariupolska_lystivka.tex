% vim: keymap=russian-jcukenwin
%%beginhead 
 
%%file 30_08_2023.stz.news.ua.misto_marii.1.mariupolska_lystivka
%%parent 30_08_2023
 
%%url https://mistomariupol.com.ua/uk/mariupolska-lystivka
 
%%author_id news.ua.misto_marii
%%date 
 
%%tags 
%%title Маріупольська листівка
 
%%endhead 
 
\subsection{Маріупольська листівка}
\label{sec:30_08_2023.stz.news.ua.misto_marii.1.mariupolska_lystivka}
 
\Purl{https://mistomariupol.com.ua/uk/mariupolska-lystivka}
\ifcmt
 author_begin
   author_id news.ua.misto_marii
 author_end
\fi

Для багатьох маріупольців знайомими є зображення Маріуполя імперської доби.
Серед найбільш відомих зображень є листівки з собором Святого Харлампія,
Марії-Магдалинівським собором, Олександрівським сквером та торговими рядами на
вулиці Торговій. Ці та інші зображення знаходяться на маріупольських листівках,
які за рахунок різноманітності та виразності зображень є дуже цінними джерелами
для багатьох галузей історичної науки.

Беручи до уваги, що наше місто два рази було зруйновано внаслідок війн та
міських боїв, а також у місті відбувались процеси перебудови – зображення на
листівках є єдиними задокументованими свідоцтвами минулого. Саме завдяки цим
листівкам у нас є можливість подивитись на ті архітектурні об'єкти, які були
втрачені у 1930-х та 1940-х роках.

\ii{30_08_2023.stz.news.ua.misto_marii.1.mariupolska_lystivka.pic.1}

Що стосується самих листівок, то вони створювалися на основі фотографій
місцевих фотографів: А. Стояновського, А. Целинського та М. Улахова.
Замовниками листівок були власники місцевих магазинів С. І. Приходько-Пихненко
та Н. М. Нерофіді, власники електро-типографії брати Гольдріни та інші місцеві
підприємці. Одним з основних видавців та розповсюджувачів листівок було
Маріупольське повітове земство.

\ii{30_08_2023.stz.news.ua.misto_marii.1.mariupolska_lystivka.pic.2}

До 1917 року, випуск поштових листівок з фотографіями Маріуполя, як і у
багатьох приморських містах, був спрямований на зовнішнього споживача: туристів
та людей, що відвідували наше місто транзитом. У зв'язку з цим, більша частина
листівок зображувала найбільш розповсюджені сюжети: вокзал, готель
\enquote{Континенталь}, базар, панорами міста, порт, вулиці Торгова та Єкатерининська.

\ii{30_08_2023.stz.news.ua.misto_marii.1.mariupolska_lystivka.pic.3}

Але, разом з розповсюдженими листівками існують досить рідкісні зразки з
унікальними зображеннями життя міста та повіту. Рідкісність окремих листівок
зумовлена кількома факторами: малою кількістю тиражів, розсіяністю паперових
екземплярів, давністю та широкою діяльністю самовидавців.

\ii{30_08_2023.stz.news.ua.misto_marii.1.mariupolska_lystivka.pic.4}

Перші маріупольські листівки з'явилися наприкінці ХІХ століття після
розпорядження Міністерства внутрішніх справ Російської імперії, яким було
надано дозвіл на відправлення поштою відкриті листи та бланки приватного
виготовлення.

\ii{30_08_2023.stz.news.ua.misto_marii.1.mariupolska_lystivka.pic.5}

Дореволюційні листівки та поштові карточки мали єдиний формат розміру – 9х14
см, одна сторона листівки використовувалася для написання тексту, а інша – мала
фотографію та підпис. Значна частина листівок яка дійшла до наших часів у
матеріальному та цифровому вигляді має поштові штемпелі, адреси та інші
атрибути поштового відправлення. Географія розповсюдженості маріупольських
листівок є досить вражаючою, до 1917 року листівки відправлялися поштою в
Німеччину, Велику Британію, Францію, Турцію та Америку. Вражаючим є факт того,
що окремі листівки здійснили і зворотну подорож та повернулися до міста
відправлення.

\ii{30_08_2023.stz.news.ua.misto_marii.1.mariupolska_lystivka.pic.6}

Не менш цікавим є те, що на цих листівках є приклади ділового та побутового
листування між людьми іншої історичної доби. Активно зустрічаються звертання до
\enquote{Його високородія...}, розшифровка та пояснення зображень.

\ii{30_08_2023.stz.news.ua.misto_marii.1.mariupolska_lystivka.pic.7}

Але для нас, найбільшу цінність складає можливість порівняння масштабів змін
які відбулися у нашому місті протягом 150 років у галузі архітектури, культури,
інфраструктури, промисловості  та міського господарства одного з найбільших
міст на узбережжі Азовського моря.

% "Русскій Провіданс" – один з прабатьків металургійного комбінату Ілліча
\ii{30_08_2023.stz.news.ua.misto_marii.1.mariupolska_lystivka.pic.8}

% Вид на Слобідку з моря. Можна бачити 3 найбильших храми міста: (зліва на право) Марії-Магдалинівський собор, собор Святого Харлампія та церква Святих Костянтина та Єлени
\ii{30_08_2023.stz.news.ua.misto_marii.1.mariupolska_lystivka.pic.9}

% Район Гавань
\ii{30_08_2023.stz.news.ua.misto_marii.1.mariupolska_lystivka.pic.10}

% Маріупольська базарна площа, на задньому плані заводські колонії
\ii{30_08_2023.stz.news.ua.misto_marii.1.mariupolska_lystivka.pic.11}

% Залізнодорожний вокзал
\ii{30_08_2023.stz.news.ua.misto_marii.1.mariupolska_lystivka.pic.12}

% Паровоз на околицях Маріуполя
\ii{30_08_2023.stz.news.ua.misto_marii.1.mariupolska_lystivka.pic.13}

% Реальне училище
\ii{30_08_2023.stz.news.ua.misto_marii.1.mariupolska_lystivka.pic.14}

% Готель "Континенталь", повз проходить свиня свійська (а точніше – "світська")
\ii{30_08_2023.stz.news.ua.misto_marii.1.mariupolska_lystivka.pic.15}
