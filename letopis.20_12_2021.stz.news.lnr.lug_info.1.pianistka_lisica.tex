% vim: keymap=russian-jcukenwin
%%beginhead 
 
%%file 20_12_2021.stz.news.lnr.lug_info.1.pianistka_lisica
%%parent 20_12_2021
 
%%url https://lug-info.com/news/pianistka-iz-ssa-vpervye-vystupila-v-luganskoj-filarmonii
 
%%author_id 
%%date 
 
%%tags 
%%title Пианистка из США впервые выступила в Луганской филармонии
 
%%endhead 
\subsection{Пианистка из США впервые выступила в Луганской филармонии}
\label{sec:20_12_2021.stz.news.lnr.lug_info.1.pianistka_lisica}

\Purl{https://lug-info.com/news/pianistka-iz-ssa-vpervye-vystupila-v-luganskoj-filarmonii}

Известная пианистка из США Валентина Лисица впервые выступила в Луганской
академической филармонии в концертной программе академического симфонического
оркестра. Об этом сообщил отдел информационной политики и рекламы учреждения
культуры.

\ii{20_12_2021.stz.news.lnr.lug_info.1.pianistka_lisica.pic.1}

\enquote{В первом отделении программы артистка в сопровождении академического
симфонического оркестра под управлением заслуженного деятеля искусств ЛНР
Александра Щурова исполнила концерт для фортепиано с оркестром № 3 Сергея
Прокофьева. Второе отделение было посвящено творчеству Сергея Рахманинова,
прозвучал его концерт для фортепиано с оркестром № 2}, – говорится в сообщении.

\ii{20_12_2021.stz.news.lnr.lug_info.1.pianistka_lisica.pic.2}

По окончании концерта пианистку поблагодарили министр иностранных дел ЛНР
Владислав Дейнего, министр культуры, спорта и молодежи ЛНР Дмитрий Сидоров и
директор филармонии Вера Геций, которые вручили гостье благодарственные грамоты
и цветы.

Лисица отметила, что испытала \enquote{невероятное вдохновение от публики, от работы с
оркестром в его родном городе}.

\enquote{Ранее мы уже выступали вместе в Крыму. Я счастлива, что теперь состоялся мой
дебют в Луганске. Прокофьев всегда по-особенному звучит на земле Донбасса. Мы
понимаем, насколько важна русская музыка, она в нашей крови, она – частичка
нас. Зрители, которые были сегодня в зале, чувствуют эту музыку особенно остро
в этой ужасной ситуации, которая происходит, в ситуации страшных потерь,
борьбы, несчастья. И в то же время присутствует надежда, понимание того, что у
нас есть такая музыка, за которую мы тоже боремся}, – сказала она.

Артистка добавила, что будет рада вновь приехать в Луганск с концертами.
