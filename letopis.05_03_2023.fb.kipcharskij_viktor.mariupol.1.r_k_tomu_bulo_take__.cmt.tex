% vim: keymap=russian-jcukenwin
%%beginhead 
 
%%file 05_03_2023.fb.kipcharskij_viktor.mariupol.1.r_k_tomu_bulo_take__.cmt
%%parent 05_03_2023.fb.kipcharskij_viktor.mariupol.1.r_k_tomu_bulo_take__
 
%%url 
 
%%author_id 
%%date 
 
%%tags 
%%title 
 
%%endhead 

\qqSecCmt

\iusr{Zhanna Savelyeva}

Психологічний портрет мародера це щось для наукових досліджень: ти можеш
померти зараз чи завтра, дім сгорить, але преш машинку із останніх сил.

\begin{itemize} % {
\iusr{Віктор Кіпчарський}
\textbf{Zhanna Savelyeva} 

Мабуть це в генах: чесні і сміливі люди завжди гинуть першими і дуже часто або
не встигають залишити нащадків, або їх дітей виховують інші - не такі чесні і
не такі сміливі.

В той же час більше шансів вижити, дати і виховати "потомство" в важкі (в усі
часи!) мають крадії, пристосуванці, підлабузники, то що.

Це стара і завжди актуальна тема:

\obeycr
Талантам нудно помогать -
Бездарності пробьются сами.
Або
Добро должно быть с кулаками,
Ножом, кастетом и клыками...
Насправді Станіслав Куняев написав:
Добро должно быть с кулаками.
Добро суровым быть должно,
чтобы летела шерсть клоками
со всех, кто лезет на добро.
\restorecr

\iusr{Zhanna Savelyeva}
\textbf{Віктор Кіпчарський} 

в доньчиній квартирі винесено усе, навіть клятий унітаз, все, окрім важкого
шкафа... У місті 70\% житла нема і жителів. А унітази знято, світильники вирвано
з коренем. А касові бокси з супермаркетів? Вони ж важать як танк... це щось за
межею свідомості

\iusr{Віктор Кіпчарський}
\textbf{Zhanna Savelyeva} 

Це зробили люди, яки вчилися у тих самих школах, що й ми або наші діти.

Чому одним пішли захищати країну, а інші служити окупантам?

На 23 лютого був концерт, який вели три курви: усі три - з Маріуполя.

Що з нами не так?

\iusr{Zhanna Savelyeva}
\textbf{Віктор Кіпчарський} 

причиною "складного", "важкого" клімату у місті та великої кількості
"особистостей не спотворених інтелектом" були металургічні підприємства.
Містоутворюючі. Моя думка.

\iusr{Віктор Кіпчарський}
\textbf{Zhanna Savelyeva} 

Як на мене, коли греки розібралися, куди їх завели і почали тікати, навколо
грецьких селищ почали селити німців (Шанталь, Каплани), донских (Талаковка) та
українських (Нікольське - хутір Гладкого) козаків, болгар, сербів: народи з
різною мовою та релігією. Все це погіршило навіть не стільки виникнення великих
заводів (звідусіль люди їхали "на шахти"), скільки те, що Маріуполь був і ж
кінцевою станцією залізниці: ті, хто їхав дійсно заробити гроші, сходили на
попередніх станціях, а менш тверезі - спали у вагонах "до кінцевої".

Плюс: "обожнюю груповуху - завжди можна "сачканути"" - не бачити результатів
власної праці у кінцевому результаті - не бути зацікавленим у якості власною
праці. Я пересиджу у тіньку, а працює хай "той парень".

Ну і Ви вірно звернули увагу на інтелект: випускник ПТУ отримував зарплатню у
рази більше інженера, не кажучи вже про вчителя.

\iusr{Sergey Drovorub}
\textbf{Zhanna Savelyeva} 

Таких дослідженнь є мабуть багато.

Колись давно порадили і там є пояснення. Читав російською:

\url{https://chtyvo.org.ua/authors/Fromm_Erich/Maty_chy_buty7/}

\ifcmt
  igc https://i2.paste.pics/36a9137ee07dcb4c831afbc870ea3652.png
	@width 0.8
\fi

\iusr{Zhanna Savelyeva}
\textbf{Віктор Кіпчарський} 

в пко завода працювали ветерани комуністичної праці, більшість з російським
корінням, як і у цехах. Розповідала одна жінка як будували у 60ті Стан 3600.
Приїздили (або зганяли) з усіх куточків срср, з західної України також. "Буду я
тягати за троячку отую тачку" - говорили заробітчани з західних рубежів, та
поверталися додому. З цього можна заключити не те, що ліниві, а те що цінували
свій труд, вимагали більшого. Залишалися ті, хто згодні були на менше.

\iusr{Віктор Кіпчарський}
\textbf{Zhanna Savelyeva} 

А згадайте Федю: "Кто не работает - тот ест! Запомни, студент" і багато
"студентів" це запам'ятали... І дітям своїм переказали...

\iusr{Марина Солошенко}
\textbf{Віктор Кіпчарський} 

у нас теж мародери хозяйнували, але їх таки затримували ( громадяни міста)
знімали штани, та прив'язували до стовпа, їх дальнішу участь вже закон
вирішував, зараз все гаразд.

\iusr{Марина Солошенко}
\textbf{Віктор Кіпчарський} 

мене це також хвилює, в Бердянську на цей час багато маріупольців, вони
розповідають, що це українські війська бомбили і нищили місто, а мій брат з
жінкою, виїзжав з міста 6 квітня і по дорозі на Мангуш, потрапив на український
блокпост, де їх машину розстріляли і вони ледве доїхали до росіян! Ось і
докажи, що це не так, спілкуватись з ними більше не хочу!

\iusr{Віктор Кіпчарський}
\textbf{Марина Солошенко} Це секта розп'ятого хлопчика у трусиках - розмовляти з ними нема сенсу. Вони вірять!

\iusr{Zhanna Savelyeva}
\textbf{Марина Солошенко} питаю таких: чому думаєте, що це були українці? Так форма ж чорна. Чому думаєте що в українців чорна форма? "Ой, не знаю!"

\iusr{Віктор Кіпчарський}
\textbf{Zhanna Savelyeva} Ви що, не пам'ятаєте "Правий сектор"?

\iusr{Zhanna Savelyeva}
\textbf{Віктор Кіпчарський} на війні ж піксель. Чи я помиляюся...

\iusr{Віктор Кіпчарський}
\textbf{Zhanna Savelyeva} 

З цим питанням не до мене...

Одне скажу: волонтери можуть дістати будь-що. Тому...

Але чорний???

Хіба що готуються до звільнення вугільних шахт ...

\iusr{Віктор Кіпчарський}

Навздогін: після розповідей "очевидців" про негрів в Урзуфі на базі Азову -
мене важко чимось здивувати: хіба що я зустріну "очевидця", який вміє знімати
на мобілку...

\end{itemize} % }

\iusr{Светлана Водзянская-Живогляд}

Вибачте, я у Вашу історію додам трохи своїх спогадів у коментарях, можна?

\begin{itemize} % {
\iusr{Віктор Кіпчарський}
\textbf{Светлана Водзянская-Живогляд} Чом би й ні?

\iusr{Светлана Водзянская-Живогляд}

У цей день ми все таки зламали супротив мого хрещеного і вмовили вивезти товар
з ринку ( все своє життя вони були приватними підприємцями, може не дуже
успішними часами, але починали зі старої розкладушки на якій в день вони
торгували, а вночі я на ній спала.) Задача була вивезти 4 точки під верх
забитих речами. Ми грузили все в ящики, пакети, коли все скінчилося то
притягнули з дому простирадла і скручували вузлами. В мене страшно боліло все
тіло, але я тягла скрізь сльози розуміючи що моїм хрещеним буде набагато гірше
в разі втрати товару бо це їх хліб і все що нажито за життя...

\iusr{Светлана Водзянская-Живогляд}

В той день я познайомилася зі сторожем ринку, добрий чоловік, трохи на підпитку
з сумними очіма. Я трохи з ним побалакала, спробувала підтримати, він
посміхався розказуючи яку він вдома робить настоянку з вишні, я пообіцяла
принести на пробу свою ( сама майже не пью, куми дуже люблять її). Пакували
речі мене наче в спину гнали, я постійно казала \enquote{швидше, ще швидше}, вечорі
Богові та своїм янголам охоронцям казала \enquote{дякую!} Бо через 2 години саме туди
було скинуто декілька авіа бомб...

\iusr{Светлана Водзянская-Живогляд}

А ще на ринку з дозволу охоронця ми пішли до зоомагазину і взяли кілька мішків
корму для собак і кішок тому що тварин багато, а брати корм нема де.. на
кожному був цінник і я собі записала скільки я винна хозяйці ( в мене як у
постійної покупчині є її номер, на зв'язок вона так і не вийшла). На
зворотньому шляху зайшли у Єву, вона була вскрита вже, там були наші військові,
ми попросили памперси для сина, памперсів вже не було, але хлопець завів нас на
склад і з нами разом намагався знайти щось схоже, знайшли урологічні прокладки,
він сказав брати стільки скільки можемо унести бо коли все скінчиться невідомо,
а від холоду дитина пісяється постійно. Дякую тому хлопцю, бо ними ми і
спасалися!!!

\end{itemize} % }
