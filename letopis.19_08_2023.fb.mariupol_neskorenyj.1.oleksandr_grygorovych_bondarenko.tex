%%beginhead 
 
%%file 19_08_2023.fb.mariupol_neskorenyj.1.oleksandr_grygorovych_bondarenko
%%parent 19_08_2023
 
%%url https://www.facebook.com/100066312837201/posts/pfbid02yFD1TBnWqJ1yWQyCEwfqGATKbSxfXsvvYs1FtzWL569yys3ZJrjEUcA8GVwW5wkLl
 
%%author_id mariupol_neskorenyj
%%date 19_08_2023
 
%%tags 
%%title Олександр Григорович Бондаренко
 
%%endhead 

\subsection{Олександр Григорович Бондаренко}
\label{sec:19_08_2023.fb.mariupol_neskorenyj.1.oleksandr_grygorovych_bondarenko}

\Purl{https://www.facebook.com/100066312837201/posts/pfbid02yFD1TBnWqJ1yWQyCEwfqGATKbSxfXsvvYs1FtzWL569yys3ZJrjEUcA8GVwW5wkLl}
\ifcmt
 author_begin
   author_id mariupol_neskorenyj
 author_end
\fi

Продовжуємо знайомитися з художниками, які беруть участь у виставковому проєкті
\enquote{Маріуполь нескорений}. @igg{fbicon.heart.white.middle}

Сьогодні це український художник, член Спілки художників України з 1988 року,
один із засновників об'єднання маріупольських художників-авангардистів
\enquote{Маріуполь 87} - Alexander Bondarenko.️😊

✴️ В його творах і зараз можна почути діалог між художниками-заснов\hyp{}никами
новаторських течій і сучасним мистецтвом. Усе творче життя Олександр Григорович
Бондаренко опановує певні категорії виразності – конструкцію, паузу
(порожнечу), ритм, колір, світло. Саме за цим принципом створені картини митця.
Його абстрактні композиції викликають різні асоціації. В творах митця –
ірраціоналізм, відхід від ілюзорно- предметного зображення, абсолютизація
чистого враження та самовираження митця засобами геометричних фігур, ліній,
кольорових плям. 

Як і багато маріупольських художників, Олександр Бондаренко після початку
повномасштабної війни, залишався у місті, в своїй квартирі, поруч зі своїми
картинами. Приморський район, де жив митець, спочатку не дуже обстрілювався
ворогом. Ситуація змінилася у середині березня 2022 року, коли ворог з
корабельної артилерії став гатити по місту.  У ті дні  Приморський район став
одним з найбільш небезпечних районів міста. 😪

16 березня  Олександру Григоровичу вдалося виїхати з міста. А наступного  дня
ворожий снаряд влучив у будинок, де на п'ятому поверсі  жив митець, в отворі
стелі стало видно небо, нерозірваний снаряд стирчав на даху будинка. Про це
пізніше Олександру Григоровичу розповіли сусіди... 

✨️Завдяки допомозі одеського художника Сергія Савченка, йому вдалося зв’язатися
з дніпровськими художниками, які йому виділили чудову майстерню, де він і зараз
працює. Олександр Григорович вважає це великою вдачею. 

🖼🎨 У проєкті \enquote{Маріуполь нескорений} беруть участь дві роботи митця. З ними
можна бути познайомитися 21 серпня на відкритті виставки у Київській галереї
мистецтв \enquote{Лавра} і скласти особисту думку про творчість художника.

Департамент культурно-громадського розвитку Маріупольської міської ради \par
Diana Tryma \par
Галерея мистецтв \enquote{Лавра} \par
Костянтин Чернявський \par
Група \enquote{Національна Спілка Художників України} \par
Маріупольська міська рада \par
КУ \enquote{Маріупольський краєзнавчий музей} \par
Місто Марії \par
\#Маріупольнескорений \#виставка \#культурнадеокупація \#Маріуполь \#Київ \#художники\par
ЯМаріуполь\par
