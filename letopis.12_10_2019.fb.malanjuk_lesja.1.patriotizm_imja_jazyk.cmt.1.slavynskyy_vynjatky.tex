% vim: keymap=russian-jcukenwin
%%beginhead 
 
%%file 12_10_2019.fb.malanjuk_lesja.1.patriotizm_imja_jazyk.cmt.1.slavynskyy_vynjatky
%%parent 12_10_2019.fb.malanjuk_lesja.1.patriotizm_imja_jazyk.cmt
 
%%url 
 
%%author_id 
%%date 
 
%%tags 
%%title 
 
%%endhead 

\paragraph{Pavlo Slavynskyy - Теоретично можливі винятки ...}

\begin{itemize} % {
\iusr{Pavlo Slavynskyy}
Теоретично можливі винятки - коли батьки назвали російським ім'ям, а людина свідома, тільки паспорт не хоче змінювати.
Але в 99\% це саме так.

\begin{itemize} % {
\iusr{Kyjeslav Borščahiveć}
\textbf{Павло Славинський} мене обурює 3 сторінка паспорта, де моє ім'я російською передають як "Павел". Питання, чого західнослов'янські імена не адаптують, а українські та білоруські перекладають

\iusr{Igor Noga}
Не існує ніяких "російських імен" Тому що самі узкіє це фікція. Діагноз я б навіть сказав  @igg{fbicon.smile}  Назвіть хоч одне.

\iusr{Ольга Когнітенко}
\textbf{Павло-Києслав Блізніченко} , та поміняйте вже той паспорт на ІД картку  @igg{fbicon.face.smiling.eyes.smiling} . Бачи ли б ви як у флюрографістки око сіпається, коли нама в паспорті росіянського варіанту  @igg{fbicon.laugh.rolling.floor} .
\end{itemize} % }

\iusr{Сергій Денісенко}

От я дуже давно створив акаунт, ще коли спілкувався переважно російською. Тепер
і прибрав би це написання, та не можу знайти, як саме. Створювати новий акаунт
не хочеться. Виходить, кожен бачить ім'я тією мовою, що налаштована в нього.

\begin{itemize} % {
\iusr{Леся Маланюк}
Я бачу «Сергій Денісенко».)

\iusr{Сергій Денісенко}

Але в розмовах іноді, коли відповідають, вилізає посилання на "Сергей ...".
Залежить від налаштувань. Одразу видно мову інтерфейсу в людини, навіть якщо
пише українською. Своєрідний індикатор

 @igg{fbicon.smile} 

\iusr{Pavlo Slavynskyy}
\textbf{Сергій Денісенко} Я це свідомо використовую. У мене ім'я для українців - одне, для іноземців - інше. По відповідях одразу бачу, як налаштований інтерфейс у співрозмовника.

\iusr{Олеся Гапанюк}
\textbf{Леся Маланюк} я також

\iusr{Костянтин Собіченко}
\textbf{Сергій Денісенко}, 

але ж в налаштуваннях, наскільки пам'ятаю, можна змінити ім'я повністю. У вас
там фігурує десь "Сергей"? Бо я, наприклад, бачу лише "Сергій Денісенко". А на
вашій сторінці - "Сергій Денісенко (Serhii Denisenko)". Звідки ж тоді береться
"Сергей"?... Цікава тема. Також неодноразово помічав, що фб-імена людей інколи
набувають інакшого вигляду. Приміром, людина підписана як Іван Петренко, а в
коментарях, коли хтось відповідає на його репліки, активізується чомусь вже
варіант Ivan Petrenko... або й того гірше - Иван.)


\iusr{Kyjeslav Borščahiveć}
\textbf{Павло Славинський} я своє ім'я налаштував лише українською та латинкою, тож всі люди з російським інтерфейсом бачать ім'я латинкою

\iusr{Сергій Денісенко}
\textbf{Костянтин Собіченко} десь "сидить" те, що було введено при створенні акаунту. Як до нього дістатися, не знайшов. Українською та латиною - то "альтернативне написання".

\iusr{Костянтин Собіченко}
\textbf{Павло-Києслав Блізніченко}, а як робиться оце налаштування додаткове, латинкою, наприклад?

\iusr{Людмила Батишкіна}
\textbf{Сергій Денісенко} А чому ДенІсенко, а не ДенИсенко.

\iusr{Kyjeslav Borščahiveć}
\textbf{Костянтин Собіченко} в налаштуваннях можна вибрати "підпис иншими мовами"

\iusr{Сергій Денісенко}
\textbf{Людмила Батишкіна} наслідки совкової транслітерації. Перший паспорт отримував у сересері, там "первинною" була рос.мова, українською лише дублювали.

\iusr{Людмила Батишкіна}
\textbf{Сергій Денісенко} Та це зрозуміло. Тепер переоформити проблематично. Багато возні. Документів багато переоформляти треба
Знаю по зятю.

\end{itemize} % }

\iusr{Yura Yakubovsky}
То все Християнські Імена. І правильно має бути так: Александер, Надія, Інґвар, Катеріна.

\begin{itemize} % {
\iusr{Леся Маланюк}
Тут не про імена, а про те, якою мовою вони написані.

\iusr{Yura Yakubovsky}
\textbf{Леся Маланюк} ми не знаємо ЯК людина записана в паспорті. Може то справді у неї таке імя.

\iusr{Леся Маланюк}
Та невже? Є такі імена Надежда чи Игорь?

\iusr{Yura Yakubovsky}
\textbf{Леся Маланюк} я їм в паспорти не заглядав. Що таке імя? Це показник психологічного здоровя батьків дитини.

\iusr{Василь Іванович}
Що Ви вкладаєте в "християнські імена"?

\iusr{Yura Yakubovsky}
\textbf{Василь Іванович} 

те і вкладаю. Більшість імен сучасних українців - це Імена, дані на честь
Християнських Святих. Вони у свою чергу поділяються на Загальнохристиянські
(зустрічаються серед усіх християн), окремо Католицькі імена (зустрічаються в
основному серед католиків) і окремо православні (зустрічаються в основному
серед православних народів). Наприклад ваше імя Василій - це Християнське імя,
проте в основному Православне. Також в Українському народі присутні язичницькі
словянські імена (тобто зустрічаються лише серед словян) та інші імена.


\iusr{Василь Іванович}

Я не Василій, а Василь... Якщо москвописка "братія" не вміє вимовити це слово,
то хто їм дохтор?.. Я Вас лиш прошу не протиставляти сучасну форму і зміст
нашої віри її попереднім формам і змістам, які всі є нашими, і більш нічиїми,
на цій Мідгард-Землі...

\iusr{Yura Yakubovsky}
\textbf{Василь Іванович} ви саме Василій. Так це імя пишеться в оригінальній грецькій формі. Василь і Вася - то лише скорочення. Чоловічі Християнські імена всі в основному закінчуються на -ій.

\iusr{Леся Маланюк}
\textbf{Юра Якубовський} (\textbf{Yura Yakubovsky}) А якщо людина пише у Фб не Василій, а Василий?

\iusr{Василь Іванович}
Якій ще грецькій мові! Ви маєте на увазі "старогрецьку", тобто - геленську, яка є мовою орійською, санскритом, українською?

\iusr{Василь Іванович}
\textbf{Леся Маланюк} От і його ім'я - це Ур (Ор, Ар, Ір) з йотизованим "Й" -
Йур, Юр. Зменшувально-ласкальна форма від нього - Юрко. Отак ми кличемо його.

\iusr{Yura Yakubovsky}
\textbf{Леся Маланюк} треба дивитися як в грецькому ориґіналі.

\iusr{Yura Yakubovsky}
\textbf{Василь Іванович} яка у вас освіта?

\iusr{Василь Іванович}
\textbf{Юра Якубовський} філологічна, дипломатична і історично-самоучна (що найцікавіше)... Років 40 займаюся історіографією, етнолігнгвістикою і іншим таким цікавеньким - Білінський аддихаєд.

\iusr{Yura Yakubovsky}
\textbf{Василь Іванович} імя Юрко завжди ненавидів. Моє імя Юрій. Походження цього імені цікаве, але це довго розписувати.

\iusr{Василь Іванович}
Дарма. Юрій - це ж Орій, який списом (арімійського-китайського) дракона убиває nach!

\iusr{Yura Yakubovsky}
\textbf{Василь Іванович} ні. Юрій - це поєднання імен Ярослав і Георгій.

\iusr{Василь Іванович}
\textbf{Юра Якубовський} 

Ось-ось! Це ОДНЕ Й ТЕ Ж. Ур-Юр, Ор-Йор, Ар-Яр, Ір-Їр. Jor - прочитайте
англійською (Джор), іспанською (Хор), французькою (Жор)... Це просто Сонце,
Сонячний, ... - і ще одно, але я думаю, Ви поки-що не готовий то сприйняти...

\iusr{Vitalik Stanislav}
\textbf{Василь Іванович} - Віталій ,розпишіть будьте добрий

\iusr{Завгородній Микола-пономар}
\textbf{Юра Якубовський} Інгвар-це скандинавське ім'я.

\iusr{Ivan Prylutskyj}

Що за маячня. Лише одне з них - Катерина - можна назвати можливо християнським.
Олександр - дохристичнське грецьке. Ігор взагалі нехристиянське варязьке, Надія
- питомо наше, дохристиянське...


\iusr{Ivan Prylutskyj}

Всі народи , беручи чужинські імена, спотворюють їх до невпізнання згідно своєї
вимови. І лише х...ли-чухраїнці намагаються витримати всі канони чужинської
мови!

\iusr{Zoya Jakivna}
\textbf{Ivan Prylutskyj} не підтримую, випадково натиснути

\iusr{Микола Романченко}
\textbf{Yura Yakubovsky} Олександр

\iusr{Mykola Krushnitski}
\textbf{Юра Якубовський} грецькою буде Васілеос, а ви перекручуєте

\iusr{Костянтин Собіченко}

Я щось також Інгваром здивований - скандинавське ж нібито. А ще більше - оцими
християнськими історіями. Можна подумати, що грецьке - то все християнське.

\iusr{Yura Yakubovsky}
\textbf{Ivan Prylutskyj} 

Катеріна - Християнське. Були аж дві Святі з таким іменем. Свята Катеріна
Римська і Свята Катеріна Сієнська. Александер - одночасно грецьке імя
Александра Великого (і двох його попередників) і християнське імя (бо було
кілька римських пап з таким іменем). Інґвар/Ігор - Скандинавське, але і
християнське, бо був такий наш правитель князь Ігор ІІ, і його Церква
канонізувала. Надія - поняття, на честь якого Християнська Церква дозволила
називати дітей і адаптовувати це імя відповідно до своєї мови. На Заході теж є
це імя - Хоуп. Поняття Надія завжди було популярне серед християн в усьому
Світі, бо це сутність самого Ісуса Хреста.

\iusr{Yura Yakubovsky}
\textbf{Завгородній Микола-пономар} Ні. Скандинавське імя - Інґвар. Записується, як Ingvar. Через G. Такого імені як Інгвар (Inhvar) - не існує.

\iusr{Yura Yakubovsky}
\textbf{Ivan Prylutskyj} хіба??? В УСІХ народів є імя Александер. І лише українці чомусь спотворюють це імя до Олександр. Хоча там і звуку такого немає.

\iusr{Yura Yakubovsky}
\textbf{Микола Романченко} Александер. Досить псувати прекрасне імя. Там навіть звуку О немає. Лише А.

\iusr{Юрій Іщук}
\textbf{Yura Yakubovsky} 

ага, Катеріна. Літературною українською мовою є мова Тараса Григоровича
Шевченка! І твір його безсмертний називається "КатерИна"! А Ваша КатерІна- то
десь у Італію! І Олександр , а не Александр! Не викручуйте мову під себе...

\end{itemize} % }

\iusr{Володимир Герасименко}
Ви праві

\iusr{Руслан Бащенко}

О, згоден на 100\%. Теж не розумію, навіщо українці пишуть свої імена
російською. Вірніше, розумію - малоросійська звичка, інерція, "так нада"...
Сумно

\iusr{Беляев Алексей}

Не согласен. Меня зовут Алексей , но это не мешает мне защищать нашу Украину. А
котось звати Тарас і т.і. так тільки кричати можуть "ганьба" а як зброю взяти в
руки так і всееееее......

\begin{itemize} % {
\iusr{Леся Маланюк}
А що тобі заважає назватися Олексієм?  @igg{fbicon.smile} 

\iusr{Костянтин Собіченко}
\textbf{Беляев Алексей}, якби ви вважали Україну своєю, то були б Олексієм. А так ви просто найманець, москвин на службі в Україні. Хоча й це, звісно, непогано вже.

\iusr{Pavlo Slavynskyy}
\textbf{Беляев Алексей} А що вам заважає писати українською?

\iusr{Костянтин Собіченко}
\textbf{Павло Славинський}, русскіє корні.)

\iusr{Ольга Храмова}
\textbf{Беляев Алексей} Олексію, здоров*я вам і Ангела Охоронця у вашій важкій праці захищати рідну Україну, повертайтеся живим. Прізвище своє пишіть як хочете, а ім*я пишіть українською всім ворогам назло, Олексій в перекладі Син Божий.

\iusr{Ivan Matviyiv}
\textbf{Беляев Алексей} 

ще один язікій захисник. Жити на землі народу і не знати його мову, або не
хотіти спілкуватися, краще помовчи, так буде чесніше.

\iusr{Беляев Алексей}
\textbf{Костянтин Собіченко} 

такий найманець як я був мій дід Петр який в роки Другої Світової на Первом
Украинском фронте первым вошёл в занятый фашистами Львов , та це йому не
заважало спилкуватись оба мовами

\iusr{Беляев Алексей}

А розповідати хто тут патріот а хто ні в контексті мови я вважаю це
розпалювання війни. Из -за этих разногласий ненужных погибло уже много хороших
людей та справжніх патриотів

\iusr{Костянтин Собіченко}
\textbf{Беляев Алексей}, краще б він туди не заходив.)

Розпалювання війни - це продовжувати утверджувати в Україні москвинський язик,
будь-де, в будь-якій формі. Навіть підписуючи ним свій фб-профіль.

\iusr{Орелі Боян}
\textbf{Беляев Алексей} 

захищать, так. Але і перетворювати Україну в НеоМатрьошкостан, теж. Україна, це
не лише територія. Це і мова, і культура, і традиції.

Ви язиком вбиваєте мову і весь духовний спадок українців. І замінюєте його на
спадок матрьошок.

Тобто такі собі внутрішні хуйлята. Великому Хуйлу не віддамо, самі правити
будемо. А українці і так паймут. Я вас правильно зрозуміла?

\iusr{Орелі Боян}
\textbf{Беляев Алексей} 

краще б не заходив до Львова. Те, що робили московські окупанти у Львові,
німчура і в кошмарах не бачила. Серед двох окупантів з німцями можна було жити.
А з московитами лише вмирати. Тож, не бачу приводу для гордості. Дід-окупант.
Внук досі не українець.

\iusr{Беляев Алексей}

А Вашими взглядами Донбасс лучше не освобождать.....

\iusr{Оксана Петришина}
\textbf{Беляев Алексей} 

Справа не тільки в комунікуванні чи написанні імені. Ви говорите російською,
думаєте нею ж, читаєте кого? Пушкіна чи Шевченка? Ахматову чи Костенко? Увесь
ваш світ - це російський світ. І ви захищаєте Україну, ризикуєте життям, але ви
будуєте в Україні другу Московщину.

\iusr{Орелі Боян}
\textbf{Беляев Алексей} я звільняю, але не Донбас, українців. В кожного своя війна.

\iusr{Pavlo Slavynskyy}
\textbf{Беляев Алексей} Львів німці залишили за 3 дні до підходу радянської армії, там ще ціла історія зі змагання ОУН та АК за прапор була. Куди ваш дідусь входив, ще раз?

\iusr{Беляев Алексей}
Краше прочитати

% -------------------------------------
\ii{fbauth.beljajev_aleksej.kramatorsk.ukraina.gorlovka.policia.mvd}
% -------------------------------------

\ifcmt
  ig https://scontent-frt3-1.xx.fbcdn.net/v/t1.6435-9/72642024_968725403504663_5722446105422069760_n.jpg?_nc_cat=102&ccb=1-5&_nc_sid=dbeb18&_nc_ohc=1tflxS3YmH0AX_wTTmY&_nc_ht=scontent-frt3-1.xx&oh=d40790092a5152bb14fa1ffb77820874&oe=618B5AA3
  @width 0.4
\fi

\iusr{Беляев Алексей}
Как-то так!!!!

\iusr{Andriy Borodavko}
\textbf{Беляев Алексей}
І - що?

\iusr{Руслана Курах}
такі, як Алєксєй, просто псують патрони українцям: і на фронті, і дірявлять тил

\iusr{Роман Волошинюк}

Російська мова в Україні і її носії це в першу чергу символ російської окупації
а носії це представники окупантів!!! І пам"ятаємо, що в Криму і лугандонії саме
"російськомовні" зрадили Україну, що в принципі не дивує!

\iusr{Ivan Matviyiv}
\textbf{Беляев Алексей} , краще б твій дід не заходив у Львів. Знаємо про радянську армію все. Не треба було нас визволяти, ніхто про це їх не просив.

\iusr{Inna Shch}
Ненароком потрапила на цей допис.
Це треба так вміти цькувати людину, яка розмовляє російською, але воює за Україну. Ще такої дурні, як тут написано не читала. Українці, агов, а чи то не люди, які вдягають вишиванку і розмовляють українською в раді вже 28років сидять і щодня зраджують Україну? Чи то не українці, їздять на заробітки до росії. Чи то не українські артисти їздять до росії на гастролі. Може вже досить? Так мовне питання болюче. Але крім того, що треба хотіти розмовляти українською, ще треба і в серці мати Україну.

\iusr{Беляев Алексей}
Дякую Вам велике за цю промову!!!

\iusr{Беляев Алексей}
\textbf{Inna Shchukina} Вони зрозуміють цю проблему коли їх дитина поїде воювати и побачать що разом з нею буде стояти побратим котрий скаже: " Привет я Саня . Я твой второй номер". "Слава Україні".

\iusr{Орелі Боян}
\textbf{Inna Shchukina} бачу аналізувати то теж не ваше. Людина вперто вважає, що москвояз в Україні - норма. А те, що захищає - індульгенція на вживання язика.
Тепер аналіз.
Якщо нащадки росіян захищають Україну, вони мають право на те, щоб тут був москвояз?
Чи захищають Україну і розуміють, що це країна українців і дотримуються мови і культури, і традицій ?
Якщо я воюю і імєю права, то питання, за що воює ? За свою нову Московію?
Друге, якби в Україні були лише українці та ті, хто поважає Україну і її право на саморозвиток, проблеми були б?
Проблема в Україні не українці, а нащадки окупантів, які вважають, що імєют права і не хочуть зрозуміти, що їх права на язик по той бік кордону, а тут їх права нав'язані кров' б, голодом і терором.
У Вас Стокгольмський синдром?
Чи не розумієте про що мова?
Крім етноменшини
московитів є і інші, але ні одна не вважає, що їх мова має бути в Україні на тих же правах, що і українська, або і замість української. Лише ці. Бо їм на підкірку прописали "імєєм права".

\iusr{Орелі Боян}
\textbf{Беляев Алексей} за що дякуєте? Що вас захистили від українців? За що воюєте?

\iusr{Орелі Боян}
\textbf{Inna Shchukina} а тепер по факту. Ви, або втратили коріння, або така ж московка, як і пан Бєляєв.

\iusr{Inna Shch}
\textbf{Орелі Боян} 

ви смішна справді))) то може приїдете до Одеси і навчите як має бути? Може до
ради будете виходити щодня і розповідати, що люди, які все життя розмовляли
російською повинні з наступної секунди перестати так говорити? Може у вас вийде
викорінити те, що впроваджувалось століттями серед українців.

\iusr{Inna Shch}
\textbf{Орелі Боян} мені все одно ваша думка. Але ви якраз добре працюєте заради Кремля, розповідаючи яка ви щира українка, а інші бруд під ногами. На все добре, з шовіністами не веду дискусій.

\iusr{Орелі Боян}
\textbf{Inna Shchukina} а Адєса . Бачила ваших. Місто манкуртів. Асобєнниє гражданє. І не українці, і не росіяни. Не пойми хто. По суті манкурти. Ви з Беляєвим знайшли одне одного.

\iusr{Орелі Боян}
\textbf{Inna Shchukina} бруд під ногами? На Кремль? З таким вінегретом до того, хто це зможе розібрати і пояснити. Не маю ні бажання, ні часу. Схоже чула. 10 жовтня 14 годин вєщало.

\iusr{Елена Норенко}
\textbf{Беляев Алексей} Вас звати ОЛЕКСІЙ! Лесик, Олесь. Така традиція вживання імені є більш притаманя для українців. Так само, якщо жінка відчуває себе українкою, то вона Надія, а не Надєжда. Ось про що йшлося)))

\iusr{Беляев Алексей}
Дірявлять таких як Ви , а я державу захищаю

\iusr{Люда Люда}
\textbf{Беляев Алексей} продаючи її москалям

\iusr{Костянтин Собіченко}
\textbf{Inna Shchukina}, 

за те що воює - цю людину не цькують. Її, якщо й цькують (хоча цього ніхто
насправді не робить), - то лише за нерозуміння того, за що, власне, вона воює.
Вона, людина ця, однією рукою відмежовується від окупантів, а іншою - їхні ж
ідеї й поширює. Як і ви, зрештою. Отака роздвоєність також ні до чого доброго
не призводить.

\iusr{Ігор Кушнірчук}
\textbf{Беляев Алексей}, таваріщь, ви захищаєте руцкій мір в Україні, а не Україну. Ви п'ята колона хуйла. І якщо ви цього не розуміє, то не означає, що це не так.

\iusr{Ігор Кушнірчук}
\textbf{Беляев Алексей}, от твій дід окупант, а ти внук окупанта й внутрішній ворог України.

\iusr{Сергій Жижко}
Пане Олексію.

За Вашу участь у боях за Україну - повага і подячність.

Але це не знімає необхідність бути патріотом України не тільки у бойових діях.

Людина, яка готова віддати життя, померти за Україну, яка бачила причини і
трагедії війни і не прийшла до висновку, що необхідно перемогти себе у
важливому - прийняти українськість у всьому, така людина ще не стала вповні
українцем.

А війна тому, що русскій мір воює за себе і міткою, знаком цього міра є русскій
язик у тому числі як жаргон совка.

\end{itemize} % }

\iusr{Беляев Алексей}
Так ділити нас не треба. ИЗ ЗА ЭТОГО И ВОЙНА У НАС.....

\begin{itemize} % {
\iusr{Леся Маланюк}
Не через московію?

\iusr{Беляев Алексей}
\textbf{Леся Маланюк} через наш розум

\iusr{Костянтин Собіченко}
\textbf{Беляев Алексей},

війна через недолугість малоросійську, через "какую
разніцу на каком язикє", чим так вміло завжди користувалась Московія. От вона й
вклала у ваші вуста оті смисли про "не треба ділити", хоча сама цим постійно і
займалась якраз, нав'язуючи українцям свій язик, щоби ті стали інакшими:
спочатку трішки не такими українцями, а потім взагалі неукраїнцями. А потім -
таке стадо вже не являє собою ніякої загрози, можна брати голими руками.

\iusr{Беляев Алексей}

Я сам з Горлівки та не був там вже 5 років так як я там "предатель", я розумію
що вже ніколи туди не повернусь , але там пройшло моє дитинство, 35 років мого
життя , там жили мої друзі ,там поховані мої родичі. І мені тепер жити з цим
все моє життя. А Ви погодились би втратити все : житло,друзів, родичів, не мати
змоги поховати рідних - заради України......

\iusr{Беляев Алексей}
А я зробив свій вибір!!! У 2014 році взяв у руки зброю і почав вирішувати чиєсь мовне питання!!!!

\iusr{Костянтин Собіченко}
\textbf{Беляев Алексей}, це добре, що взяли зброю. Але мовне питання в першу чергу треба вирішити своє.

\iusr{Орелі Боян}
\textbf{Беляев Алексей} у вас є два варіанти уникнути поділу. Перший говорити українською і нарешті зрозуміти, що тут Україна, а не НеоМатрьошкостан. Другий емігрувати з України.

\iusr{Беляев Алексей}
Хорошо судить когда спишь на своей кровати у себя дома!!!!

\iusr{Орелі Боян}
\textbf{Беляев Алексей} 

якби ви говорили українською і вся Горлівка, Хуйло б не прийшло. Ви вважаєте,
що лише постраждали? А ми всі живемо і радіємо? У всіх нас вкрадено 5 років
життя. У всіх нас гинули друзі, рідня, знайомі. Всі ми замість жити своє життя
маємо воювати. Кожен на своєму фронті. І в кожного нема спокою. Бо колись
емігранти вирішили що з України можна зробити Московію, замість зрозуміти, що
тут Україна.

\iusr{Орелі Боян}
\textbf{Беляев Алексей} 

ви знаєте чим кожен з нас мав жертвувати? А наші роди яку ціну заплатили за ваш
язик? Це для вас війна в 2014 почалася. А для нас триста з гаком років тому. А
останні сто років ми мали пекло від ваших. Тому так, ми спимо в своїх ліжках,
поки що.

\iusr{Оксана Петришина}
\textbf{Беляев Алексей} У нас війна, бо ви не вивчили українську мову!!!

\iusr{Руслана Курах}
\textbf{Беляев Алексей} 

"Хорошо судить когда спишь на своей кровати у себя дома!!!!" - ти диви, яке
нахабне.. так це через вас, защемлених стільки жервт, стільки кровопролиття, тож
платіть, паскуди, і не скигліть! воно ще й розмахує, замість посмертного
відчуття провини... тварюка така.. пішов на родіну, в гробу я бачила таких
"защітнікаф"! потвора кінчена.

\iusr{Беляев Алексей}
Паскуди твої діти якщо ти так говориш

\iusr{Тетяна Бевська}
\textbf{Беляев Алексей} Ви пішли захищати Батьківщину (нехай і малу - Горлівку) від російського окупанта, а не мовне питання вирішувати.

\iusr{Беляев Алексей}
Саме так!!!!

\iusr{Орелі Боян}
\textbf{Беляев Алексей} 

а хто від таких як ви захистить Україну? Адже таких, як ви-багато. І ви своїми
руками вбиваєте Україну. Очі відкрийте, нарешті. Ви захищаєте московський
гібридний світ і вбиваєте Україну. Нема України москвоязичної, це-колонія
московитів. Україна-українська. Все інша окупована Україна. На даний момент
окупували майже всю територію. І такі як ви навіть не усвідомлюєте, що нащадки
вбивць українців, а не освободители. І доки ви це не зрозумієте, доти Ви проти
України.

\iusr{Орелі Боян}
\textbf{Тетяна Бевська} а він який окупант? Не російський? Просто він - внутрішній окупант. А то зовнішні. А все це разом - гібридна війна.

\iusr{Люда Люда}
\textbf{Беляев Алексей} не із-за цього, подумайте, згадайте..

\iusr{Ігор Кушнірчук}
\textbf{Беляев Алексей}, о московитські методички пішли в рух. Хто б сумнівався, що ви споживач московитського лайна.

\iusr{Yuriy Kozik}
\textbf{Беляев Алексей} тоді зміни своє імя на фб, бо для мене це написано ворожою москальською мовою. Чи ти саме так себе позиціонуєш?

\iusr{Беляев Алексей}
Саме так для Вас шановні!!!

\iusr{Беляев Алексей}
И писать в ФБ свое имя я буду как свободный человек !!!! От маразма и дебилов...

\iusr{Ігор Кушнірчук}
\textbf{Беляев Алексей}, от ти й проявив себе московите. Ти і є ворог України.

\iusr{Ігор Гримайло}
\textbf{Беляев Алексей}
Війна не через мову, а через Путіна, через його найманців, через його поставки зброї та боєприпасів в Донбас.
Невже це для вас новина?

\iusr{Галя Волошин}
\textbf{Фрейр Бинтмакерь} а ваше ім'я якою написане?

\iusr{Галя Волошин}
\textbf{Беляев Алексей} 

мовним питанням путін тільки прикривається для виправдіння воєнної агресії. І
ще мовне питання використовують, щоб розсварити нас і довести до громаддянської
війни.

Насправді плани у нього великі і це зовсім не захист язика))).

Біда в тому, що ви "російськомовні українці" ненавидите Україну і українську
мову більше за того самого путіна. Бо як інакше пояснити той несамовитий
спротив переходу на українську мову? Варіантів лише два:

\begin{itemize}
  \item 1- ненависть до України
  \item 2- нездатність вивчити мову, тупість. @igg{fbicon.shrug} 
\end{itemize}

\iusr{Тетяна Бевська}
\textbf{Беляев Алексей} 

а коли станете захисником Батьківщини - України, то зникне питання мови. В
Польщі - польська, в Чехії - чеська, в Угорщині - угорська, а в Україні тільки
українська.

\iusr{Ігор Кушнірчук}
\textbf{Галя Волошин}, моє ім'я українці бачать, а решті не обов'язково.

\iusr{Галя Волошин}
\textbf{Фрейр Бинтмакерь} я українка. Без ваших викрутасів проживу і українкою бути не перестану

\iusr{Сергій Тростинський}
\textbf{Беляев Алексей} 

війна не через те що путін гіркіна привів? Через те що всетаки ділять нас? А ви
не діліться, об'єднайтесь навколо державної мови, і все, згідно вашої теорії війна
закінчиться.

\end{itemize} % }

\iusr{Беляев Алексей}
Усі ми діти ГОСПОДНІ!!!

\begin{itemize} % {
\iusr{Орелі Боян}
\textbf{Беляев Алексей} українці діти Божі. Над ними нема господина. Лише Бог.
\end{itemize} % }

\iusr{Беляев Алексей}
Кому як зручніше....

\iusr{Беляев Алексей}
Власно мені подобається і так і так...

\begin{itemize} % {
\iusr{Руслана Курах}
пішов за порєбрік. ВОРОГ

\iusr{Микола Владзімірський}
та звичайне троленятко воно

\iusr{Беляев Алексей}
Из-за таких как Вы война идёт и люди гибнут каждый день. Писаки

\iusr{Беляев Алексей}
Приїдь та подивись ворогу в очі....

\iusr{Беляев Алексей}
Писать все мастики а когда 152 рядом упадет обсераются все одинаково

\iusr{Ірина Мельник-Демченко}
\textbf{Микола Владзімірський} цікаво, їх у тролошколі всіх вчать такі дурнуваті ави ставити?

\iusr{Надія Синявська}
\textbf{Беляев Алексей} оце сів путін, почитав фб, і -,,ти диви, які писаки!,, а піду но я на них війною, ато бач, пишуть! так, Олексію?

\iusr{Костянтин Собіченко}
\textbf{Беляев Алексей}, і тут урівняв - знову какая разніца. А чому москвин, коли впаде 152, обсирається лише своєю мовою? Йому чомусь обирати не доводиться. Чому?
\end{itemize} % }

\iusr{Олекс Музичко}

Натомість відбувається гірше: навіть українки соромляться імені ГАННА -
поголівно Анни. Чоловіки вже соромляться імен Євген - поголівно Євгенії.
Соромляться імені Микита. Не теоретизую, а бачу по студентах. Тотальне
зросійщення. Ця нещасна недодаержава немає жодних шансів. Ну то може турки,
вєтнамці, китайці краще впораються на цій благодатній землі. Бо мають гідність.

\begin{itemize} % {
\iusr{Анна Брикова}
\textbf{Музичко Олександр} 

ну, по-перше, Анна і Ганна - це фонетичні варіанти одного і того самого імені,
і жоден з них не суперечить українському правопису. По-друге, чому Ви вирішили,
що це сором? Я, наприклад, переконана, що сором - це називати свою Батьківщину
"недодержавою".

\iusr{Орелі Боян}
\textbf{Музичко Олександр} 

дивлячись де. Західна Україна завжди Анни. Підозрюю, що причина в
католицизмі...

\iusr{Serhii Bryhar}

Особливо ображаються на Кирило. Це прямо як матюк). "Я Кірілл, - кажуть, - а
тупую укрАінскую вєрсію нє воспрінімаю". Знаю Владімірів, які категорично проти
того, щоб бути Володимирами. Про Олександрів уже навіть мовчу... найпоширеніше
нині взагалі "Алєкс"). Тому так, все може і не так швидко, але цілком впевнено
котиться в сраку! А тепер щодо моїх переконань. Пару слів... Я вважаю, що зараз
потрібно рятувати принаймні умовно здорове! Івано-Франківськ ще не Одеса,
звісно, але все може бути... На всій території держава, як на мене, вже
неможлива, але це не означає, що її потенціал вичерпано. Держава - це не
території, держава - це люди, об'єднані спільною ідентичністю, спільними
цінностями і завданнями. Так, адекватніша частина цього суспільства, перед усім
україномовна, теж значною мірою задурманена, але тих людей можливо приводити до
тями, бо в них є українська ідентичність, але ми також маємо території, де, в
принципі, можна просто замінити прапор, а все інше вже готове...

\iusr{Руслана Курах}
Анна Ярославна...мабуть, теж соромилася свого імені. Олександре, посоромились
би Ви краще російських репостів, або переклали б урешті..Де Ваша гідність?

\iusr{Орелі Боян}
\textbf{Руслана Курах} 

тоді не говорили ще Ганна. Це ж перші часи християнізації. А Ганна то вже
пізніша форма, коли чужі імена підлаштовували ппід українську мову. Ганна не
єдина форма було і Гандрій, і Гартем, Галина і спрощення Горпина-Агріпінна... А
у всіх іменах прикривала Г. І це не лише в українській. В німецькій Ханна... В
англійській Генрі - Анрі - Андрій...

Але те, що українці частково не сприймають українізовану форму імен, то це є,
тут навіть доводити нема чого.

\iusr{Руслана Курах}
\textbf{Орелі Боян} Анна - цілком українське ім'я

\iusr{Сергій Шумський}
\textbf{Орелі Боян} 

в німецькій «х» на письмі передається «ch». «h» - звучить ближче до
українського (польського, чеського, навіть івритського, від якого, власне hanna
і походить) «г».

До речі, любителям «анн» варто знати, що anna і hanna - це не одне і те ж.
Hanna - давньогебрейське (щось із Богом пов‘язане). Anna (anne) - тюркська
(ординська) «мати».

Ну, це вже кому що до вподоби... Як на мене, що те, що инше - невластиві
українцям чужоземні імена.

Плутанина з «ганна»-«анна» привнесена в Україну московитами, які через
відсутність у них звуку «г» взагалі в перекладах гебрейських текстів в іменах
«Ганна» викинули першу літеру (звук). Так у давніх гебреїв і з‘явилися «Анни»,
чому ті, певно, дуже би здивувалися... До речі, поширене західноукраїнське
«Анна» теж пов‘язана з москвофільським рухом 19-го сторіччя, коли попи там, на
прохання парафіян, почали активно використовувати московицькі святки. Там і
досі позиції МП доволі міцні, особливо зваживши на т.з. русинський рух і його
очільників...

І ще: оце ось поширена останнім часом передача фонетики іноземних "Hanna"
українською, як "Ханна" - це теж вплив колоніяльного під-московицького минулого
оцих "користувачів", які так роблять. Адже московити, не мавши повноцінного
звуку "г" ("h") перекладають цей поширений в світі звук то як "х", то як "ґ"
(наприклад, латинське homo у них то "хомо" (коли сапієнс), а то "ґомо" (коли
сексуаліст)).  Тому, шануймося, товариство. Тримаймося свого: "Hanna"
українською звучить як "Ганна". Ніякої "Ханни" у світі не існує... Ну, а "homo"
- завжди українською буде "гомо". Не уподібнюймося неповноцінним сусідам...

\iusr{Тереза Ониськів}
\textbf{Музичко Олександр} 

взагалі Ганна і Анна це різні імена, хоча чомусь побутує думка, що це одне ім'я
різними мовами. Для простого доказу наведу фільм Hannah Montana, яка точно не
Anna. Складніші пояснення самі знаєте, де шукати

\iusr{Костянтин Собіченко}
\textbf{Serhii Bryhar}, так, Франківськ - ще не Одеса, але вже перші натяки присутні. Власне, не лише про Івано-Франківськ мова - це вся Галичина така: ще говорить українською (звичка, інерція), але в душі вже готові до какой разніци. Проте, до тями їх повернути куди легше, згоден.

\iusr{Руслана Курах}
Ханна і Ен. не знаю, мені Алла - більше російське, ніж Анна, а більше в нас немає жіночих імен-паліндромів(

\iusr{Олекс Музичко}
\textbf{Тереза Ониськів} Я про те, що Ганни, ідентифікують себе саме як Анни. А виверти можна вигадати, як і виправдання деукраїнізації хоч 10000000000 штук. Малоросійська сутність від цього є незмінною.

\iusr{Олекс Музичко}
\textbf{Руслана Курах} Немає. Уся до вас перейшла, світоч ви наш ясний. Сподіваюся не лише у ФБ, як більшість?

\iusr{Руслана Курах}

моя мала гарно малює. було б прекрасно, якби вона стала всесвітньо відомою
дизайнеркою, чи архітекторкою, придумавши собі креативне лого, чи гарно
написане ім'я, де перша і остання літери "А" у вигляді Ейфелевих веж... але вам,
Олександре, малоросійство затнулося. кому що болить.. менше отого раша-лайна
споживайте, яке ширите, має допомогти. без образ.

\iusr{Олекс Музичко}
\textbf{Орелі Боян} 

З вашою збоченою логікою ви дуже скоро оголосите українську мову "вивертом
польською", як ваші чорносотенні попередники. Ось через таких як ви люди і не
сприймають думки про існування питомих форм українських імен і українські імена
зникають, адже росіяни не влаштовують собі такого дебілізму і не розсуджують,
що вони мають бути повально Іоаннами, а не Ванями-Іванами.

\iusr{Руслана Курах}
іноді бажання відрізнятись від російського перетворюється в абсурд.. це щодо
написання слів, імен.. якось слід і міру знати. в мене взагалі якесь тюркське
ім'ячко, тож мовчу

\iusr{Олекс Музичко}
\textbf{Костянтин Собіченко} 

Так в Одесі теж половина населення, це колишні західняки, але хитрі, такі, що у
ФБ сміливі, а потім опиняючись в Одесі, готові до зради, бо сутнісно вони і там
такі

\iusr{Орелі Боян}
\textbf{Руслана Курах} воно запозичене. Але для Західної України воно звичне. Там ніколи не говорили Ганна. Анничка, Анна, Анечка..

\iusr{Орелі Боян}
\textbf{Музичко Олександр} кожен в інших себе бачить. В мене в активному словнику навіть слова збеченець нема, у Вас бачу присутнє.

\iusr{Людмила Маркевич}
\textbf{Музичко Олександр} , Ганна, Анничка, Ганнуся...- це гарно... А от ,,Анічка"- це жах...

\iusr{Ольга Сав'як}
\textbf{Орелі Боян} На Західній Україні називали і Ганна , і Анна (як матір Богородиці). Ганна (Анна) давньоєврейське , Channa від chanan - він був милостивий , виявляв ласку.

\iusr{Світлана Самохвал}
\textbf{Музичко Олександр} Шанс завжди є, не перебільшуйте!

\iusr{Ірина Бусь}
\textbf{Музичко Олександр} А як могло б звучати: Ганнусенька - як пісенька!!!@igg{fbicon.heart.red}

\iusr{Дарія Матвіїв-Веклин}
\textbf{Сергій Шумський}

моя свекруха була 1916 року народження, галичанка, але називалася Анною. В
Карпатах, наприклад, нікого не називали в побуті ні Ганною, ні Ганею, ні Гандзею, ні
Ганнусею. Були -Анни, Анці, Аннички ще задовго до совєтів. Колись одна пані
звинувачувала мене в дефіциті патріотизму так як їй видалося замало українським
написання мого імені.


\iusr{Сергій Шумський}
\textbf{Дарія Матвіїв-Веклин} я про часи «совєтів» мову не веду. 19-те сторіччя і москвофільсьський рух в Галичині й Закарпатті почалися задовго до совєтів.

\iusr{Atmeshvar Anatoly}
Анна - це західноукраїнський варіант Ганни.

\end{itemize} % }

\iusr{Костянтин Собіченко}

Це дикий, дикомалоросійський, страшнохохляцький (не знаю, як ще сказати)
феномен. Загагдка загадок. Я знаю людей, від яких ніколи не чув жодного
москвинського слова, проте, тут, у ФБ, вони підписані саме москвинською мовою.

Насправді, якщо розібратися добре, не така це вже й загадка...

Треба москвинам віддати-таки належне, потрудилися на славу. Багато чого не
вдалося їм, бідолахам, але й зроблено чимало. І що цікаво - продовжують
працювати, завзято, методично, з покоління в покоління. От де національна ідея
поставлена! Нам би навчитися.

\begin{itemize} % {
\iusr{Надія Пройдак}
\textbf{Костянтин Собіченко} Підписуюсь під кожним словом.

\iusr{Орелі Боян}
\textbf{Костянтин Собіченко} читала, що це прямі наслідки Голодомору і терору. Комплекс меншовартості нав'язали. Українська досі сприймається, як недомова, на підсвідомому рівні.

\iusr{Орелі Боян}
Це ж як концтабір. Порядки нав'язали, закріпили. Потім колючку зняли і пішли, але є порядки закріплені.

\iusr{Костянтин Собіченко}
\textbf{Орелі Боян}, так, очевидно, маємо справу саме зі сферою підсвідомого. Бо іноді в інший спосіб і не поясниш суть цього явища.

\iusr{Тарас Корпало}
\textbf{Костянтин Собіченко} 

Ніякого "підсвідомого" не існує. Це шарлатанська вигадка фрейдів. Є різні
фізіологічні процеси, є суто автоматичні акти ВНС тощо, що впливають на
мислення, але називати їх "підсвідомим" (якимось особливим мисленням, що
відбувається у сфері "невідкритого" і тому не доступне для рефлексії мислення
"самовідкритого") підстав нема. Хоча ті ж американські університети десь
розколоті 50/50: одні - Фрейд шарлатан, другі - великий психоаналітик. Оскільки
там вважається владою, що посилати ідіоток коштом страхування до
психоаналітиків "корисно для суспільства", то так воно і продовжуватиметься.
Ліберастія взагалі схильна казати, що психічно хворих не існує.

\iusr{Лілія Григорівна}
\textbf{Костянтин Собіченко} думаю ЗМІ та еліта мають докласти зусиль. Чомусь у Львові мова звучить лагідно і мило. Імена,нам ,,обрусілим,‘‘звучать дивно, але гарно та природно.
Про що говорити, в Києві вчителька з дітками спілкується чужою мовою з підкреслено московським акцентом. А ім‘я Ярина з вуст вчительки схоже на образу @igg{fbicon.shrug} 
Може перебільшую.

\iusr{Лілія Григорівна}
\textbf{Костянтин Собіченко} треба навчитись поважати себе.

\iusr{Олекс Музичко}

вони вам зараз напишуть, що ім'я Ганна то тільки для Ганн, а уся інші мають
бути Аннами, і при тому неодмінно Ярославнами. Вона їм сама про те казала під
час спіритично-порнографічного сеансу.


\iusr{Олекс Музичко}
\textbf{Орелі Боян} Точно і тільки завдяки таким як ви, що вигадують виправдання малоросам

\iusr{Руслана Курах}
\textbf{Музичко Олександр} як нечемно

\iusr{Руслана Курах}
і самозабанився. ех

\iusr{Светлана Антонюк}
\textbf{Музичко Олександр} вся полемика - бредовый бред, мои дорогие УКРА1НЦ1! Займитесь образованием!

\iusr{Українською Будь Ласка}
\textbf{Светлана Антонюк} а тепер ще раз людською мовою

\iusr{Леся Маланюк}
\textbf{Свєтлана Антонюк}, а ви не українка?

\iusr{Светлана Антонюк}
\textbf{Леся Маланюк} украинка, поэтому стыдно за таких лесь...

\iusr{Светлана Антонюк}
Будь Ласка чистейший рассизм

\iusr{Леся Маланюк}
Украінка? Гібрид ви, манкурт, запроданка, але тільки не українка.
Якби ви були українкою, то називались би Світланою і писали б українською мовою. Служіть і далі пуйлу, він вам за це подякує. А ще ліпше — їдьте до нього на московію, там вам «стидно за такіх лесь» не буде.

\iusr{Українською Будь Ласка}
\textbf{Светлана Антонюк} ти навіть московітської не знаєш, співчуваю. «рассизм»))

\iusr{Светлана Антонюк}
\textbf{Леся Маланюк} ну права я - дура дурой ЦЯ леся

\iusr{Леся Маланюк}
Сама написала — сама себе й похвалила.))

\iusr{Светлана Антонюк}
\textbf{Леся Маланюк}  @igg{fbicon.bottle.popping.cork}{repeat=4} мабуть

\iusr{Леся Маланюк}
Свєта, пий бояришнік — це твоє.))

\iusr{Олекс Музичко}
\textbf{Леся Маланюк} ось через таких як ця манкуртіха в нас і громадянська війна по всій Україні, а на Сході ще й агресія Росії.

\iusr{Ігор Кушнірчук}
\textbf{Светлана Антонюк}, ви не українка. Українці серед українців послуговуються українською. Це по-перше. А по-друге - ви не знаєте, що таке раса, бо до чого тут расизм, московитко?

\iusr{Ivan Prylutskyj}
\textbf{Светлана Антонюк} Ні, не так: тожеукрАінка! ))

\end{itemize} % }

\iusr{Олег Щербан}
 @igg{fbicon.100.percent} 

\iusr{Марія Скрипух}
Погоджуюсь з Вами. Побільше би українців мали таку позицію. Може, колись і малороси отямляться, дай, Бог.

\iusr{Оксана Петришина}

\ifcmt
  ig https://scontent-frx5-2.xx.fbcdn.net/v/t39.1997-6/s480x480/14050144_1775288802711824_1454378351_n.png?_nc_cat=1&ccb=1-5&_nc_sid=0572db&_nc_ohc=B0tC8L092akAX9JY6Oo&_nc_ht=scontent-frx5-2.xx&oh=d5658bc0762105e5a76e140d49ae48c3&oe=61697955
  @width 0.4
\fi

\iusr{Оксана Ковалишин}
Тут і "дискутувати" нема про що.

\iusr{Lubomyr Koval}
 @igg{fbicon.hands.applause.yellow}{repeat=3} 

\iusr{Марта Данилюк}

\ifcmt
  ig https://scontent-frt3-1.xx.fbcdn.net/v/t1.6435-9/72750524_1217005315167368_4422138777963593728_n.jpg?_nc_cat=102&ccb=1-5&_nc_sid=dbeb18&_nc_ohc=b12FDyvtn6kAX_Hva4K&_nc_ht=scontent-frt3-1.xx&oh=355276846101b08740ace3e21831c6b8&oe=618CA063
  @width 0.7
\fi

\iusr{Руслана Курах}
Хочеться до всього додати пости й репости приколів російською... Це якийсь дисонанс..

\begin{itemize} % {
\iusr{Gulya Vanya}
\textbf{Руслана Курах} це особливий мазохізм...

\iusr{Руслана Курах}
\textbf{Іван Гуля} ментально то їм близьке, виходить... бо у здорового українця є відторгнення.. і що прикро, - я таке часто спостерігаю у цілком патріотичних україномовних людей. сюр якийсь
\end{itemize} % }

\iusr{Руслана Курах}
або повдягаються у вишиванки, а підписані на юсупавих, і книги постять "Мая пєрвая любофь" і всякий російський мотлох

\iusr{Роман Волошинюк}
І я впевнений, що ці "російськоязичниє тоже украінци" 100\% голосували за українофоба Зе...

\iusr{Жорж Рибка}
Дурний тебе піп хрестив...

\begin{itemize} % {
\iusr{Леся Маланюк}
Кого?

\iusr{Костянтин Собіченко}
Мені також про попа цікаво стало. )

\iusr{Оксана Захарчин}
\textbf{Жорж Рибка} ну, Жоржем точно дурний піп охрестив!

\iusr{Руслана Курах}
\textbf{Оксана Захарчин}  @igg{fbicon.beaming.face.smiling.eyes} 

\iusr{Ольга Жвава}
\textbf{Костянтин Собіченко} раніше імена давали попи. Згадайте Панаса Мирного твір "Хіба ревуть воли..?". Головний герой Нечипір. Так піп назвав, бо з походженням у дитини, на думку попа, було не дуже добре. Отже, дурний піп хрестив;)))

\iusr{Ольга Жвава}
\textbf{Оксана Захарчин} а ще Нечуй-Левицький: " Микола Потім знов побачив Нимидору (так звали дівчину всі, хоч насправді піп дав їй ім'я Минодора, бо був злий на її батька)....
\end{itemize} % }

\iusr{Ольга Жвава}
Нікіта та Крістіна;))

\begin{itemize} % {
\iusr{Юрій Іщук}
\textbf{Ольга Жвава} то називайте Микита та Христя...

\iusr{Ольга Жвава}
\textbf{Юрій Іщук} брикаються;)) , не подобається - у паспартє так напісана;)

\iusr{Владимир Маковеенко}
\textbf{Ольга Жвава} Микита та Христина))

\iusr{Костянтин Собіченко}
Владимир Маковеенко, і Володимир Маковеєнко.)
\end{itemize} % }

\iusr{Микола Владзімірський}

\url{https://www.facebook.com/ZaMovuUa/photos/a.1754657478088448/2241812339372957/?type=3&theater}

\ifcmt
  ig https://scontent-frx5-1.xx.fbcdn.net/v/t1.6435-9/40257884_2241812342706290_3073506905147047936_n.png?_nc_cat=105&ccb=1-5&_nc_sid=730e14&_nc_ohc=tn6OeHKyEFsAX8JJTSa&_nc_ht=scontent-frx5-1.xx&oh=5f2792f350676b335f6e944f08e20f2a&oe=6189AB2A
  @width 0.4
\fi

Першою ознакою українця у Facebook
є прізвище, написане українською, або його транслітерація
латинськими літерами.
Зроби так!
Якщо Українка чи Українець.
Пошир цю думку!

\iusr{Беляев Алексей}
\textbf{Inna Shchukina} 

Вони зрозуміють цю проблему коли їх дитина поїде воювати и побачать що разом з
нею буде стояти побратим котрий скаже: " Привет я Саня . Я твой второй номер".
"Слава Україні".

\begin{itemize} % {
\iusr{Леся Маланюк}
А ця дитина скаже: «Привіт, Сашку! Ми ж українці, правда? І мова наша — українська. Не бійся говорити рідною, а я тобі допоможу. Шлях до хорошої української — через погану українську. Разом ми і цю прикрість здолаємо».
Героям слава!

\iusr{Ігор Кушнірчук}
\textbf{Леся Маланюк}, війна триватиме доти доки з цього боку є захисники руцкаґа міра. А всі московитомовні і є ними.

\iusr{Михаил Гуйтур}
\textbf{Ігор Кушнірчук}... і без системної в тому числі і насамперед на державному рівні боротьби з російськомовністю на протязі мінімум трьох поколінь...

\iusr{Костянтин Собіченко}
\textbf{Михаил Гуйтур}, ви вже побороли, Михайлом стали? Чи нє звучіт?)

\iusr{Михаил Гуйтур}
\textbf{Костянтин Собіченко} ...так став на іншому акаунті... а боротьба продовжується за дорогу до правди.. і світла... як і раніше...

\iusr{Дмитро Горда}
\textbf{Леся Маланюк} ну, "привіт" скаже вже не зовсім українець - добряче засимільований українець.

\iusr{Леся Маланюк}
\textbf{Дмитро Ґорда-Горицвіт} А чим же «привіт» погане? Що за обмеження?

\iusr{Костянтин Собіченко}
\textbf{Михаил Гуйтур}, я не бачу іншого, бачу лише москвина Міхаіла. ) А світла тут якось і не багато зовсім.

\iusr{Дмитро Горда}
\textbf{Леся Маланюк} 

я не писав, що погане. НІкого не обмежував. Лише висловив жаль, що жага
повернути рідне не вмерла, але мова вже неукраїнська: велика сила лексики,
сталих виразів за роки незалежности з одного боку калькувалася на ходу, з
другого - перекручувалася. Якби, опановуючи англійську, ми таким самим робом
механічним перекладали з рідної, англійці ніяковіли б слухаючи.

\end{itemize} % }

\iusr{Оксана Тарасенко}
100\%

\iusr{Александр Рудас}
Лесю!!!
Дискусія була про все, навіть про мою власну сторінку!!!)))
Та помилку Ви не виправили і на моє запитання, стосовно дописувачів, що зареєстровані латиницею, так і не відповіли...
Соромно Лесю, змішувати грішне з праведним!)))
Раз є помилка, то її варто виправити, а не поливати брудом опонента....!)))
Щасти Вам!)

\begin{itemize} % {
\iusr{Леся Маланюк}
Яку помилку? Соромно має бути не мені, а вам, бо називаєтесь як москвин, а не українець. Удачі!

\iusr{Александр Рудас}
\textbf{Леся Маланюк}
Лесю!)

\begin{itemize}
  \item - Не я "називаюсь", а батьки назвали!)))
  \item - Яке значення Ви вкладаєте в слово "москвин", та що воно означає!?)
  \item - Чому не виправили помилку в пості, стосовно наголосу в слові "сенсорний"!?
  \item - Чому Ви не поважаєте моєї рідної мови (української), та при особовому зверненні пишете "Ви", з маленької літери!?
  \item - Чому не бачу Вашого "праведного" гніву стосовно тих, хто зареєстрований не українською, а латиницею!?
\end{itemize}

І на останок!

В моїй рідній мові, відсутнє слово "удачі", і воно є російським!!! Тому
нагадую, що в моїй рідній - українській мові, є переклад цього слова!!!!)

І це слово - "щасти"!!!)))

Учіть рідну мову Лесю!

Щасти Вам!)

\iusr{Леся Маланюк}
\textbf{Алєксандр Рудас}, спеціально для вас:
\url{http://sum.in.ua/s/udacha}

\iusr{Леся Маланюк}
\url{https://youtu.be/4geLnLH-oJ4}

\iusr{Леся Маланюк}
Як я можу виправити помилку в чужому дописі?

\iusr{Леся Маланюк}
А в паспорті ви як записаний?

\iusr{Александр Рудас}
\textbf{Леся Маланюк}
Ви уважно прослухали стосовно займенника...!?!?))))
Саме те, що я Вам і казав...!)))
Усіть рідну мову Лесю!)
В нагоді стане....!))))

\iusr{Александр Рудас}
\textbf{Леся Маланюк}
Підтримати звернення до автора!
От саме так і виправляються помилки...)

\iusr{Александр Рудас}
\textbf{Леся Маланюк}
В мене паспорт ще старого зразка - двома мовами!)
Та жодного напису латиницею....!)))

\iusr{Леся Маланюк}
Уважно. «Лише в ділових документах або вітальних листівках».
То що там у паспорті? Українською як написано?

\iusr{Александр Рудас}
\textbf{Леся Маланюк}
Вибачте, що на картинці російською, та знання інших мов лише збагачує справжню людину...
Надзвичайно влучно сказано!
Щасти Вам!
Розмова з Вами - просто марна трата часу!)
Своїх помилок Ви визнавати не бажаєте....

\ifcmt
  ig https://scontent-frx5-1.xx.fbcdn.net/v/t1.6435-9/72488844_3007986542604969_3972669136177201152_n.jpg?_nc_cat=110&ccb=1-5&_nc_sid=dbeb18&_nc_ohc=YoHyaWVdXYAAX9ja7KV&_nc_ht=scontent-frx5-1.xx&oh=c9fcd05e725767716c52c360211cb8a8&oe=618BE4D7
  @width 0.4
\fi

\iusr{Леся Маланюк}
Не вибачаю.

\iusr{Александр Рудас}
\textbf{Леся Маланюк}
Скільки мов знаєте, рівно стільки раз Ви людина....!)

\iusr{Костянтин Собіченко}
\textbf{Александр Рудас}, 

кого тільки не закличе на допомогу малорос, аби лише малоросом і лишитися.) Це
нібито Гьоте... нібито... й нібито його цитата. Неповна, щоправда. ) Щодо тебе
- взагалі жодної великої літери не застосував би.)

Це ж треба: вчити рідну мову закликає той, хто не хоче нею навіть ім'я своє
написати. Мегахохляцтво.

О боги! Ще й картинка про "умнава чєлавєка"...)) Це ти, чи що?!)) Збагачуйся,
московити тебе вже збагатили - скоро нагодують по повній.)

\iusr{Бажан Козаченко}
\url{https://www.facebook.com/groups/2145849382411277/permalink/2343085536020993/}

\iusr{Александр Рудас}
\textbf{Костянтин Собіченко}
То Ви хамло ще той...!!!)))

Лишайтесь собі хоч мало хоч великоросом! А хочте, то і Мегахохляком - на Ваш
власний розсуд!!!

А я українець! І пишаюсь цим! Єдине за що соромно - то лише за те, що в Україні
ще живуть такі чорнороті хамлюки як Ви....!)

Нагадаю Вам ще одну мудрість:

"Якщо Господь хоче обділити людину, то перш за все забирає в неї розум"...

То ж не гнівіть Бога, бо забере те що ще лишилось....

\iusr{Леся Маланюк}
Українець Алєксандр.)))

\iusr{Костянтин Собіченко}
\textbf{Леся Маланюк}, він! )) От де класика, от де буйно малоросійство цвіте! Просто хрестоматійний приклад...) Ти йому в очі плюй - а він казатиме, що дощ іде.)

\iusr{Александр Рудас}
\textbf{Костянтин Собіченко}
Таки образив та обділив Вас Господь....

Чому ж ви разом з Лесею, не шматуєте тих, в кого реєстрація латиницею та не
називаєте їх квітучим малоамериканством!?))))

Чи кишечка тоненька на хазяйнів своїх тявкати...!?)))

Ото ж Бо...

Живіть і далі в своєму світі хамства, блюзнірства та ілюзій....

Щасти вам - малоамерикоси....!)))

\iusr{Костянтин Собіченко}
\textbf{Александр Рудас}, 

так, не дав мені розуму... не дав... Аби ж то зрозуміти мені, що насправді
треба Константіном було підписатися!.. Але ні, немає того... немає, лише
тявкання порожнє ( з Лесею разом), та служба важка на хазяйнів осоружних. )

\iusr{Оксана Захарчин}
\textbf{Александр Рудас} 

знаєте, зацікавилася Вашою дискусією з п. Лесею, і пішла на Вашу сторінку,
хотіла розглянути зблизька Ваше "правильне" обличчя. І От що виявилося.
Виявляється ,Вас справді нарекли російським іменем не просто так: 70\%
матеріалу, що Вам подобається, писаний російською. Не здивуюсь, коли Ви і в
побуті нею послуговуєтеся. І манера дискусії у Вас москальська: зразу претензії
до опонента і якнайшвидше поставити його на коліна!

Знаєте, шановний, кажуть, нема гіршого, коли мідний тазик корчить із себе
кришталеву вазу! Майте сміливість називатися тим, чим Ви є насправді!

\iusr{Ігор Кушнірчук}
\textbf{Александр Рудас}, 

українською "ви" вживають в більшості випадків з маленької літери. Напевно то
був не той особливий випадок, коли потрібно було вживати саме "Ви".

\iusr{Оксана Захарчин}
\textbf{Александр Рудас} а Вам, "щирому" українцю, не соромно ходити з таким паспортом?

\iusr{Ольга Жвава}
\textbf{Костянтин Собіченко} Ґете;)

\iusr{Костянтин Собіченко}
\textbf{Ольга Жвава}, я колись читав різні думки з цього приводу... О, я знаю в кого запитати! Якщо не забуду - відповім. )

\iusr{Александр Рудас}

\obeycr
Шановні форумчани!
Перепрошую, що намагався достукатись до вашого розуму та душі...!
Дякую вам, за ту величезну роботу, яку ви провели в країні.
Зокрема в царині відбудови промисловості та боротьбі з корупцією!)
Завдяки саме вам, всі проблеми країни успішно вирішені і лишилось вирішити лише питання наголосів, та хто і як зареєстрований в соцмережах!)))
Саме ви відкрили мені очі, на той вишуканий рівень хамства та невігластва, який ще панує в нашому суспільстві!(
Вельми вдячний вам за все!
Просто нагадую - в Югославії все також запалало з "мови"...
Пишайтесь - в Україні також не обійшлося без цього та вашої допомоги....
Поки не пізно, схаменіться! Досить сіяти розбрат та поливати брудом!!!
Як дітям та онукам в очі дивитись будете.....!?(
\restorecr

\iusr{Олекса Краснюк}
\textbf{Александр Рудас} все починається з малого. Називати себе українцем - та бути ним - ці дві різні речі.

% -------------------------------------
\ii{fbauth.rudas_aleksandr.vasilkov.ukraina}
% -------------------------------------

\textbf{Олекса Краснюк}

Якщо з огляду заколоту мовного конфлікту - то згоден з Вами і це абсолютно
вірно!

Та плекають душу дві речі:

- Таких як Ви псевдоукраїнців - зовсім мало!

- Справжніх українців, які за мир та спокій в країні, незалежно від векторів
розвитку держави - переважна більшість!

Бо з огляду на новітню історію розвитку Європейського континенту - займатися
такими речами, як мовний розбрат, може або злочинець (псевдопатріот), або
людина без розуму....

Історія - річ циклічна! І щоразу, коли ми забуваємо за попередні події - вона
має здатність повторюватись!

І на останок, стосовно "бути" та "називати"!!!)

Вчіть друже рідну мову!)))

Потрібно писати та вимовляти: -"...це дві різні речі"!!!)))

А не так, як написали Ви: - "...ці дві різні речі"!

Щасти Вам!

Най вічно буде Україна!!!

\iusr{Олекса Краснюк}
\textbf{Александр Рудас} мир та спокій в якій саме Україні ? В Українському Автономному Окрузі роспедерації ? Бо ви якраз і намагаєтесь саме це й зробити .

\iusr{Олекса Краснюк}
\textbf{Александр Рудас} 

ви б краще щось би потрібне зробили б . Прийшли б , наприклад , до військкомату
, організували б собі трирічну відпустку від справ - в поля Донбасу. А потім би
розповідали про мир . А так - ви звичайне сцикло , що мріє подарувати Неньку
хуйлові - ще й прикриваєтесь нами.

\iusr{Ігор Кушнірчук}

Про який розбрат ви тут верзете? Відколи це московити та малароси брати
українцям? Це вони кликали хуйла в Україну, це вони зупиняли українських
вояків, перекриваючи дороги військовим колонам. Це вони хочуть "невійни",
шляхом здачі українських теренів ворогу. Але це називається зрада національних
інтересів і капітуляція а та "невійна", яка наступає після цього називається
окупація та рабство.

\iusr{Александр Рудас}
\textbf{Ігор Кушнірчук}
У Вас шановний серйозний розлад зору, якщо Ви стверджуєте, що я саме про це писав...!((

\iusr{Александр Рудас}
\textbf{Олекса Краснюк}
)))))
Ви вважаєте, що хамство прикрашає справжнього чоловіка....!?))))

\iusr{Костянтин Собіченко}
\textbf{Александр Рудас}, о, то ти ще тут!...) Сподобався допис? Не відпускає тема? Заворушились ізвіліни малоросійські? ) Я радий.)

\iusr{Олекса Краснюк}
\textbf{Александр Рудас} а я не справжній чоловік . Я диванний воєн. Мені можна. А все ж - чому досі не у військкоматі ?

\iusr{Олекса Краснюк}

До речі - хамство , на вашу думку - це називати речі своїми іменами ? Коли у військо ?

\iusr{Ігор Кушнірчук}
\textbf{Александр Рудас}, 

о то ви не тільки руцкій мір в Україні захищаєте, а ще й брешете -
"...займатися такими речами, як мовний розбрат, може або злочинець
(псевдопатріот)"...

Дякую за демонстрацію.

\iusr{Александр Рудас}
\textbf{Костянтин Собіченко}

\obeycr
Чому ж Ви радий...!?(
Може тому, що батьки не займались Вашим вихованням і не навчили, як Ви маєте звертатись до незнайомої людини!?(
Та замість поваги до оточуючих, привчили до зверхності та хамства!?)
Сумнівна радість...
Стосовно Вашого "не відпускає"!!!! Не ототожнюйне Ваш особистий фізіологічний стан, коли Вас "не відпускає" - з адекватними громадянами...!)))
Тепер щодо "ізвілін"!)
Повірте - мене "рух" Ваших малоамерикосівських звивинин - абсолютно не турбуює!)
Ваше завдання підливати масла в вогонь війни, моє ж - відшліфовувати до ідеалу нашу рідну мову!
І я, як і будь-який інший пересічний українець, просто зобов'язаний вказати автору на допущені ним помилки!
Бо наша мова, це не Ваш любий Порошенко на декілька років.... Мова - це на віки!!!!!)
І автор допису, має відповідально ставитись, до поширюваної ним інформації!
А загалом - шкода мені Вас!
Хамло, ще й невіглас....
Щасти Вам!
Най вічно буде Україна!!!!
\restorecr

\iusr{Александр Рудас}
\textbf{Олекса Краснюк}
Тоді маю вибачитись!)
Я не відразу зрозумів, що "Сцикло" то Ваше ім'я і Ви просто представились....!(

\iusr{Александр Рудас}
\textbf{Ігор Кушнірчук}

\obeycr
А вказати автору допису на помилку, це ознаки захисту "російського світу"!?))))
То у Вас геть важко з головою....!((((
Мав надію, що разом ми зробимо все, щоб звільнити нашу рідну мову від слів - кальок з іноземних мов!
Та бачу, що просто розкидав перлини перед...
Прикро!
А ще більш прикро, що Ви так і заклякли на рівні розвитку мушлі....
\restorecr

\iusr{Костянтин Собіченко}
\textbf{Александр Рудас}, 

слухай, опудало ти нікчемне, почитай свій текст і виправи помилки. Чи гадаєш,
що їх там немає?)

"Мова - це навіки". Лише не для тебе, малоросе. Живи, ковтай сало, витирай руки
масні об шаровари (але, щоб не бачив ніхто), і розказуй про "нашу рідну мову",
чужою своє ймення підписуючи. Смердючий хохол - це ти, якщо не знав, кого так
можна назвати.) Вітаю! Жизнь твоя удалась.)

\iusr{Александр Рудас}
\textbf{Костянтин Собіченко}

\obeycr
Дякую, смердючий малоамерикосе!)
Ваше жития, як нікчемного хамла та недоумка - також вдалося!))))
Щасти Вам!
А до лікаря, все ж таки зверніться!
Можливо ще не все втрачено....
\restorecr

\iusr{Александр Рудас}
\textbf{Олекса Краснюк}

\obeycr
Шановний пане Сцикло!
Маю визнати, що попри Ваше рафіноване хамство - Ви мене все ж таки заінтригували!)
Звідки цифра "3 роки"!?)
Це саме стільки, скільки не вистачило Порошенку повністю роздерти Україну!? Чи можливо це термін побудови теміналів для прийому сланцевого скрапленого газу зі Сполучених Штатів!?)
Напевно це ті довгострокові програми Вашого заробітку на "кровавих бронежилетах" та на списаній ще в радянські часи бронетехніці, що Ви та Ваші колеги, продаєте на передову, як нову....!?
Така собі "оговорочка по Фрейду"....
От на таких дрібничках і "паляться" малоамерикоси та українські америко-патріоти, а по простому - українські псевдопатріоти....
Вибачте! Та надалі бридко підтримувати з Вами діалог!
Щасти Вам!
Най вічно буде Україна!
\restorecr

\iusr{Ігор Кушнірчук}
\textbf{Александр Рудас}, 

непане маніпуляторе, моя особа тут ні до чого. Тут ваша брехня розглядається. Я
вам привів цитату вашого допису, де ви спочатку писали одне, а потім инше. А ви
переходите на особистість, що свідчить про відсутність аргументів за темою
обговорення.

\iusr{Александр Рудас}
\textbf{Ігор Кушнірчук}

\obeycr
Дякую!
За національністю я українець!)
Тому - "непан", а просто - "добродій"!
Тому, враховуючи, що я не поляк, то звертайтесь до мене, як до українця!!! Та пишіть мені саме українською!!!)
А саме! В українській мові, є літера " і"!!! Тому застосовувати в слові "інше", прошу саме нашу, українську літеру " і", а не її російський аналог - "и"!!!
Наступне!)))
Яку саме мою "брехню" Ви розглядаєте!?) Саме те, що я вказав на брехню автора допису, що слово "сенсорний", в українській мові має цілих два наголоси, а не один, як вказав автор!!!!???)))
То якщо Ви цього не знаєте, то це не моя "брехня", а звичайне Ваше невігластво!!!))))
Вивчайте мову, невіглас!)
\restorecr

\iusr{Костянтин Собіченко}
\textbf{Александр Рудас}, а чого ти залип тут так надовго, малоросику? Ніби ж то не твоя тєма.. Чи твоя?)
То ти ще й "за національністю українець"?)
Гомеричний сміх.) А чого ж ти, нікчемо, не наважишся на справжній український крок?)

\iusr{Александр Рудас}
\textbf{Костянтин Собіченко}

\obeycr
Ви все намагаєтесь здивувати мене своєю нікчемністю та хамтвом!?) То дарма...
Не переймайтесь, на жаль - Ви вже безнадійний...! Використовуючи Ваші ж слова - Ваш розум, на жаль, "залип" на рівні мушлі.... То ж до Вас у мене жодних запитань, а відтак і відповідей....
Та я маю донести, тим адекватним співвітчизникам, що лишаються в обговоренні, що наша мова має бути чиста та дзвінка, як джерельна вода!
І я не маю права змовчувати, коли в ту воду плюють неуки, хамлюки та звичайні F20, ставлячі невірні наголоси та нав'язуючі широкому загалу своє суб'єктивне бачення!!!
\restorecr

\iusr{Ігор Кушнірчук}
\textbf{Александр Рудас}, 

ви далеко не "добродій", а радше -злодій. Українською ви також не володієте, бо
знали б, що "инший", то не помилка, а один з варіянтів написання.

Я вказав вам на те, що в одному місці ви стверджували одне, а иншому инше, десь
таки збрехали. Та ви знову перемкнулися на мою скромну особу, хоча до чого тут
я, якщо брехун ви?

\iusr{Леся Маланюк}
\textbf{Алєксандр Рудас}, невірні наголоси? Кому вони невірні?)

\iusr{Костянтин Собіченко}
\textbf{Леся Маланюк}, не заспокоїться ніяк, просвітитель великий, знавець вірності в питаннях наголосів.)
Це до якого рівня єзуїтства треба дожитися, щоби розповідати про "нашу чисту та дзвінку мову", чужою підписуючись? Але нічого - живе людина, все в порядку в неї!)

\iusr{Ігор Кушнірчук}
\textbf{Леся Маланюк}, а де тут про наголоси було згадано до коментаря цього пана? Це просто демагогія - пересмикування.

\iusr{Александр Рудас}

\obeycr
Ідіть з миром...
Дарма я дав себе затягти в демагогію з ідіотами!
Радіє серце лише від однієї думки - що таких як ви, лише троє на сторінці...!)
Ви найкращі в царині ідіотів та недоумків...!)
Щасти вам!)
Можете не відповідати, більше моїх спроб розбудити ваш розум не буде!)
На привеликий жаль, це даремна справа...(
Паскудьте мову мого Кобзаря, і возвеличуйте мову свого Порошенка та Парубія і надалі - воля ваша....
Час - найкращий суддя!!!
Най вічно буде Україна!!!
\restorecr

\iusr{Костянтин Собіченко}
\textbf{Александр Рудас}, 

на прИвеликий жаль, є такі ідіоти, як ти.) Це біда, про яку ще Кобзар (не твій)
писав, давши вам, чмошникам, правильну характеристику як рабам , підножкам і
грязью Москви, вас виродків назвавши. Часи змінилися, але ви, салоїди тупі,
лишилися. І житимете вічно, на превеликий жаль.

Вали до біса! Ти вже й так осточортів усім, жополиз москвинський.

До психіатра, між іншим, запишись - бачу, потреба є нагальна.)

\iusr{Леся Маланюк}
\textbf{Фрейр Бинтмакерь} (\textbf{Ігор Кушнірчук}) 

Про наголоси був допис у групі «Українська мова для всіх» і, на думку цього
пана, там допущено помилку. Чомусь він вирішив, що я можу, але не хочу її
виправити. Виправити помилку в чужому дописі!)))

\iusr{Костянтин Собіченко}
\textbf{Леся Маланюк}, я ж кажу - як мінімум, відділення психіатрії найближчої поліклініки давно вже плаче на ним.)

\end{itemize} % }

\iusr{Vadym Ursul}
Абсолютно.




\end{itemize} % }
