% vim: keymap=russian-jcukenwin
%%beginhead 
 
%%file 17_08_2021.stz.news.ua.mrpl_city.1.dokumentalni_vystavy_istage_2021
%%parent 17_08_2021
 
%%url https://mrpl.city/blogs/view/pro-dokumentalni-vistavi-festivalyu-istage-2021
 
%%author_id demidko_olga.mariupol,news.ua.mrpl_city
%%date 
 
%%tags 
%%title Про документальні вистави фестивалю IStage 2021
 
%%endhead 
 
\subsection{Про документальні вистави фестивалю IStage 2021}
\label{sec:17_08_2021.stz.news.ua.mrpl_city.1.dokumentalni_vystavy_istage_2021}
 
\Purl{https://mrpl.city/blogs/view/pro-dokumentalni-vistavi-festivalyu-istage-2021}
\ifcmt
 author_begin
   author_id demidko_olga.mariupol,news.ua.mrpl_city
 author_end
\fi

IStage 2021 – п'ятиденний фестиваль-лабораторія ідей актуального театру
завершився, а маріупольці й досі перебувають під сильним враженням від
побаченого. Кожна з фестивальних вистав піднімає питання актуальних і
злободенних проблем і є документальною.

\ii{17_08_2021.stz.news.ua.mrpl_city.1.dokumentalni_vystavy_istage_2021.pic.1}

Одним з найбільш унікальних та незвичних спектаклів фестивалю став \emph{\textbf{Незалежний
театральний проєкт}} \emph{\textbf{\enquote{Осінь на Плутоні}}}, який вразив багатьох маріупольців.  Це
погляд на буття та цінності через призму людини, яка капітулювала перед часом і
опинилася в полоні власної пам'яті та власного болю, людини, яка ще не
\enquote{померла}, але й не \enquote{живе}. Робота запрошує глядачів подивитись в очі власним
страхам і здійснити онтологічну експедицію в старість як таку. Режисер і автор
проєкту \emph{\textbf{Сашко Брама}} – львівський незалежний театральний діяч, драматург,
режисер, який займається експериментальним, пошуковим театром на базі
документалістики – розповів, що для нього цей проєкт почався у 2016 році, коли
він волонтером потрапив до будинку престарілих. Автори вистави впродовж року
перебували в тісній взаємодії з мешканцями геріатричного пансіонату і цей
досвід був трансформований в художнє висловлення. Назва є символічною. \enquote{Осінь}
означає певну пору в житті людини, коли треба збирати урожай, коли все в'яне,
тобто це алегорія старості, а Плутон це холодне віддалене небесне тіло, яке
символізує пансіонат, що  залишається без ув аги, адже люди там дуже самотні,
тобто Плутон це метафора холоду та старості. Ця вистава створена на перетині
документального та лялькового театру, перформативних практик та просторового
аудіо-дизайну. Для акторів участь в подібному проєкті – дуже цікавий досвід.
Вони вважають цю виставу потрібною і актуальною.

\ii{17_08_2021.stz.news.ua.mrpl_city.1.dokumentalni_vystavy_istage_2021.pic.2}

Водночас \emph{\textbf{Вікторія Федорів}} координаторка фестивалю IStage 2021
[Фестиваль-лабораторія ідей актуального театру] наголосила, що подібні проєкти
є дуже значущими для суспільства, адже вони допомагають людям переосмислити
важливі проблеми сучасності.

Багато й інших вистав в рамках фестивалю сподобалися маріупольцям. Серед них і
\enquote{Однокласники} театру \enquote{Віримо}, спектакль  \enquote{Клас} Київського академічного
театру драми і комедії на Лівому березі Дніпра, \enquote{Страждання на Гончарівці}
театру \enquote{Нефть}, вистава \enquote{Нові шрами} Дикого театру тощо.

\ii{17_08_2021.stz.news.ua.mrpl_city.1.dokumentalni_vystavy_istage_2021.pic.3}

Одне з найбільших обговорень викликав спектакль \enquote{Клас}\par\noindent Київського академічного
театру драми і комедії на Лівому березі Дніпра складається з документальних
текстів і з особистих розповідей акторів, які задіяні у спектаклі. У виставі
піднімаються теми булінгу, шкільних проблем, війни, ставлення до старшого
покоління і політичної ситуації в країні. Авторами є режисер Стас Жирков,
драматург Павло Ар'є та актори вистави. Режисер вистави \emph{\textbf{Стас Жирков}} зазначив,
що 
\begin{quote}
\em\enquote{це не тільки про булінг, це взагалі про те, як людина починає своє життя в
школі і як закінчує в будинку для престарілих скажімо так. Це такий театр, який
робиться тут і зараз під час репетицій}.
\end{quote}

\ii{17_08_2021.stz.news.ua.mrpl_city.1.dokumentalni_vystavy_istage_2021.pic.4}

Ще в  програмі фестивалю маріупольці побачили одне вже відоме їм  ім'я
драматурга, за п'єсою якого в маріупольському драматичному театрі з 2017 року
йшла вистава \enquote{Слава героям}. Павло Ар'є, чиї твори вже стали культовими,
приїхав до Маріуполя вперше. Він є одним з авторів спектаклю \enquote{Клас} і вважає,
що наразі серед всіх українських міст Маріуполь має найбільше значення.
Драматург наголосив, що він 

\begin{quote}
\em\enquote{готовий вірити і сприймати ті речі, що говорить
Маріуполь, бо це важливо. Це набагато важливіше, ніж те, що зараз скаже Львів,
або Івано-Франківськ}.
\end{quote}

\ii{17_08_2021.stz.news.ua.mrpl_city.1.dokumentalni_vystavy_istage_2021.pic.5}

\ii{17_08_2021.stz.news.ua.mrpl_city.1.dokumentalni_vystavy_istage_2021.pic.6}

Цікаво, що актори зробили виставу ближчою до Маріуполя завдяки неочікуваним
імпровізаціям. Вони згадували місто в своїх діалогах і часто зверталися до
глядачів. Насправді для артистів – ця вистава ніби сповідь, адже вони
розповідають власні історії. Глядачі були у захваті як від ідеї, так і від
самої гри акторів. Такого театрального дійства маріупольці ще не бачили.

\ii{17_08_2021.stz.news.ua.mrpl_city.1.dokumentalni_vystavy_istage_2021.pic.7}

А вистава \enquote{Нові шрами} \enquote{Дикого театру} проводиться за підтримки ООН Жінки в
Україні. \emph{\textbf{Наталія Сиваненко}} режисерка Дикого театру розповіла, що її вистава
складається з абсолютно реальних історій. Режисерка їздила по притулкам, що
надають допомогу жінкам, які потерпають від домашнього насилля і записувала
їхні історії. Ця вистава є потужним соціальним проєктом.

Загалом IStage 2021 Фестиваль-лабораторія ідей актуального театру став важливою
мистецькою подією для маріупольців та познайомив глядачів з новими театральними
формами і змусив замислитися над актуальними проблемами нашого повсякденного
життя.
