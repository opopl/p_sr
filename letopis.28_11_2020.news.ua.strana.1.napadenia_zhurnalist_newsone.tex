% vim: keymap=russian-jcukenwin
%%beginhead 
 
%%file 28_11_2020.news.ua.strana.1.napadenia_zhurnalist_newsone
%%parent 28_11_2020
 
%%url https://kiev.strana.ua/303567-neizvestnyj-napal-na-zhurnalistku-newsone-na-tseremonii-pamjati-zhertv-holodomora-v-kieve-video.html?fbclid=IwAR393UWTZ3kJhXgyEKjACc2pVpYIoWcK7rhgDbFPSZlXSWJuxiST2rxzD7Y
 
%%author 
%%author_id 
%%author_url 
 
%%tags 
%%title На журналистку NEWSONE напали на церемонии памяти жертв Голодомора в Киеве. Видео
 
%%endhead 
 
\subsection{На журналистку NEWSONE напали на церемонии памяти жертв Голодомора в Киеве. Видео}
\label{sec:28_11_2020.news.ua.strana.1.napadenia_zhurnalist_newsone}
\Purl{https://kiev.strana.ua/303567-neizvestnyj-napal-na-zhurnalistku-newsone-na-tseremonii-pamjati-zhertv-holodomora-v-kieve-video.html?fbclid=IwAR393UWTZ3kJhXgyEKjACc2pVpYIoWcK7rhgDbFPSZlXSWJuxiST2rxzD7Y}

\ifcmt
pic https://kiev.strana.ua/img/article/3035/67_main-v1606575989.jpeg
caption Неизвестный сломал журналистке микрофон. Скриншот из видео  
\fi

\index[names.rus]{Товкес, Виолетта!Журналистка NEWSONE, нападение}

Во время освещения событий на церемонии чествования памяти жертв Голодомора в
Киеве неизвестный напал на журналистку телеканала NEWSONE Виолетту Товкес.

Видео инцидента опубликовано на YouTube-канале вещателя.

Так, корреспондентка телеканала NEWSONE рассказывала о мероприятиях по случаю
Дня памяти жертв Голодомора в прямом эфире. В 16:03 к ней подошел парень в
маске и выбил микрофон и поминальную свечу, после чего начал толкать
журналистку. Свои действия мужчина сопроводил яростными криками. Спустя
некоторое время нападавший ушел.

\ifcmt
pic https://kiev.strana.ua/img/forall/u/11/52/157357_(1).png
\fi

После происшествия Товкес показала разбитый микрофон и заявила, что намерена
обратиться к правоохранителям по факту нападения на нее.\Furl{https://youtu.be/zcgPOhSY5Zw}

В свою очередь вещатель обнародовал заявление посвященное сегодняшнему
нападению.

\zqq{Телеканал NEWSONE требует от правоохранительных органов возбудить уголовное
производство по факту нападения на журналистку телеканала, который произошел во
время прямого эфира 28.11.2020 и привлечь виновного к ответственности (ст.
345-1 Уголовного кодекса Украины). Телеканал также берет на контроль
расследование этого факта насилия. Будем благодарны за предоставление сведений
и информации о личности нападавшего. Анонимность и вознаграждение гарантируем},
- заявила пресс-служба телеканала.

Напомним, по данным Национального союза журналистов Украины представителей этой
профессии в нашей стране избивают каждые 5 дней.

Ранее мы писали, что недавно Нацполиция открыла горячую линию для сообщений о
нападениях на журналистов.
