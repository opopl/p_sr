% vim: keymap=russian-jcukenwin
%%beginhead 
 
%%file 07_01_2022.fb.fb_group.story_kiev_ua.2.olga_aleksandrovna.cmt
%%parent 07_01_2022.fb.fb_group.story_kiev_ua.2.olga_aleksandrovna
 
%%url 
 
%%author_id 
%%date 
 
%%tags 
%%title 
 
%%endhead 
\zzSecCmt

\begin{itemize} % {
\iusr{Татьяна Оржеховская}
Благодарю. С Рождеством Христовым. Всех благ

\ifcmt
  ig@ name=scr.woman.hat
  @width 0.2
\fi

\iusr{Андрій Чмир}
Тут с цифрами что-то: 22 года после 15 ти лет брака?

\iusr{Елена Сачивко}
Спасибо большое! Супер история

\iusr{Людмила Каштан}
Спасибо. Очень интересно. Узнала много нового. Кстати, моему деду в госпитале вручала Георгия Великая Княгиня Ольга.

\iusr{Людмила Драб}
В моей библиотеке 2 книги из этой серии с Вашими автографами.
Во всем,что выходит из под Вашего пера есть одна общая черта-любовь к Городу и его истории.Творческих Вам успехов!С Рождеством Христовым!

\iusr{Наталія Кудря-Маршал}
Виктор Геннадиевич, Ваши истории - всегда что-то особенное!

\iusr{тетяна Галькевич}
Спасибо

\iusr{Людмила Гнынюк}
Благодарю!
Очень интересно!
Светлая память...
С Рождеством Вас!!!

\ifcmt
  ig https://scontent-frt3-1.xx.fbcdn.net/v/t39.30808-6/271490242_1521097318266849_9135980586626656357_n.jpg?_nc_cat=104&ccb=1-5&_nc_sid=dbeb18&_nc_ohc=RT6Fx8LMxNoAX_vmdix&_nc_ht=scontent-frt3-1.xx&oh=00_AT-HvUv-mNfZvtiSouyQhS5Yl5oxURN4WD_zSP56cqAsAg&oe=61E49534
  @width 0.2
\fi

\iusr{Олена Шелест}
Благодарю. Замечательный рассказ.

\iusr{Людмила Руденко}
Спасибо Вам огромное.

\iusr{Наталия Марченкова}
Огромное спасибо

\iusr{Надежда Владимир Федько}
Чудова історія!
Я не монархіст, але ніколи не пробачу Імператору відречення від престолу і віддання чудової цивілізації на поругання інородцям-більшовикам!

\iusr{Татьяна Шиверская}
Спасибо

\iusr{Татьяна Шиверская}

\ifcmt
  ig https://scontent-frx5-2.xx.fbcdn.net/v/t39.1997-6/s168x128/11891339_897114570361979_1916032859_n.png?_nc_cat=1&ccb=1-5&_nc_sid=ac3552&_nc_ohc=TYitxsH59NMAX-1sMQo&_nc_ht=scontent-frx5-2.xx&oh=00_AT8jvpSpTaf2tAiGycjUtyrWekA9VFTlGy8PVvY8cZz0ZA&oe=61E48A72
  @width 0.1
\fi

\iusr{Николай Шевчук}
Читаючи цю історію забув про телевізор.

\iusr{Анна Сидоренко}
Спасибо Вам большое, читать было интересно.

\iusr{Tamara Pankratova}
Дякую за цікаву розповідь! А де можна придбати Вашу книгу!?

\ifcmt
  ig https://scontent-frx5-2.xx.fbcdn.net/v/t39.1997-6/s168x128/93027172_222645632401274_7176243611145601024_n.png?_nc_cat=1&ccb=1-5&_nc_sid=ac3552&_nc_ohc=PdAMOeUhJZQAX_bz8kX&_nc_ht=scontent-frx5-2.xx&oh=00_AT9eaW9TXAIwPgnxMqrLuERToo5ba6XduuyHQB0ZkO3D3g&oe=61E3CD30
  @width 0.1
\fi

\iusr{Elena Samoylenko}

\enquote{Что ни делает дурак, все он делает не так}. Сначала обустроить липовый брак, а
потом героически бороться с его последствиями. Героически помогать раненым,
служа общественному благу, но мыться ходить в шикарный приватный вагон
родственника, потому что у него то как раз вода была, в отличие от остального
общества. Любить Киев и опять таки служить общественному благу, но выбираться
из города ночью тайными тропами, боясь возмездия того самого общества, которому
ты так верно служила. И я даже не говорю про эту конкретную в.к., я обо всей
этой семейке.

\begin{itemize} % {
\iusr{Татьяна Бригинец}
\textbf{Elena Samoylenko} она ходила бы немытой, если бы знала, что ранит ваше чувство классовой гармонии. \enquote{Сама такая}

\begin{itemize} % {
\iusr{Elena Samoylenko}
\textbf{Tetiana Brihinets} 

да в том то и дело, что ничем бы она не поступилась ни ради меня ни ради вас. В
жизни бы она не ходила грязная и не отказалась ни от частички комфорта. Не было
бы того вагона с горячей водой, ей бы ее верные служанки из Днепра воду бы
носили. Меня как раз и удивляет, как им удавалось жить совершенно отдельно от
страны, которой они управляли. Прекраснодушные Романовы сначала годами
создавали весь тот ад, а потом очень удивились, оказавшись в его середине.


\iusr{Татьяна Бригинец}
\textbf{Elena Samoylenko} как вы лихо обобщаете - \enquote{Романовы}. Пепел Клааса стучит?

\iusr{Elena Samoylenko}
\textbf{Tetiana Brihinets} 

почему обобщаю? История рассказанная в топике для Романовых не уникальна. В.к.
Елизавета Федоровна и в.к. Сергей Александрович например. Выйти замуж за гея,
чтобы потом стать великомученницей. Это так по-романовски. А основательница
киевского покровского монастыря кстати тоже Ольденбургская Александра Петровна,
если не ошибаюсь. Поженились с великим князем. Он завел любовницу, заимел от
нее пятерых детей. Она жила со своим духовником, в чем он ее впоследствии
открыто и обвинил. Вы себе представляете масштаб этого трындеца в царственном
благочестивом семействе?

\end{itemize} % }

\iusr{Виктор Киркевич}
\textbf{Elena Samoylenko} 

Озлобление по поводу хождения в.к. вагон к родственнику, свидетельствует о
незнании истории и предвзятого отношения, навязанного \enquote{советским просвещением}!
А насчет \enquote{горячей воды}, то значительное большинство киевлян ощущают её
отсутствие из-за недобросовестной работы коммунальных слуб... Или \enquote{семейка}
виновата?

\begin{itemize} % {
\iusr{Elena Samoylenko}
\textbf{Виктор Киркевич} для автора такого отличного очерка вы слишком линейно мыслите. Жаль.

\iusr{Виктор Киркевич}
\textbf{Elena Samoylenko} 

Я автор 37 книг о Киеве, где все построено на реальности. Все мои творения
проникнуты любовью к Киеву, Украине и защите справедливоти. Я ищу в своих
персонажах светлое, а вы - пороки и омерзение. Мне \enquote{жаль}...

\iusr{Elena Samoylenko}
\textbf{Виктор Киркевич} 

вы считаете что \enquote{искать в персонажах светлое} - это великое
достоинство? Вы прожили много лет и до сих пор не знаете, что в реальности люди
состоят из светлого и темного? В таком случае мне жаль вас разочаровывать, но
ваши творения не построены на реальности.


\iusr{Нина Дунь-Иванова}
\textbf{Elena Samoylenko} 

Каждый видит реальность по-своему. Есть оптимисты. есть пессимисты. Для одного
на поверхности светлое, а у других - темное. К чему спорить?!!

\iusr{Elena Samoylenko}
\textbf{Нина Дунь-Иванова} 

я понимаю, о чем вы говорите. Кто-то видит везде розовых пони, а кто-то
предупреждает этих вот смотрящих в небо об открытом люке под ногами. Я бы
назвала этих людей не оптимистами и пессимистами а инфантилами и реалистами.
Потому что открытый люк под ногами останется таким даже если вы будете очень
оптимистично смотреть на жизнь. Но инфантилы с розовыми пони нужны обществу,
кто ж спорит. Как раз о них эта статья. Эти прекрасные люди вынуждены были
выбираться из города, \enquote{который они так любили} ночью едва ли не в одном нижнем
белье. Отличная цена за \enquote{оптимизм}. И это им еще повезло. Других еще более
оптимистичных \enquote{оптимистов}, которые остались в \enquote{любимом городе}, чуть позже
расстреляли в подвалах Октябрьского дворца, а тела вывезли в Быковню в машинах
с надписью \enquote{Мясо}.

\iusr{Виктор Киркевич}
\textbf{Elena Samoylenko} 

Я пишу и в этом моё \enquote{великое достоинство}. О \enquote{моих творениях} не судите, так
как с ними не знакомы. Наш спор начался о \enquote{светлой личности} Ольге, но вы и там
нашли \enquote{гадость}! Я бы мог рассказать много хорошего о прокуроре Руденко, с
которым дружили мои родители и он часто бывал у нас дома... Но не хочу с вами,
пишущей всякий вздор, общаться!

\iusr{Elena Samoylenko}
\textbf{Виктор Киркевич} 

вы знаете, я как-то даже не сомневаюсь, что вы бы могли \enquote{рассказать
много хорошего} о прокуроре Руденко, редкостном упыре. Вы - дитя советской
номенклатуры, это я уже поняла. Извините, я не буду больше общаться с
человеком, называющим собственные книги \enquote{творениями}.
@igg{fbicon.laugh.rolling.floor}{repeat=3} 


\iusr{Виктор Киркевич}
\textbf{Elena Samoylenko} \enquote{творения} - это взято с вашего послания, как всегда мерзкого!

\iusr{Нина Дунь-Иванова}
\textbf{Elena Samoylenko} 

Я поставлю ограждение возле открытого люка и пойду выгуливать розового пони!
Нельзя жить в постоянном негативе, ненависти к чужой горячей ванне. То, что
достижение для одного может вызывает усмешку у другого. Никто не ходил в чужой
обуви... Не нам судить, не нам осуждать.


\iusr{Elena Samoylenko}
\textbf{Нина Дунь-Иванова} ну я и говорю - всеобщее добро, розовые пони, вот это все. Успехов вам.
\end{itemize} % }

\iusr{Maryna Chemerys}
\textbf{Elena Samoylenko} Людина звикла підтримувати особисту гігієну - що в цьому поганого? Чи комусь було б легше, якби вона не милась?

\iusr{Elena Samoylenko}
\textbf{Maryna Chemerys} я зовсім не про гігієну говорю. Я говорю, що ніколи вони не були разом з тими, яким нібито допомагали. Ніколи вони не жили в тому світі, яким керували.

\iusr{Наталья Толстова}
\textbf{Elena Samoylenko} . Певно, Ви легко переносите відключення води. Дійсно, які буржуазні пережитки кожного дня вмиватись, а може ще й душ приймати.
\enquote{Только дураки}... Ось і постріляли майже всіх.

\iusr{Elena Samoylenko}
\textbf{Наталья Толстова} постріляли їх не за підтримання особистої гігієни, хочу вас запевнити.

\iusr{Ольга Каверзнева}
\textbf{Elena Samoylenko}, злоба і заздрість - не кращі супутники життя

\iusr{Elena Samoylenko}
\textbf{Ольга Каверзнева} ви психолог? Чи просто хочете про це поговорити?


\end{itemize} % }

\iusr{Людмила Старовойтенко}
Спасибо за рассказ и фото интересно

\iusr{Мария Бутковская}
Спасибо большое. Очень интересно ! Успехов вам.

\iusr{Татьяна Кец}
Спасибо большое! Очень интересно! Люблю Ваши повествования.

\iusr{Нина Гордийчук}
Спасибо за интересный рассказ.

\iusr{Валентина Ефимова}
Спасибо за рассказ.  @igg{fbicon.hands.applause.yellow} @igg{fbicon.heart.red}

\iusr{Лариса Олейникова}

Спасибо большое за столь интересный рассказ из жизни императорской семьи. Как
многого мы не знаем, но, благодаря Вам, прикасаемся, познаём.


\iusr{Ольга Dzhun}
Спасибо

\iusr{Vadym Makhnytskyy}

Венчание Великой Княгини Ольги Александровныи и ротмистра Ахтырского 12-го
гусарского полка Николая Александровича Куликовского состоялось 17 (4) ноября
1916 года в Трёхсвятительской ( Васильевской) церкви в Киеве, построенной на
Ярославовом «Большом дворе» князем Святославом Всеволодовичем в 1183 году и
находившейся на улице Десятинная 8.

\ifcmt
  ig https://scontent-frx5-1.xx.fbcdn.net/v/t39.30808-6/271605733_4676586542454204_296649351481489757_n.jpg?_nc_cat=105&ccb=1-5&_nc_sid=dbeb18&_nc_ohc=8fc__911d4oAX8diHiz&_nc_ht=scontent-frx5-1.xx&oh=00_AT9uu2Mz-48k1WzEzVpbk8Zq9HrSvYk9ma8w6KdpI2IdPQ&oe=61E40997
  @width 0.2
\fi

\begin{itemize} % {
\iusr{Виктор Киркевич}
Спасибо! В воспоминаниях в.к. Александра Михайййловича пишется, что \enquote{они ехали долго}... От Дворца.

\iusr{Vadym Makhnytskyy}
\textbf{Виктор Киркевич} Великий князь Александр Михайлович в Киеве жил в собственном поезде, стоявшем на железнодорожных путях недалеко от Киевского вокзала, поэтому ему дорога и показалась долгой.
\end{itemize} % }

\iusr{Тамара Нарижная}
какая интересная и трагическая судьба целой семьи

\iusr{Лариса Петрова}
Спасибо большое, очень интересно!

\iusr{Елена Гладкая}
Спасибо!

\end{itemize} % }
