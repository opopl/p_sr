% vim: keymap=russian-jcukenwin
%%beginhead 
 
%%file 20_12_2021.fb.lirnyk_sashko.1.scenarij_vsramsja_ta_ne_zdamsja
%%parent 20_12_2021
 
%%url https://www.facebook.com/permalink.php?story_fbid=6726378700736417&id=100000930615293
 
%%author_id lirnyk_sashko
%%date 
 
%%tags 
%%title 1. СЦЕНАРІЙ СЕРІАЛУ  КОРОТКОМЕТРАЖНИХ  ФІЛЬМІВ  «ВСРАМСЯ, ТА НЕ ЗДАМСЯ!»
 
%%endhead 
 
\subsection{1. СЦЕНАРІЙ СЕРІАЛУ  КОРОТКОМЕТРАЖНИХ  ФІЛЬМІВ  «ВСРАМСЯ, ТА НЕ ЗДАМСЯ!»}
\label{sec:20_12_2021.fb.lirnyk_sashko.1.scenarij_vsramsja_ta_ne_zdamsja}
 
\Purl{https://www.facebook.com/permalink.php?story_fbid=6726378700736417&id=100000930615293}
\ifcmt
 author_begin
   author_id lirnyk_sashko
 author_end
\fi

\obeycr
1. СЦЕНАРІЙ СЕРІАЛУ  КОРОТКОМЕТРАЖНИХ  ФІЛЬМІВ  «ВСРАМСЯ, ТА НЕ ЗДАМСЯ!»(робоча назва)
Фільм 1.
«СІНЄВА».( на реальних подіях)
Кабіна машини (бусік). Вид через лобове скло. 
Грає музика. Якась російська попса.
В кабіну залазить водій в бронежилеті, рукавичках, чорній шапочці і з тактичними окулярами на лобі. На броніку жовті стрічки скотчу ЗСУ. Водій віддихується, скидає рукавички , витирає руки ганчіркою. Відкидається на сидінні. Дістає цигарку і припалює.
Нахиляється і вмикає радіо голосніше. Музика грає і закінчується.
РАДІО. ГОЛОС ВЕДУЧОЇ
В ефіре радио «Комета ФМ». 
Температура в Донецке около ноля. 
Слабый снег. В Горловке туман. Без осадков.…»
В кабіну з іншої сторони залазить «Кузьміч» - боєць в бронежилеті, пікселі і касці. Теж із жовтими стрічками.
Він плюхається на сидіння і знімає каску. Витирає мокрого лоба.
КУЗЬМИЧ
Фффу!
ВОДІЙ
Ну шо там?
КУЗЬМІЧ
Закінчили. Ледве впихнули всіх. Зараз Сашка двері прикрутить, і  можна їхати.
ГОЛОС ДИКТОРА (на тлі легкої музики)
Рабочий полдень. Как, только что, стало известно, бойцы батальона «Восток» провели успешную операцию против укропов. Мы поздравляем наших защитников с очередной победой. Продолжаем нашу программу. Телефон прямого эфира…  »
Боєць витягає мобільний телефон і набирає.
ВОДІЙ
Кузьмич, ти кому?
КУЗЬМИЧ
А от зараз на радіо подзвоню!
В кабіну залазить «Сашка» -другий боєць в броніку і пікселі. На голові каска. Сашка вмощується біля Кузміча і стягує каску. Кузьміч прикладає до вуха телефон.
ВОДІЙ
Ну шо, Сашка? Всьо?
САШКА(махає рукою)
Ага!
Водій заводить машину і рушає.
Сашка щось шукає під броніком. Намацує і добуває звідти якусь річ. Сашка штурхає Кузьміча і показує йому те,що витяг.
САШКА
Кузьміч! Гля!(співає) Фром сувенір то сувенір ай лів!
КУЗЬМІЧ
Та тихо ти! Я в Донецкьке радио додзвонився!
Сашка закриває вказівним пальцем рота і посміхається, демонструючи увагу. 
Радіо грає легку музику.
КУЗЬМІЧ (в трубку)
В прямой эфир звоню…Э.. Что будем слушать? А я не знаю…Комбат, батяня ,комбат, можна? Ну, батяня комбат, Любэ. Да…
Или «Синева» у вас есть, это…
Кузьміч помічає те,що тицяв йому Сашка. І очі його оживають.
КУЗЬМІЧ
О! Точно! Давайте «Синеву»!  Ага! Хорошо! Добро. Остаюсь.
РАДІО (чоловічий голос)
Рабочий полдень!(звучить музика)
ГОЛОС ВЕДУЧОЇ (з московським акцентом)
Продолжаю выполнять ваши музыкальные заявочки и принимаю телефонный звоночек. Здравствуйте!
КУЗЬМІЧ
А…Здравствуйте!
ВЕДУЧА
Как вас зовут, представтесь!
КУЗЬМІЧ
Меня зовут «Позывной Кузьмич»!
ВЕДУЧА( сміється)
Очень приятно, «Позывной Кузьмич»! 
Действительно приятно, что к нам в эфир дозвонился наверное наш защитник, ополченец!Кому будет ваш привет или поздравление? 
Выбранная вами песня прозвучит сразу после ваших слов!
КУЗЬМІЧ
Я хочу передать привет третьому батальйону двадцять п'ятої окремої повітряно десантної бригади, і  бажаю всім ополченцям піти вслід за Моторолой! Слава Україні! Донецьк будет свободним!
ВСІ В МАШИНІ
Героям слава!
В ефірі починає грати «Расплєскалась синева, расплескалась! По тельняшкам разлилась, по беретам…»
Бійці в машині сміються. Кузьміч роздивляється голубий десантний берет, який йому тицьнув Сашка.
На береті  емблема батальону «Восток»- смугасте георгієвське тло і Георгій на коні із списом.
САШКА
Красава, Кузьміч!
ВОДІЙ
Слава ВДВ! Слава Україні!
Бусік із прапором України на бортах їде, і ззаду видно відкриті задні двері, прикручені дротом ,щоб не теліпалися. А з кузова стирчать скидані на купу руки,ноги в берцях, тєльняшки і голубі берети батальону «Восток».
\restorecr
