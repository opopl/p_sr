% vim: keymap=russian-jcukenwin
%%beginhead 
 
%%file 06_01_2022.stz.news.ua.den.1.hto_my_taki
%%parent 06_01_2022
 
%%url https://day.kyiv.ua/uk/blog/suspilstvo/hto-my-taki
 
%%author_id ljubka_andrii
%%date 
 
%%tags ukraina,identichnost,narod,nacidea,nacia,vopros,obschestvo,strana,kniga
%%title Хто ми такі?
 
%%endhead 
\subsection{Хто ми такі?}
\label{sec:06_01_2022.stz.news.ua.den.1.hto_my_taki}

\Purl{https://day.kyiv.ua/uk/blog/suspilstvo/hto-my-taki}
\ifcmt
 author_begin
   author_id ljubka_andrii
 author_end
\fi

Якщо коротко, то про роман Артема Чеха я хочу сказати всього дві речі. Перша:
премію «Книга року BBC» йому дали цілком заслужено. Друга: цей роман я не хотів
би перечитати ще раз.

Звучить парадоксально, правда? Але так часом є, коли говоримо про книжки на
болючі теми. А роман Артема Чеха – це не тільки класичний роман виховання про
хлопчика, це ще й роман про виростання цілої країни зі смердючої твані 90-х
років. Зрештою, можна так і сказати: це український роман про 90-ті.

А те десятиліття для українців було катастрофічним, і багато наших теперішніх
проблем коріняться саме в реальності дев’яностих. Мене часом дивує, що ми
говоримо про травму бездержавності, комунізму, воєн і Голодомору – як про
основні наші проблеми, що позначилися на ментальності народу і заклали в нашу
психіку руйнівні фрейми поведінки. Але водночас скромно мовчимо про 90-ті, про
це десятиліття приниження людської гідності, деградації до рівня фізичного
виживання, яка болюче позначилася на всіх нас. Так, 90-ті – це та травма, про
яку нам соромно й незручно говорити, ми воліємо про це мовчати, ніби воно
відбулося не з нами. І саме це мовчання є ознакою нашого глибокого
посттравматичного синдрому.

І саме цю стіну мовчання пробиває Артем Чех у своєму новому романі. Хоча,
підозрюю, такої глобальної мети письменник перед собою не ставив – він просто
розповідає історію одного дитинства, токсичних стосунків на фоні загального
розпаду і занепаду, бідності й звиродніння.

А це вже ми, читачі, можемо проектувати й масштабувати цю життєву історію:
Черкаси – на всю Україну, контуженого ветерана афганської війни Фелікса – на
травмоване комунізмом українське суспільство, маленького хлопчика Тимофія – на
паростки нового й незалежного українського суспільства, які, втім, не мають з
кого брати приклад, тому закономірно проходять усі сім кіл життєвого пекла.

Читаючи роман Чеха, я весь час упізнавав себе й власне дитинство, хоча виріс не
в Черкасах, не в промисловому місті, не мав контуженого афганця і дачі в селі,
але загалом – за настроєм, за атмосферою безвиході – це книжка й про мене. А
значить, з огляду на всі ці різниці, і загалом про наше покоління – людей, чиє
дитинство й роки формування припали на 90-ті. Недарма навіть описаний у книзі
перший вияв патріотичних почуттів у нас такий самий – я також співав гімн після
товариського матчу Україна-Росія в 1999 році, також пам’ятаю той гол Шевченка і
до болю чітке розуміння, що найбільшою образою на світі буде програти росіянам
у футбол. Саме росіянам – хоча в ті роки я ще не дуже цікавився політикою.

Впізнавати себе в героєві книжки – це велика похвала для її автора. А коли
роман болить тобі майже на фізичному рівні – це означає, що написаний він
по-справжньому добре. Без наворотів і прикрас, гола правда. І ця правда
терапевтична, бо визнання й проговорення своєї травми допомагає якщо не
позбутися її, то принаймні приборкати хворобливі наслідки.

Тому роман «Хто ти такий?» – справді важливий текст, адже з нього українське
суспільство може розпочати складну розмову про непривабливі сторінки нашого
минулого. Так, про них не хочеться згадувати і не хочеться розповідати чужим
людям. Але носити їх у собі – ще гірше, бо прихований біль буде муляти й
проявлятися в нас ще десятиліттями. Просто вирізати 90-ті з пам’яті теж не
вийде, бо без цієї декади ми не зможемо дати відповідь на головне питання:  «А
хто ми такі?»
