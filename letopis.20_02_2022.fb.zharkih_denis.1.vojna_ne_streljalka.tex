% vim: keymap=russian-jcukenwin
%%beginhead 
 
%%file 20_02_2022.fb.zharkih_denis.1.vojna_ne_streljalka
%%parent 20_02_2022
 
%%url https://www.facebook.com/permalink.php?story_fbid=3192839967596065&id=100006102787780
 
%%author_id zharkih_denis
%%date 
 
%%tags vojna
%%title Война не стрелялка
 
%%endhead 
 
\subsection{Война не стрелялка}
\label{sec:20_02_2022.fb.zharkih_denis.1.vojna_ne_streljalka}
 
\Purl{https://www.facebook.com/permalink.php?story_fbid=3192839967596065&id=100006102787780}
\ifcmt
 author_begin
   author_id zharkih_denis
 author_end
\fi

Война не стрелялка

Для начала два воспоминания. Мне лет 8-9 разговариваю с дедом, участником
войны, полная грудь медалей:

- Деда, а ты истребителем на войне был сколько немцев сбил?

- Лично ни одного. 

-??

- Пойми, глупенький, мне никто задачу сбить побольше немецких самолетов не
ставил. Моя задача, чтобы они не бомбили конвои, которые шли к нам от
союзников. И когда я был в воздухе конвои приходили целыми. А чтобы с ними
было, если бы мы гонялись за каждым немцем? Мы, конечно, с ними сражались, но
их задача была разбомбить корабли, а наша - не пустить их к кораблям. В войне
нужно побеждать, а не убивать. 

\ii{20_02_2022.fb.zharkih_denis.1.vojna_ne_streljalka.pic.1}

Последние слова перевернули мою детскую психику. Получалось, что дед прошел
почти всю войну (начал с обороны Киева), был на очень горячих участках,
постоянно участвовал в воздушных боях, но никого не убил, не сбил ни одного
вражеского самолета. За что же ему давали ордена и медали? Дед охотно объяснил
- за организацию системы ПВО. Он делал все, чтобы у нас был перевес в воздухе,
чтобы противник встречал отпор, даже там, где не ожидал. Ночные полеты,
дальние, порядок передачи конвоев, организация воздушных боев и перехватов,
новая техника, обеспечение, спасение сбитых летчиков, было чем заниматься на
войне, было за что очень ценили боевые товарищи и начальство. В войне, объяснял
дед, победит не тот, кто больше всех убьет, а кто лучше организован, кто
выполнит поставленные задачи и достигнет поставленных целей.  Так у меня
формировалось понимание, что реальная война далека от кино и стрелялки. 

\ii{20_02_2022.fb.zharkih_denis.1.vojna_ne_streljalka.pic.2}

Воспоминание второе. 2014 год. Одна весьма эмоциональная особа начала меня
стыдить за то, что я не поддерживаю Яценюка, а вокруг него должна объединиться
нация для победы над врагом. Народный фронт тогда уверенно занял первое место
на выборах, и... пролетел на следующих, как фанера над Парижем. Стыдно
оказалось не мне, а этой патриотке, которой я некоторое время напоминал ее
скандальное поведение. 

С Яценюком все было ясно с самого начала. Проще спрятаться от врага в одеяле,
чем за Великой Яценюковской стеной, которая была обыкновенным разводом на
деньги. Яценюк оказался не просто плохим организатором, он оказался популистом
и жуликом. А удалось ему обмануть массу людей, поскольку у них не было такого
деда, как у меня (мне повезло). 

В СССР, возможно, была не самая передовая техника, не самая красивая военная
форма и не лучшее питание у солдат. Но там понимали в мобилизации и организации
боевых действий. Всю эту культуру Украина потеряла за последние 30 лет. Нам
помогли ее потерять. Нужно понимать, что любое полицейское государство в
военном отношении всегда слабо. 

Сопротивление эффективное не столько тогда, когда звучат патриотические вопли и
выносят множество знамен (посмотрите \enquote{Триумф воли}), а когда у солдата не
воруют из котелка, а если находят вора, то расстреливают на месте. Когда бедные
не роются на помойках, а богатые в это время не жируют в ресторанах. Когда
делают ставку не на чудо оружие, а на реальных граждан, и в СМИ не
транслируется ненависть к \enquote{пятой колонне}, а идут сухие сводки о победах и
поражениях. 

И последнее. В Германии в конце войны был использован фольксштурм. Гражданским
раздавали оружие, надеясь, что они задержат наступление Советской Армии.
Оружие-то раздали, а организация обороны была слабенькая. Да и откуда ей было
взяться, ведь стыдить и обличать в непатриотизме и предательстве гораздо проще,
чем наладить питание, доставку боеприпасов и понимать военную обстановку в
хаосе отступления. Потому все это оказалось неэффективным. Более того,
фольксштурм сдавали свои. И для этого были мотивы. Вот сидит такой герой с
винтовкой или фаустом, и стреляет по войскам из окна. Войска разворачивают
самоходку и сносят весь дом, со всеми жителями. Гибнет не только солдат
фольксштурма, гибнут женщины, дети старики. 

И немцы, которые, как известно, любят порядок, и не любят, когда их дома сносят
с жителями, сами сдают фольксштурм. Но даже если не сдали - кончились патроны,
есть нечего, воды нет. Куда деваться? А очаги сопротивления оцепили, с оружием
не выйти. Только с поднятыми руками или закосить под гражданского.

А вот в СССР никакого фольксштурма не было. Были партизаны и подпольщики.
Только сейчас память о них оскверняют. Просто фольксштурм это побольше
вражеских солдат убить, а партизаны - войну выиграть. Но вспомним Яценюка,
нужно ли ему было что-то выигрывать? Он все, что хотел, то и получил. Развел
лохов, и получил.
