% vim: keymap=russian-jcukenwin
%%beginhead 
 
%%file 24_09_2022.stz.news.ua.donbas24.1.mrpl_ce_ukr_akcia.txt
%%parent 24_09_2022.stz.news.ua.donbas24.1.mrpl_ce_ukr_akcia
 
%%url 
 
%%author_id 
%%date 
 
%%tags 
%%title 
 
%%endhead 

«Маріуполь — це Україна»: як в містах України проходила національна акція
(фото)

Загальнонаціональна акція відбулася за ініціативою маріупольської спільноти у
тих містах, де створені центри «Я — Маріуполь»

24 вересня у Києві, Дніпрі, Львові, Вінниці, Черкасах, Одесі,
Івано-Франківську, Тернополі та Калуші пройшла всеукраїнська акція «Маріуполь —
це Україна». Зважаючи на заходи безпеки, не змогли провести акції лише в
Кривому Розі та Запоріжжі. У кожному місті маріупольці зібралися біля центрів
«ЯМаріуполь» з синьо-жовтими прапорами, щоб виступити проти псевдореферендумів
та розповісти свої історії. Маріупольська спільнота обрала саме цю дату для
проведення акції-мітингу, адже щорічно в останню суботу вересня у Маріуполі
відбувалося святкування Дня міста. Цей день об'єднав всіх маріупольців, нехай і
в різних містах України. Вони вкотре одноголосно заявили про те, що Маріуполь
був, є та буде українським.

Читайте також: Почуйте голос Маріуполя: історії людей, яким пощастило евакуюватися з блокадного міста

У Києві акція розпочалась виступом мешканки Маріуполя, депутатки міської ради,
підприємиці Ольги Пікули.

«Цього року ми не відкриваємо нові набережні, нові дитячі майданчики чи
лікарні. Нас позбавили цієї можливості. Всім нам було дуже важко вибиратися з
міста. Але ми вижили і дуже вдячні нашим захисникам, які, впевнена, незабаром
звільнять Маріуполь і ми всі зможемо повернутися та з новими силами почати його
відбудовувати та відновлювати», — підкреслила Ольга. 

Читайте також: Зростання руху опору в Маріуполі — як молодь протистоїть окупантам

Переважна більшість маріупольців, а це понад 200 тисяч людей, наразі
перебувають на підконтрольній території України. Всі вони вже зробили свій
вибір. Кожен маріуполець має свою історію виживання у блокадному місті.
Російські окупанти знищили їхні будинки та принесли руйнацію в місто, але
виїхавши з Маріуполя, люди не втратили віри, що в місто можливо повернутися.
Головне, щоб його звільнили! 

«Я готовий зробити все можливе для відновлення найбільшого Палацу культури на
Східній Україні. „Український дім“ ще відкриє свої двері і почне нове життя.
Головне повернути Маріуполь до України, а відбудувати і відновити все зможемо!
Для цього є і сили, і бажання», — наголосив директор МПК «Український дім» Ігор
Павлюк.

«Перша лікарня Маріуполя теж відновить свою роботу з поверненням Маріуполя до
України. Всі лікарі нашої лікарні одразу ж повернуться до міста і розпочнуть
роботу», — зазначила директорка Міської лікарні Маріуполя № 1 Лариса Мамаєва.

Читайте також: Alyona Alyona та Ukraїner представили кліп, присвячений
Маріуполю (ВІДЕО)

В окупованому місті й досі залишається багато маріупольців, які не визнають
фейкових референдумів. Всі вони чекають на ЗСУ та вірять в Україну, проте не
можуть про це вільно говорити через небезпеку.

«Я продовжую спілкуватися з людьми, які лишилися в Маріуполі. Вони чекають
деокупацію, але з різних причин не змогли виїхати. Деякі лишилися через власний
стан здоров'я, деякі - через стан здоров'я батьків. А цей референдум — це
справжній фарс, адже він проходить під дулами автоматів», — розповів
маріуполець, який нараз перебуває в Києві, Владислав Колесніков.

На мітинги-акції в різних містах України прийшло дуже багато дітей, молоді і
людей похилого віку, які читали вірші, ділилися власними переживаннями та
історіями. Всі вони зараз перебувають в різних містах, але їх об'єднує сильне
бажання повернутися до українського Маріуполя та віра, що це станеться
найближчим часом.

«Я народився і виріс у Маріуполі і не уявляю свого життя без рідного міста.
Хочу жити в Україні. Вся моя сім'я повернеться до Маріуполя, коли його
звільнять, адже не уявляємо свого життя без нього», — поділився думками 9-ти
річний маріуполець Денрис Креньов.

Нагадаємо раніше Донбас24 розповідав про онлайн-виставку української молоді.

Ще більше новин та найактуальніша інформація про Донецьку та Луганську області
в нашому телеграм-каналі Донбас24.

ФОТО: з відкритих джерел
