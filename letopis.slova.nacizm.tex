% vim: keymap=russian-jcukenwin
%%beginhead 
 
%%file slova.nacizm
%%parent slova
 
%%url 
 
%%author 
%%author_id 
%%author_url 
 
%%tags 
%%title 
 
%%endhead 
\chapter{Нацизм}
\label{sec:slova.nacizm}

%%%cit
%%%cit_head
%%%cit_pic
%%%cit_text
При этом, чему именно эти люди помогали - как-то стало забываться. Планы
\emph{нацистов} на Украину и другие республики европейской части СССР шокируют - от
выселения десяткой миллионов людей до расовой сегрегации и даже стерилизации.
\enquote{Страна} вспомнила, что готовили \emph{нацисты} Украине.  \enquote{Розовые}
планы Розенберга.  В основе отношения \emph{нацистов} к другим народам лежала их
\enquote{расовая теория} - псевдоучение о \enquote{высших} и \enquote{низших}
расах. По этой теории немцы были \enquote{высшей расой} (арийцами), а славяне
принадлежали к \enquote{низшей расе}.  Это был ключевой момент, который и
определял политику \emph{нацистов} к украинцам и другим славянам.  При этом отличия в
отношениях к отдельным славянским народам у \emph{нацистов} носили исключительно
тактический характер, исходя из текущих задач. Например, немцы поддерживали
хорватов против сербов. Отделили от Чехии Словакию, превратив ее в
марионеточное государство. Украинских же националистов хотели использовать в
борьбе против СССР и Польши
%%%cit_comment
%%%cit_title
\citTitle{22 июня - 80 лет нападения на СССР. Что немцы готовили для украинцев}, 
Максим Минин, strana.ua, 22.06.2021
%%%endcit


%%%cit
%%%cit_head
%%%cit_pic
%%%cit_text
І \emph{нацистська} влада виявилася не менш жорстокішою, ніж сталінська – як за
Сталіна можна було поплатитися свободою чи навіть життям за недоречно сказане
слово чи за «колосок», так і за Гітлера, і єдина «втіха» для селян була в тому,
що українську мову німці знали набагато гірше від більшовиків, і не завжди
могли зрозуміти та покарати за те, що почули. А українці тим іноді
користувалися. От як це було в Березані, що на Київщині.  Розповідають, що тут
був один дідусь, в якого за селом був баштан, і щоліта дід будував на баштані
собі курінь, в якому днював та ночував, охороняючи свої кавуни. І як звик він
лаяти кожного, хто йшов через його баштан у мирні часи, так робив і у воєнні,
незважаючи навіть на те, чи озброєні є ті люди, на кого він накидався зі своєю
лайкою
%%%cit_comment
%%%cit_title
\citTitle{Березань і Стародубщина під нацистською окупацією. Селяни швидко зрозуміли, хто є хто}, 
Ігор Роздобудько, www.radiosvoboda.org, 22.06.2021
%%%endcit

%%%cit
%%%cit_head
%%%cit_pic
%%%cit_text
22 червня 1941 р. \emph{нацистська} Німеччина напала на свого найближчого союзника, з
яким розпочала в Європі Другу світову війну – Радянський Союз. Радянське
керівництво та особисто І. Сталін виявилися повністю непідготовленими для такого
розвитку подій, помилково очікуючи напад ударних сил Вермахту на
Великобританію, хоча розвідка доповідала про можливість раптового військового
нападу саме на СРСР. Цей прорахунок державного керівництва дорого коштував усій
країні, а особливо простим радянським людям
%%%cit_comment
%%%cit_title
\citTitle{80 років нападу фашистської Німеччини. Україна дала все від себе для Перемоги над нацизмом - ХВИЛЯ}, 
Олександр Левченко, analytics.hvylya.net, 22.06.2021
%%%endcit

%%%cit
%%%cit_head
%%%cit_pic
%%%cit_text
В преддверии рокировок в украинском кабмине стало известно о назначении
главного правосека всея Украины Дмитрия Яроша внештатным советником
главнокомандующего ВСУ Валерия Залужного. Сообщил об этом в соцсети сам Ярош, а
через несколько дней появилась информация, также неофициальная, о том, что с
поста советника оголтелого \emph{нациста} сняли в течение суток после назначения. Что
это было? Видимо, в Киеве пытались прозондировать ситуацию и понять, насколько
готовы к сращению якобы новой власти и радикалов общество и заокеанские
хозяева. Обществу на Украине давно все до фонаря, а вот из Вашингтона и,
похоже, европейских столиц экспериментаторов из "слуг народа" быстро одернули.
Видимо, не готовы пока "партнеры" допустить откровенный удар по имиджу
"демократического государства", борющегося с "агрессором". И это несмотря на
то, что имеющий достаточный авторитет среди \emph{нацистских} отморозков Ярош вполне
смог бы помочь украинскому военному командованию решить вопросы с интеграцией
радикалов в армию и последующим списанием на них всех преступлений, творимых
ВФУ в Донбассе
%%%cit_comment
%%%cit_title
\citTitle{НЕДЕЛЯ ГЛАЗАМИ ЭКСПЕРТА: Ротация блокираторов, экобомбист и комедия Гайдая}, 
, lug-info.com, 07.11.2021
%%%endcit
