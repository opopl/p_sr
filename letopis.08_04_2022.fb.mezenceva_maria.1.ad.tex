% vim: keymap=russian-jcukenwin
%%beginhead 
 
%%file 08_04_2022.fb.mezenceva_maria.1.ad
%%parent 08_04_2022
 
%%url https://www.facebook.com/maria.mezentseva/posts/5330848443600852
 
%%author_id mezenceva_maria
%%date 
 
%%tags 
%%title Сьогодні ми побували там, де люди пережили справжній ад
 
%%endhead 
 
\subsection{Сьогодні ми побували там, де люди пережили справжній ад}
\label{sec:08_04_2022.fb.mezenceva_maria.1.ad}
 
\Purl{https://www.facebook.com/maria.mezentseva/posts/5330848443600852}
\ifcmt
 author_begin
   author_id mezenceva_maria
 author_end
\fi

Сьогодні ми побували там, де люди пережили справжній ад. Наші колеги з
Європаламенту (їх було >30) повинні це бачити і відчувати, оскільки вони
приймають різнння щодо нашого спільного майбутнього і реагують на сучасні
злочини росії проти людяності, військові злочини і невигадані історії, за якими
стоять реальні люди, сім'ї і їх страждання, рани яких ще дуже довго не зможуть
загоїться. 

Ірпінь...

Нещодавно мер Олександр Маркушин розповів, що російські окупанти у місті влаштували справжні звірства – вони розстрілювали людей, а потім їхали по тілах танками, після звільнення Ірпеня довелось збирати останки лопатами. Місцева команда влади та волонтерів зробила неймовірне. Зі 100 тис було населення було евакуйовано 95 тис. Всім відомий «дерев'яний містечок» працював вище будь-якої фізично можливої пропускної спроможності. Окремо хочу відзначити які мужні жінки в команді - це секретарка міськ ради і депутатки-жінки. Вони разом з волонтерами винесли весь цей біль на собі і тепер з посмішками на обличчях, хоча і сумними, готові перейти до відбудови міста.

\ii{08_04_2022.fb.mezenceva_maria.1.ad.pic.1}

Ірпінь, маючи 30\% окупованої території, став форпостом шляху до Києва. Міст з
фото-це символ цієї війни, тому є ідея залишити його в тому стані, яким він є
зараз і зробити там музей сучасної війни. А поряд побудувати новий. Єдине
питання - кошти. 

Буча... Абсолбтно не мілітарне місто, не можна зрозуміти чому така жорстокість
відбувалася саме тут. 

Дівчат та жінок гвалтували

Ми місце, де захоронено вже 260 тіл в 1 могилі. 

Зараз мега-важлива психологічна допомога! В кожному спілкуванні з міжнародними
партнерами і ЗМІ - закликаю до партнерства в цьому напрямку!  

Люди пережили пекло!

З 30 тис населення зараз в місті лишилося. ~3700 людей. 

З 4-5 тис людей, які постійно залишалися  в окупованій Бучі - кожен 10й вбитий
або закатований.

Багато тіл не були ідентифіковані, процес триває. Процедура ексгумації важка. 

Сьогодні ще 18 тіл було ідентифіковано. У всіх несумісні з життям поранення, їх
просто розстріляли.

В дитячому таборі «променистий» після того як вигнали окупантів 15 чоловік
знайшли закатованими із простріляним головами.

Майже кожна родина в Бучі має втрати серед родичів, рідних, друзів.

«Будівлі ми відновимо, Але людей не повернути, ніхто це не забуде!» - сказав
секретар місткої ради, який мужньо залишався в місті весь цей час! 

«Ті, хто пізнав своїх мертвих родичів хочуть єдиного - щоб тих, хто це зробив -
їх було покарано!» - додав він. 

Я не знаю де знайти справжні слова. І скільки ще «Буч» нам доведеться пережити. 

Я знаю як сильно страждають люди на Харківщині і чекають коридорів. 

Ми всі плачемо всередині, але збираємо сили для перемоги цього нелюдського,
незрозумілого, нелогічного ворога , який нищить все добре, розвинуте, справжнє
і так любиме всіма нами- нашу домівку - Україну! 

Ми повинні вистояти!

Молимося, віримо, сподіваємося!

Наші партнери їхали дорогою назад Абсолютно мовчки. 

Впевнені, ця подорож була важливою і потрібною.
