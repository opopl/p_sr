% vim: keymap=russian-jcukenwin
%%beginhead 
 
%%file 04_05_2021.fb.institut_serdca_kiev.1.operacia
%%parent 04_05_2021
 
%%url https://www.facebook.com/Heart.Institute.Kyiv/posts/3568382220052560
 
%%author 
%%author_id 
%%author_url 
 
%%tags 
%%title 
 
%%endhead 
\Purl{https://www.facebook.com/Heart.Institute.Kyiv/posts/3568382220052560}

#досвідрятує
👉В Інститут серця МОЗ України максимально впроваджені сучасні мініінвазині операції. 


\ifcmt
  pic https://scontent-frt3-2.xx.fbcdn.net/v/t1.6435-9/182484386_3568373756720073_8922738577850197089_n.jpg?_nc_cat=101&ccb=1-3&_nc_sid=730e14&_nc_ohc=EGkaCuPquZsAX-E2eTC&_nc_ht=scontent-frt3-2.xx&oh=6d5f9061c46366351f999692f484ea8f&oe=60C3557E
\fi


Днями було прооперовано  пацієнта із вираженою мітральною  недостатністю ішемічного характеру.

Пацієнт 52 років 10 місяців тому переніс гострий інфаркт міокарда, було проведено стентування коронарних артерій. Але з часом відбулося формування недостатності мітрального клапана внаслідок розширення фіброзного кільця. 
✅Провели оперативне втручання: пластика мітрального клапана із бокової мініінвазивної торакотомії (правосторонній міжреберний доступ 5-7 см), пацієнту було імплантовано штучне опорне кільце для покращення коаптації (замикання) стулок мітрального клапана.
🔹Вибір саме мініінвазивного доступу дозволяє людині в післяопераційному періоді отримати швидший та легший процес повернення до звичайного життя.  
Завдяки мінімальному травмуванню не потрібно чекати поки зростеться грудна кістка і вже за 8-10 днів можна повернутись до розумних фізичних навантажень, а за місяць — до попереднього звичайного способу життя. 
Дана методика супроводжується косметичним ефектом — це має великий вплив на психологічний  стан людини, що в комплексі працює для якісної, швидкої, комфортної післяопераційної реабілітації.
Операційна бригада:
Олег Зеленчук, Макс Ротарь (Maxim Rotari), Пресс В., Mykola Goncharenko, Інна Срипка.
Кардіолог: Наталья Ященко
Реаніматолог: Игорь Кузьмич
