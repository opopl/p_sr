% vim: keymap=russian-jcukenwin
%%beginhead 
 
%%file topics.vojna.my.6.collect.news
%%parent topics.vojna.my.6.collect
 
%%url 
 
%%author_id 
%%date 
 
%%tags 
%%title 
 
%%endhead 

%12:34:05 15-08-22
Названа примерная стоимость ущерба России от удара по аэродрому в Крыму
На аэродроме «Саки» базируется 43-й отдельный морской штурмовой авиаполк Черноморского флота. За несколько дней до взрыва, судя по спутниковым снимкам, вблизи аэродрома стояло более 20 самолетов. Преимущественно истребители Су-30СМ и разведчики Су-24МР, один тяжелый военно-транспортный Ил-76 и несколько вертолетов. Если так, то перед взрывом на аэродроме стояла авиатехника на сумму $650 млн – $850 млн.
На снимках после взрыва много авиатехники уничтожено. По подсчетам Forbes, российские самолеты примерно стоят:
✅ Су-30СМ – $50 млн
✅ Су-24МР – $6 млн
✅ Ил-76 – $86 млн.
✅ Истребитель Су-30СМ – $50 млн
В зависимости от различных данных, сумма потерь может варьироваться от $55 млн до до $450 млн:
✅ По данным Украины, Россия потеряла 9 самолетов на аэродроме в Новофедоровке. Не какие именно - неизвестно. Потому сумма потерь может варьироваться от $55 млн до $450 млн.
✅ Если взять за основу разведданные других источников, уничтожены не менее 11 самолетов - 5 истребителей Су-30СМ и 6 Су-24МР. Тогда сумма ущерба - около $300 млн.
✅ Если потери следующие - 8 Су-27, 4 Су-30СМ, 5 Су-24, 6 Ми-8 и Ил-20 - общая сумма ущерба около $380 млн.
🇺🇦 Украина Сейчас


%12:34:01 15-08-22
Окружение Путина направило Западу сигнал о желании вести переговоры.

Идут уже 173-е сутки полномасштабной войны, которую Россия ведет в Украине, в западных СМИ появилась информация, что некий высокопоставленный кремлевский чиновник тайно обратился к Западу, с просьбой помочь положить конец этой войне. Якобы он заявил, что Кремль в панике и отчаянно нуждается в прекращении войны.

Об этом пишет британское издание The Daily Mirror.

Информация, переданная этим кремлевским чиновником была распространена среди западных спецслужб. Отмечается, что старшие офицеры и чиновники, близкие к президенту России Владимиру Путину, встревожены западными санкциями и спадом экономики, вызванным войной.

Некоторые кремлевские деятели также обеспокоены все более тревожными рисками, такими как боевые действия на Запорожской атомной электростанции в Украине.

«Представитель путинского ближайшего окружения направил Западу сигнал о желании вести переговоры. Настроение кремлевской элиты — паническое», - сказано в отчете об информации от кремлевского чиновника, которого назвали «одним из столпов режима».


%16:27:37 16-08-22
Стало известно, что россияне хранили на уничтоженной базе под Джанкоем .

Незадолго до взрывов там значительно усилили ПВО.

Незадолго до взрывов в районе Джанкоя российские оккупанты значительно усилили ПВО. Предварительно там насчитывалось около 50 единиц различной техники.

Об этом сообщает СтратКом ВСУ.

В районе аэродрома вблизи Джанкой было развернуто, по предварительным данным:

не менее 8 ЗРС С-400;

3 ЗРК С-300;

5 ЗРПК «Панцирь-С1».

К тому же в районе железнодорожной станции «Азовская» производилась разгрузка:

17 ББМ;

10 САУ «Гвоздика»;

7 САУ «Мста-С»;

2 САУ «Пион»;

3 РСЗО БМ-21 «Град», а также боеприпасов.

«Кроме того, на аэродроме было сосредоточено 440 ТПК (транспортно-пусковые контейнеры, - ред.)», – говорится в сообщении.

Как добавляют в ведомстве, эти ТПК там были сконцентрированы в таком количестве не столько для защиты воздушного пространства, сколько для отправки в Херсонскую и Запорожскую области для обстрелов Николаева и Запорожья.

«Особенно накануне и непосредственно 24 августа (День Независимости Украины - ред.)», - подчеркивают в СтратКоме.

Напомним, утром 16 августа во временно оккупированном Крыму прогремели взрывы. В Джанкойском районе сдетонировало хранилище боеприпасов. 
(H)

%12:16:22 18-08-22
Айболитт Айболитыч
"Явка обязательна": в Петербурге массово рассылают повестки. Невзоров предупредил о силовом привлечении к войне
АВТОР: ИННА ЛЕВИЦКАЯ ЖУРНАЛИСТ, РЕДАКТОРI15 АВГУСТА 202214:35
Жители центра Санкт-Петербурга в массовом порядке получают повестки в военкоматы с указанием, что явка обязательна, сообщает местное издание "Фонтанка". Читатели издания рассказали журналистам, что видели в почтовых ящиках десятки повесток, адресованных людям разных возрастов: самому младшему 38 лет, самому старшему — 49. По их словам, с этим явлением они сталкиваются впервые.
"Вы обязаны 13–18 августа 2022 года явиться в военный комиссариат Адмиралтейского и Кировского районов города Санкт-Петербурга по адресу: Рижский проспект, д. 12, к. 28, для уточнения данных", — говорится в повестке. На документе стоят печать и подпись — военный комиссар Адмиралтейского и Кировского районов Р. Валеев. Отправители просят принести с собой паспорт и военный билет.
Как сообщают журналисты, на Рижском, 12, действительно находится военкомат, а Руслан Валеев с 2019 года служит в этом военкомате начальником отделения подготовки и призыва граждан на военную службу. В "Фонтанке" добавили, что в Петроградском районе сотрудники военкомата обзванивают состоящих на военном учете граждан.
"Так-с. Именно об этом я честно предупреждал три недели назад. Сейчас пока еще не "раскачались", но по военкоматам и РУВД уже разосланы "пакеты", в которых "мобилизаторам" даны права на силовое привлечение населения к войне. Как в "ДНР-ЛНР". В Питере все будет, разумеется, так же тупо и беззаконно, как и там. Никакие "отмазы" работать не будут. За свою трусость, за свое молчание тысячи питерцев оденутся в черные пластиковые мешки", — так прокомментировал эту новость российский оппозиционер, публицист Александр Невзоров.
