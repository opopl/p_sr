%%beginhead 
 
%%file 11_01_2023.fb.rojz_svitlana.kyiv.1.ya_khochu_zapam_yata
%%parent 11_01_2023
 
%%url https://www.facebook.com/svetlanaroyz/posts/pfbid02rXrSs4bNhLVrVriU2n1kfBL8XjUKQT19MgqbLyP1MVAWrxDRS6bCsmQam5pcm8ZFl
 
%%author_id rojz_svitlana.kyiv
%%date 11_01_2023
 
%%tags 
%%title Я хочу запам'ятати вчорашній вечір
 
%%endhead 

\subsection{Я хочу запам'ятати вчорашній вечір}
\label{sec:11_01_2023.fb.rojz_svitlana.kyiv.1.ya_khochu_zapam_yata}

\Purl{https://www.facebook.com/svetlanaroyz/posts/pfbid02rXrSs4bNhLVrVriU2n1kfBL8XjUKQT19MgqbLyP1MVAWrxDRS6bCsmQam5pcm8ZFl}
\ifcmt
 author_begin
   author_id rojz_svitlana.kyiv
 author_end
\fi

Я хочу запам'ятати вчорашній вечір. Наче нічого важливого і не відбувалося, але
було трепетно важливе. Донька попросила записати її в онлайн художню школу. В
понеділок мало бути перше заняття, але його перенесли на вчора, бо у викладача
не було світла. А вчора майже весь день не було світла у нас. Ми цілий день
грілися біля пічки, заощаджували енергію акумуляторів, щоб включити доньці
лампочку, підживлювати комп'ютер та організувати їй інтернет. Вона прийшла із
школи втомленою, але так радісно  сіла у екрана зі своїми фарбами, альбомом,
олівцями. Я залишилась в кімнаті, щоб допомогти, якщо б були проблеми із
зв'язком. 

Я не бачила екрана, тільки слухала. Як викладач вітався з дітьми, вони були з
різних міст України.  У багатьох теж не було світла. Він присвятив першу
зустріч котикам (безпрограшний варіант 🙂. І першу лекційну частину  показував
котиків різних художників, різних стилів, різних епох. Трохи розповідав про
них. 

Я відклала текст, над яким працювала... Слухала і думала. 

Ворог майже рік намагається вкрасти у нас теперішнє і майбутнє. А попри все -
діти, без світла в домівках, слухають лекцію про мистецтво, малюють своїх
котиків, і роблять самі внесок в своє теперішнє і майбутнє. Я думала, тільки б
зараз не було тривоги - так не хотілось, щоб їх такий красивий онлайн простір
щось турбувало.

Я думала про безсилля ворога. І про нашу силу. 

Коли починалась лекція, кілька дітей розмовляли російською. Викладач їх не
виправляв, просто проводив заняття. Під кінець - всі перейшли на українську. 

Наші котики

Обіймаю, Родино ❤️ Як хочу Перемоги
