% vim: keymap=russian-jcukenwin
%%beginhead 
 
%%file 05_12_2021.fb.tolkachev_aleksej.1.spivbesida
%%parent 05_12_2021
 
%%url https://www.facebook.com/oleksiy.tolkachov/posts/4885603421459331
 
%%author_id tolkachev_aleksej
%%date 
 
%%tags filosofia,obschestvo,skovoroda_grigorii,sobesedovanie,ukraina
%%title СПІВБЕСІДА
 
%%endhead 
 
\subsection{СПІВБЕСІДА}
\label{sec:05_12_2021.fb.tolkachev_aleksej.1.spivbesida}
 
\Purl{https://www.facebook.com/oleksiy.tolkachov/posts/4885603421459331}
\ifcmt
 author_begin
   author_id tolkachev_aleksej
 author_end
\fi

\ii{05_12_2021.fb.tolkachev_aleksej.1.spivbesida.pic.1}
\obeycr
СПІВБЕСІДА
- Добрий день, Григорію. 
- Добрий день.
- Мене звати Микола. Ви відгукнулися на позицію \enquote{копірайтер}. Ще цікавить робота?
- Так.
- Доречі, це у вас прізвище чи псевдонім?
- Прізвище.
- Доволі незвичайне! В резюме ви пишете, що навчались в Києво-Могилянській Академії на філософському факультеті, але не здобули вищої освіти.
- Так, плани на життя змінились. 
- В 1751 році ви працювали вчителем поетики в колегіумі. Яка причина звільнення?
- Викладав поетику за власною програмою. Це не дуже подобалось керівництву.
- Зрозуміло. А з Харківського колегіуму чого звільнились?
- Причина звільнення у трудовій книжці - \enquote{не захотів прийняти постриг}.
- Тобто, ви не бачите себе в духовній службі?
- Ні.
- А як відноситесь до Бога?
- Щоб пізнати Бога, треба пізнати самого себе! 
- Слушна думка. Ви повернулися до Харківсього колегіуму, а потім знову звільнилися.
- Так, через конфлікт з єпископом.
- Зрозуміло. В резюме вказано, що ви володієте польською, німецькою та грецькою мовами.
- Так, а ще церковнослов'янською.
- Які тексти пишете?
- Філософські трактати, вірші, пісні, байки.
- Якою мовою?
- Переважно, слов'яноукраїнською. Але можу і латинською.
- Нам потрібен копірайтер для написання рекламних текстів. Пропоную невелике технічне завдання. Я вам кажу тему, а ви мені - слоган. Наприклад, реклама курсів з тайм-менеджменту.
- З усіх утрат втрата часу найтяжча.
- Добре, реклама книгарень.
- Одне мені тільки близьке, вигукну я: о школо, о книги!
- Непогано! Продаж дзеркал.
- З видимого пізнавай невидиме!
- Послуги коучингу.
- Збери всередині себе свої думки і в собі самому шукай справжніх благ.
- Ресторанний бізнес.
- Всяка їжа і пиття смачні й корисні, але треба знати час, місце і міру.
- Кардіохірургія.
- Без ядра горіх ніщо, так само як і людина без серця.
- Добре, і наостанок - реклама політичної партії.
- Демон проти демона не свідчить, вовк вовчого м'яса не їсть.
- Григорію, дуже добре! Підкажіть, які у вас інтереси окрім основної роботи?
- Колись співав в операх, літургіях, на маскарадах. Люблю мандрівки.
- Останнє питання - чим зараз заробляєте на життя?
- В основному, фріланс. Приватні уроки.
- Дякую за інформацію. Ми вам зателефонуємо.
***
- Добрий день. Це Григорій. Я проходив співбесіду на позицію "копірайтер". Хотів би поцікавитись підсумками.
- Добрий день, Григорію. Тексти у вас гарні, але ми не можемо взяти вас в штат компанії.
- Чому?
- Тест показав, що ви занадто вільнодумна людина. А ще - ви часто змінюєте місця роботи. Це суперечить політиці компанії.
(С) Anton Ivakhnoff - автор тексту
\restorecr

\ii{05_12_2021.fb.tolkachev_aleksej.1.spivbesida.cmt}
