% vim: keymap=russian-jcukenwin
%%beginhead 
 
%%file topics.vojna.my.11.filmy.1.mind_games
%%parent topics.vojna.my.11.filmy
 
%%url 
 
%%author_id 
%%date 
 
%%tags 
%%title 
 
%%endhead 

Здесь я пишу о своих впечатлениях о фильме Игры Разума (Mind Games), год 2021.
Я раньше смотрел другой фильм, тоже Игры Разума по русски, а по английски
Beautiful Mind, то есть Прекрасный Разум, о гениальном математике из Принстона,
Джоне Неше (John Nash), который являлся реально живущим человеком, который
действительно был гениальным математиком, страдающим умственным расстройством
(шизофрения), но который в итоге сумел научиться жить со своим душевным
недугом, в итоге вернувшись к полноценной работе ученого. По его биографии и
был снят замечательный фильм, который был даже удостоен в свое время Оскара.
Очень интересная и поучительная жизненная история, замечательная и тонкая игра
актеров, всем рекомендую посмотреть! Так вот, находясь под впечатлением от
этого фильма, я ожидал увидеть что-то похожее в фильме 2021 года с таким же
названием, но в этом фильме мои ожидания не оправдались, потому что я увидел
совершенно другую историю, гораздо более трагичную, ужасную, довольно тяжелую
для психики, можно даже сказать, выворачивающую душу наизнанку, причем
счастливого конца (хеппи энда), как это обычно бывает в американских фильмах,
здесь нет. Признаюсь, тяжело было заставить себя досмотреть этот фильм до
конца.

Ну что ж, начнем. Фабула фильма такова. Молодой армейский психолог, Тара Мейс,
к тому же молодая, красивая, жизнерадостная девушка, что то делает у себя дома.
Внезапно раздается звонок в дверь, Тара идет открывать дверь, там стоит
мужчина, Тара говорит - спасибо, вы там что то привезли, то есть, как будто
посыльный или курьер что то привез, вроде там речь шла про пиццу. Тара говорит
ему, не ожидая подвоха, но внезапно ей в лицо пшикают чем-то, и сцена
заканчивается, очевидно, пшикают чем-то, что влияет на сознание, Тара теряет
сознание, и потом уже обнаруживает себя в фургоне, прикованной цепью. Она
обнаруживает, что ее похитили, неизвестно кто. Тара в шоке и ужасе от того, что
происходит, и пытается собраться с мыслями. Ситуация ужасная, потому что
непонятно кто, зачем и почему решил ее похитить и поместить в фургон, приковав
ее к тому же цепью.

Параллельно, идет рассказ о работы Тары, а она, как уже было сказано, работает
армейским психологом в армии США, а именно, ее основная задача -
психологическая помощь военнослужащим, особенно служащим элитных войск
(спецназа), в решении их психологических проблем. Попутно она еще встречается с
молодым парнем, и периодически есть сцены с этим парнем. А что касается работы
психологом. Сидя уже запертой в фургоне, Тара начинает мучительно вспоминать
эпизоды своей работы с военными заключенными, а эти разговоры обычно
происходили в камере, где с одной стороны сидит психолог (Тара Мейс), а с
другой - сам заключенный, которого Тара спрашивала о психологических проблемах,
и с кем Тара общалась, беседовала, тет-а-тет, за круглым столом, с одной
стороны Тара, а с другой - заключенный, сидящий за столом напротив Тары, с
руками, положенными на стол, и закованными в наручники.

Тара сидит в фургоне, плачет, думает, как выбраться, спастись. Ее мучают мысли,
кто ее похитил, зачем, с какой целью. Попутно, она начинает вспоминать свою
работу психологом, и свой опыт общения с заключенными. Первый случай - Эдди, молодой парень, на вид лет 30.
Далее как происходил диалог, дословно. Итак, обстановка. Тюремная камера, Эдди, Тара,
стол, Эдди сидит в наручниках, идет разговор.

Начало вставки разговора.

Эдди - почему?

Тара - потому что это тебе поможет

Эдди - миссия засекречена

Тара - Да, конечно, Эдди, но... у меня есть необходимый допуск. И все, о чем мы
поговорим, останется тайной (\emph{после некоторой паузы}) Слушай... ты можешь не
говорить, о чем не хочешь (\emph{Тара задумчиво смотрит в сторону, прикусив губы,
потом говорит}), но расскажи о своих чувствах. Что ты думаешь, чтобы мы вместе
это решили

Эдди - МЫ. Мне нравится, как ты это сделала

Тара - что сделала? 

Эдди - то, как ты взяла мою проблему, и сделала ее нашей (\emph{показывает на Тару,
усмехается, презрительно, внимательно потом смотрит на Тару и говорит дальше}).
Ты хороша. Но каким это образом ты сделаешь так, что мои демоны будут заперты
за дверью? (внимательно и испытующе смотрит на Тару)

Тара - ты должен мне довериться, Эдди. Я хочу тебе помочь. 

Эдди - довериться?! Неплохо, док! Ты же знаешь, что именно так мы говорили
местным. Когда мы шли через их деревню... и хотели, чтобы они рисковали своими
жизнями, чтобы бороться в нашей войне. Довериться нам...  Поверьте нам!...
(\emph{язвительно, руки сложил вместе, как будто для молитвы}) Мы пришли помочь
(\emph{Тара смотрит молча}) Ты знаешь, как и я, что это полная чушь. Мы готовим
их к войне. А когда президент говорит "хватит", мы возвращаемся домой в Америку
(\emph{Тара молчит, опустила глаза, Эдди продолжает говорить}). А те местные
должны смотреть, как их дочерей насилуют, и ... (\emph{Тара поднимает глаза на
Эдди}) ... убивают как чертовых свиней!

Тара - Эдди, ты делал свою работу. Ты не можешь верить, что все эти люди - твоя
ответственность.

Эдди - Ну да, хорошо!

Тара - ... если все правда было так плохо, почему ты хочешь вернуться? 

Эдди - а это хороший вопрос! Молодец! ( \emph{Насмешливо-иронически; Тара
молчит, Эдди собирается с мыслями, потом продолжает, наклоняясь немного вперед}
) ... потому что это единственное, что я умею. Я очень...  очень хорош в этом. 

Тара - ладно, Эдди. Давай ты расскажешь мне о том, как ты попал в специальные
войска. Расскажи немного про свою жизнь... У тебя есть братья и сестры? Чем
зарабатывает твой отец? Расскажи про свою мать, дай мне узнать тебя получше. 

Эдди - серьезно?!  Четыре профессии, которые я презираю... хочешь узнать, какие?

Тара - Да...
 
Эдди - Ладно. Первая - адвокаты, терпеть их не могу! Вторая - врачи. Третья -
старые добрые продавцы машин.  И четвертая, моя любимая из всех - психологи.
Потому что вы все делаете вид, будто знаете больше меня. И верите, в своем
извращенном уме, что мне нужна ваша помощь! Но нет!

Тара - Эдди, я хочу тебе помочь!

Эдди - ты читала мое досье. Ты знаешь, что в семь лет я остался сиротой. И
знаешь, что я закончил школу с отличием. Знаешь, что у меня диплом бакалавра по
преподаванию. А еще знаешь, что спустя три года я пошел в армию. А после этого
еще через два года меня взяли в зеленые береты. Ты знаешь обо мне все! Но ты
хочешь сидеть, смотреть мне в глаза, и играть в эти гребаные Игры Разума!!! О
том, что я чувствую, и не заставляет ли меня тот факт, что мой отец был
министр, сомневаться в том, что я делаю!

Тара - твой отец...

Эдди - мой отец был механиком! Но это ты уже знаешь. Док! Я устал убивать. Но
это все, что я знаю. Все, что умею.

Тара - Эдди, я читала твое досье. И еще я знаю, что твоя команда спасла многих
заложников. Да, ты лишал людей жизни, но при этом спас гораздо больше. Жизней,
которые были бы потеряны, если бы не ты. Ты сделал много хорошего

Эдди - у нас есть девиз в спецвойсках, Де Абрессо Либре - Свободу Угнетенным!
Подумай об этом!

( \emph{Конец Разговора Тары с Эдди} )

Тара сидит, думает, не связан ли ее разговор с Эдди с тем, что ее похитили, и
для чего...  Вспоминает, что Эдди служил в каких то психологических
спецвойсках...  А к чему это. А к тому, что спустя некоторое время в фургон
входит ее похититель - жесткий мужчина лет 40-45, лысый, с бородой, жесткими
безжалостными глазами - и говорит, смотря на Тару - Де Абрессо Либре,
Кассандра!  Тара, недоуменно отвечает - я не Кассандра! Вы взяли не ту! А
похититель - нет, тебя зовут Кассандра.  Тара начинает понимать, что, возможно,
ее захватчик по факту спецназовец. Похититель далее продолжает...  Тебя зовут
Кассандра. Сейчас и здесь, тебя зовут Кассандра... Потом показывает пальцем на
себя - а меня зовут Брюс.  Понятно? Тара - да, я Кассандра, а тебя зовут Брюс.
Брюс отвечает - хорошо! Тара начинает его умолять, отпусти меня, отпусти! Брюс
молча смотрит... и потом говорит... не сейчас... Всему Свое Время... потом
уходит, закрывает дверь фургона, и Тара снова остается одна...

... Тара сидит в фургоне, на поле, со слезами на глазах, поникшая,
раздавленная, с распущенными волосами... Куда же подевалась веселая улыбчивая
девушка Тара, невеста своего парня Брайена, с восхищением любовно оглядывающая
свое только что сделанное белое, необычайно красивое свадебное платье, висящее
на стене в ее доме! Куда подевалась спокойная, собранная Тара Мейс, армейский
психолог, чей долг помогать военнослужащим преодолевать их психологические
проблемы? Это все исчезло... испарилось...  осталась похищенная Тара... в
шоковом состоянии, не знающая, что делать, как выбираться из ее заточения...
не знающая, что с ней будет дальшее, в полном отчаянии, сидя на полу,
пытающаяся сообразить свои дальнейшие действия... 

... Как только Тара осознает, что она в плену в фургоне, появляется ее
визуальный образ как армейского психолога - строгого, невозмутимого, в
армейской форме, подтянутой и сосредоточенной, ее обычный профессиональный
образ. Тара, как армейский психолог, как служащий армии США, начинает
советовать Таре-заложнице, что делать, что думать, какие есть варианты. Думай,
Тара, думай! - повторяет раз за раз Тара-психолог Таре-пленнице. 






