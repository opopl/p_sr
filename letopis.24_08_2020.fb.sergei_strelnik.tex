% vim: keymap=russian-jcukenwin
%%beginhead 
 
%%file 24_08_2020.fb.sergei_strelnik
%%parent 24_08_2020
 
%%endhead 
\subsection{Сергей Стрельник --- Доброе утро друзья!}
\url{https://www.facebook.com/permalink.php?story_fbid=3388917331138478&id=100000607340736}
  
\vspace{0.5cm}
 {\ifDEBUG\small\LaTeX~section: \verb|24_08_2020.fb.sergei_strelnik| project: \verb|letopis| rootid: \verb|p_saintrussia| \fi}
\vspace{0.5cm}
  
Доброе утро друзья!

Сегодня, 24 августа хочу поздравить вас с Днем разжалования планеты Плутон, Днем извержения Везувия, Международным днем странной музыки и Днем падающей звезды.

Других праздников у меня для вас, к сожалению, нет.

\begin{multicols}{2}
	\obeycr
Мы от совести независимы
Стали все в девяносто первом.
Подменили нам ложью истину,
А потом у всех сдали нервы.

На искусственном чувстве гордости
Вышли комплексы каждой нации.
Без границ стали только подлости,
И бездонность - у деградации.

Но с каким отмечаем пафосом
Каждый год нашей общей трагедии!
Флаги - разные, но под чей фасон
В политической недокомедии?

Двадцать девять лет - преступления,
Слёзы, голод, кровь и предательства.
Двадцать девять лет - исступления
От насилья и надругательства.
Двадцать девять лет - безысходности,
Безнадёжности и отчаянья.
Двадцать девять лет - безнародности,
Масс бездушных во мгле скитания.
Как поминки, я независимость
Отмечаю, и стопка накрыта.
Оказалось, свобода лишь - видимость,
Так, за что же страна убита?  
	\restorecr
\end{multicols}

\index[index.authors.rus]{Стрельник, Сергей}

