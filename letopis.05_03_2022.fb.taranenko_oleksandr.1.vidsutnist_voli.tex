% vim: keymap=russian-jcukenwin
%%beginhead 
 
%%file 05_03_2022.fb.taranenko_oleksandr.1.vidsutnist_voli
%%parent 05_03_2022
 
%%url https://www.facebook.com/olestaran/posts/4962406610462905
 
%%author_id taranenko_oleksandr
%%date 
 
%%tags 
%%title Відсутність волі
 
%%endhead 
 
\subsection{Відсутність волі}
\label{sec:05_03_2022.fb.taranenko_oleksandr.1.vidsutnist_voli}
 
\Purl{https://www.facebook.com/olestaran/posts/4962406610462905}
\ifcmt
 author_begin
   author_id taranenko_oleksandr
 author_end
\fi

Моє поле бою — це інтернет: блокування нескінчених ворожих акаунтів, створення
сторінок у соцмережах та розповсюдження інформації про події в Україні для
заходу, та для росіян. За останній тиждень я залишив тисячі коментарів, та вів
безліч диалогів. Так от, НАЙБІЛЬШУ ЗНЕВАГУ викликає те, що проходить червоною
лінією через всю цю комунікацію — це повна ВІДСУТНІСТЬ ВОЛІ. 

\ii{05_03_2022.fb.taranenko_oleksandr.1.vidsutnist_voli.pic.1}

Ще можна якось зрозуміти, коли це пише який небудь «сєртіфєцирований мастєр
ногтєвого сєрвіса» з Санкт Петербургу (увесь інстаграм забитий такими
коментарями, навіть не буду скріншоти прикладати), але від чого волосся стає
дибки, так це від того, що кажуть родичі та матусі полонених або вбитих
російських солдат — там нескінчене «а что ми можєм сдєлать?». 

\ii{05_03_2022.fb.taranenko_oleksandr.1.vidsutnist_voli.pic.2}

Вони почувають себе так САМІ — це глибинна внутрішня відсутність права щось
змінювати в своєму житті, та в своїй країні. І доки вони не знайдуть в собі цей
внутрішній стрижень, на який зможуть спертися та почати творити свою долю, їм
ніхто не допоможе.

\ii{05_03_2022.fb.taranenko_oleksandr.1.vidsutnist_voli.cmt}
