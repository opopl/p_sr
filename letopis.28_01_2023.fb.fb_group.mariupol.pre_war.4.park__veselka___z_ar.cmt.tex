% vim: keymap=russian-jcukenwin
%%beginhead 
 
%%file 28_01_2023.fb.fb_group.mariupol.pre_war.4.park__veselka___z_ar.cmt
%%parent 28_01_2023.fb.fb_group.mariupol.pre_war.4.park__veselka___z_ar
 
%%url 
 
%%author_id 
%%date 
 
%%tags 
%%title 
 
%%endhead 

\qqSecCmt

\iusr{Олеся Зыкова}

Как красиво! Не верится что ничего этого уже нет

\begin{itemize} % {
\iusr{Іванова Яна}
\textbf{Олеся Зыкова} 

Веселка на щастя існує, це нас там немає (

У ній напис навпроти ЦНАПу розфарбували прапором ворога.

\end{itemize} % }

\iusr{Клавдия Макарова}

Мы дійсно були щасливі бо бачили цю красу і вірили, що так буде завжди, так нам
обіцяли.

\begin{itemize} % {
\iusr{Ігор Ярмоленко}
\textbf{Клавдия Макарова} 

І ще особливро приємно, що таку красу тоді зробили на Лівому, бо завжди все
робили майже тільки в центрі. Це потім вже площу Свободи і парк Гурова зробили,
а так позитивні зміни здебільшого бачив лише центр.

\begin{itemize} % {
\iusr{Клавдия Макарова}
\textbf{Ihor Yarmolenko} 

Я так і не побачила площу на лівому, казали дуже красиво. Так було ще багато
роботи у місті, але він вже був красивий і ми могли їм гордитися.

\iusr{Ігор Ярмоленко}
\textbf{Клавдия Макарова} Вірно! Об'єктивно Маріуполь кардинально змінився за 2015-2021 роки. Якщо порівняти, наприклад, 2013 і 2021 - два різних міста.

\iusr{Ігор Ярмоленко}

До речі, в мене є в архівах 2013 рік - треба буде теж згодом викласти,
доведеться порушити хронологію (я почав викладати з 2015, і сьогодні вже 2019).

\end{itemize} % }

\end{itemize} % }

\iusr{Оля Томина}

Любимый парк. Часто там гуляли с ребенком 😭

\iusr{Светлана Надеждина}

Земляки, разом до Перемоги!
