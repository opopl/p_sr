% vim: keymap=russian-jcukenwin
%%beginhead 
 
%%file saint_russia.buzina.1.geroi_mirovogo_urovnja
%%parent saint_russia
%%url https://buzina.org/povtorenie/2675-geroi-mirovogo-urovnia.html
 
%%endhead 

\subsection{Истории от Олеся Бузины. Герои мирового уровня}

\url{https://buzina.org/povtorenie/2675-geroi-mirovogo-urovnia.html}

Гренадеры Фридриха Великого. Как и Украина, Пруссия не имела естественных
границ, но смело вступала в бой против трех держав сразу

Вкус к подвигам можно воспитать только на примере великих свершений.

В любом человеке живет жажда подвига. Кто-то хвастается тем, сколько может
выпить. Другой лезет в горы и покоряет вершины. Взлететь выше всех, пробежать
быстрее всех и… убить больше всех — это в природе человека. Мужской подвиг
всегда был прямо пропорционален количеству отрезанных голов. Скифы цепляли
скальпы побежденных врагов к уздечке лошади. Асы Второй мировой рисовали
отметки о победах на своих самолетах. Наши — красными звездочками на фюзеляже.
Немецкие — черными палочками на хвосте. Все это — знаки смерти, как и зарубки
на снайперской винтовке. "Нет большего счастья для мужчины, — сказал Чингисхан,
— чем убить врага, угнать его скот и овладеть его женщинами". Обычно стесняются
процитировать его крылатое выражение целиком. Я тоже сомневался поначалу,
смущать ли мне публику? Но решил, что публика достаточно подготовлена. Великий
завоеватель добавил, что счастье еще и в том, чтобы уничтожить всех детей
врага, которые успели вырасти выше колеса телеги.

Конечно, это варварское счастье. Но, к примеру, нынешняя американская
геополитическая мечта ничем, в принципе, не отличается от счастья Чингисхана.
Только вместо скота "угоняют" нефть. А детей — сжигают напалмом, не особенно
разбираясь, кто из них выше колеса "Форда" или "Крайслера". Вот и вся разница.
Разглядывая фотографии героев, отмеченных орденами и медалями, мы должны
понимать, что это снимки удачливых убийц.

Я понимаю, что сейчас на меня набросятся "гуманисты", обвинив в прославлении
насилия. Но они не правы. Я не прославляю насилие. Я просто обращаю внимание на
то, что в современном мире оно занимает не меньшее место, чем добро и
милосердие. Одно не существует без другого. Страна, утратившая вкус к
героическому, забывшая подвиги предков, обречена на умирание. Мне больно
слышать, что какие-то макаки в очередной раз захватили украинское судно. Мне
противно осознавать, что предыдущий президент, так много говоривший о
патриотизме, отдал кусок украинского шельфа возле острова Змеиный потомкам
карпатских угонщиков скота из Румынии. Мне омерзительно наблюдать, как вместо
культа героического Украине навязывают культ жертв и "терпил"..

Нас, советских школьников, воспитывали на других примерах. Мы читали книжки о
пионерах-героях. Гордились взятием Берлина. Мечтали быть пограничниками и
служить в армии. Именно поэтому никто из нас не будет воспринимать как героев
рекордсменов отсидки в схронах и мастеров нападения из-за угла. Можно назвать
УПА армией. Но нельзя назвать ни одной выигранной этой "армией" битвы. Можно
прославлять Шухевича и Бандеру. И даже присвоить им звание "Герой Украины". Но
они так и останутся мелкими террористами, мало интересными мировой табели
рангов. Кто такой Бандера по сравнению с тем же Иваном Кожедубом, по нескольку
раз в день поднимавшимся в воздух на боевой вылет и сбившим 62 немецких
самолета? Никто. Пустое место. Что такое УПА по сравнению с партизанами
Ковпака? Толпа дезертиров, боявшихся служить в приличных армиях. Ни разу УПА не
осмелилась вылезти из карпатских или волынских лесов. А мой земляк Ковпак,
родившийся в Котельве возле Ворсклы (это самый восток Украины), совершил свой
Карпатский рейд, взорвав нефтепромыслы и наведя шороху в "ведомстве" Шухевича.
Вот это — подвиг! Но, кроме того, Ковпак еще и умел быть добрым: на прием к
нему, когда после войны знаменитый партизанский командир стал зампредседателя
Президиума Верховного Совета УССР, мог прийти любой колхозник. И Сидор
Артемьевич решал его проблемы!

Советские герои были настолько мощны, что сам Геббельс сказал однажды в 1945-м:
"Сверхчеловек пришел с Востока". Немцы себя считали "сверхлюдьми". Они
поставили гигантский мировой эксперимент, решив померяться силами с другими
нациями. Вывод Геббельса совершенно логичен: если победителей Франции, Польши,
Дании, Бельгии, Норвегии разбили советские люди (те, кого нынешние "либерасты"
называют "совками"), значит, эти "совки" и были подлинными "сверхлюдьми".

Фридрих II однажды сказал перед атакой: "Солдаты, вы что, собираетесь жить
вечно?"

А уж немцы имели вкус к героическому. Они воспитывали его веками. Это в
нынешней Украине образец для подражания — гетман Полуботок, спрятавший свою
мифическую бочку золота в британском банке. А Германия росла на примере
прусского короля Фридриха Великого. Когда в 1745 году этот великий полководец,
воюя одновременно с австрийцами, саксонцами и французами, остался совершенно
без средств, все ждали, что он поднимет руки. Но король срезал даже серебряные
пуговицы со своего мундира. Все, чем он обладал, было превращено в деньги на
армию. Вот как это описывает историк XIX века Федор Кони: "Фридрих увидел, что
при таких обстоятельствах он может полагаться только на самого себя. Надлежало
увеличить силы Пруссии, и на этот предмет он не пощадил ни государственной
казны, ни даже собственного достояния. Из казначейства было вынуто 6 миллионов
талеров, со всего государства сделан поземельный побор в полтора миллиона. Вся
серебряная утварь, украшавшая дворец: канделябры, столы, люстры, камины и даже
серебряные духовые инструменты были обращены в деньги. Каждую ночь двенадцать
гайдуков переносили вещи на лодки и отправляли их на монетный двор. Все
делалось тихо и скрытно, чтобы не возбудить в народе беспокойство и опасений
таким явным признаком государственной нужды. Но эти распоряжения дали королю
возможность увеличить войско и обеспечить его на долгое время всем
необходимым".

В тот же год Фридрих добился полной победы над своими противниками, обложил их
контрибуцией и не только вернул затраты, но и остался с прибылью. Какой Мазепа
способен на такое?

О Фридрихе Великом нам вообще или ничего не рассказывали, или изображали его
как авантюриста, едва уцелевшего во время Семилетней войны, благодаря
преждевременной смерти главного врага – императрицы Елизаветы Петровны. Но ведь
в ту войну Фридрих умудрялся метаться со своей армией между французами,
австрийцами и русскими, выигрывать сражения и тут же перебрасывать войска на
другой фронт. Один против троих! Если его и спасло чудо, то он его заслужил.

Представьте на месте Фридриха любого украинского гетмана. Сколько бы слез и
соплей пролили наши историки по поводу "проклятих вороженьок"! Сколько было бы
сказано, что земля у нас не та, что на ней плохо обороняться, что нет
естественных преград… Но ведь в Пруссии тоже не было никаких естественных
преград, кроме… железных солдат старого Фрица, как его называли! Русские
полководцы хорошо понимали, с кем имели дело и у кого учились. Знаменитый
фельдмаршал Румянцев — ветеран Семилетней войны и победитель турок при
Екатерине II — до самой смерти сохранил культ прусского короля, с которым
сражался в молодости. Именно его он считал лучшим европейским полководцем
своего времени.

ФЕЛЬДМАРШАЛ БЛЮХЕР — ОБРАЗЕЦ ДЛЯ ПОДРАЖАНИЯ ВСЕМ ПЕНСИОНЕРАМ 

"Старый черт" Блюхер вылез из-под лошади и вломил Наполеону

На победах Фридриха Великого выросли целые поколения немцев. Его невероятно
ценил Наполеон. Удивительно, но рука прусского короля дотянулась до императора
даже из могилы. Победную точку в наполеоновских войнах поставил прусский
фельдмаршал Блюхер. Именно он успел на помощь англичанам при Ватерлоо. В 1815
году, когда произошла эта битва, Блюхеру было уже 73 года. По нашим временам,
более чем пенсионный возраст. Наполеон называл Блюхера "старым чертом".
Естественно, когда-то "черт" был молодым. Он начал военную службу совсем юным
"чертенком" — в 14 лет, когда Наполеона еще не было на свете. В восемнадцать
Блюхер стал офицером в армии Фридриха Великого. И с тех пор до самой смерти
служил Пруссии. Ему не всегда везло. Он терпел поражения. Но это не смущало
его. Накануне битвы при Ватерлоо Блюхер потерпел поражение от Наполеона при
Линьи. Фельдмаршала придавило лошадью, вместе с которой он свалился. Он
повредил ногу и потерял сознание. Все подумали, что "старый черт" мертв. Но дед
пришел в себя, вылез из-под коня, привел в порядок армию и повел ее на звук
орудийных выстрелов. Он явился к Ватерлоо в самый нужный момент — когда
англичане уже собирались удирать под ударами французов. Какой пример любому
нашему пенсионеру! Как он наглядно доказывает, что даже на восьмом десятке
жизнь только начинается. Ведь именно в этом возрасте Блюхер вошел в историю как
победитель Наполеона. Кто бы его знал, если бы фельдмаршал не нашел в себе сил
выбраться из-под коня? А на дворе было начало XIX столетия — без антибиотиков,
пенсий и бесплатных кабинетов урологов в районных поликлиниках, куда
выстраиваются сегодня очереди из наших потенциальных "блюхеров", не
подозревающих, что они еще "о-го-го"! В самом, что ни на есть, призывном
возрасте!

ФРАНЦУЗСКИЙ И НЕМЕЦКИЙ "ВАРЯГИ" — КОРАБЛИ "МСТИТЕЛЬ" И "ЭМДЕН" 

Крейсер "Эмден". Германский рейдер не только повторил подвиг "Варяга", но и
превзошел его

Мне приятно вспоминать разбойничьи подвиги запорожских казаков. То, что они
брали штурмом Кафу и высаживались в Стамбуле, наполняет меня гордостью. Хоть
базар турецкий ограбили, и то хорошо! Но не стоит видеть в этом нечто
исключительное. С точки зрения любой, даже не очень великой морской державы,
запорожцы — просто мальки. В лучшие времена Венецию населяло всего 120 тысяч
жителей. Но этот маленький город в XIII—XV веках держал под контролем все
Средиземное море! И сегодня в морском музее Венеции можно провести целый день.
Четыре его этажа доверху наполнены венецианской морской славой — модели
кораблей, пушки, приборы, формы, карты сражений и даже галера в натуральную
величину, на которой правитель Венеции осуществлял обряд "венчания с морем".
Увидев все это, я подумал: из чего можно соорудить украинский морской музей?
Какие экспонаты в него войдут: заржавевшая подводная лодка "Запорожье" и
шаровары гетмана Сагайдачного? Да и те еще нужно или найти, или подделать.

О какой "самостийной" морской славе мы можем говорить, если даже подвиг
знаменитого "Варяга", в экипаже которого среди прочих были и украинцы, не
является чем-то уникальным. Еще в 1794 году французский линейный корабль
"Мститель" в одиночку вступил в бой с британской эскадрой. Потеряв все мачты и
треть экипажа, французы прибили флаг просто к корме и отказались сдаться,
предпочтя смерть плену. Триста пятьдесят матросов и офицеров погибли, навсегда
став примером для подражания французскому флоту.

Германский крейсер "Эмден" в 1914 году считался устаревшим кораблем. Но он
умудрился за два месяца рейдерства в Индийском океане потопить несколько
десятков английских торговых судов, русский крейсер "Жемчуг", французский
миноносец "Мушкет", осуществить бомбардировку Мадраса и героически погибнуть в
бою с новейшим австралийским крейсером "Сидней". Несколько моряков с "Эмдена"
во главе с лейтенантом Мюке захватили старую полусгнившую яхту и через весь
Индийский океан добрались до Аравии, а там, пересев на верблюдов, пробились
через банды арабских племен на родину. Вот это были "пираты"!

БЕЗ ИМПЕРСКОГО НАСЛЕДИЯ МЫ — НИЧТО 

Памятник князю Владимиру. Империя начиналась в Киеве!!!

Если Украина откажется от своего славного прошлого в составе Российской империи
и СССР, она окажется ни с чем. Ведь нельзя же, забыв о маршалах и великих
ученых, всерьез поклониться австрийским капралам. Нам говорят, что это было
неэстетичное, "неевропейское" прошлое. Но я отвечу словами русского публициста
XIX столетия князя Кельсиева: "Что же греха таить? Мы не красивы и не станем
перед прочими славянами хвастаться изяществом и своею особенною гуманностью. Но
у нас есть одно качество, которого ни у южноруссов, ни у прочих славянских
племен решительно не хватает. Качество это весьма не мудрое и, пожалуй, не
завидное: мы — сила...

При всех наших недостатках… и при всем том, что теперь так некрасиво деется у
нас, на матери, на Святой Руси, у нас дело все-таки идет вперед и вперед,
все-таки, спотыкаясь и сворачивая вправо и влево, русское государство не
чахнет, не сохнет, а развивается со дня на день. Тяжел наш путь, трудны наши
шаги, не легка наша борьба, но мы идем и идем так, как шли пятьсот лет тому
назад наши предки во времена татарщины; мы идем и мы выйдем, а то, что мы
выйдем, не только мы знаем сами, но это знает и Запад, который от каждого
нашего шага вперед, шага неслышного, молчаливого, не сопровождавшегося никакими
трезвонами, никаким благовествованием, приходит в ужас, в негодование и
становится в недоумении перед этой страшной силой, которую мы собою
представляем. Южноруссы, как и все братья славяне, могут нам говорить, что мы
идем не изящно, что наш шаг неуклюж, что наши пальцы заскорузлы, что на ногах
наших мозоли, что на спинах наших до сих пор еще свежи следы старинного кнута,
но мы идем, идем и еще раз идем. Землю пашут железной сохой, соха из красного
дерева никуда годиться не будет. Чтоб вбить гвоздь, нужен молоток железный, и
такие молотки только кузнецы делают, а никак не ювелиры. Мы создаем
государство, мы строим дома из кирпича, а не деревянные.

Изо всего славянства одни мы, великорусы, сумели выступить вперед, правдой и
неправдой сложили свое государство, с Новгородом обошлись невежливо, со Псковом
поступили мы грубо, но та дикая сила, которая бродит в нас, раз нас вывезла из
татарского плена, другой раз вывезла нас из плена немецкого, а третий раз
вывезет нас из плена европейского".

Чтобы говорить с такой прямотой, нужно быть сильным. А мир ценит только силу. И
только сильным нужно подражать. Иначе всю жизнь вы проведете в ментальном
схроне, задыхаясь от собственных выделений.

Олесь Бузина, 1 октября 2010 года
