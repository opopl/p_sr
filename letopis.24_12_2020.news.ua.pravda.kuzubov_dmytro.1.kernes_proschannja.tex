% vim: keymap=russian-jcukenwin
%%beginhead 
 
%%file 24_12_2020.news.ua.pravda.kuzubov_dmytro.1.kernes_proschannja
%%parent 24_12_2020
 
%%url https://www.pravda.com.ua/articles/2020/12/24/7277992/
 
%%author 
%%author_id kuzubov_dmytro
%%author_url 
 
%%tags kernes_gennadii,death
%%title 110 тисяч осіб, вантажівки з квітами і незапланований феєрверк. Як Харків прощався з Кернесом
 
%%endhead 
 
\subsection{110 тисяч осіб, вантажівки з квітами і незапланований феєрверк. Як Харків прощався з Кернесом}
\label{sec:24_12_2020.news.ua.pravda.kuzubov_dmytro.1.kernes_proschannja}
\Purl{https://www.pravda.com.ua/articles/2020/12/24/7277992/}
\ifcmt
  author_begin
   author_id kuzubov_dmytro
  author_end
\fi

\index[names.rus]{Кернес, Геннадий!Мэр Харькова! прощание, УП, 24.12.2020}

В ночь с 16 на 17 декабря в Берлине от последствий коронавируса умер мэр
Харькова Геннадий Кернес.

За три месяца до этого его госпитализировали в клинику "Шарите". Это не
помешало ему, ни разу не появившись перед избирателями, в третий раз выиграть
выборы городского головы, а его партии — получить большинство в горсовете и
облсовете.

В среду, 23 декабря, мэра похоронили в Харькове. Несмотря на мороз и пандемию,
попрощаться с ним, по данным полиции, пришли 110 тысяч человек.

Пышные, в стиле брежневского застоя похороны стали последним подарком
многолетнему электорату Кернеса — избирателям среднего и старшего возраста с
советским бэкграундом и фантомными болями по "союзу нерушимому". При этом,
проститься с ним пришла и молодежь, некоторые харьковчане были семьями с
маленькими детьми на руках.

"УП" рассказывает, что происходило в последние месяцы эпохи Кернеса и как мэра
провожали в последний путь.

\subsubsection{Дистанционная победа на выборах и смерть в Берлине}

Геннадия Кернеса последний раз видели на публике 23 августа, в День Харькова. 

Мэр открывал на площади Свободы самый большой в Украине "сухой фонтан". До
осени 2014-го на этом месте стоял памятник Ленину, за снос которого Кернес
когда-то обещал "сломать две руки и две ноги" "свободовцу" Игорю Мирошниченко.

Во время открытия фонтана мэр был без маски. В начале пандемии и объявленных
Киевом ограничений городской голова и вовсе выступал ковид-диссидентом —
отказывался закрывать метро и школы. Однако после угрозы уголовной
ответственности пошел на попятную. 

\ifcmt
pic https://img.pravda.com/images/doc/0/3/037c1ec-----1.jpg
caption Кернес на своем последнем публичном мероприятии, Фото: Суспільне Харків
\fi

15 сентября спикер Харьковского горсовета Юрий Сидоренко подтвердил, что Кернес
заболел коронавирусом и находится на лечении в 17-й городской больнице в
стабильно тяжелом состоянии. По информации УП, коронавирусом мэр заразился от
своей сиделки.

На следующий день специальный самолет с Кернесом на борту вылетел из
харьковского аэропорта. 17 сентября в 00:07 по киевскому времени борт
приземлился в Берлине. Мэра доставили в Германию в противоковидной капсуле, во
время полета он был подключен к аппарату высокопотоковой подачи кислорода. 

Еще через неделю, 24 сентября, пасынок Кернеса Родион Гайсинский опубликовал в
Instagram фотографию медицинского заключения из "Шарите". В справке с подписью
директора было написано, что пациент "прогрессирует каждый день" и в клинике
"надеются на его полное выздоровление". 

\ifcmt
pic https://img.pravda.com/images/doc/a/1/a183345-----22.jpg
caption Справка из "Шарите", Фото: Павел Фукс/Telegram
\fi

2 октября друг Кернеса, бизнесмен Павел Фукс написал, что городской голова
получил отрицательный тест на коронавирус и он надеется, что "через неделю
Геннадий Адольфович всем передаст привет". 

Кернес так и не вернулся в Харьков в обещанный срок. За время его отсутствия
мэра неоднократно "хоронили" в соцсетях. И все эти месяцы его окружение
всячески пыталось доказать, что мэр идет на поправку и скоро вернется в
Харьков.

\ifcmt
pic https://img.pravda.com/images/doc/3/8/3844cc1-----3.jpg
caption Кернес и Терехов, Фото: Игорь Терехов/Facebook
\fi

10 ноября Фукс
опубликовал\Furl{https://www.pravda.com.ua/news/2020/11/10/7273101/} фото
Кернеса с пасынком Родионом Гайсинским — первое за время его отсутствия.

\ifcmt
pic https://img.pravda.com/images/doc/a/d/adae31f-----4.jpg
caption Кернес на лечении в клинике "Шарите" с Родионом Гайсинским, Фото: Павел Фукс/Telegram
\fi

В начале декабря Фукс анонсировал возвращение Кернеса в Харьков к концу года.
Спустя два дня состоялась первая сессия горсовета нового созыва. Кернеса на ней
не было, а глава Харьковской ТВК Ольга Мозговая сообщила, что ему не нужно
принимать присягу, потому что он "избран городским головой не в первый раз, уже
принимал присягу 24 ноября 2010 года, что подтверждается записью в трудовой
книжке". 

При этом новым секретарем горсовета избрали вице-мэра Игоря Терехова — его
кандидатуру якобы предложил Геннадий Кернес, прислав соответствующий документ с
электронной подписью. 

За кандидатуру Терехова, которого ранее также рассматривали на должность нового
губернатора области, проголосовали все партии, за исключением блока бывшего
губернатора Юлии Светличной: Блок Кернеса "Успешный Харьков", ОПЗЖ,
"Европейская солидарность", "Слуга народа", "Партия Шария". Так Терехов стал
исполняющим обязанности мэра.

10 декабря новоизбранный секретарь рассказал\Furl{https://suspilne.media/87193-do-kinca-roku-harkivska-miskrada-zberetsa-na-pozacergovu-sesiu-kernesa-cekaut/} о планах горсовета провести
внеочередную сессию до конца года, а будет ли присутствовать на ней Кернес,
зависит от "врачей и его самочувствия". При этом, обсуждая с журналистами
здоровье мэра, уповал на бога.

"Даст бог, до внеочередной сессии вернется", — сказал Терехов и упомянул бога
еще трижды.

Буквально на следующий день стало известно, что у Кернеса отказали обе почки.\Furl{https://www.pravda.com.ua/news/2020/12/11/7276646/} В
горсовете добавили, что избранный мэр находится на гемодиализе, но "сможет
выполнять свои обязательства".

А утром 17 декабря Фукс сообщил, что Кернес умер\Furl{https://www.pravda.com.ua/news/2020/12/17/7277247/} от последствий коронавируса.

"Последствия коронавирусной инфекции дали серьезные осложнения на работу
жизненно важных органов и систем его организма", — объяснил он.

Смерть Кернеса подтвердили его жена Оксана Гайсинская,\Furl{https://www.instagram.com/p/CI492ndJhSW/} на которую пять лет
назад он переписал свои активы, и пасынок Родион Гайсинский.\Furl{https://www.instagram.com/p/CI4952EBw0-/}

\ifcmt
pic https://img.pravda.com/images/doc/2/d/2d6de06-----5.jpg
cpx Геннадий Кернес и Оксана Гайсинская в парке Горького — одном из главных имиджевых проектов мэра. Реконструировали парк накануне Евро-2012 на кредитные средства. В конце сентября "Проминвестбанк" подал в суд о взыскании с участников сделки — парка Горького и харьковского метрополитена — 6,56 миллионов евро и 52,768 миллиона гривен задолженности по кредиту, Фото: Геннадий Кернес/Instagram
\fi

Экс-нардеп, бывший мэр и бывший губернатор Харьковской области Михаил Добкин
написал,\Furl{https://t.me/s/dobkinmm} что "давно простил" Кернеса. В последние годы бывшие соратники
практически не общались, а после болезни Кернеса Добкин даже собрался в мэры
Харькова. Но в итоге снял свою кандидатуру\Furl{https://www.pravda.com.ua/news/2020/10/4/7268743/} в пользу Кернеса. 

\ifcmt
pic https://img.pravda.com/images/doc/8/f/8f36f24-----6.jpg
caption Кернес во время лечения в Израиле вместе с Добкиным, май 2014-го, Фото: Геннадий Кернес/Instagram
\fi

Проигравший Кернесу на выборах Фельдман, с которым мэр конфликтовал из-за
самого крупного рынка Харькова Барабашово,\Furl{https://www.pravda.com.ua/articles/2020/09/25/7267089/} опубликовал его фотографию без
подписи.\Furl{https://www.facebook.com/pochta.feldman/posts/5386765764670700}

В Харькове объявили трехдневный траур\Furl{https://www.pravda.com.ua/news/2020/12/17/7277306/} с 17 по 19 декабря и отменили все
развлекательно-концертные и спортивные мероприятия в эти дни, а вскоре третий
день траура перенесли на день похорон. Для сравнения, когда в сентябре в
Харьковской области разбился самолет с курсантами и погибло 26 человек, траур
был однодневный.\Furl{https://www.pravda.com.ua/news/2020/09/26/7267767/}

В Киеве реакция на смерть многолетнего мэра Харькова была сдержанной. В офисе
президента отметили,\Furl{https://www.facebook.com/president.gov.ua/posts/2229212300556958} что "отношение к Геннадию Кернесу в нашем обществе не
было, да и не могло быть однозначным", и его жизненный путь "точно исключает
однозначность в его оценках".

"Однако у него было видение развития своего города, и именно как городской
голова он работал, вкладывая в Харьков все свои эмоции и компетенции. Люди это
ценили", — прокомментировали в ОП.

Спикер Верховной Рады Дмитрий Разумков выразил соболезнования семье,\Furl{https://www.city.kharkov.ua/uk/news/dmitro-razumkov-visloviv-spivchuttya-rodini-gennadiya-kernesa-46260.html} в
Верховной Раде память Кернеса почтили минутой молчания.\Furl{https://www.youtube.com/watch?v=dLPb7MYNc5U} Зеленский лично на
смерть мэра никак не отреагировал.

\ifcmt
pic https://img.pravda.com/images/doc/0/c/0cc2733-----7.jpg
caption Кернес c будущим президентом Зеленским, ноябрь 2013-го, Фото: Геннадий Кернес/Instagram
\fi

\subsubsection{Похороны по сценарию Кушнарева }

Накануне похорон в соцсетях появилась информация, что руководители коммунальных
предприятий города собирают на прощание с Кернесом бюджетников. Игорь Терехов
опроверг эту информацию, назвав ее "кощунственной". 

"Любовь харьковчан к этому человеку такая безграничная, что не нужно никого
просить даже выйти попрощаться, люди сами хотят, люди сами придут", — заверил
он.

Сценарий похорон Кернеса прописали практически идентично похоронам другого
известного харьковского политика — бывшего губернатора области и нардепа от
"Партии регионов" Евгения Кушнарева, который погиб в 2007-м в результате
ранения на охоте. На его похоронах присутствовал весь тогдашний политический
истеблишмент, в том числе и Кернес, тогда — секретарь горсовета.

\ifcmt
pic https://img.pravda.com/images/doc/6/d/6d98d30-----8.jpg
caption Кернес и нардеп Дмитрий Святаш на похоронах Кушнарева, Фото: УНИАН
\fi

Всего за три года до этого Кернес был ярым антагонистом Кушнарева. В 2004-м, в
разгар Оранжевой революции, он поддерживал Виктора Ющенко и вместе с Арсеном
Аваковым скандировал на площади Свободы "Кушнарева геть!". 

\video{https://youtu.be/NGOXCYb5oaY}

\begin{leftbar}
  \begingroup
    \em\Large\bfseries\color{blue}
    После ранения в 2014-м Кернес утверждал в суде, что в покушении на него "торчат уши Авакова"
  \endgroup
\end{leftbar}

Однако уже в 2005-м стал симпатиком Виктора Януковича и вступил в "Партию
регионов", став однопартийцем Кушнарева. 

\ifcmt
pic https://img.pravda.com/images/doc/5/a/5a39f73-----9.jpg
cpx Незадолго до начала Евромайдана Кернес вместе с Добкиным поехал в Грозный на День рождения Рамзана Кадырова. В 2018-м на вопрос о дружбе с Кадыровым он ответил "Я не отказываюсь ни от каких своих дружеских отношений с теми, кто для вас может быть плохим героем", Фото: Геннадий Кернес/Instagram
\fi

Во время Евромайдана Кернес до последнего был верен Януковичу, однако после его
побега в Ростов сказал, что "Виктор Янукович — уже история", а "врагов надо
делать друзьями". 

\video{https://youtu.be/hwKs5K2pMbg}

\begin{leftbar}
  \begingroup
    \em\Large\bfseries\color{blue} После победы Евромайдана Кернес уехал в
        Россию,\Furl{https://www.pravda.com.ua/rus/news/2014/02/23/7015968/} а затем улетел в Женеву на встречу с Коломойским. После этого
        вернулся в Харьков и поддержал уже новую власть
  \endgroup
\end{leftbar}

Кернес умел найти общий язык с любой центральной властью. Петра Порошенко во
время его пребывания в должности главы государства он также поддерживал и
открыто агитировал за него на выборах.\Furl{https://www.mediaport.ua/kernes-pered-drugim-turom-pidtrimuyu-togo-kogo-pidtrimuvav} Однако после победы Владимира Зеленского
заявил о поддержке новоизбранного президента.

Как и в день похорон Кушнарева,\Furl{https://www.mediaport.ua/news/city/38439/segodnya_pohoronyi_evgeniya_kushnareva} для Кернеса рано утром назначили отпевание в
Благовещенском соборе, затем — прощание в фойе Харьковского национального
театра оперы и балета, следом — похороны на аллее почетных граждан кладбища №2,
рядом с могилой Кушнарева.

И.о. мэра пообещал, что во время всех процедур будут соблюдены все карантинные
меры.

"Обязательно будет соблюден масочный режим, — пообещал Терехов. — Будем просить
о соблюдении социальной дистанции. […] Любой харьковчанин, любой приезжий в
город Харьков, тот, кто хочет попрощаться, становится в живую очередь и
прощается с Геннадием Адольфовичем".

За несколько дней возле Оперного театра и кладбища начали усиленно работать
коммунальщики: чистить дороги, срезать ветки деревьев и красить лавочки.

\ifcmt
pic https://img.pravda.com/images/doc/1/2/1255b5d-----10.jpg
cpx Косметический ремонт лавочек возле Харьковского национального театра оперы и балета за несколько дней до прощания с Кернесом, Фото: h\_harkov
\fi

\subsubsection{Красная дорожка, венок от \enquote{новой} дочери и километровая очередь}

Масштаб и торжественность прощания с Кернесом напоминает документальные хроники
о прощании с вождями давно ушедшей эпохи. 

В день похорон Центральная улица города Сумская и часть центра перекрыта, а к
Харьковскому национальному театру оперы и балета можно проехать только по
пропускам. 

По периметру театра дежурят нацгвардейцы и полицейские. Для обеспечения порядка
на церемонии прощания привлекли 284 полицейских и 150 бойцов Нацгвардии, и еще
до 130 полицейских — на похоронах. 

На фасаде театра без привычных афиш висит огромный портрет мэра с его лозунгом
"Любить Харьков — работать для людей!". Слева и справа — большие экраны и
баннеры с изображением роз и свечей.

\ifcmt
pic https://img.pravda.com/images/doc/b/c/bc41930-bbk-5800.jpg
cpx Фото: Сергей Бобок
\fi

За полчаса до начала церемонии прощания к тыльной стороне театра подъезжает
кортеж с катафалком. Из динамиков начинает играть траурная музыка,
подтягиваются первые люди с цветами.

Через несколько минуту у входа останавливается еще один кортеж. Из него выходят
нардепы Евгений Мураев, Вадим Новинский, за ними — человек с охапкой роз.
Появляется и Михаил Добкин в темных очках Porsche.

"Я православный человек, когда кто-то уходит из близких или тех людей, с
которыми ты имел отношения… — говорит он журналистам. — Конечно же, я где-то
что-то совершал, что требовало извинения. Я не смог извиниться при жизни.
Извинился после смерти. Думаю, что он меня простил. Надеюсь, что простил".

Желающие проститься с городским головой поднимаются по ступенькам к главному
входу.

\ifcmt
pic https://img.pravda.com/images/doc/3/5/353251e-bbk-5754.jpg
cpx Фото: Сергей Бобок
\fi

"Там наша машина стоит, Maybach увидишь, не ошибешься", — инструктирует кого-то
по телефону женщина в длинной шубе.

На входе в театр бабушки выдают маски и антисептики. На ступеньках —
многочисленные венки: от мэра Днепра Бориса Филатова, от нового губернатора
Айны Тимчук, депутата Бориса Колесникова, экс-премьера Владимира Гройсмана,
бизнесмена Александра Ярославского. С последним Кернес долго не общался, но в
прошлом году помирился и даже отмечал вместе с ним Новый год. Чуть выше — от
коммунальных предприятий и отеля Кернеса National, в котором он жил.\Furl{https://www.pravda.com.ua/rus/articles/2016/06/10/7111399/}

\ifcmt
pic https://img.pravda.com/images/doc/7/a/7a365ef-bbk-5953.jpg
cpx Фото: Сергей Бобок
\fi

У входа в партер — гроб с телом мэра. От входа к нему по ступенькам расстелена
красная ковровая дорожка. На стене между красных портьер висит портрет Кернеса. 

\ifcmt
pic https://img.pravda.com/images/doc/b/1/b1ba5e9-bbk-5459.jpg
cpx Фото: Сергей Бобок
\fi

Оркестр играет траурную музыку. Прощающиеся с городским головой по очереди
подходят к гробу, кладут цветы и крестятся.

\ifcmt
pic https://img.pravda.com/images/doc/3/f/3fceaea-bbk-5596.jpg
cpx Фото: Сергей Бобок
\fi

У гроба и возле ступенек стоят женщины и мужчины с красными повязками на руках
— так маркируют тех, кто задействован в организации. Среди них — глава
Новобаварского района города Татьяна Цибульник, у которой два года назад нашли
незадекларированные драгоценности на 600 тысяч гривен,\Furl{https://2day.kh.ua/glava-administratsii-rajona-harkova-solgala-v-deklaratsii-prokuratura-otkryla-ugolovnoe-delo} близнецы-охранники
Кернеса и волонтеры "из народа". Заместитель городского головы, дочь покойного
депутата "Партии регионов" Татьяна Чечетова-Терашвили раздает волонтерам
указания.

\ifcmt
pic https://img.pravda.com/images/doc/7/0/705a0e0-bbk-5645.jpg
cpx Фото: Сергей Бобок
\fi

У гроба — венки от приближенных: горсовета, облсовета, и.о. мэра Игоря
Терехова, семьи Павла Фукса, родственников. Взгляд останавливается на венке
"любимому папе от Сонечки" — накануне похорон у Кернеса "нашлась" 11-летняя
дочь, сейчас она живет в Лондоне вместе со своей матерью. Журналисты между
собой шепчутся, что в прошлом году ее видели рядом с городским головой на
мемориале, но тогда ее запретили снимать.

Слева от гроба под инфракрасным обогревателем разговаривают Терехов, Новинский,
Мураев, Фукс и Виталий Кличко. Вскоре к ним подходит сын Кернеса,
новоиспеченный депутат облсовета Кирилл Кернес.

"Большая часть киевлян и харьковчан показывают, как они относятся к мэру,
который набрал достаточное количество голосов, учитывая, что в Харькове порядок
и в Харькове есть, с чего брать пример, — заверяет журналистов Кличко. 

Я помню, как мы [с Кернесом] много раз ссорились о том, кто построит лучшие
центры административных услуг — Харьков или Киев. Это такие, знаете,
соревнования, кто сможет предоставить лучший сервис для жителей наших городов.
Это такие добрые вещи, которые должны быть между мэрами городов".  

\ifcmt
pic https://img.pravda.com/images/doc/9/e/9e39241-bbk-5478.jpg
cpx Фото: Сергей Бобок
\fi

Ближе к 9 у гроба появляются еще один сын мэра Даниил Привалов, пасынок Родион
Гайсинский и Оксана Гайсинская. На голове у нее — черный платок. Родион — в
перчатках. У гроба в несколько рядов стоят стулья. Семья мэра занимает первый и
второй ряды, Кличко усаживается в четвертый, Фукс продолжает стоять.

Оркестр играет все торжественней. Сидоренко раздает новоприбывшим VIP-гостям
маски. Цветов становится все больше. Женщина в перчатках скептически смотрит на
усыпанный листьями от букетов пол и пытается пройти через ограждение.

"Что вы хотите?" — спрашивает ее волонтер

"Листья пособирать!" — робко отвечает она.

"Собирайте!".

Театр все больше становится похожим на мавзолей. Желающие проститься с Кернесом
все подходят и подходят. Бабушка с палочкой с трудом взбирается по ступенькам,
кладет гвоздики и кланяется семье мэра. К гробу подвозят человека на коляске.
Неожиданно он встает с нее и трясущимися руками возлагает на море цветов свой
букет ромашек.

Большая часть людей заходит проститься с мэром с центрального входа, с бокового
тянутся делегации чиновников. Пока по ступенькам поднимаются депутаты горсовета
и директоры департаментов, семья мэра исчезает в кулуарах театра. Вскоре Кирилл
Кернес возвращается и остается один у гроба отца. Оркестр замолкает. В углу
рядом с телом Кернеса усаживаются четверо оркестрантов и затягивают грустную
мелодию. Девушка в очереди снимает маску и промокает ею глаза.

В боковых дверях появляется экс-глава облсовета Сергей Чернов и другие
чиновники областного управления. Очередь от главного входа становится все
плотнее, у гроба возникает пробка. 

С Кернесом прощаются полицейские, католические и православные священники, люди
в военной форме. Пожилой мужчина с палкой долго крестится и кланяется гробу, а
затем поднимает в воздух кулак. Оркестр играет "Реквием" Моцарта.

Ближе к 11 в фойе появляются сопредседатель ОПЗЖ Юрий Бойко и мэр Одессы
Геннадий Труханов.

"У этого человека было много чему учиться, особенно его мужскому характеру и
воле, воле к работе, к жизни, — утверждает Труханов. — Практически год назад, в
январе, у меня был день рождения и Геннадий [Кернес] прилетел ко мне на день
рождения. Он нашел в себе силы и возможность прилететь. Эти поступки говорят о
том, что это человек с большим сердцем, подтверждал это своими делами".

Волонтеры и охранники мэра то и дело забирают цветы, которые приносят и
приносят, на улицу. Там их сгружают в грузовики и увозят. К театру стоит
очередь примерно в 700 метров, которая тянется через сад Шевченко и
заканчивается почти у самой станции метро Университет. У дороги припаркованы
автозаки.

\ifcmt
pic https://img.pravda.com/images/doc/3/2/326758f-bbk-5898.jpg
cpx Фото: Сергей Бобок
\fi

В полдень из храма Жен-Мироносиц напротив ХНАТОБа, который построил несмотря на
протесты местных жителей Кернес, а проинвестировал Павел Фукс,\Furl{https://www.city.kharkov.ua/ru/news/gennadiy-kernes-u-zvanni-pochesniy-gromadyanin-kharkova-nemae-politiki-24944.html} раздается
колокольный звон. 

На двух экранах крутят трансляцию церемонии. Периодически ее перерывают кадры с
живым Кернесом. В одном из роликов ему звонит пенсионерка и просит помочь. Он
записывает номер и обещает сделать все, что в его силах. Ощущение от общей
картины абсолютно сюрреалистичное.

В очереди девушка с пышным букетом роз рыдает навзрыд, наверх по ступенькам
несут венок от Александра Грановского, который в прошлом году баллотировался на
досрочных парламентских выборах, но не прошел\Furl{https://www.pravda.com.ua/news/2019/07/21/7221546/} и уехал в Израиль. Возложив
цветы, друзья мэра и чиновники тянутся к выходу. Но простых симпатиков здесь
намного больше. 

\ifcmt
pic https://img.pravda.com/images/doc/5/7/573b6fb-bbk-5779.jpg
cpx Фото: Сергей Бобок
\fi

Лейла раньше была администратором хостела, сейчас работает в охране. Проводить
Кернеса она пришла "даже несмотря на мигрень", простояла час в очереди.

"Для нас это очень близкий родной человек, — говорит она. — За простыми
неродными людьми так не ходят. Говорят: "Кто будет лучшим мэром теперь?" Я
думаю, что лучше Кернеса только Кернес. В такой мере любить свой город, своих
родных харьковчан — а он нас считал своими детьми — я думаю, никто не сможет".

Бизнесмен Владимир не знал Кернеса лично, но пришел "отдать ему честь".

"Я, к сожалению, не знал Геннадия Адольфовича как близкого, но сколько он
сделал для города! Из последних сил боролся, старался, — утверждает Владимир. —
[Главное его достижение —] это город Харьков. Чистота, порядок, инфраструктура,
развитие, парк Горького, все остальные городские объекты".

\subsubsection{\enquote{Сталина хоронили с помпой, а потом развенчали культ}}

На улице напротив театра  припарковано несколько полицейских машин и автобусов
для журналистов. 

Мимо одного из автобусов проходит мужчина с зеркальным фотоаппаратом и
мальчиком со штативом. Это волонтер Борис Редин. В прошлом году он "воевал" с
Кернесом за палатку участников Евромайдана на площади Свободы. Ее установили в
2014-м, а в прошлом году мэр назвал ее "грязной" и предложил снести. Однако
активистам удалось отстоять палатку.

В сам театр Борис не заходил, пришел посмотреть на происходящее со стороны, с
собой взял сына Вовчика.

"Сталина хоронили с помпой, даже в Мавзолей положили, а потом развенчали культ,
— говорит он. — Сейчас похороны, ничего нельзя говорить плохого. Но пройдет
время, повылазят все проблемы, которых намного больше, чем украшений по
Харькову. Кто-то обязательно посчитает, сколько что стоит, почему не сделана
коммунальная сфера — это все вспомнят. И тогда заговорят по-другому. Но это
будет потом". 

\ifcmt
pic https://img.pravda.com/images/doc/8/a/8ad3721-bbk-5792.jpg
cpx Фото: Сергей Бобок
\fi

В час дня, когда автобус с журналистами паркуется у входа кладбища, начинает
идти снег. У входа дежурит полиция и пара бабушек с гвоздиками. 

"Без Кернеса Харьков осиротел", — жалуется одна из них другой.

"Вы не сотрудники СБУ, не пресса, не народные депутаты?" — спрашивает у бабушек
полицейский и после того, как они отрицательно кивают, наставляет: "Тогда
отойдите".

К кладбищу подъезжает грузовик с полным кузовом цветов и исчезает за поворотом.

"А здесь можно стоять?" — не унимаются бабушки.

"Пока постойте там, чтобы не было столпотворения", — указывает полицейский на
соседнюю сторону улицы.

Там уже выстроилась толпа зевак — преимущественно пенсионеров. Среди них стоит
депутат горсовета от ОПЗЖ Андрей Лесик. Четыре года назад Кернес с помощью
охраны выставил его из зала горсовета и исключил из своей фракции "За Харків,
за Відродження"\Furl{https://city.kharkov.ua/ru/news/andriy-lesik-viklyucheniy-iz-fraktsii-za-kharkiv-za-vidrodzhennya-v-miskradi-31749.html?fbclid=IwAR0LOFh2PheH_cq4KG65KuyZPWcEPp4nBpTJi6H5VQmPfL1HpvOxZ4-QOME} "за дешевый популизм и самопиар".

Около 14:00 к кладбищу начинают съезжаться чиновники. 

Подъезжает еще один грузовик с цветами. Ко входу приносят венок "Другу от Коли
Луганского". Внутрь пропускают человека в шинели с флейтой. У одного из
операторов громко падает штатив с камерой. 

Наконец подъезжает траурная процессия. Из катафалка выносят гроб.

"Давайте вшестером!" — предлагает один из охранников-близнецов и вместе с
братом берет гроб первым.

\ifcmt
pic https://img.pravda.com/images/doc/c/8/c869fe2-bbk-6058.jpg
cpx Фото: Сергей Бобок
\fi

Гроб ставят на постамент. Вокруг тут же образовывается толпа. 

"Мальчик, нельзя здесь стоять, здесь могила!" — строго делает женщина замечание
ребенку, ставшему на могильную плиту.

Траурная музыка смешивается с жужжанием дрона в небе.

\ifcmt
pic https://img.pravda.com/images/doc/0/7/07ce469-bbk-6194.jpg
cpx Фото: Сергей Бобок
\fi

Священники начинают кадить ладаном и нараспев читать молитву.

Внезапно не по сценарию в небе начинает взрываться фейерверк. Чуть позже в
полиции пообещают выяснить, кто это сделал. В горсовете же скажут,\Furl{https://suspilne.media/90700-u-miskradi-vidreaguvali-na-feerverk-pid-cas-pohovanna-kernesa/} что "люди
вероятно хотели продемонстрировать свое плохое отношение", но Кернес "достоин
салюта".

\ifcmt
pic https://img.pravda.com/images/doc/6/d/6deccc1-bbk-6840.jpg
cpx Фото: Сергей Бобок
\fi

К могиле подходит девушка с розовыми дредами. В этот момент ей звонят на
вайбер, она отвечает и начинает вести прямую трансляцию. В одной руке держит
телефон, другой — кидает песок на крышку гроба. 

"Геннадию Адольфовичу земля пухом! — рыдает она в трубку и наводит камеру вниз.
— Мама, это Геннадий Адольфович наш, его больше нет, вместо него никого нет и
не будет!".

На противоположной стороне от нее покачиваясь, истошно кричит женщина в шубе:
"Самый сильный, самый лучший, самый бесстрашный мэр!"

Через мгновение обеих выводит под руки полиция.

Могильщики начинают интенсивно работать лопатами и успевают засыпать яму до
наступления темноты.

\ifcmt
pic https://img.pravda.com/images/doc/d/5/d512120-bbk-7156.jpg
cpx Фото: Сергей Бобок
\fi

\subsubsection{Памятник Кернесу и туманная перспектива выборов}

Игорь Терехов пообещал увековечить память Кернеса — посмертно присвоить ему
звание почетного гражданина Харькова и поставить в его честь памятник. Кроме
того, назвать его именем улицу или проспект и одну из станций харьковского
метрополитена.

"Я считаю, что мы обязаны, чтобы в городе Харькове появился памятник Геннадию
Адольфовичу Кернесу, это наш святой долг, и я буду настаивать на этом, — заявил
он на брифинге 21 декабря. — Сегодня я дал задание нашим архитекторам, чтобы
они подготовили несколько локаций, где можно поставить памятник. Мы вынесем
этот вопрос на широкое обсуждение".

\ifcmt
pic https://img.pravda.com/images/doc/0/3/0396adf-bbk-5436.jpg
cpx Фото: Сергей Бобок
\fi

На сайте президента Украины зарегистрировали петиции об установке памятника, а
также о присвоении и не присвоении Кернесу звания Героя Украины. На сайте
горсовета появилось 12 петиций об увековечении памяти Кернеса в городской
инфраструктуре.

\begin{leftbar}
  \begingroup
    \em\Large\bfseries\color{blue}
\video{https://youtu.be/YVPKVFqG4rc}

Будущий президент Владимир Зеленский пародирует ролик с Геннадием Кернесом, где
тот отчитывает директора КП "Харьковгорлифт" Анатолия Кандаурова, в
котором мэр называет чиновника "сучьим псом" и грозится "умножить его
на ноль"
  \endgroup
\end{leftbar}

По закону, в течение 15 дней после смерти городского головы, секретарь
горсовета должен обратиться в Верховную Раду о назначении внеочередных выборов.

В ответ Верховная Рада в течение 90 дней назначает их дату. Вместе с тем, глава
правления гражданской сети "Опора" Ольга Айвазовская отмечает, что этот вопрос
"может быть растянут во времени на годы, что демонстрировала неоднократная
практика"

Сразу после смерти Кернеса Терехов в лучших традициях эпохи советского застоя
пообещал "продолжить дело Геннадия Адольфовича". Учитывая, что при голосовании
за Терехова как за секретаря горсовета депутаты поддержали его почти
единогласно, вопрос, состоятся ли в Харькове новые выборы мэра, остается
открытым.

Если история чему-то и учит, так это тому, что обещания над гробом очередного
советского лидера были исключительно ритуальными и никого ни к чему не
обязывали. 

Дмитрий Кузубов, фотографии — Сергей Бобок
