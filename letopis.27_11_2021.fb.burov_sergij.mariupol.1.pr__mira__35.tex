%%beginhead 
 
%%file 27_11_2021.fb.burov_sergij.mariupol.1.pr__mira__35
%%parent 27_11_2021
 
%%url https://www.facebook.com/permalink.php?story_fbid=pfbid0ofaASshDR5RSkRKoma613YNAFPyPSzuSEEy2JCEE5nqHLP2GJiYBLW4sqgZZLjijl&id=100005851476252
 
%%author_id burov_sergij.mariupol
%%date 27_11_2021
 
%%tags mariupol,gorod,mariupol.istoria,istoria
%%title Пр. Мира, 35
 
%%endhead 

\subsection{Пр. Мира, 35}
\label{sec:27_11_2021.fb.burov_sergij.mariupol.1.pr__mira__35}

\Purl{https://www.facebook.com/permalink.php?story_fbid=pfbid0ofaASshDR5RSkRKoma613YNAFPyPSzuSEEy2JCEE5nqHLP2GJiYBLW4sqgZZLjijl&id=100005851476252}
\ifcmt
 author_begin
   author_id burov_sergij.mariupol
 author_end
\fi

Пр. Мира, 35. Зам. директора Краеведческого музея, историк Раиса Петровна Божко
установила в свое время, что дом в дореволюционное время принадлежал Федору
Ивановичу Данилову. Справа в первом этаже этого дома находилась Кондитерская и
булочная Ивана Яковлевича Марапульцева. От мариупольских старожилов довелось
слышать восторженные воспоминания о заведении Ивана Яковлевича. Говорили, что
вокруг распространилась необыкновенная \enquote{симфония} свежеиспеченного сдобного
теста, украшенного запахами ванили, корицы и других специй. 

\ii{27_11_2021.fb.burov_sergij.mariupol.1.pr__mira__35.pic.1}

Из собственных воспоминаний. В 50-е годы бывало, купишь билет на сеанс в
кинотеатр \enquote{Победа}, и бегом через дорогу в бывшее заведение Марапульцева, Там
торговали только что приготовленными прекрасными чебуреками (чир-чирами) и
другими изделиями из теста.

%\ii{27_11_2021.fb.burov_sergij.mariupol.1.pr__mira__35.cmt}
