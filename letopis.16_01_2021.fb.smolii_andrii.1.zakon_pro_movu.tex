% vim: keymap=russian-jcukenwin
%%beginhead 
 
%%file 16_01_2021.fb.smolii_andrii.1.zakon_pro_movu
%%parent 16_01_2021
 
%%url https://www.facebook.com/groups/promovugroup/permalink/900723007168228/
 
%%author 
%%author_id 
%%author_url 
 
%%tags 
%%title 
 
%%endhead 
\subsection{Сьогодні дійсно історична дата для української мови}
\Purl{https://www.facebook.com/groups/promovugroup/permalink/900723007168228/}
\ifcmt
  author_begin
   author_id smolij_andrij
  author_end
\fi

Сьогодні дійсно історична дата для української мови - 16 січня сфера громадяни
України отримують реальний механізм обслуговування українською мовою. 

Мова йде про припинення багаторічної дискримінації та відвертих знущань над
більшістю українців.

Проте, це також і далекоглядна, стратегічна мета - Україна в очах всього світу
й у самого українського суспільства має перестати бути «малоросією» чи «другою
московією». Нарешті тисячі іноземців також зможуть побачити, що це Незалежна
держава, зі своєю мовою і своїми законами, зі своєю ідентичністю та своїм
правом голосу в цивізізованому світі! 

Чомусь питання «якою мовою» не стоїть в Польщі чи Угорщині, у Німеччині чи
Франції, як і у більшості держав світу. Чомусь політики жодної держави не
кричать «какая разніца», телеканали не знущаються над єдиною державною мовою. 

У нас все навпаки. 30 років ми ходимо по колу і 30 років нам нав‘язують
наративи Кремля, які планомірно та впевнено знищують не лише мову, а й всю
країну! 

Тому, сьогодні історична дата! 

Мовний закон - це історичний Закон. 

Ми маємо відчути Україну! Світ має знати що це дійсно Україна, а не «малоросія»
чи Росія-2.

\begin{itemize}
\verb|Валентина Васильченко|

Ой, зачекайте радіти, зараз почнуть таку боротьбу, що мало не покажеться.
Закони якісь вносити і всілякі зміни до Конституції. Шо мало у
нас бужанських-дубінських?

\end{itemize}
