% vim: keymap=russian-jcukenwin
%%beginhead 
 
%%file 27_09_2021.fb.ukrainian_people_magazine.1.uschemlenie_mova_jazyk
%%parent 27_09_2021
 
%%url https://www.facebook.com/UkrainianPeopleMagazines/posts/3043542192633938
 
%%author_id ukrainian_people_magazine
%%date 
 
%%tags jazyk,kiev,mova,sssr,ukraina
%%title Розказують байки, як утискають у нас росіян. Я ж розповім, як це роблять з українцями
 
%%endhead 
 
\subsection{Розказують байки, як утискають у нас росіян. Я ж розповім, як це роблять з українцями}
\label{sec:27_09_2021.fb.ukrainian_people_magazine.1.uschemlenie_mova_jazyk}
 
\Purl{https://www.facebook.com/UkrainianPeopleMagazines/posts/3043542192633938}
\ifcmt
 author_begin
   author_id ukrainian_people_magazine
 author_end
\fi

Є такий хороший старий анекдот.

Іде бабця з онуком біля пам’ятника Богданові Хмельницькому.

Внук запитує:

«Бабушка, скажи, пажалуйста, а што ето за дядя на лошаді?» – «Панімаєш, внучєк,
здєсь кагда–то жилі украінци, ето бил іх вождь».

Ось так мені живеться…

Відомий кіноактор Іван Гаврилюк, про своє життя у Києві написав таке:
«Найбільший страх у столиці України – російська мова на кожному кроці. Я мав
розмови з політв’язнями, які просиділи 20–30 років. Вони казали: «Іване, до
концтабору звикаєш». До цього морального концтабору я звикнути не можу…».

\ifcmt
  ig https://scontent-yyz1-1.xx.fbcdn.net/v/t1.6435-9/243163326_3043542139300610_2166990475075222506_n.jpg?_nc_cat=109&_nc_rgb565=1&ccb=1-5&_nc_sid=730e14&_nc_ohc=GHwUmaUiFuwAX-zm7IK&_nc_ht=scontent-yyz1-1.xx&oh=91c4ad45e250ef5d138e391818a01f4f&oe=6178C896
  @width 0.4
  %@wrap \parpic[r]
  @wrap \InsertBoxR{0}
\fi

Розказують байки, як утискають у нас росіян. Я ж розповім, як це роблять з
українцями.

31 січня 1967 року. Я на головній пошті столиці посилаю мамі телеграму до
Львова. Пишу ж, звісно, рідною мовою. Дівчина за склом кидає мені назад і каже
буквально таке: «Ви ваабщє можетє напісать на нармальном язикє?» Це в Україні
мені говорить якась шмаркачка!

Чим прекрасна юність – ти не оглядаєшся і не думаєш про наслідки. Я лівша,
розбиваю рукою скло і спокійно виходжу. Ніхто не наздоганяв...

Весна 1997–го. Минуло тридцять років. Знову ж Київ. У аптеку на
Червоноармійській, де я купував ліки для Батька, зайшов полковник української
армії. Російською мовою він задав декілька запитань, на що йому відповіли
лагідною українською. Тоді він сказав дослівно те, що й дівчинка на пошті: «А
ви ваабщє можетє разгаваривать на нормальном языке?» Я бив цього полковника, як
гамана! Розтоптав його мобільний телефон і пішов. Ніхто не наздоганяв...

Ось так я виживаю. Я мав розмови з політв’язнями, які просиділи 20–30 років.

Вони казали: «Іване, до концтабору звикаєш».

До цього морального концтабору я звикнути не можу.

Немає такого дня, щоб я у Києві з приводу своєї західноукраїнської мови не
відчув ворожого чи негативного ставлення.

На кіностудії імені Довженка колись мене не затверджували на ролі через моє
«антісавєтскоє ліцо».

Я себе відчуваю одним із найкращих представників української національної
меншості у місті Києві.»

Лора Підгірна

На фото - український митець, народний артист України

Іван Гаврилюк, актор з "антісовєцкім ліцом"

\ii{27_09_2021.fb.ukrainian_people_magazine.1.uschemlenie_mova_jazyk.cmt}
