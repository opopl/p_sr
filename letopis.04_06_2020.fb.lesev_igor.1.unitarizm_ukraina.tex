% vim: keymap=russian-jcukenwin
%%beginhead 
 
%%file 04_06_2020.fb.lesev_igor.1.unitarizm_ukraina
%%parent 04_06_2020
 
%%url https://www.facebook.com/permalink.php?story_fbid=3256085921089155&id=100000633379839
 
%%author_id lesev_igor
%%date 
 
%%tags centralizacia,decentralizacia,gosudarstvo,gosudarstvo.unitarnoje,regiony,strana,ukraina
%%title О приболевших унитаризмом
 
%%endhead 
 
\subsection{О приболевших унитаризмом}
\label{sec:04_06_2020.fb.lesev_igor.1.unitarizm_ukraina}
 
\Purl{https://www.facebook.com/permalink.php?story_fbid=3256085921089155&id=100000633379839}
\ifcmt
 author_begin
   author_id lesev_igor
 author_end
\fi

О приболевших унитаризмом

Некомфортно – это всегда об Украине. О нас всех, кто окопался по самые разные
стороны баррикад. И даже о нашей элитке в предвыборный период. Потому что из
зоны харчевого комфорта вылететь в нашей всегда и всем недовольной стране во
время выборов – это как два пальца в рот. Можно даже не в рот.

\ifcmt
  pic https://scontent-frt3-2.xx.fbcdn.net/v/t1.6435-9/101782345_3256085614422519_3800245569545306112_n.jpg?_nc_cat=101&ccb=1-5&_nc_sid=730e14&_nc_ohc=2H8kiepJpdcAX9cGvwg&_nc_ht=scontent-frt3-2.xx&oh=3e0f9d2fa68b9480b16027ae0dcf016b&oe=61A519C0
  @width 0.7
\fi

Украина некомфортна, потому что она – не страна возможностей. Украина – это
сначала Киев, и потому уже все остальное. При этом окно возможностей для Всего
Остального все больше сужается. Днепр – это уже не «Южмаш», а Николаев далеко
не город корабелов. Уникальность украинских регионов размывается. Крупные
города от мелких отличаются именно концентрацией населения. Там может быть
больше ТЦ и продуктовых магазинчиков. Но чем принципиально Полтава отличается
от Гайсина непосвященные хрен скажут.

И ведь причина некомфортности очевидна. У регионов нет субъектности. Ни
хозяйственной, ни политической. Вся Украина давно уже превратилась в придаток
Большого Киева. При этом политическая система в стране построена таким образом,
что она практически не предусматривает делегирование интересов регионов его
представителями.

Вот смотрим. Руководителей областей и районов местные не выбирают. Это прямые
назначенцы из Киева. Региональные элиты формируются только на уровне мэров и
местных советов. Но опять же, мэр – это хозяйственная должность, а депутаты
местных советов не имеют вообще никакой политической субъектности. В 100%
русскоязычном Запорожье местные депутаты не могут открыть ни одной
русскоязычной школы. А ведь это самый мелкий пример.

По сути, единственным неформальным институтом политического представительства
регионов в центре остаются мажоритарные депутаты. Представительство это во
многом условно, хотя бы потому что никакого правила оседлости для таких
депутатов нет. Сергей Березенко, который, возможно, никогда ранее не был в
Чернигове, в прошлом созыве Рады представлял именно округ этого города. И вот
сколько раз Березенко в 2020 году бывал в Чернигове?

И все же, при выборах депутатов-мажоритарщиков ориентация на свояков была
доминирующей. Чтобы избраться от Харькова или Одессы там нужно было если не
родиться, то хотя бы пожить.

И вот что проталкивают сейчас? Формирование нового парламента только по
партийным спискам. Типа, для усиления парламентаризма. Объяснять, почему так не
будет, смысла нет. Классический парламентаризм возможен только при
существовании классических партий. У нас таких нет. Ни одной. Партии в лучшем
случае у нас функционируют в складчину, как ОПЗЖ, а чаще это авторские проекты.
Чтобы попасть в проходной список такой партии, есть только один универсальный
способ – занести бабло. Все остальное – воля случая и стечение благоприятных
обстоятельств.

Соответственно, и представлять такие депутаты будут не какую-то часть общества,
а своего партийного лидера. Точно также, как губернатор области служит не
местной громаде, а президенту – лицу, которое определяет его судьбинушку.

И самое главное. Регионы окончательно потеряют свое представительство. Вот
давайте просто представим, что в Раде текущего созыва вдруг пропали все
депутаты-мажоритарщики. Остались только списочники. Я не поленился на сайте ЦИК
посмотреть их фактическое проживание и смотрите, какая вышла картинка.

Слуга народа – 124 списочника. 76 депутатов проживают в Киевском регионе (Киев
и область), 48 – из регионов. И сразу скажу, это самая чудесная картинка,
потому что у других совсем мрак.

Но и здесь ведь не все так просто. Во-первых, СН шла под брендом-запросом
«новые лица». Это была их главная фишка избирательной кампании. Они в списки не
взяли ни одного экс-депутата. Поэтому формирование списка, в том числе и с
региональным уклоном, было от скудности возможностей.

Во-вторых, никто ведь не ожидал, что только по списку удастся провести аж 124
депутата. Реально ж прогнозы были на 80-90 человек. А поэтому следует обратить
внимание лучше на эти цифры - 8/17/37. Из первой десятки 8 депутатов, из 20-ки
– 17 и из первых 50-ти – 37 нардепов с киевской пропиской. Например, от родной
для Зе Днепропетровской области по спискам СН в парламент попало только 5
депутатов. А от Киевского региона – 76. И это у Днепра еще третье по
численности представительство. Четыре области в вообще не представлены
списочниками, а остальные в большинстве имеют по 1-2 представителям. И во
многом за счет того, что после первой сотки список СН уже формировался по
принципу хрен пойми кого набрали, и набирали как раз из регионов.

Идем дальше, ОПЗЖ. И вот здесь самая характерная картинка. Кто такие ОПЗЖ в
нашем народном представлении? Бывшая Партия регионов, ее клон. Там 37
списочников. И 32 депутата проживают в Киевском регионе! Партия, которая типа
представляет юго-восток, обитает поголовно в Киеве.

«Батькивщина». Самые древние мастодонты. 24 списочника и 20 из Киевского
региона. И это условно 20. Тот же мультимиллионер Тарута по прописке типа
проживает в Мариуполе. Ребята, кто там из Мариуполя, вы часто с ним на улицах
вашего города пересекаетесь?

«Европейская солидарность». Еще одни очень интересные персонажи. История ЕС
проходила у нас на глазах. Это история персонального падения Петра Порошенко,
который в 14-м начал с победы в первом туре, а в 19-м победил только во
Львовской области. Хорошая такая история человека из Зазеркалья. Но именно
фактор Львовской области во многом предопределил формирование списка Порошенко.
И вроде бы мы должны тут увидеть тотальное представительство Галичины. Но что
мы видим на самом деле? У ЕС 23 списочника. 15 с киевской пропиской и только 5
со Львова. Львовская область голосует за Петра, а депутаты все равно живут в
Киеве.

Более того, там и «проживание» то еще. Парубий и Кубив оставили за собой только
номинально львовские прописки, хотя давно уже плотно осели в Киеве. Или Чийгоз.
У него «местом проживания» вообще указан крымский Бахчисарай. Т.е., по факту
вся порошенковская гвардия практически в полном составе на постоянной основе
живет в Киеве.

Ну и «Голос». Типа еще одна партия из Галичины. Но это опять именно «типа». 17
списочников и 14 с киевской пропиской. Со Львова только один.

И смотрите. Это ведь данные на начало избирательной кампании. Абсолютное
большинство новоизбранных депутатов, включая мажоритарщиков, через 5 лет после
первой каденции остаются в Киеве. У нас не только элита оторвана от народа. У
нас вся элита формируется на основе одного города и одного Большого Киевского
региона. Вся остальная страна – это обслуга этого региона.

Отсюда некомфортность и второсортность остальных регионов. Города с глубинной
историей, вроде Одессы, Харьков и Львова пока еще сохраняют свою идентичность.
Но их вторичность по отношению к Киеву только усиливается. В этом плане Украина
очень здорово наследуют Россию, где есть Большая Москва, чуть-чуть шарма Питера
и все остальное. Это херовая дорожка.

А в Украине эта дорожка еще дремучей. Потому что после окучивания Большого
Киева местная элитка становится компрадорской. У нее появляются офшорные счета,
недвижка за бугром и детки не в юа-вузах. Это абсолютно массовое явление для
любого украинского чиновника/политика, перешагнувшего личную планку в лям
зелени.

Поэтому и получается картина, когда все регионы в говне, потому что вся страна
работает на Киев. Но ведь и Киев по большому то счету тоже в говне – здесь ведь
не Париж и не Нью-Йорк. Киев тоже вахтово-транзитный город, только более
облагороженный. Но общая некомфортность страны основана именно на этой
сложившейся политической вертикали – регион – Киев – Забугор, где регионы
превратились в колониальные придатки столицы, а сама столица в виде Большого
Киева – колониальный придаток Забугорщины. И естественно, что в такой пищевой
цепочке политическая элита Украины будет зубами выгрызать сохранение
унитаризма.

\ii{04_06_2020.fb.lesev_igor.1.unitarizm_ukraina.cmt}
