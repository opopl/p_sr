% vim: keymap=russian-jcukenwin
%%beginhead 
 
%%file 25_11_2021.fb.kosarenko_stas.zaporozhie.1.ljudi
%%parent 25_11_2021
 
%%url https://www.facebook.com/s.v.kosarenko/posts/10227800995794456
 
%%author_id kosarenko_stas.zaporozhie
%%date 
 
%%tags chelovek,tipologia
%%title Люди делятся на доноров, акцепторов и паразитов
 
%%endhead 
 
\subsection{Люди делятся на доноров, акцепторов и паразитов}
\label{sec:25_11_2021.fb.kosarenko_stas.zaporozhie.1.ljudi}
 
\Purl{https://www.facebook.com/s.v.kosarenko/posts/10227800995794456}
\ifcmt
 author_begin
   author_id kosarenko_stas.zaporozhie
 author_end
\fi

люди делятся на доноров, акцепторов и паразитов. Донор - высшая ступень
развития личности - человек, достигший понимания того, что нужно помогать
другим и делиться с ближним, пусть даже и незнакомым. Не важно, какой у него
уровень благосостояния, важно, какое у него сердце. 50 гривень в помощь
беженцам от воспитательницы детского сада ценнее, чем фура продовольствия от
олигарха, а когда говоришь, оставь, мол, своему ребёнку конфет купи, обижается
и отвечает, что там нужнее. И в этот момент она и правда богаче всех, потому
что имеет возможность дарить. И где-нибудь среди горы сумок уставшая Аристова
вздохнёт и улыбнётся людской доброте.

\ifcmt
  ig https://scontent-frx5-1.xx.fbcdn.net/v/t39.30808-6/257484562_10227800995514449_1395985595181360042_n.jpg?_nc_cat=111&ccb=1-5&_nc_sid=730e14&_nc_ohc=e_60UJCWhy4AX_ZXu7P&_nc_ht=scontent-frx5-1.xx&oh=43a79ab0883052002281c45f5489a05d&oe=61AAA05C
  @width 0.4
  %@wrap \parpic[r]
  @wrap \InsertBoxR{0}
\fi

Доноры редко могут принимать сами, они до последнего упираются, но кто ж их в
коме да под капельницей спрашивать будет, хотя, как правило, это единственная
возможность всучить им хоть что-то для них лично.

Акцепторы - те, кто умеет принимать с благодарностью. Они иногда сами переходят
на ступень выше и при возможности становятся донорами. Благодарность в их
глазах, улыбки на лицах их детей - вознаграждение донору. Не земные поклоны, не
грамоты и медали и не лобзание рук. Улыбки. Остальное раздражает. Нормальному
человеку трудно выслушивать. какой он спаситель и как он тут вовремя со всем
вот этим. Он знает, что вовремя, он затем и приехал.

Не всегда дающий находит отклик, не всегда его действия понятны окружающим.
Очень часто на него ополчаются те, кто лишён радости быть донором, - паразиты.
Это люди, которые почему-то считают, что им все должны и, когда им приносят
кусок хлеба, требуют, чтобы он был правильно упакован. Они - самое тяжёлое
испытание для донора, именно из-за них можно лишиться веры в то, что дарить
правильно. Те, кто это испытание выдерживает, остаются на Пути навсегда.

Ещё одна разновидность паразитов - люди, которым всё это пофигу. Они живут в
своём благополучном мире на одном глобусе с дающими и нуждающимися, но не
замечают эти вспышки положительной энергии, хотя мир вокруг питается именно ей.
Именно она - основа для всех религий, то самое Добро, которое побеждает,
возможность для любого человека возвыситься, не унижая тех, кто рядом, не за
счёт кого-то, а над собой вчерашним. Стать лучше для себя самого. Ведь очень
часто никто, кроме донора не знает о том, что он дарит.

Делайте мир лучше, ему сейчас это как-то особенно нужно.

\ii{25_11_2021.fb.kosarenko_stas.zaporozhie.1.ljudi.cmt}
