% vim: keymap=russian-jcukenwin
%%beginhead 
 
%%file 22_07_2021.fb.rudomanov_aleksandr.1.kafedra_bilchenko_vata.cmt
%%parent 22_07_2021.fb.rudomanov_aleksandr.1.kafedra_bilchenko_vata
 
%%url 
 
%%author_id 
%%date 
 
%%tags 
%%title 
 
%%endhead 
\subsubsection{Коментарі}
\label{sec:22_07_2021.fb.rudomanov_aleksandr.1.kafedra_bilchenko_vata.cmt}

\begin{itemize} % {
\iusr{Сергей Дубильер}
Никаких шансов. Борцуны за права "притесняемых и обездоленных" категорий
населения мгновенно найдут, что неэффективный работник относится к категории
страдальцев от феминизма, эйджизма, расизма, гомофобии и др., др., др. И
увольнять его нельзя

\begin{itemize} % {
\iusr{Yurii Lev}
\textbf{Sergey Dubiljer} Роботу всім піда..сам! )

\iusr{Сергей Дубильер}
\textbf{Yurii Lev} "...і кожен з них народ, всі богоносці..." (с)  @igg{fbicon.wink} 
\end{itemize} % }

\iusr{Aleksei Parnowski}
Він не фактично радянський, а просто радянський. Рік ухвалення — 1971.

\iusr{Зоя Звиняцковская}

Чесно - туди цьому всьому і дорога. Бо якого Дерріди в педагогічному
університеті раптом з‘явився осередок вивчення філософії та культурології? Може
вони б зосередились на педагогіці, було б непогано? Зараз у нас в кожному
швацькому коледжі кафедра філософії та культурології - дуже зручно, фіг його
знає, що це таке, складай програму довільно і викладай що хочеш, а грошики зі
студентів капають, а дипломи справжні.


\iusr{Дмитрий Мукойда}

В Германии-Франции-Швеции тоже нельзя увольнять людей. На самом то деле есть
обратная проблема так называемый моббинг-буллинг-боссинг на работе. Одной из
форм является саботаж работы сотрудника тоесть создание условий его
немродуктивности. В Американском праве можно увольнять в один день но если
потом сотрудник обратится в суд то моббинг-боссинг может быть поводом для очень
существенных санкций и компенсаций

\begin{itemize} % {
\iusr{Inna Zvyagintseva}
\textbf{Дмитрий Мукойда} В Германии фиксированные контракты  @igg{fbicon.frown} 

\iusr{Дмитрий Мукойда}
\textbf{Inna Zvyagintseva} право в романо-германской системе построено на НПА как основном источнике права. Так вот законы там защищают сотрудника очень сильно. А контракты всегда под законами а не над ними
\end{itemize} % }

\iusr{Pavlo Moskalenko}
На Заході звільнити держслужбовця ще важче.

\begin{itemize} % {
\iusr{Олександр Рудоманов}
\textbf{Pavlo Moskalenko} але у нас і працівника важко

\iusr{Inna Zvyagintseva}
\textbf{Pavlo Moskalenko} В Австрії й звичайного працівника набагато важче

\iusr{Владимир Раднянко}
\textbf{Олександр Рудоманов} Треба переводити на контракт, як вчителів пенсійного віку, всі нюанси там прописати терміном на 1 рік. Щось не так, не продляємо і все, до побачення....

\iusr{Катерина Стулень}
\textbf{Pavlo Moskalenko} Але там є контракт. І неефективному працівникові просто не пролонгують контракт з ним, і все - його нема.

\iusr{Катя Кондратюк}
\textbf{Pavlo Moskalenko} і що тут хорошого? В Італії саме лівацьке трудове законодавство. І чим закінчилось протистояння , наприклад, фіату? Заводи винесли в східну Європу, італійці залишились з правами, але без роботи(

\iusr{Pavlo Moskalenko}
\textbf{Катя Кондратюк} я не сказав, що це добре.

\iusr{Inna Zvyagintseva}
\textbf{Katya Kondratyuk} Заводи винесли, тому що робоча сила дешевша
\end{itemize} % }

\iusr{Андрій Жидков}

Так у нас якість законів загалом дуже невисока. Ось по цьому прикладу. Закон
про освіту та закон про вищу освіту містять вимоги виховання поваги до держави
і т.інш. Але ніяких механізмів примусити працівника виконувати ці вимоги немає.
Це ж достатнь проста річ. В контакті прописати "дотримуватися закону... при
незгоді - звільнення". Все. Але маємо, що маємо.

\begin{itemize} % {
\iusr{Марфа Соломенцева}
\textbf{Андрій Жидков} спробуй довести, що не дотримується, це ж така суб'єктивна річ ця повага ...

\iusr{Андрій Жидков}
\textbf{Marfa Solomenceva} ну це як з нашими сєпарами. Якщо людина прямо на камеру не скаже "я посягаю на територіальну цілісність держави", то все інше "нещитається"
\end{itemize} % }

\iusr{Andriy Servetnyk}

Саме так. Не буде порядку без можливості в будь-який момент звільнити
працівника. Не потрібен - отримуй компенсацію за угодою сторін і шукай собі
місця деінде. А зараз просто вакханалія, особливо з різноманітними чиновниками,
які поновлюються і поновлюються на роботі через суди...

\begin{itemize} % {
\iusr{Alena Morgan}
\textbf{Andriy Servetnyk} цікаво про компенсацію за згодою сторін))))
\end{itemize} % }

\iusr{Ganna Bezgina}
Але якщо дати можливість звільняти на раз два, то які запобіжники в ситуаціях
коли працівника просто хочуть позбавитись?

\begin{itemize} % {
\iusr{Олександр Рудоманов}
\textbf{Ganna Bezgina} у мене особисто думка, що якщо роботодавець, який дає роботу хоче звільнити працівника, якому він платить зп, то він має право це робити.
Бо в магазині можна вибрати не гнилі продукти, а нормального робітника ні? Але дати тижня за 2 попередження.

\iusr{Михаил Дикун}
\textbf{Ganna Bezgina} роботодавець повинен мати можливість позбавитись незручного працівника, щоб не було такої ситуації, як ця

\iusr{Ganna Bezgina}
\textbf{Олександр Рудоманов} тоді працівник практично позбавляється своїх прав. Роботодавець матиме право без будь якої причини звільнити працівника.

\iusr{Ольга Войтко}
\textbf{Ganna Bezgina} Раніше можна було взяти на випробний термін, щоб зрозуміти якій той працівник насправді. А зараз береш кота в мішку, а звільнити не можеш... Особливо це стосується маленьких ФОП де реорганізацією не пахне, і комісій у складі 3-х людей (без того кого звільнити) скласти не можна, бо тільки сам ФОП і один робітник.

\iusr{Ganna Bezgina}
\textbf{Михайло Дикун} це фактично перетворює працівника в раба.

\iusr{Валентина Шапошник}
Цікаво, якому це працівнику цікаво працювати там, де його не цінують? Як може бути нормальним ККД від таких взаємин?

\iusr{Ganna Bezgina}
\textbf{Валентина Шапошник} очевидно ви забуваєте про ситуацію на ринку праці. Притомних робочих місць на всіх не вистачає нажаль.

\iusr{Ganna Bezgina}
\textbf{Ольга Войтко} а зараз норму про випробувальний термін змінено?

\iusr{Дмитро Кучерак}
\textbf{Ganna Bezgina} вам же пояснили, що людина, яка платить гроші вправі вирішувати, хто їй потрібен і ті

\iusr{Ganna Bezgina}
\textbf{Дмитро Кучерак} в жодній країні немає такого не щоб звільнли людину без причини.

\iusr{Гена Дем'янчук}
А чого роботодавець зобов'язаний вас там тримати? В нього є право обирати собі працівників з якими йому вигідно і комфортно працювати

\iusr{Денис Милий}

Право на працю є одним з базових прав людини. Держава створює умови для його
реалізації та захищає. Не подобається це роботодавцю - він завжди може
припинити діяльність зовсім. Це ж добровільна справа, на відміну від дотримання
трудового законодавства.


\iusr{Ольга Войтко}
\textbf{Ganna Bezgina} так, її немає


\iusr{Олександр Рудоманов}
\textbf{Валентина Шапошник} \textbf{Виктор Задворнов}, розкажіть про те, як вас не могли звільнити, хоча ви нічого не робили  @igg{fbicon.smile} 
\iusr{Сергей Дубильер}
\textbf{Олександр Рудоманов} в теории есть практика ежегодного перезаключения договора. Человек дорабатывает до конца законтрактованного периода и не получает новый контракт

\iusr{Ganna Bezgina}
\textbf{Гена Дем'янчук} а тому що існують закони. Ми ж не в феодальному суспільстві живемо

\iusr{Гена Дем'янчук}
\textbf{Денис Милий} За те ніхто в здоровому розумі не хоче офіційно наймати нових працівників. Бо совкова держава зуміла і там все зарегулюватт

\iusr{Денис Милий}
\textbf{Гена Дем'янчук} якщо я зараз вас попрошу перерахувати цих "всіх" - ви цього не зможете зробити. Тому що це м'яко кажучи не так)

\iusr{Гена Дем'янчук}
\textbf{Ganna Bezgina} Закони встановлені ще совковими окупантами такий собі аргумент

\iusr{Валентина Шапошник}
\textbf{Олександр Рудоманов} , а чого це мене не могли звільнити? Як тільки сказали, що я не потрібна, відразу встала і вийшла.

\iusr{Romko Krytko}
\textbf{Ganna Bezgina} якщо працівника хочуть позбавитись то його позбавляться з будь-якими законами...

\iusr{Віктор Револьвенко}
\textbf{Денис Милий} ага розкажіть це медсестрам з 4 тисячами грн зп. Оце право на працю, не хочеш мінімалку на важкій роботі звільняйся, і ти зможеш знайти лише таку ж роботу на мінімалку.

\iusr{Денис Милий}
\textbf{Віктор Махно} а я хіба казав, що держава в нас справляється на відмінно із покладеною на нею функцією створення належних умов для реалізації права на працю? Ні. Але це не означає, що роботодавці не повинні дотримуватися трудового законодавства. Це не виправдання.

\iusr{Romko Krytko}
\textbf{Денис Милий} цікаво, якщо усі роботодавці припинять діяльність зовсім,
як ви будете проводити свою популістично-соціалістичну політику? За чий кошт?
Ви, як депутат оберіть собі помічника, який нічого не робитиме, платіть йому
заралату і ні в якому разі не звільняйте бо у нього є право на халяву. Як у 80\%
усіх хто працює на державних роботах, ось їхній і ваш результати роботи є
показником життя українців.

\iusr{Макс Макс}
\textbf{Ольга Войтко} можна скласти договір

\iusr{Ольга Войтко}
\textbf{Макс Гусейнов} договір не можна, а об'язково. А от після договору фіг звільниш.

\iusr{Олександр Равчев}
\textbf{Олександр Рудоманов} в нас більше 50\% працівників - державна власність)

\iusr{Anatolii Oliynyk}
\textbf{Олександр Рудоманов} ІМХО, з своєї кишені платить приватний роботодавець. І звільняти він має право одноосібно.
В держструктурах інша ситуація. Держава забезпечує право на працю і вона ж платить. Тому звільняти може керівник через рішення трудового колективу. І це рішення працівник має право оскаржити в суді.
По іншому - буде беспредел.

\iusr{Ganna Bezgina}
\textbf{Гена Дем'янчук} німецькі почитайте


\iusr{Олександр Рудоманов}
\textbf{Anatolii Oliynyk} є департамент, який нічого не робить
Чому як платник податків, з моєї кишені йдуть туди гроші?

\iusr{Markijan Hrynovets}
\textbf{Олександр Рудоманов} як викривач корупції, якого вже 2чі за це звільняли - скажу, що бажання роботодавця не може бути причиною для звільнення. Звільняти можна і треба за невідповідність вимогам до місця праці (напрофесіоналізм) або невиконання роботи. Особливо це стосується бюджетної сфери, де кожен начальник - цар зі звичаєвим правом.

\iusr{Андрей Пращарук}
\textbf{Ganna Bezgina} якби особисто мене хотіли б «позбавитися», я б навряд чи хотів лишатися працювати у такому колективі. Єдине, якщо я б розумів, що я виконував роботу добре, а звільнили мене просто так. То я б міг вимагати якусь компенсацію, але б я точно не повернувся б туди працювати назад.
Якщо говорити про скорочення штату, то напевне було б добре виплачувати скороченим працівникам компенсацію у вигляді 1-2 місячної зп, щоб була можливість спокійно знайти нову роботу.

\iusr{Inna Zvyagintseva}
\textbf{Олександр Рудоманов} Ок, а якщо звільнення не пов'язане з поганою роботою? Наприклад, хочуть звільнити вагітну жінку? Які будуть запобіжники в даній ситуації?


\iusr{Олександр Рудоманов}
А хіба в декреті працюють, а не беруть типу відпустку?

\iusr{Елена Ена}
\textbf{Romko Krytko} , Ви дуже помиляєтесь. Навіть великі компанії з юристами, я не кажу вже про ФОП, мають зі звільненням працівників складності, якщо виник конфлікт.

\iusr{Romko Krytko}
\textbf{Єна Олена} я знаю, що є складності, але немає неможливого.

\iusr{Елена Ена}
\textbf{Romko Krytko} , Ви оптиміст. А як з такими законами жити реалістам?)

\iusr{ходаковский костя}
\textbf{Олександр Рудоманов} примеров масса... полиция, попробуй там кого то уволить

\iusr{Михаил Дикун}
\textbf{Ganna Bezgina} ні, кому не подобається - написав заяву і вперед на нову роботу. У нас звільнитися за власним бажанням дуже просто і зараз, а от звільнити людину дуже складно. І це треба міняти

\iusr{Bogdan Gerasymchuk}
\textbf{Ganna Bezgina} для того щоб роботодавець не хотів позбавитись
працівника , то останній не має бути мудаком ... як у 90\% випадків . А радянський
трудовий кодекс 1974 по моєму , ліпиться до сьогодення як можливість охороні
праці розірвати будь якого роботодавця , навіть самого правильного і чесного .
І ще , перед тим як щось вимагати у того хто бізнес створив , для початку
потрібно пройти його шлях в створити своє , відпрацювати ефективно хоча б
рочків 5 , а потім вчити роботодавця розуму ....

\iusr{Михаил Дикун}
\textbf{Денис Милий} до чого тут право на працю? Звільнили людину - він пішов на нову роботу, його ніхто права на працю не позбавляє

\iusr{Ganna Bezgina}
\textbf{Олександр Рудоманов} до декрету можуть звільнити жінку.

\iusr{Ganna Bezgina}
\textbf{Bogdan Gerasymchuk} почитайте не радянські закони Европи.

\iusr{Катя Кондратюк}
\textbf{Ganna Bezgina} «тоді працівник практично позбавляється своїх прав. Роботодавець матиме право без будь якої причини звільнити працівника.»
Ви хоч задумуєтесь, що кожна двостороння угода має бути добровільною і рівноправною. Це основа римського права і кодексу Наполеона. Якщо працівник має право легко звільнитись, таке ж право має бути і роботодавця.

\iusr{Катя Кондратюк}
\textbf{Денис Милий} ну шо за... так само, як держава гарантує право на працю, так само гарантує право на підприємництво і найм. Але право - це не гарантія. Неможливість законно звільнити працівника - це якби покупця змушували купувати ті самі продукти щоразу в магазині, бо він колись мав необережність їх купити, адже держава також захищає право на підприємницьку діяльність. Захищати радянський кодекс, який регулював стосунки між держмонополією і фактично рабами і натягувати його на ринкову економіку - це таке нерозуміння природи добровільності договору і рівноправності сторін, що аж страшно(

\iusr{Катя Кондратюк}
\textbf{Ganna Bezgina} для фопа нема.

\iusr{Bogdan Gerasymchuk}
\textbf{Ganna Bezgina} я живу в Україні , і вимушений жити за реаліями що існують , що мені дадуть закони Европи , за мого життя я сумніваюсь що ми зможемо називатись европейцями . Та й я вважаю , що в кожній країні повинні бути свої адекватні правила , навіщо тупо копіювати щось і в когось . То не є панацеєю, доказано Німеччиною з їхнім «утриматись від критики північного потоку 2»

\iusr{Ganna Bezgina}
\textbf{Katya Kondratyuk} всьому свій час. Те що зараз вам здається дуже легким і правильним з часом зміниться. Я не буду з вами сперечатись, я як людина в віці знаю в яких ситуаціях може опинитись людина.

\iusr{Ganna Bezgina}
\textbf{Bogdan Gerasymchuk} при чому тут трудове право до Північного потоку? І захист робітника це не мінус.

\iusr{Катя Кондратюк}
\textbf{Ganna Bezgina} якраз проблема в тому, що час змінився, а кодекс - ні. Те, що писалось для індустріальних трудових відносин в умовах державної монополії, не працює в ринковій постіндустріальній економіці. Не думаю, що ви старші, за кодекс, тому всі випадки несправедливості якраз на совісті утопічного кодексу.
Ок, хочете, щоб людину не міг звільнити роботодавець, тоді ви мали б топити за те, що і людина не може звільнитись коли їй заманеться. Ринкова економіка працює на принципах взаємовигоди і добровільності. Ну не можуть марксистські принципи, закладені в кодексі, працювати в ринковій економіці!

\iusr{Ganna Bezgina}
\textbf{Katya Kondratyuk} назвіть мені будь ласка цивілізовану країну де роботодавець може просто звільнити людину.

\iusr{Катя Кондратюк}
\textbf{Ganna Bezgina} США.

\iusr{Ganna Bezgina}
\textbf{Katya Kondratyuk} ну це не правда. Там теж укладаються трудові угоди. Існує захист профспілковий. І попереджувати про звільнення у більшості випадків потрібно за 60 днів. Луснувши пальцями там теж не звільниш.

\iusr{Катя Кондратюк}
\textbf{Ganna Bezgina} не маніпулюйте, ніхто не казав, що не потрібно попереджати. Чи не потрібно угод.
Профспілковий захист - це де? В Детройті? Захистили? Подобається результат? Якось в силіконовій долині всі ці смузі бари і столи для тенісу працівникам з’явились не завдяки профспілкам, а завдяки ринку.
Була б можливість попередити і звільнити як в США, цієї дискусії б не було. Саме тому в нас і є ця прекрасна практика разом з заявою на прийом брати підписану заяву на звільнення. Це лицемірство, коли навіть в Адміністрації президента є така практика, а лохторату розказують про захист пролетаріату.

\iusr{Катя Кондратюк}
\textbf{Денис Милий} «Право на працю є одним з базових прав людини. Держава створює умови для його реалізації та захищає. Не подобається це роботодавцю - він завжди може припинити діяльність зовсім. Це ж добровільна справа, на відміну від дотримання трудового законодавства.» це тому адміністрація президента сама практикує наперед підписані заяви про звільнення?

\iusr{Inna Zvyagintseva}
\textbf{Олександр Рудоманов} Жінка має право на декрет за певний час до пологів, здається за 75 днів. Є чимало роботодавців, які воліють позбутися жінки на більш ранньому терміні вагітності

\end{itemize} % }

\iusr{Михаил Дикун}
Такі зміни і не ухвалять, як мінімум у найближчій перспективі, бо у нас досі всі граються у соціалізм

\begin{itemize} % {
\iusr{Олександр Рудоманов}
\textbf{Михайло Дикун} та наче пробують, але проблема в тому, що основний виборець - соціаліст по факту, але різні переконання ідеології

\iusr{Михаил Дикун}
\textbf{Олександр Рудоманов} отож, і такі закони - очевидні електоральні втрати

\iusr{Yurii Lev}
\textbf{Олександр Рудоманов} в конституції записано, що ми соціальна країна ...

\iusr{Alena Morgan}
\textbf{Михаил Дикун} весь цивілізований світ загрався в соціалізм!!)))
\end{itemize} % }

\iusr{Алексей Задорожный}

Я вообще не понимаю этого идиотизма. И профсоюзы тоже не перевариваю.
Руководство любого предприятия должно иметь полное право самостоятельно решать,
какие люди на них работают и на каких условиях. А сотрудник либо соглашается на
условия работодателя, либо ищет другого.

\begin{itemize} % {
\iusr{Роман Химич}
\textbf{Алексей Задорожный} сразу видно, что вы настоящий европеец!

\iusr{Катя Кондратюк}
\textbf{Roman Khimich} а справжня Європа будувалась на соціалізмі?

\iusr{Роман Химич}
\textbf{Катя Кондратюк} давайте уточнимо часові проміжки побудови, аби не використовувати поняття соціалізм до часів феодалізму, наприклад

\iusr{Катя Кондратюк}
\textbf{Roman Khimich} та будь який, коли економіка європейських держав була першою в світі. Коли темпи росту були не в межах статистичної похибки.

\iusr{Роман Химич}
\textbf{Катя Кондратюк} після ПСВ Стара Європа навернулася до соціалізму. Після ДСВ він став домінуючим напрямком. Соціальна держава, держава загального благополуччя грунтувалися саме на соціалістичних підходах

\iusr{Катя Кондратюк}
\textbf{Roman Khimich} там у всіх по-різному) соціалістичною стала тільки російська імперія. Найдовше противилась соціалізму Великобританія. І це прямо корелює з добробутом населення цих країн

\iusr{Роман Химич}
\textbf{Катя Кондратюк} мабуть варто визначити, що саме ми розуміємо під соціалізмом.
Про ВБ дуже дивно чути, 50-70 роки це типовий соціалізм, купа держпідприємств, височезні податки і т.і.

\iusr{Катя Кондратюк}
\textbf{Roman Khimich} а потім приходить Маргарет Теччер) там хоч чергувались ліві і консерватори. Тому так, вб найменш ліва. Та ж Італія куди лівіша...

\iusr{Роман Химич}
\textbf{Катя Кондратюк} прийшли Тетчер і що? Ковід продемонстрував, що СОЗ ВБ один з найслабкіших в Європі, на рівні отієї Італії. Про ФРН чи Північну Європу годі й казати

\iusr{Катя Кондратюк}
\textbf{Roman Khimich} до чого тут ковід? Тоді за вашою логікою, Куба справилась, яка лікарів, як рабів, віддає в користування іншим країнам, а оплату забирає собі. Якраз в Італійська медицина найменше постраждала від лівацтва. Лікуватись там дуже дорого, багато приватної медицини... до того ж соз не один ковід характеризує, є ще смертність, заходи профілактики тощо

\iusr{Роман Химич}
\textbf{Катя Кондратюк} я уникаю дискусій, в яких використовуються кліше на кшталт "лівацтво", "фашизм", "вата" і т.і. Не обговорюю чиюсь Віру Істинну і т.і.
Якщо Куба демонструє високі показники суспільного здоровя, значить вона справляється краще за країни, в яких ці показники нижче. Якою ціною - окреме питання.
Наведіть, будь ласка, дані, які характеризують стан СОЗ, подивимось разом
\end{itemize} % }

\iusr{Марфа Соломенцева}

Я таку тітоньку, навіть після реорганізації, мусила відновити на роботі через
12!!! років. Вона виграла всі суди, а тому числі Європейський. І отримала
компенсацію від України.

\begin{itemize} % {
\iusr{Olga Kakaulina}
\textbf{Marfa Solomenceva} в тому то і справа, що завжди в таких випадках порушень процедури вдосталь, для відновлення. Взяти приклад із звільненням поліцейських, які через суд відновились на посадах. А потім ще й платники податків виплачують компенсацію.


\iusr{Олександр Рудоманов}
\textbf{Olga Kakaulina} один з методів в УЗ, це домовленість з кадровиками, щоб вас не правильно звільнили  @igg{fbicon.smile} 

\iusr{Olga Kakaulina}
\textbf{Олександр Рудоманов} так таке не тільки там, на жаль.


\iusr{Олександр Рудоманов}
\textbf{Olga Kakaulina} в приватному бізнесі теж


\iusr{Олександр Рудоманов}
?

\iusr{Olga Kakaulina}
\textbf{Олександр Рудоманов} за приватний не скажу, бо не знаю. А ось держструктури, то стандартна практика((((

\iusr{Юрий Белоусов}
Мабуть погано звільняли. До речі, і цю падругу також звільняють з порушеннями, з неймовірно низької якості наказом, з яким її попереджають. І тут звичайно може бути і сидіння в центрі зайнятості на допомозі до 2 років, і поновлення на роботі, в тому числі через Європейський суд

\iusr{Alexander Kleimenov}
\textbf{Marfa Solomenceva} а якщо б ця людина була не ватою, а викривала б корупцію на факультеті чи у виші, її б роботодавець (якщо виш державний, то держава?) теж мав би право звільнити на раз-два?


\iusr{Олександр Рудоманов}
\textbf{Юрій Білоусов} чому?

\iusr{Марфа Соломенцева}
\textbf{Alexander Kleimenov} не знаю про раз-два, підстава ж має бути. До речі, в державному органі це все і було. А про корупцію, то з моєї кафедри викладача впіймали на хабарі і засудили умовно. Він потім знов працював через декілька років

\iusr{Ірина Єгорченко}
\textbf{Alexander Kleimenov} тоді б точно звільнили б на раз два без розмов. І без порушень щоб жоден суд не відновив

\iusr{Ірина Єгорченко}
Ту ж Пархоменко звільняли з кулька методом ліквідації факультету, суд присудив їй якісь гроші невеликі і все

\iusr{Роман Химич}
\textbf{Alexander Kleimenov} так, звичайно.

\end{itemize} % }

\iusr{Halyna Slotska}

Та так не тільки у Союзі, так у цілому світі. Звільнення з постійної роботи йде
через суди і виплати компенсацій. А як провести різницю між неугодним
працівником і « дурачком». А якщо неефективність буде зумовлена виконанням
заздалегідь важких завдань, потребуючих часу. Наскільки розумію мова йде про
людину, яка вела себе публічно неетично на займаній посаді, порушуючи
конституцію та закони Украіни. От це і є величезною підставою на звільнення.
Питання, чому суд не використовує такі пункти, в чому причина?

\begin{itemize} % {
\iusr{Олександр Рудоманов}
\textbf{Галина Слоцька} політичні погляди або відсутність рішення суду по криміналу не є підставою для звільнення.
Чому, якщо роботодавець дає роботу, то він ще має думати, як не звільняти того, хто йому не приносить гроші?

\iusr{Halyna Slotska}
\textbf{Олександр Рудоманов} Якщо я не помиляюся там мова йде про ту дивну викладачку?


\iusr{Олександр Рудоманов}
\textbf{Галина Слоцька} так, але по ній немає рішення суду.
Тому лише такий мезанізм(

\iusr{Halyna Slotska}
Якщо чесно, то таких людей можна звільнити тільки через реорганізацію. Найоптимальніший вихід. Питання до тих хто її влаштував на цю посаду.

\iusr{Дмитрий Мукойда}
\textbf{Олександр Рудоманов} потому что ты сам можешь создать условия для непродуктивности сотрудника. И что сам подставил сам уволил?
\end{itemize} % }

\iusr{Inna Zvyagintseva}

Для посади зав. кафедри є ряд вимог. Якщо викладачка посідала цю посаду, роблю
висновок, що керівництво вишу вона влаштовувала. Але трапився скандал з ватними
поглядами... Невже в когось є сумніви, що на подібну посаду в пост-радянському
виші не призначають за прозорим конкурсом?

\begin{itemize} % {
\iusr{Julia Zinovjeva}
\textbf{Inna Zvyagintseva} вона ж ніби завкафом не була, я просто багато років вела практику у студентів з тої кафедри, ну то ти, певно, деякі перли читала в ФБ, це було важко, і студентів шкода, але одна викладачка зрозуміла, що практика трохи спасає загальний рівень і дуже просила мене виділяти гьдину на тиждень на гуляння музеєм...Там і погляди ватні, і рівень викладання низький
\end{itemize} % }

\iusr{Дмитрий Бондаренко}
То я колись на шахті працював, там не могли звільнити курв з відділку, який самоспаси з жетонами видають, бо вони відмічали у системі, що людина поїхала у шахту, а по факту воно бухає лежить десь і зп йому тікає.
З останнього що я чув, баба через пів року поновилась через суд з виплатою зп за весь час і ще компенсацію вибила.

\iusr{Sergei Kovalchuk}
Ну да ? 3 письма о нарушениях ,опозданиях и прочее , и увольняетса легко по статье с записью в трудовой. Кадровики просто ленивые или кумовство. Кзот никто не отменял.

\iusr{Андрій Чвалюк}
Ліквідація кафедри також не панацея, якщо було допущено порушення численних норм трудового зак-ва.

\iusr{Юрий Белоусов}
В сенсі? А які підстави звільнення Ви бачите в цьому конкретному випадку? Що вона не так робить як викладач?

\begin{itemize} % {
\iusr{Олександр Рудоманов}
\textbf{Юрій Білоусов} та я цю ситуацію використав, щоб показати, що у нас застаріле законодавство. Бо всі хочуть звільняти по щолчку пальців, а таке не можливо.

\iusr{Юрий Белоусов}
\textbf{Олександр Рудоманов} так і правильно. Нашвидкоруч написаний наказ про звільнення (бо я так хочу) потім летить у суді. Це нормально, це правильно. До будь-якої справи, в тому числі звільнення працівника, треба належним чином готуватися


\iusr{Олександр Рудоманов}
\textbf{Юрій Білоусов} має бути легші механізми звільнення, ніж є зараз


\iusr{Олександр Рудоманов}
\textbf{Юрій Білоусов} я суджу по уз, якщо що


\iusr{Олександр Рудоманов}
Чувак просто ходить на роботу, і його не можуть звільнити

\iusr{Юрий Белоусов}
\textbf{Олександр Рудоманов} 

у нас зараз на 99\% зайнятих людей трудове законодавство взагалі не
поширюється, працівників гноблять і викидають куди хочуть і як хочуть,
штрафують, обмежують, заганяють у фопи. А Ви про лояльні процедури звільнення?
Кодекс законів про працю застосовується хіба ще до державного сектору, а
особливо чиновники дуже люблять це використовувати. Особливо в частині
компенсації за вимушений прогул (хоча жоден із судів не запитав заявника, а
чому ти, голубе сизий, вимушено простоював, чому ніде не пішов працювати за час
незайнятості, щоб прогодувати себе або свою сім'ю.

У мене тоді наступне питання: якої підстави для звільнення не вистачає в
законодавстві, щоб задовольнити потреби роботодавця?

\iusr{Аурел Зайчук}
\textbf{Юрій Білоусов} робить з студентів "вату"

\iusr{Юрий Белоусов}
\textbf{Аурел Зайчук} а що в нас за студенти такі нестійкі до утворення вати? У нас 22\% людей хочуть в митний союз. Їх також будете звільняти?


\iusr{Олександр Рудоманов}
\textbf{Юрій Білоусов} розумієте, проблема в тому, що держпідприємства не можуть собі такого дозволити, тому стають ще більш неефективними.
Звільнення без пояснення причин, наприклад,але і без виплати 2х окладів.

\iusr{Юрий Белоусов}
\textbf{Олександр Рудоманов} гарна підстава. Можна розглянути. Але для цього не потрібно приймати новий кодекс. І ця пропозиція вже давно могла зайти в парламент і бути прийнятою. Таким чином можна було "люструвати" всіх чиновників. Нехай з п'ятьма окладами


\iusr{Олександр Рудоманов}
\textbf{Юрій Білоусов} так я ж про що. Зміни необхідні, але ніхто нічого не хоче міняти

\iusr{Аурел Зайчук}
\textbf{Юрій Білоусов} і би цім 22\% ще й гормядянства позбавляв.

\iusr{Юрий Белоусов}
\textbf{Аурел Зайчук} якби такі механізми були закладені в законодавстві, то яка гарантія, що Ви б залишилися із громадянством?

\iusr{Аурел Зайчук}
\textbf{Юрій Білоусов} стовідсоткова

\iusr{Роман Химич}
\textbf{Олександр Рудоманов} якщо людина не ходить на роботу то це прогул. Два прогули, два попередження і на вихід. В чому проблема УЗ?

\end{itemize} % }

\iusr{Сергій Фазульянов}
Вона точно була завкаф?

\begin{itemize} % {

\iusr{Олександр Рудоманов}
\textbf{Сергій Фазульянов} нач каф

\iusr{Сергій Фазульянов}
\textbf{Олександр Рудоманов} а де про це пишуть. Бо я бачиі тільки професорка на кафедрі, але аж ніяк не завеаф


\iusr{Олександр Рудоманов}
\textbf{Олександр Одєгов} я і коментарях


\iusr{Олександр Рудоманов}
Але все одно це найпростіший алгоритм

\iusr{Сергій Фазульянов}
\textbf{Олександр Рудоманов} алгоритм так. Але все ж пересічних професор кафедри без докторського ступеню

\iusr{Максим Рожков}
\textbf{Сергій Фазульянов} вона була проффесорка, не завкафедри
\end{itemize} % }

\iusr{Alexander Andrianov}

А если пойти в своих размышлениях хоть чуточку дальше, и представить, что эти
новые изменения в трудовое законодательство, за которые вы так ратуете, будут
распространяться не только на увольнение этой ватной преподавательницы, а НА
ВСЕХ?

В том числе, и на случаи, когда ватным будет являться уже работодатель (а
сейчас, например, в позеленевших госструктурах такое - массово)?

\begin{itemize} % {
\iusr{Олександр Рудоманов}
\textbf{Alexander Andrianov} та при чому тут саме вона? Це як показник, що роботодавець не може звільнити неефективного працівника легко. Це і треба міняти

\iusr{Alexander Andrianov}
\textbf{Олександр Рудоманов} - меняйте. Усильте дов6oебов у власти ещё и новым трудовым кодексом, допускающим рабство. А там, потихоньку, и до каннибализма дойдем - чё, он же существенно упрощает социальные отношения, не?

\iusr{Олександр Рудоманов}
\textbf{Alexander Andrianov} чому ви це дивитесь з погляду на владу? Он люди відкоментили з бізнесу, що срака з нинішнім тк

\iusr{Alexander Andrianov}
\textbf{Олександр Рудоманов} - м-да, ходить по граблям - карма нашего народа...

\iusr{Олександр Рудоманов}
\textbf{Alexander Andrianov} тому і не міняють цей кодекс, який родом з 71 року

\iusr{Катя Кондратюк}
\textbf{Alexander Andrianov} держслужба має свої додаткові закони. Там інший порядок прийняття і звільнення

\iusr{Alexander Andrianov}
Вивчить цю тематику досконаліше - це все, що можу запропонувати з цього приводу. Не купуйтесь на "мільярд дерев" та інші лозунги; обмірковуйте всі нюанси. Якщо не бажаєте загнати країну та самих себе в ще більшу дупу.

\iusr{Василь Фалес}
\textbf{Alexander Andrianov} если тебя работодатель захочет
сделать рабом то ему по\&уй какой кодекс... особенно когда у него есть свой
карманный профсоюз типа бананового (ПЗТС) на УЗ. И если ты не хочешь быть
рабом, то тебе придется свои права защищать только через суд. Для того чтобы ты
убедился посмотри внимательно на этот документ в котором наглядный пример как
можно с любого работника сделать раба принудить его работать почти в два раза
больше нормы каждый месяц на протяжении длительного периода и при этом не
оплачивать за часы отработанные сверх установленной нормы которые должны были
оплачиваться в двойном размере.

\ifcmt
  ig https://scontent-frx5-2.xx.fbcdn.net/v/t39.30808-6/221905554_151110483773558_5808239251494093938_n.jpg?_nc_cat=109&ccb=1-5&_nc_sid=dbeb18&_nc_ohc=6iVXJgSbxCkAX_FHCAZ&_nc_oc=AQmQw_j3SA7hCjYskWiLEWxB8pI0V-lO1WYXh7oGmLdYWC9ynvNKqKtH6iSRCDh0W0E&_nc_ht=scontent-frx5-2.xx&oh=f7e4ed4677f30c2f30e29d5775efd6f3&oe=61496191
  @width 0.3
\fi

\iusr{Alexander Andrianov}
\textbf{Василь Фалес} - заe6ись; можно смело отменять Трудовой кодекс, да, Вася?
* Хуже нє будєт! Хоть паржом!

\ifcmt
  ig https://scontent-frx5-1.xx.fbcdn.net/v/t39.1997-6/p370x247/851585_592620644106400_451249332_n.png?_nc_cat=100&ccb=1-5&_nc_sid=0572db&_nc_ohc=8H3faEsfGh8AX9uQ5_6&_nc_ht=scontent-frx5-1.xx&oh=dac32cd6c861da76fd98449e9bc4372d&oe=614A373D
  @width 0.2
\fi

\iusr{Василь Фалес}
\textbf{Alexander Andrianov} а зачем нужен такой кодекс если работодателю на него на плевать?

\iusr{Alexander Andrianov}
\textbf{Василь Фалес} - да и Конституция тогда - какбэ, тоже не нужна - уж коли и ею жопу подтерли...

\iusr{Роман Химич}
\textbf{Олександр Рудоманов} в чому полягає неефективність цієї викладачки окрім того, що від її поглядів в когось підгора?

\end{itemize} % }

\iusr{Олександр Рак}

В університетах не зовсім так працює, більшість беруть на річні контракти, а
лояльних - на безстрокові. Більченко мабуть була з останніх, бо інакше з нею
просто могли не укладати новий договір


\iusr{Yurii Lev}

Є проблема. В нас не можуть приймати закони, які звужують права громадян. І тут
це теж можуть питання підняти.  "Згідно з положеннями ч. 3 ст. 22 Конституції
України при прийнятті нових законів або внесенні змін до чинних законів не
допускається звуження змісту та обсягу існуючих прав і свобод."

\begin{itemize} % {
\iusr{Alena Morgan}
\textbf{Yurii Lev} це точно пролема?Російськомовним же звузили права...Не зупиняйтесь))Головне, більше ініціатив від народу!))

\iusr{Yurii Lev}
\textbf{Alena Morgan} Ну тут типу як: закон як дишло, куди попхнеш, туди і вийшло. Могли визнати прийнятий закон не конституційним.
Як там було: конституційний суд визнав конституційну реформу неконституційною...
Але тут складніше. Бо як кажуть вірно, люлям таке в основній масі не сподобається. А такі популісти як ЮВТ наприклад цим точно склристаються і будуть використовувати таку вдалу для них тему, тому непомічено це навряд чи пройде.
Звісно залежить і який висновок дасть КСУ.
Тому так, тут поблема є.
\end{itemize} % }

\iusr{Vitalii Ovcharenko}

Позавчора 20 липня, ветерани і громадські активісти Донеччини провели мітинги
біля ДонОДА (у Краматорську) проти призначення Вадима Ляха на пост голови
війського-цивільної адміністрації Слов'янська.

Дивіться на відео, хто такий Вадим Лях та чого проти нього протестують ветерани
- нажимайте на посилання:

\url{https://www.youtube.com/watch?v=lne3zRcLqmo}

\iusr{Василий Сергиенко}
тут є дві сторони медалі! Якщо упростити звільнення, то повірь, більше
нормальниз звільнять, ніж сєпарських підара.......вї

\begin{itemize} % {
\iusr{Олександр Рудоманов}
\textbf{Василий Сергиенко} слухай, якщо людина спеціаліст, то зможе знайти роботу

\iusr{Василий Сергиенко}
\textbf{Олександр Рудоманов} теоритично так. Але якщо людина віддала пів життя роботі, вклала туди своє здоров*я, сили, навіть якщо і матеріально туди вклала (меблі, якісь побутові речі) і потім от так просто їй потрібно шукати нове місце?


\iusr{Олександр Рудоманов}
\textbf{Василий Сергиенко} і що? Її змушували вкладати?
Чому роботодавець має утримувати неефективного робітника через повагу?

\iusr{Василий Сергиенко}
\textbf{Олександр Рудоманов} ми тільки що казали про спеціаліста)


\iusr{Олександр Рудоманов}
\textbf{Василий Сергиенко} ну я виходжу з позиції, що спеціаліст знайде кращу роботу)

\iusr{Петро Степанець}
Підтримую, якщо людина з головою то завжди найде собі роботу.
\end{itemize} % }

\iusr{Назар Гнатович}

що за дурня? З кафедри можна звільнити дуже просто - тупо не продовживши
контракт і все). Зара викладачі сидять на контрактах річних, може якщо хтось
блатний або крутий виш, то там від 3 до 5 років контракт

\begin{itemize} % {
\iusr{Олександр Рудоманов}
\textbf{Назар Гнатович} якщо ти починав працювати не на контракті, то переведення на контракт це погіршення умов праці, що не допускається і визнається незаконним  @igg{fbicon.smile} 


\iusr{Олександр Рудоманов}
\textbf{Назар Гнатович} але там єдине шо питанні в термінах

\iusr{Назар Гнатович}
\textbf{Олександр Рудоманов} уже років із 20 як мінімум у вишах контракт

\iusr{Олександр Рудоманов}
\textbf{Назар Гнатович} я суджу по іншій держструктурі, якщо що) там контракти на 20 років))
\end{itemize} % }

\iusr{Nata Shunya}

У знайомих була ситуація. Переїхали заради непоганої роботи, на якій дружина
працювала півроку, в область, де базувалась компанія. Чоловік не працював і
сидів з маленькою дитиною. Покинули своє добре обжите вигідне житло в Києві.
Заплатили новому орендатору подвійну оренду + рієлтору. Через тиждень дівчина
була звільнена з роботи через дрібний конфлікт у колективі, БО робота
неофіційна і керівництво дозволяє собі розкидатися кадрами як йому заманеться,
не утруждаючи себе підтримкою здорової атмосфери у колективі. Результат - ціла
молода сім'я без грошей і у повній дупі. Без ТК подібні ситуації при і так не
простому житті в країні були б буденністю. Як можна казати, що працівникам не
потрібен захист, а роботодавець "у своєму бізнесі має право робити що
завгодно?"  @igg{fbicon.face.eyebrow.raised} 

\begin{itemize} % {
\iusr{Henry Becker}
\textbf{Nata Shunya} нормально, така ситуація це завжди ризик. Вони могли залишитися на старому місці і ні чим не ризикувати. В мене теж була аналогічна - переїхав поробив 2 тижні, звільнили, але нічо викрутився

\iusr{Nata Shunya}
\textbf{Henry Becker} тому так і живемо, бо люди толерують подібне ставлення до себе.

\iusr{Roman Lebed}
\textbf{Nata Shunya} якщо робота неофіційна, то нічого дивного

\iusr{Михаил Дикун}
\textbf{Nata Shunya} нормальними кадрами ніхто не розкидається, навпаки роблять все, щоб їх утримати. далі робіть висновки самостійно

\iusr{Nata Shunya}
\textbf{Roman Lebed} так про це і мова. Що закон і офіційність це є захист. А всім хочеться якоїсь анархії  @igg{fbicon.smile} 

\iusr{Nata Shunya}
Михайло Дикун та звісно, світ же чорно-білий

\iusr{Nata Shunya}
І роботодавець ніколи не вчиняє погано, не тупить, не буває дезінформований. І гарний робітник ніколи не втрачає роботу. Світ рожевих поні та єдинорогів.

\iusr{Roman Lebed}
\textbf{Михайло Дикун} ти, я бачу, не працював у колективі з їбанутим керівником)

\iusr{Михаил Дикун}
\textbf{Roman Lebed} та всяке було, коли мені не подобається робота, керівник чи ще щось, я звільняюсь. Таке ж право має бути у керівника

\iusr{Roman Lebed}
\textbf{Михайло Дикун} бувають різні ситуації. приміром, хтось купив посаду за
бабки і тепер розчищає місця для родиців. це особливо акутально для
держструктур. бізнесу все ж таки потрібні простіші правила, ніж зараз. якби
вони були - може й більше людей працевлаштовували б офіційно.

\iusr{Катя Кондратюк}
\textbf{Nata Shunya} розказати вам вагон ситуацій, коли працівник тупо не виходить на роботу, підставляючи всю компанію і завдаючи збитки в рази більші, чим його зарплата. Але ж ліваки набіжать і скажуть, що люди не кріпаки, ага.

\iusr{Катя Кондратюк}
\textbf{Nata Shunya} «Що закон і офіційність це є захист. А всім хочеться якоїсь анархії  @igg{fbicon.smile} ». Навіть адміністрація президента зізналась, що всі написали заяви на звільнення без дат, щоб в любий момент можна було звільнити. Якщо працівник хоче мати право в будь який момент піти з роботи, то й роботодавець мусить мати таке право. Рівність сторін правочину - це основа основ.

\iusr{Inna Zvyagintseva}
\textbf{Katya Kondratyuk} Таке право вже є. Називається "за згодою сторін"

\iusr{Катя Кондратюк}
\textbf{Inna Zvyagintseva} не таке. Працівник має право піти і без згоди роботодавця. В роботодавця цього права нема. Працівник має право звільнитись без причини, роботодавець не має права звільнити без причини.

\iusr{Inna Zvyagintseva}
\textbf{Katya Kondratyuk} Згода сторін. Якщо причини немає, то домовлятися з працівником, виплачувати йому добру компенсацію, щоб захотів пити, і звільняти. Чи ви думаєте, що в США в один день можна виставити на вулицю з однією зарплатою? Ні, так не працює. Для інших причин є інші статті

\iusr{Катя Кондратюк}
\textbf{Inna Zvyagintseva} ви справді порівнюєте трудовий кодекс СРСР з
законодавством США? Ви уявляєте, щоб в США поновилася завідуюча садком, яка
налила нашатирний спирт колезі в горло? Про це і допис. В 99\% працівник зможе
поновитись на робочому місці. Законно. І ця ватниця поновитися. А вся країна і
далі працюватиме з наперед написаними заявами на звільнення або через фопів. бо
не може в 21 столітті працювати кодекс, написаний для державної монополії.

\iusr{Inna Zvyagintseva}
\textbf{Katya Kondratyuk} А ви можете уявити ситуацію, аби в Україні як компенсацію виплатили зарплату за півроку-рік? Нашатирний спирт - це злочин, й тут має йтися не про звільнення, а про кримінальну відповідальність

\iusr{Катя Кондратюк}
\textbf{Inna Zvyagintseva} кримінальна відповідальність не є підставою для звільнення) це був відомий кейс. Зарплату за півроку-рік виплачують, якщо це було передбачено контрактом або колективним договором.
Ну і аналогічно, скільки працівників виплатили компенсації роботодавцям за півроку-рік за свої раптові звільнення?

\iusr{Inna Zvyagintseva}
\textbf{Katya Kondratyuk} І що, в Штатах виплачують за раптові звільнення? Ви ж пропонуєте дивитись на іноземний досвід. У випадку з завідувачкою кафедри, садком - проблема в системі. Ці люди настільки в неї інтегровані, що не виходить посунути навіть судом. На відміну від звичайних людей, які не захищені

\iusr{Катя Кондратюк}
\textbf{Inna Zvyagintseva} ніхто ж не проти гарантій і зобов’язань. Особливо в бізнесі. Проблема в тому, що вони односторонні. Якщо прописуємо компенсацію працівнику за звільнення, то така сама має бути і прописана і роботодавцю. Якщо працівник має право звільнятись з попередженням, бо просто не подобається, таке саме право має бути і роботодавця. Ринок праці сьогодні зовсім інший, чим той, що був в 71 році, коли єдиним роботодавцем була державна монополія і робота в працівника була єдиним засобом для виживання. Просто не може вільний ринок і монополія регулюватись за одними й тими самими правилами. Це суперечить економічній теорії.

\iusr{Inna Zvyagintseva}
\textbf{Katya Kondratyuk} А ви ніколи не думали, що позиція працівника й роботодавця нерівноцінна? І що є така річ як мінімальні соціальні гарантії?

\iusr{Inna Zvyagintseva}
\textbf{Katya Kondratyuk} Не було жодної монополії. Працівник міг працювати хоч в колгоспі, хоч на заводі, хоч в конторі. Якщо не хотів - міг вести домашнє господарство в селі й за рахунок нього жити. Ви не в курсі існування тіньового ринку в СРСР? А він був й мав чималі обсяги

\iusr{Inna Zvyagintseva}
\textbf{Katya Kondratyuk} Вільного ринку немає в жодній цивілізованій країні. Усюди є трудове законодавство, десь жорсткіше, десь ліберальніше. Поцікавтеся "вільним" ринком праці в Австрії. Й нічого, прекрасно живуть

\iusr{Катя Кондратюк}
\textbf{Inna Zvyagintseva} до чого тут система? Люди в єспл виграють справи! Реформа міліції провалилась через трудовий кодекс! Систему хотіли змінити! Атестаціями займались сотні волонтерів. Все перекреслив трудовий кодекс! Ви справді думаєте, що ніхто не шукав способу звільнити мєнтів. Його просто не було! Вони поновлювались, бо всі боялись зачепити трудовий кодекс, він же для «простих людей»!

\iusr{Inna Zvyagintseva}
\textbf{Katya Kondratyuk} Треш. Якщо реформа судів та поліції проведена неграмотно, то давайте дозволимо анархію та гуляй-поле. За всіх волонтерів розписуватись не буду, але сама знаю супер-профі-волонтерів без профільної освіти, досвіду й найменшого уявлення про сферу роботи

\iusr{Inna Zvyagintseva}
\textbf{Katya Kondratyuk} Приклад Більченко та поліції ілюструє той сумний факт, що попри реформи система лишилася стара. Вона виштовхує адекватних людей й притягує неадекватів. Трудове законодавство тут ні до чого, потрібно створити умови, щоб їх і не тягнуло поновлюватись

\iusr{Катя Кондратюк}
\textbf{Inna Zvyagintseva} поновлювались вони якраз по трудовому кодексу! Крапка. Вікно можливостей закрилось. Не готова продовжувати дискусію з людьми, які вірять, що профнепридатним мєнтам хтось має створювати якісь умови, щоб не тягнуло... з такими поглядами звільнення взагалі потрібно виключити з трудового кодексу, бо ж роботодавець не створив якісь там умови для когось, от вони бідні і так погано працюють.

\iusr{Inna Zvyagintseva}
\textbf{Katya Kondratyuk} Мова про цінності інституції. Якщо цінності - це хабарництво й відсутність правосуддя, академічної доброчесності - можна скільки завгодно звільняти одну Більченко, а на її місце прийде така сама панянка.

\end{itemize} % }

\iusr{Михаил Дикун}
В принципі, навіть коментарі під цим постом доводять, що в соціалістичному болоті ми надовго  @igg{fbicon.frown} 

\begin{itemize} % {
\iusr{Roman Lebed}
\textbf{Mykhailo Dykun} ну дядя, це правда, що звільнити людину і в цивілізованих країнах не так просто  @igg{fbicon.smile} 
\end{itemize} % }

\iusr{Женя Шумахєр}

Ні, така стаття у ТК існує. Але ефективність професорки на рівні. Із
професійними знаннями, у неї теж все добре. Аморальність її вчинку?! Ну, це
закладу треба спочатку це їй пред'явити, потім ввести санкції і у разі її
незгоди, довести це у суді належними доказами і фактами. Це називається –
демократія! Це практикують цивілізовані, розвинені економічно і
інституціонально країни. А в Сомалі, прикладом, розстрілюють. Що і пропонує
зробити велика кількість моїх співгромадян.

\begin{itemize} % {
\iusr{Олександр Рудоманов}
\textbf{Женя Шумахер} ну ви чесно читали пост далі першого абзаца? Питання не в конкретній ситуації, а в застарілому законодавстві

\iusr{Женя Шумахєр}
\textbf{Олександр Рудоманов} читайте уважно комент.
\end{itemize} % }

% -------------------------------------
\ii{fbauth.san_marino_ruslan.kiev.ukraina.lazer.rezka.gravirovka}
% -------------------------------------

Трудовий кодекс у нас взагалі щось зліплене з гівна та палок... Був у мене
випадок в Одесі. Працівниця почала красти гроші, товар наліво продавати. Я
приїхав після інвентаризації щоб на місці розібратися заіідки недостача та де
мої гроші. Але ця лазудро тупо втікла з офіса, коли дізналася, що я приїхав в
Одесу. На дзвінки не відповідала, взагалі перестала виходити на зв’язок. Ну не
їхати ж мені додому та виловлювати тим самим наражаючись на неприємності....
Звільнив та забув. Але ця шльондра через 3 місяці подала в суд, що їй не
віддали трудові книжку та вона не змогла знайти нову роботу та тому не
працювала. Найняла адвоката, як це в Одесі люблять робити всі кому не влом. Суд
прийняв її сторону не зважаючи на надані докази її зловживань, крадіжку
інвентарної техніки компанії та присудив виплатити цій шльондрі з. п. за 3
місяці.  Добре, шо зараз трудові книжки практично відмінили але у нас все
роблять якось коиво та косо тому...

\begin{itemize} % {
\iusr{Олександр Рудоманов}
\textbf{Руслан Сан-Марино} фігасє

\iusr{Ganna Bezgina}
\textbf{Руслан Сан-Марино} договір про матеріальну відповідальність був підписаний?

\iusr{Руслан Сан-Марино}
\textbf{Ganna Bezgina}, звісно був.

\iusr{Inna Zvyagintseva}
\textbf{Руслан Сан-Марино} Ви не звертались до поліції?

\iusr{Руслан Сан-Марино}
\textbf{Inna Zvyagintseva}, зверталися але вони без грошей не працюють. Це було десь 10 років тому.

\iusr{Inna Zvyagintseva}
\textbf{Руслан Сан-Марино} Але зараз вже працюють? Без грошей навіть заяву не приймали?

\iusr{Тарас Тарасов}
\textbf{Руслан Сан-Марино} У мене було декілька подібних випадків. Одного разу робітник, який виявився торчком прямо у офисі, "навів" на нас прокуратуру, мусорів, після звільнення. Звісно, нічого не відбулося, крім "кишкомотства". Іншого разу, звільнена злодійка намагалася шантажувати. Социалізьма, вона така...
\end{itemize} % }

\iusr{Вася Ігнатюк}

Перший раз бачу таку критику, саме через те, що кодекс на
стороні працівника, бо через якусь ватницю, треба робити зміни), мислення
державного мужа.

\begin{itemize} % {
\iusr{Олександр Рудоманов}
\textbf{Вася Ігнатюк} я в уз такого бачив, як гівна
\end{itemize} % }

\iusr{Andrzej Leszkiewicz}

Знаю історію, коли хвору на голову (коли хотіла поводилася як ненормальна, коли
їй було треба грала роль жертви і скаржилася в усі можливі інстанції)
прибиральницю не могли звільнити (суд поновлював) кілька років, поки юрист не
назбирав товсту пачку паперових доказів...


\iusr{Myatz Net}

До зміни кодексу, мабуть єдиний оптимальний варіант - контракт, але це
додаткові витрати на їх ведення/супроводження. Та і зміна "традиційних методів
менеджменту" (з ручними профспілками, "горизонтальними зв'язками" через
працевлаштування родичів/протеже місцевих чиновників всіх "мастєй" і т.д.) для
багатьох роботодавців, деформованих "соціалізмом", теж "стрес". Тому - метод є,
потреба невисока реально, "вспливають" лише окремі кейси, а загалом - всіх
влаштовує


\iusr{Володимир Шабатура}

Вся біда в тому, що ми не знаємо своїх прав і обов'язків. А адвокати це ті, що
знаходять "лаз" і використовують його даючи своє тлумачення закону. Роботодавці
не дуже хочуть витрачатись на юристів і тому в цьому питанні , на жаль, вони в
програші. Я не спец., але більш чим певен, що в данному випадку є дуже чітке
обгрунтування для звільнення об яке всі "маневри" адвокатів розіб'ються. Якщо
цей процес ,(звільнення) робиться згідно законодавства, з чітким дотриманням
всіх норм, то ніхто не "обійде"... Потрібно просто краще знати законодавство.
(В нас зараз суддя виносить рішення частіше погоджуючись з версією яка
виглядить більш грунтовно і на жаль часто не законною.) А доказати протилежне
дуже важко...

\iusr{Руслан Степура}
Ліквідували і знову відродять...одним наказом @igg{fbicon.wink} 

\iusr{Halyna Slotska}
Тимчасові контракти - не вихід. Нідерланди- три контракти підряд без перерви
більше 6 місяців на одному місці означає постійний контракт.

\iusr{Андрей Иваниченко}

Правила. Мы так же сокращали должность офицера-психолога, чтобы уйти наглухо
тупую (и блатную, после 2014 года предателя) офицера-психолога Татьяну Харланов

\href{https://myrotvorets.center/criminal/xarlanova-tatyana-nikolaevna}{%
Харланова Татьяна Николаевна, myrotvorets.center%
}

\iusr{Дмитрий Шадрин}
А що за навчальний заклад?

\begin{itemize} % {
\iusr{Олександр Рудоманов}
\textbf{Дмитрий Шадрин} Драгоманова
\end{itemize} % }

\iusr{Василий Сергиенко}
А ти уяви, якщо сєпар жінка і має неповнолітню дитину! то її навіть Президент не звільнить )))))). Читай Трудовий Кодекс)))

\iusr{Катерина Стулень}

100\%. Така ж фігня в армії. Відвертий бовдур просто сидить на посаді і через
якийсь час вже треба йому чергову зірку давати і переводити на вищу посаду. Так
до пенсії сидять. Є одна частина, де ідіотів кожен раз просто виштовхували еа
вищу посаду в главк, щоб не заважав. Частина реально класна, херово, коли ті
ідіоти приїздять з перевірками)))


\iusr{Александр Каменев}

З викладачами зараз простіше - всі працюють по строковому трудовому договору
(контракту). Тому ніж звільняти по статті, простіше не подовжити контракт без
будь-яких пояснень.


\iusr{Oleg Khoroschak}

Якщо є можливість - збирається комісія, оформлюється актом недолік і
звільняється за статтею із занесенням в трудову книжку. Трішки хоботно, але..
Як правило, працівник сам пише "за власним бажанням", коли пояснюєш
перспективу.


\iusr{Vitaliy Grinchyshyn}
створити відділ "забанені" і переводити туди усіх )

\begin{itemize} % {
\iusr{Ольга Войтко}
\textbf{Vitaliy Grinchyshyn} і платити зарплату)
\end{itemize} % }

\iusr{Вікторія Валевська}
Вона не просто ватна, вона істота, схиблена на руському мирі, таких на гарматний постріл не можна до студентів допускати.

\iusr{Іван Косенко}
Всьому виною довбаний лівацький популізм.

\iusr{Александр Одегов}
Она кстати ебанутая была задолго до.

\iusr{Alena Morgan}

Що стосовно рейтингу публікацій пані? Ви знаєте, що в неї чи не найвищий
рейтинг за цитування наукових статей був?Це і є KPI для вчених. Про її
ефективність - це ваші особисті думки, вигадки?На чому базуються висновки
взагалі?

Друге питання: чому в Німеччині закон захищає працівників, чому всі розвинені
країни до цього дійшли, а ви виступаєте на боці дикого капіталізму? Нащо це
вам?

\begin{itemize} % {
\iusr{Олександр Рудоманов}
\textbf{Alena Morgan} де я написав про її ефективність? Там про неї одне речення, перше, вступне лише

\iusr{Alena Morgan}
\textbf{Олександр Рудоманов} одне речення, яке видає те, що ви не в темі зовсім.


\iusr{Олександр Рудоманов}
\textbf{Alena Morgan} ок

\iusr{Alena Morgan}
\textbf{Олександр Рудоманов} і ще одне: лежачого не б'ють.


\iusr{Олександр Рудоманов}
\textbf{Alena Morgan} добивають?
\end{itemize} % }

\iusr{Андрей Рачев}
Увольнять за инакомыслие - это национализм и фашизм

\begin{itemize} % {
\iusr{Ірина Литвиненко}
\textbf{Andrzej Raczov} а за сепаратизм і держзраду - реальні статті кримінального кодексу в будь-якій країні  @igg{fbicon.wink} 
\end{itemize} % }

\iusr{Евгений Викторович}
Біда...

\iusr{Ivan Sorochan}
Кзот треба нахуй викинути на мороз.

\iusr{Svyatoslav Alexandrovich}
30 лет независимости и живём по совковому трудовому кодексу ?!
Ну так чё удивляться тогда.. значит страна такая.

\end{itemize} % }

