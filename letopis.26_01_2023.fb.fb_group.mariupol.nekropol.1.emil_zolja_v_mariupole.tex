%%beginhead 
 
%%file 26_01_2023.fb.fb_group.mariupol.nekropol.1.emil_zolja_v_mariupole
%%parent 26_01_2023
 
%%url https://www.facebook.com/groups/278185963354519/posts/926103015229474
 
%%author_id fb_group.mariupol.nekropol,marusov_andrij.mariupol
%%date 26_01_2023
 
%%tags 
%%title Эмиль Золя в … Мариуполе
 
%%endhead 

\subsection{Эмиль Золя в ... Мариуполе}
\label{sec:26_01_2023.fb.fb_group.mariupol.nekropol.1.emil_zolja_v_mariupole}
 
\Purl{https://www.facebook.com/groups/278185963354519/posts/926103015229474}
\ifcmt
 author_begin
   author_id fb_group.mariupol.nekropol,marusov_andrij.mariupol
 author_end
\fi

\textbf{Эмиль Золя в ... Мариуполе}

\emph{Джавелина} и \emph{Илон} (Маск) – среди тех редких имен, которыми украинцы называли
своих детей в 2022 году. В честь американского противотанкового комплекса и
гениального изобретателя, предоставившего украинской армии доступ к сети
интернет Starlink.

Век назад приоритеты в Украине были скорее европейскими. В конце 1898 года
рождается мальчик \emph{Эмиль-Золя}. Именно такое имя дали своему сыну Фроим Пинхусов
Цукерман, херсонский мещанин, живший в Мариуполе, и его супруга Лиза Абрамова -
в честь знаменитого французского писателя Эмиля Золя. Раввин молчаливо одобрил
их выбор - несмотря на вопиющую не-каноничность имени.

Эту информацию \enquote{раскопал} в метриках мариупольской синагоги Илья Луковенко,
историк-религиовед и неутомимый исследователь Еврейского и Старого городского
кладбищ Мариуполя.

Почему же отец выбрал именно Эмиля Золя? А не знаменитого Виктора Гюго, его современника, например?

Во-первых, Золя был широко известен среди читающей публики Российской империи.
Переводы его романов часто печатались несколькими литературными журналами
одновременно. Настоящий \enquote{повелитель душ} тогдашней интеллигенции.

Но, кажется, для Фроима и Лизы (и раввина?) важнее было другое. В 1898 году
Эмиль Золя выступил с открытым письмом к французскому премьеру-министру
\enquote{Я обвиняю!} (J'accuse), в котором обвинил правительство в
фальсификации \enquote{дела Дрейфуса}, капитана-еврея, заподозренного в
шпионаже в пользу немцев и приговоренного к пожизненной ссылке.

Франция да и вся Европа мгновенно раскололась на два лагеря – яростных
антисемитов и сторонников справедливого суда над Дрейфусом...

Поступок Золя был беспрецедентным – оказалось, что писатель-интеллектуал
способен влиять на решения правительства и даже его смещать силой своего слова!
В защиту кого? Эльзасского \enquote{ино\hyp{}родца}-еврея?! Неслыханная дерзость.

Буквально через месяц после публикации суд приговорил Золя к тюрьме и штрафу
\enquote{за клевету} на французскую армию. Писатель был вынужден бежать в
Англию. Он смог вернуться в Париж в июне 1899-го. Через месяц после смерти
своего тезки в Мариуполе (мальчик прожил 5 месяцев и был похоронен на городском
Еврейском кладбище).

Эмиль умер в 1902 году от отравления угарным газом. Существует версия, что это
было преднамеренное убийство - как месть за его позицию в \enquote{деле
Дрейфуса}.

В 1908 году его останки перезахоронили в парижском Пантеоне - в одной крипте с
Александром Дюма и Виктором Гюго. Альфред Дрейфус тоже пришел на церемонию (в
1906 году он был полностью оправдан) и едва избежал смерти - в него стрелял
праворадикальный журналист-антисемит и ранил его в руку...

А что же Фроим Цукерман в Мариуполе? В январе 1907 года у него рождается сын, и
он снова называет его... \emph{Эмиль-Золя}!
