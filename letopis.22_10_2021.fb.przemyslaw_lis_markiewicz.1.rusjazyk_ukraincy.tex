% vim: keymap=russian-jcukenwin
%%beginhead 
 
%%file 22_10_2021.fb.przemyslaw_lis_markiewicz.1.rusjazyk_ukraincy
%%parent 22_10_2021
 
%%url https://www.facebook.com/permalink.php?story_fbid=4681979711823158&id=100000335259707
 
%%author_id przemyslaw_lis_markiewicz
%%date 
 
%%tags identichnost',jazyk,mova,ukraina,ukraincy,ukrainizacia
%%title НЕ ІСНУЄ РОСІЙСЬКОМОВНИХ УКРАЇНЦІВ. КРАПКА
 
%%endhead 
 
\subsection{НЕ ІСНУЄ РОСІЙСЬКОМОВНИХ УКРАЇНЦІВ. КРАПКА}
\label{sec:22_10_2021.fb.przemyslaw_lis_markiewicz.1.rusjazyk_ukraincy}
 
\Purl{https://www.facebook.com/permalink.php?story_fbid=4681979711823158&id=100000335259707}
\ifcmt
 author_begin
   author_id przemyslaw_lis_markiewicz
 author_end
\fi

НЕ ІСНУЄ РОСІЙСЬКОМОВНИХ УКРАЇНЦІВ. КРАПКА

Читаю собі різні відгуки після мого вчорашнього діалогу з Мананою Абашидзе. Не
всі, звісно, схвальні. Обурені, як завжди, так звані «російськомовні українці»,
бо я їх не визнаю українцями. Авжеж. І ніколи не буду, бо немає такої категорії
національності, як «неукраїнськомовні українці». Ви – просто росіяни і годі. Ви
так любите торочити про «исторически сложилося» і воно має виправдовувати вашу
неукраїнськомовність у 2021 році, на 8 рік війни.

Я вам усе розтолкую на прикладі Польщі. У нас ніколи не було «російськомовних
поляків», бо поляка зрусифікувати  - не можливо. Мабуть одиниці – так, але не в
таких масштабах, які ви собою в Україні представляєте, неуважаємиє малороси.
Але в Польщі ми мали інші явища, вони називалися: 1) велетенський, за всю
історію Польщі, наплив до Польщі німців, 2) Kulturkampf дев’ятнадцятого та
двадцятого сторіччя. Про що ідеться? До Польщі завжди прибували німці. Були
різні сплески міграції, зокрема за часів Казимира Великого (коли Краків був
німецькомовним містом), теж за часів Речі Посполитої, коли цією державою
правили шведи чи німці. Наплив німців до Польщі був найбільшим з Пруссії та
Саксонії. Це було і ще за часів РП, але теж і пізніше. Так виникло місто
Жирардів біля Варшави, засноване німцями, моє село, де я живу біля Познані,
було засноване 750 років тому німцями. З 40\% моїх предків – поховані на
кладовищах у селах і містечках заснованих німцями. Від 1795 року по 1939, 95\%
жителів західнопольських земель (Шльонськ, Великопольща, Помор’я) вільно
володіло німецькою мовою. Зокрема ближче до 1945. До 1918 року на тих теренах
початкова освіта німецькою була обов’язковою для всіх громадян. Німецька
культура була теж для більшості поляків привабливою. Зокрема філософія і
література. Польськи письменники (серед яких найбільш відомий – Stanisław
Przybyszewski) творили німецькою У Дрездені жив, писав і видавав свої історичні
романи Крашевський. Під враженням та впливом німецьких композиторів була
більшість наших видатних композиторів: Падеревський, Шимановський,
Нововейський. Найгірший германізаційний гніт почався на зламі 19 та 20 сторіччя
і дійсно почалася ненависть поляків до німців, яка вкрай посилилася після 1939
року.

Але навіть у ті часи на заході Польщі (домінація німецької мови стосувалася
здебільшого західної Польщі) не було «німецькомовних поляків», бо основною
рисою, яка відрізняла поляка від німця, була мова спілкування. Не було
«німецькомовних поляків», так як не було жінок з прутнями. Бо людина з прутнем
то чоловік, не жінка (зазвичай). Було досить багато ополячених німців (їх було
набагато більше ніж онімечених поляків, онімечені поляки напевне жили в
Німеччині, якщо були іммігрантами). Але зазвичай вони вже визнавали себе
поляками і це сильно підкреслювали після того, як Гітлер напав на Польщу.
Наприклад генерал Унруґ. Ополячений німець з Великопольщі, який уже давно
(майже все своє життя) вважав себе поляком. Був польським генералом, польською
розмовляв гірше ніж німецькою, але сказав, що 1 вересня 1939 року він забув, як
розмовляти німецькою. У таборі розмовляв з німцями через польського
перекладача, коли відвідували його родичі, розмовляв з ними англійською. Коли
почався Майдан, я – поляк – однозначно і категорично вирішив більше російською
(зокрема в Україні) не говорити. А ви – малороси? Коли настане ця дата для вас?
Ви будете верзти, що ви – українці, навіть якщо Путін буде в Києві, а ви будете
російською мовою висловлювати своє «фе». А наприклад львів’янин – Рудольф
Вайґль. Він був ополяченим німцем, професором університету. І коли німці
увійшли до Львова, його не розстріляли як інших польських науковців, але
пропонували йому гарні посади, він відмовляв. Не вбили його, бо він був їм як
вірусолог – потрібний. Коли йому пропонували підписати фолькслісту, він сказав:
«Людина раз за життя обирає собі національність, я уже вибрав», а коли
погрожували смертю, відповідав: «Я, як біолог, стикаюся з явищем смерті і вона
мені не страшна. Єдине, що можете вдіяти з моєю польською національністю, то
мене вбити. Але якщо я вам потрібний, мусите мене терпіти як польського
професора польської національності». Вам доходить, малороси? Чи ще більше
прикладів? Німецька мова після 1945 року (протягом 5 років) у Польщі ЩЕЗЛА. А
російська протягом 8 років не щезла в Україні через вас, ваш шовінізм, ваші
крики і ваше нестримне бажання нищити українство.

\begin{cmtfront}
\uzr{Костянтин Лях}

Цілком підтримую, інакше не може бути... Ці люди без Батьківщини, без прапора)
Дякую Лис Маркевич за чудову рецензію...
\end{cmtfront}

А ви – малороси? Навіть якщо ваші предки – етнічні росіяни, то ви мали би вже
давно зукраїнізуватися в Україні, а не російщити українців. Це не природно. І
настільки не природно, що вас треба називати незукраїнізованими росіянами
(малоросами), а не «російськомовними» українцями. Ви обурюєтеся, що я
категорично не даю вам права почувати себе «російськомовними українцями». Це не
я не даю права. Просто такого не існує. Птах чоловічої статі з обручкою
орнітолога на нозі не стає завдяки обручці одруженою людиною. Щур, який
народився у хліві, не стане вепром завдяки місцю народження. Баба з базару, яка
торгує яблуками, виноградом, бананами та гранатами не стане військовим, бо вона
має кавказькі гранати. Скрипалем не станеш, бо вмієш посмикати себе по
причандалах, тому що скрипка має смичок. Якщо ви хочете, аби вас називали
українцями, то заговоріть українською а російську покиньте назавжди, бо Імперія
Зла вбиває тепер щодня українців а частину території – окупує.

\begin{cmtfront}
\uzr{Irena Jasinowska}

Або українець, або ж українець, а то все решта МОСКАЛИКИ.... До сьогодні
ненавиджу цю мову, а особливо, як поляки зачинають говорити до тебе на
російській, то аж мною натягає(
	
\end{cmtfront}

\ii{22_10_2021.fb.przemyslaw_lis_markiewicz.1.rusjazyk_ukraincy.cmt}
