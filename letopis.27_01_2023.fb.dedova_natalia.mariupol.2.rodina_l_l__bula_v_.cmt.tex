% vim: keymap=russian-jcukenwin
%%beginhead 
 
%%file 27_01_2023.fb.dedova_natalia.mariupol.2.rodina_l_l__bula_v_.cmt
%%parent 27_01_2023.fb.dedova_natalia.mariupol.2.rodina_l_l__bula_v_
 
%%url 
 
%%author_id 
%%date 
 
%%tags 
%%title 
 
%%endhead 

\qqSecCmt

\iusr{Раиса Ильинична}

Это было все так.

\begin{itemize} % {
\iusr{Natalya Dedova}
\textbf{Раиса Ильинична} чекаю на Вашу історію. ❣️

\iusr{Раиса Ильинична}
\textbf{Наталя Дєдова} Збираюся.
\end{itemize} % }

\iusr{Anhen Maria}

Лілія сусідка моєї матусі. Це жах. Дом від моєго всього одна зупинка чотири
хвилини ходьби,а я не могла під обстрілами навіть пройти пів дороги. 8 березня
я дала ключик від квартири мами молодятам з маленькою дитиною, роздобула для
малечі памперси та хотіла свіжий коржик віднести. Так і залишилися памперси
вдома. Рада, що Ліля та родина живі та змогли виїхати. 🙏Це жахливо, що ми всі
пережили. 💔💔

\iusr{Ольга Самойлова}

А мені пощастило навіть обійняти Лілю тут, в Німеччині! Вона сама нас тут
знайшла, щедро накрила стіл і ми жадібно ділилися історіями виживання в нашому
багатостраждальному місті М!💔

Дякую вам Наталя за вашу надважливу справу і тобі Ліля за гостинність і таку
добру душу!🙏❤️🕊️

\begin{itemize} % {
\iusr{Liliya Liliya}
\textbf{Ольга Самойлова} ми радовались каждому Мариупольцу 😘😘😘. Только потеряв всё, начинаешь ценить каждую мелочь, каждую секундочку. Олечка, обнимаю 😘😘😘

\iusr{Natalya Tiktinskaya}
\textbf{Liliya Liliya} 

рада узнать, что вы выбрались! Мы тоже выбрались. Жаль, что многим не суждено
было выжить и выехать. Помнишь, с Катей и Димой в группе была Даша Горбенко? На
их дом упала бомба. Погибли Даша, ее мама(Вика) и младшая сестра(Яна).

26 января у Даши был бы День Рождения. Навсегда 22.

\iusr{Liliya Liliya}
\textbf{Наталья Тиктинская} приветик, Наташенька! Я очень рада что вы живы и здоровы! 🙏🙏🙏🙏мы уже выехали.

Помню, царствие небесное Даше!🙏🙏🙏🙏

ещё погиб Артём Кравченко. Они вместе и в садике были и в школе. Катя просила
меня узнать о нём. Я несколько раз ходила на Кронштадтскую, где он жил с
бабушкой, и в один день смотрю... Идёт мне на встречу бабушка его. Я ей так
улыбаюсь. Быстро иду на встречу ,говорю ей, как я рада, что вы живы! Как Артём?
Девчонки (Катя и Маришка) так переживают за него. А она бросается мне в
объятия, так рыдает... Я передать не могу.. Мурашки... Артём погиб... Вместе со
своей девушкой и ещё несколько друзей и их дети. В частном доме прямое
попадание в подвал. Это был ужас... Это же наши дети!! 22 года! Жить и жить!!
За что??

\end{itemize} % }

\iusr{Svetlana Marflyuk}

які щасливі люди! їх вже повернули в Україну! а ми тільки мріємо про це

\iusr{Анна Марухненко}

Как страшно это вспоминать, но именно так все было. Дочка эти коржики даже
ночью просила, говорила, что вкусные.

\begin{itemize} % {
\iusr{Natalya Dedova}
\textbf{Анна Марухненко} жду Вашу историю. 💛

\iusr{Виктория Тимченко}
\textbf{Анна Марухненко} Аня, расскажи Наталье свою историю.
\end{itemize} % }

\iusr{Ельвіра Та Григорій Ілюшенко}

Боже, бедные детки! Нам, взрослым, было тяжело и страшно, а как им...

\iusr{Ольга Птичка}

1.5 литровая бутылка воды на неделю ...

\iusr{Наталья Гринчак}
\textbf{Наталя Дєдова}, дай Бог Вам здоровья и сил, благодарю Вас 100000 раз за Ваше дело 🙏❤️

\iusr{Marina Pisarenko}

Саме так, коли побачили хліб їли і плакали і зараз крихти хліба збираємо і їмо.

Наш будинок тиждень бомбили, а ми у підвалі, взагалі не було можливості щось
приготовити, всі схудли на 10-15 кг.

Це ніколи не можливо забути, а тим більше пробачити.

\begin{itemize} % {
\iusr{Natalya Dedova}
\textbf{Marina Pisarenko} розкажіть свою історію. 💛

\iusr{Marina Pisarenko}
\textbf{Natalya Dedova} 

Навіть не знаю, дякувати богу ми залишилися живі, діти та наші дві матері, ми
виїхали 1 травня з міста, бо розуміли, що треба тікати інакше за буханку хліба
нас просто здадуть.

Життя в окупації та проходження фільтрації, це жахіття, особливо, коли я, як я
думала, ще працювала в міській раді, тому, збируся і звісно розповім.

\end{itemize} % }

\iusr{Viktoria Sü-g}

Обіймаю Ліля тебе міцно

\begin{itemize} % {
\iusr{Liliya Liliya}
\textbf{Viktoria Sü-g} нашлись 😘😘😘обіймаю 🤗
\end{itemize} % }

\iusr{Елена Сафонова Тимошенко}

Читаю історії і ллються сльози, сльози спогади всього що було там після 24
лютого, після того як прийшли нас \enquote{асвабаждать} Спогади як пекли хліб для
хлопців на блокпости (муку встигли зберегти зі своєї пекарні), варили їжу,
гарячий чай. А потім на вогнищі під обстелами смажили лаваш на вогнищі, щоб на
довше вистачило борошна, для старих і діток з двору.. Спогади за приліт і
контузію.

\begin{itemize} % {
\iusr{Natalya Dedova}
\textbf{Елена Сафонова Тимошенко} чекаю на Вашу історію. 💛
\end{itemize} % }

\iusr{Наталия Волосова}

Это страшные воспоминания.... Я знаю, как это страшно, когда ты не можешь
накормить ребёнка, не можешь объяснить почему всё это происходит и в чем дети
виноваты. А самое страшное было слышать от доченьки: \enquote{Мама, я не хочу
умирать...} Мы с доченькой и с семьёй моей сестры выехали из Мариуполя 15
марта, когда российские танки заезжали в город через 17-й микрорайон. Мы ехали
на машине с разбитыми стеклами, а на встречу нам двигалась колонна фашистских
танков. С нашей стороны это был огромный риск. Но мы не хотели оставаться в
этом аду. Мы готовы были рисковать ради наших детей, ради будущего. Уже прошло
10 месяцев, но ещё не было ни одного дня, чтобы мы не вспоминали то, что мы
пережили за эти 20 дней полной изоляции, ужаса, голода и страха за жизни наших
детей... Это навсегда останется в нашей памяти кровоточащей раной.

\begin{itemize} % {
\iusr{Natalya Dedova}
\textbf{Наталия Волосова} расскажите Вашу историю для проекта \#голосамирных.

\iusr{Наталия Волосова}
\textbf{Natalya Dedova} , да, конечно. Если это важно... Ведь таких историй сотни тысяч. И у каждого история по-своему страшная...

\iusr{Natalya Dedova}
\textbf{Наталия Волосова} важно. Пишу у приват.

\iusr{Натали Кина}
\textbf{Наталия Волосова} 

Ми 15 березня виїздили через Приморський мкрн. Багато р@шистів зустріли, у
вантажівках, але танки ні. Без скла уявляю собі,як ще й мерзли... В нас тільки
заднє вибило

\iusr{Юлия Биатова}
\textbf{Наталия Волосова} Тоже никогда не забуду слова дочери: \enquote{Мама, неужели я умру. Мне ведь всего 15}....Не забуду и не прощу

\iusr{Anna Shalimova}

Якось вночі, в березні, при черговому обстрілі та бомбардуванні моя доня
спитала мене \enquote{Мам, скажи, а сьогодні ми не помремо}? Я не знаю, як я
стрималася, щоб не завити вголос... Наступного дня ми поїхали в нікуди... Це
було 16 березня...

\iusr{Iryna Datsenko}
\textbf{Наталия Волосова} 

це важливо. Повірте, багато українців і людей з закордону мають почути кожну
історію виживших. Усе має бути задукоментовано для нашадків щоб пам'ятали хто є
ворог і чого від них чекати

\iusr{Ольга Генова Преображенская}
\textbf{Наталия Волосова},

мы выезжали из Мелитополя 17 марта и присоединялись к мариупольцам в Токмаке. С
нами в колонне ехали машины без стекол, с простреленным кузовами. На это было
страшно смотреть! Мы буксировали до Запорожья машину у которой закончился
бензин. В Запорожье в центре приема беженцев люди из Мариуполя просто молчали, а
в глазах-пропасть... Мы сидели за столом с ребятами, которые четыре дня шли
пешком с котом и собакой, пока их не подобрали на дороге. Не прощу никогда! Всегда
буду помнить ту дорогу до Запорожья с торчащими хвостами снарядов и сожжеными и
машинами, 12 блокпостов орков. На одном посте хотели застрелить моего Алабая и
провоцировали моего сына-подростка на конфликт! Самое отвратительное, что реакция
моих родителей и родни, живущих в России никакая! Не прощу безразличия! Не прощу
того горя, которое они принесли в мою страну! Очень жду освобождения и Победы!
Обнимаю!

\iusr{Наталия Волосова}
\textbf{Ольга Генова Преображенская}, 

да, это была очень страшная дорога на Запорожье. Мы ехали по объездной дороге ,
потому что перед Запорожьем был взорван мост. Мы ехали около двух часов в
страхе. Через час после нашего проезда по этой дороге обстреляли колонну машин
из Мариуполя, которые также как и мы хотели доехать до Запорожья. Погибло много
людей. Мы все очень верим в нашу ПОБЕДУ!!!! Вся Украина очень ждёт этот день! И
мы дождемся!!!!! Желаю всем нам мирного неба!!! Обнимаю всех!

\iusr{Natalya Dedova}
\textbf{Ольга Генова Преображенская} жду историю. 💙💛

\end{itemize} % }

\iusr{Elena Kozaeva}

Дуже знайоме.... Досі не розумію, як ми ще вижили, харчуючись підножним
кормом.... До війни перебирали, хліб не їли... Тепер їмо!😪

\begin{itemize} % {
\iusr{Liliya Liliya}
\textbf{Елена Козаева} 

до війни ми і рибу з душком навіть не вім котам давали їсти, щоб не потрвилися..
А під час війни на душок не звертали увагу, И те що впала на брудний пол, де
лазити щури - з'їли, бо вибору не було...

\iusr{Tania Tainna}

Конечно, рассказать про кота из Базмута, что попал в конце ноября ко мне, это
мелочь.

Хо Через 25 руки. 8 лет, домашний любимец, разве что филешку без соли и специй
варёную ел.  Сейчас лопает почти все.. кашу, лук с морковкой, бисквит.. лишь бы
чуточку пахло вкусно и годится.  Лечим до сих пор ЖКТ(

\iusr{Марина Тимофеева}
\textbf{Liliya Liliya} как вы?, читаю и до слез. Я выбралась 6 апреля. У каждого своя жуткая история.
\end{itemize} % }

\iusr{Elena Novik}

После всего пережитого в Мариуполе я больше не могу выбрасывать остатки
продуктов в урну. Вспоминаю как эти кусочки чего-либо могли бы спасти людей от
голода😭

Голод, холод, обстрелы с воздуха, суши и моря...это всё очень тяжело
вспоминать🥺

\begin{itemize} % {
\iusr{Natalya Dedova}
\textbf{Елена Новик} чекаю на історію для \#голосимирних.

\iusr{Татьяна Гирина}
\textbf{Елена Новик} 

Вы правы, не могу выбрасывать теперь продукты, собака и кот помогают быть
дисциплинированными по остаткам еды!

Хлеб вообще табу, категорично запрещаю допускать порчу хлеба, контролирую каждый
кусочек, чтобы не пропал...

Каждый кусочек мог тогда спасти чью-то жизнь...

\iusr{Ирина Куцан}
\textbf{Елена Новик}, 

я тоже, просто не поднимается рука, сразу весь пережитый ужас перед глазами. В
этот момент вспоминаю свою прабабушку, она всегда рассказывала про военный
голод 2 мировой 😔, только тогда не могла понять и прочувствовать...

Какое же это страшное чувство, когда твои дети хотят есть и ты не понимаешь,
что делать ?! Безысходность ....

\end{itemize} % }

\iusr{Ирина Штучка}

Нас сусіди навчили готувати на багатті коржики з расолу, що залишився від
огірків чи помідор консервованих.... расол, вода, сода та борошно... діти
раділи дуже цій вишуканій страві....

\iusr{Яна Михейко}
\textbf{Liliya Liliya} 

а про розбомблену фуру з картоплею, яка була напівмерзла, напівзапечена чули?
Багатьох маріупольців ця картопля врятувала від голоду...

\iusr{Янбаева Янбаева}

Половина чашки воды утром...

\iusr{Галина Мотрук}

Как это страшно! Я с ужасом вспоминаю время бомбёжки и жизни там... До сих пор
удивляемся, как остались живы...

\iusr{Maya Levit}
💔

\iusr{Ella Soifer}

Сегодня уместно вспомнить и жертв Холокоста. как раз на международном уравне
вчера почтили память невинно убиенных. но эта трагедия ничему не научила
человечество. войны по всей планете - это ужасно. очередная жертва Украина... и
этому не будет конца?

\begin{itemize} % {
\iusr{Gregory Reikhman}
\textbf{Ella Soifer} Украина - жертва цепи военных преступлений рашизма
\end{itemize} % }

\iusr{Naomy Khaskin-piskunov}

Кошмар, который пережили эти люди!

\iusr{Ryslana Ryslana}
😭😭😭😭😭

\iusr{Марина Пушкина}

Я всегда боялась увидеть голодные глаза детей, это самое страшное что могло быть

\iusr{Олеся Шляпина}

Люди я вас просто всех люблю, мы ни когда этого не забудем.

\iusr{Леся Соколова}

Мы смогли вырваться только в конце июня оттуда. Выжили каким-то чудом. Голод и
холод, постоянные взрывы от падающих бомб на Азовсталь, подвал и уничтоженная
квартира, все это отбирало веру в то, что можно выжить. Мы тоже с февраля хлеб
впервые увидели только 28 апреля. Как страшно это вспоминать.

\begin{itemize} % {
\iusr{Natalya Dedova}
\textbf{Леся Соколова} расскажите Вашу историю для голосов мирных. 💙
\end{itemize} % }

\iusr{Елена Михайлюк}

Да, а мы хлеб получили 4 апреля, я никогда не думала что так его люблю. До сих
пор вспоминаю зеленый чеснок с солью.

\begin{itemize} % {
\iusr{Natalya Dedova}
\textbf{Елена Михайлюк} жду Вашу историю.
\end{itemize} % }

\iusr{Elena Bilous}

А наш під'їзд так сильно об'єднався у той час, і супчики якісь вигадували, і
якісь коржики жарили (замість хліба). Спочатку всі свої холодильники
спустошили, потім морозильні камери, потім всі шкафчики... гуртом, дякую кожному, за
підтримку, за те що були разом!

\begin{itemize} % {
\iusr{Natalya Dedova}
\textbf{Елена Билоус} чекаю Вашу історію. Хочу почути рецепти коржиків. ❣️

\iusr{Elena Bilous}
\textbf{Наталя Дєдова} яку саме історію, що саме, та у якому форматі Ви хочете почути? Трошки не зрозуміла.

\iusr{Natalya Dedova}
\textbf{Елена Билоус} як виживали в Маріуполі з 24 лютого. По Скайпу. Коли Вам буде зручно. Просто розкажіть свою історію виживання.

\iusr{Elena Bilous}
\textbf{Наталя Дєдова} зрозуміла, я напишу Вам

\iusr{Ольга Черемных}
\textbf{Елена Билоус}, и у нас так же. Только не подъезд а все кто были в подвале...
\end{itemize} % }

\iusr{Инна Шитова}

У меня чувство голода пропало, просто кусок в горло не лез вообще, спать тоже не
могла, иногда проваливалась в сон не надолго, первый раз уснула нормально 13
апреля, когда выбралась из этого ада.

\iusr{Polina Burmistrova}
Кожне слово у саме серце... немов переживаю ті дні ще раз...😪

\iusr{Оксана Спивак}

\ifcmt
  igc https://scontent-frt3-2.xx.fbcdn.net/v/t39.1997-6/319153374_878828900219928_9087338544983114164_n.webp?stp=dst-webp_s180x540&_nc_cat=100&ccb=1-7&_nc_sid=ac3552&_nc_ohc=aca3_bU7VHYAX-hU6gd&_nc_ht=scontent-frt3-2.xx&oh=00_AfD7RYeR5dIZ3yrufZ9Qpban-AgeBlMDHUiW1LO0fnOm8w&oe=640E5590
	@width 0.1
\fi

\iusr{Valentina Dmitrieva}

Апокаліпс у моєму рідному Маріуполі... Душать сльози 😪

\iusr{Адвокат Юлія Башкірова}

Я помню, когда я услышала первый раз запах свежеиспеченного хлеба) думала
потеряю сознание от этого запаха ((( В тот момент я вспоминала бабушку, которая
пережив войну и ссылку в Германию всю оставшуюся жизнь сушила сухари и хранила
их

\begin{itemize} % {
\iusr{Валентина Вайновская}
\textbf{Адвокат Юлія Башкірова} і я.......
\end{itemize} % }

\iusr{Tatyana Tserakhto}

До 24 февраля хлеб я практически не ела. Домой покупала 2-3 днепровские булки.
Иногда хватало на неделю. А вот потом.... Потом тоже не хотелось. Есть не
хотелось. Воротило от запаха дыма, которым пропитывалась еда, одежда и волосы.
Ела чтобы выжить, чтобы были силы. Но когда пришлось спуститься в подвал дома
спасаясь от бомбежки нашла пакет с маленькими сухариками, оставленными
выехавшей ранее соседкой - обрадовалась очень. вкуснее тех сухариков с
абрикосовым вареньем я с тех пор ничего не помню))

\begin{itemize} % {
\iusr{Natalya Dedova}
\textbf{Tatyana Tserakhto} расскажите Вашу историю выживания. Для проекта Голоса мирных. По Скайпу.

\iusr{Tatyana Tserakhto}
\textbf{Наталя Дєдова} я не пользуюсь скайпом

\iusr{Natalya Dedova}
\textbf{Tatyana Tserakhto} його можна встановити. Ми пишемо тільки через скайп. Такі технічні вимоги.

\iusr{Tatyana Tserakhto}
Тогда нет

\end{itemize} % }

\iusr{Ольга Черемных}

А мои дети с голоду ели хлеб с плесенью вернее уже сухари с плеснью и очень ему
радовались особенно мой годовалый сын в темноте и холоде с маленьким кусочком
сухарика так как кормить их было нечем и весь подвал ел 1 раз в день супчик и
лепешку которые готовили на костре под бомбёжками. Так и выживали и взрослые и
дети...

Этот ад не забудем и не простим...

\iusr{Olga Grakovsky}

Просто ужас что пережили люди и есть ещё сейчас люди в таком положении. За такие
страдания кремлевским преступниками прощать нельзя. Они должны понести суровое
наказания.
