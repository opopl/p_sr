%%beginhead 
 
%%file 12_04_2023.fb.fb_group.750_hudozhnykiv.1.mozhe_htos_pobachyt_svij_vytriv_verhovna_rada
%%parent 12_04_2023
 
%%url https://www.facebook.com/groups/1013559558679300/posts/6006084059426800
 
%%author_id fb_group.750_hudozhnykiv,makarova_kateryna.kyiv.designer.artist.magnokvit
%%date 12_04_2023
 
%%tags fb.hashtag.#фестивальписанок,pysanka,festival.kiev.pysanky,verhovna_rada,kiev
%%title Може хтось побаче свій витвір - виставка писанок біля Верховної Ради
 
%%endhead 

\subsection{Може хтось побаче свій витвір - виставка писанок біля Верховної Ради}
\label{sec:12_04_2023.fb.fb_group.750_hudozhnykiv.1.mozhe_htos_pobachyt_svij_vytriv_verhovna_rada}
 
\Purl{https://www.facebook.com/groups/1013559558679300/posts/6006084059426800}
\ifcmt
 author_begin
   author_id fb_group.750_hudozhnykiv,makarova_kateryna.kyiv.designer.artist.magnokvit
 author_end
\fi

Може хтось побаче свій витвір 💛💙

\href{https://www.facebook.com/verkhovna.rada.ukraine/posts/pfbid02XGUNKCieKZq9S3J62V9f4eVgjjUqXnQfFd8upSwhCvCE7CgCN4S7wdf5fLewewD8l}{%
У Верховній Раді України відкрилася виставка врятованої частини унікальних
писанок \enquote{Писанки відродження}, що демонструвалася у зруйнованому окупантами
\enquote{Добропарку} у Макарівському районі на Київщині, Верховна Рада України, facebook, 10.04.2023%
}

\ifcmt
  ig https://i2.paste.pics/a3e4e9591e365400a3b81657b81425b6.png
  @wrap center
  @width 0.8
\fi

\begin{center}
	\begin{fminipage}{0.9\textwidth}
\href{https://www.rada.gov.ua/news/razom/235113.html}{%
У Верховній Раді України відкрилася виставка врятованої частини унікальних писанок \enquote{Писанки. Відродження}, %
Прес-служба Апарату Верховної Ради України, rada.gov.ua, 10.04.2023%
}

У внутрішньому дворику Верховної Ради України відкрилася виставка врятованої
частини унікальної колекції писанок, що демонструвалася у зруйнованому
окупантами \enquote{Добропарку} у Макарівському районі на Київщині.

Презентовані в Українському Парламенті \enquote{Писанки. Відродження} були експонатами
Всеукраїнського фестивалю писанок у Києві, який бере свій початок із 2016 року.
Головною метою незвичайного арт-перформансу була підтримка і популяризація
українських народних традицій, розвиток народного мистецтва, вивчення культури
та історії Українського народу. 

У 2016 році 374 художники з усієї країни розписували метрові полотна у формі
яєць, які були представлені на Софійській площі у Києві. А вже у 2017 році
експозиція нараховувала 585 писанок.

У 2019-2022 роках постійним місцем виставки писанок став унікальний дендропарк
\enquote{Добропарк}, що на Київщині. До початку повномасштабного вторгнення тут
демонструвалося  більше 800 писанок та арт-об'єктів.

У перші дні війни ворог нещадно завдав удару по \enquote{Добропарку}, де перебувало у
цей час більшість писанок. Нині, з-понад 800 арт-об'єктів, вціліли лише 120,
частина з яких сьогодні експонується у внутрішньому дворику Верховної Ради
України.

\textbf{Довідково:} експозиція підготовлена в межах всеукраїнського проекту
PYSANKA.REBORN разом із комунікаційним агентством AVERINA та \enquote{Добропарк}
спільно з Апаратом Верховної Ради України.


	\end{fminipage}
\end{center}

%\ii{12_04_2023.fb.fb_group.750_hudozhnykiv.1.mozhe_htos_pobachyt_svij_vytriv_verhovna_rada.cmt}
