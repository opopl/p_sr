% vim: keymap=russian-jcukenwin
%%beginhead 
 
%%file slova.strana
%%parent slova
 
%%url 
 
%%author 
%%author_id 
%%author_url 
 
%%tags 
%%title 
 
%%endhead 
\chapter{Страна}
\label{sec:slova.strana}

Авіація давно лежить в основі економічного успіху багатьох країн світу, з
перших днів польоти допомагали налагоджувати міжнародні торгові зв’язки та
створювати життєво важливі внутрішні зв’язки, які \enquote{зшивають} \emph{країну}. В
авіації, як і в інших галузях, які знаходяться не в фокусі топ-чиновників, за
роки накопичилося багато проблем.  Сьогодні ми отримали реальний шанс
кардинально покращити ситуацію в галузі, адже Міністром інстраструктури став
Олександр Кубраков, людина яка не на словах знає, що таке дійсно будувати
\emph{країну} кожного дня, під керівництвом якого було збудовано і реконструйовано
близько 9 тис. км доріг тільки державного значення,
\citTitle{Сім кроків до державної політики розвитку та розбудови української авіації}, Георгій Зубко, www.epravda.com.ua, 08.06.2021

\citEntry{%
По той же причине: нет ясности со \emph{страной}. Будет или нет? В поведении
власти прочитывается четкий сигнал: она сама не верит в жизнеспособность
государства, которые управляет. А потому занимается тотальным ограблением
остатков, распродавая все за бесценок.  Четвертое - все отвлекающие
общественное внимание маневры власти, типа санкций, суда над Медведчуком и
партией Шария, референдумы по Донбассу и статусу олигархов и пр. и пр. и пр.
имеют очень короткий ресурс эффективности. И к осени, когда люди увидят рост
цен на все и вся в полтора раза, едва ли станут предохранителем от массовых
выступлений. Возможно - бунта. Ресурс терпения у пересичных близится к нулю. С
переходом в отрицательные значения по отношению к Зе-банде. Так что вангую
смену власти осенью этого - зимой-весной следующего года
}{%
\citTitle{На глазах нарастает грандиозный кризис украинской государственности}, Валерий Песецкий, strana.ua, 09.06.2021
}

Прямо на наших глазах происходит беспрецедентный дерибан \emph{Страны}.  Одноходовые
схемы Блогера. Здравствуй, Люба. Я вернулся. Три месячных бана в этом году, а
ведь на календаре только июнь. Каждый второй месяц в карцере. 100\%-е
показатели пятилетки Петра.  Хотя какие тут могут быть претензии? «Нормы
сообщества ФБ» - это правила гопников. Это их площадка и их район, и ты всегда
виноват уже тем, что не вышел мордой.  Только давайте не забывать базовое.
Черные списки неугодных формируют в Офисе Зеленского. И это его шарашка
непосредственно стоит за банами оппонентов. А не забывать будем ради главного.
Когда «настанет день, настанет час», и мы это чучело, как ранее его
предшественника, выкинем на помойку.  Правда, когда этот «день и час»
настанет, подозреваю, у нас тут уже останется Мозамбик.  Пацанва даже не
утруждает себя выдумыванием хотя бы двухходовых схем по распилу \emph{страны}. Толпа
мародеров, которая выносит торговый центр, на их фоне смотрится более
респектабельно,
\citTitle{Эта власть построила нам кунсткамеру в смартфоне}, Игорь Лесев, strana.ua, 09.06.2021

Вот так \emph{Страна} и скатывается к краю финансовой пропасти. СУПЕРХЛЕБНАЯ
ИДЕЯ ДЛЯ УКРАИНСКИХ АГРОБАРОНОВ, КАК ПО-МАКСИМУМУ ОБОГАТИТЬСЯ И ОДНОВРЕМЕННО
ОБНУЛИТЬ ГОСБЮДЖЕТ-2021. Они нас тащат к краю самой глубокой финансовой
пропасти; такое ощущение, что завтра уже не будет. Судите сами, только один ряд
беспристрастных цифр из открытой отчетности крупных экспортеров агропродукции,
свидетельствующий о минимальных размерах уплаченных ими налогов и максимальных
заработках на возмещении НДС. Вот данные Государственной налоговой службы о
том, насколько больше денег из госбюджета вернули экспортерам АПК, и сколько
они заплатили налогов,
\citTitle{Государственная налоговая служба обнародовала шокирующие цифры возмещения НДС}, Александр Гончаров, strana.ua, 09.06.2021


%%%cit
%%%cit_pic
%%%cit_text
Дело в том, что маленьким по территориям \emph{странам} значительно легче организовать
свою жизнь, найти общественный консенсус, разработать привлекательные
инвестиционные идеи. Крохотный Сингапур с населением 5 млн чел генерирует 102
тыс дол на душу. Еще эффективнее такие \enquote{малыши} как Катар и Кувейт.
Процветают и сравнительно небольшие Швейцария и Тайвань.  \emph{Маленькой стране}
проще создать эффективную налоговую систему и привлечь капитал.  Несколько лет
назад российская IT-компания разработчик игр Game Insight перенесла офис в
Литву. Ее выручка — \$100 млн в год и почти вся она в твердой валюте. Для Литвы
с ее 2.8 млн чел населения это была огромная сумма. Если бы этому примеру
последовала \enquote{Лаборатория Касперского} с выручкой в \$667 млн (2013
год), то Литва закрыла бы треть дефицита торгового баланса (€1,9 млрд). Еще
пара таких компаний, и нет проблемы
%%%cit_comment
%%%cit_title
\citTitle{Мы живем в мире, когда размер перестал иметь значение}, 
Андрей Головачев, strana.ua, 13.06.2021
%%%endcit

%%%cit
%%%cit_head
%%%cit_pic
%%%cit_text
Я думаю, \emph{Страна}, в которой сегодня не удалось еще создать глупого
человека - робота, это Россия. Нам постоянно нужно поделать что-то для души,
потому что хочется. Находясь на работе, в большинстве своем мы думаем только об
одном, когда этот чертов день закончится и побыстрее уже свалить отсюда. Для
нас работа находится далеко не на первом месте. В первую очередь для нас важно
пожить, вечно поспорить, что-то обсудить, в целом чувствовать себя живым.
Россия, как и во все времена, сегодня полна яркими и живыми личностями, каждая
из которых имеет свой уникальный характер. Начиная от простого работяги и
вплоть до нерадивых чиновников. Вот это всегда и напрягало запад - русская
непокорность и Воля. Иначе говоря - проявление характера.  Вот друзья, в чем
заключается по мнению брата главное отличие человека русского от человека
запада. И я с ним пожалуй соглашусь
%%%cit_comment
%%%cit_title
\citTitle{После 11 лет жизни в США понял самое важное на мой взгляд отличие между американцами и русскими}, 
За Пределами Онлайна, zen.yandex.ru, 06.06.2021
%%%endcit

