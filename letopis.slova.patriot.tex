% vim: keymap=russian-jcukenwin
%%beginhead 
 
%%file slova.patriot
%%parent slova
 
%%url 
 
%%author 
%%author_id 
%%author_url 
 
%%tags 
%%title 
 
%%endhead 
\chapter{Патриот}

%%%cit
%%%cit_pic
%%%cit_text
Главное - не поступились принципами!  Оскандалиться везде - главная линия
поведения \emph{ура-патриотической} тусовки на международной арене. Пренебрегать
установленными правилами, если они мешают из любого международного мероприятия
делать перформанс для своей секты в социальных сетях - постоянный принцип. От
позорных выходок Гончаренко в ПАСЕ до шума по поводу националистического клича
на спортивной форме.  И если к ужасу настоящих \emph{патриотов} своей страны наша
сборная запомнится на Евро-2020 только этим, то та секта этому даже не
огорчится - зато они «не смогли поступиться принципами»
%%%cit_title
  
%%%endcit


%%%cit
%%%cit_pic
%%%cit_text
Трудно было не заметить, как старательно изображал из себя \emph{ультрапатриота} глава
федерации футбола Павелко. И трудно было с этого не поржать, понимая, что
мутный постсоветский бизнесмен Павелко космически далек от всей этой
бандеровщины. Его настоящее отношение к ней – что-то между равнодушием и
презрением
%%%cit_comment
%%%cit_title
\citTitle{Сев на украинские потоки, так важно одеть вышиванку},
Вячеслав Чечило, strana.ua, 13.06.2021
%%%endcit

%%%cit
%%%cit_pic
%%%cit_text
Теперь это возвращается бумерангом. Нищая перенаселенная Индия стала идеальным
инкубатором для формирования нового, более заразного штамма, который уже пришел
в Европу и США, провоцируя новую волну пандемии.  Впрочем, фармацевтические
корпорации от этого не расстроятся, а \emph{украинские патриоты} все равно не снимут
запрет на «Спутник». Потому что жизнь и здоровье простых людей заботят их
меньше всего на свете
%%%cit_comment
%%%cit_title
\citTitle{Индийский штамм коронавируса не испугает украинских патриотов}, 
Андрей Манчук, strana.ua, 13.06.2021
%%%endcit

%%%cit
%%%cit_head
%%%cit_pic
%%%cit_text
Послышались претензии от \emph{патриотов}, которые специализируются на видеоиграх.
Сообщество GameStreetUA, которое выступает за украиноязычную локализацию игр,
написало критический пост о \enquote{Сталкере-2} и его отношению к украинскому
языку. \enquote{Кто как не украинская компания должна понимать вред от
обрусевшей (в оригинале - зросійщеної) передачи украинских топонимов?}, -
задаются вопросом активисты. \enquote{И русским языком в трейлере, и
подзаголовком игры украинская компания способствует распространению ложных
стереотипов о российском Чернобыле среди иностранцев}, - бьет тревогу паблик.
Правда, он считает, что GSC сделала это \enquote{невольно}. Но тем не менее
авторам поста \enquote{грустно}. Они призывают исправить \enquote{ошибку} до
релиза игры
%%%cit_comment
%%%cit_title
\citTitle{Сталкер 2 Сердце Чернобыля - почему хейтят видеоигру украинских разработчиков}, 
Екатерина Терехова, strana.ua, 16.06.2021
%%%endcit

