% vim: keymap=russian-jcukenwin
%%beginhead 
 
%%file 15_05_2021.fb.felbush_olena.1.2014_maidan
%%parent 15_05_2021
 
%%url https://www.facebook.com/olena.felbush/posts/4098485480228477
 
%%author 
%%author_id 
%%author_url 
 
%%tags 
%%title 
 
%%endhead 
\subsection{Тогда, в начале 2014 я была до смерти напугана}
\Purl{https://www.facebook.com/olena.felbush/posts/4098485480228477}

Тогда, в начале 2014 я была до смерти напугана. Волна чудовищного насилия, -
чреватого реальным фашизмом, - катилась от Майдана и Киевской обл. по всей
Украине.

Я думала о том, что можно написать Марго в Париж - историку фашизма - чтоб
большой мир знал, что здесь происходит.

Я услышала новость о том, что Путин не исключает возможности введения войск, и,
да, надеялась... Ибо полагала, что военные всё-таки живут по закону, а то, что
творилось тогда в постмайданье все мыслимые и немыслимые законы рушило.

Я старалась быть осторожной. На мне была мама... У меня была Философия. Но...

Что-то во мне взметнулось против этой осторожности и за то, ради чего она была.
И я просто сделала этот пост, потому что знала, ни осторожность не поможет, ни
неосторожность не спасет... Но - в неосторожности больше Истины.

Тогда я ещё верила в Донбасс, в себя, в Высшие силы... И потому сказала той
Украине то, что хотела и должна была сказать. Тогда я ещё так не боялась.

А теперь - я боюсь. Боюсь того, что со мной могут сделать в ДНР. Моей столице,
моей республике, моей любимой Родине. Больше всего боюсь...

Да, только теперь я нашла в себе силы признать - Донбасс погибнет. И только
теперь я смогла смириться с тем, что и я, как я надеялась - его лучшее детище,
сгину просто.

Я любила свою маму. Я любила свою Родину. Я была преданна своему призванию.

А теперь я просто боюсь. Открыть рот и сказать правду. Умирать под пытками. Сдохнуть как дерьмо.

С Оксана Тимофеева и Евгения Бильченко.

\url{https://www.facebook.com/olena.felbush/posts/611584902251903}
05-03-2014
Пост для тех моих ФБ-друзей, кто всерьез задается вопросом о том, а что же делать с этим Юго-Востоком.

Я родилась и выросла в Донецке и горжусь этим. Я наполовину немка и наполовину
русская, и этим я горжусь тоже. Я никогда не была сексуально, национально,
расово или религиозно озабоченной, а потому я с трудом понимаю людей, у которых
пунктик на какой-либо подобной почве. Мне всегда было присуще то, что
называется «гражданские добродетели», а потому мне трудно иметь дело с людьми,
которые плохо себе представляют, что это такое. Я получила неплохое
образование, о чем вы можете справиться в моем аккаунте, и я далеко не дура,
что, я надеюсь, вы заметили по публикациям в моей хронике. Но, главное – я
всегда сама решала, в каком направлении развивать свои способности и каким
образом приумножать свою культуру, я сама формировала свою систему ценностей и
культивировала свою духовность.  Я – типичная дончанка, поэтому со мной легко
договориться по любому вопросу. Но такую, как я – проще убить, чем изменить.
