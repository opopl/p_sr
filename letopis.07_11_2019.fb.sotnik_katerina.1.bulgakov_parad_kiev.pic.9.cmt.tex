% vim: keymap=russian-jcukenwin
%%beginhead 
 
%%file 07_11_2019.fb.sotnik_katerina.1.bulgakov_parad_kiev.pic.9.cmt
%%parent 07_11_2019.fb.sotnik_katerina.1.bulgakov_parad_kiev
 
%%url 
 
%%author_id 
%%date 
 
%%tags 
%%title 
 
%%endhead 

\iusr{Катерина Сотник}

– «Крутясь, волнуясь, напирая, давя друг друга, лезли к балюстраде, стараясь
глянуть в бездну собора, но сотни голов, как желтые яблоки, висели тесным,
тройным слоем. В бездне качалась душная тысячеголовая волна, и над ней плыл,
раскаляясь, пот и пар, ладанный дым, нагар сотен свечей, копоть тяжелых лампад
на цепях. Тяжкая завеса серо-голубая, скрипя, ползла по кольцам и закрывала
резные, витые, векового металла, темного и мрачного, как весь мрачный собор
Софии, царские врата. Огненные хвосты свечей в паникадилах потрескивали,
колыхались, тянулись дымной ниткой вверх. Им не хватало воздуха.

В приделе алтаря была невероятная кутерьма. Из боковых алтарских дверей, по
гранитным, истертым плитам сыпались золотые ризы, взмахивали орари. Лезли из
круглых картонок фиолетовые камилавки, со стен, качаясь, снимались хоругви.
Страшный бас протодиакона Серебрякова рычал где-то в гуще. Риза, безголовая,
безрукая, горбом витала над толпой, затем утонула в толпе, потом вынесло вверх
один рукав ватной рясы, другой. Взмахивали клетчатые платки, свивались в жгуты.

— Отец Аркадий, щеки покрепче подвяжите, мороз лютый, позвольте, я вам помогу.

Хоругви кланялись в дверях, как побежденные знамена, плыли коричневые лики и
таинственные золотые слова, хвосты мело по полу.

— Посторонитесь...

— Батюшки, куда ж?

Толстый, багровый Толмашевский угасил восковую, жидкую свечу и камертон засунул
в карман. Хору в коричневых до пят костюмах, с золотыми позументами, колыша
белобрысыми, словно лысыми, головенками дискантов, качаясь кадыками, лошадиными
головами басов, потек с темных, мрачных хор. Лавинами из всех пролетов, густея,
давя друг друга, закипел в водоворотах, зашумел народ.

Из придела выплывали стихари, обвязанные, словно от зубной боли, головы с
растерянными глазами, фиолетовые, игрушечные, картонные шапки. Отец Аркадий,
настоятель кафедрального собора, маленький щуплый человек, водрузивший сверх
серого клетчатого платка самоцветами искрящуюся митру, плыл, семеня ногами в
потоке. Глаза у отца были отчаянные, тряслась бороденка.

\obeycr
— Крестный ход будет. Вали, Митька.
— Тише вы! Куда лезете? Попов подавите...
— Туда им и дорога.
— Православные!! Ребенка задавили...
— Ничего не понимаю...
— Як вы не понимаете, то вы б ишлы додому, бо тут вам робыть нема чого...
— Кошелек вырезали!!!
— Позвольте, они же социалисты. Так ли я говорю? При чем же здесь попы?
— Выбачайте.
— Отлитургисали, можно сказать...
— На каком же языке служили, отцы родные, не пойму я?
— На божественном, тетка.
— От строго заборонять, щоб не було бильш московськои мовы.
— Что ж это, позвольте, как же? Уж и на православном, родном языке говорить не разрешается?
— Большевика держить, козаки! Шпиён! Большевицький шпиён!
— Це вам не Россия, добродию.
— Ох, боже мой, с хвостами... Глянь, в галунах, Маруся.
— Иван Иванович, тут бы полиции сейчас наряды, помните, бывало, в двунадесятые праздники... Эх, хо, хо.
— Николая вам кровавого давай? Мы знаем, мы все знаем, какие мысли у вас в голове находятся.
— Отстаньте от меня, ради Христа. Я вас не трогаю.
— Геть! В Россию! Гэть з Украины!»
(М.Булгаков. Белая гвардия).
\restorecr
