% vim: keymap=russian-jcukenwin
%%beginhead 
 
%%file 03_11_2021.fb.fb_group.story_kiev_ua.1.kiev_i_piter.cmt
%%parent 03_11_2021.fb.fb_group.story_kiev_ua.1.kiev_i_piter
 
%%url 
 
%%author_id 
%%date 
 
%%tags 
%%title 
 
%%endhead 
\subsubsection{Коментарі}

\begin{itemize} % {
\iusr{Катя Бекренева}

Мой муж ездил на Кировский в командировки, в то время его родной брат был в
Питере почти год на курсах при ЛПИ. Рассказывал как гуляли по городу, а потом
"брали Зимний ") Город очень красивый.


\iusr{Валентина Луконина}

Чудесно написано. А о парадных - вообще бальзам на раны! Спасибо, Пётр!

\iusr{Анжелика Ангел}
В Питере говорят "парадная".

\begin{itemize} % {
\iusr{Юрий Никитин}
\textbf{Анжелика Ангел} увы, не все. Особенно в спальных районах или в новых «человейниках» все реже и реже услышишь это слово.

\iusr{Анжелика Ангел}
\textbf{Юрий Никитин} просто в спальных районах нет "черных" лестниц)
\end{itemize} % }

\iusr{Ольга Сергиенко}
Люблю родной Киев, а Питер моё место силы. Мне его не хватает очень сейчас

\iusr{Наталья Емельянова}
Есть еще один город, где употребляют слово "парадное". Это Одесса.

\begin{itemize} % {
\iusr{Петр Кузьменко}
\textbf{Наталья Емельянова}, не зря она Южная Пальмира. И "сюдой", в моём рассказе услышанное в Ленинграде, оттуда.

\iusr{Larisa Olesyuk}
\textbf{Петр Кузьменко} вообще, если услышали тудой-сюдой, то не от коренного жителя Северной Пальмиры

\iusr{Петр Кузьменко}
\textbf{Larisa Olesyuk} , не спорю. Но, приятно! @igg{fbicon.wink} 

\iusr{Елена Калинина}
\textbf{Наталья Емельянова} разве что кто- нибудь понаехавший.

\iusr{Татьяна Досаева}
Слово парадное употребляли в Николаеве и в ..... Ташкенте

\iusr{Петр Кузьменко}
\textbf{Татьяна Досаева}, про Николаев не знаю. А про Ташкент узнаю у лучшего друга, коренного Ташкентца.
\end{itemize} % }

\iusr{Екатерина маковецкая}
спасибо)))) спасибо)))) моя бабушка ленинградка, а прожила всю жизнь на Подоле)))) и мое сердце на Андреевском спуске и на Невском)))

\iusr{Леся Сагайдачная}

Когда как говорю, могу подъезд, могу парадное. В Питере парадная. Но не суть) я
почему то никогда его не любила и схожести с Киевом не ощущала. В Москве - да,
в Питере нет. Не знаю почему.

\begin{itemize} % {
\iusr{Петр Кузьменко}
\textbf{Леся Сагайдачная} , а я не люблю Москву. Совсем. Не ощущаю себя там комфортно. Не город совсем. Какой-то базар-вокзал.

\iusr{Леся Сагайдачная}
\textbf{Петр Кузьменко} ну я там бывала когда ещё и Арбат был. И иная Москва. Но и Киев изменился. Просто в Москве половина семьи живёт. Хотя в Питере друзья, я действительно не могу это объяснить

\iusr{Оля Лемишко}
\textbf{Петр Кузьменко} Согласна с Вами- Москва холодная, спешащая, не уютная. А Питер какой-то светлый, добрый, приветливые лица. Возможно сейчас все по другому, ведь я и там , и там была ещё в 80.

\iusr{Леся Сагайдачная}
\textbf{Оля Лемишко} а вот у меня с точностью до наоборот. Питер холодный, а Москва широкая что-ли)) и весёлая. Но опять таки, давно всё было. Уже в 90-х иначе всё стало

\iusr{Оля Лемишко}
\textbf{Леся Сагайдачная} Возможно это ещё и зависит от того, с кем нам пришлось общаться ( родственников и близких друзей в это число не включаем), время года и даже погоду. Хорошей ночи Вам!

\iusr{Леся Сагайдачная}
\textbf{Оля Лемишко} а возможно)) да) хотя в Питере у меня очень хорошие друзья были из рок.клуба их. И Сайгон)) вам тоже
\end{itemize} % }

\iusr{Larisa Olesyuk}

Больше спасибо за воспоминание и такой доброжелательный пост. Очень тронута. В
юности для меня существовали три города: Киев, Ленинград и Москва. И я ездила
по этому Бермудскому треугольнику и наслаждалась тройной красотой.

\iusr{Мария Бойчук}

И так будет! И сейчас ездят те, кто знает, что их здесь любят и ждут. И не
только из Питера, но из Москвы и др. городов России. Постепенно народ
прозревает, сбрасывает с лица вуаль черной пропаганды, которая все еще пытается
пудрить мозги людей. Но здравый рассудок побеждает. Ничто не вечно под Луной.
Пройдет и это...

\begin{itemize} % {
\iusr{Мария Бойчук}

Почему? Я знаю многих россиян, которые уже открыли глаза и были просто удивлены
тем, как здесь у нас, в Киеве, все не так, как им рассказывают. Это от чистого
сердца. Я не понимаю, что Вам не понравилось?

\end{itemize} % }

\iusr{Iryna Morozovs'ka}

В 1998 была в круизе Москва-С.Петербург-Москва. В Питере поразили две вещи, не
считая достопримечательностей. Во-первых, к кому бы мы ни обращались, все не
просто показывали дорогу, а становились нашими гидами, причем совершенно
бесплатно и очень доброжелательно. А во-вторых, огромное количество пьяных
мужчин, женщин, молодежи и даже подростки и это было очень печально видеть.
Сейчас уже нет, мою эмпатию вырубили, когда "паребрик" пришел в мой Донецк и
видеть его в Киеве не горю желанием. Не может быть миролюбивого
демократического города в ортократическом государстве.

\iusr{Леся Сагайдачная}
\textbf{Iryna Morozovs'ka} Угу

\iusr{Алексей Добрынин}
черный пес петербург...

\iusr{Петр Кузьменко}
\textbf{Алексей Добрынин} ... морда на лапах...

\iusr{Jarek Yaroslav}

мы ездили с мамой на круизе по Скандинавии и у нас была остановка в Санкт Петербурге - город волшебный.
и я помню белые ночи это просто сумасшествие - энергия так прет хочется всю ночь гулять, творить, жить  @igg{fbicon.heart.growing} 

\iusr{Olena Elovskaya}

А мы тоже с классом ездили жили в школе на Электросиле. Зимой на Новый год.
Тогда в городе стоял жуткий мороз —40*,а в поезде, когда мы ехали было —10* (
трубу отопления прорвало). Замерзли, но было очень интересно. Кстати в
Ленобласти есть местность под названием Оболонь.

\iusr{Ольга Сергиенко}
Спасибо за этот пост. Это не модно сейчас. Но так не хватает.

\iusr{Константин Богуславский}

Питер и Киев и правда похожи. Для меня в Киеве есть огромный минус. Я когда
приезжаю в Киев и вижу это у меня даже портится настроение. Люди зачем-то
массово уродуют прекрасные старинные здания жлобскими самодельными балконами.

В Питере этого нет. Вообще. То есть совсем. В Питере нет старинных зданий, где
человек захотел и повесил себе балкон из белого говно-пластика. Речь про центр
и про старые здания, естественно. Ну может где-то и есть, но я как-то не
встречал.

\begin{itemize} % {
\iusr{Лариса Кушниренко}
\textbf{Konstantin Boguslavskiy} . 

Не представляю прекрасные здания Питера изуродованные пластиком застеклённых
царьбалконов. Еще не помню жутких графитти. Но, к сожалению, последний раз была
в 2013.

\begin{itemize} % {
\iusr{Ольга Кирьянцева}
\textbf{Лариса Кушниренко} 

таких жутких графитти и "самодельных" балконов нет, пожалуй, больше нигде. В
Новосибирске - точно. Если рисуют, то только в глубине дворов, на старых
заборах или эстакадах.

\iusr{Наталия Попко}
\textbf{Ольга Кирьянцева} 

В Новосибирске нет старых зданий(старый город -это особнячки), а здесь сносился
прилегающий дом и открывалась старая стена. Вид скажем не очень... Или старый
подъезд между домами, которому уже 150 лет... Здесь совсем другое...

\iusr{Ольга Кирьянцева}
\textbf{Наталия Попко} 

да, Новосибирску всего 130 лет. Но я о другом: люди здесь живут такие же, наши
современники, но почему-то они не уродуют центр города (да и окраины) нелепыми
балконами и уродскими рисунками.

\iusr{Наталия Попко}
\textbf{Ольга Кирьянцева} 

Ну, скажем ,застеклены балконы у Вас тоже по разному и сумасшествия с вагонкой у
вас нет. Это я согласна. Новосибирск очень похож на Киев, особенно район улицы
Челюскинцев-это как Соломинка у вокзала, где я выросла. Я была очень поражена.

\end{itemize} % }

\iusr{Лариса Кушниренко}

Для меня, города разные. Но, Петербург просто прекрасен. А то, во что
превращается центр Киева, вызывет у меня печаль. Особенно "радуют" застекленные
балконы с развешанными кондиционерами и антеннами, и графитти.

\end{itemize} % }

\iusr{Elena Kharchenko}

Моё детство до 7 лет проходило по переменно в этих двух городах. Так получилось .
Но Киев не сравнить по красоте и архитектурной гармонии ни с одним городом мира .
Ленинград холодный, абсолютно плоский.
Сейчас уже изменились оба не в лучшую сторону.

\iusr{Наталия Ковалева}
Спасибо за рассказ!

А я в 1987 году ездила с подругой почти на два месяца на курсы по
вычислительной машине "Роботрон"!

Побывали и на выставках и в музеях и в театрах и по пригородам поездили!

Жили на квартире, нам предоставили комнату, так еще и командировочные, потом
нам еще прислали по 120 рублей, я даже не представляю сколько бы это стоило.

В последние дни я еще поехала на электричке в Репино.

Как раз была красивая зима!!!

Ехали в Ленинград долго, сутки, а назад уже самолетом!

\iusr{Марианна Носаль}

Красивый город, а я к сожалению так и не съездила, собиралась в 2012, не вышло,
а потом как-то уже и страшно ездить для меня, не решилась.

\begin{itemize} % {
\iusr{Kit Gross}
\textbf{Марианна Носаль} 

я тоже боялся в Москву несколько лет назад лететь. Поехал. Никаких там проблем
не было. Нигде. Никаких косых взглядов, намеков, шуточек, хамства, злобы, хотя
во всех документах четко написано место рождения - Киев. В самом центре
гостинница Украина и на ней это написанно большими светящимися буквами, город
герой Киев на Красной Площади, вокзал Киевский и много чего другого
украинского. Повторяю, ни вражды, ни злости. Одно только отмечу, никто,
наверное на всякий случай, не поднимал тему политики, что правильно. Мы даже на
Украине друг с другом согласиться ни в чем не можем.

\iusr{Kira Dva}
\textbf{Марианна Носаль} 

абсолютно безопасно. Была в 2019 году. А когда перед отьездом, на набережной
Фонтанки, бросали монетки в Чижика- Пыжика, меня попросили перевести на
украинский знаменитое "Чижик-пыжик, где ты был?

\obeycr
На Фонтанке водку пил.
Выпил рюмку, выпил две
Закружилось в голове".
\restorecr

\end{itemize} % 

\iusr{Ludmila Debourdeau}

Благодарю вас за этот пост, такой тёплый и оптимистичный, очень понравилось.
Разделяю ваши чувства, хотя в Питере никогда не была. Не довелось. Но надеюсь
когда-нибудь для себя открыть этот прекрасный город Достоевского. Знаю его по
книгам, стихам, фильмам. Но вы правы, я думаю киевляне своей любовью схожи с
петербуржцами. Была у меня подруга-петербурженка, которая взахлёб с большой
любовью рассказывала о своём любимом городе, совсем как мы киевляне.

\iusr{Игорь Козиянчук}

Ау??? Какая похожесть??? Даже если сильно захотеть то не получится. Где в
Питере зелень? Парки? Скверы? Где горы, яры, буераки? Абсолютно
противоположность по растительности , по местности. Может быть по архитектуре?
Ровные выстроенные ряды в престольном городе. Центр культурв того времени -
эрмитаж, русский музей, кунскамера. Или может в Киеве есть хоть что-то похожее
на Кронштат??? Я уже не говорю про белые ночи и пронизывающий промозглый холод
рсенью и весной. Это два абсолютно разных города, мира, мировоззрения.

\begin{itemize} % {
\iusr{Алексей Добрынин}
то притензии к автору - Киеву 1500 лет Питер был тогда в сперматозоиде)))

\begin{itemize} % {
\iusr{Петр Кузьменко}
\textbf{Алексей Добрынин} , вообще-то Киеву гораздо больше 1500. Но,даже по Вашей, партийно-притянутой версии, уже 1539. Тут не о возрасте городов, об их ауре и особенностях.

\iusr{Алексей Добрынин}
\textbf{Петр Кузьменко} согласен с вами, был не прав ...))
\end{itemize} % }

\iusr{Петр Кузьменко}
\textbf{Игорь Козиянчук} , 

сколько людей, столько и мнений. Об этом я и написал в начале. А Кронштадт так
правильно пишется. Доводилось там бывать. Но, это уже другая история.

\begin{itemize} % {
\iusr{Игорь Козиянчук}
\textbf{Петр Кузьменко} ??? Два семиотически одинаковых места в городах. Стрелка Васьльевского острова и Влалимирская горка с Андреевским спуском. Где похожесть? Даже если оооочень напиться
Это как говорят, что русские, которые приезжают как топ звезда рф Пореченков, на сафари поубивать украинцев, наши братья. ( Если вдруг щакрадется сомнение - спросите"братика" что он сделал, когда увидел видео как во время перемирия Пореченков стреляет в Украинцев на украине? И вам ответят........ Ничего. Разве что слова. Но не перестали ходить в театр, где он играет и не перестали смотреть фильмы с ним, не байкотировпли убийцу)

\iusr{Петр Кузьменко}
\textbf{Игорь Козиянчук} те кто пришёл на нашу землю со злом и оружием не русские и не ленинградцы или москвичи. Нечисть не имеет национальности и территориальной идентификации. Или Вы до сих пор ассоциируете немцев и Германию с нацизмом, а итальянцев и Италию с фашизмом?

\iusr{Aleksandr Obukhov}
\textbf{Игорь Козиянчук}
Ублюдков с обеих сторон предостаточно.
Но мы на их территроию не ходили, однозначно.

\iusr{Петр Кузьменко}
\textbf{Aleksandr Obukhov} с этим полностью согласен. @igg{fbicon.hands.shake} 

\iusr{Игорь Козиянчук}
\textbf{Петр Кузьменко} 

немцы и итальянцы сейчас не поддерживают фашистских убийц. В отличии от
питерцев и москвичей, которые ежедневно платят гонорары убийцу Пореченко. И да.
Европейцам стыдно за прошлое. А русские восхваляют иои тихо молчат, делая вид,
что они не причем. Сколько процентов из русских выступили против войны россии
против украины? ( Это когда военные рф сбивали самолеты с украинцами, в гуково
со своей территории растреливали украинских военных)??? Единицы. А остальные...
Именно с их молчаливого согласия....

\iusr{Петр Кузьменко}
\textbf{Игорь Козиянчук} 

Вы вспоминаете Пореченкова, мелкую экранную мразь, и забываете о великом
Макаревиче, великолепной Ахеджаковой и талантливого Невзорова. Да, выступить
против существующей системы трудно. Это настоящие герои, а героев всегда мало.
Однако, посмотрите, по прошествии времени (оптимистично надеюсь, что мы с Вами
доживём до этого) россияне будут относится к нынешнему режиму и его
преступлениям так же, как сейчас итальянцы относятся к фашизму и немцы к
нацизму.

\iusr{Игорь Козиянчук}
\textbf{Петр Кузьменко} 

Да. Герои есть. Но их ничтожное меньшенство по сравнению с великими
Смоктуновскими, Баталовіми, Михалковіми. Уві.

\iusr{Игорь Козиянчук}
\textbf{Петр Кузьменко} 

Если бы хотя ыі каждый 20!!!! выступил против войны, против завхвата крыма...
Вы же прекрасно это понимаете, что войны бы не было. Не было бы смертей. А
так.... Молчат или поддерживают. И только единицы, как с ветрояными мельницами
пытаютс остановить кровь. Во времена фашизма единицы тоже выступали против. А
большинство молчало. И это молчание привело к трагедии, геноциду тогда и
приводит сейчас..

\iusr{Игорь Козиянчук}

И на всякий случай напомню, что из каждых 100 рулей, которые вы потратите в
Писере сейчас, 1 вернется к вам в в виде пули. И убьт или вас, или вашего
соседа. Увы. Россия сделала такую реальность. А каждый концерт в Питере
Макаревича - это целый снаряд, который к нам прилетит. Увы. Таков механизм
хунты. Не важно ты за или против, но полюбому кормишь монстра

\iusr{Петр Кузьменко}
\textbf{Игорь Козиянчук} 

Вы перечислили выживших из ума стариков, когда-то бывших кумирами, и добавили
сюда жополиза (извините за грубость) Михалкова. Но у них это фамильное.
Вспомните его папочку. Брат стыдился фамилии, пока тоже не вошёл в
соответствующий возраст...

\iusr{Петр Кузьменко}
\textbf{Игорь Козиянчук} с этим не спорю. Но, мой пост вовсе не об этом!

\iusr{Петр Кузьменко}
\textbf{Игорь Козиянчук} согласен. Это психология людей плюс пропаганда и спецпропагпнда.
\end{itemize} % }

\end{itemize} % }

\iusr{Ольга Чекрыгина}

Оба города прекрасны. Но для меня они совсем разные. Попытаюсь объяснить
почему. Это ощущение появилось давно, как только попала в тогда ещё Ленинград.
Но сформулировать это отличие смогла только через много лет, попутешествовав по
Европе.

Петербург, так же как и Париж - это литературные произведения, писатели,
художники. Когда туда попадаешь - все знакомо. Но для меня все это выглядит,
как декорации.

А вот Киев, Иерусалим и Рим - это настоящее, живое. Идёшь и ощущаешь, что здесь
до тебя ходили Княгиня Ольга, Олег, Ярослав Мудрый. Здесь Владимир крестил
народ. А моя дочь крещена в той же церкви, что и Михаил Булгаков. Такое же
чувство "настоящести" я испытала в Иерусалиме у могилы Давида и в Риме, гуляя
по Форуму и глядя на этот город с холмов. И Питер, и Киев - неповторимы. Их
общая черта - люди, которые любят свой город и гордятся им, и могут
рассказывать о них без конца

\iusr{Юрій Марков}

Спасибо за интересный рассказ ...Многие позиции автора не просто совпадают, но
повторяют мои прямо точь в точь. После 7-го класса, нас школьников в Питер
возила наша учительница физкультуры - классный руководитель с ее напарницей
(тоже физрук)... Впечатления зашкаливали: белые ночи, Невский проспект,
Исаакиевский собор, фонтаны Петродворца, Летний сад... Позже, уже взрослым, я
много раз бывал там. Мариинский театр и Валерий Гергиев... В составе труппы
украинские фамилии: Шевченко, Огновенко, Михайленко... Театралы Питера и Киева
помнят известного дирижёра Константина Симеонова, который был главным дирижёром
оперных театров Мариинского (тогда Кировского) и Национальной оперы Киева
(тогда Киевский театр оперы и балета им. Т. Шевченка). Кстати, самое первое
место погребения Тараса Григорьевича Шевченка было на Смоленском кладбище в
Питере (там и сейчас имеется такой большой валун-памятник с соответствующей
надписью). Первые анестезиологи Украины - А. Трещинский, А. Маловичко ездили на
стажировку в клинику П. Куприянова. И потом завязались самые теплые отношения
между питерскими и киевскими анестезиологами...

\begin{itemize} % {
\iusr{Ирина Алексеенко}
Киевские театралы помнят ежегодные гастроли Ленинградского театра современного балета.
\end{itemize} % }

\iusr{Костянтин Усенко}

Уявлення про парадні народилось в Києві й Пітері в другій половині 19 сторіччя
одночасно з багатоквартирними "дохідними домами".

Декілька входів були властиві й дворянським родинним гніздам, але найчастіше
вони не так виокремлювались функціонально.

Натомість в багатоквартирних будинках до кухні кожної квартири був чорний хід,
яким заносили не тільки їжу, а й вугілля з дровами, воду, потім гас... А що
виносили, то вже й поготів...

І так само було парадне, котрим користувались власники того житла та їх гості.

В дитинстві, вперше потрапивши до хати в Малютинці, я був вкрай здивований, що
вхід до хати один! Потім вже звик, що це скорше норма, якщо нема прислуги.

\iusr{Ігор Печерський}

Цікаво. Ви трохи зачепили "струньі моей души" і тему міста, яке мені теж не
зовсім чуже. Скажу вам так. Можна милуватись фасадом Пітера, як екскурсант,
можна трохи з місяць покопирсатися у ньому, перебуваючи у відрядженні. Я прожив
у Пітері шість років у період свого студентства. А це я вам скажу неабиякий
досвід. 

Спочатку мешкав у гуртожитку. Поруч р. Карповка, Кіровський
(Кам'яноостровський) проспект та острів. Тоді у Гастрономі, у черзі перший раз
стикнувся з явищем "скинутся на троих". У Києві такого ніколи не побачиш.
Абсолютно незнайомі люди, навіть різного соціального статусу: інтелігент в
окулярах з портфелем у білій сорочці, якийсь алкач і напрацьований втомлений
гегемон. Потім жив у з'йомних кімнатах. Правда, перед цим знайшов свою другу
половину, там народився перший син. Максимум перший рік, перебуваючи у цій
перлині, теж захоплювався, милувався: ах! Ермітаж, ах! Кунсткамера, Зимній сад,
Петергоф, Пушкін, Гатчина і т.і. 

Згодом почав вже знайомитися і з зафасадним боком міста. Інколи сам собі казав,
що повісив би того Петра за ..., що він у цьому абсолютно гнилому місці
прорубав вікно до Європи. 

Почнемо з будинків чудової архітектури. Тоді, кінець 70-х, початок 80-х всі
будинки (накшталт ефекту Бруклінського мосту) постійно знаходились у
риштуваннях для оновлення фасадів. І коли ці риштування переставляли на
наступний будинок, попередній вже починав "сипатися". Сіль і вологість свою
справу знають і добре з нею пораються. Проживаючи у "колодязі" будинку
дореволюційної побудови, стикнувся з таким явищем як комарі-папанінці. 

Зима, за вікном -20С, а у під'їзді (все ж такі були і парадні і під'їзди)
комарі літають. Жах! Виявилось, що у підвалі нашого будинку постійно стоїть
вода. Постійно! Коли народився син, я ознайомився ще з одним боком життя. По
перше, нестача сонця призводить до нестачі вітаміну Д, і в результаті у 90\%
народжених у Пітері дітей, за словами лікарки, вроджений рахіт та інша хронь чи
хрень. Про центральну дитячу лікарню ім. Філатова навіть страшно згадувати. Це
без перебільшення був для нас з дружиною серіал жахів.

Тепер, що стосується народонаселення, тут я вам скажу, дуже багато факторів
вплинуло і на його національний склад і на характер. Тут і місто-порт, тут і
столиця "північна пальміра", тут і блокада, тощо. 

Невеличкий спомин. Сиджу
якось на лавочці на проспекті, чекаю дівчину, поруч сідають дві бабусі і
починають розмовляти за життя, слово за слово аж тут чую кажуть, що після війни
дуже багато "хохлів панаєхало" і тому нібито щось стало погано. 

Пітерці, до речі, у більшості, позбавлені такої якості як снобізм, з яким я
неодноразово стикався, спілкуючись з жителями маскви. А дівчина, на яку я
чекав, знімала кімнату у хазяйки Галини Прокопівни, яка, до речі, дійсно після
війни перебралася з Києва у Пітер. Така чудова бабуся, рафінована інтелігентка,
син в неї у Маріїнінському театрі дирижером служив. Вона, коли я приходив,
завжди просила розповісти про Київ. Знаю також багато своїх однокурсників(ць),
які приїхали з різних республік і у різний спосіб залишилися у Пітері.

Ще пригадалось. Що до Парадних. Поки жив у гуртожитку, був в мене
дружок-однодумець, ми з ним любили гуляти Пітером глибоко вночі, коли вже на
вулицях взагалі нікого. Так от ми полюбляли заходити у Парадні старовинних
будинків (вони тоді ще не запиралися) і вивчаючи т.з. вхідну групу, залишки
ліпнини, висувати версії: тут напевно був камін, тут висіло велике дзеркало.
Мріяли якось зібратись та полазити пітерськими горищами. Ще пригадалось. Була в
нас знайома подружня пара, інститут кинули. Вона (з Полтави) пішла торгувати
квасом, він (з Ізюма) спочатку двірником, потім міліціонером. Дали їм кімнату
на Кіровському проспекті. Кімната в плані п'ятикутна, чимала. Вікна 2м висоти з
фігурними ручками для замикання. Коли почали ремонт і зняли шпалери, під ними
опинилися газети 1913 року. Так Юра казав, що не ремонтував, а читав газети.

P.S. Я б там не залишився жити ні за які "коврижки". Не моє місто. Навіть не
тягне відвідати. Нема Київського затишку, якоїсь камерності, якщо хочете, -
провінційності і відчуття... ну не знаю, мабуть "курортності". І ця відмінність
- океан.

\begin{itemize} % {
\iusr{Алексей Добрынин}
гарна розповідь ))

\iusr{Ігор Печерський}

\textbf{Алексей Добрынин} - життя... Оці відмінності та незвичності дуже врізаються у пам'ять.

\iusr{Dmitriy Grunergarten}
\textbf{Ігор Печерський} 

дуже цікава розповідь. Вважаю, що зараз нема нам сенсу їх "олюднювати", спільні
риси шукати. Здається мені, що саме небезпечне - їхнє гарне. Бо їхнє лайно -
проукраїнська людина й так не сприйме, а гарне...

\iusr{Лариса Салий}
\textbf{Ігор Печерський} 

легко и с интересом прочла. Меня как туриста всегда возбуждали и восхищали
другие города, но с каким же удовольствием возвращаюсь в Киев.... домой

\end{itemize} % }

\iusr{Anna Vaks}
Спасибо за пост! Мечтаю побывать в Санкт Петербурге. И твёрдо намерена исполнить свою мечту. (Киевлянка)

\iusr{Дарья Рекунова}

Мне нравился этот город.

В свое время я туда частенько каталась, как на поезде, так и стопом)))

Тоже очень впечатлил гостинный двор. Он там чистенький и красивенький)

А еще очень впечатлили люди, точнее их ритм жизни. Вот представьте: метро, час
пик, открываются двери вагона. А из вагона вышли люди... и
пошли... медленно, размеренно. И никто, НИКТО! не побежал

\iusr{Нина Опольская}

Благодарю за пост! Согрели душу воспоминаниями тех чудесных времён.... Все
осталось в памяти и никто не испортит, никогда! Два великих города, схожих по
своей уникальности... Два гостеприимных города, приветливых и в тоже время
неповторимых.. Написали ну прямо в душу.... Дружили и будем дружить... Ведь города
ни в чем не виноваты... все люди... своими умишками ...Ах, и не хочу об
этом... Обхожу стороной ...как и не было. ..Думаю, Киеву нравится общение как и
раньше. С Питером и другими городами... и не нарушить этих связей... Все
временное недоразумение... надеюсь, разум восторжествует, хотя у нормальных людей
он и не пропадал... Люблю тебя, мой Город, вечной любовью! Уважаю Питер с его
историей и достопримечательностями!

\iusr{Ирина Сагидаева}

Мы с мужем любим Питер, хотя каждый был в свое время там у своих друзей. Только
по полит. соображениям не переезжаю, я- украинка, вільно спілкуюсь українською
також. А Питер для меня по ощущениям схож, очень с Киевом.

\begin{itemize} % {
\iusr{Ирина Сагидаева}

\ifcmt
  ig https://scontent-frx5-2.xx.fbcdn.net/v/t1.6435-9/251930198_4826246574085857_2470579038520107152_n.jpg?_nc_cat=109&ccb=1-5&_nc_sid=dbeb18&_nc_ohc=VhoDxVgI4-8AX9uXlYV&_nc_ht=scontent-frx5-2.xx&oh=77676774e8f21ed4d94309cecf03be8a&oe=61AA4D29
  @width 0.3

	ig https://scontent-frt3-1.xx.fbcdn.net/v/t1.6435-9/252638036_4826246720752509_2256227657850966586_n.jpg?_nc_cat=104&ccb=1-5&_nc_sid=dbeb18&_nc_ohc=44SQcAxVy4YAX8mY24L&_nc_ht=scontent-frt3-1.xx&oh=b9a4e7ab87b4f69c676029abe7c3b2f1&oe=61A91E2C
  @width 0.3

	ig https://scontent-frx5-1.xx.fbcdn.net/v/t1.6435-9/252553282_4826246880752493_4333723871886986491_n.jpg?_nc_cat=100&ccb=1-5&_nc_sid=dbeb18&_nc_ohc=0IOsJsoEftUAX-dPuWH&_nc_ht=scontent-frx5-1.xx&oh=1571d49232686786da909a8bead152c4&oe=61A82EF2
  @width 0.3
\fi

\end{itemize} % }

\iusr{Леся Фандралюк}

Я никогда не бывала в Питере, но у меня были близкие друзья-коренные
ленинградцы. И мы понимали друг друга с полуслова, мы очень похожи.

\iusr{Юрий Симоненко}

Огромнейшее спасибо, отличный пост. В далёкие семидесятые попал командировку в
Ленинград и сразу влюбился в этот замечательный город. Давно собираю книги о
двух городах – Киев и Ленинград-Петербург. Уже 2004-2005 побывал там 8 раз,
правда каждый раз по одному дню. Но повезло увидеть Царское Село и сказочный
Петергоф. Последние приобретения по теме -

\ifcmt
  ig https://scontent-frt3-1.xx.fbcdn.net/v/t1.6435-9/251838795_970163160205234_7952506186254051707_n.jpg?_nc_cat=106&ccb=1-5&_nc_sid=dbeb18&_nc_ohc=uYC60u9YQsUAX_M1eXr&_nc_ht=scontent-frt3-1.xx&oh=229de525ff7e6ba08de54d2b36d5f705&oe=61A9918A
  @width 0.4
\fi

\iusr{Kit Gross}

Есть много знакомых ленинградцев, хотя слово "поребрик" было для них знакомо,
но то что они так называют бровку у них вызывало удивление. Бордюр скорее.

\iusr{Marina Lavrow}

Дякую за вашу допис. Так склалося, що я до Пітера потрапила один оаз, аж
дорослою, із свою 7-річною донькою. Це рідне місто мого коханого тата. До
блокади там була русько-українсько-польська родина на 100+ осіб. А по війні
лишилася одна тітка Вероніка. Вона майже сотню родичів усіх віків на санчатах
перевозила на цвинтар. Тато мене оберігав. А Пітер завжди для мене залишається
«истоком». Хоча населення в тому 1990 році мене вразило неприємно. В метро мою
мали штовхали, бігли всі, як всюди. В Ермітажі пихата бабуся-біле кока мега
задекларувала, зо семирічна дитина ну просто не може цікавитись
імпресіоністами. Дуже гарне місто Пітер, але мені воно на Києв ну ніяк не
схоже.

\iusr{Олег Андреев}

В Киеве лишь осколки от исторической грандиозности Петербурга.

\begin{itemize} % {
\iusr{Катерина Луцько}
\textbf{Олег Андреев} господин, а в якому це році з'явилась ця історія і її грандіозність? Нагадайте, будь ласка ,і хто її будував?
\end{itemize} % }

\iusr{Михайлина Голуб}

Для меня Питер остается удивительной каменной сказкой. Довелось мне побывать
там дважды: в 1970 году и в 1991. И в первый мой приезд оказалось, что я так
изучила его по книгам, что передвигалась, во всяком случае по центру,
совершенно свободно. Не буду говорить о красоте этого города - до меня было
сказано столько, что и не прочитать, не изучить все поосто невозможно. Красоту
Питера осознаешь с первого взгляда. Я не знаю, с каким городом его можно
сравнить. Как по мне, разве что с Прагой.

Наш Киев - совершенно другой. Если Питер парадный, то Киев домашний. Питером
можно восхищаться бесконечно, а в Киеве хочется жить. А вот питерцы по своим
качествам, манерам разговора напоминают не киевлян, а скорее харьковчан. Я даже
слышала, что Харьков в свое время называли украинским Ленинградом.

Оба города прекрасны. И все же, каждый по своему. И это очень хорошо.

\iusr{Нелли Стельмах}

ІМХО. Недоречний пост , автор і сам це розуміє.

\begin{itemize} % {
\iusr{Natasha Levitskaya}
\textbf{Нелли Стельмах}
Без ИМХО - если пост недоречний для вас, у вас есть право выбора не читать. Пост вне политики!

\iusr{Taras Rybaruk}
\textbf{Natasha Levitskaya} Ну, ну. У нас часто багато що не стосується політики, як не дивно. Така собі - маніпуляція “малоросів”, яким все украінське чуже.

\iusr{Володимир Шуневич}
\textbf{Нелли Стельмах} Ще й як доречний! Більше б таких. А лиш злоба , ворожнеча, бруд. Так мир ніколи не настане.
\end{itemize} % }

\iusr{Людмила Комарова}

Большая часть детства прошла там. Общее есть. Оно не в деталях, в обоих городах
есть удивительная гармония, у каждого своя. И в общении какая-то особенная
деликатность.

\iusr{Георгий Майоренко}
"Люблю тебя, Петра творенье...." Лучше Пушкина не скажешь! Отличный пост.

\iusr{Сергей Кравцов}
Упомянутую фразу "Идите сюдой пару кварталов" вы не услышите в Питере. От питерцев во всяком случае.

\begin{itemize} % {
\iusr{Елена Соловьева}
\textbf{Сергей Кравцов} да, это точно. Я родилась и выросла в Ленинградской области. Питер считаю родным городом, а замуж вышла за киевлянина. Вначале слух резало его тудой-сюдой , быстро отучила, а сейчас даже в Киеве не слышу, чтобы так говорили. Уже 37 лет прожила в Киеве

\iusr{Сергей Кравцов}
\textbf{Елена Соловьева} сейчас про речь киевскую приходится молчать. А когда-то можно было узнать киевлянина по ее особому шарму. Сейчас это стало реликтом.

\iusr{Наталия Попко}
\textbf{Сергей Кравцов} И в Киеве от коренных киевлян тоже... Глупость какая....
\end{itemize} % }

\iusr{Maxim Kaminsky}

Я киевлянин, жена из Питера. Познакомились в штатах двадцать пять лет назад.
Она приезжала со мной в Киев один раз, на четыре дня. Когда уезжали обратно в
штаты - она плакала в аэропорту. Вот как то так.

\begin{itemize} % {
\iusr{Галя Александрова}
\textbf{Maxim Kaminsky} не по теме, простите, в Тетереве на базе Киевэнерго в детстве не отдыхали? Хорошая компания была и самые теплые воспоминания, Максим Каминский был, а вдруг))
\end{itemize} % }

\iusr{Lora Berdichevsky}

Да... Неожиданно.. Тот, кто хотя бы раз бывал в Питере, а ,тем более, жил ,никогда
не забудет этот город .Мой студенческий период в жизни связан с Питером
(Ленинградом) и этот "стольный град" навсегда поселился в моей памяти и в моем
сердце. Мойка,"лягушка",третий корпус Бонча "Висла"- пивбар, Дом
книги, Петропавловская, крепость, Эрмитаж
Исакиевский, Казанский.... Росси, Растрелли, Монферран.. Я живу в необычном месте Я
живу в необычном городе Необычный здесь воздух ,невский По которому сырость
бродит ....(1976)


\iusr{Зинченко Виктор}
Прекрасный слог...

\iusr{Александр Антоненко}
Голод поймут только те кто его пережил

\iusr{Людмила Романова}

Это же надо! Недавно меня посещали подобные мысли. Я помню, как удивилась моя
племянница в Питере, когда услышала от меня слово парадное @igg{fbicon.grin} 

\begin{itemize} % {
\iusr{Марина Соколова}
\textbf{Людмила Романова}

Вот именно, удивилась. Пять лет жила в Ленинграде, учась в Институте- никогда не
говорили там "парадное"! "Парадная" дверь, парадный подъезд (вспомните
Некрасова).

\iusr{Людмила Романова}
\textbf{Марина Соколова} я живу 60 лет в Киеве и с детства именно так помню это слово @igg{fbicon.grin} 
\end{itemize} % }

\iusr{Ирина Самардак}

Да разные они ,и Киев ,и Питер, совершенно разные , как и каждый город ,
самобытен и интересен по своему ,со своими прелестями и недостатками, есть
конечно общие черты ,особенно в современной ,а так разные.. А насчёт ездить ,не
надо было соваться в суверенное государство из-за своего поребрика с колоннами
техники, то так бы и ездили

\begin{itemize} % {
\iusr{Світлана Святюк}
\textbf{Ирина Самардак} ни о чем...

\iusr{Ирина Самардак}
\textbf{Світлана Святюк} да действительно пост не о чем ,города все разные ,у каждого неповторимое лицо и история ,есть отдельные элементы архитектуры общие для всех городов ,но сравнить Киев с Питером ,все равно ,что Киевский торт с кулебякой ,у каждого блюда свой вкус
\end{itemize} % }

\iusr{Vadim Vadim}
Автору респект, скромно напомню о линиях.

\iusr{лариса погосова}

Последний раз была в Питере в 2012 году. В глаза бросалось отсутствие зелени и
...обменнок. Как -то у нас с этим попроще. Очень красивый Питер! Жила далеко от
центра, но чистота в подъездах приятно удивляла. И заменены везде окна в старых
домах. В метро очень мало назойливой рекламы. Зато висел список слов, где нужно
правильно ставить ударение. Общепит отвратительный У нас бы за такие блюда
повара убили сразу. Конечно, есть дорогие рестораны, но я говорю о сетях типа «Два
гуся» или Домашняя кухня.

На Невском прекрасное старое Кафе Север. Но старожилы говорят, что ассортимент
давно не тот. Девушки в метро скромнее одеты. Да и наши красивее!

Очень дорогие учебники и книги @igg{fbicon.cry} .Чисто в городе. А раньше Киев был чище...

\begin{itemize} % {
\iusr{Aleksandr Obukhov}
\textbf{лариса погосова}

Хоть и был последний раз в Питере в 1995, соглашусь со всем сказанным на все 100\%.

В Киеве не хватает интеллигентности в прямом смысле слова, которая бы улучшала
приезжих в него учиться, работать. Но происходит наоборот - оселячивание
столицы.

(Ничего не имею против жителей села, в коем проходило моё летнее детство и
приучение к труду)

\begin{itemize} % {
\iusr{лариса погосова}
\textbf{Aleksandr Obukhov}

Я обратила внимание на питерских продавцов в обычных
магазинах.:вежливость, интеллигентность, отсутствие полное хамства и быдлячества!

В магазине при ИФЗ выбирала сервиз. Пока выбирала ,пока бегала искать обменный
пункт, образовалась очередь. Мне ещё паковали, что б в дороге не
разбила. Образовалась очередь. Никто слова не сказал. У нас бы меня прибили за это
время ,и черепки бы вслед летели. Я и за собой замечаю нетерпимость. В Питере
особая культура отношений. Когда в Эрмитаже третий раз вышла не в тот зал, просто
расплакалась от собственной тупости. Как меня успокаивала смотрительница зала!

-Голубушка, да не расстраивайтесь вы так. Вот так и так пройдите. А лучше
приезжайте осенью-меньше туристов.»

Уже тогда билет на поезд стоил чёртову тучу денег...

\iusr{Aleksandr Obukhov}
\textbf{лариса погосова}

Вот, вот. Сдерживался я от употребления слова. Но в Киеве сейчас именно
былячество и происходит.

Это горе, которое можно исправить лишь десятелетиями, увы.

\iusr{лариса погосова}
\textbf{Aleksandr Obukhov}

Я живу в центре, всю жизнь жила на Шота Руставели и там же в школе работала. Мы
возили детей в Питер, Москву. Водили в музеи, театры...

Этого всего уже нет. По разным причинам. Процветает то, о чем я написала раньше. Во
всех направлениях.

Не хочу комментировать, сейчас начнётся...

Сколько тут коренных киевлян?

Тех, кто любит и ценит каждую тропинку и скверик?!

\end{itemize} % }

\end{itemize} % }

\iusr{Татьяна Панкова-Балог}

Прекрасный пост. Спасибо огромное. Люблю заочно Питер и мечтаю там побывать
Спасибо ещё раз!

\iusr{Виктор Гладский}
Да, были парадные и черный ход. Киев, Дмитриевская 65. Увы, уже нет.

\iusr{лариса погосова}
\textbf{Виктор Гладский}
Шота Руставели 22, был парадный и чёрный вход...

\iusr{Semyon Belenkiy}
Все написано правильно, но по моему, могу ошибаться, в Питере говорят - параднАя, а в Киеве-параднОе.

\begin{itemize} % {
\iusr{Semyon Belenkiy}
Из Интернета: «Парадная».

«Кстати, о подъездах, точнее о парадных. Ведь для жителей Петербурга подъезд
означает процесс, москвичи же уверены, что так называется место, к которому
имеется возможность подъехать.

А теперь о том, почему именно «парадная», а не «парадное»? Хотя второе слово
взято из словаря (наверное, написанного обитателями столицы), но в старых домах
можно заметить таблички с номерами квартир с надписью «Парадная №».

\end{itemize} % }

\iusr{Сергей Кравцов}

Множество великих питерцев жили некоторое время в Киеве и наоборот. К примеру
Деникин, будучи еще полковником жил неподалеку от Львовской площади, на Большой
Житомирской.

\iusr{Тамара Кинчая}

Очень люблю Питер и считаю, что -ленинградцы» или «питерцы»(как хотите) это
особая нация!

\iusr{Нина Бондаренко}

@igg{fbicon.hands.applause.yellow}  Багато разів була в місті на Неві. Ще знаю
: і там і тут завод "Більшовик" .

\begin{itemize} % {
\iusr{Сергей Кравцов}
\textbf{Нина Бондаренко} Там это бывший Обуховский завод. А в Киеве с Обуховым иные ассоциации совсем

\iusr{Нина Бондаренко}
\textbf{Сергей Кравцов} В Києві працювала з батьком на "Більшовику", а в Ленінград на
"Більшовик " їздила у відрядження. Так сталось у житті. Також була на практиці переддипломній. І ще багато разів. Місто дуже привітне миле і красиве.

\iusr{Сергей Кравцов}
\textbf{Нина Бондаренко} Там давно метро уже рядом. И к набережной можно подойим.
\end{itemize} % }

\iusr{Наталя Марчук}

Те що зараз спостерігаю , то дійсно жителі міст і Києва і Пітера інакомислячі
 @igg{fbicon.face.relieved}{repeat=3}  Найменший відсоток підтримки влади РФ саме у Пітері . Ну і оскільки мені
довелось пару раз там бувати , то ще таке слово як судок ( лоток) застосовують
і там і тут )))) А найколоритніше - це греча та кура  @igg{fbicon.face.relieved}{repeat=3} .

\iusr{Наталия Воробьева}

Мне понравился пост, большое спасибо. Мне нравится все ,где упоминается о
Ленинграде и его людях, соответственно и о Санкт- Петербурге.

\iusr{Тамара Ар}

Наконец то! Благодарю за добрый, человечный искренний пост! Питер был, есть и
будет культурной столицей на все времена, Киев - точно, также! Поэтому,
отбросив политику, надо принять, что оба этих города пережили революции,
восстания, войны и смены власти, но сохранили культуру, историю и духовность!

\iusr{Мария Иванова}

В Одессе тоже парадные) Да, и у Южной Пальмиры больше сходства с Питером.

Прекрасно написано, спасибо.

\iusr{Татьяна Лобачевская}

\ifcmt
  ig https://scontent-frx5-2.xx.fbcdn.net/v/t39.1997-6/s168x128/106015589_1000496863739351_8977234205464927799_n.png?_nc_cat=1&ccb=1-5&_nc_sid=ac3552&_nc_ohc=-KzKC_fOEC0AX-NBCXa&_nc_oc=AQmcCTwUA5SWblB7W8LoL84FGnoMAawik_CcormqRGyewgWYaElIhALB7qKdEQbjrPY&_nc_ht=scontent-frx5-2.xx&oh=d1f73528711f2594aac929b056338ab1&oe=61898A0C
  @width 0.2
\fi

\iusr{Oksana Dekhtiar}
Питер, Ленинград.. любовь моя..

\iusr{Світлана Проценко}

Ценить, Любить, Гордиться Красотой, Искать Источник Единения в Своём Городе
-Киева и с Санкт-Петербургом, донести Мнение Другим—это Поступок!!!! Спасибо  @igg{fbicon.hands.pray} 

\begin{itemize} % {
\iusr{Тамара Ар}
Вы правы!
\end{itemize} % }

\iusr{Ольга Мікова}

Эти два города в своей основе абсолютно разные - Киев город тёплый, по
настоящему древний и настоящий, Питер -холодный, построенный рассчетливо и
напоказ, фальшивая красота, совершенно несравнимые города

\begin{itemize} % {
\iusr{Виктория Луцкая}
\textbf{Ольга Мікова} 

мне Ленинград напомнил гранитную могилу ,ни одного дерева, сплошной камень, ни
малейшего сходства с Киевом не заметила, но мы ходили только по центру, может на
окраинах похоже?

\iusr{Ольга Мікова}
\textbf{Виктория Луцкая} согласна с вам)

\iusr{Тамара Ар}
\textbf{Ольга Мікова} Фальшивое - это навязанное мнение! А Питер - это культура, красота и история!

\iusr{Тамара Ар}
\textbf{Ольга Мікова} Есть культурная душа города, в этом Питер и Киев схожи! Как бы ни старались говорить негатив про Вечный Город!

\begin{itemize} % {
\iusr{Ольга Мікова}
\textbf{Toma Ar} 

Киев, кстати, с большой буквы надо писать )это ваше мнение-для вас так, а для
меня Питер, как и вся тн российская культура -псевдокультура, все вторично, и
никто никогда С.-Петербург не называет Вечным городом, Вечный город -это Рим

\iusr{Тамара Ар}
\textbf{Ольга Мікова} не стоит вам привязываться к элементарной описке! Понятно, если вам больше не к чему!

\iusr{Тамара Ар}
\textbf{Ольга Мікова} политизированность достала!

\iusr{Тамара Ар}
\textbf{Ольга Мікова} Если вы не знаете поэтов и писателей русского классицизма, который принадлежит к мировому, то это ваши проблемы!

\iusr{Ольга Мікова}
\textbf{Toma Ar} печальные последствия, когда люди ограничиваются только знанием "русского"классицизма)

\iusr{Тамара Ар}
\textbf{Ольга Мікова} Не только, но отчитываться перед вами не собираюсь

\iusr{Ольга Мікова}
\textbf{Toma Ar} хочу уточнить -если бы "не только"-тогда бы вы знали точное место в мировой культуре того, чем вы так восхищаетесь)

\iusr{Тамара Ар}
Нивелирование истины, обычное состояние для таких как вы
\end{itemize} % }

\iusr{Aleksandr Obukhov}
\textbf{Ольга Мікова}
Соглашусь со всем, за исключением "фальшивая красота".
Это просто иная красота.
А Киеве сейчас стараются задушить присущую ему теплоту и особенно это древний дух.

\begin{itemize} % {
\iusr{Ольга Мікова}
\textbf{Aleksandr Obukhov} 

вторичное в том, что Пётр І увлекся голландской и немецкой архитектурой и начал
возводить у себя подобное, понятно, что были коррективы ,учитывая местность,
характер климата и прочее( а иногда и не учитывая))))пригласил иностранных
архитекторов)))) и возвел свою мечту, построил город-призрак на болотах)открыл
"окно в Европу))))все там вторичное, как и вся почти российская культура ,за
исключением этнографии именно народностей, населявших эту территорию, но какое
отношение они имеют хотя бы к слову "русский",это очень серьёзный вопрос и как
ни парадоксально звучит, тут политика далеко на не последнем месте)хотя конечно,
можно просто бездумно любоваться красивой архитектурой....

\iusr{Aleksandr Obukhov}
\textbf{Ольга Мікова}
Это уже другая история и не к первичному посту.
\end{itemize} % }

\iusr{Петр Кузьменко}
\textbf{Ольга Мікова}, 

сколько людей, столько и мнений. Культура общения состоит в том, чтобы не
навязывать своё и уважительно относиться к мнению оппонента, даже если не
принимаешь его.

\end{itemize} % }

\iusr{Андрей Кавардаков}

Для начала хотелось бы что бы киевляне поехали отдыхать в свой Крым, у их
перестали убивать на востоке своей страны жители соседней ( и
питерцы/ленинградцы в частности), а потом можно и про северную Пальмиру
помечтать

\iusr{Наташа Грудцына}

На весенних каникулах в далёком 1969 году у нас была такая же поездка-обмен в
Ленинград на поезде. Повезла нас учительница русского языка и литературы,
классный руководитель СШ 16 Лидия Георгиевна. Обосновались мы также в
спортивном зале одной из школ новостроек. Только спали мы на матах, а
учительница на границе между девочками и мальчиками. Каждый день у нас были
интереснейшие прогулки по улицам и музеям Ленинграда, а также поездки в
Петергоф и Царское село. Этот было незабываемо! А различие между нашими
городами было и прежде всего в погоде - в Киеве всё пахло весной, там -
промозглая сырость, снег. Еда в столовых хуже, чем у нас. Зато музеи, дворцы,
какой-то особый шарм, реки Нева и Мойка нас просто очаровали. Ленинград -
настоящий город-музей. Но свой родной Киев, уютный, зелёный, тёплый я бы ни за
что не променяла на Питер.

\end{itemize} % }
