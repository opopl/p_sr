% vim: keymap=russian-jcukenwin
%%beginhead 
 
%%file slova.prazdnik
%%parent slova
 
%%url 
 
%%author 
%%author_id 
%%author_url 
 
%%tags 
%%title 
 
%%endhead 
\chapter{Праздник}
\label{sec:slova.prazdnik}

%%%cit
%%%cit_head
%%%cit_pic
\ifcmt
  pic https://avatars.mds.yandex.net/get-zen_doc/4389079/pub_60c455751d6b9f44132f00cf_60c49487aa55f742547b9456/scale_1200
	caption Симферополь, проспект Кирова, 11.06.2021 г.
\fi
%%%cit_text
Друзья, мы живём в самой большой и самой прекрасной стране мира!  Недаром наш
президент сказал очень правильные слова: \enquote{Это не Россия находится между
западом и востоком. Это запад и восток находятся слева и справа от России} С
\emph{праздником}! Россия - форева, Россия - виват! Всё, мы пошли
\emph{праздновать}
%%%cit_comment
%%%cit_title
\citTitle{Самая-самая Россия: северная, холодная, загадочная, непобедимая}, 
Крымские Посидушки, zen.yandex.ru, 12.06.2021
%%%endcit

%%%cit
%%%cit_head
%%%cit_pic
\ifcmt
  pic https://storage.lug-info.com/cache/4/5/2dc2e503-e114-4274-9a25-7b307b9351ed.jpg/w700h474
\fi
%%%cit_text
Министерство труда и социальной политики ЛНР проинформировало о выходных днях в
период \emph{празднования} Дня Святой Троицы. Об этом сообщили в ведомстве.  \enquote{В связи
с \emph{празднованием} 20 июня Святой Троицы жителей Республики в этом году ждут три
выходных дня – с 19 по 21 июня, то есть праздничный день 20 июня переносится на
21 июня. Соответственно, следующая рабочая неделя продлится 4 дня – с 22 по 25
июня}, - говорится в сообщении.  Ранее Министерство разъяснило, какие дни будут
выходными в 2021 году
%%%cit_comment
%%%cit_title
  
%%%endcit


%%%cit
%%%cit_head
%%%cit_pic
%%%cit_text
У 1982, на сторічний ювілей Джойса, в Дубліні відзначили Блумсдей офіційно,
після чого масштаб \emph{святкувань} щороку зростав. Проведення \emph{свята} було скасовано
єдиний раз, у 2006, через жалобу за прем'єр-міністром Ірландії Чарльзом Хогі.
В рамках \emph{свята} по всьому світу проходять читання \enquote{Улісса}, а в Дубліні учасники
свята повторюють маршрути персонажів, підтверджені табличками на стінах
будинків. Ентузіасти обряджені в костюми тієї епохи, замовляють страви, як у
романі - смажені баранячі нирки, склянку бургундського, бутерброд з італійським
сиром.  Є схоже \emph{свято} і в Італії – Дантеді, що означає \enquote{День Данте}.
Відзначається щорічно 25 березня, в дату, коли в 1300, згідно з усталеним
припущенням, Данте \enquote{трапив у похмурий ліс густий} – це рядок з початку
\enquote{Божественної комедії}.  Веду до того, що схоже \emph{свято} цілком могло би існувати
і в нас. Наприклад, \emph{День садка вишневого}, десь між 19 і 30 травня. З карнавалом
і мільйоном страв з вишні, з гулянням по Шевченковим місцям і концертами,
виставами та кіносеансами і ще завгодно чим – аби фантазії вистачило.
Література з прокляття шкільної програми має стати \emph{святом}
%%%cit_comment
%%%cit_title
\citTitle{Блукання Блума, ліс Данте, садок Шевченка}, 
Дмитро Десятерик, day.kyiv.ua, 16.06.2021
%%%endcit

%%%cit
%%%cit_head
%%%cit_pic
%%%cit_text
Гарант не постеснялся блокировать работу КСУ и публично поднимает вопрос об
отторжении Донбасса. Конституция все это запрещает, но кому какое дело?
Вспоминать о ней принято лишь раз в году – сегодня. А так, конечно, с
\emph{праздником}!
%%%cit_comment
%%%cit_title
\citTitle{Ранее, чтобы выглядеть прилично, власть меняла Конституцию / Лента соцсетей / Страна}, 
Максим Могильницкий, strana.ua, 28.06.2021
%%%endcit

%%%cit
%%%cit_head
%%%cit_pic
\ifcmt
  pic https://img.strana.ua/img/article/3458/krestnyj-khod-upts-93_main.jpeg
  width 0.4
	caption Крестный ход спускается на Крещатик с улицы Трехсвятительской. Фото УПЦ 
\fi
%%%cit_text
Сегодня УПЦ провела крестный ход, посвященный дню Крещения Руси.
\emph{Праздник} этот в последние годы обрел и политическое значение.  В этот
день разбиваются пропагандистские клише о том, что Украинская православная
церковь - "не украинская". Количество верующих УПЦ многократно превышает число
мирян у ПЦУ и любых политических манифестаций, которые проводятся в Киеве в
последние годы.  Этот год не стал исключением. "Страна" представляет главные
выводы из сегодняшнего крестного хода.  1. В прошлом году крестный ход по
причинам карантина не проходил. А в этом он состоялся - и перекрыл по
количеству показатель 2019 года.  Причем тут сходятся данные и самой церкви, и
полиции. УПЦ в прошлый раз сообщала, что на ход вышли 300 тысяч человек.
Сегодня в конфессии насчитали 350-400 тысяч участников
%%%cit_comment
%%%cit_title
\citTitle{Почему на улицы выходит все больше верующих УПЦ. 5 выводов после Крестного хода}, 
Максим Минин, strana.ua, 27.07.2021
%%%endcit

%%%cit
%%%cit_head
%%%cit_pic
%%%cit_text
По логике вещей, Зеленскому, как человеку, скажем так, далекому от православия,
было бы лучше обойти комментариями прошедший в Киеве огромный крестный ход в
честь 1033-летия Крещения Руси. Или же как президенту государства, верующие
граждане которого в большинстве своём православные прихожане УПЦ МП, тепло и
просто поздравить их с \emph{праздником}.  Но не тут-то было.  Зеленский
высказал претензии, что верующие на крестном ходе были без масок. Ну да, ведь
бандерлоги на день рождения своего кумира ходят исключительно в масках! Как и
ожидалось, Зеленский тупо и нагло соврал, назвав количество участвующих в
\emph{праздновании} «55 тысяч человек», хотя, по данным организаторов,
подтверждённых не только визуальным контролем, но и сведениями МВД, крестным
ходом прошли рекордные для Киева 350 тысяч верующих
%%%cit_comment
%%%cit_title
\citTitle{Аттракцион невежества, или Как натянуть историю на глобус Украины — Одна Родина}, 
Дарья Меньшова, odnarodyna.org, 29.07.2021
%%%endcit

