% vim: keymap=russian-jcukenwin
%%beginhead 
 
%%file books.bredberi_rej.rasskazy.ulybka
%%parent books.bredberi_rej.rasskazy
 
%%url 
 
%%author_id 
%%date 
 
%%tags 
%%title 
 
%%endhead 

\section{Улыбка}
\label{sec:books.bredberi_rej.rasskazy.ulybka}

На главной площади очередь установилась ещё в пять часов, когда за выбеленными
инеем полями пели далекие петухи и нигде не было огней. Тогда вокруг, среди
разбитых зданий, клочьями висел туман, но теперь, в семь утра, рассвело, и он
начал таять. Вдоль дороги по двое, по трое подстраивались к очереди ещё люди,
которых приманил в город праздник и базарный день.

Мальчишка стоял сразу за двумя мужчинами, которые громко разговаривали между
собой, и в чистом холодном воздухе звук голосов казался вдвое громче.

Мальчишка притопывал на месте и дул на свои красные, в цыпках, руки, поглядывая
то на грязную, из грубой мешковины, одежду соседей, то на длинный ряд мужчин и
женщин впереди.

— Слышь, парень, ты-то что здесь делаешь в такую рань?-сказал человек за его
спиной.

— Это моё место, я тут очередь занял,-ответил мальчик.

— Бежал бы ты, мальчик, отсюда, уступил бы своё место тому, кто знает в этом
толк!

— Оставь в покое парня,-вмешался, резко обернувшись, один из мужчин, стоящих
впереди.

— Я же пошутил.-Задний положил руку на голову мальчишки. Мальчик угрюмо
стряхнул её. — Просто подумал, чудно это — ребёнок, такая рань а он не спит.

— Этот парень знает толк в искусстве, ясно?-сказал заступник, его фамилия была
Григсби.- Тебя как звать-то, малец?

— Том.

— Наш Том, уж он плюнет что надо, в самую точку — верно. Том?

— Точно!

Смех покатился по шеренге людей.

Впереди кто-то продавал горячий кофе в треснувших чашках. Поглядев туда, Том
увидел маленький жаркий костер и бурлящее варево в ржавой кастрюле. Это был не
настоящий кофе. Его заварили из каких-то ягод, собранных на лугах за городом, и
продавали по пенни чашка, согреть желудок, но мало кто покупал, мало кому это
было по карману.

Том устремил взгляд туда, где очередь пропадала за разваленной взрывом каменной
стеной.

— Говорят, она улыбается,- сказал мальчик.

— Ага, улыбается,-ответил Григсби.

— Говорят, она сделана из краски и холста.

— Точно. Потому-то и сдается мне, что она не подлинная. Та, настоящая,я
слышал,была на доске нарисована, в незапамятные времена.

— Говорят, ей четыреста лет.

— Если не больше. Коли уж на то пошло, никому не известно, какой сейчас год.

- Две тысячи шестьдесят первый!

- Верно, так говорят, парень, говорят. Брешут. А может, трехтысячный! Или
пятитысячный! Почем мы можем знать? Сколько времени одна сплошная катавасия
была... И достались нам только рожки да ножки.

Они шаркали ногами, медленно продвигаясь вперед по холодным камням мостовой.

-Скоро мы ее увидим?-уныло протянул Том.

-Еще несколько минут, не больше. Они огородили ее, повесили на четырех латунных
столбиках бархатную веревку, все честь по чести, чтобы люди не подходили
слишком близко. И учти, Том, никаких камней, они запретили бросать в нее камни.

-Ладно, сэр,

Солнце поднималось все выше по небосводу, неся тепло, и мужчины сбросили с себя
измазанные дерюги и грязные шляпы.

— А зачем мы все тут собрались?-спросил, подумав, Том.-Почему мы должны
плевать?

Тригсби и не взглянул на него, он смотрел на солнце, соображая, который час.

— Э, Том, причин уйма.-Он рассеянно протянул руку к карману, которого уже давно
не было, за несуществующей сигаретой. Том видел это движение миллион раз.-Тут
все дело в ненависти, ненависти ко всему, что связано с Прошлым. Ответь-ка ты
мне, как мы дошли до такого состояния? Города — груды развалин, дороги от
бомбежек — словно пила, вверх-вниз, поля по ночам светятся,радиоактивные... Вот и
скажи, Том, что это, если не последняя подлость?

— Да, сэр, конечно.

— То-то и оно... Человек ненавидит то, что его сгубило, что ему жизнь поломало.
Так уж он устроен. Неразумно, может быть, но такова человеческая природа.

— А если хоть кто-нибудь или что-нибудь, чего бы мы не ненавидели?-сказал Том.

— Во-во! А всё эта орава идиотов, которая заправляла миром в Прошлом! Вот и
стоим здесь с самого утра, кишки подвело, стучим от холода зубами — ядовитые
троглодиты, ни покурить, ни выпить, никакой тебе утехи, кроме этих наших
праздников, Том. Наших праздников...

Том мысленно перебрал праздники, в которых участвовал за последние годы.
Вспомнил, как рвали и жгли книги на площади, и все смеялись, точно пьяные. А
праздник науки месяц тому назад, когда притащили в город последний автомобиль,
потом бросили жребий, и счастливчики могли по одному разу долбануть машину
кувалдой!..

— Помню ли я, Том? Помню ли? Да ведь я же разбил переднее стекло — стекло,
слышишь? Господи, звук-то какой был, прелесть! Тррахх!

Том и впрямь словно услышал, как стекло рассыпается сверкающими осколками.

— А Биллу Гендерсону досталось мотор раздолбать. Эх, и лихо же он это сработал,
прямо мастерски. Бамм! Но лучше всего,-продолжал вспоминать Григсби,-было в тот
раз, когда громили завод, который еще пытался выпускать самолеты. И отвели же
мы душеньку! А потом нашли типографию и склад боеприпасов — и взорвали их
вместе! Представляешь себе. Том?

Том подумал.

— Ага.

Полдень. Запахи разрушенного города отравляли жаркий воздух, что-то копошилось
среди обломков зданий.

— Сэр, это больше никогда не вернётся?

— Что — цивилизация? А кому она нужна? Во всяком случае не мне!

— А я так готов ее терпеть,-сказал один из очереди.-Не все, конечно, но были и
в ней свои хорошие стороны...

— Чего зря болтать-то! — крикнул Григсби.-Всё равно впустую.

— Э,-упорствовал один из очереди,-не торопитесь.- Вот увидите: ещё появится
башковитый человек, который её подлатает. Попомните мои слова. Человек с душой.

— Не будет того, — сказал Григсби.

— А я говорю, появится. Человек, у которого душа лежит к красивому. Он вернет
нам — нет, не старую, а, так сказать, ограниченную цивилизацию, такую, чтобы мы
могли жить мирно.

— Не успеешь и глазом моргнуть, как опять война!

— Почему же? Может, на этот раз все будет иначе.

Наконец и они вступили на главную площадь. Одновременно в город въехал
верховой, держа в руке листок бумаги.Огороженное пространство было в самом
центре площади. Том, Григсби и все остальные, копя слюну, подвигались вперед —
шли, изготовившись, предвкушая, с расширившимися зрачками. Сердце Тома билось
часто-часто, и земля жгла его босые пятки.

— Ну, Том, сейчас наша очередь, не зевай!

По углам огороженной площадки стояло четверо полицейских — четверо мужчин с
жёлтым шнурком на запястьях, знаком их власти над остальными. Они должны были
следить за тем, чтобы не бросали камней.

— Это для того,-уже напоследок объяснил Григс-би,-чтобы каждому досталось
плюнуть по разку, понял, Том? Ну, давай!

Том замер перед картиной, глядя на нее.

-Ну, плюй же!

У мальчишки пересохло во рту.

— Том, давай! Живее!

— Но,-медленно произнес Том,-она же красивая!

— Ладно, я плюну за тебя!

Плевок Григсби блеснул в лучах солнца. Женщина на картине улыбалась
таинственно-печально, и Том, отвечая на её взгляд, чувствовал, как колотится
его сердце, а в ушах будто звучала музыка.

— Она красивая,- повторил он.

— Иди уж, пока полиция...

— Внимание!

Очередь притихла. Только что они бранили Тома- стал как пень!-а теперь все
повернулись к верховому.

— Как её звать, сэр?-тихо спросил Том.

— Картину-то? Кажется, «Мона Лиза»... Точно: «Мона Лиза».

— Слушайте объявление- сказал верховой.- Власти постановили, что сегодня в
полдень портрет на площади будет передан в руки здешних жителей, дабы они могли
принять участие в уничтожении...

Том и ахнуть не успел, как толпа, крича, толкаясь, мечась, понесла его к
картине. Резкий звук рвущегося холста... Полицейские бросились наутек. Толпа
выла, и руки клевали портрет, словно голодные птицы. Том почувствовал, как его
буквально швырнули сквозь разбитую раму. Слепо подражая остальным, он вытянул
руку, схватил клочок лоснящегося холста, дернул и упал, а толчки и пинки
вышибли его из толпы на волю. Весь в ссадинах, одежда разорвана, он смотрел,
как старухи жевали куски холста, как мужчины разламывали раму, поддавали ногой
жёсткие лоскуты, рвали их в мелкие-мелкие клочья.

Один Том стоял притихший в стороне от этой свистопляски. Он глянул на свою
руку. Она судорожно притиснула к груди кусок холста, пряча его.

— Эй, Том, ты что же!-крикнул Григсби. Не говоря ни слова, всхлипывая, Том
побежал прочь. За город, на испещренную воронками дорогу, через поле, через
мелкую речушку, он бежал и бежал, не оглядываясь, и сжатая в кулак рука была
спрятана под куртку.

На закате он достиг маленькой деревушки и пробежал через неё. В девять часов он
был у разбитого здания фермы. За ней, в том, что осталось от силосной башни,
под навесом, его встретили звуки, которые сказали ему, что семья спит — спит
мать, отец, брат. Тихонько, молча, он скользнул в узкую дверь и лёг, часто
дыша.

— Том? — раздался во мраке голос матери.

— Да.

— Где ты болтался? — рявкнул отец.-Погоди, вот я тебе утром всыплю...

Кто-то пнул его ногой. Его собственный брат, которому пришлось сегодня в
одиночку трудиться на их огороде.

— Ложись!-негромко прикрикнула на него мать.

Ещё пинок.

Том дышал уже ровнее. Кругом царила тишина. Рука его была плотно-плотно прижата
к груди. Полчаса лежал он так, зажмурив глаза.

Потом ощутил что-то: холодный белый свет. Высоко в небе плыла луна, и маленький
квадратик света полз по телу Тома. Только теперь его рука ослабила хватку.
Тихо, осторожно, прислушиваясь к движениям спящих, Том поднял её. Он помедлил,
глубоко-глубоко вздохнул, потом, весь ожидание, разжал пальцы и разгладил
клочок закрашенного холста.

Мир спал, освещённый луной.

А на его ладони лежала Улыбка.

Он смотрел на неё в белом свете, который падал с полуночного неба. И тихо
повторял про себя, снова и снова: «Улыбка, чудесная улыбка...»

Час спустя он все ещё видел её, даже после того как осторожно сложил её и
спрятал. Он закрыл глаза, и снова во мраке перед ним — Улыбка. Ласковая,
добрая, она была Там и тогда, когда он уснул, а мир был объят безмолвием, и
луна плыла в холодном небе сперва вверх, потом вниз, навстречу утру.
