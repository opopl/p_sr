% vim: keymap=russian-jcukenwin
%%beginhead 
 
%%file 20_12_2021.stz.news.ua.radiosvoboda.1.kievljane_ugroza_vojna
%%parent 20_12_2021
 
%%url https://www.radiosvoboda.org/a/photo-de-hovatysia-kyianav-u-razi-povnomashtabnoi-viyny/31617616.html
 
%%author_id nuzhnenko_sergіj
%%date 
 
%%tags 
%%title Російська агресія. Де ховатися киянам у разі повномасштабної війни? 
 
%%endhead 
\subsection{Російська агресія. Де ховатися киянам у разі повномасштабної війни?}
\label{sec:20_12_2021.stz.news.ua.radiosvoboda.1.kievljane_ugroza_vojna}

\Purl{https://www.radiosvoboda.org/a/photo-de-hovatysia-kyianav-u-razi-povnomashtabnoi-viyny/31617616.html}
\ifcmt
 author_begin
   author_id nuzhnenko_sergіj
 author_end
\fi

«В Сполучених штатів залишилося чотири тижні, аби запобігти вторгненню Росії в
Україну», – йдеться у повідомленні телеканалу CNN, який посилається на розмову
із джерелом в адміністрації президента Байдена.

Розмови про продовження скупчення російських військ навколо кордону тривають
уже кілька місяців. Разом з тим зростають ризики повномасштабного вторгнення
сил Росії вглиб України. 

Наприкінці листопада міжнародні аналітики запевняли, що Росія стягує у бік
українського кордону техніку «у кількості, яка насторожує».

\ii{20_12_2021.stz.news.ua.radiosvoboda.1.kievljane_ugroza_vojna.pic.1}

Тим часом керівництво української розвідки презентує ймовірний план
повномасштабних наступальних дій у бік українських міст. За словами керівника
Генерального управління розвідки Міністерства оборони України Кирила Буданова,
ймовірна повномасштабна атака запланована на кінець січня, початок лютого 2022
року. Вона, ймовірно, включатиме у себе авіаудари, артилерійські удари,
бронетанкові атаки й десантно-штурмові дії.

Як інформує видання CNN, посилаючись на власні джерела, навколо кордону з
Україною зібралося щонайменше 50 батальйонно-тактичних груп. Чи вирішив нині
Володимир Путін захоплювати знову території України, поки не відомо.

\subsubsection{Готуватися до найгіршого, і сподіватися на краще}

Вище політичне та військове керівництво країни запевняють, що отримали
інформацію про скупчення військ на кордонах України від міжнародних партнерів і
заявляють, що готові стримати агресію. Утім чи готові українські міста до
можливих агресивних бойових дій? Зрештою, чи готова до цього столиця?

На тлі розмов про потенційну повномасштабну атаку сусідньої країни, українська
столиця готується до можливої необхідності укривати цивільне населення від
бомбардування чи надзвичайної ситуації.

\ii{20_12_2021.stz.news.ua.radiosvoboda.1.kievljane_ugroza_vojna.pic.2}

Дніпровський район Києва. Спальний частина міста. Журналістам демонструють оде
з підготовлених укриттів у будівлі житлово-експлуатаційного комітету 409.

Це приміщення розраховане на 350 осіб для тимчасового укриття. Тут, запевняють
у департаменті муніципальної безпеки КМДА, є все необхідне для тимчасового
перебування людей: резервуари з водою, санвузли, вентиляційна система, окремі
вхід та вихід. Є і додатковий евакуаційний вихід. Крім цього, є автономна
система вентиляції, що запускається вручну.

«Це загальне стандартне сховище, яке зроблене на початку 80 років по єдиному
стандарту», – каже директор департаменту муніципальної безпеки КМДА Роман
Ткачук.

Загалом у Києві налічують близько 500 таких об'єктів. Запевняють, усі вони
обладнані не гірше того, що демонструють журналістам. Однак, саме це приміщення
розраховане для людей, які перебувають на підприємстві. Цивільне населення, що
живе поруч, не зможе потрапити в це укриття. Для них призначені приміщення у
житлових будинках.

\ii{20_12_2021.stz.news.ua.radiosvoboda.1.kievljane_ugroza_vojna.pic.3}

Одне з таких приміщень нам показують за кілька кварталів від попереднього, в
тому ж районі столиці. У старій будівлі із масивними цегляними стінами, більш
відомій як «сталінка», у підвальному приміщенні розташоване сховище. На вході
повідомлення про номер та тип укриття, а металеві двері зачинені на замок.
Працівники ЖЕКу відчиняють двері, аби продемонструвати, що все дійсно діюче.

На вході старий дисковий телефон, протигази і плакати часів Холодної війни.
Таке укриття розраховане на не тривале перебування цивільного населення, туди
зможуть потрапити усі, кого застали бойові дії на вулиці, а також мешканці
навколишніх будинків.

\subsubsection{Замість укриття – книгарня або барбершоп}

Окрім сховищ підвищеного рівня захисту у столиці налічують понад 6 тисяч споруд
подвійного призначення, що придатні для укриття цивільного населення. Частина
укриттів розташовані підвальні приміщення житлових будинків. Утім, місто
стикається із проблемою, що такі приміщення перейшли у приватну власність. І
часто на місці укриття можна зустріти книгарню, тату-салон, барбершоп чи
продуктовий магазин.

«На об’єктах подвійного використання дуже багато в підвальних і напівпідвальних
приміщень розташовані магазини, якісь перукарні, банки, що завгодно. Ми
доводимо через районні адміністрації інформацію, що балансоутримувачі в разі,
якщо буде задіяно сигнал «Увага всім», тобто в місті завиють сирени, потрібно
відчинити двері і надати вільний доступ для захисту мешканців нашого міста»,
– каже Роман Ткачук. 

\ii{20_12_2021.stz.news.ua.radiosvoboda.1.kievljane_ugroza_vojna.pic.4}

Разом з тим, укритися від збройної агресії можна у підземних переходах,
паркінгах та, найголовніше, в столичній підземці. Столичне метро розраховане
укрити населення від будь-яких загроз.

Нині усі приміщення, передбачені для укриття населення переважно зачинені. Все
це для того, аби у мирний час туди не мали доступу сторонні особи й приміщення
не захаращувалося. У департаменті муніципальної безпеки запевняють, у разі,
якщо виникне загроза збройного протистояння у місті чи будь-якої іншої
надзвичайної ситуації, протягом 15 хвилин усі укриття будуть доступні для
населення.

\subsubsection{Час йде на хвилини}

Військовий у запасі, підполковник Збройних сил України, учасник бойових дій на
Сході України Вячеслав Скоряк сумнівається, що українські міста нині можуть
себе почувати в безпеці.

«Маючи під боком такого неадекватного сусіда, не можна вважати захищеним жодне
місто в Україні, мені здається. Саме на цьому боці Дніпра», – каже Скоряк.

\ii{20_12_2021.stz.news.ua.radiosvoboda.1.kievljane_ugroza_vojna.pic.5}

За його словами, населенню необхідно знати, де в його місті розташоване
укриття. Не менш важливо розуміти, що для сховища можна використовувати
будь-яке приміщення під землею. Попри те, що зазвичай ні авіація, ні військові
цілеспрямовано не руйнують житловий сектор, а атакують, в першу чергу, опорні
пункти збройного спротиву, або ж критично важливу інфраструктуру, ризики все ж
існують. Це він побачив перебуваючи в зоні бойових дій.

«Візьмемо відоме село Чорнухине. В 2014 році жителі, у яких були підвали,
тільки в них і жили. За цей рахунок тільки й рятувалися».

Об’єкти для укриття цивільного населення активно готували ще на початку агресії
Росії в Криму та на Донбасі. З тих пір ці приміщення регулярно перевіряють на
придатність. Попри те, що сховища побудовані в середині минулого століття, вони
й досі придатні для тимчасового укриття населення. Де вони розташовані? КМДА
підготували детальну карту для населення, аби ознайомитися із
потенційними місцями укриття неподалік своїх помешкань.
