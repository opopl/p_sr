% vim: keymap=russian-jcukenwin
%%beginhead 
 
%%file 23_01_2022.stz.news.ru.odnarodyna.1.kiev
%%parent 23_01_2022
 
%%url https://odnarodyna.org/article/kiev-ulicu-tankovuyu-pereimenovali-soedinyonnye-shtaty
 
%%author_id rusov_jurij
%%date 
 
%%tags kiev,pereimenovanija,posolstvo_usa.kiev,ulica.kiev.tankovaja,usa
%%title Киев: улицу Танковую переименовали Соединённые Штаты
 
%%endhead 
 
\subsection{Киев: улицу Танковую переименовали Соединённые Штаты}
\label{sec:23_01_2022.stz.news.ru.odnarodyna.1.kiev}
 
\Purl{https://odnarodyna.org/article/kiev-ulicu-tankovuyu-pereimenovali-soedinyonnye-shtaty}
\ifcmt
 author_begin
   author_id rusov_jurij
 author_end
\fi

В середине 1950-х годов микрорайон Нивки стал градостроительной гордостью
украинской столицы. Дело в том, что в Киеве именно с него начала воплощаться
новая жилищная политика государства, инициированная Никитой Хрущёвым, – быстрое
возведение многоквартирных домов для трудящихся, не имевших полноценного жилья.
Во время строительства, происходившего в 1950-60 гг., на табличках новых улиц
появились имена советских, партийных и государственных деятелей, героев фронта
и тыла. То есть это была изначальная топонимика, преимущественно без
переназваний. Тогда же в микрорайоне открылись ясли, детсады, школы,
предприятия бытового обслуживания, кинотеатр.

\ii{23_01_2022.stz.news.ru.odnarodyna.1.kiev.pic.1}

Нивки были известны с середины XIX века как дачная местность с одноимённым
хутором, располагавшимся вблизи тогдашнего западного въезда в Киев. Рядом на
полях росли хлеба, почему одна из версий названия хутора и утверждает, что
поселение получило его в связи с нивами, колосившимися здесь. В городскую черту
его включили в 1923 году. Но только после войны, в 1948 году, стартовала его
активная застройка – поначалу это было индивидуальное строительство. До сих пор
на здешних улицах встречаются старые маленькие хатынки с покосившимися
заборами; видимо, потомки тех, кто их соорудил их 70 лет назад, не решаются
пока продавать своё наследство, тем более, что земля в данном районе – не
центральном, но весьма привлекательном, постоянно дорожает.

\ii{23_01_2022.stz.news.ru.odnarodyna.1.kiev.pic.2}

Кстати, о тогдашней популярности Нивок свидетельствует наличие здесь
трехэтажного здания с небольшой колоннадой и балконом в стиле старорусских
помещичьих усадеб; с таких балконов помещики, как мы читали в литературе, имели
привычку общаться с крестьянами, чтобы все могли слышать барина. Но в
описываемом доме проживал... цыганский барон. С большой долей уверенности рискну
утверждать, что сам факт существования в советскую пору такого довольно
шикарного особняка говорил о дальновидности власти, умевшей находить подход к
национальным меньшинствам.

\ii{23_01_2022.stz.news.ru.odnarodyna.1.kiev.pic.3}

Стоит особняк на улице Мамасалы Тешебаева (1923–1984) – уроженца Киргизии,
красноармейца, который за подвиги при освобождении  Киева был удостоен звания
Героя Советского Союза. В 1985 году его именем назвали бывшую Станкозаводскую
улицу (Советская власть имела на это полное право, поскольку прежнее название
тоже давала она).

\ii{23_01_2022.stz.news.ru.odnarodyna.1.kiev.pic.4}

Украинские декоммунизаторы в пылу кампании по переделыванию топонимики Киева не
посмели посягнуть на имя героя-киргиза. То ли из боязни поссориться с
Киргизией, то ли вообще будучи в неведении, кто он такой. А вот другую улицу,
носившую имя большевика Ивана Бабушкина,  где возвышается, согласно
утверждениям справочников, новый особняк цыганского барона,  переименовали. И
теперь на табличках, что на фасадах зданий, красуется имя Марка Безручка –
петлюровца и начальника штаба «Сечевых стрельцов» под управлением националиста
Евгения Коновальца, организатора террора и диверсий.

Позднее Нивскую застройку, как и другие, ей подобные, по всему СССР стали
презрительно именовать «хрущобами». Память человеческая коротка, и отдельные
квартиры со всеми удобствами, куда народ радостно въезжал после жизни в бараках
и общежитиях, спустя некоторое время стали казаться ущербными – и метраж
маловат, и санузлы совмещены, и...

А ведь это прекрасный, практически дачный массив в черте города, в близкой
досягаемости от центра. На Нивках находится большой лесопарк «Дубки» с речкой
Сырец, где любят гулять горожане из разных районов. Здесь растет одно из
выдающихся деревьев Киева, так называемый «Дуб эколога Стеценко», помнящий, по
утверждениям историков, Петра Первого.

\ii{23_01_2022.stz.news.ru.odnarodyna.1.kiev.pic.5}

А во втором парке, возле метро, радуют глаз озёра. Здесь можно погулять,
покататься на лодках, пообедать в кафе и повеселиться.

\ii{23_01_2022.stz.news.ru.odnarodyna.1.kiev.pic.6}

Увы, но, многие представители нового поколения, похоже, забыли о благодарности
тем, кто их дедов и прадедов переселил из бараков в благоустроенные квартиры,
да ещё в таком прекрасном районе. Сей неутешительный вывод напрашивается при
ознакомлении с результатами голосований на выборах в течение последних лет.
Неоднократно жители Нивок выбирали во власть представителей национал-радикалов
из «Свободы». Поэтому нет ничего удивительного в том, что граждане довольно
безразлично реагируют на переименование родных улиц. Вот лишь несколько
примеров.

Центральную магистраль района, десятилетиями носившую имя Александра Щербакова
(начальника Совинформбюро с 24 июня 1941 года, начальника Главного
политуправления РККА, заместителя народного комиссара обороны СССР) «подарили»
украинскому этнографу Даниилу Щербаковскому, покончившему жизнь самоубийством.
Видимо, созвучную фамилию подбирали, дабы не раздражать старожилов.

\ii{23_01_2022.stz.news.ru.odnarodyna.1.kiev.pic.7}

Улицу Вильгельма Пика, немецкого коммуниста, переиначили в Ружинскую. Невскую
сделали Нивской. Ту самую Саратовскую, где растет знаменитый дуб, планируют
«отдать» балетмейстеру Павлу Вирскому.  Но если с Саратовской и Невской всё
понятно, причиной переименования Танковой является... обращение посольства США к
городским властям столицы Украины. Американцам не нравилось название улицы, на
которую переехало посольство США. Вопрос решался, не поверите, в неотложном
порядке, поскольку Штаты собирались открывать новый офис посольства, а
аргументировали спешное переименование заменой устаревшего названия на более
актуальное. Интересно, как бы сам Игорь Сикорский отнёсся к манипуляциям вокруг
его имени?

Так что США уже и улицы в Киеве «под себя» переименовывают...

\ii{23_01_2022.stz.news.ru.odnarodyna.1.kiev.pic.8}

Посольство США в Киеве и переименованная под его давлением улица Танковая...

На заглавном фото: Нивки и Киев с высоты птичьего полета

Записки киевлянина
