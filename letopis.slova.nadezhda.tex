% vim: keymap=russian-jcukenwin
%%beginhead 
 
%%file slova.nadezhda
%%parent slova
 
%%url 
 
%%author_id 
%%date 
 
%%tags 
%%title 
 
%%endhead 
\chapter{Надежда}

%%%cit
%%%cit_head
%%%cit_pic
%%%cit_text
Лишь попытка реализовать образовательный проект в Украине, помочь людям
раскрыть глаза на многие процессы в стране.  Сказать то, о чем все молчат.
Позапрошлая передача - монетарный суверенитет и инфляция.  Прошлая - трудовая
бедность и социальная сегрегация.  Обе передачи имели хорошие просмотры с
учетом того, что это не классическое обливание грязью в стиле нынешних ток-шоу.
Сейчас на ютубе можно посмотреть Соловьева и Скабееву, но нельзя посмотреть
нашу передачу (сразу скажу - она будет выходить и дальше).  И я догадываюсь
почему.  Наша передача дает альтернативный взгляд и всегда завершается словами:
"І памятайте, економічному диву в Україні бути".  А это самое страшное для
"цензоров" - люди не должны получить \emph{надежду} и альтернативу....
%%%cit_comment
%%%cit_title
\citTitle{Люди не должны ни в коем случае получить надежду и альтернативу / Лента соцсетей / Страна}, 
Алексей Кущ, strana.news, 29.11.2021
%%%endcit
