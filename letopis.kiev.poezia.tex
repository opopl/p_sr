% vim: keymap=russian-jcukenwin
%%beginhead 
 
%%file kiev.poezia
%%parent kiev
 
%%url 
 
%%author 
%%author_id 
%%author_url 
 
%%tags 
%%title 
 
%%endhead 

О Киев-град, где с верою святою
Зажглася жизнь в краю у нас родном,
Где светлый крест с Печерскою главою
Горит звездой на небе голубом,
Где стелются зеленой пеленою
Поля твои в раздольи золотом,
И Днепр-река, под древними стенами,
Кипит, шумит пенистыми волнами!
Как часто я душой к тебе летаю,
О светлый град, по сердцу мне родной!
Как часто я в мечтах мой взор пленяю
Священною твоею красотой!
У Лаврских стен земное забываю,
И над Днепром брожу во тьме ночной:
В очах моих все русское прямое —
Прекрасное, великое, святое.
Уж месяц встал; Печерская сияет;
Главы ее в волнах реки горят;
Она душе века напоминает;
Небесные там в подземелье спят;
Над нею тень Владимира летает;
Зубцы ее о славе говорят.
Смотрю ли вдаль — везде мечта со мною,
И милою всё дышит стариною.
Там витязи сражались удалые,
Могучие, за родину в полях;
Красою здесь цвели княжны младые,
Стыдливые, в высоких теремах,
И пел Баян им битвы роковые,
И тайный жар таился в их сердцах.
Но полночь бьет, звук меди умирает;
К минувшим дням еще день улетает.
Где ж смелые, которые сражались,
Чей острый меч, как молния, сверкал?
Где та краса, которой все пленялись,
Чей милый взгляд свободу отнимал?
Где тот певец, чьим пеньем восхищались,
Ах, вещий бой на всё мне отвечал!
И ты один под башнями святыми
Шумишь, о Днепр, волнами вековыми!

Козлов Иван

https://www.facebook.com/groups/story.kiev.ua/permalink/1674586756071452/
31.05.2021

Тина Тука
https://www.facebook.com/groups/736908309839306/user/100010758256475/

\ifcmt
  pic https://scontent-lga3-2.xx.fbcdn.net/v/t1.6435-9/193339301_1420759811625947_6129633900350245662_n.jpg?_nc_cat=108&ccb=1-3&_nc_sid=825194&_nc_ohc=aYXR-ba9uK4AX-rPKPq&_nc_ht=scontent-lga3-2.xx&oh=e2c8f7597e4d57ae119911fb6da3fd9f&oe=60DD4B6E
\fi

С праздником, КИЕВ!!!
Люблю тебя, ночной мой город,
Укрытый пологом туманных звезд.
Мне с детства ты безумно дорог, - 
Свидетель грусти, смеха, слёз.
Люблю за чуть прохладный воздух,
Пронизанный лучами фонарей, 
Где над брусчаткой виснут грозди,
Абсурдно-вычурных теней.
Люблю за лужи в желтых светофорах,
За золотую дымку у пустых витрин.
Неспешные прохожих разговоры,
И редкое бурчание машин.
За джаз из окон ресторанов,
Глаза влюблённых за столом,
За хруст упавших под ноги каштанов,
За много взглядов, мало слов…


%https://story.kiev.ua/publikacii/boris-lushelskij-istorii-sozdanija-kieva-posvjashhaetsja/
