% vim: keymap=russian-jcukenwin
%%beginhead 
 
%%file 13_09_2021.fb.nikonov_sergej.4.bilchenko_lekcia.cmt
%%parent 13_09_2021.fb.nikonov_sergej.4.bilchenko_lekcia
 
%%url 
 
%%author_id 
%%date 
 
%%tags 
%%title 
 
%%endhead 
\subsubsection{Коментарі}
\label{sec:13_09_2021.fb.nikonov_sergej.4.bilchenko_lekcia.cmt}

\begin{itemize} % {
\iusr{Галина Шулаева}

Нет у меня к Жене сочувствия.Ни грамма нет.Её вина такая же неизбывная, как
наша в 1991 году.Тогда нам было примерно столько же лет, сколько Жене.Нисколько
с себя не снимаю ответственности, но тем не менее некто злой и умелый сделал
так, что нашей великой страны не стало...

И уже в 2014 году, наученные своим горьким опытом, битые-перебитые
последствиями 1991-го Жене криком кричали:"Остановись! Не тяни хотя бы
детей-студентов за собой, они тебе верят".Искалеченные судьбы тех, кто в живых
остался, сама неприкаянная...

Перечитала свою же первую фразу ..Эх, была бы она рядом, приобняла и сказала
по-матерински:" Ну что, коза, понаделали мы дел каждая в своем времени.Никого
чужие ошибки не учат.Давай исправлять."Да пребудут с нами Якобсон, Лотман,
Межуев и Тульчинский! "Глядишь лет через тридцать( история повторяется
циклично) какую революционерку и остановишь на пути к "светлому будущему".

\begin{itemize} % {
\iusr{Alex Solovyov}
\textbf{Галина Шулаева} Она-то как раз исправляет.

\iusr{Андрей Лесто}
\textbf{Галина Шулаева} да хрен кого остановишь, когда шило в жопе и романтизм в душе. В опыта нет.
\end{itemize} % }

\end{itemize} % }

