% vim: keymap=russian-jcukenwin
%%beginhead 
 
%%file 24_01_2022.stz.news.ua.mrpl_city.1.personalna_vystavka_anny_kashkovoj
%%parent 24_01_2022
 
%%url https://mrpl.city/blogs/view/personalna-vistavka-anni-kashkovoi
 
%%author_id demidko_olga.mariupol,news.ua.mrpl_city
%%date 
 
%%tags 
%%title Персональна виставка Анни Кашкової
 
%%endhead 
 
\subsection{Персональна виставка Анни Кашкової}
\label{sec:24_01_2022.stz.news.ua.mrpl_city.1.personalna_vystavka_anny_kashkovoj}
 
\Purl{https://mrpl.city/blogs/view/personalna-vistavka-anni-kashkovoi}
\ifcmt
 author_begin
   author_id demidko_olga.mariupol,news.ua.mrpl_city
 author_end
\fi

\ii{24_01_2022.stz.news.ua.mrpl_city.1.personalna_vystavka_anny_kashkovoj.pic.1}

18 січня в рамках проєкту \enquote{Скарби талановитої молоді Донеччини}, присвяченого
80-ти річчю Донецької області, у МПК \enquote{Чайка} відбулося відкриття персональної
вистави \emph{\enquote{Зимові фантазії}} \emph{\textbf{Анни Кашкової}}, талановитої учениці Маріупольської
школи мистецтв. Загалом січень 2022 року вийшов дуже плідним для Маріуполя на
художні виставки. Проте саме виставка \enquote{Зимові фантазії} для більшості
маріупольців стала найбільш вражаючою та унікальною.

Малює Анна з раннього дитинства. Свою першу роботу \enquote{Квітуче поле} створила у 4
роки і посіла друге місце у міському конкурсі образотворчого мистецтва. Наразі
Анна навчається у 10 класі ЗОШ №41. Вона життєрадісна, чуйна людина, яку
приваблює краса навколишнього світу, тому дівчина часто подорожує, відвідує
художні музеї та виставки, бере участь у різних майстер-класах. Роботи
художниці (а їх вже більше 70) були представлені у Фінляндії, Польщі, Франції,
Німеччині, Греції, Македонії, Болгарії та Грузії. Загалом Анна стала учасницею
25-ти міжнародних конкурсах і майже завжди була лауреатом І премій та володарем
Гран-прі. На цій виставці головним став саме зимовий сюжет, адже це улюблена
пора року Анни. Працює художниця дуже швидко. Вона може і за тиждень створити
велике полотно. Дівчину надихають талановиті митці та їхні унікальні картини і
підтримка близьких та рідних людей. Хрещена мати і тітка Анни Кашкової \textbf{Ірина
Верлан} розповіла, що дівчина з раннього дитинства виявляє усидливість, терпіння
і працьовитість. Вона пише переважно у світлих, теплих тонах, передає позитив,
настрій та теплоту своєї душі.

\ii{24_01_2022.stz.news.ua.mrpl_city.1.personalna_vystavka_anny_kashkovoj.pic.2}

Це вже третя персональна виставка живопису юної художниці, на якій можна
побачити 28 унікальних творчих робіт. Тут і зимові пейзажі, і маріупольська
архітектура, і неповторні портрети. Девіз Анни: \emph{\enquote{Розфарбуй життя
яскравими кольорами}}, тому її картини дуже яскраві, насичені та містять
своєрідний сюжет і колорит. Вражають її художня майстерність та технічне
виконання. Особливу увагу привертають до себе \emph{\enquote{Старовинний
будинок Гампера}}, \emph{\enquote{На околиці рідного міста}},
\emph{\enquote{Бурхливе море}}. Це пейзажі рідного міста Маріуполь, яке дуже
любить юна художниця.  Її викладачка, відома маріупольська художниця
\emph{\textbf{Любов Макаренко}} пишається успіхами своєї учениці. Вона
наголосила, що

\begin{quote}
\em\enquote{для своїх 15 років Анна
дуже цілеспрямована. Дівчина знає, чого хоче. Пробує себе у всіх техніках: і в
акварелі, і в гуаші, і в акрилі, і в маслі, і в графіці, тобто діапазон і
сюжетів, і технік у неї неймовірно великий. Молодість художниці, енергія,
радість, жвавість – у всіх її роботах проглядається дуже яскраво та виразно.}
\end{quote}

Любов Василівна розповіла, як Анна остаточно вирішила стати художницею.
Виявляється, що на неї вплинув інший учень Маріупольської школи мистецтв \emph{\textbf{Антон
Кобзєв}}. Дівчинка була вражена роботами юного художника і заявила, що, якщо вона
не почне малювати як він, то зовсім покине цю справу. Анна Кашкова не стала
малювати як Кобзєв, у неї зовсім інша техніка, але їх обох об'єднує унікальний
талант і неймовірна працездатність. Насправді не кожен досвідчений художник має
стільки робіт, скільки наразі створила п'ятнадцятирічна Анна.

\ii{24_01_2022.stz.news.ua.mrpl_city.1.personalna_vystavka_anny_kashkovoj.pic.3}

Наша героїня стала переможницею у 24-ти Всеукраїнських та обласних
фестивалях-конкурсах і 16 міських. Отримала безліч дипломів, медалей та кубків.
Бере активну участь у житті міста. Зокрема, у 2018 році проілюструвала книги
до100-річчя Василя Сухомлинського, а у 2020 році  ілюструвала казку у книзі
Наталі Спесивцевої \enquote{Небесні казки}.

Анна побувала на Всеукраїнських пленерах у Львові, Одесі, Хмельницькому,
Івано-Франковську, Вінниці, Коблево, Хмельнику, Урзуфі, Спілкувалась з відомими
українськими художниками, що дало їй нові емоції, незабутні враження та
енергію. З вересня 2021 року отримує стипендію Маріупольської міської ради.
Рідні в усьому підтримують талановиту і обдаровану дівчину.

Персональна виставка Анни Кашкової \enquote{Зимові фантазії} на другому поверсі МПК
\enquote{Чайка} триватиме до 11 лютого. До речі, наразі в рамках проєкту \enquote{Скарби
талановитої молоді Донеччини} двері Міського палацу культури \enquote{Чайка} відкриті
для всіх талановитих маріупольців, які мають бажання представити свої роботи
широкому загалу.

