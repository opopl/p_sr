%%beginhead 
 
%%file 11_08_2023.fb.muzej_istorii_ukrainy.kyiv.1.diorama_kyiv_10_13_st
%%parent 11_08_2023
 
%%url https://www.facebook.com/mist.museum/posts/pfbid02RRKwyad9hpyaN4cjMsx1ZkHjhGbk6DkW7QSbEciCWHk9CVHbjYARBy2nA6ymjSWUl
 
%%author_id muzej_istorii_ukrainy.kyiv
%%date 11_08_2023
 
%%tags 
%%title Діорама Київ 10-13 століть
 
%%endhead 

\subsection{Діорама Київ 10-13 століть}
\label{sec:11_08_2023.fb.muzej_istorii_ukrainy.kyiv.1.diorama_kyiv_10_13_st}

\Purl{https://www.facebook.com/mist.museum/posts/pfbid02RRKwyad9hpyaN4cjMsx1ZkHjhGbk6DkW7QSbEciCWHk9CVHbjYARBy2nA6ymjSWUl}
\ifcmt
 author_begin
   author_id muzej_istorii_ukrainy.kyiv
 author_end
\fi

У Національному музеї історії України є місце, яке притягує увагу абсолютно
всіх відвідувачів. Це місце, де можна довго стояти і давати волю своїй
фантазії, уявляючи давній Київ, що розкинувся на Старокиївській горі, та
могутню давньоруську державу, столицею якої він був. Це – зала з діорамою \enquote{Київ
10 – 13 століть}, об'ємною реконструкцією панорами стародавнього міста. 

\ifcmt
  ig https://scontent-fra3-1.xx.fbcdn.net/v/t39.30808-6/366066099_811766500644781_760681870373418797_n.jpg?_nc_cat=103&ccb=1-7&_nc_sid=730e14&_nc_ohc=N1x2KsA9gvIAX9Y4_dw&_nc_ht=scontent-fra3-1.xx&oh=00_AfC9iKh8Rx7efdFDVxGyRF6t2lzwN6ArIr6UU-UdO1AFGg&oe=64DA7704
  @wrap center
  @width 0.9
\fi

Розповімо вам кілька фактів про цей дивовижний мистецький витвір.

\begin{itemize}
\item 1 Створена діорама художником Олексієм Казанським у 1977 році. Митець працював
над цим масштабним проєктом понад три роки! Макет було виготовлено на основі
детального вивчення середньовічної топографії міста, а також результатів
археологічних досліджень. 

\item 2 Висота діорами – 3,5 м, ширина – 4,8 м, глибина 2,5 м. При створенні художник
використовував  дерево, гіпс, скло, полотно, папір, олію, гуаш і навіть
пластилін. 

\item 3 На діорамі з ювелірною майстерністю та деталізацією відтворено величне
давньоруське місто Київ. Об'ємна модель наочно демонструє, як розвивалось місто
за часів князювання Володимира Великого, Ярослава Мудрого та його синів. А
також який вигляд раніше мали Десятинна церква, Софійський собор, Золоті
ворота, Михайлівський Золотоверхий собор, Поділ, де протікали річки Глибочиця
та Почайна.
\end{itemize}

📍Роздивитись деталі діорами та дізнатись більше про середньовічний Київ
запрошуємо вас у суботу, 12 серпня, на Екскурсія вихідного дня: Київ 10-13
століть. Початок о о 14:00.

На екскурсію потрібно попередньо записатися.

Телефон для запису – (044) 278-48-64 або 093 855 61 16 (Вайбер та Телеграм).

Адреса музею: м. Київ, вулиця Володимирська, 2.

\par\noindent\rule{\textwidth}{0.4pt}

Вартість – 150 грн, для школярів, студентів, пенсіонерів – 100 грн.
