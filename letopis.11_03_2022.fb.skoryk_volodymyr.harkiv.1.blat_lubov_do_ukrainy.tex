% vim: keymap=russian-jcukenwin
%%beginhead 
 
%%file 11_03_2022.fb.skoryk_volodymyr.harkiv.1.blat_lubov_do_ukrainy
%%parent 11_03_2022
 
%%url https://www.facebook.com/volodymyr.skoryk/posts/5164641480242180
 
%%author_id skoryk_volodymyr.harkiv
%%date 
 
%%tags 
%%title Мій блат - це любов до України
 
%%endhead 
 
\subsection{Мій блат - це любов до України}
\label{sec:11_03_2022.fb.skoryk_volodymyr.harkiv.1.blat_lubov_do_ukrainy}
 
\Purl{https://www.facebook.com/volodymyr.skoryk/posts/5164641480242180}
\ifcmt
 author_begin
   author_id skoryk_volodymyr.harkiv
 author_end
\fi

Я завжди був пацифістом. Ні, я не ховався від військкомату, я не відкидав
можливості йти служити. Але це було не моє. Я свідомо відмовився від воєнки в
університеті, хоч документально був придатний і міг бути покликаний до
строкової служби. Не робив собі білий квиток, хоч і міг його зробити, і мама та
рідні мене переконували в цьому, бо є захворювання серця, яке давало мені це
право. Я залишив собі шанс боронити свою країну. Про те, що її треба буде
боронити, я розумів з кінця 90-х, зі школи, коли вивчав історію, хоч і відганяв
від себе цю думку, але холодною головою розумів, що це невідворотньо і вони не
полишать нас. Така вже сутність цієї (і будь-якої) імперії, яка вкрала в нас
історію. Без нас їх немає.

\ii{11_03_2022.fb.skoryk_volodymyr.harkiv.1.blat_lubov_do_ukrainy.pic.1}

Тут мене вже кілька разів питали, як так сталося, що я потрапив до тероборони
та отримав зброю, який я блат використав?)) Мій блат - це любов до України. Хоч
я і міг скористатися для цього зв'язками в патріотичних колах, але це не мій
шлях. Перший день війни я був розгублений та не міг робити нічого, на другий
день пішов до свого військкомату, але виявилося, що найвеличніші вояки звідтіля
здатні лише довбати повістками та брати хабарі в мирний час, словом,
військкомат був вже евакуйований. Подивився на зачинені двері та розкидані
папери та пішов собі додому. На третій день з'явилася інформація про тероборону
в ХОДА, я пішов взнати, що там та як, навіть речей особистих не взяв. І все. З
тих пір я вдома ще не був. Спочатку я прийшов в якості водія, а їм обов'язково
давали зброю, навіть якщо ти не хотів і не мав досвіду. Як раз після того, як в
нас в ХОДА прилетіли снаряди, я дорогою заскочив на 30 секунд до бомбосховища,
де переховувалися Оля з дітьми, поцілував їх наостанок та благав негайно
забиратися з міста, бо боронити країну набагато легше, коли рідні в безпеці.
Потім автомати в таких, як я, забирали, але на той час у мене досвіду трохи
з'явилося і я попросив залишити собі свою зброю. Командир вже бачив мене в
справі, то не заперечував.

Чи моє це? Ні. Я й досі не відчуваю себе військовим. Але сидіти осторонь я не
можу. Бо хто, як не ми!? Коли, як не зараз!?

\#українськавесна

\ii{11_03_2022.fb.skoryk_volodymyr.harkiv.1.blat_lubov_do_ukrainy.cmt}

