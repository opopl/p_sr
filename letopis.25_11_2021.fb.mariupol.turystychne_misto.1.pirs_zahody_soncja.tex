%%beginhead 
 
%%file 25_11_2021.fb.mariupol.turystychne_misto.1.pirs_zahody_soncja
%%parent 25_11_2021
 
%%url https://www.facebook.com/mistoMarii/posts/pfbid0DcwCPb7jJKeDeS2YSMxYwgYtew2wE4Lj4XohSVHq24zvxF3Y1DrA7jYe6LPSJF66l
 
%%author_id mariupol.turystychne_misto
%%date 25_11_2021
 
%%tags 
%%title Оновлений пірс  в Маріуполі - найкраще місце для споглядання заходів сонця
 
%%endhead 

\subsection{Оновлений пірс  в Маріуполі - найкраще місце для споглядання заходів сонця}
\label{sec:25_11_2021.fb.mariupol.turystychne_misto.1.pirs_zahody_soncja}

\Purl{https://www.facebook.com/mistoMarii/posts/pfbid0DcwCPb7jJKeDeS2YSMxYwgYtew2wE4Lj4XohSVHq24zvxF3Y1DrA7jYe6LPSJF66l}
\ifcmt
 author_begin
   author_id mariupol.turystychne_misto
 author_end
\fi

Оновлений пірс  в Маріуполі - найкраще місце для споглядання заходів сонця. 

Цього літа було завершено реконструкцію другої частини пірса, і він став
популярним простором для неспішних прогулянок з сім'єю чи наодинці.

\#ТутВарто:

☀️зустріти світанок\par
🥤зробити перерву на каву \par
🦆спостерігати за чайками та качками\par
🌊послухати звуки міста біля скульптури рожевої хвилі\par
📸зробити дзеркальне селфі біля футуристичного кубу\par

Дістатися сюди можна з центру маршрутом №105 (до зупинки \enquote{Санаторій Металург}),
тролейбусом № 4 або пройтися 20 хвилин пішки від Грецької площі проспектом
Металургів до Приморського бульвару/

Пирс Мариуполь
