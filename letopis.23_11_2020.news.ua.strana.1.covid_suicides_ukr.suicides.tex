% vim: keymap=russian-jcukenwin
%%beginhead 
 
%%file 23_11_2020.news.ua.strana.1.covid_suicides_ukr.suicides
%%parent 23_11_2020.news.ua.strana.1.covid_suicides_ukr
 
%%url 
 
%%author 
%%author_id 
%%author_url 
 
%%tags 
%%title Самоубийства в Украине из-за коронавируса
 
%%endhead 

\subsubsection{Самоубийства в Украине из-за коронавируса}
\label{sec:23_11_2020.news.ua.strana.1.covid_suicides_ukr.suicides}

В Украине за время пандемии было зафиксировано немало случаев самоубийств из-за
коронавируса. Мы приводим те, которые освещались в СМИ. Не исключено, что их
было значительно больше.

В рамках всех расследований возбуждены дела по статье 115 УК Украины -
\enquote{Умышленное убийство}.

\textbf{Киев, 78-летняя женщина}. Трагедия случилась 20 ноября в больнице №12,
которая находится на улице Профессора Подвысоцкого в Печерском районе
столицы.

По данным источников \enquote{Страны} в силовых структурах, женщина была
обнаружена на пятом этаже - где лечатся от ковида. Тело находилось в
петле, медики нашли его уже без признаков жизни.

\textbf{Запорожье, 36-летняя женщина}. У украинки был не коронавирус, а подозрение
на него.

19 ноября женщина ожидала в коридоре врача на четвертом этаже. Но внезапно
пошла и выпрыгнула из окна. По некоторым данным, она страдала психическим
заболеванием. 

\textbf{Сумы, 53-летняя женщина}. Она была доставлена в одну из больниц Глуховского
района Сумской области. Диагноз - коронавирус и двусторонняя пневмония.

Во время госпитализации состояние было тяжелое. Однако ей удалось дойти до
туалета и там повеситься. Тело обнаружила соседка по палате 11 ноября.  Родные
покойной утверждают, что не замечали раньше у нее суицидального поведения. 

\textbf{Луцк, 76-летний мужчина} покончил жизнь самоубийством в одной из больниц
города 29 октября. У него был ковид. Детали не уточняются. 

\textbf{Харьков, 69-летний мужчина} выпрыгнул из окна четвертого этажа 25 октября. 

Пациент находился в реанимации с подтвержденным ковидом. Болезнь протекала
в очень тяжелой форме. Он был на аппарате ИВЛ.

\textbf{Киев, 60-летняя женщина}. Пациентка 20 октября выбросилась из
окна четвертого этажа инфекционного отделения. Женщине диагностировали
многочисленные травмы.

Скончалась она не сразу - медики боролись за до утра следующего
дня. Больная ковидом скончалась в реанимации от несовместимых с жизнью
травм. 

\textbf{Киев, мужчина}. Заболевший коронавирусом пациент лежал в городской
больницы №8 на Оболони. 

По данным СМИ, 20 октября мужчина надел костюм и выбросился из окна.
Смерть наступила мгновенно. Предсмертной записки при нем обнаружено не
было. 

\textbf{Днепр, 45-летний мужчина}. Мужчина работал врачом в больнице имени
Мечникова, ему диагностировали ковид. Покончить жизнь самоубийством он
решил с помощью оружия. 

Медик пребывал не на работе - он застрелился в своей квартире на улице
Гоголя 9 октября.\Furl{https://dnepr.strana.ua/294222-v-dnepre-vrach-bolnitsy-mechnikova-zarazilsja-koronavirusom-i-zastrelilsja.html} Рядом с тело врача лежал пистолет. 

\textbf{Киев, мужчина} покончил жизнь самоубийством 22 сентября, выпрыгнув из окна
клинической больнице №4 в Соломенском районе столицы. Он, как и предыдущие
граждане, лечился от ковида. 

\textbf{Киев, мужчина} вышел из окна\Furl{https://kiev.strana.ua/291043-iz-okna-kievskoj-bolnitsy-za-dvoe-sutok-vypalo-dva-cheloveka-.html} в той же киевской больнице №4 21 сентября.
Нельзя исключать, что он мог послужить примером для пациента, который
самоубился там же и таким же способом днем позже. 

\textbf{Киев, мужчина} не смог продолжить борьбу с коронавирусом. Пациенту больницы
в Днепровском районе столицы исполнился 81 год. Он выпрыгнул из пятого
этажа 6 августа,\Furl{https://strana.ua/news/282814-pozhiloj-muzhchina-vypal-iz-okna-bolnitsy-v-kieve.html} смерть была моментальной. 

\textbf{Тернополь, 39-летний мужчина}. Он не был болен коронавирусом, но находился
на обсервации в больнице. Трагедия случилась в самом начале пандемии, 24
апреля. Накануне украинец вернулся из Польши, где был на заработках.
Мужчина был женат, воспитывал четвертых детей.

Признаков ковида и ухудшения самочувствия у него не было. Накануне
очередного обхода врачей он заперся в палате. Нашли пациента без признаков
жизни. Способ самоубийства не сообщался. 

