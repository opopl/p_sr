%%beginhead 
 
%%file 21_11_2020.fb.fb_group.mariupol.nekropol.1.otchet_volonter_rabota_14_15_nojabrja_2020
%%parent 21_11_2020
 
%%url https://www.facebook.com/groups/278185963354519/posts/416737319499382
 
%%author_id fb_group.mariupol.nekropol,arximisto
%%date 21_11_2020
 
%%tags 
%%title Отчет о волонтерской работе в Некрополе 14-15 ноября 2020
 
%%endhead 

\subsection{Отчет о волонтерской работе в Некрополе 14-15 ноября 2020}
\label{sec:21_11_2020.fb.fb_group.mariupol.nekropol.1.otchet_volonter_rabota_14_15_nojabrja_2020}
 
\Purl{https://www.facebook.com/groups/278185963354519/posts/416737319499382}
\ifcmt
 author_begin
   author_id fb_group.mariupol.nekropol,arximisto
 author_end
\fi

\textbf{Отчет о волонтерской работе в Некрополе 14-15 ноября 2020}

\textbf{Открытия и находки}

Елена Сугак обнаружила несколько обломков надписей из старинной усыпальницы
рядом со склепом Александра Хараджаева. Постепенно приближаемся к ответу на
вопрос - а кто же был в ней похоронен?

Александр и Наталия Шпотаковская обнаружили и расчистили ограду захоронений
семейства Альбертов (рядом со склепом Спиридона Гофа). Кем они были -
неизвестно. Поиск по метрикам результатов не принес. Будем благодарны за любую
информацию!

Наконец, Андрей Марусов, Александр и Наталия обнаружили основание для памятника
рядом с древними плитами (одна из них с датой - 1848). Этот участок вообще
толком не исследовался, поэтому его обследование - в наших ближайших
приоритетах.

\textbf{Благоустройство}

Мы посадили на древнем участке и аллее восемь саженцев кленов, подаренных
ландшафтным дизайнером Станиславом Косоноговым (Березовским).

Maryna Holovnova и Андрей Марусов покрасили на выходных бОльшую часть
заброшенных крестов на древнем участке.

Мы восстановили пару разваленных советских надгробий. К сожалению, еще два
бетонных памятника не удастся привести в порядок – бетонные блоки слишком
тяжелые, чтобы их можно было поднять вручную.

Илья Луковенко расчистил подножие склепа Спиридона Гофа от земли и мусора. 

\textbf{Использование пожертвований}

На сегодня мы получили 6 150 грн. благотворительных взносов на восстановление
Некрополя, включая недавнее пожертвование в 1 500 грн. от команды Вежа - Vezha
Creative Space (выручка от экскурсий). Мариупольский ландшафтный дизайнер
Станислав Косоногов (Березовский) подарил восемь саженцев канадского клена.

Огромное спасибо всем благотворителям!

Мы потратили 1 396,92 грн.%
\footnote{Internet Archive: \url{https://archive.org/details/13_11_2020.fb.arximisto.mrpl_nekropol.pozhertvovania_otchet_1}}

\textbf{Дальнейшие планы}

Погода вносит свои коррективы в волонтерскую работу. Из-за снегопада мы
вынуждены пропустить эти выходные.

Завершение покраски оград и крестов на заброшенных могилах древнего участка,
очистка от зарослей древнего участка, разметка территории для прокладывания
дорожек - среди ближайших приоритетов.

Мы также рассматриваем возможность установки мощных светильников на солнечных
батареях. Еще летом благодаря поддержке предпринимателя-мецената Сергей
Бурлаков мы купили и установили светильники на крест усыпальницы Найденовых, а
также около ста небольших фонарей - вдоль ключевых аллей Некрополя. К
сожалению, через пару месяцев все небольшие фонари были украдены, но крест
по-прежнему сияет ночью!

Мы раздумываем над установкой нескольких светильников на опорах линии
электропередачи (их трудно украсть). Тем более, что цены на них упали...

\#mariupol\_necropolis\_report
\#mariupol\_necropolis\_spending
