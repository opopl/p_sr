% vim: keymap=russian-jcukenwin
%%beginhead 
 
%%file 18_12_2020.news.ua.obozrevatel.petik_marina.1.russia_vaccine_view_from_ukraine
%%parent 18_12_2020
 
%%url https://www.obozrevatel.com/society/temperatura-bol-v-sustavah-i-mushki-pered-glazami-rossijskie-dobrovoltsyi-rasskazali-kak-perenesli-privivku-sputnikom.htm
 
%%author Петик, Марина
%%author_id petik_marina
%%author_url 
 
%%tags covid_vaccine
%%title В России начали "вакцинировать" от коронавируса: люди рассказывают о мучительных эффектах, им угрожают
 
%%endhead 
 
\subsection{В России начали \enquote{вакцинировать} от коронавируса: люди рассказывают о мучительных эффектах, им угрожают}
\label{sec:18_12_2020.news.ua.obozrevatel.petik_marina.1.russia_vaccine_view_from_ukraine}
\Purl{https://www.obozrevatel.com/society/temperatura-bol-v-sustavah-i-mushki-pered-glazami-rossijskie-dobrovoltsyi-rasskazali-kak-perenesli-privivku-sputnikom.htm}
\ifcmt
	author_begin
   author_id petik_marina
	author_end
\fi

В России началась кампания по вакцинации граждан от коронавируса препаратом
собственной разработки "Спутник V". Однако сами жители РФ не выстраиваются в
очереди за инъекциями. Бюджетники жалуются, что их заставляют делать прививку
под страхом увольнения. А добровольцы рассказывают о побочных эффектах вакцины.

Подробнее об этом читайте в материале OBOZREVATEL.

\subsubsection{Увольняются и делают прививки от гриппа}

После того, как в России начали прививать "Спутником V" сотрудников больниц,
поликлиник, школ, социальной сферы, граждане начали заявлять о том, что
делается это добровольно-принудительно. В некоторых учреждения работники даже
пишут заявления об увольнении.

Так в поликлинике №3 города Москвы заявления об увольнении подали сразу 11
медицинских сотрудников. Поводом стал приказ руководства привить всех медиков
этого учреждения вакциной от COVID-19.

Об увольнении медработников из-за принудительной вакцинации от коронавируса
заявила правозащитница, член Общественной наблюдательной комиссии Москвы Марина
Литвинович.

\ifcmt
pic https://i.obozrevatel.com/gallery/2020/12/17/massovo-uvolnyayutsya-vrachi.png
\fi

Департамент здравоохранения Москвы бросился опровергать информацию, сообщив,
что приказ не принуждает медицинских работников к обязательной вакцинации.

\ifcmt
tab_begin cols=2
	pic https://i.obozrevatel.com/gallery/2020/12/17/screenshot24.png
	pic https://i.obozrevatel.com/gallery/2020/12/17/screenshot13.png
tab_end
\fi

Та же ситуация и в российских школах. "У нас в школе единицы изъявили желание
привиться "Спутником". Многие опасаются, что он не прошел все испытания –
результаты тестирований до сих пор толком никто не видел. О них только говорят,
но официально ничего не показывают. Поэтому многие просто ждут, как будет
работать эта вакцина. Будут ли "побочки", будут ли привитые люди болеть. Я тоже
подожду полгодика, не хочу рисковать", – рассказала OBOZREVATEL учительница
начальных классов одной из школ Москвы.

Врач-педиатр из Санкт-Петербурга коронавирусом переболела еще весной. "Он
прошел у меня в легкой форме, я сидела дома на изоляции, муж даже еду приносил
к дверям, чтобы не контактировать. Сейчас прививаться не хочу, у меня антител
достаточно. Но на работе дали понять, что надо. Поэтому многие у нас взяли и
сделали прививки от гриппа. После нее вакцину от COVID-19 можно делать не
ранее, чем через месяц-два. Так что еще посмотрим", – рассказывает OBOZREVATEL
врач.

\subsubsection{Очередей на вакцинацию нет}

Тем временем в российских поликлиниках, где начали делать прививки очередей
нет. Как рассказывают те, кто уже сходил на укол, один флакон вакцины рассчитан
на пять человек и хранить его открытым можно не более 2 часов. Поэтому медики
обзванивают тех, кто изъявил желание и записался в электронную очередь, и
спрашивают, придут ли они.

\ifcmt
pic https://i.obozrevatel.com/gallery/2020/12/17/gettyimages-1230149337.jpg
\fi

"Просят прийти в определенное время, чтобы собрать сразу группу из 5 человек и
всех уколоть. Иначе вакцина пропадет. Мне медсестра сказала, что многие
записавшиеся отказываются по разным причинам. Я сходила, сделала укол. Не могу
сказать, что что-то почувствовала, болело только место укола, слегка знобило в
первый день. Через три недели сделаю вторую инъекцию – говорят, тогда могут
проявиться симптомы, температура. Посмотрим", – рассказывает Елена, воспитатель
детского сада из Москвы.

Для того, чтобы убедить россиян в безопасности "Спутника", в соцсетях массово
создают группы добровольцев, которые вроде как приняли участие в "народном
испытании" этой вакцины. Сейчас там появляются те, кто уже сделал прививку
после того, как власти РФ запустили "Спутник" в массы. Они делятся своими
впечатлениями.

Некоторые из них пишут, что почти ничего не почувствовали. Правда, части
добровольцев вводили плацебо, поэтому никаких симптомов могло и не быть.

\ifcmt
pic https://i.obozrevatel.com/gallery/2020/12/17/gettyimages-1230145575.jpg
\fi

Другие сообщают, что в течение суток была высокая температура. Также пишут, что
болит место укола, бывает трудно поднять руку. Насколько вакцина защитит от
коронавируса, пока никто не знает.

\ifcmt
pic https://i.obozrevatel.com/gallery/2020/12/17/posledstviya-privivki.png
\fi

Но бывалые добровольцы успокаивают сомневающихся и советуют обязательно делать
прививку и не обращать ни на кого внимания.

\subsubsection{Температура и мушки перед глазами}

В Телеграм-канале "Результаты вакцинации" один из добровольцев Adrian milandr
сообщает, что заболел коронавирусом сразу после введения второй дозы Спутника.
Хотя отмечает, что перенес болезнь легко: пять дней, температура 37,2 и потеря
обоняния на 12 дней.

\ifcmt
pic https://i.obozrevatel.com/gallery/2020/12/17/screenshot4.png
\fi

Еще один доброволец, Юрий, сообщил, что первый укол сделал 6 ноября. После
этого у него появилась головная боль, позже – озноб, боли в суставах,
температура поднялась до 38. Всю ночь он не мог уснуть, мелькали "мушки" перед
глазами, давление повысилось до 145/90. Также появилась тахикардия. Симптомы
ушли через пару дней.

После второй инъекции симптомы были те же: бессонница, тахикардия, боли в
глазных яблоках, интоксикация, температура 38,3. Он также отмечает, что после
второго укола было даже тяжелее, чем после первого.

\ifcmt
tab_begin cols=2
	pic https://i.obozrevatel.com/gallery/2020/12/17/yurij-dobrovolets.png
	pic https://i.obozrevatel.com/gallery/2020/12/17/screenshot36.png
tab_end
\fi

Еще один участник испытаний "Спутника" под именем Михаил Грищенко написал, что
они участвовали в эксперименте вместе с женой. И после прививки оба заболели.
По его мнению, им ввели плацебо.

"Мы как пионэры сообщили в телемедицинский центр, понимая, что данная инфа
должна быть им интересна, как потенциальное достижение конечной точки, что у
нас срочно возьмут мазки (на ПЦР-тест – Авт). Но этого не произошло. Апелляция
к здравому смыслу исследователя не помогла. В связи с чем возникает вопрос: мы
в двойном слепом плацебо-контролируемом РКИ участвуем или в большом
бессмысленном спектакле", – написал Грищенко

\ifcmt
pic https://i.obozrevatel.com/gallery/2020/12/17/nedovolnyij-platsebo.png
\fi

Позже он сообщил, что на 6-й день им все-таки предложили пройти ПЦР тест.

\subsubsection{Объявили в розыск}

Многодетная мать из Москвы Екатерина Серова оказалась участником исследования
"Спутник V". Она сделала инъекцию во время испытаний на третьем этапе. И
оказалась в сложной ситуации. Как описывает сама москвичка, на работе у нее
требуют справку о прохождении теста на антитела к COVID-19. Ей пришлось пройти
проверку и в результате – у нее повышенное содержание антител, что означает,
что она болеет коронавирусом. Несмотря на это, ей вкололи вторую дозу Спутника.
В итоге, она и ее семья оказались в розыске как люди, распространяющие
инфекцию.

"Это реальная катастрофа! Нас вакцинируют, а потом бросают на произвол судьбы!
Мне стыдно за всех, кто проводит исследования, начиная с института Гамалеи,
департамента здравоохранения и заканчивая 109 поликлиникой города Москвы.
Вытрепанные нервы и подорванное здоровье. Тысячу раз взвесьте, прежде чем
сделать прививку от коронавируса", – предупреждает женщина.

\ifcmt
pic https://i.obozrevatel.com/gallery/2020/12/17/rasskaz-dobrovoltsa21.png
\fi

\subsubsection{Почему не доверяют Спутнику}

Спешное заявление российских властей о создании чуть ли не первой вакцины от
коронавируса еще в августе 2020 года вызвало скандал среди мировых медиков и
ученых.

Как оказалось, россияне провели не полные клинические испытания своей вакцины,
решив внедрять ее в массы.

По заявлению российского министра здравоохранения Михаила Мурашко, в стране уже
привито более 100 тысяч человек. И это притом, что третья фаза испытаний еще не
была завершена.

\ifcmt
pic https://i.obozrevatel.com/gallery/2020/12/17/gettyimages-1230149428.jpg
\fi

На самом деле перед регистрацией этого препарата в испытаниях приняли участие
всего 76 волонтеров не старше 60 лет. Для сравнения, в испытаниях вакцины от
"Пфайзер" приняли участи 456 добровольцев.

Также создатели "Спутника" утверждают, что эффективность их вакцины составляет
95\%. Это тоже вызывает сомнение, поскольку нет открытых официальных данных
третьего этапа клинических испытаний.

Пока купить "Спутник V" из европейских стран решилась только Венгрия. В Украине
же министр здравоохранения Максим Степанов заявил, что мы не будем закупать
"Спутник", пока не завершится третий этап клинических испытаний и данные не
будут опубликованы ВОЗ. Напомним, третий этап носит масштабный характер, в нем
принимают участие десятки тысяч людей.

\begin{itemize}

\iusr{Сергей Коротовский}

Какая длинная бессмысленная ложь! Ведь у многих есть близкие, знакомые,
родственники в России. Зачем обоз врет? Кто-то заплатил.  Кстати в российском
Крыму, как и во всех регионах идет массовая вакцинация Спутником! Для бедных
стран разработали Спутник-лайт, дешевый, но с меньшей эффективностью. Для
возраста 60+ испытывается эпиваккорона. Построили новый завод на 10 млн доз в
месяц. Продолжайте врать!

\iusr{Свірський Андрій}

Сергей Коротовский А зачем на Хуйлостане 7 лет как врут про Украину???

\iusr{Куксов Михаил}

Особенно ПО ИДИОТСКИ выглядит заметка о " страданиях" *Многодетная мать из
Москвы Екатерина Серова"! После вакцинации - рост антител, и это ей создало
проблемы! А со здоровьем всё в порядке! А что по этой проблеме думают больные в
больницах Киева?

\iusr{Куксов Михаил}

КСТАТИ, строчка в Обоссе *люди рассказывают о мучительных эффектах* Тоже очень
выглядит по-идиотски, ЕСЛИ ПРОЧИТАТЬ О МУЧЕНИЯХ БОЛЬНЫХ \verb|КОВИД_19|

\iusr{Свірський Андрій}

Хуйло не положено прививать такой вакциной, а скоту она к стати.

\iusr{Oleg Salnikov}

скоту вообще никакой вакцины не положено. потому вам запад и отказал в своей вакцине и российскую закупать запретил.

\iusr{Куксов Михаил}

На сегодня в России. вместе с добровольцами уже около полмиллиона привитых .
Могло у кого-то быть ухудшение самочуствия? Да! Вспомните ПРИВИВКИ ОТ ОСПЫ.
Болела рука, появлялась повышенная температура. И совершенно не удивительно,
что в бандеровском Обоссе появилась эта статья КАК ПОПЫТКА убедить украинцев,
что ОТКАЗ от российской вакцины - это не команда из сша, а желание не навредить
здоровью украинцев " Смотрите, российская вакцина плохая! Спокойно ждём
американскую хорошую!"

\iusr{Свірський Андрій}

Kuksov Michael ОТКАЗ от российской вакцины - это команда из сша, а принять
вакцыну это команда от Хуйла или его кума Мертвечука???

\iusr{Эдуард Пухальский}

Свірський Андрій бля... ты дурак... Я вот читаю коммент Михаила и твоя реакция
на него.... Так вы на разных полюсах умственного развития находитесь. И, увы,
не в твою пользу. Не делай так больше. Пусть у россиян не будет повода
говорить, что украинцы все дебилы. Они ведь по тебе могут судить обо всей
стране. А я, например, не хочу, чтобы меня из-за тебя считали дебилом. Хотя,
согласен, площадка неудачная выбрана. Это не тот сайт, где умные черпают
информацию.

\iusr{Максим Исаев}

Эдуард Пухальский, не, дурак это ты! Поэтому и живёшь ты в Стране Дураков!

\iusr{Oleg Salnikov}

Помните что писал обоз про крымский мост. - сначала что его построить в
принципе не возможно, потом, что это всё снято на мосфильме, потом, что его
вот-вот снесет льдами. То-же самое и про вакцину. Сначала утверждали что её
вообще не существует, теперь признали что она есть. но она очень плохая. Ждём
продолжения марлезонского балет.

\iusr{Dmitrieva Larysa}

Ну чого тобі нейметься у своїй параші, усе заздрощі беруть, що у нас, в Україні
краще, слухай своє лебедіне озеро, до французського марлезонського балету, вам,
азіатам, як, до Київа, рачки, все, що не робите, стається через дупу.
Щеплюйтесь і далі своєю гидотою, скоро самі виздохнете, без стороннего
втручання.

\iusr{Oleg Salnikov}

И на обозе никогда не покажут мнение о российской вакцине действительно
профессионалов. \url{https://www.youtube.com/watch?v=mQaFm7oNwRk&t=8s}

\iusr{Максим Исаев}

Прививку не делают тем, кто переболел Ковид (хотел сделать - отказ), тем кто
планирует аллергическую реакцию, тем кто планирует беременность в ближайшие 3
месяца, тем у кого меньше месяца прошло с момента прививки от гриппа и тем, кто
чувствует недомогание (простуда и т.п.). После прививки возможно легкое
недомогание, похожее на усталость в течение 5 дней (у меня коллеги уже сделали
прививки).

\iusr{Сергей Воронов}

Как всегда - ни один из ярых защитников мочи пукина не написал: "вот Я
привился, все хорошо, никаких последствий, берите с меня пример", ну или хотя
бы : "а вот Я завтра иду прививаться, пошли вместе". Да и сам рашинский кащей
не торопится прививаться. К чему бы это? Сейчас нам расскажут, что они бы всей
душой, но им не дают такой возможности. Правильно? Особенно, если учесть, что
некоторые, типа михуя, спрятались от этого чуда под диванами за тысячи
километров от раши. А чего ждать от сволочей...

\iusr{Вадим Бурмистров}
Чучело, тебе хоть ссы в глаза всё божья роса.

\iusr{Oleg Salnikov}
ничего. тебя пиндосовской мочой уколят. от неё уже шестеро ласты склеили. даст
бог, и ты следом отправишься.

\iusr{Максим Исаев}
Народу уже немеряно прививку сделало. Я тут уже писал, что переболевшим не
делают, поэтому зря ездил. Со мной в смене парень сделал недели две назад.
Слабость, легкое недомогание было дней пять. В январе делать вторую часть. Ну и
что ты ещё хочешь узнать, чудо бандеровское?

\iusr{Сергей Воронов}
Максим Исаев да не нужно рассказывать про "парня". Всего навсего: пусть хоть
один, кому реально сделали прививку как положено - двумя дозами, так и напишет
"МНЕ сделали. Все нормально, поэтому всем говорю - нормальная вакцина!". А то
собралисб любители поп....ть на общую тему и распотякивают про великолепну
вакцину. А самим элементарно ИЗБЕГАЮТ вакцинирования, потому что раши - ТРУСЫ,
причем ПОДЛЫЕ! Вот и все дела. И больше не фиг пи....ть!

\iusr{Oleg Salnikov}

«Спутник V» Как она действует, какие имеет побочные эффекты и чем отличается от
зарубежных аналогов? Разработать первую в мире вакцину против коронавируса
помог созданный в 90-е задел.«Спутник V» создан в короткие сроки и в августе
был зарегистрирован как первая в мире вакцина от COVID-19. Это стало возможно
потому, что ничего революционного в ней нет, только проверенные алгоритмы.
Препарат представляет собой векторную вакцину на основе аденовируса человека —
такие разработки существуют в России с середины 90-х
годов.Научно-исследовательский центр эпидемиологии и микробиологии имени
Гамалеи — создатели «Спутника» — работали, например, над вакциной от лихорадки
Эбола.\Furl{https://polonsil.ru/blog/43267602743/Nazvanyi-glavnyie-osobennosti-vaktsinyi-Sputnik-V-?utm_referrer=mirtesen.ru}

\iusr{Oleg Salnikov}

Ряд зарубежных вакцин действуют по иному принципу, нежели «Спутник V». В
частности, американские препараты Pfizer и Moderna представляют собой так
называемые РНК-вакцины. Специалисты относят их к препаратам генной терапии. При
этом опыта применения таких прививок у человека раньше никогда не
было.РНК-вакцины могут вызвать редкие побочные эффекты, которые пока до конца
не исследованы. Среди них ученые называют аутоиммунные реакции и образование
тромбов. Выявить полную картину можно будет только после тестирования
препаратов на большом количестве добровольцев.

\iusr{Oleg Salnikov}

В начале декабря американское Управление по санитарному надзору за качеством
пищевых продуктов и медикаментов (FDA) сообщило, что у четверых добровольцев,
получивших вакцину Pfizer, развился паралич Белла. При этом расстройстве
наблюдается временная неработоспособность лицевого нерва. Среди других побочных
явлений ученые зафиксировали лихорадку, утомляемость, головную боль и мышечную
боль.

\end{itemize}
