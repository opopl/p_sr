% vim: keymap=russian-jcukenwin
%%beginhead 
 
%%file 20_04_2021.fb.bilchenko_evgenia.1.trud_bog_duh
%%parent 20_04_2021
 
%%url https://www.facebook.com/yevzhik/posts/3819537761414639
 
%%author 
%%author_id 
%%author_url 
 
%%tags 
%%title 
 
%%endhead 

\subsection{БЖ. О долге, науке, творчестве и любви}
\Purl{https://www.facebook.com/yevzhik/posts/3819537761414639}


\ifcmt
  pic https://scontent-mad1-1.xx.fbcdn.net/v/t1.6435-9/175886647_3819537654747983_2299349323011661072_n.jpg?_nc_cat=101&ccb=1-3&_nc_sid=8bfeb9&_nc_ohc=tvGRvz7P_7AAX9TPDwX&_nc_ht=scontent-mad1-1.xx&oh=5f5e217de1c50282db9376e892f208a3&oe=60A3AA5B

  pic https://scontent-mad1-1.xx.fbcdn.net/v/t1.6435-9/173535098_3819537721414643_4838824223113964409_n.jpg?_nc_cat=110&ccb=1-3&_nc_sid=8bfeb9&_nc_ohc=suQ-hoz3KEsAX9Np7l4&_nc_ht=scontent-mad1-1.xx&oh=eb90db887ef7e120db273da3b79c5e72&oe=60A49BE2
\fi

Не тщеславия ради пишу: наградами не интересуюсь - ради истины События духа.
Знаете, что больше всего радует в тебе Бога? Труд. Не результат, а просто труд.
Навынос. С критикой знатоков. С советами тех, кто мудрее. Само творчество. Само
созидание красоты. Которая мир таки спасет. Каким бы ни был результат труда,
его творение - Богоподобно.

Вот сегодня мне написали из крупного известного международного журнала, где
предложили и платформу, и бесплатные статьи с индексами Scopus and Web of
Science, и общие конференции, и, главное, - доброе человечье слово. Я за ночь
вместо 25 написала им 50 тысяч знаков, шалунья: меня никто не цензурировал и не
сокращал, совсем, ни в кейсах, ни в мировоззрении, хотя я и вступила в
академическую традицию добротной, конструктивной полемики с коллегами-профи,
традицию, которой когда-то и Киев славился...

Потому что в настоящем творчестве и в подлинном труде, что в науке, что в
поэзии, не хватка, деньги, размер и форма, а глубина, ум, старание и смысл
имеют значение. Правила работу я много: я не стесняюсь слушаться умных
советчиков. 

Стефания Данилова

 , о тебе отдельно: это не роза голубого цвета моей майки "В Питере - пить":
 это цвет нашей будущей книги "Бахтинский снег", которую чудо родит нам, как
 чадо, после защиты твоей диссертации с 94 процентами оригинального текста. 

Это - любовь. Её не сломать. Это - Philia Sophia, Святому Кириллу невестою из
лазури и злата явившаяся в летописях до всех их симулякров.
