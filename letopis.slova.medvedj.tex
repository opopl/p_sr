% vim: keymap=russian-jcukenwin
%%beginhead 
 
%%file slova.medvedj
%%parent slova
 
%%url 
 
%%author 
%%author_id 
%%author_url 
 
%%tags 
%%title 
 
%%endhead 
\chapter{Медведь}
\label{sec:slova.medvedj}

%%%cit
%%%cit_head
%%%cit_pic
%%%cit_text
Это относительно безвредно продолжалось, пока \enquote{гуляли на окраине леса}.
Но как только зашли вглубь - столкнулись с \emph{медведем}, у которого были
свои соображения на счет того, кто в лесу хозяин (кстати, пример с
\emph{медведем} - любимый у Путина).  В таком варианте, нужно было
\enquote{убить \emph{медведя}}, но как это сделать, если в ружье один патрон и
тот холостой?  Можно выстрелить и напугать зверя, а если он не испугается? В
таком случае, есть иной вариант - играть на гармошке и танцевать.  Говорят,
\emph{медведи} очень любопытные и увидев такое, будут стоять на задних лапах и
угрожающе помахивать лапами
%%%cit_comment
%%%cit_title
\citTitle{Байдену приходится играть на гармошке и танцевать перед русским медведем}, 
Алексей Кущ, strana.ua, 18.06.2021
%%%endcit

%%%cit
%%%cit_head
%%%cit_pic
%%%cit_text
Ось уже прийшли на місце. \emph{Медведяче} леговище — то був високий, тільки від
південного боку з трудом доступний горб, покритий грубезними буками й
смереками, завалений вивертами й ломами. Від півночі, заходу і сходу вхід і
вихід замикали високі скалисті стіни, немов величезною сокирою вирубані з тіла
велетня Зелеменя і відсунені від нього за кільканадцять сажнів; сподом попід ті
стіни вузькою щілиною шумів і пінився студений гірський потік. Таке положення
улегшувало нашим ловцям роботу; вони потребували тільки обсадити не надто
широкий плай від південного боку і тим плаєм поступати чимраз далі догори, а
звір, не маючи іншого виходу, мусив конечно попастися в їх руки і на їх ратища
%%%cit_comment
%%%cit_title
\citTitle{Захар Беркут}, Іван Франко
%%%endcit
