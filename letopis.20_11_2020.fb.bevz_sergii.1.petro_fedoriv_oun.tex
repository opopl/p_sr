% vim: keymap=russian-jcukenwin
%%beginhead 
 
%%file 20_11_2020.fb.bevz_sergii.1.petro_fedoriv_oun
%%parent 20_11_2020
 
%%url https://www.facebook.com/spollloh/posts/208939830682702
 
%%author Бевз, Сергій
%%author_id bevz_sergii
%%author_url 
 
%%tags 
%%title 110 років тому народився Петро Федорів («Дальнич»)
 
%%endhead 
 
\subsection{110 років тому народився Петро Федорів («Дальнич»)}
\label{sec:20_11_2020.fb.bevz_sergii.1.petro_fedoriv_oun}
\Purl{https://www.facebook.com/spollloh/posts/208939830682702}
\ifcmt
	begin_author
   author_id bevz_sergii
	end_author
\fi

\index[names.rus]{Федорів, Петро!Керівник Служби безпеки ОУН на Закерзонні}

Рівно 110 років тому, 20 листопада 1910, народився Петро Федорів («Дальнич») –
визначний діяч українського визвольного руху, керівник Служби безпеки ОУН на
Закерзонні. Він належить до когорти вкрай маловідомих загалу борців за волю
України. Втім, саме завдяки таким людям як Петро Федорів УПА по праву
вважається найпотужнішим партизанським рухом Європи. Адже на їхніх плечах
лежала відповідальність за організацію визвольної боротьби на місцях.

\ifcmt
pic https://scontent.fiev6-1.fna.fbcdn.net/v/t1.0-9/126808407_208938520682833_8632196785163794778_n.jpg?_nc_cat=109&ccb=2&_nc_sid=8bfeb9&_nc_ohc=mgeUL7yol1kAX86AOk0&_nc_ht=scontent.fiev6-1.fna&oh=af20a392af0f2fb742000d7136cda90f&oe=5FE55AF5
caption Петро Федорів («Дальнич»), Керівник Служби безпеки ОУН на Закерзонні
\fi

Петро Федорів прийшов у світ на Бережанщині у звичайній українській родині того
часу. З гімназійних років він долучається до націоналістичного руху. Навчався
на юридичному факультеті Львівського університету. Паралельно бере дієву участь
в роботі «Пласту» та «Просвіти». За революційну діяльність в ОУН тричі
заарештовується польською окупаційною владою.

Після вибуху Другої світової війни Петро Федорів займається розбудовою мережі
ОУН на Лемківщині. 1941 він бере участь в спробі відновлення української
державності. На Київщині Петра Федоріва заарештовують німці. Проте у Львові
йому вдалося втекти з ув’язнення, щоб продовжити боротьбу. Провід ОУН належно
оцінив здібності Петра Федоріва. З 1942 «Дальнич» працює на керівних посадах в
Службі безпеки ОУН.

Найбільш насичений період боротьби для Петра Федоріва розпочинається навесні
1945 року. Він очолює Службу безпеку ОУН на землях Закерзоння (Лемківщина,
Холмщина, Надсяння) в найскладніший період для місцевих українців. Після так
званого «визволення» польські шовіністи спільно з московськими окупантами
починають цілеспрямовану депортацію українського населення краю. Петро Федорів
як референт СБ доклав дієвих зусиль для захисту українців Закерзоння від терору
ворожих режимів.

«Дальнич» розбудував мережу СБ ОУН на Закерзонні, вдосконалив структуру
підпільної спецслужби, створив велику мережу пунктів зв’язку із центральним
проводом націоналістів за кордоном. Василь Кук характеризував Петра Федоріва як
одного з найефективніших співробітників Служби безпеки визвольного руху,
завдяки плідній праці якого вдалося викрити чимало зрадників. Саме СБ ОУН під
проводом «Дальнича» забезпечувала розвідувальний супровід закордонних рейдів
УПА. Не менш продуктивною була діяльність Федоріва.

1947 р. Петра Федоріва, приспавши снодійним газом, захопили польські окупанти.
Він мужньо тримався на допитах, відмовившись від співпраці з ворогом. Також
відкинув можливість просити помилування в окупантів. 1950 р. Петро Федорів був
розстріляний за свою незламність.

Героїчною боротьбою «Дальнич» назавжди вписав своє ім’я в історію України. Він
є прикладом гідного служіння Батьківщині до останнього подиху.

Вічна слава герою!
