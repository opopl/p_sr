% vim: keymap=russian-jcukenwin
%%beginhead 
 
%%file 14_04_2021.fb.zharkih_ekaterina.1.kosmonavtika_ussr_vyshyvanki
%%parent 14_04_2021
 
%%url https://www.facebook.com/permalink.php?story_fbid=129260019214668&id=112669904207013
 
%%author 
%%author_id 
%%author_url 
 
%%tags 
%%title 
 
%%endhead 

\subsection{День Космонавтики}
\Purl{https://www.facebook.com/permalink.php?story_fbid=129260019214668&id=112669904207013}


\ifcmt
  pic https://scontent-msp1-1.xx.fbcdn.net/v/t1.6435-9/173546595_129259835881353_4325606721189369290_n.jpg?_nc_cat=105&ccb=1-3&_nc_sid=8bfeb9&_nc_ohc=CPZP0o7sXUQAX91reKp&_nc_ht=scontent-msp1-1.xx&oh=177c326c4426c268169f753466a16b1a&oe=609DFC3F
\fi


Весь мир вчера отметил один из величайших праздников в истории человечества – День Космонавтики. 

Какую роль в своё время сыграла УССР в развитии космонавтики и запуске первого
искусственного спутника Земли и затем первого космонавта планеты Земля? По
сути, самую непосредственную, ведь Сергей Павлович Королёв, уроженец Житомира,
и был основателем практической космонавтики, а также одним из основных
создателей советской ракетно-космической техники, обеспечившей стратегический
паритет и сделавшей СССР передовой ракетно-космической державой. 

Прежде чем украинский и русский народы приступят к “дележу” этого светила науки
по национальному признаку, стоит подметить, что “поле его достижений - это
советская космическая программа, и это тоже очевидный факт. Не было тогда
другого уровня для этой работы, космос - это не народное творчество», по словам
Алексея Тимофеева (2021).

А чего же добилась независимая и постсоветская Украина, получив «то самое
проклятое наследие Союза» в виде Южного машиностроительного завода имени
А.Макарова, также известного как Южмаш?

По словам нынешних сотрудников Южмаша, мощность завода позволяла изготавливать
до ста штук ракет Р-36М, также известных как “Сатана”, изобретение которой
сделало в своё время бессмысленной американскую систему противоракетной
обороны. Срок использования боевой ракеты “Сатана” без существенных отступлений
- до 30 лет, что подчёркивает уникальное качество этого продукта, некогда
выпускаемого в Днепропетровске. И в последний раз её изготовили на Южмаше.
Может, это субъективная точка зрения одного из учёных, работающего на этом
некогда легендарном заводе, но живи мы дружно с Россией, мы бы и дальше
изготавливали самые дешёвые и в то же время, добротные ракеты, которые можно
было бы запускать в научно-исследовательских и коммерческих целях.

И это одна грань всего потенциала  Южмаша, который мы разменяли на
патриотическое нытье о прошлом.  Есть над чем задуматься за без малого тридцать
лет независимости и семи лет вышиватництва без скафандров…
