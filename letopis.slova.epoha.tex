% vim: keymap=russian-jcukenwin
%%beginhead 
 
%%file slova.epoha
%%parent slova
 
%%url 
 
%%author_id 
%%date 
 
%%tags 
%%title 
 
%%endhead 
\chapter{Эпоха}
\label{sec:slova.epoha}

%%%cit
%%%cit_head
%%%cit_pic
%%%cit_text
Она рассматривала ситуацию рубежа веков как точку бифуркации,
смысло-культурного взрыва.  Сейчас человечество формально находится примерно в
той же точке, что и Серебряный век.  Вот только взрыв есть, а культуры нет.
Неужели Юрий Михайлович где-то ошибся?  Когда в самом начале этого поста я
набирал на клавиатуре фамилию Гаспаров, Т9 автоматически исправил её на
Газпром.  Потом Малевича - на Мальвину...  Чем вам не управляемый цифровыми
техногиями взрыв с неуправляемыми последствиями?  И это мы ещё не совокупились
безвозвратно с Искусственным интеллектом.  Собственно говоря, в этом и
заключается вся суть нашей \emph{эпохи} - \emph{эпохи} взбесившегося словаря
%%%cit_comment
%%%cit_title
\citTitle{Наша эпоха - это время взбесившегося словаря / Лента соцсетей / Страна}, 
Владислав Михеев, strana.news, 11.11.2021
%%%endcit
