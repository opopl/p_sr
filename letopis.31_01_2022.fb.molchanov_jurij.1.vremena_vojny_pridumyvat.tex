% vim: keymap=russian-jcukenwin
%%beginhead 
 
%%file 31_01_2022.fb.molchanov_jurij.1.vremena_vojny_pridumyvat
%%parent 31_01_2022
 
%%url https://www.facebook.com/george.molchanov.9/posts/4642247935888138
 
%%author_id molchanov_jurij
%%date 
 
%%tags informacia,infvojna,internet,obschestvo,psihologia,vojna
%%title Да, мы живем во времена, когда войны можно просто придумывать
 
%%endhead 
 
\subsection{Да, мы живем во времена, когда войны можно просто придумывать}
\label{sec:31_01_2022.fb.molchanov_jurij.1.vremena_vojny_pridumyvat}
 
\Purl{https://www.facebook.com/george.molchanov.9/posts/4642247935888138}
\ifcmt
 author_begin
   author_id molchanov_jurij
 author_end
\fi

В тему, которую поднял игорь гужва о том, могут ли вроде столь уважаемые и
крупные западные СМИ гнать столь откровенные фейки «про вот-вот войну».

А чему тут удивляться? Раз человек постепенно погружается в виртуал, раз
инвестируются десятки миллиардов (ФБ, Гугл и прочие) в метавселенные, то почему
бы и войнушку виртуальную не затеять? Развязать, повоевать и победить. Можно и
награды победителям выписать.

Сколько сейчас рядовой гражданин планеты проводит времени в гаджетах? Четверть,
треть, половину своей жизни? Телефон – это уже неотъемлемая часть нашего
организма. С ним засыпаем, с ним просыпаемся. С ним народ за столом, и с ним,
простите, в клозете.

Если люди там знакомятся, переживают бурные романы, дружат, ссорятся, рождаются
и умирают, то почему бы им не повоевать разок-другой?

И сложно в этом винить журналистов. Они точно так же живут в этом виртуальном
мире. И выносят его на страницы своих материалов. Они ведь искренне верят в то,
что пишут. Сначала создают образы, а потом в них свято верят. Не со зла,
нативно.

«Ну как же нет угрозы войны, когда она есть? - Есть? А кто об этом сказал? –
Как кто? Вон все об этом говорят! Джен Псаки опять же! – А факты? А
противоположные заявления? – Факты? Да вон их сколько! Вот и карты есть, и
снимки со спутника! – Так там же нет ничего конкретного! – Как нет, их издание
X показало, а издание Y подтвердило!!!» И т.д. и т.п.

Так создаются революции, так смещаются президенты и Игорь прав в том, что об
этом запросто можно спросить у Трампа.

Да, мы живем во времена, когда войны можно просто придумывать.

Правда есть в этом и одна ужасная деталь. Такие виртуальные маневры могут
приводить к последствиям совершенно реальным. И ладно бы, если речь шла только
об экономическом или имиджевом ущербе. Но такие вещи могут приводить к
совершенно реальным конфликтам и жертвам.

И легко ли будет засыпать участникам шоу, когда они будут знать, что благодаря
в том числе и их кончикам пальцев на клавиатуре, лежит где-то в могиле невинный
ребенок?

Впрочем, думаю да, легко  @igg{fbicon.frown} 
