% vim: keymap=russian-jcukenwin
%%beginhead 
 
%%file 19_12_2021.fb.baumejster_andrej.kiev.filosof.1.chudesnyj_doktor_kuprin_rozhdestvo.cmt
%%parent 19_12_2021.fb.baumejster_andrej.kiev.filosof.1.chudesnyj_doktor_kuprin_rozhdestvo
 
%%url 
 
%%author_id 
%%date 
 
%%tags 
%%title 
 
%%endhead 
\zzSecCmt

\begin{itemize} % {
\iusr{Андрей Баумейстер}
Электронная версия дневника Николая Ивановича Пирогова: 

\url{https://imwerden.de/pdf/pirogov_dnevnik_vracha_2008.pdf}

\begin{itemize} % {
\iusr{Валерий Зиновьев}
\textbf{Андрей Баумейстер} Как раз сейчас читаю эту книгу. Очень познавательная, даже выписываю цитаты из нее.


\iusr{Андрей Баумейстер}
\textbf{Валерий Зиновьев} прекрасно
\end{itemize} % }

\iusr{Мартин Тихолаз}

О, спасибо, повод улыбнуться -и мне нравился этот рассказ Куприна)). Вообще-то,
ведь в детстве \enquote{подгружают} классикой не зря - интересные метаморфозы
происходят с возрастом, когда снова возвращаешься к текстам детства. И очень
приятное дополнение - воспоминания)). А, иногда, и важное первое впечатление от
автора становится решающим для дальнейшего чтения

\begin{itemize} % {
\iusr{Андрей Баумейстер}
\textbf{Мартин Тихолаз} да, конечно. Но часто понимание приходит позже
\end{itemize} % }

\iusr{Олена Предко}

Дякую Андрію Олеговичу, що згадали про оповідання \enquote{Чудесний лікар}. Ще в
дитинстві мене глибоко вразило це оповідання Олександра Купріна, зачарувало
чудо в його лікарському оприявненні. Дивовижно, але через багато років я в ролі
опонента осмислювала, аналізувала феномен \enquote{чудо} в прекрасній докторській
дисертації Гудими Ігора Петровича \enquote{Феномен чуда у християнському світорозумінні
та практиці: філософсько-релігієзнавчий аналіз}. Хлопчики, які розглядали
дивовижну вітрину, в деякій мірі забувши про листа - листа, від прочитання
якого залежала життя сім'ї. Подальша кумуляція безвиході, розпачу. О.Купрін, як
прекрасний майстер психологічних інтенцій, зумів показати в деякій мірі
демістифікацію чуда, його персоналізацію й врешті -решт його божественну дію.
Направду, прекрасне \enquote{розкодування} та \enquote{розчаклування} чуда О.Купріним!

\iusr{Денис Васильев}
Дякую.

\end{itemize} % }
