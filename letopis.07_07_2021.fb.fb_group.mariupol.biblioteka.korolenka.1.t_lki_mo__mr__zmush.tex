%%beginhead 
 
%%file 07_07_2021.fb.fb_group.mariupol.biblioteka.korolenka.1.t_lki_mo__mr__zmush
%%parent 07_07_2021
 
%%url https://www.facebook.com/groups/1476321979131170/posts/4077478045682204
 
%%author_id fb_group.mariupol.biblioteka.korolenka,kibkalo_natalia.mariupol.biblioteka.korolenko
%%date 07_07_2021
 
%%tags mariupol,mariupol.pre_war,2021,kultura,issku,vystavka,hudozhnik,biblioteka,kartina
%%title Тільки мої мрії змушують мене жити! - Виставка живопису - Анастасія Стеценко
 
%%endhead 

\subsection{Тільки мої мрії змушують мене жити! - Виставка живопису - Анастасія Стеценко}
\label{sec:07_07_2021.fb.fb_group.mariupol.biblioteka.korolenka.1.t_lki_mo__mr__zmush}
 
\Purl{https://www.facebook.com/groups/1476321979131170/posts/4077478045682204}
\ifcmt
 author_begin
   author_id fb_group.mariupol.biblioteka.korolenka,kibkalo_natalia.mariupol.biblioteka.korolenko
 author_end
\fi

«Only my dreams кeep me alive!» (Тільки мої мрії змушують мене жити!) - так
16-річна маріупольчанка Анастасія Стеценко вирішила назвати свою першу
персональну виставку живопису. В експозиції Мобільної галереї «Світ захоплень»
Центральна міська публічна бібліотека ім. В.Г. Короленка м. Маріуполь Настя
представила понад 40 творчих робіт: великих і маленьких за розміром, написаних
в реалістичній манері маслом, пастеллю, гуашшю, аквареллю. Дівчина починала
осягати ази образотворчого мистецтва у маріупольської художниці Галини
Гарькіної (члена НСХУ), у 2020-му році вступила до художньої школи ім. А.І.
Куїнджі. У Анастасії Стеценко – міцна та надійна група підтримки з батьків та
друзів, а це – дуже важливо для починаючого художника! На вернісаж Анастасіі
прийшли також хлопці з музикального творчого об'єднання «НЕДОШЛО». Виконання
авторських треків Георгієм Комендантовим, Олегом та Миколою Висоцькими
«Картина», «Липовая», «Корабли», «МRPL FEST» та ін. зробило виставку живопису
ще більше свіжою, барвистою, щирою. Картини Насті - це різноплановість
почуттів, ідей, інтересів, образів. Це милі натюрморти і привабливі пейзажі,
цікаві портрети і ціла серія анімалістичних робіт: леви і жираф, фламінго і
сови, жаби і вовк, представники водного світу. В них відчувається легкість,
плавність, а чудові образи птахів та звірів пробуджують доброту і бажання
захищати «братів менших». Вернісаж в Мобільної галереї пройшов у теплій
«молодіжній» атмосфері.
