% vim: keymap=russian-jcukenwin
%%beginhead 
 
%%file 30_03_2022.stz.news.ua.pravda.2.ohtyrka.3.likarnja
%%parent 30_03_2022.stz.news.ua.pravda.2.ohtyrka
 
%%url 
 
%%author_id 
%%date 
 
%%tags 
%%title 
 
%%endhead 

\subsubsection{На другий день я вирішив піти в лікарню}

Війна прийшла в Охтирку 24-го числа, від нас до границі – трохи більше 30-ти
кілометрів. Пішло все – танки, \enquote{Гради}.

Спочатку, між 12-ою і 14-ою годинами зайшла техніка, почалися перші бої на
вході в місто. Вони ж то були впевнені, що пройдуть! Йшли на парад через нашу
Україну, на гусеницях, на броні. А ми їх тут розбили повністю, вщент! 

Я не буду говорити, хто саме їх зупинив – військові, правоохоронці чи
тероборона. Але в нас усе було, і ми до всього були готові.

25-го лютого почались \enquote{Гради}... Під обстрілами загинуло дуже, дуже
багато мирних мешканців, діти... Загинула 7-річна дівчинка. Багато людей було
поранено.

26-го лютого на нас вперше в країні скинули вакуумну бомбу. Усього їх було три.
Знов багато людей померли, деякі згоріли живцем – від них нічого не залишилося,
навіть кістки погоріли. 

\ii{30_03_2022.stz.news.ua.pravda.2.ohtyrka.3.likarnja.pic.1}

У вакуумної бомби ж специфічна дія. Коли вона вибухає, скло з вікон вилітає не
всередину в будівлю, а назовні. Тому в нас усе місто у склі. Дуже багато
понищеної інфраструктури. 

Потім кожного дня все ставало гірше і гірше.

День на 7-8-ий
\href{https://www.pravda.com.ua/news/2022/03/3/7327841/}{розбомбили ТЕЦ}. Це
був наш найчорніший день, у місті зупинилося все тепло.

\ifcmt
  ig https://img.pravda.com/images/doc/7/b/7b334f6----.jpg
	@caption Розбиті будівлі Охтирської ТЕЦ
  @wrap center
  @width 0.6
\fi

8-го березня розбомбили центр міста, міську раду.

На другий день війни я вирішив піти в лікарню. Насправді, будучи міським
головою, я і так там працював у лікарні – робив усі складні операції. Це було у
вихідні або після роботи – офіційно мені ж не можна суміщати, та й часу не
було.

Але коли на другий день у місто прилетіли \enquote{Гради}, і було багато
поранених, я пішов допомагати.

Ми прооперували трьох людей. У двох чоловіків були мінно-вибухові поранення
гомілки та стегна – їх обох я знав, у жінки був перелом стегна – її я не знав.

Один з чоловіків помер на другий чи третій день після операції, він мав
політравму – пошкоджені нирка, живіт, плюс перелом гомілки.

Щодо жінки, я впевнений, що вона була російською наводчицею, яка випадково
попала під російські \enquote{Гради}. Просто не думала, що вони накриють. У перший день
я чомусь думав, що вона військова чи дружина військового. А на другий день
зрозумів, що це військова, але не наша.

Після операції її евакуювали на Полтаву – я доповів, куди треба, в органи. Далі
це їхня відповідальність. Кожен повинен займатися своїм.

Власне, тому я сьогодні вже й не оперую, бо моя основна задача – це все ж
життєзабезпечення міста. Але продовжую консультувати. Кожного дня ходжу в
лікарню. Лікарі поки справляються самі, завідувач відділення – мій учень.
