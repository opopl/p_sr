% vim: keymap=russian-jcukenwin
%%beginhead 
 
%%file kiev.sofia
%%parent kiev
 
%%url 
 
%%author_id 
%%date 
 
%%tags 
%%title 
 
%%endhead 

%https://www.facebook.com/groups/story.kiev.ua/posts/2104494903080633/

%https://www.m-hrushevsky.name/uk/Fiction/Prose/USvjatojiSofiji.html

СОНЦЕ СОФІЇ 
Купол Софії, брама, дзвіниця…
В грудень злетілись перші синиці, 
Небом і снігом ніжноблакиті
В час урочистий у оксамиті…
Післяосінній, передріздвяний, 
Днями короткий, світлом багряний,
Дивом і святом рідним, щасливим,
Запахом хвойним гірко-щемливим…
Так не було ще ніколи. Ніколи.
Навіть тоді, як стояли довкола
Зими лихі, і страшні, і ворожі…
Війни завжди божевіллями схожі.
Запахи гару у стінах, в клітинах,
Смерті в руїнах як в павутинах…
Місто тримає вічна опора,
Прагнення волі і непокора.
Древня Софія Київська - святість
Світло і мудрість, предків крилатість,
Тисячоліття – пам'ять і мрії,
Серце держави – сонце Софії.
Шаблівська В.  2.12.2022
