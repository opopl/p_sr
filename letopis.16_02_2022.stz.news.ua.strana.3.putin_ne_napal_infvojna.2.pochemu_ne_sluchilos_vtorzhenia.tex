% vim: keymap=russian-jcukenwin
%%beginhead 
 
%%file 16_02_2022.stz.news.ua.strana.3.putin_ne_napal_infvojna.2.pochemu_ne_sluchilos_vtorzhenia
%%parent 16_02_2022.stz.news.ua.strana.3.putin_ne_napal_infvojna
 
%%url 
 
%%author_id 
%%date 
 
%%tags 
%%title 
 
%%endhead 

\subsubsection{Почему не случилось \enquote{вторжения}?}

Здесь главенствует три основных точки зрения. 

Первая - западная. Там говорят уже, что даты \enquote{наступления} могут
варьироваться, но войска РФ с границ пока не уходят, а значит, нападение может
случиться в любой момент.

То есть, \enquote{загадочный} Путин специально путает Запад, перенося даты
\enquote{вторжения} (так это подает британский таблоид The Sun, комментируя
провал своего же прогноза). 

Это значит, что кампания по \enquote{нападению Путина} будет продолжена, но
теперь уже, что называется, с открытой датой. И, собственно, заявления запарных
официальных лиц ровно об этом и говорит. Они, очевидно, и дальше намерены
нагнетать панику вокруг Украины, несмотря на провал своих прежних прогнозов.

Вторая - российская. В Москве уже несколько месяцев говорят, что никакой
операции против Украины не готовят. А вмешаться могут лишь если ВСУ пойдут в
наступление на Донбассе. 

Вчера, кстати, Россия начала отводить войска от своих юго-западных границ. Это
была явная демонстрация, что никто нападать не собирается. После чего в
Британии начался шквал публикаций о \enquote{вторжении в три часа ночи}. 

То есть цель этих публикаций была блокировать любые сигналы о разрядке и
вернуть градус напряжения на прежний уровень или выше. 

Третья - украинская. Она частично совпадает с российской: в Киеве давно
говорят, что признаков создания атакующих группировок на территории РФ они не
видят.

Но поддерживают и антироссийскую рамку, используя тему \enquote{вторжения} как
способ заявить Западу о себе, выбить какие-то преференции, оружие и деньги. 

Правда, пока Украина лишь теряет деньги на поддержание гривны в связи с
паникой. Страны Запада уже заявили о выделении новых кредитов, но эти суммы не
перекроют убытки от панического бегства инвесторов. 

Если украинская и российская точки зрения обе выглядят как
\enquote{оборонительные}, то западная позиция - чисто наступательная. Именно за
океаном и в Британии последовательно и по какому-то одному им понятному графику
повышают градус напряженности. 

И крайне интересно, зачем это делают. 
