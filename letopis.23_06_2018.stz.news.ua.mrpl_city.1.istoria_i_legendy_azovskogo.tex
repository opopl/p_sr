% vim: keymap=russian-jcukenwin
%%beginhead 
 
%%file 23_06_2018.stz.news.ua.mrpl_city.1.istoria_i_legendy_azovskogo
%%parent 23_06_2018
 
%%url https://mrpl.city/blogs/view/istoriya-i-legendi-azovskogo-morya
 
%%author_id demidko_olga.mariupol,news.ua.mrpl_city
%%date 
 
%%tags 
%%title Історія і легенди Азовського моря
 
%%endhead 
 
\subsection{Історія і легенди Азовського моря}
\label{sec:23_06_2018.stz.news.ua.mrpl_city.1.istoria_i_legendy_azovskogo}
 
\Purl{https://mrpl.city/blogs/view/istoriya-i-legendi-azovskogo-morya}
\ifcmt
 author_begin
   author_id demidko_olga.mariupol,news.ua.mrpl_city
 author_end
\fi

\begin{raggedright}
\bfseries\em\color{blue}
Море і небо – два символи нескінченності.

Джузеппе Мадзіні
\end{raggedright}

\href{https://kirillovka.ks.ua/azov-sea/}{Азовське море} має свої, тільки йому притаманні риси та особливості, легенди та
власну історію. Вражає, що воно є найменшим за розмірами, але в історії посідає
доволі значне місце. І дійсно, воно менше Аральського в півтора рази,
Каспійського – в десять разів, а Чорного – в одинадцять разів. Максимальна
глибина Азовського моря – всього тринадцять з половиною метрів. За кількістю
рослинних та тваринних організмів Азовському морю немає рівних у світі. За
рибопродуктивністю, тобто кількістю риби на одиницю площі, Азовське море в 6,5
раза перевершує Каспійське, в 40 разів – Чорне і в 160 разів - Середземне море.
За віддаленістю від океану Азовське море є найконтинентальнішим морем планети.

\ii{23_06_2018.stz.news.ua.mrpl_city.1.istoria_i_legendy_azovskogo.pic.1}

Нинішні риси Азовське і сусідні з ним моря придбали тільки під час третинного
періоду, тобто близько мільйона років тому, коли на землі тільки з'явилася
людина. До цього воно було частиною величезного стародавнього праокеану,
названого геологами Тетіс.

Оскільки Азовське море знаходилося на перетині історичних шляхів народів, воно
змінило в минулому безліч назв. Стародавні греки називали його \textbf{\enquote{Меотида}}, що
означає – \enquote{годувальниця}, римляни за низький рівень води прозвали його \textbf{\enquote{Палюс
Меотіс}} (\enquote{Меотійське болото}), скіфи – \textbf{\enquote{Каргулак}} (\enquote{багате рибою}), меоти –
\textbf{\enquote{Тіміріндою}} (\enquote{матір'ю моря}), генуезці та венеціанці називали його \textbf{\enquote{Маре
Фане}}, араби – \textbf{\enquote{Бахр-ель-Азов}}, а слов'яни – \textbf{\enquote{Сурозьким}} або \textbf{\enquote{Синім}}...

Не виключається припущення, що свою сучасну назву воно отримало приблизно в
середині XIII століття – від імені давнього торгового центру Азак, який виріс
на місці античного Танаїсу, зруйнованого Золотою Ордою. Є версія, що колись
арабський народ ази, аси (азовці) населяв узбережжя сучасного Азовського моря і
заснував там місто з відповідною назвою – Азов. Можливо, у стародавнього
населення Азова та Азовського моря дійсно текла \enquote{гаряча} арабська кров. Адже з
багатьох письмових джерел відомо, що стародавні азовці любили війни,
відрізнялися жорстокістю і помстою. Загалом щодо назви Азовського моря існує
багато припущень, але остаточного висновку немає. Можна бути впевненим в тому,
що коріння його або в гідронімії, або в етнонімії.

\ii{23_06_2018.stz.news.ua.mrpl_city.1.istoria_i_legendy_azovskogo.pic.2}

\emph{Далеко не всім відомо, що про Азовське море складено безліч легенд.}

Одна з легенд присвячена двом сестрам, які жили зі старим батьком – рибалкою
біля нашого моря. Їхня мати давно померла. Старшу сестру звали \textbf{Азою}, а другу,
меншу, – \textbf{Золотокосою Піщанкою}. Сестри були такі красиві, що хто їх бачив, той
про сон забував: все про них думав. А дівчата шукали своїх суджених
перебірливо, ніхто із місцевих парубків не припадав їм до серця. Аза щодня
сиділа на березі моря, на високій кручі, та все виглядала когось. Мабуть, свого
судженого, який поплив у далекі світи і там, як переказували люди, загинув від
ворожої шаблі. І ось одного разу, коли дівчина сиділа на березі моря, трапився
сильний вітер-буран. На морі піднялися височенні хвилі. Бігли вони до берега,
били в кручі й страшно стогнали. Аж раптом відколовся від кручі великий шмат
землі і разом із Азою впав у розбурхані хвилі. Побачила це Золотокоса Піщанка –
та й кинулась з гори у море, щоб врятувати старшу сестру. Так вони обидві й
загинули. 

\textbf{Читайте також:} 

\href{https://mrpl.city/news/view/na-poberezhe-mariupolya-sozdadut-mesto-dlya-svidanij-s-morem}{%
На побережье Мариуполя создадут место для свиданий с морем, Яна Іванова, mrpl.city, 18.06.2018}

Ранком наступного дня, коли море заспокоїлося, повернувся з гостей старий
рибалка, вийшов на берег і побачив, що нема його дочок на кручі, а на тому
місці, де любила сидіти Аза, - свіжий обвал. Глянув батько униз – а там, під
самою кручею, такий золотий пісок іскриться на сонці, що аж очі засліплює! А
море – тихе-тихе і таке лагідне, як його діти... Тоді зрозумів усе нещасний та
й гірко заплакав...

З того часу і море наше почали називати Азовським, бо ж утопилася в ньому
красуня Аза. А довгих піщаних кіс у цьому морі тому так багато, що разом з Азою
втопилася її молодша сестра – Золотокоса Піщанка.

Незважаючи на трагічний кінець легенди, вона дуже романтична, заворожує уяву
людини та викликає захоплення загадковістю Азовського моря.

\ii{23_06_2018.stz.news.ua.mrpl_city.1.istoria_i_legendy_azovskogo.pic.3}

Цікаво, що на думку фахівців пісок на берегах Азовського моря, має цілющі
властивості. Медики радять не менше 1,5-2 годин проводити на ньому.

Кажуть, що всі дороги в Маріуполі ведуть до моря. І це не дивно, адже
розташоване місто на самому березі унікального Азовського моря, яке зберігає
свої таємниці та надихає красою та самобутністю. Назву міста етимологічно
пов'язують теж з морем. Мовляв, \enquote{марі} – це море, а \enquote{поль} – \enquote{поліс}, тобто
\enquote{місто}. І виходить – місто біля моря.

\emph{Джерело: \url{http://mrpl.city/}}

%\ii{insert.author.demidko_olga}
