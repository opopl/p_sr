% vim: keymap=russian-jcukenwin
%%beginhead 
 
%%file 19_11_2021.fb.zadorozhnaja_natalia.1.skazka_sofia_1.cmt
%%parent 19_11_2021.fb.zadorozhnaja_natalia.1.skazka_sofia_1
 
%%url 
 
%%author_id 
%%date 
 
%%tags 
%%title 
 
%%endhead 
\subsubsection{Коментарі}

\begin{itemize} % {
\iusr{леся благая}
Чудо - чудное , диво-дивное ..... @igg{fbicon.face.smiling.hearts}  @igg{fbicon.hand.ok} 

\iusr{Жанна Самарская}
Держать эту радость в сердце как можно дольше!

\begin{itemize} % {
\iusr{Natalia Zadorozhnaya}
\textbf{Жанна Самарская}
Жанна, да - это смальта радости жизни, тепло моё осеннее. Главные фрагменты витража...
\end{itemize} % }

\iusr{Константин Павляк}

Киев при Рюриковичах был шведской колонией, правили здесь шведские наместники, именуемые себя \enquote{князьями},
Ингигерда - прынцесса шведская....

\begin{itemize} % {
\iusr{Natalia Zadorozhnaya}
\textbf{Константин Павляк}, и? Не реставрировать, не беречь?

Вся история из компромиссов и сложных моментов. И во всякое время её под себя в
удобном им аспекте пытаются «подправить» новоделы-современники.

Помимо того, что Ингигерда была шведской принцессой, она стала женой Ярослава
Мудрого и матерью Анны, ставшей королевой Франции.

Хотя, мне в принципе не совсем понятно предназначение комментария. Ведь
главное, чему этот пост посвящен - бескорыстный и продуктивный труд
современников (молодежи!) по реставрации уникальных памятников истории и
культуры. Готовность одних это делать, тратя силы и талант, других вкладывать
средства. А это само по себе явление весьма достойное на мой взгляд. И ещё -
умению выказать им благодарность за всё торжественной презентацией проекта.


\iusr{Константин Павляк}
\textbf{Natalia Zadorozhnaya} , и что?

Вы с таким благоговением пишете о Ярославе \enquote{Мудром}, как будто бы он является
Вашим племянником  @igg{fbicon.smile} 

И вы с таким таким трепетом пишете о его дочери Анне, которая стала королевой
Франции, как будто бы в этом есть какая-то Ваша личная заслуга  @igg{fbicon.smile} 

\ifcmt
  ig https://scontent-frx5-1.xx.fbcdn.net/v/t39.30808-6/258061053_2249278508704217_8176978676513319493_n.jpg?_nc_cat=105&ccb=1-5&_nc_sid=dbeb18&_nc_ohc=qZlyBfnqpNMAX823Toa&_nc_ht=scontent-frx5-1.xx&oh=cef6c8312b31d87f5af5bc461b3010b9&oe=61BB9B85
  @width 0.4
\fi

\iusr{Natalia Zadorozhnaya}
\textbf{Константин Павляк}, 

я пишу о людях, которые своими силами, за свои средства, без финансовой
поддержки государства, реставрируют памятники культуры и ведут скрупулёзную
работу по восстановлению исторических данных. Об умении радоваться результатам
творческого труда.

«О личной заслуге» и расставленных Вами акцентах (в моём понимании и крайне
бестактных и выдуманных Вами в рамках Ваших взглядов и субъективных ощущений и
оценочных суждений), о «трепете» и прочей ахинее - увольте. Это даже не смешно.
Выискивайте и дальше трепет и прочая-прочая.

Но я буду Вам весьма признательна, если Вы просто не будете делать это здесь, в
моей ленте, пытаться излагать Ваши фантазии в подобной форме.

Скучайте не здесь, ок?

А вот это вот всё с маркерами, цитированием и проч. - зачем? Цель?

Было о празднике, о выполненной интересной работе, исследованиях, о духовной
музыке, о памяти - и тут, упс! Свежая струя из доброжелательности.

Стоит ли?!

\iusr{Gala Kony}
\textbf{Константин Павляк} 

это право автора, на своей странице писать, ЧТО и КАК... А ВАМ мне так и хочется
сказать в рифму слово из 5 букв, первая из которых 5 в русском
алфавите... извините (снова руфмую...) хорошего воскресенья, здоровья и
терпимости( или заимствованного \enquote{толерантности})

\end{itemize} % }

\end{itemize} % }
