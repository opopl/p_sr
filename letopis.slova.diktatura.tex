% vim: keymap=russian-jcukenwin
%%beginhead 
 
%%file slova.diktatura
%%parent slova
 
%%url 
 
%%author 
%%author_id 
%%author_url 
 
%%tags 
%%title 
 
%%endhead 
\chapter{Диктатура}
\label{sec:slova.diktatura}

%%%cit
%%%cit_head
%%%cit_pic
%%%cit_text
«\emph{Диктатура} инклюзивности и толлерантности» открыто объявленная над главной
фабрикой нарративов и смыслов западного общества Голливудом показывает и
доказывает, что такой путь является и приемлемым, и необходимым. Причитания над
свободой выражения художников, свободой слова и мысли оказались обыкновенной
лицемерной пропагандой. Западное общество на защиту этих ценностей не встало,
еще раз продемонстрировав черты \emph{тоталитарной диктатуры}, продвигающей свое
влияние в области этических ценностей вполне силовыми методами
%%%cit_comment
%%%cit_title
\citTitle{Революция Духа – единственный путь спасения России}, 
Юрий Барбашов, voskhodinfo.su, 30.06.2021
%%%endcit

%%%cit
%%%cit_head
%%%cit_pic
%%%cit_text
Чи зможе радикальний націонал-патріот, спонукуваний бажанням стерти з пам’яті
близько 70 років СРСР, повернувшись у часи \enquote{до русифікації}, – близько 70 років
тих поколінь, що виросли в СРСР, здобували освіту, будували пліч-о-пліч свою
ідеологію – суміш навіяних ідей \emph{диктатури}, страху і мрії про краще майбутнє, –
приголомшених розвалом їхньої країни, але живих, як мої батьки; чи зможе
радикальний націонал-патріот прийняти, що Україна вже не \enquote{садок вишневий}, а
покоління ці не можна винищити, адже ці роки і покоління вже частина нашої
історії, нашого сьогодення і нашої ідентичності?
%%%cit_comment
%%%cit_title
\citTitle{Українці не розуміють одне одного не через мову, а через небажання слухати, чути і сприймати}, 
Юлія Мендель, www.pravda.com.ua, 07.07.2021
%%%endcit

%%%cit
%%%cit_head
%%%cit_pic
%%%cit_text
В подписанной Путиным Стратегии национальной безопасности указана главная
угроза его режиму - культурная вестернизация. Как всегда, украдено у Совка.
Помните: \enquote{Сегодня ты играешь джаз, а завтра родину продашь}. Это как раз об
угрозе культурной вестернизации для \enquote{национальной безопасности} СССР.
В стратегии много обычной смешной чепухи. Разведенный \emph{диктатор-отравитель},
открещивающийся от своих законных и внебрачных детей, дворцов, офшоров и
наемных убийц \enquote{ихтамнетов}, собирается защищать \enquote{традиционные ценности}, к
которым относит \enquote{крепкую семья, приоритет духовного над материальным, гуманизм,
милосердие и справедливость}
%%%cit_comment
%%%cit_title
\citTitle{Как разведенный диктатор-отравитель собирается защищать \enquote{традиционные ценности}}, 
Игорь Эйдман, opinions.glavred.info, 06.07.2021
%%%endcit

%%%cit
%%%cit_head
%%%cit_pic
%%%cit_text
\enquote{Батька} Білоруського федерального округу РФ Олександр Лукашенко
доручив прикордонним військам повністю перекрити український кордон.
\enquote{Граніца на замкє}, Карацупа з Мухтаром знову на бойовій вахті.
Формальна мотивація – \enquote{величезний потік зброї} з України до неіснуючих
білоруських партизан. Що це – параноя переляканого диктатора? На жаль, набагато
гірше, - пише Євген Дикий для Оbozrevatel.  Іржава \enquote{залізна завіса}
опускається на Поліські болота, і виступає в ролі театральної завіси.
\enquote{Фініта ля комедіа}, спектакль під назвою \enquote{білоруська
демократична революція} остаточно завершено. Починається нова вистава, в жанрі
трагедії, під умовною назвою \enquote{корейсько-кадирівська Білорусь}
%%%cit_comment
%%%cit_title
\citTitle{Лукашенко перекрив білорусам останню \enquote{дорогу життя}}, 
Євген Дикий, gazeta.ua, 05.07.2021
%%%endcit

%%%cit
%%%cit_head
%%%cit_pic
%%%cit_text
Зеленский не тянет на роль \emph{диктатора}. Также, как и Бутусов на роль
мученика.  Ситуация не сравнима. Но вспомнилось, как Ахматова говорила о
процессе над Бродским: какую биографию делают нашему рыжему!  Современная
Украина все больше напоминает позднесоветскую рыхлую \emph{недодиктатуру}.
Ощущение такое, что дальше нас ждёт новое "лебединое озеро" и новые
девяностые...  Реверсивное развитие превращает украинскую историю в "день
сурка".  Это как будущее без будущего. Вы заметили, что даже разговоры об
Украине как проекте в экспертной среде практически прекратились?  Ефремовская
утопия в 70-х, можно сказать, стала последним советским "образом будущего".
По-моему, даже последней утопией в мире.  Кризис позднего СССР был кризисом
проектности. Потому что слиться в экстазе с затоваренной полкой американского
супермаркета - это не проект
%%%cit_comment
%%%cit_title
\citTitle{Зеленский не тянет на роль диктатора. Как и Бутусов на роль мученика / Лента соцсетей / Страна}, 
Владислав Михеев, strana.news, 29.11.2021
%%%endcit
