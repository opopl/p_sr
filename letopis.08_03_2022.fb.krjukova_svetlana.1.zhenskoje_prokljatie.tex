% vim: keymap=russian-jcukenwin
%%beginhead 
 
%%file 08_03_2022.fb.krjukova_svetlana.1.zhenskoje_prokljatie
%%parent 08_03_2022
 
%%url https://www.facebook.com/kryukova/posts/10159987561723064
 
%%author_id krjukova_svetlana
%%date 
 
%%tags 
%%title Женское проклятье, особенно коллективное - мощная сила
 
%%endhead 
 
\subsection{Женское проклятье, особенно коллективное - мощная сила}
\label{sec:08_03_2022.fb.krjukova_svetlana.1.zhenskoje_prokljatie}
 
\Purl{https://www.facebook.com/kryukova/posts/10159987561723064}
\ifcmt
 author_begin
   author_id krjukova_svetlana
 author_end
\fi

\obeycr
Женское проклятье, особенно коллективное - мощная сила. Сегодня киевские женщины просили вместо цветов и конфет - жестокого наказания для убийц, которые принесли в их дома страх, смерть и ужас. 
Семью расстреляли в машине при попытке эвакуации ...
Ребёнок умер от обезвоживания под завалами дома... 
... Утром Киев засыпал снег. Мелкий, дробными крошками, на горизонте показалось солнце. Перелётные птицы укрылись под козырьками крыш. Теперь шум города это их стаи - птицы шепчутся и переговариваются о чем-то своём. Ругаются и поют на птичьем. Их нежные трели на фоне жестоких новостей - холодящий душу контраст. 
Интересно, что птицы знают о ракетах? 
Где прячутся во время сирен? 
Слышат ли они эти звуки? 
Известно ли птицам, о чем вопит воздушная тревога?
Реклама аренды офисов на столичных фасадах выглядит как насмешка. 
На билбордах почти не осталось наружки из той, другой, вчерашней жизни. 
Появилась другая, военная. Белым по чёрному:
Русский корабль иди н@хуй!
Русский солдат, не становись убийцей!
Как ты сможешь смотреть в глаза своим детям! 
И другой призыв от военных, вселяющий надежду: Держитесь, ещё немного и мы победим. 
Столичные недострои смотрятся зловеще и обнадеживающе одновременно. Тут планировался чей-то быт, чьи-то семьи мечтали размножаться и жить счастливой жизнью в столице. 
Недострои. Разгромленные дома. Опять эти контрасты. 
И вот снова сирена. Мужчины в помятой тачке притормаживают у проходящих мимо женщин. Те шарахаются и ... замирают. Парень дарит розы, просовывая их прямо из окна. Они раздобыли где-то охапку цветов и теперь разъезжают по городу и дарят прохожим женщинам. За то, что рядом, за то, что верят, за верность, преданность и терпение. 
В подпольном кафе, открывшимся по случаю праздника только и разговоров, что о войне. 
13-ом дне войны. 
Будут ли бомбить Киев? 
Когда переговоры?
Почему не закроют небо?
Одна из мыслей, попавших в уши заела как мантра. Пара киевлян:
- Сегодня на наших глазах разворачивается не Третья мировая, которую так боятся страны НАТО, упорно не закрывающие небо. 
- А как?
- ПЕРВАЯ ЦИВИЛИЗАЦИОННАЯ. 
Первая и Вторая мировые - были за территории, за ресурсы, за порабощение людей.
Первая цивилизационная - это битва между Добром и Злом, Светом и Тьмой, Будущим
и глубоким прошлым, прогрессом и дремучестью... И поэтому весь цивилизованный
мир сегодня с нами - против той орды, которая аж подпрыгивает на своей \enquote{красной
ядерной кнопке} и \enquote{скрепях}, потому что ничего больше у нее нет...
Цветочный киоск под домом закрылся строго за час до наступления комендантского часа - все розы, котики, пионы, тюльпаны - разобрали до одной. Остались лепестки на полу, да пара нелепых розовых медведей с меня ростом.
В этом городе, окружённым со всех сторон злом и нечистью, остаётся так много любви и веры. 
С праздником, дорогие женщины!
Мудрости вам и терпения.
8.03.2022
\restorecr

\ii{08_03_2022.fb.krjukova_svetlana.1.zhenskoje_prokljatie.cmt}
