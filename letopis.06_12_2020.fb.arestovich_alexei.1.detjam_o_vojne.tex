% vim: keymap=russian-jcukenwin
%%beginhead 
 
%%file 06_12_2020.fb.arestovich_alexei.1.detjam_o_vojne
%%parent 06_12_2020
 
%%url https://www.facebook.com/alexey.arestovich/posts/3876955605701824
 
%%author Арестович, Алексей
%%author_id arestovich_alexei
%%author_url 
 
%%tags ukraina,vojna,deti
%%title Как я рассказываю своим детям о войне
 
%%endhead 
 
\subsection{Как я рассказываю своим детям о войне}
\label{sec:06_12_2020.fb.arestovich_alexei.1.detjam_o_vojne}
\Purl{https://www.facebook.com/alexey.arestovich/posts/3876955605701824}
\ifcmt
	author_begin
   author_id arestovich_alexei
	author_end
\fi

\ifcmt
pic https://scontent-frt3-1.xx.fbcdn.net/v/t1.0-9/130287900_3876947039036014_2411256524228121370_n.jpg?_nc_cat=102&ccb=2&_nc_sid=8bfeb9&_nc_ohc=fgtzJCk4logAX9Un_iD&_nc_ht=scontent-frt3-1.xx&oh=ada9456a76e5bde4bdacb548964d7a10&oe=5FF5D849
caption Как я рассказываю своим детям о войне, Алексей Арестович
\fi

\obeycr
Как я рассказываю своим детям о войне:
1. Когда им три-четыре года: 
- Злые драконы напали на нашу страну. Солдаты сражаются с ними, чтобы защитить детей.
- И умирают?
- И иногда умирают.
2. Когда им четыре-пять:
- Драконы умеют притворяться танками и самолетами, чтобы им удобнее было воевать и нападать на солдат.
- А на детей и мам не нападут?
- Нет, солдаты их не пустят.
3. Пять лет:
- Драконы притворяются людьми. Садятся в танки и самолеты и нападают на людей. 
- А солдаты?
- А солдаты умеют их узнавать и не пускают.
- Как не пускают?
- Сражаются.
- А солдаты защищают только своих детей?
- Всех детей. И мам. И бабушек, и дедушек. А есть ещё мамы - солдаты. 
- И умирают? 
- Да, и иногда умирают.
4. После пяти лет:
- Папа, а злые драконы все ещё притворяются людьми?
- Нет, уже нет.
Есть просто люди, которые сами стали драконами. 
А есть люди, которые, сражаясь, остаются людьми.
\restorecr

\begin{itemize}

\item \fbusr{Алекс Смарт}
\enquote{Дети,я много врал вам, начиная с трех лет}

\item \fbusr{Gelatin V Kapsulah}
Учитывая, что психологический возраст среднестатистического украинца составляет
что то около 16 лет, то таки да. Детьми только и называть)

\item \fbusr{Олександр Бурейко}
Як розповісти про реальну війну Зеленському? Мабуть не легше, ніж тим дітям?

Спробуйте йому пояснити, що якщо його цінності насправді такі як наші - Майдан,
війна це захист від окупантів, українська ідентичність, тощо то
пора і врешті-решт варто зрозуміти, що Порошенко більше
союзник, ніж опонент і тим більше не ворог.

Та й демонстрація власних комплексів на рівні кварталівських жартів про Рошен,
не личить офіційному лідеру держави. Поясніть йому чи його
наближеним як радник з інформаційних питань. Бо таке враження,
що ІПСО за вплив на президента ми програємо.

\item \fbusr{Natik Alyshev}
Дай Бог, чтобы эти дети уже увидели мир своей стране.

\item \fbusr{Dembel Fox}

Я когда выезжал в 14 из Луганска, после очередного \enquote{блокпоста}, когда нас
проверяли полупьяные любители рузЬкого мира, 3-х летняя дочка
спросила:
\begin{itemize}
  \item - а это плохие дяди с автоматами ?
  \item - нет, они не плохие, они несчастные, просто их заколдовал злой волшебник Пу...
\end{itemize}

\item \fbusr{Shole Mahmudova}
Как не хочется говорить про войну. Как тяжело объяснить детям сущность войны! Как жаль, что нашим народам выпала такая участь, никак не жить нам спокойно, не дают «большие силы» нормальным народам жить без войн и смертей!
Терпения и мира Украине!

\item \fbusr{Misha Pyvovarchuk}

Яка милозвучна мова драконів, і як класно що публічні особи можуть собі
дозволити використовувати цю мову на широкий загал!  P. S.
Мовчазна згода жертви на драконівські правила, мрія дракончиків

\item \fbusr{Leyla Gahramanova} 

Это катастрофа мирового масштаба, когда брат ненавидит брата. Этого представить
невозможно. Получается, что власти России сеют вражду
специально, но то правительство, а что же народ, читаешь
комменты, да там вообще ад. Машина пропаганды России это похуже
чем атомная бомба. Хотя у меня полно друзей русских и
украинцев, они не в счет, московские русаки шовинисты, а
украинцы всегда были теплее и роднее

\item \fbusr{Luda Dudnichenko}

В страшному сні не приходило, що Україна доживе до такого кошмару. Саме
обідне,,що багато насєлєнія не розуміють біду , яка впала на
нас. Сьогодні зранку включила ТБ,щоб подивитись на вшанування
воїнів . Пройшла по всіх каналах- тиша. Ітільки на першому
каналі радіо була повна трансляція - привітання
президента,нагородження героїв живих і які загинули. Тільки
ньюсван показало фрагменти. Йде війна, а ми прикрили
коронавірусом всі інші проблеми. В якій це країні виступ
Президента ігнорують засоби інформції?. Як це розуміти? В
країні починають голови піднімати вороги України. Це очевидно

\item \fbusr{Алексей Вовченко}

Победивший дракона сам становится драконом. Ну это если дети будут смотреть кино Захарова

\item \fbusr{Вадик Ладик}
ну за сім років діти добряче виростають!!!!

\item \fbusr{Michael Tschirka}

І як же ж ти розповідаєш 42- р. мальчіку - "янелоху" про війну . ... " "поганиє
порошенковскІе укробандеровци напалі на мірних лугандонов -рускоязичніков
...бойся -бой , малчік ! они повсюду !

\item \fbusr{Кирилл Полевой}

Расскажите своим детям как на седьмом году войны украинский канал легализирует
главаря "ЛНР", зама Плотницкого, и объясните детям, почему СБУ на это никак не
реагирует

\ifcmt
pic https://scontent-frx5-1.xx.fbcdn.net/v/t1.0-9/129914858_3407868022697670_1793477446282995682_o.jpg?_nc_cat=105&ccb=2&_nc_sid=dbeb18&_nc_ohc=6XpwgSon44gAX-EM22G&_nc_ht=scontent-frx5-1.xx&oh=b974e442ff74f167d84977b2f096a908&oe=5FF49579
width 0.3
fig_env wrapfigure
\fi
\end{itemize}
