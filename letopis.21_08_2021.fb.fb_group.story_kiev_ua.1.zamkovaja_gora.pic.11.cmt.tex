% vim: keymap=russian-jcukenwin
%%beginhead 
 
%%file 21_08_2021.fb.fb_group.story_kiev_ua.1.zamkovaja_gora.pic.11.cmt
%%parent 21_08_2021.fb.fb_group.story_kiev_ua.1.zamkovaja_gora
 
%%url 
 
%%author_id 
%%date 
 
%%tags 
%%title 
 
%%endhead 

\iusr{Татьяна Сирота}
\figCapA{На Замковой горе}

\iusr{Зинаида Савицкая}
Что же это?

\iusr{Татьяна Сирота}
\textbf{Зинаида Савицкая} Не знаю. Находится недалеко от капища перед входом на кладбище.

\iusr{Alex Maler}
\textbf{Татьяна Сирота} один из склепов. Замурованный. Их там десятки.

\iusr{Татьяна Сирота}
\textbf{Alex Maler} Спасибо. Буду знать.

\iusr{Зинаида Савицкая}
Спасибо, рисунок славянский. Как здорово Вы гуляете и подмечаете все детали!

\iusr{Сергей Романцов}
Славянский орнамент - символ предков. Используется еще и как оберег для детей.

\iusr{Лана Свирид}
Видимо мы сюда не дошли(

\iusr{Татьяна Сирота}
\textbf{Лана Свирид} Но Вы ведь не спускались с Замковой на Фроловскую., потому и не дошли. @igg{fbicon.wink} 

\iusr{Лана Свирид}
\textbf{Татьяна Сирота} Да.. пошли той же дорогой что заходили.

\iusr{Татьяна Сирота}
\textbf{Лана Свирид} А меня не очень вдохновлял спуск по той крутой лестнице, что на Андреевский. Поэтому пошли, хоть и через кладбище, но более пологим спуском. @igg{fbicon.face.upside.down} 

\iusr{Лана Свирид}
\textbf{Татьяна Сирота} молодцы! В следующий раз тоже пойдем пологим спуском.

\iusr{Elena Antonova}
О, вот и до кладбища добрались...

\iusr{Татьяна Сирота}
\textbf{Elena Antonova} Точно! @igg{fbicon.face.upside.down} 

\iusr{Елена Гребенник}
Алатырь
