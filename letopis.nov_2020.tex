% vim: keymap=russian-jcukenwin
%%beginhead 
 
%%file nov_2020
%%parent 2020
 
%%url 
%%author 
%%tags 
%%title 
 
%%endhead 

\ii{01_11_2020}
\ii{02_11_2020}
\ii{03_11_2020}
\ii{04_11_2020}
\ii{05_11_2020}
\ii{06_11_2020}
\ii{07_11_2020}
\ii{08_11_2020}
\ii{09_11_2020}
\ii{10_11_2020}

\ii{11_11_2020}
\ii{12_11_2020}
\ii{13_11_2020}
\ii{14_11_2020}
\ii{15_11_2020}

\ii{16_11_2020}
\ii{17_11_2020}
\ii{18_11_2020}

\ii{19_11_2020}
\ii{20_11_2020}

\ii{21_11_2020}
\ii{22_11_2020}
\ii{23_11_2020}
\ii{24_11_2020}
\ii{25_11_2020}
\ii{26_11_2020}
\ii{27_11_2020}
\ii{28_11_2020}
\ii{29_11_2020}
\ii{30_11_2020}

Київський Привид

В Повітрі Летить

Київський Привид

Збиває за Мить

Збиває за Київ

Народ, Україну

Вбиває паршиву рашистську скотину

Київський Привид

В Повітрі Летить

Збиває за Мить

Збиває за Київ

Народ, Україну

Вбиває паршиву рашистську скотину

Він Гордість Країни

Він Слава Вінту

Десятки рашистів відправив до трун

Душа його світла

Боронить Своїх

А Дух його Сильний

Сильніш ворогів

Київський Привид

В Повітрі Летить

Збиває за Мить

Збиває за Київ

Народ, Україну

Вбиває паршиву рашистську скотину

Київський Привид

Слава тобі

Київський Привид

Сили тобі

https://mrpl.city/blogs/view/mariupolska-poetesa-oksana-stomina-zhittya-mizh-ryadkiv

"ОСТАННІЙ ПРАПОР" Сл. Оксана Стоміна, муз./вокал В'ячеслав Купрієнко
https://www.youtube.com/watch?v=Ia4hrxAffgg

https://www.facebook.com/oksana.stomina/posts/pfbid0DRVzFrdHRUSa3iX1yPqrP7P5Aj4bbH33uDCJgzo7rfMThWon8m9mqTwMj4n8F4H9l
https://civilvoicesmuseum.org/stories/%22pid-kozhnim-budinkom-v-mariupoli-yakij-mi-rozgribali-bulo-80100-trupiv%22

З ДНЕМ МАТЕРІ!

Вітаю усіх мам і матусь зі Святом!!!

Ех... не вмію я напевне робити дуже гарні поздоровлення... Особливо коли йде
страшна війна, і стільки матерів вже не дочекаються з фронту своїх синів або
дочок, і не тільки з фронту... скільки трагічних історій по всій Україні, і з
Маріуполя теж...  

Але тут є багато чого ще сказати...  

Отже... по-перше,
невеличка така собі сільська оповідка...  короче, яким чином я встиг
поздоровити мою Маму двічі, причому один раз таким собі секретним способом...
Оце я значить в селі зараз. Зараз у нас світить Сонечко, дуже гарна погода. Тут
у мене в селі також є мої батьки...  Отже... коли я до них приїжджаю, вони
зазвичай потирають ручки... бо є кого припахати на сільські роботи... отже.
Вчора я взяв пилку...  спиляв декілька невеличких кущика, навіть деревця... які
заважали на городі. А сьогодні я все це відніс на одну таку Кучу, де всі такі
деревця або спиляні кущики можна полишити...  Отже... відніс я значить оті
деревця на оту Кучу...  потім, йду собі назад. Аж тут бачу, сидить собі сусідка
на лавці... старенька вже така бабця, років здається їй вже за 80... тітка
Віра. У неї ми іноді купляємо молоко та яйця... Короче, вона така... мене
побачила здалеку і одразу кличе так голосно, - на всю вулицю, - а ну Ваня йди
сюди!!! (а говорить ота бабуся Віра досить голосно - незважаючи на свій
поважний вік...) Я такий (!!!!!!!!!!?????????? що там таке сталось?!! )...  ну
добре, підхожу, сідаю на лавочку... там, кажу, з днем Матері вас, тьотя Віра,
ляляля... а вона мені - Ваня, потрібна твоя допомога! Я такий - (!????????) - а
тітка Віра і каже далі... - ну дивись, ось тута у мене телефон, там я отримала
в вайбері гарне поздоровлення - там, фоточка з квіточками + гарне відео... від
моєї подруги, я хочу все це розіслати також іншим моїм рідним та подружкам (ну
тобто, напевне хотіла розіслати... ну наприклад також своїй дочці... ну... я не
знаю всіх тих людей, декотрих напевне й знаю по селу, але вони в неї були
підписані таким чином... що я й не знав, хто це конкретно... ) а я їй кажу - а
це все просто - тицяєте отута, тицяєте отам, і всє дєла! А вона каже - ой Ваня,
я навіть в окулярах, нічого не бачу, не розуміюсь на тому! Ну що поробиш, стара
я вже!...  Допоможи мені, ти ж оті всі діла знаєш, правда!  Ну добре... оце
значить сиджу я на лавочці із тією бабусею, моєю сусідкою, яка не тільки не має
фейсбука, а навіть не знає, як в тому вайбері розіслати повідомлення... але з
іншої сторони, молоко та яйця, які ми в неї купляємо, просто неймовірні...  ну
і допомагаю... оце от значить, бачите картинка, її ж вам ваша власна подруга
надіслала?  Так! Бачу! А оце отут зверху справа такі три цяточки бачите?  Оце
вже я не дуже бачу, але кажи, може й запам'ятаю! Оце от три цяточки не бачить,
а Ваню за сто метрів - бачить...  парадокс, та й годі!  Але добре... Ну...
врешті-решт каже, що так! Так, бачу!  Бачу оті три цяточки зверху справа,
продовжуй, Ваня!  От... оті три цяточки... Ось бачите? Тицяю тута - тут
відкривається менюшка - бачу, так!  продовжуй, Ванєчка! Отут в цій менюшці -
далі - ось написано ПОДІЛИТИСЯ - бачите? Бачу!  Далі - натиснули поділитися...
ось тута написано ВАЙБЕР. Бачите? Бачу! Ось...  Далі... ось випадає список
контактів, тут є по списку ... Ось тута є наприклад Ганна, а ось тута Галина, а
ось тута є Андрій та Василь (їм не треба відсилати!! - одразу застерігає бабуся
- і... ну... я теж думаю, що... поздоровляти Андрія або Василя із днем Матері
напевне не треба... в цьому я з тіткою Вірою повністю згоден... ) потім... є
якісь незрозумілі номери - ну просто номери і все +380...22 - ото я бабусю
питаю - а цей номер треба чи ні? Хто оце взагалі!? А бабуся собі так замислено
лоба потирає і думає... а хто ж це - хто ж це... а! - це теж моя подружка, ми з
нею колись...  давай, надсилай, так!!!  А оцей - далі питаю - оцей! - що на
...57 - оцей!? - майже кричить тітка Віра - нєєєє, це взагалі наш голова
сільради, ти що!?  (ех, забула підписати, що то наш голова сільради...)

Ну короче, познайшов я оті усі номери, повідправляв! І гарну картиночку, там -
квіточки, слова, поздоровлення...  потім, невеличке відео із музикою також
відправив... А потім питаю... а щось я не бачу в вашому списку моєї мами!  А
бабуся - ой, ну звісно, я твою маму, Ваня, звісно знаю! Як же не знати твоєї
мами та твого татка, Ваню! Їх тут всі добре знають!  Але твоя мама в мене ніяк
не підписана, то може, ти знаєш, як її додати і теж надіслати?? Я кажу - ну
так, звісно, зараз зроблю... Ну і от, ввожу номерок моєї мами в телефоні отої
бабці, копіюю оті самі відосики та картинки, надсилаю, готово! 

Оце от щодо того, як можна поздоровити свою власну маму двічі, причому один раз
так, що мама навіть і не буде знати про це? А ось так :) один раз значить я її
поздоровив просто так, явним чином...  а один раз, виходить, таким от секретним
чином... а в кінці знову виходить та ж тема, про яку я вже почав писати - тобто
- про Розриви Реальності, тому що є Сукупна Реальність, в якій ми живемо усі
разом... а є також окремішні Реальності різних людей, які часто-густо просто не
перетинаються ніяким чином... - тому що... 

(1) мама моя звісно отримала поздоровлення від бабці Віри... ну і звісно, мама
моя також надіслала поздоровлення у відповідь... але вона НЕ знає, хто їй
насправді надіслав оте повідомлення із гарними картинкою та відео 

(2) бабця Віра, зі своєї сторони, попрохала мене допомогти... але вона нічого
абсолютно не знає про мої дописи в фейсбуці, і взагалі, їй навіть важко
надіслати просте повідомлення в Вайбері

(3) оті всі люди із списку бабці Віри, яким я надіслав повідомлення із відео та
картинкою, НЕ знають, що повідомлення їм насправді надіслав Ваня на прохання
бабці Віри, вони ж бачать тільки контакт бабці Віри в своєму телефоні, і все! А
де той Ваня? Деякі з них напевне взагалі не знають, що в одному селі із ними
живе Іван, такий собі київський програміст, спеціаліст із таких або таких мов
програмування, який, крім того, що нещодавно знайшов Кроликів на останньому
поверсі Музею Історії Києва і т. д., - і слово Кролик у людей в селі можливо
буде визивати лише одну асоціацію із справжніми живими кроликами і все, а про
те, що кроликів можна також малювати на пластику у тисячі найрізноманітніших
способів (навіть от я недавно бачив наприклад Шоколадного Зайця - пам'ятаєте
пісню про шоколадного мерзавца?) і виставляти на Великдень у великих містах як
от у Києві на Софійській Площі  або ж у Маріуполі на Театральній Площі коло
ДрамТеатру, вони й найменшої гадки не будуть мати... 

(4) а ще з іншої сторони, я от повикладав отут у себе на сторінці різні
відео... із бузком, бджолами або ранковою яєчнею із грудінкою або ж кавою, що
вибігає... але...  ці відео нічого не кажуть про те, що десь поруч живе бабця
Віра, яка все про всіх знає, хоча вона може в житті ні разу не заходила в
фейсбук або тим більше у тік-ток або інстаграм... а коли я наприклад виставив
купу відео із музикою, то знову ж таки ці відео  нічого не кажуть про те, що я
записував ці відео тоді, коли...  короче, всі мої кудись пішли чимось
займатись... і я значить залишився сам-на-сам із пічкою...  і таким чином мені
ніхто не заважав вже записувати оті відео...  От я згадав тут ще добре напевне
всім відому оповідку про слона, вставлю тут...

.................................. ПРИТЧА ПРО СЛОНА ..................................  
Давним-давно в маленькому місті жили-були
шість сліпих мудреців. Одного разу в місто привели слона. Мудреці захотіли
побачити його. Але як? «Я знаю, - сказав один мудрець, - ми обмацати його».
«Хороша ідея, - сказали інші, - тоді ми будемо знати, який він - слон». Отже,
шість чоловік пішли дивитися слона. Перший обмацав велике плоске вухо. Воно
повільно рухалася вперед-назад. «Слон схожий на віяло!» - Закричав перший
мудрець. Другий мудрець помацав ноги слона. «Він схожий на дерево!» - Вигукнув
він. «Ви обидва не маєте рації, - сказав третій, - він схожий на мотузку». Ця
людина намацав слонячий хвіст. «Слон схожий на спис», - вигукнув четвертий.
«Ні, ні, - закричав п'ятий, - слон, як висока стіна!» Він говорив так,
обмацуючи бік слона. Шостий мудрець смикав слонячий хобот. «Ви всі неправі, -
сказав він, - слон схожий на змію». - «Ні, на мотузку!» «Змія!» «Стіна!» «Ви
помиляєтеся!» «Я правий!» Шестеро сліпих кричали один на одного цілу годину. І
вони ніколи не дізналися, як виглядає слон.
....................................................................

Ну і до речі... походу, виходить, що... я через оту бабцю поздоровив заодно
купу-силенну людей, не тільки мою маму у другий (секретний!) раз, але також... як
знайомих, так і таких, про яких я в перший раз щось почув...  (а списочок у
бабці великий!! - бо бабця Віра всіх знає і без всяких фейсбуків... все життя
прожила в селі, ще й німців пам'ятає, напевне ... - от... прийшлось значить оті
всі поздоровлення надсилати у декілька заходів...)



