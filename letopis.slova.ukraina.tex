% vim: keymap=russian-jcukenwin
%%beginhead 
 
%%file slova.ukraina
%%parent slova
 
%%url 
 
%%author 
%%author_id 
%%author_url 
 
%%tags 
%%title 
 
%%endhead 

Так и живем в \emph{Украине}: удочек нет у власти для простых \emph{украинцев},
и рыбы нет для страждущих, \textbf{Еще никому из президентов не удалось объединить
Украину}, Александр Гончаров, strana.ua, 24.05.2021

Євробачення 2021: \emph{Україна} посіла п'яте місце, beztabu.net, 23.05.2021

\emph{Украину} нужно лишить права оценивать выступления участников от
страны-агрессора: это аморально, Игорь Кондратюк, obozrevatel.com, 23.05.2021

Уберите российский сапог и \emph{украинская} культура побьет все мировые
рекорды, Елена Кудренко, obozrevatel.com, 23.05.2021

\emph{Украинские} туристы подрались в самолете: мама младенца озвучила свою
версию конфликта, Марина Петик, obozrevatel.com, 24.05.2021

Не просто \emph{Україна}, а \emph{Україна}-Русь. Як Михайло Грушевський захищає
державні кордони?  radiosvoboda.org, 23.05.2021

Навіть попри те, що Путін не почав нову фазу завоювань в \emph{Україні} –
останній епізод кілька тижнів тому змусив людей у Вашингтоні і Києві
понервувати,
radiosvoboda.org, 24.05.2021

«Дайте нам грошей!»: большая \emph{украинская} мечта о великой халяве, from-ua.com, 24.05.2021

«Евровидение» в Европрессе: Италия — победитель, Германия и Британия  «пасут
задних», а \emph{Украину} назвали «кошмарной танцевальной угрозой», sharij.net,
24.05.2021

Зато выступление GoA от \emph{Украины} с песней «Шум» назвали одним из лучших
выступлений вечера,
sharij.net, 24.05.2021

Паралимпийская сборная \emph{Украины} заняла первое место на чемпионате Европы,
завоевав 37 золотых медалей, sharij.net, 23.05.2021

Кладбище империй: Чему научила США и \emph{Украину} война в Афганистане,
sharij.net, 24.05.2021

Для \emph{Украины}, где миллионы людей с разных сторон, Холодная война долго не сможет быть холодной,
\textbf{Дело Протасевича - еще один повод раздуть Холодную войну}, Денис Жарких, strana.ua, 24.05.2021

Вот я читаю посты некоторых российских патриотов, и не нахожу особой разницы
между ними и патриотами \emph{украинскими},
\textbf{Дело Протасевича - еще один повод раздуть Холодную войну}, Денис Жарких, strana.ua, 24.05.2021

Тобто. вже рік тому можна було бачити, що Північний потік 2 для \emph{України}
- програна справа, \textbf{Наш МИД живет в каком-то выдуманном мире}, Владимир
Воля, strana.ua, 24.05.2021

У келіях Печерського монастиря запалав світильник \emph{української} культури. Саме
тут бере свій першопочаток давня \emph{українська} література, художнє мистецтво,
медицина. Нестор-літописець – перший історик \emph{України}-Руси, автор Повісті, що є
головним джерелом вивчення \emph{української} історії, Агапій – перший відомий лікар,
Аліпій – перший живописець...
radiosvoboda.org, 06.05.2017

Шановні туристи, ми не заперечуємо вагомості внеску цього київського духовного
центру до скарбниці всіх східнослов'янських культур, але наголошуємо, що Лавра
– це передусім феномен \emph{української} культури».
radiosvoboda.org, 06.05.2017

Відвідувачі можуть запитати, чому у тексті Повісті Нестора Літописця (мова
твору церковнослов'янська) древня обитель іменується лише Печерською. Один із
найкращих знавців давньоруських літописів професор Василь Яременко стверджує –
це лише доказ того, що розмовною мовою автора була \emph{українська}. Саме тому
«\emph{українська} лексика ллється суцільним потоком у Повісті: жито, колодязь,
подружжя, туга, печера…» – аргументує професор.
radiosvoboda.org, 06.05.2017

\emph{Украина} была одной из главных составных частей Советского Союза — чрезвычайно
развитой в экономическом отношении, многолюдной и комфортной республикой.
\emph{Украинцы} по сути являлись второй имперской нацией, а выходцы из республики со
своими многочисленными группами поддержки порою оказывались на самых вершинах
власти, \textbf{Надоела Украина? Ежедневное «иди и смотри»}, Константин Кеворкян, ukraina.ru, 24.05.2021

Скоро в \emph{Украину} ворвется лето: синоптики заявили о \enquote{шикарной} погоде и сделали детальный прогноз, obozrevatel.com,
25.05.2021

На реках и водоемах с начала года утонуло более 300 \emph{украинцев}: причины жуткой
статистики, obozrevatel.com, 25.05.2021

\emph{Україна} і ЄС готують рішення про припинення польотів до і над Білоруссю.
Якими можуть бути інші санкції? radiosvoboda.org, 25.05.2021

«Це незграбно і підло. Винуватці зриву спецоперації \emph{українських}
спецслужб мають бути встановлені і покарані. А спроби їх «відмазати» марні», –
написав у твітері колишній міністр, radiosvoboda.org, 25.05.2021
