% vim: keymap=russian-jcukenwin
%%beginhead 
 
%%file 30_10_2019.fb.makarevich_ekaterina.1.ulybka_iskrennjaja.cmt
%%parent 30_10_2019.fb.makarevich_ekaterina.1.ulybka_iskrennjaja
 
%%url 
 
%%author_id 
%%date 
 
%%tags 
%%title 
 
%%endhead 
\subsubsection{Коментарі}
\label{sec:30_10_2019.fb.makarevich_ekaterina.1.ulybka_iskrennjaja.cmt}

\begin{itemize} % {
\iusr{Igor Vol}
+

\iusr{Александр Пасховер}

Когда меня под пытками "испанского сапога", заставят удалить с моей фейсбучной
ленты Екатерину Макаревич, я буду терпеть до последнего. Ну и тогда не
уступлю. Катя, как правило у вас очень теплые посты. Спасибо.

\begin{itemize} % {
\iusr{Екатерина Макаревич}
\textbf{Александр Пасховер} спасибо)) ну и у меня тоже бывают слабости, когда не всегда могу о теплом) но спасибо, что прощаете эту слабость и продолжаете верить в теплоту:)) меня эта вера тоже вытаскивает из разных не очень состояний))
\end{itemize} % }

\iusr{АННА Гукова}
И как называется книга?

\begin{itemize} % {
\iusr{Екатерина Макаревич}
\textbf{АННА Гукова} Экхарт Толле - Сила момента сейчас
\end{itemize} % }

\iusr{АННА Гукова}

Это напомнило мне ситуацию, когда я работала в дет.доме, и одна из старших
девочек обижал моих, они были не то чтоб помладше, но послабее. Так вот, в один
момент, проходя по коридору мимо неё, я приставила её к стенке, посмотрела в
глаза и сказала твёрдым голосом: если ещё раз она кого то обидит, я с ней
поговорю отдельно.

Я до сих пор помню эти глаза...
А она до сих пор говорит, что я её любимый воспитатель.
РS больше 10-ти лет прошло с того момента.
Есть что-то неуловимо настоящее в искренности.

\iusr{Юра Федоришин}

Наверное это и есть Событие, где событийствуют двое. Совпадение в пространстве
и во времени производит Встречу , которая рождает материал для дальнейшего
построения личной жизни души. Вполне возможно что именно так , во мгновении и
происходит встреча с Богом. Припомнился М. Бубер с акцентом на способности "я"
встретиться с "ты". Очень тепло подано, спасибо.

\iusr{Алла Супрун}
Дякую!

\iusr{Гоша Калужский}
очень трогательно

\iusr{Max Righter}
Живя не в рф понимаешь, как просто ничего не значащее "привет, как дела?" или просто улыбка поднимают настроение и заряжают энергией.

\begin{itemize} % {
\iusr{Екатерина Макаревич}
\textbf{Max Righter} 

это не касается страны, мне кажется, все дело во взгляде. В способности
замечать посреди серой будничности что-то простое, но и при этом живое, что с
тобой случается. Этому можно научиться, думаю, в любом стране. Другой вопрос,
где-то легче этим делиться с другими, где-то сложнее, поскольку они погружены в
разборки между собой. Но и даже в такой ситуации страны можно такое видеть,
просто нужно захотеть быть внимательным к мелочам  @igg{fbicon.smile} 

\iusr{юлия корниенко}
\textbf{Екатерина Макаревич} все наш взгляд

\iusr{Екатерина Макаревич}
\textbf{юлия корниенко} угу

\iusr{юлия корниенко}
\textbf{Екатерина Макаревич}

\ifcmt
  ig https://scontent-mxp1-1.xx.fbcdn.net/v/t39.1997-6/s168x128/64420393_2844260942268688_1163146473118892032_n.png?_nc_cat=1&ccb=1-5&_nc_sid=ac3552&_nc_ohc=Qu_wZOq-YxEAX_g5SVS&_nc_ht=scontent-mxp1-1.xx&oh=db1dbe053dcdd2e201e46868eb5113b5&oe=618DCC21
  @width 0.2
\fi

\end{itemize} % }

\iusr{Lena Chornohuz}

Как искренне и чисто. Спасибо, Вы подарили мне улыбку сегодня.

\iusr{Андрей Николаевич Гаазе}

\ifcmt
  ig https://scontent-mxp1-1.xx.fbcdn.net/v/t39.1997-6/s480x480/47169705_726019901117278_1665333462619389952_n.png?_nc_cat=1&ccb=1-5&_nc_sid=0572db&_nc_ohc=k3mWU0JmLRwAX_PNLZl&_nc_ht=scontent-mxp1-1.xx&oh=fffee7b3edbd72e3f28d808359002016&oe=618DCA7C
  @width 0.2
\fi

\iusr{Inna Kayuta}

А что за книга? Я помню как удавалось орущих детей в метро улыбкой успокоить и
перестроить на позитивную волну. Но это должна быть очень зарядная и искренняя
улыбка. Здесь в америке мы очень ощущаем наши привычки закрываться от
постороннего взгляда. А улыбок тут море. Просто мимо идут, но улыбнутся.

\begin{itemize} % {
\iusr{Екатерина Макаревич}
\textbf{Inna Kayuta} Экхарт Толле - Сила момента сейчас

\iusr{Inna Kayuta}
\textbf{Екатерина Макаревич} спасибо!
\end{itemize} % }

\iusr{Алексей Шальнев}
На Кипре я этому научился, люблю свой остров, свой Дом.

\end{itemize} % }
