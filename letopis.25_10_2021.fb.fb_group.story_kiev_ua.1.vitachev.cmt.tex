% vim: keymap=russian-jcukenwin
%%beginhead 
 
%%file 25_10_2021.fb.fb_group.story_kiev_ua.1.vitachev.cmt
%%parent 25_10_2021.fb.fb_group.story_kiev_ua.1.vitachev
 
%%url 
 
%%author_id 
%%date 
 
%%tags 
%%title 
 
%%endhead 
\zzSecCmt

\begin{itemize} % {
\iusr{Александр Киевлянин}
Так Витачов или Витачев?)

\begin{itemize} % {
\iusr{Оксана Дубинина}
\textbf{Александр Киевлянин} укр.- Витачів,
русск. - Витачов

\begin{itemize} % {
\iusr{Александр Киевлянин}
\textbf{Оксана Дубинина} а Витачев по-белорусски?

\iusr{Оксана Дубинина}
\textbf{Александр Киевлянин} похоже разговорное)) @igg{fbicon.wink} 

\iusr{Володимир Бойко}
\textbf{Оксана Дубинина} Просто видаліть російську назву, і всі зрозуміють)

\iusr{Eduardo April}
\textbf{Оксана Дубинина} . В Украине не может быть русского названия!

\iusr{Оксана Дубинина}
\textbf{Володимир Бойко}
як то видалити назву?
Топоніми не я власне призначала))
Я вільно пишу українською і російською мовами. І хто хоче, все розуміє.
А щодо назв, то так склалося, що кияни спокійно сприймають інформацію на українській та російській мовах.
І це нормально.

\iusr{Оксана Дубинина}
\textbf{Eduardo April} Если бы не могло, то и не было бы))
Я - за українську мову, люблю і насолоджуюся її звучанням @igg{fbicon.heart.red}.
Але я проти тиску та бездумних заперечень @igg{fbicon.thinking.face}.
Кожний топонім має назву на будь-якій мові.
Хорошого Вам дня!

\iusr{Оксана Будулуца}
\textbf{Оксана Дубинина} Я вважаю, що назви населених пунктів, не можна перекладати на іноземні мови.

\iusr{Оксана Дубинина}
\textbf{Оксана Будулуца} Оксаночко, але ж назви на інших мовах вже існують.
Наприклад, італійці кажуть назву міста Napoli («наполі»), а ми - Неаполь, вони Roma - («рома»), а ми - Рим, турки Istanbul (істанбул»), а ми - Стамбул..., та дуже багато схожих прикладів.
Через це хіба ми всі не розуміємо про що мова?))
\end{itemize} % }

\iusr{Olga Lys}
\textbf{Александр Киевлянин} Витачів

\end{itemize} % }

\iusr{Микола Веселий}
Класс! Спасибо, вспомнилось мне это, бывал там, такая красота

\iusr{Svitlana Bacumovich}
ПлОщадка панорамна!

\begin{itemize} % {
\iusr{Оксана Дубинина}
\textbf{Svitlana Bacumovich} у меня ошибка? Спешила поделиться, могла не заметить опечаток.
Спасибо, если укажете)

\iusr{Оксана Дубинина}
\textbf{Svitlana Bacumovich} надпись под фото подправила. Благодарю!)
\end{itemize} % }

\iusr{Андрей Стащук}


\ifcmt
  tab_begin cols=3,no_fig,center

     pic https://scontent-frx5-1.xx.fbcdn.net/v/t39.30808-6/248362660_1076033656534651_3404112317809813415_n.jpg?_nc_cat=105&ccb=1-5&_nc_sid=dbeb18&_nc_ohc=So803JlaAp8AX9Ot1Ws&_nc_ht=scontent-frx5-1.xx&oh=00_AT8tGbX9gSUQEJucCTSxQDVi6QP1rfgfNVw3KV6Sr9Nx1w&oe=61D4ED9D

		 pic https://scontent-frx5-1.xx.fbcdn.net/v/t39.30808-6/246768354_1076034053201278_7717160710053825900_n.jpg?_nc_cat=105&ccb=1-5&_nc_sid=dbeb18&_nc_ohc=kf6kRMyQUHsAX8rWjve&_nc_ht=scontent-frx5-1.xx&oh=00_AT-HpRMA53MkDk_81rgzVQ-0jDaSdXF8ZCWTdt1_8RiprQ&oe=61D58AA4

		 pic https://scontent-frt3-2.xx.fbcdn.net/v/t39.30808-6/247480017_1076034103201273_3550563560826547864_n.jpg?_nc_cat=101&ccb=1-5&_nc_sid=dbeb18&_nc_ohc=bht9qVo6doIAX-4a1Zm&_nc_ht=scontent-frt3-2.xx&oh=00_AT-Ad5pg9KwApIJNbhIcdFMBwQvnS9j0r9szy_nO7_D5UA&oe=61D58185

  tab_end
\fi

\iusr{Андрей Стащук}
Был уже два раза. Виды и атмосфера супер!

\iusr{Viktor Chernikov}

В мене з Вітачива тесть, та я не один день там був із сім'єю. Краєвиди там
незабутні, просто туди треба приїхати та подивитись з пагорбів на ту велич
землі та Дніпра. Колись я ходив на риболовлю до водокачки, то вниз праворуч
каплички вниз.Ох і рибаля була гарнесенько, я вам скажу, тільки от на верх
видиратись було ще це ті пригоди, але хто знає той погодиться зі мною що того
було варте!

\begin{itemize} % {
\iusr{Оксана Дубинина}
\textbf{Viktor Chernikov} 

Ви так смачно розповідаєте, що захотілося з тої Вашої риболовлі юшки свіжої
запашної @igg{fbicon.face.savoring.food} ))

Долати перешкоди - то характер! Саме такі пригоди вражають і запам‘ятовуються!

Дякую!)

\iusr{Viktor Chernikov}
\textbf{Оксана Дубинина} 

То я бачу з лайків що є люди котрі пам'ятають ту водокачку? А хто пам'ятає хату
ДІДА ПАНАСА?

\end{itemize} % }

\iusr{Olena Savych}

\ifcmt
  ig https://i2.paste.pics/6a67cc6e94b35d3d2ee7a08240326fde.png
  @width 0.2
\fi

\iusr{Лариса Проскурняк}
Краєвиди у Витачіві зачаровують.

\ifcmt
  ig https://scontent-frt3-2.xx.fbcdn.net/v/t39.30808-6/248479758_4836192879781978_2766421825766945549_n.jpg?_nc_cat=103&ccb=1-5&_nc_sid=dbeb18&_nc_ohc=ztXTslyYpFUAX-shUjv&_nc_ht=scontent-frt3-2.xx&oh=00_AT_U_qPtkG9LZ8N0mt6gWEFNk8AFY896H6BBN7rcJnpbgw&oe=61D4D4A0
  @width 0.3
\fi

\iusr{Оксана Дубинина}
\textbf{Лариса Проскурняк} @igg{fbicon.heart.red} це ж казка!)

\iusr{Ольга Кирьянцева}

Спасибо большое. Красивейшие места и природа в Обуховском районе. Может, потому
что связаны с самыми чудесными воспоминаниями со студенчеством и стройотрядом.
Вот бы ещё фото Украинки увидеть... Не планируете, случайно?) Наша бригада там
дамбу строила.

\begin{itemize} % {
\iusr{Оксана Дубинина}
\textbf{Ольга Кирьянцева} 

Специально для бывших юных строителей-студентов дамбы в Украинке пару фото и
маленькое видео @igg{fbicon.heart.red}

\ifcmt
  tab_begin cols=3,no_fig,center

     pic https://i2.paste.pics/aa8a12ac40113747085a9ae3a61269a5.png
		 pic https://i2.paste.pics/6734104c35e8ecec257ea5c8d7ba9757.png
		 pic https://i2.paste.pics/b2b3ccb708c9438bcadcd950c412af17.png

  tab_end

  tab_begin cols=2,no_fig,center,resizebox=0.8

		 pic https://i2.paste.pics/5c2c2e00f0491b22a5f7a8a3faa86ae5.png
		 pic https://i2.paste.pics/6c4b4784864b27e9b884dc4b8bfd2d65.png

  tab_end
\fi

\iusr{Оксана Дубинина}
\textbf{Ольга Кирьянцева}


\ifcmt
  tab_begin cols=3,no_fig,center

     pic https://scontent-frt3-2.xx.fbcdn.net/v/t39.30808-6/247360125_6556676437705901_1272670073825721604_n.jpg?_nc_cat=101&ccb=1-5&_nc_sid=dbeb18&_nc_ohc=88L3DuL0ZmIAX_g_IM8&_nc_ht=scontent-frt3-2.xx&oh=00_AT9qZmNuXDnyeK7q-H_9W1l064NQyFuWDsWk1YpSHQ-PiA&oe=61D59F35

		 pic https://scontent-frt3-1.xx.fbcdn.net/v/t39.30808-6/247878530_6556683174371894_5767542734725529755_n.jpg?_nc_cat=106&ccb=1-5&_nc_sid=dbeb18&_nc_ohc=6pIcPGle5IEAX_gwEZc&_nc_ht=scontent-frt3-1.xx&oh=00_AT_tgK9bnCrz-HygG_5-ul2EIAOo9jOESQ-LIvFTxGuW2Q&oe=61D4A3DC

		 pic https://scontent-frx5-2.xx.fbcdn.net/v/t39.30808-6/248700542_6556685251038353_6317233260569560402_n.jpg?_nc_cat=109&ccb=1-5&_nc_sid=dbeb18&_nc_ohc=8QGNtJXJ8EsAX_i9pbR&_nc_ht=scontent-frx5-2.xx&oh=00_AT9kcFnsGLZJbDj3C_HuVVs1ouqqHfr_rQWuMmRZvnZT1A&oe=61D62263

  tab_end
\fi

\iusr{Оксана Дубинина}
\textbf{Ольга Кирьянцева} в следующий раз сделаю фото пристани)

\ifcmt
  ig https://scontent-frt3-2.xx.fbcdn.net/v/t39.30808-6/246728910_6556691744371037_5607006264383833433_n.jpg?_nc_cat=101&ccb=1-5&_nc_sid=dbeb18&_nc_ohc=mGS9a3YkiQIAX-qsx0C&_nc_ht=scontent-frt3-2.xx&oh=00_AT9RDOj829BL3GIDPQTigc0DPcf3Q7nytVklsr0nKg_oUg&oe=61D60CD2
  @width 0.2
\fi

\iusr{Оксана Дубинина}
\textbf{Ольга Кирьянцева} 

а вот фото именно с дамбы с Украинки в сторону Конча Заспы, закат над Козинкой.
Вода тихая была, небо отражалось

\ifcmt
  ig https://scontent-frt3-2.xx.fbcdn.net/v/t39.30808-6/247534869_6556709071035971_3824531599149692899_n.jpg?_nc_cat=101&ccb=1-5&_nc_sid=dbeb18&_nc_ohc=6CllHcY8M0EAX8_BHvF&_nc_ht=scontent-frt3-2.xx&oh=00_AT-NVy_oGN9OIPtrjHhM_ID4YCCP1OOl53yD0egWL9OT9A&oe=61D5EBBF
  @width 0.3
\fi

\iusr{Ольга Кирьянцева}
\textbf{Оксана}, 

Оксаночка, Вы исполнительница мечт!)) Как мне хотелось увидеть хоть одним
глазком на \enquote{дело рук наших})) И вот оно, свершилось! Дамба жива, цела и в
полном порядке  @igg{fbicon.face.smiling.eyes.smiling}  @igg{fbicon.wink} 

\end{itemize} % }

\iusr{Марина Юсупова}
Bravo  @igg{fbicon.hands.applause.yellow} 

\iusr{Марина Юсупова}
Спасибо за интересный рассказ и прекрасные фото.
Об этом замечательном месте услышала впервые, Теперь хочется всё увидеть и ощутить самой

\iusr{Раиса Карчевская}
Прекрасный пост и замечательные фотографии Спасибо большое

\iusr{Оксана Дубинина}
\textbf{Ольга Кирьянцева}
Понимаю @igg{fbicon.face.smiling.hearts} 
Я сейчас поищу фото набережной Украинки) - покажу. Жаль, тогда только проездом были. Но обязательно в следующий раз подробнее фото сделаю

\iusr{Наталия Варавина}
Порадовали, спасибо!

\iusr{Ольга Кирьянцева}
Спасибо, Оксаночка. Какая красота! И белый парус - очень символично)).  @igg{fbicon.face.blowing.kiss}  @igg{fbicon.face.happy.two.hands} 

\iusr{Алла Чумак}

\ifcmt
  ig https://scontent-frx5-1.xx.fbcdn.net/v/t39.30808-6/247167554_621568698976512_1240766133468561537_n.jpg?_nc_cat=110&ccb=1-5&_nc_sid=dbeb18&_nc_ohc=Q6ixETsJP1cAX9_U5ZE&_nc_ht=scontent-frx5-1.xx&oh=00_AT-d3Yyz63O1OP-nyk1VNafAx6qstTPDs5-qdFrf88XW4g&oe=61D47CA7
  @width 0.3
\fi

\iusr{Аркадий Заславский}
А какое там озеро...

\iusr{Татьяна Сидорук}
Чудесно!

\ifcmt
  ig https://scontent-frx5-1.xx.fbcdn.net/v/t39.30808-6/248692058_348442280412574_3850198512204322596_n.jpg?_nc_cat=100&ccb=1-5&_nc_sid=dbeb18&_nc_ohc=xE3-4Mcp8ucAX_ZrGqk&_nc_oc=AQm-4w3OKoo4M4J0w0Txv0j-YA-I5HWofL9Q6U3tk7pYPcRFpZVP2aPXmOrGMn9vDQo&_nc_ht=scontent-frx5-1.xx&oh=00_AT_PMpnFg9uPIYKmnEritt-MDYgZatcTNfPYybQijkWA6w&oe=61D627B1
  @width 0.3
\fi

\iusr{Алла Чумак}

Были летом. Красота необычайная! Пейзажи неповторимые, не насмотреться на
разлив Днепра, высокий берег, душистое разнотравье, памятник древнему Вытичеву,
замечательную деревянную церквушку... Прекрасное ощущение причастности к давней
истории нашей земли и народа...


\iusr{Алла Чумак}


\ifcmt
  tab_begin cols=2,no_fig,center.resizebox=0.5

     pic https://scontent-frt3-1.xx.fbcdn.net/v/t39.30808-6/242909019_621574672309248_472118350196677053_n.jpg?_nc_cat=107&ccb=1-5&_nc_sid=dbeb18&_nc_ohc=Fz3V763zgRsAX-2nZlA&_nc_ht=scontent-frt3-1.xx&oh=00_AT9e3CAByv8QFETdjFEJaaTLlJ-oQGlv2awVjn6RcNglYQ&oe=61D54E70

		 pic https://scontent-frx5-2.xx.fbcdn.net/v/t39.30808-6/247354870_621574828975899_2907913145680274103_n.jpg?_nc_cat=109&ccb=1-5&_nc_sid=dbeb18&_nc_ohc=r3oi6SnoGVoAX86mj4B&_nc_ht=scontent-frx5-2.xx&oh=00_AT_pgo2BGCfqYOktY0cm3VadIjK4Un09rird0v7NeQzqkg&oe=61D5664C

  tab_end
\fi

\iusr{Olga Lys}
Дякую

\iusr{Viktoria Terpylo}
До боли знакомые склоны... мои корни из Триполья

\begin{itemize} % {
\iusr{Нина Земзина}
\textbf{Viktoria Terpylo} Фамилия очень распространенная в Триполье!

\iusr{Viktoria Terpylo}
\textbf{Нина Пугачева} да.. да... трипольская. А мамина Щур девичья.
\end{itemize} % }

\iusr{Irina Koeznetsova}

\ifcmt
  ig https://scontent-frt3-2.xx.fbcdn.net/v/t39.1997-6/p370x247/69129996_2155548331222884_2348915591252803584_n.png?_nc_cat=103&ccb=1-5&_nc_sid=0572db&_nc_ohc=9wpppW83t8QAX9G7OV1&_nc_ht=scontent-frt3-2.xx&oh=00_AT8mU91XzXV7je_B0mB5IXXNTszJHZb973acNO_RzrVacw&oe=61D51C76
  @width 0.2
\fi

\iusr{Валентина Прокопенко}
Літо 2021

\ifcmt
  ig https://scontent-frt3-1.xx.fbcdn.net/v/t39.30808-6/246780499_1289100274870261_1598500760865042377_n.jpg?_nc_cat=106&ccb=1-5&_nc_sid=dbeb18&_nc_ohc=21rjkt8YNYkAX9ZRRxR&_nc_ht=scontent-frt3-1.xx&oh=00_AT8-i0d6Scq_wPpKpKZ8TxDd2ZLt5uQUqq2mJO0lvh_A4w&oe=61D5D1B7
  @width 0.3
\fi

\iusr{Алла Баглюк}
Энергетически сильное место, время уносится в далекое прошлое. А Днепр предстает во всей красе.

\iusr{Виктория Косцевич}
Как здорово!

\iusr{Людмила Андриевская}
Надзвичайно гарно! Дніпровськи враєвиди у всій красі!

\iusr{Людмила Андриевская}

\ifcmt
  ig https://scontent-frx5-1.xx.fbcdn.net/v/t39.30808-6/246974321_906212750282354_868326967861631477_n.jpg?_nc_cat=110&ccb=1-5&_nc_sid=dbeb18&_nc_ohc=177KzUx8YGsAX-UVIOk&_nc_ht=scontent-frx5-1.xx&oh=00_AT8TltTvS2k7A61dLT_8H0q2B5i5ZTSY2CABeHHXykcyxw&oe=61D5778D
  @width 0.3
\fi

\iusr{Анастасия Сычевская}

\ifcmt
  ig https://scontent-frx5-2.xx.fbcdn.net/v/t39.1997-6/s168x128/47270791_937342239796388_4222599360510164992_n.png?_nc_cat=1&ccb=1-5&_nc_sid=ac3552&_nc_ohc=biH9FXmFm-8AX95J40s&_nc_ht=scontent-frx5-2.xx&oh=00_AT8JNn_1k8sokNJh3RkUy3lInIuINlv6Nk8mOxPZl9h8zg&oe=61D4CC76
  @width 0.1
\fi

\iusr{Ирина Иванченко}

Какие замечательные фото, Оксана! Удивительная природа, буйство осенних
красок, чудесные панорамные виды Витачова - ты славно потрудилась, благодарю за
прекрасную экскурсию!


\iusr{Оксана Дубинина}
\textbf{Ирина Иванченко}  @igg{fbicon.face.smiling.hearts} 

\iusr{Виталий Войтенко}
Там живут замечательные люди и они мои друзья

\ifcmt
  ig https://i2.paste.pics/7c2cc6af9e72ca069841c843e3af63aa.png
  @width 0.2
\fi

\iusr{Оксана Дубинина}
\textbf{Виталий Войтенко} привет Вашим друзям! Я в восхищении от этих мест @igg{fbicon.face.smiling.hearts} 

\iusr{Людмила Мозговая}
А мы летом там были...

\iusr{Irina Malshenkowa}
А кручі називали левадами!

\begin{itemize} % {
\iusr{Оксана Дубинина}
\textbf{Irina Malshenkowa} дякую! дуже гарне українське слово «левади» @igg{fbicon.heart.red}
Згодна, якби цей допис був українською, було б дуже мелодійно і гарно @igg{fbicon.face.smiling.hearts} 
Якщо буде натхнення, напишу))
\end{itemize} % }

\iusr{Людмила Джулай}
Це ж просто заповідний красовид, куди не глянь... Левади. Дякую, що нагадали.

\iusr{Таня Сидорова}

Дякую за чудові фото, я була тільки в Українці і то давно. Дякую за фото
подорож. Ось тільки прикро чути, що так мало людей там живе.

\iusr{Наталья Эгатова}
От этих видов так и веет покоем, простором и древностью! Пейзажи необыкновенно красивые!

\iusr{Ольга Собко-Нестерук}
Такое удовольствие от рассказа и фото видов села. Спасибо Едем

\iusr{Оксана Дубинина}
\textbf{Ольга Собко-Нестерук} спасибо @igg{fbicon.face.smiling.hearts}  @igg{fbicon.face.happy.two.hands} 

\iusr{Нина Земзина}

Похоже, мы были с Вами в Витачеве и Халепье в один день. Живем в Украинке и ездим
по этому маршруту очень часто, через Триполье. Прошлый раз впервые проехали из
Витачева в Халепье по грунтовке в сторону озера, которое местные называют
оз. Рица. В Халепье заехали через ул Яблунева. Это было незабываемо (местами даже
жутковато!)


\iusr{Оксана Дубинина}
\textbf{Нина Земзина} очень даже возможно)) @igg{fbicon.face.happy.two.hands}  Эмоций -море!

\iusr{Людмила Нагорная}
\textbf{Оксана Дубинина} 

Дякую за прекрасні фото, але маю зазначити, що власні назви не переводять. Отже
\enquote{Витачів}, російською \enquote{Вытачив}.

\begin{itemize} % {
\iusr{Оксана Дубинина}
\textbf{Людмила Нагорная} і Вам дякую за добрі слова!

Щодо перекладу географічних назв треба написати в іншу інстанцію, яка друкує
або створює електронні карти. Я можу висловити власну думку, а Ви спробуйте
замінити «Витачов» на «Вытачив» на всіх електронних інфосайтах. Я точно не маю
повноважень)

\end{itemize} % }

\iusr{Светлана Ландыш}
Спасибо.

\iusr{Александр Ладик}

Існує миле українське слово \enquote{КРАЄВИД}, його вживають усі, хто
спілкується українською

\begin{itemize} % {
\iusr{Оксана Дубинина}
\textbf{Александр Ладик} я спеціально обмовилась, що місцеві жителі називають «красовид», а ще взяла слово в лапки.
Вважаю, що так і з‘являються нові предучодові слова в нашому лексиконі.
Краєвиди теж дуже мило.
А головне - спілкуватися позитивно))

\iusr{Александр Ладик}
\textbf{Оксана Дубинина} Це не зрозуміло, коли пишеться російською. Здається що це \enquote{опечатка}, чи автор почув не так, як є.

\iusr{Оксана Дубинина}
\textbf{Александр Ладик} я підтверджую почуте від місцевих жителів - вони кажуть саме красовид)
\end{itemize} % }

\iusr{Kaurkovska Veronika}
Как бы отелями не застроили предприимчивые шустряки -сытопузикм

\begin{itemize} % {
\iusr{Оксана Дубинина}
\textbf{Kaurkovska Veronika} это точно!

\iusr{Andy Yankovskiy}
\textbf{Kaurkovska Veronika} уже почти)

\begin{itemize} % {
\iusr{Оксана Дубинина}
\textbf{Andy Yankovskiy} от того как огородили высоченными заборами Конча Заспу становится грустно...

\iusr{Andy Yankovskiy}
\textbf{Оксана Дубинина} 

вот так касается, в принципе там нужно место для кофе, но думаю это просто
начало, т к хозяин уже чувствует себя там царём)) при мне отдавал указания
рабочему пойти погонять людей, ктр просто рассматривают, пытаясь угадать что
будет внутри, ресторан или отель?

\ifcmt
  ig https://scontent-frt3-1.xx.fbcdn.net/v/t39.30808-6/247453994_4684549971567666_5392747150825586801_n.jpg?_nc_cat=107&ccb=1-5&_nc_sid=dbeb18&_nc_ohc=2Yxy-pNMWiMAX_23TBe&_nc_oc=AQldzGhfJWjNJ65N9wpLvZgQKemcwv1hE-cstTXuarkRD5Mc2CON6n9mEh0SenvhlIs&_nc_ht=scontent-frt3-1.xx&oh=00_AT-NOhQCsta67aRb-Z5M17G5QFjpdx1uzIApsO8CIxjVkA&oe=61D50F72
  @width 0.4
\fi

\iusr{Оксана Дубинина}
\textbf{Andy Yankovskiy} да, я тоже обратила внимание на строительные работы по благоустройству ресторана

\iusr{Оксана Дубинина}
\textbf{Andy Yankovskiy} кстати, именно это фото - стоп-кадр из видео с дрона год назад, а сейчас там активный непонятный движ со стройматериалами

\iusr{Andy Yankovskiy}
Это частный участок, но по нормам строение должны быть 5 м от границы участка, а не на самой границе))
\end{itemize} % }

\end{itemize} % }

\iusr{Andy Yankovskiy}
Ага рядом с мельницей уже ресторан упёрся в её крыло)))

\iusr{Gregory Kushnir}

Якщо не помиляюся, тут знімалася одна зі сцен українського кіно \enquote{Поводир}.. А
ще років 15 тому, відпочиваючи тут із друзями, я вперше дізнався історію про
\enquote{батька Зеленого} - зі слів місцевого старожила. Отаман Зелений прославився
діючи саме на цих теренах 100 років тому, він навіть брав Київ під свою владу
із кількома іншими отаманами, але то вже інша історія..

\begin{itemize} % {
\iusr{Оксана Дубинина}
\textbf{Gregory Kushnir} я теж дещо чула про Зеленого. Шалені часи в історії. Але одночасно й цікаві
\end{itemize} % }

\iusr{Соломаха Вита}

\ifcmt
  ig https://i2.paste.pics/04c24623dba0dbc06f1738413e76c0b6.png
  @width 0.2
\fi

\iusr{Максим Плескач}
Там красиво!!!

\ifcmt
  ig https://scontent-frx5-1.xx.fbcdn.net/v/t39.30808-6/248553368_4500545643340175_5283842529889294715_n.jpg?_nc_cat=105&ccb=1-5&_nc_sid=dbeb18&_nc_ohc=-C-ZG4kwJzYAX9nF9mj&_nc_ht=scontent-frx5-1.xx&oh=00_AT_KNBBiKICDUtZPmAt-brQI7vlpf3bGGI1tI64lhDikGg&oe=61D5E8E3
  @width 0.4
\fi

\iusr{Максим Плескач}
Суперово

\ifcmt
  ig https://scontent-frt3-1.xx.fbcdn.net/v/t39.30808-6/247263258_4500552770006129_4915469662682304848_n.jpg?_nc_cat=108&ccb=1-5&_nc_sid=dbeb18&_nc_ohc=_WUXmM7HXBMAX8T29Na&_nc_ht=scontent-frt3-1.xx&oh=00_AT_daQ5OHrAo30Cqat8EukrXVO4CcdzU6CFj4F9YrGNi-w&oe=61D4DBA3
  @width 0.2
\fi

\iusr{Елена Пойлишер}
Была там, очень красиво!

\iusr{Пятенко Надежда}

\ifcmt
  ig https://scontent-frx5-1.xx.fbcdn.net/v/t39.30808-6/247238492_1741038146094344_4036609210307687743_n.jpg?_nc_cat=100&ccb=1-5&_nc_sid=dbeb18&_nc_ohc=M2-4geDIaEwAX-AvZCn&_nc_ht=scontent-frx5-1.xx&oh=00_AT_xfoG-VBpwQxsOBXLYaoy9CwpCNx4DHnd7KPCxlaK9pw&oe=61D6314B
  @width 0.4
\fi

\iusr{Petro Bisirkin}
Бував тут не один раз, краса цієї місцевості захоплює...

\iusr{Татьяна Чута}
Очень интересно, красота.... @igg{fbicon.rose} 

\iusr{Татьяна Зубко Маркина}

\ifcmt
  ig https://scontent-frx5-2.xx.fbcdn.net/v/t39.1997-6/s168x128/93027172_222645632401274_7176243611145601024_n.png?_nc_cat=1&ccb=1-5&_nc_sid=ac3552&_nc_ohc=I471at92lsIAX_EhuGA&_nc_ht=scontent-frx5-2.xx&oh=00_AT__l2KfbgZJtH8u15aWdCyQj0yfgeIxkW-C_FEuAgziyg&oe=61D5F570
  @width 0.1
\fi

\iusr{Нина Гордийчук}
Спасибо за интересную информацию.

\iusr{Вячеслав Крячок}
Мои предки из этих мест. КРЯЧОК и ВОЛОВЕНКО !

\ifcmt
  ig https://i2.paste.pics/079ad84af37abf0c4960db65d6a321eb.png
  @width 0.2
\fi

\iusr{Вячеслав Крячок}
Родители в детстве меня отправили в Ветачов, я могу сказать что там вырос

\iusr{Александр Богданов}

\ifcmt
  ig https://i2.paste.pics/3ab4186f162064b45b4c902bf1a246cc.png
  @width 0.3
\fi

\iusr{Sasha Ty}
Цікаво. Дякую.
Мені випало відвідати Витачів коли ще греблі не існувало. Краєвиди шалені і село було як у Шевченка.
 @igg{fbicon.smile} 

\iusr{Арт Юрковская}
Уветичи превратились в Витачив.


\end{itemize} % }
