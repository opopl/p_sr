% vim: keymap=russian-jcukenwin
%%beginhead 
 
%%file 27_05_2021.fb.bib.dnr.doneck.nauchnaja_krupskoj.1.chitatel_konferencia_dnr
%%parent 27_05_2021
 
%%url https://www.facebook.com/drunbk/posts/2045124658963453
 
%%author_id bib.dnr.doneck.nauchnaja_krupskoj
%%date 
 
%%tags biblioteka,chtenie,dnr,donbass,kniga,konferencia,literatura
%%title Читательская конференция «Сделано в ДНР: по страницам книг, изданных в Республике»
 
%%endhead 
 
\subsection{Читательская конференция «Сделано в ДНР: по страницам книг, изданных в Республике»}
\label{sec:27_05_2021.fb.bib.dnr.doneck.nauchnaja_krupskoj.1.chitatel_konferencia_dnr}
 
\Purl{https://www.facebook.com/drunbk/posts/2045124658963453}
\ifcmt
 author_begin
   author_id bib.dnr.doneck.nauchnaja_krupskoj
 author_end
\fi

Читательская конференция «Сделано в ДНР: по страницам книг, изданных в
Республике».

27 мая в Русском центре Донецкой республиканской универсальной научной
библиотеки им. Н. К. Крупской состоялась читательская конференция «Сделано в
ДНР: по страницам книг, изданных в Республике». Конференция была посвящена
обсуждению сотрудничества между библиотеками, писателями и издателями;
рассмотрению пройденного творческого пути за время существования Донецкой
Народной Республики.

\ii{27_05_2021.fb.bib.dnr.doneck.nauchnaja_krupskoj.1.chitatel_konferencia_dnr.pic.1}

Открыла конференцию Наталья Чернецкая – генеральный директор издательского дома
«Edit», которая отметила, насколько важно республиканским издательствам и
писателям выходить на российский рынок. Также издатель рассказала об успехе
книг «Донбасс для \enquote{чайников}» и «2014: точка невозврата», которые победили в
нескольких престижных литературных премиях РФ.

\ii{27_05_2021.fb.bib.dnr.doneck.nauchnaja_krupskoj.1.chitatel_konferencia_dnr.pic.2}

Крайне важные и волнующие вопросы поднял председатель правления Союза писателей
ДНР, член Союза писателей России, писатель-фантаст Фёдор Березин. Автор говорил
о влиянии интернет-пространства на современное книжное дело, о падении тиражей
и исчезновении возможности переиздания произведений.

\ii{27_05_2021.fb.bib.dnr.doneck.nauchnaja_krupskoj.1.chitatel_konferencia_dnr.pic.3}

Писатель, краевед, автор книги «Донбасс для «\enquote{чайников}» Олег Измайлов
совместно с Натальей Чернецкой анонсировал выход нового издательского проекта –
«Большое открытие страны огня». Данная книга станет дополненным переизданием
советского сборника «Открытие страны огня», где собраны произведения русских
писателей, жизнь и творчество которых тесно связаны с Донбассом.

На конференции также выступили: поэт, издатель, член Союза писателей ДНР и
России Игорь Лысый; поэтесса, член Союза писателей ДНР Виктория Семибратская;
сотрудник отдела краеведения библиотеки им. Н. К. Крупской Дарья Ильченко; поэт,
член Союза писателей ДНР и России Александр Гросов, поэт, руководитель ансамбля
«Донбасский характер» Владимир Чачанидзе; поэты Александр Кечко и Алина
Николаева.

Все участники конференции поздравили сотрудников библиотеки с Днем библиотек
Донецкой Народной Республики.

Мы благодарим всех за участие и желаем творческих успехов!
