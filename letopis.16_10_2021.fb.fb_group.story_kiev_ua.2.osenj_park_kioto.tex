% vim: keymap=russian-jcukenwin
%%beginhead 
 
%%file 16_10_2021.fb.fb_group.story_kiev_ua.2.osenj_park_kioto
%%parent 16_10_2021
 
%%url https://www.facebook.com/groups/story.kiev.ua/posts/1776104402586353
 
%%author_id fb_group.story_kiev_ua,sirota_tatjana.kiev
%%date 
 
%%tags gorod,kiev,krasota,osenj,park,park.kiev.kioto,priroda
%%title Осень - это время  любоваться природой
 
%%endhead 
 
\subsection{Осень - это время  любоваться природой}
\label{sec:16_10_2021.fb.fb_group.story_kiev_ua.2.osenj_park_kioto}
 
\Purl{https://www.facebook.com/groups/story.kiev.ua/posts/1776104402586353}
\ifcmt
 author_begin
   author_id fb_group.story_kiev_ua,sirota_tatjana.kiev
 author_end
\fi

\begin{multicols}{2} % {

Осень - это время любоваться природой.

И сегодня мы с младшим внуком едем любоваться осенним парком \enquote{Киото}.

В честь побратима Киева, японского города Киото, на левом берегу Киева, вдоль
Броварского проспекта и улицей Киото (между станциями метро \enquote{Черниговская} и
\enquote{Лесная}) в 1972  году обустроили парк.

Парк выполнен с использованием традиционных для Японии деталей: здесь есть сад
камней и камень почетного сидения, пятиметровая квадратная ступа – гранитная
пагода, в которую, по традиции, принято закладывать реликвии, что благоприятно
влияет на окружающую среду, холм, повторяющий очертания горы Фудзияма.
Территория парка украшена мостиками, по которым можно перейти на небольшие
островки, японскими фонариками и множеством других элементов из страны
Восходящего солнца. Здесь растут вековые сосны, японский клен и располагается
самая длинная в Украине аллея сакур.

Бродя по аллеям это небольшого, но прекрасного парка, вдруг вспомнилось начало
стихотворения Владимира Котикова:

"Сердце нежным пламенем ожгла –

Шаль цветную на себя набросив,

Медленно, украдкой подошла,

И в глаза глядит Царица-осень."

Парк Киото.

15 октября 2021 года 

\setlength{\parindent}{0pt}
\ii{16_10_2021.fb.fb_group.story_kiev_ua.2.osenj_park_kioto.pic.1}
\end{multicols} % }

\ii{16_10_2021.fb.fb_group.story_kiev_ua.2.osenj_park_kioto.pic.2}

\begin{multicols}{2} % {
\setlength{\parindent}{0pt}
\ii{16_10_2021.fb.fb_group.story_kiev_ua.2.osenj_park_kioto.pic.3}

\end{multicols} % }
