% vim: keymap=russian-jcukenwin
%%beginhead 
 
%%file 29_01_2021.fb.fb_group.story_kiev_ua.1.devushka_ulybka.cmt
%%parent 29_01_2021.fb.fb_group.story_kiev_ua.1.devushka_ulybka
 
%%url 
 
%%author_id 
%%date 
 
%%tags 
%%title 
 
%%endhead 
\subsubsection{Коментарі}

\begin{itemize} % {
\iusr{Мария Бородатюк}
\textbf{Ирина Петрова}, и весело, и грустно как-то... Но очень живо и интересно!!! « Совпадение? Не думаю!»

\begin{itemize} % {
\iusr{Ирина Петрова}
\textbf{Мария Бородатюк} спасибо) жизнь - самый талантливый сценарист @igg{fbicon.wink} 

\iusr{Alik Perlov}
\textbf{Ирина Петрова} Случаи и совпадения!! Один из моих любимых Французских фильмов! Рекомендую к просмотру. Готовый сценарий)))) добавить , Ёлку на балконе , и всё!!! Готово к продаже!!!

\iusr{Ирина Петрова}
\textbf{Alik Perlov} благодарю, мон ами, ты всегда найдешь нужные слова! @igg{fbicon.face.grinning.squinting} 

\iusr{Alik Perlov}
\textbf{Ирина Петрова} Петя , писатель, не читатель @igg{fbicon.face.tears.of.joy}{repeat=2} 
\end{itemize} % }

\iusr{Oleksandr Dobrovolskyi}
Советское падение.  @igg{fbicon.grin} 

\iusr{Michael Guchlerner}

Трогательная история @igg{fbicon.face.sleepy} 
Пишу со знанием дела  @igg{fbicon.wink}  (ВК-12 - ВК-52 1975-1980)...

\begin{itemize} % {
\iusr{Ирина Петрова}
\textbf{Michael Guchlerner} ой, да мы ж просто родственники!!!! Вооот - это уже не совпадение, а судьба!!!!

\iusr{Michael Guchlerner}
\textbf{Ирина Петрова} это - точно! Пятихатки и Березань для меня не пустые слова.
И жену я тоже с АСП нашёл  @igg{fbicon.face.grinning.big.eyes} 

\iusr{Ірина Грушецька}
\textbf{Michael Guchlerner} практика в Пятихатках проходила под лозунгом « землемеры пьют без меры» @igg{fbicon.grin}{repeat=3} 

\iusr{Ирина Петрова}
\textbf{Michael Guchlerner} я ж и говорю -" мы с тобой одной крови, мой серый брат!"
\end{itemize} % }

\iusr{Александра Тарнавская}
Берегись велосипедиста !  @igg{fbicon.hands.applause.yellow}{repeat=3}  @igg{fbicon.bicycle} 

\begin{itemize} % {
\iusr{Ирина Петрова}
\textbf{Александра Тарнавская} ты знаешь всё! Ты помнишь всё!Скажи, есть о чем написать, правда?  @igg{fbicon.heart.eyes}  @igg{fbicon.face.blowing.kiss} 

\iusr{Ірина Грушецька}
\textbf{Ирина Петрова} пролетели годы-гады, но воспоминания остались, не вырубишь и топором @igg{fbicon.grin} 
\iusr{Александра Тарнавская}
Да уж!)

\iusr{Ирина Петрова}
\textbf{Александра Тарнавская} может, ответим Ильфу и Петрову или братьям Гримм? А? @igg{fbicon.wink}  @igg{fbicon.face.blowing.kiss}  А что? Сказки и похождения Остапа - та то семечки @igg{fbicon.wink} 

\iusr{Ірина Грушецька}
\textbf{Александра Тарнавская} и велосипед помню, и нашу уютную комнатушку, и песни под гитару... 
\enquote{и я была девушкой юной, сама не припомню когда...}

\iusr{Ірина Грушецька}
\textbf{Александра Тарнавская} а помнишь тот замечательный плакат, который мы нарисовали и ранним утром привесили на входе в столовую? « в клубе представление идёт,собрался разряженный народ.Все одеты словно ляльки, лишь один сидит у майке» отгадайте кто такой? @igg{fbicon.face.tears.of.joy}{repeat=4} 

\iusr{Ирина Петрова}
\textbf{Ірина Грушецька} дааа... злые мы были тогда! @igg{fbicon.laugh.rolling.floor} 
\end{itemize} % }

\iusr{Ирина Иванченко}
Прелестный рассказ, я тоже верю,что случайных совпадений не бывает, мы просто порой не понимаем,куда направляет нас судьба...

\iusr{Ирина Петрова}
\textbf{Ирина Иванченко} абсолютно согласна, на  @igg{fbicon.100.percent} !

\iusr{Владимир Вашето}
ТВ ! 83-84 и 86-90

\iusr{Irina Sokol}
Как будто фильм посмотрела!

\begin{itemize} % {
\iusr{Ирина Петрова}
\textbf{Irina Sokol} дякую! На серіал не потягне, але таку невеличку картину можна @igg{fbicon.face.happy.two.hands} 

\iusr{Irina Sokol}
\textbf{Ирина Петрова} да, коротенькое кино классное. Но есть такие продюсеры что могут и на 200 серий раскрутить! @igg{fbicon.grin} 

\iusr{Sima Maliuga}
Надеялась, что сойдутся с Кисевичем! Столько неожиданных встреч !

\iusr{Ирина Петрова}
\textbf{Sima Maliuga} бывает разное, тут судьба срежессировала немного другой сценарий)))
\end{itemize} % }

\iusr{Валентина Сапун}

Только вчера, перебирая в книжном шкафу, нашла твои рассказы, ещё и подумала, с
удовольствием прочла бы что-то ещё. Сегодня прочла. Совпадение...

\begin{itemize} % {
\iusr{Ирина Петрова}
\textbf{Valentina Sapun} вот, я ж и говорю @igg{fbicon.heart.eyes}{repeat=3} 

\iusr{Валентина Сапун}
\textbf{Ирина Петрова} ты чудо

\iusr{Ирина Петрова}
\textbf{Valentina Sapun} Валюша, жаль, что это не оценил виновник мокрых подушек @igg{fbicon.face.grinning.squinting} 

\iusr{Валентина Сапун}
\textbf{Ирина Петрова} Бог с ним, не достоин такой умницы, красавицы. Не жалей.
\end{itemize} % }

\iusr{Ирина Лобанова}
Пишите, захватывает с первого предложения.

\ifcmt
  ig https://scontent-frx5-2.xx.fbcdn.net/v/t39.1997-6/s168x128/118211260_659965514618313_1559572347261946985_n.png?_nc_cat=1&ccb=1-5&_nc_sid=ac3552&_nc_ohc=MPctuvpftR8AX_GIkog&_nc_ht=scontent-frx5-2.xx&oh=5bf17ad033f384dd9eed4763586c69f8&oe=61946637
  @width 0.1
\fi

\begin{itemize} % {
\iusr{Ирина Петрова}
\textbf{Ирина Николаевна Романык} благодарю! Как говорит мой брат, профессиональный литератор, если придумано название, первая и последняя фраза - остальное покатит  @igg{fbicon.face.relieved} 
\end{itemize} % }

\iusr{Ірина Коваленко}
Про КИСИ все правда, очень приятно было читать.  @igg{fbicon.face.smiling.eyes.smiling} 

\begin{itemize} % {
\iusr{Ирина Петрова}
\textbf{Ірина Коваленко} мы все из одной стаи!!!

\iusr{Ірина Коваленко}

\ifcmt
  ig https://scontent-frx5-2.xx.fbcdn.net/v/t39.1997-6/s168x128/106372140_1000683287054042_6738390578369216151_n.png?_nc_cat=1&ccb=1-5&_nc_sid=ac3552&_nc_ohc=hqBgUG2iJWQAX_mMcIx&_nc_ht=scontent-frx5-2.xx&oh=d09f64c6af721ce9282f461f5ceb1025&oe=61954366
  @width 0.1
\fi

\iusr{Ірина Коваленко}
ВК, 1981-1986.

\end{itemize} % }

\iusr{Ирина Грень}

КИСИ, сан-тех, ТВ 1. Пятихатки после 1 курса колхоз в Херсоне, Джанкой... 4
общежитие на ул. Просвещения... выпуск 1984... одногрупники отзовитесь!

\begin{itemize} % {
\iusr{Nataliya Borodina}
\textbf{Irina Gren} Я тоже училась на сантехе но на вк в те же годы.. выпуск 1984

\iusr{Ирина Грень}
\textbf{Nataliya Borodina} знакомые все лица. Совпадения?!
\end{itemize} % }

\iusr{Пани Елена}
Как красиво))))

\iusr{Ирина Петрова}
\textbf{Пани Елена} благодарю, душа моя!

\iusr{Olena Ivanenko}
Лихо! От судьбы не уйдешь...

\iusr{Ирина Петрова}
\textbf{Olena Ivanenko} так ото ж... и не обскачешь @igg{fbicon.wink} 

\iusr{Татьяна Сирота}
Прочла на одном дыхании! @igg{fbicon.hands.applause.yellow}{repeat=3} 

\iusr{Николай Мелешко}
Как-то классно написано. Здорово.

\iusr{Ирина Петрова}
\textbf{Николай Мелешко} дякую!

\iusr{Ирина Архипович}
Захватывающие совпадения!!!  @igg{fbicon.hand.ok}  @igg{fbicon.thumb.up.yellow}  @igg{fbicon.heart.eyes} 

\begin{itemize} % {
\iusr{Ирина Петрова}
\textbf{Irina Arhipovich} вот так бывает @igg{fbicon.face.happy.two.hands} 

\iusr{Ирина Архипович}
\textbf{Ирина Петрова} Да, очень интересно!! @igg{fbicon.heart.eyes} 
\end{itemize} % }

\iusr{Nataliya Borodina}
Киси...вк 11...51...1979....1984 год...пятихатки незабываемы

\iusr{Olena Savych}
Чудово!

\iusr{Вікторія Святненко}
Замечательная история...
Я когда поступала в КИСИ меня уговаривали поступать на сантехнический, а я упёрлась и только на АСУ, - первый набор!! Уууу как интересно и ново)))
И КВН, и Березань...
"Сов падения")))))@igg{fbicon.heart.suit}{repeat=3}

\begin{itemize} % {
\iusr{Michael Guchlerner}
\textbf{Вікторія Святненко} разве в 1977 был первый набор на АСУ?
Я, когда в КИСИ поступал в 1975, уже об АСУ подумывал. Но, "блат" сказал, что АСУ не может. Так я стал на 5 лет "сантехником"...

\iusr{Вікторія Святненко}
\textbf{Michael Guchlerner} можете и ошибаюсь, утверждать не буду, может, действительно, второй или третий набор, это тогда было в новинку))
\end{itemize} % }

\iusr{Марина Ермилова}

Я правильно поняла, что эта Леночка, врач с редкой красивой фамилией -
очередная подруга ходока?

\begin{itemize} % {
\iusr{Ирина Петрова}
\textbf{Марина Ермилова} нет, нет, что Вы! Это замечательный доктор, спасающая людей! Не дай Бог никому видеть пациентов этого отделения! У Юрия остались там просто знакомые, как у каждого человека, я тоже знала многих сотрудников мужа, например. У Доктора прекрасная семья, и к Юрию это не имеет никакого отношения.
\end{itemize} % }

\iusr{Марина Ермилова}
А вообще, гад он.

\begin{itemize} % {
\iusr{Ирина Петрова}
\textbf{Марина Ермилова} 

знаете, с высоты прожитых лет я усвоила , что порой то, что кажется сейчас вот
таким плохим, очень-очень плохим, неудачным, ой-ой-ой, потом оказывается
правильным. Конечно, это не относится к смерти близких людей. А все эти бытовые
жизненные события регулируются к нашему же благу. Во всяком случае, у меня,
например, так оно и происходит в жизни. Конечно, надо немало пережить, чтобы
иметь "статистику", но, вот у меня такой вывод. Мнение субъективное, не
претендующее на всеохватывающее правило.) Да нет, он не гад, он просто
самый-самый обычный мужчина, каких множество.

\end{itemize} % }

\iusr{Валентина Нагорна}
Чудова розповідь! Пишіть! Читається легко! В житті все буває!

\ifcmt
  ig https://scontent-frx5-2.xx.fbcdn.net/v/t39.1997-6/s480x480/71072682_381206996150002_1108438717300736000_n.png?_nc_cat=1&ccb=1-5&_nc_sid=0572db&_nc_ohc=-RppxoobRx4AX_hBLJd&_nc_ht=scontent-frx5-2.xx&oh=089ddfb140afa0c5d46d83d73c1fa363&oe=6194E21E
  @width 0.3
\fi

\begin{itemize} % {
\iusr{Ирина Петрова}
\textbf{Валентина Нагорна} дякую, життя - то ще той Майстер на вигадки!
\end{itemize} % }

\iusr{Виктория Ушакова}
Спасибо, я давно такого удовольствия от чтения не получала)

\begin{itemize} % {
\iusr{Ирина Петрова}
\textbf{Виктория Ушакова} 

спасибо - что ещё надо тщеславному сердцу автора, как не похвала читателя!
@igg{fbicon.face.happy.two.hands} 

\end{itemize} % }

\iusr{Olga Lazorenko}
Как всё знакомо...

\iusr{Ірина Оснач}
Ариша, а что Кися, где Юра теперь?

\begin{itemize} % {
\iusr{Ирина Петрова}
\textbf{Ірина Оснач} у всіх учасників історії все склалось по життю, дай Боже здоров'я всім! @igg{fbicon.wink} 

\iusr{Ірина Оснач}
 @igg{fbicon.smile} 
\end{itemize} % }

\iusr{Luda Abramova}

Пишется может и не очень легко, но читается легко и на одном дыхании! Несколько
минут, а какой промежуток жизни, может и не совсем простой, но описан легко и с
улыбкой!

\ifcmt
  ig https://scontent-frx5-2.xx.fbcdn.net/v/t39.1997-6/s168x128/11891339_897114570361979_1916032859_n.png?_nc_cat=1&ccb=1-5&_nc_sid=ac3552&_nc_ohc=8fhVAtEOhkMAX9WgErJ&_nc_ht=scontent-frx5-2.xx&oh=6fc5e9276538d17029e6146d83df1560&oe=61937632
  @width 0.1
\fi

\begin{itemize} % {
\iusr{Ирина Петрова}
\textbf{Luda Abramova} 

дякую, сонечко, а чи бувають ті, "прості" проміжки життя?  @igg{fbicon.wink}  Воно ж таке, примхливе @igg{fbicon.wink} 

\iusr{Luda Abramova}
\textbf{Ирина Петрова}

\ifcmt
  ig https://scontent-frt3-2.xx.fbcdn.net/v/t1.6435-9/144624891_1350193178679674_804507321720680880_n.jpg?_nc_cat=103&ccb=1-5&_nc_sid=dbeb18&_nc_ohc=aCgmWMEIToIAX9q1Wr5&_nc_ht=scontent-frt3-2.xx&oh=6059e89b81b66d54bbd259a8f131ac95&oe=61B46465
  @width 0.3
\fi

\end{itemize} % }

\iusr{Ирина Петрова}
\textbf{Frantseva Elena} дуже радію, що сподобалось @igg{fbicon.wink} 

\iusr{Елена Францева}
Ирина Петрова, да, очень!) Спасибо!) Вы молодец!!!)

\iusr{Елена Францева}

\ifcmt
  ig https://scontent-frt3-1.xx.fbcdn.net/v/t39.1997-6/s480x480/16686650_181834648976184_83008442087768064_n.png?_nc_cat=106&ccb=1-5&_nc_sid=0572db&_nc_ohc=dE6uimoEWBQAX8_lrpa&_nc_ht=scontent-frt3-1.xx&oh=b5bbe7029c551bdc4c784c855c9ad69d&oe=61949B92
  @width 0.3
\fi

\iusr{Юлія Качура}
Завязка, кульминация, хочется развязки - хэппи-энда... а вместо него что-то слегка щемящее...

\iusr{Ирина Петрова}
Всё, как в жизни... @igg{fbicon.wink} 

\begin{itemize} % {
\iusr{Тамара Градобоева}
\textbf{Ирина Петрова} Ирочка, даже я помню Спасевича, После Нового года, то ли 1980, то ли 1979, когда мы с Карлушей были у тебя в гостях

\iusr{Ирина Петрова}
\textbf{Тамара Градобоева} да, мы встречали 1980 , и не только с Карлушей @igg{fbicon.laugh.rolling.floor}  @igg{fbicon.wink}{repeat=2}  @igg{fbicon.laugh.rolling.floor}  @igg{fbicon.heart.eyes}  @igg{fbicon.face.eyes.star} 

\iusr{Людмила Федорова}
Какой чудесный рассказ! Спасибо! Вот Вы Петя, Петюня, а я была Федя и Федюня. Ох уж эти университеты @igg{fbicon.face.smiling.eyes.smiling} 
\end{itemize} % }

\iusr{Александр Молибог}

Пишет Ира Молибог. ещё не прочла этот рассказ, но уже поставила лайк, и пишу
этот коммент, впервые за многие годы, буду целый день работать с радостным
чувством: меня ждёт прекрасное чтение .. до встречи: Как причудливо порой.

\begin{itemize} % {
\iusr{Ирина Петрова}
\textbf{Александр Молибог} Ира, тёзка, хорошего дня и нескучного вечера! @igg{fbicon.face.smiling.eyes.smiling} 
\end{itemize} % }

\iusr{Александр Молибог}
\textbf{Ирина Петрова}, 

Пишет Ира Молибог. вечер 21 час. 35 мин. 25. 06. 2021 год Прочла. И все
комментарии. Великолепно, пишите, Бог одарил Вас талантом. Очень хорошее
послевкусие в душе осталось. Спасибо " Киевские истории"

\begin{itemize} % {
\iusr{Ирина Петрова}
\textbf{Александр Молибог} спасибо, так приятно  @igg{fbicon.face.happy.two.hands}  талант - это немго громко, просто могу описать то, что вижу, слышу. И поделиться с людьми.
\end{itemize} % }

\end{itemize} % }
