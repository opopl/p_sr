% vim: keymap=russian-jcukenwin
%%beginhead 
 
%%file 18_11_2021.fb.bilchenko_evgenia.3.russkij_logos_citaty
%%parent 18_11_2021
 
%%url https://www.facebook.com/yevzhik/posts/4453474928020916
 
%%author_id bilchenko_evgenia
%%date 
 
%%tags bilchenko_evgenia,chelovek,ljubov,logos,poezia,rusmir
%%title БЖ. Русский Логос в цитатах
 
%%endhead 
 
\subsection{БЖ. Русский Логос в цитатах}
\label{sec:18_11_2021.fb.bilchenko_evgenia.3.russkij_logos_citaty}
 
\Purl{https://www.facebook.com/yevzhik/posts/4453474928020916}
\ifcmt
 author_begin
   author_id bilchenko_evgenia
 author_end
\fi

БЖ. Русский Логос в цитатах

Когда мои стихи о русском Логосе вызывают у реакционных постмодернистов
приступы деменции образца 1990-х, хочется говорить не от имени своего грешного
"Я", а от я-капли в дождепаде-океане гениев:

Ольга Берггольц:

\begin{multicols}{2}
\obeycr
От сердца к сердцу
От сердца к сердцу. Только этот путь
я выбрала тебе. Он прям и страшен.
\smallskip
Стремителен. С него не повернуть.
Он виден всем и славой не украшен.
Я говорю за всех, кто здесь погиб.
В моих стихах глухие их шаги,
их вечное и жаркое дыханье.
\smallskip
Я говорю за всех, кто здесь живет,
кто проходил огонь, и смерть, и лед,
я говорю, как плоть твоя, народ,
по праву разделенного страданья...
И вот я становлюся многоликой,
и многодушной, и многоязыкой.
\smallskip
Но мне же суждено самой собой
остаться в разных обликах и душах,
и в чьем-то горе, в радости чужой
свой тайный стон и тайный шепот слушать
и знать, что ничего не утаишь...
\smallskip
Все слышат всё, до скрытого рыданья...
И друг придет с ненужным состраданьем,
и посмеются недруги мои.
\smallskip
Пусть будет так. Я не могу иначе.
Не ты ли учишь, Родина, опять:
не брать, не ждать и не просить подачек
за счастие творить и отдавать.
\smallskip
...И вновь я вижу все твои приметы,
бессмертный твой, кровавый, горький зной,
сорок второй, неистовое лето
и все живое, вставшее стеной
на бой со смертью...
\restorecr
\end{multicols}


\ifcmt
  tab_begin cols=4

     pic https://scontent-frt3-1.xx.fbcdn.net/v/t39.30808-6/258435254_4453475154687560_5949107304731894509_n.jpg?_nc_cat=106&ccb=1-5&_nc_sid=8bfeb9&_nc_ohc=3fnqMzYb01AAX9EvlVI&_nc_ht=scontent-frt3-1.xx&oh=43ae0e3731a1706919cf3b67e6657048&oe=619CB222

     pic https://scontent-frt3-2.xx.fbcdn.net/v/t39.30808-6/258723181_4453475464687529_8366223727562038433_n.jpg?_nc_cat=101&ccb=1-5&_nc_sid=8bfeb9&_nc_ohc=25IGgvbsOecAX_MCRKi&tn=lCYVFeHcTIAFcAzi&_nc_ht=scontent-frt3-2.xx&oh=7c5ae9929e91fddcae421127e130e1ce&oe=619C8513

		 pic https://scontent-frx5-1.xx.fbcdn.net/v/t39.30808-6/258610139_4453475591354183_5504807163868252125_n.jpg?_nc_cat=111&ccb=1-5&_nc_sid=8bfeb9&_nc_ohc=wnm0qqw3YO0AX9oH4PE&_nc_oc=AQkoa3BtS1gP7vt9XjeUzqWR72349NO1vja6gGErEBnwJvvAfKN_gh2khrc9n6Vi6pU&_nc_ht=scontent-frx5-1.xx&oh=af4d329fb07730b7bc210c8ee7cc2486&oe=619D2C78

		 pic https://scontent-frt3-2.xx.fbcdn.net/v/t39.30808-6/258850769_4453475684687507_2709217994952406555_n.jpg?_nc_cat=101&ccb=1-5&_nc_sid=8bfeb9&_nc_ohc=8GA9gcKDUnEAX-_eCXH&_nc_ht=scontent-frt3-2.xx&oh=43119e9c531559d8aede27d8e2b34781&oe=619D2B53

  tab_end
\fi

Федор Достоевский:

Потому что ты счастлив, ты хочешь, чтоб все, решительно все сделались разом
счастливыми. Тебе больно, тяжело одному быть счастливым! Потому ты хочешь
сейчас всеми силами быть достойным этого счастья.

PS. Самое смешное, что классические академические поэты, которые точно должны
знать, что такое Логос, но предпочитают говорить о нем подражательными словами
и называть его банальными именами, считают меня постмодернистом, и это потешно. 

Надо тем и другим читать Ольгу и дядю Федора. И все станет ясно и понятно.
Почему я тащусь от Логоса на языке своей эпохи и говорю о нем от своего имени.

\#урокиБЖ

\ii{18_11_2021.fb.bilchenko_evgenia.3.russkij_logos_citaty.cmt}
