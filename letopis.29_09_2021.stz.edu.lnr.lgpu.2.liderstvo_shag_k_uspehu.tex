% vim: keymap=russian-jcukenwin
%%beginhead 
 
%%file 29_09_2021.stz.edu.lnr.lgpu.2.liderstvo_shag_k_uspehu
%%parent 29_09_2021
 
%%url http://lgpu.org/news/8447-v-lgpu-sostoyalos-zakrytie-vtoroy-ochnoy-sessii-proekta-liderstvo-shag-k-uspehu.html
 
%%author_id 
%%date 
 
%%tags 
%%title В ЛГПУ состоялось закрытие второй очной сессии проекта «Лидерство – шаг к успеху»
 
%%endhead 
\subsection{В ЛГПУ состоялось закрытие второй очной сессии проекта «Лидерство – шаг к успеху»}
\label{sec:29_09_2021.stz.edu.lnr.lgpu.2.liderstvo_shag_k_uspehu}

\Purl{http://lgpu.org/news/8447-v-lgpu-sostoyalos-zakrytie-vtoroy-ochnoy-sessii-proekta-liderstvo-shag-k-uspehu.html}

В Луганском государственном педагогическом университете 29 сентября состоялось
закрытие второй очной сессии проекта «Лидерство – шаг к успеху» для обучающихся
10-11 классов.

\ii{29_09_2021.stz.edu.lnr.lgpu.2.liderstvo_shag_k_uspehu.pic.1}

Участниками проекта стали более ста учащиеся из одиннадцати территорий
Луганской Народной Республики.

\ii{29_09_2021.stz.edu.lnr.lgpu.2.liderstvo_shag_k_uspehu.pic.2}

Ребята участвовали в тренинговых занятиях мастер-класса, встретились с
руководителями Университета, пообщались со студентами, посетили  музеи ЛГПУ:
Музей истории университета, Археологический и Зоологический музей, Зимний сад.

\ii{29_09_2021.stz.edu.lnr.lgpu.2.liderstvo_shag_k_uspehu.pic.3}

\begin{zznagolos}
– Я рад, что стал участником проекта «Лидерство – шаг к успеху»! За время
посещения ЛГПУ, я смог научиться новым навыкам, узнать интересные приемы
общения с людьми, а также посетить очень интересные экскурсии по музеям и по
корпусам педагогического университета. Хотелось бы выразить благодарность всем
организаторам и настоящим профессионалам своего дела, которые поделились с нами
большим объемом знаний! – рассказал участник проекта, учащийся Краснодонской
средней школы №9.
\end{zznagolos}

«Лидерство – шаг к успеху» – социально-ориентированный проект, направленный на
креативное развитие молодого человека, площадка для творчества и
самореализации, платформа для раскрытия потенциала, получения компетенций,
связанных с коммуникацией, расширением границ и рамок мышления, формирования
лидерских качеств и поведения. Проект направлен на создание условий для
получения универсальных компетенций, связанных с критическим мышлением, умением
принимать решения в условиях многозадачности, успешно коммуницировать.

\ii{29_09_2021.stz.edu.lnr.lgpu.2.liderstvo_shag_k_uspehu.pic.4}

Цель проекта «Лидерство – шаг к успеху»: мотивация обучающихся 10-11 классов
образовательных организаций (учреждений) Луганской Народной Республики на выбор
профессии педагога; развитие лидерских качеств подростков путем повышения
коммуникационно-организационных компетенций.

Пресс-центр университета, фото Алексея Волобуева 

% empty
%\ii{29_09_2021.stz.edu.lnr.lgpu.2.liderstvo_shag_k_uspehu.pic.5}

