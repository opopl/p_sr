% vim: keymap=russian-jcukenwin
%%beginhead 
 
%%file 05_02_2022.stz.news.ua.strana.1.vtorzhenie_nyt
%%parent 05_02_2022
 
%%url https://strana.news/news/375235-rossija-zakanchivaet-podhotovku-vtorzhenija-v-ukrainu-the-new-york-times.html
 
%%author_id poluljah_natalja
%%date 
 
%%tags napadenie,rossia,ugroza,ukraina,vtorzhenie,vtorzhenie.ru2uk
%%title Россия находится на заключительном этапе подготовки вторжения в Украину - The New York Times
 
%%endhead 
 
\subsection{Россия находится на заключительном этапе подготовки вторжения в Украину - The New York Times}
\label{sec:05_02_2022.stz.news.ua.strana.1.vtorzhenie_nyt}
 
\Purl{https://strana.news/news/375235-rossija-zakanchivaet-podhotovku-vtorzhenija-v-ukrainu-the-new-york-times.html}
\ifcmt
 author_begin
   author_id poluljah_natalja
 author_end
\fi

После публикации Bloomberg о якобы начале российского вторжения в Украину, за
которое агентству пришлось извиняться, другое американское издание вновь
выпустило публикацию о скором \enquote{вторжении}.

\ifcmt
  ig https://img.strana.news/img/article/3752/rossija-zakanchivaet-podhotovku-35_main.jpeg
  @caption The New York Times пишет о завершении подготовки РФ ко вторжению. Фото: Минобороны России
  @wrap center
  @width 0.8
\fi

Газета The New York Times заявила, что Россия якобы находится на финальном
этапе подготовки к военной операции против Украины.

Об этом издание написало со ссылкой якобы на источники в украинском
командовании (при том, что министр обороны Украины отрицает угрозу вторжения
России).

Собеседники издания заявили о переброске в Крым за последние две недели около
10 тысяч российских военных и приведение некоторых размещенных на полуострове
подразделений в состояние повышенной боевой готовности. Также источники в
украинском командовании выразили обеспокоенность тем, что в Крым направлено
подразделение Росгвардии, которое, как они считают, может использоваться для
контроля захваченной территории.

У границы с Украиной за пределами Крыма, сказали опрошенные изданием аналитики,
наблюдается подготовка \enquote{как по учебнику} к военной операции: развертывание
госпиталей и систем связи, радиоэлектронного подавления, переброска авиации и
дополнительных воинских подразделений.

Закончить подготовку к военной операции, по оценкам экспертов, Россия может в
течение нескольких недель. Пока, отмечают они, у России возле границ
достаточное количество военнослужащих, но не полностью выстроены линии
снабжения и другая инфраструктура.

При этом собеседники издания как в украинском командовании, так и среди
экспертов считают, что по-прежнему непонятно, какие именно цели преследует
Россия, разворачивая войска у границы с Украиной.

Отметим, что Россия отрицает подготовку ко вторжению в Украину и называет
фейками слухи об этом, которые транслируют западные СМИ.

Украинские власти также отрицают угрозу вторжения России, обвиняя западные СМИ
в нагнетании паники.

Ранее \enquote{Страна} писала, как в немецком таблоиде Bild описали трехэтапный \enquote{план
вторжения РФ в Украину}.

Также мы приводили реакцию на это пресс-секретаря президента России Пескова и
МИД РФ.
