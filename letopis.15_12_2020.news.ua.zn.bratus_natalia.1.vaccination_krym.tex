% vim: keymap=russian-jcukenwin
%%beginhead 
 
%%file 15_12_2020.news.ua.zn.bratus_natalia.1.vaccination_krym
%%parent 15_12_2020
 
%%url https://zn.ua/WORLD/okkupanty-kryma-objavili-o-nachale-vaktsinatsii-ot-covid-19.html
 
%%author Братусь, Наталия
%%author_id bratus_natalia
%%author_url 
 
%%tags vaccination,krym,russia
%%title Оккупанты Крыма объявили о начале вакцинации от COVID-19
 
%%endhead 
 
\subsection{Оккупанты Крыма объявили о начале вакцинации от COVID-19}
\label{sec:15_12_2020.news.ua.zn.bratus_natalia.1.vaccination_krym}
\Purl{https://zn.ua/WORLD/okkupanty-kryma-objavili-o-nachale-vaktsinatsii-ot-covid-19.html}
\ifcmt
	author_begin
   author_id bratus_natalia
	author_end
\fi
\index[rus]{Коронавирус!Крым! вакцинация, 15.12.2020}

\textbf{На первом этапе смогут вакцинировать 500 жителей полуострова.} 

\ifcmt
  pic https://zn.ua/img/article/4395/65_main-v1608035470.jpg
  width 0.5
  fig_env wrapfigure
\fi

В оккупированном Россией Крыму объявили о начале вакцинации от коронавируса,
сообщили в пресс-службе оккупационной власти Крыма.

«Сегодня в Крыму начинается вакцинация. Количество вакцины, поступившей на
сегодняшний день в республику, позволяет обеспечить вакцинацию 500 человек.
Сегодня планируется привить 30 – 50 человек", – говорится в сообщении.

На первом этапе будут вакцинировать медработников из группы риска. В ближайшее
время в оккупированный Крым и Севастополь доставят дополнительно тысячу доз
вакцины.

Список категорий граждан, которые получат доступ к вакцине, будет расширен "по
мере поступления вакцины".

Вакцинацию будут проводить в поликлиниках.

Напомним, Украина отказалась закупать российскую вакцину от коронавируса.\Furl{https://zn.ua/UKRAINE/ukraina-ne-budet-pokupat-rossijskuju-vaktsinu-ot-covid-19-stepanov.html}
Разработка не прошла масштабного испытания, а ее эффективность и безопасность
недоказана.

Ранее сообщалось, что Россия намерена производить свою вакцину от коронавируса
"Спутник
V",\Furl{https://zn.ua/UKRAINE/rf-namerena-vypuskat-svoju-vaktsinu-sputnik-v-v-kharkove-v-ukraine-ob-etom-nicheho-ne-znajut.html}
которая не прошла необходимых сертификаций и тестирования, в Харькове на
мощностях харьковской фармкомпании "Биолек".

В то же время там отметили, что пока речь может идти о производстве для
поставок вакцины в третьи страны. В ЦОЗ говорят, что точно вакцинировать
украинцев "Спутником" не будут.
