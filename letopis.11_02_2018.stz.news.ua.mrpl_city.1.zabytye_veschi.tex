% vim: keymap=russian-jcukenwin
%%beginhead 
 
%%file 11_02_2018.stz.news.ua.mrpl_city.1.zabytye_veschi
%%parent 11_02_2018
 
%%url https://mrpl.city/blogs/view/zabytye-veshhi
 
%%author_id burov_sergij.mariupol,news.ua.mrpl_city
%%date 
 
%%tags 
%%title Забытые вещи
 
%%endhead 
 
\subsection{Забытые вещи}
\label{sec:11_02_2018.stz.news.ua.mrpl_city.1.zabytye_veschi}
 
\Purl{https://mrpl.city/blogs/view/zabytye-veshhi}
\ifcmt
 author_begin
   author_id burov_sergij.mariupol,news.ua.mrpl_city
 author_end
\fi

Дом был большой с пятью окнами по фасаду, обращенному на Торговую улицу. Три
окна освещали зал, два других – спальню. Кроме этих комнат, еще была детская,
обширная кухня-столовая с плитой, квадратным столом, за который могла усесться
обедать немалая семья, и большой славянский буфет с дверцами, украшенными
тетеревами, вырезанными из дерева, – трофеями  неведомого охотника. У каждого
отделения буфета был свой, неповторимый запах: в одном можно было почувствовать
аромат корицы, в другом – ванили, в третьем – капель датского короля
(лекарства, давно вышедшего из употребления). Самое интересное, что источников
запахов в буфете давно уже не было. Память о них хранила благородная древесина.
Еще были длинный холодный коридор, переделанный из веранды, с продолговатым
столом, предназначенным для летних трапез обитателей дома, летняя кухня,
примыкающий к ней тамбур \enquote{черного хода} и довольно глубокий погреб.

Все жилые комнаты, кроме детской, имели по две распашные филенчатые двери,
окрашенные под слоновую кость, с латунными ручками. Ручки эти перед праздниками
и накануне прихода званых гостей натирали мелом, и они ярко сверкали под лучами
солнца, пробивающимися сквозь гардины, или таинственно мерцали вечером в зыбком
пламени свечей. Такое расточительство с двумя дверными проемами на комнату
люди, знавшие дореволюционные времена по книгам да по кинофильмам, объясняли
тем, мол, когда в доме устраивалась вечеринка, распахивались все двери, и
вереница танцующих под бренчание пианино перемещалась по всему дому.  Но на
самом деле, такое планировочное решение мариупольские зодчие принимали для
того, чтобы можно было, закрывая в том или ином сочетании двери, создавать
изолированные пространства, - как бы отдельные квартиры. Тем более что в
подобных мариупольских домах было два входа: парадный и \enquote{черный}. В домах с
такой планировкой жили молодые семьи до той поры, пока не обретали собственное
жилье. Но все же чаще изолированные комнаты предназначались квартирантам...

Дом для детей, родившихся и вступивших в нем в сознательный возраст, был полон
неожиданных открытий. В передней над огромным сундуком была прибита вешалка для
зимней одежды. Когда однажды летом с нее сняли для просушки все пальто,
москвички и детские кожушки, обнаружилась карта мира с изображениями животных,
обитавших на том или ином континенте. Такую карту можно было рассматривать
часами. На дне сундука нашлись случайно два предмета: один (уже детям
известный) – качалка, похожая на ту, которой мама раскатывала тесто для
вареников или пирожков, другой – небольшая доска с грубыми зазубринами.
Оказалось, что это -  бабушкино приданое и называлось оно рубелем и каталкой.
Позже довелось увидеть их в действии: бабушка, орудуя ими, разглаживала белье,
хотя в доме уже давно имелись чугунные утюги. Вот что значит сила привычки.

Самым интересным был погреб. Там можно было найти, например, старый примус,
помятый самовар, неизвестно откуда попавшую в погреб шахтерскую лампу, латунные
тазы, в которых в конце лета и в начале осени денно и нощно варили варенья и
повидло, лохань из дубовых клепок, стянутых стальными обручами, - устройство
для стирки белья. А также кадки для засолки капусты, огурцов, помидоров и
других даров мариупольских огородников. В кадку, пока она была пуста, можно
было залезть и представить себе, что ты – князь Гвидон из пушкинской сказки. 

Вдруг обнаружилось, что когда-то в погреб был вход с улицы. И тому
подтверждение – дверь, засыпанная снаружи, но изнутри сохранившая на стеклах
надписи в зеркальном изображении. Приложив определенное усилие, можно было
прочесть: \enquote{Кавказскiя угощенiя. Люля-кебабъ, долма, кюфта, чуреки, буза}. Из
пяти слов, начертанных на запыленных стеклах, известным было одно – \enquote{буза}.
Бузу, слегка хмельной непрозрачный напиток, бабушка приготавливала под
праздники из пшена. Правда, и \enquote{чуреки} попадались на глаза, кажется, в
\enquote{Кавказском пленнике} Льва Толстого. Значит, в прошлом в погребе была
закусочная или как там ее называли в дореволюционный период истории Мариуполя.

Когда дети выросли и разъехались кто куда, часть дома стали сдавать
квартирантам. Одни из них жили по нескольку лет, другие становились на постой
на несколько дней. Временными жильцами были артисты, приезжавшие летом на
гастроли в наш город. Но это было еще в довоенное время. Иногда постояльцы,
покидая временное пристанище, намеренно или случайно оставляли какие-то вещи.
Вещи эти хранили годами, к ним нельзя было притрагиваться, поскольку хозяева
дома надеялись, что объявятся владельцы и забытое имущество нужно будет
возвращать. На чердаке, в углу, был обнаружен потертый светло-коричневый
саквояж, открыв его, можно было ощутить запах не выветрившихся за многие годы
лекарств. Стало быть, вещь принадлежала врачу, квартировавшему некогда в доме.
Не тот ли это врач, о котором взрослые рассказывали, что он для себя сам
приготавливал особым образом кефир?  На чердаке также был найден валик с
навитой на него бумажной лентой, на всей его длине были пробиты отверстия в
определенном порядке. Только много позже стало ясно – это деталь шарманки –
музыкального инструмента, который в наши дни можно увидеть в каком-нибудь
музее, а услышать о нем в песенке Николая Баскова. Может, в одной из комнат
дома жил когда-то шарманщик да забыл ленту с жалобной мелодией?

В коридоре, кроме упомянутого ранее стола, находился продавленный диван без
спинки, оббитый выцветшим зеленоватым барканом. Если приподнять его верхнюю
часть, то открывался ящик, заполненный нотами, присыпанными трухой из верхней
части дивана. Ноты эти были партиями для инструментов духового оркестра. Они
были рукописными, нотные знаки украшены кокетливыми росчерками и виньетками.
Каждая партия была помечена начертанными размашистым почерком непонятными тогда
словами, к примеру, \enquote{корнетъ-а-пистонъ 1-й}, \enquote{альтъ}, \enquote{баритонъ}, \enquote{флейта},
\enquote{геликонъ 2-й} и так далее. Из семейной легенды было известно, что в годы
Гражданской войны в доме был на постое капельмейстер  оркестра одного из полков
Добровольческой армии генерала Деникина. Под ударами наступавших на Мариуполь
то ли красных, то ли махновцев деникинцам пришлось срочно ретироваться из
города. С ними бежал и капельмейстер, оставивший на вечные времена ноты. Их-то
и водворили в диван на хранение.

С \enquote{музыкальной} темой связана еще одна история. В октябре сорок первого
года немцы заняли Мариуполь. Торговая улица была определена оккупационными
войсками как главная для временного расквартирования проходящих через город
германских войск. Однажды в дом вошел немецкий офицер. Он был средних лет,
небольшого роста, полноват, в пенсне, верхняя часть рукавов его мышиного цвета
мундира была обшита вертикальными полосками серебристого галуна. Не обращая
внимания на хозяев, он прошел в зал, подошел к стоявшему там пианино, деловито
открыл крышку, пробежавшись пальцами по клавишам, извлек бравурную мелодию.
Заметив на верхней крышке пианино альбом с нотами, полистал его. Остановился на
странице с нотами \enquote{Лунного вальса} Исаака Дунаевского из кинофильма
\enquote{Цирк}. Сыграл его с начала до конца и произнес \enquote{Wunderschön!}
- прекрасно. После чего офицер удалился. Через час-другой в доме появилось три
немецких солдата с такими же галунами на рукавах, как и у офицера в пенсне. Они
притащили с собой походные раскладные кровати, стянутые ремнями в рулоны
постели и духовые музыкальные инструменты. Это были музыканты военного
оркестра. Они прожили в доме не так уж долго, но успели оставить память о себе.
Они ставили свои бритвенные принадлежности на крышку пианино. Их следы на
полированной поверхности остались навсегда.

В один из первых дней после освобождения города средняя дочь хозяев привела в
дом двух пожилых учительниц, двух сестер – Полину Федоровну и Марию Федоровну.
У одной из них,  Полины Федоровны, она училась в школе. Старушки остались без
крова, их квартиру на Итальянской улице сожгли немецкие факельщики. Учительниц
поселили в спальне. Квартирную плату с них, естественно, не брали. Старшая из
них - Полина Федоровна зарабатывала на пропитание частными уроками русского
языка отстающим ученикам. Мария Федоровна устроилась в школу поваров, где
преподавала кулинарию. Можно только представить, как тяжело было это делать в
городе, где скудный рацион жителей состоял в лучшем случае из тюльки и хлеба,
испеченного из смеси непросеянной ржаной и пшеничной муки. Года через два Мария
Федоровна тихо умерла: легла спать, а утром не проснулась. Полина Федоровна
осталась одна. Она на глазах дряхлела. Неожиданно отыскался ее сын, бывший
офицер белой армии, оказавшийся в Чехословакии. Он окончил в Праге высшее
техническое училище, стал инженером. Он вызвал мать к себе, ей отказали в
выезде из СССР. Несмотря на уговоры и обещания поддерживать ее, она решила
уехать к дальней родственнице в другой город. На прощание она подарила внуку
хозяев дореволюционную цветную открытку с изображением Царь-пушки в Кремле.

В каждой комнате дома было по иконе: в зале – Троица, в спальне – Богородица с
Младенцем, в детской – святой Пантелеймон-целитель, в столовой – суровый лик
Николая Чудотворца. Однажды один из зятьев, коммунист и ответственный работник,
переминаясь с ноги на ногу, попросил бабушку снять иконы: \enquote{Не могу я жить в
доме, где в каждом углу иконы, поймите меня правильно}. Бабушка поняла просьбу
\enquote{правильно}. Иконы были спрятаны. Постепенно они были забыты.

Стремительно меняется мир вещей. Трудно объяснить сейчас молодому человеку, как
использовались рубель и каталка, счеты, логарифмическая линейка и арифмометр.
Что такое чернильница-непроливайка, примус и керогаз, патефон, щипцы для колки
сахара, устройства для набивки табаком папиросных гильз. За считаные годы
практически вышли из употребления пишущие машинки. Все это тоже забытые вещи.
