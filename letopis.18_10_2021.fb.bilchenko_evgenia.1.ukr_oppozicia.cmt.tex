% vim: keymap=russian-jcukenwin
%%beginhead 
 
%%file 18_10_2021.fb.bilchenko_evgenia.1.ukr_oppozicia.cmt
%%parent 18_10_2021.fb.bilchenko_evgenia.1.ukr_oppozicia
 
%%url 
 
%%author_id 
%%date 
 
%%tags 
%%title 
 
%%endhead 
\subsubsection{Коментарі}
\label{sec:18_10_2021.fb.bilchenko_evgenia.1.ukr_oppozicia.cmt}

\begin{itemize} % {
\iusr{Родион Лютый}
Главный враг Украины - это националисты. Это потерянное разумом стадо, готово всю страну залить кровью, ради своих конченных идей.

\begin{itemize} % {
\iusr{Евгения Бильченко}
\textbf{Родион Лютый} ну, я сказала, сама Украина, потому что... Это трудно объяснить, и ужасно не хочется портить утро. Массовый психоз - это чем лечить?

\iusr{Евгения Бильченко}
\textbf{Родион Лютый} чем лечить русскоязычных русофобов? Тетенек возраста моей мамы... Вот это вот всё. Оппозицию типа, такую же?

\iusr{Алексей Бажан}
\textbf{Евгения Бильченко} Правильный вопрос - это баг или фича? Начиная с Петра Бернгардовича Струве русская общественная мысль отвечает - фича.

\iusr{Евгения Бильченко}
\textbf{Алексей Бажан} если не трудно, объясни, я не поняла.

\iusr{Алексей Бажан}
\textbf{Евгения Бильченко} 

Это и в книжке Тесли есть, кстати: "В данном случае необходимо остановиться на
природе противостояния — как отмечает А.И. Миллер в ставшей уже классической
монографии по истории «украинского вопроса», «<…> восприятие украинского и
белорусского, в той мере, в какой последнее проявляло себя, национальных
движений в корне отличалось от восприятия других национальных движений в
империи. Борьба с другими национальными движениями была борьбой за сохранение
целостности империи. Борьба же с украинским движением непосредственно касалась
еще и вопроса о целостности русского народа (для тех, кто верил, что триединая
русская нация уже существует) или о том, какие территории и какое население
составят то ядро империи, которое предстояло консолидировать в русскую нацию
(для тех, кто отдавал себе отчет в том, что большая русская нация представляет
собой только проект)». Иначе говоря, данные национальные проекты оказывались
непосредственно конкурирующими друг с другом — в своей последовательной
формулировке они не допускали компромисса, реализация одного представала как
исключающая другую."

\href{https://www.colta.ru/articles/literature/11245-vse-bogi-umerli}{%
«Все боги умерли», АНДРЕЙ ТЕСЛЯ ОБ АНТОЛОГИИ ПО «УКРАИНСКОМУ ВОПРОСУ», colta.ru%
}

\end{itemize} % }

\iusr{Тим Печеніг}

\href{https://guide-israel.ru/places/10239-ариэль/}{%
АРИЭЛЬ, Город славится своим приятным прохладным климатом, зеленью и чистым воздухом, guide-israel.ru%
}

\iusr{Тим Печеніг}
Євгенія Віталіївно! Тут в бібліотеці університеті міста Аріель є ваша книга

\iusr{Евгения Бильченко}
\textbf{Тим Печеніг} верю, я же была. Просто забыла, какая.

\end{itemize} % }
