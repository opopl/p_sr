% vim: keymap=russian-jcukenwin
%%beginhead 
 
%%file 21_09_2021.fb.nikonov_sergej.4.stil_odezhdy
%%parent 21_09_2021
 
%%url https://www.facebook.com/alexelsevier/posts/1586056595072983
 
%%author_id nikonov_sergej,bilchenko_evgenia
%%date 
 
%%tags bandera_stepan,bilchenko_evgenia,budenovka,foto,nacionalizm,odezhda,rusmir,st_peterburg,svidomia,ukraina
%%title БЖ. "Евгения Витальевна, как называется стиль, в котором вы одеваетесь?"
 
%%endhead 
 
\subsection{БЖ. \enquote{Евгения Витальевна, как называется стиль, в котором вы одеваетесь?}}
\label{sec:21_09_2021.fb.nikonov_sergej.4.stil_odezhdy}
 
\Purl{https://www.facebook.com/alexelsevier/posts/1586056595072983}
\ifcmt
 author_begin
   author_id nikonov_sergej,bilchenko_evgenia
 author_end
\fi

Теперь пост о стиле одежды поэтессы. Без моих реакций, мнений, лайков или
дислайков, комментов.

БЖ. "Евгения Витальевна, как называется стиль, в котором вы одеваетесь?"

Украинские свидомые из провинций вдохновились заявлением открытой сторонницы
Степана Бандеры, которая в качестве дополнительного аргумента против моего
пребывания в университете выдвинула тезис о неправильном стиле одежды. Меня
откровенно умилил провинциализм моих пушистых братьев и во избежание дальнейших
вопросов я решила не заморачиваться ответами в личку, а обобщить всё это в виде
небольшого мастер-класса.

\ifcmt
  ig https://scontent-yyz1-1.xx.fbcdn.net/v/t1.6435-9/242480919_1586054805073162_5933207176284516498_n.jpg?_nc_cat=107&ccb=1-5&_nc_sid=730e14&_nc_ohc=98fk9CNE-50AX90-oB9&_nc_ht=scontent-yyz1-1.xx&oh=b21e963dcf2e58906a4b78eb25949d61&oe=616D7208
  @width 0.4
  %@wrap \parpic[r]
  @wrap \InsertBoxR{0}
\fi

Сейчас я сменила очень нравящийся моему дедушке цветной стиль (дед не понимал,
что такое хиппи, ему нравились радующие глаз краски) на моногромное европейское
бохо. Во-первых, потому что надевать хиппианские вещи в эпоху власти
либерализма и new lefts - это как при Брежневе униформу носить. Слишком
консервативно и догматично.

Русский мир в моем лице предпочитает западный стиль одежды, так тоже бывает.
Если он не по карману, всегда есть более дешёвые украинские марки, например,
Garne, в трикотаже им нет равных, но одежда холодная.

Когда я иду по Невскому, я ничем не отличаюсь от других девочек, даже обидно:
там все одеты ярко, ибо столица. А в украинских провинциях тетки в вышиванках с
рынка на меня глазеют. Вышитая блуза есть у меня одна: французская, стиль а ля
рус, она в 2010 году стоила 6 тысяч гривен, подарок деда на очередное
достижение в науке. Сейчас я уже не имею таких денег. Самые лучшие худи есть
еще в фирме Street Wear, а нарядные лёгкие свитера - в United Colours of
Benetton. Знаменитые менеджеры Disigual взвинтили цены так, что их бренды я
беру в украинских провинциях, где они в три раза дешевле: местные мешканцы не
понимают расшитый в индейском индийском стиле тренд и думают, что это с цыган
на вокзале сняли. Очень хороши также классические рок-вещи в лавках Питера,
принты с Босхом, например.

Самые лучшие туфли - всех моделей и фасонов - в Берлине, но мне они не по
карману, пришлось выйти, предпочитаю кроссовки "ЭКО" из Intertop, но такое
счастье с моими доходами - раз в семь лет. В Польше фирма DeeZee произвела
неплохие толстовки, недавно порадовал даже Reserved, они перестали быть
попсовыми. Vera Moda - лучшие свитера на зиму. Моему - 25 лет, ношу с
пятнадцати, вид тот же, шерсть 100 \%.

И, конечно, Русская ярмарка мастеров, блестящий hand made. Но там очень дорого,
я не решилась. Если вы предпочитаете украинскую одежду в стиле армовира, хотя
бы не носите вышиванку с джинсами - это пошлятина. Не носите вышиванку с
аппликациями - это китч и ужас. Стильно - вышитое белое на белом с
многослойными кораллами и юбкой в пол, слегка с румынским отливом, со скифскими
вставками, магазин "Мрії Марії" на Андреевском, - там отличная этника, но меня
сейчас ни этника, ни богема не вдохновляют, только рэп, милитари и бохо - под
настроения "отвали", "можем повторить" или "обними" соответственно (день на
день не приходится). Хорошая эклектика тоже должна быть не салатом, а
насыщенным супом: гармонией или нарочитой дисгармонией элементов, как в
архитектуре сецессион.

Вот куда я с удовольствием надеваю косынку и бабушкину юбку в пол - это
Церковь. А костюм я берегу только для СПбГУ. Чтобы я такое надела, нужно, чтобы
я преклонялись перед местом, куда я иду в патриархальном или в деловом.
Искренно и честно преклонялась. Пока это - только Церковь и СПбГУ.

Что носим этой осенью? Вот могу посоветовать то, что носят в Европе сейчас,
продается через Россию, там больше новинок, мне очень нравится кофточка Lava
Lamp Spilt Multicolor Cardigan за \$28.00:\par
\url{shop.shopcider.com/products/lava}

Но это чудо - дорого для меня, больной и безработной, тем более то же самое,
можно нарыть в стоке, сэконде (лучший сэконд видела я в Гермерсхайме) или в
более скромных аналогах Украины или Польши, Россия сейчас тоже - дорога
по-европейски и так же изящна, молодец, выбор для принцесс и воинов одинаково
изобретателен, моден и крут.

Надеюсь, у сознательных украинцев больше не возникнет вопросов относительно
моего стиля. Вкуса вам, дорогие banderas! Столичного, имперского, да. А то в
Мариинку или в Венскую оперу так и придёте в аппликациях с "трубы" (переход на
майдане) и в юбке времён генсека Черненко.

Илл.: фотоколлаж сторонника и читателя из России.

\begin{itemize} % {
\iusr{Светлана Пикта}
Поэтессы такие модницы
\end{itemize} % }
