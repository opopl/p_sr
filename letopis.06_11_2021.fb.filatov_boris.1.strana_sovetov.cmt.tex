% vim: keymap=russian-jcukenwin
%%beginhead 
 
%%file 06_11_2021.fb.filatov_boris.1.strana_sovetov.cmt
%%parent 06_11_2021.fb.filatov_boris.1.strana_sovetov
 
%%url 
 
%%author_id 
%%date 
 
%%tags 
%%title 
 
%%endhead 
\subsubsection{Коментарі}

\begin{itemize} % {
\iusr{Мария Пашкурова-Петренко}

Борис Альбертович, сама не даю непрошеных советов и никогда их не слушаю от других.
Также не очень люблю просить о помощи.. но сейчас без Вас не получается. Зайдите, в личку, пожалуйста. Очень нужна Ваша помощь

%\begin{itemize} % {
%\iusr{Borys Filatov}
%\textbf{Мария Пашкурова-Петренко} ага, вижу.
%Я супругу напишу сча Вашему

%\iusr{Мария Пашкурова-Петренко}
%\textbf{Borys Filatov} , спасибо


%\iusr{Borys Filatov}
%\textbf{Мария Пашкурова-Петренко} уже, пусть закинет свой телефон. Его юристы наберут в понедельник

%\iusr{Мария Пашкурова-Петренко}
%\textbf{Borys Filatov} , благодарю  @igg{fbicon.hands.pray}  всё сделаем

%\iusr{Оксана Червинская}
%\textbf{Borys Filatov} Круто, когда так просто, на прямую можно обратиться за помощью к Мэру!  @igg{fbicon.thumb.up.yellow} 
%\end{itemize} % }

\iusr{Katerina Marhitich-Filippova}

Ооо! Я очень люблю это свойство наших людей.

И, если в офлайне люди ещё хоть немного фильтруют порывы, то в онлайне полный
разброд. Ибо экран даёт ощущение безопасности и непосланности в глаза.

Очень любят прийти на страницу к чужому человеку и рассказать, как ты не так
что-то делаешь, или писать в личку и нарезать задачи, ибо переживают и они
неравнодушны.

Связываю это с тем, что нет а) понятия, что такое личные границы, и что давание
советов без спроса - это некорректно и б) у нас нет культуры в соцсетях.
Точнее, никто не задумывается о том, что она есть и ошибочно считают, если есть
мысль, то нужно всенепременно её выложить человеку. Это как прийти в чужой дом
[личную страницу] и рассказывать хозяину, что у него шторы зелёного цвета
плохие и батарею ему следует заменить и вообще, как фигово он сделал ремонт у
себя в квартире, надо бы переделать.

Но, когда человек отодвигает все вот эти порывы - таки советчики обижаются
искренне, называют хозяина хамом и не понимают, что это они сами попросту
бестактны.

\begin{itemize} % {
\iusr{Borys Filatov}
\textbf{Katerina Marhitich-Filippova} 

А ещё есть так называемые «волонтёры», которые думают, что сделав спам-рассылку
чиновникам/бизнесменам/их жёнам - они спасли ребёнка и очистили свою совесть.

Приятно заниматься добрыми делами не отрываясь от своего смартфона и тегая всех
подряд.

\iusr{Татьяна Николаевна Манько}

Я совершенно согласна с нашим МЭРОМ, ЧТО КАЖДЫЙ ГРАЖДАНИН В НАШЕЙ СТРАНЕ ПРЕЗИДЕНТ))

\iusr{Denys Shapkun}
\textbf{Katerina Marhitich-Filippova} 

только страница в соцсетях это совсем не личный дом и не сцена, с которой можно
вещать все что угодно под аплодисменты @igg{fbicon.wink} 

Это скорее большой зал с кучей круглых столов, где каждый может говорить, что
хочет. Не нравится чей-то коммент, давай бан, пусть за твоим столом будут
только приятные тебе отзывы или игнорь, не нравится чей-то стол, встал и вышел.
Но таких столов миллионы, где-то есть правила, где-то говорят что хотят. Это
всего лишь личная точка зрения как на это реагировать, обижаться, банить,
игнорить или посылать. Заставить всех мило улыбаться, хвалить и лайкать не
получится @igg{fbicon.wink} 

\iusr{Katerina Marhitich-Filippova}
\textbf{Denys Shapkun} 100\% согласно с тобой.

Я сейчас попробую конкретизировать, что именно я имею в виду:

Личная страница человека - это все таки личная страница человека, и, даже не
смотря на то, что это интернет и все публично, но правила культурного обмена
мнениями и корректной формы никто не отменял.

Когда я к кому-то прихожу на страницу почитать, то всегда помню о том, что это
его территория. И я могу быть согласна с мнением автора, а могу быть и не
согласна. И это ОК.

Но я всегда топлю за то, что при выражении несогласия или желанием поучить /
дать совет всегда нужно задать себе вопрос \enquote{а сказал бы я в такой формулировке
этому человеку лично в глаза?}. Потому что зачастую такие вот формы несогласия
в комментариях несут в себе очень неприятный характер и от них веет
\enquote{навязываением} и очень похоже на \enquote{высер}, типа \enquote{а баба Яга против, я щас вам
тут расскажу, как жить надо}, \enquote{это же интернет, ачотакова?}.

И вот как раз это бесит: интернет порождает ощущение вседозволенности, потому
что человек в глаза так никогда не скажет, ибо есть риск быть посланным
подальше, а то и больше.

Короче, я за культурный обмен мнениями за круглыми столами... и за бан за
грубости )) Вот )))

\iusr{Katerina Marhitich-Filippova}
\textbf{Denys Shapkun} 

а про непрошенные советы - это вообще отдельная тема.  @igg{fbicon.face.grinning.squinting}  Мы так к ним привыкли
с детства, что не отсекаем, что это проламыание чужих границ и агрессия.

Никогда не даю советы, если меня о них не просят. По крайне мере, отбиваю себе
руки [язык] за малейшее желание его дать ибо \enquote{у меня же жопыт}. Советы надо
раздавать, когда о них просят конкретно вас.

Это мое твердое убеждение по жизни.

\iusr{Denys Shapkun}
\textbf{Katerina Marhitich-Filippova} 

согласен, что хамство, агрессия, оскорбления неприемлемы и логично могут
вызвать бан. Но чем острее тема или публичнее личность, тем больше будет
комментариев или отзывов, которые автору могут не понравиться.

А разрешения или приглашения для таких комментов особо и не нужно, достаточно
просто сделать пост публичным, открытым для комментов.

Я так это вижу и поэтому немного удивляюсь, когда кто-то негодует непрошенным
комментариям) твоё уточнение мне понятно на 100\%)

\iusr{Ольга Хрусталёва}
\textbf{Borys Filatov} 

А с другой стороны, получая звоночки от \enquote{неравнодушных}, получишь объемную
картину проблем, волнующих сограждан. А если руководствоваться только докладами
подчиненных... они вас убаюкают рассказами о всеобщем благоденствии и
обожании... У палки два конца...

\end{itemize} % }

\iusr{Игорь Рыжков}

Не вижу ничего плохого в активности неравнодушных людей. Мудрый человек найдёт
зерно в любом совете, даже если воспользоваться им не собирается. А «клеймить»
незнакомых неравнодушных к Вам людей - прямая «расчищенная» дорога к культу.

Обидно за тех людей, которые искренне пытались поделиться с Вами своими
мыслями, а попали за это в Ваш список «неадекватных». Так Вы лишь отталкиваете
от себя активных и неравнодушных граждан.

И да... Это мой первый и последний коммент на Вашей страничке... А то и я попаду в
«неадекваты»...

\begin{itemize} % {
\iusr{Borys Filatov}
\textbf{Игорь Рыжков} 

Знаете, уважаемый, в чем у Вас и Вам подобных, основная проблема?

В том, что вы, обидчивые наши, читаете только там где вам выгодно, где не колет
глаз, где не тычут носом в собственные недостатки.

Вам пишут про социальный феномен, а вы видите культ.

Вам говорят, что мысли/мнения/дискуссия - это одно, а навязчивые советы
незнакомым людям - это другое. Но вы сразу обижаетесь и записываете себя в
«неадекваты».

Научитесь ценить чужое личное пространство даже больших начальников и/или
публичных людей и будет вам счастье.

Ну и чувства юмора Вам тоже бы не помешало, уважаемый профессор.

Хоть чуть-чуть.

\ifcmt
  ig https://scontent-frx5-1.xx.fbcdn.net/v/t39.30808-6/250946845_4510158149066047_2157942217573315440_n.jpg?_nc_cat=110&ccb=1-5&_nc_sid=dbeb18&_nc_ohc=ubd8iaGQiKsAX-PSB2v&_nc_ht=scontent-frx5-1.xx&oh=f4474a970be14d55dd17e0edc09f353b&oe=61A772E3
  @width 0.4
\fi

\iusr{Tala V Kaii}
\textbf{Игорь Рыжков}, а Вы доктор каких наук? Что-то я, наверное, пропустила

\ifcmt
  ig https://scontent-frt3-1.xx.fbcdn.net/v/t39.30808-6/253870643_2824681391085868_64457220285377735_n.jpg?_nc_cat=104&ccb=1-5&_nc_sid=dbeb18&_nc_ohc=pJ98613f-6UAX_ggUBQ&_nc_oc=AQkzypSaHMzfOrYYfY2HIx0F_ZU8CaoOJRmWhAJ4Yzy8tZt6xdGoghT2fUtwtQTwQrE&_nc_ht=scontent-frt3-1.xx&oh=0de436fd9e2b384408dbd324e60b467c&oe=61A91CE6
  @width 0.4
\fi

\iusr{Игорь Рыжков}
\textbf{Tala V Kaii}  @igg{fbicon.grin} 

\iusr{Игорь Рыжков}
\textbf{Tala V Kaii} Супер! Спасибо. Исправлю.  @igg{fbicon.smile} 

\iusr{Игорь Рыжков}
\textbf{Tala V Kaii} Спасибо.  @igg{fbicon.grin}{repeat=3} 

\iusr{Алексей Барбашов}
\textbf{Borys Filatov}
В советах нету благодати
и большей частью пользы нет,
и чем дурак мудаковатей,
тем он обильней на совет.
Игорь Губерман

\iusr{Вера Козырева}
\textbf{Tala V Kaii} Уже у Игоря Рыжкова польза оказалась от комментария под постом Филатова.
Доктор каких-то наук,,проффесор ошибки исправил. @igg{fbicon.face.grinning.sweat}{repeat=3}  Благодаря Вам. Уже хорошо
Проффесор @igg{fbicon.face.grinning.sweat}{repeat=3} 

\iusr{Eugene Biryukov}
\textbf{Игорь Рыжков} 

«активность неравнодушных» должна быть подкреплена знаниями. Иначе, это не
активность, а пресловутое «возня дурня со ступою». Когда «активный» человек,
абсолютно не знающий ПДД протыкает колеса или отламывает дворники машине,
которая припаркована по правилам, но мешающая лично ему - это хорошая
«активность»? Когда человек, не знающих, что борьба с преступностью и
правонарушениями есть обязанность органов правопорядка, абсолютно не
подчиняющимся мэрам городов, требует от мэра навести порядок с наркоманами -
это хорошая «активность»? Когда человек, закончивший в лучшем случае вуз по
специальности «автоматизация машиностроения» (а то и вовсе без высшего
образования, ибо не сподобился) дает советы мэру, как управлять городом в
период пандемии нового, практически не изученного в мире вируса (хотя, спроси
их чем отличается бактерия от вируса -не скажут) - это хорошая «активность»?
Когда люди рассуждают о кипрских офшорах , куда, якобы, уходят деньги с
парковок, хотя самого смысла слова «офшор» не понимают, и дебет от кредита
отличить не могут - это нормальная «активность»? Можно продолжать долго, но
смысл вышеизложенного, в общем, один: «активность» не всегда синоним
«полезности». Активность дурака - это всего лишь дурость. И ничего другого....
И очень зря вы «не видите в этом ничего плохого».

\end{itemize} % }

\iusr{Николай Баркарь}

Скажите пожалуйста Борис Альбертович, учитывая плохую экономическую ситуацию в
\enquote{зелёной} стране, город решил выйти из \enquote{ямы} за счёт эвакуации припаркованных
автомобилей, штрафов за незначительные нарушения, и других штрафов? Может всё
таки поймём что такого в экономики нет, и направим эту армию бездельников на
более ВАЖНЫЕ ДЕЛА? К примеру на борьбу с наркоманией, бухающими прямо на
детских площадках, охраной порядка, и т.д.?

\begin{itemize} % {
\iusr{Borys Filatov}
\textbf{Николай Баркарь} 

Сообщаю Вам персонально, что в компетенцию местного самоуправления не входит
борьба с наркотиками и охрана общественного порядка.

Это - полномочия органов внутренних дел.

Позвоните по телефону 102, а не давайте советы мэру.

Тут, кстати, о таких как Вы написано, но, видать, туго доходит.

( пошёл пить успокаивающие, понимая, что на 30-м году независимости некоторые
так и не понимают, кто и за что отвечает в государстве )


\iusr{Николай Баркарь}
\textbf{Borys Filatov} 

Да всё я понимаю, достучаться ни до кого не возможно. Поэтому и приходиться всё
с Мэром обсуждать. 102, это звонок в бездну, в бесконечность. Они ничего не
решают, решают их. Полиция ограничена в действиях. Любой \enquote{гражданин},
наделённый быдлячим поведением, с лёгкостью посылает полицейского в далёкое
сексуальное путешествие, и ничего не боится. Без обид. А на счёт борьбы с
наркотиками, это зло, которое нужно обсуждать на уровне государства, так как мы
потеряли уже два поколения. Ещё одно, и всё, государство можно распускать. Не
надо оставлять такие важные вопросы на уровне заинтересованных в этом
чиновников.

\iusr{Рус Викторович}
\textbf{Borys Filatov} 

Б. А. Вы конечно избраный нашими горожанами... МНОГОЕ сделали для города... Но
иногда Ваша манера общения не всегда воспринимается людьми.... хорошо
@igg{fbicon.beaming.face.smiling.eyes}  вам как человеку государственому, не
указывают (я так думаю.) а хотят подсказать.

Как всеДнепровско избраному.. Прислушайтесь и приймите ПРАВИЛЬНОЕ решение.
С уважением.

\iusr{Олег Головко}
\textbf{Borys Filatov} успокаивающее вряд ли поможет(. Алкоголь!!!( не совет- рекомендация!))))

\iusr{Patyka Andrii}
\textbf{Nikolai Barkar} Про ПДР в Конституції нічого немає. Отже, це обмеження прав!!! За рух на дорогах без обмежень! Хай живе Конституція!!!

\iusr{Олег Демин}
\textbf{Николай Баркарь} есть выход, - перестаньте нарушать и необходимость кормить "армию бездельников" отпадет сама собой. На сэкономленные средства высадят ёлочки  @igg{fbicon.face.tears.of.joy}  но это не точно

\iusr{Олег Фрид}
\textbf{Borys Filatov} Борис Альбертович, какие функции и полномочия у Муниципальной Варты?

\iusr{Borys Filatov}
\textbf{Олег Фрид}, это - «частная армия Филатова».
Разве об этом ещё не поведали мои оппоненты и «общественники» у них на зарплате?)))
...
Ну а если серьезно, то никакой муниципальной варты не хватит навести порядок в миллионном Городе. Нужна реформа правоохранительных органов, о которой написано в программе партии «Пропозиция».

\iusr{Андрий Костенко}
\textbf{Nikolai Barkar} \url{http://krasche.a3.kyiv.ua/}

\iusr{Людмила Селютина}
\textbf{Андрий Костенко} Краще было, краще было не ходити. Краще было, краще было не кохати, чем теперь, чем теперь покидати!

\iusr{Лия Димитрова}
\textbf{Николай Баркарь} он даже не понял вашего вопроса  @igg{fbicon.face.tears.of.joy} или лихо съехал с него ...

\iusr{Ольга Хрусталёва}
\textbf{Borys Filatov} Вы видите себя городским технарем, а народ хочет видеть вас отцом города. А "отца" должны волновать не только канализация и парковки. Тяжела шапка Мономаха...

\iusr{Наталья Редько}
\textbf{Николай Баркарь} По поводу "эвакуации"- если попытаться пройти по Вернадского с коляской...то там хочется не эвакуировать, там хочется кувалдой побить)))) Давайте не будем преУМЕНЬШАТЬ то, что мешает жить обычным жителям

\end{itemize} % }

\iusr{Алла Кушнир}
когда Вы успеваете на такие post? хорошего дня).

\begin{itemize} % {
\iusr{Borys Filatov}
\textbf{Алла Кушнир} Это не я пишу, а «литературные негры»:))
И Вам хорошего дня!

\iusr{Олексій Андрусевич}
\textbf{Borys Filatov} Ви про цих написали, про яких мій діду каже :" не знаю як, але не так!!"

\iusr{Ashot Arushanov}
І не тільки написати, а ще й відповісти кожному! Альбертич, ти красава! )
Неграм привіт!
Якось вже доїду до вас, давно не бачилися. )

\iusr{Людмила Селютина}
\textbf{Олексій Андрусевич} Надо запомнить!

\iusr{Олексій Андрусевич}
\textbf{Людмила Селютина} користуйтеся на здоров'я!
\end{itemize} % }

\iusr{Елена Ничипорчик}
 @igg{fbicon.thumb.up.yellow} а что делать ему самому советчик не знает ) а что делать меру 100\% @igg{fbicon.face.tears.of.joy} 

\begin{itemize} % {
\iusr{Borys Filatov}
\textbf{Елена Ничипорчик} Есть такая хорошая фраза, которую приписывают то ли Артуру Блоху, то ли Сюзан Эрц: «О бессмертии мечтают миллионы людей - тех самых, которые мучительно думают, чем бы занять себя в дождливый воскресный вечер»

\iusr{Елена Ничипорчик}
\textbf{Borys Filatov}  @igg{fbicon.beaming.face.smiling.eyes}  @igg{fbicon.thumb.up.yellow} 

\iusr{Ирина Ренгач}
\textbf{Borys Filatov} 

Доброе утро, Борис Альбертович!

Читая комментарии к своим постам, пожалуйста, помните о том, что их пишут не
только с аккаунтов реальных людей, но и с аккаунтов технических, таких как, к
примеру, аккаунт \enquote{Андрей Воронец}. Вчера с этого аккаунта от Вас требовали
отключить в Днепре центральное отопление.

Красный цвет в аватарке и фотографии обложки профиля \enquote{Андрея Воронца}
превалирует, чтобы аватарка бросалась в глаза. Вежливость в наличии. Ваше имя и
речевой оборот \enquote{Уважаемый Борис Филатов} в первом предложении.

Не все, кто пишет на Вашей странице, делают это по своей воле. \enquote{Андрей Воронец}
зарабатывает на Вашей известности. В его аккаунте ни одной личной фотографии,
под постами нет комментариев друзей. Типичный бот.

Психолог с университетским образованием.

Пожалуйста, не берите близко к сердцу писанину продажных психологов. У каждого
своя работа. Кто-то продаётся Вашим оппонентам. Вы же понимаете, что не всех
радует Ваше развитие. В общении с подписчиками важна Ваша реакция. Потому что
Вы - реальный, успешный \enquote{двухметровый блондин с голубыми глазами}. Кто они -
неизвестно.

Берегите нервы.
Это не совет.
Это просьба.
От диаспоры:)

\iusr{Алим Галушка}
\textbf{Ирина Ренгач} А покуда разберешься-пересортируешь, где бот, а где нормальный человек, - вот день и прошел... А тут еще куча бумаг на столе... Информационный шум - беда нашего времени.

\iusr{Ирина Ренгач}
\textbf{Алим Галушка} 

Главное - не терять нервные клетки, разговаривая с неискренними людьми.

У Бориса Альбертовича есть выбор. Он может не отвечать на комментарии. От слова
\enquote{вообще}. Так поступают многие публичные личности в соцсетях. Но Борис
Альбертович гуманист. Он открыт к дискуссии и даже в ней заинтересован, как
человек, стремящийся к развитию. Личная fb-страница Филатова популярна, судя по
количеству подписчиков. Это сильная сторона ситуации. Её слабая сторона - в
часто встречающейся анонимности комментаторов.

Поэтому, прежде, чем вступать в дискуссию, будет рациональным внимательно
изучить страницу, с которой комментарий написан. И даже постараться понять
мотивы, движущие человеком, написавшим комментарий.

Чтобы не тратить время на ветряные мельницы.

Время - одна из фундаментальных ценностей жизни.

Наравне со здоровьем.

Что касается \enquote{информационного шума}, то я полностью разделяю Ваше мнение.

Это беда и метод нейролингвистического программирования масс.

\iusr{Алим Галушка}
\textbf{Ирина Ренгач} 

Насчет анонимности комментаторов, так это больше проблема фейсбука, который во
многих отношениях зашел в тупик. То, что мэр читает комменты (надеюсь, что это
таки не \enquote{литературные негры}), так это хорошо. Видимо он пытается найти здесь
крупицы разума. Но эти золотые песчинки крайне сложно \enquote{вымывать} из тонн пустой
руды.

\iusr{Алим Галушка}
\textbf{Ирина Ренгач} 

Вы меня не поняли. Или я непонятно высказался. Фейсбук должен помогать выделять
из массива комментаторов нечто содержательное и предлагать посмотреть это
интересующемуся лицу. Хотя бы как опция. Наподобие того, как Гугл находит
нужное. Но ФБ работает в этом смысле аж вообще никак - ворох хлама прямо на
голову. Кто же с таким справится.

\iusr{Ирина Ренгач}
\textbf{Алим Галушка} 

Фейсбук - платформа для обмена информацией и общения. Если Вы будете ожидать от
людей, которые Вас не любят, хорошего к Вам отношения, Вы окажетесь в
эмоциональном тупике. Каждый диалог глубоко индивидуален.

Бисер перед свиньями не рекомендовали метать ещё две тысячи лет назад. Люди,
которые прячут свои лица за политическими лозунгами, как минимум, трусливы.

\iusr{Ирина Ренгач}
\textbf{Алим Галушка}

Фейсбук развивается так, как в этом заинтересованы его владельцы. Пользователи
могут развивать креативность своих личных аккаунтов в соответствии с принятыми
владельцами платформы правилами.

Что касается комментариев, Борис Альбертович обозначил проблему очень точно.

"Страна Советов". Советы нужно давать тогда, когда о них спрашивают.

Алим. Благодарю Вас за беседу. Хороших выходных.

\end{itemize} % }

\iusr{Жанна Линник}

Советы- это не плохо. Можно их выслушивать... можно воспользоваться... можно
проигнорировать. Но только так можно услышать , что думают люди. Иначе на
страничке останутся только жополизы.... остальных забанят. Так конечно проще.

\begin{itemize} % {
\iusr{Кирилл Мартыщенко}
Если прислушиваться ко всем мыслям людей - можно разочароваться в человечестве


\iusr{Borys Filatov}
\textbf{Жанна Линник} мой Вам совет не путать мысли и советы. Советую понять разницу между тем, о чем думают люди и что они советуют.
Хороший совет, правда?

\iusr{Ольга Шиян}
\textbf{Жанна Линник} Ну вам же человек рассказал, что он слышит! Как адекватный человек может реагировать на подобный " глас народа"? Только с помощью гидазепама.

\iusr{Жанна Линник}
\textbf{Borys Filatov} в вот видите... И Вы советы даёте. Так что это нормально. Хотя заметьте... Я совета не давала .. просто мысли по этому поводу высказала...

\iusr{Жанна Линник}
\textbf{Ольга Шиян} а разве я сказала, что сомневаюсь в том, что Борис Альбертович слышит...

\iusr{Людмила Селютина}
\textbf{Жанна Линник} Больше скажу, Борис Альбертович ещё и видит!

\iusr{Жанна Линник}
\textbf{Людмила Селютина} слава Богу!
\end{itemize} % }

\iusr{Ігор Куліковський}
Мой Вам совет - исправьте нирвану на сатори, Япония же ж!
(ну это шутка, ну правда))

\begin{itemize} % {
\iusr{Borys Filatov}
\textbf{Ігор Куліковський} , не обобщай.
Это несколько разные понятия. Мой товарищ - именно в нирване:))
Читать тут:
\url{https://www.japanese-buddhism.com/nirvana-is-satori.html}
\end{itemize} % }

\iusr{Инна Патутина}
Так вы, оказывается блондин??? По фото не скажешь.

\begin{itemize} % {
\iusr{Borys Filatov}
\textbf{Инна Патутина} Да, блондин. Мышиного цвета.
И усох немного от двух метров.

\iusr{Марина Васильевна}
я тоже пошла смотреть на блондина))


\iusr{Borys Filatov}
\textbf{Марина Васильевна} Стоп, стоп, девочки, это была шутка!!!
)))

\iusr{Инна Патутина}
\textbf{Borys Filatov} вы красавчик

\iusr{Ирина Микитась}
\textbf{Borys Filatov} С людьми, которьім надо обьяснять, что Вьі пошутили, надо бьіть осторожньім.

\iusr{Ирина Терехова}
\textbf{Borys Filatov} , поздно!)))
Все роются в архивах, которые оцифрованы изначально)))
Хороших выходных!

\iusr{Людмила Селютина}
\textbf{Инна Патутина} Если партия скажет, перекраситься и вырасти до двух метров тогда уже возможно таки да!
\end{itemize} % }

\iusr{Liliia Bohacheva}

Коли не можуть керувати своїм життям, намагаються керувати життями інших.
Так було, є і буде.
Гарних Вам вихідних, Борисе Альбертовичу.

\iusr{Ірина Жосан}
Гарне у Вас, пане, почуття гумору! З задоволенням читаю Ваші пости.

\iusr{Елена Мельниченко}

С некоторых пор читаю все ваши посты. Я не политик поэтому и не лезу во всякие
дебаты. Мне плевать какая партия будет при власти. Я, как и большинство, хочу
для страны стабильности и процветания.

Но вас я просто обожаю)))

Люблю людей которые откровенны в своих эмоциях)))

Хорошего дня.

\begin{itemize} % {
\iusr{Олена Дишловенко}
\textbf{Елена Мельниченко} ,присоединяюсь!! Редкий вид Позитивного Народного Мэра!!
\end{itemize} % }

\iusr{Игорь Эйшинский}
«Лучше бы вместо этого....» С этой фразы самые ценные советы начинаются.

\iusr{Максим Золотарёв}

В таком случае ответный \enquote{приказ} @igg{fbicon.beaming.face.smiling.eyes}. 
Всегда сохраняйте позитивный настрой, не теряйте оптимизма и не обращайте внимание на идиотов.

\iusr{Irina Zemlyanskaya}

Доброе утро, Борис Альбертович! Люди так привыкли всё критиковать и давать
ненужные советы. Сил и терпения Вам, солнечной осени

\iusr{Станислав Пономаренко}

С такими рода \enquote{советчиками} с которыми у меня более менее нормальные отношения,
я задаю один вопрос:

\enquote{А какой у тебя доход} или если сложно им понять, то \enquote{сколько ты
зарабатываешь}?

После всех остальных вопросов, \enquote{а какая разница} или \enquote{ а к чему это}, я даю
ответ: \enquote{да потому что ты не имеешь права давать советы людям, которые богаче
тебя. Если у него доход больше чем у тебя, значит он умнее и знает больше.}

Как правило потом вопросов станет меньше, или они переходят в \enquote{інше русло
розмови}.

\begin{itemize} % {
\iusr{Olga Chaplinskaya}

ну, Вы утрируете! Эти богатые часто отлично иллюстрируются поговоркой
\enquote{Дурний-дурний, а хитрий!} Ученые, занимающиеся исследованиями в областях
науки, где богатенькие не смыслят ни уха, ни рыла, преподаватели ВУЗов, художники
и писатели... ряд можно продолжить, точно не глупее богатого Яныка или Загида!
Наш дорогой автор, скорее, исключение из правил, ибо у него есть 3 в
1: интеллект, образование и материальный успех !

\iusr{Olga Chaplinskaya}

когда случается некая \enquote{всемирная жопа}, за помощью бегут не к богатым,
а к умным!

\end{itemize} % }

\iusr{Олена Дишловенко}
Браво!!!

\iusr{Таня Куян}

От чтения ваших постов получаю удовольствие, меня даже не коробит от сленга и
мата (сама такая), из песни слов не выкинуть. Хорошего, позитивного дня

\iusr{Татьяна Честнейшая}
Браво

\iusr{Катрин Сорокина}
В смысле нет рубильника?  @igg{fbicon.face.flushed} 

\begin{itemize} % {
\iusr{Олег Ткач}
\textbf{Катерина Сорокіна} дійсно. Невже у Мера міста не такий кабінет @igg{fbicon.beaming.face.smiling.eyes}{repeat=2} 
\end{itemize} % }

\iusr{Антонина Кашкина}

Почему-то люди думают, что давать советы - это хорошо. Но для другого человека
непрошенный совет является грубым вторжением в личное пространство. Вот как
начать размахивать кулаками у него перед носом. Или ему на штанину нассать. Это
означает \enquote{я умный, а ты дурак, поэтому делай вот как я скажу}. Так что каждый
раз когда вы пытаетесь открыть рот и сказать подружке \enquote{а вот я бы твоего Васю
выгнала} или \enquote{в этот салат надо класть тёртый, а не резанный картофель} вы
пытаетесь нассать ей на ногу. Некрасиво, неэтично, вызывает внутреннюю
агрессию. Если совет попросили - это другое, и отношение к словам будет другое.

\iusr{Наталья Богиня}

Собаки лают, а караван идёт вперёд, вам желаю от чистого сердца температуры
36,5 !!

\iusr{Алим Галушка}

Несомненно, что в потоке советов есть ценные. Но беда в том, что такой вал
информации переварить невозможно. Принимается общее решение, продиктованное
действительным состоянием большинства советов, - все фигня... и все
выплескивается...

\iusr{Елена Ничипорчик}

А я знаю почему так ) раньше все пизд**ли на лавках да по пивнушкам, и «кучки»
эти между собой не пересекались ) Но сейчас им дали Фейсбук и понеслась  @igg{fbicon.face.tears.of.joy} 

\iusr{Сергей Александрович}

Читаю вас и так и хочется вас поправить со Страной Советов!))))))Но нет вам виднее!
И вообще вы мер который выбрал себя сам, вы изверг, вы мер мер беспредела, ......, а тут ещё и двух метровый блондин!)))))))
Держись Борис Альбертович !
Собаки лают караван идёт!
Всем мил не будешь!
Сил вам !

\iusr{Алла Олейник}

Люди темные, как ёжики в тумане. В школах есть урок права с 9 класса по 11. И
за 3 года не учат ветви власти и кто за что отвественный. Зато там куча
академической зауми, которая вылетает из головы на раз два. А во взрослой жизни
мало кто самостоятельно может разобраться...

\begin{itemize} % {
\iusr{Олексій Андрусевич}
\textbf{Alla Oleynik} ...ще й не має бажання розбиратися...
\end{itemize} % }

\iusr{Максим Андрейчук}

По моему мнению - проблема нашего народа в том, что все они привыкли или так уж
сложилось традиционно жить по чьей то указке - то есть не привыкли быть
самостоятельными принимать какие-то решения и нести ответственность за эти
решения в которых дьявол кроется в деталях.

То есть человеку который проработал на дядю всю свою жизнь легко давать советы
не понимая, что такое организационные процессы, ответственность,
самостоятельность и т.п.

Так уж сложилось, что даже исторически, что 99,9\% наших граждан не строили свою
жизнь а просто приходили на работу которую им кто-то придумал/организовал с
графиком, с условиями работ, с проработкой нюансов и т.п. и им кажется, что
организовывать всё просто, хотя сами они понятия не имеют, что это такое.

\iusr{Наталя Пасічник}

Іноді я вас не розумію, іноді - категорично не підтримую, але частіше - просто
захоплююся!!! І, дивно, що повчати не хочеться ні в якому разі!

\iusr{Людмила Селютина}

Всегда эти любители советов и ЦУ выводили меня из душевного равновесия. Часто
им отвечала, что не надо мне рассказывать, что мне надо или не надо делать или
было надо или не надо делать. Видимо у таких людей жизнь протекает серо и не
насыщено и они от безделья и скуки бесцеремонно лезут с советами и указаниями,
о которых их не просят.


\iusr{Евгения Цветанская}
Не матюкайтесь @igg{fbicon.face.tears.of.joy} 

\begin{itemize} % {
\iusr{Mykhailo Omelchoock}
\textbf{Евгения Цветанская} це не порада, це наказ? @igg{fbicon.face.grinning.smiling.eyes} 
\end{itemize} % }

\iusr{Ольга Коваленко}
Не говорите что мне делать, и я не скажу, куда вам идти...

\iusr{Виктория Степаненко}

Понимая, что заблуждение - это, когда человек думает, что он прав, а на самом деле - не прав, я советы давать не люблю.
И, думаю, что это правильно.

\iusr{Mila Lyudmila}

Как я жалею. 29 сентября 2021г. Я была в Меноре в Хеседе по своим делам. Потом
пошла во двор, где люди вакцинировались, походила, посмотрела и не решилась. 6
октября я заболела. 15 октября положили в больницу 21 в тяжелом состоянии,тест
показал положительный, у меня ковид. В больнице больные все без прививки лежат
семьями и во всех тяжелое состояние. Температура 39, кислород в крови 86.
огромная одышка. Дышала кислородом. Становилось все хуже. Поражена пол правого
легкого. Спасибо мужу и дочери купили препарат Ремдесивир, 6 флаконов за 24000
грн, я жива. Много пью воды и находуст в движении. Муж и дочь привиты пфайзер.
Не заболели.

\iusr{Olga Olga}

Господь, умоляю, дай Борису терпения и здоровья на этом поприще. Усмири
несведущих и вложи в их пустые головы немного мозгов. Аминь.

\iusr{Елена Ничипорчик}
\textbf{Olga Olga}  @igg{fbicon.hands.pray} 

\iusr{Ольга Шиян}

И это все последствия разрушения психиатрической службы.... Сочувствую вам, как
это вынести в масштабах огромного города , не знаю. Тут в школе несколько таких
советчиков, анонимщиков и активистов, работать не дают.... Но к сожалению
подобные люди просто окружают.... Ваш пост дал надежду, что все таки, с ума
сошёл не ты лично....

\iusr{Николай Боровик}
Зато люди не равнодушны;) переживают за Вас;)

\iusr{Инна Худова}

Сил Вам и терпения -наш мер, @igg{fbicon.flame}{repeat=2} самый лучший мер! Кому не видна, его
работа, оденьте очки, неужели не видно, как город преобразился? @igg{fbicon.exclamation.mark} Борис Альбертович
берегите себя, свою семью и наш город @igg{fbicon.exclamation.mark} - а дальше из серии  - собаки лают -
караван идёт/потому, что караван знает, куда и за чем идет!! @igg{fbicon.hand.waving}  @igg{fbicon.face.smiling.eyes.smiling}  @igg{fbicon.hand.ok} 

\iusr{Владимир Петренко}
\textbf{Борис Альбертович}, я написал Вам в личку, прочтите пожалуйста.

\iusr{Оксана Широкопояс}
\textbf{Борис Альбертович}, 

спасибо вам большое  @igg{fbicon.hands.pray} за установку светофора на
перекрёстке Электрическая и 20 Лет Победы.

Он был очень важен, не только для регулировки движения, но и для безопастности
граждан.

\begin{itemize} % {
\iusr{Татьяна Соина}
\textbf{Оксана Широкопояс} плюсую @igg{fbicon.heart.red}. Муж тоже очень рад.
Ждем теперь когда подключат
\end{itemize} % }

\iusr{Олександр Лисенко}
Сильно. Особливо останнє. Дякую за допис.

\iusr{Валентина Плющий}
Последний абзац супер!

\iusr{Елена Суркова}
 @igg{fbicon.biceps.flexed} Браво!

\iusr{Елена Адонина}

Я разделяю Ваши эмоции, но всё таки мэру не надо использовать \enquote{плохие} слова в
своих высказываниях, хотя я понимаю, что это доступнее для отдельной категории
наших людей)


\iusr{Наталія Ковальова}

Ну и скромняга  @igg{fbicon.face.tears.of.joy}  @igg{fbicon.thumb.up.yellow}  - \enquote{я - двухметровый блондин с голубыми глазами}.
Ален Делон скромно покуривает в недоумении ..... @igg{fbicon.thinking.face} 

\begin{itemize} % {
\iusr{Владимир Выговский}
\textbf{Наталія Ковальова} Делон зростом 1,60, не блондин, хоча і з синіми очима.

\iusr{Валентина Попеня}
\textbf{Владимир Выговский} рост Делона 1,77

\iusr{Владимир Выговский}
\textbf{Валентина Попеня} ну, це все міняє.
\end{itemize} % }

\iusr{Олена Кривенко}
Приказ -  @igg{fbicon.thumb.up.yellow}{repeat=3} 

\iusr{Николай Слободянюк}
Держитесь \textbf{Borys Filatov} ))!!!

\iusr{Andrew Makarov}
Доброе утро! Вот такого на нашей улице не было, пожалуй, с конца 80-х годов прошлого века. Мэрии - спасибо.

\ifcmt
  ig https://scontent-frx5-2.xx.fbcdn.net/v/t39.30808-6/254326168_3927905017312153_8176051212958990912_n.jpg?_nc_cat=103&ccb=1-5&_nc_sid=dbeb18&_nc_ohc=q6cffAkWBqgAX-5scuN&_nc_ht=scontent-frx5-2.xx&oh=7a06de21624245d188b4a35aac082e31&oe=61A87FB3
  @width 0.4
\fi

\iusr{Алєся Юріївна}

* І десь у кутку, тихо плачуть Ваші радники* можуть же і звільнити тепер)))

Доречі про \enquote{совєтчіков}: як правило, це люди, які самі ніколи ні чого не
роблять, лише чекають, коли хтось інший щось зробить, щоб потім почати
\enquote{савєтовать})))

\iusr{Оксана Воронежская}
 @igg{fbicon.face.grinning.sweat}  @igg{fbicon.hands.applause.yellow}  благодарю, подняли настроение! Особенно про архивы ... @igg{fbicon.laugh.rolling.floor} 

\iusr{Olga Kovalchuk}

Уважаемый Борис Альбертович. С огромным удовольствием не только Вас всегда
читаю, но и с нетерпением жду издания \enquote{Записок мера} ...Блестяще изложенные
истории, размышления займут достойное место среди произведений современных
прозаиков. Прошу не принимать, как совет @igg{fbicon.smile} 

\iusr{Елена Жарикова}

Да, самое ужасное, что порой людей интересует только то, что под носом!
Собственник говорит, что покрасьте мне стену перед окном, чтоб я выглянул и
красиво!

Я ему отвечаю, что деньги на другие, очень важные работы по дому нужны, красота
подождёт! Вот посмотрите какая щитовая у вас в подъезде: новая, все поменяно,
просто ВСЁ! На что он отвечает, что а мне это не надо! Мне стену покрасьте!

И такие люди дают советы нашему МЭРУ, да мэру с большой буквы!


\iusr{Татьяна Семенова}

Здорово, что Вы и ваши близкие живы и здоровы. Хорошо, что у вас хорошее
настроение и отличное чувство юмора. Это поможет и дальше на полную
вкладываться на работе ибо нам, обывателям всегда что-то надо, ах, мы такие
внезапные и непредсказуемые. В общем хороших выходных.

\iusr{Татьяна Филипповна}

Как бы заставить кого нужно)) узаконить установку подобных счетчиков тепла на
каждую батарею. Тогда мерзнущие давали бы жару, а экономные одевали носки. Это
фото из Германии, даже на полотенцесушителе стоят. Помогите внедрить! Просьба
большая, не совет, еслишо))

\ifcmt
  ig https://scontent-frx5-1.xx.fbcdn.net/v/t39.30808-6/253948364_10227185633575941_7414433628395428102_n.jpg?_nc_cat=100&ccb=1-5&_nc_sid=dbeb18&_nc_ohc=zjmhR_HkEscAX9PCP3P&_nc_ht=scontent-frx5-1.xx&oh=0ece820a7f193bf65f8e57e2d4c84080&oe=61A85EB6
  @width 0.4
\fi

\begin{itemize} % {
\iusr{Yury Lyakh}
\textbf{Tetyana Pylypivna} разводка труб в совецких домах не позволит в большинстве своем индивидуально крутить каждую отдельную батарею, пострадает весь стояк в подъезде.

\iusr{Татьяна Филипповна}
\textbf{Yury Lyakh} 

В этом немецком доме разводка стояков, как в хрущевках. Поэтому и счетчиков
много, а не один на квартиру. В каждой комнате отдельный стояк. Я в своей,
кстати, когда меняла забитые батареи, на каждой из них поставила краны. Только
смысл их закрывать? Экономить кому? Ведь мне все-равно отопление по нормам (
здесь надо бы нецензурно выразиться, да ладно.) считается. А у них считаются
теплокалории. Хололная вода, что у нас, в Киеве, гоняется по батареях, циферки
в окошке на счетчике не ускорит. Видимо, поэтому, такие счетчики и не
узаконивают. Какую по счету хатынку ахметка старшему сыну купил намедни? А вы\enquote{
низзя, низзя} ))) Я еще советский инженер, еслишо))

\ifcmt
  ig https://scontent-frt3-1.xx.fbcdn.net/v/t39.30808-6/254301531_10227186239671093_724938546845346380_n.jpg?_nc_cat=106&ccb=1-5&_nc_sid=dbeb18&_nc_ohc=7k_Hl_83iecAX8DnQpn&_nc_ht=scontent-frt3-1.xx&oh=5a4a079a87312dd01c7453d7c569d356&oe=61A8A945
  @width 0.4
\fi

\iusr{Татьяна Филипповна}
Узаконивание только таких счетчиков, даже не учитывая много чего хорошего и до этого, потянет на президенство, аж бегом, не?

\iusr{Niina Ohrimenko}
\textbf{Tetyana Pylypivna} 

У порядку обміну досвідом. В Естонії за законом кожен багатоквартирний будинок
повинен бути об'єднанням власників квартир. Власники і вирішують про
встановлення таких лічильників на батареї і рубильників, які вмикають-вимикають
батареї. За власні гроші. Дані від лічильників автоматично передаються на
комп'ютери, які розраховують вартість опалення для кожної квартири. Міська
влада до цього не притична. В Естонії майже щомісяця перебувають українські
чиновники/делегації різного рівня, які, напевно, приїздять сюди за досвідом і
порадами.

\iusr{Татьяна Филипповна}
\textbf{Niina Ohrimenko} Їх такий досвід не влаштовує. Бо як же тоді так нахабно красти?

\end{itemize} % }

\iusr{Кристина Варава}

Чем выше человек по статусу, тем больше человек начинает ему давать советы. Как
распоряжаться деньгами ( на что тратит на благотворительность, например) не
туда инвестирует, не тех берет в команду и этот список бесконечный. Просто люди
не привыкли нести ответственность за себя, а давать советы это ж бесплатно:) и
значимость проявил личную и ответственности ноль. Так и живём. Большому кораблю
большое плаванье. Успехов вам. Читаю ваши посты как дневник города. Тут и
эмоции, и боли, и тенденции и просто личность в мегаполисе.

\iusr{Таня Буряк}

Проще всего давать советьі. Особенно если их не спрашивают. И єто раздражает.
Но... Иногда советьі полезньі. Убедиться, что поступаешь правильно, послушать и
поступить по-своему или узнать мнение других

\iusr{Oxana Shamray}
про це ще Чуковський писав \enquote{...и такая дребедень целый день, то тюлень позвонит, то олень..}

\iusr{Sergei Kurchenko}
Да, все прекрасно @igg{fbicon.heart.red} Но жить стало опасно, бедные и богатые по разным берегам... М.Барских

\iusr{Алексей Барбашов}

В стране полно специалистов, знающих как управлять страной лучше
политиков-бездарей! Но, к сожалению, они все уже работают дворниками,
парикмахерами, таксистами и др.  @igg{fbicon.wink} 

\iusr{Дрозд Любов}
Та нє, спать треба з дружиною. @igg{fbicon.face.smiling.eyes.smiling} .Вона у Вас класна!

\iusr{Наталія Баласинович}
Талант викладати думки,пишаюсь

\iusr{Natali Kovalenko}
Без советов, спасибо за красоту в городе @igg{fbicon.thumb.up.yellow}{repeat=2} 

\iusr{Марина Маламуж}

Советы сейчас превратились в хотелки. За частую Совет мой, а выполнение твое.
Но если наоборот, то сразу получаешь в ответ, ты чего......


\iusr{Лидия Опаренко}
Собаки лают а караван идёт

\iusr{Kostiantyn Boiaryntsev}

Правильный ответ советчикам: \enquote{Я бы вам посоветовал мне не советовать!}

\iusr{Вячеслав Тиховлис}

Борис Альбертович, можно вопрос, а не совет. Когда подрядчик по гарантии
заменит асфальтовое покрытие на Южном мосту, которое было уложено весной и
сейчас стало еще хуже, чем было весной?

\iusr{Lyubov Shepilova}

Спасибо за утреннюю исповедь, впечатляет  @igg{fbicon.thumb.up.yellow} 

Но что интересно, большинство комментариев от женского пола @igg{fbicon.thinking.face}  Более того,
открыто заявляют: \enquote{От чтения ваших постов получаю удовольствие...} А вот это
уже настораживает. А где же мужская половина в это время?  @igg{fbicon.face.grinning.big.eyes}{repeat=3} 

Позитивного всем настроения @igg{fbicon.face.smiling.halo}, суббота на дворе и прекрасная осень @igg{fbicon.rainbow} 

\iusr{Ярослав Бровченко}
Дам вам хороший совет - не обращать на чужие советы внимание)

\iusr{Надежда Журавель}
Дааа, без юмора на Вашем посту никак не выжить!!! Всем с диванов виднее как и что нужно делать. @igg{fbicon.fist.raised} 

\iusr{Игорь Богдашев}

Я думал, что у мера в кабинете есть рубильники ''свет'' ''газ'' ''отопление''
''интернет'', а в чем тогда крутость мера, если даже рубильников нету?

\iusr{Ада Фрідман}
Люблю читати Філатова!)

\iusr{Светлана Мороз}
ТЕРПЕНИЯ! @igg{fbicon.fist.raised}{repeat=2} Пусть пишут,что хотят !
@igg{fbicon.shrug} Город меняется в лучшую сторону!!! @igg{fbicon.hands.pray}
СПАСИБО!!! @igg{fbicon.fist.raised}{repeat=2}
@igg{fbicon.hands.applause.yellow}{repeat=3} 

\iusr{Анна Редько}
Одни эксперды и поучатели ... )))

\iusr{Слава Иванов}
Это «они» ещё не начали рассказывать что земля плоская ...
@igg{fbicon.thinking.face}  @igg{fbicon.face.tears.of.joy}  @igg{fbicon.globe.showing.americas} 

\iusr{Виктория Поварницкая}
Сил вам @igg{fbicon.face.relieved} 

\iusr{Антон Березовский}
"Не давай советы меру Днепра, не став мером Днепра"
Конфуций до вакцинации

\iusr{Инна Куликовская}
Подняли настроение)) Вы крутой Мэр!)

\iusr{Віктор Журавель}
Ну і Бог з ними, як кажуть: собака гавкає, а караван іде. На всіх нервової системи не вистачить.

\iusr{Svitlana Dashkevich}

тут згодна, верещали як різані батареї холодно, тепер вони не гріють, а насчітають
ого го, та ви елопи на дворі +16 на кой ляд вони второхтілись холодні, відповідь
бо тепло, ну дебілізм пре через край

\iusr{Vitalii Pavliuk}
Ох, и энергии у вас, БА... ! Откуда подпитываетесь?
Телефон из рук выпадает при чтении.
Трезв, если шо.

\iusr{Наталія Шуварська}
Ви там, тойво, тримайтеся чи шо )

\iusr{Вячеслав Стадник}
Зато можно поржать не покупая билетик в цирк. 24/7

\iusr{Лада Хомячок}

как к Вам, Борис Альбертович, можно попасть на личный приём? писала заявку на
ОСОБИСТИЙ ПРИЙОМ ДО МІСЬКОГО ГОЛОВИ, но дали ссылки только на электронную
заявку... написала... но попадёт ли к Вам она?!

\iusr{Dennis Belichenko}
Я тоже Борю Холода помню с детства, у нас отцы дружили. Мой папа твоего тоже знал в те времена. Мы потом в Америку уехали в 90х

\iusr{Елена Донская}
У Вас железные нервы. Как Вам удаётся оставаться спокойным после всего этого бреда, который Вам пишут?

\iusr{Лускань Алина}
Подскажите пожалуйста почему на улице Звонкой не включили отопление ?

\iusr{Lena Anele}
Это пять)))))), особенно про отопление)))))
\iusr{Igor Bez}
Ну так с дивана легче всего п\#₽ть @igg{fbicon.face.tears.of.joy}{repeat=3} 

\iusr{Тетяна Ярмак}
А жаль((

\iusr{Сергей Перевозчиков}
А городской архив отцифровали?))

\iusr{Роман Матофій}
\textbf{Borys Filatov} Обещали Чебанову заасфальтировать Четверку. Еще перед выборами.

\begin{itemize} % {
\iusr{Роман Матофій}
Жаль, что нет ответа.
\end{itemize} % }

\iusr{Andriy Zharov}
Надо так:
Падлы !!! Вакцинируйтесь !!!!

\iusr{Володимир Білий}
Страна советов и экспертов)))любой знает всё))))

\iusr{Михайло Майстренко}

У людей з ОФШОРНИМИ РАХУНКАМИ, БІЗНЕСОМ І НЕРУХОМІСТЮ ЗА КОРДОНОМ, які сидять
НА ДЕРЖАВНИХ І ГРОМАДСЬКИХ ПОСАДАХ, зовсім інше ставлення до СВОГО
СУСПІЛЬСТВА! Вони просто заробітчани і раби своїх кубишок!!!


\iusr{Boris Ivanov}
так и знал, что Вас триггернет тот мужик с отоплением  @igg{fbicon.face.tears.of.joy} 
отличный пост!

\iusr{Татьяна Диденко}
Я, не могу не вставить 5 копеек, советую и приказываю Вам, как старший друг, оставайтесь таким как Вы есть!!!

\iusr{Ира Иваница}

\ifcmt
  ig https://scontent-frt3-2.xx.fbcdn.net/v/t39.30808-6/254703695_1931480217032125_5777365100885601117_n.jpg?_nc_cat=101&ccb=1-5&_nc_sid=dbeb18&_nc_ohc=6iW4Dl1grLgAX9fmGQ6&_nc_ht=scontent-frt3-2.xx&oh=8dd596ed9c46e650ccf0e3c9e035c0b8&oe=61A960A3
  @width 0.4
\fi




\end{itemize} % }
