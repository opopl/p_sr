% vim: keymap=russian-jcukenwin
%%beginhead 
 
%%file 06_11_2021.fb.bogatyrev_daniil.1.den_osvobozhdenia_kieva
%%parent 06_11_2021
 
%%url https://www.facebook.com/bogatyriov.daniil/posts/1068080230672990
 
%%author_id bogatyrev_daniil
%%date 
 
%%tags 1943,1943.kiev.osvobozhdenie,6_nov,gorod,istoria,kiev,osvobozhdenie,prazdnik,ukraina
%%title В День освобождения Киева
 
%%endhead 
 
\subsection{В День освобождения Киева}
\label{sec:06_11_2021.fb.bogatyrev_daniil.1.den_osvobozhdenia_kieva}
 
\Purl{https://www.facebook.com/bogatyriov.daniil/posts/1068080230672990}
\ifcmt
 author_begin
   author_id bogatyrev_daniil
 author_end
\fi

В День освобождения Киева.

Многие в сети удивляются тому, что в \enquote{поздравлении} с Днём освобождения Киева,
Владимир Зеленский заявил об этом дне, как об \enquote{истории безмерного равнодушия и
жестокости больших начальников}, намекая на старые либерастические мифы о том,
что \enquote{забросали трупами} и всё такое. 

Однако, меня подобные заявления из уст Зеленского как раз не удивляют. В
современной Украине официальной идеологией является украинский национализм.
Именно из его постулатов исходят ВСЕ украинские чиновники в своих публичных
оценках событий прошлого, включая Великую Отечественную Войну. Поэтому, не
ждите от них почтения к праздникам, связанным с теми событиями. Даже на фото,
прикреплённом к \enquote{поздравлению} Зеленского видно, что в зале славы на Лютежском
плацдарме стоит почётный караул в мазепинках и форме, стилизованной под форму
УПА. Пока в Украине существует \enquote{постмайданный} политический режим, максимум
чего мы можем ожидать от его функционеров в \enquote{военные} праздники, связанные с
советским прошлым - это жирный плевок в это прошлое. Скажите спасибо, что пока
с портретами \enquote{Гитлера-освободителя} не ходят. Так что, \enquote{возмущение} постом
Зеленского полагаю лицемерным. Все мы в курсе, в какой стране живём и чего
ждать от её властей. 

И да, всех нормальных людей поздравляю с праздником! Не с очередным \enquote{днём
скорби и соплей}, а именно с праздником!

Больше читайте на моём канале по ссылке и подписывайтесь: 

\url{t.me/polit_bogatyr}

\ii{06_11_2021.fb.bogatyrev_daniil.1.den_osvobozhdenia_kieva.cmt}
