% vim: keymap=russian-jcukenwin
%%beginhead 
 
%%file slova.nacionalist
%%parent slova
 
%%url 
 
%%author 
%%author_id 
%%author_url 
 
%%tags 
%%title 
 
%%endhead 
\chapter{Националист}

На Светлану и ее супруга Вадима поступили доносы в Генеральную прокуратуру
Украины и СБУ. После того, как правоохранительные органы не нашли в действиях
Светланы Пикты состава преступления, многодетная мать подверглась травле со
стороны \emph{националистов} радикальной группировки С14. Опасения за свою
жизнь и жизнь детей вынудили семью в конце концов, принять решение о
переселении в Россию, а именно – в Ярославскую область,
\textbf{Мужественные киевляне стали нашими земляками}, rweek.ru, 20.07.2018

%%%cit
%%%cit_pic
%%%cit_text
Но дело не в «испорченных русских», а в названии страны. «Украина» - это
полонизированное слово «Окраина», и совсем непонятно, почему западные украинцы,
вошедшие в состав Российской империи, а затем и СССР, предпочли называть свою
страну именно этим совсем не презентабельным словом. Но оно прочно укоренилось
в лексиконе, а большевики после 1917 года пошли на поводу у \emph{националистов},
переименовав Малороссию в Украину (Украинская ССР). Определение «окраина» было
забыто – \emph{националистам} очень сильно хотелось побыстрее идентифицироваться
%%%cit_title
\citTitle{Почему современную Украину назвали «окраиной», а не более престижно - Киевской Русью?}, 
Исторический Понедельник, zen.yandex.ru, 22.02.2021 
%%%endcit

