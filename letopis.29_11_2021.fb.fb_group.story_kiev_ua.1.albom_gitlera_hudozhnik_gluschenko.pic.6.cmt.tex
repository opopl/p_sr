% vim: keymap=russian-jcukenwin
%%beginhead 
 
%%file 29_11_2021.fb.fb_group.story_kiev_ua.1.albom_gitlera_hudozhnik_gluschenko.pic.6.cmt
%%parent 29_11_2021.fb.fb_group.story_kiev_ua.1.albom_gitlera_hudozhnik_gluschenko
 
%%url 
 
%%author_id 
%%date 
 
%%tags 
%%title 
 
%%endhead 

\begin{center}
\figCapA{Угода про ненапад між Німеччиною та Радянським Союзом (нім.
Deutsch-sowjetischer Nichtangriffspakt; рос. Договор о ненападении между
Германией и Советским Союзом) — розрахована на 10 років міждержавна угода,
датована 23 серпня 1939 року, підписана у Москві міністром закордонних справ
Німеччини Йоахімом фон Ріббентропом та Головою Ради народних комісарів,
народним комісаром закордонних справ СРСР В'ячеславом Молотовим у присутності
члена Політбюро ЦК ВКП(б), секретаря ЦК ВКП(б), члена Виконавчого комітету
Комінтерну Йосипа Сталіна та посла Німеччини Вернера фон дер Шуленбурга. В
літературі і пропаганді Угода отримала назву «Пакт Молотова — Ріббентропа».}	
\end{center}
