% vim: keymap=russian-jcukenwin
%%beginhead 
 
%%file 02_12_2020.news.ua.strana.2.ukr_songs_youtube_stats
%%parent 02_12_2020
 
%%url https://strana.ua/news/304242-morhenshtern-little-big-basta-i-zivert-kakuju-muzyku-slushali-ukraintsy-v-2020-hodu.html
 
%%author 
%%author_id 
%%author_url 
 
%%tags 
%%title 
 
%%endhead 

\subsection{Ни одной песни на украинском языке. Что украинцы слушали на YouTube в уходящем году}
\Purl{https://strana.ua/news/304242-morhenshtern-little-big-basta-i-zivert-kakuju-muzyku-slushali-ukraintsy-v-2020-hodu.html}

14:42, 2 декабря 2020
\ifcmt
pic https://strana.ua/img/article/3042/42_main.jpeg
caption Кадр из клипа "Краш" Клава Кока \& Niletto. Скриншот YouTube 
\fi

В уходящем 2020 году украинцы слушали в основном российских исполнителей в
YouTube. А если и украинских - то русскоязычных.

Молодежные мелодии, под которые можно танцевать, набирали миллионные
просмотры в том числе и благодаря украинской аудитории. Таковы данные
компании Google по итогам уходящего года. 

В топ-10 клипов, которые неизменно занимали первые места в трендах, только
две видеоработы украинских артистов - это украинская группа Artik \& Asti и
дуэт Дмитрия Монатика с Верой Брежневой. 

Все остальное - это российские поп- и рэп-исполнители, а также панк-рейв-группы
и премьерные видеоклипы российских блогеров. Рейтинг этого года возглавила
песня \enquote{Краш} российской блогерши-певицы Клавы Кока и российского исполнителя
Niletto.

Разбирались в главных музыкальный трендах уходящего года и анализировали,
какая доля российской музыки приходится на украинскую аудиторию.

\subsubsection{Топ-10 музыкальных видео 2020 года в Украине}

Клава Кока \& Niletto - Краш (official video) - этот сингл
российских исполнителей был выпущен 10 июня на лейблах Black Star Inc и
Zion Music. Видео набрало около 115 млн просмотров. Видеоклип, где дуэт
исполнителей из Екатеринбурга выступил в роли мультперсонажей, выполнен в
стиле компьютерной игры. Режиссером клипа выступил Дмитрий Климов. 

В тексте песни повторяются слова: \enquote{Ты мне не враг, но с тобой не дружу. Я
твой личный маньяк. Я тебя придушу}. 

IFrame

\textbf{Little Big - Hypnodancer (Official Video)} - песня
российской панк-рейв-группы была выпущена 8 мая 2020 года на лейблах
Warner Music Russia и Little Big Family. 

Песня записана на английском языке. Клип набрал около 150 млн просмотров и
попал в тренды YouTube в 26 странах. В Украине занял 1 место в трендах.  Клип
снимали в Санкт-Петербурге, в нем можно увидеть таких российских знаменитостей:
шоумена Александра Гудкова, фронтмена группы The Hatters Юрия Музыченко,
вокалистку группы \enquote{Ленинград} Флориду Чантурию, блогеров Руслана
Усачева и Данила Поперечного. 

По сюжету клипа - главные герои в лице Ильи Прусикина и Юрия Музыченко танцуют
на столах в казино гипнотические танцы. В результате такого баттла происходит
ограбление игровых заведений. 

\textbf{Artik \& Asti - Девочка танцуй (Official Video)} - украинская поп-группа,
основанная в 2010 году. Исполняет сингл на русском языке. Режиссером клипа
стал украинский клипмейкер Алан Бадоев.

В составе группы - два артиста Артём Умрихин (псевдоним Artik), который
является продюсером и исполнителем, а также вокалистка Asti (Анна Дзюба).
Группа часто выступает с гастролями в России и Европе. 

Просмотров клипа \enquote{Девочка танцуй} свыше 62 млн. 


Mia Boyka \& Егор Шип - Пикачу - российские артисты, 18-летний Егор Шип
и 23-летняя Миа Бойка, больше известны как блогеры и звезды тик-тока.
Впрочем, это не помешало им снять забавный видеоклип на звукозаписывающей
студии лейбла \enquote{Black Star Inc}. Премьера песни появилась в сети 30 июля и
тут же стала главным хитом лета.

Клип набрал 89 млн просмотров, а текст песни, посвященной культовому
персонажу из аниме 90-х, глубоко въелся в мозг меломанам. Эту веселую
песенку молодежь с удовольствием качала себе на рингтонг.

В тексте хита есть слова: \enquote{Я сижу на камушке, где ж ты моя лапушка, тут и
там ищу, чтоб не потерять. Ты мой Пикачу, выбираю тебя! Пика пика чу. Так
с тобой хочу чуть-чуть}.


\textbf{Little Big - Uno - Russia - Official Music Video - Eurovision 2020} -
премьера нашумевшего клипа и испано-английского сингла, с которым Little
Big планировали ехать на Евровидение-2020 (конкурс отменили из-за пандемии
коронавируса, Ред.), вышел 12 марта под лейблами Warner Music Russia и
Little Big Family. 

Видео набрало почти 169 млн просмотров. 

В клипе снялись четверо постоянных участников группы \enquote{Little Big} - Илья
Прусикин, Софья Таюрская, Сергей Макаров и Антон Лиссов. В съемках также
участвовали фронтмен группы \enquote{The Hatters} Юрий Музыченко, солистка группы
\enquote{Ленинград} Флорида Чантурия и танцор Дмитрий Красилов.

Образы в клипе в стилистике 1970-1980-х годов пришлись по вкусу всем
блогерам, в том числе иностранным, которые посвятили клипу комичные
интернет-мемы.

В целом клип стал пародией на самый типичный номер \enquote{Евровидения},  котором
высмеиваются все клише, связанные с конкурсом.

\textbf{Morgenshtern \& Элджей - Cadillac} - эта песня российских
хип-хоп-исполнителей Алишера Моргенштерна и Элджея была выпущена 9 июня
2020 года на лейбле Zhara Music. Кстати, 9 июня Алишер объявил днем
Кадиллака, а 17 июля вышли ремиксы песни.

Релиз клипа на трек состоялся 9 июля, в день рождения Элджея. Он набрал
83,2 млн просмотра. 

Его режиссёрами стали Ян Боханович и Роман Нехода, а продюсером Slava
Marlow.

\textbf{Dabro - Юность} - российской поп-группе удалось взорвать YouTube
романтической композицией \enquote{Юность}. Клип набрал 75,8 млн просмотров и
сделал братьев Ивана и Михаила Засидкевичей известными.

Интересно, что Засидкевичи - уроженцы Казани, где обычно снимают и
записывают все свои клипы. Иван Засидкевич получил музыкальное образование
в Донецке, в Музыкальном училище ДМУ. Он владеет многими инструментами:
труба, ударные, фортепиано, балалайка.

Михаил окончил музыкальную школу по классу аккордеона. Совместно братья
создают ремиксы и пишут музыку для других исполнителей.

IFrame

\textbf{MORGENSHTERN \& Тимати - El Problema} - клип российских рэп-звезд вышел 18
сентября 2020, спродюсирован Slava Marlow. В песне используются 808-е басы
и скрипка - по просьбе Тимати, поскольку он закончил музыкальную школу по
классу скрипки.

В октябре Slava Marlow и Моргенштерн опубликовали видеоролик на YouTube, в
котором раскрыли главный секрет песни. В бас был добавлен звук пердежа
Моргенштерна. По заверению музыкантов, Тимати не знал о данной особенности
бита. 

Сейчас у клипа 71,1 млн просмотров. 

\textbf{Баста \& Zivert - неболей} - этот хит объединил настоящих мастодонтов
российской рэп-эстрады. Песня получилась немного меланхоличной, а
чувственный клип покорил сердца миллионов зрителей. 

Сейчас у клипа 39,4 млн просмотров. 

IFrame

\textbf{Monatik \& Вера Брежнева} - ВЕЧЕРиНОЧКА закрывают десятку топовых клипов.
Украинские исполнители представили совместный клип 11 июня. Его съемки
проходили недалеко от Барселоны в горах. 

Режиссер клипа - одесситка Таню Муиньо. Стильные образы Дмитрия и Веры
создавали известные стилисты Ольга Жижко и дизайнер Иван Фролов.

У клипа 24,1 млн просмотров. 

Какую музыку слушают в Украине 

Итак, 80\% музыки, которую слушает украинский потребитель в YouTube,
принадлежит российскому шоу-бизнесу. Из 10 клипов, которые выбились в топ,
только 2 клипа украинских исполнителей. В Украине места на музыкальном
олимпе закрепились за российскими Little
Big, Niletto, Morgenshtern, Баста, Zivert и другими.

Да и тот факт, что в рейтинг попала работа Artik \& Asti, нельзя назвать
исключительно заслугой талантливой украинской группы, которая хоть и была
создана в Украине, но сегодня работает в России и популярность получила
именно в этой стране.

Украинским продуктом можно назвать только клип Монатика с Брежневой,
замыкающий рейтинг YouTube. Правда, снимался он за границей, а текст песни
написан на русском языке.

В целом клипы явно продвигают молодежь и блогеры с приличной аудиторией в
соцсетях. Часто клипы - это пародии и пранки, откровенный стеб, дань
тусовочной жизни подростков. Герои в клипах - не без атрибутов хайповости:
со множеством татуировок, ярким макияжем или волосами. Песни будто
отзеркаливают настроения современных подростков, которые искренне делятся
любыми эмоциями - от запредельного веселья до драматических любовных
переживаний. 

Ситуация, в которой украинские слушатели отдают предпочтение российским
или англоязычным исполнителям, не меняется из года в год. В 2019 году
журналист Анастасия Товт проводила опрос молодых украинцев в центре Киева.
Уже тогда было понятно, что русский рэп буквально завоевал сердца
украинской аудитории. "Пошлая Молли", Фараон, Скриптонит, Т-Fest, Макс
Корж, ЛСП, Моргенштерн - вот что играет в наушниках украинцев. В ТОП-100
iTunes год назад украинской была только "Плакала" группы KAZKA.

IFrame

Интересно, что ситуация по рейтингу YouTube по сравнению с прошлым годом
изменилась для украинского языка в худшую сторону - в 2019-м в списке самых
популярных песен было больше украинских исполнителей (4), а также встречались
две песни на государственном языке. 

То есть можно с уверенностью говорить, что на украинскую молодежь
националистические веяния последних лет влияют весьма слабо. Качественный
культурный продукт и общность языка выводят российскую музыку в топ интересов
украинцев. 


