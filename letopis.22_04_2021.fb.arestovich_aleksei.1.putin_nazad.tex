% vim: keymap=russian-jcukenwin
%%beginhead 
 
%%file 22_04_2021.fb.arestovich_aleksei.1.putin_nazad
%%parent 22_04_2021
 
%%url https://www.facebook.com/alexey.arestovich/posts/4259109537486427
 
%%author 
%%author_id 
%%author_url 
 
%%tags 
%%title 
 
%%endhead 
\subsection{Усе це брязкання зброєю - чистий стратегічний блеф, прикриття власної слабкості}
\Purl{https://www.facebook.com/alexey.arestovich/posts/4259109537486427}

\ifcmt
  pic https://scontent-cdg2-1.xx.fbcdn.net/v/t1.6435-9/177217885_4259108137486567_3893970902988090775_n.jpg?_nc_cat=104&ccb=1-3&_nc_sid=8bfeb9&_nc_ohc=kvX72HUOcC4AX9hEkOe&_nc_ht=scontent-cdg2-1.xx&oh=93db42a063e420b3e67266794698661b&oe=60A91E1F
\fi

- Замість зловісного жаху в зверненні до Федеральних Зборів РФ, ми почули від
Путіна пасивно-агресивну промову дуже ображеної та трохи переляканої людини.

Ключове послання для внутрішнього порядку денного - «єдність» (боїться протестів).

Ключове зовнішне послання - «безконфліктний світ» (боїться реального протистояння). 

Нервове спирання лівою рукою на трибуну, пошук надійної опори.

Переминання з ноги на ногу, відхиляння тіла назад, підсвідома спроба дистанціюватись від болючих питань.

Доктрина «безконфліктного існування» означає, що РФ не тягне економічно не те,
що повномасштабну війну, навіть, «просту» гонку озброєнь.

Усе це брязкання зброєю - чистий стратегічний блеф, прикриття власної
слабкості, розрахунок (з Китаєм за спиною), що Україна та НАТО здригнуться та
дадуть задню.

А ми не дали.)

Детальніше: \url{https://youtu.be/H6KssRewTd4}

Мої коментарі: 

00:00 – Гості та тема ефіру.
06:53 – Досягнення на засіданні ТКГ. Промова Путіна про самозахист, а не про напад. 
22:04 – Нагода для України перевірити свою готовність до війни. Чому не буде нового фронту проти України.
52:35 – Що потрібно для втілення військової кампанії Путіна?
01:01:56 – Росію попередили про військову відсіч у разі нападу на Україну.
01:20:41 – Які питання і справи об’єднують українських політиків.
