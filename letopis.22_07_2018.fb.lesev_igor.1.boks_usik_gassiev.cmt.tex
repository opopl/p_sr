% vim: keymap=russian-jcukenwin
%%beginhead 
 
%%file 22_07_2018.fb.lesev_igor.1.boks_usik_gassiev.cmt
%%parent 22_07_2018.fb.lesev_igor.1.boks_usik_gassiev
 
%%url 
 
%%author_id 
%%date 
 
%%tags 
%%title 
 
%%endhead 
\subsubsection{Коментарі}

\begin{itemize} % {
\iusr{Долгов Олег}
На мой взгляд, Там это бой смотрели ровно с таким же интересом, как если бы Гассиев дрался с чехом или англичанином.

\begin{itemize} % {
\iusr{Владимир Волков}
В гостях сейчас в Ленинградской области у родственников. Болели все за Усика!)

\iusr{Игорь Лесев}
\textbf{Владимир Волков} трибуны на Олимпийском Усика тоже НИ РАЗУ не освистали, что какбэ намекает, как в РУ относятся к политике

\iusr{Владимир Волков}

Усик начал нравится в первую очередь после того, как категорически отмел
политику, как не пытались его спровоцировать журналисты!
Молодец! Здесь это тоже ценят!

\iusr{Долгов Олег}
Более того, крымчанина Усика многие воспринимают как своего. Тут и болели так: наш флаг - не наш флаг, при этом желая победы обоим.

\iusr{Владимир Волков}

Сейчас некоторые "особоактивные" украинские граждане уже пытаются перекрутить
на свой лад высказывания Усика про развитие украино-российских дружественных
отношений!(((


\iusr{Долгов Олег}
\textbf{Владимир Волков} 

Ой, да щас найдут чтеца по губам, который покажет на какой минуте Усик шептал
"Славу". Там пэрэмога пэрэможная. Мне больше всего веселит гордость о
прозвучавшем в Московии гимне Украины, как будто он звучал откуда то с небес, а
не сами москали его включили.

\end{itemize} % }

\end{itemize} % }
