% vim: keymap=russian-jcukenwin
%%beginhead 
 
%%file 15_10_2021.fb.lgaki.1.pokorjajem_kinovertikal
%%parent 15_10_2021
 
%%url https://www.facebook.com/AkademiyaMatusovskogo/posts/4691855147545230
 
%%author_id lgaki
%%date 
 
%%tags donbass,kino,kultura,lgaki,lnr,lugansk,rossia,saratov
%%title Покоряем «Киновертикаль»
 
%%endhead 
 
\subsection{Покоряем «Киновертикаль»}
\label{sec:15_10_2021.fb.lgaki.1.pokorjajem_kinovertikal}
 
\Purl{https://www.facebook.com/AkademiyaMatusovskogo/posts/4691855147545230}
\ifcmt
 author_begin
   author_id lgaki
 author_end
\fi

Покоряем «Киновертикаль»

Студентка третьего курса кафедры художественной анимации нашей Академии Анна
Вольвак вернулась из Саратова, где работала в составе жюри VIII Открытого
фестиваля-конкурса детского и юношеского кино «Киновертикаль».

\ifcmt
  pic https://scontent-mxp1-1.xx.fbcdn.net/v/t1.6435-9/245244457_4691852910878787_2201626649492581765_n.jpg?_nc_cat=106&ccb=1-5&_nc_sid=730e14&_nc_ohc=YohZWv_yjmIAX-Oe76H&_nc_oc=AQmJ_em6bV0BDZDb70teukiF1hChpNULKonG3GkRWtlUrqytn3bsEB6dDV3dxYc5kDM&_nc_ht=scontent-mxp1-1.xx&oh=deab9b028ec87745f1124168eed4823e&oe=61B02DDE
  @width 0.7
\fi

— В прошлом году Анин авторский мультфильм «Что у кошки для чего» стал
победителем фестиваля в номинации «Свободный полет», и поэтому в нынешнем году
ее пригласили войти в состав жюри и лично присутствовать на фестивале, —
рассказала педагог кафедры и руководитель анимационной студии «Юла» нашей
Детской академии искусств Анна Вегера. 

\ifcmt
  tab_begin cols=2

     pic https://scontent-mxp1-1.xx.fbcdn.net/v/t1.6435-9/245836958_4691852907545454_1666232629720776977_n.jpg?_nc_cat=100&ccb=1-5&_nc_sid=730e14&_nc_ohc=8MXo0gqiAagAX__fvtt&_nc_ht=scontent-mxp1-1.xx&oh=87499af54c2ab69c88f0f7bb909721bd&oe=61B0A636

     pic https://scontent-mxp1-1.xx.fbcdn.net/v/t1.6435-9/246051731_4691852914212120_9083550097498745325_n.jpg?_nc_cat=106&ccb=1-5&_nc_sid=730e14&_nc_ohc=JE0A3__Bh-oAX9C5Nt4&_nc_ht=scontent-mxp1-1.xx&oh=0862d814eaae961f25c0925374ea3666&oe=61AD2E01

  tab_end
\fi

К слову, Анна Вольвак – выпускница «Юлы».

— В команду жюри набираются победители конкурса прошлого года. Участвовать в
работе этой команды – трудоемкий, но интересный опыт! — говорит студентка. —
Помимо эмоций от самой поездки, фестиваль тоже оставил море впечатлений.
Мегагостеприимные организаторы, новые знакомства, премьера анимационного фильма
Гарри Бардина «Песочница»... В общем, пища для мозгов на долгое время вперёд!
Спасибо фестивалю и организаторам за впечатления, возможности и вдохновение на
новые работы!

Ждем новых мультфильмов Ани.

Фото из личного архива Анны Вольвак.
