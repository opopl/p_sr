% vim: keymap=russian-jcukenwin
%%beginhead 
 
%%file 06_04_2021.fb.kuchinskij_sergej.1.liderstvo.cmt
%%parent 06_04_2021.fb.kuchinskij_sergej.1.liderstvo
 
%%url 
 
%%author 
%%author_id 
%%author_url 
 
%%tags 
%%title 
 
%%endhead 
\subsubsection{Коментарі}

\begin{itemize}
%%%fbauth
%%%fbauth_name
\iusr{Олексій Прилипко}
%%%fbauth_url
%%%fbauth_place
%%%fbauth_id
%%%fbauth_front
%%%fbauth_desc
%%%fbauth_www
%%%fbauth_pic
%%%fbauth_pic portrait
%%%fbauth_pic background
%%%fbauth_pic other
%%%fbauth_tags
%%%fbauth_pubs
%%%endfbauth
 

З точки зору системи управління якістю, лідер - це людина, яка бере на себе
відповідальність за результат, розуміє, як досягти цього результату в поточній
бізнес ситуації, виходячи з наявних ресурсів, забезпечує ресурси для досягення
результату (персонал, обладнаня, час ...) , підтримує ризик орієнтовний підхід
та роботи на результат, забезпечує досягнення результату, підтримує
співробітників в їх зусиллях в досягненні результату, допомогає покращувати
діяльність..

З моєї точки зору, лідерство потрібно може 5-10\% дорослого населення. Навряд
це всім потрібно в школі, але цим навичкам в обов'язковому порядку потрібно
навчати середню управлінську ланку та вище.

\begin{itemize}
%%%fbauth
%%%fbauth_name
\iusr{Sergey Kuchinsky}
%%%fbauth_url
%%%fbauth_place
%%%fbauth_id
%%%fbauth_front
%%%fbauth_desc
%%%fbauth_www
%%%fbauth_pic
%%%fbauth_pic portrait
%%%fbauth_pic background
%%%fbauth_pic other
%%%fbauth_tags
%%%fbauth_pubs
%%%endfbauth
 
\textbf{Oleksiy Prylypko} Киевский гуру Проектного менеджмента проф.Бушуев
двигал понятие "Динамического лидерства"(и, кажется, он же его и придумал?).
Там сознательная ситуационная передача лидерства "из рук в руки" под задачу в
больших проектах, вроде Укрытия. Аналогичные наметки и дискуссии были в
интернете у военных США. Когда любой лейтенант может становиться лидером
крупной операции в зависимости от ситуации и задачи. Если, как прогнозируют
некоторые, будущее за сетевым предпринимательским обществом на основе сливок
прекариата, тогда ситуационное динамическое лидерство станет массово
востребованным в динамических локусах предпринимательской сети. Тогда,
"лидерство", наряду с "предпринимательством", есть смысл тотально преподавать в
школах. Сейчас это звучит, как "преждевременная заумь", но кто знает ))

%%%fbauth
%%%fbauth_name
\iusr{Олексій Прилипко}
%%%fbauth_url
%%%fbauth_place
%%%fbauth_id
%%%fbauth_front
%%%fbauth_desc
%%%fbauth_www
%%%fbauth_pic
%%%fbauth_pic portrait
%%%fbauth_pic background
%%%fbauth_pic other
%%%fbauth_tags
%%%fbauth_pubs
%%%endfbauth
 
\textbf{Sergey Kuchinsky} можливо, але лідерство націлене на результат. Дуже
сумніваюсь, що школярі розуміють, чого вони хочуть досягти і як. Знову ж, твої
приклади базуються на людях, які прагнуть результату, і в яких є для цього
ресурси. Я згоден, що лідерству треба навчати, але не в школі , а в ВНЗ, або
окремими тренінгами, і не всіх, а тільки тих хто готовий прийняти на себе цю
роль


%%%fbauth
%%%fbauth_name
\iusr{Sergey Kuchinsky}
%%%fbauth_url
%%%fbauth_place
%%%fbauth_id
%%%fbauth_front
%%%fbauth_desc
%%%fbauth_www
%%%fbauth_pic
%%%fbauth_pic portrait
%%%fbauth_pic background
%%%fbauth_pic other
%%%fbauth_tags
%%%fbauth_pubs
%%%endfbauth
 
\textbf{Oleksiy Prylypko} Ну, когда-то и таблица умножения(да и умение
читать-писать) была "не для всех", а "секретным знанием посвященных египетских
жрецов". Возможно, что то же самое произойдет с недлявсехным "лидерством" и
"предпринимательством". Если упростить и считать, что лидерство это всего лишь
навык привлекать и объединять людей под решение той или иной задачи. И если
прогнозы сбудутся и массовым станет предпринимательский прекариат, то весь
новый (пост)средний класс бросится учить своих детей базовому лидерству. Все
зависит от того, какими будут экономика и общество. И по этим же странным
прогнозам менеджерские иерархии тоже будут уходить. Как уходят индустриальный
пролетариат, бизнес-школы и т.п.

%%%fbauth
%%%fbauth_name
\iusr{Олексій Прилипко}
%%%fbauth_url
%%%fbauth_place
%%%fbauth_id
%%%fbauth_front
%%%fbauth_desc
%%%fbauth_www
%%%fbauth_pic
%%%fbauth_pic portrait
%%%fbauth_pic background
%%%fbauth_pic other
%%%fbauth_tags
%%%fbauth_pubs
%%%endfbauth
 
\textbf{Sergey Kuchinsky} 

з точки зору алгебри, мені подобається модель, коли дитина повинна визначитись,
вона йде до ВНЗ, або отримує робітничу професію. Тому що зараз індустріальні
підприємства стогнуть через відсутність людей, що можуть щось робити руками.

А що стосовно freelance, то якщо це індивідуал, то навіщо йому лідерство? Але
якщо він керівник і в нього є підлеглі, за добробут яких він взяв на себе
відповідальність, от тут йому необхідні лідерські навички

%%%fbauth
%%%fbauth_name
\iusr{Sergey Kuchinsky}
%%%fbauth_url
%%%fbauth_place
%%%fbauth_id
%%%fbauth_front
%%%fbauth_desc
%%%fbauth_www
%%%fbauth_pic
%%%fbauth_pic portrait
%%%fbauth_pic background
%%%fbauth_pic other
%%%fbauth_tags
%%%fbauth_pubs
%%%endfbauth
 
\textbf{Oleksiy Prylypko} 

Думаю, что им уже стонать недолго )) Когда люди массово пошли в города, то
сельхоз тоже некоторое время "стонал". Пока не научился обходиться 1-м
процентом населения. Насколько я помню теорию, одной из причин появления
корпораций называли транзакциональные издержки. Которые сегодня сильно
уменьшаются. То есть, издержки содержания постоянной рабочей силы становятся
меньше издержек от найма и быстрой инкорпорации поденщиков. Зачем лидерство
фрилансу? Чтобы решать более сложные задачи, а не быть более или менее
квалифицированным поденщиком. У специалиста могут быть 2 высших формы развития
и роста - профессиональная(супер-специалист) и социальная(предприниматель). В
мире тотального фриланса(если он будет) те, кто будет владеть элементарным
"лидерством" и "предпринимательством" будут зарабатывать намного больше. Если
при этом упростить получение этих навыков до уровня школьного обучения чтению и
письму, то может получиться любопытно.

%%%fbauth
%%%fbauth_name
\iusr{Олексій Прилипко}
%%%fbauth_url
%%%fbauth_place
%%%fbauth_id
%%%fbauth_front
%%%fbauth_desc
%%%fbauth_www
%%%fbauth_pic
%%%fbauth_pic portrait
%%%fbauth_pic background
%%%fbauth_pic other
%%%fbauth_tags
%%%fbauth_pubs
%%%endfbauth
 
\textbf{Sergey Kuchinsky} 

ми згодні в одному - що це навичок потрібний керівнику і цьому потрібно вчити.



%%%fbauth
%%%fbauth_name
\iusr{Sergey Kuchinsky}
%%%fbauth_url
%%%fbauth_place
%%%fbauth_id
%%%fbauth_front
%%%fbauth_desc
%%%fbauth_www
%%%fbauth_pic
%%%fbauth_pic portrait
%%%fbauth_pic background
%%%fbauth_pic other
%%%fbauth_tags
%%%fbauth_pubs
%%%endfbauth
 
\textbf{Oleksiy Prylypko} Возможно, что лидеры будут
ситуационно-врЕменными(динамическое лидерство). То есть, чуть ли не каждый
сможет неоднократно становиться "директором на час" ))

\end{itemize}

%%%fbauth
%%%fbauth_name
\iusr{Владимир Калюжный}
%%%fbauth_url
%%%fbauth_place
%%%fbauth_id
%%%fbauth_front
%%%fbauth_desc
%%%fbauth_www
%%%fbauth_pic
%%%fbauth_pic portrait
%%%fbauth_pic background
%%%fbauth_pic other
%%%fbauth_tags
%%%fbauth_pubs
%%%endfbauth
 
шляпа Наполеона

\end{itemize}

