% vim: keymap=russian-jcukenwin
%%beginhead 
 
%%file insert
%%parent body
 
%%url 
 
%%author_id 
%%date 
 
%%tags 
%%title 
 
%%endhead 

% Почуйте голос Маріуполя: історії людей, яким пощастило евакуюватися з блокадного міста, Алевтина Швецова, donbas24.news, 21.09.2022
\ii{insert.read_also.shvecova.pochujte_golos_ludej_evak}
% Alyona Alyona та Ukraїner представили кліп, присвячений Маріуполю, Алевтина Швецова, donbas24.news, 18.09.2022
\ii{insert.read_also.shvecova.alyona_alyona_ukrainer_klip_mrpl}
% Твоя сила — прагнути перемоги: триває онлайн-виставка української молоді, Алевтина Швецова, donbas24.news, 24.09.2022
\ii{insert.read_also.shvecova.tvoja_syla_pragnuty_peremogy_onlain_vystavka_molodi}
% Вулиця Богдана Ступки в Покровську: яким був життєвий шлях українського митця, 25.07.2022
\ii{insert.read_also.shvecova.vulycja_bogdana_stupky_pokrovsk}


% Євген Целік пройшов пішки з Києва до Запоріжжя, щоб зібрати кошти на ЗСУ, 10.08.2022
\ii{insert.read_also.elina_prokopchuk.jevgen_celik_pishky_koshty_zsu}
% Історія сталевої Нави — захисниця Маріуполя розповіла про весілля на Азовсталі та російський полон, 05.05.2023
\ii{insert.read_also.elina_prokopchuk.istoria_stalevoj_navy}
% Як під час війни працює Маріупольський державний університет?, 10.05.2022
\ii{insert.read_also.elina_prokopchuk.jak_pid_chas_vijny_pracjuje_mdu}


\ii{insert.author.demidko_olga}
\ii{insert.author.demidko_olga.eng}

\ii{insert.author.ivanova_jana}
\ii{insert.author.shvecova_alevtina}
\ii{insert.author.veremeeva_tetjana}

\ii{insert.read_also.demidko.viktoria_lisogor}
\ii{insert.read_also.demidko.sosnovskij}
\ii{insert.read_also.demidko.stomina}
\ii{insert.read_also.demidko.chernov}
\ii{insert.read_also.demidko.osypenko}
\ii{insert.read_also.demidko.cherepchenko}
\ii{insert.read_also.demidko.krjachok}
\ii{insert.read_also.demidko.saburov}
\ii{insert.read_also.demidko.mariupolchanky}
\ii{insert.read_also.demidko.kozhevnikov}
\ii{insert.read_also.demidko.rudenko}
\ii{insert.read_also.demidko.ukrainceva}
\ii{insert.read_also.demidko.sladkovy}
\ii{insert.read_also.demidko.zhyvko}
\ii{insert.read_also.demidko.zabavin}
\ii{insert.read_also.demidko.zajceva}
\ii{insert.read_also.demidko.ponomareva_feja}
\ii{insert.read_also.demidko.minjajlo}
\ii{insert.read_also.demidko.levchenko}
\ii{insert.read_also.demidko.arhangelska}
\ii{insert.read_also.demidko.shevchenko_chas_dlja_sebe}
\ii{insert.read_also.demidko.djuzhok}
\ii{insert.read_also.demidko.abalmasova}
\ii{insert.read_also.demidko.kyrnyckyj_policeman}
\ii{insert.read_also.demidko.kateryna_karaben_fortun}
\ii{insert.read_also.demidko.chufarov}
\ii{insert.read_also.demidko.kiseljova_nina_mykolaivna}
\ii{insert.read_also.demidko.balabanov}
\ii{insert.read_also.demidko.magdalyc}
\ii{insert.read_also.demidko.dejnychenko}
\ii{insert.read_also.demidko.otchenashenko}
\ii{insert.read_also.demidko.nelli_iskakova}
\ii{insert.read_also.demidko.mangush_koneferma}
\ii{insert.read_also.demidko.dobrunova_anzhelika}
\ii{insert.read_also.demidko.viktor_orlov}
%Історія одного кохання: дівчина-вогонь та перший красень
\ii{insert.read_also.demidko.istoria_odnogo_kohannja}
%Єлизавета Бірюкова, чи "Маріупольський список Шиндлера"
\ii{insert.read_also.demidko.elizaveta_birjukova}
\ii{insert.read_also.demidko.5_misc_priazovja}
\ii{insert.read_also.demidko.tetjana_dubinina}
\ii{insert.read_also.demidko.grecka_teatralna_studia}
\ii{insert.read_also.demidko.teatralna_artil_dramkom}
% Народний театр "Театроманія"
\ii{insert.read_also.demidko.teatromania}
% Про жіноче обличчя війни...
\ii{insert.read_also.demidko.zhinoche_oblycchja_vijny}
% Володимир Кульбака: видатний маріупольський археолог і улюблений викладач
\ii{insert.read_also.demidko.kulbaka}
% Незвичні хобі та "суперздібності" театральних діячів Маріуполя
\ii{insert.read_also.demidko.hobi_zdibnosti}
% Перший недержавний театр Маріуполя "Terra Incognita" (Свій театр для Своїх)
\ii{insert.read_also.demidko.terra_incognita}
\ii{insert.read_also.demidko.240_rokiv_legendy_mista_2}



% ------------------
% Як в Маріуполі проходив фестиваль «Театральна брама» — деталі
\ii{insert.read_also.demidko.donbas24.teatr_brama_detail}
% 7 липня єдиній народній артистці України в Маріуполі Світлані Отченашенко виповнилося б 77 років...
\ii{insert.read_also.demidko.donbas24.otchenashenko_77_rokiv}
% «Театроманія» продовжує розвивати та підтримувати культуру Маріуполя, 06.09.2022
\ii{insert.read_also.demidko.donbas24.teatromania_rozv_pidtrym_kulturu_mrpl}
% У Києвi покажуть виставу про Марiуполь, 20.07.2022
\ii{insert.read_also.demidko.donbas24.u_kyevi_pokazhut_vystavu_pro_mariupol}
% Які театральні проєкти та культурні заходи, присвячені Маріуполю, реалізуються закордонними митцями, 31.07.2022
\ii{insert.read_also.demidko.donbas24.jaki_teatr_proekty_mrpl_zakordon_mytci}
% "Тримаємось разом" — маріупольський театр "Conception" представить нову виставу, 13.10.2022
\ii{insert.read_also.demidko.donbas24.trymajemos_razom_mrpl_teatr_conception_nova_vystava}
% Злочини росіян проти культурної спадщини Донбасу, 11.08.2022
\ii{insert.read_also.demidko.donbas24.zlochyny_rosian_proty_kulturn_spadchyny_donbasu}
% Донбаські легенди, які передаються з покоління в покоління, 03.07.2022
\ii{insert.read_also.demidko.donbas24.donbaski_legendy}
% Інститут нацпам'яті спростовує міфи про Другу світову війну, 15.09.2022
\ii{insert.read_also.demidko.donbas24.uinp_mify_druga_svitova_vijna}
% Як мешканці Приазов'я в минулому карали нечесних чиновників-хабарників, 25.07.2022
\ii{insert.read_also.demidko.donbas24.jak_meshkanci_priazovja_karaly_chynovnykiv_habarnykiv}
% Унікальні факти про театральну культуру Приазов'я, 27.03.2023
\ii{insert.read_also.demidko.donbas24.unikalni_fakty_teatr_kultura_priazovja}
% У Києві відновили маріупольську виставу, 02.12.2022
\ii{insert.read_also.demidko.donbas24.u_kyevi_vidnovyly_mariupolsku_vystavu}
% Чи любив Марко Кропивницький Приазов'я: до дня створення українського реалiстичного театру, 27.10.2022
\ii{insert.read_also.demidko.donbas24.chy_ljubyv_marko_kropyvnyckyj_priazovja_stvoren}
% Маріупольський театр "Conception" показав у Тернополі свою виставу, 12.04.2023
\ii{insert.read_also.demidko.donbas24.mrpl_teatr_conception_vystava_ternopil}
% «Життя переселенське» — маріупольський театр підготував нову прем'єру, 28.02.2023
\ii{insert.read_also.demidko.donbas24.zhyttja_pereselenske_premjera}
% "Від папірусу до гаджету" — новий проєкт Марії та Олександра Сладкових, 25.10.2022
\ii{insert.read_also.demidko.donbas24.vid_papirusu_do_gadzhetu_sladkovy}
% Маріупольці вразили своєю прем'єрою киян, 29.07.2022
\ii{insert.read_also.demidko.donbas24.mariupolci_vrazyly_kyjan}
% У Києвi вiдбулося прощання з героїчним марiупольцем, 06.06.2023
\ii{insert.read_also.demidko.donbas24.u_kievi_vidbulos_proschannja_iz_geroichnym_mariupolcem}
% У Києві презентували музичний відеокліп «Кохання сталь», 15.05.2023
\ii{insert.read_also.demidko.donbas24.muzychnyj_videoklip_kohannja_stal}
% Маріупольська арт-резиденція PostMost відновлює свою діяльність, 11.10.2022
\ii{insert.read_also.demidko.donbas24.mrpl_art_rezidencia_postmost_vidnov_dijalnist}
% Як у Німеччині створюються футболки, присвячені Маріуполю — деталі, 06.09.2022
\ii{insert.read_also.demidko.donbas24.futbolki_nimecchyna_mrpl}
% Поранене та ув'язнене місто у роботах маріупольського художника, 24.10.2022
\ii{insert.read_also.demidko.donbas24.kartyny_oleksandr_lukjanov}
% Полонянка з Азовсталі Валерія Суботіна розкаже про одруження під обстрілами та полон — подробиці, 10.05.2023
\ii{insert.read_also.demidko.donbas24.polonjanka_azovstal_valeria_subotina_rozskazhe_odruzhennja_polon}
% У Києві відбувся захід на підтримку військовополоненої з Азовсталі, 17.12.2022
\ii{insert.read_also.demidko.donbas24.kyiv_zaxid_pidtrym_polonena_azovstal}
% Надихаючі історії про непересічних жінок Приазов'я, 07.03.2023
\ii{insert.read_also.demidko.donbas24.nadyhajuchi_istorii_zhinky_priazovja}
% У столиці захисниця Маріуполя презентувала нові вірші, 14.05.2023
\ii{insert.read_also.demidko.donbas24.u_stolyci_zahysnycja_mrpl_prezent_novi_virshi}
% Юні маріупольці виступили з концертом у Києві, 15.05.2023
\ii{insert.read_also.demidko.donbas24.juni_mrplci_koncert_kyiv}
% Зі сльозами на очах — зустріч, що об'єднала маріупольців, 16.05.2023
\ii{insert.read_also.demidko.donbas24.zi_sljozamy_na_ochah_zustrich_objednala_mrplciv}
% В Німеччині проходять творчі вечори, присвячені Маріуполю, 02.11.2022
\ii{insert.read_also.demidko.donbas24.nimecchyna_tvorchi_vechory_oksana_stomina}
% В українських містах проходять тематичні прогулянки, присвячені Маріуполю, 14.11.2022
\ii{insert.read_also.demidko.donbas24.ukr_mista_tematychni_proguljanky_prysv_mrpl}
% Студентка МДУ розповідає про Маріуполь в Центрально-Європейському університеті, 07.11.2022
\ii{insert.read_also.demidko.donbas24.studentka_mdu_rozpovidaje_mrpl_central_evr_univ}
% Серце Маріуполя — Маріупольський університет — активно допомагає постраждалим маріупольцям, 02.08.2022
\ii{insert.read_also.demidko.donbas24.serce_mrpl_mdu_dopomagaje_mrplci}
% Студент МДУ Олександр Денисов завоював 14 медалей на двох чемпіонатах Німеччини, 14.09.2022
\ii{insert.read_also.demidko.donbas24.student_mdu_oleksandr_denisov_14_medalej}
% «Архів війни» — хто збирає свідчення маріупольців у Києві, 02.03.2023
\ii{insert.read_also.demidko.donbas24.arhiv_vijny_hto_zbyraje_svidchennja_mrplciv_kyiv}
% Рік повномасштабної війни: події, які увійшли в історію України, 24.02.2023
\ii{insert.read_also.demidko.donbas24.rik_vijny}

\ii{insert.read_also.demidko.teatr_okupacia_1}
\ii{insert.read_also.demidko.teatr_okupacia_2}

\ii{insert.read_also.burov.doktor_praskovja_smirnaja}
\ii{insert.read_also.burov.rena_sajenko}
\ii{insert.read_also.burov.istoria_staryj_korpus}
\ii{insert.read_also.burov.kazancev}
\ii{insert.read_also.burov.arihbaeva}
\ii{insert.read_also.burov.ivan_sidorchuk}
\ii{insert.read_also.burov.portnihi}

\ii{insert.read_also.dkm.berkova}
