% vim: keymap=russian-jcukenwin
%%beginhead 
 
%%file slova.civilizacia
%%parent slova
 
%%url 
 
%%author 
%%author_id 
%%author_url 
 
%%tags 
%%title 
 
%%endhead 
\chapter{Цивилизация}
\label{sec:slova.civilizacia}

%%%cit
%%%cit_head
%%%cit_pic
%%%cit_text
В том то и дело, что и это зачаточное «мировое правительство» и негативные
явления, которые приписывают его всепроникающему влиянию – порождены
естественным, логически объяснимым, давно предсказанным развитием \emph{цивилизации}.
Сверхконцентрация капитала в руках узкого круга лиц, супермонополизм –
предсказаны на основании анализа процессов, происходящих в капиталистической
экономике, еще в середине XX века. Соответственно и образование группы сверх
влиятельных владельцев капитала, возможности которых превосходят интересы даже
самых крупных национальных государств – естественный результат
сверхмонополизации экономики
%%%cit_comment
%%%cit_title
\citTitle{Революция Духа – единственный путь спасения России}, 
Юрий Барбашов, voskhodinfo.su, 30.06.2021
%%%endcit

%%%cit
%%%cit_head
%%%cit_pic
%%%cit_text
Возьмем ту же Олимпиаду, там нет русских, ибо нет флага, не звучит гимн, но они
всюду на Олимпиаде. Более того, очевидно, что русские – это основа Олимпиады,
все взоры только к ним… к тем, которых нет.  Сказано, не строй дом свой на
песке, нужен крепкий фундамент, нужна основа, но где же взять ее, когда несет
по жизни без смысла и знания? Современная \emph{цивилизация} погрязла в быте и
достижения прибыли, превратилась в песок времени, отказалась от основ
%%%cit_comment
%%%cit_title
\citTitle{Русских нет нигде, потому что они везде}, Вестник, zen.yandex.ru, 03.08.2021
%%%endcit

%%%cit
%%%cit_head
%%%cit_pic
%%%cit_text
Ни расы, ни нации сами по себе не враждуют, поскольку они не онтологичны и не
транзитологичны — они враждуют лишь каждый в своей \emph{цивилизационной} ориентации,
да еще и не самостоятельно избираемой.  Йозеф Геббельс в свое время утверждал:
«Самое главное это правильно выбрать себе врага».  Враг суть проекции всего
самого плохого, что есть у того, кто создает себе врага. То есть все, что есть
у нас чудовищного, плохого, омерзительного, постыдного нужно собрать и
поместить в другого, которого обозначить врагом. Враг суть антагонист создателя
врага. Враг требует ресентимента: подозрения, зависти, реванша, мести, злобы,
ненависти, агрессии
%%%cit_comment
%%%cit_title
\citTitle{Враг}, Сергей Дацюк, analytics.hvylya.net, 18.11.2021
%%%endcit
