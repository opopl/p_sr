% vim: keymap=russian-jcukenwin
%%beginhead 
 
%%file 30_09_2021.fb.jermolenko_vladimir.1.raznoobrazie_imperia.cmt
%%parent 30_09_2021.fb.jermolenko_vladimir.1.raznoobrazie_imperia
 
%%url 
 
%%author_id 
%%date 
 
%%tags 
%%title 
 
%%endhead 
\subsubsection{Коментарі}

\begin{itemize} % {
\iusr{Tetiana Smiletko}

можливо, парламентська республіка. побільше не олігархічних медіа, як-от ATR. а
ще книжок і культурних продуктів з різних культ просторів. завершення
децентралізації. і побільше українців заохочувати об’єднуватися в ГО, які
репрезентують різні національні спільноти...

\begin{itemize} % {
\iusr{Volodymyr Yermolenko}
\textbf{Tetiana SmiLetko} не певен, що парламентська республіка - це наш
варіант. В такій розмаїтій країні треба це розмаїття балансувати. Парламентська
республіка може довести розмаїття до розпаду.
\end{itemize} % }

\iusr{Марічка Муска}
Збережу для перечитування. Але от написали на тему мого дня)). Теж прийшлося думати про багатонаціональність і розмаїття

\iusr{Ivan Box}

під враженням від «локальна історія», випуск «Назва “Україна”: як з’явилась, що
означає і чи потрібна “Русь” | Ярослав Грицак |»: «…модерн неможливий без
націй…».


\iusr{Mariana Sadovska}
Знову ви точно називаєте те, про що я думаю, сперечаюся , розмовляю останніми днями. Дякую

\iusr{Віталій Імамов}
Як? - відділити державу від ідеологій.
Ідеології, як і віросповідання мають бути особистою справою кожного. При цьому партії можуть бути з ідеологіями. Важливо що рішення, які творить держава не мають обгрунтовуватися ідеологією.

\iusr{Zlat Dubniak}

Мені здається, ключове питання мультикультуралізму в Україні - це російське
питання. І об нього спотикається багато наших прихильників ідеї громадянської
'нації' й ідеї нейтралітету в національному питанні. Вони не розуміють, що
всьому російському в Україні треба ще заслужити повагу, оскільки їхня культура
прийшла на ці землі не діалогом, а війною. А наразі російське може викликати в
українців якщо не ненависть, то дуже холодні почуття. Отже, ще одна особливість
нашого розмаїття в тому, що у нас абсурдною є ідея повної міжнаціональної
толерантності і солодкого єднання з усіма під ряд.

\begin{itemize} % {
\iusr{Volodymyr Yermolenko}
\textbf{Zlat Dubniak} 

це правда. Бо "русскій мір" використовує розмаїття для того, щоб стерти
розмаїття. Використовує свободу для того, щоб знищити свободу. Використовує
неімперське розмаїття для того, щоб утвердити імперське нерозмаїття. - але
водночас я вірю в громадянську націю і політичну націю, яка матиме місце для
україномовних, російськомовних, кримськотатарськомовних, угорськомовних і т.д.
- водночас будучи саме українською політичною нацією. Це складно, але це
можливо

\iusr{Ivan Box}
\textbf{Zlat Dubniak} 

виглядає так, що повага не найбільш точна точка прикладання зусиль (для
порозуміння наприклад. не зрозуміло якої поваги бракує/надлишок до пушкіна чи
достоєвського). адже для мільйонів українців, послуговування російською сталося
не через відсутність/присутність поваги (до російської мови/культури). це
сталося за відсутності вибору й вкорінення традиції. просто (але й небезпечно)
усе пояснити насиллям. білоруське й українське зросійщення можна описати
тотожним насиллям, але ж результати геть різняться.

\end{itemize} % }

\iusr{Yura Gaidai}

Саме сьогодні слухав Ніла Ферґюсона про те, що історія людства є, по-суті,
історією імперій, і що ігнорувати цього не можна, зокрема і в моделюванні
майбутнього. А деколонізація навчальних програм веде до "відміни" історії.

\iusr{Pavlo Shevelo}
А спробуймо ще інакше:
чи дійсно щось КОНКРЕТНЕ перешкоджає збереженню множинності в теперішній республіці Україна (при всій її недосконалості)?
ДІЙСНО?
Якщо так, то що саме?
Пам'ятаймо наскільки важливо правильно поставити (себто зформулювати) запитання.

\begin{itemize} % {
\iusr{Pavlo Shevelo}
уточнення:
ні, мені аж ніяк не йдетьмя про підважування тез обговорюваного посту (постингу), натомість пропоную ословити (вербалізувати) ризики та/або загрози для щонайменше збереження, а ще краще розвитку багатоманіття в Україні.
\end{itemize} % }

\end{itemize} % }
