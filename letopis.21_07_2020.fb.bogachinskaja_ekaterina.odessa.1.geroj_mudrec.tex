% vim: keymap=russian-jcukenwin
%%beginhead 
 
%%file 21_07_2020.fb.bogachinskaja_ekaterina.odessa.1.geroj_mudrec
%%parent 21_07_2020
 
%%url https://www.facebook.com/permalink.php?story_fbid=1179749069040201&id=100010153568113
 
%%author 
%%author_id bogachinskaja_ekaterina.odessa
%%author_url 
 
%%tags chelovek,filosofia,mudrost,obschestvo,odessa,ukraina,zhizn
%%title Героем сегодняшнего времени станет тот, кто сумеет остановится на золотой середине
 
%%endhead 
 
\subsection{Героем сегодняшнего времени станет тот, кто сумеет остановится на золотой середине}
\label{sec:21_07_2020.fb.bogachinskaja_ekaterina.odessa.1.geroj_mudrec}
 
\Purl{https://www.facebook.com/permalink.php?story_fbid=1179749069040201&id=100010153568113}
\ifcmt
 author_begin
   author_id bogachinskaja_ekaterina.odessa
 author_end
\fi

Героем станет тот, кто победит блеск и нищету куртизанки и останется взирать на мир все таким же спокойным взглядом мудреца.

Когда-то одна книга на меня произвела столь сильное влияние, что осталось лишь
название в голове и некий посыл о том, что блеск и нищета всегда где-то рядом 

Название было «Блеск и нищета куртизанки».

Блеск и нищета куртизанки – сколько всего в этих словах. 

Посмотрите вокруг нас сегодня?

\ifcmt
  pic https://scontent-lga3-1.xx.fbcdn.net/v/t1.6435-9/109953725_1179749002373541_4896853330496750327_n.jpg?_nc_cat=110&ccb=1-3&_nc_sid=8bfeb9&_nc_ohc=esfoPwdeNkAAX_JK8Xb&_nc_ht=scontent-lga3-1.xx&oh=c635bb7581f4b4132caea6f6226896c1&oe=60FE1F6C
  width 0.4
\fi

Разве мы не живем в мире внешнего блеска и душевной нищеты?

Разве мы все не превратились в куртизанок, заложников рекламных слоганов,
которые сами же сочинили?

Человек такое существо, что сам не ведает порой до чего доведут его помыслы. 

До блеска, нищеты или остановятся где-нибудь на середине пути?

Героем сегодняшнего времени станет тот, кто сумеет остановится на золотой
середине. 

Он познает нищету, он познает блеск, но ни там ни там не найдет покоя, потому
что истина всегда будет где-то посередине. 

Мы живем в обществе, которое разобщено не столь финансовой разницей в статусах,
сколько нищими и богатыми душами. 

Блеск и нищета куртизанок вокруг слепит и не дает дышать полной грудью. 

Иногда кажется, вот вздохнешь, напьешься полные легкие морского воздуха, а
потом снова приступ удушья, когда видишь разбитое корыто у южных берегов.

И снова блеск памяного селфи ослепит очередную парочку молодых и любопытных
глаз. 

И снова они пройдут мимо, по дороге накладывая фильтр в Instagram. 

Где же взять такой фильтр, чтобы не видеть ни блеска, ни нищеты?

Чтобы оставаться в гуще событий и лишь взглядом прожитого долгую жизнь мудреца
взирать на все это где-то издалека?

- Как долго я живу на этом свете, - промолвил он. - А он все так же равнодушен
ко всему, что они творят.

Герой говоришь?

Да где ж ему взяться то в стране дураков, вечно дерущихся за объедки с
праздничного стола ее былого величия… 

\verb|#Герой_сегодняшнего_времени|

\ii{21_07_2020.fb.bogachinskaja_ekaterina.odessa.1.geroj_mudrec.cmt}

