% vim: keymap=russian-jcukenwin
%%beginhead 
 
%%file 07_11_2020.sites.ru.zen_yandex.asov_alexandr.1.russ_idea_kitezh
%%parent 07_11_2020
 
%%url https://zen.yandex.ru/media/id/5eeaf725a3dca453cfdd4b58/zolotoi-grad-na-holme-ili-sokrovennyi-grad-kitej-russkaia-ideia-5f96b5cb24d0d15a66b32dd4
 
%%author 
%%author_id asov_alexandr
%%author_url 
 
%%tags russkaja_ideja
%%title "Золотой град на холме" или "сокровенный град Китеж"? (русская идея)
 
%%endhead 
 
\subsection{\enquote{Золотой град на холме} или \enquote{сокровенный град Китеж}? (русская идея)}
\label{sec:07_11_2020.sites.ru.zen_yandex.asov_alexandr.1.russ_idea_kitezh}
\Purl{https://zen.yandex.ru/media/id/5eeaf725a3dca453cfdd4b58/zolotoi-grad-na-holme-ili-sokrovennyi-grad-kitej-russkaia-ideia-5f96b5cb24d0d15a66b32dd4}
\ifcmt
  author_begin
   author_id asov_alexandr
  author_end
\fi

\index[rus]{Русь!Русская идея, 07.11.2020}

Русская цивилизация всегда жила великими идеями, искала смыслы своего
существования... И посему у нас, с крушением СССР, с его идеей
интернациональной-коммунистической, образовался вакуум идей... Но Россия не
может так долго жить, она задыхается в безыдейном и бездуховном пространстве.

А разворот к идеям либерального периферийного капитализма, либо, например,
реставрация идей самодержавия, — уже показали свою несостоятельность. Потому
что они не воспринимаются, да и не являются, идеями подлинно великими, кои
только и могут оправдать существование в настоящем и в будущем такого мощного
государства, по-прежнему самого большого по площади и объединяющего множество
столь разноликих народов...

Любая великая цивилизационная идея проистекает из некого сакрального центра и
выстраивает вокруг себя целый мир и материальный, и духовный...

Изначально великие идеи были религиозными, и каждая имела (и имеет) свой центр.
Три мировые религии — христианство, ислам и иудаизм полагают своим духовным
центром Иерусалим, а ещё — возникшие на развалинах Римской империи духовные
центры — Ватикан и падший Константинополь. 

\ifcmt
  pic https://avatars.mds.yandex.net/get-zen_doc/1368767/pub_5f96b5cb24d0d15a66b32dd4_5f970a9e1772f52b507bc647/scale_1200
  caption Рим. Ватикан.
  width 0.4
\fi

Мир ислама имеет духовным центром Мекку, хотя многие значимые события
коранической истории и для мусульман происходят в том же Иерусалиме... Все сии
течения противостоят друг другу, хотя имеют единый корень.

\ifcmt
  pic https://avatars.mds.yandex.net/get-zen_doc/3908958/pub_5f96b5cb24d0d15a66b32dd4_5f970ada1772f52b507c3814/scale_1200
  caption Мекка. Кааба.
  width 0.4
\fi

Более древние духовные традиции чаще имеют центрами не города, а горы: ведизм
всех толков (индуизм, буддизм и т.п.) имеет центром гору Кайласа, или ещё более
древний центр гору Меру где-то у нас на Севере, — нередко полагают, что это
земная ось на полюсе.

Древние греки полагали таковым центром Олимп, а славяне — Алатырь-гору
(Эльбрус) , как образ той же Мировой горы духовного мира.

В наше время, в эпоху индустриальную и постиндустриальную, "золотой миллиард" и
периферийный капитализм своим центром, политическим и духовным (в масонском
смысле), полагают Вашингтон, построенный по образу и подобию языческого
Древнего Рима с величественными храмами, возрождёнными "чудесами античного
мира", форумом и Капитолием. 

\ifcmt
  pic https://avatars.mds.yandex.net/get-zen_doc/3985268/pub_5f96b5cb24d0d15a66b32dd4_5f970b4e24d0d15a6678d6dd/scale_1200
  caption Вашингтон. Капитолий.
  width 0.4
\fi

Противостоять этому центру и сей идеологии в России предполагала Москва, как
наследник Константинополя, т.н. "Третий Рим", "а четвёртому не бывать" (по
утверждению старца Филофея).

В имперский период таковым центром стал Петербург, "Северная Пальмира",
построенная, как и Вашингтон, по масонскому плану, причём образцами для него
стали древние античные города востока, противостоящие "римлянам".

Коммунистическая идеология, имеющая истоком, духовным и историческим,
революционные масоно-языческие течения Запада, и то же франк-масонство в его
радикальном изводе (с идеалами Великой Французской революции), — своим новым
политическим и духовным центром стала полагать революционную Москву.

И именно Красная Москва, стала ориентиром для всего мира, ищущего альтернативу
либерально-капиталистической идеологии, ведущей мир к упадку и войнам. С
развалом же Советского Союза, таковой полюс был утрачен...

\ifcmt
  pic https://avatars.mds.yandex.net/get-zen_doc/3938527/pub_5f96b5cb24d0d15a66b32dd4_5f970bb424d0d15a66799c93/scale_1200
  caption Москва. Кремль.
  width 0.4
\fi

* * *

Что же теперь? Каково место России в сём капиталистическом и либеральном мире?

Для америко-центричной модели мироустройства России уготовано место в лучшем
случае источника ресурсов, мозгов, то есть место дальней колонии. Центр —
"Золотой град на холме", или Новый Рим, то есть Вашингтон; мы же — варвары,
вечная, экзистенциальная угроза "цивилизованному миру", западу. И этого не
изменить — ибо это по сути религиозная догма, символ веры, и сие не зависит от
реальности, от эпохи.

Да, мы ныне — ресурсная колония Запада, попытка же обрести самостоятельность в
сём мире, жёстко пресекается. А посему Запад время от времени, ещё с языческих
времён, бряцая оружием, идёт в поход на "варваров", рождает фашизм.

Какая же идея нам нужна, какая идея уже зарождается или возрождается внутри
нашего общества? Общие черты её уже ясны. Очевидно, что она должна включать в
себя консервативные морально-нравственные ценности, идеи социальной
справедливости, а также сбережения самобытности всех народов нашей страны, и
прежде всего русского народа, как взявшего на себя тяжкую ношу народа
государствообразующего. Это и есть та опора, от коей следует отталкиваться, идя
в будущее...

Этим критериям отчасти отвечала идеология коммунистическая, и большинство тех,
кто жил в Советском Союзе, с ностальгией вспоминают то время. Теперь же идее
"Золотого града на холме" может противостоять лишь идея "Сокрытого града
Китежа", питавшая во все времена все идеологии нашей земли, "русского мира"...
И об этой идее ещё нужно говорить, её нужно раскрывать отдельно...

* * *

В недавно вышедшей книге "Фантастикум..." (М. 2020) есть герой, глава
сверхсекретной спецслужбы, коя ещё с советских времён ведёт наблюдения за
религиозными организациями и масонскими орденами. Он занимается распутыванием
серии преступлений, убийств, кражи таинственных манускриптов, кои совершили
адепты древних учений. И в конце-концов он понимает, что за всеми сиими
преступлениями стоит его давний враг...

И он также размышляет на тему "красной религии", противостоящей "чёрным
учениям"... Приведу здесь фрагмент сей книги, который затрагивает эту тему. 

\subsubsection{Из книги \enquote{Фантастикум} (о \enquote{красной идее})}


\ifcmt
  pic https://avatars.mds.yandex.net/get-zen_doc/3986532/pub_5f96b5cb24d0d15a66b32dd4_5f970f274dcc5c613ccec545/scale_1200
  width 0.4
\fi

\begingroup
   \em
(... )И сейчас он думал, что напрасно сие осталось под спудом, скрывалось
официозом поздней советской идеологии. При том, что практика коммунизма
являлась по сути религиозной: ведь в ней имеется культ вождей и мучеников
революции, также обожествление партии, канонизация партийной символики,
обрядов, праздников...

А раз так, то, может быть, стоило начать модернизировать и саму доктрину, дабы
вернуть красной идее её мистическое измерение, освободив её для духовных
исканий и придав глубину...

(...)А вообще-­то, в сей теме было не так уж много нового. Кое-­что ему было
давно знакомо, к примеру, по курсам философии и истмата («исторического
материализма»). Например то, что коммунистические идеи о всеобщем равенстве, об
отказе от собственности, общности имущества, — древнейшие на земле.
Коммунистическим же называют и общинный строй так называемого «Золотого века»,
«когда боги жили среди людей».

На основе коммунистических идей издревле создавались религиозные секты, их
брали на вооружение повстанческие народные движения, повсюду и во все времена
шедшие под красными знамёнами.

К этим идеям люди возвращались и в античности, и в средневековье. Достаточно
вспомнить «утопических социалистов», и при том католических священников Томаса
Мора, причисленного к лику святых, а также Кампанеллу, — весьма почитаемых
«красными философами». Они занимают почётные места в «пантеоне советских
святых», их имена были высечены на «Обелиске революционным мыслителям»,
установленном в 1918 году в Александровском саду у Кремля по ленинскому плану
монументальной пропаганды. 
\endgroup

\ifcmt
  pic https://avatars.mds.yandex.net/get-zen_doc/1077599/pub_5f96b5cb24d0d15a66b32dd4_5fa6368c8eb5b23a30eb8e70/scale_1200
  caption Обелиск революционным мыслителям. у Кремля в Москве.
  width 0.4
\fi

\begingroup
\em
Кстати, по тому же плану и тогда же были установлены: статуя с обелиском
языческой богине Свободе (Либертас), напоминающей американскую Статую Свободы,
на Советской площади, — там, где теперь стоит статуя Юрия Долгорукого; а ещё
рельеф той же богине на Сенатской башне Кремля, — там, где затем поставили
Мавзолей.

Но что говорить о такой древности, об истоках «красной идеи» в античности и
средневековье! По тому же ленинскому плану устанавливались статуи и таким
предтечам советской «красной религии», как деятели Великой Французской
революции: Марат, Дантон и Робеспьер — первые двое к тому же видные масоны. А
ведь они почитаются и на Западе, как буржуазные революционеры.

* * *

(...)А затем прошлое откликнулось эхом через сто лет, и уже в России. Здесь
якобинцами, сокрушающими церковь, стали большевики­-ленинцы.

Реставрация церкви, обновление её и приспособление для нужд новой власти,
произошло при Сталине, а термидор и контр­революция — при Горбачёве и
Ельцине... И так история катится далее, её трясёт на кочках, и всё уже готово
идти вразнос... 
\endgroup

