%%beginhead 
 
%%file 31_03_2023.fb.mahovska_iryna.hudozhnik.mariupol.1.tetjana_tkachenko_galak_knizhka
%%parent 31_03_2023
 
%%url https://www.facebook.com/iryna.makhovska/posts/pfbid02SUR5ryedajwjfqpGZE5FYPLHJnH7AXtUZtiRaSf8Hue7iZUA18ENwSQ6p3GNTZo7l
 
%%author_id mahovska_iryna.hudozhnik.mariupol,tkachenko_galak_tetjana.mariupol
%%date 31_03_2023
 
%%tags mariupol,kultura,kniga,literatura
%%title Ну, ось вона. Моя перша книжка. Тоненька, недосконала, але щира (Тетяна Ткаченко-Галак)
 
%%endhead 

\subsection{Ну, ось вона. Моя перша книжка. Тоненька, недосконала, але щира (Тетяна Ткаченко-Галак)}
\label{sec:31_03_2023.fb.mahovska_iryna.hudozhnik.mariupol.1.tetjana_tkachenko_galak_knizhka}
 
\Purl{https://www.facebook.com/iryna.makhovska/posts/pfbid02SUR5ryedajwjfqpGZE5FYPLHJnH7AXtUZtiRaSf8Hue7iZUA18ENwSQ6p3GNTZo7l}
\ifcmt
 author_begin
   author_id mahovska_iryna.hudozhnik.mariupol,tkachenko_galak_tetjana.mariupol
 author_end
\fi

I am grateful to
\href{https://www.facebook.com/profile.php?id=100034532776327}{Тетяна Ткаченко-Галак}, 
teacher and just a extraordinary person, for a wonderful book
of poems written with a sincere heart. And I was passionate about making a
cover for this book, with the help of a photo of my painting - our 🇺🇦city of
Mariupol is on it, how beautiful it was before the war. As well as Azovstal and
other places in Mariupol ...

Here is  the text of the author's publication:

* * *

Ну, ось вона. Моя перша книжка. Тоненька, недосконала, але щира. Довго
вагалася, чи потрібна вона взагалі ( дорого, війна, складнощі). Але
усвідомивши, що від мене може нічого не залишитися у цьому світі, вирішила дати
їй життя. Довго шукала і відновлювала свої загублені рукописи. Не всі знайшла.
Не всі публікую. Так само тут нема моїх пісенних текстів. Вони надійно
зберігаються у нашій з  Наталкою Павелко творчості) Хочу, щоб українське слово
множилося, щоб моїй родині було приємно від цієї книжки. Щоб моїм друзям було
тепліше. Щоб не забувся Маріуполь. Саме його вечірній пейзаж ви бачите на
обкладинці, яку для мене люб'язно розробила Iryna Makhovska, за що я їй
безмежно вдячна. Теплі роботи цієї маріупольської художниці так само можна
знайти в якості ілюстрацій. Це краєвиди нашого міста, які вже ніколи такими не
будуть, але навічно закарбувалися в наших серцях. 

Дякую кожному, хто надихав. Люблю вас, друзі.

P.S. Книга видана без спонсорської підтримки, мінімальним накладом. Тому на
всіх не вистачить))  У продажу нема і не буде. Залишу там, де потрібен мій
слід. Для решти зацікавлених - можливо переслати на електронну пошту.

( photos by Тетяна Ткаченко-Галак )

%\ii{31_03_2023.fb.mahovska_iryna.hudozhnik.mariupol.1.tetjana_tkachenko_galak_knizhka.cmt}
