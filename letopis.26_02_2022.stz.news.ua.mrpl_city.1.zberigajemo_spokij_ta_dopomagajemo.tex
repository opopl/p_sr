% vim: keymap=russian-jcukenwin
%%beginhead 
 
%%file 26_02_2022.stz.news.ua.mrpl_city.1.zberigajemo_spokij_ta_dopomagajemo
%%parent 26_02_2022
 
%%url https://mrpl.city/blogs/view/zberigaemo-spokij-ta-dopomagaemo
 
%%author_id demidko_olga.mariupol,news.ua.mrpl_city
%%date 
 
%%tags 
%%title Зберігаємо спокій та допомагаємо!
 
%%endhead 
 
\subsection{Зберігаємо спокій та допомагаємо!}
\label{sec:26_02_2022.stz.news.ua.mrpl_city.1.zberigajemo_spokij_ta_dopomagajemo}
 
\Purl{https://mrpl.city/blogs/view/zberigaemo-spokij-ta-dopomagaemo}
\ifcmt
 author_begin
   author_id demidko_olga.mariupol,news.ua.mrpl_city
 author_end
\fi

Останні декілька днів виявилися найбільш неспокійними для Маріуполя і загалом
України за всю нашу новітню історію. Безмежно вдячна тим людям, які збирають
допомогу для наших військових. Сьогодні вільний простір \enquote{Halabuda} став
найбільшим центром волонтерського руху в Маріуполі. Тут можна пройти курси щодо
надання першої медичної допомоги, навчитися зберігати спокій і допомогати тим,
хто нас захищає!

Як наголосив волонтер і засновник вільного простору \enquote{Halabu\hyp{}da} Дмитро Чичера,
наразі всю діяльність трансформували до 2014 року. Всі раніше заплановані
заходи довелося скасувати, адже зараз всі сили спрямовані на допомогу армії,
тому що це вкрай важливо і для нашої територіальної оборони, і для ЗСУ, і для
національної гвардії. Допомога надається в різних напрямах. Це допомога і
водою, і якимись засобами гігієни, і медикаментами.

\ii{26_02_2022.stz.news.ua.mrpl_city.1.zberigajemo_spokij_ta_dopomagajemo.pic.1}

Водночас, якщо спочатку допомога вільного простору \enquote{Халабуда} була спрямована
на підрозділи на першій лінії оборони, то зараз до волонтерів звернулися
представники територіальної оборони, адже серед мешканців Маріуполя виявилося
чимало охочих захищати власне місто. Голова громадської організації товариства
сприяння обороні України Ігор Казначаєв підкреслив, що \enquote{небезпека гуртує
українців і вже 24 лютого багато маріупольців пішли до воєнкомату та
записувалися в лави тероборони}.

Водночас у вільному просторі \enquote{Халабуда} проводяться курси з надання першої
медичної допомоги, на які можуть потрапити всі охочі маріупольці. Проводить
заняття депутатка Маріупольської міської ради, парамедик Катерина Сухомлинова.

\ii{26_02_2022.stz.news.ua.mrpl_city.1.zberigajemo_spokij_ta_dopomagajemo.pic.2}

Наразі ці заняття є досить затребуваними і своєчасними. Зокрема, директорка
кінофестивалю \enquote{Кіно і ти} Ольга Новікова зазначила, що абсолютно всі
маріупольці і загалом українці повинні в такий складний час пройти подібні
курси, адже ніхто не знає, в якій ситуації він може опинитися вже через 5
хвилин. На заняттях розповідають, як робити штучне дихання, відбуваються
тренування на манекенах, показують як правильно накладати пов'язки, джгут, в
разі проведення ампутації при осколкових рішеннях, як зупиняти кровотечі та
фіксувати пошкодження різного характеру.

\ii{26_02_2022.stz.news.ua.mrpl_city.1.zberigajemo_spokij_ta_dopomagajemo.pic.3}

Також саме у вільному просторі \enquote{Halabuda} збираються волонтери та волонтерки,
які створюють спецодяг для військових. Це не масове виробництво. Роблять все
руками. Заготівлі залишилися ще з 2014-2015 року. Шили їх у Києві, відшивали у
Маріуполі. Нині виготовляють і військові куртки. Вони трошки відрізняються від
спецодягу. Виготовленням одягу та сіток займаються дівчата та жінки Маріуполя.
Важливо, що кожен намагається робити все, що в його силах.

У Маріуполі відсутня паніка і мешканці  намагаються зберігати спокій. Про це
говорять і іноземні журналісти, які зараз перебувають у місті. У Халабуді я
зустріла журналіста французької газети \enquote{LE MONDE} Емануеля Гріспена. Він
наголосив, що \enquote{не бачив  жодного випадку паніки. Так, люди чекають біля
банкомату, біля заправних станцій. Але хаосу ніякого немає. У магазинах є
продукти і все необхідне}. Емануель підкреслив, що його це дуже дивує. Він
порівняв Маріуполь з іншими містами і зазначив, що у французів реакція була б
зовсім інакшою. Французький журналіст залишився вражений реакцією та спокоєм
населення.

\ii{26_02_2022.stz.news.ua.mrpl_city.1.zberigajemo_spokij_ta_dopomagajemo.pic.4}

На жаль, через припинення роботи Нової пошти ускладнилися поставки з інших міст
до Маріуполя. Проте, незважаючи на це і маріупольці, і мешканці з інших міст
та країн дуже активно намагаються допомагати і підтримувати українських
військових.

Дмитро Чичера розповів, що кожного дня відбуваються поставки продуктів, води,
медикаментів для військових. Зараз вже почали писати представники з інших
держав, зокрема представники Польщі, Австралії, Канади, Німеччини та Франції
також намагаються допомагати.

Всі охочі маріупольці можуть допомогти військовим медикаментами, ковдрами,
продуктами  за адресою вулиця Нільсена, 60, де і розташований вільний простір
Халабуда. Або залишити продукти у маріупольських магазинах, які відсьогодні
готові передавати їх військовим.
