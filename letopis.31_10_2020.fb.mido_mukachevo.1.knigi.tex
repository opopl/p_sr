% vim: keymap=russian-jcukenwin
%%beginhead 
 
%%file 31_10_2020.fb.mido_mukachevo.1.knigi
%%parent 31_10_2020
 
%%url https://www.facebook.com/mido.mukachevo/posts/3572978219435662
%%author myroslav dochynec
%%tags knigi,books,kiev
%%title У книг теж буває золотий час
 
%%endhead 

\subsection{У книг теж буває золотий час}

\Purl{https://www.facebook.com/mido.mukachevo/posts/3572978219435662}
\Pauthor{Дочинець, Мирослав}

У книг теж буває золотий час.

Зателефонував незнайомець.

«Я пенсіонер із Києва. Люблю читати».
«Я теж».
«Люблю ваші книги. Хочеться, щоб їх читало більше людей».
«І я не проти».
«Треба щось для цього зробити».
«Я роблю – пишу ці книги».
«Я би хотів долучитися до поширення».
«Не будете ж ви їх носити по електричках…»
«Ні, я їх даруватиму».
«Добра ідея. І книги гарний подарунок, але не такий вже й дешевий. Та ще й для пенсіонера».
«А хіба вартісне буває дешевим?!» – сказав він.
«Не знаю, я не фахівець у сфері цін і тарифів».
«Зате я фахівець. Тому замовляю п’ятсот примірників вашої нової книжки – «Золотий час».
«П’яток примірників?» – перепитав я.
«Ні, п’ятсот».

Книжки вже в Києві. Вони підуть в усі 118 бібліотек столиці. А також для
численних колег, друзів та знайомих родини Олега Суворова.

Не пам’ятаю, з якого разу він нарешті дав мені короткі відомості про себе.
Народився 1968 року під Києвом. Мав вісім років, коли батьки розлучилися.

Виховувався поза родиною. У школі навчався без фанатизму, зате залюбки займався
спортом, у 18 років виконав нормативи кандидата в майстри з важкої атлетики.
Перша професія – збиральник корпусів суден на Ленінській кузні.  Потім армія,
прикордонні війська, там ще й куховарити навчався, бо не було кому. Запрошували
готувати для офіцерів та делегацій. Після армії рік працював у ресторані на
ВДНГ. Вступив до Торгово-економічного університету, за балами успішності став
кращим студентом факультету. 1993 року одержав грант на навчання в США –
фінанси, менеджмент. Закінчив університет Грейсленда. По тому п’ять років
працював менеджером та керівником у різних українських та зарубіжних компаніях.
2000 рік став переломним. Народився син Михайлик.

Започаткував власний маленький бізнес – магазин автозапчастин. Через рік продав
його і разом із дружиною відкрили торгову компанію Фуд Сервіс із продажу
устаткування для харчової промисловості. Цим і займаються впродовж
дев’ятнадцяти років. У колективі 31 чоловік. Імпортують товари з чотирьох
європейських країн. Компанія посідає лідируючі позиції в сегменті. Народилося
ще троє дівчат – Софійка, Марійка, Мирославка. Двоє старших дітей навчаються в
Лондоні, менші захоплюються дизайном і театром. Виростають працелюбними,
уважливими, чуйними. Родина Суворових цінує і береже природу, саджають дерева,
допомагають громаді села під Києвом, де мешкають.

Ось така біографія українця, звичайної-незвичайної української сім’ї.

Є добро на світі, казала моя баба Марта. Бо є доброчинці.

Доброго часу вам, пане Олеже, при доброму здоров’ї біля добрих людей!
