% vim: keymap=russian-jcukenwin
%%beginhead 
 
%%file 21_11_2021.fb.marshalok_igor.ukraina.1.ukraino_ja_v_tebe_virju
%%parent 21_11_2021
 
%%url https://www.facebook.com/permalink.php?story_fbid=1224320238096291&id=100015549784426
 
%%author_id marshalok_igor.ukraina
%%date 
 
%%tags den_gidnosti_svobody,maidan2,revgidnosti,ukraina
%%title Україно, я в тебе вірю!
 
%%endhead 
 
\subsection{Україно, я в тебе вірю!}
\label{sec:21_11_2021.fb.marshalok_igor.ukraina.1.ukraino_ja_v_tebe_virju}
 
\Purl{https://www.facebook.com/permalink.php?story_fbid=1224320238096291&id=100015549784426}
\ifcmt
 author_begin
   author_id marshalok_igor.ukraina
 author_end
\fi

Україно, я в тебе вірю!

21 листопада в Україні згадують про події Революції Гідності (2013) і
Помаранчевої революції (2004), відзначаючи День Гідності та Свободи

СІМ років тому, 21 листопада 2013 року, в Україні почався Євромайдан. У той
листопадовий четвер сотні людей вийшли на Майдан Незалежності з мирним
протестом проти раптового скасування колишньою владою наміру підписати Угоду
про асоціацію з Європейським Союзом. Чому почався Євромайдан. Євромайдан в
Україні почався після того, як український уряд під керівництвом Миколи
Азарова вирішив призупинити процес підписання Угоди про Асоціацію з Євросоюзом 

\ii{21_11_2021.fb.marshalok_igor.ukraina.1.ukraino_ja_v_tebe_virju.pic.1}

Під час правління Януковича влада любила забороняти проведення опозиційних
мітингів під різними приводами за допомогою рішень судів. Так сталося і цього
разу - Окружний адміністративний суд Києва заборонив проведення масових акцій
в центрі Києва до 7 січня 2014 року. Однак, в цей же день ввечері, незважаючи
на оперативно ухвалену судом заборону, на Майдані Незалежності в Києві
зібралися більше тисячі осіб.  

Саміт у Вільнюсі відбувся 28-29 листопада, і по мірі його наближення акції
протесту в країні спадали. І хоча багато хто до моменту закриття саміту
сподівався, що українська влада передумає і все-таки підпише УА з ЄС, цього не
сталося. А вже в ніч на 30 листопада, відразу після завершення Вільнюського
саміту, сталося те, чого ще не було в історії незалежної України. Силовики
спецпідрозділу «Беркут» отримали наказ розігнати протестувальників, які все ще
залишалися на Майдані. «Беркутівці» серед ночі під час силового розгону побили
десятки активістів, переважно молодь.

Згодом влада пояснювала рішення розігнати протестувальників необхідністю
підготувати площу до новорічних свят, зокрема - встановити ялинку. Тією
самою, кістяк якої потім став одним із символів Євромайдану 

Такий жорстокий спосіб розправи з незгодними, небачений Україною раніше, не
залишив байдужими мільйони українців. І вже з самого ранку після «кривавої
ночі» на Майдані 30 листопада, обурені свавіллям влади громадяни почали
збиратися на Михайлівській площі. 

Народне Віче у Києві 1 грудня зібрало, за
різними оцінками, від 500 тисяч до 1 мільйона осіб. Також на акції протесту у
зв'язку з розгоном Євромайдану люди вийшли у багатьох містах України. Так
Євромайдан переріс у Революцію гідності. 

Протестуючі змогли "відбити" у міліціонерів Майдан Незалежності, де київські
комунальники вже почали встановлювати новорічну ялинку та готувати місце для
ярмарку. Охороняти площу від мітингувальників поставили бійці Внутрішніх
військ, якими переважно були солдати строкової служби. Вони особливого опору не
чинили. 

Зайнявши Майдан, протестувальники звели барикади, щоб запобігти повторенню
розгону, що стався в ніч на 30 листопада. Однією із знакових подій початку
Революції гідності стала дата 11 грудня. Тоді, вночі, спецназівці знову
спробували розігнати протестувальників. У ніч на середу, коли силовики знову
штурмували Майдан, вперше за кілька сотень років дзвони Михайлівського собору
забили на сполох.

Я ніколи не забуду ніч одинадцятого грудня. На Майдані не так багато
протестувальників, а тут черговий штурм силовиків. Задзвонили Михайлівські
дзвони, і  відвертий, майже охриплий голос Руслани: Києве! Вставай!
Прокидайся!  І Бог почув заклики і треба було бачити як з усюд йшли люди.
Відстояли, Так почалася Революція Гідності. Пам ятаймо і боронімо нашу Україну:
ЗАВЖДИ.

Тому треба всім добре подумати, чи зумієм ми отак просто віддати все, набуте
віками нашими пращурами? Чи не буде соромно перед ними, які мужньо й хоробро
відстоювали свою державність, які йшли до кінця, які не здавались, які
врешті-решт перемогли і вибороли ту незалежність. 

Недарма Ліна Костенко казала: «Якогось ранку прокинемось зовсім в іншій
державі. Бо проспали свою».

Як там у пісні: 

\obeycr
Не спи, моя рідна земля!
Прокинься, моя Україно!
Відкрий свої очі у світлі далеких зірок!
Це дивляться з темних небес
Загиблі поети й герої –
Всі ті, що поклали життя за майбутнє твоє.
\restorecr

Давайте не будемо розчарувати тих, що дивляться на нас з небес, давайте не
проспимо нашу Україну. У нас є МРІЯ, і в нас є СИЛИ...  

Доведемо вкотре незламність нашого духу!!! Україно, я в тебе вірю!
