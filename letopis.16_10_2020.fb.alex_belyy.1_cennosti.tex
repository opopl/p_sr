% vim: keymap=russian-jcukenwin
%%beginhead 
 
%%file 16_10_2020.fb.alex_belyy.1_cennosti
%%parent 
 
%%endhead 

\subsection{Наши Ценности}
\label{sec:16_10_2020.fb.alex_belyy.1_cennosti}

Вектор внешней политики всегда влиял на внутренние процессы. Понятие «западные
ценности» у нас появилось ещё при Горбачёве. Многие СМИ начали очернять все,
чем тогда жила наша страна, и на перебой начали нас убеждать, что западные
ценности и есть идеал существования человечества.

После обретения независимости Украина  не раз меняла свой курс и на сегодняшний
день вроде бы определилась, давайте посмотрим, что же такое эти «ценности».
Демократия, выборность власти на альтернативной основе - основа политического
устройства в западном мире.  Но и здесь различия очень большие.  Президентская
или парламентская система власти и принципы избирательного законодательства
делают понятие «западные ценности» каким-то очень размытым.  Прямые выборы
главы государства есть не во всех странах.  Самая запутанная система выборов в
США.  Там президента выбирают не люди и не депутаты, а выборщики.  Такого нет
больше нигде.  И зачастую кандидат, набравший меньше голосов, становится
лидером.  Но это ещё не все ценности, к которым нас так усиленно вели.  Свобода
слова оказывается только на словах.  Цензура в СМИ хуже, чем в последние годы
Союза.  Пропаганда насилия в кино и на телевидении стала нормой.  Но и наука не
остаётся в стороне.  Переписывание истории носит кощунственный характер.  Наши
деды уже не герои, а оккупанты, фашизм победили не они.  Уже дошло до того, что
некоторые страны пытаются ещё себе и деньги требовать за то, что их освободили
от фашизма.  Национализм и насильственная украинизация в нашей стране - это
тоже не без западной помощи, и это тоже «западные ценности».  Уровень
пропаганды стал просто Сталинским.  Что же нам ещё навязывают под лозунгом
«западные ценности»?  Отказ от семейных ценностей, понятие «мама» и «папа» уже
стали нарицательными на Западе.  Пропаганда нетрадиционной культуры стала очень
агрессивной, тех, кто с ней не согласен, обвиняют в ущемлении прав меньшинств.
Но это меньшинство по факту стало диктовать правила большинству.  В Европе и в
ряде других стран закрываются церкви, зато массово открываются гей-клубы.  Есть
ещё одна особенность этих ценностей.  Одна - для себя, другая - на экспорт.
Тут вообще все весело.  Они всегда поддерживают любых политиков, кто
обслуживает их интересы.  Даже откровенные нацисты им подходят.  Вы не
заметили, что как только кого-то из чиновников или политиков обвиняют в
преступлениях, в основном, в коррупционных,  они начинают вопить, что их
преследуют по политическим мотивам и валят на Запад.  Очень наши не бедные
граждане любят бежать в Англию.  Их там с удовольствием принимают и дают
защиту.  Таких фамилий просто масса.  Где и как они заработали свои миллионы -
их не спрашивают.  И этими ценностями нас пичкают уже много лет.  Как только мы
стали на путь этой западной ловушки, мы стали терять свою экономику, свою
мораль и свои традиции.  Нас становится все меньше и меньше, а жизнь наша
превратилась в выживание.  Мы живем не так, как мы привыкли и хотим, а как нам
велят эти «западные ценности».  «Право сильного» они используют не стесняясь.
Продажность элит и их зависимость от Запада ставит нас в положение рабов.
Права у нас забрали, а обязанности нам добавили.  Нам уже и бусы не дают, а
зачем?  Мы уже все сдали без боя.  Если ты сам запрягся, не удивляйся, что тебя
считают лошадью и погоняют. Да, кнут нам оставили, а вот пряник - только
избранным.  Но и у них начались необратимые процессы, в США и Европе - кризис в
экономике и кризис в политической системе.  Там все больше ситуация напоминает
нашу перед развалом СССР.  Массовые беспорядки и кровь на улицах уже никого не
удивляют.  Расслоение общества проходит не только по экономической, но и по
идеологической разнице во взглядах.  Либеральный мир теряет своё влияние, и на
его место приходят где-то традиционные, а где-то радикальные взгляды.  Мир на
пути перемен, куда пойдём мы?  

Где наше место в мире? 

Что и как будут учить наши дети, на каком языке нам говорить?  

Может, сами будем решать, как нам
жить?  Может, ну их, эти «ценности», обойдёмся без них.  

Алекс Белый
