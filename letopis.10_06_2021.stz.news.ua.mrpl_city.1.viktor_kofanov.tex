% vim: keymap=russian-jcukenwin
%%beginhead 
 
%%file 10_06_2021.stz.news.ua.mrpl_city.1.viktor_kofanov
%%parent 10_06_2021
 
%%url https://mrpl.city/blogs/view/pamyati-viktora-kofanova
 
%%author_id demidko_olga.mariupol,news.ua.mrpl_city
%%date 
 
%%tags 
%%title Пам'яті Віктора Кофанова
 
%%endhead 
 
\subsection{Пам'яті Віктора Кофанова}
\label{sec:10_06_2021.stz.news.ua.mrpl_city.1.viktor_kofanov}
 
\Purl{https://mrpl.city/blogs/view/pamyati-viktora-kofanova}
\ifcmt
 author_begin
   author_id demidko_olga.mariupol,news.ua.mrpl_city
 author_end
\fi

\ii{10_06_2021.stz.news.ua.mrpl_city.1.viktor_kofanov.pic.1}

8 червня пішов з життя талановитий художник і графік України і зокрема
Маріуполя \emph{\textbf{Віктор Іванович Кофанов}}. Про це на своїй сторінці у Facebook
повідомив маріупольський мистецтвознавець та колекціонер, друг художника
\emph{Олександр Михайлович Чернов.} Він наголосив, що 
\begin{quote}
\em\enquote{останнім часом художник тяжко
хворів... Віктор Іванович так мріяв про виставку, присвяченої його 80-річчю...
Маестро був добрим, щедрим і гостинним чоловіком. Близько 100 своїх робіт він
подарував нашому Художньому музеї ім. А. І. Куїнджі, багато робіт і в моєму
зібранні...}. 
\end{quote}
Саме спогади та знання Олександра Михайловича Чернова допомогли
мені при написанні цієї статті.

Народився Віктор Іванович 6 березня 1941 року в Маріуполі. Середню освіту
здобув у маріупольській середній школі № 41. З раннього дитинства мріяв стати
художником. Батьки ставилися до цього без ентузіазму, однак і не заважали сину.

Початкову художню освіту отримав в художній студії при Палаці культури ім. К.
Маркса у Кальміуському (тоді Іллічівському) районі. Першим вчителем став
місцевий художник і викладач \emph{Ктиторов Микола Юхимович}, який зумів виховати
багатьох професійних художників. Втім починав працювати Віктор Кофанов не як
художник, а на місцевому заводі важкого машинобудування слюсарем. Згодом відбув
службу в лавах радянської армії та демобілізувався у 1963 році. Професійну
художню освіту отримав в Саратовському художньому училищі у 1964–1968 роках. В
одному з інтерв'ю Віктор Іванович розповів: 

\begin{quote}
\em\enquote{Я вступав до Ворошиловградського
відділення в художники. Все здав на 5, а диктант, по-моєму, написав на трійку.
Директор сказав: \enquote{Нам не грамотні художники не потрібні}. Я забрав документи,
поїхав. Потім вступав в Рязань після армії, з Рязані перейшов в Саратов}.
\end{quote}

\ii{10_06_2021.stz.news.ua.mrpl_city.1.viktor_kofanov.pic.2}

У Саратові його вчителями стали \emph{Михайло Іванович Просянкін} та \emph{Валентин
Сергійович Успенський}. Всіх своїх педагогів Віктор Іванович завжди згадував з
вдячністю. Саратовське художнє училище закінчив з відзнакою. Захистивши диплом
на \enquote{відмінно}, Кофанов відправився в Ленінградську академію мистецтв. Але там
іспит з живопису склав на \enquote{три}. Екзаменаційну картину забрав до Маріуполя.
Незважаючи на важкий старт, художник згодом досяг великих результатів.
Повернувшись до Маріуполя, працював у Бюро наочної агітації заводу \enquote{Тяжмаш}.

\ii{10_06_2021.stz.news.ua.mrpl_city.1.viktor_kofanov.pic.3}

З 1972 року Кофанов брав активну участь у художніх виставках різного рівня
(міських, обласних, республіканських, всесоюзних). Перший успіх прийшов до
Кофанова в 1972 році, коли він взяв участь у Всесоюзній художній виставці
молодих художників в Москві. А вже наступного року його гравюра \emph{\enquote{Вулиця А. І.
Куїнджі в Жданові}} була з успіхом представлена на республіканській художній
виставці \emph{\enquote{Меморіал А. І. Куїнджі}}, що пройшла в його рідному Маріуполі. Сьогодні
робота  є експонатом Художнього музею ім. А. І. Куїнджі. Успіх надихає
художника, він продовжує плідно працювати. На багатьох республіканських
виставках, він показує свої нові роботи – індустріальні пейзажі: \enquote{Конвертерний
цех будується} (1975 рік), \enquote{В котловані} (1975 рік), \enquote{Крани в судноремонтному}
(1977 рік) та багато інших. Згодом Віктор Іванович був прийнятий до місцевих
художньо-виробничих майстерень художнього фонду України.

Був у творчих відрядженнях в Прибалтиці, Узбекистані, на острові Хортиця, на
півночі європейської Росії. Звідусіль художник привозив численні малюнки,
етюди, акварелі. На початковому етапі більше працював художником-графіком.
Відмінною рисою в його творчості є володіння технікою кольорового офорта –
рідкісного і складного способу гравіювання. Загалом Віктор Іванович чудово
працював у всіх техніках графіки: офорт (чорно-білий), кольоровий офорт,
акварель, пастель,  малюнок, ксилографія, ліногравюра, гравюра на картоні.
Активно працював у цьому напрямі до середини 1980-х років. З 1992 року стає
членом Національного союзу художників України.

% pushkin
\ii{10_06_2021.stz.news.ua.mrpl_city.1.viktor_kofanov.pic.4}

Віктор Іванович дуже любив поезію. Цікаво, що художник і сам писав вірші.
Найбільше його надихав Олександр Сергійович Пушкін. У 1980 –1990-их роках
народилася ціла серія картин, присвячена великому поетові. Художник працював
над нею довго, відповідально і натхненно. Як результат – образи Пушкіна в різні
моменти його життя і завжди в оточенні природи, причому в усі пори року. Назви
картин говорять самі за себе: \enquote{Похмуриий час}, \enquote{Місячна ніч}, \enquote{Зимовий вечір},
\enquote{На Чорній річці}, \enquote{Жовтень}. На полотнах поет і природа живуть в гармонії.

У 1997 році художник починає проводити персональні виставки. Довгий час Кофанов
залишався одним з останніх могікан худож\hyp{}ників-графіків. Останнім часом віддавав
перевагу живопису. Улюблені жанрами стають пейзаж і натюрморт. Своєрідною
колірною метафорою домагається мальовничої декоративності. Картини і гравюри
Віктора Івановича неодноразово публікувалися в журналах і газетах.

Мистецтвознавець Олександр Чернов називав Віктора Івановича \emph{\textbf{символом
маріупольського живопису}}. Він зазначив, що 

\begin{quote}
\em\enquote{залишаючись переконаним
традиціоналістом, художник дотримувався методу реалістичної школи, при цьому
вмів творчо осмислити традицію і досить вільно нею розпорядитися. Його живопис
оригінальний, пізнаваний. Перші роботи відрізнялися стриманою гамою, сюжети
художник вибирає непомітні. З роками палітра освітлюється, з'являється чистий,
локальний колір. У композиції пейзажів з'являється епічний образний лад.
Трактуванням пейзажного мотиву з широким охопленням простору художник створює
відчуття неминущої величі природи. І тоді, здавалося б, найзвичайнісінькі
мотиви, які ми інколи не помічаємо, у нього піднімаються до поетичної висоти.
Однак не завжди художник досягав бажаного результату і тоді починалися \enquote{муки
творчості}. Майстер знову і знову повертався до свого твору}.
\end{quote}

\ii{10_06_2021.stz.news.ua.mrpl_city.1.viktor_kofanov.pic.5}

Серія робіт, на яких зображені Маріуполь, Приазов'я, Узбекистан, Литва –
відображають величезну любов автора до природи. Громадський активіст і художник
Микола Шевченко підкреслив, що 

\begin{quote}
\em\enquote{картини Кофанова настільки відображають майстра
і настільки мають власну структуру, власний формат, що його сплутати з кимось
просто неможливо}.
\end{quote}

\ii{10_06_2021.stz.news.ua.mrpl_city.1.viktor_kofanov.pic.6}

Наразі роботи Віктора Івановича зберігаються у фондах Маріупольського
краєзнавчого музею та в приватних українських і зарубіжних колекціях. У
Кофанова бажання творити було внутрішньою потребою. Одного разу в інтерв'ю його
запитали: пішов би Віктор Іванович в художники, якби знав, який тернистий шлях
на нього чекає? 

\begin{quote}
\em\enquote{Якби в юності все знав заздалегідь, все одно б пішов в художнє
училище. Чи багато художнику треба? Мені б знати, що є надійний дах над
головою, та проблеми б життєві не терзали, і – творити!} 
\end{quote}

– відповів він... Творити – було життєвим кредом художника. Сьогодні талант
маріупольця продовжить жити в його роботах. Сподіваюся, що найближчим часом
можна буде відвідати виставку робіт, присвячену пам'яті унікального художника
Віктора Івановича Кофанова.

\emph{\textbf{Представлені картини зберігаються в Особистому архіві О. М. Чернова.}}
