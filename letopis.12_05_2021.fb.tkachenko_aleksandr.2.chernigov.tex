% vim: keymap=russian-jcukenwin
%%beginhead 
 
%%file 12_05_2021.fb.tkachenko_aleksandr.2.chernigov
%%parent 12_05_2021
 
%%url 
 
%%author 
%%author_id 
%%author_url 
 
%%tags 
%%title 
 
%%endhead 
\subsection{Сьогодні відбулось перше виїзне засідання Уряду у Чернігові}
\label{sec:12_05_2021.fb.tkachenko_aleksandr.2.chernigov}

Сьогодні відбулось перше виїзне засідання Уряду у Чернігові. Поїздки в регіони
завжди потрібні, адже комунікація на місцях дозволяє зрозуміти стан справ для
прийняття подальших рішень. 

І перше, з чого хочеться почати: Чернігів недооцінений, як туристичне, так і
культурне місто. Факт, що історія та Чернігівщина – слова-синоніми, – гадаю,
вам відомий. Але чи переконались ви в цьому особисто? Якщо ні – змініть це не
затягуючи. Сам був тут давно, проте точно вирішив приїхати сюди з малим
найближчим часом.


\ifcmt
tab_begin cols=2

  pic https://scontent-frt3-1.xx.fbcdn.net/v/t1.6435-9/186484345_4101343779933193_7598894699348433407_n.jpg?_nc_cat=109&ccb=1-3&_nc_sid=730e14&_nc_ohc=DmwkLIgr4pcAX_pFkEr&_nc_ht=scontent-frt3-1.xx&oh=2df79fb1c48521834ab73592cd80e3d4&oe=60C474BB

	pic https://scontent-frt3-1.xx.fbcdn.net/v/t1.6435-9/186037480_4101344253266479_4387331951717046540_n.jpg?_nc_cat=106&ccb=1-3&_nc_sid=730e14&_nc_ohc=I10aP1JABHMAX8Ze1Kl&_nc_ht=scontent-frt3-1.xx&oh=91d505fcfee348e70e3b38f9956c7f45&oe=60C65DC1

tab_end
\fi


✅ Дитинець – сакральне місце для кожного чернігівця. Розглядаємо його реконструкцію, як один із перших об’єктів у межах Великої Реставрації. Відчути дух історії – от за чим однозначно варто їхати сюди. А маршрут вже спланований Національним архітектурно-історичним заповідником «Чернігів Стародавній»:
📍 Спасо-Преображеньский собор, як і Софія Київська, один із найдавніших в Україні. Стіни збереглись у початковому вигляді навіть після Другої світової війни. Як і внутрішня архітектура розташування приміщення. Заходиш і усвідомлюєш відлуння часу. 
📍Будинок полкової канцелярії – унікальна споруда, бо зберігалась у первісному вигляді. Так само потенційно один із перших об’єктів Великої Реставрації. Проблема №1 – теперішній стан вимагає негайних реставраційних робіт. І проблема №2 – 100 тис музейних предметів, на жаль, залишаються на полицях приховані від загалу через відсутність належного приміщення для їхньої демонстрації. Серед них понад 150 гетьманських універсалів, колекції стародруків, родинні архіви. Хочемо це змінити. Вже працюємо над пошуком потрібного для фондів приміщення. 
📍 Чернігівці будували дуже вправно. Ще одним свідченням цього є Борисоглібський собор. На відміну від Спасо-Преображеньського – пережив кілька значних руйнувань і реставрацій, через що змінював свій зовнішній вигляд. Але наразі пам’ятку відновлено в автентичному вигляді. Усередині справжній європейський шедевр – срібний іконостас, побудований за кошти Івана Мазепи.
📍 Відреставрований після пожежі у роки Другої світової війни Колегіум – шедевральний зразок українського бароко і надгарна архітектурна споруда міста. А ще Катерининська і П’ятницька церкви, Єлецький монастир, Троїцько-Іллінський монастир та інші пам’ятки архітектури національного значення. 
📍Окремо звертаю увагу на Антонієві печери з підземною церквою, де ховали ченців-затворників. Археологи до сих пір вражені цим унікальним збереженням.
✅ Про Чернігів культурний теж є чимало розповісти. 
📍Чернігівський історичний музей ім. Василя Тарновського – єдиний музей всього чернігівського краю і один з найстаріших музеїв країни. Фондосховище експонатів – надрідкісне. Понад 160 тисяч об’єктів, серед яких і цінності національного значення. Наприклад, перша підвіска князя Мстислава, вулик Петра Прокоповича, найбільша колекція Шевченкіани з автографом Тараса Григоровича, кіот чернігівської ікони Іллінської Божої Матері та багато інших унікальних об’єктів.
📍 Чернігівський драмтеатр ім. Тараса Шевченка – колектив та команда використовують наявний ресурс на повну і хизуються аншлагами. У що щиро вірю. До речі, театр має гарний досвід співпраці з УКФ: вистава «Небезпечні зв’язки» поставлена завдяки гранту фонду.
📍Чернігівський фаховий музичний коледж ім. Левка Ревуцького став альмаметір’ю для багатьох видатних музикантів. Директор настільки вболіває за своїх учнів, що власними зусиллями започаткували у коледжі музей. Такі підходи дуже надихають.
📍Філармонійний центр фестивалів та концертних програм. Об’єкту пощастило більше, тому що реставраційні роботи розпочато ще минулого року. На черзі – фасад, заміна покриття тротуару. Далі – глядацька зала, оновлення звукопідсилюючого, освітлювального, мультимедійного обладнання, заміна театральних крісел та облаштування місць перебування осіб з інвалідністю.


\ifcmt
tab_begin cols=2
  pic https://scontent-frx5-1.xx.fbcdn.net/v/t1.6435-9/185513385_4101344426599795_4239979896685380344_n.jpg?_nc_cat=105&ccb=1-3&_nc_sid=730e14&_nc_ohc=XUYop7xmfeMAX8ATx5I&_nc_ht=scontent-frx5-1.xx&oh=443662cdcbf3f469945f0aaf9957c7b6&oe=60C74CCE

	pic https://scontent-frt3-1.xx.fbcdn.net/v/t1.6435-9/186031915_4101346449932926_2587992093334935330_n.jpg?_nc_cat=109&ccb=1-3&_nc_sid=730e14&_nc_ohc=qBi4nPzKys0AX_78cpw&_nc_ht=scontent-frt3-1.xx&oh=f0aa399151fd442f8f47eb30cf757ba7&oe=60C53111

	pic https://scontent-frt3-1.xx.fbcdn.net/v/t1.6435-9/185716787_4101348406599397_4795890632619551393_n.jpg?_nc_cat=108&ccb=1-3&_nc_sid=730e14&_nc_ohc=Gjv3WmysA6wAX9RL31s&_nc_ht=scontent-frt3-1.xx&oh=cee4909b9e9cfa05c7a754826ac9dcbc&oe=60C787AD

	pic https://scontent-frt3-1.xx.fbcdn.net/v/t1.6435-9/186295419_4101348349932736_2573050675223845047_n.jpg?_nc_cat=107&ccb=1-3&_nc_sid=730e14&_nc_ohc=GFGfF3xMTUMAX9VQRRg&_nc_ht=scontent-frt3-1.xx&oh=6628e1dac6968a7c824fcb782505a7f6&oe=60C69B5D

tab_end
\fi

