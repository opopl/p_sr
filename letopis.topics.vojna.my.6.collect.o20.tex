% vim: keymap=russian-jcukenwin
%%beginhead 
 
%%file topics.vojna.my.6.collect.o20
%%parent topics.vojna.my.6.collect.gen
 
%%url 
 
%%author_id 
%%date 
 
%%tags 
%%title 
 
%%endhead 

Краткая история первой русско-украинской войны
Война между РФ и Украиной началась не в феврале 2022, а в феврале
2014 года, когда воспользовавшись гражданскими беспорядками в Украине
президент РФ направил российские войска для блокировки баз ВСУ в Крыме.
Последующий т.н. референдум “легализировал” военную оккупацию.
Почти бесшумный захват полуострова ободрил Кремль и на Донбасс были
направлены стрелковы, чтобы поднять ”народное восстание" против
киевских “нацистов” и очередным референдумом присоединить Донецкую и
Луганскую область - тем самым создав коридор к Крыму.
Но тут украинцы очухались от своих внутренних разборок и стали так
бить “донецких сепаратистов”, что пришлось на помощь им посылать
российских отпускников-бурятов и весь российский военторг.
Почему Путин тогда не послал в открытую российскую армию на
“спецоперацию” не ясно, но возможно, что свою роль сыграл не вовремя
сбитый российским БУКом малазийский Боинг.
Может быть Кремль испугался, что открытое вторжение российской
армии на Донбасс вкупе со сбитым гражданским Боингом приведет к прямому
военному вмешательству НАТО, к чему Россия ни тогда, ни сейчас не
готова.
После этого война вступила в состояние “странной” и находилась в
этом состянии почти 8 лет.
У Украины не было достаточно сил, чтобы выпихнуть “сепаратистов”
туда, откуда они пришли плюс Запад уговаривал Киев быть хорошим и
плюшевым, а Москва все надеялась, что Украина вот-вот встанет на колени и
вернется в московские конюшни...
Почему Путин решил в конце 2021 решить вопрос военным путем и
перевел странную войну на горячие рельсы знает наверно только он, его
подушка и советник его подушки.
Может быть ВВП убедил сам себя, что Зеленский клоун, что киевское
правительство абсолютно недееспособно и что украинская армия побежит при
первом же звуке российских сапог.
Может быть ВВП действительно серьёзно болен и решил напоследок
оставить свой коричневый след на страницах истории.
Может быть РПЦ убедила своего верного сына и полковника, что армия
это единственный способ вернуть неблагодарных хохлов в лоно русской
православной церкви.
О реальной причине мы можем сейчас только гадать.
Украина смогла не только устоять, но и оттогнать российские войска
от своей столицы, война по большей части превратилась в позиционную и
РФ начинает терять свое преимущество в тяжелой артиллерии и воздушных
силах.
Пока молчаливое большинство в РФ молчит и вопрос заключается
только в том - как долго оно будет молчать и отводить свои глаза от
свежих могил на российских кладбищах.
Как долго продлится эта первая русско-украинская война сейчас
никто сказать не может.
У Украины есть цель - изгнать агрессора со своей территории и
начать восстанавливать свою страну.
У России в этой войне нет ни одной конкретно сформулированной и
достижимой цели......а войну без цели даже если и есть ракеты выиграть
не возможно.

