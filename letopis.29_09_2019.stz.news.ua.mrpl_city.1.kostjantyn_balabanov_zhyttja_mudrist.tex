% vim: keymap=russian-jcukenwin
%%beginhead 
 
%%file 29_09_2019.stz.news.ua.mrpl_city.1.kostjantyn_balabanov_zhyttja_mudrist
%%parent 29_09_2019
 
%%url https://mrpl.city/blogs/view/kostyantin-balabanov-zhittya-osyayane-mudristyu
 
%%author_id demidko_olga.mariupol,news.ua.mrpl_city
%%date 
 
%%tags 
%%title Костянтин Балабанов: життя, осяяне мудрістю
 
%%endhead 
 
\subsection{\large Костянтин Балабанов: життя, осяяне мудрістю}
\label{sec:29_09_2019.stz.news.ua.mrpl_city.1.kostjantyn_balabanov_zhyttja_mudrist}
 
\Purl{https://mrpl.city/blogs/view/kostyantin-balabanov-zhittya-osyayane-mudristyu}
\ifcmt
 author_begin
   author_id demidko_olga.mariupol,news.ua.mrpl_city
 author_end
\fi

\begin{quote}
\em З нагоди 70-річчя ректора Маріупольського державного університету Костянтина
Васильовича Балабанова.	
\end{quote}

\ii{29_09_2019.stz.news.ua.mrpl_city.1.kostjantyn_balabanov_zhyttja_mudrist.pic.1}

4 жовтня відзначає свій ювілей ректор Маріупольського державного університету,
доктор політичних наук, талановитий педагог, дипломат, громадський діяч,
справжній патріот і просто чудова людина Костянтин Васильович Балабанов.
Сучасний вигляд Маріуполя і загалом Приазов'я вже неможливо уявити собі без
МДУ, а сам університет без його ректора. Життєвий шлях нашого героя дійсно
осяяний мудрістю, адже високий професіоналізм в роботі, вміння переконувати
будь-якого співрозмовника і відстоювати свою точку зору в кабінетах найвищого
рівня дозволило Костянтину Васильовичу домогтися значних успіхів у всіх сферах
діяльності. Робота для ректора – не професія, а покликання. Кожне його слово
вагоме і цінне. Після зустрічі з цією невтомною, енергійною людиною переймаєш
невичерпне завзяття й ентузіазм, які вона випромінює.

\ii{29_09_2019.stz.news.ua.mrpl_city.1.kostjantyn_balabanov_zhyttja_mudrist.pic.2}

Народився майбутній науковець у грецькій родині шанованих освітян в с.
Малоянисоль Нікольського району Донецької області. Незаперечним авторитетом в
сім'ї був батько – Василь Васильович Балабан – директор Малоянисольської школи.
Фронтовик, людина кришталевої чесності, який виховав цілу плеяду талановитих
педагогів, присвятивши своє життя педагогічній діяльності. Мати – Раїса
Амвросіївна – понад півстоліття працювала вчителем молодших класів. Батьки
намагалися прищепити своїм дітям – Ларисі та Любі, Кості та Василю – любов до
книг та жагу до знань.

\ii{29_09_2019.stz.news.ua.mrpl_city.1.kostjantyn_balabanov_zhyttja_mudrist.pic.3}

Змалечку Костя був допитливим і старанним хлопцем. З усіх предметів найбільше
полюбляв історію. Цікаво, що з дитинства хлопець вмів з легкістю поєднувати
кілька справ, при цьому йому вдавалося усюди досягати високих результатів. Так,
проводячи багато часу в оточенні книг, він встигав і виконувати свої обов'язки
по дому, і розважатися з друзями. Після закінчення школи зі срібною медаллю
Костянтин вирішив вступати на історичний факультет Донецького державного
університету. Подолавши складний конкурс, хлопець стає студентом, який
відрізнявся від інших неабиякою амбітністю та надмірною відповідальністю. У цей
період весь час хлопця забирали студентське братство, навчання та книги. Попри
юний вік, Костянтин відразу стає лідером академічної групи, яка згодом два роки
поспіль стає кращою в університеті.

\ii{29_09_2019.stz.news.ua.mrpl_city.1.kostjantyn_balabanov_zhyttja_mudrist.pic.4}

У 1970 р. Костянтин Балабанов закінчив університет, після чого працював
учителем історії в сільській середній школі. У системі вищої освіти з 1982 р. –
асистент, старший викладач, доцент в ДонДУ. Сфера наукових інтересів Костянтина
Балабанова охоплює близьку йому діяльність молодіжних організацій. У 1984 році
закінчив аспірантуру ДонДУ, а в 1992-му – докторантуру при Київському
державному університеті ім. Шевченка. У цьому ж році очолив Маріупольський
гуманітарний коледж, відкритий в 1991 р.

% portrait black white
\ii{29_09_2019.stz.news.ua.mrpl_city.1.kostjantyn_balabanov_zhyttja_mudrist.pic.5}

Загалом 1992 рік став переломним в житті Костянтина Васильовича. Він отримує
пропозицію очолити гуманітарний коледж при Донецькому державному університеті у
Маріуполі. З такою ініціативою виступає ректор ДонДУ Володимир Павлович
Шевченко, який одним з перших розгледів у Костянтині Балабанові непереможний
запал лідера і великий творчий потенціал. Ідея створення навчального закладу,
який би сприяв відродженню як україн\hyp{}ської, так і грецької мови і культури в
регіоні з чималою діаспорою греків та зміцненню дружби українського і грецького
народів, виникає у Володимира Павловича ще під час тривалих душевних розмов у
гостинному домі родини Балабан. Не всім відомо, що Костянтин Васильович не
відразу прийняв пропозицію. І це не дивно, адже докторантура забирала багато
часу, тим більше життя в Донецьку, з яким пов'язані і теплі спогади, і близькі
люди, залишати не хотілося. Однак долею судилося саме йому взятися за таку
нелегку, але вкрай важливу справу як розбудову та розвиток єдиного в Маріуполі
гуманітарного університету. Цікаво, що саме університет подарував чоловіку
доленосну зустріч. Саме тут він познайомився зі своєю майбутньою дружиною
Євгенією.

Костянтин Балабанов стає першим доктором політичних наук на Донеччині. Його
наукові напрацювання базуються на особистому досвіді керівництва закладом вищої
освіти. Автор унікальної моделі міжнародної співпраці, реалізованої у межах
інституту, він доводить її ефективність щоденною невтомною та результативною
працею. Костянтин Васильович має талант здібного керівника і природний дар
згуртовувати людей навколо себе. Можливо, саме завдяки цим якостям йому вдалося
створити міцний і злагоджений колектив. Однак спочатку було неймовірно складно,
адже зарплати були зовсім низькі. Щоб утримати людей, ректору доводилося
просити брата, щоб той привозив картоплю, помідори, масло, сало, щоб якось
допомогти викладачам...

Попри низку проблем, у 1993 році мудрий керівник домігся того, щоб коледж був
перетворений в Маріупольський гуманітарний інститут Донецького державного
університету. За ці роки ЗВО став справжнім центром гуманітарної освіти,
розвитку української мови і духовності, відродження еллінізму на Південному
Сході України. Природжений лідер, блискучий організатор і справжній
перфекціоніст, в будь-якій найскладнішій ситуації він бере відповідальність
лише на себе. У найкоротші терміни, незважаючи на несприятливу економічну
ситуацію в країні, ректор ініціював і завершив будівництво нових навчальних
корпусів, що на ті часи було рівносильно подвигу!

Енергійна діяльність ректора МДУ надихала та водночас вражала не тільки
маріупольців. Зокрема, за оцінкою американського журналу \enquote{Time} (1997 р.)
Костянтина Балабанова визнано одним з 8 видатних греків зарубіжжя, які
заслужили світове визнання завдяки своїй діяльності в галузі освіти, культури
та економіки.

% greeks list
\ii{29_09_2019.stz.news.ua.mrpl_city.1.kostjantyn_balabanov_zhyttja_mudrist.pic.6}

Як закономірний результат злагодженої роботи колективу і послідовного курсу
ректора на поліпшення якості освітніх послуг 17 червня 2004 р. розпорядженням
Кабінету міністрів України Маріупольський гуманітарний інститут отримав
університетський статус. А рівно через 3 місяці К. В. Балабанов підписав у
Болоньї найважливіший для європейської вищої школи документ – Велику Хартію
університетів – і, увійшовши в десятку провідних вузів України, що удостоїлися
такої честі, здійснив справжній прорив в освітній простір Європи. Завдяки
організаторським здібностям Костянтина Васильовича, в 2006 році МДГУ став
першим вузом України, обраним до Ради директорів Європейської організації
публічного права, до якої входять 17 європейських країн і понад 70
університетів з усього світу.

\ii{29_09_2019.stz.news.ua.mrpl_city.1.kostjantyn_balabanov_zhyttja_mudrist.pic.7}

МДУ – єдиний в Європі ЗВО (крім університетів Греції та Кіпру), в якому
створено спеціалізований факультет грецької філології та перекладу, де понад
300 студентів вивчають як спеціальність мову, культуру та історію Греції.
Університет є базовим закладом вищої освіти на Сході України з вивчення
італійської мови.

Особливо ректор пишається колективом викладачів і студентів, які успішно
поєднують досвід старшого покоління з молодістю і креативністю молодшого.
Костянтин Васильович наголошує, що гарні, розумні, креативні студенти – це
обличчя університету, міста і країни. Водночас він підкреслює, що
\enquote{силу, мужність, волю і мудрість керівника дали йому люди, з якими він
працює}.

\ii{29_09_2019.stz.news.ua.mrpl_city.1.kostjantyn_balabanov_zhyttja_mudrist.pic.8}

Щорічно понад 100 студентів університету беруть участь в освітніх та культурних
програмах Греції, Кіпру, Польщі, Італії, США, Великобританії, Чехії, Франції,
Бельгії, Німеччини, Нідерландів.

МДУ активно співпрацює з Федерацією грецьких товариств України. За підтримки
федерації на базі університету систематично проводяться всеукраїнські олімпіади
школярів з новогрецької мови, історії, культури Греції та греків України,
семінари для вчителів та конкурси \enquote{Учитель року}.

\ii{29_09_2019.stz.news.ua.mrpl_city.1.kostjantyn_balabanov_zhyttja_mudrist.pic.9}

1 липня 2008 року Державна акредитаційна комісія України прийняла рішення
присвоїти Маріупольському державному університету четвертий рівень акредитації,
а за два роки він отримує статус класичного університету. Безперечні успіхи
вузу, його авторитет і міжнародне визнання безпосередньо пов'язані з
конструктивною роллю, високими діловими і моральними якостями Костянтина
Васильовича Балабанова.

\ii{29_09_2019.stz.news.ua.mrpl_city.1.kostjantyn_balabanov_zhyttja_mudrist.pic.10}

2010 рік запам'ятався значущою науковою подією – Костянтина Балабанова обирають
членом-кореспондентом Національної академії педагогічних наук України за
відділенням вищої школи. Вагомий науковий доробок, а також результати
керівницької та дипломатичної діяльності ректора МДУ високо оцінює вітчизняна
освітянська спільнота. Тепер він офіційно долучається до когорти людей, чиї
наукові розробки визначають майбутнє української вищої освіти.

% with zelenskii
\ii{29_09_2019.stz.news.ua.mrpl_city.1.kostjantyn_balabanov_zhyttja_mudrist.pic.11}

Якщо 28 років тому про гуманітарний коледж знали хіба що у Донецькому регіоні,
то зараз Маріупольський державний університет щороку відвідують десятки
делегацій. Вони приїжджають на запрошення та з власної ініціативи, адже
університет зарекомендував себе як партнер, співпраця з яким є плідною та
взаємовигідною. Тільки за останні 3 роки університет прийняв близько 100
іноземних делегацій найвищого рівня. Зокрема, приймали Єврокомісара Йоганнеса
Гана, міністрів закордонних справ Швейцарії, Греції, Литви, послів США, Японії
та більшості країн ЄС. \emph{\enquote{Нам є, що показати}}, – говорить ректор МДУ, і це
звучить гордо. Загалом історія Маріупольського державного університету –
приклад того, яких результатів можна досягти за дуже короткий період, якщо
докласти максимум таланту, мудрості і працьовитості.

\ii{29_09_2019.stz.news.ua.mrpl_city.1.kostjantyn_balabanov_zhyttja_mudrist.pic.12.wife_daughter}

Незважаючи на нестримний темп життя, постійну зайнятість, Костянтину
Васильовичу вдається всюди встигати, для кожного знайти добре слово і допомогти
корисною та цінною порадою. А найбільшою підтримкою і натхненням для нього є
найбільш рідні люди – кохана дружина Євгенія та донька Женя, які невимовно
пишаються досягненнями свого чоловіка і батька.

Вражає не тільки життєва мудрість нашого героя, глибокі знання, але й науковий
досвід, адже він – фаховий спеціаліст в області теорії і історії міжнародних
відносин, міждержавного співробітництва і проблем євроінтеграції – є автором
понад 300 наукових статей, книг і підручників. Всі нагороди і здобутки мудрого
керівника одразу і не порахуєш. Костянтин Балабанов – кавалер ордена України
\enquote{За заслуги} I, II і III ступенів, ордена Ярослава Мудрого V ступеня, ордена
Української православної церкви \enquote{Рівноапостольного князя Володимира} I і II
ступенів, ордена \enquote{Честь} (одна з найбільш високих державних нагород в Греції)
та орденів Італійської Республіки \enquote{Зірка Італійської Солідарності} і \enquote{Зірка
Італії}.

\ii{29_09_2019.stz.news.ua.mrpl_city.1.kostjantyn_balabanov_zhyttja_mudrist.pic.13}

Костянтин Васильович є Почесним громадянином м. Асклепіон (Греція), м.
Маріуполя, Нікольського району і с. Малоянісоль, має почесний міжнародний титул
\enquote{Посланник еллінізму}. З нагоди відзначення 10-річчя діяльності консульства на
чолі з Костянтином Балабановим його дипломатичний статус було підвищено до
Почесного Генерального консула Республіки Кіпр у Маріуполі. Це найвищий
дипломатичний ранг, який іноземна держава може присвоїти громадянину України.
На сьогодні в Україні більше 100 почесних консульств, і лише три з них,
включаючи дипломатичне представництво Кіпру в Маріуполі, мають статус Почесного
Генерального консульства.

\ii{29_09_2019.stz.news.ua.mrpl_city.1.kostjantyn_balabanov_zhyttja_mudrist.pic.14}

Завдяки зусиллям, стратегічному мисленню, далекоглядності й організаторським
здібностям ректора Маріупольський державний університет входить в число
провідних вузів країни з великим потенціалом, зі своїми науковими школами і
чудовими традиціями, добре організованим працьовитим колективом, який успішно
готує висококваліфіковані кадри. З упевненістю можна стверджувати, що без
Костянтина Васильовича Балабанова в Маріуполі не існувало б такого
прогресивного і потужного університету.

Найвищий професіоналізм, талант, безмежна відданість справі, працьовитість і
наполегливість у досягненні поставлених цілей дали змогу нашому герою досягти
значних успіхів у його діяльності, здобути шану та повагу серед колег та
студентства.

Впевнена, що досвід, знання і чудові людські якості Костянтина Васильовича
будуть і у подальшому спрямовані на розвиток Маріупольського державного
університету, Маріуполя та України загалом. Нехай і надалі у всіх починаннях
супроводжують розуміння і підтримка з боку колег та однодумців, а професійна
діяльність буде сповнена задоволення і творчими перемогами.

\ii{29_09_2019.stz.news.ua.mrpl_city.1.kostjantyn_balabanov_zhyttja_mudrist.pic.15}

\textbf{Улюблене місце в Маріуполі:} Приморський парк.

\textbf{Улюблена книга:} класична мемуарна література.

\textbf{Порада маріупольцям:} 

\begin{quote}
\em\enquote{Маріуполь – це місто працелюбних і талановитих людей. Бережіть і розвивайте його, любіть і пишайтеся своїм містом!}.
\end{quote}

\ii{insert.read_also.demidko.kiseljova_nina_mykolaivna}
