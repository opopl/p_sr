% vim: keymap=russian-jcukenwin
%%beginhead 
 
%%file 19_11_2021.tg.tkachev_jurij.1.revgidnosti
%%parent 19_11_2021
 
%%url https://t.me/dadzibao_talk/290635
 
%%author_id tkachev_jurij
%%date 
 
%%tags 
%%title 
 
%%endhead 
%Общество принуждения, несвободы, страха, лицемерия и лжи

\subsection{О \enquote{Дне достоинства и свободы}}
\label{sec:19_11_2021.tg.tkachev_jurij.1.revgidnosti}

\Purl{https://t.me/dadzibao_talk/290635}

\ifcmt
 author_begin
   author_id tkachev_jurij
 author_end
\fi

О "Дне достоинства и свободы". 

Можно много чего писать про последствия победы Евромайдана, но одна вещь
произошла очень чётко: свобода слова и воззрений в Украине резко закончилась. 

До Майдана на самом деле Украина в этом плане была достаточно свободной
страной: люди почти любых взглядов с точки зрения права их высказывать были
почти равны. Коммунисты, анархисты, националисты и "социал-националисты", ЛГБТ
и анти-ЛГБТ, бандеровцы и сталинисты, пророссийские и русофобы были вынуждены
так или иначе сосуществовать в информационной среде на равных правах. Сколько
было эфиров, разных "дискуссионных клубов"! Вспоминать словесные баталии в них
сегодня даже немножко смешно. Даже на небезызвестном телеканале АТВ в студии
часто появлялись ярые политические украинцы, а автору этих строк неоднократно
приходилось сталкиваться с оппонентами в эфирах других каналов... 

После Майдана общественно-политические взгляды быстро поделили на дозволенные и
недозволенные, причём публичное высказывание недозволенных взглядов моментально
влекло последствия. В некоторых случаях это могли быть уголовные дела
(сепаратизм, госизмена, статьи в рамках закона о декоммунизации), но чаще
использовались другие методы. 

Скажем, госслужащие и работники бюджетных учреждений (врачи, учителя) за
"неправильные" взгляды моментально лишались работы. Даже те, кто не находились
в прямой зависимости от государства, могли столкнуться с проблемами: мы помним
скандалы с музыкантами при отборе на Евровидение, со спортсменами после
Олимпиады...

Даже те, кто не зависели от государства вообще никак, тоже сталкивались с этим:
есть масса примеров травли частных предприятий, сотрудники которых допускали те
или иные "неправильные" высказывания. Зачастую это также заканчивалось
увольнением.

Появился "Миротворец". Это сейчас к нему принято относиться скорее
юмористически, а в 2014-2015 году попадание туда реально грозило
неприятностями: многие люди лишились, к примеру, банковских счетов вместе с
лежащими на них деньгами. Да и сегодня присутствие на Миротворце может помешать
вам получить кредит или стать основанием для отказа в устройстве на работу в
структуры вроде тех же банков или, скажем, "Укрзализныци". 

Определённые вещи стали табуированными в большинстве СМИ. Например, Майдан
запрещено называть госпереворотом. Почему? Потому. Потому что свобода. И
достоинство. СМИ, которые допускают подобные высказывания, сами становятся
табуированными: от санкций и исков о лишении лицензий до просто мелкого
геморроя на местах типа "ой, нам запрещено давать вам комментарии, извините". 

Доходило до смешного. Сотрудника Госаудитслужбы вызвали на ковёр за то, что он
лайкал (!) мои посты в Фейсбуке: "Ну вы же понимаете, какое сейчас время?".
Свобода, ага.

А потом кстати меня из Фейсбука вообще удалили без объяснения причин. 

Ну и конечно же, Украина не была бы Украиной, если бы "новую реальность" не
использовали в мелких кулуарных разборках с исключительно меркантильными
целями. Один чиновник записывает своего начальника поющим "Как упоительны в
России вечера" и пишет донос вышестоящему начальству. Ну, то есть в известных
кругах стучали всегда. Но теперь стук приобрёл некий совершенно неповторимый,
непередаваемый колорит и аромат!

И ведь не то чтобы это было прямо ужасно, непереносимо и опасно. В реальности
некие реальные угрозы существуют лишь для публичных людей, публично же
придерживающихся неправильных взглядов. Да и с ними, в конце концов, можно
сосуществовать, по своему примеру говорю. Есть в этом даже некий извращённый
кайф, эдакий чисто мальчишеский «вызов системе». 

Но... противно. Миллионы людей привыкли думать над каждым словом, привыкли
взвешивать, что, как и в чьём обществе можно сказать, а чего говорить не нужно
или нельзя ни в коем случае. 

Прямым следствием победы "Революции достоинства", а точнее, государственного
переворота, уничтожившего Украину в том виде, в котором она была создана в 1991
году, стало становление вот этого общества принуждения, несвободы, страха,
лицемерия и лжи. 

Добились. Здобулы. Наскакали. 

Кушайте теперь, не обляпайтесь.
