% vim: keymap=russian-jcukenwin
%%beginhead 
 
%%file 18_02_2022.fb.fb_group.story_kiev_ua.1.gorod_dozhd
%%parent 18_02_2022
 
%%url https://www.facebook.com/groups/story.kiev.ua/posts/1864059407124185
 
%%author_id fb_group.story_kiev_ua,sirota_tatjana.kiev
%%date 
 
%%tags gorod,kiev,pogoda
%%title В любимом Городе идёт дождь...
 
%%endhead 
 
\subsection{В любимом Городе идёт дождь...}
\label{sec:18_02_2022.fb.fb_group.story_kiev_ua.1.gorod_dozhd}
 
\Purl{https://www.facebook.com/groups/story.kiev.ua/posts/1864059407124185}
\ifcmt
 author_begin
   author_id fb_group.story_kiev_ua,sirota_tatjana.kiev
 author_end
\fi

В любимом Городе идёт дождь...

А мы, от Бессарабки по Крещатику и Богдана Хмельницкого, идём \enquote{на
дело}...))) А именно, в Природоведческий музей.

\ii{18_02_2022.fb.fb_group.story_kiev_ua.1.gorod_dozhd.pic.1}

Малый давно туда просился. Уж очень ему хотелось посмотреть на скелет мамонта.

Этот музей — один из крупнейших научно-природоведческих музеев мира.

\raggedcolumns
\begin{multicols}{3} % {
\setlength{\parindent}{0pt}

% pic 4
\ii{18_02_2022.fb.fb_group.story_kiev_ua.1.gorod_dozhd.pic.2}
\ii{18_02_2022.fb.fb_group.story_kiev_ua.1.gorod_dozhd.pic.2.cmt}

\ii{18_02_2022.fb.fb_group.story_kiev_ua.1.gorod_dozhd.pic.5}

\ii{18_02_2022.fb.fb_group.story_kiev_ua.1.gorod_dozhd.pic.6}
\ii{18_02_2022.fb.fb_group.story_kiev_ua.1.gorod_dozhd.pic.6.cmt}

\ii{18_02_2022.fb.fb_group.story_kiev_ua.1.gorod_dozhd.pic.7}
\ii{18_02_2022.fb.fb_group.story_kiev_ua.1.gorod_dozhd.pic.7.cmt}

\ii{18_02_2022.fb.fb_group.story_kiev_ua.1.gorod_dozhd.pic.8}
\ii{18_02_2022.fb.fb_group.story_kiev_ua.1.gorod_dozhd.pic.9}
\ii{18_02_2022.fb.fb_group.story_kiev_ua.1.gorod_dozhd.pic.10}

\end{multicols} % }

Создан в 1966 году как единый экспозиционно-территориальный комплекс в составе
Геологического, Палеонтологического, Зоологического, Ботанического и
Археологического музеев.

10 декабря 1996 года указом Президента Украины ему  был предоставлен статус
Национального.

Находится он в здании, которое строилось для Ольгинской гимназии и в котором
разместится  ей так и не пришлось.   

Проведя в музейных залах часа три (а может и больше) и получив огромное
удовольствие от увиденного, мы в прекрасном настроении вернулись домой.

17 февраля 2022 года
