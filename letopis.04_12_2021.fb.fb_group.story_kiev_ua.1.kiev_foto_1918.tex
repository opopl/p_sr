% vim: keymap=russian-jcukenwin
%%beginhead 
 
%%file 04_12_2021.fb.fb_group.story_kiev_ua.1.kiev_foto_1918
%%parent 04_12_2021
 
%%url https://www.facebook.com/groups/story.kiev.ua/posts/1811758855687574
 
%%author_id fb_group.story_kiev_ua,ljashenko_vadim
%%date 
 
%%tags 1918,foto,gorod,kiev
%%title Киев - фотографии 1918 года
 
%%endhead 
 
\subsection{Киев - фотографии 1918 года}
\label{sec:04_12_2021.fb.fb_group.story_kiev_ua.1.kiev_foto_1918}
 
\Purl{https://www.facebook.com/groups/story.kiev.ua/posts/1811758855687574}
\ifcmt
 author_begin
   author_id fb_group.story_kiev_ua,ljashenko_vadim
 author_end
\fi

В 2014 году, я системно собирая материалы о Киеве, обнаружил фотографии 1918
года. Их снимали кайзеровские фотографы, когда в городе наступила весна и
солнце растопило лед и снег.

\ii{04_12_2021.fb.fb_group.story_kiev_ua.1.kiev_foto_1918.pic.1}

\begin{itemize}
  \item 01. Днепр 29 марта 1918 года.
  \item 02. Пароход и церкви.
  \item 03. Прибытие парохода.
  \item 04. Софиевский собор, 1 апреля 1918 года.
  \item 05. Военный Николаевский Собор.
  \item 06. Михайловский Златоверхий собор.
  \item 07. Софийский собор.
  \item 08. Киево-Братский Богоявленский монастырь.
  \item 09. Рынок у Киево-Братского Богоявленского монастыря.
  \item 10. Мариинский дворец.
  \item 11. Трапезная церковь.
  \item 12. Памятник Искре и Кочубею (на Арсенальной).
  \item 13. Свято-Троицкая надвратная церковь.
  \item 14. Киевская гавань.
\end{itemize}

\ii{04_12_2021.fb.fb_group.story_kiev_ua.1.kiev_foto_1918.pic.2}
\ii{04_12_2021.fb.fb_group.story_kiev_ua.1.kiev_foto_1918.pic.3}
\ii{04_12_2021.fb.fb_group.story_kiev_ua.1.kiev_foto_1918.pic.4}

