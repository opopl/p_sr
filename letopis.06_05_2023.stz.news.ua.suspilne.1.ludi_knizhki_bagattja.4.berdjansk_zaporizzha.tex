% vim: keymap=russian-jcukenwin
%%beginhead 
 
%%file 06_05_2023.stz.news.ua.suspilne.1.ludi_knizhki_bagattja.4.berdjansk_zaporizzha
%%parent 06_05_2023.stz.news.ua.suspilne.1.ludi_knizhki_bagattja
 
%%url 
 
%%author_id 
%%date 
 
%%tags 
%%title 
 
%%endhead 

\subsubsection{Бердянськ-Запоріжжя. Дві тисячі за таксі та перевірки кадирівців}

У Бердянську нас поселили у квартиру. Там ми просиділи шість діб. Кожного ранку
ходили в спорткомплекс записувались на автобуси. Щоранку о 6-й ранку треба було
приходити, щоб попасти та виїхати до Запоріжжя. Наша черга не підходила, тоді
зовсім не пускали автобуси у Бердянськ. І був вже відчай, була істерика.
Істерія була страшна. Чоловіки кажуть: ми не знаємо, що робити, бо ніхто ще не
знає цю дорогу, ніхто не хоче на машинах їхати, бояться. Перевізники ще не
їздили туди. На своїх машинах виїжджали люди, а нам треба було чекати тільки
автобуси, більш ніяк.

\ii{06_05_2023.stz.news.ua.suspilne.1.ludi_knizhki_bagattja.pic.8}

В якийсь ранок нам пощастило. Мені зателефонувала моя співробітниця й каже:
бігом вибігайте аж на кільце, туди за Бердянськ, бо в місто вони не заїдуть і,
може, це навіть останній раз, коли вони приїхали. Ми викликали таксі, ціни були
баснословні, їхати 7-10 хвилин, а просили 2 тисячі гривень. Ми доїхали до
кільця, пройшли блокпост, там стояли буряти чи ямали, точно не знаю, заскочили
у ці автобуси. А вони вже були набиті, бо не тільки ми, маріупольці, виїжджали,
виїжджало багато бердянців. Там хоч і було все ціле, але місто було окуповане,
і вони прекрасно розуміли, що буде.

Так в тісноті да не в обіді доїхали ми до Василівки. Повзли довго, цілий день
ми були в дорозі, як стемніло, як наступила комендантська година у Василівці,
сказали: ви нікуди далі не їдете, залишаєтесь тут.

\begin{qqquote}
Ночували у полі, там вже були кадирівці, виходили, тикали автоматами,
перевіряли всіх чоловіків, дивились їхні татуювання, документи. Дівчат більш
менш не чіпали, а чоловіків роздягали через кожні п'ять хвилин. Кудись їх
забирали на допити, потім повертались вони, то побиті, то ще якісь.	
\end{qqquote}

Рано в ранці, ми думаємо, має скінчиться комендантська година і поїдемо. Шоста
година ранку, дев’ята, десята, дванадцята — а ми все стоїмо та стоїмо. Як нам
сказали, що чи викуп якийсь затребували, чи якісь продукти. Це вирішувалось на
якомусь високому рівні. Й десь в четвертій вечора нас, всю колону, випустили.

Десь о шостій вечора, це було, здається, коли ми перетнули оцю Василівку й
перейшли в це село, яке вже було під контролем України, там нас вже
супроводжувала машина, ми вже не зупинялись, летіли вже як навіжені до того
"Епіцентру" до Запоріжжя. Дуже багато було поранених, їх одразу ж пересадили в
швидкі допомоги. Діти кричали. Ситуація була дуже жахлива.

Ось так 16 березня я вибралась з Маріуполя і 26 березня лише я потрапила до
Запоріжжя. Там мене чекав чоловік біля \enquote{Епіцентру}, я зареєструвалась, він
посадив нас у машину й відвіз до родичів в Полтавську область. Там ми більш
менш приходили до себе, до тями.
