% vim: keymap=russian-jcukenwin
%%beginhead 
 
%%file slova.sevastopol
%%parent slova
 
%%url 
 
%%author 
%%author_id 
%%author_url 
 
%%tags 
%%title 
 
%%endhead 
\chapter{Севастополь}
\label{sec:slova.sevastopol}

%%%cit
%%%cit_pic
%%%cit_text
Республика (государство) Крым и город федерального значения \emph{Севастополь},
народы Крыма 7 лет назад сделали свой выбор в пользу России. Тем более Путин
сказал, что вопрос принадлежности Крыма закрыт окончательно. Если же сборная
Украины пройдет в 1/4 финала, который будет в Санкт-Петербурге, тогда
действительно уже нужно будет принимать какие то решения. А форма сборной
Украины по версии зрителей британского телеканала Sky Sports признана худшей из
всех форм 24 сборных, которые участвуют в чемпионате. А Вы как считаете стоит
реагировать России?
%%%cit_comment
%%%cit_title
\citTitle{Какая разница на каких футболках Крым, главное он в составе России}, 
Добрый Человек, zen.yandex.ru, 11.06.2021
%%%endcit

%%%cit
%%%cit_pic
\ifcmt
  pic https://avatars.mds.yandex.net/get-zen_doc/751940/pub_60ccb9cfb4fa377376c682b3_60cd1dc8aa9d6d2116e1efe3/scale_1200
\fi
%%%cit_text
Довелось мне недавно осуществить свою давнишнюю мечту и побывать в Городе-герое
\emph{Севастополь}, о котором я много слышал, но никак не мог попасть туда. На самом
деле, в городе есть на что посмотреть, а особенно меня впечатлил морской порт и
мощные военные суда. Приятно смотреть как город утопает в зелени, а большое
количество скверов и парков позволяет наслаждаться свежим воздухом и испытать
умиротворение.  В одном из таких парков я и познакомился с коренным жителем
\emph{Севастополя} в 5 поколении, как он упомянул позже в нашем диалоге. Не
буду называть его имя, потому что это не суть важно. Так вот, рассказ этого
дедули растянулся на целых 2 часа, потому что наверное своим вопросом о том,
как изменился \emph{Севастополь} за последние 7 лет, когда стал частью России, я задел
то, о чем он много и часто думал. Что же рассказал пожилой \emph{севастополец} и что
именно изменилось с тех пор, как город стал российским? А также, почему
украинцы впадают в дикость, когда слышат про изменения с его слов? Давайте по
порядку...
%%%cit_comment
%%%cit_title
\citTitle{«Я севастополец в 5 поколении»: изменения «Города-героя» при России за 7 лет приводящие в дикость украинцев}, 
За Пределами Онлайна, zen.yandex.ru, 19.06.2021
%%%endcit

