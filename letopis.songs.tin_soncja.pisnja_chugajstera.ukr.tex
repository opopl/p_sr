% vim: keymap=russian-jcukenwin
%%beginhead 
 
%%file songs.tin_soncja.pisnja_chugajstera.ukr
%%parent songs.tin_soncja.pisnja_chugajstera
 
%%endhead 

\subsubsection{Українська}
\label{sec:songs.tin_soncja.pisnja_chugajstera.ukr}

У предвічних хащах, де не тліє цвіт болота,
Де лягла Батия Чорно-Золота кіннота,
Часом в білій свиті зачарований блукає
Лісовик-Чугайстер і стиха промовляє:

Якщо можеш, завітай до мене ти,
Якщо хочеш, спий моєї води,
Якщо Віриш, розкажи мені,
Про що ти мрієш, коли живеш на самоті...

Брами павутиння від сосни і до берези –
Вартові чекання --- Долі зламані терези.
Полум’яна рута життєдайно розквітає,
Та ніхто не бачить! Ніхто того не знає!

Стогне хижим плачем трясовина на Поліссі,
Попелом укрите листя у древлянськім лісі.
В безнадії квола похилилася тополя,
Дме тривожний вітер з отруєного поля.

Все, що наснилось більше не вернеться –
Казкові мрії в небутті.
Занадто рано зупинилось серце:
Ми вмерли надто молоді.

Мій голос з неба чорний крук принесе,
Спливе по Прип’яті вінок.
Чекайте вістки під Похмурим Сонцем,
У сяйві жалібних зірок.
