%%beginhead 
 
%%file 17_10_2021.fb.fb_group.mariupol.biblioteka.korolenka.1.shchodnya_mi_robimo_
%%parent 17_10_2021
 
%%url https://www.facebook.com/groups/1476321979131170/posts/4390677287695610
 
%%author_id fb_group.mariupol.biblioteka.korolenka,kibkalo_natalia.mariupol.biblioteka.korolenko
%%date 17_10_2021
 
%%tags mariupol,mariupol.pre_war,kniga
%%title Щодня ми робимо маленькі чи великі вчинки, щодня прагнемо стати в чомусь ще кращими…
 
%%endhead 

\subsection{Щодня ми робимо маленькі чи великі вчинки, щодня прагнемо стати в чомусь ще кращими…}
\label{sec:17_10_2021.fb.fb_group.mariupol.biblioteka.korolenka.1.shchodnya_mi_robimo_}
 
\Purl{https://www.facebook.com/groups/1476321979131170/posts/4390677287695610}
\ifcmt
 author_begin
   author_id fb_group.mariupol.biblioteka.korolenka,kibkalo_natalia.mariupol.biblioteka.korolenko
 author_end
\fi

Щодня ми робимо маленькі чи великі вчинки, щодня прагнемо стати в чомусь ще
кращими... В фондах Центральна міська публічна бібліотека ім. В.Г. Короленка м.
Маріуполь є чудове видання - Уляна Скицька «\#НАШІ на карті світу. Історії про
людей, якими захоплюється світ» (Львів, 2019). Серед героїв книжки є Богдан
Гаврилишин...

Богдан Дмитрович Гаврилишин – видатний український, канадський, швейцарський
економіст, громадський діяч, меценат, іноземний член НАН України. Працюючи на
світовому рівні як член Наглядової ради Міжнародної академії мистецтва та науки
Богдан Гаврилишин ще на початку 2010-х дійшов висновку: дуже важливим є
усвідомлення та прийняття кожним членом суспільства індивідуальної
відповідальності на ВСІХ рівнях. У 2014 р. професор створив змістовний та
лаконічний документ – «Декларацію відповідальності людини», що може стати
дороговказом до більш демократичного, інклюзивного та ефективного суспільства. 

19 жовтня, у день народження великого українця, відзначається Всеукраїнський
День відповідальності людини – нове, але важливе свято для України. Стало
традицією вшановувати пам'ять Богдана Гаврилишина Хризантемовою акцією. В
Маріуполі теж долучилися до цієї події завдяки Ользі Близнюк – амбасадору Фонду
Родини Богдана Гаврилишина в Донецькій області та користувачу ЦМПБ ім. В.Г.
Короленка. Біля будівлі бібліотеки Ольга посадила чарівні хризантеми. 

\#ДеньВідповідальності \#дію\_відповідально
