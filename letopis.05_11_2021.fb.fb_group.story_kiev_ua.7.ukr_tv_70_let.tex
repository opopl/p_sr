% vim: keymap=russian-jcukenwin
%%beginhead 
 
%%file 05_11_2021.fb.fb_group.story_kiev_ua.7.ukr_tv_70_let
%%parent 05_11_2021
 
%%url https://www.facebook.com/groups/story.kiev.ua/posts/1790929294437197/
 
%%author_id fb_group.story_kiev_ua,pestun_maksim.kiev.ukraina.pisatel
%%date 
 
%%tags 70_let,istoria,sssr,televizor,ukraina,ussr
%%title Сегодня исполнилось 70 лет Украинскому телевидению
 
%%endhead 
 
\subsection{Сегодня исполнилось 70 лет Украинскому телевидению}
\label{sec:05_11_2021.fb.fb_group.story_kiev_ua.7.ukr_tv_70_let}
 
\Purl{https://www.facebook.com/groups/story.kiev.ua/posts/1790929294437197/}
\ifcmt
 author_begin
   author_id fb_group.story_kiev_ua,pestun_maksim.kiev.ukraina.pisatel
 author_end
\fi

Сегодня исполнилось 70 лет Украинскому телевидению. Для всех жителей УССР это
явилось весьма знаменательным событием. Чем еще было скрасить досуг советскому
работнику, особенно вдали от центральных городов, где были театры и другие
очаги культуры?  Молодежь бегала на танцы по субботам, а в будние дни...

Люди старшего поколения познавали мир через печатное слово. Газеты "Правда",
"Известия", "Труд"...

\ifcmt
  ig https://scontent-frt3-1.xx.fbcdn.net/v/t39.30808-6/253777490_3049294675340693_4123493451255021351_n.jpg?_nc_cat=102&ccb=1-5&_nc_sid=825194&_nc_ohc=nQnMnsiI-hIAX8j2C85&_nc_oc=AQmdyCvgGsINMj3bluktd4Qg-w6dK8X0uvZgIsMWrokmR3gfNvSUWyrx_3dmeCvnQS0&_nc_ht=scontent-frt3-1.xx&oh=de2f0fc14fedb7e48bcb9099957e065b&oe=618D63EF
  @width 0.4
  %@wrap \parpic[r]
  @wrap \InsertBoxR{0}
\fi

Нужно отметить, что все они имели двойное назначение, в основном сугубо
утилитарное. Говорили, что в "Правде" нет известий, а в "Известиях" правды...

И вот постепенно, почти в каждом доме, а со временем и в хатах зажглись
"голубые огоньки". Сначала их трудно было назвать "голубыми".  Через мутноватую
жидкость линз с трудом можно было разобрать происходящее на экране, но со
временем диагональ экрана постоянно росла. Примерно, как сейчас происходит  со
смартфонами. 

Можно сказать, что появление телевидения совершило коренной переворот в досуге
граждан.

Через экран они познавали мир, следили за "загнивающим Западом" в программах
Зорина и других политических обозревателей. Болели за Анжелу Девис и Нельсона
Манделлу. Сочувствовали рабочим, угнетаемым мировым капитализмом. Гордились
"нашей" помощью народам Африки. Конечно, основное время на экране показывали
передовиков производства и достижения советского народа на пути к коммунизму.
Этими достижениями, хоть и гордились, но практически не смотрели...

Зато на праздники все семьи собирались у экрана, чтобы посмотреть и послушать
праздничный концерт. А на Новый Год концерт был особенный - "Голубой огонек."
Это была самая долгожданная  передача в году. Перед этим все собравшиеся у
экрана, с наполненными бокалами ждали поздравление секретаря ЦК КПСС. Думаю,
мало найдется тех, кто не чокался с ним... 

Конечно, был еще "Кабачок 13 стульев" с несравненной пани Моникой, паном
Гималайским, паном Директором, пани Каролиной и другими.  Это  был глоток
другой, западной жизни, которую большинство телезрителей представляло себе
только из передач. 

В первые годы после появления телевизор был такой редкостью, что обычно его
смотрели всем двором, вынося на улицу или балкон одной из квартир.

Имеющие счастье иметь такое счастье, не знали покоя ни днем ни вечером.
Особенно, когда транслировали парад или футбольный матч...

Сейчас трудно представить, что для переключения программ, когда их появилось
несколько. Я помню три,  две Московские и одна Киевская, нужно было вставать и
идти переключать их вручную. Но главная фишка была в том, что эти переключатели
(программы шли не  подряд, а через пол дюжины щелчков) быстро ломались у всех,
и нужно было использовать плоскогубцы. Думаю, все читатели старшего поколения
помнят, что в детстве "пультами дистанционного управления" служили именно они.

Сегодня с приходом эры интернета роль телевидения отошла на второй план. А
тогда трудно было представить себе какой-либо дом без телевизора. Телевизор
служил объединяюще-рaзъединяющим началом для всей семьи.

Основные проблемы возникали, когда показывали интересную программу, типа "в
мире животных" , а по другой программе шел футбол. Или мультики. Если главе
семейства удавалось отбиться от "животных" в пользу футбола, то от "мультиков
для ребенка" далеко не всегда...

Отдельно нужно сказать о телесериалах. Я помню несколько знаковых. Это, любимый
в детстве,"Четыре танкиста и собака" и позже всенародный "Семнадцать мгновений
весны". Когда по телевизору шел премьерный показ очередной серии со Штирлицем,
улицы пустели. Ни в какое сравнение это ни шло с перестроечной "Санта
Барбарой". Ее смотрели люди старшего поколения, которые не сразу нашли себя в
новых реалиях. Наиболее активная часть населения "делала бабки", как говорил
тогда будущий президент "самой миролюбивой страны..."

Про вечную "Иронию судьбы" я промолчу. Это такой же символ Нового года, как
салат оливье и шампанское. 

Многие в то время начали создавать частные телестудии.

Одни из первых студий вне системы Гостелерадио, построила моя компания
"Джерело". Мы брали списанные ПТС (передвижные телевизионные станции) с 4
студийными камерами Томпсон, восстанавливали их, и на этой базе создавали
маленькие студийные телецентры. Один из первых находился на улице Ленина в
институте повышения квалификации работников телевидения, где я был главным
инженером, еще один в городе Славутич, который развивался быстрыми темпами
после Чернобыльской катастрофы. Несколько подобных телецентров моя компания
построила в Прибалтике.

Много воспоминаний накопилось у меня за годы работы на телевидении. От самых
первых дней после института до назначения коммерческим директором НТКУ.

Вот одно из них:

 После окончания Электроакустического факультета КПИ, бывшего Киноинженерного,
 единственного сохранившегося обломка Киноинженерного института на Бессарабке,
 я пошел работать на студию «Укртелефильм».

Сначала меня, как молодого специалиста по звуку, направили работать «по
специальности» в отдел теплоснабжения...

До сих пор у меня сохранились самые теплые воспоминания о моем первом месте
работы. Маленький одноэтажный, больше похожий на дачный, домик на отшибе утопал
в зелени студийного сада.

Рядом был Русановский канал, вдоль которого было так здорово гулять почти в
полном одиночестве. А весной в ухоженном саду "Укртелефильма" цвели яблони.

С благодарностью вспоминаю своего первого начальника Тисьменецкого. Он не
устраивал молодому работнику дедовщину и не ограничивал в приходе и уходе. Все
равно, я ровным счетом ничего не понимал в теплотехнике и никто не сомневался,
что я там временно.

Самым замечательным было то, что я тогда жил практически на той же улице, где
находилась студия. На улице Флоренции. Так ее назвали в честь города-
побратима. 

	Однажды я наблюдал там совершенно сюрреалистическую картину.

Как-то днем я шел домой и увидел на абсолютно пустой улице, расставленные вдоль
канала несколько лотков с дефицитными продуктами, за которыми стояли аккуратно
одетые в свежую форму девушки с кружевными чепчиками на голове. И ни одного
покупателя.

Я почувствовал себя мальчиком, попавшим в волшебный город, который создал
старик Хоттабыч.

- Это все реально можно купить? - спросил я недоверчиво у симпатичной
продавщицы.

 Да - сказала она.

- И икру и кофе и….?

А как это?... Моему удивлению не было предела.

- Делегация из Флоренции на один день посетила Киев – прервала она мой вопрос.

Надо заметить, что делегацию так и не привезли, они «застряли» на других
мероприятиях и про улицу-побратим не вспомнили…

А еще на Флоренции в «Совминовских домах» жили сотрудники консульства США. Так
я впервые увидел не на картинке огромные американские автомобили.

Но вернемся к телевидению.

Меня вскоре перевели на Крещатик 26 в отдел «Кинохроники», который снимал на
кинопленку все, что происходило в стране.

Там я стал звукооператором, а потом, когда перестроечный «ветер перемен» достиг
Киева, меня, единственного на то время с высшим образованием в Звукоцехе,
выбрали «Старшим звукооператором». В мои обязанности входило, на общественных
началах, распределять сотрудников на съемки. С теплом вспоминаю своего первого
наставника Аркадия Левитина, который работал на телевидении  с первых лет его
становления и других сотрудников. Многих из них уже нет в живых.

Вообще, работа в "КиноХронике" была очень интересной. Сегодня ты снимал
коровники, утопая по колено в грязи в далеком селе, куда еле добирался
студийный Рафик, а завтра мог работать на зеркально начищенном паркете в
посольстве или Верховном Совете.

А еще с киногруппой я объездил всю страну. Не только каждый маленький городок в
Украине, но и все столицы республик СССР.

На этих съемках я познакомился со многими замечательными писателями,
сценаристами, поэтами, актерами, художниками, режиссерами и просто выдающимися
людьми того времени.

Такими как Дмитрий Лихачев, Лев Кулиджанов, Сергей Параджанов, Олег Янковский,
Расул Гамзатов, Виталий Коротич, Людмила Гурченко, Сергей Иванов, ставший
впоследствии моим другом,  и многими-многими другими.

Как «молодому, подающему надежды» руководителю, партийное руководство студии
предложило мне вступить в ряды членов партии. Без этого, говорили они,
дальнейший рост невозможен. Это был, как бы, ритуал посвящения. Ну, думаю,
вступил в одно, вступлю и в другое...

Собрали комсомольское собрание и молодежь отдела решила не выдвигать меня
кандидатом в славные ряды КПСС.

Мотивировали они это дословно так: «как же можно его в партию?.. если у него и
так уже машина лучше, чем у нашего министра!»

Надо сказать, что я не расстроился, а вскоре, буквально через год, многие
сотрудники побежали сдавать и публично рвать на камеру свои красные книжечки,
"но это уже совсем другая история"...

\ii{05_11_2021.fb.fb_group.story_kiev_ua.7.ukr_tv_70_let.cmt}
