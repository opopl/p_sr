% vim: keymap=russian-jcukenwin
%%beginhead 
 
%%file slova.druzhba
%%parent slova
 
%%url 
 
%%author 
%%author_id 
%%author_url 
 
%%tags 
%%title 
 
%%endhead 
\chapter{Дружба}
\label{sec:slova.druzhba}

%%%cit
%%%cit_head
%%%cit_pic
%%%cit_text
У проросійському болоті велика заруба. Чубляться Портнов/Лукаш і Шарій. Ще
вчора вони були найліпшими \emph{друзями}, а сьогодні вже один одного тваринками
називають. Що ж сталося? Там всього намішано. І президентські амбіції
Портнова, і \enquote{туалетна історія} Шарія, і бажання їхніх учорашних спільних \emph{друзів}
вислужитись перед Зеленським, десь і Медведчук є на радарах
%%%cit_comment
%%%cit_title
\citTitle{Шарія кинули всі. Втече до Москви?}, Олексій Братущак, blogs.pravda.com.ua, 20.06.2021
%%%endcit

%%%cit
%%%cit_head
%%%cit_pic
\ifcmt
  tab_begin cols=2
		width 0.4
     pic https://strana.ua/img/forall/u/0/36/8870e29b30de20a65403b30c30b4b37560dc7956d45f7795881761.jpg
		width 0.46

     pic https://strana.ua/img/forall/u/0/36/2021-07-01_09h35_48.png
  tab_end
\fi
%%%cit_text
Подробности нападения. 29 июня на матч Украина-Швеция, который завершился
выходом украинской сборной в четвертьфинал Евро-2020, пришли не только
болельщики команд-соперниц, но и российский фанат Денис Смоленский вместе со
своими \emph{друзьями} - украинцем и поляком. Трое \emph{товарищей} выбрали
яркие образы, чтобы показать приверженность к разным странам, но в то же время
единство.  Россиянин выбрал одежду с российской символикой, которая отличала
его на \enquote{сине-желтом} секторе - он был в красной футболке сборной
России, с флагом РФ и в шапке-ушанке.  И, как видно по снимкам, в начале матча
украинские болельщики спокойно реагировали на болельщика, укутанного в триколор
%%%cit_comment
%%%cit_title
\citTitle{На Евро 2020 избили российского фаната Дениса Смоленского, что известно об инциденте}, 
Анна Копытько, strana.ua, 01.07.2021
%%%endcit

%%%cit
%%%cit_head
%%%cit_pic
\ifcmt
  pic https://strana.ua/img/forall/u/0/36/2021-07-03_14h11_09.png
	width 0.4
\fi
%%%cit_text
Главное, чтобы хорошо играли. Первые, кого мы встретили были две
\emph{подружки}, которые говорят на разных языках. Одна - на русском, другая -
на украинском. Они ответили, что для них вообще не принципиально, на каком
языке будут говорить на пресс-конференции украинские футболисты. Главное -
чтобы было понятно.  \enquote{Все равно, кому как удобно. Кто-то говорит на
украинском, кто-то - на русском.  Может, кто-то не знает украинский... А
вообще, вроде бы классно играли. Я видела в Инстаграме все истории были только
о футболе}, - получили мы первый ответ
%%%cit_comment
%%%cit_title
\citTitle{Что говорят украинцы о пресс-конференциях футболистов на русском языке. Опрос Страны}, 
Антонина Белоглазова, strana.ua, 03.07.2021
%%%endcit

%%%cit
%%%cit_head
%%%cit_pic
\ifcmt
  pic https://strana.ua/img/forall/u/0/36/2021-07-03_14h15_31.png
	width 0.4
\fi
%%%cit_text
Двое \emph{товарищей} также были солидарны с девушками и предложили в первую очередь
разобраться с теми, кто критикует украинских футболистов за язык.  При этом
ребята в повседневной жизни общаются на украинском языке.  \enquote{Как им удобно,
пусть так и говорят. В принципе много людей на русском языке говорят}, -
ответили нам. И добавили, что критику за язык в основном считают хайпом
%%%cit_comment
%%%cit_title
\citTitle{Что говорят украинцы о пресс-конференциях футболистов на русском языке. Опрос Страны}, 
Антонина Белоглазова, strana.ua, 03.07.2021
%%%endcit

