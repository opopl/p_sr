% vim: keymap=russian-jcukenwin
%%beginhead 
 
%%file 14_09_2021.fb.nikonov_sergej.1.latinica_danilov
%%parent 14_09_2021
 
%%url https://www.facebook.com/alexelsevier/posts/1581332005545442
 
%%author_id nikonov_sergej
%%date 
 
%%tags chelovek,danilov_aleksei,kirillica,kultura,latinica,rnbo,ukraina
%%title О переходе на латиницу
 
%%endhead 
 
\section{О переходе на латиницу}
\label{sec:14_09_2021.fb.nikonov_sergej.1.latinica_danilov}
 
\Purl{https://www.facebook.com/alexelsevier/posts/1581332005545442}
\ifcmt
 author_begin
   author_id nikonov_sergej
 author_end
\fi

О переходе на латиницу. Пока это только заявление, которое многие воспринимают
как плод размышлений Данилова. Но это только кажется его тупость. В случае
реализации это будет сознательным действием против собственной культуры. Суть
проста. Национальная письменность - основа национального мировосприятия. Ибо
мировосприятие выражается словами в речи, на бумаге или на клавиатуре. Общее
знание речи дает доступ ко всем национальным текстам, а значит и смыслам. А
значит и к убеждениям, и к действиям. Латиница не чужда, но не считается родной
для большей части нашего народа. Мы знаем языки, но и украинская и русская
культура сегодня основаны на кириллице. А с русской культурой мы воюем в угоду
западным патронам под предлогом сплочения против врага. Но какого сплочения? Мы
не строим новейшие заводы, (включая оборонные). Нет у нас плана с лозунгом "Все
для фронта, все для победы". Не слышал чтобы всерьез модернизировали старые. Не
возрождаем, а убиваем науку. Зато проклинаем русскую культуру, изгоняя ее или
загоняя в нужные рамки. Стало быть надо побороться и с кириллицей, вводя людей
в западное информационное и смысловое поле. 

Добавлю, что это делается под предлогом модернизации общества. Какая
модернизация, если уничтожаются основы общества. Создается модернистский анклав
государства-придатка западной цивилизации. Придатка, а не равного партнера. 

Я гражданин Украины, категорически выступаю против такого статуса. И против
двух перспектив: страну убивают свои правители или ее уничтожают другие
государства. КАКОЙ ВЫХОД: НА ЛИЧНОМ УРОВНЕ  ОСТАВАТЬСЯ ЧЕЛОВЕКОМ, ПОДДЕРЖИВАЯ
СВОЮ КУЛЬТУРУ И ОБРАЗОВАНИЕ. НА ГРУППОВОМ - НЕСТИ КУЛЬТУРУ В СВОЕМ ОКРУЖЕНИИ. С
ЛЮБОВЬЮ К НЕЙ И УВАЖЕНИЕМ К ЛЮДЯМ И ИХ ОСОБЕННОСТЯМ. НА ГОСУДАРСТВЕННОМ -СМЕНА
ВСЕЙ ВЛАСТИ. Чем скорее, тем лучше. Ещё лучше - с уголовным наказанием. Но это,
боюсь, утопия. При любом раскладе оставайтесь людьми собственной культуры:
русской, украинской, еврейской, иной.

\ii{14_09_2021.fb.nikonov_sergej.1.latinica_danilov.cmt}
