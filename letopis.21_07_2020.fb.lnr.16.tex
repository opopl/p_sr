% vim: keymap=russian-jcukenwin
%%beginhead 
 
%%file 21_07_2020.fb.lnr.16
%%parent 21_07_2020
 
%%endhead 
\subsection{Киев пытается привлечь детей из Крыма и Донбасса к обучению в украинских образовательных учреждениях}
\label{sec:21_07_2020.fb.lnr.16}
\url{https://www.facebook.com/groups/LNRGUMO/permalink/2863871240391098/}
  
  
Киев пытается привлечь детей из Крыма и Донбасса к обучению в украинских
образовательных учреждениях. Для этого Верховная Рада внесла поправки в
законопроект «О высшем образовании». Теперь абитуриенты, если они желают
учиться на Украине (исключительно на украинском языке), не должны будут при
поступлении проходить тесты Внешнего независимого оценивания.Однако в стране,
где последние 6 лет идет массовая украинизация, а также переписывается история,
к такой новости не все отнеслись положительно. Так, например, киевский
политолог Юрий Грицак заявил, что перед поступлением в украинские вузы детей из
Донбасса и Крыма нужно проверять на детекторе лжи. По его словам, это якобы
вопрос внутренней безопасности и уменьшения общественного напряжения. Эту
ситуацию в комментарии корреспонденту Новостного агентства «Харьков» оценил
экс-депутат Верховной Рады Алексей Журавко, назвав несколько причин того,
почему крымчане и дончане не поедут учиться на Украину.

«Это очередная глупость. Стоит понимать, что вся безопасность сегодня зависит
от таких людей, как этот так называемый политолог, и всей украинской верхушки
власти. Самое больше зло для Украины --- это они.  Никакой речи не может идти о
проверке дончан и крымчан на детекторе лжи, потому что в данный момент они
ведут себя намного достойнее, правильнее и любяще по отношению к своей Родине.
А Родина для крымчан --- это Россия, у Донбасса --- донецкая земля. Они борются за
свою безопасность, а Украина, к сожалению, никак не борется за нее» .По его
словам, Киев, наоборот, такими заявлениями чиновников и лжеполитологов сильнее
отдаляет от себя Крым и Донбасс, разрушая оставшуюся Украину в тех границах, в
которых она сегодня существует. Журавко уверен, что чем больше таких людей со
странными предложениями будет на Украине, тем быстрее развалится страна. В
связи с этим он предложил помещать людей, которые заявляют подобное, в
специализированные учреждения, где лечат психические расстройства, считая, что
это было бы реальным шагом к безопасности страны.  «Я уверен, что ни крымчане,
ни дончане своих детей учиться на Украину отправлять не будут. Особенно в такую
трудную минуту, когда страной управляют дураки. Насколько я знаю, многие жители
Донбасса обучаются и получают свидетельство об окончании сначала местных
образовательных учреждений, после чего они едут учиться в Российскую Федерацию.

Поймите правильно, сегодня большой риск еще связан и с тем, что такой
репрессивный орган, как СБУ, продолжает свою работу, это угроза для жителей
Крыма и Донбасса. Ведь неизвестно, что потом будет с их детьми на территории
Украины. Впрочем, я не думаю, что родители настолько глупы, что не понимают,
какая ситуация сегодня сложилась на Украине», - сказал бывший депутат Верховной
Рады.

«Тем более что на сегодняшний день украинское образование неконкурентоспособно.
Сами подумайте, ведь то, что они сделали с украинским языком, с
перепрофилированием учебных заведений и обучением по выдуманной истории с
насаждением так называемых новых героев. Поэтому образование на Украине уже не
может быть конкурентоспособным. При этом они попирают и оскверняют этим
настоящую историю и все, что было хорошего на Украине», - подытожил Алексей
Журавко в комментарии корреспонденту Новостного агентства «Харьков».
