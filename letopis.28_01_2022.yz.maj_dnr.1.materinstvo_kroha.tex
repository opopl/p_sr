% vim: keymap=russian-jcukenwin
%%beginhead 
 
%%file 28_01_2022.yz.maj_dnr.1.materinstvo_kroha
%%parent 28_01_2022
 
%%url https://zen.yandex.ru/media/id/5f8f226b1fe36c1d9e02a36b/navstrechu-krohe-programma-podderjki-materinstva-v-dnr-61f40d7c93c6b5521962f90a
 
%%author_id yz.maj_dnr
%%date 
 
%%tags deti,dnr,donbass,materinstvo,zhizn
%%title Навстречу крохе! Программа поддержки материнства в ДНР
 
%%endhead 
 
\subsection{Навстречу крохе! Программа поддержки материнства в ДНР}
\label{sec:28_01_2022.yz.maj_dnr.1.materinstvo_kroha}
 
\Purl{https://zen.yandex.ru/media/id/5f8f226b1fe36c1d9e02a36b/navstrechu-krohe-programma-podderjki-materinstva-v-dnr-61f40d7c93c6b5521962f90a}
\ifcmt
 author_begin
   author_id yz.maj_dnr
 author_end
\fi

Глава Донецкой Народной Республики Денис Пушилин акцентировал внимание на том,
что Программа социально-экономического развития 2022-2024 касается не только
экономики, но и социальной сферы. В том числе и медицины.

В связи с тем, что демографическая ситуация в нашей Республике, в первую
очередь из-за войны, непростая, ее необходимо решать оперативно и системно.

\ii{28_01_2022.yz.maj_dnr.1.materinstvo_kroha.pic.1}

«Мы с вами выстояли и из пепла восстанавливаем ту жизнь, которую хотели
построить, ту жизнь, которой будут гордиться наши дети», - как-то сказал Денис
Владимирович, а говоря о Программе 2022-2024 отдельно акцентировал на
необходимости разработать государственную программу по поддержке материнства.

И вот сегодня эта программа презентована!

С 20 февраля 2022 года будет запущена трехлетняя программа «Навстречу крохе»,
которая будет оказывать дополнительную поддержку женщинам при беременности и
родах с момента постановки на учет будущей мамы до родов малыша.

Цитирую Главу Минздрава ДНР Александра Оприщенко:

\begin{zznagolos}
«Программа утверждена Приказом Минздрава ДНР, и рассчитана на 3 года –
2022-2024 гг., однако планируется продолжение данной программы с изменениями и
дополнениями с учетом приобретаемого опыта в процессе реализации Программы».	
\end{zznagolos}

Главное: Программа «Навстречу крохе» предусматривает гарантированное бесплатное
обеспечение беременной лекарственными препаратами, расходными материалами, а
также проведением необходимых анализов и осмотров врачами.

Для того, чтобы каждая будущая мама знала о своих правах, при постановке на
учет ей будет выдан «Сертификат будущей мамы», в котором подробно указаны не
только этапы ее беременности, но и информация:

- когда необходимо осуществить визиты к врачу,

- какие провести анализы,

- чем будет обеспечена беременная при естественных родах, и, в случае
необходимости, проведения родоразрешения кесаревым сечением,

- советы и рекомендации для обеспечения благополучного вынашивания ребенка.

Важно:

1. Все, что предусмотрено Сертификатом, предоставляется бесплатно, т.е. за счет
бюджетного финансирования.

2. Программа распространяется на всех будущих мамочек, находящихся на учете,
вне зависимости от срока.

Для более подробной информации предлагаю вам пройти по ссылке на сайт Минздрава
ДНР, где можно ознакомиться с соответствующим Приказом и Сертификатом для
будущей мамы.

А я позволю себе привести текст обращения к будущим мамочкам из этого
сертификата:

\begin{zzquote}
	
Дорогая будущая мама!

Весь мир меняется вокруг Вас в один миг и это случается тогда, когда Вы
осознаете, что носите в себе новую жизнь! Если Вы держите в руках, этот
сертификат - примите наши поздравления - совсем скоро Вы станете мамой! Это
настоящее волшебство и, пожалуй, одно из величайших чудес: в течение 9 месяцев
внутри Вас развивается маленькое создание, а потом на свет появляется новый
человечек. Однако, прежде, чем это случится, Вам и Вашим близким предстоит
пройти непростой путь, в котором мы будем рядом с Вами - помогать и
поддерживать Вас в стремлении родить здорового ребенка и сохранить свое
здоровье! Это время, когда Вы должны привыкнуть к особенному, новому для себя
положению.

Сертификат будущей мамы «Навстречу крохе» - это документ, который выдается
каждой беременной женщине, находящейся под наблюдением в учреждении
здравоохранения, подведомственного Министерству здравоохранения Донецкой
Народной Республики, гарантирующий бесплатное оказание минимального
необходимого объема медицинской помощи женщине во время беременности, в родах и
в раннем послеродовом периоде (при условии физиологического течения
беременности, родов и послеродового периода, либо наличия минимальных
отклонений от физиологического течения)

Дальше в сертификат записываются данные будущей мамыи другие важные детали. В
разделах №1 «Беременность. В ожидании чуда» и №2 «Здравствуй, малыш» подробно
расписаны советы и рекомендации будущим родителям, начиная с первого визита к
врачу, и есть даже талоны на посещение врача!

\end{zzquote}

А заканчивается Сертификат прекрасными словами:

«Министерство здравоохранения Донецкой Народной Республики желает здоровья Вам
и Вашему малышу!»

Полностью «Сертификат будущей мамы» в виде вложенного PDF файла на моем
Телеграм канале МАЙ ДНР:

\url{https://t.me/MAYDNR/37881}

А вот памятка для будущих мамочек о порядке действий для участия в программе
поддержки материнства "Навстречу крохе":
