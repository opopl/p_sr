% vim: keymap=russian-jcukenwin
%%beginhead 
 
%%file slova.igra
%%parent slova
 
%%url 
 
%%author_id 
%%date 
 
%%tags 
%%title 
 
%%endhead 
\chapter{Игра}

%%%cit
%%%cit_head
%%%cit_pic
\ifcmt
  tab_begin cols=3

     pic https://storage.lug-info.com/cache/f/9/59c92d91-7684-4655-8c81-360b067f9227.jpg/w1000h616

     pic https://storage.lug-info.com/cache/a/5/dc8b5cca-2135-4265-8b02-841035ab444f.jpg/w1000h616

     pic https://storage.lug-info.com/cache/3/0/71c42b1b-557a-4f45-84c5-1d33c9ec4d23.jpg/w1000h616

  tab_end
\fi
%%%cit_text
Команда знатоков, набравшая наибольшее количество баллов, прошла в суперфинал,
где 29 октября сыграет против телезрителей.  Ответы участников оценивали
депутат Молодежного парламента ЛНР Алина Волкова, заместитель начальника отдела
защиты социально-экономических прав и гарантий трудящихся Федерации профсоюзов
ЛНР, помощник депутата Народного Совета ЛНР Александр Погорелый и помощник
координатора проекта "Лидеры Луганщины" ОД "Мир Луганщине" Анастасия
Николотова. Также за \emph{игрой} следила депутат Народного Совета ЛНР Зинаида
Наден
%%%cit_comment
%%%cit_title
\citTitle{Команда профсоюзных активистов из Луганска прошла в суперфинал игры "Что? Где? Когда?}, 
, lug-info.com, 25.10.2021
%%%endcit

%%%cit
%%%cit_head
%%%cit_pic
%%%cit_text
Хочу відразу зазначити, що подібні \emph{ігри} з історичними персонажами не є
українською особливістю. Виявилося, що достатньо українцям вийти поза межі
традиційної матриці, як вони стають прагматичними і навіть успішними. Прикладів
цьому безліч. Це і трудова еміграція до Північної Америки другої половини ХІХ
століття, і політична еміграція ХХ століття. Де українці, що вийшли поза межі
національно-політичної традиційної парадигми, ставали успішними бізнесменами,
науковцями, правниками. Успішно інтегрувалися в тамтешнє суспільство. А їхні
нащадки ставали міністрами, депутатами і губернаторами провінцій
%%%cit_comment
%%%cit_title
\citTitle{Мертві герої проти живих людей}, 
Василь Расевич, zaxid.net, 29.10.2021
%%%endcit

