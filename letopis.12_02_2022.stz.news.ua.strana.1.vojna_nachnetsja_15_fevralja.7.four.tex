% vim: keymap=russian-jcukenwin
%%beginhead 
 
%%file 12_02_2022.stz.news.ua.strana.1.vojna_nachnetsja_15_fevralja.7.four
%%parent 12_02_2022.stz.news.ua.strana.1.vojna_nachnetsja_15_fevralja
 
%%url 
 
%%author_id 
%%date 
 
%%tags 
%%title 
 
%%endhead 

\subsubsection{4. Давление на Украину}

Последний сценарий - это давление непосредственно на украинское руководство
через падение гривны и закрытие рынков капитала на военных слухах. 

Спектр целей может быть самым разным. От ультимативных требований по внутренней
повестке, выгодной американцам - вплоть до того, что у Зеленского не будет
иного выхода, как резко ускорить мирное решение вопроса Донбасса. Чтобы
наверняка снять угрозу военной эскалации с Россией. 

Ведь если \enquote{завтра война}, то тех, кто увидит \enquote{зраду}, например, в Минских
соглашениях или других формах украино-российского компромисса, станет заметно
меньше.

Тем более, что американцы не устают повторять, что вмешиваться в потенциальную
войну не станут - чтобы не спровоцировать ядерный конфликт с Россией. То есть
Украину не защитят. В таком разрезе даже обещанный Зеленским референдум по
\enquote{Минску-2} может дать неожиданный результат. 

Кстати, неслучайно Байден заявил о выезде граждан США из Украины сразу после
провала вчерашних переговоров \enquote{Нормандской четверки}. 

Впрочем, в России, судя по недавним заявлениям представителя Кремля Дмитрия
Козака, в такую вероятность не особенно верят. Вчера и сегодня он критиковал
Европу и США за то, что те не давят на Киев по Минским соглашениям. 

К тому же, не исключен вариант, что от Украины как раз требуют не пойти на
мирные условия, а начать новую войну - в рамках одного из описанных сценариев
(2 и 3). 

В любом случае есть ощущение, что мы близки к развязке, которая случится уже на
следующей неделе. События сейчас максимально ускорились.

