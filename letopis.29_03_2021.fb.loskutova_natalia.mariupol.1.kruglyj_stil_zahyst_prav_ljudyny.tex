%%beginhead 
 
%%file 29_03_2021.fb.loskutova_natalia.mariupol.1.kruglyj_stil_zahyst_prav_ljudyny
%%parent 29_03_2021
 
%%url https://www.facebook.com/1427894275/posts/pfbid0BMNkhT1howQDKFontunsf2jnnNJViGD9XtFMBHoFdroxMpgvg6FEQSuw7qsbDLuXl
 
%%author_id loskutova_natalia.mariupol
%%date 29_03_2021
 
%%tags mariupol.pre_war,mariupol
%%title Участь викладачів ФІМ у презентаційному круглому столу, присвяченому захисту прав людини
 
%%endhead 

\subsection{Участь викладачів ФІМ у презентаційному круглому столу, присвяченому захисту прав людини}
\label{sec:29_03_2021.fb.loskutova_natalia.mariupol.1.kruglyj_stil_zahyst_prav_ljudyny}

\Purl{https://www.facebook.com/1427894275/posts/pfbid0BMNkhT1howQDKFontunsf2jnnNJViGD9XtFMBHoFdroxMpgvg6FEQSuw7qsbDLuXl}
\ifcmt
 author_begin
   author_id loskutova_natalia.mariupol
 author_end
\fi

Участь викладачів ФІМ у презентаційному круглому столу, присвяченому захисту прав людини

29.03.2021 у Фонді Розвитку Маріуполя відбувся презентаційний круглий стіл та
майстер-клас з використання симуляційних ігор для вивчення прав людини у школі,
у якому взяли участь викладачі факультету іноземних мов Наталія Лоскутова та
Марина Нетреба. Цей захід проходив у рамках проєкту \enquote{SIMSCHOOL – СИМУЛЯЦІЙНІ
ІГРИ ДЛЯ ВИВЧЕННЯ ПРАВ ЛЮДИНИ}, що реалізується командою CRISP (Німеччина) та
EdCamp Ukraine у партнерстві з Європейським Союзом, метою якого є розробка
нового підходу до вивчення прав людини. 

\ii{29_03_2021.fb.loskutova_natalia.mariupol.1.kruglyj_stil_zahyst_prav_ljudyny.pic.1}

У заході брали участь небайдужі до
питання прав захисту людини освітяни: директори шкіл м. Маріуполя та
Волноваського району, вчителі, педагоги-організатори, соціальні педагоги та
викладачі МДУ. Під час круглого столу було презентовано посібник \enquote{Культура й
практика викладання прав людини: поради, вправи, симуляційні ігри}, у якому
надано теоретичний матеріал з опису концептуальної тріади ефективної освіти з
прав людини, а також вправи та симуляційні ігри про права людини. 

\ii{29_03_2021.fb.loskutova_natalia.mariupol.1.kruglyj_stil_zahyst_prav_ljudyny.pic.2}

Під час заходу освітяни мали змогу не лише прослухати теоретичний курс, але й
також долучитися до імплементації вправ та симуляційних ігор. Було протестовано
такі симуляційні ігри як \enquote{Шарада}, \enquote{Взаємозв'язок прав людини},
\enquote{Зроби крок}. Понад усе запрошеним сподобалася симуляційна гра
\enquote{Правила школи}, під час якої учасники взаємодіяли у трьох різних
командах (учні, викладачі, батьки), яким слід було дійти згоди щодо вироблення
нових шкільних правил. Гра сприяла розвитку ціннісного ставлення до прав людини
та почуття відповідальності, справедливості й солідарності. Наприкінці
учасникам було надано поради щодо імплементації симуляційних ігор у їхніх
навчальних закладах. 

\ii{29_03_2021.fb.loskutova_natalia.mariupol.1.kruglyj_stil_zahyst_prav_ljudyny.pic.3}

Загалом захід виявився дуже позитивний, ефективним та
практичним. Учасники опанували нові підходи до освіти з прав людини та отримали
майже покрокову інструкцію про те, як говорити з учнівством про права людини
так, аби це було не перерахуванням законодавчих актів, а живими діями, які
можна було б застосувати у щоденній практиці реального життя в різних його
сферах. Тож попереду – застосування нових знань!

\ii{29_03_2021.fb.loskutova_natalia.mariupol.1.kruglyj_stil_zahyst_prav_ljudyny.pic.4}

%\ii{29_03_2021.fb.loskutova_natalia.mariupol.1.kruglyj_stil_zahyst_prav_ljudyny.cmt}
