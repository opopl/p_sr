%%beginhead 
 
%%file 11_08_2021.fb.arximisto.1.kommunalschiki_ochistili_ot_zaroslej_allei_nekropolja
%%parent 11_08_2021
 
%%url https://www.facebook.com/arximisto/posts/pfbid0MVsoa5XHByksPEz9Sd4TJaJHGDXURPZw72KgKXXZsqugJsHc27FnvyA6KANhL7hzl
 
%%author_id arximisto
%%date 11_08_2021
 
%%tags 
%%title Коммунальщики очистили от зарослей аллеи Мариупольского Некрополя
 
%%endhead 

\subsection{Коммунальщики очистили от зарослей аллеи Мариупольского Некрополя}
\label{sec:11_08_2021.fb.arximisto.1.kommunalschiki_ochistili_ot_zaroslej_allei_nekropolja}

\Purl{https://www.facebook.com/arximisto/posts/pfbid0MVsoa5XHByksPEz9Sd4TJaJHGDXURPZw72KgKXXZsqugJsHc27FnvyA6KANhL7hzl}
\ifcmt
 author_begin
   author_id arximisto
 author_end
\fi

Коммунальщики очистили от зарослей аллеи Мариупольского Некрополя

\#новости\_архи\_города

По просьбе волонтеров работники Комбината коммунальных предприятий очистили от
зарослей две ключевые аллеи Мариупольского Некрополя (Старого городского
кладбища) – аллею, ведущую к усыпальнице Найденовых, и аллею к памятнику Ильи
Кечеджи-Шаповалова, шурина Архипа Куинджи и одного из первых театралов
Мариуполя.

Из-за небывалых ливней вся территория Некрополя бурно заросла бурьянами и
чертополохом, по словам Андрей Марусова, директора общественной организации
\enquote{Архи-Город}, которая координирует волонтерскую деятельность по исследованию и
восстановлению Некрополя. 

Как следствие, мариупольцы с огромным трудом пробираются к могилам своих
родственников. Проведение экскурсий по Некрополю также стало крайне
затруднительным. Между тем, интерес к Некрополю как к памятке
историко-культурного наследия Мариуполя растет. В связи с проведением Мега
Йорты в Мариуполе в сентябре этого года мы ожидаем очередной всплеск интереса,
поскольку Некрополь – это настоящий Пантеон приазовских греков...

Мы благодарим за содействие Ксению Сухову, секретаря городского совета, Галину
Степанкову и Юлию Симоненко, руководителей Комбината, а также Петр Андрющенко,
советника городского головы по промоции города.
