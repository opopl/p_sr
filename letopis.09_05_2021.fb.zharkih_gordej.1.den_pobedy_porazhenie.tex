% vim: keymap=russian-jcukenwin
%%beginhead 
 
%%file 09_05_2021.fb.zharkih_gordej.1.den_pobedy_porazhenie
%%parent 09_05_2021
 
%%url https://www.facebook.com/permalink.php?story_fbid=2594404254198659&id=100008873284825
 
%%author 
%%author_id zharkih_gordej
%%author_url 
 
%%tags 9_may,den_pobedy,kiev,pamjat,pobeda,pobeditel,porazhenie,ukraina,veteran.vov,vov
%%title День Победы, который показывают, как день поражения
 
%%endhead 
 
\subsection{День Победы, который показывают, как день поражения}
\label{sec:09_05_2021.fb.zharkih_gordej.1.den_pobedy_porazhenie}
\Purl{https://www.facebook.com/permalink.php?story_fbid=2594404254198659&id=100008873284825}
\ifcmt
 author_begin
   author_id zharkih_gordej
 author_end
\fi

День Победы, который показывают, как день поражения.

Сегодня я увидел на одном из каналов концерт в честь 9 мая и в начале спросонья
даже и не понял, что это концерт в честь Дня Победы (показалось, что очередная
мулька типа повторяющейся Юрмалы). То певцы пели о том, как плохо, что
случилась война (месседж "спасибо, что не умер"), то устраивались откровенные
свистопляски с танцующими попсовиками нулевых годов, что выглядит так, как
будто это очередная смешная песенка про неверную жену и любовника. Я никогда не
смотрел подобные концерты: они всегда поражали меня своим унынием и совершенно
неверным представлением, что сделали наши великие предки. 

\ifcmt
  pic https://scontent-mia3-1.xx.fbcdn.net/v/t1.6435-0/p180x540/184270913_2594404200865331_305120644909546357_n.jpg?_nc_cat=104&ccb=1-3&_nc_sid=730e14&_nc_ohc=sBesEOxQoDIAX9iGf-2&_nc_ht=scontent-mia3-1.xx&tp=6&oh=a58a9c7eda0d7554ca8c46e864317a15&oe=60D2062D
	caption Вечный Огонь, Парк Славы, Город-Герой Киев
\fi

В их смыслах гуляет либо то, что наши солдаты нехотя выстояли войну, либо то,
что это извращенцы, которым лишь бы девушек покуражить. Не было никогда
показано, что это победители самой кровавой войны в истории человечества и что
они не только мужественно выстояли, но еще и разбили самую сильную армию Европы
у них дома. Казалось бы, концерт в честь святого праздника, но он играет
красками выгодными только тем, кто из года в год повторяет тезисы: «завалили
трупами», «в войне мы победили случайно». Я, как правнук победителей, очень
оскорблен подобным жлобством на нашем ТВ: это мое естественное национальное и
семейное право видеть своих предков героями, а о них всякие менеджеришки
вытирают ноги таким недочеловеческим действом. 

Еще сегодня я был в Парке Славы для возложения цветов в честь своих
предков-победителей и не увидел там ни одного молодого человека (может, они все
же были, но я лично их не видел). Этот праздник стал, на мое великое
разочарование, тусовкой стариков и женщин бальзаковского возраста, хотя его
должны нести потомки победителей – внуки и правнуки. Этот праздник, как смысл,
скорее важен детям и молодым людям как пример для будущих наших побед и
свержений. Такие образы, как «мужчина-воин» и «женщина, которая верно ждет его
с войны», - это не только красиво, это еще являет собой и наши праевропейские
символы, которые идут с давних веков - и 9 мая должен нести также. 

Я, как потомок победителей, не имею право позиционировать своих предков иначе,
как героев из вагнеровских опер или богатырей из  былин. Я чувствую моральную
ответственность и буду учить с женой своих будущих детей этому: мои дети должны
знать историю и быть достойными своих пращуров. И после унтерменшных «великих
фильмов о великой войне», где наши солдаты – это извращенцы и преступники, мы
удивляемся, почему связь с потомками утрачена. 

Мы - не жертвы, мы - потомки победителей, у нас всегда должны гореть глаза к
новым свершениям. Просто посмотрите на фотографии победителей: они красивы и
устремлены, и лично я чувствую трепет, когда смотрю на них. Однако мне
предлагают образы бухих зеков, которые пошли на войну, потому что пистолет в
спину сунули, - образы мерзавцев, придуманные мерзавцами. Пусть такие образы
сгорят в огне истории, и я надеюсь на свершение этого. С праздником Победы,
друзья и близкие.

\emph{Сергей Никонов}

подробнее отвечу на пост потом, но молодежь была и много. Может она просто
отмечает иначе.

\emph{Алексей Кучма}

Ветеранов по пальцам одной руки можно посчитать (живых). Все проходит и все
возвращается. Через 30 лет об этой войне будет не больше воспоминаний, чем
сегодня, о первой мировой. Тоже предки погибали, тоже герои были. Героев и
сейчас хватает. Мне искренне жаль людей погибших в той бойне, но я ведь к этому
имею уже совсем косвенное отношение. Мои предки ещё и от татаро-монгол
пострадали и муки терпели адские (перед глазами образ из "Рублева" Тарковского,
когда смолу бусурмане влили ключарю Патрикию в соборе Успенском, во Владимире ,
за христианский подвиг), но ведь государство не будет создавать из
татаро-монгольского нашествия - мощнейший символ. При всем уважении к
празднующим - ходить в один и тот же день на военный парад - это идеологичный
маркер. К тому же, что должно удивлять, когда у нас на улицах шествия факельные
и ветеранам вчера зиговали (пруфы легко нагуглить), тут уж не надо быть сильно
далёким, что бы понимать, что Украина современная, на стороне нацистов, потому
и нет праздника, потому и мерзавцы. А жертвы - так как исходя из такой логики,
Мы - проиграли (коллективная травма нацистов). И дело не в индивидуальных
предпочтениях. Нас всех разделили по интересам, а пирожок беспонтовый лежит
бесхозный, беснуется.
