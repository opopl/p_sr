% vim: keymap=russian-jcukenwin
%%beginhead 
 
%%file 30_08_2021.fb.helgis_gerus.1.semja_ukrainizacia_mova
%%parent 30_08_2021
 
%%url https://www.facebook.com/ol.geras.98/posts/808711979809806
 
%%author_id helgis_gerus
%%date 
 
%%tags deti,docherj,jazyk,mater,mova,muzyka,semja,ukrainizacia
%%title Щоб з півночі не сказали, що між нами немає ніякої різниці і ми адіннарот
 
%%endhead 
 
\subsection{Щоб з півночі не сказали, що між нами немає ніякої різниці і ми адіннарот}
\label{sec:30_08_2021.fb.helgis_gerus.1.semja_ukrainizacia_mova}
 
\Purl{https://www.facebook.com/ol.geras.98/posts/808711979809806}
\ifcmt
 author_begin
   author_id helgis_gerus
 author_end
\fi

\obeycr
Не знаю, навіщо я це пишу, але просто щоб для себе зафіксувати момент.
Моя абсолютно зросійщена доня-восьмикласниця вже понад місяць у побуті, вдома розмовляє лише українською.
Для нас це дивина, бо наша говірка — цк Тернівський діалект Слобожанщини, тому чистою українською ми не розмовляли ніколи.
А тут — літературна українська, хтозна звідки взялася, бо у нашому регіоні у побуті її катма.
Ніхто їі не вмовляв , ніхто не примушував, лише дотично коли я задав питання, вона посерйознішала і сказала, "щоб з півночі не сказали, що між нами немає ніякої різниці і ми адіннарот".
Я ледь не впав під стіл  @igg{fbicon.smile}  
Ну, аргумент так собі, але вражає те, що ми взагалі на ці теми ніколи не розмовляємо.
Отже, доводиться підтягувати розмовну українську, бо хоча я досить впевнено пишу, чисто розмовляти майже не можу.
Ось так і сталася дивина, коли доня-школярка подала приклад, як треба розвиватися у цьому питанні, бо соромно відчувати, що вона, котру ніколи окрім школи ніде не вчили, розмовляє краще за батьків  @igg{fbicon.smile} 
P.S Її улюблений український РОК-гурт — Брати Гадюкіни.
Я в приємному шоці, короч  @igg{fbicon.smile} 
\restorecr

\ii{30_08_2021.fb.helgis_gerus.1.semja_ukrainizacia_mova.cmt}
