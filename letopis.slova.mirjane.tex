% vim: keymap=russian-jcukenwin
%%beginhead 
 
%%file slova.mirjane
%%parent slova
 
%%url 
 
%%author 
%%author_id 
%%author_url 
 
%%tags 
%%title 
 
%%endhead 
\chapter{Миряне}
\label{sec:slova.mirjane}

Образовательными проектами дело не ограничится.

«Мы, миряне Украинской православной церкви, хотим защитить наш мир, наши семьи, наших детей от этой вакханалии безнравственности и духовного разложения», — говорится в сообщении.

Как защитить?— «Очень просто: объединившись и выступив с активной позицией — за нашу веру, за наши ценности, за нашу Церковь».

Подчёркивается, что «Миряне» — это не только онлайн-проект, это люди — люди, которые «исповедуют православную веру, уважают свою историю, чтут традиции и берегут культуру своих предков».

В своём видеообращении Василий Макаровский предложил мирянам ответить самим себе на вопрос: в какой стране они хотят жить, какие ценности для них важны — «семейные ценности, христианские ценности, или ценности национальной вражды, или, может быть, ЛГБТ?».

