% vim: keymap=russian-jcukenwin
%%beginhead 
 
%%file 17_03_2021.fb.fb_group.story_kiev_ua.1.sobaki
%%parent 17_03_2021
 
%%url https://www.facebook.com/groups/story.kiev.ua/posts/1619858551544273
 
%%author_id fb_group.story_kiev_ua,kuzmenko_petr
%%date 
 
%%tags kiev,kievljane,semja,sobaka,zhivotnoje
%%title Нам до собак еще расти... Чтоб вровень встать с их благородством!
 
%%endhead 
 
\subsection{Нам до собак еще расти… Чтоб вровень встать с их благородством!}
\label{sec:17_03_2021.fb.fb_group.story_kiev_ua.1.sobaki}
 
\Purl{https://www.facebook.com/groups/story.kiev.ua/posts/1619858551544273}
\ifcmt
 author_begin
   author_id fb_group.story_kiev_ua,kuzmenko_petr
 author_end
\fi

Навеяно прекрасной публикацией Светланы Поклад. 

\begin{zznagolos}
\obeycr
Нам до собак еще расти... Чтоб вровень встать с их благородством!
А им, вовек, не доползти, до человеческого скотства...
\smallskip
Александр Розенбаум.
\restorecr
\end{zznagolos}

Свои собаки в моей жизни появились довольно поздно. Сколько помню себя, всегда
просил у родителей завести собаку. Конечно, как все мальчишки моего времени,
мечтал о Джульбарсе или Мухтаре из культовых советских фильмов. Очень хотел
овчарку, но был согласен на любую, даже самую малюсенькую собачку. Совсем
маленьких зауважал, когда малышом увидел как самоотверженно защищала своего
хозяина карманная собачка директора маминой столовой Героя Советского Союза
дяди Вани Кудрина. 

Но, к сожалению, мама с папой были неумолимы. Повзрослев я понял почему. Как
завести собаку, сначала в девятиметровой комнате коммунальной квартиры на
Андреевском спуске, а позже на Автозаводской, с болеющим и почти не выходящим
из госпиталей, больниц и санаториев отцом и всегда работающей мамой и мной,
подростком мечущимся между спортом, книгами, друзьями, школой и любимым
Подолом? Так что мне оставалось только тискать и баловать собак своих школьных
и дворовых друзей. 

Потом случилась школьная любовь, ставшая единственной в
жизни. У Нелли был пинчер Лялька, потом появился метис пуделя и скотч-терьера
Рэм. Я их очень полюбил и они отвечали взаимностью. В военно-морском училище
было не до собак. Тем более Лялька ушла в собачий Рай, а Рэм стал нашей
семейной собакой. Его преклонный возраст и спешный, в связи с Чернобыльской
катастрофой, отлёт всей нашей молодой семьи к месту офицерской службы папы, то
есть меня, сделали невозможным путешествие пса с нами. 

Когда, спустя несколько
лет нелёгкой службы с любимым личным составом срочной службы, я смог перевести
дух и вспомнить о семье, то практически с удивлением обнаружил, что наша жизнь
почти в тайге у моря имеет свои немалые радости и преимущества. Красивое белое
судно - океанограф моё новое место службы, без матросов - срочников на борту,
подарило мне даже свободное время. Я вступил в военно - охотничье общество и
купил дефицитное в то время ружьё в одном из киевских домашних отпусков в
легендарном магазине \enquote{Мисливсьтво та рибальство} на Крещатике. 

С просьбами маленькой дочери о собаке вспомнилась детская мечта. Да и как
новоиспечённому охотнику без четвероногого помощника? Долго выбирая среди собак
охотничьих пород, подходящих для дальневосточного зверья - будущих
предполагаемых объектов добычи, а также компактностью (однокомнатная квартира)
коммуникабельностью и покладистым нравом мы обратили свой семейный взор на
таксу. Но подошли к обретению нового члена семьи со всей ответственностью и
добросовестностью. Была перечитана и изучена масса специальной литературы,
найдены контакты кинологических экспертов - таксятников дальневосточного и
союзного масштаба, освоены азы дрессуры и обучения большой и серьёзной собаки,
на коротких лапках.  В общем, собираясь в очередной отпуск в родной Город мы
уже знали, что наш пёс будет киевлянином. Он будет от рабочих дипломированных
родителей - чемпионов, с прекрасными охотничьими качествами и длиннющими
родословными. 

К этому меня уже
обязывали знакомства в кинологическом мире и данное мной обещание влить свежую
кровь в таксячье поголовье Дальнего Востока. Предшествующие длительные
телефонные разговоры с московскими и киевскими специалистами не прошли даром.
Уже на третий день с прилёта домой мы были в кабинете кинологов Украинского
общества охотников и рыболовов на Нестеровском переулке при осмотре щенков
интересующего нас помёта. 

Незабываемая профессиональная и добродушная киевлянка старой закалки Вера
Андреевна Масс, с которой мы познакомились тогда воочию, осматривала пятерых
чёрно - подпалых тупомордых и длинноухих маленьких увальней. Мы влюбились во
всех. 

Через неделю рано утром, когда дочь ещё мирно спала, мы с женой отправились на
Оболонь за щенком. Как хотелось чтобы вагон метро ехал быстрей! Быстро нашли
требуемый дом и квартиру. На звонок отреагировал бас Таси, мамы малышей. В
комнате с разбежавшимися щенками я стал экспериментировать с брошенной на пол
связкой ключей, громкими хлопками и прочими вычитанными хитростями. 

Неля в это время присела и рассматривала малышей. Из трёх кобельков ей
приглянулся более мелкий, изящный. Пока мы с ней обсуждали кандидатуру будущего
члена семьи, самый большой мальчик спокойно устроился и пригрелся под широкой
длинной юбкой супруги. 

Выбор своей будущей хозяйки и семьи пёс сделал сам. Потом мы везли его на Подол
в метро и щенок смело смотрел по сторонам одаривая искромётными взглядами
улыбающихся попутчиков. Катюша полюбила собачку сразу и беззаветно. Хотела
играть с Арисом (помёт таксят в УООиР был зарегистрирован на букву \enquote{У}
и полное имя нашего такса было Ульрих фон Арис) постоянно. Однако я твёрдо
вознамерился вырастить дисциплинированного охотничьего пса. В какой-то мере это
мне удалось. Но, поскольку я веду повествование в Киевских историях,
останавливаться подробно на охотничьих и выставочных достижениях нашего питомца
не стану. Главное, что родился и провёл первые два с половиной месяца своей
жизни всеобщий любимец в Киеве. Потом для него был год возмужания в
дальневосточной тайге. 

В очередной киевский отпуск мы привезли \enquote{качка}, с играющими под
лоснящейся шкурой мускулами. Знающего основные команды и безукоризненно
исполняющего их (для папы). И разгильдяя, облизанного и зацелованного моими
девочками, когда я на службе. Искусственных нор и притравочных станций для
норных собак в окрестностях Владивостока тогда не было и обучать собаку работе
со зверем в норе мне пришлось в Буче. Весь отпуск, несколько раз в неделю мы с
Арисом ездили на притравочную станцию к нормастеру Валерию Павловичу и его
помощникам.

Эти уроки пёс усвоил крепко и на всю жизнь. Вернулись к Тихому океану мы даже с
рабочим дипломом по лисе добытым на киевских соревнованиях молодняка. 

Потом, рядом с Владивостоком, наши друзья создали охотничье кинологическое
хозяйство \enquote{Маньчжур}. Мы построили в лесу и за 200 метров от морского
пляжа искусственную нору с лисовином Пиней и площадку для притравки лаек и
других пород собак с медведем - белогрудкой Кешей. 

Собаководство стало моим серьёзным увлечением, и
я стал экспертом второй категории охотничьего собаководства норных пород. А наш
Арис стал показывать начинающим собакам, как нужно работать в норе. Помимо
тренировок и состязаний на искусственной норе мы стали охотиться в тайге. 

Пёс блистал и на выставках. Пошли вязки, алиментные щенки. Деньги, которые это
приносило стали серьёзным подспорьем в непростое время начала 90-х. 

Ещё через год мы привезли на родную землю уже чемпиона. Для пополнения
поголовья охотничьих норных собак Приморья из отпуска мы везли обратно корзину
со щенками. Там были киевляне вельш-терьер Реббека, длинношёрстная такса Хелси,
и жесткошёрстные братик и сестра из Луцка. 

С жесткошёрстной девочкой Люкси мы впоследствии не смогли расстаться. Так у нас
появилась ещё одна такса. На её воспитание у меня было уже меньше времени и
возможностей. Она многому научилась у Ариса, но норы невзлюбила. Прекрасно
охотилась на воде и на суше. Была замечательным следопытом. 

Наши таксы успели получить рабочие дипломы на соревнованиях по кровяному следу,
по вольерному кабану и даже по медведю.  Cобаки работали в паре. Из моего
довольно нудного для не собачников рассказа заинтересованный читатель без труда
сделает вывод, что для наших абсолютно добродушных к людям собачек все другие
четвероногие любого размера были априори добычей. Представьте себе, как
непросто было нам, особенно Катюше. 

Ей приходилось гулять с нашими питомцами почти всегда. Мама была занята работой
и подкинувшим массу нелёгких проблем папой. Кстати меня в то тяжёлое время Арис
выручал не только деньгами от вязок. 

Расскажу один забавный случай. Как-то в нашем корпусе госпиталя затеяли ремонт
трубопроводных коммуникаций. Разворотили дыры в санузлах на всех пяти этажах. И
откормленные госпитальные крысы стали ночами свободно гулять на четвёртом этаже
в нашем нейрохирургическом отделении.

Помимо недопустимой в медучреждении антисанитарии они обращали в панику женский
медперсонал и пациенток. Верхом бесцеремонности стал поступок залезшей ночью на
койку моего соседа по палате упитанной крысищи. Вопль Саши слышал весь корпус.

Он был сухопутным офицером и к соседству с этими милыми созданиями, в отличие
от нас - моряков, не привык. После его обоснованных и, большей частью цензурно,
аргументированных протестов наша невропатолог предложила начальнику отделения.
- \enquote{У Петра Константиновича есть охотничьи таксы. Пусть Нелли привезёт
на пару суток. Травить крыс, при наличии большинства лежачих пациентов, не
представляется возможным}. Спросили моё мнение. Я сказал, что можно
попробовать. Гулять с Арисом будет санитар Дима. Еду псу на пару дней Нелли
привезёт. Позвонили. И после обеда жена приехала ко мне с собакой. Арис прожил
в отделении трое суток. Днём дремал у меня под койкой и гулял с Димой в
госпитальном парке, гоняя больших собак и позволяя матросу - санитару заводить
знакомства с их миловидными хозяйками. А ночами пёс выходил на охоту. 

Дверь в нашу палату была открыта и я иногда громко шипел на погавкивающего от
азарта пса. Жалоб на возню и рычание от пациентов не было. Зато каждое утро под
моей кроватью был выложен с десяток добытых трофеев. Моего такса полюбили все. 

Когда мы уезжали домой в Киев через пару лет, подаренные щенки наших собак
стали домашними любимцами моего спасителя главного нейрохирурга, невропатолога,
и одной из замечательных мед сестричек. 

В Киеве, конечно, наши свободолюбивые и решительные питомцы доставили немало
хлопот жителям Почайнинской, Щекавицкой, Волошской, Валов и ещё нескольких
соседних улиц и кварталов. Мы старались по возможности чаще вывозить
четвероногих друзей в Пущу, в Вишенки на дачу кумовей, просто в лес
куда-нибудь. Люкси превратилась в абсолютно домашнюю собаку с характером кошки.
Арис продолжил выставочную и состязательную карьеру.

Когда искусственная нора появилась на Трухановом острове мы стали чаще возить
его на притравочную станцию. Как загорались глаза друга уже в автомобиле, на
подъезде! Правда после нескольких жёстких хваток зверя в норе, с фразой
нормастера - \enquote{Крокодилы сегодня не работают!}, Ариса часто не пускали в работу.

В эти дни он просто дышал любимыми запахами, слушал лай, наблюдал работу других
собак и был счастлив. Говорят, что собаки похожи на хозяев и часто разделяют их
участь. Первый раз у Ариса отказали здание ноги, когда ему было лет 7. Лечили
его, как и хозяина, человеческие нейрохирурги Сакского военного клинического
санатория. Пёс не просто встал на ноги. 

Как-то в Киеве раздался звонок
домашнего телефона нашей подольской квартиры. Звонила незабвенная Вера
Андреевна, главный таксячий эксперт Киева. - Ребята! У нас республиканские
состязания на Трухановом. Арис будет? Жена попробовала отвертеться. - Он давно
диванный пёс. Года два назад был паралич, еле вылечили. Я сказал поедем,
тряхнём стариной! Конечно пёс был на вершине счастья. Как будто не было
прожитых почти десяти лет, страшной болезни, диванной гиподинамии. 

Нас, хозяина и пса, на норе встретили тепло и радостно по-свойски. Мне
разрешили даже посидеть рядом со столами судейской коллегии и посмотреть работу
парочки собак.  Потом, невзирая на однотонный пронзительный вой ожидающих своей
очереди ягд-терьеров, некоторые из которых находились даже в подвшенных на
сучьях окрестных деревьев рюкзаках (собачники - заводчики этой достойной
охотничьей породы меня поймут) моего четвероногого пенсионера вне очереди
пустили в нору.

По старой памяти лис был предварительно сменён на свежего, а нормастер
оживился. Положенные для преследования круги в норе - восьмёрке Арис намотал
довольно быстро, хотя и прыгали с грохотом верхние крышки. Пёс здорово
раздобрел на домашних вкусняшках. Потом контакт и мгновения хватка по месту. Он
даже начал привычно тащить к выходу успокоившегося опытного зверя, как в
юности. Наше с Арисом самолюбие польстила фраза Веры Андреевны сказанная
присутствующим: - \enquote{Вот смотрите, что значит опыт. Лежала собака
несколько лет на диване, пирожки грызла. Пришла и сработала превосходно!} Потом
мы получили кубок за первое место среди такс и за второе общее и довольные
уехали домой.

Это была последняя гастроль нашего любимца такого представительного уровня.
Доживали свой век наши таксы довольно мирно и спокойно, гуляя по Подолу и
иногда выезжая на природу. В совсем пожилом возрасте Арис перенёс инфаркт, а у
Люкси стала развиваться опухоль соска. Псу в 13,5 лет снова парализовало задние
конечности. Медикаментозно удалось слегка их восстановить. Он не таскал лапы за
собой, а прыгал опираясь на две негнущиеся ноги. Однако всё ещё отчаянно лаял
на всех четвероногих. Своей смертью нашим любимцам умереть всё же не пришлось.

Когда Арису было 15, а Люкси 14 лет какие-то нелюди травили собак на Подоле.
Варварским способом, обрызгивая непахнущим ядом углы зданий, подножия столбов и
деревьев. Собак погибло тогда очень много. Вет клиники боролись. Некоторых
молодых спасли. Арис перенёс тяжёлую операцию. Однако возраст и слабое сердце
подвели. В ту же ночь любимца не стало. 

Я был рядом и никогда не забуду его прощающихся глаз. Люкси пережила друга
всего на две недели. Отравление спровоцировало обострение онкозаболевания.
Собак с тех пор у нас с женой не было. А дочь с семьёй завели сначала западно -
сибирскую лайку красавца Клайда и, не так давно, всё же таксу Терри. Позавчера
Клайду было 6 лет, а вчера исполнился год Терри. 

Мы их очень любим, тискаем, когда гостим у детей и внуков, а также   непременно
регулярно приветствуем по видеосвязи. Породу, после ухода наших любимых такс,
через пару лет, нам пришлось сменить кардинально. Проходя ординатуру в одной из
детских киевских больниц дочь однажды принесла с дежурства маленькую
шотландскую вислоухую кошечку окраса поинт. Котейку отдал своему лечащему
доктору Екатерине Петровне мальчик, попавший в больницу с астматическим
бронхитом в третий раз за короткий промежуток времени. Причиной болезни
оказалась аллергическая реакция на шерсть его любимицы. Кошечка напоминала
героя мультфильма \enquote{Котёнок по имени Гав}.  Нелли растаяла сразу. Я
свыкался с кошкой долго. Теперь Масяне почти 13 лет и она чувствует себя
хозяйкой в доме и, по меньшей мере, герцогиней. А я всё-таки мечтаю о маленькой
жесткошёрстной миниатюрной кроличьей таксе. 

Таким получился рассказ киевлянина о животных - киевлянах, которым пришлось
попутешествовать с хозяевами, но всегда возвращаться домой.

\ii{17_03_2021.fb.fb_group.story_kiev_ua.1.sobaki.cmt}
