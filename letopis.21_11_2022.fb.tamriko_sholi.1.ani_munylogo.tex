% vim: keymap=russian-jcukenwin
%%beginhead 
 
%%file 21_11_2022.fb.tamriko_sholi.1.ani_munylogo
%%parent 21_11_2022
 
%%url https://www.facebook.com/tamriko.sholi/posts/pfbid06gedNmUx6Xwb1qHHbDAb6WhuMbitutTgvnQs4dtwB1HZ1JMRnPeG1TyPm4Xu3ZGkl
 
%%author_id tamriko_sholi
%%date 
 
%%tags 
%%title Ані минулого, ані майбутнього. За що варто триматися?
 
%%endhead 
 
\subsection{Ані минулого, ані майбутнього. За що варто триматися?}
\label{sec:21_11_2022.fb.tamriko_sholi.1.ani_munylogo}
 
\Purl{https://www.facebook.com/tamriko.sholi/posts/pfbid06gedNmUx6Xwb1qHHbDAb6WhuMbitutTgvnQs4dtwB1HZ1JMRnPeG1TyPm4Xu3ZGkl}
\ifcmt
 author_begin
   author_id tamriko_sholi
 author_end
\fi

Я наразі проживаю дивний, але важливий досвід - я працюю з біженцями та
вимушеними переселенцями. Не як координатор, бо вони - мої колеги. Так, після
майже 9-річної перерви я знову в редакції, в команді, однак цього разу вона на
100\% (окрім засновників) складається з тих, хто знає на власному досвіді, що
воно таке - війна в твоїй країні. Мої колеги з різних країн, про всіх і про сам
проєкт розкажу іншим разом, зараз хочу про один з наших діалогів.

Ми їхали декілька годин у потязі, поверталися з робочої поїздки, та
спілкувалися. Мій колега з Афганістану сказав:

- Мабуть, декому я можу здаватися трошки інфантильним через мої деякі дитячі
бажання. Але тут все просто: в мене не було дитинства. Потім роки навчання,
вимушена еміграція через смертельну загрозу, інтеграція в новій країні з нуля,
а потім я зрозумів, що  прямо зараз хочу і можу деінколи подарувати собі трошки
дитинства, яке в мене відібрали.

Я поклала ці слова у кишеню і днями розмірковувала над ними. У мене також було
вкрай дивне дитинство у Луганську 90х. Чи то було дитинство взагалі? Чи дається
це зараз взнаки у моєму дорослому житті? Чи здаюся я іншим де-не-де
інфантільною? Неважливі. Неважливі відповіді на ці питання. Адже

в нас немає по-справжньому ані минулого, ані майбутнього. 

Найцінніше, що ми маємо у будь-яку епоху і у будь-який час, - це ЗАРАЗ.

Це те, що я відчуваю і хочу зробити - прямо зараз. 

Навіть якщо це комусь може здатися інфантильним. Бо чи дорослішаємо ми взагалі
- то є теж велике питання.

\ii{21_11_2022.fb.tamriko_sholi.1.ani_munylogo.orig}
\ii{21_11_2022.fb.tamriko_sholi.1.ani_munylogo.cmtx}
