% vim: keymap=russian-jcukenwin
%%beginhead 
 
%%file slova.okkupacia
%%parent slova
 
%%url 
 
%%author 
%%author_id 
%%author_url 
 
%%tags 
%%title 
 
%%endhead 
\chapter{Оккупация}
\label{sec:slova.okkupacia}

%%%cit
%%%cit_head
%%%cit_pic
%%%cit_text
В \emph{оккупированных} городах Украины издавались газеты на украинском языке,
а помощникам нацистов из местных разрешали забирать имущество своих жертв -
например, евреев. Также они получали льготы по ведению мелкого бизнеса вроде
торговли (часто это были \enquote{отжатые} у тех же евреев или поляков магазины или
хозяйства).  Но любые претензии на политическое самоуправление давились на
корню. И в этом смысле \enquote{украинофильские} воззрения Альфреда Розенберга так и
остались теорией. Хотя, разумеется, никаким украинофилом он не был. Основой его
теории по славянскому вопросу был раскол среди ключевых народов СССР. И главным
\enquote{раскольником} Розенберг видел Украину 
%%%cit_comment
%%%cit_title
\citTitle{22 июня - 80 лет нападения на СССР. Что немцы готовили для украинцев}, Максим Минин, strana.ua, 22.06.2021
%%%endcit


%%%cit
%%%cit_head
%%%cit_pic
%%%cit_text
Тут прилетела ещё одна новость, напрямую не связанная с газопроводами, но она
тоже про истерики и хотелки Табаки: сейм Литвы призвал власти требовать от
России компенсацию за \enquote{\emph{советскую оккупацию}}. У меня инсайд, я
знаю точную цифру, сколько Вильнюс получит хоть в рублях, хоть в евро, хоть в
баксах за \enquote{\emph{оккупацию}}. Обращайтесь, подскажу.  На этом все, если
вам понравилось прочитанное, то подписывайтесь, комментируйте, ставьте лайки
%%%cit_comment
%%%cit_title
\citTitle{Вой коллективного Табаки}, 
Мак Сим, zen.yandex.ru, 29.06.2021
%%%endcit

%%%cit
%%%cit_head
%%%cit_pic
%%%cit_text
Если сложить вместе поздравления первых лиц и методички Института нацпамяти, то
получается, что 28 октября 1944 г. наши соотечественники в составе войск
антигитлеровской коалиции в ходе немецко-советской войны завершили изгнание
нацистов с территории Украины, чем и завершили ее повторную \emph{оккупацию}. При этом
неназванные \emph{оккупанты} "стали примером для нового поколения украинских
защитников и защитниц, которые обороняют Украину на востоке".  Так что у всех,
кто не хочет тоже сойти с ума, пробираясь наощупь в этом безобразном мире, День
освобождения Украины от фашистских захватчиков по-прежнему случился в ходе
Великой Отечественной войны усилиями 1-го и 4-го Украинских фронтов РККА.  С
праздником!
%%%cit_comment
%%%cit_title
\citTitle{Если принять риторику власти по освобождению Украины, можно сойти с ума / Лента соцсетей / Страна}, 
Дмитрий Заборин, strana.news, 29.10.2021
%%%endcit

%%%cit
%%%cit_head
%%%cit_pic
%%%cit_text
После развала СССР, когда Малая Россия (Украина) стала «незалежной», снова
начался период агрессивной украинизации всего русского. Напомню, что периоды
активной украинизации, ксенофобского украинского нацизма, были связаны с
властью Центральной рады и Директории после революции 1917 года в России,
германской \emph{оккупацией} во время Первой и Второй мировой войн, политикой
радикальных революционеров, большевиков, которые в 1920-е и начале 1930-х годов
взращивали украинскую интеллигенцию, «язык» в противовес «великорусскому
шовинизму»
%%%cit_comment
%%%cit_title
\citTitle{Шевченко без украинизма}, , topwar.ru, 09.03.2019
%%%endcit

%%%cit
%%%cit_head
%%%cit_pic
%%%cit_text
Подготовка к путинской агрессии у нас ведется по заветам Петра Алексеевича –
власть мечется по миру, прося больше оружия и больше денег на армию. Но есть
большие сомнения в эффективности подобных действий.  Политика последних семи
лет привела к тому, что часть страны (и не очень маленькая) чувствует себя как
под \emph{оккупацией}. Их история объявлена неправильной. Их предки – врагами. Их язык
вытесняется на законодательном уровне, а дети учат в школе стихи, в которых они
не понимают половины слов. Миллионы людей де-факто стали гражданами второго
сорта, пораженными в элементарных правах – их можно оскорблять и даже убивать.
Когда они выйдут на проспект Бандеры с цветами навстречу российским танкам – в
чем именно их можно будет обвинить?
%%%cit_comment
%%%cit_title
\citTitle{Чувствуют ли граждане Украины эту страну своей собственной? / Лента соцсетей / Страна}, 
Вячеслав Чечило, strana.news, 06.12.2021
%%%cit_url
\href{https://strana.news/opinions/365636-chuvstvujut-li-hrazhdane-ukrainy-etu-stranu-svoej.html}{link}
%%%endcit
