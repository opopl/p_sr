% vim: keymap=russian-jcukenwin
%%beginhead 
 
%%file 12_10_2019.stz.news.ua.mrpl_city.1.mariupol_zhivopisec_petr_kot
%%parent 12_10_2019
 
%%url https://mrpl.city/blogs/view/ko-dnyu-hudozhnika-mariupolskij-zhivopisets-petr-kot
 
%%author_id burov_sergij.mariupol,news.ua.mrpl_city
%%date 
 
%%tags 
%%title Ко Дню художника: мариупольский живописец Петр Кот
 
%%endhead 
 
\subsection{Ко Дню художника: мариупольский живописец Петр Кот}
\label{sec:12_10_2019.stz.news.ua.mrpl_city.1.mariupol_zhivopisec_petr_kot}
 
\Purl{https://mrpl.city/blogs/view/ko-dnyu-hudozhnika-mariupolskij-zhivopisets-petr-kot}
\ifcmt
 author_begin
   author_id burov_sergij.mariupol,news.ua.mrpl_city
 author_end
\fi

В год, когда все советские люди и прогрессивное человечество, если верить
тогдашним советским газетам, радио и телевидению, готовились достойно встретить
50-ю годовщину Октябрьской Революции, по приглашению первого секретаря горкома
Владимира Михайловича Цыбулько в наш город приехала группа выпускников
Киевского художественного института. Среди них был и \textbf{Петр Кот} - молодой
живописец, с отличием окончивший вуз. Позади у него было тяжелое детство и не
менее тяжелая юность, служба на флоте, учеба в Днепропетровском художественном
училище и, наконец, в Киевском художественном институте. Молодым художникам
через короткий срок, кажется, месяцев через семь, предоставили квартиры. Хоть и
небольшие, но по тем временам достаточно благоустроенные. Несмотря на то, что
условия для работы были неплохими и в коллективе местных художников были
мастера довольно высокого уровня, часть прибывших из Киева художников в разные
годы покинули наш город. Петр Кот остался. С той поры и до самой кончины, до
октября 2003 года, жизнь его была связана с Мариуполем.

\ii{12_10_2019.stz.news.ua.mrpl_city.1.mariupol_zhivopisec_petr_kot.pic.1}

Первые впечатления Петра Алексеевича от нового для него места жительства: \enquote{Меня
поразил большой промышленный город, его заводы. Но больше всего поразили люди,
которые там работали}. Первое близкое знакомство молодого художника с этими
людьми состоялось в доменном цехе завода \enquote{Азовсталь}. Там, рядом с огнедышащей
домной, в грохоте механизмов, в тревожном сумраке литейного двора, временами
озаряемом красными отсветами потоков расплавленных чугуна и шлака, он рисует
портреты горновых, газовщиков, мастеров. Эти рисунки источают горячее дыхание
цеха. Петр Кот создавал образы своих героев энергичными, мощными, но в то же
время экономными касаниями угля или карандаша к бумаге. Просто на большее у
него не было времени. Рабочие позировали ему в короткие промежутки между
выпусками из печи чугуна. Впрочем, можно ли назвать позированием считанные
минуты отдыха рабочих, еще не отрешившихся от схватки с расплавленным металлом.

Позже Петр Алексеевич напишет портреты доменщиков Георгия Гриненко и Дмитрия
Овчаренко, но уже в своей мастерской. Конечно, в этих работах, написанных
маслом, иной подход. Углубленное постижение сущности человека создается и
тщательной лепкой лица, и выбором позы, в наибольшей степени характеризующей
портретируемого. Естественно, не только рабочие доменного цеха запечатлены на
холстах художника. Среди героев его картин и судокорпусник Виктор Ряпалов, и
инженер-металлург Александра Сидоренко, и крановщица Любаша, кстати, на
редкость лиричный и тонкий по колориту портрет. За годы работы в Мариуполе кто
только не побывал в мастерской талантливого портретиста: сталевары и юристы,
ученые и моряки, строители и ветераны войны. Иногда он писал там, где жили и
трудились его модели: хлебороба Михаила Зиновьева - у стены его крестьянского
дома, рыбаков Николая Пацёру и Ивана Винниченко - на берегу их кормильца,
Азовского моря. Здесь Петр Кот продемонстрировал не только мастерство в
области портрета, но и незаурядный талант пейзажиста. Природа на этих полотнах
в полной гармонии с настроением и фактурой портретируемого, а не только некий
фон, как это порой бывает.

С трепетным чувством любви и почтительности выполнены портреты земляков –
хлеборобов Киевщины – тружеников, умудренных жизненным опытом и одаренных
судьбой великим терпением. Это и портрет матери художника Параски Зиньковны, и
лесника Анастасия Лепехи, и сельского столяра Тимоха Зайца, и колхозника Дмитра
Махтеевича. Их чаяния, их думы, их мироощущение и их быт художнику были
известны. Он ведь сам родился в крестьянской семье. В семье, где главными
приоритетами были трудолюбие, честность и преданность \enquote{\em рідному краю и рідній
мові}. Видно, еще в родном гнезде были заложены в него и трудолюбие, и упорство
в достижении цели, и уважительное отношение к личности. Оттуда тяготение к
людям, одержимым в труде, людям с цельными натурами.

Особое место в творчестве мастера занимает серия портретов ветеранов
Отечественной войны. Как-то он сказал: \emph{\enquote{Тут у меня как бы долг перед отцом.
Отец у меня не вернулся с войны, погиб. И мне, конечно, хотелось поближе
познакомиться с ветеранами, как бы с друзьями отца. Это были интересные люди,
прошедшие сквозь горнило боев, израненные, но духовно не сломившиеся. Я получил
огромное удовлетворение над их портретами}}. Всего один пример. Глядя на
полковника Морозова – участника гражданской и Отечественной войн,
запечатленного на холсте, нельзя не поверить в искренность слов художника.
Перед зрителем предстает воин, испытавший все искусы жизни: и ратный труд, и
славу, и горечь разочарований, и боль ран. Но вся его душевная и мыслительная
работа – там, внутри его, а грубые, будто вырубленные из кремня черты лица – та
крепостная стена, за которую доступ посторонним заказан. И руки, сцепленные в
замок с силой, доступной лишь человеку с непреклонной волей – еще одна защита
от непрошеного сочувствия. Нужно отметить, что в портретах Петра Алексеевича
руки характеризуют человека едва ли не с той же силой, что и лицо.

В череде образов, оставленных Петром Котом для истории Мариуполя, много
личностей, бывших в городе, как говорится, на виду: знатных сталеваров, членов
партии с большим стажем, участников революции. Кроме этих формальных
признаков, они еще были тружениками и воинами. Людьми, пусть заблуждающимися,
но верящими в идею всеобщего равенства и братства трудящихся. \emph{\enquote{Все они для
меня представляли большой интерес, они остались героями, какими в
действительности и были}}, - это тоже слова П. А. Кота, сказанные в одной из
телепередач. Редкая, а особенно в наше время, способность не менять обличье в
зависимости от обстоятельств. Проходят годы, меняются официальные взгляды на
историю, ниспровергаются идеалы. В недавних обожествленных вождях находят
черты злодеев. И только самоотверженные труженики и отважные защитники Родины
на все времена остаются героями. Их-то и запечатлел на своих полотнах и
рисунках Петр Алексеевич Кот.

Как человек, чье детство и юность прошли среди живописной природы Украины, Петр
Алексеевич не мог не обратиться к пейзажу, как самостоятельному жанру. Это
видно по числу полотен с изображением милых его сердцу мест: речушка Басанка,
виды хутора Тымки, левады, опушки леса, вековые ивы. Он писал одни и те же
уголки родной ему земли в конце лета и осенью, в разное время дня, но чаще
утром. Кот не стремился создать пейзаж эффектный, поражающий зрителя броским
колоритом или необычным композиционным приемом. Безукоризненное владение
полутоновыми соотношениями позволяло ему передать в своих картинах то состояние
природы, то настроение, которое довелось испытать ему самому. Петр Алексеевич
объяснял свои принципы в живописи: \emph{\enquote{Мои учителя Карп Демьянович Трохименко,
Виктор Григорьевич Пузырьков, Николай Степанович Боровский, Илья Насонович
Штильман говорили, что чем меньше в картине надуманности, а больше жизни, тем в
большей мере она может быть отнесена к произведению искусства. Натура – вот
источник моего вдохновения, источник тем, источник познания. То, из-за чего я
каждый день беру в руки карандаш или кисть}}.

Он был человеком неравнодушным. Его заботила судьба хуторов на Украине и то,
что безвозвратно уходит в прошлое старинный, веками сложившийся уклад в
украинских селах, что оскудели стада коров. Всегда старался помочь своим
собратьям-художникам. Не только стремился, но и помогал. Хлопотал, чтобы
облегчить их быт, сохранить для будущих поколений их произведения, особое
внимание он уделял участникам Отечественной войны. Он поддерживал молодых
талантливых художников. За время его руководства Мариупольской городской
организацией Союза художников Украины ее состав увеличился с двенадцати до
тридцати семи членов Союза.

Творениями заслуженного художника Украины Петра Алексеевича Кота любуются
посетители картинных галерей Феодосии, Бердянска, Афин, Красного Луча, Яготина,
частных коллекций Украины, России, США, Италии, Франции, Тайваня, Болгарии,
Польши. И только для жителей Мариуполя, где прошли самые плодотворные годы
творчества талантливого художника, его произведения были почти недоступны. За
несколько месяцев до своего ухода из жизни он с грустью сказал: \enquote{\em Была бы в
Мариуполе картинная галерея, подарил бы я городу портреты тех, кто составил
славу нашего города. Шесть почетных граждан позировали мне, Герои Советского
Союза, просто достойные люди...} Желание Петра Алексеевича было исполнено. Теперь
с его работами можно встретиться в Художественном музее имени А. И. Куинджи.

\textbf{Читайте также:} \href{https://mrpl.city/blogs/view/makarenko-lyubov-vasilivna-najgolovnishetse-samopovaga}{%
Макаренко Любов Василівна: \enquote{Найголовніше – це самоповага!}, Ольга Демідко, mrpl.city, 11.10.2019}
