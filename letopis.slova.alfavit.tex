% vim: keymap=russian-jcukenwin
%%beginhead 
 
%%file slova.alfavit
%%parent slova
 
%%url 
 
%%author 
%%author_id 
%%author_url 
 
%%tags 
%%title 
 
%%endhead 
\chapter{Алфавит}
\label{sec:slova.alfavit}

%%%cit
%%%cit_head
%%%cit_pic
%%%cit_text
\begin{itemize}
\item 4. Ни в коем случае не следует давать местному населению более высокое образование. Если мы совершим эту оплошность, мы сами породим в будущем сопротивление против нас. Поэтому, по мнению фюрера, вполне достаточно обучать местное население, в том числе так называемых украинцев, только чтению и письму.
\item 5. Ни в коем случае мы не должны какими бы то ни было мероприятиями развивать у местного населения чувства превосходства! Необходимо делать как раз обратное!
\item 6. Вместо нынешнего \emph{алфавита} в будущем в школах надо ввести для обучения латинский шрифт.
\item 7. Немцы должны быть обязательно удалены из украинских городов. Даже размещение их в бараках вне городов лучше, чем поселение внутри городов! Ни в коем случае не следует строить русские (украинские) города или благоустраивать их, ибо местное население не должно иметь более высокого жизненного уровня.
\end{itemize}
Немцы будут жить в заново построенных городах и деревнях, строго изолированных
от русского (украинского) населения. Поэтому дома, строящиеся для немцев, не
должны быть похожи на русские (украинские). Мазанки, соломенные крыши и т.д.
для немцев исключаются.
%%%cit_comment
%%%cit_title
 \citTitle{22 июня - 80 лет нападения на СССР. Что немцы готовили для украинцев}, Максим Минин, strana.ua, 22.06.2021
%%%endcit


%%%cit
%%%cit_head
%%%cit_pic
%%%cit_text
Азбука структурно более совершенный \emph{алфавит}, чем абецадло. Хотя
латиница, надо признать, в этом плане вторая в рейтинге.  Наряду со славянскими
языками, кириллица особенно благоприятно-удобный \emph{алфавит}, например, для
тюркских языков. Фонемы Ы, Э, Ч, Ш, Щ, более употребимы у тюрок (особенно у
исконных хунно-тюрок: тувинцев, хакасов, фуюйских кыргызов, шорцев, алтайцев,
казахов, киргизов и нек. др.), чем у многих иных народов нашей планеты.  Азбука
несравненно практичнее и удобнее абецадла прежде всего тем, что звуку в ней
соответствует буква. В латинице же немало фонем, изображаемых сочетанием двух
(а то и нескольких) литер. Например SH и CH — (Ш и Ч) в английском варианте.
Или SCHTSCH — (Щ) в немецком
%%%cit_comment
%%%cit_title
\citTitle{Кириллица VS латиница: что лучше? / Статьи}, 
Александр Абакумов, fraza.com, 17.09.2021
%%%endcit

