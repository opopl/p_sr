% vim: keymap=russian-jcukenwin
%%beginhead 
 
%%file 14_12_2020.news.ru.vesti.golovko_alexei.1.vaccination_sputnik_v
%%parent 14_12_2020
 
%%url https://www.vesti.ru/article/2498738
 
%%author Головко, Алексей
%%author_id golovko_alexei
%%author_url 
 
%%tags sputnik_v,vaccine,covid
%%title Врачи, офицеры, учителя: в РФ началась вакцинация от COVID-19
 
%%endhead 
 
\subsection{Врачи, офицеры, учителя: в РФ началась вакцинация от COVID-19}
\label{sec:14_12_2020.news.ru.vesti.golovko_alexei.1.vaccination_sputnik_v}
\Purl{https://www.vesti.ru/article/2498738}
\ifcmt
	author_begin
   author_id golovko_alexei
	author_end
\fi

\video{https://player.vgtrk.com/5516dd2a-52a6-9b42-9119-f57f9873924e}

\ifcmt
pic https://cdn-st1.rtr-vesti.ru/vh/pictures/xw/307/895/7.jpg
\fi

\begin{leftbar}
	\begingroup
		\em Масштабная вакцинация от коронавируса проходит уже по всей стране.
				Сегодня "Спутником V" начали прививать не только врачей и учителей, но
				и работников МФЦ, сферы культуры, услуг и торговли. По данным
				соцопросов, сделать прививку сейчас готовы вдвое больше россиян, чем в
				начале осени. И как сообщили сегодня создатели вакцины, "Спутник V" на
				100\% защищает от тяжелых случаев COVID-19.
	\endgroup
\end{leftbar}

\index[rus]{Коронавирус!Россия!Вакцинация по всей стране, 14.12.2020}

Теперь это по-настоящему всероссийская процедура. Вакцина от коронавируса
доставлена в каждый регион страны. В Орле первые 200 доз – для местных врачей.

"Вирус никуда не денется, поэтому решил привиться. Подготовлен, думаю, что всё
будет хорошо", – пояснил Андрей Батраченко, детский хирург.

В Петербурге военные летчики и офицеры ПВО пришли на прививку прямо с боевого
дежурства.

В Нальчике "Спутником V" прививаются социальные работники.

А на Чукотке – очередь из школьных учителей.

"Чтобы не нести инфекцию в дом, к близким, мы вакцинируемся и от гриппа
ежегодно, и от пневмококка в этом году и вот впервые от ковида", – говорит
Валентина Синкевич, директор Чукотского института развития образования и
повышения квалификации.

Надежность вакцины постоянно уточняют ученые. Чем больше дней проходит после
испытаний среди добровольцев, тем больше у специалистов данных и поводов
утверждать, что заявленная ранее эффективность в 91,4\% – это не предел.

"На 42-й день эффективность превышала 95\%, и мы ожидаем, что эффективность
будет выше не на 21-й, а на более поздние дни. Мы говорим, что 91,4\% – это
некая низшая граница эффективности", – сообщил Кирилл Дмитриев, генеральный
директор Российского фонда прямых инвестиций.

Но чтобы препарат сработал – особое внимание транспортировке. От склада до
прививочного пункта вакцину везут в машинах-рефрижераторах. Каждое перемещение
ампул – строго регламентировано.

Дальше хранение – только в специальных сертифицированных холодильниках при
температуре не выше -18 градусов.

"За два часа, как будем прививать сотрудников, достается из холодильников и как
только кристаллы льда исчезли, начинаем прививать", – рассказала Валерия
Бруснева, главный врач городской клинической поликлиники №1 Ставрополя.

Проведенный на днях опрос показывает – почти половина россиян уже готовы
сделать прививку в ближайшее время. Еще больше тех, кто уверен, что вакцинация
остановит эпидемию. Две трети знают о вакцине "Спутник V" и три четверти из них
следят за ее испытаниями.

А в Белгороде прививки ставят врачи, уже испытавшие препарат на себе: "Мы все
живы-здоровы, у нас чудесный иммунитет. Иммунный статус просто зашкаливает".

Но пока вакцинация набирает обороты – главное внимание по-прежнему – соблюдению
санитарных норм.

"Больных много, в связи с тем, что люди перестают носить маски", – сообщила
врач.

В Томске полицейские решили проверить, кто это шумит после полуночи в закрытом
баре. Работа в это время сейчас запрещена. А тут и кальянная и компьютерный
клуб. Все клиенты – конечно же, без масок. Заведению теперь выпишут немалый
штраф.

Но больше тех, кто не вредит, а помогает. В Ижевске волонтеры на машинах
подвозят медиков. Разносят лекарства тем, кто на изоляции.

"Если есть такая возможность хотя бы два часа в день, хотя бы один час,
выделите, это совсем не сложно", – говорит Дарья Байраншина, волонтёр.

В Челябинске появились 10 новых машин для развоза врачей неотложки. На вызовы,
к людям с высокой температурой эти автомобили ездят круглосуточно.

В Геленджике ПЦР-тесты пришли сдавать Снегурочки и деды Морозы. Массовые
мероприятия в регионе запрещены, но у этих сказочных персонажей особая миссия.

"У нас есть категории детей, которые ждут дедов Морозов – это дети инвалиды,
дети сироты", – рассказала Елена Василенко, заместитель главы муниципального
образования города-курорта Геленджик.

А в Липецке новогоднее настроение врачам решили создать местные лесники. Во все
больницы и поликлиники они привезли 360 елок. Праздник состоится даже вопреки
эпидемии.
