% vim: keymap=russian-jcukenwin
%%beginhead 
 
%%file 23_02_2022.tg.tkachev_jurij.2.sprashivajut_storona_ukrainy
%%parent 23_02_2022
 
%%url https://t.me/dadzibao/5466
 
%%author_id tkachev_jurij
%%date 
 
%%tags __feb_2022.vtorzhenie,patriotizm,rossia,ukraina
%%title Спрашивают: «Как ты можешь в такой ситуации не быть на стороне Украины?»
 
%%endhead 
 
\subsection{Спрашивают: «Как ты можешь в такой ситуации не быть на стороне Украины?»}
\label{sec:23_02_2022.tg.tkachev_jurij.2.sprashivajut_storona_ukrainy}
 
\Purl{https://t.me/dadzibao/5466}
\ifcmt
 author_begin
   author_id tkachev_jurij
 author_end
\fi

Спрашивают: «Как ты можешь в такой ситуации не быть на стороне Украины?».
Отвечу публично.

У меня украинский паспорт, другого нет и не было. Всю жизнь я старался быть
образцовым гражданином: соблюдал законы, занимался благотворительностью,
участвовал в субботниках, сдавал кровь и это всё. По мере возможности занимался
общественной деятельностью. Платил налоги даже тогда, когда мог этого не делать
(и когда многие из тех, кто сегодня вопят о патриотизме, этого и не делают). 

Но это не помешало мне получить статус врага государства. Да, мне пока повезло
больше, чем тому же Бузине, моим друзьям, погибшим 2 мая или тем, кто провёл
долгие годы в украинских тюрьмах. Но если вы всерьёз думаете, что слежка,
прослушка и фабрикуемые уголовные дела способствуют росту симпатий к
государству, то вы ошибаетесь. Попробуйте хотя бы месяц попросыпаться с
чувством радости, что разбудил вас будильник, а не настойчивый стук в дверь в 6
утра. 

И это притом, что ни до, ни после 2014 года я не нарушил ни одного закона
Украины. Да, мои взгляды сильно расходятся с позицией украинского государства.
И я против его политики по большинству пунктов. Но моё право придерживаться
каких угодно взглядов гарантировано не только бумажкой, которая у Украины
вместо конституции, но и штуками вроде Декларации прав человека. Никаких
противозаконных действий я не предпринимал. По сути, всё, в чём я виновен – это
неправильный образ мыслей.

И более того. В Украине регулярно звучат публичные утверждения, что люди с
моими взглядами вообще должны быть понижены в правах, высланы из страны, а то и
убиты. Я не помню случая, когда за такие высказывания кто-либо был привлечён к
ответственности, т.е. Украина как государство такое мнение разделяет, или как
минимум не возражает.

Да и о чём можно говорить, когда не наказаны люди, не на словах, а на деле
убивавшие инакомыслящих 2 мая 2014 года? И хватит уже рассказывать, что это
было просто несчастное стечение обстоятельств. В условиях, когда
непосредственно перед приходом погромщиков с Куликова поля отозвали дежуривших
там правоохранителей, а ГСЧС целенаправленно затягивали отправку пожарных машин
к зданию, где заживо горели люди, такие заявления отдают откровенным цинизмом.

Впоследствии же правоохранительные органы Украины вместо проведения
расследования обстоятельств трагедии занимались целенаправленным уничтожением
улик, что считается соучастием в преступлении.

Мне говорят: не надо путать страну и государство, народ и власть. А я и не
путаю. Язык и культура моего народа целенаправленно вытесняются государством на
обочину культурной и общественной жизни; в учебных заведениях за
государственный счёт идёт пропаганда второсортности моего народа и враждебного
к нему отношения. 

«Надо сплотиться для защиты от внешнего врага»? Это, простите, от кого? От
России? А мне, простите, Россия пока что ничего плохого не сделала. В отличие,
снова-таки, от. Не Россия все эти годы грозила мне и таким как я лишением
гражданства, высылкой, лагерями, тюрьмой и убийствами. В России не запрещают
мой язык и культуру. Не Россия убивала и сажала моих друзей за неправильные
взгляды. И не Россия прямо сейчас планирует зачистки в т.ч. и меня лично с
использованием банд контролируемых спецслужбами «активистов».  Пусть же мне
назовут хотя бы одну причину для того, чтобы в этой ситуации я был «на стороне
Украины»? Защитить её, чтобы она и дальше могла унижать и угнетать таких как я?
Простите, мазохизм – это не моё.

Достаточно с Украины и того, что в текущей ситуации я не буду участвовать в
войне (на своём, информационном фронте) с противоположной стороны. И делать я
этого не буду не из каких-то мифических симпатий к Украине, а просто потому,
что у меня есть долг перед моими читателями – говорить им правду, или по
крайней мере то, во что верю я сам, и максимально ограждать их от пуль и
снарядов когнитивной войны. 

И кстати, если, как меня уверяют, Украина в этой ситуации – на стороне добра и
правого дела, то просто правды ей должно быть более чем достаточно.
