% vim: keymap=russian-jcukenwin
%%beginhead 
 
%%file 13_05_2022.fb.fb_group.tipichnoe_xtz.1.info_dlja_ludei_v_polshu
%%parent 13_05_2022
 
%%url https://www.facebook.com/groups/xtzevskie/posts/3174974392745820
 
%%author_id fb_group.tipichnoe_xtz
%%date 
 
%%tags 
%%title Інформація для людей, які їдуть до Польщі від війни
 
%%endhead 
 
\subsection{Інформація для людей, які їдуть до Польщі від війни}
\label{sec:13_05_2022.fb.fb_group.tipichnoe_xtz.1.info_dlja_ludei_v_polshu}
 
\Purl{https://www.facebook.com/groups/xtzevskie/posts/3174974392745820}
\ifcmt
 author_begin
   author_id fb_group.tipichnoe_xtz
 author_end
\fi

Добрий ранок

Інформація для людей, які їдуть до Польщі від війни.  Одним із прикордонних
переходів (пішохідний та транспортний ), через які можна перетнути
українсько-польський кордон, є Смільниця – Крощенко. 

Найближчий пункт прийому - Łodyna 41 розташований поблизу пункту пропуску через
кордон у с..Łodyna.

\ii{13_05_2022.fb.fb_group.tipichnoe_xtz.1.info_dlja_ludei_v_polshu.pic.1}

У Приймальному пункті можна зупинитися відпочити, з'їсти  гарячу їжу, отримати
медичну допомогу, отримати польську сім картку,, випрати одяг та отримати всю
необхідну інформацію щодо подальшого перебування в Польщі чи можливість
переїзду в інші країни. 

Ми також надаємо необхідний набір з продуктів та засобів для перебування у
перші дні в Польщі, такі як: косметика, засоби гігієни, одяг, продукти для
дітей та корм для домашніх тварин.

\ii{13_05_2022.fb.fb_group.tipichnoe_xtz.1.info_dlja_ludei_v_polshu.pic.2}

До пункту прийому можна приїхати власним транспортом, автобусом.

Надаємо спальні місця в окремих кімнатах, маємо медичний пункт, дитячу кімнату,
душові, туалети, кухню з їдальнею та пральню.

Адреса для навігації  @igg{fbicon.blue.square}  Łodyna 41, 38-700 Łodyna

Контактна особа +48794647177 Володимир (Telegram) 

Instagram @Merkulovvov

\ii{13_05_2022.fb.fb_group.tipichnoe_xtz.1.info_dlja_ludei_v_polshu.pic.3}
\ii{13_05_2022.fb.fb_group.tipichnoe_xtz.1.info_dlja_ludei_v_polshu.pic.4}

\begin{itemize} % {
\iusr{Наталья Гладыш}
Скажите с украинским паспортном границу с Польши в Украину можно пройти

\iusr{Владимир Меркулов}
\textbf{Наталья Гладыш} да можно, только нужно объяснить причину, отсутствия паспорта загран

\iusr{Наталья Гладыш}
\textbf{Владимир Меркулов} если я его вообще не делала

\iusr{Анна Яковлева}
Никто на границе не спрашивает почему нет загранпаспорта.

\iusr{Владимир Меркулов}
\textbf{Наталья Гладыш} выпускают!! Выехать можно!

\end{itemize} % }
