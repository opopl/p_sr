% vim: keymap=russian-jcukenwin
%%beginhead 
 
%%file 22_12_2020.news.ua.pravda.1.sputnik_v_pobichni_effekty
%%parent 22_12_2020
 
%%url https://www.pravda.com.ua/news/2020/12/22/7277824/
 
%%author 
%%author_id 
%%author_url 
 
%%tags 
%%title Росіяни поскаржилися на побічні ефекти від вакцини "Супутник V"
 
%%endhead 
 
\subsection{Росіяни поскаржилися на побічні ефекти від вакцини \enquote{Супутник V}}
\label{sec:22_12_2020.news.ua.pravda.1.sputnik_v_pobichni_effekty}
\Purl{https://www.pravda.com.ua/news/2020/12/22/7277824/}

\ifcmt
pic https://img.pravda.com/images/doc/2/4/240aa2c-spytnik-v.jpg
\fi

Росіяни, які першими вирішили зробити щеплення від коронавірусу російською
вакциною "Супутник V", поскаржились на побічні ефекти від неї.

\textbf{Джерело:} видання "Медуза",\Furl{https://meduza.io/feature/2020/12/22/zhene-govoril-chto-u-menya-oschuschenie-budto-ya-beremennyy?utm_source=telegram&utm_medium=live&utm_campaign=live} журналісти якого поспілкувалися зі щепленими людьми

\textbf{Деталі:} Вакцинація проти коронавірусу в Росії почалася 5 грудня. У більшості
регіонів зробити щеплення можуть тільки лікарі, вчителі та інші групи ризику.
Але як мінімум у Москві і Підмосков'ї вже зараз вакцинуватися може практично
будь-хто. Однак, зазначається, що охочих небагато.

За словами заступника гендиректора клінінгової компанії, 36-річного Василя, він
вирішив зробити щеплення відразу ж, як тільки дізнався про надходження перших
вакцин в поліклініку. Нічого спеціально про вакцину не вивчав, чув краєм вуха
по телевізору, що її винайшли російські вчені.

Він розповів, що через кілька годин після щеплення почало "ламати" ноги і очі –
навіть дивитися було боляче, особливо в сторони, пізніше почало морозити і
трясти, піднялася температура до 38,6 градуса.

За словами Василя, 8 днів у нього було відчуття, що він застудився. Чоловік
вирішив робити інший укол через 21 день від першої вакцинації.

27-річний співробітник НДУ ВШЕ з Москви Михайло також розповів про побочні
ефекти – у нього через кілька годин після щеплення піднялася температура і
почала боліти шкіра, як при грипі, боліла поясниця і була сильна втома. Але це,
за його словами, тривало недовго, далі він почувався добре. Чоловік планує
робити другу вакцину.

Про подібні побочні ефекти – лихоманку, підвищену температуру, ломоту в тілі,
головний біль заявили й інші люди, які зробили щеплення російською вакциною
"Супутник V".

\textbf{Передісторія: }

\begin{itemize}
\item У Росії троє медиків з Алтайського краю заразилися COVID-19 після щеплення російською вакциною "Спутник V".

\item 26 листопада творці російської вакцини "Супутник V" запропонували
				шведсько-британській компанії AstraZeneca скомбінувати препарати нібито
				для збільшення ефективності.

\item Президент РФ Володимир Путін розпорядився вже з 7 грудня почати масову
				вакцинацію жителів країни від коронавирусу російським препаратом
				"Спутнік V".
\end{itemize}
