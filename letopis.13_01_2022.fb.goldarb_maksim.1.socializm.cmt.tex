% vim: keymap=russian-jcukenwin
%%beginhead 
 
%%file 13_01_2022.fb.goldarb_maksim.1.socializm.cmt
%%parent 13_01_2022.fb.goldarb_maksim.1.socializm
 
%%url 
 
%%author_id 
%%date 
 
%%tags 
%%title 
 
%%endhead 
\zzSecCmt

\begin{itemize} % {
\iusr{Валентина Тышкевич}
Согласна на  @igg{fbicon.100.percent} 

\iusr{Ольга Булатова}
Похоже на путь Китая. Я не говорю, что это плохо. На мой взгляд - это адекватный и справедливый путь.

\iusr{Ольга Авдонина-Липий}
Спасибо за Новогодние пожелания на православный Старый Новый Год! Пусть они сбудутся!!!!!!!

\iusr{Олег Григорьев}
Да , необходимо !

\iusr{Владимир Зинченко}
Когда это случится? когда начнем действовать? или только говорить будем?!

\begin{itemize} % {
\iusr{Максим Гольдарб}
\textbf{Владимир Зинченко} «только говорить» - это Вы обо мне? Или о себе?

\iusr{Владимир Зинченко}
\textbf{Максим Гольдарб} Я вообще о всей стране-а гниды и дальше живут и жируют!
\end{itemize} % }

\iusr{Станислав Прокопчук}
Современный мир давно это понял. Социально-ориентированная экономика вот идеал к которому мы должны стремится.

\iusr{Galina Naugolnaya}
И когда это будет? Доживём ли до этого?

\begin{itemize} % {
\iusr{Максим Гольдарб}
\textbf{Galina Naugolnaya} обязательно. Делать ещё это будете )

\iusr{Galina Naugolnaya}
\textbf{Максим Гольдарб}, 

спасибо, Максим, Вы меня всегда вдохновляете, а то иногда такая безнадёга
берет! Счастья Вам, успехов в Вашем праведном деле, я с Вами! Если можно, хочу
пообщаться по поводу пенсий госслужащих, вышедших на пенсию в 2008-2010г.г. -
надо подумать, что можно сделать! Уходила на пенсию в 2009 г. по госслужбе и
пенсия на тот момент была в три раза больше минимальной. А с реформой ревы ,
т.е. без осовременивания, пенсия осталась на том уровне и составляет 2800 грн.,
хотя теперь сотрудники на моём рабочем месте получают 15000 грн. А мы как нищие
теперь, как - будто никогда и не работали, хоть и с высшим образованием и
должности занимали и работали, выполняя задания государства! За газ должны
платить зимой 3000, 00 грн. и это только за газ + другие коммунальные платежи.
Прости, что излила свою душу.


\iusr{Galina Naugolnaya}

Простите, что в последнем предложении, обратилась на \enquote{ты}, так
получилось, просто не дописала в конце слова \enquote{те}.

\end{itemize} % }

\iusr{Сергей Падалкин}
Нет. Не отдадут империалисты ни недра, ни заводы, ни энергетику, ни землю.

\begin{itemize} % {
\iusr{Виктор Павлухин}
\textbf{Сергей Падалкин} 

Вы пророк?

сша вылезая из одного кризиса попадает в другой и так без конца! То что они там
что то делают???? Кому идет на пользу? они всю Планету тянут если не в могилу,
то в деградацию цивилизации точно!

СССР просуществовал всего 70 лет, а сша всего 240, они в Мире, организовали за
это время, 209 военных и других конфликтов, рекомендую узнать ( поизучять
методом сравнения гугла и Яндекса) сколько штаты украли ноу-хау у Союза в
серьёзных отраслях и исследованиях ( микроэлектроника, авиастроение,
аэродинамика, томография, мобильная связь, гелеоэнергетика, компьютерные
технологии, ядерные технологии в энергетике материаловедение, химия и тд и тп),
только в 90-х они оцифровали все научные библиотеки ведущих НИИ в Москве под
недосмотром алкаша ельцина, так тогда они украли в ЦАГИ научные отчеты по
термозащитным плиткам для Бурана, но не сумели их правильно приклеить и их
наказала жизнь катастрофами Шатла, они украли в 1998 г в Украине 2 Советских
(Российских) ТВЭЛа и скопировали их ( вестенгаузен) ,( они не соизмеримо
совершеннее американских хотябы потому, что шестигранники в сечении, а их
квадрат) потом ради эксперимента поставили их в реактор Южно-Украинской АЭС,
что чуть было не кончилось вторым Чернобылем, спасибо нашим Кулибиным, которые
их извлекли из активной зоны).

Пока эти СУКИ, наши грантоеды проамериканские коллаборанты, будут оккупировать
Киев ни чего в это стране не изменится, а это сможем сделать МЫ, выключив в
голове своих сограждан внутренности от телевизора и начнём восстанавливать им
серое вещество мозга!

\iusr{Сергей Падалкин}
\textbf{Виктор Павлухин} 

Всё верно говорите. НО. Социализм - это общественная собственность на средства
производства. Отдадут олигархи пром.гиганты? Если полмиллиона вооружённых
силовиков их защищает от народа.

\iusr{Сергей Падалкин}

Я, конечно, обеими руками за. И за обновлённый социализм, и за обновлённый
СССР. Но уже сегодня даже за эту фразу демоны грозят тюрьмой.

\end{itemize} % }

\iusr{Виктор Скрынников}
Вперёд, к победе социализма!

\iusr{Яна Латышева}

Звучит отлично. Идея понятна. Вопрос реализации.

Всегда основные составляющие в любом проекте - ресурсы, инструментарий и,
конечно же, управление.

К сожалению, в Украине для реализации идеи сейчас острый дифицит всех трех
составляющих  @igg{fbicon.shrug}  @igg{fbicon.face.pensive} 

\end{itemize} % }
