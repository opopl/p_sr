% vim: keymap=russian-jcukenwin
%%beginhead 
 
%%file 04_12_2021.fb.pushkina_ksenia.kramatorsk.1.dekabrj
%%parent 04_12_2021
 
%%url https://www.facebook.com/kseniya.pushkina.397/posts/619585639483492
 
%%author_id pushkina_ksenia.kramatorsk
%%date 
 
%%tags dekabrj,dobrota
%%title Декабрь особенный месяц
 
%%endhead 
 
\subsection{Декабрь особенный месяц}
\label{sec:04_12_2021.fb.pushkina_ksenia.kramatorsk.1.dekabrj}
 
\Purl{https://www.facebook.com/kseniya.pushkina.397/posts/619585639483492}
\ifcmt
 author_begin
   author_id pushkina_ksenia.kramatorsk
 author_end
\fi

Если согнуть корочку мандарина, то брызнет липкий, сладкий и ароматный сок
цедры — это декабрь.

\ii{04_12_2021.fb.pushkina_ksenia.kramatorsk.1.dekabrj.pic.1}

Если согнуть веточку ели и испачкаться в её смоле так, что пальцы будут липнуть
и пахнуть хвоей еще пару дней — это декабрь.

Если под ногами скрипит снег,

если на рукавах и на ресницах не тают снежинки, а за окном ничего не видно
из-за снегопада — это декабрь.

Если так хочется возвращаться мыслями в детство, вспоминать добрые истории,

перебирать старые фотографии — это декабрь.

Если сладкого хочется чаще, если спать хочется больше, если думаешь о
подарках... - это тоже декабрь.

Декабрь особенный месяц.

\ii{04_12_2021.fb.pushkina_ksenia.kramatorsk.1.dekabrj.pic.2}

Позвольте ему проникнуть в ваши сердца...

Позвольте ему ветром напевать вам любимые мелодии, снегопадом отправлять в
детство,

свечами разжигать чувства,

а сладким напоминать о доброте.
