% vim: keymap=russian-jcukenwin
%%beginhead 
 
%%file slova.raketa
%%parent slova
 
%%url 
 
%%author 
%%author_id 
%%author_url 
 
%%tags 
%%title 
 
%%endhead 
\chapter{Ракета}

%%%cit
%%%cit_head
%%%cit_pic
%%%cit_text
Пятая нерешаемая проблема России – это её страсть ко всему глобальному и равнодушие к малому.
Нет в мире, должно быть, другого народа, который бы так гордился полётами в
космос, масштабом своей территории, завоеванием или, если угодно, освобождением
других народов, словом, любыми историческими сюжетами, где виден размах – и в
то же время был так беспомощен в повседневном улучшении той скромной
реальности, что дана не Гагарину или Жукову, а мелкому руководству и населению
какого-нибудь жилого квартала. Мы можем навести порядок в Сирии, но не в
Рязанской области, и корни этой драмы находятся где-то намного дальше, чем
принято думать, не только в глупости или жадности конкретного Иван Иваныча.
Должно быть, в самом устройстве русской культуры есть что-то глубоко кочевое,
но не в смысле «кочевников», какими их видит исторический миф. Скорее, русский
кочевник – это военный, переезжающий из одних казарм в другие, крестьянин,
сжигающий лес, чтобы засеять поле, но через несколько лет двинуться дальше,
казак-конкистадор, чиновник, перемещаемый на огромные расстояния распоряжениями
сверху, беглый крепостной или ссыльный преступник, ищущий работы в городе
колхозник, нынешний вахтовый продавец или охранник. Русские не дружат с
оседлостью, им вечно что-то мешает как следует обустроиться на одном месте –
нашествия, стихийные бедствия, власть, – но если бы можно было помечтать, то
борьба с борщевиком, которым зарастает страна, кажется мне важнее запуска
\emph{ракеты}
%%%cit_comment
%%%cit_title
\citTitle{Семь нерешаемых проблем России}, Дмитрий Ольшанский, 15.07.2021
%%%endcit

