% vim: keymap=russian-jcukenwin
%%beginhead 
 
%%file 19_01_2022.yz.italomania_i_gastromania.1.etruski_russkie
%%parent 19_01_2022
 
%%url https://zen.yandex.ru/media/italy/etrusski-eto-russkie-ili-net-poprobuem-razobratsia-alfavit-tochno-pohoj-61e80efaf9f75d0fb45a8f17
 
%%author_id yz.italomania_i_gastromania
%%date 
 
%%tags alfavit,etruski,pismennost',russkie
%%title Этрусски - это русские или нет?
 
%%endhead 
 
\subsection{Этрусски - это русские или нет?}
\label{sec:19_01_2022.yz.italomania_i_gastromania.1.etruski_russkie}
 
\Purl{https://zen.yandex.ru/media/italy/etrusski-eto-russkie-ili-net-poprobuem-razobratsia-alfavit-tochno-pohoj-61e80efaf9f75d0fb45a8f17}
\ifcmt
 author_begin
   author_id yz.italomania_i_gastromania
 author_end
\fi

Древний спор о происхождении этрусков не могли разрешить на протяжении
нескольких веков. Учёные пытались найти ответы, изучая культуру, историю,
обычаи и языки. Но не так давно в дело включились генетики и похоже их
исследования положат конец догадкам о том, кто же всё-таки этруски – русские
или итальянцы?

\ii{19_01_2022.yz.italomania_i_gastromania.1.etruski_russkie.pic.1}

\subsubsection{Что вообще известно об этрусках?}

Этруски – довольно загадочная цивилизация, о которой сохранилось не так много
прямых фактов. Известно лишь, что этот могущественный народ жил ещё в доримской
Италии начиная с 9 века до н.э.

Их язык и культура вызывают неподдельный интерес у археологов и историков,
заставляя искать всё новые доказательства их принадлежности к той или иной
культуре.

\ii{19_01_2022.yz.italomania_i_gastromania.1.etruski_russkie.pic.2}

\begin{zznagolos}
В основном информация об их традициях, обычаях и религии дошла до нас благодаря
рассказам древних римлян. 	
\end{zznagolos}

Современные исследования полагают, что родиной этрусков были Причерноморские
степи. Но по каким-то причинам (возможно, голод или неурожай) им пришлось
покинуть родные края в начале бронзового века и двинуться на север Италии.

\subsubsection{Какими были этруски?}

Мужчины носили короткие волосы и коротко стригли бороду, любили устраивать
зрелищные бои и париться в бане. Женщины же напротив имели длинные волосы,
которые носили распущенными и тщательно за ними ухаживали. 

\begin{zznagolos}
Что интересно, в отличие от римлянок, этрусски обладали большей свободой и даже
могли вступать в бой с мужчинами.	
\end{zznagolos}

В качестве одежды выступала накидка или рубашка, а в качестве обуви – сандалии.

Особо эта цивилизация преуспела в строительстве и медицине (среди найденных
предметов при раскопках были обнаружены даже зубные протезы). А в 8 веке до
нашей эры начало стремительно развиваться ювелирное дело, что подтверждают
найденные многочисленные украшения из драгоценных камней в виде колец,
браслетов и булавок.

Особый интерес этот народ проявлял к силам природы, которым устраивались
жертвоприношения, а также загробному миру, о чём говорят роскошно украшенные
саркофаги и гробницы.

Известно, что этруски и римляне были союзниками. А период, когда Этрурия
перестала быть независимой и до момента получения этрусками римского
гражданства (в 89 году до нашей эры), назывался этрусско-римским.
