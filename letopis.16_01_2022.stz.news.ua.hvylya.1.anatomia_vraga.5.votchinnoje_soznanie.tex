% vim: keymap=russian-jcukenwin
%%beginhead 
 
%%file 16_01_2022.stz.news.ua.hvylya.1.anatomia_vraga.5.votchinnoje_soznanie
%%parent 16_01_2022.stz.news.ua.hvylya.1.anatomia_vraga
 
%%url 
 
%%author_id 
%%date 
 
%%tags 
%%title 
 
%%endhead 

\subsubsection{Вотчинное сознание, как опора вертикали власти}
\label{sec:16_01_2022.stz.news.ua.hvylya.1.anatomia_vraga.5.votchinnoje_soznanie}

\enquote{Что угодно богу, то угодно государю} (с) - именно так византийская \enquote{подушка}
легитимности определила всю дальнейшую институциональную эффективность
московской властной вертикали. Причем, \textbf{она остается статичной вне зависимости
от исторического периода России}, будь это царство Михаила Романова, империя
Петра Первого или Николая Первого, СССР времен семинариста Сталина, создавшего
религиозную версию \enquote{марксизма} со \enquote{святыми мощами} в Мавзолее, либо же
Российской Федерации современного нам президента Путина. Это жесткая \enquote{монархия
вотчинного сознания}, где все люди, живущие на ее территории, являются как бы
\enquote{пришлыми холопами} на территорию \enquote{царя}, который владеет землей по
\enquote{природной}, сакральной традиции \enquote{власти от Бога} и \textbf{является единственным
паттерном \enquote{защиты от хаоса}}. Поэтому спорить с вертикалью власти - подвергать
сомнению существование государства. Это то, чего о России не понимают на западе
с его \enquote{непонятными свободами}. Во времена Ивана IV (Грозного) даже бояре в
челобитных, писали - \enquote{се яз, холоп твой}, а впервые привилегированное сословие
в России перестало быть \enquote{холопами} только в XVIII веке, по указу \enquote{демократа}
императора Петра III о \enquote{вольности дворянству}.

