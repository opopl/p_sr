% vim: keymap=russian-jcukenwin
%%beginhead 
 
%%file topics.vojna.rashizm.analiz.1.russkaja_mentalnost
%%parent topics.vojna.rashizm.analiz
 
%%url 
 
%%author_id 
%%date 
 
%%tags 
%%title 
 
%%endhead 

Русская ментальность — тщеславие, презрение к другим народам, превосходство, авторитет власти — все они никуда не денутся и после смерти Путина
Украинцы — искренние, добрые и отходчивые. Тем более, что в современном информационном обществе темы слишком быстро сменяют друг друга, влияя на память и чувства.
Поэтому и сохраняется риск того, что не завтра, но через несколько лет или десятилетий, мы в очередной раз услышим о «братьях», «хороших русских», обманутых Путиным и тому подобное.
Нет, вину за эту войну, за смерть и страдания должны нести все они.
Ведь Владимир Путин только сумел оформить то, что подсознательно было в мозгу рядового россиянина после «геополитической катастрофы ХХ-го века» — найти путь к их чувствам, удачно использовать черты русского характера.
Эти черты, эта русская ментальность — тщеславие, презрение к другим народам, превосходство, авторитет власти — все они никуда не денутся и после смерти Путина.
А значит не исчезнет и почва для рашизма, и угрозы для нас в ближайшем будущем и перспективе.
Кроме того, всегда будут те, кто поражение в войне с Украиной будут считать случайностью, стечением обстоятельств, следствием заговора мирового еврейства, или даже следствием мягкости политики самого Путина.
Проблема значительно глубже, чем один человек.
