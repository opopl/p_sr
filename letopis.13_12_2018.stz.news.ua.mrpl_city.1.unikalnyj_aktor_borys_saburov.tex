% vim: keymap=russian-jcukenwin
%%beginhead 
 
%%file 13_12_2018.stz.news.ua.mrpl_city.1.unikalnyj_aktor_borys_saburov
%%parent 13_12_2018
 
%%url https://mrpl.city/blogs/view/unikalnij-aktor-mariupolskogo-teatru-boris-saburov
 
%%author_id demidko_olga.mariupol,news.ua.mrpl_city
%%date 
 
%%tags 
%%title Унікальний актор маріупольського театру Борис Сабуров
 
%%endhead 
 
\subsection{Унікальний актор маріупольського театру Борис Сабуров}
\label{sec:13_12_2018.stz.news.ua.mrpl_city.1.unikalnyj_aktor_borys_saburov}
 
\Purl{https://mrpl.city/blogs/view/unikalnij-aktor-mariupolskogo-teatru-boris-saburov}
\ifcmt
 author_begin
   author_id demidko_olga.mariupol,news.ua.mrpl_city
 author_end
\fi

В маріупольському театрі працювали актори, чия творчість надихала весь
колектив. Серед тих, хто над усе сприяв розвитку театру в Маріуполі, був
заслужений артист Узбецької РСР, народний артист Української РСР Борис Сабуров
– один з визначних і славетних артистів театру радянських часів. Блискуче
комедійне обдарування, невичерпна фантазія, уміння не повторюватися – усе це
приваблювало серця вдячних глядачів і суворих критиків до Б. Сабурова.

Народився Борис Олександрович 21 лютого 1912 року в Єкатеринбурзі. Семирічну
школу закінчив у Тюмені, після закінчення школи змінив безліч занять: працював
на судноверфі, на сірниковій фабриці, навчався живопису, був конюхом. Мабуть, є
натури, яким усе в житті цікаве і все хочеться спробувати самому. Так, через
любов до коней він стає конюхом, завдяки захопленню живописом – вчиться в
художній студії. Вже коли був актором Магнітогорського театру імені О. Пушкіна,
Борис Олександрович просто так, для себе, оволодів майже всіма театральними
спеціальностями – від виробника бутафорії до кравця. Але найбільшою любов'ю,
долею, покликанням стала для нього професія актора. Часто нашого героя називали
\emph{\enquote{людиною щасливої долі}}, але мало хто замислювався, як непросто
було Борису Олександровичу, не маючи спеціальної освіти, опановувати таємниці
неймовірно цікавого, але водночас складного ремесла.

\ii{13_12_2018.stz.news.ua.mrpl_city.1.unikalnyj_aktor_borys_saburov.pic.1}

\textbf{Читайте також:} \emph{Мариупольцы смогут посетить необычную балетную постановку}%
\footnote{Мариупольцы смогут посетить необычную балетную постановку, Олена Онєгіна, mrpl.city, 10.12.2018, \url{https://mrpl.city/news/view/mariupoltsy-smogut-posetit-neobychnuyu-baletnuyu-postanovku-foto} }

У 1928 році за комсомольським закликом поїхав на будівництво Магнітогорського
металургійного комбінату, будував там першу комсомольську домну і коксові печі,
працював газівником. Після робочого дня брав участь в агітбригаді \enquote{Синя блуза},
грав у різних сценках. Незабаром завод направив Сабурова в Театр робітничої
молоді (ТРАМ), там його помітили та запросили до допоміжного складу
Магнітогорського драматичного театру імені Пушкіна.

\ii{13_12_2018.stz.news.ua.mrpl_city.1.unikalnyj_aktor_borys_saburov.pic.2}

У 1931 році відбулися перші виходи Бориса Олександровича на сцену – в масовках,
в епізодах, потім перші ролі в основному складі акторів. Від перших ролей у
масових сценах до епізодів заголовних ролей - ніби по сходинках, дедалі вище й
вище, по крихітці, збираючи знання й досвід, він відточував майстерність.
Загалом в 1939–1964 роках Борис Олександрович Сабуров був актором театрів
Алма-Ати, Ташкента, Одеси, Усть-Каменогорська, Львова, Кишинева, Миколаєва
(1954–1964).

З 1964 року до 1992 року талановитий артист працював у Донецькому обласному
російському театрі (м. Маріуполь). Сабурову було притаманне загострене почуття
часу, вміння в будь-якій ролі знаходити больові точки зіткнення з тогочасними
проблемами, тому під час роботи Б. Сабурова у маріупольському театрі складно
було знайти виставу без його участі.

\ii{13_12_2018.stz.news.ua.mrpl_city.1.unikalnyj_aktor_borys_saburov.pic.3}

Однак найпомітнішими ролями в цьому театрі для актора стали ролі діда Щукаря
(\enquote{Піднята цілина}), Христофора Блохіна (\enquote{Казки старого
Арбату}), Бога (\enquote{Святий і грішний}), блазня (\enquote{Король Лір}),
дяді Миті (\enquote{Любов і голуби}). Роль діда Щукаря з \enquote{Піднятої
цілини} була неначе спеціально придумана для цього актора, який настільки вдало
перевтілився і завоював всю увагу глядачів, що до кінця вистави Щукар став
головним героєм вистави. Якщо дотримуватися рамок амплуа, то за всіма ознаками
Бориса Олександровича слід зарахувати до категорії гострохарактерних комедійних
акторів. І справді, коли актор грав на сцені, важко, просто неможливо було
залишатися байдужим. Запас його знахідок невичерпний, діапазон величезний – від
легкого гумору до гротеску, буфонади. І завжди все акторськи виправдане, як у
хорошому живопису: мазок до мазка, і кожний – єдино точний. Порівняння це не
випадкове. Б. Сабурову самою природою дароване уміння відчувати кожну роль у
кольорі та лінії. Спочатку він створює ескізи гриму персонажа, потім шукає
костюм і пластичний малюнок ролі, і, як правило, художнє чуття ніколи його не
підводило. А коли вже готова певна оболонка, в неї вкладається живе й трепетне
людське серце.

\textbf{Читайте також:} \emph{В Мариуполе можно посетить новогоднюю резиденцию приключений}%
\footnote{В Мариуполе можно посетить новогоднюю резиденцию приключений, Анастасія Папуш, mrpl.city, 09.12.2018, \url{https://mrpl.city/news/view/v-mariupole-projdet-novogodnyaya-rezidentsiya-priklyuchenij}}

\ii{13_12_2018.stz.news.ua.mrpl_city.1.unikalnyj_aktor_borys_saburov.pic.4}

За розповідями самого актора, він йшов до розуміння внутрішнього світу героя
через точно знайдені зовнішні характеристики. І тут йому допомагало вміння
малювати, чим він власне і займався всюди, де тільки було можливо: на
репетиціях, вдома, на прогулянці. Наприклад, працюючи над образом Щукаря в
шолоховській \enquote{Піднятій цілині}, він зробив сотню начерків свого героя,
і будь-яка побутова подробиця, портретна деталь була надзвичайно важлива: як
дід злазить з печі, як їсть, який у нього ніс і які ґудзики на сорочці. І
костюм, і грим Б. Сабуров знаходив сам, і якщо це йому повністю вдавалося, він
відчував себе на сцені затишно, по-домашньому, ніби вже по-справжньому і
назавжди зрісся душею зі своїм героєм.

\ii{13_12_2018.stz.news.ua.mrpl_city.1.unikalnyj_aktor_borys_saburov.pic.5}

У кіно Сабуров вперше знявся на Одеській кіностудії в 1958 році у фільмі про
шахтарів \enquote{Зміна починається о шостій}. Всього за свою кар'єру Борис
Олександрович зіграв приблизно в 40 фільмах, глядачеві він найбільш відомий за
ролями в картинах: \enquote{Цілуються зорі} (зіграв головну роль Єгорич), \enquote{Крижана
внучка} (дід Веремій), \enquote{Щовечора після роботи} (вчитель Іван Никанорович).

Згадуючи роботу з Б. Сабуровим, заслужена артистка України С. Немчук,
зазначила: \emph{\enquote{При ньому не можна було грати слабо, якось спускатися нижче своїх
можливостей, що називається, \enquote{схалтурить}. Він просто цього не допускав,
обов'язково говорив про це}}.

Видатний актор назавжди увійшов в аннали історії театрального мистецтва України
й дотепер залишається найбільш яскравим і унікальним актором в історії
маріупольського театру. Борис Олександрович Сабуров пішов з життя 10 липня 1992
року. Похований на Старокримському кладовищі в Маріуполі. Народна артистка
України Світлана Іванівна Отченашенко, яка довгий час працювала поруч з Борисом
Олександровичем, зазначила: \emph{\enquote{Він народний артист не тільки за званням, але й по
безмірній любові й вдячності всіх, хто його знає}}.

\textbf{Читайте також:} \emph{Маріупольчанки, які надихають...}%
\footnote{Маріупольчанки, які надихають..., Ольга Демідко, mrpl.city, 23.11.2018, \url{https://mrpl.city/blogs/view/mariupolchanki-yaki-nadihayut}\par Internet Archive: \url{https://archive.org/details/23_11_2018.olga_demidko.mrpl_city.mariupolchanky_jaki_nadyhajut} }

\emph{Всі представлені світлини зберігаються в Поточному архіві Донецького
академічного обласного драматичного театру (м. Ма\hyp{}ріуполь) та особистому архіві
О. М. Чернова.} 
