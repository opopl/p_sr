% vim: keymap=russian-jcukenwin
%%beginhead 
 
%%file 27_11_2020.fb.fb_group.story_kiev_ua.1.nakipelo
%%parent 27_11_2020
 
%%url https://www.facebook.com/groups/story.kiev.ua/posts/1525338420996287
 
%%author_id fb_group.story_kiev_ua,ugrjumova_viktoria.kiev.pisatel
%%date 
 
%%tags kommunizm,kultura,nacionalizm,sssr,travlja,ukraina
%%title НАКИПІЛО!
 
%%endhead 
 
\subsection{НАКИПІЛО!}
\label{sec:27_11_2020.fb.fb_group.story_kiev_ua.1.nakipelo}
 
\Purl{https://www.facebook.com/groups/story.kiev.ua/posts/1525338420996287}
\ifcmt
 author_begin
   author_id fb_group.story_kiev_ua,ugrjumova_viktoria.kiev.pisatel
 author_end
\fi

НАКИПІЛО!

Я народилася у країні, де моїх предків знищували лише тому, що вони народилися
у іншій країні, якої не стало у 1917, у якій їх знищували тому, що вони
народилися в інших країнах, які ця, зникла вже у 1917 році імперія захоплювала.

\ii{27_11_2020.fb.fb_group.story_kiev_ua.1.nakipelo.pic.1}

Я народилася у країні, де вчених, які заперечували можливість повертати річки у
зворотному напрямку та попереджали, що нищівне ставлення до землі та річок
призведе до екологічної катастрофи, суворо запитували – ви що - проти народу?
проти партії? проти щастя пролетарів у всьому світі? Я народилася у країні, де
комсомольські та партійні ватажки могли задзьобати будь-кого, мотивуючи вищими
лозунгами. І у мене враження, що вона не зникла, та країна. Я пишу текст про
шоколад та запахи, а меня відповідають, що це, мабуть, прихована реклама КПРС.
Якщо згадати про басейн Динамо - скажуть, що ти пропагуєш радянський спосіб
життя. Не дай тобі боже згадати чисту воду Дніпра, або його зелені схили. Якщо
написати, що Гете - великий письменник, ті самі люди скажуть, що це про фашизм.
Є такі персони, які візьмуть у Тулуз-Лотрека - те що він хворів на сифіліс, а з
Достоєвського пам’ятають лише, що він був антисемітом - більше нічого. Ви
нудні,хочу сказати я. Дуже нудні, марні, ви марнуєте своє життя і намагаєтесь
відняти безцінний час чужого життя на ці безглузді комсомольсько-партійні
суперечки. Ви так і залишилися ватажками піонерських дружин - не тими, веселими
та заводними, а тими, яких всі мовчки зневажали, бо вони вміли лише бігати до
вчителів та шепотіти на вухо - хто що про кого погано сказав або подумав, бо
вони напевне знають, хто що думає. Що я можу вам побажати? Здоров’я та щастя у
вашому нудному, такому злому світі - бо кожен з нас обирає по собі – я з
Тулуз-Лотрека - його картини, з Достоєвського геніальні романи, з Гете - Фауста
та його наукові праці. Але хтось пам’ятає лише кілька рядків з Хорст Вессель, і
це його власне право. Що вже тут поробиш? Одна радість - киян, які заходять на
цю сторінку, аби читати історії про Київ, набагато, набагато більше

\ii{27_11_2020.fb.fb_group.story_kiev_ua.1.nakipelo.cmt}
