% vim: keymap=russian-jcukenwin
%%beginhead 
 
%%file slova.denj
%%parent slova
 
%%url 
 
%%author 
%%author_id 
%%author_url 
 
%%tags 
%%title 
 
%%endhead 
\chapter{День}
\label{sec:slova.denj}

%%%cit
%%%cit_head
%%%cit_pic
%%%cit_text
Новый миф утверждает, что 22 июня вспоминать нечего, поскольку \enquote{для украинцев
война началась в 1939 году}. Только я не знаю другого такого \emph{дня}, который бы
обозначил перелом в судьбе конкретно украинцев. Зато мы все знаем \emph{день}, который
разделил жизнь 170 миллионов людей на \enquote{до} и \enquote{после}. И беду сделал общей, и
войну, и Победу. Понятно, что всегда есть кто-то особенный. Так и здесь есть
те, кто жил в Польше. Или те, кто ликовал и радовался, бежал помогать строить
\enquote{новый порядок}, рассчитывая на персональную миску и будку. Или за свою личную
обиду спешил мстить, но уляпался по шею. У них действительно свой счет времени
и в общий смысл они никак не укладываются.  Да только какое нам до них дело?  У
сотен миллионов живущих внутренние часы безо всяких подсказок отсчитывают те
бесконечные минуты самого длинного \emph{дня}. Это живет в нашей крови. И его
нельзя отменить постановлением очередного бестолкового правительства
%%%cit_comment
%%%cit_title
 \citTitle{У сотен миллионов живущих дата 22 июня 1941 года - в крови}, 
 Дмитрий Заборин, strana.ua, 22.06.2021
%%%endcit

