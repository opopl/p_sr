% vim: keymap=russian-jcukenwin
%%beginhead 
 
%%file letters.sosnovsky
%%parent letters
 
%%url 
 
%%author_id 
%%date 
 
%%tags 
%%title 
 
%%endhead 

%todo

Доброго вечора, дякую за Вашу відповідь! Дуже шкода, що книжки згоріли...
Але... щодо книжок... насправді, достатньо мати на руках лише один екземпляр (у
кого-небудь з маріупольців), щоби відтворити її. Тобто (1) відсканувати,
оцифрувати, виокремити текст та графіку, зібрати заново - це технічна вже
робота.  Щодо книжок Бурова. Наскільки я бачу по каталогу - вони є в основному
фонді НБУВ (Бібліотеки Вернадського), тому цю книжку можна відтворити заново.
Щодо Оксани Стоміної її книжки, я вже про це писав у своєму пості - я згадав ту
книжку також. І якщо екземпляри цієї книжки є у Оксани або у кого небудь, тут
лише потрібна індивідуальна воля зробити оцифровку або домовитись з
видавництвом про передрук (і я впевнений, що попит буде, і досить значний)...
Насправді, я бачу таке. Я сподіваюсь, Ви мені вибачите, якщо що, як людині
сторонній. Мені, коли я пишу про Маріуполь, насправді важко, тому що я
знаходжусь в позиції людини, яка не бачила всіх цих жахів, більш того, емоційно
ніяк з Маріуполем не пов'язана (я вже писав про це вище). І особисто мене війна
сильно не зачепила, як і мою сім'ю. І тому я завжди ризикую, що мої слова
будуть різкими або недоречними з точки зору маріупольців. Але... так чи інакше,
тему книжок треба піднімати, бо від цього залежить подальша доля Маріуполя,
оскільки чисто військового звільнення мало для повноцінного відродження
Маріуполя, яке буде потребувати величезних духовних зусиль, і допомоги всієї
України (крім чисто економічних чинників, як от бюджет на побудову будинків і
т.д. ). І насправді, вже зараз можна було б зробити книжок десять, або і
більше. (1) про блокадний Маріуполь - спогадів, свідчень вже на тисячі сторінок
(2) про мирний Маріуполь - це теж дуже-дуже важливо, тому що... ніхто (в
масштабах всієї України) особливо нічого не знає про Маріуполь, окрім того, що
це місце страшної трагедії і місце героїзму полку Азов. А крім того - в
масовому сприйнятті Маріуполь не існує. Він не існує у масовій свідомості як
туристичне цікаве місто, він не існує в масовій свідомості як високо-культурне
місто з дуже цікавою історією і т.д. Мені дуже прикро це казати, але це так. І
тому... виходить зачароване коло. Люди не знають про Маріуполь - не читають -
Маріуполь випадає із інформ-простору - маріупольці у свою чергу не можуть
вибратись із відчаю і т.д., мають відчуття, що про них забули і т.д. Звичайно,
є дуже багато хороших людей і організацій, які допомагають маріупольцям, в
Києві теж багато чого відбувається, але... на жаль, в масштабах України цього
мало. Ну добре... Щодо мене, як я сказав вище, я програміст. До речі, я Ваші
фото вже бачив - навіть записав пости з ними. Моя ідея власне зараз, збирати
докупи інформацію про Маріуполь. Тоді 
