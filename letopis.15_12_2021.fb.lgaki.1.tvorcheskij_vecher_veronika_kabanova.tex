% vim: keymap=russian-jcukenwin
%%beginhead 
 
%%file 15_12_2021.fb.lgaki.1.tvorcheskij_vecher_veronika_kabanova
%%parent 15_12_2021
 
%%url https://www.facebook.com/AkademiyaMatusovskogo/posts/4894433683954041
 
%%author_id lgaki
%%date 
 
%%tags donbass,kabanova_veronika.pevica.lnr,lgaki,lnr,lugansk,muzyka
%%title Творческий вечер Вероники Кабановой
 
%%endhead 
 
\subsection{Творческий вечер Вероники Кабановой}
\label{sec:15_12_2021.fb.lgaki.1.tvorcheskij_vecher_veronika_kabanova}
 
\Purl{https://www.facebook.com/AkademiyaMatusovskogo/posts/4894433683954041}
\ifcmt
 author_begin
   author_id lgaki
 author_end
\fi

Мы знаем, в чем сила ее.

Творческий вечер Вероники Кабановой собрал в концертном зале Академии
Матусовского меломанов со всей Республики.

Вероника Кабанова – самая молодая заведующая кафедрой в главном творческом вузе
Донбасса. А знающие люди говорят, что и в соседних ДНР, России и других странах
едва ли есть прецеденты. Эту должность прекрасная вокалистка, обладающая редким
меццо-сопрано, лауреат международных конкурсов заняла в 2019-м, когда ей было
всего 28.

\ii{15_12_2021.fb.lgaki.1.tvorcheskij_vecher_veronika_kabanova.pic.1}

— На наших глазах рождается новый тип творческой личности, которая сочетает в
себе несколько аспектов и успешна во многих сферах: Вероника Кабанова и певица,
и педагог, и философ — она прекрасно защитила кандидатскую диссертацию,
углубилась в науку. И это обеспечивает ей высокий профессиональный уровень, —
подчеркивает декан факультета музыкального искусства Академии Матусовского,
кандидат искусствоведения Светлана Черникова. — Ее приглашали многие театры, но
Вероника осталась верна нашей Академии, Луганской Народной Республике и своему
делу. И мы очень гордимся тем, что у нас воспитываются, вырастают в
профессионалов и работают такие люди.

\ii{15_12_2021.fb.lgaki.1.tvorcheskij_vecher_veronika_kabanova.pic.2}

Вероника — выпускница нашего вуза, класс — профессора кафедры вокала,
заслуженной артистки Украины Людмилы Колесниковой. Педагог помнит ее 17-летней
юной девочкой, которая пришла к ней, окончив музыкальную школу. 

— Была очень скромной, послушной, старательной, отличницей, — рассказала
Людмила Анатольевна. — Первый год становления был непростым, голос
меццо-сопрано – очень редкий и полноценно оформляется обычно годам к тридцати,
а Вероника тогда была еще очень молода. Кто-то говорил даже, что она не очень
музыкальна. Но у нас была цель... Мы много занимались, в том числе дополнительно,
даже летом. И поэтому она и открылась, и состоялась.

\ii{15_12_2021.fb.lgaki.1.tvorcheskij_vecher_veronika_kabanova.pic.3}

И сегодня на сцене большого концертного зала музыкального корпуса Академии
Вероника Кабанова в сопровождении Молодежного симфонического оркестра вуза
(художественный руководитель коллектива — Сергей Йовса, дирижер — Александр
Ковальчук) показала не все, но многие грани своих таланта и мастерства. В Сети
есть материалы вроде «10 самых сложных партий меццо-сопрано»... Так вот многие из
них гости творческого вечера «За все, о музыка, тебя благодарю!» услышали
сегодня вечером. Например, арию Принцессы Эболи из «Дона Карлоса» Верди,
«Хабанеру» из «Кармен» Бизе, арию Любаши из «Царской невесты»
Римского-Корсакова... А еще были блестящие фрагменты из «Сильвы» Имре Кальмана –
сольно и в дуэте со студентом уже ее, доцента кафедры вокала, кандидата
философских наук Вероники Кабановой класса, Алексеем Нестеровым. Песня Джудитты
из одноименной оперетты Легара... В финале же прозвучала ария Марицы из оперетты
Кальмана.

— Мы работаем с Вероникой Кабановой, наверное, уже больше десяти лет. В списке
совместно реализованных творческих проектов – многочисленные концертные
программы на сцене филармонии, в которых она участвовала, постановка Оперной
студии Академии Матусовского 2017 года – опера «Кащей Бессмертный»
Римского-Корсакова, в которой Вероника блистала в роли Кащеевны, — говорит
дирижёр Луганского академического симфонического оркестра Луганской
академической филармонии, заслуженный деятель искусств ЛНР Александр Щуров. —
Она активно развивается, все время стремится постигать новые вершины,
замечательный человек, прекрасный музыкант, работать с которой для меня –
огромное удовольствие!

А для всех, кто ценит настоящую музыку, удовольствие – ее слушать! Спасибо за
него!

Фото –  Марина Машевски.
