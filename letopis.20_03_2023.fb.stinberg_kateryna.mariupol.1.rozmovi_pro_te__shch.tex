%%beginhead 
 
%%file 20_03_2023.fb.stinberg_kateryna.mariupol.1.rozmovi_pro_te__shch
%%parent 20_03_2023
 
%%url https://www.facebook.com/frankishekate/posts/pfbid034591gjYvcyWkoiKZqf7QggDyK9cqVicudau1odq3yDvoWKjJKAi1z1JTDr9DoAefl
 
%%author_id stinberg_kateryna.mariupol
%%date 20_03_2023
 
%%tags mariupol,mariupol.war,dnevnik
%%title Розмови про те, що треба їхати, шли чи не з 8 березня
 
%%endhead 

\subsection{Розмови про те, що треба їхати, шли чи не з 8 березня}
\label{sec:20_03_2023.fb.stinberg_kateryna.mariupol.1.rozmovi_pro_te__shch}

\Purl{https://www.facebook.com/frankishekate/posts/pfbid034591gjYvcyWkoiKZqf7QggDyK9cqVicudau1odq3yDvoWKjJKAi1z1JTDr9DoAefl}
\ifcmt
 author_begin
   author_id stinberg_kateryna.mariupol
 author_end
\fi

Розмови про те, що треба їхати, шли чи не з 8 березня. Останньою краплиною
стала інформація від сусідів, що люди, які вчора виїхали, добралися до
Бердянська та відправили СМС. Це вже було щось. Тим більше, що саме в цей день
їхали й інші сусіди.

На 19 березня хоча б по разу палав кожен будинок в полі зору. Деякі - кілька
разів. Наш не палав, але в такому розкладі це було питання часу (як потім
з'ясувалось, годин).

Сказати, що боялися дороги — це нічого не сказати, але було відчуття, що краще
ризикнути та виїхати чи ніж перебувати в постійній небезпеці. 

Страшніше за все виявилось забрати автівку з гаража. По гаражах прилітало. На
щастя, саме в наш не потрапило, але він чи не єдиний із десятка повністю вцілів
і навіть двері не відкрило вибухом. Дуже боялися, що заклинило замок чи що весь
той мотлох, який свекор збирав над автівкою, завадив їх. Ні, не завалив та не
заклинило. Весь цей час в 30 метрах на Будівельниках йшли вуличні бої. 

Автівка завелася ми швидко виїхали, доїхали до під'їзду, покидали речі. Чи не в
останню мить я буквально мізинцем (бо руки були повністю зайняті) захопила
вишиту ікону. Прийшлося залишити автокрісло Кирила та новій м'якенький махровий
халатик. Його особливо шкода. 

Виїхали дворами на Зелинського, потім - на Бахчиванджи, Нахімова та в порт.
Всюди одне й те саме - зруйновані дома, вибите скло, сліди пожеж. 

Найбільше боялися, що уламками поріжемо колеса. Що робити в такому випадку,
навіть не думали. Проте, в автівці вперше за два тижні змогли зігрітися та, що
дуже важливо, зарядити телефон, бо куди ми їхали - просто не знали. Їхали не
кудись, а звідки...

Думали, чи не заїхати до дома на Черемушках, та потім вирішили не ризикувати. Я
сказала, що що б ми там не побачили - цілий будинок, не цілий - це нічого не
змінить, потрібно їхати. 

Місто було повністю порожнє, а ось за містом ми побачили величезну чергу на
блокпості. Над Маріуполем - чорна димна хмара. 

Найстрашніше, що паралельно автівкам йшли люди. З дітьми на руках, з кофрами на
колесах, з рюкзаками, з переносками. Здавалось, вони йдуть нескінченим потоком,
таким саме нескінченим потоком їхали автівки. І мені було невимовно шкода, що у
нас не бус, не автобус, не потяг, який би міг взяти бодай частину цих людей... 

Щойно я включила телефон, він задзвонив в мене в руках, а я навіть не бачила,
хто дзвонить через дуже яскраве весняне сонце. По голосу впізнала
\href{https://www.facebook.com/anton.ianchenko}{Anton Yanchenko}. Він дуже
здивованим голосом спитав щось типу \enquote{Катю, це ти?}, а я почала ржати ти
питала, а кого ще він очікував почути, дзвонячи на мій номер.  Який майже три
тижні був поза зоною... 

Так ми виїхали з Маріуполя рік тому.

%\ii{20_03_2023.fb.stinberg_kateryna.mariupol.1.rozmovi_pro_te__shch.cmt}
