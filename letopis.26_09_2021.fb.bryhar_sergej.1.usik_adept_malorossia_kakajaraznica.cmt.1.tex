% vim: keymap=russian-jcukenwin
%%beginhead 
 
%%file 26_09_2021.fb.bryhar_sergej.1.usik_adept_malorossia_kakajaraznica.cmt.1
%%parent 26_09_2021.fb.bryhar_sergej.1.usik_adept_malorossia_kakajaraznica.cmt
 
%%url 
 
%%author_id 
%%date 
 
%%tags 
%%title 
 
%%endhead 
\paragraph{Екатерина Вигилёва - Скріншоти Анатолій Штефан, Андрій Смолій}
\label{sec:26_09_2021.fb.bryhar_sergej.1.usik_adept_malorossia_kakajaraznica.cmt.1}

\begin{itemize} % {
\iusr{Екатерина Вигилёва}

\ifcmt
  ig https://scontent-yyz1-1.xx.fbcdn.net/v/t39.30808-6/242777354_2606190666193937_1128731019661716705_n.jpg?_nc_cat=102&_nc_rgb565=1&ccb=1-5&_nc_sid=dbeb18&_nc_ohc=ZAjuaYuGPo4AX9yb65L&_nc_ht=scontent-yyz1-1.xx&oh=8262b9f15c9245a0adf2a6059a9ccc91&oe=615A8380
  @width 0.4

	ig https://scontent-yyz1-1.xx.fbcdn.net/v/t39.30808-6/242615944_2606190712860599_900541053348246341_n.jpg?_nc_cat=110&_nc_rgb565=1&ccb=1-5&_nc_sid=dbeb18&_nc_ohc=JprBH2VZ4TwAX9AfcDu&_nc_ht=scontent-yyz1-1.xx&oh=1735215566f8418af4a3fe7d564d58b1&oe=615B5932
  @width 0.3
\fi

\begin{itemize} % {
\iusr{Serhii Bryhar}
\textbf{Катерина Вігільова} А я вболівав. Проти прихвостня "русскогоміра" (правда, не активно, і взагалі був у нічному поїзді.. тому дізнався про все уже зранку)

\iusr{Марія Боголюбова}
\textbf{Катерина Вігільова} фсбусік
\end{itemize} % }

\iusr{Артем Коломієць}
Погоджуюся з Вами

\iusr{Тетяна Попович}
Так, згідна з Вами)

\iusr{Владимир Дубровский}

Усік повернув пояса чемпіонів, котрі Джошуа забрав у Володимира Клічко!!!

И во вторых у него в Крыму родственники живут пока еще.

\begin{itemize} % {
\iusr{Олег Кобизський}
\textbf{Vladimir Doubrovski} що ти таке верзеш? Ти людина думаюча взагалі? Чи відрізнити квадратне від круглого не в змозі

\iusr{Владимир Дубровский}
Розумие, спидчуваю ... слабо в Крим на 900 и одну добу?

\iusr{Вуйко Місь}
\textbf{Vladimir Doubrovski} І що? У Данила Гайдамахи теж батьки живуть в окупованому "чемними андрофагами" , Криму. І він - українець ,а не таке лайно , як русік.

\iusr{Владимир Дубровский}
\textbf{Вуйко Місь} Якщо русик поднимет жовто-блакитний прапор на батковщине в Криму то сядет от 3х до 5-ти

\iusr{Dmytro Mashkovskyi}

Усік вже давно міг вивезти батьків з Криму, але цього не зробив. Та й
цілуватись з Ондатрієм його ніхто не примушує, все полюбовно. Так що аргументи
Владіміра розбиваються об факти.

\end{itemize} % }

\iusr{Инна Комбарова}
Згодні з Вами ! ЄДНАЄМОСЬ!

\iusr{Андрій Винниченко}

Перепалка - це якийсь русизим, який мабуть першим Олесь Гончар використав.
Українською немає слова перепалка. Українською перепалка перекладається як
престрілювання, стрілянина, ну в крайньому разі нехай перестрілка. А в
переносному значенні перепалка перекладається, як сварка, лайка, суперечка.


\iusr{Олег Кобизський}
От таких малорусиків ой як багато

\ifcmt
  ig https://scontent-yyz1-1.xx.fbcdn.net/v/t1.6435-9/243068052_3080921222234280_2587242779968199566_n.jpg?_nc_cat=106&ccb=1-5&_nc_sid=dbeb18&_nc_ohc=gy1VMkEDQQ8AX-Uo8Tn&_nc_ht=scontent-yyz1-1.xx&oh=7f085a7eeaf505067d462cd9881ed30b&oe=617C6DDE
  @width 0.3
\fi

\iusr{Roman Romanow}

а є й така пазіція (камнєкідатілям зразу кажу - чілавєк воював за нас.
осознанно. і зара не дає спуску). отакоєот

\ifcmt
  ig https://scontent-yyz1-1.xx.fbcdn.net/v/t39.30808-6/243023615_4374142636040249_2320983751027905326_n.jpg?_nc_cat=102&_nc_rgb565=1&ccb=1-5&_nc_sid=dbeb18&_nc_ohc=gfHJs2k342AAX_10fQ2&_nc_ht=scontent-yyz1-1.xx&oh=0666a9dda34d57398494df418d38297c&oe=615AB7A8
  @width 0.8
\fi

\begin{itemize} % {
\iusr{Volodymyr Tarnavskyi}
Хароша пазіція... Какі камні)

\iusr{Roman Romanow}
\textbf{Volodymyr Tarnavskyi} just like it is

\iusr{Volodymyr Tarnavskyi}
Да) рагулі вони такі: вчора путіна клянуть, сьоня за санька болєют) і все по колу повторюється) історично)

\iusr{Roman Romanow}
по-перше в слові хуйло 6 помилок. а по-друге у городі бузина....

\iusr{Volodymyr Tarnavskyi}
Ну конєшно) головне путіна правильно написати, а там хоч трава не рости...ой, і за усіка можна поболєть)
\end{itemize} % }

\iusr{Євгенія Чуприна}

Я радію перемозі Усіка саме через те, що бачу в тому стратегічну перспективу.
Так, він дурник і його заплутали хитрі попи, але сам факт, що кримчанин є
українським спортсменом, є промовистим, і навіть його дурнуваті погляди
працюють на концепцію - що б не казав, що б не думав, а все одне обрав Україну
так чи інакше

\begin{itemize} % {
\iusr{Roman Romanow}
\textbf{Євгенія Чуприна} абсолютно логічна позиція. піддержую десь на 73\%)))

\iusr{Serhii Bryhar}
\textbf{Євгенія Чуприна} Так просувати концепцію малоросійсьтва в Росії було б відверто дико. Тут він, як би не було прикро це визнавати, перебуває на своєму місці. Тут поле боротьби, і він вельми серйозна бойова одиниця... Але, на жаль, як говорив ще один "монстр жанра", він "с той сторони".

\iusr{Orysya Klymovych}
\textbf{Євгенія Чуприна}
Промовисті - на передовій.

\iusr{Євгенія Чуприна}
\textbf{Serhii Bryhar} звідки така впевненість, що він з тієї сторони? Спочатку ходив з оселедцем і витатуював тризуб. Потім раптом почав говорити про адіннарод. Потім знов забігав і заговорив українською. Про які погляди взагалі йдеться і хто всерйоз буде на них орієнтуватися. Але чувак виступає від України, і це точно
\end{itemize} % }

\iusr{Сергій Гук}

Я і не уболівав ні за, ні проти, і не тому, що Усік малорос чи українець.

По-перше, не варто нас ділити саме на дві категорії як колись
Хвильовий-Фітільов чи московська пропаганда, це нам шкодить. Треба б розуміти,
що категорій набагато більше, а не дві.

Є певні люди, котрі й як українці сваряться з українцями не за малоросійство, а
за багато різних речей, взяти ту ж партійщину.

По-друге, вважав би диким бажати Усіку поразки. Переміг, і добре, бо багатьом у
нинішньому депресняку потрібні якісь позитивні емоції з галузі видовищ,
спортивних у тому числі.

Чув думку, що цей поєдинок був домовлений, аби пізніше Джошуа зміг узяти
реванш. Якщо це так, то по-третє, давайте не перетворювати самих себе на гіршу
частину шоу нібито бою.

\iusr{Женя Малахова}

Дякую Вам за непохитну позицію, силу духу і сміливість!! притомні українці з ВАМИ!

\iusr{Олександра Решотко}

Вітаю, Пане Сергію. Погоджуюсь з Вашими висновками. Найсумніше, що ідентичними
епітетами- бандерівка, махрова націоналістка і т.д. мене нагородили тиждень
тому, але це були не фанати рУсіка, а захисники лгбт, коли я написала, що
Якби на позір не здавалося, що між адептами девіантних проявів і рашистами -
прірва, і одні і другі близнюки.

\iusr{Галина Іванова}

\href{https://youtu.be/RB-zYEcJUm8}{%
"Патріот" Усик насправді не переміг Джошуа, а Медведчук – український націоналіст, %
STERNENKO, youtube, 26.09.2021%
}

\iusr{Mykola Revuk}

Питання в тому що ми не хочемо визнавати що є українці з іншою точкою зору...
Радіти перемозі Усика.. треба помірковано як перемозі України в спорті... як
перемозі Шахтаря.. не прив‘язуючись до особистостей і не роблячи конфліктів там
де вони не доречні.

\begin{itemize} % {
\iusr{Serhii Bryhar}
\textbf{Mykola Revuk} Я сподіваюся, ви не думаєте, що я накинувся на когось із кулаками. Просто спробував продемонструвати інший бік медалі. А закінчилося все тими фразами, які я, власне, процитував. Дискусії не вийшло)..
\end{itemize} % }

\iusr{Алла Шубрікова}
Ой, та чи надовго він українець...

\iusr{Сергей Джоболда}
+++

\iusr{Олексій Пономаренко}

Навіть Яніна Соколова вболювала за Алєксандра бо він же Україну прославляє!!!

А назвати Яніну не думаючою язик не повертається.

То ж і я собі думаю чи я такий дурний чи ринг лікує ватників від московських скрєп (

\begin{itemize} % {
\iusr{Serhii Bryhar}
\textbf{Олексій Пономаренко} 

Нікого ринг не лікує. А прославляти державу потрібно не кулаками, а реформами і
перетворювання та та відходом від галімного совка. Кулаки - це добре, але як
доповнення, а якщо країна може радіти лише через те, що боксер забрав пояси -
вона у великій халепі.

От виходить якесь дике, хоч і вправне як спортсмен, каже "я не Олександр, а
Алєксандр"... а потім нормальні, здавалося б, українці, це спокійно хавають і
плескають в долоні та кажуть: "це наш хлопець, Алєксандр)))". Та ніфіга воно не
наше. Він свій бік уже обрав. Підтримка таких як він - це підтримка Малоросії.
Можна помилитися, але не варто перетворювати подібне в системність...

Щодо Соколової... Вона говорила багато різного. І дуже вірного, як на мене, -
теж. Знаю, до речі, що вони з Усіком живуть в одному жк. Може про щось
розмовляли, може вона зробила якісь висновки. Може їй щось сподобалося... Але я
думаю, що такі, як Усік не змінюються! 

Зараз от просто стратегічна пауза, бо так вигідніше для іміджу... Скоро знову
повалять скрєпи і русскіймір...


\iusr{Олексій Пономаренко}
\textbf{Serhii Bryhar} я підтримую вас повністю.

Тож Алєксандр це ватний манкурт який працює в інформаційному просторі на
користь росії та ця перемога в ринзі Усика це ще одна поразка України в
гібридній війні.

З цим мені все зрозуміло.

Я не розумію чому частина проукраїнського суспільства це не в змозі усвідомити.

І в приклад я наводжу Яніну Соколову.... вона ж проукраїнська, свідома
патріотка і ось тобі такі зашквари з Усиком....

Я не розумію її позицію з цього приводу

\iusr{Olena Dmytrenko}
Ой тільки не треба популяризувати Путінське словко скрєпи. Ніяк без маскфи?

\iusr{Yurchenko Sergiy}
\textbf{Олексій Пономаренко} навіть хто?
Коли вона стала "супер-патріоткою" чи є такою за межами "ефірів" ?

\iusr{Наталя Даценко}
\textbf{Олексій Пономаренко} Яніна порошенківське, а порох скритий запутінець. Коло замкнулось.

\iusr{Natalie Gunko}
\textbf{Олексій Пономаренко} 

Ваша Яніна набрала в рота гівна, коли на закритті Одеського кінофестивалю
британський режисер зі сцени сказав «Спасіба Расія! Спасіба Адєса». Патріотка
липова.

\iusr{Олексій Пономаренко}
\textbf{Natalie Gunko} 

по-перше вона настільки ж моя наскільки і ваша.

По-друге з приводу одеського фестивалю на мою думку та виходячи з того
матеріалу який я бачив то Соколова зреагувала адекватніше ніж більшість
присутніх там одеситів(хоча відео обрізане і я так і не зрозумів чим все
закінчилося)

По-третє для мене Яніна зашкварилась ще на ситуації з футболом та Ніцой.

Отож мені цікаво як так сталося що журналістка з проукраїнськими поглядами (а у
неї безліч цікавих проукраїнського контенту: інтерв'ю та сюжетів) починає
поступово йти в напрямку какая разніца на каком язикє та мишебратья-адінарод?

\end{itemize} % }

\iusr{Олександр Масляник}
краще б програв!

\iusr{Андрій Смолій}

Друже, якщо навіть деякі проукраїнські публічні люди, журналісти, активісти
сьогодні написали який «усік» молодець, то я не знаю що далі казати.

Сумно, що ті хто мав би захищати країни сьогодні раптом верещать «спорт внє
палітікі».

Я тоді вже мовчу за «насєлєніє», яку взагалі живе в реальності байдужості до
своєї країни.

\begin{itemize} % {
\iusr{Viacheslav Harchenko}
\textbf{Андрій Смолій} значить ці "деякі проукраїнські публічні люди, журналісти, активісти" не такі вже й проукраїнські, як видають із себе.

\iusr{Павло Редзель}
\textbf{Андрій Смолій} треба підтримати Сергія - важко бути одному проти зграї кретинів, навіть якщо за нього тільки одна Людина...перевірено ,,на власній шкірі".
\end{itemize} % }

\iusr{Андрій Смолій}

І все ж я думаю що цей рУсік не дурень, а гарно проплачений та завербований агент фсб

\begin{itemize} % {
\iusr{Serhii Bryhar}
\textbf{Андрій Смолій} А якщо так, то приводів підтримувати його ще менше. Я б сказав: їх менше нуля..

\iusr{Руслан Шеремета}
\textbf{Андрій Смолій} 

В кожної людини є його власні переконання - які йому хтось вклав до мізків. А
інших прикладів не навів... От і все. Не всі можуть бути настільки патріотами -
як Сліпак.


\iusr{Андрій Смолій}

\textbf{Serhii Bryhar} 

у чувака, який є мільйонером вся сім‘я живе в окупованому Криму. Це вже про
щось свідчить. Всі проукраїнські люди які мали можливість - вже виїхали. Інших
попакували по тюрмах. А його дружина взяла паспорт РФ. І це мова про мільйонера
що може переїхати будь куди

\iusr{Serhii Bryhar}
\textbf{Андрій Смолій} 

А про дружину - все просто: "она етнічєская россіянка, і імєла подноє моральною
право сдєлать такой вибор". От приблизно таке мені розповідали опоненти у
відповідь не подібні до твоїх тези. Так а стоп, - говорю я, - то що, нормально,
що там уся рідня, не аж тут проблеми? А чому їх не пакують? Значить - усє за
РФ? Логічно?... Це багатьох дратує... На питання "а навіщо ж тоді вболівати за
це консервоване московське лайно?", від якого смердить русскімміром, змістовної
відповіді немає. Є лише: хватіт радікалізма))).


\iusr{Михайло Василик}
\textbf{Руслан Шеремета}

Такими, як СЛІПАК і не можуть бути усі, але, принаймі не падати так низько за
30 років НЕЗАЛЕЖНОСТІ і на сьомому році війни з московією...


\iusr{Руслан Шеремета}
\textbf{Михайло Василик} А хто і як має виховувати патріотизм?

\iusr{Руслан Янісевич}
\textbf{Андрій Смолій} 

ні, він просто зазомбований чувак, який повірив що його власні здобутки це не
результат його ж кропіткої праці, а молитви кгбешних попів.


\iusr{Андрій Смолій}

\textbf{Руслан Янісевич} не думаю.

\iusr{Руслан Янісевич}
\textbf{Андрій Смолій} ну таке, кожен має право мати свою думку. Головне не сперечатись за такі дурниці.

\iusr{Михайло Василик}
\textbf{Руслан Шеремета}
Це дуже непросте питання. Навряд, чи хтось може дати вичерпну і однозначну відповідь.

\iusr{Руслан Шеремета}
\textbf{Михайло Василик} 

Виховувати патріота має власний приклад інших патріотів, чиновники і вибрана
влада, родина і школа. Але - маємо під 70\% просто телепнів. На всіх посадах і
становищах, в тому числі і шкільні вчителі.

\end{itemize} % }

\iusr{Olesia Vessna}

Важко утримувати полярні позиціі в цілому. Про цей бій та ситуацію, яка
викликала такий резонанс почитайте у Игорь Ларин. Відсторонитись від накалу
емоцій і спробувати зрозуміти... це важко, але не неможливо.

\iusr{Владислав Гриневич}

Щоб Ви не вважали, але, принаймні формально, він Українець. Яка перепалка тут
може бути? Хіба Вам треба саме вона.

\begin{itemize} % {
\iusr{Serhii Bryhar}
\textbf{Владислав Гриневич} 

Шарій теж має український паспорт! І що? Вважатимемо його українцем? Це такий
цікавий шлях: "він прославляє Україну"... Мені не все одно, хто це. Я пробую
показати інший бік, а у відповідь отримую агресію. Ну що ж, витримаю і піду
далі. А головне в тому, що поняття малорос і українець усе більше й більше
змішуються. Ну то какаяразніца, чьоужтам...


\iusr{Павло Пастушенко}
\textbf{Владислав Гриневич} хіба вам гнилі яблука і добрі однакові на смак, добродію. А то не Українець, а Малорос

\iusr{Владислав Гриневич}
\textbf{Serhii Bryhar} Шарій - інша тема, давайте ще Яника згадаємо

\iusr{Владислав Гриневич}
\textbf{Павло Пастушенко} є хтось краще? Браття Клички?
Я погоджуюся у цілому, але...

\iusr{Ганна Счасна-Гарус}
\textbf{Владислав Гриневич} він- мскаль
..

\iusr{Людмила Демянчук}
\textbf{Ганна Счасна-Гарус} Малорос без роду і племені , завербований, ймовірно, і проплачений Москвою.

\iusr{Владислав Гриневич}
\textbf{Ганна Счасна-Гарус} може бути(
\end{itemize} % }

\iusr{Orysya Klymovych}
\textbf{Владислав Гриневич}
Ага - паспорт із тризубом не проміняв  @igg{fbicon.thinking.face} 

\begin{itemize} % {
\iusr{Oleksandr Chumak}
\textbf{Orysya Klymovych} його дружина проміняла український паспорт на російський, діти - громадяни РФ.

\iusr{Orysya Klymovych}
\textbf{Oleksandr Chumak}
бавиться в патріота  @igg{fbicon.face.confused} 

\iusr{Людмила Демянчук}
\textbf{Orysya Klymovych} Піар, банальний піар !
\end{itemize} % }

\iusr{Anatoliy Lustyk}
Це не фейсбук, у поїзді через вусика могли б постраждати.

\iusr{Volodymyr Tarnavskyi}

От я теж, Сергію, помітив цей момент - він прославляє Україну. Це от таке наше
совкове африканське мислення - в країні хуйово і повний безлад, але нам зараз
дядя в трусах медальку завоює на алімпіаді чи на ринзі, і вже не так на душі
хріново буде. Ну ібажання постійно перед нормальними європами та америками
похизувтися. Не розуміють бідні, що поважають англо-сакси всякі не за
стрибучість, співочість, чи боксючість, а за розвинене суспільство, процвітаючу
економіку, наукові і творчі досягнення. А все інше - фентезійні досягнення
самих аборигенів.

\iusr{Євген Бренчик}

Як Галичанин скажу , що у нас таких дійсно дуже багато ! Здається в
україномовні , і не мпцешники , і ла-ла-ла-ла , але в головах така каша , що
навіть з родичами стараєшся стримуватися ...Та то і не дивно , бо мої сусіди
навколо в Сочі на заробітках , дітей забрали , син хотів в військовий виш , так
уже не хоче , бо рускі платять добре ... А інші в Москві мають мережу
автомагазинів ,працюють на них там наші , покутяни , купа знайомих бойків у
Сибірах на нафтових і газових промислах та просто валять ліс . Для них Росія -
годувальниця і з їхніх слів , до них там дуже добре ставлення та оплата і
ніякого тиску через бандерівське походження . Цим їх Росія і купує , заманює та
асимілює , а приїхавши додому , вони своїми вихваляннями , зманюють інших їхати
туди і любити і поацювати на братушек . От так от весело .

\begin{itemize} % {
\iusr{Людмила Демянчук}
\textbf{Євген Бренчик}
Це наш менталітет такий, аби заробити і немає значення де і у кого.
Не всі такі, але продажних заробітчан величезний \%.
\end{itemize} % }

\iusr{Paweł Majstrenko}

Згадалося: "Може він і сучий син, але це наш сучий син" (Франклін Рузвельт)

\begin{itemize} % {
\iusr{Alla Krasowska}
\textbf{Paweł Majstrenko}
Так само свого часу казали донецькі щодо Януковича: "Знаємо, що він бандит, але то наш бандит".

\iusr{Paweł Majstrenko}
\textbf{Alla Krasowska} знаю. це, звісно, не виправдання
\end{itemize} % }

\iusr{Paweł Majstrenko}

Я теж не особливо розділяю цей ажіотаж навколо перемоги, але, справедливості
заради, на далекій відстані, в міжнародній історії України, в історії світового
боксу, мало хто згадає його політичну позицію, але всі будуть пам'ятати чергову
звитягу українського спортсмена.

\begin{itemize} % {
\iusr{Людмила Демянчук}
\textbf{Paweł Majstrenko} А чи буде тоді Україна на карті світу ?

\iusr{Paweł Majstrenko}
\textbf{Людмила Демянчук} якщо це жарт - то не смішно, а якщо серйозно, то я не розділяю таку риторіку. Так мислить вата. Існування України не можна всерйоз піддавати сумніву - це прояв меншовартості.
\end{itemize} % }

% -------------------------------------
\ii{fbauth.vasilik_mihail.ivano_frankovsk.ukraina}
% -------------------------------------

Мене дивує, що сьогодні частина українців і досі не вийшли з того
постокупаційного синдрому меншовартості. Кому ще сьогодні не зрозуміло, що усік
не є українцем. Волею історичних подій і певним збігом обставин він став
громадянином України. Так, він використовує сьогодні український прапор, але
чисто із споживацьких меркантильних міркувань. 

Була б інша ситуація, він без будь якого примусу підняв би московського БЄСІКА
над собою. Неможливо абстрагувати боксера усіка від його московитсткої
сутності. Він, навпаки, є відвертим носієм руцкомірних цінностей і не є
гордістю українців, а навпаки, реальним ворогом України.

P.S. звичайно, мені простіше це сприймати, бо ще з молоком матері я засвоїв, що
кацап і комуняка це одвічні вороги. Я змалку радів коли совітських спортсменів
перемагали представники будь-якої країни. А московська імперія вкладала у спорт
величезні капіталовкладення, бо це була ВЕЛИКА ПОЛІТИКА і совєтський спорт мав
бути на висоті. Але навіть це далеко не завжди допомагало, особливо, в
технологічно складніших видах спорту.

\begin{itemize} % {
\iusr{Світлана Гончарова}
\textbf{Михайло Василик} Недавно вживала поняття споживацького та меркантильного ставлення школи й батьків до дітей.

\iusr{Сергій Манелюк}
\textbf{Михайло Василик} Повністю Вас підтримую і поділяю Ваші цінності. Теж з дитинства радів, коли совіт програвав. І зараз вболіваю за всіх, хто змагається з московією.
\end{itemize} % }

% -------------------------------------
\ii{fbauth.goncharova_svetlana.ukraina.katareni67}
% -------------------------------------

Не дивуйтеся. Панування Московії потім совка дало свої результати. Маємо
населення какаяразніца України в кількості десь 80 відсотків і тільки 20
відсотків українців. Автор допису і жіночка 50 років і виявилися українцями. А
решта населення. Гниле морально і фізично, безпринципне, пристосоване, підле. І
горе українцям жити між цим болотом. І важко будувати Україну для українців.
Кругом населення какаяразніца.

\iusr{Леся Струтинська}

\ifcmt
  ig https://scontent-yyz1-1.xx.fbcdn.net/v/t1.6435-9/243263026_573535220756630_8304201537014325045_n.jpg?_nc_cat=100&_nc_rgb565=1&ccb=1-5&_nc_sid=dbeb18&_nc_ohc=sbMY73M0tuUAX9HqMpN&_nc_ht=scontent-yyz1-1.xx&oh=7075c78e1f22dbcbd98d4f96ff93ef22&oe=617CB4D6
  @width 0.5
\fi

\iusr{Леся Струтинська}

Треба було там перевести тему на вакцинацію. Ті ЗЕбожілі антивакцинатори би
швидко за усрУсіка забули...

\ifcmt
  ig https://scontent-yyz1-1.xx.fbcdn.net/v/t1.6435-9/243105441_573535644089921_1784562368937326169_n.jpg?_nc_cat=102&_nc_rgb565=1&ccb=1-5&_nc_sid=dbeb18&_nc_ohc=Pa6GKdSKnI8AX8puCJ9&_nc_ht=scontent-yyz1-1.xx&oh=8457c36c5754b55d1000a1cafd8e12d1&oe=617CD021
  @width 0.3
\fi

\iusr{Любов Янішевська}

\ifcmt
  ig https://scontent-yyz1-1.xx.fbcdn.net/v/t39.1997-6/p480x480/91521538_1030933857302751_5093925307199520768_n.png?_nc_cat=1&ccb=1-5&_nc_sid=0572db&_nc_ohc=7ESG5mFS1dgAX81O16f&_nc_ht=scontent-yyz1-1.xx&oh=dc7851b8406f3d0f4068971573859cbb&oe=615AD5F7
  @width 0.2
\fi

\iusr{Тетяна Михальська}
Підтримую вас

\iusr{Назарій Біленький}

Україна і її мова, символи і суть повинні бути принципово вищими ніж будь які
люди і будь які дії цих людей. Люди помирають і програють, а Україна була є і
буде для всіх мільйонів українців!


\iusr{Надія Умриш}
Так...

\iusr{Наразі На Часі}
глорімани хєрові

\iusr{Sergiu Devdyk}

Цікава тема, якою не варто знехтувати, це те, що потрібно більше на спортсменів
тиснути, щоб вони на камери, та й загалом давали інтерв'ю державною мовою(не
подобається це ''визначення''). Бо це хоч , не хоч а ''козерь'' в борні за
ідентичність.

Треба ж визнати, що те що започаткувала Пані Лариса має жити і діяти!

\begin{itemize} % {
\iusr{Юрий Филипюк}
\textbf{Sergiu Devdyk} 
Потрібно, щоб публічні люди казали на камеру, і не один раз, що путін ху@ло,
Крим анексований московією, а на Донбасі війна з тією ж московією (можуть усе
це говорити навіть московітською).

\iusr{Sergiu Devdyk}
\textbf{Jura Filipjuk} 

Не бачу мудрості і прикладу для дітей в Ваших словах Пане Юрію!

А про Анексію і Війну не може бути примусу, бо тоді не буде відомо хто є хто.
Але нашою вони забов'язані говорити.

\iusr{Юрий Филипюк}
\textbf{Sergiu Devdyk} 

Від того, що усік заговорив українською, українцем він не став. І не стане. Для
молоді, і не лишень, він кумир. І з його життєвою позицією "мі же братья"
потрібно боротись різними методами. У тому числі ігнором.

\iusr{Sergiu Devdyk}
\textbf{Jura Filipjuk} 

дивна позиція в більшості українців, хочуть зробити з ''гівна коника''. Я не
кажу, що він має стати українцем, чи ним є, а те що молодь дивиться на нього, і
та молодь розуміє , що українці натиснули на нього(на спортсменів), тобто
вийшли з позиції влади і сили. То ж вони оберуть, підуть за силою!

А , вам не надоїло вже доводити, що ми з моск@лями не брати? Для мене це
зрозуміло, і не вважаю за доцільне з гонору комусь, це доводити!


\iusr{Юрий Филипюк}
\textbf{Sergiu Devdyk} Мені про те, що ми з москалями не брати, говорити нічого. Усіку про це розкажіть, прибравши власний гонор.

\iusr{Sergiu Devdyk}
\textbf{Jura Filipjuk} а нашо мені той Усік. Шо він є, шо його нема . То його проблеми.

\end{itemize} % }

\iusr{Ruslan Pankevych}

Можливо, варто почати з корупції? Особливо у ВР,на Банковій, обласних та
місцевих рад?

Там багато хто ходить у вишиванках і співає українські пісні і при цьому
обкрадає Україну крадучи з бьюджету , пенсій, будівництва і т д Чи вони не
приносять Україні більши шкоди?

Чи це « наші хлопці» і якщо у вишиванках то їм можна?

Не там ми ворогів шукаємо, мені здається

\begin{itemize} % {
\iusr{Марта Козак}
\textbf{Ruslan Pankevych} на городі бузина , а в Києві дятько. До чого ваша писанина? Вже вам вишиванки завинили . От народ.

\iusr{Volodymyr Tarnavskyi}

А може варто почати з найпростішого - яку державу ми будуємо, з якими
символами, сенсами і баченнями тощо. А давайте боротись з корупціонерами у
вишиванкам - це 100\% російський наратив. Я от бачу найбільшу шкоду для
економіки держави в олігархах, які обкрадають державу на мільярди доларів. І
вони всі російськомовні. Це конкретні люди, і конкретна проблема. Але про них
роспропаганда не розкаже. Простіше боротись з корупціонерами саме у вишиванках
і україномовних, хоча більша частина державних грошей розкрадається
російськомовними депутатами, олігархами, бізнесменами і наближеними до них озу.

\iusr{Ruslan Pankevych}
\textbf{Марта Козак} 

вишиванки точно не завинили. Але не завжди, ті що у вишиванках є справжні
патріоти.

Усик не краде гроші з бюджету, не продає свій голос на голосуванні у ВР і не
заберае пенсію, зарплату, чи допомогу через схеми .І не купляє собі золоті
туалети за вкрадені кошти.

Але про них не пишуть і їх не судять

\iusr{Ruslan Pankevych}
\textbf{Volodymyr Tarnavskyi} 

ви здуріли?))
Яка різниця в чому вдіті ті, що обкрадають і дурять??
Чи у вишиванках чи у валянках чи у костюмах від котюр?
Я пишу лиш те, що треба оцінювати за вчинками а не за одягом чи мовою чи вірою.
А злодії одінуть все шо завгодно і навчуться хоч китайскої при потребі…

\iusr{Volodymyr Tarnavskyi}

Ну так а логіка писати ідіотський коментар про злодіїв у вишиванках, а потім
стверджувати, що злодій є злодій, і неважливо якою він мовою балакає?
Взагалі-то Усика чмирять не за мову, а за конкретні антидержавні слова і вчинки
- від підтримки московської церкви, до невизнання окупації Криму і маячнею про
триєдиний народ. Але тут з'являєтесь ви і розповідає те всім, що варто ловити
україномовних корупціонерів. Як кажуть в нас ні в кут, ні в двері.

\iusr{Ruslan Pankevych}
\textbf{Volodymyr Tarnavskyi} 

ну так добре відволікати народ від злодіїв та корупціонерів находячи тих, хто
має інший погляд.

Дійсно, до чого тут злодії та корупціонери?)

Все нормально,-хай і дальше обкрадають і дурять:) Я ми будемо про Усика, який
під український прапор і під українську мову здобув таку нагороду.

Цікаво, а що Ви зробили для України чи українців ?

\iusr{Volodymyr Tarnavskyi}

В мене таке враження, що ви ведете внутрішній діалог із собою чи тренує те
промову перед виборцями.

Ви пишете про мовних корупціонерів у вишиванках. Вам відповідають, що писати
про мовний аспект корупції - дурість і невігластво, бо кейс Усика зовсім не про
мову, а про інші речі. В питанні російськомовних військових - аргумент хороший.

Тут - не приший кобилі хвіст. Тоді ви змінюєте тактику і починаєте розповідати,
що потрібно боротися з корупцією взагалі, незважаючи на мову. А хто каже, що
непотрібно?:) І як може кейс Усика відволікати від справ корупції? Ви що
неспроможні окрім боксу моніторити інші теми? Маразм.

Щодо "що ви зробили для українців?" То це аргумент підлітка. Зробив те саме, що
й ви - займаюся власною справою, сплачую податки (у вашому випадку не
впевнений). І будь ласка, вимикайте оцю дешеву популістську риторику пронизану
російськими наративами - ви не на мітинзі бюджетників.

\end{itemize} % }

\iusr{Maria Ratti}
Перекінчик цей Усик, свідок московицького паханату, якому вірус скаженого совка з'їв мізки!
Нема чому радіти...

\iusr{Martinika Shelest}
Запроданець Усик.
Сепарська консерва.

\iusr{Лиза Ребар}

\ifcmt
  ig https://scontent-yyz1-1.xx.fbcdn.net/v/t1.6435-9/243046360_6141880799187480_6659801162331630771_n.jpg?_nc_cat=104&_nc_rgb565=1&ccb=1-5&_nc_sid=dbeb18&_nc_ohc=lozxkMUSlXwAX8tJbzy&_nc_ht=scontent-yyz1-1.xx&oh=9f28008503d7a7566d6f929e0fa1e39f&oe=6179C542
  @width 0.3
\fi

\begin{itemize} % {
\iusr{Myroslava Vesna}
\textbf{Ліза Ребар} " патріоти" не зрозуміють...
\end{itemize} % }

\iusr{Світлана Швидка-Чирибан}
на жаль... їх таки 73\%...

\iusr{Myroslava Vesna}

Я Вас підтримаю і буду другою жіночкою " за 60". Очевидно, не всі обтяжують
свої сірі клітинки - а навіщо?! Краще покричати, лайкнути і гордо проголосити,
що найбільший патріот! Дальше свого носа не бачать і не ХОЧУТЬ бачити!

\iusr{Gmailo Petrovich Markian}

Після другої світової, багатьох німецьких учених прийняли в США, не дуже
зважаючи на те що вони працювали на Гітлерівську Німеччину. Сотні тисячі людей,
їдуть туди і шукають те місце де їх чекають, де їх оцінять, і де їх люблять.

Напревеликий жаль,Україна ще не є тою державою, в якій працюють для створення
хорошого іміджу на міжнародній арені, підтримують визначних людей, шанують
патріотів, забезпечують тих хто усе віддав за свою країну. Совкова система і її
партапарат глибоко впився в тіло народу, і висмоктує останні сили та краплі
крові із молодого організму нашої держави. 

Тотальна корупція уже фальш-патріотів, синків партійних главарів, недолугі та
зажерливі чинуші, рвуться до влади, щоб відпилити якнайбільший шмат державного
пирога. Тому у такій ситуації, якщо ми справжні патріоти, потрібно докладати
всіх сил, щоб не відганяти людей, а навпаки приваблювати. Те що Усик прихильник
МП це не тільки його промах, це промах наших політиків, державотворців,
суспільства, народу, духовенства в кінці кінців. 

Тому перш ніж жертися один з одним, через кістку розбрату, яку нам завжди
підкидали наші окоянні брати з за парєбріка, подумаймо як ми можемо єднатися,
допомагати один одному, деактивовувати корупцію, кумівство, бюрократію які
нищать наш народ. От тоді до нас самі будуть іти іноземці щоб стати українцями
а українці будуть любити свій народ і свою державу!

\begin{itemize} % {
\iusr{Олег Габалевич}
\textbf{Gmailo Petrovich Markian} чудовий пост! Це зло що нищить Україну то є система. Це та голова ворожої країни, яку залишила її комуністична влада, знищивши українську еліту. І ця псевдоеліта сьогодні донищує країну, перетворивши її в заповідник совку.
\end{itemize} % }

\iusr{Олексій Будяк}

Пардон, ми маємо купатися в морі Усикової душі, чи просто користуватись плодами
його професійної діяльності?

\begin{itemize} % {
\iusr{Юрий Филипюк}
\textbf{Олексій Будяк} Ми маємо розуміти хто він такий - українець чи московіт. Для мене особисто московіт, для вас - какая разніца.

\iusr{Serhii Bryhar}
\textbf{Олексій Будяк} Він замерає будь-які плоди професійної діяльності. Аж настільки, що користуватися ними починають вороги.

\iusr{Олексій Будяк}
\textbf{Jura Filipjuk} 

Ви хочете сказати, що московитська нація дуже багата на таланти і народила
зокрема й такого талановитого боксера, як Усик? Мушу не погодитись. По крові
Усик - українець. За самоідентифікацією - "х..ол" і "ґражданін міра". За суттю
- найманець-пристосуванець. Останнє - типова риса професійних спортсменів не
лише з пост-радянського простору, а й їхніх колег з усього світу. Винятком є
якраз "не-усики".

Тож коли український боксер українського походження здобуває чемпіонські пояси,
які свого часу роками збирав інший українець - Володимир Кличко... До перемоги
українського спорту це все-таки має певне відношення.


\iusr{Олексій Будяк}
\textbf{Serhii Bryhar} 

Людей часто підводить плутанина між поняттями "хороша людина", "патріот",
"розумна людина", "професіонал своєї справи" і ще цілим рядом схожих понять,
які десь перетинаються, але вцілому не є тотожними. Дуже гарно, коли людина
суміщає у своїй особистості усі ці риси. Для того, хто претендує на роль лідера
держави, ця умова є дуже бажаною. А в простих смертних та, чи інша риса може
випадати. Так патріота, який не дуже схильний замислюватись, можна змусити
повірити, буцімто Порошенко - "кривавий барига". А від навіть розумного
патріота, проте позбавленого професіоналізму, не буде жодної користі, якщо його
поставити на чолі профільного міністерства.

Боксер Усик, з його хитродупістю, замаскованою під "православну" упоротість, не
підходить на роль героя, на роль взірця. Особливо, в часи війни з Росією. А ось
на роль результативного спортсмена, який підкинув в нашу депресивну дійсність
трохи позитиву у вигляді досягнення ( рангом вищого, ніж в наших футболістів )
- цілком.

\iusr{Юрий Филипюк}
\textbf{Олексій Будяк} 

Ніколи не належав до категорії "какая разніца". Усик, який александр,
промосковітський холуй і мені гидко його дивитись та слухати. Соромно, що над
ним височіє прапор України.

\end{itemize} % }

\iusr{Регіна Лялюшко-Лозицька}
На превеликий жаль- це так і є.




\end{itemize} % }
