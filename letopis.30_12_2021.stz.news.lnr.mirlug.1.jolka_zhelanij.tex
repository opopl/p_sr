% vim: keymap=russian-jcukenwin
%%beginhead 
 
%%file 30_12_2021.stz.news.lnr.mirlug.1.jolka_zhelanij
%%parent 30_12_2021
 
%%url https://mir-lug.info/novosti-proektov/v-ramkah-akczii-yolka-zhelanij-rebyata-pobyvali-za-kulisami-luganskogo-czirka
 
%%author_id news.lnr.mirlug
%%date 
 
%%tags donbass,novyj_god,jolka,lnr,lugansk,deti,cirk
%%title В рамках акции «Ёлка желаний» ребята побывали за кулисами луганского цирка
 
%%endhead 
\subsection{В рамках акции «Ёлка желаний» ребята побывали за кулисами луганского цирка}
\label{sec:30_12_2021.stz.news.lnr.mirlug.1.jolka_zhelanij}

\Purl{https://mir-lug.info/novosti-proektov/v-ramkah-akczii-yolka-zhelanij-rebyata-pobyvali-za-kulisami-luganskogo-czirka}
\ifcmt
 author_begin
   author_id news.lnr.mirlug
 author_end
\fi

Заместитель министра культуры, спорта и молодёжи ЛНР Роман Олексин от имени
ведомства в рамках акции «Ёлка желаний» Общественного движения «Мир Луганщине»
исполнил новогодние желания восьми ребят, которые мечтали побывать в цирке.

\ii{30_12_2021.stz.news.lnr.mirlug.1.jolka_zhelanij.pic.1}

Карина Сопельникова из Новосветловки Краснодонского района, Эльвира Коломийцева
и ее брат Рома из Антрацитовского района, Алексей Чернявский, Анастасия
Молоканова с братом и сестрой и Тимур Халиуллин из Ровеньков побывали в
закулисье Луганского государственного цирка и увидели новогоднее цирковое
представление.

\ii{30_12_2021.stz.news.lnr.mirlug.1.jolka_zhelanij.pic.2}

Директор учреждения культуры Дмитрий Касьян провёл для ребят экскурсию по музею
цирка и показал, что происходит за кулисами до начала циркового представления.
Ребята своими глазами увидели подготовку артистов к выступлению. Смогли
пообщаться с Дедом Морозом, сфотографироваться с огромным хаски, суровым
медведем и весёлым капуцином.  Восторгу ребятни не было предела. Громкий
детский смех наполнил всё закулисье.

\ii{30_12_2021.stz.news.lnr.mirlug.1.jolka_zhelanij.pic.3}

– Я очень рад тому, что нам удалось исполнить такую светлую мечту и показать
ребятам цирк. Сегодня они узнали, как цирк устроен, его историю, что находится
за кулисами, увидели дрессированных животных. И это очень здорово! Мечты
сбываются, – отметил Дмитрий Касьян.

\ii{30_12_2021.stz.news.lnr.mirlug.1.jolka_zhelanij.pic.4}

Роман Олексин подчеркунл, что коллектив Министерства культуры, спорта и
молодёжи ЛНР был рад исполнить желания юных жителей республики.

– Я сегодня испытал незабываемые впечатления, когда увидел радость детей.
Ребята в рамках акции «Ёлка желаний» захотели увидеть цирк. Желания у них были
самые разнообразные – посмотреть закулисье, увидеть цирковых животных, побывать
на представлении. Мы собрали их всех вместе и исполнили мечту каждого.  Также
мы приготовили небольшие сюрпризы – сладкий стол и новогодние подарки, –
отметил заместитель министра.

Юная Эльвира Коломийцева рассказала, что попросила у Деда Мороза, чтобы он с
младшим братом побывала в цирке. Для ребят это первое в их жизни цирковое
представление. Раньше они видели выступления только гастролирующих артистов в
местном Дворце культуры.

– Вот моя мечта и сбылась. Я очень довольна. Я на всю жизнь это запомню. Для
меня это счастье. Я хочу счастливое детство, ведь какое будет детство, такой и
будет жизнь, – сказал она.

\ii{30_12_2021.stz.news.lnr.mirlug.1.jolka_zhelanij.pic.5}

Шестилетний брат Эли Рома тоже очень счастлив. Сегодняшний день надолго
останется в его памяти. Особенно мальчику понравился озорной капуцин, который
хлопал ребёнка по ладошке, а в ответ Рома заливался звонким смехом. Даже когда
мальчик вспоминает об этом он не может сдержать смех.

Мама ребят Дарья Коломийцева рассказала, что её семья узнала про акцию «Ёлка
желаний» по радио и ребята сразу же решили написать письмо.

– Дочка загадала желание побывать в луганском цирке. Через некоторое время нам
позвонили с очень радостной новостью,  сказали, что мы едем всей семьёй в цирк.
Мы здесь впервые. Нам очень понравилось. Был очень тёплый приём, нам очень
приятно, что детская мечта осуществилась. Огромное спасибо организаторам акции,
Министерству культуры, спорта и молодёжи ЛНР, а также Луганскому
государственному цирку, – сказала она.

Напомним, что реализация акции «Ёлка желаний» продлится до 28 февраля.  Для
участия в акции желающим необходимо снять шар с ёлки, которая находится в
Луганске по адресу: ул. Карла Маркса, 7. Время работы ёлки желаний по будням с
9.00 до 18.00.
