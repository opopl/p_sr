% vim: keymap=russian-jcukenwin
%%beginhead 
 
%%file 07_02_2023.fb.fb_group.vrazhennja.ua.1.davaite_pro_pozitiv_.cmt
%%parent 07_02_2023.fb.fb_group.vrazhennja.ua.1.davaite_pro_pozitiv_
 
%%url 
 
%%author_id 
%%date 
 
%%tags 
%%title 
 
%%endhead 

\qqSecCmt

\iusr{Olga Angelova}

\enquote{Полліанна} Елеонор Портер

\begin{itemize} % {
\iusr{Катерина Черкашина}
\textbf{Olga Angelova} о, це справді позитив)

\iusr{Olga Kazak}
\textbf{Ольга Ангелова} 

Дуже крута книга. Коли мені її рекомендували, я навіть і здогадатися не могла,
що книга орієнтована здебільшого на молодшого читача може викликати таке
захоплення у дорослих

\iusr{Yulia Mozhova}
\textbf{Olga Angelova} Свого часу дуже мені допомогла. Потрапила до рук у потрібний момент

\iusr{Леся Ткаченко}
\textbf{Olga Angelova} прочитала у дитинстві і досі іноді перечитую. Книга, яка відроджує жагу до життя.
\end{itemize} % }

\iusr{Катерина Черкашина}

Цікаве питання. Мені здається, що \enquote{повертає бажання жити} - це про вивільнення
негативних емоцій, про те, щоб проридатися. І тут підійде будь-яка книга, яка
дозволить прожити емоційний катарсис. Більше того, мені здається, що простіше
книга, то краще.

Я пам'ятаю два рази, коли книга попала мені в потрібен момент і викликала хвилю емоцій.

Це були \enquote{ПС. Я люблю тебе} Сесилії Ахерн і \enquote{Лісова пісня} Лесі Українки)

А, згадала. Із недавнього це була \enquote{Життя PS} Лєри Бурлакової

\iusr{Алла Безрук}

В. Франкл \enquote{Людина у пошуках справжнього сенсу}

\begin{itemize} % {
\iusr{Senchyshyn Myroslava}
\textbf{Алла Безрук} О, тааааак
\end{itemize} % }

\iusr{Саша Серпень}

Сяйво та Доктор Сон. Стівен Кінг.

\iusr{Iva Bilka}

Зараз мені прилетить, але колись в депресивні підліткові роки мені дуже
допомогла віднайти бажання жити \enquote{Вероніка вирішила померти}. Правда спочатку
фільм, а потім книга. Багато моментів тоді у мене співпало з гг.

А для натхнення час від часу почитую твори Пагутяк. Дуже подобається її стиль.

\begin{itemize} % {
\iusr{Вікторія Адаменко}
\textbf{Iva Bilka} тільки хотіла писати те саме про \enquote{Вероніку}. А ви вже написали...
Дуже часто згадую цю книгу

\iusr{Алла Безрук}
\textbf{Iva Bilka}, свого часу і мені ця книга подобалась, є над чим замислитись..

\iusr{Цікава Ґава}
\textbf{Iva Bilka} я і досі маю цю трилогію на полиці. Вона правда російською, тому ховаю. Це єдині русняві книги + Піколт, від яких я поки що не можу позбавитися.
\end{itemize} % }

\iusr{Катерина Черкашина}

Бажання до творчості мені будила Велика магія від Ліз Гілберт

\iusr{Nastia Dzhula}

Щоденник Бріджит Джонс - єдина книга яку перечитувала разів 5-6

\iusr{Антоніна Федорович}

Вільна Гленон Дойл

\begin{itemize} % {
\iusr{Людмила Тараненко}
\textbf{Антоніна Федорович} зараз читаю. Не все мені \enquote{лягає}, але дуже гарно написано

\iusr{Антоніна Федорович}
\textbf{Людмила Тараненко} звісно, що не все, бо в кожного свій досвід, свої тригери.
Я просто проридала над книгою. Чесно

\iusr{Антоніна Федорович}

Ось

\url{https://www.facebook.com/groups/vrajenniya.ua/permalink/1377166519308450}

\iusr{Надія Омельчук}
\textbf{Антоніна Федорович} наскільки ж ми всі різні) Я її так і не осилила - продала)
\end{itemize} % }

\iusr{Юлия Шевченко}

Піднесені вітром М. Мітчел та уся серія про Мумі Тролей

\iusr{Inga Lazoryak}

Джейн Ейр і Віднесені вітром 📚❤ назавжди

\begin{itemize} % {
\iusr{Pisachova Marina}
\textbf{Инга Лазоряк} Джейн Ейр❤️ читала разів сто) і дивилася всі екранізації, улюблена з Маклом Фасбендером

\iusr{Aryna Andriichenko}
\textbf{Pisachova Marina} прийшла написати про Джейн Ейр. Перечитую може раз на рік з дитинства. Щоправда, з віком інакше вже сприймаєш персонажів (типу містер Рочестер - абьюзер і расист, Джейн трохи занадто емоційна), але все рівно люблю.

\iusr{Pisachova Marina}
\textbf{Aryna Andriichenko} про містера Рочестера 100\%

\iusr{Aryna Andriichenko}
\textbf{Inga Lazoryak} раніше теж Віднесені вітром любила. Зараз описи природи здаються занадто довгими, головна пара - двоє нещасних людей, що через свій інфантилізм не змогли стати щасливими разом, ну і відвертий расизм авторки теж дістає. Що лишилось добрим - це психоаналіз вчинків героїв і опис історичних подій
\end{itemize} % }

\iusr{Катерина Черкашина}

Позитивними вважаю книги з Дискосвіту Тері Пратчетта.

І всю творчість Маркеса. А ще проза Йогансена)

\begin{itemize} % {
\iusr{Олена Косовська}
\textbf{Катерина Черкашина} Пратчетт - безпрограшний варіант, коли хочу чогось крутого і не знаю, чого. От дереш будь-яку його книгу і в яблучко.
\end{itemize} % }

\iusr{Євген Агєєв}

\enquote{Нескінчений жарт}

\iusr{Людмила Тараненко}

Ой, ну це були різні книги в різний час. Буває, що в цей конкретний момент тобі
трапляється потрібна книга. Ніколи не вгадаєш, що саме зараз зайде.

Я люблю книги Плохія.

Колись мені дуже зайшли книги Стефані Маєр із серії \enquote{Сутінки}.

На початку повномасштабного вторгнення я прочитала декілька творів Ремарка і
\enquote{Марусю} Шкляра. Вони були ну дуже \enquote{на часі}

\iusr{Pisachova Marina}

Із книг «Чаликушу» Решат Нурі Гюнтекін. Фільми про Гаррі Поттера. Передивилася
весною, коли виїхала з Чернігівщини із зони бойових дій

\begin{itemize} % {
\iusr{Svitlana Kibich}
\textbf{Pisachova Marina} так, і в мене \enquote{Чаликушу}. Читала-перечитувала разів 10. Фільм теж сильний, особливо вдало підібрана музика.

\iusr{Anna Daineko}
\textbf{Pisachova Marina} оооо, дякую, що нагадали! Пішла копистатися у шафі) Чаликушу - те, що мені сьогодні вкрай потрібно)
\end{itemize} % }

\iusr{Діана Веллс}

\enquote{Енн із Зелених Дахів}

\iusr{Liudmyla Holenko}

Регіна Бретт \enquote{Будь дивом}

\iusr{Senchyshyn Myroslava}

Біблія, Толкін, Клайв Люїс, Томас Мертон....

\iusr{Olha Ivashchenko}

Маруся Квітки-Основьяненка

\iusr{Oksana Borodenko}

Енн із Зелених Дахів і ... тільки не смійтеся... Мумі Тролі

\begin{itemize} % {
\iusr{Tetiana Protsiv}
\textbf{Oksana Borodenko} а чому сміятись!? В мене це най-най-книга! Перечитуємо з малим всі три книги вже разів п'ять! Я їх просто обожнюю! Вище писала саме про Мумі Тролів!
\end{itemize} % }

\iusr{Yaroslav Diakun}

Сила моменту Тепер, Екгарт Толле

\iusr{Iryna Zvorska}

Абрикосова книгарня
Шлях мирного воїна

\iusr{Тетяна Черниш}

Трилогія перший закон Аберкромбі. Перші книги, які після початку великої війни
змусили мене загубитися у читанні.

\begin{itemize} % {
\iusr{Максим Степанюк}
\textbf{Тетяна Черниш} оо, він прикольно пише. В спину йому дихає Желязни, але Аберкромбі, явно краще. Дійсно норм автор.

\iusr{Тетяна Черниш}
\textbf{Максим Степанюк} хроніки Абмера це мій наступний етап

\iusr{Максим Степанюк}
\textbf{Тетяна Черниш} а, я за них по приколу узявся. 150 грн, то не гроші. Ну я і взяв.
\end{itemize} % }

\iusr{Arsenii Troian}

Книги лише все це забирають, особливо сили. Це все, про що ви зазначили, мені
дає музика

\iusr{Taisiya Nakonechna}

Тревор Гадсон «Про що нас запитує Бог»

\iusr{Bakina Kristina}

Сліди на дорозі Маркус! Книжки Сайгона.

\iusr{Ирина Март}

\enquote{Сад Гетсиманський}, \enquote{Жовтий князь}. Це перше що згадалось.

\iusr{Олена Макарчук}

Джоджо Моєс \enquote{До зустрічі з тобою}. Я ледве не розплакалася, коли виявила її в
евакуаційному наплічнику. Автоматично кинула а потім хопа!

\begin{itemize} % {
\iusr{Svitlana Salamatina-Kovalenko}
\textbf{Олена Макарчук} я плакала і над книгою, і над фільмом. І відмовилася читати продовження...

\iusr{Олена Макарчук}
\textbf{Svitlana Salamatina-Kovalenko} правильно що відмовилися. Зіпсувала

\iusr{Svitlana Salamatina-Kovalenko}
\textbf{Олена Макарчук} не знаю...донька читала,сподобалося. Але я не хотіла.

\iusr{Олена Макарчук}
\textbf{Svitlana Salamatina-Kovalenko} не варто. Нічого не втратите. А я не буду читати продовження Самотності в мережі

\iusr{Olena Laur-Kozko}
\textbf{Svitlana Salamatina-Kovalenko} а книга від фільму дуже відрізняється? Бо я тільки фільм бачила.

\iusr{Svitlana Salamatina-Kovalenko}
\textbf{Olena Laur-Kozko} ні)

\iusr{Олена Макарчук}
\textbf{Svitlana Salamatina-Kovalenko} фільм дивитися теж не буду. Я не люблю таке. Або-або. Якщо читала книгу - не дивлюся фільм. І навпаки

\iusr{Svitlana Salamatina-Kovalenko}
\textbf{Олена Макарчук} я зазвичай теж так)
\end{itemize} % }

\iusr{Жанна Прядко}

Дуже допомогла книга Джоді Піколт \enquote{Книга двох шляхів}. Вчасно. Якраз
після смерті мами.

\begin{itemize} % {
\iusr{Цікава Ґава}
\textbf{Жанна Прядко} обожнюю Піколт. Цю видавав КСД російською чи Ви читали англійською?

\iusr{Жанна Прядко}
\textbf{Цікава Ґава} на жаль, російською. Але авторку - обожнюю.

\iusr{Цікава Ґава}
\textbf{Жанна Прядко} і я. А як книга називалася російською, бо не пригадую такої назви

\iusr{Жанна Прядко}
\textbf{Цікава Ґава} Книга двух путей

\iusr{Цікава Ґава}
\textbf{Жанна Прядко} це точно було КСД чи все-таки Азбука?

\iusr{Жанна Прядко}
\textbf{Цікава Ґава} точно купляла в КДС. Але чи вони видавали, не скажу. Бо зараз не вдома
\end{itemize} % }

\iusr{Valentina Mihaylenko}
\textbf{Джен Ейр} і все (не лише детективи) Агати Крісті

\iusr{Леся Тиркус}

Мирослав Дочинець \enquote{Вічник. Або сповідь на перевалі духу}

\iusr{Світлана Шинкарук}

Гаррі Поттер) з нею я можу відгородитися від усього поганого)

\begin{itemize} % {
\iusr{Olena Tovstopiat}
\textbf{Світлана Шинкарук} я теж🙆

\iusr{Наталія Антоновська}
\textbf{Світлана Шинкарук} і у мене так😄
\end{itemize} % }

\iusr{Юля Кучер}

«Дівчина з тату дракона» Стіг Ларсон

\iusr{Юлія Ільніцька}

\enquote{Бог завжди подорожує інкогніто} Гунеля)

\iusr{Іванна Анчаківська}

Не зовсім книги але є такі журнали Колесо Життя - ще ті мотиватори)))

\iusr{Yulia Tsidylo}

Подивлюся, бо в мене такої книги немає, мабуть... Хоча одна з найпозитивніших
це \enquote{Смажені зелені помідори...}

\iusr{Єлена Макарова}

Книги Агати Крісті

\iusr{Татьяна Тарасюк}

Алхімік

\iusr{Максим Степанюк}

Конан-варвар та Heavy metal \#1 і Етноґрафічний збірник за 1916й

\iusr{Дарина Возна}

\enquote{Хроніки Нарнії} моє все, більше нічого не зустрічала, щоб аж так надихало

\iusr{Оксана Комарянська}

Біблія

\begin{itemize} % {
\iusr{Svetlana Kureeva}
\textbf{Оксана Комарянська} +++ ♥️♥️♥️
\end{itemize} % }

\iusr{Anastasia Fess}

Маленький Принц. Настільки люблю цю історію, що збираю її в різних виданнях.

\iusr{Галина Якименко}

Довгий час це була книга \enquote{Звіяні вітром}.

\iusr{Yulia Mozhova}

Дюна Герберта, саме перша книга. Таємничий острів Жюля Верна. Кульбабове вино
Бредбері. І Гаррі Поттер. Хоча для кожного моменту є \enquote{своя} книга

\iusr{Lory Yas}

Збірка поезій Павла Вишебаби \enquote{Тільки не пиши мені про війну}, актуальчик так би мовити

\iusr{Natalia Skorokhoda}

Рей Бредбері \enquote{Кульбабове вино} та інші твори

Луїза Елкотт \enquote{Маленькі жінки} дві частини

Дж. Роулінг \enquote{Гаррі Поттер} особливо 1 частина

Клайв Льюїс \enquote{Хроніки Нарнії}

Елізабет Гілберт \enquote{Їсти, молитися, кохати}

P S. А ще мене вражає, що під цим дописом є хмара людей зі схожими уподобаннями:)

Не хочу писати список з 10 авторів, бо всі вони є у коментарях інших читачів)

\iusr{Паша Діскаленко}
\textbf{Олена Печорна} \enquote{В затінку земної жінки}. Наріне Абгарян \enquote{Далі жити}

\iusr{Olena Tovstopiat}

Лавр. Сергій Водолазкін (з автором можу помилятися). Проте після прочитання тої
книги, инші не йшли ні до серця ні до голови.

\iusr{Anna Kosaryeva}

Урсула Ле Гуін

\begin{itemize} % {
\iusr{Kim Bommie}
\textbf{Anna Kosaryeva} \enquote{Чарівник Земномор'я} чи якась інша книга? ☺️

\iusr{Anna Kosaryeva}
\textbf{Kim Bommie} і уся трилогія про Зеда (там сам чарівник земноморя, гробниці Атуана, на останньому березі) і весь Хайнський цикл.

\iusr{Ольга Колесникова}
\textbf{Anna Kosaryeva} аввввв я її обожнюю!😍🥰
\end{itemize} % }

\iusr{Ольга Анцибор}

Перечитала майже всі коментарі і була засмучена. Книги українських авторів
майже ніхто не назвав. Тільки кілька людей і то читали класиків. А сучасних
українських авторів ніхто. Шкода. А мені дають наснагу до життя і натхнення до
творчості романи нашої сучасниці, прекрасної письменниці, моєї подруги( чим я
дуже пишаюся ) Валентини Михайленко. Не можу навіть виділити якийсь твір, всі
її романи чудові, захоплюючі, в них саме життя і з радощами і з болем.
