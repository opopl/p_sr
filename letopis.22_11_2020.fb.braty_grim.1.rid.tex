% vim: keymap=russian-jcukenwin
%%beginhead 
 
%%file 22_11_2020.fb.braty_grim.1.rid
%%parent 22_11_2020
 
%%url https://www.facebook.com/2306674846060629/photos/a.2306853576042756/3527898570604911/
%%author 
%%author_id 
%%tags 
%%title 
 
%%endhead 

\subsection{Марковичі - козацький рід}
\Purl{https://www.facebook.com/2306674846060629/photos/a.2306853576042756/3527898570604911/}

\ifcmt
pic https://scontent-waw1-1.xx.fbcdn.net/v/t1.0-9/126891901_3527898577271577_2269508540420112147_n.jpg?_nc_cat=104&ccb=2&_nc_sid=9267fe&_nc_ohc=34_SvemS6UgAX-jcGcc&_nc_ht=scontent-waw1-1.xx&oh=9699e79edc04fd56568a4d46da937e23&oe=5FE123C1
\fi

Побутує думка, що українській нації не щастило сформувати власну еліту, бо
нація без держави не здатна виплекати свою еліту через відсутність традиції і
поколіннєвої тяглості елітного середовища. 

Є підстави так вважати, адже українську еліту або знищували під корінь, або
асимілювали загарбники.

Тим більше нам варто знати та вивчати родоводи славних українських родів.

Фахівці стверджують, що більшість українців знають заледве своїх дідів чи
прадідів. Бо нащадків нашої еліти, справжніх українських аристократів,
продовжувачів гетьманських родів викосили ще в 1930-ті. А саме вони вони
творили українську історію.

Серед них - Марковичі, вони ж Маркевичі в їхній чернігівській лінії, які
належать до найвизначніших родів XVIII-XIX століть.  Марковичі вважали своїм
предком напівлегендарного Аврама, який на початку ХVII століття оселився на
Прилуччині.

Його син, підприємець Марко Аврамович, володів значними маєтностями у Прилуках
та Пирятині і власним коштом звів Пречистенську церкву у Прилуках.

Сини Марка належали до козацької старшини: Андрій був лубенським полковником і
генеральним підскарбієм, Іван - прилуцьким сотником та полковим суддею, 

Федір теж прилуцьким сотником і бунчуковим товаришем.

Дочка Марка Анастасія стала дружиною гетьмана Івана Скоропадського, на якого
мала значний вплив, так що козаки говорили: «Іван носить плахту, а Настя
булаву». Так що Настя ймовірно була навіть фактичною правителькою України. 

Старший син Андрія Яків започаткував чернігівську лінію роду Марковичів, до
якої, зокрема, належали історики Яків та Олександр.

Другий син Андрія Семен став засновником роменської лінії. До неї належали
етнограф, член Кирило-Мефодіївського братства Опанас Маркович, який взяв шлюб з
письменницею Марко Вовчок, та його небіж, письменник Дмитро Маркович, що став
генеральним прокурором Української народної республіки та сенатором Української
держави.

Натомість Федір Маркович заснував прилуцьку лінію, представники якої у ХІХ
столітті стали називатися Маркевичами.

Історик, етнограф, поет, музика та композитор Микола Маркевич приятелював з
Тарасом Шевченком, і саме йому поет присвятив свій вірш «Бандуристе, орле
сизий».

Приятелем Шевченка був і Андрій Маркевич, юрист і сенатор, який вже у віці
сімдесяти семи років домігся дозволу видати перше в Російській імперії повне
видання творів поета.

Онук Марка Аврамовича Яків навчався у Києво-Могилянській академії, де був улюбленим учнем її ректора Феофана Прокоповича.

Згодом Яків Маркович служив у війську в ранзі бунчукового товариша, а коли його
батько Андрій з козаками відправився на Ладогу рити канали, гетьман
Скоропадський призначив Якова наказним лубенським полковником.

1725 року Яків брав участь у дворічному Дербентському поході в Персію, в якому
його батько був наказним гетьманом. Пізніше він був призначений генеральним
підскарбієм.

Яків Маркович був одним з найосвіченіших людей Гетьманщини і добре володів
польською, німецькою, російською, французькою та латинською мовами. Він
перекладав з латини, власноручно переписував книги, зокрема Літопис Граб’янки,
писав вірші та прозу.

1715 року Яків Маркович став продовжувати записи історичних подій під назвою «Кройніка», які вів його тесть Павло Полуботок.

1717 року Яків почав власний «Щоденник» і вів його упродовж наступних
півстоліття, заповнивши загалом 10 книг. У «Щоденнику» він описував розвиток
економіки Гетьманщини та її торгівельні зв’язки з Польщею, Запоріжжям, Росією,
Кримом та країнами Західної Європи, наводив відомості про чисельність козацьких
полків і їхні походи.

Після смерті Якова Марковича «Щоденник» був виданий тричі, а перше видання
здійснив у російському перекладі його онук Олександр 1859 року.

І ще один з Марковичів, книга котрого стала помітною сторінкою в українській
історії. До Чернігівської лінії Марковичів належав і онук Якова Марковича, теж
Яків, який народився 27 жовтня 1776 року у родинному маєтку неподалік Пирятина
на Полтавщині. Він навчався у Московському університетському пансіоні, згодом
працював у Колегії закордонних справ у Санкт-Петербурзі.

Там 1798 року 22-річний Яків Маркович-молодший видав першу частину своєї
історичної праці під назвою «Записки про Малоросію, її жителів та виробництва».
Це – книжка-вершина української історіографії 18 століття, яка буквально
пронизана просвітницьким духом з масою використаної літератури.

Тож маємо приклад надзвичайно розгалуженого українського роду, який дав Україні
істориків, державних діячів, науковців, який пронизав фактично все життя
українського Лівобережжя XVIII-XIX століть, тобто тих земель, де найбільше
зберігся дух українського автономізму і де найраніше почалося модерне
українське національне відродження.

Джерело:
\url{https://www.radiosvoboda.org/a/953883.html}

