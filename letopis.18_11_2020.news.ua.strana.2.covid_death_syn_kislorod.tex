% vim: keymap=russian-jcukenwin
%%beginhead 
 
%%file 18_11_2020.news.ua.strana.2.covid_death_syn_kislorod
%%parent 18_11_2020
 
%%url https://kiev.strana.ua/301674-pod-kievom-muzhchina-umer-ot-koronavirusa-pozhertvovav-kislorod-synu-.html
%%author 
%%tags 
%%title 
 
%%endhead 

\subsection{В больнице Киевской области умер больной коронавирусом украинец, который пожертвовал свой кислород сыну - волонтер}
\Purl{https://kiev.strana.ua/301674-pod-kievom-muzhchina-umer-ot-koronavirusa-pozhertvovav-kislorod-synu-.html}

16:12, сегодня

\ifcmt
pic https://kiev.strana.ua/img/article/3016/74_main-v1605707718.jpeg
caption https://kiev.strana.ua/img/article/3016/74_main-v1605707718.jpeg
\fi

В больнице Белой Церкви умер больной коронавирусом мужчина, который из-за
нехватки кислородных концентраторов пожертвовал своими минутами на
аппарате в пользу сына, также больного Covid-19.

Об этом сообщила волонтер Дана Яровая на своей странице в Фейсбуке.

При этом в Киевской ОГА говорят, что, по их информации, такого случая в
области не было.

Дана Яровая рассказала, что в больницах Киевской области не хватает
кислородных концентраторов. В итоге больным дают дышать кислородным
концентратором по 15 минут в день.

"Сегодня я опять искала кислородный концентратор... Концентратор для
человека, который не лечится, как я и многие пациенты, дома. Концентратор
для человека, которого не берут в реанимацию только потому, что нет мест,
и который находится в больнице не в далеком областном центре Урюпинска, а
в городе Киев... Так вот, человеку с сатурацией, которому показана
реанимация, отказывают в месте в реанимации, ибо некуда, и дают кислород,
внимание(!), на 15 минут в сутки. Вдумайтесь! 15 минут в сутки!" -
написала волонтер.

Далее она написала, что ей стало известно о том, что у ее знакомого в
районном центре, отец и сын попали в больницу. И те же 15 минут в сутки им
давали кислород.

"В районном центре отец и сын попали в больницу. И те же 15 минут в сутки
им давали кислород. Отец отказался от своих 15 минут в день в пользу сына.
Отец умер, сына продолжают лечить. Только вот не знаю, за умершего отца
ему дают еще 15 минут кислорода, или уже нет", — написала Яровая.

По ее словам, 15 минут кислорода для человека, который постоянно
задыхается, очень мало. "Кто не пережил признаки одышки во время
коронавируса, не поймет. У меня был концентратор, но на подкорке все равно
этот жуткий страх, что в следующий раз ты можешь не вдохнуть. Я задыхалась
настолько, что не могла говорить. И еще раз, это я была с концентратором.
А теперь представьте человека, который задыхается сутки и ему 15 минут
дают подышать кислородом", — рассказала волонтер.

\ifcmt
tab_begin
	caption Пост Даны Яровой в Facebook
	pic https://kiev.strana.ua/img/forall/u/11/1/Screenshot_36(7).jpg
	pic https://kiev.strana.ua/img/forall/u/11/1/Screenshot_37(6).jpg
tab_end
\fi

При этом в посте она не уточнила, в каком именно райцентре это произошло.
В пресс-службе Киевской ОГА сообщили, что в Киевской области такого случая
не было.

"Сейчас подняли всю информацию - нет в Киевской области такого случая. И
это может подтвердить директор департамента охраны здоровья" - сообщил
руководитель управления коммуникаций КОГА Сергей Куница.

Позже в комментарии "Стране" волонтер уточнила, что речь идет о больнице
№3 в городе Белая Церковь Киевской области.

\ifcmt
pic https://kiev.strana.ua/img/forall/u/0/34/%D0%A1%D0%BD%D0%B8%D0%BC%D0%BE%D0%BA(384).JPG
\fi

Главный врач указанного медучреждения эту информацию опровергла. "В
подчиненном мне медучреждении такого случая не было", -  процитировали
главврача на сайте Киевской облгосадминистрации.

"Некоторые электронные средства массовой информации сообщили, что в одной
из больниц Белой Церкви, Киевской области, умер больной коронавирус
человек, который уступил свой кислород также больному сыну. Данная
информация не соответствует действительности" - говорится в официальном
сообщении.

По информации департамента здравоохранения Киевской областной
государственной администрации, на сегодня в области из 2185 коек,
отведенных под коронавирусных пациентов, к кислороду подключены 1355. При
этом заполнены 825 коек, свободными остаются 530 кроватей.

"Проблемы с обеспечением кислородом пациентов не существует", - сообщила
департамента охраны здоровья КОГА Елена Ефименко.

Напомним, ранее сообщалось, что в Житомире от коронавируса умерла женщина,
которая жаловалась мэру на отсутствие кислорода в реанимации.\Furl{https://kiev.strana.ua/news/301457-v-zhitomirskoj-bolnitse-ot-koronavirusa-umerla-zhenshchina-zajavljavshaja-o-nekhvatke-kisloroda.html}

Также мы подробно рассказывали, почему ковидные больницы в Украине стали
адом для больных и врачей.\Furl{https://kiev.strana.ua/articles/analysis/298581-pochemu-kovidnye-bolnitsy-stali-adom-i-dlja-patsientov-i-dlja-vrachej.html}

%Популярное
%1
%"Хотите меня ударить?". На канале Порошенко внезапно "закошмарили" его
%соратницу Федыну. Анатомия скандала
%2
%В России десятки мужчин бросились прыгать на эскалаторе ради спасения
%девочки. Видео
%3
%Игорь Уманский: "В правительстве решили: лучше мы будем грабить на
%дорогах, чем Степанов на медицине"
%4
%Украинцы без стажа не смогут получать пенсии с 2021 года
%5
%"Родился украинцем - умер москалем". Как националисты радовались смерти
%режиссера Романа Виктюка
%Частная жизнь
%[IMG]
%Гео Лерос о своем отпуске в Египте: "Я отдыхал 4 дня в номере, по цене
%чуть больше $200 в сутки"
%Страна » Киев » В больнице Киевской области умер больной коронавирусом
%украинец, который пожертвовал свой кислород сыну - волонтер
%Читайте также
%* В Украине новый смертельный антирекорд. За сутки от Covid-19 умерли
%256 человек
%* В столице за сутки зафиксировали 1 213 случаев коронавируса, число
%умерших перевалило за тысячу - Кличко
%* В Житомире от коронавируса умерла женщина, которая жаловалась мэру на
%отсутствие кислорода в реанимации


