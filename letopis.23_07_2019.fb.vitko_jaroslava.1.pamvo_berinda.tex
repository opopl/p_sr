% vim: keymap=russian-jcukenwin
%%beginhead 
 
%%file 23_07_2019.fb.vitko_jaroslava.1.pamvo_berinda
%%parent 23_07_2019
 
%%url https://www.facebook.com/yaroslava.vitko/posts/2328441830567077
 
%%author_id vitko_jaroslava
%%date 
 
%%tags berinda_pamvo,filologia,jazyk,jazykoznanie,mova,ukraina
%%title Хвилинка історії мовознавства, нічого особистого - Памво Беринда
 
%%endhead 
 
\subsection{Хвилинка історії мовознавства, нічого особистого - Памво Беринда}
\label{sec:23_07_2019.fb.vitko_jaroslava.1.pamvo_berinda}
 
\Purl{https://www.facebook.com/yaroslava.vitko/posts/2328441830567077}
\ifcmt
 author_begin
   author_id vitko_jaroslava
 author_end
\fi

Хвилинка історії мовознавства, нічого особистого. 

Памво Беринда - український мовознавець, лексикограф, письменник, видатна
постать українського освітньо-культурного руху першої половини ХVII століття.
Автор праці, яку можна вважати одним із перших словників української мови, -
"Лексіконъ славенорωсскїй альбо Именъ тлъкованїє" (1627 р). 

Так от, в його праці зустрічається ідіома "слуга народу" зі значенням
"мучитель, кат, цекляр".

\ifcmt
  ig https://scontent-frx5-1.xx.fbcdn.net/v/t1.6435-9/67154669_2328441803900413_8775057814285451264_n.jpg?_nc_cat=105&ccb=1-5&_nc_sid=8bfeb9&_nc_ohc=1HPHVy_YaT8AX_vpQND&_nc_ht=scontent-frx5-1.xx&oh=0e7785b3c7fb3f731f4277c72d65c60a&oe=61658746
  @width 0.8
\fi

А ви вірите в знаки й символи?
