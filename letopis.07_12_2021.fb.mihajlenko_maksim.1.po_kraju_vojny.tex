% vim: keymap=russian-jcukenwin
%%beginhead 
 
%%file 07_12_2021.fb.mihajlenko_maksim.1.po_kraju_vojny
%%parent 07_12_2021
 
%%url https://www.facebook.com/max.mykhaylenko/posts/10158613235126762
 
%%author_id mihajlenko_maksim
%%date 
 
%%tags napadenie,putin_vladimir,rossia,ugroza,ukraina,vojna
%%title ПО КРАЮ ВОЙНЫ ИЛИ КАТАСТРОФЫ ИЗ МЕЛОЧЕЙ
 
%%endhead 
 
\subsection{ПО КРАЮ ВОЙНЫ ИЛИ КАТАСТРОФЫ ИЗ МЕЛОЧЕЙ}
\label{sec:07_12_2021.fb.mihajlenko_maksim.1.po_kraju_vojny}
 
\Purl{https://www.facebook.com/max.mykhaylenko/posts/10158613235126762}
\ifcmt
 author_begin
   author_id mihajlenko_maksim
 author_end
\fi

ПО КРАЮ ВОЙНЫ ИЛИ КАТАСТРОФЫ ИЗ МЕЛОЧЕЙ

У Владимира Путина и его присных - натуральная истерика. Как старый
провинциальный опер, он-то хотел взять \enquote{склонного к демагогии}, как явно
написано в соответствующем досье начала 80-х, Джозефа Байдена \enquote{на слабо}. А
теперь бывший дрезденский топтун сам оказался в клещах масштабной ответной
реакции, в которой больше не получается \enquote{играть на имидж}. Ему-то, по сути,
необходимо было показать своей ОПГ, составленной из тех, кто разокрал активы
РСФСР и севших им на шею рэкетиров - что он всё еще решает их дела на
единственно важном для них уровне. Пусть методом сомалийских пиратов, пусть
будучи \enquote{Верхней Вольтой} с ракетами, пусть изображением из себя маниака
\enquote{держите меня семеро}.  Да только вот никакой Путин не маниак. Разумеется, он
живёт в пузыре, надутом за 20 лет придворными сочинителями, в котором
вымирающая Россия вовсе не бродяжит по помойкам 4-го мира, а покоряет Луну,
Марс и Альфу Центавра, но по своей природе он этакий \enquote{жучок}, из тех, которые
своим вещизмом и схематозами подточили СССР. 

По классическому определению, он типичный \enquote{кадавр, неудовлетворённый
желудочно}, которому были чужды идеи милленаристской жреческой секты
большевиков, а впоследствии, и инопланетной \enquote{экономики социализма}, построенной
на дефиците. Владимир Владимирович - активный участник разграбления огромного
промышленно-логистического комплекса города Ленинграда и Ленинградской области
- отнюдь не склонен к большим идеям, и даже имитировать их у него выходит
плохо, ведь юрфак ЛГУ прошёл для него у каскадёров и в секции дзюдо.  

В конце концов, хорошо служа клепто-прорабам перестройки, и обеспечивая их
пенсии, Путин дорос до статуса одного из лидеров транснационального
криминального синдиката, изначально строившегося на теневых инвестициях
ибероамериканской наркомафии и ее западноевропейских банкиров в постсоветские
рынки, и смог сделать российское крыло доминирующим в этом синдикате. 

В частности, благодаря виктимности немцев и паралича их системы охраны
конституционного порядка. С этим он и войдёт, несомненно, в историю, в том же
смысле, в котором вошли в неё некоторые пиратские адмиралы Карибов или боссы
итало-американской the Mob. Если бы Путин был пошире в кости, даже внешне
немного напоминал бы сегодня Сальваторе Луканию (Lucky Luciano).  Но как только
дело доходит до сферы подлинно политического - у \enquote{Лаки Пучиано} начинаются
проблемы неразрешимого характера, потому как в голове у него мешанина из
речекряков нескольких разрушенных в ХХ веке на территории северной Евразии
цивилизаций. 

При этом больше всего он боится тех, кого он ограбил - карибские пираты грабили
чужих, условные сицилийцы \enquote{помогали друг другу} (отсюда и cosa nostra - \enquote{наше
дело}), а вот Московия всегда стояла на порабощении большинства собственного
населения и захвате чужих земель с целью \enquote{наживы-выживания} (отсюда и это
вечное убожество).  

В 2013-14 годах (2004-5 год - это несколько другая история, хотя тоже
малоприятная для Москвы) Путин наотмаш получил по морде - используя сленг его
МИДа - от украинцев. В последний раз Кремль так получал только от поляков в
начале 20-х годов, то есть сто лет назад. Поэтому в длящемся уже много лет - и
уверяюю вас, оно еще немало лет продлится - противостоянии, на кону оказалась
крайне специфическая (в том смысле, в котором она имеется у Талибана или
сообщества на севере Йемена) государственность самой Московии-постРСФСР-постРФ. 

Однако, технически ситуация и впрямь много лет напоминает изначальный
\enquote{сетап} Первой мировой - воевать в Европе особенно никто не хочет. Но,
как тихо приговаривал, прихлёбывая чаёк в Цюрихе другой Владимир - Ильич,
\enquote{а придётся}.  

Ведь с какой стороны не подходи к Украине - с прогрессистской, как к
\enquote{витрине}, с консервативно-мистической, как к сердцу древней империи, и
связанной с ней стороны геополитической - как к Хартленду, с палеологистической
- как к огромному сухопутному Гибралтару, с историцистской - как к Новой
Польше...- более ценного приза на планете сегодня нет, и пока не растаят льды,
не возникнет. 

Может ли разрешить эти накопившиеся противоречия реконфигурация баланса сил?
Ещё лет десять назад можно было бы сказать \enquote{да}, но ныне, из-за
дегенеративной примитивности управляющего контура РФ - однозначного ответа нет. 

Подобная неопределённость в мире, в котором сложилась диспропорция между
носителями мировоззрения смотрящего в звёзды и носителями архаичного патогена
перманентного насилия - заставляет украинцев, как и многие народы за тысячи лет
человеческой истории до них, а некоторые - и сегодня, окончательно перейти к
мобилизационной модели коллективного сознания на несколько поколений вперёд.

\ii{07_12_2021.fb.mihajlenko_maksim.1.po_kraju_vojny.cmt}
