% vim: keymap=russian-jcukenwin
%%beginhead 
 
%%file 30_04_2020.fb.fb_group.story_kiev_ua.2.kievskie_mozaiki.pic.1
%%parent 30_04_2020.fb.fb_group.story_kiev_ua.2.kievskie_mozaiki
 
%%url 
 
%%author_id 
%%date 
 
%%tags 
%%title 
 
%%endhead 

\ifcmt
	  ig https://scontent-frx5-1.xx.fbcdn.net/v/t1.6435-9/95559919_3174547625912156_6325568219402731520_n.jpg?_nc_cat=105&ccb=1-5&_nc_sid=b9115d&_nc_ohc=RJaGwjnfQCQAX_YPnYD&_nc_oc=AQnyVO_YkCXV_lc6Gyn3CZ7jqGFoYL5xwlmUpVn6MHYeldXuRDZe6h8YqGQNFZjPp7g&_nc_ht=scontent-frx5-1.xx&oh=a7dd1cccf20d53b80d78b903699e62f2&oe=61B41D51
	  @width 0.4
\fi

\iusr{Ирина Петрова}
Дякую КІ!!! Зараз все виглядає чудово! Ми круті!

\iusr{Oksana Bilous}
Це Хмельницького, 26?

\iusr{Ирина Петрова}
Цей будинок під адресою Володимирська 51-53, Хмельницького 26 наступний, оцей старовинний з балкрончиками (на фото праворуч видно)

\iusr{Инна Шмыгина}
На заднем плане Киевпроект

\iusr{Ирина Петрова}
Так!

\iusr{Александр Шукевич}
Родная Мозаика !

\iusr{Светлана Юшина}
Пано на здании бывшей гостиницы Эрмитаж, улице Богдана Хмельницкого, 26. Киевпроект ниже, на Богдана Хмельницкого, 16-22

\iusr{Ирина Петрова}

так, звісно, він нижче. Але на фотографії можна бачити його верхівку. Як раз
вийшло на панно.

\iusr{Татьяна Данилова}

\ifcmt
  ig https://scontent-frx5-2.xx.fbcdn.net/v/t39.1997-6/s168x128/17634213_1652591098100624_731967241620291584_n.png?_nc_cat=1&ccb=1-5&_nc_sid=ac3552&_nc_ohc=5DRC_VFO4xQAX9D71Py&_nc_ht=scontent-frx5-2.xx&oh=2ca1108f6b97d1d1e1e40c9f25bb7f28&oe=61946877
  @width 0.1
\fi

\iusr{Nina NinaNina}

громадськість зібрала кошти на реставрацію. обличчя панно С. Кириченка
«Українська пісня» на будівлі колишнього готелю «Ермітаж» (садиба Я. Бернера)

\iusr{Ирина Петрова}

Поштовхом для цих дій була активність учасників групи "Киевские истории"! За що
їм щира подяка, уклін!

\iusr{Nina NinaNina}

Так, саме в групі про це читала, і розказую туристам, як кияни бережуть місто!!
