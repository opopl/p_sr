% vim: keymap=russian-jcukenwin
%%beginhead 
 
%%file 25_08_2021.fb.zhuravko_aleksej.1.nezavisimost_kolonia
%%parent 25_08_2021
 
%%url https://www.facebook.com/permalink.php?story_fbid=1540349012976732&id=100010049438147
 
%%author 
%%author_id zhuravko_aleksej
%%author_url 
 
%%tags kolonia,nezalezhnist,ukraina
%%title Украина за тридцать лет своей независимости превратилась в колонию и стала страной третьего мира
 
%%endhead 
 
\subsection{Украина за тридцать лет своей независимости превратилась в колонию и стала страной третьего мира}
\label{sec:25_08_2021.fb.zhuravko_aleksej.1.nezavisimost_kolonia}
 
\Purl{https://www.facebook.com/permalink.php?story_fbid=1540349012976732&id=100010049438147}
\ifcmt
 author_begin
   author_id zhuravko_aleksej
 author_end
\fi

Дорогие друзья! 

За тридцать лет своей независимости Украина стала самой бедной страной
континента, с деиндустриализованной экономикой, разрушенной инфраструктурой,
стагнирующим образованием и полностью деградировавшей социальной сферой.
Основой идеологической надстройки украинского государства стал синтез рыночного
фундаментализма, антикоммунизма и национализма, «густо приправленный
цивилизационными мифами и антипролетарскими предрассудками». 

Демографические потери Украины, где на рубеже девяностых проживало 52 миллиона
граждан, сравнимы с населением далеко не самой маленькой европейской страны — и
давно превысили количество прямых и опосредованных жертв Голодомора. При этом
темпы депопуляции постоянно растут — о чём вслух говорят, как украинские, так и
международные социологи. 

\ifcmt
  pic https://scontent-cdt1-1.xx.fbcdn.net/v/t1.6435-9/239621232_1540348976310069_5246428429670893731_n.jpg?_nc_cat=105&_nc_rgb565=1&ccb=1-5&_nc_sid=8bfeb9&_nc_ohc=mJT3aYe6nPwAX_OeLgd&_nc_ht=scontent-cdt1-1.xx&oh=3d68720df5164548799bd236d3ea86d9&oe=614C9C42
  width 0.4
\fi

Украина стала чем-то «вроде черной дыры континента», в большинстве случаев ее
рассматривают как глубоко периферийную страну, которая поставляет на мировой
рынок трудовых мигрантов, карпатский лес, контрабандные сигареты и нелегально
добытый янтарь, вместе с ручейками утекающего из зоны боевых действий оружия, и
торговлю наркотиками. 

После развала СССР страна развивалась в соответствии с «рыночной реставрацией
девяностых». То есть после распада СССР эксплуатировались общественные
богатства, накопленные в советский период, а наиболее ценными активами
государственной собственности завладели будущие олигархи. При этом беднеющее
государство постепенно сбрасывало с себя непосильный груз социальных гарантий
советского времени. 

Новая экономическая реальность с неизбежностью формировала соответствующую
идеологическую надстройку. Ее основой явился синтез рыночного фундаментализма,
антикоммунизма и национализма, густо приправленный цивилизационными мифами и
антипролетарскими предрассудками. Эта крайне правая по своей сути идеология
обозначила основную линию государственной пропаганды, и вместе с тем стала
настоящей гражданской религией новой украинской интеллигенции — в широком
спектре от откровенных нацистов до прогрессистски настроенных либералов. И она
доказала свою эффективность, обеспечивая политическое господство над страдающим
от «реформ» обществом. 

Уважаемые друзья, погуглите, что построено во времена «советской оккупации» с
1945-го по 1975-й в УССР. Сделайте то же самое за период с 1991-го по 2021-й.
А теперь сравните и подумайте о том, что вы праздновали 24-го августа. На мой
взгляд, нечего, к сожалению. 

Украина за тридцать лет своей независимости превратилась в колонию и стала
страной третьего мира. А может быть даже и хуже. Ломать же не строить. А вот
что останется будущему поколению, об этом, смотрю, на Украине никто не думает.
И это очень страшно и печально.

С уважением, 
Алексей Журавко
