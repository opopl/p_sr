% vim: keymap=russian-jcukenwin
%%beginhead 
 
%%file 14_01_2022.fb.sebastjanovich_pavel.1.sporidnenist.cmt
%%parent 14_01_2022.fb.sebastjanovich_pavel.1.sporidnenist
 
%%url 
 
%%author_id 
%%date 
 
%%tags 
%%title 
 
%%endhead 
\zzSecCmt

\begin{itemize} % {
\iusr{Andrey Kravets}
Работа, семья, ипотека....

\iusr{Queen Jane}
Футбол

\iusr{Dmitry Belous}
Это признак инфантильности нациции

\begin{itemize} % {
\iusr{Александра Лутохина}
\textbf{Dmitry Belous} Именно. Как научила воспитателька в детском саду, так и делать. Вместо воспитательки правительство, всего-то.

\iusr{Павел Гурнік}
\textbf{Dmitry Belous} насамперед ознака інфантильності і патерналізму най-активних майже-громадян.
Інші беруть з них приклад
\end{itemize} % }

\iusr{Олена Пархуня}

Ну, т. е, если коротко... мы все разные, и всегда такие были.. А политика-это
способ найти что-то для объединения людей в группы.. Найти вопросы(и ответы),
которые устроят большинство. И только.. Сегодня (21 век, однако) просто все
быстрее выходит наружу, и очень много информационного д\#рьма Но, с этим надо
учиться жить..(21 век, однако)). Думаю, телевизор, 70 лет назад, тоже вносил
пугающие изменения в инфополе.. прошлое уже..

\iusr{Roman Sozansky}

А це нормально для будь-якого суспільства. Тому що всі дуже різні, культурно,
освітньо, за способом мислення, за особистими якостями і т.і. Я вже вам казав,
вам особисто подам руку, незважаючи на різні погляди в гуманітарній сфері. Про
це я готовий з вами дискутувати скільки завгодно і вважати, що ви погано
розбираєтесь в гуманітарних питаннях. Маю на це право. Але це не заперечує
того, що як людина, ви мені видаєтесь гідним чоловіком і я завжди про це
пам'ятатиму перед усім. А суперечки, дискусії, незгоди - це нормально. Цим і
відрізняється здорове суспільство від хворого. В здоровому організмі кров
кипить і лейкоцити з еритроцитами постійно влаштовують гармидер. А в важко
хворому - штиль і ніхто ні з ким не сперечається. Подивіться на Росію, Північну
Корею, Білорусь. Ви такого \enquote{єднання} хочете? Такого камертону? Тут в рамках
однієї сім'ї, стосунків кількох людей важко найти повне порозуміння, порою, а
він хоче, щоб 40 млн людей думали і дихали в унісон. Дивак-чоловік.


\iusr{Ігор Можайко}
\enquote{Бла-асловения!.. Всем сердцем!.. Нежность испытывал, нежность, понимаешь ты, мужицкая морда?}

\iusr{Володимир Брунько}

ЛОМы не те. Если технологично, по-деловому, как в бизнесе, то нужно искать
ЛОМов (продавцы), покупать их, обучать, сделать все компоненты маркетинга,
отработать бюджет на продвижение \& go!

Покупать не за деньги. Хотя и это нужно. Покупать - это перевербовывать.

\iusr{Святослав Дубина}
Сегодня в 16:30 будем проявлять волю  @igg{fbicon.face.wink.tongue}  Паша, не дрейфь, настроим мы общий камертон

\iusr{Зиновій Бокійчук}

...органічно було би, коли такий тон виробляється в суспільстві, а не
сподіватися на вождів. Джерелом гармонії завжди є стосунки між людьми. Тому я
би поставив питання: Як повернути цим стосункам природну гармонію?

\begin{itemize} % {
\iusr{Олена Пархуня}
\textbf{Зиновій Бокійчук} 

она всегда с людьми.. Не может кто-то забрать гармонию, или вернуть её.. Просто
это не ваша гармония.. Вы хотите только (!)такую, как вам удобно. Мне она зачем?


\iusr{Зиновій Бокійчук}
\textbf{Олена Пархуня} Якщо Вас влаштовує так як є зараз, то вам гармонія дійсно не потрібна. Я допускаю, що знайдуться люди яким подобатися какафонія.

\iusr{Павел Себастьянович}
\textbf{Зиновій Бокійчук} да, камертон должен начать звучать из общества. Может он и звучит, не слышим.
\end{itemize} % }

\iusr{Oleksandr Diubanov}
Ценностных приоритетов для себя никто не выделяет.

\begin{itemize} % {
\iusr{Олена Пархуня}
\textbf{Oleksandr Diubanov} нит. Просто у каждого они свой. Свои, не ваши.. Это и напрягает, вас..

\iusr{Oleksandr Diubanov}
\textbf{Олена Пархуня} поэтому и каждый лишь сам за себя. Но вы ошибаетесь в том что меня это напрягает. Другое слово.
\end{itemize} % }

\iusr{Марьяна Шевцова}

Разделяй и властвуй... именно этим занимаются все правители... они разделили
народ, науськали, натравили запад на восток, объявили войну русскому языку, на
котором говорит почти вся Украина, допустили, изобрели позорную гибридную войну
в Донбассе, ограбили народ налогами тарифами и погнали нищих и обездоленных на
войну защищать свои сверхприбыли ... они выдавили лучших работников в Европу и
хотят теперь их ограбить налогами, они установили себе космические зарплаты за
узаконенное безделье...и этому беспределу наших избранников не видно края...

\iusr{Людмила Халанская}
\textbf{Марьяна Шевцова} Вы забыли добавить: а мы всё это терпим и местами даже поощряем.

\iusr{Олена Пархуня}

Вы умеете слова складывать в стройные конструкции.. Но, сегодня, это уже не
сильно работает.. т. е призывы не работают совсем.. Работают действия,
харизма, деньги... В итоге будет так:

Если просто... конечно(стихи не мои)
Но в тему @igg{fbicon.monkey.see.no.evil} 
"Я почистил карму,
Выхожу в астрал..
Но пришлось вернуться-
Кот опять наср@ лал..

\begin{itemize} % {
\iusr{Микола Чепурний}
\textbf{Олена Пархуня} 

Вам осталось задаться вопросом: а почему народ клюёт на харизмы, деньги (как
пряник) и действия (как кнут) ? Вопрос риторический и ответ очевиден: пряник
\enquote{на халяву}, а кнут - что бы безмозглые не разбежались.

\end{itemize} % }

\iusr{Татьяна Нанкина}
Бедность, одна из причин.

\iusr{Микола Чепурний}
\textbf{Татьяна Нанкина} А причина бедности - глупость ?!

\iusr{Татьяна Нанкина}
\textbf{Микола Чепурний}, так

\iusr{Алена Петренко}

Утеряна духовность, превалирует материальный мир, \enquote{разучились} читать и
думать, гаджеты, вместо помощи, используются для праздной развлекаловки в
основном, ушло живое общение между нами, нас пожирает одиночество и плюс к этому
два года масочной шизофрении, страха.....

\iusr{Наталия Давиденко}

Все нации одинаковые, единства нет нигде. В любом вопросе. Хотя я за то, чтобы
мы были терпимее и добрее друг к другу.

\begin{itemize} % {
\iusr{Микола Чепурний}
\textbf{Наталия Давиденко} Ой ли -одинаковые ?! Есть нации, наученные пророками \enquote{терпеть и быть добрыми взаимно}

\iusr{Людмила Шумлянська}
\textbf{Микола Чепурний} кто эти святые?
\end{itemize} % }

\iusr{Виктор Вовин}

Спорідненість это про родство, наверное, больше, как явное так и возможное.

А вот эмпатии в нашем обществе критично не хватает, ее проявление многие
воспринимают с недоверием или удивлением, в зависимости от собственного уровня
таковой..


\iusr{Natali Korchinskaya}

Меня поражает одно, за 8 лет ни одного движения не было против войны, ни
активистов, ни родителей детей которых потеряли, никого и это страшно!!

\begin{itemize} % {
\iusr{Iren Semivolos}
\textbf{Natali Korchinskaya} до росіян-загарбників звертайтесь. Чи ви хочете, щоб українців просто вбивали, а вони не захищались?!
\end{itemize} % }

\iusr{Ольга Михайлова}

После массированного наступления украинских активистов на права (и обычаи
коммуникации!) русскоязычных, с хамством, хейтингом, отчуждением по типу
\enquote{чемодан, вокзал, Россия}... - сложно представить душевное единение. Мол, мы
добились своего - а теперь не просто смиряйтесь, а радуйтесь и нас любите?

\begin{itemize} % {
\iusr{Павел Себастьянович}
\textbf{Ольга Михайлова} да вроде попустило

\iusr{Павел Гурнік}
\textbf{Ольга Михайлова} 

Права описані в Конституції і відповідних їй законах, мої і ваші. Просто
додержуємся, не скиглимо, не пред'являємо претензій поза законом, любові не
треба. \enquote{Єднання} це не про любов-секс-душу, а про громадянське. Окрім тільки
ступінь виховання і \enquote{комунікації} між тими, де ступінь схожа. Все дуже просто.

\end{itemize} % }

\iusr{Анатолий Денисенко}

Прежде чем отвечать на этот вопрос, давайте себе представим такую
идеалистическую картину. Абсолютно все важные решения, определяющие развитие
общества и экономики в нашей стране на всех уровнях от ВР, ОП, КМ, ГП и до
городского сельского совета, принимаются в соответствии с утверждённой
известной всем методикой и являются оптимальными, т. е., наилучшими, с точки
зрения полезности для общества. Что будет следствием этого? Через полгода мы
будем иметь другую страну.

- Во-первых, не будет необходимости критиковать решения руководства на всех
уровнях (ведь все решения будут правильными); всю энергию можно будет
направлять на выполнение правильных решений наилучшим образом;

- уйдут «собачьи бои» с теле- и радиоканалов, политики будут меньше оскорблять
друг друга; ведь на ток-шоу, в основном, критикуются решения руководства
страны.

- снимется психологическая нагрузка с лиц, принимающих решения, у них появится
больше времени для реальных дел, а не для разработки вариантов и их оценки; по
методике они только утверждают наилучший представленный им вариант решения.
Например, командующий армией сам не разрабатывает план операции (не принимает
решение на операцию), он только утверждает его, после представления ему этого
плана начальником штаба. А сам план разрабатывают специалисты и эксперты по
определённой методике.

У читателей может возникнуть вопрос: а всегда ли можно найти такой вариант
решения, который удовлетворит всех заинтересованных? На основе моего
многолетнего опыта отвечу: да, можно. Нужно только создать систему принятия
оптимальных решений. А это совсем не сложно. Я много лет работал в такой
системе, а затем сам обучал принятию оптимальных решений слушателей Военной
академии.

Таким образом, наши соотечественники будут чувствовать себя жителями единой
мощной страны (это будет следствием), если все важные решения в стране на всех
уровнях будут оптимальными (наиболее правильными и нужными), принимаемыми по
известной всем научно обоснованной методике при участии наилучших в стране
специалистов и экспертов. Тогда весь народ будет такое правительство
поддерживать, качественно выполнять его решения и чувствовать себя единым.


\iusr{Михаил Пластун}
Особисто мені подобається - людяність...

\begin{itemize} % {
\iusr{Константин Соловьев}
\textbf{Михаил Пластун} какая людяність? Примитивов, рабов государства, граждан и т.д.?

\iusr{Михаил Пластун}
\textbf{Константин Соловьев} возлюбите ближнего своего, как самого себя...

\iusr{Константин Соловьев}
\textbf{Михаил Пластун} не агитируйте за любовь, как коммунизм. Я давно получил вакцинацию от дури превращения публичных отношений на основе \enquote{любви} - вообще.
Филеа - да. Но у греков различений любви было много.
Но вакцину я получил из заметок Маркса и Энгельса Циркуляр против Криге
\url{https://fil.wikireading.ru/29561}
\end{itemize} % }

\iusr{Маша Пашина}
Вся сегодняшняя политика построена на разобщении! О какой споридненности можно мечтать?(((

\iusr{Svitlana Salamatova}
Потому что все ждут Камертона, но не желают им становиться)

\iusr{Виктор Позументщиков}
В стране - насилие... в первую очередь экономическое...

\iusr{Наталия Федорова}
Доний пытается)

\iusr{Константин Соловьев}

Потому, что воля ( при нормальных условиях) проявляется на основе
интеллектуальных моментах, что порождает необходимое понимание и логики

На рациональной основе могут быть соучастия в адекватных формах. Отсутствие
рассудочности, ума, разума - исключает дееспособность.


\iusr{Даниил Монин}
Потому что зрада расходится легче  @igg{fbicon.smile} 

\iusr{Микола Чепурний}

Слово \enquote{спорРІДненість} еще есть в словаре, но уже теряется в лексиконе. \enquote{Иваны
не помнящие РОДства} одурманены теми, кому хочется \enquote{загребать жар чужими
руками}, разделять и властвовать. Люди не помнят РОД, не понимают родства
потому, что не формируется научное мировоззрение системой образования и
воспитания. 

1. Нет осознания ЕДИНСТВА у всех людей человеческого мозга, как функционала с
ЕДИНСТвенной целью - ОБЩЕственного формирования и приРОДного отбора самых
продуктивных и самообучаемых экземпляров. 

2. Нет осознания истинного предназначения СЛОВА, как людского способа
достижения поРАЗУМЕНИЯ своего соРОДства. Нет единой терминологии. 

3. Нет шкалы истинных человеческих ценностей и моральных законов. 

4. Нет осознания причины существования \enquote{конфликта интересов}. Разные
интересы и разное видение своих задач=проблем должны быть по законам Природы,
но КОНфликтов разум не должен допускать...по его предназначению. 

5. Нас на уроках литературы не учат находить КОНтекст, находить КОНструктив при
помощи языков, понимать вредность слепого КОНформизма........

И все 6-666 пунктов всех причин разногласий имеют единую причину - мы все
\enquote{инвалиды детства} из-за все еще отсутствующей научной системы
воспитания детей. Подавляющее число учителей не являются Учителями ибо не имеют
просвещенности, а \enquote{работники культуры} не знают что такое
культурология, ибо \enquote{за 30 сребренников} нанимаются на службу власти
заниматься театром, кино, песнями да плясками... \enquote{Все продается и все
покупается!}

\iusr{Oleksandr Kramarenko}

Павел, где была спорідненість, когда шла речь о дельфинах и ультразвуке 2 года
назад? Что-то ее не было тогда видно***((

\iusr{Павел Себастьянович}
\textbf{Oleksandr Kramarenko} там же, где и сейчас

\iusr{Людмила Шумлянська}
спорідненості з ворами при владі? стараємося, а вони не хочуть

\iusr{Татьяна Тимошенко}

Как раз в эти дни видно, как мы не можем определиться, что же нам ближе по
духу-христианские праздники западного обряда или восточного, а может советские
дед мороз и снегурочка, а может языческие святки и маланка, какая уж тут
\enquote{спорідненість}?


\iusr{Катерина Здоровцова}

Хоть и есть стремление нации к лучшему, но Напряжение от того, что постоянно
что-то происходит вызывает, мягко говоря, истощение. Вот, об этом так и говорят

\ifcmt
  ig https://scontent-frt3-2.xx.fbcdn.net/v/t39.30808-6/271935081_5443916065635735_903457071836188841_n.jpg?_nc_cat=103&ccb=1-5&_nc_sid=dbeb18&_nc_ohc=Jvl-y1yG_xkAX9SB4pO&_nc_ht=scontent-frt3-2.xx&oh=00_AT9lUrqKvuseERycIK1EcjpJVXN4TcZHbBOHCwHTmALnRQ&oe=61E8315E
  @width 0.3
\fi

\iusr{Олег Романюк}
\textbf{Pavel Sebastianovich}, 

дивно, а де була ваша \enquote{спорідненність} п\enquote{ять років попередньої влади? що вам
було не так, а зараз вже неможливо? Коли олігархічні ЗМІ та проросійські сили
знищували репутацію п'ятого президента - де була ваша толерантність та
\enquote{спорідненність}, в той час, коли країна з нульовими ЗСУ організовувала
спротив, об'єднувала міжнародну спільноту проти агресора, відновлювала
економіку?

\begin{itemize} % {
\iusr{Павел Себастьянович}
\textbf{Олег Романюк} там же, где и сейчас

\iusr{анна анфимова}
\textbf{Олег Романюк} 

Про яку репутацію 5-го президента йдеться? Ви маєте на увазі репутацію
сивочолого чолов'яги, який здав донецький аеропорт і забув про це повідомити
мужніх кіборгів? А вони мужньо, до останнього подиху боронили той клаптик землі
з вежою? Чи, може, того, хто тиснув руку та обіймав такого собі вбивці на
прізвисько ...ло? Ні, мабуть того, хто приймав парад під час Іловайську? І все
це - репутація? Хіба що слово \enquote{репутація} може бути і з від'ємним значенням.
Нікого ні з ким не порівнюю, тим більше, президентів. Не везе Україні з
власними гавелами, мабудь, ще не народилися.


\iusr{Олег Романюк}
\textbf{Anna Anfimova} , ви до сих пір вірите у ці кацапські брехні? дивно, що ви не згадали про вбитого брата та Ліпецьку фабрику.
\end{itemize} % }

\iusr{Sergey Rusanych}
*звучит 9я Бетховена*

\iusr{Павел Гурнік}
необхідні успішні приклади, а їх нема навіть в малих об'ємах

\iusr{Yartsev Anatoliy}
Большое имущественное расслоение, проблема для кооперации разных слоев населения.

\iusr{Семен Березюк}

70 років гулагів і репресій... це соціальний синдром який цементується
жебрацькою економікою.  але все зміниться. Потрібно робити революцію,
соціально-економічну революцію, а фундаментом для неї стане демарж.



\end{itemize} % }
