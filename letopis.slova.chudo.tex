% vim: keymap=russian-jcukenwin
%%beginhead 
 
%%file slova.chudo
%%parent slova
 
%%url 
 
%%author 
%%author_id 
%%author_url 
 
%%tags 
%%title 
 
%%endhead 
\chapter{Чудо}
\label{sec:slova.chudo}

%%%cit
%%%cit_head
%%%cit_pic
%%%cit_text
І при цьому Зеленський так \emph{начудив} і так сильно не довіряє своєму оточенню, що
одночасно хоче і боїться йти на дострокові перевибори президента. А це останній
шанс для нього переобратися. Адже далі ситуація буде лише погіршуватися, а
значить у виборця буде танути бажання його переобрати. Але шостий президент не
хоче відповідати за все, що \emph{начудив}, і не хоче з'їжджати з держдачі. Він уже
звик. І царювати звик. І звик уже не виступати перед можновладцями в лазні і на
корпоративах, а вершити їхні долі без закону, але за допомогою РНБО. Але все
одно Зеленський боїться йти на дострокові перевибори і залишати Україну на 90
днів на Разумкова. Адже їхні стосунки сьогодні, м'яко кажучи, напружені. І мало
що спікер, який на період виборів стане в.о. президента, може зробити. Та й те,
що всі \enquote{слуги} Зеленського швидко присягнуть Разумкову, Володимир Олександрович
розуміє. Ось і боїться, що замість переобрання на другий термін може отримати
перший. За рішенням суду. Адже начудив.  І дуже скоро у влади не буде бажання
відповідати словами видатного лідера сучасності, за версією Єрмака і Мендель, –
\enquote{какая разница?}. Тому що \emph{начудили}. По-поганому
%%%cit_comment
%%%cit_title
\citTitle{Шостий президент так начудив, що наздоганятимуть і його, і його свиту}, 
Бopиcлaв Бepeзa, gazeta.ua, 24.06.2021
%%%endcit

