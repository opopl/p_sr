% vim: keymap=russian-jcukenwin
%%beginhead 
 
%%file 24_12_2021.fb.fb_group.story_kiev_ua.1.imperatrica_maria_fedorovna.cmt
%%parent 24_12_2021.fb.fb_group.story_kiev_ua.1.imperatrica_maria_fedorovna
 
%%url 
 
%%author_id 
%%date 
 
%%tags 
%%title 
 
%%endhead 
\zzSecCmt

\begin{itemize} % {
\iusr{Машинець Валентина}
Дуже цікаво

\iusr{Людмила Тесленко}

Интересная история, несущая в себе смыслы.

А отречение от русскости - это крючкотворство и занудство
комментаторов, прячущихся в пустых страничках своих историй.

\begin{itemize} % {
\iusr{Юрий Стебельский}
\textbf{Людмила Тесленко} наоборот, отречение от своих корней - имхо это очень интересное историческое явление. Нечасто такое можно наблюдать.

\begin{itemize} % {
\iusr{Людмила Тесленко}
\textbf{Юрий Стебельский} При пересадке растений иногда просто необходимо подрезать подгнившие корешки для укрепления и их дальнейшего здорового роста .
Не отрекаются любя.

\iusr{Taras Manolov}
\textbf{Yuriy Stebelskiy} будьте послiдовним. Панi Людмила Тесленко має вiдверто українське прiзвище; мусите тепер наполягати, щоб вона писала українською. Виправляйтесь скорiше  @igg{fbicon.smile} 

\iusr{Юрий Стебельский}
\textbf{Taras Manolov} а господин Мустафа Найем на каком языке должен писать? Он украинец или кто? А Юрий Ехануров? А Жан Беленюк? Ему на каком положено писать? Вроде бы чистокровный украинец, судя по фамилии. Не говоря уже о месье Зеленском.
Им как быть на Руси?

\iusr{Taras Manolov}
\textbf{Yuriy Stebelskiy} так це ви з собою сперечаєтесь! Це ж вас обходить хто якою мовою розмовляє, не мене! Це ви примушуєте людей вживати мову, яку самі вважаєте їм за доречну, це не до мене!

\iusr{Taras Manolov}
\textbf{Yuriy Stebelskiy} Русь, до речі, якщо ви не в курсі, це і є Україна, її серце. Нам просто прийшлося взяти собі назву Україна, щоб нас з москалями не плутали. Хочете називатися росіянами, хто ж вам заборонить; але історія ніде не ділася. Саме українська - це питома руська мова, мова русинів-українців, не зважаючи що там на Московії кажуть.
\end{itemize} % }

\iusr{Ludmila Karpuk}
\textbf{Людмила Тесленко} 

русская история искупана вся в крови других народов и инакомыслящих, а далее, и
своего царя, его детей, а потом плачут - подавай им, дескать царя, ибо без него
невмоготу.. это ведь все в лучших русских традициях - подло убивать безоружных
людей, отвозить в вечную мерзлоту в Сибирь, строить концлагеря ГУЛАГи, и прочие
адские изобретения упыря сталина

\begin{itemize} % {
\iusr{Людмила Тесленко}
\textbf{Ludmila Karpuk} 

Согласна с вами. Но у меня Дед, пропавший без вести в годы ВОВ, которого я никогда
не видела, был русским.

И я от него никогда не отрекусь, как от одного из своих корней.

И это-совсем другая история.


\iusr{Willy Poindexter}
\textbf{Ludmila Karpuk} что за традиции ? Кого это в Империи убивали до прихода новых вандалов - большевиков? И не путайте жертву и убийцу - Россию и ссср .

\iusr{Саша Неу}
\textbf{Ludmila Karpuk} Так так везде... вся история Европы полита кровью и колоний....

\iusr{Саша Неу}
\textbf{Ludmila Karpuk} 

вы про Конго и их колонистов Бельгию почитайте.... про Бенгалию и их колонистов
англичан почитайте... так для общего развития ..на Солженицыне и Суворове
летиратура не заканчивается.


\iusr{Ludmila Karpuk}
\textbf{Willy Poindexter} согласна, но ведь это те же люди, которые населяли Россию ещё за царя
\end{itemize} % }

\iusr{Sergiy Pohvata}
\textbf{Людмила Тесленко} отречение от чего простите?))

\begin{itemize} % {
\iusr{Людмила Тесленко}
\textbf{Sergiy Pohvata} якобы от русскости.
А стоит отречься от снобизма, подлости, предательства, подпитывающих это.

\iusr{Sergiy Pohvata}
\textbf{Людмила Тесленко} как всё сложно у вас тут)

\iusr{Людмила Тесленко}
\textbf{Sergiy Pohvata} Внутри человека все сложно.
\end{itemize} % }

\iusr{Олег Курилов}
\textbf{Taras Manolov} 

по поводу Руси, согласен с вам, по поводу: \enquote{прийшлося взяти собі назву Україна,
щоб нас з москалями не плутали} полный бред.... То что украинский, немного
ближе к Древнерусскому - бесспорно. Потому что это язык селян, который меньше
подвергся изменениям. Кстати, у учёных идёт спор какой язык ближе к
Древнерусскому : украинский или беларусский. Это два близкородственных языка с
единым корнем. Кто такие русины- украинцы, незнаю. Наверное вы имеете ввиду
жителей Закарпатья... повыдумывают новых терминов..... Малоросы! вот правильное
название  @igg{fbicon.smile}. И выдумывать ничего не надо!

\end{itemize} % }

\iusr{Ирине Вильчинская}

моя прабабця розповідала, що часто бачила Марію Федорівну біля Олександрівської
лікарні. Вона, з кимось із дівчат, вдягнені у форму сестер милосердя, йшли до
лікарні у супроводі солдата, що ніс кошик з гостинцями, фруктами. Вони
опікувалися декількома койками у цій благодійній лікарні і постійно
навідувалися туди. І це було дуже зворушливо. Кияни оцінили.

\begin{itemize} % {
\iusr{Ирине Вильчинская}

Я б не наголошувала тут на національній темі (зараз в мене капці полетять!). Я
про зразки соціальної поведінки з боку пануючого класу (соціальних еліт), які
мають задавати моральний тон суспільству своєю поведінкою. Шляхетність у
вчинках (нехай навіть показна)- це приклад для наслідування іншими прошарками
суспільства. Вона формує колективний менталітет суспільства та демонструє
моделі соціальної поведінки \enquote{гомо моралис}. Порівняння з нашим сьогоденням та
соціальною поведінкою наших сьогодняшніх (умовних) \enquote{еліт} - не на їхню користь.
Щось не проглядаються серед волонтерської спільноти масової присутності
представників т. з. еліт. За виключенням одиниць. Тільки й усього. А з боку
історичності - це зафіксований факт. Деякі з Романових (далеко не усі!)
відчували прихильність до Києва. Згадайте Єлизавету, завдяки якій ми й досі
милуємося Андріївською церквою та Маріїнським палацем, засновницю Покровського
\enquote{княжого монастиря} Олександру Петрівну, не говорячи вже про Марію Федорівну.

\end{itemize} % }

\iusr{Oleg Berezhinskiy}

Цікаво, скільки людей того часу закидали принцесі Марії Дагмар \enquote{отричение от
датскости}, бо вона ж почала розмовляти російською, навіть Федірівною стала  @igg{fbicon.smile} 

\begin{itemize} % {
\iusr{Helen Bela}
\textbf{Oleg Berezhinskiy},

становясь женой русского императора, невеста( а они все были импортных
королевских кровей;)) принимала православие, а традиционной иконой-
покровительницей являлась Феодоровская икона Божией матери. Поэтому все жены
русских царей становились Фёдоровнами, традиционно. И никто никому ничего не
\enquote{закидав}. И разговаривать она могла в семье хоть на французском, хоть
на родном, но, как русская императрица, естественно, выучивала русский, опять-
таки, традиционно.

\iusr{Oleg Berezhinskiy}
\textbf{Helen Bela} "Она по-русски плохо знала,
Журналов наших не читала,
И выражалася с трудом
На языке своём родном..."  @igg{fbicon.smile} 
\end{itemize} % }

\iusr{Тина Шевченко}

Каждый раз, когда заходит тема Романовых, не устаю поражаться тому, как
специально запущенная брехня и ложь вьелись накрепко во все, що касается имени
и памяти Григория Распутина, а правильнее Григория Нового!!! То, что Мария
Федоровна поддалась этой лжи и клевете, тоже внесло свою лепту в гибель ее
сына, внучек, внука и невестки.. Уверена, ее душа уже давно это поняла...

\begin{itemize} % {
\iusr{Игорь Сирадчук}
\textbf{Тина Шевченко} Ви би виклали другу історію про Григорія, бо ніде немає. Цікаво.

\begin{itemize} % {
\iusr{Тина Шевченко}
\textbf{Игорь Сирадчук} 

Якщо по справжньому цікаво, то особисто Вам дам правдиву інфу.. А окремо для
\enquote{диванних істориків} викладати не буду... Не дозріли..


\iusr{Игорь Сирадчук}
\textbf{Тина Шевченко} Будь ласка, дякую.

\iusr{Александр Лепетуха}
\textbf{Тина Шевченко} Куда нам)))

\iusr{Тина Шевченко}
\textbf{Игорь Сирадчук} Дуже рада, що хоч одна людина в цій группі хоче знати правду..

\iusr{Тина Шевченко}
\textbf{Игорь Сирадчук} 

Я написала!!! Тут, в комментах, в кінці!!! Хто хоче дійсно знати правду, той в
любих умовах вийде на неї, як я.. А Вам ще раз дякую!!! Ви - справжній!!!

\end{itemize} % }

\iusr{Вера Тоцкая Усикова}
Его фамилия Новых!

\iusr{Тина Шевченко}
\textbf{Вера Тоцкая Усикова} Умница!!!! Да!!!!

\iusr{Тина Шевченко}
\textbf{Игорь Сирадчук} 

Добре, викладаю для тих, хто дозріли!!!! В Ютубі серія історично-документальних
фільмів \enquote{Григорий Новый. Мученик за Христа и Царя}. Це сама найкраща інформація,
вся вкупі!!! Я дізнавалась цю правду поступово, коли жила 10 років в Пітері і
служила там в лікарнях сестрою милосердя/волонтером/.. Самі россіяни цього не
знають, бо мозок їм забабахали віковою брехнею!!!!!

\begin{itemize} % {
\iusr{Диана Бикулова}
\textbf{Тина Шевченко} , - 

прямо так и никто.... Нет ни в его фамилии,ни в его истории ничего
нового.. разве что для Вас. А считатт его святым - это дело веры и отсутствия
логики и знаний о том времени и людях. И фильм наверняка заказной, скорее всего
Церьковь постаралась. Распутин полностью творения Синода.


\iusr{Светлана Красовская}
\textbf{Тина Шевченко} 

як можна дивитись таку муть. Прочитайте Радзінського \enquote{Распутін}. Там
викладені історичні факти.

\iusr{Тина Шевченко}
\textbf{Светлана Красовская} 

Щассс.. Радзінський авторитет для тих, хто зомбований.. Я не начитана, а
обізнана.. Підіть, послужіть/без грошей/10-15 років в пітерських лікарнях, а
потім поспілкуємось!!! Згода?! Щось мені підказує, що ви на це не підете.. Вас,
тих хто не має поняття ні в чому, тонна, але це немає ніякого значення для
історичної Правди!!!

\end{itemize} % }

\iusr{Юрий Крючков}
\textbf{Тина Шевченко} Распутин был талантливый аферист, большой развратник и умелый манипулятор.

\begin{itemize} % {
\iusr{Тина Шевченко}
\textbf{Юрий Крючков} 

Это вам доставило радио ОБС/Одна Баба Сказала/. Сидите со своим бабьим
\enquote{мнением} на диване, чего вы сюда лезете?! Я пишу не для таких, как вы.. Вы
читать-то вообще, сээээээр, умеете????!!!!???

\iusr{Юрий Крючков}
\textbf{Тина Шевченко}  @igg{fbicon.face.yawning} 

\iusr{Тина Шевченко}
\textbf{Юрий Крючков} Ну да, радио ОБС вам это хорошо вложило в ушки..
\end{itemize} % }

\end{itemize} % }

\iusr{Светлана Дубински}

\ifcmt
  ig https://i2.paste.pics/1ba0d4aab33ac4f0edfa94b32756041d.png
  @width 0.2
\fi

\iusr{Olha Lychova}
Дякую за цікаву розповідь!!!

\iusr{Ирине Вильчинская}

и снова сеча! казалось бы - просто исторический факт. Ну любили многие Киев, он
никого не оставлял равнодушным. В том числе и представители царской фамилии. И,
заметьте, они сюда переезжали, сбегая от своих \enquote{семейных очагов}. По разным
причинам - Мария Федоровна с невесткой не ладила (по слухам), Александра
Петровна - от несчастливой женской судьбы, Елизавета Петровна - ну полюбила она
украинца и через него прочувствовала душу Города... Даже венчаться здесь хотела.
Ведь и цари и царицы - тоже люди и им не чужды свои симпатии и антипатии. И все
они, оседая здесь, привнесли что-то хорошее - церкви и монастыри, которые до
сих пор ласкают глаз и греют душу, достойное поведение, \enquote{запавшее} в памяти у
горожан. Давайте сохранять и собирать то доброе и хорошее, что Город накопил в
себе за свою долгую и непростую историю!

\begin{itemize} % {
\iusr{Regina Kostetska}
\textbf{Irina Vilchinskaya} Спасибо Вам за эти слова! Я думаю, что очень многие с Вами согласны!
\end{itemize} % }

\iusr{Willy Poindexter}

Принцесса Дагмар сначала была невестой Наследника Николая Александровича,
старшего сына императора Александра Освободителя ( см фото). Однако Цесаревич
скоропостижно скончался от менингита и уже на смертном одре в Ницце соединил
руки своей невесты и брата Александра. Так принцесса Дагмар стала невестой
будущего императора. Это было в 1865. А уже в 1868 она благополучно
разрешилась от бремени сыном Николаем.

\ifcmt
  ig https://scontent-frt3-1.xx.fbcdn.net/v/t39.30808-6/269885568_1080244802712785_3293261419248073491_n.jpg?_nc_cat=106&ccb=1-5&_nc_sid=dbeb18&_nc_ohc=Y0vyhBrbJVQAX94uqPg&_nc_ht=scontent-frt3-1.xx&oh=00_AT9zXRtBeE6JJ553qFlJkANR-lYq9DL46AVTICVrPVpaBg&oe=61CF4EB0
  @width 0.3
\fi

\iusr{Вадим Горбов}

Киевляне всегда были очень переменчивы в своих симпатиях и предпочтениях в 1917
году. Впрочем и сейчас. Они «тепло вітали» вдовствующую императрицу, они носили
на руках Керенского по Городу летом 1917, встречая его цветами и оркестром на
вокзале...

\iusr{Татьяна Оржеховская}
Благодарю за интересные исторические факты

\iusr{Юрий Панчук}

Цікаво, що Марія Федорівна займалася гімнастикою. Мене завжди дивувало, що вона
навіть в похилому віці виглядала як молода жінка (десь читав, що навіть
пластику собі робила, здається у спогадах Фелікса Юсупова). Про Олександра
Третього є думка, що фактично це він намітив загибель династії Романових. Після
вбивства свого батька, Олександр ІІІ почав затягувати гайки і практично знищив
ті свободи, які започаткував Олександр ІІ. Микола ІІ продовжував лінію батька,
не хотів ділитися владою і статками ні з ким і довів свою родину і все
дворянство до загибелі. Кажуть, що теперішні міліардери і близько не дотягують
до багатств Миколи ІІ. В РІ навіть спеціальна держслужба була, яка доглядала
нерухомість царя (зараз в будинку її Київського філіалу МОЗ).

\begin{itemize} % {
\iusr{Галина Полякова}
\textbf{Юрий Панчук} З іншого боку, є думка, цілком грунтовна, що саме за царювання Олександра Третього Росія розбудовувала свою економіку і сягнула дуже пристойного рівня.
\end{itemize} % }

\iusr{Наталя Кудря}

ніби ж недавно був пост адміна про недопустимість ворожнечі - а тут юрій
стебельський з своїм українофобством... зіпсував такий цікавий пост як жовчю
плюнув

\begin{itemize} % {
\iusr{Юрий Стебельский}
\textbf{Наталя Кудря} 

не надо приписывать. Я люблю Украину. И если уточнять, то люблю сильнее очень
многих.  Кто по настоящему любит Украину, тот любит и русскую Украину, и
галицийскую Украину, и русинскую Украину, и татарскую Украину.  Народ Украины
интернационален и многонационален. Потому следует любить все нации, живущие
века на Украине.

\begin{itemize} % {
\iusr{Наталя Кудря}
\textbf{Yuriy Stebelskiy} 

я ні слова про інші нації не написала!! хотіла перечитати ваш комент- немає -
мабуть таки порушує правила групи! а хто найдужче \enquote{любив} Україну з росіян- так
це ленін!! він так і написав (на пам'ятнику було теж) -що без України росія не
можлива - ну зараз ситуація якраз в тему... да - і про віки там в тій цитаті теж
щось було  @igg{fbicon.wink} 


\iusr{Людмила Адаменко}
\textbf{Юрий Стебельский} удушение в объятиях «любви»!
\end{itemize} % }

\end{itemize} % }

\iusr{Татьяна Курченко}
Це пост про Київ і кохання! Дякую.

\iusr{Надежда Калинская}
Цікаво,дякую!

\iusr{Татьяна Трофименко}
Цікаво !! Але ще б цікавіше були історіїї героїв України.. Дякую

\begin{itemize} % {
\iusr{Ирине Вильчинская}
\textbf{Татьяна Трофименко} 

Як на мене - велика історія завжди складається з \enquote{пазликів} маленьких історій.
А \enquote{ступінь} героїзму - ну хто ж її виміряє? Я вважаю, що і героїзм - він
різний, різнобічний. Хтось захищає нас від ворога на фронті, хтось спасає наших
вояків на фронті (кожен раз збираючи та везучи туди зібране по крихтам -
обладнання, смаколики), хтось - підтримує у шпиталях ( ці люди не дуже
піаряться, лише, коли потрібна допомога хлопцям), хтось - підгодовує
безхатченків, хтось вбиває своє здоров я, відбиваючи залишки зелених зон від
скажених забудовників, хтось, б ється з бюрократами за спасіння історичних пам
яток. А хтось - рятує життя чотирилапих, хвостатих, пернатих... Я вже більше 10
років бачу стареньку жіночку, дуже інтелігентного вигляду, яка із дня у день,
двічі йде з пакуночком і підгодовує котячих у десятку будинків. У будь -яку
погоду. Коли не бачу декілька днів, розумію, що вона хворіє... якби хтось сказав
їй, що це - героїзм, вона б не зрозуміла! Я була вражена існуванням
\enquote{мальтійської кухні}, яка майже рік після Майдану годувала на Подолі усіх
нужденних. Хлопці й дівчата кожний день готували їжу, годували, лікували. Кияни
з усіх куточків зносили продукти, ліки, речі. Якими мірками виміряти це? Кияни
завжди були чуйними до людської біди і швидко відгукувалися. Справжній героїзм
- він без \enquote{панфар} і голосних лозунгів. Але він - завжди поряд з нами. Бо
справжній героїзм -він просто з людським обличчям і чистим сумлінням та
співчуттям.

\begin{itemize} % {
\iusr{Нина Светличная}
\textbf{Ирине Вильчинская} 

Мальтійська кухня почалась саме на Майдані. Стояли біля Поштамту, годували,
обігрівали. А потім на Подолі не просто допомогали одягом, посудом, навіть
будматеріалами для біженців. Для дітей влаштовували свята, курси з вивчення
мови.

\ifcmt
  ig https://scontent-frt3-1.xx.fbcdn.net/v/t39.30808-6/269912280_3027626947490094_4013780535486858539_n.jpg?_nc_cat=107&ccb=1-5&_nc_sid=dbeb18&_nc_ohc=zoFKjUqmuUUAX9mhFTC&_nc_ht=scontent-frt3-1.xx&oh=00_AT-4qKpc2s5J-17Ex0sCkntvg_NmXOaSvJWbGlsuUIe80A&oe=61CEFC5C
  @width 0.4
\fi

\iusr{Ирине Вильчинская}
\textbf{Світлична Ніна} 

дякую за доповнення. Я їх вперше побачила саме на Подолі. Сама декілька разів
відносила продукти і бачила, як від метро люди йшли просто чередою - несли
усе, чим могли поділитися. Одна жіночка похилого віку тягла клумак із теплим
одягом завбільшки її самої...

\end{itemize} % }

\iusr{Валентина Прибутько}
\textbf{Татьяна Трофименко} от і напишіть про героїв України, а ми почитаємо

\end{itemize} % }

\iusr{Анна Сидоренко}
Спасибо, очень интересная история.

\iusr{Ирина Третьякова}
Благодарю автора! Очень интересно!

\iusr{Надія Чуй}
Дякую! Історія кохання на все життя!

\iusr{Ирина Скапцова}

Это очень интересно. Мы должны знать все о нашем прошлом. Какая разница о ком? Мы
все гости здесь. И кому какая участь предрешена, мы не знаем. Будем ли мы в
дальнейшем так беспокоиться о тех, кто сильно нуждается в нашей помощи.


\iusr{Ирина Скапцова}
А очень бы хотелось, чтобы о нас побеспокоились. Нашем будущем!!!??????

\iusr{Kaurkovska Veronika}

Тепер розумію, чому Маріїнський палац називають \enquote{царським}. Він же досить
невеликий, в порівнянні із іншими.

\begin{itemize} % {
\iusr{Ирине Вильчинская}
\textbf{Kaurkovska Veronika} 

Єлизавета приїздила до Києва на прощу до Лаври. Їй настільки сподобався Київ,
що вона забажала вінчатися із Розумовським саме тут. Під цю подію збудували
Андріівську церкву - також, \enquote{царську} - іграшку! Вона ж зовсім невеличка
всередині і навіть у ті часи її не використовували постійно для служб. Вона й
була \enquote{іграшковою}. Але вінчання тут так і не відбулося. Та й Маріїнський палац
був збудований під резиденцію царських особ, якщо вони будуть і надалі
відвідувати Київ... Тому вони з Андріівською церквою в одному стилі і у
кольоровій гамі...

\end{itemize} % }

\iusr{Елена Семененко}
Расказивала Елизавета Сергеевна у нас била библия
И её тоже конфисковали при
Другой власти при таких же обстоятельствах

\iusr{Tetyana Dmytriyeva}
Спасибо большое за публикацию. Замечательная женщина. Очень приятно знать, что ей было хорошо в Киеве, Крыму. Спасибо!

\iusr{Willy Poindexter}

Вдовствующая императрица Мария Фёдоровна и её камер-казак Тимофей Ящик. Копенгаген, 1924 год
За четыре года до смерти. Тимофей дожил в Дании до 50-х, женится. Есть потомки. Фото ниже, извините

\iusr{Willy Poindexter}

\ifcmt
  ig https://scontent-frx5-2.xx.fbcdn.net/v/t39.30808-6/269908938_1080445982692667_3866078358562015894_n.jpg?_nc_cat=109&ccb=1-5&_nc_sid=dbeb18&_nc_ohc=LE3Er3Dh_88AX87ZRtM&_nc_ht=scontent-frx5-2.xx&oh=00_AT-zJtqX6UnUwNPOR7mntbU_ct62aB_NaQaMHC1xj3l5Dg&oe=61D0B1ED
  @width 0.3
\fi

\begin{itemize} % {
\iusr{Галина Полякова}
\textbf{Willy Poindexter} Замечательно! Спасибо!
\end{itemize} % }


\iusr{Лариса Мысник}
Дуже цікаво!

\iusr{Любовь Белоцерковец}
Цікава історія. Дякую!

\ifcmt
  ig https://scontent-mxp1-1.xx.fbcdn.net/v/t39.1997-6/s168x128/16781161_1341101952618574_7704631035023065088_n.png?_nc_cat=1&ccb=1-5&_nc_sid=ac3552&_nc_ohc=G4LwlFdwHK8AX-U2TW3&_nc_ht=scontent-mxp1-1.xx&oh=00_AT_3QYFory6r5iCWKY0pQ1TXAp0bKE8Uy7U_WM9CHoDIjQ&oe=61CFBFEB
  @width 0.1
\fi

\iusr{Елена Мирошниченко}

Бабушка рассказывала про визит Марии Федоровны в Фундуклеевскую (Мариинскую)
гимназию в 1916 году. Неделю готовились, реверансы репетировали, ну а видели в
основном юбки фрейлин. Мария Федоровна действительно очень маленького роста
была...

\begin{itemize} % {
\iusr{Неда Окопова}
\textbf{Елена Мирошниченко} Моя тоже!

\iusr{Елена Мирошниченко}
\textbf{Неда Окопова}, училась в гимназии в это время или маленького роста была?
\end{itemize} % }

\iusr{Нина Бондаренко}
На одному подиху прочитала вашу оповідь. Дякую. Дуже цікаво.

\iusr{Анатолий Викторович}
А чим похвалиться нинішній російський \enquote{імператор}?
Тільки кров'ю...

\begin{itemize} % {
\iusr{Лидия Щепкина}
\textbf{Анатолий Викторович} Ви влучно пiдмiтили...

\iusr{Оксана Євсєєнко}
\textbf{Анатолий Викторович} 

Якщо у воюючій країні 73\% виборців захотіли на посаді \enquote{Імператора} бачити
людину, яка навіть власну чоловічу гідність публічно зневажала заради нових
гонорарів (грошей, тобто продавав (гра на роялі без штанів), та 4-рази
ухилялась від повісток з воєнкомату - то які запитання можуть бути до т.зв
нинішнього, бо він і правда нічого не обіцяв. А запропонувати суспільству щось
високодуховне розбещена людина апріорі не здатна.

\begin{itemize} % {
\iusr{Zyama Syu}
\textbf{Оксана Євсєєнко} 

Не за того імператора йшла мова, ви помилилися.. А нашого... не треба поливати
брудом серед читачів з інших країн, ми якось самі розберемось. Дякую за
розуміння.

\iusr{Людмила Адаменко}
\textbf{Оксана Євсєєнко} секта порохоботів!

\iusr{Ирина Петрова}
\textbf{Ольга Туниковская} хамство в групі неприпустимо  @igg{fbicon.face.unamused} 
\end{itemize} % }

\end{itemize} % }

\iusr{Arcadia Olimi}

Спасибо!!!
Очень интересно!!! @igg{fbicon.sun}

\iusr{Галина Местечкина}
Життя, прожите недарма!

\iusr{Марина Плахотнюк}
Спасибо, очень интересно

\iusr{Ljudmila Luist}

В Хельсинки от главной улицы города отходит Dagmarinkatu(katu-улица), в честь
датской принцессы и супруги русского монарха.

\begin{itemize} % {
\iusr{Ирине Вильчинская}
\textbf{Ljudmila Luist} 

%\index[rus]{Мудрый, Ярослав! Князь Киевский}

У нас многое из нашей истории изъято, забыто... В честь дочери Ярослава Мудрого
Олисавы, которую привычнее сегодня называть Елизаветой, а у викингов, королевой
которых она стала, ее звали королевой Эллисив (ELLISIV DRONNING ), имя которой
-Золотоволоски из Гардарики - вписано в Висы радости, сложенные ее будущим
супругом Гаральдом Суровым, названы улицы в нескольких норвежских городах
-Осло, Бергене, Тронгейме, Ставангере. А в Киеве такой улицы НЕТ! Как нет улиц
с именами ее сестер, тоже европейских королев - Анны и Анастасии. История - она
остается не только в памяти человеческой, она \enquote{цепляется} и вписывается в
названия городов, улиц, местностей, урочищ, закрепляясь и сохраняясь для
будущих поколений!

\begin{itemize} % {
\iusr{Олег Курилов}
\textbf{Ирине Вильчинская} 

а зачем называть улицы Киева в честь каких то дочерей Ярослава Мудрого -
заграничных королев? Они у них были королевами, пусть и называют там улицы.
Если им нечего делать. Что такого выдающегося они сделали? Давайте всех дочерей
всех князей вспоминать. Чего уж мелочиться? Чем таким вышеупомянутые дамы лучше
других? А те дочери которые вышли замуж за Половецких ханов их будем вспоминать
или они не \enquote{правильные} не за тех вышли замуж? :)))

\iusr{Ирине Вильчинская}
\textbf{Олег Курилов} 

у каждого свои резоны и приоритеты. Кому-то дороги полосатые ленточки, ставшие
символом страны-агрессора, кому-то странички отечественной истории, \enquote{вырванные}
из исторического контекста. Что касается дочерей Ярослава, то, если бы они
были просто \enquote{заграничными королевами}, коих было множество, вряд ли зарубежные
хроники и саги сохранили о них столько информации. О том, что они выдающегося
сделали - поинтересуйтесь. А мы свою историю, вероятно, стараниями таких же
мизогинистов - \enquote{историкописцев}, теперь вынуждены собирать по пазликам из тех же
европейских хроник...

\iusr{Галина Полякова}
\textbf{Олег Курилов} Вы имеете в виду Армию Власова? У них ведь была эта ленточка!

\iusr{Олег Курилов}
\textbf{Галина Полякова} 

Что значит и у них была такая ленточка? Это было у них флагом? Как элемент
формы? Ну так у всех национальных дивизий СС в том числе и украинской 14
дивизии, и в подразделениях \enquote{Хиви} в опозновательных знаках и шевронах были
национальные флаги. Французские, Норвежские, Литовские, Эстонские, Армянские,
Азербаджанские и т.д. И что? Предлагаете отказаться от нашего украинского
флага? Извините, очень дешёвый аргумент! Что бы опорочить георгиевскую ленту.
Кстати, а вы знаете что все офицеры УНР (не только Украинской Держави, а именно
УНР) даже уже будучи в эммиграции, с гордостью носили ордена Георгия у кого они
были  @igg{fbicon.wink}. Прикольно смотреть, Железный крест за зимний поход и Георгиевский
крест рядом. А орден Симона Петлюры почему то не хотели носить..  @igg{fbicon.smile} 
(Омельянович-Павленко)

\iusr{Ирине Вильчинская}
\textbf{Олег Курилов} 

Королева Анна была Анной КИЕВСКОЙ, что обозначено даже на постаменте ее
памятника во Франции. Хорошо, что Вы узнали, наконец, о ней. А знаете ли, что в
отличие от своего венценосного супруга, она была грамотной и подписывала
документы своим именем, а не крестиком. Кстати, именно благодаря ее
просвещенности, французы узнали имя Филлип, так назвала она своего сына и это
имя стало \enquote{королевским}. 

Она основала монастырь и в аббатстве до сих пор о ней хранится добрая память,
ежегодно там отмечают день ее памяти. Елизавета стала
\enquote{прародительницей} многих европейских королевских фамилий через свою
дочь и ее потомков. Анастасия, королева Венгрии, основала множество монастырей,
играла не последнюю роль в политической жизни не только Венгрии, но и Европы,
помогая оградить страну от влияния Рима. 

Вообще, все три дочери Ярослава были не просто королевскими женами, а
личностями и влиятельными политическими фигурами при своих мужьях в силу
просвещенности и ума. Жаль, что Вы этого не знали. А что до переходов на
личности, то я высказала свою оценку ваших взглядов, исходя из публичной
информации и вашего же поста. И, да, мне неприятна вражеская символика в
контексте сегодняшних событий, она несколько изменила свою семантику и
воспринимается иначе, чем Вы это преподносите. Я тоже уважаю память своих обоих
дедов, погибших в той войне, свекра, прошедшего всю войну, отца, успевшего
пошагать по военным дорогам, но я не смешиваю в одно - ту и эту историю. 

И есть еще одна историческая личность, окутанная тайной. Агата. Жена
английского короля Эдуарда. Известно, что женился он в Киеве, но кем была она -
дочерью ли Ярослава или его сестрой -историки до сих пор спорят. Но о святой
Агате помнят в Туманном Альбионе. 

Есть дуб, где вела свою жизнь отшельницы святая Агата, исцеляла, помогала людям
и место это, и память о ней хранятся. А вот в наших летописях о женщинах этих
информации нет. И это еще раз говорит о том, что монахи-летописцы просто
\enquote{вычеркивали} целые страницы истории, расписывая доблестные достижения
лишь мужчин.

\iusr{Ljudmila Luist}
\textbf{Irina Vilchinskaya} 

Спасибо Вам за такие просвящающие ответы. Читаю с упоением, но в силу высокого
возраста память слабеет, да ещё Covid перенесенный добивает. Моим хобби всегда
было чтение, обожала все историческое. С уважением к Вашим познаниям и что
делитесь ими

\ifcmt
  ig https://i2.paste.pics/fe9b6fdf55b23c2029883754e8f96150.png
  @width 0.2
\fi

\iusr{Олег Курилов}
\textbf{Ирине Вильчинская} 

\enquote{была Анной КИЕВСКОЙ, что обозначено даже на постаменте ее памятника во
Франции.}  @igg{fbicon.face.tears.of.joy}  в летописях она значится как -
русская от слова Русь. Для вас это принципиально важно? Не вижу разницы.
Хотя... Есть! Государство и земля называлась именно Русью. По этому и Анна была
Руской, как Бланка именно Кастильской, Изабелла - Баварской, Клеменция -
Венгерской и т. д. Названием города никто не именовал, слишком мелко. Ну а на
памятнике, современная коньюктура, не болеее. 

\enquote{А знаете ли, что в отличие от своего венценосного
супруга, она была грамотной и подписывала документы своим именем, а не
крестиком.} Думаю это написано в каждом школьном учебнике, и это очень давно
известно и что? Все дети Ярослава получили хорошее образование. И? \enquote{Кстати,
именно благодаря ее просвещенности, французы узнали имя Филипп, так назвала она
своего сына и это имя стало \enquote{королевским}} Отлично, просто это имя не было
популярным на Севере Франции и в Королевской семье, в виду не популярности
греческих имён. А вот на Юге Франции, в связи с контактами и торговлей с
Византией, это имя знали. Вы считаете это огромное достижение назвать сына -
Филиппом? 

\enquote{Она основала монастырь и в аббатстве до сих пор о ней хранится добрая
память, ежегодно там отмечают день ее памяти.} Замечательно! Только каждая
вторая королева, грубо говоря, основывала монастыри или что то им строила
@igg{fbicon.wink}. Изучите и вы удивитесь  @igg{fbicon.smile}.
\enquote{Елизавета стала \enquote{прародительницей} многих европейских
королевских фамилий через свою дочь и ее потомков.} - Елизавета (Олисава).
Отлично! Кстати, а вы почему то не упоминаете о том, что через два года после
того как её муж - Харальд, стал королём Норвегии, он женился на другой? Это то
же её достижение? А по поводу родства королей. Хорошо притянуто за уши  @igg{fbicon.smile} .
Через половецкую княжну - мать Ярослава Мудрого, половцы прародители \enquote{многих
европейских королевских фамилий}.  @igg{fbicon.smile} . \enquote{Вообще, все три дочери Ярослава были не
просто королевскими женами, а личностями и влиятельными политическими фигурами}
Да ну? именно влиятельными? политическими? Долго описывать патриархальный уклад
того времени. \enquote{Жаль, что Вы этого не знали.} - конечно же я не знал  @igg{fbicon.smile}. ну а по
поводу последнего. У нас совершенно разные взгляды. Но я не разу об этом не
упомянул, так как считаю неуместным совать сюда свои полит. взгляды. Ну это
такое....

\iusr{Ирине Вильчинская}
\textbf{Ljudmila Luist} 

Я скажу больше. Анна Ярославна увезла во Францию с собой Евангелие, которое до
сих пор хранится в городе Реймсе под названием Реймское Евангелие. В этом
городе по традиции короновались французские короли и клятву они приносили над
этой реликвией. Сегодня Реймское Евангелие переиздано и стало доступно многим
благодаря совместной работе французской стороны и украинских историков,
музейщиков, издателей. Один из экземпляров представлен в Софии Киевской, на
родине Анны. А также, на подворье заповедника \enquote{София Киевская} ежегодно
расцветают великолепные розы, выведенные во Франции в память об этой женщине
-Anne de Kiev (Oracarli, Anna Regina). Розы эти необыкновенно красивы и имеют
просто волшебный аромат. Цветут до поздней осени. Вот такая она \enquote{просто
королева}... и еще. Вторая часть ее жизни еще более романтична и авантюрна, чем
первая, \enquote{королевская}! Но это уже другая история...

\ifcmt
  ig https://scontent-frx5-1.xx.fbcdn.net/v/t39.30808-6/270043035_4922800597750641_1341768562158550152_n.jpg?_nc_cat=105&ccb=1-5&_nc_sid=dbeb18&_nc_ohc=7FAE0QrqNnkAX-FoM5S&_nc_ht=scontent-frx5-1.xx&oh=00_AT8-TVIqWNCwflN93uY7UMmctpPWisHEQiZqE8ptqlBbkA&oe=61D0B763
  @width 0.3
\fi

\iusr{Ljudmila Luist}
\textbf{Irina Vilchinskaya} 

Сейчас вспомнила,что про эти розы  @igg{fbicon.rose}  читала в КИ. Очень благодарна дочери, что
она открыла для меня эту группу. Окно в прошлое, которое не хочу забыть.


\iusr{Ирине Вильчинская}
\textbf{Ljudmila Luist} 

Киев всегда открыт для друзей и всегда с радостью их принимает! А киевлян
\enquote{бывших} не бывает, это доказывает и группа КИ.

\iusr{Галина Полякова}
\textbf{Олег Курилов} 

Юпитер сердится... Позвольте возвратиться к Вашему собственному комментарию, в
котором Вы соотносите эту ленточку с ВОв. А мой скромный вопрос, не РОА ли Вы
имеете в виду, Вы обрушиваете на меня все сразу, даже украинский национальный
флаг. Георгиевскую ленточку в годы Второй мировой войны советские
военнослужащие не надевали, не имели, не признавали.

\iusr{Ирине Вильчинская}
\textbf{Олег Курилов} 

На все ваши контраргументы даже не стану силы тратить, а вот о втором \enquote{браке}
Гаральда - это интересно. Дело в том, что это не был брак в полном смысле слова.
Тора была дочерью влиятельного конунга и стала наложницей Гаральда, а не женой.
Ни в одной из саг она не названа дроннинг - королева. Гаральду необходимо было
укрепить свое влияние среди норвежской \enquote{верхушки} и обзавестись наследниками
мужского рода. Что он и сделал, приблизив к себе Тору. У них с Елизаветой были
лишь две дочери (по-видимому, близняшки). Другой вопрос, что по местным законам
ВСЕ дети конунга признавались равными -и рожденные от законных жен, и рожденные
от наложниц. Поэтому сыновья Торы и Гаральда со временем тоже стали королями, а
дочь Елизаветы - королевой и не одного королевства!

\end{itemize} % }

\iusr{Олег Курилов}
\textbf{Галина Полякова} 

\enquote{Георгиевскую ленточку в годы Второй мировой войны советские военнослужащие не
надевали} это всего лишь ваше не знание, или вы просто забыли. Георгиевскую
ленту носили на Ордене Славы! Самом популярном солдатском Ордене! И именно эта
ленточка символизирует орден Георгия и его наследника - Орден Славы, в которых
есть именно эта лента! Отвечая на ваш вопрс, нет, не Роа имею ввиду. К Власову
отношусь плохо, а к подонку - Шухевичу ещё хуже. Так что совсем нет  @igg{fbicon.wink} 

\begin{itemize} % {
\iusr{Галина Полякова}
\textbf{Олег Курилов} Мы с Вами находимся по разные стороны баррикад, поэтому дальнейший диалог не имеет смысла.

\iusr{Олег Курилов}
\textbf{Галина Полякова}  @igg{fbicon.smile}   @igg{fbicon.face.tears.of.joy}  @igg{fbicon.thumb.up.yellow} 

\iusr{Ljudmila Luist}
\textbf{Galina Poliakova} Мудрое решение, на себе испытала.

\iusr{Олег Курилов}
\textbf{Галина Полякова} 

вся печаль в том, что про Власова вы знаете ( при чём из него никто не пытается
героя делать). А про это .... Шухевича, помимо служению немцам, уничтожавшего
мирное население!!! ( сделали супер героя  @igg{fbicon.frown}  ) Вы скромно уходите от ответа  @igg{fbicon.frown} 

\end{itemize} % }

\iusr{Олег Курилов}
\textbf{Ирине Вильчинская} 

\enquote{Гаральду необходимо было укрепить свое влияние среди норвежской \enquote{верхушки}}
Именно! нужна была влиятельная жена! Наконец то мы подходим к сути.
Влиятельной, Королева может быть или во время регенства, или если за ней стоят
влиятельные родственники! Русь не на Францию, не на Норвегию никак не влияла.
Потому что была далеко! А вот на Венгрию могла влиять! По этому, как вы
правильно уточнили, Харольду нужна была влиятельная жена из влиятельной семьи!
Елизавета была никто, просто иноземка и у неё не было никакого влияния! Именно
по этому он и женился второй раз! Кроме того, дочь Елизаветы к нам никакого
отношения не имеет! Она уже норвежка из совершенно другой династии!

\begin{itemize} % {
\iusr{Ирине Вильчинская}
\textbf{Олег Курилов} 

Ваше желание уничижить и принизить роль киевских княжен и роль киевского князя
в мировой политике того времени мне уже понятна. Как и понятно не очень
глубокое знание этого вопроса. Киевский двор, несмотря на географическую
отдаленность (весьма относительную, как для варягов) был очень мощным союзником
многих претендентов на норвежские владения. Не одно поколение варягов находили
здесь не только денежную службу в княжеской дружине, но и дети конунгов часто
оставлялись на попечении Киева, пока их родители бились с соперниками за трон.
Тот же Гаральд после разгромной битвы при Стикласдатире и гибели там Олафа,
добрался до Киева и здесь \enquote{зализывал раны}. Кстати, сын Олафа, Магнус как раз
оставался при киевском дворе, пока отец и дядя дрались на севере. Еще и тема
Олафа интересна. Не получив обещанную Ингигерд в жены (ее папа резко передумал
и отдал ее Ярославу), он, тем не менее, оставил своего сына при дворе своего,
более удачливого, \enquote{соперника}. Такие связи в дальнейшем очень были полезны обеим
сторонам в политическом смысле- дети, вырастая, сохраняли добрые отношения с
приютившим их Киевом. Ярослав предоставлял военную помощь Гаральду в походах на
южных соседей, на Константинополь. Так что поддержка Киева была всегда на
стороне Гаральда, но, безусловно, ему была нужна еще и поддержка "ближнего"
круга. Политика делает свое дело. Но в последнюю битву за английский престол
при Стамфордбридже он взял с собой Елизавету с дочерьми и сына Олафа, вероятно,
в надежде на победу и представление новым подданным своей королевы. Не Тору, а
Елизавету.

\iusr{Олег Курилов}
\textbf{Ирине Вильчинская} 

Det er likevel et problem at Stuvs vers er det siste pålitelige vitnemålet vi
har om Ellisiv. Selv om sagaene lar henne være med til Norge, inneholder de
ingen tradisjoner – eller gir overhodet ikke noen opplysninger – om hennes
samliv med Harald, eller om hennes rolle som norsk dronning. Attpåtil forteller
sagaene at Harald ca. 1048 tok seg en \enquote{meddronning}, nemlig Tora
Torbergsdatter, som tilhørte en av de aller fremste ættene i Norge. Tora ble
mor til Haralds to sønner, Olav og Magnus, som etterfulgte faren som konger.

Перевод с Норвежского. 

\enquote{Это все еще проблема, что стихи Стува - последнее надежное
свидетельство, которое у нас есть об Эллисиве. Хотя саги разрешают ей приехать
в Норвегию, они не содержат никаких преданий - и вообще не предоставляют
никакой информации - о ее сожительстве с Харальдом или о ее роли норвежской
королевы. Кроме того, саги рассказывают, что Харальд ок. В 1048 году была взята
«соправительница», а именно Тора Торбергсдаттер, принадлежавшая к одной из
самых выдающихся семей Норвегии. Тора стала матерью двух сыновей Харальда,
Олафа и Магнуса, которые унаследовали его отец как короли.} :))) Вот вам версия
от самих норвежцев :)))

\iusr{Олег Курилов}
\textbf{Ирине Вильчинская} 

Далее: 

\enquote{Ваше желание уничижить и принизить роль киевских княжен} НЕТ! Нет, я просто
хочу сказать, что там нет ничего этакого выдающегося!!! У нас какая то проблема
\enquote{меншевартости} какую то заурядную королеву мы почему то считаем большим нашим
достижением!! Зачем???? 

\enquote{роль киевского князя в мировой политике} - НИГДЕ я
такого не говорил! Может процитируете? Я с большим уважение отношусь к Ярославу
Мудрому! Вы просто не понимаете ничего!!! к огромному сожалению! Я указывал
удалённость Руси от Норвегии и Франции, и специально оговорился, что в Венгрии
Русь имела влияние! Но вы не внимательно читаете оппонента. И придумываете за
меня отсебятину  @igg{fbicon.frown} . Китайский император велик! Но если он отдаст свою дочь
замуж за короля Германии, у неё будет влияния меньше, чем у дочери герцога
Баварского это же элементарно! \enquote{мне уже понятна.} - Боюсь, что вам ничего не
понятно  @igg{fbicon.frown} . 

\enquote{Киевский двор, несмотря на географическую отдаленность (весьма
относительную, как для варягов) был очень мощным союзником многих претендентов
на норвежские владения.} Ну что ж, давайте подумаем, как мог повлиять киевский
князь на норвежцев? Он их сосед? Нет! Он мог отправить свои войска или дружину
в помощь кому то? НЕТ! Не было такого ни разу. Он мог чем то повлиять на
норвежских эрлов? НЕТ. Какое влияние? Багдадский Халиф Велик! Кокое он имел
влияние на скандинавию? Никакого, окромя того что там деньги зарабатывали.

\enquote{Ярослав предоставлял военную помощь Гаральду в походах на южных соседей} Где и
когда. А вот отсюда столо интересно расскажите об этом. ОН отправлял войска в
Данию? Не помню такого! \enquote{на Константинополь.} Что за чушь! Гаральд ходил в
Константинополь на зароботки, а не в военный набег. У вас всё смешалось  @igg{fbicon.smile} 

\iusr{Ирине Вильчинская}
\textbf{Олег Курилов} 

версии от норвежцев - это версии, которые может выдвигать всякий желающий. У
нас есть письменные источники Снорри Стурулсона, который собрал впервые саги и
зафиксировал все устные предания норвежцев. Это наиболее ранние сведения,
наиболее приближенные к тем, легендарным временам. И нет причин в них
сомневаться, тем более, что саги эти, в отличие от наших летописей, достаточно
откровенно повествуют о не самых лицеприятных событиях своей
истории... Например, о первой встрече Ингигерд с Ярославом. когда она схлопотала
от него совсем не благородную пощечину за недипломатическое замечание в его
адрес... Так что переубеждать Вас я не собираюсь - Вы слишком зациклены на
своем. Источники говорят сами за себя. Нигде не упоминается Тора с приставкой
\enquote{королева}, Наложница она и есть наложница, хоть и с правами.

\iusr{Олег Курилов}
\textbf{Ирине Вильчинская} 

Я вам привёл норвежский источник. То есть родной к этой истории. Посмотрел у
соседей Англичан там говорится о Жене Торе и их замужестве. Но это не суть, и
не важно. Это такие мелочи. Пусть Елизовета первая королева или там
единственная, совсем не так уж важно. Короче говоря, ни о каком особом влиянии
в Норвегии у Елизаветы даже норвежцы не видят. Неважно пусть она там и
королевой была ( хотя сами норвежцы это оспаривают  @igg{fbicon.smile}  ). Но раз она не смогла
родить королю сына, ни о каком влиянии уж точно говорить не приходится! Мы
именно об этом говорим. Кстати, вот: \enquote{В 1034 году Харальд со своей дружиной
(около 500 человек) поступил на службу к византийскому императору.} Какой там
военный поход на Константинополь с 500 человек! Смешно! И пожалуйста, не
приписывайте мне ваших предположений!!! Я же этого не делаю! Это опять говорит
о слабости вашей позиции  @igg{fbicon.wink}  И ещё раз. Нам и нашей Русью есть, чем гордиться
помимо каких то, чьих то королев в каких то странах.  @igg{fbicon.wink}  Князьями от Святослава
до Александра Невского и Данила Галицкого. Софиевским Собором, нашей мозаикой,
и т.д.

\iusr{Ирине Вильчинская}
\textbf{Олег Курилов} 

Мы говорим о разном времени. В 1034 году Гаральд, не получивший согласия на
брак с Олисавой у Ярослава, оскорбившись, идет на службу к византийскому
императору наемником деньжат заработать. Заработал, нужно сказать, совсем не
плохо - все заработанное целыми флотилиями отсылал в Киев, на хранение. А потом
еле успел ноги унести от посягательств императрицы Зои. Вернувшись через десять
лет и женившись на Золотоволоске, в 1043 году пошел на Константинополь с сыном
Ярослава поквитаться за убийство там русича. С подачи тестя. Так что военные
союзы с Киевом не сбрасывались со счетов. И, между прочим, когда Ярослав, по
просьбе норвежцев вернуть им Магнуса (сына Олафа Толстого) правителем, выдвинул
свои условия - сохранить,сберечь его воспитанника в вечных борениях за власть,
они поклялись и клятву сдержали. Магнус умер своей смертью из-за слабого
здоровья!

\iusr{Олег Курилов}
\textbf{Ирине Вильчинская} 

Уважаемая Ирине. \enquote{Так что военные союзы с Киевом не сбрасывались со счетов.} ВЫ
правильно говорите, безспорно, НО это только в том случае, когда норманы
находились в наших владениях. Да это сработало. им нужна поддержка и надёжный
тыл. А когда он был в Норвегии? ведь не зря он жениля на Торе! Как вы правильно
указали раннее, ему нужна была поддержка и влияние! И эту поддержку и влияние
Ярослав Мудрый и его дочь не могли предоставить. А могли предоставить местные
могущественные феодалы, что я и пытаюсь Вам терпеливо обьяснить уже какой час.
 @igg{fbicon.smile}  А когда он пошёл в Англию ему предоставила поддержку Русь? Нет! Хоть она и
могущественнее и сильнее Норвегиии в пять раз. Она очень далеко! А вот в случае
с Польшей, Венгрией, Волжской Булгарией и даже Византией, уже совсем другое
дело.

\end{itemize} % }

\iusr{Ирина Ракоча}
\textbf{Олег Курилов} 

Анна была не русской а дочерью а князя Киевской Руси, а это разные вещи и не
путайте праведное с грешным. Вы пытаетесь казаться умнее чем есть на самом
деле.

\begin{itemize} % {
\iusr{Олег Курилов}
\textbf{Ирина Ракоча} 

Великолепно! А кто был по вашему киевский князь Ярослав Мудрый? или там князь
Владимир Мономах? Вы путаете современные названия национальностей, с названием
жителей Руси! Они все тогда были рускими, руськими, короче жителями Руси.
Термин Киевская Русь придуман в 19 веке как раз русскими по национальности
@igg{fbicon.face.tears.of.joy} , историками Р.И.  @igg{fbicon.smile}  Ещё раз,
житель тогдашней Руси - руский (руський), СОВСЕМ НЕ РАВНО современному названию
национальности - русский. . 

П.С. вы сначала лучше изучите вопрос, а потом уже
переходите на личности, хотя даже в этом случае не стоит этого делать
@igg{fbicon.wink}   @igg{fbicon.smile}. Всего самого хорошего!

\end{itemize} % }

\end{itemize} % }

\iusr{Людмила Таркан}
Дякую за цікаву розповідь!

\iusr{Ирина Петрова}

Якраз декілька днів тому були у Маріїнському палаці з подружкою \textbf{Світлана
Карбовська} потім глянула історію про перебування імператриці у Києві напередодні
революції... жахливі часи... @igg{fbicon.face.unamused} 


\end{itemize} % }
