% vim: keymap=russian-jcukenwin
%%beginhead 
 
%%file 16_07_2020.news.dnr.tri_color_center.1
%%parent 16_07_2020
 
%%endhead 

\def\sectitle{В ДНР почтили память мирных жителей, погибших при ударе
украинской авиации по городу Снежное}

\subsection{\sectitle}

\url{https://www.facebook.com/groups/LNRGUMO/permalink/2849401448504744/}
\url{https://tricolor-center.ru/stati/44017-v-dnr-pochtili-pamyat-mirnyx-zhitelej-pogibshix-pri-udare-ukrainskoj-aviacii-po-gorodu-snezhnoe.html?fbclid=IwAR3xQIn1-MDRuZYlL-rs6MtgK-amTl0iP7D-oFeWrxclAcIbQH5CwFy12v4}
  
\vspace{0.5cm}
{\ifDEBUG\small\LaTeX~section: \verb|16_07_2020.news.dnr.tri_color_center.1| project: \verb|letopis| rootid: \verb|p_saintrussia|\fi}
\vspace{0.5cm}

\ii{fig.16_07_2020.news.dnr.tri_color_center.1}

В ДНР вспоминают жертв авиаудара украинской армии по городу Снежное. Сегодня,
15 июля, шестая годовщина с того страшного дня.

15 июля 2014 года в 06:30 штурмовик Су-25 ВС Украины нанес удар по центру
Снежного, выпустив не менее шести ракет. В результате авианалета обрушилась
часть жилого дома по улице Ленина, были повреждены близлежащие здания.
Двенадцать мирных жителей погибли, не менее десяти человек, в том числе
ребенок, получили ранения.

«Я спал с мамой, когда украинский самолет сбрасывал на нас бомбы. Мы ничего не
слышали и не видели. Папа выжил, а на маму две плиты упало, а на меня одна. Я
лежал и кричал: «Помогите, помогите!». Услышали люди мой голос, и скорая
приехала и помогла мне, а мама умерла», --- цитирует раненого в тот день Богдана
Ястребова Донецкое агентство новостей.

«Это была первая крупная трагедия, --- вспоминает подробности того дня
замначальника местного государственного пожарно-спасательного отряда Виталий
Белов. --- К месту прибыли спасатели и извлекли пять пострадавших и тела восьми
погибших горожан. В спасательной операции принимали более 100 человек личного
состава».

По словам Белова, в разборе завалов принимали участие и обычные горожане,
бросившиеся на помощь, несмотря на опасность возобновления обстрелов. Взрывы
были слышны во всем городе, поэтому к месту трагедии пришли сотни неравнодушных
людей.

«Жители объединились все сразу. Если беда пришла, как иначе, мы же русские
люди, коллектив, если беда какая, то все стараются помочь, вместе, сообща.
Нельзя дома отсиживаться: сегодня к соседу пришла беда, завтра к тебе, и тебе
кто-то поможет», --- заявил Александр Бондаренко, служивший на тот момент
замкоменданта города.

«В полседьмого утра пролетел самолет и сбросил бомбы, возле налоговой снаряд не
разорвался, а если бы разорвался, то мой бы подъезд тоже пострадал, --- добавила
жительница соседнего дома Валентина Ивановна, очевидица трагедии. --- Повсюду
были разбросаны части тел, руки, ноги, столько крови было. Было очень страшно».

Годовщина удара украинской авиации по Снежному в 2014 году является
напоминанием о том, что Киев готов пойти на любые преступления, чтобы подчинить
Донбасс. Об этом сегодня заявил глава ДНР Денис Пушилин.

«Каждый год этот день возвращает нас к осознанию того, что Киев готов на любые
преступления, чтобы сломить волю жителей Донбасса. Украинские силовики
целенаправленно пускали ракету за ракетой по городу, зная, что там находятся
дети, старики, женщины. Такое нельзя забыть и невозможно простить», --- написал
он в своем телеграм-канале.

Сегодня в Снежном у дома №14 по улице Ленина собрались жители Республики, чтобы
почтить память земляков, погибших в результате украинской агрессии.


