% vim: keymap=russian-jcukenwin
%%beginhead 
 
%%file 05_03_2021.fb.molchanov_jurij.1.lvov_izbienie_mova
%%parent 05_03_2021
 
%%url https://www.facebook.com/george.molchanov.9/posts/3616161285163480
 
%%author_id molchanov_jurij
%%date 
 
%%tags izbienie,jazyk,lvov,mova,muzyka,muzykant,ukraina,__mar_2021.lvov.izbienie.muzykanty.jazyk
%%title По избиению во Львове
 
%%endhead 
 
\subsection{По избиению во Львове}
\label{sec:05_03_2021.fb.molchanov_jurij.1.lvov_izbienie_mova}
 
\Purl{https://www.facebook.com/george.molchanov.9/posts/3616161285163480}
\ifcmt
 author_begin
   author_id molchanov_jurij
 author_end
\fi


По избиению во Львове.

Сериал с пещерными рефлексиями на языковую тему, активно снимаемый на телефоны
после принятия мовного закона, демонстрирует серую банальность: чем интенсивнее
съемочный процесс, чем больше роликов в тик-токах, фб и ютубах, тем ближе
медный таз, которым обычно накрывается идеологический тоталитаризм.

За доказательствами в карман лезть не нужно. Та же эпоха недавнего юбиляра
Горбачева (утрировано): чем отчаяннее была пропаганда сухого закона, тем больше
флаконов из-под тройного одеколона валялось в подворотнях. Чем громче "слава
КПСС", тем явственнее "пошли нах!".

Нелепость ситуации с языком была предсказуема. Поскольку игра в патриотизм в
исполнении Порошенко хоть и до отвращения лицемерная, но она плюс-минус в
органике. Петр Алексеевич первым оседлал этого коня и уже успел к нему
прирасти. Поэтому зеленые аматоры ура-патриотизма обречены быть карикатурой БПП
даже в том случае, если они вдруг, на манер Дзержинского, решат удариться в
зеленый террор. Это все-равно будет карикатура. А что может быть более скучным,
чем карикатура на такое явление, как Сывочолый?

Каждый видосик с участием мовных хунвейбинов на улицах, каждая истерика в
общепитах и торговых точках, каждый случай травли и буллинга, каждое
слюноотделение всяких дроздовых, каждая гадость и желчь бесчисленных
медианичтожеств дают ощутимый суммарный эффект идиосинкразии у тихого
большинства. Особенно на фоне экономической и социальной феерии за окном.
Потому что эта самая идиосинкразия подстерегает каждого, кто выключает
телевизор с фейсбуком и переводит взгляд от «революция тривае» на суровую
реальность. Неизбежно!

Каждое решение, каждый шаг правителей, оторванный от суровых каждодневных
реалий экономики и социалки озлобленного народа, будет только усиливать этот
эффект. В прогрессии.

А резвые «патриоты», кто за деньги, а кто по дурости, будут и дальше с грохотом
загонять власть в гетто своей повестки, умножая на ноль весь каркас института
власти, как таковой. 

Каким бы патриотичным орнаментом не был вышит ее фИговый листок.

\ii{05_03_2021.fb.molchanov_jurij.1.lvov_izbienie_mova.cmt}
