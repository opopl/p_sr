% vim: keymap=russian-jcukenwin
%%beginhead 
 
%%file 21_07_2021.fb.bryhar_sergej.1.mova_odessa_vokzal
%%parent 21_07_2021
 
%%url https://www.facebook.com/serhiibryhar/posts/1746088302257781
 
%%author Бригар, Сергей
%%author_id bryhar_sergej
%%author_url 
 
%%tags jazyk,mova,odessa,ukraina,ukrainizacia,vokzal
%%title Мова - Одесса - Железнодорожный Вокзал
 
%%endhead 
 
\subsection{Мова - Одесса - Железнодорожный Вокзал}
\label{sec:21_07_2021.fb.bryhar_sergej.1.mova_odessa_vokzal}
 
\Purl{https://www.facebook.com/serhiibryhar/posts/1746088302257781}
\ifcmt
 author_begin
   author_id bryhar_sergej
 author_end
\fi

Іду через Одеський залізничний вокзал. Заходжу в другу касову залу. На зустріч
прямують жінка з малим хлопчиком. З ними - великий рюкзак і валіза. Жінка
зупиняється, і з дуже відчутним українським акцентом звертається до мене
російською: 

"Ізвінітє, а ви нє моглі би подсказать, гдє нам здєсь найті камєру
схор...ой...збєрєженія? А то у нас пока нє получИлосЯ найтИ".

- Звісно, можу. Це не складно, - відповідаю я: - повертаєте он у той прохід,
потім прямо, вдовж адмінприміщень, побачите вихід в місто, минаєте його,
проходите повз кафе, і відразу за ним - камера зберігання.

\obeycr
- Ох, як гарно. Дякую вам дуже. 
- Будь ласка. Але ж розмовляйте українською - всі все чудово зрозуміють!
- Добре. Це ж я просто думала, що то Одеса, і все оцево.
- Так Одеса - це теж Україна.
- Я все розумію. І дякую вам ще раз.
- А ви, якщо не секрет, звідки?
- Зі Львова...
\restorecr

Аналогічна історія відбулася зі мною два тижні тому у Львові. Але є ключова
відмінність. І полягає вона в тому, що гість міста Лева із Запоріжжя навіть не
спробував перейти на українську. Хоча розмовляли ми близько 10 хвилин. От він
навіть не подумав. Тобто "це ж Львів, і все оцево" йому по цимбалах. Він всюди
гордо несе з собою свій "русскій язик".

\ifcmt
  pic https://scontent-lga3-1.xx.fbcdn.net/v/t39.30808-6/217888003_1746087355591209_1536558217953487498_n.jpg?_nc_cat=108&ccb=1-3&_nc_sid=8bfeb9&_nc_ohc=krZ1ZKNccNcAX8g72YA&_nc_ht=scontent-lga3-1.xx&oh=73a10863de4ab2bb70088d60958084f4&oe=610366AD
  width 0.4
\fi

От, будь ласка: ми, україномовні - невпевнені, якісь залякані, закомплексовані;
і вони, зросійщені (чи росіяни за самовизначенням) - абсолютно впевнені, часто
навіть нахабні... Як кажуть в одному південному місті: "це дві великі різниці".
Серед нас тих, хто підлаштовуються "под большинство" значно більше, ніж серед
них тих, хто підлаштовується "під більшість".

По суті, невпевненим україномовним потрібно брати приклад з них! Так, я
розумію, що це складно, бо за ними стоїть хоч і гнила, але все ще сильна і
небезпечна імперія, а за нами - лише наші однодумці. Але потрібно
"опірюватися". Це дуже важливо. Якщо все залишиться так, як є зараз, уже скоро
україномовні масово підлаштовуватимуться під них не лише в Одесі, Харкові чи
Києві, а й значно західніше...


\ii{21_07_2021.fb.bryhar_sergej.1.mova_odessa_vokzal.cmt}
