% vim: keymap=russian-jcukenwin
%%beginhead 
 
%%file 04_03_2023.fb.kipcharskij_viktor.mariupol.1.r_k_tomu__bulo_take_.cmt
%%parent 04_03_2023.fb.kipcharskij_viktor.mariupol.1.r_k_tomu__bulo_take_
 
%%url 
 
%%author_id 
%%date 
 
%%tags 
%%title 
 
%%endhead 

\qqSecCmt

\iusr{Ирина Краевая}

Це страшно читати. Я розумію, що до дня не можу відтворити той жах, у якому ми
опинитися у перші два тижні війни, все як кремезний жах. Одне пам'ятаю. Дуже
гучно луплять завжди, нема їжі та води... й все це треба добувать серед тих
бомбувань... Велика дяка людям, хто допоміг. Велика та незабутня

\begin{itemize} % {
\iusr{Віктор Кіпчарський}
\textbf{Ирина Краевая} 

В жодному разі не намірювався лякати або вихвалятися. Це просто мій щоденник на
смартфоні, у який я записував те, що відбулося за добу. Зараз я лише перекладаю
українською мовою, а доповнюю після \enquote{навздогін}, аби відокремити \enquote{тодішне} від
\enquote{сучасного}.

\iusr{Віктор Кіпчарський}
\textbf{Ирина Краевая} 

Так, велика подяка власникам маленьких \enquote{шопиків} (магазинчиків), які їздили містом, аби привезти нам їжу.

\iusr{Віктор Кіпчарський}
\textbf{Ирина Краевая} 

Гупало з усіх боків, вікна, а подекуди і будинок дриготіли, але переляку до
істерики майже не було. Принаймні, у мене. У жінок... були часом червоні очі.

Дуже тішило, коли після особливо гучного \enquote{гупання} чотирирічна онучка спокійно
промовляла тоненьким дитячим голосом: \enquote{ісходящій} - дідусь поганому не навчить!

Молилися і вірили, що біда омине нас.

І родичі молилися за нас.

Навіть відшукали \enquote{правильного батюшку}, аби помолився за нас.

Так сталося, що наш двір був чи не найспокійнішим у місті - і це диво.

І до літа в нашому будинку нікого не вбило і навіть ніхто не помер!

\iusr{Марина Солошенко}
\textbf{Віктор Кіпчарський} 

у мене, коли в Харкові було дуже гучно, молодший онук завжди казав \enquote{ну ми ж
цілі, все гарно}, прятали його в подушки, хліб роздавали, привозили сир,
картоплю, крупи, це нас трохи заспокоювало. Надвигались морози і мої думки були
о маріупольцямх, як без тепла їх пережити?! Нелюди! Вони розуміють, що ми їх
ніколи не простим?!

\iusr{Віктор Кіпчарський}
\textbf{Марина Солошенко} 

За їхньою логікою, це ми маємо просити прощення, бо, як сказала вчора в
Нью-Делі сумна коняка: \enquote{роSSія вимушена захищатися, бо на неї напала
Україна}!!!

Поки вікна і дах були цілі, морози були не такі страшні, як нестача їжи...

\end{itemize} % }

\iusr{Светлана Водзянская-Живогляд}

Моїм великим страхом теж були залишки еутіроксу, бо він зник одним з перших 😞
у цей день чоловік пішов забирати машину яка стояла у гаражі під шлаковою,
зустрів там людей які навпаки поїхали туди ховати машину і він їх підкинув
додому на Кірова, через 2 дні зустріли їх у черзі за продуктами на Щироку кумі,
вечорі там було кілька прильотів і машини в них більше не було 😞 а нам у
вигляді допомоги на багатодітну родину видали 4 чабати у той день

\begin{itemize} % {
\iusr{Светлана Водзянская-Живогляд}
\textbf{Наталия Павлова} 

в мене теж волосся висипалося сильно, після того як виїхали то не могли знайти
еутірокс ніде, я в паніці бігала по всіх аптеках і не було ніде :(( приїхала у
Іспанію і на пальцях пояснювала провізору що мені треба. Як вона здивувалася
дівчині що плаче тримаюся звичайний гормон у руках ...

\end{itemize} % }

\iusr{Светлана Водзянская-Живогляд}

День страшного мародерства. Нам сусіди сказали що дають харчі на Метро, я
приїхала туди і не змогла себе змусити вийти з машини. То було МАРОДЕРСТВО!!!
люди тягнули стільці, якісь речі.. мене паралізувало від жаху, мама поруч
заплакала, це і привело мене до тями. Я вдавила газ у підлогу щоб не бачити
цього...

\begin{itemize} % {
\iusr{Віктор Кіпчарський}
\textbf{Светлана Водзянская-Живогляд} В нас мародерства ще не було - може через те, що в Дружбі були поліцейські?

\iusr{Віктор Кіпчарський}
\textbf{Наталия Павлова} Найжахливіше в цьому те, що мародерили не заїжджі, а свої.

\iusr{Светлана Водзянская-Живогляд}
\textbf{Віктор Кіпчарський} можливо, в центрі вже почалося...

\iusr{Светлана Водзянская-Живогляд}
\textbf{Наталия Павлова} просто хапали все що бачили і те що треба і ні

\iusr{Віктор Кіпчарський}
\textbf{Наталия Павлова} Чим форма ТрО відрізняється від форми ЗСУ?
Я пишу про те, що бачив на власні очі, чого і Вам бажаю.

\iusr{Светлана Водзянская-Живогляд}
\textbf{Наталия Павлова} нажаль так

\iusr{Віктор Кіпчарський}
\textbf{Наталия Павлова} 

Мій Друг (колишній мій студент), пішов у тероборону, був поранений, поморозив
ноги, йому відрізали пальці і зараз він у полоні. Я не впізнав його на фото -
настільки він постарів.

Ще один мій Друг (теж колишній студент), пішов у тероборону, привозив їжу у
госпіталь на Нептуні, у ТерраСпорт. Виїхав з Маріуполя, пішов у ЗСУ і загинув.

З розповідями про те, як \enquote{дамбілі бамбас} йдіть, будь ласка, на якусь іншу сторінку.

Це моя сторінка і вирішую, що на ній може бути, а чого бути не може.

Вважайте, що у Вас від мене два останні попередження.

\iusr{Віктор Кіпчарський}
\textbf{Наталия Павлова} у моєму чорному списку.

\end{itemize} % }
