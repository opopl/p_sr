% vim: keymap=russian-jcukenwin
%%beginhead 
 
%%file 20_11_2021.fb.drobotenko_nikolaj.kiev.1.spuski_pozdravlenia
%%parent 20_11_2021
 
%%url https://www.facebook.com/nick.drobotenko/posts/4368597526570814
 
%%author_id drobotenko_nikolaj.kiev
%%date 
 
%%tags alpinizm,gory,ukraina
%%title О СПУСКАХ С ВЕРШИН и традициях поздравлений
 
%%endhead 
 
\subsection{О СПУСКАХ С ВЕРШИН и традициях поздравлений}
\label{sec:20_11_2021.fb.drobotenko_nikolaj.kiev.1.spuski_pozdravlenia}
 
\Purl{https://www.facebook.com/nick.drobotenko/posts/4368597526570814}
\ifcmt
 author_begin
   author_id drobotenko_nikolaj.kiev
 author_end
\fi

О СПУСКАХ С ВЕРШИН и традициях поздравлений.

(лонгрид, как модно сейчас выражаться)

(для специфического круга читателей – так выражаться не модно, но я это сделал)

***

Вопрос - что на скрине объединяет сообщения авторов?

Понятно - все они поздравительные. Наши парни совершили уникальное восхождение
на Аннапурну ІІІ в Гималаях, чего до них за 40 лет разных попыток не смогли
сделать иные международные экспедиции. Но не на этом я фокусирую взгляд. А на
том, что во ВСЕХ месседжах присутствует оборот «Вот теперь…»

\ifcmt
  pic https://scontent-frx5-1.xx.fbcdn.net/v/t39.30808-6/259166853_4368941019869798_6682328784805035614_n.jpg?_nc_cat=110&ccb=1-5&_nc_sid=730e14&_nc_ohc=dqJBUt14dQgAX8pnHQH&_nc_ht=scontent-frx5-1.xx&oh=74621ea41f93a5b3ecaae47fded0e01c&oe=61A558B6
  @width 0.8
\fi

И об этом загадочном «Вот теперь» жизнь дала повод повести разговор.

***

Представлю тех, кто употребил попавший в фокус речевой оборот (в представлении
они не нуждаются, но я соблюду ряд формальностей):

- Марина Коптева – обладатель престижной награды «Золотой ледоруб» за
первопрохождение в 2011 году стены Great Trango Tower (38 дней на маршруте) в
составе женской команды - Марина Коптева, Галина Чибиток, Анна Ясинская (не
считая иных сложнейших знаковых восхождений).

- Геннадий Копейка. Первый украинец, ступивший на вершину самого опасного
восьмитысячника К2, участник экспедиции на Южную стену Лхоцзе 1990 года –
пройденный командой маршрут не давался Рейнхольду Месснеру несколько раз, после
чего тот назвал его «маршрутом XXІ века».

- Тарас Поздний – восходитель на восьмитысячники и организатор восхождений по
всему миру.

То есть люди, мягко говоря, понимающие горы, только после спуска восходителей в
базовый лагерь написали поздравления с употреблением слов «Вот теперь».

А началось всё с первого сообщения о достижении украинской командой вершины на
Фейсбук-странице Черкасской федерации альпинизма и скалолазания (процитирую):

«Сегодня 06.11.2021 в 6:19 пришла смс - "Есть вершина" + координаты 

Вячеслав Полежайко, Никита Балабанов , Михаил Фомин сделали первопроход на
непальскую вершину Аннапурна III ( Annapurna III 7555 метров).

Целью команды стал не пройденный юго-восточный гребень горы.

Более трёх десятилетий Юго-Восточное ребро Аннапурны III оставалось одной из
последних "серьёзных нерешенных проблем в Гималаях"»

Конец цитаты.

Это была информационная бомба. Но в тексте новости - никаких поздравлений (!)
Только сухая констатация факта. Однако по перепостам, комментам пошёл вал
ликования - «Поздравляем!!», «С горой, парни!! Круто!!»

И для меня нет ничего хуже, чем над ринувшимися поздравлять сразу прекрасными
искренними людьми открыть вентиль с холодной водой.

***

Так уж вышло, что я кое-что понимаю в статистике несчастных случаев в
альпинизме.

До развала Союза каждый НС строго фиксировался, разбирался (даже если речь шла
о травме), и сводился под конец года в печатный бюллетень «Анализ несчастных
случаев в альпинизме». Эти ежегодные бюллетени распространялись по альплагерям
и контрольно-спасательным пунктам.

Так вот с тех пор у меня остались в памяти цифры – ориентировочно 70% НС в
горах происходило на спуске. И лишь 30% - на подъёме.

Не буду далеко ходить за примерами – украинский альпинизм, последнее пятилетие.

1. Четыре месяца назад - 4 июля сего года, при спуске с вершины Тетнульд на
Кавказе (2Б к/тр) гибнет киевлянин Анатолий Мрачковский. Срыв без страховки на
фирновом склоне.

2. Полгода назад – 2 мая текущего года, Яна Кривошея совершает скальное
восхождение соло на гору Geyik Sivrisi в Турции (категория трудности неизвестна
– Яна поехала на скалы одна). После отправленного маме сообщения в Viber «Я на
вершине!» - срыв на спуске с фатальным исходом.

3. Два года назад – 28.11.2019 при спуске дюльфером в Татрах с вершины Видлова
Вежа (маршрут 3А к/тр) гибнут инструктор Владимир Михалко и участница группы
Ирина Скрылева.

4. Три года назад – в июле 2018 года, на спуске с вершины Тетнульд (тот же
маршрут, что в п.1 перечня) - cрыв связки с гребня. Гибнет участник восхождения
Евгений Александров, напарник получает серьёзные травмы.

Можно долго говорить о причинах каждой трагедии и ошибках, которые к ним
привели. Я сейчас об ином.

Мне не известно НИ ОДНОГО несчастного случая с украинскими альпинистами НА
ПОДЪЁМЕ в тот же период (карпатские инциденты, в том числе гибель в 2019-м от
переохлаждения двух человек при подъёме на обсерваторию горы Попиван
Черногорский, не привожу – это ПВД /походы выходного дня/, а не альпинизм – там
своё поле непаханое).

Подводим черту. Пять погибших НА СПУСКАХ с вершин за последнее пятилетие и НИ
ОДНОГО – на подъёме. Т.е. 100% НС случились на спусках. Так вышло.

И я хотел бы, чтоб вот эта закономерность – на спусках происходит бОльшая часть
аварий в горах - отложилась у всех в голове. Как останавливающий фактор для
эмоций «болельщиков». Но главное - как предостережение восходителям. Как стимул
продолжать быть сконцентрированным и после вершины, на спуске.

Закономерность, статистика, которую я призываю запомнить, давно имеет своё
объяснение - расслабление («Мы на вершине, ура!»), снижение концентрации
внимания, накопленная усталость, спад присутствовавшего на подъёме адреналина,
пренебрежение страховкой (чтоб побыстрее спуститься) «да тут чуть-чуть вот
пройти…»  и т.д. и т.п. При этом спуск обычно просматривается хуже подъёма.

***

А теперь - к заголовку поста.

Те, кто кое-что понимает в горах – НЕ ПОЗДРАВЛЯЮТ восходителей, пока те не
спустятся вниз.

Альпинизм – не спринт. Вершина – не финиш. Оттуда не снимет волшебник в голубом
вертолёте, показав кино-эскимо.

Вершина – полдела. По рискам аварий так треть. Впереди – не менее ответственная
часть восхождения под названием «спуск».

Сопереживающие могут слать лучи добра и поддержки восходителям в любой форме –
«удачного спуска!», «ждём внизу с нетерпением!», «держим кулаки!», пр. и др.,
но поздравлять – РАНО.

Поздравлять ПОСЛЕ спуска – традиция, которую многие, как показал опыт, не знают
и которой, в век стремительного опопсячивания альпинизма, как оказалось, не
учат.

***

Но можете традициям и не следовать. Просто есть слова-маркеры. И если Кличко
(он у нас альпинист теперь, в курсе?) гордо заявляет: «Я покорил вершину
Казбека высотой 5 тысяч метров!» - сразу ясно, что выпускник современного
альпинистского коммерческого инкубатора Виталий Кличко не приобрёл понимания
гор вот вообще. Казбек этим летом лениво плюнул лавиной прямо на путь
стандартной линии восхождения. И если б там в тот момент оказался «покоритель»
Кличко – отпели бы уже «покорителя». А Киевом управлял другой мэр.

«Я взошёл», «я совершил восхождение» - сдержанная корректная лексика. А не
«покорил», как любят массово вещать журналисты, которых переучивать – только
портить, их ляпами  нужно расслабленно наслаждаться.

С поздравлениями на вершине – та же история. Которую я в статистике уже
рассказал. 

В общем выбирайте – быть похожими на Кличко в альпинизме или учиться у
Коптевой, Копейки и Позднего. В конце концов выбор каждого – скатываться в
попсу или нет. Я не навязываю. Сами решайте.

***

А сейчас опишу личные ощущения после опубликованной SMS «Есть вершина!». Изложу
ход своих мыслей.

***

Внутреннее ликование было первой реакцией, несомненно. Но мгновенно мысль –
только б не расслабились при осознании, что решили альпинистскую мировую
проблему последних сорока лет. Лишь бы теперь не допустили ошибку…

Путь вверх был сложнейшим. И у меня роились вопросы.

- Уровень общей усталости?

- Сколько на группу осталось продуктов?

- Сколько газа? Без еды еще можно как-то тянуть, но без воды, растапливаемой
изо льда на горелке – никак.

- Что с обморожениями? Если есть - насколько критичны?

- Там сложный спуск, дюльфера на семи тысячах в холод и ветер… Хватит ли
расходников на организацию своих станций – петель, крючьев, закладных
элементов?

…

В вале грянувших сетевых поздравлений у меня в голове шёл вал вопросов. Я
напряженно отслеживал сообщения.

- Апдейт 6 ноября на странице Черкасской ФАиС. От парней смс: «в 14:13 -
спустились до 6700м. Ночуем. Все ок.»

Дьявол, это родным вы можете писать «всё ок» (и правильно делаете), а я вижу
прогноз – ветер на вершине до 115 км/час. На 6700 будет меньше, но всё равно –
это жесть.

- Апдейт 7 ноября. Сообщение в 16:49 – «мы на 6400м. Дюльферяем»

Чччёрт… сброс всего трёхсот метров от предыдущей ночёвки. Значит всё очень
непросто. Идёт техническая работа. Посыл флюидов: «Парни, дюльфера… предельно
внимательно!!»

А в голове – разбор НС в Татрах два года назад с группой Михалко (общий опыт
Михалко в горах – сорок лет, инструкторский – тридцать). Группой была допущена
ошибка именно при организации дюльфера. Дюльфер продёрнулся со спускавшейся в
это время последней участницей. Результат - двое погибших, несмотря на
колоссальный опыт Володи (я знал его по 80-м) - он в этом инциденте тоже погиб.

«Ребята, очень внимательно!.. Очень!..»

- Апдейт 8 ноября в 23:50 – «Всё ок. Техническое всё. Спустились на 5400.
Завтра план на траву».

Сплюнуть три раза! Ещё чуть-чуть, парни!.. С 5400 если бежать в БЛ на
заплетающихся ногах к еде будете, не хряснитесь уже носом где-нибудь, не говоря
уж о целых конечностях. Чтоб под конец без сюрпризов. 

- Апдейт 9.11 в 11:55 – «Привет. Всё хорошо, мы уже сняли БЛ и летим в
Катманду. Все живы)»

***

Всё окончено.

И на странице Черкасской ФАиС, председателем которой является один из
восходителей - Вячеслав Полежайко, появляются в хронике заключительные слова:
«Ура! С горой парни!!! Ждем с нетерпением дома»

Вот они, поздравления. СЕЙЧАС. А не ранее.

***

А у меня пост-синдром.

В первых же интервью слышу то, чего опасался.

Восхождение длилось дольше расчётного времени. Брали еду на 12 дней, провели на
горе 18. В предпоследний день закончился газ. Обморожения есть, но они не
критичны (на видео - волдыри пальцев рук). Физиономии обветрены, но вполне в
норме – трещины губ совсем мелочь.

Потери в весе:

Миша – минус 12 кг.

Никита – минус 13 кг.

Слава – минус 16 кг.

Но они дотянули. Молодцы – вне сомнений. Можно выдохнуть и произнести
сакраментальное «ВОТ ТЕПЕРЬ…».

***

… вот теперь я сделаю лирическое отступление.

***

Наприкінці минулого тижня мені зателефонував персональний тренер з української
мови, широко відомий у вузьких колах вельмишановний пан Ігор Чаплинський.

Те що він альпініст вищого гатунку люди «в темі» знають давно. Але широкому
загалу пан Ігор відкрився, оголосивши на всю Україну два роки тому на честь
свого 60-річчя персональний конкурс «На краще сходження, мандрівку або пригоду»
з призовим фондом у 60 тисяч гривень. І пан Чаплинський чесно конкурс провів,
гроші виплатив (таких конкурсів в історії українського альпінізму ще не було).

А «добив» він питання своєї популярності ще в ширших колах, коли в 61 рік
народив п’яту донечку, оголосивши про цю подію в Фейсбуці (то ж я не розголошую
таємницю).

Ми, звісно, вболіваємо за пана Чаплинського, який зробив вже п’ять спроб
народити синочка, щоб передати йому альпіністське залізяччя, яке випадає в
нього з полиць і дівчатам геть не потрібне. Бажаємо пану на шляху до мети не
спинятися. І готові щиро вітати з народженням кожної наступної донечки аж до
ста років, бо дівчатка в нього виходять просто красуні. А там і красень-хлопчик
десь повинен з’явитись – по теорії ймовірності шанси ж ростуть, час ще є, а
натхнення там хоч відбавляй. Але любимо ми пана Чаплинського не за це. А за те,
що він відвертий і щирий. Якщо людина – гімно, пан Ігор так і каже – «гімно» (я
пізнаю нові співочі слова). А якщо гарна людина, то так і каже – «Оце – класний
чувак» (обожнюю його солов’їну). Але я не про це.

Ми ж тут про альпінізм розмірковуємо. То мені до вподоби його двійкова система
числення в бесідах про наявних людей в альпінізмі - «То не альпініст» і «Оце
альпініст». Все. Тумблер. Без напівтонів. Не зважаючи на паперові розряди,
звання, посади у якихсь федераціях. «Ми ж знаємо – хто альпініст, а хто
балалайка» - запам’ятав я колись їм сказану фразу (вивчаю ж мову старанно). 

І ось, зателефонувавши, пан Ігор після тренування мене українською в вільній
бесіді по всіх сферах життя, наприкінці каже – «Напиши щось про хлопців, які на
Аннапурну сходили. Класні чуваки. Альпіністи».

Все. Це топ. Визнання. Знак якості. Абсолют.

Відповідаю - «Звісно, я й сам розумію. Сам щиро хочу їх привітати. Не люблю
тексти писати - ти ж знаєш, але тут випадок особливий. Обов’язково щось
напишу».

***

Ось і пишу…

***

Теперь перейду на иностранный язык, щоб мій тренер з української не узнал, что
именно будет написано у меня в тексте – він надає хлопцям копняків самостійно і
сам їм особисто все скаже.

А я выражу ощущения личные - наблюдателя, далёкого от современного альпинизма,
и глядящего на процессы со стороны. Буду говорить так, будто смотрю ребятам в
глаза.

Михаил Фомин

Никита Балабанов

Вячеслав Полежайко

Парни, у меня в связи с вашим восхождением есть три внутренние констатации.

- Первая.

Вот поверьте, я понятия не имею о ваших званиях на бумаге – оформляли вы
мастеров спорта, не оформляли... никогда не интересовался. В двоичной системе
Чаплинского (которую я разделяю всецело) это не важно. Просто констатирую - при
том, что пройденный вами маршрут не могла ранее пройти ни одна МЕЖДУНАРОДНАЯ
экспедиция, вы являетесь мастерами спорта международного класса по факту. Когда
МСМК - не запись в бумажке, а фактическое положение дел.

Есть ещё звание «Заслуженный мастер спорта» - тоже ваше по факту. В 2016-м вы
уже заслужили «Золотой ледоруб» за восхождение на Талунг, и я не удивлюсь, если
эта награда по итогам 2021 года вновь достанется вам. Просто буду держать
кулаки.

- Вторая констатация.

Сейчас с коммерческой попсовизацией альпинизма, не приучающей людей в горах
ДУМАТЬ, любое восхождение можно купить. От клиента требуется лишь здоровье и
деньги. Из технических навыков – только жумар толкать по верёвке. Об остальном
- думают гиды. Клиент получает фоточки, имидж «покорителя», слывёт крутым
альпинистом. И количество таких мастеров жумаринга (МЖ, МЖМК в отличие от МС и
МСМК) от года к году всё больше (кстати, забыл поздравить Виталия Кличко с
«покорением» Казбека – сейчас это делаю. Жаль на вершине поздравить не смог –
был бы в тренде). И я понимаю, что слывущие уже авторитетами в альпинизме,
выросшие в оллинклюзиве «покорители» пяти-семи-восьмитысячников ледоруб от
ледобура ещё отличат. Но ПОНИМАТЬ горы не будут, и научить никого не смогут
никак. Потому что не прошли школу без нянек.

С вами – иное. У вас приобретен опыт планирования и реализации автономных
самостоятельных восхождений. Самостоятельного распознавания опасностей, рисков
(что считалось обязательным в альпинизме моего времени, а сейчас приходится
этому удивляться). Вы можете других НАУЧИТЬ ПОНИМАТЬ ГОРЫ. Вы – МЫСЛИТЕ в
горах. Вот это – сверхважно. Инструкторы альпинизма. По факту.

- Констатация третья.

Мировой альпинизм – мир имён, а не «корочек». И в историю альпинизма вы вписали
свои имена.

«Полежайко, Фомин, Балабанов решили задачу, которая не поддавалась решению
сорок лет». Это – страница истории. Ваша персональная. Именная.

«Полежайко, Фомин, Балабанов – те, кто совершили первопроход ЮВ гребня
Аннапурны ІІІ».

Всё, точка. Вас не нужно представлять титулами. У вас ИМЕНА в мировом
альпинизме.

Так что примите скромные, но искренние поздравления от старого китайца,
сидящего на берегу реки и наблюдающего за течением жизни. Традиционное, краткое
- с горой, парни!

***

Будущих супер-маршрутов в жизни вам уже нажелали. А я пожелаю одно. Пожелаю
всегда возвращаться. «Супер» будет гора или «не супер». Гималаи, Карпаты – без
разницы.

Для меня «супер» - вы, личности. И «супер» - когда все возвращаются. 

Жму руку вам, всем троим. Жму крепчайше.
