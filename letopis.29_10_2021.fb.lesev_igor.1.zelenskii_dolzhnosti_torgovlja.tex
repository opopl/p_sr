% vim: keymap=russian-jcukenwin
%%beginhead 
 
%%file 29_10_2021.fb.lesev_igor.1.zelenskii_dolzhnosti_torgovlja
%%parent 29_10_2021
 
%%url https://www.facebook.com/permalink.php?story_fbid=4727900350574364&id=100000633379839
 
%%author_id lesev_igor
%%date 
 
%%tags korupcia,politika,strana,ukraina,zelenskii_vladimir
%%title Как у Зеленского торгуют должностями
 
%%endhead 
 
\subsection{Как у Зеленского торгуют должностями}
\label{sec:29_10_2021.fb.lesev_igor.1.zelenskii_dolzhnosti_torgovlja}
 
\Purl{https://www.facebook.com/permalink.php?story_fbid=4727900350574364&id=100000633379839}
\ifcmt
 author_begin
   author_id lesev_igor
 author_end
\fi

Как у Зеленского торгуют должностями

У нас бытует устойчивое мнение, что украинцы, идя на выборы, регулярно
демонстрируют свою, ну скажем так, туповатость. Из всех вариантов выбирают
самый худший, а если все варианты одинаково плохи, придумают другой, который
будет кратно еще более отвратный. И все же это не совсем так.

Прямо скажем, в самоорганизации мы с вами нихера не преуспели. Мы, конечно, не
сомалийцы. Те, напомню, пошли в разнос, как и мы, в славном 1991-м. И до сих
пор демонстрируют постапокалипсис в отдельно взятом регионе, где «одна страна»
нарисована только на карте.

\ifcmt
  ig https://scontent-lhr8-1.xx.fbcdn.net/v/t39.30808-6/250925068_4727900057241060_8167590118251696571_n.jpg?_nc_cat=109&ccb=1-5&_nc_sid=730e14&_nc_ohc=mF44jlw0Kz8AX_heYYJ&_nc_ht=scontent-lhr8-1.xx&oh=7af9daa32a24fd8135b8cfd0a32abfe8&oe=6183E86F
  @width 0.4
  %@wrap \parpic[r]
  @wrap \InsertBoxR{0}
\fi

Но сравните современную Украину и, скажем, ФРГ образца 1975 года. Немцы через
30 лет после войны. Гигантские территориальные потери. И это ведь не только
условные ГДР. Австрия, Силезия, Померания, Восточная Пруссия – это все были
легально признанные мировым сообществом земли Германии. И там везде в 1945-м –
руины. А что не развалено – вывезено. И репарации. И оккупация. И унизительные
ограничения.

И вот ФРГ-75. Площадь меньше 250 тыс. кв. км. Украина без Крыма и части
Донбасса почти в два с половиной раза больше. Но уже в 1975 – это 4-я экономика
в мире. Больше, чем у победителя Британии, и больше, чем у нарисованного
победителя Франции.

Короче, мы – не немцы. Но ведь и не албанцы, и не сомалийцы. И не пуштуны. Я
все-таки не могу представить даже самого отмороженного нарика из Украины,
который будет держаться за шасси улетающего самолета.

Но вот как-то все равно ничего не клеится. 30 лет просто насмарку. Если бы мы
были заводом или фирмой, Украину уже б 50 раз закрыли. У стран-неудачниц просто
нет такой опции. Иначе вся Африка, за исключением 5-6 стран, там была б уже
закрыта.

Еще раз, вопросов к коллективному украинцу тоже хватает. Самое нечитающее в
Европе сообщество. А еще сообщество, в котором в 21 веке с умными заточками
отрицают вакцинацию. Или сообщество, где вприпрыжку отдают свои суверенные
институции – а это основа любой государственности – под внешнее управление.
Вопросы есть. Потому что в головах очень много критически опасных вавок.

Но есть и объективные причины, почему у нас из коллективного украинца делают
эдакого туповатого обывателя. Коллективный украинец отстранен от реального
управления государством. Ну или почти отстранен. Исключения составляют только
города, в которых мы имеем возможность напрямую выбирать мэра. Все остальное –
это потемки в назначениях. И эти потемки еще и активно продаются.

По сути, элитка в стране после очередных выборов получает право на 5-летнюю
синекуру. А дальше уже столуется на свое бесстыжее усмотрение, где интересы
аборигенов рассматриваются по остаточному принципу. Да и то вынужденно, чтобы
быдло не начало неуправляемо быковать.

Приведу несколько примеров, как в стране Зеленского торгуют ключевыми
должностями. Самый свежий – довыборы в Раду по Херсонской области. Там лидирует
представитель местного боса-коммерса Колыхаева Геннадий Лагута. От «слуг» идет
губернатор Козырь, шансы которого на фоне Лагуты – нулевые. И «вдруг» Офис
назначает Лагуту губернатором Херсонской области, а сам новоназначенный на
брифинге поддерживает кандидатуру «слуги» Козыря.

История и так мерзкая сама по себе. Персонажи насрали на своего же избирателя.
Представьте в той же Германии нечто подобное. Там бы политическая карьера таких
хитрожопов закончилась моментально. Собственно, потому немцы и строят Германию,
а не Украину. Но это ведь еще и чистой воды коррупционная история. Здесь явный
конфликт интересов и явная продажа должности в обмен на возможность заполучить
в Раде еще одну зеленую кнопочку. Это – политическая коррупция. А возможно и не
только политическая.

Другой пример. Осенью прошлого года Зеленский назначает председателем
Черниговской ОГА 29-летнюю Анну Коваленко. У девочки профильное образование
«театроведение». Согласитесь, чертовски необходимая специализация для
губернатора. А еще Аня удачно вышла замуж за генерал-полковника Руслана
Хомчака, который на момент ее назначения был главнокомандующим ВСУ. Но летом
увольняют Хомчака, а неделю спустя и Коваленко лишается своего кресла. Вот как
думаете, был в назначении Коваленко элемент непотизма? Или она лишилась своего
поста только потому, что стала путать пьесы Уильяма Шекспира с Теннеси
Уильямсом?

Черкасская область. Там губернатором на момент назначения становится еще один
29-летний пацанчик Саша Скичко. Финалист «Минуты славы», ведущий разных шоу, в
общем, кто смотрит дур-ТВ, вполне знаком с этим парнягой. А еще Саша хорошо
женился на дочери долларового мультимиллионера Леонида Юрушева, который сегодня
во многом и заправляет Черкасской областью. И вот уже Сашко в стране Зеленского
становится губернатором этой самой области. С-совпадение, или пацан просто
талантлив сам по себе? Конечно же, талантлив. Год назад перед самым назначением
Саша не сумел в интервью назвать годовой бюджет Черкасской области. Только
настоящие таланты могут себе подобное позволить.

И пару анонсов предстоящих назначений. Министром обороны у нас в ближайшее
время планируют назначить… Алексея Резникова. Редкий прохвост, который потерся
и с Катеринчуком, и с Власенко, и с Кличко. Сейчас трется с Зеленским и нет
сомнения, что дальше будет тереться с кем-то еще.

И все же Резников – это юрист с очень специфическим шлейфом. Последние два года
он отвечает типа за реинтеграцию Донбасса. Достижения грандиозные. Если бы не
название его министерства, никто бы и не знал в стране, что у нас происходит
возврат Донбасса. И все же – где Резников, а где армия? Каким боком он вообще к
ней? И не потому ли его назначают, что оборонка – это более 300 ярдов бюджета,
которые удобнее окучивать через юриста-схематозника, а не через какого-то
кадрового военного?

Да, а на место Резникова планируют поставить Ирину Верещук. Если ее сравнивать
с Резниковым, то, собственно, ничего критично провального не произойдет. Трудно
развалить то, что и так не работает. И все-таки, в каком месте Верещук, а в
каком – интеграция Донбасса? Человек работал мэром Равы-Русской (наверное,
превратила городок в нечто среднее между провинцией Швейцарии и Франции), затем
попала в «слуги» и попыталась стать мэром Киева, выяснила, где находится
Приорка, полетала на зонтике и собрала на мэрских выборах потрясающие 5\%
голосов киевлян. Отличный путь женщины-воина. И вот теперь ее поставят
«реинтегрировать Донбасс». Такая заковыристая зеленая логика.

Собственно, так у нас построена вся система власти. Назначения не просто
непрозрачные – они вызывающе идиотские и откровенно коррупционные. И если бы
такая система работала в той же Германии, немцы точно так же за одно поколение
разучились создавать наукоемкую продукцию, начали бы выращивать рапс и вели б в
соцсетях дискуссию о мировом вакцино-заговоре. Сообщество, которое не принимает
участия в управлении и распределении благ страны, превращается в созерцателей
своего же падения.

\url{https://t.me/Lesev_Igor}
