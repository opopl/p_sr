%%beginhead 
 
%%file 12_01_2023.fb.dedova_natalia.mariupol.1.nadezhda_sukhorukova
%%parent 12_01_2023
 
%%url https://www.facebook.com/permalink.php?story_fbid=pfbid0WbmCvc6wqdEEf64P8muJqA7X7iXvSJijKHkRaBzJ7kYT7PA7i3LdceapkuBCLzggl&id=100007662284921
 
%%author_id dedova_natalia.mariupol,suhorukova_nadia.mariupol
%%date 12_01_2023
 
%%tags mariupol,dusha,chelovek,tragedia
%%title Надежда Сухорукова - Мы - бывшие смертники, пережившие второе рождение
 
%%endhead 

\subsection{Надежда Сухорукова - Мы - бывшие смертники, пережившие второе рождение}
\label{sec:12_01_2023.fb.dedova_natalia.mariupol.1.nadezhda_sukhorukova}
 
\Purl{https://www.facebook.com/permalink.php?story_fbid=pfbid0WbmCvc6wqdEEf64P8muJqA7X7iXvSJijKHkRaBzJ7kYT7PA7i3LdceapkuBCLzggl&id=100007662284921}
\ifcmt
 author_begin
   author_id dedova_natalia.mariupol,suhorukova_nadia.mariupol
 author_end
\fi

\#міймаріуполь

\#війна

\#надежда \#маріуполь

Надежда Сухорукова

Мы - бывшие смертники, пережившие второе рождение.

Все, кто не умер в Мариуполе, пытаются выжить в обычном мире.

Война разбросала нас по разным городам и странам.

Если устроить перекличку, то голоса мариупольцев зазвучат отовсюду.

Даже с вражеской территории.

Это вызывает удивление и ярость.

Они знают, как нас убивали рашисты, почему же отправились на темную сторону?

Я поняла, что никогда не смогу передать словами весь мариупольский  ад. 

Рассказать о нем невозможно.

Я не знаю, как будет выглядеть на бумаге физическое чувство безысходности.

Как объяснить, что означает апатия смерти? 

Когда ты на грани, между небом и землей, а вокруг - почерневший от горя и
страха город.

Крыши домов поднимаются от взрывов, а твоя "крыша"  съезжает от страха.

Это состояние передал Эдвард Мунк.

На его картине "Крик" изображена моя душа, в окруженном рашистами  Мариуполе. 

Многотонные бомбы, бесконечные снаряды, подлые мины и постоянный дикий крик от
одного конца города до другого. 

Кричат те, кто умирает. 

Кто пытается выжить. 

Кому страшно и у кого больше нет сил справляться с молчанием.

Кричит от отчаяния тот, кто сдается смерти, и, от напряжения, тот, кто еще
борется.

Шепчут те, кто медленно угасает от голода и обезвоживания.

Причитает мать, потерявшая ребенка. 

Плачет малыш над мертвой матерью. 

Ему очень холодно, а она уже не обнимает и не согревает его.

Невозможно передать словами страх, холод, отчаяние, ночной кошмар, нежелание
видеть реальность. 

Слова не смогут сжать ледяными пальцами сердце и остановить взгляд  на одной
точке.

На точке между ужасом жизни и страхом смерти.

В Мариуполе не было сна. 

Не было тишины. 

Не было света. 

Солнце закатилось за обугленные дома. 

В Мариуполе не было реальности. 

В Мариуполе был крик.

Кричали наши души.
🖤

😶 Останнє худі, яке ми купили Сашкові, було саме з цим принтом. Купували в
серпні 2021, в Києві, коли були у відпустці. Мені не сподобався цей вибір Віті
та сина. Недобра картинка... 

😶 Остання вистава, яку ми дивилися разом, була "Крик", Драмком. 5 лютого. Тоді
і зробили це фото вдвох. Останнє. 

🥀
