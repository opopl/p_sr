% vim: keymap=russian-jcukenwin
%%beginhead 
 
%%file slova.realnost
%%parent slova
 
%%url 
 
%%author 
%%author_id 
%%author_url 
 
%%tags 
%%title 
 
%%endhead 
\chapter{Реальность}

%%%cit
%%%cit_head
%%%cit_pic
%%%cit_text
І з часом починаєш помічати її вплив на те, що відбувається в твоїй
\emph{реальності}.  Не їсти після 18-ої, уступати місце старшим, не матюкатись,
кожного дня робити зарядку, не їсти мучного, не давати хабарів, не перевищувати
швидкість, не обманювати, відкладати 10\% з кожного заробітку, перейти на
українську, ніколи не запізнюватись, не звинувачувати нікого, не відповідати
злом на зло тощо – будь-яке рішення поводити себе певним чином може стати
фактором, що трансформує наше життя.  Сюди відносяться всілякі Кодекси честі
всіх часів і народів, самурайство в будь-якій сфері, релігійне подвижництво,
яке давало можливість будувати та трансформувати своє життя, – як персональне,
так і соціальне
%%%cit_comment
%%%cit_title
\citTitle{Чи можна примусити збірну України розмовляти державною мовою?}, 
Євген Лапін, www.pravda.com.ua, 06.07.2021
%%%endcit
