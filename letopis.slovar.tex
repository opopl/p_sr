% vim: keymap=russian-jcukenwin
%%beginhead 
 
%%file slovar
%%parent body
 
%%url 
 
%%author 
%%author_id 
%%author_url 
 
%%tags 
%%title 
 
%%endhead 
\chapter{Словарь}

\url{http://slovardalja.net/word.php?wordid=15339}
МАЙДАН

МАЙДАН м. площадь, место, поприще; | возвышеная прогалина и | стоящий на ней
лесной завод: смолокурня, дегтярня, поташня, смолевой, селитряный майдан,
завод, работающий на воле; см. буда; | сборное место; | станичная, сборная
изба; | охотничья хижина в лесу, на прогалине; | ниж. суводь, водоворот на
широком плесу; | торг, базар, или место на нем, где собираются мошенники, для
игры в кости, в зерн, орлянку, карты, откуда пожеланье: талан на майдан! что в
Сиб. значит: удачи на ловлю! | Южн. курган, древняя могила. Майданище ср. род
городища, особ. в лесу, или где был лес, и будний, смолевой майдан. Майданный,
к майдану относящ. Майданный курган, могила, разрытый, раскопанный сверху, с
котловиною. - подъямок, род ларя, для стока смолы, под курною печью. Майданить,
майданничать, мошенничать, промышлять игрою; | мотать, прогуливать и
проигрывать свое. Майданник, майданщик, мошенник, шатающийся по базарам,
обыгрывающий людей в кости, зерн, наперсточную, в орлянку, в карты. На всякого
майданщика по десяти олухов. Не будь олухов, не стало б и майданщиков.

