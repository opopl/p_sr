% vim: keymap=russian-jcukenwin
%%beginhead 
 
%%file 02_06_2018.stz.news.ua.mrpl_city.1.vokzal_mariupolja
%%parent 02_06_2018
 
%%url https://mrpl.city/blogs/view/vokzal-mariupolya
 
%%author_id burov_sergij.mariupol,news.ua.mrpl_city
%%date 
 
%%tags 
%%title Вокзал Мариуполя
 
%%endhead 
 
\subsection{Вокзал Мариуполя}
\label{sec:02_06_2018.stz.news.ua.mrpl_city.1.vokzal_mariupolja}
 
\Purl{https://mrpl.city/blogs/view/vokzal-mariupolya}
\ifcmt
 author_begin
   author_id burov_sergij.mariupol,news.ua.mrpl_city
 author_end
\fi

В те давние времена, когда к Мариуполю еще не была подведена железная дорога,
товары, поступающие в город по сухопутью, доставлялись чумацкими возами. Наш
великий земляк \textbf{Архип Иванович Куинджи} изобразил и возы с поклажей, и волов, и
бредущих рядом чумаков на известной картине \enquote{Чумацкий тракт в Мариуполе}. Она
была написана в 1875 году, а через семь лет завершили строительство участка
Екатерининской железной дороги от станции Еленовка до нашего города.

\ii{02_06_2018.stz.news.ua.mrpl_city.1.vokzal_mariupolja.pic.1}

Тогда же было построено здание вокзала. Сначала возвели его центральную часть,
в которой помещались кассы, зал ожидания, кабинеты начальника и дежурного по
станции, комната телеграфиста. К слову скажем, что к этому служащему железной
дороги мариупольцы приходили сверять свои карманные часы, конечно, только те, у
которых таковые имелись. Постепенно центральная часть обросла пристройками, по
две с каждой стороны. Все помещения соединялись между собой. Законченный вид
вокзал станции Мариуполь приобрел, очевидно, к первому десятилетию XX века.
Свидетельство тому - почтовые открытки с его изображением, относящиеся именно к
этому периоду времени.

Старожилы, вероятно, помнят это одноэтажное вытянутое здание, выкрашенное в
грязновато-зеленый цвет. На его центральной части со стороны перрона был
укреплен начищенный до блеска колокол. Дежурный по станции в красной форменной
фуражке периодически подходил к нему, дергал за веревку, звоном оповещая
отъезжающих и провожающих об отправлении поезда. Августовский номер
\enquote{Мариупольского справочного листка} - издавалась некогда такая газета - за 1900
год сохранил для будущих поколений расписание пассажирских поездов той поры.
Почтово-пассажирский поезд №3 отправлялся до станции Дебальцево в 11 часов 49
минут ночи, а товарно-пассажирский №7 - до Воронежа в 8 часов 50 минут утра.
Ходило еще три поезда до порта и обратно. Как видите, не так уж часто нарушался
покой и тишина на перроне вокзала, если, конечно, не считать проходящих мимо
товарных поездов с углем и пшеницей, следовавших в порт.

\ii{02_06_2018.stz.news.ua.mrpl_city.1.vokzal_mariupolja.pic.2}

Очень шумно было здесь в декабре 1905 года, когда бастовали железнодорожники,
поддержавшие декабрьское вооруженное восстание в Москве, и 22 ноября 1917 года
- в этот день красногвардейцы отбили у своих идейных противников вагон с
оружием. В память об этих событиях на здании вокзала были установлены мраморные
доски с соответствующими надписями. Их почему-то сняли.

У каждого поколения мариупольцев свои воспоминания об этом месте встреч и
расставаний. У кого-то отложилось в сознании, как впервые вступил на землю
незнакомого города с фанерным чемоданчиком в руке и комсомольской путевкой на
строительство \enquote{Азовстали} в кармане. Кому-то запомнилось, как в окне вагона
мелькнуло лицо отца, уезжающего на фронт. Мелькнуло, чтобы исчезнуть навсегда.
Перед внутренним взором другого всплывут картины эвакуации. Баулы, чемоданы,
узлы, нервозная суета, плач малых детей, тревога взрослых, едущих в
неизвестность. А вот то, что осталось в памяти \textbf{Валентины Войцеховской
(Тимченко)}: \emph{\enquote{В 42 году меня, пятнадцатилетнюю девочку, и моих сверстников
угоняли в Германию. Весь перрон был запружен людьми, вокруг гул, рев, крики,
слезы матерей. Все время трехлетнего пребывания в Германии я видела перед
глазами маму, которую поддерживали брат и отец, чтобы она не потеряла сознание.
Еще помню, что кто-то играл на баяне и мальчик семи-восьми лет пел какую-то
печальную песню}}. Никогда не забудут жители близлежащих домов и кварталов,
переживших в военное лихолетье бомбежки и обстрелы станции, здание вокзала,
сожженное во время войны, с кое-как сохранившимся крайним левым крылом, если
смотреть с привокзальной площади.

\textbf{Читайте также:} 

\href{https://mrpl.city/news/view/v-mariupole-zheleznuyu-dorogu-hotyat-ubrat-podalshe-ot-poberezhya}{%
В Мариуполе железную дорогу хотят убрать подальше от побережья, Ярослав Герасименко, mrpl.city, 13.02.2018}

Но не только печальные воспоминания связаны со старым мариупольским
железнодорожным вокзалом. \textbf{Инна Сергеевна Демьянова}, для которой Слободка - ее
малая родина, со светлой грустью поведала о небольшом круглом скверике,
украшавшем некогда привокзальную площадь. В центре его был фонтан - цапля, из
клюва которой вверх вырывалась струя воды, а из четырех лягушек с четырех
сторон исторгались струйки поменьше. Этот небольшой уголок природы был окружен
металлическим заборчиком и бордюром из аккуратно подстриженного кустарника,
почти вплотную к чаше фонтана стояло несколько садовых скамеек. На них теплыми
летними вечерами любили отдыхать обитатели соседствующих с вокзальной площадью
дворов. Вокруг скверика было проложено трамвайное кольцо, на котором время от
времени появлялся трамвай третьего маршрута, связывающего центр города с
вокзалом.

Своеобразной достопримечательностью окрестных мест был ресторан в здании
вокзала. Внутренность его была темновата из-за тяжелых темно-малиновых штор.
Командированные, завершив свои дела в нашем городе, перед отправлением поезда
под рюмочку прохладной водки торопливо ели наваристый украинский борщ и
по-особенному приготовленные котлеты. Здесь распивалось шампанское при проводах
закадычных друзей, сюда приходили после поздних прогулок по Большой - так в
обиходе называли нынешний проспект Мира - пижонистые молодые люди, чтобы
освежиться бокалом-другим свежего \enquote{жигулевского} пива; а то и чем-нибудь
покрепче.

В начале 70-х годов старое здание вокзала решили заменить новым. Вместо него в
1974 году появилось то сооружение, которое знают наши современники. Архитекторы
задумали его легким и воздушным как внутри, так и снаружи. Стены зала ожидания
на втором этаже украсили своими работами местные художники. Мозаику,
посвященную мужественному труду металлургов, талантливо и с большой экспрессией
выполнили монументалисты \textbf{Валентин Константинов} и \textbf{Лель Кузьминков}.
Противоположная к ней стена была украшена деревянным резным панно художника
\textbf{Алексея Бочарова}, изображающим стилизованную карту железных дорог страны.
Сейчас это панно исчезло.

\textbf{Читайте также:} \href{https://mrpl.city/news/view/mariupol-monumentalnyj-top-10-mozaik-goroda-foto}{%
Мариуполь монументальный: ТОП-10 мозаик города, Яна Іванова, mrpl.city, 13.12.2016}

До событий Революции достоинства поезда из нашего города ходили в Воронеж,
Брянск. Минск, Москву и Санкт-Петербург? Сейчас в расписании движения
пассажирских поездов значатся пять направлений – Киев, Львов, Одесса, Харьков и
Бахмут. Но, как и раньше, здесь дают справки, как доехать до той или иной
станции, где сделать пересадку, продают билеты на поезда ближнего и дальнего
следования, объявляют об их прибытиях и отправлениях. Пусть чуть меньше стало
пассажиров, чем было это раньше, а стало быть, и встречающих и провожающих - на
то есть свои причины, но вокзал, как и прежде, остался местом встреч,
расставаний и... воспоминаний.

\ii{02_06_2018.stz.news.ua.mrpl_city.1.vokzal_mariupolja.pic.3}
