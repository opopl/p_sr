% vim: keymap=russian-jcukenwin
%%beginhead 
 
%%file slova.zhizn
%%parent slova
 
%%url 
 
%%author 
%%author_id 
%%author_url 
 
%%tags 
%%title 
 
%%endhead 
\chapter{Жизнь}

Самый потрясающий наш разведчик — это Дмитрий Быстролетов: его \emph{жизнь}
похожа на авантюрный роман, в котором чего-чего, а приключений хватало,
\textbf{«Разведка уродует души и \emph{жизни}» Советский разведчик добывал
секреты по всему миру. За что его сослали в лагеря?} lenta.ru, 29.05.2021

Почну здалеку. Якось так сталося, що останнім часом в стрічці мені стали
траплятися прекрасні мотивуючи дописи про те, як потрібно перестати боротися за
життя і почати просто \emph{жити}. Автори порівнюють звичку пручатися
обставинам, які неможливо змінити, чи не з вадами, через які людина не може
насолоджуватися \emph{життям}. Таких дописів все більше. Можливо, це викликано
якоюсь внутрішньою втомою. Для людини, яка останнім часом тільки те і робить,
що протистоїть: окупантам, обставинам, постійним особистим втратам це дуже
приваблива пропозиція - припинити спротив та сприймати все як є. Але таке
можливо тільки, якщо проблеми перебільшені, а звичка пручатися скоріше нагадує
вередування з приводу чи без. Не можна насолоджуватися \emph{життям}, коли саме
\emph{життя}, гідність, або свобода під загрозою. Я звикла боротися, тому що
знаю, що безнадія - це не об'єктивна реальність. Це суб'єктивний стан готової
до поразки душі, \citTitle{Попереду виснажлива боротьба. Мені потрібна ваша
підтримка та допомога}, Лариса Волошина, gazeta.ua, 09.06.2021

Life is not a Battle to Win, but a Melody to Sing,
\textbf{Two Steps From Hell - Secret Melody},
youtube, 19.02.2018

%%%cit
%%%cit_head
%%%cit_pic
%%%cit_text
Интересно тут, правда, другое. Новые хозяева \emph{жизни} заимели карманные армии и
ловят рыбку в мутной воде, точно зная, что государство на их «шалости» глядит
сквозь пальцы.  Думают, что схватили Бога за бороду, но по факту держат волка
за уши. Стоит лишь самую малость зацепить старших товарищей за живое и вот уже
«товарищ майор» потрошит ваши кубышки, тайники и схроны.  Не расслабляйтесь,
дорогие «патриоты». Рано или поздно коса найдет на камень и кому-то
влиятельному захочется вытащить из-под полы ваши грязные секретики. И вот тогда
«госбезопасность», которая глядит сквозь пальцы, но очень внимательно,
припомнит ваши имена
%%%cit_comment
%%%cit_title
\citTitle{Не расслабляйтесь, дорогие патриоты! Рано или поздно, коса найдет на камень}, 
Максим Могильницкий, strana.ua, 20.06.2021
%%%endcit

