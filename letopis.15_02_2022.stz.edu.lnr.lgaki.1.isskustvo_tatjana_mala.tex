% vim: keymap=russian-jcukenwin
%%beginhead 
 
%%file 15_02_2022.stz.edu.lnr.lgaki.1.isskustvo_tatjana_mala
%%parent 15_02_2022
 
%%url https://lgaki.info/novosti/iskusstvo-ne-rozhdaetsya-sluchajno
 
%%author_id lgaki
%%date 
 
%%tags donbass,hudozhnik,isskustvo,kultura,lgaki
%%title Искусство не рождается случайно...
 
%%endhead 
 
\subsection{Искусство не рождается случайно...}
\label{sec:15_02_2022.stz.edu.lnr.lgaki.1.isskustvo_tatjana_mala}
 
\Purl{https://lgaki.info/novosti/iskusstvo-ne-rozhdaetsya-sluchajno}
\ifcmt
 author_begin
   author_id lgaki
 author_end
\fi

Послушать художницу Татьяну Малу, выставка работ которой открылась сегодня в
Академии Матусовского, так у нее все – легко и просто. Случайно родилась
авторская техника: делала на линолеуме клише для гравюр и поняла, что можно их
и сами использовать. А если еще золотом клише полить, то лучше видна фактура, а
потеки, пятна добавляют результату необычности и тайны... Любимая работа?
«Невеста»! Потому что родилась сама собой: прикладываешь бумагу к стеклу, и
получается оттиск. Просто же!

\ii{15_02_2022.stz.edu.lnr.lgaki.1.isskustvo_tatjana_mala.pic.1}

На самом деле, конечно же, всё получается «просто» и «само собой» только у тех,
кто не замечает никаких сложностей и объемов проделанного труда, потому что
занят любимым, своим делом. Татьяна Васильевна вначале окончила Луганское
художественное училище, потом полиграфическую академию во Львове, начала
строить свою профессиональную карьеру в 2000-м, в 2008-м была принята в Союз
художников, участвовала в международных выставках, представила не одну
персональную, в том числе за границей. И к сегодняшнему дню она – кандидат
педагогических наук, доцент кафедры графического дизайна нашей Академии.
Нынешняя выставка ее работ в стенах нашего вуза – первая, в ЛГАКИ она преподает
с 2020 года.

\ii{15_02_2022.stz.edu.lnr.lgaki.1.isskustvo_tatjana_mala.pic.2}

— И мы очень рады, что у нас есть такой педагог, такой художник, — подчеркнула,
представляя автора выставки, декан факультета изобразительного и
декоративно-прикладного искусства Академии Наталья Феденко.

\ii{15_02_2022.stz.edu.lnr.lgaki.1.isskustvo_tatjana_mala.pic.3}

На полотнах в выставочном холле большей частью – женские портреты, колоритные,
экзотичные, очень интересные.

— Мне всегда были интересны разные народы, представители разных
национальностей, — рассказала Татьяна Мала. — Всегда хотелось в портретах
передать этот их особый характер, харизму, самобытность.

Насколько здорово это получилось, приходите посмотреть в Академию Матусовского
в корпус на Красной площади, 7.

Фото – Марина Машевски.
