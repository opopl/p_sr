% vim: keymap=russian-jcukenwin
%%beginhead 
 
%%file 17_04_2021.fb.backandalive.1.kladbische
%%parent 17_04_2021
 
%%url https://www.facebook.com/backandalive/posts/1984052861752141
 
%%author 
%%author_id 
%%author_url 
 
%%tags 
%%title 
 
%%endhead 

\subsection{Коли батьки ховають дітей — це справжня трагедія, так просто не має бути}
\label{sec:17_04_2021.fb.backandalive.1.kladbische}
\Purl{https://www.facebook.com/backandalive/posts/1984052861752141}

Коли батьки ховають дітей — це справжня трагедія, так просто не має бути. 

\ifcmt
  pic https://scontent-frt3-1.xx.fbcdn.net/v/t1.6435-9/174731396_1984052141752213_1125584058487368060_n.jpg?_nc_cat=106&ccb=1-3&_nc_sid=730e14&_nc_ohc=6wUNmHluDcYAX9BrWTf&_nc_ht=scontent-frt3-1.xx&oh=e76493d5144ee15d7aca3c40183a0d96&oe=60A0A5B9
\fi

Складно собі уявити почуття рідних, які втратили дитину через те, що якісь
покидьки вирішили погратися у реставраторів імперії й прийшли зі своїми
божевільними ідеями «русского міра» до тебе додому під звуки танкових гусениць
та вибухи «Градів». 

На фото батько двох українських героїв, чернівчанин Юрій Мамчій. Він прийшов на
могилу до свого сина Станіслава, якого російські окупанти вбили у 2016 році
поблизу Авдіївки. 4 дні тому ворожий обстріл забрав життя другого його сина —
Олексія. 

Немає жодних слів, які здатні це описати — є лише невимовний біль, який ніколи не вщухне…

Від початку квітня ще 9 українських воїнів загинули у боротьбі за свободу та
незалежність України. Ми схиляємо голови перед їхнім подвигом. 

Вічна слава вам, наші герої! Ми ніколи не забудемо і не пробачимо!

Фонд \enquote{Повернись живим}
