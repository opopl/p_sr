% vim: keymap=russian-jcukenwin
%%beginhead 
 
%%file 14_05_2021.fb.klimkin_pavel.1.ukraina_antirossia
%%parent 14_05_2021
 
%%url https://www.facebook.com/pavloklimkin.ua/posts/1442812996066052
 
%%author 
%%author_id 
%%author_url 
 
%%tags 
%%title 
 
%%endhead 

\subsection{Анти-Россия - Путин - Климкин}

Слова Путіна сьогодні про нас як \enquote{анти-Росію} не варто недооцінювати.
Це не чергова знахідка спічрайтерів. Путін каже про нас як постійне джерело
проблем для Росії та її антипода. Саме такої лексики він ще не застосовував.
Відповідь Кремля буде асиметричною та дуже жорсткою.

Юрий Черняк

Україна може існувати тільки в форматі анти-Росії. В інших форматах то вже буде не Україна.

Soy Sahar

Юрий Черняк какие еще государства в истории долго и эффективно существовали с одной лишь целью - быть Анти?

Glib Ivanov

Це перемога. І без тролінгу. Бо до цього було - одін народ/браття/трі
сестри/общяя історія. Ми переходим в сет-ап північна-південна корея, де ми
південна. І це добрий сигнал.

Игорь Дорошенко

Glib Ivanov це не перемога і не зрада. Це риторика Кремля до якої треба
ставитися дійсно дуже обережно. Анти-Росія, це все одно ототожнення з Росією.
Ми не можемо бути Анти-Росією, бо ми не є Росією від слова зовсім. Ми Україна і
новий штапм як анти-росія це ще один засіб маніпулювання свідомістю.

Glib Ivanov

Игорь Дорошенко це відміний штамп. Який тільки полегшує нашу soft power і для внутрішнього і для зовнішнього споживача. Ізраїль анти палестина/антиіран, Південна Корея анти Північна, і так далі. Як показує практика після того як ти стаеш «анти» і «прихвостнем сша» то все йде добре. Та ж Корея і Ізраїль це доводять. А от коли ти країна друг, братній народ, одін народ і тд з Рашкою то ти стаєш Венесуєлою, Кубою і іншими сірими зонами, де в прямому сенсі нема чого їсти

Игорь Дорошенко

Glib Ivanov ну скоріше за все, мені це видається маніпуляцією. Бо приставка
"анти" вона формує негатив. Тобто, в цьому дискурсі слово Україна не
вживається, а якась Росія і "Анти-Росія". Україна де?

Glib Ivanov

Игорь Дорошенко я не знаю для кого «анти» в конексті Росії формує негатив. Це
коли казати що для сирійного вбивці, поліція це анти серійний вбивця. Коли тебе
країна террорист, з манікальною владою, і дрімочою внутрішною аджендою називае
«анти» це як в топ список потрапити, поруч із США і Чехію яких сьогодні
офіційно РІШЕНЯМ кабміну затвердили як держав ворогів(((

Игорь Дорошенко

Glib Ivanov я дуже сумніваюся. Добре, вважайте так, як вам подобається. Але ж
знаєте.... кремлівська пропаганда вона ж така підступна.

Glib Ivanov

Игорь Дорошенко ну краще бути ворогом росії, в офіційній методичці, ніж братнім
народом. Бо що стало з братніми абхазами, русско-говорящіми лугандонцями, і
іншими так би мовити офіційно дружніми народами ми бачимо. А вороги якось
тримаются. Мене більше лякає коли пропаганда кидає сюди наратив що українці
гарні, чемні і браття росіан, а українська держава (влада) фашисти і нас просто
треба звільнити. От мені лячніше від того що нас будут звілняти, ніж від того
що нас назовуть ворогом

Игорь Дорошенко

Glib Ivanov ну майже так. Тільки я трохи не про це. Це закладає в масову
свідомість те, що існує тільки Росія і якась вигадана Анти-Росія. Тут мета
вибити з пропаганди Україну, бо згадки про Україну - це підтвердження факту
існування України, з яким Путін не може змиритися. Тому для пропаганди замість
назви існуючої суверенної держави Україна вони вигадали Анти-Росію.

Игорь Дорошенко

Проти суверенної держави Україна воювати якось не те. Тому треба вигадати щось
нове. І тут з'являється "Анти-Росія". І вони підуть воювати і вбивати вже не
Україну а якусь там "Анти-Росію". Це ж багато чого в свідомості змінює. Тим
паче в свідомості ватників, москвофілів та самих громадян Росії.
