% vim: keymap=russian-jcukenwin
%%beginhead 
 
%%file topics.vojna.my.11.filmy.2.beautiful_mind
%%parent topics.vojna.my.11.filmy
 
%%url 
 
%%author_id 
%%date 
 
%%tags 
%%title 
 
%%endhead 

\section{О Фильме Игры Разума (Beautiful Mind)}

Недавно я посмотрел и даже несколько раз пересмотрел прекрасный фильм Игры
Разума (Beautiful Mind - буквально, Прекрасный Разум - 2001 год - удостоенный
Оскара).  Это фильм про математика-ученого, Джона Неша, реально жившего,
который известен в науке, тем, что создал теорию игр (game theory),
математическую теорию, которая ныне активно используется во многих областях
человеческой жизни, за которую ему в 1994 присудили Нобелевскую Премию по
экономике.  Джон Неш также известен тем, что на какой то период своей жизни он
заболел шизофренией, но он смог побороть в итоге свой душевный недуг, в
конечном итоге вернувшись к активной жизни в науке. Вот про жизнь Неша в науке,
а также про то, как Неш заболел, боролся со своим недугом, и как он научился
справляться со своей болезнью, и был снят этот фильм. А касательно содержания
фильма, и впечатлений после просмотра, я перейду немного позже, сейчас хотелось
бы сделать небольшое философское отступление по поводу роли точных наук, и в
частности математики, в нашей современной жизни.

... Говорят, что математика - это царица наук, и это действительно так, хотя
тут есть парадокс, потому что математика - это в каком то смысле вообще не
наука, если принять точку зрения, что наука - это все, что проверяется и
подтверждается опытом, и что наука исследует явления природы. Ну ладно,
математика царица наук или нет, это не так важно, как важно то, что математика
лежит в основе нашего сегодняшнего понимания мира, потому что физические
законы, которые управляют всем, что происходит вокруг, выражены через
математические уравнения и формулы. И исторически развитие наших представлений
об окружающем мире, шло параллельно с развитием математики.  Более того,
технологии, которые мы пользуемся каждый день, даже например простой
выключатель света, покоятся на действии законов физики, записанных
математическим языком, как например уравнения электродинамики Максвелла.  В
общем, математика повсюду, она не видна совершенно, ее не замечают, но она
работает - и в том, как работает компьютер или смартфон, и в том, как и почему
движется вагон Киевского метро, и почему лифт в нашем доме едет вверх или вниз,
и почему, если набрать ванну полную воды в своей квартире, а потом плюхнуться
туда, то весь пол будет мокрым, и почему, если бросить яблоко с балкона нашей
квартиры на десятом этаже, оно неизбежно упадет через несколько секунд на
асфальт, громко шмякнувшись, и до смерти испугав рядом проходящую старушку, и
даже также в том, почему самолет взлетает, когда мы летим в отпуск в Египет,
хотя казалось бы, если он такой тяжелый, так как он способен летать? 

\ifcmt
  tab_begin cols=3,no_fig,center

     pic https://i2.paste.pics/fa62372d549789835275d055915e7ef5.png
		 @caption Исаак Ньютон, английский физик и математик (1642 - 1727)

		 pic https://i2.paste.pics/b0451b1c82d51957a71bfad9bb089f0c.png
		 @caption Джеймс Кларк Максвелл, шотландский физик и математик (1831 - 1879)

		 pic https://i2.paste.pics/60a463d15e5875fe82582334fa50512f.png
		 @caption Людвиг Больцман, австрийский физик-теоретик (1844-1906)

  tab_end
\fi

Вот...  математика и физика лежит в основе нашего сегодняшнего технологичного и
цифрового мира, и без Исаака Ньютона, открывшего закон всемирного тяготения,
Джеймса Кларка Максвелла, записавшего уравнения электродинамики, или же Людвига
Больцмана, внесшего вклад в понимание законов термодинамики, а также многих-многих других
ученых, внесших самый разнообразный и неоценимый вклад в копилку наших знаний о
законах, управляющих поведением природных явлений, мы бы по прежнему бы бегали
туда-сюда с факелами, свечками, ездили бы на конях, и также мерли бы пачками
немеряно от оспы, чумы, и прочих напастей, как это и происходило например в
Средневековье.  А чтобы поехать из Винницы, скажем в Киев на паломничество к
Святым Церквям, да, как это было в столетии этак восемнадцатом, нам бы пришлось
трястись целую неделю в бричке, потому что купить билет на поезд Винница-Киев
было никак нельзя, потому что и поездов, и вокзалов тогда никаких не было! А
насчет того, чтобы созвониться по скайпу со старым школьным товарищем, ныне
загорающим на своем собственном ранчо в Техасе, так речи вообще не было! Так
что слава математике! Слава физике!  Слава точным наукам, и ученым, благодаря
которым у нас есть столько всяких технологий, облегчающих и делающих приятней
нашу жизнь!

... А теперь, после такого философско-исторически-лирического отступления, (
которое возможно некоторым покажется довольно забавным и смешным ) перейду
собственно к фильму.

Фильм, как уже было сказано, о жизни американского математика Джона Неша.
Начало отводит нас к тому времени (конец сороковых годов двадцатого столетия),
когда Неш приезжает в первый раз в Принстонский Университет (Princeton
University), один из самых известных и мощных университетов в США, по
стипендии, чтобы учиться на степень доктора философии (PhD), то есть кандидата
наук по нашему.  Здесь он встречает своих будущих друзей Бендера, Сола, и также
Мартина Хансона, - все они математики. Он также попадает на приветственную
лекцию декана факультета, где декан говорит о важности математики и точных
наук, а также экономики и медицины в начинающемся противостоянии и холодной
войне глобальных сверхдержав - США и СССР.  Лекция про то, что математика
помогла выиграть вторую мировую войну (взлом нацистских шифров), и про то, что
противостояние между сверхдержавам происходит не только в плане гонки
вооружений, но также по линии интеллектуальных свершений на ниве точных наук, и
что математики в Принстоне должны сыграть важную роль в этой новой холодной
войне. 

\ifcmt
  ig https://i2.paste.pics/a1c92ce3708fd5f3d08de3a581df3a62.png
  @caption Математики выиграли войну! Математики разгадали японские шифры и создали атомную бомбу. Математики - вроде вас.
  @wrap center
  @width 0.8
\fi

... Далее, Неш, молодой аспирант, уже имеющий репутацию гения, отправляется
обустраиваться в свою комнату в университетском общежитии. Далее происходит
такая сцена. Неш, весь в мечтах о своих будущих научных свершениях, и горящий
желанием тут же начать творческую работу над своими идеями, в своей комнате
придвигает свой письменный стол к окну, потом садится за него, намереваясь как
следует поработать.  Он видит, что снаружи внизу в университетском дворе
собирается команда студентов поиграть в мяч, есть две команды, которые
расходятся по разные стороны перед началом игры. Неша интересует математическое
осмысление движения команд, его ключевая мысль, то, что его занимает все время
- управляющая динамика - теория, которую ему только предстоит разработать.  Так
вот, Неш придвигает письменный стол к окну, смотрит на команды, и начинает на
окне рисовать математическое отображение того, что происходит снаружи между
командами - линию, разделяющую визуально команды, какие то другие знаки и
символы.  Он поглощен этим занятием, все его мысли об этом. 

\ifcmt
  tab_begin cols=3,no_fig,center

		 pic https://i2.paste.pics/c539b7cb15972a8126ed817f671437e7.png
		 @caption Чарльз вваливается в комнату Неша, - давай поделим сию келию на двоих!

		 pic https://i2.paste.pics/090cffb8d99ee7998c636bd30c8ecfad.png
		 @caption Неш удивленно смотрит на незваного развязного гостя, - на двоих?! только не это!

     pic https://i2.paste.pics/e65bf043c8d68ab79b88767d5e7497c7.png
		 @caption Комната Неша, Неш сидит за столом, позади фамильярный вальяжный Чарльз Гармен

  tab_end
\fi

Внезапно в комнату
врывается какой то молодой полураздетый рубаха-парень, вальяжный весь такой,
навеселе, фамильярный; его зовут Чарльз Гармен, он видит Неша возле окна, и тут
же знакомится с ним.  Неш удивленно на него смотрит, потом садится за стол,
намереваясь продолжать свою математическую работу. Не тут то было!  Чарльз все
время болтает, о каком то приеме, о какой то горячей встрече с какой то
красоткой, в общем, без умолку из него сыпятся слова.  Потом, видя, что Неш на
него не обращает внимание, Чарльз пытается привлечь его внимание, сначала
пытаясь украсть печенье из коробки Неша, стоящей на столе, а когда Неш нервно
захлопывает своей правой рукой эту самую коробку с печеньем, Чарльз вообше
запрыгивает на стол перед Нешем, и насмешливо смотря на него, говорит, - так
оказывается, мой сосед зануда! Ну что ж, надо растопить лед! - И достает
фляжку, очевидно, с каким то крепким напитком внутри.  Интересно, что пьют
студенты в Принстоне при первом знакомстве, - водяру? или коньяк? или может
быть виски или же бренди?  Я не знаю, к сожалению, а фильм об этом умалчивает,
что же именно было внутри этой самой небольшой железной фляжки. 

\ifcmt
  tab_begin cols=3,no_fig,center

     pic https://i2.paste.pics/c35fef1cb2fed886e4f23f45e31876f2.png
		 @caption Неш рисует формулы на окне
		 
		 pic https://i2.paste.pics/69eeed5ffeb48dd4db1105f957a49aae.png
		 @caption Чарльз бегает по комнате, болтает, пытается привлечь внимание Неша

		 pic https://i2.paste.pics/5bca779e8b8e635195b0692d5759d758.png
		 @caption Прелюдия к доверительной беседе на свежем воздухе с фляжкой с неизвестной жидкостью

  tab_end
\fi

Далее... Неш и Чарльз на крыше болтают дружески, а фляжка им в этом очень помогает. Неш признается,
что не любит людей, и что это взаимно, и что его учительница ему когда-то даже говорила, что Джон
родился с двойной порцией мозгов, и в половиной сердца... Он делится своими мечтами о своей научной работе,
взволнованно говорит, что ему нет дела до семинаров, и до тупой зубрежки, ему позарез нужна идея, оригинальная идея,
чтобы состояться в его работе, чтобы его заметили в научных кругах... 

\ifcmt
  tab_begin cols=4,no_fig,center

     pic https://i2.paste.pics/3b55aa0648932b7fe4623d4ee60507bc.png
		 @caption Чарльз - раз трудно сломать лед, может растопим?

		 pic https://i2.paste.pics/d8f97a55d88f71f3439ac74407c43098.png
		 @caption Беседа на крыше

		 pic https://i2.paste.pics/4facd95bbd7a6f0a0b2d63c7071a7cf9.png
		 @caption Беседа на крыше

		 pic https://i2.paste.pics/077c17da891de9424e2f8044d2c0b21b.png
		 @caption Беседа на крыше

  tab_end
\fi

Далее... идет забавная сцена... На улице... Друзья Неша играют в какую то игру
на доске с округлыми шашками, тут рулит веселый Мартин Хансон с сигаретой в
зубах, который все время всех обыгрывает, - кто следующий, - давай!
Проигравшие игроки плюются и встают из-за доски. Тут они все замечают Неша,
который бегает туда-сюда с карандашом и записной книжкой, задом наперед,
туда-сюда... Эй Неш! - что ты делаешь?! - Неш отвечает - я исследую движение
голубей, я пытаюсь разгадать законы движения птиц... Мартин зовет его, давай
сюда, будем играть! 

\ifcmt
  tab_begin cols=3,no_fig,center

     pic https://i2.paste.pics/d106a3652d2412f62eb445889bee52a8.png
		 @caption Мартин Хансон - трусы, все трусы, никто не примет мой вызов!

		 pic https://i2.paste.pics/f7497efe00b9efad49206ec16c252297.png
		 @caption Неш исследует поведение голубей
		 
		 pic https://i2.paste.pics/bc69d1f4217c891d574282e4a188aaf2.png
		 @caption Голуби, движение которых исследует Неш

  tab_end
\fi

Мартин подкалывает Неша, - Неш подавляет всех своим гением, а со мной сразиться
ему смелости не хватает!  Неш, в конце концов не выдерживает и идет сражаться с
Мартином, и что вполне естественно, - потому что Мартин очевидно в ударе, - и
он просто рвет на части всех, кто садится с ним играть, - Мартин выигрывает, -
походу во время игры задавая Нешу ехидные вопросы. Неш, раздраженный,
огорченный проигрышом, рывком встает из-за доски, и нервно, опрокинув доску с
шашками по дороге, уходит, не оглядываясь...

\ifcmt
  tab_begin cols=3,no_fig,center

     pic https://i2.paste.pics/06e098e0879eeba082ded5da924542be.png
		 @caption Мартин вызывает Неша на игру

		 pic https://i2.paste.pics/39c59fd90aeadfa5698f04c6efaf2e08.png
		 @caption Мартин Хансон и Неш играют, друзья смотрят

		 pic https://i2.paste.pics/57abf680a22c8bba94f3dfd3602a716f.png
		 @caption Игра, вид сверху

  tab_end
\fi

Далее... Неш в библиотеке, у окна, снова строчит уравнения... Темно уже, вечер,
тихо в библиотеке... немногочисленные студенты сидят, что то читают, готовятся, изучают,
решают... Мягко светят лампы, тишина... А Неш мучается у окна со своими уравнениями...  все окно уже
исписал... И тут к измученному Джону подходит Чарльз, как будто из ниоткуда
появляется...  Эй, ты уже тут два дня сидишь! - да... Мартин уже статьи
написал, а у меня ничего нет еще, уныло своему другу отвечает Неш... 

\ifcmt
  tab_begin cols=3,no_fig,center

     pic https://i2.paste.pics/653a1ad2b0b393814d2f6038ce692f3b.png
		 @caption Неш в библиотеке

		 pic https://i2.paste.pics/f9d95fb87a3474a921d5d965a2ba809c.png
		 @caption Неш расписывает окна уравнениями, описание стаи голубей, дерущихся за крошки

		 pic https://i2.paste.pics/03207529a601d360df16948de6ca2d3d.png
		 @caption Уравнения на окне, женщина гонится за вором, укравшим у нее кошелек

  tab_end
\fi

Неша мучает здесь такое... из двоих победить может только один, найти бы равновесное состояние...
когда победа не главное... и нет проигравших... как бы это повлияли на конфликты, разницы курсов...
Неш мечтает... Чарльз возвращает его в реальность - Джон, ты когда последний раз ел?
Джон, стрепетнувшись, понимает, что пора завязывать на сегодня с вычислениями, и идет в бар
отдыхать с друзьями.

\ifcmt
  tab_begin cols=3,no_fig,center

     pic https://i2.paste.pics/0fe76420dd04bd81b322e5442d642626.png
		 @caption Джон и Чарльз в библиотеке

		 pic https://i2.paste.pics/630b3868fc184a51caf3106d71227206.png
		 @caption Чарльз - Нешу: когда последний раз ел?

		 pic https://i2.paste.pics/c2c61aa0fc94ec1780315fcdfb49d059.png
		 @caption Неш покидает библиотеку, идет развлекаться: я уважаю пиво!

  tab_end
\fi

Далее, сцена в баре... Джону улыбается какая то красотка в углу бара, его
друзья поднакивают Джону - ей, Неш, смотри, она на тебя смотрит!! Джон,
набравшись храбрости, идет к красотке. И тут конфуз.  Он не знает, что говорить
в таких случаях, молчит... и в итоге выдает пару откровенно глупых фраз, после
чего красотка залепливает ему пощечину, с возгласом - хам!! - и с возмущением
покидает бар с подружками.  Друзья Неша смеются.

\ifcmt
  tab_begin cols=3,no_fig,center

     pic https://i2.paste.pics/ff9fc7b817f5be2b2567ea1362acc50a.png
		 @caption Декан, разговор с Джоном

		 pic https://i2.paste.pics/235fddfa725fafc2ffc4b61880576e6d.png
		 @caption Джон Неш, разговор с деканом

		 %pic https://i2.paste.pics/c237e1bbbd81dba0e2b591954e6ce43c.png
		 pic https://i2.paste.pics/2b71b969f8720fb6347ed06704759e9e.png
		 @caption Сцена признания с ручками

  tab_end
\fi

Далее... в университете... Джон идет вместе с деканом факультета, декан ему
говорит об успехах студентов... кого куда рекомендовать...  декан говорит
Джону... вы прогуливаете семинары... я не могу вас рекомендовать, мне очень
жаль... Отдельно стоит отметить сцену, где Джон с деканом заходят в
университетское кафе и видят такую картину... за столом сидит какой то
уважаемый профессор, ему по очереди подносят и кладут на стол ручки, как знак
признания заслуг в науке... волнующая сцена...  Джон смотрит... декан его
спрашивает - Джон, что вы видите? - Джон отвечает - это признание...

\ifcmt
  tab_begin cols=3,no_fig,center

     %pic https://i2.paste.pics/0e3ad97b09273e68e124c967a1cd2098.png
		 pic https://i2.paste.pics/265273b6356b4ac407c1eb8538d1cdfe.png
		 @caption Джон в отчаянии - я не вижу!

		 pic https://i2.paste.pics/7070e9dc8ed9047703521df07e37e75c.png
		 @caption Джон в отчаянии - я не могу... потерпеть неудачу! 

		 pic https://i2.paste.pics/b3a6a7d39a64c3129e68a4a96468a7d6.png
		 @caption Летящий стол вниз на улицу, бумажки с вычислениями разлетаются в сторону

  tab_end
\fi

Джон, потрясенный тем, что его отвергнули... с новой силой принимается за свои
вычисления...  Уже наступила зима... Джон с своей комнате вовсю трудится...
через силу, через край...  у него не получается... Джон пытается снова и снова,
яростно рисуя уравнения на окнах своей келии... но все равно не получается...
Джон в отчаянии, почти что плачет... я не вижу, не вижу, как решить эту
задачу... Он близок к нервному срыву... Тут на помощь ему приходит Чарльз...
он выводит его из состояния отчаяния, прибегнув к простому способу, просто
нахер выкинуть письменный стол Джона, разбив попутно окна, на которых Джон
рисовал свои уравнения... Стол с грохотом летит вниз, бумажки, испещренные
уравнениями, летят в разные стороны... Чарльз с Джоном со смехом подбегают к
окну и смотрят вниз... Даа, красиво так стол упал, красиво разбился вдребезги!
Джон ожил, он смеется, вместе с Чарльзом, довольно и весело выглядывая из окна на
случайных прохожих - не волнуйтесь, это мой стол!

\ifcmt
  tab_begin cols=2,no_fig,center

     pic https://i2.paste.pics/f8265aafafeb95ad68356bead88ded0a.png
		 @caption В баре, та самая сногсшибательная девушка

		 pic https://i2.paste.pics/3c969817f89a7cd0b6a93184d576e1e6.png
		 @caption В баре, Хансон: вспомни уроки Адама Смита!

  tab_end
\fi

Далее... снова сцена в баре... Джон с друзьями отдыхает, бильярд, пиво,
курево... все такое...  Снова женский вопрос, вездесущий и непременный,
особенно для студентов-математиков Принстона...  Высокая, сногсшибательная
девушка-блондинка появляется на горизонте... с компании своих подружек...
Лакомая добыча! Все глаза сразу же оборачиваются, господа, посмотрите, ойойой!
А Джон... Джон сидит за столиком отдельно, с кипой бумаг, снова таки с новым
азартом, что то вычисляет... Ему говорят, эй, Джон, бросай свои вычисления, тут
есть дело поважней! Джон отрывается от своих бумажек, и они с друзьями садятся
за стол, начинают обсуждать подходы, стратегии... решения этого самого женского
вопроса...  Какую же стратегию выбрать, чтобы заполучить девушек?  Разговор
перескакивает на Адама Смита, отца-основателя экономики - при конкуренции
личные амбиции служат общему благу!  Значит, - добавляет один из собравшихся, -
каждый сам за себя, господа! Другой добавляет, - а отшитым достаются
подружки... Третий добавляет вальяжно, судя по всему, - Мартин, - меня не
отошьют! И дальше, дальше, идет обсуждение, смакование, обдумывание... Ммм, как
же их всех заполучить!  А Неш слушает, слушает... идея созревает! Да, важная
идея созревает! Она наша... у него преимущество, но стоит ему заговорить,
Мартин ехидно говорит насчет Неша, имея в виду предыдущий конфуз Неша, - и ....

\ifcmt
  ig https://i2.paste.pics/b59713ed125a4f4bb8b618a5cfbee53f.png
  @caption Неш объясняет свою революционную идею
  @wrap center
  @width 0.8
\fi

... а Неш тут и говорит, - Адам Смит устарел! Что!? - оборачиваются на него
друзья, - а Неш продолжает, - если все рванем к ней, - то помешаем друг другу,
- и она не достанется никому; тогда мы займемся подружками, и они оттолкнут
нас, - никто не хочет быть вторым сортом. А вот если ее никто не заметит, мы не
будем толкаться, и не оскорбим других девушек, - так мы выиграем! Лишь так
получим женщин! Далее, Неш, вдохновленный, почуяв, нащупав верную идею, в
ударе!  Он растолковывает друзьям, - Адам считал, что лучше всего... когда
каждый член группы действует в своих интересах, - это правда, но не вся! - на
деле, результат будет оптимальный, если каждый член группы сделает как лучше
для себя, и для группы! Хансен ржет в ответ, - если таким способом хочешь
получить ее, то иди к черту! А Неш... А неш, довольный, счастливый Неш, ему в
баре делать больше нечего, плевать на женщин, ведь он придумал идею! Он
собирает бумажки, и бежит из бара записывать, по дороге чуть не столкнувшись с
той самой блондинкой, которая послужила для него зацепкой в его новой
революционной теории! Но кому нужны девушки, если уже есть управляющая
динамика, революционная теория, которая, возможно, перевернет сами основания
экономики, опровергнет самого Адама Смита, отца-основателя экономики! Неш
стремглав бежит в свою келию записывать, записывать...

... И вот, теория готова, записана... изложена в виде статьи... Джон берет
статью, идет к декану... показывает ему... декан перелистывает... Джон в
смущении, нерешительности...  что скажет строгий ученый, первый судья его
работы... не пошло ли все прахом, не разнесут ли его работу сейчас вдребезги? А
декан... перелистывает... смотрит на вычисления...  а потом говорит... знаете,
Джон... с таким размахом вы написали... ну, я уверен, вас возьмут куда угодно!
Полный успех! Джона зачисляют в группу Уилера, ведущую научную группу в МТИ
(Массачусетстком Технологическом Институте), - вместе с его двумя
друзьями-математиками, - Солом и Бендером, за которых Джон просит декана.
Полный успех! А Мартин... ехидный веселый соперник Неша... Мартин останется на
факультете, где в итоге займет место декана, но это уже потом, потом... 

\ifcmt
  tab_begin cols=3,no_fig,center

     pic https://i2.paste.pics/04012b9cdbbd0a18aae4f2c7e723f799.png
		 @caption Неш в Пентагоне с военными

		 pic https://i2.paste.pics/af37c2baaaeebf1440192545732bc650.png
		 @caption Комната в Пентагоне, Неш начинает разгадывать шифр из цифр на большом экране

		 pic https://i2.paste.pics/3c8bd37e2239df607f7c7134e8454d7b.png
		 @caption Неш в процессе разгадывания шифра, цифры на экране

  tab_end
\fi

Пентагон, 1953 год... пять лет спустя... Неш отправляется туда, чтобы помочь
военным в разгадке шифров...  где приводят в комнату, где ему предстоит
разгадать шифр... Целый день он неотрывно смотрит на светящийся экран...  цифры
бегают перед его глазами, складываясь в комбинации... военные ждут его...
ждут... день проходит...  в итоге, Неш приходит к разгадке, и показывает на
карты координаты... предположительно точки проникновения русских в США...  на
вопрос Неша, что перевозят русских, военный ничего не говорит... Неш выводят
обратно из Пентагона...

\ifcmt
  tab_begin cols=3,no_fig,center

     pic https://i2.paste.pics/74223e964b7584a97c88fd538ad0bfcb.png
		 @caption Неш на семинаре, дает задание для студентов

		 pic https://i2.paste.pics/c968cabdeb0a08e181ed8b31654726a4.png
		 @caption Студентка открывает окно, семинар Неша

		 pic https://i2.paste.pics/0176467db6b02dc0c29e409519bf439f.png
		 @caption Рабочие уходят на перерыв после просьбы студентки

  tab_end
\fi

Неш вернулся в МТИ, в свою лабораторию... ему напонимают, что ему также нужно
вести семинары... скучное занятие, которым должны заниматься сотрудники
лаборатории Уилера...  обучать будущие лучшие умы Америки... Джон с отвращением
смотрит на учебник, с которым ему предстоит вести семинар, с большой неохотой
он входит в комнату, где ему предстоит вести семинар... Войдя в комнату, он
выкидывает учебник в корзину, и решает, что проще всего дать очень трудную
задачу студентам, которые многие из них решить вообще не смогут... Выбросив
учебник...  пишет на доске формулы... Далее, момент... На улице какой то шум,
там рабочие что то ремонтируют, а в комнате жарко. Студенты просят открыть
окно, чтобы не так жарко дышалось, а Неш, наоборот, требует, чтобы окно было
закрыто, поскольку иначе его голос - голос преподавателя - не будет слышен как
следует.  Тут поднимается эффектная красивая стройная темноволосая
девушка-студентка - и разрешает ситуацию простым способом - она открывает окно,
и просит рабочих сделать перерыв. Рабочие с радостью соглашаются, и уходят на
обед. Вуаля! Ситуация разрешена, а Неш с усмешкой говорит... что, как известно,
вариационное исчисление подразумевает наличие множества решений... Студентка же
внимательно-внимательно так смотрит на Неша, и мы догадываемся, что
неспроста...


\ifcmt
  tab_begin cols=3,no_fig,center

     pic https://i2.paste.pics/99f0b212060f3498d43a13adad4c860e.png
		 @caption Встреча Неша и Уильяма Парчера возле здания МТИ

		 pic https://i2.paste.pics/dd4db51ff1f809092569386bd355e900.png
		 @caption Неш и Уильям Парчер

		 pic https://i2.paste.pics/6cccbb0bc069dcf489e964bb48f9bec0.png
		 @caption Секретная лаборатория, куда Неша приводит Парчер

  tab_end
\fi

Ну ладно, со студенткой мы еще увидимся, а сейчас следующий эпизод. Вечер, Неш
выходит из здания своего института, и тут сталкивается с каким то типом,
которого он раньше уже видел в Пентагоне, его зовут Уильям Парчер, Старший
Брат.  Он представляется Нешу, и они неторопливо идут, общаясь. Парчер говорит
о том, что Пентагон высоко оценил работу Неша. Также, оказывается, Парчер
раньше курировал работу Оппенгеймера - американского физика-теоретика - по
проекту создания атомной бомбы в Лос-Аламосе (который был инициирован письмом
Альберта Эйнштейна президенту Рузвельту, и который в итоге закончился созданием
атомной бомбы, и атомными бомбардировками Хиросимы и Нагасаки).  Они идут по
направлению к заброшенным зданиям за пределами МТИ, и оказывается, что в этих
заброшенных зданиях на самом деле размещается секретная дешифровальная
лаборатория Парчера.  Неш в недоумении и даже восхищении осматривает
лабораторию - люди снуют туда-сюда, какие то установки, радиограммы постоянно
звучат... Работа на национальную безопастность Америки кипит вовсю... Парчер
приглашает Неша в свой кабинет и там рассказыват детально, что есть угроза...
русские собираются взорвать атомную бомбу на территории США, и что очень важно,
чтобы Неш помог им, - в дешифровке сообщений, которые скрыты в обычных газетах
и журналах. Неш соглашается сотрудничать. Далее, Парчер и Джон возвращаются
назад в основной зал лаборатории, и Парчер объясняет, что нужно делать Нешу.
Нешу также вставляют в руку специальный имплантант с меняющимися светящимися
цифрами, - биологический код доступа. Неш теперь шпион, и работает на
национальную безопастность Америки! Неш... возвращается к себе... начинает
собирать журналы, газеты, все время черкая... исследуя тексты, сообщения,
картинки, рекламу...  пытаясь найти спрятанные зашифрованные сообщения... эта
тайная работа поглощает все его время, занимает его мысли, засасывает его
полностью... Он занимается этим у себя на работе и дома...


\ifcmt
  tab_begin cols=3,no_fig,center

     pic https://i2.paste.pics/a2b7947bb32dc793388840cabde441b3.png
		 @caption Девушка (Алисия) входит в кабинет Неша

		 pic https://i2.paste.pics/a92592a5cdae417e61d75459ed4254b6.png
		 @caption Застанный нежданной гостьей врасплох Неш

		 pic https://i2.paste.pics/e476b7cb86a135d1db1e74239863ee0c.png
		 @caption Алисия показывает решение Нешу

  tab_end
\fi

За этим занятием в его кабинете его и застает та самая очаровательная девушка
(ее зовут Алисия), которая так просто решила проблему с окном на семинаре.... а
сейчас она входит в кабинет Неша...  оказывается, решив ту самую сложную
задачу, которую Неш дал на доске студентам.  Неш в недоумении, и указывает на
огрехи в решении, надеясь избавиться от настойчивой непрошеной посетительницы.
Но не тут то было! Алисии нравится Неш, и она приглашает его поужинать вместе,
в итоге они встречаются на званом вечере, - у них свидание.

Так... детали свидания мы пропустим... кратко... Джон и Алисия встречаются на званом ужине...
куча важных людей, все роскошно, красиво, большой зал... бокалы с шампанским, важные шишки, губернаторы,
все такое... потом Неш и Алисия выходят наружу, уже вечер... смотрят на небо, созвездия, красиво, романтично...
свидание удалось!

\ifcmt
  tab_begin cols=3,no_fig,center

     pic https://i2.paste.pics/b35e12bef8d7db281cd90d0bb2622cf1.png
		 @caption Секретная работа Неша над журналами и газетами

		 pic https://i2.paste.pics/5706a3029f65383b482d93a85c04a89d.png
		 @caption Неш "разгадывает" журналы и газеты

		 pic https://i2.paste.pics/51b53529b77630a23d91ac7b329b753d.png
		 @caption Совершенно секретно!

  tab_end
\fi

Далее... Неш продолжает свою работу над расшифровкой журналов и газет...
секретной работой на правительство США... он вычисляет, вычисляет... потом
результаты своей работы складывает в конверты с пометкой "совершенно секретно",
и с этими конвертами отправляется в какой то особняк возле леса или парка, где
на воротах стоит датчик доступа, который Неш открывает посредством своего
имплантанта, приложив руку к датчику; уже темно... в окнах особняка светятся немногочисленные окна... 
наконец... ворота открываются... где-то гавкает собака.... и Неш торопливо, и
нервно оглядываясь, крадется к почтовому ящику внутри территории особняка;
положив конверты в почтовый ящик, Неш уходит назад через ворота. 

Далее, еще одно свидание Алисии и Неша... уже на берегу озера... день...
красиво все так!  Тут мы тоже опустим детали, в общем, все снова отлично, дело
движется дальше! И неуклюжий и даже тупой как пробка способ Неша ведения
разговоров с девушками абсолютно не помеха Алисии, поскольку она влюблена в
Неша, и ей лишние церемонии ни к чему!

\ifcmt
  tab_begin cols=3,no_fig,center

     pic https://i2.paste.pics/0f286a8967fc224aa9f630b974d07ea8.png
		 @caption Марси, племянница Чарльза Гармена, друга Неша

		 pic https://i2.paste.pics/b4f6fd5169436346edfc0b46beed6da6.png
		 @caption Неш и Марси в парке университета. Марси: вы такой смешной, мистер Неш!

		 pic https://i2.paste.pics/b67bc69e77fe93b982322a0152e66928.png
		 @caption Марси, Чарльз и Джон

  tab_end
\fi

Далее... снова университет, днем... Неш что-то вычисляет, сидя на траве возле
дерева...  Тут его внимание отвлекает смешливая девочка, ее зовут Марси,
оказывается, она племянница друга Джона Чарльза, которого мы уже встречали
раньше. Тут же неподалеку и Чарльз, и Джон очень рад видеть Чарльза снова!
Оказывается, Чарльз в Гарварде, читает лекции. Друзья гуляют по парку,
общаются, делятся.  Неш признается, что встретил девушку, которую любит, и она
его тоже, и он обсуждает с другом, стоит ли ему жениться...

\ifcmt
  tab_begin cols=3,no_fig,center

     pic https://i2.paste.pics/b72616186f3db908c5f929dfbfeba0d2.png
		 @caption В кафе, Неш просит руки Алисии

		 pic https://i2.paste.pics/758e8d02ffa80b2596af9985211c773f.png
		 @caption Свадьба Неша и Алисии, надпись на машине с сердечком

		 pic https://i2.paste.pics/fa3d398b95a634f0dcef6b3ba616b966.png
		 @caption Свадьба, Неш и Алисия выходят вместе

  tab_end
\fi

Далее, сцена с Алисией в кафе... Неш опаздывает... Алисия терпеливо ждет его за
столиком... Неш просит руки Алисии... причем в необычный, свойственный для Неша способ, впрочем...

Неш: наши отношения предполагают долговременность? Я жду подтверждений достоверных эмпирических данных!

Алисия: прости... но сначала я должна... пересмотреть свои девичьи грезы о
любви. Эм... Подтверждения?  Достоверные данные... Ладно. Скажи, велика ли
Вселенная?

Неш: бесконечна!

Алисия: откуда ты знаешь?

Неш: но все данные на это указывают!

Алисия: но это не доказано!

Неш: нет...

Алисия: ты сам не видел?

Неш: нет...

Алисия: с чего же ты уверен?

Неш: я не уверен, я верю!

Алисия: С любовью точно так же. Итак... тебе неизвестно, хочу ли я выйти за тебя...

Ну что ж, в итоге Неш узнал, хочет ли Алисия выйти за Неша замуж! Потому что
потом идет свадьба!  Юхуууууууууууу! Праздник, шикарная невеста, Неш в красивом
костюме, его друзья рядом с ними.

Далее, Неш и Алисия поженились... а Неш продолжает свою секретную работу...
здесь случается неожиданность... когда Неш в очередной раз доставляет свои
конверты в особняк...  вечером... он выходит из особняка... неожиданно видит,
как подъезжает Уильям Парчер на черной машине.  Он приказывает Нешу садиться
вместе с ним, говорит, что за ними следят... Погоня... Кто то другие мчатся за
ними... Погоня... Стрельба... темными улицами Кембриджа, штат Массачусетс... в
итоге, Парчер стреляет, убивает нападавших... машина преследователей уходит под
воду... Для Неша, кошмарное, ужасное происшествие.  Неш возвращается домой...
подавленный, в шоке... Алисия спрашивает, что с ним... Неш запирается у себя в
комнате... Неш становится нервным, дерганным, испуганным... Его все пугает вокруг... 

\ifcmt
  tab_begin cols=4,no_fig,center

     pic https://i2.paste.pics/ad3486f7a008a9d17474eae185f3edab.png
		 @caption Доктор Розен, врач-психиатр

		 pic https://i2.paste.pics/eaf77d3372458557ebfb56fd10d92deb.png
		 @caption Неш в безумном состоянии

		 pic https://i2.paste.pics/d8955c7ee7462fedd9a51e746a798393.png
		 @caption Неш: это русские! это русские!

		 pic https://i2.paste.pics/123999c4724dbd59b21dcb4be77ae255.png
		 @caption Неша ловят и потом увозят в психиатрическую больницу

  tab_end
\fi

В общем, в итоге оказывается, Неш болен шизофренией... На лекции в Гарвардском
университете, куда он приезжает на конференцию по математике... он говорит про
теорию чисел и теорию относительности Эйнштейна...  внезапно ему кажется, что
его преследуют... Он начинает бежать, его ловят какие то мужчины...  он кричит,
что они русские... в итоге, его пакуют и отправляют в психиатрическую
больницу...  Неш болен... Здесь на сцену выходит доктор Розен, лечащий
врач-психиатр Джона... Он говорит Алисии, что у Джона шизофрения....  Сцена в
комнате с Розеном и Джоном... Джон видит Чарльза... в той же комнате!
Оказывается, что Чарльз, его лучший друг - это мираж, галлюцинация, которую
видит только Неш...  Алисия начинает действовать... она выясняет, куда Джон
отвозил свои секретные конверты...  заброшенный обветшалый особняк... где никто
не живет... Алисия обнаруживает те самые конверты, которые никто не забирал...
Джон отправлял свои донесения впустую...  

\ifcmt
  tab_begin cols=3,no_fig,center

     pic https://i2.paste.pics/18d2106be73ca43ce36b897da7fd3585.png
		 @caption Алисия и Неш в больнице

		 pic https://i2.paste.pics/9c1abbcaaa8a7cec1917508163d1064e.png
		 @caption Алисия Нешу: нет никакого Парчера! Все это тебе только кажется!

		 pic https://i2.paste.pics/4b9a7f5f37fdd33ce7c1596fb1a7d56b.png
		 @caption Джона готовят к проведению интенсивной терапии инсулиновых шоков

  tab_end
\fi

Алисия приносит эти конверты на встречу в больнице со своим мужем...  Джон
начинает понимать, что это все была иллюзия, обман, игра его собственного
разума...  Он расковыривает свою руку, в которую якобы был вживлен имплантант,
и оказывается... что никакого имплантанта нет... рука вся в крови... после
этого случая медсестра срочно зовет доктора Розена...  и Нешу назначают курс
интенсивных инсулиновых шоков.... Ужасное, мучительное лечение, когда пациента
привязывают за руки и ноги к больничной койке, и когда его тело трясется вовсю,
пока по его телу проходят шоки... 

%1:20

После интенсивного лечения Джона выписывают, он снова дома. Занимается своими
делами...  Алисия дома, ухаживает за ним... он пытается работать, но ему надо
принимать таблетки, которые прописал доктор Розен.  Таблетки... тормозят
работу, тормозят работу его разума, он не может как следует сосредоточиться,
работа идет.  Еще... в личной жизни, в качестве мужа Алисии... у него тоже не
получается... у Алисии случается нервный срыв, Алисия и Джон охладели друг к
другу, Алисия держится лишь благодаря своему чувству долга как матери, у
которой маленький ребенок на руках. 

Неспособный сосредоточиться на работе из-за таблеток, Неш постепенно начинает
увиливать от лечения. Алисия дает ему таблетки, он их не принимает, а
складывает в ящике стола... Ему кажется, что это поможет ему в его работе...
но, в какой то момент ему начинают снова мерещиться цифры, шифры в журналах на
его столе. Он начинает снова сходить с ума.  А произошло это так. Возле его
дома есть небольшой заброшенный сарайчик, наверное в метрах триста от их дома,
в лесу.  Как то раз, вечером, Джон, как обычно, кладет таблетки в стол, вместо
того чтобы выпить их с водой и проглотить, как прописал доктор, и внезапно ему
чудится, что кто то бросил камень и бежит из дома. Джон стремглав бросается
вслед за этим неизвестным, он выбегает за пределы дома... Темнота, лес... А
неизвестный куда то убегает... Джон бежит вслед. Внезапно в его лицо светят
яркие фонари, везде военные вокруг, появляется снова Парчер - Старший Брат,
галлюцинация на самом деле, - которая уже мучала Джона раньше. Парчер ему
говорит, что Неш нужен по прежнему стране, - и что они организовали
дешифровальную лабораторию прямо в заброшенном сарайчике. Джон и Парчер
проходят в сарай, там вовсю кипит работа, - военные, дешифровальщики, какие то
радио аппараты, звуки, в общем, работа на правительство США кипит вовсю (как мы
уже обсуждали, это все иллюзия, галлюцинация Джона, но сам Джон этого не
понимает). Потом... Джон пытается возразить Парчеру, дескать, вы же не
существуете!! Парчер его перебивает, да, Розен, конечно, это же шарлатан!! И
Неш... успокоенно соглашается, говорит, а я боялся, что вас не существует... В
этот раз, Джон снова, и с последний раз, становится рабом своих собственных
шизофренических фантазий... Он снова начинает кромсать журналы, обклеивая все
стены сарайчика исписанными вырезками... Тайно от ничего не подозревающей
Алисии... пока следующий случай не вскрывает то, что Джон снова заболел...

А дело было так, уже днем, в их доме... Погода пасмурная... порывы ветра,
облака... вот-вот начнется дождь...  Алисия говорит, что ей надо собрать белье,
висящее снаружи, а Джон ей говорит, давай, а я пока посмотрю за сыном, я его
искупаю, не волнуйся... Алисия соглашается и идет на улицу забирать висящее на
веревках белье... Джон идет с сыном в ванную комнату, включает воду... Воду
набирается... Ребенок лежит на спине... А уровень воды все ближе к тому уровню,
когда ребенок начнет захлебываться... Джон занят чем то другим... Он увидел
Чарльза - его галлюцинацию - и Чарльз ему говорит, не волнуйся, я посмотрю за
ребенком... Джон уходит в другую комнату. А Алисия... внезапно слышит какие то
звуки...  со стороны заброшенного сарайчика... она что то начинает
подозревать... идет туда... открывает двери сарайчика... и выясняет, что весь
сарайчик обклеен вырезками... так же само, как это было в рабочей комнате Джона
в МТИ, когда он в первый раз подписался работать на свою галлюцинацию Парчера.
Алисия понимает, что Джон снова заболел, и стремглав бежит назад в дом,
вспомнив, что она оставила ребенка Неша. И действительно, она успевает вовремя,
потому что еще чуть-чуть, и ребенок захлебнулся бы от воды! А Джон невозмутимо
стоит в другой комнате, и говорит Алисии - да не волнуйся, Чарльз посмотрит за
ребенком! Какой Чарльз!? - кричит Алисия, - здесь нет никого!!!

