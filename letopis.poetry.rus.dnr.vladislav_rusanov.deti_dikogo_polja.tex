% vim: keymap=russian-jcukenwin
%%beginhead 
 
%%file poetry.rus.dnr.vladislav_rusanov.deti_dikogo_polja
%%parent poetry.rus.dnr.vladislav_rusanov
 
%%url https://stihi.ru/2019/03/15/9890
%%author 
%%tags 
%%title 
 
%%endhead 

\subsubsection{Мы дети Дикого Поля...}
\Purl{https://stihi.ru/2019/03/15/9890}
\index{Поэзия!Образы!Поле}
\index{Поэзия!Образы!Дикое Поле}

Мы — дети Дикого Поля,
разбойничьей, злой удачи.
Мы малочувствительны к боли,
мы редко и скрытно плачем.

Несли нас пегие кони,
земля под копытом горела,
умели пустою ладонью
отбить печенежьи стрелы.

Теперь наши степи — пашни,
нарыты в них шахты-норы,
натыканы телебашни,
куда ни поедешь — город.

Теперь мы не головы рубим,
а рубим горючий камень.
Дымятся заводов трубы,
гремят и кузни, и станы.

Но войны степным пожаром
в начале нового века
пылают заревом ржавым,
и льются из крови реки.

Ни выкрики конной лавы,
ни поступь цепи пехотной...
Здесь делят виновных и правых,
на русских идёт охота.

Здесь снова до смерти спорят
селяне и рудокопы,
до пресного, мелкого моря
тянутся нитки окопов.

Попробуй пустою ладонью
отбить осколок снаряда.
И скачут пегие кони,
и степь, словно простынь, смята...

Попробуй не чуять боли,
попробуй не видеть горя
в пределах Дикого Поля
у малосольного моря.

2019
