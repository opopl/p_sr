% vim: keymap=russian-jcukenwin
%%beginhead 
 
%%file 08_06_2019.stz.news.ua.mrpl_city.1.mariupolskie_pljazhi
%%parent 08_06_2019
 
%%url https://mrpl.city/blogs/view/mariupolskie-plyazhi
 
%%author_id burov_sergij.mariupol,news.ua.mrpl_city
%%date 
 
%%tags 
%%title Мариупольские пляжи
 
%%endhead 
 
\subsection{Мариупольские пляжи}
\label{sec:08_06_2019.stz.news.ua.mrpl_city.1.mariupolskie_pljazhi}
 
\Purl{https://mrpl.city/blogs/view/mariupolskie-plyazhi}
\ifcmt
 author_begin
   author_id burov_sergij.mariupol,news.ua.mrpl_city
 author_end
\fi

\ii{08_06_2019.stz.news.ua.mrpl_city.1.mariupolskie_pljazhi.pic.1}

Кто первым пришел туда, где волна за волной накатывается на песчаный берег,
выбрасывая белую пену, пучки водорослей, пестрые раковины отмерших моллюсков,
обкатанные осколки камней, щепки древесины, сплошь облепленные мелкими
ракушками, а иногда и обрывки рыбацких снастей? Не для того, чтобы добывать
рыбу, и не для того, чтобы направить свой челн в морской простор. А с
единственной целью – погрузить свое изнуренное жарой тело в прохладную воду,
омыть его, а потом, подрагивая от озноба, упасть ничком на теплый песок и
нежиться под лучами солнца. Кто был первым пляжником у побережья, где стоит
ныне Мариуполь — человек из эпохи неолита, киммериец, скиф, сармат, гот, гунн,
хазарин, печенег, половец или представитель других народов, кочевавших в
бескрайней приазовской степи, и порой выходивших и к берегу моря? А может
первым был запорожский казак Кальмиусской паланки или грек-переселенец из
Крыма? История по этому поводу хранит гордое молчание.

\textbf{Читайте также:} 

\href{https://mrpl.city/news/view/e-kstremalno-i-s-veterkom-v-mariupole-poyavilsya-kajtserfing-foto}{%
Экстремально и с ветерком: в Мариуполе появился кайтсерфинг, Олена Онєгіна, mrpl.city, 05.06.2019}

Интересные сведения о мариупольских пляжах сравнительно недавнего времени, —
действительно, что там сто — сто двадцать лет по сравнению с несколькими
столетиями, тем более с тысячелетиями, — имеются в статье заведующей отделом
Мариупольского краеведческого музея Раисы Петровны Божко, опубликованной в 2005
году в литературно-публицистическом сборнике \enquote{Лукоморье}. Раиса Петровна, в
частности, приводит цитату из обзора жизни Мариуполя за 1896 год корреспондента
ростовской газеты \enquote{Приазовский край} А. Попова, будущего известного писателя
Александра Серафимовича: \emph{\enquote{Решено городом устройство морских купален, но дело
тормозится губернским начальством, а между тем купальни необходимы: на взморье
можно видеть, как дамы, лошади, дети и кавалеры купаются вместе}}.

Автор приведенных выше строк, видимо, все же сгустил краски. Очень может быть,
мариупольцы и купались вместе с лошадьми, но чтобы благонамеренные и
богобоязненные жительницы нашего города предстали нагишом перед посторонними
мужчинами – это вряд ли могло быть. Да и о купальнях в обзоре говорится так,
будто они никогда не существовали. Между тем на карте Мариуполя 1888 года
купальни обозначены и располагались они примерно там, где сейчас стоят
сооружения бывшего яхт-клуба комбината \enquote{Азовсталь}. Может быть, ко времени,
когда в наш благополучный город приехал А. Серафимович, они обветшали и их
снесло в море? Для современного читателя, наверное, стоит объяснить, что
представляли собой купальни. Это были дощатые строения, огораживающие площадку
суши и прилегающего к ней участку моря. Заметим, что пользование купальнями
было платным.

Р. П. Божко привела название интересного для людей, интересующихся историей
города, документа, разработанного и утвержденного Мариупольской городской
думой, со следующим названием: \emph{\enquote{Обязательные для жителей г. Мариуполя
постановления о мерах к восстановлению порядка и безопасности во время купания
людей и лошадей}}. Раиса Петровна подчеркнула, постановления начинались двумя
запретами: \emph{\enquote{Лошадей купать в особом месте отдельно от купающихся людей.
Купаться лицам мужского и женского пола вместе строго воспрещается}}. Может,
действительно местные любители морских ванн купались всем скопом? Или эти
постановления о купаниях упреждали возможные нарушения нравственности. Кто его
знает?

Это послание из прошлого дает возможность воспроизвести картину мариупольских
пляжей на рубеже XΙX и XX веков. Для купания было отведено два места. Чтобы
добраться для первого из них, нужно было пройти вниз по Гамперовскому спуску
или спуститься по лестнице, связывающей начало Торговой улицы со Слободкой, а
затем двигаться по Морской улице, претерпевшей два переименования: в советское
время она носила имя Розы Люксембург, а сейчас называется Земской. Проследовав
по нескольким относительно коротким кварталам Слободки и перейдя через
железнодорожную линию, уже было можно увидеть море, а через короткий, не
сбавляя шаг, промежуток времени и окунуться в него.

Что касается второго участка для купания, то он располагался в устье Клиновой
балки, и пешком добраться до него можно было по довольно крутому спуску из
Городского сада, поскольку знакомая нам лестница была сооружена уже в советское
время. Вот что пишет Р. П. Божко об обустройстве этого участка: \emph{\enquote{Слева от дороги
хозяева могли купать лошадей. Им отводилось пространство в сто саженей (213,4
м). Отступая пятьдесят саженей от лошадиного пляжа, разрешалось купаться
мужчинам. А вот женщины могли располагаться не менее чем за сто саженей от
мужского пляжа. Чтобы не возникало недоразумений, в указанных местах
предписывалось устанавливать специальные знаки, ограничивающие территорию:
черные столбы для лошадей, красные — для женщин, белые — для мужчин}}. Из
цитируемой статьи мы узнаем, что разрешалось купаться еще и на бирже, правда,
только мужчинам. Биржей в нашем городе называли площадь, примыкающую к устью
Кальмиуса, где располагались склады, амбары и лавки купцов. В советское время
это место стали называть гаванью. Как и сейчас, мариупольцы с наслаждением
погружали свои тела в волны ласкового моря, старались как можно лучше загореть.
Нередко это благое стремление оздоровить себя заканчивалось обугливанием кожи.
Особенно часто это происходило с людьми приезжими. Тот, прежний, пляж отличался
от нынешнего, пожалуй, лишь тем, что мужчины и женщины, как уже было сказано,
принимали морские ванны раздельно, а потому купальными костюмами не
пользовались.

\textbf{Читайте также:} 

\href{https://mrpl.city/news/view/problemnye-plyazhi-mariupolya-vosstanovyat-po-frantsuzskoj-tehnologii-znamenitogo-kurorta-la-roche}{%
Проблемные пляжи Мариуполя восстановят по французской технологии знаменитого курорта La Roche?, mrpl.city, 04.06.2019}

Кроме прямого назначения - купания - мариупольские пляжи использовались для
занятий физкультурой. До наших дней дошла фотография, запечатлевшая длинную
шеренгу красноармейцев 238-го стрелкового полка, делающих зарядку. Известный
мариупольский скульптор Иван Савельевич Баранников в молодые годы был заядлым
фотографом. Благодаря ему можно увидеть застывшие упражнения
акробатов-любителей 30-х годов ХХ века, таких же, каким он был сам. Поскольку
на пляжах было не так уж много народу, молодые люди в перерывах между заплывами
становились в круг и играли в волейбол. Нередко можно было увидеть и заядлых
игроков, склонившихся над шахматной доской, в окружении зевак.

Несмотря на невзгоды и тяжести быта, мариупольцы всегда старались выкроить
время летом, чтобы, говоря на местном наречии, сбегать на море \enquote{скупнуться}.
Есть косвенные доказательства тому. Одно из них позаимствуем из уже
упоминавшейся статьи Р. П. Божко: \emph{\enquote{В 1918 году, во время оккупации города
австрийскими войсками, в городскую думу обратились представители Союза бывших
офицеров – участников Первой мировой войны. Они просили разрешить им очистить
участок побережья и устроить там платный пляж, обнесенный забором, где всего за
15-20 копеек отдыхающие смогли бы купаться и принимать солнечные ванны, не
опасаясь за сохранность одежды}}. Еще одно доказательство – объявление,
напечатанное 23 июля 1943 года в газетенке, издаваемой в оккупированном
гитлеровцами Мариуполе. Вот его содержание: \emph{\enquote{Участок пляжа перед солдатским
санаторием, обозначенный надписями, отведен для раненых и больных немецких
солдат. Гражданскому населению купаться на этом участке пляжа запрещено. Кто не
выполнит этого распоряжения, будет оштрафован. Городской комендант}}...

В середине 50-х годов ХХ века часть берега моря неподалеку от Городского сада в
народе получило название студенческого пляжа. Это было место встреч
студентов-мариупольцев, приезжавших домой на каникулы из Харькова,
Ростова-на-Дону, Москвы, Ленинграда и других городов, где они учились в
тамошних вузах. Но основную массу составляли студенты Ждановского
металлургического института, а также старшеклассники и учащиеся техникумов.
Одним словом, молодежь. Обменивались новостями, знакомились, шутили, играли на
гитарах, \enquote{записывали пульку}, т.е. играли в преферанс. Нередко можно было
увидеть молодых людей с односторонним загаром. Так и знай — это преферансист
несколько часов не менял положения своего тела в азарте игры.

Как добирались до купален и пляжей? До революции — кто беднее и детвора –
пешком. Состоятельные люди могли позволить себе проехаться на извозчике. В
адрес-календаре \enquote{Весь Мариуполь и его уезд} за 1910 год приведена такса для
легковых извозчиков г. Мариуполя, утвержденная Мариупольской городской думой в
1908 году. Поездка к купальням против Морской улицы или обратно должна была
обходиться от 20 до 35 копеек, в зависимости от комфортности экипажа, к
купальням против Клиновой балки - от 25 до 45 копеек. Поездка к купальням и
обратно со стоянкой не более часа стоила почти в два раза дороже.

От старожилов приходилось слышать, что в двадцатые годы прошлого столетия по
Нижнему портовскому шоссе — теперь это Приморский бульвар — проходил один из
первых мариупольских автобусных маршрутов \enquote{Город - Порт}. И по нему за
неимением автобусов курсировали даже по тем временам допотопные грузовики, в
кузовах которых на деревянных лавках рассаживались пассажиры. С тридцатых годов
вдоль пляжей начали ходить трамваи четвертого маршрута. После 1970 года их
сменили троллейбусы. Правда, трамвай шел из порта в город через Слободку, а
троллейбус поворачивает, как известно, на проспект Металлургов.

Вот такая получилась краткая история мариупольских пляжей.

\textbf{Читайте также:} 

\href{https://archive.org/details/01_06_2019.sergij_burov.mrpl_city.doroga_v_port}{Дорога в порт, Сергей Буров, mrpl.city, 01.06.2019}
