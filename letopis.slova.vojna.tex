% vim: keymap=russian-jcukenwin
%%beginhead 
 
%%file slova.vojna
%%parent slova
 
%%url 
 
%%author 
%%author_id 
%%author_url 
 
%%tags 
%%title 
 
%%endhead 
\chapter{Война}

С \emph{войной} в родной стране никак нельзя примириться. О книге, которая
приобрела новый вкус после семи лет конфликта. Ще у 2018 році на книжковому
форумі купила книгу Серые пчёлы Андрея Куркова!  Гортала, тоді не пішла. Зараз,
коли минає вже сім років цієї жахливої \emph{війни}, а світла в кінці тонелю
влада так і не засвічує, просто ковтнула цю оповідь.  Ця \emph{війна} так
болить ....  Просто замисліться і відчуйте. Не змиряймося з такою ситуацією..
Ніяк. Не можна!,
\textbf{С войной в родной стране никак нельзя примириться},
Марина Ставнийчук, strana.ua, 07.06.2021

%%%cit
%%%cit_pic
%%%cit_text
\emph{Воїн}, який захищає Україну, звертається до дівчини з окопів: \enquote{Я
запрошую тебе, твого чоловіка, офіційно до себе на позицію, де ми
\enquote{прохолоджуємося в тіньочку під деревцем, ножками махаємо} } і додає:
\enquote{Поки ти працюєш, відпочиваєш, гуляєш, спиш вдома поруч зі своїм
чоловіком}.  Ще один \emph{воїн} каже: \enquote{Я кинув роботу, батько трьох
дітей, прийшов захищати свою країну}. Жінки-\emph{військовослужбовці} з фронту
також звернулися до Яни: \enquote{Мороз, погода, непогода, мінус 25. Коли по
тобі гатять дві з половиною години, мінометний обстріл. Коли ти чуєш ці виходи,
чуєш свист, рахуєш ті секунди, доки летить ця міна}. Журналісти хотіли
передати запрошення від \emph{бійців} Яні особисто, але за місцем реєстрації
б'юті-майстрині не знайшли. Де Яна нарощує вії журналісти також не довідались,
офіційних даних про її салон і чи сплачує вона податки немає
%%%cit_comment
%%%cit_title
\citTitle{Українські воїни звернулися до горе-блогерки}, 
gazeta.ua, 14.06.2021
%%%endcit

