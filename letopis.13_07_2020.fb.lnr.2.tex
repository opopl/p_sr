% vim: keymap=russian-jcukenwin
%%beginhead 
 
%%file 13_07_2020.fb.lnr.2
%%parent 13_07_2020
 
%%endhead 

\subsection{Киев готовится воевать в Донбассе ещё четверть века}
\label{sec:13_07_2020.fb.lnr.2}
\url{https://www.facebook.com/groups/LNRGUMO/permalink/2841332549311634/}
  
Украинский вице-премьер-министр по вопросам реинтеграции Алексей Резников
сделал очередное заявление по поводу нежелания Киева выполнять Минские
соглашения.

«О Минских соглашениях я даже процитирую госпожу Ангелу Меркель: «...они не
высечены на камне». Поэтому можно сказать, что они сегодня не соответствуют
реалиям, в которых была Украина, когда заключались те соглашения как
политико-правовые договорённости. Потому что сегодня сроки, которые
предусмотрены в Минских соглашениях --- 2014-2015 годы --- уже истекли, а эти
Минские соглашения не реализованы и стали во многом неактуальны», --- заявил
Резников в эфире телеканала «Украина 24».

Накануне же своего заявления в эфире украинского телеканала Резников отметился
статьёй на сайте Atlantic Council, в которой продвигает ту же идею: наступило
время пересмотреть Минские соглашения, и обращается с призывом к Западу
поддержать этот пересмотр.

«Западу стоит поддержать пересмотр Минских соглашений, поскольку они не
отражают нынешних реалий и потребностей по урегулированию конфликта на
Донбассе… Сейчас не время догматически цепляться за существующие
договорённости. Напротив, творческие подходы необходимы для обеспечения
глобальной безопасности и предотвращения дальнейшей европейской
дестабилизации»,– написал вице-премьер-министр Украины и первый замглавы
украинской делегации в ТКГ Алексей Резников в своём блоге для Atlantic Council.
Что служит ещё одним доказательством того, кто именно является заказчиком
пересмотра Минских соглашений.

Одним из важных аргументов для пересмотра Резников считает тот факт, что
Украина была «в невыгодной ситуации». Сегодня же, по мнению вице-премьера, «у
нас есть оружие и боеспособная армия, поэтому Минские соглашения нуждаются в
пересмотре», добавляя как бы невзначай, что и «джавелины» уже подвезли на
передовую, чего ж ждать?

Напомним, это тот Резников, который не так давно глубокомысленно заявлял, что
«капитуляцию гитлеровская Германия подписала, находясь в невыгодном положении,
иначе бы условия капитуляции были совсем другие». И по этой реплике сразу
становится понятно, на чьей стороне пан Резников.

А дальше он рассказывает читателям Atlantic Council, как Украина, оказывается,
изо всех сил старалась в течение года соблюдать минские договорённости --- и
войска отводила, и даже переформатировала свою делегацию в ТКГ, но ничего не
помогает, Россия (не «сепаратисты» Донбасса, а именно Россия) «продолжает
нарушать соглашение о прекращении огня». Поэтому ничего не остаётся --- только
пересмотр минских договорённостей.

«Дальнейшее промедление приведёт только к дополнительным издержкам», --- делает
вывод Резников.

За последние дни Киев вообще отметился целым рядом странных заявлений.
Например, глава МИД Украины Дмитрий Кулеба выдал фразу: «Украина ведёт войну за
мир в Донбассе». То есть ВСУ открывают артиллерийский огонь по жилым кварталам
Донбасса ради того, чтобы там наступил мир? И только ради скорейшего
наступления мира убивают мирных жителей --- стариков, женщин, детей? До такого
даже гитлеровские нацисты не додумались.

А новый, так сказать, «переговорщик от Донбасса» Денис Казанский, который
почему-то сидит на переговорах вместе с киевской делегацией, и вовсе
потребовал, чтобы Донецкая Народная Республика отдала Дебальцево, а границы
были восстановлены на момент 12 февраля 2015 года, день подписания Минских
соглашений.

Напомним, Дебальцево полностью освободили от ВСУ 18 февраля.

Ну так договоритесь с самым главным «переговорщиком» украинским, что делать –
отказываться от Минских соглашений или границу «двигать».

Резников спустя пару дней сделал ещё одно заявление.

По его мнению, необходимо всего-то 25 лет для полной и безопасной реинтеграции
Донбасса, нужно лишь подождать, когда «сменится одно поколение жителей
Донбасса».

«Безопасная реинтеграция займёт минимум лет 25. Это одно поколение должно
измениться, когда мы с вами сможем сказать, что восстановленный контроль на тех
территориях со стороны украинской власти приведёт также к преодолению
разрушения того раздора в головах, который там есть. Поэтому нужно просто
приготовиться подождать»,– сказал Резников. При этом справедливо напоминает,
что в Донбассе уже пойдут в школу дети, которые знать не знают «что такое
свободная Украина». Но забыл добавить, что хорошо знают, с какой стороны летят
снаряды и мины, умеют на слух определить калибр артиллерии, знают, как лучше
спрятаться при обстреле, а у многих сохранятся на всю жизнь шрамы от украинских
снарядов. Неужели господин Резников верит в то, что через 25 лет эти дети всё
забудут? Или просто приучает украинцев, что война продлится ещё четверть века?

После этого заявления Резникова парламентская фракция «Оппозиционная платформа
– За жизнь» потребовала от президента Украины Владимира Зеленского полностью
переформатировать состав украинской делегации в ТКГ и прекратить обеспечивать
интересы «партии войны».

««Мир --- за год» --- под таким лозунгом победил Владимир Зеленский и его
политическая команда. Но сегодня ключевые переговорщики от Украины разрушают
Минские соглашения --- единственную правовую основу для мирного урегулирования и
готовят украинское общество к бесконечному конфликту на Донбассе. Никто не
давал Владимиру Зеленскому и его ставленникам мандата на то, чтобы четверть
века водить страну по пустыне конфликта. Настал момент сделать выбор --- дать
стране мир или уйти с извинениями»,– говорится в заявлении ОПЗЖ.

А глава МИД Российской Федерации Сергей Лавров, выступая на совместной с
президентом Сербии Александром Вучечем пресс-конференции в Белграде, напомнил,
что Минские соглашения должны выполняться по той простой причине, что подписаны
представителем Киева, а затем утверждены Совбезом ООН. Согласно же
международному праву, все резолюции Совбеза ООН обязательны к выполнению.

«Заявления, которые мы слышим от официальных украинских представителей о том,
что эти решения не носят обязательный характер, --- это всё от лукавого либо от
незнания предмета, такое тоже я вполне допускаю», --- считает Сергей Лавров.

Заместитель руководителя администрации президента РФ Дмитрий Козак также
прокомментировал заявления вице-премьера Украины.

«В нашем, да и в общепринятом в мире понимании любые публичные высказывания
вице-премьера, министра, других должностных лиц --- это и есть официальная
позиция государства. Если в ближайшее время это заявление будет подтверждено
или не будет опровергнуто теперь уже премьер-министром или президентом Украины,
то это действительно означает выход Украины из Минских соглашений, а
следовательно, и из минского и «нормандского» форматов переговоров по
урегулированию конфликта в Донбассе. Это ещё и значит, что президент Украины
должен будет дезавуировать свою подпись полугодовой давности под итоговым
коммюнике парижского саммита 9 декабря 2019 года», --- отметил Козак.

Тем временем война в Донбассе угрожает выйти на новый уровень: глава СММ ОБСЕ в
Киеве Яшар Халит Чевик сделал заявление о том, что в Донбассе наблюдается
пятикратное увеличение военной техники вдоль всей линии разграничения.

Полина Судоплатова
  
