% vim: keymap=russian-jcukenwin
%%beginhead 
 
%%file slova.trevoga
%%parent slova
 
%%url 
 
%%author 
%%author_id 
%%author_url 
 
%%tags 
%%title 
 
%%endhead 
\chapter{Тревога}
\label{sec:slova.trevoga}


%%%cit
%%%cit_head
%%%cit_pic
%%%cit_text
Доводилось ли вам обнаруживать себя посреди ночи, или с утра пораньше, за
механическим пролистыванием социальных сетей, когда пора бы уже спать, или
просыпаться, а вы всё скролите фиды, без явной цели, по наитию руки, то и дело
тянущейся к смартфону, чтобы посмотреть: \enquote{а, вдруг, там что}?
Компульсивное использование сетей, думскроллинг или депрессивный гедонизм – это
общее место в жизни прекариата. Исследования показывают, что залипая в
смартфонах, мы пытаемся преодолеть \emph{тревогу}, депрессию, стресс, одиночество;
заполнить пустоту и отвлечься от будней; получить удовольствие, которое из раза
в раз оказывается скоропостижным, ускользающим, неудовлетворительным.
Всё это – часть глобального психокризиса, чьими всадниками являются скука,
невроз, депрессия и \emph{тревога}
%%%cit_comment
%%%cit_title
\citTitle{Залипая в соцсетях, мы пытаемся преодолеть тревогу, депрессию, стресс, одиночество}, 
Анатолий Ульянов, strana.ua, 12.07.2021
%%%endcit

