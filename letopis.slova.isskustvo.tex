% vim: keymap=russian-jcukenwin
%%beginhead 
 
%%file slova.isskustvo
%%parent slova
 
%%url 
 
%%author_id 
%%date 
 
%%tags 
%%title 
 
%%endhead 
\chapter{Исскуство}

%%%cit
%%%cit_head
%%%cit_pic
\ifcmt
  tab_begin cols=3
     pic https://avatars.mds.yandex.net/i?id=7a1c4796f8691dfd466990fdc2e9e560-5233605-images-thumbs&n=13
     pic https://www.france-voyage.com/visuals/photos/%E3%83%91%E3%83%AA%E3%81%AE%E3%83%8E%E3%83%BC%E3%83%88%E3%83%AB%E3%83%80%E3%83%A0%E5%A4%A7%E8%81%96%E5%A0%82--36168_w1000.jpg
		 pic https://i.pinimg.com/originals/30/d0/05/30d005e3c11e45c2280fa141172f0f44.jpg
  tab_end
\fi
%%%cit_text
А втім, усі ці відтінки й відмінності стосуються зовнішнього вигляду будови.
\emph{Мистецтво} змінило тут тільки оболонку. Сама будова християнського храму
лишилася недоторканою. Це той же внутрішній скелет, те ж послідовне
розташування частин. Яким би різьбленням та скульптурою не була прикрашена
оболонка Собору, завжди під нею знаходиш хоча б у зародковому, початковому
стані римську базиліку. Вона незмінно розташовується на землі за постійним
законом. Це ті ж два навхрест пересічені нефи, верхня частина яких, заокруглена
в абсиду, утворює хори. Це завжди ті самі низенькі бічні нефи для процесій
усередині храму або для каплиці, щось на зразок бічних проходів, з якими
центральний неф сполучається через проміжки між колонами. На цій непорушній
основі кількість каплиць, порталів, дзвіниць, шпилів змінюється безліч разів,
відповідно до фантазії століття, народу, \emph{мистецтва}. Забезпечивши всі вимоги
церковної відправи, архітектура робить те, що їй забагнеться. Статуї, вітражі,
розети, арабески, різьблені прикраси, капітелі, барельєфи — все це вона поєднує
згідно з тим правилом, яке їй до вподоби. Звідси дивовижна зовнішня
різноманітність цих будов, в основі яких лежить так багато порядку та єдності.
Стовбур дерева — незмінний, розгалуження — химерні
%%%cit_comment
%%%cit_title
\citTitle{Собор Паризької Богоматері}, Віктор Гюго
%%%endcit

%%%cit
%%%cit_head
%%%cit_pic
%%%cit_text
Мир словно перевернулся. На телевизионных шоу спасают голодающего
политавантюриста и взывают к гуманизму, и в это же самое время – миллионы живут
ниже прожиточного минимума, старики побираются у магазинов и памятников, а
152-м калибром бьют по сепаратистам, по ходу разрушая и убивая рикошетом своих
земляков. Оттуда – такой же огонь, но с другим пафосом.  Ложь в лицо и
артистичное двуличие превратились в \emph{искусство}, которому аплодируют. Экономика
«в тени», жизнь «в полу-тени», совесть – вообще «не в тренде». Бесконечные
разборки, обвинения и разоблачения, а в лицах – ненависть и неприятие (на
гектар рядом не...). Вся эта ненависть переносится на улицы, в аудитории, на
базары и на митинги. И как это все может жить вместе успешно и в «мире»? Ищу
ответы...  Не знаю, может все написанное – уже банальное брюзжание... Но что-то в
стране не так, если так массово доверяют свою судьбу проходимцам. Слушают что
«попроще».  Живут в режиме «выживания» - личного, бизнесового,
государственного. И судя по состоянию нашего майданометра, готовы поиграть в
судьбу еще пару раз как минимум.  Вот это все – наверное и есть какое-то
специфическое прохождение этого самого «конца света». В том смысле, что в
толкотне за выживание в «темные годы» будет еще долго идти поиск спички,
фонарика, костра, который бы хоть на время «осветил». А значит, это еще
надолго, и с пока еще ненаписанной утопией про завтра
%%%cit_comment
%%%cit_title
\citTitle{Пришел конец старому теплому украинскому миру. Начинается что-то другое / Лента соцсетей / Страна}, Андрей Ермолаев, strana.news, 29.11.2021
%%%endcit
