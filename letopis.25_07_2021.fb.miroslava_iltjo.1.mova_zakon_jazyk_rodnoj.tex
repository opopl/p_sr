% vim: keymap=russian-jcukenwin
%%beginhead 
 
%%file 25_07_2021.fb.miroslava_iltjo.1.mova_zakon_jazyk_rodnoj
%%parent 25_07_2021
 
%%url https://www.facebook.com/muroslavamix/posts/1776640635856886
 
%%author Ильтьо, Мирослава
%%author_id miroslava_iltjo
%%author_url 
 
%%tags jazyk,language,language.native,mova,ukraina,ukrainizacia
%%title У мєня рускій радной! – Не факт. Він у вас, радше, побутовий
 
%%endhead 
 
\subsection{У мєня рускій радной! – Не факт. Він у вас, радше, побутовий}
\label{sec:25_07_2021.fb.miroslava_iltjo.1.mova_zakon_jazyk_rodnoj}
 
\Purl{https://www.facebook.com/muroslavamix/posts/1776640635856886}
\ifcmt
 author_begin
   author_id miroslava_iltjo
 author_end
\fi

\begin{itemize}
\item – Я перепрошую, дотримуйтеся, будь ласка, закону про обслуговування державною мовою.
\item – У мєня рускій радной!
\item – Не факт. Він у вас, радше, побутовий. Це не літературна російська, а
український територіальний варіант російської мови. Вона не може бути рідною
для зросійщених українців.
\end{itemize}

‼️Бо в 16 столітті Українська Церква перебула під Костантинополем, а в той же
час Московська Церква самовільно оголосила свою автокефалію. До всього щастя на
українські книги на польській території відбувалися гоніння і вилучення.

‼️У 17 столітті цар Михайло з Філаретом тупо спалили всі примірники
надрукованого в Україні «Учительного Євангелія» Кирила Ставровецького, який
роком раніше перейшов в унію. Ще через кілька десятків років Патріарх Іоаким
наказав повидирати з українських книжок листки. Розпалили багаття на вулицях
Мацкви і почали палити твори видатних українських богословів, бо Іоаким засцяв,
що Петро Могила, Лазар Баранович та Інокентій Ґізель йому великі конкуренти.

‼️У 18 столітті Петро І не напрягається палити, просто забороняє друкувати
українською, а з усіх церковних книг українські тексти прибирає, ніби їх там і
не було. Петро ІІ йде ще далі: наказує переписати з української мови на
російську всі державні постанови й розпорядження. А Синод видає розпорядження
вилучити зі шкіл українські букварі. Приходить Катерина II і забороняє
викладати українською мовою в Києво-Могилянській академії. А через якийсь час
Митрополит Самуїл розпоряджається, щоб  російська стала там єдиною мовою
викладання.

Зруйнована Запорізька Січ і закриті українські школи при полкових козацьких
канцеляріях, бо вєлік руській язик!

‼️У 19 столітті за царським указом всі україномовні школи заборонені і введена
обов’язкова цензура на будь-яку літературу.

Погром Кирило-Мефодієвського товариства,  заборона найкращих творів Шевченка,
Куліша, Костомарова та інших. Валуєвський циркуляр про заборону давати
цензурний дозвіл на друкування україномовної духовної і популярної освітньої
літератури. Прийняття Статуту про початкову школу, за яким навчання має
проводитись лише російською мовою.

Емський указ Олександра ІІ про заборону друкування та ввозу з-за кордону
будь-якої україномовної літератури, а також про заборону українських вистав та
друкування українських текстів з нотами, а саме народних пісень. Олександр III
взагалі забороняє хрещення українськими іменами.

‼️У 20 столітті Столипін забороняє діяльність  будь-яких українських організацій
і культурних товариств, видавництв, читання лекцій українською, створення
будь-яких неросійських клубів. Микола ІІ видає наказ про скасування української
преси. Приходить до влади Сталін і починається криваве місиво – знищуються
більшість українських письменників. Вже в УРСР вчителям російської мови
порівняно з вчителями мови української платять суттєво вищі зарплати. Усі музеї
переходять на російську. ЦК КПРС видає постанову про закріплення російської
мови як загальнодержавної. Роками пізніше російській мові надається статус
офіційної.

А зараз взагалі Росія напала на Україну під прикриттям захисту російськомовного
населення. І оце все місиво криваве уже сім років в тому числі й за мову.

Циркулярами, указами, заборонами, законами, анафемами, покараннями шість
століть поспіль Росія нищить українську мову. 

Тож ваша рідна, скоріше за все, таки українська. А російська – звична для
користування. 

Просвітницька лекція тривалістю 4 хвилини. 

Це не займе ні у вас багато часу, ні у співрозмовника. Але оптику трохи
змінить, може.

І от ще. 

Різні регіони України піддавалися різному рівню зросійщення, але волинські
малороси особливо дратують, бо якраз у нас російська споконвіку в мізерній
меншості, на щастя. Не було примусової зміни прізвищ на російській манер, як от
в інших регіонах. Тому оця тяглість до прогинання перед співрозмовником, який
говорить російською, вона навіть не про поглинання російською української, а
про самовільне згодовування всього того, що потребує не просто бережного
ставлення, а й захисту.


\ii{25_07_2021.fb.miroslava_iltjo.1.mova_zakon_jazyk_rodnoj.cmt}
