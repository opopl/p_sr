% vim: keymap=russian-jcukenwin
%%beginhead 
 
%%file 05_04_2022.fb.smirnov_vadim.odessa.rieltor.1.berislav_rajon.cmt
%%parent 05_04_2022.fb.smirnov_vadim.odessa.rieltor.1.berislav_rajon
 
%%url 
 
%%author_id 
%%date 
 
%%tags 
%%title 
 
%%endhead 
\zzSecCmt

\begin{itemize} % {
\iusr{Вадим Смирнов}

Очень сложно сдерживать свои эмоции, но пост постарался сделать максимально
сдержанным, без оценочных суждений.

Бог всё видет. Каждому по делам и заслугам его воздасться.

\iusr{Svitlana Senik}
Це не войни, це падаль яка прийшла на нашу святу землю.

\iusr{Bohdan Khomchenko}

1. Якщо цей текст для українців, то чому мокшанською?

2. Якщо цей текст для світової спільноти, то чому мокшанською?

\begin{itemize} % {
\iusr{Вадим Смирнов}
\textbf{Bohdan Khomchenko} 

що для Вас важливіше - лихо, яке прийшло на нашу землю, чи мова на якій про яке
намагаються донести світові?

Питання життя чи смерті нівелює перед мовним питанням для Вас, чи як???

Всі ми люди, усі під богом.

\iusr{Bohdan Khomchenko}
\textbf{Вадим Смирнов} 

у вас \enquote{бог} з маленької літери, або, іншими словами \enquote{у вас ус атклєілся}.

Ну, і додам для \enquote{світові} — the world prefers english.

\iusr{Вадим Смирнов}
\textbf{Bohdan Khomchenko} 

можете продовжувати вінчати терням, а можете зробити щось корисне. У нас завжди
є вибір. Поки живемо. Поки б'ється серце.

\iusr{Bohdan Khomchenko}
\textbf{Вадим Смирнов} я роблю весь час, коли не сплю.
\end{itemize} % }

\end{itemize} % }
