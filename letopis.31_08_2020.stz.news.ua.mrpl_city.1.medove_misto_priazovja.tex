% vim: keymap=russian-jcukenwin
%%beginhead 
 
%%file 31_08_2020.stz.news.ua.mrpl_city.1.medove_misto_priazovja
%%parent 31_08_2020
 
%%url https://mrpl.city/blogs/view/medove-misto-priazovya
 
%%author_id demidko_olga.mariupol,news.ua.mrpl_city
%%date 
 
%%tags 
%%title "Медове місто" Приазов'я
 
%%endhead 
 
\subsection{\enquote{Медове місто} Приазов'я}
\label{sec:31_08_2020.stz.news.ua.mrpl_city.1.medove_misto_priazovja}
 
\Purl{https://mrpl.city/blogs/view/medove-misto-priazovya}
\ifcmt
 author_begin
   author_id demidko_olga.mariupol,news.ua.mrpl_city
 author_end
\fi

\ii{31_08_2020.stz.news.ua.mrpl_city.1.medove_misto_priazovja.pic.1}

Незважаючи на кінець серпня, справжні поціновувачі відпочинку біля моря можуть
продовжити насолоджуватися останніми теплими днями і у вересні. Особисто я не
дуже хочу прощатися з літом, тому вирішила написати блоги про найбільш відомі
та водночас бюджетні курорти на Азовському морі. Почну з Білосарайської коси –
села, розташованого в Першотравневму районі Донецької області, що має багату
історію і відрізняється особливим мікрокліматом. Не всім відомо, що в перекладі
з тюркської мови Білосарайська коса означає \enquote{медове місто}. Враховуючи всі
переваги коси та особливості її розташування, думаю, село і справді стало для
багатьох саме \enquote{медовим}.

Історія коси бере початок з середніх віків, тоді на території села
розташовувалося місто Білгород. У XIII столітті монголо-татари захопили місто і
перейменували його в Білосарай. Монголо-татари торгували там з генуезькими і
венеціанськими купцями. Поступово місто прийшло в занепад, але під час
переселення греків в Приазов'я наприкінці XVIII століття ці землі почали
освоювати заново. 1835 рік вважають роком заснування села. На території коси
розташований найстаріший маяк Північного Приазов'я. Вже 184 роки вогонь маяка
вказує кораблям заповітний шлях до берега. Цікаво, що всередині маяка збережені
старовинні круті сходи, двері з клеймом французькою мовою, що свідчить про те,
що вони були виготовлені фірмою Ейфеля у Парижі в 1882 році.

Не дивно, що Білосарайську косу називають островом середземноморської фауни і
флори. На території коси розташовується орнітологічний заповідник (з 1995 року)
та ландшафтний заказник АН України загальною площею 616 гектарів, який входить
до національного природного парку \enquote{Меотида}. Тут налічується близько 158
різновидів птахів, зокрема 25 з яких занесено до Червоної книги України:
мартини, крижні, кулики, гуси, лебеді. У флорі заказника зареєстровано 217
видів рослин, серед яких і червонокнижні холодок коротколистий та тюльпан
змієлистий.

Розваги на Білосарайській косі також доповнить екскурсія до термального
радонового джерела. У місцевих мешканців та приїжджих популярна рибалка,
оскільки прибережні води багаті на рибу.

\ii{31_08_2020.stz.news.ua.mrpl_city.1.medove_misto_priazovja.pic.2}

На думку жителів села, відпочивати на Білосарайці найкраще у травні та вересні,
особливо для людей, що мають проблеми зі здоров'ям, і для тих, хто не
переносить спеку. Температура повітря в цей період помірна, а вода нагрівається
до комфортної температури 20–23 С. Водночас в цей час небагато людей, тому
пляжі чисті, велика кількість вільних місць в пансіонатах, дуже привабливі ціни
та гарна погода.

Мікроклімат Білосарайської коси є дуже корисним для здоров'я, адже тут
дивовижним чином злилися воєдино запах степів і моря, а повітря насичене
озоном, бромом, йодом, різними солями та ефірними оліями. Дихати таким ароматом
дуже корисно: морська вода віддає в повітря йод і озон, а трави – фітонциди,
які вбивають шкідливі бактерії і дезінфікують легені, що дуже корисно для тих,
хто страждає на бронхіальну астму, легеневі захворювання і часті ГРЗ.

Сьогодні село – це приватний сектор з дво- і триповерховою забудовою. По обидва
боки центральної вулиці Білосарайської коси так багато приватних будинків,
котеджів, пансіонатів і санаторіїв, що в нумерації будинків використовується
практично весь алфавіт. Наразі в селі немає джерел прісної води, тому питна
вода або привізна, або відфільтрована зі свердловини. На косі є ринок, кафе,
магазини і сезонні атракціони. Також  сьогодні село прикрашає безліч цікавих
пам'ятників, присвячених історії та головним здобуткам коси. У селі працює
декілька аптек і фельдшерський пункт біля бази відпочинку \enquote{Наша дача}.
Неподалік від пошти відкрився дитячий зоопарк, в якому можна побачити
африканських страусів, осликів, зайців, єнотів, куниць, нутрій, лисиць,
камерунських коз, африканських кур, в'єтнамських поросят.  На території
зоопарку розташовані батут та гойдалки, на яких діти можуть  безкоштовно
покататися. Господарі зоопарку дбайливо піклуються про всіх тварин, пам'ятаючи,
що для багатьох з них вони стали тимчасовою домівкою. Так наприкінці серпня
вони відпустять сову, яку принесли до них зовсім маленькою і травмованою. Зараз
птах набрався сил і готовий повернутися в степ. Та ж доля чекає й на лисицю,
яку господарям врятували від браконьєрів, але вони розуміють, що волелюбній
тварині все ж комфортніше на свободі. 

Відпочивальників на косі завжди багато. Тут Азовське море неглибоке, вода добре
прогрівається, тому місце приваблює передусім сім'ї з дітьми. Ціни відпочинку в
пансіонатах зовсім різні (від 100 грн до 1100). Все залежить від рівня комфорту
та побажань туристів. Якщо порівнювати з цінами інших курортів на Азовському
морі, Білосарайка стала більш дорогим задоволенням, ніж, наприклад, Мелекіне.
Але просторі піщані пляжі, тепле море і особливий мікроклімат завжди
приваблюватимуть своєю унікальністю та самобутністю.
