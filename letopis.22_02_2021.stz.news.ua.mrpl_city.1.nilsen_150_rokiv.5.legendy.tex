% vim: keymap=russian-jcukenwin
%%beginhead 
 
%%file 22_02_2021.stz.news.ua.mrpl_city.1.nilsen_150_rokiv.5.legendy
%%parent 22_02_2021.stz.news.ua.mrpl_city.1.nilsen_150_rokiv
 
%%url 
 
%%author_id 
%%date 
 
%%tags 
%%title 
 
%%endhead 

\subsubsection{Легенди, пов'язані з Нільсеном та його будівлями}

Існує легенда, що під час Другої світової війни вежа не зазнала серйозних
пошкоджень завдяки оригінальній методиці, використаній під час будівництва.
Мешканці міста вважали, що її врятували курячі яйця, які міцніше поєднували
цемент, ніж цегла. Правда це, чи тільки легенда, для нас важливо, що вона все ж
таки вціліла.

Будинок легендарного архітектора, в який він переїхав з родиною, залишивши
щойно збудований будинок з левами, не тільки безжалісно руйнує час, ще й
переслідує низка легенд.

Деякі містяни йому вже придумали назву – \emph{будинок з ридаючими німфами}
(розташований поруч з Міським садом). Саме німфи, якими архітектор прикрасив
фасад, і породили чутки та похмурі легенди про привидів за його цегляними
стінами.

\ii{22_02_2021.stz.news.ua.mrpl_city.1.nilsen_150_rokiv.pic.4}

За однією з версій, популярної у місцевих старожилiв, німфи Нільсена з'явилися
в пам'ять про дочку архітектора, яка начебто померла від тифу. А ці німфи
оплакували померлу дівчину.

Інші особливо вразливі містяни стверджували, що чули (особливо вночі) як з
будинку доносився жіночий плач, що є актуальним, якщо подумати про стан будівлі
і термінову необхідність її реставрації...
