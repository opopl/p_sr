% vim: keymap=russian-jcukenwin
%%beginhead 
 
%%file 06_01_2022.fb.zharkih_denis.1.ekonomicheskij_smysl_anti_rossii
%%parent 06_01_2022
 
%%url https://www.facebook.com/permalink.php?story_fbid=3160015067545222&id=100006102787780
 
%%author_id zharkih_denis
%%date 
 
%%tags antirossia,ekonomika,ideologia,rossia,ukraina
%%title Экономический смысл Анти-России
 
%%endhead 
 
\subsection{Экономический смысл Анти-России}
\label{sec:06_01_2022.fb.zharkih_denis.1.ekonomicheskij_smysl_anti_rossii}
 
\Purl{https://www.facebook.com/permalink.php?story_fbid=3160015067545222&id=100006102787780}
\ifcmt
 author_begin
   author_id zharkih_denis
 author_end
\fi

Экономический смысл Анти-России

Любая революция начинается с экономических требований - да нам есть нечего, да
сколько можно эти цены терпеть, так как вы нам мало платите, где кружевные
трусики, почему не завезли и тому подобное. Но далее в нее вдыхают политический
смысл. Он может быть направлен на требования бунтующих, а может вообще никак с
ними не связан. 

Политический смысл революции 1917 года (и февральской и октябрьской) поиск
нового управления страной. Уже в 1905-7 пробовали народовластие, советы. От
этого не отказались. В 1907 система получила поражение, но толкнула страну к
парламентаризму, была созвана Дума. С парламентаризмом не вышло, опять советы,
опять провал, но появляется жесткий контроль одной партии, и система начинает
выстраиваться.  Но просили ли революционные матросы, рабочие и крестьяне диктат
одной партии? Да как-то не особо, они о другом думали. 

Свидетелем трех украинских революций я был, и хорошо видно, как социальные
требования подменялись политическими, при этом далекими от требований
протестующих. События конца 80-х  были вызваны глупыми реформами управления. Но
протестующим подкинули политический смысл независимости. Украина получила
независимость, но проблем стало не меньше, а больше.

Первый и второй Майдан имели антироссийскую направленность. Первый был мягким и
бескровным, и более был за западный вектор, чем против России. А вот второй уже
был не столько за западный вектор, сколько антироссийским. Впрочем, и
украинская независимость, вместо поиска новых эффективных систем управления,
сразу же скатилась в антироссийскую риторику.  

То есть схема такая: Акт первый - возмущаемся социально-экономическим
положением, Акт второй - обвиняем в этом кого-то, Акт третий - мстим
обвиняемым. Украинская идеология  вела одну линию - во всех бедах Украины
виновата Россия, ее влияние и наследие. Это ничем не отличается политического
смысла в Прибалтике, бунтов в Беларуси и Казахстане. 

Значит, кто-то эту линию методично внедряет, кто-то использует возмущение
народа (иногда очень справедливое), вкладывая его энергию в разрушение
межнациональных связей. И тут ничего нового. Национал-социалисты в Германии
тоже начинали с социальных лозунгов, но потом оказалось, что на социальную
поддержку могут рассчитывать только представители титульной нации. Прочие нации
не просто второй сорт, они объявлены врагами, и должны быть либо порабощены,
либо уничтожены. 

Фактически экономика Третьего Рейха - экономика грабежа соседей. Как только
Гитлеру дали по соплям, и грабителей погнали восвояси, Рейх был обречен.
Нынешние антироссийские настроения и политические смыслы тем притягательны, что
точно также узаконивают грабеж. Мы ж не у своих забираем, а у русских, свое,
кровное, что они у нас отобрали.

А дальше схема несколько изящнее фашизма. Отобранное складываем в надежном
месте западной демократии. Потом западная демократия прозревает, объявляет
борца коррупционером, и забирает активы себе. Вот и вся схема антироссийских
настроений. Не благодарите.

\ii{06_01_2022.fb.zharkih_denis.1.ekonomicheskij_smysl_anti_rossii.cmt}
