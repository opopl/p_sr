% vim: keymap=russian-jcukenwin
%%beginhead 
 
%%file plaintex.cmds.dimen
%%parent plaintex.cmds
 
%%endhead 

\subsection{dimen}
\url{https://www.tug.org/utilities/plain/cseq.html#dimen-rp}

dimen 	Registers
Internal Quantity

\subsubsection{Synopsis:}

\begin{verbatim}
	\dimen<8-bit register number>=<dimen>
\end{verbatim}

\subsubsection{Description:}

THERE are 256 dimen registers: \verb|\dimen0| to \verb|\dimen255|. Each may hold a
dimension. The commands \verb|\advance|, \verb|\multiply|, and \verb|\divide| allow limited
arithemetic with the registers. A dimen register may be placed in a count
register. TeX converts a dimension to a number by assuming units of sp
(scaled points) [118]. The dimen registers obey TeX's grouping structure.
So, changes to a register inside a group will not affect the value of the
register outside the group unless \verb|\global| is used with the register [119].
The command \verb|`\showthe\dimenn'| writes the value for \verb|`\dimenn'| to the
terminal and to the log file [121]. The register \verb|\dimen255| is available for
temporary storage [122].

\subsubsection{Example:}

\begin{verbatim}
	1. \dimen1=10pt
	2. \dimen2=1.5\dimen1
	3. \dimen3=-\dimen2
	4. \dimen4=0.66667\dimen3
	5. The dimens are: \the\dimen1, \the\dimen2, \the\dimen3, \the\dimen4.\par
	6. \dimen1=1pt
	7. \dimen2=1sp
	8. \count1=\dimen1
	9. \count2=\dimen2
	10. The counts are: \the\count1, \the\count2.\par
\end{verbatim}

Produces: See typeset version.

\subsubsection{Comments:}

Lines 1-4 show how easy it is to set one \verb|\dimen| register equal to a
fractional multiple of another register.  Lines 6-9 show how \verb|\dimen|
registers are converted to \verb|\count| registers.

TeXbook References: 118-119. Also: 118-122, 271, 276, 346-347, 349, 360, 363, 395.

Related Primitives: dimendef, ifdim, advance, multiply, divide, count.

For Additional Examples, see: divide, prevdepth 
