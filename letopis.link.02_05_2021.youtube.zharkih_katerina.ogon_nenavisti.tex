% vim: keymap=russian-jcukenwin
%%beginhead 
 
%%file link.02_05_2021.youtube.zharkih_katerina.ogon_nenavisti
%%parent 25_11_2021.fb.zharkih_katerina.1.nasilie_v_ukraine.cmt
 
%%url 
 
%%author_id 
%%date 
 
%%tags 
%%title 
 
%%endhead 

\href{https://youtu.be/50UdMGoAouQ}{%
2 мая. Огонь ненависти, который сжигает Украину. Екатерина Жарких, youtube, 02.05.2021%
}

\begin{multicols}{2}
2 мая 2014. Одесса. 

В этот день случилось немыслимое в цивилизованном мире — люди одних взглядов
сожгли заживо людей других взглядов. Своих сограждан. Мужчин и женщин. А кого
не успели — добивали собственными руками. Это событие — наглядный пример, как
можно стравить людей с помощью популизма, деления на своих и чужих и
беспринципных СМИ. Одно из самых страшных вещей этой ситуации — это то, что
заметная доля общественности ликовала от этого события, как от великой победы.
И тогда, в 14-м, и не один год после они были счастливы и глумились над смертью
своих собратьев, над близкими и родственниками погибших. Мы позволили ненависти
овладеть нами, и в результате мы имеем войну. В первую очередь в головах и
сердцах. Мы не добьёмся процветающей Украины, пока не полюбим нашего брата,
сестру. 

Любовь — всему начало, Любовь — это Бог. А ненависть и злость — низменные
чувства, которые ведут нашу страну к смерти. Мы все-таки не каннибалы. Я
надеюсь, что не каннибалы. Если мы будем идти таким путём, в Украине просто не
останется людей, они сгорят духовно в бесконечных склоках или сожгут друг друга
в огне войны. Или сбегут из этого ада в другие, более комфортные, более
человечные страны. Если мы не найдём силы любить друг друга, мы потеряем всё,
да и самой Украины не станет. Давайте вспомним погибших в этот день, их не
вернуть, но мы ещё можем спасти свои души от огня ненависти, а нашу Украину —
от смерти в этом бессмысленном огне. 

Екатерина Жарких
\end{multicols}

