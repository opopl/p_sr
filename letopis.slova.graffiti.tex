% vim: keymap=russian-jcukenwin
%%beginhead 
 
%%file slova.graffiti
%%parent slova
 
%%url 
 
%%author 
%%author_id 
%%author_url 
 
%%tags 
%%title 
 
%%endhead 
\chapter{Граффити}
\label{sec:slova.graffiti}

%%%cit
%%%cit_head
%%%cit_pic
\ifcmt
  pic https://img.pravda.com/images/doc/c/3/c3d2a14-001---------.jpg
	caption Напис на стіні Софії: \enquote{Блуд створив у святий день}; Фото: із особистих архівів Вячеслава Корнієнка
	width 0.4
\fi
%%%cit_text
Але цінність \emph{графіті} не тільки в цьому. За літописним текстом людини не
видно.  А от графіті дозволяють побачити за літерами людське.  – Приміром,
переживання людини: \enquote{Купріяне, гарний же ти, скільки мені ще даремно
розпалятися}. Або \enquote{того не пиши, того не проклинай, а Івану погано}.
Читаю, а у мене перед очима старий буркун, який всім незадоволений і навіть
нашкрябати йому не дають, тож він жаліється.  Хтось довіряв стіні свої мрії:
Свята Софія, допоможи мені стати художником.  Або прокльони: Козьма тать – ти
вкрав м'ясо. Бодай тобі ноги сплутало. Амінь!
%%%cit_comment
%%%cit_title
\citTitle{Непозбувний лох, стіна-психолог та вічна молитва. Про що говорить Київ на стінах, парканах і асфальті}, 
Олена Струк, www.pravda.com.ua, 02.07.2021 
%%%endcit

