% vim: keymap=russian-jcukenwin
%%beginhead 
 
%%file 03_01_2022.fb.fb_group.story_kiev_ua.1.ev_lobanovskij.cmt
%%parent 03_01_2022.fb.fb_group.story_kiev_ua.1.ev_lobanovskij
 
%%url 
 
%%author_id 
%%date 
 
%%tags 
%%title 
 
%%endhead 
\zzSecCmt

\begin{itemize} % {
\iusr{Кучеревский Геннадий}
 @igg{fbicon.thumb.up.yellow} 

\iusr{Лариса Максименко}

\iusr{Наталия Вигерина}
Спасибо за интересный рассказ! С Новым Годом!

\iusr{Сергій Криськов}

И Вам, спасибо за оценку и, С Новым Годом!

\iusr{Тарас Єрмашов}
\enquote{... Евгений Васильевич – тренера, т.е. своего брата}. Тобто, вони були зовнішньо схожі?

\iusr{Сергій Криськов}

Та подивіться ж на фото! Звісно, схожі. Але для художньої самодіяльності це не
має значення.

\iusr{Valentina Urban}

Я прекрасно помню здание этого Института, который был построен где-то то в
середине 60- х. Я жила рядом в пятиэтажке и чтобы попасть на ул.Щорса
приходилось всегда проходить мимо этого здания.

С Новым годом, спасибо за интересный рассказ!

\iusr{Лариса Лобановська}

Хочу добавить, что Лобановский проработал там практически всю жизнь, с 18 лет
до самой смерти в 66... А институт был очень дружный и веселый, народ как
приходил туда после учебы, обычно в пищевом, так и оставался навсегда, поэтому
все друг друга прекрасно знали и дружили. Плюс неплохие зарплаты, довольно
быстро можно было получить квартиру (по тогдашним меркам), ну и, конечно, Куба!
На Кубе перебывали почти все сотрудники, а это чеки, машины... Я не знаю, как
сейчас, но еще лет 10 назад институт работал, хоть и в очень урезанном
виде,чего нельзя сказать о подавляющем большинстве проектных институтов.

\begin{itemize} % {
\iusr{Сергій Криськов}
\textbf{Лариса Лобановська} 

Лариса, спасибо и с Новым годом Вас! Теперь это частная лавочка, 44 сотрудника,
а название сохранилось: Укргіпроцукор. У них есть сайт.

\iusr{Лариса Лобановська}
\textbf{Сергій Криськов} 

это не совсем частная лавочка, еще в 90-е, как на всех предприятиях, институт
приватизировали сотрудники. Потом в году, наверное, 2006-2007 была неприглядная
история с рейдерским захватом института, но институт все-таки отстояли. Как
раз, я так понимаю, нынешний директор Шкабара Е. сумел таки сохранить институт.
Хотя что там сейчас, честно, не знаю.

\iusr{Алла Федоренко}
\textbf{Сергій Криськов} 

Скажите, пожалуйста, в каком году Вы пришли на Укркинохронику и в каком отделе
Вы работали, это не любопытство, просто я и мой муж работали на студии, муж
ассистентом оператора, я техн.редактор отдела готовой продукции, в ноябре 2021
года студии исполнилось 90 лет, созванивались, хотели встретиться, нас очень
мало осталось, к сожалению, но, увы, начался новый виток ковид и встречу
пришлось отложить. Если Вам не трудно, напишите, пожалуйста, будем ждать.
Спасибо. Пусть Новый год принесет всем здоровья и добра!

@igg{fbicon.heart.red}{repeat=2} @igg{fbicon.hands.pray}{repeat=3} 
\end{itemize} % }

\iusr{Володимир Довгуник}

Цікаво!!! Такого з Вашої біографії я не знав, а от про кіностудію знаю. Дякую.
Горжусь що був Вашим Колегою по отельні.

\begin{itemize} % {
\iusr{Сергій Криськов}
\textbf{Володимир Довгуник} Спасибо, Володя, с Новым годом! Про киностудию будет рассказ, готовлю его к дню рождения О.К. Антонова

\begin{itemize} % {
\iusr{Алла Федоренко}
\textbf{Сергій Криськов} Очень интересно, отзовитесь, пожалуйста!!! @igg{fbicon.heart.red}

\iusr{Сергій Криськов}
\textbf{Алла Федоренко} Я в сети

\iusr{Алла Федоренко}
\textbf{Сергій Криськов} 

Я выше Вам написала, нам с мужем очень интересно знать, мы же бывшие
кинохроники, муж работал в группе Косинов Александр, Писанко Игорь и Юра, они
снимали, а я получала разрешения на показ фильмов, то есть работала с цензурой,
муж пришел на студию летом 1975 года, а я в июне 1982 года,

\iusr{Сергій Криськов}
\textbf{Алла Федоренко Косинова} помню, его на киностудии знали все.
Я там работал всего лишь 2 года, 1977-79
В цехе диафильмов, сначала фотографом, потом и.о. фоторедактора.
Возможно, знаете Евгения Солопко - это мой одноклассник и друг.

\iusr{Алла Федоренко}
\textbf{Сергій Криськов} 

Спасибо Вам, Сергей, конечно знаем Женю Солопко, могу сбросить Вам фото из его
фильма, для ВГИКа по свету снимал, начало года 1984, зима, а ему надо было
сессию сдавать ехать, вот Женя меня и попросил сняться в его частевке!!

\ifcmt
  ig https://scontent-frx5-1.xx.fbcdn.net/v/t39.30808-6/271244156_978205263074391_5434021412543900315_n.jpg?_nc_cat=100&ccb=1-5&_nc_sid=dbeb18&_nc_ohc=MZT-C-47Ta0AX9an2GQ&_nc_ht=scontent-frx5-1.xx&oh=00_AT84OJOikxURHlL91EVc_ULbRRIvhY6erKER_JO9oT0OcQ&oe=61E03C2C
  @width 0.2
\fi

\iusr{Алла Федоренко}
\textbf{Сергій Криськов} Если Вы видитесь с Женей, передайте ему большой привет от Юрия и Аллы Пташник. Мне очень понравилась Ваша история, хочется продолжения, спасибо, что откликнулись!!!!

\iusr{Сергій Криськов}
\textbf{Алла Федоренко} Спасибо, передам. Он переехал на ПМЖ в село, купил там дом с участком. Общаемся только по тлф, в Киеве он давно не был, соцсетями принципиально не пользуется.

\iusr{Алла Федоренко}
\textbf{Сергій Криськов} Мы тоже уехали в родное село мужа ещё в 2004 году, он вышел как Чернобылец на пенсию и в апреле будет 18 лет, как мы стали селянами  @igg{fbicon.grin} !
Спасибо Вам за чудесную ночку!!!

\iusr{Сергій Криськов}
\textbf{Алла Федоренко} И Вам спасибо, рад общаться, рад вспомнить молодость... Я сейчас общаюсь на 2 фронта, т. к. нашлись еще бывшие сослуживцы.

\end{itemize} % }

\end{itemize} % }

\iusr{Александр Венге}
Хорошая во всех отношениях история. Спасибо.

\iusr{Анна Сидоренко}
Да, хорошая и правдивая история из жизни и ребята хорошие на фото, молодость......

\iusr{Tamara Pankratova}
Цікаво...

\iusr{Irena Moskalenko}
I worked with in same department couple of years from 1978 after graduate from KPI
The men behind Lobanovsky and 2 others from the left looks familiar
Do you remember they’re name’s.

\begin{itemize} % {
\iusr{Сергій Криськов}
\textbf{Irena Moskalenko} Крайний слева (1) - не помню.
\begin{itemize}
  \item 1. (мне напомнили) И. Хомяк
  \item 2. зам.нач.отдела В. Мохорт
  \item 3. ст. инж. И.Н. Барбарчук
  \item 4. Е.В. Лобановский
  \item 5. рук.бриг. Г.Д. Бобровник
  \item 6. ст.техник М. Макиевский
  \item 7. ст.инж. В. Мурашко
  \item 8. ст.инж. П.Е. Гандюхин
  \item 9. рук.бриг. М.В. Терехов
  \item 10. ст. инж. Волченко (Вовченко?)
\end{itemize}
\end{itemize} % }

\iusr{Irena Moskalenko}
спасибо

\iusr{Олена Медведева- Прицкер}

С Новым Годом! Уважаемый Сергей! Спасибо, очень интересные воспоминания о
проектном институте, где я тоже трудилась, позже Вас, с 1988г. По 1997г.
Евгений Васильевич Лобановский был в то время директором \enquote{Гипросахара}.
Любимый и уважаемый человек. Когда начали развал сахарной промышленности и
уничтожение этого института, сердце директора не выдержало и он скончался.
Проектные институты закончили свою деятельность, все сотрудники остались без
работы и у каждого началась какая- то другая жизнь. Но при встречах всегда
повторяли - если бы Лобановский остался жив - мы бы не остались без работы.

\begin{itemize} % {
\iusr{Сергій Криськов}
\textbf{Олена Медведева- Прицкер} 

Да, очень-очень жаль. В последний раз я увиделся с Евгением Васильевичем в
середине 90-х, возле дворца \enquote{Украина}. Мы встретились в очереди на сдачу
ваучеров мошенникам.

\end{itemize} % }

\iusr{Tata Femina}
Печально как(

\iusr{Ольга Опанасенко}

Я Опанасенко Ольга работал в институте с 1980 по 1989 год. Проектную работу не
любила но часто вспоминаю интересных, мудрых, интеллигентных людей, работающих
в институте. Особенно помню безумно интересные, подмеченные вовремя смешные
истории, новогодние выступления каждого отдела института. Безусловно всю
дружескую атмосферу работников института поддерживал Лобановский Е.В. Ещё помню
я ехала домой со своей подругой в 11 троллейбусе и что- то громко обсуждали
ситуацию, связанную с Лобановский. А потом оглянувшись назад я увидела самого
Лобановского и тут душа ушла в пятки. Он улыбнулся а я думала что завтра мне
капец. Но все обошлось, благодаря его чувства юмора и доброй души. Светлая
память о нем.

\end{itemize} % }
