% vim: keymap=russian-jcukenwin
%%beginhead 
 
%%file 22_11_2021.fb.zolotjko_jaroslava.kiev.zhurnalist.1.mysli_o_evromaidane.cmt
%%parent 22_11_2021.fb.zolotjko_jaroslava.kiev.zhurnalist.1.mysli_o_evromaidane
 
%%url 
 
%%author_id 
%%date 
 
%%tags 
%%title 
 
%%endhead 
\subsubsection{Коментарі}

\begin{itemize} % {
\iusr{Марина Гусарь}

Мысли об экономике вообще вызывают чувство безысходности. Сколько предприятий
закрылось за последние годы?! Среди них и те, которые работали больше 100 лет.
Все пережили - революции, войны. Вот только новые украинские реалии не смогли
пережить. А как может выжить страна, в которой уже почти ничего не работает. Мы
будем аграрным гигантом, обещали нам. Прошло 8 лет. Мы экспортируем сахар
(вспомним Терещенко, Бродского и пр. сахарозаводчиков), картофель, а теперь уже
даже сахар и огурцы. Вот такой получился праздник

% -------------------------------------
\ii{fbauth.osetinskij_vitalij.melitopol.ukraina.krivoj_rog.apteka.kopijka}
% -------------------------------------

Ярослава, отозвалось болью каждое слово. И мне грустно. Что вспоминаю я в этот
день? Лица погибших парней в погонах (да, протестующих тоже жаль, но я мысленно
был по другую сторону баррикад). Слякотный февральский день у горисполкома
вместе с ребятами из ВВ. Коктейли Молотова. Дальше - отчаяние. Никаких светлых
надежд не было. Только чернота и отчаяние. Ненависть вокруг. Визит
националистов ко мне на работу. Угрозы. Вызовы в СБУ...

\begin{itemize} % {
\iusr{Дмитрий Руцкий}
\textbf{Виталий Осетинский} по какому поводу Вас вызывали в СБУ?

\iusr{Виталий Осетинский}
\textbf{Дмитрий Руцкий} 

у нас с 2011 года вполне легально работала общественная организация. Казачий
кадетский класс. Вроде скаутского клуба для школьников. Ходили с ребятами в
походы, учились ориентироваться на местности и т. д. В 2014 началась "охота на
ведьм". Вылезло на свет Божий намеренно "сознательных" граждан, считающих своим
гражданским долгом непременно на кого-нибудь настучать. Не понравились чем-то и
мы. Флагами не махали, в вышиванки не рядились. К счастью, офицеры СБУ
оказались вменяемыми. Я написал расписку, что "обязуюсь не вести антиукраинскую
деятельность" и все на этом закончилось. Но деятельность организации пришлось
прекратить, чтобы не подставлять под удар детей и их родителей.

\end{itemize} % }

\iusr{Лидия Сивай}

Просто для сравнения. 73\% выбрали нынешнего главу государства. Вероятный вывод:
27\% были за других, т. е. против этого. Удивительно, но как большинство
представляло себе артиста-комика в качестве руководителя страны? А теперь
сравнение. Ксения Собчак также баллотировалась в президенты, она в чём-то
сродни артисту-комику, однако её выбрали 0,0... \%. Точную цифру не помню. И
ещё. Очень бы хотелось, чтобы страдания людей прекратились, люди, несмотря на
то что не разобрались, не заслуживают такого "европейского" уровня жизни!

\begin{itemize} % {
\iusr{Юрий Зущик}
\textbf{Лидия Сивай} мы все получаем то, что заслуживаем. исключений из этого правила не вижу
\end{itemize} % }

\iusr{Лидия Сивай}

Дополнение. Кстати, уровень жизни в Эстонии, стране ЕС, в разы ниже, чем в
других странах. Почему? Да потому что это бывшая советская территория, и она
рассматривается "западными партнёрами" под другим узлом зрения, как,
собственно, и все бывшие советские территории. В Эстонии такая мизерная пенсия,
что не работающего пенсионера не встретишь и днём с огнём не найдешь. У кого
позволяют силы, всё работают. Моей подруге уже 71 и её мужу столько же, оба
работают и не представляют свою жизнь только на пенсионные выплаты. При этом,
конечно, быт получше, медицина отлажена, причём хорошо поставлена электронная
связь медиков с пациентами - не надо сидеть в очереди и вообще бегать за
рецептами. По электронке выдают и в аптеке. Правда, первых три дня болезни
вообще не оплачиваются, а оплата начинается с 4-го дня, причём сначала 70\%
оклада и только потом, вдруг болезнь затянется, то, возможно, оплата составит
80\%. И оклады у них раза в три поменьше, чем, в частности, в Финляндии.

\begin{itemize} % {
\iusr{Ярослава Золотько}
\textbf{Лидия Сивай}, 

в Польше с начала пандемии семейные врачи открывают больничные по телефону т.к.
у них на дом не ходят, но и пациентов с подозрением на ковид не гонят в
поликлиники. А у меня знакомый в Киеве 5 часов (!!!) сидел к дежурному врачу
чтобы открыть больничный. Вопрос - скольких он за это время успел заразить. В
Чехии врачи не ходят к пациентам на дом, да, но и пациенты не сидят под дверью
по 5 часов. Разговаривала с нашими гражданами, которые туда перебрались,
говорят максимум ожидания - 10 минут.

\iusr{Лидия Сивай}
\textbf{Ярослава Золотько} 

Да, это европейская действительность. Просто нашим людям надо бы себя с начала
почувствовать европейцами, А потом строить европейский образ жизни. Но со своим
менталитетом 'европу' не построить. Не то внутреннее содержание. Но оно,
возможно, и не нужно. Вполне достаточно обыкновенного гуманизма, причём
свойственного нашему народу. Люди! Посмотрите на себя! Вы можете многое! Не
берите пример не известно с кого! Это моё мнение.

\end{itemize} % }

% -------------------------------------
\ii{fbauth.zuschik_jurij.kiev.ukraina.fotograf}
% -------------------------------------

это все потому, что мы живем на зоне. и правят нами воры. а если говорить
по-умному, то у нас слишком высокая коррупционная рента - она закладывается в
цены товаров и услуг, денег на развитие банально не хватает, все
разворовывается

\end{itemize} % }
