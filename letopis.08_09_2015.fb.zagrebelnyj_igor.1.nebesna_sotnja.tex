% vim: keymap=russian-jcukenwin
%%beginhead 
 
%%file 08_09_2015.fb.zagrebelnyj_igor.1.nebesna_sotnja
%%parent 08_09_2015
 
%%url https://www.facebook.com/permalink.php?story_fbid=1486253628365952&id=100009439885823
 
%%author_id zagrebelnyj_igor
%%date 
 
%%tags 2014,maidan2,nebesna_sotnja,ukraina
%%title Небесна Сотня
 
%%endhead 
 
\subsection{Небесна Сотня}
\label{sec:08_09_2015.fb.zagrebelnyj_igor.1.nebesna_sotnja}
 
\Purl{https://www.facebook.com/permalink.php?story_fbid=1486253628365952&id=100009439885823}
\ifcmt
 author_begin
   author_id zagrebelnyj_igor
 author_end
\fi

Ще з весни 2014 року мене страшенно нервувало те, що з так званої Небесної
сотні роблять тєрпіл. Так, ніби це якісь \enquote{учасники мирного протесту}, яких
розстріляли чи забили кийками мусора через власні садистично-маніакальні
настрої.

Такий погляд на загиблих майданівців цілком вкладається в парадигму політичного
мислення, яку відстоюють ті, хто прийшов до влади по крові Небесної сотні. Адже
у цій ліберально-демократичній парадигмі легітимною є жертва, але аж ніяк не
герой. В її межах симпатія належить тому, хто постраждав, але не тому, що сам
використовує насильство у благих цілях.

Насправді Небесна сотня об'єднує у своїх рядах і перших, других — і тих, хто
реально \enquote{постраждав безвинно}, і тих, хто загинув, так би мовити, зі зброєю в
руках.

Яника вдалося вигнати не тому, що з боку майданівців загинуло чоловік двісті чи
більше, а тому, що загинуло чимало бійців \enquote{Беркуту}, міліціонерів, солдатів
внутрішніх військ (тітушок ніхто не рахує, хоч їх також загинуло чимало).  Тому
що була зламана їхня воля до спротиву, і Яника не було кому захищати. Це було
досягнуто через бруківку, пляшки з запалювальною сумішшю, саморобні гранати,
вогнепальну зброю, якої ставало все більше і більше (власне, хід подій вирішило
те, що під ранок 20 лютого на Майдані з'явилося залізяччя, привезене з Західної
України).

Але офіційним \enquote{герменевтам Майдану} невигідно говорити правду. Зараз, прийшовши
до влади, вони зацікавлені, щоб люди і надалі сприймали Небесну сотню як
зборище тєрпіл. В інакшому випадку виходитиме, що вони прийшли до влади
внаслідок збройного перевороту (нехай і погано озброєного), внаслідок
радикального порушення монополії держави на насильство, внаслідок численних
убивств співробітників державних силових відомств. А вони всього цього бояться
— бояться збройного перевороту, бояться порушення монополії на насильство,
опікуються своїми цепними псами (чого лиш варте посмертне нагородження
ВВшиників, які загинули під парламентом).
