% vim: keymap=russian-jcukenwin
%%beginhead 
 
%%file slova.mentalnost
%%parent slova
 
%%url 
 
%%author 
%%author_id 
%%author_url 
 
%%tags 
%%title 
 
%%endhead 
\chapter{Ментальность}
\label{sec:slova.mentalnost}

%%%cit
%%%cit_pic
%%%cit_text
Полноценная реинтеграция оккупированных территорий наступит только после
\emph{ментальной} реинтеграции жителей Крыма и Донбасса — замминистра Министерства
оккупированных территорий Замлинский. Как вы понимаете, речь не о том, чтобы
Украину сделать такой, чтобы она стала \emph{ментально} близкой для миллионов людей в
Крыму и Донбассе. Речь о том, что \enquote{мы научим русню Украину любить}.
Реинтеграция по ним — это \emph{принудить ментально} миллионы людей к декоммунизации,
украинизации, бандеризации и НАТО. Сделать всех бандеровцами.  Но этим
\enquote{реинтеграторам} за 7 лет не удалось \enquote{реинтегрировать} ни меня,
ни миллионы украинцев в самой Украине в свою \emph{бандеровскую ментальность}. И пока
МинВОТ возглавляет кучка недалёких галичан типа Резникова и вот этого
Замлинского, Крым и Донбасс будут \emph{ментально} с каждым днём всё дальше от
Украины. Вот такая она — реинтеграция от Зеленского
%%%cit_comment
%%%cit_title
\citTitle{Замминистра оккупированных территорий назвал условия реинтеграции}, 
Александр Скубченко, strana.ua, 12.06.2021
%%%endcit


%%%cit
%%%cit_head
%%%cit_pic
%%%cit_text
Чому зовнішня політика України є настільки провінційною? Відповідь полягає в
тому, що в українському суспільстві був сформований жіночий тип \emph{ментальності}.
Українці дуже емоційні, і на відміну від американців і німців - менш
раціональні та прагматичні. Наприклад в США домінує чоловічий тип \emph{ментальності},
який формує прагматизм американців. На першому місці для американців завжди
стоїть особиста вигода. Українцям же які є носіями жіночого типу \emph{ментальності},
як і будь якій жінці, подобаються компліменти та яскраві образи, які викликають
емоції
%%%cit_comment
%%%cit_title
\citTitle{Що чекає на Україну?}, Стефан Закревський, 
analytics.hvylya.net, 19.06.2021
%%%endcit

%%%cit
%%%cit_head
%%%cit_pic
%%%cit_text
Вы поймите, это \emph{ментальные} структуры, на которых нельзя построить страну. Это
чужды будущему структуры. Чужды. Так нельзя. Это аморально. Такой подход ведет
к неспособности очищаться сегодня, потому что, что такое коррупция? Коррупция –
это детский сад по сравнению с тем, в чем оказалась сегодня Украина.  Что вы
имеете ввиду?  Потому что, когда у вас элита страны договаривается с элитой
враждебного государства за бизнес-преимущество расколоть страну это уже не
коррупция, это уже даже не измена, это враждебные действия. Можем ли мы сегодня
очиститься? То есть, как бы Гена Друзенко и Юра Романенко они же так ставят
вопрос: мы способны от этого очиститься или нет? Это вопрос сегодня будущего
%%%cit_comment
%%%cit_title
\citTitle{Сергей Дацюк: Украина сегодня  - не просто попрошайка, она на мусорнике истории}, 
Сергей Дацюк; Людмила Немыря, hvylya.net, 28.06.2021
%%%endcit

%%%cit
%%%cit_head
%%%cit_pic
%%%cit_text
Ми багато що успадкували від Радянського Союзу. Більше, ніж хотілось би
сучасній, свідомій людині. Наш спадок є і матеріальним, і \emph{ментальним}.
\emph{Ментальний} зв'язок з минулим якраз і полягає в тому, що держави, яка б поважала
особистість, у нас нема. Навіть після 30 років незалежності.  А щодо
матеріального спадку, то ми «доїдаємо» його і по цей день. Не створюючи нічого
принципово нового. І це також характеризує \emph{менталітет} влади: не пробувати, не
намагатися, не здійснювати прориви. Не жити самим (або жити лише на крадене) і
не давати жити іншим – власним громадянам.  Ми користуємося тією самою
житлово-комунальною системою, яка перейшла до нас від СРСР. Нинішні ОСББ важать
небагато, якщо їм доводиться мати справу з латаними і недолатаними трубами
епохи раннього Брєжнева чи пізнього Хрущова. В країнах Балтії спромоглися
замінити всі радянські комунікації, а в Україні – чомусь ні. У нас і досі тече
вода, придатна хіба що для технічних потреб, по трубах, які пам’ятають період
застою
%%%cit_comment
%%%cit_title
\citTitle{Весь сенс нашої боротьби}, Михайло Поживанов, lb.ua, 25.11.2021
%%%endcit
