% vim: keymap=russian-jcukenwin
%%beginhead 
 
%%file 18_01_2022.tg.tkachev_jurij.1.specoperacia_amerika
%%parent 18_01_2022
 
%%url https://t.me/dadzibao/5085
 
%%author_id tkachev_jurij
%%date 
 
%%tags infvojna,napadenie,rossia,ugroza,ukraina,usa
%%title Слухи о нападении Путина могут быть американской спецоперацией
 
%%endhead 
\subsection{Слухи о нападении Путина могут быть американской спецоперацией}
\label{sec:18_01_2022.tg.tkachev_jurij.1.specoperacia_amerika}

\Purl{https://t.me/dadzibao/5085}
\ifcmt
 author_begin
   author_id tkachev_jurij
 author_end
\fi

Собственно, распространение американцами слухов о грядущем нападении России
могут быть \enquote{вещью в себе}: уже само по себе распространение таких слухов
создаёт невыгодную для России информационную обстановку во всём мире. И не
знаю, как насчёт всего мира, а вот в Украине американцам удалось напугать
граждан преизрядно: такого уровня нервозности я не припоминаю с 2014-го.

И если за пределами Украины происходит что-то хоть немного похожее, то на
экономику России это неизбежно окажет негативное влияние даже без всяких
санкций: даже внутрироссийский бизнес почти наверняка повременит с реализацией
каких-то инвестиционных проектов, предпочтя придержать деньги до более
определённых времён. Что же до иностранцев, то о них и говорить не приходится,
а иностранные деньги (включая, скажем, кредиты крупных международных банков)
России и российскому бизнесу всё ещё нужны.

И если считать нагнетание военных слухов антироссийской операцией самим по
себе, то многие наблюдаемые странности обретают простое объяснение.

У этого плана, впрочем, есть один веский недостаток: это вряд ли сработает
долго. Человеческое общество и его отдельные сегменты не могут долго жить в
состоянии невроза: даже по той же Украине можно судить, что продержать людей в
настроении ожидания неизбежного российского нападения можно от силы полгода,
максимум год (но это уже с риском перемещения в психушку на постоянной основе).
Потом \enquote{ваша рыжая морда примелькается и вас начнут попросту бить}, как это
произошло в Украине, на которую \enquote{Путин нападал на следующей неделе} по пару раз
в год 8 лет. И где этому уже почти перестали верить, и потребовалось
вмешательство пока не дискредитировавших себя американцев для того, чтобы это
прокатило на этот раз.

То есть, по сути в рамках этой концепции России стоит просто запастись
терпением и переждать волну. Благо резервы для этого есть: возможно, нынешняя
ситуация - как раз отличный повод распечатать кубышку Фонда национального
благосостояния для поддержания экономики путём прямых и косвенных инвестиций.

И второй аспект: экономически данная кампания бьёт не только по России, но и по
Украине. Мы с вами уже говорили об обвале украинских еврооблигаций, теперь вот
пишут об активном выводе средств из украинских внутренних облигаций:
иностранные держатели с начала года сократили своё присутствие в них на 5\%. 

Стоит отметить, впрочем, что панические настроения будут по украинской
экономике слабее, чем по российской. Как я уже говорил, первой жертвой таких
штук становится рынок инвестиций, который в Украине и так практически мёртв; а
то, что мертво, как известно, умереть не может. Впрочем, одними инвестициями
всё не ограничится: если, к примеру, не удастся остановить начавшийся рост
курса доллара, то это уже ударит по внутреннему потребительскому рынку. А с
учётом дороговизны газа, этот удар практически не будет компенсирован
экспортёрами, обычно получающими рост прибылей при ослаблении курса нацвалюты. 

Однако все эти вещи вряд ли будут волновать кого-то кроме самих украинских
граждан. А в том, что их благополучие американцы принесут в жертву ради своих
тактических целей, у меня нет сомнения. 

Плохо быть иностранной колонией. По возможности избегайте этого.

\url{https://t.me/dadzibao/5086}

...но хуже всего если, как в известной пословице, миром управляет не тайная
ложа, а явная лажа.

Представим себе на минутку, что американская или британская разведка
действительно на тех или иных основаниях пришли к выводу, что Россия готовится
напасть на Украину. В конце концов, разведчики тоже способны ошибаться, не так
ли? 

Представим себе теперь, как на сообщения о готовящемся российском нападении
должны реагировать в самой России в ситуации, когда сами они знают, что
нападать не собираются? А одним-единственным образом: как внешнеполитическую
провокацию. Зачем? Ну, затем, чтобы оправдать, к примеру, украинское
наступление на Донбассе, а точнее, заранее дискредитировать возможный
российский на него ответ. Это, чёрт возьми, логично, да мы с вами тоже так
подумали, не так ли? 

Что начинает делать Россия в ситуации, когда уверена, что Украина готовится к
войне? Правильно, начинает готовиться к ответке. 

И вот уже сами американцы, видя эту подготовку, получают \enquote{блестящее}
подтверждение докладам своей разведки: \enquote{Да, Путин таки собирается напасть}. И
начинают принимать свои контрмеры: спешно начинают перебрасывать в Украину
технику, \enquote{вешают} свои военные самолёты.

Как эту натовскую активизацию воспринимают в России, где точно знают, что
нападать не намерены? Да понятно как: как подтверждение факта подготовки к
операции в Донбассе. Ну и можно ожидать более активных контрмер: приведение
войск в повышенную готовность и так далее.

Что не останется без внимания американцев, которые... Ну, то есть ситуация
начинает накручиваться, накручиваться и накручиваться, причём по сути
бесконтрольно со стороны обоих участников.

И вот уже ситуация оказывается наэлектризованной настолько, что на линии фронта
кому-то достаточно особенно шумно испортить воздух, чтобы это приняли за
условный сигнал начала большой войны. 

Которой, в рамках обсуждаемой нами гипотезы, стороны не готовились и оно вроде
как \enquote{само насралось}. 

Самое же печальное здесь в том, что жертвой этой ситуации так или иначе станут
жители Украины и Донбасса. 

Хреново быть иностранной колонией, по возможности избегайте этого,
