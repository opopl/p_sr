% vim: keymap=russian-jcukenwin
%%beginhead 
 
%%file 22_10_2021.fb.bilchenko_evgenia.3.moja_prezentacia_v_moskve
%%parent 22_10_2021
 
%%url https://www.facebook.com/yevzhik/posts/4368111636557246
 
%%author_id bilchenko_evgenia
%%date 
 
%%tags bilchenko_evgenia,moskva,prezentacia,rossia
%%title БЖ. Моя презентация в Москве: ущипните меня!
 
%%endhead 
 
\subsection{БЖ. Моя презентация в Москве: ущипните меня!}
\label{sec:22_10_2021.fb.bilchenko_evgenia.3.moja_prezentacia_v_moskve}
 
\Purl{https://www.facebook.com/yevzhik/posts/4368111636557246}
\ifcmt
 author_begin
   author_id bilchenko_evgenia
 author_end
\fi

БЖ. Моя презентация в Москве: ущипните меня!

Всё самое интересное - без меня происходит, пока я в депрессии лечусь
гидазепамом, коньяком, антибиотиками и позолоченной осенью в Киеве. Надеюсь,
это не сон. Мой драгоценный издатель Игорь Савкин сейчас сообщает, что
состоялось представление моей книги в Высшей школе экономики на знаменитой
Октябрьской конференции. 

\ifcmt
  tab_begin cols=4

     pic https://scontent-frt3-2.xx.fbcdn.net/v/t1.6435-9/247464925_4368111309890612_2316028845218898061_n.jpg?_nc_cat=101&ccb=1-5&_nc_sid=8bfeb9&_nc_ohc=LaRCjMiVMjAAX9Q4g3_&_nc_ht=scontent-frt3-2.xx&oh=492725ad654cc55aae1772f765336df5&oe=619B889D

     pic https://scontent-frt3-1.xx.fbcdn.net/v/t1.6435-9/247852318_4368111393223937_1019419155198828306_n.jpg?_nc_cat=106&ccb=1-5&_nc_sid=8bfeb9&_nc_ohc=znMQJz8i_MsAX8FttJm&_nc_ht=scontent-frt3-1.xx&oh=cddf24f6480e320b7c4f23c6a2818cf7&oe=6198A0E5

		 pic https://scontent-frx5-1.xx.fbcdn.net/v/t1.6435-9/247690310_4368111463223930_8181584423392080742_n.jpg?_nc_cat=105&ccb=1-5&_nc_sid=8bfeb9&_nc_ohc=e6CVKGJysCkAX_kRh8k&_nc_ht=scontent-frx5-1.xx&oh=d99b834184281a61ad51ec1156eede17&oe=61985CBC

		 pic https://scontent-frt3-1.xx.fbcdn.net/v/t39.30808-6/247728738_4368111789890564_5153812350571764270_n.jpg?_nc_cat=107&ccb=1-5&_nc_sid=8bfeb9&_nc_ohc=ytbHK2nWcCsAX-6Gjwz&tn=lCYVFeHcTIAFcAzi&_nc_ht=scontent-frt3-1.xx&oh=ef8e2d47bf16b44a098d5c4fe3732b68&oe=6178C8D3

  tab_end
\fi

На фото: профессор А.Ю. Сунгуров, директор департамента политологии ВШЭ, и моя
юношеская любовь - профессор Григорий Тульчинский, который казался мне старше,
я же его с 17 лет читаю. А он - вот какой молодой и озорной. Помню, как те, кто
уволил меня из НПУ, молились на него, ругали меня за "неправильное" его
истолкование в докторской (из-за этого даже хотели защиту отложить) и... дружно
его предали после 2014. А он написал мне позитивную рецензию. Ну, не Сократ ли,
не культура ли высокой академической дискуссии? Также в книге присутствуют
интерпретации Стефания Данилова и Александра Секацкого: это едино и полярно
одновременно, хорошие Весы.

Честно? Я никогда не верила в подобное. Не верила по двум причинам: что такие
люди обратят на меня внимание и что моя нервная традиционалистская и
марксистская критика неолиберализма и американизма зацепит респектабельную
гуманитарную и, собственно, либеральную тусовку. А я ещё и транскультуру
критикую, и ничего: мы говорим, мы обмениваемся разными мнениями, мы ищем
истину как дух целого.

Ощущаю себя дурной, недоразвитой, маленькой и нерукопожатной, как будто
откуда-то из красивого дворца вдруг из динамика орет мой хриплый Егор Летов. То
ли комплексы, то ли не верю своим глазам, то ли ощущение, что меня УСЛЫШАЛИ.
Это стоило увольнения меня за "контакты с русским миром" и всех страданий
фрилансера, потому что это давно искомый мной профессиональный академизм. 

Мне прямо вот хочется всех соборно любить и сказать огромное спасибо лично
Игорю, издательству "Алетейя" и всем-всем-всем за этот диалог, за внимание к
мнению Иного, за способность услышать. Россия, спасиБог. Ты крутая и классная.
Реально. \textbf{\#russianbrand}
