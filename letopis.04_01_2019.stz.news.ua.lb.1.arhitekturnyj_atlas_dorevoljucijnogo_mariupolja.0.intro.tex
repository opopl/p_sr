% vim: keymap=russian-jcukenwin
%%beginhead 
 
%%file 04_01_2019.stz.news.ua.lb.1.arhitekturnyj_atlas_dorevoljucijnogo_mariupolja.0.intro
%%parent 04_01_2019.stz.news.ua.lb.1.arhitekturnyj_atlas_dorevoljucijnogo_mariupolja
 
%%url 
 
%%author_id 
%%date 
 
%%tags 
%%title 
 
%%endhead 

\begin{quote}
\bfseries
В массовом сознании за Мариуполем давно и железно закрепился образ флагмана
украинской металлургии. В последние годы к этому добавился еще более суровый
имидж переднего края боевых действий. Приазовский город часто упоминается в
контексте военных событий или промышленных показателей. Но где-то там, под
прахом веков и слоем заводской пыли, есть ещё один Мариуполь – старый город
шумных базарных площадей, крутых мощеных улочек и солидных купеческих
особняков. Он заслонен артефактами индустриализации в историческом,
географическом и ментальном смыслах.

В отличие от заводских труб, старый Мариуполь легко не заметить, потому что
архитектурных свидетельств существования этого города осталось не так уж и
много. Вернее, большая их часть находится в непрезентабельном состоянии, а то,
что можно с гордостью демонстрировать - скорее разрозненные фрагменты, по
которым трудно представить целое. Судьба старого Мариуполя такая же сложная и
исковерканная, как и судьба современного.

LB.ua продолжает серию архитектурных гидов по украинским городам.
\end{quote}

Мариуполь появился на картах в конце ХVІІІ века. В 1882 году к побережью
Азовского моря от станции Еленовка дотянулась ветка Екатерининской железной
дороги. А в 1889 заработал современный глубоководный порт. Эти два
инфраструктурных проекта послужили прологом к большому будущему. Они сделали
город привлекательным для иностранных инвесторов, а местному купечеству открыли
мировые рынки. Мариуполь стал расти, как на дрожжах. Задымили два
металлургических завода, открывались иностранные консульства и торговые
представительства, в лавках можно было купить товары со всей Европы. Вскоре
появились местные миллионеры. Интересно отметить, что историко-культурный центр
Мариуполя всю досоветскую эпоху совпадал со старыми укреплениями Кальмиусской
паланки.

Позже по идеологическим причинам связи с козацким прошлым были разрушены. От
этого центра в западном направлении исходили три главных городских улицы,
образовывая в плане трезубец. Именно в этом районе стали появляться самые
значимые архитектурные объекты. Похвастать согласованностью архитектурных
решений Мариуполь не может, так как строительство велось хаотично, в
соответствии со вкусами и желаниями каждого отдельного заказчика. Большинство
построек того периода выполнено в эклектичном стиле, но иногда можно четко
заметить влияние неоклассицизма, модерна и даже неоготики. Кирпичная застройка
рубежа ХІХ – ХХ веков это как раз та самая \enquote{мариупольская архитектурная
старина}, которая досталась в наследство современным горожанам.

К началу прошлого века многие \enquote{акулы} местного бизнеса уже могли себе позволить
масштабное капитальное строительство. Спустя сто лет частные особняки и
коммерческая недвижимость богатейшей части городской элиты составили
значительную часть архитектурного наследия Мариуполя.

