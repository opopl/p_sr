% vim: keymap=russian-jcukenwin
%%beginhead 
 
%%file 11_01_2022.fb.menendes_enrike.1.ukraina_2030
%%parent 11_01_2022
 
%%url https://www.facebook.com/e.menendes/posts/6797911273584574
 
%%author_id menendes_enrike
%%date 
 
%%tags 2030,donbass,ukraina,vojna
%%title 2030 год. Украина
 
%%endhead 
 
\subsection{2030 год. Украина}
\label{sec:11_01_2022.fb.menendes_enrike.1.ukraina_2030}
 
\Purl{https://www.facebook.com/e.menendes/posts/6797911273584574}
\ifcmt
 author_begin
   author_id menendes_enrike
 author_end
\fi

2030 год. Украина.

На Донбассе наступил мир. В 2027 году были подписаны прорывные соглашения между
Украиной и представителями Л/ДНР о прекращении вооружённого конфликта и
реинтеграции ОРДЛО в Украину. Гарантами соглашения выступили Германия, Франция
и Российская Федерация.

Украина представляет собой государство с нейтральным и внеблоковым статусом. В
Конституции страны записано не участие в различных военных союзах и запрет на
размещение иностранных военных баз на её территории.

Новое административно-территориальное деление Украины подразумевает создание 8
макро-регионов. Все имеют статус автономии, которая подразумевает значительное
влияние на жизнь этой территории. Один из этих регионов – автономная республика
Донбасс, которая включает территории Донецкой и Луганской областей на 2013 год.

В стране два официальных языка – украинский и русский. Некоторые региональные
парламенты приняли статус других языков в качестве региональных.

Все названия городов, улиц и памятники утверждаются на региональном уровне.
Существуют общенациональные и чисто региональные памятные даты.

Украина является кандидатом на вступление в ЕС. Между тем, подписаны новые
торговые соглашения с участием Евросоюза и России, которые регулируют выгодные
условия торговли между всеми партнёрами.

Ключевыми торговыми партнёрами Украины, в порядке объема, являются ЕС, РФ,
Китай и страны Ближнего Востока. За 2029-2030 финансовый год, объём украинского
экспорта вырос по всем четырём направлениям. Годовой бюджет профицитный. Сальдо
внешнеторгового баланса положительное.

Создан международный трибунал по военным преступлениям на Донбассе. Десятки
военных преступников с обеих сторон попали под его действие. Игорь Гиркин
(Стрелков) выдан Российской Федерацией, поскольку доказана его вина в
совершении тяжких преступлений. Бывший глава украинского СНБО Александр
Турчинов встречает свой 66-й день рождения в тюрьме, где он отбывает срок за
преступления против украинского народа.

Средняя зарплата по стране эквивалентна 1100 долларам в месяц. Этого удалось
добиться благодаря масштабным программа реиндустриализации экономики и
возрастающей отдаче инвестиций. Вовсю идёт восстановление Донбасса на основе
новой промышленности. Это драйвер всей украинской экономики.

Вопрос Крыма ещё не решён окончательно, но Украина и РФ подписали пакет
соглашений, согласно которому вводится управление регионом на основе концессии.
В Крыму уравнены российское и украинское гражданство. Украина согласилась на
компенсацию за потерю госсобственности в Крыму в обмен на открытие
логистических возможностей и признание нахождения в Крыму Черноморского флота
РФ. Остальные вопросы решаются в рабочем порядке.

Обанкротились ботафермы, которые сеяли ненависть в социальных сетях. Некоторые
из самых активных деятелей, получили реальные сроки за разжигание розни и
призывы к насилию. Уроки толерантности и примирения проводятся в каждой школе
на территории Украины.

Украинские военные, которые участвовали в АТО получили серьёзные пенсии и
надбавки, при условии, что они не замешаны в военных преступлениях. Бывшие
военнослужащие Л/ДНР получили возможность вступить в ряды ВСУ, приняв присягу.
Они тоже имеют военные пенсии, но относительно привелегий до сих пор идут
прения в региональном парламенте  - федеральное правительство в Киеве
собирается ветировать эту инициативу. Многие бывшие военнослужащие Л/ДНР служат
в органах внутренних дел и республиканской гвардии, принимая украинскую
присягу.

Вы скажете я романтик? Мечтатель? Возможно, вы будете правы. Но разве это не
достойная цель, чтобы мечтать о ней?

\ii{11_01_2022.fb.menendes_enrike.1.ukraina_2030.cmt}
