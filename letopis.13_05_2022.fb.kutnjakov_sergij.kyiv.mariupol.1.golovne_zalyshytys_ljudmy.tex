%%beginhead 
 
%%file 13_05_2022.fb.kutnjakov_sergij.kyiv.mariupol.1.golovne_zalyshytys_ljudmy
%%parent 13_05_2022
 
%%url https://www.facebook.com/100000943996304/posts/pfbid02zGMz5aCbyEDbSYmPtaVy7Jg7hMJ2jMA5Gb1ZwJC1wi1UWKXAaC6Lg4uMLgRkbT3ol
 
%%author_id kutnjakov_sergij.kyiv.mariupol,demidko_olga.mariupol
%%date 13_05_2022
 
%%tags 
%%title Пам'ятаємо, що нині всім складно, але головне - залишитись Людьми
 
%%endhead 

\subsection{Пам'ятаємо, що нині всім складно, але головне - залишитись Людьми}
\label{sec:13_05_2022.fb.kutnjakov_sergij.kyiv.mariupol.1.golovne_zalyshytys_ljudmy}
 
\Purl{https://www.facebook.com/100000943996304/posts/pfbid02zGMz5aCbyEDbSYmPtaVy7Jg7hMJ2jMA5Gb1ZwJC1wi1UWKXAaC6Lg4uMLgRkbT3ol}
\ifcmt
 author_begin
   author_id kutnjakov_sergij.kyiv.mariupol,demidko_olga.mariupol
 author_end
\fi

Сьогодні на дитячому майданчику почула від співгромадянки, що зараз весь світ
повинен допомагати нам - українцям, давати  безкоштовне житло і годувати
нас...🤯😪  На її питання, коли планую повертатися в Україну, сказала, що вже
влітку хочу жити з батьками в якомусь українському місті... Вона дивилась на
мене, як на божевільну. 😅 Після сьогоднішньої розмови знайшла  ПАМ'ЯТКУ
ПОРЯДНОГО ПЕРЕСЕЛЕНЦЯ, яка наразі потрібна всім українцям, адже за нами всі
спостерігають і з'ясовують хто ми такі. І ми привозимо із собою найгірше чи
найкраще, що є всередині нас.

І все ж потрібно пам'ятати: 😊❗

👉1. Нам нічого не винні там, де нас приймають. Винні тільки ми, зазирнувши в
booking і дізнавшись ціни на свій номер в готелі або апартаменти, де оселилися.

👉2. Ми не жертви у чужій країні. Ми постраждали у своїй. А тут люди щедрі та
гостинні, і ми у ГОСТЯХ.

👉3. Не розраховуйте, що вас хтось  годуватиме. Майте свій план рестарту свого
життя.

👉4. Ми – українці, а не орда, тому залиште по собі будь-яке місце кращим, ніж
було до вас. І навчіть цьому дітей – залишити \enquote{гарний слід} на згадку.

👉5. Будьте вдячні за все. Найкращий час показати дітям, що українці – це
вдячні люди.

Пам'ятайте, ніхто не дізнається, що ви ввічливі та вдячні за все, поки не
почнете про це говорити.

👉6. Не смикайте господарів без приводу. Гугл в допомогу. Звертайтеся з
питаннями до тих, хто допомагає, лише коли йдеться про специфічні особливості
даної місцевості.

👉7. Ви − прекрасні люди. Не забувайте це, і покажіть найкращі сторони себе та
дітей. Будьте приємними у спілкуванні, безконфліктними та терплячими. Якщо у
вас запити й рівень життя вищий, ніж вам запропонували, попрацюйте і зробіть
там \enquote{палац}.

👉8. Будьте корисні будь-де. Ваші діти запам'ятають не ваші слова, а ваші
справи за критичних обставин.

👉9. Вивчайте мову тієї країни, куди приїхали. Хоча б із поваги до людей, хто
нас прийняв, вивчіть хоч 10-20 побутових фраз.

👉10. Більшість повернеться до своєї країни та рідного міста. Зробіть усе, щоб
після вас сказали, що українці – це найкращі люди. І надбайте за цей час для
себе нових \enquote{рідних та близьких} людей. 

І пам'ятаємо, що нині всім складно, але головне - залишитись Людьми🙏🇺🇦

© \href{\urlDemidkoIA}{Olga Demidko}, мешканка Маріуполя, зараз живе в м. Прага, Чехія

\ii{insert.author.demidko_olga}
