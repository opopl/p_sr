% vim: keymap=russian-jcukenwin
%%beginhead 
 
%%file 03_12_2020.fb.patrioty_ukrainy.1.slobodjanjuk_ruslan_mykolajovych_35
%%parent 03_12_2020
 
%%url https://www.facebook.com/groups/741289219573471/?notif_id=1607337432116411&notif_t=group_r2j_approved&ref=notif
 
%%author 
%%author_id 
%%author_url 
 
%%tags 
%%title 
 
%%endhead 

\subsection{НАЗАВЖДИ 35…}
\Purl{https://www.facebook.com/groups/741289219573471/?notif_id=1607337432116411&notif_t=group_r2j_approved&ref=notif}

Сьогодні, 2 грудня, 41 рік від дня свого народження мав би відзначати СЛОБОДЯНЮК Руслан Миколайович.

\index[names.rus]{Слободянюк, Руслан Миколайович!Герої АТО!1979-12.04.2015}

Народився в 1979 році у м. Кіровоград на Кропивниччині, громадянин України.
Захищав нашу країну в складі 72-ї окремої механізованої бригади ЗС України у
званні солдата. Загинув 12 квітня 2015 р. в зоні проведення АТО в Донецькій
області. Похований на алеї почесних поховань Рівнянського кладовища.

\ifcmt
pic https://scontent.fiev6-1.fna.fbcdn.net/v/t1.0-9/128387421_389122792426944_6298925370526134171_n.jpg?_nc_cat=107&ccb=2&_nc_sid=825194&_nc_ohc=S5XstcVQbxEAX_Y6hjh&_nc_ht=scontent.fiev6-1.fna&oh=df21643b192e070d809a029d183f340f&oe=5FF28BAD
caption СЛОБОДЯНЮК Руслан Миколайович
\fi

Слободянюк Руслан Миколайович - стрілець 7-ї механізованої роти 3-го
механізованого батальйону 72-ї окремої механізованої бригади Сухопутних військ
Збройних Сил України, солдат.

Народився 2 грудня 1979 року в місті Кіровоград (нині - Кропивницький).
Закінчив 9 класів загальноосвітньої школи №18 міста Кіровограда у 1995 році та
професійно-технічне училище №8 міста Кіровограда.

Проходив строкову військову службу в лавах Збройних Сил України.

Після демобілізації повернувся у місто Кіровоград. Працював на місцевих
підприємствах, зокрема на підприємстві «Креатив».

У лютому 2015 року мобілізований до лав Збройних Сил України. Пройшов військову
підготовку у 169-му навчальному центрі Сухопутних військ Збройних Сил України
(військова частина А0665, селище міського типу Десна Козелецького району
Чернігівської області). Служив стрільцем 7-ї механізованої роти 3-го
механізованого батальйону 72-ї окремої механізованої бригади Сухопутних військ
Збройних Сил України (військова частина А2167, місто Біла Церква Київської
області).

З весни 2015 року брав участь в антитерористичній операції на сході України.

12 квітня 2015 року солдат Слободянюк під час виконання бойового завдання
підірвався на розтяжці у районі села Старогнатівка Волноваського району
Донецької області.

18 квітня 2015 року похований на Алеї Слави Ровенського кладовища міста
Кіровоград (нині - Кропивницький).

Нагороджений орденом «За мужність» III ступеня (16.01.2016- посмертно).

21 серпня 2015 року в місті Кіровоград (нині - Кропивницький) на будівлі
загальноосвітньої школи №18 (вулиця Юрія Коваленка, 9а), де навчався Руслан
Слободянюк, йому відкрито меморіальну дошку

Патріоти України!
