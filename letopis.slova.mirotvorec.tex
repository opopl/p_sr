% vim: keymap=russian-jcukenwin
%%beginhead 
 
%%file slova.mirotvorec
%%parent slova
 
%%url 
 
%%author 
%%author_id 
%%author_url 
 
%%tags 
%%title 
 
%%endhead 
\chapter{Миротворец}
\label{sec:slova.mirotvorec}

%%%cit
%%%cit_head
%%%cit_pic
%%%cit_text
Глава Венгерского МИД поднял в ПАСЕ вопрос о существовании в Украине печально
известного сайта \emph{«Миротворец»}.  Так, в частности, министр осудил этот
позорный ресурс, назвав его «списком врагов, которые должны быть наказаны». И
ведь дело говорит иностранец! По европейским меркам, такие списки -
средневековая жуть.  Не так давно мониторинговая комиссия ООН по правам
человека призывала Верховную Раду инициировать закрытие \emph{сайта
«Миротворец»}. Там, дескать, размещают персональные данные людей, создавая
угрозу их безопасности.  Примерно того же от украинского руководства добивался
МИД Германии. Аккурат после того, как в базу \emph{«Миротворец»} попал
экс-канцлер ФРГ Герхард Шрёдер.  Даже украинский омбудсмен и замминистра
информационной политики Украины говорили об этой проблеме, но воз и ныне там.
\emph{«Миротворец»} неприкосновенен
%%%cit_comment
%%%cit_title
\citTitle{Глава Венгерского МИД поднял в ПАСЕ вопрос о существовании в Украине сайта Миротворец}, 
Максим Могильницкий, strana.ua, 23.06.2021
%%%endcit

