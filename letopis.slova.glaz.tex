% vim: keymap=russian-jcukenwin
%%beginhead 
 
%%file slova.glaz
%%parent slova
 
%%url 
 
%%author 
%%author_id 
%%author_url 
 
%%tags 
%%title 
 
%%endhead 
\chapter{Глаз}
\label{sec:slova.glaz}

%%%cit
%%%cit_head
%%%cit_pic
%%%cit_text
К Кабмину у омбудсмена претензий оказалось гораздо меньше, чем к депутатам, но
его \emph{зоркий глаз} и острый слух всё же сумел углядеть и услышать, как некоторые
представители правоохранительных органов Украины посмели на телевизионных
ток-шоу изъясняться на русском.  Добрался украинский шпрехен-фюрер и до
местного самоуправления, где тоже нашел вопиющие факты общения на русском. В
городских и местных советах массово звучит русский язык и происходит это не в
районах Донбасса, подконтрольных Украине, а в самых что ни есть западных
районах страны. Отдельные претензии пофамильно омбудсмен предъявил мэрам
Мариуполя, Днепра и Одессы
%%%cit_comment
%%%cit_title
\citTitle{Speak на мове please}, 
Terra Incognita, zen.yandex.ru, 30.04.2021
%%%endcit

%%%cit
%%%cit_head
%%%cit_pic
%%%cit_text
Посовещались меж собой и решили - \emph{ОЧИ} назовём глазами. Люди перестанут видеть
\emph{ОЧЕвидное}. Буду смотреть, но - не будут видеть. Вежды - веками. Закроем людям
\emph{ОЧИ} веждами и - мир ослепнет. И вот уж мир блукает в потёмкам своего
невежества. Слепой - НЕВЕЖДА - не вижу, живу наощупь.  Лобное место (самое
высокое место называлось ЛОБНЫМ) переместим пониже и пусть это будет лоб, где
по - настоящему - ЧЕЛО. Всё. Образ исказился.  (Мы -то - природные. Нам нужно
всё представить). Вот и вышло такое СВЕТОПРЕДСТАВЛЕНИЕ. Казусное. Липовое.
Кривда 
%%%cit_comment
%%%cit_title
\citTitle{Чтобы мир извратить, нужно просто исказить исКОНную Суть Слова}, 
Ирина, zen.yandex.ru, 22.06.2021
%%%endcit

%%%cit
%%%cit_head
%%%cit_pic
%%%cit_text
Мирослава вхопилась за груди і відвернула \emph{очі}. Тухольські пасемці на сплетені з
галуззя мари положили трупа, а за ним потягли і медведя. Понура мовчанка
залягла над товариством. Велика калюжа крові блискотіла до сонця і нагадувала
всім, що тут іще перед хвилею стояв живий чоловік, батько дітям, веселий,
охочий і повний надії, а тепер з нього лишилася лише безформна купа кровавого
м'яса. У великої часті бояр відійшла охота до ловів.  — Цур їм, тим проклятим
медведям! — говорили деякі.— Нехай тут хоч жиють, хоч гинуть собі, чи ж нам для
них наражувати своє життя?
%%%cit_comment
%%%cit_title
\citTitle{Захар Беркут}, Іван Франко
%%%endcit

%%%cit
%%%cit_head
%%%cit_pic
%%%cit_text
Хол вел их большей частью по вершинам холмистых возвышенностей, но иногда
приходилось переходить от одного хребта до другого, чтобы не попасть \emph{на
глаза} дровосеку или фермеру. Опасности не было, наоборот, их скорее ждало бы
радушие и гостеприимство, но все же они решили, пока возможно, никому не
показываться.  Известие о такой пестрой компании разнеслось бы быстро, а это
уже было бы опасно.  Спустившись с холмов в глубокие долины, они попали в
другой мир. Здесь деревья росли тесно, были больше, на крутых склонах
проступали каменистые обнажения, в руслах быстрых ручьев лежали массивные
булыжники. Высоко над головой на вершинах холмов ветер шумел листвой, но внизу
ветра не было. В тишине густого леса оглушительно звучал шорох обеспокоенной
ими белки или взрыв крыльев вспугнутой куропатки, которая как призрак
стремительно поднималась в воздух
%%%cit_comment
%%%cit_title
\citTitle{Зачарованное паломничество}, Клиффорд Саймак
%%%endcit
