% vim: keymap=russian-jcukenwin
%%beginhead 
 
%%file articles.ternovii_vinec_ukrainy
%%parent body
%%url http://www.berdnyk.com.ua/statti1.html
 
%%endhead 

\subsection{Терновий вінець України: Росії минулій, сучасній, майбутній – відкрите послання}

Олесь Бердник. «Терновий вінець України (Росії минулій, сучасній, грядущій —
відкрите послання)» // ж. «Сучасність» (Мюнхен), 1985, №10 – с.7-12.

\url{https://fantlab.ru/work270904}
\url{http://www.berdnyk.com.ua/statti1.html}

``Росіє!

...325 років тому у Переяславі ми дали одне одному лицарське слово на
побратимство, на єдність, на вірність. Кожен з нас вщ того тривожного дня
посіяв безліч різних зерен у землю кількох поколінь.

Вже четверте століття, Росіє, ми йдемо спільним ланом, пожинаючи врожай,
заповіданий прадідами, який же то урожай?

На сцені соціяльного базару буде дана однозначна відповідь: Україна і Росія —
рівні серед рівних, дві великі сестри, зростають. квітнуть і йдуть до сяючих
вершин. Україна дає стільки-то сталі, чавуну та вугілля, хліба і сала, випускає
ось таку лявину книг, має стільки-то мільйонів студентів, учених та героїв
праці!

Проте ці плякатні визначники, Росіє, втомили дух України. Вона відкидає їх з
огидою і показує на своє чоло.

Глянь на моє чоло, Росіє,— ти побачиш терновий вінець? Так, результат
багатовікової спілки наших народів — Голгота України — окраденої, замученої,
оббріханої, розп'ятої.

Не поспішай лютувати, Росіє! Задумайся і згадай минуле: твоє духовне падіння і
наша ганьба почались в той самий день, коли ми, не розпізнавши гадючого духу
московських тиранів, відкрили Золоті Ворота України до тебе, Росіє, і що
приплинуло до мене? Все відбувалось на історичному полі, нічого не можна
приховати.

Ти отримала наші багатющі землі, Росіє, а на додачу — трудящі руки і мистецькі
душі, рівних яким мало у світі. Ти безжально пожирала наші багатства і
безсоромно смоктала творчий геній України, привласнюючи собі пріоритет і славу.
Ти прикрила убогість і нікчемність своїх царів і опричників нашою піснею, нашою
науковою думкою, звитягою наших лицарів. А натомість?..

Ти зруйнувала колиску свободи — Січ Запорізьку, дивовижний витвір еволюції,
який міг би на кілька століть наблизити епоху волі і народовладдя. Ти
привласнила собі все, що було пов'язане з історією запорізьких лицарів духу —
клейноди, архіви, легенди, пісні. Ти наклала вето на саму пам'ять про них, бо
жахаєшся їхнього воскресіння у духосфері сучасности.

В цьому діялозі не варто перераховувати всі факти та імена, архіви твоїх
жандармських катакомб мають все, щоб освіжити пам'ять. Тому згадаю лише
основне.

До спілки з тобою український нарід виборював свою суверенність, свою волю і в
тій борні гартував душу, творив пісню і високу мисль космічного всеохоплення.
Ми не грабували чужих багатств і не захоплювали чужих земель. Жадаючи волі, ми
шанували волю сусідів. Ми не мурували в'язниць, не городили кордонів, не
творили кріпаків з вільних громадян. І коли якийсь український старшина ставав
вельможею і захоплювався февдальними привілеями шляхти, то він сам одривався
від духу матері України, стаючи її ворогом і прислужником опресивних сусідів.

Наша щирість веліла бачити у сусідів такий же духовий стрій, як і в собі. Це
була жахлива помилка.

Закон з'єднаних посудин ілюструє історичну ситуацію нашого єднання. Духовий та
економічний вакуум Московщини нездоланно смикнув до себе все, чим була багата і
славна Україна. А щоб волелюбне серце народу не могло дати відсіч зрадникам і
дворушникам — необхідно було знищити віковий корінь сили і вільности — інститут
козацтва та кобзарства — ці два крила українського генія.

О, як ти безжально, Росіє, нищила, обкарнувала, висмикувала пір'їни з тих
райдужних крил!

Крило козацьке ти одсікла одразу — і найжорстокіше! А розсіяні залишки лицарів
розвіяла у безвість: непокірними загатила фінські та сибірські болота, а
покірнішими — обставила свої кавказькі кордони та почала завойовувати апетитні
східні території.

Крило кобзарське відсікти було важче: ампутація розтяглася на кілька віків, тим
більше, що джерело творчости пливло з бездонної криниці серця народнього.
Проте, твій чаклунський дух, Росіє, знав, що то — основне завдання, бо доки
вимахує в повітрі кобзарське крило — може бути регенероване, відроджене
одсічене козацьке крило.

Ти почала випивати, висушувати творчу криницю України. Як гусінь, оповили нашу
землю Петрові і Катеринині ублюдки та свої перевертні. Колись найосвіченіший в
Европі нарід опустився в найнижчі круги інферно: пекельні слуги не знали, як ще
дошкульніше ударити по серцю України, щоб померла нарешті райдуга на творчому
крилі.

В ту критичну добу лише явище Шевченка врятувало Україну від деградації і
воскресило райдужне крило генія рідного народу. То був дивовижний порив джерела
творчости із надрів Духосфери.

Лютість ворожого духу Росії була вражаюча. Проте вже пізно було щось робити:
заборона лише роздмухувала вогнище відродження. Гадючий чаклунський дух обрав
інший шлях — шлях визнання і введення в свою програму. Пророк, з яким
обнімається ворог, втрачає більшість своїх революційних сил.

Могутні соціяльні потрясення XX віку не принесли Україні воскресіння: всі
творчі сили були втягнуті у хитру гру політичних шахраїв: деякі з нас лягли на
полі бою, деякі впали з роздробленими черепами у підвалах ЧК та гестапо, деякі
загинули в сибірських таборах, деякі обрали шлях Переяслава — шлях ганьби і
прислужництва.

Досвід трьохсотлітнього прислужництва та приниження показав: одна помилка, одне
хитання дає ланцюгову реакцію помилок та падінь у грядущому. Те, що зроблено
сьогодні — неможливо переробити завтра. Чорні зерна зради і страху виростають в
чортополох занепаду, з якого годі шукати виходу.

Висновок такий: кожен нарід повинен сам вирішувати свою долю, не дозволяючи
іншим народам перебирати на себе керівну ініціятиву.

Краще вмерти героєм, ніж животіти кріпаками!

Ми мали такий лицарський заповіт і — зневажили його! Століття рабства —
розплата за зраду духової волі!

Але ти, Росіє, не радій, не танцюй над купою обсмиканого райдужного пір'я! Ті
обрубані крила не приросли органічно до твого зміїного тулуба. Ти обтяжила себе
злочинами, падіннями, зрадами, від яких тебе не очистять всі твої святі і
подвижники. Ти стала гігантською в'язницею народів, і не захотіла зруйнувати ту
в'язницю після Жовтневої Революції, а ще сильніше стиснула в правиці скіпетр
жорстокости.

Той скіпетр ти ще, ще і ще невблаганно опускала на голову причинної України —
божевільної Діви, котра так необачно наділа на свій палець обручальне кільце.

Хто зміряє океан муки, в якому пливе Україна? Хто опише страждання мільйонів
померлих від штучного голоду в 33 році? Хто зможе розповісти про ридання тих,
хто був безневинно розстріляний у 37-39 роках? Хто обніме духовним оком
незміряний світ приниження, безправности, деградації, ув'язнень, безвісних
смертей, голодувань, втрати ідеалів,— той світ, який став історичним фантомом
України, її прокляттям, її невпинною реальністю?!

Тисячі вбитих поетів, мистців, мислителів... Ще тисячі і тисячі втомлених,
підкуплених, застрашених!

В той час, як всі народи землі прагнуть до волі і знаходять шляхи до неї, ти,
Росіє, накинула аркан на шию України і безжально душиш її, щоб вибити з неї
пам'ять про славне минуле, про космічне покликання її буття!

І навіть тепер, коли син України Корольов відкрив для тебе Браму Космосу,
навіть в такий духовний час ти не хочеш зм'якшити своє жорстоке серце, Росіє!
Всі кращі сини українського духу знов в неволі, на засланні, під невсипущим
оком жандармів! Кожен, хто сказав тобі слово Правди,— відчув удар твого
скіпетра жорстокости. Незламний Мороз — за що ти караєш його? Хіба за героїчний
захист української культури?! Ніжний Сверстюк, який творчим духом оглядав
риштування українського собору душі,— невже ти так жахаєшся його? Правдивий
Лісовий, який добровільно пішов на ешафот, щоб сказати тобі слово
перестороги...

А за що ти опустила меч кари на Миколу Руденка — космічного поета, устами якого
Бог дав тобі слово відродження і грядущого знання;

за що ти так тяжко скривдила його і його побратимів, котрі прийняли на себе
місію захисту поневолених — Лук'яненка, Тихого, Матусевича, Мариновича, Вінса?

Світличний, Чорновіл, Стус, Калинець, Стасів, Шабатура — десятки героїчних
чоловіків і жінок України, вся вина яких лише в тому, що вони правдиво мислять
і діють,— що вони вчинили тобі, Росіє?!

Ми не станемо рахувати втрати! Дух України знову і знову народить нас для
страшного двобою із законом Неволі і Жорстокости! Але ти, Росіє, повинна в цей
грізний історичний час визначити свою стежку і свій духовний статус! Пам'ятай —
ось для тебе настав час вирішення: воскреснути серед вільного кола народів або
впасти в запустіння і забуття!

Нехтувати цим попередженням не слід — ні сила, ні страх інших народів не
врятують тебе від тієї долі, яка чигала на всіх насильників,— від долі повного
знищення!

... Кров людська — не вода! Земля не приймає її, і кожна краплина волає до
неба. Хай вона впаде на тебе, Росіє Петра і Катерини, Сталіна і Берії, Росіє
сучасних безликих! Хай спопелить вона гадючу шкіру Дракона, щоб спляча красуня
могла прокинутись до життя Духу!

Я свідчу на Суді Божому ім'ям України — свідчу за зраджених і замучених
козаків, за знедолених кріпаків, за ганьблену пісню, за принижену думку, за
численні покоління, котрі вмирали на чужому полі, не виконавши народньої місії,
за мільйони померлих від голоду, за тисячі розстріляних сучасників моїх, за
мільйони забутих, страждаючих від суму, відчаю та безвиході!

Дай відповідь Богові, Росіє, і прийми альтернативу:

Лише повна воля для скованих тобою народів звільнить тебе! Лише Епоха Духовних
Республік, Свят Народів, вільних від політично-економічних кайданів чужих
ідеологій відкриє для нас життя суверенности і обопільної дружби. Україна не
бажає більш тягти чужу колісницю до термоядерної прірви, до повної творчої
деградації!

Я виходжу до тебе у пустельне вранішнє поле, двоголовий орле Росії! Я
сам-на-сам викликаю тебе на ґерць. як це велось в казкові часи! Діва-Україна
благословила мене на подвиг і сказала на прощання — бийся без щита!

І я стою супроти тебе, древній Драконе, з відкритими грудьми—але безстрашно!
Йди сюди — з своїми в'язницями, бюрократичними бандами, царями, вождями,
стукачами, провокаторами! Ти не зможеш перемогти мене, бо я — Безсмертний Дух
України!...''.
