% vim: keymap=russian-jcukenwin
%%beginhead 
 
%%file 09_09_2017.stz.news.ua.mrpl_city.1.istoria_pervyj_den_goroda
%%parent 09_09_2017
 
%%url https://mrpl.city/blogs/view/istoriya-pervyj-den-goroda
 
%%author_id burov_sergij.mariupol,news.ua.mrpl_city
%%date 
 
%%tags 
%%title История: Первый День города
 
%%endhead 
 
\subsection{История: Первый День города}
\label{sec:09_09_2017.stz.news.ua.mrpl_city.1.istoria_pervyj_den_goroda}
 
\Purl{https://mrpl.city/blogs/view/istoriya-pervyj-den-goroda}
\ifcmt
 author_begin
   author_id burov_sergij.mariupol,news.ua.mrpl_city
 author_end
\fi

\ii{09_09_2017.stz.news.ua.mrpl_city.1.istoria_pervyj_den_goroda.pic.1}

Впервые мариупольцы отмечали День города - своеобразные именины Мариуполя  – 13
сентября 1987 года. Операторы народной любительской киностудии \enquote{Пламя} Дворца
культуры металлургов комбината имени Ильича Юрий Цымбалюк, Николай Чухлиб и
Анатолий Новиков  запечатлели на цветной кинопленке фрагменты этого нового для
нашего города праздника, смонтировали их. К делу авторы отнеслись очень
серьезно: отснятый материал был проявлен не где-нибудь, а на Киевской студии
научно-популярных фильмов, в то время славящейся новейшим технологическим
оборудованием.

\ii{09_09_2017.stz.news.ua.mrpl_city.1.istoria_pervyj_den_goroda.pic.2}

Фильм получился яркий, мажорный и, конечно, \enquote{идейно выдержанный}. Нужно не
забывать, что в 1987 году еще был Советский Союз, а руководящей и направляющей
силой -  КПСС. Название фильму дали без затей – \enquote{День города}. Он
демонстрировался в местной киносети в качестве киножурнала к художественным
фильмам и имел довольно приличный успех у зрителей, особенно тех, кто заметил
себя или своих близких среди празднующего народа.

\ii{09_09_2017.stz.news.ua.mrpl_city.1.istoria_pervyj_den_goroda.pic.3}

Копия этого произведения кинолюбителей дошла до наших дней благодаря двум нашим
землякам: Вилию Котенко и бывшему директору кинотеатра \enquote{Савона} Жанне Данченко.
Вилий Михайлович, будучи директором Конторы кинопроката, собирал все фильмы: и
художественные, и документальные, - которые имели хотя бы отдаленное отношение
к нашему городу. Он же  организовал хранение работ местных кинолюбителей. Он
выписал из Госфильмофонда СССР копии рабочего материала кинохроники, снятой в
Мариуполе в довоенные годы, и уникальный киносюжет о первых днях освобождения
города в сентябре 1943 года. Все это богатство содержалось в идеальном порядке.
Заметим, что перечисленное выше вовсе не входило в его обязанности. Так,
веление сердца и любовь к родному городу.

\ii{09_09_2017.stz.news.ua.mrpl_city.1.istoria_pervyj_den_goroda.pic.4}

Но в середине девяностых годов прошлого столетия отлаженная как часы система
союзного кинопроката развалилась, нашлись шустрые люди, которые по-своему
распорядились одним из лучших в Украине предприятием, созданным усилиями В.М.
Котенко, который к тому времени, к сожалению, преждевременно ушел из жизни.
Неизвестно куда делось оборудование, остался без хозяина богатейший фильмофонд.
И тогда-то и пригодилась деловая хватка и инициативность Жанны Игоревны. Она
приняла на хранение все фильмы. Их перевезли в руководимое ею зрелищное
учреждение. И надо сказать, что фильмокопии не лежали мертвым грузом,
периодически их использовали для ретропросмотров, а хроника широко используется
всеми тремя городскими телекомпаниями, к сожалению, как правило, не
упоминается, откуда они взяты.

Но вернемся к фильму, снятому 13 сентября 1987 года. Некоторое представление о
нем может дать дикторский текст к нему.

"Свидетелем многих событий была эта старая башня. Сегодня у нее особый день. В
памятном сентябре сорок третьего года войска Южного фронта  изгнали
гитлеровцев из Мариуполя. Крупный промышленный и портовый центр Юга страны
вновь обрел свободу. Теперь навсегда. И не случайно для празднования Дня города
выбрали первый осенний месяц. Готовились заранее. Об \enquote{именинах} города писали
местные газеты, развернулось социалистическое соревнование в его честь, были
изготовлены эмблемы, установлены панно, оформлены районы города. И когда утром
13 сентября 1987 года площади и улицы расцветились яркими флагами, загремела
медь оркестров, горожан это не удивило. Все знали  - праздник города начался.
Его  центр образовался у здания городского театра. Сюда принесли знамя города,
здесь прозвучали торжественные речи, вру­чены награды победителям трудовых
вахт.

Тут же встречали дорогих гостей - участ­ников освобождения Мариуполя от
фашистской нечисти. У памятных монументов были возложены цветы. Четко печатая
шаг, восхищая зрителей выправкой, прошли шеренги воинов. Тысячи подписей
поставлено под обращением в защиту мира, против ядерной опасности. В
мартеновском цехе комбината имени Ильича перед плавкой мира начался митинг. А
праздник все ширился. Пионеры встретились с прославленным сталеваром - дважды
Героем Социалистического Труда Григорием Яковлевичем Горбанем.

Художники в сквере у театра установили мольберты, поэты читали лучшие свои
стихи. Хотелось сказать, что стар и млад вышли на улицы. Но наш город рабочий,
и в дни торжеств должны работать домны и мартены, без остановки катать металл
прокатные станы, железнодорожники отгружать готовую продукцию, а моряки
выводить суда в рейсы. Для курсантов морской школы этот день особенно памятный
- он совпал с торжественным ри­туалом посвящения их в моряки.

Строится город, растут этажи, появляются новые адреса. Новый профилакторий
трудящихся \enquote{Ждановтяжмаша}. Новый Дворец пионеров, новый видеосалон. Его
открытие ждановские кинофикаторы  приурочили к общегородскому  праздни­ку.
Ярмарка. В разноцветных палатках бойкая торговля сладостями, от мангалов тянет
аппетитным дымком. В веселой кутерьме то и дело щелкают затворы  фотоаппаратов.
Где как не среди яркого гулянья сделать фотографию на память? Постарались
повара и официанты ресторана \enquote{Украина}. Огородили  площадку плетнем, развесили
на кольях глечики. Чем не шинок? Здесь можно отведать украинские блюда, по­пить
чаю из пыхтящего самовара. А шинкарки просто загляденье - все в лентах,
намистах, в вышитых сорочках. Не вместить в десятиминутный фильм все события
этого дня. Народная пляска сменялась концертом рок-ансамбля, лирическая песня -
бравурными звуками духового оркестра, по проспектам двигался веселый карнавал.
Праздник удался".

Что же увидели зрители? Прежде всего, панораму утреннего города и, конечно же,
старую водонапорную башню. Только после этого появилось название фильма. За ним
следовало море знамен союзных республик, группа ветеранов, школьницы с
гирляндой из хвои и цветов, смена пионерского караула у памятника Макару Мазаю,
кавалькада всадников в буденовках как воспоминание о гражданской войне,
гуляющая публика. И «джентльменский набор» парадного фильма тех времен. Искры
от разрезаемого металла, плавка в мартеновском цехе, состав с готовой
продукцией металлургического завода, моряки в рубке судна, комбайн среди
пшеничного поля, люди на стройке и т.п.

Как ни странно, в дикторском тексте прозвучала только одна фамилия – дважды
Героя Социалистического Труда  Григория Яковлевича Горбаня. А ведь в объективы
кинокамер попали и другие известные люди. Это Герои Социалистического Труда
сталевар Владимир Павлович  Клименко, строительница Мария Федосеевна Короткая,
генеральный директор комбината имени Ильича Николай Алексеевич  Гуров,
вручающий грамоты и передовикам производства, ильичевский сталевар Валентин
Лазаренко,  Вилий Михайлович Котенко на открытии видеосалона – новинке тех
времен. А еще поэт Павел Александрович Бессонов, читающий стихи на эстраде
Городского сада, уже тогда известный художник-медальер Ефим Викторович Харабет,
заместитель главного редактора \enquote{Приазовского рабочего} Валерий Сидорович
Комаров, который нацелил свой фотоаппарат на гуляющую  публику в Театральном
сквере. И лишь в одном кадре мелькнуло руководство города. Случайность это или
закономерность, сейчас трудно сказать.

В финале зрители увидели красавиц-украинок  в сорочках-вышиванках, с веночками
на головках, в намистах и караваями хлеба в руках. В фильме много панорам
города, снятых с верхних точек, запечатлевших вид нашего города конца
восьмидесятых годов XX века, что, конечно, увеличивает ценность работы
кинолюбителей. Да и в целом она представляет собой в известном смысле
исторический документ, тем более что сейчас уже имеются ее цифровые копии на
дисках, что может обеспечить сохранность на долгое время. В последующие годы
Дни города проводились и многолюднее, и ярче, и масштабнее, и с другим
смысловым наполнением, но этот – образца 1987 года – был первым, и хорошо, что
о нем осталась некая материальная память.
