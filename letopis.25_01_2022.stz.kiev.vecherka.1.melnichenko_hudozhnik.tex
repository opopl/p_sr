% vim: keymap=russian-jcukenwin
%%beginhead 
 
%%file 25_01_2022.stz.kiev.vecherka.1.melnichenko_hudozhnik
%%parent 25_01_2022
 
%%url https://vechirniy.kyiv.ua/news/60725
 
%%author_id katajeva_maria
%%date 
 
%%tags 90_let,hudozhnik,isskustvo,kiev,ukraina
%%title Знаковий київський художник та скульптор відзначає 90 років
 
%%endhead 
 
\subsection{Знаковий київський художник та скульптор відзначає 90 років}
\label{sec:25_01_2022.stz.kiev.vecherka.1.melnichenko_hudozhnik}
 
\Purl{https://vechirniy.kyiv.ua/news/60725}
\ifcmt
 author_begin
   author_id katajeva_maria
 author_end
\fi

\begin{zznagolos}
Сьогодні Володимиру Мельніченку виповнилося 90 років.
\end{zznagolos}

Митець народився 25 січня 1932 року в Києві. Протягом німецької окупації Києва
жив у дитячому інтернаті. У 1950 році закінчив Київську художню школу ім. Т.
Шевченка.

\ifcmt
  ig https://vechirniy.kyiv.ua/uploads/2022/01/25/123.jpg
	@caption Сьогодні Володимиру Мельніченку виповнилося 90 років. Фото: АРВМ
\fi

1951 року вступив до Київського державного художнього інституту, навчався в
майстерні на першому курсі М. Шаронова, Іржаковського, на третьому курсі у
Тетяни Яблонської, з четвертого — майстерня Карпа Трохименка.

У 1954 році Володимир Мельніченко разом з дружиною Адою Рибачук їде на
Баренцове море (Арктика) на переддипломну практику. Північ справила велике
враження на художників та значно вплинула на світогляд, творчий доробок і на
все їхнє подальше життя. На півночі митці створили багато робіт про життя,
філософію та цивілізацію ненців. Наразі це одні з небагатьох художні
антропологічні матеріали з історії цього народу. Деякі роботи, натхнені
Північчю, кияни мали змогу побачити у «Ягалереї» у 2019 році. 27 вересня 1959
року було засновано Перший художній музей в Арктиці. Роботи Ади Рибачук і
Володимира Мельніченка становили основу колекції музею, вони подарували 118
творів живопису і графіки.

Володимир Мельніченко працював у творчому тандемі з Адою Рибачук. Усі спільні
роботи — монументальні твори, архітектура, скульптура та фільми — підписували
абревіатурою АРВМ. Ада Рибачук та Володимир Мельніченко працювали з
архітектором Авраамом Мілецьким як творча група з 1960 до 1982 року. У спільних
роботах головна роль була за Адою Рибачук та Володимиром Мельніченко: елементах
творчості, дизайну, синтезу кольорової пластики та художньо-декоративних
образів з архітектурою.

Декілька їх реалізованих проєктів Володимира Мельніченка та Ади Рибачук можна
побачити у Києві. Так, у 2019 році відреставрували мозаїчний комплекс фонтанів
«Зорі та Сузір’я», що розташований перед Київським Палацом дітей та юнацтва. А
минулого року унікальні мозаїки відродили на Центральному автовокзалі.Також
митці 13 років працювали над Стіною Пам’яті на Байковому кладовищі. Її спіткала
сумна доля — радянська влада поховала рельєфи під шаром бетону, оголосивши їх
«чужими принципам соціалістичного реалізму». У 2021-му році нарешті почали
відкривати з-під бетону її фрагменти.

Сьогодні збереженням культурної спадщини художників Ади Рибачук та Володимира
Мельніченка опікується спільнота АРВМ. Для охочих періодично проводять
екскурсії в майстерню Володимира Мельніченка.

Марія КАТАЄВА, «Вечірній Київ»

