% vim: keymap=russian-jcukenwin
%%beginhead 
 
%%file 11_01_2022.yz.vestnik.1.russkie_idut
%%parent 11_01_2022
 
%%url https://zen.yandex.ru/media/vestnikviktor/russkie-idut-ili-ne-idut-pochemu-tak-proishodit-61dd2007cd29b04c1430da3f
 
%%author_id yz.vestnik
%%date 
 
%%tags russkie,rossia
%%title Русские идут или не идут, почему так происходит
 
%%endhead 
\subsection{Русские идут или не идут, почему так происходит}
\label{sec:11_01_2022.yz.vestnik.1.russkie_idut}

\Purl{https://zen.yandex.ru/media/vestnikviktor/russkie-idut-ili-ne-idut-pochemu-tak-proishodit-61dd2007cd29b04c1430da3f}
\ifcmt
 author_begin
   author_id yz.vestnik
 author_end
\fi

Здравия всем, друзья.

Во всех СМИ Европы и США постоянно говорят, что Русские идут и даже уточняют,
куда они идут, и я решил понять, почему так происходит, ведь я-то, как русский
человек прекрасно понимаю, что никуда мы не идем, более того, даже обидно от
этого бывает.

И вообще, во всем Мире много говорят о России, как об агрессоре, а мы, русские
никак не понимаем этого, ведь мы то точно знаем, что мы самый миролюбивый
народ, вон помогли Казахстану и радуемся, как дети. Но на Западе по какой-то
причине создают образ русского человека, человека сильного, бескомпромиссного,
бесстрашного, угрюмого и верят в этот образ, и страшно его боятся.

И ведь не зря боятся, этим образом Запад создает условия напряжения жизненных
процессов, которые влияют на русского человека положительно, стабилизирует его
энергетическое тело, поднимает Силу Духа.

\begin{zznagolos}
Сила Духа требует проявления и как сказано во всех русских сказках, Иван,
который сидит на печи, слазит с нее и идет туда, куда глаза глядят. А куда у
него глаза глядят спросонья? На Кудыкину гору, или туда, куда Макар телят не
гонял, или туда, не знай куда, чтобы найти то, не знаю, что.	
\end{zznagolos}

В такие моменты жизни пробуждения русского человека и он начинает смотреть куда
глаза глядят, даже если он никуда не идет и сидит у себя дома, все вокруг
ощущают, что идет, что русские вокруг. Эта особенность от того, что для
русского человека, что далеко, то близко, а что близко, то далеко, одним словом
все с ног на голову. Вот где она, та же Кудыкина гора? Не далеко и не близко,
на то она и Кудыкина.

И вот в такое время, когда русский человек просыпаются, хоть и сидит дома, все
равно идет. И обычно находятся те, кто пытаются остановить этот несуществующий
поход.

И каким образом они пытаются это сделать? Остановить образ, который сами же и
создали и чаще всего методом уничтожения источника этого придуманного страшного
образа Русского человека, не понимая, что они пытаются не русского человека
остановить, а свой страх перед созданным образом. Одним словом, метод один,
который всегда оборачивается войной.

\begin{zznagolos}
И вот тут-то они сталкиваются уже не с придуманным образом, а с реальным
Русским человеком, и эта новая реальность кардинально отличается от
придуманного образа страха, который сами же и создали, потому что поднимается
уже реальный человек и вот тут-то уже реально наступает – Русские идут.	
\end{zznagolos}

Так было, так есть и так будет.

Чем опасен для противника русский человек? Непредсказуемостью,
нестандартностью, неопределенностью, но это только они так считают, для нас это
нормальное осознание Мира, не логическое, а импульсивное в режиме синтеза и
анализа. Мы с детства живем в такой атмосфере, в полу хаотической, в которой
постоянно создаем свою, пусть маленькую реальность, и в ней сами пишем правила,
пусть и параллельно с общепринятыми.

Еще одно отличие русского человека от европейца. Европейцы всегда ходят от
пункта А до пункта В, а русский человек всегда идет к цели. Если цели нет, то
он ее создает, например, Победа, или хотя бы «туда, не знай куда», пусть
неопределенная, но цель.

Еще одно качество Русского человека, которого очень сильно боятся – это сама
справедливость, он болен справедливостью, а что это такое и как это, определяет
сам.

Например, русский человек не завоевывает никогда, ибо это несправедливо, он
может только освобождать. Опять же пример Казахстана, он пришел туда, чтобы
восстановить справедливость, а не завоевывать. И сейчас поднимается не для
того, чтобы кого-то завоевать, потому что это несправедливо, а для того, чтобы
помочь и установить справедливость всюду. Во всяком случае, он так считает и
переубедить его очень трудно.

\begin{zznagolos}
Не сердите русских, они долго запрягает, быстро едут, а едут они не от пункта и
до пункта, а до полной и безоговорочной Победы. Нет у него иной цели и быть не
может, нам не нужна земля чужая, хватает и своей, цель еще не определили, но
будет она только одна – вся Земля – Прекрасная, Изумительная и Таинственная,
освобожденная, а не завоеванная. И что радует, что освобождение началось и
очень быстро.	
\end{zznagolos}

Коллективный Запад решил, что победили Россию, более того, они даже нас в этом
почти убедили, но мы вспомнили, что это невозможно, удивились и встаем и не для
того, чтобы завоевать, чтобы освободить планету от несправедливости, и мы
сделаем это и без войны, ибо знаем как.

Мы прошли одну дату нашего предсказания, импульс прекрасный, год Великих
Перемен начался. Нам Любить и нам Жить, нам Творить Жизнь Прекрасную, АУМ!

НАБЛЮДАЕМ, друзья – 22. 02. 2022 Всем желаю веры и крепости Духа.

Приглашаю на сайт «Истина времени». Читать так, открываем любую тему, любой
текст – это ответ. Или в отдел Книги, можно отдохнуть при чтении или чему-либо
научиться.

Ставьте лайки, подписывайтесь

До скорой встречи.
