% vim: keymap=russian-jcukenwin
%%beginhead 
 
%%file 02_10_2021.fb.gorovyj_ruslan.1.poezd_kiev_konstantinovka_barvinok.cmt
%%parent 02_10_2021.fb.gorovyj_ruslan.1.poezd_kiev_konstantinovka_barvinok
 
%%url 
 
%%author_id 
%%date 
 
%%tags 
%%title 
 
%%endhead 
\subsubsection{Коментарі}

\begin{itemize} % {
\iusr{Наталья Мельничук}
Так ось чому у них барвінок на могилах садять !

\begin{itemize} % {
\iusr{Евгения Макарова}
\textbf{Наталья Мельничук} у нас теж садять

\iusr{оксана ткач}
Все просто, барвінок застилає землю і не дає бур'янам рости.
\end{itemize} % }

\iusr{Алексей Меркулов}
Говорять чомусь "БарвЄнково", має бути БарвінкОве.

\begin{itemize} % {
\iusr{Volodymyr Lysechko}
\textbf{Alexey Merkulov} СлАвянск))

\iusr{Алексей Меркулов}
\textbf{Volodymyr Lysechko} місцеві так говорять)) немісцеві паляться на СлавЯнск

\iusr{Олена Семибратова}
Пєски з тої ж серії мабуть.

\iusr{Ruslan Gorovyi}
\textbf{Олена Семибратова} кримскоє, блд

\iusr{Олена Семибратова}
\textbf{Ruslan Gorovyi} Ага. Та та ж сама Канстантіновка.

\iusr{Volodymyr Lysechko}
\textbf{Alexey Merkulov} Місцеві трошки неграмотні))

\iusr{Тетяна Бурлак}
\textbf{Alexey Merkulov} чєлюсть інша, не вимовляє

\iusr{Tatyana Misiyuk}
\textbf{Олена Семибратова} ще «СнЕжное»- палево для немісцевих

\iusr{Anastasia Yevglevska}
Місцеві кажуть ще «Барвєновка» і «Краматоровка»

\end{itemize} % }

\iusr{Игорь Крамар}
Руськоязична щелепа, буває. Тест паляниці не пройде ))

\begin{itemize} % {
\iusr{Марина Ніколаєнко}
\textbf{Garry Kramar} ще слово -черевики- погано вимовляють.

\iusr{Orlando Glum}
\textbf{Garry Kramar} щелепи легко вправляються лівим та правим хуком...
\end{itemize} % }

\iusr{Елена Кабачок}
ДЕ ви без кінця таке находите?

\begin{itemize} % {
\iusr{Ruslan Gorovyi}
\textbf{Елена Кабачок} я просто спілкуюся, а не мовчу... говорун

\iusr{Михаил Кузнецов}
\textbf{Ruslan Gorovyi} Це як у Андрія Данилка спитали, де він знайшов образ Вєрки Сердючки. А він каже: "Просто я живу поруч із Володимирським ринком".

\iusr{Владимир Пучков}
\textbf{Ruslan Gorovyi}  @igg{fbicon.wink} 
\href{https://youtu.be/A_elSrJbi1M}{%
Птица говорун отличается умом. Отличается сообразительностью Тайна 3 планеты, ВСТАВОЧКА, youtube, 19.03.2021%
}

\end{itemize} % }

\iusr{Александр Воробьев}
Наступна станція "Могильна"

\begin{itemize} % {
\iusr{Михаил Кузнецов}
\textbf{Александр Воробьев} У Ставищанському районі є село Гостра Могила.

\iusr{Tatyana Misiyuk}
\textbf{Михаил Кузнецов} ще Савур Могила ж є
\end{itemize} % }

\iusr{Людмила Лещук}
Маґільнікаває

\iusr{Тамара Свеженцева}
Бляць. Коли чую таке Барвєнкаває, хочеться перегризти горло, чесно

\begin{itemize} % {
\iusr{Ruslan Gorovyi}
\textbf{Tamara Svezhentseva} розумію

\iusr{Павло Редзель}
\textbf{Tamara Svezhentseva} кофе, пасіба, сімчасов,скікичасов...повбивав би.

\iusr{Taras Chkan}
\textbf{Tamara Svezhentseva} ПЄскі  @igg{fbicon.smile} 

\iusr{Тамара Свеженцева}
\textbf{Taras Chkan} , так. Матрьошкоголові безтолочі перевели на свій лад.

\iusr{Михаил Кузнецов}
\textbf{Tamara Svezhentseva} Я вас розумію. Я так спииймаю, коли кажуть: "Станція мєтро ЛЄБЄДСКАЯ".
А одного разу прочитав на ОЛХ "мєтро лєбєдскоє"...

\iusr{Taras Chkan}
\textbf{Михайло Кузнецов} а ще ж є, бляць, СхОдніца... Повбивав би
\end{itemize} % }

\iusr{Василь Лапа}
"Барвінок" був і російською також.

\begin{itemize} % {
\iusr{Ruslan Gorovyi}
\textbf{Vasil Lapa} був... та бач?

\iusr{Біллі Рубін}
\textbf{Василь Лапа} а разве есть в русском слово барви?

\iusr{Василь Лапа}
Біллі, ти бот, скоріш за все, з Ольгіно, тому ти йдеш.... ну тьі понял.

Для українців, які, можливо, не зовсім мене зрозуміли, дописую більш розлого.

В Україні за часів совку був щомісячний журнал "Малятко". Для маленьких дітей.

Та "Барвінок" для більш дорослих. Обидва мені виписувала моя російськомовна
мама (ну, по черзі, за моїм віком).

І одного разу я отримав "Барвінок" російською.  @igg{fbicon.face.screaming.in.fear} 

Вікіпедія

"Перший номер «Барвінку» відкривався однойменним віршем Н.Забіли — як
своєрідним поетичним маніфестом видання. У 1950–98 рр. дублювався
російськомовним випуском.

\iusr{Василь Лапа}
Приблизно так

\ifcmt
  ig https://scontent-mia3-2.xx.fbcdn.net/v/t1.6435-9/244355702_2908208756096474_1904674187267051618_n.jpg?_nc_cat=109&_nc_rgb565=1&ccb=1-5&_nc_sid=dbeb18&_nc_ohc=0rYSiYfdjiQAX-Lb-pr&_nc_ht=scontent-mia3-2.xx&oh=d39139cf51914424942e293763769eeb&oe=618227B1
  @width 0.4
\fi

\iusr{Тетяна Сочка}
\textbf{Vasil Lapa}, і це геть не означа, що у московському нарєчії було слово "барвинок"))) Вони втуляться навіть в неіснуюче, а ви вірите))

\iusr{Василь Лапа}
Я десь стверджував протилежне?

\end{itemize} % }

\iusr{Сергій Гривняк}
Клас!!!!

\iusr{Olena Shevchenko}

насправді це є класичний приклад примусової русифікації за часів совка. бо
місто і район справді називаються - БарвЕнково, БарвЕнківський район

\begin{itemize} % {
\iusr{Ruslan Gorovyi}
\textbf{Olena Shevchenko} жесть

\iusr{V.M. Varga}
Неправда, українською — Барвінкове
\url{https://barvinkove-miskrada.gov.ua/}

\iusr{Olena Shevchenko}
вибачте, не знала. просто живу поряд з цим районом і називають його саме БарвЕнкове. була впевнена що так і пишеться

\iusr{Анна Саутина}
\textbf{Olena Shevchenko} це дифтонг

\iusr{V.M. Varga}
І району такого вже нема

\end{itemize} % }

\iusr{V.M. Varga}
Питання до Укрзалізниці

\ifcmt
  ig https://scontent-mia3-1.xx.fbcdn.net/v/t1.6435-9/244409028_940723389812773_3260023710971660508_n.jpg?_nc_cat=101&_nc_rgb565=1&ccb=1-5&_nc_sid=dbeb18&_nc_ohc=rWUgF02bAuYAX9jr1bm&_nc_ht=scontent-mia3-1.xx&oh=ee7f15ccc003e381d504653238c28f9b&oe=61839C33
  @width 0.4
\fi

\begin{itemize} % {
\iusr{Ксеня Мальована}
\textbf{V.M. Varga} Дістала вже ця Укрзализныця. Постійно відкривається саме російський варіант @igg{fbicon.anger} 
\end{itemize} % }

\iusr{Леся Комарницька}
Клааас!!!  @igg{fbicon.face.tears.of.joy} Де у нас барвінок, то у них могила.

\begin{itemize} % {
\iusr{Anastasia Holubenko}
\textbf{Леся Комарницька}

\ifcmt
  ig https://scontent-mia3-1.xx.fbcdn.net/v/t1.6435-9/244413721_4930547193642057_1748320735721992926_n.jpg?_nc_cat=104&_nc_rgb565=1&ccb=1-5&_nc_sid=dbeb18&_nc_ohc=WGcEYrSRH9oAX-8Ino4&_nc_ht=scontent-mia3-1.xx&oh=b2ed2d0f52f25aa9ad0f48174077cbd8&oe=6182338C
  @width 0.4
\fi

\iusr{Ірина Арсієнко}
\textbf{Леся Комарницька} тільки зараз у голові клацнуло: у нас на Донбасі часто барвінок на могилах якраз і саджають. То, мабуть, це таки ... цапська традиція  @igg{fbicon.face.smirking}{repeat=2} 

\iusr{Natali Blashchyshena}
\textbf{Ірина Арсієнко} та ні... бачила його на цвинтаря там, де цими ...цапами навіть не пахло.
Взагалі якась ритуальна рослина. На весілля з нього досі в багатьох регіонах, на караваї плетуть вінки...

\iusr{Ірина Арсієнко}
\textbf{Natali Blashchyshena} дякую) Треба би про нього почитати

\iusr{Ruslan Gorovyi}
Фраза «стелися барвінку, буду поливати» заграла новими барвами.

\iusr{Леся Комарницька}
\textbf{Ruslan Gorovyi}  @igg{fbicon.thumb.up.yellow}{repeat=3} 

\iusr{Natali Blashchyshena}
\textbf{Ruslan Gorovyi} а фраза з пісні GO-A "зеленим барвінком тай застелися" тож, напевне, російською гарно переклали :-)) вийшла готика...

\iusr{Anton Marek}

В нашому селі до 1947 року хоронили своїх родичів у власних садах, то барвінок
я бачив тільки на цих могилах. А вже в 1985 році ці покинуті сади разом з тими
могилами згорнули в ярок, зробили поле, посіяли пшеницю два рази та й
бросили. Зараз на цих могилах дачі побудовані. Жах просто.

\end{itemize} % }

\iusr{Сашка Трончук}

Отак, с самава утра запутав і злізеш з потяга, а їй шше в наперділому вагоні
їздити і їздити @igg{fbicon.laugh.rolling.floor}{repeat=3} 

\iusr{Nataliya Kucher}

Сьогодні п'ю цитрамон зранку і в процесі запивання згадала "натощак" і наше
"натщесерце". Не вимовлять, це точно!)

\begin{itemize} % {
\iusr{Павло Редзель}
\textbf{Наталія Кучер} можна натще)))

\iusr{Лариса Прядкіна}
\textbf{Nataliya Kucher}
Хоча, здавалося б, серце тут і зовсім ні до чого)
\end{itemize} % }

\iusr{Павло Редзель}
Тяжко, тяжко з такими баранами про барвінок.... запутуются)))

\iusr{Олена Тяпка}
Гарно з раночку таке почитати.

\iusr{Любов Ломакіна}
Знайомі фсьо місця.

\iusr{Olha Akulova}
Класний метод співставлення: барвінок/могильник

\iusr{Любов Крижановська}
І не скрутить морду від такого !?

\iusr{Anna Popovych}
Дайтє білєт да Лашадінава!
Нема такої станції
Дайтє білєт!
Ну нема!
Лааадно- один до Кобеляк...

\begin{itemize} % {
\iusr{Оксана Баран}
\textbf{Anna Popovych} адін білєт в Кабєлякі

\iusr{Anna Popovych}
\textbf{Оксана Баран} написала, як хлопці у Волновасі розказали)))

\iusr{Татьяна Вербава}
\textbf{Anna Popovych} білетів не було, а от квиткі є! 0

\iusr{Anna Popovych}
\textbf{Татьяна Вербава} до Лашадінава? Гммм...

\iusr{Татьяна Вербава}
\textbf{Anna Popovych} про Кобеляки я не вела мову @igg{fbicon.smile} 

\iusr{Anna Popovych}
Кумедні ви, тітонько, я мала *виривати*контекст ,на догоду вам у чужій стрічці?
\end{itemize} % }

\iusr{Богдан Марциняк Волошин}
а за нєй Іванка как магільнік вйоцца!)))

\begin{itemize} % {
\iusr{Андрій Ігнатюк}
\textbf{Богдан Марциняк} Волошин пісня заграла новими фарбами )))
\end{itemize} % }

\iusr{Valya Yelizarova}
Понапридумують ото.

\iusr{Марина Трембач}
Згадується, що Барвінкове було названо на честь козака Барвінка.
У 80ті роки минулого століття пам'ятаю назву російською "Барвенково", не "БарвенковоЕ". Люди там спілкувалися суржиком, де більшість слів було українських

\iusr{Маріна Омелянюк-Сахарук}
Це геніально....)))з самого ранку доза позитиву від Вас... Дякую...)))) ☺ ️  @igg{fbicon.flame} 

\iusr{Eugene Lynnyk}

Жахіття. Це ж треба так мову спаплюжити. До речі, я читав якусь фантастику, в
перекладі на російську, так там якраз барвінок не перекладали, воно й писалося
"венок из барвинков в волосах".

\begin{itemize} % {
\iusr{Лариса Прядкіна}
\textbf{Eugene Lynnyk}
Адже "вінок з могильнику у волосах" якось не дуже романтично звучить, хіба що для заупокійного співу))
\end{itemize} % }

\iusr{Shamryk Andriy}
Супер!)))))

\iusr{Галина Галина}
Я з цього міста і бісить теж як так можна ??

\iusr{Тим Златкин}

Там бар стояв на перехресті, "У Вєнковіча" називався, відомого пивовара, потім
гостиний двір поруч побудували, згодом з'явилося селище, знатна історія рідного
краю  @igg{fbicon.face.upside.down} 


\iusr{Lina Zhuk}

Десь біля Трипілля під Києвом. Їде маршрутка в ті краї і на борту кілька
студентів. Хтось з них питає, а де буде зупинка Пєтушкі? Всі пасажири і водій
здивовані і перепитують: які Пєтушкі? Нема тут такого населеного пункту. Довго
перепитують, а потім !!! хтось здогадується: то вам Півні треба?  @igg{fbicon.smile}  І смішно
було і не дуже.

\iusr{Юрій Величко}
 @igg{fbicon.thumb.up.yellow} 

\iusr{Nazar Pariychuk}

З тієї опери - мені знайомий казав з Харкова: «пашлі зайдьом в РУКАВІЧЬКУ» @igg{fbicon.face.tears.of.joy} 
купім паєсть». Кажу, нема такого слова в російській @igg{fbicon.face.grinning.smiling.eyes}  кажи вже в «пєрчатку»))))
як не хочеш українською говорити

\begin{itemize} % {
\iusr{Наталія Гурошева}
\textbf{Nazar Pariychuk} чому ж, є. Є "перчатка", і є "рукавица", "рукавичка".

\iusr{Nazar Pariychuk}
То не українізм ?) Вперше «чую»)

\iusr{Ogorodniychuk Iryna}
\textbf{Наталія Гурошева} «варєжка» чи «варєшка» там у них замість рукавички)))

\iusr{Наталія Гурошева}
\textbf{Ogorodniychuk Iryna} і варежка є, і рукавиця.
\end{itemize} % }

\iusr{Valerya Boeva}
Барвінкове завжди було, райцентр на Харківщині. Там моя мама народилася.

\iusr{Taras Zarubko}
Супроти ,, барвінок- могільнікі" , ,,чорнобривці - бархатцьі " - це сємєчкі !

\iusr{Larisa Prokopchyk}

Їхала в потязі, сусіди по купе *атдихать єхалі в СхОдніцу. Кажу , може в
Східницю? Нєт, в СхОдніцу. Як в тім анекдоті про піііво...повбивала б))

\iusr{Михайло Заїка}
Добре, що про село згадала... , а не про дірєвню...

\iusr{Михайло Заїка}

Софіївська Борщагівка, 200*-ий рік. Я питаю місцевих "Гдє тут улица
ГайдАра?..." - "нема такої"- відповідають -" Ну, он там (ч) лєніна, здєсь
Гагаріна, а он там де гайдарА... " - " кого-кого? " - "гайдарА, кажу!.. "

\iusr{Оксана Колечко}

\ifcmt
  ig https://scontent-mia3-2.xx.fbcdn.net/v/t39.1997-6/s180x540/100848399_3233314540115934_4088949586100486144_n.png?_nc_cat=105&ccb=1-5&_nc_sid=ac3552&_nc_ohc=p3ki3yBvcKsAX8WvGER&tn=lCYVFeHcTIAFcAzi&_nc_ht=scontent-mia3-2.xx&oh=a3838cb85e37033350fa0792869a6260&oe=6160330D
  @width 0.3
\fi

\iusr{Лидия Кирилова}
Перевод дурацкий. Это просто насмешка.

\begin{itemize} % {
\iusr{Андрей Железняк}
\textbf{Лидия Кирилова} над чем?
\end{itemize} % }


\end{itemize} % }
