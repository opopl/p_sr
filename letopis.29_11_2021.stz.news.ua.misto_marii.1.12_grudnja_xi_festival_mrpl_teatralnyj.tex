% vim: keymap=russian-jcukenwin
%%beginhead 
 
%%file 29_11_2021.stz.news.ua.misto_marii.1.12_grudnja_xi_festival_mrpl_teatralnyj
%%parent 29_11_2021
 
%%url https://mistomariupol.com.ua/uk/mariupol-teatralnyj-yaki-vystavy-predstavlyat-na-hi-festyvali
 
%%author_id news.ua.misto_marii
%%date 
 
%%tags 
%%title ХІ фестиваль аматорських театральних колективів "Маріуполь Театральний"
 
%%endhead 
 
\subsection{ХІ фестиваль аматорських театральних колективів \enquote{Маріуполь Театральний}}
\label{sec:29_11_2021.stz.news.ua.misto_marii.1.12_grudnja_xi_festival_mrpl_teatralnyj}
 
\Purl{https://mistomariupol.com.ua/uk/mariupol-teatralnyj-yaki-vystavy-predstavlyat-na-hi-festyvali}
\ifcmt
 author_begin
   author_id news.ua.misto_marii
 author_end
\fi

\ifcmt
  ig https://i2.paste.pics/PI11K.png?trs=1142e84a8812893e619f828af22a1d084584f26ffb97dd2bb11c85495ee994c5
  @wrap center
  @width 0.7
\fi

12 грудня розпочинається програма ХІ фестивалю аматорських театральних
колективів \enquote{Маріуполь Театральний}. Програма  розрахована на різновікові групи
населення – від маленьких маріупольців до дорослих містян.

Впродовж тижня з 12 по 18 грудня 9 місцевих колективів представлять свої
вистави у різних жанрах та форматах: від комедії до драми.

12 грудня о 12 годині дня буде представлена вистава \enquote{Маленька Баба Яга} від
театральної групи \enquote{Грані}. У спектаклі буде розкрита історія маленької дівчинки
та мудрого ворона, які разом вчаться чаклувати та, одного разу, потрапляють на
чарівне свято. Вистава буде проходити на сцені Маріупольського театру ляльок.

\ii{29_11_2021.stz.news.ua.misto_marii.1.12_grudnja_xi_festival_mrpl_teatralnyj.pic.1}

12 грудня о  17:00 - \enquote{Сільські комедії} театру авторської п'єси
\enquote{Conception} на  сцені ЦСМ \enquote{Готель Континенталь}.

Чотири неповторні гумористичні історії однієї сільської глибинки про яскравих
місцевих жителів. Персонажі зберуться на одній сцені, щоб зробити глядацький
вечір сповненим добрим сміхом, позитивними емоціями та приємними враженнями.

\ii{29_11_2021.stz.news.ua.misto_marii.1.12_grudnja_xi_festival_mrpl_teatralnyj.pic.2}

12 грудня 18:00 – \enquote{Аляска} від Першої театральної школи-студії на сцені
ЦСМ \enquote{Готель Континенталь}.

\enquote{Аляска} – це історія про школярку, яка разом з матір'ю переїде з
великого міста у тихе маленьке містечко. Вона про пошуки щастя та самого себе.
Вистава висвітлює актуальні проблеми підлітків, соціалізацію молоді, аб'юзивні
стосунки, а також тему забуття власної історії та мови. Перформативна вистава
поєднує у собі елементи документального театру та народжений з досліджень,
інтерв'ю першої театральної школи-студії міста Маріуполя, які стали голосом
підлітків тут і зараз.

Прем'єра перформансу \enquote{Аляска} відбулася цього року в рамках проєкту
\enquote{Марафон міжнародних резиденцій} культурно-мис\hyp{}тецьких ініціатив
\enquote{Діалог мовою мистецтва} за підтримки Україн\hyp{}ського культурного фонду.

\ii{29_11_2021.stz.news.ua.misto_marii.1.12_grudnja_xi_festival_mrpl_teatralnyj.pic.3}

15 грудня 19:00 – \enquote{У відкритому морі} від Театроманії на сцені ЦСМ \enquote{Готель Континенталь}.

Трое елегантно одягнених чоловіків дрейфують на плоту у відкритому морі. Дивом
переживши корабельну аварію, вони залишилися практично без провізії. Коли
їстівні запаси добігають кінця, перед джентльменами постає питання, як вижити.

\ii{29_11_2021.stz.news.ua.misto_marii.1.12_grudnja_xi_festival_mrpl_teatralnyj.pic.4}

16 грудня 19:00 – \enquote{А ви буваєте в театрах?} від театру \enquote{Фенікс} на сцені ЦСМ \enquote{Готель Континенталь}.

В основу вистави увійшли три твори Аркадія Аверченко, три істоpії про любов до
високого мистецтва, любові між чоловіком і жінкою, любові між артистом і
режисером, актором і глядачем. Всі герої вистави хочуть бути щасливими, вони,
кожен по своєму, \enquote{рвуться до світлого, променистого життя} і у кожного про
нього своє уявлення. Але всіх іх об'єднує те, що без любові і творчості немає
справжнього життя.

\ii{29_11_2021.stz.news.ua.misto_marii.1.12_grudnja_xi_festival_mrpl_teatralnyj.pic.5}

17 грудня 17:00 – \enquote{Ввечері третього дня} від Театру Юного Актора на
сцені Маріупольського театру ляльок.

Bистава \enquote{Увечеpі третього дня} – про дітей різного віку, які проводили
літні канікули в одній з садиб. Гімназисти, підлітки, діти з почутями,
переживаннями і мріями, які жили на початку 19 століття. Спектакль поставлений
за розповідями Надії Теффі, що увійшли до збірки \enquote{Золоте дитинство} і
дитячим історіям Аркадія Аверченко.

\ii{29_11_2021.stz.news.ua.misto_marii.1.12_grudnja_xi_festival_mrpl_teatralnyj.pic.6}

18 грудня 11:00 – \enquote{Як Петрушка вчитись не хотів} від Маріупольського театру ляльок на його ж сцені.

Кукольна вистава засновується на п'єсі дитячого письменника Маршака, в якій
оповідається історія хлопчика, що хотів співати та веселитися, тому вирішив
замість школи піти в літній сад на зустріч пригодам.

\ii{29_11_2021.stz.news.ua.misto_marii.1.12_grudnja_xi_festival_mrpl_teatralnyj.pic.7}

18 грудня 17:00 – \enquote{Лев, чаклунка та платтяна шафа} від театру
\enquote{Тезіс} на сцені Маріупольського театру ляльок.

В основі цього сюжету розташована історія про чотирьох дітей, які за допомогою
чарівної шафи потрапляють і не менш чарівну країну, яку їм потрібно врятувати
від жахливої Білої чаклунки.

\ii{29_11_2021.stz.news.ua.misto_marii.1.12_grudnja_xi_festival_mrpl_teatralnyj.pic.8}

18 грудня 19:00 – \enquote{Гра} від Театральної артелі \enquote{ДрамКом} на сцені ЦСМ \enquote{Готель Континенталь}.

Пізній вечір ... двоє чоловіків у старовинному особняку ... Один молодий і гарячий,
другий – у літах ... Один – знаменитий письменник, другий – не дуже успішний
тураген ... Вони одні у величезному будинку ... Що можуть робити вони удвох, у
величезному будинку... Що між ними?... Звісно, гра!

\ii{29_11_2021.stz.news.ua.misto_marii.1.12_grudnja_xi_festival_mrpl_teatralnyj.pic.9}

Усі вистави будуть проходити у Центрі сучасного мистецтва \enquote{Готель
Континенталь} та Маріупольського театру ляльок, які знаходяться за однією
адресою – вул.  Харлампіївська, 17/25.

Квитки можна придбати у касах та на сайті \enquote{Карабас}\footnote{\url{https://mariupol.karabas.com/hall/palac-kulturi-molodezhnij}} або у
адміністратора за телефоном 067 900 17 25.

\begin{quote}
\em
Заходи відбуватимуться з дотриманням усіх епідеміологічних норм. Для
відвідування глядачам необхідно буде пред'явити паспорт вакцинації або
негативний ПЦР-тест.	
\end{quote}
