% vim: keymap=russian-jcukenwin
%%beginhead 
 
%%file 13_09_2021.fb.baumejster_andrej.kiev.filosof.1.filosofia_istoria_socializm_marks.cmt
%%parent 13_09_2021.fb.baumejster_andrej.kiev.filosof.1.filosofia_istoria_socializm_marks
 
%%url 
 
%%author_id 
%%date 
 
%%tags 
%%title 
 
%%endhead 
\subsubsection{Коментарі}

\begin{itemize} % {
\iusr{Андрей Баумейстер}
\url{http://gska2.rada.gov.ua/site/const/universal-4.html}

\iusr{Богдан Козак}
Заувага щодо форми.

Коли до Андрія Олеговича застосовують шаблони, Андрію Олеговичу це не
подобається, як і кожному іншому. Але сам Андрій Олегович не проти використати
шаблони/стереотипи/узагальнення щодо замовчування та фальсифікації. Хтось там
замовчує і хтось там фальсифікує. Є таке враження… Хотілося б таке враження… на
фоні якого критика чогось мала б сенс.

Заувага щодо змісту.

Ніхто не заперечує і не замовчує про соціалістичність інтелігенції та політиків
України в кінці ХІХ ст. – на початку ХХ ст. Я, наприклад, пам'ятаю це зі школи.
Я пам'ятаю про своє обурення, коли дізнався, що Грушевський розпустив армію (!)
під час війни (!!!) (більш дебільного вчинку не можна було зробити).

Мова йде про спадкоємність державності, а не про повернення в точку відліку. Я
сьогодні не такий, як у школі, але я успадковую все своє шкільне. Пам'ять про
школу важлива мені, як і будь-якому іншому. А виходить так, що наполягання на
моїй пам'яті – це "ідеологізація, міфологізація та фальсифікація". Чудни дєла
Твоі Госпаді!

Заувага щодо книжних полиць.

Немає видань Лєніна і Троцького (біда-то яка!), а отже їх замовчують. Тут
відверта маніпуляція, бо на українських полицях взагалі (за великим рахунком)
нічого немає. Книжний ринок України супер блідий та бідний у порівнянні з
Німеччиною та Францією. Я рахую так, спочатку треба насичувати ринок всім
пристойним, а останньою чергою книгами терористів та вбивць.

\begin{itemize} % {
\iusr{Андрей Баумейстер}
\textbf{Богдан Козак} 

но книги Грушевского же есть? Как и биографии Петлюры?

Маркс, все-таки, в книжных должен быть. Это была моя первая мысль. 

Вторая мысль, ещё проще: школьникам (и не только им) рассказывают, что пока
здесь мирно строили европейскую страну, с северо-востока вторглись страшные
банды Муравьёва. Большевиков возглавлял, кстати, не только Муравьёв. Но и
родной сын классика украинской литературы Юрий Коцюбинский. Если я не ошибаюсь,
улицу его имени переименовали в улицу Винниченко. О чем я? О понимании того
факта, что левые (причём радикально-левые) идеи взращивались в Украине со
второй половины 19 века. И поэтому большевизм нашёл здесь в начале 20-го года
благодатную почву. Не в Чехии, не в Польше, не в Турции, а в Украине. И это
нужно понимать.  И учиться у истории. И не изобретать себе другое прошлое,
похожее, иногда, на современные идеологические клише. И быть последовательными
в декоммунизации.  Или более мудрым и осторожными (потому что при
последовательной, а не избирательной декоммунизации, украинская литература
может существенно пострадать. Как и искусство вообще, вместе с архитектурой).

\iusr{Denys Zhadiaiev}
\textbf{Богдан Козак} 

я зараз на одній катедрі філософії, де не тільки томи
Леніна, Сталіна, а й платівки з постановами ('Так победим!"). Нема дефіциту!
Але хотілося б чогось з Оксфорду та інших традицій.


\iusr{Андрей Баумейстер}
\textbf{Denys Zhadiaiev} 

звучит красиво. Но, во-первых, речь шла об украинском
переводе Маркса и влиянии левых идей в Украине 19-начала 20 века. А, во-вторых,
в Киевском национальном, Могилянке, УКУ и других ведущих университетах, уже
давно "другие традиции". Зачем ломиться в открытую дверь? Что касается
"оксфордского уровня" науки и преподавания, то, уж извините, есть то, что есть.
Одними желаниями тут не обойтись. Тут бы наука сохранилась в стране. А вы про
Оксфорд...

\iusr{Denys Zhadiaiev}
\textbf{Андрей Баумейстер} Декотрою мірою наука в нас, не в книгах на полицях і не в їх кількості, в принципі. Я читав Лєніна та Маркса і не переконався у виваженості одного, та у оригінальності другого. Звісно, Capital сама по собі цікава назва (може бути і столицею!), але філософія це не про те, скільки і як їсти (економіка), а незвичайний погляд на звичайні речі. У Маркса було запозичено систему Гегеля зі зміною полюсів. Так, мені здається, може кожен студент...

\iusr{Ярослав Крюк}
\textbf{Богдан Козак} ,гарний аналіз тексту товариша Андрія  @igg{fbicon.fist.raised}  @igg{fbicon.flag.ukraina}

\iusr{Богдан Козак}
\textbf{Ярослав Крюк} дякую)

\iusr{Богдан Козак}
\textbf{Andrii Baumeister} 

1) видання Грушевського є, а нормальне видання Драгоманова немає (хоча є гарна
біографія Ушкалова і більше нічого), а українське середньовіччя з Могилянкою -
практично відсутнє. але я погоджуюсь з Вами, що Маркс має бути.

2) Вот імена! Соціалізм в Україні (від Лесі України до Винниченка) орієнтувався
на Маркса і Прудона, тобто на європейські цінності гуманізму й співстраждання
пригнобленим, а в Росії була пропаганда і смерть, смерть і пропаганда. Суцільна
брехня і тисячі вбивств.

В Україні були демократичні процеси, а з Росії йшла орда окупації, яка нікого
не питалась і ніяких інтересів українців не враховувала.

В Росії переміг не Маркс, а терор - улюблена забава Лєніна.

\end{itemize} % }

\iusr{Edgar Leitan}

В Германии теперь, увы, очень сильный перекос именно с крайне левую сторону.
Вся мощь государственнй пропаганды идёт на поддержку именно левых идей, причём
повсеместно, с самого детского сада: постоянная критика "проклятого
капитализма", "взять всё и поделить", а в последнее время и критическая расовая
теория и прочие прелести. При этом даже умеренно консервативное мышление
является крайне нежелательным, консерваторы не допускаются без купюр и
издевательских идеологических поучений до телевидения и т. д.

Но напрасно Вы будете искать консерватизм среди "зачищенной" и выхолощенной
меркелевской CDU. Последняя всё больше напоминает умеренное крыло СЕПГ
(хонеккеровской SED), в отличие от радикальной Die Linke и совсем безумных
Зелёных. Ну и безостановочная борьба gegen Rechts, против правизны.

Однако вся эта дежурная критика капитализма прекрасно сочетается с работой того
же соцдемократа Шрёдера на Газпром, и т. п.

\begin{itemize} % {
\iusr{Андрей Баумейстер}
\textbf{Edgar Leitan} думаю, не только Германия. Речь идёт о левом крене в политике официального Брюсселя

\iusr{Edgar Leitan}
\textbf{Andrii Baumeister} Именно так. Всё даже не крайне консервативное, а просто любое нелевое шельмуется, всеми силами и, как говорят, "из всех утюгов". Нет даже намёка на разумный диалог мировоззрений и идей. Крайная левизна в нашем ЕС - это уже просто как политика Партии в СССР. От прежде консервативно-центристских народных партий остались одни лишь именования "христианских демократов".

\iusr{Edgar Leitan}
Ну а избирателям продают лишь старую консервативную риторику. Тем, кто скучает по старым временам.
\end{itemize} % }

\iusr{Михайло Бойченко}

Готували нове видання Капіталу Маркса українською, більша частина роботи була
виконана, але пару років тому чомусь усе застопорилося. Може хтось знає, чому?

\begin{itemize} % {
\iusr{Андрей Баумейстер}
\textbf{Михайло Бойченко} эй, может кто-то знает, отзовитесь)))! Наверное, денег не хватило

\iusr{Юра Чопко}
\textbf{Andrii Baumeister} Бінго!

\iusr{Михайло Бойченко}
\textbf{Andrii Baumeister} хотілося би серйознішого ставлення. Наприклад, якщо ця тема дійсно хвилює, інвестуйте у видання


\iusr{Андрей Баумейстер}
\textbf{Михайло Бойченко} это уже что-то новое... Может дадите полный список, куда инвестировать?

\iusr{Михайло Бойченко}
\textbf{Andrii Baumeister} це Ви виявляєте інтерес, я нічого не нав'язую  @igg{fbicon.face.happy.two.hands} 

\iusr{Андрей Баумейстер}
\textbf{Михайло Бойченко} конечно, это же очевидно
\end{itemize} % }

\iusr{Viktor Levandovsky}

По моему мнению, вопрос поставлен по сути о зрелости социума и претензии к его
инфантильной (infant - грудничек) картине мира.

Упрощенные варианты мессаджа:

\begin{itemize}
  \item 1) ребенок, стань взрослым 
  \item 2) взрослый, оставь детство
\end{itemize}

Вот только нету точки отсчета и меры - наш социум все еще в детстве, и нужно
ждать взросления, или это инфантильный взрослый, кому пора развинуться.

Верно ли я Вас понял?

\begin{itemize} % {
\iusr{Андрей Баумейстер}
\textbf{Viktor Levandovsky} где-то так
\end{itemize} % }

\iusr{Юра Чопко}

Не бачу потреби виставляти Маркса і, тим більше, Леніна на полиці книжкових
магазинів. Їх можна легко купити по ціні мотлоху прямо з підлоги, точніше, з
асфальту, на Петрівці, хоча б. Крім того, треба враховувати нечувані тиражі тих
письменників за радянських часів. В багатьох бібліотеких, як от в моїй, до
прикладу, їх писаннячка все ще непогано збереглися тесь в далеких, темних
кутках.

Шановний Автор плутає ідеї наступності і закостенілості. Не треба багато
зусиль, аби довести, що наші закони доволі суттєво відрізняються від
УНРівських. Гадаю, коли ми говоримо про наступництво, треба розуміти, перш за
все, наступництво в прагненні бути вільним народом, а не задвірками котроїсь з
імперій. В такому розумінні наша війна за Донбас цілком подібна з війною УНР з
більшовиками.

Та і закони наші не такі вже й різні. Правда, в нас існюють приватні банки. Але
існує також визначення держави в Конституції як "соціально", що б це не
означало. Існують статті про право на освіту, охорону здоров'я, свободу
об'єднуватися в профспілки. Все це можна розглядати як спадщину УНР, вартує
чогось ця спадщина чи ні.

Найсерйозніше питання - це питання, чому народ повівся на соціалістичні ідеї.
Моя думка така, що, насправді, ті ідеї були не такі вже й соціалістичні.
Парцеляція панської землі була здійснена ще французькими революціонерами, котрі
від того не стали менш "буржуазними". На селі, а там жила переважна більшість
українців, існувала кричуща несправедливість в тому розумінні, що той, хто її
обробляв, одержував мізерну частину прибутку. Це треба було ліквідувати. І той,
хто це перший зробив би, завоював би владу.

Ленін і Троцький були не останніми політиками. І зовсім не дарма їх першими
декретами були про мир і землю. Тобто, вони задовольнили найпекучіші народні
прагнення. За що народ, зрештою, привів і втримав їх при владі.

\begin{itemize} % {
\iusr{Андрей Баумейстер}
\textbf{Юра Чопко} 

купити "як мотлох"? Серьёзно? Любимая история в моем кругу, как Маркса
переводили на украинский с русского языка. Может пришло время переводить с
немецкого? И тогда, возможно, это будет называться философской классикой в
украинских переводах?

\iusr{Юра Чопко}
\textbf{Andrii Baumeister} 

Та хто ж заваджає! Перекладайте, на здоров'я. Лише не бачу попиту на таку
"філософію". І саме тому, що в нас, все таки, ще не зовсім відбило
"историческую память".

Врахуйте, що в нас лівацькі ідеї багато менше поширені, ніж в Німеччині. Тому й
ліві книжки продаватимуться значно гірше, ніж там.

Сподіваюся Ви, як правий консерватор, не дуже шкодуєте за таким станом речей.

П.С. Не можу втерпіти щоб вкотре не попросити пояснити, що воно таке - "правий
консерватор" у Вашому розумінні.

\iusr{Gleb Nemo}
\textbf{Юра Чопко} 

а ще одним із перших декретів був про антисемітизм, скорочення алфавіту, а
також розглядалося питання стосовно того, щоб жінка була спільною.

\iusr{Андрей Баумейстер}
\textbf{Юра Чопко} 

у меня есть видео на эту тему. Будет время, запишу ещё одну беседу. Что
касается "непопулярности" левых идей в современной Украине, позволю себе не
согласиться. Они чрезвычайно популярны, но ушли "вглубь сознания". Тем они
опаснее. Лучше артикулированные левые идеи (интеллектуалами и политиками), чем
стихийный и придавленный ресентимент. Когда прорвет, будет поздно


\iusr{Юра Чопко}
\textbf{Andrii Baumeister} 

Знову таки, важливо відрізняти ліві ідеї і популізм.

Нещодавно з подивом довідався, що моїй племінниці в католицькому університеті
викладають франкфурктську школу. Поза межами цієї поважної, хоч і дуже молодої
установи, здається, ніхто неомарксизмом особливо не цікавиться.

Популізм, зрозуміло, представлений значно ширше в нашій політиці. Не в останню
чергу, Юлією Тимошенко з її "тисячею". Але яка така особлива від неї загроза, я
не доберу. Можливо, Вам в Києві краще видно, де воно збирається "прорвати".

\end{itemize} % }

\iusr{Gleb Nemo}

Не плохо было бы и Mein Kampf на полки в свободный доступ, и Вернера Мазера (не
шучу). При отправке этого комментария Фейсбук отправил предупреждение, что я
могу нарушать нормы сообществ ))

\begin{itemize} % {
\iusr{Андрей Баумейстер}
\textbf{Gleb Nemo} в Германии уже продаётся
\end{itemize} % }

\iusr{Гліб Парфьонов}

Так ведь и короткая гражданская война между Гетьманатом Скоропадского и
петлюровцами была больше как раз завязана на идеологической борьбе, нежели на
том что Гетьман якобы взял и поддержал белогвардейцев. Чего стоят хотя бы
переговоры одного из заговорщиков - Винниченко, с Москвой в том ключе что они
подпишут мир (а Украина тогда вела гибридную войну с РСФСР, не смотря на
подписанный прелиминарный мир) и будут строить революционное будущее, хотя на
самом деле большевики как мы знаем обманули Петлюру и Винниченко. Потому надо
все говорить прямо и четко, тогда и станет ясно почему тот же Булгаков не любил
Петлюру и не приписывать ему украинофобию. Ведь все дело в контексте


\iusr{Maria Lypych}
\textbf{Андрій Олегович} дякую за аналіз!

Не може бути сучасної форми націоналізму, щоб народ/держава була об’єднана
сучасними і ефективними ідеями, а не застарілими неауктуальними на часі?

Хочу сказати два моменти, про застарілу економіку, що мені відізвалося, і
духовність.

В нас ще і до недавна зберігалася така застаріла ментальність про невміння
поводитися з грошима, отримання прибутку - моральне зло, уявлення в межах
традиційної економіки.

І духовність/мораль/релігія(християнство) яке було (і ще мабуть трохи є) дуже
застаріле, в нас була ментальність як сказав П. Полонський у Вашій розмові, що
Бог дав статичні закони, закінчений варіант, а ви лише маєте їх виконувати, в
нас так і було, всее, мовчати, виконуй те що сказано і все, а є динамічний
підхід, я за динамічний, «динаміка як невід’ємна частина божественності», коли
я це почула про статичні закони мені істина відкрилася про те що в мене
відбувалося.

\iusr{Kostya Shneytor}

У человека, который обманывает себя о своём прошлом, и настоящее - не совсем
настоящее, и будущее о котором подумать страшно. Это касается и каждого из нас
в отдельности и, в той же, или даже большей степени - любые человеческие
сообщества, и человечество в целом. Чтобы жить полноценно, нужно всегда видеть
ситуацию, как она есть, а не воображаемую картинку. Для этого, во-первых -
нужно принять свою историю во всей полноте.


\iusr{Анна Шаловинская}

Большая часть комментариев... как-то не туда... я прочла мысль автора в том, что
понимание максимально объективной истории(насколько это возможно об истории,
которая всегда, при любой власти переписывалась) обязательно и необходимо.
Уничтожение культурного и исторического наследия(любого) и искусственные
запреты всегда приводят к пагубным результатам. Можно не принимать какой-либо
идеологии, оспаривать, но прежде чем это делать... её нужно изучать и знать. В
контексте образования любого человека. Наших детей и нас самих сейчас лишают
очень многого( да, конечно, мы можем покупать книги, исхитряться находить
качественную литературу. Правильно пропагандировать украинскую литературу. Но
меня очень расстраивает качество украинских переводов и тренд на «жуткий
украинский», на некачественное содержание...( я до сих пор предпочитаю литературу
на русском или английском ибо крайне мало украинских аналогов(есть, появляются,
но крайне мало).

\iusr{Владимир Демьянов}

Андрей, достанется Вам. Вспомните Жванецкого: "Где ты купил эту дрянь? Почём?
Сколько-сколько?? Ну и надули же тебя!" Этот и сам понимает, что лоханулся, но
злится не на того, кто надул, а на того, кто ему глаза пытается разуть.

\iusr{Dmytro Kuzmenko}

Стосовно замовчування лівацтва Петлюри чи Винниченка, то я з таким не стикався,
радше навпаки. І в школі і в університеті про це постійно нагадували, що ось де
ми мали правильний соціалізм, але програли війну неправильному. Коли я вчився,
то мене більше обурювало, що замовчується Скоропадський. Соціалісти 1920-тих
мають зв'язок із 1990-тими, бо ж незалежну Україну спершу будувати колишні
комуністи. Але симпатії народу до соціалізму розвіялися за сто років майже
безслідно. Після 2004 і, тим більше, після 2014 героєм боротьби за незалежність
став Павло Скоропадський, його праці (і книги про нього) зараз активно
видаються, разом із роботами Донцова, навіть справжнього правого консерватора
Липинського пробували видавати. Нарешті вже можна сказати, що Гетьманат не
замовчується, як у 1990-ті. Чи означає зміна пріоритетів і цінностей, що забули
за соціалізм Винниченка й Петлюри? Ні, радше зрозуміли, що причиною поразки у
боротьбі за незалежність був саме цей соціалізм. І що в нинішній Україні це все
неактуально, тож і впливових лівих (за діями, а не словами) партій у нас уже
немає й навряд чи будуть.

А стосовно Маркса, то про які видання Маркса можна говорити, якщо в нас навіть
Арістотеля українською в нормальному обсязі немає, не кажучи вже за Гегеля. От
коли будуть всі важливі праці Арістотеля, Гобса, Юма, Гегеля, Шопенгауера,
Ніцше, Гайдеґера, Мілла, К’єркеґора, Пірса й Рассела, то після них і до Маркса
черга дійде.

\iusr{Alexander Levine}

Удивительная на первый взгляд дихотомия получается между крайне агрессивном
процессом дистилляции сознания масс (со стороны либерального 'новолева' ) и
существующими в свободном доступе материалами и первоисточниками, касательно
исторической памяти. Хотя ответ может быть довольно простым: до этих 'трёх
шкафов с литературой о третьем рейхе' средний благочестивый оптимист просто
физически не может добраться, потому как на пути его гаджет может предложить
что-то более нейтрально ориентированное)). Но это просто торжество
инклюзивности?

\iusr{Мкртич Мкртчян}

\ifcmt
  ig https://scontent-lga3-1.xx.fbcdn.net/v/t1.6435-9/241904819_6060854643985557_7543552305416102888_n.jpg?_nc_cat=100&ccb=1-5&_nc_sid=dbeb18&_nc_ohc=C6V50gT8t1IAX9R9ag-&_nc_ht=scontent-lga3-1.xx&oh=3110766dc9d78ef0e35bba8c6434183d&oe=61718D36
  @width 0.5
\fi

\iusr{Volodymyr Matkovskyi}

В Україні з виданням книг взагалі біда. Мабуть більше ніж 10 років шукав книгу
Рагнера-Форгрімлема "короткий теологічний словник". Так і не знайшов. Потім, в
Чехії зайшов в перший католицький магазин і купив. А Ви, переживаєте про Маркса
і Леніна яких можна скачати в інтернеті.


\iusr{Владимир Величко}

Ви потужно розмірковуєте. Але це дуже дуже складно для сьогоднішньої влади.
Вони більшу половину, що Ви написали просто не зрозцміють, а прості люди і
подавну...

\iusr{Бикреев Сергей}
Зачем вурдалаков печатать?

\iusr{Юрий Чернявский}

Если национал-социалисты строили социализм для отдельно взятой нации, марксисты
за глобальную социалистическую революцию для рабочего интернационала, то
единственной неопробованной позицией остаётся корпоративистский анархизм -
социализм, корпоративизм соединяющий защиту рабочего класса с разрешением на
частную собственность как у нацистов, но в рамках не чисто-кровной нации
объедененной в одно централизованной государство, а в рамках объедененных
соседских (территориальных) общин, а не кровных общин. Получается лоскутное
одеяло, patchwork-и из корпоративистских общин. Этакой синтез укрианского Гуляй
поля и Холодного яра.


\iusr{Віталій Козаренко}
все так було і буде і історію і філосорфію та і інші розділи наук будуть завжди
використовувати в залежності від нужд політиків

\iusr{Vitail Fisun}

Я смотрю как Вас, Андрей, интеллектуала деформирует от широкой росийской
имперской идентичности к «несогласию» с бурным восстановлением «узкой»
украинской нац. индентичности, которая конечно содвалась по проектам... Вы точно
философ?

\iusr{Tim Vlas}

Група 239, яка створила Україну в 1989-1996 були комуністами, соціалістами.
Різні соціалістичні та більшовицькі партії засідали у ВРУ до 2014 року.

Партія "Слуга народу" є соціал-демократичною чи правою?

"Голобородько" прийшов до влади на відверто соціалістичних гаслах. Тому його
підтримали 73\%.

Варто розрізняти гасла і політику більшовиків минулого, сучасних
необільшовиків. НЕП у попередній версії відомо чим закінчився.



\end{itemize} % }
