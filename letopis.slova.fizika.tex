% vim: keymap=russian-jcukenwin
%%beginhead 
 
%%file slova.fizika
%%parent slova
 
%%url 
 
%%author_id 
%%date 
 
%%tags 
%%title 
 
%%endhead 
\chapter{Физика}

%%%cit
%%%cit_head
%%%cit_pic
%%%cit_text
\emph{Физики-теоретики} отметили свой успех бутылкой коньяка. В память о работе
им выдали в качестве сувенира по номерному ключу от замка электронной системы
предохранения бомбы. Считается, что инициированное Сахаровым снижение мощности
бомбы в два раза позволило исключить серьезное радиоактивное загрязнение, а
также разрушения на площадках полигона. При этом ядерщики убеждены, что взрыв
при полной загрузке вызвал бы ядерный смерч, способный накрыть территорию
размером с Владимирскую область.  60 лет назад руководство СССР и вместе с ним
вся страна ликовали. Всего через полгода после полета Юрия Гагарина в космос
удалось показать миру наличие сверхмощного оружия. По мнению Хрущева, взрыв
«Царь-бомбы» должен был отрезвить американских «ястребов», показав им
научно-технический потенциал СССР
%%%cit_comment
%%%cit_title
\citTitle{«Дом будто ножом срезало» 60 лет назад СССР взорвал «Царь-бомбу» — самую мощную в истории. Что помнят о взрыве очевидцы?: Общество: Россия: Lenta.ru}, Дмитрий Окунев, lenta.ru, 30.10.2021
%%%endcit
