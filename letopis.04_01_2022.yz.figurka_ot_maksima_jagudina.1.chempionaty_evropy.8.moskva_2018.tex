% vim: keymap=russian-jcukenwin
%%beginhead 
 
%%file 04_01_2022.yz.figurka_ot_maksima_jagudina.1.chempionaty_evropy.8.moskva_2018
%%parent 04_01_2022.yz.figurka_ot_maksima_jagudina.1.chempionaty_evropy
 
%%url 
 
%%author_id 
%%date 
 
%%tags 
%%title 
 
%%endhead 
\subsubsection{Чемпионат Европы 2018 (Москва)}
\label{sec:04_01_2022.yz.figurka_ot_maksima_jagudina.1.chempionaty_evropy.8.moskva_2018}

Мы видим, как на протяжении всей истории турнира результаты неизменно растут,
после долгих лет отсутствия прогресса у женщин как в технике, так и презентации
Россия вывела женское фигурное катание на абсолютно новый уровень и продолжает
с каждым годом этот уровень повышать. 

\ii{04_01_2022.yz.figurka_ot_maksima_jagudina.1.chempionaty_evropy.8.moskva_2018.pic.1}

Это был первый Чемпионат Европы в Москве с 1965 года. Сделать пьедестал
полностью российским снова помешала легендарная \textbf{Каролина Костнер}, которой к
моменту этого турнира был уже 31 год без нескольких недель. При этом для
третьего места Каролине хватило трех тройных прыжков и одного каскада за
произвольную программу - уровень судейской лояльности по отношению к Каролине
достиг запредельной величины.

\ii{04_01_2022.yz.figurka_ot_maksima_jagudina.1.chempionaty_evropy.8.moskva_2018.pic.2}

Такую Костнер \textbf{Маша Сотскова} должна была опережать, тем более перед этим
турниром она великолепно выступила в финале Гран-при и на Чемпионате России,
завоевав две серебряные медали. В итоге Маша допустит небольшие ошибки в обеих
программах и этого хватит, чтобы проиграть Каролине. На Олимпиаде и Чемпионате
Мира будут восьмые места и это станет последним большим успехом Маши, через
сезон в 19 лет она завершит спортивную карьеру.

На этом Чемпионате Европы случилось событие о котором даже боялись подумать
персональные поклонники \textbf{Евгении Медведевой} - прервалась ее победная серия,
которая к этому моменту уже насчитывала 13 личных турниров подряд. Победить
помешали последствия тяжелой травмы, ошибка на двойном акселе в короткой
программе и чуть меньшее число прыжков во второй половине в произвольной
программе, чем у основной соперницы. Этот чемпионат Европы станет третьим и
последним в карьере Жени. Через месяц будет драматичная Олимпиада, второе место
и в командном, и в личном турнире, затем скандальный переезд в Канаду к Орсеру,
где Евгения завоюет свою последнюю медаль на главных стартах в Японии на
Чемпионате Мира 2019. Последний соревновательный прокат Женя провела в возрасте
20 лет значительно задержавшись в большом спорте относительно большинства
других наших звездных фигуристок последнего десятилетия.

Чемпионкой Европы 2018 в итоге стала 15-летняя дебютантка взрослых стартов
\textbf{Алина Загитова}, выигравшая в своем первом же сезоне три главных старта из
четырех и лишь на последующем Чемпионате Мира выступившая неудачно. От
новаторского тренерского штаба Алины мы увидели контент со всеми прыжками во
второй половине и риттовыми каскадами в обеих программах, что стало еще одним
шагом на пути к предстоящим революционным усложнениям.
