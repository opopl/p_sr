%%beginhead 
 
%%file 20_12_2021.fb.arximisto.1.mariupolcy_podarili_pervoj_gimnazii_fotografii
%%parent 20_12_2021
 
%%url https://www.facebook.com/arximisto/posts/pfbid0kToTx7ML4o7iPeRYUJbjty2bftSpR9nrsGh42VdcEk3pJFKmdFBfMQRwzuVkMierl
 
%%author_id arximisto
%%date 20_12_2021
 
%%tags 
%%title Мариупольцы подарили Первой гимназии уникальные старинные фотографии
 
%%endhead 

\subsection{Мариупольцы подарили Первой гимназии уникальные старинные фотографии}
\label{sec:20_12_2021.fb.arximisto.1.mariupolcy_podarili_pervoj_gimnazii_fotografii}

\Purl{https://www.facebook.com/arximisto/posts/pfbid0kToTx7ML4o7iPeRYUJbjty2bftSpR9nrsGh42VdcEk3pJFKmdFBfMQRwzuVkMierl}
\ifcmt
 author_begin
   author_id arximisto
 author_end
\fi

Мариупольцы подарили Первой гимназии уникальные старинные фотографии

\#новости\_архи\_города

Мариупольцы Ярослав Федоровский и Андрей Марусов подарили музею Первой гимназии
(Мариупольской Мариинской женской гимназии) более десяти уникальных фотографий
гимназисток 1902-1909 годов, выкупленных на антикварном
интернет-аук\hyp{}ционе. Они призывают жителей города создавать клубы меценатов
при городских музеях для пополнения их коллекций и превращения в современные
культурные \enquote{магниты} Мариуполя.

Фотографии и документы когда-то принадлежали Надежде Гавриловне Никошевой,
выпускнице Мариинки 1907 года. Она родилась в 1889 году в семье зажиточных
мещан-греков. 

По данным краеведа Сергея Катрича, в ее роду были также итальянцы Томазо,
которые построили гостиницу \enquote{Континенталь}, первую частную
электростанцию в городе и т.д. 

Новогодний подарок меценатов включает двенадцать фотографий гимназисток и их
друзей, аттестат Надежды Никошевой об окончании полного курса гимназии и
присвоении ей звания домашней учительницы, а также коллективное фото выпускниц
вместе с преподавателями и начальницей гимназии Александрой Генглез.

Все желающие смогли ознакомиться с ними на публичной презентации, проведенной
на выходных в кофейне Story в Старом Мариуполе.

Эти фотографии станут нашей жемчужиной, поскольку в коллекции музея немного
оригиналов дореволюционных документов, как отмечает Анна Дежец, директор музея.
Из-за отсутствия средств музей не в состоянии их приобретать самостоятельно.

Мариинская женская гимназия была несправедливо забыта в советские времена.
Между тем, именно ее выпускницы стали основой корпуса учителей Мариуполя и
Приазовья в начале XX века, как подчеркнул Ярослав Федоровский. Чтобы
стимулировать изучение ее истории, весной этого года я заказал в архивах и
подарил музею копию списка ее выпускниц 1903-1916 годов. А потом мы выяснили,
что на интернет аукционах продается достаточно много документов, фотографий и
даже вещей, которые прямо относятся к истории гимназии...

Меценатом может стать каждый, убежден Андрей Марусов, директор ГО \enquote{Архи-Город}.
Стоимость дореволюцинных фотографий не так уж и велика – в среднем, около ста
гривен. У нас на примете еще несколько ценных документов и фотографий. Мы
призываем всех желающих – давайте вместе сделаем музей Первой гимназии
настоящим культурным \enquote{магнитом} Мариуполя! 

И дело даже не в приобретении уникальных артефактов, по мнению А. Марусова. Во
всем мире клубы меценатов и попечительские советы при музеях и галереях – это
площадки для их развития. Они связывают музеи с ценителями истории и культуры и
позволяют им оставаться интересными даже во времена, когда все вокруг
стремительно меняется. Присоединяйтесь!

Напомним, что этим летом общественные активисты собрали необходимые средства и
инициировали предоставление зданию Первой гимназии статуса памятника
архитектуры. Подготовленная заявка была рассмотрена управлением культуры при
Донецкой облгосадминистрации и в настоящее время дорабатывается инициаторами с
учетом замечаний, высказанных чиновниками.

========================================

Мы благодарим администрацию кофейни Story за пространство для проведения презентации!

Мы будем рады предоставить более подробную информацию об инициативе! Контакты:
arximisto@gmail.com, 096 463 69 88 (Андрей Марусов), 0667256777 (Ганна Дежец).

Списки выпускниц Мариинской женской гимназии 1903-16 гг. см. здесь \url{https://cutt.ly/4Y6bZlI}
\footnote{Internet Archive: \url{https://archive.org/details/mariinka_graduates_1903_1916_final}}
