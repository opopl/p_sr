% vim: keymap=russian-jcukenwin
%%beginhead 
 
%%file 02_08_2021.fb.volga_vasilii.1.uzhe_ne_bolit
%%parent 02_08_2021
 
%%url https://www.facebook.com/Vasiliy.volga/posts/2957977334519679
 
%%author Волга, Василий Александрович
%%author_id volga_vasilii
%%author_url 
 
%%tags obschestvo,ukraina
%%title УЖЕ НЕ БОЛИТ.
 
%%endhead 
 
\subsection{УЖЕ НЕ БОЛИТ.}
\label{sec:02_08_2021.fb.volga_vasilii.1.uzhe_ne_bolit}
 
\Purl{https://www.facebook.com/Vasiliy.volga/posts/2957977334519679}
\ifcmt
 author_begin
   author_id volga_vasilii
 author_end
\fi

УЖЕ НЕ БОЛИТ.

Даже не знаю сколько сотен моих подписчиков из России и других стран Мира
задали мне в разных интерпретациях один и тот же вопрос: «Если триста пятьдесят
тысяч человек вышли на Крестный Ход в Киеве, то почему ни один их них не вышел
на защиту Монумента Славы во Львове, когда его сносили нацисты?». 

Отвечу просто. Потому, что уже не болит. 

Тридцать лет не прошли даром, и тем, кто настойчиво продолжает верить в «один
народ», предлагаю еще раз и более внимательно посмотреть на нашу новую,
созданную не так давно, только лишь за тридцать последних лет, украинскую
социальную действительность. 

У народа, как и у отдельного человека есть личность, душа, если хотите. И так
же, как личность человека способна к раздвоению, к расщеплению, к болезни, так
же и душа народа, способна дробиться и болеть. 

И что характерно. Стоит только народной душе заболеть, стоит только бесам
вселиться хоть в какую-то часть народной души и начать процесс дробления, стоит
только из единого народа выделиться первому болезненному ростку новой личности,
пусть даже самому маленькому, хромому, бледненькому – дальше этот процесс
остановить бывает уже почти невозможно. Иногда помогают экзорцисты, иногда
лоботомия, но иногда не помогает уже ничего. 

Когда Верховная Рада Украины приняла Закон, запрещающий преподавание на русском
зыке, почти ни один человек не вышел с протестом. 

Когда русских на Украине объявили вне закона, ни один человек не вышел с
протестом. 

Когда по всей стране сносят памятники нашим отцам и дедам, освобождавшим нашу
страну от нацизма, когда прославляют нацистов, когда палачи становятся героями,
никто из нас не выходит с протестом. 

Нам не страшно. Нам безразлично. 

Почему? А потому, что в душе украинского народа больше нет той нравственной матрицы, которая есть, например, у народа русского.

Великая Отечественная Война была не только военно-политическим событием, она
была событием Нравственно-Волевым! Народы Советского Союза, сплотившиеся в
единый Советский народ, который во всем мире для простоты называли народом
Русским, явили небывалые высоты самоотверженного служения Родине, патриотизма,
жертвенности и массового героизма, когда казах закрывал своим телом русского,
русский спасал жизнь украинцу, жертвуя собой, как это было с моим дедом под
Сталинградом.

Именно это для сегодняшней Росси и есть нравственным фундаментом, который и
определяет Добро и Зло в современном мире. Об этом снимают фильмы, пишут
статьи, романы, поэмы, об этом говорят первые лица государства. Т.е.: наши отцы
отдавали свои жизни за то, чтобы никогда больше не было войны, чтобы никогда
больше ни одна нация, ни один народ не решил вдруг, что он имеет какое-то особе
право по отношению к другим народам, а все другие народы ему должны, что
самопожертвование ради общего дела – это правильно, что Победа одна на всех –
это правильно, что мы за ценой не постоим и никто из нас за ценой не постоит, и
«нет большей любови, как кто душу свою положит за други своя».

Это там, друзья, у вас, в России. И когда кто-то залазит в эту нравственную
матрицу своими грязными гробовыми ручками, словно приведение из тридцатых годов
прошлого века Германии, то тут же болью во всем народном теле отдается это
касание к живой народной душе. И это невозможно. Это невозможно даже в Трептов
Парке в Германии, где стоит величественный памятник Русскому Солдату,
рассекающему мечом арийскую свастику. Но это возможно на Украине. Ибо душа
украинского народа, которая когда-то, лет так триста тому назад начала
отделяться от единой души Русского народа, сегодня достигла высшей степени
своего автономного существования и утвердилась в нем, верит в него, им есть. 

Тридцать последних лет нравственная матрица украинского народа была подвержена
системному тлетворному воздействию. Заболела, заразилась, пошла трупными
пятнами. Кровушку пустили в 2014 году, и кровушка эта сделала свое дело. Теперь
нет того, что болело бы в украинской душе, когда туда залазят руки нациствующих
приведений. Просто болеть нечему. Ибо Великая Отечественная Война – это не
святая война, а захватническая, и захватила нас Москва, а не Берлин. Берлин
обещал нам свободу. Так чему же радоваться 9 Мая? Нечему. Отчего скорбеть,
когда уничтожают памятники воинам освободителям? Это ведь каратели, захватчики,
палачи. И тлеет тихонько радость неподдельная в украинской новой душе, когда
уничтожают Жукова, Кожедуба, Гастелло, Матросова, когда уничтожают памятник
солдату, погибшему при освобождении Львова от немецко-фашистских захватчиков,
ибо захватчики они только для русских, для нас же, для украинцев, они –
освободители от русских. 

Так думают не все на Украине, но большинство. А те же, кто так не думает,
комфорт свой ценят гораздо выше, чем убеждения. Убеждения у таких украинцев
что-то вроде успокоительной таблетки на ночь. Пососал, и в койку. Я ведь не
такой. Я с убеждениями. Но комфорт мне дороже. Боюсь я, что больно мне будет,
если вдруг выступать против тех других буду. Их вон сколько и силища какая за
ними, а за мной кто? Никого. Так зачем же?!

И засыпает новый украинец со спокойной совестью. 

Так понятно, друзья?

\ifcmt
  pic https://scontent-cdg2-1.xx.fbcdn.net/v/t1.6435-9/229839703_2957978621186217_7495233060556934415_n.jpg?_nc_cat=107&ccb=1-3&_nc_sid=730e14&_nc_ohc=-vJwDbh_bAYAX9jQ2QG&_nc_ht=scontent-cdg2-1.xx&oh=73567df986716b2285a0b85ee55e9939&oe=61318DED
  width 0.4
\fi
