% vim: keymap=russian-jcukenwin
%%beginhead 
 
%%file 30_03_2022.fb.skyba_jakubova_ivanna.harkiv.1.mama
%%parent 30_03_2022
 
%%url https://www.facebook.com/ivanna.skyba/posts/4939947469374597
 
%%author_id skyba_jakubova_ivanna.harkiv
%%date 
 
%%tags 
%%title Ох, мама, все одно ти рано чи пізно дізнаєшся
 
%%endhead 
 
\subsection{Ох, мама, все одно ти рано чи пізно дізнаєшся}
\label{sec:30_03_2022.fb.skyba_jakubova_ivanna.harkiv.1.mama}
 
\Purl{https://www.facebook.com/ivanna.skyba/posts/4939947469374597}
\ifcmt
 author_begin
   author_id skyba_jakubova_ivanna.harkiv
 author_end
\fi

Ох, мама, все одно ти рано чи пізно дізнаєшся. Я вже 10 днів у Харкові, і я
щаслива кожної хвилини свого буття, щаслива настільки безмежно, що мені навіть
не соромно, що я щаслива під час війни, бо мабуть тільки так і можна вижити в
час війни — бути щасливою. 

% 1-3
\ii{30_03_2022.fb.skyba_jakubova_ivanna.harkiv.1.mama.pic.1}

Я захлинаюся щастям від першого блокпосту на в'їзді в область, від першого
офіцера, який побажав мені добре доїхати, від літнього вояка з ліхтариком, який
світив мені між їжаків, шоб я не в'їбалася в темряві, від першої парковки біля
дому моїх найдорожчих сьогодні друзів. 

Я щаслива бачити, як під моїми вікнами пацани ганяють мяча, як над Лопанню
цвітуть первоцвіти і розмірено пливуть качечки, я млію від звуку ППО, солодкого
як оргазм коханої жінки, звуку який означає «по-лу-чи-лось», від сяючої чистоти
вулиць, від цього величезного натхненного мурашника, який тягне й тримає. 

Я нарешті відсипаюся (сьогодні не спиться, бо надто багато незавершених справ
на завтра), не схоплююся серед ночі курити на кухню від незрозумілих новин, я
взагалі вдвічі менше курю, я сплю без трусів з вірою в ЗСУ (й знову-таки, ППО). 

% 4-5
\ii{30_03_2022.fb.skyba_jakubova_ivanna.harkiv.1.mama.pic.2}

Я їм дешевий авокадо, насмажила котлет, зварила суп з залишками сушених
ізюмських (!!!) грибів і навіть компот. 

Я щаслива обіймати своїх старих і нових друзів, і тих, хто друзями не були, але
стали, не уявляти собі страшні картини, а бачити Місто, Яке Стоїть, Місто
Залізобетон.

% 6-8
\ii{30_03_2022.fb.skyba_jakubova_ivanna.harkiv.1.mama.pic.3}

Держпром у заході сонця світиться, мам. Він золотий, мій ріднесенький.

У перший день я розібрала на складі кількадесят коробок, які сама ж і
відправляла з Хмеля — вони приїхали буквально за добу переді мною, і я була
щаслива що можна не дзвонити і не питати. 

% 9-11
\ii{30_03_2022.fb.skyba_jakubova_ivanna.harkiv.1.mama.pic.4}

Я була щаслива топити 120 по місту, позбавленому світлофорів і односторонніх
вулиць. Щаслива прокинутися якось в шостій ранку у друзів, всунути ноги в
кросівки і сісти за кермо ще до першої кави. Щаслива зварити каву від коханих
дружечків на своїй власній кухні. Я сплатила комунальні платежі, бо я маю
гарячу ванну, гарячу ванну, мам!

% 12-13
\ii{30_03_2022.fb.skyba_jakubova_ivanna.harkiv.1.mama.pic.5}

Я вожу містом іноземних журналістів, які провели 12 років на найстрашніших
війнах світу і які, сьорбаючи мій суп, заливаються від захвату нашими ЗСУ, які
надирають сраку русні. 

% 14-15
\ii{30_03_2022.fb.skyba_jakubova_ivanna.harkiv.1.mama.pic.6}

Мам, я сьогодні постриглась у перукарні!

Я нарешті виключила цей довбаний аякс, який доводив мене до сказу в
Хмельницькому. Я сплю в солодкій тиші, знаючи, що зранку я буду потрібна.

У мене розвязалися всі вузлики, які мене тисли більше року. Я навчилася простим
речам і складним речам. Здається, я ще ніколи в житті не була щасливою
настільки повно.

Днями я, дасть бог, зустрінуся з нашими посилочками з різних місць і ручками
сама їх роздам, руки в руки. 

Мама, Харків — це Щастя. Це мій Ніжний Залізобетон

\#щоденники\_війни

\ii{30_03_2022.fb.skyba_jakubova_ivanna.harkiv.1.mama.cmt}
\ii{30_03_2022.fb.skyba_jakubova_ivanna.harkiv.1.mama.cmtx}
