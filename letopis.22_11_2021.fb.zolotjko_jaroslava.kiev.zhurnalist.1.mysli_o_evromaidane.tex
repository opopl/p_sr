% vim: keymap=russian-jcukenwin
%%beginhead 
 
%%file 22_11_2021.fb.zolotjko_jaroslava.kiev.zhurnalist.1.mysli_o_evromaidane
%%parent 22_11_2021
 
%%url https://www.facebook.com/yaroslava1411/posts/1764408520412332
 
%%author_id zolotjko_jaroslava.kiev.zhurnalist
%%date 
 
%%tags maidan2,ukraina
%%title Пару мыслей о Евромайдане...
 
%%endhead 
 
\subsection{Пару мыслей о Евромайдане...}
\label{sec:22_11_2021.fb.zolotjko_jaroslava.kiev.zhurnalist.1.mysli_o_evromaidane}
 
\Purl{https://www.facebook.com/yaroslava1411/posts/1764408520412332}
\ifcmt
 author_begin
   author_id zolotjko_jaroslava.kiev.zhurnalist
 author_end
\fi

Пару мыслей о Евромайдане... Я понимаю, что многие не разделяют моего мнения, и
возможно, кто-то расфрендится, но, как говорится, всем подряд нравятся только
посты с котятами...

\ifcmt
  pic https://scontent-frt3-1.xx.fbcdn.net/v/t39.30808-6/258015768_1764407200412464_7618664946526287324_n.jpg?_nc_cat=102&ccb=1-5&_nc_sid=730e14&_nc_ohc=vrmGAQZjJKgAX-qLbO2&_nc_ht=scontent-frt3-1.xx&oh=97fe140d3fc7a3970b70f35513fff61f&oe=61A6A6D2
  @width 0.7
  %@wrap \parpic[r]
  %@wrap \InsertBoxR{0}
\fi

Наблюдения за массовыми торжествами по поводу годовщины Революции Достоинства
меня натолкнули на мысль: ''а что мы отмечаем?''. Я могу понять, почему люди
вышли на Майдан, были надежды и мечты о европейском будущем. Но понятие
европейских ценностей, так же, как и достоинства, невозможно без определенного
уровня благосостояния, которое обеспечивает людям относительную свободу. Прошло
8 лет, но уровень жизни среднестатистического украинца не вырос.

Мы самое бедное государство в Европе, мало того, в отличие от времен
Януковича/Азарова у нас нет ресурсов даже для прохождения отопительного сезона
и уже начались веерные отключения электричества.  Что имеем - африканские
зарплаты и европейские тарифы. 

Реформы. На реформы выделяются огромные суммы, а сами реформы в основном
остаются на бумаге. Те, что и были проведены, не улучшили жизнь украинцев. А
ведь реформы проводятся именно для этого. В частности, медреформа под соусом
''европейской'', к примеру, только уменьшила доступ к медицине маломобильных
граждан. Вызвать так называемого семейного врача теперь невозможно даже
маломобильным старикам. Кстати, уже был случай, когда в Киеве на Оболони
бабушка в возрасте за 80 умерла по дороге в поликлинику. Врачей в крупных
городах не хватает катастрофически, запись на прием за 2 недели. Заболел - что
делать, иди к врачу, сиди в живой очереди. Вот и встречаются под кабинетом у
семейника пациент с подозрением на ковид, беременные, старики, здоровые за
сертификатами/справками и т.д.

Свобода взглядов. С каждым годом, я все чаще задумываюсь о том, что не хочу
жить в Украине - не вижу здесь будущего для своего ребенка. И дело не только в
экономике и медицине, а в повсеместном делении людей на ''касты'' по языку,
отношению к историческим событиям, вере, которого раньше не было. Если коротко,
то политика государства сводится к ''кто не скачет, тот москаль/рука
Кремля/сепар(нужное подчеркнуть)''. Теперь есть журналисты правильных СМИ и
неправильных, которые закрываются без суда и следствия, только по решению СНБО.

Символы. Не имею ничего против вышиванки, но я против принуждения, когда людям
говорят: ''Придешь на работу без вышиванки, не выпустим в эфир'', когда в
школах чмырят ребенка, если он пришел не в вышитой сорочке в День вышиванки. И
тут возникает вопрос: люди, которые так ненавидят совок, чем они поступают
лучше, чем партийные деятели тех времен? Как говорится, теми же методами...

Экономика. В 21 веке государство в центре Европы не имеет энергоресурсов для
прохождения отопительного сезона - на ТЭЦ нет даже минимального запаса угля,
веерные отключения электричества уже начались в регионах  и под Киевом. А
Кличко не исключает отключений и в самом Киеве. Пенсионеры с пенсией в 2 тыс.
грн получают счет за коммуналку в разы больше.

P.S. Сегодня мне грустно: если в первые годы еще была надежда.., то сейчас ее
уже не осталось. Я рада, что не была участником Евромайдана, а оставалась лишь
сторонним наблюдателем. Как говорится, не очаровывайтесь, чтобы не
разочаровываться.

\#Украина \#Евромайдан \#енергоресурси \#энергоресурсы \#экономика \#Euromaidan
\#економіка

\ii{22_11_2021.fb.zolotjko_jaroslava.kiev.zhurnalist.1.mysli_o_evromaidane.cmt}
