% vim: keymap=russian-jcukenwin
%%beginhead 
 
%%file 08_09_2021.fb.neborakova_bogdana.1.jazyk_institut_francia_kursy
%%parent 08_09_2021
 
%%url https://www.facebook.com/neborakova/posts/4382186261866214
 
%%author_id neborakova_bogdana
%%date 
 
%%tags francia,jazyk,mova,ukraina,ukrainizacia
%%title У 2021 році в Києві мене за замовчуванням хочуть навчати російською мовою. Серйозно?
 
%%endhead 
 
\subsection{У 2021 році в Києві мене за замовчуванням хочуть навчати російською мовою. Серйозно?}
\label{sec:08_09_2021.fb.neborakova_bogdana.1.jazyk_institut_francia_kursy}
 
\Purl{https://www.facebook.com/neborakova/posts/4382186261866214}
\ifcmt
 author_begin
   author_id neborakova_bogdana
 author_end
\fi

Цього літа я почала вчити французьку і вперше зрозуміла, про що говорив
Калинець фразою «Україна також і для українців».

Записалася в липні на курс-інтенсив А1.1 при Французький інститут в Україні /
Institut français d'Ukraine в Києві. Викладачка виявилася російськомовна. Я ж
відчула, як створюю дискомфорт своїм небажанням вникати в спряжение, глагольі,
падежи та інше наречие. Я не розумію, що це, не мала потреби розуміти і не можу
уявити цю потребу в майбутньому. Бажання слухати російську, якою не користуюся,
в мене теж немає.

\ifcmt
  ig https://scontent-mia3-1.xx.fbcdn.net/v/t1.6435-9/241638452_4382194605198713_1626581703092204065_n.jpg?_nc_cat=106&ccb=1-5&_nc_sid=730e14&_nc_ohc=8j34l0u6dgsAX_C0VHu&_nc_ht=scontent-mia3-1.xx&oh=2607386126f05b3908e089d9fb9653ea&oe=61847F05
  @width 0.4
  %@wrap \parpic[r]
  @wrap \InsertBoxR{0}
\fi

Я одразу спитала себе: що робити? Ну, думаю, якщо викладачка виявиться вкрай
сильною, потерплю 10 занять інтенсиву. Викладачка, на мою думку, не виявилася,
на 3 занятті ми вкотре отримали роздаткові матеріали з граматики російською — і
я написала до адміністрації курсів.

Перед тим читаю ч.7 ст.21 ЗУ «Про забезпечення функціонування української мови
як державної»:

7. Викладання іноземної мови в закладах освіти і на курсах з вивчення іноземних
мов здійснюється відповідною іноземною або державною мовою.

Думаю: ого, це ж у мене є механізм впливу навіть.

Але вирішую на закон не посилатися, скаргу не писати, потай заняття не
записувати і докласти зусиль для того, щоб усе пояснити делікатно. Мені
подобається ходити вулицею Гончара на живі заняття з мови, я сподіваюся ще зо
два-три роки це робити. Тому прошу перевести мене в іншу групу — через те, що
викладачка веде заняття французькою та російською (якої, як виявляється, я не
розумію), а не українською, плюс маю деякі інші зауваги.

Попри заповненість українськомовної групи, мене переводять, я доучуюся.
Приходить час осіннього семестру. Я вже навчена тим, що треба з’ясувати мову
викладача, пишу запит про це на курси, бо потрапити до точно україномовного не
можу, потрібна онлайн-група, мій же — тільки «в класі». 

Намагаюся з’ясувати єдине: хто з онлайн-викладачів вестиме заняття французькою
та українською, а не французькою та російською. Двічі отримую пропозицію ходити
до викладача, в якого я і була, офлайн. І врешті довідуюся, що серед чотирьох
груп дистанційного вивчення рівня А1.2 є одна викладачка, «яка може давати
пояснення українською».

Пишу у відповідь листа з таким текстом:

«Тобто, україномовні заняття можуть бути лише очними ("у класі") з NN?

Поясніть, будь ласка, що значить "може давати пояснення українською"? Мовою "за
замовчуванням" є російська?

Прошу зрозуміти мене правильно, я хочу потрапити до викладача, який не веде
занять російською і не поширює роздаткові чи пояснювальні матеріали російською,
і хочу зрозуміти максимально широке коло опцій.» 

І вже 12 днів не отримую на нього відповіді. 

Чи хочу я починати війну і змушувати саме цих людей вчити мене французької з
використанням української? Ні. Мова справді є сенситивною сферою, і осад я
отримала не найкращий для того, щоб наступні три роки тріумфувати, якщо
переконаю викладачів говорити українською, та ловити їхнє невдоволення від
цього. Це не те саме, що приготувати мені кави і просити розмовляти українською
чи купити сукню і просити розмовляти українською. Інша справа, що мова — це
стосунок, який має мінятися, щоб в Україні таких ситуацій ставало менше. Я бачу
в цьому дискримінацію україномовних українців, які не зобов’язані знати як
російську мову загалом, так і тим паче її галузеву термінологію.

PS. Я вирішила спробувати попрацювати дистанційно на курсах французького
альянсу у Львові. Водночас я б хотіла, щоб ситуація змінювалася — а
україномовна особа не відчувала себе нав’язливою надокучливою проблемою для
мовної школи у столиці України в 2021 році. 

Відчуваю зараз дві емоції: сором за себе, що активніше і одразу не висловила
своє ставлення і говорила завуальовано, не пояснюючи те, що існує закон.
Думала, що це надто жорстко. Водночас відчуваю непристойність цієї ситуації. У
2021 році в Києві мене за замовчуванням хочуть навчати російською мовою.
Серйозно?

\ii{08_09_2021.fb.neborakova_bogdana.1.jazyk_institut_francia_kursy.cmt}
