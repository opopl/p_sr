% vim: keymap=russian-jcukenwin
%%beginhead 
 
%%file 06_09_2021.fb.borodavko_andrej.1.druzhba_narodiv_mova
%%parent 06_09_2021
 
%%url https://www.facebook.com/bo.contact/posts/2368450146622136
 
%%author_id borodavko_andrej
%%date 
 
%%tags druzhba,druzhba_narodov,harkov,jazyk,kiev,moskovia,mova,narod,rossia,taksi,taksist,uzbekistan
%%title Отака от совєцька й московитська дружба народів
 
%%endhead 
 
\subsection{Отака от совєцька й московитська дружба народів}
\label{sec:06_09_2021.fb.borodavko_andrej.1.druzhba_narodiv_mova}
 
\Purl{https://www.facebook.com/bo.contact/posts/2368450146622136}
\ifcmt
 author_begin
   author_id borodavko_andrej
 author_end
\fi

Віз нас із дружиною (Київ) в суботу зранку таксист років 40 з чимось, який із
дивною вимовою руского язика з якогось дива сказав про нашу чудову літературну
мову.

На що спитав, звідки він до нас прихав. "Я тут жіву", - почув у відповідь.

Врешті-решт з'ясувалося, що він приїхав з Харкова, громадянин України, а вимова
- то нашарувалися фарсі з Узбекистану (не жарт), звідки він походить, із
акцентами походженням з московії, де він тривалий час жив.

Цей рецензент якості української мови щось мені про те, яка дружба народів була
у совку, як всі були одним народом.

Спитав прямо, чи називали його в обличчя за совка або в московії оті добрі люди
"4\#ркою" та/чи "40rno3@д-м"?

"Да, - життєрадісно й на позитиві відповів таксист, - і так, і так називали!
Але ж я був своїм 4-ою! І всі люди жили дружно".

Отака от совєцька й московитська дружба народів.

\#київськетаксі

\ii{06_09_2021.fb.borodavko_andrej.1.druzhba_narodiv_mova.cmt}
