%%beginhead 
 
%%file 29_08_2022.fb.stanislavskyj_ivan.mariupol.fotograf.zhurnalist.1.u_mene_na_stol__test
%%parent 29_08_2022
 
%%url https://www.facebook.com/I.Stanislavsky/posts/pfbid033NAHL1CwRTErU2emCCyy8Lwq2YbgLQpbAUd8zjZSPN6pXJ2onD5B8h2YP9EFUs8tl
 
%%author_id stanislavskyj_ivan.mariupol.fotograf.zhurnalist
%%date 29_08_2022
 
%%tags mariupol,mariupol.pre_war,mozaika,isskustvo,monumentalizm
%%title У мене на столі тестовий видрук фотопутівника "Маріуполь монументальний"
 
%%endhead 

\subsection{У мене на столі тестовий видрук фотопутівника "Маріуполь монументальний"}
\label{sec:29_08_2022.fb.stanislavskyj_ivan.mariupol.fotograf.zhurnalist.1.u_mene_na_stol__test}

\Purl{https://www.facebook.com/I.Stanislavsky/posts/pfbid033NAHL1CwRTErU2emCCyy8Lwq2YbgLQpbAUd8zjZSPN6pXJ2onD5B8h2YP9EFUs8tl}
\ifcmt
 author_begin
   author_id stanislavskyj_ivan.mariupol.fotograf.zhurnalist
 author_end
\fi

У мене на столі тестовий видрук фотопутівника \enquote{Маріуполь монументальний}. Ця
робота, напевно, закриє тему збереження даних про спадок маріупольських
митців-монументалістів, принаймні для мене. Тут дійсно зібрано все, про що мені
вдалося дізнатися. Інформацію для фотопутівника я почав збирати навесні 2020
року, а навесні 2022 вже був готовий його друкувати в Маріуполі. Але так
сталося, що він виходить восени й у Львові. Проте для мене головне що він таки
виходить, що інформація, яку я так довго збирав, не залишиться у мене в столі.

Отже, Маріуполь Монументальний це:

- 128 сторінок

- 197 світлин

- 144 твори у 13 техніках (тільки мозаїк майже пів сотні)

- Тверда обкладинка та така ж гарна якість друку як і в \enquote{Усіх віддтінків}

Схема розповсюдження буде така сама як і з \enquote{Усіма відтінками}, лишаєте заявку в
коментарях під цим постом - отримуєте фідбек в пп.

Орієнтовна вартість книги 600 грн.

Перші примірники будуть готові у другій половині вересня.
