%%beginhead 
 
%%file 31_12_2022.fb.tomchuk_dmytro.1.mi_teper_khto_de__ta
%%parent 31_12_2022
 
%%url https://www.facebook.com/dmitriy.tomchuk/posts/pfbid02hPkXeRsEwhGVprUJdo6fDokcmncE8nB6irrXCS4f7oCFgoiNLaZ44VURCvJbHhojl
 
%%author_id tomchuk_dmytro
%%date 31_12_2022
 
%%tags 
%%title Ми тепер хто де. Та я часто тепер не певний, де я  сам та хто я є
 
%%endhead 

\subsection{Ми тепер хто де. Та я часто тепер не певний, де я  сам та хто я є}
\label{sec:31_12_2022.fb.tomchuk_dmytro.1.mi_teper_khto_de__ta}

\Purl{https://www.facebook.com/dmitriy.tomchuk/posts/pfbid02hPkXeRsEwhGVprUJdo6fDokcmncE8nB6irrXCS4f7oCFgoiNLaZ44VURCvJbHhojl}
\ifcmt
 author_begin
   author_id tomchuk_dmytro
 author_end
\fi

Ми тепер хто де. Та я часто тепер не певний, де я  сам та хто я є. Це через те,
що рік, що минає, розкидав нас світом та розкидав наші світи. Але ми є. Десь,
якось, за межею зрозумілого, але ми є. 

Дякувати Богові, здебільшого особисто в  мене люди живі та цілі. Умовно.
Найбільший крінж цього року - слухати, як вбиваються щодо нашого життя знайомі
за кордоном. Чомусь усе, що з нами коїться, їм здається дуже страшним. Й це
найсмішніше, бо вони там, а  ми тут. Та все хочеться мені відповісти їм: "Та не
вбивайтеся ви так, ви так все рівно не вб'єтеся". Але це буде нечемно. Тому я
мовчу. Щодо нас - ми тут ну геть нічого страшного у нашому житті не бачимо. Але
мені все важче описати комусь з них, як насправді ми тут живемо. Бо наші світи
розходяться все більше. Ця прірва більшатиме, та вороття немає. Але можна
співпрацювати й так, це не головне.

А головне от що. Щоб ті, хто в містах, зустріли новий рік з електричним світлом
та у себе вдома, а не у сховищах. Щоб ті хто в окопах - хай вони зустрінуть
так, як в них вийде, хочу тільки, щоб це була просто тиха над ними ніч. Нехай
буде тиша та зірки. Щоб ті, хто за кордоном... Я не можу уявити, як вони
зустрічатимуть. Можливо, просто пригорнуть дітей, що заснули в них на колінах.
Я тільки сподіваюся, що вони не плакатимуть у цю ніч.

Мої побратими проведуть цю ніч на нулі. Тисячі військових, з якими я пересікся
у цьому році, всі їхні обличчя в мене перед очима. Багатьох вже нема тут, але я
сподіваюся, що вони святкуватимуть теж - десь там. Мої колеги, співробітники,
партнери, інвестори - про деяких я навіть тепер не знаю нічого. Я завжди писав
їм вітання з новим роком, але цього разу не знаю, що їм сказати. Мої читачі -
вони тримали у цьому році нас на собі, надсилаючи нам добрі слова,
благословення та гроші.  

За звичкою я думаю, як три сотні разів до того: "Пережити цю ніч". Та раптом
розумію що це я  - про новорічну ніч.

Але я сподіваюся, що 2023 принесе нам щастя. Виявляється, ми прожили життя, не
знаючи, яке воно - щастя.  Але тепер кожний з нас знає, яким саме воно буде, та
що має відбутися, щоб бути щасливим. Нехай це трапиться у 2023.

Ми прожили 2022, й ми перемогли. Ми живі, хоча були всі умови для смерті. Це
означає, що ми перемогли, й це перемога життя над смертю. Нашого життя над
нашою смертю. Бог дасть, у 2023 прийде день, коли нам не доведеться
розмірковувати більш над тим, чи будемо ми живі завтра. Це єдине моє бажання на
Новий рік - щоб цей день прийшов саме у 2023 для кожного з нас. З нього ми
почнемо нове життя, як збиралися багато разів, та нічого не виходило, бо
відкладали на понеділок. Цього разу не відкладатимемо.

Це буде щасливий рік, не чомусь там, а тому, що ми повною мірою проплатили у
2022 щастя не тільки на 2023, але на сто років вперед. 

Я вітаю всіх, хто читає ці слова, з Новим роком. Радий,що ми всі живі.
Сподіваюся, що всі ми житимемо довго та з легким серцем. Я хочу жити з легким
серцем, бо весь 2022 я з легким серцем воював, виконуючи заповіт Скаженого Пса
Джеймса Меттіса. А тепер хочу просто жити. Та хочу, щоб жили всі ми. Саме таку
задачу я ставлю 2023.

Нехай у 2023 ми принесемо світові мир.

З Новим роком!
