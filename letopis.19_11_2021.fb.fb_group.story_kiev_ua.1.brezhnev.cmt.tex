% vim: keymap=russian-jcukenwin
%%beginhead 
 
%%file 19_11_2021.fb.fb_group.story_kiev_ua.1.brezhnev.cmt
%%parent 19_11_2021.fb.fb_group.story_kiev_ua.1.brezhnev
 
%%url 
 
%%author_id 
%%date 
 
%%tags 
%%title 
 
%%endhead 
\zzSecCmt

\begin{itemize} % {
\iusr{Андрей Павлов}
Настоящий мужик! Любил охоту, машины, женщин, застолья. Другим давал жить.

\begin{itemize} % {
\iusr{Sergej Grishakov}
\textbf{Андрей Павлов} ..члены только и жили...

\iusr{Виктор Задворнов}
\textbf{Sergej Grishakov} а дочка-то лучше всех
\end{itemize} % }

\iusr{Анна Николаевна}
Светлая память Леониду Ильичу Брежнєву  @igg{fbicon.hands.pray} 

\begin{itemize} % {
\iusr{Igor Neronov}
\textbf{Анна Николаевна} Нам солнца не надо, нам Партия светит, нам хлеба не надо- Работу давай!!!)))

\iusr{Igor Neronov}
\textbf{Анна Николаевна} 

Эй, любители, а особенно любительницы эпохи Брежнева, ну- ка, вспомните, чем Вы
пользовались вместо прокладок, как десятками раз штопали колготки и носили их
под брюки, какого качества мышьяк закладывали Вам в пломбы, синие куры в пайке
и обязательно 300 грамм в додачу костей на кило замороженного мяса, родителей,
горбатившихся всю жизнь на квартиру и проживание всей семьей с бабушками и
дедушками на 50 метрах? Вспомнили??? Будь проклято и забыто это время!!!!!!

\begin{itemize} % {
\iusr{Арт Юрковская}
\textbf{Igor Neronov} 

не смешите. Сейчас ползарплаты за еду надо отдать, в больницу лучше не попадать
,а то никто без денег лечить не будет. За тепло не может пол Украины
заплатить. Чтобы квартиру купить - надо взять кредит, если еще дадут, а если не
платишь - и квартиру потеряешь и всем на тебя начхать. Когда слышат, что тебе
56-57 лет - на работу уже не возьмут. Пенсии отменилив будущем вообщ, про
8-часовый день с двумя выходными можно забыть. Вот это время - будь проклято! А
советским есть, что вспомнить и сравнить.


\iusr{Igor Neronov}
Так, работайте!!! Сидя жопой на диване, будете голодать в любое время!!

\iusr{Светлана Манилова}
\textbf{Igor Neronov}, Вы сильно не расходитесь, пожалуйста. Комментарии не должны нарушать правил группы...

\iusr{Natasha Levitskaya}
\textbf{Igor Neronov}

Вам уже делала замечание за грубость, но комментарий ваш удалили. Видимо, вы не
успели прочитать. Ещё раз вам предупреждение - лексику измените в комментариях!


\iusr{Igor Neronov}
\textbf{Natasha Levitskaya} ок, но очень затруднительно ))

\iusr{Natasha Levitskaya}
\textbf{Igor Neronov}

Здесь не ваша личная территория. И не надо устраивать диванные \enquote{бои без
правил}, даже если вам и затруднительно. В группе более 100 тыс. участников.


\iusr{Catherine Murashchyk}
\textbf{Igor Neronov} особенно войну в Афганистане вспомните.

\iusr{Igor Neronov}
\textbf{Vit Naz} кто на кого учился ...)))

\end{itemize} % }

\iusr{Алексей Назаров}

Войну в Афганистане вспомню я. Потому, что на неё загнали моего отца. Путём
тупого шантажа. Когда подошла его очередь на квартиру, ему сказали: или в
Афганистан, или забудь о квартире. Прапощику без жилья, у которого жена и двое
детей. И вернулся он оттуда с хорошим таким посттравматическим синдромом.
Хорошо, хоть вернулся относительно целым. Уже одного этого хватило,чтобы навеки
проклясть эту сучью власть, маразмирующего бровеносца и его наследников. А у
меня за время моей жизни при ней накопилось неплохое количество счетов к ней.


\iusr{Янина Ерошкина}
\textbf{Igor Neronov} А ты работаешь? @igg{fbicon.wink}  @igg{fbicon.face.tears.of.joy} 

\begin{itemize} % {
\iusr{Igor Neronov}
\textbf{Янина Ерошкина} и день и ночь..)

\iusr{Янина Ерошкина}
\textbf{Igor Neronov} На невидимом фронте? @igg{fbicon.face.tears.of.joy}  @igg{fbicon.laugh.rolling.floor} 

\iusr{Igor Neronov}
\textbf{Янина Ерошкина} наша Служба и опасна и трудна, и на первый взгляд как - будто не видна...
\end{itemize} % }

\end{itemize} % }

\iusr{Григорий Владимирович}

2 раза в Борисполе видел на расстоянии около 20 метров, потом с Олейником и
Недригайлом сопровождали в Залесье.

\begin{itemize} % {
\iusr{Катя Шерман}
Вы прикоснулись к вечности  @igg{fbicon.smile}  Григорий Владимирович

\iusr{Григорий Владимирович}
\textbf{Катя Шерман} Хорошо когда есть что вспомнить хорошее.
\end{itemize} % }

\iusr{Max Gopencko}
До указа «Об усилении борьбы с пьянством и алкоголизмом» оставалось 3,5 года....

\iusr{Николай Бурчин}
А сигареты с ментолом в те годы были в продаже? @igg{fbicon.face.nerd} 

\begin{itemize} % {
\iusr{Igor Neronov}
\textbf{Николай Бурчин} Нет, только газеты \enquote{Правда} и
\enquote{Известия} разрезанные на четвертушки, вместо туалетной бумаги, лежали
в сортире...))

\begin{itemize} % {
\iusr{Николай Бурчин}
\textbf{Igor Neronov} жаль, жаль.. у нас была бумага  @igg{fbicon.smile} 

\iusr{Igor Neronov}
\textbf{Николай Бурчин} Бумага -то была, но газетку подкладывали в унитаз, чтобы фекалии не пачкали дорогущий чешский, а кому повезет, финский фаянс..))))

\iusr{Sergej Grishakov}
\textbf{Igor Neronov} \enquote{Правды} нет, остался \enquote{Труд} в неограниченном колличестве.

\iusr{Надежда Владимир Федько}
\textbf{Igor Neronov} 

Ми жили в комунальній квартирі. Коли мені було 5 років, то в мої обов'язки
входило рвати на шматки газети для туалету. При цьому була категорична
настанова - портрети \enquote{вождів} виривати і класти в стопку окремо. Одного разу я
застав сусідку за знищенням портретів - вона рвала їх на дрібні клаптики і
кидала у відро зі сміттям, яке потім виносилося у дворовий сміттєзбірник.

\end{itemize} % }

\iusr{Виктор Анисков}
\textbf{Николай Бурчин}
Были  @igg{fbicon.wink}  Я в 77-78 курил Данхил )

\begin{itemize} % {
\iusr{Николай Бурчин}
\textbf{Виктор Анисков} круто! добывали как-то или продавался в табачных киосках?

\iusr{Виктор Анисков}
\textbf{Николай Бурчин}
Бывший сотрудник предложил купить. Я и иногда ему заказывал )
\end{itemize} % }

\iusr{Татьяна Ховрич}
\textbf{Николай Бурчин}, 

появились в период Олимпиады-80. Потом не исчезали (Sallem, New Port). Правда,
стоили намного дороже, чем наш \enquote{Космос}. Если не ошибаюсь, по 1.5 руб. А может,
ошибаюсь в цене.

\begin{itemize} % {
\iusr{Illya Davydov}
\textbf{Tatyana Hovrych} а вот если простые Столичные прокапать корвалолом,то будут ментоловые

\iusr{Татьяна Ховрич}
\textbf{Illya Davydov}, небольшая поправочка: жидким валидолом из круглой капсулки.

\iusr{Ирина Левина}
\textbf{Татьяна Ховрич} \enquote{MORE} - 110мм (зелененькая пачка) продавались повсеместно, а вот Данхил привозили моряки из загранки))

\ifcmt
  tab_begin cols=4,no_fig,center

     pic https://scontent-frx5-1.xx.fbcdn.net/v/t39.30808-6/258336884_972076243379545_7103405831031703650_n.jpg?_nc_cat=111&ccb=1-5&_nc_sid=dbeb18&_nc_ohc=nO2hp_QJlM8AX8LiGEd&_nc_ht=scontent-frx5-1.xx&oh=00_AT_SSi7MA2NS3XJI2kyt9Z8IeELa36WvsN3PYnhm3C4RiA&oe=61C95645

		 pic https://scontent-frt3-1.xx.fbcdn.net/v/t39.30808-6/257406640_972076786712824_9094888357543117531_n.jpg?_nc_cat=106&ccb=1-5&_nc_sid=dbeb18&_nc_ohc=oqy6h5YwG0oAX90Ats2&_nc_ht=scontent-frt3-1.xx&oh=00_AT_rnLtPQELHwE7ObdJ8AK60o-7q4UvZ6NFodstON-c9ZA&oe=61C95879

		 pic https://scontent-frt3-2.xx.fbcdn.net/v/t39.30808-6/258775559_972077903379379_2210019873837989737_n.jpg?_nc_cat=103&ccb=1-5&_nc_sid=dbeb18&_nc_ohc=zF6D4a7dawkAX-Sqz1Z&_nc_ht=scontent-frt3-2.xx&oh=00_AT8D7F7bj9XCpC69UFpBeaR5YTzjJ55pHYt8mewr4HwQaw&oe=61C91D68

		 pic https://scontent-frx5-2.xx.fbcdn.net/v/t39.30808-6/258751539_972078156712687_5108058481969474308_n.jpg?_nc_cat=109&ccb=1-5&_nc_sid=dbeb18&_nc_ohc=FTuDj6aKhpoAX_jirzs&_nc_ht=scontent-frx5-2.xx&oh=00_AT9NKKfueKoTECbeHusBV2TNxWPgaVOw8fTBPrNUiV_pOw&oe=61C8D56C

  tab_end
\fi

\iusr{Татьяна Ховрич}
\textbf{Ирина Левина}, 

да, именно. Баловали себя в дни \enquote{студенческой ЗП}: покупали в каждую стипендию
разные импортные сигареты. Но зелёные \enquote{Мore} - самые любимые: во-первых,
ментоловые, во-вторых, длинненькие, долго курятся. @igg{fbicon.face.wink.tongue} 


\iusr{Надежда Владимир Федько}
\textbf{Татьяна Ховрич} Дякую пані Тетяно за чудову колекцію і інформацію!

\iusr{Татьяна Ховрич}
\textbf{Надежда Владимир Федько} , колекція - не моя, а Ірини Левіної.

\iusr{Надежда Владимир Федько}
\textbf{Татьяна Ховрич} Моя подяка пані Ірині))

\iusr{Надежда Владимир Федько}
\textbf{Ирина Левина} Дякую за колекцію і підтримку))

\iusr{Татьяна Ховрич}
Ирина, Вы - коллекционеруете?

\iusr{Ирина Левина}
\textbf{Татьяна Ховрич} нет, но люблю путешествовать по интернету во времени и пространстве )))

\end{itemize} % }

\iusr{Alexander Nejuvoj}
\textbf{Николай Бурчин} Были наши помню год 65-й. \enquote{Пчёлка} назывались.

\begin{itemize} % {
\iusr{Аркадий Израилевский}
\textbf{Alexander Nejuvoj} \enquote{Пчёлка} были ароматизированые, а не ментоловые.

\iusr{Alexander Nejuvoj}
\textbf{Аркадий Израилевский} Я был маленький и ещё не курил. Но пахли они вкусно. Ароматизированные или с ментолом,-это уже технология???
\end{itemize} % }

\iusr{Николай Бурчин}

Я совсем не помню ментоловые сигареты в СССР, наверное таки не очень часто
встречались или не пользовались особой популярностью. В начале 90-х помню пошла
мода на них. Помню длинные черные More...  @igg{fbicon.smile} 

\iusr{Нина Светличная}
\textbf{Николай Бурчин} Моя приятелька \enquote{робила} будь-які сигарети ментоловими. На сигарету наносила трішки бальзаму \enquote{Звездочка})

\iusr{Igor Neronov}
Не так.. правды- нет, известия- закончились, остался труд за три копейки..)

\iusr{Виталий Колычев}
Явские \enquote{золотое руно} и 100мм тоже были \enquote{золотое руно}

\iusr{Igor Popell}
\textbf{Mykola Burchin} я демобилизовался в ноябре 1981г. В продаже какое-то время были
финские сигареты. Видимо, остатки поставок на олимпиаду 80г. Помню, что иногда
покупал ментоловый \enquote{Newport} (или \enquote{Salem}?) за 1,50р. или за 1р. Были еще
Marlboro и Bond.

\begin{itemize} % {
\iusr{Irena Visochan}
\textbf{Igor Popell} 

Какая-то финская водка была? с клюквой? или ликёр? И, конечно, а як жеж, без
сигаретки с ментолом? (в сумочке, Salem,, Да, это после олимпиадные остатки
роскоши.)


\iusr{Igor Popell}
\textbf{Irena Visochan} водку не застал. Видимо, всю выпили до моего дембеля ))) А вот Fanta была. Жвачки Kalev появились.

\iusr{Надежда Владимир Федько}
\textbf{Irena Visochan} Фінську горілку з клюквою пам'ятаю...

\iusr{Irena Visochan}
\textbf{Igor Popell} 

Проконсультировалась с главным спецом, с моим Игорем. Так вот-' финской водки
не было, а был финский клюквенный ликёр.


\iusr{Irena Visochan}
\textbf{Надежда Владимир Федько} мой муж сказал, что это был ликер, финский клюквенный,

\iusr{Igor Popell}
\textbf{Irena Visochan} профессионализм не убьешь!

\iusr{Irena Visochan}
\textbf{Igor Popell}  @igg{fbicon.hearts.two} 

\iusr{Надежда Владимир Федько}
\textbf{Irena Visochan} У Вашого чоловіка чудова пам'ять!

\iusr{Irena Visochan}
\textbf{Надежда Владимир Федько} Да. Надюша, спасибо!

\iusr{Надежда Владимир Федько}
\textbf{Irena Visochan} Надійки вже нема! Сторінку веду я, її чоловік, Володимир.
\end{itemize} % }

\iusr{Олег Дурбалов}
Были, ST Moritz

\iusr{Юрий Мисик}
\textbf{Николай Бурчин} да были

\iusr{Catherine Murashchyk}
\textbf{Mykola Burchin} были! Сама курила

\iusr{Ivan Tsurkan}
\textbf{Николай Бурчин} в магазинах «Березка», «Каштан»

\iusr{Лариса Павловская}
\textbf{Mykola Burchin} В киоске на углу Прорезной(Свердлова тогда) и Крещатика продвали импортные сигареты, помню болгарские

\iusr{Татьяна Шаповалова}
\textbf{Николай Бурчин} Нет. Но мы их курили. Всё можно было \enquote{достать}.

\iusr{Иванна Кудрина}
\textbf{Николай Бурчин}
Были... в Москве точно... More, в зелёной упаковке - ментоловые.

\iusr{Gennady Henry Sergienko}
\textbf{Mykola Burchin} Кто помнит \enquote{Запашнi} - нашу альтернативу ментолу?  @igg{fbicon.smile} 

\begin{itemize} % {
\iusr{Генадий Пинский}
\textbf{Gennady Henry Sergienko} Несколько не хуже.

\iusr{Gennady Henry Sergienko}
\textbf{Генадий Пинский} Любимые в школьные годы  @igg{fbicon.smile} 

\iusr{Ирина Левина}
\textbf{Gennady Henry Sergienko} Запашнi (ГОСТ-58, Прилуксткая ТФ) @igg{fbicon.wink} 

\ifcmt
  ig https://scontent-frt3-1.xx.fbcdn.net/v/t39.30808-6/259557211_972132860040550_5959224906510151616_n.jpg?_nc_cat=106&ccb=1-5&_nc_sid=dbeb18&_nc_ohc=VFC1bixzbK8AX9XQ1qQ&_nc_ht=scontent-frt3-1.xx&oh=00_AT92NB4IZozJu8Lvc6ElHZ7iGK2j9RD75SMN3m4XouCJqw&oe=61CA1DAA
  @width 0.4
\fi

\end{itemize} % }

\iusr{Геннадий Дудко}
\textbf{Николай Бурчин} склоне всего это были остатки с Олимпиады 80. Цена была 1,50 р. Да если хотелось такие покурить, то ,, достать,, можно было без проблем.

\end{itemize} % }

\iusr{Sergej Grishakov}
Видел \enquote{настоящего мужика} на расстоянии вытянутой руки в Киеве на проспекте Воссоединения во время приезда Никсона.
Народу \enquote{выгнали} тогда на улицы-немерянно.

\begin{itemize} % {
\iusr{Tanya Podkolzina}
\textbf{Sergej Grishakov} , ага. Тоже там стояли, не по доброй воле)

\iusr{Анна Николаевна}
\textbf{Tanya Podkolzina} 

Я тоже видела Леонида Ильича и Щербицкого на привокзальной площади, когда он
приезжал открьівать знаменитьій монумент Родина- мать. Я бьіла молодой, мне
бьіло прикольно и интересно. А потом мьі пошли с друзьями вьіпить по чашечке
кофе с пироженьім Картошка. Бьіло вкусно и запомнилось.

\end{itemize} % }

\iusr{Валерій Чистяков}
За здоровье Ильича - накатили сгоряча!

\begin{itemize} % {
\iusr{Luda Draganova}
\textbf{Валерій Чистяков} ...Накатили сгоряча - получили строгача. Вот такие именины Леонида Ильича!
\end{itemize} % }

\iusr{Светлана Дубински}
Лучшие времена переживала страна.

\begin{itemize} % {
\iusr{Надежда Владимир Федько}
\textbf{Светлана Дубински} Я так не вважаю!

\begin{itemize} % {
\iusr{Светлана Дубински}
\textbf{Надежда Владимир Федько} Все конечно условно. Не сажали и не убивали. Вот в этом \enquote{большая заслуга} этих времен

\iusr{Надежда Владимир Федько}
\textbf{Светлана Дубински} 

Угу. У 1972 пройшла перша хвиля арештів і посадок української інтелігенції.
Напередодні Олімпійських ігор у Москві (1980) друга \enquote{зачистка}...

В 1974-му Солженіцина викинули з СРСР. Така ж доля і Віктора Некрасова.
Лупиноса, Плюща і Григоренко закрили в психіатричну лікарню... Буковський
відсидів, а потім його обміняли на Луїса Корвалана. І це тільки окремі особи з
величезного списку репресованих за політичні погляди.

\iusr{Zoya Sharykova}
\textbf{Надежда Владимир Федько} 

А більшості людей було так зручно не замислюватись над тим... З‘явилися «фантики»
у вигляді Фанти, сигарет (не палю, тому їх не перелічую), копченої ковбаски,
яку частіше стали «викидати» у зв‘язку з Олімпіадою - от і здалося, що щасливі
роки... Прізвища тих, кого ув’язнювали, забороняли навіть вимовляти. Запитати б
про ті «щасливі» роки у їхніх родин...


\iusr{Надежда Владимир Федько}
\textbf{Zoya Sharykova} 

Запитати треба також у родин тих 15 000 солдат, що полегли в Афганістані. І тих
десятків тисяч, що залишилися інвалідами.


\iusr{Zoya Sharykova}
\textbf{Надежда Владимир Федько} 

Так, це взагалі окрема сторінка, цей мартиролог на їхній совісті... Мені
доводилось зустрічатися з хлопцями-афганцями, писати про них (я журналіст), і я
ті їхні рани, фізичні й душевні, бачила зблизька. Скількох уже немає, а ті, хто
є, так і не позбулися синдрому... За що їхні життя поламані? Я особисто ще можу
шукати і знаходити приводи для оптимізму, а вони так і не навчились... Тих, хто
зумів спекулювати на цій темі, не маю на увазі, війна відкривала й підленькі
душі, але вони в меншості.

\iusr{Надежда Владимир Федько}
\textbf{Zoya Sharykova} 

Згідно довідника \enquote{Военно-политическая спецоперация СССР в Афганистане}
(2008) безповоротні втрати - 14 453, з яких 1739 покінчили життя самогубством!

Загальні санітарні втрати 469 685 осіб. (стор. 215)

\iusr{Надежда Владимир Федько}
\textbf{Zoya Sharykova} Як журналісту можу порекомендувати Вам такий довідник:

С.В. Червонопиский, А.А. Костыря, В.Г. Сироштан. Военно-политическая операция
СССР в афганистане (25 декабря 1979 - 15 февраля 1989 гг.). Словарь-справочник.
2-е изд. переработанное и дополненное. Киев - 2008.

\ifcmt
  ig https://scontent-frx5-1.xx.fbcdn.net/v/t39.30808-6/259683616_4716323241760430_7488078940293679502_n.jpg?_nc_cat=110&ccb=1-5&_nc_sid=dbeb18&_nc_ohc=WB6BReHzdQYAX-h4j7q&_nc_ht=scontent-frx5-1.xx&oh=00_AT88Zqyi1zAxY2gtmYzYfO_GdvBl4o_rg_05sZlvDcIB8Q&oe=61C89D55
  @width 0.4
\fi

\iusr{Zoya Sharykova}
\textbf{Надежда Владимир Федько} 

Дякую! Червонописького знаю дуже мало й офіційно (після двох заходів), а Володю
Сіроштана добре й особисто, правда, давно не спілкувались. Знайду довідник,
хоча вже давно про це не пишу. Скажу Вам відверто про те, про що мало кому
говорила: коли сталось оте путлерівське вторгнення до нас під істеричним
девізом «спасти русский народ», якось воно печально нагадало те радянське
бажання «помочь афганскому народу»... Тільки тепер ми помінялися місцями, ми їх
не запрошували і зараз змушені відбиватися від «порятунку»... Якось нагадує
кармічну відповідь... Але ж ні, ту сусідню країну ця карма не наздогнала. Хіба що
все попереду...

\iusr{Надежда Владимир Федько}
\textbf{Zoya Sharykova} 

Форму покарання обирає Господь! Наша проблема - це \enquote{внутрішня еміграція},
тобто, люди, які живуть в Україні, але вважають, що повинні жити в Росії.

\iusr{Олег Курилов}
\textbf{Надежда Владимир Федько} 

число погибших в Афганистане преувеличено немного. Вы путаете военные потери и
общие. Самоубийц никогда не вносили в военные потери, даже сейчас. Ввод войск в
Афганистан был ошибкой СССР, но обвинять его за это... это геополитика, и не
важно какой строй был тогда в СССР, это зависило от множества разных
факторов.... Скажем так если страна входит в тройку-пятёрку ведущих стран. Она
обязана поступать так или перестанет быть ведущей страной.

\iusr{Надежда Владимир Федько}
\textbf{Олег Курилов} Рекомендую відкрити на с. 215 і прочитати! Всі претензії до авторів довідника!

\ifcmt
  ig https://scontent-frx5-1.xx.fbcdn.net/v/t39.30808-6/259683616_4716323241760430_7488078940293679502_n.jpg?_nc_cat=110&ccb=1-5&_nc_sid=dbeb18&_nc_ohc=WB6BReHzdQYAX-h4j7q&_nc_ht=scontent-frx5-1.xx&oh=00_AT88Zqyi1zAxY2gtmYzYfO_GdvBl4o_rg_05sZlvDcIB8Q&oe=61C89D55
  @width 0.4
\fi

\iusr{Олег Курилов}
\textbf{Надежда Владимир Федько} 

спасибо за рекомендацию, но в сети не смог найти это издание. Поискал в других
источниках. В принципе цифры приводятся похожие на Ваши ( те, что вы привели).
В общие санитарные потери входят заболевшие разными инфекционными
заболеваниями.

\end{itemize} % }

\iusr{Нина Светличная}
\textbf{Светлана Дубински} 

І де ж ці часи \enquote{переживались}? У таборах, де сидів Стус? Чи у
психлікарнях, куди засаджували незгодних?

\iusr{Надежда Калинская}
\textbf{Светлана Дубински} \enquote{Лучшие}, але переживала, виживала!

\end{itemize} % }

\iusr{Светлана Мороз}
кто там не жил не поймет

\iusr{Тарас Єрмашов}

Хм... а де співробітники \enquote{Укрхудожпрому} у 1981 році брали для щоденних
буденних перекурів цигарки з ментолом і ароматизований тютюн для люльки? В
\enquote{Гавані} хіба кубинські сигари й цигарки були з імпорту, та ще якісь єгипетські
в кіосках. Більш-менш у вільному продажу в Києві асортимент (\enquote{Бонд стріт},
\enquote{ЛД}, \enquote{Президент}, ментолові \enquote{Імпала} тощо) з'явився з 1988.

\begin{itemize} % {
\iusr{Надежда Владимир Федько}
\textbf{Тарас Єрмашов} \enquote{Чорний ринок}... У фарцовщиків...Здається (сам не курю, тому точно не знаю) імпортні сигарети і тютюн для трубки продавалися в \enquote{Каштані} на чеки.

\begin{itemize} % {
\iusr{Анна Николаевна}
\textbf{Nadegda Volodymyr Fedko} у фарцовщиків все було в рази дорожче. А значить не погано жили.

\iusr{Тарас Єрмашов}
\textbf{Надежда Владимир Федько} 

На щодень, для робочих перекурів, по їх цінам? )

Якщо й викроювали на \enquote{Мальборо} з куцої зарплатні (навряд чи в
художників-оформлювачів вона, як і забезпечення, рівнялася з обкомівською), то
курили під урочисті події, щоб підвищити соц. статус і т.п.

Я ж кажу: у 1988-90 рр. так могли вже в туалеті курити (хоч імпортні цигарки
були дорогими порівняно з вітчизняними й солгарськими - 4-5 крб. - але цілком
доступними), в 1981 - ніяк (хіба що якась корпоративна подія була - вечірка, як
в \enquote{Службовому романі}, зустріч Нового року та под. Та й то цей імпорт був у
одного-двох, і той пригощав інших, зображаючи симулякр буржуазного шику, як
Еллочка-Людожерка косплеїла Вандербільдиху).

\iusr{Надежда Владимир Федько}
\textbf{Анна Николаевна} Дивлячись з чим порівнювати!

Десь із рік тому назад була у мене дискусія з одною жінкою, ностальгуючою по
СССР, яка й досі не сприймає незалежності України. Їй 54 роки. Я запропонував
їй поділити аркуш паперу навпіл... На лівій стороні написати її зарплатню, коли
їй було 24 роки, що вона мала на той час і що вона могла купити. А в правій
частині аркуша написати, як ц неї зараз зарплатня, що вона має і що може
купити.

Вона завзято почала писати... Але через декілька хвилин кинула і чесно
зізналася, що зараз вона має набагато більше і набагато більше може собі
купити.

Люди мого покоління плутають ностальгію за молодістю з ностальгією по життю в
СССР.

\iusr{Анна Николаевна}
\textbf{Nadegda Volodymyr Fedko} ні я не сумую за СРСР. Але це наша історія, наше життя. І в пам'яті залишилося багато чого хорошого.

\iusr{Надежда Владимир Федько}
\textbf{Анна Николаевна} 

Все залежить від того, скільки сигарет викурювати в день. Якщо пару штук за
кавою, то пачки вистачить на 10 днів. А купити три-чотири пачки на місяць не
такі вже й значні витрати.

\iusr{Анна Николаевна}
\textbf{Nadegda Volodymyr Fedko} погоджуюсь  @igg{fbicon.rose} 

\end{itemize} % }

\iusr{Сергій Авдєєв}
\textbf{Тарас Єрмашов} Друже, 1981: Столичні - 40 к., Salem (Finland) - здається 2 р. 20 к.:

\ifcmt
  ig https://scontent-frt3-1.xx.fbcdn.net/v/t39.30808-6/258872305_4673136336132936_4275509036836123223_n.jpg?_nc_cat=104&ccb=1-5&_nc_sid=dbeb18&_nc_ohc=NRbialgTvyoAX9l-9lc&_nc_ht=scontent-frt3-1.xx&oh=00_AT-vuWAlBxe704aI1vaI0plq6Apba6An8ng5WM-di7GACQ&oe=61C9C157
  @width 0.3
\fi

\begin{itemize} % {
\iusr{Тарас Єрмашов}
\textbf{Сергій Авдєєв} 

Їх пам'ятаю (не безпосередньо в 1981, звісно - я тоді стикався із значно
прозаїчнішими марками курива, вітчизняними \enquote{Прима} - 16 к., \enquote{Ватра} - 25 к.),
які ми, пуцьвіріньки, іноді тихцем куштували. Та історику реконструювати хід не
таких вже й давніх подій - справа неважка). Але 2,20 для тих часів і простолюду
(хоч і творчого) - все одно дорого, не на щодень. Не кажучи вже про
ароматизований тютюн для люльки (хоча в моєму оточенні ніколи не було жодного
дорослого, що курив би її).

\iusr{Сергій Авдєєв}
\textbf{Тарас Єрмашов} 

А для \enquote{форсу творчого}  @igg{fbicon.smile}  Звісно, що посполиті не могли собі дозволити такий
щоденний \enquote{аристократизм}, але певні \enquote{елітні} категорії \enquote{такі да}

\iusr{Надежда Владимир Федько}
\textbf{Тарас Єрмашов} А пляшка горілки за 2,87 чи за 3,12? А шампанське за 4, 17? А портвейн за 1,62? На цьому фоні цигарки не такі вже й дорогі...
\end{itemize} % }

\end{itemize} % }

\iusr{Татьяна Соловьева}

Через год, в этом же возрасте Леонида Ильича не станет(

\iusr{Тарас Нечепорук}

Будучи студентами проходили археологічну практику у закинутому селі на березі
Дністра. Вся моя група щось розкопувала, а я працював на промивці(з труби тече
вода, а я тримаю сито, щось схоже на роботу золотошукачів). Я там був один
пацан, а всі інші професійні археологи - чоловіки. Постачали мене цигарками і
пивом від якого на спеці я п'янів дуже швидко. Одного дня керівник після обіду
дав команду всім збиратись і йти у місцеву забігайлівку. Усі поскидали робочий
інструмент мені в олноколісну тачку яку я завжди мав обов' язком волочити на
нашу \enquote{базу}. Як автор цього тексту я також почув \enquote{смаленим} і
затіяв утечу @igg{fbicon.face.tears.of.joy}  та вона була не такою вдалою. Плентався я ухмілівший з тачкою
позаду усіх стежкою і на повороті до \enquote{корчми} надумав тихенько звернути
в сторону. Компанія була у них своя, кому я там треба? Але керівник помітив мій
маневр і запитав - А ти куди? - я зробив вигляд наче переплутав шось і
поплентався в корчму. Там пив на рівні з усіма, а під вечір зібрав силу у кулак
і зумів доволочити тачку додому.  Інструмент вдалось зберегти.

\begin{itemize} % {
\iusr{Elena Shinovska}
\textbf{Тарас Нечепорук} а це до чого?

\iusr{Тарас Нечепорук}
\textbf{Elena Shinovska} я був у схожій з автором ситувції і мав намір втекти з \enquote{примусової попойки}
\end{itemize} % }

\iusr{Олег Витенко}
Если женщина красива
И в постели горяча,
Это личная заслуга Леонида Ильича)

\iusr{Надежда Владимир Федько}
\textbf{Олег Витенко} Пам'ятаю!

\iusr{Олег Витенко}
Где-то читал, что Лёня всегда курил одну марку, \enquote{Новость}.

\begin{itemize} % {
\iusr{Надежда Владимир Федько}
\textbf{Олег Витенко} 

\enquote{Новость} по спец заказу. Набивались сигареты \enquote{Новость} отборным табаком
Вирджиния и только из листьев верхних побегов. При этом, у сигарет был
удлиненный фильтр для более качественной фильтрации дыма.

\ifcmt
  ig https://scontent-frx5-1.xx.fbcdn.net/v/t39.30808-6/259213501_4713312832061471_6647078316162545017_n.jpg?_nc_cat=111&ccb=1-5&_nc_sid=dbeb18&_nc_ohc=zLG7GkDQp_UAX_Ev5pv&_nc_oc=AQlulznbHa2zU_qZqLIVvLuiMX1PTZW61s8NODSm-tk7fXK0AAa7XJFCc3VCEEsQsnU&_nc_ht=scontent-frx5-1.xx&oh=00_AT9uFgV_JwhzFu-UNZtZSWOyX_hqMNFu0QenDFB1Jb-NuQ&oe=61C8F5F4
  @width 0.4
\fi

\iusr{Олег Витенко}
\textbf{Надежда Владимир Федько} Спасибо за информацию

\iusr{Надежда Владимир Федько}
\textbf{Олег Витенко}

\ifcmt
  ig https://scontent-frx5-2.xx.fbcdn.net/v/t39.30808-6/259533398_4713393318720089_6431615917781952195_n.jpg?_nc_cat=109&ccb=1-5&_nc_sid=dbeb18&_nc_ohc=SLy40E7EbJYAX-WkN7X&_nc_ht=scontent-frx5-2.xx&oh=00_AT8SHSQd5p2OLjEzqSp9soFJBhkVgUCyj0Q_5eaG9kReFA&oe=61C9E297
  @width 0.3
\fi

\end{itemize} % }

\iusr{Олег Витенко}

Ну и тогда расскажу один забавный случай. В 81-м году Брежнев открывал памятник
Родине-матери. А из моих окон, с левого берега, открывается вид на данный
шедевр. Мой друг стащил у папы подзорную трубу и бинокль. Весьма добротная
техника была. И мы пытались увидеть Брежнева. Нам было по девять лет. Но как мы
не старались, нам это не удалось. Рядом крутилась младшая сестра, пятилетнего
возраста, и тоже очень хотела попробовать данные девайсы в действии. В
конце-концов она получила подзорную трубу, высунулась в окно и как заорёт: \enquote{Я
вижу Брежнева, он машет красным флагом!}. Мы бросились смотреть. Естественно,
что это были фантазии ребенка @igg{fbicon.face.tears.of.joy}{repeat=3} 

\iusr{Иван Сторчоус}

\ifcmt
  ig https://scontent-frx5-2.xx.fbcdn.net/v/t39.30808-6/258758178_3159568550995156_5192246563157166106_n.jpg?_nc_cat=109&ccb=1-5&_nc_sid=dbeb18&_nc_ohc=M-QxVFyJVuAAX_ua_vG&_nc_ht=scontent-frx5-2.xx&oh=00_AT-aTnftKIun0ly_cHlswYnOgKooYRaAABSUgc7hzXzxzw&oe=61C91B1A
  @width 0.3
\fi

\begin{itemize} % {
\iusr{Надежда Владимир Федько}
\textbf{Иван Сторчоус} 

В тролейбусе, на сиденьи, что над колесом, сидит мужчина и читает газету.
Верхний край газеты загнулся и виден край черной рамки. Сидящие напротив двое
мужчин вытягивают шеи, пытаясь увидеть, кто же в рамке...

Мужчина поднимает глаза от газеты и тихо говорит.

- Нет. Помпиду!
\end{itemize} % }

\iusr{Тапуа Nesterenko}
Славный рассказ

\iusr{Олег Шульженко}
Видимо добрым мужиком был Леонид Ильич, курил Новость под Marlboro, гонял на подаренных Вилли Брандтом Mercedes.
Наверное и догадывался, що трохи не тим шляхом йдемо....

\iusr{Николай Варченко}
\textbf{Олег Шульженко} не догадывался, он знал это точно...

\iusr{Roman Subotin}
У нас дед-сосед тоже любил накатить в д р Брежнева. Декабрьскими вечерами болтше делать нефиг было

\iusr{Руслан Дьогтяр}
 @igg{fbicon.face.womiting} 

\iusr{Петр Кузьменко}

Интересное приключение! Но всё же бросать собутыльников как-то не по мужски!
Нужно было действовать по старому правилу: \enquote{Чем больше выпьет комсомолец, тем
меньше выпьет хулиган!}. Что там было пить? Пол банки шила на троих здоровых
мужиков. Один из которых, тем более, закалённый годами боёв с зелёным змием
особист. @igg{fbicon.laugh.rolling.floor}  Да и такой юбилей Дорогого Леонида Ильича повод отличный и
законный! @igg{fbicon.thumb.up.yellow}  @igg{fbicon.laugh.rolling.floor} 

\begin{itemize} % {
\iusr{Yurii Kadochnikov}
\textbf{Петр Кузьменко} У новорічну ніч 1997 чи 1998, одного комсомольця, що втомився, я протягав пару годин по Оболоні, але знайшов його адресу та здав родичам на руки. Нажаль, того вечора я був у формі і мусив реагувати)))

\iusr{Руслан Саченко}
\textbf{Петр Кузьменко} не факт, что в шкафу не было добавки
\end{itemize} % }

\iusr{Клим Форманчук}

За Леонида Ильича и я бы выпил ! Память моя не рисует мне жизнь в годы его
правления в печальных тонах. Мне всё тогда было очень даже комфортно. Главное,
как не банально это звучит, была уверенность в завтрашнем дне. Мы просто жили!
А сейчас, в большинстве своём, существуем ...

\begin{itemize} % {
\iusr{Анна Борисенко}
\textbf{Клим Форманчук} да его за один только Афганистан можно ненавидеть. Столько горя принесли в страну и своих сколько положили и что они пережили и не пережили в той десятилетней войне.

\begin{itemize} % {
\iusr{Света Богданець}
\textbf{Анна Борисенко} Да, но сегодня хуже чем тогда....

\iusr{Клим Форманчук}
\textbf{Анна Борисенко}, 

а почему тогда наши (СА) спешно вошли в Афганистан, знаете ?! Примерно по той
же причине сейчас многие \enquote{инструкторы} и росс.вооружение на востоке Украины.
Россия не любит граничить со своими прямыми врагами. Вот и тогда СССР не хотел
у южной границы базы НАТО. Жертвы неизбежны в любой войне и конфликте. Правда,
кто-то на тех жертвах ярды зарабатывает.

\iusr{Надежда Владимир Федько}
\textbf{Света Богданець} Чим гірше?!

\iusr{Олег Курилов}
\textbf{Анна Борисенко} 

наверное не соглашусь с вами. А причём здесь Брежнев к Афганистану? Это была не
его инициаитива, он даже не настаивал на этом решении, а поддержка этого
решения была коллегиальна. Если я не ошибаюсь, он просто завизировал это
решение. Кроме того, когда принимали это решение, думали что это на несколько
месяцев, максимум на год. Ошиблись - да, безспорно! Грубо просчитались и не
учли местную специфику - да безспорно! Нужно было более тоньше играть, не так
топорно. Ну так, от ошибок никто не застрахован. Ну так и Наполеон зря полез в
Испанию и Россию :)))). П,С. Хотя, конечно, как глава государства он несёт
ответственность за такие полные просчёты. Но на тот момент это не казалось чем
то сложным и затяжным. на всякий случай, не являюсь не фанатом не ненавистником
Брежнева. Стараюсь смотреть на вещи непредвзято и без эмоций.

\iusr{Анна Борисенко}
\textbf{Клим Форманчук} 

а что Украина была прямым врагом России, или Афганистан был прямым врагом? Или
Чехословакия в 1968 году была врагом? Прибалтика в 91? Грузия в 2008? Путину
невыносима мысль, что рядом будет свободное демократическое государство,ведь
народ в его стране посмотрит, и не дай Бог тоже захочет. У совка были те же
соображения. А Брежнев не был таким душкой, каким нам теперь его рисуют. Путем
госпереворота пришёл к власти, при нем расцвела карательная медицина,
коррупция, дефицит, блат, развал экономики.


\iusr{Анна Борисенко}
\textbf{Света Богданець} мне сегодня лучше чем в совке.

\iusr{Клим Форманчук}
\textbf{Анна Борисенко}, 

никакое крепкое государство не будет терпеть возле себя соседей которые
являются подконтрольными той страны, которая всегда есть соперником первой.
Коротко, но думаю понятно.


\iusr{Надежда Владимир Федько}
\textbf{Клим Форманчук} У Вас типово імперське мислення!

\iusr{Клим Форманчук}
\textbf{Надежда Владимир Федько}, 

це не імперських замашки, а правда життя. Тільки не пишіть про Сергіїв Посад !
Вже набридло відповідати, що \enquote{не завжди за парканом знаходиться те, що на ньому
намальовано}.

\iusr{Надежда Владимир Федько}
\textbf{Клим Форманчук} 

У нас в Україні вистачає імперців і не з \enquote{Сергієва Посада}. Я не збираюся Вас
виховувати чи перевиховувати. Але бажаючи випити за Брежнєва" Ви автоматично
стаєте моральним співучасником злочинів, які були вчинені комуністичним
режимом. І які чиняться зараз правлячим режимом Росії. Бог Вам суддя.

\iusr{Надежда Владимир Федько}
\textbf{Клим Форманчук} 

А щодо \enquote{Сергіїв Посад}... Якщо людина живе в Україні, а в акаунті вказує місце
проживання Москву, то однозначно, що вона ідентифікує себе з Росією, а не з
Україною. Україна для неї чужа! Вона духовно схвалює і політику СРСР, і
політику Росії! Вона мріє побачити російські танки на Хрещатику!

Але цього не буде! МИ не дозволимо!

\iusr{Клим Форманчук}
\textbf{Надежда Владимир Федько}, 

с вами всё ясно. Я с 1956 года с Киевом, и все его радости и боль прошли со
мной. А вы судя по всему в Киеве если и были, то проездом. И пожалуйста без
ответа,- не нарывайтесь. Всех благ !

\end{itemize} % }

\iusr{Артем Скрыпник}
Это Украина что ли была врагом??

\begin{itemize} % {
\iusr{Клим Форманчук}
\textbf{Артем Скрыпник} , а вы откуда ?! P.S. Я не люблю по чужим страницам шастать.

\iusr{Клим Форманчук}
\textbf{Артем Скрыпник} , а вы как думаете ? Или не знаете сколько вооружения и техники зашли на восток Украины с России ?! Как и сколько гробов пошли по обе стороны оттуда.

\iusr{Артем Скрыпник}
\textbf{Klim Formanchyk} Украина не была врагом РФ до нападения в 2014

\iusr{Клим Форманчук}
\textbf{Артем Скрыпник} , а где вы в моём комменте прочитали что была ?! Я с 1956 года (г.р.) с Украиной. А вот с НАТО, куда так стремятся все украинские лидеры, Россия была и есть врагом. Лидеры пиарятся, а люди гибнут.

\iusr{Надежда Владимир Федько}
\textbf{Артем Скрыпник} Для Росії Україна була ворогом з 1991-го...

\iusr{Олег Курилов}
\textbf{Надежда Владимир Федько} хорошая шутка! спасибо, посмеялся.
\end{itemize} % }

\end{itemize} % }

\iusr{Микола Веселий}

Помню, как плакала по-настоящему наша классная руководительница, когда
показывали его похороны. Почетный караул у траурного портрета был из учеников
школы. Большой портрет с черной рамкой был и на клумбе напротив сегодняшнего
ТРЦ \enquote{Гулливер}. Несколько дней, пока был траур. А потом переименовали
теперешнюю Соломенскую площать в площадь Брежнева

\iusr{Vadim Vadim}

Весело жил в годы правления Леонида Ильича и сейчас неплохо, но скучно))кстати,
особисты не валяются, а лежат в засаде))

\iusr{Богдан Дмитерко}

Лучший руководитель из всех за всю историю Советов и Независимой. Леонид Иллич
Брежнев, дорогой, мы тебя помним

\iusr{Тоня Олешко}
Даже комментировать не хочется... Как вспомню (те времена) - так и вздрогну. Враньё на вранье и брехней управляет!!!

\begin{itemize} % {
\iusr{Jimmy Stounton}
\textbf{Тоня Олешко} да чё ты знаешь то...

\iusr{Тоня Олешко}
\textbf{Vladimir Mykhalchuk} 

дело не в том, что мы слышим из \enquote{уст} этих подонков, а в том, что мы можем
узнать правду, а верить лжи - это дело каждого. Тогда мы учили
марксизм-ленинизм и шаг в сторону - расстрел или Соловки

\iusr{Сергей Челапко}
\textbf{Тоня Олешко} мы тоже в лесу жили, там же и питались

\iusr{Тоня Олешко}
\textbf{Vladimir Mykhalchuk} 

я ничего не путаю. А мягкое пошлепывание? Вам нравится жить во враньё? Мне нет

Черновол, Стус, Лукьяненко, Хельсинская группа и т. д.

И бесплатного ничего не было, кто-то всегда платил, и сейчас не можем очухаться
от этого \enquote{бесплатного}, отрыжка \enquote{меньшего брата}. Фууууу....

Единственное с чем могу смириться- это с ностальгией за молодыми годами,
здоровье, энергия.... но от старого маразматика это не зависимо, к счастью. Ещё
помню анекдоты про Брежнева и его свиту

\iusr{Olena Vitvitska}
\textbf{Vladimir Mykhalchuk} це ви серйозно про рай? Про політв´язнів часів гарьачьо любимага нічого не чули?

\iusr{Willy Poindexter}
\textbf{Vladimir Mykhalchuk}

\ifcmt
  ig https://scontent-frx5-1.xx.fbcdn.net/v/t39.30808-6/257466454_1058955014841764_9069583011048288851_n.jpg?_nc_cat=110&ccb=1-5&_nc_sid=dbeb18&_nc_ohc=GWtcBiwV_7kAX8o99Jd&_nc_ht=scontent-frx5-1.xx&oh=00_AT8CbdMDmJEGi2rdREUk6Wbzw7hEuVaKbMF3NXUS7MxgsA&oe=61C9CC93
  @width 0.4
\fi

\end{itemize} % }

\iusr{Олена Шелест}
Благодарю. Классные воспоминания о непростых временах.

\iusr{Андре Ковальчук}
Мда... жили в шоколаде и не догадывались....

\begin{itemize} % {
\iusr{Надежда Владимир Федько}

Дивлячись з чим порівнювати!

У вересні 1992-го я був у Брюсселі, в штаб-квартирі НАТО. Ми були першою
українською делегацією після розпаду СРСР.

Коли я походив вулицями міста... Подивився на гарно одягнених, усміхнених
людей... Походив по магазинам... То зрозумів, у якому концтаборі пройшли 45
років мого життя.


\iusr{Надежда Владимир Федько}

Станіславе, вітаю))
\end{itemize} % }

\iusr{Андрей Кужеев}
классный рассказ, ностальгия (не пью больше 20 лет!)

\iusr{Александр Бессарабов}
Что там говорить, было время...

\iusr{Елена Хлистунова}
Ну если бы не интернациональный долг в Афгане... а так нормально: медицина
бесплатно, обучение бесплатно... жить можно

\iusr{Толик Михальченко}
В Афганистан полезли-непонятно зачем... думаю военные это затеяли а Брежнев по старости-согласился...

\begin{itemize} % {
\iusr{Надежда Владимир Федько}
\textbf{Толик Михальченко} 

Як розповідав нам один з ветеранів, полковник ГРУ, була геополітична ідея - в
прикордонному районі Афганістану організувати \enquote{приєднання} цього району до
СРСР. Але західна розвідка про це дізналася... І місцевий шейх, який вів
переговори щодо реалізації ідеї, був ліквідований. Ветеран розповів нам це на
одному з засідань \enquote{Товариства ветеранів розвідки Військово-морського флоту} у
2011 р.

Він був офіцером оперативної розвідки 797 Червонопрапорного кабульського
розвідцентру 40-ї армії.


\iusr{Олег Курилов}
\textbf{Надежда Владимир Федько} какой бред вы написали! что курите? Кто такой Хафизулла Амин знаете? боюсь что нет! какое ещё присоединение к СССР???? какая там западная разведка??? идите учите мат. част....

\iusr{Андрей Чекховской}
Полезли по указанию хозяев британии. Там намечалось обьединение государств. Рашка три мировые воевала за британию.
\end{itemize} % }

\iusr{Oleg Kukshyn}

Кто-то из близкого окружения Леонида Ильича доложил ему, что про него сочиняют
анекдоты.

- Это хорошо.

Сказал Брежнев.

- Значит народ меня любит.

\begin{itemize} % {
\iusr{Aleksandr Mitryaev}

\obeycr
- Леонид Ильич, та какое у Вас хобби?
- Собираю о себе анекдоты.
- И много Вам удалось собрать?
- Два с половиной лагеря.
\restorecr


\iusr{Анна Николаевна}
\textbf{Aleksandr Mitryaev} Такий анекдот більше підходить для правління Сталіна. При Брежнєву були піонерськи табори відпочинку і студентські стройотряди.

\iusr{Aleksandr Mitryaev}
Мне 75 лет и я пережил все. До 1991 года за кривое слово в сторону КПСС могли запраторить ....
\end{itemize} % }

\iusr{Катя Шерман}

Уважаемые администраторы, сообщество превращается в филиал «ссср, прекрасная
страна в которой мы жили».

\begin{itemize} % {
\iusr{Надежда Владимир Федько}
\textbf{Катя Шерман} Доносітєльство - генетична риса радянського періоду))

\begin{itemize} % {
\iusr{Катя Шерман}
\textbf{Надежда Владимир Федько} 

В смысле я доношу? Так это же публичный текст. И даже просмотренный админом.
Это скорее мое мнение. Или Вы пошутили а я не поняла. Сложно не видя человека
улавливать сарказм  @igg{fbicon.smile} 


\iusr{Надежда Владимир Федько}
\textbf{Катя Шерман} 

А як ще зрозуміти Ваше звернення до Адмінів, як не публічний донос)) За мною
гріха немає! Я не хвалив СССР! А за схвальні коментарі щодо Брежнєва і життя
того періоду я не несу відповідальності. Учасники групи висловлюють свою точку
зору.


\iusr{Ирина Петрова}
\textbf{Надежда Владимир Федько} 

якщо так, і Ви \enquote{чисті}, то чому так розхвилювались? \enquote{Публічний донос} - це
чистий оксюморон) і ще - наші адміни дуже уважно і ретельно стежать за всіма
публікаціями, тому все і завжди бачать.


\iusr{Надежда Владимир Федько}
\textbf{Ирина Петрова} Я не розхвилювався)) Просто стало неприємно...
\end{itemize} % }

\iusr{Dima Kievsky}
Как видите - не только советского  @igg{fbicon.wink}  Или советский народ ещё жив?

\end{itemize} % }

\iusr{Катя Шерман}

Каким местом в вашем рассказе Киев? Админы, это по правилам разводить здесь
ностальгию не по теме сообщества?

\begin{itemize} % {
\iusr{Надежда Владимир Федько}
\textbf{Катя Шерман} Я розповів історію, яка була у Києві зі мною, корінним киянином.
Покажіть мені, будь ласка, що в моїй розповіді є похвалою СРСР!
P.S.
А за коментарі я не несу відповідальність!

\begin{itemize} % {
\iusr{Катя Шерман}
\textbf{Надежда Владимир Федько} извините, если грубо высказалась. Вот из-за комментариев и разозлилась.

\iusr{Надежда Владимир Федько}
\textbf{Катя Шерман} 

Так це реальність! Дуже багато людей у віці 60+ ностальгують по СССР, хоча
логічно пояснити свою ностальгію навряд чи зможуть. На мій погляд, це
ностальгія по молодості, по щасливих днях першого кохання, закінченню ВУЗа... і
т.д.


\iusr{Катя Шерман}
\textbf{Nadegda Volodymyr Fedko} под такими текстами люди находят подтверждение что они не одни и идут голосовать за условного Бойко (на самом деле за свою молодость которой не вернуть)
\end{itemize} % }

\iusr{Александр Дем}
\textbf{Катя Шерман} Читать надо внимательно. На ул Тургеневской

\iusr{Natali Potapenko}
\textbf{Катя Шерман}
Згодна, я, читаючи розповідь і коментарі, згадую політв'язнів тих часів

\end{itemize} % }

\iusr{Игорь Гаврилов}

Всего семьдесят пять, а каким глубоким, выжившим из ума стариком он нам
виделся. Что с людьми делал коммунизм, мрак.

\begin{itemize} % {
\iusr{Юрий Блохин}
Не коммунизм, а порочная система, это разные вещи.

\iusr{Игорь Гаврилов}
\textbf{Юрий Блохин} Порочная коммунистическая система.

\iusr{Юрий Блохин}

Ну как вариант, порочная демократическая система, прикрыться можно любым
названием, подмена понятий главный инструмент в управлении людьми.

\end{itemize} % }

\iusr{Георгий Готесман}
\enquote{Ему за нас - и деньги, и два ордена. А он - от радости - всё бил по морде нас} (В. Высоцкий)

\iusr{Анатолій Домбровський}

Коли у Леоніда Ілліча запитали чи він збирає анекдоти про себе? Так, забираю.
Вже два лагері працюють, скоро третій відкриєм.

\begin{itemize} % {
\iusr{Елена Аксенова}
\textbf{Анатолій Домбровський} , это анекдот про Сталина !!!

\iusr{Анатолій Домбровський}
\textbf{Елена Аксенова} Я слыхал его в брежневское время. Все повторяется, только в другой \enquote{интерТРИПАЦИИ}.

\iusr{Анатолій Домбровський}
\textbf{Елена Аксенова} История повторяется, у меня снова нет денег. Из юношеских афоризмов.

\iusr{Irina Somova}
\textbf{Анатолій Домбровський} Это анегдот с бородой...
\end{itemize} % }

\iusr{Alexey Novozhylov}
Як добре що цього більше нема!  @igg{fbicon.face.smiling.eyes.smiling} 

\begin{itemize} % {
\iusr{Надежда Владимир Федько}

На жаль, не була проведена люстрація функціонерів партійного апарату КПУ/КПРС! От і борсаємося досі в тенетах минулого.

\iusr{Alexey Novozhylov}
\textbf{Nadegda Volodymyr Fedko} ми не перші, а хто мав провести люстрацію тих людей, такі самі як вони? Згадайте тогочасну Україну, українську максимум на 5\%. @igg{fbicon.face.worried} 

\iusr{Надежда Владимир Федько}
\textbf{Alexey Novozhylov} Так! Вибір президентами Кравчука і Кучми яскраво демонструє політичну орієнтацію населення.
\end{itemize} % }

\iusr{Ігор Можайко}
звідки тодішнім митецям було знати, що то було на миколая, а завтра-день вєчєка?

\iusr{Ирина Петрова}

В тому далекому 1982- му році ми прогулювали роботу ( типа \enquote{творчий день у
бібліотеці}) з моїми дружками з ІГМ АН, щось купили у гастрономі, ну, як
\enquote{щось} - якусь дуже нехитру закуску, пам'ятаю лише трьохлитрову банку зелених
помідорів!))) Нашо їх такими баняками робили - не уявляю. І ось сидимо ми вдома
у одного з друзів, він тоді холостяком ще був, картопельки насмажили, якась
ковбаска та помідори!!! Звикли телевізор вмикати, щоб галдів фоном. І щось
зрозуміти не можемо - щось таке незвичне... але ж не зациклились. Ввечері немає
концерту до Дня міліції(((( і тут вже повідомили...ну, чесно кажучи, у 25
особого розпачу не було, а ось з траурного мітингу мене з одним з тих друзяків
вигнали, бо він мене смішив, на вухо шепотів анекдоти, а коли труна стукнулась
- все, я не стрималась. Парторг \enquote{запропонував} покинути залу...мені((( а Мишко
лишився((( ввечері пом'янули... мені оце зараз, в свої вже не 25, трішки
соромно, бо смертенька, будь кого, то справа не для сміху... хай простить мене
р. Б. Леонід... його гріхи йому Господь сам порахує...

\iusr{Леонид Антонов}

\obeycr
Прошла зима,
Настало лето
И я с волнением шепчу
СПАСИБО партии за это
И Леониду ильичу!
\restorecr

\begin{itemize} % {
\iusr{Світлана Савдерова}
\textbf{Леонид Антонов} начало не помню, а дальше: и все мы ваши дети и все мы ваши внуки и нам любое дело по плечу - спасибо родной партии за это и лично Леониду Ильичу .... ( это не шутка. Звучало на всех детских приветствиях всех партийных мероприятий ).
\end{itemize} % }

\iusr{Александр Дем}
Да, а жили при коммунизмё и не заметили...

\begin{itemize} % {
\iusr{Надежда Владимир Федько}

Чомусь \enquote{комуністичний рай} дуже пильно охороняли, щоб ніхто не втік у \enquote{капіталістичний ад}!
А з \enquote{капіталістичного аду} чомусь ніхто, навіть безробітні і бездомні, не хотіли жити при комунізмі!

\iusr{Вита Вовченко}
\textbf{Александр Дем} Не заметили то, чего не было)))
\end{itemize} % }

\iusr{Ирина Сидорук}
Да такого руководителя для простого народа наверно небудет

\iusr{Alexey Novozhylov}
\textbf{Ирина Сидорук} сподіваюсь що нєбудєт.  @igg{fbicon.beaming.face.smiling.eyes}  @igg{fbicon.rose} 

\iusr{Елена Черняховская}
Плакали, не представляли как жить дальше

\begin{itemize} % {
\iusr{Hanna Melnyk}
\textbf{Olena Chernyakhivska} Кто плакал, а многие радовались !!!

\iusr{Елена Черняховская}
\textbf{Hanna Melnyk} Думаю, меньшинство.
\end{itemize} % }

\iusr{Диана Харитонова}

На момент смерти Брежнева я была еще в садике. Несколько групп привели в
актовый зал, где был телевизор, и посадили нас смотреть это шоу. Несколько
детей в итоге вывели в истерике.  @igg{fbicon.frown} 

\begin{itemize} % {
\iusr{Денис Гнатюк}
\textbf{Диана Харитонова} А ще \enquote{В гостях у сказки} відмінили, падонкі. )

\iusr{Диана Харитонова}
\textbf{Денис Гнатюк} І як це стосується до моєго мессага-спогада, друже? Щось я не зрозуміла. (стала у позу, руки в боки)

\iusr{Денис Гнатюк}
\textbf{Диана Харитонова} Нагадала мені дитячі травми. Я тоді в санаторії був у Ворзелі, тож нам був ще той облом. Отак і стають антисоааєтчіками )

\iusr{Irina Somova}
\textbf{Диана Харитонова} \enquote{Шоу}, слово-то какое в садике выучили.......

\iusr{Диана Харитонова}
\textbf{Irina Somova} это уже термин меня взрослой
\end{itemize} % }

\iusr{Светик Усенко}
Спасибо, дорогой Леонид Ильич!

\iusr{Alexey Novozhylov}
\textbf{Светик Усенко} Дарагой!

\iusr{Геннадий Непомнящий}

Если сравнивать*то* выпить с сегодняшним катанием самокатом по кабмину. Думаю
что стрелка весов всё-таки склонится к *застою*

\iusr{Luda Draganova}
А фото - наче він прямо селфі зробив!

\iusr{Alex Volya}
\textbf{Luda Draganova} и тогда уже делали  @igg{fbicon.selfie} ))

\iusr{Dima Kievsky}
Нехорошо бегать с \enquote{поля боя}.... Во все времена  @igg{fbicon.wink} 

\iusr{Alex Vinokur}

Ничто так не возвышает личность, как активная жизненная позиция

Леонид Брежнев

***

\obeycr
Не расскажешь всё в один присест.
Вспомнился фрагмент из жизни прежней.
Шёл какой-то съезд КПСС,
Выступал на нём товарищ Брежнев.
\smallskip
Мудро, при генсековском жезле,
Избегая сложных дефиниций,
Говорил о разном, в том числе
Про активность жизненных позиций.
\smallskip
Ясно, он не думал обо мне,
Поважнее были адресаты,
Но меня устроило вполне,
Тезис оказался очень кстати.
\smallskip
Он помог решить один вопрос,
Не детализирую конкретно.
Я как личность даже чуть подрос,
Внешне, правда, было незаметно.
\restorecr

\iusr{Владимир Шипов}

Вот только при дорогом Леониде Ильиче и пожили, никто уже расстреливал, вешал,
казнил, пытал, ссылал в ГУЛАГ, по ночам не ездил \enquote{черный воронок}... Выучились,
стали людьми.

\begin{itemize} % {
\iusr{Константин Кострыкин}
\textbf{Владимир Шипов} 

Верно! При нем был коммунизм. Затем \enquote{барыги} недра приХватезировали и села с
агропромышленным комплексом поросли бурьянами как Припять. Скоро и там
прихватизируют.


\iusr{Gary Sorokin}
\textbf{Владимир Шипов} в псих больницы недовольных сажали Закрытого типа
\end{itemize} % }

\iusr{Dima Kievsky}

История с пьянкой напомнила другую - как нас, студентов, отправили осенью на
субботник, листья убирать и там, на свежем воздухе, довольно большая фракция во
главе \enquote{бывшего старшины} отпраздновала со спиртом. И вот, старшина не
рассчитал силы и его за этим застукала администрация... Было комсомольское
собрание, старшину осуждали не только \enquote{кандидаты в члены}, но и вовремя
сбежавшие собутыльники! Проголосовали за выговор: все за, один против.... Это
был я, пожалуй единственный, кто с ними не пил. Вот такая была мораль в поздний
застой. Сейчас, конечно, всё другое и люди сейчас другие да?  @igg{fbicon.wink} 


\iusr{Paul Berman}

Маленькая ремарка: \enquote{особист}-это сотрудник военной контрразведки. Здесь не тот
случай....

\iusr{Надежда Владимир Федько}

Між собою співробітники іменували його \enquote{особістом}.

\iusr{Андрей Миндеев}

Широкой, щедрой души оказался представитель первого отдела. Предательски, трусливо
и гадко поступил сбежавший, посмеивающийся автор, над добротой и искренностью
чувств угощавшего. @igg{fbicon.face.grinning.sweat} 

\begin{itemize} % {
\iusr{Надежда Владимир Федько}

Вважаю це привітом з СССР))

\iusr{Ирина Петрова}
\textbf{Андрей Миндеев} оце ж таке! До суто київської групи така жвава цікавість людини з іншого краю землі . Популярність групи не може не радувати! @igg{fbicon.grin} 

\begin{itemize} % {
\iusr{Надежда Владимир Федько}
\textbf{Ирина Петрова} 

Оце подумав... Як же гидко я поступив... )) Треба було йти з товаришами по
чарці до самого кінця банки... разом впасти на підлогу... розділити заточення у
витверезнику... а вийшовши на свободу з чистою совістю опохмелитися і бути
гордим, що вистояв у цьому нелегкому випробуванні!

Мене, негідного, в такий священний для кожної радянської людини день, запросив
розділити трапезу сам начальник Першого відділу! А я не оцінив широку і щедру
душу чекіста! Немає мені прощення!))

(І скупа чоловіча сльоза скотилася по моїй неголеній щоці!)

\iusr{Андрей Миндеев}
\textbf{Ирина Петрова} 

Какого хрена \enquote{хлюпались на стулья и наслаждались закуской: салом,
краковской колбасой, квашенной капусткой, первой, второй, третьей....
нахаляву?} @igg{fbicon.face.grinning.sweat}  Послали бы честно, сразу или в
процессе, по адресу ненавистного особиста. \enquote{Герой} бля, с фигой в
кармане! А приятеля то за, что бросили одного по предательски?

\end{itemize} % }

\end{itemize} % }

\iusr{Татьяна Жалнина}

Спокойная жизнь настала, но предательство, развал государства, жаль это позже
повторилось в Украине

\begin{itemize} % {
\iusr{Alexey Novozhylov}
\textbf{Tatyana Zhalnina} 

Україна поволи повертає свою культуру і історію, і стає Україною українською, а
не оккупантсько- колоніальною, а також вчиться вибудовувати стосунки з усіма
іншими країнами нашої планети, а не тихесенько сидіти за жирною росіянською
дупою і виконувати їх забаганки. Окрім російщини, в цьому світі є величезна
кількість інших країн і прекрасних культур, і нема ані часу, ані бажання
звертати увагу на руськємір.  @igg{fbicon.beaming.face.smiling.eyes}{repeat=3}  @igg{fbicon.rose} 

\iusr{Надежда Владимир Федько}
\textbf{Alexey Novozhylov} Ментально ми абсолютно різні!

\iusr{Надежда Владимир Федько}
\textbf{Татьяна Жалнина} Зрада була і є з боку національної меншини, яка вважає себе панівною!
\end{itemize} % }

\iusr{Надежда Владимир Федько}

Давня мудрість говорить: «Про покійників або правду, або нічого!»

Зі спогадів Трояка М.З, генерал-майора, заступника Голови Комітету держбезпеки УРСР (1967-1978).

***

Це було на початку грудня 1966 року. Туманна холодна погода. Л. Брежнєв з Ю.
Андроповим їдуть потягом до Праги через Чоп для того, щоб підтримати Новотного,
який ледь-ледь тримається при владі, бо його атакує Дубчек із прибічниками. <...>

Ільницький запропонував зайти до бенкетного залу перекусити. Брежнєв погодився,
а там вже були накриті столи і чекали запрошені – москвичі та секретарі обкому
з Ужгорода, осіб десь вісімнадцятеро. Брежнєв побачив стіл і каже: «О, тут и
водочка, и селедочка». Після другої чарки Леонід Ілліч запропонував розповісти,
як і хто почав агітувати за воз’єднання Закарпаття з Україною. Треба сказати,
що на той час у Брежнєва була дуже добра пам’ть – він називав прізвища
закарпатських комуністів, які тоді вели роботу щодо воз’єднання.

Застілля підходило до кінця. Начальник Чопського райвідділу подав мені знак, що
колеса переставлені, і потяг готовий в дорогу. Я сказав про це секретарю
Андропова, той – Андропову, а той – Брежнєву. А Брежнєв йому у відповідь: «Юра,
а давай никуда не ехать, пусть те чехи сами разбираются». Явно що то був жарт.
Андропов знову до нього – там чехи за Тисою вже чекають. А Брежнєв – не
поїдемо. Тоді Андропов вжив останній аргумент: «Пошли к вагону, там ожидают
«трусики». Брежнєв зразу пожвавішав і пішов. То були дві подавальниці з
управління охорони. Я їх бачив, коли поїзд повертався із Праги і в Ужгороді
було дано обід на честь Генсека. Дійсно, то були дівчата «хоч куди». Коли я
запитав заступника начальнbrа Дев’ятого управління Центру про їхні обов’язки,
то він коректно відповів, що вони подавальниці. Ото і все...

***

Джерело:

Борис Шульженко. Особистість і час. // Документи, спогади, матеріали. За
загальною редакцією професора Ю. І. Шаповала. – АДЕФ-Україна, Київ, 2006. С.
286-288.


\iusr{Oleg Ivanenko}
Эпоха Застолья

\iusr{Надежда Владимир Федько}

Навіть уявити собі не міг, що мій спогад про банальну п'янку викличе таку увагу
і стільки коментарів...  Цікавий спектр думок.

\iusr{Надежда Владимир Федько}

До відома всіх читаючих мої історії - я корінний киянин!

\iusr{Олег Романович Фещенко}

ХА-ХА-ХА - абсолютно правдивая история. Как будто окунулся опять в те времена!
И ведь практически на каждом предприятии, в каждом творческом коллективе сидел
такой тип (смотрящий от НКВД) и пыжился от своей важности. А на самом деле был
шестерка обыкновенная.

\iusr{Tamara Pankratova}

\ifcmt
  ig https://i2.paste.pics/28c02fb0ca2015b47bbd5f2a85a888fa.png
  @width 0.1
\fi

\iusr{Андрей Долохов}
Лёня сделал селфи


\end{itemize} % }
