\isubsec{2_2_Generating_PDF_from_TeX}{Generating PDF from \ \TeX}

\ \TeX\ users discovered PDF at an early stage, and problems relating to TEX and 
PDF creation are now well understood. There are three broad areas to cover: how 
 
to create PDF, how to ensure good-quality font rendering, and how to add
extra information like hypertext links and navigation buttons. We will
deal with these issues in the following sections. 

\isubsubsec{2_2_1_Creating_and_manipulating_PDF}{Creating and manipulating PDF}

PDF documents can be created in four ways: 

\begin{itemize}
  \item  Convert existing PostScript files to PDF using a ``distiller''
    program. The Adobe Acrobat Distiller is the most powerful and
    sophisticated of these, but \verb|Ghostscript| also performs well, as
    does NikNak []. This approach means that you can create PDF from any
    application that can produce PostScript (that is, almost everything). 
  \item  Use Adobe's PDFWriter printer driver for VVindows and Macintosh to produce 
    PDF from any normal application like a word processor or spreadsheet. 
  \item  Use Adobe's Acrobat Capture software to offer a workflow in which existing 
    printed pages are scanned, put through an optical character recognition system, 
    and the result saved as PDF. There is a clever feature by which words that 
    cannot be recognized are preserved as bitmaps, resulting in a reliable-loo/eing 
    PDF file that might be a mixture of real text and small bitmaps. 
  \item  Use an application that writes PDF directly. In the \ \TeX\
    world we have \pdfTEX 
    (see \refsec{2_4_Generating_PDF_directly_from_TeX} on page 67), MicroPress' VTEX  [], and a DVI 
    driver, Mark Wicks's dvipdfm. 

\end{itemize}

The most common method by far is the first, since almost all text
formatting software can write good PostScript these days. The second is
not really recommended for serious work, since it gives no opportunity
to add hypertext links automatically and gives no control over features
like compression and sampling of art work. It is useful for quick and
dirty work, however. The third method is rather specialized and is
really suitable only for large-scale projects converting legacy
documents with experienced staff controlling the quality. The last
method is, naturally, the ideal one, but there are few examples of
suitable applications. The reason is not hard to findAcrobat Distiller
is an excellent piece of software with great flexibility, and there is
not much incentive to develop new back-ends for applications. 
%%page page_51                                                  <<<---3
Users of Acrobat Distiller should carefully check how they set up the
application. Two of the ``Job Options'' panels \footnote{
We describe
and illustrate the Windows version here, but the Macintosh version is
very similar; UNIX users need to consult their documentation to see how
configuration files need to be written, or how the command line options
are used. 
} are particularly important: the one that sets compression and graphics
sampling (\reffig{2-1}), and the one that determines which fonts are
embedded (\reffig{2-2}). We will talk more about fonts in the next 
section, but be aware that Acrobat Distiller is manipulating included
bitmap figures. You need to be sure that it is doing what you expect. It
defaults to a behavior that produces small files, and this may not
necessarily be your priority. 

Manipulating PDF files after creation is beyond the scope of this
chapter. Suffice it to say that there is a large and rich selection of
plugins for Acrobat to perform all sorts of jobs, including prepress
functions, security enhancement, and marking-up comments. Merz (1998),
s 53-55, has a useful summary, and there are extensive catalogs
maintained by Adobe [] and independent vendors (for instance, []). 

\isubsubsec{2_2_2_Setting_up_fonts}{Setting up fonts}

One of the most confusing issues in both PostScript and PDF is the
handling of different types of fonts. A PDF-producing application can
deal with a font in one of three ways: First it can take the entire font
and embed it in the file; second it can make a subset font of just those
characters used in the document and embed that subset; or third it can
simply embed some summary details about the font (such as its name, its
metrics, its encoding, its type --- sans serif, symbol, for example --- and clues
about its design) and rely on the display application to show something
plausible. This last strategy is preferred for documents that are to be
delivered on the Web, since it creates the smallest files. The display
application can work again in several ways. It can try to find the named
fonts on the local system; it can simply substitute system fonts as
intelligently as possible; or it can use Multiple Master fonts to mimic
the appearance of the original font. 
 
%%page page_52                                                  <<<---3
 
Unfortunately for \ \TeX\  users, their systems have traditionally
depended on the use of fixed resolution bitmap (that is, \verb|.pk|)
fonts, since \ \TeX\  was established before scalable fonts were a
usable reality. These are embedded in PostScript output as Type 3 fonts
(see Goossens et al. (1997), Section 10.3 for a full description of
PostScript's font types). Acrobat Distiller cannot deal with these fonts
intelligently because there are no font descriptors available. It leaves
them embedded in the PDF file, and Acrobat renders them very poorly They
would print reasonably well if the original resolution were high enough. 

The contrast between three types of font display can be seen in Figures 2.3, 
2.4, and 2.5. The first two figures were both set in Monotype Baskerville, but in 
\reffig{2-4}, the font was not embedded in the PDF file, so Acrobat constructed 
a Multiple Master instance to match Baskerville as best it could. \reffig{2-5} uses 
\ \TeX's Computer Modern, but because bitmap PK fonts were used, the result is 
almost unreadable. 

Avoiding the problem of bitmap fonts in \ \TeX\  output is clearly
vital. If you intend to produce good-quality PDF, you need to find Type
1 (or TrueType, although this format is less well supported by most DVI
drivers) versions of all the fonts that you intend to use and then
inform the driver that it should use them. How this is done depends on
the DVI driver; Y\&Y's \verb|dviwindo| and \verb|dvipsone| drivers, for
instance, support (except in \emph{extremis}) \emph{only} Type 1
scalable fonts and can access whatever is installed in Adobe Type
Manager. For the widely used dvips driver (see Goossens et al. (1997),
Section 11 for more details), it is necessary to make sure that the
fonts are listed in the file \verb|psfonts.map| or in a \verb|map| file
referenced by a configuration file. For instance, to ensure that
Computer Modern is treated properly, the \ \texlive\  [] distribution
has a \verb|dvips| configuration file \verb|config.cms| that loads
\verb|cms.map|; that file contains the following lines: 

\begin{verbatim}
cmb1O CMB1O <cmb10.pfb 
cmbsy1O CMBSY1O <cmbsy10.pfb 
cmbx1O CMBXIO <cmbx10.pfb 
cmr1O CMR1O <cmr10.pfb 
cmr12 CMR12 <cmr12.pfb 
cmr17 CMR17 <cmr17.pfb 
logo1O logo1O <logo10.pfb 
logo8 logo8 <logo8.pfb 
\end{verbatim}

Usage would be something like 

\begin{verbatim}
latex myfile 
dvips myfile -Pcms -o myfile.ps 
\end{verbatim}

to prepare a PostScript file (\verb|myfi1e.ps|) that can be fed through Acrobat Distiller 
to make a PDF file. 
 
%%page page_53                                                  <<<---3
%%page page_54                                                  <<<---3
 

Note that in the last lines of the map file, the font name is in
lowercase (\verb|logo8|, as opposed to \verb|CMR10|). This is significant and happens
because the fonts come from different sources. Most components of
Computer Modern were originally put into Type 1 format by Blue Sky
Research in the 1980s, subsequently enhanced by Y\&Y, and then made
freely available in 1996 through an arrangement brokered by the American
Mathematical Society; these fonts all have uppercase names. Other
members of the family (added by Taco Hoekwater, for example) have
lowercase names. 

Confusingly, there is another set of Computer Modern Type 1 fonts prepared by 
Basil Malyshev in the early 1990s (the first version was named Paradissa and a subsequent revision, BaKoMa). These 
fonts have \emph{lowercase} names, and are found in 
some \ \TeX\  distributions. If you are confused about which versions you have, you 
need to examine one of the pfb files and look at the copyright notice. 

Many commonly used public domain technical fonts \emph{have} been converted to 
Type 1 format; among those available in \ \TeX\  archives are 

\begin{itemize}
  \item All of the Computer Modern family (including \LaTeX\  additions); 
  \item The American Mathematical Society fonts; 
  \item The St. Mary's Road symbol fonts; 
  \item The RSFS script font; 
  \item The TIPA phonetic fonts; and 
  \item The XY-pic fonts. 
\end{itemize}
 
%%page page_55                                                  <<<---3
 
Type 1 and TrueType versions of the ``European Computer Modern'' (\verb|ec|) fonts are 
available from MicroPress []. However, there are also two alternatives 

\begin{enumerate}
  \item The \verb|ae| package provides virtual fonts that match the
    \verb|ec| fonts as much as possible and draws on the original
    Computer Modern fonts. There are a few missing 
    characters, like guillemets, but this package is fine for many users. 
  \item  The commercial European Modern font set by Y\&Y [] is a set of 
    high-quality, fully hinted fonts that can fully replace \verb|ec|.
    \footnote{ This book uses the European Modern sans serif and typewriter fonts. }
\end{enumerate}

There is one final, but very important issue, to consider. If you use
commercial fonts (for example, Adobe or Monotype fonts that you have
purchased, Y\&Y's 
Lucida Bright, or European Modern), you cannot embed the entire font in a PDF 
file and then gaily make it available on the Internet. This would clearly break your 
licensing conditions because other people can extract the fonts from your file. You 
must, at a minimum, \emph{subset} the fonts, and possibly (for example, in the case of small 
vendors like Y\&Y to whom font piracy presents a serious threat) pay additional 
license fees. Y\&Y also insists that you must change Acrobat Distiller's subsetting 
mechanism. By default it does \emph{not} subset the font if more than
35\% of the characters are used. You should set this to 99\% (see \reffig{2-2}) to ensure that Distiller 
always subsets unless \emph{every} character is used. \footnote{
It is generally impossible to use 100\% of most text fonts, since the encoding vector does not usually 
let you access all glyphs contained in a font. 
} This has, besides, the desirable quality of making the document smaller. 

%%page page_56                                                  <<<---3

\isubsubsec{2_2_3_Adding_value_to_your_PDF}{Adding value to your PDF}

Creating a PDF image of your normal printed page is one thing; making an
electronic document that takes advantage of all the features of PDF is
another. At a minimum, cross-references need to have PDF hypertext links
added. However, many people also expect the possibility of automatic
bookmarks (the optional PDF ``table of contents'' on the left side of
the display), that URLs be active links, and the possibility of adding
new arbitrary links. The features can be added in four ways (in
ascending order of preference): 

\begin{enumerate}
  \item  By laboriously adding manual links in Acrobat Exchange. This option is error 
    prone and has to be repeated each time the document changes. 
  \item  By running an application that tries to guess linking from information in the 
    file. It is not a very reliable method. 
  \item  By having your application embed special PostScript code in the output that 
    can be recognized by Acrobat Distiller and turned into links, for example. 
  \item  By having your application generate PDF code directly. This will correspond 
    to the cross-reference information in the source. 
\end{enumerate}

The third method is the most widely used and is well supported by Adobe. 
Acrobat Distiller recognizes a special PostScript command, \verb|pdfmark|, and this is 
used as a hook to insert a vast amount of functionality into a PDF file. As an example, 

\begin{verbatim}
[ /Color [1 O 0] /H /I /Border [0 O O] /Subtype /Link 
/Action << /S /GOTO /D (figure.1) >> /Rect [100 254 125 266] 
/ANN pdfmark end 
\end{verbatim}

creates a hypertext link at the rectangle defined by \verb|Rect| to a point in the document 
named \verb|figure.1|, and 

\begin{verbatim}
[ /Count 0 /Action << /S /GoTo /D (section.2) >7 /Title 
(Introduction) /OUT pdfmark end 
\end{verbatim}

creates a bookmark entry with the text \verb|Introduction| pointing at
the destination \verb|section.2|. It is not our intention in this book
to describe the \verb|pdfmark| commands, since the majority of users
will not create them directly. However, Adobe has produced good
documentation [], Thomas Merz [] has an excellent freely available
primer (a chapter of Merz (1998)), and D. P. Story's Web site [] has a
detailed and well-presented tutorial on using pdfmark directly in TEX. 

One question that arises, however, is how these \verb|pdfmark| commands get into 
the PostScript file. \ \TeX\  users have it rather easy, as almost all DVI to PostScript 
drivers allow for the insertion of raw PostScript into the output stream.
\footnote{ Merz (1998), Chapter 6, describes techniques for other applications like Microsoft Word and 
FrameMaker.}

%%page page_57                                                  <<<---3

Using \verb|dvips|, for instance, you can use \verb|\special{ps:: ....}|
to insert any PostScript code you like. It is, however, unlikely that
you will write these commands in your \LaTeX\ document since it is
easier to do one of the following: 

\begin{enumerate}
  \item Use the Hyper\TeX\  \verb|\special| (see Appendix B.1 on page 403) commands to insert higher-level commands, which a driver converts to the necessary \verb|pdfmark| commands; 
  \item Use driver-specific \verb|\special| commands, such as those supported by \verb|dviwindo| 
    or \verb|VTEX|; or 
  \item Better yet, operate at the level of generalized \LaTeX\  commands in a macro 
    package, which translate to whatever mechanism is appropriate for your setup. 
\end{enumerate}

The last approach is that taken by \verb|hyperref| and is described in
Section 2.3. In a similar way, users of programs like Microsoft Word,
FrameMaker, and PageMaker trap their existing cross-referencing
mechanisms and write pdfmark commands into the output PostScript file.
The completely open and programmable nature of \ \TeX\  makes our
application particularly amenable to such an approach. 

Apart from the \verb|hyperref| package, the following packages are
``PDF-aware'': 

\begin{itemize}
  \item Packages that produce \verb|\special| commands according to the
    Hyper\TeX\  conventions, such as Michael Mehlich's \verb|hyper|, are
    PDF-aware. The resulting DVI file can be processed with the \verb|-z| option
    of \verb|dvips| to make a rich PostScript file for Acrobat Distiller. 
  \item The Con\TeX T macro package by Hans Hagen has very full support for PDF 
    in its generalized hypertext features. 
  \item Rich PDF from \verb|texinfo| documents can be created with \verb|pdftexinfo.tex|, 
    which is a slight modification of the standard \verb|texinfo| macros. This is part 
    of the \pdfTEX\  distribution and works only with that system. 
  \item A similiar modification of the \verb|webmac|, called
    \verb|pdfwebmac.tex|, allows production of hypertext PDF versions of
    programs written in \verb|WEB|. This is also part of the \pdfTEX\
    distribution. 
\end{itemize}

Finally, we must not forget what in many ways is the best solution -- an
application that writes PDF directly. We will look at one such solution,
\pdfTEX\ , in \refsec{2_4_Generating_PDF_directly_from_TeX}, which provides access to all features of PDF.
The \pdfTEX\  program adds a number of primitives to the \ \TeX\
language that can be used directly. In practice, however, most people
will find it easier to continue with the familiar\ \LaTeX\ syntax
supported by the \verb|hyperref| package since this has a driver that
maps all the commands to the new \pdfTEX\  primitives. 

%%page page_58                                                  <<<---3
