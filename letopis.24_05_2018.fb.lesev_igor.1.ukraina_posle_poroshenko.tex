% vim: keymap=russian-jcukenwin
%%beginhead 
 
%%file 24_05_2018.fb.lesev_igor.1.ukraina_posle_poroshenko
%%parent 24_05_2018
 
%%url https://www.facebook.com/permalink.php?story_fbid=1928922477138846&id=100000633379839
 
%%author_id lesev_igor
%%date 
 
%%tags obschestvo,politika,poroshenko_petr,ukraina
%%title Украина после Порошенко
 
%%endhead 
 
\subsection{Украина после Порошенко}
\label{sec:24_05_2018.fb.lesev_igor.1.ukraina_posle_poroshenko}
 
\Purl{https://www.facebook.com/permalink.php?story_fbid=1928922477138846&id=100000633379839}
\ifcmt
 author_begin
   author_id lesev_igor
 author_end
\fi

Украина после Порошенко

Ожидания 2019 года и сегодня уже кажущимся неминуемое непереизбрание Порошенко,
и в Украине, и в самопровозглашенных республиках Донбасса, и в самой Москве
необоснованно завышены.

\ifcmt
  ig https://scontent-frt3-1.xx.fbcdn.net/v/t1.6435-9/33386624_1928922343805526_2262557409681604608_n.jpg?_nc_cat=106&ccb=1-5&_nc_sid=730e14&_nc_ohc=d_4PgootNRsAX9G49jT&_nc_ht=scontent-frt3-1.xx&oh=ab428a939934c8f0c63e7588bb5b7bde&oe=61BB00EC
  @width 0.4
  %@wrap \parpic[r]
  @wrap \InsertBoxR{0}
\fi

В России уверены, что придут «адекватные» сменщики, с которыми для начала можно
будет установить прямой диалог, а в перспективе наконец-то договориться по
Крыму, что автоматом решит вопрос санкций. В ЛДНР уверены, что со сменой
Порошенко закончится ад в их повседневной жизни. Ну а в Украине вариации
ожиданий зависят от личных политических позиций и занимаемого места под
солнцем. Одни надеяться вернуться во власть. Другие порешать с новым Первым и
сохранить занятые позиции. Это если о больших и средне-больших человеках. Есть
еще категория, которая хочет отомстить. Таких много, включая кровников. Ну а
большинство просто хотят «чуть пожить для себя».

В общем, типичные ожидания патерналистических обществ России, ЛДНР и Украины.

Но шутка 2019 года заключается в том, что ничего принципиально не изменится. Мы
ведь с вами не все дети Петра Алексеевича, и не входим в ситуативную комнату.
Да, пару сотен особо одиозных будут раздеты (но опять же, не нами и не в пользу
нас), кому-то придется уехать и стать канадским или испанским подданным. Но на
среднюю температуру по палате это никак не повлияет.

Первое, что следует уяснить – никакого антимайдановского реванша, а значит и
изменения внешнеполитического под-американского курса не произойдет. Не
произойдет во многом благодаря нашему чудесному соседу, облагородившему Крым и
четко подметившему, что на Донбассе не так встали. Цивилизационное
противостояние в стране до 14 года шло нос в нос, тютелька в тютельку.
Большевики границы УСССР хоть и давненько нарезали, но сделали это так, что к
1991 году противостояние проходило как на аптечных весах. Поэтому вернуться
экс-регионалам возможности нет никакой. Майдан был не против коррупции,
воровства Семьи или бедности. Майдан – это против прорусскости и
провосточности. И если вдруг встанет вопрос, вот такой как сейчас Порошенко и
реально кто-то из экс-регионалов, никаких сомнений в конечном выборе быть не
должно.

Из этого выплывает второе – после 2019 года будут эти самые минус Порошенко и
особо прикормленные. При этом о хреновом Яценюке или бездарном Гройсмане все мы
забудем уже через несколько месяцев после смены власти. Ну вот чисто
гипотетически, завтра уже наступило и Юля во главе ЮА. Кто будет премьером?
Соболев? Власенко? Может, Кужель? Смешно ведь, правда? Но у Юли с подбором
кадров такой же швах, как и у Пети. Один в один с одинаковым мусорным пакетиком
профессионалов работают.

Но, скорее всего, будет третий вариант с передачей кадровой политики
замечательным ребятам-помощникам заштатных конгрессменов из Вашингтона. И нам
они подгонят новых яресько, супрун и прочих омелянов. Просто вместо Пети будет
Юля. Или Толя. Или Слава. А может, чем черт не шутит, Виталя. И все вот так как
сейчас есть, только с уменьшением ресурсной базы. Но чтобы этот новый Сизиф не
делал, он будет заканчивать всегда одинаково. Как это уже сделали Леня, потом
другой Леня, потом Витя, потом новый Витя, теперь Петя и… нет, повторно Пети
уже точно не будет и это единственное, что отличает действующего Первого от его
попередников.

\ii{24_05_2018.fb.lesev_igor.1.ukraina_posle_poroshenko.cmt}
