% vim: keymap=russian-jcukenwin
%%beginhead 
 
%%file topics.vojna.my.6.collect.54
%%parent topics.vojna.my.6.collect.moe
 
%%url 
 
%%author_id 
%%date 
 
%%tags 
%%title 
 
%%endhead 

Николай Гоголь, Тарас Шевченко, Леся Украинка, Иван Франко, Павел Загребельный,
Мария Пригара, Иван Коцюбинский, Олександр Олесь, Марко Вовчок, Григорий
Сковорода - это вообще философ. Всех тут не перечислишь, и это дело вкуса, на
самом деле, кому что нравится. Театры, да полно. В Киеве только театров 30,
наверное есть, или больше. Я точно не считал. Я скажу те, где я был сам.
Например, Театр Франка, Театр Леси Украинки, Учебный Театр Университета
Карпенко Карого, где студенты дают представления, просто агонь. Художники - тот
же Тарас Шевченко, Илья Репин, Катерина Билокур с ее картинами цветов, Мария
Приймаченко, да полно всяких разных художников, да, еще Иван Марчук, ну и так
далее. Вообще, это дело вкуса, как я сказал. Другие скажут еще больше, чем я. А
ты что, думала, мы пингвины и живем на куске льдины, что ли, а

ну, это тоже дело вкуса. Мне вот вообще больше всего нравились постановки
Учебного Театра Университета Карпенко Карого, я на их выступления раз десять
ходил, на Шекспира, что там еще было... Тартюф. Что еще там было... Не помню,
современные какие то пьесы были. А! Вот еще, Маруся Чурай была, ее Лина
Костенко написала, поэма в стихах. В общем, много всего есть, много, очень
много

%Татьяна Соколоваreplied to ivan
%21:42
и вообще то мы не зомби..вы просто так оскорбляете называя нас алкашами и так
далее..да и вас я понимаю,это ваша родина украина и не знаю даже..вообще сейчас
трудно о чём то судить..

так где же вы понимаете, вы и такие как вы ж все время говорите, что у нас нет
Родины, Флага и так далее, что мы нищеброды и так далее. А ведь я киевлянин и
всю жизнь живу в Киеве. А мои родители и бабушки и дедушки тоже киевляне. А
Киевляне - это Сила, это Дух, это Культура. Знаменитые люди во всем мире
родились в Киеве или имеют корни из Киева. Вон даже тот самый Булгаков, или же
Голда Меир, или же Игорь Сикорский. Киев - это все мое. Киев - это Вечный
Город, это Город Андрея Первозванного и Князя Владимира, понимаете, да. А вы
претесь в Киев на танках, вместо того чтобы приехать на поезде, и кидаете бомбы
на этот Город. Дым от бомб на фоне памятника Владимиру Крестителю - это же
просто неслыханно какое варварство. Кидать бомбы на Киев, да, на Святой Город
Киев с его Лаврой и Софией - это как бы мусульмане обстреливали Мекку. Как вы
думаете, Город вам это простит, а. Не простит вам этого Киев, не простит.
