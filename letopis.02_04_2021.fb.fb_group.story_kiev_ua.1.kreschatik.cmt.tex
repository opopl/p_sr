% vim: keymap=russian-jcukenwin
%%beginhead 
 
%%file 02_04_2021.fb.fb_group.story_kiev_ua.1.kreschatik.cmt
%%parent 02_04_2021.fb.fb_group.story_kiev_ua.1.kreschatik
 
%%url 
 
%%author_id 
%%date 
 
%%tags 
%%title 
 
%%endhead 
\subsubsection{Коментарі}

\begin{itemize} % {
\iusr{Vladimir Glikin}

\ifcmt
  ig https://i2.paste.pics/c8614490ab23e57428977ecf0aa332b9.png
  @width 0.2
\fi

\iusr{Игорь Бойко}
Музыка кошмарная

\begin{itemize} % {
\iusr{Мила Подлесная}
\textbf{Игорь Бойко} Отчего же? Типичная для того времени.

\iusr{Игорь Бойко}
\textbf{Мила Подлесная} Для того времени много чего было типично. Это не меняет ее кошмарности.

\iusr{Виктор Петровский}
\textbf{Игорь Бойко} сейчас музыка для души? Что аж срать хочетса извините за грубое слово. Просто душа сейчас поёт и радуетса.

\begin{itemize} % {
\iusr{Игорь Бойко}
\textbf{Виктор Петровский} Я что-то говорил про сейчас? У автора был выбор. Автор выбрал хрень.

\iusr{Виктор Петровский}
\textbf{Игорь Бойко} я понял, хрени сейчас помоему выше крышы

\iusr{Игорь Бойко}
\textbf{Виктор Петровский} Говорите о современной хрени в другом месте, я тут при чем?

\iusr{Елена Червоная}
\textbf{Виктор Петровский} хорошего воспитания Вам музыка не дала(

\iusr{Виктор Петровский}
\textbf{Игорь Бойко} я что к вам с претензиями? Мы наверно не так поняли друг друга. Извините если что.

\iusr{Игорь Бойко}
\textbf{Виктор Петровский} Всего доброго, тоже не имел намерения вас задеть.

\iusr{Виктор Петровский}
\textbf{Елена Червоная} За грубость я извинился.

\iusr{Елена Червоная}
\textbf{Виктор Петровский} ну и хорошо. @igg{fbicon.smile}  @igg{fbicon.hot.beverage} 
\end{itemize} % }

\end{itemize} % }

\iusr{Ната Бузань}

\ifcmt
  ig https://i2.paste.pics/3ea38b7a28ccdc0ed454a0787afa9227.png
  @width 0.2
\fi

\iusr{Мила Подлесная}

Глядя на такие видео, осознаешь, что прожил не одну, а несколько жизней, причем
очень разных.

\begin{itemize} % {
\iusr{наталья колесник}
\textbf{Мила Подлесная} а сколько еще проживем @igg{fbicon.face.grinning.smiling.eyes}  каждый день, как на войне

\iusr{Вася Лейкин}
\textbf{наталья колесник} Теперь и мы можем представить, что творилось в сердцах людей в далёком 1917....

\ifcmt
  ig https://scontent-frx5-2.xx.fbcdn.net/v/t39.1997-6/s168x128/10333122_298592793654253_1393149385_n.png?_nc_cat=1&ccb=1-5&_nc_sid=ac3552&_nc_ohc=mveKXQAb0-EAX9AMwCW&_nc_oc=AQn-l89ij5ICcVqBORP4agU_gNQDcP9wxSAwcAYRkoTSjHFJh6rGL0LHYnWJD4h6Ij4&tn=lCYVFeHcTIAFcAzi&_nc_ht=scontent-frx5-2.xx&oh=00_AT8IQpm553Bc_ETdrztVv9k-OYk8O9ziLln9cQ2UPuRqgQ&oe=61BCE147
  @width 0.1
\fi

\end{itemize} % }

\iusr{Алла Поповская}
Замирает сердце,как хочется вернутся...... @igg{fbicon.heart.red}{repeat=2}

\iusr{Марина Лабунец}
Каштаны!!!!!!!!!!!!

\iusr{Татьяна Задорожная}

Ого немногочисленных, людей дуже багато, немногочисленно авто це да, и ще
невелик1 каштан, бо коли 82 роц1 це вже велетн1

\begin{itemize} % {
\iusr{Леонид Красиловский}
\textbf{Татьяна Задорожная} зачекайте ще пару років і по Хрещатику тільки собаки бездомні будуть бігати((((
\end{itemize} % }

\iusr{Надежда Баторевич}

\ifcmt
  ig https://i2.paste.pics/9b53d25f3d4f3631c17230458a0210ad.png
  @width 0.2
\fi

\iusr{Анатолий Лернер}
Да, это был ещё Киев...

\iusr{Наталья Ершова}
Да это был Киев наш!!! Киевляне меня поймут!!!

\iusr{Людмила Кралина}
Грустно.

\iusr{Полина Микульская}
Ностальгия

\iusr{Юрий Панчук}

У меня тоже есть раннее воспоминание, как мы с бабушкой переходили Проспект
Победы поверху и не по переходу примерно напротив ЗАГСа. Помню один грузовичок
ГАЗ 51 даже остановился, хотя мы могли подождать и бабушка сказала \enquote{Вот хороший
водитель}. Тогда было по две полосы в каждую сторону и подземных переходов еще
не построили, ходил трамвай.


\iusr{Катя Теркина}
Какая зелёная улица

\iusr{Надежда Василенко}
Это город, который имеет лицо. А не лоскутное одеяло, утопленное в бетоне

\iusr{Леонид Красиловский}
А когда мне было два годика то по Крещатика конки запряженные бегали.....)))))
и Сталину ещё год жизни был отведён ....

\iusr{Людмила Иванова}
А музыка из Золотой мины!

\iusr{Людмила Иванова}
Еще были каштаны на Кресте!просто тогда не было Кличко с елками

\begin{itemize} % {
\iusr{Светлана Сергеева}
\textbf{Людмила Иванова} Ели в Киеве тоже были помимо каштанов. Голубые ели тоже были прекрасным украшением города.
\end{itemize} % }

\iusr{Тамара Мельничук}
Це прекрасно але в той час транспорту було на процентів 90 менше

\begin{itemize} % {
\iusr{Ольга Поканевич}
\textbf{Тамара Мельничук} Ага, зігнали з усього Києва. І людей теж.  @igg{fbicon.face.woozy} 

\iusr{Виктор Петровский}
\textbf{Тамара Мельничук} потому что работа была у людей на местах. Извините за грубое слово, какого хрена человеку ехать в Киев, мучитса, снимать квартиру, койко место. Всю страну уничтожили, ветер гуляет по полям и селам. И что делать людям? Все в Киев тут хоть не здохнешь. Так и живём 30 лет, а что дальше?
\end{itemize} % }

\iusr{Ирина Архипович}
Очень жаль, что нет!!! Ах, как хочется вернуться!!!!.....

\iusr{Aleksy Ivanov}
Moy Kiev... noviy ne znayu... chuvstvuyu sebya chujakom... za 30 let vse izmenilos

\iusr{Людмила Киенко}
Теперь,,каштаны Киева,, только в песне остались.

\iusr{Сергей Хромешкин}

\ifcmt
  ig https://scontent-frx5-2.xx.fbcdn.net/v/t39.1997-6/s168x128/70049209_2494354163990780_8707503201399603200_n.png?_nc_cat=1&ccb=1-5&_nc_sid=ac3552&_nc_ohc=nFi3Z53RGOoAX-bijcH&_nc_ht=scontent-frx5-2.xx&oh=00_AT8zNBg6RbZpThzDvC9CUDGwP_nEDxx3s5p-ln3KulRdVA&oe=61BE37E6
  @width 0.1
\fi

\iusr{Сергей Хромешкин}
Да был...

\iusr{Владимир Каледин}
Не вернуть уже того счастливого времени! На превеликий жаль!!!

\begin{itemize} % {
\iusr{Виктор Петровский}
\textbf{Владимир Каледин} истинные слова
\end{itemize} % }

\iusr{Виктор Петровский}

Подумаешь, Барыги начали в конце 80 красть, воруйте чтобы вас удавило. Но город
оставте в покое, зелень вырубали, всё уничтожили. Я уверен что половина этих
Барыг, киевляне родившиеся в нашем родном Киеве. Ну что нельзя было уничтожать?


\iusr{Юра Савчак}
Магія часу

\begin{itemize} % {
\iusr{Yuri Arkadyev}
\textbf{Юра Савчак} - машина времени, тёзка  @igg{fbicon.hand.ok}  @igg{fbicon.thumb.up.yellow}  @igg{fbicon.hand.waving} 
Как раз в прошлом, ХХм веке её и изобрели...
Называется интернет...
\end{itemize} % }

\iusr{Андрей Долохов}
Помню, весь Крест был зеленый. Кому мешали каштаны?

\begin{itemize} % {
\iusr{Наталья Ершова}
\textbf{Андрей Долохов} нелюди деньги отмывали, одно вырубали, другое сажали и так по кругу!!!

\iusr{Natalya Tarasenko}
\textbf{Андрей Долохов} Чужакам. @igg{fbicon.face.disappointed} 
\end{itemize} % }

\iusr{Irina Gavrilenko}
Мені тоді було два роки. Цікаво дивитися. Київ, мій Київ... Машини неймовірні. Всі ж такі правопорушники дорожнього руху.

\iusr{Олена Угнівенко}
Можно было Крещатик перейти не на светофоре)

\iusr{Елена Остапенко}

\ifcmt
  ig https://scontent-frx5-2.xx.fbcdn.net/v/t39.1997-6/s168x128/47270791_937342239796388_4222599360510164992_n.png?_nc_cat=1&ccb=1-5&_nc_sid=ac3552&_nc_ohc=vYZtHV62k4wAX_xwE90&_nc_ht=scontent-frx5-2.xx&oh=00_AT_zFyP5suouMvQGkPRs1JJE_z0hPrAaadhnA4wkg6XjAA&oe=61BD1176
  @width 0.1
\fi

\iusr{Лина Агаркова}
Спасибо. Счастливое время в любимом Городе

\iusr{Алла Кириленко}
Класс! мне тоже было 2 годика! @igg{fbicon.face.grinning.smiling.eyes} 

\begin{itemize} % {
\iusr{Arkadi Romansky}
\textbf{Алла Кириленко} а я родился. И счастлив, что родился в таком замечательном городе. И глубоко несчастлив видя, как его изуродовали.

\iusr{Светлана Недайбида-Бучко}
\textbf{Алла Кириленко} И мне!

\iusr{Алла Кириленко}
\textbf{Светлана Недайбида} я 18 июля

\iusr{Светлана Недайбида-Бучко}
\textbf{Алла Кириленко} А я - 13 сентября


\iusr{Алла Кириленко}

\ifcmt
  ig https://i2.paste.pics/82286c31a0e048d47ddc830086af9b9c.png
  @width 0.2
\fi

\end{itemize} % }

\iusr{Polina Potakh}
а мне было 12 лет и я обожала и обожаю Крещатик хотя живу уже в Сиднее с 1989 года..

\iusr{Лариса Гаврилюк}
Де наші дерева каштани???

\begin{itemize} % {
\iusr{Maria Dakicat}
\textbf{Лариса Гаврилюк} Та ось же!

\ifcmt
  ig https://scontent-frt3-1.xx.fbcdn.net/v/t1.6435-9/168034464_2895297067421695_7830589022707284700_n.jpg?_nc_cat=108&ccb=1-5&_nc_sid=dbeb18&_nc_ohc=svbyy63ofiQAX_qWTIK&_nc_ht=scontent-frt3-1.xx&oh=00_AT9BUApWz8W42P8sxmjUbom9KLY2FYk1Eo9T_hcS6S3eKQ&oe=61DF40DC
  @width 0.4
\fi

\iusr{Maria Dakicat}
А ще ось.

\ifcmt
  ig https://scontent-frx5-1.xx.fbcdn.net/v/t1.6435-9/168502599_2895298034088265_8160538025604346181_n.jpg?_nc_cat=111&ccb=1-5&_nc_sid=dbeb18&_nc_ohc=_K6LdWLoSOwAX-mYybw&_nc_ht=scontent-frx5-1.xx&oh=00_AT-FfScb0rAOS3vLQgP8iuwluRdv0yp0Jar3jeKBwmx6JQ&oe=61DE1027
  @width 0.4
\fi

\iusr{Лариса Гаврилюк}

\ifcmt
  ig https://scontent-frx5-2.xx.fbcdn.net/v/t39.1997-6/p480x480/106011002_953858235076534_2503726066745003202_n.png?_nc_cat=1&ccb=1-5&_nc_sid=0572db&_nc_ohc=aeGvVQ9s4YgAX9I1CCm&tn=lCYVFeHcTIAFcAzi&_nc_ht=scontent-frx5-2.xx&oh=00_AT_cqPOv5FHVRsVcgkYtQDp3ijhlk6Mv6voNntzcTQ_hnQ&oe=61BD6090
  @width 0.2
\fi

\end{itemize} % }

\iusr{Надежда Лабик}

Это и мой Киев. В том году у меня родился сын. Так ничего не стоило прогуляться
с коляской с пл. Победы до памятника Ленину /там рядышком жили мои
родственники/

\iusr{Yuri Arkadyev}

...Киев-67  @igg{fbicon.thinking.face} ...ну привет, родной город... через 4е
дня мне \enquote{...адын год...}  @igg{fbicon.wink}
@igg{fbicon.face.smiling.sunglasses} 

\ifcmt
  ig https://scontent-frx5-2.xx.fbcdn.net/v/t1.6435-9/168557338_10226644783172779_3030679425313623338_n.jpg?_nc_cat=109&ccb=1-5&_nc_sid=dbeb18&_nc_ohc=QFAElrX0MUsAX8nTh6D&_nc_ht=scontent-frx5-2.xx&oh=00_AT9jCJF4O7lfybZbJx_rSnynpH8uV9HCuTMMLCkT6_WEag&oe=61DCC72C
  @width 0.4
\fi

\iusr{Евгений Степаненко}
Чистый город счастливых людей

\iusr{Раиса Карчевская}
Прекрасное видео. Огромное спасибо

\iusr{Лена Смовженко}
Как красиво, уютно и зелено! Но смотришь-и почему-то становится грустно...

\begin{itemize} % {
\iusr{Бабич Лариса}
\textbf{Лена Смовженко} разделяю и поддерживаю. Тоже такие слова пришли, но вы успели первая.

\iusr{Владимир Степаныч}
 @igg{fbicon.100.percent} \%. ФИЛЬМ УЖАСОВ @igg{fbicon.ogre} 

\iusr{Валерий Маслов}
\textbf{Лена Смовженко}: и времена молодости, и времена спокойствия, развития, созидания и надежд...
\end{itemize} % }

\iusr{Наталья Конончук-Шахрур}
Супер .Скучаю .

\iusr{Oksana Kirichenko}

Я в тот году только родилась, но ностальгирую по тому, старому и зеленому,
Киеву! Всегда любила в детстве с родителями гулять по Крещатику! Сейчас уже
совсем не тот Крещатик...

\iusr{Светлана Макаренко}
Ностальгия до слез. @igg{fbicon.face.sleepy} 

\ifcmt
  ig https://i2.paste.pics/37fd67c3d833d36f51603bab86335872.png
  @width 0.2
\fi

\iusr{Настасья Старинец}
Как стильно  @igg{fbicon.cat.heart.eyes}  @igg{fbicon.face.smiling.hearts} 

\iusr{Ольга Волынец}

А мне в 1967 уже 15 лет))) Вот такие свадебные машины были возле фотографии на
бывшй Ленина, бегала с подружками на них смотреть))) А каштаны давали тень
летом, красота была. Сейчас видимо экология и \enquote{выхлопники} убили каштаны. @igg{fbicon.cry} 

\iusr{Владимир Степаныч}

ЧТО ЗА ЛЮДИ ЕДУТ В ЭТИХ ОГРОМНЫХ СУПЕРКАРАХ ПО ПРОЕЗЖЕЙ ЧАСТИ?
И КАКАЯ ЖЕ ПРОПАСТЬ МЕЖДУ НИМИ И ТЕМИ, КТО ТОЛПОЙ ГРЕБЁТ ПО ТРОТУАРУ!!!

\iusr{Юра Ключник}
Як було чудово! Після цього року, ще цілих 20 років вбивали, катували і гноїли
по тюрмах українську інтелігенцію!

\begin{itemize} % {
\iusr{Виктор Петровский}
\textbf{Юра Ключник} это звери. Но здесь о Киеве речь, о его красе, зелени. О том что был цветущий наш Киев. О ГОРОДЕ. А убивали Украинскую Нацию. Демоны Ада.
\end{itemize} % }

\iusr{Алла Канаева}
Тоска по прошлому @igg{fbicon.face.pensive} 

\iusr{Галя Твердохлебова}
\textbf{Алла Канаева} эх .. @igg{fbicon.cry} 

\iusr{Светлана Дубински}
Таким и запомнился Крещатик!@igg{fbicon.heart.red}

\iusr{Natalya Tarasenko}
Какая красивая музыка...

\begin{itemize} % {
\iusr{Iryna Dashkovska}
\textbf{Natalya Tarasenko} Музыка из фильма «Золотая мина», композитора Исака Шварца

\iusr{Natalya Tarasenko}
\textbf{Iryna Dashkovska} Спасибо, Ирина. (Уже и фильм захотелось посмотреть с такой музыкой).  @igg{fbicon.heart.growing} 

\iusr{Iryna Dashkovska}
\textbf{Natalya Tarasenko} кстати фильм 77 года, с прекрасными актерами - Даль, Киндинов, Глузский, Удовиченко.

\iusr{Natalya Tarasenko}
\textbf{Iryna Dashkovska} Что-то выплывает из памяти... Даль, Глузский - теперь точно посмотрю! Спасибо, Ирина. @igg{fbicon.bouquet} 
\end{itemize} % }

\iusr{Владимир Крылов}
Мой. Киев. Город который вспоминаю. Родился и живу

\iusr{Nataliya Borodina}
Как дядька не спеша перешел дорогу прям перед Волгой газ 21

\iusr{Наталья Швед}
Какой же он был красивый!!! Особенно когда цвели каштаны на Крещатике!!!

\ifcmt
  ig https://i2.paste.pics/b083b1aa43168f94e84254728a97f2f7.png
  @width 0.2
\fi

\iusr{Дмитрий Пушкарев}
еще нет этого пресного навязчивого красного цвета

\iusr{Ирина Соколова}
Время летит быстро, изменяются города, люди, мода... Останется история

\iusr{Галина Святненко}
И нет 1967 года...

\iusr{Maryna Sadchenko}
Как раз по нему ... стою, трудно движение со скоростью 1 км/ч назвать движением  @igg{fbicon.wink} 

\iusr{Dimitri Statnikov}

Кому нужна была эта зелень если город в то время был рассадником паскудников из
КГБ и прочих мохровых Активистов кот повально подозревали людей, арестовывали и
следили.. Мою бедную маму кот в то время работала гидом и переводчиком в
Интуристе постоянно подозревали и таскали к кураторам на Владимирскую. Она
чудом не села в тюрьму ни за что, а другим девушкам работавшим с ней повезло
меньше..

\begin{itemize} % {
\iusr{Арт Юрковская}
Всем остальным 2 миллионам киевлян.

\iusr{Oleg Kukshyn}
Только как тут взаимосвязаны зелень и КГБисты - совсем непонятно.
Киев тогда был городом-парком, сейчас такого уже нет.

\iusr{Светлана Токарева}
\textbf{Dimitri Statnikov} А причём тут политика?
\end{itemize} % }

\iusr{Властелин Кота Мяу}
И какой то Вася переходит дорогу где приспичило

\iusr{Tatiana Loukianova}
Я тоже так любила делать....

\iusr{Коневцева Натали}
Мой родной и любимый город @igg{fbicon.heart.red} но почти без каштанов на Крещатике  @igg{fbicon.cry} 

\iusr{Nadejda Popova}
Совершенно другой темп жизни

\iusr{Semyon Belenkiy}
«Як тебе не любити,
Києве Наш...»

\begin{itemize} % {
\iusr{Михаил Чартин}
\textbf{Semyon Belenkiy} К большому сожалению, эта любовь в прошлом.

\iusr{Semyon Belenkiy}
\textbf{Mikhail Chartin} Да уж, ты прав Мишенька
\end{itemize} % }

\iusr{Алексей Ткаченко}

Интересно что доброго он нашёл в те времена. Разве что был ребёнком и всё
казалось в розовом цвете. Кстати тогда водители никогда не пропускали пешеходов
даже на зебре.

\begin{itemize} % {
\iusr{Дмитрий Бартюк}
\textbf{Алексей Ткаченко} да, сбивали на хрен и ехали дальше. Да и самих зебр вообще не было, если честно. И электричества. Я помню. Это был ужас.

\iusr{Eva Evidze}
\textbf{Алексей Ткаченко} 

Але які були каштани! Вони створювали густу крону, і по Хрещатику можна було
пройти повністю в тіні дерев. А ще були густі-густі кущі навколо газонів і
клумб. Там ще стояли дуже зручні лавки. Але якийсь ідіот першими викорчував ці
гарні кущі. Потім інші ідіоти прибрали каштани із їхніми чудовими кованими
решітками з візерунками, які захищали корені дерев. А ще Хрещатик весь час
поливали. Особливо пам'ятаю, як вранці мама вела мене з метро Хрещатик до
зупинки 18-го тролейбуса, а асфальт був весь мокрий але чистий. И повітря було
чистим.

\begin{itemize} % {
\iusr{Алексей Ткаченко}
\textbf{Eva Evidze} и колбаса по 2.20

\iusr{Лариса Олейникова}
\textbf{Алексей Ткаченко} И не только 2.20, была и 1.60. Завидно?

\iusr{Alexander Fefelov}
\textbf{Лариса Олейникова} бред. Чему завидовать?

\iusr{Sofia Shutaya}
\textbf{Eva Evidze} глрод был добрым, без спешки, чистым и безопасным.

\iusr{Света Медецкая}
\textbf{Алексей Ткаченко} зато вкусная была!!!)))
\end{itemize} % }

\iusr{Elena Garam}
\textbf{Алексей Ткаченко} наверно Вам не понять

\end{itemize} % }

\iusr{Надежда Смаглюк}
Какой зеленый, уютный, теплый город. В то же время такой монументальный !

\iusr{Таня Гур}
Как жаль, что нет каштанов...

\iusr{Maria Dakicat}

От тоді були часи! Не те, що зараз! )))) Тоді трава була зеленішою, і каштани
були кращі, а машини, а люди )))) Пенсіонери з групи так люблять ностальгувати
по часам совка, я помітила.)

\begin{itemize} % {
\iusr{Петр Кузьменко}
\textbf{Maria Dakicat} і Ви, колись, повірте будете ностальгувати за теперешнім часом, коли ще можна ходити лише в масці, та дихати повітрям без ізолюючого протигазу, або балонів з повітрям за спиною скафандру. Все пізнається у проівнянні...

\begin{itemize} % {
\iusr{Maria Dakicat}
\textbf{Петр Кузьменко} Я й зараз по вулицях в масці не ходжу. Навіщо зомбі-апокаліпсіс уявляти? Чи вам так приємніше, коли думаєте,що ви пожили в супер-часи,а ми будемо мучитись? Київ зовні зараз став набагато краще, ніж був навіть 30 років тому.

\iusr{Петр Кузьменко}
\textbf{Maria Dakicat} 

саме так. Хрещатик зеленіше, автомобілей меньше, Місто \enquote{прикрашене}
баготоповерховими монстрами на колись зелених схилах тоді ще чистого та
повноводного Дніпра. І на моєму Андріївському узвозі будівля \enquote{красеня} - театру
та зовсім новий пам'ятник Гоголю, схожий на голанського неформала. Може
продовжити перелік досягнень які зробили Київ \enquote{набагато кращим}?

\iusr{Maria Dakicat}
\textbf{Петр Кузьменко} Діалог не має сенсу. Мені, доречі, подобається будівля театру на Андрієвському. Трабл в тому, що пенсіонери хочуть залишити місто таким, яким воно було в їхньому дитинстві. Тоді давайте усі разом поностальгуємо за Києвом геть без машин часів Київської Русі )

\iusr{Петр Кузьменко}
\textbf{Maria Dakicat} нажаль, діалог не має сенсу. Особливо після Вашого допису про захват будівлею Театру на Подолі. Років через 40 Ви будете на моєму боці...

\iusr{Светлана Манилова}
\textbf{Maria Dakicat}, в \enquote{Киевских историях} нет такой возрастной группы, которая бы вспоминала времена Киевской Руси... @igg{fbicon.smile} 

\iusr{Maria Dakicat}
\textbf{Светлана Манилова} Это запрещено правилами группы?  @igg{fbicon.thinking.face} А то я бы поностальгировала . Одни церкви да дома одноэтажные. Лепотааа  @igg{fbicon.face.grinning.big.eyes}  Не то, что нынешнее племя

\iusr{Светлана Манилова}
\textbf{Maria}, нет, конечно не запрещено. Просто, как я уже написала выше, вспоминать этот период (период Киевской Руси) некому... @igg{fbicon.smile} А вот свидетелей того времени, о котором вспоминает автор, в группе много.

\iusr{Петр Кузьменко}
\textbf{Maria Dakicat} чтобы предметно ностальгировать о древних временах нашего Города внимательно прочтите посты участника и автора нашей группы Юрия Никитина. Полагаю, будет не лишним.

\iusr{Alexander Fefelov}
\textbf{Петр Кузьменко} місто не може не розвиватись. Невже це так важко зрозуміти. Хоча каштанів жаль.

\iusr{Петр Кузьменко}
\textbf{Alexander Fefelov} Ви бували у Празі, у Відні? Чому так важко зрозуміти, що місто може розвиватися цивілізовано?

\iusr{Света Медецкая}
\textbf{Maria Dakicat} ну у вас и вкус!!!)))

\iusr{Alexander Fefelov}
\textbf{Петр Кузьменко} 

Бував скрізь. І Ваше порівняння досить дивне. Особливості розташування,
кількість населення, а головне, що місто після війни треба було оновлювати
треба ж приймати до уваги. Я жив на Смирнова-Ласточкина 10, практично 19
сторіччя, поряд Киянівський провулок, - це було майже село у центрі міста. На
Гончарах- Кожем'яках жив товариш фактично в 19 сторіччі, топили дровами, туалет
і вода на вулиці. У нас на Львівський теж були дома без води та каналізації, з
сараями з дровами. Так у столиці в 21 ст. не повинно бути і майже не буває в
пристойних країнах. Важко уявити велике місто з шматками сел прямо у центрі. А
таких місць було дуже багато. Я часто проїзжаю Софію та Бухарест, порівняно з
Києвом, повірте, це глибока переферія. Розумію Вас, хотілося б порівнювати Київ
з Лондоном або Парижем. Але... завдяки комунякам і іншим реальним причинам не
Гонконг і не Брюгге. Наше місто, якщо порівнювати з менш крутими, навіть
європейськими, столицями, - супер, зовсім не соромно. Якби ще не пострадянських
мери, що народились не в Києві.. от про що треба думати... як можна оцінити
роботу Омельченко, при якому спотворили Майдан та знищили Сінний ринок,
будівництво деяких хмарочосів, що закривають Лавру, при Космосі, будівництво
острівків \enquote{смерті} на дорогах, замість ремонту, при Віталіку? Якби тільки
загублені каштани були головною проблемою. Мабуть при Ярославі чи Воломирі
зелені та річок-озер в Києві було набагато більше. Але зараз 21 сторіччя і 5
млн мешканців. Будьте реалістом.

\iusr{Петр Кузьменко}
\textbf{Alexander Fefelov} 

тут я цілком згоден з Вами у більшості питань! Я сам народився та виріс у
комунальній квартирі на Андріївському узвозі 2. Повністю підтримую, що мерію та
ключові посади міского землелерозподілу та містобудування повинні посідати
небайдужі та обізнані кияни. Реконструкція та забудова сучасного Міста не
повинна перетворюватися на знищення його пам'яток та зелених насаджень.


\iusr{Master Valerii Boiko}
\textbf{Alexander Fefelov} А чого в Гончари Кожумяки заглядати, Пашка з нашого класу жив в провулку Десятинному в дерев'яному будинку, туалет на подвір'ї)))

\iusr{Alexander Fefelov}
\textbf{Valerii Boiko} 

да, Валера, точно. Я и забыл. А ведь Паша жил ну прям в самой центральной точке
Киева, причем ещё и древнего. Во дворе раскопки все время вели. И что, надо
было оставлять эту хату там? Может и не самый красивый в мире дом построили на
ее месте, но, в принципе, неплохой. Город, безусловно, должен развиваться. Хотя
немного ностальгии все же внутри присутствует  @igg{fbicon.smile}  блин

\iusr{Master Valerii Boiko}
\textbf{Alexander Fefelov} 

Ещё и Слава с первого класса жил на Андреевском в таком же деревянном доме,
помню потолки низкие были и все деревянное красного цвета. 2-3 мин ходьбы от
школы нашей. Эх Ностальжи....)


\iusr{Maria Dakicat}
\textbf{Петр Кузьменко} 

Ось вам Відень. Серед старовинних будівель нормально себе почуває новодєл. І
ніхто не охає: Пропала Мальвина, невеста моя .)

\ifcmt
  ig https://scontent-frt3-2.xx.fbcdn.net/v/t1.6435-9/168398163_2895944747356927_1970153126459682874_n.jpg?_nc_cat=103&ccb=1-5&_nc_sid=dbeb18&_nc_ohc=8CuVTXu-mwoAX9shunQ&_nc_ht=scontent-frt3-2.xx&oh=00_AT_7ctCb8qytUVimaf4gJW4-Rp6mt2t9y3iR17z554Tvuw&oe=61DE5B67
  @width 0.4
\fi

\iusr{Петр Кузьменко}
\textbf{Maria Dakicat} 

це далеко не центр Відня. Цей район щось на кшталт нашої Троєщини. Порівняйте.
До того ж жодна будівля, як в центрі так і на \enquote{задвірках} в Австрії, ніколи не
буде капітально відремонтована, чи не дай Боже знесена без широкого обговопення
та погодження з громадськістю. Тим більше будова чогось нового.


\iusr{Maria Dakicat}
\textbf{Петр Кузьменко} 

Ви все знаєте краще. Вам видніше, де у Відні Троєщина )) Панове, ваше дитинство
та юність не означаюсь, що все тоді було красиве, а зараз уродське. Це просто
ви були тоді красівші.)) А Київ змінюється, не залежачи від чиєїсь ностальгії і
від того, хто коли народився чи народив. Складно прийняти думку, що ми
старішаємо. Але це життя. Ми починаємо старішати душею, коли не сприймаємо
сучасність і вона нас починає дратувати.

\iusr{Петр Кузьменко}
\textbf{Maria Dakicat} 

да ми старійшаємо. Але радіємо життю. Приймаємо сучасність такою як вона є. Але
коли руйнують та паплюжать Святе, ми, відповідальні кияни, не будемо дивитись
на це з захопленням від таких \enquote{нововведень}! Так само думають та діють наші
діти та онуки. Те що Місто змінюється чудово. Але змінюватися воно повинно на
краще. Це в ідеалі. А зараз вже нехай \enquote{Аби не гірше...}


\iusr{Ludmyla Chechel}
\textbf{Maria Dakicat} Бодай ти ніколи не дожила до пенсійного віку! Амінь.

\iusr{Maria Dakicat}
\textbf{Ludmyla Chechel} Точняк, бабусю, прокляніть мене ))) Які добрі в нас пенсіонери )))

\iusr{Ludmyla Chechel}
\textbf{Maria Dakicat} я не твоя бабуся. і не пенсіонерка.

\end{itemize} % }

\iusr{Светлана Манилова}
\textbf{Maria}, вспоминают свое детство, молодость. И имеют на это право, тем более, что тематика нашей группы это предопределяет.

\iusr{Света Медецкая}
\textbf{Maria Dakicat} я - не пенсионер, но очень люблю ностальжи!!!))) Изуродовали Киев!!!)))

\begin{itemize} % {
\iusr{Maria Dakicat}
\textbf{Света Медецкая} 

Остаётся только застрелиться по поводу изуродованного Киева.)) Или
повеситься.)) Люди, прогресс есть прогресс. Города меняются, люди стареют и
умирают. Нет смысла застревать в прошлом, не лучше ли радоваться новому,
тому, что приносит современность? Всем удачи и добра! А Киев красавчик. Был,
есть и будет.@igg{fbicon.heart.red}

\end{itemize} % }

\iusr{Alla Rujizkaia}
\textbf{Maria Dakicat} а ще ,люблять смайлiк зi сльозкою,..

\iusr{Alexander Fefelov}
\textbf{Alla Rujizkaia} Ну ладно Вам. У меня вот противоположная позиция и никаких смайликов со слезой, несмотря на то, что мои прабабушки ещё в Киеве жили  @igg{fbicon.smile} 

\iusr{Светлана Манилова}

Уважаемые участники! Тон вашего общения неуместен в нашей группе. Постарайтесь
соблюдать правила, чтобы не было закрыто комментирование публикации
администрацией группы. Заранее спасибо!

\iusr{Светлана Манилова}

\href{https://www.facebook.com/groups/story.kiev.ua/posts/1563739173822878}{Группа \enquote{Киевские истории} - территория позитива, и я не дам никому перевести её в площадку для разжигания ненависти сторонников разных идеологий, Олег Коваль, facebook, 04.01.2021}

\end{itemize} % }

\iusr{Светлана Александренко}
Мое детство!

\iusr{Игорь Лысенко}
красиво

\iusr{Наталия Попова}
Как хорошо стало

\iusr{Татьяна Высоцкая}

Думаю, напрасно вы хороните Киев. Правда, не стоит! Он не такой, как был 40 лет
назад, но и мы не такие. Мы приняли и телефоны в кармане, пользуемся всеми благами
цивилизации. Разрешите и городу меняться. Он красивый, очень, это я слышу очень
часто от его гостей.

\begin{itemize} % {
\iusr{Светлана Токарева}

Разрешите с вами не согласиться. То что было не в какое сравнение не идёт.
Грязный, убогий, злые озабоченный лица...

\begin{itemize} % {
\iusr{Татьяна Высоцкая}
\textbf{Светлана Токарева} 

В каких местах убогий? Да, мусора стало больше, но вряд ли вы лично хотите идти
в магазин с авоськой, гремя кефирными бутылками и банкой для сметаны. Каждое
время выглядит по- своему.)

\iusr{Петр Кузьменко}
\textbf{Татьяна Высоцкая} 

прогрессивная Европа переходит или уже перешла на холщовые авоськи и стеклянные
бутылки, отказавшись от пластика. А Вы до сих пор гордитесь горами
непеработанного мусора, полигоны с которым занимают уже одну двадцатую
территории страны... @igg{fbicon.face.sad.but.relieved} 

\iusr{Татьяна Высоцкая}
\textbf{Петр Кузьменко} 

Вы начинаете раздражать, приписывая мне то, чего я не говорила. Скопируйте и
покажите мне мои слова, где я восхищаюсь горами мусора.

\end{itemize} % }

\iusr{Петр Кузьменко}
\textbf{Татьяна Высоцкая} 

нынешние гости не видели прежний Город. И уже, к сожалению, не смогут увидеть.
А, что касается технологических новшеств, то первый в мире компьютер и первый
компакт - диск и самый большой самолёт и ещё много чего было создано в том
зелёным, чистом, уютном Киеве.

\begin{itemize} % {
\iusr{Татьяна Высоцкая}
\textbf{Петр Кузьменко} 

Зато я видела и своим ощущениям верю больше, чем вашим. Просто вы ход времени
принимаете кусками, здесь нравится, а здесь рыбу заворачивали. Меняете коньки
на квадроциклы, убогие квартирки с удобствами на улице на квартиры улучшеной
планировки и тут же грустите, что нынче все худо. Это старость.)


\iusr{Петр Кузьменко}
\textbf{Татьяна Высоцкая} возможно. Но Вы пишу с большой буквы. Значит и в школе нас учили лучше...

\iusr{Татьяна Высоцкая}
\textbf{Петр Кузьменко} 

А наши дедушки носили парики и панталоны. Если честно, то для меня не имеет
значения с какой буквы вы мне напишите вы. Мы же не на диктанте, к чему эти
придирки?)


\iusr{Петр Кузьменко}
\textbf{Татьяна Высоцкая} ну, что тут ответить? В наше время грамотность была одним из качеств культурного человека...

\iusr{Татьяна Высоцкая}
\textbf{Петр Кузьменко} Одно из качеств культурного человека- не долдонить незнакомым людям свои замечания. У вас, видно, культурка особая. Спокойной ночи, бывший киевлянин.

\iusr{Петр Кузьменко}
\textbf{Татьяна Высоцкая} бывших киевлян не бывает. Доброй ночи, леди невидимка.

\iusr{Kate Levenko}
\textbf{Петр Кузьменко} 

Если вы настолько грамотны, то должны знать, что слово \enquote{вы} не обязательно
писать с большой буквы, а лишь в том случае, если хотите выразить свое глубокое
уважение к собеседнику. На просторах соцсетей обращение на \enquote{Вы} явно избыточно
и неуместно. Ну и культурные люди не попрекают других людей грамотностью - это
моветон.

\end{itemize} % }

\iusr{Alexander Fefelov}
Не соглашусь. И сейчас наш город классный

\end{itemize} % }

\iusr{Татьяна Ларченко}
Как много народа!

\iusr{Наталия Платонова}

Я заканчиваю школу, любимую, незабвенную ССМШ им. Лысенко и уезжаю, но никогда
нигде не будет так, как в Киеве...

\iusr{Aleks Faershtein}

Да....... Это был город мечты многих людей! Мне — 13, вся жизнь впереди, через
год — первая любовь!!! Каштаны, масса людей на Крещатике, все спешат, запах
мира и спокойствия вокруг, начало 6-дневной войны Израиля с арабскими
«соседями»... Никто не мог подумать, что через 24 года я окажусь на Ближнем
Востоке, через 10 лет я женюсь, через 11 родится сын, а через 16 — доча...
Мирная, спокойная, сытая и беззаботная жизнь... Как жаль, что все это пролетело
шальной пулей и кануло в никуда... Родной Киев, родной Крещатик, родные папа,
мама (земля вам пухом), тихая и мирная Украина... Жаль, что нет машины времени
и я не смогу хотя бы на часок заглянуть
домой!!!!!!! @igg{fbicon.synagogue}  @igg{fbicon.wedding}  @igg{fbicon.mosque}  @igg{fbicon.sparkles}
@igg{fbicon.dizzy}  @igg{fbicon.wilted.flower}  @igg{fbicon.dove}
@igg{fbicon.butterfly}  @igg{fbicon.unicorn}  @igg{fbicon.lady.beetle}  
@igg{fbicon.baby.chick.front.facing}  @igg{fbicon.couple.with.heart}  @igg{fbicon.hands.pray} @igg{fbicon.hand.victory} @igg{fbicon.call.me.hand} 
@igg{fbicon.thumb.up.yellow}  @igg{fbicon.love.you.gesture}  @igg{fbicon.hands.applause.yellow} 

\iusr{Aleks Faershtein}


\ifcmt
  tab_begin cols=3,no_fig,center

     pic https://scontent-frx5-1.xx.fbcdn.net/v/t1.6435-9/168255252_1372800953064747_8288088889925904836_n.jpg?_nc_cat=100&ccb=1-5&_nc_sid=dbeb18&_nc_ohc=7rY8GjR-044AX8E2UkG&_nc_ht=scontent-frx5-1.xx&oh=00_AT_5F_OgmGlliZsCLGyZHZ14iD3l_sBFcQakCVYfTwSCtQ&oe=61DFBA9F

		 pic https://scontent-frt3-1.xx.fbcdn.net/v/t1.6435-9/167937755_1372801056398070_429750041699050036_n.jpg?_nc_cat=102&ccb=1-5&_nc_sid=dbeb18&_nc_ohc=BHGg4ajTmpQAX9YQcCa&_nc_ht=scontent-frt3-1.xx&oh=00_AT_YxR6O5ok-Zk4rMEVlIMIk7tk835UHiF17VOCmSNWCCg&oe=61DF99E2

		 pic https://scontent-frx5-1.xx.fbcdn.net/v/t1.6435-9/167987478_1372801226398053_2297836402662229953_n.jpg?_nc_cat=100&ccb=1-5&_nc_sid=dbeb18&_nc_ohc=3B4NFLENe3cAX-3kr6U&_nc_ht=scontent-frx5-1.xx&oh=00_AT9-G6NBO7a5ae4CRqy12YuC-CwVOAzg4f2SUDoZOPOP3Q&oe=61DFC4F3

  tab_end

  tab_begin cols=3,no_fig,center

		 pic https://scontent-frt3-1.xx.fbcdn.net/v/t1.6435-9/167944574_1372801346398041_8153150818584870721_n.jpg?_nc_cat=107&ccb=1-5&_nc_sid=dbeb18&_nc_ohc=Q42gnHsCaLwAX95-J66&_nc_ht=scontent-frt3-1.xx&oh=00_AT_TNRGnx_VhXc8VejUFOGiIFdQan3EQ5zERVWVuoy3mUg&oe=61DCEC77

		 pic https://scontent-frx5-1.xx.fbcdn.net/v/t1.6435-9/168300892_1372801443064698_1305726814553637187_n.jpg?_nc_cat=110&ccb=1-5&_nc_sid=dbeb18&_nc_ohc=T5idySgE7pwAX_boDcJ&_nc_ht=scontent-frx5-1.xx&oh=00_AT_dYfzHBorAkaiRMwcWF0_z4igYTeMkN7JNaa7oh3DoKQ&oe=61DDCDA4

		 pic https://scontent-frx5-2.xx.fbcdn.net/v/t1.6435-9/168026836_1372801513064691_4441816860180475455_n.jpg?_nc_cat=109&ccb=1-5&_nc_sid=dbeb18&_nc_ohc=z8gChYQ5ShUAX-5c_o-&_nc_ht=scontent-frx5-2.xx&oh=00_AT9XNmDklyUI4yHtfxRf6HZa5HSAodAjOOy7rQUd0tZFtw&oe=61DE1CB4

  tab_end

\fi

\iusr{Aleks Faershtein}

\ifcmt
		 ig https://scontent-frt3-1.xx.fbcdn.net/v/t1.6435-9/167436739_1372801879731321_9197301105863225787_n.jpg?_nc_cat=106&ccb=1-5&_nc_sid=dbeb18&_nc_ohc=3xdELLmy_Y4AX9vmzLK&_nc_ht=scontent-frt3-1.xx&oh=00_AT8DAe5RcW-HPGhFtAR20i87CWkxEDkIao-LkfbBYnquSQ&oe=61DF631F
		 @width 0.4

\fi

\begin{itemize} % {
\iusr{Yuri Arkadyev}
\textbf{Aleks Faershtein} на этом фото Вы очень похожи на моего одноклассника,
\textbf{Олег Згурский}  @igg{fbicon.wink} ...и хоть между нами 12 лет разницы
но и мы и Вы родились в год  @igg{fbicon.horse}  @igg{fbicon.flame}, а это многое объясняет
@igg{fbicon.hand.ok}  @igg{fbicon.thumb.up.yellow} 
\end{itemize} % }

\iusr{Aleks Faershtein}

\ifcmt
  ig https://scontent-frt3-2.xx.fbcdn.net/v/t1.6435-9/168042611_1372802043064638_1186180301062636015_n.jpg?_nc_cat=101&ccb=1-5&_nc_sid=dbeb18&_nc_ohc=7jK-qx5ZEp8AX8x0aQD&_nc_ht=scontent-frt3-2.xx&oh=00_AT81Z8WqZU0QATyOtdyQ21MuH4AeCCQ4HL9D_FMSvAXK7w&oe=61DC8DC1
  @width 0.4
\fi

\iusr{Aleks Faershtein}

\ifcmt
  ig https://i2.paste.pics/a69d87f7071fb0026f02573131ff47b0.png
  @width 0.4
\fi

\iusr{Багинская Светлана}
Какое чудесное видео, тёплое и милое.......

\iusr{Nataliya A Bogatova Payne}
Нет рекламы @igg{fbicon.heart.eyes} 

\begin{itemize} % {
\iusr{Наталья Денисова}
\textbf{Nataliya A Bogatova Payne} как Вы правы! Какой же Киев красивый без рекламы!

\iusr{Yuri Arkadyev}
\textbf{Nataliya A Bogatova Payne} просто в то Время она была просто не нужна  @igg{fbicon.wink} ...вся реклама была в \enquote{Правде}, \enquote{Известиях} и \enquote{Вечернем Киеве}...да и Бульвар был ещё без Гордона  @igg{fbicon.wink}  @igg{fbicon.face.smiling.sunglasses} 
\end{itemize} % }

\iusr{Эльвира Земблевич}
На первых секундах этого видео ЗИС кабриолет-мой отчим везет свадьбу.

\iusr{Наталья Агранат}
Вы только посмотрите сколько народа Как все. зелено И где эти все люди сегодня

\iusr{Евгения Мамичева Яцкевич}
Его и вправду нет... он в молодости нашей, лишь осветляет душу изредка...

\iusr{Константин Павляк}
Спокойный, зелёный город....
зеленского нет даже в проекте...

\iusr{Алла Холодова}

Меня ещё тоже нет, только в 71 родилась, как я любила с подругами гулять по
этому Крещатику, сейчас даже не хочу

\begin{itemize} % {
\iusr{Слоницький Віктор Юрійович}
\textbf{Алла Холодова} Дед Степан говорил в шутку вытрищатик. Ещё он рассказывал что сторона главпочтампта гуляли студенты и гувернантки, а сторона метро для серьёзных людей.

\iusr{Алла Холодова}
\textbf{Слоницький Віктор Юрійович}

\ifcmt
  ig https://scontent-frx5-2.xx.fbcdn.net/v/t39.1997-6/s480x480/14050144_1775288802711824_1454378351_n.png?_nc_cat=1&ccb=1-5&_nc_sid=0572db&_nc_ohc=W1vW1mw9tZcAX-OqIVK&_nc_ht=scontent-frx5-2.xx&oh=00_AT9KP7q7mjFnNU3RUyYGIUiucB56YWRANpaYocQQhIxypg&oe=61BE8215
  @width 0.2
\fi

\end{itemize} % }

\iusr{Marina Lavrow}
Мне 7 лет. Дед ездит на серой Победе: их много в этом видео.

\iusr{Кирило Анатолїйович Козлов}

Я родился в 1968 году и то же помню Крещатик тех лет. действительно, светлое
пятно в памяти навсегда. именно таким запомнился и таким и буду помнить до
конца своих дней! хотя, лично Мое время, время молодости это середина 80-х.


\iusr{Inna Zalutsky}
Мы жили на Красноармейской 5, сразу за рынком, напротив магазина Минерал (?).

\iusr{Марина Винарская}
Это был совсем другой город, другая страна, другие люди..

\iusr{Марина Винарская}
Красавец город

\iusr{Татьяна Парахневич}
Я 1966 г. Поню. Бабуля в горсовете работала. Так что я, к ней ходила на работу. Какое все живое было.

\iusr{Liliana Travkina}
прямо плакать хочется, и жилось здесь и дьшалось....

\iusr{Марина Винарская}
Мы все скучаем по тем временам. Все осталось в те годы и друзья, и первая любовь,

\begin{itemize} % {
\iusr{Алексей Абрамов}
\textbf{Марина Винарская} вы не правы сей час лучше , виталик у руля су ка
\end{itemize} % }

\iusr{Андрей Соломатин}
Мой год рождения, мой любимый город, пасаж детский мир, с мамой и сестрой ходили,

\iusr{Irina Kovorotnaya}
Спасибо за прекрасное видео!))

\iusr{Alla Alyeksyeyenko}
Мне 5 лет..... как же хочется прогуляться.....

\iusr{Juri Stefan Dubrovski}

мне ще 11-ть и я не знаю английского как бы хотелось, хотя отец занимается со
мной с 5-ти лет и еще понятия не имею о немецком, папа мог бы со мной
заниматься паралельно, но он проморгал), а вот этот язык мне понадобился
больше... правда намного позже, ... помню такой Хрешатик, зеленный, спокойный,
человечный ...время было такое


\iusr{Петр Кузьменко}

Фотографии Киева того самого 1967 года.


\ifcmt
  tab_begin cols=3,no_fig,center

     pic https://scontent-frx5-1.xx.fbcdn.net/v/t1.6435-9/167759396_4039226912796229_6593582304090414678_n.jpg?_nc_cat=105&ccb=1-5&_nc_sid=dbeb18&_nc_ohc=RW1juYPFtYoAX-hO9TR&_nc_ht=scontent-frx5-1.xx&oh=00_AT_hMwcOnRvv6NtuETaL66DSLGBrRj8nUDMvGlkPmGcVOQ&oe=61DFBB07

		 pic https://scontent-frt3-1.xx.fbcdn.net/v/t1.6435-9/168868847_4039227236129530_6773688496873253018_n.jpg?_nc_cat=104&ccb=1-5&_nc_sid=dbeb18&_nc_ohc=brVZgkDSlKEAX_Kghgp&_nc_ht=scontent-frt3-1.xx&oh=00_AT--VfPwK5a1oJJvoYAoh3po5Uv9ISNb9yBeQCfJCdTicg&oe=61E01D13

		 pic https://scontent-frt3-2.xx.fbcdn.net/v/t1.6435-9/168113148_4039227566129497_4673063604080017383_n.jpg?_nc_cat=103&ccb=1-5&_nc_sid=dbeb18&_nc_ohc=a8Z-fL0drmAAX9slbe7&_nc_ht=scontent-frt3-2.xx&oh=00_AT-7rsyN3PmAsuRWYiyq7jm3DRj32Dp-McESDwSs2ay5bg&oe=61DE615C

  tab_end
\fi

\iusr{Sem Broh}
Сейчас выйдешь на крешатик как в африке побывал!)))

\iusr{Master Valerii Boiko}

Когда люди становяться стариками, когда начинают ныть,..... вот раньше, ....вот
у нас, ....вот не то что сейчас, ...И трава зеленее и хлеб вкуснее и жена
помоложе))). Да, все течет, все меняется!. Но раньше в Турции и Египте вы не
отдыхали, на Мальдивы не летали, по Европе не катались, на хороших машинах не
ездили. Мир смотрели через призму первого канала и постанов ЦК КПСС Радуйтесь
жизни панове, уж какая есть, такая есть. А каштаны на Крещатике не только от
меров растут, но и от нашего желания их посадить или создания инициативной
группы по их возрождению. А в целом Киев стал лучше чем в 67 году. Конечно не
без уродств, но это всегда в нашей жизни. !

\begin{itemize} % {
\iusr{Ирина Науменко}
\textbf{Valerii Boiko} И сейчас далеко не все на Мальдивах. А в Европу можно было поехать по туристической путевке .А вот в Крыму, на Кавказе могли отдыхать все.

\begin{itemize} % {
\iusr{Larysa Ivashuk-Panchiy}
\textbf{Ирина Науменко} все , это кто? Дети партработников на халяву. А люди из сел где отдыхали? И за что?

\iusr{Ирина Науменко}
\textbf{Larysa Ivashuk-Panchiy} Городские жители. Я в селе не жила. А у селян на книжках по 100 тысяч попропадало. Они деньги на книжку складывали.

\iusr{Serhii Yushko}
\textbf{Ирина Науменко} всі? Не смішіть, за які кошти???

\iusr{Larysa Ivashuk-Panchiy}
\textbf{Ирина Науменко} может вы страну перепутали? Городские жители - это кто? Освободители и их семьи, те котырые захватили чужое имущество, или вы о рабочих , что за копейки ишачили за третисортную еду?

\iusr{Ирина Науменко}
\textbf{Larysa Ivashuk-Panchiy} Это городские. Я говорю о тех, кто жил в Киеве.

\iusr{Ирина Науменко}
\textbf{Larysa Ivashuk-Panchiy} Это вы о каких годах. Где захватили имущество? Это вы о 1917 что ли? На картинке 60е годы. При чем тут захват имущества.

\iusr{Larysa Ivashuk-Panchiy}
\textbf{Ирина Науменко} и я о тех, кто жил в Киеве. Семья честных врачей, без дополнительного дохода не могла на Кавказ летать. Это точно.

\iusr{Ирина Науменко}
\textbf{Larysa Ivashuk-Panchiy} Могла семья честных врачей летать. Я тоже из семьи не рабочих, а служащих. Так что этот разговор в пользу бедных. На этой высокой ноте переписку закончим.

\iusr{Master Valerii Boiko}
\textbf{Ирина Науменко} Давайте честно, не все могли на Кавказ поехать и не все могли в Крыму отдыхать. Дети тех, кто могли это делать и сейчас без проблем по Европе и Мальдивах катаются!

\iusr{Ирина Науменко}
\textbf{Valerii Boiko} 

Я раньше могла, а сейчас не могу. В разные времена все люди жили по разному
.Сейчас кто то живёт, а кто то по мусорникам лазит. Вот этого в 60е годы точно не
было.

\iusr{Мила Подлесная}
\textbf{Larysa Ivashuk-Panchiy} 

Не жила в селе, мои родители были врач и инженер, но я до 90-х побывала в трех
странах за рубежем, а в своей стране была во многих республиках.


\iusr{Дарья Трощановская}
\textbf{Valerii Boiko} 

ну я могу с уверенностью сказать что каштаны растут от мэров и вода течёт и
чисто вокруг в зависимости от администрации... В Стамбул вы бы и не поехали 20
лет назад. Там было грязно, воду давали раз в месяц на улице, стояли толпы
людей с бидонами и баклагами, а хранили они её в ванных, в которых мы мылись в
Киеве к примеру. А золотой рукав был грязнее чем наш старий Дніпро, пришла
администрация эрдогана и начала менять все для людей и к лучшему, теперь всем в
кайф прогуляться по Стамбулу и новое поколение даже не помнит как было мерзко и
грязно в 90х...

Все познаётся в сравнении... И со временем... Сейчас когда администрация
сменилась там опять горы мусора на улицах.

\iusr{Larysa Ivashuk-Panchiy}
\textbf{Ирина Науменко} естественно, ибо теперь распределение не в вашу пользу! А если б по справедливости было в 60 годы- то и тогда б бедствовали

\iusr{Ирина Науменко}
\textbf{Larysa Ivashuk-Panchiy} 

Дорогая, отстань. Я и тогда не бедствовала и сейчас не бедствую. А вы явно
обделены вниманием. Поэтому вам нечем заняться, только комментарии строчить.

\end{itemize} % }

\iusr{Larysa Ivashuk-Panchiy}
А в 60 годы жильем были все обеспечены за счет кого?

\begin{itemize} % {
\iusr{Ирина Науменко}
\textbf{Larysa Ivashuk-Panchiy} Квартиры в Советское время государство выдавало их, к вашему сведению, строили тогда.

\iusr{Larysa Ivashuk-Panchiy}
\textbf{Ирина Науменко} уверенна, что так уже и не поймете, что квартиры, бессплатные спортивные кружки, отдыхи по санаториях за счет государства Не было. Ибо откуда государство может взять средства- только от людей. Не доплачивая, а некоторых труд вообще не оплачивая(например трудодни за черточку)
\end{itemize} % }

\iusr{Ольга Морозова}
\textbf{Valerii Boiko} Киев занимает почетное 3 место среди самых загрязненных столиц мира, путешественник ненаблюдательный.
\end{itemize} % }

\iusr{Наташа Лычаченко}

\ifcmt
  ig https://scontent-frx5-2.xx.fbcdn.net/v/t39.1997-6/p480x480/105941685_953860581742966_1572841152382279834_n.png?_nc_cat=1&ccb=1-5&_nc_sid=0572db&_nc_ohc=_w9z4sOXh30AX8PmqhT&_nc_ht=scontent-frx5-2.xx&oh=00_AT8IBgZzaqg_YF1_PW-CKMXtygs91dN3IoQhlVmI2tWtaw&oe=61BDEC0B
  @width 0.2
\fi

\iusr{Жанна Гнатовская}
Смотрю и плачу, что мы потеряли...

\iusr{Vadim Tkach}

Дрес- код довоєнного покоління - вважалось престижно капелюх і костюм, жінки
частіше носили хустки, на селі були в моді картузи- восьмиклинки і галіфе, у
жінок часто плюшеві жупани. Поступово прикид змінився на кепі і шапки.

\iusr{Вадим Сухомлин}

\ifcmt
  ig https://scontent-frx5-2.xx.fbcdn.net/v/t39.1997-6/s168x128/93118771_222645645734606_1705715084438798336_n.png?_nc_cat=1&ccb=1-5&_nc_sid=ac3552&_nc_ohc=lc93vluqWdkAX_GnZio&tn=lCYVFeHcTIAFcAzi&_nc_ht=scontent-frx5-2.xx&oh=00_AT8SmB1Nez50QRTPSSFGfm54nT-zbkBRV4VBrTXZKRHR7g&oe=61BCDCBF
  @width 0.1
\fi

\iusr{Виталий Войтенко}
Як тебе не любити,Києве мій! @igg{fbicon.heart.sparkling} 

\iusr{Люда Невзгляд}
Какое прекрасное было время для меня, моих. , родителей, друзей

\iusr{Люда Невзгляд}
Замечательный, красивый, уютный город. Тёплый, зелёный. Я счастливая.

\iusr{Ol Na}
 @igg{fbicon.heart.suit}{repeat=3}

\iusr{Анатолий Чернышов}

Эльвира Земблевич! Ваш отчим был кавказцем или у него только были характерные
усы и кепка-грузинка? Несколько раз видел этот ЗИС на площади Толстого, где
молодожены фотографировались в ателье.

\begin{itemize} % {
\iusr{Эльвира Земблевич}
\textbf{Анатолий Чернышов} он был украинцем, но стиляжничал. Он работал на киностудии Довженко и даже снимался в кино. \enquote{Акваланги на дне}. Оттуда его кепка.


\ifcmt
  tab_begin cols=2,no_fig,center

     pic https://scontent-frt3-2.xx.fbcdn.net/v/t1.6435-9/169668630_5198964140173522_3669524274259263073_n.jpg?_nc_cat=103&ccb=1-5&_nc_sid=dbeb18&_nc_ohc=NGKqcd8KJg4AX_iBnlK&_nc_ht=scontent-frt3-2.xx&oh=00_AT_QOO2MDh0cpGK5YPY_cuOkAfiOdt2Kg9rbfeN93SMVlQ&oe=61DFF38E

		 pic https://scontent-frx5-2.xx.fbcdn.net/v/t1.6435-9/169393793_5198964466840156_5977965090442674035_n.jpg?_nc_cat=109&ccb=1-5&_nc_sid=dbeb18&_nc_ohc=Rgl9921oPfAAX9WNk98&_nc_ht=scontent-frx5-2.xx&oh=00_AT8ctGwQE36o5FKrmFYviGQh-NCh5sVLKtmMRDyuYnF3Vw&oe=61DE29D1

  tab_end
\fi

\end{itemize} % }

\iusr{Antoniy Malynovskiy}

не грустите товарисчи)))) зелень такая штука она то есть то ее нет))) например
на фотках начало 20 века холмы лысые, деревьев почти нет... а вот реки и леса
вокруг Киева засраны капитально((( і давно така херня, і нікому діла нема про
екологію в Країні....ні зеленим, ні голубим ні різнокольоровим.....

\begin{itemize} % {
\iusr{Inna Kaduchenko}
\textbf{Antoniy Malynovskiy} А ще б узяли тай відмовились би усі разом від автомобілів, електрики, телефонів, комп'ютерів - була б дуже прикладна, дійова любов до минулого.
\end{itemize} % }

\iusr{Olga Rossokha}

Сегодня, вернее, уже вчера вечером мне пришлось, так сказать, прогуляться по
Крещанику. Крещатик с туями выглядит... печально и уныло. Как бы не пытались
оправдаться руководители разных лет ныне, загубленные каштаны - их вина. О
причинах, которые привели к гибели каштанов, писать не буду. Это отдельная
тема. Жаль, что так произошло. Нет теперь под землёй места мощной корневой
системе деревьев...

\ifcmt
  ig https://scontent-frx5-1.xx.fbcdn.net/v/t1.6435-9/168508910_3700890566675480_4651569265364435170_n.jpg?_nc_cat=111&ccb=1-5&_nc_sid=dbeb18&_nc_ohc=7N-FoDHeEdgAX8VqYgV&_nc_ht=scontent-frx5-1.xx&oh=00_AT8WLew_LjZJOHIYrgrmV1LgPgnnhWA6CS99xL5ScgUNAA&oe=61DD66DE
  @width 0.4
\fi

\iusr{Inna Nikolaevna}
да, ато были только у шишек и спекулянтов, бандитов и блатных

\iusr{Татьяна Новза}
Город моего детства, хорошо то как, очень.

\iusr{Viktoriia Ishchuk}
Так атмосферно)) Каштанов вроде бы больше было, да?

\iusr{Татьяна Новза}

Тут про какие-то курорты, Египеты и прочее, при чем это сдесь, Киев это дом и
лучше чем дома мне никогда и нигде не было 10 дней на море и все, домой в Киев.

\begin{itemize} % {
\iusr{Aleks Faershtein}
\textbf{Татьяна Новза} Вы абсолютно правы!!! Мы с детьми ездили в Евпаторию и на нашу базу отдыха в Скадовск, на залив Азовского моря... И действительно ничего лучше этого не было! Мы за последние 30 лет объездили много стран, но такого кайфа как было в 70-90 на неустроенных курортах Крыма и Кавказа НЕТ!!! Было классно после пляжа зайти в городскую столовку, съесть котлету с пюре, стакан сметаны, гороховый суп и компот... Да, и главное, мы были молоды и все было впереди....... @igg{fbicon.hands.pray}  @igg{fbicon.100.percent}  
@igg{fbicon.hand.victory} @igg{fbicon.face.zany}  @igg{fbicon.face.wink.tongue}  @igg{fbicon.heart.broken}  @igg{fbicon.heart.with.ribbon}  @igg{fbicon.hand.ok} 
\end{itemize} % }

\iusr{Ирина Форгаме}
Красивый, чистый, зеленый . А что сейсас?

\iusr{Анна Ковалева}
Спасибо. И я могла возвращаться домой с работы в Гипрограде.

\iusr{Марина Набока}

тихий, тихий, а я умудрилась тогда под машину попасть выходя из троллейбуса...
нет не мажор наехал, тогда такие на машинах не ездили, а черная волга с
начальником, еще его шофер приходил, винился, т. е. не погоняющий его
начальник, а водила попал ..., возле ЦУМа было много людей, все кричали \enquote{девушка
запишите свидетелей} а я тихо рыдала, глупая, испуганная, юная, не понимающая
в какой стране живу- потом оказалась виноватой, сказали бегала по Крещатиеку по
мостовой....


\iusr{Светлана Оликсеенко}
Все в белом!

\end{itemize} % }
