% vim: keymap=russian-jcukenwin
%%beginhead 
 
%%file 18_01_2021.fb.zharkih_katerina.1.mova_terpily
%%parent 18_01_2021
 
%%url https://www.facebook.com/kate.zharkih.5/posts/5551153354910723
 
%%author 
%%author_id 
%%author_url 
 
%%tags 
%%title 
 
%%endhead 
\subsection{Мовный закон утверждает 2 вещи}
\Purl{https://www.facebook.com/kate.zharkih.5/posts/5551153354910723}
\ifcmt
  author_begin
   author_id zharkih_katerina
  author_end
\fi

\index[rus]{Мова!Закон 16 января 2020!Kate Zharkih's opinion, 18.01.2020}

МОВНЫЙ ЗАКОН УТВЕРЖДАЕТ 2 ВЕЩИ:

\begin{itemize}
\item 1. ВЫ БЫДЛО БЕЗ ПРАВ. 
\item 2. РЫНОК В СТРАНЕ ОТМЕНЯЕТСЯ, ВМЕСТО НЕГО МОВНЫЙ ГУЛАГ 
\end{itemize}

Когда мне говорят, что тему русского языка нельзя поднимать в Украине, дабы
«патриотическое» меньшинство на зарплате не возбудилось и не набросилось на
тебя — у меня внутри всегда возникает протест. 

\ifcmt
  pic https://scontent-mad1-1.xx.fbcdn.net/v/t1.0-9/139708321_5551123404913718_1641708224855532737_n.jpg?_nc_cat=108&cb=846ca55b-311e05c7&ccb=2&_nc_sid=8bfeb9&_nc_ohc=3Jwq_DUwhW0AX-zeqPl&_nc_ht=scontent-mad1-1.xx&oh=24ac86d601897166c89bedab150fec47&oe=602AC5A8
  width 0.4
\fi


И это не только желание сказать таким советчикам, что вы, граждане-товарищи,
просто терпилы! Ведь такие одёргивания летят именно от русскоязычных украинцев.
Позиция терпилы привела к второсортности, если не третьесортности
русскоязычного украинца, права которого записано в 3,8,10 и 24 статьях
Конституции. 

Где тут демократия? Где тут Европа? 

Моя страна похожа на племя бабуинов, где нельзя быть другим, иметь иное мнение
и взгляды. И мне обидно, что огромную часть украинцев заставляют себя
чувствовать дерьмом. И испытывать вину за то, что они не поддерживают
пропаганду и дурацкие законы, которые принуждают ломать себя, строить из себя
то, чем не являешься. 

В сфере обслуживания нужно думать о клиенте, это элементарная истина любого
бизнеса. Да, переходить на язык потребителя. Учитывая, что все его знают и
понимают. Но если не понимаете или не хотите - есть государственный, не беда,
никто не обидется и не поднимет скандал. Спровоцирует это только принудиловка и
хамство, и это адекватная реакция здорового человека, у которого есть
достоинство. Из под палки ничего хорошего не будет. Ни развития ни любви. Как и
в человеческих отношениях.

Я, к примеру, слабо представляю, что приходит к вам англоязычный клиент, а вы
такие «тільки державною!». Что сделает клиент? Уйдёт к конкурентам. 

Этим законом не только деформируется мышление, когда частный бизнес должен
служить не клиенту и собственной прибыли, а обслуживать идиотскую (и
противозаконную на самом деле) политику власти. Ужас в том, что это
преподносится как норма, как достижение. И молчаливое большинство, которое
насилуют за его же деньги с налогов, уже почти согласно, что так и надо.  

Задумайтесь, это же просто убивает рынок. Раз так, то признайтесь, что у нас не
капитализм, что у нас нет частной сферы услуг, что теперь все рестораны, отели,
салоны – подразделение тоталитарного Минкульта. Согласны ли на это все частные
предприниматели? Сомневаюсь.

Но самое обидное, что нам хотят закрыть рот. Расчесать все извилины в одном
направлении, и чтобы строго на мове. 

Наша страна в крови, долгах и грязи, какая к черту мова? Мовой не закончишь
войну, не накормишь людей, не уменьшишь тарифы. Зато это рай для стукачей и
скандалистов, которые разожгут ещё больший огонь в стране. 

Пока мы тут все возмущаемся о мовном законе, продолжают воровать из нашего
кармана. Пора давать отпор. Хватит быть терпилами, нас больше. Каждый, кто
чувствует так же, знайте, я с вами.
