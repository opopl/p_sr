% vim: keymap=russian-jcukenwin
%%beginhead 
 
%%file 27_10_2022.stz.news.ua.donbas24.1.chy_ljubyv_marko_kropyvnyckyj_priazovja_stvorennja_ukr_realistic_teatr.2.furor
%%parent 27_10_2022.stz.news.ua.donbas24.1.chy_ljubyv_marko_kropyvnyckyj_priazovja_stvorennja_ukr_realistic_teatr
 
%%url 
 
%%author_id 
%%date 
 
%%tags 
%%title 
 
%%endhead 

\subsubsection{Чому вистави Марка Кропивницького викликали справжній фурор?}

Влітку 1908 року антрепренер М. Кононенко запросив М. Кропивницького приїхати в
Маріуполь. У міській газеті \enquote{Маріупольське життя} повідомлялося, що Марко
Кропивницький приїздить на гастролі і що він виступить у ролі \emph{Виборного}
(\enquote{Наталка Полтавка}), \emph{Карася} (\enquote{Запорожець за Дунаєм}), а у виставі \enquote{Доки сонце
зійде — роса очі виїсть} зіграє одразу дві ролі — колишнього селянина-кріпака
Максима Хвортуни й поміщика Воронова. Вистави мали проходити у приміщенні
театру братів Яковенків і всі квитки були негайно розпродані. Неперевершений
майстер сцени цілковито виправдав очікування публіки, яка щиро винагороджувала
гру Марка Кропивницького бурхливими оплесками, що переходили в овації. Вистави
відбувалися при переповнених залах. \emph{Через репертуар \enquote{батько української сцени}
показував глядачеві реалістичну правду дійсності, розкривав соціальні вади і
протиріччя тогочасного життя, чим \textbf{завоював у Маріуполі та загалом Приазов'ї
безліч прихильників.}} Крім запланованих трьох виступів, актор погодився взяти
участь ще в двох виставах. Він дуже любив Маріуполь та особливу атмосферу, яка
панувала в залі, коли він виступав, адже глядачі його обожнювали.

\ii{insert.read_also.demidko.donbas24.trymajemos_razom_mrpl_teatr_conception_nova_vystava}

% Марко Кропивницький
\ii{27_10_2022.stz.news.ua.donbas24.1.chy_ljubyv_marko_kropyvnyckyj_priazovja_stvorennja_ukr_realistic_teatr.pic.5}

На закінчення гастролей 19 серпня 1908 року пройшов блискучий бенефіс Марка
Кропивницького у виставі \enquote{Глитай, або ж Павук}.

Не минуло й двох років після цієї визначної події в культурному житті міста, як
у Маріуполь прийшла трагічна звістка про смерть корифея української сцени. У
той час у місті працювала трупа Т. Левченка та А. Матусина. Вона організувала в
театрі братів Яковенків \textbf{\emph{\enquote{Вечір пам'яті Марка Кропивницького}}}. По закінченню
програми була показана вистава за п'єсою М. Кропивницького \enquote{Дай серцю волю,
заведе в неволю}. Це свідчить про особливе ставлення не тільки глядачів, але й
театральної трупи міста до Марка Кропивницького та високу оцінку його внеску в
розвиток театрального мистецтва Приазов'я.

Раніше Донбас24 розповідав, як в Маріуполі
\href{https://archive.org/details/12_09_2022.olga_demidko.donbas24.festival_teatralna_brama_detali}{проходив
фестиваль \enquote{Театральна брама}}.%
\footnote{Як в Марiуполi проходив фестиваль \enquote{Театральна брама} - деталi, Ольга Демідко, donbas24.news, 12.09.2022, \par%
\url{https://donbas24.news/news/yak-v-mariupoli-proxodiv-festival-teatralna-brama-detali}, \par%
Internet Archive: \url{https://archive.org/details/12_09_2022.olga_demidko.donbas24.festival_teatralna_brama_detali}%
}

Ще більше новин та найактуальніша інформація про Донецьку та Луганську області
в нашому \href{https://t.me/donbas24}{телеграм-каналі Донбас24}.

ФОТО: з відкритих джерел.

