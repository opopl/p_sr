% vim: keymap=russian-jcukenwin
%%beginhead 
 
%%file 13_12_2021.stz.news.ua.vovka.1.tochka_nolj
%%parent 13_12_2021
 
%%url https://vovka.com.ua/?p=10285
 
%%author_id blog_obo_vsem_i_niachem
%%date 
 
%%tags 
%%title Точка ноль
 
%%endhead 
\subsection{Точка ноль}
\label{sec:13_12_2021.stz.news.ua.vovka.1.tochka_nolj}

\Purl{https://vovka.com.ua/?p=10285}

\ifcmt
 author_begin
   author_id blog_obo_vsem_i_niachem
 author_end
\fi

\begin{multicols}{2} % {
\setlength{\parindent}{0pt}

Этот пост из серии «Заметки на полях роликов», однако из-за особой важности
выходит под персональным названием.

Владимир Александрович в пятницу вывалил на Ютьюбчике кагбэ отвлекающий видос,
про какую-то левую хрень типа жила-была страна «Киевская Русь — Украина». Форма
правления — демократия через вече, на котором женщины князей выбирали, а потом
отзывали если чо там им не нравилось.

Короче малопонятная лютая дичь, в которой даже разбираться лень.

А сам тем временем пригласил к себе одну журношлистку с одного савсем жэ не
олигархического телеканала и надавал ей интервью.

И вот в этом интервью стало уже совсем не смешно.

Само интервью вышло каким-то скомканным шоле. Сидит человек, пытается родить
умную мысль, и тутраз, и просто перескакивает на отвлеченную тему. И так 22
минуты с хвостиком.

Но в нем прозвучали две вещи, которые кагбэ намекают, что Украина опять
приближается к точке ноль.

Для тех, кто не знает шо это, кратенько объясню.

Точка ноль, это звено в цепи событий, которое обрушает привычное мироустройство
и далее жизнь идет по совершенно новому сценарию. Характерной особенностью
точки ноль является то, что нет возможности вернуться обратно. Мало того, чем
больше будут прилагать усилий по возвращению, тем хуже будет результат в
последствии.

В этом ее главное отличие от точки невозврата. Они похожи, но это не тождества.
Точка невозврата не предопределяет будущее, она перечеркивает прошлое. Точка
ноль ставит четкий коридор, по которому уже нет развилок, вправо, влево, вбок.

Чтобы долго не размазывать философскую кашу по тарелке, наглядный пример — это
супружеская измена.

Жила семья, все было хорошо.

Затем кто-то сходил налево. Это была точка невозврата, но не точка ноль. Точка
невозврата, потому как это только вопрос времени, когда эта измена повторится
вновь, ну и ясно, шо это уже не семейная жизнь, а форменное лядство. Но кагбэ
выбор еще есть, не большой но все же.

А вот точка ноль — это когда вторая половина узнает об этом факте.

Жить как раньше не получится. Можно конечно с покер-фейсом делать вид, что
ничего не произошло, но результат в таком случае будет еще разрушительнее.

Возвращаясь к интервью, замечу что оба события болтаются вокруг референдумов.

Причем один из них никакой хрен не проводил, а следовало бы.

Вторым жэ Владимир Александрович с загадочным лицом пугал, и так не сильно
отчехляющую по жизни тетку, а тут еще такое. Так шо она пропустила это мимо
ушей, и вот сильно зря.

Могла бы оказать услугу стране и выспросить, а шо конкретно он имел в виду. Ибо
я не уверен, шо понял его правильно, но если таки да, то проводить такой
референдум категорически нельзя, по моему ошибочному мнению.

Но начнем с менее важного.

Если кто не следит за ходом выяснения кто кому Рабинович, между Вашингтонским
обкомом и башнями Кремля, так я могу ошибочно предположить, шо одна из тем
видеоконференции касалась пресловутого «Расширения НАТО на Восток».

Опять жэ, для тех кто не знает шо это.

Есть версия, что когда-то, еще при развале совка, в НАТО пообещали, правда
якобы на словах, не расширяться вплотную к границам русских. Но кагбэ это
обещание, если оно вообще было, старательно игнорируется, шо очевидно,
например, по троебалтам.

Было оно, не было, установить истину не представляется возможным. Горбачев в
привычной для себя манере переобувается на лету, то яростно подтверждая, то
яростно жэ опровергая сам факт таких договоренностей. В НАТО отрицают, русские
утверждают.

Все как всегда, в принципе.

Нутакот требование русских «письменно» зарубить вступление Украины и Грузии в
Альянс с моей точки зрения тупо дипломатическая уловка.

В НАТО уже подустали повторять, шо они в гробу видали Украину с ее проблемами.

И дело не в противостоянии с русскими. Дело в самой Украине.

Обозначенных Альянсом 2\% бюджета от ВВП, выделяемых на оборону и так не все
страны-участницы вытягивают. Греция, Эстония, США, Франция, Польша, Турция и
Румыния, вроде, но наверное кого-то не вспомнил. А их там 28 стран, если чо.

И в этих странах (кроме Турции) не фантики, а вполне себе деньги, по отношению
к которым украинская гривна ну кагбэ в пролете.

Даже сейчас, выпрыгивая из трусов и сдавая в долговое рабство еще не рожденных
украинцев, бюджет Украины на оборону чутка больше, чем у Бельгии. Зато площадь
Бельгии, которую надо оборонять, это шота там 5\% от площади Украины.

Вопчем понятная математика. Это даже без учета коррупции.

Такое счастье даром никому не надобно на содержании, даже если бы не было
никакой агрессии и аннексии.

Именно поэтому, как мне показалось, «письмо трех» в 2008 году было встречено ну
скажем так с прохладой со стороны Франции и Германии.

Ну и в результате этих предъяв получается дипломатическая растяжечка.

Письменно пообещать не жениться — это плевок в сторону невесты, которую уже
отматросили, но заявление в ЗАГС еще не подавали.

Отсутствие такого письменного обещания — это слив диалога в тупик, ну типа: «О
чем нам говорить, если вы свои жэ слова не можете на бумаге изложить и
расписаться?»

По словам Владимира Александровича сонный дедуля утверждает, шо вступление в
НАТО это исключительно воля украинского народа.

А вот как раз народ этот никто и не спросил.

Референдума не было, хотя вон в тех странах, которые уже в НАТЕ, референдумы
проходили. И в некоторых по нескольку раз, если чо.

В (на) Украине этот момент решили пропустить в виду неоднозначности
волеизъявления. А че их спрашивать, этих людишек? Че они там ваще понимают?

Раися шо? Атакуэ? Атакуэ! Значит нада в НАТУ, она нас защитит.

Поменяли Конституцию и готово. Встречай нас Альянс, мы уже идем.

Так вас же не звали! Да пофигу, мы уже настроились.

И плевать, шо по Конституции источником власти в (на) Украине есть народ.
Единственным кстати источником, других не обозначено.

Но это еще не самый цимес.

Как я уже писал, Владимир Александрович вращая глазами затулил, шо он «не
исключает референдума относительно Донбасса в целом».

И вот тут я припух конкретно. Че ща началось, то?

Дело в том, что когда-то чуть ли не единственный умный человек среди этих
селюков и клептоманов Вячеслав Черновол осторожно высказывался за перекроение
устройства страны. Очень осторожно и он в частности указывал, что пока еще не
время.

Суть его идеи сводилась к замене территориально-административного устройства с
унитарной формы на федеративную.

Кому интересно можете вот тут ознакомиться:

\href{https://www.youtube.com/watch?v=yhx3hXitop0}{%
Вячеслав Чорновол о федерализации Украины, История новейшего времени, youtube, 06.02.2017%
}

\end{multicols} % }

\ifcmt
  tab_begin cols=3,no_fig,center

     pic https://i2.paste.pics/25e070e33c10025186a748ca0115b907.png
		 pic https://i2.paste.pics/cb9c30274ebc3e37650cbee533f95b09.png
		 pic https://i2.paste.pics/73e6f7eb5d9ea427f180842bcbc1c421.png

  tab_end
\fi

\begin{multicols}{2} % {
\setlength{\parindent}{0pt}

Такой референдум чисто теоретически имеет право быть. Если назрела ситуация в
стране, если есть запрос у людей, если есть смелость у власти.

Но чтобы жить как Швейцария, нужно быть Швейцарией. Это там по любому чиху
пилят референдум. И страна представляет собой кантоны, в которых ВНЖ для
иностранцев из одного кантона не гарантирует получение такого же ВНЖ в другом в
автоматическом режиме путем замены карточки.

Короче Украина не Швейцария и никогда ею не будет, поэтому все эти разговоры в
пользу бедных. Они ушли вместе с Черноволом, и как мне кажется, его именно за
это и грохнули, хотя я могу и ошибаться.

А референдум по отдельно взятому Донбассу это ваще шо? Все очень напоминает
сепаратизм. Только уже со стороны человека, занимающего пост президента.

И если только поставят так вопрос, то все. Туши свет, сливай воду. Это будет
очередная точка ноль. Даже если только заикнуться.

На сейчас жители неподконтрольных территорий демонизированы и загнаны под
плинтус.

Ну типа вот они сами козлы виноваты, Путина звали, флагами махали. И ваще они
не украинцы.

Ну как не украинцы? Паспорта украинские? Значит украинцы.

Флагами махали и Путина звали отдельные Иванов, Петров и Сидоров, а не все
поголовно жители Донецкой и Луганской областей.

И вот эти вот товарищи, если следствие докажет, а суд вынесет решение, могут
называться сепаратистами, с последующей реакцией.

А не все население, просто потому что они тугие, кривые, косые или как там их
еще называют.

Никто не обязан любить свое государство. Это нигде не прописано. Нет в
Конституции статьи «Каждый гражданин обязан любить вышиванку, Бандеру и
Украину, патаму шо это и есть Украина».

Это личное дело отдельно взятого гражданина.

А вот государство наоборот обязано делать все, что написано в Конституции для
всех граждан Украины, без исключения. За это наемные менеджеры, вот эти все
чинуши, депутаты, менты, судьи и прочая шваль, получают зряплату, которую ему
платит народ Украины.

Да, да, именно народ скидываясь в бюджет. Эти деньги не с неба туда падают.
Даже кредиты, которые все равно придется отдавать тому же народу. Их не спишут,
с большой долей вероятности.

Так не работает «вон те, какие-то не такие, давайте мы их прореферендим».Через
очень короткий промежуток времени есть неиллюзорный шанс получить еще 100500
таких же «не таких» граждан.

Почему я говорю именно про очередную точку ноль? Потому что одна уже была,
только на нее не обращают внимания. Причем сознательно игнорируя.

Ща проиллюстрирую.

19 февраля 2014 года была захвачена воинская часть на территории города Львова.
Мало того, были захвачены здания СБУ и МВД.

Вот пруф.

\end{multicols} % }
