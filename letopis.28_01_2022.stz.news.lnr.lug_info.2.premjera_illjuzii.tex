% vim: keymap=russian-jcukenwin
%%beginhead 
 
%%file 28_01_2022.stz.news.lnr.lug_info.2.premjera_illjuzii
%%parent 28_01_2022
 
%%url https://lug-info.com/news/luganskij-russkij-dramteatr-predstavil-prem-eru-spektakla-illuzii
 
%%author_id news.lnr.lug_info
%%date 
 
%%tags donbass,kultura,lnr,lugansk,premjera,teatr
%%title Луганский русский драмтеатр представил премьеру спектакля "Иллюзии"
 
%%endhead 
 
\subsection{Луганский русский драмтеатр представил премьеру спектакля \enquote{Иллюзии}}
\label{sec:28_01_2022.stz.news.lnr.lug_info.2.premjera_illjuzii}
 
\Purl{https://lug-info.com/news/luganskij-russkij-dramteatr-predstavil-prem-eru-spektakla-illuzii}
\ifcmt
 author_begin
   author_id news.lnr.lug_info
 author_end
\fi

Фото: Марина Сулименко / ЛИЦ

Луганский академический русский драматический театр имени Луспекаева представил
премьеру спектакля \enquote{Иллюзии} по пьесе современного драматурга Ивана Вырыпаева.
Об этом передает корреспондент ЛИЦ.

Постановщиком выступил российский режиссер Бари Салимов, который уже реализовал
вместе с луганской труппой несколько творческих проектов.

\ii{28_01_2022.stz.news.lnr.lug_info.2.premjera_illjuzii.pic.1}

\enquote{Иллюзии} состоят из небольших рассказов-монологов, каждый из которых ставит
под сомнение содержание предыдущего. В первом монологе зрители слышат речь
умирающего старика, который горячо благодарит жену за счастливо прожитые в
браке 52 года и ее любовь, которая наполнила смыслом всю его жизнь и сделала из
него человека. Но уже в следующем рассказе эта самая жена перед смертью
признается другу мужа, что всю жизнь любила его одного. Тот, воодушевленный,
бежит к своей жене и сообщает, что полвека их совместной жизни были ошибкой, и
на самом деле он всегда любил соседку, хотя сам не подозревал об этом. А
оскорбленная жена насмешливо заявляет, что всю жизнь была любовницей его друга,
того самого, что умер в самом начале и, получается, врал своей жене даже на
смертном одре. Два мужчины и две женщины запутываются в собственных иллюзиях,
но разрывает эту иллюзорную цепь внезапная развязка пьесы.

\ii{28_01_2022.stz.news.lnr.lug_info.2.premjera_illjuzii.pic.2}

Заведующая литературной частью театра Ирина Цой отметила, что пьеса
рассказывает зрителю \enquote{о любви и нелюбви}.

\enquote{Сколько граней у чувства, которое живет в нашем сердце? Существует ли формула
любви? Вырыпаевский текст в руках свободомыслящего и не боящегося
экспериментировать режиссера Бари Салимова приобрел уникальную форму,
наполнился глубокими смыслами}, - сказала завлит.

\ii{28_01_2022.stz.news.lnr.lug_info.2.premjera_illjuzii.pic.3}

По ее мнению, актерам Андрею Ткаченко, Наталье Бутко, Анне Приходкиной и
Владиславу Иванову \enquote{удалось в многозначных диалогах открыть глубины подсознания
и заставить каждого не только чувствовать, а пережить происходящее как часть
собственного бытия}.

Ткаченко рассказал, что благодаря утонченной сценографии, которую создала
\enquote{чуткий художник} Дарья Красноштан, спектакль приобрел романтическую атмосферу:
сцены любви, в том числе и трагические, зритель застает \enquote{в серебристых оттенках
лунного света, озаряющегося яркими живописными всплесками}.

Иванов рассказал, что, озвучивая \enquote{прекрасные монологи Вырыпаева}, артисты
\enquote{погружаются в космическую атмосферу и ощущают связь со Вселенной}.

По задаче режиссера, артисты не вживаются в роли, как это происходит
традиционно, а рассказывают о персонажах зрителю так, как сами их понимают.

\enquote{Несмотря на такую необычную форму постановки Бари Салимова, темы поднимаются в
постановке вечные – это взаимоотношения мужчины и женщины}, - подчеркнул актер.

Бутко выразила уверенность в том, что после просмотра спектакля зрителям
обязательно захочется позвонить своим родным и близким и сказать им слова
любви.

Очередной показ спектакля состоится 29 января в 16:00.

Напомним, ранее артисты театра знакомились с творчеством Вырыпаева в рамках
самостоятельных работ проекта \enquote{Культурная среда}. 

Дополнительную информацию можно получить по телефонам: (0642) 50 20 31, (0642)
50 11 43, на сайте театра или на его страницах в социальных сетях \enquote{Фейсбук},
\enquote{ВКонтакте} и \enquote{Инстаграм}.
