% vim: keymap=russian-jcukenwin
%%beginhead 
 
%%file 07_02_2023.fb.fb_group.vrazhennja.ua.1.davaite_pro_pozitiv_.cmt
%%parent 07_02_2023.fb.fb_group.vrazhennja.ua.1.davaite_pro_pozitiv_
 
%%url 
 
%%author_id 
%%date 
 
%%tags 
%%title 
 
%%endhead 

\qqSecCmt

\iusr{Olga Angelova}

\enquote{Полліанна} Елеонор Портер

\begin{itemize} % {
\iusr{Катерина Черкашина}
\textbf{Olga Angelova} о, це справді позитив)

\iusr{Olga Kazak}
\textbf{Ольга Ангелова} 

Дуже крута книга. Коли мені її рекомендували, я навіть і здогадатися не могла,
що книга орієнтована здебільшого на молодшого читача може викликати таке
захоплення у дорослих

\iusr{Yulia Mozhova}
\textbf{Olga Angelova} Свого часу дуже мені допомогла. Потрапила до рук у потрібний момент

\iusr{Леся Ткаченко}
\textbf{Olga Angelova} прочитала у дитинстві і досі іноді перечитую. Книга, яка відроджує жагу до життя.
\end{itemize} % }

\iusr{Катерина Черкашина}

Цікаве питання. Мені здається, що \enquote{повертає бажання жити} - це про вивільнення
негативних емоцій, про те, щоб проридатися. І тут підійде будь-яка книга, яка
дозволить прожити емоційний катарсис. Більше того, мені здається, що простіше
книга, то краще.

Я пам'ятаю два рази, коли книга попала мені в потрібен момент і викликала хвилю емоцій.

Це були \enquote{ПС. Я люблю тебе} Сесилії Ахерн і \enquote{Лісова пісня} Лесі Українки)

А, згадала. Із недавнього це була \enquote{Життя PS} Лєри Бурлакової

\iusr{Алла Безрук}

В. Франкл \enquote{Людина у пошуках справжнього сенсу}

\begin{itemize} % {
\iusr{Senchyshyn Myroslava}
\textbf{Алла Безрук} О, тааааак
\end{itemize} % }

\iusr{Саша Серпень}

Сяйво та Доктор Сон. Стівен Кінг.

\iusr{Iva Bilka}

Зараз мені прилетить, але колись в депресивні підліткові роки мені дуже
допомогла віднайти бажання жити \enquote{Вероніка вирішила померти}. Правда спочатку
фільм, а потім книга. Багато моментів тоді у мене співпало з гг.

А для натхнення час від часу почитую твори Пагутяк. Дуже подобається її стиль.

\begin{itemize} % {
\iusr{Вікторія Адаменко}
\textbf{Iva Bilka} тільки хотіла писати те саме про \enquote{Вероніку}. А ви вже написали...
Дуже часто згадую цю книгу

\iusr{Алла Безрук}
\textbf{Iva Bilka}, свого часу і мені ця книга подобалась, є над чим замислитись..

\iusr{Цікава Ґава}
\textbf{Iva Bilka} я і досі маю цю трилогію на полиці. Вона правда російською, тому ховаю. Це єдині русняві книги + Піколт, від яких я поки що не можу позбавитися.
\end{itemize} % }

\iusr{Катерина Черкашина}

Бажання до творчості мені будила Велика магія від Ліз Гілберт

\iusr{Nastia Dzhula}

Щоденник Бріджит Джонс - єдина книга яку перечитувала разів 5-6

\iusr{Антоніна Федорович}

Вільна Гленон Дойл

\begin{itemize} % {
\iusr{Людмила Тараненко}
\textbf{Антоніна Федорович} зараз читаю. Не все мені \enquote{лягає}, але дуже гарно написано

\iusr{Антоніна Федорович}
\textbf{Людмила Тараненко} звісно, що не все, бо в кожного свій досвід, свої тригери.
Я просто проридала над книгою. Чесно

\iusr{Антоніна Федорович}

Ось

\url{https://www.facebook.com/groups/vrajenniya.ua/permalink/1377166519308450}

\iusr{Надія Омельчук}
\textbf{Антоніна Федорович} наскільки ж ми всі різні) Я її так і не осилила - продала)
\end{itemize} % }

\iusr{Юлия Шевченко}

Піднесені вітром М. Мітчел та уся серія про Мумі Тролей

\iusr{Inga Lazoryak}

Джейн Ейр і Віднесені вітром 📚❤ назавжди

\begin{itemize} % {
\iusr{Pisachova Marina}
\textbf{Инга Лазоряк} Джейн Ейр❤️ читала разів сто) і дивилася всі екранізації, улюблена з Маклом Фасбендером

\iusr{Aryna Andriichenko}
\textbf{Pisachova Marina} прийшла написати про Джейн Ейр. Перечитую може раз на рік з дитинства. Щоправда, з віком інакше вже сприймаєш персонажів (типу містер Рочестер - абьюзер і расист, Джейн трохи занадто емоційна), але все рівно люблю.

\iusr{Pisachova Marina}
\textbf{Aryna Andriichenko} про містера Рочестера 100\%

\iusr{Aryna Andriichenko}
\textbf{Inga Lazoryak} раніше теж Віднесені вітром любила. Зараз описи природи здаються занадто довгими, головна пара - двоє нещасних людей, що через свій інфантилізм не змогли стати щасливими разом, ну і відвертий расизм авторки теж дістає. Що лишилось добрим - це психоаналіз вчинків героїв і опис історичних подій
\end{itemize} % }

\iusr{Катерина Черкашина}

Позитивними вважаю книги з Дискосвіту Тері Пратчетта.

І всю творчість Маркеса. А ще проза Йогансена)

\begin{itemize} % {
\iusr{Олена Косовська}
\textbf{Катерина Черкашина} Пратчетт - безпрограшний варіант, коли хочу чогось крутого і не знаю, чого. От дереш будь-яку його книгу і в яблучко.
\end{itemize} % }

\iusr{Євген Агєєв}

\enquote{Нескінчений жарт}

\iusr{Людмила Тараненко}

Ой, ну це були різні книги в різний час. Буває, що в цей конкретний момент тобі
трапляється потрібна книга. Ніколи не вгадаєш, що саме зараз зайде.

Я люблю книги Плохія.

Колись мені дуже зайшли книги Стефані Маєр із серії \enquote{Сутінки}.

На початку повномасштабного вторгнення я прочитала декілька творів Ремарка і
\enquote{Марусю} Шкляра. Вони були ну дуже \enquote{на часі}

\iusr{Pisachova Marina}

Із книг «Чаликушу» Решат Нурі Гюнтекін. Фільми про Гаррі Поттера. Передивилася
весною, коли виїхала з Чернігівщини із зони бойових дій

\begin{itemize} % {
\iusr{Svitlana Kibich}
\textbf{Pisachova Marina} так, і в мене \enquote{Чаликушу}. Читала-перечитувала разів 10. Фільм теж сильний, особливо вдало підібрана музика.

\iusr{Anna Daineko}
\textbf{Pisachova Marina} оооо, дякую, що нагадали! Пішла копистатися у шафі) Чаликушу - те, що мені сьогодні вкрай потрібно)
\end{itemize} % }

\iusr{Діана Веллс}

\enquote{Енн із Зелених Дахів}

\iusr{Liudmyla Holenko}

Регіна Бретт \enquote{Будь дивом}

\iusr{Senchyshyn Myroslava}

Біблія, Толкін, Клайв Люїс, Томас Мертон....

\iusr{Olha Ivashchenko}

Маруся Квітки-Основьяненка

\iusr{Oksana Borodenko}

Енн із Зелених Дахів і ... тільки не смійтеся... Мумі Тролі

\begin{itemize} % {
\iusr{Tetiana Protsiv}
\textbf{Oksana Borodenko} а чому сміятись!? В мене це най-най-книга! Перечитуємо з малим всі три книги вже разів п'ять! Я їх просто обожнюю! Вище писала саме про Мумі Тролів!
\end{itemize} % }

\iusr{Yaroslav Diakun}

Сила моменту Тепер, Екгарт Толле

\iusr{Iryna Zvorska}

Абрикосова книгарня
Шлях мирного воїна

\iusr{Тетяна Черниш}

Трилогія перший закон Аберкромбі. Перші книги, які після початку великої війни
змусили мене загубитися у читанні.

\begin{itemize} % {
\iusr{Максим Степанюк}
\textbf{Тетяна Черниш} оо, він прикольно пише. В спину йому дихає Желязни, але Аберкромбі, явно краще. Дійсно норм автор.

\iusr{Тетяна Черниш}
\textbf{Максим Степанюк} хроніки Абмера це мій наступний етап

\iusr{Максим Степанюк}
\textbf{Тетяна Черниш} а, я за них по приколу узявся. 150 грн, то не гроші. Ну я і взяв.
\end{itemize} % }

\iusr{Arsenii Troian}

Книги лише все це забирають, особливо сили. Це все, про що ви зазначили, мені
дає музика

\iusr{Taisiya Nakonechna}

Тревор Гадсон «Про що нас запитує Бог»

\iusr{Bakina Kristina}

Сліди на дорозі Маркус! Книжки Сайгона.

\iusr{Ирина Март}

\enquote{Сад Гетсиманський}, \enquote{Жовтий князь}. Це перше що згадалось.

\iusr{Олена Макарчук}

Джоджо Моєс \enquote{До зустрічі з тобою}. Я ледве не розплакалася, коли виявила її в
евакуаційному наплічнику. Автоматично кинула а потім хопа!

\begin{itemize} % {
\iusr{Svitlana Salamatina-Kovalenko}
\textbf{Олена Макарчук} я плакала і над книгою, і над фільмом. І відмовилася читати продовження...

\iusr{Олена Макарчук}
\textbf{Svitlana Salamatina-Kovalenko} правильно що відмовилися. Зіпсувала

\iusr{Svitlana Salamatina-Kovalenko}
\textbf{Олена Макарчук} не знаю...донька читала,сподобалося. Але я не хотіла.

\iusr{Олена Макарчук}
\textbf{Svitlana Salamatina-Kovalenko} не варто. Нічого не втратите. А я не буду читати продовження Самотності в мережі

\iusr{Olena Laur-Kozko}
\textbf{Svitlana Salamatina-Kovalenko} а книга від фільму дуже відрізняється? Бо я тільки фільм бачила.

\iusr{Svitlana Salamatina-Kovalenko}
\textbf{Olena Laur-Kozko} ні)

\iusr{Олена Макарчук}
\textbf{Svitlana Salamatina-Kovalenko} фільм дивитися теж не буду. Я не люблю таке. Або-або. Якщо читала книгу - не дивлюся фільм. І навпаки

\iusr{Svitlana Salamatina-Kovalenko}
\textbf{Олена Макарчук} я зазвичай теж так)
\end{itemize} % }

\iusr{Жанна Прядко}

Дуже допомогла книга Джоді Піколт \enquote{Книга двох шляхів}. Вчасно. Якраз
після смерті мами.

\begin{itemize} % {
\iusr{Цікава Ґава}
\textbf{Жанна Прядко} обожнюю Піколт. Цю видавав КСД російською чи Ви читали англійською?

\iusr{Жанна Прядко}
\textbf{Цікава Ґава} на жаль, російською. Але авторку - обожнюю.

\iusr{Цікава Ґава}
\textbf{Жанна Прядко} і я. А як книга називалася російською, бо не пригадую такої назви

\iusr{Жанна Прядко}
\textbf{Цікава Ґава} Книга двух путей

\iusr{Цікава Ґава}
\textbf{Жанна Прядко} це точно було КСД чи все-таки Азбука?

\iusr{Жанна Прядко}
\textbf{Цікава Ґава} точно купляла в КДС. Але чи вони видавали, не скажу. Бо зараз не вдома
\end{itemize} % }

\iusr{Valentina Mihaylenko}
\textbf{Джен Ейр} і все (не лише детективи) Агати Крісті

\iusr{Леся Тиркус}

Мирослав Дочинець \enquote{Вічник. Або сповідь на перевалі духу}

\iusr{Світлана Шинкарук}

Гаррі Поттер) з нею я можу відгородитися від усього поганого)

\begin{itemize} % {
\iusr{Olena Tovstopiat}
\textbf{Світлана Шинкарук} я теж🙆

\iusr{Наталія Антоновська}
\textbf{Світлана Шинкарук} і у мене так😄
\end{itemize} % }

\iusr{Юля Кучер}

«Дівчина з тату дракона» Стіг Ларсон

\iusr{Юлія Ільніцька}

\enquote{Бог завжди подорожує інкогніто} Гунеля)

\iusr{Іванна Анчаківська}

Не зовсім книги але є такі журнали Колесо Життя - ще ті мотиватори)))

\iusr{Yulia Tsidylo}

Подивлюся, бо в мене такої книги немає, мабуть... Хоча одна з найпозитивніших
це \enquote{Смажені зелені помідори...}

\iusr{Єлена Макарова}

Книги Агати Крісті

\iusr{Татьяна Тарасюк}

Алхімік

\iusr{Максим Степанюк}

Конан-варвар та Heavy metal \#1 і Етноґрафічний збірник за 1916й

\iusr{Дарина Возна}

\enquote{Хроніки Нарнії} моє все, більше нічого не зустрічала, щоб аж так надихало

\iusr{Оксана Комарянська}

Біблія

\begin{itemize} % {
\iusr{Svetlana Kureeva}
\textbf{Оксана Комарянська} +++ ♥️♥️♥️
\end{itemize} % }

\iusr{Anastasia Fess}

Маленький Принц. Настільки люблю цю історію, що збираю її в різних виданнях.

\iusr{Галина Якименко}

Довгий час це була книга \enquote{Звіяні вітром}.

\iusr{Yulia Mozhova}

Дюна Герберта, саме перша книга. Таємничий острів Жюля Верна. Кульбабове вино
Бредбері. І Гаррі Поттер. Хоча для кожного моменту є \enquote{своя} книга

\iusr{Lory Yas}

Збірка поезій Павла Вишебаби \enquote{Тільки не пиши мені про війну}, актуальчик так би мовити

\iusr{Natalia Skorokhoda}

Рей Бредбері \enquote{Кульбабове вино} та інші твори

Луїза Елкотт \enquote{Маленькі жінки} дві частини

Дж. Роулінг \enquote{Гаррі Поттер} особливо 1 частина

Клайв Льюїс \enquote{Хроніки Нарнії}

Елізабет Гілберт \enquote{Їсти, молитися, кохати}

P S. А ще мене вражає, що під цим дописом є хмара людей зі схожими уподобаннями:)

Не хочу писати список з 10 авторів, бо всі вони є у коментарях інших читачів)

\iusr{Паша Діскаленко}
\textbf{Олена Печорна} \enquote{В затінку земної жінки}. Наріне Абгарян \enquote{Далі жити}

\iusr{Olena Tovstopiat}

Лавр. Сергій Водолазкін (з автором можу помилятися). Проте після прочитання тої
книги, инші не йшли ні до серця ні до голови.

\iusr{Anna Kosaryeva}

Урсула Ле Гуін

\begin{itemize} % {
\iusr{Kim Bommie}
\textbf{Anna Kosaryeva} \enquote{Чарівник Земномор'я} чи якась інша книга? ☺️

\iusr{Anna Kosaryeva}
\textbf{Kim Bommie} і уся трилогія про Зеда (там сам чарівник земноморя, гробниці Атуана, на останньому березі) і весь Хайнський цикл.

\iusr{Ольга Колесникова}
\textbf{Anna Kosaryeva} аввввв я її обожнюю!😍🥰
\end{itemize} % }

\iusr{Ольга Анцибор}

Перечитала майже всі коментарі і була засмучена. Книги українських авторів
майже ніхто не назвав. Тільки кілька людей і то читали класиків. А сучасних
українських авторів ніхто. Шкода. А мені дають наснагу до життя і натхнення до
творчості романи нашої сучасниці, прекрасної письменниці, моєї подруги( чим я
дуже пишаюся ) Валентини Михайленко. Не можу навіть виділити якийсь твір, всі
її романи чудові, захоплюючі, в них саме життя і з радощами і з болем.

\begin{itemize} % {
\iusr{Олена Макарчук}
\textbf{Ольга Анцибор} я читаю переважно сучасну українську літературу. Але суто констатую факт: під руку трапилася ця

\iusr{Natalka Shvchk}
\textbf{Ольга Анцибор} вперше чую це ім'я...

\iusr{Ольга Анцибор}
\textbf{Natalka Shvchk} Яке?

\iusr{Natalka Shvchk}
\textbf{Ольга Анцибор} Валентина Михайленко

\iusr{Ольга Анцибор}
\textbf{Natalka Shvchk} письменниця має в своєму доробку понад 20 романів, була призеркою \enquote{Коронації слова}.

\iusr{Ольга Анцибор}


\ifcmt
  tab_begin cols=2,no_fig,center,separate

     pic https://scontent-frt3-2.xx.fbcdn.net/v/t39.30808-6/329752166_2084333445103910_3685786631642476488_n.jpg?_nc_cat=110&ccb=1-7&_nc_sid=dbeb18&_nc_ohc=ifElgHUjy4UAX_aRUrq&_nc_ht=scontent-frt3-2.xx&oh=00_AfA2NFFFRTKhW5yfC-AZWr2JG7trMHpuMs0HBxHBVwOr5A&oe=641AE39D

		 pic https://scontent-frt3-2.xx.fbcdn.net/v/t39.30808-6/329711295_784030396686884_8572678015006289060_n.jpg?_nc_cat=110&ccb=1-7&_nc_sid=dbeb18&_nc_ohc=nD2vNmU7LdEAX8-D5I7&_nc_ht=scontent-frt3-2.xx&oh=00_AfA_vR6Lq2gvf2-bWfeFI8-jNj7cIWIPdNfKYA9AUWbr6w&oe=641AE681

  tab_end
\fi

\iusr{Ольга Анцибор}

Нещодавно В, Михайленко презентувала свою дитячу книгу \enquote{Скарб під
знаком риби} юним і дорослим читачам в польському місті Катовіце.

\iusr{Natalka Shvchk}
\textbf{Ольга Анцибор} вибачайте, для мене Коронація слова - не знак якості. Це не означає, що я щось маю проти вашого висновку. Не оцінюю те, чого не читала.

\iusr{Галина Степанова}
\textbf{Ольга Анцибор} 

я наприклад тільки слухаю книги. У вашої подруги, Валентини Михайленко немає
аудіозапису. Але... Хочу запропонувати такий варіант. Я давно хочу записати чиюсь
книгу у аудіоформаті. Боюсь, що якщо зроблю запис чиєсь книги без згоди
автора, то по рушу авторське право. Тому, запропонуйте мою ідею своїй
подрузі, можливо вона погодиться.

\begin{itemize} % {
\iusr{Ольга Анцибор}
\textbf{Галина Степанова} 

я навіть не знаю, чи є у Валентини Микитівни аудіокниги. Спитаю. У мене теж є
роман, один поки що. Читачі захоплені ним. Але тут бачу читач з такими
вишуканими смаками, що більше тяжіють до зарубіжної літератури, а я пишу про
непросте кохання простої української жінки. А у Валі я спитаю, тоді напишу Вам.

\ifcmt
  igc https://scontent-frt3-2.xx.fbcdn.net/v/t39.30808-6/329771868_897117951483235_1008522111023756404_n.jpg?_nc_cat=111&ccb=1-7&_nc_sid=dbeb18&_nc_ohc=9Fat5zE2SfIAX_Q1-TJ&_nc_ht=scontent-frt3-2.xx&oh=00_AfAQtGCyZJubsK3hrOp3ArzGBa1IA9BRjA5DfjFTfV8o5A&oe=641B8574
	@width 0.4
\fi

\iusr{Natalka Shvchk}
\textbf{Ольга Анцибор} справа не у вітчизняному, чи зарубіжному. Швидше - жанр мені такий не заходить. До речі, не одне опитування у цій групі показало, що українських авторів ми читаємо дуже багато, особливо в останній рік.

\end{itemize} % }

\iusr{Валентина Кузнецова}
\textbf{Ольга Анцибор} чому ж ніхто? "Абрикосова книгарня" названа, авторка українка.

\iusr{Валентина Кузнецова}

Як же я забула, про Скарлет... Звісно що багато років мене надихали згадки
уривків з книги Віднесені вітром. А ще, Жюль Верн Таємничий острів.

\iusr{Леся Тиркус}
\textbf{Ольга Анцибор} 

Всі ці слова можу сказати про твори Мирослава Дочинця - великого письменника
сучасності❤❤❤ Усі його твори заслуговують уваги. Люблю. Читаю. Чекаю нових
видань.

\iusr{Ольга Анцибор}
\textbf{Леся Тиркус} я теж його люблю.

\iusr{Леся Тиркус}
\textbf{Ольга Анцибор} дякую, що поділилися. Пошукаю книги Валентини Михайленко, бо її книг не читала.

\iusr{Юрій Куций}
\textbf{Ольга Анцыбор} 

Якось, на жаль, не траплялися такі в період, коли я прочитав найбільше
літератури (1987-2002). Зараз читаю рідше, але за списком, \enquote{складеним} у
пам'яті ще в 90-ті... І теж українські автори трапляються рідко - переважно
європейські та американські... Подерв'янського читал - але його творчiсть
викликає бажання спалити... і не лише книги...

Із сучасного українського читав лише Дяченко – але вони, як зараз кажуть,
несправжні українські автори, бо писали російською. Ну і «Східний синдром» Юлії
Ілюхи читав – просто тому, що знаю автора особисто. Ось і все із сучасного
українського...

\begin{itemize} % {
\iusr{Ольга Анцибор}
\textbf{Юрий Куцый} ні, автора не паліть, хай собі матюкається.

\iusr{Юрій Куций}
\textbf{Ольга Анцыбор} це бажання... Не намiр... 🙂

\iusr{Юрій Куций}
\textbf{Ольга Анцыбор} я там трохи дописав комент - трохи все ж читав. Але зовсiм небагато.

\iusr{Ольга Анцибор}
\textbf{Юрий Куцый} дякувати Богу, бо ж людина.

\iusr{Юрій Куций}
\textbf{Ольга Анцыбор} для мене, вибачте, Лесь - щось на зразок мокшанського Юза Олешковського або, боронь Боже, Вєнички Єрофєєва... Має право на існування, напевно - але зовсім не моє...
\end{itemize} % }

\iusr{Алла Безрук}
\textbf{Ольга Анцыбор}, 

зараз багато хороших творів українських авторів, які варті уваги. Але питання
стояло про настільну книгу, про книгу, яка з поміж усіх відрізняється своєю
силою, змістом, мотивацією, яка є для нас книгою життя. Різниця між просто
хорошими книгами відчутна)

\begin{itemize} % {
\iusr{Ольга Анцибор}
\textbf{Алла Безрук} я вважаю, що \enquote{хороша книга} це як і \enquote{правильне життя}. Я
колись купила книгу Ірени Карпа, бо керувала на той час творчим об'єднанням і
хотіла знайомити своїх гуртківців з новою і \enquote{модною} літературою. Стала
читати, їдучи в електричці. І наткнулася на такий крутий мат, і не один, що
почервоніла і стала озиратися, чи ніхто не бачить які гидкі матюки я читаю. А
комусь же це подобається!!! Колись Ірена була в нашому місті і моя донька брала
в неї інтерв'ю. Мене вразили її відповіді настільки, що я навіть не буду тут їх
повторювати. Мабуть я вже несучасна і ніц не розумію, що красиво, а що гидко.
\end{itemize} % }

\end{itemize} % }

\iusr{Roman Topka}

Історія України від Діда Свирида

\iusr{Olga Bespalova}

\enquote{Дівчинка, з якою дітям не дозволяли водитися}, Ірмгард Койн. Дитяча, але про серйозні речі і з чудовим гумором.

\iusr{Цікава Ґава}

Від початку повномасштабного вторгнення і ще місяців 4-5 могла читати лише
ілюстрованого Поттера, особливо ночами чи під час тривог

\iusr{Anna Prysiazhniuk}

Перечитую рідко, але щось подібне є. Наприклад, \enquote{Спитайте Мієчку} - ні разу не
настільна, навіть її не маю, але допомогла мізки переключити з новин весною і
дала втікти у власне safe place. \enquote{Око Світу} повернула бажання дізнатися, що
там далі. А взагалі, Ремарка лікує мені депресію))

\begin{itemize} % {
\iusr{Ira Ch}
\textbf{Anna Prysiazhniuk} те саме з Мієчкою. Прям пам'ятаю як ночами читала і кайфувала вперше від лютого

\iusr{Наталя Кончаківська}
\textbf{Ira Ch} Я теж.
\end{itemize} % }

\iusr{Дана Досяк}
\textbf{Катерина Мотрич} \enquote{Ніч після сходу сонця}. Рекомендую

\iusr{Olga Oguy}

\enquote{Раковий корпус} Солженіцин

\iusr{Ntina Ntoubrova}

Василь Тибель \enquote{Бурштин}. Про таку мразоту в мєнтовських погонах, що коли все
скінчилося хепіендом - було полегшення. Хочеться, якщо екранізують - щоб усюди
де в кадрі будуть менти - поряд ставили портрет Арсена Авакова... ((

\begin{itemize} % {
\iusr{Jana Alhimina}
\textbf{Ntina Ntoubrova} Коментар заслуговує на найвищу оцінку 🤣👍👍👍
\end{itemize} % }

\iusr{Olena Vermizova}

Annie Barrows - The Guernsey Literary and Potato peel pie society

Susanna Kearsley - A desparate fortune

Не те, що я до них повертаюсь, але якось приємний осад залишили і відчуття
надії

\iusr{Наталка Андрощук}
\textbf{Анатолій Дімаров} \enquote{на коні і під конем} ❤️ (на похоронах качура кожен раз ридаю від сміху)

\iusr{Tetiana Protsiv}

Моя книга - це три книги про Мумі-тролів! Насправді познайомилась з ними вже
коли читала молодшому. А в житті моїх дітей вони з'явились з дитинства і я дуже
цьому радію! В нас кожен із сім'ї схожий на когось з Мумі сім'ї! І це така
терапія ті книги! Їх варто читати і в дорослому віці!

\begin{itemize} % {
\iusr{Олена Косовська}
\textbf{Tetiana Protsiv} все чекаю, коли перекладуть решту творів Туве Янссон. В неї настільки гарна проза, дуже шкода, що її досі немає українською. А ще є неймовірна книга про неї Туули Кар'ялайнен. Так хочеться, щоб це все було українською.
\end{itemize} % }

\iusr{Volodymyr Zhuk}

МУР Андрій Любка. Досить позитивно.

\iusr{Nataliya Konovalenko}

Ірвінг Стоун \enquote{Муки і радості}👍

\iusr{Оксана Матвійчук}

Твори Ф.Саган

\iusr{Валентина Кузнецова}

Різдво з червоним кардиналом.
Кульбабове вино.
Ідеальне Різдво для собаки.

\iusr{Oleksandra Holyaka}

Чайка

\iusr{Марина Карпова}

Спитайте Мієчку

\iusr{Олександр Горічев}

Що? Невже ніхто ще не назвав? Буду першим тоді. Сильм. Сильмариліон. \enquote{...серед
плачу знаходимо місце для радості, а під тінню смерті живе довгочасне світло.}

І ще Жан-Крістоф Ромена Ролана.

\iusr{Анна Таланова}

Зараз буде неочікувано... але «Твердиня» Макса Кідрука.

\iusr{Liubov Kovalchuk}

\enquote{Гаррі Поттер} врятував психіку в перші місяці війни. Періодично перечитую
\enquote{Тягар пристрастей людських} Моема, а також \enquote{Темні матерії} Пулмана.

\iusr{Ольга Процик}

Обожнюю Поттеріану! А в плані поштовху до творчості - \enquote{Прислуга}

\iusr{Світлана Єдинець}

Повертають радість всі книги Фанні Флег. \enquote{Смажені зелені помідори} найслабша з
них, на мою думку. Я їх не перечитую, і вони не мої настільні, бо я таке не
дуже люблю.

Але був період в житті, коли читала, і свою функцію вони виконали. Так що
рекомендую. Повертають спокій, радість і віру в людей.

\begin{itemize} % {
\iusr{Natalka Shvchk}
\textbf{Світлана Єдинець} Рай десь поруч 🖤
\end{itemize} % }

\iusr{Надія Омельчук}

\enquote{Гаррі Поттер} однозначно. Завжди дивлюся або читаю, коли смутно зовні
і всередині.

З останнього - дуже зайшла книга \enquote{Дар} Едіт Еґер, читаю з маркером,
страшенно задоволена, що придбала.

\iusr{Kim Bommie}

Енн із Зелених Дахів

\iusr{Liuba Khala}

'Книга про тіло' від акторки Камерон Діаз

\iusr{Лариса Дебеляк-Козак}

Цитадель

Юнні роки. Шлях Шеннона

А. Кронін

\iusr{Оксана Лагойко}

Гаррі Поттер🤩

\iusr{Yuliia Diachenko}

Думаю, шо це \enquote{Прислуга} Кетрін Стоккет. Не те щоб натхнення з неї черпати, але
книга для мене і сумна і весела водночас. І нагадує, шо на зміну темним часам
приходить завжди світло.

Тут дехто пише, шо рятувався в перші місяці війни книгами, і це добре. А я
навпаки не могла нічого читати півроку. Вперше була така пауза, відколи
навчилася читати.

\iusr{Turkovska Zoya}

Дар" Ева Едгер

\iusr{Анна Хома}

\enquote{Розмови з Богом} Ніла Уолша

\iusr{Oksana Kos}

\enquote{Бог ніколи не моргає}

Регіни Бретт

\iusr{Taras Vaschuk}

Міллер і Буковскі. Вони вселяють бажання жити, бо там така срака, що розумієш:
твої проблеми і травми - то ніщо 🤣

\begin{itemize} % {
\iusr{Ekaterina Provozina}
\textbf{Taras Vaschuk} я пам'ятаю свої відчуття від цього))
and now sometimes I'm interviewed, they want to hear about
life and literature and I get drunk and hold up my cross-eyed,
shot, runover de-tailed cat and I say,"look, look
at this!"
but they don't understand, they say something like,"you
say you've been influenced by Celine?"
"no," I hold the cat up,"by what happens, by
things like this, by this, by this!"

\iusr{Taras Vaschuk}
\textbf{Ekaterina Provozina} yep. They ain't got a shit...

\iusr{Олена Косовська}
\textbf{Taras Vaschuk} Поки не дочитала до кінця, то промайнула думка, що я якого не того Буковскі читала, що прямо повертає бажання жити 😆

\iusr{Taras Vaschuk}
\textbf{Олена Косовська} ну спершу він плвертає бажання блювоти, а потім уже жити 🤣
\end{itemize} % }

\iusr{Yaroslav Rudnitsky}

22.11.63 Кінг, Гуситська трилогия Сапковський, Правік та інші часи Токарчук.

\iusr{Natalia Belichuk}

Проект щастя. Гретхен Рубін. Почала читати в червні - зайшла до сестри полити
квіти, і знайшла її на книжковій полиці. Тоді, на 5 місяць війни, дуже хотілось
щастя, щоб не скотитись у прірву відчаю. Поки читала, написала свій проект
щастя, почала вести щоденник одного речення для емоційної розрядки. Недавно
робила ревізію - в мене було 35 пунктів в моєму проекті, за 8 місяців виконала
26, інше - в процесі. Але не стала настільною, бо я віддала її на аукціон. Може
ще комусь пригодиться. Наразі це єдина мотиваційна книга, яка мене змотивувала.
Настільні книги в мене кулінарні.

\iusr{Victoria Skrybka}

А от мене рятувала Таємниця старого Лами, коли я була в передчутті війни.

Оскар Уайльд в тінейджерстві. Згодом Маркес.

Взагалі у кожної книги є свій час

\iusr{Alyana Alya}

Послухаю

\iusr{Алла Саковець}

Мої рекомендації (взяла б ці книги з собою, якби летіла на Марс в одному
напрямку)))):

Улас Самчук "Волинь" - цю книгу просто потрібно!!! прочитати кожному українцю і
після цього уже все в житті буде по-іншому...

Книги Ірен Роздобудько - перечитувала багато разів і щоразу як вперше- Гудзик,
Амулет Паскаля, Все, що я хотіла сьогодні, Шості двері, Дванадцять.

На диво- Люко Дашвар - але лише одна ця її книга - Рай.Центр. Інші її книги-
ні-ні-ні!!!

Гаррі Поттер - особливо перші книги.

Макс Кідрук - Бот, Бот-2

Тетяна Байда-Барбелюк

"Море і соняхи" - не можна відірватись, і це про нашу Одесу і Волинь!!!

Володимир Лис "Століття Якова" - книга справжня і глибока, але в жодному разі
не фільм!!!

Міла Іванцова "Живі книги" - до дірок.

Маркес "Сто років самотності" - було за що дати Нобеля.

Вся Агата Крісті.

Ден Браун "Джерело"

Екзюпері "Маленький принц"

Поезії Ліни Костенко

На диво - ліричний Іван Франко, оце прочитала його "Сойчине крило", ходжу
другий тиждень під враженням! Які толстоєвські і пушкіни??? Коли у нас є такий
Франко?

Ондржей Секора "Муравлик Ферда" - улюблена книга дитинства мого і моїх дітей.

Андерсен - казки "Русалочка" і "Ялинка".

І ще - Брати Капранови, їх книга "Кобзар 2000" (версія для жінок), з нього
оповідання"Катеринка" не відпускає - коротко, але дуже глибоко!

\begin{itemize} % {
\iusr{Natalka Shvchk}
\textbf{Алла Саковець} а чому на Марс не Колонію, а Ботів?))

\iusr{Алла Саковець}
\textbf{Natalka Shvchk} Чомусь дуже мене зачепило, коли читала Бота, це була перша книга з Кідрукових, хоча пізніше багато що читала в нього...

\iusr{Ольга Стащук}
\textbf{Алла Саковець} \enquote{Перехресні стежки} Франка, то є 👍💯‼️

\iusr{Olena Poznikhirenko}
\textbf{Ольга Стащук} так то супер!
\end{itemize} % }

\iusr{Sanya Malash}

\enquote{Українська пунктуація} Світлани Шабат-Савки

\iusr{Natalka Shvchk}

Павло Глазовий. Сміхологія. Не настільна, але точно повертає радість бодай на
мить)) До речі, читала ці гуморески усю вагітність. І мій син прекрасно розуміє
мій гумор)) не знаю, чи це гени, чи це виховання, чи все таки це книжка)))

Загалом, мабуть до тих, які про жити і творити, про баланс я б віднесла саме
поезію. Будь-яку. І за жанром, і за темпом, і за римою. Різноманітна ритміка
строф уже повертає баланс, вводить в медитацію...

\iusr{Julia Methqal}

Колись, 11 років назад, для мене стала такою книга \enquote{Несвятые святые и другие
рассказы}. Зараз дивлюсь на неї і рука не піднімається ні взяти перечитати, ні
викинути. Перечитати не можу, тому що автор російський архімандрит, а всі ми
добре знаємо роль парашної церкви зараз, і викинути не можу, бо там такі чудові
історії, свідчення того, що Бог з нами завжди і що дійсно є чудеса.

П. С. Тільки не сваріть за коментар

\begin{itemize} % {
\iusr{Леся Тиркус}
\textbf{Julia Methqal} це наше життя❤, те, що минуло❤

\iusr{Julia Methqal}
\textbf{Леся Тиркус} от зараз, коли так паскудно на душі, хочеться взяти, перечитати ті оповідання, бо вони дійсно лікують душу, але рука не піднімається
\end{itemize} % }


\iusr{Oksana Lokatyr}

Ліна Костенко💛 Усе, що писала - дуже сильне і для мене найдієвіше в усі часи

\iusr{Євгенія Лисак}

Гаррі Поттер. ці книги і фільми витягали часто з депресії і взагалі- в будь якій
незрозумілій ситуації читаю чи дивлюся) а на другому місці \enquote{Старосвітські
батюшки та матушки}

\iusr{Тетяна Громова}

Володар перснів. Почала вірші перекладати після неї, зроду таким не займалась.
І свого часу Хроніки Нарнії витягли за вуха з депресії

\iusr{Yaroslava Stulova}

Радикальне пробачення, Коліна Тіппінга 🤔

\begin{itemize} % {
\iusr{Daria Morgun}
\textbf{Yaroslava Stulova} о, це та, яку я викинула в смітник)))

\iusr{Yaroslava Stulova}
\textbf{Daria Morgun} кожному своє 😅 А я періодично читаю і мене дійсно мотивує і допомагає)
\end{itemize} % }

\iusr{Nataliya Kovalyova}

Марія Кореллі \enquote{Скорбота Сатани}. В підлітковому віці справила сильне враження.

\iusr{Марія Ружевич}

\enquote{Давайте про позитив...} Що \enquote{давайте}? Скільки можна?

\begin{itemize} % {
\iusr{Zav Andyz}
\textbf{Марія Ружевич} Я теж терпляче чекаю перевидання Книги пам'яті полеглих за Україну.
\end{itemize} % }

\iusr{Надя Сенькевич}

Дж. Уайнхолд та Б. Уайнхолд усі розділи

\iusr{Ірина Токар}

Багато, важко визначити навіть зараз. Від дитинства, довгий проміжок часу, це була книжка "Два капітани".

Улюбленою з улюблених була і лишається "Маруся Чурай" Л.Костенко. Книжка
"Холодне ❤️" В.Піддубного, у певний період часу, дала мені друге дихання. Тож
тепер їх дві )

\iusr{Daryna Bukach}

Біблія

\iusr{Іванна Стеф'юк}

Дуже добре пам'ятаю цей момент. Слава Курилов \enquote{Сам в океані}.

\iusr{Ганна Гановченко}

Книги Світлани талан та букова земля

\iusr{Олег Шапошніков}

\enquote{Сага про Форсайтів}. Там абзаци, від яких перехоплює подих. І оповідання Григора Тютюнника.

\iusr{Олена Мороз}

Кожного разу це нова книжка. Зазвичай, мене рятує Кінг, бо після жахів
навколишнє не таким жахливим здається. В юності (в достт вже не дитячому віці)
мене рятували казки народів світу, а зараз не можу казки читати.

\begin{itemize} % {
\iusr{Світлана Єдинець}
\textbf{Олена Мороз} я після читання Кінга починаю сильно вірити в Бога. Серйозно.
\end{itemize} % }

\iusr{Стефанія Томин}

\enquote{Сам в океані} Курилова

\iusr{Мар'яна Шерстило}

\enquote{Тигролови} сама життєствердна історія

Дає сили і віру в те що нічого не має неможливого

\iusr{Стас Масельский}

Даррелл, Херріот, анімалістика взагалі. Та взагалі гарна, захоплююча книга

\iusr{Svetlana Kureeva}

Біблія: особливо Книга Псалмів, Євангелія, весь Новий заповіт, Пісня над піснями, Притчі Соломона ♥️♥️♥️

Ми читаємо і дитячу за щоденним планом і звичайну )))

\iusr{Аня Мельничук}

Поезія Ліни Костенко.

Маленький принц.

\iusr{Танюха Даниленко}

Ірен Роздобудько "Все, що я хотіла сьогодні"

\iusr{Даша Глушак}

У мене таких авторів два: це Ремарк і Кінг. Якщо в житті повна лажа, беру котрогось із них!

\iusr{Катерина Костюк}

Одеська Сага від Ю. Верба.

\iusr{Наталя Чорна}

Усі книги Мирослава Дочинця.

\iusr{Оксана Олейник}

Поезія Ліни Костенко.

\iusr{Kateryna Solovyova}

"Подорож до Шамбали" Юрги Іванаускайте. Неймовірно крута історія!

\iusr{Hanna Marysh}

Алхімік✨

\iusr{Таня Лазарук}

Елеонор Портер "Поліанна" - книга про гру в радість, найпозитивніша. Рекомендую
для читання всім.❤️

\begin{itemize} % {
\iusr{Oleksa o'Zori}
Ой, повністю підтримую! Головне правило Поліанни мене не раз виручило в житті ❤
\end{itemize} % }

\iusr{Пуделко Марія}

Фільм ,,Знайомтесь, Джо Блек.,,з Бред Пітт іЕнтоні Хопкінс..А книг кілька, в кожний час різні.

\iusr{Валентина Зайцева}

Книга з українською душею.прочитала " на одному подиху"

\ifcmt
  igc https://scontent-fra3-1.xx.fbcdn.net/v/t39.30808-6/329876440_910904887022825_4579908431300176683_n.jpg?_nc_cat=102&ccb=1-7&_nc_sid=dbeb18&_nc_ohc=PA4RXUx5ucMAX8V4kpI&_nc_ht=scontent-fra3-1.xx&oh=00_AfD9XnikYQqx1DWYND_d-TQR4k8havLHPav39He5sIdI6A&oe=641AA1F1
	@width 0.4
\fi

\begin{itemize} % {
\iusr{Julia Methqal}
\textbf{Валентина Зайцева} у неї є книга Дивна така любов. Дуже сподобалася! Цю візьму до waiting list, дякую

\iusr{Валентина Зайцева}
\textbf{Julia Methqal} а я ✏Дивну любов 😉, Вітрова гора більш для підлітків, але читається 👌
\end{itemize} % }

\iusr{Людмила Павлоцкая}

Елінор Портер «Полліанна».

\iusr{Марина Рудницька}

«Вільгельм Майстер» Гете

\iusr{Iryna Oleksandrivna}

"Прогулянка" Роберт Вальзер в перекладі Андруховича.

Це щось настільки просте і позаземне водночас. Обожнюю цю книгу❤️

\iusr{Natalya Lapina}

У погані часи перечитую новели О'Генрі

\iusr{Юрій Козьмін}

Джеральд Даррелл 🙂

\iusr{Валентина Кузнецова}

Як же я забула, про Скарлет... Звісно що багато років мене надихали згадки
уривків з книги Віднесені вітром. А ще, Жюль Верн Таємничий острів.

\iusr{Anna Daineko}

Джейн Ейр, Малюк і Карлсон, Біблія і поезія Halyna Kruk

\iusr{Oksana Starko}

Минулого року читати не могла... Але наприкінці, в грудні, прорвало - на одному
подиху \enquote{ковтнула} книгу молодої української авторки Анни Грувер \enquote{Її порожні
місця}. Це щось надзвичайне! А потім побачила і придбала в книгарні \enquote{Вибрані
афоризми} Григорія Савича, проілюстровані Олександром Ройтбурдом - книга на всі
часи, під різний настрій і стан душі... І прорвало! Зараз читаю Гіларі Мантел
\enquote{Вулфголл}.

\iusr{Світлана Кравчук}

Чарльз Мартін "Гора між нами"

\iusr{Оксана Токарська}

"Овод" у 13 років. У далеких 80-х...

\iusr{Юрій Куций}

Мої найперші книги (прочитані у 7-8 років), завдяки яким я люблю і тварин, і
літературу, це \enquote{Моя сім'я та інші звірі} Даррелла, \enquote{Про всіх створінь — великих
і малих} Герріота та \enquote{Плямистий сфінкс} Джой Адамсон. Це - i досі мої настільні
книги...

\iusr{Grazhyna Martseniuk}

Сергій Жадан!... будь яка збірка поезії... кожна у мене з закладочками на улюбленому❤️

\iusr{Nataly Korotenko}

Новели С. Цвейга

\iusr{Ярина Піддубняк}

Не те, щоб настільна, але \enquote{Я, Побєда і Берлін} мені подобається. Я відволікаюсь
від негативу. Із останніх перечитаних - \enquote{Повія}, тепер думаю, який гарний серіал
б зняли за цим твором.

\begin{itemize} % {
\iusr{Ирина Гузий}
\textbf{Ярина Піддубняк} екранізація з Л.Гурченко теж цікава
\end{itemize} % }

\iusr{Олеся Шостак}

"Усиновлена" Nelya Romanovska

\iusr{Maksym Petiakh}

джоді піколт "чуже серце"🤗🦋 рекомендую

\begin{itemize} % {
\iusr{Олена Макарчук}
\textbf{Maksym Petiakh} читала її дві книги. Про хлопця з аутизмом і про педофіла в рясі

\iusr{Maksym Petiakh}
\textbf{Олена Макарчук} а там будь яку бери її книгу і крутезна🤗 мотивуюче чтиво

\iusr{Інна Яцюра}
\textbf{Maksym Petiakh} це точно мотивуюче... 🤭

\iusr{Олена Макарчук}
\textbf{Інна Яцюра} а Ви дарма так. Книги реально круті. Добре що потроху перекладають українською. А до цього були лише російською.

\iusr{Maksym Petiakh}
\textbf{Інна Яцюра} хм, кожному своє🤔
\end{itemize} % }

\iusr{Alla Trushyk}

"Мандрівний замок Хаула"

\iusr{Олександра Решотко}

Мирослав Дочинець "Криничар"

\iusr{Ирина Гузий}

Богдан Коломійчук всі книжки про пригоди комісара Вістовича.

\iusr{Інна Самошевська}

Поліанна 🌷

\iusr{Інна Яцюра}

Дочитую В. Шкляра "Марусю". Бачу повторення сьогоднішніх подій, що відбувалися
у 1918 році... Ті самі біди.

\iusr{Надія Козак}

Ця книга не те щоб залишила глибокі враження, а докорінно змінила моє життя і
тепер вже важко сказати, звідки досі беруться сили.

\ifcmt
  igc https://scontent-frt3-2.xx.fbcdn.net/v/t39.30808-6/329464159_783880746431035_5971672341925443658_n.jpg?_nc_cat=110&ccb=1-7&_nc_sid=dbeb18&_nc_ohc=7UXCdAR-9aQAX8wODoS&_nc_ht=scontent-frt3-2.xx&oh=00_AfB-y3T-ShfksJGsJlLmMydjH4T0N9aRaYtpE6iGfxayeQ&oe=641A007A
	@width 0.4
\fi

\iusr{Мирослава Барабаш}

Галина Вдовиченко \enquote{Пів'яблука. Інші пів'яблука} - люблю ці книги за атмосферу,
за дружбу, за тиху радість, яку вони мені дають. А ще, з донечкою, читаємо і
перечитуємо, теж від пані Галини, \enquote{Містельфи} та серію про 36 і 6 котів.

\iusr{Наталія Шварц}

Завжди повертає до життя \enquote{Триб} Мирослава Дочинця, нашого сучасного письменника
із Закарпаття, про життєву філософію столітнього мудреця Андрія Ворона.

\iusr{Maksym Petiakh}

Джозеф Кемпбелл \enquote{Тисячоликий герой}. 😉 цікавеньке)

\iusr{Олена Косовська}

Коли нічого не хочеться, і навіть читання не приносить радість (а часто це
єдине, що тримає психіку на плаву), я читаю Бредбері. От взагалі, не важливо,
що я в нього читаю, але він настільки повертає мені бажання жити і щось робити.
Ще Курт Воннеґут, але то трохи про інше, у нас з ним особливий вид медитації,
але якщо нічого читати не хочеться, то його хочеться завжди, але з ним не
з'являється бажання жити, з ним навпаки замотуєшся в кокон, але тоді
відрождуєшся, так що можна сказати, що також своєрідне повернення до бажання
жити. Туве Янссон та Астрід Ліндґрен мають особливий вплив на мене, дають ту
життєдайну енергію і віру в добро, яких так часто не вистачає. Але щоб виділити
якусь конкретну книгу, мабуть, не зможу. В нас більше з авторами особливі
стосунки.

\iusr{Оленка Чижмар}

Сестра Емануель \enquote{Меджугор'є. Торжество серця};\enquote{ Звіяні вітром}
і \enquote{Повнолітні діти}

\iusr{Zav Andyz}

Моріс Метерлінк \enquote{Мудрість і доля}

\iusr{Креденець Наталія}

Дочинець! Мирослав.. читала запоєм все, що змогла дістатина на той час: «Лад»,
«Криничар», «Вічник», «Горянин» .. Розчарував тільки трохи « Мафтей». З
теперішніх яскравих вражень - Додж Бату « Моцарт».

\iusr{Onyshchuk Alex}
\textbf{Ольга Токарчук} \enquote{Книги Якова}, Юрій Рибчинський \enquote{Чорні морелі}, Гашек \enquote{Швейк}.

\iusr{Larisa Lemeha}

\enquote{Чаликушу Корольок-пташка співуча}

Автор Решат Нурі Ґюнтекін
