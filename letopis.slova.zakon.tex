% vim: keymap=russian-jcukenwin
%%beginhead 
 
%%file slova.zakon
%%parent slova
 
%%url 
 
%%author 
%%author_id 
%%author_url 
 
%%tags 
%%title 
 
%%endhead 
\chapter{Закон}
\label{sec:slova.zakon}

%%%cit
%%%cit_pic
%%%cit_text
Максим, \enquote{надо определиться с правовым статусом}? Это - как? Вы не
учитываете того факта, что наука История - это не наука Физика, в которой раз и
навсегда появившиеся (открытые людьми) \enquote{законы} действительны как в
настоящем, так и в прошлом, так и будут действовать и в будущем, разве что
только расширяясь и развиваясь дальше, по мере расширения знаний в этой
\enquote{естественной} науке (\emph{Закон сохранения Материи}, например). Но в
истории людей законы в их обществах пишет не Природа (естество), а люди.
Поэтому в науке Право нет тех \emph{Законов}, которые действуют сейчас, но действовали
в прошлом и будут действовать и в будущем тоже. Именно эта особенность
\enquote{человеческих} (гуманитарных и юридических) \emph{Законов} закреплена в
одном из \enquote{базовых} принципов этой науки - \emph{Закон} не имеет обратной силы.
Иными словами нельзя (!) с точки зрения современного Права рассматривать и
оценивать события и действия людей в прошлом, когда они жили и действовали в
другом \enquote{правовом поле}, и согласно действовавшим в то время \emph{Законам} и
правилам, были вполне \enquote{законопослушными} людьми, не делавшими ничего
\enquote{предосудительного}.  Но это не означает того, что человек и
человеческие общества развиваются \enquote{\emph{беззаконно}}, т.е. совершенно
случайно и хаотично. Это означает только то, что люди, не смотря на все свои
попытки в прошлом и настоящем, просто еще не открыли эти \enquote{базовые}
\emph{Законы} своего развития, действовавшие в прошлом, действующие в
настоящем, и которые будут действовать и в будущем. Но эксперимент под
названием \enquote{Жизнь на планете Земля} еще не закончен, поэтому шанс
открыть эти \emph{Законы} у людей тоже еще сохраняется
%%%cit_comment
Алексей Джалалов
%%%cit_title
\citTitle{Кто основал Москву? Вы даже удивитесь - ее основал великий киевский князь в XII веке и заселил ее украинцами}, 
Исторический Понедельник, zen.yandex.ru, 18.02.2021
%%%endcit

%%%cit
%%%cit_head
%%%cit_pic
%%%cit_text
В общем, если хотите, можете сами глянуть отчёт на 100 с лишним страниц. Ну и
как вы поняли, Украина \enquote{образцово демократическое} це-европейское государство,
в котором полным ходом внедряют европейские ценности, соблюдают права человека
и лучше всех знают как нужно поступать с основным \emph{законом} страны -
Конституцией. Правильно, \emph{основной закон}, по-украински, годится только на то,
чтобы в него рыбу заворачивать.  Выводы делать вам. Благодарю за внимание
%%%cit_comment
%%%cit_title
\citTitle{Speak на мове please}, 
Terra Incognita, zen.yandex.ru, 30.04.2021
%%%endcit

%%%cit
%%%cit_head
%%%cit_pic
%%%cit_text
Да и новые \emph{законопроекты} \enquote{слуг народа} и правительства
относительно будущего Донбасса прямо сужают права для жителей \enquote{ДНР} и
\enquote{ЛНР}, если они вернутся в состав Украины. Например, таков
\emph{законопроект} о \enquote{коллаборантах}. Также граждан Украины, у
которых найдут российский паспорт, Кабмин предложил интернировать - то есть
поступать с ними, как с военнопленными и без всякого суда изолировать. Это
избирательное поражение в правах, что прямо запрещено Конституцией
%%%cit_comment
%%%cit_title
\citTitle{День Конституции Украины 28 июня - какие статьи нарушаются сильнее всего}, 
Оксана Малахова; Максим Минин, strana.ua, 28.06.2021
%%%endcit

%%%cit
%%%cit_head
%%%cit_pic
%%%cit_text
Одиозный закон все-таки признали конституционным.  Какая-то чиновница
Конституционного суда с характерной фамилией Ковбасюк радостно сообщила нам,
что запуганные властью судьи признали нормальным жуткий закон о тотальной
украинизации. При этом даже авторы этого треша несколько лет признавали
неконституционность своего творения и призывали не дать КСУ даже браться за его
анализ.  Потом слуги под давлением нациков отказались от своих несмелых попыток
слегка смягчить самые одиозные формы дискриминации русскоязычных
%%%cit_comment
%%%cit_title
\citTitle{Украинизаторы активно машут лопатой на похоронах своей страны}, 
Олег Волошин, strana.ua, 14.07.2021
%%%endcit
