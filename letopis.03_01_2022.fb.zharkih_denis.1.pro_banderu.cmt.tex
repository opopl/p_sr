% vim: keymap=russian-jcukenwin
%%beginhead 
 
%%file 03_01_2022.fb.zharkih_denis.1.pro_banderu.cmt
%%parent 03_01_2022.fb.zharkih_denis.1.pro_banderu
 
%%url 
 
%%author_id 
%%date 
 
%%tags 
%%title 
 
%%endhead 
\zzSecCmt

\begin{itemize} % {
\iusr{Raisa Savchenko}
Перепгнут. Перекрасятся.
Развернуться. Поклонятся.
Изогнутся. А аромат какой будет!?!?

\iusr{Сергей Нефедьев}

Ну, не скажите, Денис, не скажите! В.И.Ленин написал громадное количество
статей и писем. Посмотрите на полное собрание его сочинений, сколько томов! А
что написал С.Бандера? С наследием В.И.Ленина работать проще: один план ГОЭЛРО
можно было разбирать пол-семестра. А что делать с Бандерой? Здесь нужна
творческая работа - додумать самому, что же хотел \enquote{построить Степан}? Почему у
него не получилось что-то построить? Кто ему помешал? Нет! Что не говорите, а с
Бандерой гораздо сложнее работать идеологам. Наверное потому у них такие
зарплаты...

\begin{itemize} % {
\iusr{Сергей Никонов}
\textbf{Сергей Нефедьев} , 

это для государственников проще работать с Лениным. Им нужны идеи
государственного строительства, которые не даст Бандера А для тех, кто просто
хочет управлять толпой нужны Гитлер, Бандера или идеолог охоты на каких-то птиц
в Китае. Мол, перебьем каких-то людей или животных и заживем хорошо. Бандера в
основном делал то, что разрешали нацисты. Они же ничего особого не строили.
Разве-что для нужд Германии. У меня свои версии насчет зарплат идеологам: мало
отпетой сволочи и абсолютно беспринципных а грязной и лживой агитационной
работы много,.2. Они работают не только над идеологией. 3. Распиливают
средства. 4. Стрессовая у них работа. Посудите сами: \enquote{проклятые} Шарии, Путин
со своими сложными идеологиями, \enquote{пятая колонна} из честных, но умных, народ все
более трудно внушаемый пошел.


\iusr{Сергей Нефедьев}
\textbf{Сергей Никонов} Ну, я же с сарказмом написал...

\iusr{Сергей Нефедьев}
\textbf{Сергей Никонов} 

Да, окромя всего, что вы перечислили, нужно же ещё и убедить охлос, что им
живётся сегодня лучше, чем когда-либо. Многие верят...
\url{https://youtu.be/JnhA22gWu60}

\end{itemize} % }

\iusr{Юрий Александров}

Бандеровцам всё таки нужно место где бы они жили и пользовались своей
идеологией, надо место куда бы их можно было бы высылать

\begin{itemize} % {
\iusr{Сергей Вартанян}
\textbf{Юрий Александров} КАК ВЫ СМОТРИТЕ НА БИРОБИДЖАН

\iusr{Igor Bog}
\textbf{Сергей Вартанян} Польша ближе и территориально и идеологически... И работа уже есть!

\iusr{Сергей Вартанян}
\textbf{Igor Bog} какой то вы не гуманный к полякам

\iusr{Igor Bog}
\textbf{Сергей Вартанян} почему? Там уже полно этого добра прижилось, просто и семьи подтянутся!

\iusr{Сергей Вартанян}
\textbf{Igor Bog} повезло полякам с исторической памятью не то что у галичан

\iusr{Igor Bog}
\textbf{Сергей Вартанян} ... Да там примерно то же самое! Историческая память исключительно соответствующая актуальной повестке...

\iusr{Сергей Вартанян}
\textbf{Igor Bog} против москалей, я правильно понял

\iusr{Igor Bog}
\textbf{Сергей Вартанян} ну на данный момент да, ещё и коммунистов! А вообще-то они всех ненавидят!

\iusr{Сергей Вартанян}
\textbf{Igor Bog} все им должны

\iusr{Igor Bog}
\textbf{Сергей Вартанян} улавливаете основное сходство идеологии?  @igg{fbicon.wink} 

\iusr{Сергей Вартанян}
\textbf{Igor Bog} вспомнил ЭТО 

\url{https://www.youtube.com/watch?v=YCpeU6_2gAw}

\iusr{Юрий Александров}
\textbf{Сергей Вартанян} нет, только не туда

\end{itemize} % }

\iusr{Сергей Вартанян}

меня удивил наш историк коммунистической партии в институте, который после
развала СССР стал ярым почитателем УПА

\iusr{Антон Колесников}
\textbf{Сергей Вартанян} 

Среди моих бывших преподавателей и коллег подобных персонажей оказалось более
чем. Конформисты-флюгеры и просто политические проститутки.

\iusr{Сергей Вартанян}
\textbf{Антон Колесников} именно, что политические проститутки - всем дают кто заплатит

\iusr{Даниил Богатырёв}

Да, Вы правы. Сейчас поймал себя на мысли, что из всех советских философов
по-настоящему значимым и выдающимся могу назвать, разве что, Александра
Зиновьева. Да и тот диссидент. Разумеется, в советские годы творил и
по-настоящему выдающийся философ Алексей Лосев. Но он, строго говоря, был не
советским, а русским философом, так как продолжал традиции русской религиозной
философии, а университет закончил ещё до революции - в 1915 году. Вот и
получается, что довлеющая над философским образованием в СССР марксистская
идеология, в её догматизированном виде, отнюдь не способствовала раскрытию
настоящих талантов, вместо них выдвигая на первые роли посредственных
конформистов.


\iusr{Сергей Вартанян}
\textbf{Даниил Богатырёв} четко в десятку - вместо них выдвигая на первые роли посредственных конформистов - приспособленцев-ренегатов

\iusr{Евгения Савицкая}

Подождете? Увы. Поздно уже. Время ушло. И если не случится воссоединения двух
братских народов, как кому то хочется, развал страны неизбежен.

К слову сказать, братским я бы взяла в кавычки, п.ч мало кто уже так считает. И
по праву - после известных событий с 14 года по и по сей день.

\begin{itemize} % {
\iusr{Igor Bog}
\textbf{Евгения Савицкая} За семь лет оголтелой пропаганды, можно не только братьев, а и родителей ненавидеть приучить!  @igg{fbicon.wink} 

\iusr{Евгения Савицкая}
\textbf{Igor Bog}
Вы правы.

\iusr{Петр Вдович}
\textbf{Евгения Савицкая} я тоже считаю что точка бифуркации уже пройдена.
Объединение страны невозможно.

\iusr{Евгения Савицкая}
\textbf{Петр Вдович}
А это и не случиться. Невозможно. Слишком много беды....

\end{itemize} % }

\iusr{Олександр Михельсон}

как-то сильно этот пост сводится к примитивизации, котоаря (якобы) шибко не
нравится автору. впрочем, ведь в тексте справедливо замечено, что поклонение
идолам - оно всегда одинаковое, независмо от цвета идолов.

\iusr{Vogel Paul}

Украинская наци философия очень нудная и не то что вторичная, а пятиричная,
если можно так сказать. Я вообще без понятия, что там привлекает людей?
Философия \enquote{ауе} и то более романтична и обаятельна. А зло как известно обязано
быть обаятельным.

\iusr{Прошкин Виктор}
Хороший текст!

\iusr{Алексей Папушников}

Приспособленцы без морали и стержня всегда были, во все времена - им по
барабану кому орать \enquote{Слава}, они всегда выбирают победителя и продают, когда
чувствуют нового, сильнее прежнего. А держится всегда на простых людях со
стержнем, которые не изменяют своим моральным, человеческим ценностям. Это они
на своих плечах переносят все тяготы страшного времени, остаются всегда верны
своим убеждениям. Те кто афиширует свою смену убеждений - типичный
приспособленец.

\iusr{Сергей Анпилов}
Идолов сотворили...

\iusr{Сергей Вартанян}
\textbf{Сергей Анпилов} 

Не сотвори себе кумира — жить своей головой, не поклоняться бездумно кому-либо,
любой идее, мысли, не следовать моде, быть собой, иметь собственное мнение, ..


\iusr{Сергей Вартанян}
продадут они того Бандеру, а, главное. Украину и украинский народ за хороший харч.

\iusr{Евгений Юрьевич}
Самое важное - это нажиться. Ограбить и убить прикрываясь национальной идеей.
Кого не убили - пороботить, удовлетворяя свои животные инстинкты.
Все просто.

\iusr{Антон Колесников}

Мы с тобой, Денис, учились на одном факультете и примерно в одни и те же годы,
причем в отличие от тебя, я вообще учился на самом суперидеологическом
отделении нашего идеологического факультета (нк). Так что с горечью вынужден
согласиться с твоими наблюдениями, при том, что мне лично было не наплевать,
что учить. Но таких, как я, было абсолютное меньшинство - хоть среди
преподавателей, хоть среди студентов. Увы.

\iusr{Евгений Юрьевич}
\textbf{Антон Колесников} О... НК..

\iusr{Владимир Епифанов}
и где же такой бессовестный вуз ?

\iusr{Николай Терещенко}
Точно подмечено. От перемены мест слагаемых, совесть не меняется. Она или есть, или её нет.

\iusr{Нина Спиридонова}
Телепрограммы им сменить - недели хватит

\iusr{Виктория Дубицкая}
Это Украина. В Москае все же было не так.

\begin{itemize} % {
\iusr{Денис Жарких}
\textbf{Виктория Дубицкая} А что учение Маркса-Энгельса-Ленина отстояли? СССР сохранили? Не знал. Вот оно чо, Михалыч...

\iusr{Виктория Дубицкая}
\textbf{Денис Жарких} Нет, учение Маркса - это отделение Научного коммунизма. Это было дно. Наверное. А на философском факультете преподавали интеллектуалы. За этим на факультет и шли.

\iusr{Денис Жарких}
\textbf{Виктория Дубицкая} У нас они тоже были. Только это не решило ничего.

\iusr{Виктория Дубицкая}
\textbf{Денис Жарких} С марксизмом все сложнее, его просто уже не было. Была схема без смысла. Его и за 30 лет существования компартии без СССР не появилось. У нас. А на западе есть.
\end{itemize} % }

\iusr{Николай Гончаров}

Денис ! Я и раньше подозревал что в Киеве даже при СССР с политическим
образованием была полная задница. Особенно когда киевское военно-морское полит
училище перекрасилось в соросовско-могилянскую академию. Да и это не мудрено,
главный укр идиолог кравчук. Мне посчасливилось учиться в Актюбинском ВЛУ ГА.
Кроме специальных дисциплин которые были на довольно высоком уровне у нас была
очень сильная кафедра общественных дисциплин. Её сила заключалась в её
интернациональности. Историю КПСС преподовал Максим Мажитович Мергенов казах.
МЛ философию Ли Хан сен Григорьевич кореец. Политическую экономию Фаттахова
Алсу Вазыховна татарка. И именно там нам популярно объяснили что такое
пантюркизм и панисламизм, именно там я узнал о \enquote{подвигах} Украинской ГКЦ (
униатской ) во время ВОВ и за что её прикрыли в СССР. Так что политический идол
в лице бандеры это просто раздражитель в большой политической игре, который
хорошо подходит для необразованных и слабообразованных ( хуторянских ) масс.
Такой же как при кравчуке был Тарас Григорьевич и \enquote{золото полуботка}. И я
думаю что вскоре подорожают \enquote{фарбы и пендзлики}, патриоты начнут
перекрашиваться.

\begin{itemize} % {
\iusr{Петр Вдович}
\textbf{Николай Гончаров} нет, не начнут.

\iusr{Николай Гончаров}
\textbf{Петр Вдович} Начнут, ещё как начнут. Как бывшие украинские \enquote{коммунисты} в 1991.
\end{itemize} % }

\iusr{Денис Жарких}
Не знал, что все держалось на Актюбинске.

\begin{itemize} % {
\iusr{Николай Гончаров}
\textbf{Денис Жарких} А кто сказал что держалось ? Подход к обучению был другой, благодаря кадрам.

\iusr{Денис Жарких}
\textbf{Николай Гончаров} У нас тоже были и хорошие кадры, но это не меняло систему.

\iusr{Николай Гончаров}
\textbf{Денис Жарких} 

Подход был в ВУЗе другой, интернациональный. За \enquote{чурка} или \enquote{хохол} можно было
и из училища вылететь, если такое при \enquote{батьках} ляпнуть. Это не \enquote{понадусе}.

\end{itemize} % }

\iusr{Кира Берестенко}
Цинично, но точно.

\iusr{Петр Вдович}
И да... и нет.
В первой части... согласен. И касательно идеологий Ленина и Бандеры и их оценок.
Во второй... нет, не согласен. Объединение страны уже невозможно.

\iusr{Игорь Филиппович Шпилей}
\textbf{Петр Вдович} так страна и не разъединена, кроме оккупированной части. Зачем объединять то, что едино. Бред кремлебота.

\iusr{Руслан Паличев}

100\%! Изучение сочинений Маркса-Энгельса-Ленина-Сталина требует времени, ума и
огромного труда. А сочинения Бандеры вообще существуют? Можно вообще не
напрягаться!)

\iusr{Мирослава Александровна Бердник}

Очень точно

\iusr{Игорь Филиппович Шпилей}

Я тоже учился в идеологическом вузе, при чем коммунистическом в кубе, но меня
не удалось так жестко зазомбировать, как профнепригодного политтехнолога Денис
Жарких

\iusr{Елена Клименко}

Мої батьки жили при різних ідеологіях, збагачується тільки той, хто краде, при
любій владі, a от чи довго буде існувати країна при таких апетитах, то питання
часу.

\iusr{Elena Raygorodetsky}

Ну почему вопрос не в Бандере? Как раз в нем, в его геноцидной идеологии -
примитивной, как вы справедливо пишете, до ужаса, но и до ужаса притягательной.
Нам пофиг, кто и как набивает карманы, нам слишком часто об этом напоминают те
же бандеровцы, нас волнует гуманитарная политика - язык. культура, история. А
єто славянское триединство полностью разрушают бандера и бандеровцы, да к тому
же пропагандируют вечный, удобный, эффективный геноцид. Так что в нем именно, в
Бандере все дело.

\iusr{Владимир Павлович}
Супер

\iusr{Виктор Цыганов}

вы точно перепутали факультеты... философский был один из последних по благам
которые удавалось выпросить у власти.... что же касается сравнения то они
всегда хромают... ваши - не исключение

\end{itemize} % }
