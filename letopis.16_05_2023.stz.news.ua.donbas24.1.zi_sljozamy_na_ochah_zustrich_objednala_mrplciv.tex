% vim: keymap=russian-jcukenwin
%%beginhead 
 
%%file 16_05_2023.stz.news.ua.donbas24.1.zi_sljozamy_na_ochah_zustrich_objednala_mrplciv
%%parent 16_05_2023
 
%%url https://donbas24.news/news/zustric-z-navoyu-objednala-mariupolskix-pereselenciv-z-riznix-mist-ukrayini-video
 
%%author_id demidko_olga.mariupol,news.ua.donbas24
%%date 
 
%%tags 
%%title Зі сльозами на очах — зустріч, що об'єднала маріупольців
 
%%endhead 
 
\subsection{Зі сльозами на очах — зустріч, що об'єднала маріупольців}
\label{sec:16_05_2023.stz.news.ua.donbas24.1.zi_sljozamy_na_ochah_zustrich_objednala_mrplciv}
 
\Purl{https://donbas24.news/news/zustric-z-navoyu-objednala-mariupolskix-pereselenciv-z-riznix-mist-ukrayini-video}
\ifcmt
 author_begin
   author_id demidko_olga.mariupol,news.ua.donbas24
 author_end
\fi

\ii{16_05_2023.stz.news.ua.donbas24.1.zi_sljozamy_na_ochah_zustrich_objednala_mrplciv.pic.front}
\begin{center}
  \em\color{blue}\bfseries\Large
  Маріупольці у Києві подякували захисниці свого міста
\end{center}

13 травня у Червоній залі Будинку кіно відбулася \href{https://archive.org/details/10_05_2023.olga_demidko.donbas24.polonjanka_azovstal_valeria_subotina_rozskazhe_odruzhennja_polon}{\emph{творча зустріч}}%
\footnote{Полонянка з Азовсталі Валерія Суботіна розкаже про одруження під обстрілами та полон — подробиці, Ольга Демідко, donbas24.news, 10.05.2023, \par%
\url{https://donbas24.news/news/polonyanka-z-azovstali-valeriya-subotina-rozkaze-pro-odruzennya-pid-obstrilami-ta-polon-podrobici}, \par%
Internet Archive: \url{https://archive.org/details/10_05_2023.olga_demidko.donbas24.polonjanka_azovstal_valeria_subotina_rozskazhe_odruzhennja_polon}%
} з поетесою, журналісткою та захисницею Маріуполя Валеріює Суботіною (Карпиленко) на
позивний Нава. Цей захід зібрав не тільки друзів, знайомих, посестер і
побратимів Валерії, він об'єднав маріупольців, які були в Києві чи приїхали зі
Львова, Івано-Франків\hyp{}ська, Дніпра, щоб подякувати мужній і сталевій дівчині,
яка не так давно вийшла з полону.

\ii{insert.read_also.elina_prokopchuk.istoria_stalevoj_navy}
\ii{16_05_2023.stz.news.ua.donbas24.1.zi_sljozamy_na_ochah_zustrich_objednala_mrplciv.pic.1}

Друзі Валерії виходили на сцену, аби поділитися життєвими історіями, які їх
поєднують з військовослужбовицею, зачитували її вірші зі збірки \enquote{Квіти і
зброя}. Також присутні присвячували жінці власні поезії, натхненням для яких
стала вражаюча сила Нави та кохання подружжя Суботіних. Всіх присутніх до
глибини душі вразив музичний відеокліп \enquote{Кохання сталь}. Пісню, присвячену
драматичній історії кохання, виконала народна артистка України \href{https://archive.org/details/15_05_2023.olga_demidko.donbas24.muzychnyj_videoklip_kohannja_stal}{\emph{Марія Бурмака}}.%
\footnote{У Києві презентували музичний відеокліп \enquote{Кохання сталь}, Ольга Демідко, donbas24.news, 15.05.2023, \par%
\url{https://donbas24.news/news/u-kijevi-prezentuvali-muzicnii-videoklip-koxannya-stal}, \par%
Internet Archive: \url{https://archive.org/details/15_05_2023.olga_demidko.donbas24.muzychnyj_videoklip_kohannja_stal}%
}

Маріупольці приїхали з рідних міст України, щоб зустрітися з Валерією та
особисто подякувати їй за самовідданість, неабияку мужність і захист. На цьому
заході вони зустрілись з багатьма співмістянами і були дуже розчулені такою
зустріччю.

\begin{leftbar}
\emph{\enquote{Я переповнена щастям від зустрічі з такими рідними обличчями. Наче знову в
Маріуполі. І головне, я зустрілася і обнялася зі Сталевою Незламною
Навою!}}, — підкреслила маріупольчанка Вікторія Константіновська.
\end{leftbar}

\begin{leftbar}
\emph{\enquote{Безмежно щаслива від того, що ми всі побачились! Я приїжджала до Києва
спеціально на цю зустріч! Вже вдома, на Львівщині, а думками і далі з кожним з
вас!}}, — зазначила поетеса Надія Умриш.
\end{leftbar}

\begin{leftbar}
\enquote{Чудовий вечір. Дуже хвилюючий. Дякую, що не було усяких там
\enquote{посадових осіб}. Дуже душевно}, — зауважила маріупольчанка Ольга
Гапоненко. 
\end{leftbar}

\ii{insert.read_also.demidko.donbas24.kyiv_zaxid_pidtrym_polonena_azovstal}

\href{https://archive.org/details/video.14_05_2023.donbas_novyny.zustrich_valeria_subotina}{%
Відео: Зустріч з Валерією Суботіною, Донбас Новини, 14.05.2023}%
\footnote{\url{https://archive.org/details/video.14_05_2023.donbas_novyny.zustrich_valeria_subotina}} %
\footnote{\url{https://www.youtube.com/watch?v=epmwFanEkOw}}

\ifcmt
  ig https://i2.paste.pics/PSA12.png?trs=1142e84a8812893e619f828af22a1d084584f26ffb97dd2bb11c85495ee994c5
  @wrap center
  @width 0.9
\fi

Крім цього, під час заходу виступило багато українських митців. Зокрема, пісні
виконав актор-військовослужбовець Дмитро Лінартович, композиції на твори
Григорія Сковороди та Василя Стуса — лірник і бандурист Тарас Компаніченко, а
Марія Бурмака заспівала пісні \enquote{Поцілуй мене на прощання}, \enquote{Повернись живим} та
\enquote{Перемога}.

\ii{16_05_2023.stz.news.ua.donbas24.1.zi_sljozamy_na_ochah_zustrich_objednala_mrplciv.pic.2}
\ii{16_05_2023.stz.news.ua.donbas24.1.zi_sljozamy_na_ochah_zustrich_objednala_mrplciv.pic.3}

Водночас відомі захисники України, актори, поети і письменники, які не
відвідали захід, прочитали вірші онлайн. Серед них: Катерина Поліщук (Пташка),
Сергій Жадан, Римма Зюбина і Павло Вишебаба. До речі, останній був
однокурсником Валерії.

\ii{insert.read_also.demidko.donbas24.nadyhajuchi_istorii_zhinky_priazovja}
% zhadan
\ii{16_05_2023.stz.news.ua.donbas24.1.zi_sljozamy_na_ochah_zustrich_objednala_mrplciv.pic.4}
\ii{16_05_2023.stz.news.ua.donbas24.1.zi_sljozamy_na_ochah_zustrich_objednala_mrplciv.pic.5}

Валерія прочитала \href{https://archive.org/details/14_05_2023.olga_demidko.donbas24.u_stolyci_zahysnycja_mrpl_prezent_novi_virshi}{три нових вірші}%
\footnote{У столиці захисниця Маріуполя презентувала нові вірші, Ольга Демідко, donbas24.news, 14.05.2023, \par%
\url{https://donbas24.news/news/u-stolici-zaxisnicya-mariupolya-prezentuvala-novi-virsi}, \par%
Internet Archive: \url{https://archive.org/details/14_05_2023.olga_demidko.donbas24.u_stolyci_zahysnycja_mrpl_prezent_novi_virshi}%
} та поділилася своїми планами з усіма
присутніми, наголосивши, що вона продовжуватиме писати.

\begin{leftbar}
\emph{\enquote{Планую працювати над прозою, тому що обов'язково потрібно написати про
пресслужбу \enquote{Азовсталі}. Те, про що я можу написати зсередини, то, чим я
займалася, я була з тими хлопцями поруч постійно. І звичайно, що треба писати
про полон: про те, як ми не зламалися}}, — наголосила Валерія Суботіна.
\end{leftbar}

\ii{16_05_2023.stz.news.ua.donbas24.1.zi_sljozamy_na_ochah_zustrich_objednala_mrplciv.pic.6}

Нагадаємо, раніше Донбас24 розповідав, що \href{https://archive.org/details/15_05_2023.olga_demidko.donbas24.juni_mrplci_koncert_kyiv}{\emph{юні маріупольці виступили}}%
\footnote{Юні маріупольці виступили з концертом у Києві, Ольга Демідко, donbas24.news, 15.05.2023, \par%
\url{https://donbas24.news/news/yuni-mariupolci-vistupili-z-koncertom-u-kijevi-foto}, \par%
Internet Archive: \url{https://archive.org/details/15_05_2023.olga_demidko.donbas24.juni_mrplci_koncert_kyiv}%
} в Києві з концертною програмою.

Ще більше новин та найактуальніша інформація про Донецьку та Луганську області
в нашому телеграм-каналі Донбас24.

ФОТО: Юрія Вереса

\ii{insert.author.demidko_olga}
%\ii{16_05_2023.stz.news.ua.donbas24.1.zi_sljozamy_na_ochah_zustrich_objednala_mrplciv.txt}
