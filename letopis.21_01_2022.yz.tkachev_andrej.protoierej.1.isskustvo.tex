% vim: keymap=russian-jcukenwin
%%beginhead 
 
%%file 21_01_2022.yz.tkachev_andrej.protoierej.1.isskustvo
%%parent 21_01_2022
 
%%url https://zen.yandex.ru/media/andreytkachev__official/nam-nujno-iskusstvo-ili--ne-nujno-61ea84fa1d3f6c1cec713988
 
%%author_id tkachev_andrej.protoierej
%%date 
 
%%tags cerkov,chelovek,evangelie,isskustvo,lermontov_mihail,poezia
%%title Нам нужно искусство? Или – не нужно?
 
%%endhead 
\subsection{Нам нужно искусство? Или – не нужно?}
\label{sec:21_01_2022.yz.tkachev_andrej.protoierej.1.isskustvo}
 
\Purl{https://zen.yandex.ru/media/andreytkachev__official/nam-nujno-iskusstvo-ili--ne-nujno-61ea84fa1d3f6c1cec713988}
\ifcmt
 author_begin
   author_id tkachev_andrej.protoierej
 author_end
\fi

Музыка лечит людей до сегодняшнего дня. Об этом знал Авиценна. У него в
трактате о медицине есть отдельная глава: «О лечебных свойствах музыки». 

Сегодня ни в одной медицинской академии этому не учат. А музыка и лечит, и
калечит. От одной музыки можно заболеть. Другой музыкой можно излечиться. 

\ii{21_01_2022.yz.tkachev_andrej.protoierej.1.isskustvo.pic.1}

Творчество – это вообще величайшая вещь. У нас, когда большевики насильственной
своей безбожной рукой половину храмов позакрывали, половину разрушили, то
проповедь Евангелие продолжалась в Третьяковской галерее. Там висела Богоматерь
Владимирская. 

Она «просто» висела. А возле нее «просто» ходили. И некоторые ходили, замирали
и останавливались. Смотрели на Нее. Она – на них. 

Помните, как на «Владимирской» грустно смотрит Богородица ... 

Люди ездили, например, к храму Покрова на Нерли. 

Чего там смотреть, вроде бы? А храм там стоит со времен Андрея Боголюбского. 

И люди туда едут и едут. И смотрят на него и смотрят. 

Проповедь Евангелие сохранилась благодаря церковному искусству. 

Вот и вопрос – надо это или не надо? 

А один священник дореволюционный однажды вышел на проповедь и хотел людям
сказать слово о молитве. Но сказал им другое: «В минуту жизни трудную теснится
в сердце грусть. Одну молитву чудную твержу я наизусть. Есть сила благодатная в
созвучье слов простых. И дышит непонятная святая сила в них. С души как бремя
скатится сомненья далеко, и дышится, и плачется, и так легко-легко. Аминь.» 

И – ушел. Это был Лермонтов. 

Прочел стих. Вот тебе и проповедь. Хотя Лермонтов – не самый церковный поэт.
Но, тем не менее, у него есть очень много нежных проникновенный христианских
мотивов в поэзии. 

Так что – как? Что скажете? Нужно нам искусство или не нужно? 

\ii{21_01_2022.yz.tkachev_andrej.protoierej.1.isskustvo.cmt}
