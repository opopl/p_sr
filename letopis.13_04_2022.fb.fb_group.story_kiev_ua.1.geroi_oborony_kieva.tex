% vim: keymap=russian-jcukenwin
%%beginhead 
 
%%file 13_04_2022.fb.fb_group.story_kiev_ua.1.geroi_oborony_kieva
%%parent 13_04_2022
 
%%url https://www.facebook.com/groups/story.kiev.ua/posts/1902168473313278
 
%%author_id fb_group.story_kiev_ua,mak_poznyak
%%date 
 
%%tags 
%%title ГЕРОЇ ОБОРОНИ КИЄВА
 
%%endhead 
 
\subsection{ГЕРОЇ ОБОРОНИ КИЄВА}
\label{sec:13_04_2022.fb.fb_group.story_kiev_ua.1.geroi_oborony_kieva}
 
\Purl{https://www.facebook.com/groups/story.kiev.ua/posts/1902168473313278}
\ifcmt
 author_begin
   author_id fb_group.story_kiev_ua,mak_poznyak
 author_end
\fi

@igg{fbicon.crossed.swords} ГЕРОЇ ОБОРОНИ КИЄВА.

▫️ Не рідко доводиться чути докори на адресу білорусів, що вони є поплічниками
росіян у цій війні і мало чим допомагають українцям. Маючи друзів у Білорусі і
знаючи, що не мало білоруських хлопців боронять Україну зі зброєю в руках ще з
2014 року — завжди прагну відновити справедливість і просвітити критиків. Вже
котрий день поспіль, не дає спокою настирлива думка, аби трохи розповісти
киянам про білоруський батальйон імені Кастуся Калиновського, який сформували
безпосередньо для оборони Києва. І ось цей час настав.

\ii{13_04_2022.fb.fb_group.story_kiev_ua.1.geroi_oborony_kieva.pic.1}

▫️ @igg{fbicon.military.medal}  Сьогодні (13.04.2022), указом президента, посмертно відзначений найвищим
державним званням «Герой України» — мінчанин, заступник командира батальйону
ім. Калиновського, старший сержант Олексій Скобля на прізвисько \enquote{Тур}.

▫️ 13-го березня загін \enquote{Тура} втрапив у засідку під селом Мощун поблизу Києва.
Олексій прикривав відхід своєї групи, був поранений у стегнову артерію та не
зміг вибратись з поля бою живим. Наступного дня йому мало виповнитись 32.
Протягом тижня до власної загибелі, використовуючи професійні медичні навички,
Олексій Скобля врятував десятьох поранених побратимів. Він захищав Україну з
2015 року, отримав українське громадянство і одружився у Києві.

▫️ Батальйон імені Кастуся Калиновського був сформований на основі тактичної
групи \enquote{Білорусь} на початку березня 2022 року. Чисельність — понад 500 бійців.
Батальйон ім. Калиновського брав активну участь у звільненні Ірпеня та
Лук'янівки, бився за Бучу і Мощун.

▫️ Окрім Героя Олексія Скоблі — вшанування пам'яті варті двоє інших білорусів,
що поклали життя за Україну.

▫️✝️ 27-го березня під Ірпенем, внаслідок артилерійського обстрілу, був
смертельно поранений 32-річний Зміцер Апанасович (позивний \enquote{Терор}). У той день
загін Апанасовича знищив три ворожі БМП.

▫️✝️ Найпершим же білорусом, хто загинув від початку повномасштабного вторгнення,
став ще один мінчанин — 27-річний Ілля Хренов (позивний \enquote{Литвин}), котрий
боронив нашу Неньку з 2014-го. Ілля отримав смертельне поранення під час боїв
за Бучу 3-го березня, проти значно переважаючих сил ворога. Близько 40 наших
піхотинців з кулеметами і гранатометами очікували на 10 одиниць російської
бронетехніки, але її виявилось понад 70 одиниць... У нерівному бою наші
оборонці зазнали великих втрат і були змушені відступити.

▫️ 24-го лютого Ілля Хренов також найпершим серед білоруських добровольців
записав відео із закликом до білорусів ставати на захист України.

 @igg{fbicon.speech.baloon}  \enquote{Українці — мій братній народ, тому я не міг сидіти на місці, коли їх почали
вбивати}, — відповідав Ілля на питання про мотиви рішення взяти до рук зброю.

Інтерв'ю від побратима Іллі Хренова про фатальну битву — читайте на моїй
персональній сторінці.

Замість епілогу публікації скажу, що я проти поспішного перейменування деяких
станцій метро, особливо станції \enquote{Мінська}, бо не мало мінчан гідно захищають
наш славний Київ і не жаліють власного життя. Після нашої перемоги ми маємо
обов'язково допомогти білорусам у звільненні їхньої батьківщини від
лукашенківського режиму.

@igg{fbicon.fist.raised}  Жыве Беларусь!

\ii{13_04_2022.fb.fb_group.story_kiev_ua.1.geroi_oborony_kieva.cmt}
