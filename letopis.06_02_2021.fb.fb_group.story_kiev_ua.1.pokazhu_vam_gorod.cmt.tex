% vim: keymap=russian-jcukenwin
%%beginhead 
 
%%file 06_02_2021.fb.fb_group.story_kiev_ua.1.pokazhu_vam_gorod.cmt
%%parent 06_02_2021.fb.fb_group.story_kiev_ua.1.pokazhu_vam_gorod
 
%%url 
 
%%author_id 
%%date 
 
%%tags 
%%title 
 
%%endhead 
\zzSecCmt

\begin{itemize} % {
\iusr{Людмила Ареф'єва}

Спасибо огромнейшее за то, что поделились своей работой 40-ней
давности... как-будто на 10 минут вернулась в юность... тронуло до слез...

\iusr{Федор Лебедев}
\textbf{Людмила Ареф'єва} - спасибо и Вам! Позже, загружу другой свой фильм о Киеве.

\iusr{Людмила Ареф'єва}
\textbf{Федор Лебедев} буду ждать!

\iusr{Наталия Ковалева}
Спасибо, вы вернули меня в старый Киев!

\iusr{Елена Сидоренко}

Спасибо Вам, Фёдор! Вы вернули меня в молодость, в тот Киев, который отличается от
нынешнего. Очень хороший фильм, не верится, что это работа молодого
режиссера - такие красивые панорамы города, люди, среди которых ищещь знакомые
лица... а закат какой! @igg{fbicon.heart.beating}. Благодарю! @igg{fbicon.heart.red}

\iusr{Светлана Манилова}
\textbf{Елена}, не Феликс, а Федор! @igg{fbicon.smile} 

\iusr{Елена Сидоренко}
\textbf{Светлана Манилова} исправила. Ошиблась неспроста! @igg{fbicon.beaming.face.smiling.eyes} 

\iusr{Федор Лебедев}
\textbf{Елена Сидоренко} - и Вам спасибо большое за добрые слова! Через несколько дней загружу другой фильм о Киеве, который делал с братом, - он намного лучше.

\iusr{Елена Сидоренко}
\textbf{Федор Лебедев} Фёдор! Спасибо ещё раз! И консультант Халепо, и текст он читал @igg{fbicon.heart.beating} 

\iusr{Vadim Basovskiy}
Спасибо, хороший фильм, тем более, что автору было всего 19 лет. Конечно нужно сделать поправку на время.

\iusr{Lesik Machynsky}
\textbf{Vadim Basovskiy} на времена и нравы...

\iusr{Константинова Наталия}

Именно таким я покинула Киев 40 с лишним лет назад, каждый год стараюсь
приехать, видела и вижу изменения города, не всегда в лучшую сторону, но всё
равно, пожив и побывав в Киеве, его уже нельзя забыть никогда. Спасибо за
фильм

\iusr{Alla Pavlova}
Спасибо, Фёдор, Вы показали ещё мой Киев! Сейчас это уже чужой город!

\iusr{Андрей Шиян}
Спасибо!

\iusr{Вера Хохлова}
Спасибо, Вам !)

\iusr{Людмила Волжанка}
Добрый, приветливый, теплый Киев, очень жаль, что многое изменилось....... К сожалению не в лучшую сторону

\iusr{Larisa Tverdokhleb}
Спасибо большое!!! Вы любите Киев и это чувствуется в Вашем фильме! Киев нельзя не любить - это наш Город!

\iusr{Ирина Нищимная}

Спасибо Вам, фильм прекрасен, как и наш любимый город,,как приятно вернуться в
детство, юность и пройтись родными улицами,, как же не хватает сейчас таких
фильмов на тв, они ведь показывают как нужно любить и беречь наш прекрасный
Вечный Киев!

\iusr{Федор Лебедев}
\textbf{Ирина Нищимная} - Благодарю Вас за добрые слова и желаю крепкого здоровья и благополучия!

\iusr{Светлана Блаус}

Как будто машина времени вернула меня на 40 лет назад! Какое счастье увидеть
город юности, любви и надежды! Нет слов...спасибо!

\iusr{Федор Лебедев}
\textbf{Светлана Блаус} - и Вам спасибо! Всего Вам Доброго!

\iusr{Ольга Волынец}

Вопрос к автору. Есть кадр, Святой Владимир на фоне Днепра, но нет пешеходного
моста, а он открыт в 1957 году, может другой ракурс. А в целом, молодцы, окунули в
прошлое красавца Киева! @igg{fbicon.hands.applause.yellow} 

\iusr{Федор Лебедев}
\textbf{Ольга Волынец} - 

спасибо за комментарий! Я вчера после публикации видео вспомнил,что это съёмки
не 1980, а лето 1982 года. В пешеходный мост в том фильме мелькает. В кадре с
Владимиром он не виден из-за точки съёмки. Всего Вам хорошего!


\iusr{Ольга Волынец}
\textbf{Федор Лебедев} ,спасибо!

\iusr{Петр Кузьменко}

Огромное спасибо за возвращение в юность! Целый водоворот чувств и эмоций
вызвал Ваш фильм. Благодарю!


\iusr{Федор Лебедев}
\textbf{Петр Кузьменко} - и Вам спасибо! Всего Вам доброго желаю!

\iusr{Тамара Нарижная}

Спасибо за чудесный фильм Как приятно увидеть все памятники которых сейчас уже
нет и испытываешь гордость от того что живешь всю жизнь в таком красивом городе

\begin{itemize} % {
\iusr{Lesik Machynsky}
\textbf{Тамара Нарижная} по ильичам плачь???

\iusr{Федор Лебедев}
\textbf{Тамара Нарижная} - и Вам спасибо! Всего Доброго!

\end{itemize} % }

\iusr{Kristinka Medvedeva}

Автор, огромное Вам спасибо за прекрасный фильм, красивые, трогательные кадры!!!
Аплодирую Вам и Вашим помощникам за такое памятное для всех киевлян
кинопроизведение!!!

\begin{itemize} % {
\iusr{Федор Лебедев}
\textbf{Kristinka Medvedeva} - 

и Вам спасибо за высокую оценку этой скромной работы! На неделе выставлю
намного лучший фильм - \enquote{Город Архангела Михаила}, который лет 15 назад о Киеве
сделал мой брат Михаил. Вот тем фильмом действительно, можно гордиться.

\iusr{Kristinka Medvedeva}
\textbf{Федор Лебедев} с нетерпением буду ждать фильм!
\end{itemize} % }

\iusr{Оксана Дубинина}
Фёдор, спасибо! браво!)))) Это профессиональный фильм, тем более с поправкой на
технические возможности. А с чего снимали? С вертолёта?

\iusr{Федор Лебедев}
\textbf{Оксана Дубинина} - спасибо за отзыв. Да, на час арендовали у ГАИ вертолёт. Я об этом писал в предисловии.

\iusr{Татьяна Сирота}

Спасибо, Федор!
Нахлынули воспоминания.
Как же мы были тогда молоды!
А фильм о Киеве сегодняшнем очень нужен. Снимайте его!
Наш город был, есть и будет прекрасен всегда.

\iusr{Федор Лебедев}
\textbf{Татьяна Сирота} - взаимно! Благодарю Вас и всех добрых зрителей!

\iusr{Шурик Барков}
Классно, прям окунулся в атмосферу Киева своего детства, хоть оно было то на 10 лет позже.
Спасибо!

\iusr{Федор Лебедев}
\textbf{Шурик Барков} - благодарю Вас за комментарий! Успехов желаю Вам!

\iusr{Руслана Колесник}
Спасибо огромное!!)

\iusr{Cvetlana Dubodelova}

Спасибо за воспоминания, как жаль, что многие изменения произошли не в лучшую
сторону, центр Киева, Крещатик, потерял свою индивидуальность, своё лицо, а
какие фонтаны были..., Ещё раз спасибо за прекрасный фильм о любимом городе

\iusr{Федор Лебедев}
\textbf{Cvetlana Dubodelova} - благодарю за просмотр и добрые слова! Успехов Вам!

\iusr{Нина Воронцова}

Я, хорошо помню Киев, того времени, ведь тогда мне было 18 лет, и я любила
гулять по городу. А, сейчас он изменился, и сейчас другая эпоха.

\iusr{Ольга Шамрай}
Рада была увидеть и услышать Анатолия Трофимовича Халепо. Спасибо

\begin{itemize} % {
\iusr{Федор Лебедев}
\textbf{Ольга Шамрай} - и вам спасибо. Меня тут поправили, - это съёмки не 80-го года, а 82-го...

\iusr{Ольга Шамрай}

Уважаемый \textbf{Федор Лебедев}, для меня точность даты не столь важна. Важно, что Вы
это подарили: культура памяти, съемка, уважение к материалу. Для меня это
самое важное. И память о людях, которые меня учили @igg{fbicon.heart.red}

\iusr{דמיטרי קוגן}
\textbf{Ольга Шамрай} 

и Татьяну Золоторёву тоже. Мой отец тогда работал в экскурсоводом и состоял в
исторической секции которую возглавлял Анатолий Трофимович. На все заседания
секции папа брал меня тогда девятиклассника с собой. Никогда не забуду
экскурсию по Подолу которую по приглашению Халепо проводил Брайчевский

\end{itemize} % }

\iusr{Татьяна Анисимова}

Фотографий много о Киеве у меня. Могу поделиться. Снимаю с 13 лет. Сейчас мне
59. Коренная киевлянка. Ваша тема мне интересна.

\begin{itemize} % {
\iusr{Федор Лебедев}
\textbf{Татьяна Анисимова} 

- благодарю Вас! Давайте спишемся - созвонимся. Я вообще, хочу сделать
объявление в этой группе, - чтобы люди писали или в группу или мне лично свои
небольшие воспоминания о той, прежней жизни. Я бы сделал серию роликов с их
рассказами и фотографиями. Это важно для молодёжи. Мой тел. 098-131-33-00,
почта: \url{flebedev@ukr.net}

\iusr{Татьяна Анисимова}
\textbf{Федор Лебедев} ок

\iusr{Татьяна Анисимова}
0937743260. \url{anistaten@gmail.com}
\end{itemize} % }

\iusr{Volodymyr Nekrasov}

Прекрасний фільм! Велике Вам дякую! Захоплюючі ракурси. Деякі вже стали
хрестоматійними, а деякі нові, авторські. І це у 19 років!!! Геніально!

\begin{itemize} % {
\iusr{Федор Лебедев}

Дякую Вам! Але, вже після публікації я пригадав, що ці зйомки відбувались не у
1980, а у 1982 році. Я вже був великим хлопцем, - 21 рік, на останньому курсі
навчався заочно на кінознавця.

\iusr{Volodymyr Nekrasov}
\textbf{Федор Лебедев} 

я у свої що 19 що 21 був іще дитиною  @igg{fbicon.smile}  Не можу також не
поцікавитись звідки музичний супровід? Спеціально для цього фільму чи взятий
звідкись?

\iusr{Федор Лебедев}
Музику я брав у фонотеці \enquote{Київнаукфільму}, де робив монтаж. Музика з
радянських кінофільмів, але я знаю, з яких...
\end{itemize} % }

\iusr{Лариса Кучерова}

Спасибо, Федор, что можно увидеть тот Киев, мой Киев. И люди, люди.. горожане.
Как же хорошо окунуться в воспоминания..

\iusr{Федор Лебедев}
\textbf{Лариса Кучерова} - спасибо и Вам! На след. неделе опубликую другой фильм о Киеве, - он интереснее и снят лучше.

\iusr{Людмила Крошка}
Спасибо!!! Окунулась в детство!!

\iusr{Элеонора Лозовская}
Очень, очень трогательно.. хочу продолжения... не задерживайте Федор..

\iusr{Федор Лебедев}
\textbf{Элеонора Лозовская} - благодарю Вас! На неделе размещу другой фильм о Киеве. Он намного лучший.

\iusr{Natasha Levitskaya}

Спасибо! Прекрасный фильм и Киев такой уютный и теплый!
Как хорошо и профессионально вы сняли в 19 лет! Просто, браво!

\iusr{Федор Лебедев}
\textbf{Natasha Levitskaya} - Благодарю Вас!

\iusr{Сергей Евсеенко}
Просто Супер. Щедеврально.

\iusr{Федор Лебедев}
\textbf{Сергей Евсеенко} - благодарю! Успехов Вам!

\iusr{Неля Цекалова}
Прекрасное чувство ностальгии, благодарю...

\iusr{Ольга Белозовская}
Великолепно! Спасибо большое!

\iusr{Алла Лукьянченко}
Спасибо! Это настольгия! Чудесный Город, чудесные люди, чудесное время, хоть и не легкое

\iusr{Giorgi Begeluri}

Sovershenno izumitelni film! Nikogda ne biv v kieve pdumal, esli prediotsa
pobivat v vashem chudnom gorode to vsio mne budet znakoma. Bravo!

\begin{itemize} % {
\iusr{Федор Лебедев}
\textbf{Giorgi Begeluri} - благодарю Вас за просмотр и отзыв! Желаю Вам успехов и здоровья!

\iusr{Giorgi Begeluri}
\textbf{Федор Лебедев}
Eto vam spasibo za takoi xoroshi film pro vash xoroshi gorod. Privet iz tbilisi!
\end{itemize} % }

\iusr{Сизова-Билодид Валентина}
Дякую за спогади.

\iusr{Сергей Чеховский}
Восьмидесятые .....  @igg{fbicon.thinking.face} 

\iusr{Svetlana Voloshina}
Благодарю за Киев моего детства! @igg{fbicon.hands.pray} 

\iusr{Маргарита Евтушевская}
Дякую Вам Майстре! Моє дитинство, мій рідний Киів, спогади.... Браво

\iusr{Федор Лебедев}
\textbf{Маргарита Евтушевская} - дякую і Вам за перегляд і високу оцінку! Бажаю Вам успіхів!

\iusr{Гала Алданькова}
Прекрасный фильм!

\iusr{Ирина Ермакова}
Спасибо, очень интересно

\iusr{Федор Лебедев}
\textbf{Ирина Ермакова} - благодарю Вас! Желаю успехов!

\iusr{Светлана Недайбида-Бучко}
Спасибо огромное за воспоминания! @igg{fbicon.hands.applause.yellow}{repeat=3}
@igg{fbicon.hearts.two}  @igg{fbicon.rose} 

\begin{itemize} % {
\iusr{Федор Лебедев}
\textbf{Светлана Недайбида} - спасибо Вам за просмотр! Желаю успехов и здоровья!

\iusr{Светлана Недайбида-Бучко}
\textbf{Федор Лебедев} В самом конце 80-х я начала работать в Бюро путешествий нештатным экскурсоводом. Спасибо еще раз огромное за воспоминания и эмоции!
Я и сейчас работаю гидом. С удовольствием поделюсь фотографиями.
\end{itemize} % }

\iusr{Надежда Сенчило}
Большая благодарность вам, расчувствовали до слез, как все было гармонично в
родном Киеве. Ностальгия.

\iusr{Федор Лебедев}
\textbf{Надежда Сенчило} - Спасибо за добрые слова! Всего Вам доброго!

\iusr{Георгий Майоренко}
Очень теплый фильм! Спасибо! До слез. Талантливо, профессионально.

\begin{itemize} % {
\iusr{Федор Лебедев}
\textbf{Георгий Майоренко} - благодарю Вас за высокую оценку этого скромного фильма! Желаю Вам всего Доброго!

\iusr{Георгий Майоренко}
\textbf{Федор Лебедев} Все отлично! Как говаривал один мой знакомый: - \enquote{Скромность мое оружие!})) Творческих успехов!
\end{itemize} % }

\iusr{Ирина Попова}
Спасибо большое за этот трогательный фильм. Это Киев моей молодости

\iusr{Федор Лебедев}
\textbf{Ирина Попова} - благодарю Вас! Успехов Вам желаю!

\iusr{Руслана Оксаненко}
Отлично снято, спасибо!

\iusr{Федор Лебедев}
\textbf{Руслана Оксаненко} - благодарю! Всего Вам Доброго!

\iusr{Руслана Оксаненко}
И Вам!  @igg{fbicon.love.letter} 

\iusr{Олег Сушков}
 @igg{fbicon.thumb.up.yellow} 

\begin{itemize} % {
\iusr{Федор Лебедев}
\textbf{Олег Сушков} - Спасибо за просмотр и оценку!

\iusr{Вячеслав Гарбулин}

Был зелёный, чистый и красивый город, родной и любимый Киев!

А сейчас полностью наоборот! Грязь и бездействуют службы!!! И строят что
попало! Увы, но это факты

\end{itemize} % }

\iusr{Инга Болгова}

Искренние слова любви и благодарности автору! Именно таким я впервые увидела
Киев, полюбила его, и живу здесь более лет тридцати лет Жаль только, что
современное градостроительство, а не время, с каждым годом уносит частичку того
города, из фильма. Остаётся только верить в чудесный Дух города, который
охраняет его на протяжении тысячелетий. Автору - браво!

\begin{itemize} % {
\iusr{Федор Лебедев}
\textbf{Инга Болгова} - благодарю Вас за просмотр и тёплые слова! Желаю Вам крепкого здоровья и всего наилучшего!
\end{itemize} % }

\iusr{Мила Свиридова}
Спасибо большое!

\iusr{Наталия Прусова}
Замечательный фильм! Спасибо автору за его творчество! Обязательно нужно создать картину сегодняшнего города!

\iusr{Татьяна Желдубовская}
Спасибо

\iusr{Мария Аронова}
Спасибо!

\iusr{Mykola Anfinogenov}

Спасибо за фильм за память, о том что прошло, это всегда в нашем серце. Будьте
успешным во всем, все всегда держалось на аматорах и интузиастах. Это спокон
веков.!

\begin{itemize} % {
\iusr{Федор Лебедев}
\textbf{Mykola Anfinogenov} 

- благодарю Вас за такие хорошие пожелания! И Вам успехов и здоровья желаю!

\iusr{Олена Потильчак}

Коля, я опять заплакала. Я вспомнила школу, Киевпроект, Дом архитекторов, как
мы большой компанией ездили к тебе на дачу. Я вспомнила нашу аллейку на
Липской. Институтскую. Вроде бы эти улицы есть и их уже нет. Я никак не могу
принять то, что время и родных, любимых уже не вернуть. Коля, ты держись. И я
буду держаться, потому, что нам еще надо встретиться здесь, на аллейке. Береги
свое здоровье.

\iusr{Mykola Anfinogenov}
\textbf{Олена Потильчак} Благодарю теба, я обязательно постараюсь !
\end{itemize} % }

\iusr{Анна Анна}
Спасибо, за воспоминания о нашем Киеве!

\iusr{Diana Melnik}
Спасибо Вам большое. Создайте, пожалуйста, картину сегодняшнего дня.

\begin{itemize} % {
\iusr{Федор Лебедев}
\textbf{Diana Melnik} - благодарю Вас! Планирую на своих уроках со студентами-видеоблогерами скоро начать съёмки. В прошлом году нам помешал карантин. А сейчас в планах сделать несколько оригинальных видео о современном городе.
\end{itemize} % }

\iusr{Oksana Kashenko}
Киев в вашем фильме такой тёплый....
Я как будто побывала в своём детстве
Эта смешнючая девочка с бантом и мороженым напомнила мне меня :)))

\iusr{Федор Лебедев}
\textbf{Oksana Kashenko} - благодарю Вас за просмотр! Желаю Вам всего самого хорошего!

\iusr{Ірина Кот}

Очень трогательно, прямо слезы наворачивались во время просмотра. Чувствуется
атмосфера тех лет. Очень уютно и душевно. Спасибо.

\begin{itemize} % {
\iusr{Федор Лебедев}
\textbf{Ірина Кот} - благодарю Вас за просмотр и высокую оценку! Всего Вам Доброго желаю!

\iusr{Ірина Кот}
\textbf{Федор Лебедев} Вам спасибо, за любовь к Киеву. Творческих Вам успехов и всех благ!

\iusr{Федор Лебедев}
\textbf{Ірина Кот} - взаимно! Благодарю и желаю Вам всего наилучшего!
\end{itemize} % }

\iusr{Евгения Курбатова}
Спасибо огромное за фильм!! Много, к сожалению, уже не увидим сегодня!

\iusr{Олександр Кульбашний}
Спасибо большое

\iusr{Олена Потильчак}

Это время ухода моего детства. Время юности. С одной стороны, лучше бы не
смотрела, больно. С другой стороны очень хорошо, что посмотрела. Это как опять
увидеть очень родного человека, которого уже нет. Спасибо автору поста за ваш
авторский прекрасный фильм.

\iusr{Федор Лебедев}
\textbf{Олена Потильчак} - благодарю Вас за просмотр и высокую оценку! Желаю Вам успехов и самого наилучшего!

\iusr{Радмила Васильева}
спасибо

\iusr{Людмила Гончар}

Это лучшее, лучшее видео про Киев, да и вообще про нас! Спасибо. Прекрасное
посвящение любимому городу и великим людям.

\begin{itemize} % {
\iusr{Федор Лебедев}
\textbf{Людмила Гончар} - 

большое спасибо за просмотр и очень высокую оценку нашего не большого и не
нового фильма! В пятницу я планирую загрузить в эту же группу любителей Киева
(если Администратор разрешит) другой, ещё лучший фильм о Киеве, - \enquote{Город
Архангела Михаила}. Надеюсь, что он Вам понравится! Всего Вам Доброго!

\end{itemize} % }

\iusr{Виктория Чупыра}
Спасибо, этот фильм как из прошлой жизни, ностальгия... слезы...

\iusr{Ирина Ноздренко}

Спасибо за фильм трогательный до слёз, созданный с огромной любовью к городу,
несущий ( теперь уже ностальгическую ) энергетику Киева тех лет. Спасибо за
анонс продолжения о современном Киеве. Ждём))

\iusr{Roman Tymin}
Класс

\iusr{Tim Liabakh}

Фильм - огонь. Жаль, что нищета советского обывателя не позволила иметь много
отснятого Киева тех лет. Тем ценнее такие фильмы.

\begin{itemize} % {
\iusr{Федор Лебедев}
\textbf{Tim Liabakh} - 

спасибо большое за Вашу высокую оценку! Но я с Вами совсем не согласен в том,
что советский обыватель жил в нищете. Жили простые люди в начале 80-х годов не
сравнить, с сегодняшним уровнем жизни. Я лично коммунистом не был, но получил
бесплатно 2 высших образования, бесплатную квартиру и работал режиссёром и
оператором много и интересно. До развала СССР в Киеве работало 5 киностудий.
Зайдите в кино архив им. Пшеничного - им хранилищ не хватает, чтобы уместить
старую кинохронику.

\iusr{Tim Liabakh}
\textbf{Федор Лебедев} 

я про то, что количество отснятого на бытовые кино и видеокамеры гораздо
меньше. Что касается жизни простого тогда человека, и всего бесплатного, то я
бы не хотел вдаваться в эту полемику. Я не много застал того времени, но
прекрасно помню как жили мои родители, бабушки и дедушки.

Спасибо вам за фильм. Если так много отснятого, то конечно его нужно цифровать.
Понимаю, что это безумно дорого и неокупаемо.

\iusr{Галина Якушик}
А колись знімали, і не плакали що дорого

\iusr{Tim Liabakh}
\textbf{Галина Якушик} а хто зараз плаче, шо дорого знімати? Зараз на любий телефон знімати можна
\end{itemize} % }

\iusr{Эля Коваль}
Спасибо Вам человеческое, от всех киевлян, помню город таким, а сегодня
показала свои детям.

\iusr{Федор Лебедев}
\textbf{Эля Коваль} - благодарю Вас за просмотр и добрые комментарии! Успехов всей семье и здоровья желаю!

\iusr{Людмила Порунова}
Київ - моє улюблене місто, але при чому тут \enquote{русские}?

\begin{itemize} % {
\iusr{Olena Machekhina}
\textbf{Людмила Порунова} 

У дописі зазначена дата створення фільму - 1982-й рік. Якби там не було б
тогочасних ідеологічних шаблонів і кадрів з пам'ятниками Леніну, фільму б не
було. Тим паче він був створений на замовлення, тобто сюжет і текст замовлені.

Фільм хороший, теплий, створеній з любов'ю до Києва

\end{itemize} % }

\iusr{Roman Subotin}

Bolshoe spasibo i za publikacoyu filma i za vospiminaniya! Udachi v realizacii idei !

\iusr{Аркадий-Лариса Малюга}
Здорово!

\iusr{Аркадий-Лариса Малюга}
Сбережем его для наших нащадков.

\iusr{Федор Лебедев}
\textbf{Аркадий-Лариса Малюга} - благодарю Вас за просмотр и оценку!

\iusr{Ганна Путова}

Класс. Моя мама - городской экскурсовод, я с 5 лет с ней была на колесах. Как
раз в бюро она работала с 1978 по 1993 г. Здесь - их маршруты, из коллеги
молодые.

\iusr{Федор Лебедев}
\textbf{Ганна Путова} - благодарю Вас за просмотр и отзыв!

\iusr{Галина Местечкина}

Який подарунок знову почути голос незабутнього Анатолія Трофимовича і пройти
разом з ним вулицями рідного міста! Сердечна подяка!

\iusr{Федор Лебедев}
\textbf{Галина Местечкина} - спасибо Вам за просмотр и добрые комментарии!

\iusr{Галина Местечкина}

Спасибі, Федоре, за чудовий фільм і збереження голосу нашого незабутнього
Анатолія Трофимовича! Чекаю продовження вже досвідченого Майстра!

\iusr{Федор Лебедев}
\textbf{Галина Местечкина} - благодарю за просмотр и высокую оценку! Примите от меня Добрые пожелания!

\iusr{Елена Палий}
Огромное спасибо !!!!!!

\iusr{Федор Лебедев}
\textbf{Елена Палий} - благодарю Вас за просмотр и слова благодарности! Всего Вам хорошего и доброго желаю!

\iusr{Evgenia Lazartchuk}
Спасибо, замечательный фильм!

\iusr{Zoya Nechay}

Дякую автору за теплий фільм і щемливі спогади. Окреме сердечне спасибі за
збереження голосу і можливість ще раз почути-побачити Анатолія Халепо. До
сліз..

\iusr{Федор Лебедев}
\textbf{Zoya Nechay} - и я Вас благодарю и желаю Вам всего самого Доброго!

\iusr{Zoya Nechay}
\textbf{Федор Лебедев} Дякую!

\iusr{Lena Andreeva}
Спасибо за память @igg{fbicon.hearts.revolving} 

\iusr{Федор Лебедев}
\textbf{Lena Andreeva} - спасибо и вам за просмотр! Всего Доброго Вам желаю!

\iusr{наталья колесник}
Все такое родное и далекое @igg{fbicon.heart.blue}  @igg{fbicon.heart.yellow} 

\iusr{Mila Bagry-Gaskill}
Beautiful shots, lovely and so nostalgic @igg{fbicon.heart.red}

\iusr{Федор Лебедев}
\textbf{Mila Bagry-Gaskill} - благодарю Вас! Желаю всего хорошего!

\iusr{таня козырева}
Да, это был город для жизни!

\iusr{Федор Лебедев}
\textbf{таня козырева} - спасибо за просмотр!

\iusr{Людмила Заец}
Фильм классный, увидела город детства, спасибо

\begin{itemize} % {
\iusr{Федор Лебедев}
\textbf{Людмила Заец} - благодарю! Успехов Вам!

\iusr{Людмила Заец}
\textbf{Федор Лебедев} фотографии вам надо без людей ?

\iusr{Федор Лебедев}
\textbf{Людмила Заец} - Спасибо! Мне нужны виды старого Киева (до революции 17 года).

\iusr{Людмила Заец}
\textbf{Федор Лебедев} понятно
\end{itemize} % }

\iusr{Valentyna Komerzhenko}
Це неймовірно!! Дякую!

\iusr{Федор Лебедев}
\textbf{Valentyna Komerzhenko} - благодарю Вас за комментарий! На неделе выставлю ещё один наш фильм о Киеве, - он намного лучше. Всего Вам самого наилучшего желаю!

\iusr{Ніна Панченко}
Спасибо, прекрасный фильм, приятные воспоминания.... ностальгия по городу юности....

\iusr{Yuliya Kolotilova}
Спасибо, какой был город@igg{fbicon.heart.red}

\iusr{Ян Приворотский}
Словно снова в детстве побывал...

\iusr{Валерия Руденко-Ярощук}
Браво! Очень интересно!

\iusr{Наталия Багрянцева}
Благодарю. Замечательно.

\iusr{Александр Шукевич}
Чудесный фильм, о Чудесном Родном Городе! Огромная Благодарность !

\begin{itemize} % {
\iusr{Федор Лебедев}
\textbf{Александр Шукевич} - и я Вас благодарю за просмотр и за тёплый отзыв! Успехов Вам!

\iusr{Александр Шукевич}
\textbf{Федор Лебедев} предлагаю Дружбу в F.
\end{itemize} % }

\iusr{Vitalij Vartiainen}
Как же город изменился до неузнаваемости! Замечательный исторический фильм!

\iusr{Anastasia Raint}
Невероятно!!!! Потрясающе!!!!!  @igg{fbicon.face.smiling.hearts}  @igg{fbicon.face.happy.two.hands} 
Спасибо большое за \enquote{дверцу} в прошлое @igg{fbicon.face.happy.two.hands} 

\iusr{Федор Лебедев}
\textbf{Anastasia Raint} - спасибо Вам за просмотр и такие комментарии! Успехов Вам больших желаю!

\iusr{Алла Голуб Завада}
Спасибо большое Вам за фильм, который многих вернул в детство. Очень скучаю по тому Киеву.

\iusr{Федор Лебедев}
\textbf{Алла Голуб Завада} - Благодарю Вас! Всего хорошего Вам желаю!

\iusr{Наталья Шатыко}

Какой прекрасный фильм! Спасибо вам за память! Я помню этот город именно
таким! Помню эти экскурсии! Фонтаны, площади, каштаны... Трогательно до слез,
душа щемит... Спасибо огромное за ваш профессионализм! Здоровья крепкого,
успехов!

\iusr{Федор Лебедев}
\textbf{Наталья Шатыко} 

- благодарю Вас за просмотр и за высокую оценку этой работы! Судя по
многочисленным комментариям, мы не зря с подростками таскали аппаратуру по
Киеву. Всего Вам Доброго!

\iusr{Татьяна Вальдовская}
Спасибо! Окунулась в детство и юность!

\iusr{Лариса Пекарська}
Спасибо, за встречу смоим детством.

\iusr{Клавдия Плаксий}
Великолепный красивый фильм. Вы большой молодец! Спасибо вам!

\iusr{Светлана Вербицкая}
Спасибо большое, увидела свой город детства, и места мимо которых проходила каждый день.

\iusr{Анна Яненко}
Дякую! Підкажіть, а чи є десь у доступі згаданий Ваш фільм про Києво-Печерську
лавру?  @igg{fbicon.smile} 

\begin{itemize} % {
\iusr{Федор Лебедев}
\textbf{Анна Яненко} 

- дякую Вам! Ні, фільму про Лавру в Ютубі немає. Але я виставлю його у цій
групі десь, через тиждень. Його назва: \enquote{Летопись в камне} (1980рік).

\iusr{Анна Яненко}
\textbf{Федор Лебедев} дякую! Чекатиму  @igg{fbicon.smile} 

\iusr{Анна Яненко}
\textbf{Федор Лебедев} 

на сайті Національного заповідника \enquote{Києво-Печерська лавра} є рубрика \enquote{На згадку
про Києво-Печерську лавру}, де є чимало фотографій і поштівок. За бажанням
можна сконтактувати з авторами рубрики для отримання якісніших відбитків  @igg{fbicon.wink} 

\iusr{Федор Лебедев}
\textbf{Анна Яненко} дуже Вам дякую!
\end{itemize} % }

\iusr{Anatoliy Kanevsky}
Спасибо огромное за прекрасный фильм навеявший столько приятных и незабываемых воспоминаний. @igg{fbicon.heart.red}{repeat=3}  @igg{fbicon.heart.eyes}{repeat=3}  Сохраню

\iusr{Федор Лебедев}
\textbf{Anatoliy Kanevsky} -благодарю Вас за просмотр и добрые слова! Всего Вам доброго и будьте здоровы!

\iusr{Nadya Harlash}
Очень талантливая работа. Спасибо Вам огромное  @igg{fbicon.face.smiling.eyes.smiling} 

\iusr{Людмила Серая}
Замечательній фильм. Получила удовольствие. Спасибо. Удачи Вам!

\iusr{Татьяна Барабаш}
Спасибо вам большое! Приятно вернуться во время своей юности! Мой любимый и родной город!

\iusr{Asya Postoyenko}

Фільм дійсно дуже гарний, динамічний і гармонійний, а як для 21-річного юнака -
так і зовсім дивовижний. Я його показуватиму своїм штатівцям, що вчать
россійську - тому що відео з такою чіткою дикцією знайти дуже важко.. зараз всі
лопочать)

\iusr{Федор Лебедев}
\textbf{Asya Postoyenko} - дякую! Бажаю Вам щастя!

\iusr{Татьяна Губарь}
Спасибо за экскурсию по-любому городу, там где прошло детство, юность и молодость.

\iusr{Людмила Пироцкая}
Хороший фильм. согретый любовью к родному древнему и вечно молодому Киеву!!! @igg{fbicon.hand.ok}  @igg{fbicon.hands.pray} 

\iusr{Наталия Исхакова}
Ждём с нетерпением новый фильм! Спасибо!

\iusr{Наталья Ершова}
Спасибо за прекрасный фильм!!!

\iusr{Светлана Юшина}

Получила большое удовольствие! Прекрасный фильм, большое Вам спасибо! Мастерски
снят Киев, текст непревзойдённого Анатолия Трофимовича Халепы (мой Учитель)!
Время начала наших экскурсий. @igg{fbicon.heart.suit}

\iusr{Вероника Иванова}
Київ нашої юності! Нічого краще не бачила  @igg{fbicon.face.smiling.hearts} 

\iusr{Федор Лебедев}
\textbf{Вероника Иванова} - дякую Вам за перегляд і дуже високу оцінку!

\iusr{Лиса Патрикеевна}

Замечательный город детства. После таких фильмов понимаешь, что это уже другой
мир, другие люди и другой город, которые остались в прошлом

\iusr{Валентина Шахова}

Чудовий фільм, питання чому «Киев для русского человека начало государства» (не
дослівно, ) чому «русского», не українці 40 років назад населяли, наразі, Київ

\end{itemize} % }
