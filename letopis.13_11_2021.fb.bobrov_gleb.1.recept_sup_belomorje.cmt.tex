% vim: keymap=russian-jcukenwin
%%beginhead 
 
%%file 13_11_2021.fb.bobrov_gleb.1.recept_sup_belomorje.cmt
%%parent 13_11_2021.fb.bobrov_gleb.1.recept_sup_belomorje
 
%%url 
 
%%author_id 
%%date 
 
%%tags 
%%title 
 
%%endhead 
\subsubsection{Коментарі}
\label{sec:13_11_2021.fb.bobrov_gleb.1.recept_sup_belomorje.cmt}

\begin{itemize} % {
\iusr{Алексей Матвеев}

Спасибо Глеб, обязательно попробую с молоком. Я обычно рыбу варю очень долго, а
потом ещё настаиваю на горячей печи, ну это в деревне, и только с перлухой.
Молотят все!)) Отдельное спасибо за совет борьбы с пеной на кастрюле.


\iusr{Николай Ардальянов}
Глеб молодец, что назвал это суп..... уха готовится по другой технологии...

\iusr{Alexander Tumanov}

Народная практика, однако, советует пользовать не менее 5 сортов рыбы.
Неизгладимое впечатление произвела на меня уха, еденая в Венгрии, именно из 5
сортов рыбы и без всяких наполнителей, типа ракушек или крупы. С большим
количеством острого красного перца.

\begin{itemize} % {
\iusr{Глеб Бобров}
Вот, я об этом тоже думаю, Саш! В следующий раз добавлю в рыбный набор из молок, красной рыбы и тресковых - сельдь и лакедру.
\end{itemize} % }

\iusr{Наталья Паваляева}

\ifcmt
  ig https://scontent-frx5-2.xx.fbcdn.net/v/t39.1997-6/p480x480/105941685_953860581742966_1572841152382279834_n.png?_nc_cat=1&ccb=1-5&_nc_sid=0572db&_nc_ohc=dMmZHIhxBS8AX-dHIb0&_nc_ht=scontent-frx5-2.xx&oh=2e70a53db06234fede2a96c8ba0d6a8a&oe=61AA258B
  @width 0.1
\fi

\iusr{Елена Промыслова}
Завидую...
Мужу запрещены все бульоны кроме овощного, так что у нас только постные супы. А ведь хочется!  @igg{fbicon.face.confused} 

\begin{itemize} % {
\iusr{Nadya Carpenter}
Сделайте себе отдельно:)

\iusr{Елена Промыслова}
\textbf{Nadya Carpenter}
Так ему же тоже хочется, мне мужа жалко.
\end{itemize} % }

\iusr{Евгений Марченко}

Глеб Леонидыч, салам! Как азиатский сведловчанин луганчанину. Вот всё ж
зашибись, но зачем молоко в рыбий суп?

И второе, моркву на тёрке, либо как корейскую, мелко соломкой пошинковать?

Рахмет тебе душевнейший, однако попытаюсь разобраться без применения молочки
)))

\begin{itemize} % {
\iusr{Глеб Бобров}
Весь север от Беломорья до Балтики и выше готовит рыбные супы с молоком - это классика.

\iusr{Евгений Марченко}
\textbf{Глеб Бобров}
Йок вопросов. Я живу в Азии. Чуть южнее и восточнее. Просто для меня это... непонятно.

\iusr{Глеб Бобров}
Евгений с чего это Ёбург - вдруг Азия )))))))))))

\iusr{Евгений Марченко}
\textbf{Глеб Бобров}

Глеб Леонидыч, между нами. Официальная граница Европы -Азии проходит по
пермской трассе в 12-ти верстах западнее города. Чуть не доезжая до объездной.
Я там в огороды езжу еженедельно. Ты уж мне поверь пожалуйста на слово.

Фактическая граница проходит ещё на 12 вёрст западнее, на своротке на
Первоуральск. Там тоже есть знак.

Свердловск это Азия. Как бы кто как к этому факту ни относился )))

\iusr{Евгений Марченко}

\ifcmt
  ig https://scontent-frx5-2.xx.fbcdn.net/v/t39.30808-6/254944658_3957199841048316_8719938668755779080_n.jpg?_nc_cat=109&ccb=1-5&_nc_sid=dbeb18&_nc_ohc=DB9RhuKSsUEAX9R4-uu&_nc_ht=scontent-frx5-2.xx&oh=f0314f6e23c6c8e338988d350d141781&oe=61A92815
  @width 0.4
\fi

\iusr{Евгений Марченко}
И да, я азиат. Русский. Мне нисколько не стыдно.

\end{itemize} % }

\iusr{Андрей Третьяков}
Я после того, как у Похлёбкина рецепт прочёл, подсел на калью

\begin{itemize} % {
\iusr{Глеб Бобров}
У меня пока руки и голова не дошла да кальи.

\iusr{Андрей Третьяков}
\textbf{Глеб Бобров}
Очень просто и вкусно

\iusr{Глеб Бобров}
Андрей, обязательно попробую и доложу по результатам.

\end{itemize} % }

\iusr{Дмитрий Винокуров}
Супер, \textbf{Глеб Бобров} !
Интересно, завтра попробую. Тебе пора за кулинарную книгу браться. В смысле писать)

\begin{itemize} % {
\iusr{Глеб Бобров}
После Похлёбкина и Сталика???
Не, я не эпигонствую - готовлю для души ))))))))

\iusr{Дмитрий Винокуров}
\textbf{Глеб Бобров} и напиши для души)))
\end{itemize} % }

\iusr{Filin Iglowl}
две форельки и линь

\begin{itemize} % {
\iusr{Глеб Бобров}
у нас форелька от 550 руб. за кг, а линь изредка бывает вяленой за вообще фантастические деньги )))))))))))
\end{itemize} % }

\iusr{Андрей Хамхидько}
 @igg{fbicon.hands.applause.yellow}  \textbf{\#времяпервыхблюд}

\iusr{Пономаренко Валентина}
Хоть издавай в сафьяновой обложке!!

\iusr{Олеся Зимина}
Возьмём на заметку! Давно я уху не готовила... Спасибо!)

\begin{itemize} % {
\iusr{Глеб Бобров}
Это суп и ни разу не уха ))))))))))

\iusr{Олеся Зимина}
\textbf{Глеб Бобров} Я в этих тонкостях не разбираюсь!))) Для меня любой суп из рыбы - уха!))))
\end{itemize} % }

\iusr{Евгений Тухто}

Ел неделю назад на Волге рыбный суп по местному рецепту. Они его тоже ухой не
считают. Своеобразной фишкой были оладьи к супу из сазаньей икры.


\iusr{Юрий Олейников}
Пока дочитал, слюной истёк)))

\iusr{Иван Богданов}

Глеб, на пальцах.... Никак не пойму - как меняется вкус бульона от обжарки лука
и моркови? Вот просто ... что изменяется во вкусе, если а). не жарить-подпекать
и б). жарить подпекать?

\begin{itemize} % {
\iusr{Nadya Carpenter}

А) размазанная структура, блеклый вкус супа в итоге.

Б) более яркий выраженный вкус, текстура плотная у овощей, масло при обжарке
хорошо вбирает вкусы и запахи- специй, овощей, и удерживает их в блюде.

Индусы вообщемвсе специи пассеруют быстренько, потом это масло заливают в блюда

\iusr{Иван Богданов}

Стоп! Я говорил не о пассировке. Это само собой разумеется. А именно о "подпечь
овощи" для бульона. Пассерую я чаще всего вот так... топленое коровье и хорошее
растительное.

\ifcmt
  ig https://scontent-frx5-1.xx.fbcdn.net/v/t39.30808-6/255083263_6545015448874477_9083064503041085749_n.jpg?_nc_cat=110&ccb=1-5&_nc_sid=dbeb18&_nc_ohc=fHMMnrYmX3wAX8YAwSr&_nc_ht=scontent-frx5-1.xx&oh=36318b32d338e156c3370822575b6b3e&oe=61A95A5F
  @width 0.4
\fi

\end{itemize} % }

\iusr{Илиан Калина}

Морковкой порадовали - тоже всегда перекладываю, и не жалею. Молоко озадачило.
Может молоки?
ладно, шутка.

Насчет зелени это зря. Петрушку и зеленый лук покрошить и сыпать только в
тарелку, но стебель петрушки целиком в суп смело за пять минут до остановки. И
укроп кстати также. Потом выкинуть да и все.

\begin{itemize} % {
\iusr{Глеб Бобров}

Укроп забьет весь вкус, здесь он нежный и сливочный. Петрушку я не ем - в ней
много витамина К влияющего на свертываемость крови и тромбообразования, - не с
моими диагнозами, короче....

\end{itemize} % }

\iusr{Егерь Лесник}
Десертная ложка с горкой соли на 5 литров объёма - это не многовато?

\iusr{Глеб Бобров}
шесть литров...

\iusr{Марат Кулиев}
Звучит вкусно!

Но первую лаврушку я бы минут через пять, после начала кипения, вытащил. Аромат
из неё выварится быстро, а потом только горечь.

\iusr{Глеб Бобров}
горечь двух лаврушек на шесть литров не чувствуются...

\iusr{Александр Борисов}

Здорово! Но всё же лучшая уха на костре, как минимум двойная(тройная) и
заправленная перегоревшей щепкой(древесным углём)...

\end{itemize} % }
