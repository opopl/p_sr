% vim: keymap=russian-jcukenwin
%%beginhead 
 
%%file 24_01_2022.fb.fb_group.svit_kriz_objektyv.1.kostogryz_fotosessia_ptahy
%%parent 24_01_2022
 
%%url https://www.facebook.com/groups/o.bohdanets.photographer/posts/682962326066348
 
%%author_id fb_group.svit_kriz_objektyv
%%date 
 
%%tags foto,priroda,ptica,ukraina
%%title Костогризи - птахи - фотосесія
 
%%endhead 
 
\subsection{Костогризи - птахи - фотосесія}
\label{sec:24_01_2022.fb.fb_group.svit_kriz_objektyv.1.kostogryz_fotosessia_ptahy}
 
\Purl{https://www.facebook.com/groups/o.bohdanets.photographer/posts/682962326066348}
\ifcmt
 author_begin
   author_id fb_group.svit_kriz_objektyv
 author_end
\fi

Сьогодні мені пощастило протягом тривалого часу поспостерігати за парою птахів
вид яких зветься Костогриз (лат. Coccothraustes coccothraustes) (англ.
Hawfinch). Це досить поширений вид, в Україні налічується близько 400 тис. пар,
але дуже потайливий, через що складний до спостережень та вивчення. Через свою
скритність він дійсно рідко потрапляє до об'єктиву, тому побачити його в зимову
пору вдвічі приємніше. 

\ii{24_01_2022.fb.fb_group.svit_kriz_objektyv.1.kostogryz_fotosessia_ptahy.pic.1}

Перш за все цей вид дуже гарний, як і решта
представників родини В'юркових (Fringillidae). Розміром зі шпака звичайного,
але доволі кругліший, особисто мені нагадує великого снігура. Коли дивишся на
нього в анфас він виглядає доволі грізно. 

\ii{24_01_2022.fb.fb_group.svit_kriz_objektyv.1.kostogryz_fotosessia_ptahy.pic.2}

Перш за все кидається в очі масивний
дзьоб, велика голова і товста шия. Зветься костогризом він не дарма, адже
живиться кісточками черемхи та вишні, менш охоче поїдає горобину та бузину.
Завдяки своєму неймовірно міцному дзьобу він лускає кісточки вишні наче
насіння. 

\ii{24_01_2022.fb.fb_group.svit_kriz_objektyv.1.kostogryz_fotosessia_ptahy.pic.3}

Сьогодні я на власні в цьому впевнився, спочатку на прикладі аронії, а
згодом на вишні. Апетит у нього дуже не слабий, дзьоб постійно замурзаний, але
птах час-від-часу його ретельно чистить об гілля та сніг. Якщо подивитися на
нього в профіль то одразу стає зрозуміло, що він не таки й однотонний. Але
найбільше вражає його спина і те як складені крила, саме це мені подобається
найбільше, форма та забарвлення його складених крил. Нічого подібного на
просторах Полісся я поки що не зустрічав. 

\ii{24_01_2022.fb.fb_group.svit_kriz_objektyv.1.kostogryz_fotosessia_ptahy.pic.4}

Орнітологи зазначають, що під час
годівлі цей вид дуже агресивний, як до особин свого, так і інших видів. На
щастя в моєму випадку все інакше, птах дозволив заглядати йому фотоапаратом
мало не в дзьоб. Відзняв все від і до, а під кінець взагалі знахабнів і з
відстані в два метри, своїм калькулятором, зробив йому портретну фотосесію (яку
винесу в окремий допис). Після цього я одразу зловив себе на думці - за
характером майже снігур, такий же флегматичний та компанійський, просто інший
ззовні. Все це відбувалося під кінець фото-дня, тому це була вишенька (а кому
кісточка) на торті.

\ii{24_01_2022.fb.fb_group.svit_kriz_objektyv.1.kostogryz_fotosessia_ptahy.pic.5}
