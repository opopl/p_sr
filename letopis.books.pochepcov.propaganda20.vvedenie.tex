% vim: keymap=russian-jcukenwin
%%beginhead 
 
%%file books.pochepcov.propaganda20.vvedenie
%%parent books.pochepcov.propaganda20
 
%%url 
 
%%author_id 
%%date 
 
%%tags 
%%title 
 
%%endhead 

\section{Введение}

Информационные технологии меняют не только мир, но и способы воздействия на
него. Чем сильнее становятся информационно-коммуникативные технологии, тем
большей становится их результативность. И поскольку воздействие может быть не
только открытым, но и скрытым, то у человека остается все меньше возможностей
ему противостоять. Сегодня практически все строится на информационных и
виртуальных интервенциях: от политических выборов и покупок в магазине до
рейдерского захвата предприятия и ведения войны. Фильмы и телесериалы
захватывают наши мозги в мирной жизни, политтехнологи делают это же во время
выборов, а проведение информационных и психологических интервенций обеспечивает
результат войны в физическом пространстве.

Усилив информационный инструментарий, человечество почти ничего не сделало,
чтобы подготовить к этому население. Робкие слова о медиаграмотности не могут
решить проблему. Создав опасность «огня» в информационных потоках, человечество
не побеспокоилось об институте новых информационных пожарных того же уровня,
что и возможная опасность.

Автор выражает признательность сайту Детектор.медиа  за многолетнее
сотрудничество.
