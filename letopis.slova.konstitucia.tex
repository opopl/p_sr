% vim: keymap=russian-jcukenwin
%%beginhead 
 
%%file slova.konstitucia
%%parent slova
 
%%url 
 
%%author 
%%author_id 
%%author_url 
 
%%tags 
%%title 
 
%%endhead 
\chapter{Конституция}
\label{sec:slova.konstitucia}

%%%cit
%%%cit_head
%%%cit_pic
%%%cit_text
Почему защита русского языка - лишь пустая декларация в \emph{Конституции}?
Почему свободное использование, развитие и защита русского языка в Украине —
это лишь пустая декларация в \emph{Конституции Украины}?  Почему у Зеленского
хотят ввести уголовную ответственность за оскорбление украинского языка, но на
оскорбление русского языка у Зеленского всем плевать?  Почему нам —
украинизация всех сфер общественной жизни, а \enquote{Кварталу 95} и сериалам
Зеленского — спецзакон, позволяющий русский язык?  Есть ли проблемы с русским
языком и гуманитарными правами в Украине и какой статус для русского языка
нужен в Украине?
%%%cit_comment
%%%cit_title
\citTitle{Почему на оскорбление русского языка у Зеленского всем плевать?}, 
Александр Скубченко, strana.ua, 21.06.2021
%%%endcit

%%%cit
%%%cit_head
%%%cit_pic
%%%cit_text
В політичній сфері, мусимо розпочати процес інклюзивної Конституанти. Бо в
існуючому політико-правовому дизайні Республіки не збудуєш. Так само варто
говорити про відповідальне громадянство і нові принципи побудови управління в
країні. Ми маємо нарешті розірвати пуповину з пострадянською спадщиною у
вигляді \emph{Конституції} 1996 року та ухвалити новий Основний Закон!  Хто зробить
все це? Нові політичні суб'єкти.  І не потрібно мені тикати під ніс \enquote{зеленими}
та різними потойбічними \enquote{голосами}.
Існуючі політичні гравці в ситуації кризи та потужних викликів демонструють
повне ідейне та організаційне банкрутство
%%%cit_comment
%%%cit_title
\citTitle{Маніфест реаліста або час збирати каміння. Воно нам скоро знадобиться - ХВИЛЯ}, 
Виталий Кулик, analytics.hvylya.net, 17.06.2021
%%%endcit
