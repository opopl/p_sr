% vim: keymap=russian-jcukenwin
%%beginhead 
 
%%file 02_12_2020.news.ru.lenta_ru.mozzhuhin_andrei.1.pribaltika.poligon_holokosta
%%parent 02_12_2020.news.ru.lenta_ru.mozzhuhin_andrei.1.pribaltika
 
%%url 
 
%%author 
%%author_id 
%%author_url 
 
%%tags 
%%title 
 
%%endhead 

\subsubsection{Полигон холокоста}

Вы в книге напоминаете, что наиболее чудовищные масштабы массовые расправы над
евреями приняли в Литве. Почему именно эта республика стала главным полигоном
холокоста в Прибалтике?

Потому что именно в Литве еще с дореволюционных времен существовала самая
большая в Прибалтике еврейская община. К июню 1941 года там проживало, по
разным оценкам, 225-265 тысяч местных евреев (литваков) и около 15 тысяч
еврейских беженцев из Польши, появившихся там в 1939 году. Поскольку Литва была
полностью оккупирована вермахтом уже в первую неделю Великой Отечественной
войны, сбежать оттуда просто никто не успел. Результат этого был чудовищным.

\begin{leftbar}
	\bfseries
	\centering
{\Large\color{orange}220 000}\par
				{\large евреев}\par
было уничтожено в Литве за время немецкой оккупации
\end{leftbar}

Это 95 процентов от довоенной численности местной еврейской общины и 4/5 от
всех евреев, живших в Прибалтике до 1941 года. В процентном отношении эти цифры
— самые большие потери еврейского населения не только среди союзных республик,
но и всех европейских государств в границах до начала Второй мировой войны.

Как именно в Литве убивали евреев, наглядно иллюстрирует один документ, который
я цитирую в книге. Думаю, нелишним будет его повторить и тут. Один немецкий
военный стал свидетелем еврейского погрома в Каунасе 24 июня 1941 года. Вот его
рассказ о расправах над местными евреями:

\begin{fancyquotes}
Молодой мужчина (это был литовец) в возрасте примерно 16 лет, с засученными
рукавами, был вооружен железным ломом. К нему подводили человека из стоявшей
рядом группы людей, и он одним или несколькими ударами по затылку убивал их.
Таким образом он менее чем за час убил всех 45-50 человек <…> После того как
все были убиты, молодой мужчина положил в сторону лом, пошел за аккордеоном и
взобрался на лежавшие рядом тела убитых. Став на гору, он заиграл литовский
национальный гимн. Поведение стоявших рядом гражданских лиц, среди которых были
женщины и дети, было невероятным — после каждого удара ломом они аплодировали,
а когда убийца заиграл литовский гимн, толпа подхватила его
\end{fancyquotes}

Еще существовали гетто в Вильнюсе, Каунасе и других городах — по сути,
перевалочные пункты перед отправкой в лагеря уничтожения. Были и расстрельные
ямы в Панеряе и многих других местах Литвы. Практически все это осуществлялось
руками местных коллаборантов — немцы не делали «грязную работу». В Латвии и
Эстонии было то же самое.

\lenta{То есть приведенное вами свидетельство говорит не об единичном случае, а о повсеместной картине?}

После ухода Красной армии и еще до прихода вермахта жестокие еврейские погромы
охватили всю Литву. «Временное правительство Литвы», возникшее в последних
числах июня 1941 года вследствие восстания «Фронта литовских активистов»
(подготовленного абвером), в течение нескольких дней успело напринимать
множество пронацистских законов, в том числе и расовый.

\ifcmt
tab_begin cols=3
	caption Книга «Прибалтика. 1939-1945 гг. Война и память», Центральный архив ФСБ России

	pic https://icdn.lenta.ru/images/2020/10/20/16/20201020160225334/pic_e1e84650e2f99a08941e6b964f9ad5ab.jpg
	caption Раскопки могил расстрелянных нацистами жителей Латвии

	pic https://icdn.lenta.ru/images/2020/10/20/16/20201020160442295/pic_82016a82ed17d3240a73258da7872aac.jpg
	caption Акт эксгумации на месте массового захоронения жертв нацистов, составленный членами Чрезвычайной комиссии по расследованию злодеяний немецко-фашистских оккупантов по г. Слока Рижского уезда Латвийской СССР. 28 апреля 1945 г.

	pic https://icdn.lenta.ru/images/2020/10/20/16/20201020160445453/pic_cccc84b04813b00f0f89565a0221485a.jpg
	caption Список расстрелянных нацистами жителей Латвии
tab_end
\fi

Потом, конечно, немецкие оккупационные власти его ликвидировали. Им в принципе
не нужны были никакие «правительства» на территории «Остланда».

Я читал вашу книгу и слушаю вас сейчас, поэтому не могу не спросить: откуда
была такая звериная ненависть к евреям?

Насчет национализма — он был укоренен и в независимых государствах Прибалтики.
Достаточно вспомнить, например, что в Латвии, руководимой Улманисом, стержнем
официальной идеологии стал тезис «Латвия — государство для латышей».

Это следствие внутренних социальных комплексов, которые умело канализируются в
политических целях. Удобно найти внутреннего врага, обвинить его в своих бедах
и выместить на нем все раздражение от собственных же ошибок и неудач. А уж если
этот «сконструированный» враг в чем-то успешнее тебя, то ненависть многократно
усиливается.

В исторической литературе описано множество аналогичных инцидентов, в том числе
документально подтвержденных. Есть книга Марии Рольникайте «Я должна
рассказать», основанная на ее дневниковых записях о Вильнюсском гетто. Эти
записи удалось вынести за пределы гетто ее учителю-литовцу. Она — узница
Вильнюсского гетто, откуда в Освенцим были депортированы ее мать, сестра и брат
(отец воевал на фронте). Нацисты отправили Марию в концентрационный лагерь
Штуттгоф. Оттуда в 1945 году ее вынесли на руках наши красноармейцы. Они
случайно заметили едва дышавшую девочку, лежавшую среди трупов.

В книге Рольникайте очень ярко описано, как литовские соседи и одноклассники, с
которыми раньше всегда были добрые отношения, внезапно не только переставали
здороваться, но и начинали травить и издеваться. Одно из самых интересных
исследований последнего времени на эту тему — монография Руты Ванагайте,
литовской журналистки и историка, под названием «Свои» (другой перевод на
русский язык названия этой книги — «Наши» — прим. «Ленты.ру»). Она на
материалах устной истории и документов анализирует эту проблему, до сих пор
неизжитую литовским обществом. Настолько неизжитую, что Ванагайте в Литве после
выхода книги подвергли остракизму и стали изымать ее книги из магазинов. Книга
переведена на русский и многие другие языки.

