% vim: keymap=russian-jcukenwin
%%beginhead 
 
%%file 25_01_2022.fb.fb_group.story_kiev_ua.1.vysockij_i_kiev.cmt
%%parent 25_01_2022.fb.fb_group.story_kiev_ua.1.vysockij_i_kiev
 
%%url 
 
%%author_id 
%%date 
 
%%tags 
%%title 
 
%%endhead 
\zzSecCmt

\begin{itemize} % {
\iusr{Елена Мельникова}
Вечно жив, вечно любим...

\iusr{Tatyana Smirnova}
Спасибо. Интересно, еженедельник знала много о бабушке

\iusr{Tatyana Smirnova}
Ничего не знала, телефон чудит

\iusr{Людмила Краснюк}

Какая удивительная киевская бабушка была у Владимира Высоцкого! @igg{fbicon.heart.red} Спасибо,
Оксана, что поделились с нами такой интересной историей ! @igg{fbicon.hands.pray}  @igg{fbicon.hearts.two} 

\iusr{Оксана Денисова}
\textbf{Людмила Краснюк} Спасибо Вам, рада, что интересно!

\iusr{Александр Лазаретник}
Шедевр. Второго Высоцкого нет. И никогда не будет.

\iusr{Оксана Денисова}
\textbf{Александр Лазаретник} Согласна!

\iusr{Евдокия Зуева}
Светлая память!

\iusr{Вика Нурибекова}

Говорили, что в архиваз студии Довженко были записи его концертов в том числе и
в НИИ, для студентов - для небольшой аудитории. Было около сотни бабин.

\begin{itemize} % {
\iusr{Gregory Kushnir}
\textbf{Вика Нурибекова}, 

я та моя сестра виросли на цих бобінах!.. В залі сміх не замовкав, десь навіть
до істерики.. Мама ще тоді казала, що хтось тайкома позаписував ці монологи та
пісні

\iusr{Вика Нурибекова}
\textbf{Gregory Kushnir} у нас были все бабины 94 штук, но пожар все уничтожил.

\iusr{Вика Нурибекова}
\textbf{Gregory Kushnir} 

а цитаты и сейчас актуальны : \enquote{настоящих буйных мало - вот и нету вожаков},
\enquote{зачем аборигены сьели Кука - молчит наука}, \enquote{поэты ходят пятками по лезвию
ножа и режут в кровь свои босые души}, \enquote{я не люблю фатального исхода, от жизни
никогда не устаю} и можно продолжать и продолжать...


\iusr{Gregory Kushnir}
\textbf{Вика Нурибекова}, 

усе це, та багато інших пісень Володимира Семеновича в нашій родині розібрані
на цитати)). Навіть у колі друзів дитинства це вже меми)), бо ці бабіни (у нас
із десяток) були не раз переписувані)

\end{itemize} % }

\iusr{Алена Милько}
Респект автору!

\iusr{Оксана Денисова}
\textbf{Алена Милько} Спасибо!

\iusr{Zyama Syu}
Таких, как Владимир Высоцкий больше нет! @igg{fbicon.heart.red}{repeat=3}

\iusr{Регина Бочковская}
Спасибо большое за рассказ о бабушке. Светлая память Владимиру и его замечательной бабушке.

\iusr{Оксана Денисова}
\textbf{Регина Бочковская} Спасибо Вам!

\iusr{Инна Смирнова}
Красиво как написали, спасибо!!!

\iusr{Оксана Денисова}
\textbf{Инна Смирнова} Спасибо Вам!

\iusr{Jacob Sukhoy}

Нужно было эмигрировать в Чикаго что бы попасть на выступление Высоцкого это
было в Январе 1979 года в этот день выпало немного снега хотя в эту зиму
1978/1979 годов была одна из самых больших снежных осадков за историю
города. Вот так город Чикаго встретил Высоцкого.  @igg{fbicon.hands.pray} 

\begin{itemize} % {
\iusr{Оксана Денисова}
\textbf{Jacob Sukhoy} Спасибо за Ваши воспоминания!

\iusr{Natalia Bastun}
\textbf{Jacob Sukhoy} interesting!!!!
\end{itemize} % }

\iusr{Валентина Белозуб}

Работая на Украинском телевидении (Крещатик 26), была свидетелем записи песен
Владимира Высоцкого в нашей студии! Незабываемые воспоминания, поскольку ни
капли звёздности в поведении Владимира не было! Работа, работа и уважение к
окружающим !!! Светлая память!!!

\begin{itemize} % {
\iusr{Оксана Денисова}
\textbf{Валентина Белозуб} Спасибо Вам большое, очень интересно!

\iusr{Валентина Белозуб}
\textbf{Оксана Денисова} Оксаночка, это был незабываемый концерт с единственным исполнителем Высоцким. Эмоции переполняли ...

\iusr{Оксана Денисова}
\textbf{Валентина Белозуб} Ох, завидую белой завистью @igg{fbicon.heart.red}

\iusr{Елена Шестопал}
\textbf{Валентина Белозуб} ,здорово!!!!

\iusr{Ольга Патлашенко}

А еще, несмотря на свой очень плотный график, в этот первый гастрольный приезд
в Киев в 71 году Вл. Сем. нашел возможность и время выступить перед студентами
и преподавателями КТИЛПа в нашем актовом зале. И мне, студентке 2 курса,
повезло побывать на этом незабываемом концерте! Концерт, который длился 2 часа,
прошел на одном дыхании.

И сегодня, в его день рождения, мы опять вспоминаем нашего кумира. Спасибо автору публикации

\iusr{Елена Дарова}
\textbf{Валентина Белозуб} А когда это было?

\end{itemize} % }

\iusr{Инна Валентиновна}

Я ходила на его концерт с мамой в Киеве. Столько милиции было вокруг. Такой
ажиотаж, столько народу! Спасибо за воспоминание. А Высоцкий действительно один
и неповторим. Киев имеет к нему непосредственное отношение по линии любимой
бабушки.

\begin{itemize} % {
\iusr{Оксана Денисова}
\textbf{Инна Валентиновна} Как Вам повезло, видеть и слышать Высоцкого вживую! Завидую по- хорошему!

\iusr{Инна Валентиновна}
\textbf{Оксана Денисова} согласна, это счастье видеть и слышать его.
\end{itemize} % }

\iusr{Natasha Levitskaya}

С Днём рождения, Владимир Семёнович! Помним!
Нам повезло - мы жили с вами в одно время!
Светлая и добрая память!@igg{fbicon.heart.red}

\iusr{Августина Тытянчук}
Светлая память!!!

\iusr{Igor Neronov}
Господи, а есть кто-то из «великих», в жилах которого нет иудейской крови, хотя
бы в третьем поколении?)))

\begin{itemize} % {
\iusr{Татьяна Демидова}
\textbf{Igor Neronov} нет, конечно. Сам Христос и Матерь божья были евреями. Народ избранный, умный, сильный духом своим

\iusr{Igor Neronov}
\textbf{Татьяна Демидова} Вы заглядывали в их паспорт? Или тетя Роза нашептала?))))

\iusr{Маргарита Марченко}
\textbf{Igor Neronov} конечно нет!

\begin{itemize} % {
\iusr{Igor Neronov}
\textbf{Маргарита Марченко} вот, так и раскрываются подробности пятой графы \enquote{Б} класса...))))

\iusr{Anna Vasilyuk}
\textbf{Igor Neronov} вирішили побути ложкою дьогтю в бочці з медом?!! @igg{fbicon.face.tears.of.joy} дякувати не треба.

\iusr{Igor Neronov}
\textbf{Anna Vasilyuk} 

Ну, не нравится мне Высоцкий, ни песни его, ни страсть к алкоголю и наркотикам,
так что, мне закатить глаза к небу и прохрипеть про \enquote{черный пистолет}?)))

\iusr{Татьяна Демидова}
\textbf{Igor Neronov} что вы знаете про его страсти? Только то, что вам написали. Хотя, Пушкин тоже не всем нравился....

\iusr{Маргарита Марченко}
\textbf{Igor Neronov} коНЭшно!
\end{itemize} % }

\end{itemize} % }

\end{itemize} % }

