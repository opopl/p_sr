% vim: keymap=russian-jcukenwin
%%beginhead 
 
%%file 21_08_2021.fb.fb_group.story_kiev_ua.1.zamkovaja_gora.pic.1.cmt
%%parent 21_08_2021.fb.fb_group.story_kiev_ua.1.zamkovaja_gora
 
%%url 
 
%%author_id 
%%date 
 
%%tags 
%%title 
 
%%endhead 

\iusr{Татьяна Сирота}

{\em\color{blue}\bfseries Памятник Николаю Васильевичу на Андреевском спуске у подножия Замковой горы}

\iusr{Елена Соколова}
А пера в руке уже нет...

\iusr{Татьяна Сирота}
\textbf{Елена Соколова} А я и не заметила.

\iusr{Елена Соколова}
\textbf{Татьяна Сирота} 

просто у меня есть фото с пером, а 2 недели назад была на Андреевском и мне
сказали, что нет пера ( наши \enquote{любители} сняли ). Смотрю, таки да...


\iusr{Татьяна Сирота}
\textbf{Елена Соколова} Вот и как с этим бороться?!!! @igg{fbicon.anger}{repeat=3} 

\iusr{Елена Соколова}
\textbf{Татьяна Сирота} Не знаю. Обидно. @igg{fbicon.sun.with.face} 

\iusr{Ирина Чуприна}
Элегантная девушка из того Николая Васильевича получилась.

\iusr{Alex Maler}

Смотрю я на этот \enquote{памятник} - никто не пробовал вот так сесть на что-либо такой
же высоты и вытянуть именно так ноги, закинув одну на другую?

Я, лично, не поленился скопировать эту позу...

Результат был плачевным, я просто упал на спину.

Скульптор - халтурщик.

\iusr{Татьяна Сирота}
\textbf{Alex Maler} 

Не повезло Вам! У меня получилось. При этом, чувствовала себя достаточно
комфортно. @igg{fbicon.wink}{repeat=3} 
