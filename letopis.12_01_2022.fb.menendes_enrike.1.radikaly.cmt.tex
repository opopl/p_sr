% vim: keymap=russian-jcukenwin
%%beginhead 
 
%%file 12_01_2022.fb.menendes_enrike.1.radikaly.cmt
%%parent 12_01_2022.fb.menendes_enrike.1.radikaly
 
%%url 
 
%%author_id 
%%date 
 
%%tags 
%%title 
 
%%endhead 
\zzSecCmt

\begin{itemize} % {
\iusr{Юрий Лукшиц}

Просто текст актуальный, отсюда активность. Сейчас многие пытаются предсказать
будущее Украины. А с комментаторами в \enquote{Фейсбуке} лично я веду себя очень
просто. На хамство, угрозы и т.д. - жалоба плюс блокировка. С такими
бескультурными персонажами нет никакого смысла общаться.

\iusr{Иван Воробьев}

А что, гетман Ходкевич сейчас у вас считается героем или зрадником? Вообще в
долине Днепра это известный персонаж? У нас - не очень, если что, мало кто
поймет скрытый смысл идиомы.  @igg{fbicon.smile} 

\iusr{Максим Дудник}

Может, эффект порохоботов и фейсбука в 19м означает, что в фейсбуке, и правда,
собралась самая умная часть народа? (Быстро убегает...) (Возвращается) А если
эта - самая умная, то на других вообще смотреть страшно... (убегает еще
быстрее)

\iusr{Энрике Менендес}
\textbf{Максим Дудник} Сектанты априори не бывают самыми умными )

\iusr{Виктория Пасечник}

важная поправка. украинская аудитория фб - 16 млн активных пользователей (на
средину 2021 года). это аж никак не абсолютное меньшинство

\begin{itemize} % {
\iusr{Энрике Менендес}
\textbf{Виктория Пасечник} думаете это реальная цифра?

\iusr{Виктория Пасечник}
она фигурирует +/- в нескольких источниках одинаковая. видимо, да 

\href{https://ain.ua/ru/2021/01/25/ukrainskij-facebook-instagram-2021}{%
За полгода украинский Facebook вырос до 16 млн, а Instagram — до 14 млн: исследование, %
ain.ua, 25.01.2021%
}

\iusr{Ренат Якубов}
\textbf{Виктория Пасечник} 

В это число явно включены люди, кои заходят в ФБ раз в пол-года, чтоб принять
участие в каком-нибудь розыгрыше. Формально да, они на ФБ есть, фактически -
нет. Я уж не говорю о том, что большинство даже сравнительно активных
пользователей не лезут за пределы френдленты, наипаче в
общественно-политические сра... дискуссии. И правильно делают.

\end{itemize} % }

\iusr{Александр Лапин}

Хорошо бы ещё СМИ перестали выдавать вопли праздношатающихся на фейсбучной
помойке за новости @igg{fbicon.wink}  А то ведь действительно дожились до того, что если ты
что-то доброе, нужное людям и правильное сделал, но не рассказал об этом в
Facebook, то вроде как ничего и не сделал  @igg{fbicon.face.tears.of.joy}  А будущее всё равно за Telegram...

\iusr{Энрике Менендес}
\textbf{Александр Лапин} да, СМИ ужасающие по качеству. Читать нечего

\iusr{Kasmir Charan}
скільки разів в коментах назвали \enquote{сепаром})

\iusr{Энрике Менендес}
\textbf{Kasmir Charan} бесчисленное число раз )


\iusr{Валерий Прявинов}
Загнать за Можайск...  @igg{fbicon.smile}  :):)

\begin{itemize} % {
\iusr{Энрике Менендес}
\textbf{Валерий Прявинов} это выражение часто использует один мой знакомый олигарх. От него и прицепились ))) подходит ко многим ситуациям

\iusr{Владимир Михайловский}
\textbf{Энрике Менендес} Правильно \enquote{За Можай}. В этом случае и особо озабоченные будут меньше нервничать.
\end{itemize} % }

\iusr{Daniel Tchikin}

\enquote{Конструктивные граждане} – миф. Если речь о жизни и смерти, а конфликт крайне
острый – всё равно человек, вовлечённый в события, рано или поздно окажется по
одну из сторон баррикад.

Вот ты вчера \enquote{типа сделал попытку проявить нейтральность}, а по факту озвучил
на 100\% российскую повестку дня (не \enquote{прилепинско-мильчаковскую}, а
официальную, кремлёвскую). Это как яркий пример невозможности нейтралитета
изнутри.

\begin{itemize} % {
\iusr{Энрике Менендес}
\textbf{Daniel Tchikin} ну, если эту повестку поддерживает около 30\% граждан Украины, то у меня для тебя плохие новости ))

\iusr{Daniel Tchikin}
\textbf{Энрике Менендес}, 

ну, во-первых, твои цифры взяты с потолка, а во-вторых, сколько бы их ни было -
их не большинство.

Всегда будут, в товарных количествах, \enquote{носители иного мнения}. В Великобритании
в 1940-м было немало сторонников мира с Германией. В США существовало мощное
движение сторонников Нейтралитета. В СССР без пинка со стороны властей едва ли
набралось бы достаточно желающих сражаться с немцами.

\enquote{Йес, а что толку?}, как говорится.


\iusr{Вячеслав Чанов}
\textbf{Daniel Tchikin}  @igg{fbicon.face.rolling.eyes} на последних президентских выборах за такую повестку проголосовали 75\% украинских избирателей, внезапно.
P. S. Про СССР особенно смешно)))

\iusr{Daniel Tchikin}
\textbf{Viacheslav Chanov} , уверены, что за такую? Их кандидат обещал им федерализацию, двуязычие, запрет на вступление в НАТО, дружбу с РФ и суд над военно-политическим руководством страны?

\iusr{Вячеслав Чанов}
\textbf{Daniel Tchikin} 

\enquote{Их кандидат} обещал украинцам мир в результате реализации минских соглашений и
двуязычие на региональном уровне. Другое дело, что у реальных хозяев страны
были другие планы. И победивший кандидат резко сменил и политику, и риторику.

Примечательно, что вы отделяете себя от большинства украинцев.


\iusr{Daniel Tchikin}

Про СССР разрешаю посмеяться. Заградотряды ведь нужны были, чтобы прогонять в
тыл раненых и уставших - а то так рвались в бой, что ни есть, ни отдыхать, ни
лечиться не хотели, приходилось силой заставлять:)

\iusr{Daniel Tchikin}
\textbf{Viacheslav Chanov} , 

ссылку на обещание \enquote{двуязычия на региональном уровне} в студию, пожалуйста. А
то тезис из серии \enquote{Трамп обещал признать Крым территорией РФ}.

А мир в результате реализации Минских соглашений принципиально недостижим - так
как стороны толкуют их диаметрально по-разному. Их не для того подписывали.


\iusr{Daniel Tchikin}

Кстати, кандидат с перечисленными мною программными пунктами тоже на выборы
шёл. Но даже во второй тур не вышел. Бойко его фамилия. Это к вопросу о том,
какую повестку дня поддерживает большинство украинцев.

\iusr{Вячеслав Чанов}
\textbf{Daniel Tchikin}  @igg{fbicon.laugh.rolling.floor}{repeat=3} вы вообще Минские соглашения читали?
Там всё чётко, по пунктам, в том порядке, в котором эти пункты должны быть реализованы.
Другой вопрос, что проигравшую в 2015м году сторону не устраивает вариант, что придётся учитывать интересы победителя, поэтому реализацию соглашений всячески тормозят

\iusr{Daniel Tchikin}
\textbf{Viacheslav Chanov} , а что, в войне уже есть победитель? И кто же он?

\iusr{Вячеслав Чанов}
\textbf{Daniel Tchikin} да, оба мои дедушки, которых \enquote{гнали в бой заградотрядами} (а на самом деле они пошли в военкомат добровольцами), над вами бы посмеялись.

\iusr{Daniel Tchikin}
\textbf{Viacheslav Chanov} , бывали разные. Были и упёртые сталинисты, которые с именем Его в атаку ходили. Были добровольцы, которым плевать на государство, но за страну готовы драться. Но полно было и мобилизованных насильно. И не думаю, что 30\% - а куда больше.

\iusr{Вячеслав Чанов}
\textbf{Daniel Tchikin}  @igg{fbicon.laugh.rolling.floor}{repeat=3} вы \enquote{думаете}, а я просто знаю. Я старше вас, общался со многими людьми того поколения и хорошо знаю, что было у них в голове в момент нападения нацистов. Так вот, за \enquote{заградотряды} вы бы от них получили по морде сразу, как только попытались бы соврать.

\iusr{Daniel Tchikin}
\textbf{Viacheslav Chanov} , по факту, по морде получали прямо после атаки те, кто слишком громко орал "За Сралина!!!". И из личных бесед, и из опубликованных воспоминаний.
Но, если нравится думать, что 95\% населения были сталинолизами - на здоровье. К предмету дискуссии это не относится

\iusr{Вячеслав Чанов}
\textbf{Daniel Tchikin} зато за Бойко проголосовали на \enquote{освобождённом} Донбассе, после чего выборы на контролируемой Киевом донбасской территории перестали проводить вообще  @igg{fbicon.beaming.face.smiling.eyes} 

\iusr{Вячеслав Чанов}
\textbf{Daniel Tchikin} хорошо врать про войну с Гитлером, когда большинство свидетелей уже умерло. Типа никто не возразит)) а ещё можно законодательно запретить возражать вранью, потому что \enquote{российский нарратив}))

\iusr{Daniel Tchikin}
\textbf{Viacheslav Chanov} , 

давайте остановимся на том, что за него проголосовали. 11,6\%, если не ошибаюсь.
Вот именно столько граждан поддержали на выборах ту точку зрения, которую
Энрике изложил в своём резонансном посте. То есть, конечно, каждый 9-й
гражданин - вроде и немало. Но недостаточно, чтобы ради них ставить всю страну
с ног на голову


\iusr{Вячеслав Чанов}
\textbf{Daniel Tchikin} То есть вы признаете, что Минские соглашения украинская сторона выполнять изначально не собиралась?
Спасибо за откровенность @igg{fbicon.beaming.face.smiling.eyes} 

\iusr{Daniel Tchikin}
\textbf{Viacheslav Chanov} , 

пока что законодательно запретили распространять неудобную правду о той войне
только в РФ. Параллельно закрыв общество \enquote{Мемориал}, с формулировкой, дескать,
сведения о репрессиях могут создать СССР \enquote{имидж террористической страны}.

\iusr{Daniel Tchikin}
\textbf{Viacheslav Chanov} , как и российская сторона. Которая вообще теперь утверждает, что она - \enquote{не сторона}.

\iusr{Вячеслав Чанов}
\textbf{Daniel Tchikin} 

 @igg{fbicon.laugh.rolling.floor} Так на Украине вообще врут, что дивизия
 Ваффен-СС \enquote{Галичина} на самом деле с Гитлером воевала
 @igg{fbicon.laugh.rolling.floor}  \enquote{за незалежнiсть}
 @igg{fbicon.laugh.rolling.floor}  до этого даже \enquote{Мемориал} не
 додумался.

\enquote{Мемориал} 

закрыли за то, что он нацистов и их помощников пытался представить как
\enquote{незаконно репрессированных} @igg{fbicon.beaming.face.smiling.eyes} 


\iusr{Вячеслав Чанов}
\textbf{Daniel Tchikin} 

на стороне республик воевали не только граждане России, но и граждане Чехии,
Сербии, Бразилии, Франции, Испании... Считать ли эти страны \enquote{сторонами в
войне}?

\iusr{Daniel Tchikin}
\textbf{Viacheslav Chanov} , ну, разумеется. Среди репрессированных, с точки зрения путинщины, вообще незаконных не было: кто не нацист - тот помощник, и никаких гвоздей:)
Кстати, так, для прояснения ситуации - вы лично в какой стране живёте?

\iusr{Вячеслав Чанов}
\textbf{Daniel Tchikin} в своей стране я живу, в своей @igg{fbicon.beaming.face.smiling.eyes} 
А вас в Донецке \enquote{оккупанты} ещё не репрессировали? Или вы не в Донецке живете?

\iusr{Daniel Tchikin}
\textbf{Viacheslav Chanov} , ну, если в Донецк завозили сербские танки или чешские боприпасы, а боевики НВФ состояли на испанском денежном довольствии - тогда, разумеется, считать

\iusr{Daniel Tchikin}
\textbf{Viacheslav Chanov} , это вопрос дискутабельный.
Но как Ваша страна называется?

\iusr{Вячеслав Чанов}
\textbf{Daniel Tchikin}  @igg{fbicon.thinking.face} как же так, вы \enquote{заукраинец} и вас в Донецке никто не трогает? Или вы не в Донецке живете?

\iusr{Daniel Tchikin}
\textbf{Viacheslav Chanov} , вы на вопрос не ответили

\iusr{Вячеслав Чанов}
\textbf{Daniel Tchikin} моя родина - СССР. И я никуда не эмигрировал.  @igg{fbicon.beaming.face.smiling.eyes} 

\iusr{Daniel Tchikin}
\textbf{Viacheslav Chanov} , то есть, вы живёте в СССР? И кто же нынче генеральный секретарь КПСС? Или по-прежнему обязанности президента исполняет М. С. Горбачёв?

\iusr{Вячеслав Чанов}
\textbf{Daniel Tchikin}  @igg{fbicon.beaming.face.smiling.eyes} теперь ваша очередь на вопросы отвечать.
Вы сейчас в Донецке живете или просто наврали в профиле?

\iusr{Daniel Tchikin}
\textbf{Viacheslav Chanov} , 

жить в Донецке, публично декларируя даже самые умеренно-проукраинские взгляды,
невозможно. Вот Энрике, несмотря на всё свою соглашательскую риторику и
лояльность к боевикам, был изгнан \enquote{в 36 часов с одним чемоданом}.


\iusr{Вячеслав Чанов}
\textbf{Daniel Tchikin} То есть в профиле вы врёте, насчёт исторических вопросов вы врёте и насчёт того, что ваши взгляды и есть проукраинские вы тоже врёте.
Понятно.

\iusr{Daniel Tchikin}
\textbf{Viacheslav Chanov} , мне, собственно, наплевать, что понятно тем, кто даже своё лицо стесняется показать, и на простой вопрос прямо ответить. В отличие от вас, мне не страшно свою позицию заявить публично, не прячась.

\iusr{Елена Кузина}

Официальная кремлевская повестка это Украина кандидат на вступление в ЕС? Где
Россия просто по тихому регулирует какие то соглашения , чтобы Украина могла
беспепятственно вступить в ЕС?

Или Крым куда Россия должна впустить Украину на правах концессии хотя может
никогда это не делать. И т.д и т.п

А России это все зачем?

Ради не вступления в НАТО так и так Украина туда не вступит, ибо не дадут.

По моему Энрике как раз озвучил реально самую проукраинскую ура патриотическую
повестку в которой сплошные профиты для Украины.

Назвать ее прокремлевской это надо очень постараться


\iusr{Вячеслав Чанов}
\textbf{Daniel Tchikin} 

я и мои друзья свою позицию заявили публично, когда в 2014 году защищали
памятники Ленину от боевиков с Майдана. И я точно знаю, что такие как вы - не
Украина. Вы маленькое меньшинство, которое держится у власти враньём и насилием
над большинством украинцев. Без иностранной поддержки вас бы уже давно смели.


\iusr{Олександр Костюченко}
\textbf{Viacheslav Chanov} І податки в сересер платите?

\iusr{Daniel Tchikin}
\textbf{Елена Кузина} , \enquote{кандидатом} можно быть и 40 лет, как Турция. Такая формулировка РФ вполне устроила бы.
Что до всего остального - у нас с Вами (и с Энрике, судя по всему) просто разное понимание, что есть патриотизм, и как будет лучше для Украины

\iusr{Daniel Tchikin}
\textbf{Viacheslav Chanov} , Донецк - это Украина?

\iusr{Вячеслав Чанов}
\textbf{Daniel Tchikin} Украина - это СССР @igg{fbicon.beaming.face.smiling.eyes} 

\iusr{Остап Петренко}
\textbf{Вячеслав Чанов} , 

послушай, совок, басни о той далёкой войне, твои дидываивали, вся эта
историческая хрень про дивизии СС /не СС, уже большинство населения не
интересуют, так как сейчас идёт другая война, и на ней тоже воюют чьи-то
сыновья, мужья, отцы, деды, не цепляют уже мифы про ту далёкую войну


\iusr{Вячеслав Чанов}
\textbf{Остап Петренко} та понятно, вам чтобы гнать украинцев на эту войну, надо врать про ту. Причём врать много и постоянно.

\iusr{Остап Петренко}
\textbf{Вячеслав Чанов} , 

украинцы свою землю защищают от россиян, которых тоже гонят на эту войну, а кто
и за русскомирную идею воюет, а во вранье вашу российскую пропаганду никто не
перещеголяет

\iusr{Вячеслав Чанов}
\textbf{Остап Петренко}  @igg{fbicon.laugh.rolling.floor}{repeat=3}  

как же так, воюют \enquote{россияне}, но как только идёт большой обмен пленными, так
Киев в Донецк и Луганск всё сплошь граждан Украины отдаёт на обмен. Как в
Рождество 2017 года. Из 720 пленных, отпущенных тогда Киевом, 705 - граждане
Украины.

\iusr{Олексій Соціалістичний}
\textbf{Остап Петренко} \enquote{дидываивали}, \enquote{совок} это все что, нужно о таких демократах ,патриотах и евроинтеграторах как вы

\iusr{Вячеслав Чанов}
\textbf{Олексій Соціалістичний} 

они просто хотят интегрироваться в Новую Европу образца 1942 года и до сих пор
не поняли, что эта Эуропа закончилась в мае 1945го

\iusr{Остап Петренко}
\textbf{Вячеслав Чанов} , 

ну там же теперь у военнослужащих желанные для них российские паспорта, а
ментально они всегда были с Россией. А воевали непосредственно с Российскими
войсками когда они заходили батальонными группами во время Иловайска (тогда их
в плен брали, десантничков всяких), и во время Дебальцево, но танкисты воевали
на дистанции, это же не пехота, в плен попасть тяжёло

\iusr{Остап Петренко}
\textbf{Олексій Соціалістичний} , и что, тебе стало легче от такого знания?

\iusr{Вячеслав Чанов}
\textbf{Остап Петренко} конечно, людям как Олексій Соціалістичний всегда легче, когда они по отношению к нацистам не испытывают иллюзий.

\iusr{Остап Петренко}
\textbf{Вячеслав Чанов} , ну и пускай себе радуется!

\iusr{Вячеслав Чанов}
\textbf{Остап Петренко} а то есть как минимум пехота республик из украинских граждан состоит? Ценное признание, так глядишь и дойдёт, что против Киева не Россия воюет.

\iusr{Остап Петренко}
\textbf{Вячеслав Чанов} , 

ты про их российские паспорта прочёл, или тут видим, тут не видим? Слушай, мне
не интересно это размазывание, тем более что меня устраивает текущая ситуация,
ещё бы стрелять перестали и было бы вообще хорошо, все были бы довольны. Может
по стрельбе и договорятся и заморозят конфликт, и славно!


\iusr{Вячеслав Чанов}
\textbf{Остап Петренко} в 2014 году? Российские паспорта у жителей Донбасса?
Пойди проспись!
А насчёт прекращения стрельбы сделать элементарно - нужен просто приказ киевских генералов.

\iusr{Остап Петренко}
\textbf{Вячеслав Чанов} , 

и московских, чтобы своим шестёркам в \enquote{республиках} приказали. Если бы
Москва не вмешалась в 14-м, то было бы на всём Донбассе как сейчас в Славинске,
Краматорске и Мариуполе, но получилось как получилось, теперь все довольны и
счастливы, я надеюсь...


\iusr{Вячеслав Чанов}
\textbf{Остап Петренко} если бы в 2014м реально \enquote{Москва вмешалась}, граница бы проходила по Днепру минимум)))
Не веришь - у Саакашвили спроси. В 2008 как раз реальная российская армия с Грузией воевала а не выдуманная))

\iusr{Яна Дрюченко}
\textbf{Остап Петренко} ви розмовляєте з істотою, яка все ще живе в СРСР. Ви впевнені, що цьому бідосі можна щось пояснити? Навіть, якщо він тут «не на службе»

\iusr{Остап Петренко}
\textbf{Вячеслав Чанов} , всё, надоел, кремлебот, ПНХ!

\iusr{Вячеслав Чанов}
\textbf{Остап Петренко} я так и знал, что против фактов возразить нечего))) но для \enquote{украинцев} это нормально)) им от фактов пердак разрывает))

\iusr{Вячеслав Чанов}
\textbf{Яна Дрюченко} так и вы за счёт наследия СССР живете. Но скоро пилить станет вам нечего. И вороги больше ни газа, ни электричества не подадут.

\iusr{Остап Петренко}
\textbf{Вячеслав Чанов} , 

послушай, мне нравится что твоя Россия деградирует, изолируется, что она
разругалась практически со всеми соседями, что внутри неё живёт такое
количество наций и народностей, которые, мягко говоря, не любят друг друга,
надеюсь что во время, либо после трансфера власти вся путинская вертикаль очень
быстро посыплется, проблемы возникнут похлеще казахстанских, только помочь ей
будет некому, а добивать будет кому...


\iusr{Вячеслав Чанов}
\textbf{Остап Петренко} хороший у вас манямирок, завидую)))

\iusr{Яна Дрюченко}
\textbf{Viacheslav Chanov} про деградацію чия б корова мукала, московіт.

\end{itemize} % }

\end{itemize} % }
