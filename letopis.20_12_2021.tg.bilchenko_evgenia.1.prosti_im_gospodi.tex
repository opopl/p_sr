% vim: keymap=russian-jcukenwin
%%beginhead 
 
%%file 20_12_2021.tg.bilchenko_evgenia.1.prosti_im_gospodi
%%parent 20_12_2021
 
%%url https://t.me/bilchenkozhenya/5359
 
%%author_id bilchenko_evgenia
%%date 
 
%%tags bilchenko_evgenia,travlja
%%title БЖ. Прости им, Господи, ибо не ведают, что творят
 
%%endhead 

\subsection{БЖ. Прости им, Господи, ибо не ведают, что творят}
\label{sec:20_12_2021.tg.bilchenko_evgenia.1.prosti_im_gospodi}

\Purl{https://t.me/bilchenkozhenya/5359}
\ifcmt
 author_begin
   author_id bilchenko_evgenia
 author_end
\fi

БЖ. Прости им, Господи, ибо не ведают, что творят

Близкие не понимают. Не хотят открыть глаза. Я понимаю. Дни мои сочтены. На
Украину я не вернусь, я не могу и не хочу. Я думала, что того, что я сделала
там после покаяния, для России будет достаточно. Оказалось, недостаточно. Это
вечный штрафбат, и он должен кончиться моей смертью. Обе стороны толпы
удовлетворит только моя смерть. Ужасно то, что я ненавижу свою жизнь и не могу
сама ее прервать: религия не позволяет, страх, Бог, я смотрю на окно и не
могу... 

\ii{20_12_2021.tg.bilchenko_evgenia.1.prosti_im_gospodi.pic.1}

Люди - не дальновидны. Даже те, кто меня поддерживает. Они думают, что бан
решит проблемы. В сетевом обществе бан - это пшик. Виртуальное структурирует
реальную жизнь. Нет никакой "карающей десницы власти": властью в 21 веке
управляет толпа, это низовой контроль, это демократия. Товарищ Сталин,
помогите, где вы... Сталин бы или расстрелял, или простил, и все бы кончилось.
Где же этот суровый Отец, батя, чтоб раз и навсегда пресечь цепи сетевого
общества?

Украинцы, русские, кто при памяти, вы помните, почти год назад, я выступила в
сети в защиту прав русских с критикой закона о языке? Стерненко поднял
виртуальную травлю. За ней последовали: скандал, увольнение, общественные суды,
допросы, нападения и, как следствие, изгнание в Россию. Служба занятости
Украины выдала мне волчий билет во все вузы как носителю русского мира.

Я не извинилась перед ними. Я вела борьбу с украинскими националистами до
конца: все пять лет после покаяния и уже насмерть - весь 2020 год. Но и этого
мало. Все российские СМИ мне тогда звонили и освещали это, сочувствовали,
прощали мне прошлое. Я верила, я и сейчас верю. Я реально верю России.

С 2014 по 2016 годы я более-менее поддерживала курс нынешней Украины. Как и
большинство населения сейчас. Я это делала искренно, пока не поняла, что правда
не на стороне Украины. Да, для зрелого решения мне нужно было побывать на этой
войне лично. Иначе по сети у меня не выходит. Одного не понимаю... Почему ДО
покаяния я СВОБОДНО катала по России с концертами, а мой текст "Кто я" читали
поэты на площадях. Где были патриоты? Я бы ту себя сейчас сюда бы не пустила.
Но патриоты молчали, а по России грохотали мои стихи. Мне страшно признаться: а
если, это такая игра, так все договорено, так и надо? Все роли распределены в
матрице. Иноагентам - можно. Покаявшимся преданным детям русского мира -
нельзя.

И вот я стала "пророссийской". Так это называется? Не здесь, в России, а там на
Украине. Ни за какие деньги. Только потери. Пять лет я теряла и теряла: одну
работу за другой. Организовывала какие-то кружки, ездила, сопротивлялась. Не
переезжала. За это в России мне закрыли все выступления, и началась дикая
травля с обеих сторон. Я вышла за флажки: нельзя каяться. Надо быть врагом,
чтобы тебя уважали, так? Но я не могла иначе. Я нашла правду и не могла от нее
отступиться, любой ценой. Любой. Правда на стороне России, на которую напал мир
посредством Украины как инструмента. Потому началось уничтожение русских на
Украине. Последний год я пошла на большое испытание, я верила, что все сделала
правильно... Все, что я пишу сейчас я говорила везде на Украине в полный голос.

В итоге два моих паблика зеркалят друг друга. Украинска толпа кричит: "Чемодан,
вокзал, Россия", а российская: "Чемодан, вокзал, Украина". Оба крыла толпы
сходятся на одном: "Предателей нигде не любят". Оба крыла толпы на моей почве
подружились. То, что сейчас сделали русские патриоты, украинские СМИ репостят
как назидание украинцам: Россия страшна, не кайтесь. Пять лет антифашистской
борьбы на Украине выброшены на помойку, они не зачлись.

Россия могла бы на мне выиграть информационную войну, но она проигрывает.

Прости, Господи, моим русским гонителям, ибо не ведают, что творят.

Что я передала, если я обрела истину через покаяние и поплатилась? Что?

Сравниваю: на Украине мне нельзя работать, свободно передвигаться, опасно, но
ещё можно писать в сети и продавать свои рукописи, перебиваясь хоть как-то. Ещё
там есть врачи, за деньги. И там можно выступать раз в году с охраной. В России
нельзя: работать, просить о помощи (ни то, ни то), лечиться, ибо нет полиса,
выступать. Любить Россию тоже нельзя: не имею права, не заслужила. Для поэта
сцена - это жизнь, это душа. Голос мой утопает в ненависти. Я испытываю ярость,
когда то, что происходит в ВК подхватывается украинцами аж до Гордона:
"Приїжджай, Ївго, ми тебе стрінемо".

Я думала, я верила, что годовая травля со стороны украинских националистов
поможет тем русским, что не прощают, понять, что я делала. И хотя бы не бить
лежачего и не заставлять его каяться не только за свои, но и за чужие грехи, за
грехи, которых я не совершала.

Банальное мужество не позволяет мне прятаться. Я устала прятаться. И закрывать
страницы тоже. Но я понимаю по своему состоянию, что это последний этап. Я
вышла напрямую на грань. Я не прошу никакой помощи ни у кого. Это судьба, это
удел. Когда я ехала, я знала, меня же все время преследовали в России, которую
я выбрала не ради коврижек, а вопреки всему, по любви. Я ждала, когда здесь
добьют. Я знала! Мне потому так больно, что по любви, и больно, что российские
гонители пляшут на заказ украинских СМИ.

Я по карте иду за флажки адской правдой в этой адской информационной войне.
