% vim: keymap=russian-jcukenwin
%%beginhead 
 
%%file 03_11_2021.fb.fb_group.story_kiev_ua.1.kiev_i_piter
%%parent 03_11_2021
 
%%url https://www.facebook.com/groups/story.kiev.ua/posts/1789730974557029/
 
%%author_id fb_group.story_kiev_ua,kuzmenko_petr
%%date 
 
%%tags gorod,kiev,rossia,sravnenie,st_peterburg,ukraina
%%title Киев и Питер
 
%%endhead 
 
\subsection{Киев и Питер}
\label{sec:03_11_2021.fb.fb_group.story_kiev_ua.1.kiev_i_piter}
 
\Purl{https://www.facebook.com/groups/story.kiev.ua/posts/1789730974557029/}
\ifcmt
 author_begin
   author_id fb_group.story_kiev_ua,kuzmenko_petr
 author_end
\fi

Предвижу негативные отзывы многих одногруппников на эту публикацию и разделяю
их. Согласен. Этот пост не ко времени. Но, давайте оставим в стороне (только в
этой статье, пока читаем) войну, политику и справедливые чувства отторжения
всего, что связано с северным соседом Украины. 

Я хочу написать о схожести и различии между Киевом и Санкт-Петербургом. И
главным образом между киевлянами и ленинградцами - питерцами. Я намеренно не
пишу жителями наших городов. Потому, что именно киевляне и питерцы как никто
иной, понимают огромное различие между горожанами и жителями, между народом и
населением. Стоп. Опять скатываюсь к жизни общества, а социология и политика
родные сёстры. 

Итак. Позволю себе
процитировать пост Наташи Левицкой "Вот такой берёзовый листопад я увидела
утром возле парадного...". Именно ПАРАДНОГО. Никто, кроме киевлян и
ленинградцев, не употребляет это название. Это малюсенький штрих, которых
довольно много. 

\ifcmt
  tab_begin cols=3

     pic https://scontent-frt3-2.xx.fbcdn.net/v/t1.6435-9/252705994_4682788385106742_2224278969348492717_n.jpg?_nc_cat=103&ccb=1-5&_nc_sid=b9115d&_nc_ohc=eeKhkrgjlzEAX_zPtdo&_nc_ht=scontent-frt3-2.xx&oh=fbc39a9571afe71d416d0cbd2358874b&oe=61A7867B

     pic https://scontent-frt3-1.xx.fbcdn.net/v/t1.6435-9/248945237_4682788448440069_6661484803431502304_n.jpg?_nc_cat=102&ccb=1-5&_nc_sid=b9115d&_nc_ohc=v9hDCL5VueoAX-QtJxb&_nc_ht=scontent-frt3-1.xx&oh=11ab9f4b9c8e7d7126bc7663e8898541&oe=61A9CD1F

		 pic https://scontent-frx5-1.xx.fbcdn.net/v/t1.6435-9/252470227_4682788501773397_7325567013358380469_n.jpg?_nc_cat=105&ccb=1-5&_nc_sid=b9115d&_nc_ohc=z1UNwoXPjs0AX805JeB&_nc_ht=scontent-frx5-1.xx&oh=1659e5a9725d81cd92064b4e698a2d93&oe=61AA66E7

  tab_end
\fi

Впервые я очутился в Ленинграде ещё подольским школьником - семиклассником. У
нас была поездка - экскурсия по обмену между школами Киева и Ленинграда. А
может только Подола и Лиговки, не знаю. Мы приехали на поезде, с нашими
преподавателями, и жили в спортивном зале ленинградской школы. Там были
установлены обычные панцирные кровати, с типовыми советскими матрасами и
бельём. С нашим заселением в спортзал произошла курьёзная история, о которой не
могу не поведать читателям. Сопровождали и возглавляли нашу группу учеников
классный руководитель Рита Николаевна Котляр и преподаватель украинского языка
и литературы Раїса Романівна Шарц. Мальчиков в группе было меньше, чем девочек,
но всё - же достаточное количество (как сказал герой великого Яковченко в
прекрасном фильме "Вий"). 

\ifcmt
  tab_begin cols=3

     pic https://scontent-frx5-1.xx.fbcdn.net/v/t1.6435-9/252771785_4682788698440044_2198950010761898596_n.jpg?_nc_cat=110&ccb=1-5&_nc_sid=b9115d&_nc_ohc=QpNwJqokS0QAX9VSsBQ&_nc_ht=scontent-frx5-1.xx&oh=43742063640e85d4fd3e4173450fe593&oe=61A79AAF

     pic https://scontent-frt3-2.xx.fbcdn.net/v/t1.6435-9/252903921_4682788778440036_4787932020736237001_n.jpg?_nc_cat=101&ccb=1-5&_nc_sid=b9115d&_nc_ohc=NG0jg_Zm9CYAX-dIaOu&_nc_ht=scontent-frt3-2.xx&oh=bbac42682aa48003f0bac0a25955c115&oe=61AB2DE9

		 pic https://scontent-frx5-1.xx.fbcdn.net/v/t1.6435-9/252864170_4682788831773364_7182829272536250219_n.jpg?_nc_cat=111&ccb=1-5&_nc_sid=b9115d&_nc_ohc=FPM6Ev1L6IwAX-__x68&tn=lCYVFeHcTIAFcAzi&_nc_ht=scontent-frx5-1.xx&oh=e082dadca56a953c30624005cea4c68d&oe=61A7F99F

  tab_end
\fi

Когда нас привели в отгороженную для мальчиков часть спортзала, встал вопрос
"Где же преподаватель - мужчина, который поселится с мальчишками и будет
следить за их безмятежным сном белыми ночами?" Этот казус нисколько не смутил
Раїсу Романівну, яка впевнено проголосила: "Я заміню чоловіка!" І, дійсно,
замінила. Даже покурить ночью (водился за некоторыми из нас такой грешок уже в
отрочестве) мы пробирались, лишь заслышав богатырский храп нашей грозной
наставницы. Прошу прощения у руссоязычной части читателей за украиноязычную
вставку, но пишу как было. Киевляне поймут. 

\ifcmt
  tab_begin cols=3

     pic https://scontent-frt3-1.xx.fbcdn.net/v/t1.6435-9/249242167_4682788945106686_1237758368996291605_n.jpg?_nc_cat=102&ccb=1-5&_nc_sid=b9115d&_nc_ohc=gQybu6D3tzkAX8Pd-xF&_nc_ht=scontent-frt3-1.xx&oh=32463527eb25ace6b7ce653217a3fc87&oe=61A76445

     pic https://scontent-frt3-1.xx.fbcdn.net/v/t1.6435-9/250349439_4682789215106659_3288916988678004519_n.jpg?_nc_cat=106&ccb=1-5&_nc_sid=b9115d&_nc_ohc=2DyNivo2U8wAX94CGTj&_nc_ht=scontent-frt3-1.xx&oh=caa5f1f21edf81e7baccbf1ed500fa6d&oe=61AAB155

		 pic https://scontent-frx5-2.xx.fbcdn.net/v/t1.6435-9/252762961_4682789298439984_6068263326052334896_n.jpg?_nc_cat=109&ccb=1-5&_nc_sid=b9115d&_nc_ohc=Cih29y4w4TcAX8h5kH9&tn=lCYVFeHcTIAFcAzi&_nc_ht=scontent-frx5-2.xx&oh=8c3e79ad6111f5eaeafeb56f4bfd99fb&oe=61A98387

  tab_end
\fi

Тогда мы ознакомились со всеми значимыми экскурсионными достопримечательностями
Ленинграда и его окрестностей с помощью групповодов - старшеклассников той же
лиговской школы. Имена двух хороших ленинградских ребят, влюблённых в свой
город и хорошо знающих его историю, к сожалению, стёрлись в моей памяти. Но, я
хорошо их помню и общее фото на память пылиться где-то на антресоли подольской
квартиры. Предполагаю, что в то же время, по набережной Днепра, Андреевскому
спуску, Киево - Печерской лавре и Софиевскому собору ленинградских школьников с
Лиговки водили старшеклассники нашей школы, не только выполняя общественное
поручение, но и от души, рассказывая и показывая во всей красе величественный
Киев. 

\ifcmt
  tab_begin cols=2

     pic https://scontent-frt3-1.xx.fbcdn.net/v/t1.6435-9/252540436_4682789421773305_3087772108760484232_n.jpg?_nc_cat=107&ccb=1-5&_nc_sid=b9115d&_nc_ohc=YdxmhiPm2lUAX-v7LA7&tn=lCYVFeHcTIAFcAzi&_nc_ht=scontent-frt3-1.xx&oh=f4c05a69db1f950896ceb531fa2958bb&oe=61A953E5

     pic https://scontent-frt3-2.xx.fbcdn.net/v/t1.6435-9/252786169_4682789518439962_1888792843670023982_n.jpg?_nc_cat=101&ccb=1-5&_nc_sid=b9115d&_nc_ohc=PJd2uChdiOgAX_TQxiX&_nc_ht=scontent-frt3-2.xx&oh=2eb44d26a7654298f920b3a04f504e1b&oe=61A8AE30

  tab_end
\fi

После первого посещения мне ещё очень много раз приходилось бывать в Ленинграде
- Санкт-Петербурге. Хорошо изучить и даже по-своему полюбить этот прекрасный
город. Однако, я снова отошёл от основной нити своего повествования за что
нижайше прошу прощения у читателей. Так, в чём же похожи киевляне и
ленинградцы? (Мне будет удобней называть жителей Северной Пальмиры так, да
простит меня придирчивый читатель). 

\ifcmt
  tab_begin cols=2

		 pic https://scontent-frx5-2.xx.fbcdn.net/v/t1.6435-9/252404705_4682789628439951_6283323375286324366_n.jpg?_nc_cat=109&ccb=1-5&_nc_sid=b9115d&_nc_ohc=rwJ-mVMXYf0AX_QmZmD&tn=lCYVFeHcTIAFcAzi&_nc_ht=scontent-frx5-2.xx&oh=659783a065e60f8c15af13834341a286&oe=61A82E7E

     pic https://scontent-frt3-1.xx.fbcdn.net/v/t1.6435-9/252771787_4682789735106607_5934599253636517714_n.jpg?_nc_cat=102&ccb=1-5&_nc_sid=b9115d&_nc_ohc=0hu-UaPoXBoAX-ww_fy&_nc_ht=scontent-frt3-1.xx&oh=ad5e5f5651a375894e6cb73828f85b47&oe=61AABF6C

  tab_end
\fi

Итак, про "парадное" я уже писал. Тут можно вспомнить и бордюры, которые
киевляне упорно называли и называют по-своему бровкой, а ленинградцы
поребриком. И то, что, спросив где-нибудь на Невском "Как пройти на Фонтанку?",
вы часто услышите "Идите сюдой пару кварталов". Не задумывались почему такой
ответ? Думаю, потому, что Санкт-Петербург сразу проектировался и строился
прямоугольными кварталами, как и позже, восстановленный после страшного пожара
начала 19 века, киевский Подол. Кстати, подольский Гостиный двор младший брат
питерского. 

И, уж если речь зашла об архитектуре, как не отметить величественные творения
Франческо Бартоломео Растрелли, самую лучшую и близкую для меня родную
величественную Андреевскую церковь в Киеве и прекрасный Смольнинский монастырь
в Санкт-Петербурге. Уютный и живописный Мариинский и монументальный Зимний
дворцы в этих городах. Про Питер - колыбель трёх революций и город с самыми
радикальными и протестными, в отношении власти, настроениями известно многим.
Что говорить о Киеве, с нашей студенческой революцией на граните, и двумя
Майданами, последний из которых, народный и кровавый запомнил весь мир. Ну вот
не могу я без политики. Но это схожесть, простите. 

Напишу и об отличиях между городами - столицами (бывших не бывает). В Киеве
теплее, легче дышится, даже сейчас, невзирая на все "старания" многих поколений
КМДА, больше зелени, и конечно, больше свободы.

Санкт-Петербург, напротив, трогает больше всего строгостью форм и пейзажей,
своей монументальностью. Что питерскому гостю бросается в глаза в первую
очередь? В Киеве много горок. Шагать придётся то вверх, то вниз. Настоящая
проблема для девушек, которые уже через 20-30 минут попросят отвести их в
первое попавшееся кафе или просто купить кофе в ближайшем ларьке. 

В Санкт-Петербурге подъёмы и спуски встречаются лишь на эстакадах, виадуках и
мостах, а ларьков с кофе и фаст-фуд ом практически нет. Зато там много
небольших кафе, где можно укрыться от назойливого дождя. За пределами своих
исторических центров и Ленинград - Санкт-Петербург и Киев практически
неотличимы от всякого другого города - миллионника. Подобие ленинградских
микрорайонов с их близнецами в других городах даже легло в основу знаменитого
рязановского фильма любимого уже не одним поколением советских и постсоветских
людей. 

Словом, тема эта обширная и развивать её можно очень долго. Но я не хочу
утомлять читателя своими эпистолярными завихрениями. Скажу лишь напоследок, что
очень бы хотелось, чтобы наши внуки (о детях говорить даже не буду) могли
беспрепятственно, открыто и доброжелательно ездить из Киева - столицы свободной
демократичной Украины в демократичный, миролюбивый и настоящий Санкт-Петербург.
И питерцы, с открытыми сердцами и общечеловеческими ценностями в душе приезжали
любоваться нашим прекрасным вечным Городом.
