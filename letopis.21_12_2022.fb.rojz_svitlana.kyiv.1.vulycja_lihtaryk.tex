% vim: keymap=russian-jcukenwin
%%beginhead 
 
%%file 21_12_2022.fb.rojz_svitlana.kyiv.1.vulycja_lihtaryk
%%parent 21_12_2022
 
%%url https://www.facebook.com/svetlanaroyz/posts/pfbid0U28HLbyH2tNzaQ8CDnK5UZ5D8M2dhGzabyBapjvcNHrixUYceVKtTrp5fCsGEFDWl
 
%%author_id rojz_svitlana.kyiv
%%date 
 
%%tags 
%%title Із сьогоднішнього. Йдемо в темряві по вулиці з ліхтариком
 
%%endhead 
 
\subsection{Із сьогоднішнього. Йдемо в темряві по вулиці з ліхтариком}
\label{sec:21_12_2022.fb.rojz_svitlana.kyiv.1.vulycja_lihtaryk}
 
\Purl{https://www.facebook.com/svetlanaroyz/posts/pfbid0U28HLbyH2tNzaQ8CDnK5UZ5D8M2dhGzabyBapjvcNHrixUYceVKtTrp5fCsGEFDWl}
\ifcmt
 author_begin
   author_id rojz_svitlana.kyiv
 author_end
\fi

Із сьогоднішнього. Йдемо в темряві по вулиці з ліхтариком. Біля дороги снігова
гірка, від неї тінь. Кажу донці: дивись, тінь - наче кремль. - кремль, а що це?
а! пам'ятаю, я малювала, як він горить 🙂 

Ми без зв'язку. Вже звично, щоб поговорити з моїми батьками - їдемо по трасі,
ловимо мережу. Це найскладніше зараз - не знати, що з рідними.

А дома пічечка гріє і шурхотить дровами. Коли вмикається світло, ми всі так
вигукуємо "Ура!!!" Що кішка налякано підстрибує на місці. 

В нашому сільському чаті збирають гроші та речі родинам, які цього потребують,
знаходять і віддають господарям котиків, всім селом збирають іграшки, щоб
прикрасити ялинку біля сільради. Перегукуються - на яких вулицях є світло. 

Село повністю темне - кілька вікон світяться під гул генераторів, на вулиці
взагалі людей немає. А чат в вайбері живий і турботливий. 

У нас дома горять свічки та ліхтарики. Пахне мандаринами. Я почала збирати і
сушити мандаринову шкірку, додаю її до свічок - щоб аромат був сильнішим.
Влітку я не змогла зробити свій звичний "літній збір" - з листя калини, малини,
смородини, полуниці, ожини, вишні, м'яти, пелюстками троянд, лавадною....
Зроблю зимовий - з шкіркою мандаринів, апельсинів, гвоздикою і буду змішувати з
чаєм та кидати в печиво. 

Зараз у нас грає різдвяна музика. 

Я намагаюсь слідкувати, щоб всі канали сприйняття - мої та рідних були
задіяні... Звертаю свідомо увагу на смаки, звуки, аромати, доторки...щоб бути
живими. Щоб в темряві і невизначеності  не впадати в анабіоз. 

Сьогодні між сиренами ми розмальовували з донькою вікно. Будемо вітати кожну
секунду світла, що з сьогоднішнього дня додаватиметься. 

І скоро, сподіваюсь, коли включать світло, будемо кричати: Ура!!! Світло! 

І чекаю, коли будемо кричати: Ура! Ми Перемогли. 

Обіймаю, Родино ❤️ як вірю в Перемогу

\ii{21_12_2022.fb.rojz_svitlana.kyiv.1.vulycja_lihtaryk.orig}
\ii{21_12_2022.fb.rojz_svitlana.kyiv.1.vulycja_lihtaryk.cmtx}
