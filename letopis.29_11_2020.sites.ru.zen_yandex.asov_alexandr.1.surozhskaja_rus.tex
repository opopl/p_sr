% vim: keymap=russian-jcukenwin
%%beginhead 
 
%%file 29_11_2020.sites.ru.zen_yandex.asov_alexandr.1.surozhskaja_rus
%%parent 29_11_2020
 
%%url https://zen.yandex.ru/media/id/5eeaf725a3dca453cfdd4b58/otkrytie-surojskoi-rusi-5fa04a2c49e00863eb7c8f3b
 
%%author 
%%author_id asov_alexandr
%%author_url 
 
%%tags russia,surozh,istoria
%%title Открытие Сурожской Руси
 
%%endhead 
 
\subsection{Открытие Сурожской Руси}
\label{sec:29_11_2020.sites.ru.zen_yandex.asov_alexandr.1.surozhskaja_rus}
\Purl{https://zen.yandex.ru/media/id/5eeaf725a3dca453cfdd4b58/otkrytie-surojskoi-rusi-5fa04a2c49e00863eb7c8f3b}
\ifcmt
	author_begin
   author_id asov_alexandr
	author_end
\fi

\ifcmt
  pic https://avatars.mds.yandex.net/get-zen_doc/1579326/pub_5fa04a2c49e00863eb7c8f3b_5fbc1edc0b4af8014994607a/scale_1200
	caption В Сурожской крепости с сыном и экскурсоводом (а потом соавтором) А. Осташко
\fi

\subsubsection{Сказка или быль?}

Всегда манили эти земли. Сказкой казалась легендарная Сурожская Русь в древней
Тавриде, напоенной солнцем, ласкаемой тёплым Чёрным морем. Тем самым, что в
древние русы именовали Сурожским.

И когда я читал волшебные дощечки "Велесовой книги", представлялись мне русские
воины, идущие через сухие степи и поросшие дубами и гарбом горы. И вот они
штурмуют неприступные стены великого Сурож-града, берут одну стену, потом
вторую. И ведет их гнев за поруганные святыни, ибо волхвы отправляя воинов на
рать говорили, что в Суроже, великом и древнем русском городе, захваченном
греками, "наши боги повержены во прах..."

Да, согласно "Велесовой книге", Сурож был русским городом, столицей Сурожской
Руси на Черном море, где славяне жили с 8-го века до н.э. чуть не два
тысячелетия. И сей город греки не раз захватывали, разоряли сла­вянские храмы,
самые богатые и прославленные на Руси. Но вновь и вновь приходили русские
воины, и брали приступом его стены, и освобождали Сурожскую Русь.

После издания "Велесовой книги", ко мне часто обращались люди, зна­комые с
трудами историков и археологов, описывающих древнюю историю Причерноморья и
Тавриды (Крыма). И они порой возмущались: "Сурожская Русь в Крыму? Славянские
боги в великолепных храмах на горах, на берегу Черного моря? Но об этом нет
никаких исторических и археологических данных!"

И в самом деле, не мечта ли это? Не морок ли? Если открыть любой исторический
справочник, там находишь только сведения о том, что город Судак (именуемый в
русских летописях XIV-го века Сурожем) был основан во 2-ом веке н.э. и причем
аланами. Потом были там византийцы, готы, греки, потом венецианцы и генуэзцы,
потом татары и турки... А о славянах ни слова.

В специальных трудах если в связи с Крымом славяне и упо­минаются, то только
как пришлые купцы. Ну может быть были торговые пред­ставительства, пусть даже
славянские кварталы в городах на берегу Черного моря. И только. Да иногда
вспоминается поход князя Бравлина на Сурож в 7-м веке н.э., мол сей славянский
князь взял город, крестился там и ушел.

Вновь и вновь обращаюсь к строкам "Велесовой книги". И вновь передо мной встает
образ великого славянского града. Столицы мощного кня­жества, самого южного,
соприкасающегося и воюющего многие века и с античными греками, и с римлянами, и
с готами. И при этом города высочайшей культуры. Одни описания сурожских храмов
что стоят! Где все это?

Очевидно, что многочисленые свидетельства "Велесовой книги" о славян­ском
Суроже, да и вообще о славянской Тавриде, разительно не со­от­вет­ствуют
общепринятому мнению. И если будет доказано, что и в этом случае "Велесова
книги" права, то мы получим новое историческое до­ка­зательство подлинности
дощечек.

Не следует забывать и то, что ныне (как и в древности) на сии земли претендуют
многие народы. В нашем веке о Крыме как о своей ис­то­ри­ческой родине заявляли
немцы и татары, да и турки не могут забыть о том, что Крым был частью
Оттоманской империи. И только славяне не говорили о Крыме, как о древней
славянской земле, не смотря на то, что мы на это имеем не меньше оснований.

\subsubsection{Первое путешествие в Сурож}


И вот отпуск, еду (тогда это было впервые) в Судак. Машина, петляя между гор,
вкатила в Су­дакскую долину, полого спускающуюся к морю, оставила позади
вино­градники и табачные плантации, затем проехала по главной улице самого
Судака. Мелькают надписи: ресторан "Сурож", футбольная команда "Сурож",
кинотеатр "Сурожский" и т.д.

А вот после небольшого подъёма показалась на вершине Крепостной горы, широком
наклоненном плато, что высоко взлетело над морем, сама знаменитая Генуэзская
крепость.

\ifcmt
  pic https://avatars.mds.yandex.net/get-zen_doc/1888829/pub_5fa04a2c49e00863eb7c8f3b_5fbc22986ea65c24b319b8b8/scale_1200
	caption Сурожская-Шннкэзская крепость
\fi

И, разумеется, в первый же день я пошёл осматривать Крепостную гору и крепость.
Величественные стены Судакской крепости в самом деле впечатляют. Ранее я видел
их только издали, с моря, тогда они показались мне какой-то средневековой
итальянской сказкой (как мне говорили, их построили генуэзцы). Теперь я их
увидел вблизи, теперь их можно было пощупать руками.

С чем сравнить их? По красоте и по стилю, разве что с Мос­ковским Кремлем (к
строительству коего тоже приложили руки ита­льянцы). Да и по мощи стен! Высокие
и толстые стены дважды опоясывают всю гору, красиво вписываясь в природой
созданный идеальный для крепости ланд­шафт. Крепость возведена на скалах,
окруженная с трех сторон про­пастями, а с четвертой крутым склоном, у основания
коего ранее был ров и протекала река. А башни крепости представляют собой
настоящие средневековые замки, частью рес­таврированные в наше время. И
про­тянулись сии стены вокруг плато на несколько километров. Грандиозное
строение!

Но не менее потрясает и то, что за стенами. Холмы, холмы, холмы... Под ними
погребен нераскопанный древний город. Каждый холм когда-то был домом или
дворцом. И почти ничего не касалась лопата археолога... Как мне потом
объяснили, здесь культурный слой составляет 15 (!) метров. То есть под
развалинами одного города погребены развалины другого, а под ним третьего и так
далее. Это настоящая русская Троя, где люди (и как очевидно, предки славян)
жили тысячи лет!

А теперь здесь осталось только одно неразрушенное здание. Это мечеть,
построенная уже турками на месте христианского храма. И в ней теперь музей. И
можно видеть в нем кое-какие экспонаты, раскопанные здесь, надгробия... Но
очевидно, что раскопок здесь по сути не велось, была только разведка местности.
В советское время мало кому был интересен город расцвет коего приходился, как
тогда думали, на генуэзское время. Тогда боль­ше интересовались античностью. На
это выделялись большие средства, работали археологические партии, а на
средневековье средств уже не хва­тало, не было и соответствующих экспедиций.

А зря! Тем более, что судя по всему здесь был не только средневековый, но и
античный город... А по "Книге Велеса", Сурож был к тому же и единственным в
Причерноморье славянским городом, и не просто городом, а столицей княжества!

\subsubsection{В гостях у хранителя крепости}

Современный город Судак у подножия горы живет своей обычной курортной жизнью. И
мало кто, кроме приезжих, интере­суется древностями.

Однако я в тот первый свой приезд нашел одного такого человека. Это был старый
хранитель крепости Алексей Михайлович Горшков. Ему тогда было за семьдесят,
длинная седая борода и ясные глаза придавали ему сходство с отшельником или
волхвом. Старенькая квартира его была заставлена картинами (он ещё и художник),
завалена книгами по истории, материалами из раскопок.

Я тогда нашел самого знающего сии проблемы археолога, ибо Алексей Ми­хайлович
сорок лет занимался изучением древностей Судака, участвовал во всех раскопках,
которые были на его земле, да и написал множество на­учных работ, боль­шая
часть коих осталась в рукописях (он показал 20 написанных им, но
неопубликованных книг).

И это встреча была настоящей удачей. Алексей Михайлович оказался не только
самым авторитетным археологом и знатоком местных древностей, но и
единомышленником.

Он уже сорок лет бился над тем, чтобы доказать именно славянскую историю
древнего Сурожа. По его убеждению Сурож всегда, со времени основания, был
славянским купеческим городом, столицей купеческой республики подобной
Новгородской. И только после того как турки вырезали весь город в конце 15-го
века, и город был полностью разрушен, славяне отсюда ушли. И причем турки
разрушили до основания сей город именно потому, что опасались могучего и уже
подымающегося после веков татарского владычества северного соседа. Им не нужен
был славянский город на их земле. Остальные крымские города не были славянскими
по пре­имуществу.

Алексей Михайлович показал и свою работу, в коей он рассказывал о 20 (!)
исторических названиях древнего Судака (самые известные: аланское Сугдея,
генуэзское: Солдайя, турецкое: Судак, русское: Сурож). Эти названия он носил в
разные периоды своей богатой истории. Здесь и славянские, и греческие, и
аланские, и скифские, и римские, и татарские, и готские, и турецкие имена. И за
каждым именем стоит своя история. Каждый народ, живший на этой земле, давал
городу свои имена. А жили здесь десятки народов. Сурож во все времена был
многонациональным, но славянская община всегда была преобладающей, коренной.
Потому и в русских летописях сей город всегда значился под своим родным именем:
Сурож.

Трудно подыскать другой такой случай, когда славяне славянским име­нем называли
неславянский город. Царь-град? Но это особый случай, этот город, столица
Визанатийской империи, был городом сакральным, и сла­вянское имя он получил от
сказочного Царь-града, бывшего в эпические времена в этих же местах. Нет,
Сурож-град всегда считался славянским, русским городом!

И как же теперь понять то, что обычно историки и археологи не считают Сурож
городом русским? 

\subsubsection{Свидетельствуют дощечки}

Я показал Алексею Михайловичу копии дощечек "Книги Велеса", перевёл древний
текст...

\begin{leftbar}
  \begingroup
    \em\Large\bfseries\color{blue}
"Мы — славные потомки Дажьбога, родившего нас через корову Земун. И потому мы —
кравенцы: скифы, анты, русы, борусины и су­рожцы. Так мы стали дедами русов, и
с пением идем во Сваргу синюю". (Род IV, 4:3).
  \endgroup
\end{leftbar}

Вот! Это древнейшее свидетельство о сурожцах. Сурожцы здесь при­знаются
потомками небесной Коровы Земун и Дажьбога. А согласно славянским мифам,
Дажьбог был внуком Велеса, рожденного Коровой Земун. То есть русы также и
потомки Велеса, а одно из са­кральных имен Велеса: Тавр. И значит это имя
именно "бык". То есть тавры, древние жители Крыма-Тавриды - это и есть сурожцы.
Их же "Велесова книга" именует тиверцами.

Так! Алексей Михайлович обрадовался, ибо он также считал тавров предками
славян. Оказывается, археологически культура тавров (керамика, например)
совершенно идентична более поздней культуре тиверцев-уличей, живших у устья
Днестра в раннем средневековье. Археологи всегда на­ходили сию связь, но не
решались признать тавров и тиверцев одним пле­менем. Осторожность всегда
выглядит более научно. Легче отметить оче­видную идентичность археологической
культуры, чем признать какой-либо древний народ предком славян!

Таврами их называли греки, а сами себя они называли по именам своих
прародителей. Во-первых: русами, ибо были потомками Роси, матери Дажьбога.
Во-вторых тиверцами, ибо были также потомками Тавра-Велеса. И, ко­нечно,
сурожцами, то есть потомками Сурьи-Солнца, который по­читался как отец Велеса.

Напомню: Сурья и Земун родили Велеса, потом Велес а Азова родили Рось, а она от
Перуна родила Дажьбога. От Дажьбога и пошли русы, коих и "Книга Велеса" и
"Слово о полку Игореве" именует Дажьбожьими внуками. Сурожцы же помнили и о
первом отце Солнце-Сурье, в честь коего они и назвали Сурож-град. И другое их
древнее название: уличи напоминает о том, что они разводили в ульях пчёл, и
делали из меда священный сол­нечный напиток: сурицу.

Алексей Михайлович также вспомнил: "Не об этом ли говорили и греки, производя
тавров и скифов от коровы Ио? Само имя древних жителей Тавриды, тавров обычно
толкуется как коровичи, ибо "таврос" по- гречески также значит "бык"".

Прекрасно! Но только что же мы все к грекам обращаемся, — возмутился я явной
несправедливостью. Не менее интересные, да к тому же и гораздо более древние
свидетельства о таврах-уличах мы можем найти в "Махабхарате".

Да-да! В книге царских собраний "Сабха-парве" (кн. 21) "О завоевании северных
стран" можно найти рассказ о том, как после знаменитой битвы на поле Курукшетра
герой Арджуна двинулся на заво­евание северных кимерийских стран
Кимпуруша-варша, лежа­щих за Гима­лаями. И покорил некие племена "улука" рядом
с "син­дами", кои покло­нялись Индре-Парджанье (то есть Перуну).

Очевидно, речь идет именно об уличах-таврах, которые жили рядом с синдами (это
кавказские синды из окрестностей Синдики, древней Анапы). А между прочим, это
было в III тысячелетии до н.э.! Чуть не за две с половиной тысячи лет до того
как в Тавриде появились греки. И "Велесова книга" также помнит о сём походе
Арджуны, коего именует древ­ним сла­вянским и арийским князем Яруной, пришедшим
из Пенжа. Именно он, согласно традиции, распространял в сих землях ведическую
веру, традицию почитания Вышня-Всевышнего.

И что же мы не помним об этой священной истории? Об этом походе, между прочим,
и ныне рассказывают в Индии, ибо там чтут ведическую тра­дицию. А у нас об этом
не знают.

\subsubsection{Античный Сурож}

Да что там говорить о ведических древностях! Алексей Михайлович разводит
руками: до сих пор у нас не умолкают споры о том существовал ли Сурож и во
времена античные.

На чем основано везде растиражированное утверждение, что Сурож был основан
аланами во II веке н.э.? Оказывается есть одна строчка, приписка в одной
христианской рукописи XII века, в Синаксарии\footnote{
Опубл. Антоний-архимандрит. Заметки 12-15 вв., относящиеся к Крымскому городу Сугдее, приписки греческого Синаксария //ЗООИД - 1863 г. Т. 5 - С. 535-628//.
} ("Житии святых"),
сохранившейся в монастыре на острове Халки, в Турции. Там написано буквально
следующее: "Построена крепость Сугдеи в 5720 г.". Нетрудно вычислить, что речь
идет о 212 г. н.э.. Замечу, здесь не сказано об основании города аланами, к
тому же тех, кого в 3-м веке именовали аланами, были русколанами, то есть
кавказскими русами и аланами (скифами), хлынувшими тогда в крымские города.

Но это же всего только свидетельство об очередном строительстве, или
перестройке крепости. И только. А ведь был и античный город на месте Сурожа.

Да! Подтверждаю я. Вот и в "Велесовой книге" сказано о том, как греки в 4-ом
веке до н.э. захватили Сурож. 

\begin{leftbar}
  \begingroup
    \em\Large\bfseries\color{blue}
"Когда наши пращуры сотворили Сурож, начали греки приходить гостями на
наши торжища. И, прибывая, все осматривали, и, видя землю нашу, посылали
к нам множество юношей, и строили дома и грады для мены и торговли. И
вдруг мы увидели воинов их с мечами и в доспехах, и скоро землю нашу они
прибрали к своим рукам, и пошла иная игра. И тут мы увидели, что греки
празднуют, а славяне на них работают. И так земля наша, которая четыре
века была у нас, стала греческой." (Троян II, 2).
  \endgroup
\end{leftbar}

Так Сурож стал греческим. Через четыре века после его основания в 8-ом веке,
согласно тем же дощечкам. То есть это было в 4-ом либо 5 ом веке до н.э. И
кстати, зная это, мы вполне можем определить кто совершил этот завоевательный
поход. Только войска афинского полководца Перикла в это время (а именно в 437
году до н.э.) сделали рейд по причерноморским городам.

Именно! Алексей Михайлович сразу расцвел: это именно то, что он так долго
искал: еще одно подтверждение античной праистории Сурожа. Ока­зывается,
согласно его исследованиям, именно афиняне основали (а теперь мы можем сказать,
не основали, а захватили) древний город на месте Сурожа. Его так и назвали
Афинион, в честь Афин, откуда явились тогда греки. Это античное имя Сурожа
можно увидеть в любом зарубежном (но не отечественном) справочном пособии по
истории Причерноморья. А у нас об этом знают только единицы.

Нам известны античные имена всех городом Крыма и Кавказа, только Сурож в это
список не входит. Почему? Загадка.

\ifcmt
  pic https://avatars.mds.yandex.net/get-zen_doc/1873182/pub_5fa04a2c49e00863eb7c8f3b_5fbc26820b4af801499d9a24/scale_1200
	caption Сурожская крепость
\fi

А между тем, как рассказал Алексей Михайлович, развалины Афиниона ещё в 70-е
годы здесь искал Павел Шульц, и кое-что нашел на пляже у пансионата Сокол.
Находили здесь и античную керамику, даже целые амфоры. Но предубеждение будто
не было сего античного города до сих пор существует.

А ведь это славная страница истории не только Причерноморья, но и всей Руси.
Времена Перикла именуют расцветом или Золотым веком Эллады. Именно тогда в
греческом мире стали популярны сюжеты из скифской и киммерийской истории,
легенды о гипербореях. Именно тогда стала развиваться античная астрология,
появи­лись названия и греческие толкования созвездий, среди коих мы можем найти
образы северных зверей. Да и всю астрологическую науку греки почитали
заимствованными у гипербореев. Не в славянских ли, а именно в сурожских, храмах
они получили сии знания?

Античная история Руси ещё ждёт своих историков, да и археологов. И лучшего
места, чем окрестности Сурожа, для сих исследований не су­ще­ствует. До сих пор
не было раскопок на соседней с Крепостной - горе по имени Болван (то есть Идол,
или Кумир), где как видно и находились славянские храмы и статуи богов.

\subsubsection{Средневековая история}

О средневековой истории Сурожа известно несколько больше, ибо с этого времени
он существует и "официально". Но и здесь "Велесова книга" существенно дополняет
общепринятую картину древней истории сего древнего славянского княжества.

В начале 3-го века, в 212 году н.э., сарматы-аланы (надо полагать с
русколанами) пришли на помощь угнетённым славянам и изгнали греческих
поработителей. Тогда же на месте деревянных и насыпных укреплений Сурожа
возвели каменные стены. Город стал тогда по алански также называться Сугдеей
(от иранского «sugda», «священный»). Потом, на краткое время власть в городе
перешла к готам, сия местность была завоевана в 237 году конунгом Книвом.
Кстати, есть основания полагать, что в то время граница между прагерманскими
(вандальскими) и пра­славянскими (венедскими) родами еще не была проведена,
потому в дощечках «Книги Велеса», в отличие от германских анналов, сей конунг
имеет также славянское имя Кий Готский (вообще среди имен готских королей того
и более позднего времени мы находим немало славянских имён).

Затем Сурож (Суренжаньское княжество) входит в империю Гер­манареха, внука
Книва (врем. правл. 351-375 г. н.э.), то есть становится частью Остроготии.
Начинаются гото-сла­вянские войны, описание коих занимает немалую часть
«Велесовой книги». В конце-концов великий князь Русколани Бус из рода Белояров
отвоевал Тавриду и освободил суренжан от очередных поработителей. Сурож вместе
со всей Тавридой стал вновь славянским краем.

В это время Сурож пережил небывалый расцвет, он стал чуть не самым значительным
городом Причерноморья, стал спорить по своему значению, по торговле, по
количеству жителей и храмов с Константинополем. Никогда со времени Буса Сурож
не достигал таких вершин славы. С тех пор, вначале медленно и незаметно, он
клонился к упадку. А в то время была полностью застроена Крепостная гора,
великолепный храмовый комплекс занимал Болванную гору, также ограждённую
крепостными стенами. 

\ifcmt
  pic https://avatars.mds.yandex.net/get-zen_doc/1874110/pub_5fa04a2c49e00863eb7c8f3b_5fbc29ba4b9b1b331d3cd4fe/scale_1200
\fi

В сих храмах, по свидетельству «Велесовой книги», славили сол­нечных богов
Сурью, Хорса и Ярилу. Также там были святилища пра­родителей солнечной династии
славянских царей: Яра, Ария Оседня, Яруны, Арияны и самого Буса Белояра.

Алексей Михайлович, изучив по "Велесовой книге" сию историю, добавил, что есть
и некоторые археологические подтверждения сих сведений. Он мне показал книгу Е.
И. Лопушинской "Крепость в Судаке", которая вышле в Киеве в 1991 году. В сей
книге осторожно утверждается, что Приморское укрепление, раскопанное Фронджуло
М.А. в 1968 году \footnote{Фронджуло М. А. Раскопки в Судаке. К. Наукова думка,
1976.} принадлежит не 6-му, как он считал, а 3-4 вв. н.э., то есть временам
Буса Белояра. Она даже полагала, что тогда стены были возведены на манер
византийских, и сделала это вывод на основании сходства кладки с кладкой
Золотых Ворот в Константинополе.

Очевидно, что со времен Буса Белояра в Суроже также начало крепнуть
христианство. В начале, надо полагать в виде арианства, принесенного готами, а
потом и в виде византийского православия. Хри­стианскую религию византийцы
стали использовать с целью нового порабощения и эллинизации Причерноморья. И
это быстро привело византийцев к успеху. Уже через двести лет после времени
Буса, то есть в 580-590-е годы н.э. князь Криворог из рода Белояров был
вынужден совершить поход с Воронежской Руси на Сурож, ибо в нем греки захватили
власть, славянские храмы были разорены и боги славян были «повержены в прах».
Это был поход в защиту древней веры.

Белояр Криворог победил греческое (т.е. христианское) войско, но это уже мало
что значило. С этого времеии Сурож, согласно «Книге Велеса», перестал считаться
русской землей, не смотря на то, что в нем продолжли жить в основном славяне,
ибо по вере он стал преимущественно греческим. Белояр Криворог только прекратил
преследования славян древнего вероис­поведания и восстановил служение в
некоторых храмах. Где, судя по всему, оно продолжалось до 7-го века н.э., когда
понадобилось вновь на защиту древних святынь идти вначале Бравлину I (около 660
г. н.э.), а потом и Бравлину II (790 г. н.э.).

Славяне, коренные суренжане, наряду с греками и готами, потом жили в Суроже
(Сугдее, Солдайе) по-видимому вплоть до времени турецкого заво­евания, то есть
до конца 15-го века н.э., постепенно уменьшаясь числом. Они пережили и время
владычества Византии, и Хазарии, и Дешт-и-Кыпчака, и Золотой Орды. В 13-14-м
веках в Суроже появились венецианцы и генуэзцы, которые отреставрировали старую
крепость, построенную еще во времена Буса Белояра и возвели новые башни на
месте старых, разрушенных временем. При этом сохранилась старая планировка
Сурожской крепости. Две линии ее крепостных стен (вначале деревянных, а потом
каменных) упоминаются ещё «Велесовой книгой».

Кстати, сохранились многочисленные документы того времени, и со­гласно им тогда
город приходил в упадок, генуэзцам и венецианцам не хватало средств даже на
реставрацию крепости, да и сам городской гарнизон составлял не более трех
десятков человек (в то время как для обороны крепости требовались тысячи
воинов). И во всех документах того времени говорится об упадке города, об его
запустении из-за конкуренции с соседней Кафой (Феодосией)... А современные
историки приписывают строительство города, мощных укреплений к этому времени!
Нет, расцвет Сурожа принадлежит более древним временам.

И только после того как турки в 1475 году взяли Сурож, крепость была разрушена,
а население бежало. Возможно, в очередной раз на Русь. На Руси выходцами из
Сурожской Руси основано два Сурожа (в Брянской и Витебской областях), а также
Судогда, и может быть даже много ранее Суздаль во Владимирской области. Три
столетия древний Сурож лежал в развалинах при турках, а новый город Судак стал
расти уже после прихода в Крым русских войск в 18 столетии.

И была забыта древняя история сего града да и всей Черноморской Руси-Русколани.
Вспомним ли мы об этом? Будем ли изучать сию историю в школах? Будут ли
археологические экспедиции в этих краях? Русская Троя ждет своих
первооткрывателей. Холмы Сурожа хранят великие тайны русской праистории.

\ii{29_11_2020.sites.ru.zen_yandex.asov_alexandr.1.surozhskaja_rus.comments}
