% vim: keymap=russian-jcukenwin
%%beginhead 
 
%%file 07_12_2020.sites.ru.zen_yandex.asov_alexandr.1.tajna_kazachestvo
%%parent 07_12_2020
 
%%url https://zen.yandex.ru/media/id/5eeaf725a3dca453cfdd4b58/taina-proishojdeniia-kazachestva-5fcba3a3702d845a131bdb1f
 
%%author 
%%author_id asov_alexandr
%%author_url 
 
%%tags kazaki,russia
%%title Тайна происхождения казачества
 
%%endhead 
 
\subsection{Тайна происхождения казачества}
\label{sec:07_12_2020.sites.ru.zen_yandex.asov_alexandr.1.tajna_kazachestvo}
\Purl{https://zen.yandex.ru/media/id/5eeaf725a3dca453cfdd4b58/taina-proishojdeniia-kazachestva-5fcba3a3702d845a131bdb1f}
\ifcmt
	author_begin
   author_id asov_alexandr
	author_end
\fi


\ifcmt
  pic https://avatars.mds.yandex.net/get-zen_doc/1873427/pub_5fcba3a3702d845a131bdb1f_5fcbc9117e300d7cca23a5fa/scale_1200
	caption Казак на бивуаке
\fi

\index[rus]{Казачество!Тайна происхождения, 07.12.2020}

Россия — это могучее древо, возросшее и раскинувшее свою крону на половину
Евразийского континента благодаря стараниям предков наших, среди коих важную
роль играли казаки.

Древо это имеет много крепких корней. Одни корни уходят на Север, в легендарную
Гиперборею, иные за Урал и в Сибирь, в ранние цивилизации скифские и сакские, а
иные на юг — в цивилизации киммерийские и сармато-русколанские.

«Веды Руси» называют Древнюю Русь — Русколанью. Говорится, что создавалась она
«две теме» — двадцать тысяч лет. А до её создания русы жили в легендарном
Семиречье, землях южноуральских и степях, раскинувшихся от Алтая и Китая до
Каспия, а ещё ранее — на Севере. После своего создания Русколань занимала земли
от Волги и Северного Кавказа до Дуная.

Золотые, или Трояновы века Русколании закончились во время Великого переселения
народов, в IV веке н.э. «И тогда пели о походах отцов своих, о том, что когда
Русколань пала ниц из-за сражений с готами и гуннами, тогда Киевская Русь
создалась антами» («Велесова книга» Лют IV, 6:5). То есть, по Ведам, эпоха
Русколани — это эпоха, предшествовавшая образованию Руси Киевской.

Ядро же сей до-Киевской Руси, а также священную столицу Русколании, град Кияр,
Веды Руси помещают в землях Подонья и Северного Кавказа, Пятигорья, где потом
против русов-русколан выступили гунны и иные сменившие их находники: хазары,
половцы, печенеги и так далее, вплоть до крымских татар и турок-османов.

Сам же Кияр Древний, священную столицу Русколани, «Веды Руси» размещают вблизи
горы Великой Алатыря, которую ныне именуют Эльбрусом. 


\ifcmt
  pic https://avatars.mds.yandex.net/get-zen_doc/1572663/pub_5fcba3a3702d845a131bdb1f_5fcbccb18f8c7853ed5a6066/scale_1200
	caption Кияр у Эльбруса. Коллаж А. Асова (по рис. М. Преснякова)
\fi

\begin{leftbar}
  \begingroup
    \em\large\bfseries\color{blue}
		«И тогда все сыновья Ария, князья Кий, Щек и Хорив, поднялись на Белую
				Алатырь-гору, и увидели, что место это исполнено всяких благ, и есть
				тут реки, изобильные рыбой, а по берегам их тянутся поля и леса, где
				много зверей и птиц всяких родов. И построили они здесь град Великий
				Кияр, названный так в память старшего брата, имя коего Кий. А потом
				братья Щек и Хорив потекли каждый в свой край. И от тех трёх братьев
				три племени славян происходят» («Ярилина книга» II, 6:1). 
  \endgroup
\end{leftbar}

Подтверждает эти рассказы и вся совокупность имеющихся у нас исторических
свидетельств.

Память о том, что среди предков русов были роксаланы, сохранилась в нашем
летописании. Так, в Густынской ле­­тописи сказано, что от первых прародителей
«родишася Венеди, Анты, Аляны, Росканы, Раскаляны, аки бы Русь и Аляны... се си
вси единого суть народа и языка си ест словянского».

К сожалению, не все древние летописи дошли до нас, многие были уничтожены во
времена не столь давние, когда в отечественной науке главенствовали норманисты.

Однако мы имеем и более ранние тому свидетельства зарубежных
пи­сателей-историков. Ещё античные и раннесредневековые историки и писатели
помещали предков славян — венедов, роксалан и антов — в Северном Причерноморье
и на Северном Кавказе, и об этом будет сказано подробнее. Эта традиция была
продолжена в Позднее Средневековье, а затем и в Новое Время, когда складывалась
европейская историческая наука.

Так, итальянский историк и философ Филиппо Буанокореи Каллимах (1437—1496)
определил границы расселения славян в области, «именуемой Сарматией»: от
Чёрного моря по рекам Днестр и Днепр до Дона и Рифейских (Уральских) гор. В
жизнеописании кардинала Збигнева он написал, что в Сарматии славяне проживают с
давних времён, «именуясь венетами, роксаланами, склавонами».

В те годы не только сарматское и роксаланское происхождение русов не вызывало
сомнений. Очевидным казалось и то, что пришли русы с Кавказа. А подтверждением
тому, что прародина славян находилась в древности на Северном Кавказе, являлось
для историков той эпохи проживание с древности и вплоть до их времени на
Северном Кавказе наследников роксалан — казаков, или черкасов (черкесов), как
тогда именовали и южнорусские, и малорусские роды.

Впоследствии, уже в XIX веке, имя черкесов перешло и на часть союзных русским
адыгов — современных черкесов, которых уже с древнейших времён связывали с
русскими черкасами отношения не только союзнические, но и соседские, и
родственные, ибо было много смешанных браков.

И даже более того, историки и землеописатели указывали тогда не только на
проживание на Северном Кавказе и в Подонье донских казаков-черкасов, но также
на независимое от Московии княжество Пятигорское — наследника древней
Тмуторокани и Русколани. Оказывается, священная область Пятигорья, ядро Древней
Руси-Русколани, с древнейших времён и вплоть до XVI—XVII веков, до оттеснения
казаков из этих мест турками-османами, была населена потомками русколан —
казаками- черкасами и пятигорцами.

* * *

Современные отечественные историки не знают о длительной предыстории казачества
на Северном Кавказе, ничего не рассказывают и о Пятигорском княжестве. Однако
достаточно заглянуть в книги историков XVI и XVII веков, а также на
географические карты того времени, как всё становится на свои места.

Так, швейцарский историк Конрад Геснер (1516—1558) в книге «Митридат» (1555)
перечислил 60 народов славянского языка, среди коих он назвал и «Черкаси
Пятигорцы около Чёрного моря» (Circasi Quinquemontani circa Pontum alias
Ciercassi).

О том же писал и польский историк и культурный деятель Мацей Меховский
(1457—1523). В «Трактате о двух Сарматиях» (1517) он различил Европейскую и
Азиатскую Сарматии. В Европейской Сарматии проживали, по Меховскому, вандалы и
славяне (западные, южные и восточные). Причём среди восточных славян он выделил
группы русов-черкасов (малороссов), их земли в то время входили в состав Речи
Посполитой, а также московитов. И тех, и других он почитал потомками роксалан.
За Доном же, в Азиатской Сарматии, Меховский знал и иные роды черкесов, коих
тогда именовали также «пятигорскими черкасами». «Дальше, к югу,— писал он,—
есть ещё какие-то остатки черкесов. Это весьма дикий и воинственный народ, по
происхождению и языку — русские».

От тех же сармат и даже, более точно, северокавказских алан,
предводительствуемых царём Аланом II (он же, полагаю, Арий Оседень), производил
все славянские народы великий польский историк Мацей Стрыйковский (1547— 1582). 

\begin{leftbar}
  \begingroup
    \em\large\bfseries\color{blue}
«Наши старшие хронисты, — замечал он, — начало славянского народа производят
прежде всего от Иафета, сына Ноя... А иные строят длинную генеалогию,
восходящую к Алану II, который первый пришёл из Азии в Европу. Он имел четырёх
сыновей, из которых старшим был Вандал...»
  \endgroup
\end{leftbar}

Более подробно и обстоятельно о русах и славянах Северного Кавказа рассказал
также австрийский посол в Москве Сигизмунд Герберштейн. В 1527 году он выпустил
книгу «Записки о московитских делах», получившую большую известность на Западе.
В этой книге он сделал заключение о близости языков, которыми пользуются
«поляки и русские, властвующие на весьма широком пространстве, и
черкесы-пятигорцы у Понта (Чёрного моря)...».

О черкесах же он сообщил также: «Русские утверждают, что это христиане, что они
живут независимо по своим законам, а церковную службу выполняют по греческому
обряду на славянском языке, которым они и пользуются главным образом в жизни».

И это 1527 год! До присоединения Пятигорья к России ещё двести пятьдесят лет!
Между Пятигорским княжеством и Московской Русью на тысячу вёрст раскинулась
Золотая Орда, вассальная к Турции, в Диком Поле за Доном кочуют крымчаки и
ногайцы.

\ifcmt
  pic https://avatars.mds.yandex.net/get-zen_doc/3684252/pub_5fcba3a3702d845a131bdb1f_5fcbc37c702d845a134cbc89/scale_1200
	caption Сигизмунд Гербершейн на фоне своей карты. Над г. Бештау крест и надпись "черкасы пятигорские". Рядом "черкасы попули", то есть "народ черкасы".
\fi


Своё сочинение Сигизмунд Герберштейн снабдил картой, где он вполне определённо
очертил границы расселения донских черкас, а также древнего славянского
Пятигорского княжества на Северном Кавказе. Он даже указал крест, стоящий на
самой высокой вершине Пятигорья, то есть на горе Бештау.

Известны и иные географические карты той эпохи. Так, пятигорских черкасов на
Северном Кавказе мы обнаруживаем на карте Дженкинсона, крупного деятеля
английской «Московской компании», который четырежды пересёк Россию от Белого
моря до Астрахани в 1557—1562 годах. Составлена эта карта была для практических
торговых нужд, и не по преданиям, поэтому она совершенно достоверна.


\ifcmt
  pic https://avatars.mds.yandex.net/get-zen_doc/3957666/pub_5fcba3a3702d845a131bdb1f_5fcbd0f5c26ad131b682d36b/scale_1200
	caption Карта Гесселя Герритса.
\fi

В те же годы на карте голландского картографа Гесселя Герритса, составленной
ещё в 1613 году на основе русской карты 1523 года, мы находим народ пятигоров
(Petigori) в районе Пятигорья, а чуть восточнее обозначены черкасы (Circassi),
которые на сей карте помещены в землях алан (асов). И заметим, кабардинцы
(адыги) на сей карте обозначены западнее пятигоров (явных славян-русских) и
черкас.

В те же годы хорватский писатель Мавро Орбини, бывший архимандритом Рагузским,
назвал пятигорцев и черкесов большим и могущественным народом и поставил их в
ряд с иными великими славянскими народами. В книге «О славе славян», изданной
на итальянском языке в Пезаро в 1601 году (Orbini M. «De gli Slavli et
progressaro dell loro»), Орбини написал: \emph{«К нации и языку славянскому относятся
не только те, кто проживает в Далмации, Иллирике, Истрии и на Карпатах, но
также и многие другие большие и могущественные племена: Болгары, Рации или
Рассияне, Сербы... Пятигорцы, которые жилища имеют на пяти горах, Руссы,
Подолии, Поляне, Московиты и Черкасы...»}

Спустя полвека турецкий путешественник и землеописатель Эвлия Челеби (по матери
абхазец) рассказал об осаде турками Азова, где сидели донские и запорожские
казаки. Поведал он и о легендах, согласно которым древние города Крыма были
основаны славянами. И он же в 1666 году достиг на Северном Кавказе страны,
которую он назвал «Черкесией», но там уже русских казаков-черкесов (черкасов)
не застал, ибо они уже были оттеснены в степи турками и татарами. Оставшиеся же
подвергались ассимиляции.

Московия уже тогда считала эти земли своими, ибо в 1553 году кабардинцы и
казаки «били челом» русскому царю Ивану IV о принятии их в подданство, но
османы и татары не дали осуществиться этому. И потому после более чем столетия
битв с турками и вассальными им крымскими татарами пятигорцы и черкасы были
вытеснены из этого края к Дону, за Терек и на Волгу.

Война за свои исконные земли между турками, крымскими татарами и русскими
казаками с тех пор не прекращалась. Русские и турецкие войска не раз проходили
по этой земле. И потомки пятигорцев, волжские казаки под предводительством
атамана Савельева, в союзе с калмыками, отвоевали свои древние земли в
следующем столетии, а именно — в 1769 году, во время очередной русско-турецкой
кампании. Затем , с 1793 года по указу императрицы Екатерины эти земли
заселялись также запорожскими казаками.

***

Для чего здесь столько внимания уделено именно этому краю? Да только по той
причине, что в современных учебниках отечественной истории эти земли почитаются
русскими (казачьими) только с конца XVIII века, то есть со времени нового их
заселения, так же как не признаются предками русских и казаков древние
насельники сих мест: сарматы и русколаны.

А ведь именно здесь ведические манускрипты помещают столицу обширнейшего и
могучего древнего государства — Русколани, предшественника Руси Киевской. И тут
следует заметить, что даже в первых учебниках по русской истории, ещё не
переписанных норманистами, русколаны не раз назывались предками русов.

Действительно, если открыть Синопсис 1735 года, первый напечатанный гражданским
шрифтом учебник русской истории, то мы прочтём: «Под тем Сараматским именем все
прародители наши Славенороссийские, Москва, Россы, Поляки, Литва, Поморяне,
Волынцы и прочая заключаются... От тех же Сарматских и Славянороссийских осад
(поселений) тот же народ Росский изыде, от него же неции нарицахуся Россы, а
иные Аланы, а потом прозвашася Роксаланы, аки бы Росси и Аланы...»

А вот утверждение одного из первых российских историков Василия Никитича
Татищева (1686—1750). Он в своей «Истории Российской с самых древнейших времён»
написал следующее: «Оттуда кимбри вышли, где ныне черкесы и пятигоры обитают, и
храбрые суть люди. Язык с нами имеют один». И заметим, что Татищев это написал
до того, как Кавказ вошёл в состав Российской империи. И уже тогда было
известно, что в Пятигорье жили народы — черкасы и пятигорцы, говорившие на
русском языке.

Великий русский просветитель и основатель антинорманизма и славянофильского
направления в отечественной исторической науке Михаил Васильевич Ломоносов
(1711—1765) в своём капитальном труде «Древняя Российская история от начала
российского народа до кончины великого князя Ярослава Первого...» (1766) так
обобщил сведения предшествующих историков о происхождении русов и славян: 

\begin{leftbar}
  \begingroup
    \em\large\bfseries\color{blue}
«Не тщетно западные нынешних веков писатели российский народ за Роксалан
признают... Аланов и Роксаланов единоплемёнство из многих мест древних
историков явствует, и разность в том состоит, что Алане — общее имя
целого народа, а Роксалане — речение, сложенное от места их обитания,
которое не без основания производят от реки Раа, так у древних
писателей слывёт Волга. Плиний Аланов и Роксаланов вместе полагает.
Роксалане у Птолемея переносным сложением называются Аланорси. Имена
Аорси, и Роксане или Россане у Страбона, точное единство Россов и
Аланов утверждает: к чему достоверность умножается, что они обои
славенского поколения были; за тем, что Сарматов единоплемёнными от
древних писателей засвидетельствованы...»
  \endgroup
\end{leftbar}

В XVIII веке против этой устоявшейся традиции выступили только основатели
псевдонаучного норманизма — Байер, Миллер и Шлёцер. И далеко не сразу их мнение
стало официальным для российской науки. Даже Николай Михайлович Карамзин
(1766—1826), представивший официальную версию отечественной истории в своей
книге «История государства российского», издание которой было закончено в 1829
году, только приводит мнение «норманистов» о происхождении казаков от беглецов
из России, но он его не разделяет и полагает, что они только «считались
таковыми».

Он же даёт многие летописные свидетельства о предках казаков: чёрных клобуках,
берендеях и черкасах. Более того, в V главе тома VIII, он утверждает, что даже
при Иване Грозном на Северном Кавказе существовали христианские черкасские
княжества: «Кроме ногаев... и донских казаков, царь (Иван IV) имел на юге
усердных слуг в князьях черкасских: они требовали от нас полководца, чтобы
воевать Тавриду, и церковных пастырей, чтобы просветить всю их землю учением
евангельским. То и другое желание было немедленно исполнено. Государь послал к
ним бодрого Вишневедского и многих священников, которые в дебрях и на скатах
гор Кавказских, основав церкви, обновили там древнее христианство».

Потом эта традиция была продолжена в трудах замечательного учёного, яркого
представителя антинорманизма и славянофильства, а также автора многих
гимназических учебников истории, выходивших в конце XIX и в начале XX веков
Дмитрия Ивановича Иловайского (1832—1920). Он посвятил немало капитальных
трудов истории Азово-Черноморской Руси, а о роксаланах в своей книге «Начало
Руси» (1882) он писал: \emph{«Не может быть никакого сомнения в том, что Рось, или
Русь, и Роксаланы это одно и то же название, один и тот же народ. Роксаланы
иначе выговаривалось Россаланы. Это название сложное... Оно означает Алан,
живших по реке Рокс или Рос».}

Наиболее же подробно, с привлечением редчайших источников, а также казацких
преданий, историю казаков, в том числе северокавказских и черноморских, изложил
Е.П. Савельев в своей книге «Древняя история казачества», вышедшей в 1915 году
в Новочеркасске (она была переиздана в наше время в Москве). И он также
утверждал происхождение казаков от роксалан, сармат и иных древних народов.
Также он показал родство казаков и древних новгородцев, а также вятских
ушкуйников и рязанцев. 

И потому, полагаю, необходимо восстановить историческую справедливость, а кроме
того, вспомнить достижения старой школы, которая отстаивала античную историю
Руси и видела прародителей русов и казков в роксаланах, а предшественницей
Древней Руси — Великую Сарматию или Роксаланию, которую ныне благодаря
«Велесовой книге» мы называем Русколанией.

\ii{07_12_2020.sites.ru.zen_yandex.asov_alexandr.1.tajna_kazachestvo.comments}
