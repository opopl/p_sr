% vim: keymap=russian-jcukenwin
%%beginhead 
 
%%file 09_01_2020.stz.news.ua.mrpl_city.1.dyva_trapljajutsja
%%parent 09_01_2020
 
%%url https://mrpl.city/blogs/view/diva-traplyayutsya
 
%%author_id demidko_olga.mariupol,news.ua.mrpl_city
%%date 
 
%%tags 
%%title Дива трапляються...
 
%%endhead 
 
\subsection{Дива трапляються...}
\label{sec:09_01_2020.stz.news.ua.mrpl_city.1.dyva_trapljajutsja}
 
\Purl{https://mrpl.city/blogs/view/diva-traplyayutsya}
\ifcmt
 author_begin
   author_id demidko_olga.mariupol,news.ua.mrpl_city
 author_end
\fi

Дорогі читачі, якщо вірити в дива, вони обов'язково траплятимуться частіше,
особливо на новорічні свята. Проте героїня мого нарису, маріупольчанка \textbf{Людмила
Калініченко}, ніколи не вірила в дива.

Ця зворушлива історія, яка сталася з нею 5 років тому, може стати повчальною
для кожного з нас. Доля часто випробовувала Люду, але жінка звикла долати всі
труднощі самотужки. У 2015 році вона втратила одразу все найдорожче. Через
недбалість сусідів у неї згорів будинок, того ж року померли батьки. Лише
неймовірно сильний і вольовий характер Людмили допоміг їй витримати всі життєві
негаразди і жити далі.

Напередодні Нового Року 37-річна маріупольчанка розуміла, що має тільки роботу,
яка раніше розкривала потенціал Люди, а тепер стала лише рутинним обов'язком.
Саме тому вона вирішила після роботи більше знаходитися серед людей, які
чекають новорічного дива і можуть заразити гарним настроєм. Найбільше їй
подобалася ялинка у центрі міста. Там частіше можна було почути дитячий сміх і
побачити щасливих людей.

До Нового року залишилося два дні, а святкового настрою ще не було, тому вона
відправилася знову до ялинки. Але чомусь того дня людей було небагато, мабуть,
всі займалися підготовкою до найважливішого свята. Людмила не помітила, як до
неї підійшов переодягнутий Дід Мороз і спитав, що саме вона хотіла б отримати
під ялинку. Жінка посміхнулася і люб'язно відповіла, що її бажання
нездійсненне. Цієї ж миті Люда подумала, що ніколи не вірила в Діда Мороза... 

Однак цей переглянутий чоловік не відходив і наполягав озвучити побажання.
Людмила, яка звикла нікому не говорити про свої справжні надії і сподівання,
вирішила все ж таки звернутися до нього: \emph{\enquote{Дорогий Діду Морозе, не міг би ти
мені надіслати трошки щастя в особистому житті, будь ласка, я ж у тебе сто
років нічого не просила}}.

Так щиро жінка давно не говорила. Переодягнутий Дід Мороз подивився Людмилі
прямо в очі і пообіцяв все виконати. Потім він підійшов до хлопчика шести
років, що стояв поруч з ялинкою. Люда почула, як хлопчик попросив, щоб мама
зв'язала светра, теплого і м'якого. Вона дуже здивувалася, адже діти завжди
просять іграшки. Їй було цікаво, чому хлопчик попросив саме це, але підходити
до нього вона не наважувалася. Про себе вирішила, що дитина дуже мудра,
попросила конкретний светр, але, чому саме від мами. Все одне дивне прохання,
подумала жінка. Вона вже забула і про свято, і про власне бажання...

Вся її увага була прикута до цього маленького хлопчика, який стояв такий
серйозний і чомусь сумний. Раптом до нього підійшов чоловік, взяв за руку і
повів у бік зупинки. Люда сама того не очікувала, але одразу ж пішла за ними.
Вона чула, як чоловік переконував хлопчика, що у них будуть найкращі Новий рік
і Різдво, а відсутність мами зовсім не зіпсує новорічний настрій. Але хлопчик
мовчав. На щастя, вони сіли в тролейбус, що підходив і нашій героїні. Вона
встала на їхній зупинці і подумала, що обов'язково з ними познайомитися.
Подібного з Людою ще ніколи не траплялося.

Але її так захопили бажання хлопця, що його життя стало для неї цікавішим, ніж
домашні справи. Пізніше жінка дізналася, що мама Андрія, так звали хлопця,
покинула сім'ю і поїхала закордон влаштовувати нове життя. Люді вдалося
познайомитися з Андрієм та його батьком Вадимом випадково. Вони нещодавно
переїхали на нову адресу і вдвох прийшли до її перукарні 31 грудня. Хлопчик
одразу помітив в куточку стола жінки недов'язаний светр. Людмила розуміла, що
Андрій хотів отримати светр саме від мами, тому вирішила відправити його нібито
від неї, чи просто передати. Але вийшло все трохи інакше. І якщо познайомилися
вони тільки на Новий Рік, Святвечір, Різдво і Старий новий рік святкували вже
втрьох.

Зустріч з Андрійком біля ялинки для Люда вважала невипадковою. Вона чомусь
вирішила, що вже не встигне створити сім'ю, але після подій 2015 року Людмила
не тільки зрозуміла, що сильно помилялася, але й почала всім серцем вірити в
дива...
