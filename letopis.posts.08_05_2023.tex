% vim: keymap=russian-jcukenwin
%%beginhead 
 
%%file posts.08_05_2023
%%parent posts
 
%%url 
 
%%author_id 
%%date 
 
%%tags 
%%title 
 
%%endhead 

%https://www.facebook.com/ivan.ivan.kyiv/posts/pfbid02Vx9F11K6Ls4vyTsfjvQxYRvQ1RFLcACm3kYzZLtke2ZEumw6H3bnU3GGpQJgv7BPl

\ifcmt
  ig https://scontent-fra5-2.xx.fbcdn.net/v/t39.30808-6/346057036_629174292038210_4409116727650085222_n.jpg?_nc_cat=107&ccb=1-7&_nc_sid=730e14&_nc_ohc=hzLA9VtLPRcAX9h8LUq&_nc_ht=scontent-fra5-2.xx&oh=00_AfB62avtIR_JDrsgRp8dBsfoNrNkaXqUYOaGyMNs4QKE4w&oe=645ECC47
  @width 0.4
  @minipage 0.4
  @wrap \parpic[r]
\fi

Оце от виставив зараз два альбоми...
Один - Вітер Віє, другий - Вогонь Горить...
Я короче зараз в Селі. Тут нічого немає, тобто, немає:
- Київського Метро
- власне Києва
- немає моєї Київської квартири разом із усіма моїми улюбленими книжками, поличками, купою усіляких шаф,
де захована також купа усього, зокрема велосипедні насоси та велосипедні покришки і камери
- немає ні Музею Чорнобиля, де я нещодавно був
- немає ні Музею Історії Києва, де я теж був, і знайшов походу
Кроликів на останньому поверсі, а зараз ці Кролики втекли на природу в Добропарк і десь там зараз всі тусуються і щось там між собою перетирають... Всім раджу зїздити в Добропарк... тому що (1) там дуже-дуже гарно (2) там Тюльпани, і (3) там Кролики, і частина з них була свого часу на Театральній Площі в Маріуполі в 2019, а ще до того... на Софійській Площі на виставці Кролів та Писанок в 2018 році... а ще до того... хтось десь ночами не спав, і розмальовував отих нещасних кролів...
- немає ні Музею Історії України... де я також був у березні... і там на четвертому поверсі діє постійна виставка про Маріуполь та про Азовсталь, та про Героїв Азову, які полягли за Маріуполь... на третьому чи на другому поверсі тоді там  була виставка про Григорія Сковороду... встиг побувати там в останню мить перез закриттям виставки... на першому поверсі є виставка про Київ у березні 2022 року, і як орки намагались його захопити... а де ті орки зараз?...
правильно... орки в оркостані... але скоро оркостан впаде... і не буде вже ніяких орків та мордорів, а буде перемога України, повна і беззастережна перемога України, я в цьому абсолютно впевнений... ну і там також є експозиція мозаїк Десятинної Церкви, дуже-дуже цікаво!
- немає ні Байкового Кладовища, де я теж нещодавно був... 
на прощанні із Володимиром Мельниченко, автором Стіни Пам'яті... це такий величезний барельєф, який за часів комуністичної влади був замурований суцільно в бетон... там в цій композиції 33 барельєфи здається... і дотепер лише один барельєф звільнений з полону... дуже сумна історія насправді щодо тої Стіни Пам'яті...
- крім того, тут (в селі) немає війни... От я знаю, що у Києві вночі вила сирена напевне, і купа всього налетіло... От друг мені щойно подзвонив... і почав емоційно так говорити в трубку - от ти знаєш, що тут було, що тут було!!!!!!!!!!!!!!!!!... ааааааааааа!!!!... Ну і... Крім того, я подивився новини, і звичайно війна так само йде по всій Україні, і це все жахливо, що було і що сталось зараз на Одещині, і також наприклад в Києві... Але... тут в селі всього цього немає... тут в селі прямо зараз мир... от зараз напевне хтось страшно образиться і побіжить... ну я не знаю... от зараз у мене 65 друзів... а може скоро буде 60 від вечір... бо хтось подумає, ото вже якийсь божевільний просто, нєєєє, він мені не потрібен в моїй стрічці точно... або ж взагалі всі зараз розбіжаться... і я залишусь один-сам-по-собі хм... ну що ж поробиш... я не п'ятак, щоби усім подобатись одночасно... як казав свого часу Маяковський... есть на свете много всяких вкусов и вкусиков, кому нравлюсь я, а кому Кусиков...
- тут також немає ні будинків великих, ні тролейбусів, ні...
короче... тут дійсно нічого немає. Як кажуть американці або англійці, In the Middle of Nowhere, тобто... Посередині Нічого.. От я колись подорожував по Америці... там є купа всього дуже цікавого, але найцікавіше там природа... Тому що у американців дуже молода держава, і історія їхня налічує всього 400 років як держави... чи щось типу того... а до того часу там були індіанці і все... але природа їхня просто фантастична... національні парки Єлоустоун або Йосеміті... або ж Флоріда... або ж... національний парк Долина Смерті... і ось якраз у цьому останньому парку я був свого часу... взяв машину напрокат і все обїздив, всю Каліфорнію... від гамірного Лос-Анжелеса до ось таких от національних парків із марсіанським пейзажем... І там у тому парку... Долині Смерті тобто... якраз був один такий кемпінг, де можна було зупинитись, заплатити дєнюжку, поставити машину і палатку... І це місце так і називалось... In the Middle of Nowhere.... 
- тобто... тут дійсно нічого немає. НУ НІЧОГІСІНЬКО ТУТА НЕМАЄ. НУ КАПЄЦ ПРОСТО.
А ЩО Ж ТУТ Є. А ЩО Ж ТУТА ДІЙСНО Є ЯКЩО НІЧОГО НЕМАЄ... ні нічних клубів з дівчатами-у-яких-ногі-ростуть-від-самих-ушей, якщо немає ні Київського метро... ні Київських Каштанів або ж Дніпра? ... І що ж це таке за місце, де нічого немає... але водночас все є... і що ж власне тута є?
А ось що... Пташечки... поля, гаї... трохи треба проїхати... і буде озерце... водичка... можна рибку половити... також... тут є ліси...
ну короче тут все є, що треба. І тому... до речі... я назвав один із своїх альбомів Дорога Туди Де Нічого Немає і Все Є.
От... Дехто напевне спитає себе подумки... а що це все за маячня, і навіщо все це пишеться, якщо в країні йде страшна війна? Чи оця людина точно в своєму розумі, чи точно у цієї людини всі вдома? А от дійсно... чи всі в мене вдома?... 
Ну добре, щиро зізнаюсь одразу, в мене не всі вдома... Далеко не всі вдома... і це я думаю точно видно по моїй сторінці... Але...
Спитайте себе самі, ну тобто, ті, хто будуть мати такі питання до мого посту... або взагалі до мене як особистості... а чи у Вас самих або у Вас або у того або у сього всі вдома... І виявиться, що зараз... ми всі в одній лодці, практично... бо звісно є дуже багато людей... у яких... всі дома, це точно... Ну тобто... Як жили собі до війни... так і живуть... начебто і війни ніякої нема... І все у них в порядку... і все вони знають, що і як і т.д... Але... є також дуже багато людей... які надзвичайно постраждали від цієї жахливої війни... безліч нещасть, горя... втратили все... втратили друзів, батьків, сестер, дітей власних... Я от знаєте...
свого часу записав пост Наталії Дєдової... 130 сторінок вийшло... (отой пост є в моїй папочці https://mega.nz/folder/NjJ0mLab#SR5BVg9x-Oxsd71im5gSXw) така собі Книжка... Суцільний Жах і Біль на кожній сторінці... бо на кожний сторінці тобі хтось посміхається з маріупольців... то якась дівчина, то хлопчик, то якийсь дядько... то хтось разом.. і тут ще припис... оцього розірвало... оцей підірвався на міні ... оцей.. отого... просто якийсь нескінченний Крик і Біль... але... знаєте... я свої сльози вже виплакав... там в тому пості від 10 січня цього року, якщо подивитесь уважно, Наталія Дєдова так само пише - я свої сльози вже виплакала...
І я от думаю...
Якщо почати питати по людям... і питати кожного... привіт, як ти? а у тебе всі вдома? а він або вона відповість - ех, війна забрала того, забраго сього... ті вже ніколи не повернуться додому... Ех... а я вчора цілий день плакав... бо загинув щойно мій найкращий друг... а у мене війна забрала мого чоловіка... і я навіть не знаю... де його тіло... І от... виявиться... що... у мене точно не всі вдома... ну, просто собі якийсь шалений київський програміст... майже як отой Шалений Заєць із Аліси... з програмістами або шахістами або математиками таке часто буває... що вони перенапружуються. не знають міри в каві, сигаретах... не сплять ночами, поки роблять свої наукові відкриття... ну і... врешті решт просто зходять з розуму... але... от знаєте... я на це скажу... в принципі майже у всіх інших також не всі вдома... на жаль...
#УнасУвсіхНеВсіВдома ось такий от хештег тута вставлю...
Так... Ніколи не думав... що Аліса в Країні Чудес виявиться настільки актуальною, тому що... там такий самий діалог...
............................
Так, цього разу помилитися було годі: перед нею було найсправжнісіньке порося! Носитися з ним далі було безглуздо. Вона опустила порося на землю і вельми втішилась, коли воно спокійнісінько потрюхикало собі в хащу.
— З нього виросла б страх яка бридка дитина, — подумала Аліса. — А як порося, по-моєму, воно досить симпатичне. [61]
І вона стала пригадувати деяких знайомих їй дітей, з яких могли б вийти незгірші поросята ("Аби лиш знаття, як міняти їхню подобу", — подумала вона), і раптом здригнулася з несподіванки: за кілька кроків від неї на гілляці сидів Чеширський Кіт.
Кіт натомість лише усміхнувся.
"На вигляд наче добродушний, — подумала Аліса, — але які в нього пазурі та зуби!.. Варто з ним триматися шанобливо".
— Мурчику-Чеширчику, — непевно почала вона, не знаючи, чи сподобається йому таке звертання.
Кіт, однак, знову всміхнувся, тільки трохи ширше.
"Ніби сподобалося," — подумала Аліса, й повела далі:
— Чи не були б ви такі ласкаві сказати, як мені звідси вибратися?
— Залежить, куди йти, — відказав Кіт.
— Власне, мені однаково, куди йти... — почала Аліса.
— Тоді й однаково, яким шляхом, — зауважив Кіт.
-... аби лиш кудись дійти, — докінчила Аліса.
— О, кудись та дійдеш, — сказав Кіт. — Треба тільки достатньо пройти.
Цьому годі було заперечити, тож Аліса спитала дещо інше:
— А які тут люди проживають?
— Он там, — Кіт махнув правою лапою, — живе Капелюшник, — а отам (він махнув лівою) — [62] Шалений Заєць. — Завітай до кого хочеш: у них обох не всі вдома*.
— Але я з такими не товаришую, — зауважила Аліса.
— Іншої ради нема, — сказав Кіт. — У нас у всіх не всі вдома. У мене не всі вдома. У тебе не всі вдома.
— Хто сказав, що у мене не всі вдома? — запитала Аліса.
— Якщо у тебе всі вдома, — сказав Кіт, — то чого ти тут?
Аліса вважала, що це ще ніякий не доказ, проте не стала сперечатися, а лише спитала:
— А хто сказав, що у вас не всі вдома?
— Почнімо з собаки, — мовив Кіт. — Як гадаєш, у собаки всі вдома?
— Мабуть, так, — сказала Аліса.
— А тепер — дивися, — вів далі Кіт. — Собака спересердя гарчить, а з утіхи меле хвостом. А я мелю хвостом — спересердя, а від радощів гарчу. Отже, у мене не всі вдома.
— Я називаю це мурчанням, а не гарчанням, — зауважила Аліса.
— Називай як хочеш, — відказав Кіт. — Чи граєш ти сьогодні в крокет у Королеви?
— Я б залюбки, — відповіла Аліса, — тільки мене ще не запрошено.
— Побачимося там, — кинув Кіт і щез із очей.
Аліса не дуже й здивувалася: вона вже почала звикати до всяких чудес. Вона стояла й дивилася
*В Англії широко популярні приказки "mad as a hatter", "mad as a March hare ("божевільний, як капелюшник", "казиться, як березневий заєць ). Саме ці приказки й "породили" Керролових персонажів. Капелюшники й справді часто божеволіли: отруєння ртуттю, яку вони використовували, обробляючи фетр, часто спричинювало галюцинації.[63]
на ту гілку, де щойно сидів Кіт, коли це раптом він вигулькнув знову.
— Між іншим, що сталося з немовлям? — поцікавився Кіт. — Ледь не забув спитати.
— Воно перекинулося в порося, — не змигнувши оком, відповіла Аліса.
— Я так і думав, — сказав Кіт і знову щез.
Аліса трохи почекала — ану ж він з'явиться ще раз, — а тоді подалася в той бік, де, як було їй сказано, мешкав Шалений Заєць.
— Капелюшників я вже бачила, — мовила вона подумки, — а от Шалений Заєць — це значно цікавіше. Можливо, тепер, у травні, він буде не такий шалено лютий, як, скажімо, у лютому...
Тут вона звела очі й знову побачила Кота.
— Як ти сказала? — спитав Кіт. — У порося чи в карася?
— Я сказала "в порося", — відповіла Аліса. — І чи могли б ви надалі з'являтися й зникати не так швидко: від цього йде обертом голова.
— Гаразд, — сказав Кіт, і став зникати шматками: спочатку пропав кінчик його хвоста, а насамкінець — усміх, що ще якийсь час висів у повітрі.
— Гай-гай! — подумала Аліса. — Котів без усмішки я, звичайно, зустрічала, але усмішку без кота!.. Це найбільша дивовижа в моєму житті!
До Шаленого Зайця довго йти не довелося: садиба, яку вона невдовзі побачила, належала, безперечно, йому, бо два димарі на даху виглядали, як заячі вуха, а дах був накритий хутром. Сама садиба виявилася такою великою, що Аліса не [64] квапилася підходити ближче, поки не відкусила чималий кусник гриба в лівій руці й довела свій зріст до двох футів. Але й тепер вона рушила з осторогою.
"А що, як він і досі буйний?" — думала вона. — Краще б я загостила до Капелюшника!"
