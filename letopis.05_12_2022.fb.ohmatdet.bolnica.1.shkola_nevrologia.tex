% vim: keymap=russian-jcukenwin
%%beginhead 
 
%%file 05_12_2022.fb.ohmatdet.bolnica.1.shkola_nevrologia
%%parent 05_12_2022
 
%%url https://www.facebook.com/ndslohmatdyt/posts/pfbid016Jxg5JqDhnfddjkHPtv7AZnShBuSifCeabAWfGi8NxQ9itb1hXo91aW37ZMyFy5l
 
%%author_id ohmatdet.bolnica
%%date 
 
%%tags 
%%title В Охматдиті пройшла І Мультидисциплінарна школа неврологів
 
%%endhead 
 
\subsection{В Охматдиті пройшла І Мультидисциплінарна школа неврологів}
\label{sec:05_12_2022.fb.ohmatdet.bolnica.1.shkola_nevrologia}
 
\Purl{https://www.facebook.com/ndslohmatdyt/posts/pfbid016Jxg5JqDhnfddjkHPtv7AZnShBuSifCeabAWfGi8NxQ9itb1hXo91aW37ZMyFy5l}
\ifcmt
 author_begin
   author_id ohmatdet.bolnica
 author_end
\fi

⚡️В Охматдиті пройшла І Мультидисциплінарна школа неврологів⚡️

🔻1 грудня відбулася І Мультидисциплінарна школа неврологів НДСЛ «Охматдит» МОЗ
України «Розсіяний склероз та міастенія у дітей». Участь у конференції взяли
понад тисяча спеціалістів з різних регіонів України. 

Під час конференції були розглянуті такі теми: 

✔️Клініко-діагностичні критерії та лікування розсіяного склерозу у дітей згідно нових стандартів.\par
✔️ Нейровізуалізація розсіяного склерозу у дітей.\par
✔️ Клінічна та лабораторно-інструментальна діагностика, лікування міастенії у дітей.\par
✔️ Нейровізуалізація тимом при міастенії у дітей.\par
✔️ Тимектомія як метод хірургічного лікування при міастенії у дітей.\par
✔️ Проведення плазмофорезу при розсіяному склерозі та міастенії у дітей.\par

🔻Під час конференції виступили провідні спеціалісти України. Участь взяли
також іноземні спікери: Dr Ming Lim, який очолює відділення дитячої нейронауки
в Лондонській дитячій лікарні Evelina London Children’s Hospital та дитячий
невролог-консультант в Оксфордській дитячій лікарні Oxford Children’s Hospital
— Dr Sithara Ramdas.

🔻Попри війну, лікарня Охматдит продовжує науково-практичну роботу. Дякуємо
учасникам за постійний розвиток та безперервну працю. 

Запис трансляції за посиланням:

\url{https://youtu.be/FX2iJrdS8M8}

📸Більше фото за посиланням:

\ii{05_12_2022.fb.ohmatdet.bolnica.1.shkola_nevrologia.orig}
