% vim: keymap=russian-jcukenwin
%%beginhead 
 
%%file 16_04_2021.fb.respublikalnr.2.kosmos
%%parent 16_04_2021
 
%%url https://www.facebook.com/groups/respublikalnr/permalink/797325424236584/
 
%%author 
%%author_id 
%%author_url 
 
%%tags 
%%title 
 
%%endhead 

\subsection{Газета "Республика" (№15, 2021г).  Свой взгляд. Освоение стратосферы и космоса – это наша история}
\Purl{https://www.facebook.com/groups/respublikalnr/permalink/797325424236584/}

Дата первого полета в космос отныне стала новой временной точкой отсчета. ДО и
ПОСЛЕ. ДО полёта в заоблачную высь и ПОСЛЕ взгляда человека на голубую планету
Земля с космической высоты.

\ifcmt
  pic https://scontent-mxp1-2.xx.fbcdn.net/v/t1.6435-9/174805925_122959133217528_1884652516997879483_n.jpg?_nc_cat=105&ccb=1-3&_nc_sid=825194&_nc_ohc=7Qw6qm5R0UsAX8eZ_lL&_nc_ht=scontent-mxp1-2.xx&oh=513ad708fc8d4cdb2ef92cc3c218e73b&oe=60A262A4
\fi

12 апреля 1961 года гражданин Советского Союза, 27-летний старший лейтенант
Юрий Алексеевич Гагарин на космическом корабле «Восток» впервые в мире совершил
орбитальный облёт Земли, открыв эпоху пилотируемых космических полётов. 

В день полёта Гагарина после первых сообщений по радио тысячи людей высыпали на
улицы городов в радостном приподнятом настроении. Обнимались, поздравляли друг
друга – посланец двухсотмиллионной страны первым начал освоение космического
пространства! Вся планета, раскрыв глаза, смотрела на нас.

Полёт, длившийся всего 108 минут, стал мощным прорывом в освоении космоса. Имя
Юрия Гагарина стало широко известно в мире, а сам первый космонавт досрочно
получил звание майора и звание Героя Советского Союза. Это – день триумфа науки
и всех тех, кто сегодня трудится в космической отрасли. День авиации и
космонавтики 12 апреля ежегодно отмечает весь мир. А в текущем году дата
первого космического полёта юбилейная – 60 лет.

ЭПОХА ПРОРЫВА

Вы никогда не задумывались, почему именно в СССР был произведён бросок в
космос? Это всё идёт оттуда – из времени первых пятилеток. Все, в том числе и
великий Генеральный конструктор С. П. Королев, формировались там же, в эпохе
30-х годов. Атмосферу того времени можно уловить, посмотрев фильмы тех лет,
прочитав книги, послушав песни. Вставала «страна героев, страна мечтателей,
страна учёных». Это было время мобилизации и развития, охватившее массы.
Миллионы молодых, впервые получивших образование, горели энтузиазмом. Они были
уверены, что им по плечу любые фантастические с точки зрения техники проекты.

Именно в 1930-е стали авторитетными труды учителя из провинциальной Калуги К.
Э. Циолковского – основоположника теоретической космонавтики, который
предсказал появление Гагарина: «Я свободно представляю первого человека,
преодолевшего земное притяжение и полетевшего в межпланетное пространство. Он
русский. По профессии лётчик. У него отвага умная, лишённая дешёвого
безрассудства. Представляю его открытое русское лицо, глаза сокола». 

В 1936 году на экраны нашей страны вышел фантастический фильм «Космический
рейс» Василия Журавлева. Он потряс воображение советских людей. Сделанный с
привлечением лучших ученых, этот фильм рассказывал о том, как космический
корабль с двумя пилотами должен стартовать на Луну в 1946 году! Ракетоплан в
картине стартует на фоне панорамы будущей Москвы, смоделированной лучшими
архитекторами! Этот фильм помогал создавать и Циолковский. Он сделал к нему
множество рисунков и схем, показывая, как должен стартовать аппарат, как он
должен прилуняться, как будет вести себя экипаж в полёте. Поразительно, но
именно Циолковский предложил ввести в фильм посадку возвращающегося на Землю с
помощью гигантского парашюта ракетоплана. Когда 45 лет спустя этот фильм
показали советским космонавтам, они дружно зааплодировали гениальному
предвидению. Впоследствии альбомы с рисунками и теоретическими выкладками,
сделанными при создании фильма, издадут в виде научного труда. Случай
беспримерный в мировой кинематографии!

В такой атмосфере реальности росли будущие покорители космоса.

\subsubsection{«Всё выше, выше и выше…»}

Небо звало первых лётчиков и стратонавтов. Июльским вечером 1927 года пилот
Александр Иванович Жуков на авиетке «Буревестник» инженера В. П. Невдачина
установил мировой рекорд высоты, поднявшись над землей на 5 тысяч метров и
побив прежний рекорд в 1 200 м. Этим рекордом Советское государство в очередной
раз заявило о своём стремлении стать одной из ведущих мировых авиационных
держав. Имена героических лётчиков были у всех на устах: Чкалов, Каманин,
Водопьянов, Леваневский… И будут ещё рекорды по высоте и дальности полётов.

А исследования стратосферы, то есть высоты более 11 км? Для этого применялись стратостаты – специально оборудованные аэростаты. 30 сентября 1933 года стратостат «СССР-1» конструкции К. Д. Годунова совершил полёт на высоту 19 км, установив новый мировой рекорд. Рвавшиеся в небо знали, что идут на риск, полёты не всегда оканчивались благополучно. В столице ДНР есть памятник четырём стратонавтам – Я. Г. Украинскому, С. К. Кучумову, П. М. Батенко и Д. Е. Столбуну, совершившим полёт на стратостате «ВВА-1» в июле 1938 года и погибшими над Донецком при выполнении задания. Но никакая опасность не останавливала новое пополнение покорителей неба.

Наш Луганск всегда был авиационным городом. В годы расцвета авиации в нём
находились аэропорт, лётное военное училище, учебный авиационный центр ДОСААФ.
В 1930 году была создана 11-я военная школа пилотов, позднее получившая имя
Пролетариата Донбасса, а в 1933 году – аэроклуб. На лётных полях аэроклуба
получали путевку в небо многие замечательные летчики, героически защищавшие
небо страны в годы Великой Отечественной войны. Среди выпускников Луганского
аэроклуба – дважды Герой Советского Союза генерал-лейтенант А. И. Молодчий и 17
Героев Советского Союза.

А сколько улиц и жилых массивов нашего города носят имена лётчиков, космонавтов
и аэростатчиков! Например, Н. Гудованцев, А. Ритслянд, В. Лянгузов, И. Паньков,
Д. Градус, С. Демин, чей подвиг был увековечен в названиях улиц на Городке
завода ОР. Кем они были? Большинство луганчан об этом не знает, да и сами имена
воздухоплавателей по истечении многих десятков лет были забыты, а фамилии,
благодаря устной народной обработке, претерпели изменения – Годуванцев,
Градусов, Рислянд... А ведь на самом деле эти отважные люди, посвятившие свою
жизнь покорению воздушной стихии и погибшие при спасении гораздо более
известных сейчас своих современников – полярников-папанинцев, имеют самое
непосредственное отношение к Донбассу. Слово о них.

\subsubsection{Стратонавты}

В начале ХХ века наряду с развитием самолётостроения наступила эра дирижаблей
или управляемых аэростатов – аппаратов легче воздуха. Строившиеся в 1930-х
годах жёсткие дирижабли представляли собой гигантские воздушные корабли с
дальностью беспосадочного полёта 10–15 тыс. км, с полезной нагрузкой до 90
тонн. В Советском Союзе первый дирижабль был построен в 1923 году. Позднее была
создана специальная организация «Дирижаблестрой», которая построила и сдала в
эксплуатацию более десяти дирижаблей мягкой и полужёсткой систем. В 1937 году
крупнейший советский дирижабль «СССР-В6» установил мировой рекорд
продолжительности полёта – 130 часов 27 минут. 

Дирижабли использовались в народном хозяйстве. На них испытывались методы
борьбы с лесными пожарами, учёта лесов и уничтожения малярийных очагов. Для ВВС
проводились прыжки парашютистов. На дирижабли возглавлялись большие надежды по
доставке грузов и перевозке пассажиров на дальние расстояния. Существенная роль
отводилась также агитационным полётам и обучению экипажей.

Одним из наиболее опытных аэронавтов был Николай Семёнович Гудованцев. Выходец
из рабочей семьи, любознательный юноша увлёкся авиацией и особенно
дирижаблестроением, куда и направил свой путь. После соответствующей учёбы Н.
С. Гудованцев получил звание инженера-механика дирижаблестроения и с тех пор
был участником всех крупных перелётов дирижаблей. Некоторые из этих рейсов
выполнялись над Донбассом, с бивуачной стоянкой в Сталино, как назывался тогда
Донецк. 

В 1937 году за высокие лётные качества и организаторские способности 28-летний
Гудованцев был назначен командиром эскадры дирижаблей и утверждён Совнаркомом
Союза членом совета при начальнике Аэрофлота.

В феврале 1938 года поступили сообщения, что льдина, на которой дрейфовала
полярная экспедиция И. Д. Папанина, раскололась. Полярников надо было спасать.
Гудованцев внёс предложение использовать свой корабль. Вечером 5 февраля
дирижабль «СССР-В6» вылетел из Москвы в сторону Мурманска. Следуя по маршруту и
имея регулярную связь по радио с Москвой, дирижабль пролетел свыше 20 часов. К
19 часам 6 февраля он приближался к станции Кандалакша (277 км до Мурманска) и
тут связь с ним внезапно оборвалась. Дирижабль потерпел аварию. 

При расследовании причин выяснилось, что корабль шёл на высоте 300–450 м при
густой облачности, что в условиях наступившей темноты ухудшало видимость. По
маршруту движения попалась не обозначенная на имевшейся у экипажа полётной
карте-десятивёрстке выпуска 1904 года высокая скала, в которую с ходу врезался
дирижабль и, переломившись, рухнул на неё. 

Из 19 человек экипажа погибли 13 и среди них – командир корабля Н. С.
Гудованцев, второй командир И. В. Паньков, первый помощник командира С. В.
Дёмин, второй помощник командира В. Г. Лянгузов, первый штурман А. А. Ритсланд,
бортсиноптик Д. И. Градус. В честь отважных аэронавтов решением исполкома
Ворошиловградского горсовета были названы несколько улиц, которые и сейчас
носят имена героев.

Трагедия «СССР-В6» вызвала большой резонанс в стране и наряду с другими
подобными катастрофами в мире послужила закату «эпохи дирижаблей». Но не
стремлению человека в небесную высь, к которой шли, преодолевая все преграды.

\subsubsection{Люди-легенды}

3 марта 1960 года министр обороны СССР Р. Я. Малиновский выпустил приказ № 0031
«Временное положение о космонавтах». Отряд космонавтов Военно-воздушных сил
СССР был сформирован по результатам отбора среди лётчиков истребительной
авиации. В него вошли двадцать человек – молодые офицеры с отменным физическим
и психологическим здоровьем, отличными умственными способностями, готовых
выполнять любые приказы и участвовать в любых испытаниях. С 25 марта начались
регулярные занятия по программе подготовки космонавтов.

Кто они были, те, первые, о которых ходили легенды, а их имена знал каждый?
Юрий Гагарин родом из деревни Клушино Смоленской области, сын плотника и
доярки. Окончил ремесленное училище по специальности формовщик-литейщик,
Саратовский индустриальный техникум, занимался в Саратовском аэроклубе. Был
направлен в Оренбургское военное авиационное училище, по окончании которого
служил в морской авиации. И биографии других во многом схожи, просты и близки.
Из двадцати претендентов на первые полёты отобрали шестерых: Юрий Гагарин,
Герман Титов, Григорий Нелюбов, Андриян Николаев, Павел Попович и Валерий
Быковский. По очерёдности государственной комиссией первым был выбран Юрий
Гагарин. Да и в проведённом опросе среди других членов команды космонавтов они
тоже указали на него.

12 апреля 1961 года стало триумфом Страны Советов. С того дня и положение
Гагарина резко изменилось. Он стал не только первым летавшим космонавтом, но и
публичной фигурой мирового масштаба. Он объехал 30 стран, и везде его ждал
восторженный приём, он встречался с главами государств и правительственными
чиновниками, профсоюзными и общественными деятелями, военными и учёными,
обычными гражданами и детьми. И надо сказать, испытание внезапно обрушившейся
славой Юрий Гагарин выдержал с честью.

…Космические полёты продолжались, рос и отряд космонавтов. В него влились и
уроженцы Луганщины, что решили связать свою жизнь с космосом. Это Владимир
Афанасьевич Ляхов, он родился в Антраците в 1941 году; Георгий Степанович
Шонин, который родился в Ровеньках в 1935 году. К тому же было время, когда наш
тепловозостроительный завод получил задание особой важности: по спецзаказу были
созданы тепловозы, которые работали на Байконуре и вывозили на стартовую
площадку космические корабли. 

В Луганском краеведческом музее представлена экспозиция, посвящённая
землякам-космонавтам. Среди уникальных экспонатов – скафандр, в котором
Владимир Ляхов побывал в открытом космосе, переданный им в дар музею. За три
космических полёта, общая продолжительность которых составила более 333 суток,
он три раза выходил в космос (суммарная продолжительность выходов 7 часов 8
минут). Родина достойно оценила его подвижнический труд – В. А. Ляхов дважды
удостоен звания Героя Советского Союза с вручением медалей «Золотая Звезда»,
награждён рядом отечественных и иностранных орденов и медалей.

Знаменитый космонавт никогда не забывал родную Луганщину. Особенно часто он
приезжал в Антрацит. На родине у него осталось много друзей – Владимир Ляхов
был простым, общительным, доброжелательным, любил хорошую шутку. Таким его
запомнили многие. Он стал почётным гражданином Луганщины и почётным президентом
региональной общественной организации «Луганское землячество» г. Москвы. По
этой линии было сделано много. Когда же пришла гроза 2014 года, самым важным
стал вопрос оказания гуманитарной помощи родному краю, в организации которой
Ляхов принимал самое активное участие...

К сожалению, сегодня земляков-космонавтов нет среди нас. Годы и напряжённый
труд взяли своё. Но такие люди будут вечно жить в наших сердцах. Люди с большой
буквы.

Государственным унитарным предприятием Луганской Народной Республики «Почта
ЛНР» выпущен блок художественных почтовых марок к 60-летию первого полета в
космос и Дню космонавтики. На марках размещены портретные фотографии Героев
Советского Союза: первого в истории человечества космонавта Юрия Алексеевича
Гагарина; окончившего в 1941 году Ворошиловградскую школу военных летчиков
имени Пролетариата Донбасса космонавта Георгия Тимофеевича Берегового.
Иллюстративной основой для блока марок стала репродукция картины военного
летчика, летчика-космонавта, инженера и художника Алексея Архиповича Леонова
«Наша прекрасная планета».

Среди торжеств, проводимых в честь 60-летия первого космического полёта,
поступило сообщение: 9 апреля 2021 года со стартовой площадки № 31 космодрома
Байконур выполнен пуск ракеты-носителя «Союз-2.1а» с пилотируемым кораблём «Ю.
А. Гагарин» (Союз МС-18) и экипажем длительной экспедиции МКС-65. 

Штурм космического пространства продолжается.

Александр АКЕНТЬЕВ

ГАЗЕТА "РЕСПУБЛИКА" (№15, 2021г).

\verb|#газета #республика #изучаем_историю #развитие_космонавтики|
