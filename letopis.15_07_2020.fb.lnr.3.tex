% vim: keymap=russian-jcukenwin
%%beginhead 
 
%%file 15_07_2020.fb.lnr.3
%%parent 15_07_2020
 
%%endhead 
  
  
\subsection{На очереди --- бордели. На Украине легализован игорный бизнес}
\label{sec:15_07_2020.fb.lnr.3}
\Purl{https://www.facebook.com/groups/LNRGUMO/permalink/2846882902089932/}
\Pauthor{Пономарева, Анна}

\cite{15_07_2020.fb.lnr.3}

\index[rus]{Казино}

Горе-президента Владимира Зеленского еще долгие годы будут проклинать потомки.
Чтобы наворотить столько дел за короткий период пребывания у власти, надо
обладать особым умением. И проблема не только в масштабной распродаже всей
Украины, которую организовал глава государства. Сегодня Верховная рада с легкой
руки гаранта приняла закон «О государственном регулировании деятельности
организации и проведения азартных игр», легализующий в стране игорный бизнес.

Отныне на Украине расцветут казино, включая онлайн, букмекерские конторы,
появятся залы игровых автоматов (что для полуголодного населения особенно
актуально). Будет разрешено проведение онлайн-игр в покер, при этом турниры по
спортивному покеру азартной игрой не считаются.  Правда, для самоуспокоения
были предложены некоторые нормы, которые позволят держать в узде азартные
инстинкты украинцев, но на деле они и гроша ломаного не стоят. Например,
запрещено допускать к играм недееспособных и тех, чья дееспособность
ограничена, а также людей в состоянии алкогольного или наркотического
опьянения. Зная украинскую действительность, вряд ли около каждого игрового
автомата будет стоять проверяющий и смотреть, вменяем клиент или готов отдать
последнее. Еще один важный момент, на который обращают внимание эксперты, —
модуль онлайн-мониторинга казино и залов игровых автоматов появится только с
2022 года. Что это означает --- к гадалке не ходи: игорный бизнес будет трудиться
бесконтрольно. И, естественно, никто не подумает о том, что необходимо
ограничивать зависимых игроков, как о том поют правила закона. Более того,
отсутствие контроля за игрой позволяет легализовать теневые доходы депутатам и
госчиновникам. В своих декларациях эти умельцы начнут вписывать доходы, как
выигрыши. А, они, как известно, налогами не облагаются.

Так что все планы киевских мудрецов по поводу того, что легализация игорного
бизнеса даст дополнительно в бюджет более \$400 млн в год и позволит ускорить
рост ВВП страны до 1\%, вызывают исключительно смех, равно как и обещания
направить вырученные средства через спецфонд на поддержку спорта, медицины и
образования. «А как иначе: хочешь развивать спорт --- крути рулетку; хочешь
образование --- раздавай карты. И медицина без азарта вообще не медицина.
Молодцы, так держать!» --- написала в Facebook официальный представитель
Министерства иностранных дел РФ Мария Захарова.

На Украине сказки о миллионах на несчастье людей вызывают негодование. «Игорный
бизнес --- чрезвычайно социально опасен. И пострадают от него те люди, которые и
так живут за чертой бедности и будут выносить последнее из своих домов в
надежде на легкие деньги. Кроме того, и само государство от такого решения не
выиграет --- законопроект прописан в интересах конкретных групп, которые
«крышуют» подпольные казино», --- заявила депутат от «Оппозиционной платформы —
За жизнь» Наталья Королевская.

По словам бывшего депутата Верховной рады Украины Алексея Журавко, вместо того,
чтобы заниматься решением реальных проблем государства и спасать его население
от бедности, украинские власти ввергают страну в очередной хаос.

«Я считаю, что это очередной бред. Наша страна и так нищая. А что такое игорный
бизнес? Это, я вам напомню, один из основных источников бедности граждан,
которые превращаются в своего рода наркоманов. Они будут выгребать остатки
пенсий своих стариков, продавать квартиры, закладывать дома и имущество --- и все
ради того, чтобы испытать эйфорию возле рулетки и за карточным столом», —
цитирует ФАН Алексея Журавко, по мнению которого, негативные последствия от
легализации игорного бизнеса полностью обнулят все налоговые поступления и
«рост» ВВП. И, как следствие, в не очень благополучной Украине начнется разгул
бандитизма, грабежей и даже убийств. Предприимчивые граждане, как в 90-е годы,
станут «скручивать колеса с машин, воровать магнитолы, таскать стекла и
запчасти, а потом нести все это в ломбард».  Понятно, что легализация казино и
игровых залов --- это далеко не первый шаг в политике самоуничтожения, которую
проводят местные власти, да и не последний. Следующим логичным решением
украинского парламента вполне может стать легализация борделей. «Я уже ничему
не удивлюсь. Примут этот закон, чипируют всех проституток и будут наживаться
еще и на этом. Таким образом, наши девочки превратятся в товар, которым будут
пользоваться развратные украинские (да и не только!) богатеи», --- убежден
Алексей Журавко.  И мало ли, что украинцы не согласятся с таким ноу-хау. С
легализацией игорного бизнеса тоже не все было гладко, согласно соцопросам,
более 60\% населения не поддерживало возвращение азартной индустрии «ни при
каких условиях». Но власти закон продавили (голоса «за» --- 224 из необходимых
248 --- отдали депутаты от «Слуги народа»), и украинцы проглотили это молча.

