% vim: keymap=russian-jcukenwin
%%beginhead 
 
%%file 19_02_2019.stz.news.ua.mrpl_city.1.viktoria_lisogor_beregynja_fondu_mariupolja
%%parent 19_02_2019
 
%%url https://mrpl.city/blogs/view/viktoriya-lisogor-bereginya-knizhkovogo-fondu-mariupolya
 
%%author_id demidko_olga.mariupol
%%date 
 
%%tags kultura,mariupol,mariupol.biblioteka.korolenka,mariupol.pre_war
%%title Вікторія Лісогор: берегиня книжкового фонду Маріуполя
 
%%endhead 
 
\subsection{Вікторія Лісогор: берегиня книжкового фонду Маріуполя}
\label{sec:19_02_2019.stz.news.ua.mrpl_city.1.viktoria_lisogor_beregynja_fondu_mariupolja}
 
\Purl{https://mrpl.city/blogs/view/viktoriya-lisogor-bereginya-knizhkovogo-fondu-mariupolya}
\ifcmt
 author_begin
   author_id demidko_olga.mariupol
 author_end
\fi

У Маріуполі живе чимало талановитих, непересічних і яскравих особистостей.
Місту пощастило, що так багато людей невпинно працюють задля його розвитку і
процвітання. Деякі приклади вражають і надихають водночас. Мова піде про дуже
енергійну і працьовиту, мудру і виважену, креативну і відповідальну
\emph{\textbf{Вікторію Олександрівну Лісогор}}. Сьогодні вона займає важливу
посаду, від якої залежить духовний розвиток маріупольців. Наша героїня очолює
Централізовану бібліотечну систему міста, а це означає, що на її тендітні плечі
лягли складні завдання: посприяти розвитку бібліотек і зберегти книжковий фонд
Маріуполя. Пропоную познайомитися з Вікторією Олександрівною поближче.

\ii{19_02_2019.stz.news.ua.mrpl_city.1.viktoria_lisogor_beregynja_fondu_mariupolja.pic.1}

Народилася Вікторія 7 листопада 1973 року в радгоспі Приазовський
Першотравневого району, поруч з Маріуполем (тоді ще Жданов). Її мама працювала
зоотехніком, а тато - водієм. Росла вона у дружній родині, сумувати не
доводилося, адже був і молодший брат. Цікаво, що наша героїня Маріуполь
полюбила з дитинства. Оскільки вона росла у селі, то кожна поїздка до міста
була для неї подією. Дівчинку вабило величезне місто своїми вогнями, своєю
величчю. Вони з братом дуже раділи, коли в новорічні свята батьки відвозили їх
на новорічні вистави чи новорічну ялинку. Особливо запам'яталися Вікторії
поїздки на море, яке вона полюбила з дитинства. Загалом Маріуполь у неї
викликає дуже багато гарних, радісних і світлих почуттів та незабутніх
спогадів. Дуже любить гуляти історичною частиною міста.

\textbf{Читайте також}: \href{https://mrpl.city/blogs/view/razom-mi-zmozhemo-vse-dopomozhemo-kozhnomu-mariupoltsyu-pidkoriti-vezhu}{%
Разом ми зможемо все: допоможемо кожному маріупольцю "підкорити" Вежу, %
mrpl.city, 19.02.2019%
}

Сьогодні Вікторія Олександрівна живе з чоловіком і дочкою. На всі звершення і
перемоги її надихає саме родина. Дочка вже досить доросла, вона викладач в
Технічному ліцеї, чоловік працює в правоохоронних органах. Вікторія дуже любить
слухати музику, дивитися цікаві фільми (стежить за новинками) і гуляти берегом
моря.

З дитинства у Вікторії і думки не було стати бібліотекарем. Вона дуже хотіла
бути лікарем. Навіть намагалася вступити до маріупольського медичного училища.
Але доля розпорядилася інакше. Не пройшовши конкурс в медучилище, для того щоб
не втрачати рік, вона пройшла курси секретаря-друкарки в ДТСААФі. Одного разу,
проходячи повз бібліотеки ім. Короленка, випадково побачила оголошення про
прийняття на роботу секретаря. Зайшла тільки поцікавитися вакансією, а
«залишилася» там на довгі роки. Пропрацювавши кілька років, вступила до
Донецького училища культури, а після закінчення – до Харківської академії
культури. Отримала професію за фахом «Бібліотекознавство і бібліографія»,
кваліфікація за дипломом – бібліотекар-бібліограф. У бібліотеці ім. Короленка
вона працює вже 28 років (з 1991 року), про свій вибір не жалкує, більшість
свого часу віддає саме роботі. На посаді директорки Центральної бібліотечної
системи Вікторія Олександрівна не так довго, але це той щасливий випадок, коли
людина на своєму місці. Підлеглі дуже поважають свого керівника, мабуть, тому
за досить короткий час було створено злагоджений і дружній колектив, який охоче
розвивається і намагається діяти в ногу з часом. Сама Вікторія Олександрівна
кожного дня вчиться чомусь новому, освоює різні процеси. Вона дуже вимоглива до
себе та інших, як кажуть ті, хто оточує, у неї «синдром відмінниці». Часто
боїться зробити щось не так, когось підвести. Але завдяки підтримці як сім’ї,
друзів, так і колег все виходить якраз «на відмінно»!

\textbf{Читайте також}: \href{https://mrpl.city/blogs/view/oleksandr-chernov-hranitel-hudozhnih-tsinnostej-mariupolya}{%
Олександр Чернов: хранитель художніх цінностей Маріуполя, %
mrpl.city, 12.02.2019%
}

Вікторія підкреслила, що кожна книга, журнал або газета мають свого читача.
Перш ніж комплектувати фонд бібліотек або робити передплату на періодику,
необхідно вивчити потреби читачів. Тому до цих процесів Вікторія Олександрівна
і її колеги ставляться дуже уважно, намагаються задовольнити будь-який «каприз»
відвідувачів.

Наша героїня поділилася смаками маріупольських читачів. Молодь зараз більше
цікавиться історичною, філософською та психологічною літературою. Доросла
аудиторія більше полюбляє наукову та пізнавальну. Однак детектив і любовний
роман теж ніхто не скасовував. Цікаво, що в Маріуполі дуже затребувана сучасна
українська література, зокрема Сергій Жадан, Андрій Кокотюха, брати Капранови,
Ірен Роздобудько та інші.

\ii{19_02_2019.stz.news.ua.mrpl_city.1.viktoria_lisogor_beregynja_fondu_mariupolja.pic.2}

Вікторія Олександрівна наголошує, що сучасна бібліотека відрізняється від
радянських бібліотек, тому це дві абсолютно різні історії. Ніхто й подумати не
міг, що в бібліотеці може функціонувати ігротека, кіноклуби, клуб англійської
мови, клуби за інтересами, спортивна кімната і навіть студія крою та шиття. І
це все це не заважає справжнім любителям книги, яка не просто стоїть на полиці,
а залишається головною цінністю кожної бібліотеки. Вікторія підкреслила, що
ніяка електронна книга не замінить новенького з запахом друкарської фарби
видання улюбленого автора. А шелест сторінки – взагалі неймовірне відчуття.
\emph{«Книга завжди була і залишиться живою і затребуваною»}, – зазначає Вікторія.

\textbf{Читайте також}: \href{https://mrpl.city/blogs/view/mariupolets-evgen-sosnovskij-take-krasive-zhittya}{%
Маріуполець Євген Сосновський: таке красиве життя, %
Ольга Демідко, mrpl.city, 19.01.2019}

Всього Центральна бібліотечна система для дорослих об'єднує 11 бібліотек
(Центральна бібліотека і 10 бібліотек-філій). Вікторія Олександрівна вважає, що
кожен колектив – це маленька бібліотечна сім'я. А загалом у бібліотеці
випадкових людей не буває. Зараз працюють і молодь, і люди більш зрілого віку.
Наступність поколінь вже дуже звична справа. Але й більш зрілі співробітники
намагаються не поступатися, освоюють нові інформаційні технології, тому
молодість і зрілість йдуть пліч-о-пліч.

\ii{19_02_2019.stz.news.ua.mrpl_city.1.viktoria_lisogor_beregynja_fondu_mariupolja.pic.3}

Останні два роки Вікторія Олександрівна дуже активно працює з Міжнародною
організацією з міграції (проекти громади для ВПО). Отримавши корисний досвід, у
2017 році колективу вдалося виграти у цього донора грант, завдяки чому
покращилася матеріально-технічна база. До того ж, у бібліотеці ім. Короленка
абсолютно безкоштовно проходять майстер-класи з різних технік прикладного
мистецтва, школа фінансової грамотності, психологічні курси, школа комп’ютерної
грамотності тощо. Наприкінці 2018 року колектив бібліотеки був визнаний однією
з кращих громад і виграв ще два гранти на суму понад 300 тис. грн. Завдяки цій
перемозі було відкрито спортивний куточок, почала працювати школа (йога для
чайників), студія крою та шиття та інше.

\textbf{Читайте також:} \href{https://mrpl.city/blogs/view/volodimir-kulbaka-vidatnij-mariupolskij-arheolog-i-ulyublenij-vikladach}{%
Володимир Кульбака: видатний маріупольський археолог і улюблений викладач, %
mrpl.city, Ольга Демідко, 21.12.2018%
}

У Вікторії Олександрівни є свій власний Еверест. Вона дуже хоче побачити
центральну бібліотеку міста суперсучасною, оснащеною передовою технікою,
RFID-технологіями, з терміналами самовидачі книг та іншими унікальними речами,
які зроблять бібліотеку більш комфортною і затишною. А найголовніше
–затребуваною і сповненою читачами. Оскільки сьогодні у бібліотеці ім.
Короленка відбувається глобальна реконструкція, за якою ретельно і дуже уважно
стежить Вікторія Олександрівна, особисто я впевнена, що цей Еверест буде все ж
таки підкорений.

\textbf{Улюблені книги Вікторії Олександрівни:} «Гордість та упередження» Джейн Остін.

\textbf{Улюблені фільми:} «Віднесені вітром» з неповторною Вів'єн Лі.

\textbf{Життєвий девіз:} «Йди далі. Усе найкраще ще попереду».

\textbf{Порада всім маріупольцям:} «Головне - любити своє місто. І, звичайно ж, берегти».

\textbf{Читайте також:} \href{https://mrpl.city/news/view/morskuyu-svinku-i-detej-nagradili-v-mariupole-za-selfi-s-knigoj-foto}{%
Морскую свинку и детей наградили в Мариуполе за селфи с книгой, %
Ганна Хіжнікова, mrpl.city, 03.10.2018%
}
