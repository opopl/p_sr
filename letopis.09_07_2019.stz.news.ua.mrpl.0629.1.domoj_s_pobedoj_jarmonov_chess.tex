% vim: keymap=russian-jcukenwin
%%beginhead 
 
%%file 09_07_2019.stz.news.ua.mrpl.0629.1.domoj_s_pobedoj_jarmonov_chess
%%parent 09_07_2019
 
%%url https://www.0629.com.ua/news/2447525/domoj-s-pobedoj-mariupolskij-sahmatist-zavoeval-titul-cempiona-foto
 
%%author_id 
%%date 
 
%%tags chess,kultura,mariupol,mariupol.pre_war,peremoga,sport
%%title Домой с победой. Мариупольский шахматист завоевал титул чемпиона, - ФОТО
 
%%endhead 
 
\subsection{Домой с победой. Мариупольский шахматист завоевал титул чемпиона, - ФОТО}
\label{sec:09_07_2019.stz.news.ua.mrpl.0629.1.domoj_s_pobedoj_jarmonov_chess}
 
\Purl{https://www.0629.com.ua/news/2447525/domoj-s-pobedoj-mariupolskij-sahmatist-zavoeval-titul-cempiona-foto}

\ii{09_07_2019.stz.news.ua.mrpl.0629.1.domoj_s_pobedoj_jarmonov_chess.pic.1}

Сегодня, 9 июня, на железнодорожном вокзале Мариуполя было многолюдно. Горожане
и представители департамента молодежи, культуры и спорта встречали выдающегося
мариупольского шахматиста  Игоря Ярмонова.

Он побывал на чемпионате в Словакии, где в 5-ый раз стал чемпионом мира по
шахматам среди спортсменов с нарушением опорно-двигательного аппарата.

Президент федерации шахмат Мариуполя Владимир Гутников отметил, что знает Игоря
более 20 лет. Его первым тренером был отец, который научил парня играть в
шахматы в 6 лет.  \enquote{Это человек, который всего достиг сам. У него столько кубков
и медалей, что их уже можно измерять килограммами}, - сказал он.

Начальник управления спорта горсовета Алексей Казанцев подчеркнул, что Игорь
Ярмонов - это лицо шахмат Мариуполя и благодаря ему к шахматам тянутся дети, а
городская власть имеет вдохновение развивать этот вид спорта. \enquote{Благодаря таким
людям, которые показывают результат, шахматы получают новый виток развития в
нашем городе}, - сказал он.

На вокзале Игоря Ярмонова встретили аплодисментами и цветами. Даже незнакомые
люди специально останавливались, чтобы поздравить чемпиона.

Как рассказала супруга Игоря Галина, для Игоря это был 19-й чемпионат и он
приехал на него в статусе фаворита. Перед шахматистом была огромная
ответственность - защитить свой титул. Он сразу же захватил лидерство и
подтвердил свой статус сильнейшего.

Галина говорит, что ее муж посвящает каждую медаль чему-то выдающемуся. В
прошлом году к 240-й годовщине города, а сегодняшнюю победу - родителям. \enquote{Он
стремился, хотел выиграть, и у него все вышло. Если этот человек ставит цели -
он достигает их}, - рассказывает Галина Ярмонова. Она добавила, что ее муж
уделяет шахматам максимальное количество времени и готовится к новым победам.

Напомним, ранее сообщалось, что Игорю Ярмонову могут присвоить звание \href{https://www.0629.com.ua/news/2440768/sahmatistu-igoru-armonovu-hotat-prisvoit-zvanie-pocetnyj-grazdanin-mariupola}{\enquote{Почетный
гражданин Мариуполя}}.
