%%beginhead 
 
%%file 28_03_2022.fb.wishnevskaja_helen.mariupol.1.v_mariupole_ya_ne_pl
%%parent 28_03_2022
 
%%url https://www.facebook.com/helen.wishnevski/posts/pfbid0FvccTred1g2UgWrBKWrS3iyqtA1fF8QbVsbi7MbYYBJZuTAc1QgijVDzqb4ppPC6l
 
%%author_id wishnevskaja_helen.mariupol
%%date 28_03_2022
 
%%tags mariupol,mariupol.war
%%title В Мариуполе я не плакала. Как-то некогда было
 
%%endhead 

\subsection{В Мариуполе я не плакала. Как-то некогда было}
\label{sec:28_03_2022.fb.wishnevskaja_helen.mariupol.1.v_mariupole_ya_ne_pl}

\Purl{https://www.facebook.com/helen.wishnevski/posts/pfbid0FvccTred1g2UgWrBKWrS3iyqtA1fF8QbVsbi7MbYYBJZuTAc1QgijVDzqb4ppPC6l}
\ifcmt
 author_begin
   author_id wishnevskaja_helen.mariupol
 author_end
\fi

В Мариуполе я не плакала. Как-то некогда было. Дров напили, потом наколи, потом
костёр разведи, поесть приготовь - и всё это вперемешку с постоянными
прятаниями при звуке прилетающего снаряда. В силу своей природной трусости (
да-да, я как тот Зайчишка из мультика - постоянно всего боюсь, жуткая трусиха)
костёр я разводила нему подъезда, как все. После прилёта града под мой подъезд
в нём остались только мы с мамой и Валешкой на своём пятом этаже да ещё одна
семейная пара, которая берегла квартиру детей, но они практически постоянно
были в подвале. Мне с сыном бегать в подвал - сами понимаете. Тем более и
находился он не в моём подъезде. В общем, сидели мы дома. А кормить-то своих
надо. А поджилки-то трясутся. Ну Хелен Валериевна и приняла ответственное
решение - готовить на площадке. Ну а что - окна повылетали в подъезде,
вентиляция замечательная, никого, кроме нас, считай, нет - пуркуа бы и не па?
Да, но нужны дрова. И тут очень пригодились вылетевшие рамы. Притащила парочку,
поставила на площадке. Пошла поменять сыну памперс. Вышла через 15 мин, чтобы
распилить их - нет рам, с*издили. Ну дом-то пятиэтажный, пошла ещё парочку
принесла, тут же преобразовала их в дрова и заботливо занесла в квартиру. А то
шастают тут всякие, топливо воруют. Вспомнила, что где-то на балконе чудом не
выброшенное после ремонта железное ведро. Насыпала туда песка наполовину,
накидала дровишек - и ву а ля! Почти мангал, йопт! Даже борщ получалось сварить
и картохи пожарить. Ну вот когда тут плакать? А счс прорвало. Смотрю, во что
превратили мой Мариуполь и плачут не глаза. Плачет сердце. Да, в моём районе
было горячо, да, видела, что город весь в чёрном дыму, но осознать масштабы до
сих пор не могу. Глаза видят, мозг не принимает. 

P. S. Правило бумеранга никто не отменял....

%\ii{28_03_2022.fb.wishnevskaja_helen.mariupol.1.v_mariupole_ya_ne_pl.cmt}
