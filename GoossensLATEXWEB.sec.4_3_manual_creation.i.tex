
\isubsec{4_3_manual_creation}{Manual creation of hypertext elements}

A relatively small number of low-level commands provides the foundation
on top of which the desired hypertext outcome can (usually) be tailored.
These commands provide direct access to HTML tags, the means to produce
hypertext pages, the support needed to establish hypertext links, and a
method to request a specific appearance of the content. 

\isubsubsec{4_3_1_Raw_hypertext_code}{Raw hypertext code}

Many useful features can be achieved by brute force inclusion of small fragments of 
HTML code. The command 

\begin{verbatim}
\HCode{content} 
\end{verbatim}

is handy on such occasions, because it holds the processing of its
content just to macro expansion. This restricted mode of operation
outputs the characters in raw format instead of replacing them with the
corresponding symbols of the current font (see \refsec{4_6_7_The_font_control_files} for a
description of how \texht works with fonts). 

\begin{verbatim}
                                        Make it 
Make it                                 \HCode{<STRONG>}\texttt 
<STRONG>&1t;STRONG&gt;</STRONG>.        {<STRONG>}\HCode{</STRONG>}. 
\end{verbatim}

The command 

\begin{verbatim}
\Hnewline 
\end{verbatim}

may be used to force line breaks within the parameters of the \verb|\HCode| commands. 

The alternative 

\begin{verbatim}
\HChar{html-character-code} 
\end{verbatim}

command, on the other hand, introduces raw characters at a rate of one
character per instance of the command. For example, in the default
setting the\ \LaTeX\ special tilde character \verb|~| is translated to
\verb|\HChar{160}|. 

It should be noted that, unlike the use of \verb|\HCode|, a parameter of the \verb|\verb| 
command of\ \LaTeX\ is not necessarily translated to exactly the same format in the 
HTML file. All that is required is to present to the reader something that looks 
like the source content. The same holds for the text enclosed within a verbatim 
environment. 

\isubsubsec{4_3_2_Hypertext_pages}{Hypertext pages}

The documents are automatically partitioned into hypertext pages, based
on their logical structure and the options you have chosen. For
brute-force creation of hypertext pages, the pair of commands 

\begin{verbatim}
\HPage{entry-anchor} 
\EndHPage{} 
\end{verbatim}

is available. 

The page may incorporate a navigation button with a command of the form 

\begin{verbatim}
\ExitHPage {exit-anchor} 
\end{verbatim}

for establishing a backward path to the parent of the page. 
 
\begin{verbatim}
\HPage{Enter} 
and then \ExitHPage{exit} 
\EndHPage{} 
the hypertext page. 
\end{verbatim}

If the parameter of \verb|\ExitHPage| is empty, the navigation buttons
employ the anchors supplied in the \verb|\HPage| commands. 

%%page page_188                                                  <<<---3

\isubsubsec{4_3_3_hypertext_links}{Hypertext links}

HTML uses the tag 

\begin{verbatim}
<A HREF="target-file#target-loc" NAME="cur-loc" parameters > anchor </A> 
\end{verbatim}

for representing hypertext links. Each tag of this kind 
assumes knowledge of where the target file is located and the desired entry point 
into the file. In addition, each of these tags provides a name for the current location 
and allows for other parameters. 

The following command offers a similar functionality, while automatically
deriving the local target files when they are not explicitly given. 

\begin{verbatim}
\Link[target-file parameters]{target-loc}{cur-loc}anchor\EndLink 
\end{verbatim}

Specifically, when \emph{target-file} is empty, \texht assumes the target file
belongs to the current document, and it takes it upon itself to find the file.
A file containing a location named \emph{target-loc} is searched for by the
\verb|\Link| command. 

The component \[\emph{target-file parameters}\] is optional, if both
\emph{target-file} and \emph{parameters} are empty. When parameters is not
empty, it must be preceded by a space. 

\begin{verbatim}
                An external link to 
                \Link[http://www.tug.org 
                ID="ORG"]{}{}TUG\EndLink. 

                An internal \Link 
                {to}{}link\EndLink{} to 
                \Link{}{to}here\EndLink. 
\end{verbatim}

Within the parameters of the \verb|\Link| command, the special characters
\verb|~|, \verb|_|, and \verb|%| should be entered using the commands
\verb|\string~|, \verb|\string_|, and\verb|\%| respectively. 

\isubsubsec{4_3_4_css}{Cascading Style Sheets}

HTML is a language for identifying structures within hypertext
documents; it resembles the way the \verb|\section| instructions of \
\LaTeX\  are employed to identify logical structures within source
documents. In both cases, the entities do not deal with how the
structures are to be presented to the readers, delegating such details
to specifications provided elsewhere through special purpose languages. 

Cascading Style Sheets (CSS) [] is a language for specifying
presentations for HTML entities. \texht collects such requests within a
special file named \hbox{\emph{jobname}.css} and issues some requests of
its own. 

It is beyond the scope of this chapter to describe the CSS language. However, it 
should be realized that the language is easy to learn and use by authors with a basic 
knowledge of HTML and a little familiarity with desktop publishing terminology. 

%%page page_189                                                  <<<---3

The command 

\begin{verbatim}
\Css{CSS code} 
\end{verbatim}

is provided for requesting presentation of document elements.  Given
source code of the form 

\begin{verbatim}
\section{Header} Text.
\par 
More text 
\end{verbatim}

\texht will produce HTML output like this: 

\begin{verbatim}
<H2>1 Header</H2> <P CLASS="noindent"> Text. 
<P CLASS="indent"> More text 
\end{verbatim}

and CSS code like this: 

\begin{verbatim}
P.noindent { text-indent: 0em } P.indent { text-indent: 1.5em } 
\end{verbatim}

A request to color the title green can be made with a command of the form 

\begin{verbatim}
\Css{ H2 { color: green } }
\end{verbatim}

A CSS file can be specifically requested with the environment 

\begin{verbatim}
\CssFile[list-of-files] 
content 
\EndCssFile 
\end{verbatim}

CSS code can be imported from other files (\emph{list-of-files}), as well as explicitly given 
within the environment. 

The \verb|\EndPreamble| command calls this environment to create a CSS file, if the 
user does not make an earlier request for such a file. 

The CSS file should include the comment \verb|/* css.sty */| on a separate line so 
that \verb|t4ht|, the \verb|tex4ht| postprocessor (see \refsec{4_6_4_A_look_at_t4ht}), can identify it as a place to 
put the content of the \verb|\Css| instructions. Without such a line, that content will be 
ignored. The filename and the initial content of the file can be reconfigured with 
the \verb|\Configure{CssFile}{filename}{content}| command. 

\emph{Inline} CSS code can be created using the following environment: 

\begin{verbatim}
\Css content\EndCss 
\end{verbatim}

Consider the following source and configuration files: 

\iii{vb_page_189}

%%page page_190                                                  <<<---3
This will create the following HTML and CSS files: 

\begin{verbatim}

<! --- try.htm1 -->                   /* try.css */ 
<STYLE TYPE="text/css">               H2 { color : blue; } 
H2 { color : red; }                   H2 { color : green; } 
</STYLE> 

\end{verbatim}
