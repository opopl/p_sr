% vim: keymap=russian-jcukenwin
%%beginhead 
 
%%file 21_01_2022.stz.news.ua.fromua.1.lnr_dnr_marionetki.2.rezervacii
%%parent 21_01_2022.stz.news.ua.fromua.1.lnr_dnr_marionetki
 
%%url 
 
%%author_id 
%%date 
 
%%tags 
%%title 
 
%%endhead 

\subsubsection{Печерні резервації «руского міра»}

У Європі вважають, що Росія може не наважитися юридично інтегрувати окуповані
території Донбасу до свого складу. Але принципової різниці це вже не матиме. Бо
де-факто відбудеться повне поглинання «республік» у склад РФ. Інтеграція
економіки, інфраструктури та культурного життя ДНР і ЛНР з Росією зайшла надто
далеко. Причина цього – тривала відсутність української влади і українського
впливу на тимчасово окупованих територіях.

Не треба бути видатним експертом, щоб оцінити реальний стан справ у ДНР і ЛНР.
Ці штучні утворення за майже вісім років існування перетворилися у справжні
печерні резервації «руского міра». Відділені від України та решти світу, вони
живуть у власному соціокультурному вимірі.

Відокремлення Донбасу від України почалося ще перед початком російської
агресії. Опитування громадської думки до 2014 року показували, що жителі цього
регіону часто мали власну думку на багато політичних питань, яка не збігалася з
рештою держави. Причина – вплив російського інформаційного простору на
свідомість населення. Окупація Донбасу почалася з окупації свідомості людей. А
вже потім прийшли російські бойовики і проголосили свої «народні республіки».

Сьогодні ДНР і ЛНР живуть в особливій реальності. Велика частина
проукраїнського населення покинули ці богом забуті території. А люди, які
залишилися, або не мають варіантів і можливостей виїхати, або є прихильниками
інтеграції з Росією і щиро ненавидять все українське. Людей на окупованій
частині Донбасу щодня годують байками про українських нацистів і фашистів. З
самого дитсадка там вчать ненавидіти Україну. У школах та у місцевих ЗМІ
пропагують вороже сприйняття нашої держави. Молодь буквально накачують
мілітаризмом і ворожістю до України. Існує лише одна точка зору. І вона
обов’язково антиукраїнська.

ДНР і ЛНР – це уламки радянського минулого з елементами російського
імперіалізму і шовінізму. Резервації, де панує власна альтернативна викривлена
версія реальності. У місцевих медіа новини з України – це наче новини з зони
стихійного лиха, де панує голод, холод і масове зубожіння. Де при владі
нацисти, які переслідують і вбивають за російську мову. Не забувають там і про
США та НАТО, які для пропагандистів ДНР і ЛНР є справжнім втіленням
всесвітнього зла. Якщо послухати, що каже місцева самопроголошена влада про
світ, то складається враження: Північноатлантичний Альянс створювався з метою
поневолити вільний народ Донбасу. А основна геополітична мета Вашингтона –
збройна інтервенція проти Донецька та Луганська.

\ii{21_01_2022.stz.news.ua.fromua.1.lnr_dnr_marionetki.2.rezervacii.pic.3}

У цих печерних оазисах фейків та маніпуляцій панує розгул українофобії і
возвеличується радянське минуле. Святкуються радянські свята і російські
історичні дати. Москва там давно сприймається як втілення справедливості і
основна надія на порятунок. Місцева влада не приховує свого бажання увійти до
складу Російської Федерації. І всіляко заохочує отримання російських паспортів
та економічну співпрацю з державою-агресором. Офіційно України вже давно майже
немає на цій території. Хоча це не означає, що всі жителі ДНР/ЛНР – палкі
любителі «руского міра» і вірять у байки, які їм щодня розповідають окупанти і
їх пособники. Однак публічно висловити іншу думку неможливо. Бо ризикуєш
втратити роботу, житло, майно, потрапити у в’язницю і отримати ярлик «пособника
українських карателів».
