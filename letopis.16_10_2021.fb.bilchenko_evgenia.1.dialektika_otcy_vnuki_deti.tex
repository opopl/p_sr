% vim: keymap=russian-jcukenwin
%%beginhead 
 
%%file 16_10_2021.fb.bilchenko_evgenia.1.dialektika_otcy_vnuki_deti
%%parent 16_10_2021
 
%%url https://www.facebook.com/yevzhik/posts/4349706408397769
 
%%author_id bilchenko_evgenia
%%date 
 
%%tags bilchenko_evgenia,dedushka,pamjat,poezia,semja
%%title БЖ. Диалектика: деды, отцы, внуки - 1, 2, 3
 
%%endhead 
 
\subsection{БЖ. Диалектика: деды, отцы, внуки - 1, 2, 3}
\label{sec:16_10_2021.fb.bilchenko_evgenia.1.dialektika_otcy_vnuki_deti}
 
\Purl{https://www.facebook.com/yevzhik/posts/4349706408397769}
\ifcmt
 author_begin
   author_id bilchenko_evgenia
 author_end
\fi

БЖ. Диалектика: деды, отцы, внуки - 1, 2, 3

Я читаю стихи одного молодого питерского поэта. Она совсем ещё небывалая
(оттого - настоящая), свежая и только начинает свой полет, не догадываясь о
муке, ожидающей ее на пути русского слова. Не могу удержаться и публикую её
текст, срезонировавший мне, потому что он - о Деде. Отцы, которые выросли на
Америке и Бродском: вот - мы, внуки. Тезис, антитезис и синтез. Дух,
потреблятство и снова дух. Фото прилагаю - моего родного молодого деда. Тоже
военного, инвалида боевых действий, но моложе: полковника, чернобыльского
героя. А с нацистами у меня прабабушка воевала. 

\ifcmt
  ig https://scontent-lga3-1.xx.fbcdn.net/v/t39.30808-6/246164844_4349706548397755_8229741525950723012_n.jpg?_nc_cat=110&ccb=1-5&_nc_sid=8bfeb9&_nc_ohc=-0TrLId5REMAX-YPfzo&_nc_ht=scontent-lga3-1.xx&oh=e81b97e9dfd110b86adb7763ddd1f5dc&oe=61727C2B
  @width 0.4
  %@wrap \parpic[r]
  @wrap \InsertBoxR{0}
\fi

 * *
в память о моём дедушке Иване Павловиче Гирман
Мой дед не любил говорить о войне,
только когда приходил к нам дядя.
А я в полусне
молчала, из кухни прокуренной глядя,
и слушала речь о вещах, непонятных мне.
Мой дед
изучал все газеты, но не любил
то, что случалось на той стороне экрана
(вернее, на той стороне страны).
Смотрел передачи про дальние страны.
Что до войны:
бабушка крикнет: «Иван, подай мухобойку»,
а мне продолжает рассказывать:
«Были года,
враг наступал. Победи он,
не стало бы стойких,
ни школ, ни заводов, ни стройки,
ни даже соседа Кольки,
и было бы горько от рабского, злого труда».
Потом говорит: 
«А наш дед, когда не был дедом, 
сражался с фашистом».
Я думала: вот это да!
Когда-то он был трудовик, уважаемый в школе,
чинил табуреты и мог бы построить печь.
Но дедушки нет, и ломается всё.
Поневоле
подумаешь: где это время со вкусом соли,
и эта несложная  речь?
Автор: Маргарита Чекунова
\textbf{\#этопитердетка}
