% vim: keymap=russian-jcukenwin
%%beginhead 
 
%%file slova.slon
%%parent slova
 
%%url 
 
%%author 
%%author_id 
%%author_url 
 
%%tags 
%%title 
 
%%endhead 
\chapter{Слон}
\label{sec:slova.slon}

%%%cit
%%%cit_head
%%%cit_pic
%%%cit_text
В Китае, в провинции Юньнань стадо диких азиатских \emph{слонов} весной покинули
заповедник и начали миграцию на север. Животные прошли уже более 500
километров, в начале они шли по малонаселенной местности, а сейчас дошли до
столицы провинции - Куньмина.  Об этом сообщает The Guardian и Global Times.
Когда именно \emph{слоны} покинули заповедник неизвестно, первые упоминания об этом
появились в апреле. Что стало причиной того, что животные ушли из места, где
для них были созданы все условия и вернутся ли они домой - также пока
неизвестно. Изначально заповедник покинуло 16 \emph{слонов}, но затем двое из них
вернулись обратно. В пути родился \emph{слоненок}, таким образом, на данный момент по
провинции идет 15 \emph{слонов}
%%%cit_comment
%%%cit_title
\citTitle{В Китае слоны покинули заповедник и начали миграцию}, Эллина Либцис, strana.ua, 09.06.2021
%%%endcit

