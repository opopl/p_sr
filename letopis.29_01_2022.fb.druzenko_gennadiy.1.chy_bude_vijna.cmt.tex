% vim: keymap=russian-jcukenwin
%%beginhead 
 
%%file 29_01_2022.fb.druzenko_gennadiy.1.chy_bude_vijna.cmt
%%parent 29_01_2022.fb.druzenko_gennadiy.1.chy_bude_vijna
 
%%url 
 
%%author_id 
%%date 
 
%%tags 
%%title 
 
%%endhead 
\zzSecCmt

\begin{itemize} % {
\iusr{Александр Герасько}

З вчорашнього потоку беззв'язних думок Зе напрошується думка, що Зе хоче
капітуляції, а тут його Байден і Борис заважать. Бо якжеш тепер сказати, що ти
програв на тлі, зокрема військової допомоги.

\begin{itemize} % {
\iusr{Gennadiy Druzenko}
\textbf{Александр Герасько} в мене склалось протилежне враження. Що однозначно свідчить про брак логіки і послідовності в рефлексіях нашого гаранта.

\iusr{Александр Герасько}
\textbf{Gennadiy Druzenko} не зовсім зрозумів Вас

\iusr{Микита Василенко}
\textbf{Gennadiy Druzenko} Спасибо, получил эстетическое удовольствие.

\iusr{Gennadiy Druzenko}
\textbf{Александр Герасько} в мене склалось враження (і не в мене одного), що саме Байден штовхає нас в пастку виконання Мінська, а Зеленський впирається. Хоча з перформансу нашого гаранта годі робити однозначні висновки.

\iusr{Александр Герасько}
\textbf{Gennadiy Druzenko} я не виключаю нічого адже у великій політиці місцю моралі немає. Можливо Президент щось і хотів натякнути, але в нього точно не вийшло і вряд чи буде в майбутньому. Я не знаю, як Він збирається досидіти ще 2.5 роки у такому стані, стані нездоровому.
\end{itemize} % }

\iusr{Alex Yurchak}
як завжди влучно

\iusr{Dmytro Bintsarovskyi}

Ось і Ви про Карибську кризу згадуєте. Це випадково чи Ви справді бачите
паралелі? Якщо останнє, де американські ядерні ракети в Україні? Як Україна
мілітарно загрожує Росії? Я розумію, навіщо ці паралелі Росії: щоб відчути себе
знову над-державою, повернутись у часи, коли СРСР був зі США \enquote{на рівних}. Але
для чого педалювання Карибської теми Україні та Заходу?

\begin{itemize} % {
\iusr{Gennadiy Druzenko}
\textbf{Dmytro Bintsarovskyi} 

я згадую книжку \enquote{The Guns of August: The Outbreak of World War I}, яка описує
як Європа вистрелила собі в скроню. Тому не бачу, як Ваші запитання повʼязані з
моїм текстом. Дисципліна дискусії передбачає обговорення заявленої теми, а не
будь-яких цікавих проблем та міркувань.

\iusr{Dmytro Bintsarovskyi}

Ну, згадка про Карибську кризу у Вашому дописі була зовсім не обов'язковою -
тому я припустив, що вона там невипадково (і чемно перепитав: випадково чи ні?)
До того ж ця тема розкручується на Хвилі, де Ви публікуєтеся. Ну гаразд,
проїхали

\end{itemize} % }

\iusr{Ganna Stulska}
яка це \enquote{Велика Росія} на початку 17 століття!?

\iusr{Ольга Михайлова}

Не могу согласиться, что украиский ресентимент хочет равенства среди всех. В
отношении европейцев - да, равенства, а в отношении русских очевидно
культивация превосходства. Русские как этнос подвергаются постоянному
обесцениванию, наездам на достоинство и т.п. Такой этнической групы, как
русские, в публичном пространстве Украины попросту нет, его интересы не
учитываются, предлогом чего служит огульное уравнение русских Украины и жителей
России.

\end{itemize} % }
