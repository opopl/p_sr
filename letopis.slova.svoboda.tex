% vim: keymap=russian-jcukenwin
%%beginhead 
 
%%file slova.svoboda
%%parent slova
 
%%url 
 
%%author 
%%author_id 
%%author_url 
 
%%tags 
%%title 
 
%%endhead 
\chapter{Свобода}
\label{sec:slova.svoboda}

%%%cit
%%%cit_pic
\ifcmt
  pic https://avatars.mds.yandex.net/get-zen_doc/1641049/pub_60be21f47b0ba72b6f672a58_60be2b04e14854002e7c596f/scale_1200
\fi
%%%cit_text
Посольство США в Украине через Twitter позволило футбольной ассоциации страны
разместить на форме националистический лозунг \enquote{СУГС} и контур страны с Крымом.
Мужики, я восхищён. Королева восхищена. Все мы тут рыдаем. Следующий шаг -
добиться разрешения каждому футболисту сделать себе татуировку на спине, ну или
где удобно, где Украина с Крымом.  Или даже с Крымом и с Кубанью.  И во время
матча вся команда - р-раз! - и эту татуировку показывает стадиону, задирая (или
приспуская) форму. И русские плачут от страха и зависти.  «Крым - их! - кричат
русские в ужасе. - Смотрите! Видите! Вон их Крым!» Мы никогда не научимся быть
столь же \emph{свободными}
%%%cit_title
\citTitle{Футбольная ассоциация Украины разместила конутр страны с Крымом}, 
Захар Прилепин, zen.yandex.ru, 07.06.2021
%%%endcit

%%%cit
%%%cit_pic
%%%cit_text
Ох, труден в осмыслении триединый и (ныне трехглавый) русский народ. Одни себя
в пограничники записали, окраины от москалей стерегут, другие никак не решат:
из каких именно краев себе на холку «русь» варяжскую призвали и что это вообще
значит. А государство со столицей в Минске задачку на два умножило. Никак не
выяснят, почему она — Белая. Русь которая... Схематично проблему так
сформулируем: земли современного государства Белоруссия еще в Средневековье
назывались — Белая Русь. Историки XIX считали (по сей день имеют
последователей), что название происходит от цвета волос и повседневной одежды
белорусского народа (холстина неокрашенная). Более пафосно звучит утверждение:
название «Белая» подчеркивает \emph{древнюю свободу}, кристальную чистоту и
незамутненность, независимость данных земель. Они длительное время не знали
«ига татаро-монгольского», польско-литовского и московского. Как там сказано...

%%%cit
%%%cit_pic
%%%cit_text
Никакой свободы \emph{Слова} больше не будет, цензура становится нормой для
информационной политики Свободного Мира, а единое пространство глобального
интернета уже вскоре разделят на огороженные друг от друга сегменты. Новая
Берлинская стена – только распространившаяся практически повсеместно. А если в
Евросоюзе заблокируют Telegram, его немедленно запретят в свободной
демократической Украине – вдобавок к уже запрещенным VKontakte, Одноклассникам,
Яндексу, Касперскому, итп. Потому что это станет отличным поводом расправиться
с неугодными для власти тг-каналами. Свободный Мир запрещает – и мы тоже.  Не
говорите, что вас не предупреждали
%%%cit_comment
%%%cit_title
\citTitle{Власти Германии грозят заблокировать Telegram}, Андрей Манчук, strana.ua, 13.06.2021
%%%endcit

%%%cit
%%%cit_pic
%%%cit_text
Целый глава партии \enquote{Слуга народа} Корниенко ответил Путину, что в
Украине никого не ущемляют по языковому фактору: \enquote{Так \emph{свободно},
как чувствуют себя русскоговорящие люди в Украине, они наверное, нигде себя в
мире не чувствуют, разве что, в России}.  Для того, чтобы понять, как это
сегодня — \enquote{\emph{так свободно}}, достаточно вспомнить \emph{свободы}
для русского языка в Украине в 2013 и сравнить их с сегодняшними
\enquote{\emph{свободами}}. Нормы Конституции Украины относительно языков
остались прежними, а вот \emph{свобод} — практически не осталось.
\enquote{\emph{Свобод}} для русского языка в Украине у меня не больше, чем в
любой другой стране мира — только говорить на нём. Получать воспитание,
образование и информацию на русском — никаких \emph{свобод} в Украине не
осталось. Но в любой стране мира я — иностранец, а дома, в Украине?
%%%cit_comment
%%%cit_title
\citTitle{Глава партии Слуга народа доказывает Путину, что в Украине не ущемляют права русскоязычных}, 
Александр Скубченко, strana.ua, 14.06.2021
%%%endcit

%%%cit
%%%cit_head
%%%cit_pic
%%%cit_text
Придет день, и с вас смогут спросить за любую картинку в бложике.  Пока
политическая цензура сворачивает \emph{сетевые свободы}, рыночек откусывает
все, что от них осталось. Придет день, и с вас смогут спросить за любую
картинку в бложике - потому что у нее будет хозяин, который захочет денег. И не
только за картинку, потому что приватизируют все, до чего можно дотянуться.
Либертарианская идеология блокчейна вполне логично работает на новый
виртуальный тоталитаризм: "NFT (Non-Fungible Token, или невзаимозаменяемый
токен) работает на основе блокчейна и является по сути уникальным сертификатом,
привязанным к цифровому объекту — картине, ролику, гифке, тексту. От
криптовалюты его отличает та самая «невзаимозаменяемость»: условный биткоин
можно обменять на точно такой же биткоин, но два идентичных NFT найти нельзя
%%%cit_comment
%%%cit_title
\citTitle{Рынок откусывает все, что еще осталось от сетевых свобод}, 
Андрей Манчук, strana.ua, 18.06.2021
%%%endcit


%%%cit
%%%cit_head
%%%cit_pic
%%%cit_text
Целый пласт статей запрещает любые виды дискриминации и поражения в правах.
Статья 21 говорит, что \enquote{все люди \emph{свободны} и равны в своем достоинстве и
правах}.  Решениями СНБО без суда и следствия происходит поражение в правах
десятков украинских граждан. Без публичного предъявления каких-либо
доказательств их вины и без соответствующих решений судов.  Статья 24
устанавливает, что: \enquote{Не может быть привилегий или ограничений по признакам
расы, цвета кожи, политических, религиозных и других убеждений, пола,
этнического и социального происхождения, имущественного состояния,
местожительства, по языковым или другим признакам}.  О языковой дискриминации
мы уже говорили. Но есть и региональная, при которой пенсии тем же переселенцам
выплачиваются по более усложненной процедуре, чем остальным гражданам
%%%cit_comment
%%%cit_title
\citTitle{День Конституции Украины 28 июня - какие статьи нарушаются сильнее всего}, 
Оксана Малахова; Максим Минин, strana.ua, 28.06.2021
%%%endcit

