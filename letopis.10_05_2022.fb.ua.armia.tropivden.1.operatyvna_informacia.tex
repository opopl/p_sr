% vim: keymap=russian-jcukenwin
%%beginhead 
 
%%file 10_05_2022.fb.ua.armia.tropivden.1.operatyvna_informacia
%%parent 10_05_2022
 
%%url https://www.facebook.com/TROPIVDEN/posts/137822568827394
 
%%author_id ua.armia.tropivden
%%date 
 
%%tags 
%%title Оперативна інформація станом на 06.00 10.05.2022 щодо російського вторгнення
 
%%endhead 
 
\subsection{Оперативна інформація станом на 06.00 10.05.2022 щодо російського вторгнення}
\label{sec:10_05_2022.fb.ua.armia.tropivden.1.operatyvna_informacia}
 
\Purl{https://www.facebook.com/TROPIVDEN/posts/137822568827394}
\ifcmt
 author_begin
   author_id ua.armia.tropivden
 author_end
\fi

Оперативна інформація станом на 06.00 10.05.2022 щодо російського вторгнення

Розпочалася сімдесят шоста доба героїчного протистояння Українського народу
російському воєнному вторгненню.

Триває повномасштабна збройна агресія проти України. 

\ii{10_05_2022.fb.ua.armia.tropivden.1.operatyvna_informacia.pic.1}

Противник проводить наступальні дії на сході нашої держави з метою встановлення
повного контролю над територією Донецької та Луганської областей та утримання
сухопутного коридору між зазначеними територіями та окупованим Кримом. 

Основні зусилля авіаційного угруповання противник зосереджує на підтримці дій
підрозділів у східній операційній зоні: на Слобожанському, Донецькому напрямках
та в районі заводу \enquote{Азовсталь}.

Триває застосування артилерії практично уздовж усієї лінії зіткнення.

Зберігається висока ймовірність завдання ракетних ударів по об’єктам цивільної
та військової інфраструктури на всій території України.

Не виключається ймовірність проведення диверсій на об'єктах хімічної
промисловості України з подальшим звинуваченням у цьому підрозділів Збройних
Сил України.

На Волинському і Поліському напрямках противник активних дій не проводив. Ознак
формування наступальних угруповань не виявлено. Визначені підрозділи збройних
сил білорусії виконують завдання з прикриття українсько-білоруського кордону в
Брестській та Гомельській областях.

На Сіверському напрямку противник продовжує забезпечувати посилену охорону
ділянки українсько-російського кордону в Брянській та Курській областях.
Здійснив обстріли із застосуванням реактивних систем залпового вогню району
прикордонних населених пунктів Велика Писарівка, Білопілля, Краснопілля та
Юнаківка Сумської області.

Очікується продовження проведення демонстраційних дій противником вздовж
ділянки державного кордону України з метою сковування дій підрозділів Збройних
Сил України та недопущення їх перегрупування на інші напрямки.

На Слобожанському напрямку ворог підтримує у повній готовності до застосування
визначені сили та засоби протиповітряної оборони на території Бєлгородської
області. Продовжує зосереджувати зусилля на обороні зайнятих рубежів з метою
недопущення просування підрозділів наших військ у напрямку державного кордону
України. Проводить повітряну розвідку.

На Ізюмському напрямку противник здійснював поповнення боєприпасів, пального та
матеріально-технічних засобів.

Противник продовжує підготовку до наступальних дій в напрямках населених
пунктів Лиман та Сєвєродонецьк. 

Тривають бої за населені пункти Воєводівка, Тошківка та Нижнє Сєвєродонецького
району Луганської області, а також Кам'янка Ясинуватського району Донецької
області.

В Маріуполі ворог артилерією та ударами авіації продовжує знищувати
інфраструктуру заводу \enquote{Азовсталь}. Тривають бойові дії.

На Південнобузькому напрямку противник активних бойових дій не проводив.
Здійснював обстріли підрозділів наших військ. Веде підготовку до відновлення
штурмових дій з метою покращення тактичного положення. 

Корабельні угруповання ворога продовжують виконувати завдання з ізоляції
районів бойових дій, ведення розвідки, нанесення ракетних ударів по важливим
об’єктам цивільної та військової інфраструктури нашої держави, підтримки
підрозділів на приморському напрямку та блокування цивільного судноплавства.

За попередню добу на території Донецької та Луганської областей захисниками та
захисницями України успішно відбито п'ятнадцять атак ворога, знищено один
зенітний ракетний комплекс, 9 танків, 3 артилерійські системи, 25 одиниць
бойової броньованої техніки, 3 одиниці спеціальної інженерної техніки та 3
автомобіля ворога.

Віримо у Збройні Сили України! Разом до перемоги! 

Слава Україні!

\href{https://www.facebook.com/GeneralStaff.ua}{Генеральний штаб ЗСУ/General Staff of the Armed Forces of Ukraine}

