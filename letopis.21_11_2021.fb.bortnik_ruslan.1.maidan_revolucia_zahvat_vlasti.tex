% vim: keymap=russian-jcukenwin
%%beginhead 
 
%%file 21_11_2021.fb.bortnik_ruslan.1.maidan_revolucia_zahvat_vlasti
%%parent 21_11_2021
 
%%url https://www.facebook.com/ruslan.bortnik.3/posts/10161448424299689
 
%%author_id bortnik_ruslan
%%date 
 
%%tags maidan2,obschestvo,politika,revolucia,ukraina
%%title Майдан - революция и захват власти
 
%%endhead 
 
\subsection{Майдан - революция и захват власти}
\label{sec:21_11_2021.fb.bortnik_ruslan.1.maidan_revolucia_zahvat_vlasti}
 
\Purl{https://www.facebook.com/ruslan.bortnik.3/posts/10161448424299689}
\ifcmt
 author_begin
   author_id bortnik_ruslan
 author_end
\fi

Майдан начинался как политическая акция, шел как Революция, а закончился
неконституционным и примитивным захватом власти. 

Поэтому эти события и оценивают по-разному.  

Не каждый успешный Майдан надо называть "революцией" - посмотрите на Латинскую
Америку или Африку.

Революция - это немного другое - это полная смена элит, принципов и механизмов
функционирования государства, отношений между обществом и государством, а не
просто иерархии в верхах ...

Поэтому это просто "Майдан". Классический. Украинский. Когда внизу - бунт,
посредине  - политика, а вверху - коррупция. 

Политики пришедшие к власти на волне Майдана на десятилетия ввергли страну в
политический  и правовой хаос, конфликт и войну, бедность, депопуляцию и
десуверенизацию, сохранили "коррупционные" достижения предшественников;
практически уничтожили перспективы построение правого Государства и
объединённого, благополучного Общества.

\begin{cmtfront}
\uzr{Sergey Kuchinsky}

\begin{multicols}{2}
\obeycr
Чи щось удалось за тих 8 років?
Країна ж не всі ті роки змарнувала?
Натикано купу прутнІв-прапорів,
Бандера, найбільші тризуби із сала...
\smallskip
Від поступу цього у душах бринить
Та щастя неповне, така вже природа.
Бо поки рагуль на Бандеру трочить,
Сусіди будують міста та заводи.
\restorecr
\end{multicols}
	
\end{cmtfront}

\ii{21_11_2021.fb.bortnik_ruslan.1.maidan_revolucia_zahvat_vlasti.cmt}
