% vim: keymap=russian-jcukenwin
%%beginhead 
 
%%file 28_01_2022.fb.ljamec_sergej.1.verhovna_rada_nedeesposobna
%%parent 28_01_2022
 
%%url https://www.facebook.com/anserua/posts/5822400491110206
 
%%author_id ljamec_sergej
%%date 
 
%%tags politika,ukraina,verhovna_rada
%%title Верховная Рада - недееспособна
 
%%endhead 
 
\subsection{Верховная Рада - недееспособна}
\label{sec:28_01_2022.fb.ljamec_sergej.1.verhovna_rada_nedeesposobna}
 
\Purl{https://www.facebook.com/anserua/posts/5822400491110206}
\ifcmt
 author_begin
   author_id ljamec_sergej
 author_end
\fi

Верховная Рада - недееспособна.

Это очевидно для тех, кто наблюдает за парламентом. Но большинству населения
нужно доказывать. Ок, вот доказательство.

Сегодня Верховная Рада заслушала Кабинет министров.

Чего можно было ожидать? Парламент гулял почти месяц, и собирается гулять еще
полмесяца. Логично, что Рада как верховный орган власти в Украине обязан
заниматься самыми важными для страны вопросами.

\ii{28_01_2022.fb.ljamec_sergej.1.verhovna_rada_nedeesposobna.pic.1}

Что за вопросы? Нас активно пугают нападением России - раз. В Украине
энергетический кризис и негде брать газ - два. Безумная инфляция, в том числе
за счет роста мировых цен - три. Это лишь острые вызовы, а есть еще и
хронические болячки. Как минимум, три повода серьезно спросить с правительства,
готово ли оно к вызовам. Собственно для этого и задуман формат - доклад Кабмина
перед парламентом.

Что оказалось по факту: Верховная Рада опять сделала Кабмину \enquote{теплую ванную}.
Депутаты позволили чиновникам из Кабмина мирно выступить - и спокойно идти
\enquote{работать} дальше. Ничего при этом не выяснив. Ни по войне, ни по экономике, ни
вообще.

Дело не в том, что не хватает депутатов. Голосов как раз достаточно. За закон о
национальном сопротивлении, например, сегодня проголосовали 278 депутатов. В
том числе, 198 \enquote{слуг} из 203 зарегистрировавшихся. Остальное понемного домазали
депутаты. Статистика говорит, что \enquote{депутаты болеют} и \enquote{депутаты в отъезде} -
просто отговорка.

Вопрос в другом: парламент не является самостоятельным.

В 2019 году, когда в Раде засело монобольшинство, нас это даже радовало.
Существовали ожидания, что вот сейчас начнем жить по-другому. Президент новый и
свой, парламент новый и его.

Розовые очки спали уже через полгода, а окончательно - спустя пару лет.
Владимир Зеленский оказался для избирателей чужим. А вот парламент остался
подчинен президенту и Офису президента - органу власти с крайне ограниченным
уровнем разумности. Так мы получили уродливую конструкцию.

В силу естественных причин, Офис президента не может родить ни одной путной
идеи. А парламент - идет в фарватере политического слабоумия. При этом Кабмин
тоже подчинен Офису и ничего путного не может предложить. Зато все друг друга
покрывают.

Замкнутый круг, который нам \enquote{повезло} наблюдать в то самое время, когда
российские войска стоят на границе, а в мире пахнет новым финансовым кризисом.

Этот узел напоминает знаменитый гордиев. Его уже нельзя распутать, настолько
тесно сплелись между собой все эти персонажи. Его можно только разрубить. В
законный спосіб - но безжалостно и решительно. Иначе не будет вообще ничего,
одно болото имени Офиса президента.
