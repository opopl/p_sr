% vim: keymap=russian-jcukenwin
%%beginhead 
 
%%file 19_12_2021.fb.fb_group.story_kiev_ua.1.detstvo_shkola.pic.1.cmt
%%parent 19_12_2021.fb.fb_group.story_kiev_ua.1.detstvo_shkola
 
%%url 
 
%%author_id 
%%date 
 
%%tags 
%%title 
 
%%endhead 

\iusr{Надежда Владимир Федько}
Типова шкільна форма мого дитинства.

\iusr{Володимир Луць}
Така шкільна форма у хлопців була до 64-го.

\ifcmt
  ig https://scontent-frx5-2.xx.fbcdn.net/v/t39.30808-6/268822123_1660549004337126_4213811849285814425_n.jpg?_nc_cat=109&ccb=1-5&_nc_sid=dbeb18&_nc_ohc=86S7gVD0-3gAX8UOo1-&_nc_ht=scontent-frx5-2.xx&oh=00_AT_HGT8MLHphotyxpIwIOw-2ThrYpzYOIUWUsVC6UjnUjQ&oe=61C808DC
  @width 0.3
\fi

\iusr{Михайло Наместник}
\textbf{Володимир Луць} но были и ремни

\iusr{Володимир Луць}
\textbf{Михайло Наместник} Ремінь йшов з гімнастьоркою. А ця вважалась парадною.

\iusr{Михайло Наместник}
\textbf{Володимир Луць} це для мене загадка, бо гімнастьорки я не пам'ятаю, а свій ремінь був точно. Буду шукати фото тих літ.

\iusr{Володимир Луць}
\textbf{Михайло Наместник} Доречі на першій світлині пряжка ременя розвернута догори.

\iusr{Юлия Кулинская}
В моєї мами є точно таке ж фото, я аж збільшила, чи це не вона.

\iusr{Квітницька Ганна}
Портфели тонковаты, по сравнению с нынешними)))

\iusr{Михайло Наместник}
може і тонкуваті, але в мене й досі праве плече нижче за ліве  @igg{fbicon.frown} 

\iusr{Надежда Владимир Федько}
\textbf{Квітницька Ганна} У першому - другому класах підручників було мало) Буквар, щоденник, пара трійка зошитів...
Ну, я ще носив книжку, яку читав...
