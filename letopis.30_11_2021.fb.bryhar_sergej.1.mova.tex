% vim: keymap=russian-jcukenwin
%%beginhead 
 
%%file 30_11_2021.fb.bryhar_sergej.1.mova
%%parent 30_11_2021
 
%%url https://www.facebook.com/serhiibryhar/posts/1845823818950895
 
%%author_id bryhar_sergej
%%date 
 
%%tags identichnost',jazyk,mova,ukraina
%%title Отже, як не крути, а ключовий чинник - мова
 
%%endhead 
 
\subsection{Отже, як не крути, а ключовий чинник - мова}
\label{sec:30_11_2021.fb.bryhar_sergej.1.mova}
 
\Purl{https://www.facebook.com/serhiibryhar/posts/1845823818950895}
\ifcmt
 author_begin
   author_id bryhar_sergej
 author_end
\fi

"Єсть сєло, а єсть люді. Сєло і люді - ето нє одно і то же. І іх как-то вообщє
нє жалко", - (розмовляючи з якимись невідомими мені особами) говорить знайомий
чоловік, якому близько 60 років, син видатного одеського науковця, вихідця із
"срєднєй полоси Расєі".

\ifcmt
  ig https://scontent-lhr8-2.xx.fbcdn.net/v/t39.30808-6/261140354_1845823788950898_4867470227232057717_n.jpg?_nc_cat=101&ccb=1-5&_nc_sid=8bfeb9&_nc_ohc=lpWBLJN1g98AX_c8Ang&_nc_ht=scontent-lhr8-2.xx&oh=fcb2afb2d52e4ac53fe60deb4cdf12af&oe=61B43CC8
  @width 0.4
  %@wrap \parpic[r]
  @wrap \InsertBoxR{0}
\fi

Ні, я не став заперечувати. Ми давно вже майже ні про що не розмовляємо. Ми
ідейні вороги.

Я для нього - "сєло з тєлячьєй мовой". Він для мене - огидний наслідок
слабкості української держави.

Власне, з його нащадками, тобто фактично моїми ровесниками, мені теж немає, про
що говорити. Це вже третє покоління ідейних ворогів українства.

Але я собі от що подумав: "Московити в першій половині ХХ століття і московити
в першій половині століття ХХІ-го - одне і те ж! Вони не змінюються. Але одна
справа, що вони не змінюються там, у себе - це логічно, і зовсім інша - що вони
не змінюються тут, в Україні... Думаю, якби обставини склалися відповідним
чином, ті, що там, дійсно могли б і повторити жахіття 1932-1933 років чи новий
1937-й. А ці, що тут, могли б і докластися... І от що цікаво, ми, люди зі
стійкою національною ідентичністю, для них теж "сєло, которого нє жалко". Ми,
всі оці "аборигени", в їхньому розумінні, маємо або стати росіянами, або...
може і припинити існування. От вам, якщо коротко, то і весь світогляд.

%\ii{30_11_2021.fb.bryhar_sergej.1.mova.cmtfront.1}
\ii{30_11_2021.fb.bryhar_sergej.1.mova.scr.1}


Але вони такі не тому що зроблені з "особливого тіста", а тому що українська
держава для них схожа на теплу ванну. Вона не примушує ставати громадянами!
Вона не вимикає їхнє шовіністичне російське "живлення", і не підключає до
свого...

Замість висновку маю може й дещо утопічні, але загалом достатньо достатньо
важливі думки.  

Напевно, хтось скаже: "про що ти взагалі? - поглянь-но, що коїться на верхівці
влади". Та бачу... Але ж я дивлюся на історичний процес у широкій перспективі,
тому й розумію, що "шостий" все одно буде викинутий на звалище історії, в
компанію до "четвертого", а ось далі дуже важливо знати, що робити з людьми,
які мешкають в Україні, і ненавидять її як окрему державу, вважають частиною
Росії; та й з тими, хто хоч і тихо, але живе у просторі російських наративів -
теж.

Зрештою, не можна наступати на ті ж самі граблі... Нічого не робити - віддати
ініціативу ворогу!

В Європі є держави, досвід яких нам може бути дуже навіть корисний.
Псевдограмадяни (ті, хто хочуть такими бути) не повинні мати ті ж права, що й
громадяни справжні (а фактично, в нашій погано реформованій УРСРівській
реальності оті псевдограмадяни почуваються навіть краще, ніж справжні
громадяни).

Паспорти негромадян? Так. І я переконаний, що вони би стали не стільки
реальністю, скільки "страшилкою" для кожного "русского", який хоче жити і
розвиватися, а не боротись за ідеали "русского міра", жертвуючи перспективами.

Пам'ятаю, один дивний, як на мене, естонський футболіст, який грав на своєму
рівні дуже і дуже непогано, але відмовлявся здавати іспит з естонської мови і,
відповідно, отримувати нормальний паспорт (для місцевої вати це, звісна річ,
особистість культова), через що і не зіграв жодного матчу за національну
збірну, в кінці одного з інтерв'ю заявив: "мої діти навчаються і розмовляють
естонською, тому в них немає жодних проблем".

От бачите, як просто - все зав'язується саме на мові. Іспит - це і є відбір.
Скоріше навіть символічний, але не лише. Ну а загалом, звісно ж, держава має
створити такі умови, щоби реалізовуватися без використання державної мови було
неможливо. 

Може когось уже і не зміниш, але в них виростуть діти, які не стануть
жертвувати своїми перспективами. 

Отже, як не крути, а ключовий чинник - мова. І нікуди від цього не підеш. Хіба
що до Москви...

\ii{30_11_2021.fb.bryhar_sergej.1.mova.cmt}
