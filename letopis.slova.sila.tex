% vim: keymap=russian-jcukenwin
%%beginhead 
 
%%file slova.sila
%%parent slova
 
%%url 
 
%%author 
%%author_id 
%%author_url 
 
%%tags 
%%title 
 
%%endhead 
\chapter{Сила}
\label{sec:slova.sila}

%%%cit
%%%cit_head
Россия-Сила
%%%cit_pic
%%%cit_text
Друзья, сегодня, в единственный летом большой и значимый праздник - День России
- давайте говорить о России. Нашей огромной, любимой, иногда проблемной, но всё
равно очень и очень привлекательной стране.  Аналитику, сухие цифры - в
сторону. Или совсем чуть-чуть. Интересна общая картина огромной,
\emph{сильной}, значимой на планете державы и в то же время тихой и застенчивой
Руси
%%%cit_comment
%%%cit_title
\citTitle{Самая-самая Россия: северная, холодная, загадочная, непобедимая}, 
Крымские Посидушки, zen.yandex.ru, 12.06.2021
%%%endcit

%%%cit
%%%cit_head
%%%cit_pic
%%%cit_text
Влада не спроможна добити \enquote{п'яту колону} Медведчука, змінити правила гри в
політикумі, що призводить до самовідтворення політичної корупції та влади
олігархата. Опозиція не в стані підняти загальноукраїнський протест. Навіть
нормально організувати збір підписів за референдум не змогли. Я вже не каджу
про виведення людей на вулицю проти тарифів, зниження рівня життя,
антисоціальних законів тощо. Суцільна імпотенція парламентських партій.  А
політичний процес не терпить пустот та імпотенції. Досить швидко з'явиться
\emph{сила}, яка пред'явить своє право на владу, або візьме її \emph{силою}
%%%cit_comment
%%%cit_title
\citTitle{Маніфест реаліста або час збирати каміння. Воно нам скоро знадобиться - ХВИЛЯ}, 
Виталий Кулик, analytics.hvylya.net, 17.06.2021
%%%endcit

На свете существует только две \emph{силы}: доллары и литература (М.А. Булгаков, «Зойкина квартира»)

