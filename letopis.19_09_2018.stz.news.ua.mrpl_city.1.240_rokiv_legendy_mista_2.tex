% vim: keymap=russian-jcukenwin
%%beginhead 
 
%%file 19_09_2018.stz.news.ua.mrpl_city.1.240_rokiv_legendy_mista_2
%%parent 19_09_2018
 
%%url https://mrpl.city/blogs/view/do-240-richchya-mariupolya-legendi-mista-chastina-ii
 
%%author_id demidko_olga.mariupol,news.ua.mrpl_city
%%date 
 
%%tags 
%%title До 240-річчя Маріуполя: Легенди міста. Частина II.
 
%%endhead 
 
\subsection{До 240-річчя Маріуполя: Легенди міста. Частина II.}
\label{sec:19_09_2018.stz.news.ua.mrpl_city.1.240_rokiv_legendy_mista_2}
 
\Purl{https://mrpl.city/blogs/view/do-240-richchya-mariupolya-legendi-mista-chastina-ii}
\ifcmt
 author_begin
   author_id demidko_olga.mariupol,news.ua.mrpl_city
 author_end
\fi

\ii{19_09_2018.stz.news.ua.mrpl_city.1.240_rokiv_legendy_mista_2.pic.1}

\begin{quote}
\em Продовжуючи знайомство з легендами Маріуполя, пропоную поринути в загадкове минуле нашого міста...	
\end{quote}

Представлені легенди знайдені у розвідках визначного краєзнавця Аркадія
Проценка та зібрані завдяки розповідям маріупольців.

\begin{center}
\textbf{Назва міста}
\end{center}

Назву Маріуполя часто пов'язують з морем: \enquote{марі} – це море, а \enquote{поль} – \enquote{поліс},
тобто \enquote{місто}. І виходить – \enquote{місто біля моря}. Але згідно з легендою, яка
виникла на самому початку XIX століття, назву \enquote{Маріуполь} греки принесли з
Криму. Раз в місті оселилися греки, значить, і назву місту дали вони, вважали
багато авторів, які писали про місто. У Бахчисараї є передмістя Майрем
(Маріам), де розташовувався Успенський монастир, реліквією якого була ікона
Одигітрія – Богородиці Марії з немовлям Христом. Походження назви передмістя
Бахчисарая Майрем трактується так: передмістя названо в честь Богородиці Марії,
оскільки монастир Успіння Богородиці присвячений саме їй. Або названий він на
честь ікони Марії-Богородиці з немовлям Христом.

\ii{19_09_2018.stz.news.ua.mrpl_city.1.240_rokiv_legendy_mista_2.pic.2}

Укладачі цієї легенди вважали, що якщо Маріуполь заснували саме жителі селища
Майрем, то і назва міста Маріуполя виникла на честь Марії-Богородиці, або на
честь ікони Матері Одигітрії.

\textbf{Читайте також:} \emph{В честь кого назван наш город?}%
\footnote{В честь кого назван наш город?, Сергей Буров, 04.08.2018, mrpl.city, \url{https://mrpl.city/blogs/view/v-chest-kogo-nazvan-nash-gorod}} %
\footnote{Internet Archive: \url{https://archive.org/details/04_08_2018.sergij_burov.mrpl_city.v_chest_kogo_nazvan_nash_gorod}}

\begin{center}
\bfseries Будинок митрополита Ігнатія	
\end{center}

Є в Маріуполі одна примітна будівля на розі вулиць Торгової та Митрополитської
(точна адреса: Митрополитська, 7). Цей невеликий, в півтора поверхи, будинок,
швидше за все, не привертає до себе увагу.

\ii{19_09_2018.stz.news.ua.mrpl_city.1.240_rokiv_legendy_mista_2.pic.3}

Звичайно, це не та споруда, яка значиться в списках пам'яток архітектури, але
це не означає, що вона не заслуговує на увагу. Коли підходиш до будинку ближче,
то мимоволі починаєш передчувати якусь історію за цими стінами. І передчуття не
обманює. З цим будинком пов'язані цікаві сторінки історії Маріуполя.

Наприкінці XVIII століття будинок належав митрополиту Ігнатію з роду
Гозадінових, він є найдавнішою будівлею в нашому місті. Звідси в народі і
отримала назву вулиця. Митрополит прожив в будинку сім років. Після його смерті
в 1786 році будинок перейшов його спадкоємцям.

\textbf{Читайте також:} \emph{К 240-летию Мариуполя: перекресток на Торговой}%
\footnote{К 240-летию Мариуполя: перекресток на Торговой, Сергей Буров, 08.09.2018, mrpl.city, \url{https://mrpl.city/blogs/view/k-240-letiyu-mariupolya-perekrestok-na-torgovoj-1}} %
\footnote{Internet Archive: \url{https://archive.org/details/08_09_2018.sergij_burov.mrpl_city.k_240_letiu_mariupolja_perekrestok_na_torgovoj}}

\begin{center}
\bfseries Віддані грифони	
\end{center}

\ii{19_09_2018.stz.news.ua.mrpl_city.1.240_rokiv_legendy_mista_2.pic.4}

За однією з легенд зруйнований маєток на вулиці Італійській не зміг знайти
нових господарів через привид Петра Петровича Регира, якого колись оберігали
\enquote{грифони}. Їх можна побачити і зараз над вікнами другого поверху. Ці крилаті
міфологічні істоти повинні були слугувати не тільки прикрасою будинку, але й
оберігати золото свого господаря. Але, на думку маріупольців, саме грифони не
відпустили свого видатного господаря, який не покинув маєток разом із сім’єю.
Він залишився, незважаючи на існуючу небезпеку, приниження від більшовиків і
загадкову смерть. Цікаво, що після смерті Петра Регира наступні господарі у
маєтку довго не затримувалися, поки він не перетворився на руїни. Один
маріупольський старожил вважає, що в цьому винні саме крилаті грифони, які
залишилися віддані Петру Петровичу, хоча й не змогли його врятувати.

\textbf{Читайте також:} \emph{Маріупольський особняк з грифонами}%
\footnote{Маріупольський особняк з грифонами, Ольга Демідко, mrpl.city, 28.08.2017, \url{https://mrpl.city/blogs/view/mariupolskij-osobnyak-z-grifonami} } %
\footnote{Internet Archive: \url{https://archive.org/details/28_08_2017.olga_demidko.mrpl_city.mariupol_osobnjak_z_grifonami}}

\begin{center}
\bfseries Загадковий Куїнджі	
\end{center}

Весь образ уродженця Маріуполя, художника А. І. Куїнджі не вписувався в рамки
звичних уявлень: і манера, і прийоми живопису, і поведінка, і зовнішність – все
вражало. Чи не випадково навколо його імені було багато таємничого,
загадкового... Люди вірили в особливу силу цієї людини, складали про нього
легенди.

Його творча оригінальність органічно виходила з непересічності його натури. Він
залишався для сучасників загадкою, і спроби розгадати її, як правило,
поступалися місцем найнеймовірнішим припущенням.

Слава Куїнджі розрослася настільки швидко, що заздрісники стали поширювати
чутки, нібито він – колишній маріупольський ретушер-самозванець, який після
смерті справжнього художника викрав його твори.

Сьогодні часом буває важко відрізнити дійсні події його життя від вигаданих.
Поширенню домислів сприяли також суперечливість особистості живописця і його
вчинків.

За твердженням друзів художника, йому не було чужим почуття справедливості,
співчуття до слабких, любов до талановитих, прагнення до повної незалежності.
Однак загальний характер спогадів та свідчень про нього суперечливий, інколи
вони виключають одне одного. Життя і творчість Куїнджі так суперечливо
сприймалися сучасниками, що, здається, мова йде про різних людей.

\ii{19_09_2018.stz.news.ua.mrpl_city.1.240_rokiv_legendy_mista_2.pic.5}

Так виникла легенда про двох Куїнджі. Вона склалася ще за життя художника як в
колі друзів, так і недругів. У ній протиставлялися Куїнджі-людина і
Куїнджі-художник.

Куїнджі – просто людина, практичний грек з Маріуполя, якого бідність привчила
відмовляти собі у всьому, а Куїнджі-художнику успіх дав змогу практично
здійснити мрію, мати великі статки і стати незалежним.

Цікавий парадокс: Куїнджі скуповував землі, будинки і чекав, коли вони зростуть
в ціні, а потім дарував їх. З боку це виглядало, як нажива. Так народилася
легенда про його скупість, а він був абсолютно позбавлений цього недоліку.
Наприклад, анітрохи не жалкуючи про це, він подарував суспільству художників
свій маєток на південному березі Криму, поблизу станції Кікінеїз, який
оцінювався в значну суму.

Сучасники з подивом дивилися на те, як він соромливо вручав невідомому потрібну
тому суму грошей. При цьому на питання чи знайомі вони, він відповідав, що ні,
але ж гроші незнайомцю потрібніші...

\begin{center}
\textbf{Загублене дзеркало}
\end{center}

Існує легенда, що на старовинному цвинтарі Маріуполя, яке вважається недіючим,
на одному з пам'ятників чи склепів заховано унікальне і заповітне дзеркало.
Варто в нього подивитися людині похилого віку і до неї повернуться втрачені
роки, а разом з ними - надія на спробу виправити скоєні помилки. Також за
легендою завдяки дзеркалу можна повернути кохану людину, втрачений успіх і
здійснити будь-яке бажання. Але, незважаючи на багато спроб маріупольців знайти
загадкове дзеркало, дотепер воно залишається лише легендою...

\textbf{Читайте також:} \emph{Таємниці старовинного маріупольського цвинтаря}%
\footnote{Таємниці старовинного маріупольського цвинтаря, Ольга Демідко, mrpl.city, 20.11.2017, \url{https://mrpl.city/blogs/view/taemnitsi-starovinnogo-mariupolskogo-tsvintarya}} %
\footnote{Internet Archive: \url{https://archive.org/details/20_11_2017.olga_demidko.mrpl_city.tajemnyci_cvyntar}}
