% vim: keymap=russian-jcukenwin
%%beginhead 
 
%%file 21_05_2021.fb.buzhanskii_max.1.volnosti_nacia
%%parent 21_05_2021
 
%%url https://www.facebook.com/permalink.php?story_fbid=1949860681845068&id=100004634650264
 
%%author 
%%author_id buzhanskii_max
%%author_url 
 
%%tags nacia,nacidea,ukraina,vyshyvanka
%%title Код нашей нации, никакая не вышиванка, а "вольности"
 
%%endhead 
 
\subsection{Код нашей нации, никакая не вышиванка, а \enquote{вольности}}
\label{sec:21_05_2021.fb.buzhanskii_max.1.volnosti_nacia}
\Purl{https://www.facebook.com/permalink.php?story_fbid=1949860681845068&id=100004634650264}
\ifcmt
 author_begin
   author_id buzhanskii_max
 author_end
\fi

Теперь, когда все уже красиво сфотографировались и не менее красиво высказались, можно сказать пару слов неприятной правды.

Не нужно, но можно.

Код нашей нации, никакая не вышиванка, а \enquote{вольности}.

Полное и солидарное от Запорожья до Львова презрение к закону и порядку.

А вышиванка, безусловно красивая и исторически достоверная рубаха, призванная это презрение к законности хоть чего-либо как то скрыть.

От нас самих.

Согласитесь, гордиться вышиванкой намного приятней, чем гордиться вечным
хаосом.

Raisa Prisetska

ГОрдитися вишитою сорочкою не варто. Факт. Бо вона є або її немає. Як власне
коріння.Як мамина любов в тій заполочі. Як память поколінь. "роль" вишиванки не
в "сокритіїї" законності. Якщо це незрозуміло, це не проблема історичної
вишитої сорочки. І Вишиванка не відповідає за тих манкуртів, які використовують
її саме з такою метою!

Олександр Щербаков

код нации-это ее историческая миссия.Украинцы еще не осознали,зачем существует
Украина и какова ее миссия.

Виталий Триль

Если законы, написанные и принятые людьми , получившими такое право в
результате предвыборного обмана граждан , не способствуют развитию большинства
а являются инструментом сохранения власти кучки проходимцев то вполне
естественно что такие законы граждане будут пытаться игнорировать . Имея
примером пренебрежение законами со стороны представителей власти . В Украине
люди презирают не закон а представителей закона .

Сергей Удовик

Вольности у нас обусловлены пограничьем, что буквально и обозначает слово
украина, и психология пограничья это бегство от государства. Со свободой это
мало связано, поскольку свобода подразумевает ответственность, и ей надо
учиться

Александр Диденко

Ввшиванка, однозначно, красивая! Мы, украинцы, очень наивные и романтичны! Но,
вчера наблюдал, как сгоняли студентов на марш вышиванок в Харькове, а
отметившись, они тут же ныряли в ближайшую станцию метро. Ничего не поменялось
в методах!

Александр Шибанов

Просторная рубаха с орнаментом- древняя одежда практически ВСЕХ
народов.Уникальности не вижу.Уникальность вижу в
Амосове,Королеве,Бубке,Патоне!А вольности?Ну так мы же везде кричим о козацьком
роде.А это кто- несогласные с властью да беглые каторжники!

Михаил Журавлев

А взяли и запретили русский язык

Тамара Логинова

Когда больше нечем гордиться..

Андрей Васильович

Вышиванку уже вознесли и обесценили!!! Это уже как паспорт без печати,нет
вышиванки и ты уже не украинец!!! А то что это должно идти из души,так то
такэ!!! Началось уже у кого дороже,так как у кого что длинней,,,,, А взять и
похвастаться,что сам,сама вышила,так это тоже такэ!!! Печалька(

Виктор Михайлов

В точку, полностью согласен, но хочу добавить, что постоять друг за дружку в
беде тоже никогда не было желания...

Eugene Demenok

Все буде вишиванка!:)

Eugene Demenok

Если бы кто-нибудь сказал, например, французам, что генетический код их нации -
какая-то вышитая рубашка, интересно было бы посмотреть на реакцию.

Леся Куруч

Евгений Деменок Они бы просто не поняли. Бред необъясним.

Владимир Глядченко

Який ще Марс, нам - упоротим нацистам важніше вишиванка, борщ і щоб кацапам
погано жилося. Це ж національна ідея - краще ми в лайні жити будемо, тільки щоб
сусідам насолити

Наталия Литвиненко
Сегодня это стало фетишем, выставляется напоказ, словно для того, чтобы клеймить друг друга нам, а это неправильно и губит наши души.
Мы сами, только сами, своими руками себя уничтожаем, своей ненавистью и злобой, ищем врагов не там, не уважаем друг друга- в этом и есть главная печаль.
Не замечаем, что позволяем злу поселиться в наших сердцах и скрыться под масками лицемерия и фальши.
Света и Добра всем нам.
