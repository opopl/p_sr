% vim: keymap=russian-jcukenwin
%%beginhead 
 
%%file slova.konstitucia
%%parent slova
 
%%url 
 
%%author 
%%author_id 
%%author_url 
 
%%tags 
%%title 
 
%%endhead 
\chapter{Конституция}
\label{sec:slova.konstitucia}

%%%cit
%%%cit_head
%%%cit_pic
%%%cit_text
Почему защита русского языка - лишь пустая декларация в \emph{Конституции}?
Почему свободное использование, развитие и защита русского языка в Украине —
это лишь пустая декларация в \emph{Конституции Украины}?  Почему у Зеленского
хотят ввести уголовную ответственность за оскорбление украинского языка, но на
оскорбление русского языка у Зеленского всем плевать?  Почему нам —
украинизация всех сфер общественной жизни, а \enquote{Кварталу 95} и сериалам
Зеленского — спецзакон, позволяющий русский язык?  Есть ли проблемы с русским
языком и гуманитарными правами в Украине и какой статус для русского языка
нужен в Украине?
%%%cit_comment
%%%cit_title
\citTitle{Почему на оскорбление русского языка у Зеленского всем плевать?}, 
Александр Скубченко, strana.ua, 21.06.2021
%%%endcit

%%%cit
%%%cit_head
%%%cit_pic
%%%cit_text
В політичній сфері, мусимо розпочати процес інклюзивної Конституанти. Бо в
існуючому політико-правовому дизайні Республіки не збудуєш. Так само варто
говорити про відповідальне громадянство і нові принципи побудови управління в
країні. Ми маємо нарешті розірвати пуповину з пострадянською спадщиною у
вигляді \emph{Конституції} 1996 року та ухвалити новий Основний Закон!  Хто зробить
все це? Нові політичні суб'єкти.  І не потрібно мені тикати під ніс \enquote{зеленими}
та різними потойбічними \enquote{голосами}.
Існуючі політичні гравці в ситуації кризи та потужних викликів демонструють
повне ідейне та організаційне банкрутство
%%%cit_comment
%%%cit_title
\citTitle{Маніфест реаліста або час збирати каміння. Воно нам скоро знадобиться - ХВИЛЯ}, 
Виталий Кулик, analytics.hvylya.net, 17.06.2021
%%%endcit


%%%cit
%%%cit_head
%%%cit_pic
%%%cit_text
Сторонники введения этого механизма говорят, что в больницах и так идут
огромные теневые поборы с пациентов, и лучше, чтобы эти средства платились
легально. Видимо, опции существенно поднять зарплаты медикам и потом вынудить
их прекратить поборы с помощью правоохранительной системы - власти даже не
рассматривают. И готовят решения, которые прямо нарушают \emph{Конституцию}.  К слову,
о медицине косвенно говорит еще и 22 статья. Там написано, что при утверждении
новых законов не допускается сужение содержания и объема существующих прав и
свобод. Однако медицинская реформа урезает такие права. Как, собственно, и
пенсионная реформа. Да и языковая
%%%cit_comment
%%%cit_title
\citTitle{День Конституции Украины 28 июня - какие статьи нарушаются сильнее всего}, 
Оксана Малахова; Максим Минин, strana.ua, 28.06.2021
%%%endcit

%%%cit
%%%cit_head
%%%cit_pic
%%%cit_text
С Праздником друзья! Желаю всем, чтобы когда-то \emph{Конституция} стала действующим
Законом, а не мечтой и символом!  Но, пока-что, мы в Уип Уип проанализировали и
насчитали, что из 161 статьи \emph{Основного Закона Украины} на сегодня системно и
последовательно нарушаются 125 - около 78\%. 36 статьи \emph{Основного Закона},
которые \enquote{условно выполняются}, содержат в основном номинальные или отсылочные
нормы. Из первых двух разделов \emph{Конституции} полностью выполняется только статья
20 \enquote{Государственные символы}.  \emph{Конституция} - это очень тонкий инструмент, и
если хотя бы одна её статья цинично и публично нарушается, то это ставит под
угрозу разрушения весь документ
%%%cit_comment
%%%cit_title
\citTitle{Из 161 статьи Конституции Украины системно нарушаются 125 / Лента соцсетей / Страна},
Руслан Бортник, strana.ua, 28.06.2021
%%%endcit

%%%cit
%%%cit_head
%%%cit_pic
%%%cit_text
А уже в 2014-м на волне хаоса и беззакония нардепы «вернули» \emph{Конституцию}
образца 2004-го постановлениями парламента. Затем они избавились от тех судей
КСУ, которых могли уволить по своей квоте, а ГПУ дали указание преследовать
всех, кто имел отношение к Решению КСУ о возврате к \emph{Конституции} образца 96-го
года.  Так «революционеры» сперва надругались над \emph{Основным Законом}, а после
растоптали его. И это логично! Если оценивать акты «майданного» периода на
предмет \emph{конституционности}, то выяснится, что решения, принятые незаконно
назначенными руководителями, не являются легитимными, а многих «уважаемых»
людей следует немедля отправить в барак
%%%cit_comment
%%%cit_title
\citTitle{Ранее, чтобы выглядеть прилично, власть меняла Конституцию / Лента соцсетей / Страна}, 
Максим Могильницкий, strana.ua, 28.06.2021
%%%endcit

