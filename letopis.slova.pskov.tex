% vim: keymap=russian-jcukenwin
%%beginhead 
 
%%file slova.pskov
%%parent slova
 
%%url 
 
%%author 
%%author_id 
%%author_url 
 
%%tags 
%%title 
 
%%endhead 
\chapter{Псков}

%%%cit
%%%cit_pic
%%%cit_text
Вот теперь что-то становится понятно. Так как все потомки Рюрика, и вообще все
норманны называли себя русами, то и получается, что Владимир был русским
князем, а вовсе не каким-то там украинским. Так что он на 100\% был
Рюриковичем, то есть вообще не славянином, а прямым потомком людей, пришедших
откуда-то с севера.  На закуску следует теперь выяснить, был ли на самом деле
Владимир киевским князем. Известно, что он родился под \emph{Псковом}, куда княгиня
Ольга отослала беременную от его сына Святослава сноху, бывшую в то время ее
приближенной, то есть фавориткой
%%%cit_title
\citTitle{Почему киевский хан Владимир Креститель считается украинским князем, когда он даже славянином не был?},
Исторический Понедельник, zen.yandex.ru, 05.01.2021 
%%%endcit

