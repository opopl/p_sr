% vim: keymap=russian-jcukenwin
%%beginhead 
 
%%file 31_12_2013.fb.bilchenko_evgenia.1.doneck_shahter_ermitazh
%%parent 31_12_2013
 
%%url https://www.facebook.com/yevzhik/posts/610177332350714
 
%%author Бильченко, Евгения
%%author_id bilchenko_evgenia
%%author_url 
 
%%tags bilchenko_evgenia,donbass,doneck,ukraina
%%title Шахтерский Эрмитаж: Донецк, который меня удивил
 
%%endhead 
 
\subsection{Шахтерский Эрмитаж: Донецк, который меня удивил}
\label{sec:31_12_2013.fb.bilchenko_evgenia.1.doneck_shahter_ermitazh}
\Purl{https://www.facebook.com/yevzhik/posts/610177332350714}
\ifcmt
 author_begin
   author_id bilchenko_evgenia
 author_end
\fi

Шахтерский Эрмитаж: Донецк, который меня удивил.

Вот и состоялись мои поэтические гастроли в Донецке. Однако все по порядку. По
приглашению Ани Ревякиной, которая великолепно выполнила функции арт-куратора,
несмотря на то, что у нее самой была премьера спектакля по ее поэтической
пьесе, я выступила в камерном зальчике одного из донецких арт-кафе перед
молодой творческой публикой города, среди которой, однако, были не только юные
слушатели, но и не чуждые авангарду люди постарше.


\ifcmt
  tab_begin cols=3
  width 0.3

     pic https://scontent-lga3-1.xx.fbcdn.net/v/t1.18169-9/1525483_610175802350867_438607498_n.jpg?_nc_cat=111&ccb=1-3&_nc_sid=730e14&_nc_ohc=EmSV4xPvRagAX-9GZpu&_nc_oc=AQmmJtXlATiwbo5rtvYA-dKEYjPHIvVXmZVOd7ySV9On0M1UrJugVROuexUJitpjcbQ&_nc_ht=scontent-lga3-1.xx&oh=50974632befe671c009f841bee8b3365&oe=60ED1E9F

     pic https://scontent-lga3-1.xx.fbcdn.net/v/t1.18169-9/1546365_610175835684197_1592485571_n.jpg?_nc_cat=102&ccb=1-3&_nc_sid=730e14&_nc_ohc=ekKSJGzpyEMAX8XcF9p&_nc_ht=scontent-lga3-1.xx&oh=8f0f8e2f6e637bc8ba58ccd237323787&oe=60ED2D49

     pic https://scontent-lga3-1.xx.fbcdn.net/v/t1.18169-9/944828_610175759017538_341958810_n.jpg?_nc_cat=101&ccb=1-3&_nc_sid=730e14&_nc_ohc=II0O9EqedckAX_dogiT&_nc_ht=scontent-lga3-1.xx&oh=7565839dd03182858930e176570e5fb3&oe=60EC38FB

  tab_end
\fi

Гастролей донецких я ждала, надо признаться, как ни одну из своих последних
поездок. Со смешанным чувством страха и надежды. Чего мне только не говорили
друзья и коллеги перед поездом - что киевские, что донецкие. "Сними
желто-голубую ленточку, а то прибьют!", "Избегай контактов с милицией!", "Не
читай стихи о Майдане!" - и т.д. и т.п. Ленточки я, разумеется не сняла. Скорее
я бы избавилась от знаков отличия где-нибудь во Львове, где ими никого не
удивишь. От милиции, за исключением демонстративно вежливой просьбы не курить
на вокзале под табличкой с перечеркнутой сигаретой, где я дымила, как паровоз,
− ничего худого я не слышала. 


\ifcmt
  tab_begin cols=3
  width 0.3

     pic https://scontent-lga3-1.xx.fbcdn.net/v/t1.18169-9/1505086_610175699017544_437649118_n.jpg?_nc_cat=108&ccb=1-3&_nc_sid=730e14&_nc_ohc=v9JGcZek2jQAX9l5OxO&_nc_ht=scontent-lga3-1.xx&oh=0f7fbba6e8c38f6780153228b88e41ec&oe=60ECB582

     pic https://scontent-lga3-1.xx.fbcdn.net/v/t1.18169-9/578451_610175722350875_545126885_n.jpg?_nc_cat=103&ccb=1-3&_nc_sid=730e14&_nc_ohc=jxIBtmELJikAX9BS242&_nc_ht=scontent-lga3-1.xx&oh=19457c902d194057148309b280392494&oe=60EC9A79

     pic https://scontent-lga3-1.xx.fbcdn.net/v/t1.18169-9/1480504_610175755684205_389057938_n.jpg?_nc_cat=101&ccb=1-3&_nc_sid=730e14&_nc_ohc=6dBR6D-mgLMAX-wUulL&_nc_ht=scontent-lga3-1.xx&oh=10b30bc68664ebfd866f63e4a2836bde&oe=60EBE7A6

  tab_end
\fi


Возможно, мне просто повезло, не спорю. Но все это время меня не покидало
смутное ощущение  Защиты Сверху. Чтобы получить эту защиту, надо просто  делать
свое. Не роптать, не дергаться, не искать виновных, не перекладывать
ответственность.  Просто идти и ощущать Путь - прислушиваться к его советам и
принимать его подсказки. 

\ifcmt
  tab_begin cols=3
  width 0.3

     pic https://scontent-lga3-1.xx.fbcdn.net/v/t1.18169-9/1526609_610175942350853_814920732_n.jpg?_nc_cat=111&ccb=1-3&_nc_sid=730e14&_nc_ohc=vU6liScaCAgAX8njOYd&_nc_ht=scontent-lga3-1.xx&oh=0184d481e4c61cb9f410213c4c944a51&oe=60EBEBBC

     pic https://scontent-lga3-1.xx.fbcdn.net/v/t1.18169-9/1551751_610175959017518_1671568837_n.jpg?_nc_cat=105&ccb=1-3&_nc_sid=730e14&_nc_ohc=U8tDpcZSoFcAX-WeIQB&_nc_ht=scontent-lga3-1.xx&oh=319544481333263fa7125572c8411ba2&oe=60ECE9E8

     pic https://scontent-lga3-1.xx.fbcdn.net/v/t1.18169-9/1503998_610175962350851_6598585_n.jpg?_nc_cat=105&ccb=1-3&_nc_sid=730e14&_nc_ohc=TvKjG3trsU4AX-_uTwj&_nc_ht=scontent-lga3-1.xx&oh=01d6ded74d2f497cbbc45315488e704b&oe=60EC3CD1

  tab_end
\fi

Далее - об отдельных вехах этого пути.

Пытаться умалчивать о хрестоматийно известной проблеме Востока и Запада в
Украине - по меньшей мере пошло. Дилемма эта существует, − и ни для кого не
секрет, что оппозиция регионов имеет не только (и не столько) геополитический
характер, но прежде всего духовную, ментальную природу. 

Донбасс - это страна в стране, со своими культурными кодами и мыслительными
матрицами, отличными не только от Киева и Львова, но и (кстати!) от других
восточных областей. И читать эту матрицу можно только изнутри, в специфическом
внутреннем контексте формул и символов, которыми здесь привыкли
коммуницировать. К примеру, здешней ментальности свойственен почти
романтический при всей кажущейся грубости максимализм. 

\ifcmt
  tab_begin cols=2
		width 0.3

     pic https://scontent-lga3-1.xx.fbcdn.net/v/t1.18169-9/1525417_610176252350822_314524360_n.jpg?_nc_cat=108&ccb=1-3&_nc_sid=730e14&_nc_ohc=jTfyGMnJi_IAX9kg4Xh&tn=lowUrFCbCbt-jOWu&_nc_ht=scontent-lga3-1.xx&oh=cccc2e5a474695304fb79e2912a8763b&oe=60ECA14A
		width 0.26

     pic https://scontent-lga3-1.xx.fbcdn.net/v/t1.18169-9/1512479_610176282350819_1063391370_n.jpg?_nc_cat=106&ccb=1-3&_nc_sid=730e14&_nc_ohc=CtY078PXT9QAX-cXvTq&_nc_ht=scontent-lga3-1.xx&oh=b91c596b89035768a76c960184d1daea&oe=60EB97A3

  tab_end
\fi

Если ты - бандит, то уже - самый настоящий, классический, со всеми
признакам авантюрного капитализма эпохи "красных пиджаков". Если - националист,
то экстремальности твоего патриотизма позавидует, порой, сам житель Косова. И в
то же время именно здесь происходит ломка стереотипов. 

За два дня я поняла, как мощно у нас работает механизм редукционного обобщения
- суждения о целом по его части, на основании которой делается удобная в
обращении, но довольно вульгарная «типичная» картинка. Например, одним из
стереотипных  образов является образ безграмотного "нового украинца". А мне
встретился бизнесмен, который разбирается в поэзии Бродского, интересуется
текстами Упанишад и пишет талантливые стихи. На фоне массового распространения
в среде нашей поэтической интеллигенции графоманской полуобразованности, такие
случаи - крайне показательны. 

Во время концерта, который длился два часа, надела украинский веночек. Когда я
в нем появилась, услышала от одного из столиков шепот: «Ой, а я − против
Майдана". И тут же чье-то остерегающее: "Давай не будем обострять, не  об этом
сейчас". 

Во время чтения декламировала старое и новое, русское и украинское, включая
тексты о революции. Попросила не хлопать. люди согласились, слушали молча,
очень внимательно. Но однажды, несмотря на мои запреты, спонтанные
аплодисменты, переросшие в откровенную поддержку, все же случились. Это было
после текста, посвященного беркутовцу, который избил студентку. Комментарии –
излишни.

Когда вечер закончился, мне сказали столько теплых слов, сколько я не получала,
наверно, за весь год. У некоторых были слезы на глазах. Я сама боялась
расплакаться от счастья взаимного понимания. А меня поняли и поверили. Нет, не
изменили убеждений, но приняли и разделили боль. И я просто кожей ощущала
маленький бродячий огонек Свободы, который зажигался глубоко внутри и
передавался от сердца к сердцу.

А потом... потом был Эрмитаж. Частный особняк на берегу реки, хозяин которого –
человек явно не бедный – тем не менее, читает Евтушенко, не растопыривает
пальцы веером, не матерится и, как и мы, «сумасшедшие нищие хиппи с Майдана»,
осуждает бандитскую власть. Имея несусветную кучу увешанных статуями мраморных
комнат за спиной, он сидит и пьет водку в крохотной советской кухоньке,
закусывая дешевым сыром  и галетами. Мы с моим неизменным донецким издателем
(три четвертых моих книг вышли именно здесь, а не в фонде имени Степана
Бандеры, как кто-то думает))) – у него в гостях и говорим о Родине. Мы поем
украинские песни, спорим и надеемся на победу добра. 

Мы - это: поэт, историк, издатель, бизнесмен, геолог... четверо усталых
взрослых людей, готовых защищать студентов до последнего вздоха, - и одна
девчонка, пережившая свои тридцать три, но не утратившая ни студенческой веры,
ни детской наивности. В огромном зале бьют часы среди мраморных кариатид. У
плиты крепчает добротный русский, на котором звучат объяснения в любви…
«националистическому» Майдану. Внизу под широкой верандой течет какая-то река.
Кажется, Кальмиус. Утром с похмелья я почему-то называю её Мангупом. Ничего
удивительного: та же атмосфера священной крепости. 

Вечером следующего дня – нежданный подарок: мощнейшее выступление знаменитого
Сергея Жадана, собравшее полную местного населения  площадь у памятника
Шевченко. Много украинской речи, девчонки, держащие флаг, танцы, чистая,
сознательная и честная радость Слова ... Все, как на Майдане, только в меньшем
масштабе − и без патетических графоманов на большой сцене. Потом случайно
встречаю Сергея в автобусе. Кто-то когда-то дал ему мою книгу – подойти не
решилась: я давно не испытывала чувство школьного стеснения. Приятное ,черт
побери, чувство. 

Когда я в сине-голубой ленточке заходила в троллейбус или магазин, люди
смотрели на меня удивленно, но не агрессивно. Скорее в их улыбках читалось
недоуменное смущение и, как мне показалось, − какая-то робкая солидарность.
Хотя, возможно, я хочу видеть больше, чем есть, не спорю. Впрочем, я и так
получила здесь слишком многое, чтобы еще что-то додумывать.

Я понимаю, каково это, - гордиться своим народом. Это - очень естественно, как
дышать, и не требует дополнительных волевых усилий.

\ii{31_12_2013.fb.bilchenko_evgenia.1.doneck_shahter_ermitazh.cmt}
