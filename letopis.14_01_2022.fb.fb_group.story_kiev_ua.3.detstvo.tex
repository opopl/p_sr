% vim: keymap=russian-jcukenwin
%%beginhead 
 
%%file 14_01_2022.fb.fb_group.story_kiev_ua.3.detstvo
%%parent 14_01_2022
 
%%url https://www.facebook.com/groups/story.kiev.ua/posts/1840108102852649
 
%%author_id fb_group.story_kiev_ua,bondareva_iryna.kyiv
%%date 
 
%%tags detstvo,kiev,kievljane,pamjat
%%title Час спогадів дитинства...
 
%%endhead 
 
\subsection{Час спогадів дитинства...}
\label{sec:14_01_2022.fb.fb_group.story_kiev_ua.3.detstvo}
 
\Purl{https://www.facebook.com/groups/story.kiev.ua/posts/1840108102852649}
\ifcmt
 author_begin
   author_id fb_group.story_kiev_ua,bondareva_iryna.kyiv
 author_end
\fi

Дні, що перед Новим роком і святкові дні після, - це час спогадів
дитинства... Моє дитинство від народження до 13 років пройшло на вулиці
Борщагівській...

Малою я дуже любила хворіти... Перш за все тому, що знала точно, - мені
дозволять їсти улюблені малесенькі мариновані опеньки і вишневе варення! 

\ii{14_01_2022.fb.fb_group.story_kiev_ua.3.detstvo.pic.1}

Гриби восени збирали в лісі, маринували і закатували. Відкривали банки, в
основному, тільки на свята до столу. Вишневе варення я могла їсти на сніданок,
обід і вечерю, тому мене обмежували. 

Ну а, якщо дитина хворіла... То тут вже відмови не було!!! 

І от, наївшися грибочків і закусивши вареньком, я умощувалась зручненько у
великих подушках, під величезною пухнастою ковдрою і просила бабусю дати мені
товстенний сімейний альбом... І відправлялася у паралельний світ... З
невимовним трепетом повільно перегортала сторінки, довго і уважно вдивлялася в
обличчя бабусь, прабабусь, дідусів, прадідусів, їхньої численної рідні, яких я
ніколи не знала... Особливе задоволення мені доставляли світлини, на яких мої
батьки, їхні брати і сестри, бабусі і дідусі були малими дітьми... Годинами
сиділа я над тим альбомом, поки не засинала з ним в обіймах... 

Наступним  \enquote{номером програми} було розглядання, а пізніше й читання деяких
прекрасних книжок з великої дідусевої бібліотеки. Найулюбленішими були казки
Андерсена, величезна товста книга  М. Ю.  Лєрмонтова і  \enquote{Пригоди барона
Мюнхаузена}. В цих книжках були неймовірної краси ілюстрації! Роздивляючись їх
і занурюючись у той чарівний світ, - на кого я тільки не перетворювалась і ким
тільки не була в своїй уяві! В яких пригодах і мандрах тільки не побувала! 

А ще я радісно хворіла тому, що до мене викликали улюблену лікарку Валерію
Назарівну Грушевську, Царство їй небесне... Вона була дружиною дідусевого
найліпшого, ще з часів гімназії, друга Анатолія, племінника Михайла Сергійович
Грушевського. Вони дружили усе своє, на жаль, дуже коротке життя. Мій дідусь
помер у 48 років, лишивши 40-річну бабусю з чотирма дітьми... Анатолій пережив
його не надовго... 

Валерія Назарівна працювала у лікарні ім.Калініна лікарем-педіатром у
пологовому відділенні, де я і народилась... Вона лікувала, майже, всіх дітей в
нашій родині до глибокої своєї старості. Таких лікарів, якою була Валерія
Назарівна, тепер не буває... В усякому разі, я не зустрічала. Таку любов несла
в собі ця жінка до маленьких людей, до своєї справи, до свого служіння... Її
важко було назвати красивою зовні, але душа, особистість були неймовірно
красиві! Вона розмовляла дуже тихо, лагідним голосом, такими ніжними
інтонаціями, що дитина одразу була сповнена довіри. Які в неї були люблячі
руки! Як вона відчувала кожну дитину! 

Я чекала її з нетерпінням, хоча лікарів взагалі боялася страшенно. Коли вона
прослуховувала легені, заглядала в горло, щось лагідно, спокійно запитувала, -
я перебувала під якимось чудовим гіпнозом, у повній гармонії. 

Після огляду і рекомендацій бабуся пригощала Валерію Назарівну, вони пили чай з
пирогами і розмовляли... Я вже в повній ейфорії слухала ці розмови, спогади про
покійних чоловіків, про батьків... Журчання їхніх тихих голосів заколисувало
мене і я дрімала потроху щасливіша зі щасливих... 

Зовсім вже перед сном бабуся натирала мене якимись мастилами, кутала у свою
теплу хустину, поїла травами з медом... і кип'яченим молоком... 

Яка ж то була гидота - оте гаряче молоко з содою, маслом і медом! І неминуче з
пінками!!! Як таке можливо було пити?! 

Довго і нудно мене умовляли, потім вже мали місце якісь погрози... У кінці
кінців, крізь сльози, якимсь дивом в мене вливали кілька ложок, але не більше! 

Промайнув час... Я лікувала сина, потім онучку... Багато чого пам'ятала з того,
що казала Валерія Назарівна... І ніколи більше не зустрічала серед лікарів хоч
трохи когось схожого на неї... 

Опеньки я не мариную, не закатую, хоч і дуже люблю гриби. Варення з вишні варю
завжди, але воно має схильність закінчуватися ще до Нового року... не без моєї
помічі...  @igg{fbicon.face.wink.tongue} 

Молоко так і не полюбила... 

Сімейні альбоми із задоволенням переглядаю і сьогодні...особливо, коли
хворію... Там усередині, на тих світлинах такі прекрасні, рідні обличчя із
далекого- далекого, майже нереального, іноді здається, життя... 

Піду вип'ю чаю з імбірем, лимоном та медом... А ще у морозилці є дуже
смачненька свіжа малина і вишня, пересипані  трохи цукром....
@igg{fbicon.face.smiling.eyes.smiling}  @igg{fbicon.face.savoring.food} 

\ii{14_01_2022.fb.fb_group.story_kiev_ua.3.detstvo.cmt}
