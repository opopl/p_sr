% vim: keymap=russian-jcukenwin
%%beginhead 
 
%%file 11_09_2021.stz.news.ua.tyzhden.1.kraina_balzak_vozrast
%%parent 11_09_2021
 
%%url https://tyzhden.ua/Columns/50/252991
 
%%author_id nestelejev_maksim
%%date 
 
%%tags ukraina,strana
%%title Країна бальзаківського віку
 
%%endhead 
\subsection{Країна бальзаківського віку}
\label{sec:11_09_2021.stz.news.ua.tyzhden.1.kraina_balzak_vozrast}

\Purl{https://tyzhden.ua/Columns/50/252991}

\ifcmt
 author_begin
   author_id nestelejev_maksim
 author_end
\fi

Матеріал друкованого видання \textSelect{№ 36 (719) від 8 вересня}

\textSelect{Відгриміли паради, відлунали привітання, минули святкування і настав час
визнати сувору правду: нам уже тридцятка. Тож виправдовуватися, що ми молода
держава, вже пізно. Бо настав наш бальзаківський вік, як заведено називати таку
кількість років після виходу роману Оноре де Бальзака «Тридцятирічна жінка»}

Три декади незалежності України — радісна віха для нас і прикре непорозуміння
для наших супостатів, які впродовж віків «неситим оком зазирали» на нашу
«оновлену землю», відмовляючись бачити тут окрему державу і націю. Крім того,
нашу боротьбу за незалежність завжди ускладнювали свої ж таки краяни:
запроданці, корисні ідіоти, зрадники та загалом «несвідомі людські маси». Й у
першій половині ХХ століття це різностороннє протистояння мало безліч визначних
битв і вирішальних моментів, які найкраще змогли зафіксувати не підручники з
історії, а мистецтво.

У 1907 році Винниченко в назві оповідання «Уміркований і щирий» спохопив ті два
стародавні психологічні типи, які, живучи в Україні, по-різному бачать роль
українця. Знаково, що двоє персонажів потрапили до в’язниці зовсім не через
національне питання, а через те, що надмірна балакучість одного та мовчазна
згода другого роблять із них соціалістів, ворогів, яких тоді боялися більше за
всяких там українців. Щирий «надміру любить Україну», а «уміркований»
переконаний, що «нам, українцям, воно якраз і не до лиця. Нам більше треба
політикою брать: там промовчати, там ухилитися, там потанцювати під дудку
дужчих». І радує хіба те, що друзі трималися разом до кінця, а щирому ще й
попротестував у буцегарні: «Геть, чортова кацапня, з наших українських тюрмів!
Чого поналазили сюди?!» Уже за тридцять років таке покричати він не зміг би, бо
його змусили б замовкнути кулею в потилицю.

Читайте також: \href{https://tyzhden.ua/Columns/50/252254}{Книжність, Максим Нестелєєв, tyzhden.ua, 03.07.2021}

Те, що сталося за десять років після Винниченка, описав Юрій Тютюнник у книжці
«Революційна стихія». Революцію 1917-го він застав, перебуваючи в гарнізоні під
Сімферополем, де згодом постала проблема «освідомлення» українців, що
перебували в царській армії. І на першому зібранні всіх солдатів-«уроженців»
українських ґуберній відбулася така символічна сцена: «Прибуло щось до семи
тисяч. Відкривши віче, я запропонував: — Хто поміж вами українці, піднесіть
руку до гори! Піднеслося не більше трьохсот рук. — Малороси! Піднесіть руки!
Піднесло руки коло половини присутніх. — Хахли! Піднесіть руки! Знов піднесла
руки добра третина».

Пізніше Тютюнник став генералом-хорунжим армії УНР, а згодом під псевдонімом
Юрко Юртик — одним із сценаристів геніальної «Звенигори» (1927) Довженка,
першої частини його кінотрилогії. А робити другу частину — «Арсенал» (1929) —
він уже не зміг, бо того самого року його арештували. Й «Арсенал» лишився
промовистим втіленням того, як драматично життя розійшлося з мистецтвом.
Режисер, як відомо, брав участь у показаному січневому повстанні 1918 року,
коли більшовики з заводу «Арсенал» почали свій збройний наступ на Центральну
Раду, однак на нашому боці. Тоді як у фільмі він відтворив усе з ворожої точки
зору, тому-то й українці у нього недолугі й карикатурні. І питання, яке
ставлять у кіно, тепер набуло геть інших смислів — «Ти перекинув нашу «Вільну
Україну»?».

\begin{zznagolos}
І хотіла меншість незалежності, а більшість хотіла, аби її не чіпали, але стали
незалежними всі. І ті, хто мріяв, і ті, кому байдуже. І вже Україна
бальзаківського віку, а визвольні змагання досі тривають, бо мрія — найбільший
ворог байдужості	
\end{zznagolos}

22 січня 1918 року проголосили незалежність Української Народної Республіки, і
найвиразніше цю добу описав Підмогильний в оповіданні «Гайдамака». У ньому до
армії УНР прибився гімназист Олесь, якому було все одно, з ким і проти кого
воювати, бо й жити йому не дуже хотілося. Тож він «постановив полізти туди, де
вбивають». На початку оповідання показано, як український військовий переконує
селян «озброїтись і сполучитись з ними, щоб укупі боронити інтереси нашої
рідної Вкраїни». А далі письменник одним реченням підбив передчасні підсумки
нашої тодішньої національної поразки: «Селяни згоджувались, що боронити Україну
треба, але озброїтись не хотіли і в село не пустили».

1926 року Остап Вишня написав гумореску «Чухраїнці», де з сумним гумором
зобразив наш народ як такий, що вже сам не пам’ятає свого коріння: «— Якої ви,
лорди, нації? Вони, почухавшись, одповідають: — Та хто й зна?! Живемо в
Шенгерієвці. Православні».

І 1941 року Улас Самчук нібито продовжив Вишню, невесело висновуючи та
запитуючи: «Нарід чи чернь?». А далі — 50 непростих років аж до 24 серпня.
Пройшовши дев’ять кіл зачарувань і розчарувань, 1991 року ми зрештою цю
незалежність здобули, точніше відновили, і тоді ж зі здивуванням виявили, що
боротьба за свободу на цьому не припинилася. І «злії люде» і далі «присипляють»
нашу увагу, розпорошуючи наші сили та ширячи зловорожу пропаганду.

Читайте також: \href{https://tyzhden.ua/Columns/50/250443}{Ходи, королево, ходи, Максим Нестелєєв, tyzhden.ua, 05.12.2021}

Минали роки, і жили на нашій території руси, русини та козаки, і боялися наші
недруги мазепинців, петлюрівців і бандерівців. Але й дотепер мешкають у нашій
країні одночасно малороси, хохли й українці — і сила наша в цьому, і горе. І
хотіла меншість незалежності, а більшість хотіла, аби її не чіпали, але стали
незалежними всі. І ті, хто мріяв, і ті, кому байдуже. І вже Україна
бальзаківського віку, а визвольні змагання досі тривають, бо мрія — найбільший
ворог байдужості. 
