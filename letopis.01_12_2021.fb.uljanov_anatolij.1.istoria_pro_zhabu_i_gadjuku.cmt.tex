% vim: keymap=russian-jcukenwin
%%beginhead 
 
%%file 01_12_2021.fb.uljanov_anatolij.1.istoria_pro_zhabu_i_gadjuku.cmt
%%parent 01_12_2021.fb.uljanov_anatolij.1.istoria_pro_zhabu_i_gadjuku
 
%%url 
 
%%author_id 
%%date 
 
%%tags 
%%title 
 
%%endhead 
\subsubsection{Коментарі}
\label{sec:01_12_2021.fb.uljanov_anatolij.1.istoria_pro_zhabu_i_gadjuku.cmt}

\begin{itemize} % {
\iusr{Анатолий Ульянов}

Новость:

\ii{link.29_11_2021.bbc.1.golodna_tusa}

Реакция:

\href{https://www.facebook.com/tamara.zumka/posts/10227302865898844}{%
Хеловін був, Чорна п’ятниця була, Голодомору - не було, Tamara Zlobina, facebook, 30.11.2021%
}

\href{https://www.facebook.com/opavlenko/posts/10226952966598088}{%
З цікавістю спостерігаю, як розгортається історія навкруги \enquote{голодної туси} блогерів, %
Oksana Pavlenko, facebook, 30.11.2021%
}

\href{https://life.pravda.com.ua/columns/2021/11/30/246660/}{%
"Голодна блогерська туса": чи є в Instagram місце пам’яті про Голодомор?, Тамара Злобіна, life.pravda.com.ua, %
30.11.2021%
}

Голодный пенсионер:

\href{https://dilo.net.ua/novyny/u-kyyevi-didus-yiv-hlib-zalyshenyj-na-memoriali-pam-yati-zhertvam-golodomoru/}{%
У Києві дідусь їв хліб, залишений на Меморіалі пам’яті жертвам Голодомору, dilo.net.ua, 28.11.2021%
}

\uzr{Жителі та гості Києва зібралися 27 листопада на території Національного музею
\enquote{Меморіал жертв Голодомору}, щоб вшанувати пам'ять загиблих.}

\begin{multicols}{2}

\ifcmt
  ig https://dilo.net.ua/wp-content/uploads/2021/11/1638052098-8471.jpg
  @width 0.4
\fi

Літній чоловік збирав хліб та продукти біля скульптури \enquote{Гірка пам’ять дитинства}.

Відповідне відео оприлюднив Telegram-канал \enquote{Столиця}.

Зазначається, що літнього чоловіка помітили вже після офіційних поминальних
заходів у День пам’яті жертв Голодомору.

На кадрах добре видно, що пенсіонер ходив навколо скульптури серед людей та
збирав в сумку продукти, які залишили ті, хто прийшов пом’янути жертв геноциду.
При цьому літній згорблений чоловік їв хліб, залишений на Меморіалі.

\end{multicols}

\iusr{Валерий Коваленко}

Сколько этому пенсионеру лет? Примерно 70? Значит, в 1991 году ему было лет 40
примерно. И голосовал он, скорее всего, за нэзалэжнисть...

\begin{itemize} % {
\iusr{Анатолий Ульянов}
мы ничего не знаем об этом человеке, кроме того, что ему хочется есть

\iusr{Olexandr Shchegolev}

\textbf{Валерий Коваленко} ну да, за незалежність. Подальше от голодухи. Что не так?

\href{https://vakhnenko.livejournal.com/175302.html}{%
Американская продовольственная помощь Российской Федерации, vakhnenko.livejournal.com, 24.11.2014%
}

\iusr{Olexandr Shchegolev}
\textbf{Анатолий Ульянов} мы не знаем хочется ему есть или нет. Может быть стопицот вариантов трактовки его действий.

\iusr{Анатолий Ульянов}
\textbf{Olexandr Shchegolev} 

человек собирает куски хлеба, и ест их у тебя на глазах. его голод вызывает у
тебя сомнения? ок. какой из вариантов трактовки его действий кажется тебе
наиболее вероятным? что ты видишь на этом видео?

\iusr{Валерий Коваленко}
\textbf{Olexandr Shchegolev} , убежали?

\iusr{Olexandr Shchegolev}
\textbf{Валерий Коваленко} пока нет( руцкэ медведь держит крепко. Но назад уже дороги нет и это радует.

\iusr{Olexandr Shchegolev}
\textbf{Анатолий Ульянов} 

не собирает куски хлеба, а взял кусочек. Что его к этому толкнуло? ХЗ, на вид
человек нездоров. Поди знай что у него в голове.

\end{itemize} % }

\iusr{Владимир Могилевский}
Свобода, это не смена хозяина

\iusr{Наталья Плужник}
Жлобы \enquote{розкуркулылы} инфлюенсера, ибо память предков не дремлет

\iusr{Олег Шеленко}

Мені дивно, що ці люди, які декларують \enquote{Моє тіло - моє діло} коли заходить про
автономію інших, в подібних випадках, починають вести себе як якийсь
націоналістичний ИГИЛ. При цьому тут мова не про якісь речі, що можуть шкодити
суспільству, якимось людям, а про вірність різновиду світської \enquote{релігії}. При
цьому в 99\% коли б мова зайшла про День Перемоги, память героїв Великої
Вітчизнної ітп від них просто потоки бруду і зневаги

\iusr{Alsan Amir}

Ну право, устроить вечеринку можно, но не глумилово же. Есть же элементарная
тактичность, как давно заметил, у русскоязычных чувство такта куда-то
потерялось. Видимо 90-е дают знать и камеди -клаб.

\iusr{Luca Brasi}
Почему так много красного.

\iusr{Евгений Никульников}
Хорошо подмечено!

\iusr{Mikola Vilni}

Пенсионеры конечно же в жопе, и только Толик и Китай им поможет! Срочно! Брекин
ньюс! Толик переезжает завтра на пмж в коммунистический Китай! ЗАвтра жжже.

Анатолий, 100 \$ и завтра будет видео как пенсионерка ест хлебушек у святынь
ватностана. Только нормальные люди такой мразотой, в отличии от ваты, не
занимаются.

\end{itemize} % }
