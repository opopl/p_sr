% vim: keymap=russian-jcukenwin
%%beginhead 
 
%%file 24_02_2023.fb.miroshnichenko_mihail.mariupol.1.moi_dnevnik__kotorii.cmt
%%parent 24_02_2023.fb.miroshnichenko_mihail.mariupol.1.moi_dnevnik__kotorii
 
%%url 
 
%%author_id 
%%date 
 
%%tags 
%%title 
 
%%endhead 

\qqSecCmt

\iusr{Tatyana Kondratenko-Spetsialnaya}

Этот ад мы не забудем никогда, и при воспоминании сразу наворачиваются слёзы.
Чего нам только не пришлось пережить в эти дни, в Мариуполе. Сама смогла
выехать с больным, после инсульта дядей 21 марта. Хорошо хоть сына с беременной
на 7 месяце невесткой успела отправить последним эвакуационным поездом, а сама
осталась, чтобы забрать дядю. Но больше возможности выехать не было. Слава Богу
мы живы. Теперь ждём Победы. Все буде Україна.

\iusr{Елена Сенатосенко}

\ifcmt
  igc https://i.paste.pics/9c74f62aa473100e581614e2e7b3a5de.png
	@width 0.1
\fi

\iusr{Михаил Мирошниченко}
\textbf{Сена С.Е.В.}

\ifcmt
  igc https://scontent-fra3-1.xx.fbcdn.net/v/t39.30808-6/332308674_1368180787290629_2651771968573528356_n.jpg?stp=cp6_dst-jpg&_nc_cat=105&ccb=1-7&_nc_sid=dbeb18&_nc_ohc=mvuuj_Xq7dAAX8WsdJY&_nc_ht=scontent-fra3-1.xx&oh=00_AfAgUtoS_K5Wee0p9NClUSKyF02PcpmsZ2B-tM4JGU5Njw&oe=6407CB0E
	@width 0.4
\fi

\iusr{Осташко Сергей}

Утро 24/2022. Примерно в 7-30 я стоял у окна и пил кофе, когда над домом одна
за другой пролетели 2 крылатые ракеты. Была переменная облачность и их было
отчетливо видно в течении нескольких секунд. Через пару минут - звук взрыва.
Это была атака на арсенал в Калиновке (15км).

Я не испугался только потому, что был растерян, мы не включаем телевизор уже 7
лет, новостей я не знал. Включил компьютер и понял, что школа для детей -
отменяется. Работа мне и так не грозила, я работал из дома и передавал дела
после увольнения. Были планы принципиально поменять направление деятельности и
стиль жизни. Поменял (.

Поток новостей ужасал: ракетный атаки, попадания в гражданские объекты, колонны
вражеской техники со всех направлений, воздушные бои, десант и бои под Киевом,
колонны беженцев, расстрел рашистами гражданских...

Винница очень далека от прифронтовых районов. Но, в течении суток город оброс
блокпостами. Объездные дороги и переулки были перекрыты, блокпосты фильтровали
колонны. Трассы и выезды были забиты машинами беженцев: через Винницу идут
направления на Молдавию и на Львов. Все заправки были опустошены.

Выезжать было поздно. Пол-бака топлива в машине. Двое детей. Никаких идей куда
ехать. Решили ждать определенности, благо, что в Виннице тихо.

\iusr{Тетяна Вішинська Грузіна}

24.02.22

Проснулась от звуков взрывов, первая мысль- ну вот маски и сняты, теперь будет
видно кто есть кто.

Жестокий убийца открыл своё лицо, но сам отождествляет себя с благородным
воином, сражающимся за «правое дело»...

Почувствовала ненависть к идеализированию какой-либо личности.

Посмотрела на севоё испуганное и заспанное лицо в зеркале и порадовалась, что я
далека от идеального образа и улыбнулась от этой мысли.

Каждый имеет право обманываться, результаты этих ложных представлений о себе, о
мире, о других людях как правило имеют весьма болезненные последствия, а
прозрения от тьмы лжи, зачастую ещё и мучительные.

Надеюсь мое прозрение будет смягчаться осознанным стремлением видеть ясно,
осознавать причины и решать в имеющихся условиях текущие жизненные задачи...

И первая задача, которая пришла, была: позаботиться о еде и воде, узнать, что с
этими ресурсами у близких.

Делать все что от меня зависит, чтоб сохранять свою жизнь и целостность и
помогать в этом тем, кто рядом.

И чтоб задачи решались без паники провела круг Рей-Ки с намерением :

«Все что происходит со мной, ведёт меня в Свет и Любовь.»

Эту мысль я пронесла в блокадном Мариуполе или может она меня несла, каждый
раз, возвращалась к ней, когда видела прямую угрозу жизни.

Немного спокойнее становилось, зная куда и зачем я движусь.

Так я смягчала себе осознание объемов жестокости и лжи.

Ещё я приняла на себя ответственность успеть сказать люблю, тем к кому чувствую
любовь и подлецу, что он подлец.

Так я в контакте с соседом, который однажды сказал возле костра, что с ДНР
начнётся порядок, ясно дала понять, что эта мысль меня удивляет, как можно
ждать порядка от жестокого неадеквата и насильника.

И что считать порядком, на что мне начал орать этот самый сосед со своей женой,
что они просто хотят жить…

« все, что происходит со мной ведёт меня в Свет и Любовь»

Я осознала что ненавижу страх смерти, он делает человека рабом.

Приняла для себя решение использовать эту энергию страха, как разумную
осторожность.

Говорить подлецу, что он подлец можно и глазами, особенно полезно это делать,
если этот подлец вооружён и запрограммирован убивать всех, кто представляет
угрозу для него и спец операции, которую он выполняет.

Так выезжая из обугленного от пожарищ разрушенного подлецами родного города, я
не сдерживая рыдания смотрела в глаза направляя своё сознание как можно глубже
в этом внимании я стремилась коснуться Души этих подлецов, которые предпочли
поддерживать чудовищный акт насилия и жестокости, называя его благородным
словом: «освобождение»

Те у кого ещё не умерла Душа, если мне удавалось ее коснуться, то как правило
личности в нарядах военных отводили взгляд тупя его в Землю.

Так я понимала Моё послание от Души к Душе было получено.

« То, что происходит со мной, ведёт меня в Свет и Любовь»

Было ясно, что подлое, только Эго, согласившееся с подменой понятий во власть
которого попала Душа, желающая освобождения, и она неприметного сбросит, подлое
Эго, при этом не все останутся жить живой Душой в теле.

Уж очень давно и подло ее предали.

И я не знаю останусь ли в теле я.

И вместимся тем, я продолжаю удерживать длинную мысль:

«Все, что происходит со мной ведёт меня в Свет и Любовь»

После моего выезда 16 марта,

мои любимые родные люди в лице дочки и мамы исчезли на 2 месяца из моего поля
зрения и образовался информационный вакуум.

Я осознавала, что каждый свободен использовать своё право жить и умереть,
выбирая место, способ и состояние.

Самое трудное было согласиться именно с этим .

Осознаю, что любимые свободны от моих представлений и ожиданий по отношению к
ним.

И я свободна от представлений и ожиданий относительно меня.

И вместе мы можем находить общие представления, цели и ценности, чтоб быть
близкими по Духу, не только по крови и находиться в физическом контакте.

Позднее после их выхода на связь, я осознала, что для истиной близости
расстояние не помеха.

Больно только моему Эго, которое хочет играть свои роли с теми, кто этим ролям
замечательно подигрывал в привычных радостных сценариях ...

Значит будут и другие жизненные сценарии у нас- теперь хочу счастливые, даже на
каком-то расстоянии!

У меня это получается лучше и лучше с каждым вдохом и выдохом.

«То, что происходит со мной, ведёт меня в Свет и Любовь»

...

\iusr{Виктория Баркалая}

Да, этот день мы не забудем никогда. 23 отмечала свой день рождения, п утром
позвонили и сказали, что началась война...

\begin{itemize} % {
\iusr{Михаил Мирошниченко}
\textbf{Виктория Баркалая} с днем Рождения, пусть все мечты сбудутся!

\iusr{Виктория Гуторова}
\textbf{Виктория Баркалая} мы и 18.03 никогда не забудем... воздух, море, артиллерия, мины... и 24.03 не забудем... Мы вообще ничего не забудем.
\end{itemize} % }

\iusr{Татьяна Стасенко}

\ifcmt
  igc https://scontent-fra3-1.xx.fbcdn.net/v/t39.30808-6/332334344_504479701882797_1670856326605134551_n.jpg?_nc_cat=102&ccb=1-7&_nc_sid=dbeb18&_nc_ohc=6XNe_zh1JPgAX-Xq_A8&_nc_ht=scontent-fra3-1.xx&oh=00_AfBGq9KmNYRsFusUv0sTkhQjZK_MnN7693oaNo3MjrIqrw&oe=6407F5BA
	@width 0.4
\fi

\iusr{Алексей Кокарев}

Слов нет. Миха это просто не знаю как описать пз.

\begin{itemize} % {
\iusr{Михаил Мирошниченко}
\textbf{Алексей Кокарев} это только начало.

\begin{itemize} % {
\iusr{Алексей Кокарев}
\textbf{Михаил Мирошниченко} красавчик

\iusr{Алексей Кокарев}
\textbf{Михаил Мирошниченко} написать можно много, а исправить будет уже нельзя, вернуть знакомых и друзей, ну каждый будет в ответе за то что сделал. Всем мира и чистого неба.

\iusr{Irina Lavrinenko}
\textbf{Михаил Мирошниченко} в тот жуткий период сложно и рискованно было выезжать из города.. Спасибо, Миша, за помощь нашим друзьям,когда вывез их отца..

\iusr{Михаил Мирошниченко}
\textbf{Ирина Лавриненко} привет Ирочка. Есть такой духовный закон: кто помогает людям, тому помогает Бог! Большой привет Андрею и сыновьям!
\end{itemize} % }

\end{itemize} % }

\iusr{Татьяна Кулагина}

Ребята, мы, кто выжили в Мариуполе - прошли самые жестокие испытания ада на земле.

Спасибо Богу, Вселенскому разуму, ангелам, природе, всем людям, которые были
рядом. Только обьединившись и помогая друг другу - мы выжили в полной
атисанитарии ❗По видимому нашим душам надо было получить и такой опыт. Он
горький. Миша, спасибо тебе большое за то, что вывез нас с этого пекла.

\begin{itemize} % {
\iusr{Михаил Мирошниченко}
\textbf{Татьяна Кулагина} Рад был вам помочь.

\ifcmt
  igc https://scontent-fra5-2.xx.fbcdn.net/v/t39.30808-6/332356511_1174354466608669_6193088306083073401_n.jpg?_nc_cat=109&ccb=1-7&_nc_sid=dbeb18&_nc_ohc=O9KDV5vrsN0AX_5yVbL&_nc_ht=scontent-fra5-2.xx&oh=00_AfDwmmpoYWHH54tUo_ypNqsFIo_gFTsYSwr__Du0StCPgA&oe=64082225
\fi

\iusr{Татьяна Кулагина}
\textbf{Михаил Мирошниченко} 

Бог создал людей для развития, а развиваться мы можем, когда друг друга слышим
и учитываем интересы ближних!

Это и есть помощь - взаимоуважение, тогда всем тепло и комфортно 🌞

\end{itemize} % }

\iusr{Liubov Sukhomlyn}

Мы тоже 16.03.2022 г выехали с Мариуполя. Никогда не забуду этот день, какой
страх мы пережили, когда ехали по центру к Драмтеатру, это было последнее утро,
когда мы видели его целым.

\iusr{Tkach Alina}

Спасибо, Михаил за то, что вывезли папу из Мариуполя! Вы наш герой!🙏💪🏻❤️

\iusr{Валентина Блакитная}

Меня тоже 16 марта дети с ранением вывезли с МАРИУПОЛЯ ,,,, ЛЕЖА В ГРУЗОВОМ АВТО
,,ГОРОДА РОЗРУШЕННОГО НЕ ВИДЕЛА----ВИДЕЛА ТОЛЬКО САМОЛЕТЫ КОТОРЫЕ БОМБИЛИ МОЙ
ПОСЕЛОК,,, ЭТОТ КОШМАР И УЖАС С НАМИ ДО КОНЦА НАШИХ ДНЕЙ,,, НЕТ ПРОЩЕНИЯ
асвабадителям ,,, ТАКОЕ ОСВОБОЖДЕНИЕ ИМ В КАЖДЫЙ ГОРОД И ДОМ, БОЛЕЕ 100 ТЫС
УБИТЫХ И НЕОПОЗНАННЫХ,,,,

\iusr{Ксения Сластенова}

все верно сказано, мне казалось это какойто квест на выживание, я ловила себя на
мысле ну доожно же хоть день становиться легче, не может же быть так сложно,
каждый день выживать и какието новые испытания, уровни, а легче не
становиться, легче только стало когда выехали из Мариуполя и потом всевышний
начал нам помогать во всем, в Мариуполе тоже я видела божью помощ, но все равно
было очень сложно
