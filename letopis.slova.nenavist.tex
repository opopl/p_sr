% vim: keymap=russian-jcukenwin
%%beginhead 
 
%%file slova.nenavist
%%parent slova
 
%%url 
 
%%author 
%%author_id 
%%author_url 
 
%%tags 
%%title 
 
%%endhead 
\chapter{Ненависть}
\label{sec:slova.nenavist}

%%%cit
%%%cit_head
%%%cit_pic
%%%cit_text
Защитник футбольного клуба \enquote{Ростов} Игорь Калинин признался, что после смены
украинского гражданства на российское стал получать угрозы от жителей Украины.
Об этом он рассказал в интервью Ютуб-каналк \enquote{ФАК.Подкаст}.  \enquote{Когда я поменял
гражданство, писали, угрожали. Но я к диванным критикам отношусь... ну пуская
пишут, если им делать нечего, кроме как сидеть строчить гадкие сообщения. Для
меня главное - мнение моей семьи, моих родных, моих родителей, моих партнеров
по команде}, - сказал 25-летний спортсмен.  Он добавил, что из-за угроз
перестал читать комментарии, а потом и вовсе удалился из всех соцсетей.  \enquote{Я не
знаю, почему народ у нас такой... Столько \emph{ненависти}!} - сказал Калинин
%%%cit_comment
%%%cit_title
\citTitle{Футболист Калинин рассказал об угрозах украинцев после смены гражданства}, 
Наталья Полулях, strana.ua, 25.06.2021
%%%endcit

%%%cit
%%%cit_head
%%%cit_pic
%%%cit_text
Ясен перец, что попытки вытравить Великую Отечественную из сознания и
устраивать танцы на костях и памяти миллионов, погибших и выживших, а потом
отстроивших страну ценой невероятных лишений и усилий, могут предпринимать
только отъявленные моральные уроды и мерзавцы и за хорошее место возле корыта.
А попытки выдавить эту войну из сознания, обозвать ее «нарративом Путина»,
переименовать в разного рода расхожую жвачку типа «вторая мировая»,
«советско-германская» предпринимаются уже далеко не первый раз. Но каждый раз
это вызывает отторжение и ответную \emph{ненависть}. Самую настоящую классовую
\emph{ненависть}. В самом прямом смысле и со всеми вытекающими...
%%%cit_comment
%%%cit_title
\citTitle{Краткий словарь грантоедов под редакцией СНБО / Лента соцсетей / Страна}, 
Александр Карпец, strana.news, 26.10.2021
%%%endcit

%%%cit
%%%cit_head
%%%cit_pic
%%%cit_text
Оттуда же не покидает нас и ощущение, что все жители бывших республик СССР
«наши». «Как же могли \emph{возненавидеть} друг друга братские народы?» —
спрашивают сами у себя замечательные добрые люди и не понимают. «Мы же один
народ, почему столько \emph{ненависти}?» — искренне удивляются другие.  Ну,
во-первых, \emph{ненавидят} далеко не все. В той же России граждане, которые не
любят украинцев за то, что они украинцы, малочисленны и маргинальны. И даже на
сотрясаемой агрессивной националистической пропагандой Украине такая
\emph{ненависть} совершенно не повсеместна. Кстати, как научные исследования,
так и практика жизни однозначно говорят, что язык вражды чаще всего моментально
улетучивается, когда объект враждебных высказываний прямо перед тобой, а не
где-то там за тридевять земель или в интернете
%%%cit_comment
%%%cit_title
\citTitle{Разделенная Украина: что с нами происходит?}, Павел Волков, ukraina.ru, 31.10.2021
%%%endcit

%%%cit
%%%cit_head
%%%cit_pic
%%%cit_text
Маємо звернутися до всіх органів влади – президента, міністерств тощо. Нам
дійсно мають повернути наші канали. На цих каналах має йти мова про нашу
незалежність, наше майбутнє, про нашу мову, про стан нашої культури. Повинні
там розкривати нашого ворога – його побут, його відмінність від українців.
Треба врешті заборонити канал "Наш". Я вже не кажу про пресу. Ми маємо
формувати \emph{ненависть} до цього режиму – московського, царського, путінського.
Так, як це роблять вони у своїх передачах, що не слово, то ненависть до
України.  Також ми повинні покінчити з олігархами. Треба згадати тактику
Рузвельта і Черчилля, які зібрали олігархів і сказали: "Гроші в оборону.
Переможемо – повернемо!" і забрали в них мільярди доларів.  А що сьогодні
робить влада? Вона боїться української єдності, усвідомлення необхідності
боротьби з заклятим споконвічним ворогом і тому веде таку політику.  Про
Голодомор ми чуємо лише в останні тижні листопада. А це має бути щотижнева
передача. А також про ту саму війну, в якій ми стільки втратили, воюючи з
гітлерівським та сталінським режимами. Народ всього цього не знає"
%%%cit_comment
%%%cit_title
\citTitle{Як виграти інформаційну війну? – Слово Просвіти}, ,slovoprosvity.org, 01.11.2021
%%%endcit
