% vim: keymap=russian-jcukenwin
%%beginhead 
 
%%file 30_04_2020.fb.fb_group.story_kiev_ua.2.kievskie_mozaiki.pic.14
%%parent 30_04_2020.fb.fb_group.story_kiev_ua.2.kievskie_mozaiki
 
%%url 
 
%%author_id 
%%date 
 
%%tags 
%%title 
 
%%endhead 

\ifcmt
  ig https://scontent-frx5-2.xx.fbcdn.net/v/t1.6435-9/94951901_3174547782578807_8840867871585206272_n.jpg?_nc_cat=109&ccb=1-5&_nc_sid=b9115d&_nc_ohc=Bx_InGYj_28AX8XEbYV&_nc_ht=scontent-frx5-2.xx&oh=8b18224a8ce3f77e156138c18e67b01e&oe=61B3D5AF
  @width 0.4
\fi

\iusr{Ирина Петрова}
Трипільські мотиви

\iusr{Инна Шмыгина}
Проспект Победы

\iusr{Nina NinaNina}

пр. Перемоги - мозаїки на перших будинках виготовлені, які всі у цьому рядочку
будинків над колишнім воєнторгом у 1967-68 роках разом з будівлею попереду і
так і планувалось, ці дві - це вже початок 1980-х, не памятаю точно дату. На
першій мозаїці аутентичні трипільські мотиви (№ 17), вони часто зустрічаються у
розписах глиняного посуду, а про другий вже написала коментарій - стилізований
каштан серед зірок, але то як кому фантазія підкаже, поява саме такої теми
