% vim: keymap=russian-jcukenwin
%%beginhead 
 
%%file 06_07_2020.fb.fb_group.story_kiev_ua.1.fotografia_vospominania.pic.5
%%parent 06_07_2020.fb.fb_group.story_kiev_ua.1.fotografia_vospominania
 
%%url 
 
%%author_id 
%%date 
 
%%tags 
%%title 
 
%%endhead 

\ifcmt
  ig https://scontent-frt3-1.xx.fbcdn.net/v/t1.6435-9/107828526_3380588525308064_190673264393232774_n.jpg?_nc_cat=104&ccb=1-5&_nc_sid=b9115d&_nc_ohc=Y3HAuMY7-MQAX_0hCl1&_nc_ht=scontent-frt3-1.xx&oh=4e8fb97824cd1083f17d2a63dc1368f0&oe=61B5C109
  @width 0.4
\fi

\iusr{Ирина Петрова}

Кто помнит такие качели - две лошадки на концах длинной доски, перекинутой
через опору

\iusr{Татьяна Оксаненко}
\textbf{Ирина Петрова} 

Такие качельки были и в моем детстве, почему-то больше всего любила на ней
кататься на Подоле, во дворе, он же малюсенький сквер, там (Фруктовый
переулок)жила мамина подруга, а виделись они очень часто. Мой папа очень любил
Житний рынок, часто отаваривались там, реже на Сенном

\iusr{Ирина Петрова}
\textbf{Татьяна Оксаненко} 

вооот! Модные аттракционы нашего детства! Какое милейшее название - Фруктовый
@igg{fbicon.face.happy.two.hands} 

\iusr{Татьяна Оксаненко}
\textbf{Ирина Петрова} Потом, по-моему, этот переулок, стал Ярославский, если я не ошибаюсь...

\iusr{Елена Андрийченко}
\textbf{Ирина Петрова} очень популярны и востребованы были. Всегда рядом стояла очередь желающих "прокатиться"!

\iusr{Olena Ivanenko}
Да-да, именно такие с лошадками были на детских площадках и во дворах...

\iusr{Олена Медведева- Прицкер}

Да, деревянные качели - лошадки были на всех детских площадках, но меня водили
кататься в Полицейский садик на Красноармейской.

\iusr{Анна Ковалева}
\textbf{Олена Медведева- Прицкер} А девчушка похожа на тебя, подружка!
