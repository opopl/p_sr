% vim: keymap=russian-jcukenwin
%%beginhead 
 
%%file 20_11_2021.fb.fb_group.story_kiev_ua.2.malenkaja_podolskaja_devochka.cmt
%%parent 20_11_2021.fb.fb_group.story_kiev_ua.2.malenkaja_podolskaja_devochka
 
%%url 
 
%%author_id 
%%date 
 
%%tags 
%%title 
 
%%endhead 
\subsubsection{Коментарі}

\begin{itemize} % {
\iusr{Ольга Клюева}
Срочно найти парня!

\iusr{Светлана Манилова}
\textbf{Ольга}, может, он есть в нашей группе. @igg{fbicon.smile} 

\iusr{Antonina Chepiga}
Какие вы хорошенькие!

\iusr{Светлана Манилова}
\textbf{Antonina}, спасибо! @igg{fbicon.smile} 

\iusr{Ольга Кирьянцева}
Думаю, что мальчик смущен не меньше вашего)). Детские чувства -самые чистые и искренние).

\begin{itemize} % {
\iusr{Светлана Манилова}
\textbf{Ольга}, Вы правы.

\iusr{Виктор Задворнов}
\textbf{Светлана Манилова} прекрасные незнакомки на фото. А мальчик смущен не менее девочек. "Все прошло, как с белых яблонь дым."
\end{itemize} % }

\iusr{Виктор Задворнов}

\enquote{Мы будем петь и смеяться, как дети}. Выйдя на пенсию, все чаще вспоминаю
начало этой песни. Никакой политики. А теперь о детсадовских воспоминаниях.
Микроклимат в группе, конечно, зависит от воспитательницы и няни. Мне повезло:
меня всегда любили дети и я всегда любил детей.

\iusr{Татьяна Степанова}
Маленькие красотки @igg{fbicon.face.smiling.hearts} 

\iusr{Светлана Манилова}
\textbf{Татьяна}, я таки маленькая. @igg{fbicon.smile} 

\iusr{Анна Шустерман}
Такое же платьице было на утреннике ещё у кого-то из нашей группы. По-моему у
Татьяны Кутишенко, почему-то запомнилось.)

\begin{itemize} % {
\iusr{Тетяна Кутишенко}
\textbf{Анна Шустерман}, очень похожи:))
У Светы ещё контрастные детали есть!
По-модному!
Но платье+бант - точно похожи.

\ifcmt
  ig https://scontent-lhr8-1.xx.fbcdn.net/v/t39.30808-6/259876536_1256898911460861_7716560404369435698_n.jpg?_nc_cat=108&ccb=1-5&_nc_sid=dbeb18&_nc_ohc=eXWCipFAxgkAX_F-KWS&_nc_ht=scontent-lhr8-1.xx&oh=cfb1f61d135fad454ddfcce6b5fc6a22&oe=61B4EC5C
  @width 0.4
\fi

\begin{itemize} % {
\iusr{Анна Шустерман}
\textbf{Тетяна Кутишенко} , вот, именно про это платьице, Танечка, я и вспомнила.) Значит Ковид ещё не всю память отшиб. @igg{fbicon.wink} 

\iusr{Светлана Манилова}
\textbf{Тетяна}, по-моему, это одно и то же платье, только у Вас без бантиков. @igg{fbicon.smile} 

\iusr{Тетяна Кутишенко}
\textbf{Светлана Манилова}, твёрдо-капроновое:))

\iusr{Светлана Манилова}
\textbf{Тетяна}, именно! @igg{fbicon.thumb.up.yellow} 
\end{itemize} % }

\end{itemize} % }

\iusr{Olena Korenets Nebuchadnezzar}
Девчонки - прелесть!

\iusr{Валентина Турунцева}

Детский сад 100 на Борисоглебской улице.Заведущая Дралова Бетти Евсеевна, как
же давно это было, столько лет прошло, а воспоминания остались навсегда. Какой
же был чудесный детсад!!!!

\begin{itemize} % {
\iusr{Светлана Манилова}
\textbf{Валентина}, Бетти Евсеевна- мама сотрудника моего папы, поэтому я и оказалась в этом садике. @igg{fbicon.smile} 

\iusr{Валентина Турунцева}
\textbf{Светлана Манилова} Бетти Евсеевна была супер заведующая, благодаря ей детсад был самый лучший.Это помнят все сотрудники и воспитанники и родители.

\iusr{Светлана Манилова}
\textbf{Валентина}, знаю!
\end{itemize} % }

\iusr{Петр Кузьменко}

Светик! У мальчика знакомое лицо. Похоже он учился в параллельном классе нашей
100 школы ( кстати, тебе не кажется, что цифра 100 тебя преследует с раннего
детства). Если мне не изменяет зрительная и общая память, его фамилия Резников
или Резник. И учился он в В классе. Мы, если ты помнишь, в Б. Кстати, он
эмигрировал одним из первых, в 7-им или 8-м. Куда, не помню.

\begin{itemize} % {
\iusr{Светлана Манилова}
\textbf{Петр}, надо спросить у Вали, мы с ней были в одной группе. @igg{fbicon.smile} 

\iusr{Петр Кузьменко}
\textbf{Светлана} уточни. Я, вроде бы, помню по физиономии.

\iusr{Светлана Манилова}
\textbf{Валентина}, прошу помощь зала! В данном случае твою. @igg{fbicon.smile} 

\iusr{Валентина Штомпиль}
\textbf{Светлана Манилова} лицо, вроде бы, знакомое, но фамилию не помню, к сожалению

\iusr{Светлана Манилова}
\textbf{Валентина}, спасибо!

\iusr{Елена Свец Баркан}
\textbf{Петр Кузьменко} по-моему его имя Миша

\iusr{Петр Кузьменко}
\textbf{Елена} я не помню...
\end{itemize} % }

\iusr{Раиса Карчевская}
Прекрасное фото

\iusr{Lara Iwalg}

Какие замечательные снимки! И выражения лиц обеих девочек просто необыкновенно - на первом скованность, на втором - раскованность
Спасибо за рассказ @igg{fbicon.heart.red}

\begin{itemize} % {
\iusr{Светлана Манилова}
\textbf{Lara Iwalg}. спасибо! Именно так! @igg{fbicon.heart.red}
\end{itemize} % }

\iusr{Rimma Turovskaya}
Трогательная история. @igg{fbicon.grin} 

\iusr{Павел Пауль}
66 садик водницкий наш был на Фрунзе за речным техникумом. Фотогграффии придется поискать.

\iusr{Петр Кузьменко}

И вступает \enquote{элита}. 98 садик у Флоровского монастыря и, частично, в самом монастыре.

\iusr{Лариса Захарченко}
Так мило!!!

\iusr{Леся Лагуна}

Какие приятные воспоминания! Спасибо Вам! Фото прелестные.

Я с таким удовольствием, после очередного рассказа - Воспоминания, читаю
комментарии- как приятно, что вы ( многие ) друг друга знаете с детства ,
учились вместе, жили рядом!! Это здорово  @igg{fbicon.hands.raising} 

Мой садик на Жилянской ( 39 или 41). Все там перестроили, здание 3-х этажное
моего садика есть. Правда там какой- то банк. Это уже не важно. Фотографий
море, шикарные были утренники (1965-1970).

Благодарю Вас за прекрасные Воспоминания ! @igg{fbicon.thumb.up.yellow}
@igg{fbicon.hands.applause.yellow}  @igg{fbicon.cat.heart.eyes} 

\iusr{Светлана Манилова}
\textbf{Леся}, спасибо! Делитесь с нами Вашими воспоминаниями.

\iusr{Наталия Дудник}
Прочла с удовольствием. Как всё мило и тепло.

\iusr{Светлана Манилова}
\textbf{Наталия}, спасибо! Тепло, идущее с тех времен. @igg{fbicon.smile} 

\iusr{Алла Гершкович}
Ещё роддом вспомните.

\begin{itemize} % {
\iusr{Олег Коваль}
\textbf{Алла}, что Вас не устраивает? Вы хотите об этом поговорить?

\begin{itemize} % {
\iusr{Алла Гершкович}
\textbf{Олег Коваль} 

почему не устраивает? Я просто высказала свое мнение. Имею право, так, как и вы
на свое. Зачем тогда ставить все эти воспоминание в ФБ? Высказываются все, кто
как думает. Или вы предпочитаете, чтобы с вами все соглашались? Не бывает так.

\iusr{Олег Коваль}
\textbf{Алла}, 

давайте я Вам напомню. Вы сейчас находитесь в группе \enquote{Киевские истории}, и эта
публикация на все 100\% соответствует тематике нашего паблика. Скорей всего, не
ведая того, Вы заблудились в просторах Фейсбука. Это наша территория, нравится
Вам это или нет.

\iusr{Алла Гершкович}
\textbf{Олег Коваль} 

да, ради бога, я не претендую на вашу территорию. Ну, а в ФБ заблудиться
запросто. Ну, удалите меня, пусть вам полегчает.

\iusr{Олег Коваль}
\textbf{Алла}, Вы меня опять таки не поняли. Я хочу, что бы Вам полегчало.

\iusr{Natasha Levitskaya}
\textbf{Алла Гершкович}

\enquote{имею право и высказываются все, как думают} - лучше иногда не думать вообще и тем более высказываться!
Зато у вас есть право выбора - читать то, что вам интересно!


\iusr{Алла Гершкович}
\textbf{Олег Коваль} 

да у меня все впорядке. Не переживайте. В Киеве бываю регулярно. Это мой родной
город. Правда, с каждым разом, все меньше его узнаю. Будьте здоровы!


\iusr{Алла Гершкович}
\textbf{Natasha Levitskaya} вот я и читаю. Не вам судить.

\iusr{Олег Коваль}
\textbf{Алла}, благодарю! Взаимно и хороших выходных.

\iusr{Алла Гершкович}

\ifcmt
  ig https://i2.paste.pics/2057bc998e3078df636058936947e421.png
  @width 0.2
\fi

\end{itemize} % }

\iusr{Светлана Манилова}
\textbf{Алла}, \enquote{в свое время мы подумаем и об этом...} @igg{fbicon.smile} 

\iusr{Алла Гершкович}
\textbf{Светлана Манилова} подумайте.

\iusr{Светлана Манилова}
\textbf{Алла}, идея хорошая. Но роддом Зайцева здесь уже вспоминали не раз. @igg{fbicon.smile} 

\begin{itemize} % {
\iusr{Алла Гершкович}
\textbf{Светлана Манилова} у меня в этом роддоме дочка родилась, а сейчас он где-то на Виноградаре.

\iusr{Светлана Манилова}
\textbf{Алла}, да, на Мостицкой.

\iusr{Алла Гершкович}
\textbf{Светлана Манилова} наверное. Я сейчас плохо в Киеве ориентируюсь. Когда-то было по другому.
\end{itemize} % }

\iusr{Ирина Петрова}

Ой, я теж, теж !!! Народилась у пологовому будинку Жовтневої лікарні! А садочок
139 був на площі Франко, і нікого не можу знайти ні з палати пологового, ні з
садочка! @igg{fbicon.laugh.rolling.floor}{repeat=3} 

\end{itemize} % }

\iusr{Ирина Архипович}

Обожаю такие фото!! Чудесные моменты детства!!  @igg{fbicon.hand.ok}
@igg{fbicon.thumb.up.yellow}  @igg{fbicon.face.happy.two.hands} 

\iusr{Анна Сидоренко}
\textbf{Светлана Манилова} 

вы хорошенькая девочка на фотографии и, что характерно, узнаваемая. Даже в
таком возрасте запоминаются моменты на всю жизнь, наш мозг работает на нас...
Спасибо.

\begin{itemize} % {
\iusr{Светлана Манилова}
\textbf{Анна}, спасибо Вам! Но Вы мне льстите... @igg{fbicon.wink} 

\iusr{Анна Сидоренко}
Чтобы вы знали, Светлана, я лесть ненавижу, зря вы так...

\iusr{Светлана Манилова}
\textbf{Анна}, я в хорошем смысле.@igg{fbicon.heart.red}
\end{itemize} % }

\iusr{Ирина Кудря}
Не побоюсь слов, отныне это моя любимая история о вас. @igg{fbicon.heart.red}

\iusr{Светлана Манилова}
\textbf{Ирина}, спасибо, Ирочка! @igg{fbicon.heart.red}

\iusr{Tamara Malezhyk}
Светлана, спасибо за искренний и трогательный рассказ, сопровождаемый
фотографиями с далекого детства. Успехов и Здоровья.

\iusr{Светлана Манилова}
\textbf{Tamara}, спасибо!@igg{fbicon.heart.red}

\iusr{Ольга Чекрыгина}

Очень нравится, когда фотографии имеют свою историю. Взглянул и сразу всплывают
в памяти звуки, запахи, эмоции. И вспоминаются давно забытые имена. И хочется
поделиться этим чувством.

Вот и Ваша небольшая история вернула многих из нас в свое детство. Спасибо.

\begin{itemize} % {
\iusr{Светлана Манилова}
\textbf{Ольга}, спасибо Вам! Для этого мы и собрались в этой группе. @igg{fbicon.smile} 
\end{itemize} % }

\iusr{Ludmila Karpuk}
Очень хорошая.. на обеих фото..

\iusr{Ирина Иванченко}
Очень милые фотографии, спасибо, улыбнули, Светлана.

\begin{itemize} % {
\iusr{Светлана Манилова}
\textbf{Ирина}, спасибо! Этого и хотелось. @igg{fbicon.smile} 

\iusr{Ирина Иванченко}
\textbf{Светлана Манилова} ,надеюсь, Света, вас не обидели мои шутки под фото, с Ч Ю у вас порядок.

\iusr{Светлана Манилова}
\textbf{Ирина}, в отсутствии его замечена не была! @igg{fbicon.face.tears.of.joy} Наоборот, мне нравится, спасибо!
\end{itemize} % }

\iusr{Тетяна Поляніна}

\ifcmt
  ig https://i2.paste.pics/7162ba31e0543c2855207df13fb56aca.png
  @width 0.3
\fi

\iusr{Кретов Андрей}
Спасибо!

\iusr{Ксения Литвин}
Красиво @igg{fbicon.heart.sparkling} 

\iusr{Аня Трегубова}
Спасибо за откровение. Так душевно!

\iusr{Светлана Манилова}
\textbf{Аня}, спасибо за отклик! @igg{fbicon.smile} 

\iusr{Арт Юрковская}
А у меня было такое же платье!

\begin{itemize} % {
\iusr{Светлана Манилова}
\textbf{Арт Юрковская}, выше в комментариях у одной нашей участницы тоже, как оказалось, было аналогичное. @igg{fbicon.smile} 
\end{itemize} % }

\iusr{Ольга Лубягина}
Знатный кавалер и такие хорошенькие принцессы @igg{fbicon.heart.with.ribbon} 

\iusr{Наталья Лаврухина}
Такие милашечки девченочки.

\iusr{Ирина Петрова}

Светочка, вот даже, если бы фотография появилась без подписи -  @igg{fbicon.100.percent}  я бы узнала
Вас! Чудная, атмосферная история! Может, и правда, найдется этот мальчишечка ,
ведь у нас, в КИ всё возможно!!! @igg{fbicon.heart.eyes}  @igg{fbicon.wink} 

\begin{itemize} % {
\iusr{Светлана Манилова}
\textbf{Ирина}, 

спасибо! Правда, я уже сама себя с трудом узнаю...
@igg{fbicon.face.tears.of.joy} А мальчишечка пусть найдется. Тогда это будет
новая очередная история \enquote{Киевских историй}. @igg{fbicon.smile} 

\iusr{Ирина Петрова}
\textbf{Светлана Манилова} интересно бы сделать челлендж \enquote{Сорок лет спустя} @igg{fbicon.wink} 

\iusr{Светлана Манилова}
\textbf{Ирина}, \enquote{пятьдесят лет спустя}. @igg{fbicon.smile} 

\iusr{Ирина Петрова}
\textbf{Светлана Манилова} Свет, мы не на аукционе - кто больше @igg{fbicon.face.tears.of.joy}{repeat=3}  
деффачкам вообще \enquote{двадцать с половиной спустя} @igg{fbicon.face.tears.of.joy} 
\end{itemize} % }

\end{itemize} % }
