% vim: keymap=russian-jcukenwin
%%beginhead 
 
%%file 15_05_2021.fb.ravreba_maximus.1.bulgakov_kiev
%%parent 15_05_2021
 
%%url https://www.facebook.com/permalink.php?story_fbid=1675045966219125&id=100011412138905
 
%%author 
%%author_id 
%%author_url 
 
%%tags 
%%title 
 
%%endhead 

\subsection{Как Воланд подарил Мастеру Маргариту и покой}
\label{sec:15_05_2021.fb.ravreba_maximus.1.bulgakov_kiev}
\Purl{https://www.facebook.com/permalink.php?story_fbid=1675045966219125&id=100011412138905}

\ifcmt
  pic https://scontent-frt3-1.xx.fbcdn.net/v/t1.6435-9/186448924_1675045802885808_2111577378671165322_n.jpg?_nc_cat=108&ccb=1-3&_nc_sid=730e14&_nc_ohc=SkzQ6973OV8AX-vCZEg&_nc_ht=scontent-frt3-1.xx&oh=a79718a47633e970b1d8fb698a8b9627&oe=60C4E0DD
\fi

Как Воланд подарил Мастеру Маргариту и покой.

Ко дню рождения Михаила Афанасьевича.

Если бы писатели писали все, как есть, интересных романов бы не существовало. Жизнь штука скучная, в ней нет гротеска. 

ПОЭТОМУ БЕЛАЯ ГВАРДИЯ - ВЫМЫСЕЛ.

Шурин Мастера, Леонид Карум с которого написан Тальберг не был подлецом и не бросал жену и сестру писателя Варвару. Прочитав роман рыжая Лена-Варвара порвала с братом и вычеркнула его из жизни. 
Домовладелец Листовничий-Василиса не был трусом и ничтожеством. Он был храбрый человек, сидел в ЧК, стоял у стенки, бежал с парохода-тюрьмы и сгинул.
А каналья-гетман Павел Петрович Скоропадский не бежал переодевшись таинственным майором фон Шраттом. А подал в отставку, сел на паровоз и уехал в Германию.
ЖИЗНЬ ПРОЗАИЧНА.
Но талантливый литератор превращает ее в культурный миф и всегда оставляет придуманную историю вместо реальной. Так, что персонажам реальной истории, остается только ждать участи. Писатель может сделать их героями.
А МОЖЕТ И НЕ СДЕЛАТЬ.
Булгаков мог то, и другое. Это качество очень ценил в талантливом драматурге Булгакове товарищ Воланд-Сталин, который мог из кого угодно сделать Мастера. А мог и не сделать. Не волнуйтесь, товарищи! Тов. Сталин осыпал драматурга Булгакова милостями, сделал его Мастером, подарил в Москве покой, Маргариту и уютную квартиру поэта Мандельштама, который не оправдал оказанного ему высокого доверия трудящихся всех стран. Но нельзя быть добрым. Доброта тов. Воланда свела Мастера в могилу, ибо будучи гедонистом, Мастер заострял перо своего таланта внутривенными инъекциями. Морфiй. Профессор чорной магии об этом прекрасно знал из донесений тов. Агранова, сотрудника ОГПУ. Но больше от коллег и сотрудников гениального Мастера из Массолита, где он и сам работал. Воланду ничего не стоило прекратить эти опасные опыты, арестовав Мастера на основании любого доноса и спасти ему этим жизнь. Но Миша Булгаков был слаб и ничтожен, как Василиса. Он не выдержал бы даже ареста, а не то, что Бутырки. И Воланд махнул рукой, поручив его дни судьбе. Через полгода Мастер умер в прекрасной, московской квартире по улице Фурманова 3/5, в двух шагах от Кремля, Арбата и Пречистенки. Чуть раньше там был арестован Мандельштам. 
Нехорошая квартирка...
ПРОЩАЛИСЬ С МАСТЕРОМ В МАССОЛИТЕ.
Потому что Михаил Булгаков был Берлиоз.
А на доме Турбiныхъ в Киеве, из окна кабинета Миши вывешен желто-синий, петлюровский флаг. В прихожей объявление:
ПРОРОССИЙСКИМ ВХОД НЕЖЕЛАТЕЛЕН.
Может уже и не висит, но история любит жестоко шутить.

Сандро Кириак

А вот и цитата из того времени по нынешнее и в будущее:
«А зачем оно было? Никто не скажет. Заплатит ли кто-нибудь за кровь? Нет. Никто. Просто растает снег, взойдёт зелёная украинская трава... выйдут пышные всходы... задрожит зной над полями, и крови не останется и следов. Дёшева кровь на червонных полях, и никто выкупать её не будет. Никто».
(«Белая гвардия»).

Volodimyr Victorovich

Но единственное, что может спасти смертельно раненного кота, — это глоток
бензина. Поэтому протестую, Булгаков бессмертен!
