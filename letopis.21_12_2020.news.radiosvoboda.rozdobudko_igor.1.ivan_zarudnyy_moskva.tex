% vim: keymap=russian-jcukenwin
%%beginhead 
 
%%file 21_12_2020.news.radiosvoboda.rozdobudko_igor.1.ivan_zarudnyy_moskva
%%parent 21_12_2020
 
%%url https://www.radiosvoboda.org/a/31010955.html
 
%%author 
%%author_id rozdobudko_igor
%%author_url 
 
%%tags moskva,zarudnii_ivan,arhitektura
%%title Киянин Іван Зарудний – архітектор, який запровадив у Москві стиль українського бароко
 
%%endhead 
 
\subsection{Киянин Іван Зарудний – архітектор, який запровадив у Москві стиль українського бароко}
\label{sec:21_12_2020.news.radiosvoboda.rozdobudko_igor.1.ivan_zarudnyy_moskva}
\Purl{https://www.radiosvoboda.org/a/31010955.html}
\ifcmt
	author_begin
   author_id rozdobudko_igor
	author_end
\fi

\index[cities.rus]{Москва!Архитектура, Украинское барокко, Иван Зарудный, 21.12.2020}
\index[names.rus]{Зарудный, Иван!Архитектор, 21.12.2020}

\ifcmt
pic https://gdb.rferl.org/86B9430C-2E8E-4E60-8CCA-CB89CC743EC7_cx0_cy1_cw0_w1597_r1_s.jpg
caption Церква Тихвінської ікони Божої Матері Донського монастиря у Москві, збудована за проєктом українського архітектора Івана Зарудного 
\fi


\begingroup
	\large\em\bfseries
(Рубрика «Точка зору»)

Події Визвольної війни в Україні під проводом Богдана Хмельницького та подальше
існування України в єдиній із Росією державі для політичного життя України мали
негативні наслідки. Але для культурного розвитку Росії вони виявилися вкрай
важливими. Через Україну, завжди до того замкнена для західного впливу Росія
увійшла в контакт із більш розвиненою європейською культурою, а українські
митці та церковні діячі на довгі роки стали керманичами розвитку нової,
європейської культури в Росії.
\endgroup

\begin{leftbar}
	\begingroup
		\em\large\color{orange}\bfseries 
		У ХVII столітті стиль українського бароко можна побачити в Росії скрізь
	\endgroup
\end{leftbar}

Через Україну ознайомилася Росія тоді і з архітектурним стилем європейського
бароко, бо в Україні на ті часи припав розквіт цього стилю, так званого
українського або козацького бароко. У ХVII столітті стиль українського бароко
можна побачити в Росії скрізь, від столиці донського козацтва міста \textbf{Черкаська},
що був заснований за сто років до того українськими козаками-черкасцями, які
прийшли сюди разом зі своїм гетьманом \textbf{Дмитром Вишневецьким}, і до столиці
тодішнього російського Сибіру, міста \textbf{Тобольська}, де збудований в ті роки
місцевий кремль став найсхіднішим взірцем українського архітектурного стилю в
Росії.

\ifcmt
pic https://gdb.rferl.org/BE68A9D5-8325-46CD-822A-CDCC50F67BA2_w1023_r0_s.jpg
caption Воскресенський військовий собор донських козаків, споруджений на початку ХVIII століття за українськими взірцями. Це перший кам’яний собор на Дону

pic https://gdb.rferl.org/742AFEFB-4919-4CBC-BD58-383873CA5435_w1023_r0_s.jpg
caption Воскресенський військовий собор донських козаків, споруджений на початку ХVIII століття за українськими взірцями, в колишній столиці донського козацтва – місті Черкаську, що було засноване українськими козаками-черкасцями (нині Старочеркаська станиця Ростовської області)

pic https://gdb.rferl.org/BCA4EEC9-CB17-4A1E-81CF-EF5AE48681C1_w1597_r0_s.jpg
caption Воскресенський військовий собор донських козаків, споруджений на початку ХVIII століття за українськими взірцями, в колишній столиці донського козацтва – місті Черкаську, що було засноване українськими козаками-черкасцями (нині Старочеркаська станиця Ростовської області)

pic https://gdb.rferl.org/C71EBF71-C0B2-46BE-9281-CE6AE52511EF_w1023_r0_s.jpg
caption Софійсько-Успенський собор Тобольського кремля, зразок стилю українського бароко в російському Сибіру

pic https://gdb.rferl.org/28B1DA26-BFE0-4F0B-BF29-F0AE5E52CDC1_w1023_r0_s.jpg
caption Софійсько-Успенський собор Тобольського кремля у стилі українського бароко
\fi

Розбудовувалася в українських традиціях і тодішня Москва, де під впливом
українського, козацького бароко, виник місцевий, оригінальний стиль так званого
«московського бароко». Архітектором, що насадив цей стиль у Москві, був
виходець з України \textbf{Іван Зарудний}.

Мені, як мешканцю Москви, завжди милують очі та серце ці будівлі, чий виразний
стиль питомо відрізняється від усіх інших будівель російської столиці. Тут
відчуваються і традиції українського бароко, і щось інше, притаманне особисто
Зарудному.

\begin{leftbar}
	\begingroup
		\em\large\color{orange}\bfseries 
Гетьман Іван Мазепа у 1690 році направив Зарудного до Москви, де він і створив найбільші архітектурні шедеври
	\endgroup
\end{leftbar}

Народився Іван Зарудний у Києві, близько 1670 року, в заможній козацькій
родині. Навчався у \textbf{Києво-Могилянській академії}. Після цього студіював в Італії,
вивчав європейське декоративне мистецтво. Після повернення поступив на службу
до канцелярії гетьмана \textbf{Івана Мазепи}. На прохання \textbf{Петра I} гетьман Мазепа у 1690
році направив Зарудного до Москви, де він і створив найбільші архітектурні
шедеври.

З найбільш відомих – так звана «Меншикова вежа». Фаворит Петра I, \textbf{Олександр
Меншиков} дуже хотів, аби його особиста церква у Москві мала шпиль більший, ніж
шпиль Петропавлівського собору в Петербурзі. За легендою, це дуже не
сподобалося царю Петру, і шпиль з будівлі Зарудного було знято (за іншою
версією, в шпиль потрапила блискавка та спалила його).

\ifcmt
tab_begin cols=3
	caption Башня Меньшикова, Москва, Архитектор Иван Зарудный

pic https://gdb.rferl.org/355B963D-1CBA-4BB1-9EEF-B60C7B4AC501_w650_r0_s.jpg
caption Первинний вигляд «Меншикової вежі» у Москві, яку було споруджено за проєктом українського архітектора Івана Зарудного. Була замовлена у 1707 році

pic https://gdb.rferl.org/45ADBF52-1489-44C2-9876-20182BDB5659_w650_r0_s.jpg
caption Сучасний вигляд «Меншикової вежі» у Москві, яку було споруджено за проєктом українського архітектора Івана Зарудного

pic https://gdb.rferl.org/F093AC33-187C-48EC-A87D-1D10110E352A_w650_r0_s.jpg
caption Сучасний вигляд «Меншикової вежі» у Москві, яку було споруджено за проєктом українського архітектора Івана Зарудного

tab_end
\fi

В подальші роки Меншикова вежа стала місцем зборів московських масонів, а в
наші роки є Подвір’ям Антіохійської православної церкви.

\index[names.rus]{Меньшиков, Александр!Фаворит Петра Первого}

\ifcmt
pic https://gdb.rferl.org/9B365E9A-A9A4-4290-9860-CEB20892AF0F_w650_r0_s.jpg
caption Олександр Меншиков, на замовлення якого Іван Зарудний збудував «Меншикову вежу» в Москві
\fi

Побудував Зарудний і Спаський собор Заіконоспаського монастиря поблизу Кремля –
тут перебувала відома Слов’яно-греко-латинська Академія, перший заклад вищої
освіти у Росії.

\ifcmt
pic https://gdb.rferl.org/5834D7FF-FAD2-404A-A588-9BFA21890FA5_w650_r0_s.jpg
caption Спаський собор Заіконоспаського монастиря на території Слов’яно-греко-латинської академії у Москві. Собор споруджено за проєктом українського архітектора Івана Зарудного
\fi

З інших його відомих споруд, що збереглися в Москві до нашого часу, можна
назвати церкву Іоанна Воїна на Якиманці (навпроти сучасного посольства
Франції), та храм Тихвінської ікони Божої Матері Донського монастиря.

\ifcmt
pic https://gdb.rferl.org/D10AE557-ABF5-4BD3-A8C5-1540CFFCC7E1_w1023_r0_s.jpg
caption Церква Івана Воїна на Якиманці (Москва)

pic https://gdb.rferl.org/5663EB2C-3DB2-4BBD-AA10-57DAA0F69FAA_w1023_r0_s.jpg
caption Церква Тихвінської ікони Божої Матері Донського монастиря (Москва)
\fi

Працював Іван Зарудний також у Петербурзі та Таллінні, де створив чудовий
іконостас, що є зараз у Преображенському.

\ifcmt
pic https://gdb.rferl.org/F5B313DA-E480-4DF6-A262-74283D285285_w1023_r0_s.jpg
caption Естонія. Іконостас Івана Зарудного у Преображенській церкві в Таллінні. Храм Естонської православної церкви Константинопольського патріархату
\fi

Слава Зарудного в Росії була настільки велика, що у 1707 році він очолив
ізографічну палату при Сенаті, тобто став головою усіх архітекторів та
художників Російської імперії. Але, разом з тим, до України його зайвого разу
намагалися не відпускати. Помер Іван Зарудний у Петербурзі 1727 року, а пам’ять
про нього живе у його творах і зараз. Пишаємося його творами і ми, російські
українці.

\begingroup
	\em
\textbf{Ігор Роздобудько} – історик, перекладач, член Малої Ради Громади українців Росії

Думки, висловлені в рубриці «Точка зору», передають погляди самих авторів і не
конче відображають позицію Радіо Свобода
\endgroup
