% vim: keymap=russian-jcukenwin
%%beginhead 
 
%%file 24_11_2021.fb.taksjur_jan.1.radikalnoje_sredstvo
%%parent 24_11_2021
 
%%url https://www.facebook.com/taksyur/posts/4775502959162489
 
%%author_id taksjur_jan
%%date 
 
%%tags 2014,maidan2,pamjat,rasskaz,taksjur_jan.pisatel,ukraina
%%title РАДИКАЛЬНОЕ СРЕДСТВО
 
%%endhead 
 
\subsection{РАДИКАЛЬНОЕ СРЕДСТВО}
\label{sec:24_11_2021.fb.taksjur_jan.1.radikalnoje_sredstvo}
 
\Purl{https://www.facebook.com/taksyur/posts/4775502959162489}
\ifcmt
 author_begin
   author_id taksjur_jan
 author_end
\fi

Написал семь лет назад. Потом забыл. А в связи с недавней годовщиной вспомнил.

\headCenter{РАДИКАЛЬНОЕ СРЕДСТВО}

\obeycr
– Садитесь и расслабьтесь. Что вас беспокоит?
– Видите ли, мне какой-то голос...
– Слышите голоса?
– Нет-нет, только один голос.
– И что он вам говорит?
– Да разное. Чаще всего, задаёт вопросы.
– И это вас беспокоит?
– Честно говоря, беспокоит. Потому что я не всегда могу дать ответ.
– Ну, например, сегодня о чём он спрашивал?
– О европейской ориентации. Мол, наши земляки во время революции показали своё стремление в Европу.
– Именно так. Всё правильно.
– Да, но он спрашивает, в чём это проявилось.
– Разве вы не помните? Это же было не так давно.
– А я забыл. Видимо, от недоедания. И, знаете, так ещё провокационно ставит вопрос: что для тебя важнее, ассоциация с Европой или колбаса?
– А вы что?
– Я говорю, конечно, ассоциация. Тем более, купить колбасу я не могу себе позволить.
– И голос тогда умолкает?
– Где там! Продолжает мучить вопросами.
– О чём?
– О преступном режиме.
– В каком смысле?
– В том смысле, что преступный режим, слава богу и нашим стратегическим партнёрам, мы свергли, а жить стало ещё тяжелее.
– Странно. Вы получили «Памятку участника революции»?
– Может, и получал. Я же говорю, от недоедания стал всё забывать. И достижения революции, и где эта памятка, тоже не помню.
– Хорошо, вот вам новая. И ещё я выпишу таблетки для памяти. Они быстро восстанавливают в сознании все наши достижения.
– Огромное спасибо! Разрешите, я сразу одну приму. О! Совсем другое дело! Теперь я всё припоминаю. И памятку эту красивую узнаю. Тут же всё написано!
– Конечно.
– А он меня изводил! Особенно, вот этим четвёртым вопросом. А здесь сразу ответ можно прочитать!
– Кстати, чтобы снять напряжение, можете прочесть ответ вслух.
– С удовольствием! «Если какой-то нестойкий элемент спрашивает вас, чем предыдущие воры и денежные мешки были хуже, чем новые и сто̀ило ли из-за новых проливать кровь и разрушать государство, отвечайте: новых признаёт цивилизованный мир, у них приятные лица и они мои единомышленники».
– Совершенно верно.
– Ой!
– Что случилось?
– Только что он спросил, какая разница, какой ворюга тебя грабит, симпатичный или несимпатичный? А тут нет ответа. Что ему сказать?
– Скажите, что разница колоссальная.
– И всё?
– Всё. Сказали?
– Сказал.
– Ну? 
– Смеётся. Говорит, что я идиот. Что повёлся на ложь тех, кто делит мою землю. Что с помощью продажный писак мне запихали в голову, будто я лучший на планете. И, если я покрашу свой забор в правильные цвета, моя жизнь станет волшебной и прекрасной.
– А вы загляните в памятку.
– Минутку. Ага, есть! "Если спрашивают, почему после покраски забора гривна не пошла вверх, отвечайте: кремлёвское чудовище обрушило курс".
– Абсолютно точно.
– Смотрите-ка, здесь и о рыбе в Днепре есть, и о наводнениях! И всё это проклятое чудовище?
– К сожалению.
– А как зовут этого злодея?
– Примите ещё таблетку.
– Спасибо. И всё-таки, что это за голос во мне говорит?
– Вас нужно обследовать. Сделать анализы. Как врач, пока ничего сказать не могу.
– А может, это моя совесть?
– Совесть? Нет, уважаемый. Совесть – это галлюцинация. В Европе её давно ликвидировали химически.
– Ой, ой! А он сейчас говорит: да, я твоя совесть!
– Тогда на ночь проглотите ещё вот эту синюю таблетку и он, наконец, заткнётся. Вы куда сейчас направляетесь?
– Иду на праздник. Юбилей Свержения Преступного Режима.
– Какого именно? У нас часто свергаются преступные режимы.
– Не помню. С момента последнего свержения, я ничего не ел.
– Что ж, до свиданья. Счастливого пути.
– И вам всего хорошего.
  Пациент покинул кабинет, а доктор подошёл к окну. За окном шли демонстранты. Слышались голоса: «Да здравствует Трёхмесячный Юбилей Свержения Преступного Режима! Вперёд к новым победам!».
– А потом они вернутся домой и начнут терзаться сомнениями, – подумал доктор, глядя на колонну с самодельными плакатами. – Вот если бы найти такое лекарство, такую вакцину, которую можно было бы прописать всем и разом прекратить страдания.
  Он вошёл в небольшую лабораторию рядом с кабинетом, продолжая размышлять.
– Нужно искать. Потому что после таблеток, памяток и манифестаций они, голодные, возвращаются в свои подвалы, и желудочный сок, который не смог выделиться, начинает влиять на мозг. Отсюда сомнения, вопросы. Я просто обязан найти какой-нибудь радикальный способ. Раз и навсегда. И тогда мистер Муланд оплатит моё открытие. Не обязательно деньгами. Кому они сейчас нужны? Можно продуктами питания.
Доктор представил огромную толпу счастливых людей. Не знающих сомнений, поскольку приняли его чудодейственное лекарство.
– А будут ли они людьми? – неожиданно спросил голос внутри.
– Кто это? – заволновался доктор.
– Ты знаешь.
– Я сейчас приму таблетку и тебя не станет!
  Он проглотил синюю таблетку и закрыл глаза. Он знал, что голос скоро замолчит. Но действия хватит часа на два-три. Потом снова мука.
– Нет, нужно что-то окончательно подавляющее, – решил доктор, открывая глаза. – Чтобы эти голоса замолкли навеки. И нужно писать докладную мистеру Муланду. Моё изобретение может решить судьбу целой страны. Пусть даёт аванс. Может, не кинет, сволочь, как в прошлый раз. В конце концов, не могут же они кидать постоянно. Будем на это надеяться. Во всяком случае, больше нам надеяться не на что.
2014 г.
\restorecr

\ii{24_11_2021.fb.taksjur_jan.1.radikalnoje_sredstvo.cmt}
