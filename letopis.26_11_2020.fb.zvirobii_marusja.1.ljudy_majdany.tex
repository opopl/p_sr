% vim: keymap=russian-jcukenwin
%%beginhead 
 
%%file 26_11_2020.fb.zvirobii_marusja.1.ljudy_majdany
%%parent 26_11_2020
 
%%url https://www.facebook.com/marusyazvirobiy1/posts/3458481807533498
 
%%author Звіробій, Маруся
%%author_id zvirobii_marusja
%%author_url 
 
%%tags 
%%title Першим відкриттям, що мене шокувало, були люди Майдану
 
%%endhead 
 
\subsection{Першим відкриттям, що мене шокувало, були люди Майдану}
\label{sec:26_11_2020.fb.zvirobii_marusja.1.ljudy_majdany}
\Purl{https://www.facebook.com/marusyazvirobiy1/posts/3458481807533498}
\ifcmt
	author_begin
   author_id zvirobii_marusja
	author_end
\fi

Першим відкриттям, що мене шокувало, були люди Майдану. Це був культурний шок.
Я не уявляла, що навколо мене є стільки надзвичайно мужніх громадян. Тоді мені
здавалось, що вся країна така як майданівці і я стала нею надзвичайно пишатися.
А ті нові для мене воїни вмотивували мене безстрашно стати поряд з ними в
оборону.

\ifcmt
pic https://scontent.fiev6-1.fna.fbcdn.net/v/t1.0-9/127841615_3458480907533588_1725130671955332739_o.jpg?_nc_cat=104&ccb=2&_nc_sid=730e14&_nc_ohc=VLQYdXWMZRoAX85_wi3&_nc_ht=scontent.fiev6-1.fna&oh=af63552e524b7f585ba6bbd57b7f987e&oe=5FF076C3
\fi

Друге відкриття було неприємне, під час виборів зеленського, я побачила скільки
в цій державі ватників, какойразніци і \enquote{однаково як буде, аби не Порошенко, хоч
паржом}. Надзвичайна недалекоглядність і відсутність найелементарнішої
аналітики передвиборчих процесів. Ну вибачим то звісно і тепер якось будемо
вирулювати, але це був теж шок для мене. Це не та країна, яка зробила майдан. В
ній вже відбулися певні глибинні зміни. Перемогла малоросія з всраними
пропагандою мізками.

Третій шок відбувається зараз. Я бачу забагато страху в очах. І це стадія дуже
небезпечна. Адже наші надмірно толерантні громадяни, нездатні рубитися на рівні
з агресивною гопотою, яка тягне країну до краху і тут не вийде по-хорошому. То
є більшовики, наглі, грубі, і зовсім не толерантні. Толерантність це лише ген
рабства, піднятий в нас нашим ворогом до рівня чесноти. Какая разніца, ми ж
такі благородні, да, ні з ким намагаємось не сваритись? 

Вважаю толерантність, як і рускій язик, інструментами окупації територій, які
росія дуже ефективно використовує проти нас. 

Пишатися своєю толерантністю і небажанням іти на конфлікти сьогодні, то
найбільша тупість. Адже це величезний самообман, то просто приховання свого
страху. Тваринного страху, перед МВС, СБУ, судами, владою, опонентами. 

Я ненавиджу такий страх. Дуже сильно відчуваю його навколо себе сьогодні. І
саме в ньому тепер наш параліч, який не дає здвинути ситуацію з місця. 

Окрім того, хочу пояснити тим, в кого різко прокинувся, або й завжди був
синдром білого пальтішка, якщо у вас нема ворогів, ви ні з ким не сваритесь,
можливо ви... покидьок??? Адже це означає, що ви не закрили рота брехуну, який
би в такому випадку на вас образився, що ви не помстилися зраднику, не наказали
злочинця і повз будь-якої несправедливості пропливаєте мовчки. Так розквітає
несправедливість, корупція, злочинність і сепаратизм. За мовчазної згоди і
небажання сваритись чи боязні боротись. А потім в плачевній ситуації держави
винні всі, окрім самих себе.

Будьте мужніми, адже на нас дивляться наші діти, вони завтра будуть такими,
якими сьогодні бачать нас. 

Поборемось!

\subsubsection{Комментарии}

\begin{itemize}
\item \textbf{Людмила Крижановська}
Ніяких компромісів, каzapів будемо нищити до віку!

\item \textbf{Роман Дмитрович Давиденко}
Немає страху, є ненависть до тих хто опустився до рівня блазнів і тим підриває віру.

\item \textbf{Ирина Барицкая}
Поборемось, однозначно! Але народу потрібні лідери для боротьби, інакше не буде
толку. Мають бути ті, які кинуть клич і організують. Тоді вже все
і здвинеться... 

\item \textbf{Bogdan Severin}

Про толерантність - прямо в ціль! 
Херове це діло. Воно людину демотивує щось робити.

\item \textbf{Олена Єрьоменко}

Не бійся ворогів - у гіршому випадку вони можуть тебе вбити.
Не бійся друзів - у гіршому випадку вони можуть тебе зрадити.
Бійся байдужих - вони не вбивають і не зраджують, але тільки з їхньої мовчазної згоди існують на землі зрада і вбивство.
- Бруно Ясенський , «Змова байдужих» 1937

\item \textbf{Марія Маринченко}

Люди Майдану до кінця сформували мою ідентичність, національні цінності. Це
було щось надзвичайно важливе і особливе, коли я там кожною
клітинкою свого тіла відчувала, що я - не одна,а зі мною разом
- весь патріотичний Український Народ. Той, хто там не був,
ніколи цього не зрозуміє.
\end{itemize}
