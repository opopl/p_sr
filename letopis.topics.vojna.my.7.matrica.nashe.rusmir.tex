% vim: keymap=russian-jcukenwin
%%beginhead 
 
%%file topics.vojna.my.7.matrica.nashe.rusmir
%%parent topics.vojna.my.7.matrica.nashe
 
%%url 
 
%%author_id 
%%date 
 
%%tags 
%%title 
 
%%endhead 

\paragraph{22:18:58 20-08-22 Наталія Владюкreplied to Владимир}
%img Screenshot from 2022-08-20 22-18-26.png

Ви що дійсно настільки придурки тупі? Мені вас шкода, що ви тупі дебіли нічого
не розумієте. Бажаю вам вашого руського міру який ви принесли нам, в кожну хату
, доти скільки ви будете ходити по цій землі. Що іваше покоління лягало і
вставало з цим миром

\paragraph{21:13:56 21-08-22 Tatjana Latviete}
%img Screenshot from 2022-08-21 21-14-23.png

Есть мир нормальных людей. В нем действуют простые и понятные истины: война – это преступление против человечества, свобода лучше, чем несвобода, человеческая жизнь – бесценна. В нем родители любят детей и заботятся об их будущем. И есть «русский мир». В нем все вывернуто наизнанку, а потом перекручено и разорвано. В нем война воспринимается как необходимость и благо. В нем требуют больше убийств. В нем родители определяют своим детям путь жертв ради своих идей.
Вот семья с новенькой белой «Ладой». Они купили ее на деньги, выплаченные за убитого сына, который поехал воевать в Украину. Белая «Лада» – символ идеи этой части русского мира. Где существование серо и уныло и смерть на войне становится смыслом жизни. Родители отправляют детей на смерть – за деньги, за Путина, за то, чтобы не дать соседней стране жить по своему выбору. Они не смогли дать им другого будущего.
Вот российский сенатор, который отрекается от собственной дочери, потому что она не поддерживает «спецоперацию». Это тоже обитатель «русского мира», довольно типичный. Он предает своего ребенка, чтобы не запятнать карьеру. В этом отсеке «русского мира» деньги и власть превыше всего.
Вот главный идеолог этого самого «русского мира» Александр Дугин. «Кирилл и Мефодий фашизма», интеллектуал, призывавший к войне, фанатик построения евразийской империи любой ценой. Ценой оказалась жизнь его дочери. Педагогика созерцания трупа является важнейшей частью духовного созревания личности, утверждал Дугин. Стоит ли сочувствовать тому, кто получил столь страшную возможность для духовного развития? Все-таки да. Как отцу, потерявшему дочь. Все идеи рассыпаются в прах, когда смерть приходит в собственный дом. Но это тоже был выбор родителя – именно такое будущее. Когда он из ребенка растил солдата русского мира, готовил бойца идеологического фронта, погружая в поток своих идей, ценностей и верований.
«Русский мир» беспощаден, он требует бесконечных жертв и пожирает всех, включая собственных обитателей. Он лишает будущего, он не привносит в действительность ничего, кроме смерти.
