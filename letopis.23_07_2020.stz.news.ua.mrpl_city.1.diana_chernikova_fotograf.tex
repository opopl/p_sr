% vim: keymap=russian-jcukenwin
%%beginhead 
 
%%file 23_07_2020.stz.news.ua.mrpl_city.1.diana_chernikova_fotograf
%%parent 23_07_2020
 
%%url https://mrpl.city/blogs/view/diana-chernikova-prosto-lyubit-sebe-kozhna-z-nas-osobliva
 
%%author_id demidko_olga.mariupol,news.ua.mrpl_city
%%date 
 
%%tags 
%%title Фотограф з Маріуполя Діана Чернікова: "Просто любіть себе, кожна з нас особлива!"
 
%%endhead 
 
\subsection{Фотограф з Маріуполя Діана Чернікова: "Просто любіть себе, кожна з нас особлива!"}
\label{sec:23_07_2020.stz.news.ua.mrpl_city.1.diana_chernikova_fotograf}
 
\Purl{https://mrpl.city/blogs/view/diana-chernikova-prosto-lyubit-sebe-kozhna-z-nas-osobliva}
\ifcmt
 author_begin
   author_id demidko_olga.mariupol,news.ua.mrpl_city
 author_end
\fi

У липні своє професійне свято відзначили творчі люди, які по-іншому бачать світ
через оптику. Наразі в Маріуполі працює безліч креативних і професійних
фотографів, кожен з яких має свій стиль і оригінальний підхід до своїх моделей.
Але я вирішила написати про дівчину, яка допомагає маріупольчанкам розкритися
по-новому, відчути себе найпривабливішими і повірити, що вони дійсно особливі.

%\ii{23_07_2020.stz.news.ua.mrpl_city.1.diana_chernikova_fotograf.pic.1}

\ifcmt
  tab_begin cols=2,no_fig,center,separate,no_numbering

  ig https://mrpl.city/uploads/posts/redactor/aobw9djk4ml3ovsb.jpg
  ig https://mrpl.city/uploads/posts/redactor/8kxtqodl48fb9nob.jpg

  tab_end
\fi

Мова піде про \emph{Діану Чернікову – жіночого фотографа Маріуполя.} На своїй сторінці
в Instagram вона підкреслює, що з нею комфортно працювати і, що вона розкриє
жіночність своїх моделей. Ці слова є цілковитою правдою, адже вона справжня
майстриня чарівних перевтілень дівчат Маріуполя, час з якою летить непомітно. В
арсеналі Діани більше 50 різних ракурсів та прикладів для позування, які
допомагають розкрити модель і показати її найкращі сторони, а також свій набір
інструментів для того, щоб швидше налагодити контакт. У Діани моделі абсолютно
різні, як і фотографії, які роблять щасливішими своїх власниць. І незважаючи на
різний вік, особливості статури, моделей нашої героїні об'єднує вміння вражати
природною красою. Хоча, коли дивишся на готову світлину Діани, не завжди думаєш
про процес зйомки, адже всі дівчата неначе зійшли з обкладинок дорогого глянцю.
Втім, щоб отримати таке неймовірне фото, потрібно повністю довіритися
фотохудожниці, яка допоможе не переживати перед камерою, порадить, що одягнути
і як показати себе з кращих сторін. Фотозйомка у Діани – це і свого роду
психологічне розвантаження. Її легкість, вишуканий смак, готовність до
справжньої співпраці і бажання отримати унікальне фото – роблять справжні
дива...

%\ii{23_07_2020.stz.news.ua.mrpl_city.1.diana_chernikova_fotograf.pic.2}

Народилася Діана в Маріуполі, який вже давно став для дівчини найулюбленішим і
найріднішим містом. За освітою вона перекладач. Займатися фотографією почала ще
в школі, в старших класах. На час навчання в університеті залишила улюблену
справу. Сказати що мріяла стати фотографом, дівчина не може. Спочатку у неї
були інші цілі. Загалом вона самоучка, ніякої спеціальної школи фотографії не
закінчувала, але завжди відрізнялася бажанням постійно працювати над
вдосконаленням своїх навичок. Дуже цінний досвід вона отримала, поки працювала
у столичній компанії фотографів різного рівня в одній з фотостудій. Там був
великий потік людей, дуже різних, з різним статусом і характером, різного віку
(від одного місяця до 80 років). І до кожного потрібен був свій підхід. Так
Діана навчилася ще більше розуміти людей, намагалася до кожного підібрати
індивідуальний підхід.

\ii{insert.read_also.demidko.5_misc_priazovja}
\ii{23_07_2020.stz.news.ua.mrpl_city.1.diana_chernikova_fotograf.pic.3.vmeste}

Чотири роки дівчина \enquote{набивала руку} і клієнтську базу, без перерви, працюючи
майже щодня, крім карантину. Сьогодні фотографія займає більшу частину життя
Діани, адже, крім зйомок є ще і постобробка, яка триває не одну годину.
Фотохудожниця розуміє, що все приходить з досвідом і наразі у неї в планах
пройти безліч курсів, що посприяють вдосконаленню її майстерності. Однак
дівчина, сама того не усвідомлюючи, вже досягла дуже високого рівня. Вона вміє
не тільки з усіма знайти спільну мову, але й виявити найпривабливіші риси та
підібрати оптимальний ракурс для кожної моделі. Пози самі з'являються в її
голові як картинка, все залежить від образу моделі, від стилю. Є в арсеналі
Діани і стандартні пози, але у кожної свій ракурс, тому навіть стандартні
виглядають по-різному. Вражає, що після роботи з Діаною у дівчат підвищується
самооцінка і вони стають впевненішими.

\ii{23_07_2020.stz.news.ua.mrpl_city.1.diana_chernikova_fotograf.pic.4}

Окрім фотографії, яка останнім часом займає весь вільний час дівчини, вона
також любить читати, вивчати нову інформацію, малювати, готувати якісь
смаколики та займатися квітами. Діана обожнює Маріуполь. Де б вона не була, їй
завжди хочеться повернутися додому, а місто зараз стало привабливішим, тому
з'явилося більше гарних місць для фотосесій.

\ii{23_07_2020.stz.news.ua.mrpl_city.1.diana_chernikova_fotograf.pic.5}

Надихають маріупольчанку люди, їхня діяльність, творчість. Кожна модель для неї
особлива, в кожній є своя родзинка. Також вона відчуває велику віддачу від
дівчат, тому вважає, що фотограф і клієнт – це завжди команда. Дівчина
найбільше радіє, коли люди повертаються знову і знову, радять її своїм рідним
та друзям.

\ii{23_07_2020.stz.news.ua.mrpl_city.1.diana_chernikova_fotograf.pic.6}

Сім'я і близькі завжди підтримували Діану. Загалом їхня підтримка стала першим
поштовхом до створення кар'єри фотографа. Сьогодні вона вважає себе
по-справжньому щасливою, адже робота приносить їй колосальне задоволення. На її
думку, важливо, щоб кожна людина займалася тим, що їй подобається, тоді всі
зможуть відчути себе щасливішими.

\ii{23_07_2020.stz.news.ua.mrpl_city.1.diana_chernikova_fotograf.pic.7}

%\ifcmt
  %tab_begin cols=2,no_fig,center,separate,no_numbering

  %pic https://mrpl.city/uploads/posts/redactor/z78uukpz3m6p8opd.jpg
  %pic https://mrpl.city/uploads/posts/redactor/kdltrukkdkbxj0tb.jpg

  %tab_end
%\fi

\begingroup
\em
\textbf{Улюблене місце:} море.
\par\bigskip

\textbf{Улюблений фільм:} \enquote{Любов і голуби} (1985 рік).
\par\bigskip

\textbf{Улюблена книга:} \enquote{Магія ранку} Гела Елрода.
\par\bigskip

\textbf{Порада маріупольчанкам:} 
\par\bigskip

\begin{quote}
\enquote{Головне – не соромитися і перестати переживати! Якщо на вас дивляться люди, як
ви фотографуєтесь десь в людному місці, вони забудуть про вас через 15 хвилин,
а фотографії ви будете переглядати ще дуже довго з цього місця. Немає межі
досконалості, всі ми завжди хочемо виглядати краще, але найголовніше – любити
себе, адже кожна з вас особлива! Я вважаю, що людей нефотогенічних немає,
просто не знайшли потрібний ракурс. І, щоб полюбити себе, перестати
комплексувати, потрібно ходити на фотосесії, і дивитися потім на себе з боку в
різних образах і жанрах. Просто спробуйте!}.
\end{quote}
\endgroup

\ii{insert.read_also.demidko.tetjana_dubinina}
\ii{23_07_2020.stz.news.ua.mrpl_city.1.diana_chernikova_fotograf.pic.8}

\textbf{\emph{Фото з архіву Діани Чернікової.}}
