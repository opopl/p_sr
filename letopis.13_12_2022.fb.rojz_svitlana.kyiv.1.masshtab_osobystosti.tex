% vim: keymap=russian-jcukenwin
%%beginhead 
 
%%file 13_12_2022.fb.rojz_svitlana.kyiv.1.masshtab_osobystosti
%%parent 13_12_2022
 
%%url https://www.facebook.com/svetlanaroyz/posts/pfbid0rf3d4izqVf9a1RPJQLB4YSQDLiGz2EYnq1VdrYRCKCgjoaoE53j7AhdfyVedD8aTl
 
%%author_id rojz_svitlana.kyiv
%%date 
 
%%tags 
%%title Масштаб особистості
 
%%endhead 
 
\subsection{Масштаб особистості}
\label{sec:13_12_2022.fb.rojz_svitlana.kyiv.1.masshtab_osobystosti}
 
\Purl{https://www.facebook.com/svetlanaroyz/posts/pfbid0rf3d4izqVf9a1RPJQLB4YSQDLiGz2EYnq1VdrYRCKCgjoaoE53j7AhdfyVedD8aTl}
\ifcmt
 author_begin
   author_id rojz_svitlana.kyiv
 author_end
\fi

Масштаб особистості. Зараз в багатьох розмовах виникає ця тема. Думаю, ми
завжди його відчуваємо за всіма словами та діями людей. Напевно, він більше,
ніж межі професії. Напевно, поєднаний з особистою силою - яка пов'язана з
рівнем відповідальності, який ми можемо на себе взяти. Можливо, з тим
внутрішнім об'ємом, який вже досліджували. 

Коли мене питають - \enquote{ну навіщо \enquote{він/вона} це робить, як їм
вдається таке створювати? Напевно щось компенсує} І починаються пошуки чогось,
щоб знецінило внески цих людей чи проектів - я відповідаю відверто: Може ця
людина набагато масштабніша за нас і ми просто в собі не можемо її вмістити? 

Коли \enquote{масштаб} людини набагато більший - це як коли попадаєш в простір, в якому
не бачиш кордонів, його складно проконтролювати, і це викликає тривогу. 

Ми ставимо свої внутрішні \enquote{мітки} всередині людини. Відзначаючи її дії, слова,
ставлячи діагнози чи наділяючи часто своїм змістом. Шукаючи її слабкість, іноді
зловтішаючись, коли вона робить помилки, виявляється \enquote{просто людиною}, бо це
наближає її до нас.  Іноді намагаючись її спростити для себе, а іноді наділяючи
надмірною вагою чи сенсами - проектуючи на неї свою не прийняту і не засвоєну
силу.

А вона і є просто людиною. Але людиною, яка ризикнула взяти більше
відповідальності, ризикнула більше бути \enquote{собою}. 

Я люблю зустрічі з людьми, які \enquote{більше за мене}. Коли торкаюсь до їх думок,
проектів - це робить і мене більше. 

Я кілька разів слухала Нобелівську промову Олександри Матвійчук. Хотілось
слухати стоячі і плакати, і аплодувати кожній фразі. Бо масштаб, який
відчувався за кожною її фразою - величезний. 

Ми живемо в складний і страшний час - але жити саме в час, коли твориться
історія, коли відбувається зсув у всіх площинах, коли ти відчуваєш, як світ
бачить велич твоєї Країни і ти щодня щохвилини відчуваєш велич тих, знайомих і
незнайомих, хто її захищає, зміцнює -  це дивний дар для нас. Я хочу це
запам'ятати кожною клітиною. 

Можливо, ці часи роблять всіх нас масштабніше? 

Колись я запам'ятала фразу (можливо, це фраза Ошо, не можу знайти зараз автора) 

\enquote{Ти не один. Саме в тобі доля цього світу}.

Можливо, ми стаємо \enquote{масштабнішими}, коли відчуваємо, як своїми думками, діями,
виборами пов'язані із цією долею? 

Обіймаю, Родино ❤️ як бажаю нам відчувати свою силу. І як хочу Перемоги

%\ii{13_12_2022.fb.rojz_svitlana.kyiv.1.masshtab_osobystosti.orig}
%\ii{13_12_2022.fb.rojz_svitlana.kyiv.1.masshtab_osobystosti.cmtx}
