% vim: keymap=russian-jcukenwin
%%beginhead 
 
%%file 08_04_2021.fb.makarenko_oksana.1.bog_ukraincy
%%parent 08_04_2021
 
%%url https://www.facebook.com/permalink.php?story_fbid=2894967667455203&id=100008259933050
 
%%author 
%%author_id 
%%author_url 
 
%%tags 
%%title 
 
%%endhead 

\subsection{ОТАК - хто Він для людей – це зрозуміло а хто для Нього – люди?}
\Purl{https://www.facebook.com/permalink.php?story_fbid=2894967667455203&id=100008259933050}

отак прокинутися о шостій чи навіть о п'ятій ранку
і дивитися як поступово прозорішає фіранка
глибоко вдихати і видихати – дооовго-предооовго
думати про Бога і про себе про себе і про Бога
як Йому живеться – ну просто цікаво – одразу ніде і всюди
хто Він для людей – це зрозуміло а хто для Нього – люди?
їхня любов і ненависть – це добро і зло – весь сукупний досвід
і коли Він їх зазвичай перераховує – ввечері чи вдосвіта
сидить собі такий у кімнаті із голубими шпалерами
і нехай у нього буде помічник – секретар – скажімо Валєра
от Він задумується і каже – Валерію… а що там у нас українці?
той приносить книгу і розкриває на потрібній сторінці
Господь тоді трохи супиться підпирає бороду мружить очі
гортає сторінку за сторінкою – так нібито не дуже й хоче
чиркають комети об стелю над самісінькими головами
тут Він її закриває і каже – Валерію… своїми словами
і Валерій трохи картинно виходить на середину кімнати
розправляє плечі відводить ногу – Ви хотіли історій? – нате!
і починає розказувати – спочатку впівголоса – дуже тихо
про те що жити в Україні – це водночас і щастя і лихо
його голос росте і кріпне і за якусь мить виходить на ноти
ніби це шабельна атака – ніби лавою йде кіннота
і ти розумієш що від неї вже немає спасіння
тільки смерть а по смерті – обіцяне воскресіння
він розказує все так – ніби вже й сам українець
в тисячному поколінні – може навіть десь із-під Вінниці
і увесь його рід – до останнього – у останньому коліні
височіє за ним безмовно – молитовні – уклінні
він сумлінно перераховує всіх живих і мертвих
перераховує всі зради і нібито даремні жертви
а тоді пояснює що жертва не буває даремною
особливо в цій країні яка од віку була буремною
він розказує про козацтво про чиїсь багатство і бідність
про кожну люблячу пару і кожну вдалу вагітність
перераховує всіх поранених страчених і спасенних
тих хто віддавав нажите і тих хто крав казенне
він проходить село за селом місто за містом
від його слів у кімнаті вже мало місця – їм тісно
він розказує про прилаштованих і про геть неприкаяних
про гнаних і гонителів про Авелів і Каїнів
і Господь забуває про все на світі – навіть про каву
і так Йому все незвично! і так Йому все цікаво!
це ж почути одночасно про стільки плачу і сміху!
і нервово гризе від хвилювання на вказівному ніготь
і коли Валерій закінчує повисає тотальна тиша
тільки чутно як в сусідній галактиці шкребеться миша
ніби точить щось смачне і аж попискує від щастя
і провисає над ними небо – як хустка – квітчасте
так минає хвилина за хвилиною година за годиною
і от саме зараз Господа можна сплутати з людиною
– нагадай мені завтра… треба б зайнятись українцями…
і про отого… як його… ну ти розказував… з Вінниці
Сергій Татчин
