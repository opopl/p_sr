%%beginhead 
 
%%file 16_03_2023.fb.eu4business_sme_ukraine.1.mystectvo_tancju_mariupol_vijna
%%parent 16_03_2023
 
%%url https://www.facebook.com/eu4business.sme.ukraine/posts/pfbid08WYywWjkXagVfNJUtvxuDpUsSPkLksyd3VvfBQjcXaWoDu6jw8J45nBZytcSqDGGl
 
%%author_id eu4business_sme_ukraine
%%date 16_03_2023
 
%%tags mariupol,tanec,horeografia,kultura
%%title Історія успіху: Мистецтво танцю під час війни
 
%%endhead 

\subsection{Історія успіху: Мистецтво танцю під час війни}
\label{sec:16_03_2023.fb.eu4business_sme_ukraine.1.mystectvo_tancju_mariupol_vijna}

\Purl{https://www.facebook.com/eu4business.sme.ukraine/posts/pfbid08WYywWjkXagVfNJUtvxuDpUsSPkLksyd3VvfBQjcXaWoDu6jw8J45nBZytcSqDGGl}
\ifcmt
 author_begin
   author_id eu4business_sme_ukraine
 author_end
\fi

\#історія\_успіху: Мистецтво танцю під час війни

У Маріуполі 💃Лілія Полякова мала власну школу класичної та сучасної
хореографії та мріяла відкрити ще спортивну студію. І дівчині це вдалося! Та
вже за півтора місяця після заснування омріяної студії Лілія була змушена
покинути рідне місто і справу свого життя. Все через те, що Росія розпочала
свою жорстоку війну проти України. Новою домівкою для хореографині став Київ,
де згодом Лілія вирішила розпочати свій бізнес з нуля. Суттєвою підтримкою у
відкритті нової школи став мікрогрант від програми \#EU4Business

✨✨✨✨✨ Усе почалося з мрії

Значну частину свого життя Лілія Полякова присвятила танцям. Дівчина багато
разів перемагала у міжнародних конкурсах і прагнула навчати цього мистецтва
інших. Тому Лілія стала дипломованою хореографинею і у 2017 році відкрила
професійну школу класичної та сучасної хореографії Хореографічна школа \#Fouetté
[фуете] для дітей і дорослих в українському місті Маріуполь.

Головною родзинкою Fouetté можливість дітям та мамам одночасно тренуватися в
сусідніх залах. За п’ять років існування школа здобула великих успіхів і
багатьох перемог у танцювальних конкурсах. Учні Лілії ставали призерами
всеукраїнських та міжнародних фестивалів. 

На початку 2022 року в школі навчалися понад 200 учнів і викладали четверо
вчителів. Тоді у січні під назвою \enquote{Fouetté} була ще відкрита спортивна студія
для дітей і дорослих, Тут можна було навчатися інших напрямків, як-от брейкданс
і спортивно-естрадні танці.

\enquote{Відкрити власну студію було мрією, яка здійснилася всього на півтора місяця.
Ще була мрія організувати власний танцювальний фестиваль-конкурс, яку, на жаль,
справдити не встигли}, — згадує Лілія.

💥💥💥💥💥 День, який змінив усе

Рано-вранці 24 лютого 2022 року Маріуполь прокинувся від вибухів одним серед
перших міст України. Всі плани, сподівання та мрії були зруйновані російськими
снарядами.

Школа хореографії \enquote{Fouetté} припинила роботу з початком повномасштабної війни.
Лілія сподівалася, що за кілька днів або тижнів ситуація стабілізується, але
цього так і не сталося. Завдяки тому, що студія була розташована у підвальному
приміщенні, для багатьох вона стала справжнім прихистком. Працівники та учні
школи разом зі своїми родинами тривалий час переховувалися там від обстрілів. 

Сама ж Лілія виїхала з Маріуполя одразу з початком бойових дій, а у квітні
переїхала до Києва після деокупації області.

\enquote{Спочатку було страшно навіть уявити, як починати все заново у великому місті.
Та ще не покидали надії повернутись додому. Це, напевно, був найскладніший
крок. У травні я вирішила, що вже досить сподіватись на краще, і для початку
можна орендувати окремий зал}, — розповідає підприємниця.

👣👣👣👣👣 Відновити бізнес з нуля

За словами Лілії, орендувати повноцінне приміщення і зробити в ньому ремонт
тоді не було фінансової можливості. Адже професійна танцювальна школа
зобов'язувала мати гідний вигляд і відповідний інвентар. Але доля наче сама
підкинула дівчині рішення: Лілія випадково побачила інформацію про мікрогрант
від програми EU4Business на сайті Центру підтримки та розвитку малого та
середнього бізнесу Маріуполя.

Хореографиня уперше брала участь у конкурсі на отримання гранту, але завдяки
своєму досвіду вона чітко розуміла, які кошти та дії потрібні для відкриття
нової студії. Звістка про перемогу приємно здивувала Лілію. Згадує, що була
неймовірно щаслива отримати позитивну відповідь і одразу ж почала пошук
необхідного приміщення.

\enquote{Серед викликів було знайти саме велике приміщення з необхідною кількістю
просторих залів. До того ж, там мало бути укриття, адже зараз це — необхідність
для провадження діяльності}, — розповідає Лілія.

І зробити це вдалося! Завдяки грантовим коштам було орендоване приміщення
загальною площею 141 кв. м, яке повністю відповідає потребам школи. Тепер
Fouetté знову має два окремих зали, щоб одночасно проводити заняття для дітей
та їхніх батьків. Частина мікрогранту також була витрачена на ремонт і
закупівлю необхідного обладнання та інвентарю. Ще завдяки допомозі Лілія
працевлаштувала ВПО з Маріуполя на посаду адміністратора для ліпшої комунікації
та пошуку клієнтів.

\enquote{Мікрогрант допоміг відновити діяльність спортивно-танцювальної студії та
задовольнити потреби людей у тренуваннях для дітей і дорослих. Від початку
повномасштабної війни багато студій були вимушені зачинитись, але попит на
відвідування подібних закладів у Києві є і зростає}, — зазначає підприємниця.

🩰🩰🩰🩰🩰 Нове життя — нові досягнення

Від листопада 2022 року професійна школа хореографії \enquote{Fouetté}
розпочала свою повноцінну діяльність у Києві. 

Зараз у студії викладають двоє тренерів; із напрямів — є класична хореографія,
контемпорарі, модерн, також акробатика і розтяжка. Навчаються вже кілька груп
діточок і дорослих, загалом понад 40 учнів. До речі, троє з них — із Маріуполя.

\enquote{До мене на тренування змогли повернутись діти, які теж були вимушені покинути
Маріуполь. Тепер вони мають можливість відвідувати улюблені заняття у новому
місті. Для дітей зараз вкрай важливо відчувати частинку дому поруч. А фізичний
розвиток завжди має велике значення}, — ділиться Лілія.

Нові клієнти приходять до студії здебільшого завдяки соцмережам або за
рекомендаціями. Зараз головна ціль Лілії — зібрати великий колектив, який
Fouetté мала у Маріуполі. А для цього треба дати новому місту дізнатися про
школу. Також у найближчому майбутньому дівчина планує відновити групи для дітей
з особливими потребами. За словами Лілії, для цих діток фізичний розвиток
просто необхідний, адже він стимулює розумову діяльність. Ще у планах — знову
разом з учнями брати участь у всеукраїнських і міжнародних конкурсах.

Хореографиня сподівається, що згодом вдасться реалізувати ще одну мрію —
організувати власний танцювальний фестиваль. І, звісно ж, відновити діяльність
студії в українському Маріуполі.

«Здорова, сильна, спортивна нація — це дуже круто! Тому ми будемо й надалі
нести в маси користь спорту та танців як для дітей, так і для дорослих», —
говорить Лілія.

ℹ️ℹ️ℹ️ℹ️ℹ️ Довідкова інформація:

Мікрогранти на підтримку українського підприємництва стали доступними в межах
програми міжнародної співпраці EU4Business: конкурентоспроможність та
інтернаціоналізація МСП, яка співфінансується 🇪🇺Європейським Союзом та
🇩🇪Урядом Німеччини і реалізується німецькою федеральною компанією GIZ Ukraine
cпільно з Міністерство цифрової трансформації України, Міністерство економіки
України, українським порталом державних послуг онлайн Дія, українським
національним проєктом із розвитку підприємництва та експорту Дія.Бізнес та
українською державною установою Офіс з розвитку підприємництва та експорту.
Партнером-виконавцем ініціативи мікрогрантів є Міжнародна благодійна
організація Фонд Східна Європа.

\#eu4business, \#MovingForwardTogether, \#sme\_support\_EU\_GIZ,
\#BIOs\_support\_EU\_Germany\_GIZ

Фото: © Лілія Полякова
