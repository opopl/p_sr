% vim: keymap=russian-jcukenwin
%%beginhead 
 
%%file 22_02_2022.fb.goldarb_maksim.1.prezident_minsk
%%parent 22_02_2022
 
%%url https://www.facebook.com/maksim.goldarb/posts/2817374025238449
 
%%author_id goldarb_maksim
%%date 
 
%%tags minsk_dogovor
%%title У президента было два с половиной года для исполнения «Минска»
 
%%endhead 
 
\subsection{У президента было два с половиной года для исполнения «Минска»}
\label{sec:22_02_2022.fb.goldarb_maksim.1.prezident_minsk}
 
\Purl{https://www.facebook.com/maksim.goldarb/posts/2817374025238449}
\ifcmt
 author_begin
   author_id goldarb_maksim
 author_end
\fi

У президента было два с половиной года для исполнения «Минска», возврата
Донбасса, установления мира, наказания кровавых барыг. 

Для этого народ Украины дал ему всё: беспрецедентное доверие, власть, желаемых
депутатов и министров, фору, в конце концов. 

Если ему не нравился «Минск» (о чем он почему-то сообщил нам только сейчас,
назвав его «бездарно составленным»), не вопрос - у него было ДВА года на
выполнение им самим подписанного Парижского коммюнике по внедрению мирной
формулы Штайнмайера. 

Даже в эти несколько последних дней у него была реальная возможность выполнить
всего два действия, которые бы помогли вернуть мир и Донбасс: отменить его
экономическую блокаду и съездить туда, по-настоящему, по-президентски,
по-государственному устанавливая отношения с краем. 

Ни за эти годы, ни за эти дни он не сделал НИЧЕГО: ни для мира, ни для
поверивших ему людей, ни для возврата территорий, ещё более усугубив внутренний
раскол страны и потерю суверенитета. 

К сожалению, время, которое можно и нужно было бы использовать на
восстановление и воссоединение страны, бездарно и безвозвратно упущено. 

Вся абсолютно ответственность за происшедшее - на президенте зеленском и его
окружении.
