% vim: keymap=russian-jcukenwin
%%beginhead 
 
%%file slova.jumor
%%parent slova
 
%%url 
 
%%author_id 
%%date 
 
%%tags 
%%title 
 
%%endhead 
\chapter{Юмор}
\label{sec:slova.jumor}

%%%cit
%%%cit_head
%%%cit_pic
%%%cit_text
\emph{Гумор} як технологія формує легке ставлення до життя. В тому числі до серйозних
суспільних явищ, зокрема до виборів, зміни влади. Якщо всі політики стають
об’єктом сміху, якщо всі вони однакові, себто однаково смішні, то яка різниця,
кого з них обрати? Зрештою, навіщо до цього ставитися відповідально, краще
обирати «по приколу»...  Коли таке ставлення до життя вже виховане телевізійним
гумором, то все решта стає справою маніпулятивно-політичної техніки.  Цікаво,
що самі діячі політичного гумору добре розуміють, що в реальних умовах України
\emph{гумор} – це щось значно більше, ніж просто \emph{гумор}. Це зброя. Алеся Бацман,
головний редактор інтернет-видавництва «Гордон», сказала в ефірі «Україна-24»,
що їй скаржилися деякі провідні актори «Кварталу 95», що після того, як
Зеленський став президентом, він їм суворо заборонив жартувати щодо себе і
своєї політики... Навіщо це робити, якщо \emph{гумор} поза політикою і зовсім не
впливає на політичну ситуацію в країні?  Дуже цікаво висвітлив цю тему сатирик
Дмитро Чекалкін. Пан Чекалкін у минулому дипломат, перший секретар українського
посольства в Тель-Авіві, вихованець Інституту військових перекладачів, знавець
івриту і арабської мови. Він розповів про те, що після того як його сатиричний
проект «Веселі яйця» мав великий успіх під час Помаранчевої революції,
російський політтехнолог Глєб Павловський негайно порушив у Москві питання про
створення для України \emph{гумористичного} проекту з протилежними настановами. І тоді
з’явився «Квартал 95»...  Хоча його натхненники навряд чи очікували того
фантастичного успіху, котрий він здобув у 2019 році...
%%%cit_comment
%%%cit_title
\citTitle{Гумор як політична технологія та зброя}, 
Ігор Лосєв, day.kyiv.ua, 07.12.2021
%%%cit_url
\href{https://day.kyiv.ua/uk/blog/polityka/gumor-yak-politychna-tehnologiya-ta-zbroya}{link}
%%%endcit
