% vim: keymap=russian-jcukenwin
%%beginhead 
 
%%file 28_12_2022.fb.rojz_svitlana.kyiv.1.foto_pidsumky
%%parent 28_12_2022
 
%%url https://www.facebook.com/svetlanaroyz/posts/pfbid02rhtipEsj9sf7cvifAZr6FxM3gcouPz4zUgoqeXgGJvwQioDMC55przrhS3YiVT8ml
 
%%author_id rojz_svitlana.kyiv
%%date 
 
%%tags 
%%title Мої підсумки року в цьому фото
 
%%endhead 
 
\subsection{Мої підсумки року в цьому фото}
\label{sec:28_12_2022.fb.rojz_svitlana.kyiv.1.foto_pidsumky}
 
\Purl{https://www.facebook.com/svetlanaroyz/posts/pfbid02rhtipEsj9sf7cvifAZr6FxM3gcouPz4zUgoqeXgGJvwQioDMC55przrhS3YiVT8ml}
\ifcmt
 author_begin
   author_id rojz_svitlana.kyiv
 author_end
\fi

Друг, який багато років живе в іншій країні, запитав: "як ти? які твої підсумки
року?" Я надіслала йому без слів це фото з галереї телефону. Він відповів "вау!
така фактура, композиція та освітлення. І кіт у тебе милий". А коли я розповіла
про те, що, насправді, на цьому фото, він після паузи сказав, що заплакав. 

Це, насправді, страшне символічне фото, яке я робила минулого місяця під час
обстрілів. Коли зовсім поруч із нами були вибухи, ракети летіли на Київ. Ми з
донькою, сином і кішкою сиділи в підвалі. Кішка, яка перша біжить в підвал при
небезпеці,  була в своєму укритті між своїх "2 стін" поруч із нами. Кішка, яку
ми везли в Івано-Франківськ і назад в Київ, яку оберігаємо, як всіх - своїх.
Яка переживає війну разом з нами. Коли якось наші знайомі з впевненістю сказали
про нас і сім'ю брата: ну ви ж покинули своїх котів, коли їхали до Франківська
на початку війни, куди ж ви з ними? - для мене це було настільки образливо - як
хтось міг подумати, що ми залишимо- зрадимо своїх?!

Поруч із кішкою ліхтарик, який я купувала для доньки. Щоб вона не боялась в
темряві. Він не практичний, але своєю казковістю створює таку атмосферу
затишку. Дивно казати - атмосфера затишку в підвалі під час обстрілів. 

А сам ліхтарик стоїть на домашній тактичній аптечці, яку для нас збирав
чоловік. Щоб, якщо щось впаде - ми мали змогу врятуватись самі і допомогти
іншим.

Це красиве страшне фото. Як і війна велично страшна. Велична - цінностями, які
висвітлює. Колись в дитинстві я казала бабусі про війну: ви жили в героїчні
часи, у вас була можливість проявити свою силу. Дурко. Краще я б її питала,
якою силою їм вдалося боротись, зберегтися  і жити далі.  

Мої підсумки року в цьому фото - вони не про кількість втілених проектів, вони
про те, що тепер я знаю, що означає боятись, але при цьому - берегти, бути
вірною, піклуватись, стати "двома емоційними стінами", для тих, кому потрібно,
бути вдячною тим, хто став такими "стінами" для мене, Вірити. І знати, блін,
про вміст тактичної аптечки. І молитись, щоб це не знадобилось. 

І, так, наша Шира дуже мила. 

Обіймаю, Родино ❤️ як хочу для нас найголовнішого "підсумку" - Перемоги
