% vim: keymap=russian-jcukenwin
%%beginhead 
 
%%file 06_08_2021.fb.kovaljov_andrej.1.detsad_mova
%%parent 06_08_2021
 
%%url https://www.facebook.com/andriy.kovalyov.5/posts/10223819810514872
 
%%author Ковалев, Андрей
%%author_id kovaljov_andrej
%%author_url 
 
%%tags deti,detsad,gorod,kiev,mova,ukraina
%%title Cередовище комунікації між дітьми в дитячому садочку нашого сина
 
%%endhead 
 
\subsection{Cередовище комунікації між дітьми в дитячому садочку нашого сина}
\label{sec:06_08_2021.fb.kovaljov_andrej.1.detsad_mova}
 
\Purl{https://www.facebook.com/andriy.kovalyov.5/posts/10223819810514872}
\ifcmt
 author_begin
   author_id kovaljov_andrej
 author_end
\fi

Хотів би поділитися нашим досвідом, як ми створюємо середовище комунікації між
дітьми в дитячому садочку нашого сина. Сподіваюся, це буде корисним для дітей
та батьків Києва та інших міст України.

Наш з Iryna Briazgun 3,5-річний син Матвій розмовляє українською мовою, для
нього вона – основна. Другою мовою, якою я регулярно з ним говорю, яку він
часто чує в Ютюбі та в дитячому садочку, є англійська. Російську мову Матвій
чує не часто та зараз практично її не розуміє.

\ifcmt
  pic https://scontent-cdt1-1.xx.fbcdn.net/v/t39.30808-6/232021925_10223819720112612_4691012050544245247_n.jpg?_nc_cat=110&ccb=1-4&_nc_sid=730e14&_nc_ohc=Beqtg7Sb-DUAX-A5ru9&_nc_ht=scontent-cdt1-1.xx&oh=84176d8a55dbefad5bc293e625a0325e&oe=61156F53
  width 0.4
\fi

Ми (я в Чернігові, а Іра в Києві) виростали в російськомовному середовищі і
нашою основною мовою спілкування до 2005 року була російська. Це змінилося
після Помаранчевої революції, коли ми поступово перейшли на українську і
вирішили, що для наших дітей основною мовою спілкування буде українська, другою
та міжнародною – англійська.

Серед моїх друзів та знайомих досить багато людей після Помаранчевої, а
особливо після Революції гідності або самі перейшли на українську, або вирішили
виховувати своїх дітей українською. І це – чудово! З 2014 року на вулицях Києва
дійсно все частіше чутно українську, на українську вже не обертаються здивовано
люди в магазинах і офісах. 

Мені здається, що серед більшості киян вже є (ще поки неозвучений) консенсус,
що української має звучати все більше в публічному просторі і нею мають все
більше говорити наші діти. За моїми спостереженнями на дитячих майданчиках з
дітьми до 5 років 15-20\% батьків вже говорить (або намагається говорити)
українською, це ще не 50 на 50, однак і не тотальна меншість, як це було в 90-х
та на початку 2000-х.  

Кожен з нас – киян, що хотіли б щоб їх діти більше говорили українською,
намагається створювати для своїх дітей українськомовне середовище, в якому
дітям було б комфортно соціалізуватися та пізнавати світ. Чи є вже таке
українськомовне середовище в Києві? Думаю, що ні. Динаміка – позитивна, однак
поки що не можна сказати, що українськомовним дітям комфортно перебувати на
вулиці, в гуртках або взаємодіяти з іншими людьми, які в більшості розмовляють
російською.

Минулого року Матвій пішов до дитячого садка. Українська мова є основною мовою
навчання в садочках та школах Києва та інших міст. Однак, починаючи з першого
дня в садочку ми спостерігаємо за встановленням неофіційної соціальної норми:
українська стає мовою навчання та тією мовою, якою звертається вихователь до
дітей, російська – мовою неформального спілкування дітей між собою та коли діти
звертаються до вихователів. З 10 дітей у групі Матвія (3-4 річних), українська
є основною для 2, інші 8 дітей розмовляють вдома російською. Ми поспілкувалися
з батьками інших дітей та зрозуміли, що вони очікують, що дітей навчать
розмовляти українською в садочку. Однак, практика показує, що таке “навчання”
працює неефективно. Ось деякі наші спостереження за Матвієм та дітьми його
групи:

1) діти завчають одно-, двох-слівні відповіді на запитання вихователів (“Яка це
фігура?” – “Трикутник”, “Назвіть цей колір” – “Червоний”), у них не формується
навичики активної комунікації, побудови зв’язних речень українською мовою. 

2) українською діти майже не вигадують нових слів, мемів, якихось смішних
фразочок, не повторюють один за одним звуків, майже не грають в голосові ігри,
хоча гра з голосом та звуками, звукоімітація – дуже природня дитяча поведінка

3) ситуація постійного перебування в звучанні двох мов, коли є мова навчання
(українська) і домашня мова (російська) призводить до того, що як у
українськомовних так і російськомовних дітей мова взаємної комунікації
спрощується до слів, що найбільш схоже звучать обома мовами, обидві мови значно
спрощуються. Це може посилюватись в майбутньому, коли вже дорослі люди говорять
про те, що їм важко знаходити потрібні слова українською або вони соромляться
“некрасивості” своєї мови

4) в старшій групі (5-6 років) багато дітей не готові до школи через брак
навичок висловлювання українською. Директорка однієї приватної київської школи
нещодавно сказала мені, що 60-70\% дітей, що приходять до них в перший клас
досить погано можуть висловити розгорнуту думку у відповідь на питання
українською.

Враховуючи ці спостереження ми разом з вихователями садочка запропонували іншим
батькам впровадити щотижневий “день української мови” (для нас це - четвер). 

\begin{itemize}
\item - вихователі попросили батьків протягом цього дня з ранку до вечора
				спілкуватися з дітьми українською, підтримувати намагання дітей
				говорити українською в домашньому комфортному середовищі, щоб вона
				також ставала мовою їх повсякденного життя

\item - мультфільми, музика, книжки в цей день також українською

\item - в цей день вихователі ненав’язливо виправляють дітей, коли вони
				говорять щось російською (“А ти знаєш як це буде українською?”, “А
				можеш повторити те, що ти сказав українською?”, “Давайте згадаємо
				українські слова на літеру “А”” і т.д.). 
\end{itemize}

Всі батьки сприйняли цю ініціативу позитивно або нейтрально, негативу не було.

Що, можливо, підтверджує мою тезу про сформований неофіційний консенсус у
Києві.

Зараз йде вже другий місяць цієї ініціативи і ми та вихователі бачимо перші позитивні результати:

\begin{itemize}
\item 1) майже всі батьки намагаються в цей день з дітьми практикувати українську. Це покращує мову як дітей так і батьків 🙂
\item 2) деякі діти почали самі нагадувати батькам, що в цей день треба говорити українською, тобто діти починають самі шукати мовної практики 🙂
\item 3) дітям стало простіше говорити між собою українською. Якщо раніше Матвій
часто чув навколо себе російську, то зараз починає поступово формуватися
українськомовне середовище і це нас дуже тішить 🙂

\item 4) деякі батьки після запуску днів української мови сказали, що давно думали
про те, щоб більше з дітьми говорити українською, але якось не наважувались, а
прохання вихователів їх надихнуло
\end{itemize}

Звісно, такий “день” класно було б мати кожного дня, але треба рухатись
поступово 🙂 Думаю, надалі ми запропонуємо збільшувати кількість таких днів на
тижні. 

Мені здається, ця ініціатива буде корисною майже всім батькам та дітям Києва та
інших українських міст. Якщо ви б хотіли запустити подібну ініціативу в своєму
садочку або школі ми будемо раді поділитися нашим досвідом з батьками,
вихователями та вчителями. Напишіть про це в коментарях або мені в
повідомлення.

Також ми спільно з друзями плануємо невдовзі запустити розмовні клуби
української для батьків та дітей, на яких у дружній атмосфері ви зможете
попрактикувати українську не соромлячись помилок або “сказати щось не так”.
Якщо вам цікаво, напишіть також про це і ми надішлемо скоро анонс.

Авторка картинки та крутої серії про кота Інжира - Олена Павлова (с).
