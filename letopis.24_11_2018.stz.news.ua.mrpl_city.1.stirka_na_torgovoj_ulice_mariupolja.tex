% vim: keymap=russian-jcukenwin
%%beginhead 
 
%%file 24_11_2018.stz.news.ua.mrpl_city.1.stirka_na_torgovoj_ulice_mariupolja
%%parent 24_11_2018
 
%%url https://mrpl.city/blogs/view/stirka-na-torgovoj-ulitse-mariupolya
 
%%author_id burov_sergij.mariupol,news.ua.mrpl_city
%%date 
 
%%tags 
%%title Стирка на Торговой улице Мариуполя
 
%%endhead 
 
\subsection{Стирка на Торговой улице Мариуполя}
\label{sec:24_11_2018.stz.news.ua.mrpl_city.1.stirka_na_torgovoj_ulice_mariupolja}
 
\Purl{https://mrpl.city/blogs/view/stirka-na-torgovoj-ulitse-mariupolya}
\ifcmt
 author_begin
   author_id burov_sergij.mariupol,news.ua.mrpl_city
 author_end
\fi

Как стирали белье на Торговой улице, да и во всем в Мариуполе, когда не было
стиральных машин и стиральных порошков? Да руками. Были, конечно,
приспособления: корыта из оцинкованного железа, выварки, доски с рифленой
поверхностью, а кое у кого доставшееся от старого дореволюционного мира лохани,
сработанные из дубовых клепок, скрепленных двумя железными обручами. Основными
моющими средствами были хозяйственное мыло, щелок, а в годы лихолетья – зола от
будылья подсолнухов. Существовало еще одно экзотическое стиральное средство –
жавелевая вода. Но это уж из очень давних времен.

\textbf{Читайте также:} 

\href{https://mrpl.city/news/view/v-ukraine-chtyat-pamyat-zhertv-golodomora-video}{%
В Украине чтят память жертв Голодомора, Олена Онєгіна, mrpl.city, 24.11.2018}

 Большая стирка в семье была событием экстраординарным. С утренней зарей в
выварке грели воду, а потом целый день женские руки в мыльной воде терли о
стиральную доску до изнеможения простыни, наволочки, пододеяльники, полотенца и
прочее белье. Затем полоскание, вываривание в мыльном или щелочном растворе,
снова полоскание, подсинивание и накрахмаливание и, наконец, сушка. Все эти
операции, кроме последней, требовали воду, много воды. Правда, в Мариуполе уже
в довоенные годы и еще раньше во дворах, а у некоторых и в домах, были
водопроводные краны. Но немцы, покидая город в сентябре сорок третьего года,
взорвали электростанцию на Мало-Фонтанной улице, вывели из строя насосы,
которые обеспечивали город живительной влагой.

\ii{24_11_2018.stz.news.ua.mrpl_city.1.stirka_na_torgovoj_ulice_mariupolja.pic.1}

Воду источали карстовые трещины известняков-ракушечников на склонах улицы. Этот
родник со времен заселения города называли Большим фонтаном. Именно им
воспользовался гражданский инженер и городской архитектор Мариуполя Виктор
Александрович Нильсен, чтобы спроектировать и построить водопровод. Если уж об
этом сооружении здесь сказано, назовем дату пуска его в строй – 25 марта 1911
года.

После того как из города выбили оккупантов, из всего водопровода действующим
остался только каптаж - цилиндрический кирпичный резервуар для вывода на
поверхность подземных вод. Но поскольку воду из каптажа нечем было откачивать,
она, перетекая через край, изливалась в довольно глубокую канаву, образуя за
кирпичной стеной небольшое озерцо. Вода в нем даже в теплую погоду была
ледяной, ведь она была проточной. Конечным пунктом ее движения был Кальмиус.
Вот в этом озерце женщины полоскали белье. Хорошо намочив его, укладывали на
камни и нещадно колотили вальками. Что такое валёк? Википедия дает следующее
определение этому нехитрому устройству - \emph{деревянная прямоугольная пластина для
выколачивания белья во время стирки, слегка изогнутая, с рукояткой.}

\textbf{Читайте также:} 

\href{https://mrpl.city/news/view/stoletnyuyu-istoriyu-vodonapornoj-bashni-v-mariupole-pomestili-v-korotkij-rolik-video}{%
Столетнюю историю водонапорной башни в Мариуполе поместили в короткий ролик, Ярослав Герасименко, mrpl.city, 21.03.2018}

С озерцом связано еще одно воспоминание. В нем водились небольшие юркие рыбки,
снующие в толще воды. Они были небольшими, длиной пять-шесть сантиметров.
Ребятне иногда удавалось выловить самодельными сачками несколько рыбешек. Улов
помещали в пол-литровую или литровую стеклянную банку, наполненную водой, и со
всеми предосторожностями несли домой. Но вот незадача – рыбки почему-то недолго
жили в неволе. Много позже довелось узнать, что их название – гамбузия, что их
предки – обитатели Южной Америки, а завезены были в Европу для борьбы с
малярией - бичом тех лет. Дело в том, что личинки малярийных комаров,
разносчиков болезни - любимое лакомство гамбузий. Снижая численность личинок,
маленькая представительница ихтиофауны уменьшала число заболевших малярией...

Были женщины, которые занимались стиркой профессионально, – прачки. В
благополучные периоды в истории они стирали для людей состоятельных. В годы
лихолетья, когда успешно было покончено с эксплуататорским классом, ремесло их
не приносило никакого дохода, и они перебивались как могли. Правда, когда
наступила НЭП, нашлась и прачкам работа. Но у ленинской Новой экономической
политики век оказался недолгим, нэпманы разорились, а некоторых из них
отправили в места отдаленные, и снова прачки оказались не у дел. Потом война,
послевоенная разруха, сами понимаете, некому было заказывать стирку...

\textbf{Читайте также:} 

\href{https://mrpl.city/news/view/dlya-uluchsheniya-kachestva-vody-v-mariupole-zadejstvovan-sputnik-i-minskie-peregovory}{%
Для улучшения качества воды в Мариуполе задействован спутник и минские переговоры, mrpl.city, 25.10.2018}

В соседнем дворе, в небольшой квартире, устроенной в дореволюционном
магазинчике, жила Дарья Игнатьевна. Сухонькая подвижная женщина в годах с живым
взглядом василькового цвета глаз и руками, изъеденными щелоком. Стирка была ее
профессий еще с тех пор, когда отец привез ее из близлежащего села в услужение
к мариупольской барыне. Эта особа, имя которой давно забыто, не только научила
Дашу правильно стирать, но и со временем отдала замуж за дальнего родственника
Иону. Он был городовым, - для молодежи сообщим, что это – царского времени
полицейский. Несмотря на то что брак был как бы по расчету, он оказался
счастливым. Молодая жена подарила мужу сына-первенца, а затем, с разницей по
два года, еще двух мальчишек. Теперь Дарья занималась только своим домашним
хозяйством и детьми.

Но размеренная жизнь была низвергнута в пропасть, когда главу семейства
мобилизовали на германскую войну – так называли в народе Пер\hyp{}вую мировую. В мае
1915 года австрийская шрапнель сразила Иону наповал. На скудное пособие,
положенное от царя, вдове-солдатке с тремя малолетними детьми прожить было
трудно. И Дарья стала обстирывать чужих людей, брала стирку из гостиницы
\enquote{Россия}, отстоящей недалеко от ее дома. Состоятельные заказчицы ценили Дарью
за то, что она могла не только хорошо выстирать блузки и платья из тончайшего
батиста или шелка, но и тщательнейшим образом разгладить без единой морщинки
многочисленные рюшки, плиссировки, воланы, оборки, жабо, кружевные отделки,
прошвы и другие украшения, свойственные одежде модниц начала ХХ века. А ведь у
прачки с Торговой улицы не было электрического утюга с устройством, с помощью
которого можно за секунду можно установить температуру нагрева для глажения то
шерстяной ткани, то крепдешина, то заполонившей весь мир синтетики.
Инструментом Дарьи Игнатьевны были тяжеленные утюги, греющиеся денно и нощно на
плите, зимой в доме, а летом на печурке, сложенной во дворе. Она обучила своему
мастерству и невестку – жену старшего сына, громкоголосую полноватую кубанскую
казачку...

\textbf{Читайте также:} \href{https://archive.org/details/08_09_2018.sergij_burov.mrpl_city.k_240_letiu_mariupolja_perekrestok_na_torgovoj}{К 240-летию Мариуполя: перекресток на Торговой, Сергей Буров, mrpl.city, 08.09.2018}

В освобожденном от оккупантов Мариуполе еще не рассеялся дым пожарищ, а на
заводах уже кипела работа по их восстановлению. Значительная часть городских
строений была сожжена. На Торговой улице гитлеровские поджигатели успели
предать огню только двухэтажные дома. Но не все. Один из них во дворе городской
бани уцелел. На его втором этаже азовстальцы устроили учреждение с вывеской –
\enquote{Дом приезжих}. Там ночевали люди, командированные по разным неотложным делам
на восстанавливаемую Южную Магнитку. Вот в \enquote{Доме приезжих} и нашли себе работу
Дарья Игнатьевна и ее невестка. Плоды их трудов были налицо. Через весь двор
были протянуты веревки, а на них парусами развевались снежной альпийской
белизной простыни, пододеяльники, наволочки. Правда, их безупречную чистоту
слегка портили казенные штампы с надписью: \enquote{Азовсталь. ДП}...

В середине пятидесятых годов прошлого уже ХХ века на Торговой улице в доме № 39
с дверью на срезанном углу, обращенной как раз на перекресток улиц Торговой и
Фонтанной, был устроен магазин электротоваров. Там на видном месте стояли
припорошенные пылью две или три стиральных машины \enquote{Рига-55}, произведение
Рижского электромеханического завода (РЭЗ). На них долгое время не находилось
покупателей. Бытовало мнение, что эти устройства при стирке нещадно рвут вещи.
В разряд неходового товара машины попали еще и потому, что стоили они дорого –
850 рублей за штуку. Сумма по тем временам для многих была фантастической.
Летом 1956 года папа как-то зашел в магазин - то ли за лампочкой, то ли за
пробками-предохранителями. В поле его зрения попала стиральная машина. В тот
исторический момент было принято решение – нужно ее купить. Назанимали у
родственников необходимые деньги. И вот уже бак, верх которого – из нержавейки,
а низ – из листовой стали, покрытой эмалью цвета топленого молока, стоит во
дворе. Все его обитатели собрались у нашей обновы. В форточку продели шнур от
машины, присоединили к розетке. На летней печке нагрели в выварке воду, слили
ее в машину. Для испытания взяли халат дяди Кости, сшитый из баракана,
потерявшего свой первородный цвет. В этот предмет одежды дядя облачался, когда
надо было делать самую грязную домашнюю работу, например, трусить сажу из
грубки, ремонтировать водопровод или еще что-нибудь такое. Халат опустили в
горячую мыльную воду. Папа включил машину. Когда она отработала положенное
время, зацепив деревянными щипцами, вытащили постиранную вещь. И тогда все
увидели ее истинный цвет – оранжевый с коричневыми закорючками. И, главное,
халат остался неповрежденным. \enquote{Рига-55} была верным помощником в семье на
протяжение тридцати лет, а затем подарена дальним родственникам. Родственники
уехали из города, и дальнейшая судьба этой стиральной машины неизвестна.

\vspace{0.5cm}
\begin{minipage}{0.9\textwidth}
\textbf{Читайте также:} 

\href{https://mrpl.city/news/view/mariupol-podaril-pereselentsam-novuyu-stiralnuyu-mashinu}{%
В Мариуполе переселенцам подарили новую стиральную машину, mrpl.city, 17.03.2017}
\end{minipage}
\vspace{0.5cm}

Эра стиральных машин с активатором продолжалась очень долго. Лишь в 1981 году
Кировский завод \enquote{Электробытприбор} произвел первую партию из 100 машин модели
\enquote{Вятка-автомат-12}. Постепенно предприятие нарастило их выпуск, но из-за
высокой цены (495 рублей, правда, позже ее снизили до 400 рублей) в застойное
время ее покупали только очень, очень высокооплачиваемые люди. Сейчас во многих
мариупольских домах успешно пользуются автоматическими стиральными машинами
разных фирм и конструкций, магазины наперебой зазывают их потенциальных
покупателей. Но это – сегодняшний день, а это повествование посвящено прошлому.
На этом и закончим.
