% vim: keymap=russian-jcukenwin
%%beginhead 
 
%%file 08_09_2021.fb.bryhar_sergej.1.mova_jazyk_surzhik
%%parent 08_09_2021
 
%%url https://www.facebook.com/serhiibryhar/posts/1782593251940619
 
%%author_id bryhar_sergej
%%date 
 
%%tags jazyk,mova,obschestvo,surzhik,ukraina
%%title Ні! Будь-яка українська краща, ніж будь-яка російська!
 
%%endhead 
 
\subsection{Ні! Будь-яка українська краща, ніж будь-яка російська!}
\label{sec:08_09_2021.fb.bryhar_sergej.1.mova_jazyk_surzhik}
 
\Purl{https://www.facebook.com/serhiibryhar/posts/1782593251940619}
\ifcmt
 author_begin
   author_id bryhar_sergej
 author_end
\fi

"Дама" старшого віку питає мене: "а ви разговаріваєтє на "чістом" укрАінском,
ілі на суржикє?"

"Ну все, - думаю, - почалося: зберися, і - в бій".

Довелося вчини по-одеськи - відповісти питанням на питання: "а ви розмовляєте
літературною, довершеною російською, чи діалектом, який увібрав у себе впливи
усіх мов, що побутували на півдні України?"

Наче ж просте питання, однак відповіді я так і не отримую.

Вирішую добивати:

- Ваше мовлення схоже на мову Достаєвського, чи принаймні якогось Пєлєвіна?

Ігнор... Натомість мені починають розповідати ось таке: "ти понімаєш, мнє очєнь
даже нравітся укрАінскій, но протівно, когда слишу вот етот сєльскій суржик..."

Доводиться парирувати: "цікаво вже те, що після століть московської окупації
українська мова взагалі збереглася, і більше того, досі є однією з
найпоширеніших мов Європи - думаю, скільки б Московія не старалася, вбити її
неможливо..."

От коли говориш приблизно так - спокійно, без агресії, але впевнено, - усі ці
"поборники "чистої" української", як правило, замовкають. Іноді бісяться. Але,
зрештою, все одно припиняють своє "філософствування".

Можна, звісно, і просто послати, але це опція - на крайній випадок. Як на мене,
злити опонента, повільно його "розігрівати", значно цікавіше і приємніше. Вони
мають розуміти, що ми не слабкі!...

Коли мій малий повертається з прогулянки, і в його лексиконі з'являються різні
там "папа", "самольотік", "окошко" (не скажу, що так буває часто, але
трапляється), я одразу ж намагаюся це виправити. Миритися з тим усім, "бо це ж
Одеса", не в моїх правилах! 

Я хочу, щоб наш рівень мовлення був достатньо високим. Але це моя справа. 

Взагалі ж, коли хтось починає розповідати про "протівний сєльскій суржик", я
розумію, що переді мною московська зараза. Принцип тут простий: "ви говорітє на
плохом укрАінском, потому давайте лучше - на русском" (поганою російською? -
"нє важно, главноє - на русском")

Ні! Будь-яка українська краща, ніж будь-яка російська!

\ii{08_09_2021.fb.bryhar_sergej.1.mova_jazyk_surzhik.cmt}
