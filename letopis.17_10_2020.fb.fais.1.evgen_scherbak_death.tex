% vim: keymap=russian-jcukenwin
%%beginhead 
 
%%file 17_10_2020.fb.fais.1.evgen_scherbak_death
%%parent 17_10_2020
%%url https://www.facebook.com/MCFUkraine/posts/2432141126911199
%%tags fais,death,evgen scherbak
 
%%endhead 

\subsection{Від нас пішов Євген Вікторович Щербак...}
\label{sec:17_10_2020.fb.fais.1.evgen_scherbak_death}

\url{https://www.facebook.com/MCFUkraine/posts/2432141126911199}

16 жовтня 2020 року на 83 році життя померла талановита людина - Євген
Вікторович Щербак, організатор розвитку альпінізму та скелелазіння на
Криворіжжі, кандидат у майстри спорту з альпінізму та скелелазіння. Свій шлях
спортсмена-альпініста пройшов в складі збірної команди Дніпропетровської
облради «Авангард». У складі команди він брав участь в експедиціях на Памірі й
Тянь-Шані, підкорив пік Ісмоіла Сомоні (Комунізму), пік Леніна, пік
Корженевській, пік Хан-Тенгрі. За його плечима перемоги у Чемпіонатах України з
альпінізму та у ветеранських Чемпіонатах України зі скелелазіння. 

За свою тренерську кар'єру виховав 15 кандидатів в майстри спорту і три майстри
спорту.  Зробив багато для популяризації скелелазіння як активного відпочинку і
ведення здорового способу життя. За ці роки сформувалась стабільна і сильна
секція скелелазіння під керівництвом Євгена Вікторовича, яка стала місцем
проведення міських, обласних, всеукраїнських змагань.

У 2018 році вийшла збірка віршів Євгена Щербака за назвою «Лирические
маршруты». В одному з останніх своїх віршів він написав такі рядки:

Душа стремиться выше к звёздам,
А тело в землю норовит
И, как над нами не хлопочут
Мы отвергаем их усилия спасти.

У наших серцях, пам'яті світлий образ Євгена Вікторовича залишиться назавжди.
Федерація альпінізму і скелелазіння України висловлює щирі співчуття рідним,
близьким, друзям Євгена Вікторовича. Вічна йому пам'ять.
