% vim: keymap=russian-jcukenwin
%%beginhead 
 
%%file slova.taksi
%%parent slova
 
%%url 
 
%%author_id 
%%date 
 
%%tags 
%%title 
 
%%endhead 

\chapter{Такси}

%%%cit
%%%cit_head
%%%cit_pic
%%%cit_text
Сьогодні я їхав з \emph{таксистом}, який авторитетно заявив: «Скоро буде
переворот. До влади прийдуть військові і наведуть порядок!». На моє запитання,
а хто ж очолить хунту, він відповів обтічно: «Побачимо…». Дату і час перевороту
повідомляти відмовився, хоча відчувалося: знає.  Як сказав би Тарас Григорович,
«Коли ми діждемося Піночета з новим і праведним декретом? А діждемось-таки
колись!»
%%%cit_comment
%%%cit_title
\citTitle{Даже таксисты живут ожиданиями государственного переворота}, 
Константин Бондаренко, strana.news, 24.10.2021
%%%endcit

%%%cit
%%%cit_head
%%%cit_pic
%%%cit_text
\emph{Такси} вертикально поднялось в воздух. Гаал смотрел в изогнутое
прозрачное окно, наслаждаясь ощущением полета в машине и инстинктивно
уцепившись за сидение водителя. Люди внизу стали походить на муравьев в
растревоженном муравейнике.  Затем они сделались невидимыми.  Впереди
показалась стена. Она начиналась высоко в воздухе, а конца ее не было видно. В
стене зияли широкие дыры туннелей. \emph{Такси} подлетело к одной из этих дыр и
нырнуло внутрь. На секунду Гаал удивился, как это водитель узнал, что нужно
именно сюда. Темноту вокруг рассекали лишь одинокие сигнальные огни.
\emph{Таксист} резко снизил скорость, и Гаал подался вперед, стараясь
восстановить равновесие. Затем машина вынырнула из туннеля и затормозила на
одном из уровней.  — Отель «Люксор», — сказал водитель.  Он помог Гаалу
выгрузить багаж, с деловым видом получил десять кредиток на чай, взял
очередного пассажира и тут же поднялся в воздух.  Трантор... в начале 13-го
тысячелетия его возможности достигли больших высот. Являясь центром имперского
правления в течение тысяч столетий, будучи расположенным в центральных районах
Галактики среди наиболее плотно населенных и индустриально развитых миров
системы, он не мог не стать самым значительным и богатым государством, которое
когда-либо видела человеческая раса
%%%cit_comment
%%%cit_title
\citTitle{Основание}, Айзек Азимов
%%%endcit
