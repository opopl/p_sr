% vim: keymap=russian-jcukenwin
%%beginhead 
 
%%file 28_05_2021.fb.arestovich_alexei.1.myshlenie_bulychev_kir
%%parent 28_05_2021
 
%%url https://www.facebook.com/alexey.arestovich/posts/4369260933137953
 
%%author Арестович, Алексей
%%author_id arestovich_alexei
%%author_url 
 
%%tags 
%%title Уровень, на котором зарождается человеческая мысль, как правило, не осознается человеком
 
%%endhead 
 
\subsection{Уровень, на котором зарождается человеческая мысль, как правило, не осознается человеком}
\label{sec:28_05_2021.fb.arestovich_alexei.1.myshlenie_bulychev_kir}
\Purl{https://www.facebook.com/alexey.arestovich/posts/4369260933137953}
\ifcmt
 author_begin
   author_id arestovich_alexei
 author_end
\fi

\begin{itemize}
\item - Уровень, на котором зарождается человеческая мысль, как правило, не
осознается человеком.  Поэтому, проблема маршрутизации мысли, поднятия ее из
скрытых чувственных структур в глубине подсознания, наверх, к оформлению в виде
структурированных и понятных себе и другим слов, образов и символов никогда не
будет банальной.

Со времён изобретения левой политической идеологии, основными идеями которой
является общественный прогресс, освобождение от традиции и справедливое
распределение коллективный Запад, мечется между левыми и правыми, одинаково
отягощённый (и внутренне разрушенный) нигилизмом.

Ноосферный проект, старательно подозреваемый своими оппонентами в левом
происхождении, трудами своих отцов-основателей Тейяра де Шардена и И.Ефремова,
на самом деле, избавлен от родового проклятия левизны:

\item - попытки подменить реальные эволюционные структуры в человеке и человечестве
рациональными, но сформулированными, без учёта биологии живого, схемами
общественного переустройства.
\end{itemize}

\ifcmt
  pic https://scontent-frt3-2.xx.fbcdn.net/v/t1.6435-9/191507750_4369256543138392_5714725392168550602_n.jpg?_nc_cat=103&ccb=1-3&_nc_sid=8bfeb9&_nc_ohc=wQ9A5C_KrVkAX-X-K4j&_nc_ht=scontent-frt3-2.xx&oh=ff0e81c53642bb312c6c2381d1da382b&oe=60D97650
\fi

Шарден, будучи учёным-антропологом с мировым именем, четко выводит логику
сверх-разума из неизбежной логики развития живого.

Ефремов, будучи палеонтологом с мировым именем, четко ставит зависимость не
только общественного прогресса, но и самого вида homo, от нашей способности
сделать эволюцию живого управляемой нами производительной силой общества. 

Кир Булычев, автор сценария «Через тернии у звёздам», даром что выдающийся
историк-востоковед, в фильме, который мы будем очень тщательно разбирать завтра
с разных точек зрения (психологической, символической, общественной,
художественной) порождает практически идеальный сценарий репрезентации проблемы
«скрытое чувственно-биологическое - рационально-идеальное», выписывая и судьбу
главных героев и соотвествующий контекст, и изящный символизм проблемы со
свойственным ему художественным обаянием.

Приглашаю прогуляться этой дорогой вместе.

Среди тем обсуждения, тема направления, путей и проблем преобразования Человека
и человечества, и характера препятствий этому, займёт непоследнее местно. 

Причём - в проекции на проблемы развития индивидуальной и коллективной психики,
путях развития Разума. 

Приходите. Завтра, 17.00-21.00, онлайн:

\url{https://youtu.be/Chz9FZ3rCJU}

Записаться:

\begin{itemize}
\item рус: \url{https://go.apeiron.school/cherez-ternii-ru-2905RF}
\item укр: \url{https://go.apeiron.school/cherez-ternii-ua-2905RF}
\end{itemize}
