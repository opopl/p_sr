% vim: keymap=russian-jcukenwin
%%beginhead 
 
%%file 11_10_2021.fb.golovko_mihail.1.svobodovec_gordost
%%parent 11_10_2021
 
%%url https://www.facebook.com/mihaylo.holovko/posts/4217948208303786
 
%%author_id golovko_mihail
%%date 
 
%%tags gordost',patriotizm,politika,ukraina,vo_svoboda
%%title Пишаюся тим, що я — свободівець!
 
%%endhead 
 
\subsection{Пишаюся тим, що я — свободівець!}
\label{sec:11_10_2021.fb.golovko_mihail.1.svobodovec_gordost}
 
\Purl{https://www.facebook.com/mihaylo.holovko/posts/4217948208303786}
\ifcmt
 author_begin
   author_id golovko_mihail
 author_end
\fi

Пишаюся тим, що я — свободівець! Для мене за честь — бути частиною великої
команди однодумців, великої родини, яка будує Україну, наповнену українським
змістом! 30 років — це значний відрізок, не лише для життя людини, а й для
організації. Кожен із нас, свободівців, може пригадати безліч історій,
пов'язаних із тим, скільки ми разом, пліч-о-пліч, пройшли: горнило боротьби з
агресором, відстоювання цінностей України, захист територіальної цілісності
нашої держави, революції, війна...

\ifcmt
  tab_begin cols=3

     pic https://scontent-frx5-2.xx.fbcdn.net/v/t39.30808-6/245014831_4217906658307941_2259873251595272971_n.jpg?_nc_cat=109&ccb=1-5&_nc_sid=8bfeb9&_nc_ohc=edBBXEwng7cAX-ltusP&_nc_ht=scontent-frx5-2.xx&oh=431ff0d8d44585c44f57f3180045af7f&oe=61927936

     pic https://scontent-frt3-2.xx.fbcdn.net/v/t39.30808-6/245185487_4217906738307933_3784126642646029394_n.jpg?_nc_cat=101&ccb=1-5&_nc_sid=8bfeb9&_nc_ohc=TEM54Mf_7qIAX8zDjGU&_nc_ht=scontent-frt3-2.xx&oh=9c35490a29d1c0769b3d559128b34621&oe=6191289B

		 pic https://scontent-frx5-1.xx.fbcdn.net/v/t39.30808-6/244776583_4217906851641255_3821923462264347585_n.jpg?_nc_cat=105&ccb=1-5&_nc_sid=8bfeb9&_nc_ohc=b_ctS7uk7ZoAX-SOnGB&_nc_ht=scontent-frx5-1.xx&oh=b890568e5baa6b921ba3c4ba8ad778a6&oe=6191E87B

  tab_end
\fi

У кожного зі свободівців своя історія. Я прийшов в організацію ще будучи
студентом. Коли повернувся із заробітків, зрозумів, що необхідно змінювати
країну й прагнув потрапити до лав націоналістичної організації, яка має в собі
ідею нації та прагне наповнити Україну українським змістом. З-поміж існуючи
обрав «Соціал-національну партію України». У 2002-му познайомився з Віталієм
Хоркавим і Олегом Сиротюком. У 2003-му до Тернополя завітав Олег Тягнибок, у
2004-му СНПУ змінила назву на ВО «Свобода» й він очолив її. 

Тернопільщина завжди відзначалася своїм патріотизмом. У 2000-их ми почали
активно працювати й розвивати місцевий осередок націоналістичної організації.
Ми мали неймовірну жаху в серці та віру в ту справу, якою займаємося. У 2006-му
вперше отримали значну підтримку на виборах від жителів Тернополя. Вдалося
сформувати фракцію у міській раді, я мав за честь тоді бути депутатом. 

\ifcmt
  tab_begin cols=3

     pic https://scontent-frt3-1.xx.fbcdn.net/v/t39.30808-6/245211391_4217906978307909_7802621980529803467_n.jpg?_nc_cat=104&ccb=1-5&_nc_sid=8bfeb9&_nc_ohc=uZofICobHG8AX-92aN1&_nc_ht=scontent-frt3-1.xx&oh=8d0ddbc3c7be06c09edff544a534683c&oe=6192D1D7

     pic https://scontent-frx5-1.xx.fbcdn.net/v/t39.30808-6/245089278_4217907118307895_5581417629310557556_n.jpg?_nc_cat=100&ccb=1-5&_nc_sid=8bfeb9&_nc_ohc=KzEEDpPnOEgAX8-kb_8&_nc_ht=scontent-frx5-1.xx&oh=d04fe50585074bb0606a5d1448851092&oe=6191CAC5

		 pic https://scontent-frx5-2.xx.fbcdn.net/v/t39.30808-6/244703381_4217907228307884_3333518887667403751_n.jpg?_nc_cat=109&ccb=1-5&_nc_sid=8bfeb9&_nc_ohc=RcgNNfpZ8VkAX8YSPeA&tn=lCYVFeHcTIAFcAzi&_nc_ht=scontent-frx5-2.xx&oh=abaa285226516959463d65aa93bc36da&oe=6193033C

  tab_end
\fi

Ми знайомилися з тернополянами та жителями сіл області. Інколи доводилося в
30-градусний мороз роздавати газети, чи організовувати акції, протести, але це
було не складно, адже таким чином гартувався наш дух. 

В 2009-му році настав переломний момент. Із гаслом: «Перемоги розпочнуться з
Тернопілля!» свободівці перемогли на виборах до Тернопільської обласної ради та
вперше отримали націоналіста на посаді голови Тернопільської обласної ради —
Олексія Кайду. Про нас заговорили не лише на Тернопільщині, а й по всій
Україні. Продовженням історії успіху став 2010-ий рік, коли ми виграли вибори
до місцевих рад та отримали першого голову-націоналіста міста обласного
значення — Сергія Надала, який у минулому році втретє переміг на виборах,
заручившись підтримкою рекордної кількості жителів. Це говорить про те, що ми
системно працюємо. Правильно працюємо. Що ми є тією силою, яка виконує свої
обіцянки і є правдивою організацією. 

Пригадую дуже визначний для нас, українців, 2012-ий рік. Вперше націоналісти
отримали фракцію у Верховній Раді України. «Свобода» змінювала країну. Ми вели
за собою людей. Ми були стрижнем Революції гідності. Ми змінювали країну. Ці
зміни закладалися в ідеї партії, яку в 1991-му році створили наші батьки —
засновники: Олег Тягнибок, Анатолій Вітів, Андрій Міщенко. Першу програму
написали на основі книги Ярослава Стецька, нашого земляка, який проголошував
відновлення Української держави. До слова, Тернопільська обласна рада
проголосила 2022-ий рік Роком Ярослава Стецька, ми заснували найвищу нагороду
імені Ярослава Стецька. Й, це також певного роду наша ідеологічна перемога.
Мені надзвичайно приємно вручити почесну відзнаку Тернопільської обласної ради
«За заслуги перед Тернопільщиною» імені Ярослава Стецька Олегу Тягнибоку. 

Те, що в 2014-му нам не дали змоги потрапити до Верховної Ради України, для
більшості партій означало б кінцем. Але, не для націоналістів. Не для нас. Ми
не одноразовий проект, більшість з яких уже українці й не згадають, як
називалися, а — потужна сила, наповнена ідеєю побудови процвітаючої України. І,
переконаний, нам це під силу. Адже, хто як не ми, — націоналісти в змозі
побудувати державу, за яку віддавали життя наші предки, за яку ми боролися на
Революції, за яку гинули наші побратими на війні... Ми, націоналісти, — це та
сила, яка веде за собою націю. Ми творимо історію та будуємо Україну. Чи не усі
постулати, закладені в нашій програмі, сьогодні реалізовані. Це — заборона
комуністичної ідеології, відстоювання української мови, визнання Української
Повстанської Армії, наших Героїв, і Покрови Пресвятої Богородиці державним
святом. Це — наші перемоги, здобутки системної роботи. 

Попереду — нові виклики та нові завдання. Нові загрози. Ми бачимо, як
піднімається питання наступу на нашу землю. Ми бачимо, як намагаються знищити
цінності українського народу. Українську традиційну сім'ю. Українську церкву. У
нас — багато роботи. Ніхто, крім нас не зможе вести за собою людей та зберегти
Україну для наших дітей. Ніхто, крім нас, не в змозі йти проти течії, піднімати
найгостріші питання, адже всі слухають політтехнологів, а ми змінюємо націю!
