% vim: keymap=russian-jcukenwin
%%beginhead 
 
%%file 30_12_2021.fb.fb_group.story_kiev_ua.1.rizdvo_derevo.pic.1.cmt
%%parent 30_12_2021.fb.fb_group.story_kiev_ua.1.rizdvo_derevo
 
%%url 
 
%%author_id 
%%date 
 
%%tags 
%%title 
 
%%endhead 

\iusr{Maksym Oleynikov}

1901 рік. Праворуч від міської Думи – будинок Дворянських зборів, де до
революції ставили різдвяну ялинку і проходили святкові бали.

\iusr{Maksym Oleynikov}

Початок 1950-х рр. Будинок колишніх Дворянських зборів на тодішній площі
Калініна, в якому містився тоді Будинок вчителя. Будівля знесена у 1975р. для
будівництва Будинку профспілок.

\ifcmt
  ig https://scontent-frx5-1.xx.fbcdn.net/v/t39.30808-6/271110527_979112082678494_2273719112613879823_n.jpg?_nc_cat=105&ccb=1-5&_nc_sid=dbeb18&_nc_ohc=bnLsHFT8YZYAX-NjI7r&_nc_ht=scontent-frx5-1.xx&oh=00_AT8HU_-g647IuQ_4Jnz9rCWE5NPfil8nuAHzokjbz5yZ8w&oe=61D24682
  @width 0.4
\fi

\iusr{Аліса Забой}
\textbf{Maksym Oleynikov} була там в дитинстві кілька разів на новорічних ялинках...

\iusr{Людмила Глебова}
\textbf{Maksym Oleynikov}, пам'ятаю. Здається на другому поверсі там був магазин чоловічого одягу - купували батькові костюм.

\iusr{Татьяна Новза}

Шкода, знищили таку красу. І правду кажучи, будинок профспілок набагато гірший.
Страшний, незграбний.
