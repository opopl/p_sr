% vim: keymap=russian-jcukenwin
%%beginhead 
 
%%file 09_09_2017.stz.news.ua.mrpl_city.2.unikalni_budivli_na_vulyci_georgiivskij
%%parent 09_09_2017
 
%%url https://mrpl.city/blogs/view/unikalni-budivli-na-vulitsi-georgiivskij
 
%%author_id demidko_olga.mariupol,news.ua.mrpl_city
%%date 
 
%%tags 
%%title Унікальні будівлі на вулиці Георгіївській
 
%%endhead 
 
\subsection{Унікальні будівлі на вулиці Георгіївській}
\label{sec:09_09_2017.stz.news.ua.mrpl_city.2.unikalni_budivli_na_vulyci_georgiivskij}
 
\Purl{https://mrpl.city/blogs/view/unikalni-budivli-na-vulitsi-georgiivskij}
\ifcmt
 author_begin
   author_id demidko_olga.mariupol,news.ua.mrpl_city
 author_end
\fi

Гуляючи вулицями рідного Маріуполя, часто ми не звертаємо уваги на те, що
бачимо щодня. Місто приховує від неуважного погляду свою красу і свої таємниці,
адже кожна будівля має власні неповторні портретні риси. Різні деталі,
аксесуари на будинку, як прикраси на жінці, то з'являються, то зникають.
Змінюється настрій будівлі, вираз обличчя. Сьогодні зміни відбуваються особливо
швидко. Зриваються старі вивіски, вікна змінюються на пластикові, стираються
важливі риси. Не всім маріупольцям відомо, що в нашому місті є цікаві будівлі,
чиї деталі вражають самобутністю та унікальністю. Важливим є дізнатися про це
зараз, коли вони ще зберегли свою історико-культурну цінність і кожен з нас
може більш уважно придивитися до них...

Будинки, що мають незвичайні деталі, знаходяться на вулиці Гергіївській – одній
з найдавніших вулиць міста. Ми розглянемо дві будівлі купця Натана Рябінкіна,
які були побудовані наприкінці XIX – на початку XX ст. Обидва будинки
збереглися до наших днів, маючи індивідуальні художні риси, є цінними
пам'ятками маріупольської архітектури.

\ii{09_09_2017.stz.news.ua.mrpl_city.2.unikalni_budivli_na_vulyci_georgiivskij.pic.1}
\ii{09_09_2017.stz.news.ua.mrpl_city.2.unikalni_budivli_na_vulyci_georgiivskij.pic.2}
\ii{09_09_2017.stz.news.ua.mrpl_city.2.unikalni_budivli_na_vulyci_georgiivskij.pic.3}

Будинок № 11 по вулиці Гергіївській, побудований на початку XX ст., прикрашають
цегляні маскарони – декоративні рельєфи у вигляді маски, що зображують людське
обличчя або голову тварини. Фігурна кладка незвичайних античних масок,
складена, швидше за все, зі спеціальної формувальної цегли, незважаючи на
руйнування і пізніші грубі переробки, надає будівлі своєрідну чарівність.
Цікавим є той факт, що найголовніша функція маскаронів – сакральна (священна).
Вони були оберегами, талісманами, \enquote{відвертали} зло. Можливо, князь Натан
Рябінкін вірив, що подібні архітектурні деталі стануть не тільки прикрасою
будівлі, але й зможуть оберігати його від життєвих проблем та примножувати
матеріальні статки.

\ii{09_09_2017.stz.news.ua.mrpl_city.2.unikalni_budivli_na_vulyci_georgiivskij.pic.4}
\ii{09_09_2017.stz.news.ua.mrpl_city.2.unikalni_budivli_na_vulyci_georgiivskij.pic.5}
\ii{09_09_2017.stz.news.ua.mrpl_city.2.unikalni_budivli_na_vulyci_georgiivskij.pic.6}
\ii{09_09_2017.stz.news.ua.mrpl_city.2.unikalni_budivli_na_vulyci_georgiivskij.pic.7}

Інший будинок (№13), який також належав Натану Рябінкіну, має дуже виразні
кахлі, що використовувалися для оздоблення зовнішніх стін будинків. Не тільки в
Маріуполі, але й по всій території України аналогів подібній будівлі не знайти.
Оскільки на будинку представлені всі типи кахлів: рівні й рельєфні, глазуровані
й теракотові (не покриті емаллю). Верхню частину будинку прикрашають глиняні
фігури персонажів грецької міфології та навіть мушкетера, що надають будівлі ще
більшої самобутності.

\ii{09_09_2017.stz.news.ua.mrpl_city.2.unikalni_budivli_na_vulyci_georgiivskij.pic.8}

Вулиця Георгіївська має багато унікальних споруд з власними таємницями та
цікавими історіями. Проте не можна пройти повз будівлі синагоги, руїни якої
вражають і непокоять водночас. Вона була побудована наприкінці XIX ст. та стала
центром культурного і громадського життя єврейської общини. При ній була
заснована Талмуд-Тора (єврейський релігійний навчальний заклад для хлопчиків з
малозабезпечених сімей), в якій викладали єврейську мову, російську грамоту,
первісну арифметику і навіть Біблію.

\ii{09_09_2017.stz.news.ua.mrpl_city.2.unikalni_budivli_na_vulyci_georgiivskij.pic.9}

Цікавим є той факт, що будівля синагоги була єдиною храмовою спорудою, якій
пощастило вціліти під час влади більшовиків та окупантів. Після відступу німців
у будівлі послідовно розміщувалися: гімнастична зала, контора Гіпромеза,
лікарня, медичне училище, вечірньо-заочна школа, школа юних моряків. У середині
1980-х років було прийнято рішення розмістити в синагозі картинну галерею.
Проте взимку 90-х років після занадто сильного снігопаду дах не витримав ваги
снігового покриву і почав руйнуватися. Частину будівельних матеріалів
розтягнули мешканці навколишніх будинків, тим самим остаточно її зруйнувавши.
Руїни, що залишилися від однієї з найстаріших будівель Маріуполя, можна
спостерігати і сьогодні.

Релігійна громада іудеїв неодноразово зверталася до Маріупольської міськради з
проханням відновити історичну пам'ятку міста, але до цих пір безрезультатно.
Стурбовані подібним станом речей маріупольські активісти намагаються проводити
суботники та арт-заходи на території синагоги, таким чином привертаючи увагу
громадськості до зруйнованої будівлі.

Однак ще не все втрачено, адже у липні 2017 р. український культуролог і
громадський діяч Наталя Мусієнко внесла синагогу в онлайн-реєстр громадського
проекту \#SOSмайбутнє, головна мета якого – кардинально змінити ситуацію
спільними зусиллями громади. Залишається сподіватися, що і влада міста не буде
стояти осторонь від вже давно наболілої проблеми.

Інколи, гуляючи вулицями рідного міста, корисно зупинитися і озирнутися
навколо. Можливо, поруч знаходиться будівля, чиї архітектурні деталі неповторні
та унікальні... А може, ваша увага стане в нагоді іншій будівлі, яка вже давно
чекає на допомогу і негайні дії...

\ii{09_09_2017.stz.news.ua.mrpl_city.2.unikalni_budivli_na_vulyci_georgiivskij.pic.10}
\ii{09_09_2017.stz.news.ua.mrpl_city.2.unikalni_budivli_na_vulyci_georgiivskij.pic.11}
