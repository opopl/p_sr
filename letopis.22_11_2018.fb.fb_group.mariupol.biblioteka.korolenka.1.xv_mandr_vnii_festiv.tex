%%beginhead 
 
%%file 22_11_2018.fb.fb_group.mariupol.biblioteka.korolenka.1.xv_mandr_vnii_festiv
%%parent 22_11_2018
 
%%url https://www.facebook.com/groups/1476321979131170/posts/1965196416910388
 
%%author_id fb_group.mariupol.biblioteka.korolenka,gerasimova_nadia.mariupol
%%date 22_11_2018
 
%%tags mariupol,festival,kultura
%%title XV Мандрівний фестиваль Docudays UA
 
%%endhead 

\subsection{XV Мандрівний фестиваль Docudays UA}
\label{sec:22_11_2018.fb.fb_group.mariupol.biblioteka.korolenka.1.xv_mandr_vnii_festiv}
 
\Purl{https://www.facebook.com/groups/1476321979131170/posts/1965196416910388}
\ifcmt
 author_begin
   author_id fb_group.mariupol.biblioteka.korolenka,gerasimova_nadia.mariupol
 author_end
\fi

21 листопада в Центральній міській бібліотеці ім. В.Г. Короленка відбулася
прес-конференція з нагоди відкриття XV Міжнародного мандрівного фестивалю
документального кіно про права людини DOCUDAYS UA в місті Маріуполі та
прилеглих районах. На прес-конференції були присутні читачі бібліотеки,
представники місцевих ЗМІ, члени громадської організації «Приазовська
правозахисна група», священник Української православної церкви Київського
патріархату Отець Роман, директор Централізованої бібліотечної системи для
дорослих  В. О. Лісогор, заступник голови Маріупольського товариства сліпих,
заступник голови міського комітета з доступності Ольга Кочетова та ін. 

Розпочав прес-конференцію Голова правління громадської організації «Приазовська
правозахисна група», координатор фестивалю Docudays UA  Григорій Курачицький.
Він розповів, що у  жовтні місяці  стартував  XV Міжнародний  мандрівний
фестиваль документального кіно про права людини Docudays UA та триватиме по
грудень місяць включно. Найкращі документальні фільми про права людини зможуть
побачити мешканці понад 250 міст і містечок  України.  До міста Маріуполя
фестиваль завітав всьоме. Головна тема цьогорічного фестивалю має назву «Рівні
рівності» та присвячена суспільним викликам, пов'язаним з поняттям рівності.
Фестивальні покази – це не тільки перегляд, а ще і  пропозиція подискутувати
про поширені стереотипи та визначити, як сповідувати практики рівності в нашому
повсякденні, зрештою – «виміряти» рівні рівності: культурні, релігійні,
гендерні, рівності перед законом. Вікторія Лісогор розповіла, що Центральна
бібліотека невипадково обрана партнером фестивалю.  Бібліотекари не перший рік
з власної ініціа-тиви діють в напрямку розвитку рівних можливостей для всіх
категорій населення. Для цього багато зроблено та планується зробити. В
бібліотеці проводяться безкоштовні юридичні консультації, діє сервіс з
електронної доставки документів, організовано чотири комп'ютеризованих робочих
міста для людей з вадами зору, діють клуби за інтересами тощо. Отець Роман
розповів, що взаємодія громадськості, влади і духовенства під час фестивалю
Docudays UA  сприяє сталому розвитку громадянського суспільства. Світлана
Селіверстова, координаторка Міжнародного фестивалю документального кіно про
права людини Docudays UA,  розповіла про те, що на фестиваль цього року
представлено 62 документальних фільми з 36 країн світу. Жителі Маріуполя
зможуть подивитися 15 стрічок з колекції фестивалю. Серед усіх фільмів – фільм,
який отримав приз глядацьких симпатій на Docudays UA-2018 «Віддалений гавкіт
собак» (режисер Сімон Леренґ Вільмонт). Крім того, стрічка озвучена
тифлокоментарем, що робить її доступною для людей незрячих та з вадами зору. Ця
інформація зацікавила заступника голови Маріу-польського товариства сліпих –
Олену Кочетову. Вона відзначила, що дуже важливо робити доступними для всіх
кіносеанси, і крок за кроком прибирати перепони, які заважають людям жити
повноцінним життям. Родзинкою прес-конференції стала виставка чудових картин
дівчинки Ганни, яка навчається інклюзивно і створює дома прекрасні картини.
Натхнення і теми для творчості їй надає те, що вона дуже багато читає. І ту
маленьку частину своїх робіт, що була сьогодні представлена, вона створила під
враженням від прочитання слов'янської міфології. 

Наприкінці всі присутні подивилися фрагмент фільму «Віддалений гавкіт собак»,
прем'єрний показ якого відбудеться  22 листопада о 16.00 в кінотеатрі
«Перемога».

Детальніше з програмою регіональних показів можна ознайомитися на офіційному
веб-сайті Мандрівного фестивалю Docudays UA : \url{www.traveling.docudays.org.ua}.
