% vim: keymap=russian-jcukenwin
%%beginhead 
 
%%file 08_09_2021.fb.tovt_anastasia.1.strana_ua_sankcii_rnbo
%%parent 08_09_2021
 
%%url https://www.facebook.com/anastasija.tovt/posts/4345240448894325
 
%%author_id tovt_anastasia
%%date 
 
%%tags cenzura,danilov_aleksei,rnbo,sankcii,strana_ua,svoboda_slova,ukraina,zhurnalist
%%title Мы, журналисты, будем бороться за нашего общего ребёнка - за издание Страна.ua
 
%%endhead 
 
\subsection{Мы, журналисты, будем бороться за нашего общего ребёнка - за издание Страна.ua}
\label{sec:08_09_2021.fb.tovt_anastasia.1.strana_ua_sankcii_rnbo}
 
\Purl{https://www.facebook.com/anastasija.tovt/posts/4345240448894325}
\ifcmt
 author_begin
   author_id tovt_anastasia
 author_end
\fi

Сегодня я вместе со Svetlana Kryukova выступала с речью на заседании
парламентского Комитета  по свободе слова. 

Мы обратились к народным депутатам с призывом дать свою оценку блокировке СМИ
путём введения санкций СНБО.

По итогу комитет не выдал резолюцию с осуждением санкций против «Страны».
Комитет согласился лишь обратиться в СНБО, чтобы те назвали причину введения
санкций.

\ifcmt
  ig https://scontent-frx5-1.xx.fbcdn.net/v/t39.30808-6/241392670_4345240892227614_6437933053091828236_n.jpg?_nc_cat=105&_nc_rgb565=1&ccb=1-5&_nc_sid=8bfeb9&_nc_ohc=q6T8GAxme0AAX_Avc-9&_nc_ht=scontent-frx5-1.xx&oh=b921691fdc21c3f4edd37c807b0bb3fc&oe=61466134
  @width 0.4
  %@wrap \parpic[r]
  @wrap \InsertBoxR{0}
\fi

Понятно, что это лишь формальность. Вразумительного ответа от СНБО ни мы, ни
депутаты, ни общество не получит. Потому что любое обвинение нужно доказывать
только через суд. А не через мутные решения СНБО.

При этом наше издание уже две недели живет под санкциями, борется с
ддос-атаками и оспаривает блокировку сайта провайдерами. 

Те, кого это не коснулось, часто не понимают, что это вообще значит - санкции
СНБО? 

Попробую объяснить на простом примере. 

Представьте, что у вас родился ребёнок. Вы его выкормили, вырастили, воспитали.
Он подрос и стал заметно выделяться на фоне других детей. Он быстрее, сильнее,
успешнее. Он острый на язык, часто язвительный, критичный к окружающим - так
же, как к себе. Он яркий, дерзкий и успешный. 

И вот однажды к вам в дом приходит группа людей, волей случая наделённых
властью. 

И заявляет, что ваш любимый ребёнок больше не может жить. И говорят: мы его
отбираем. 

Первая реакция - шок. Как такое может быть? Почему, за что? 

У власти нет ответа на эти элементарные вопросы. Просто - ваш ребёнок больше не
может существовать. Основания? Просто они так решили. 

Публично они начинают рассказывать, что вы своего ребёнка как-то не так
воспитывали, ваш ребёнок плохой, слишком дерзкий и непохожий на других. Но это
все - выдумки и попытки оправдаться за обыкновенный произвол. 

Издание «Страна.ua» - наш ребёнок. Пять лет мы, журналисты, его растили,
вкладывали в него свой талант, амбиции и профессионализм. Каждый месяц нас
читало больше 24 млн человек. Мы входим в топ-3 самых популярных новостных
изданий в Украине. Мы украинское медиа, которое никогда не нарушало закон.
Никаких подозрений, уголовных дел и официальных обвинений против нас не было. 

Да, мы были критичны к власти - причем любой. Но мы всегда давали власти слово.
Всегда у них была возможность высказаться, дать свою позицию. 

Например. Когда «слугу народа» Никиту Потураева ударила Ирма Крат, я лично была
одной из первых журналисток, которая позвонила члену комитета по свободе слова,
Татьяне Цыбе, и уточнила все подробности происшествия.

Точно так же мы поступили, когда член комитета по свободе слова Евгений Брагар
посоветовал бабушке продать собаку, чтобы оплатить коммуналку. «Страна» сделала
интервью и с Брагаром, и с той бабушкой, и с ее собакой  @igg{fbicon.face.smirking}  

Не потому, что нам кто-то из политиков нравится или не нравится. А потому что
это наша работа. Именно так и должны поступать профессиональные СМИ. 

Что мы получили взамен? Санкции и блокировку нашего издания. 

Без суда и следствия. Просто потому, что мы кому-то не нравимся. Но разве
симпатии или антипатии могут становиться основанием для блокировки медиа в
современной правовой стране, где представители власти так часто  говорят, что
Украина - это Европа? 

Конечно, нет. Это внесудебная расправа, больше уместная для авторитарного
средневекового государства.

Но мы, журналисты, будем бороться за нашего общего ребёнка - за издание
Страна.ua. Мы продолжаем и продолжим работать, несмотря на блокировки и
санкции. И наше слово найдёт путь к своему читателю.
