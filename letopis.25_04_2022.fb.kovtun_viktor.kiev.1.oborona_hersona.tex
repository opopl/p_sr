% vim: keymap=russian-jcukenwin
%%beginhead 
 
%%file 25_04_2022.fb.kovtun_viktor.kiev.1.oborona_hersona
%%parent 25_04_2022
 
%%url https://www.facebook.com/permalink.php?story_fbid=550770589789428&id=100045694758586
 
%%author_id kovtun_viktor.kiev
%%date 
 
%%tags 
%%title Оборона Херсона очима очевидця..
 
%%endhead 
 
\subsection{Оборона Херсона очима очевидця..}
\label{sec:25_04_2022.fb.kovtun_viktor.kiev.1.oborona_hersona}
 
\Purl{https://www.facebook.com/permalink.php?story_fbid=550770589789428&id=100045694758586}
\ifcmt
 author_begin
   author_id kovtun_viktor.kiev
 author_end
\fi

Оборона Херсона очима очевидця.. 

.. Глава Херсонской ОГА в 2016-2019 годах Андрей Гордеев рассказал, что
российская армия уже 3 марта была в Херсоне из-за хаотичной обороны и
отсутствия координации. Власть областного уровня в лице губернатора Геннадия
Лагуты исчезла из региона в первый день войны. Об этом он рассказал в
комментарии для «Новой газеты»... (вірити йому чи ні.. Справа кожного..) 

\ii{25_04_2022.fb.kovtun_viktor.kiev.1.oborona_hersona.pic.1}

".. Представители полиции, СБУ также оставили город без защиты, сбежав в первых
рядах. Действующий глава Херсонской областной госадминистрации Геннадий Лагута
пропал с напутствием для мэра: «Я в этом не участвую».— Глупость или
предательство. Возможно, и то, и другое. Значит, у нас есть четыре линии
обороны. Первое — перешеек между Херсонской областью и Крымом, он весь
полностью заминирован должен был быть. И мне, кстати, сказали, что за неделю до
(ВОЙНЫ — ред.) его разминировали какого-то хера. Дальше — погранцы, но задача
погранцов не вступать в бой, а увидеть факт (Войны — ред.), дать сигнал и
отойти, отодвинуться и так далее. Потом уже идет наша оросительная система.
Оросительная система — она проектировалась в Советском Союзе Министерством
обороны и Министерством сельского хозяйства. Сама структура — как бы форма
канала — не предусматривает его пересечения никаким видом транспорта. В любом
случае, это утруднение движения. То есть логика такая: взрываются мосты, они
все останавливаются, и наша артиллерия (бьет — ред.). Потом третья линия
обороны — это уже сама речка Днепр. До последнего держится Днепр, мост
Антоновский. Если что, он взрывается, и Херсон как бы вообще вне военных
действий, моста нет, мы только охраняем линию воды. Хер его кто перейдет, тот
Днепр. И последняя линия обороны — Каховский плацдарм. Идет эвакуация Новой
Каховки, там выселяются люди, и держится каховский плацдарм. Там делаются
редуты, капониры. На все это отводится сутки. Вот эти сутки нам надо было
остановить их на входе. Не получилось, — пояснил Андрей Гордеев.

Он добавил, что украинские силы, находясь в меньшинстве, удерживали Антоновский
мост, перекинутый через Днепр, два дня. В основном это была территориальная
оборона, которую так и не организовали перед началом войны... —

Началась война, а у нас ни одного блокпоста нет, ни одного! Все хотят что-то
делать, никто не знает, что делать. Никем никто не управляет. Я захожу к этому
комбригу теробороны, говорю: «Дима, что такое?» Даже карты Херсона у него не
было на столе. Понимаешь? Добровольцы прибегали в течение дня, стояли около
военкомата, не знали, что делать. К ним никто не выходил. В 7 часов вечера
подвезли автобус, и их увезли в село на Днепрянское, где должна была быть база
теробороны. В 7 утра того же дня было совещание по обороне области. Комбриг
Ищенко говорит: «Ребята, <...>, мне нужен грузовик, забрать оружие со склада к
себе на базу». Те говорят: «Все хорошо, жди, сейчас будет». Он звонит в 8 утра:
«Где мой грузовик?» Говорят: «Подожди, сейчас будет». Якименко он звонил,
замгубернатора по обороне. Звонит в 9 утра, звонит в 10 утра — подожди. А в 11
утра Якименко ему говорит: «Извини, обратись к мэру, потому что я вообще сейчас
далеко, в эвакуации». То есть это в то время, пока он обещал транспорт, он
просто убегал с администрацией вместе. В 11 утра комбриг уже понимает, что у
него нет машины. И он случайно узнает, что едет машина с оружием в Алешки —
догоняет 59-ю бригаду, которую там разбили. Он говорит: «Я ее останавливаю и
скидываю с нее 660 автоматов с боекомплектами». И все, она поехала. Хорошо, что
хоть это взял, потому что машина не доехала, ее расхерачили на дамбе. И с этой
машины выдают 400 автоматов в первый день добровольцам на Днепрянском. На
Антоновском мосту идет бой. Надо взять под охрану мост — типа, все туда. И они
пешком где-то километра три пошли, на этот Антоновский мост. Пришли туда, там
ночь уже была, глухая ночь. По ним <...>, они увидели кишки и как деру дали все!
Где-то половину автоматов вернули, половина так, без вести пропала. Как и люди.
Ну, в смысле, где-то залегли на дно, забоялись, — рассказал об обстановке
первого дня войны Андрей Гордеев.

«В итоге на вооружении у подразделений терробороны осталось пару сотен
автоматов и 50 реактивных гранат РПГ-18. С этим арсеналом бойцы попытались
устроить засаду на российскую бронетехнику в Сиреневом парке Херсона. 36 из них
погибли. В общей сложности сообщается почти о семи десятках погибших участников
территориальной обороны... Батюшка наш херсонский похоронил 67 ребят из
теробороны, неопознанных, сам. Они просто как собаки, на 20 сантиметрах вот так
глиной присыпаны, без гробов, без ничего! Вот батюшка их всех пофотографировал,
нумерацию сделал, сделал в телеграме закрытый чат. Победим, говорит, потом
будем их всех доставать, — рассказал экс-глава Херсонской ОГА...

P. S... Слов просто нет.. ну не хочется  верить.. Поэтому на 60 день войны нужна
правдивая информация от \enquote{власти}...живые а тем более мёртвые должны знать
фамилии.. И всем Необходима помощь Арестовича...
