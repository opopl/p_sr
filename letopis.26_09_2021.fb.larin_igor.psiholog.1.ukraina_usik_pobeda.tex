% vim: keymap=russian-jcukenwin
%%beginhead 
 
%%file 26_09_2021.fb.larin_igor.psiholog.1.ukraina_usik_pobeda
%%parent 26_09_2021
 
%%url https://www.facebook.com/PsychologLarin/posts/4618914334814934
 
%%author_id larin_igor.psiholog
%%date 
 
%%tags __sep_2021.usik.pobeda.dzhoshua,boks,cerkov,chelovek,detstvo,hristianstvo,mirovozrenie,obschestvo,patriotizm,pobeda,pravoslavie,psihologia,sport,ukraina,upc,usik_aleksandr,vzroslenie
%%title Вчера, на боксерском ринге, для всего мира победила Украина, именно Украина и только для самой Украины это не так
 
%%endhead 
 
\subsection{Вчера, на боксерском ринге, для всего мира победила Украина, именно Украина и только для самой Украины это не так}
\label{sec:26_09_2021.fb.larin_igor.psiholog.1.ukraina_usik_pobeda}
 
\Purl{https://www.facebook.com/PsychologLarin/posts/4618914334814934}
\ifcmt
 author_begin
   author_id larin_igor.psiholog
 author_end
\fi

Вчера, на боксерском ринге, для всего мира победила Украина, именно Украина и
только для самой Украины это не так.

Мы живем каждый своей жизнью, своей болью, своими чувствами.

И своим опытом. 

Тем который сложился из наших предыдущих контактов. 

Наша психика это, по сути, история наших контактов.

Качественного опыта, сегодня,  у нас у всех, катастрофически мало.

И именно сейчас мы его живём и формируем.

Все вместе.

\ifcmt
  ig https://scontent-lga3-1.xx.fbcdn.net/v/t1.6435-9/243003975_4618939601479074_7292917941260411148_n.jpg?_nc_cat=101&ccb=1-5&_nc_sid=8bfeb9&_nc_ohc=O7td6qNc-tMAX8TXzRg&_nc_ht=scontent-lga3-1.xx&oh=7023927b6fd21929d3632a49432de577&oe=61773FA5
  @width 0.4
  %@wrap \parpic[r]
  @wrap \InsertBoxR{0}
\fi

Как вы поступите с сыном-подростком, воткнувшимся в поганую историю? 

Правильно, как любящий родитель, вы попытаетесь разобраться, что его вовлекло в
неё. 

Из каких чувств он,  интеллигентный, воспитанный и подающий надежды юноша,
вечером ушёл из тёплой, уютной квартиры и оказался  в плохой компании? 

И если вы тот самый осознанный родитель, вы попробуете разобраться и рано или
поздно придёте к тому, что в этой истории большая часть ответственности за
происходящее лежит именно на вас. 

И только на вас.

А если, в этот не простой момент его становления, вместо своей любви, внимания
и заботы,  вы станете давать ему оценку и учить жизни, то ваш сын выберет то
место и тех людей, где найдёт то понимание и наполнение, которое ему необходимо
и которое ему дадут уже другие фигуры, которые он назначит родительскими.

А если вы, не дай бог,  будете, запутавшемуся и не поддержанному вами ребёнку,
постоянно говорить что он мудак, вы потеряете его вовсе.

Он уйдёт. 

И не просто уйдёт. Ещё и отречётся.

Уйдёт с теми и туда где его, как ему видится, греют, любят и самое главное
слышат. 

Кстати, об такую форму питаются всякого рода секты.

И когда нибудь вы увидите своего мальчика, активным участником той самой
преступной компании и искренне будете недоумевать, почему в такой прекрасной,
интеллигентной семье родилось такое недоразумение.

Вы и здесь будете снимать с себя ответственность.

Будете выискивать кривые гены в пятых поколениях, совершенно отказываясь
осознать простую вещь.

Этой беде дали взойти именно вы. 

Это ваших рук дело.

И именно вы не оставили ему другого выбора.

Александр великолепный боец, яркая личность и похоже, очень тёплый человек. 

Но одновременно мне не близко то что в последнее время артикулирует Усик.

Но это не его позиция. 

Это важно осознавать нам. Как бы взрослым. 

И не пиздить человека за ее не сформированность. 

Её формирование шаги другого порядка. В том числе и наши.

Внутренние качели видны невооруженным глазом.

Полярные шаги от целования флага, до любви с церковью, которая открыто
враждебна этому флагу, от гопака и оселедца, до Иисуса, могут говорить о
непростых внутренних процессах.

И именно в таких процессах человеку нужен человек.

В психологии такое состояние называется пограничным. Особенно ярко и явно оно
наблюдается у подростков в момент формирования личностного Я.

В профессиональном спорте мирового уровня история внутренней не зрелости
качественного и профессионального спортсмена-это обычная история. 

Это плата за успех.

Но, никогда и не при каких условиях нельзя отталкивать от себя ребёнка.

Даже, если этот ребёнок несёт полную чушь-он ребёнок.

За любого ребёнка нужно драться. 

Особенно если этот ребёнок наш.

Потребность в эмоциональной привязанности к родительской фигуре это вопрос
жизни и смерти. 

Её, человек ищет всю свою жизнь.

Не дадите вы, дадут другие.

Что, сейчас похоже и происходит.

А если вы конкурируете с ребёнком, а вы конкурируете, тогда вы сами ещё дети.

Один из самых фундаментальных маркёров зрелой особи,  идентичности , это умение
выдерживать детские аффекты.

Нельзя бить ребёнка за отсутствие позиции. 

Важно помочь ему её сформировать. 

Позиция может появиться только у зрелой, плотной и сформированной личности.

В противном случае, человек отсутствие этой плотности компенсирует интроектами.

Чужими знаниями, мыслями, мировоззрением.

И в этот момент формирования личности очень важно кто именно окажется рядом.

Мы вот, выбрали кричать: «Ату его!»

Только помните, пожалуйста, что происходит с подростком когда от него
отворачиваются значимые для него люди.

И кто в итоге за это в ответе.

PS

Спасибо за бой. Невероятно эмоциональное зрелище.

Пояса едут домой.
