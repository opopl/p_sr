% vim: keymap=russian-jcukenwin
%%beginhead 
 
%%file 06_01_2022.stz.news.lnr.lug_info.3.poslanie_rozhdestvo
%%parent 06_01_2022
 
%%url https://lug-info.com/news/rozdestvenskoe-poslanie-arhierea-luganskoj-eparhii-mitropolita-panteleimona
 
%%author_id 
%%date 
 
%%tags rozhdestvo,lnr,donbass,vera,cerkov
%%title Рождественское послание архиерея Луганской епархии митрополита Пантелеимона
 
%%endhead 
\subsection{Рождественское послание архиерея Луганской епархии митрополита Пантелеимона}
\label{sec:06_01_2022.stz.news.lnr.lug_info.3.poslanie_rozhdestvo}

\Purl{https://lug-info.com/news/rozdestvenskoe-poslanie-arhierea-luganskoj-eparhii-mitropolita-panteleimona}

"Христос рождается, — славьте!

Христос с небес, — встречайте!

Христос на земле, — возноситесь!

Пойте Господу, вся земля, и с

веселием воспойте, люди:

ибо Он прославился".

(1-я песнь канона Рождества Христова)

Возлюбленные о Господе священники и диаконы, честное иночество, дорогие братья
и сестры!

В этот торжественный и священный день искренне поздравляю вас со светлым
праздником Рождества Господа Бога и Спаса нашего Иисуса Христа!

\ii{06_01_2022.stz.news.lnr.lug_info.3.poslanie_rozhdestvo.pic.1}

Когда пришла полнота времени, Бог Слово принял человеческую плоть и родился в
городе Вифлееме от Приснодевы Марии для спасения всего рода человеческого.
Согласно изречению блаженного Феофилакта Болгарского: \enquote{Слово, оставаясь тем,
чем Оно было, стало тем, чем оно не было}, — в этом открылась великая
благочестия тайна Боговоплощения (1Тим. 3:16). Пытливый человеческий разум во
все времена пытался в это чудо проникнуть, но эту тайну можно воспринять только
верой. Как говорит святитель Иоанн Златоуст: \enquote{Все наши важнейшие догматы чужды
умствований и доступны только вере}.

Поэтому сегодня все мы, верные христиане, собравшись в храмах Божиих,
присоединяемся к бесчисленному сонму ангелов небесных и воспеваем вместе с ними
песнь, содержание которой открывает нам тайну строительства нашего спасения:
\enquote{Слава в вышних Богу, и на земли мир, в человецех благоволение} (Лк. 2:14). Все
силы небесные радуются о рождении Спасителя, так как в пришествии Христа в этот
мир, Бог открывает спасение людям. В этой радости мы приветствуем друг друга и
воздаем славу нашему Богу.

Родившийся Христос открывает человеку путь к восстановлению мира с Богом,
приобретения мира душевного, который некогда был утрачен через грехопадение. В
тайне боговоплощения Христос дает нам возможность примириться со своей
совестью, примириться друг с другом, в конце концов — получить прощение грехов,
которые и являются первопричиною всякой брани, всякого неустройства и всякого
зла. Таким образом, вместе с ангелами мы воспеваем Родившегося Господа Иисуса
Христа как Примирителя неба и земли, как Примирителя людей с Богом, как
Примирителя человека с человеком.

Второй причиной радостного славословия Бога есть благоволение Божие к людям: «в
человецех благоволение». Это Божие благоволение открывается нам, дорогие братья
и сестры, в Единородном, возлюбленном Божием Сыне, о Котором Отец Небесный
провозгласил: \enquote{Сей есть Сын Мой возлюбленный, о Нем же благоволих} (Мф. 3:17).
Святитель Филарет (Амфитеатров) так поучает нас об этом: \enquote{Чтобы обратить
благоволение Божие на человеков, Сын Божий Сам воплощается и дается нам;
естество человеческое приемлет на Себя; соединяет его на всю вечность со Своим
Божеством; послушанием Своим Богу Отцу исправляет непослушание человеков;
беспредельным смирением Своим укрощает нашу гордость; Божественными заслугами
страданий и смерти Своея, избавляет нас от гнева Божия за грехи наши}.

Подчас бывает так, что люди радуются этому празднику только земной радостью,
еще не вкусив радости духовной и не получив опыта богообщения, не вполне
осознавая важность великого торжества Рождения Спасителя. Видя подобное
отношение, нам не стоит расстраиваться или унывать, так как Бог через Свое
воплощение и рождение, призывает каждого человека ко спасению, дает возможность
каждому выбрать себе житейский путь. Пришествие в этот мир Христа Спасителя
произошло не по принуждению, но исключительно по Его доброй воле. Как говорит
святитель Григорий Чудотворец, епископ Неокесарийский: \enquote{Бог восхотел и снисшел,
совершая спасение людей, ибо в воле Божией — жизнь всех людей}.

Важно подчеркнуть, что подлинная любовь проявляется только в свободной воле,
поскольку в любви нет принуждения. Именно с таким величайшим почтением и
благоволением Бог относится и к нашему выбору, к свободной воле человека. Вот
как говорит об этом святитель Андрей Кесарийский: \enquote{Мое присутствие не
насильственное, ибо Я стучусь в двери сердца и радуюсь вместе с отверзающими об
их спасении}.

Рождение Христа Спасителя имеет прямое отношение к каждому человеку. После
такого осознания просто невозможно, немыслимо жить без Бога. Именно об этом
говорит святитель Иоанн Златоуст: «Бог пришел на землю, а человек — на небо,
все соединилось. Бог принимает мою плоть, чтобы освятить меня; дает мне Своего
Духа, чтобы спасти меня». И это воистину так, Бог рождается на земле как
человек! Но для чего? \enquote{Чтобы мы получили жизнь через Него} (1Ин. 4:9). В этих
словах апостол Иоанн Богослов говорит не только о физической жизни, он говорит
о жизни вечной, которая начинается уже здесь на земле, конечно, если человек
проживает ее с Богом. Подтверждая эту мысль, преподобный Иустин (Попович)
поучает: \enquote{Господь по неизмерной любви становится человеком и навсегда остается
Богочеловеком в человеческом мире. Жизнь человеческая только Бого-жизнь, как
жизнь в Боге, приобретает свой вечный смысл. А вне Бога жизнь — это самая
нелепая бессмыслица, преисполненная обиды и горечи. Наша жизнь только в Боге
обретает свой единственно разумный, единственно логический смысл}.

Сегодня к каждому из нас звучат слова праздничного богослужения: \enquote{Христос с
небес, — встречайте!}. То есть — Христос сходит с небес, люди, выходите на
встречу! Здесь усматривается некоторое сходство с притчей о десяти девах, ибо и
там, в полуночи, раздавался глас: \enquote{Се жених грядет, исходите в сретение его}
(Мф. 25:6). Но разница в том, что в притче приходит Судия, а в Рождестве
Христовом — Спаситель, Который не одномоментно совершает суд, но дает еще время
каждому из нас избавиться от безумства и приобрести мудрость.

К этой встрече с Христом мы готовились заблаговременно: убирали свои жилища,
обновляли свою одежду, готовили вкусную пищу. Казалось бы, что по внешним
признакам все указывает на то, что у нас совершается праздник. И такая
прозаичная, земная часть торжества, видимо, не излишня, но для Господа не она
ценна. Чего ждет Христос от каждого из нас?

Отвечая на этот вопрос, давайте посмотрим, как родился Христос! Богомладенец
явился миру в вертепе, Его обвили в пелены и положили в ясли (Лк. 2:12). В этом
усматривается кротость, смирение и умаление Царя Славы. \enquote{Научитесь от Меня, ибо
Я кроток и смирен сердцем...} (Мф.11:29) — говорит Спаситель. Поэтому, дорогие
братья и сестры, — достойной встречей Родившегося Христа будет наша внутренняя
благонастроенность духа, очищение своего сердца от грехов, молитвенность,
подлинное смирение и кротость, — это будет залогом приобретения благодати.
Приложим усилия, чтобы встретить Христа по примеру непосредственных свидетелей
этого чуда: \enquote{Встречайте Господа вместе с пастырями желанием спасения, вместе с
Иосифом и Богоматерью — благоговейным успокоением в Нем, вместе с волхвами —
охотным шествием вслед Его, вместе с Ангелами — благодарным славословием Ему},
— поучает святитель Феофан Затворник.

Дорогие отцы, братья и сестры, еще раз поздравляю вас с великим и светозарным
праздником Рождества Христова! В этот радостный день призываю вас поклоняться
Богомладенцу не только словами, но и своими добрыми делами, исполнением Его
заповедей. Если мы доверимся Господу и постараемся Ему послужить деятельной
любовью и милосердием, то очень многое изменится не только в нас самих, но и
вокруг нас. Желаю всем вам благоденственной и мирной жизни, доброго здравия и
необходимых, неизреченных благ от Господа нашего Иисуса Христа.

Христос рождается! Славите его!

С любовью о Богомладенце Иисусе Христе,

митрополит Луганский и Алчевский Пантелеимон
