% vim: keymap=russian-jcukenwin
%%beginhead 
 
%%file 17_01_2022.stz.news.ua.radiosvoboda.1.ukraina_rik_foto.pic.9_10
%%parent 17_01_2022.stz.news.ua.radiosvoboda.1.ukraina_rik_foto
 
%%url 
 
%%author_id 
%%date 
 
%%tags 
%%title 
 
%%endhead 


\ifcmt
  tab_begin cols=2,no_fig,center

     pic https://gdb.rferl.org/8c8e0000-0aff-0242-663b-08d9d90e8928_w1023_s.jpeg
		 @number 9
		 @caption_begin
3 липня. Фан-зона на Спортивній площі в Києві під час прямої трансляції
матчу 1/4 фіналу Чемпіонату Європи-2020 з футболу між збірними України
та Англії на екранах ТРЦ Gulliver в Києві. Більше фотографій про те, як
близько 20 тисяч уболівальників спостерігали за матчем у центрі Києва, у
\href{https://www.radiosvoboda.org/a/photo-vbolivalnyky/31339971.html}{фоторепортажі}
		 @caption_end

		 pic https://gdb.rferl.org/8c8e0000-0aff-0242-660b-08d9d90e886b_w1023_s.jpeg
		 @caption_begin
12 липня. Люди на акції
\href{https://www.radiosvoboda.org/a/u-kyyevi-pochaly-demontazh-modernistskoyi-budivli-kvity-ukrayiny/31355106.html}{протесту
«Квіти України» в Києві}. Активісти виступали проти знесення будівлі
пізньорадянського модерну, розташованої на вулиці Січових Стрільців 
		 @caption_end
		 @number 10

  tab_end
\fi
