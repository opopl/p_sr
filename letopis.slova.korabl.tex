% vim: keymap=russian-jcukenwin
%%beginhead 
 
%%file slova.korabl
%%parent slova
 
%%url 
 
%%author_id 
%%date 
 
%%tags 
%%title 
 
%%endhead 
\chapter{Корабль}

%%%cit
%%%cit_head
%%%cit_pic

\ifcmt
  tab_begin cols=3

     pic https://regnum.ru/uploads/pictures/news/2021/10/22/regnum_picture_16348942292527105_normal.JPG
     pic https://regnum.ru/uploads/pictures/news/2021/10/22/regnum_picture_16348943002812285_normal.JPG
		 pic https://regnum.ru/uploads/pictures/news/2021/10/22/regnum_picture_16348942172373491_normal.JPG

  tab_end
\fi
%%%cit_text
Иван Федорович Крузенштерн — мореплаватель, учёный, один из учредителей
Русского географического общества, человек, совершивший первое русское
кругосветное плавание. Экспедиция 1803−1806 годов под командованием И. Ф.
Крузенштерна и Ю. Ф. Лисянского на \emph{кораблях} «Надежда» и «Нева» расширила
горизонты российской науки во многих областях знания — гидрологии и
гидрографии, картографии, биологии и этнографии
%%%cit_comment
%%%cit_title
\citTitle{«Крузенштерн. Вокруг света» — фоторепортаж}, 
Наталья Стрельцова, regnum.ru, 22.10.2021
%%%endcit

