% vim: keymap=russian-jcukenwin
%%beginhead 
 
%%file 09_02_2022.stz.news.lnr.lug_info.1.konkurs_tehnika_molodezh
%%parent 09_02_2022
 
%%url https://lug-info.com/news/organizatory-ob-avili-pobeditelej-konkursa-tehniceskih-proektov-sredi-ucasejsa-molodezi
 
%%author_id news.lnr.lug_info
%%date 
 
%%tags donbass,konkurs,lnr,molodezh,obrazovanie
%%title Организаторы объявили победителей конкурса технических проектов среди учащейся молодежи
 
%%endhead 
 
\subsection{Организаторы объявили победителей конкурса технических проектов среди учащейся молодежи}
\label{sec:09_02_2022.stz.news.lnr.lug_info.1.konkurs_tehnika_molodezh}
 
\Purl{https://lug-info.com/news/organizatory-ob-avili-pobeditelej-konkursa-tehniceskih-proektov-sredi-ucasejsa-molodezi}
\ifcmt
 author_begin
   author_id news.lnr.lug_info
 author_end
\fi

Организаторы определили победителей VIII республиканского конкурса технических
проектов \enquote{Идея – проект – решение}. Об этом с места события передает
корреспондент ЛИЦ.

\ii{09_02_2022.stz.news.lnr.lug_info.1.konkurs_tehnika_molodezh.pic.1}

Организованный Республиканским центром научно-технического творчества (РЦНТТ)
при поддержке Министерства образования и науки ЛНР конкурс прошел на базе
столичного многопрофильного педагогического колледжа Луганского
государственного педагогического университета (ЛГПУ).

\ii{09_02_2022.stz.news.lnr.lug_info.1.konkurs_tehnika_molodezh.pic.2}

Присутствующая на открытии мероприятия первый заместитель министра образования
и науки ЛНР Ольга Долженко отметила, что \enquote{в ходе защиты своего технического
проекта конкурс предоставляет обучающимся возможность продемонстрировать свою
изобретательскую деятельность}.

\ii{09_02_2022.stz.news.lnr.lug_info.1.konkurs_tehnika_molodezh.pic.3}

Заслушав ряд докладов, она отметила глубокие технические знания и яркие идеи
представителей общеобразовательных школ и учреждений среднего специального
образования в области высоких технологий.

\ii{09_02_2022.stz.news.lnr.lug_info.1.konkurs_tehnika_molodezh.pic.4}

\enquote{Когда они представляют свои работы, чувствуешь, с одной стороны, как мало ты
знаешь, а с другой стороны, гордость за то, что наши дети в таком возрасте
имеют такие знания, что у них есть свои идеи, и они эти идеи уже воплощают в
жизнь}, - сказала замминистра.

\ii{09_02_2022.stz.news.lnr.lug_info.1.konkurs_tehnika_molodezh.pic.5}

Долженко выразила уверенность в том, что большинство конкурсантов в будущем
станут членами Малой академии наук, \enquote{где продолжат развивать свои таланты}.

\ii{09_02_2022.stz.news.lnr.lug_info.1.konkurs_tehnika_molodezh.pic.6}

Защита конкурсных проектов прошла по четырем направлениям: \enquote{Транспортная
техника и оборудование; авиация и космонавтика}, \enquote{Информационные и
коммуникационные технологии}, \enquote{Техническое решение проблем охраны окружающей
среды и энергосбережения}, \enquote{Экспериментальные и конструкторские разработки для
использования в учебной деятельности, промышленности и сельском хозяйстве}.

\ii{09_02_2022.stz.news.lnr.lug_info.1.konkurs_tehnika_molodezh.pic.7}

В состав жюри вошли сотрудники Луганского государственного университета (ЛГУ)
имени Владимира Даля, педагогического колледжа ЛГПУ и РЦНТТ, которые отметили
\enquote{актуальность тематики проектов конкурсантов и успешную реализацию идей}.

\ii{09_02_2022.stz.news.lnr.lug_info.1.konkurs_tehnika_molodezh.pic.8}

\enquote{В рамках нашей секции свои работы представители девять человек из разных
городов Республики. Работы были очень интересные, они были представлены из
различных сфер информационных технологий – разработки сайтов, программных и
аппаратных продуктов. Они были выполнены на достаточно высоком уровне и
получили заслуженные оценки}, - рассказал старший преподаватель кафедры
компьютерных систем и сетей ЛГУ имени Даля, член жюри секции \enquote{Информационные
технологии} Константин Зорин.

Он подчеркнул, что конкурсанты проявили грамотность и многогранные знания во
многих областях науки, добавив, что у ряда представленных проектов после
доработок есть шанс реализоваться в жизни.

Заместитель директора по научной работе центра, председатель жюри конкурса
Дмитрий Бончук рассказал, что благодаря участию в конкурсе учащиеся развивают в
себе такие качества, как кропотливость, усидчивость, терпение, любовь к технике
и любознательность.

Так, ученик 11-го класса Лутугинской специализированной школы Денис Овсянников
представил на суд жюри работу из области зеленых технологий.

\enquote{Целью моего проекта была реализация обеспечения загородного участка, коттеджа
или дачи электроэнергией без экологического загрязнения – в виде солнечных
панелей. Мы вместе с моим преподавателем занимались этим долгие месяцы, это
была труднореализуемая идея ввиду малодоступности специализированной
информации. Но в итоге мы все-таки сделали эту работу такой, какой задумывали},
- рассказал он.

Школьник отметил, что побудило его приступить к этой работе забота о будущем
окружающей среды.

\enquote{Каждый человек должен понимать, что природа – это исчерпаемый ресурс, поэтому
нужно внедрять чистые технологии, за которыми, я уверен, будущее}, - подчеркнул
изобретатель.

Денис планирует продолжить модернизацию своего проекта, \enquote{чтобы внедрить его на
практике во благо Республики}.

Победителями конкурса стали: студент Луганского колледжа строительства,
экономики и права Алексей Концевич, который представил проект "Разработка
Системы автоматической парковки в направлении "Транспортная техника и
оборудование; авиация и космонавтика"; студентка Луганского колледжа
информационных технологий и предпринимательства Наталья Краснобай, защищавшая
проект \enquote{Apprentice} в направлении \enquote{Информационные и коммуникационные
технологии}; студент Луганского колледжа строительства, экономики и права
Максим Завгородний, выступивший с проектом \enquote{Разработка детектора газа} в
направлении \enquote{Техническое решение проблем охраны окружающей среды и
энергосбережения}; обучающийся кружка \enquote{Юный изобретатель и рационализатор}
Брянковской городской станции юных техников Никита Варушкин,
продемонстрировавший проект \enquote{Медицинский комплекс диагностики и лечения
пациентов для сельской местности} в направлении \enquote{Экспериментальные и
конструкторские разработки для использования в учебной деятельности,
промышленности и сельском хозяйстве}.

Обладатели призовых мест получили дипломы Министерства образования и науки ЛНР,
а также ценные подарки центра – рюкзаки, канцелярские и компьютерные
принадлежности. Участники конкурса, не ставшие в этом году призерами, получили
на память сувенирную продукцию.

\enquote{Этот конкурс собрал на сегодняшний день 55 участников из самых разных
уголочков нашей Республики. Конечно же, конкурс вызывает восторг, потому что мы
вновь увидели талантливое будущее нашей Республики}, – отметила директор РЦНТТ
Лейда Лоскутова.

Она подчеркнула, что все представленные проекты \enquote{не оторваны от жизни, а
направлены на улучшение жизни в разных ее областях}, в том числе в
здравоохранении, образовании, культуре и технике.

Также директор центра отметила и труд наставников молодых изобретателей,
\enquote{которые сумели найти в детях искру таланта и развить ее, научили детей ставить
перед собой сложные цели и достигать их}.

\enquote{Это люди, которые тратят свое здоровье, силы и нервы на то, чтобы состоялся
каждый ребенок, на то, чтобы у каждого ребенка была своя победа. И хочется
поздравить сегодня нас с теми новыми талантливыми звездочками, которые зажглись
на небосклоне Луганской Народной Республики}, - добавила педагог.

Она выразила уверенность в том, что конкурс из года в год будет собирать все
больше талантливой учащейся молодежи.

Напомним, мероприятие было организовано в рамках дней науки и техники, которые
проходят с 7 по 13 февраля во всех образовательных учреждениях и организациях
Республики под единой темой: \enquote{От Архимеда до Королева}. Цель - создание условий
для социализации и самоопределения учащейся молодежи, повышение роли
технического творчества во всестороннем гармоничном развитии личности и
стимулирование интереса к инновационным разработкам.
