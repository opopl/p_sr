% vim: keymap=russian-jcukenwin
%%beginhead 
 
%%file slova.teatr
%%parent slova
 
%%url 
 
%%author 
%%author_id 
%%author_url 
 
%%tags 
%%title 
 
%%endhead 
\chapter{Театр}

%%%cit
%%%cit_head
%%%cit_pic
%%%cit_text
«Вісь Сковороди» – це мандрівна \emph{театральна майстерня}, яка є спробою дослідження
й занурення в культуру епохи бароко, адже бароковий пласт культури є одним із
найвиразніших та найвагоміших в Україні, що творить місток діалогу із Західною
Європою. Також майстерня зосереджується на розкритті знакової постаті для
української культури Григорія Сковороди крізь \emph{театральне дійство}, спів та
музику.  Про все це ZAXID.NET поговорив з ініціатором проекту та керівником
Шкільного театру Євгеном Худзиком
%%%cit_comment
%%%cit_title
\citTitle{«Нам би хотілось закрутити всю Україну на цій осі»}, 
Катерина Сліпченко, zaxid.net, 29.07.2021
%%%endcit
