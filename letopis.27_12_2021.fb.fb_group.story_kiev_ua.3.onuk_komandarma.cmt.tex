% vim: keymap=russian-jcukenwin
%%beginhead 
 
%%file 27_12_2021.fb.fb_group.story_kiev_ua.3.onuk_komandarma.cmt
%%parent 27_12_2021.fb.fb_group.story_kiev_ua.3.onuk_komandarma
 
%%url 
 
%%author_id 
%%date 
 
%%tags 
%%title 
 
%%endhead 
\zzSecCmt

\begin{itemize} % {
\iusr{Evgen Zachepylenko}
Остер май 1969 год.

\begin{itemize} % {
\iusr{Надежда Владимир Федько}
\textbf{Evgen Zachepylenko} Я в цей час дослужував на Кубі (1967-1969).

\iusr{Evgen Zachepylenko}
\textbf{Надежда Владимир Федько} в мое время тоже туда отправляли дослуживать...

\iusr{Надежда Владимир Федько}
\textbf{Evgen Zachepylenko} Я знаю. У мене один з приятелів служив у 1971-1973-му. Але вони вже служили у напіввійськовій формі. А ми ще були в цивільному.
\end{itemize} % }

\iusr{Надежда Владимир Федько}

Ця історія трапилася на початку грудня 1966-го, тому і згадалася напередодні 2022-го.

\iusr{Сергій Савелій}

Я служив пізніше ( 73- 75) і не в Острі, але так згадалися і курилка ( я також
не курив, але сидів за компанію), і наряди на кухню - і чергові, і
позачергові... Особливо, коли зміну приймали вредні кабельники, які не любили
нас - радіорелейників. Проводили пальцем по верхній мисці - \enquote{жирна} - всю
пірамідку з 10 мисок руйнували і на всіх столах - \enquote{перемивати...}

\begin{itemize} % {
\iusr{Надежда Владимир Федько}
\textbf{Сергій Савелій} 

Знайоме! На двох - 500 мисок, 500 кружок, 500 ложок, 50 кастрюль для супу, 50
мисок для \enquote{бацил}, 50 блюдець для масла... Посуд з алюмінію. Тільки сніданок
закінчився... починаємо мити... Майже перед обідом закінчуємо. З миючих засобів
- господарське мило і сода!

\iusr{Сергій Савелій}
\textbf{Nadegda Volodymyr Fedko} Інколи можна було випросити у кухарів порошок гірчиці...

\iusr{Надежда Владимир Федько}
\textbf{Сергій Савелій} 

А котли мити... Одягаєш старе (підмінне) х/б... Скидаєш чоботи... залазиш у
котел і драїш... Потім вилазиш... з шлангу вимиваєш.

Приходить повар... проводить пальцем по стінці котла...

Жирно! Перемити!

\iusr{Игорь Морочковский}
\textbf{Сергій Савелій} 

75-77, Гатчина, нас, кабельщиків, при передачі чергувань \enquote{сношали} стаціонщикі,
а радіорелейниками у нас була дружба. Доречі, дідовщини у нас також майже не
було.

\iusr{Вадим Скоп}
73-75, Р-408. Учебка под Полтавой, а затем Монголия.

\begin{itemize} % {
\iusr{Сергій Савелій}
\textbf{Vadim Skop} А нам в полк урядового зв'язку прийшли Р- 410 М \enquote{Альбатрос}, тропосферки...

\iusr{Вадим Скоп}
\textbf{Сергій Савелій} 

Р-410 к нам тоже пришли в конце 74-го. Но перед дембелем не было желания
переучиваться. На них работала уже \enquote{молодёжь}. У нас, на \enquote{точке} были ещё
линейщики и спутник. Со всеми были хорошие отношения, жили в одной казарме.

\iusr{Сергій Савелій}
\textbf{Vadim Skop} 

Я написав пісню \enquote{Армейская}, з таким приспівом \enquote{Мы тропосферками стреляем в
небосвод, На разворотах замесив гектары грязи, Сигналу трассу без проблем
пробьет Гвардейский полк правительственной связи!} Є що згадати... З прийдешніми
святами, годок!

\iusr{Вадим Скоп}
\textbf{Сергій Савелій} Дякую. І Вам не хворіти.
\end{itemize} % }

\end{itemize} % }

\iusr{Надежда Владимир Федько}

Десна (Остер) літо 2017-го... Я в жовтій \enquote{британці}.

\ifcmt
  ig https://scontent-frx5-1.xx.fbcdn.net/v/t39.30808-6/270202444_4855536254505794_1834943867048460512_n.jpg?_nc_cat=100&ccb=1-5&_nc_sid=dbeb18&_nc_ohc=UisDsbrXGp0AX-mDQzK&_nc_ht=scontent-frx5-1.xx&oh=00_AT8FcemCiqCjuD1EgGCEf1ocR50FwXkF9wvcxrdc8Z-FRw&oe=61E4C450
  @width 0.3
\fi

\begin{itemize} % {
\iusr{Alex Vines}
\textbf{Надежда Владимир Федько} 

Привет, я 2 года прослужил в Остре, 300й полк. И на губу меня на 7м суток
посадил зам комдива. Уже дембеля, 2го Мая.

\end{itemize} % }

\iusr{Надежда Владимир Федько}

У 2018-му я тиждень читав лекції молоді... Намагався знайти свою казарму... НЕ
знайшов. Все дуЖЖе змінилося за піввіку.

\begin{itemize} % {
\iusr{Evgen Zachepylenko}
наша была по центру одноэтажная саперы инженерные войска ВУС 0501.

\iusr{Evgen Zachepylenko}
когда была СА меня как офицера запаса гоняли на переподготовку в КИСИ и летом в лагерь на 40 дней ...
\end{itemize} % }

\iusr{Alexey Novozhylov}

Навіть і не знаю що сказати, я служив в тій армії у 83-85 роках, але до сих
пір, коли мені інколи сниться армія, я прокидаюсь в шоковому стані. @igg{fbicon.beaming.face.smiling.eyes}{repeat=2} 

\begin{itemize} % {
\iusr{Надежда Владимир Федько}
\textbf{Alexey Novozhylov} 

Ми щодня після підйому і туалету бігали 1 кілометр. Спочатку це було важко..
Потім звикли. Можу сказати, що фізична підготовка плюс добре харчування,
отримані в Острі, дали мені на все життя гарне здоров'я.

\end{itemize} % }

\end{itemize} % }
