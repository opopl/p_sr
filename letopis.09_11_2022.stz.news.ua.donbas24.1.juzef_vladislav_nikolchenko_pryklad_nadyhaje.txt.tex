% vim: keymap=russian-jcukenwin
%%beginhead 
 
%%file 09_11_2022.stz.news.ua.donbas24.1.juzef_vladislav_nikolchenko_pryklad_nadyhaje.txt
%%parent 09_11_2022.stz.news.ua.donbas24.1.juzef_vladislav_nikolchenko_pryklad_nadyhaje
 
%%url 
 
%%author_id 
%%date 
 
%%tags 
%%title 
 
%%endhead 

Маріуполь,Україна,Мариуполь,Украина,Mariupol,Ukraine,Mariupol.MSU,Маріуполь.МДУ,Мариуполь.МГУ,Освіта,Образование,Education,date.09_11_2022
Ольга Демідко (Маріуполь)
09_11_2022.olga_demidko.donbas24.juzef_vladislav_nikolchenko_pryklad_nadyhaje

Юзеф-Владислав Нікольченко — заслужений працівник культури України, приклад
якого надихає (ФОТО)

9 листопада в Україні відзначили Всеукраїнський день працівників культури та
майстрів народного мистецтва

Щорічно 9 листопада разом із Днем української писемності та мови вітають
працівників культури України. З цієї нагоди хочеться розповісти про культурного
діяча, чия унікальна діяльність на ниві культури надихнула багатьох учнів,
студентів, викладачів та працівників культури. Ім'я Юзефа-Владислава
Мойсейовича Нікольченка вже багато років відоме далеко за межами не тільки
Маріуполя, а й України, воно є символом порядності, патріотизму, вірності своїй
справі. В одній особі поєдналися непересічний талант і велика працелюбність,
відповідальність і професіоналізм, інтелігентність і гострий розум, вражаюча
ерудиція та мудрість.

Читайте також: Історія життя видатного археолога Володимира Кульбаки

Сторінки біографії

Юзеф-Владислав Мойсейович народився 2 листопада 1946 року в містечку Сарни
Рівненської області. Батько був військовим лікарем в Сарнах, а мама стала
першою лікаркою-педіатром, яка прибула в Сарни в 1944 році після його
звільнення від німецько-нацистських загарбників. Подвійні імена дітей були
пов’язані з католицькими традиціями сім'ї батька. У брата Юзефа Мойсейовича
(так звертаються до нього колеги і студенти) також було подвійне ім'я —
Борис-Євген. До речі, вони з братом були двійнятками, хоча інтереси у
майбутньому у них були різними — Борис більше захоплювався природничими
науками, а Юзеф — гуманітарними.У 1951 році сім'я переїхала до Рівного, мама —
Олександра Митрофанівна Нікольченко — була призначена завідувачкою дитячого
відділення Рівненської обласної лікарні. А через рік отримала посаду головного
дитячого лікаря Рівненської області, яку обіймала до смерті у 1979 році У 1969
році вона отримала почесне звання Заслужений лікар України. Саме мама робила
все можливе, щоб її сини після закінчення середньої школи отримали вищу освіту
за їх особистим вибором і виросли гідними людьми. Адже батько у 1963 році
залишив сім'ю.

Читайте також: «Укрпошта» випустила нову марку з Кримським мостом: коли та як
купити

Любов до історії та культури

З майбутньою професією Юзеф-Владислав визначився у 5 класі, коли історію
стародавнього світу почала викладати блискуча та неймовірна вчителька Євгенія
Володимирівна Клібанова. Він й досі пам’ятає її слова на уроці з історії
Стародавнього Єгипту: «Ось з цієї хвилини ви, дорогі діти, будете на все життя
закохані у історію…». Завдячуючи Євгенії Володимирівні, її методиці викладання
та глибоким знанням хлопець закохався в історію і одразу ж визначивcя з
майбутньою професією. З 5 класу майбутній історик почав багато читати. Саме
бібліотеки — міські і домашня — формували його як гуманітарія і сприяли
постійному самовдосконаленню. Та й не лише читання поглинало весь вільний час
майбутнього історика. Було ще й захоплення спортом. Між шурхотом книжних
сторінок проявилося ще одне уподобання — фехтування. У 1960−1970-х роках він
був неодноразовим учасником республіканських та всесоюзних змагань із
фехтування на шпагах.

Читайте також: Окупанти в Маріуполі майже знищили мурал «Мілана»: навіщо вони це роблять (ФОТО)

Перша спроба вступити на історичний факультет у 1964 р. не вдалася, оскільки
тоді історія сприймалася як ідеологічна дисципліна, була певна квота, яка не
перевищувала 20\% для випускників шкіл. Юзеф-Владислав тоді склав історію СРСР
на 5, російську мову та літературу теж на 5, але твір з літератури — на 4. І,
як це не дивно, цих високих балів йому не вистачило для вступу до університету.
Але юнак завжди був дуже наполегливим та цілеспрямованим. У 1965 році він
вступає на I курс Харківського державного університету ім. М. Горького (тепер —
Харківський національний університет імені В. Н. Каразіна). Всі три іспити юнак
склав на «відмінно». На курсі було 50 студентів, серед яких переважна більшість
були вже дорослі люди — 1933, 1935, 1938, 1939, 1940, 1942, 1943 та 1944 років
народження, які досить добре пам’ятали трагічні події Другої світової війни та
страшні часи німецько-нацистської окупації. Курс у переважній більшості
складався з наполегливих та успішних студентів. Серед колег-студентів Юзефа
Мойсейовича, випускників курсу, були нині видатні вчені-історики України:
доктор історичних наук, професор Руслан Пиріг та український історик і
громадський діяч, доктор історичних наук, завідувач кафедри новітньої історії
України Запорізького національного університету Федір Турченко.

На думку Юзефа Мойсейовича, його, як майбутнього професійного археолога,
історика і культуролога сформував історичний факультет Харківського державного
університету та його блискуча викладацька і наукова когорта, зокрема професори
В. І. Астахов, П. І. Гарчев, К. Е. Гриневич, В. М. Довгопол, А. П.
Ковалевський, С. М. Королівський, О. О. Кучер, Б. К. Мигаль, І. К. Рибалка, С.
І. Сідєльніков, І. Л. Шерман, К. К Шиян, Б. А Шрамко, А. Г. Слюсарський. На той
час харківська історична школа була найкращою в УРСР.

Читайте також: Маріупольські художниці представили нову виставку в Дортмунді
(ФОТО)

Діяльність в галузі культури

З грудня 1970 року, Юзеф Мойсейович почав працювати у Рівненському обласному
краєзнавчому музеї, де очолив археологічну експедицію та відділ охорони
пам’яток історії та культури. Музейний період його трудової діяльності був
позначений значними досягненнями у дослідженні археологічних пам’яток на
території Рівненської та Волинської областей, у музейній справі, краєзнавстві
та науковій діяльності. З 1973 року по червень 1981 року він працював
заступником директора музею з наукової роботи. У червні 1981 року Юзефа
Мойсейовича було призначено заступником начальника Рівненського обласного
управління культури. А в жовтні 1991 році, вже після проголошення незалежності
України, він був на сесії обласної ради одноголосно обраний начальником
управління культури Рівненської обласної державної адміністрації. Працював на
цій посаді до червня 1998 року. Це були важкі часи. Юзеф Мойсейович робив усе
можливе і неможливе, щоб зберегти в області мережу і кадри установ та закладів
культури і мистецтва; розвивати професійне мистецтво і традиційну культуру,
бібліотечну і музейну справу, забезпечити збереження і вивчення культурної
спадщини нашого народу. Його сумлінна праця була високо поцінована державою: 23
березня 1998 року Указом Президента України Л. А. Кучми йому було присвоєне
почесне звання Заслуженого працівника культури України.

Працюючи в музеї і обласному управлінні культури, Юзеф Мойсейович за
сумісництвом працював на посаді доцента у Рівненському державному інституті
культури, де викладав навчальні дисципліни «Музеєзнавство і краєзнавство»,
«Історію культурно-освітньої роботи», «Організацію діяльності установ культури»
та інші. У 1998 році йому було присвоєне звання доцента з культурології.

Читайте також: Спілка археологів України засудила претензії рф на українську
культурну спадщину

Робота у Маріупольському державному університеті

У 1995 р. Юзеф Мойсейович та його дружина Тамара Марківна отримали привабливу
пропозицію від ректора Маріупольського гуманітарного інституту Костянтина
Васильовича Балабанова переїхати до Маріуполя і працювати викладачами
інституту. Особиста зустріч, спочатку Тамари Марківни, а пізніше Юзефа
Мойсейовича з Костянтином Васильовичем викликала у них відчуття захопленості
від його високого професіоналізму як організатора і керівника авторитетного в
Україні вишу, самовідданого ентузіаста розвитку вищої освіти в державі. Ніяких
сумнівів не було — треба їхати до Маріуполя! Це рішення приймалося Юзефом і
Тамарою Нікольченками у зеніті їхнього авторитету на Рівненщині, у віці, коли
їм обом вже було за 50! Вони вірили у те, що їхня подальша доля повинна бути
пов’язаною з Маріуполем і Маріупольським гуманітарним інститутом.

Спочатку, у січні 1995 року, до Маріуполя переїхала Тамара Марківна працювати
доцентом кафедри української мови та літератури, а згодом, з 28 серпня 1998
року ї Юзеф Мойсейович розпочав свою викладацьку і наукову діяльність у
Маріупольському гуманітарному інституті на кафедрі історичних дисциплін, де
викладав курси «Давня історія України» та «Історія української та зарубіжної
культури». З січня 2000 року до травня 2002 року він виконував обов’язки
завідувача кафедри історичних дисциплін.

Читайте також: У Дії з’являться дипломи та атестати: коли можна скористатися

У 1999 році К. В. Балабановим було прийняте рішення щодо створення факультету
заочного навчання. Його деканом призначили Юзефа Мойсейовича. Він очолював цей
факультет з вересня 1999 року до вересня 2009 року. Це був час його плідної
співпраці з деканатами і кафедрами. За 10 років значно зросла кількість
спеціальностей і студентів заочної форми навчання. З багатьма з них і сьогодні
він підтримує добрі відносини. У 2003 році на історичному факультеті було
створено кафедру культурології та інформаційної діяльності, на яку Юзеф
Мойсейович перейшов працювати і де працює донині. Він високопрофесійний і
авторитетний викладач і науковець. За статистикою Google Academy — автор понад
200 наукових праць і навчальних видань, з високим показником рівня їхнього
цитування, учасник багатьох наукових форумів різних рівнів, постійний керівник
науковою діяльністю студентів. Сьогодні у Юзефа Мойсейовича щаслива велика
родина: дві доньки, два зяті, дві онучки та правнук.

«Сім'я — це головне, що є у кожної людини. Вона завжди була і є поруч, вона
завжди підтримує і надихає», — наголошує Юзеф-Владислав Нікольченко

Читайте також: Серце Маріуполя — Маріупольський університет — активно допомагає
постраждалим маріупольцям (ВІДЕО)

З першого року роботи у навчальному закладі Юзеф Мойсейович користується
великою і заслуженою повагою студентів, у якому вони вбачають не тільки
доброзичливого викладача-професіонала, а й старшого друга і порадника.

«Мій шлях культуролога розпочався 4 роки тому саме на лекції Юзефа Мойсейовича
„Історія світової культури“. Я дуже вдячна за особистий внесок Юзефа
Мойсейовича в долю, розум кожного студента», — зазначила студетка-магістр
спеціальності «Культурологія» Маріупольського державного університету.

«Юзеф Мойсейович допоміг мені обрати спеціальність та натхненно вивчати
культурологію», — наголосила студентка 3 курсу спеціальності «Культурологія»
Софія Шустенко.

Читайте також: Студент МДУ Олександр Денисов завоював 14 медалей на двох
чемпіонатах Німеччини

Як змінилося життя після 24 лютого? 

Після обстрілу терористами мікрорайону «Східний» у січні 2015 року, під який
потрапив і будинок, у якому жили Юзеф Мойсейович, Тамара Марківна і сім'я
доньки Марії, вони без вагань відкинули пропозиції друзів з Рівного повернутися
назад. Незважаючи на всі випробування, вони вирішили, що їхнє місце саме в
Маріуполі, який став для них за чверть століття рідним містом.

«У Маріуполь я закохався. Обожнював Східний, де прожив 24 роки. Вважаю, що
маріупольці пережили багато яскравих подій і вистояли в найтрагічніші хвилини.
Тепер настав час рухатися вперед та пам’ятати, що Маріуполь є невід'ємною
частиною великої держави», — розповів викладач. 

Після повномасштбаного вторгнення Росії в Україну Юзефу Мойсейовчу та його
дружині, Тамарі Марківні, довелося декілька місяців перебувати у підвалі. Їм
вдалося виїхати завдяки колезі, яка вивезла подружжя з окупованого міста.
Чоловік не вірив до кінця, що місто можуть знищити і поки перебував у
Маріуполі, навіть зайшов у зруйнований корпус історичного факультету
Маріупольського державного університету, щоб на власні очі побачити, чи вціліло
щось. Знищення міста, яке стало справжнім домом — не могло не відобразитися на
здоров'ї викладача, проте він не дозволив собі довго відпочивати, адже
найголовніше — це повернення до роботи зі студентами.

Наразі Юзеф Мойсейович мешкає з дружиною у Рівному і дуже щасливий, що має
вірних друзів в Україні і щирих колег у Маріупольському державному університеті
— на кафедрі культурології, на історичному та інших факультетах і, безумовно,
вдячних студентів. Він до нестями закоханий в улюблену справу, якій віддав
майже 53 роки життя. Навіть в іншому місті заслужений працівник культури
України продовжує прищеплювати любов до культури Маріуполя і загалом України.
Звертаючись до студентів, особливо першокурсників, Юзеф Мойсейович завжди
наголошує, що без улюбленої справи життя не може бути цікавим.

Раніше Донбас24 розповідав, про студентку МДУ Софію Шустенко, яка розповідає
про Маріуполь в Центрально-Європейському університеті.

Ще більше новин та найактуальніша інформація про Донецьку та Луганську області
в нашому телеграм-каналі Донбас24.

ФОТО: з відкритих джерел.
