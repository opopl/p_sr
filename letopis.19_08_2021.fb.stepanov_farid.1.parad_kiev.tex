% vim: keymap=russian-jcukenwin
%%beginhead 
 
%%file 19_08_2021.fb.stepanov_farid.1.parad_kiev
%%parent 19_08_2021
 
%%url https://www.facebook.com/StepanovFarid/posts/628113664830449
 
%%author Степанов, Фарид
%%author_id stepanov_farid
%%author_url 
 
%%tags kiev,parad,ukraina,vsu
%%title ЗАЩИТНИКИ - Или: репетиция парада
 
%%endhead 
 
\subsection{ЗАЩИТНИКИ - Или: репетиция парада}
\label{sec:19_08_2021.fb.stepanov_farid.1.parad_kiev}
 
\Purl{https://www.facebook.com/StepanovFarid/posts/628113664830449}
\ifcmt
 author_begin
   author_id stepanov_farid
 author_end
\fi

ЗАЩИТНИКИ.  Или: репетиция парада.

- І як воно: жінці у армії? – я остановился возле строя военных, среди которых выделялась одна девушка.
- Чудово! – искренняя улыбка. Девушка поправляет берет.
- Ви одна серед такої кількості … чоловіків.

Девушка вновь задорно улыбается. Ей приятно такое внимание.

- Маю велику повагу.
- Увагу? – не расслышал из-за шума Крещатика.
- Повагу – смеётся. – І увагу теж.

\ifcmt
  tab_begin cols=3

     pic https://scontent-cdt1-1.xx.fbcdn.net/v/t39.30808-6/240115650_628111291497353_5248771361818586722_n.jpg?_nc_cat=103&_nc_rgb565=1&ccb=1-5&_nc_sid=8bfeb9&_nc_ohc=i8UDGb8q160AX_9G-IM&_nc_ht=scontent-cdt1-1.xx&oh=eca7036575e59368e688b037a9b11385&oe=61237DDD

     pic https://scontent-cdg2-1.xx.fbcdn.net/v/t39.30808-6/239961555_628111341497348_5534736797626584181_n.jpg?_nc_cat=102&ccb=1-5&_nc_sid=8bfeb9&_nc_ohc=ILIZfwUbJ28AX9-ixbK&tn=lCYVFeHcTIAFcAzi&_nc_ht=scontent-cdg2-1.xx&oh=b0e1723d81561629f887174ffbbf6b3a&oe=61245B53

     pic https://scontent-cdg2-1.xx.fbcdn.net/v/t39.30808-6/238538023_628111394830676_8517558190732370597_n.jpg?_nc_cat=102&_nc_rgb565=1&ccb=1-5&_nc_sid=8bfeb9&_nc_ohc=RZGKZBiW7fEAX_vZKzH&_nc_ht=scontent-cdg2-1.xx&oh=2fa2ac948df9d7036b841f7fac66cfee&oe=6123EEEA

  tab_end
\fi

Теперь уже смеются все: девушка, её товарищи по службе и я. Командир тоже
смеётся. И просит девушку привести форму в порядок. Девушка сразу становится
серьёзной. Ещё раз поправляет берет. Украдкой взглянув на меня: ну как?
Поднимаю большой палец: отлично! И ловлю ревнивый взгляд её командира.

\ifcmt
  tab_begin cols=3

     pic https://scontent-cdt1-1.xx.fbcdn.net/v/t39.30808-6/239943018_628111444830671_5610177879493798644_n.jpg?_nc_cat=110&_nc_rgb565=1&ccb=1-5&_nc_sid=8bfeb9&_nc_ohc=qDuA80AL-XYAX-EQkht&tn=lCYVFeHcTIAFcAzi&_nc_ht=scontent-cdt1-1.xx&oh=dec12d16ecabe4a033bedd6c2aa7410b&oe=6123053A

     pic https://scontent-cdg2-1.xx.fbcdn.net/v/t39.30808-6/240160778_628111511497331_1053774969163899303_n.jpg?_nc_cat=102&ccb=1-5&_nc_sid=8bfeb9&_nc_ohc=oPUkUXR8NpQAX-Tr3yV&tn=lCYVFeHcTIAFcAzi&_nc_ht=scontent-cdg2-1.xx&oh=fcdc5e18501217a5a7933d37ef98f5f3&oe=6123775E

     pic https://scontent-cdg2-1.xx.fbcdn.net/v/t39.30808-6/240122539_628111558163993_5034298932634524624_n.jpg?_nc_cat=104&_nc_rgb565=1&ccb=1-5&_nc_sid=8bfeb9&_nc_ohc=NWTpnDzrwnYAX-GXp4h&_nc_ht=scontent-cdg2-1.xx&oh=82052479abcc167cca9c29f34ff4af4d&oe=61237950

  tab_end
\fi


А шум на Крещатике не смолкает. Военные наполнили его весёлым гомоном: смех,
шутки, телефонные разговоры с близкими: «Да, поел». Рядом со спортивным
магазином парочка организовала лёгкий перекус. Нежничают влюблённые, словно не
замечая окружающих прохожих. Чуть в стороне военные музыканты. Каждый со своим
инструментом: одни протирают блестящие бока духовых, другие - умудряются
вздремнуть среди этого шума, ударник отбивает на коленке военный марш.

\ifcmt
  tab_begin cols=2

     pic https://scontent-cdg2-1.xx.fbcdn.net/v/t39.30808-6/240122539_628111558163993_5034298932634524624_n.jpg?_nc_cat=104&_nc_rgb565=1&ccb=1-5&_nc_sid=8bfeb9&_nc_ohc=NWTpnDzrwnYAX-GXp4h&_nc_ht=scontent-cdg2-1.xx&oh=82052479abcc167cca9c29f34ff4af4d&oe=61237950

     pic https://scontent-cdt1-1.xx.fbcdn.net/v/t39.30808-6/239969079_628111661497316_5266115327004174173_n.jpg?_nc_cat=103&_nc_rgb565=1&ccb=1-5&_nc_sid=8bfeb9&_nc_ohc=WO9XJLzAw70AX-_ADjP&tn=lCYVFeHcTIAFcAzi&_nc_ht=scontent-cdt1-1.xx&oh=f824c74bd178ab863a8ee16ad94bcf0c&oe=6122BF81

     pic https://scontent-cdg2-1.xx.fbcdn.net/v/t39.30808-6/240191390_628111721497310_6358983082459558539_n.jpg?_nc_cat=102&ccb=1-5&_nc_sid=8bfeb9&_nc_ohc=RxjZZgK1JEUAX-RCGUH&tn=lCYVFeHcTIAFcAzi&_nc_ht=scontent-cdg2-1.xx&oh=c55415c0958cb1bc5b0f4db6ae391ca1&oe=6122E7C2

  tab_end
\fi

Молоденький солдат позирует родителям. Отец делает фото. Мама светится счастьем
– мой сын. Две девушки в форме, на скамейке напротив метро «Крещатик», что-то
сравнивают на экранах своих смартфонов. Смеются. Под фонарём сидит бывалый
воин. В руках сабля на груди медаль. Он с высоты своего опыта смотрит на
снующую вокруг него молодёжь. Он расслаблен, но огромные руки крепко сжимают
оружие.  

\ifcmt
  tab_begin cols=3

     pic https://scontent-cdg2-1.xx.fbcdn.net/v/t39.30808-6/239961315_628111784830637_78978558249648706_n.jpg?_nc_cat=102&_nc_rgb565=1&ccb=1-5&_nc_sid=8bfeb9&_nc_ohc=RZytw2DJgE0AX-7yMif&_nc_ht=scontent-cdg2-1.xx&oh=fdd19dbedbba1e408329670d475b1d3a&oe=6124ACE1

     pic https://scontent-cdt1-1.xx.fbcdn.net/v/t39.30808-6/240016437_628111841497298_8038626094131548064_n.jpg?_nc_cat=109&_nc_rgb565=1&ccb=1-5&_nc_sid=8bfeb9&_nc_ohc=V4oN4xiQiMQAX_p7bFJ&_nc_ht=scontent-cdt1-1.xx&oh=b2dfd179f6edb66f97f5ee3a3d544c30&oe=6123BC26

     pic https://scontent-cdt1-1.xx.fbcdn.net/v/t39.30808-6/240109964_628111891497293_6000439152730413105_n.jpg?_nc_cat=106&_nc_rgb565=1&ccb=1-5&_nc_sid=8bfeb9&_nc_ohc=vZYVXnRsy-UAX8FR_xL&_nc_ht=scontent-cdt1-1.xx&oh=25b7064192b8f25dc618bd86c343a802&oe=61243388

  tab_end
\fi

\obeycr
- Вставай, нас хотят сфотографировать. – военный потрепал загривок большого лохматого пса.
- Можно?
- Конечно. – мне даже показалось, что он сказал псу: «Улыбайся».
\restorecr

Киваю им с благодарностью. И замечаю ещё одного бойца. С необычной для военного
причёской – хвостиком, выбивающемся из-под берета. Выгоревшие на солнце волосы
и брови. Рука уверенно держит древко знамени. По взгляду читается, что никакой
враг не отберёт у него это знамя.

А вокруг этого импровизированного военного бивака живёт своей жизнью летний
город. Мамы с трудом удерживают детей, готовых потрогать всё. Влюблённые
парочки, держась за руки, гуляют среди военных. Девушки с восхищением смотрят
на происходящее – девушки любят мужчин в форме. Их парни же, всем своим видом,
пытаются показать своим девушкам, что они не менее мужественны. Особые эмоции
всё это действо вызывает у бабушек и иностранцев. В руках смартфоны, в глазах –
восторг. В небе проносится тройка истребителей. Все дружно поднимают головы,
провожая их взглядом. В районе Главпочтамта истребители разлетаются в стороны,
оставляя в небе след, похожий на раскрывшейся цветок. На Крещатике
аплодисменты.

Звучит команда и весь этот хаотичный человеческий муравейник приходит в
движение. Проходит минута и на Крещатике выстраиваются идеально ровные колонны.
«Гвардія!!! Гвардія!!!» - слышится многоголосый крик со стороны Прорезной. И
только собаки пограничников ещё долго переговаривались между собой возле ЦУМа
на своём собачьем языке.

Я смотрел на эти мужественные, немного суровые, но открытые лица и меня
переполняла гордость. Военная служба нелёгкое испытание. Тем более, когда твоя
страна воюет. Но за боевой подготовкой, марш-бросками, строевой и нарядами,
военные остаются обыкновенными людьми. Со своим маленькими житейскими
радостями, бытовыми проблемами, влюблённостями, заботами и простым человеческим
счастьем.

Невольно вспомнил свою военную службу. Но это уже совсем другая история.
