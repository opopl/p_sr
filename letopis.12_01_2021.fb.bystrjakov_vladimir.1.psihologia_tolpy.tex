% vim: keymap=russian-jcukenwin
%%beginhead 
 
%%file 12_01_2021.fb.bystrjakov_vladimir.1.psihologia_tolpy
%%parent 12_01_2021
 
%%url https://www.facebook.com/permalink.php?story_fbid=1705695149583608&id=100004294186082
 
%%author 
%%author_id bystrjakov_vladimir
%%author_url 
 
%%tags obschestvo,psihologia,tolpa
%%title Психология масс
 
%%endhead 
 
\subsection{Психология масс}
\label{sec:12_01_2021.fb.bystrjakov_vladimir.1.psihologia_tolpy}
 
\Purl{https://www.facebook.com/permalink.php?story_fbid=1705695149583608&id=100004294186082}
\ifcmt
 author_begin
   author_id bystrjakov_vladimir
 author_end
\fi

\index{Психология!Толпа}

Этот материал - только для тех, кто любит \enquote{ковыряться} в сути происходящих
событий и в человеческой психологии. ПРОЧТИТЕ ЭТИ УМНЫЕ МЫСЛИ! (не мои)

Друзья мои! Тем, кто любит логику и точность формулировок. Ниже - несколько
цитат из книги «Психология масс» французского социального психолога и социолога
Гюстава Лебона (Gustave Le Bon)

1. Толпа никогда не стремилась к правде; она отворачивается от очевидности, не
нравящейся ей, и предпочитает поклоняться заблуждению, если только заблуждение
это прельщает ее. Кто умеет вводить толпу в заблуждение, тот легко становится
ее повелителем; кто же стремится образумить ее, тот всегда бывает ее жертвой.

2. Утверждение тогда лишь оказывает действие, когда оно повторяется часто и,
если возможно, в одних и тех же выражениях. Кажется, Наполеон сказал, что
существует только одна заслуживающая внимания фигура риторики — это повторение.
Посредством повторения идея водворяется в умах до такой степени прочно, что в
конце концов она уже принимается как доказанная истина.

3. Читая постоянно в одной и той же газете, что А — совершенный негодяй, а В —
честнейший человек, мы в конце концов становимся сами убежденными в этом,
конечно, если только не читаем при этом еще какую-нибудь другую газету,
высказывающую совершенно противоположное мнение. Только утверждение и
повторение в состоянии состязаться друг с другом, так как обладают в этом
случае одинаковой силой.

4. Единственные важные перемены, из которых вытекает обновление цивилизаций,
совершаются в идеях, понятиях и верованиях. Крупные исторические события
являются лишь видимыми следствиями невидимых перемен в мысли людей.

5. История указывает нам, что как только нравственные силы, на которых
покоилась цивилизация, теряют власть, дело окончательного разрушения
завершается бессознательной и грубой толпой, справедливо называемой варварами.
Цивилизации создавались и оберегались маленькой горстью интеллектуальной
аристократии, никогда — толпой. Сила толпы направлена лишь к разрушению.
Владычество толпы всегда указывает на фазу варварства.

6. Управляют толпой не при помощи аргументов, а лишь при помощи образцов. Во
всякую эпоху существует небольшое число индивидов, внушающих толпе свои
действия, и бессознательная масса подражает им.

7. Верить в преобладание революционных инстинктов в толпе — это значит не знать
ее психологии. Нас вводит тут в заблуждение только стремительность этих
инстинктов. Взрывы возмущения и стремления к разрешению всегда эфемерны в
толпе. Толпа слишком управляется бессознательным и поэтому слишком подчиняется
влиянию вековой наследственности, чтобы не быть на самом деле чрезвычайно
консервативной. Предоставленная самой себе, толпа скоро утомляется своими
собственными беспорядками и инстинктивно стремится к рабству.

8. Толпа, способная мыслить только образами, восприимчива только к образам.
Только образы могут увлечь ее или породить в ней ужас и сделаться двигателями
ее поступков.

9. Могущество слов находится в тесной связи с вызываемыми ими образами и
совершенно не зависит от их реального смысла. Очень часто слова, имеющие самый
неопределенный смысл, оказывают самое большое влияние на толпу. Таковы,
например, термины: «демократия», «социализм», «равенство», «свобода» и т. д.,
до такой степени неопределенные, что даже в толстых томах не удается с
точностью разъяснить.

Внимание! Приведенные выше цитаты отобраны из книги, которая была написана еще
до Первой мировой войны!

\ii{12_01_2021.fb.bystrjakov_vladimir.1.psihologia_tolpy.cmt}
