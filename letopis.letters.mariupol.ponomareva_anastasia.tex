% vim: keymap=russian-jcukenwin
%%beginhead 
 
%%file letters.mariupol.ponomareva_anastasia
%%parent letters
 
%%url 
 
%%author_id 
%%date 
 
%%tags 
%%title 
 
%%endhead 

ох! цікаво, що ж там всередині! ) до речі, є такий Парадокс (апорія) Зенона...
древні греки мучались з ним, бо тоді ще не було відповідного математичного
апарату для розв'язання того парадоксу... (це щодо тої формули, що третя зліва
від кружечка посередині - така штука в математиці називається нескінченний ряд.
що рівномірно сходиться)... Короче, є черепаха і Ахіл. Черепаха повзе повільно,
а Ахіл відомий своєю швидкістю... Питання, яким мучались древні греки, таке -
коли Ахіл дожене черепаху?  Черепаха стартує, проходить відстань. Ахіл начебто
її наздогнав, але черепаха знову відповзла. Ахіл знову стрибнув, але черепаха
знову відповзла і т.д. 
