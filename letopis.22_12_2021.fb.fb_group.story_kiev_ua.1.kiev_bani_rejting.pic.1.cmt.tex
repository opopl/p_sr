% vim: keymap=russian-jcukenwin
%%beginhead 
 
%%file 22_12_2021.fb.fb_group.story_kiev_ua.1.kiev_bani_rejting.pic.1.cmt
%%parent 22_12_2021.fb.fb_group.story_kiev_ua.1.kiev_bani_rejting
 
%%url 
 
%%author_id 
%%date 
 
%%tags 
%%title 
 
%%endhead 

\figCapA{1. 1980-е. Вход в \enquote{Караваевские} бани}

\iusr{Станіслав Теліцький}

Власне, у цих лазень була назва \enquote{Галицькі лазні}. Безліч разів бував там у
60-70-і. Доводилось там спілкуватися ще з дореволюційними завсідниками. Парна
там була розкішна, дуже люта. Внизу були перукарня, буфет і аптечний кіоск.
Висіли великі репродукції картин Шишкіна і Айвазовського. Якщо не помиляюсь,
фото зроблено в середині 70-х.

\iusr{Yuriy Bubnov}
\textbf{Станіслав Теліцький} Але це фото \enquote{Караваївської} лазні на Пушкінській.

\iusr{Станіслав Теліцький}
\textbf{Yuriy Bubnov}, 

даруйте! В лазні на Пушкінській не бував. Але фасад дуже подібний. Ви ж знаєте
- будівлі на Жилянській вже давно немає. Була б панорама більша, напевно, і сам
би зрозумів свю помилку.  @igg{fbicon.face.pensive}  Сердечно дякую за теплі спогади! Ми ж з Вами ходили
одними \enquote{стежками} Про \enquote{Галицькі}. Знаєте, що поруч знаходилпсь Галицька площа
(на цьому місці зараз стоїть універмаг \enquote{Україна}.

\iusr{Yuriy Bubnov}
\textbf{Станіслав Теліцький} Вот это наша \enquote{Галицкая} баня, 1956 год.

\ifcmt
  ig https://scontent-frt3-2.xx.fbcdn.net/v/t39.30808-6/269743186_1079430936182353_1437278127859207674_n.jpg?_nc_cat=103&ccb=1-5&_nc_sid=dbeb18&_nc_ohc=15AmhSmyj3gAX_TTNv9&_nc_ht=scontent-frt3-2.xx&oh=00_AT8o-CakueJi2KxysWIHkrgp0vdlYHgubhRKOBt5hRd4Jw&oe=61CB0B7C
  @width 0.4
\fi

\iusr{Станіслав Теліцький}
\textbf{Yuriy Bubnov}, 

там було два входи. Ближче до Старовокзальної - каса і прохід через залу з
буфетом і аптекою. Дальні розташовувались проти сходових маршів, що вели до
чоловічого і жіночого відділень, а також - жо номерів з ванними чи парними. Я
не помиляюсь?  ☺ ️ 

\iusr{Yuriy Bubnov}
\textbf{Станіслав Теліцький} Насколько я помню, так и было. Запомнилось объявление возле касс: \enquote{Семейных номеров нет}. @igg{fbicon.face.wink.tongue} 

\iusr{Регина Бочковская}
\textbf{Yuriy Bubnov} \enquote{Галицкая}, а мы называли баня на Степановской. Ходили с удовольствием, правда очень не нравились очереди. @igg{fbicon.wink} 
