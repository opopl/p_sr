% vim: keymap=russian-jcukenwin
%%beginhead 
 
%%file 30_08_2020.ussr_1983
%%parent 30_08_2020
 
%%endhead 

\subsection{СОВЕТУЮ ПОЧИТАТЬ!! Это было в СССР, 1983 год}

\url{https://www.facebook.com/BatushkaV/posts/1548600201987289}
\url{https://lt.sputniknews.ru/society/20200222/11319864/SMI-uznali-kak-zhivet-devushka-perenesshaya-replantatsiyu-stop-pervoy-v-USSR.html?fbclid=IwAR27Ur0mwiJm0d1N_BeAP9k3zdpGho-6iRfmeCGbK9Q5Ilbb9l-np0h4zvk}

\index[rus]{СССР}
\index{Date!1983}

Время, когда никто деньги на операции не собирал. А когда маленькой литовской
девочке отец нечаянно косилкой отрезал ступни, ей был предоставлен самолет и
персональный воздушный коридор до Москвы. А в Москве за ее ножки боролся
грузинский хирург. И никому в голову не приходило поинтересоваться
национальностью...

В советской Литве жила девочка Раса. Летом 83-го года с ней произошло
несчастье: ее отец-тракторист работал в поле и случайно косилкой ей отрезало
ступни обеих ножек. Расе было 3 года. Умереть — да и только, от потери крови и
болевого шока. Но уже через 12 часов дочка тракториста из колхоза «Вадактай»
лежала на холодном операционном столе в столице СССР.

Для Ту-134, по тревоге поднятому той пятничной ночью в Литве, расчистили
воздушный коридор до самой Москвы. Диспетчеры знали — в пустом салоне летит
маленький пассажир. Первое звено «эстафеты добра», как написали литовские
газеты, а вслед за ними и все остальные. Ножки, обложенные мороженой рыбой,
летят на соседнем сиденье. В иллюминаторах — московский рассвет, на взлётном
поле — с включённым двигателем столичная «скорая».

А в приёмном покое детской больницы молодой хирург Датиашвили — вызвали прямо
из дома, с постели — ждёт срочный рейс из Литвы. «Она, не она», навстречу
каждой машине с красным крестом. «Начальство не давало добро: никто не делал
ещё таких операций, — вспоминает Датиашвили. — Пойдёт что не так — мне не
жить». 12-й час с момента трагедии...

— Вынесли на носилках крошечное тельце, сливающееся с простынёй. Кричу: ноги
где? Ноги переморожены, на пол падает рыба... Рамаз Датиашвили говорит, что
оперировал "на одном дыхании". Сшивал сосудик с сосудом, артерию с артерией,
нервы, мышцы, сухожилия. Через 4 часа после начала операции выдохлись его
помощники, которых он еле нашёл в спящей Москве: медицинская сестра Лена
Автонюк («у неё экзамены, сессия») и сослуживец доктор Бранд.

Рамаз шил один: ещё сухожилие, ещё один нерв. «Я как по натянутой проволоке
шёл: стоит оглянуться — и упадёшь...». Через 9 часов, когда были наложены
последние швы, маленькие пяточки в ладонях доктора потеплели. Пропасть была
позади... ". Тогда за Расу переживала вся страна.

Загудела проснувшаяся Москва: не было в мире таких прецедентов. «Только в
социалистической стране могло такое произойти», — отстучал кто-то восторженную
телеграмму председателю «Вадактая».

История трёхлетней Расы Прасцевичюте и уникальная операция советских врачей,
которая изменила жизнь одного ребёнка, стали символом сплочённости СССР...
