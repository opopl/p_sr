% vim: keymap=russian-jcukenwin
%%beginhead 
 
%%file 21_07_2020.fb.lnr.12
%%parent 21_07_2020
 
%%endhead 
\subsection{Путин подписал закон о повышении пенсий родным погибших военных}
\url{https://www.facebook.com/groups/LNRGUMO/permalink/2863845473727008/}
  
\vspace{0.5cm}
{\small\LaTeX section: \verb|21_07_2020.fb.lnr.12| project: \verb|letopis| rootid: \verb|p_saintrussia|}
\vspace{0.5cm}

Андрей Аркадьев
20.07.2020

Повышенная пенсия будет выплачиваться родственникам участников ВОВ, Героев
Советского Союза и России в случае потери кормильца.

Закон, утверждающий повышенные пенсии для семей погибших военных, подписал
президент РФ Владимир Путин. Документ опубликован на официальном
интернет-портале правовой информации.

Согласно закону, после смерти военного, повышенную пенсию будут получать все
нетрудоспособные члены его семьи.  Повышение выплат также коснется сотрудников,
уволенных с военной и иной приравненной к ней службы (все категории указаны в
45-й статье закона «О пенсионном обеспечении военнослужащих»).

Речь идет о членах семей участников Великой Отечественной войны, Героев
Советского Союза и Российской Федерации, ветеранов боевых действий, жителей
блокадного Ленинграда, инвалидов с детства вследствие военной травмы,
олимпийских чемпионов и некоторых других категорий.
  
