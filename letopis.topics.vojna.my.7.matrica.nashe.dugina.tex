% vim: keymap=russian-jcukenwin
%%beginhead 
 
%%file topics.vojna.my.7.matrica.nashe.dugina
%%parent topics.vojna.my.7.matrica.nashe
 
%%url 
 
%%author_id 
%%date 
 
%%tags 
%%title 
 
%%endhead 

\paragraph{11:35:08 22-08-22 Світла Правда}

⚡️ Дорогие подписчики и гости канала! Вечером 20 августа погибла Дарья Дугина дочь идеолога "русского фашизма" Александра Дугина. Мы располагаем информацией, с подробностями о гибели Дарьи, подлинность которой не вызывает у нас никаких сомнений. Более того, уверены, что у многих из тех, кто знает лично Александра Дугина наш материал только подтвердит предположения.
Около месяца назад Александр Дугин имел очный разговор со своим хорошим знакомым генералом ФСБ и этот разговор был предусмотрительго генералом записан. Во время общения Дугин сокрушается о "неправильном пути" который выбрало российское руководство в проведении специальной военной операции, о недостаточных мерах, об отсутствии решимости у Путина "вести настоящую, большую войну до полного разгрома врага" и т.д. Дугин начал с того, что идеология без готовности автора идти на большие жертвы, лишь утопия. Дугин сказал, что много думал о том почему так, что удерживает Путина от "решительных действий " и как это можно исправить. Далее, Александр рассказал случай, который произошёл с ним во время прогулки: прогуливаясь по улицам в размышлениях, как помочь президенту "спасти страну от падения в бездну" и дать импульс Путину переосмыслить ситуацию и сподвигнуть его к решительным действиям. Дугин увидел черно-белого котенка, который сидел у кого-то во дворе на заборе и при виде Александра направился по забору к нему махнув лапой "испарился", а пройдя ещё буквально сто пятьдесят метров он увидел очень похожего котенка, только мертвого. И тогда, по мнению Дугина, он понял, что это "знак судьбы". Он никчёмный человек, если не готов пожертвовать ради торжества своих идей самым дорогим, что у него есть.
Дугин сам предложил принести в жертву свою дочь, чем вызвал непонимание у собеседника. Но Александр говорил довольно ярко и убедительно. Это он лично сказал генералу о том, что у Путина тоже дочери, и смерть его (Дугина) дочери от рук "западных спецслужб", он воспримет как личную трагедию и горе, и в этот момент наступит перелом в его создании, якобы Путин должен понять, что враг видит слабость, что враг ни перед чем не остановится и следующими могут быть дочери президента, поэтому врага надо уничтожать на поле боя и вырвать победу. Запись разговора была передана напрямую секретарю Совета Безопасности РФ Николаю Патрушеву, после чего к нему вызвали вышеупомянутого генерала. Спустя несколько дней состоялся ещё один разговор генерала с Дугиным, который уже записывали сотрудники ФСБ, и в этом разговоре Дугин без сомнений подтвердил свое предложение и решимость.
Генерал предложил Александру сделать подлог, организовать якобы убийство дочери Дугина, а в реальности вывести Дарью, подложив обезображенный до неузнаваемости труп посторонней женщины, но Дугин решительно отмел это предложение сказав, что если вскроется правда это будет катастрофа для всех. В последствии с Александром связывались ещё минимум два раза и он подтверждал решимость. Покушение готовилось силами ФСБ и заняло две недели. Дата была выбрана не случайно и это событие уже действительно преподнесли Путину как месть "западных кураторов Навального за его неудавшееся отравление", что конечно полный бред, но на президента подействовало и довольно впечатляюще. Сам Дугин не знал ни деталей покушения, ни сроков, для него это был своеобразный "сюрприз". Какие последствия будут у этого громкого убийства увидим уже в ближайшее время. Вот только подобные жертвы обычно напрасны, эмпатия президента Дугиным сильно переоценена. Генерал С В Р



\ii{topics.vojna.my.7.matrica.nashe.dugina.2}
\ii{topics.vojna.my.7.matrica.nashe.dugina.3.odin_furer}
\ii{topics.vojna.my.7.matrica.nashe.dugina.4.zhertva}
\ii{topics.vojna.my.7.matrica.nashe.dugina.5.pole}
\ii{topics.vojna.my.7.matrica.nashe.dugina.6.sobchak}
