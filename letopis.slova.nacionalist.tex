% vim: keymap=russian-jcukenwin
%%beginhead 
 
%%file slova.nacionalist
%%parent slova
 
%%url 
 
%%author 
%%author_id 
%%author_url 
 
%%tags 
%%title 
 
%%endhead 
\chapter{Националист}
\label{sec:slova.nacionalist}

На Светлану и ее супруга Вадима поступили доносы в Генеральную прокуратуру
Украины и СБУ. После того, как правоохранительные органы не нашли в действиях
Светланы Пикты состава преступления, многодетная мать подверглась травле со
стороны \emph{националистов} радикальной группировки С14. Опасения за свою
жизнь и жизнь детей вынудили семью в конце концов, принять решение о
переселении в Россию, а именно – в Ярославскую область,
\textbf{Мужественные киевляне стали нашими земляками}, rweek.ru, 20.07.2018

%%%cit
%%%cit_pic
%%%cit_text
Но дело не в «испорченных русских», а в названии страны. «Украина» - это
полонизированное слово «Окраина», и совсем непонятно, почему западные украинцы,
вошедшие в состав Российской империи, а затем и СССР, предпочли называть свою
страну именно этим совсем не презентабельным словом. Но оно прочно укоренилось
в лексиконе, а большевики после 1917 года пошли на поводу у \emph{националистов},
переименовав Малороссию в Украину (Украинская ССР). Определение «окраина» было
забыто – \emph{националистам} очень сильно хотелось побыстрее идентифицироваться
%%%cit_title
\citTitle{Почему современную Украину назвали «окраиной», а не более престижно - Киевской Русью?}, 
Исторический Понедельник, zen.yandex.ru, 22.02.2021 
%%%endcit

%%%cit
%%%cit_pic
%%%cit_text
Зеленский не смог Украину сшить, он продолжает ее раскалывать по \emph{национальному},
языковому, региональному и массе других признаков, но расколотая страна никогда
не может быть ни сильной, ни богатой. Украинский безмозглый \emph{национализм}
уже стоил Украине Крыма и Донбасса, но дураки склонны во всем винить других. И
своей дуростью и желанием присосаться ко всем Украина надоела всем - и на
Западе, и на Востоке
%%%cit_title
\citTitle{Зеленский продолжает раскалывать Украину по целому ряду признаков}, 
Денис Жарких, strana.ua, 13.06.2021
%%%endcit

%%%cit
%%%cit_pic
%%%cit_text
У нас доминирующая идеология – \emph{украинский национализм}, а \emph{национальной элиты},
которая бы связывала свою и своих детей судьбу с Украиной, - нет, т.к. давно
политические и бизнес лидеры свили свои семейные гнезда далеко за рубежом.
Невероятно, но факт.  Ну, и сколько еще протянет такое государство? На сколько
лет хватит терпения обнищавшему и одурманенному народу безразлично смотреть на
масштабную коррупцию и казнокрадство?  Не поэтому ли украинская экономика и
реальные доходы населения постоянно падают, а средний класс – исчезает? И далее
нас ждут еще более тяжелые времена
%%%cit_title
\citTitle{Украинский национализм - есть, а национальной элиты - нет}, 
Александр Гончаров, strana.ua, 13.06.2021
%%%endcit

%%%cit
%%%cit_head
%%%cit_pic
%%%cit_text
Отметим, что в Украине нет возможности получать образование на русском языке. В
рамках политики украинизации все ВУЗы переведены на украинский язык. В том
числе и в русскоязычных регионах.  В ответ на видео девушки \emph{националисты} начали
травлю девушки.  \enquote{От имени нашего сообщества призываем все компетентные органы,
СМИ, активистов и неравнодушных граждан немедленно отреагировать на эти
антиукраинские заявления Маргариты}, - сказано в сообщении одной из групп
\enquote{мовных активистов}
%%%cit_comment
%%%cit_title
\citTitle{Студентка негативно высказалась об украинском языке}, , kharkov.strana.ua, 16.06.2021
%%%endcit

%%%cit
%%%cit_head
%%%cit_pic
%%%cit_text
Операцію з убивства Коновальця розпочали в серпні 1933 року. Чекіст П.
Судоплатов проник у середовище \emph{націоналістів} через колишнього старшину СС,
галичанина Василя Хом'яка. Останній представив його як розчарованого в
комунізмі та готового долучитися до \emph{націоналістів}. Контактував з дуже багатьма
\emph{націоналістами}. Тричі зустрічався з Коновальцем і спілкувався про розвиток
\emph{націоналістичного руху} в УРСР
%%%cit_comment
%%%cit_title
\citTitle{Евген Коновалець: від життя до смерти у безсмертя (тези за
телепрограмою \enquote{Ген українців})}, 
Ірина Фаріон, blogs.pravda.com.ua, 14.06.2021
%%%endcit

%%%cit
%%%cit_head
%%%cit_pic
\ifcmt
  pic https://img.strana.ua/img/article/3405/na-kontserte-basty-31_main.jpeg
  width 0.4
\fi
%%%cit_text
Некоторые люди, которые пришли на концерт Басты, очень резко реагируют на акцию
\emph{националистов}. В выражениях не стесняются.  \enquote{Идите на концерт, че вы приперлись
бл@ть?! Стоят они бл@ть, плакаты достали! Еб@нутые. Войны бл@ть до Басты не
было?}, - возмутилась одна из пришедших на концерт.  В свою очередь,
\emph{националисты} утверждают на том, что российский рэпер якобы поддерживал аннексию
Крыма Россией
%%%cit_comment
%%%cit_title
\citTitle{На концерте Басты начались перепалки с националистами}, , kiev.strana.ua, 25.06.2021
%%%endcit

%%%cit
%%%cit_head
%%%cit_pic
%%%cit_text
\enquote{Я не считаю, что украинский народ - недружественный. Я вообще считаю,
что российский и украинский - один народ. (...) Народы, у которых языки
отличаются больше, чем украинский и русский, дорожат единством}, - сказал
Путин.  Также он заявил, что в Украине \enquote{выдавливают русский язык из
реальной жизни}.  \enquote{Везде хватает узколобых людей, крайних
\emph{националистов}. Они действуют от чистого сердца, но не от большого ума. Я
не считаю украинский народ недружественным. А сегодняшнее руководство Украины
явно недружественное}, - сказал президент РФ
%%%cit_comment
%%%cit_title
\citTitle{Путин назвал украинцев и россиян единым народом}, , strana.ua, 30.06.2021
%%%endcit


%%%cit
%%%cit_head
%%%cit_pic
%%%cit_text
После того, как автор победного гола матча Украина-Швеция на Евро-2020 Артем
Довбик на пресс-конференции отвечал на русском языке, на него обрушились с
критикой \emph{националисты}.  Нападающего сборной Украины, который побил
рекорд легендарного Мишеля Платини, раскритиковала скандальная писательница
Лариса Ницой, которая и ранее называла украинских футболистов
\enquote{кончеными московитами}. А языковой омбудсмен Креминь потребовал от
футболистов общаться на украинском даже после завершения карьеры
%%%cit_comment
%%%cit_title
\citTitle{Зеленский о русских и русском языке. Что будущий президент говорил в 2014 году}, 
Екатерина Терехова, strana.ua, 03.07.2021
%%%endcit

%%%cit
%%%cit_head
%%%cit_pic
\ifcmt
tab_begin cols=2
  width 0.4
  caption В этом году в Украине выступает много российских звезд, strana.ua

  pic https://img.strana.ua/img/article/3412/pojushchij-revansh-v-79_main.jpeg

  pic https://strana.ua/img/forall/u/0/25/%D0%A1%D0%BD%D0%B8%D0%BC%D0%BE%D0%BA_%D1%8D%D0%BA%D1%80%D0%B0%D0%BD%D0%B0_2021-06-30_%D0%B2_19.55_.01_.png
  width 0.5
tab_end
\fi
%%%cit_text
В этом году в Украине выступит рекордное количество российских артистов.
Беглый просмотр афиш дает понять, что в этом сезоне на украинских сценах будет
часто звучать российская музыка. За одно только лето Украину посетят десятки
топовых российских артистов: Валерий Меладзе, Face, Мумий Тролль, Фараон, Артур
Пирожков, Jah Khalib, Noize MC, Jony, Hammali\&Navai, Скриптонит, T-Fest.
Многие из них дадут не один, а сразу несколько концертов в разных городах
Украины.  Также в Украине выступит целый \enquote{десант} российских стендап-комиков.
При этом \emph{националисты} пытаются кошмарить российских артистов и их поклонников,
но безуспешно
%%%cit_comment
%%%cit_title
\citTitle{Поющий реванш. В Украине этим летом выступят десятки российских артистов}, 
Анастасия Товт; Полина Пронина, strana.ua, 01.07.2021
%%%endcit

%%%cit
%%%cit_head
%%%cit_pic
%%%cit_text
«Сто сімдесят років тому, на світанку доби \emph{націоналізму}, німецький поет Генріх
Гайне виявив красномовну розбіжність між двома типами колективних почуттів:
"Нам [німцям], — писав він, — наказали бути патріотами, і ми стали патріотами,
тому що ми робимо все, до чого змушують нас наші правила. Однак не варто
думати, що це такий самий патріотизм, як і те почуття, що так само називається
тут, у Франції.  Патріотизм француза означає, що в нього зігрівається серце, і
завдяки цьому теплу воно розтягується та збільшується, тож його любові вистачає
не лише на найближчого родича, а й на всю Францію, увесь цивілізований світ.
Патріотизм німця означає, що його серце скорочується і стискається, як шкіра на
холоді, і тоді німець починає ненавидіти все іноземне, не хоче більше бути
громадянином світу, європейцем, а лише провінційним німцем".»
%%%cit_comment
%%%cit_title
\citTitle{Чтобы не наступать на грабли, украинцам нужно изучать историю / Лента соцсетей / Страна}, 
Геннадий Друзенко, strana.news, 31.10.2021
%%%endcit

%%%cit
%%%cit_head
%%%cit_pic
%%%cit_text
Николай Кузнецов погиб смертью храбрых в неравном бою с отрядом УПА под
Львовом.  Олесь Бабий же известен тем что написал марш украинских
\emph{националистов}, а также ряд книг под псевдонимом Хмелик. Родился в селе под
Калушем, служил в Австрийской армии. Жил и работал в Праге. Сидел в тюрьме в
Дрогобыче. Был освобождён во время оккупации Украины. В 44 году, перед
освождением Украины от фашистов переехал в Германию, потом в США, в Чикаго, где
работал в университете и в редакции украинского журнала. Умер в Чикаго в 1975
году. Никогда не был в Киеве
%%%cit_comment
%%%cit_title
\citTitle{Почему в Киеве не должна быть улица в честь Кузнецова? / Лента соцсетей / Страна}, 
Евгений Хурсин, strana.news, 08.11.2021
%%%endcit

%%%cit
%%%cit_head
%%%cit_pic
\ifcmt
  tab_begin cols=4
     pic https://avatars.mds.yandex.net/get-zen_doc/1362956/pub_617c92ab93121f356cd92b06_617cd52908a7eb3ffe366e5b/scale_1200
     pic https://avatars.mds.yandex.net/get-zen_doc/1592246/pub_617c92ab93121f356cd92b06_617cce38f7b11960ad674e84/scale_1200
     pic https://avatars.mds.yandex.net/get-zen_doc/1888829/pub_5d5c198b98fe7900ac172a45_5d5c199ecfcc8600ac894364/scale_1200
     pic https://avatars.mds.yandex.net/get-zen_pictures/4012878/901063758-1635669515797/orig
  tab_end
\fi
%%%cit_text
И снова здравствуйте. Знаете, порой слушаю я шумерских политических деятелей и
задумываюсь над вопросом, а что они такое курят? Вот честно, на трезвую голову
весь тот бред, который они несут, придумать невозможно.  Вот взять, к примеру,
Гоголя Николая Васильевича, того самого величайшего РУССКОГО писателя, который
хоть и родился в с Васильевке Полтавской губернии, но жил и творил в Питере.
Так вот потомки шумеров все никак поделить Николая Васильевича не могут.
Сначала, министр культуры и информационной политики Украины Александр Ткаченко
заявил, что Гоголь — это украинский писатель, а «российская пропаганда» его
присвоила, «принудительно записав в русские».  А теперь и гражданка Фарион, та
самая украинская \emph{националистка}, судя по всему страдающая от Логореей,
"прославившаяся" своими русофобскими высказываниями, такими как ...  Заявила,
что Гоголь является очень трагической личностью в украинской культуре.  А весь
трагизм в том, что Николай Васильевич «сошел с ума» из-за... русского языка.
Дескать, он "...думал на украинском языке, а писал по-русски...", вот и не
выдержала тонкая душевная организация великого писателя. Что? Тетка, ты похоже,
совсем с ума сошла.  Ну это ладно, от злобной старухи другого ожидать не
приходится. Но вот знаете, в чем странность. Меня многие комментаторы (под
моими статьями об украинских дегенера..х), уверяют, что власть это одно, а вот
обычные украинцы так не думают. Однако комментарии по видосиком с интервью,
говорят об обратном
%%%cit_comment
%%%cit_title
\citTitle{Шумерско-украинские страсти вокруг Гоголя}, 
Рассказы дяди Вовы, zen.yandex.ru, 30.10.2021
%%%endcit
