% vim: keymap=russian-jcukenwin
%%beginhead 
 
%%file slova.chistota
%%parent slova
 
%%url 
 
%%author 
%%author_id 
%%author_url 
 
%%tags 
%%title 
 
%%endhead 
\chapter{Чистота}
\label{sec:slova.chistota}

%%%cit
%%%cit_head
%%%cit_pic
%%%cit_text
То есть, вы считаете, что если кого-то мы посадим в тюрьму, то к нам вернется
национальное достоинство?  Нет. Недостаточно. Я же говорю об \emph{очищении}. То есть,
что такое \emph{очищение}? \emph{Очищение} – это жертва. Два человека уходят в тюрьму, это
осознается как неправильная ментальная структура, которая завладела нами, а
остальные проходят ситуацию дересиментализации, \emph{очищения}, переосмысления, что
это путь тупиковый, что нельзя за признание языка прощать коррупцию,
стяжательства, грабеж. Нельзя за эти вещи прощать. Если ты говоришь
по-украински, то ты можешь заниматься стяжательством, коррупцией и грабежом.
Нет, нельзя! Вот понимаете, это жертва, эти в тюрьму, эти – \emph{очищение}
%%%cit_comment
%%%cit_title
\citTitle{Сергей Дацюк: Украина сегодня - не просто попрошайка, она на мусорнике истории}, 
Сергей Дацюк; Людмила Немыря, hvylya.net, 28.06.2021
%%%endcit

