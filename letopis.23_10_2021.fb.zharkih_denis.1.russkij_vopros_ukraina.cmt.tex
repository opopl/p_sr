% vim: keymap=russian-jcukenwin
%%beginhead 
 
%%file 23_10_2021.fb.zharkih_denis.1.russkij_vopros_ukraina.cmt
%%parent 23_10_2021.fb.zharkih_denis.1.russkij_vopros_ukraina
 
%%url 
 
%%author_id 
%%date 
 
%%tags 
%%title 
 
%%endhead 
\subsubsection{Коментарі}

\begin{itemize} % {
\iusr{Денис Наан}
На Украине.

\iusr{Dmitry Lesnoy}
Плохо, что потихоньку.

\iusr{Александр Супрунов}
Раб мечтает не о свободе, а о своих рабах. Украина сама мечтала стать империей, и она ей стала, причем хуже чем совок.

\begin{itemize} % {
\iusr{Андрей Лесто}
\textbf{Александр Супрунов} чем-чем стало? Оно даже реально самостоятельным государством стать не смогло, не то, не империей.
\end{itemize} % }

\iusr{Василь Гутак}
Так і є, стадія "дати люлей Империи" - в нас триває.

\begin{itemize} % {
\iusr{Сергей Семенов}
\textbf{Василь Гутак} поэтому империя будет вынуждена дать люлей, чтобы привести в чувство. Грузины поняли свое место. Кто следующий? ВВП ответил.

\iusr{Василь Гутак}
\textbf{Сергей Семенов} , ну якщо так - то пропадає вся струнка теорія-конструкція Дениса, шо Україна то як Канада, Австралія і т. п. , які залишилися англомовними але вийшли з впливу Імперії.
Як у Вас кажуть - труси одіньте, або хрестик знімайте.

\iusr{Проказница Мартышка}
\textbf{Василь Гутак} Коли вже чоботи в Тихому океані помиєш? @igg{fbicon.face.tears.of.joy}{repeat=5} 

\iusr{Василь Гутак}
\textbf{Проказница Мартышка} , а я тут при чім? Ми обговорюємо допис Дениса, і тему яку Він підняв.
Вас чомусь цікавлю я особисто?

\iusr{Проказница Мартышка}
\textbf{Василь Гутак} Не отмазывайтесь - не в военкомате!

\iusr{Василь Гутак}
\textbf{Проказница} Мартышка , бот?
Допис не читали, бачу.
\end{itemize} % }

\iusr{Богдан Моторин}

В России 7 год пытаются натянуть на уши определенную догму о том, что условный
"русский" украинец - это их собственность и карт-бланш на всякие противоправные
действия в отношении Украины. Но, это делает не только Россия , но и другие
соседние государство. Проблема в том, что власть и в целом государство не
делает не каких шагов в сторону перекрытия таких вот волн "исторической
привязанности к определенному государство". Власть всё делает для того, чтоб
магнит общества отталкивался от друг друга. Хотя решение данного вопроса -
банален и прост. Ждём-с мудрых политиков на горизонте или создаём их сами.

\begin{itemize} % {
\iusr{Елена Лобанова}
\textbf{Богдан Моторин} 

создавать "новых политиков" в пробирке планируете? А в случае со "ждём-с"
уверены, что манна небесная упадёт прямо вам на головы и поможет вам
элементарно выжить в процессе ожидания чуда появления новых политиков? Майдан
похоронил, то хрупкое равновесие, которое помогало существовать Украине, как
единому государству, и мирным путём эту страну уже не сшить воедино. Скорее,
государство Украина постепенно истлеет, как ветхая тряпка, и прекратит своё
существование в качестве государственного образования.


\iusr{Андрей Лесто}
\textbf{Богдан Моторин} иллюзии. Других политиков не будет. Среда не та.
\end{itemize} % }

\iusr{Сергей Вартанян}
а вообще это ниша Бужанского, он у нас историк депутат, который не стоит на страже благосостояния граждан страны, а изучает историю

\iusr{Игорь Кириченко}

Националистам не следует трогать три темы в Украине и всё будет более или менее
хорошо в остальном. Это тема русского языка и национального вопроса, тема
результатов и победы в Великой Отечественной войне и тема православной церкви.
Как только нормализуются эти вопросы всё потом пойдёт своим чередом.

\begin{itemize} % {
\iusr{Ирина Ольховка}
А что будут раскачивать тогда?!)) Им денюжки на чем-то зарабатывать нужно?))) Не на работу же идти,право!)

\iusr{Полина Дьяченко}
\textbf{Игорь Кириченко} это три кита на которых стоит украинский национализм. Больше им и трогать)) нечего по сути

\iusr{Проказница Мартышка}
\textbf{Игорь Кириченко} До первого Майдана так и было. А в 2014 году сорвались с цепи - теперь ныкается Украина, как бродячая собака, подъедая объедки за "цивилизованными" странами!
\end{itemize} % }

\iusr{Наталья Ковтун}

А мне видится так. Те страны, о которых упоминали вы, Денис, действительно
изначально были колониями британцев, которые потом ушли. Теперь дружат... Всё
такое. А американцы, вообще, на других землях ☝ ️ , а не на территории
Великобритании, создали себя такими какие они есть. Сплотились вокруг бога под
названием "злато". И все, кому дорог этот Бог- прекрасно себя там чувствует. По
всему Миру таких людей много. Потому и стремятся и гордятся своими "зелёными
картами" многие.... А вот шотландцам и ирландцам- как-то не уживаются со своей
империей. Я не историк и не политолог. Просто читатель. И мне кажется, что
Украинского Мира не может быть, потому, что мы на территории Русской Империи -
всё пытаемся построить украинскую. Не я! Считаю, что строить украинское- это
шаг назад, или даже больше и глубже. Но точно не прогресс! Я не слышала за эти
последние 8лет(а может и даже больше), чтобы кто нибудь из политиков
переодевшихся во всё украинское, предложил бы какой-то путь! Хоть в капитализм,
хоть в социализм, хоть какую-то систему ведения государственности! Их главное
жизненное кредо " Я не мо@каль!"

\begin{itemize} % {
\iusr{Проказница Мартышка}
\textbf{Наталья Ковтун}  @igg{fbicon.thumb.up.yellow}  @igg{fbicon.hands.applause.yellow}{repeat=3} 
\end{itemize} % }

\iusr{Юрий Лукшиц}

Либо будут уважаться права всех этнических групп в Украине, либо "украинская
Украина" будет существенно меньше нынешних границ.

\begin{itemize} % {
\iusr{Елена Лобанова}
\textbf{Юрий Лукшиц} если, в принципе, сохранится в виде государственного образования
\end{itemize} % }

\emph{Полина Дьяченко}

Не надо называть то, что во власти « элитами». Очень важно называть все своими
именами и нужными словами. Называя этих ушлепков « элитой» вы даете им
некоторые моральные преимущества которых они не достойны и уже достойны не
будут. И если люди их будут считать элитой и ожидать от них действий как от
элиты то опять повсеместное разочарование и все, что за этим следует.

\iusr{Тамара Адоевцева}
Элита- подразумевается ЛУЧШИЕ. А где Вы в элите видите лучших ? Одно.....

\iusr{Юрий Александров}

На данном этапе Украина России не нужна, в России это поняли, на Украине нет.
Ну зачем сейчас, в период роста России нужны какие то территории? Крым,
понятно, это безопасность, но остальное не надо. Как развиваться если надо
будет восстанавливать?

\iusr{Игорь Подопригора}
а нужно ли?

\iusr{Наталья Асуховская}
Точно, ни ума , ни фантазии, зато понты @igg{fbicon.laugh.rolling.floor}{repeat=3} 

\end{itemize} % }
