% vim: keymap=russian-jcukenwin
%%beginhead 
 
%%file 02_03_2023.stz.news.ua.donbas24.1.arhiv_vijny_hto_zbyraje_svidchennja_mrplciv_kyiv.txt
%%parent 02_03_2023.stz.news.ua.donbas24.1.arhiv_vijny_hto_zbyraje_svidchennja_mrplciv_kyiv
 
%%url 
 
%%author_id 
%%date 
 
%%tags 
%%title 
 
%%endhead 

Ольга Демідко (Маріуполь)
02_03_2023.olga_demidko.donbas24.arhiv_vijny_hto_zbyraje_svidchennja_mrplciv_kyiv
Маріуполь,Україна,Мариуполь,Украина,Mariupol,Ukraine,date.02_03_2023

«Архів війни» — хто збирає свідчення маріупольців у Києві (ФОТО)

Всі, хто став жертвою та свідком воєнних злочинів, можуть розповісти про це
світові

Влітку 2022 року в Україні було започатковано проєкт «Архів війни», створений
ГО «DOCU DAYS». Це об'єднана база відео та аудіоматеріалів про війну в Україні.
Представники проєкту зберігають та упорядковують воєнну хроніку. За допомогою
найпростіших засобів документування вони намагаються зробити воєнні злочини рф
видимими та покарати винних. У Києві в рамках «Архіву війни» маріупольці
діляться пережитим досвідом. Їхні свідчення збирають представники проєкту,
зокрема і актор театру та кіно, педагог і тренер Ігор Аронов.

Читайте також: Не бачать світла та місяцями не миються — як живуть діти у
Бахмуті та куди їх евакуюють

Про координатора проєкту

Про проєкт «Архів війни» Ігорю Аронову розповіла знайома і він вирішив
долучитися. До речі, актор колись хотів стати футбольним журналістом. Ще у
студентські роки Ігор працював журналістом у «Спорт-Експрес в Україні». З
огляду на це він мав досвід спілкування з людьми. Збирати свідчення почав
наприкінці червня. Вже опитав приблизно 50 — 60 осіб. Зустрічався з херсонцями,
чернігівцями, мешканцями Київської області. Проте найбільше спілкувався саме
маріупольців. Ці історії пов'язані з граничними для людського організму
психологічними, емоційними чи фізичними навантаженнями.

«Насправді ці свідчення з перших вуст — є унікальною базою. Це важливо і для
нашої історії, і для подальших трибуналів. Маріупольці пережили великий стрес,
тому їхні розповіді відрізняються від інших», — наголосив Ігор.

У актора є своя історія, адже повномасштабне вторгнення змусило змінити всі
плани, відмовитися від багатьох мистецьких проєктів. Зокрема, він є автором
освітньо-дослідницького проєкту «Актор без прикриття», в рамках якого Ігор
проводить психофізичні воркшопи для акторів та всіх зацікавлених. Водночас
Аронов працював тренером у художніх фільмах і помічником режисера в театрі. У
березні 2022 року йому запропонували бути тренером акторської майстерності у
художньому фільмі. У квітні і травні планував закінчити зйомки в художньому
повнометражному українському фільмі «В зеніті», де у нього була одна з головних
ролей.

Читайте також: Мінус вісім одиниць: як десантник спалив російську бронетехніку
під Мар'їнкою (ВІДЕО)

Коли почалося повномасштабне вторгнення, Ігор брав участь у воркшопі з
традиційного співу, що проходив у Карпатах. Він не одразу міг потрапити до
рідного Києва і намагався робити все можливе, щоб бути корисним. Оскільки актор
не мав жодного військового досвіду, вирішив займатися волонтерською діяльністю.
Стала у нагоді багаторічна робота в Польщі, адже більше 10 років він
реалізовував там мистецькі проєкти. Так, протягом першого місяця війни Аронов
виконував багато волонтерської та координаторської роботи, спрямованої на
допомогу українцям в Польщі. Це була і цілеспрямована допомога біженцям, і
організація притулків та постачання військових речей з Польщі.

«Я викладав польську мову для українців та українську мову для поляків для всіх
охочих. Також я перекладав польські театральні п'єси для субтитрів, які ставили
в польських театрах для українців. Намагався спрямувати свою енергію в
правильному напрямі», — розповів актор.

Читайте також: Жителька Красногорівки чудом вижила після того, як ворожий
снаряд поцілив у її будинок (ВІДЕО)

Про важливість проєкту «Архів війни»

Загалом, за словами Ігоря Аронова, важливо наразі знайти способи, які
допоможуть прийняти нову реальність та намагатися бути корисними і щодня робити
все можливе для нашої спільної перемоги. Після отримання свідчень актор
продовжує спілкуватися з багатьма маріупольцями. Намагається бути їм чимось
корисним.

«Хочу зрозуміти, чим я можу допомогти, чим буду корисним конкретно цій людині»,
— поділився Ігор.

Читайте також: 20-річна дівчина із Закарпаття чекає на первістка від 60-річного переселенця з Маріуполя

Найбільше актора вразила історія відомого маріупольця В'ячеслава Долженка, який
самотужки створив у Маріуполі унікальний музей, а у березні 2022 року дивом
врятував і свою 91-річну маму, і себе. Чоловік бачив, як російські окупанти
знищили і рідне місто, і справу всього його життя.

«Зараз В'ячеслав живе єдиною мрією — повернутись до Маріуполя і зробити кращий
музей, ніж він мав до цього. І багато хто вже пообіцяв свої предмети до
майбутнього музею. Вважаю, що про цю дивовижну людину мають знати», —
підкреслив Ігор.

Загалом цей проєкт дуже вплинув на актора, змінилися життєві цінності. Маючи
цей безпосередній контакт з таким горем і болем, вирішив реалізовувати нові
проєкти. Зокрема, Ігор планує створити соціально-мистецький проєкт, спрямований
на реабілітацію маріупольських студентів. Така ідея з'явилася у Ігоря давно,
але саме після спілкування з маріупольцями, які ділилися трагічними спогадами
виживання під час облоги міста чи виїзду з нього, він вирішив, що такий проєкт
набуває особливої актуальності саме зараз.

«У моєму житті часто з'являються певні обставини, які мене до чогось
підштовхують. Загалом зараз я відчуваю, що я повинен жити своїм життям і робити
все, щоб побудувати таке суспільство, державу, країну, які хотіли б побачити
загиблі герої», — підсумував Аронов.

Читайте також: Двоє хлопчиків з Рубіжного знайшли мати, яка безвісти зникла
(ФОТО)

Для того, щоб долучитися до проєкту «Архів війни» та розповісти свою історію,
можна написати Ігорю Аронову особисто в Facebook аба заповнити Google форму.
Адже сьогодні кожен відзнятий кадр, кожне голосове повідомлення — це частина
нової історії, яку ми творимо разом. А надані свідчення зможуть бути
використані проти агресора в кримінальних судах, журналістських розслідуваннях
та подальших наукових і мистецьких роботах.

Нагадаємо, раніше Донбас24 розповідав, що в Україні створили сайт, який
допоможе ознайомитися з наслідками війни.

Ще більше новин та найактуальніша інформація про Донецьку та Луганську області
в нашому телеграм-каналі Донбас24.

Фото: Степана Рудика, Афіни Хайї, Анастасії Телікової та з відкритих джерел
