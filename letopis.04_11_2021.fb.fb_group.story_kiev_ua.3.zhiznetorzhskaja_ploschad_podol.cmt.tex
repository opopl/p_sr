% vim: keymap=russian-jcukenwin
%%beginhead 
 
%%file 04_11_2021.fb.fb_group.story_kiev_ua.3.zhiznetorzhskaja_ploschad_podol.cmt
%%parent 04_11_2021.fb.fb_group.story_kiev_ua.3.zhiznetorzhskaja_ploschad_podol
 
%%url 
 
%%author_id 
%%date 
 
%%tags 
%%title 
 
%%endhead 
\subsubsection{Коментарі}
\label{sec:04_11_2021.fb.fb_group.story_kiev_ua.3.zhiznetorzhskaja_ploschad_podol.cmt}

\begin{itemize} % {
\iusr{Петр Кузьменко}

Ниночка, БЛАГОДАРЮ! Замечательный пост о родных и близких для меня местах
родного Города. Там я ходил ещё ребёнком, за руку с мамой. Там прошли моё
детство и юность. Я писал об этом в своих постах в нашей замечательной группе о
Житнем базаре и о Подоле. И, ещё, в тему Вашей публикации. Реплика "Цирюльник
из-за канавы" в исполнении великого Яковченко в легендарном фильме "За двумя
зайцами", как раз и означала, что "паликмахерская" Галахвастова находилась с
другой стороны рынка.  @igg{fbicon.laugh.rolling.floor} 

\begin{itemize} % {
\iusr{Нина Опольская}
\textbf{Петр Кузьменко} ой, точно, напомнили, Петр, супер @igg{fbicon.hearts.revolving}  @igg{fbicon.100.percent} 
\end{itemize} % }

\iusr{Генадий Пинский}

Я жил на улице Житнеторжская дом №8. КВ 11. Это фотография моего дома. Кто ещё
проживал в этом доме с 1945 по 1969 годы?

\begin{itemize} % {
\iusr{Нина Опольская}
\textbf{Генадий Пинский} Великолепно Геннадий, ваш Дом, как это трогательно @igg{fbicon.hearts.revolving} 
\end{itemize} % }

\iusr{Елена Мельникова}
Спасибо автору.

\begin{itemize} % {
\iusr{Нина Опольская}
\textbf{Елена Мельникова} Благодарю Вас, Елена, за внимание @igg{fbicon.hearts.revolving} 
\end{itemize} % }

\iusr{Яков Любомирский}

Фото, просто супер. Вспомнил свой дом на В. Валу 10, кинотеатр Колос, Рынок. Спасибо Большое!

\begin{itemize} % {
\iusr{Нина Опольская}
\textbf{Яков Любомирский} Воспоминания ...Приятно, что напомнила Вам! Благодарю за внимание @igg{fbicon.hearts.revolving} 
\end{itemize} % }

\iusr{Елена Сидоренко}

Спасибо большое! Любимый Подол и Житний рынок нужно беречь, но, к сожалению, рынок
уничтожают... а он ведь с времён Киевской Руси.. @igg{fbicon.cry}  Многие дома нуждаются в
реставрации, а некоторые отремонтировали так, что дом потерял своё лицо, свой
облик - надстроили этажи, изменили фасад, оторвали кусок, когда строили метро...

\ifcmt
  ig https://scontent-frx5-1.xx.fbcdn.net/v/t39.30808-6/252624735_2684871315140787_7314376294916448670_n.jpg?_nc_cat=111&ccb=1-5&_nc_sid=dbeb18&_nc_ohc=gSZua4nyX0IAX--sW2E&_nc_ht=scontent-frx5-1.xx&oh=90191fdd3b7470cdaac40083b6ba8102&oe=618AE0C8
  @width 0.4
\fi

\begin{itemize} % {
\iusr{Нина Опольская}
\textbf{Елена Сидоренко} Вам благодарность !

\iusr{Елена Сидоренко}
\textbf{Нина Опольская} 

это Вам, Ниночка, благодарность. Я не родилась на Подоле, но очень люблю это место
на земле. Уезжаю - скучаю.... Очень хочу, чтобы не сносили старые дома, хранили
историю. На всех сохранившихся дореволюционных домах я бы повесила таблички -
кто построил, кто тут жил. Таблички должны быть прочные, но недорогие, бронзовые и
медные долго не проживут.

\ifcmt
  ig https://scontent-frx5-1.xx.fbcdn.net/v/t39.30808-6/253298047_2685024405125478_5876254563891099182_n.jpg?_nc_cat=105&ccb=1-5&_nc_sid=dbeb18&_nc_ohc=daB0VLvXfrsAX_GmwQr&_nc_ht=scontent-frx5-1.xx&oh=c5eb8f9cee78b334c8823efb9d16a58b&oe=61898E47
  @width 0.7
\fi

\iusr{Нина Опольская}
\textbf{Елена Сидоренко} Так согласна с Вами, Елена!

\end{itemize} % }

\iusr{Gary Sorokin}
Очень интересно! Почему я живя в Киеве этого не знал?

\begin{itemize} % {
\iusr{Нина Опольская}
\textbf{Gary Sorokin} Познавайте Киев, этот великий Город @igg{fbicon.hearts.revolving} 
\end{itemize} % }

\iusr{Лилия Момотюк}

Ниночка, спасибо большое! Это место Моей малой Родины! Именно там)! Чудо прямо какое-то). Публикация попала прямо в укромный уголок моей души, где живут воспоминания детства моего Подола

\begin{itemize} % {
\iusr{Нина Опольская}
\textbf{Лилия Момотюк} Очень рада, что напомнила! Детские воспоминания всегда с нами и очень сильны @igg{fbicon.hearts.revolving}  Благодарю за внимание!

\iusr{Лилия Момотюк}
\textbf{Нина Опольская} благодарю и Вас! Вы имеете чудесный слог литературного письма
\end{itemize} % }

\iusr{Антонина Мостовенко}
Интересный рассказ.

\begin{itemize} % {
\iusr{Нина Опольская}
\textbf{Антонина Мостовенко} Спасибо большое за внимание @igg{fbicon.hearts.revolving} 
\end{itemize} % }

\iusr{Аркадий-Лариса Малюга}

Для каждого киевлянина подол это правда малая родина и маленькие магазинчи́ки по
которым любили побегать и склоны где катались на санках.

\begin{itemize} % {
\iusr{Нина Опольская}
\textbf{Аркадий-Лариса Малюга} Подольские истории Города, вероятно, самые сильные по энергетике, в памяти всегда

\iusr{Борис Лушельский}
\textbf{Аркадий-Лариса Малюга}
О да.
Ведь без Подола, Киев не возможен.
Как Святой Владимир, без креста.
У каждого из нас свои воспоминания о Киеве ( в общем) и о Подоле ( в частности ).
\end{itemize} % }

\iusr{Тамара Миколаенко}

И вообще самые лучшие воспоминания своей молодости и детства, очень люблю
Подол, жила на ул. Ратманского

\begin{itemize} % {
\iusr{Борис Лушельский}
\textbf{Тамара Миколаенко}
А моя мама ,до 1963 года, жила на ул. Верхний Вал.

\iusr{Alla Ravikovych}
\textbf{Тамара Миколаенко} 

Сейчас это Веденская, и наши дома давно снесены. Еще в 1986 году коренных подолян
выселили на Троещину, а ещё раньше на Оболонь, так что сегодня на Подоле живут
совсем другие люди. А остатки коренных подолян разбросаны кто по всему Киеву, а
кто совсем далеко.

\begin{itemize} % {
\iusr{Тамара Миколаенко}
\textbf{Alla Ravikovych}

\ifcmt
  ig https://scontent-frx5-2.xx.fbcdn.net/v/t39.1997-6/s168x128/47270791_937342239796388_4222599360510164992_n.png?_nc_cat=1&ccb=1-5&_nc_sid=ac3552&_nc_ohc=9Iscxzr94EsAX8s2Wnd&_nc_ht=scontent-frx5-2.xx&oh=16668db0d0d3b2c15b8688a0b8599b32&oe=6189A6F6
  @width 0.2
\fi

\end{itemize} % }

\end{itemize} % }

\iusr{Борис Лушельский}
Без комментариев. Прекрасное повествование . Благодарю .

\iusr{Людмила Волжанка}
"И так хочется сберечь эти исторические места для будущих поколений!" - жаль, что этого не понимают временщики у власти

\iusr{Роза Миколаївна Кириченко}

Спасибо от коренной Подолянки. Родилась в больнице Зайцева , до войны жила на
Константиновской, где мелькомбинат, с 1943 на Нижнем валу, 37. Скучаю по Подолу,
радуюсь каждой весточке о родном Подоле. Спасибо автору!!!

\begin{itemize} % {
\iusr{Елена Мельникова}
\textbf{Роза Миколаївна Кириченко} Мне кажется весь Подол рождался в больнице Зайцева.

\iusr{Роза Миколаївна Кириченко}
\textbf{Елена Мельникова} Мой брат, моя дочь, племянник и племянница тоже там родились

\iusr{Нина Опольская}
\textbf{Роза Миколаївна Кириченко} Благодарю от всей души за внимание! Подол - самый красивый на свете!
\end{itemize} % }

\iusr{Vitali Andrievski}
А в кинотеатре "Колос" если проезжал трамвай часто пленка рвалась

\end{itemize} % }
