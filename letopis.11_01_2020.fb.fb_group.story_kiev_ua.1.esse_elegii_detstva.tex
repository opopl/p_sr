% vim: keymap=russian-jcukenwin
%%beginhead 
 
%%file 11_01_2020.fb.fb_group.story_kiev_ua.1.esse_elegii_detstva
%%parent 11_01_2020
 
%%url https://www.facebook.com/groups/story.kiev.ua/posts/1245492932314172/
 
%%author_id fb_group.story_kiev_ua,petrova_irina.kiev
%%date 
 
%%tags detstvo,gorod,kiev,pamjat
%%title Из эссе "ЭЛЕГИИ ДЕТСТВА"
 
%%endhead 
 
\subsection{Из эссе \enquote{ЭЛЕГИИ ДЕТСТВА}}
\label{sec:11_01_2020.fb.fb_group.story_kiev_ua.1.esse_elegii_detstva}
 
\Purl{https://www.facebook.com/groups/story.kiev.ua/posts/1245492932314172/}
\ifcmt
 author_begin
   author_id fb_group.story_kiev_ua,petrova_irina.kiev
 author_end
\fi

Из эссе "ЭЛЕГИИ ДЕТСТВА"

Ах, сколько утекло с той поры дней, великое множество событий произошло, более
ярких и впечатляющих, но, хранит память детства милые истории, каковые и
представляю на суд моего снисходительного Читателя.

Над древним Городом отшумели годы военного лихолетья. Послевоенные голод,
холод, разруха постепенно отступали, восставал из руин и горя, хорошел Город.
Оттаивали и людские души, утихала боль потерь, высыхали слезы.

\ifcmt
  tab_begin cols=3

     pic https://scontent-frt3-1.xx.fbcdn.net/v/t1.6435-9/81922519_2918257654874489_4658369997947535360_n.jpg?_nc_cat=108&ccb=1-5&_nc_sid=b9115d&_nc_ohc=8aBglMlEbj8AX_gzhsd&_nc_ht=scontent-frt3-1.xx&oh=aadd4c9fd67c4c6847f40c55984bd033&oe=61B4B397

     pic https://scontent-frx5-2.xx.fbcdn.net/v/t1.6435-9/82974841_2918257914874463_8240172229443190784_n.jpg?_nc_cat=109&ccb=1-5&_nc_sid=b9115d&_nc_ohc=j9xMtwa5osIAX_EFl2D&_nc_ht=scontent-frx5-2.xx&oh=6baf34c835642c23efe05c4cf5c4d126&oe=61B49D53

		 pic https://scontent-frt3-2.xx.fbcdn.net/v/t1.6435-9/81897332_2918258151541106_7545835567125102592_n.jpg?_nc_cat=103&ccb=1-5&_nc_sid=b9115d&_nc_ohc=BNENLcOJTAsAX9YLY12&_nc_ht=scontent-frt3-2.xx&oh=b14fbb68b2673a7590b1f8a68b1c8592&oe=61B44108

  tab_end
\fi

И вот на дворе 60-е. Оттепель... Звенящее слово весны и свободы, свежего воздуха
и брызнувшего солнца. И даже внезапная нехватка хлеба, очереди на углу у
булочной, после еще не забытых голодных лет, не так уж и страшны.

Так прекрасен был восстановленный Город. Из всех городов мира он отличался
каким-то особенным уютом, самая главная его площадь была вся в цветах, клумбах,
газонах. Весной тюльпаны и гиацинты заливали её красно-белой пеной, а летом
строгие канны и розы всех цветов радовали глаз и сердце горожан. Это уже потом
её изуродуют правители-временщики и архитекторы-вандалы, а в те годы бегали по
площади аппетитные краснобокие троллейбусы и стояли будки с бравыми
регулировщиками в белоснежных кителях.

\ifcmt
  tab_begin cols=4

     pic https://scontent-frx5-1.xx.fbcdn.net/v/t1.6435-9/82360157_2918258611541060_2854657069917667328_n.jpg?_nc_cat=111&ccb=1-5&_nc_sid=b9115d&_nc_ohc=8oF9wR-VsKAAX95Zkp7&_nc_ht=scontent-frx5-1.xx&oh=51de8897e6ad50af43a9aa5fcbd9af7e&oe=61B326AB

     pic https://scontent-frt3-2.xx.fbcdn.net/v/t1.6435-9/81724596_2918258811541040_8948691917449723904_n.jpg?_nc_cat=101&ccb=1-5&_nc_sid=b9115d&_nc_ohc=u8fuejSE7JkAX82h_7C&_nc_ht=scontent-frt3-2.xx&oh=68bc28ac61d4e7da8e2512d70b66537e&oe=61B3AE19

		 pic https://scontent-frt3-1.xx.fbcdn.net/v/t1.6435-9/81831187_2918259121541009_4843712388119134208_n.jpg?_nc_cat=107&ccb=1-5&_nc_sid=b9115d&_nc_ohc=igFUDXHE2xEAX9sXN7b&_nc_ht=scontent-frt3-1.xx&oh=42cc13199477e13d383d05935480eff8&oe=61B29BE0

     pic https://scontent-frx5-1.xx.fbcdn.net/v/t1.6435-9/82069069_2918259264874328_6302596977662099456_n.jpg?_nc_cat=100&ccb=1-5&_nc_sid=b9115d&_nc_ohc=wZtyROF-OfEAX9wMMM4&tn=lCYVFeHcTIAFcAzi&_nc_ht=scontent-frx5-1.xx&oh=f8f7adb9b2b7e85d8b0aba179f129f96&oe=61B47BB5

  tab_end
\fi

В самом сердце прекрасного города жила маленькая, тихая улочка с причудливым
для здешних краёв названием – Меринговская (с ударением на «о»). Иногда
произносили с ударение на «е» и в моем детском воображении несся табун меринов
с горящими глазами и капающей с удил пеной. Так было в фильме, который я
увидела по чуду тех лет – телевизору, но об этом позже.

В названии улочки остался на земле след от бывшего владельца большой усадьбы
профессора Ф.Ф. Меринга. Задолго до бурных лет начала всеобщего счастья,
равенства и братства, построил он много добротных и удобных домов с красивыми
лепными фасадами. Квартиры в домах сдавали внаем, назывались дома – доходными.
Множество разного люда нашли здесь кров и пристанище. Царственную величавость
Анны Горенко, вулканический темперамент Эренбурга, вековую печаль древнего
народа в глазах Мандельштама - чего только не повидали на своем веку
белокипящие каштаны и золотые липы тихой улочки. Как-то, решив отдать дань
нерушимой дружбе всех народов-братьев великой Страны, присвоили улочке имя
великого сына Востока Фирдоуси. Но, чуждое для здешнего уха, оно не прижилось,
и вновь стала улочка Меринговской.

Во время войны дома тут пострадали крепко, разруха Крещатика костлявой рукой
коснулась и нашей улочки. Некоторые дома исчезли с лица земли, а несколько
оставшихся коробок стен решили возродить. На восстановление пригнали пленных
немецких солдат. Они работали старательно, с кровноврожденной тщательностью,
украсть мешок цемента или десяток-другой кирпичей не могли из-за генного
неприятия воровства и богобоязненности. Да, если бы, и поступились принципами,
зачем в лагере для военнопленных цемент и кирпич? Пребывание там временное,
недвижимостью обзаводиться неразумно. Вот и приобрели крепость, аккуратность и
красоту возрожденные дома с ампирной лепкой, цветочными орнаментами, головами
дев и драконов. Вновь же построенные были уже по-современному скучны, «без
архитектурных излишеств».

Дом под номером 7 (счастливое число!) был прекрасен расстрелиевской покраской –
бирюзовый с белоснежной отделкой, изящен снаружи, уютен внутри. Предназначался
он для семей генералов и офицеров железнодорожного корпуса. Генералы и
полковник получали квартиры на «престижных» третьих, четвертых и пятых этажах,
а первые, вторые и шестые этажи были коммунальным пристанищем семей офицеров в
чинах майоров и капитанов. веселые, большие семьи дворников, сантехников и
прочего рабочего люда обживали полуподвалы, или, как принято было их называть,
цокольные помещения. О, сколько разных, забавных и печальных историй мог бы
поведать этот дом-красавец со свежеблестевшими по весне, отмытыми и натертыми
газетами до блеска, окнами. Впрочем, сотни таких историй может рассказать
каждое людское жильё, но сейчас о другом...

Теперешние толкователи истории называют те года застоем, тоталитаризмом,
засилием идеологии и разными мудреными терминами. Так оно, возможно, и было.
Но, память рисует весьма милые картинки того времени. Запах мастики, ёлки и
мандаринов, сугробы ростом с маму с протоптанными дорожками, веселые ручьи,
несшиеся с крутой улицы-соседки, ванильный аромат куличей, прянный запах
пасхальной буженины, дирижабль на школьной спортплощадке, милицейские заслоны в
майские и ноябрьские праздники, обойдя которые, прорваться проходными дворами и
подъездами, на Крещатик в дни парадов, было делом чести местных мальчишек.
Память детства не может хранить политический и идеологический дух того времени,
да это и не к чему. Об этом написаны сотни умных томов. Здесь же речь идет о
дорогом сердцу, домашнем укладе старого дома на зеленой киевской улочке.

В большой четырехкомнатной квартире на втором этаже судьба соединила жизнь трех
семей. Супружеская чета дикторов украинского радио с редко звучавшей тогда в
быту прекрасной украинской речью. Семья была большая – муж и жена, две дочери,
мать жены, две незамужние сестры жены, занимали они две комнаты, бывшие
гостиную и столовую. Капитан железнодорожных войск с женой и двумя сыновьями
ютились в небольшой комнатке, служившей когда-то спальней. Самую маленькую
комнатушку, давешний кабинет, занимали экономист нефтебазы Наум Львович с
супругой. Они были немолоды и бездетны.

Много людей – много отношений. Особо мирно в квартире не жили, все время кто-то
с кем-то против кого-то дружил. Мужчины в «кастрюлькины распри» старались не
вникать, на крики и жалобы женушек особо не реагировать, мол, покричат,
поскандалят и уймутся. Общались мужики дружно за кружечкой пивка или рюмочкой
красной бормотухи в угловом «гадюшничке». Обычная жизнь миллионов коммуналок
послевоенного времени.

Наум Львович был из многодетной небогатой еврейской семьи. Но, замашки и
привычки имел барские. Да и должность обязывала – экономист нефтебазы был
уважаемым человеком. Жену звал ласково Иринушкой, но, держал в ортодоксальной
строгости. Их комнатушка окнами и крохотным балкончиком выходила во двор. На
этом балкончике была сооружена голубятня, но, не ради удовольствия погонять
заливистым свистом умных птиц в весеннем небе, держал с десяток сизарей в
клетке Наум Львович. Неееет, в запоздалом мальчишестве его не упрекнешь. Он
очень любил голубей, нежно и трепетно любил, но... в виде рагу. Несчастная
Иринушка, содрогаясь от жалости и отвращения, лишала жизни невинных посланцев
мира, тушила с овощами и приправами нежное мясо, скрывая невидимые миру слезы.
Соседи возмущались, чертыхались, плевались. С антисемитской ненавистью
вспыхивали, за спиной Иринушки, слова, что, дескать, «... и Христа распяли!»
Особым гневом полнилась жена капитана, фамилия которого, по странному капризу
судьбы, была ... Голубятников!

Невзирая на это, Иринушка исправно трудилась над тушкой очередной жертвы
гастрономической привязанности Наумчика. Ибо, если к положенному часу в
подогретой тарелке не было бы представлено нежное рагу с гарниром из кабачков,
все злобные шпыняния соседок показались бы райским пением в сравнении с
гневными воплями мужа. Ровно в 17-30 по московскому времени, о чем извещал
голос соседа-диктора из черной тарелки репродуктора на стене, на круглом столе,
покрытом скатертью тяжелого шелка с китайскими фазанами и драконами, должны
были находиться: жесткокрахмальная льняная салфетка в начищенном бронзовом
кольце, серебряная стопочка и хрустальный графинчик-лафитничек с водочкой, две
закусочные и одна хлебная тарелочки, два мельхиоровых столовых прибора – вилка
и нож для мяса, и то же для рыбы, отдельная вилочка с ножичком для десерта. Где
приобрел знания о такой сервировке сын многодетной еврейской семьи из местечка
под Бердичевом, сказать невозможно. Итак, ровно в 17-30 на пороге возникал,
возвратившийся с работы, Наумчик. Пробок тогда в Киеве не было, ходили много
пешком, время рассчитывать было несложно. В 17-40 с чистовымытыми руками он
восседал за стол, для аппетита (как-будто он страдал его отсутствием!) выпивал
первую рюмочку, а в 17-43 в двери комнаты появлялась Иринушка, неся в заранее
подогретой тарелке вышеописанное рагу. Начиналась Трапеза. Во время оной
Иринушкино место было определено на полшага позади правого локтя Наумчика. И
если все было по чину – мясо протушено до искомой мягкости, специй добавлено до
надлежащей душистости и кабачки чуть похрустывали, то Иринушка представлялась к
награде долькой от десерта, состоявшего из сочного яблока или благоухающей
груши. Но, если в гастрономическую рапсодию вкрадывалась фальшивая нотка
пересола или недоперченности, тарелки летели наземь и гнев Наумчика был слышен
даже в кухне, к вящему удовольствию злорадствующих соседушек.

На соседней улице находился, хорошо известный в Киеве, колбасный магазин. Об
этом заведении можно написать много томов, он держался несокрушимым оплотом
дефицита все годы генсеков, партсъездов, застоя и прочих социалистических
изысков. И, увы, бесславно пал в борьбе неравной с капиталистической
действительностью 90-х.

Теперь в зеркальных витринах стоят тупые манекены в костюмчиках с ценниками,
количество циферок в которых, в аккурат равно количеству цифр в телефонном
номере бывшего колбасного рая. Но все это будет очень потом, а в благодатные
60-е по дороге с нефтебазы в «Колбасный на Карла Маркса» заходил Наумчик и
покупал 200 граммов розовой, как ушко эльфа, «Докторской». Здесь не место
гимнам «тем» колбасам, но, поверьте, это было ВКУСНО.

И, завернутая в пергаментную бумажку, колбаска волновала воображение нашего
экономиста всю недолгую дорогу домой. В строго определенное время, а точнее – в
17-40 бумажка разворачивалась на закусочной тарелочке, мельхиоровое
совершенство фирмы «SOLINGEN» резало нежную плоть на аккуратные брусочки, кои
служили отличной закуской под кошерную водочку. Верная Иринушка, стоя на
обычном месте, взирала на эту картинку с тоской. И, однажды, случилось
неслыханное! Брусочек розового волшебства уже плыл ко рту хозяина, и вдруг
раздался тихий вопль души: «Наумчик, а можно мне кусочек?» Оторопев от
небывалой дерзости, чуть не подавившись колбаской, Наумчик воззрился на
нарушительницу устоев трапезы. Прошли долгие секунды и обалдевший муж гневно
взрыкнул: «Ну, ты... на ВСЁ компаньон!!!»

Справедливости ради заметим, что колбаски Иринушке так и не досталось.

Эту вопиющую историю обиженная Иринушка поведала Валюше, соседке из квартиры
напротив, и своей сестрице Нюре, гуляя с ними в скверике около театра Франко.

Нюра была несколько глуховата, говорить с ней надо было громко, вот поэтому
история стала достоянием гуляющей там же общественности. Фраза «Ну, ты на всё
компаньон!» долгие годы была в нашем микрорайоне фольклорным оборотом,
обозначавшем вопиющее нахальство.

\ii{11_01_2020.fb.fb_group.story_kiev_ua.1.esse_elegii_detstva.cmt}
