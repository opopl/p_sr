% vim: keymap=russian-jcukenwin
%%beginhead 
 
%%file 14_09_2021.fb.drobot_tatjana.kiev.1.pohod_v_gosti
%%parent 14_09_2021
 
%%url https://www.facebook.com/tetyana.drobot.1/posts/4562805857102806
 
%%author_id drobot_tatjana.kiev
%%date 
 
%%tags chelovek,gosti
%%title Где-то на просторах сети увидела пост про поход в гости
 
%%endhead 
 
\subsection{Поход в гости}
\label{sec:14_09_2021.fb.drobot_tatjana.kiev.1.pohod_v_gosti}
 
\Purl{https://www.facebook.com/tetyana.drobot.1/posts/4562805857102806}
\ifcmt
 author_begin
   author_id drobot_tatjana.kiev
 author_end
\fi

Где-то на просторах сети увидела пост про поход в гости. 

Когда ты приходишь в гости, первое, это тебя пригласят за стол, и пока ты не
успела опомниться, отправят мыть руки. 

\ii{14_09_2021.fb.drobot_tatjana.kiev.1.pohod_v_gosti.pic.1}

Моя мама и бабушка всегда говорили, что первое, что ты должна сделать, это
накормить человека, ведь ты не знаешь, когда он последний раз ел..

С автором у меня, по ходу, возраст одинаковый, ближе к пенсии, чем к стипендии:

«Я застала ещё времена, когда ты приходишь в гости и первое, что делали - это
тебя кормили..

Неважно, откуда ты пришла, неважно сколько времени на дворе, неважно сколько ты
пробудешь.

Первое - накормить.

Сначала спрашивали, нет, даже не спрашивали, говорили - у нас, скажем, сегодня
пюре с котлетами, будешь? 

Если отказывалась - на тебя озадаченно смотрели и спрашивали, что будешь? Мол,
готовить отдельно придется.

Поэтому считалось невежливым отказываться.

Ели вместе, за одним столом. Одновременно беседовали, знакомились ближе. Заодно
смотрели, как ешь, это многое говорит о человеке.

Потом пили чай. Тоже вместе.

Тут важный момент. К чаю желательно было что-то вкусное принести "к столу".
Можно, конечно, и так, с пустыми руками... но... Это тоже многое говорит о
человеке.

Пили чай и тоже беседовали.

Потом бывало так, что мужчины уходили в комнату, а женщины оставались - вместе
прибирались и мыли посуду.

Считалось вежливым для гостя предложить свою помощь.

И вежливым для хозяйки - отказать, но все-таки принять.

И знакомство продолжалось.

Такой ритуал вежливости.

И лишь затем расходились и можно было заняться, зачем пришёл.

Сейчас ушла эта культура - ходить в гости.

А ведь было ещё немало нюансов, которые уже и позабылись.

То, что и составляет ощущение времени.

Например, было принято, если гость первый раз в доме - показать дом. Устроить
такую экскурсию. 

С подробным показом библиотеки или чем еще богаты - пластинок, марок,
календариков...

Все надо было непременно показать накормленному гостю.

От гостя требовалось вежливо показывать интерес.

В чужом доме было не принято критиковать порядки дома.

И много таких подробностей...

Сейчас вспомнишь - целая культура ушла...»

Поэтому не удивляйтесь, если вы заглядываете ко мне в гости, а я с порога: 

- У меня сегодня борщ и вареники!
