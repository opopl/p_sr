% vim: keymap=russian-jcukenwin
%%beginhead 
 
%%file 30_10_2020.fb.denis_zharkih.1.human
%%parent 30_10_2020
 
%%url https://www.facebook.com/permalink.php?story_fbid=2841541549392577&id=100006102787780
%%author 
%%tags 
%%title 
 
%%endhead 

\subsection{Человека надо любить таким, каким он есть - Денис Жарких}

\Purl{https://www.facebook.com/permalink.php?story_fbid=2841541549392577&id=100006102787780}

Человека надо любить таким, каким он есть. Ангела любить просто, но вот
незадача, ангелов-то не бывает. Часто люди любят не человека, а его
идеализированный образ, делают из него идола, божка, а потом этого идола
низвергают и проклинают. И оказывается, что реально-то не любили этого
человека, а потом ненависть вызвал тоже не реальный человек, а тоже некоторый
образ, который к реальному человеку не подходит.

С народом, его историей происходит тоже самое. Не стоит идеализировать любой
народ, его историю. нет народа ангелов, нет народа праведников. Но и нет народа
дьяволов, нет народа злодеев. Любить свой народ, но знать его позорные главы в
истории, ошибки, грехи могут только люди с большим сердцем и чистой душой. Вот
чем больше таких людей будет, тем будет лучше и сам народ, тем больше в нем
будет любви, уважения, порядочности и благородства. А сегодня мы воспитываем
ублюдков, которые любят не свой народ, а свое больное представление о нем. Чем
это закончится? Да тем же, что и было - неверием в народную мудрость,
самобичивением и брезгливостью по отношению к своим. Сколько раз это в нашей
истории было, сколько можно ходить по этому кругу от национальной спеси до
национального самоуничижения и обратно? Реального человека любить трудно.
Реальный народ еще труднее. Но пока мы любим, мы живем, и пока нас любят мы
живем в народе. Закончится настоящая любовь - закончимся и мы.  2017
