% vim: keymap=russian-jcukenwin
%%beginhead 
 
%%file slova.rod
%%parent slova
 
%%url 
 
%%author_id 
%%date 
 
%%tags 
%%title 
 
%%endhead 
\chapter{Род}
\label{sec:slova.rod}

%%%cit
%%%cit_head
%%%cit_pic
\ifcmt
  tab_begin cols=2
     pic https://avatars.mds.yandex.net/get-zen_doc/62917/pub_612a3ba09f19a00797a78d20_612a3c7643414f1b773971cc/scale_1200
     pic https://avatars.mds.yandex.net/get-zen_doc/1542444/pub_612a3ba09f19a00797a78d20_612a3d700e2b051ebe1b2627/scale_1200
  tab_end
\fi
%%%cit_text
Царство Русское еще не образовалось, не построили еще люди, жившие в то далекое
время ни городов, ни прекрасных храмов и самого названия у той огромной
территории тогда не было.  А был \emph{род}, который составляли семьи,
произошедшие от одного человека - родоначальника, близкие роды объединялись в
племена, а объединение родственных племен образовывало народ. У каждого народа
– своя культура, обычаи, свой язык и свои отличия во внешности.  Но как бы ни
хотел каждый \emph{род} жить самостоятельно и обособленно, нужно было постоянно
защищаться от врагов, диких зверей, погодных стихий и других напастей, которых
в то время было в изобилии.  И стали понемногу появляться поселения с крепкими
домами и высокими стенами вокруг, на которых круглые сутки дежурили воины.
Постепенно народы заселяли самые лучшие районы, как правило, вдоль рек, озер
%%%cit_comment
%%%cit_title
\citTitle{Кто жил на землях будущей Москвы? Славяне появились намного позже. Кто же вы, предки коренных москвичей?}, 
География \& История, zen.yandex.ru, 19.09.2021
%%%endcit

%%%cit
%%%cit_head
%%%cit_pic
%%%cit_text
Поляне же жили в те времена отдельно и управлялись своими \emph{родами}; ибо и
до той братии (о которой речь в дальнейшем) были уже поляне, и жили они все
своими \emph{родами} на своих местах, и каждый управлялся самостоятельно. И
были три брата: один по имени Кий, другой - Щек и третий - Хорив, а сестра их
Лыбедь. Сидел Кий на горе, где ныне подъем Боричев, а Щек сидел на горе,
которая ныне зовется Щековица, а Хорив на третьей горе, которая прозвалась по
имени его Хоривицей. И построили город в честь старшего своего брата, и назвали
его Киев. Был вокруг города лес и бор велик, и ловили там зверей, а были те
мужи мудры и смыслены, и назывались они полянами, от них поляне и доныне в
Киеве
%%%cit_comment
%%%cit_title
\citTitle{Повесть Временных Лет}, Нестор Летописец
%%%endcit
