% vim: keymap=russian-jcukenwin
%%beginhead 
 
%%file 14_02_2022.fb.ukr.nauchnaja_shkola_bazis.1.olimpiada_fizika
%%parent 14_02_2022
 
%%url https://www.facebook.com/sciencebasis/posts/1272812123213888
 
%%author_id ukr.nauchnaja_shkola_bazis
%%date 
 
%%tags deti,fizika,nauka,obrazovanie,shkola,ukraina
%%title Вітаємо наших призерів олімпіад із фізики!
 
%%endhead 
 
\subsection{Вітаємо наших призерів олімпіад із фізики!}
\label{sec:14_02_2022.fb.ukr.nauchnaja_shkola_bazis.1.olimpiada_fizika}
 
\Purl{https://www.facebook.com/sciencebasis/posts/1272812123213888}
\ifcmt
 author_begin
   author_id ukr.nauchnaja_shkola_bazis
 author_end
\fi

Вітаємо наших призерів олімпіад із фізики! 

Учні 8-11 класів взяли участь у Київській міській олімпіаді з фізики, де
показали одні з найкращих результатів:

- Тимофій Колісник, Тимофій Стаднійчук та Уляна Ронська - І місце

- Микита Голубецький - ІІ місце

- Андрій Ведмідь, Дар’я Рєзнікова, Ілля Савенко, Назар Черпак та Ян Железняк -
ІІІ місце

\ii{14_02_2022.fb.ukr.nauchnaja_shkola_bazis.1.olimpiada_fizika.pic.1}

Хочемо відзначити, що усі учні «Базису» відмінно зробили експериментальну
частину олімпіади. А Уляна Ронська та Ян Железняк набрали за експеримент
максимально можливу кількість балів, що буває надзвичайно рідко. Пишаємося
такими високими результатами!

В той же час учні 7-х класів здобули призові місця на районній олімпіаді з
фізики:

- Владислав Деревко, Марк Бірюков - ІІ місце

- Андрій Мусієнко, Олександр Рогач, Василь Братаніч, Ілля Лунін, Софія
Карташова, Артем Храпач - ІІІ місце. 

Дякуємо усім учасникам і бажаємо їм вмотивованості та нових перемог! 

\#Basis \#фізика
