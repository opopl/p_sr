% vim: keymap=russian-jcukenwin
%%beginhead 
 
%%file slova.knjaz
%%parent slova
 
%%url 
 
%%author 
%%author_id 
%%author_url 
 
%%tags 
%%title 
 
%%endhead 
\chapter{Князь}
\label{sec:slova.knjaz}

%%%cit
%%%cit_head
%%%cit_pic
%%%cit_text
За кілька років після хрещення \emph{князь} Володимир Святий вирішив розмежувати на
Русі небесне і земне. У 995/996 році він видав Десятинній церкві грамоту на
отримання десятини, яка до 1011 року перетворилася на перший церковний статут.
Незважаючи на свою назву, документ стосувався устрою всієї державної влади,
оскільки розмежував юрисдикцію світських і церковних судів, а також віддав
священикам контроль мір і ваг. Всі інші питання залишилися у віданні \emph{князя}.
Пізніше свій церковний статут видав \emph{князь Ярослав}, а потім канонічне право
остаточно відокремилося від державного
%%%cit_comment
%%%cit_title
\citTitle{Українській Конституції – 2300 років}, 
Сергій Громенко, gazeta.ua, 26.06.2021
%%%endcit

%%%cit
%%%cit_head
%%%cit_pic

\ifcmt
  tab_begin cols=2
     width 0.4
     pic https://img.strana.ua/img/article/3410/zelenskij-vnes-v-33_main.jpeg

     pic https://strana.ua/img/forall/u/11/52/%D0%B3%D0%B5%D1%80%D0%B1.jpg
  tab_end
\fi

%%%cit_text
Главным элементом Большого герба, согласно проекту, является Знак \emph{Княжеского}
Государства Владимира Великого (малый Государственный Герб Украины) золотого
цвета, размещенный на синем пятистороннем щите с закругленными нижними боковыми
углами с золотой каймой.  Над щитом - изображение \emph{великокняжеского} венца
(короны) Ярослава Мудрого и пурпурно-золотой шатер в виде растительного
орнамента. Щит держат: с левой стороны - лев (герб Галицко-Волынского
\emph{княжества}), с правой - воин-казак с ружьем (герб Войска Запорожского) под щитом
- лента из двух горизонтальных полос синего и желтого цветов, под лентой - два
золотых колоска пшеницы, переплетенные гроздью калины пурпурного цвета со
стилизованным листьями пурпурно-золотого цвета
%%%cit_comment
%%%cit_title
\citTitle{Зеленский внес в Раду законопроект о большом гербе. Эскиз}, 
, strana.ua, 29.06.2021
%%%endcit

%%%cit
%%%cit_head
%%%cit_pic
%%%cit_text
Устав от такой жизни, \emph{князь} Дмитрий Михайлович с двумя своими старшими
сыновьями Борисом и Давыдом сначала поступает на службу к нижегородскому \emph{князю}
Дмитрию Константиновичу на должность тысяцкого. А после он появился в Москве в
1366—1369 годах и поступил на службу к \emph{князю} Дмитрию Ивановичу (будущему
Донскому). Воинский талант Дмитрия Михайловича был видимо оценен по
достоинству, он играл ключевую роль в реформировании московским \emph{князем} своего
войска и вскоре \emph{князь} Дмитрий даже выдал свою сестру Анну за своего нового
воеводу.  В 1371 году произошёл военный конфликт между Дмитрием Донским и
рязанским \emph{князем} Олегом Ивановичем. В битве при Скорнищево московское войско,
которым командовал \emph{князь} Дмитрий Михайлович, нанесло сокрушительное поражение
\emph{князю} Олегу, который был вынужден временно отказаться от титула \emph{князя}
Рязанского - на его место был посажен пронский князь Владимир Дмитриевич,
союзник московского \emph{князя}
%%%cit_comment
%%%cit_title
\citTitle{Военный гений Руси}, Илья Duke, zen.yandex.ru, 30.10.2021
%%%endcit

%%%cit
%%%cit_head
%%%cit_pic
\ifcmt
  tab_begin cols=3
     pic https://avatars.mds.yandex.net/get-zen_doc/4447382/pub_613f571f199c4d2e89cea60a_613f57c914adf418808a3013/scale_1200
     pic https://avatars.mds.yandex.net/get-zen_doc/168279/pub_613f571f199c4d2e89cea60a_613f57ddc5a13154046886fc/scale_1200
		 pic https://avatars.mds.yandex.net/get-zen_doc/4776500/pub_613f571f199c4d2e89cea60a_613f57eb199c4d2e89d01168/scale_1200
  tab_end
\fi
%%%cit_text
Летописные источники пестрят сообщениями: киевские \emph{князья} неоднократно «воевали
страну Корсунскую». «Русы» имели огромный интерес к этому региону, закладывали
собственные колонии и базы. Постоянно нагнетая своё присутствие в торговых и
военных делах Крыма, Причерноморья и Приазовья. Эти опорные пункты появились
еще до Оскольда, Рюрика, Олега, примерно в самом начале IX века. Данные точные,
археологические. Но говорить о «государстве русов» не приходится. Хазары и
Византия заметили бы.  А если вчитаться в описания патриарха Фотия: речь идёт о
«неизвестном народе».  Собственных соседей ромеи знали в лицо, это точно. То
есть... опять Киев, так получается? Довольно далече от Проливов. Хорошо, пусть
так. «Киевские русы» могли в IX веке иметь свои колонии, базы и опорные пункты
в Крыму и Приазовье.  Чем-то радикально отличаясь от дунайской, каспийской,
черноморской «руси»? Чем, скажите на милость...
%%%cit_comment
%%%cit_title
\citTitle{Начало «Русского Мира»: есть общий праздник для России, Украины и Белоруссии}, 
Исторические напёрстки, zen.yandex.ru, 14.09.2021
%%%endcit
