% vim: keymap=russian-jcukenwin
%%beginhead 
 
%%file 25_08_2021.fb.ovsjannikov_vitalij.1.otvetstvennost
%%parent 25_08_2021
 
%%url https://www.facebook.com/vibhutisaktidas.vitalii/posts/10216844847828052
 
%%author Овсянников, Виталий
%%author_id ovsjannikov_vitalij
%%author_url 
 
%%tags chelovek,otvetstvennost',psihologia,svoboda
%%title СКОЛЬКО ОТВЕТСТВЕННОСТИ МЫ ГОТОВЫ СЪЕСТЬ
 
%%endhead 
 
\subsection{СКОЛЬКО ОТВЕТСТВЕННОСТИ МЫ ГОТОВЫ СЪЕСТЬ}
\label{sec:25_08_2021.fb.ovsjannikov_vitalij.1.otvetstvennost}
 
\Purl{https://www.facebook.com/vibhutisaktidas.vitalii/posts/10216844847828052}
\ifcmt
 author_begin
   author_id ovsjannikov_vitalij
 author_end
\fi

СКОЛЬКО ОТВЕТСТВЕННОСТИ МЫ ГОТОВЫ СЪЕСТЬ.

Свобода! Сладкое и любимое слово… Просто слово. Потому, что свободы как таковой
никто и не нюхал. Ну, разве что, мгновения, когда забрался на вершину холма… и
пока не вспомнил про жену и детей, работу и кредит – насладился свободой. Потом
взял своё коромысло и побрел по проторенной тропе. 

\ifcmt
  pic https://scontent-cdg2-1.xx.fbcdn.net/v/t39.30808-6/240124479_10216844843867953_1217822809481441146_n.jpg?_nc_cat=107&_nc_rgb565=1&ccb=1-5&_nc_sid=730e14&_nc_ohc=nWE4btGhyX0AX9P4tuO&_nc_ht=scontent-cdg2-1.xx&oh=4a51797500991d09d1d47bdfbd09d2c1&oe=612D39D6
  width 0.4
\fi

Желание свободы и хомут ответственности находятся в постоянном конфликте. И это
очень мешает жить. Есть религии, которые тысячелетиями выпиливают из сознания
своей паствы такое понятие как совесть. Потому, что совесть, по их понятиям,
очень мешает стать сверхчеловеком, стать избранными. 

Но мало кто задумывался, что свобода по сути равна ответственности. То, за что
взял ответственность, в этом и свободен. У нас перевертыш: свобода – это
безответственность. Поэтому, так «хорошо» перекинуть ответственность на кого-то
другого… За свое здоровье – на медицину, за свой финансовый доход – на
работодателя, за свои желания – «что мама скажет…» В итоге, так приятно
спихнуть на Бога и звезды всё остальное – ну, вот такой я и… и… и…! 

Часто в психологической работе человек видит себя связанным по рукам и ногам.
Это эмоционально-энергетическое состояние удерживает от движения по жизни.
Вроде, зона комфорта, а в душе как в невычищенном курятнике. И смердит чувством
вины. 

И виноватый ищет кто виноват! Все отношения превращаются в переживания чувства
вины и обвинения других.  «Сами криво посадили, а Тюлюлюй виноват…» 

Нет жизни с чувством вины. Это ад. Но, как правило, у богатого воображения
всегда есть варианты. У большинства людей вина надуманная, т.е. нереальная.
Потому, что реальная вина может быть исправлена. Кто из нас может исправить
первородный грех Адама? Тогда зачем попы своей пастве мозг компостируют? 

Где ответственность из чувства вины, там нет свободы. Это манипуляция. На
вопрос: «В чем моя ответственность?» - ответ: «А во всём! Начиная от делов
Адама и заканчивая Путиным». Когда границы ответственности не определены,
человек в ступоре. Ответственность для него, как кость в горле: не глотнуть, не
выплюнуть. И его воображение придумывает и рисует самые безумные картинки.

Поэтому, «прожевать» и «проглотить» мы можем только ту ответственность, которая
конкретна. Нормальная жизнь человека начинается с понимания: в чем состоит его
ответственность, и что таковой не является. Плюс к этому, ответственность
должна иметь смысл. Извечный конфликт поколений в изменении вектора смыслов.
Поэтому, дети естественно уходят от своих родителей, потому что не могут нести
бремя ответственности родительских смыслов. Они хотят чувствовать себя не
«обязанными», а - «вовлеченными». 

И, самое важное, мы подсознательно (или интуитивно) несем ответственность перед
высшей ценностью: идеей, идеалом, Богом, самим собой… Мы свободны в выборе
своей высшей ценности, а значит свободны в ответственности перед ней. Поэтому,
ответственность – это признак персональной зрелости, через которую человек
доказывает свое «совершеннолетие». Кому доказывает? Да, самому себе…

К одному мужчине пришел друг. Сели на кухне. Сделали бутерброды, чай. Потом,
хозяин помыл посуду, аккуратно сложил всё на место. Гость сказал: «Я тоже
помогаю своей жене». «А я нет! - ответил хозяин. – Поддерживать порядок в
квартире, где я живу, - это так же и моя ответственность. Это не помощь, и не
обязанность. Это часть моей жизни…»

\begin{verbatim}
	#makerпсихология
	#дневникмыслей
\end{verbatim}
