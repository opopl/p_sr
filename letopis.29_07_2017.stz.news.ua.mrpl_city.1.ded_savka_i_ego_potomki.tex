% vim: keymap=russian-jcukenwin
%%beginhead 
 
%%file 29_07_2017.stz.news.ua.mrpl_city.1.ded_savka_i_ego_potomki
%%parent 29_07_2017
 
%%url https://mrpl.city/blogs/view/ded-savka-i-ego-potomki
 
%%author_id burov_sergij.mariupol,news.ua.mrpl_city
%%date 
 
%%tags 
%%title Дед Савка и его потомки
 
%%endhead 
 
\subsection{Дед Савка и его потомки}
\label{sec:29_07_2017.stz.news.ua.mrpl_city.1.ded_savka_i_ego_potomki}
 
\Purl{https://mrpl.city/blogs/view/ded-savka-i-ego-potomki}
\ifcmt
 author_begin
   author_id burov_sergij.mariupol,news.ua.mrpl_city
 author_end
\fi

Образ прадеда Саввы Васильевича складывался в голове постепенно, подобно тому,
как кусочки картона с краями причудливой формы собираются детскими руками в
картинку. Самое интересное, что его упоминание, впервые услышанное, было
воспринято ребячьим умом как одно слово – \enquote{Дедсавка}, как имя еще не
встречавшегося человека из многочисленной родни. Лишь много позже, когда одна
из теток с улыбкой рассказывала о том, как она, тогда пятилетняя девочка, сидя
на коленях у дедушки Савки, расчесывала деревянным гребнем его бороду, а потом
заплетала ее в косичку с ленточкой. Вот тогда-то \enquote{Дедсавка} разделилось на два
слова, как бы сказал просвещенный филолог, на степень родства и собственное
имя.

Из рассказов взрослых о пращуре было узнано, что родился и жил он на хуторе
Руда, что прилепился к местечку Варва, откуда родом и дедушка Петя, и бабушка
Таня. Уже будучи на пенсии, роясь в одном из томов Энциклопедического словаря
Брокгауза и Ефрона, в глаза бросилась статья \enquote{Варва}. Из статьи, набранной
мелким, но вместе с тем удивительно четким шрифтом, довелось узнать, что
названный выше населенный пункт принадлежал Лохвицкому уезду Полтавской
губернии. Что Варва – \enquote{одно из древнейших поселений в крае и упоминается в
летописях как место значительной победы Владимира Мономаха  в 1079 году над
половцами}.

Судя по тому, что дед Савка приходил на заработки в Мариуполь, приходил в
буквальном смысле этого слова – крестьянин не мог позволить себе роскошь
тратить деньги на  переезд по железной дороге от ближайшей к Варве станции
Пирятин до города на Азовском море – был он человеком выносливым. А то, что в
Мариуполе работал дубщиком, свидетельствовало о его жилистости и силе. Слово
\enquote{дубщик} давным-давно вышло из повседневного употребления и не попало в поле
зрения Владимира Даля и других лексикографов, и, вероятно, имело сугубо местное
происхождение. Оно возникло от слов \enquote{дуб}, \enquote{дубок}. Большой знаток мореплавания
на Азовском море покойный Владимир Андреевич Павлий в своей книге, посвященной
этой теме, приводит такое определение: \enquote{Дуб, дубок - наиболее распространенный
тип парусного судна на Азовском и Черном морях с двумя и даже тремя мачтами с
косыми парусами длиной, до 30 метров}.

Как известно, через порт в устье Кальмиуса шел поток товарного зерна,
выращенного в Северном Приазовье. Однако недостаточная глубина реки не
позволяла подходить морским судам под погрузку прямо к причальной стенке.
Приходилось мешки с зерном грузить на дубки и с помощью этих мелкосидящих
парусников доставлять их к бортам стоящих на рейде пароходов. И вот дубщики,
выражаясь казенным языком, выполняли все грузовые операции: от берега до дубка
и от дубка на пароход. Кроме того, часть дубщиков еще и управляли
\enquote{плавсредством}.

Понятно, что дед Савка не для развлечения преодолевал расстояние в несколько
сот верст от родного ему хутора до города у моря, а затем  весь световой день
таскал на себе мешки в два с половиной пуда, отдыхая лишь во время переходов
дубка к пароходу и обратно. Конечно, все делалось ради дополнительного
заработка, хотя, по мнению бабушкиного свояка - деда Мирона по прозвищу
Кобыляцкий козак: \enquote{Дiд Савка був дуже заможний - вiн i так все мав, та йому
завжди грошей було замало}. Но на самом деле не жадность толкала землероба
ходить с хутора Руда в далекий Мариуполь. Он смотрел вперед. У него было пятеро
сыновей и дочка. С дочкой было все ясно – отдаст ее замуж и дело с концом. А
сыновья подрастут, женятся – семейный надел нужно будет делить на пять частей.
Каждому достанется клочок земли, который вряд ли сможет прокормить семью, когда
она разрастется. Вот Савва Васильевич и решил подзаработать, да и прикупить
сыновьям землицы.

Но человек предполагает, а Господь располагает. Петр еще мальчишкой перебрался
в Мариуполь, примкнул к артели земляков-портных. Начинал с  подмастерья,
научился ремеслу, и стала ему кормилицей швейная машина \enquote{Зингер}. Аксентий,
женившись, переехал к брату в Мариуполь, жил у него с молодой женой и дочуркой.
Устроился работать каталем на \enquote{Русский Провиданс}. Работа, правда, была очень
тяжелой, зато денежной. Подкопил деньжат, отец выделил кое-что, брат помог - и
купил Аксентий землю в селе Никольском, в недавнем  Володарске. С отцом
семейства остались Иван, Мыкыта и младший - Поликарп, решили  - \enquote{гуртом краще
буде}. Так что делить землю не пришлось.

Предусмотрительность деда Савки проявилась и в таком факте. Когда ему едва
перевалило за пятьдесят, он сгородил из добрых дубовых плах себе  труну, или
по-русски гроб, затащил его до времени на горище. С того момента невестки стали
с великим страхом подниматься на чердак по хозяйственным делам, да и то - по
двое. И здесь Савва Васильевич просчитался. Пока он жил во время голода 1921
года у Петра в Мариуполе, кто-то воспользовался тем, что он заготовил себе. К
этому времени сохранилось еще одно семейное предание. Одряхлевшего деда Савку в
хорошую погоду невестка усаживала на стул перед воротами. Наблюдение за
прохожими и проезжающими извозчиками было его развлечением. Однажды Татьяна
Кирилловна  с ужасом услыхала такую тираду свекра: \enquote{Добродію, підійдіть, будь
ласка, до мене. Я вам ось що скажу, якби не ця леворюція, мій Петька був би
городським головою}. \enquote{Батько, хіба таке можна говорить, - взмолилась невестка,
- зараз прийдуть хлопці з ЧК та замордують і Петра, і мене, та й вас не
помилують}.

Умирать дед Савка поехал на родину. Грех сказать, но ему повезло, он не увидел,
как раскулачивали заможных селян – тружеников, как землю и скот его и его
сыновей забрали в колхозы, как умер совсем молодым Аксентий, оставив вдову с
четырьмя малыми детьми на руках...

Ну а что с внуками деда Савки? Были среди них много воинов в Великую
Отечественную войну, защищавших страну от фашистов, был и военный врач, был и
комбайнер – Герой Социалистического Труда, и доцент университета, и директор
техникума, и просто добросовестные пахари и рабочие. Внучки тоже не подкачали -
это известная в России художница-керамистка, преподавательницы географии,
языков немецкого, английского...
