%%beginhead 
 
%%file 19_10_2020.fb.fb_group.mariupol.nekropol.1.otchet_volonter_rabota_17_18_oktjabrja_2020
%%parent 19_10_2020
 
%%url https://www.facebook.com/groups/278185963354519/posts/390699035436544
 
%%author_id fb_group.mariupol.nekropol,arximisto
%%date 19_10_2020
 
%%tags 
%%title Отчет о волонтерской работе 17-18 октября 2020
 
%%endhead 

\subsection{Отчет о волонтерской работе 17-18 октября 2020}
\label{sec:19_10_2020.fb.fb_group.mariupol.nekropol.1.otchet_volonter_rabota_17_18_oktjabrja_2020}
 
\Purl{https://www.facebook.com/groups/278185963354519/posts/390699035436544}
\ifcmt
 author_begin
   author_id fb_group.mariupol.nekropol,arximisto
 author_end
\fi

\bigskip
\textbf{Отчет о волонтерской работе 17-18 октября 2020}
\bigskip

Каждый день мы обнаруживали старинную плиту! И каждый день мы садили цветы...

\emph{\textbf{Находки и открытия}}

Титаническая работа \href{https://www.facebook.com/profile.php?id=100004930356528}{Илья Луковенко}%
\footnote{\url{https://www.facebook.com/profile.php?id=100004930356528}}
по очистке и исследованию склепа Спиридона Гофа близка к завершению. Уже
понятно, что вход был с левой стороны. Он перекрывался массивными плитами. При
подзахоронении плиту поднимали, вносили в склеп гроб и \enquote{запечатывали}
плитами...

За склепом Спиридона Гофа под мусором Елена Сугак обнаружила и расчистила
старинную мраморную (?) плиту. К сожалению, без надписи и даты. Кажется, она
относится к концу 19-началу 20 века...

В воскресенье Андрей Марусов и Елена Сугак обнаружили и расчистили еще одну
древнюю плиту - в шаге от памятника Гавриила Гофа, плиты Сарры Гоф и двух
массивных древних плит! На ней ни надписи, ни даты. Рядом же нашли обломок
верхней части плиты более позднего времени...

На сегодня именно на участке вокруг склепов Хараджаева-Гофа сосредоточено
наибольшее количество древних плит - семь. От самой древней в Некрополе плиты
Гавриила Сахаджи 1834-го до нескольких плит без надписей и дат. И, похоже, это
еще не конец...🙂

\emph{\textbf{Благоустройство}}

В субботу мы посадили сентябринки и ромашки - спасибо \href{https://www.facebook.com/profile.php?id=100014346646926}{Ірина Носова}%
\footnote{\url{https://www.facebook.com/profile.php?id=100014346646926}}
за подарок! В воскресенье - белые нарциссы...

Мы наконец-то начали расчистку мусорных куч за склепом Спиридона Гофа....

Огромная благодарность всем волонтерам за самоотверженный труд!

\emph{\textbf{О ресурсах}}

Мы исчерпали все ресурсы на расходные материалы... А хотим сделать еще
простейшие дорожки на участке Хараджаева-Гофа (агроволокно плюс гравий),
высадить мускари \textbackslash\ крокусы вокруг склепов, а также пару-тройку \enquote{благородных}
деревьев вместо \enquote{вонючек} и акаций, очистить от зарослей пространство вокруг
участка...

Так что на неделе обратимся ко всем неравнодушным за помощью...

\emph{\textbf{О планах на 24-25 октября}}

Предварительное время сбора волонтеров - в 13:00 возле белого Памятного креста
в субботу и воскресенье. Более подробный анонс опубликуем чуть позже.

До встречи, друзья!

Контактный телефон 096 463 69 88.

\#mariupol\_necropolis\_report
