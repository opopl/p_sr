% vim: keymap=russian-jcukenwin
%%beginhead 
 
%%file 15_09_2018.stz.news.ua.mrpl_city.1.k_240_letiu_mariupolja_ulica_fontannaja
%%parent 15_09_2018
 
%%url https://mrpl.city/blogs/view/k-240-letiyu-mariupolya-ulitsa-fontannaya
 
%%author_id burov_sergij.mariupol,news.ua.mrpl_city
%%date 
 
%%tags 
%%title К 240-летию Мариуполя: Улица Фонтанная
 
%%endhead 
 
\subsection{К 240-летию Мариуполя: Улица Фонтанная}
\label{sec:15_09_2018.stz.news.ua.mrpl_city.1.k_240_letiu_mariupolja_ulica_fontannaja}
 
\Purl{https://mrpl.city/blogs/view/k-240-letiyu-mariupolya-ulitsa-fontannaya}
\ifcmt
 author_begin
   author_id burov_sergij.mariupol,news.ua.mrpl_city
 author_end
\fi

Эта старинная мариупольская улица получила собственное имя 28 сентября 1876
года вместе с другими улицами и площадями, имевшимися к тому времени в городе.
Нужно думать, что имя это существовало в народе и раньше. До строительства в
первое десятилетие ХХ века электростанции и городского водопровода Фонтанная
прямиком вела к Большому фонтану - главному источнику питьевой воды Мариуполя.
Но потом вокруг электро- и насосной станций водопровода соорудили кирпичный
забор, который перегородил торную дорогу к живительной влаге, а вот название
улицы осталось.

Фонтанная улица довольно продолжительное время была окраинной. На плане 1811
года на ней обозначено несколько строений, расположенных не далее
Харлампиевской улицы. К 1889 году у нее появилась соседка – Евпаторийская
улица, а сама она продвинулась до Константиновской (современная  улица
Нильсена). В 1903 году ее дома достигли Бахмутской (ныне – проспект
Металлургов). Протяженность современной улицы Фонтанной – 1 километр 320
метров.

\ii{15_09_2018.stz.news.ua.mrpl_city.1.k_240_letiu_mariupolja_ulica_fontannaja.pic.1}

К концу 60-х годов прошлого столетия улицу Франко в Центральном районе и
Веселую в Кальмиусском кварталы многоэтажных домов соединили в единую городскую
магистраль. Так на карте города появился проспект Металлургов. И с той же карты
исчезли названия \enquote{Франко} и \enquote{Веселая}. Ну, с Веселой, Бог с ней, а вот лишиться
городского объекта имени классика украинской литературы Ивана Яковлевича Франко
было как-то неудобно. И тогда пришла счастливая мысль – присвоить имя Каменяра
единственной из мариупольских улиц, сохранившей с дореволюционных времен свое
первоначальное название – Фонтанной. Когда в 90-е годы вернули городским
объектам их исконные названия, снова появились новенькие таблички: \enquote{Ул. Фонтанная}.

\textbf{Читайте также:} \href{https://mrpl.city/blogs/view/do-240-richchya-mariupolya-legendi-mista-chastina-i}{%
До 240-річчя Маріуполя: Легенди міста. Частина I, Ольга Демідко, mrpl.city, 12.09.2018}

Патриархальный вид улицы южного провинциального города здесь сохранялся до 90-х
годов ХХ века. Сейчас многое изменилось. Фасады многих домов давно не видели ни
мастерка штукатура, ни кисти маляра. Среди уютных мещанских домишек возвышаются
сооружения - нечто среднее между трансформаторной будкой и складом, некоторые
дома надстроены нелепыми мансардами, на углу рядом с Греческой стоит
покосившийся домик как символ бренности всего сущего.

\ii{15_09_2018.stz.news.ua.mrpl_city.1.k_240_letiu_mariupolja_ulica_fontannaja.pic.2}

Перейдем от общих рассуждений к некоторым конкретным объектам. Улица Фонтанная
начинается от дома № 1, который стоит против электроподстанции. Номера 3 и 5,
может, чем-то и примечательны, но чем, автору этих строк неизвестно. А вот о
номере 7 (\emph{на фото}) стоит поговорить. В этом доме жил в начале ХХ века священник
Василий Баев. Он окончил Мариупольское духовное училище, затем
Екатеринославскую семинарию. Василий Кириллович Баев, будучи рукоположен в
священники, служил в церквях сел Мангуш, Старый Крым, затем в Мариупольском
Харлампиевском соборе. Он был законоучителем в частной гимназии Неонилы
Саввичны Дарий, Мариупольском техническом училище. Отец Василий ушел из жизни в
1924 году в возрасте пятидесяти восьми лет.

\textbf{Читайте также:} 

\href{https://mrpl.city/news/view/v-vodonapornoj-bashne-mariupolya-otkroyut-biblioteku-foto}{%
В водонапорной башне Мариуполя откроют библиотеку, Ярослав Герасименко, mrpl.city, 14.09.2018}

Дом отца Василия и стоящее перед ним строение послужили натурой для художников.
Вероятно, первой обратила внимание на этот уголок старого Мариуполя художница
Людмила Московченко. Много позже пришел сюда с мольбертом Суммани Мамедов.

\ii{15_09_2018.stz.news.ua.mrpl_city.1.k_240_letiu_mariupolja_ulica_fontannaja.pic.3}

Перейдем Торговую улицу. Дом № 11. Он сразу бросается в глаза. Несмотря на то
что ему более ста лет, выглядит он молодцевато. Свежая светло-голубая с белым
окраска, расчищена затейливая кладка наружных стен. Говорят, будто принадлежал
этот особнячок купцу Алакозову. Так ли это? Документальных подтверждений нет.
Напротив - дом № 10. Мрачноватое строение с табличкой, свидетельствующей, что в
нем находится штаб-квартира комитета самоорганизации населения
\enquote{Старогородской}.

\ii{15_09_2018.stz.news.ua.mrpl_city.1.k_240_letiu_mariupolja_ulica_fontannaja.pic.4}

Для того чтобы узнать дату постройки двухэтажного дома № 44, не нужно рыться в
архивах, поскольку она указана на фронтоне - 1911 г. Вот что довелось узнать от
приветливых жильцов этого старинного здания – супружеской пары. Дом был
построен неким купцом, во дворе были конюшня и каретный сарай, а сейчас в них
гаражи. Под домом большой подвал. Сразу же после освобождения Мариуполя от
немецко-фашистских захватчиков в этом особняке расположился Молотовский
райисполком. К сведению, до \enquote{разоблачения} Хрущевым так называемой
антипартийной группы Жовтневый район нашего города назывался Молотовским, а
теперь стал Центральным. В 1953 году на бывшей улице Карла Либкнехта, 39 было
построено трехэтажное здание, туда перебрались райкомы партии и комсомола, а
также исполком Молотовского района. А в опустевший дом вселили девять семей, в
наши дни там осталось только пять. Соседний дом № 46 - здесь жил известный
мариупольский кардиолог Алексей Никитич Пархоменко.

\ii{15_09_2018.stz.news.ua.mrpl_city.1.k_240_letiu_mariupolja_ulica_fontannaja.pic.5}

Через дорогу возвышается дом № 53 (\emph{на фото}). Так и не удалось узнать, кто его
построил, кто обитал в нем до захвата города фашистами. Но достоверно известно,
что во время оккупации здесь квартировал с семьей Николай Комровский,
назначенный немцами бургомистром Мариуполя. Комровский был этническим немцем, с
началом войны чудом не выселенный в восточные районы страны. Жить ему тут
судилось не очень долго. Пришлось срочно бежать из города вместе с
гитлеровцами. После освобождения Мариуполя и до 1953 года особняк занимал
Молотовский райком партии. Рядышком, во дворе, во флигеле с шаткой деревянной
лестницей и комнатушками с низкими потолками располагался райком комсомола.
После 1953 года в доме № 53 был районный отдел социального обеспечения,
управление № 219 треста \enquote{Донбассмеханомонтаж}, а сейчас он принадлежит
неизвестному собственнику.

\textbf{Читайте также:} 

\href{https://mrpl.city/news/view/v-mariupole-pokazali-chto-proishodit-za-vysokim-zaborom-v-tsentre-goroda-foto}{%
В Мариуполе показали, что происходит за высоким забором в центре города, Ганна Хіжнікова, mrpl.city, 11.09.2018}

Угловой дом № 68. Он интересен тем, что, во-первых, в 50-е годы ХХ века стал
первым на улице трехэтажным строением с четырехэтажной угловой вставкой, а
во-вторых, здесь долгие годы жил известный живописец и монументалист, член
Национального союза художников Украины, краевед, общественный деятель, почетный
гражданин Мариуполя Лель Николаевич Кузьминков.

Дом № 85 – единственная на Фонтанной девятиэтажка – дань крупнопанельному
жилищному строительству в нашем городе. Она соседствует со строением № 87.
Здесь располагается отдел полиции. А до этого в этом здании была школа № 5, а
еще раньше, до революции – высшее начальное училище для мальчиков, открытое,
как подсказала заместитель директора краеведческого музея Р. П. Божко, по закону
1912 года. Здесь учились четыре года, и документ о его окончании позволял
продолжать учебу в гимназии. Проезжая часть Фонтанной, упираешься в решетчатый
забор стадиона общеобразовательной школы № 65. Далее домов с адресами этой
улицы нет.

Фонтанная улица в пределах между улицами Торговой и Архипа Куинджи никогда не
была оживленной, а в наши дни, кажется, совсем обезлюдела. Редкие прохожие, еще
более редкие автомобили.
