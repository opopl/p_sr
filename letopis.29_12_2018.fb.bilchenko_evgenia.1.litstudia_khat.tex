% vim: keymap=russian-jcukenwin
%%beginhead 
 
%%file 29_12_2018.fb.bilchenko_evgenia.1.litstudia_khat
%%parent 29_12_2018
 
%%url https://www.facebook.com/yevzhik/posts/1960495327318901
 
%%author Бильченко, Евгения
%%author_id bilchenko_evgenia
%%author_url 
 
%%tags akter,bilchenko_evgenia,literatura,poezia,teatr,teatr.kiev.khat,ukraina
%%title Литстудия БЖ при участии актеров КХАТ: отчет.
 
%%endhead 
 
\subsection{Литстудия БЖ при участии актеров КХАТ: отчет.}
\label{sec:29_12_2018.fb.bilchenko_evgenia.1.litstudia_khat}
\Purl{https://www.facebook.com/yevzhik/posts/1960495327318901}
\ifcmt
 author_begin
   author_id bilchenko_evgenia
 author_end
\fi

Литстудия БЖ при участии актеров КХАТ: отчет.

Ну, раком-боком начали мою идею-фикс по возрождению традиций Таганки в Украине:
наконец-то вместе собрали поэтов и актеров. Честно говоря, опыт трудный. Актеры
давят на эмоции и подачу, поэты - на технику и мастерство. Первые в глазах
вторых рискуют прослыть дилетантами, вторые в глазах первых - занудами. Чтобы
как-то примирить лошариков и сухариков, долго автопати-думаем над единством
содержания, формы и тела поэта по Гумилёву. Теперь по отдельным гостям.

\ifcmt
  tab_begin cols=2

     pic https://scontent-mxp1-2.xx.fbcdn.net/v/t1.6435-9/48958846_1960494590652308_1548914555835383808_n.jpg?_nc_cat=106&ccb=1-3&_nc_sid=8bfeb9&_nc_ohc=lasK2R16axkAX8Fu62k&_nc_ht=scontent-mxp1-2.xx&oh=6fefcc16924add6c24bc48b0c04d91a1&oe=60D5CFAF

     pic https://scontent-mxp1-2.xx.fbcdn.net/v/t1.6435-9/49121038_1960494850652282_3744951567871115264_n.jpg?_nc_cat=105&ccb=1-3&_nc_sid=8bfeb9&_nc_ohc=qMd9N_HPXA4AX_GjVxi&_nc_ht=scontent-mxp1-2.xx&oh=9d6e80405e3aa89ceb933dda4916744b&oe=60D72713

  tab_end
\fi

\ifcmt
  tab_begin cols=2

     pic https://scontent-mxp1-2.xx.fbcdn.net/v/t1.6435-9/49203464_1960494663985634_2380295288037310464_n.jpg?_nc_cat=105&ccb=1-3&_nc_sid=8bfeb9&_nc_ohc=6esdBb4yxLYAX_pyCff&_nc_ht=scontent-mxp1-2.xx&oh=0202b8e55c22230f24d33a3208fbfe90&oe=60D65ADC

		 pic https://scontent-mxp1-2.xx.fbcdn.net/v/t1.6435-9/48991622_1960494710652296_3652164851049955328_n.jpg?_nc_cat=103&ccb=1-3&_nc_sid=8bfeb9&_nc_ohc=c3C3ezxOgZYAX-vFrBU&_nc_oc=AQlV9QG5nWxzWV6p562Oik1tPpMdw_Tq5A2pyZNIdryEone4h-ecYdELdGi0kJ4iIBk&_nc_ht=scontent-mxp1-2.xx&oh=3827eb2483b84d307822f885a2898704&oe=60D77D4B

  tab_end
\fi


\begin{itemize}
  \item ВЯЧЕСЛАВ РАССЫПАЕВ - он еще придумал новые методы деконструкции языка. Немыслимо.
  \item ВИКТОР ГРИЗА - убедилась, что мы понимаем друг друга с полувзгляда, и это информация для нас двоих: аудитория к ней не готова. Донбасс.
  \item АНДРЕЙ ЧУПАХИН - четкая прямота. Дай Бог, ты не обиделся на наше с Волковым: "Добавь полифонии, чел".
  \item АЛЕКСЕЙ КРУШЕЛЬНИЦИЙ - жду, когда ты впадешь в полозковский сплин, столько радости, что становится страшно, чел.
  \item ВЛАД АРТЁМОВ - перспективный молодой поэт-актер. Как только перестанет косить под Есенина, родится для литературы. Дала список, че читать.
\end{itemize}

\ifcmt
  tab_begin cols=3

     pic https://scontent-mxp1-2.xx.fbcdn.net/v/t1.6435-9/49612775_1960494820652285_8337811821222166528_n.jpg?_nc_cat=108&ccb=1-3&_nc_sid=8bfeb9&_nc_ohc=sC4F_40w8t0AX9PMtJs&_nc_ht=scontent-mxp1-2.xx&oh=bd6dd2706ecabc4012d2af36902f4eab&oe=60D5CDF7

		 pic https://scontent-mxp1-2.xx.fbcdn.net/v/t1.6435-9/49103483_1960494920652275_2004815786593484800_n.jpg?_nc_cat=100&ccb=1-3&_nc_sid=8bfeb9&_nc_ohc=exud5qBWi2MAX--Hb8k&_nc_ht=scontent-mxp1-2.xx&oh=449d9a9ec3d9460d3474fd47ccd374a3&oe=60D68533

		 pic https://scontent-mxp1-2.xx.fbcdn.net/v/t1.6435-9/48429802_1960495057318928_5380148812637536256_n.jpg?_nc_cat=100&ccb=1-3&_nc_sid=8bfeb9&_nc_ohc=Rh0ld3sY_KkAX_DJA6g&_nc_ht=scontent-mxp1-2.xx&oh=90036b2a8094fa1faff40ff5b1d130e5&oe=60D600D5

  tab_end
\fi

\begin{itemize}
  \item ТАТЬЯНА КАРАБЕТ - обалленный автор из Приднестровья. Школа: нечто между Долгаревой, Ревякиной и луганским "Станом".
  \item КАТАРИНА СИНЧИЛЛО - ну, что я могу сказать о примадонне? Она вдохновляет. Ее наивность на фоне чувственности трогает.
  \item КОЛЯН АРЕСТОВ - начал петь харьковских поэтесс.
  \item АЛЕКСЕЙ КОСТРОМИН - явный рост вглубь. Чел начал давать архетипы.
  \item ВАДИМ ВОЛКОВ - тут без комментариев. Учитесь у мастера.
\end{itemize}

\ifcmt
  tab_begin cols=1
		 pic https://scontent-mxp1-2.xx.fbcdn.net/v/t1.6435-9/48422717_1960495120652255_2607365931189403648_n.jpg?_nc_cat=105&ccb=1-3&_nc_sid=8bfeb9&_nc_ohc=l4lWu4mdKJMAX907o0m&_nc_ht=scontent-mxp1-2.xx&oh=6178db970d7d4123f759d93450abdb10&oe=60D5A90F
		 width 0.5
  tab_end
\fi

\begin{itemize}
  \item ЮЛИЯ БОГДАНОВА - та же фигня. Она талантище. 
  \item ОЛЕГ МАКСИМЕНКО - впервые я увидела у него режиссуру выступления: он подобрал разностилевые вещи и этим пленил актеров.
  \item РОБЕРТ КРИВОШЕЕВ - нет, он не делает фэнтези в стихах. Он сам фэнтези, а в рюкзаке у него - живые ястребы.
  \item ИРИНА РЫПКА - вся архаика Сибири, откуда она сама, баснословно прекрасный мелос.
  \item МАРГО ГЕЙКО - девушка давно переросла куртуазный маньеризм, но еще в это не верит.
  \item ЛЮДМИЛА ТРОЯНОВСКАЯ - школа Кабира и Салминой. Добротная современная поэзия.
  \item ЖЕНЯ ЧИСТЫЙ - альтернатива кабировщине. Традиции а ля Артем Сенчило.
  \item ДЕНИС ВАСИЛЬЕВ - добрый умный православный рэпер с добрыми умными православными рассказами. Степень подтекста наших с ним общений так же выношу за скобки: аудитория у меня начинает шугаться Востока.
  \item ЮРИЙ АНИСИМОВ - алкогольный перформанс. Юра, было весело, потому что семирижды начатый стих в итоге вышел вполне приличным.
  \item ДМИТРИЙ НИКИШИН - оказывается, он пишет не только глубокую музыку, но и умные тексты.
\end{itemize}

\ifcmt
  tab_begin cols=3

		 pic https://scontent-mxp1-2.xx.fbcdn.net/v/t1.6435-9/48944820_1960495203985580_3375800193931804672_n.jpg?_nc_cat=106&ccb=1-3&_nc_sid=8bfeb9&_nc_ohc=ibY1WSReWcoAX8rGzoX&tn=ntrKbsW_7ChXu3v-&_nc_ht=scontent-mxp1-2.xx&oh=5ae9a68f7e11fdc8cc6f0f3bdc29ef79&oe=60D6D9B9

		 pic https://scontent-mxp1-2.xx.fbcdn.net/v/t1.6435-9/49074673_1960495250652242_3279063193825050624_n.jpg?_nc_cat=106&ccb=1-3&_nc_sid=8bfeb9&_nc_ohc=rPBX76QuN7cAX8teSzR&_nc_ht=scontent-mxp1-2.xx&oh=97d5c76d56d3941bb5082c873bc6edbc&oe=60D597CA

  tab_end
\fi

Вроде никого не забыла. 

PS. Потом еще пошли на одну студию, где было много украинского фолка и тут -
опа - я со своим русским и православным рок-рэпом. Запомнились слова забавной
девушки с претензией на жительницу ПариЗЖу (лучше бы занималась украинским
фолком): \enquote{Стихи впечатлили, но руки я вам не подам}. \enquote{Ну, и
хрен с тобой}, - сказала я вслух, а про себя подумала: \enquote{Ниче, в Праге я
таких барышень доводила до слез и полной расшивки. И они человеками
становились}. Всем понимания и челосущности.
