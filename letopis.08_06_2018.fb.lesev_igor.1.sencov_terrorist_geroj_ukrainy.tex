% vim: keymap=russian-jcukenwin
%%beginhead 
 
%%file 08_06_2018.fb.lesev_igor.1.sencov_terrorist_geroj_ukrainy
%%parent 08_06_2018
 
%%url https://www.facebook.com/permalink.php?story_fbid=1944104222287338&id=100000633379839
 
%%author_id lesev_igor
%%date 
 
%%tags geroizm,sencov_oleg,terrorizm,ukraina
%%title Сенцов – террорист или герой Украины?
 
%%endhead 
 
\subsection{Сенцов – террорист или герой Украины?}
\label{sec:08_06_2018.fb.lesev_igor.1.sencov_terrorist_geroj_ukrainy}
 
\Purl{https://www.facebook.com/permalink.php?story_fbid=1944104222287338&id=100000633379839}
\ifcmt
 author_begin
   author_id lesev_igor
 author_end
\fi

Сенцов – террорист или герой Украины?

Тему закинул Мураев, против которого ГПУ уже оперативно возбудила два уголовных
дела. Но все же, давайте попытаемся ответить на этот вопрос.

\ifcmt
  ig https://scontent-frt3-2.xx.fbcdn.net/v/t1.6435-9/34814419_1944103245620769_51331799854350336_n.jpg?_nc_cat=103&ccb=1-5&_nc_sid=730e14&_nc_ohc=r5-s01knfG4AX_jIGRU&_nc_ht=scontent-frt3-2.xx&oh=dce186c787fd10c97ac720195a23e903&oe=61B75B0D
  @width 0.4
  %@wrap \parpic[r]
  @wrap \InsertBoxR{0}
\fi

Для начала поставим следующий вопрос. Почему Украина защищает Сенцова? Ну, то
что гражданин, понятно. А по сути? Украина считает, что найденная взрывчатка на
месте задержания Сенцова – это подстава и провокация? Кстати, защита Сенцова
именно так и считает. Но тогда другой вопрос, - а в чем тогда геройство
Сенцова, если его просто подставили? Мы же не считаем героем студента Игоря
Индило, убитого в далеком 2010 году в ментовке. И покойная Оксана Макар –
придушенная и неудачно сожженная – тоже не герой Украины. Это жертвы. Даже
казак Гаврилюк, раздетый на холоде беркутовцами – жертва ментовского произвола,
а не «герой Майдана». Поэтому, если Сенцову накидали какой-то взрывоопасной
дряни в рюкзачок – он жертва.

Но из Сенцова украинские власти вся равно усиленно лепят героя. Почему? А может
быть потому, что теоретически замысел Сенцова – если такой имел место быть –
что-либо взорвать на территории РФ и есть геройство? Но тут опять же очень
скользкая дорожка для официального Киева. Молодогвардейцы в Великую
Отечественную не были террористами, хотя их немцы судили и казнили именно как
террористов. Не был террористом и Йозеф Аллербергер, положивший из снайперской
винтовки за войну более двух с половиной сотен советских солдат. И
молодогвардейцы, и Аллербергер – это герои войны, хоть и стоящие по разные
стороны баррикад.

И теперь мы переходим к ключевому вопросу. Теоретическое намерение гражданина
Украины уничтожить коммуникации, инфраструктурные сооружения, объекты
жизнеобеспечения или военнослужащих Российской Федерации – это акт терроризма
или нет? Давайте упростим вопрос. Если гражданин Украины заедет на территорию
РФ (скажем, в Белгород), убьет российского военнослужащего и вернется в
Украину, будет ли считаться по нашим действующим законам это преступлением?

А ведь по Сенцову вопрос не упрощается, а усложняется. Его задержали со
взрывчаткой на территории Крыма, а не в Белгородской области. Крым сами знаем
чей. Поэтому вопрос официальному Киеву звучит так: допускается ли гражданину
Украины самовольно планировать/организовывать уничтожение инфраструктуры на
временно оккупированном Крымском полуострове? Это преступление или геройство?
Честно ответив на этот вопрос, прояснится и статус Сенцова. А пока что эти
вопросы пусть задает в украинском суде Мураев.
