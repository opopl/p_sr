% vim: keymap=russian-jcukenwin
%%beginhead 
 
%%file 11_09_2021.fb.fb_group.story_kiev_ua.1.doktor_tri_raza
%%parent 11_09_2021
 
%%url https://www.facebook.com/groups/story.kiev.ua/posts/1750243641839096
 
%%author_id fb_group.story_kiev_ua,miljko_alena
%%date 
 
%%tags kiev
%%title Доктор. Я видела доктора три раза
 
%%endhead 
 
\subsection{Доктор. Я видела доктора три раза}
\label{sec:11_09_2021.fb.fb_group.story_kiev_ua.1.doktor_tri_raza}
 
\Purl{https://www.facebook.com/groups/story.kiev.ua/posts/1750243641839096}
\ifcmt
 author_begin
   author_id fb_group.story_kiev_ua,miljko_alena
 author_end
\fi

% -------------------------------------
\ii{fbauth.miljko_alena.kiev.ukraina.pisatel}
% -------------------------------------

Доктор. 

Я видела доктора три раза. 

В первый раз, когда он стоял в майских сумерках на крыльце  корпуса
Александровской  больницы и курил. Было прохладно, цвели каштаны. Еще утром я
забыла ключи от квартиры, и, чтобы попасть домой, мне пришлось ехать за ними к
маме почти через весь город. 

\ifcmt
  ig https://scontent-frt3-1.xx.fbcdn.net/v/t39.30808-6/241799303_1329513564130326_3444856603022210533_n.jpg?_nc_cat=108&_nc_rgb565=1&ccb=1-5&_nc_sid=825194&_nc_ohc=Cg2TIcbltDsAX_4uIk-&_nc_ht=scontent-frt3-1.xx&oh=48b725a50dc3e274db432f5e9afa38ba&oe=6141D7B5
  @width 0.4
  %@wrap \parpic[r]
  @wrap \InsertBoxR{0}
\fi

Он стоял на крыльце больницы и курил, а я стояла в темноте и видела, как
озаряет свет из приоткрытой двери его силуэт. Смотрела и не могла отвести
взгляд. Он казался суперменом. Невероятно красив. К тому же белый халат, джинсы
и белые туфли. 

Я смотрела на него и от смущения не знала, как начать разговор. Мне казалось
неудобным попросить его позвать маму. Он заметил меня сам:

– Что тебе, девочка, нужно?

Я сказала. Он сходил и позвал ее. Она вышла недовольная, поругала меня. При
нем. За невнимательность. Ведь как можно выходить из квартиры, не проверив,
есть ли ключи в сумке? Я стояла, пригвожденная к месту, и на меня сверху вниз
смотрел доктор. Смотрел и улыбался. А мне было стыдно, что мама делает мне
замечание при нем. 

Смотрел он, как потом оказалось, не просто так, а изучал наше с мамой сходство.
Когда я повзрослела, периодами сильно была похожа на маму в таком же возрасте.
А девочкой я больше напоминала папу. Когда я уже ушла с ключами, доктор в шутку
сказал маме: «Не ваша работа».

В метро я вспоминала каждый его жест, каждое слово. То, как он держал сигарету.
Как смотрел. Как улыбался. Я была очарована.

Прошло несколько лет. Однажды мама сказала, что сегодня у нее в гостях будет ее
коллега. Я и предположить не могла, что гостем будет тот самый доктор, которого
я видела когда-то в мае на крыльце больницы. У меня дрожали руки, когда он в
прихожей подавал свой плащ и спрашивал, куда пройти. 

Они с мамой разговаривали о работе. Я сидела в своей комнате, сердце стучало,
как у мышонка. Я вслушивалась в обрывки их беседы и нечего  не понимала. У него
оказался красивый грудной голос. Потом мама позвала меня попрощаться с гостем,
и я, как воспитанная девочка, вышла сказать «до свидания». Он переступил порог
и улыбнулся. Мама, собираясь закрыть за ним дверь, спохватилась и вслед ему
крикнула, что он забыл книги по хирургической стоматологии, за которыми и
приходил. Он, стоя уже на ступеньках лестницы, ответил, что это повод прийти
еще. 

«Еще» не наступило никогда. 

Позже из маминого разговора с кем-то по телефону я совершенно случайно узнала о
трагических событиях в его жизни. Оказывается, он раньше жил и работал в
Новосибирске. Имел жену и детей. Потом встретил женщину, которая настояла на
его разводе. К ней он и приехал в Киев. В новой семье у него родилось два сына.
Жена его занимала очень высокую должность . И расцвете карьеры ее посадили за
взятки на десять лет. Старший , а затем и младший  сын стали наркоманами, один
из них заболел ВИЧ, а сам доктор спился. Имущество у них конфисковали, и в один
день он стал просто бомжом.

С тех пор прошло много времени. Накануне пасхальных праздников, ярким
апрельским утром, мы с мамой делали покупки на Владимирском рынке. Уже шли к
выходу. Рядом с дверями стоял сгорбленный мужчина в старом клетчатом пиджаке и
спортивных штанах, просил милостыню. Волосы почти седые, лицо серое. Небрит. В
глазах пустота. Мне показалось, что ему как будто вовсе и не нужно подаяние, он
просто стоял с протянутой рукой. Мы прошли мимо. Потом мама остановилась и
быстро вернулась к попрошайке. Она дала ему большую милостыню и пошла назад. Он
вслед спросил: 

– Кто вы? 

Мама сама себе тихо ответила: «Не важно». 

Мы сели в мою машину, мама задумалась. Я спросила, чем же ее так впечатлил этот
бомж? И она рассказала о докторе, с которым оперировала когда-то вместе. И о
том, какой он был талантливый и блестящий хирург. И я вдруг поняла, что она
говорит именно о том докторе, которого я видела в детстве на крыльце больницы.
Я не могла поверить в то, что с ним такое произошло. Я вспоминала его белый
халат, руки, улыбку и красивый голос.

Прошло еще года четыре. Маму пригласили на День медицинского работника в
больницу, где она проработала тридцать пять лет. Она долго готовилась к этой
встрече и даже сшила новое платье. Ей так хотелось увидеть бывших коллег,
узнать, кто и как живет сейчас. Я купила – на том же, кстати, рынке – цветы и
конфеты и подвезла ее до больницы.

После официальной части сотрудников пригласили на фуршет, где можно было
пообщаться со знакомыми. Вернулась мама поздно, уставшая, но счастливая от
встреч, общения и поздравлений. Я спросила, присутствовал ли тот доктор.
Оказалось, что нет. А потом она рассказала продолжение его истории. 

Когда они еще работали вместе, в него была влюблена одна санитарка. Так сильно,
что не смела на него даже глаз поднять. Когда он проходил мимо, у нее все
валилось из рук. Но, исполняя свои обязанности, девушка всегда уделяла ему
особое внимание. Утюжила его халаты, подавала чай, убирала в ординаторской. И в
каждое его дежурство ездила в больницу из пригорода. Все знали о том, что
санитарка  влюблена в доктора, над ней не смеялись – сочувствовали. Когда же
она узнала, что он стоит с протянутой рукой на рынке, то приехала и забрала его
к себе. Своей заботой и любовью за несколько лет вернула его к нормальной
жизни. Он даже подтвердил врачебную категорию и стал работать
хирургом-стоматологом в амбулатории в ее маленьком пригороде.

Что это – сострадание? Безусловная любовь к ближнему? Или женское упорство:
ждать и дождаться своего?
