% vim: keymap=russian-jcukenwin
%%beginhead 
 
%%file 23_09_2021.fb.nikonov_sergej.3.bilchenko_tusovka
%%parent 23_09_2021
 
%%url https://www.facebook.com/alexelsevier/posts/1587779861567323
 
%%author_id nikonov_sergej,bilchenko_evgenia
%%date 
 
%%tags bilchenko_evgenia,kultura,obschestvo,poezia,travlja,tusovka,ukraina
%%title БЖ. Тусовке
 
%%endhead 
 
\subsection{БЖ. Тусовке}
\label{sec:23_09_2021.fb.nikonov_sergej.3.bilchenko_tusovka}
 
\Purl{https://www.facebook.com/alexelsevier/posts/1587779861567323}
\ifcmt
 author_begin
   author_id nikonov_sergej,bilchenko_evgenia
 author_end
\fi

\ifcmt
  ig https://scontent-lga3-1.xx.fbcdn.net/v/t1.6435-9/242788486_1587777354900907_4197950038268677752_n.jpg?_nc_cat=107&_nc_rgb565=1&ccb=1-5&_nc_sid=730e14&_nc_ohc=PkiO0Qgd5EsAX_sqGkj&_nc_ht=scontent-lga3-1.xx&oh=fb15ae7c2fc46b33f0d54e678e2250a7&oe=6171D17D
  @width 0.4
  %@wrap \parpic[r]
  @wrap \InsertBoxR{0}
\fi

У каждого бывают падения и взлеты. Поэты могут написать о них. А я здесь в роли молчаливого историка или библиографа, переписчика биографии. Молчу без мнений.
БЖ. Тусовке
А помните эти славные фестивальные времена,
Когда я шла впереди толпы, а за мною скакали все?
Дьявол ввел меня в прелесть: ведь мне казалось, я не одна,
И за мной готовы бежать друзья по огневой росе.
Как охотно из моих рук получали свои дипломы вы!
Не очередь, - давка имела место к свободному микрофону.
Моё имя, как знак, превратилась в символ, ломаный языком молвы.
У меня разрывались входные двери, мессенджеры, смартфоны.
Каждый хвастал, что он знаком со мной, у меня бывал.
Каждый не преминул отметить, что наряд у меня красивый.
После феста мой дом превращался (покотом спим) в вокзал,
То есть, просто в вокзал, а не в: "Вокзал-чемодан-Россия".
Все мои лекции были открытыми: скрывать было вправду нечего.
Правда была курящей, на курилке отважно липли.
Правда была настоящей и почти что не изувеченной.
Мой дух пытались споить за мои же, обдирая меня, как липку.
Всё на своё: грамоты, антологии, книги, залы.
Бесплатные входы, бесплатные души - символы самобытности.
Я вас прощаю, иудофилы. Я уже всё сказала.
Мне больше добавить нечего - даже, если б и было что.
Меня ещё раздражало, что все вы страстно хотели
Фото со мной на память: желающих было много.
От поцелуев ваших живого места на теле
Не оставалось, а от следов - терялась моя дорога.
На самом деле ангел меня надоумил и вразумил,
Чего стоит ваша верность, ваша богемная суета.
Чего стоит ваш Парнас, который меня поил,
Но не споил, не сломал, не сгрёб и живую не снял с креста.
Но вы не сердитесь, что я молчу в ответ на звонки сейчас.
Они бывают ещё раз в месяц: ваши застенчивые "Алло".
"Как у тебя дела?" - "Родня, я счастливей вас".
Я встретила Бога,
Живого Бога...
Мне сказочно повезло.
23 сентября 2021 г.
