% vim: keymap=russian-jcukenwin
%%beginhead 
 
%%file 13_10_2021.fb.aljona_ptaha.1.pesnja_bandera_shkolniki
%%parent 13_10_2021
 
%%url https://www.facebook.com/permalink.php?story_fbid=2060936207400425&id=100004521976128
 
%%author_id aljona_ptaha
%%date 
 
%%tags bandera_stepan,deti,kiev,shkola,ukraina
%%title Песня про Бандеру - киевские школьники
 
%%endhead 
 
\subsection{Песня про Бандеру - киевские школьники}
\label{sec:13_10_2021.fb.aljona_ptaha.1.pesnja_bandera_shkolniki}
 
\Purl{https://www.facebook.com/permalink.php?story_fbid=2060936207400425&id=100004521976128}
\ifcmt
 author_begin
   author_id aljona_ptaha
 author_end
\fi

Скажу, все таки, пару слов по поводу киевских школьников, триумфально
исполнивших песню "Батько наш Бандера", чтобы досадить учительнице.

Не по поводу самих школьников скажу, а вообще, о песнях.

Песня эта, не просто не новая, а очень старая.

Калька с идеологического убожества тридцатых годов- "Сталин- отец народов".

Ровно та же самая история, только одну фамилию заменили на другую, и
многочисленные народы на один.

Страну погрузили ментально в 20е-30е годы прошлого века в худшем их проявлении.

Не вчера погрузили, факт, за пять лет каденции Порошенко, но и за прошедшие с
неё 2 года ничего не сделали для того, чтобы её из этого бреда вытащить.

И бред ведь тут не в личности даже покойного Степана и его мелкой исторической
роли, бред в подходе, в том, как современных людей, детей и взрослых,
превращают в комсомольцев тридцатых, с безумно горящими глазами.

Что с этим делать?

Уже ничего, ничего не сделаешь так, как это делается у нас, невидимая рука
рынка отрегулирует, заменив проблемы дрессировки учителей более насущными
проблемами.

А вот что стоило бы сделать, так это написать сочинение.

Детки, вижу, вы поёте, что Степан Бандера ваш Батько, герой и что там ещё
принято в культах мелких личностей.

Ок.

Вот программа ОУН(Б), где Б обозначает как раз покойного Степана, тут важно не
ошибиться, ибо были версии.

Программа начала лета 41 года, самый пик стремительного полёта надежд и мыслей,
вот сейчас перекурим, пока вермахт и айнзацгруппы справятся, и пойдём управлять
державой.

Пишите сочинение, что в этой программе вам нравится, дорогие школьники, и
почему.

Какие именно пункты.

Потом пишите дальше, как именно, по вашему, этого можно было бы достичь.

И, в конце, последний вопрос, обязательный для всех.

Мы ведь не знаем, что можно написать, одни напишут одно, другие другое, все ж
разные.

А вопрос последний- общий на всех.

А вы, вот именно ты, Маша, Вася, Петя, хотели бы, чтобы с вами сделали то, что
вы написали и так, как вы написали?

Это же не с кем то другим, это с вами лично, с младшими братом и сестрой, с
бабушкой и дедушкой, мамой и папой.

Точно хотите?

Я думаю нет, не точно.

И думаю, что если начать разбираться в деталях, целях и средствах, не захочется
не только пробовать, и петь не захочется.

Но это уже другая история, не про нас.

Слова не мои, но поддерживаю на 1000!!!
