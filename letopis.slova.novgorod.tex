% vim: keymap=russian-jcukenwin
%%beginhead 
 
%%file slova.novgorod
%%parent slova
 
%%url 
 
%%author 
%%author_id 
%%author_url 
 
%%tags 
%%title 
 
%%endhead 
\chapter{Новгород}
\label{sec:slova.novgorod}

%%%cit
%%%cit_pic
%%%cit_text
Родившись где-то в районе 942 года, Владимир стал \emph{князем Новгородским}, потому
что Киевом правили его старшие братья – попеременно. Когда Ярополк убил своего
брата Олега, в Киев явился с дружиной Владимир и убил Ярополка. Вот так
наследный Новгородский князь стал князем Киевским. Кстати, титул \emph{Новгородского
князя} остался при нем тоже. А потом он стал и каганом, то есть ханом, равным
будущему императору Чингисхану.  Вот так русский (рус, русич) князь-каган
захватил украинский Киев и крестил всех его языческих жителей. А если бы не
захватил?  Ну, тогда пришлось бы московитам впоследствии отвоевывать Киев у
хазар или половцев, а может даже у самого Чингисхана, имевшего на него свои
виды
%%%cit_title
\citTitle{Почему киевский хан Владимир Креститель считается украинским князем, когда он даже славянином не был?},
Исторический Понедельник, zen.yandex.ru, 05.01.2021 
%%%endcit

%%%cit
%%%cit_pic
%%%cit_text
Воевал киевский князь за свою жизнь множество раз. Мы привыкли представлять
себя воителем прежде всего его отца Святослава. Но сын оказался не менее грозен
и легок на подъем. Он воевал с поляками и хазарами, с радимичами и булгарами, с
печенегами и ятвягами. Даже на греческой принцессе он женился только после
военного похода, силой принудив грекам отдать ему обещанную невесту.  Но
последний свой поход Владимир собирался провести не против соседей, а против
родного сына Ярослава. Того, который позже станет известен как Ярослав Мудрый.
Страна, которую князь сколачивал воедино столько лет железными гвоздями,
трещала по швам. Сыновья, разосланные по разным городам, уже примеряли на себя
роль властителей и косо поглядывали на отца. Ярослав, сидевший в
\emph{Новгороде}, дерзко отказался присылать Владимиру дань.  Отец собирался
примерно наказать сына-бунтовщика. Именно поэтому он, уже лежа на смертном
одре, приказал готовить маршрут, по которому двинутся его войска на мятежный
\emph{Новгород}. Князь все еще надеялся оправиться и навести порядок в своем
государстве и в своей семье
%%%cit_comment
%%%cit_title
\citTitle{Мостите мост. Что означали последние слова князя Владимира?}, 
Русичи, zen.yandex.ru, 09.06.2021
%%%endcit
