% vim: keymap=russian-jcukenwin
%%beginhead 
 
%%file 16_01_2022.stz.news.ua.hvylya.1.anatomia_vraga.2.triggery_starta
%%parent 16_01_2022.stz.news.ua.hvylya.1.anatomia_vraga
 
%%url 
 
%%author_id 
%%date 
 
%%tags 
%%title 
 
%%endhead 

\subsubsection{Триггеры старта}

Чисто технически, возвеличиться Москве помогло не только близкое и длительное
взаимодействие с Золотой Ордой, но и косвенно Киев, в скором времени ушедший в
литовский государственный проект вместе с большей частью территории современной
Украины. В 1299 году хан Золотой Орды Тохта, вступив в противостояние с
мятежным беклярбеком Ногаем, разорил улус (владения) Ногая, к которому, кстати,
относилось все Среднее Поднепровье с Киевом. Согласно летописи, после штурма
города \enquote{весь Киевъ разыдеся}.

Именно после этого страшного разорения Киева ханом Тохтой в 1299 году киевский
митрополит Максим переносит свою резиденцию, так называемое \enquote{седалище}
из Киева во Владимир-на-Клязьме, по сути, смещая \enquote{управленческую
администрацию} сакрального центра клерикализма Руси из потерявшего свое
политическое значение Киева на северо-восток в \enquote{стольный город}
Залесской Руси - Владимир - то есть - в ближайшую орбиту будущей экспансии
Москвы. \textbf{Это был первый технический триггер.}

\textbf{"... в 1299 г. митрополит Максим, не стерпев насилия татарского, собрался со всем
своим клиросом и уехал из Киева во Владимир на Клязьму...", что послужило началом
дальнейшего приобретения Москвой значения церковной столицы Руси уже в годы его
преемника, митрополита Петра." (с) Ключевский}

Ещё одним триггером, в том же 1299 году, стал переезд значительного количества
знати со своими \enquote{служилыми людьми} из опустошенного татарами Киева на
территорию Московского княжества. Этот \enquote{переезд} сразу сказался на увеличении
военной силы Москвы и ее управленческого потенциала. Именно рутинная
необходимость обеспечивать содержание возросшего числа \enquote{служилых людей} стала
одной из причин первых системных экспансионистских действий младшего сына
Александра Невского - московского князя Даниила.
