% vim: keymap=russian-jcukenwin
%%beginhead 
 
%%file 13_11_2020.fb.tanya_naumchuk.1.dva_sleda
%%parent 13_11_2020
 
%%url https://www.facebook.com/manya.naumchuk/posts/785551092303114
%%author 
%%author_id 
%%tags 
%%title 
 
%%endhead 

\subsection{Два следа (Притча)}
\label{sec:13_11_2020.fb.tanya_naumchuk.1.dva_sleda}
\Purl{https://www.facebook.com/manya.naumchuk/posts/785551092303114}
\Pauthor{Наумчук, Таня}
\index[writers.rus]{Наумчук, Таня!Два следа, притча}

\ifcmt
pic https://scontent-waw1-1.xx.fbcdn.net/v/t1.0-9/125165118_785551062303117_2949422791541466349_n.jpg?_nc_cat=101&ccb=2&_nc_sid=8bfeb9&_nc_ohc=z38vT3kNiswAX8TxNXG&_nc_ht=scontent-waw1-1.xx&oh=8f96bedcc9bdba3431019701189969a5&oe=5FDDC178
caption Два следа (Притча)
\fi

\begin{multicols}{2}
\obeycr
Однажды Человек спросил у Бога:
--- Легка ли будет у меня дорога?
--- Легка, когда меня ты помнить будешь,
Ни на минуту не забудешь.
--- А как я буду видеть, что ты рядом?
--- Тебе  это знать совсем и не надо,
Ко мне обращайся в молитвах всегда
И я не оставлю тебя никогда.
И в горе с тобою я буду, и в счастье,
В телесных недугах, в душевном ненастье.
А, если захочешь, так к морю пойдешь,
И там , на песке, ты два слЕда найдешь.
Идут дни за днями, проходят года...
Тут, вдруг, Человека настигла беда,
Он Бога просил, чтоб помог его горю.
Когда все прошло, он отправился к морю.
На мокром песке там следы отискал;
И было все так, как Господь предсказал.
Знать был Он с ним рядом, его не оставил
И щедро устами он Бога пославил.
Огромное счастье имел Человек
И с радостью Бога прославил на век!
Отправился к морю увидеть следы.
Нашел те два слЕда опять у воды.
Но, вдруг, Человека настиг тяжкий недуг...
А был ли Господь с ним тогда, или нЕ был??
Он тяжко страдал, сил и чувства лишался,
Но, Господа славить и тогда не гнушался.
Когда же здоровье к нему возвратилось,
Увидеть следы ему вновь захотелось.
О, Боже! У кромки воды был один всего след...
Вставал из-за моря багряный рассвет.
---- О , Бог мой! Когда я был болен
Оставил меня Ты!! Ты мной не доволен!??
Вдруг с неба послышался голос могучий:
-- Могу объяснить тебе я этот случай,
На мокром песке не твоих ног следы,
То след моих ног у кромки воды.
Когда ты болел, не стоял на ногах
Тебя я в то время носил на руках!
12.11.2020г.  ( Наумчук Таня)
\restorecr
\end{multicols}

