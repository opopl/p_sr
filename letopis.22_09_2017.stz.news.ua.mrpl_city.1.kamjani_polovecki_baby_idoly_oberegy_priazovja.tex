% vim: keymap=russian-jcukenwin
%%beginhead 
 
%%file 22_09_2017.stz.news.ua.mrpl_city.1.kamjani_polovecki_baby_idoly_oberegy_priazovja
%%parent 22_09_2017
 
%%url https://mrpl.city/blogs/view/kamyani-polovetski-babiidoli-oberegi-priazovya
 
%%author_id demidko_olga.mariupol,news.ua.mrpl_city
%%date 
 
%%tags 
%%title Кам'яні половецькі баби – ідоли-обереги Приазов'я
 
%%endhead 
 
\subsection{Кам'яні половецькі баби – ідоли-обереги Приазов'я}
\label{sec:22_09_2017.stz.news.ua.mrpl_city.1.kamjani_polovecki_baby_idoly_oberegy_priazovja}
 
\Purl{https://mrpl.city/blogs/view/kamyani-polovetski-babiidoli-oberegi-priazovya}
\ifcmt
 author_begin
   author_id demidko_olga.mariupol,news.ua.mrpl_city
 author_end
\fi

Археологічні артефакти, що знаходяться на території Приазов'я і Маріуполя,
зокрема, мають власну історію і для багатьох можуть стати справжнім
відкриттям.

\ii{22_09_2017.stz.news.ua.mrpl_city.1.kamjani_polovecki_baby_idoly_oberegy_priazovja.pic.1}

Не всім відомо, що половецькі скульптури, які в народі мають назву \enquote{кам'яні
баби}, мають унікальне значення і дають можливість більше дізнатися про
середньовічних кочовиків – половців, що справили величезний вплив на всі
сторони економічного, соціально-політичного і культурного життя Київської Русі
епохи феодальної роздробленості. У Маріуполі три  половецьких скульптури
знаходяться біля краєзнавчого музею, але більшість ідолів зберігаються на
подвір'ї музею.

Згідно літописам половці вперше з'явилися біля кордонів Русі в 1055 р. і,
уклавши мирний договір, пішли в степ. З цього моменту почалася двохсотлітня
епопея ворожнечі і дружби цього кочового народу з Руссю.

Головним джерелом для вивчення історії, побуту, культури половецького народу
були і залишаються їхні статуї – кам'яні баби. Термін \enquote{кам'яні баби} (baba) в
перекладі з тюркського говору – пращур. А можливо називання кам'яних статуй
\enquote{бабами} походить від слова \enquote{pählaban}, що перекладається з перських діалектів
як \enquote{богатир, атлет}.

\ii{22_09_2017.stz.news.ua.mrpl_city.1.kamjani_polovecki_baby_idoly_oberegy_priazovja.pic.2}

У Приазов'ї число кам'яних баб може наближатися до 600. На частині з них можна
виявити певні етнографічні межі і, відповідно, приналежність до конкретного
народу. Половецькі статуї (XI – XIV ст.) виготовлені майстрами в місцевих
кар'єрах з граніту, піщанику, вапняку. Фігури (що стоять або сидять)
відтворюють канонізований образ чоловіків і жінок з певним набором прикмет:
головним убором, зачіскою, одягом, взуттям, прикрасами, зброєю. На голові у
чоловіків – воїнів, одягнутих в каптан, півсферичний шлем, а жіночі фігури
увінчані капелюхом. Цікаво, що не лише у жінок, а деколи і у чоловіків
підкреслені голі груди – символ сили. Всі статуї обов'язково \enquote{тримають} біля
живота судину в руках. Виходячи з вірувань половців, він наповнений священним
напоєм. Судина символізувала і присутність духа померлого. З точки зору
анімістичних уявлень тюрок, шаман міг укласти душу померлого в кам'яну статую.
Етнографічні відомості дозволяють збудувати сценарій давніх обрядів \enquote{за участю}
статуй.  Виготовлена до дня поминань, статуя встановлювалася біля дерев'яного
житла, що стоїть на кам'яній опорі (огорожі), яка будувалася в першу чергу.
Статуя була уособленням померлого, який немов був присутнім серед учасників
тризни на власних поминаннях. Пізніше проводили повторні поминання, після яких
статую розбивали, звільняючи душу померлого, а житло спалювали. Після нього
залишалася огорожа, яку закидали камінням. Так була використана частина статуй
і в Приазов'ї, більшість з них мають значні дефекти: втрачена голова,
пошкоджені руки, ноги.

\ii{22_09_2017.stz.news.ua.mrpl_city.1.kamjani_polovecki_baby_idoly_oberegy_priazovja.pic.3}

Обличчя деяких жіночих статуй облямовують \enquote{роги} – вони, як вважають учені,
виконували не тільки роль декору, але й мали магічне значення. Високі
\enquote{капелюхи} з \enquote{рогами} були найбільш святковим вбранням жінок. \enquote{Роги} мали і
смислове значення. Вони були, ймовірно, символом дівчини-нареченої. Наречена у
половців ототожнювалася з бараном, а наречений і його дружки з вовками,
зобов'язаними наздогнати дівчину.

Зображення жінок несли різне семантичне навантаження: мати, знатна пані і
навіть войовниця. Фігури чоловіків – із зброєю і військовим спорядженням – це
зображення вождів, а сидячі статуї відображують старійшин родів. У половців був
свій пантеон божеств, який також відбивався в статуях. Наприклад, жіночу статую
половці пов'язували з богинею Умай,  чоловіча фігура ототожнювалася з
верховним божеством Кам.

\ii{22_09_2017.stz.news.ua.mrpl_city.1.kamjani_polovecki_baby_idoly_oberegy_priazovja.pic.4}

Половці були язичниками і не виключено, що вони присвячували скульптури саме
пантеону божеств. Не даремно класик персидської поезії Нізамі, спираючись на
розповіді своєї дружини-половчанки, відзначав, що всі племена половців
згинаються перед цією єдиною у своєму роді скульптурою, вершник залишає перед
нею стрілу, пастух, який заведе до неї отару, опускає перед нею вівцю. Дійсно,
і на території Приазов'я у підніжжя половецьких статуй знаходили кістки
баранів, а один раз – кістки дитини – дівчинки. Жіночі статуї обожнювали, їх
боялися, у них просили допомоги. І все ж таки не можливо визначити остаточно
кому вони ставилися. Відомо, що мешканці Приазов'я вірили у здатність скульптур
оберігати та приносити удачу. Так, у 2011 р. жителі с. Тельманово не захотіли
передавати бабу до музею, боялися, що після цього від них відвернеться удача і
влітку не підуть дощі, які так потрібні для вдалого урожаю.

Цікавим є той факт, що кам'яним бабам половців присвячені пісні, картини,
фільми, - вони вже давно стали надбанням культури людства.

\ii{22_09_2017.stz.news.ua.mrpl_city.1.kamjani_polovecki_baby_idoly_oberegy_priazovja.pic.5}

Сьогодні гостро стоїть проблема збереження половецьких скульптур, час руйнує
безцінні пам'ятки минулого, якими багате Приазов'я. Необхідно прийняти всі міри
для подальшого збереження половецької культури, яка уособлює звичаї, релігійні
уявлення та неповторне мистецтво вже неіснуючого народу і є складовою частиною
історії нашого регіону.

\textbf{\em Половецькі скульптури в образотворчому мистецтві:}

\ii{22_09_2017.stz.news.ua.mrpl_city.1.kamjani_polovecki_baby_idoly_oberegy_priazovja.pic.6}
\ii{22_09_2017.stz.news.ua.mrpl_city.1.kamjani_polovecki_baby_idoly_oberegy_priazovja.pic.7}
\ii{22_09_2017.stz.news.ua.mrpl_city.1.kamjani_polovecki_baby_idoly_oberegy_priazovja.pic.8}
\ii{22_09_2017.stz.news.ua.mrpl_city.1.kamjani_polovecki_baby_idoly_oberegy_priazovja.pic.9}

У 2015 р. в рамках проекту \enquote{Маріуполь – це Україна} магістр історії Наталя
Черкун завдяки кам'яним половецьким бабам реконструювала костюми кочовиків.
Фіналісти конкурсу \enquote{Маріуполь – це Україна} отримали можливість взяти участь у
фотопроекті \enquote{Край, епоха – разом ми}, де вони перевтілилися у половців та
половчанок.

\ii{22_09_2017.stz.news.ua.mrpl_city.1.kamjani_polovecki_baby_idoly_oberegy_priazovja.pic.10}
\ii{22_09_2017.stz.news.ua.mrpl_city.1.kamjani_polovecki_baby_idoly_oberegy_priazovja.pic.11}
\ii{22_09_2017.stz.news.ua.mrpl_city.1.kamjani_polovecki_baby_idoly_oberegy_priazovja.pic.12}
