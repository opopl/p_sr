% vim: keymap=russian-jcukenwin
%%beginhead 
 
%%file 18_01_2022.fb.miheev_vladislav.1.pafos_tomos_bitva.cmt
%%parent 18_01_2022.fb.miheev_vladislav.1.pafos_tomos_bitva
 
%%url 
 
%%author_id 
%%date 
 
%%tags 
%%title 
 
%%endhead 
\zzSecCmt

\begin{itemize} % {
\iusr{Олег Климов}
Короче, меньше слов - больше тел (за решеткой;)))

\iusr{Юрий Малик}

\ifcmt
  ig https://scontent-frx5-2.xx.fbcdn.net/v/t39.30808-6/272024987_4652807298165151_2868941557904312452_n.jpg?_nc_cat=109&ccb=1-5&_nc_sid=dbeb18&_nc_ohc=PR_iC_JtDBMAX8mXeOs&_nc_ht=scontent-frx5-2.xx&oh=00_AT_z0wIeLmunQopbfVmf1iolz41SHDBP4el5rx_DTUksnQ&oe=61EB19BC
  @width 0.2
\fi

\iusr{Владимир Михайловский}
Самая значимая часть выражения \enquote{украинские интеллектуалы} - кавычки.

\iusr{Наталия Иванова}
Очень точно, Владик! Позор на все наши головы!

\iusr{Alexander Klymenko}

Владислав, доброе утро. Конечно со многим согласен. По или Зе? Здравая реакция
может быть только одна - СДД, тут раввин прав. Отдельное спасибо за смелую
цитату Делёза, который отважился опровергнуть диалог, как форму тысяч лет
философствования, метод Сократа, Платона... И действительно, в этом отношении
он почувствовал, как радикально меняется что-то важное в мире. Диалог уходит
из-за интуитивного, на уровне животного инстинкта понимания, что больше нет
будущего. Поэтому растет нетерпимость, взаимная агрессия, ненависть и
противоречия. Повсюду усиливается. А вот индивидуальное, отдельное творчество,
где-то в стороне у нас как раз возможно, на границе миров, в эпицентре безумия,
парадоксально. Оно и есть у тех единиц, кто во внутренней эмиграции, кто номада
по духу. Хорошо знаю и мировое и, например, российское современное искусство, и
уверен - у нас есть три-пять художников высшего мирового уровня. И в их
творчестве нет ни грамма пропаганды и выгодной власти лжи. Увы, публичная
литература, философия, кино и история в Украине, оказались насквозь
пропагандистскими, брехливыми и примитивными. Как минимум на виду только те в
этой сфере, кого поощряет режим.

\end{itemize} % }
