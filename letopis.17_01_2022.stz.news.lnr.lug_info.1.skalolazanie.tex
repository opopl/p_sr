% vim: keymap=russian-jcukenwin
%%beginhead 
 
%%file 17_01_2022.stz.news.lnr.lug_info.1.skalolazanie
%%parent 17_01_2022
 
%%url https://lug-info.com/news/bolee-50-sportsmenov-prinali-ucastie-v-cempionate-i-pervenstve-lnr-po-skalolazaniu
 
%%author_id news.lnr.lug_info
%%date 
 
%%tags 
%%title Более 50 спортсменов приняли участие в чемпионате и первенстве ЛНР по скалолазанию
 
%%endhead 
\subsection{Более 50 спортсменов приняли участие в чемпионате и первенстве ЛНР по скалолазанию}
\label{sec:17_01_2022.stz.news.lnr.lug_info.1.skalolazanie}

\Purl{https://lug-info.com/news/bolee-50-sportsmenov-prinali-ucastie-v-cempionate-i-pervenstve-lnr-po-skalolazaniu}
\ifcmt
 author_begin
   author_id news.lnr.lug_info
 author_end
\fi

Более 50 спортсменов стали участниками открытого первенства и чемпионата ЛНР по
скалолазанию, которые прошли на базе Республиканского физкультурно-спортивного
общества (РФСО) \enquote{Динамо} МВД ЛНР. Об этом сообщил Центр общественных связей
ведомства.

\enquote{На базе физкультурно-спортивной организации 
\enquote{Динамо} МВД ЛНР состоялись два
значимых спортивных мероприятия: открытое первенство ЛНР и чемпионат Республики
по скалолазанию. Всего на старты вышли более 50 участников}, – говорится в
сообщении.

В соревнованиях приняли участие команды РФСО \enquote{Динамо}, Республиканского центра
детско-юношеского туризма и краеведения, Луганского учебно-воспитательного
объединения \enquote{Академия детства}, а также учреждений внешкольного и
дополнительного образования Свердловска и Первомайска.

Нижняя возрастная планка для спортсменов, вышедших на старт первенства, – 2016
год рождения. В чемпионате принимали участие взрослые, состоявшиеся спортсмены,
достигшие 21 года. Верхняя возрастная граница составляла 35 лет.

Участники состязались в номинациях \enquote{Трудность} и \enquote{Скорость}. По их результатам
и определялся окончательный командный результат. Победителем чемпионата стала
команда РФСО \enquote{Динамо} МВД ЛНР.

Председатель Федерации альпинизма и скалолазания ЛНР Валерий Коноваленко
отметил, что первенство и чемпионат являются не только спортивными
мероприятиями, но и механизмом отбора для участия в соревнованиях более
высокого статуса, в частности, на территории России.

\enquote{Мы достаточно давно и достаточно успешно там выступаем. Однако именно
республиканские старты являются лакмусом, проверочным камнем, на котором мы
тестируем методики, которые использовали тренеры, и оцениваем фактический
уровень спортсменов}, – пояснил он.

В МВД рассказали, что начале февраля в Луганске состоится открытое
детско-юношеское первенство РФСО \enquote{Динамо} по скалолазанию в дисциплине
\enquote{боулдеринг}. Планируется, что, помимо спортсменов из ЛНР, в нем примут участие
команды из Донецка, Горловки и Макеевки.
