% vim: keymap=russian-jcukenwin
%%beginhead 
 
%%file 02_01_2022.fb.druzenko_gennadiy.1.bandera.cmt
%%parent 02_01_2022.fb.druzenko_gennadiy.1.bandera
 
%%url 
 
%%author_id 
%%date 
 
%%tags 
%%title 
 
%%endhead 
\zzSecCmt

\begin{itemize} % {
\iusr{Yaroslav Znych}

Дуже популярна на сторінках Тік-току, і не тільки, пісня \enquote{Батько наш -
Бандера, Україна - мати} - не є автентичною повстанською піснею. Бандера не був
батьком у 40-50-ті і зараз не може бути батьком України: не той масштаб і не та
ідеологія, яка допоможе об'єднати українців і збудувати країну Мрії. PS ♨️
Фольклористка Інституту філології КНУ Ярина Закальська говорить, що
\enquote{...це варіант пісні УПА «Ой у лісі, лісі, під дубом зеленим, там
лежить повстанець тяженько ранений». За мелодією оригінальна і нова пісні майже
однакові, проте дещо відрізняються тексти.} PPS @igg{fbicon.face.nerd} Дивно,
як на мене, що немає/невідомий автор сучасної частини тексту пісні, яка стала
популярною навіть у Московії.

\iusr{Zinoviy Svereda}

Справа не у Бандері. Справа в підході. Просто історію пишуть тогочасні
переможці. Пілсудський ким був під час періоду, коли Росія заволоділа частиною
Польщі? і як його трактувала Росія? Терорист і т д. бо він підривав російських
жандармів і т д...Засновники і борці за незалежність Ізраїлю що робили?
взривали британських вояк. і т д... І це в кожній країні... просто істоію
пишуть переможці того періоду. Якщо б українська держава в той час з'явилась,
то країни-сусіди б ніяких претензій б не ввисували щодо. того, хто є героєм
України. А так 40 років радянської пропаганди, і не тільки, бо багато
держав-сусідів мають територіальні претензії до України: не хочуть визнати, що
украінці теж мають право на самовизначення. і мають право трактувати свою
історію так, як потрібно згідно принципу, що кожен народ має право на свою
державу.

\iusr{Марія Тановицька}

Як казав шановний Віталій Портніков (в далекому 2011 році): \enquote{Політиків, навіть
таких, які зазнали повної поразки, треба залишати історикам, а моральні
приклади, такі - як Шептицький, треба залишати людям}

\iusr{Богдан Панкевич}

Ви оцінюєте поширення пісні про Бандеру та мітинги в день його народження як
глорифікацію. Натомість переважна більшість виконавців та учасників нічого про
реального Бандеру та його методи не знають. Тому їхні дії це лише протистояння
рускому міру і московській брехні про Бандеру та Україну. Це звичайна
контрпропаганда на користь Україні. Вона нервує московитів? Значить досягає
мети. А переосмислення прийде згодом.


\iusr{Сергій Худий}

Тобто поява банерів у Хмельницькому і Тернополі це справа рук ФСБ, я вірно Вас
зрозумів?

\ifcmt
  ig https://scontent-frx5-1.xx.fbcdn.net/v/t39.30808-6/270474644_4818781901477789_5835023908090389262_n.jpg?_nc_cat=100&ccb=1-5&_nc_sid=dbeb18&_nc_ohc=-QmvUifKe4wAX8OzyuG&_nc_ht=scontent-frx5-1.xx&oh=00_AT8aY1dk2PNttVmEsL4KzsPcIrOf2xrNN9PXydUQl0mq-A&oe=61D81C84
  @width 0.3
\fi

\iusr{Gennadiy Druzenko}
\textbf{Сергій Худий} 

це справа рук корисних ідіотів. Принаймні те, що «Свободу» за часів Януковича
фінансував Клюєв, який чомусь втік до Росії, - для мене доконаний факт.

\iusr{Денис Гнатюк}

Ну ось на жаль не стратили ми ригів/суддів/мусорів у 2014-му. Сильно то помогло?

\iusr{Gennadiy Druzenko}
\textbf{Денис Гнатюк} 

нам - так. Бо якщо б почали страти, це б дало Путіну карт-бланш на наведення
порядку в Україні. Чи - у разі перемоги - перетворило б Україну на аналог Куби

\iusr{Alexander Lotosh}
Гарно та глибоко!

\end{itemize} % }
