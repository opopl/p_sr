% vim: keymap=russian-jcukenwin
%%beginhead 
 
%%file 02_12_2021.fb.smolij_andrij.1.kumiry
%%parent 02_12_2021
 
%%url https://www.facebook.com/permalink.php?story_fbid=4636407583063674&id=100000834337338
 
%%author_id smolij_andrij
%%date 
 
%%tags kultura,kumir,muzyka,nebesna_sotnja,rossia,ukraina,ukraincy
%%title Це кумири українців. 8 років війни. 8 років з часу загибелі Небесної Сотні
 
%%endhead 
 
\subsection{Це кумири українців. 8 років війни. 8 років з часу загибелі Небесної Сотні}
\label{sec:02_12_2021.fb.smolij_andrij.1.kumiry}
 
\Purl{https://www.facebook.com/permalink.php?story_fbid=4636407583063674&id=100000834337338}
\ifcmt
 author_begin
   author_id smolij_andrij
 author_end
\fi

Це кумири українців. 

8 років війни. 8 років з часу загибелі Небесної Сотні. 8 років боротьби за свою
країну.

Мені соромно. І гидко до глибини душі. 

Сьогодні Spotify та Apple music називають ТОП-100 найпопулярніших пісень 2021
року в Україні. 

\ii{02_12_2021.fb.smolij_andrij.1.kumiry.pic.1}

Жодної української пісні. Точніше є одна. ОДНА ЗІ 100. Більшість - рашистський
непотріб, який продукує Кремль і все що з ним пов‘язане. 

Моргенштерн, Скриптоніт, Інстасамка, Король і Шут, Лобода і десятки інших…

Знаєте яке найпопулярніше відео в Україні? «ДВОРЄЦ ПУТІНА». 

А це точно Україна? Ми точно не в Білорусі чи на Росії? Ми точно 8 років воюєму
з ворогом, що десятками і сотнями років знищував нашу державність та
ідентичність? 

Розумієте, ми вже в котрий рік «приїхали». І таке враження що виходу з цього
замкнутого кола немає. Паростки українізації, чи повернення до свого що почали
закладатись у 2016-2018 роках множать на нуль. Нам кажуть «какая разніца», нам
втовкмачують в голови «сватів» та «кварталів», нашу молодь отруюють російським
болотом, до нас сотнями знов заїжджають російські гастролери…і так можна
перелічувати дуже і дуже довго. 

Путні аплодує. Україна знов у культурній пастці Кремля. А головне - там
українська молодь, для якої сучасна українська культура - це щось чуже і
незрозуміле.

Весілля, заходи, вечірки, свята звичайних українців просто нашпиговані
російським. Навіть не усвідомлюючи цього, ми в своїй країні створюємо фактичний
російський простір.

Втрутитись мала б держава. Але держава мовчить. Ба більше - своїм мовчанням
сприяє плодоносносності російського культурно-інформаційного вірусу.

Ви думаєте звідки оті сумнозвісні блогери плюють на Голодомор? Звідки на
«тусовках» цих мільйонників саме російська музика? От саме звідти, що російське
стало незамінною частиною життя країни, яка бореться з агресором. 

Подвійні стандарти чи подвійна реальність. Дві суспільні реальності, які не
вкладаються в адекватну голову. Але вони є, існують і розвиваються. Одна
реальність - де Захисники та волонтери, активісти та патріоти захищають свою
країну. Інша - де суспільна більшість живе «русскім міром». 

Вихід? Заборонити культурне російське. Не давати рашистам тут «шастать».
Вкладати мільярди гривень не у ефемерне «Велике крадівництво», а у розвиток
свого, українського. Того, що буде тримати всю країну на вістрі процесів, а не
російського занепаду. 

Але поки лише навпаки. 

Навіть в ці хвилини, в ці години, тисячі і мільйони наших співвітчизників
споживають російське і отруюють наше з вами майбутнє. 

Не варто дивуватись тому кого вибирають і кого приводять до влади. І як би це
сумно це звучало:

поки в головах русскоміровське л**но ми не вийдемо з порочного пола постійних
помилок і трагедій...

\ii{02_12_2021.fb.smolij_andrij.1.kumiry.cmt}
