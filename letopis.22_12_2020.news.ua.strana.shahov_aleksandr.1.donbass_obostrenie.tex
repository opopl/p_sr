% vim: keymap=russian-jcukenwin
%%beginhead 
 
%%file 22_12_2020.news.ua.strana.shahov_aleksandr.1.donbass_obostrenie
%%parent 22_12_2020
 
%%url https://strana.ua/news/308068-obostrenie-na-donbasse-17-20-dekabrja-chto-ob-etom-izvestno.html
 
%%author 
%%author_id shahov_aleksandr
%%author_url 
 
%%tags 
%%title В пять раз больше обстрелов. Почему на Донбассе снова началось обострение боев
 
%%endhead 
 
\subsection{В пять раз больше обстрелов. Почему на Донбассе снова началось обострение боев}
\label{sec:22_12_2020.news.ua.strana.shahov_aleksandr.1.donbass_obostrenie}
\Purl{https://strana.ua/news/308068-obostrenie-na-donbasse-17-20-dekabrja-chto-ob-etom-izvestno.html}
\ifcmt
	author_begin
   author_id shahov_aleksandr
	author_end
\fi

\ifcmt
pic https://strana.ua/img/article/3080/obostrenie-na-donbasse-68_main.jpeg
cpx Иллюстративное фото Минобороны Украины 
\fi

На Донбассе перед новогодними праздниками начал просыпаться фронт. Количество
обстрелов увеличилось, стороны рапортуют о раненых и погибших.

В Украине эту тенденцию никак не комментируют, хотя она подтверждается
объективными данными ОБСЕ. Причем цифры нарушений в отчетах Миссии разительно
выше тех, которые приводит штаб Операции объединенных сил.

То есть обострение пытаются закамуфлировать под рядовые "провокации
противника", на которые украинские войска традиционно дают отпор. 

При этом о погибших пока заявляют лишь сепаратисты, а ОБСЕ обвиняет Украину в
том, что армия выводит тяжелое оружие из мест хранения, предусмотренных
перемирием.

Проходит ли перемирие испытание на прочность или же ситуация скатывается в
войну - разбиралась "Страна". 

\subsubsection{Что говорят в ОБСЕ}

В последние дни на Донбассе действительно идет резкий рост боестолкновений. Об
этом свидетельствуют  отчеты миссии ОБСЕ. 

Если брать период с прошлого по этот понедельник, то мы увидим, что со
стандартных 30-40 нарушений в сутки статистика ухудшилась да 100-200 обстрелов. 

Вот как выглядела эта "эволюция" в датах отчетов Миссии:

\begin{itemize}
  \item 14 декабря вышел отчет с 45-ю нарушениями
  \item 15-го - 21 нарушение
  \item 16-го - 39 нарушений
  \item 17-го - 46 нарушений
  \item 18-го - 84 нарушения
  \item 19-го - 234 нарушения 
  \item 21-го - 99 нарушений
\end{itemize}

То есть рост напряженности начался в четверг вечером (отчет от 18 декабря).
Тогда Миссия заявила о 30 взрывах под Мариуполем и о 19 - в районе Донецкой
фильтровальной станции.

Также в ОБСЕ отметили, что украинские войска разместили на Луганщине
зенитно-ракетный комплекс - он стоял на территории, откуда такое вооружение
следовало отвести. А в Донецкой области ВСУ вывели из ангаров десять самоходных
гаубиц, что также противоречит условиям отвода. 

Со стороны "ДНР" и "ЛНР" таких же нарушений в этот день не было. А в среду
наблюдатели ОБСЕ впервые с лета заметили воронку от минометного обстрела вблизи
подконтрольного сепаратистам поселка Дачное к западу от Луганска. 

Но перейдем к отчету за пятницу - когда число нарушений подскочило практически
в пять раз по сравнению с началом недели. 

Основная доля обстрелов пришлась на Донецкую область - 232 из 234-х. Что видно
и по карте. Большая часть боев опять пришлась на район Мариуполя.

\ifcmt
pic https://strana.ua/img/forall/u/0/92/%D0%BA%D0%B0%D1%80%D1%82%D0%B0_%D1%81%D0%BC%D0%BC.png
\fi

Также в этот день Миссия нашла в Луганской области семь украинских установок
залпового огня, которые по железной дороге завезли на территорию, откуда это
вооружение, наоборот, следовало отводить.

Параллельно наблюдатели обнаружили один ЗРК сепаратистов, расположенный в жилом
секторе одного из населенных пунктов Луганщины.

\ifcmt
pic https://strana.ua/img/forall/u/0/92/%D0%BE%D1%82%D0%B2%D0%BE%D0%B4.png
\fi

В Донецкой области ОБСЕ не досчиталось украинского тяжелого вооружения в
местах, где оно должно храниться. То есть, по всей вероятности, эта техника
вернулась на фронт. 

\ifcmt
pic https://strana.ua/img/forall/u/0/92/%D0%B2%D0%BE%D0%BE%D1%80%D1%83%D0%B6%D0%B5%D0%BD%D0%B8%D0%B5.png
\fi

В субботу обострение продолжилось. Большая часть взрывов пришлась на поселок
Спартак, который находится под контролем "ДНР". 

Всего с момента установления перемирия ОБСЕ зафиксировала более четырех тысяч
нарушений. Из них почти две тысячи - это взрывы: то есть стрельба из
артиллерийских систем и детонация мин. И еще две тысячи - автоматные очереди. 

При этом, конечно, это все равно в разы меньше, чем за 2019 год: тогда в
среднем было по 800-900 нарушений в сутки. Но и цифры по итогам пяти месяцев
перемирия все равно показательные. Обе армии находятся в боеготовом состоянии и
постоянно прощупывают противника обстрелами. 

Это значит, что за время перемирия коренного перелома в противостоянии не
произошло: подтягивается оружие, идут обстрелы. И ситуация в любой подходящий
для этого момент может сорвать в новое обострение. Начатки которого мы уже
наблюдали 17-20 декабря.

\subsubsection{Что говорят в Киеве, "ЛДНР" и США}

В Украине говорят о нескольких раненых вследствие боев конца минувшей недели.
Но в целом тревожной риторики от командования ООС и политического руководства в
Киеве не звучит. За один из самых тяжелых дней декабря - 18 число -
зафиксировали всего пять нарушений. 

Иное дело в непризнанной "ДНР". Там заявили, что в последние три дня идет
серьезное обострение, а 18 декабря - когда и ОБСЕ зафиксировала резкое
ужесточение боев - погибли двое служащих "Народной милиции". Это произошло на
юге, под Мариуполем.

В "МИД ДНР" заявили, что перемирие на грани срыва, и новогодний режим тишины,
соответственно, тоже. "Мы оставляем за собой право адекватного ответа на
действия вооруженных формирований Украины", - предупредили в Донецке.

То есть, как видим, о том, что идет обострение в основном говорят сепаратисты.
И эти данные подтверждаются отчетами ОБСЕ (по крайней мере, касаемо обстрелов). 

В то же время как именно ситуация сорвалась в новые обстрелы, да еще и накануне
Нового года и Рождества - вопрос открытый. 

Интересно, что пока Украина фиксирует обострение как рядовой процесс, по его
поводу внезапно выступило посольство США. Сегодня оно призвало Россию
"прекратить агрессию" и напомнило про двух раненых украинских солдат и одного
гражданского. 

\ifcmt
pic https://strana.ua/img/forall/u/0/92/%D0%BF%D0%BE%D1%81%D0%BE%D0%BB%D1%8C%D1%81%D1%82%D0%B2%D0%BE(9).png
\fi

\subsubsection{Почему началось новое обострение}

В предыдущий раз фронт пришел в движение после местных выборов. Мы рассказывали
о том, как почти сразу после первого тура 25 октября на фронте пошли убитые и
раненые.\Furl{https://strana.ua/news/298108-obostrenie-na-donbasse-30-oktjabrja-chto-proiskhodit.html}

Та ситуация показала, что перемирие само по себе вещь очень шаткая и без
развития успеха на политическом треке ничего по большому счету не изменит. 

Прошло почти два месяца, и обострение повторяется. Поводом для обстрелов могло
тактически стать что угодно. Однако стратегически - это по-прежнему все то же
отсутствие политического прогресса между сторонами.

Причем накануне по многим дипломатическим позициям пошел откат назад. В
Трехсторонней контактной группе провалилась идея согласовать "дорожную карту"
выполнения Минских соглашений. В ответ Россия привезла в ООН представителей
"ДНР" и "ЛНР",\Furl{https://strana.ua/news/304355-sovbez-oon-s-dnr-i-lnr-zajavlenija-po-ukraine-video-zasedanija-2122020.html} показав Киеву, что готова и сама решить судьбу "республик".

А в декабре против почти всех "слуг народа" Кремль ввел санкции.\Furl{https://strana.ua/news/306237-sanktsii-rossii-protiv-deputatov-sluhi-naroda-pochemu-ikh-vveli.html} Раньше
подобная "черная метка" посылалась Петру Порошенко - когда стало окончательно
понятно, что с ним ни о чем договориться невозможно. 

Все это было приурочено к годовщине Нормандского саммита, решения которого были
на 90\% не исполнены. Украина в частности даже близко не подошла к выполнению
пунктов политического характера: она не вынесла на обсуждение проекты законов
об особом статусе Донбасса и не узаконила "формулу Штайнмайера". 

При этом весь год из Киева шли официальные заявления, что Минские соглашения
нужно поменять. На что Путин несколько дней назад озвучил свой ответ:\Furl{https://strana.ua/news/307185-putin-na-press-konferentsii-rasskazal-o-perspektivakh-mira-v-donbasse.html} этого не
произойдет. И Россия будет настаивать на нынешней редакции Минска-2. 

Таким образом Кремль сделал заявку даже не столько на позицию в отношениях с
Украиной - сколько на "красные линии" для США, где к власти приходит новый
президент Джо Байден. И при котором переговоры с Москвой по Донбассу могут
возобновиться. 

Однако, чем больше стрельбы на фронте - тем тяжелее эти переговоры будут
протекать. И тем более вероятно, что Вашингтон займет жесткую позицию в
отношении Москвы. И тем менее вероятно, что Россия и США смогут прийти к
каким-то компромиссам.

Аналогичная ситуация была сразу после прихода к власти Трампа. Когда именно
обострение боев под Авдеевкой побудило президента США сделать первый звонок
Петру Порошенко. И вскоре разговоры, что при новом президенте начнется прорыв в
мирных переговорах сошли на нет.
