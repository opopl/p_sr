%%beginhead 
 
%%file 17_01_2023.fb.ohmatdet.bolnica.1.vpav_z_30_metrovo__v
%%parent 17_01_2023
 
%%url https://www.facebook.com/ndslohmatdyt/posts/pfbid02iBcurnVghZ2wd4cmpNYgVZXnGQTLntkrg3ZJJ9LTAs274H45a8MzJAkbFo46grbdl
 
%%author_id ohmatdet.bolnica
%%date 17_01_2023
 
%%tags medicina,ohmatdet.bolnica
%%title Впав з 30-метрової висоти та вижив: в Охматдиті врятували підлітка з Чернігівської області
 
%%endhead 

\subsection{Впав з 30-метрової висоти та вижив: в Охматдиті врятували підлітка з Чернігівської області}
\label{sec:17_01_2023.fb.ohmatdet.bolnica.1.vpav_z_30_metrovo__v}

\Purl{https://www.facebook.com/ndslohmatdyt/posts/pfbid02iBcurnVghZ2wd4cmpNYgVZXnGQTLntkrg3ZJJ9LTAs274H45a8MzJAkbFo46grbdl}
\ifcmt
 author_begin
   author_id ohmatdet.bolnica
 author_end
\fi

⚡️Впав з 30-метрової висоти та вижив: в Охматдиті врятували підлітка з
Чернігівської області⚡️

15-річний хлопчик внаслідок нещасного випадку впав з 30-метрової вишки (це
висота дев’ятиповерхового будинку). Дитина була госпіталізована у важкому стані
у міську лікарню, через 5 днів — до Охматдиту, для порятунку життя. Хлопець
поступив у стані керованої коми у відділення анестезіології та інтенсивної
терапії з ЕКМО. Йому діагностували політравму: забій головного мозку, закриту
травму грудної клітини, забій легень та нирок, розрив печінки, закриті
багатоуламкові переломи обох стегон зі зміщення уламків. Лікарі кажуть, шанси
вижити після такого падіння у хлопця були мінімальні.💔

При поступленні в лікарню хлопець не рухав ногами. В Охматдиті йому ургентно
провели сегментарну внутрішню фіксацію хребта та закриту репозицію перелому
стегон. Два місяці за життя та здоров’я хлопця боролася величезна кількість
медиків: хірурги, реаніматологи, ортопеди-травматологи, торакальні хірурги,
нейрохірурги, неврологи, спеціалісти відділення діагностики, анестезіологи,
реабілітологи, педіатри, фізіотерапевти, психологи, медичні сестри усіх
відділень.🙏🏻

Завдяки роботі мультидисциплінарної команди, життя хлопчика вдалося врятувати.
Його стан стабілізувався. Через два місяці лікування, у дитини почав
відновлюватися рух у лівій кінцівці. Хлопець вже рухає пальцями та навіть
згинає коліно. І це — величезна перемога при таких тяжких і численних травмах.
Дитина продовжила реабілітацію у санаторії. Зараз хлопчик вже самостійно
сидить, активні рухи в ногах відновилися.🙌🏻

Лікарі закликають батьків бути уважними до своїх дітей, бо причини подібних
падінь з висоти можуть бути різними. Часто підлітки намагаються таким чином
вкоротити собі віку, або піддаються сучасним трендам і залазять на небезпечні
висоти, аби зробити селфі. Іноді випадковість стає причиною подібних трагедій.
Наша лікарня є опорним центром лікування політравми у дітей, спеціалісти
цілодобово готові надавати високоспеціалізовану допомогу.⚡️
