% vim: keymap=russian-jcukenwin
%%beginhead 
 
%%file slova.tvorchestvo
%%parent slova
 
%%url 
 
%%author 
%%author_id 
%%author_url 
 
%%tags 
%%title 
 
%%endhead 
\chapter{Творчество}

%%%cit
%%%cit_head
%%%cit_pic
%%%cit_text
Після такого одкровення українська влада постала перед засадничим вибором: 1)
або цілком в дусі ОУН (чи якщо завгодно Лукашенка) почати репресії проти
"колаборантів" та "іноземних агентів" і в такий спосіб остаточно перетворитись
на українську версію "русского мира" (тільки з протилежним геополітичним
вектором), або 2) запропонувати українцям іншу ідентичність, в якій ключовими
будуть не мова, віра та ненависть до москалів, а \emph{творчість},
інклюзивність, конкурентоспроможність та відкритість до світу. Ідентичність
майбутнього, а не минулого.  І найгірший вихід з цієї дилеми – це заховати
голову в пісок і не зробити жодного вибору
%%%cit_comment
%%%cit_title
\citTitle{Фіаско етнічної українізації}, 
Генадій Друзенко, analytics.hvylya.net, 28.07.2021
%%%endcit

