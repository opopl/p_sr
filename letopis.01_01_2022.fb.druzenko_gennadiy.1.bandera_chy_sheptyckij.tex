% vim: keymap=russian-jcukenwin
%%beginhead 
 
%%file 01_01_2022.fb.druzenko_gennadiy.1.bandera_chy_sheptyckij
%%parent 01_01_2022
 
%%url https://www.facebook.com/gennadiy.druzenko/posts/10158691949433412
 
%%author_id druzenko_gennadiy
%%date 
 
%%tags bandera_stepan,ideologia,nacionalizm,oun,sheptickij_mitropolit.ukr,ukraina
%%title БАНДЕРА ЧИ ШЕПТИЦЬКИЙ?
 
%%endhead 
 
\subsection{БАНДЕРА ЧИ ШЕПТИЦЬКИЙ?}
\label{sec:01_01_2022.fb.druzenko_gennadiy.1.bandera_chy_sheptyckij}
 
\Purl{https://www.facebook.com/gennadiy.druzenko/posts/10158691949433412}
\ifcmt
 author_begin
   author_id druzenko_gennadiy
 author_end
\fi

БАНДЕРА ЧИ ШЕПТИЦЬКИЙ?

Іноді, конструюючи національний історичний наратив, важко уникнути принципового
вибору: кого ми воліємо бачити як предмет для наслідування? В кому ми впізнаємо
національний архетип? Кого слід зарахувати до українського пантеону?

Оскільки комуністичний період становлення української державності наразі
затаврований як окупаційно-колоніальний (хоча вважати його таким можна тільки в
мазепинському сенсі: \enquote{самі себе звоювали}), звернемо погляд на міжвоєнну
Галиччину, в якій наразі модно шукати українських героїв. Тим паче і привід
чудовий – 113 років від дня народження творця й провідника ОУН(б) Степана
Бандери.

Я ніколи не приховував: Бандера – не мій герой. Бо символом українського опору
\enquote{совєтам} людину, яка ніколи не воювала проти СРСР, зробила саме радянська
пропаганда. І глорифікація провідника ОУН(б) у сучасній Україні – лише ознака
того, що ми як були так і залишаємось глибоко вторинними щодо наративів, які
пишуть у Москві. Тільки якщо до певного часу в Україні домінувало бажання
наслідувати Москву, то тепер превалює таке ж вторинне за своєю природою
прагнення, аби все було з точністю до навпаки аніж у Москві. Якщо для Кремля
Бандера – диявол во плоті, значить для нас він має бути ледве не батьком
української нації. Діти та й годі...

Втім у тридцять років, що минули від часу проголошення української
незалежності, давно варто навчитись критичній саморефлексії, яка і відрізняє
дорослих від дітей. А вміння критично поглянути на своє минуле нагадує нам, що
Степан Бандера (на відміну від Коновальця чи Шухевича або того ж Івана Бабія,
про якого мова піде нижче) ніколи не воював проти \enquote{совєтів}, але воював не
тільки проти поляків, а й проти братів-українців і на його руках є не тільки
польська, а й українська кров.

Одною з жертв молодих радикалів з ОУН (крайовим провідником якої у 1933-34
роках був саме Степан Бандера) став український педагог та директор Української
гімназії у Львові Іван Бабій, колишнній  старшина Української галицької армії,
що активно виступав за мирне співіснування українців та поляків у Другій  Речі
Посполитій.

Так, за півтора місяці до атентату на Івана Бабія, крайовий провідник ОУН був
заарештований польською владою, і на час вбивства перебував у в‘язниці. Але
вирок Бабію «трибунал ОУН» виніс раніше. Та й власне стиль \enquote{дискусії} з
опонентами з українського табору цілком у дусі Степана Андрійовича та його
адептів.

А тепер почуємо зрілий та відповідальний голос того, кого б я волів бачити
справжнім кумиром сучасного українства і хто б мав заслужено обіймати одне з
центральних місць в українському пантеоні, – Митрополита Андрея Шептицького. У
відповідь на вбивство молодими українськими радикалами українця Івана Бабія
Митрополит Андрей оприлюднив у газеті «Діло» за 5 серпня 1934 року заклик
«Голос митрополита». Далі – цитата:

«Директор Бабій упав жертвою українських терористів, дрож жаху потрясла цілим
народом. Убивають зрадливим способом найліпшого патріота, заслуженого
громадянина, знаменитого педагога, знаного й ціненого всіми приятеля, опікуна й
добродія української молоді. Вбивають без ніякої причини, хіба лише тому, бо їм
не подобалася виховна діяльність покійного. Вона була перешкодою в злочинній
акції втягування середнє-шкільної молоді в підпольну роботу. Якщо так є, то всі
заслужені й розумні українці впадуть з рук скритовбийців, бо нема розумного
українця, який не противився би такій злочинній акції. Нема педаґоґа ані
вчителя, який не стверджував би, що допускається тяжкого злочину проти молоді
той, хто відводить її від праці, а втягає в підпілля. Нема одного батька ані
матери, які не проклинали би провідників, що зводять молодих на бездоріжжя
злочину.

Якщо хочете зрадливо вбивати тих, що противляться вашій роботі, прийдеться вам
всіх учителів і професорів, що працюють для української молоді, всіх батьків і
матерей українських дітей, усіх настоятелів і провідників виховних українських
інституцій, усіх політиків і громадських діячів. А передовсім прийдеться вам
скритовбийствами усунути перешкоди, які у вашій злочинній і глупій роботі
ставляє духовенство разом з Єпископами.

Скритовбийник доконавши зрадливо підлого душогубства вважав себе може за героя,
хоч сповнивши свій злочин утікав, як чоловік, що стидається свого поступку.
Утікав перед дочасною карою, думав, що самогубством втече перед суспільною
відповідальністю і вічною карою. Достойний ученик провідників українських
терористів, що безпечно сидячи за границями краю наші діти вживають до убивства
їх батьків, а самі в безіменній ареолі геройства тішаться вигідним життям
стягаючи жертви заграничних патріотів призначені для народу, котрого добро
нищать.

Злочинну роботу українських терористів, яку ведуть божевільні, осуджує, осудила
нераз і не перестане осуджувати вся українська преса й всі українські політики
без огляду на партійну приналежність. А все-ж є ще такі, що не здають собі як
слід справи, до якого ступня злочинна і безглузда ціла та робота. Тому усіх
товаришів і учнів покійного дир. Бабія запрошую до зложення прилюдного
свідоцтва про його прикмети і заслуги. Нехай цілий народ ясно видить, якою
дорогою він, а якою дорогою його убійники бажали вести нашу молодь.

Поміж незвичайними прикметами покійного була та рідка прикмета, яку виробляв
він і в молоді – відвагу. Знаючи, на що наражається, той старшина української
армії сповняв тяжкий обовʼязок для наших дітей і з пожертвуванням добра
власного і добра родини не покидав становища. До жовніра, що не вмів утікати,
зза плота стріляв не так нещасний збаламучений і засліплений убийник, як радше
боягузи, що втікають перед карою, відповідальністю, опінією, таксамо як у війні
боягузи втікали з фронту».

А тепер – ідіть святкуйте День народження того, кого митрополит Андрей назвав
\enquote{провідником українських терористів}. Тільки не забувайте (зокрема і
греко-католицькі священники та вірні), що глорифікація Бандери – це зрада
Шептицького.
