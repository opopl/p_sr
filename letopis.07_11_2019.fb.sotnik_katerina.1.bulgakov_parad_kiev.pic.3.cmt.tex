% vim: keymap=russian-jcukenwin
%%beginhead 
 
%%file 07_11_2019.fb.sotnik_katerina.1.bulgakov_parad_kiev.pic.3.cmt
%%parent 07_11_2019.fb.sotnik_katerina.1.bulgakov_parad_kiev
 
%%url 
 
%%author_id 
%%date 
 
%%tags 
%%title 
 
%%endhead 

\iusr{Катерина Сотник}

– «Гремели страшные тяжкие колеса, тарахтели ящики, за десятью конными куренями
шла лентами бесконечная артиллерия. Везли тупые, толстые мортиры, катились
тонкие гаубицы; сидела прислуга на ящиках, веселая, кормленая, победная, чинно
и мирно ехали ездовые. Шли, напрягаясь, вытягиваясь, шестидюймовые, сытые кони,
крепкие, крутокрупые, и крестьянские, привычные к работе, похожие на беременных
блох, коняки. Легко громыхала конно-горная легкая, и пушечки подпрыгивали,
окруженные бравыми всадниками.

[...]

— Эх... эх... вот тебе и пятнадцать тысяч... Что же это наврали нам.
Пятнадцать...  бандит... разложение... Господи, не сочтешь. Еще батарея... ещё,
ещё... (думал Николка Турбин).

[...] Лязг, лязг, лязг. Глухие раскаты турецких барабанов неслись с площади
Софии, а по улице уже ползли, грозя пулеметами из амбразур, колыша тяжелыми
башнями, четыре страшных броневика.

Броневики, гудя, разламывая толпу, уплыли в поток туда, где сидел Богдан
Хмельницкий и булавой, чернея на небе, указывал на северо-восток. Колокол еще
плыл густейшей масляной волной по снежным холмам и кровлям города, и бухал,
бухал барабан в гуще, и лезли остервеневшие от радостного возбуждения мальчишки
к копытам черного Богдана. А по улицам уже гремели грузовики»...

(М.Булгаков. Белая гвардия).
