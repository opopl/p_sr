% vim: keymap=russian-jcukenwin
%%beginhead 
 
%%file 16_10_2020.fb.ukr_antarctic_centre.1.zajavky
%%parent 16_10_2020
%%url https://www.facebook.com/AntarcticCenter/posts/1497645670445305
 
%%endhead 

\subsection{Сувора романтика: кількість охочих працювати на станції «Академік Вернадський» збільшується}
\label{sec:16_10_2020.fb.ukr_antarctic_centre.1.zajavky}

\url{https://www.facebook.com/AntarcticCenter/posts/1497645670445305}

Завершився прийом заяв на конкурс до складу зимувального загону 26-ї
Української Антарктичної Експедиції, який проведе рік на станції «Академік
Вернадський».

«Якщо минулого року ми отримали 88 заявок, то цього року вже 103. Можливо,
цьому сприяло те, що вперше за 20 років нам спільно з Міністерством освіти та
науки України вдалося збільшити оплату праці полярників, і тепер вона хоча б
частково компенсує екстремальні умови праці, на які погоджуються зимувальники.
Також не може не радувати, що жінки звикли, що на конкурсі дійсно рівні умови
для всіх, і прагнуть працювати в Антарктиді. Вони подавалися переважно на
наукові посади» - зазначив Євген Дикий, директор НАНЦ.

Загалом на конкурс із відбору полярників до 26-ї УАЕ подали заявки 103
кандидати: 84 чоловіки та 19 жінок. В середньому це - понад 8 претендентів на 1
місце, адже в загін зимівників набирають 12 фахівців.

Найбільше охочих цього року бути системним адміністратором на станції (21
кандидат), далі вже традиційно кухар (15 кандидатів) та лікар (14 кандидатів).
На всі інші посади отримано: 32 заявки на 7 наукових посад (майже по 5
кандидатів на місце), 9 заявок --- на механіка та 7 --- на дизеліста. 

Остаточний склад 26-ої УАЕ буде визначено у кінці 2020 року. Кандидати мають
пройти співбесіду, медогляд, психологічне тестування. Після цього переможці
конкурсу відправляться на командний тренінг, а далі --- в довгий шлях до
Крижаного континенту.

\ifcmt
pic https://scontent.fiev21-2.fna.fbcdn.net/v/t1.0-9/121958591_1497645267112012_5845727662768923405_o.jpg?_nc_cat=111&_nc_sid=730e14&_nc_ohc=wCMqUTMv5AEAX_PIzjg&_nc_ht=scontent.fiev21-2.fna&oh=9ea8b1baa8bc38b99f1e7b1eb32672aa&oe=5FB0D39E
\fi

