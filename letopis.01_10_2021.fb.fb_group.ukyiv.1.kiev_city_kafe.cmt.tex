% vim: keymap=russian-jcukenwin
%%beginhead 
 
%%file 01_10_2021.fb.fb_group.ukyiv.1.kiev_city_kafe.cmt
%%parent 01_10_2021.fb.fb_group.ukyiv.1.kiev_city_kafe
 
%%url 
 
%%author_id 
%%date 
 
%%tags 
%%title 
 
%%endhead 
\subsubsection{Коментарі}
\label{sec:01_10_2021.fb.fb_group.ukyiv.1.kiev_city_kafe.cmt}

\begin{itemize} % {
\iusr{Володимир Наконечний}

Неясно яким способом інформації може бути передана цим "горе-підприємцям", адже
шпилька на карті веде в чисте поле. У них є щось типу сторінки чи сайту?

\begin{itemize} % {
\iusr{Ксеня Подзігун}
\textbf{Володимир Наконечний} 

я шукала по назві (ґуґлила), вони ще тільки но відкриваються, ремонт
закінчують, до них був заклад "южний півопровод ". Це як МАФ. Забігайлівка

\iusr{Ксеня Подзігун}
\textbf{Володимир Наконечний}
Південний пивопровід
073 883 8383\par
\url{https://maps.app.goo.gl/D9xsp9vC5aBbRaXX7}

Це я найшла через ґуґл місце. Попередній заклад, який перетворюється в кієв
сіті. Тому і нема ще в інтернеті нового закладу, бо він офіційно ще не
відкрився

\begin{itemize} % {
\iusr{Володимир Наконечний}
\textbf{Ксеня Подзігун} тоді буде важко якось адресно щось передати через інтернет. Раджу моніторити коли з'явиться десь щось іхнє і тоді повертайтеся, підмога з'явиться)

\iusr{Ксеня Подзігун}
\textbf{Володимир Наконечний} добре, дякую!
\end{itemize} % }

\iusr{Сергій Шум}
От скотиняки! Можливо є сенс направити звернення на 1551? \textbf{Олександр Рудоманов} підкажеш?

\end{itemize} % }

\iusr{Іван Гонта}
\textbf{Ксеня Подзігун}, все ж не зрозумів пропозиції. По шпильці на мапі немає закладів.

\begin{itemize} % {
\iusr{Володимир Наконечний}
\textbf{Іван Гонта} вище відповідь з цього приводу є

\iusr{Іван Гонта}
\textbf{Володимир Наконечний} я зайшов в гугл, і закрив той старий заклад, скоро він пропаде з мапи. Так відповів гугл.

\begin{itemize} % {
\iusr{Ксеня Подзігун}
\textbf{Іван Гонта} , і скоро на карті з'явиться новий заклад, про який я написала. Тільки як відкриється
\end{itemize} % }

\end{itemize} % }

\iusr{Billy Kubrik}
Таки City. Друге слово правильне в них.

\iusr{Сергій Даценко}
Яке ж то city, то pueblo

\iusr{Lana Poroh}

Виставила в групу Борщагівка. Подивіться які ватники в тій групі. Побажали мені
знайти чоловіка а не виставляти це в групу, обізвали нациком, сказали що з
головою в мене проблема. А ви і далі думайте що Київ то україномовне місто.

\begin{itemize} % {
\iusr{Ксеня Подзігун}
\textbf{Lana Poroh} 

а мені в особисті пишуть матюки всякі і кажуть, що Київ та Борщагівка з покон
віків був російськомовним. Хоча 40 років тому хатки стояли і селяни розмовляли
українською чи суржиком. Я запитую: ви корінний, а він каже ні. Я йлму і
подібним :що ж ви приїджджий можете говорити

\iusr{Ксеня Подзігун}
\textbf{Lana Poroh} рашка Київ населяла своїм гражданамі. Щоб перемішати людей з лайном і тепер це л..но роблять подібні заяви

\begin{itemize} % {
\iusr{Lana Poroh}
\textbf{Ксеня Подзігун} коли був Ющенко та ПОП то більшість розмовляли а не говорили. А зараз всі стали голобородьки.
\end{itemize} % }

\end{itemize} % }

\iusr{Анатолій Паладійчук}

\href{https://youtu.be/8yn9b83qs4k}{%
Підкаура feat Воєвуцький. Київ-сіті (Pidkaura feat Voevudsky. Kyiv-city), STVeryBullMusic, youtube, 21.12.2020%
}

\iusr{Túrin Turambar}
Така ж сама х.рня є біля Повітрофотського проспекту

\iusr{Володимир Олівець}
"Сіті" лунає ще менш українсько, Київ- місто!

\iusr{Solomia Shtephan}
Хай з цим розбираються "картко-киянівці"

\end{itemize} % }
