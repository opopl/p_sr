% vim: keymap=russian-jcukenwin
%%beginhead 
 
%%file songs.ukraina
%%parent songs
 
%%url 
 
%%author 
%%author_id 
%%author_url 
 
%%tags 
%%title 
 
%%endhead 

%https://pisni.club/pisni/patriotic/15905-poema-pro-ukrainu.html

Слова Шевченка... Наче кроки
Безсмертних дум його, слова:
"Реве та стогне Дніпр широкий,
Сердитий вітер завива".

Ти, Україно,
Невмируща, якщо і діти, і старі
У бій ідуть в лиху годину,
А з ними – всюди кобзарі.

Так було... Нелюди з гестапо,
Забивши, кинули у став
Сліпого, сивого Остапа
За те, що "Думку" заспівав....
За те, що "Думку" заспівав.

Палало все... Та не згоріла
Остапа "Думка", як зоря,
Бо мудра пісня мала крила
Душі сліпого кобзаря.

Слова Шевченка... Наче кроки
Майбутніх дій – його слова...
"Реве та стогне Дніпр широкий,
Сердитий вітер завива".

Гнобила Неньку "вища" раса...
А біля славних берегів
Могилу генія-Тараса
Громили зграї ворогів...

Та час прийшов! На вражу силу
Піднялись Єдність і Добро,
Звільнивши Неньку нашу сиву,
І мудру "Думку", і Дніпро...
Звільнили рідний наш Дніпро.

На бій пішла уся Вкраїна!
І знову – вільна Русь свята,
Безсмертна наша Україна,
На довгі, радісні літа!

"Реве та стогне Дніпр широкий,
Сердитий вітер завива,
Додолу верби гне високі,
Горами хвилі підійма". 
