% vim: keymap=russian-jcukenwin
%%beginhead 
 
%%file 23_05_2021.fb.bilchenko_evgenia.1.najdenysh
%%parent 23_05_2021
 
%%url https://www.facebook.com/yevzhik/posts/3917556714946076
 
%%author Бильченко, Евгения
%%author_id bilchenko_evgenia
%%author_url 
 
%%tags 
%%title БЖ. Найденыш
 
%%endhead 
 
\section{БЖ. Найденыш}
\label{sec:23_05_2021.fb.bilchenko_evgenia.1.najdenysh}
\Purl{https://www.facebook.com/yevzhik/posts/3917556714946076}
\ifcmt
 author_begin
   author_id bilchenko_evgenia
 author_end
\fi


\ifcmt
  pic https://scontent-lga3-2.xx.fbcdn.net/v/t1.6435-9/189580000_3917556668279414_9034469803891359801_n.jpg?_nc_cat=110&ccb=1-3&_nc_sid=8bfeb9&_nc_ohc=NSovD-1Dy3gAX_BahOU&_nc_ht=scontent-lga3-2.xx&oh=7526014e68025e7121044860440fd7a3&oe=60CE3F70
\fi


Гуляя как-то раз в одном лесу,
Приросшему к низовьям девы Волги,
Я встретил смерть, забывшую косу,
И зайчиком напуганного волка.
Ещё я видел листья без корней
И корни без стволов, что в виде лямок
Держали землю, бегая по ней,
Но всё это меня не удивляло.
Ещё я видел небо, и его
Тащили вниз, как бабка с дедкой - репу,
Грибы без шляп, но с детской головой.
Короче говоря, я видел скрепы.
Ещё я встретил там богатыря,
Но он был мёртв, и, сколько ни будил я,
Всё было зря. Вечерняя заря
Над ним, звеня, справляла динь-динь-динь-ю.
И я сказал: "Позор тебе, очнись.
Тут - тишь да гладь, а там - война и поле".
Мой вопль в три эха подхватила высь,
И мать-земля открыла третий полюс.
Как трудно было - дохлого нести:
Родную тушу весом в два центнера,
Но есть у слабых только два пути:
Быть слабым или - честь, любовь и вера.
И я его, спаси Христос, донёс,
Средь поля положив, как клад бесценный.
И был обстрел. А после был допрос.
Но обошлось изгнанием со сцены.
К нему вернувшись, пальцы теребя
За нервы, заусенцы, нити, сваи,
Увидел я в бойце своём - себя.
И он был жив.
Точнее, я - живая.
23 мая 2021 г.
Илл.: Валерий Кандалинцев. Богатырка-поленица.
