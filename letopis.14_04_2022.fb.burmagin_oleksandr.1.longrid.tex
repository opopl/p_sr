% vim: keymap=russian-jcukenwin
%%beginhead 
 
%%file 14_04_2022.fb.burmagin_oleksandr.1.longrid
%%parent 14_04_2022
 
%%url https://www.facebook.com/oleksandr.burmagin/posts/5837782119570302
 
%%author_id burmagin_oleksandr
%%date 
 
%%tags 
%%title 28.02.2022 - 14.04.2022 - лонгрід.
 
%%endhead 
 
\subsection{28.02.2022 - 14.04.2022 - лонгрід.}
\label{sec:14_04_2022.fb.burmagin_oleksandr.1.longrid}
 
\Purl{https://www.facebook.com/oleksandr.burmagin/posts/5837782119570302}
\ifcmt
 author_begin
   author_id burmagin_oleksandr
 author_end
\fi

28.02.2022 - 14.04.2022 - лонгрід. 

Протягом цього часу мав честь бути у складі добровольчого батальону \enquote{Свобода}
112-ї бригади територіальної оборони Києва.  

...Після першого шоку 24-го лютого і евакуації жінок та дітей в безпечні місця,
ми з друзями вирішили швидко рухатись на Київ. Столиця, як і вся країна, була
під ударами і з декількох боків на неї сунула орда... 26-го числа дорога із
заходу в напрямку Києва була практично пустою. На під"ізді нас зустрів вже
підірваний міст у Стоянці та нашорошені, практично безлюдні вулиці міста. З
окружної було добре видно житловий будинок, в який буквально за пару годин
перед тим влучила ракета...   

\ii{14_04_2022.fb.burmagin_oleksandr.1.longrid.pic.1}

Пару днів ми шукали можливість долучитись до руху спротиву. Всюди величезні
черги з бажаючих взяти в руки зброю, запис даних і слова - чекайте дзвінка.
Зелене світло було тільки для тих, хто проходив військову службу, мав статус
учасника бойових дій. Після трьох спроб в різних пунктах мобілізації, потрапили
до набору і врешті-решт складу батальону \enquote{Свобода}. 

Словами звісно важко передати атмосферу в середовищі людей, в переважній
більшості вчора цивільних, які свідомо вирішили боронити від ворога Київ. Як і
складність для командирів організувати мотивованих, але малодосвідчених людей в
бойовий підрозділ. Проте, зараз можу впевнено сказати, що і вони і ми
впорались. З кожнем днем руки у всіх більш впевнено тримали зброю, більш
ефективно виконувались команди, покращувалась загальна злагодженність
підрозділу. 

Цьому сприяла не тільки атмосфера звісно, і хороші командири, а й величезна
команда людей, яким хочеться, принагідно гречно і красно подякувати.
Насамперед, волонтерам, які забезпечували і забезпечують підрозділ...насправді,
важко перелічити, чим не забезпечують. Інструкторам, які часто через
\enquote{@обтваюмать, синок}, але швидко змушували працювати тебе зі зброєю максимально
ефективно, а різні за кількістью групи бійців діяти, як єдиний організм. Також,
мабуть, назавжди в пам'яті залишаться тренінги з тактичної медицини (турнікет -
це не тільки метро, а й можливість однією рукою самому собі закрити важку
кровотечу при поранені) та лекціі військового психолога. Низькій уклін і всім,
хто налагоджував і забезпечував пристойні побутові умови батальону.

Особисто, не вважаю, що маю якійсь військові здобутки. Так, ми стояли на
Оболоні в очікуванні можливих проривів орків з боку Вишгорода, і мали задачу
прикривати за потреби ЗСУ, спецпідрозділи МВС. Так, ми декілька днів були під
мінометними обстрілами в Ірпіні, виконуючи задачі \enquote{2-ї лінії} і потім тримали
одну розв'язку на в'ізді в Гостомель. Через день стало зрозуміло, що орки вже
забігли за обрій. Загалом, рідко хто виходив зі строю, коли командири казали
\enquote{потрібно...}. Мабуть, це і є найбільше досягнення, наразі - бути готовим.

Переважна більшість побратимів далі рухатиметься на Схід. Хто лишається, -
будуть величезним резервом в разі можливої нової навали з півночі. Це вже не ті
цивільні, які в кінці лютого, початку березня прийшли записуватись до тербатів.
Я ж, із завтрашнього дня повертаюсь до інформаційного фронту, де також вистачає
задач і викликів. Буду радий бачити побратимів та колег на каву-чай
(Національна рада України з питань телебачення і радіомовлення, Прорізна, 2,
к.413).

Насамкінець, друзі - необхідність тримати стрій нікуди не поділась. Війна
триває, захисники та захисниці України далі потребують підтримки та участі
кожного. Перемога - внеску кожного, в тій сфері, де він може допомогти.
Згуртованість, єдність та підтримка один одного - запорука звільнення нашої
землі від окупантів. 

Київ - вистояв, вистоїть і Україна!

\ii{14_04_2022.fb.burmagin_oleksandr.1.longrid.cmt}
