% vim: keymap=russian-jcukenwin
%%beginhead 
 
%%file 27_11_2020.news.ru.lenta_ru.1.soc_seti
%%parent 27_11_2020
 
%%url https://lenta.ru/articles/2020/11/27/social_networks/?utm_source=RCM-F6FE&utm_campaign=rcmdc24bdc5
 
%%author 
%%author_id 
%%author_url 
 
%%tags 
%%title «Их возможности безграничны»
 
%%endhead 
 
\subsection{«Их возможности безграничны»}
\label{sec:27_11_2020.news.ru.lenta_ru.1.soc_seti}
\Purl{https://lenta.ru/articles/2020/11/27/social_networks/?utm_source=RCM-F6FE&utm_campaign=rcmdc24bdc5}

\index[rus]{Социальные сети!Аналитика}

\begin{center}
	\begingroup
		\bfseries\em\Large\color{orange}
		Социальные сети имеют огромную власть над всем миром. Чем это опасно?
	\endgroup
\end{center}

\ifcmt
pic https://icdn.lenta.ru/images/2020/11/26/17/20201126172453871/detail_e6ac06e482f14a49e6b1e9397a60b660.jpg
cpx Фото: Reuters
\fi

\begin{leftbar}
	\bfseries\color{blue}\large
Социальные сети давно перестали быть просто безобидными сервисами для обмена
сообщениями и впечатлениями — теперь это могущественные платформы,
которые формируют мировую повестку. Их мощь и влияние на пользователей
с каждым годом только усиливаются. Они способны привести к власти даже
президентов. Насколько реально вмешательство IT-гигантов в мировые
политические процессы и кто противостоит их власти — в материале
«Ленты.ру».
\end{leftbar}

\subsubsection{Теневой мир}

Об опасностях, поджидающих пользователей в социальных сетях, говорят
практически с момента их появления. И если ранее исследователи чаще высказывали
опасения по поводу депрессивного и деструктивного контента, плохо влияющего на
пользователей, то сейчас все чаще речь идет о фальсификации картины мира у
миллионов юзеров. Еще десять лет назад активист Эли Паризер придумал термин
«пузырь фильтров», который определил как ситуацию, в которой алгоритмы
IT-платформ искажают получаемую нами информацию в сети. По его словам, их
механизмы умышленно настроены так, чтобы пользователь потреблял больше того,
что ему нравится, и чаще встречал точку зрения, которую разделяет. Тогда
Паризер предположил, что это может привести к расколу в обществе. То есть
задуманный как демократическая площадка для всех, вне зависимости от доходов,
социального положения, национальности и гендера, интернет на самом деле лишь
усилит неравноправие и разногласия в обществе.

Журналисты и исследователи сходятся во мнении, что худшие прогнозы сбылись, но
пузыри фильтров — лишь компонент сложившейся в мире печальной ситуации, причины
и следствия которой сейчас все менее и менее ясны. Частично проблема
заключается в том, что теперь трудно понять, что случилось раньше: поляризация
общества или усугубление положения с помощью соцсетей. Многие предполагают, что
это самоусиливающаяся система, которая питается от собственных мощностей. «Где
я живу, кто мои друзья и какие медиа я использую, — все определяет то, что я
вижу, а затем формирует решения, которые я принимаю о том, какие медиа я
потребляю, где жить и с кем дружить», — заявил недавно Паризер.

Согласно недавнему исследованию, опубликованному в журнале Science, наиболее
ярко такой раскол виден на политической арене. Статья, написанная совместно
специалистами из шести различных областей, доказывает, что носители тех или
иных политических взглядов с высокой вероятностью будут ненавидеть тех, кто не
разделяет их точку зрения, и посчитают их «аморальными», «неприятными». В
результате, по словам ученых, возникает так называемое «политическое
сектантство», при котором юзеры не ищут точки соприкосновения, а сразу готовы
повесить ярлык на оппонента. Это может быть опасно для общества. Разные стороны
действуют как будто в разной реальности, оперируя различными наборами фактов и
реакциями на них. Социальные сети в этом случае лишь усугубляют ситуацию.
Особенные опасения исследователи высказывали относительно крупных политических
процессов, например, выборов. Издание BuzzFeed недавно опубликовало
исследование «индекса политического стресса», который продемонстрировал, что
нынешняя нестабильность соответствует периоду подготовки к гражданской войне.

Среди исследуемых компонентов использовались такие статистические данные, как
социальное неравенство и недоверие к правительству, объем госдолга, уменьшение
заработной платы и многое другое.

\ifcmt
pic https://icdn.lenta.ru/images/2020/11/26/17/20201126173349580/pic_ca5fc7cf1d0095c4bb4d360543878737.jpg
cpx Фото: Reuters
\fi

Усложняют понимание происходящего и непрозрачные механизмы модерации. Ранее
пользователи сети с большим интересом и почти сочувствием читали откровения
администраторов, очищающих условный Facebook от языка вражды, изображений
насилия и спорных видеороликов. Их описывали как уставших и измученных людей,
пытающихся справиться со своими эмоциями и стрессом с помощью любых доступных
средств и живущих в страхе перед бывшими коллегами. Все они подписали
соглашения о неразглашении с организацией Cognizant, в которых обязались не
обсуждать свою работу и даже признавать, что Facebook является клиентом
Cognizant. Но такая секретность нужна не столько для защиты сотрудников от
недовольных пользователей, — она защищает компанию от критики ее внутренних
правил и редактуры. Модераторы обязуются не обсуждать это ни с кем, включая
близких родственников. Под запретом даже разговоры об ощущениях, которые
приносит им их работа, — что лишь усиливает тревожность и ощущение одиночества.

Сейчас же модераторы соцсетей представляются большинству пользователей иначе:
юзеры осознали, насколько велики последствия принятых ими решений.
Адъюнкт-профессор Корнеллского университета и научный сотрудник в Microsoft
Research New England Тарлтон Гиллеспи в своем исследовании «Стражи интернета»
утверждает, что модерация контента — это способ существования социальных сетей,
а не лишь одна из функций.

В то же самое время администрации платформ всегда стремятся создать
впечатление, что площадки не имеют вовсе никаких ограничений и поддерживают
свободу самовыражения для каждого. Однако пользователь из любой точки мира,
включая известных людей — журналистов, политиков и общественных деятелей, может
неожиданно стать объектом модерации или даже бана. Заместитель директора
Института стратегических исследований и прогнозов РУДН, член Общественной
палаты России Никита Данюк считает, что такие алгоритмы включены в функционал
соцсетей неспроста: «Это целенаправленное создание из социальных сетей
инструмента манипулирования не только общественным мнением, но и по факту
общественным сознанием, потому что этот информационный кокон, в который
попадает пользователь, по сути формирует не только мнение в отношении того или
иного события, он формирует в целом картину мира и мировоззрение».

\ifcmt
pic https://icdn.lenta.ru/images/2020/11/26/17/20201126174755933/pic_b7138896c3d3718c82aadcc53852eb21.jpg
cpx Фото: Reuters
\fi

По словам Данюка, вмешательство социальных сетей выражается не только в
политической модерации и цензурировании, но и в несправедливом распределении
возможностей: для одних платформы предоставляют все ресурсы для распространения
контента, а других ждут цензура и предвзятая модерация. Это приводит к тому,
что некоторую информацию пользователь попросту не может найти. Он назвал такую
ситуацию актом политической цензуры, в рамках которой гиганты запрещают
рядовому пользователю по большому счету обращаться к альтернативным источникам
информации.

\subsubsection{Пробуждение силы}

Четыре года назад все СМИ мира заключили: избрание Дональда Трампа президентом
США гарантировали социальные сети. Их усиление началось еще при Бараке Обаме,
который перед выборами 2012 года использовал для общения с избирателями
имиджборд Reddit и даже отвечал на анонимные вопросы пользователей сети на
специальной площадке. Позднее, еще находясь на посту главы государства, Обама
выступил на тему расового неравенства в YouTube, и этот выбор площадки явно
повысил его рейтинг. Победа Трампа в 2016-м казалась парадоксом: он потратил на
предвыборную кампанию почти вдвое меньше, чем его соперница, — 429,5 миллиона
долларов против 897,7 миллиона у Клинтон. Самой крупной тратой у обоих
кандидатов тогда стало размещение рекламы. Но если у Клинтон в числе самых
крупных трат оказались транспортные расходы и зарплаты сотрудникам, то штаб
Трампа сосредоточил все расходы на онлайне. Трамп сделал ставку на социальные
сети. Позже один из руководителей Facebook заявил, что компания «несет
ответственность» за избрание Трампа президентом США. Тогда вице-президент по
дополненной и виртуальной реальности Эндрю Босуорт обратился к сотрудникам с
сообщением, что избранию Трампа способствовало не иностранное вмешательство, а
его хорошо спланированная кампания.

\begin{leftbar}
	\bfseries\centering
	{\Large\color{orange}429.5}\par
миллиона долларов\par
потратил Дональд Трамп на предвыборную кампанию\par
\end{leftbar}

Позднее в своих аккаунтах Трамп совмещал образ экзальтированного и резкого в
выражениях главы государства с продвижением себя как «обычного» человека. Но
вскоре он понял, что недооценил силы социальных сетей, которые обеспечили ему
новый виток в карьере: модераторы платформ без труда лишали его подписчиков, а
также помечали его посты как фейки и опасный контент. Особенно ярким и
известным стал многолетний конфликт Трампа с платформой Twitter. Администрация
сервиса неоднократно уличала президента США в распространении дезинформации,
одобрении насилия и манипулировании.

Глава Белого дома же утверждал, что стал жертвой предвзятого отношения к
консерваторам: по его словам, сообщения правых систематически ограничиваются, а
их последователи банятся. Он уверен, что это происходит из-за ненависти к
продвигаемой им политике: «Twitter ничего не делает со всей ложью и
пропагандой, распространяемой Китаем или радикальной лево-демократической
партией», — заключил Трамп. Позднее он высказал мнение, что 230-й раздел акта о
пристойности в сфере коммуникаций должен быть отменен. Эта часть документа,
принятого еще в 1996 году, когда-то дала развитие зарубежным IT-платформам и
интернету вообще, — благодаря ему площадки не несут ответственности за то, что
размещают на них пользователи. Однако президент Трамп посчитал, что гиганты
злоупотребляют юридической защитой со стороны государства, и вскоре издал указ,
согласно которому площадки могут выйти из-под защиты 230-го акта, если будут
принимать непрозрачные решения и цензурировать контент. Но пока соцсети
существуют при прежних условиях, обращаясь с публикуемым пользователями
контентом по собственному усмотрению.

По словам первого заместителя председателя комиссии по развитию информационного
общества, СМИ и массовых коммуникаций Общественной палаты России Александра
Малькевича, IT-гиганты уже стали важнейшим инструментом в политической борьбе.
Он считает, что в минувших выборах в США именно IT-гиганты сумели победить
Трампа. «Допустим, на выборах президента побеждает Джон Джонсон. И он не сможет
ни в Twitter, ни в Facebook написать о том, что он победил, потому что Twitter
и Facebook напишут: у нас другое мнение. А Роберт Робертсон, который проиграл
выборы, напишет в Twitter, что победил, и это опубликуют. А потом десятки и
сотни мировых СМИ разберут эту информацию, и все! И Роберт Робертсон в глазах
мировых элит, да и большей части страны станет легитимным лидером. Поэтому
влияние у них очень большое и серьезное. И то, что сейчас мы увидели в Америке,
— действительно это была демонстрация их возможностей, возможности
безграничны», — пояснил Малькевич. Общественный деятель уверен, что таким
образом IT-гиганты показали всему миру, что могут цензурировать и продвигать
любой контент, и российской стороне следует в кратчайшие сроки задуматься о
том, чтобы перерезать «пуповину цифровой зависимости» и развить собственные
сервисы и стартапы.

\ifcmt
pic https://icdn.lenta.ru/images/2020/11/26/17/20201126173710151/pic_91cb37ee1ec66c618a5f9dea01128c03.jpg
caption Александр Малькевич Фото: Дмитрий Духанин / «Коммерсантъ»
\fi

Никита Данюк тоже думает, что именно американские интернет-гиганты «назначили»
нового главу государства. «Они именно назначили Байдена своим президентом,
более того, способствовали его победе неоднократно, при этом те средства
коммуникации, которые были у Трампа, который использовал часто социальные сети
в общении со своими избирателями, потому что у него очень плохие были отношения
с традиционными СМИ, — так вот, эти инструменты коммуникации точно так же
мешали Дональду Трампу», — заключил он. При этом многие зарубежные СМИ
отметили, что социальные сети в этот раз сами попали в собственную ловушку.
Так, издание Business Insider утверждает, что Facebook и Twitter во время
выборов предпринимали безуспешные попытки «лопнуть» создаваемые годами
собственные пузыри фильтров.

\subsubsection{Нужно больше золота}

Растущая мощь и сила зарубежных IT-гигантов очень беспокоит американские
власти. И дело не только в их влиянии, хотя, к примеру, генеральный директор
компании Expensify (разработчика популярного приложения, отслеживающего
расходы) Дэвид Барретт в конце октября разослал 10 миллионам клиентов письма с
призывом проголосовать за Джо Байдена, «защитив демократию». Массивы
информации, сосредоточенные в руках нескольких интернет-корпораций, огромны. В
октябре 2020 года подкомитет по антимонопольной политике юридического комитета
палаты представителей Конгресса США представил 449-страничный доклад о
действиях монополистов на IT-рынке: компаний Amazon, Apple, Facebook и Google.
Расследование злоупотреблений компаний велось почти полтора года, для этого
авторы изучили 1,2 миллиона документов и сообщений, провели почти 250 интервью
с участниками рынка и выслушали 38 свидетелей. Конгрессмены пришли к выводам,
что для укрепления в своей сфере корпорации злоупотребляли собственной властью
и неправомерно подавляли конкурентов.

Google в докладе называется монополистом в сфере поиска и онлайн-рекламы
(показатели — 90 и 41 процент соответственно), Apple обвиняют в дискриминации
других разработчиков приложений, а Amazon — в установлении гегемонии на рынке
розничных продаж. Особое внимание уделено Facebook, который, по словам
специалистов, неправомерно завладел монополией на рынке соцсетей и рекламы. Ему
принадлежит 59 процентов рынка по количеству пользователей и 20 процентов
мирового рынка онлайн-рекламы. Немало вопросов у юристов вызывает сделка по
приобретению платформой сервиса Instagram в 2012 году.

\ifcmt
pic https://icdn.lenta.ru/images/2020/11/26/17/20201126173958457/pic_309a3d352db5a7a16f90b7a116936630.jpg
cpx Фото: Dado Ruvic / Reuters
\fi

Изучив сопроводительные документы, они пришли к выводу, что приложение было
перспективным проектом, не нуждающимся ни в какой поддержке со стороны
Facebook. Также в докладе упоминается об известной «записке Каннингема» —
служебном документе, который в 2018 году подготовил для внутреннего
использования старший аналитик Facebook. Из доклада Тома Каннингема следует,
что пользователи все чаще отдают предпочтение другим продуктам, принадлежащим
корпорации, а именно Instagram и WhatsApp. Он высказал опасение, что
популярность основной платформы Facebook такими темпами будет стремительно
снижаться, и по достижении определенной точки былую мощь вернуть уже не
удастся. Бывшие сотрудники корпорации утверждают, что тогда топ-менеджмент
принял решение иначе позиционировать эти продукты, чтобы избежать внутренней
конкуренции. Однако администрация Facebook все отрицает — в противном случае им
придется признать неправомерный сговор.

Сейчас Демократическая партия США предлагает разветвить структуры IT-гигантов и
разделить их на несколько меньших компаний, а также ограничить их возможности.
Предлагается также изменить антимонопольное законодательство — к примеру,
считать антиконкурентными любые поглощения или слияния, совершаемые крупными
корпорациями. К чему это приведет, пока неизвестно: антимонопольные слушания с
участием представителей этих компаний, прошедшие летом 2020-го в Конгрессе США,
остались лишь громким событием.

Согласно недавней публикации журнала MIT Technology Review престижного
Массачусетского технологического института, крупные онлайн-платформы все еще
непрозрачны в своих решениях, что оставляет их уязвимыми для заявлений о
цензуре, а также дает возможности для дезинформации. Автор статьи, американский
социолог и сотрудница Гарвардского университета Джоан Донован, утверждает, что
крупные социальные сети должны отнестись к этому вопросу так же серьезно, как и
к своему желанию получить прибыль.

\begin{leftbar}
	\em\large
	Нам нужно больше прозрачности. Дезинформация касается не только фактов; это
				вопрос о том, кто должен говорить о фактах. Решения о честной модерации
				контента являются ключом к общественной ответственности\par
				\textbf{Джоан Донован}
\end{leftbar}

По словам исследовательницы, когда дезинформация распространяется из цифрового
мира в физический, она может перенаправить общественные ресурсы и поставить под
угрозу безопасность людей.

\subsubsection{Пути неисповедимы}

Пока американская сторона пытается разобраться с антимонопольным
законодательством, в России раздумывают над иным инструментарием для
воздействия на зарубежные интернет-платформы. В ноябре 2020-го в Госдуму был
внесен законопроект, предполагающий наказание для интернет-платформ за цензуру.

Это касается ограничения сайтам доступа к информации, опубликованной в
российских СМИ. Авторы проекта утверждают, что некоторые ресурсы необоснованно
ограничивают доступ россиян к таким данным. В случае, если законопроект
утвердят, к платформам-нарушителям можно будет применять экономические санкции
в виде штрафов, а также частичного или полного ограничения доступа. В
Роскомнадзоре это решение объяснили приоритетом российских законов над
правилами платформ.

Член Общественной палаты Никита Данюк считает, что крупные социальные сети и
онлайн-платформы будут соблюдать российские законы и не станут заниматься
цензурированием, если в их отношении будут действовать определенные меры.
Например, если им будут ограничивать трафик или запрещать заключать
государственные контракты. Обращенные к корпорациям штрафы должны выражаться не
в конкретной сумме, а в проценте от прибыли, которую они приобретают в
российском информационном пространстве от продаж и рекламирования всевозможных
товаров и услуг. «Эти инструменты работают, например, во Франции, Австралии,
Германии, и социальные сети периодически вынуждены сталкиваться с такими
ситуациями и, собственно, идти навстречу, для того чтобы не потерять очень
привлекательный рынок, чтобы не потерять прибыль, и это действительно
работает», — заявил он. По словам Данюка, такая ситуация стала возможной из-за
того, что деятельность интернет-платформ никак не регулировалась даже в США — в
стране, которая получает максимум выгоды и пользы от этих компаний.

\ifcmt
pic https://icdn.lenta.ru/images/2020/11/26/17/20201126173516842/pic_93c1c21a327d86e5f1a72b4206aead5c.jpg
cpx Фото: Dado Ruvic / Reuters
\fi

Александр Малькевич высказывается более резко, отмечая, что соцсети стремятся
«захватить максимальное количество квадратных метров на информационной поляне».
Зампред комиссии по развитию информационного общества, СМИ и массовых
коммуникаций Общественной палаты считает, что у IT-гигантов очень много
инструментов влияния и вмешательства: «Главный вывод — мы должны признать, что
это реальная проблема, это реальная угроза, дальше это будет только
разрастаться. Потому что, войдя во вкус, зачем им останавливаться? Они
понимают, что они могут действовать безнаказанно — полностью снести аккаунт
человека, заблокировать неугодный пост; при этом угодный пост продвигать всеми
возможными способами и средствами. И значит — только альтернатива! Мы должны
идти к строительству собственной интернет-экосистемы, во-первых. А во-вторых,
на нашем рынке должны работать те площадки, платформы и IT- компании, которые
соблюдают наши законы и уважительно относятся к цифровым правам граждан
России».

Эксперты также видят проблему в том, что иногда судебные решения, которые
выносятся в России, и требования от отечественного регулятора не исполняются по
целому ряду причин. Например, у них нет представительств на территории России:
по факту как юридическое лицо они в целом имеют возможность игнорировать эти
требования и не вступать в диалог. Чем закончится это противостояние с
иностранными IT-гигантами, пока неизвестно, но, как показывает практика, сразу
несколько крупных государств планируют заставить корпорации, годами
устанавливавшие свои правила в сети, подумать над своим поведением.
