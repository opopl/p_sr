% vim: keymap=russian-jcukenwin
%%beginhead 
 
%%file poetry.rus.dnr.vladislav_rusanov.ni_ellina_ni_iudeja
%%parent poetry.rus.dnr.vladislav_rusanov
 
%%url 
%%author 
%%tags 
%%title 
 
%%endhead 

\subsubsection{Нет ни эллина, ни иудея...}
Владислав Русанов

Нет ни эллина, ни иудея,
а дороги к храму отмощены
черепами. Воскликнешь «Где я?!»
пред ощеренной пастью площади.
Свет фальшфайера. Вонью тленною
растекается ненависть-сукровица.
С размозжёнными вдрызг коленами,
ты опять не поспеешь к заутренней.
Перерезанным воешь горлом ты,
хрипло булькаешь чёрными сгустками,
нет дороги тропою торною,
только стёжкой кривой да узкою.
Ныне судьбы чертой означены
и осколком навылет ранены.
То ли тёткою, то ли мачехой
крест пропившая мне Украина.
Лебезит кнутом, лупит пряником,
дланью щедрой дарует отметины.
То ли пасынки, то ли племянники,
заслоняемся тыном плетенным
из упрямства да из отчаяния,
круг рисуем со странными знаками.
Пропоют ли мне величальную
или в яму кинут собакою?

2015 
