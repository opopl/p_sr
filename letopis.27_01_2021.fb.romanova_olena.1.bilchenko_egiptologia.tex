% vim: keymap=russian-jcukenwin
%%beginhead 
 
%%file 27_01_2021.fb.romanova_olena.1.bilchenko_egiptologia
%%parent 27_01_2021
 
%%url https://www.facebook.com/olena.romanova.39/posts/2878976852374726
 
%%author Романова, Олена
%%author_id romanova_olena
%%author_url 
 
%%tags __jan_2021.bilchenko,bilchenko_evgenia,egipet,egiptologia,kultura,nauka,obschestvo,skandal,ukraina
%%title Тут всі кинулись обговорювати поведінку пані Євгенії Більченко. Я ж хочу прохи сказати про наукову якість її викладання
 
%%endhead 
 
\subsection{Тут всі кинулись обговорювати поведінку пані Євгенії Більченко. Я ж хочу прохи сказати про наукову якість її викладання}
\label{sec:27_01_2021.fb.romanova_olena.1.bilchenko_egiptologia}
\Purl{https://www.facebook.com/olena.romanova.39/posts/2878976852374726}
\ifcmt
 author_begin
   author_id romanova_olena
 author_end
\fi

Шановні колеги,

Update - враховуючи резонанс, який мало перепощення мого посту і подальші
дискусії - вирішила відкрити цей пост для всіх. Це -мої попередні враження,
сподіваюсь, що далі я напишу, чи ми напишемо щось офіційно. Як історик, не
вважаю за можливе у самому пості вже щось змінювати, крім граматичних помилок:

Тут всі кинулись обговорювати поведінку пані Євгенії Більченко.

Я ж хочу прохи сказати про наукову якість її викладання. Мені тут колеги
показали. що у неї лекції із культури Стародавнього Єгипту. Я подивилась, не
все, бо нерви не витримують.

Подивившись це все фаховим зором єгиптолога, з усією повагою до всіх
дискутантів, маю сказати, що в тих лекціях суцільна псевдонаука та її художня
творчість та невігластво. Ну по-перше, маса помилок в фактах та перекручень,
вони вивалюються таким суцільним шаром. що все розплутати практично не можливо.

Наприклад, вона всерйоз розпочинає лекцію із гумільовшини і розповідає, що
ландшафт визначає культурні архетипи і всіляке таке.

Далі, там маса \enquote{відкриттів} - наприклад про африкансьі чорні племена, яки
називалися гіксоси, і які захопили Єгипет і створили 25 династію :). І викладає
вичитану у Струве чи когось із його епігонів концепцію про Ехнатона,
революціонера, який .бореться із жахливими експлуататорами-жерцями за
пригноблені народні маси, і порівнює його реформу із реформою Петра І.

Чи що стародавні єгиптяни вірили. що на \enquote{полях дуат} росла кукурудза в ріст
людини, про мармур, яким були обкладені піраміди, чи про хірургиню \enquote{Песечет},
яка робила операції на серці, та написала хірургічний трактат. За кількістю
перлів та маячні на хвилину часу- пані Більченко переважає в декілька разів
Бебика. все просто не описати,  там і  і всі псевднаукові треши, в дусі
піраміда символізує жіноче начало, бо в основу її покладено трикутник,..., і
еманації,  і космізм, і всілякий інший символізм, який пані береться
витлумачувати, і вся та маячня, як присутня в роботах педагогів, та доморощених
культурологів, і маса радянських стереотипів.  східна деспотія, і ще маса
всього, про все не можливо розповісти в коментарях.

І справа не тільки в тому, що предмет Стародавній Єгипет та його культура - це
ніби-то специфічна тема, і не всі її добре знають.

Пані викладачка демонструє своє повне невігластво в історії, і  фактах та
процесах,і  в загальних тенденціях. Вона не здатна зрозуміти, що стародавні
суспільства і стародавні культури не такі, як сучасні, тому вона постійно
модернізує все, і у неї Ехнатон перетворюється на народного преЗедента, який
бореться із непосильними тарифами та гнітом чиновників над народом, за відміну
ліцензування на поховання, а \enquote{єгипетська молодь} прагне змін та розвитку, тощо.

Це не кажучи вже про таке, що між окремими подіями (Скажемо кінець епохи 18
династії (бл. 1305 р. до н.е ) та утвердженням 26 Саїської династії 664 р. до
н.е), які вона видає за причини та наслідки -  сотні років, тощо, плутання із
хронологією, послідовністю подій іт.д.

Варто вказати, що і із загальною ерудицією там не все добре, схоже, не лише
уроки історії у школі та у вузі прогуляла,бо не було б стільки ляпів.

Пані плутає поняття,  нагромаджує такі конструкції, які не мають сенсу, або
плутається із категоріями обєктів, християньский  аскетизм у неї походить від
процесу муміфікації.

Також з усього цього видно невміння критично мислити, невміння відрізняти
наукову інформацію від ненаукової та від псевдонаукової, як і нерозуміння того,
що вона читає, бо вона перекручує все, що можна, і єгиптяни у неї винайшли
шахи, і не лише кукурудзу привезли із Америки, та мармуром піраміди покрили, і
т. Вона черпає інформацію із \enquote{інтернету} не спиймаючи критично її.
Факт-чекінгу при підготовці до лекцій вона не робить, інакше б вона уникнула
маси тих дурних \enquote{фактів}, які вона створила своєю фантазією.

Пані все прочитане із різних джерел та незрозіміле їй, валить до однієї купи,
та дає скрізь \enquote{власне бачення} яке є не чи іншим, як дурнею та псевдонаукою.

Псевдонаукової маячні як  і різних \enquote{художніх висловів}  про Стародавній Єгипет
- в інтернеті повно, але проблема тут - це читається у виші як курс для
студентів. тут ми підходимо до двох важливих питань:

\begin{itemize}
\item 1) Чи місце у виші псевдонауці?
\item 2) Відповідальності викладача за достовірність інформації, яку він подає студентам.

Як на мою думку, псевдонауковий треш треба вичищати із вишів, і із школи теж, а
такж я вважаю, що викладач відповідальний за достовірність інформації, яку він
дає на уроці чи на лекції, і процес викладання це не можливість безконтрольного
навязування студентам власних фантазій під соусом "власної думки", "авторського
бачення", а все-таки навчання їх певним занням та виклад систематизованої
інформації, якою сучасна світова наука володіє про ту чи іншу тему, навчання,
яке має давати фахівець. Не всі можуть бути фахівцями із Стародавнього Єгипту,
чи Шумеру, чи Аккаду, чи Індії, Китаю, хетів, тощо, але для цього існує як маса
саме наукової літератури, яку треба читати замість перегляду відеороликів з
ю-тюба, як і спеціалісти, в яких можна отримати консультацію, з різних
дициплін. Не розуміє хтось щось про арабську культуру, - а треба прочитати
лекцію - то має можливість звернутися до арабістів, і т.д.

\end{itemize}

Поки що  я цей пост закрила, і пишу для друзів, може потім щось напишу більш
офіційне. Проблема тут не в політичних поглядах, як на мою думка, а у
некомпетентності  і у псвдонауці, яку пані викладає.  Якщо це ми засуджували,
коли йшлось про Кириленко, то чому маємо терпіти у пані Більченко?

Посилання на лекції і ще тут. була б вдячна, коли б інші історики також
подивились інші лекції та зробили висновок про їх рівень і щось би написали
\url{https://www.youtube.com/watch?v=u8Sb0FUv1lc}
