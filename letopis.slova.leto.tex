% vim: keymap=russian-jcukenwin
%%beginhead 
 
%%file slova.leto
%%parent slova
 
%%url 
 
%%author 
%%author_id 
%%author_url 
 
%%tags 
%%title 
 
%%endhead 
\chapter{Лето}

%%%cit
%%%cit_pic
%%%cit_text
Вроде бы простенький вопрос, но неподготовленного человека он может поставить в
тупик. Мало того, что \emph{летоисчисление} было совсем другим, так еще и с цифрами
проблема - ни арабских, ни римских в древних отечественных летописях не
встретишь. Ну и как же тогда сообщить читателю требуемую дату?
%%%cit_comment
%%%cit_title
\citTitle{Как писали даты в Древней Руси?}, 
Русичи, zen.yandex.ru, 07.06.2021
%%%endcit
