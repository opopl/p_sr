% vim: keymap=russian-jcukenwin
%%beginhead 
 
%%file 04_02_2022.fb.chemeris_vladimir.1.den_nacionalnoj_ganjby
%%parent 04_02_2022
 
%%url https://www.facebook.com/thomas.severyn/posts/3065301463685072
 
%%author_id chemeris_vladimir
%%date 
 
%%tags futbol,nacia,pozor,rossia,sport,ukraina
%%title День національної ганьби
 
%%endhead 
 
\subsection{День національної ганьби}
\label{sec:04_02_2022.fb.chemeris_vladimir.1.den_nacionalnoj_ganjby}
 
\Purl{https://www.facebook.com/thomas.severyn/posts/3065301463685072}
\ifcmt
 author_begin
   author_id chemeris_vladimir
 author_end
\fi

День національної ганьби

Не тому, що збірна України із футзалу програла. Грала збірна пасіонарно і
достойно. Але перемагає сильніший.

Це – спорт. 

Але матч Україна – Росія перетворили зі спорту на пропагандистський гній.
Стараннями «патріотів», посадовців і правих політиків.

Завдяки їм і масованій кампанії у змі «переможемо москалів!», матч із футзалу,
який раніше не викликав великого інтересу у публіки, дивилося багато українців.

І вони побачили: нашим спортсменам заборонили тиснути руки своїм спортивним
суперникам. Під час виконання гімну РФ трансляцію перервали.

Ми пам’ятаємо часи, коли поведінка радянських спортсменів регламентувалася
чиновниками від спорту, але такого, аби вільним людям забороняли
фотографуватися із спортсменами із США та говорити англійською мовою, не було.

Далі у трансляції неслися спортивні вигуки «фанатів»: «...Смерть ворогам...
уйло...  вбий... трупи...».

З трибун футзалу в Амстердамі на нас дивилося перекошене від злоби обличчя, яке
чомусь обгорнулося українським прапором.

«Ви чуєте ці вигуки? Їх у прямій трансляції чують глядачі з багатьох країн
світу! Ось яка підтримка у нашої збірної!», - радісно повідомляв коментатор
телеканалу UA Перший.

Коментатор, який говорив не «збірна Росії», а «команда у червоній формі». Не
називаючи при цьому прізвищ російських гравців.

Коментатор телеканалу, який фінансується з українського бюджету.

Глядачі з багатьох країн, слава богу, не зрозуміли те, що волали на трибунах.
Бо інакше вони б могли перенести враження від людей, які втратили людську
подобу, на всіх українців.

Але весь цей перформенс бачили багато українців.

Українців, які ці перекошені обличчя постійно бачать на вулицях українських
міст. 

Попри безперервну мантру влади і «соросят», що «фашизму у нас немає».

Українців, які у абсолютній більшості хочуть миру і свободи говорити те, що
вони думають.

І ми, українці, знаємо, що українці не такі.

І нав’язати нам таку поведінку, навіть через телевізор, не вдасться.

Але є ще одна причина називати сьогоднішній день днем національної ганьби.

YouTube заблокував акаунти Першого Незалежного та UkrLive.

«Патріоти» і «соросята» кинулись радіти з приводу цензури і знищення свободи
слова в Україні, уподобившись отарі на трибунах амстердамського стадіону...

\begin{itemize} % {
\iusr{Андрей Недзельницкий}
И сегодня переплавили памятник Суворову

\iusr{Марк Чепурной}
А Зе подобається.

\iusr{Олег Бондарь}
Еще не вечер ...
\end{itemize} % }
