%%beginhead 
 
%%file 13_11_2022.fb.galushko_denis.1.v_iskova_kampan_ya_2
%%parent 13_11_2022
 
%%url https://www.facebook.com/HalushkoDenis/posts/pfbid0nc3NNHUZi6o8aBCTy9ktG9pAB6mJkCw73dtiQs8uV8AMk3zcpG26Co8bce8jZj7il
 
%%author_id galushko_denis
%%date 13_11_2022
 
%%tags 
%%title Військова кампанія 22 року завершується поверненням Херсону
 
%%endhead 

\subsection{Військова кампанія 22 року завершується поверненням Херсону}
\label{sec:13_11_2022.fb.galushko_denis.1.v_iskova_kampan_ya_2}

\Purl{https://www.facebook.com/HalushkoDenis/posts/pfbid0nc3NNHUZi6o8aBCTy9ktG9pAB6mJkCw73dtiQs8uV8AMk3zcpG26Co8bce8jZj7il}
\ifcmt
 author_begin
   author_id galushko_denis
 author_end
\fi

Військова кампанія 22 року завершується поверненням Херсону.

Місто, громадяни які чинили опір, герої, зустріли своїх визволителів та без
світла і води намагаються їх нагодувати і пригостити, це неймовірні почуття
єдності, в ці хвилини, народу та війська. 

Піски були тим фортпостом об які ворог зламав всі свої плани і марні сподівання
на хоч який успіх в цій війні, точка неповернення і загальний вклад в початок
контрнаступу нашого війська цього літа. Піски не захопили, їх стерли в
цегляний пил та груди шифера і черепиці. 

Бахмут, місто побратим Херсона він дав змогу, стримати ворога і нанести йому ще
більш ніщивного удару. 

Могила тисячі, чмобиків, кадирівців і вагнерівців.  Безглузда різанина,
м’ясорубка факти якої згодом відкриються для московії, від якої навіть їхні
матері, що «наражають» і загальна біомаса населення будуть у шоці. 

Іде зима, вона буде безжальною для нашого ворога, дасть змогу нам стримати,
перегрупуватися, зібратися перед рішучими діями наступної весняної кампанії
23-го. 

Ще більше деморалізувати і кінцево придушити будь які спроби ворога. залишити
собі хоч найменший шмат української землі. 

Я вірю, що вона закінчиться для московитів, так само як і зима під Києвом, та
ми будемо бачити на дорозі їх замерші тулуби, що їх кинуть «своих не бросаем»
тікаючи від нашого наступу.  

В мене немає жодних ілюзій з приводу того, що ми пам’ятамимо якою ціною
далася нам ця перемога, але у воїна своя задача і нам буде потрібна ще більше
вашої підтримки цієї зими. 

Це буде боротьба за наш бойовий дух та боєздатність і в цьому ми розраховуємо
не тільки на власну вмотивованість, бажання досягти мети, яка одна на всіх, а
і ваші слова і листи  з дому, які зігрівають набагато краще ніж полум’я з
буржуйки.

Підтримуйте своїх, один одного, зима буде і тільки від нас залежить якою
саме.

Допомога нашому підрозділу.

Картка дружини в першому коменті.

Дякую всім за підтримку, разом до перемоги!
