% vim: keymap=russian-jcukenwin
%%beginhead 
 
%%file 01_09_2021.fb.murajev_jevgenij.1.mogilnik_westingauz
%%parent 01_09_2021
 
%%url https://www.facebook.com/yevgeniy.murayev/posts/3960476020722878
 
%%author_id murajev_jevgenij
%%date 
 
%%tags atomic_energy,memorandum,ukraina,usa
%%title Ядерный Могильник - Westinghouse Electric - Меморандум
 
%%endhead 
 
\subsection{Ядерный Могильник - Westinghouse Electric - Меморандум}
\label{sec:01_09_2021.fb.murajev_jevgenij.1.mogilnik_westingauz}
 
\Purl{https://www.facebook.com/yevgeniy.murayev/posts/3960476020722878}
\ifcmt
 author_begin
   author_id murajev_jevgenij
 author_end
\fi

Меморандум о 30 миллиардах долларов на 5 новых атомных энергоблоков, которые
гипотетически собирается построить в Украине американская компания-банкрот
Westinghouse Electric. Это первые «перемога-новости» из Вашингтона.

Напомню, что это тот самый Westinghouse Electric, который поставляет в Украину
свое ядерное топливо для наших энергоблоков и проводит на АЭС небезопасные
эксперименты по адаптации этого топлива. И да, это именно та компания, из-за
которой под Киевом, в Чернобыльской зоне у нас достраивают гигантский ядерный
могильник, чтобы утилизировать эти отработанные ядерные отходы. Потому что
вывозить наши отработанные сборки в Сибирь, как выяснилось, по идейным , а
скорее коррупционным соображениям нельзя.

\ifcmt
  pic https://scontent-frt3-2.xx.fbcdn.net/v/t1.6435-9/240920171_3960475544056259_8895938771856826240_n.jpg?_nc_cat=103&_nc_rgb565=1&ccb=1-5&_nc_sid=8bfeb9&_nc_ohc=Cuw9e7usIS8AX8qE4Mw&_nc_ht=scontent-frt3-2.xx&oh=52b09544c27ff3611228002997a4b495&oe=61594DE3
  width 0.4
\fi

Но и это еще не все новости. Давайте прямо говорить, что принятие Зеленским
этого аборигенского подарка от Вашингтона, это политическая взятка хозяину
перед встречей, построение ещё и энергетической зависимости от метрополии, а
так же ещё одна «Большая стройка» для лиц в орбите ОП.

Но и тут надо отметить, что план Офиса масштабней. Аккуратно перед вылетом из
Киева в Верховной раде появился проект закона, который позволит провести
корпоратизацию НАЭК «Энергоатом», который быстро одобрили в Минэнерго. А
значит, процесс сдачи нашей атомной энергетики идет полным ходом. Дальше
осталось дело техники -  протянуть законопроект через монобольшинство в
парламенте. А если, а так скорее всего и будет, инвестиций не будет, то просто
создать акционерное общество, склепать в «Энергоатоме» очередной наблюдательный
совет и рулить энергобезопасностью и крупнейшим по финансовым потокам
предприятием, контролирующим более 50\% рынка вырабатываемой электроэнергии,
отправляя многомиллионные премии родителям в США.

Об успехах этих наблюдателей уже можно судить по деятельности «Нафтогаза» и
«Укрзализници». Многое, я думаю, могут рассказать и в Ощадбанке. У каждой из
этих госкомпаний, за номинальным руководителем с украинской фамилией есть турок
Шевки Аджунер (на Укрзализнице) или британка Клер Споттисвуд (в Нафтогазе),
Коболев, отправивший многомиллионную премию маме в Штаты, без которых ни одно
решение не будет принято.

Теперь первый вопрос на логику – как думаете, эта группировка иностранных
граждан работает в наших с вами интересах?

И второй вопрос, насколько, после этой сделки Зеленского с Westinghouse,
умножится внешний долг Украины?

\ii{01_09_2021.fb.murajev_jevgenij.1.mogilnik_westingauz.cmt}
