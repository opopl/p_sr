% vim: keymap=russian-jcukenwin
%%beginhead 
 
%%file 04_10_2020.fb.medvedeva_olesja.1.s_dnem_uchitelja
%%parent 04_10_2020
 
%%url https://www.facebook.com/olesia.medvedieva/posts/1446575842200200
 
%%author_id medvedeva_olesja
%%date 
 
%%tags den.uchitelja,detstvo,foto,medvedeva_olesja,semja,ukraina
%%title С Днём учителя! Спасибо, что продолжаете, не благодаря, а вопреки!
 
%%endhead 
 
\subsection{С Днём учителя! Спасибо, что продолжаете, не благодаря, а вопреки!}
\label{sec:04_10_2020.fb.medvedeva_olesja.1.s_dnem_uchitelja}
 
\Purl{https://www.facebook.com/olesia.medvedieva/posts/1446575842200200}
\ifcmt
 author_begin
   author_id medvedeva_olesja
 author_end
\fi

Воспитанная учителями @igg{fbicon.heart.red}

Мои мама и папа - учителя русского, французского языков и литературы. Закончили
горловский ИНЯЗ, где и познакомились. 

Дальше была работа в школе. 90е, веселуха!

\ifcmt
  ig https://scontent-frt3-2.xx.fbcdn.net/v/t1.6435-9/120855958_1446567795534338_3353731750659944308_n.jpg?_nc_cat=103&ccb=1-5&_nc_sid=8bfeb9&_nc_ohc=tyJ2H8HKXaUAX-erqhk&_nc_ht=scontent-frt3-2.xx&oh=62163a25dbebc74269561619b7198fee&oe=61C593D1
  @width 0.4
  %@wrap \parpic[r]
  @wrap \InsertBoxR{0}
\fi

Папа разрабатывал новую методику преподавания русского, развивался в науке, но
спустя какое-то время ушёл из профессии и занялся строительством. Я верю, что
он все равно вернётся к преподаванию, особенно в условиях жести с языками у нас
в стране.

Мама была зам директора по воспитательной работе 11 школы, что на Комсомольце.
Как сейчас помню, мы красили краской пол в августе регулярно, с тех пор не
очень люблю что-то красить. 

В нашем районе было много неблагополучных детей, мама их воспитывала, вместо их
родителей. Очень расстраивалась, когда органы опеки забирали кого-то в дет дом
или интернат. Часто дети ночевали у нас дома, если их родители бухали. Всегда у
нас был полон дом учеников: готовили какие-то концерты, спектакли, сценки, шили
костюмы, пели. В общем что только не делали. Мама очень любила детей. Мне
иногда казалось, что их она любит больше меня, но повзрослев я поняла, что это
было не так. Когда школу решили снести, она ушла работать в ВУЗ по тому же
направлению, но так и не смогла простить закрытие школы городским властям. 

Мама и папа своим примером научили меня человеколюбию, желанию помогать
нуждающимся, уметь работать ради высшей цели, ценить учительский труд, тянуться
к знаниям и быть собой.

Я так вдохновилась их примером, что тоже пошла в педагогический, но Вселенная
выбрала для меня иной путь. 

С Днём учителя! Спасибо, что продолжаете, не благодаря, а вопреки!
