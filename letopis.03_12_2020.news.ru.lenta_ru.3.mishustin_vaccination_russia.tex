% vim: keymap=russian-jcukenwin
%%beginhead 
 
%%file 03_12_2020.news.ru.lenta_ru.3.mishustin_vaccination_russia
%%parent 03_12_2020
 
%%url https://lenta.ru/news/2020/12/03/prosba/
 
%%author 
%%author_id 
%%author_url 
 
%%tags 
%%title Мишустин обратился к губернаторам из-за массовой вакцинации
 
%%endhead 
 
\subsection{Мишустин обратился к губернаторам из-за массовой вакцинации}
\label{sec:03_12_2020.news.ru.lenta_ru.3.mishustin_vaccination_russia}
\Purl{https://lenta.ru/news/2020/12/03/prosba/}

\ifcmt
pic https://icdn.lenta.ru/images/2020/12/03/17/20201203171751354/pic_ceae488bd38edd98cf1f004971088c25.jpg
caption Михаил Мишустин Фото: Дмитрий Астахов / РИА Новости
\fi

\index[rus]{Коронавирус!Россия!Вакцинация, Мишустин, 3.12.2020}
\index[names.rus]{Мишустин, Михаил!Премьер-министр России}

Премьер-министр России Михаил Мишустин обратился к губернаторам с просьбой
лично контролировать ход массовой вакцинации и обеспечить ее удобство и
безопасность. Об этом сообщает РИА Новости в четверг, 3 декабря.

«Чем лучше пройдет вакцинация, тем быстрее сможем общаться без ограничений», —
сказал он.

По его словам, регионы уже доложили о высокой готовности к началу масштабной
вакцинации от коронавируса.

Накануне президент Владимир Путин поручил на следующей неделе начать в России
масштабную вакцинацию против COVID–19 с использованием российского препарата
«Спутник V». Первыми вакцинацию должны пройти врачи и учителя.

В Москве электронная запись на прививку будет доступна для работников
образования, здравоохранения, а также городских социальных служб уже с 4
декабря. Работа прививочных пунктов начнется с 5 декабря. Мэр столицы Сергей
Собянин заверил, что персонал прошел необходимое обучение.

