% vim: keymap=russian-jcukenwin
%%beginhead 
 
%%file 06_12_2019.stz.news.ua.mrpl_city.1.olga_suleina_budjmo_dobrishymy
%%parent 06_12_2019
 
%%url https://mrpl.city/blogs/view/olga-suleina-budmo-dobrishimi-i-uvazhnishimi-odin-do-odnogo
 
%%author_id demidko_olga.mariupol,news.ua.mrpl_city
%%date 
 
%%tags 
%%title Ольга Сулеіна: "Будьмо добрішими і уважнішими один до одного!"
 
%%endhead 
 
\subsection{Ольга Сулеіна: \enquote{Будьмо добрішими і уважнішими один до одного!}}
\label{sec:06_12_2019.stz.news.ua.mrpl_city.1.olga_suleina_budjmo_dobrishymy}
 
\Purl{https://mrpl.city/blogs/view/olga-suleina-budmo-dobrishimi-i-uvazhnishimi-odin-do-odnogo}
\ifcmt
 author_begin
   author_id demidko_olga.mariupol,news.ua.mrpl_city
 author_end
\fi

Наступна наша героїня є щирою і неймовірно чутливою людиною. Її доброта не знає
меж, але сама вона не дуже любить розповідати про себе та свою діяльність.
Сором'язлива і водночас дуже енергійна, справжня і самобутня маріупольчанка
Ольга Сулеіна наголошує, що вона немає девізу і  не любить зайвого пафосу, але
у неї є справа, за яку болить серце і душа...

\ii{06_12_2019.stz.news.ua.mrpl_city.1.olga_suleina_budjmo_dobrishymy.pic.1}

Народилася Оля в Маріуполі. Батьки працювали у ВАТ \enquote{Азовмаш}. Батько –
заступником директора, мама – конструктором. Дівчинка росла у дружній родині
разом з молодшою сестрою. З дитинства Оля любила багато читати, тому обожнювала
багату бібліотеку бабусі й дідуся. Росла активною і працьовитою дівчиною. У
школі брала активну участь в організації різних заходів. Після закінчення школи
вступила до Донецького університету на історичний факультет. Отримала
спеціальність юриста. Однак за спеціальністю не довелося працювати. 20 років
виконувала обов'яз\hyp{}ки секретаря директора в \enquote{Азовмаш}. У 1992 році вийшла
заміж. Разом з чоловіком вона вже цілих 27 років. Їм вдалося створити міцну і
щасливу сім'ю. Підтримка сім'ї жінку надихає найбільше.

\ii{06_12_2019.stz.news.ua.mrpl_city.1.olga_suleina_budjmo_dobrishymy.pic.2}

Під час війни маріупольчанка стала волонтером. Працюючи у Фонді допомоги
пораненим і сім'ям загиблих, намагалася надавати необхідну допомогу
переселенцям та пораненим в АТО. З 2015 року почала працювати в БФ \enquote{Карітас
Маріуполь}. Тут Ольга займалась проєктами в буферній зоні (Мар'янка, Гнутово,
Талаківка). Це видача палива, продуктових наборів, аптечок, обігрівачів,
буржуйок. Водночас на базі Фонду допомоги пораненим і сім'ям загиблих жінка
разом з Олександром Магдалицем відкриває пункт тимчасової перетримки собак, що
постраждали в АТО – \enquote{Чіп і Дейл}. Там було дуже багато собак (близько 200).
Собак привозили заляканими. За ними ніхто не доглядав, їх ніхто не годував,
вони постійно шукали місце, де сховатися від небезпеки. Саме тому Оля з іншими
волонтерами робили все можливе, щоб врятувати наляканих тварин. Всім їм був
потрібен особливий догляд і справа не тільки в тому, що вони потребували
довгого лікування та відновлення... Треба було показати, що їх дійсно люблять. І
тоді вони відчували себе по-іншому і починали поступово реабілітуватися. Ольга
намагалася слідкувати за подальшою долею своїх вихованців. Зокрема, нові
власники повинні були надавати звіти. Часто волонтери приходили в гості до
своїх улюбленців, щоб подивитися як вони себе почувають. Водночас жінка
продовжувала надавати необхідну допомогу переселенцям.

Нашій героїні – завдяки великому серцю і  неймовірному бажанню допомагати –
вдалося не тільки врятувати велику кількість тварин, але й подарувати багатьом
маріупольцям нових вірних і надійних друзів. Ольга була директоркою пункту
перетримки \enquote{Чіп і Дейл} до 2017 року, коли всі собаки знайшли нових господарів.
Найбільше в цій благородній і важливій справі допомагав лобіст і громадський
діяч Олександр Магдалиць. Сьогодні у Олі 8 кішок і 6 собак. Вона й наразі
готова весь вільний час допомагати всім безпритульними чотиририлапим.

\ii{06_12_2019.stz.news.ua.mrpl_city.1.olga_suleina_budjmo_dobrishymy.pic.3}

Останні 3 роки волонтерка весь час присвячує одній з головних справ свого життя
– соціальній їдальні (на Банній, 4). Завдяки спонсорській  підтримці
\enquote{Renovabis} вдавалося прогодувати понад 2000 осіб. У липні 2019 року
закінчилася фінансова  підтримка. Вдається годувати безкоштовно в день 10
чоловік за свої кошти. Годують людей без постійного місця проживання і тих, хто
опинився в скрутних життєвих обставинах. Серед них є люди з інвалідністю,
багатодітні сім'ї, одинокі пенсіонери, які опинилися на межі виживання.

\ii{06_12_2019.stz.news.ua.mrpl_city.1.olga_suleina_budjmo_dobrishymy.pic.4}

Спеціалісти \enquote{Карітас Маріуполь} розуміють, який позитивний вплив має така
підтримка для людей, адже завдяки можливості отримувати гарячі обіди вони
можуть заощадити кошти на ліки, і взагалі підтримувати на рівні своє соціальне
життя.

Маріупольчанка наголошує, що потреба в їжі завжди залишалася основною, тому
необхідно зробити все можливе, щоб соціальна їдальня отримала знову фінансову
підтримку, особливо напередодні новорічних свят.

Дочка нашої героїні, Анастасія, також є кураторкою багатьох соціальних
проєктів, активно займається кар'єрою. Ольга любить гуляти з сім'єю у Міському
саду. Її тішить, що найближчі люди її розуміють і підтримують.

\textbf{Улюблені фільми:} \enquote{Москва сльозам не вірить} (1979 рік), \enquote{Любов і голуби} (1984 рік).

\textbf{Улюблені книги:} твори Б. Акуніна і Ю. Нікітіна.

\textbf{Порада маріупольцям:} \emph{\enquote{Будьте добрішими і уважнішими один до одного! Вмійте допомогти іншим, коли це необхідно!}}
