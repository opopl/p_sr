% vim: keymap=russian-jcukenwin
%%beginhead 
 
%%file 22_01_2022.tg.tajny_talejrana.1.napadenie_rossii
%%parent 22_01_2022
 
%%url https://t.me/tainytaleyrana/40
 
%%author_id tg.tajny_talejrana
%%date 
 
%%tags infvojna,isteria,napadenie,rossia,ukraina,vtorzhenie
%%title Нападение России на Украину
 
%%endhead 
\subsection{Нападение России на Украину}
\label{sec:22_01_2022.tg.tajny_talejrana.1.napadenie_rossii}
 
\Purl{https://t.me/tainytaleyrana/40}
\ifcmt
 author_begin
   author_id tg.tajny_talejrana
 author_end
\fi

Нападение России на Украину

Ещё с конца осени не умолкают разговоры про нападение РФ на Украину по всем
фронтам, сотни тысяч солдат уже на границе, караул, в общем. Напомню, что
осенний виток паники первыми раздуло известное американское издание Politico
(\href{https://www.bbc.com/russian/features-59139673}{
СМИ и соцсети пишут о новом наращивании российского военного присутствия у границ Украины. Есть основания?, %
bbc.com, 02.11.2021%
}), сообщив о скоплении ВС РФ близ
Ельни (300 км до Украины). И, что было очевидным, вторжения не произошло. Вроде
бы, всё устаканилось, но американцы не успокаиваются и дальше нагнетают

Для начала я просто хочу вам напомнить, сколько раз объявляли про возможное
нападение 

В 2015 году Турчинов предупреждал,
(https://www.pravda.com.ua/rus/news/2015/06/11/7070980/) что \enquote{Россия готовится
к вторжению и развязыванию агрессии...}. Тогда же и действующий президент
Порошенко говорил о войне
(\href{https://hromadske.radio/ru/news/2015/07/17/poroshenko-po-informacii-razvedki-vtorzhenie-v-ukrainu-budet}{%

}),
ссылаясь на разведку

В 2017 глава МИД Климкин пугал возможным наступлением
(\href{https://www.5.ua/polityka/klimkin-rozpoviv-chomu-narazi-provedennia-zustrichi-u-normandskomu-formati-nemozhlyve-136977.html}{%
Клімкін: Війська РФ на кордоні з Україною – демонстрація готовності до повномасштабного вторгнення, %
www.5.ua, 27.01.2017%
}):

\enquote{Російські війська на кордоні з Україною – це демонстрація готовності до
повномасштабного російського вторгнення в Україну}; 

Washington Post писал
(\href{https://zn.ua/UKRAINE/putin-ne-otkazhetsya-ot-voyny-protiv-ukrainy-iz-za-bolshogo-kolichestva-pogibshih-rossiyan-wp-256547_.html}{%
Путин не откажется от войны против Украины из-за большого количества погибших россиян - WP, %
zn.ua, 08.08.2017%
}),
что Путин не откажется от войны против Украины

В 2018 ПМ Польши Моравецкий прогнозировал нападение России

(\href{https://www.dw.com/ru/%D0%BF%D0%BE%D0%BB%D1%8C%D1%88%D0%B0-%D1%81%D0%BF%D1%80%D0%BE%D0%B3%D0%BD%D0%BE%D0%B7%D0%B8%D1%80%D0%BE%D0%B2%D0%B0%D0%BB%D0%B0-%D0%BA%D0%BE%D0%B3%D0%B4%D0%B0-%D1%80%D0%BE%D1%81%D1%81%D0%B8%D1%8F-%D0%BC%D0%BE%D0%B6%D0%B5%D1%82-%D0%BD%D0%B0%D0%BF%D0%B0%D1%81%D1%82%D1%8C-%D0%BD%D0%B0-%D0%BA%D0%B8%D0%B5%D0%B2/a-46351227}{%
Польша спрогнозировала, когда Россия может напасть на Киев, dw.com, 19.11.2018%
}
)
сразу после строительства Северного Потока-2. Бывший посол США
(\href{https://apostrophe.ua/news/society/accidents/2018-12-06/v-ssha-zagovorili-o-novoy-atake-rossii-protiv-ukrainyi/148030}{%
В США заговорили о новой атаке России против Украины, apostrophe.ua, 06.09.2018%
})
в Украине говорил о новой атаке РФ, \enquote{Украине и международному сообществу надо
быть готовыми}

В 2019 представитель ГУР сообщал

(\href{https://ukraine.segodnya.ua/ukraine/rossiya-otrabotaet-vtorzhenie-v-ukrainu-ukrainskaya-razvedka-1326694.html}{%
Россия отработает вторжение в Украину – разведка, ukraine.segodnya.ua, 05.09.2019%
})

об отработке Россией вторжения в Украину, \enquote{завершается формирование трех
российских дивизий возле нашей границы}

В 2020 американские эксперты через Радио Свободу заявляли
(https://ru.krymr.com/a/analitiki-ssha-o-vozmozhnom-napadenii-rossii-na-ukrainu/30711677.html)о
возможном нападении РФ: «Необходимо следить за событиями». Тогда же сообщалось,
что из-за дефицита воды Россия готова наступать
(https://www.dsnews.ua/politics/bitva-za-vodu-pochemu-i-kak-rossiya-deystvitelno-mozhet-napast-07072020101500),
публиковались даже карты

В 2021 мы помним, как весной готовились к обороне. До сих пор одна только
болтовня

(https://news.liga.net/politics/news/rossiya-gotovit-provokatsiyu-dlya-vtorjeniya-svoih-regulyarnyh-voysk-na-donbass-razvedka)


Заметьте,
я использовал ссылки только украинских или грантоедских СМИ, тут не скажешь
про \enquote{российскую пропаганду}. И таких новостей собрать за предыдущие годы можно
больше, вам, думаю, хватит для понимания

Теперь по сегодняшнему дню

Россия нападать НЕ будет, так как смысла сейчас это делать нет. Против неё
начнется на самом деле экономическая блокада. Мы знаем, что самая великая
победа та, которая одерживается без боя, вот пока что за столом переговоров с
США и гарантиями безопасности Москва хочет выпроводить НАТО из Украины. Россию
интересуют не территории буквально, а сфера влияния (как и всех больших гос-в в
XXI веке). Хотела бы присоединить Восточную Украину - сделала бы намного раньше

И немного про нагнетателей. Президент Зеленский правильно сделал, что решил
успокоить украинцев по поводу войны и верно сказал про инвесторов, которые не
решатся вкладываться, читая подобные заявления. Но сразу же после этого идут
сообщения о Харькове и ожесточенных боях + чиновники на разных уровнях и с
разными компетенциями только усложняют дело противоречивыми заявлениями -
видно, что в госаппарате полный бардак без единой позиции

Всю эту кашу заваривают именно американские СМИ, даже Зе мягко, но обратил на
это внимание. Естественно, думают они не об Украине, а только помогают повышать
ставки перед торгами с РФ...

P.S. Имею нескольких знакомых, которые трубят на ухо о вторжении РФ. Предложил
спор на \$100. Зараза, не хотят терять деньги и сами же понимают это!)
