% vim: keymap=russian-jcukenwin
%%beginhead 
 
%%file 22_04_2022.fb.solovjov_mikita.harkov.demsokyra.1.hronika.cmt
%%parent 22_04_2022.fb.solovjov_mikita.harkov.demsokyra.1.hronika
 
%%url 
 
%%author_id 
%%date 
 
%%tags 
%%title 
 
%%endhead 
\zzSecCmt

\begin{itemize} % {
\iusr{Людмила Колесник}
Да, все переживаем за город @igg{fbicon.heart.red} ️@igg{fbicon.hands.pray} 

\iusr{Виталий Буняев}

Есть такое по настроениям. Но я не с Донбассом это связываю. А с увеличением
болезненности, что ли ... обстрелов. Были у людей иррациональные надежды, что
потеплеет - и наступит облегчение. Как бы некоторое время надежды
оправдывались, но внезапно все вернулось почти на уровень первых дней. По итогу
недоумение и ощущение неизвестной бесконечности...

\iusr{Люда Глива}
Всі розуміємо що нашим дуже важко
Ще важче буде відвоювати вже втрачене.

\iusr{Igor Sakhno}
Держитесь

\iusr{Igor Shanyuk}
А откуда они обстреливали Коротич???

\iusr{Юрий Белотицкий}
Вроде была ракета. Если \enquote{смерч}, то у них дальность до 120км.

\iusr{Юлия Соколова}
Дійсно, зараз на Донбасі вирішується наше майбутнє....

\iusr{Olga Limarova}
сорри, у Терехова пишут про газ в 20 м/э.
\url{https://www.city.kharkov.ua/ru/news/u-kharkovi-vidnovili-gazopostochannya-meshkantsyam-sche-20-bagatopoverkhivok-50486.html}

\begin{itemize} % {
\iusr{Микита Соловйов}
\textbf{Olga Limarova} Значит официальные данные противоречат друг другу. Посмотрю завтра, есть ли карта и перечень подключенных.

\iusr{Olga Limarova}
\textbf{Микита Соловйов} спасибо
\end{itemize} % }

\iusr{Yulia Denisyuk}
Да, есть неопределенность и напряженность. Чувствую сама и вижу это по своим близким

\iusr{Марина Юрьева}

Люди !!!! может что знаете про Казачью Лопань????? там бомбили сегодня,
СМИ ничего не пишут, почему-то

\begin{itemize} % {
\iusr{Oleksii Voznosymenko}
\textbf{Марина Юрьева} так там орки, давно, с самого начала еще

\iusr{Марина Юрьева}

Да. окупированая территория. Но родственники там. сегодня бомбили. И негде узнать как там люди?

\iusr{Микита Соловйов}
\textbf{Марина Юрьева} 

А откуда идея, что о таких вещах вообще могут и должны писать в СМИ? Тем более,
о взрывах на оккупированной территории.

\end{itemize} % }

\iusr{Anna Salnikova}
обстрелы Коротича меня поразили. Не слышала, чтобы так далеко по Полтавскому Шляху попадало раньше...

\begin{itemize} % {
\iusr{Микита Соловйов}

Дальнобойным они могут бить на выбор всю харьковскую область. Причем даже не
Калибрами, а Смерчами в максимально дальнобойном варианте.

\iusr{Elena Medvedeva}
\textbf{Анна Сальникова} живу в селе за 20 км от ХГ. Направление на НАШ украинский К. Три дня назад накрыли наше село смерчами
\end{itemize} % }

\iusr{Anna Salnikova}

Второй рассказ про харьков в ленте))) Хроники жизни в метро.
\url{https://www.facebook.com/victoria.ivlevayorke/posts/10160021815850987}

\iusr{Микита Соловйов}
\textbf{Анна Сальникова} Почитал. Как по мне, живая иллюстрация к моему тезису, что из метро людей переселять надо. Это уже за гранью имхо.

\iusr{Ganna Oksenenko}
Сжался в ожидании не только Харьков. Вся Украина...и та, что в Европе тоже. Все хотят домой и Перемоги

\iusr{Alex Moroz}

Кто-нибудь знает, не разрушено ли здание Института радиофизики и электроники на
Академика Проскуры на Жуках? Я когда-то давно там работал.

\iusr{Yulia Denisyuk}
Никита, спасибо за ваши посты. Всегда читаю с интересом, хотя и не всегда со всем согласна

\iusr{Зоя Лебедєва}

Дякую за новини про Харків. Пам'ятаю почуття харків'ян у 14-15 роках, місто
жило, працювало, а напруга була у людей, коли серед спілкування виривалось
тихе, - страшно. І відповідь пошепки, - так, страшно... І все, далі просто
діла, погода, начебто звичайне життя. А зараз під обстрілами, руйнуванням,
смертю, звичайно напруга велика. Харків'яни сильні, продовжують жити,
працювати.

\iusr{Rosti Kałańczuk}
Скажите, в каком состоянии ДЖД?

\iusr{Ольга Шимко}
Галя, тримайтеся. Ми теж в такій ситуації. Будемо мужніми і вірити в премогу

\iusr{Ирина Глущенко}
Не кажется

\iusr{Яна Дерега}

Ні, не здається. Життя в очікуванні, невизначеності. В мене старенькі батьки в
Куп'янську, з 7 квітня з містом та районом немає ніякого зв'язку. Вже декілька
днів у місті та районі немає світла та води. Окупанти розповсюджують інфу, що
це Україна відключила((

\iusr{Попков Максим}

Вчора літуни прорвалися, пішуть з трох одного заземліли. Мінус круглій рінок,
влучіли в велике віділеня НП. Меньше світла ночью в домівках.

\end{itemize} % }
