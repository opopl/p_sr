% vim: keymap=russian-jcukenwin
%%beginhead 
 
%%file 17_12_2022.stz.news.ua.donbas24.1.kyiv_zaxid_pidtrym_polonena_azovstal
%%parent 17_12_2022
 
%%url https://donbas24.news/news/u-kijevi-vidbuvsya-zaxid-na-pidtrimku-viiskovopolonenoyi-z-azovstali-foto-video
 
%%author_id demidko_olga.mariupol,news.ua.donbas24
%%date 
 
%%tags 
%%title У Києві відбувся захід на підтримку військовополоненої з Азовсталі
 
%%endhead 
 
\subsection{У Києві відбувся захід на підтримку військовополоненої з Азовсталі}
\label{sec:17_12_2022.stz.news.ua.donbas24.1.kyiv_zaxid_pidtrym_polonena_azovstal}
 
\Purl{https://donbas24.news/news/u-kijevi-vidbuvsya-zaxid-na-pidtrimku-viiskovopolonenoyi-z-azovstali-foto-video}
\ifcmt
 author_begin
   author_id demidko_olga.mariupol,news.ua.donbas24
 author_end
\fi

\ii{17_12_2022.stz.news.ua.donbas24.1.kyiv_zaxid_pidtrym_polonena_azovstal.pic.front}
\begin{center}
  \em\color{blue}\bfseries\Large
17 грудня у Києві провели літературні читання віршів маріупольської поетеси Валерії Суботіної
(Карпиленко), яка наразі перебуває у полоні
\end{center}

17 грудня у Будинку кіно в Києві відбулися літературні читання віршів
маріупольської поетеси та письменниці, доброволиці полку \enquote{Азов}, \href{https://donbas24.news/news/bili-ta-morili-golodom-u-poloni-zaxisniki-mariupolya-vtratili-po-60-70-kg}{\emph{української
військовополоненої}}.%
\footnote{Били та морили голодом — у полоні захисники Маріуполя втратили по 60−70 кг, Еліна Прокопчук, donbas24.news, 28.09.2022, \par\url{https://donbas24.news/news/bili-ta-morili-golodom-u-poloni-zaxisniki-mariupolya-vtratili-po-60-70-kg}}

Літературні читання \emph{\enquote{Ковток повітря для книги Валерії
Суботіної (Карпиленко) \enquote{Квіти і зброя}: читаємо з тобою}} організували рідні і
друзі Валерії. Захід присвятили перевиданню її поетичної збірки, яку
письменниця у 2021 році презентувала в Маріуполі та Києві та підтримці мужньої
маріупольчанки.

\textbf{Читайте також:} \href{https://donbas24.news/news/u-kijevi-zyavilasya-vulicya-geroyiv-polku-azov}{\emph{У Києві з’явилася вулиця Героїв полку \enquote{Азов}}}%
\footnote{У Києві з'явилася вулиця Героїв полку \enquote{Азов}, Тетяна Веремєєва, donbas24.news, 26.10.2022, \par%
\url{https://donbas24.news/news/u-kijevi-zyavilasya-vulicya-geroyiv-polku-azov}%
}

\subsubsection{Історія Валерії}

\emph{\textbf{Валерія Суботіна (Карпиленко)}} на позивний \emph{\textbf{Nava}} — маріупольська поетеса,
кандидатка наук із соціальних комунікацій, у минулому випускниця й викладачка
Маріупольського державного університету, яка разом з \enquote{Азовом} захищала рідний
Маріуполь. Історію кохання Валерії Карпиленко та прикордонника \emph{\textbf{Андрія Суботіна
(Бороди)}} записав побратим, азовець \emph{\textbf{Дмитро Козацький (Орест)}}. Пара
познайомилась, коли Валерія працювала в пресслужбі Донецького прикордонного
загону. В облозі на \enquote{Азовсталі} вирішили одружитись. Датою обрали 5 травня, бо
це день народження полку \enquote{Азов}. Проте 7 травня Андрій загинув. Валерія вже 7
місяців перебуває в полоні.

\begin{leftbar}
\emph{\enquote{Ти три дні був моїм законним чоловіком. І цілу вічність ти — моє кохання. Мій
рідний, мій турботливий, мій мужній... Ти був і є найкращий. Мені
лишилося твоє прізвище, твоя любляча родина і спогади про щасливий час
разом}}, — написала Валерія у травні.
\end{leftbar}

\textbf{Читайте також:} \href{https://donbas24.news/news/iz-rosiiskogo-polonu-povernulisya-64-ukrayinci-ta-odin-gromadyanin-ssa-foto}{\emph{Із російського полону повернулися 64 українця та один громадянин США: список звільнених}}%
\footnote{Із російського полону повернулися 64 українця та один громадянин США: список звільнених, Тетяна Веремєєва, donbas24.news, 14.12.2022, \par%
\url{https://donbas24.news/news/iz-rosiiskogo-polonu-povernulisya-64-ukrayinci-ta-odin-gromadyanin-ssa-foto}%
}

\ii{17_12_2022.stz.news.ua.donbas24.1.kyiv_zaxid_pidtrym_polonena_azovstal.pic.1}

\subsubsection{Захід, присвячений Валерії}

Напередодні Дня Святого Миколая, як уявний подарунок і сердечна підтримка для
Валерії, в Будинку кіно відбулися літературні читання \enquote{Ковток повітря для книги
Валерії Суботіної (Карпиленко) \enquote{Квіти і зброя}: читаємо з тобою}. Мати
загиблого чоловіка Валерії, Людмила Суботіна, та її друзі організували
невеликий, камерний, але дуже душевний творчий захід, присвячений перевиданню
її поетичної збірки, яку письменниця в 2021 році презентувала в Маріуполі й
Києві. На жаль, у багатьох маріупольців-переселенців книжки першого видання
збірки з підписами авторки були втрачені в зруйнованих будинках. Ведучим заходу
був актор та режисер \textbf{\emph{Станіслав Сукненко.}}

\ii{17_12_2022.stz.news.ua.donbas24.1.kyiv_zaxid_pidtrym_polonena_azovstal.pic.2}

Серед запрошених гостей заходу кобзар, лірник, бандурист, народний артист
України, лауреат Національного конкурсу кобзарського мистецтва ім. Г. Китастого
\emph{\textbf{Тарас Компаніченко}}, який прочитав вірші Валерії та заспівав душевні пісні на
підтримку\par\noindent маріупольчанки та її побратимів.

\begin{leftbar}
\emph{\enquote{У цьому горнилі зараз формується справжня еліта. Ці люди стали в оборону
національної ідентичності. Кожен з них це потужний творець. Всі вони
реалізовані люди. В обороні стоїть справжня національна еліта, яка
ніколи не буде мімікрувати, зраджувати та кудись перебігати. Бо вони
інакшими бути не можуть. Саме такій еліті володарювати на цій землі}}, —
підкреслив Тарас Компаніченко.
\end{leftbar}

\textbf{Читайте також:} \href{https://donbas24.news/news/rosiyani-planuyut-sudilishhe-nad-zaxisnikami-azovstali}{\emph{Росіяни планують судилище над захисниками Азовсталі}}%
\footnote{Росіяни планують судилище над захисниками Азовсталі, Тетяна Веремєєва, donbas24.news, 02.12.2022, \par%
\url{https://donbas24.news/news/rosiyani-planuyut-sudilishhe-nad-zaxisnikami-azovstali}%
}

\href{https://archive.org/details/video.17_12_2022.donbas_novyny.taras_kompanichenko}{%
Відео: Тарас Компаніченко, Донбас Новини, 17.12.2022}%
\footnote{\url{https://www.youtube.com/watch?v=H4vXpbnqe-Q}} %
\footnote{\url{https://archive.org/details/video.17_12_2022.donbas_novyny.taras_kompanichenko}}

\ifcmt
  ig https://i2.paste.pics/PS2FN.png?trs=1142e84a8812893e619f828af22a1d084584f26ffb97dd2bb11c85495ee994c5
  @wrap center
  @width 0.9
\fi

Вірші Валерії читали громадські активісти Маріуполя, представники міської
влади, журналісти, колеги і друзі Валерії.

\begin{leftbar}
\emph{\enquote{Я знаю Леру близько 20 років. Спочатку, як студентку, потім, як колегу,
адже ми разом викладали у Маріупольському державному університеті. Ми завжди
підтримували стосунки, тому що ми дуже споріднені. Вона говорила, що я для неї
взірець, але насправді це вона для мене справжній взірець для наслідування}}, —
наголосила \emph{\textbf{Вікторія Рогожина.}}
\end{leftbar}

\begin{leftbar}
\emph{\enquote{Знаючи Леру, можу точно сказати, що вона навіть в найгіршій ситуації може
зберігати оптимізм та віру. Головне, щоб вона повернулася з полону не тільки
живою, а щоб вона залишилася собою. Змогла зберегти свою здатність — бачити
навколо добро, яка б біда не відбувалася}}, — зауважив заступник міського голови
\emph{\textbf{Денис Кочубей.}}
\end{leftbar}

В окремому відео можна було почути вірш Валерії у виконанні відомого
українського актора \emph{\textbf{Дмитра Лінартовича.}}

\textbf{Читайте також:} \href{https://donbas24.news/news/mi-rozumili-shho-slyaxu-nazad-nemaje-prikordonnik-pro-boyi-za-mariupol-poranennya-ta-polon-video}{\emph{\enquote{Ми розуміли, що шляху назад немає} — прикордонник про бої за Маріуполь, поранення та полон}}%
\footnote{\enquote{Ми розуміли, що шляху назад немає} — прикордонник про бої за Маріуполь, поранення та полон, Наталія Сорокіна, donbas24.news, 11.11.2022, \par%
\url{https://donbas24.news/news/mi-rozumili-shho-slyaxu-nazad-nemaje-prikordonnik-pro-boyi-za-mariupol-poranennya-ta-polon-video}%
}

\ifcmt
  ig https://i2.paste.pics/PS2LL.png?trs=1142e84a8812893e619f828af22a1d084584f26ffb97dd2bb11c85495ee994c5
  @wrap center
  @width 0.9
\fi

\href{https://archive.org/details/video.17_12_2022.donbas_novyny.virsh_valerii_karpylenko_aktor_dmytro_linartovych}{%
Відео: Вірш Валерії Карпиленко у прочитанні актора Дмитра Лінартовича, Донбас Новини, 17.12.2022}%
\footnote{\url{https://archive.org/details/video.17_12_2022.donbas_novyny.virsh_valerii_karpylenko_aktor_dmytro_linartovych}} %
\footnote{\url{https://www.youtube.com/watch?v=-6UbChyxulE}}

Також представили відеозвернення Сергія Жадана, який теж зачитав один з віршів
маріупольчанки. З відеозверненням виступив і Дмитро Козацький (Орест). Чоловік
поділився спогадами про службу з Валерією, наголосив на її неповторності та
виключному таланті і висловив сподівання, що вже незабаром Nava буде вільною.
Ведучий Станіслав Сукненко прочитав зворушливого листа доктора політичних наук,
професора, радника ректора Маріупольського державного університету \emph{\textbf{Костянтина
Балабанова}} до Валерії Карпиленко.

\href{https://archive.org/details/video.17_12_2022.donbas_novyny.kvity_i_zbroja_zaxid_prysv_valerii_karpylenko}{%
Відео: Квіти і зброя, захід присвячений Валерії Карпиленко, Донбас Новини, 17.12.2022}%
\footnote{\url{https://archive.org/details/video.17_12_2022.donbas_novyny.kvity_i_zbroja_zaxid_prysv_valerii_karpylenko}} %
\footnote{\url{https://www.youtube.com/watch?v=8mVaD2cmNHo}}

\ifcmt
  ig https://i2.paste.pics/PS2UU.png?trs=1142e84a8812893e619f828af22a1d084584f26ffb97dd2bb11c85495ee994c5
  @wrap center
  @width 0.9
\fi

\subsubsection{Про нову збірку віршів}

Зі слів самої Валерії, збірка \enquote{Квіти і зброя} написана нею під час служби в
добробаті, в полку \enquote{Азов} (2015−2016 рр.), і пізніше, з 2017 по 2020 р., в
Донецькому прикордонному загоні. Весь цей час Валерія записувала історії бійців
і писала вірші про них. Кожний вірш — це спогад про хлопців і дівчат, які були
там, на Сході, всі ці 8 років російсько-української війни, і для яких є
настільки дорогою і небайдужою Україна, що вони готові віддати за неї
найдорожче — свої життя.

\ii{17_12_2022.stz.news.ua.donbas24.1.kyiv_zaxid_pidtrym_polonena_azovstal.pic.3}

\textbf{Читайте також:} \href{https://donbas24.news/news/na-yizu-davali-30-sekund-zaxisnik-mariupolya-mixailo-dianov-rozpoviv-pro-znushhannya-v-poloni}{\emph{\enquote{На їжу давали 30 секунд} — захисник Маріуполя Михайло Діанов розповів про знущання в полоні}}%
\footnote{\enquote{На їжу давали 30 секунд} — захисник Маріуполя Михайло Діанов розповів про знущання в полоні, Еліна Прокопчук, donbas24.news, 02.10.2022, \par%
\url{https://donbas24.news/news/na-yizu-davali-30-sekund-zaxisnik-mariupolya-mixailo-dianov-rozpoviv-pro-znushhannya-v-poloni}%
}

Збірка складається з чотирьох розділів: \textbf{\enquote{Пріоритети}},
\textbf{\enquote{Вона}}, \textbf{\enquote{Замовляння}} та
\textbf{\enquote{Душа}}. Кожний із них має свою атмосферу і свою історію
написання. \enquote{Пріоритети} — це про власний вибір захищати країну на
фронті та спогади про побратимів, в основі кожного вірша реальні історії
бійців. Розділ \enquote{Вона} — душа ліричної героїні, її бунтівна натура,
сумніви й переконання, розчарування і любов.  \enquote{Замовляння} не тільки за
назвою й тематикою, а й за художнім ритмом та образам — абсолютно
фольклорно-відьомське поетичне \enquote{зілля}. Останній розділ поетичної
збірки прочиняє для читачів віконце у світ душі ліричної героїні, відкриває її
пантеїстичний світогляд.

\ii{17_12_2022.stz.news.ua.donbas24.1.kyiv_zaxid_pidtrym_polonena_azovstal.pic.4}
\ii{17_12_2022.stz.news.ua.donbas24.1.kyiv_zaxid_pidtrym_polonena_azovstal.pic.5}
\ii{17_12_2022.stz.news.ua.donbas24.1.kyiv_zaxid_pidtrym_polonena_azovstal.pic.6}

Раніше Донбас24 розповідав, що Гурт Tvorchi присвятив \href{https://donbas24.news/news/gurt-tvorchi-prisvyatili-pisnyu-zaxisnikam-azovstali-video}{\emph{пісню захисникам \enquote{Азовсталі}}}.%
\footnote{Гурт Tvorchi присвятив пісню захисникам \enquote{Азовсталі}, Тетяна Веремєєва, donbas24.news, 15.12.2022, \par\url{https://donbas24.news/news/gurt-tvorchi-prisvyatili-pisnyu-zaxisnikam-azovstali-video}}

Ще більше новин та найактуальніша інформація про Донецьку та Луганську області
в нашому телеграм-каналі Донбас24.

ФОТО: з відкритих джерел та з особистого архіву Ольги Демідко


\ii{insert.author.demidko_olga}
%\ii{17_12_2022.stz.news.ua.donbas24.1.kyiv_zaxid_pidtrym_polonena_azovstal.txt}
