% vim: keymap=russian-jcukenwin
%%beginhead 
 
%%file 06_09_2021.fb.semesjuk_ivan.1.kiev_pozhar_kostel_arhitektura.cmt
%%parent 06_09_2021.fb.semesjuk_ivan.1.kiev_pozhar_kostel_arhitektura
 
%%url 
 
%%author_id 
%%date 
 
%%tags 
%%title 
 
%%endhead 
\subsubsection{Коментарі}

\begin{itemize} % {
%%%fbauth
%%%fbauth_name
\iusr{Марина Марченко}
%%%fbauth_url
%%%fbauth_place
%%%fbauth_id
%%%fbauth_front
%%%fbauth_desc
%%%fbauth_www
%%%fbauth_pic
%%%fbauth_pic portrait
%%%fbauth_pic background
%%%fbauth_pic other
%%%fbauth_tags
%%%fbauth_pubs
%%%endfbauth
 

\obeycr
Ох, як смачно ви все це описали.
Ще зранку хотіла відгукнутися.
А саме про Полісся.
Довелося сидіти там прямо у глибинці, де навіть автобусів нема, пів мясяця з італійцями.
Ну точно Терра Інкогніта - така незаймана природа - мішаний красивенний ліс, озера(нехай у кар'єрах), люди такі класні, щедрі і зараз дзвонять запрошують по гриби.
Із для мене дивного було почуто про нєшуточну війну між житомирськими та рівненьськими за ліс.
Наші(житомирькі) не пускають рівненьських до себе, бо кажуть, що ті знищили все у себе із-за бурштину, тепер до них лізуть збирати ягоди та гриби. І там війна серйозна.
Як вони люблять ліс, милуються їм.
Але є проблеми - алкоголізм дійсно зашкалює, вплодь до бабусь і через одного рак шлунку, Чернобиль.
\restorecr

%%%fbauth
%%%fbauth_name
\iusr{Дмитро Руденко}
%%%fbauth_url
%%%fbauth_place
%%%fbauth_id
%%%fbauth_front
%%%fbauth_desc
%%%fbauth_www
%%%fbauth_pic
%%%fbauth_pic portrait
%%%fbauth_pic background
%%%fbauth_pic other
%%%fbauth_tags
%%%fbauth_pubs
%%%endfbauth
 

Ця пожежа, однозначно є символічною, тому найбільше хотілося б щоб ми правильно
зрозуміли цей символізм і відповідно відреагували, бо вже так набридло
виправляти наслідки власної байдужості до сакральних речей.


%%%fbauth
%%%fbauth_name
\iusr{Олекса Манн}
%%%fbauth_url
%%%fbauth_place
%%%fbauth_id
%%%fbauth_front
%%%fbauth_desc
%%%fbauth_www
%%%fbauth_pic
%%%fbauth_pic portrait
%%%fbauth_pic background
%%%fbauth_pic other
%%%fbauth_tags
%%%fbauth_pubs
%%%endfbauth
 

До речі, європейці до середини 19 сторіччя дупля не відбивали що то за готика
така. Для них це будо потворство, несмак і груда почорнівших камінюк, яка
заважає порпатись на городі і пасти там гусей. Покі не з’явився Ежен
Віоле-ле-Дюк, не почав реставрувати собори і замки, писати книги і статті та не
пояснив популярно публіці, що це взагалі таке. І що це має неабияку культурну
цінність. І зараз, завдяки йому, у масовій свідомості сформована ця зв’язка
Готика=Європа=Сердьновіччя=Культура. До нього її не було, все це, що будували
сторіччями цілими містами і цехами, руйнувалось і зносилось тільки так.

\begin{itemize} % {
%%%fbauth
%%%fbauth_name
\iusr{Оксана Цюпа}
%%%fbauth_url
%%%fbauth_place
%%%fbauth_id
%%%fbauth_front
%%%fbauth_desc
%%%fbauth_www
%%%fbauth_pic
%%%fbauth_pic portrait
%%%fbauth_pic background
%%%fbauth_pic other
%%%fbauth_tags
%%%fbauth_pubs
%%%endfbauth
 
\textbf{Olexa Mann} Українці почали розбиратися в неоготиці, а от в романському стилі ніяк, сумно, що руїни десятинної в Києві не мають жодного мистецького, культурного, історичного значення для освічених людей нашого краю, бо вже касапи там свій рпцшний храм збудували. В центрі Києва і ніхто не переймається . Сумно все

%%%fbauth
%%%fbauth_name
\iusr{Raisa Prisetska}
%%%fbauth_url
%%%fbauth_place
%%%fbauth_id
%%%fbauth_front
%%%fbauth_desc
%%%fbauth_www
%%%fbauth_pic
%%%fbauth_pic portrait
%%%fbauth_pic background
%%%fbauth_pic other
%%%fbauth_tags
%%%fbauth_pubs
%%%endfbauth
 
\textbf{Оксана Цюпа} будівництво рпц шних церков- то чиста пролітика, проникнення в свідомість і життя люду. рпц споруда біля Ейфелевої вежі і всі приміщення її - це чисто фсб штука.. І вона вже має вплив на пересічних французів- бо то "духовний" дом. Там і оперні співаки і всякі пеци виступають. сКазати, шо наші цього не розуміють? сумніваюсь... Простіше мати архітекторів-запроданців совіцького пошибу і владу таку ж. яка нічого не тільки за кордоном, а й вдома не робить. ВІргіше, трудиться в потілиця проти України. З культури починається все, з радіо на стіні, з музеїв, з історіїї - з поваги до кожної збереженої цеглинки старовини..

%%%fbauth
%%%fbauth_name
\iusr{Valerii Zema}
%%%fbauth_url
%%%fbauth_place
%%%fbauth_id
%%%fbauth_front
%%%fbauth_desc
%%%fbauth_www
%%%fbauth_pic
%%%fbauth_pic portrait
%%%fbauth_pic background
%%%fbauth_pic other
%%%fbauth_tags
%%%fbauth_pubs
%%%endfbauth
 
\textbf{Olexa Mann} ну чого ж це, Віктор Гюґо долучився і Вальтер Скот, Фюстель де Кулянж

%%%fbauth
%%%fbauth_name
\iusr{Volodymyr Kolbasa}
%%%fbauth_url
%%%fbauth_place
%%%fbauth_id
%%%fbauth_front
%%%fbauth_desc
%%%fbauth_www
%%%fbauth_pic
%%%fbauth_pic portrait
%%%fbauth_pic background
%%%fbauth_pic other
%%%fbauth_tags
%%%fbauth_pubs
%%%endfbauth
 
\textbf{Olexa Mann} Кьольнський собор добудували вже при Пруссії після Наполеона. Перед тим він стовбичив з однією баштою і краном на ній аж з Середньовіччя. Кран перейшов прямо з полотен 15 ст на фотографії

%%%fbauth
%%%fbauth_name
\iusr{Максим Горбань}
%%%fbauth_url
%%%fbauth_place
%%%fbauth_id
%%%fbauth_front
%%%fbauth_desc
%%%fbauth_www
%%%fbauth_pic
%%%fbauth_pic portrait
%%%fbauth_pic background
%%%fbauth_pic other
%%%fbauth_tags
%%%fbauth_pubs
%%%endfbauth
 
\textbf{Olexa Mann} це якось трохи спрощений погляд, адже саме готичні собори були центром помазання монархів на трон.

%%%fbauth
%%%fbauth_name
\iusr{Олекса Манн}
%%%fbauth_url
%%%fbauth_place
%%%fbauth_id
%%%fbauth_front
%%%fbauth_desc
%%%fbauth_www
%%%fbauth_pic
%%%fbauth_pic portrait
%%%fbauth_pic background
%%%fbauth_pic other
%%%fbauth_tags
%%%fbauth_pubs
%%%endfbauth
 

\textbf{Maxim Gorban} це відбулось після втрати функціональної актуальності готичних споруд для громади.

%%%fbauth
%%%fbauth_name
\iusr{Pavlo Salyga}
%%%fbauth_url
%%%fbauth_place
%%%fbauth_id
%%%fbauth_front
%%%fbauth_desc
%%%fbauth_www
%%%fbauth_pic
%%%fbauth_pic portrait
%%%fbauth_pic background
%%%fbauth_pic other
%%%fbauth_tags
%%%fbauth_pubs
%%%endfbauth
 
Я так розумію, що повернення готики в архітектуру відбувалося паралельно до зародження національної ідентичності і концентрацією уваги на оцій всій Вальтерскотівщині

%%%fbauth
%%%fbauth_name
\iusr{Pavlo Salyga}
%%%fbauth_url
%%%fbauth_place
%%%fbauth_id
%%%fbauth_front
%%%fbauth_desc
%%%fbauth_www
%%%fbauth_pic
%%%fbauth_pic portrait
%%%fbauth_pic background
%%%fbauth_pic other
%%%fbauth_tags
%%%fbauth_pubs
%%%endfbauth
 
Неоготичні будівлі були популярні серед української національносвідомої буржуазії кінця 19ст. Наприклад неоготичні будинки Терещенків
\end{itemize} % }

\end{itemize} % }

