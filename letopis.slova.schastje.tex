% vim: keymap=russian-jcukenwin
%%beginhead 
 
%%file slova.schastje
%%parent slova
 
%%url 
 
%%author 
%%author_id 
%%author_url 
 
%%tags 
%%title 
 
%%endhead 
\chapter{Счастье}
\label{sec:slova.schastje}

%%%cit
%%%cit_head
%%%cit_pic
%%%cit_text
Нужно просто жить свободно и \emph{счастливо}.  Покидая Ливингстон - город основанный
беглыми рабами, я смотрел на остовы погибших кораблей. На птиц, что напоминают
души людей, освободившихся от рабства.  Вот почему жители этого городка на краю
Земли, выглядят более свободными и \emph{счастливыми}, чем жители «цивилизованных»
столиц? Почему в их взгляде нет злобы и высокомерия? Может потому что они не
переписывают никаких Деклараций о Свободе, а просто живут свободно и \emph{счастливо}?
%%%cit_comment
%%%cit_title
\citTitle{Для того, чтобы быть счастливыми, не надо переписывать Декларации о Свободе / Лента соцсетей / Страна}, 
Константин Стогний, strana.ua, 06.07.2021
%%%endcit


%%%cit
%%%cit_head
%%%cit_pic
%%%cit_text
По закінченні Другої світової ми будували ракети, все ще готуючись захищатися і
нападати, потім оплакували Сталіна, після його смерті з жахом вдивляючись у
нічне небо, певні, що от-от налетять американські бомбардувальники, які, як
відомо, тільки й чекали кончини великого Йосипа Віссаріоновича, щоб напасти на
СРСР; потім рахували генсеків ЦК КПРС, що помирали один за одним, спостерігали,
як слабне хватка КДБ, шукали незалежності, плакали від розпаду СРСР, раділи
цьому розпаду, ходили на мітинги, кочували від економічної кризи до політичної
і назад, корумпувалися, піддавалися пропаганді, далі страждали і боялися
мріяти.  У нашої багатостраждальної нації немає \emph{генетичного коду щастя}.
Ми ходили не до тієї школи і виконували не те домашнє завдання
%%%cit_comment
%%%cit_title
\citTitle{Українці не розуміють одне одного не через мову, а через небажання слухати, чути і сприймати}, 
Юлія Мендель, www.pravda.com.ua, 07.07.2021
%%%endcit

%%%cit
%%%cit_head
%%%cit_pic
%%%cit_text
"Поддалась давлению".  Журналист Юрий Ткачев написал, что на кону было
благополучие и карьера спортсменки.  "Понимаете, друзья, я прекрасно понимаю
эмоции. Но осуждать или не осуждать "прозрение" спортсменки могут те, кто был в
её ситуации. То есть, когда тебе в момент, когда тебе кажется, что вот оно,
\emph{счастье}, приходят и популярно объясняют, что это \emph{счастье} может очень быстро
закончиться и смениться чем-то противоположным. Что, мол, сейчас у тебя много
чего есть, а завтра не будет ничего.  Ни работы. Ни учёбы. Ни карьеры. Ни
спорта. Ничего вообще. Только ушаты говна из каждого утюга и постоянный не
слишком доброжелательный присмотр со стороны спецслужб.  Но всё это можно очень
легко исправить. Достаточно кое-что сказать и кое-что сделать
%%%cit_comment
%%%cit_title
\citTitle{\enquote{Многие вас поддерживали и теперь разочарованы}. Что пишут в сети о покаянном заявлении спортсменки Магучих}, 
Оксана Малахова, strana.ua, 12.08.2021
%%%endcit
