%%beginhead 
 
%%file 22_09_2023.fb.fb_group.mariupol.biblioteka.korolenka.1.mariupol_misto_talanovytoj_molodi
%%parent 22_09_2023
 
%%url https://www.facebook.com/groups/1476321979131170/posts/6545560382207279
 
%%author_id fb_group.mariupol.biblioteka.korolenka,lisogor_viktoria.mariupol
%%date 22_09_2023
 
%%tags 
%%title Маріуполь – місто талановитої молоді
 
%%endhead 

\subsection{Маріуполь – місто талановитої молоді}
\label{sec:22_09_2023.fb.fb_group.mariupol.biblioteka.korolenka.1.mariupol_misto_talanovytoj_molodi}
 
\Purl{https://www.facebook.com/groups/1476321979131170/posts/6545560382207279}
\ifcmt
 author_begin
   author_id fb_group.mariupol.biblioteka.korolenka,lisogor_viktoria.mariupol
 author_end
\fi

\textbf{Маріуполь – місто талановитої молоді.}

Вересень для Маріуполя та маріупольців завжди був особливім місяцем.

Після спекотного літа, відпочинку на нашому лагідному та теплому морі,
відпусток та подорожей містами України або іншими країнами світу, ми завжди
занурювались в клопітливу підготовку до важливих для нашого міста дат. Вересень
– це безліч культурних заходів, урочистих подій та фестивалів. А головне це –
День міста.

В бібліотеках також панувала своя атмосфера. Ми ретельно готувались до цього
свята, придумуючи різні форми роботи. Проводили літературні квести чи турніри,
цікаві вікторини чи конкурси, пізнавальні краєзнавчі часи тощо. Та дуже
потужними та яскравими завжди були виставки талановитих дітей, які проходили в
мобільній галереї \enquote{Світ захоплень} у Центральній бібліотеці ім. В. Г. Короленка.

Передивляючись свої врятовані архіви, я натрапила на оцифровані малюнки, які
були представлені на виставці творчих робіт учнів художнього відділення
Маріупольської школи мистецтв (викладачі Макаренко Л.В. та Головенкіна Н.В.),
до 240 річчя міста, тобто 5 років тому.

Це особливі малюнки. На них зображений наш мирній, гарний, квітучий та самий рідний Маріуполь.

Пропоную і вам зануритись у перегляд цих робіт нашої талановитої молоді.

Нажаль мало на яких є авторство. Тому якщо хтось впізнає свій, друзів або знайомих пишіть в коментарях.

Далі буде...
