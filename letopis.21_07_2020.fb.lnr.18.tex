% vim: keymap=russian-jcukenwin
%%beginhead 
 
%%file 21_07_2020.fb.lnr.18
%%parent 21_07_2020
 
%%endhead 
  
\subsection{Эксперт рассказал, до чего национализм довёл Украину}
\url{https://www.facebook.com/groups/LNRGUMO/permalink/2863803723731183/}

\vspace{0.5cm}
{\small\LaTeX~section: \verb|21_07_2020.fb.lnr.18| project: \verb|letopis| rootid: \verb|p_saintrussia|}
\vspace{0.5cm}

Украинские националисты засели во всех ключевых гуманитарных отраслях ещё со
времён президентства Виктора Ющенко и навязывали стране идеологию западничества
и раскола государства по языковому, этническому и религиозному признаку.

Об этом в эфире телеканала «Первый Казацкий» заявил политолог Руслан Бортник,
передаёт корреспондент «ПолитНавигатора».

«Украина – страна, раздираемая между востоком и западом. У нас есть правое и
левое крыло, с одной стороны правые, которые доминируют в гуманитарном
дискурсе: куда страна должна идти, как нужно любить Украину, каким языком
общаться, в какую церковь ходить и так далее.

Правые доминируют в гуманитарном дискурсе лет 15, ещё с эпохи Ющенко, а с
другой стороны, также внешнеориентированные силы, которые тоже ориентируются на
внешних игроков, и они также с реваншистскими настроениями.

Страна в этой дихотомии становится всё мельче, всё меньше, всё беднее и всё
более непривлекательной с точки зрения жизни в ней», – отметил он.

Вадим Москаленко
