% vim: keymap=russian-jcukenwin
%%beginhead 
 
%%file 08_10_2021.fb.magomedov_musa.avdeevka.1.vojna.cmt
%%parent 08_10_2021.fb.magomedov_musa.avdeevka.1.vojna
 
%%url 
 
%%author_id 
%%date 
 
%%tags 
%%title 
 
%%endhead 
\subsubsection{Коментарі}

\begin{itemize} % {
\iusr{Александр Барамиков}

Печально и грустно читать. В войне страдают мирные люди и люди, которые
повелись на речи "вождей". Очень бы хотелось узнать как беспрепятственно зашли
в Донецк колонна Гиркина.

\begin{itemize} % {
\iusr{Муса Магомедов}
\textbf{Александр Барамиков} это один из многих вопросов...

\iusr{Olena Demydiuk}
\textbf{Муса Магомедов} лично для меня это - единственный вопрос. И я хочу знать на него ответ!

\iusr{Александр Павлов}
\textbf{Olena Demydiuk} Я знаю, как он зашел в Донецк. Мы были 5 июля 2014 года с финским журналистом в Славянске и Краматорске и видели, как колонна из 150 машин зашла в Донецк. Никто, еще раз говорю - НИКТО, даже не пытался этому воспрепятствовать. Они шли, не парясь, со скоростью 60-70 км в час. По открытой степи. Остальное - думайте сами...

\iusr{Игорь Шендер}
\textbf{Александр Павлов} Так с журналистом - фотографий небось наделали и видео сняли? Показывайте, всем интересно.
\end{itemize} % }

\iusr{Елена Лимарева}

На два месяца мира дольше ! Можно позавидовать даже! У меня война началась 26
мая, когда украинская военная авиация бомбила Донецк. Первые мирные жертвы на
Привокзальной площади – женщина со смертельными ранениями в голову, её 8-летний
сын ранен... И убитый парень- парковщик.

\begin{itemize} % {
\iusr{Муса Магомедов}
\textbf{Елена Лимарева} я в это время жил в Донецке и боевые вертолёты над аэропортом видел, и дым над ним. И в дом, котором я жил прилетело, но я был на работе в это время. Но мне казалось всё это ещё можно остановить, что это не по-настоящему и вот-вот найдут решение, договорятся... Не сложилось.

\iusr{Валентина Щербак}
Да, до слёз...
мне стало страшно и непонятно , когда начался обстрел 26, 27 июля ,, и вырубился свет и на заводе перебили два ввода.. сказали , если за сутки не сделают - все конец всему. И между обстрелами - эта звовещая тишина...соседи - заводские, сутками были на работе, и говорили - Муса с нами..
Спасибо вам, Муса Сергоевич, на вас была единственная Надежда- будет жить завод , будет жить Авдеевка.!
\end{itemize} % }

\iusr{Владимир Вередин}

Можно сколько угодно сокрушаться, надеяться и верить, но я при памяти и буду
всегда помнить как ,,мирное население донбасса,, при молчаливом выжидании
самого богатого украинца-дончанина слили донбасс вместе с ,,мирным населением,,
...

\begin{itemize} % {
\iusr{Муса Магомедов}
\textbf{Владимир Вередин} про Крым ничего не хотите сказать?

\iusr{Владимир Вередин}
\textbf{Муса Магомедов} , Крым сдал Янукович при ,,харківських угодах,, ... как бы тоже донецкий персонаж ...

\iusr{Татьяна Данильченко}
\textbf{Владимир Вередин} а Турчинов с Яценбком совсем ни причём???


\iusr{Муса Магомедов}
\textbf{Владимир Иванович}, очень удобно смотреть на историю, забывая половину фактов, ту что не нравится и не ложится в Вашу картину мира. Продолжайте в том же духе, оно так проще.

\iusr{Ольга Алексеевна Хвощинская}
\textbf{Муса Магомедов} для некоторых история - это события произошедшие, а для других это трактуется как байки.
\end{itemize} % }

\iusr{Александр Мелешко}

Двадцатые числа июля дали нам всем понять, что это не просто какие то сводки в
новостях. Что вот это все где то там в Славянске, и что нас это не касается и
снова скоро будет мир и спокойствие  @igg{fbicon.face.unamused} . Первые прилеты на территорию АКХЗ, весь
город в коксовом дыму и отсутствие электроснабжения - как вчера помню свое
беспомощное состояние, то что мы все стали заложниками ситуации и абсолютно не
знаем что происходит, что нас ждёт завтра. Но при этом от каждого из нас
зависит что-то, и мы каждый можем повлиять, пусть даже случайно оказаться в
нужном времени и месте и помочь аварийной бригаде добраться на место для
устранения аварии. И каждый тот день это теперь история 

\href{https://youtu.be/a6HcNbYIeYE}{%
АКХЗ 21.07.14 экологическая катастрофа, Александр Мелешко, youtube, 21.07.2014%
}
\ii{08_10_2021.fb.magomedov_musa.avdeevka.1.vojna.cmt.text.youtube.21_07_2014.zavod}

\iusr{Муса Магомедов}
\textbf{Александр Мелешко} ты прав, 21-е.

\iusr{Наталья Рева}

23 июля? Муса, при все уважении, что ж так поздно. То же самое могу сказать о
многих своих знакомых. 13 Марта на пл Ленина убили человека, из-за спины
милиции, на все смотрели и не вмешивались. А жителей Донецка, молодых,
пожилых, с детьми, которые вышли против это русскомирского маразма, во всех
донецких сми назвали завезёнными бандеровцами. Уже два месяца, как висел
русский флаг над Ворошиловский райсоветом. А вы говорите, 23 июля.  @igg{fbicon.face.anxious.sweat} 

\begin{itemize} % {
\iusr{Natella Nedirivko}
\textbf{Наталья Рева} 

Полностью согласна... В марте я тоже не совсем все понимала, а вот 12 апреля
(захват Славянска) для меня раставил все ((((

\iusr{Муса Магомедов}
\textbf{Наталья Рева} да, на площади Ленина убили человека и мне его искренне жаль, я помню этот день. И на предыдущем проукраинском митинге я был, как и на многих других, на этом не был, к сожалению или к счастью- не знаю.
Но это ещё была не война.
Поэтому, при всём моём уважении, для меня война началась именно 23.07.14.

\iusr{Адексей Цимбаленко}
\textbf{МусаМагомедов} ОТ ЧТЕНИЯ ВАШИХ ВСПОМИНАНИЙ СТАЛО БОЛЕТЬ СЕРДЦЕ...
НО ВСПОМИНАТЬ НАДО. ЗА ВАМИ Л И Д Е Р О М СТОЯЛИ ЛЮДИ, КОТОРЫЕ МОГЛИ НЕ ВЕРНУТЬСЯ С РАБОЧЕЙ СМЕНЫ...
О НИХ НЕЛЬЗЯ ЗАБЫВАТЬ. ЭТО ВАША ОБЩАЯ ПОБЕДА!

\iusr{Elena Yudina}
\textbf{Наталья Рева}, у каждого своя дата осознания войны  @igg{fbicon.face.pensive} 
\end{itemize} % }

\iusr{Вадим Нестеренко}

10 июня 2014г. Пять дней до дня рождения. Было двухнедельное затишье и семья
приехала домой в родную квартиру на Киевском РИКе после двухмесячных гостеваний
у родственников. Но снова начались боевые действия, а мы собирались на море.
Впервые в жизни я не бубнел что жена собирает с собой практически весь
гардероб, а наоборот говорил, что может быть холодно и надо взять еще и теплые
вещи, что мы едем отдыхать, а не стирать поэтому надо брать побольше, что надо
взять все, что может пригодиться, что бы потом не искать в курортном селе.
Вспоминали как мы гуляли по городу, как появилась Дочка, как еще встречались и
жена приезжала ко мне в гости и мы ходили в донецкий планетарий... 

Катя все спрашивала вернемся ли мы после моря домой... говорил что обязательно
вернемся, но понимал что наша семья больше никогда не будет в этом доме,
понимали все, даже наша маленькая Маша. Надежда все еще грела внутри и хотелось
думать что ошибаемся. Собирались медленно, как обычно собираются домой из
недолгого отпуска в приятном месте. Ночью снова падали грады, казалось в
соседнем дворе, тогда еще не работал внутри автоматический калькулятор
расстояния до зоны обстрела. 

Тряслись окна, было страшно за жизни родных. В тайне от наших девочек
договорились с тестем, что с моря он заберет мою семью к себе. Было понятно что
это не на неделю и не на месяц, и искренне удивлялся тем кто ехал в Турцию и
другие места пересидеть этот ад, а люди просто не верили в происходящее.
Хотелось проснуться и что бы все было как раньше, хотелось что бы приехали
друзья на выходные, мы взяли вкуснейшего шашлыка на Маяке и весело чтото
обсужлали, пока наши дети иргают, но это уже было историей нашей жизни и чтото
внутри говорило об этом очень уверенно и с грустью. 

Уезжали с тяжелым чувством, никогда, ни до, ни после, мы так не ехали на отдых,
с абсолютным чувством пути в одну сторону. После было гораздо страшнее, но
именно тогда для меня закончилась мирная жизнь и началась война

\begin{itemize} % {
\iusr{Александр Барамиков}
\textbf{Вадим Нестеренко} 

тогда не думали о том, что это надолго. Ехали на море пересидеть или в далёкое
забугорье.... печально блин, просто охринеть как!! Я не жил там уже в это время,
но судьба так распорядилась, что три двоюродные сестры, родной брат остались в
Макеевке, отец в Крыму.... Эти события разделили нас как территориально, так и
политически. Спасибо Вам за свой рассказ и хочу Вам пожелать тепла и
спокойствия в сердце))

\iusr{Вадим Нестеренко}
\textbf{Александр Барамиков} спасибо, взаимно!
\end{itemize} % }

\iusr{Roman Shakhov}

Один из домов в Авдеевке... градинка пришла... разве такое хочется вспоминать?

\ifcmt
  ig https://scontent-frx5-1.xx.fbcdn.net/v/t39.30808-6/244425997_386408046493301_5250734729253091247_n.jpg?_nc_cat=111&ccb=1-5&_nc_sid=dbeb18&_nc_ohc=JL6OuRu-foAAX8zNenk&_nc_ht=scontent-frx5-1.xx&oh=973caa738043dd7365ee2daf9c36c2f4&oe=61A47F8C
  @width 0.4
\fi

\iusr{Римма Филь}
Муса! Я пишу и стираю и снова пишу! Обнимаю тебя и жму руку!

\iusr{Евгений Букша}
Муса Сергоевич. Вам респект всю эту жопушку пережили с нами на АКХЗ и решали важные вопросы для завода и города без всяких как сейчас модно удалёнок. @igg{fbicon.thumb.up.yellow} 

\iusr{Оксана Юровицкая}
Обнимаю... Ты -Человечище! Спасибо.

\begin{itemize} % {
\iusr{Стелла Айнбиндер}
\textbf{Oksana Yurovickaya} Здорово что Вы так написали ! Буквально мои мысли !!!!
\end{itemize} % }

\iusr{Daria Bielikova}
Муса, дякую за вашу діяльність тоді і там, і за те, що називаєте війну війною. Це і досі не всі вміють.

% -------------------------------------
\ii{fbauth.kolesov_igor.avdeevka.ukraina}
% -------------------------------------

При всем уважении к Наталье Емченко, я не буду поддерживать данный флешмоб,
хотелось бы всё забыть и не вспоминать. Две армии, которые периодически
наступают друг на друга, а ты между этой наковальней и молотом ходишь на работу
каждый день, дома маленькая дочь и масса вопросов, как жить дальше. Во время
войны каждый прожитый в этом городе день это маленькая жизнь, снаряды прилетали
на кого Бог пошлет. Забыть хочется все это по-быстрей

% -------------------------------------
\ii{fbauth.hvoschinskaja_olga.avdeevka.ukraina.feldsher}
% -------------------------------------

Помню мужчину с переломом голеностопа везли в горбольницу, и попали под жёсткий
перекрёстные огонь, возвращаться на завод нет смысла (уже ни так далеко от
больницы) и возле 7 школы накрыло по взрослому, Серёга, водитель спасибо тебе
за твоё терпение к моим матам  @igg{fbicon.wink}, приехали сдали пациента, ребята военные
выскочил из больницы думали что нам уже всё. Назад возвращались, видели воронку
от снаряда от которой Серёга ушёл.  @igg{fbicon.eyes}  Много чего было..

\iusr{Надежда Ткачук}
Авдос июль 2014

\ifcmt
  ig https://scontent-frx5-2.xx.fbcdn.net/v/t39.30808-6/244578137_1688775194846314_8312492812840104725_n.jpg?_nc_cat=109&ccb=1-5&_nc_sid=dbeb18&_nc_ohc=w9YcJIwceLwAX9YvB5t&_nc_ht=scontent-frx5-2.xx&oh=22a1baacc799e4c292ed41aeba0cb3bc&oe=61A58E02
  @width 0.4
\fi

\iusr{Roman Shakhov}

Так себе флешмоб... даже вспоминать не хочется... ни когда увидел первые
разрушения, ни когда первый раз попал под обстрел, ни поломанные человеческие
судьбы.

\iusr{Рус Че}

Кто бы мог подумать, что такое - вообще, ВОЗМОЖНО???
Какой-то сюрреализм...
До сих пор - в голове не укладывается это всё...(((

\iusr{Елена Севрюкова}
Кто остался в городе и не уехал прошли все ужасы и страхи.

\iusr{Мария Борщова}

В Донецке войны тоже не было бы, если бы кое кто не ездил на переговоры с
бандитами... Можно было бы решить всё за один раз, как в Одессе... Значит
многим выгодна эта проклятая война...

\iusr{Светлана Литвин}

\ifcmt
  ig https://scontent-frt3-1.xx.fbcdn.net/v/t39.30808-6/244751998_2181184168688983_5339313870564613026_n.jpg?_nc_cat=106&ccb=1-5&_nc_sid=dbeb18&_nc_ohc=BtgOjZoxxhIAX9e138F&_nc_ht=scontent-frt3-1.xx&oh=1e600bddd145d2f1fba973390a11dacc&oe=61A5004C
  @width 0.4
\fi

\iusr{Сергей Нестеренко}
Молодёжная 20

\ifcmt
  ig https://scontent-frx5-2.xx.fbcdn.net/v/t39.30808-6/244543779_4604883542921814_6969633697045937346_n.jpg?_nc_cat=109&ccb=1-5&_nc_sid=dbeb18&_nc_ohc=LIQmI0rmcawAX-QBjF-&_nc_ht=scontent-frx5-2.xx&oh=eeb0da889ebc87a1f85f45a84ec057a5&oe=61A5A342
  @width 0.5
\fi

\iusr{Сергей Нестеренко}
Молодежная 20

\ifcmt
  ig https://scontent-frt3-1.xx.fbcdn.net/v/t39.30808-6/244583894_4604889166254585_8548509243563909564_n.jpg?_nc_cat=104&ccb=1-5&_nc_sid=dbeb18&_nc_ohc=3MLMn5gbiH0AX9AiEtv&_nc_oc=AQko2D5RtbU2iqIRJh3z0U4iK2nJpZWdfJd2GSwEpcSYazIyJnC_5PB662yVPrsIe_8&_nc_ht=scontent-frt3-1.xx&oh=9aa64e53f08278d07dcf3a773936e064&oe=61A445D5
  @width 0.5
\fi

\iusr{Елена Севрюкова}
Да помню, как ездила на велике в старуху в колодец по воду, и не важно бомбили или нет, вода это жизнь.

\iusr{Кочерга Станислав}
Спасибо - это настоящие воины !!!!!!!

\iusr{Андрей Сенека}
Сильный текст.  @igg{fbicon.thumb.up.yellow}. Не забыть это время никогда.

\iusr{Сергей Нестеренко}

\ifcmt
  ig https://scontent-frt3-1.xx.fbcdn.net/v/t39.30808-6/244982981_4604889649587870_5124994272364140768_n.jpg?_nc_cat=106&ccb=1-5&_nc_sid=dbeb18&_nc_ohc=sc71hVbQ3XsAX860ue4&_nc_ht=scontent-frt3-1.xx&oh=5a7494e3400e0c2479bf8b9f4ac26a9a&oe=61A44AE1
  @width 0.4
\fi

\iusr{Александр Суркис}
Тяжело всё это читать, а видеть и жить с этим ещё хуже....

\iusr{Александр Мелешко}
Один день, который ни как не закончится. Вот уже семь лет

\iusr{Сергей Нестеренко}

\ifcmt
  ig https://scontent-frt3-1.xx.fbcdn.net/v/t39.30808-6/244648484_4604883776255124_7824281750797126997_n.jpg?_nc_cat=108&ccb=1-5&_nc_sid=dbeb18&_nc_ohc=JSc-k2n8lfEAX8NR80J&_nc_ht=scontent-frt3-1.xx&oh=6d43727d159912ba43eca7be137688d2&oe=61A4909A
  @width 0.4
\fi

\iusr{Евгения Карабаз}
Хочется запомнить и вспоминать день когда всё закончится @igg{fbicon.face.pleading} 

\iusr{Светлана Кот}
Русский мир, он такой!

\iusr{Светлана Жар}
А 2 ое мая все забыли?

\iusr{Михайло Ухман}
Дякую!

\iusr{Наталья Еременко}
Муса Сергеевич, посодействуте пожалуйста в восстановлении дома МПС 52

\iusr{Сергей Нестеренко}
Менделеева 8

\ifcmt
  ig https://scontent-frx5-1.xx.fbcdn.net/v/t39.30808-6/244746829_4604883619588473_399321853130970017_n.jpg?_nc_cat=110&ccb=1-5&_nc_sid=dbeb18&_nc_ohc=o6AZxcb9yv4AX_LaqF5&_nc_ht=scontent-frx5-1.xx&oh=7756d7a6eb2816eaf7828a4bf2b47413&oe=61A51C44
  @width 0.4
\fi

\iusr{Андрей Фищук}
Смерть російським окупантам!

\iusr{Кочерга Станислав}

Да, я богу благодарен, что моя смена 4 бригада смолоперегонного цеха осталась
жива. И спасибо большое всем моим сотрудника, что они вели себя как настоящие
мужики!!!!!!! Спасибо и ИТР Кизило. В. А. Косенко А. В. Венско. В. В



\end{itemize} % }
