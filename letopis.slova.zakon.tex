% vim: keymap=russian-jcukenwin
%%beginhead 
 
%%file slova.zakon
%%parent slova
 
%%url 
 
%%author 
%%author_id 
%%author_url 
 
%%tags 
%%title 
 
%%endhead 
\chapter{Закон}
\label{sec:slova.zakon}

%%%cit
%%%cit_pic
%%%cit_text
Максим, \enquote{надо определиться с правовым статусом}? Это - как? Вы не
учитываете того факта, что наука История - это не наука Физика, в которой раз и
навсегда появившиеся (открытые людьми) \enquote{законы} действительны как в
настоящем, так и в прошлом, так и будут действовать и в будущем, разве что
только расширяясь и развиваясь дальше, по мере расширения знаний в этой
\enquote{естественной} науке (\emph{Закон сохранения Материи}, например). Но в
истории людей законы в их обществах пишет не Природа (естество), а люди.
Поэтому в науке Право нет тех \emph{Законов}, которые действуют сейчас, но действовали
в прошлом и будут действовать и в будущем тоже. Именно эта особенность
\enquote{человеческих} (гуманитарных и юридических) \emph{Законов} закреплена в
одном из \enquote{базовых} принципов этой науки - \emph{Закон} не имеет обратной силы.
Иными словами нельзя (!) с точки зрения современного Права рассматривать и
оценивать события и действия людей в прошлом, когда они жили и действовали в
другом \enquote{правовом поле}, и согласно действовавшим в то время \emph{Законам} и
правилам, были вполне \enquote{законопослушными} людьми, не делавшими ничего
\enquote{предосудительного}.  Но это не означает того, что человек и
человеческие общества развиваются \enquote{\emph{беззаконно}}, т.е. совершенно
случайно и хаотично. Это означает только то, что люди, не смотря на все свои
попытки в прошлом и настоящем, просто еще не открыли эти \enquote{базовые}
\emph{Законы} своего развития, действовавшие в прошлом, действующие в
настоящем, и которые будут действовать и в будущем. Но эксперимент под
названием \enquote{Жизнь на планете Земля} еще не закончен, поэтому шанс
открыть эти \emph{Законы} у людей тоже еще сохраняется
%%%cit_comment
Алексей Джалалов
%%%cit_title
\citTitle{Кто основал Москву? Вы даже удивитесь - ее основал великий киевский князь в XII веке и заселил ее украинцами}, 
Исторический Понедельник, zen.yandex.ru, 18.02.2021
%%%endcit

%%%cit
%%%cit_head
%%%cit_pic
%%%cit_text
В общем, если хотите, можете сами глянуть отчёт на 100 с лишним страниц. Ну и
как вы поняли, Украина \enquote{образцово демократическое} це-европейское государство,
в котором полным ходом внедряют европейские ценности, соблюдают права человека
и лучше всех знают как нужно поступать с основным \emph{законом} страны -
Конституцией. Правильно, \emph{основной закон}, по-украински, годится только на то,
чтобы в него рыбу заворачивать.  Выводы делать вам. Благодарю за внимание
%%%cit_comment
%%%cit_title
\citTitle{Speak на мове please}, 
Terra Incognita, zen.yandex.ru, 30.04.2021
%%%endcit

%%%cit
%%%cit_head
%%%cit_pic
%%%cit_text
Да и новые \emph{законопроекты} \enquote{слуг народа} и правительства
относительно будущего Донбасса прямо сужают права для жителей \enquote{ДНР} и
\enquote{ЛНР}, если они вернутся в состав Украины. Например, таков
\emph{законопроект} о \enquote{коллаборантах}. Также граждан Украины, у
которых найдут российский паспорт, Кабмин предложил интернировать - то есть
поступать с ними, как с военнопленными и без всякого суда изолировать. Это
избирательное поражение в правах, что прямо запрещено Конституцией
%%%cit_comment
%%%cit_title
\citTitle{День Конституции Украины 28 июня - какие статьи нарушаются сильнее всего}, 
Оксана Малахова; Максим Минин, strana.ua, 28.06.2021
%%%endcit

%%%cit
%%%cit_head
%%%cit_pic
%%%cit_text
Одиозный закон все-таки признали конституционным.  Какая-то чиновница
Конституционного суда с характерной фамилией Ковбасюк радостно сообщила нам,
что запуганные властью судьи признали нормальным жуткий закон о тотальной
украинизации. При этом даже авторы этого треша несколько лет признавали
неконституционность своего творения и призывали не дать КСУ даже браться за его
анализ.  Потом слуги под давлением нациков отказались от своих несмелых попыток
слегка смягчить самые одиозные формы дискриминации русскоязычных
%%%cit_comment
%%%cit_title
\citTitle{Украинизаторы активно машут лопатой на похоронах своей страны}, 
Олег Волошин, strana.ua, 14.07.2021
%%%endcit

%%%cit
%%%cit_head
%%%cit_pic
%%%cit_text
Декларация была ПЕРВЫМ шагом на пути к независимости Украины. И именно в
Декларации были заложены основы будущего государства и нашей Конституции,
принятой только через 6 лет. Декларация о независимости Украины
продемонстрировала живущим в тогда еще Украинской ССР, что комфортно будет
ВСЕМ, кто живёт на её территории, независимо от национальности или
вероисповедания.  Спустя 31 год Декларация растоптана вместе с правами граждан.
Её просто не читали властные люди.  Одни - в силу скудности ума, другие - в
силу понимания кратковременности пребывания у власти.
P.S. Отныне гражданин Украины не имеет права выражать свои мысли, получать
услуги, учить своих детей и т.д на языке, впитанном с молоком матери.
\emph{Закон о языке} вступил сегодня в силу((
%%%cit_comment
%%%cit_title
\citTitle{Декларация о независимости обещала комфортную жизнь всем живущим в Украине}, 
Сергей Ларин, strana.ua, 16.07.2021
%%%endcit

%%%cit
%%%cit_head
%%%cit_pic
%%%cit_text
Первый русский (по национальной принадлежности) митрополит Руси Иларион в
«Слове о \emph{законе} и благодати» говорит о народе русском и Земле русской:
Ибо правили они не в безвестной и захудалой земле, но в земле Русской, что
ведома во всех наслышанных о ней четырёх концах земли.
И опять не упомянута Украина
%%%cit_comment
%%%cit_title
\citTitle{В чём нас обвиняют на сей раз}, Досужник, zen.yandex.ru, 02.08.2021
%%%endcit

%%%cit
%%%cit_head
%%%cit_pic
%%%cit_text
Тепер я оповідаю про це вільно й впевнено, тому що знаю. А тоді... я не приховую
— тоді було всього. Сумніви, вагання, тяжкі роздуми, пошуки — з чого почати? Я
вирішив розпочати н експериментів, пов’язаних з енергетикою організму, бо ж
вона складає основну сутність біологічного прояву. Я збагнув, що неможливо
ступити на новий, вищий рівень буття, не звільнившись віл архаїчного \emph{закону}
поглинання, пожирання чужої плоті чужого життя. Доки ми залежні від погоди,
їжі, вбпання пристроїв, богів, інших істот, корисних копалин, безлічі речей та
сутностей — про яку свободу можна говорити, про яку волю? Тут може бути лише
одна, умовна відносна свобода — свобода діяти із знанням справи, як говорив
Енгельс. Але ж знову — діяти в межах того ланцюга природничих констант, до
якого нас прип'ято, і тому будь-яка наша дія буде детермінована міріадами
ниток, що сповивають гуллівера людського розуму й чуття...
%%%cit_comment
%%%cit_title
\citTitle{Вогнесміх}, Олесь Бердник
%%%endcit

%%%cit
%%%cit_head
%%%cit_pic

\ifcmt
  tab_begin cols=2
     pic https://strana.news/img/forall/u/0/36/2021-10-27_10h43_29.png
     pic https://strana.news/img/forall/u/0/36/2021-10-27_10h59_13.png
  tab_end
\fi
%%%cit_text
Еще один опрошенный полагает, что парламентско-президентская республика
подходит для нашей страны идеально и хотел бы, чтобы депутаты президента хоть
как-то "держали в узде", как американцы. "Не все зависит от президента, от его
прихоти, того, что ему взбредет в голову", - уверен мужчина.  Он сожалеет, что
в парламент зашло большинство сторонников президента.  "К сожалению, это
парламентское большинство что хотят, то и делают. Какие \emph{законы} хотят, такие и
принимают. Все к этому идет (к президентскому правлению, Ред.). Они это могут
устроить. Но хорошо, что они этого не сделают. Народ не даст этого сделать.
Разумные силы Украины как-то встряхнут и Данилова, и партию "Слуги народа" и
самого президента... Нам Царь-батюшка не нужен в Украине", - подытожил он
%%%cit_comment
%%%cit_title
\citTitle{"Он уже и так диктатор". Что думают в Киеве об идее увеличить полномочия Зеленского. Опрос "Страны"}, 
Антонина Белоглазова, strana.news, 31.10.2021
%%%endcit

%%%cit
%%%cit_head
Строгость российских законов смягчается необязательностью их исполнения
%%%cit_pic
%%%cit_text
Каждый год в Киеве происходит одна и та же дурня - на зиму ограничивают
скорость на скоростных магистралях с 80 до 50 км. Раньше я возмущался, а сейчас
даже не пикну. Потому что знаю, что за этим ничего не последует. Все как ехали
сотку по проспекту Победы, так и будут ехать, включая полицию в потоке. Да,
иногда полиция будет кого-то ловить, но в целом, сбавив скорость у полиции,
дальше будут ехать, как ехали.  Это особенность нашего государства - безумие
его \emph{законов} и норм можно игнорировать необязательностью их исполнения.
Здесь все понарошку, кроме бабла, которое у вас вытягивают под тем или иным
поводом, используя безумные \emph{законы}
%%%cit_comment
%%%cit_title
\citTitle{Чем нивелируется безумие законов и норм нашего государства?}, 
Юрий Романенко, strana.news, 02.11.2021
%%%endcit

%%%cit
%%%cit_head
%%%cit_pic
%%%cit_text
Все эти племена имели свои обычаи, и \emph{законы} своих отцов, и предания, и каждые –
свой нрав. Поляне имеют обычай отцов своих кроткий и тихий, стыдливы перед
снохами своими и сестрами, матерями и родителями; перед свекровями и деверями
великую стыдливость имеют; имеют и брачный обычай: не идет зять за невестой, но
приводит ее накануне, а на следующий день приносят за нее – что дают. А
древляне жили звериным обычаем, жили по-скотски: убивали друг друга, ели все
нечистое, и браков у них не бывали, но умыкали девиц у воды. А радимичи, вятичи
и северяне имели общий обычай: жили в лесу, как и все звери, ели все нечистое и
срамословили при отцах и при снохах, и браков у них не бывало, но устраивались
игрища между селами, и сходились на эти игрища, на пляски и на всякие бесовские
песни, и здесь умыкали себе жен по сговору с ними; имели же по две и по три
жены. И если кто умирал, то устраивали по нем тризну, а затем делали большую
колоду, и возлагали на эту колоду мертвеца, и сжигали, а после, собрав кости,
вкладывали их в небольшой сосуд и ставили на столбах по дорогам, как делают и
теперь еще вятичи. Этого же обычая держались и кривичи, и прочие язычники, не
знающие \emph{закона Божьего}, но сами себе устанавливающие \emph{закон}
%%%cit_comment
%%%cit_title
\citTitle{Повесть Временных Лет}, Нестор Летописец
%%%endcit

%%%cit
%%%cit_head
%%%cit_pic
%%%cit_text
ПРО ОБЛИВАННЯ ВОДОЮ НА ВЕЛИКОДЕНЬ. Деякі з давніх \emph{беззаконників} джерелам
та озерам жертви приносили, аби помножилися плоди земні, а часом і людей у воді
топили. В деяких землях Рóссійських і до сьогодні давнього того безчинства
обновлюється пам’ять, коли під час Пресвітлого Дня Воскресіння Христового
молодь обох статей, або й старші, зібравшись, одне одного ніби задля якоїсь
забави у воду вкидають, і випадає [деяким], намовою бісівською, кинутим у воду,
чи об камінь, чи об дерево розбитися, чи втонути, і негідно втратити душу свою.
Інші ж, хоча й не вкидають у воду, то поливають водою, тому ж бісу жертву
давніх забобонів поновлюючи. Нині в звичаї втіху, а не жертви ідольські
творять, проте, краще б тому не бути.  Шостий ідол — Коляда — бог празничний.
На його честь свято велике місяця грудня 24 дня святкують 482. Хоча люди руські
й Святим Хрещенням просвітилися, й ідолів викорінили, але дехто пам’ять того
біса Коляди й до сьогодні не перестає поновлювати. Почавши з самого Різдва
Христового, повсякденно збираються на богомерзенні ігрища, пісні співають, і в
них, хоча й про Різдво Христове згадують, але тут же \emph{беззаконно} й Коляду
— давню спокусу диявольську — часто повторюючи, додають
%%%cit_comment
%%%cit_title
\citTitle{Синопсис Київський}, , litopys.org.ua 
%%%endcit
