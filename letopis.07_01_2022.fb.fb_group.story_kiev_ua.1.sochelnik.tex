% vim: keymap=russian-jcukenwin
%%beginhead 
 
%%file 07_01_2022.fb.fb_group.story_kiev_ua.1.sochelnik
%%parent 07_01_2022
 
%%url https://www.facebook.com/groups/story.kiev.ua/posts/1835165403346919
 
%%author_id fb_group.story_kiev_ua,nuribekova_vika
%%date 
 
%%tags kiev,rozhdestvo,tradicii
%%title Сочельник
 
%%endhead 
 
\subsection{Сочельник}
\label{sec:07_01_2022.fb.fb_group.story_kiev_ua.1.sochelnik}
 
\Purl{https://www.facebook.com/groups/story.kiev.ua/posts/1835165403346919}
\ifcmt
 author_begin
   author_id fb_group.story_kiev_ua,nuribekova_vika
 author_end
\fi

Гоша очень любил, когда на Святвечер вся семья собиралась у тети Томы. 

Тетя Тома была одинокая, когда началась война, ей было 15 лет, старшая сестра
Валентина, ушла на войну медсестой, оставив Томе двоих племянников, для которых
и стала второй матерью. 

\ii{07_01_2022.fb.fb_group.story_kiev_ua.1.sochelnik.pic.1}

Гоша - это Игорь, дети произносят обычно два слога из длинных слов. Игореша
звучало  длинно, так для домашних появился Гоша.

Жили всей семьей на Татарке в дедовом доме. Дом был на шесть комнат - по одной
на каждого и гостинная, где и проводились семейные трапезы.

\ii{07_01_2022.fb.fb_group.story_kiev_ua.1.sochelnik.pic.2}

Когда дедов дом снесли и на его месте построили девятиэтажку, тете Томе и ее
сестре там достались однушки. 

Гоша, уже взрослый, со своими детьми и внуками всегда в Святвечер был на
Тататарке у тети Томы. Это же семейная традиция. Такде приезжали кузены и
кузины тоже с семьями.

\ii{07_01_2022.fb.fb_group.story_kiev_ua.1.sochelnik.pic.3}

Предрождественским вечером, святый ужин или как еще называли богатая кутья,
собирались не на просто общий ужин семьи. Это был символ домашнего очага.
Считалось, что покойные родственники тоже собираются на этот ужин, поэтому и
готовили поминальные блюда. Ужин на Святвечер проходил неторопливо, а
собравшаяся за столом вся семья вела тихие и неторопливые беседы только о
хорошем. Повышать голос и ссориться нельзя ни в коем случае. 

Поскольку 6 января припадает на последний день Филипповского поста, к столу
готовили исключительно постные блюда. Конечно же, нельзя пить и спиртное! В
Святвечер на столе должен стоять только узвар.

В день перед Сочельником держали строгий пост. Разрешалось немного поесть в
обед только детям. Поэтому неудивительно, что все с нетерпением ожидали, когда
сойдет первая звезда и можно будет начать трапезу.

Ужин тетя Тома готовила весь предрождественский день. Нельзя сказать, что была
она набожной, но традиции чтила.

Готовила, обязательно с хорошим настроением, иначе еда не удастся.

Основными праздничными блюдами были кутья, постный борщ и узвар. Эти блюда
несли сокровенный поминальный смысл. 

Кутью, тетя Тома ставила вариться еще на рассвете, чтобы с первыми лучами
солнца вынуть из печи, (потом уже из духовки), а заливала ее чистейшей
\enquote{рассветной} водой, ведь готовилась \enquote{Божья пища} и пища для
духов-Лада, святых и добрых душ покойников.

Пшеницу, из которой была кутья, нельзя было мешать, потому что считалось, что
будет голодный год. 

Постный борщ был с рыбой, вернее из консерв - кильки в томате. Он варился
накануне, потому что борщ всегда вкусный на второй день.

Обязательными были вареники: с капустой, картофелем с укропом, с фасолью,
иногда с грибами (когда уже дети выросли) и конечно с вишнями. 

Пироги или как Гоша называл \enquote{колоски} делала она с зеленым луком,  укропом,
петрушкой, рисом и обязательно с вязигой. Теперь вязиги не купить да и заменить
не чем.

Пироги действительно тета Тома заплетала в колоски. Ножом не пользовалась, по
народному поверью - это к ссорам в году будущем. Поэтому тетя Тома пользовплась
специальным колесиком для теста. Хоть это и не нож, она на всякий случай
прикусывала язык, чтоб в нужный момент промолчать и не поссориться.

Еще к столу готовились бобы - белые и черные отдельно. Бобы варились без соли,
чтоб не полопались и не были твердыми, а потом при обжаривании с луком солить
уже можно было. Жарили на постном масле и ногда заливались томатом.

Кроме этого к столу были блинчики гречневые и пшеничные, рыба жареная и вареная
(заливная), селедка, рыбные тефтели и конечно всевозможные соленья.

Несмотря на то, что хлопот с приготовлением ужина было много, поддерживали
порядок и чистоту в доме.  Разолить или насорить считалось плохой приметой.

Упаси Бог, с кем-то ссориться в этот день. Наоборот, нужно помириться с
врагами, чтобы в новом году было мирно и в доме, и вне дома. 

Чтобы защитить домочадцев от болезней, по углам стола под скатерть ложили по
зубчику чеснока, который также защищал от нечистой силы. 

Только после этого застилалась скатерть, и выставлялись блюда. На
рождественский стол нельзя подавать ножи и вилки, потому все блюда нужно было
есть ложками.

Для пирогов и хлебу каждому ставили пирожковую тарелку. Хлеб и пироги принято
было отламывать кусочками.

На столе во время Сочельника непременно должна гореть восковая свеча,
знаменовавшая собой животворную силу – солнце, к которому возносится пчела. Ее
вставляли в хлеб.

Зажигая свечу, говорили: \enquote{Миры, праведное солнце, святым душечкам и нам живым,
грей землю-матушку и все живое на ней.}

Тушить свечу на Сочельник нельзя было, потому что она считалась символом жизни
и должна была догореть до конца. Когда кто-то случайно погасит свечу, то
верили, что в следующем году кто-то умрет в семье. 

Если во время ужина свеча горит ясно, то будет хороший год, если мигает и
коптить – не будет благополучия.

После ужина по народному поверью кутью и еще некоторые блюда со стола не
убирали – эта еда оставлялась для умерших родственников, которые, тоже приходят
к нам \enquote{на рождественскую кутью}. 

Вот уже четверть века нет тети Томы, нет и Гоши, но души их в Сочельник с
родней, которая их помнит и чтит их традиции.
