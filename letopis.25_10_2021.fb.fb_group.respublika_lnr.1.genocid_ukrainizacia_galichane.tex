% vim: keymap=russian-jcukenwin
%%beginhead 
 
%%file 25_10_2021.fb.fb_group.respublika_lnr.1.genocid_ukrainizacia_galichane
%%parent 25_10_2021
 
%%url https://www.facebook.com/groups/respublikalnr/posts/907627336539725/
 
%%author_id fb_group.respublika_lnr,zimina_olesja
%%date 
 
%%tags donbass,galichina,jazyk,lnr,luganschina,lugansk,mova,ukraina,ukrainizacia
%%title ГАЗЕТА "РЕСПУБЛИКА" (№42, 2021г). ГЕНОЦИД. ЛУГАНЩИНА ДЛЯ ГАЛИЧАН?
 
%%endhead 
 
\subsection{ГАЗЕТА \enquote{РЕСПУБЛИКА} (№42, 2021г). ГЕНОЦИД. ЛУГАНЩИНА ДЛЯ ГАЛИЧАН?}
\label{sec:25_10_2021.fb.fb_group.respublika_lnr.1.genocid_ukrainizacia_galichane}
 
\Purl{https://www.facebook.com/groups/respublikalnr/posts/907627336539725/}
\ifcmt
 author_begin
   author_id fb_group.respublika_lnr,zimina_olesja
 author_end
\fi

ГАЗЕТА "РЕСПУБЛИКА" (№42, 2021г).

ГЕНОЦИД. ЛУГАНЩИНА ДЛЯ ГАЛИЧАН?

Похоже, именно этим лозунгом руководствуется официальный Киев по отношению к
оккупированным ВСУ территориям Донбасса. Насильственная украинизация и
дерусификация проводилась на Украине с самого начала так называемой
независимости. Особенно активизировались эти процессы после захвата власти
«оранжевыми». С 2004 по 2017 год количество русских школ на Украине уменьшилось
в три раза, а количество обучающихся в них детей – в четыре раза.

\ifcmt
  ig https://scontent-frt3-1.xx.fbcdn.net/v/t39.30808-6/248342303_245761647603942_970723051181814880_n.jpg?_nc_cat=102&ccb=1-5&_nc_sid=825194&_nc_ohc=rSRR4Gh7fYIAX8d-06-&_nc_ht=scontent-frt3-1.xx&oh=f7584757d680cddbff1e2b628cbe5b45&oe=617AC7FB
  @width 0.4
  %@wrap \parpic[r]
  @wrap \InsertBoxR{0}
\fi

Порошенко закрепил тенденцию законом о полном запрете обучения на русском
языке. И если бы идущий во власть актёр из популярного шоу сразу заявил о своей
приверженности курсу дерусификации, большинства голосов избирателей он бы не
получил. 

Новая власть не только не отменила дерусификацию образования, но и приняла
закон о тотальной украинизации всех сфер общественной жизни. Попутно запретив и
языки малых народов – такая себе «маскировка», показывающая якобы
справедливость языкового закона. 

Теперь для поступления на госслужбу все граждане обязаны получить сертификат о
владении государственным языком. А те, кто давно занимает посты в бюджетных
организациях, обязаны доказать своё свободное владение «мовой» в её абсолютно
дикой версии «новояза». 

На подконтрольной Украине части Луганской области открыто шесть центров
сертификации, которые работают крайне медленно: запись на сдачу экзамена
проводится в последний день каждого месяца, и никак иначе. Причём даже тот, кто
займёт очередь ровно в полночь, может не попасть на вожделенную сертификацию –
много «левых» людей. Знакомые, друзья знакомых, родственники и прочие
доверенные лица, которые принесут в центр сертификации мзду в конверте, уже
записаны в очередь. 

Сам экзамен заслуживает особого рассмотрения. Во-первых, все соискатели на
получение сертификата обязаны смотреть прямо в камеру. Нельзя скосить взгляд
даже случайно: если вам захочется чихнуть, на нос сядет муха или кто-то
окликнет во время тестирования – всё, опрос прерывается и такому человеку
сертификат не выдаётся. Сидеть ровно, глаза в камеру, ни малейшего движения
глазами. Во-вторых, на 26 вопросов даётся ровно минута на один вопрос. За это
время нужно прочитать, понять, ответить «мовой». Далее нужно десять минут
говорить на предложенную тему, смотря прямо в камеру. Что это? Допрос в
гестапо? 

Получается, те, кто с детства разговаривал на «мове», а это жители западных
областей Украины и семьи националистов, будут гораздо лучше подготовлены к
экзамену. Вот так аккуратно и якобы по закону проводится замещение русских
специалистов на малообразованных жителей Галичины, которые сейчас увидели для
себя новые возможности в колонизации Донбасса. Вот и готовятся рагули управлять
донбасскими «москаликами», рассаживаются в уютные кресла госслужащих на правах
«титульной нации». Одно дело – спину гнуть на польских плантациях, а совсем
другое – стать властью со всеми полагающимися привилегиями. 

Игорь МИРТОВСКИЙ

ГАЗЕТА "РЕСПУБЛИКА" (№42, 2021г).

\#газета \#республика \#политика\_Украины \#насильственная\_украинизация
