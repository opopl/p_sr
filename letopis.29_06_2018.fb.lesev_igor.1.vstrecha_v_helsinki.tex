% vim: keymap=russian-jcukenwin
%%beginhead 
 
%%file 29_06_2018.fb.lesev_igor.1.vstrecha_v_helsinki
%%parent 29_06_2018
 
%%url https://www.facebook.com/permalink.php?story_fbid=1973559856008441&id=100000633379839
 
%%author_id lesev_igor
%%date 
 
%%tags geopolitika,helsinki,rossia,ukraina,usa,zapad
%%title По встрече в Хельсинки
 
%%endhead 
 
\subsection{По встрече в Хельсинки}
\label{sec:29_06_2018.fb.lesev_igor.1.vstrecha_v_helsinki}
 
\Purl{https://www.facebook.com/permalink.php?story_fbid=1973559856008441&id=100000633379839}
\ifcmt
 author_begin
   author_id lesev_igor
 author_end
\fi

По встрече в Хельсинки.

«Большая сделка» не состоится по ряду причин. Во-первых, никто не понимает, что
это вообще такое. Во-вторых, любые компромиссы геополитического масштаба
возможны только при уважении преемственности этих решений
сторонами-подписантами. А американцы как раз на кую вертят все свои
договоренности, как только видят, что «лед трогается». И это касается не только
примера Трампа с Ираном и климатом. В 2001-м Джордж Буш объявил в одностороннем
порядке о выходе из договора по ПРО. Кстати, именно с тех давних пор началась
новая ракетная гонка. Это чтобы не забывали.

\ifcmt
  ig https://scontent-frx5-1.xx.fbcdn.net/v/t1.6435-9/36325880_1973559776008449_4512669004866781184_n.jpg?_nc_cat=100&ccb=1-5&_nc_sid=730e14&_nc_ohc=f4Ek35jGswsAX9bGKyo&_nc_ht=scontent-frx5-1.xx&oh=492116f7c189d046d67a6f5c95ef2e16&oe=61B9A8DF
  @width 0.4
  %@wrap \parpic[r]
  @wrap \InsertBoxR{0}
\fi

Но вернемся к Хельсинки. Американцы выполняют взятые на себя обязательства
только с сильными. Это их историческая самобытность. Но теперь ситуация
усугубляется еще двумя позициями. Первое – в Вашингтоне нарушен принцип
единоначалия. Трамп пишет одни твиты, в стиле «ах как хорошо уметь
договариваться и вместе дружить». Помпео тут же комментит, что никаких
компромиссов по Крыму с русскими не будет. Заметьте, это говорит даже не
сенатор-демократ, а формально и неформально прямой подчиненный первого лица,
отвечающий за внешнюю политику. Будь сейчас жив Ричард Никсон, он бы себя
чувствовал последним лохом на планете, уйдя в отставку из-за каких-то особо
недоказанных подозрений в подслушке оппонентов. В Штатах сейчас может нести
каждый все что угодно, главное не задевать чувства чернокожих и гомосексуалов.
Вот только договариваться теперь в Штатах не с кем.

И второе – проблема не только в том, что непонятно, с кем договариваться, но
еще и нет единой платформы, на которой можно строить эти договоренности. Чтобы
русские и американцы начали договариваться, им нужно начать жить по одним
физическим законам. Собственно, вест срач межу US и RU из-за этого и
проистекает. Не из-за Крыма, Сирии, вмешательства в выборы и мертвого
Магницкого. Это осколки этого разлома. Русские и американцы живут в разных
физических измерениях. Вот если в Хельсинки хотя бы начнут говорить о том, что
нужно вернуться к обсуждению единого для всех учебника по физике, тогда в
обозримой перспективе откроется площадка по решению всех остальных вопросов.

Ну а Украина, Сирия, Крым – это частные вопросы. Из них можно раздувать
конфликты вселенского масштаба. Обе мировые войны из-за меньших недоразумений
начинались, только потому, что какой-то из сторон нужна была война. А можно все
это не замечать и порешать на раз-два, когда есть на то желание и ресурсная
база.

И последнее. Мировые противоречия не ограничиваются конфликтом между
расшатанным Западом и оборзевшей Россией. Вообще, это только одно из
направлений разлома (пусть и самое видимое). И даже если Путин с Трампом о
чем-то «чудесно договорятся» в обозримом будущем, в глобальном плане мировой
американоцентризм всех уже заипал от глухой деревушки в Бурунди до финансового
квартала в Шанхае.

\ii{29_06_2018.fb.lesev_igor.1.vstrecha_v_helsinki.cmt}
