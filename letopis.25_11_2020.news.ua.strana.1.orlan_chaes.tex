% vim: keymap=russian-jcukenwin
%%beginhead 
 
%%file 25_11_2020.news.ua.strana.1.orlan_chaes
%%parent 25_11_2020
 
%%url https://strana.ua/news/303030-foto-ohromnoho-orlana-v-polete-u-pruda-chernobylskoj-aes.html
 
%%author 
%%author_id 
%%author_url 
 
%%tags 
%%title В Чернобыльском заповеднике показали фото огромного орлана в полете у пруда атомной станции
 
%%endhead 
 
\subsection{В Чернобыльском заповеднике показали фото огромного орлана в полете у пруда атомной станции}
\label{sec:25_11_2020.news.ua.strana.1.orlan_chaes}
\Purl{https://strana.ua/news/303030-foto-ohromnoho-orlana-v-polete-u-pruda-chernobylskoj-aes.html}

22:08, сегодня

\ifcmt
pic https://strana.ua/img/article/3030/30_main-v1606334459.jpeg
caption Редкий орлан-белохвост. Фото: Чернобыльский заповедник
\fi

Сотрудники Чернобыльского заповедника показали взрослого
орлана-белохвоста, которые очень редко встречаются в Украине. 

Об этом сообщает пресс-служба Чернобыльского радиационно-экологического
биосферного заповедника в Facebook.

\enquote{Существование орлана-белохвоста прочно связано с различными водоемами и,
одновременно, лесными массивами - хотя бы небольшими, но со старыми и
раскидистыми деревьями. Ведь эта большая хищная птица - отличный рыболов,
которому деревья нужны для гнездования}, - указано в сообщении. 

Отмечается, что на территории Чернобыльского заповедника орланы чувствуют
себя замечательно - для их удобства вокруг множество водоемов, лесов, а
также минимальная человеческая деятельность, что также повышает его шансы
на выживание. Есть здесь и красивые леса, и достаточно водоемов со
всевозможной живностью, а, главное, не беспокоят люди.

Данную птицу фотограф поймал в объектив около пруда-охладителя
Чернобыльской атомной электростанции. 

\ifcmt
pic https://strana.ua/img/forall/u/11/58/photo5330286038083547234.jpg
\fi

Напомним, сотрудники Чернобыльского заповедника показали фото воробьиного
сыча, который является самым маленьким представителем сов в Украине.
Средняя сыч-воробей может поместиться в ладони взрослого человека и весит
около 80 грамм, поэтому постоянно вынужден прятаться от хищников
побольше. 

Как сообщал \enquote{Страна}, сотрудники Чернобыльского заповедника показали фото
единственной в Украине черепахи, которая проживает в наших широтах в условиях
дикой природы. 


