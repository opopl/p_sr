% vim: keymap=russian-jcukenwin
%%beginhead 
 
%%file 13_02_2022.stz.news.ua.strana.1.vozduh_blokada.1.ishod_inostrancev
%%parent 13_02_2022.stz.news.ua.strana.1.vozduh_blokada
 
%%url 
 
%%author_id 
%%date 
 
%%tags 
%%title 
 
%%endhead 

\subsubsection{Исход иностранцев из Украины. В чем причина?}

На данный момент, 31 страна потребовала у своих граждан покинуть Украину как
можно быстрее. 

\begin{itemize}
\item США
\item Нидерланды
\item Япония
\item Южная Корея
\item Израиль
\item Черногория
\item Норвегия
\item Латвия
\item Великобритания
\item Эстония
\item Канада
\item Новая Зеландия
\item Австралия
\item Финляндия
\item Бельгия
\item Германия
\item Кувейт
\item Ирак
\item Италия
\item Испания
\item Литва
\item ОАЭ
\item Саудовская Аравия
\item Швеция
\item Словения
\item Иордания
\item Палестина
\item Марокко
\item Люксембург
\item Багамы
\item Болгария
\end{itemize}

При этом Германия перенесла свое консульство из Днепра во Львов. Канада
переводит туда все свое посольство.

В то же время посольства основных стран продолжают работать, хотя и сокращают
свой персонал. А в представительстве США, как утверждает немецкая газета Bild,
жгут секретные документы и уничтожают аппаратуру.

Впрочем, к данной информации стоит относиться с известной долей скепсиса.
\enquote{Вторжение} даже сами Штаты наметили только на 16 февраля, то есть
времени вывезти документы и без уничтожения остается достаточно. Да и раньше
эту работу могли провести - учитывая, что Вашингтон эту тему развивает еще с
прошлого года. 

Также оптимистично выглядит заявление Евросоюза, который заявил, что посольства
его стран закрыты не будут (хоть и не исключил, что персонал их могут
сократить). 

Тем не менее, весь этот массовый исход - один из главных факторов нагнетания
паники вокруг Украины. Впрочем, связан он с \enquote{вторжением} лишь опосредованно.

Заявления разных стран с призывами к своим гражданам покинуть Украину могут
быть связаны с намерением страховых компаний отменить в понедельник 14 февраля
гарантии по полетам над Украиной для авиаперевозчиков.

Слухи о такой отмене становятся все более настойчивыми. Особенно после решения
голландской компании KLM прекратить полеты в Украину. 

\enquote{Действительно, международные страховщики могут снять гарантии. После заявления
KLM о прекращении работы в Украине и массовом отзыве работников посольств
страховой рынок занервничал. Иностранцы-перестраховщики могут оставить решение
по гарантиям на страховщиках, но те вряд ли те станут брать на себя такую
ответственность. Да и авиакомпании вряд ли будут рисковать пассажирами и
самолетами. Поэтому мы увидим в прямом смысле пустое небо - не будут летать ни
иностранные, ни украинские перевозчики}, - сообщил \enquote{Стране} глава Ассоциации
страхового бизнеса Украины Вячеслав Черняховский.

Кроме того, владельцы частных самолётов сказали \enquote{Стране}, что им пришло
уведомление, что в течение 48 часов страховые гарантии на их борты будут
отозваны, а потому самолёты должны покинуть до истечения этого срока Украину.

\enquote{Пару часов назад от страховой компании пришло уведомление, что в течение 48
часов самолёт нужно вывести в европейскую зону. У меня борт был не в Украине,
поэтому я принял решение его не возвращать}, - сообщил \enquote{Стране} один из
политиков, в собственности которого есть частный самолёт.

Как сообщил «Стране» источник на страховом рынке, крупнейшие британские
страховые компании, у которых перестраховываются другие страховщики, сегодня
разослали всем лессорам мира (собственники самолетов, которые отдали их в
лизинг авиакомпаниям) официальное письмо, что через 48 часов перестаёт
действовать страховое покрытие в Украине и над Украиной для любого воздушного
судна. 

\enquote{Эти уроды (британские страховые компании - Ред.) вводят воздушную блокаду. Ни
один самолёт не вылетит из Украины и не влетит в Украину примерно с середины
дня понедельника}, - сообщил «Стране» источник.

Если страховые гарантии отзовут, международное авиасообщение с Украиной будет
закрыто.

Причем это будет сделано по инициативе западных структур.

То есть, никто не сможет улететь из Украины вне зависимости от того будет ли на
самом деле анонсированное \enquote{вторжение} или нет.

Вероятно, об этом уже знают иностранные правительства, а потому и призывают
своих граждан срочно покинуть страну. 

Другими словами Запад фактически готовит с понедельника воздушную блокаду
Украины. И это вполне реальный и очень болезненный удар. В отличие от пока еще
гипотетического \enquote{вторжения Путина}. Причем этот удар по нашей стране наносит не
\enquote{страна-агрессор}, а те государства, который называют себя \enquote{союзниками
Украины}.

