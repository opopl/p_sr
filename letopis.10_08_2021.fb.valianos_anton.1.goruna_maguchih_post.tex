% vim: keymap=russian-jcukenwin
%%beginhead 
 
%%file 10_08_2021.fb.valianos_anton.1.goruna_maguchih_post
%%parent 10_08_2021
 
%%url https://www.facebook.com/temnykua/posts/4059831194086168
 
%%author Valianos, Anton
%%author_id valianos_anton
%%author_url 
 
%%tags __aug_2021.maguchih.foto.olimpiada.lasickene,goruna_stanislav.ukr.sportsmen,lasickene_maria.rossia.sportsmenka,olimpiada.tokio,politika,rossia,sport,ukraina
%%title Вчора потрапив на очі пост депотата-каратиста Станіслава Горуни
 
%%endhead 
 
\subsection{Вчора потрапив на очі пост депотата-каратиста Станіслава Горуни}
\label{sec:10_08_2021.fb.valianos_anton.1.goruna_maguchih_post}
 
\Purl{https://www.facebook.com/temnykua/posts/4059831194086168}
\ifcmt
 author_begin
   author_id valianos_anton
 author_end
\fi

Вчора потрапив на очі пост депотата-каратиста Станіслава Горуни. Звернув увагу
на нього, бо він є просто класичним набором маніпуляцій та кліше стосовно
\enquote{бідолашної дитини-спорсменки}, яку зацькували диванні експерти(о
так, де мій диван). Приблизно такіж пости написали деякі українські(?)
політики, спортсмени та громадські діячі. Чесно кажучи ці пости віють такою
наївністю, що замість ляща, автора хочеться потріпати за щічку зі словами: ууу
ти мамкін експерт. Тож сонечкі мої, сідайте навколо, дід Антон віщати буде....

\ifcmt
  pic https://scontent-cdt1-1.xx.fbcdn.net/v/t1.6435-9/223197318_4059828907419730_5699195770488682916_n.jpg?_nc_cat=110&ccb=1-4&_nc_sid=730e14&_nc_ohc=PhRAf8M_G5MAX9LouQM&_nc_oc=AQl42KoWkJVOqP5tav4C0q6kPsTrhzw7N_hbOG4-M7alnBOP07f9PLLMCHJnMNAIhxo&_nc_ht=scontent-cdt1-1.xx&oh=48443b905354a78bbca053b4fa2c532a&oe=6139EA33
  width 0.4
	fig_env wrapfigure
\fi

Спорт поза політикою, спробую пояснити максимально просто, щоб в вас мозок не
закипів. Вищезгадане твердження було б вірним тільки в одному випадку: в період
між змаганнями спортсменів тримали десь в ізоляції. Але ж ні ! Спортсмени мають
пабліки з сотнями тисяч підписників, кожне їхнє слово слухають фанати і
підхоплюють журналісти, пишуть мемуари та знімають фільми. Ой вей! І ви мені
такі реально хочете сказати, що здатні проманеврувати між каплями дощу і не
пересіктись з політиним життям країни? В мене погані новини для вас. Нажаль(для
України), разом з авторитетом у спорті, ви набуваєте авторитет у інших питаннях
про які висловлюєтесь і з якими маєте справу, а для молодого покоління ви
стаєте кумірами і вони ледь не вашим інструкціям живуть. Стасік Горуна, ау,
спорт поза політикою? Ню ню.....

Політики посварили братерські народи. З метою уникнення бану буду вживати
термін узькі люди — тобто ті, які мають узьку душу, від того злобні, завісливі та
підступні. Кажуть узькімі людьми щильно заселено Мордор. Так от в мене є родичі
і були(!)  друзі, які живуть серед узьких людей і самі є узькими. Мене завжди
дивував рівень нікчемності людей, які сумують за совком. Мільйони людей знищено,
розв’язання війни, поламані долі і покоління рабів, нездатних на самостійне
мислення! Але дешева ковбаса! Але давайте спробуємо заспокоїтись і списати це
на нездатність аналізувати історичні джерела(давно ж було). Повірили в це? Ок.

А як бути з сьогоденням? Це ж все відбувається на наших очах. Майже в кожного
громадянина України є той, хто може розповісти про цю війну! І після цього ви
будете мило пити чай з родичами, які вам казатимуть: \enquote{Гражданская
вайна, Крим наш, путін маладєц}? Ви серйозно? Недолугим любителям братерського
узького народу ніяк не вдається зрозуміти, що війну веде не держави, а війну
ведуть два діаметрально різних між собою народи. Ця війна між нами не перша і
навіть не двадцять перша і обіймаючи узького, який перед тим не стояв на
колінах з вибаченнями — ти обіймаєш ворога. 

Пане депутат Горуна, ви знаєте хто такий Темур Юлдашев? Спортсмен, який загинув
в 2014. Чемпіон призер.... якби він вижив, ви гадаєте він би обіймався з
узькими? Ні! Бо він людина і в нього є гідність, про яку ти лише в книжках
читав. Як гадаєш, наскільки вище я був би у спорті, якби в 2014 залишився в
Києві? А так 3 роки на війні, а потім рік реабілітації від наслідків. На перший
чемпіонат Європи мені збирали кошти всім світом, бо я залишився ні з чим. Мій
дім окуповано! 4 роки мого життя забрали твої друзі, до яких ти полюбляєш
їздити з семінарами. Кнопку пуску Урагану по селу Хрящувате натискав не путін, а
звичайний узький.

От якось так..... хтось частину ноги досі не відчуває, а хтось \enquote{атстаньтє ат дєвачкі}

\ii{10_08_2021.fb.valianos_anton.1.goruna_maguchih_post.cmt}
