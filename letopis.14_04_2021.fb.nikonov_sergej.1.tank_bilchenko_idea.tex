% vim: keymap=russian-jcukenwin
%%beginhead 
 
%%file 14_04_2021.fb.nikonov_sergej.1.tank_bilchenko_idea
%%parent 14_04_2021
 
%%url 
 
%%author 
%%author_id 
%%author_url 
 
%%tags 
%%title 
 
%%endhead 
\subsection{Танк 1 - в поддержку Евгении Бильченко}
\label{sec:14_04_2021.fb.nikonov_sergej.1.tank_bilchenko_idea}
\Purl{https://www.facebook.com/alexelsevier/posts/1471695689842408}

Танк 1 (она просила "танки" в поддержку. пока поддерживаю умом). Для  поддержки
Евгении Бильченко. 

Начал смотреть ее лекцию по идеологии внутри нас. Мысль. Я ещё досмотрю. Сейчас
рассуждаю упрощенно. Она рассматривает связь двух факторов: а) внешнего
воздействия различных коммуникаций, влияния смыслов, передаваемых извне. б)
архетипов, присущих человеческому мировоззрению, его глубоких внутренних основ.

Так вот, я подумал, почему могла работать советская идеология  и почему не
работает наша? Да потому, что советская была удачным сочетанием того и другого
на основе действительности успешного государства. Возьмем, к примеру, лозунг:
"За Родину, За Сталина!" Здесь все сочетается. И благородный внутренний
человеческий мотивзащиты Отечества, осознание реальных побед советского
государства и Сталин как идеологический символ, передаваемый извне. Что  у нас:
армия, язык, вера. Тут попытка экслуатации любви к Родине, как святого чувства.
Но, что в реальности она значит для носителей всех языков кроме украинского и
верующих людей? Тезис про армию как защитницу не подтверждается фактажом, таким
который пробудил во мне желание бить врага как в 1941 г. Говорят, что на нас
нападают, но с Россией торгуют, и ее нападение не показывают. Наверное, она
применяет людей-невидимок.  Эта идеология действует на эмоции, у других
пробуждает душевные травмы, склонность к насилию и т.п. Вот вам враг и бейте.
Язык. Языковая позиция может быть архетипом для этнического украинца и вызывать
положительные эмоции, реакции, представление о языке неразрывно связано с
восприятием Родины. А для русского, венгра или другого часто  ассоциируется с
внешним, чуждым ему насаждением украинского языка. Для него русский язык- язык
Души. Плюя в него, плюют в душу или в Душу.   Вера. Вера вообще малоизменчивая.
Она столетиями не меняется. Считается данным, непогрешимым. И тут достают
какой-то чуждый людям Томос и, что ещё хуже, начинают отжимать храмы силой. Не
по-христиански решают вопрос, а силой. Здесь снова в душе человека происходит
столкновение глубоких убеждений с навязываемым чуждым.
