% vim: keymap=russian-jcukenwin
%%beginhead 
 
%%file 13_02_2022.fb.molchanov_jurij.1.jedinenie
%%parent 13_02_2022
 
%%url https://www.facebook.com/george.molchanov.9/posts/4684034465042818
 
%%author_id molchanov_jurij
%%date 
 
%%tags edinstvo,napadenie,rossia,strana,ugroza,ukraina
%%title Наконец-то произошло реальное единение страны
 
%%endhead 
 
\subsection{Наконец-то произошло реальное единение страны}
\label{sec:13_02_2022.fb.molchanov_jurij.1.jedinenie}
 
\Purl{https://www.facebook.com/george.molchanov.9/posts/4684034465042818}
\ifcmt
 author_begin
   author_id molchanov_jurij
 author_end
\fi

Три типичных реакции на происходящее:

1. Паника. Кто-то озабочен спасением кэша/бизнеса/шубы, выездом на запад
страны/планеты, параллельно накачивая децибелами собственной истерики все
доступные соц.сети.

2. Отстраненность. «Да пошло оно всё! Пойду примус починять». Кто-то с
рюмкой/стаканом/бокалом, кто-то в сериале, кто-то на тусе. Отвлекаются. Но
гаджеты из рук не выпускают. И соцсети.

3. Отрицание. Все херня. Байденвсёврёт/Путинвсёврёт. Никакой войны не будет,
потому что не будет.

Во всех трех категориях есть представители всех слоев общества. Всех религий и
политических предпочтений. Те, кто стоял на майдане, кто плевать на него хотел
и те, кому был симпатичен антимайдан.

Наконец-то можно смело сказать, что в стране произошло объединение. 

\enquote{Єдина країна/единая страна} - это когда никто нихера не может понять и
один общий пушной зверек на всех.
