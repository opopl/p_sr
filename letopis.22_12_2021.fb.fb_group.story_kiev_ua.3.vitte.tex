% vim: keymap=russian-jcukenwin
%%beginhead 
 
%%file 22_12_2021.fb.fb_group.story_kiev_ua.3.vitte
%%parent 22_12_2021
 
%%url https://www.facebook.com/groups/story.kiev.ua/posts/1824132621116864
 
%%author_id fb_group.story_kiev_ua,poljakova_galina.kiev
%%date 
 
%%tags istoria,kiev,rusimperia,vitte_sergej.rusimperia,zheleznaja_doroga
%%title Сергей Юльевич Витте
 
%%endhead 
 
\subsection{Сергей Юльевич Витте}
\label{sec:22_12_2021.fb.fb_group.story_kiev_ua.3.vitte}
 
\Purl{https://www.facebook.com/groups/story.kiev.ua/posts/1824132621116864}
\ifcmt
 author_begin
   author_id fb_group.story_kiev_ua,poljakova_galina.kiev
 author_end
\fi

Я продолжаю серию миниатюр, посвященных людям, чья жизнь и деятельность были
связаны с Украиной. В Киеве, на тихой улочке Леонтовича, у входа в Министерство
Ж/Д установили памятный знак – бюст Витте. И это справедливо. 

\begin{zznagolos}
«Родись Витте американцем — он стал бы миллиардером, в диких прериях собрал бы 

неисчислимые стада, в Калифорнии открыл бы

золотую жилу, среди индейцев стал бы

вождём, среди разбойников — атаманом.»

Из воспоминаний современников
\end{zznagolos}

Сергей Юльевич Витте украинцем не был, но так и не смог избавиться от
характерного украинского акцента, чем жутко раздражал светский Петербург. Среди
его предков были голландцы, курляндцы, князья Долгоруковы... Кстати, двоюродной
сестрой была Елена Блаватская, к которой он относился весьма скептично. 

\ii{22_12_2021.fb.fb_group.story_kiev_ua.3.vitte.pic.1}

Витте закончил физико-математический факультет университета в чудном городе
Одесса и по совету дяди перебрался в Киев. И устроился на Юго-Западную железную
дорогу каким-то совершенно мелким чиновником, чтобы постигать дело с самого
низу. Вник. Изучил. И вскоре Витте орал на собственное начальство, дескать,
если они собираются угробить царя, позволяя тяжеленому царскому поезду мчаться
с огромной скоростью, то пусть избегают его участок. Аудитория внимала молча.
Заподозрив неладное, оглянулся: за его спиной стоял Александр III и тоже
внимательно слушал. Вскоре из Петербурга пришло приглашение возглавить
столичное ведомство. Витте сложил два и два и отказался, поскольку на
государственной службе платят гораздо меньше. Царь тоже сложил два и два и
пообещал доплачивать из своего личного кармана. Поехал. 

Царь не ошибся. Дела закипело. Железные дороги строились по всей стране, был
разработан Устав Российских железных дорог, железнодорожные тарифы, потом еще и
таможенные, проводилась передовая для того времени маркетинговая политика. Его
главной задачей была не столько эксплуатация железных дорог, сколько выкуп их в
казну для создания единой общегосударственной транспортной системы — по складу
мышления Витте был государственником.

Мыслил нетривиально и не гнушался деталями. По сей день чай в поезде подают в
подстаканниках. А это придумал Витте: чтобы посуда не опрокидывалась и не
съезжала на пол при движении поезда. 

Увидел печальную красавицу в театре. Влюбился. О разводе договорился с ее
мужем, подкрепив предложение пачкой денег. Царю немедленно доложили кошмарную
новость: министр Витте женится на разведенной еврейке! Александр III ответил в
свойственном ему духе: «Да хоть на козе, лишь бы дело делал». Но запомнил.
Позднее просил у Витте совета, как поступить с евреями? Проблем-то много. Тот
ответил просто: либо утопить всех в Черном море, либо предоставить им равные
права перед законом. Царь, видимо, решил не спешить. 

Потом Сергей Юльевич стал Министром финансов. Лучшего министра в России не было
ни до, ни после. Провел денежную реформу, сделав рубль конвертируемым. Таким
мощным рубль больше уже не был никогда. Сам Витте стал одним из самых уважаемых
политиков для всей Европы. Его называли «дедушкой российской индустриализации».
А для индустриализации нужны были кадры. К Витте обратилась делегация киевлян с
ходатайством об открытии в Киеве среднего технического учебного заведения.
Министр финансов ответил: «Открытие в Киеве среднего технического училища –
вчерашний день. Сегодня городу и всему вашему краю крайне необходимо инженерное
училище, то есть политехнический институт». И Киевский политех был создан.
Витте не скрывал свою любовь к Киеву.

Царь поручил Витте преподавать наследнику, будущему Николаю II, финансовое
дело. Витте оказался суровым и требовательным преподавателем: Николай II его
возненавидел. Взойдя на трон, старался пореже вспоминать, что отец завещал
следовать советам Витте.

В 1903 г. Витте был назначен на пост председателя Комитета министров. Эта
должность была фактически почётной отставкой, так как Комитет до революции 1905
года не имел никакого значения. Однако в трудную минуту, когда Россия понесла
тяжелое поражение в русско-японской войне, именно ему поручили возглавить
российскую миссию на мирных переговорах с Японией. Длительные и сложные
переговоры проходили в Портсмуте, США. Умнейший Витте сумел обворожить не
только американских посредников, но и членов японской делегации, и условия мира
стали неожиданно благоприятными для России. За эту дипломатическую победу Витте
был произведен в графское достоинство: Граф Сахалинский. 

В громадной империи назревал кризис. Витте обратился к Николаю II: «Россия
переросла форму существующего строя. Она стремится к строю правовому на основе
гражданской свободы». Додавил. Под руководством Витте был составлен манифест от
17 октября 1905 года, которым народу были дарованы свобода совести слова,
собраний, союзов и неприкосновенность личности. Российская империя получила
шанс превратиться в демократическую страну. Это было апогеем государственной
деятельности Витте. Такого безобразия ему простить не смогли и отправили в
полную отставку. 

На досуге занялся написанием мемуаров. Эта новость привела в смятение Двор.
Началась охота за рукописью, даже покушения. После смерти Витте в 1915 году его
вдова, Матильда Витте, сумела сохранить рукопись, а после революции книгу
издала, снабдив умным и смелым предисловием. 

Будем помнить Сергея Юльевича Витте, дерзкого министра-государственника. А еще
– он был киевлянином.

\ii{22_12_2021.fb.fb_group.story_kiev_ua.3.vitte.cmt}
