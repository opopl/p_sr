%%beginhead 
 
%%file 13_10_2018.fb.fb_group.mariupol.biblioteka.korolenka.1.plastilinografiya___
%%parent 13_10_2018
 
%%url https://www.facebook.com/groups/1476321979131170/posts/1902611256502238
 
%%author_id fb_group.mariupol.biblioteka.korolenka,kibkalo_natalia.mariupol.biblioteka.korolenko
%%date 13_10_2018
 
%%tags plastilin,mariupol,master_klass
%%title Пластилинография – это интересно!
 
%%endhead 

\subsection{Пластилинография – это интересно!}
\label{sec:13_10_2018.fb.fb_group.mariupol.biblioteka.korolenka.1.plastilinografiya___}
 
\Purl{https://www.facebook.com/groups/1476321979131170/posts/1902611256502238}
\ifcmt
 author_begin
   author_id fb_group.mariupol.biblioteka.korolenka,kibkalo_natalia.mariupol.biblioteka.korolenko
 author_end
\fi

Пластилинография – это интересно!

Пластилин - материал, знакомый каждому с детства. С его помощью можно создавать
настоящие шедевры – лепные картины с изображением полуобъемных предметов и
называется  это пластилинографией. 13 октября  участники клуба «Нand Мade»
Центральной библиотеки им. В.Г. Короленко занимались именно этим удивительным
видом декоративно-прикладного искуссства. Мастер-класс по пластилинографии
проводила Жанна Юрковская. Свое первое после летних каникул занятие в «Нand
Мade» она провела в образе романтической «тетушки Жаннет». Интересно, что
Татьяна Чебанова – одна из самых способных «учениц» - очень плодотворно провела
лето, создав целый «пластилиновый зоопарк». Участники мастер-класса трудились с
большим воодушевлением  и  старанием.
