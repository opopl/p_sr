% vim: keymap=russian-jcukenwin
%%beginhead 
 
%%file 24_07_2018.fb.lesev_igor.1.chto_posle_juli
%%parent 24_07_2018
 
%%url https://www.facebook.com/permalink.php?story_fbid=2013256638705429&id=100000633379839
 
%%author_id lesev_igor
%%date 
 
%%tags politika,timoshenko_julia,ukraina
%%title Что после Юли?
 
%%endhead 
 
\subsection{Что после Юли?}
\label{sec:24_07_2018.fb.lesev_igor.1.chto_posle_juli}
 
\Purl{https://www.facebook.com/permalink.php?story_fbid=2013256638705429&id=100000633379839}
\ifcmt
 author_begin
   author_id lesev_igor
 author_end
\fi

Что после Юли?

Прикинем, что победа Юлии Тимошенко на президентских неминуема. Есть, конечно,
варианты, но лично я чем больше их изучаю, тем больше возвращаюсь к базовому
тезису. Итак, ЮВТ уже следующим летом принимает присягу президента Украины. Но
что же дальше?

\ifcmt
  ig https://scontent-lhr8-2.xx.fbcdn.net/v/t1.6435-9/37744583_2013256105372149_4488077959160135680_n.jpg?_nc_cat=102&ccb=1-5&_nc_sid=730e14&_nc_ohc=d847PPtYHjwAX876id4&_nc_ht=scontent-lhr8-2.xx&oh=7cb88fa7f4714fc80a35d422d11ef786&oe=61B839D0
  @width 0.4
  %@wrap \parpic[r]
  @wrap \InsertBoxR{0}
\fi

Первое. Никакой сатисфакции не будет. Ну если, конечно, для вас сатисфакция –
это персональное раздевание таких славных персонажей, как Порошенко, Кононенко,
Луценко и ребят помельче. Юля – это уже Аваков + электрификация всей страны.
Возможно, даже Турчинов останется на своем месте. Да, поредеют перспективы
винницких – Гройсман станет к осени (следующей) рядовым парламентарием,
«фронтовики» потеряют все свои текущие профиты в виде минюста и оборонки.
Каких-то особо токсичных, вроде Пашинского, даже «упустят» на ПМЖ в Италию.
Только не забывайте, что все эти пашинские вышли из Юли. И союз НФ с Юлей также
уже ситуативно сложился.

Вторая тема – кадровая. У Юли кадров нет. Стоит посмотреть на текущий список
«Батькивщины» в Раде, чтобы понять непреложную истину – кадровый голод у
Тимошенко в разы жестче, нежели у Порошенко. По сути, она будет вертеть все тот
резерв, с косметическими правками на вип-уровне. Это значит, что качество в
лучшем случае останется таким же.

Третье. Донбасс. Все хотят там мира. Но мир на Донбасса – это путь к системному
переформатированию всей политической модели в Украине. Не будем рассуждать,
хорошо это для государственности UA или плохо. Просто отметим, что Юля
побеждает при текущей модели. А при обновленной она уходит в утиль. Поставьте
для себя личный вопрос. Или интересы любимой вами компании, или ваше место в
этой самой компании – что важнее? Вот для вас лично? Для Юли в масштабах
Украины-компании ценностные ориентиры абсолютно идентичны.

Это не значит, что по Донбассу подвижек не будет. Но будут они а) со стороны
внешних игроков и б) в очень скором времени сама Юля превратится в Петю и будет
действовать ТОЧНО ТАКЖЕ. Т.е., никак.

Социалка. Ну тут я вас просто умоляю. С приходом Юли к власти Украина начнет
выплачивать максимальные кредитные обязательства. У Тимошенко будет время,
чтобы завести в парламент еще какое-то вменяемое большинство. Затем картина
будет такая же. Нелюбимая баба-ведьма во главе государства. И парламент,
состоящий из одних ублюдков. Все как сейчас, только с экивоком на 2019 год.

Изменения формы государственного правления. Парламентская республика.
Децентрализация с признаками федерализации… Ах-ха-ха. Кто хоть мега-отдаленно
пересекался с «юльками» в работе, сам же ответит на эти вопросы – полный бред.
Юля натягивает. Но не отдает. Да, есть определенные варианты. Например,
согласованное внешнее давление, плюс внутренний олигархат. Но Юле обязательно
надо гарантировать канцлер-сидушку. Но опять же, тут куча условий. Сначала надо
выбрать новый парламент, где Юля при своих. Затем протолкнуть конституционные
изменения. Но ведь и вступают эти изменения не сразу, а после того как ЮВТ
может соскочить с поста президента в пост премьера. А для этого нужны или новые
досрочные (а кто на них пойдет?), или еще один 5-летний цикл.

А значит, все эти сусальные разговоры об уменьшении президентских полномочий
актуальны только до тех пор, пока Юля остается не президентом. Потом все будет
то же самое, только чуть-чуть наоборот.

Из всего этого не следует, что Юля хуже Пети. Хуже Пети не может быть даже сам
Петя. Но Юля все равно остается ставленником одной цивилизационной группы. Она
не решит вопросы, да и не может их решить при всем желании, которые разъединяют
страну – язык, религия, мировоззренческое инакомыслие. Юля, как и Петя, будет
все также озираться на уличную толпу, которая во многом и приведет ее во
власть. А затем эту же власть у нее максимально ограничивать. В итоге, из
умеренного популиста, Тимошенко превратится неминуемо в популиста уровня
Порошенко. Один в один, только разве чуть задница изящней.

Функция Юли – это поменять нелояльных чиновников, харчующихся с потоков, на
таких же, но своих. Возможно, они будут менее наглые и не такие тупые.
Возможно. Но для пациента цвет утки все же не столь важен. И вот тут я в конце
обязательно должен написать что-то особенно умное. Ну так ловите – за кого не
голосуй, все равно в итоге получится Петя.

\ii{24_07_2018.fb.lesev_igor.1.chto_posle_juli.cmt}
