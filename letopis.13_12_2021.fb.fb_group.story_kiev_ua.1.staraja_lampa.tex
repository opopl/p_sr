% vim: keymap=russian-jcukenwin
%%beginhead 
 
%%file 13_12_2021.fb.fb_group.story_kiev_ua.1.staraja_lampa
%%parent 13_12_2021
 
%%url https://www.facebook.com/groups/story.kiev.ua/posts/1817991488397644
 
%%author_id fb_group.story_kiev_ua,novikov_victoria.kiev
%%date 
 
%%tags akvarel,kiev,risunok
%%title Старая лампа
 
%%endhead 
 
\subsection{Старая лампа}
\label{sec:13_12_2021.fb.fb_group.story_kiev_ua.1.staraja_lampa}
 
\Purl{https://www.facebook.com/groups/story.kiev.ua/posts/1817991488397644}
\ifcmt
 author_begin
   author_id fb_group.story_kiev_ua,novikov_victoria.kiev
 author_end
\fi

Старая лампа 

Случайно попалась мне на глаза эта акварель. На ней нет ни одного предмета,
который бы не напомнил мне о дедушке и бабушке. Мы жили вместе в одной квартире
в дореволюционном доме за Костелом. 

\ii{13_12_2021.fb.fb_group.story_kiev_ua.1.staraja_lampa.pic.1}

И хотя у нас была большая комната, мы с мамой почти каждый вечер проводили у
них. Папа работал по вечерам и мы сбегали от шума печатной  машинки.

На круглом столе в центре их комнаты стояла точно такая же лампа. Я думаю, что
ею перестали пользоваться много лет назад, но она продолжала жить на столе.  На
ней лежали очки. У меня пар двадцать очков для чтения. Я никогда не могу найти
ни одной. А у дедушки у очков было постоянное место. У них вообще было всегда
всего так мало, что все вещи имели своё место. В комнатке десятиметровой был
идеальный порядок.

Дедушка признавал только подстаканник ( никаких чашек) и чай пил только
вприкуску. Чай должен был быть не просто горячим, а кипятком. Все, что холоднее
кипятка, было не чаем. Попробуй, неся чай по длиннющему коридору коммунальной
квартиры, сохранить температуру кипения?! Но бабушка каким-то образом
умудрялась ему угождать. Подстаканник и вазочка с кусковым сахаром ( очень
похожая на эту солонку) - это дедушка.  

А вот блокнот с рецептами- это бабушка. Он был такой потрёпанный, что листочки
уже не держались вместе. Они пожелтели, уголки истрепались. Она знала все свои
рецепты напамять и заглядывала, по-моему, в блокнотик, чтоб рассказать кто и
при каких обстоятельствах дал ей тот или иной рецепт. 

«Что бы мне придумать на завтра?! Вот! Сделаю кисло-сладкое. Это рецепт бабушки
Кати. Какая она была хозяйка! Ее  мясо елось губами. И не только потому, что
она делала базар на Бессарабке !»

И переходила к  истории о бабушке Кате и  ее многочисленных мужьях .

А если мы вместе смотрели фильм, то бабушка с мамой обязательно вспоминали на
кого похожи актёры. «Это же  копия Толи, который жил под нами на
Константиновской до войны! Какая у него была красавица жена! Он погиб на
фронте, а она из эвакуации поехала к сестре в Кишинёв !» 

А дедушка восклицал: «Это не актриса! Вера Холодная- вот это была актриса!
Какие это были фильмы! Тапёр, помню, играет что-то медленное и грустное, она
идёт по мосту и вдруг ОН! Тапёр замер  ....Ее лицо крупным планом ...»

И вот так в комнате, в которой был стол, а на нем старая лампа с очками,
подстаканник и вазочка с кусковым сахаром, и проходили вечера.

Чуть не забыла! Кисточки тоже были на столе. А рядом стакан с водой, чтоб их
мыть, и краски. И я что-то раскрашивала, или делала аппликации из бархатной
бумаги, которую мне покупали в в канцтоварах напротив Владимирского рынка, и
слушала все эти майсы с открытым ртом. 

Это было  так хорошо, так тепло, что я теперь даже не уверена, что это  все
было на самом деле....

©️ Victoria Novikov

@igg{fbicon.artist.palette} Любовь Титова

\ii{13_12_2021.fb.fb_group.story_kiev_ua.1.staraja_lampa.cmt}
