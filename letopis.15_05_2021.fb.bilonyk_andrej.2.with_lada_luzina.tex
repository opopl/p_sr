% vim: keymap=russian-jcukenwin
%%beginhead 
 
%%file 15_05_2021.fb.bilonyk_andrej.2.with_lada_luzina
%%parent 15_05_2021
 
%%url https://www.facebook.com/bilonyk.a/posts/3930446817068393
 
%%author 
%%author_id 
%%author_url 
 
%%tags 
%%title 
 
%%endhead 

\subsection{Феєричне побачення з найвідомішою Києвицею Lada Luzina біля пам'ятника Булгакова}
\Purl{https://www.facebook.com/bilonyk.a/posts/3930446817068393}


\ifcmt
  pic https://scontent-frt3-1.xx.fbcdn.net/v/t1.6435-0/p526x296/186553392_3930446723735069_2281843646927201395_n.jpg?_nc_cat=102&ccb=1-3&_nc_sid=8bfeb9&_nc_ohc=L94uOsd72RwAX-EgjMj&_nc_ht=scontent-frt3-1.xx&tp=6&oh=2f0f0196ba57a99065aee793b33c46a9&oe=60C476FC
\fi

Феєричне побачення з найвідомішою Києвицею Lada Luzina біля пам'ятника
Булгакова на Андріївському узвозі в 3:13 p.m.

Нарешті зима і локдаун закінчилися, і я зміг подарувати присвячений ще до Дня
письменника подарунок - листівку-репродукцію картини Вільгельма Котарбінського,
ще одного київського містика 

Варто  відмітити, що про існування самого Котарбінського я на свій сором
дізнався саме від Лади, і як тільки я побачив цю листівку з едельвейсом і
привітанням з Новим роком, я зрозумів, що вона повинна бути саме в неї.

Як правильно підмітили Лада, Михайло Афанасійович точно не образиться від
такого сусідства.
