% vim: keymap=russian-jcukenwin
%%beginhead 
 
%%file 26_11_2020.news.ru.lenta_ru.1.mass_covid_vaccination
%%parent 26_11_2020
 
%%url https://lenta.ru/news/2020/11/26/vac/
 
%%author 
%%author_id 
%%author_url 
 
%%tags covid,vaccine,sputnik_v,russia,kreml
%%title Кремль анонсировал начало массовой вакцинации от коронавируса
 
%%endhead 
 
\subsection{Кремль анонсировал начало массовой вакцинации от коронавируса}
\label{sec:26_11_2020.news.ru.lenta_ru.1.mass_covid_vaccination}
\Purl{https://lenta.ru/news/2020/11/26/vac/}

\index[rus]{Коронавирус!Россия!Вакцинация}
\index[names.rus]{Песков, Дмитрий!Пресс-секретарь Президента России}

\ifcmt
pic https://icdn.lenta.ru/images/2020/11/26/13/20201126132104392/pic_899a166731b15505cf48b4b91c08da8a.jpg
caption Фото: Виктор Коротаев / «Коммерсантъ»
\fi

Массовая вакцинация от коронавируса в России начнется до Нового года. Ее
анонсировал представитель Кремля Дмитрий Песков, передает ТАСС.

По словам пресс-секретаря президента России, вакцинация будет проходить
поэтапно и с учетом географических особенностей страны.

\begin{leftbar}
	\bfseries
	«Будет определенная этапность, но задача состоит в том, чтобы были
				минимальные этапы, и как можно быстрее обеспечить [вакциной] всех
				желающих», --- уточнил Песков
\end{leftbar}

Сроки, когда все желающие получат возможность сделать прививку от COVID-19,
пока называть рано, считает Песков. По его словам, это зависит от географии
нашей страны и особенностей хранения вакцин.  

\subsubsection{Проблемы с производством Кремлю известны}

В Кремле известно о проблемах ряда российских фармкомпаний в связи с запуском
производства вакцин.

«Кто-то более преуспел, кто-то менее, у каких-то компаний получается, у
каких-то нет, это абсолютно нормальный процесс», --- уверен пресс-секретарь
президента.

Он подчеркнул, что «постепенно увеличивается объем биореакторов, которые
задействуются, в них готовится второй компонент [вакцины]».

\subsubsection{Какой объем препарата произведут к началу вакцинации --- пока неясно}

По словам представителя Кремля, вопрос о планах по количеству вакцин следует
направить в министерство промышленности и торговли, которое ответственно за
производство препаратов для прививок.

Также Песков напомнил о состоявшемся недавно совещании главы правительства
Михаила Мишустина с производителями вакцин.

«Это то, чем сейчас занимаются штаб и наши производители постоянно», --- заверил
Песков, подчеркнув, что «таймлайны обсуждаются, но они должны при необходимости
обнародоваться самим штабом».

\subsubsection{В оборот уже выпущено более сотни тысяч доз вакцины}

Ранее вице-премьер Татьяна Голикова заявляла, что массовая вакцинация от
коронавируса в России планируется с начала нового года. По ее словам, уже
выпущены в оборот более 117 тысяч доз отечественной вакцины «Спутник V», до
конца года планируется произвести более двух миллионов доз.

По словам производителей российской вакцины от коронавируса, эффективность
«Спутник V» превысила 95 процентов уже на 42-й день после первой инъекции (21-й
день после второй инъекции). К этому времени у добровольцев сформировался
устойчивый иммунный ответ.

В России на сегодняшний день зарегистрированы две вакцины против COVID-19.
Первым регистрацию прошел препарат «Спутник V». 14 октября президент Владимир
Путин объявил о регистрации «ЭпиВакКороны» от центра «Вектор». В стране также
разрабатывается третий препарат --- этим занимается центр имени Чумакова.

\index[rus]{Коронавирус!Вакцина!Спутник V}
\index[rus]{Коронавирус!Вакцина!ЭпиВакКорона}

Планируется, что массовый выпуск вакцины начнется в феврале 2021 года.

