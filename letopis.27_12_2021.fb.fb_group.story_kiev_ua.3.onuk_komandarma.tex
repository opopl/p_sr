% vim: keymap=russian-jcukenwin
%%beginhead 
 
%%file 27_12_2021.fb.fb_group.story_kiev_ua.3.onuk_komandarma
%%parent 27_12_2021
 
%%url https://www.facebook.com/groups/story.kiev.ua/posts/1827705147426278
 
%%author_id fb_group.story_kiev_ua,fedjko_vladimir.kiev
%%date 
 
%%tags kiev,pamjat,rkka,semja,sssr
%%title 30 «дітей лейтенанта Шмідта» і один «онук командарма Федько»...
 
%%endhead 
 
\subsection{30 «дітей лейтенанта Шмідта» і один «онук командарма Федько»...}
\label{sec:27_12_2021.fb.fb_group.story_kiev_ua.3.onuk_komandarma}
 
\Purl{https://www.facebook.com/groups/story.kiev.ua/posts/1827705147426278}
\ifcmt
 author_begin
   author_id fb_group.story_kiev_ua,fedjko_vladimir.kiev
 author_end
\fi

30 «дітей лейтенанта Шмідта» і один «онук командарма Федько»...

// Прощавай «гражданка»

10 січня 1966 року мені виповнилося 19 років! Жодних планів на найближче
майбутнє я не будував – у травні або червні на мене чекав весняний призов в
Радянську Армію.

Виходячи з цього, я максимально використовував час, насолоджуючись свободою...
Катався на ковзанах на ковзанці стадіону «Динамо» та на лижах у Пуща-Водицькому
лісі; їздив на тренування по мотокросу у Феофанію; вечори проводив з Іринкою Н.
та з друзями. Про майбутнє я з дівчиною не говорив. Попереду було три роки
розлуки і клястися у вічному коханні та пов’язувати себе зобов’язаннями
вірності ми вважали недоцільним. Життя є життя. Якщо ми збережемо духовну
потребу бути разом, то відновимо стосунки і вступимо в шлюб. Якщо ж ні, то у
кожного своя доля, а ми залишимося друзями. 

У травні – червні військкомат мене не турбував і я зрозумів, що потрапляю в
осінній призов, який проходив у листопаді – грудні.

Літо пройшло у тренуваннях, поїздках на різні змагання та в розвагах. Наш
тренер, Тимошков, сказав, що у мене існує теоретична можливість потрапити у
спортивну роту і проходити службу в Києві. 

Наприкінці вересня я якось зайшов у Печерський районний військкомат, до якого
був приписаний з 1964 року, і поцікавився коли мене призвуть в армію – у
листопаді чи в грудні? Отримав відповідь, що одразу після листопадових свят.

1 жовтня 1966 року я звільнився з роботи. Однією з причин була травма нашого
тренера, який зламав ногу і всі тренування відмінили на місяць – півтора.
Оскільки на «Арсеналі» я \enquote{працював} токарем 2-го розряду у цеху № 22, який був
експериментальним і секретним, то крім підписання обхідного листка мені
необхідно було пройти бесіду у Першому відділі заводу. Пікантною обставиною
було те, що зайнятий спортивними справами я абсолютно не був у курсі, яку
експериментальну продукцію виробляли у нашому цеху. Начальник же відділу,
наголошуючи про що я не повинен нікому (навіть особістам в армії!) розповідати,
фактично розповів мені, що і для чого виробляли в цеху. Після ґрунтовної бесіди
я підписав документ про нерозголошення таємниць, що стали мені відомі за час
роботи на заводі, і отримав повний розрахунок. Гроші були непогані по тим часам
– 70 рублів СРСР! Моє «пролетарське» життя закінчилося! 

***

Перед ритуальними радянськими святами – 7-8 листопада – я отримав повістку на 9
листопада з’явитися у військкомат з паспортом, приписним свідоцтвом і двома
фотокартками (як на паспорт). 

Приходжу... Таких, як я, на подвір’ї військкомату тусується чоловік 40... Багато
хто з батьками. Я ж категорично відмовився, щоб мене проводжали родичі.
Підійшла моя черга... У мене забрали паспорт, приписне свідоцтво, фотографії і
виписали військовий квиток та обліково-послужну картку. Після чого мені видали
повістку на 10 листопада, попередили про кримінальну відповідальність за
ухиляння від призову на військову службу та сказали, щоб я прийшов на відправку
у старій одежі, яку можна викинути. Починався новий період життя – військовий!

// 30 «дітей лейтенанта Шмідта» і один «онук командарма Федько»

Усього три тижні пройшло, як я одягнув військову форму. Учбова дивізія в Острі...
Перерва між заняттями. Сидимо в «курилці»... Я не курив, але сидів з курцями, щоб
не переривати почату раніше розмову... Проходить повз «курилку» замполіт полку...
Підсів до нас, закурив... Прислухається до наших розмов... 

І раптом питає: «А скажіть, курсант Федько, командарм Федько не ваш родич»? 

Що у мене в голові перемкнуло я не знаю, але ляпнув: «Дідусь»! 

Справа в тому, що батько мій – Федір Костянтинович Федько. 

Командарм – Іван Федорович Федько.

А я Володимир Федорович Федько.

Коротше, ім’я Федір як ключик, що нібито пов’язує нашу трійцю! 

В той же час, командарм Федько народився у 1897 році, а мій батько у 1906.
Таким чином, командарм аж ніяк не міг бути моїм дідусем! (Але цю арифметику я
пізнав вже пізніше...)

Замполіт більше нічого не запитував. І я забув про цю історію.

Одними із моїх улюблених книжок юності були «Дванадцять стільців» і «Золоте
теля»! 

Цілком допускаю, що я ляпнув «Дідусь!», на запитання замполіта про
спорідненість з командармом Федько, чисто з присутнього мені з дитинства
почуття гумору. Не будемо забувати, що за три тижні я ще не звик до армійської
дисципліни!

***

// 10 діб арешту, які я провів на кухні...

Наступного року, на 23 лютого, був урочистий вечір. Полковий оркестр грає
військові марші... Під музику ми розсідаємося у великій залі клубу... Команда
піднятися... Ми встаємо і завмираємо по команді «Струнко!»... У залу вноситься
полковий прапор... Виголошуються доповіді про створення Червоної Армії та бойові
успіхи під час Великої Вітчизняної війни; про історію створення Учбової дивізії
«Остер»...

Виступають ветерани... І замполіт раптом каже: «У мене є сюрприз. В нашому полку
служить онук прославленого командарма громадянської війни Івана Федоровича
Федько – Володимир Федорович Федько. Зараз він розповість про свого дідуся»!

Я піднімаюся і кажу: «Я щиро вибачаюся, але я тоді невдало пошуткував. Я не маю
спорідненості з командармом Федько»!

Замполіт червоніє, потім блідне! Хвилина мовчання! Металевим голосом (майже як
Левітан, що читав фронтові зведення від Радянського Інформбюро) підполковник
карбує: «Курсант Федько! Рівняйсь, струнко! Десять діб арешту за нахабну брехню
про спорідненість з прославленим командармом, героєм громадянської війни Іваном
Федоровичем Федько, з відбуванням покарання на гарнізонній гауптвахті»!

Стою, руки по швах, голосно відповідаю: «Єсть! Десять діб арешту за брехню з
відбуванням покарання на гарнізонній гауптвахті!»

Після зібрання з ротою йду на вечерю, потім повертаємось в казарму. Настрій
невеселий – чекаю коли принесуть записку про арешт і мене відведуть на «губу»
(гарнізонну гауптвахту). Година тягнеться як тиждень... Нарешті старшина викликає
мене і веде до командира роти. Майор питає, як була справа... розповідаю... Він
ледь стримує посмішку і відпускає мене в казарму очікувати на свою долю.

Записки на арешт не поступило і після відбою лягаю спати з усіма. Наступного
дня, в перерві між заняттями, командир роти повідомляє, що арешт на десять діб
замінено мені на десять нарядів поза чергою на кухню. (Вже потім я дізнався, що
замполіт полку перевищив свої повноваження – він не мав права на покарання у
вигляді арешту на 10 діб! Крім того, була і його вина – він повинен був
ретельно перевірити мої слова).

Краще б я відсидів на гауптвахті! 

Наряд виглядав так... Після команди «Відбій!», коли всі лягали спати, я йшов на
кухню і поступав у розпорядження чергового по кухонному наряду до ранку!
Переодягався у підмінне х/б (робочий одяг), далі чистив картоплю (пару мішків),
моркву, цибулю... Мив і різав чищені овочі... Мив кухонний посуд... Мив котли... І так
до 5:30 ранку. Потім мився у душі, переодягався у повсякденну одежу і йшов у
свою роту, яка  піднімалася у 6:00. І цілий день занять з ротою! 

І так, через добу на кухню, пройшло двадцять діб!

Єдиною приємністю було те, що під ранок кухонному наряду старшина дозволяв
смажити картоплю на салі (свині відгодовувалися у підсобному хазяйстві при
кухні). Смаженої картоплі в раціоні харчування не було. 

Яка ж це була смакота! Смажена картопля плюс шматок свіжого сала та
свіжовипечений чорний хліб з часником!

З роками спогади про втому і недосипання розчинилися у пам’яті, а насолода від
такої нічної їжі залишилася на все життя!

***

\ii{27_12_2021.fb.fb_group.story_kiev_ua.3.onuk_komandarma.cmt}
