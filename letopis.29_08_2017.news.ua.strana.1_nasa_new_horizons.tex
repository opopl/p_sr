% vim: keymap=russian-jcukenwin
%%beginhead 
 
%%file 29_08_2017.news.ua.strana.1_nasa_new_horizons
%%parent 29_08_2017
%%url https://strana.ua/news/89603-v-ssha-hotovjat-novoe-poslanie-dlja-inoplanetjan-kotoroe-ujdet-na-bort-mezhplanetnoj-stantsii-new-horizons.html
 
%%endhead 

\subsection{В США готовят новое послание для инопланетян, которое уйдет на межпланетную станцию New Horizons}
\label{sec:29_08_2017.news.ua.strana.1_nasa_new_horizons}

\url{https://strana.ua/news/89603-v-ssha-hotovjat-novoe-poslanie-dlja-inoplanetjan-kotoroe-ujdet-na-bort-mezhplanetnoj-stantsii-new-horizons.html}

08:43, 29 августа 2017

В США начался сбор средств на отправку еще одного послания инопланетным
цивилизациям с помощью межпланетной станции NASA New Horizons, передает канал
NBC.

О сборе средств объявил Йон Ломберг, один из участников создания "Золотых
пластинок" с информацией о Земле и человечестве, которые установлены на
кораблях Voyager 1 и Voyager 2, уже вышедших за пределы Солнечной системы. 

Станция New Horizons, направляющаяся сейчас в пояс Койпера - колоссальное
скопление малых небесных тел, оставшихся после формирования Солнечной системы -
в конечном счете также выйдет в межзвездное пространство.

Как считает Ломберг, не следует упускать возможность направить в бортовой
компьютер этой станции информацию, которую когда-нибудь могли бы расшифровать
инопланетяне. В радиоимпульс, который он называет "Посланием Единой Земли",
можно было бы включить фотографии и другую информацию о нашей планете. Ломберг
и группа его сторонников уже начали собирать методом краудфандинга \$72 тысяч,
необходимые для поддержания сайта, на котором будут собирать информацию для
послания.

Если все пойдет по плану, то методом интернет-голосования будет решено, какую
именно информацию следует включить в письмо во Вселенную.

1 января 2019 года New Horizons сблизится с небесным телом, обозначенным в
астрономических каталогах как 2014 MU69 в поясе Койпера - это примерно в 1,6
млрд км от Земли, далеко за орбитой Плутона. На передачу всей собранной
информации об этом небесном теле ей потребуется около года. Только после этого
освободится память бортового компьютера для приема "Послания Единой Земли".

"Таким образом у нас есть почти два года на то, чтобы составить послание, -
отметил Йон Ломберг. - Это займет около года, а еще год нужен на то, чтобы
написать компьютерную программу, проверить ее и убедиться, что она пригодна для
бортового компьютера". Такие программы, по его мнению, можно будет использовать
и в будущем для послания другим цивилизациям с помощью тех аппаратов, которые
будут выходить за пределы Солнечной системы.

"Наши станции - образцы технического совершенства, - считает Ломберг. -
Фактически, это произведения искусства, а произведение искусства должно быть
подписано автором". 

Напомним, что послания иным мирам уже несут американские межпланетные аппараты
Voyager 1 и Voyager 2, вышедшие за пределы Солнечной системы. Они были запущены
с Земли в 1977 году. На борту аппаратов установлены позолоченные медные
пластинки, помещенные в защитные футляры. На них записана информация о
местонахождении Солнечной системы, приветствия на 55 разных языках, а также 115
изображений из области математики, физики, химии, астрономии, анатомии
человека, жизни людей, а также аудизаписи - шум ветра, пение птиц, звук
движущегося поезда и отрывки из музыкальных произведений Баха, Моцарта,
Бетховена.

Послания другим цивилизациям отправлены также с автоматическими станциями
Pioneer 10 и Pioneer 11, запущенными в 1972-1973 годах. На них установлены
идентичные таблички с зашифрованной картинкой-сообщением другим цивилизациям.
На пластины из анодированного алюминия нанесены символистические данные о
человеке, нашей планете и ее расположении. В частности, изображены мужчина с
поднятой в приветствии рукой, женщина, а за ними для сравнения в том же
масштабе - рисунок аппарата Pioneer.
