% vim: keymap=russian-jcukenwin
%%beginhead 
 
%%file 31_12_2020.stz.news.ua.mrpl_city.1.olena_berkova_mrii
%%parent 31_12_2020
 
%%url https://mrpl.city/blogs/view/olena-berkova-vtilyujte-v-zhittya-mrii
 
%%author_id demidko_olga.mariupol,news.ua.mrpl_city
%%date 
 
%%tags 
%%title Олена Беркова: "Втілюйте в життя мрії!"
 
%%endhead 
 
\subsection{Олена Беркова: \enquote{Втілюйте в життя мрії!}}
\label{sec:31_12_2020.stz.news.ua.mrpl_city.1.olena_berkova_mrii}
 
\Purl{https://mrpl.city/blogs/view/olena-berkova-vtilyujte-v-zhittya-mrii}
\ifcmt
 author_begin
   author_id demidko_olga.mariupol,news.ua.mrpl_city
 author_end
\fi

\ii{31_12_2020.stz.news.ua.mrpl_city.1.olena_berkova_mrii.pic.1}

Наприкінці 2020 року я познайомилася з творчістю жінки, яка стала особисто для
мене справжнім взірцем для наслідування. Внутрішня сила цієї талановитої та
тендітної маріупольчанки вражає. \emph{\textbf{Олена Володимирівна Беркова}}, попри всі
труднощі, навчилася не опускати руки. Сьогодні Олена Володимирівна є керівницею
танцювального колективу \enquote{Маріупольська школа класичної хореографії} ПК
\enquote{Молодіжний} та співзасновницею нового танцювально-театрального колективу, який
у вересні – жовтні у своїй програмі \enquote{OMUT} представив два одноактних
модерн-балети. Завдяки її невичерпному ентузіазму і відданості справі багато
вихованців Олени змогли повірити в себе та стати щасливішими.

Народилася наша героїня в Маріуполі в сім'ї інженера. Щоб дівчинка стала
балериною мріяла більше її мама, ніж вона сама. Оскільки під час Другої
світової війни мати Олени потрапила до дитячого будинку, вона не змогла
здійснити свою мрію стати балериною. Проте здійснила свою мрію в доньці. Коли
Оленці було десять років, мама повезла її до Москви, де вона вступила до
Московського хореографічного училища (нині – Московська державна академія
хореографії). Довгих вісім років дівчина навчалася балету, тренуючись щодня
п'ять-шість годин. Проживала майбутня балерина в інтернаті. За цей час у нашої
героїні виробилася неабияка сила волі і терпіння.

\ii{31_12_2020.stz.news.ua.mrpl_city.1.olena_berkova_mrii.pic.2}

Вже в 14 років Олена разом з іншими ученицями училища відправилася в свою першу
гастрольну поїздку – до Франції. На згадку про ті гастролі збереглася афіша, на
якій написано французькою: \enquote{L'ecole de danse de Bolshoi}, тобто \enquote{Школа танцю
Великого театру}.

Цікаво, що всесвітньо відомий танцівник, прем'єр Великого театру, народний
артист Росії Микола Цискарідзе – також випускник Московського хореографічного
училища. Олена була молодшою за Миколу, але особисто з ним знайома. Більше
того, вона часто з ним була в парі, адже Цискарідзе відрізнявся дуже складним
характером, що відлякувало інших дівчат. А от Олена – досить терпляча, тому,
коли на іспиті їх побачили вдвох, ніхто не здивувався. Іспит вони склали
блискуче. Після цього Олені Володимирівні запропонували роботу у Большому
театрі, але потрібно було закінчити навчання.

У 18 років, ставши випускницею, дівчина мала за плечима солідний досвід роботи
на сцені і безліч гастрольних поїздок – вона виступала в країнах Європи і Азії,
Південної та Північної Америки, в Африці і Японії. За розподілом балерина
потрапила до московського Державного академічного театру класичного балету. Як
солістка танцювала па-де-де, па-де-труа і сольні партії практично у всіх
балетах класичного репертуару: \enquote{Лускунчик}, \enquote{Лебедине озеро}, \enquote{Жизель}, \enquote{Дон
Кіхот}, \enquote{Спартак}. Театр давав вистави як в Росії, так і за кордоном.

Незважаючи на затребуваність в постановках, Олена вже тоді розуміла, що у
балерини, на жаль, прекрасний, але короткий шлях. Тому вирішила закінчити
юридичний факультет сучасного гуманітарного університету в Москві.

\ii{31_12_2020.stz.news.ua.mrpl_city.1.olena_berkova_mrii.pic.3}

Цікаво, що Олена пройшла навчання і в училищі сестер милосердя, в чому вона
відчувала особливу духовну потребу. Навчалася в Свято-Дмитрівському училищі
сестер милосердя при Першій міській лікарні, яка знаходиться під патронатом
Московського патріархату. Маріупольчанка працювала волонтеркою в лікарнях,
допомагаючи хворим. Саме навчання в духовно-релігійній установці допомогло
Олені зрозуміти, в чому полягає її найголовніше призначення, як і загалом
кожної жінки, а саме – \emph{бути дружиною, матір'ю.}

\ii{31_12_2020.stz.news.ua.mrpl_city.1.olena_berkova_mrii.pic.4}

Сьогодні Олена – мати трьох дітей (Володимир (17 років), Надія (12 років) та
Єлизавета (9 років) і щаслива дружина. До речі, Надія мріє піти по стопах
матері і наразі вчиться балетному мистецтву. На повернення до рідного Маріуполя
Олену надихнув саме чоловік Василь. Спочатку вона працювала як хореограф у
театрі \enquote{Ребята} при Палаці металургів, а потім очолила маріупольську філію
школи хореографічної майстерності Вадима Писарєва. З 2010 року Олена
Володимирівна очолює дитячу громадську організацію \enquote{Маріупольська школа
класичної хореографії}, основним завданням якої було об’єднання дітей і молоді
для популяризації прекрасного мистецтва балету. На сьогоднішній момент дитяча
громадська організація налічує понад 70 членів – дітей і дорослих. 10 осіб
обрали своєю професією хореографію. Три учні вже отримали професію артист
балету.

\textbf{Читайте також:} \emph{\enquote{Щелкунчик} зажжет новые звезды в Мариуполе}%
\footnote{\enquote{Щелкунчик} зажжет новые звезды в Мариуполе, ДК Молодежный, mrpl.city, 08.10.2018, \par%
\url{https://mrpl.city/blogs/view/shhelkunchik-zazhzhet-novye-zvezdy-v-mariupole}
}

\ii{31_12_2020.stz.news.ua.mrpl_city.1.olena_berkova_mrii.pic.5}

У березні 2018 року ДГО \enquote{Маріупольська школа класичної хореографії} виграла
грант громадського бюджету Маріуполя (100 000 грн.) і здійснила постановку
дитячого інклюзивного балету \enquote{Лускунчик}. Унікальність проєкту полягає в тому,
що в ньому беруть участь діти на інвалідних візках і діти з іншими формами
інвалідності. Проєкт з успіхом продовжує здійснюватися. Олена Володимирівна
наголошує, що в Маріуполі безліч проблем, пов'я\hyp{}заних з відсутністю умов для
всебічного розвитку дітей з особливими потребами, але незважаючи на відсутність
пандусів у ПК \enquote{Молодіжний} і вона, і батьки роблять все можливе, щоб діти
активно долучалися до незабутнього дійства на сцені, що, безумовно, є дуже
важливим і корисним для кожного з них.

\ii{31_12_2020.stz.news.ua.mrpl_city.1.olena_berkova_mrii.pic.6}

З 2019 року в \enquote{Маріупольській школі класичної хореографії} існує інклюзивний
хореографічний клас, де діти займаються на інвалідних візках. Олена
Володимирівна – талановита педагогиня та викладачка. Вона вміє зарядити дітей
своєю енергією, розкрити і надихнути їх. Ніколи в житті не забуду інтерв'ю з
\textbf{Алісою Бордюг} та \textbf{В'ячеславом Гайдаром}, які давали свої коментарі після гри у
виставі \enquote{Лускунчик}. Вони, як і інші діти були усміхнені та щасливі. Єдине, що
відрізняло їх від інших – це інвалідні візки і велика вдячність в очах.
В'ячеслав наголосив, що любов до балету йому прищепила саме Олена
Володимирівна, і балет вчить його пізнавати щось нове, досягати своєї мети.
Наразі керівниця дитячої організації боїться, що ПК \enquote{Молодіжний} може бути
реорганізований і діти залишаться без приміщення. Загалом вже декілька років
колектив переживає не найкращі часи, часто є забутим. Я дуже сподіваюся, що
протягом 2021 року ця прикра помилка з боку представників культурного розвитку
нашого міста буде виправлена. А сам колектив зі своєю енергійною керівницею не
думають опускати руки і сподіваються, що 2021 рік стане продуктивним та
успішним завдяки участі в нових грантах та конкурсах. Олена Беркова – справжній
генератор ідей, вона має безліч задумів, які готова втілювати і в новому
танцювально-театральному колективі разом із \emph{\textbf{Анастасією Петровою}} та \emph{\textbf{Анною
Паніотовою}}. Їхній колектив не боїться експериментувати з фарбами, звуками,
світлом, режисурою та можливостями людського тіла. Водночас в Маріуполі це
єдиний колектив, де можна побачити балетні вистави, виконані на високому
професійному рівні.

\ii{31_12_2020.stz.news.ua.mrpl_city.1.olena_berkova_mrii.pic.7}

\begingroup
\em
\textbf{Улюблена книга:}

\begin{quote}
\enquote{З літератури швидше назву автора. За глибиною розуміння людської душі ніхто не
зрівняється з Ф. Достоєвським. Дуже люблю читати православну духовну
літературу – твори Феофана Затворника, Ігнатія (Брянчанінова).
Цікавлюся християнською філософією і педагогікою, подобається читати
біографічну літературу про життя і творчість великих людей}.
\end{quote}

\textbf{Улюблений фільм:} 

\begin{quote}
\enquote{З кінематографа назву класичні фільми – \enquote{Москва сльозам не вірить} (1979),
\enquote{Службовий роман} (1977), \enquote{Любов і голуби} (1985). Із зарубіжного кінематографу
– \enquote{Амелі} (2001), \enquote{Сім'янин} (2000), \enquote{Володар перснів} (2001), \enquote{Хоббіт}
(2012)}.
\end{quote}

\textbf{Хобі:} 

\begin{quote}
\enquote{Оскільки я багато подорожувала по світу, у мене завжди була мрія показати
красу нашої планети своїм дітям. Тому я займаюся пошуком лайфхаків, як
багатодітній родині подорожувати по світу і не розоритися. Вже є невелика
колекція лайфхаків. Чекаємо кінця пандемії щоб втілити в життя наші мрії!}.
\end{quote}

\textbf{Порада-побажання маріупольцям (і насамперед собі):} \enquote{Не сумувати незважаючи ні на що! Адже щастя поруч!}.

\endgroup

\ii{insert.read_also.dkm.berkova}
