% vim: keymap=russian-jcukenwin
%%beginhead 
 
%%file 19_02_2021.fb.dmitruk_anastasia.1.rev_gidnosti
%%parent 19_02_2021
 
%%url https://www.facebook.com/anastasia.dmitruk/posts/4317961948218566
 
%%author_id dmitruk_anastasia
%%date 
 
%%tags 2014,godovschina,maidan2,obschestvo,strana,ukraina
%%title Я не відвідую події, присвячені річницям Революції Гідності
 
%%endhead 
 
\subsection{Я не відвідую події, присвячені річницям Революції Гідності}
\label{sec:19_02_2021.fb.dmitruk_anastasia.1.rev_gidnosti}
 
\Purl{https://www.facebook.com/anastasia.dmitruk/posts/4317961948218566}
\ifcmt
 author_begin
   author_id dmitruk_anastasia
 author_end
\fi

Я не відвідую події, присвячені річницям Революції Гідності. 

Це я зрозуміла сьогодні, відповідаючи на питання журналістки 5 каналу для міні
сюжету. 

І виявляється, мені зовсім не соромно. Ні, це не ігнор чи принципова позиція,
просто так випадає... 

\ifcmt
  ig https://scontent-frx5-1.xx.fbcdn.net/v/t1.6435-9/152361414_4317963174885110_6210723462971186825_n.jpg?_nc_cat=111&ccb=1-5&_nc_sid=8bfeb9&_nc_ohc=dn9ZwpqfrKcAX_HXUbe&_nc_oc=AQnjZw_sOGVtaIKYoKvoMmAtJa93BG2GtTHDGrLATlybcuNQ0mHWWJpubEbrdlZEkgI&_nc_ht=scontent-frx5-1.xx&oh=024cd29c3ce73b9f446d38802c27a48b&oe=618C9E67
  @width 0.4
  %@wrap \parpic[r]
  @wrap \InsertBoxR{0}
\fi

На мою думку, всі ці урочисті події або навпаки вшанування, знову ж про те
саме, що я писала в попередньому пості — «Бути чи здаватись?» 

Бути українцем, поважати і розвивати свою країну, плекаючи в собі цінності
Революції — це важка робота, яку треба робити щодня, а не лише у пам’ятні дати. 

Бо якщо тобі справді болить Лютий 2014, то він болить набагато частіше, ніж на
річницю...

Коли бачиш як чувирло припаркувалося на пішохідному переході - і ти викликаєш
поліцію, а не мовчки проходиш повз; коли у відповідь на «здравствуйте»,
вимагаєш обслуговувати українською мовою; коли в транспорті вмикають шансон, і
ти вимагаєш його вимкнути; коли купуєш квиток в український театр, а не на
«вечірній квартал» або «лігу сміху»; коли читаєш українські книжки і даруєш їх
друзям; коли об’єднуєшся громадою і блокуєш вирубку лісу; коли ініціюєш круті
проєкти і досягаєш їхнього втілення в життя, і тд.

Це означає бути тою країною, за яку стояв Майдан. Бути завжди важче, ніж
здаватись. 

Ми, як громадяни, з претензією на свідомість та інтелігентність, зобов’язані
набагато більше, ніж раз на рік вшановувати пам’ять і тішитись які ми були
молодці в 2014. 

Куди важливіше — хто ми є сьогодні, в 2021. І чи достатньо зробив кожен з нас,
щоб бодай якось загоїти біль від вогню і  втрат 2014?

Твори красиву Україну! Завжди. Щодня. Живим на славу. На пам’ять полеглим за її
Незалежність. 

П.с. Це єдине моє фото з Революції, яке  випадково зазнимкував Oles Badio; бо
знову ж таки, коли ти справді бунтуєш, тобі не до фотографій  @igg{fbicon.flame} 
