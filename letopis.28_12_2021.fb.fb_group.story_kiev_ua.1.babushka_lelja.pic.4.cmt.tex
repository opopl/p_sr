% vim: keymap=russian-jcukenwin
%%beginhead 
 
%%file 28_12_2021.fb.fb_group.story_kiev_ua.1.babushka_lelja.pic.4.cmt
%%parent 28_12_2021.fb.fb_group.story_kiev_ua.1.babushka_lelja
 
%%url 
 
%%author_id 
%%date 
 
%%tags 
%%title 
 
%%endhead 

\iusr{Оксана Чеснова}
Бог мой! Через что предстоит пройти этим людям..(

\iusr{Татьяна Комащенко}
\textbf{Оксана Чеснова}, глядя на дореволюционные фото, думаю о том же. Люди жили, растили детей, но кому-то это мешало. Очень печально...

\iusr{Анатолий Крупин}
\textbf{Татьяна Комащенко} 

Действительно, глядя на старые фотографии есть над чем задуматься. Жизнь была
во все времена и во все времена растили детей, иначе бы не было нас. Да и семьи
были многодетные и у некоторых семей было до 20 детей. Потом стали меньше
наполовину. Это видно по церковным записям. Например, в семье из 11 детей
выжило только 5. Но была и высока детская смертность и выживала в лучшем случае
только половина. Но если перенестись в сегодня и задуматься над тем, что будет
завтра.

\iusr{Арт Юрковская}

Вы поймите, что 90 \% людей жили как собаки, хуже животных - и дальше бы жили, если
бы их не загнали на четыре года голых и босых в морозные окопы, а их дети и жены
пошли с протянутой рукой и умирали от болезней и голода, потому что каждый
третий кормилец гнил в окопах. А зачем царь залез в войну? Война между
англичанами, французами и немцами. Зачем это все? Люди не захотели дальше
учавствовать в войне и просто ушли по домам. А кому-то это не нравилось, видете
ли. 2\% помещиков владели 40\% земли, а 90\% крестьян владели 30\% земли.

\iusr{Оксана Космина}
А що то за дівчина стоїть у третьому ряду зліва?

\iusr{Ludmila Debourdeau}
\textbf{Oksana Kosmina} понятия не имею

\iusr{Оксана Космина}
\textbf{Ludmila Debourdeau} шкода

\iusr{Maria Minyaeva}
Какие усы!
