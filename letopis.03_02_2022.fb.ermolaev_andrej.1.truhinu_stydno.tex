% vim: keymap=russian-jcukenwin
%%beginhead 
 
%%file 03_02_2022.fb.ermolaev_andrej.1.truhinu_stydno
%%parent 03_02_2022
 
%%url https://www.facebook.com/yermolaievandrey/posts/1011153949496959
 
%%author_id ermolaev_andrej
%%date 
 
%%tags gorod,kultura,obschestvo,selo,sovest,styd,ukraina
%%title Трухину стыдно?
 
%%endhead 
 
\subsection{Трухину стыдно?}
\label{sec:03_02_2022.fb.ermolaev_andrej.1.truhinu_stydno}
 
\Purl{https://www.facebook.com/yermolaievandrey/posts/1011153949496959}
\ifcmt
 author_begin
   author_id ermolaev_andrej
 author_end
\fi

Трухину стыдно?

Скандальная история с ДТП депутата от \enquote{СН} Трухина - это больше, чем история с
сокрытием преступления. Этот случай - как лакмусовая бумажка - проявляет суть и
характер политической культуры господствующей молодой политической элиты.

Нам давно и хорошо знаком принцип \enquote{не пойман - не вор}. Этот полу-бытовой
принцип отражает определенную политико-правовую культуру, связанную с
определением преступления и наказания, устойчивой правовой традицией и нормами
гражданского права. \enquote{Вор}, преступник должен быть обличен, наказан в судебном
порядке и при этом имеет право на защиту. 

Но как быть, если сама система исключает возможность \enquote{поймать}, доказать и
наказать? Если вместо презумпции невиновности утверждается принцип презумпции
неприкасаемости? И если единственным способом сдвинуть ситуацию с \enquote{мертвой
точки} является альтернативная система оглашения факта преступления, действие
общества \enquote{в обход} замороженных правовых процедур?

Для политологов и знатоков права тут огромное поле для анализа и выводов о
природе и характере политико-правовой системы. Очевидно, что такая система не
имеет ничего общего с процедурами правового государства, а ближе, скорее, к
феодальному праву и сословной неприкасаемости.

Но меня заинтересовало вот это самое \enquote{стыдно}. Сказал это человек, уличенный
видеофактами своего преступления и попытками уйти от наказания, совершая при
этом и новые преступления (попытка дать взятку, использовать свое положение,
уйти от криминальной ответственности).

Что же это за \enquote{стыдно} такое, которое стало возможно только через год, после
широкой огласки новых фактов, включая видеофиксацию поведения этого господина?

\enquote{Стыдно} - это не \enquote{виновен}. \enquote{Стыдно} - это когда не увернешься от
общественного мнения и осуждения, когда независимо от правовых последствий твой
поступок был признан как преступный самим обществом. Такая реакция характерна
для \enquote{традиционных обществ}, \enquote{провинциальных культур}, общинного уклада, -
\enquote{культуры стыда} в широком смысле. В отличие от \enquote{культуры совести}, городской
культуры с более развитым индивидуальным сознанием и чувством лично
ответственности за поступки и проступки. 

Это удивительная метаморфоза украинской политики. Молодое, амбициозное, яркое
племя \enquote{Поколения Зе} на поверку оказалось банальным сельским сбродом, которое
не способно само отвечать ни за сказанное, ни за содеянное, и которое признает
свои ошибки и даже преступления, только когда происходит вот такая вот
\enquote{огласка}. И им не совестно, им - \enquote{стыдно}, потому что уже не отопрешься, не
скроешь и не скроешься. 

Власть сейчас - это большое село, со своим кумовством, порукой, сокрытием и
прикрытием. В прошлые времена, когда традиции старого общества \enquote{культуры стыда}
были крепче любого закона, именно такие механизмы \enquote{общинного осуждения}
работали мощнее любого следствия и ареста. Измазанные заборы, толпы возмущенных
соседей-односельчан и оглашение на площадях работали похлеще судебного
приговора. Может, так и надо с нашим \enquote{политическим селом}?

Но все же Украина - не село. И это сохраняет надежду на другое будущее. И на
возрождение нашей \enquote{культуры совести}.

p.s. Трухин исключен из партии \enquote{СН}. Только сейчас. Коллективный стыд - тоже
примечательная штука. Только когда прижало...
