% vim: keymap=russian-jcukenwin
%%beginhead 
 
%%file 05_09_2018.fb.lesev_igor.1.o_medvedchuke.cmt
%%parent 05_09_2018.fb.lesev_igor.1.o_medvedchuke
 
%%url 
 
%%author_id 
%%date 
 
%%tags 
%%title 
 
%%endhead 
\subsubsection{Коментарі}

\begin{itemize} % {
\iusr{Gavrilova Irina}
отличный расклад и выводы правильные.

\iusr{Elena Pachkovsky}
Медведчук уже сказал, что не планирует идти в президенты, И главное зачем ему играть в игры Порошенко?

\begin{itemize} % {
\iusr{Евгений Отовчиц}
Затем, что они партнеры по криминальному и легальному бизнесу. Медведчук если не второй, то третий человек в стране. После Ахметова и Порошенко.

\iusr{Elena Pachkovsky}
\textbf{Евгений Отовчиц} ну и в каком бизнесе они партнеры, только конкретно?

\iusr{Евгений Отовчиц}
\textbf{Elena Pachkovsky} нефтяном и газовом.

\iusr{Elena Pachkovsky}
\textbf{Евгений Отовчиц} а по конкретнее

\iusr{Евгений Отовчиц}
\textbf{Elena Pachkovsky} поконкретнее - я не прокуратура и не следователь. Хотите наивно обманываться - ваше право. Только появление Медведчука в Минской группе уже было очень красноречиво. Прекрасный бэкграунд для возврата в большую политику, миротворец и вызволитель украинских граждан, мудрый и влиятельный Виктор Медведчук.

\iusr{Матвей Кублицкий}
\textbf{Евгений Отовчиц} наивно обманываться - ключевая фраза.. вас попросили привести доказательства - а их нет. Чуствуете каким лохом вы выглядите?

\iusr{Евгений Отовчиц}
\textbf{Матвей Кублицкий} 

давайте теперь про лоха перед зеркалом поупражняйтесь

\url{https://m.facebook.com/story.php?story_fbid=1989699344385120&id=100000352217065}
\end{itemize} % }

\iusr{Марина Прохорова}

Я не знаю, как Медведчук оценивает сам себя, но зачем ему мараться обо всех
этих Грынивыв и , прости Господи, Уколовых, я решительно не представляю.

\iusr{Матвей Кублицкий}

О! Заработало!!!! Я уж думал ты совсем в делах зашился. По сути: какая разница
кто придет к власти на след год? Этот человек по любому ничего не сможет
изменить ни в экономике, ни в политике. Он может лишь слегка смягчить отношения
с РФ - начав их нормализацию. Резко в сторону России он повернуться не сможет -
народ не поймет. Это как в Грузии нынче. Всё население молится на русских
туристов и продажи товара в Россию, а правительство продолжает мелко тявкать в
ту же сторону.

\begin{itemize} % {
\iusr{Игорь Лесев}
разница у нас таки есть)) как между пальцем и не пальцем)

\iusr{Матвей Кублицкий}
\textbf{Игорь Лесев} 

ну разве что в мелочах. Реально следующий президент не сможет освободится от
диктата США и не сможет наладить отношений с РФ. То есть во внешней политике
изменений не будет - не зависимо от президента. Можно лишь начать что-то
исправлять во внутренней - но и тут давление сильное. Особо не разгуляешься

\end{itemize} % }

\iusr{Дмитрий Коломийченко}

Между ПАПом и кумом Путена безусловно существует базовое соглашение, иначе бы
кум Путена не вышел из тени. Но при этом каждый пытается использовать друг
друга по-максимуму. ПАП отрабатывает угрозу реванша, а кум Путена медийный
ресурс ПАПа. Расчет кума Путена на окучивание наследия ПР и сильную
парламентскую фракцию в будущем. ПАПу нужно реализовать схему Ельцина,
переизбравшись через угрозу реванша. За отсутствием нормального Лебедя сойдёт и
Гриценко. ПАП на определенных условиях может помочь собрать наследие ПР
понятному и уязвимому человеку. Позже через него можно будет вести продуктивные
переговоры с русскими. Остальных претендентов в процессе обламать, чтобы знали
своё место. Против ТЮЛи остаётся поддерживать градус пиара для высокого
антирейтинга, провоцируя последнюю на радикальные заявления. Чем она будет
пугать ватников готовых голосовать за ТЮЛю против ПАПа. Тем самым разрушая
ТЮЛину заявку на общеукраинского президента. Схема игры - отсекание от всех
возможных резервов, электоральная изоляция. ТЮЛя может перебить это радикальной
постановкой вопроса региональным элитам: "Неужели вы хотите терпеть это ещё 5
лет?" И предложения им реальной децентрализации. В медийном плане отказаться от
соревнования с ПАПом в любви к родине и сосредоточится на
социально-экономических посланиях.Благо правительство в этом смысле в весьма
уязвимом положении. Перейти к прямому общению с избирателями на их языке через
соц. сети. В стиле Трампа. Политическую риторику постепенно перевести на рельсы
просвещенного консерватизма, демонстрируя себя в качестве украинского лидера
консервативной революции в Европе. Схема игры - украинский Орбан.

\begin{itemize} % {
\iusr{Игорь Лесев}
отличное дополнение

\iusr{Евгений Отовчиц}
Только Тимошенко никак не побеждает. Вообще. Без шансов.

\iusr{Дмитрий Коломийченко}
\textbf{Евгений Отовчиц} Почему не побеждает?

\iusr{Евгений Отовчиц}
\textbf{Дмитрий Коломийченко} не знаю. Судьба такая - НЕ стать Президентом.

\iusr{Дмитрий Коломийченко}
\textbf{Евгений Отовчиц} Может быть. Но эти выборы такие, если они состоятся, что невозможное возможно, как говорил один поэт.

\iusr{Михайло Бойченко}
Nice try @igg{fbicon.face.smiling.eyes.smiling} 
\end{itemize} % }

\iusr{Олег Резник}
Жизнь покажет.

\iusr{Ирина Касьянова}
Никак не подберут актёра на роль свадебного генерала...

\end{itemize} % }
