% vim: keymap=russian-jcukenwin
%%beginhead 
 
%%file 30_07_2021.fb.krym_tatar_resursnyj_centr.1.ni_kapli_okupantu
%%parent 30_07_2021
 
%%url https://www.facebook.com/CTRC2015/posts/2955847068028657
 
%%author_id krym_tatar_resursnyj_centr
%%date 
 
%%tags blokada,blokada.krym.voda,krym,okupacia,voda
%%title «Ні краплі окупанту»: КРЦ започатковує нову інформаційну кампанію
 
%%endhead 
 
\subsection{«Ні краплі окупанту»: КРЦ започатковує нову інформаційну кампанію}
\label{sec:30_07_2021.fb.krym_tatar_resursnyj_centr.1.ni_kapli_okupantu}
 
\Purl{https://www.facebook.com/CTRC2015/posts/2955847068028657}
\ifcmt
 author_begin
   author_id krym_tatar_resursnyj_centr
 author_end
\fi

@igg{fbicon.exclamation.mark}
«Ні краплі окупанту»: КРЦ започатковує нову інформаційну кампанію @igg{fbicon.exclamation.mark}

 @igg{fbicon.droplet}  Проблема води завжди була актуальною в Криму. Раніше, до 2014 року, частина
потреб півострова покривалися за рахунок Північно-Кримського каналу з
материкової України.

20 лютого 2014 року Російська Федерація здійснила акт агресії проти України.

Згідно з міжнародним гуманітарним правом держава-агресор зобов’язана нести
відповідальність за цивільне населення, яке залишилося в окупованому Криму.

Саме тому Україна перекрила Північно-Кримський канал.

\ifcmt
  ig https://scontent-mxp1-1.xx.fbcdn.net/v/t1.6435-9/217030080_2955846134695417_6022034080440536361_n.jpg?_nc_cat=104&ccb=1-5&_nc_sid=730e14&_nc_ohc=cwQYSoTHNloAX_l4ptn&_nc_ht=scontent-mxp1-1.xx&oh=86d32c052bd76cb6f492652f9dce2504&oe=6190113A
  @width 0.4
  %@wrap \parpic[r]
  @wrap \InsertBoxR{0}
\fi

 @igg{fbicon.round.pushpin}  Тема дефіциту води загострилася у 2020 році. Окупаційна влада продовжує
активно популяризувати цю «проблему», роблячи різноманітні заяви. Уже декілька
разів РФ звинувачувала у нестачі води саме Україну і закликала міжнародну
спільноту звернути увагу на «антигуманні дії» Києва щодо перекриття подачі води
в окупований Крим.

Більше того, Генеральна прокуратура РФ відправила скаргу на Україну до
Європейського суду з прав людини згідно зі статтею 33 Конвенції про захист прав
людини і основних свобод. Окупанти вимагають, щоб Україна негайно припинила
такі «кричущі порушення, як блокування надходження прісної води до Криму».

 @igg{fbicon.thinking.face}  Але кому насправді не вистачає води на півострові? Хто винен у цьому? 

Незважаючи на те, що у Криму катастрофічно не вистачає води, російська влада
продовжує закликати тисячі туристів на півострів. На жаль, ми бачимо, що триває
і мілітаризація Криму. При тому, спостерігаємо процес заміщення населення,
тобто активне заселення Криму громадянами РФ. Все це потребує використання
великих запасів води.

 @igg{fbicon.face.disappointed}  Крим просто до цього не готовий.

Ми не знаємо до чого це призведе. Саме тому Кримськотатарський Ресурсний Центр
започатковує кампанію «Ні краплі окупанту» для того, щоб проінформувати
міжнародне співтовариство і громадськість про те, що відбувається в Криму та
попередити про можливі наслідки.

 @igg{fbicon.pushpin}  Закликаємо всіх приєднатися до інформаційної кампанії КРЦ і поширити цей
плакат на своїх сторінках в соціальних мережах (Facebook, Twitter, Instagram,
Telegram) із нашим тегом (Кримськотатарський Ресурсний Центр), щоб показати
всьому світу, що ми продовжуємо боротися за наш Крим і ні за що не дамо воду
окупанту.

Плакати були розроблені Кримськотатарським Ресурсним Центром, активістами
міжнародного руху за деокупацію Криму і солідарність з кримськотатарським
народом \textbf{\#LIBERATECRIMEA} та українським художником Andriy Yermolenko

