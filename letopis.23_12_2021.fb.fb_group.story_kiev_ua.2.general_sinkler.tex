% vim: keymap=russian-jcukenwin
%%beginhead 
 
%%file 23_12_2021.fb.fb_group.story_kiev_ua.2.general_sinkler
%%parent 23_12_2021
 
%%url https://www.facebook.com/groups/story.kiev.ua/posts/1824812234382236
 
%%author_id fb_group.story_kiev_ua,poljakova_galina.kiev
%%date 
 
%%tags armia,istoria,kiev,sinkler_vladimir.ukr.general,ukraina,unr
%%title Генерал Володимир Олександрович Сінклер
 
%%endhead 
 
\subsection{Генерал Володимир Олександрович Сінклер}
\label{sec:23_12_2021.fb.fb_group.story_kiev_ua.2.general_sinkler}
 
\Purl{https://www.facebook.com/groups/story.kiev.ua/posts/1824812234382236}
\ifcmt
 author_begin
   author_id fb_group.story_kiev_ua,poljakova_galina.kiev
 author_end
\fi

Генерал Синклер недолго прожил в Киеве. Сюда привела его служба. И сюда же
привез его СМЕРШ. Прямо в Лукьяновскую тюрьму. Как и Габсбурга. Как и Габсбург,
генерал Синклер навсегда остался в Киеве. Вот только могилы его нет – где то
зарыли. Но память остается. 

\ii{23_12_2021.fb.fb_group.story_kiev_ua.2.general_sinkler.pic.1}

Ім’я генерала Володимира Олександровича Сінклера чомусь майже загубилося в
українській історії. Але саме він заклав основи українського війська. Саме він
створив, так би мовити, мозок армії – генштаб. 

Є дві версії щодо походження генерала. Хтось вважає, що він є нащадком
шведського майора Сінклера — зв'язкового старшини під рукою гетьмана Пилипа
Орлика. Існують свідчення, за якими майора Сінклера 28 червня 1739 року на
шляху з Царгорода до Стокгольма заарештували і вбили поплічники Михайла
Бестужева-Рюміна. А хтось впевнений, що шведи тут ні до чого, а його предком
був чи то англієць, чи то шотландець, який подався на службу до російського
царя Олексія Михайловича. Де правда, певно, ми вже не дізнаємося. Але є
документи – протоколи допитів у радянській Лук’янівській в’язниці у 1945-46
роках, де Володимир Сінклер сам зазначив свою національну приналежність –
українець. Звичайно, генерал поважав свій старовинний шляхетний рід, але жив у
теперішньому часі і свій вибір зробив без вагань: українець на службі України.

З діда-прадіда військовий, Володимир Олександрович отримав освіту, достатню для
блискучої кар’єри: кадетський корпус, Михайлівське артилерійське училище,
Академія Генерального штабу. Служба просувалася успішно – достойні посади у
лейб-гвардії. Але професійна і чоловіча гідність закликають 25-річного
випускника Академії Генштабу покинути Петербург і їхати до Маньчжурії, адже
йшла російсько-японська війна і про столичні принади треба було забути.

Перша Світова війна. Фронт. Звання, бойові нагороди. Важке поранення. 

Революція 1917 року. Генерал-майор Сінклер, командувач 2-го гвардійського
корпусу, звільняється і їде в Київ. Їде будувати українську армію. У листопаді
1917 року він вже у складі Українського генерального військового штабу. Він
стає членом Військово-ученого комітету Військового міністерства УНР, згодом —
Української Держави. З 1920 року очолює Генштаб УНР. 

Міняються політичні діячі, уряди... Павло Скоропадський, Симон Петлюра, Василь
Тютюнник... Але генерал служить не їм, а Україні. Саме він створив оперативний
апарат Армії УНР, створив офіцерські та унтер-офіцерські школи. Генерал Сінклер
- безпосередній розробник усіх бойових операцій. Його функції: оперативна
діяльністю, формування та організації української армії, розвідка,
контррозвідка, мобілізація і багато-багато ще чого. 

Може така поважна особа отримувала величезні статки? Ні, грошей взагалі не
було. Славетного генерала фактично утримувала дружина. Вона вишивала на
замовлення якоїсь американської фірми. З того й хліб їли. 

Після поразки і ліквідації УНР оселився в Польщі. Працював кондуктором,
маркером на копальні. Польський уряд поважав його знання та досвід і
неодноразово пропонував високі посади у Війську Польськім. Але генерал Сінклер
дякував і ввічливо відмовлявся, бо вже присягнув Україні. 

1945 року його заарештував СМЕРШ. Лук’янівська в’язниця. Допити. Смертний
вирок. Але не розстріляли, бо не встигли. У ході допитів генерал Сінклер помер.
Помер так, як і жив: гідно і мужньо.

\ii{23_12_2021.fb.fb_group.story_kiev_ua.2.general_sinkler.cmt}
