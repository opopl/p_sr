% vim: keymap=russian-jcukenwin
%%beginhead 
 
%%file 13_06_2021.fb.bilchenko_evgenia.2.krovj_kvartirnik_otzyvy_foto
%%parent 13_06_2021
 
%%url https://www.facebook.com/yevzhik/posts/3977238895644524
 
%%author Бильченко, Евгения
%%author_id bilchenko_evgenia
%%author_url 
 
%%tags bilchenko_evgenia,kniga,krov',kultura,kvartira,kvartirnik,literatura,poezia,ukraina
%%title ОТЗЫВЫ, ФОТО И КРОВЬ С КВАРТИРНИКА
 
%%endhead 
 
\subsection{Отзывы, фото и кровь с Квартирника}
\label{sec:13_06_2021.fb.bilchenko_evgenia.2.krovj_kvartirnik_otzyvy_foto}
\Purl{https://www.facebook.com/yevzhik/posts/3977238895644524}
\ifcmt
 author_begin
   author_id bilchenko_evgenia
 author_end
\fi

Вы не поверите. Вот - обложка книги \enquote{Пьета} от Iren Martynyuk. Вот -
моя рука после ухода гостей с презентации. Я случайно порезалась ножом, делая
себе и другу бутерброд. Я просто воспроизвела свою  обложку. Кровь не хочет
останавливаться уже вторые сутки: йод, спирт, зелёнка, бинты, перетяжки...
Ничего. Муж настаивает, что в больницу - не надо. Вытекло литр, наверно. Из
пальца, Карл! Это - немыслимо. Еле оттерли все: мебель, пол, раковину, новую
мою одежду для праздника, тапки и носки студента... 

Все видели: я не пила почти, просто бокал сухого вина. Символически. Я не пью
больше, это факт, я сильно ушла в церковное видение. Моя солдатская
свёртываемость крови всегда была гордостью врачей. Это - мистика. Есть ещё
объяснение: нацисты так мучают, квартирник просто их выбесил, особенно в НИИ
культурологии, где научный секретарь Инна Кузнецова (открываю имя) без научных
достижений вообще после облома судилищ в драгопеде, никак не успокоится:
преследует меня лично с ноября, с 18 января - уже в оскорбительной и подлой
форме, днём и ночью, круглосуточно. Натравливает ее муж, по фамилии Рябенький,
который в НИИ работает по её протекции, без надлежащего образования. Oxana
Chelysheva, write Thomas Wallgren to take of his partnersheep: Institute
commemorates OUN-UPA. Мне все равно, я всех прощаю, но, в общем, врач настоял
на регулярном гидазепаме. А я не хочу обманывать ученого с мировым именем. Он
верит мне. 

Может транквилизатор убить свёртываемость крови? А аспирин? Я пью его, потому
что все время в полувирусе ползаю. Сейчас ощущаю себя царевичем, сыном Николая
Второго. Еле набрала текст благодарности вам одной рукой, а мне - надо
работать: все время денег не хватает. Все время надо работать двумя руками. И
меня просто спасает научное и поэтическое письмо, Сергей Возняк  знает. Теперь
о хорошем.

Дорогие гости, спасибо за теплые отзывы и вашу ко мне деликатную доброту.
Спасибо, что поняли, что я - иная. Пьета есть Пьета... И эти стихи с кровью...
Удивительно, но, в отличие от насморка, вид такого количества крови из такого
худого пальца (\verb|#НиколайТопало| свидетель) меня вообще не трогает.
Никакого ни страха, ни отвращения, как у барышень. Странно? Закономерно? А
теперь - слово вам.

\enquote{Для того и устраиваются квартирники, чтобы видеть и всецело чувствовать
открытые души единомышленников. Презентация новой книги \enquote{Пьета}  замечательной
Евгения Бильченко  у неё дома. Всё было настолько классно, что и описания мои
ничего не дадут, это всё нужно лично увидеть воочию и прочувствовать. 

Спасибо, Женя, за то, что ты совершенно другая!!!! Спасибо за подаренную нам
всем часть твоего личного мировосприятия!!!!}

АНДРЕЙ ЧУПАХИН.

\enquote{Сложно передать эмоции от Твоей Пьеты, дорогая. Настолько, что получив
её, я до сих пор не могу подобрать слова. Ты удивительная и совсем другая.
Спасибо за то, что я была с вами вчера!} Анастасия Иванютенко.

PS. Лучше я скажу сразу. Я прощалась с вами. Люблю тех, кто не предал меня.
Гостиная царски вместила вас. И пролетарски обогрела. Идите и сопротивляйтесь.
Идите и выгоняйте их на мороз, товарищи. Вы сможете. Ничего не бойтесь. Победа
придет. Я просто сейчас - на других, повыше, левелах этой адской игры, приз
которой  - танк, рай, наш язык, Бог... Как в фильме про концлагерь Роберто
Бениньи \enquote{Жизнь прекрасна} в трактовке Славоя Жижека. Посмотрите кино.

Фото: Андрей Чупахин, Надежда Сточко-Бабий, Настенька Иванютенко.

