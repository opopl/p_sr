% vim: keymap=russian-jcukenwin
%%beginhead 
 
%%file books.shou_dzhulia.psihologia_zla
%%parent books
 
%%url 
 
%%author_id 
%%date 
 
%%tags 
%%title 
 
%%endhead 

\url{https://www.litmir.me/br/?b=676705&p=1}

Ненасытно любопытным

Кто сражается с чудовищами, тому следует остерегаться, чтобы самому при этом не стать чудовищем.
ФРИДРИХ НИЦШЕ. ПО ТУ СТОРОНУ ДОБРА И ЗЛА

Введение

Жажда

Знаменитый немецкий философ XIX века Фридрих Ницше в 1881 году писал: «Böse
denken heißt böse machen» – «Нечто становится дурным, поскольку его таким
считают»[1]. Зло становится злом только тогда, когда мы навешиваем на что-то
ярлык «зло», когда думаем, что это что-то и есть зло. Ницше утверждал, что зло
– это субъективное переживание, а не характеристика, изначально присущая
человеку, предмету или действию[2].

Данная книга изучает упомянутое переживание с научной точки зрения, предлагая
широкий спектр понятий и концепций, часто ассоциирующихся со словом «зло». Это
исследование человеческого лицемерия, абсурдности зла, обыденного безумия и
эмпатии. И я надеюсь изменить ваш взгляд на то, что значит «быть плохим».

Последние тринадцать лет, будучи студенткой, преподавательницей и
исследовательницей, я с радостью обсуждала науку зла с каждым, кто готов был
слушать. Больше всего мне нравится разрушать фундаментальные представления о
добре и зле как о белом и черном, заменяя их нюансированными научными
взглядами. Я хочу, чтобы все мы более осмысленно обсуждали поведение, к
постижению которого, как нам поначалу представляется, мы не можем и не должны
даже приближаться. Без понимания мы рискуем дегуманизировать других людей,
поставить на них крест. Мы можем и должны попытаться уразуметь то, что называем
злом.

Давайте начнем с упражнения на эмпатию к злу. Подумайте о худшем, что вы когда-либо совершали. О чем-то, чего вы, возможно, стыдитесь и из-за чего, как вы знаете, другие будут думать о вас хуже. Измена. Кража. Ложь. Теперь представьте, что все об этом узнали. Осудили вас. И стигматизировали, исходя из вашего проступка. Как вам такое?

Нам бы очень не хотелось, чтобы мир навсегда осудил нас на основании того, о чем мы сожалеем больше всего. И все же мы делаем это каждый день по отношению к другим людям. Размышляя о своих решениях, мы держим в голове определенные нюансы, обстоятельства, трудности. А думая о чужом выборе, мы часто замечаем лишь последствия поступков. Это приводит нас к тому, что мы называем людей – при всей их сложности – одним отвратительным словом. Убийца. Насильник. Вор. Лжец. Психопат. Педофил.
	
	

Эти ярлыки, которыми мы награждаем других, основаны на наших представлениях о том, кем эти люди должны быть, учитывая их поведение. Единственное слово призвано обобщить чей-то подлинный характер и опорочить человека, передать другим, что ему нельзя доверять. Он опасен. Да и не человек вовсе: скорее какая-то ужасная ошибка. Тот, кому мы не должны пытаться сочувствовать: он безнадежно испорчен, и нам никогда не удастся его понять. Такие люди за пределами нашего понимания, их невозможно спасти, они злы.

Но кто эти «они»? Вероятно, если мы осозна́ем, что мы все часто прокручиваем у себя в голове мысли и совершаем поступки, которые другие находят недостойными, это поможет нам постичь саму суть того, что мы зовем злом. Я могу гарантировать: кто-то в мире считает вас воплощением зла. Вы едите мясо? Работаете в банковской сфере? Ваш ребенок родился вне брака? То, что кажется нормальным вам, не просто не представляется таковым для других, а порой выглядит даже крайне предосудительным. Возможно, все мы злодеи. А может, никто из нас.

Как общество, мы много думаем о зле и при этом совсем не обсуждаем его по-настоящему. Ежедневно мы слышим о новых человеческих злодеяниях и поверхностно вовлекаемся в постоянный поток сообщений, из-за которых нам начинает казаться, что человечество обречено. Как говорят журналисты, кровавые новости – самые интересные. Происшествия, которые вызывают сильные эмоции, попадают в газетные заголовки и забивают ленты наших соцсетей. Увидев их за завтраком, к обеду мы уже забываем о них.

В частности, сегодня нашу жажду насилия, похоже, утолить еще труднее, чем прежде. Исследование, опубликованное в 2013 году психологом Брэдом Бушменом и его коллегами, изучавшими проявления жестокости в фильмах, продемонстрировало, что «начиная с 1950 года количество насилия в кино увеличилось примерно вдвое, а стрельбы в фильмах с маркировкой PG-13 [12A][3] стало больше, чем в фильмах с маркировкой R [15][4]»[5]. Фильмы становятся все более жестокими, даже те, что предназначены для детей. Истории о насилии и тяжелых человеческих страданиях проникают в нашу повседневную жизнь чаще, чем когда-либо.

Какой эффект это оказывает на нас? Это искажает наше восприятие распространенности преступности: мы начинаем думать, будто ее больше, чем есть на самом деле. Влияет на то, что мы называем злом. И меняет наши представления о справедливости.

Предлагаю сверить ожидания насчет данной книги. Я не погружаюсь глубоко в индивидуальные случаи. Тем, кого часто называют воплощением зла, посвящены целые тома: например, Джону Венеблсу, самому юному человеку, когда-либо обвиненному в убийстве в Великобритании[6], названному в таблоидах «Прирожденным злодеем», или серийному убийце Теду Банди из США, или «Убийцам Кену и Барби» – Полу Бернардо и Карле Хомолка из Канады. Истории этих людей, без сомнения, захватывающие, но в действительности книга не о них. Она о вас. Я хочу, чтобы вы скорее поняли собственные мысли и наклонности, чем разобрали примеры чужих преступлений.

Это также не философская книга, не религиозная, не книга о нравственности. Она призвана помочь нам осознать, почему мы совершаем ужасные поступки по отношению к другим, а не выяснить, должно ли это происходить или какие наказания будут уместны. Книга изобилует рассказами об экспериментах и теориях и обращается за ответами к науке. Она разделяет понятие зла на множество подпонятий и изучает каждое в отдельности.
	
	

Это не всеобъемлющий взгляд на зло. Для того чтобы создать что-то подобное, не хватит жизни. Возможно, вас разочарует, что я почти не обсуждаю такие масштабные темы, как геноцид, жестокое обращение с детьми, детская преступность, мошенничество на выборах, государственная измена, инцест, наркотики, формирование банд или война. Если вас это интересует, вы найдете немало книг, посвященных указанным вопросам, но данная книга – не одна из них. Она опирается на доступные источники, чтобы показать нечто неожиданное. И дает обзор разнообразных важных тем, связанных с понятием зла, которое мне кажется захватывающим, знаковым и часто остающимся без внимания.
вернуться

1

Nietzsche, F. Morgenröte: Gedanken über die moralischen Vorurteile [Утренняя заря]. Munich: DTV, 1881.
вернуться

2

Оригинальная цитата: «Die Leidenschaften werden böse und tückisch, wenn sie böse und tückisch betrachtet warden» («Страсти становятся злыми и коварными, когда их считают злыми и коварными»).
вернуться

3

PG-13 – рейтинг, которым в системе рейтингов Американской киноассоциации помечаются фильмы, не рекомендованные к просмотру детям младше 13 лет. 12А – рейтинг Британского совета по классификации фильмов, аналог рейтинга PG-13. – Прим. пер.
вернуться

4

R – рейтинг в системе рейтингов Американской киноассоциации, им помечаются фильмы, к просмотру которых дети до 17 лет допускаются только в присутствии родителей. 15 – рейтинг Британского совета по классификации фильмов, аналог рейтинга R. – Прим. пер.
вернуться

5

Bushman, B. J., Jamieson, P. E., Weitz, I., & Romer, D. ‘Gun violence trends in movies’. Pediatrics, 132 (6) (2013), pp. 1014–18.
вернуться

6

Десятилетний Джон Венеблс совершил убийство не один, а вместе со своим ровесником Робертом Томпсоном, разница в возрасте мальчиков составила три дня. – Прим. ред.

