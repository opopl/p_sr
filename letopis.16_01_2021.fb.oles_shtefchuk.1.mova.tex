% vim: keymap=russian-jcukenwin
%%beginhead 
 
%%file 16_01_2021.fb.oles_shtefchuk.1.mova
%%parent 16_01_2021
 
%%url https://www.facebook.com/komonaut/posts/10222751764616454
 
%%author 
%%author_id 
%%author_url 
 
%%tags 
%%title 
 
%%endhead 

\subsection{Мова}
\Purl{https://www.facebook.com/komonaut/posts/10222751764616454}
\ifcmt
  author_begin
   author_id shtefchuk_oles
  author_end
\fi

Вчора був на Українське радіо, де в прямому ефірі говорили про 16 січня і про
Українською в Харкові. Мапа , про те як Харків та Одеса готуються обслуговувати
українською. 

Одразу після цього зняли про це ж сюжет для Суспільне Харків.

\ifcmt
  pic https://scontent-prg1-1.xx.fbcdn.net/v/t1.0-9/139385453_10222751615252720_6127212381841936589_n.jpg?_nc_cat=101&ccb=2&_nc_sid=730e14&_nc_ohc=lDPnmQ5T2QAAX8hPk5Q&_nc_oc=AQlOoN5rJHE38a3Iu8uDZiL6oIUx1MvrPlvA5Xi9sDNPqqvC0xw4mpIOiiQtHmak2_M&_nc_ht=scontent-prg1-1.xx&oh=3047b8a2e7521923a41fdc0a138685b0&oe=6026DB42
  width 0.4
\fi

Так от, готуючись до ефіру і продивляючись мапу https://bit.ly/30j0Gpe вкотре
впіймав себе на висновку, що до нас, україномовних, здебільшого готові саме
елітні заклади.

Наприклад першим супермаркетом, що обслуговує українською в Ха був Le Silpo, а
якщо подивитися на список кафе - то це теж переважно дорогі якісні кафе. 

І з одного боку хочеться, щоб українська й надалі була ознакою якості закладу,
бо нахіба нам хамство чи неповага українською, хай залишається як є.

А з іншого - ця "елітарність" і не всім доступна і звісно набридає, бо "елітного" ремонту взуття чи молочного кіоску чи стоматолога - ще пошукати. 

Тому ми чекали завтра і я особисто маю обережно-оптимістичні очікування, бо з
одного боку бачу, що багато хто з підприємців реально готується, з іншого -
розумію, що це буде тривалий процес й ті хто хоче, має можливості не
дотримуватися норм закону, на жаль.

Прямо вчора перед ефіром побачив, як це працює - в "Чудо-маркеті" обслуговують
українською (вперше бачу там таке), переді мною касирка красиво все говорить
чоловіку поважного віку і у відповідь отримує "Дуже дякую!" 👍 

На це і сподіваюсь, що носії української нарешті її будуть застосовувати і це
буде унормовуватися і таким чином розширюватиметься україномовний простір.

І знов наголошую, що українська - це безпековий фактор, це стратегічне питання,
й це треба було зробити давно, але добре що робиться хоч зараз. 

Бо це все не для нас і навіть не для наших дітей, а щоб хоча б онуки могли вільно послуговуватися нашим багатством - мовою.

Тому щиро сподіваюсь на ваше розуміння і на те, що ви підтримаєте персонал,
який завтра вперше (або ні) заговорить українською.

Слава Україні!
