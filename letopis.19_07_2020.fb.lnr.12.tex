% vim: keymap=russian-jcukenwin
%%beginhead 
 
%%file 19_07_2020.fb.lnr.12
%%parent 19_07_2020
 
%%endhead 
  
\subsection{Чудовищная катастрофа на Донбассе: человеческие останки всё ещё находятся на месте трагедии}
\label{sec:19_07_2020.fb.lnr.12}
\url{https://www.facebook.com/groups/LNRGUMO/permalink/2858364274275128/}

\vspace{0.5cm}
{\small\LaTeX~section: \verb|19_07_2020.fb.lnr.12| project: \verb|letopis| rootid: \verb|p_saintrussia|}
\vspace{0.5cm}

Независимый американский журналист Патрик Ланкастер, с 2014 года освещающий
войну на Донбассе, в годовщину трагедии с самолётом Малайзийских авиалиний
побывал на траурном мероприятии и на месте крушения. В очередной раз Ланкастер
записал сюжет о том, что останки жертв всё ещё находятся на донбасской земле и
это нужно как-то решить.

Однако, как отмечает журналист, пока мировое сообщество 6 лет занято «поиском
виновников» (а по факту притягиванием обвинений против России), до самих жертв
и членов их семей никому нет дела.

«Почему до сих пор остаются останки жертв на месте крушения самолета MH17
спустя 6 лет после того, как рейс MH17 Малайзийских авиалиний был сбит над
Донецкой Народной Республикой?

Поскольку премьер-министр Австралии говорит, что они „ищут справедливости“ для
жертв MH17, похоже, что он и многие другие страны забыли тот факт, что на месте
крушения все еще остаются человеческие останки.

Как вспоминают жертв MH17 у Национального памятника MH17 в Амстердаме, в 6-ю
годовщину авиакатастрофы люди на территории ДНР также вспоминают жертв.

Я показываю вам эту поминальную службу здесь, и, как я делаю каждый год, я хожу
по месту крушения и документирую то, что я вижу, так что это публично записано
в надежде, что будет еще один поиск, чтобы принести все человеческие останки
домой, где им место.

Я стараюсь делать все с максимальным уважением к жертвам и их семьям».
  
