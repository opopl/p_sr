% vim: keymap=russian-jcukenwin
%%beginhead 
 
%%file 01_05_2021.fb.igrunov_vjacheslav.1.ukrainstvo_trubeckoj_nikolaj
%%parent 01_05_2021
 
%%url https://www.facebook.com/v.igrunov/posts/10159065450004508
 
%%author 
%%author_id 
%%author_url 
 
%%tags 
%%title 
 
%%endhead 

\subsection{Публикации в Ленте новостей - Владимир Баранов}
\label{sec:01_05_2021.fb.igrunov_vjacheslav.1.ukrainstvo_trubeckoj_nikolaj}
\Purl{https://www.facebook.com/v.igrunov/posts/10159065450004508}

Повторю часть цитаты, которую я постил 2 года назад. С каждым годом ее
буквалистская предсказательная точность поражает все больше. Гениальность этого
человека не поддается измерению:

"Эти же люди, конечно, постараются всячески стеснить или вовсе упразднить самую
возможность свободного выбора между общерусской и самостоятельно-украинской
культурой: постараются запретить украинцам знание русского литературного языка,
чтение русских книг, знакомство с русской культурой. Но и этого окажется
недостаточно: придется еще внушить всему населению Украины острую и пламенную
ненависть ко всему русскому и постоянно поддерживать эту ненависть всеми
средствами школы, печати, литературы, искусства, хотя бы ценой лжи, клеветы,
отказа от собственного исторического прошлого и попрания собственных
национальных святынь. Ибо, если украинцы не будут ненавидеть все русское, то
всегда останется возможность оптирования в пользу общерусской культуры.

Однако, нетрудно понять, что украинская культура, создаваемая в только что
описанной обстановке, будет из рук вон плоха. Она окажется не самоцелью, а лишь
орудием политики и, притом, плохой, злобно-шовинистической и задорно-крикливой
политики. И главными двигателями этой культуры будут не настоящие творцы
культурных ценностей, а маниакальные фанатики, политиканы, загипнотизированные
навязчивыми идеями. Поэтому, в этой культуре все, — наука, литература,
искусство, философия и т.д., — не будет самоценно, а будет тенденциозно. Это
откроет широкую дорогу бездарностям, пожинающим дешевые лавры благодаря
подчинению тенденциозному трафарету, — но зажмет рот настоящим талантам, не
могущим ограничивать себя узкими шорами этих трафаретов. Политиканам же нужно
будет главным образом одно — как можно скорей создать свою украинскую культуру,
все равно какую, только, чтобы не была похожа на русскую. Это неминуемо поведет
к лихорадочной подражательной работе: чем создавать заново, не проще ли взять
готовым из заграницы (только бы не из России!), наскоро придумав для
импортированных таким образом культурных ценностей украинские названия!"

Николай Трубецкой, 1927 г
