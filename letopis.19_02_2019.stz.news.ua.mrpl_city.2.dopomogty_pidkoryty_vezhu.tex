% vim: keymap=russian-jcukenwin
%%beginhead 
 
%%file 19_02_2019.stz.news.ua.mrpl_city.2.dopomogty_pidkoryty_vezhu
%%parent 19_02_2019
 
%%url https://mrpl.city/blogs/view/razom-mi-zmozhemo-vse-dopomozhemo-kozhnomu-mariupoltsyu-pidkoriti-vezhu
 
%%author_id mariupol.vezha_centr
%%date 
 
%%tags kultura,mariupol,mariupol.kultura,mariupol.pre_war,mariupol.vezha
%%title Разом ми зможемо все: допоможемо кожному маріупольцю "підкорити" Вежу
 
%%endhead 
 
\subsection{Разом ми зможемо все: допоможемо кожному маріупольцю \enquote{підкорити} Вежу}
\label{sec:19_02_2019.stz.news.ua.mrpl_city.2.dopomogty_pidkoryty_vezhu}
 
\Purl{https://mrpl.city/blogs/view/razom-mi-zmozhemo-vse-dopomozhemo-kozhnomu-mariupoltsyu-pidkoriti-vezhu}
\ifcmt
 author_begin
   author_id mariupol.vezha_centr
 author_end
\fi

У Вежі 157 сходинок. Стільки треба пройти, аби піднятись нагору, і стільки ж,
аби спуститись. Шість поверхів. Висота найвищого – 12 метрів. Саме така довжина
сходин між третім та четвертим поверхом.

Насправді четвертий поверх – це колишня цистерна з водою (нагадуємо, що вежу
побудували як частину першого маріупольського водогону). Потім, коли вона вже
не використовувалась для подачі води до домівок маріупольців, всі насоси з
будівлі прибрали, в цистерні вирізали отвір, закріпили до нього сходи і стали
використовувати залізний бак як ще один поверх. Це досить непересічне видовище.
Піднімаєшся на третій поверх, а він височенний (12 м), йдеш крутими сходами і
опиняєшся в цистерні, ходиш по залізному дну.

\textbf{Читайте також:} \href{https://mrpl.city/blogs/view/vezha-plani-na-majbutne}{%
\enquote{Vezha}: плани на майбутнє, Культурно-туристичний центр \enquote{Вежа}, mrpl.city, 14.11.2018}

Працівники Вежі називають перехід з третього на четвертий поверх
\enquote{водорозділом}. Підняття сходинами – нелегка справа, тому ті відвідувачі, які
надто поспішають, дуже стомлюються в цій частині Вежі, на четвертому поверсі
перепочивають. До цього ж, не всі можуть подолати страх висоти. Частина людей
навіть не зважується далі підніматися.

\subsubsection{Історії наших відвідувачів}

\textbf{\emph{Кристина, 19 років, адміністраторка:}} \enquote{Декілька разів на тиждень і дорослі, і
діти повертаються, так і не піднявшись нагору. Кажуть, що страшно. Бачать
сходини, цистерну і не можуть змусити себе йти далі. Тобто вони не потрапляють
на оглядовий майданчик і не бачать діораму міста, задля якої всі сюди й
приходять}.

В таких випадках адміністратори заспокоюють людей, розповідають, що вони не
перші, хто побоявся і в цьому не має нічого страшного. І показують фото або
відео, що ж там далі є - вгорі.

\textbf{\emph{Галина, 53 роки, відвідувачка:}} \enquote{Начебто висоти не боюсь, але побачила, як треба
йти, то сказала подрузі: \enquote{Ти підіймайся, а я тебе тут почекаю}. Шкода,
звичайно, що не змогла осилити прогулянку вежею, але не можу себе силувати.
Мені потім показали фото верхніх поверхів}.

У Вежі 157 сходинок, і не тільки страх заважає деяким людям їх подолати. На
кожному поверсі розташовані меблі, на яких можна перепочити. Ми завжди
рекомендуємо не робити стрімких рухів, берегти себе та підійматися повільно.

\textbf{\emph{Павло, 68 років, відвідувач:}} \enquote{У мене вади серця, лікарі радять не
напружуватись. Я пройшов 4 поверхи і все – щемить, коле. Зупинився, перечекав,
посидів і спустився вниз}.

Хтось не може подолати ці сходи, він не побачить те, що бачить більшість.

\textbf{Читайте також:} Що таке \href{https://mrpl.city/blogs/view/shho-take-kovorking-i-z-chim-jogo-idyat-v-mariupoli}{%
\enquote{коворкінг} і з чим його їдять в Маріуполі, Культурно-туристичний центр \enquote{Вежа}, mrpl.city, 23.01.2019}

\textbf{\emph{Олена, 22 роки, відвідувач:}} \enquote{У мене зір - мінус 7. Інтер'єр
Вежі бачу чудово, можу до всього доторкнутись. А от вид з оглядового майданчика
для мене неповний – будівлі вдалині розпливаються, краєвид моря нечіткий. Тут
мені тільки бінокль допоможе. Але я звикла, тож просто уявила собі, що мала б
побачити}.

Ми залюбки проводимо екскурсії за запитом. Якщо у Вежі не проходить якийсь
захід, і адміністратор вільний, то він розкаже вам історію будівництва Вежі,
біографію її архітектора, подальшу історію будівлі, проведе по всім поверхам та
покаже цікавинки. Родзинка екскурсії – можливість за допомогою окулярів
доповненої реальності побачити центр міста зразка 1910 року. В електронному
вигляді відтворені за старими фото будівлі проспекту Миру (тоді вул.
Єкатерининська), деяких зараз немає, а деякі змінили зовнішній вигляд
кардинально. І ми зіткнулися з проблематичним питанням – як проводити нашу
екскурсію людям з вадами слуху?

І найболючіше для нас питання: як підняти нагору відвідувачів на колясках,
милицях, з ураженням опорно-рухового апарату тощо. 

\textbf{\emph{Кирило, 17 років, відвідувач:}} \enquote{Зламав ногу, в принципі, ходжу на милицях
нормально, коли гарна погода без дощу та снігу. Гуляли з другом у центрі і
зайшли у Вежу. Друг піднявся, а я його внизу почекав, почитав книжки. Нічого,
ще місяць – і я підіймусь і побачу таки ту Вежу, про яку всі говорять}.

У випадку Кирила проблема вирішиться сама по собі через деякий час. Ми бажаємо
йому скорішого одужання та чекаємо у гості. Але є випадки, коли час, на жаль,
не допоможе.

\textbf{Читайте також:} \href{https://mrpl.city/news/view/na-turisticheskoj-trope-v-mariupole-poyavilsya-pervyj-mini-nilsen}{%
На туристической тропе в Мариуполе появился первый \enquote{Мини-Нильсен},%
Анастасія Папуш, mrpl.city, 17.02.2019%
}

\subsubsection{Чому так сталося і хто може допомогти}

Будівля 1910 року не була розрахована на багаторазовий підйом великої кількості
людей щодня. Наглядачі насосної системи того часу – це були здорові чоловіки
середнього віку, які ходили крутими сходинами. З 1936 до кінця 1950-х років у
Вежі розташовувалась пожежна станція і її працівниками, знов-таки ж, були
дорослі чоловіки.

Після набуття статусу пам'ятника архітектури у 1983 році будь-які зміни у
фасаді та інтер'єрі Вежі – це велике питання. Чи треба його змінювати та
підлаштовувати під реалії сьогодення, чи залишити максимально автентичним? І
якщо ми добавили новітні предмети інтер'єру (меблі, ковані вироби,
освітлювання), то будівництво будь-яких підйомних механізмів у невеликій за
діаметром будівлі – це вже інша справа.

Тож ми задумались, як зробити так, щоб кожен відвідувач міг комфортно
перебувати у Вежі. Звичайно ж, ми будемо звертатись до муніципалітету за
підтримкою, знаходити грантові програми, але в нас дуже багато пунктів, які
треба вирішити. І тут необхідна допомога громади.

\textbf{Читайте також:} \href{https://mrpl.city/news/view/v-tsentre-mariupolya-poyavilas-novaya-mini-skulptura-nilsena-foto}{%
В центре Мариуполя появилась новая мини-скульптура Нильсена, %
Олена Онєгіна, mrpl.city, 18.02.2019%
}

\subsubsection{В планах у нас:}

\begin{itemize}
	\item 1. Створення об'ємних моделей Вежі, її інтер'єру та діорами міста, яка
	відкривається з нашого оглядового майданчика. До них можна доторкнутися і
	відчути.
  \item 2. Створення візуальних проекцій для людей з вадами слуху. Облаштування мультимедійних пристроїв з відео про Вежу з субтитрами.
  \item 3. Створення аудіозаписів для людей з вадами зору.
  \item 4. Навчання адміністраторів мови жестів.
  \item 5. Створення відеоекскурсії по Вежі.
  \item 6. Придбання біноклів на оглядовий майданчик.
\end{itemize}

Тож ми розпочали збір коштів на екскурсії для людей з особливими потребами та
інвалідністю. Скарбничка знаходиться на першому поверсі нашого центру. Будемо
поступово, залежно від коштів, виконувати усі пункти нашого списку. Ми наразі
консультуємось з різними організаціями аби доповнити та розширити наш список
необхідних дій у цьому напрямку. Та ми відкриті для ваших коментарів,
пропозицій, ідей. Надсилайте їх на поштову скриньку \url{vezha.pr@gmail.com}. \textbf{\emph{Разом
ми зможемо все!}}

\textbf{Читайте також:} \href{https://mrpl.city/blogs/view/fotomistetstvo-mariupolya-ekskurs-u-minule}{%
Фотомистецтво Маріуполя: екскурс у минуле, Ольга Демідко, mrpl.city, 05.02.2019}
