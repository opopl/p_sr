% vim: keymap=russian-jcukenwin
%%beginhead 
 
%%file 26_10_2019.stz.news.ua.mrpl_city.1.starye_avtomobili_mariupolja
%%parent 26_10_2019
 
%%url https://mrpl.city/blogs/view/ko-dnyu-avtomobilista-pro-starye-avtomobili-mariupolya-1
 
%%author_id burov_sergij.mariupol,news.ua.mrpl_city
%%date 
 
%%tags 
%%title Ко Дню автомобилиста: про старые автомобили Мариуполя
 
%%endhead 
 
\subsection{Ко Дню автомобилиста: про старые автомобили Мариуполя}
\label{sec:26_10_2019.stz.news.ua.mrpl_city.1.starye_avtomobili_mariupolja}
 
\Purl{https://mrpl.city/blogs/view/ko-dnyu-avtomobilista-pro-starye-avtomobili-mariupolya-1}
\ifcmt
 author_begin
   author_id burov_sergij.mariupol,news.ua.mrpl_city
 author_end
\fi

Как-то наш сосед, отставной боцман торгового судна Саша, подергав-подергав свое
\enquote{Рено}, наконец-то пристроил его на стоянку в самый конец вереницы \enquote{Фордов},
\enquote{Лад}, \enquote{Ситроенов} и других изделий автопромов разных стран вдали от своего
подъезда. После этого он произнес следующую тираду: \enquote{Говорят, народ бедный, а
машину приткнуть некуда}. Здесь, конечно, Саша погорячился. Далеко не у всех
владельцев квартир нашего девятиэтажного кооперативного дома есть собственные
средства передвижения, даже если включить в этот список складные велосипеды.
Но, правда и то, что на относительно небольшом пространстве нашей придомовой
территории можно встретить автомобили фирм Франции и Германии, Южной Кореи и
Японии, России и США, Италии и даже Румынии, не говоря уже о нашей, украинской.

\ii{26_10_2019.stz.news.ua.mrpl_city.1.starye_avtomobili_mariupolja.pic.1}

\textbf{Читайте также:} 

\href{https://mrpl.city/news/view/zavodi-motor-v-den-avtomobilista-v-mariupole-provedut-ralli-i-figurnyj-pilotazh}{
%
Заводи мотор: в День автомобилиста в Мариуполе проведут ралли и фигурный пилотаж, %
Олександра Невська, mrpl.city, 24.10.2019}

Нечаянно подслушав \enquote{крик души} боцмана Саши нежданно-негаданно вспомнилось
детство, которое пришлось на время войны и первые послевоенные годы. В сознании
предстал почти зримо дедушкин дом, окно, выходящее на Торговую улицу,
подоконник, на котором можно было уютно пристроиться и смотреть, смотреть на
беспрестанно меняющиеся картины там, за стеклом. Именно рядом с домом дедушки
начиналась крутизна улицы. Автомобили, едущие вниз, притормаживали, движущиеся
вверх - прибавляли газу, извозчики же с большим усердием \enquote{поощряли} лошадей
кнутами. Соотношение гужевого транспорта к автомобильному было не в пользу
последнего. Запомнились груженые подводы и тянущие их низкорослые лошаденки.
Взрослые говорили, что привезли их из Монголии, в качестве братской помощи
Советскому Союзу, истерзанному войной. В транспортном потоке иногда можно было
увидеть линейки. Сейчас трудно объяснить людям даже среднего возраста, какими
были эти экипажи для четырех седоков, включая и возницу.

Но не подводы занимали в первую очередь внимание малолетнего обозревателя.
Машины! Вот для чего нужно было карабкаться на подоконник. Трудяги грузовики
ЗиС-5, полуторки ГАЗ–А, \enquote{детища} Горьковского автозавода, купленного в 30-е
годы прошлого, двадцатого, века за золото у американцев. Нередко мелькали
полученные по ленд-лизу самодовольные американские грузовики \enquote{Студебеккеры},
\enquote{Доджи}, \enquote{Шевроле} и малютки \enquote{Виллисы}. Порой можно было увидеть и трофейные
легковушки, \enquote{БМВ}, \enquote{Фольксвагены}, и очень уж редко - изящные \enquote{Мерседесы}.
Некоторое приевшееся однообразие транспортного потока \enquote{разбавлялось} грохотом
танка Т-34 с огромным кузовом из листовой стали, укрепленном на месте башни с
пушкой. Вот это был грузовик, что надо!

\textbf{Читайте также:}

\href{https://mrpl.city/news/view/vladeltsy-e-litnyh-avto-v-mariupolerekordsmeny-oblasti-po-uplate-transportnogo-naloga}{Владельцы элитных авто в Мариуполе – рекордсмены области по уплате транспортного налога, mrpl.city, 24.10.2019}

Вероятно, в описываемую эпоху требования \enquote{гаишников} к шоферам, - именно так
называли привычных для нас водителей автомобилей, - были совсем не строги. Во
всяком случае, наш сосед шофер Усманов, приезжая на своем \enquote{газике} пообедать,
сначала шел в рядом находящееся предприятие общепита, где из-под полы
\enquote{подавали} стаканчик. А уж потом отправлялся к своей Аглие, жене и матери их
трех мальчишек.

Довольно продолжительное время что-то не приходилось слышать, что у кого-то из
мариупольцев в частной собственности был автомобиль. Поговаривали, будто
сталевару-стахановцу Макару Мазаю была подарена \enquote{Эмка} - легковушка
отечественного производства М-1.

Но вот где-то в 47 году в нашем городе появилось несколько \enquote{Побед} ГАЗ-М-20.
Благодаря необычной форме они сразу бросались в глаза. Кажется, первые в
Мариуполе \enquote{Победы} были бежевого цвета. Позже появились и серые с шашечками
вдоль кузова – такси. Примерно в это время кое-кто из горожан купил себе это
изделие Горьковского автозавода. Некоторые обладатели этих машин хвастались,
что только в них можно было ездить, не снимая шляпы. Тогда же появились и
изделия московского автозавода - \enquote{Москвичи}, подозрительно похожие на немецкие
\enquote{Опель кадеты}.

\textbf{Читайте также:} \href{https://mrpl.city/blogs/view/na-chem-priehat-v-mariupol}{%
На чем приехать в Мариуполь?, mrpl.city, 31.03.2019}

Через какое-то время в нашем городе появились \enquote{Победы} и \enquote{Москвичи}, дверцы у
которых были сделаны не из стального листа, а из лакированной авиационной
фанеры. Кажется, это случилось в тот период, когда началась \enquote{холодная война}.
Поговаривали, будто прекратились поставки тонкого холоднокатаного листа. И
продолжалась такая ситуация до тех пор, пока на \enquote{Запорожстали} не освоили
производство листового металла, необходимого для отечественного автопрома. В
1956 году началось массовое производство легкового автомобиля среднего класса
ГАЗ-21 \enquote{Волга}. Приходилось слышать от взрослых, что первым обладателем 21-й
\enquote{Волги} у нас в Мариуполе был У. Е. Мальченко, чья профессия была -
протезирование зубов, а страсти - охота и автомобили.
