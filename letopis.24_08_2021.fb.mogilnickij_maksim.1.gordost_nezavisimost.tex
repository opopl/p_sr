% vim: keymap=russian-jcukenwin
%%beginhead 
 
%%file 24_08_2021.fb.mogilnickij_maksim.1.gordost_nezavisimost
%%parent 24_08_2021
 
%%url https://www.facebook.com/permalink.php?story_fbid=6226985490704930&id=100001806235447
 
%%author Могильницкий, Максим
%%author_id mogilnickij_maksim
%%author_url 
 
%%tags gordost',nezalezhnist,ukraina
%%title Нам искренне хотелось бы, но гордиться без причины способен лишь дурак
 
%%endhead 
 
\subsection{Нам искренне хотелось бы, но гордиться без причины способен лишь дурак}
\label{sec:24_08_2021.fb.mogilnickij_maksim.1.gordost_nezavisimost}
 
\Purl{https://www.facebook.com/permalink.php?story_fbid=6226985490704930&id=100001806235447}
\ifcmt
 author_begin
   author_id mogilnickij_maksim
 author_end
\fi

Нам ведь искренне хочется гордиться. В этот день – особенно. Гордиться нашим
добродушным, щедрым и гостеприимным народом, красотой природы и богатством
недр, экономическим ростом и научными достижениями, мудростью государства и
защищенностью гражданина.

Всем и каждому хотелось бы рассказывать о достижениях наших спортсменов,
шедеврах, созданных деятелями культуры, о качестве образования, доступности
медицины и надежности полиции. Хотелось бы.

Сунуть в мешок, а после вынести на помойку все то, что разъединяет и унижает.
Чтобы не было впредь насильственной украинизации и вездесущего «колорита» по
разнарядке, фрейдизма гигантских флагштоков и комичных «вышиваночных»
марафонов, трезубцев из тортов и гопака в небе.

Символы – штука полезная, но вовсе не первоочередная. Символы, как бы странно
это ни звучало, должны хоть что-то символизировать. И что же символизирует
гигантский флагшток за многие миллионы, склонившийся над обнищавшей страной?

Нам искренне хотелось бы, но гордиться без причины способен лишь дурак. За
символами должны стоять независимость, стабильность и могущество государства. В
противном же случае это не более, чем яркие картинки.
