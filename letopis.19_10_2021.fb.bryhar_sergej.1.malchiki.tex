% vim: keymap=russian-jcukenwin
%%beginhead 
 
%%file 19_10_2021.fb.bryhar_sergej.1.malchiki
%%parent 19_10_2021
 
%%url https://www.facebook.com/serhiibryhar/posts/1813902838809660
 
%%author_id bryhar_sergej
%%date 
 
%%tags deti,jazyk,mova,ukraina,ukrainizacia
%%title Вчора ми вже були "мальчіками". Трошки
 
%%endhead 
 
\subsection{Вчора ми вже були \enquote{мальчіками}. Трошки}
\label{sec:19_10_2021.fb.bryhar_sergej.1.malchiki}
 
\Purl{https://www.facebook.com/serhiibryhar/posts/1813902838809660}
\ifcmt
 author_begin
   author_id bryhar_sergej
 author_end
\fi

Назар сьогодні порадував.

Сидимо в пісочниці.

Підходять жінка (60+) з онучкою (3+, ближче до 4-х). 

Мала відразу ж кидається до наших іграшок.

\ifcmt
  ig https://scontent-lga3-2.xx.fbcdn.net/v/t39.30808-6/245640616_1813902812142996_5735002311565945963_n.jpg?_nc_cat=107&ccb=1-5&_nc_sid=8bfeb9&_nc_ohc=CibXCyipLxEAX-wXZfD&_nc_ht=scontent-lga3-2.xx&oh=f764fd65d8a54973e27cd0f6f895d25d&oe=6174512C
  @width 0.4
  %@wrap \parpic[r]
  @wrap \InsertBoxR{0}
\fi

Мої хлопці реагують спокійно.

- Сонєчка, так нєльзя. У тєбя єсть своі ігрушкі. Потом может помєняєтєсь.

Сонечці - по цимбалах. Все одно рветься. Бабуся її стримує. Хлопці спостерігають.

- Смотрі, Сонєчка, - вивалює з сумки всі іграшки, що там були: - Вот етім іграєшся ти, - показує пальцем, - а вон тєм - мальчікі.

Сонечка зупиняється.

- Ми не "мальчікі", - відповідає мій старший.

- А кто ви?

- Ми гарні хлопчики.

- Ааааа. Вон оно что.

- Так. Вчора ми вже були "мальчіками". Трошки.

- І що, тобі не сподобалося?

- Так, не сподобалося. Сьогодні ми знову хлопчики.

- Молодці.

- Так...
