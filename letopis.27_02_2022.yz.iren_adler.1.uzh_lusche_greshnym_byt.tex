% vim: keymap=russian-jcukenwin
%%beginhead 
 
%%file 27_02_2022.yz.iren_adler.1.uzh_lusche_greshnym_byt
%%parent 27_02_2022
 
%%url https://zen.yandex.ru/media/iren_adler/uj-luchshe-greshnym-byt-6219e5bc9c11ee526c40ff76
 
%%author_id yz.iren_adler
%%date 
 
%%tags __feb_2022.vtorzhenie,rossia
%%title Уж лучше грешным быть...
 
%%endhead 
 
\subsection{Уж лучше грешным быть...}
\label{sec:27_02_2022.yz.iren_adler.1.uzh_lusche_greshnym_byt}
 
\Purl{https://zen.yandex.ru/media/iren_adler/uj-luchshe-greshnym-byt-6219e5bc9c11ee526c40ff76}
\ifcmt
 author_begin
   author_id yz.iren_adler
 author_end
\fi

Я вам сейчас честно расскажу, почему я не написала немедленно о происходящем. В
первую очередь, для меня это, конечно, было оглушительной неожиданностью, и я
первые часы неотрывно смотрела новости. В промежутках бегала на работу. А в
других промежутках вела разъяснительную работу по два часа. Сосредоточиться на
написании статьи не было возможности. Что за разъяснительная работа? Я же тут в
очень узком кругу считаюсь этаким политическим \enquote{гуру}. И вот участники
этого узкого круга с утра четверга начали мне звонить и требовать пояснений. И
на каждого я тратила по часу, а то и больше. Разволновалась до подскока
давления (давление - это наследственное).

\ifcmt
  ig https://avatars.mds.yandex.net/get-zen_doc/1671180/pub_6219e5bc9c11ee526c40ff76_6219ef4816b4077ea9a0d503/scale_1200
  @caption Яндекс.Картинки
  @wrap center
  @width 0.8
\fi

Для начала позвонила та заполошенная мама, которой я когда-то объясняла
ситуацию с Крымом. Я проводила аналогию с дочкой, которую по расчету отдали
замуж, если помнят мои читатели. И начала эта мама чуть ли не трагически: там
ракетами по городам стреляют! Пришлось долго и нудно объяснять, что стреляют не
по городам, а по объектам военной инфраструктуры. Чуть позже объявилась моя
собственная матушка. Там почти ничего объяснять не понадобилось, но пришлось
поучаствовать в жалобах на отсутствие префронтальной доли головного мозга у
польских политиков. Потом приехала та самая моя ученица-бюрократка. И вместо
урока получилась политинформация. Хотя она тоже уже наш человек. Сама там
пытается проводить нашу линию, хотя ей, конечно, это делать нежелательно. Тем
более, как она призналась, у нее не хватает доказательной базы, она в ситуации
не разбирается. Она в общем понимает, что предпринятая Россией спецоперация
оправдана всеми обстоятельствами, но четко и ясно донести до окружающих не
может. Потом еще один ученик сразу же потребовал объяснений, что меня
порадовало. Пацан всего в шестом классе, но мыслит правильно. Потом со знакомым
кремлеботом обсуждали все происходящее часа три. Потом была та самая вменяемая
киевлянка, о которой я тоже уже рассказывала. Она мне поведала о том, как
пыталась вразумить свою многочисленную родню, и как у нее не получилось. Я в
очередной раз поразилась ее разумности и вменяемости. И вот так два дня! А
вчера вечером опять позвонил знакомый кремлебот, который помогал мне с этим
роликом к 23 февраля, и мы три часа параллельно смотрели новости и обсуждали. И
уже просто не до статьи. Но постоянно висела мысль, как и что надо написать.

Вообще, мое отношение к происходящему можно выразить одной хорошо известной
фразой из сонета Шекспира: Уж лучше грешным быть, чем грешным слыть... Гораздо
обидней, когда тебя обличают в каких-то совершенно немыслимых деяниях, даже не
пытаясь понять и выслушать, и далеко не так обидно, когда ты эти деяния
совершаешь. Тем более, что деяния эти совершаются во имя правого дела. Ну вот
стоит сейчас визг - санкции! Санкции! Да он бы и так стоял, этот визг. И
судороги были бы, и корчи, и припадки. Придумали бы еще какого-нибудь Скрипаля.
Это такой нескончаемый процесс. И люди трезвомыслящие давно уже поняли, что как
бы Россия не пыталась подружиться с Западом, у нее не получится. Потому что у
них неколебимая вера в собственное превосходство. 

\ifcmt
  ig https://avatars.mds.yandex.net/get-zen_doc/5240577/pub_6219e5bc9c11ee526c40ff76_6219f013be00b3339fb92372/scale_1200
	@caption Яндекс.Картинки
  @wrap center
  @width 0.8
\fi

Я уже рассказывала на своем примере. И не только на своем. Какая-нибудь немка с
окончанием Хауптшуле, которая всю жизнь безработная, потому что ничего делать
не умеет и не хочет, но зато ее немецкий как бы родной, невзирая на то, что она
не читала ни Гете, ни Шиллера, и вообще не знает кто это, неколебимо верует в
то, что она выше меня, говорящей на четырех языках, с университетским
образованием. Да, мой немецкий далек от совершенства, но я на нем говорю, а вот
ей даже английский не по зубам. Язык их заклятых друзей.

И вот у них это сидит! Они - высшая раса. Они все умеют и все знают. У них все
по правилам и по законам. А Россия - это страна дикарей. Вот в среду мы с
коллегами обсуждали отпуск, две немки и нас двое, вторая - из Николаева. Я с
ней много лет дружу, заскоки есть, но не критичные. И вот одна из немок вдруг
начинает выступать, что Путин отключит газ и что она Путина ненавидит. У меня
было большое желание как у кота Базилио \enquote{Щас в роже вцеплюсь!} 

\ifcmt
  ig https://avatars.mds.yandex.net/get-zen_doc/2355127/pub_6219e5bc9c11ee526c40ff76_6219efa6cddbf118aa8c3381/scale_1200
	@caption Яндекс.Картинки
\fi

Но я, естественно, промолчала. А смысл? Там ничего не докажешь и не расскажешь.
Там стена. Правда, вторая немка меня приятно удивила. Она одернула свою товарку
и сказала, что это вообще не ее ума дело, мозгов у нее на это нет и ее дело с
отпуском определиться. На следующий день я встретилась с еще одной немкой,
пожилой дамой, и та меня тоже приятно удивила. Когда я ее спросила, осторожно,
как она понимает ситуацию, она сразу ответила: \enquote{Наши ашлохи
(ругательное слово) все запороли!}

Сейчас много писать не буду. Через полчаса уроки. Но сказать хочется очень
много. И еще очень хочется пройтись по всем этим деятелям культуры и по этой
так называемой \enquote{интеллихенции}. А пока хочу сказать:

Если не вернусь, прошу к интеллигенции меня не причислять!
