% vim: keymap=russian-jcukenwin
%%beginhead 
 
%%file 03_12_2020.fb.korchynskyi_dmytro.1.zavdannja_hristiane
%%parent 03_12_2020
 
%%url https://www.facebook.com/korchynskyi/posts/3536125223119651
 
%%author Корчинський, Дмитро
%%author_id korchynskyi_dmytro
%%author_url 
 
%%tags 
%%title ПОВТОРЕННЯ - Перед українськими християнами всіх конфесій стоять кілька першочергових завдань
 
%%endhead 
 
\subsection{ПОВТОРЕННЯ - Перед українськими християнами всіх конфесій стоять кілька першочергових завдань}
\label{sec:03_12_2020.fb.korchynskyi_dmytro.1.zavdannja_hristiane}
\Purl{https://www.facebook.com/korchynskyi/posts/3536125223119651}
\ifcmt
	author_begin
   author_id korchynskyi_dmytro
	author_end
\fi

\index[rus]{Християни!Завдання, Корчинський}

ПОВТОРЕННЯ

Перед українськими християнами всіх конфесій стоять кілька першочергових завдань:

\begin{itemize}
\item 1. Заборона абортів. Вбиваючи дітей нація щоденно приносить жертву
				сатані. Цей страшний гріх лягає не лише на тих, хто здійснює, але також
				на тих, хто мириться, в першу чергу на тих, хто розуміє, на нас.
				Українські християни ігнорують проблему. Це ганьба.  Мають бути
				атаковані абортарії, відлучений від церкви і проклятий кожний депутат,
				який відмовиться голосувати за заборону.

\item 2. Активна протидія лицемірам. Найгіршими ворогами Церкви є не язичники і
				безбожники, але ті, хто ВДАЮТЬ християн. Це в першу чергу УПЦ ФСБ. Саме
				вона продукує 90\% антихристиянських настроїв у Вкраїні. Необхідно
				активізувати пропаганду тих, хто одурений цією синагогою сатани,
				боротися проти функціонерів.

\item 3. Протидія маніхейству.

\item 4. Багатократне посилення волонтерства. Християнин — це волонтер.

\item 5. Мілітаризація церковних спільнот. Кожне братство, громада, церква
				мають являти собою бойовий відділ готовий до дії у вуличній акції і в
				бою. Для цього необхідно налагодити їхнє навчання, злагодження,
				матеріально-технічне забезпечення та обмін досвідом.

\item 6. Через капеланство збільшення церковного впливу на Збройні сили, які в
				перспективі мають стати Армією Бога.

\item 7. Налагодження якнайтіснішої взаємодії між трьома гілками християнства.
				Утворення широкого християнського фронту.

\item 8. Фізична протидія інформативно-пропагандистсько-розважальним засобам
				олігархів. Практично всі електронні ЗМІ деморалізують націю і мають
				бути придушені, або колись конфісковані християнами.

\item 9. У зв’язку з загальним отупінням населення і деградацією освіти, саме
				Церква має стати засобом виховання цілісної людини. Церква — не лише
				знаряддя спасіння, але також спільнота осягнення. Внутрішньоцерковна
				освіта має стати універсальною.

\item 10. Для виконання всіх вищеперелічених завдань, необхідне збільшення
				мобілізаційного потенціалу Церкви. Солідарність, боєздатність,
				варіативність. Амінь.
\end{itemize}
