% vim: keymap=russian-jcukenwin
%%beginhead 
 
%%file 18_08_2018.stz.news.ua.mrpl_city.1.kniga_knig_o_mariupole
%%parent 18_08_2018
 
%%url https://mrpl.city/blogs/view/kniga-knig
 
%%author_id burov_sergij.mariupol,news.ua.mrpl_city
%%date 
 
%%tags 
%%title Книга книг о Мариуполе
 
%%endhead 
 
\subsection{Книга книг о Мариуполе}
\label{sec:18_08_2018.stz.news.ua.mrpl_city.1.kniga_knig_o_mariupole}
 
\Purl{https://mrpl.city/blogs/view/kniga-knig}
\ifcmt
 author_begin
   author_id burov_sergij.mariupol,news.ua.mrpl_city
 author_end
\fi

Богатыми людьми были мариупольские деловые люди Хараджаевы, очень богатыми.
Владели они кирпичными заводами, лучшими домами в городе, хлебными ссыпками и
даже пароходами. Занимались ли они благотворительностью из-за широты натуры
своей? А быть может, такова традиция была среди состоятельных людей? Хотелось
ли им, чтобы и после ухода в мир иной осталась память о них? Кто сейчас скажет?
Но жертвовали Хараджаевы немалые средства на церкви, на содержание учебных
заведений и больниц. Однако время не прислушивается к людским желаниям. Давно
стерты с лица земли мариупольские храмы, исчезли кирпичные заводы Хараджаевых,
и лишь очень редкие старожилы да некоторые дотошные краеведы помнят, где стояли
некогда хараджаевские дома и амбары.

\textbf{Читайте также:} 

\href{https://archive.org/details/14_04_2018.sergij_burov.mrpl_city.kogda_byl_osnovan_mariupol}{%
Когда был основан Мариуполь?, Сергей Буров, mrpl.city, 14.04.2018}

А памятником Хараджаевым, во всяком случае, одному из них - Давиду
Александровичу, стало дело, на которое они потратили деньги, может быть, по их
понятиям, совсем плевые. Когда весной 1892 года решено было организовать
экскурсии для мариупольских гимназистов, преподаватели мужской Александровской
гимназии принялись подготавливать тексты лекций, которые должны были предварить
каждую из экскурсий. Раздел \enquote{Топография и климат г. Мариуполя} написал
инспектор Н. Ф. Вавилов, преподаватель древних языков А. Ф. Петрашевский составил
обзор по инославным общинам, лечебным и учебным учреждениям, клубам, театру и
тому подобное, математику М. И. Кустовскому довелось составить обзор \enquote{Фабричная
и заводская деятельность г. Мариуполя}, а почетному попечителю гимназии
Д. А. Хараджаеву - \enquote{Торговля и торговые учреждения} и т.д.

\ii{18_08_2018.stz.news.ua.mrpl_city.1.kniga_knig_o_mariupole.pic.1}

Но самый большой объем книги был написан директором гимназии Григорием
Ивановичем Тимошевским. Это \enquote{Переселение православных христиан из Крыма в
Мариупольский уезд Азовской, ныне Екатеринославской губернии}, \enquote{Основание г.
Мариуполя и некоторые данные к его истории}, \enquote{Духовное и гражданское
самоуправление}, \enquote{Православные храмы}. В связи с этим один уважаемый
мариупольский краевед, к сожалению, умерший, поставил под сомнение авторство
директора гимназии. Мол, прибыл Г. И. Тимошевский в Мариуполь в 1891 году,
первая экскурсия назначена на 1 марта 1892 года, а тут еще нужно ознакомиться с
новым местом, новыми подчиненными. Когда ему писать? Где взять время? Стало
быть, не директор тексты составил. И подписанные им тексты – чистый плагиат.

\ii{18_08_2018.stz.news.ua.mrpl_city.1.kniga_knig_o_mariupole.pic.2}

Но, зная из художественной литературы нравы и обычаи интеллигенции конца XIX
века, такой поступок был невозможен. Человек, имевший университетское
образование, свободно владел пером. Но даже не в этом дело. Критик нового
директора, наверное, не знал, что Григорий Иванович, будучи директором
Симферопольской мужской гимназии, дважды организовывал экскурсии своих
воспитанников по Крыму в 1886 и в 1888 годах. По результатам экскурсий 1888 г.
был издан типографским способом отчет. И что-что, а историю Крыма, в том числе
и переселение христиан с полуострова в Северное Приазовье он знал досконально.

Экскурсии мариупольских гимназистов были проведены за одиннадцать дней - с 1
марта по 10 мая. Лекции прочитаны, но организаторам показалось, что этого мало.
Решено было издать лекции типографским способом. Сказано - сделано. А сделано
потому, что деньги на издание лекций пожертвовал местный предприниматель и
почетный попечитель гимназии Давид Александрович Хараджаев. И получилась
книга... Поражает оперативность владельца мариупольской типографии Франтова. 10
мая 1892 года была проведена последняя экскурсия, и в этом же году первые
читатели открыли пахнущую свежей типографской краской книгу. А ведь в те
времена не было линотипов, все 500 страниц набраны, да что набраны, -
отпечатаны, сшиты и переплетены вручную.

Сегодня остались считанные экземпляры \textbf{\enquote{Отчета об учебных экскурсиях
Мариупольской Александровской гимназии. Мариуполь и его окрестности}}. Именно о
нем идет здесь речь. Два или три из них хранятся в Мариупольском краеведческом
музее, какое-то количество - в частных собраниях. Говорят, есть по экземпляру
этого уникального издания в Москве, в фондах Российской национальной
библиотеки, да еще в Санкт-Петербурге, в библиотеке имени Салтыкова-Щедрина. То
есть за границей. Хоть мариупольский типограф Франтов и добыл для книги плотную
бумагу, прочные льняные нитки для сшивки, крепкий картон для переплета и
добротную кожу для корешка, неумолимое время взяло свое. Ветшает, увы, книга и
может и вовсе исчезнуть. Если и не исчезнуть, то стать доступной еще более
ограниченному кругу людей, чем сейчас. Положение с доступом к книге несколько
улучшилось после того, как усилиями доктора физико-математических наук,
профессора Стефана Алексеевича Калоерова не было отпечатано два репринтных
издания (2001 и 2007). К сожалению, тираж их невелик.

\ii{18_08_2018.stz.news.ua.mrpl_city.1.kniga_knig_o_mariupole.pic.3}

Кроме тем, названных выше, на страницах книги идет неспешное повествование о
митрополите Игнатии, возглавившем переселение, об особо замечательных событиях
- к ним отнесены, например, посещения Мариуполя императором Александром I,
сообщено даже о стоимости подарков, преподнесенных монарху верноподданными
мариупольцами. Подробно и обстоятельно написано о православных (и иных
вероисповеданий) храмах. И хорошо авторы сделали - иначе о многих из них мы бы
и не знали. Не забыты были флора окрестностей города, статистические данные,
должное отдано местному театру и его основателям, быту, нравам и обычаям
мариупольских греков. Завершалась книга нотной записью мелодий греческих песен
и словами к ним, собранными и записанными одним из преподавателей.

\textbf{Читайте также:} \href{https://mrpl.city/news/view/ne-stalyu-edinoj-mariupolskie-fabriki-i-zavody-rastvorivshiesya-vo-vremeni}{%
Не сталью единой: Мариупольские фабрики и заводы, растворившиеся во времени, Кіра Булгаковa, mrpl.city, 14.08.2018}

Создатели книги отнеслись к делу столь серьезно, что и сегодня к ее помощи
прибегают краеведы, историки, этнографы, языковеды, литераторы и журналисты,
усердно цитируя ее, если речь идет о Мариуполе, правда, не всегда и не все, к
сожалению, делают соответствующие ссылки. А книга, поверьте, интересна для
многих, очень многих читателей, особенно в Мариуполе.
