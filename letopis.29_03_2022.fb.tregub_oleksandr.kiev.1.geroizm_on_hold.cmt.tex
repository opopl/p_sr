% vim: keymap=russian-jcukenwin
%%beginhead 
 
%%file 29_03_2022.fb.tregub_oleksandr.kiev.1.geroizm_on_hold.cmt
%%parent 29_03_2022.fb.tregub_oleksandr.kiev.1.geroizm_on_hold
 
%%url 
 
%%author_id 
%%date 
 
%%tags 
%%title 
 
%%endhead 
\zzSecCmt

\begin{itemize} % {
\iusr{Olga Khvorostetskaya}

А ще Франкл писав, що вижили ті - хто мав смисл, сенс \enquote{заради чого
жити}  @igg{fbicon.heart.yellow}  @igg{fbicon.heart.blue}  сама книга має назву \enquote{У пошуках смислу}. Це мабуть
найголовніше сьогодні.

\iusr{Inna Prihodko}
\textbf{Olga Khvorostetskaya} в гарний зошит написала плани на весну та літо, перемога і море в планах є!

\iusr{Оксана Баранова}

Теж прочитала цю книгу за рік до війни Книга допомогла зберегти себе і рухатись
далі.

\iusr{Denys Sudilkovsky}
Улюблена книга випадково вчасно прочитана за місяць до війни.

\begin{itemize} % {
\iusr{Alexander Tregub}

після мемуарів з задоволенням прочитав і його наукові роботи, теж гарні

\iusr{Yulia Storozhuk}
Раджу ще \enquote{The Choice} Edith Eger.
\end{itemize} % }

\iusr{Dima Boychuk}

Героям слава!

Саша, відновлюйте роботу. Мене вже люди питають де вчитись веб-дизайну.

\iusr{Alexander Tregub}
\textbf{Dima Boychuk} вже  @igg{fbicon.smile} 

\iusr{Yana Bohdanova}

Бачила що розповсюджують новину, начебто прожектор надав доступ до 12 курсів
безкоштовно. Хотіла знайомим скинути, але не можу знайти у вас ніякої інфи.
Фейк?


\end{itemize} % }
