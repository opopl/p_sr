% vim: keymap=russian-jcukenwin
%%beginhead 
 
%%file slova.leto
%%parent slova
 
%%url 
 
%%author 
%%author_id 
%%author_url 
 
%%tags 
%%title 
 
%%endhead 
\chapter{Лето}
\label{sec:slova.leto}

%%%cit
%%%cit_pic
%%%cit_text
Вроде бы простенький вопрос, но неподготовленного человека он может поставить в
тупик. Мало того, что \emph{летоисчисление} было совсем другим, так еще и с цифрами
проблема - ни арабских, ни римских в древних отечественных летописях не
встретишь. Ну и как же тогда сообщить читателю требуемую дату?
%%%cit_comment
%%%cit_title
\citTitle{Как писали даты в Древней Руси?}, 
Русичи, zen.yandex.ru, 07.06.2021
%%%endcit

%%%cit
%%%cit_head
%%%cit_pic
\ifcmt
  tab_begin cols=3
     pic https://regnum.ru/uploads/pictures/news/2021/10/01/regnum_picture_16330825763056635_normal.JPG
     pic https://regnum.ru/uploads/pictures/news/2021/10/01/regnum_picture_16330825792553960_normal.JPG
		 pic https://regnum.ru/uploads/pictures/news/2021/10/01/regnum_picture_16330825952392120_normal.JPG
  tab_end
\fi
%%%cit_text
Будет ли \emph{бабье лето} в этом году, это еще вопрос. По прогнозам синоптиков,
температура воздуха в Подмосковье в первой половине октября не составит выше
8−11 градусов днем, ночью может опускаться до 2 градусов тепла
%%%cit_comment
%%%cit_title
\citTitle{В Москве ждут бабьего лета-2021 — фоторепортаж}, 
Наталья Стрельцова, regnum.ru, 01.10.2021
%%%endcit


