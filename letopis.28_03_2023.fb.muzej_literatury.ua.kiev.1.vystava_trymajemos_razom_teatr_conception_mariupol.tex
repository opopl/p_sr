%%beginhead 
 
%%file 28_03_2023.fb.muzej_literatury.ua.kiev.1.vystava_trymajemos_razom_teatr_conception_mariupol
%%parent 28_03_2023
 
%%url https://www.facebook.com/NMLUmuseumlit/posts/pfbid02GmNYW1UvYVLywMFTs8rdLshPKvvLK1AeLbYKnNCGSpmMXoaFrug3Tn67GB4poUZil
 
%%author_id muzej_literatury.ua.kiev
%%date 28_03_2023
 
%%tags 
%%title Поетично-театралізована вистава "Тримаємось разом" незалежного маріупольського театру авторської п’єси "Conception"
 
%%endhead 

\subsection{Поетично-театралізована вистава \enquote{Тримаємось разом} незалежного маріупольського театру авторської п'єси \enquote{Conception}}
\label{sec:28_03_2023.fb.muzej_literatury.ua.kiev.1.vystava_trymajemos_razom_teatr_conception_mariupol}

\Purl{https://www.facebook.com/NMLUmuseumlit/posts/pfbid02GmNYW1UvYVLywMFTs8rdLshPKvvLK1AeLbYKnNCGSpmMXoaFrug3Tn67GB4poUZil}
\ifcmt
 author_begin
   author_id muzej_literatury.ua.kiev
 author_end
\fi

У Міжнародний день театру в \href{\urlMuzejLiteraturyUkrainyIA}{Національному музеї літератури України} пройшла
благодійна акція \enquote{Я, театр і війна} на підтримку \href{\urlTeatrConceptionMariupolIA}{незалежного маріупольського
театру авторської п'єси \enquote{Conception}} (режисер Олексій Гнатюк) на якій було
представлену поетично-театралізовану виставу \enquote{Тримаємось разом}. Акція
проводилась спільно з Департаментом культури маріупольської міської ради та
Департаментом культурно-громадського розвитку Маріупольської міської ради її
директоркою Діаною Тримою Diana Tryma. 

До заходу також долучилась відома сучасна драматургиня Ніна Захоженко – авторка
серії драматичних новел про життя цивільних українців під час війни. Пані Ніна
співпрацює з низкою державних і незалежних театрів як діджитал-художниця та
культурна менеджерка. Вона активно волонтерить, підтримує українське військо. 

\href{\urlMariupolIA}{Маріуполь}... 

Як багато емоцій, переживань та почуттів українців пов'язані з цим знищеним,
але нескореним українським містом. Це місто із солі та сталі, місто величних
людей, місто яке стояло навіть тоді, коли плавився метал та ламався бетон. Це
місто сили, місто надії, місто Марії. 

\href{\urlMariupolDramTeatrIA}{Маріупольський театр} назавжди залишиться для нас \enquote{перлиною сходу}, місцем
талановитих і навіжено закоханих у свою справу акторів, місцем жахливого
військового злочину. Рашисти знищили будівлю, але неможливо знищити ідею,
неможливо знищити культуру. Вона розквітає в інших місцях та одного дня
обов'язково повернеться до своєї відбудованої обителі.  

\href{\urlTeatrConceptionMariupolIA}{Театр авторської п'єси
\enquote{Conception}} із \href{\urlMariupolIA}{міста Маріуполь}, – це театр
якому вдалося вирватися з окупації і подолати багато кілометрів для того, щоб
знову зібратися разом, відродити культуру міста Марії, презентувати нам свою
акторську майстерність та репертуар. Театр почав свій шлях 2019-го року у
рідному місті. Молодий театр брав участь у багатьох масштабних заходах свого
міста, а згодом підкорив всеукраїнську та міжнародну сцену. Зараз акторів
радісно зустрічають на всіх великих сценах \href{\urlUkrainaIA}{України}.

За допомогою художнього слова актори театру розповідають глядачеві, про
сучасні реалії українського народу під час війни та його незламність.
\enquote{Тримаємось разом} – це вірші, обпалені війною, які глибоко торкаються струн
людської душі, викликають сльози, гнів, біль і в той же час дарують розраду,
вселяють віру і надію в нашу Перемогу.

Це було щемливо, емоційно, неймовірно! Ми всі і плакали, і сміялись, і
ненавиділи і, що тільки не відчули за час цієї вистави. Цілий  вир емоцій.
Друзі ви неймовірні! Дякуємо Вам за Ваш виступ, що неабияк зворушив нас,
дякуємо Вам за Вашу силу, витримку, незламність, за те, що Ви є!

\enquote{З азовської сталі кується меч, що нищить будь-якого ворога}! Слава Україні!

\href{https://archive.org/details/video.28_03_2023.muzej_literatury_ukrainy.blagodijna_akcia_ja_teatr_i_vijna}{%
Відео заходу за посиланням% 
}
