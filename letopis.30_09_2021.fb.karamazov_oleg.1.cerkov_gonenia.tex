% vim: keymap=russian-jcukenwin
%%beginhead 
 
%%file 30_09_2021.fb.karamazov_oleg.1.cerkov_gonenia
%%parent 30_09_2021
 
%%url https://www.facebook.com/oleg.karamazov/posts/4263374747043793
 
%%author_id karamazov_oleg
%%date 
 
%%tags cerkov,hristianstvo,hristos,onufrii_mitropolit,pravoslavie,presledovanie,ukraina,upc,vera
%%title Время гонений – лучшее время жизни Церкви
 
%%endhead 
 
\subsection{Время гонений – лучшее время жизни Церкви}
\label{sec:30_09_2021.fb.karamazov_oleg.1.cerkov_gonenia}
 
\Purl{https://www.facebook.com/oleg.karamazov/posts/4263374747043793}
\ifcmt
 author_begin
   author_id karamazov_oleg
 author_end
\fi

Монах Онуфрий:

«Когда в Церкви покой, то это для нее, пожалуй, самое страшное время. То, что
мы переживаем сейчас, – это, на самом деле, процесс роста и возмужания.

Почву под ногами христианина можно сравнить с болотом. Нам необходимо постоянно
трудиться, двигаться вперед, ведь если стоять на месте, то можно гарантированно
утонуть. Сам того не замечая, христианин будет опускаться все ниже и ниже, пока
не захлебнется в собственном окамененном нечувствии. Таково свойство времени
покоя. Это касается как мирян, так и духовенства. Внутри все незаметно
прокисает. Скрипит, как старая половица, утреннее и вечернее бездушное
молитвенное правило. Скрипит не смазанная благодатью душа, а потом умирает, как
чахоточная старуха.

\ifcmt
  ig https://scontent-mia3-2.xx.fbcdn.net/v/t39.30808-6/243535122_4263373187043949_418247265157552454_n.jpg?_nc_cat=105&ccb=1-5&_nc_sid=8bfeb9&_nc_ohc=mvpOV3CxqI0AX8ghpbl&_nc_ht=scontent-mia3-2.xx&oh=e3dc10b4742905778e4626e396478247&oe=61615B1A
  @width 0.4
  %@wrap \parpic[r]
  @wrap \InsertBoxR{0}
\fi

Другое дело, когда градус создаваемой ,в обществе антицерковной истерии
становится таковым, что в Церкви могут остаться только те, кто ставит свою веру
в Бога выше любых других жизненных приоритетов. Церковные приспособленцы и
карьеристы, те, кто ищут в Церкви выгоды, денег, известности, почитания, кто
любит «председательствовать на собраниях» и «чтобы люди о них говорили хорошо»,
постепенно уходят туда, где они этих целей могут достигнуть. Т.е. в такое время
Церковь очищается от всего ненастоящего, что в ней есть. И это здорово. 

Христос четко обозначил парадигму взаимоотношений мира и Церкви: «Меня гнали и
вас будут», «Меня мир ненавидел и вас будет ненавидеть» и даже «убивая вас,
будут думать, что делают дело, угодное Богу». Как только мир начинает нас
любить, мы сразу же перестаем быть рабами Христа и становимся рабами мира, а
это и есть самая страшная подмена в жизни.

Смотря на свой домашний иконостас, я не вижу на нем ни одного человека, который
прожил бы благочестивую «застойную» жизнь и вошел бы в мир Света. С каждой
иконы на меня смотрит или мученическая кровь, или бескровное мученичество.
Алгоритм «работы» диавола с Церковью нам хорошо известен.  Он до наивности
прост и потому черзвычайно эффективен. В первые времена христианства сатана
кричал в тогдашних «СМИ», что «христиане – развратники и каннибалы»,  что «они
на тайных собраниях режут младенцев, пьют их кровь и потом блудят».

Конечно же, как за такое не отдать христиан на съедение львам в амфитеатр. Сто
лет назад сатана тоже учил, что христиане – это  «реакционное духовенство»,
«пособники империализма» и «враги революции». Ленина, по воспоминаниям
современников, буквально трясло от одного упоминания о Церкви и Христе. 

Через пару десятков лет духовенство расстреливали «за связь с контрреволюцией»,
за «работу на иностранную разведку», за «организацию сговора с целью свержения
советской власти» и за  «пособничество западным спецслужбам». На дворе двадцать
первый век, а аргументы у сатаны все те же. «Агенты Кремля», «работники ФСБ»,
«пособники агрессора». Пластинка та же и почерк узнаваемый.

Но настоящие христиане жили, живут и будут жить только одним – Христом и
Евангелием. Их, конечно же, можно оклеветать, убить, посадить в тюрьму, как это
уже было не раз в истории, но они будут продолжать любить Бога, Церковь и свою
Родину. Потому что, как и Бог, она у нас одна.

Время гонений – лучшее время жизни Церкви. Тогда для спасения достаточно одного
– просто быть верным Богу и нашей Матери-Церкви. На самом деле, все в этом мире
очень, очень просто. Потому что сам Бог прост. В мире есть Добро и Зло. Есть
Вселенская Любовь и есть такая же Ненависть. Спаситель дал нам заповедь любви.
Все, что несет в себе любовь, свет, добро, милосердие, сострадание, так или
иначе, берет свое начало в Боге. Потому что он Добр. Все, что несет в себе
ненависть, разрушение, войну, призыв к убийствам, жажду смерти – берет свое
начало в аду, независимо от того, в какие одежды мотивов и намерений сатана
свое зло облекает. Наше истинное гражданство – это Царство Божие, и нужно
приложить много усилий, чтобы получить там вечную прописку.

То, что мы переживаем сейчас, это, на самом деле, процесс роста и возмужания.
Зона комфорта никогда не дает динамики роста. Это касается любой системы, какая
бы она ни была. Рост возможен только в труде и сложной обстановке, в самых
неблагоприятных условиях. Тогда открываются внутренние резервы, приобретается
новый опыт, от «терпения происходит благочестие, из благочестия братолюбие, из
братолюбия любовь… А в ком нет сего, тот слеп… Посему, братия, более и более
старайтесь делать твердым ваше звание и избрание, так поступая, никогда не
преткнетесь» (2 Петр. 1: 6-10).»

\ii{30_09_2021.fb.karamazov_oleg.1.cerkov_gonenia.cmt}
