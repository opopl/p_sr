%%beginhead 
 
%%file 31_07_2021.fb.fb_group.mariupol.biblioteka.korolenka.1.yekskurs_ya_vistavko
%%parent 31_07_2021
 
%%url https://www.facebook.com/groups/1476321979131170/posts/4152954054801269
 
%%author_id fb_group.mariupol.biblioteka.korolenka,shulga_inna.mariupol
%%date 31_07_2021
 
%%tags mariupol,kultura,vystavka,isskustvo,pamjat,hudozhnik
%%title Екскурсія виставкою  "Фестиваль пам’яті"
 
%%endhead 

\subsection{Екскурсія виставкою  \enquote{Фестиваль пам'яті}}
\label{sec:31_07_2021.fb.fb_group.mariupol.biblioteka.korolenka.1.yekskurs_ya_vistavko}
 
\Purl{https://www.facebook.com/groups/1476321979131170/posts/4152954054801269}
\ifcmt
 author_begin
   author_id fb_group.mariupol.biblioteka.korolenka,shulga_inna.mariupol
 author_end
\fi

31 липня у Міському центрі сучасного мистецтва і культури ім. А.І. Куінджи
відбулась екскурсія виставкою  «Фестиваль пам'яті», присвяченій маріупольськім
художникам майстрам ковальського мистецтва Манохіну А.Б. та Сумарокову Л.Н. З
ініціативи співробітників Центральної міської публічної бібліотеки ім. В.Г.
Короленка виставку відвідали люди з функціональними порушеннями зору.
Відвідування виставки відбувалось з тифлокоментарем, який допоміг відчути красу
і неповторність виробів ковальського мистецтва. Застиглою в металі музикою
можна назвати чудові витвори майстрів. Це і підсвічники, і меблі, і зброя,
скульптурні композиції тощо. Кожен відвідувач зміг тактильно відчути кожну
роботу. Всі відвідувачі були в захопленні від виставки. За їх словами цих
вражень їм вистачить надовго.
