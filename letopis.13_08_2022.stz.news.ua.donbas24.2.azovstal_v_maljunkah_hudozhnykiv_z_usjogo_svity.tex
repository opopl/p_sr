% vim: keymap=russian-jcukenwin
%%beginhead 
 
%%file 13_08_2022.stz.news.ua.donbas24.2.azovstal_v_maljunkah_hudozhnykiv_z_usjogo_svity
%%parent 13_08_2022
 
%%url https://donbas24.news/news/azovstal-v-malyunkax-xudoznikiv-i-ilyustratoriv-z-usyogo-svitu-foto
 
%%author_id news.ua.donbas24,shvecova_alevtina.mariupol.zhurnalist
%%date 
 
%%tags 
%%title "Азовсталь" в малюнках художників і ілюстраторів з усього світу
 
%%endhead 
 
\subsection{\enquote{Азовсталь} в малюнках художників і\titlebreak ілюстраторів з усього світу}
\label{sec:13_08_2022.stz.news.ua.donbas24.2.azovstal_v_maljunkah_hudozhnykiv_z_usjogo_svity}
 
\Purl{https://donbas24.news/news/azovstal-v-malyunkax-xudoznikiv-i-ilyustratoriv-z-usyogo-svitu-foto}
\ifcmt
 author_begin
   author_id news.ua.donbas24,shvecova_alevtina.mariupol.zhurnalist
 author_end
\fi

Протистояння українських захисників і захисниць на території металургійного
гіганта — комбінату \enquote{Азовсталь} — вже увійшло в світову історію. Потужним
відображенням долі оборонців \enquote{Азовсталі} стали арти, малюнки, заклики врятувати
людей з обложеного рашистами комбінату, і ці зображення не залишали байдужими
користувачів соціальних мереж і людей з різних міст і країн, які виходили з
плакатами на мирні акції, присвячені порятунку захисників \enquote{Азовсталі}.

До 89-річчя комбінату \enquote{Азовсталь} Донбас24 підготував добірку ілюстрацій і
малюнків, які були створені художниками і художницями з усього світу. На них
зображені мужні та сміливі захисники України; цивільні мешканці та мешканки
Маріуполя, які ховалися в азовстальських бункерах від потужних російських
обстрілів, що ні на мить не вщухали; дружини та діти оборонців, які чекають
вдома на своїх рідних, а також металургійний гігант, який через
російсько-українську війну перетворився на фортецю Маріуполя.

\ii{insert.read_also.shvecova.donbas24.azovstal_fortecja_mrpl_narys_metalurg_kombinat}

Ще більше новин та найактуальніша інформація про Донецьку та Луганську області
в нашому телеграм-каналі Донбас24.

ФОТО: з відкритих джерел, арт обкладинки Dmytro Pozhirauskas.

%\ii{13_08_2022.stz.news.ua.donbas24.2.azovstal_v_maljunkah_hudozhnykiv_z_usjogo_svity.txt}
