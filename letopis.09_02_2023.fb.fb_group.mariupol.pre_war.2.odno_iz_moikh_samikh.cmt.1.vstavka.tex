% vim: keymap=russian-jcukenwin
%%beginhead 
 
%%file 09_02_2023.fb.fb_group.mariupol.pre_war.2.odno_iz_moikh_samikh.cmt.1.vstavka
%%parent 09_02_2023.fb.fb_group.mariupol.pre_war.2.odno_iz_moikh_samikh.cmt
 
%%url 
 
%%author_id 
%%date 
 
%%tags 
%%title 
 
%%endhead 

\par\noindent\rule{\textwidth}{0.4pt}

\ifcmt
  tab_begin cols=2,no_fig,center,separate
     pic https://i.paste.pics/50c8caba25c1ceb50df3b86756ddbfa8.png
		 pic https://i.paste.pics/493c3479c40add7ee1b7b54accc872de.png
		 pic https://i.paste.pics/161186bcf36d310acae699b7c8696f25.png
		 pic https://i.paste.pics/e62b37278a056dd8704e35c8a06a8e79.png
  tab_end
\fi

\begin{center}
\begin{fminipage}{0.8\textwidth}
\em

Історія пам'ятника святителю Ігнатію Маріупольському, який зруйнували росіяни

Скульптурі, від якої залишилося лише каміння, у дні війни виповнилося 25 років

Росіяни зруйнували в Маріуполі пам'ятник святителю Ігнатію на території собору
Архистратига Михаїла, передає donbas24.

Історію створення цієї монументальної роботи майстрів оприлюднив журналіст та
краєзнавець з Маріуполя Іван Станіславський.

Він опублікував унікальні світлини з архіву видатних художників Валентина
Константінова та Леля Кузьминкова, які до цього ніде не демонструвалися. Також
користувачам був представлений один з альтернативних авторських ескізів з
коментарями. Авторами проєкту були саме ці видатні художники, а виконавцем став
Анатолій Чумель.

\begin{center}
\begin{minipage}{\textwidth}
\ifcmt
  tab_begin cols=3,no_fig,center,layout=3.2,separate
     pic https://donbas24.news/storage/news/hgwjksqxll70lehn.jpg
		 pic https://donbas24.news/storage/news/j2gvygutdjqyft0h.jpg
		 pic https://donbas24.news/storage/news/aqslzbv8okbn1ows.jpg

		 pic https://donbas24.news/storage/news/jtjmoilzss80scjs.jpg
		 pic https://donbas24.news/storage/news/cgvajpjz4rkixbmr.jpg
	tab_end
\fi
\end{minipage}
\end{center}

«Маріуполь — це не тільки будинки. Матеріальні об'єкти культурної спадщини — це
дещо з того, що, на жаль, неможливо відновити чи відбудувати краще, ніж було.
Не буде більше ХІХ сторіччя з його архітектурою, немає Горської, щоб скласти
мозаїку, немає Константінова з Кузьминковим, щоб створити пам’ятник Митрополиту
Ігнатію. Тепер залишається радіти, що дещо з тієї спадщини було зафіксовано на
фото чи врятовано зі смітників, та сумувати, що це не робилось систематично й
масово», — зазначив краєзнавець.

Як повідомляла найстаріша газета Маріуполя «Приазовський робочий», пам'ятник
митрополиту, який вивів тисячі греків та заснував місто у 1778 році, був
відкритий у Лівобережному районі 25 березня 1997 року до 220-річчя міста.

Завдяки світлині 2000 року можна розгледіти деталі скульптури. Про зруйнований
пам'ятник стало відомо від спільноти «Азовські греки» ще 16 квітня.

На жаль, тепер від роботи видатних виконавців залишилися одні руїни.

Нагадаємо, що від обстрілів також постраждав сам собор Архистратига Михаїла. У
цілому ж Донецька область займає друге місце по Україні, де через обстріли
постраждали об'єкти культурної спадщини.

Також у Маріуполі Донецької області горів краєзнавчий музей — оцінити масштаби
пошкоджень та втрати фонду наразі неможливо.

Іванова Яна

\end{fminipage}
\end{center}

\par\noindent\rule{\textwidth}{0.4pt}

