% vim: keymap=russian-jcukenwin
%%beginhead 
 
%%file 11_02_2022.fb.fb_group.story_kiev_ua.1.soedinit_nesoedinimoe
%%parent 11_02_2022
 
%%url https://www.facebook.com/groups/story.kiev.ua/posts/1859102250953234
 
%%author_id fb_group.story_kiev_ua,solovjeva_tatjana.kiev
%%date 
 
%%tags kiev
%%title СОЕДИНИТЬ НЕСОЕДИНИМОЕ
 
%%endhead 
 
\subsection{СОЕДИНИТЬ НЕСОЕДИНИМОЕ}
\label{sec:11_02_2022.fb.fb_group.story_kiev_ua.1.soedinit_nesoedinimoe}
 
\Purl{https://www.facebook.com/groups/story.kiev.ua/posts/1859102250953234}
\ifcmt
 author_begin
   author_id fb_group.story_kiev_ua,solovjeva_tatjana.kiev
 author_end
\fi

СОЕДИНИТЬ НЕСОЕДИНИМОЕ

Когда поднимаешься с  Крещатика по левой стороне улицы Богдана Хмельницкого
(бывшей улицы Ленина) мимо затянутого рекламой бывшего Центрального гастронома,
что бы заглянуть в книжный магазин, не сможешь не остановиться возле Киевской
перепички. 

\ii{11_02_2022.fb.fb_group.story_kiev_ua.1.soedinit_nesoedinimoe.pic.1}

Собственно, эта горячая сосиска в тесте, со времен уже не существующей страны -
первый киевский фаст-фуд. Хот-дог! Слово такое, правда, не употребляли тогда. 

Удивительно, как это окошко в стене стало таким популярным. Тут всегда очередь,
а эта фотография - просто удача, да и почти утро еще.

\ii{11_02_2022.fb.fb_group.story_kiev_ua.1.soedinit_nesoedinimoe.pic.2}

А рядом - книжный, нет не так - КНИЖНЫЙ! Они меняют названия, а зачем название
такому магазину? Книжный - и этим все сказано. 

Запах книг. Знакомая книга и книга, которую еще не прочел. Предвкушение.
Детские. Подарочные...

Говорите, электронные книги...

Ну тогда это как театр по телевизору!

\ii{11_02_2022.fb.fb_group.story_kiev_ua.1.soedinit_nesoedinimoe.pic.3}
