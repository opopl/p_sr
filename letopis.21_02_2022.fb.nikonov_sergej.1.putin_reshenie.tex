% vim: keymap=russian-jcukenwin
%%beginhead 
 
%%file 21_02_2022.fb.nikonov_sergej.1.putin_reshenie
%%parent 21_02_2022
 
%%url https://www.facebook.com/alexelsevier/posts/1695102630835045
 
%%author_id nikonov_sergej
%%date 
 
%%tags dnr,donbass,lnr,priznanie,putin_vladimir,ukraina
%%title Путин явно нервничал, принимая решение
 
%%endhead 
 
\subsection{Путин явно нервничал, принимая решение}
\label{sec:21_02_2022.fb.nikonov_sergej.1.putin_reshenie}
 
\Purl{https://www.facebook.com/alexelsevier/posts/1695102630835045}
\ifcmt
 author_begin
   author_id nikonov_sergej
 author_end
\fi

Путин явно нервничал, принимая решение. Мы допрыгались со своей внутренней и
внешней политикой. Довоевались, доубивались. НАТО и США доупирались. А у
донбассцев может какая-то надежда на мир появится. Слушал речь Путина. Сейчас
во всех проблемах винит государственную  и национальную политику СССР. Считает,
что дали много суверенитета. Назвал СССР- исторической Россией. СССР -
исторический СССР. Он не сводился к России. Что касается сегодняшнего дня, то
он обосновывает российскую политику нашим тунеядством на бесплатных уступках.
Не буду ни спорить, ни поддерживать здесь. Но в оценке нашей верхушки он прав:
коррупция и национализм как источник проблем, разъедающих государство и
приведщих к Майдану. И он реально показал внутреннюю картину нашей страны в
2014 г. Путин хочет покарать националистов. Логично. Именно они формируют
антироссийскую повестку дня. И только вред для ситуации в целом. Сказанное
Путиным для нас не ново. Тот, кто  интересуется, давно о таких делах знал. Увы,
сейчас Путин враг и его мало-кто послушает. И, конечно, он сводил все к
западному цивилизационному выбору. Ему так выгодно. Я отвечу: тут дело не в
западных ценностях, а в западных интересах. Демократия - великая вещь, когда
реализуется с умом. Тоже самое и о рыночной экономике. Но вот интересы группы
западных государств заключались в строительстве нашего государства по их
рецептам. В строительстве Антироссии, не Украины. Об этом Путин и говорит. Как
по мне, не европейского государства с ценностями европейского общества, а
общества примитивного национализма, борющегося совсем русским.  Судя по речи,
идея Зеленского или авторов текста о создании или получении ядерного оружия
вызвала серьезную негативную реакцию в России. Фотогеничный и талантливый
артист Володя наговорил в Мюнхене. Думать надо было. Всей нашей верхушке. И да,
тут цель внешней политики - не освобождение \enquote{оккупированных} территорий, а
содействие западному миру в военном противостоянии с РФ. В этом контексте Путин
говорит о доброте России в отношениях с НАТО в 1990-2000г. Дело ещё в слабости.
Никакой доброты в ельцинском руководстве не было. Отсюда и расширение НАТО без
вступления России. В этом связи Путин правильно указывает на великое совместное
прошлое. Чтобы наше общество опомнилось. Но опомниться надо всем. И нам с нашим
раболепием перед евроатлантическим миром и России, которая способствовала уходу
Крыма. По-человечески сделано правильно. Вот государственный суверенитет
нарушать тоже не стоит. Но нас  прогнозы умных людей сбылись. Хочешь в НАТО-
теряй земли. Или в прямом смысле или теряй симпатии жителей центра , Юга и
Востока. Россия в ответ принимает меры по безопасности. То, что мы, Украина,
продолжаем преступление против Донбасса путем войны тоже верно. Как и неприятие
им нашего государства. Да Россия не решалась на отторжение ЛДНР от нас или
легитимизацию осособления территорий. Признание республик закономерно... 
