% vim: keymap=russian-jcukenwin
%%beginhead 
 
%%file 26_01_2022.fb.fb_group.story_kiev_ua.1.kiev_mosty_1941.cmt
%%parent 26_01_2022.fb.fb_group.story_kiev_ua.1.kiev_mosty_1941
 
%%url 
 
%%author_id 
%%date 
 
%%tags 
%%title 
 
%%endhead 
\zzSecCmt

\begin{itemize} % {
\iusr{Павел Руденко}
Сейчас опять враг войной на нас пойдет, придется взрывать

\begin{itemize} % {
\iusr{Артур Игнатов}
\textbf{Павел Руденко} я на левом живу. Не нужно.

\iusr{Павел Руденко}
\textbf{Артур Игнатов} паром будет)

\iusr{Артур Игнатов}
\textbf{Павел Руденко} думаю, наши мосты скоро и без войны окажутся в воде.

\iusr{Надежда Владимир Федько}
\textbf{Артур Игнатов} Без паніки)
\end{itemize} % }

\iusr{Igor Rostov}
Мой текст из ЖЖ. Удивительно, через сколько лет его прочитали.

\iusr{Jon Yossarian}

И надолго это остановило Вермахт?

Бей своих, что б чужие боялись. Нет в истории человечества более ублюдочного
образования, чем совок. @igg{fbicon.face.symbols.mouth} 

\begin{itemize} % {
\iusr{Юрий Блохин}
\textbf{Jon Yossarian} вы вообще верите тому что пишите, или это не главное ? Ведь люди вашу глупость всерьёз принять могут.

\iusr{Jon Yossarian}
\textbf{Юрий Блохин} вы на вопрос ответьте, прежде чем слюной плеваться.
Насколько задержало уничтожение мостов, своих войск и гражданских на этих мостах продвижение Вермахта на левый берег?.
Заодно откройте тайну, как помогло борьбе с оккупантами взрыв Крещатика и прилегающих районов? Осветите отдельно судьбу жильцов этих домов, а так же заложников, расстрелянных в отместку за взрывы.
Если не можете, то заткните пасть, совок.

\iusr{Юрий Блохин}
\textbf{Jon Yossarian} 

У меня нет желания отвечать на глупые вопросы, ответы на которые есть даже у
людей изучавшие военную подготовку в гражданском ВУЗе. Я так понял, что вы из
той породы людей, которые орут на каждом шаге, что не хрен было сопротивляться
, Кац предлагает сдаться .... а там пили бы \enquote{Баварское} .... манкурт вы наш  @igg{fbicon.wink} 


\iusr{Jon Yossarian}
\textbf{Юрий Блохин} ну так и заткнитесь, болтун находка для шпиона

\iusr{Надежда Владимир Федько}
\textbf{Юрий Блохин} 

Повідайте нам, будь ласка, яка мотивація була у Київського угрупування ЧА
здатися в полон в кількості 665 000 (!) Чи про це на кафедрі військової
підготовки Вам не розповідали)

\iusr{Юрий Блохин}
Я так понял что я только что раскрыл вашу тайну  @igg{fbicon.face.grinning.sweat} 

\iusr{Дмитрий Артюк}
\textbf{Юрий Блохин} Когда у человека нету аргументов, он начинает затыкать рот и плеваться.  @igg{fbicon.smile} 
\end{itemize} % }

\iusr{Dmitrii Pisanenko}
Враг сейчас не ходит, а летает! Это еще хуже, захватит и все.

\iusr{Владимир Картавенко}

За три дня до того как советские войска покинули Киев, на Рыбальском создана в
ч 20884 и в 1967 установлен памятник военным морякам - монитор Железняков.

В 1999 умер отец Путина и нам предложили заменить монитор на подлодку сс-310.
Пробную на Коса Тузла выиграли, суды тоже. Под угрозой ядерного уничтожения
народа Украины подписали соглашения. ..


\iusr{Андрій Чмир}

Какие автоматчики? Фильм посмотрите \enquote{Город, которьій предали.} Три дня ! до 19
числа в Киеве НЕ бьіло немцев. И как взорвали мост с отходящими частями. Там
воспоминания детей-очевидцев.

\begin{itemize} % {
\iusr{Надежда Владимир Федько}
\textbf{Андрій Чмир} Чудовий фільм!

\iusr{Галина Зуева}
\textbf{Андрій Чмир} На мосту были и раненые и гражданские. Взорвали вместе с нашими людьми. Зато как \enquote{красиво} пишет в мемуарах

\iusr{Надежда Владимир Федько}
\textbf{Галина Зуева} Мемуари писалися \enquote{по канону}, написаному в Ідеологічному відділі ЦК КПСС.
\end{itemize} % }

\iusr{Владимир Мареев}
Страшное дело эта война. Мама 1934 г.р. всегда плакала когда войну вспоминала.
Пережила окупацию в Чернигове.

\iusr{Oleksandr Oleksandrovich Maslov}
Шел третий месяц ВОВ...

\iusr{Ковальская Татьяна}
Преступление 20 века!

\iusr{Людмила Пащинська}
Кто считает - что надо было оставить все немцам - на блюдечке с голубой каемочкой ?

\begin{itemize} % {
\iusr{Надежда Владимир Федько}
\textbf{Людмила Пащинська} Знищення інфраструктури - військовий злочин!

\iusr{Tovarish Inka}
\textbf{Надежда Владимир Федько} Це ви зараз про сьогодення. А що стосовно оборони та військових дій в минулому столітті?

\iusr{Надежда Владимир Федько}
\textbf{Tovarish Inka} Міжнародне законодавство про правила ведення війни і військові злочини були прийняті ще у 1907 році.

\iusr{Tovarish Inka}
\textbf{Надежда Владимир Федько}

\ifcmt
  ig https://scontent-frx5-1.xx.fbcdn.net/v/t39.30808-6/s851x315/272822672_1073689263214692_1574443218841073449_n.jpg?_nc_cat=100&ccb=1-5&_nc_sid=dbeb18&_nc_ohc=p9tZX8KVI58AX9UwZEw&_nc_ht=scontent-frx5-1.xx&oh=00_AT88oOZMo1CK9DrGk-C-TqBOBuvekKmXB_YQFqtHsTfxBQ&oe=61F7EA57
  @width 0.3
\fi

\end{itemize} % }

\iusr{Людмила Пащинська}
Отступавшие польские войска в 1920 году взорвали пору Николаевский цепной мост - вот где преступление.

\iusr{Людмила Пащинська}
Сейчас вообще не образование - они еще совок ругают. А американское - то вообще - прав был Задорнов.

\iusr{Надежда Владимир Федько}

У 2012 році, працюючи в архіві Київської області, я познайомився з Борисом
Фінкільштейном. Військовий інженер А. Фінкільштейн його дятько. Борис
розповідав, що після війни дядько згадував про відхід з Києва і підрив мостів.

\iusr{Вадим Ляшенко}

Подобные воспоминания очень ценны!

\iusr{Надежда Владимир Федько}

Історично незаперечним є той факт, що відступаючи з Києва німецька армія нічого
не знищувала, не підривала, не палила!

\begin{itemize} % {
\iusr{Дмитрий Артюк}
\textbf{Надежда Владимир Федько} Не смешите, отступая враг уничтожает инфраструктуру противника. Взрывы были в Киеве и в 1943 году.

\begin{itemize} % {
\iusr{Надежда Владимир Федько}
\textbf{Дмитрий Артюк} Наведіть конкретні приклади, будь ласка!

\iusr{Дмитрий Артюк}
\textbf{Надежда Владимир Федько}

Сгорели Киевский университет, Дом Обороны, городская публичная библиотека,
электростанции, взорваны два цеха завода \enquote{Большевик}, хлебозаводы,
водопроводное хозяйство, путепроводы и ряд больших жилых зданий.

\iusr{Дмитрий Артюк}
В 1945 году Гитлер даже издал приказ о ,выженной земле,, в Германии.

\iusr{Надежда Владимир Федько}
\textbf{Дмитрий Артюк} Практично все, що Ви назвали, було підірвано НКВС, або знищено при бомбардуваннях Києва радянською авіацією у 1943-му.

\iusr{Дмитрий Артюк}
\textbf{Надежда Владимир} Федько Вам виднее, обратитесь в ДАКО, найдете много интересного.
\end{itemize} % }

\iusr{Лисенко Олег}
\textbf{Nadegda Volodymyr Fedko} В ДАКО є цілий фонд що було знищено в Києві і області.

\iusr{Надежда Владимир Федько}
\textbf{Лисенко Олег} Я працював з цим фондом! Там багато документів про те, ЩО було вивезено під час евакуації Києва і ЩО було зруйновано!
Радянськими військами!

\iusr{Дмитрий Артюк}
\textbf{Надежда Владимир Федько} а 22 июня 1941 г.не подскажите кто бомбил Киев?

\begin{itemize} % {
\iusr{Надежда Владимир Федько}
\textbf{Дмитрий Артюк} Київ не бомбили! На цю тему є військовий звіт Штабу Пінської військової флотилії...

\iusr{Дмитрий Артюк}
\textbf{Надежда Владимир Федько} 

в 1941 году мои предки жили в Киеве, я их свидетельствам больше доверяю. Тогда же
одна из бомб угодила в дом моей бабушки в Пуще Водице, во время воздушной
тревоги. В поводу бомбежке Киева, изучайте документы.


\iusr{Надежда Владимир Федько}
\textbf{Дмитрий Артюк} Якраз цим я давно вже займаюся!

Из оперативной сводки штаба 36-й ИАД от 23:00 22 июня 1941 года:

«В 4.00 22.6.1941 г. Части 36-й ИАД ПВО заняли боевое положение по тревоге с дислокацией на постоянных аэродромах.
В 07.15 произведен налет 19 самолетов противника Хе-111, направление Бровары, аэропорт Киев. С высоты Н=2000 сброшено 90 бомб осколочных и фугасных калибра 40-100 кг. 4-я эскадрилья 43 ИАП преследовала противника в районе истребления, но догнать не смогла…
Потери в результате бомбардировки аэропорта Киев в 7.15: убито 32 человека, ранено 34, контужено 3 из состава рабочих строительства и колхозников села Жуляны.
Наши потери: один самолет И-16 43 ИАП разбился на взлете, летчик погиб; один И-16 43 ИАП, преследуя противника, бомбившего Киев в 7.15, не рассчитал горючего и сел вынужденно, самолет подлежит ремонту.
Сбитых самолетов противника нет.»
***
Из «Хроники боевых действий Пинской военной флотилии…»:
«В 07 ч. 00 м. авиация противника бомбардировала аэродром 46-й отдельной авиаэскадрильи в Киеве. Сгорел ангар и 3 авиамотора».
\end{itemize} % }

\end{itemize} % }

\iusr{Людмила Пащинська}
Они вообще были прекрасными людьми........

\iusr{Людмила Пащинська}

А на счет военного злочина - так это как НАТО бомбило Югославию....Ирак.....где
тут международное право ?

\begin{itemize} % {
\iusr{Людмила Пащинська}

Как-то сейчас вспомнилось пропаганду Гебельса в Киеве с фотографиями ......вот
типа народ смотрите - ваши бомбят с самолетов город - есть
жертвы............вот такие плохие советы-немцев отгоняют ...........


\iusr{Людмила Пащинська}

14 мая 1943 года киевская газета «Нове українське слово» поместила несколько
обращений к киевлянам, сообщив, что (язык и правописание оригинала) «в ніч на
11 травня большевицькі пірати вчинили розбійницький терористичний наліт на наше
місто. В наслідок цього злочинного акту зруйновано кілька житлових будинків,
убито кілька мирних жителів, наших співгромадян, серед них — діти, жінки,
старики, інваліди... Глибокий сум і разом з тим невимовний гнів та обурення
сповнюють кожного з нас — свідків цієї злочинної справи. Своє звіряче обличчя
большевики виявляли багато разів: і тоді, коли засилали наших братів на
біломорську або колимську каторгу, і коли піддавали їх нелюдським тортурам у
катівнях сталінсько-єжовського НКВД, і коли морили штучно створеним голодом
мільйони наших селян. Так само виявили вони шалену лють, збільшену ще почуттям
безсилля, коли, залишаючи під тиском героїчної німецької армії Київ, власними
руками висадили в повітря кращі його будови, пам’ятки нашої культури...»

\iusr{Людмила Пащинська}

тоді як німецькі війська, шкодуючи мирне населення і цінуючи здобутки нашої
багатовікової праці, не завдали нашому місту ні якої шкоди. Не маючи змоги
перемогти противника у відкритому і чес ному бою, жидо-большевицькі кати широко
користуються такими методами боротьби, як удар у спину, бандитські наскоки. А
що робили ці бандити в місцевостях, які тимчасово переходили знову під їх
владу!

\iusr{Людмила Пащинська}
Нове украинське слово №113 от 15 мая. Кстати под редакцией Олени Телиги.
\end{itemize} % }

\iusr{Галина Зуева}

А в это время моя бабушка стояла с чемоданом на правом берегу и все это видела.
Только по ее словам на мосту были наши солдаты, который взлетели вместе с
мостом Она так и не смогла эвакуироваться и осталась в оккупации. А потом
взлетел на воздух Крещатик и наш до на Лютеранской,17 сгорел.


\iusr{Алена Дмитрова}

Это что? Хвалебная песня НКВДешникам, которые уродовали города и судьбы?
Печально, что вы живёте ещё тем временем, которое другие проклинают

\begin{itemize} % {
\iusr{Надежда Владимир Федько}
\textbf{Алена Дмитрова} Це історія, яку ми повинні знати і не забувати! Історія в особах! Більшовицький переворот 1917 року розколов суспільство на дві частини: на тих, хто не сприйняв більшовицької ідеології і піднявся на боротьбу проти неї, і на тих, хто вірно служив більшовицькій ідеї.
\end{itemize} % }

\iusr{Нина Бондаренко}
Дякую за розповідь і фотографії.

\iusr{Надежда Владимир Федько}

Залишимо на совісті мемуаристів свідчення про «героїчний захист мостів» та не
менш «героїчний підрив мостів» на очах декількох десятків (максимум!) німецьких
мотоциклістів...

Вдумайтесь самі: тисячі бійців ЧА тікають через мости так завзято, що нікому
протистояти максимум двом-трьом десяткам мотоциклістів, що сміливо прорвалися
до мостів, у яких на озброєнні тільки кулемети MG-34 і автомати MP-40, з вельми
обмеженим запасом набоїв.

А де бронетехніка, яка за логікою, повинна була б прикривати відхід військ до
останніх хвилин перед підривом мостів?

А де ж бронепотяги НКВС, які за логікою, повинні були б прикривати відхід
військ до останніх хвилин перед підривом мостів?

Про реальну ситуацію на 18 вересня 1941 року свідчить німецька карта… Ця карта
не передбачалася для пропагандистського піару, а була звичайним штабним
документом. З неї ми бачимо, що на 18 вересня Київське угрупування Червоної
Армії давно оточено підрозділами вермахту, а лінія Східного фронту, від
Балтійського до Чорного морів, проходить вже далеко за Києвом!

***

Відійшовши від Києва, угрупування радянських військ в кількості 655 000 (!)
осіб здалося в полон!

P.S.

Карта з фондів ДАКО.

\ifcmt
  ig https://scontent-frx5-1.xx.fbcdn.net/v/t39.30808-6/272836142_4969278743131544_3491473089677909434_n.jpg?_nc_cat=100&ccb=1-5&_nc_sid=dbeb18&_nc_ohc=7qZVEesLhD0AX_w7FtR&_nc_ht=scontent-frx5-1.xx&oh=00_AT-0tO_ETfMvtYQS6iQNf-y8jSYBv5AvpeBjstc1PmYx9Q&oe=61F706E2
  @width 0.2
\fi

\iusr{Надежда Владимир Федько}

Якщо ви дружите з математикою, то можете підрахувати, на скільки хвилин
«ураганного пулеметного огня» (як свідчить у мемуарах полковник Мажорін)
вистачить набоїв у мотоциклістів.

Кулемет MG-34 має темп стрільби 800-900 пострілів на хвилину. // Тип
боєпостачання: стрічка на 50 або 200 набоїв; або дводисковий магазин на 75
набоїв.

Пістолет-кулемет MP-40 має темп стрільби 500-550 пострілів на хвилину. // Тип
боєпостачання: магазин на 32 набої.

Максимум через 10 хвилин «ураганного вогню» боєзапаси мотоциклістів повинні
були скінчитися і у них залишалося два варіанти: а) відступити; б) здатися в
полон!

\ifcmt
  ig https://scontent-frx5-1.xx.fbcdn.net/v/t39.30808-6/272687417_4969286516464100_3277682501583614544_n.jpg?_nc_cat=111&ccb=1-5&_nc_sid=dbeb18&_nc_ohc=zV8opeDm3v4AX_1ydYA&_nc_ht=scontent-frx5-1.xx&oh=00_AT-m43DKg-fnZHLGGGnLfhwVa-bnJExxqBfVwMix6ixxqQ&oe=61F8E32F
  @width 0.3
\fi

\begin{itemize} % {
\iusr{Jon Yossarian}

скоріш за все, їх там не було зовсім - німці ввійшли у кинуте місто. А НКВДешна
сволота, як це в них і досі заведено, брехала, немов пес смердючий.

\begin{itemize} % {
\iusr{Valeriy Levitskiy}
\textbf{Jon Yossarian} 

Подивився Бабин Яр \enquote{Контекст} Сергія Лозниці. Крім іншого, там є такі кадри:
Німці займають село, виганяють селян із хат і підпалюють їх з вогнемету

A їх часто останнім часом стали уявляти, як таких собi добрих дядечків

\iusr{Jon Yossarian}
\textbf{Valeriy Levitskiy} 

і до чого кадри коли німці відступаючи, спалюють хати у селі ?

Де я казав, що німці добрі дядечки???

Що ви собі понапридумали, га???

Я кажу, що скоріш за все, німців фізично не було біля мостів, коли їх
підірвали. Що, скоріш за все, смердюча НКВДешна собака бреше, коли розказує про
мотоциклетників, що відкривали \enquote{ураганний} вогонь.

Ви ще вірити мемуарам НКВДешної сволоти? Вам мало брехні?

Ви, взагалі, можете розуміти, що ви читаєте???

Чи ви за зеленського голосували ??
@igg{fbicon.face.grinning.squinting}{repeat=3} 

\iusr{Jon Yossarian}
\textbf{Valeriy Levitskiy} а ! То ви - бот !  @igg{fbicon.beaming.face.smiling.eyes} 

\iusr{Valeriy Levitskiy}
\textbf{Jon Yossarian} 

Я не бот, если вам интересно: Я еврей с Подола, за Зеленского не голосовал, так
как уехал с Родины 26 лет назад, да и Йоссариан ник вам не подходит, вы какой то
агрессивный

\iusr{Надежда Владимир Федько}
\textbf{Valeriy Levitskiy} Вермахт не палив селянські хати!

Хати спалювалися, як покарання за зв'язок з партизанами. Цим займалися
поліційні батальйони.

\iusr{Надежда Владимир Федько}
\textbf{Valeriy Levitskiy} Тепер все зрозуміло!

\iusr{Jon Yossarian}
\textbf{Valeriy Levitskiy} тогда есть шанс, что вы поняли, что именно я написал.
И логику немцев, когда они уничтожили весь Труханов остров можно понять - враги хладнокровно уничтожали поселение, которое могло помешать обороняться от красных.
Понять, но не простить - в какой вселенной это не преступление?
При этом идиотское исполнение логичного уничтожения мостов принять невозможно, как и бесчеловечный подрыв Крещатика вместе со всеми мирными жителями.
Непрофессионализм, пренебрежение людскими жизнями - отличительные черты совка.

\iusr{Надежда Владимир Федько}
\textbf{Jon Yossarian} Усі жителі з Труханового острова були переселені в центральний Київ.

\iusr{Valeriy Levitskiy}
\textbf{Nadegda Volodymyr Fedko} 

мне уже давно все стало понятно, ещё в 1986 году в солдатах, про все и про всех, я
просто читаю эту чушь и смеюсь, все повторяется заново один к одному, история
ничему не учит


\iusr{Jon Yossarian}
\textbf{Надежда Владимир Федько} я знаю, але все ж таки хати попалили.
Взагалі, попри те, що німці таки були нелюдями, зрівнювати їх з червоними неможливо.

\iusr{Valeriy Levitskiy}
\textbf{Nadegda Volodymyr Fedko} 

Вы защищаете Вермахт почему то больше, чем сами наследники Вермахта в
Германии. Знаю о чем говорю, так как общаюсь с ними каждый день по месту
жительства. Это просто удивительно как работает мышление у людей одни
крайности. То коммунистов возносили до небес, пионеры все поголовно, теперь все
антикоммунисты Вермахт зауважали... Но так не бывает

\iusr{Jon Yossarian}
\textbf{Valeriy Levitskiy} 

о! Я думал, я рассказыаю какому то подьемнику на какой то горе, а оказывается,
рассказыаю какому-то деду  @igg{fbicon.beaming.face.smiling.eyes}{repeat=3} 

Шо ж вы живёте в таком ужасном городе? Неужели ничего получше не нашлось??  @igg{fbicon.face.grinning.squinting}{repeat=2} 

Опять внимательно не читаете, дедушка, я написал, что целиком логичное
распоряжение о взрыве мостов, было исполнено с преступным игнорированием
возможных потерь среди своих же войск и мирных жителей.

Ферштейн, Христофор Бонифатьевич ?  @igg{fbicon.beaming.face.smiling.eyes} 

Девушку можно нежно уговорить, а можно изнасиловать, результат один и тот же -
последствия разные.

Это с уважением к вашим сединам, старина  @igg{fbicon.face.grinning.squinting}{repeat=3} 

\iusr{Jon Yossarian}
\textbf{Valeriy Levitskiy} 

никто никого не защищает. История есть. И знать надо ее, а не байки красных
выродков. Это факты, а у вас - эмоции. Сплошные при том  @igg{fbicon.beaming.face.smiling.eyes} 

\iusr{Надежда Владимир Федько}
\textbf{Valeriy Levitskiy} 

Я історик, а не замполіт! Це раз. По-друге, як християнин, я не можу зводити
наклепи на людей, незалежно від того, до \enquote{своїх}, чи до \enquote{чужих} вони
стосуються.

Звинувачувати людей у злочинах, які вони не чинили є гріх!

Крім того, це злочин перед молоддю - отруювати їхні мізки брехнею!

\iusr{Valeriy Levitskiy}
\textbf{Nadegda Volodymyr Fedko} 

Надеюсь вы не обманываете, как большинство, подстраивающееся под ситуацию

\iusr{Надежда Владимир Федько}
\textbf{Valeriy Levitskiy} 

А який сенс обманювати?

Зараз достатньо інформації про те, що Гітлер був змушений прийняти рішення щодо
превентивного удару по СССР!

Друга Світова війна була запрограмована у Версалі Великобританією і Францією!
Розчленування Німеччини, пограбування репараціями і грабіж німецьких колоній!

Приєднання частини території Німеччини до Польщі і Чехословаччини - ось міни,
закладені під Європу т. зв. Версальським мирним договором.


\iusr{Надежда Владимир Федько}
\textbf{Valeriy Levitskiy} Про це навіть Ленін писав)

\iusr{Надежда Владимир Федько}
\textbf{Valeriy Levitskiy} 

Ще одним винуватцем Другої світової війни є Польща!

Третій Райх неодноразово пропонував вирішити питання із екстериторіальним
сполученням (залізниця і автобан) з відтятими німецькими територіями та
статусом міста Данциг. І кожного разу Польща, за спиною якої стояла
Великобританія, відмовлялася розмовляти на цю тему.

\iusr{Надежда Владимир Федько}
\textbf{Valeriy Levitskiy} 

Більше того, коли вже почався наступ Вермахту на Польщу, то Гітлер пропонував
Беку і керівництву Англію провести переговори... обіцяючи зупинити наступ. Ті
проігнорувати ініціативу Гітлера.

Війна потрібна була Великобританії, щоб знищити Німеччину, як конкурента.

\end{itemize} % }

\iusr{Valeriy Levitskiy}
\textbf{Jon Yossarian} Извините, но вы дурилка просто. Я таких десятки видел, мне даже вас жаль чесно
Дисскуссия окончена

\begin{itemize} % {
\iusr{Jon Yossarian}
\textbf{Valeriy Levitskiy} нет, старик, это вы мишигине копф  @igg{fbicon.face.grinning.squinting} 
Вот у нас всегда так или Эйнштейн или Яша Кедми  @igg{fbicon.beaming.face.smiling.eyes} 

\iusr{Jon Yossarian}
\textbf{Valeriy Levitskiy} вместе со старостью приходит мудрость... В вашем случае старость пришла одна  @igg{fbicon.beaming.face.smiling.eyes} 

\iusr{Надежда Владимир Федько}
\textbf{Jon Yossarian} Я думаю, що в даному випадку опонент дотримується корпоративної думки свого етносу.
\end{itemize} % }

\iusr{Valeriy Levitskiy}

Вы имеете право на собственное мнение

Я живу в Германии. Если бы вы открыто здесь в печати или в интернете
высказывались так, минимум это стоило бы вам репутации и работы.

Дискутировать не будем Хорошего дня вам !

\begin{itemize} % {
\iusr{Надежда Владимир Федько}
\textbf{Valeriy Levitskiy} Чим сучасна Німеччина відрізняється від СССР? Та ж сама політична цензура і заборона на професію!

\iusr{Надежда Владимир Федько}
\textbf{Valeriy Levitskiy} Цікава політика!

Значить, якщо я буду публічно поширювати наклепи про нібито \enquote{гомосексуалізм}
Гітлера; \enquote{казнокрадство} Герінга чи \enquote{пияцтво} Геббельса, то це вважатиметься
нормальним.

А якщо я скажу, що Гітлер був чудовим художником, то одразу попаду під цензуру
і кримінал.

Слава Богу, що в Україні ми не беремо прикладу з Європи!
\end{itemize} % }

\iusr{Jon Yossarian}
\textbf{Надежда Владимир Федько} шось мені вважається, що ви розкидуєте бісер перед свинями

\begin{itemize} % {
\iusr{Надежда Владимир Федько}
\textbf{Jon Yossarian} Просто звичайна порядність змусила мене довести дискусію до логічного кінця)

\iusr{Jon Yossarian}
\textbf{Надежда Владимир Федько} як тіки інтернет став загальнодоступний усі таргани повилазили. Ви їм нічого не доведете...
\end{itemize} % }

\iusr{Valeriy Levitskiy}

Вы видимо не понимаете меня... Эти монстры прошлого не волнуют здесь никого,
полно проблем у всех и так, люди не живут историей 3 Рейха Я раз лет 15 назад
разговорился с одним старым танкистом, попытался что то у него спросить, он
прошел на танке Украину дошел до Волги был ранее плен итд ,,Он посмотрел на
меня и сказал, запомни парень : "Война очень плохо..."и добавил: " а Наташа
хорошо !"Вот о чем народ тут думает, а не о том был ли Адольф педиком или нет.

\begin{itemize} % {
\iusr{Надежда Владимир Федько}
\textbf{Valeriy Levitskiy} Ви про піраміду Маслоу знаєте? Люди, які не цікавляться історією і власною генеалогією - примітивні особі. Потреби шлунку і примітивних видовищ замінюють для них все інше.
А танкіста я розумію... Він не довіряв Вам, тому так і сказав.

\iusr{Valeriy Levitskiy}
\textbf{Nadegda Volodymyr Fedko} я с Надеждой или Владимиром общался, просто интересно ?логика вроде не женская

\iusr{Надежда Владимир Федько}
\textbf{Valeriy Levitskiy} Це спільний акаунт моєї покійної дружини, Надійки, і мій - Володимира.
Але наші погляди багато в чому співпадали. Особливо у ставленні до комуністичної ідеології і радянської влади.
Ми не були активними борцями, оскільки не бажали провести своє життя у концтаборах. Ми просто жили своїм життям, ігноруючи компартію.
\end{itemize} % }

\iusr{Valeriy Levitskiy}

Извините за нескромный вопрос, соболезную... не хотел вас обидеть... понимаю
вас

\iusr{Надежда Владимир Федько}
\textbf{Valeriy Levitskiy} 

Дякую. Нічого страшного. За минулі півроку я вже трохи адаптувався...

Надійчині подруги мене упросили, щоб я не закривав її акаунт, а викладав на
ньому фотографії з нашого життя (ми прожили разом 47 років).

Ну, і іноді беру участь у дискусіях на цікаві теми.

\end{itemize} % }

\iusr{Надежда Владимир Федько}

Давайте відкинемо політичні і пропагандистські погляди, а розглянемо факти...

Радянська влада в особі партійного керівництва, а також армія і НКВС перед
втечею з Києва вчинили злочини: розстріл політичних в'язнів; звільнення
кримінальних злочинців; вивезення з міста всіх продовольчих запасів; вивезення
усіх медичних засобів; вивезення обладнання заводів і фабрик; мінування і
підрив Хрещатика, електростанції, водопроводу та станції перекачки нечистот
каналізаційної мережі.

Нагадаю, що в Києві на момент відходу ЧА залишилося близько 400 000 (!) осіб, в
переважній більшості жінок з дітьми і старих людей.

\iusr{Jon Yossarian}
\textbf{Надежда Владимир Федько} цікаво, хто буде це заперечувати.

\iusr{Надежда Владимир Федько}

Сказавши А, треба сказати і Б...

А тепер запитання до прихильників \enquote{совка}:

1) Хто забезпечив продовольством населення Києва. Хоча було дуже важко і
голодно, але Голодомору в Києві не було.

2) Хто гасив пожежі, коли палав Хрещатик і будинки на вулицях поруч?

3) Хто захищав спокій і життя киян від бандитів, злодіїв і підпільників?

4) Хто відроджував промислове і господарське, економічне життя в Києві?

\begin{itemize} % {
\iusr{Надежда Владимир Федько}

Ніякої ідеології! Голі факти!

\iusr{Jon Yossarian}
\textbf{Надежда Владимир Федько} вони чітко виконували обов'язки окупуючої сторони
\end{itemize} % }

\iusr{Надежда Владимир Федько}

НЕ розумію людей, які вступають в дискусію заради того, щоб облити помиями
опонента!

В даній дискусії є конкретні факти і конкретні злочини!

Навіщо \enquote{растекаться мыслию по древу} і відходити від теми публікації?

\begin{itemize} % {
\iusr{Jon Yossarian}
\textbf{Надежда Владимир Федько} 

бо воно розуміє, що воно дурне, але усе, у що воно вірить страстно є брехня і це його корьожить нє по дєтскі, то тре щось збріхать  @igg{fbicon.face.grinning.squinting} 
І воно починає воняти.
Таки дуже погано, бо чорт старий, а ума нема  @igg{fbicon.beaming.face.smiling.eyes} 
\end{itemize} % }

\iusr{Людмила Пащинська}

Бомбардировка Дрездена (нем. Luftangriffe auf Dresden, англ. Bombing of
Dresden) — серия бомбардировок немецкого города Дрезден, осуществлённых
Королевскими военно-воздушными силами Великобритании и Военно-воздушными силами
США 13—15 февраля 1945 года во время Второй мировой войны. В результате
бомбардировок около четверти промышленных предприятий города и около половины
остальных зданий (городская инфраструктура и жилые дома) было уничтожено или
серьёзно повреждено. По утверждениям американских ВВС, на несколько недель было
парализовано движение транспорта через город. Оценки количества погибших
разнились от 100 тысяч в официальных немецких отчётах времён войны до 275
тысяч. ЭТО ТОЖЕ БОЛЬШЕВИКИ СДЕЛАЛИ ?

\begin{itemize} % {
\iusr{Надежда Владимир Федько}
\textbf{Людмила Пащинська} Англія і США були союзниками СССР)

\iusr{Людмила Пащинська}
С такими друзьями и врагов не надо.
\end{itemize} % }

\iusr{Tovarish Inka}

Хто небудь, зніміть Надєжду з коня і відберіль у неї шаблюку, якою вона
розмахує, захищаючи улюблений нею фашистський режим. Дякую!

\begin{itemize} % {
\iusr{Jon Yossarian}
\textbf{Tovarish Inka} дитино, яка ж ви дурна, як то сало без хліба  @igg{fbicon.face.screaming.in.fear} 
Як це ж так, сало то ви ж любите, бачу по фото, чого ж не думаєте?  @igg{fbicon.face.pleading} 

\iusr{Надежда Владимир Федько}
\textbf{Tovarish Inka} Дякую за увагу! Я - Володимир. У Вас немає інших аргументів, як навісити ярлика \enquote{фашист})

\begin{itemize} % {
\iusr{Jon Yossarian}
\textbf{Надежда Владимир Федько} ви впевнені, що вона знає різницю між хвашистом та нацистом?  @igg{fbicon.face.smirking} 
Це як дитині з дітсадка доводити, що таке корпускулярно-волнова теорія світла - немає фундаменту, читать навчилась, а розуміти, не, ми такого не походили

\iusr{Надежда Владимир Федько}
\textbf{Jon Yossarian} Ну хочеться жінці, щоб я був \enquote{фашист}) Пробачимо їй це бажання.
У французів є приказка: \enquote{Если женщина неправа - мужчина должен извиниться!}

\iusr{Jon Yossarian}
\textbf{Надежда Владимир Федько} дякую за спілкування  ☺ ️ 
Але цю ентропію нам не порушити, нажаль  @igg{fbicon.face.pleading} 

\iusr{Надежда Владимир Федько}
\textbf{Jon Yossarian} Навзаєм! Приємно спілкуватися з адекватними розумними людьми.

\iusr{Надежда Владимир Федько}

Ліваки і ліберали ще ніяк не зрозуміють просту істину - Третій Райх потерпів
військову, а не ідеологічну поразку! Подивіться на видавничу політику Росії...
Якими шаленими накладами видаються та перевидаються книги про Третій Райх -
мемуари та інша література (геральдика, уніформа, озброєння). І порівняйте з
виданням літератури про Радянську армію.

Варто зазначити, що література про \enquote{нацистську} Німеччину не залежується на
прилавках!

Подивіться скільки молодіжних каналів в Телеграм-месенджері присвячено історії
і культурі Третього Райху - сотні і сотні. Студентська молодь самостійно (і
безкоштовно) перекладає німецькі оригінали 20-х - 40-х років і викладає їх в
Інтернет. Це величезний пласт сучасної НС-субкультури.

\iusr{Олег Курилов}
\textbf{Надежда Владимир Федько} 

Какие вам нужны аргументы?????? Кажеться я полностью во всех постах Вам
ответил. Кроме, ну совсем глупых. И зачем вы сюда идеологию приплетаете? Какое
отношение идеология имеет к истории?

\end{itemize} % }

\iusr{Tovarish Inka}
От і поспілкувалися два \enquote{розумники}: трансНадєжда і дід Мазай. Ну, хоч так...

\end{itemize} % }

\iusr{Людмила Пащинська}
Вот я и говорю- что пропаганда Гебельса в 1943 году - что сейчас пишут - один в один ..........

\iusr{Надежда Владимир Федько}

Щиро дякую всім, хто взяв участь у цікавій дискусії. Вона збагатила мої
пізнання сучасного погляду на події Німецько-радянської війни 1941-1945 рр.

\begin{itemize} % {
\iusr{Tovarish Inka}
\textbf{Надежда Владимир Федько}, і тобі прощавай, транснадіє! І їж сало, бо надєжда твоя страшна як атомна війна

\iusr{Надежда Владимир Федько}
\textbf{Tovarish Inka} Жінок бити не шляхетно, але за такі слова ТИ цього заслуговуєш!

\iusr{Надежда Владимир Федько}

Але я тебе прощаю! Звідки мізки і розум у лівачки), захисниці комуністів...

\iusr{Tovarish Inka}
\textbf{Надежда Владимир Федько} , а шо такоє, сало натерло? Чи тільки тобі дозволено фото роздивлятися? Іди проспись, хамло

\iusr{Tovarish Inka}
\textbf{Надежда Владимир Федько} , Амінь!
\end{itemize} % }

\end{itemize} % }
