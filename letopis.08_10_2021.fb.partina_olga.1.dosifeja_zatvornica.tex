% vim: keymap=russian-jcukenwin
%%beginhead 
 
%%file 08_10_2021.fb.partina_olga.1.dosifeja_zatvornica
%%parent 08_10_2021
 
%%url https://www.facebook.com/permalink.php?story_fbid=2983120112002070&id=100009123325663
 
%%author_id partina_olga
%%date 
 
%%tags cerkov,kiev,pravoslavie,vera
%%title Преподобная Досифея (Тяпкина), Киевская, затворница
 
%%endhead 
 
\subsection{Преподобная Досифея (Тяпкина), Киевская, затворница}
\label{sec:08_10_2021.fb.partina_olga.1.dosifeja_zatvornica}
 
\Purl{https://www.facebook.com/permalink.php?story_fbid=2983120112002070&id=100009123325663}
\ifcmt
 author_begin
   author_id partina_olga
 author_end
\fi

Преподобная Досифея (Тяпкина), Киевская, затворница

В пятнадцатом веке на южных подступах к Киеву, на Китай-горе монахи
Киево-Печерской Лавры создают пещерный монастырь, а затем и Лаврский скит.
Подвизалось здесь немало великих старцев, но самым знаменитым был преподобный
Досифей, прославившийся в ХVIII веке своими великими подвигами. Никто и не
подозревал, что под именем старца Досифея долгие годы скрывалась православная
подвижница Дарья!

\ifcmt
  tab_begin cols=3

     pic https://scontent-lhr8-1.xx.fbcdn.net/v/t1.6435-9/244386462_2983119958668752_8827227518234553291_n.jpg?_nc_cat=100&ccb=1-5&_nc_sid=8bfeb9&_nc_ohc=GeIkCMFInd0AX-9xJ0G&_nc_ht=scontent-lhr8-1.xx&oh=e22cf76ffb79496b09616ccee6b29060&oe=619DF476

     pic https://scontent-lhr8-1.xx.fbcdn.net/v/t1.6435-9/244406066_2983120008668747_8326707850055180259_n.jpg?_nc_cat=107&ccb=1-5&_nc_sid=8bfeb9&_nc_ohc=OKyuCOxefOMAX_CYxUo&_nc_ht=scontent-lhr8-1.xx&oh=f74c178a1e5fdc58bf7feab4119292a7&oe=619FD6AF

		 pic https://scontent-lhr8-2.xx.fbcdn.net/v/t1.6435-9/244520152_2983120072002074_3954837508271728207_n.jpg?_nc_cat=105&ccb=1-5&_nc_sid=8bfeb9&_nc_ohc=a7hgfM2AbOcAX-al1Gk&_nc_ht=scontent-lhr8-2.xx&oh=5328cd2162349f10d9b95a439e3ca17a&oe=61A0CF0D

  tab_end
\fi

Родилась она в 1721 году в семье богатых и знатных рязанских дворян Тяпкиных.
Род свой вели они от героя Куликовской битвы Василия Варгоса, ставшего потом
любимым окольничим Димитрия Донского. Неразрывные духовные связи соединяли
дворян Тяпкиных с родом святого князя. Бабушка Дарьи ушла под старость из мира
в женский Свято-Вознесенский монастырь, находившейся в Московском Кремле, и
стала мантийной монахиней Порфирией. Когда девочке исполнилось два года,
родители с маленькой Дарьей отправились на богомолье в Сергиеву Лавру. Проездом
через Москву они навестили в Вознесенском монастыре бабушку, монахиню Порфирию,
которая упросила родителей оставить внучку при ней, дабы молитвами управить
благой путь жизни ребенка. Монашеская жизнь с раннего детства оставила глубокий
след в сознании Дарьи и повлияла на всю ее дальнейшую жизнь, тем более что душа
у девочки была до крайности чуткой и трепетной, милосердной ко всем. Молилась
Дарья, как взрослая, никаких поблажек: выстаивала длинные монастырские службы
вместе со всеми, клала поклоны сколько надо, осеняя себя крестным знамением.
Пятилетняя Дарья строго держала посты, по средам и пятницам вкушала только хлеб
и воду. Рано выучилась грамоте и вечерами читала бабушке Евангелие.

В монастыре Дарья прожила семь лет. А потом пришло время забрать девочку домой
на светское воспитание, как и положено было при ее дворянском происхождении.
Новая жизнь обрушилась на Дарью бессмысленным, никчемным грузом. Пустая трата
времени пришлась девочке не по сердцу и вызывала лишь жалость ко взрослым.
Окружающие ее люди, в свою очередь, тоже недоумевали: богатая, а ходит, как
крестьянка, барыня, а спит на доске, под голову сено кладет да еще с бедняками
трапезу делит и разговаривает как с равными. Чужд был богатый родительский дом,
полный гостей, болтовня сестер, глупые мечты о завидных женихах, украшениях и
нарядах. Не нужно было Дарье и светское образование, которое непременно хотели
дать ей родители-дворяне. Для этого предполагалось выписать для юной затворницы
из столицы бойкую учительницу – француженку... Последней каплей стали разговоры
о знатном женихе. И Дарья сделала свой выбор.

Однажды в солнечный майский день Дарья попросилась пойти с сестрами в лес. Для
нее это было единственной возможностью уйти из дома. И она, ни о чем не
сожалея, ушла навсегда – от дома, от родителей, от близких ей людей… от чуждой
ей жизни. Сколько ни звали ее сестры, сколько ни аукали, так и вернулись в
слезах домой без младшей сестры.

Сначала решила она пойти в Вознесенский монастырь к своей бабушке. Но подумав,
что родственники найдут ее там и силой вернут назад, поступила иначе. Дарья
остригла волосы, приобрела на базаре мужскую одежду и, переодевшись в простого
крестьянского мальчика, направилась в Троице-Сергиеву Лавру. Долгое путешествие
изменило ее лицо, оно огрубело, стало смуглым от палящего солнца, голос охрип,
тело исхудало до невозможности. Теперь никто не смог бы признать в ней прежнюю
красавицу Дарью. Она назвалась беглым крестьянином Досифеем (по-гречески
Досифей означает «богом данный»). Посоветовавшись, старцы разрешили ей
пребывать в Лавре тайно, но не иноком, а просто послушником. Сколько
осторожности, сколько силы воли потребовалось Дарье, чтобы никто не догадался,
что она девушка.

Так прошло три года. Однажды в Лавру в поисках пропавшей дочери приехала мать
Дарьи. Долго молилась она у мощей святых угодников Божьих. Во время службы
заметила молодого послушника. Материнское сердце безошибочно угадало любимое
чадо, и она попросила монахов, чтобы юный послушник пришел к ним после службы в
гостиницу. Но решительная Дарья, не подавая виду, мигом собралась, накинула
котомку на плечи и снова пустилась в путь.

Начался беспримерный подвижнический путь послушника Досифея на родину
иночества, в Киево-Печерскую Лавру. Через два месяца он уже стоял перед ее
настоятелем и рассказывал, что с раннего детства мечтал стать монахом.
Беспаспортного беглеца отказались принять в число братии. В поисках убежища
Досифей приходит в окруженную лесами Китаевскую пустынь, где издавна селились
православные подвижники. Досифей поднялся на самый верх горы и руками вырыл
себе в земле пещеру. В ней он прожил 17 лет. Питался он только хлебом, который
изредка приносили ему иноки из Китаевой пустыни, мхом и корнями.

Слава о прозорливом и мудром отшельнике разлеталась по городам и весям
Российской империи. Тысячи людей приходили к нему за помощью. Вельможи и нищие,
старики и подростки прибегали к этому великому утешителю. Но никто из
паломников не видел его лица. Он никогда не покидал пещеры и никого не впускал
к себе. С приходящими к нему за советом и благословением беседовал только через
маленькое окошечко. Исключение было сделано только для государыни императрицы
Елизаветы Петровны, посетившей киевские святыни в 1744 году вместе с
наследником престола Петром Федоровичем и его невестой Екатериной, будущей
императрицей. Чтобы императрица смогла пешком подняться наверх Китай-горы в
пещеру Досифея, монахи срочно вбили в узкую тропинку множество деревянных
ступенек.

Долгое время беседовал преподобный Досифей с царицей. Читал и прозревал старец
в ее сердце. На всю жизнь запомнила она свою покаянную исповедь у таинственного
затворника и его советы о будущем трона, о выборе престолонаследника и его роли
в истории России...

Досифей пользовался большим уважением у киевских архиереев и монашеской братии,
но даже не был пострижен в рясофор – первый чин монашества. Елизавета Петровна
повелела немедленно совершить обряд пострижения и сама присутствовала на нем.
При постриге подвижнику было дано то же самое имя – Досифей...

Все это время дворяне Тяпкины не переставали искать свою дочь по разным
обителям России. Однажды к прозорливому старцу приехала на гору Китай его
родная сестра Агафья, чтобы узнать о судьбе пропавшей Дарье. Через узкое
окошечко благословил Досифей свою сестру и успокоил: Дарья стала монахиней,
служит Господу, живет высокой духовной жизнью. У нее свой путь, и искать ее
больше не надо. Именно от преп. Досифеи получил в Китаево благословение идти в
Саровскую обитель будущий великий старец преп. Серафим.

Когда в Российской империи было запрещено отшельничество, преподобная
переселилась на Дальние пещеры в Лавре, где ей назначили келейника по имени
Феофан (будущего известного соловецкого подвижника). Однажды Досифей послал
Феофана принести ладану и стал раздавать его по кусочкам каждому из приходящих.
«Люди станут падать на дороге, умирать как мухи… Но вы не падайте духом, дети,
а молитесь!..» И действительно, в конце 1770 года на Украине случилась моровая
язва. Многих скосила она, даже иноки в Лавре не спаслись. Уцелели только те
люди, которым Досифей подал ладан и благословил молиться.

За день до своей кончины преподобный Досифей вышел из затвора и пошел по
келлиям: прощаться с братией. Перед каждым иноком падал старец на колени и со
слезами просил прощения. Простившись со всей обителью, он затворился в своей
келлии, чтобы провести последние часы в молитве. Скончался Досифей 25 сентября
1776 года на 56-м году жизни. Напрасно наутро стучал послушник в его дверь.
Никто не откликнулся. Когда же монахи вошли в келлию, глазам их представилась
поразительная картина: тихо теплилась лампада, перед иконой стоял на коленях
Досифей и как будто молился (точно так же в 1833 году отойдет к Господу и
преподобный Серафим Саровский) – правая рука почившего старца поднята вверх в
крестном знамении. В левой – зажата записка, адресованная братии: «Тело мое
приготовлено к отшествию в вечную жизнь. Молю вас, не открывая, предать его
погребению».

Завещание старца исполнили в точности. Никто не посмел обнажить многотрудное
тело и обмыть его. С благоговением погребли его иноки в монастырской ограде на
северной стороне Свято-Троицкой церкви.

Нет ничего тайного, что не стало бы явным. Спустя несколько лет Агафья Тяпкина
снова приехала в Китаевскую пустынь, очень расстроилась, не застав старца в
живых. Попросила взглянуть на его портрет, узнала свою сестру Дарью и потеряла
сознание. Так стало известно, что великий затворник Китаевской пустыни был не
мужчиной, а девицей – преподобной Досифеей.
