% vim: keymap=russian-jcukenwin
%%beginhead

%%file 19_12_2022.fb.rojz_svitlana.kyiv.1.obijmanci
%%parent 19_12_2022

%%url https://www.facebook.com/svetlanaroyz/posts/pfbid02V4dK5Bv4EjfdTGTowywVc6Ng3BJHK7L2UWBetff86pvCXe4mNnZJfAz9SeBUVYBpl

%%author_id rojz_svitlana.kyiv
%%date

%%tags
%%title "Обійманці". Наш новий проект в подарунок дітям та дорослим

%%endhead

\subsection{\enquote{Обійманці}. Наш новий проект в подарунок дітям та дорослим}
\label{sec:19_12_2022.fb.rojz_svitlana.kyiv.1.obijmanci}

\Purl{https://www.facebook.com/svetlanaroyz/posts/pfbid02V4dK5Bv4EjfdTGTowywVc6Ng3BJHK7L2UWBetff86pvCXe4mNnZJfAz9SeBUVYBpl}
\ifcmt
 author_begin
   author_id rojz_svitlana.kyiv
 author_end
\fi

"Обійманці". Наш новий проект в подарунок дітям та дорослим. Напевно, один з
найтепліших наших з Halyna Tytysh  проектів. 12 карток з зображенням різних
обіймурок- обіймишок - обійм для різних дистанцій в контакті з описами дій, 12
розгорток для поробок, інструкція. А ще пакети стикерів з різними обійманцями
для месенджерів - щоб надсилати обійми, коли ми на відстані.

Цей проект дуже простий і одночасно глибокий. Він більше, ніж про близькість.
Хоч ми і знаємо - що саме близькість - зцілює. Саме близькі надійні стосунки
допомагають розпочати процес відновлення.

Зараз у дорослих не вистачає часу і сил вигадувати ігри, часто немає можливості
витримувати емоційне навантаження і бути в тілесному контакті.

Коли ми переживаємо такий довготривалий стрес, ми втрачаємо контакт з
тілесністю. І дорослим, і дітям, при всій потребі в близькості, може бути
складно витримати тілесний контакт. І нам важливо обережно обирати дистанцію,
поважати потреби і простір інших,  але й підтримувати близькість.

Тому я вигадала різні дистанції в грі. І діти, і дорослі можуть самі обирати,
зараз товкатися, як ведмедики, чіплятись, як мавпочки, торкатись долонями, наче
восьминіжки, чи надсилати повітряні поцілунки, як рибки. (Все це описано на
картках)

А поробки - щоб задіяти і моторику, і повернути відчуття кордонів, коли
вирізаємо та розмальовуємо, щоб робити іграшку разом, чи в подарунок комусь і
одночасно активізувати різні відділи мозку  - бо розмалювати, вирізати,
складати об'ємну фігурку - це все про синхронізацію різних відділів.

А ще нам потрібна радість від відчуття, що ми щось проконтролювали, довели до
кінця, створили - це додає нам відчуття маленької перемоги і підвищує
самооцінку.

А стікери - щоб відчувати близькість, де б не були.

Ми можемо обіймати різними обійманцями дітей після пробудження, можемо обережно
"торкатись" дистантними обіймами підлітків, можемо обирати картки наосліп,
можемо обирати свідомо, можемо робити поробки, можемо просто говорити "обіймаю,
як....".

Ви б бачили чат нашої команди при розробці - найтепліша команда та робота: я
вдячна Галині  і ГО "Смарт освіта"  за всі наші спільні проекти, Tania Stus  -
редакторка і саме Тетяна запропонувала всі назви Обіймурки, обіймишки. (Мене
вистачило лише на обійманці 🙂. Helena Iastremska  - малювала наших звіряток,
Olena Marchyshyna  - створювала розгортки та стікери, Roman Marchishin  -
займався макетуванням. Катя Костенко  - координувала нашу команду.

Завантажити всі 12 карток і розгорток можна тут: \url{https://bit.ly/3VbNZqR}

Стикерпаки з обійманцями для месенжерів .Спробуйте, неможливо зупинитись

Viber: \url{https://bit.ly/3jgJFJu}

Телеграм: \url{https://bit.ly/3BPvhhR}

Обіймаю, Родино ❤️ так сильно, наче ми вже Перемогли.

\ii{19_12_2022.fb.rojz_svitlana.kyiv.1.obijmanci.orig}
\ii{19_12_2022.fb.rojz_svitlana.kyiv.1.obijmanci.cmtx}
