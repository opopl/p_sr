% vim: keymap=russian-jcukenwin
%%beginhead 
 
%%file 05_09_2021.stz.news.ua.mrpl_city.1.marafon_mizhn_rezydencij_2021
%%parent 05_09_2021
 
%%url https://mrpl.city/blogs/view/marafon-mizhnarodnih-rezidentsij-2021
 
%%author_id demidko_olga.mariupol,news.ua.mrpl_city
%%date 
 
%%tags 
%%title Марафон міжнародних резиденцій 2021
 
%%endhead 
 
\subsection{Марафон міжнародних резиденцій 2021}
\label{sec:05_09_2021.stz.news.ua.mrpl_city.1.marafon_mizhn_rezydencij_2021}
 
\Purl{https://mrpl.city/blogs/view/marafon-mizhnarodnih-rezidentsij-2021}
\ifcmt
 author_begin
   author_id demidko_olga.mariupol,news.ua.mrpl_city
 author_end
\fi

Цьогоріч Маріуполь має, як ніколи, насичену культурну програму завдяки й
Марафону міжнародних резиденцій 2021. До міста приїжджають запрошені режисери з
Європи і спільно з місцевими акторами створюють різні культурні проєкти. Цей
Марафон має на меті завдяки сучасній театральній лабораторії мистецького
діалогу відкрити новий культурний Маріуполь для України та всього світу.

\ii{05_09_2021.stz.news.ua.mrpl_city.1.marafon_mizhn_rezydencij_2021.pic.1}

Першою із запрошених митців стала \emph{\textbf{Мадлен Бонгард}} – швейцарська режисерка,
акторка та педагогиня. Вона провела серію акторських практик, що досліджують
усвідомлення присутності. Це дозволило учасницям ще глибше вивчити свої
відчуття. Від тіла до внутрішнього почуття – актриси занурились у справжніх
персонажів. Це історія про свідомість, про внутрішній ритм та задоволення від
дослідження.

Ексклюзивною спец-подією фестивалю iStage 2021 став перформанс  \textbf{\enquote{Ви там і тут}}
– проєкт, що створений у колаборації між Маріуполем та Лондоном. У рамках
програми міжнародних театральних резиденцій режисерка \emph{\textbf{Джозі дейл Джонс}}  спільно
з командою Народного Театру \enquote{Театроманія} досліджувала соціальні теми за
допомогою слів, зображень та руху. Керівник проєкту і режисер \enquote{Театроманії}
\textbf{Антон Тельбізов} розповів, що:

\begin{quote}
\em\enquote{це дуже велика робота, яка піднімає важливі
соціальні теми, що розкриває акторів по-особливому, де кожен може побачити себе
з боку і стати учасником цієї великої вистави}. 
\end{quote}

В основу перфомансу лягли історії акторів, адаптовані під театральний формат на
онлайн-репетиціях.

\ii{05_09_2021.stz.news.ua.mrpl_city.1.marafon_mizhn_rezydencij_2021.pic.2}

Третю резиденцію представив актор і режисер-постановник Національного театру
Грецї в Афінах та Муніципального театру Пірея \emph{\textbf{Евангелос Космідіс}}. Його
перфоманс \textbf{\enquote{Аляска}} присвячено проблемам підлітків. З українцями і зокрема
маріупольцями грецький режисер співпрацює вперше, але сподівається, що це не
останній проєкт в Україні. Режисеру подобається працювати з дітьми та
підлітками, зосереджуватися на їхніх проблемах та створювати разом тексти. Він
працює з інструментами \enquote{Devised theatre}, завдяки яким, на його думку,
перфоманс вийде дійсно актуальним і цікавим: 

\begin{quote}
\em\enquote{ми почнемо з нуля без тексту та
ідей, разом з командою відкриватимемо теми, які хочемо показати глядачам. Такий
метод створює багатошаровий матеріал, тому що після того, як ми розробимо наші
перші чернетки, ми перейдемо до зустрічі та поєднання цього твору мистецтва зі
світовою літературою та культурою. Таким чином, перформанс ідеально
відповідатиме актуальним питанням}.
\end{quote}

\ii{05_09_2021.stz.news.ua.mrpl_city.1.marafon_mizhn_rezydencij_2021.pic.3}

Еван Космідіс, зосереджуючись на проблемах дітей та підлітків, допомагає їм
якнайкраще розповідати свої справжні історії аудиторії. Учасниками цього
проєкту стали актори Театру ляльок, Народного театру \enquote{Театроманія} та учні
Першої театральної школи. В основу перфомансу лягла історія про пошуки щастя, в
юному і підлітковому віці. \emph{\textbf{Ірина Руденко}} наголосила, що цей проєкт – серйозна
інвестиція  в юних акторів та безцінний досвід для її учнів.

\ii{05_09_2021.stz.news.ua.mrpl_city.1.marafon_mizhn_rezydencij_2021.pic.4}

Перфоманс \enquote{Аляска} по-справжньому міжнародний, адже дійство відбуватиметься не
тільки українською та російською мовами, а й грецькою, польською, італійською.
Актриса Народного театру \enquote{Театроманія} \emph{\textbf{Марія Бойко}} розповідає, що вона співає
пісні італійською та польською і, незважаючи на те, що ці мови дівчина не знає,
все одно розуміє, про що співає, завдяки роботі режисера.

\ii{05_09_2021.stz.news.ua.mrpl_city.1.marafon_mizhn_rezydencij_2021.pic.5}

Результатом Марафону міжнародних резиденцій 2021 стане\par\noindent створення в Маріуполі
культурного продукту, який посприяє перетворенню міста на величезний творчий
майданчик! Координаторка резиденції \emph{\textbf{Вікторія Федорів}} підкреслює, що учасникам
до вподоби незвична форма роботи з грецьким режисером і, що вона рада бачити
захват і бажання  акторів працювати над цим унікальним проєктом. Вікторія
виступає і в ролі перекладача та завжди допомагає режисеру та учасникам проєкту
зрозуміти одне одного. Втім за час перебування в Маріуполі Евангелос Космідіс
почав вчити українську та російську мови.  Всі репетиції дуже насичені,
тривають досить довго. І грецький режисер, і маріупольські актори сподіваються,
що перфоманс  \enquote{Аляска} сподобається глядачам.

Марафон міжнародних резиденцій 2021 реалізується в рамках проєкту
культурно-мистецьких ініціатив \enquote{Діалог мовою мистецтва} за підтримки
Українського культурного фонду. Перфоманс \enquote{Аляска} маріупольці зможуть
подивитися безкоштовно у другій половині вересня за умови попередньої
реєстрації, про яку буде повідомлено на сторінці в фейсбук \emph{\textbf{\enquote{Маріуполь. Велика
культурна столиця України}}}.
