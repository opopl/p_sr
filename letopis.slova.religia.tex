% vim: keymap=russian-jcukenwin
%%beginhead 
 
%%file slova.religia
%%parent slova
 
%%url 
 
%%author 
%%author_id 
%%author_url 
 
%%tags 
%%title 
 
%%endhead 
\chapter{Религия}

%%%cit
%%%cit_head
Война за Веру Христову - Страна Агностиков и Атеистов
%%%cit_pic
%%%cit_text
– У збірці є рядок: \enquote{Відважно воює за віру Христову країна агностиків та
атеїстів}. Це поетичний троп чи діагноз Україні?  – Це просто образ. Не брався
би ставити діагнози, занадто це марнославно й претензійно. Якщо ви питаєте про
\emph{релігію}, то крім агностиків та атеїстів маю велику кількість знайомих
вірян різних церков. До їхніх духовних пошуків та розчарувань ставлюсь із
повагою та розумінням. Якось так сталося, що ще з юнацького віку трапилося
бути свідком постання різних українських \emph{релігійних} громад. Тому це все
моє середовище, незалежно від моїх власних \emph{релігійних} переконань
%%%cit_comment
%%%cit_title
\citTitle{Письменник Сергій Жадан: Мені цікаво говорити про речі прості й
очевидні}, Олег Поляков, life.pravda.com.ua, 13.06.2021
%%%endcit

