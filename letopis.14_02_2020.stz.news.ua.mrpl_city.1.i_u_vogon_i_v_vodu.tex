% vim: keymap=russian-jcukenwin
%%beginhead 
 
%%file 14_02_2020.stz.news.ua.mrpl_city.1.i_u_vogon_i_v_vodu
%%parent 14_02_2020
 
%%url https://mrpl.city/blogs/view/i-u-vogon-i-v-vodudo-dnya-vsih-zakohanih
 
%%author_id demidko_olga.mariupol,news.ua.mrpl_city
%%date 
 
%%tags 
%%title І у вогонь, і в воду... (До дня всіх закоханих)
 
%%endhead 
 
\subsection{І у вогонь, і в воду... (До дня всіх закоханих)}
\label{sec:14_02_2020.stz.news.ua.mrpl_city.1.i_u_vogon_i_v_vodu}
 
\Purl{https://mrpl.city/blogs/view/i-u-vogon-i-v-vodudo-dnya-vsih-zakohanih}
\ifcmt
 author_begin
   author_id demidko_olga.mariupol,news.ua.mrpl_city
 author_end
\fi

\ii{14_02_2020.stz.news.ua.mrpl_city.1.i_u_vogon_i_v_vodu.pic.1}

Офіційно \emph{\textbf{День всіх закоханих}} існує вже більше 16 століть, хоча в Україні
популярний з 1990 року. Втім я хочу написати про людей, які ніколи не
відзначали це свято. Провівши разом майже \textbf{20 тисяч днів і ночей}, вони
зрозуміли, що для збереження міцних і теплих стосунків одних почуттів буде
замало, потрібні і повага, і терпіння, і робота над собою. А ще треба
усвідомлювати, що це саме твоя людина, з якою можна і в огонь, і в воду... 

\ii{14_02_2020.stz.news.ua.mrpl_city.1.i_u_vogon_i_v_vodu.pic.2}

У свої 25 років \emph{\textbf{Михайло}} мав чималий життєвий досвід. Він знав п'ять мов
(французьку, німецьку, польську, українську та росій\hyp{}ську), був працьовитим і
завзятим хлопцем, який не боявся труднощів. Під час війни потрапив до полону,
що не могло пройти безслідно. Однак його завжди бадьорий настрій і спокійний
характер приховували від інших душевний біль і важкі спогади. Вдень багато
працював, треба було родині допомагати: грошей часто не вистачало. Але
молодість все одно брала своє. Після роботи (працював водієм автобуса) ноги
самі несли парубка в Міський парк. Хіба всидиш вдома, коли поруч красивий парк,
де грає музика і гуляють вродливі дівчата? І одного разу хлопець запримітив на
танцях серед зграйки жвавих дівчат чарівну дівчину з гарним ім'ям Віра. Вона
красиво сміялася і якось дуже гордо і в той же час по-дитячому задирала своє
кругле підборіддя.

\ii{14_02_2020.stz.news.ua.mrpl_city.1.i_u_vogon_i_v_vodu.pic.3}

Михайло, незважаючи на свою привабливість, був досить сором'язливим і
спостерігав за Вірою перший час здалеку. Але одного разу побачив, як вона
катається на гойдалці, розгойдуючись до самого неба, вище всіх, і зрозумів: що
такої дівчини раніше ніколи не зустрічав.

Остаточно Віра підкорила Михайла наступного дня. Він разом з другом гуляв
вулицями міста і голосно співав пісні. Коли вони проходили повз кінотеатр
Перемоги, Михайло помітив Віру, яка дивилася прямо в його очі і теж разом з
подругою почала співати цю пісню. І Михайло, і його друг були зачаровані цим
неповторним голосом. Хлопець вирішив, що після цього співу у нього точно
вкрадуть Віру і він набрався сміливості нарешті познайомитися. Дівчині Михайло
відразу сподобався: симпатичний, скромний, ввічливий.

\ii{14_02_2020.stz.news.ua.mrpl_city.1.i_u_vogon_i_v_vodu.pic.4}

\emph{\textbf{Віра Солодуха}} активно займалася самоосвітою, читала наукову та художню
літературу, грала на фортепіано, займалася вокалом. У свої 22 роки вона
вважалася завидною нареченою. Молоді люди почали зустрічатися влітку 1951 році.
Через три місяці Михайло зробив коханій пропозицію. Оскільки вони не хотіли
обтяжувати свої родини, вирішили будувати своє життя самостійно. Причому
буквально... Михайло і Віра вдвох побудували два будинки. Зараз складно уявити,
щоб дівчина допомагала хлопцю під час будівництва. Але це було цілком реально.
Їхній будинок став результатом справжнього кохання і коштував їм неабияких
зусиль та нелегкої праці. Згодом вони стали батьками двох синочків. З 1959 року
жінка працювала головним бухгалтером. Роботу свою любила, але сімейний затишок
– більше. Закохані не зізнавалися одне одному, що часто відчували ревнощі,
боялися цим потурбувати. Хоча Михайло дійсно боявся, що його мудру дружину
колись заберуть у нього, а Віра не наважувалася спитати як чоловік ставиться до
жінок, що відкрито починають з ним фліртувати. Вона бачила, що увагу від інших
жінок він не сприймає і сподівалася, що й думати про це не варто. Найбільше
любили відпочивати вдвох. Часто ходили до улюбленого Міського Саду, а влітку
відпочивали в Трускавці. Щотижня Михайло влаштовував сімейне свято, де були
пісні і смачна випічка Віри. Всі проблеми на роботі чи вдома вони обговорювали.
Намагалися знайти вихід вдвох.

Однак у подружжя були свої труднощі. Третій син помер, коли йому виповнилося 1
рік і 5 місяців. Це стало справжнім ударом для сім'ї. Віра стала мовчазною і
закритою. А у Михайла не вистачало потрібних слів. Саме тоді він почав писати
їй листи, в яких захоплювався силою, красою, характером жінки, нагадував, що
тільки з нею \enquote{можна подолати будь-які випробування, \emph{тільки з нею – і в огонь, і
в воду...}}. Ці слова стали девізом їх життя.

\ii{14_02_2020.stz.news.ua.mrpl_city.1.i_u_vogon_i_v_vodu.pic.5}

Особисто мене найбільше вразило те, що подружжя завжди намагалося зберегти
теплі відносини, боролись один за одного часом навіть з собою. \emph{\textbf{Михайло
Миколайович}} після 40 років спільного життя навіть занотував 10 невеличких
правил, які допомагали їм і в подальшому сімейному житті. Впевнена, що ці
правила можуть стати в нагоді не тільки їхнім онукам та правнукам, а й всім
маріупольцям, які теж, як і ця чудова пара, намагаються зберегти свою сім’ю від
випробувань долі.

І ось власне \emph{\textbf{правила:}}

\begin{itemize} % {
\item 1. Дарувати коханій квіти, навіть якщо у вашому саду їх безліч. Ваза не повинна
стояти порожньою...

\item 2. Відпускати чоловіка з друзями тричі на тиждень

\item 3. Всі проблеми обговорювати та вирішувати разом

\item 4. Не лягати спати в сварці.

\item 5. Бути найкращими друзями для своїх дітей.

\item 6. Робити сюрпризи одне одному без приводу.

\item 7. Приймати одне одного такими, якими ви є.

\item 8. Розділяти побутові клопоти

\item 9. Проводити сімейні вечері щотижня

\item 10. Просто кохати, адже тільки кохання завжди повертає до молодості.
\end{itemize} % }

\ii{14_02_2020.stz.news.ua.mrpl_city.1.i_u_vogon_i_v_vodu.pic.6}

\begin{quote}
\large\em Дорогі читачі, всіх вітаю з Днем закоханих і від душі бажаю нехай поруч буде
потрібна і, головне, ваша людина. Нехай кожен день буде сповнений щирим
коханням, теплом і турботою!
\end{quote}
