% vim: keymap=russian-jcukenwin
%%beginhead 
 
%%file 16_10_2020.fb.fb_group.story_kiev_ua.1.panno_zhitnego_rynka.cmt
%%parent 16_10_2020.fb.fb_group.story_kiev_ua.1.panno_zhitnego_rynka
 
%%url 
 
%%author_id 
%%date 
 
%%tags 
%%title 
 
%%endhead 
\subsubsection{Коментарі}
\label{sec:16_10_2020.fb.fb_group.story_kiev_ua.1.panno_zhitnego_rynka.cmt}

\begin{itemize} % {
\iusr{Oleg Protsenko}
Тема почему ОН не стал мужем не раскрыта!)).
Спасибо за пост, интересно!

\iusr{Ирина Петрова}
\textbf{Oleg Protsenko} уплыл... к грекам @igg{fbicon.face.grinning.squinting} 

\iusr{Igor Popell}
Спасибо! Сотни раз проходил мимо и не обращал внимания. В старом времени, видимо, пропущена буква Б - Буки. Но почему? Ответ будет?

\begin{itemize} % {
\iusr{Ирина Петрова}
\textbf{Igor Popell} точно! "Буки" пропущена. Она не означала никакую цифру, она означала - "Бог'.

\iusr{Володимир Цибульський}
Не поминай всуе!
\end{itemize} % }

\iusr{Тома Храповицкая}
Спасибо Вам, никогда не присматривалась, даже не задумывалась, всё в спешке, всё некогда, спасибо, учусь!!!

\iusr{Ирина Петрова}
\textbf{Toma Zakharova} да, мы часто не останавливаем взгляд на деталях... сама часто об этом думаю)

\iusr{Марина Перепелиця}
Очень здорово! Я тоже никогда не обращала на это внимание...

\iusr{Ирина Петрова}
\textbf{Марина Перепелиця} да, а таких моментов в Киеве бесконечно много)

\iusr{Григорий Авраменко}
Клас.

\iusr{Ирина Петрова}
\textbf{Григорий Авраменко} спасибо

\iusr{Лариса Дешкова}

\ifcmt
  ig https://scontent-frx5-1.xx.fbcdn.net/v/t39.1997-6/s480x480/106271535_701635933951356_5830774832652794379_n.png?_nc_cat=110&ccb=1-5&_nc_sid=0572db&_nc_ohc=SXBFdb-4jEwAX-Mwm_c&_nc_ht=scontent-frx5-1.xx&oh=9f12097f979d2311f26238e7f72a2c64&oe=61942F9F
  @width 0.2
\fi

\iusr{Dedekalo Iryna}
Спасибо!

\iusr{Ирина Петрова}
\textbf{Dedekalo Iryna} приятно, что добавила маленькую толику в копилку интересного о нашем Городе)

\iusr{Maryna Chemerys}
Цікаво, дякую!

\iusr{Ирина Петрова}
\textbf{Maryna Chemerys} дуже приємно, що зацікавила. )

\iusr{Наталия Романенко}
Очень интересно.

\begin{itemize} % {
\iusr{Ирина Петрова}
\textbf{Наталия Романенко} спасибо! ,,В жизни все гораздо живописнее.

\iusr{Наталия Романенко}
\textbf{Ирина Петрова}

\ifcmt
  ig https://scontent-frx5-2.xx.fbcdn.net/v/t39.1997-6/s168x128/16781161_1341101952618574_7704631035023065088_n.png?_nc_cat=1&ccb=1-5&_nc_sid=ac3552&_nc_ohc=7VVbYisllDsAX_Wd3Vk&_nc_ht=scontent-frx5-2.xx&oh=26fbb3769d691fc4de785a969568c35c&oe=61946C6B
  @width 0.1
\fi

\end{itemize} % }

\iusr{Ольга Вакуленко}
Спасибо, даже не знала, что это так познавательно

\iusr{Ирина Петрова}
\textbf{Ольга Вакуленко} да, ведь очень многое мы просто видим, не вникая в сущность) а такого стооолько вокруг!

\iusr{Nadiya M Shana}
Никогда с этих позиций не рассматривала. Спасибо! Плз, детали при встрече. @igg{fbicon.face.blowing.kiss} 

\iusr{Ирина Петрова}
\textbf{Nadiya M Shana} поторопить со встречей. Есть планы - реконструкция рынка. Возможно, во время реконструкции панно не будет видно.

\iusr{Volodya Step}
Дуже цікаво, сподобалось. Чомусь ніколи не звертав уваги. Ви гарно розповіли.

\iusr{Ирина Петрова}
\textbf{Volodya Step} дякую! Наживо все набагато цікавіше!

\iusr{Ирина Комарова}

\ifcmt
  ig https://scontent-frx5-2.xx.fbcdn.net/v/t39.1997-6/s168x128/66699203_620859688421494_8793453631660621824_n.png?_nc_cat=1&ccb=1-5&_nc_sid=ac3552&_nc_ohc=67yvXghXJMIAX9Z4Iy8&_nc_ht=scontent-frx5-2.xx&oh=9f73e32c94725334ade4e582225e2814&oe=6195231F
  @width 0.1
\fi

\iusr{Lessya Ours}

Необыкновенно интересно. На этом рынке бывала мало и никогда не рассматривала
панно-чтото такое серо-бетонное, казалось мне. Теперь и ваши фото
внимательнейше изучу и, при случае, рассмотрю в реале!

Благодарю! @igg{fbicon.thumb.up.yellow}  @igg{fbicon.heart.sparkling} 

\begin{itemize} % {
\iusr{Ирина Петрова}
\textbf{Lessya Ours} да, поверьте, очень интересно вживую. Не могу вот сама прочесть некоторые надписи...
\end{itemize} % }

\iusr{Irina Popova}

век учись, век живи, а..., сколько раз там была, все бегом, бегом, а вот
благодаря вам приобщилась к культуре наших архитекторов и художников, граверов,
спасибо огромнейшее!

\begin{itemize} % {
\iusr{Ирина Петрова}
\textbf{Irina Popova} да, мы все спешим, бежим, такой наш ритм( некогда остановиться, осмотреться. Разве, что в молодости, когда вот так гуляли с другом... свидание оригинальное)
\end{itemize} % }

\iusr{Alla Sokolova Sokolova}
Интересно. Спасибр.

\iusr{Маргарита Марченко}
Действительно интересно)

\ifcmt
  ig https://scontent-frx5-2.xx.fbcdn.net/v/t39.1997-6/s168x128/93025159_222645582401279_8207007965157261312_n.png?_nc_cat=1&ccb=1-5&_nc_sid=ac3552&_nc_ohc=cx_4kk5_uMYAX92DbMp&_nc_ht=scontent-frx5-2.xx&oh=88f72e55deb184b0f8b8f338996a3bae&oe=6195CC82
  @width 0.1
\fi

\iusr{Алла Вайнерман}
Спасибо! Очень интересно! Я помню, когда рынок только строился. Но никогда не задумывались над этим панно.

\iusr{Максим Дербенев}
Спасибо. Рад что наша экскурсия сподвигла вас прогуляться по до боли вам знакомым местам!

\begin{itemize} % {
\iusr{Ирина Петрова}
\textbf{Максим Дербенев} 

Макс, это вам с Максом спасибо! Я бы сама так много и не вспомнила! А уж про
Софру и не знала!!! Почему-то не смогла кликабельно написать имена! До новых
встреч и в эфире и в жизни!  @igg{fbicon.heart.eyes}{repeat=3} 

\iusr{Максим Дербенев}
\textbf{Ирина Петрова} До встречи!
\end{itemize} % }

\iusr{Владислав Кучанский}
Неоднократно рассматривал панно, но только догадывался о истинном смысле сюжета. Спасибо. О лифте не знал.

\iusr{Ирина Петрова}
\textbf{Vladislav Kuchansky} да, это очень любопытное произведение! Рада, что многим нравится!

\iusr{Томуля Князева}

Обожаю Житний рынок. Я родилась на Воздвиженке. Мы с мамой ходили на этот
базар, но я никогда не обращала внимание. Большое спасибо, очень интересно

\iusr{Томуля Князева}

\ifcmt
  ig https://scontent-frt3-1.xx.fbcdn.net/v/t39.1997-6/s180x540/108554614_3714255988603572_399616359735863222_n.png?_nc_cat=107&ccb=1-5&_nc_sid=ac3552&_nc_ohc=IL-ytigN6mYAX-MxVtS&_nc_ht=scontent-frt3-1.xx&oh=b7f10592dbe0058f5b40896a16e0483e&oe=61941667
  @width 0.3
\fi

\iusr{Людмила Сидоренко}

Дякую за чудову розповідь. Але в наш час чомусь стає страшно, що можуть
провести декомунізацію (Ленінград, Дніпропетровськ) та демонтувати панно як в
готелі Турист☺

\iusr{Ирина Петрова}
\textbf{Людмила Сидоренко} надеемся, что такого не будет...

\iusr{Tatyana Shpak}
Спасибо большое. Очень познавательно!

\iusr{Татьяна Ларченко}
Спасибо всем за интересные рассказы и прекрасно - познавательную группу !

\iusr{Людмила Мозговая}
Но больше люблю Бесарабку...

\begin{itemize} % {
\iusr{Ирина Петрова}

Да, я, честно говоря, тоже. Родилась и выросла рядом с Бессарабкой, сколько мы
туда отходили))) Даже и сейчас люблю всё равно туда зайти, иногда просто
пройти, иногда купить творог и сало))) И даже выпить чашечку кофе.) Конечно,
всё очень изменилось, но, ведь плиточка квадратная всё ж еще есть))) Мне в
детстве она почему-то кукурузу напоминала)

\end{itemize} % }

\iusr{Ірина Литвиненко}
Цікаво дуже. Дякую  @igg{fbicon.face.smiling.eyes.smiling}  @igg{fbicon.hands.applause.yellow} 

\iusr{Ирина Николова}

Пізнавально, дякую, давайте ще, обожнюю Киів і він не перестає мене
дивувати, шалений темп, не вистачає часу, а тут історія просто перед очима:)

\begin{itemize} % {
\iusr{Ирина Петрова}
так, я теж дуже багато чого вперше почула із розповідей учасників групи. Маленьких нюансиків, штришків, особистих споминів, все знати неможливо, але хочеться)))
\end{itemize} % }

\iusr{Татьяна Гурьева}
замечательно. спасибо)

\begin{itemize} % {
\iusr{Ирина Петрова}
Танюша, спасибо! @igg{fbicon.heart.eyes} 

\iusr{Татьяна Гурьева}
\textbf{Ирина Петрова} никогда не задумывалась над сюжетом, а теперь пришло время

\iusr{Ирина Петрова}
\textbf{Татьяна Гурьева} 

знаю, что любишь Подол, вживую всё гораздо интереснее, панорамно. Не знаю,
правда ли, но, скоро, вроде, закроют на реконструкцию... не верю, что снимут,
но...


\iusr{Татьяна Гурьева}
\textbf{Ирина Петрова} все может быть, очень люблю его место)
\end{itemize} % }

\iusr{Светлана Лебедева}
Спасибо. Очень интересно. Никогда не задумывалась, что в этом панно так много смысла.

\iusr{Valeriy Levitskiy}

Ходил с бабушкой на Житний ещё до строительства крытого рынка. Рынок был с
обеих сторон от Вала

Ещё помню при открытии рынка, там при входе со стороны Хоревой стояли автоматы
для продажи бутербродов и пепси, что удивляло сильно На Валу помню ещё кинотеатр
Колос деревянный жёлтый

\iusr{Ирина Иванченко}

Как интересно, никогда не приглядывалась к этим элементам оформления, вечная
суета не даёт увидеть что-то неординарное рядом... Благодарю за рассказ.

\begin{itemize} % {
\iusr{Ирина Петрова}
\textbf{Ирина Иванченко} да, мы все бежим в своем городе. Когда куда-нибудь приезжаем - рассматриваем тщательнее, а тут - и время не хватает, и, вроде, да ещё успеем...

\iusr{Неман Иванов}
\textbf{Ирина Петрова}, точно!
Словно приморские жители, в своих летних хлопотах о туристах, забывающие о морских купаниях!

\iusr{Ирина Петрова}
\textbf{Неман Иванов} очень точное сравнение! Супер! Есть знакомые у моря, они за лето ни разу в море не зайдут. Смены аредаторов поглощает всё время @igg{fbicon.face.tears.of.joy} 
\end{itemize} % }

\iusr{Ирина Черныш}
Хочу внести уточнение. Автор панно Ирина Перова. Это абсолютно точно.

\begin{itemize} % {
\iusr{Ирина Петрова}
\textbf{Ирина Черныш} 

интересно... а есть первоисточник таких данных? Очень интересно было бы знать.
Авторство Анатолия Трофимовича Домнича описано в его биографии в Википедии.

\begin{itemize} % {
\iusr{Ирина Черныш}
\textbf{Ирина Петрова} 

дело в том, что Гугл выдает информацию, что автор панно Домнич. Но это не так.
Моя мама с 70 года работала в КиевЗНИИЭП вместе с Ириной Перовой. Ирина
работала над этим проектом, я тогда была юная, но хорошо это помню. Вчера, когда
прочла ваш пост изумилась, так как всю жизнь считала автором проекта Ирину
Перову. Рассказала маме и сегодня списалась с несколькими людьми, которые в
курсе тех давнишних событий. Исполнителем непосредственно был Александр Резуто.
А автор эскизов Ирина Перова. Заказ на оформление житнего рынка получил отдел
монументально- декларативного искусства при КиевЗНИИЭП. Прилагаю фото, которое
хранится у нас дома. Один из фрагментов панно. При желании можно найти больше.
Обещаю, что займусь этим обязательно.

\end{itemize} % }

\iusr{Ирина Черныш}

\ifcmt
  ig https://scontent-mxp1-1.xx.fbcdn.net/v/t1.6435-9/122015910_2728883290548728_2116606783546912354_n.jpg?_nc_cat=108&ccb=1-5&_nc_sid=dbeb18&_nc_ohc=Fv49Ds7Llh8AX-L4zFT&_nc_ht=scontent-mxp1-1.xx&oh=eb267b135e684e0827242bfa94fe9f30&oe=61B45FDA
  @width 0.4
\fi

\iusr{Ирина Петрова}
\textbf{Ирина Черныш} 

не могу спорить, но, тогда откуда взялось в этой истории имя Анатолия Домнича?
А Ирина Перова ещё жива? (извините, если вопрос не очень корректен).

\begin{itemize} % {
\iusr{Ирина Черныш}
\textbf{Ирина Петрова} Жива моя мама,слава богу! которая работала с Ириной бок о бок и видела как создавались эти эскизы.

\iusr{Ирина Петрова}
\textbf{Ирина Черныш} это отлично. Может, Ваша мама сможет объяснить появление имя Домнича?
\end{itemize} % }

\iusr{Ирина Черныш}
Не могу знать откуда взялось имя А. Домнича Про Ирину знаю, что жила в Германии. Жива ли ..не могу знать. Постараюсь навести справки.

\iusr{Ирина Петрова}

Честно говоря, этот эскиз, который Вы предоставили, и рисунок на панно мне
кажутся разными. Но, я не большой эксперт. А невозможна была ситуация, когда
проект готовили несколько авторов, а утвердили один? И это был проект Домнича?
Было бы очень интересно и правильно, на мой взгляд, узнать истину. Спасибо за
внимание к истории. Это ценно., @igg{fbicon.face.happy.two.hands} 

\iusr{Ирина Черныш}
То что вы видите, это один из эскизов. Таких зарисовок было много

\iusr{Ирина Петрова}
\textbf{Ирина Черныш} понятно)

\end{itemize} % }

\iusr{Александр Несытов}
Киев АртДеко, вещи которые хочешь трогать. и тепло от них

\iusr{Лида Баженова}
Спасибо за рассказ. Уникальное современное здание. Только бы не снесли как Сенной


\end{itemize} % }
