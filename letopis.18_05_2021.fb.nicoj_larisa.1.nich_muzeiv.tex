% vim: keymap=russian-jcukenwin
%%beginhead 
 
%%file 18_05_2021.fb.nicoj_larisa.1.nich_muzeiv
%%parent 18_05_2021
 
%%url https://www.facebook.com/larysa.nitsoi/posts/4308703885828377
 
%%author 
%%author_id 
%%author_url 
 
%%tags 
%%title 
 
%%endhead 
\subsection{Ніч музеїв - Піду нині в музей. Вночі}
\Purl{https://www.facebook.com/larysa.nitsoi/posts/4308703885828377}

НІЧ МУЗЕЇВ.

Піду нині в музей. Вночі. Це, як побачити зворотній бік місяця, подивитись
іншими очима на нашу історію. 

Там як? Заходиш - і перед тобою панорама \enquote{Подмосковні вечори}. Печера,
вогнище, гора кісток в кутку, коло неї голодний приручений пес. 

\ifcmt
  pic https://scontent-iad3-1.xx.fbcdn.net/v/t1.6435-9/187203285_4308703832495049_3888740948881531496_n.jpg?_nc_cat=102&ccb=1-3&_nc_sid=8bfeb9&_nc_ohc=2iZFmzHbr1MAX-Rx6EN&_nc_ht=scontent-iad3-1.xx&oh=ad5d8b6dc17fe31ae94bdc437dfec553&oe=60CDA937
\fi


Дивиться на ті кістки і думає: "Здичавілі зовсім. Нафіга я повівся на те імпортозаміщеніє? Був же свобідним вільним вовком..." 

В купі соломи спить баба з немовлятами. Нарожала.  А навколо вогню хуйлопітеки
з кремнієвими іскандерами і в шинелях із мамонта. Дивишся на то і, навіть,
шкода, шо вони не погодились із Дарвіном, не захотіли еволюціонувати й
залишились тупиковою гілкою людства.

А ще там поряд дядько тиснеться. Одяг на ньому потріпаний, личаки стоптані,
руки до крові здерті, а в брилику голодні кажани останню вошу доїдають. Сумний
дядько, бо не відав, що таке реклама, прийшов на вогонь у печері, казали, що
там добре, тепло, ніхрена робити не потрібно, і можна лаптями хлебтати, а бліни
з лопати жерти дають. А виявилось, що то плем'я охотніків і собіратєлєй: охочі
до дармового і збирають все, що погано лежить. От і сидить дядько, думу думає,
про те, що літав би соколом зараз, якби не курячими мізками думав та
хуйлопітекам не підгавкував. 

А потім ідеш у наступний зал. Тільки переступив поріг - а там та сама кам'яна доба, але в Трипіллі.

І там тобі: колесо, прялочка, станочок ткацький, кружечок гончарний. Та ще
полумиски, глечики та дзбаночки тиснені та різьблені. А узори на них - то
чисто, як на рушниках наших та по лиштві сорочечок. Йой! Яка то краса!

Заглядаєш краєм ока у попередню залю - і дякуєш Бозі, шо ти не моsкаLь і не той підгавкувач, шо пропадає коло них ні за цапову душу.

А потім ідеш далі. А там уже горно ковальське, міхи, щипці, молоти і молотки з
молоточками, величезний кадуб для води стоїть, що в ньому залізо студили, і
підкови на дровищі викладені та ще різна дрібнота, як то: наконечники для стріл
і списів, вуздечки і брязкальця, пряжки для ременів та застібки до всього.

А на стінах висять мечі та кольчуги, та броніки і шоломи високого ступеню
захисту. А що вже щити - то не намилуєшся! І на кожному сонечко відчеканене. 

І ото згадуєш ті \enquote{Подмосковниє вєчєра} в першому залі - і пишаєшся, шо предки в тебе не ликом шиті.

А далі вже заля з письмовими згадками на камені, бересті, пергаменті, папері. 

А там - зодчество : козацьке барокко, панське рококо, галицький ампір і модерн,
фортифікації і твердині, монастирі, палаци і, ясна річ, шо частний сектор з
його хатками біленькими, з тинами-перелазами та навколо хаток - паничі кручені,
соняхи на півнеба, а на обрії - то там і волошки в житі видно, ага.

І ти на ото все дивишся, козак Мамай тобі на кобзі грає, ти собі попиваєш
узварчик із мальованого кухлика, слухаєш, як хрущі над вишнями гудуть. Милота і
щастя. Бо є хліб і до хліба.

Восени сємки в Подмосков`ї на шубу соболину виміняєш, а там до Різдва й
кабанчик поспіє. Ні, кабанчика в експозиції нема. Але він мається на увазі, бо
кабанчик - то святе...

А за склом одяги і строї висять. Як в бутіку якомусь. Сорочки, льолі, запаски,
горбатки, юпки, свити, кацавейки, лапсердаки, кожушки , камізельки, кептарики.

Та ще хустки, очіпки, шапки смушеві і брилики різні. Та намисто, дукачі,
коралі, гердани, силянки і низавки...  І то всьо простий люд мав таке. 

Пани в платях шовкових ходили і корсетах- камзолах, ага. З хутрами того пса, шо
втік з-під купи кісток в "Підмосковних вечорах", шоб його не ззїли...

А потім, знаєте шо? Потім починається заля з тою епохою, котра нас зробила
сірими. Бо той дурнуватий дядько з "Підмосковних вечорів" відкрив браму
Батурина і привів хуйлопітеків в Україну. І вони зробили тут у нас темну ніч і
криваву пляму історії. 

Тупі форми, убогий зміст і жодної автентики. Мовби й не було в тих роках нічого
яскравого. Свого  рідного, українського. Окрім нашої крові. 

То екпозиція про тяжке наше пізнання, шо воно таке йоксель-моксель, йошкін кот і yobanyikatsap.

І лише в кінці - синьо-жовті фани, Щеневмерла і купа стріляних гільз, пробитих
касок, покрученого металу. То вже хлопці привезли з війни.

Директор музею вже переселив екскурсовода до бухгалтерки в кабінет і готує нову залю для нашої історії. Там будемо ми з вами. І те, що ми створимо. Руками, серцями і головами. Якщо вони в нас є.
Люди, вимкніть телевізори і йдіть до музею. Подивіться: що там в експозиції, послухайте екскурсовода, зробіть висновки. 
А далі - берімось до праці! Час творити історію. 
Поки її не продовжили писати  для нас хуйлопітеки із експозиції "Підмосковні вечори".
(с) Дзвінка Торохтушко
