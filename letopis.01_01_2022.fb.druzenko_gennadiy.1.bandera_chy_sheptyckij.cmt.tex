% vim: keymap=russian-jcukenwin
%%beginhead 
 
%%file 01_01_2022.fb.druzenko_gennadiy.1.bandera_chy_sheptyckij.cmt
%%parent 01_01_2022.fb.druzenko_gennadiy.1.bandera_chy_sheptyckij
 
%%url 
 
%%author_id 
%%date 
 
%%tags 
%%title 
 
%%endhead 
\zzSecCmt

\begin{itemize} % {
\iusr{Nikita Kimovish}
Дякую. Я підозрював, що саме так і було, проте не знав, що так трагічно.

\iusr{Slavko Polyatynchuk}
Генадію, я не знайшов у Шептицького жодної згадки про причетність Бандери до загибелі Бабія

\begin{itemize} % {
\iusr{Yaroslav Znych}
\textbf{Slavko Polyatynchuk} 

\enquote{Чи не найнапруженішим конфлікт між Шептицьким та молодими націоналістами був у
1934 році, коли вони вбили Івана Бабія, директора української академічної
гімназії. Шептицький різко їх засудив: @igg{fbicon.exclamation.mark} «[Я]кщо хочеться зрадливо забити тих,
які противляться вашій роботі, доведеться вам вбити усіх вчителів та
професорів, які працюють для української молоді, всіх батьків і матерів
українських дітей». Він нагадав їм, що «злочин є завжди злочин» і «що святій
справі не можна служити закривавленими руками» (цит. за Лисенко 2000: 38-39). В
українському суспільстві цю заяву митрополита сприйняли як надзвичайно сильне
засудження діяльності націоналістів (Паньківський 1965: 140).  @igg{fbicon.face.worried} У відповідь
ОУН, за мовчазної згоди іншого єпископа спробували усунути Шептицького з
престолу митрополита 4.}

\end{itemize} % }

\iusr{Богдан Панкевич}

Шептицький завжди засуджував злочини, особливо - вбивства. Вперше різко засудив
у 1908 році М.Січинського за вбивство намісника Потоцького. Що терором не можна
служити українській справі. Але часто голос Шептицького волав у пустелі
примітивізму і хибного розуміння патріотизму. Січинський на еміграції потім був
великим другом злочинного радянського режиму, показав своє нутро. Але це не
зупинило придурків, які декілька років тому повісили йому меморіальну таблицю
на будинку Львівської облради. На жаль Церква не популяризує величезну частину
спадщини Шептицького бо не може (чи не хоче) дієво слідувати його заповітам у
соціальній сфері, розбудові громадянського суспільства, економічному розвитку,
кооперації і т.д.

\begin{itemize} % {
\iusr{Gennadiy Druzenko}
\textbf{Богдан Панкевич} на жаль...

\iusr{Богдан Панкевич}
\textbf{Gennadiy Druzenko} 

А що тепер робити, коли термін "бандерівець" означає прихильника незалежної,
антимосковської України? Тому я, все знаючи і розуміючи - бандерівець.

\iusr{Gennadiy Druzenko}
\textbf{Богдан Панкевич} 

як казав великий українець Микола Зеров, ad fontes - повернутись до джерел.
Врешті-решт образ Бандери як головного борця з російщенням нав’язав нам Кремль.
І це спонукає до рефлексій.

\iusr{Nikita Kimovish}
\textbf{Богдан Панкевич} 

Да, но не забывайте, что большая часть украинцев относится к культу Бандеры
достаточно прохладно, однозначно негативно - приблизительно процентов сорок.
Что делать? Кстати, вы, надеюсь, в открытую войну верите? Если да, сообщите,
когда начинается в этом году или мне поездку на Крит планировать?

\iusr{Yaroslav Znych}
\textbf{Богдан Панкевич} можете бути націоналістом-патріотом @igg{fbicon.face.monocle} 

\iusr{Богдан Панкевич}
\textbf{Nikita Kimovish} 

Дуже багато українців знає власну історію у російсько-радянській брехливій
версії. І не лише українці - он президент Ізраїлю наговорив дурниць у нашій
Верховній Раді, їхні політики також у полоні старої радянської пропаганди. В
найкращих університетах світу донедавна вивчали історію Росії на основі
фальсифікату ще часів Катерини Другої. Потім так навчені студенти ставали
експертами і урядовцями. Тому ще не скоро світ і самі українці зрозуміють
правду і повірять у неї.

\iusr{Богдан Панкевич}
\textbf{Yaroslav Znych} 

У нас націоналізм означає те саме, що в інших країнах - патріотизм. Бо інші
країни не переживали такого неймовірного геноциду титульної нації.

\iusr{Nikita Kimovish}
\textbf{Богдан Панкевич} 

Пан Богдан. Вы же в университете учились. И в советское время жили, коммунистом
были, кстати? 1. Украина была столбом, причем подпорным Империи. \enquote{Родина
застоя} 70 процентов сержантско-старшинского состава РККА а после СА составляли
этнические украинцы. 40 процентов состава ВМФ - украинцы. Какой геноцид? Я уже
не говорю о культуре, о промышленности на Украине. Ваша позиция мне непонятна,
коль вы, судя по всему, образованный человек и свидетель былого расцвета
Украины. Ну, а если хотите жить в Лемберге.... Тут я вам не помощник.

\iusr{Богдан Панкевич}
\textbf{Nikita Kimovish} 

росіяни знищили у ХХ столітті близько 20 мільйонів українців. Що не завадило
іншим українцям стати опорою імперії. Так само як колись їхні предки були
економічною опорою Речі Посполитої. Зараз маємо свою державу і у ній має бути
своя правда. Не російська і не польська. Оскільки ми досить близькі культурно
до поляків та росіян то для утвердження власної ідентичності мусимо поставити
між собою і цими народами максимально високу стіну. Будь яка близькість до
росіян загрожує незалежності України. Росія розпочала війну тому, що не бачила
інших можливостей втримати Україну під своїм впливом.

\iusr{Yaroslav Znych}
\textbf{Nikita Kimovish} 

1) \enquote{...70 процентов РККА/СА, ...40 процентов состава ВМФ - украинцы.} Дозвольте
уточнити: були не українці, а укрАинцы, а частіше \enquote{хахлы}... А скільки \% було
серед старших офіцерів?! ⛔️ "...Анализ национального состава старших офицеров и
генералитета РККА демонстрирует, что чем выше было воинское звание, тем выше
был удельный вес русских.

Так, среди майоров русских числилось 66,83\%, среди подполковников — 68,86\%,
среди полковников — 73,47\%, а среди генералов в конце войны — 77,45\%.!?

2) культуру представляли Хор Верьовки і Ансамбль Вірського, а \enquote{настоящая}
культура - опера, балет були в Москві і Лєнінграді. 

3) промисловість з усім екологічним негативом була по всій УРСР і вишенька на
тортику - ЧАЕС поруч з Києвом...

\iusr{Богдан Панкевич}
\textbf{Nikita Kimovish} 

побачив, що Ви професор українського університету. І тут коментуєте російською
мовою. Поширюючи російську пропаганду. Це огидно. Припиняю дискусію.

\iusr{Yaroslav Znych}
\textbf{Богдан Панкевич} 

помиляєтеся, цей пан - професор колишнього \enquote{советского университета}...
@igg{fbicon.laugh.rolling.floor} 

\iusr{Nikita Kimovish}
\textbf{Богдан Панкевич} 

.....богданчик, ты уже старик, а уме как не было так и нет. Все равно ты ничего
не решаешь, \enquote{патриот} немощный. \textbf{Yaroslav Znych} славочка, я с ботами не
разговариваю. Принцип. Ты, кстати, мальчик или девочка, цветок запоздалый.


\end{itemize} % }

\end{itemize} % }
