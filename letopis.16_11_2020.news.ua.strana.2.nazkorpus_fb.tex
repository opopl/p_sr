% vim: keymap=russian-jcukenwin
%%beginhead 
 
%%file 16_11_2020.news.ua.strana.2.sputnik_V_peshko
%%parent 16_11_2020
 
%%url https://strana.ua/news/301374-natskorpus-v-facebook-publikuet-reklamnye-posty-nesmotrja-na-zapret-buzzfeed.html
%%author 
%%tags 
%%title 
 
%%endhead 

\subsection{Как Facebook зарабатывает на пропаганде \enquote{Нацкорпуса}, которую обещал блокировать}
\Purl{https://strana.ua/news/301374-natskorpus-v-facebook-publikuet-reklamnye-posty-nesmotrja-na-zapret-buzzfeed.html}

10:22, сегодня 

\ifcmt
pic https://strana.ua/img/article/3013/74_main.jpeg
caption Факельное шествие Нацкорпуса в Киеве
\fi

Несмотря на попытки убрать из Facebook пропаганду украинских ультраправых
групп, связанных с \enquote{Азовом} и \enquote{Нацкорпусом}, они продолжают вести свою
деятельность в соцсети. 

Об этом в колонке для издания Buzzfeed\Furl{https://www.buzzfeednews.com/article/christopherm51/neo-nazi-group-facebook} пишет журналист-расследователь Кристофер
Миллер. 

\enquote{Ультраправая группа, связанная с американскими сторонниками превосходства
белой расы, использует Facebook для вербовки новых членов, организации насилия
и распространения ультраправой идеологии по всему миру. Хотя более года назад
Facebook заблокировал на своих страницах движение \enquote{Азов} и его лидеров}, \dshM пишет Миллер.

Как утверждает журналист, Facebook продолжает получать прибыль от рекламы,
размещенной людьми, близкими к \enquote{Азову}. Последние рекламные посты выходили в
понедельник, 16 ноября. А с июля \enquote{Азов} открыл около десятка новых страниц в
Facebook.

Это делается не напрямую от организации, а с аккаунтов частных лиц. Так,
например, пользователь Алла Засядко разместила в ФБ 82 рекламных объявления за
3726 долларов.

Многие из этих объявлений призывали к уличным протестам против украинских
властей. Один из рекламных роликов призывает детей записываться на курсы
патриотической молодежи. \enquote{Подобные курсы включают обучение обращению с
огнестрельным оружием}, \dshM предупреждает Миллер.

Представитель Facebook заявил в комментарии BuzzFeed, что \enquote{Азов} по-прежнему
заблокирован в соцсети. \enquote{Мы удаляем контент, который хвалит или поддерживает
их, если мы о нем узнаём}, \dshM заявили в компании.

Однако на момент публикации материала страница \enquote{Национального корпуса} в
Facebook все еще была активна.

Издание напоминает, что в США Facebook критиковали за размещение страниц
ультраправых организаций в ходе выступлений BLM.

А в Украине, как уже детально рассказывала \enquote{Страна}, эту соцсеть модерируют
также симпатики правых. Детальнее об этом \dshM в материале Facebook и нацисты. Как
и почему цензоры соцсети блокируют критику ультраправых в Украине.\Furl{https://strana.ua/articles/rassledovania/276162-ukrainskie-tsenzory-facebook-blokirujut-kritiku-ultrapravykh-v-ukraine.html}

Как пишет Buzzfeed, Facebook заблокировал страницы \enquote{Азова} в апреле 2019 года.
Компания удалила несколько страниц, связанных с группой, в том числе те,
которыми управляют ее высокопоставленные члены. Но с 16 июля группа ведет новую
страницу \enquote{Национальный корпус}. Страница не пытается скрыть свою принадлежность
к \enquote{Азову} \dshM там открыто обсуждается деятельность и руководителей полка, есть
ссылки на веб-сайты и электронную почту \enquote{Азова}, а также размещаются фотографии
участников в форме на митингах и маршах с факелами.

\enquote{В Facebook не могут не знать, что движение \enquote{Азов} опасно. После серии жестоких
нападений на ромов и представителей ЛГБТ по всей Украине членами \enquote{Национального
корпуса} и его военизированного уличного крыла Госдепартамент США назвал
\enquote{Нацкорпус} \enquote{Азова} \enquote{националистической группой ненависти}}, \dshM напоминает
издание.

\enquote{В последние пару лет участники групп, связанных с Азовом, применяли насилие в
отношении уязвимых групп в украинском обществе и угрожали госслужащим, а
социальные сети служили важным инструментом для организации этих действий и
распространения их результатов}, \dshM сказал изданию Мэтью Шааф, который
возглавляет украинский офис правозащитной группы Freedom House.

 
