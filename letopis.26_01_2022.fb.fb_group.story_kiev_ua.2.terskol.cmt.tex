% vim: keymap=russian-jcukenwin
%%beginhead 
 
%%file 26_01_2022.fb.fb_group.story_kiev_ua.2.terskol.cmt
%%parent 26_01_2022.fb.fb_group.story_kiev_ua.2.terskol
 
%%url 
 
%%author_id 
%%date 
 
%%tags 
%%title 
 
%%endhead 
\zzSecCmt

\begin{itemize} % {
\iusr{Renata Miroshnichenko}

На эту лыжную базу в Терсколе мы ездили много лет! С друзьями :
летчиками-испытателями из Жуковского и Калининградского авиаотряда. Они меня
там ставили на лыжи @igg{fbicon.face.grinning.squinting} 

Одни из самых ярких воспоминаний жизни!)

\begin{itemize} % {
\iusr{Валентин Марченко}
\textbf{Renata Miroshnichenko} Я бы тоже поехал на какую-нибудь базу для воспитателей детсадиков или демонстраторов женской одежды дома моделей. Пусть ставят на лыжи...

\iusr{Renata Miroshnichenko}
\textbf{Валентин Марченко} мимо! Я сначала вышла замуж, а у мужа была уже традиция: \enquote{каждый год мы с друзьями...} @igg{fbicon.face.grinning.squinting} 

\iusr{Валентин Марченко}
Ну... не расстраивайтесь. Можно и с мужем, конечно. Раз уже есть..

\iusr{Renata Miroshnichenko}
\textbf{Валентин Марченко} \enquote{да если б у меня такой кот был, я б, может, и не женился!} @igg{fbicon.face.grinning.squinting} 
\end{itemize} % }

\iusr{Татьяна Фоменко}

Чегет, Терскол, Азау...Знакомые места. В 80-ые годы пристанище всех знаменитых
бардов, научной и творческой интеллигенции Москва и Питера. Очень известных
людей, знакомых только по \enquote{телевизору}, можно было встретить на склоне, в баре,
в гостинице... В номере под нами жил Юрий Визбор. Его книга Завтрак с видом на
Эльбрус - одна из любимых. Про песни молчу, их знают все. Последний раз была
там больше 20 лет назад. По многим причинам там теперь делать нечего. И, да,
публика совершенно другая

\begin{itemize} % {
\iusr{Елена Лаврова}
\textbf{Татьяна Фоменко} любимые места! В Азау прошел мой медовый месяц!!!

\iusr{Renata Miroshnichenko}
\textbf{Елена Лаврова} и мой)

\iusr{Татьяна Фоменко}
...а у нас свадебное путешествие было через перевал Донгузорун в Грузию!

\iusr{Аліса Забой}
\textbf{Татьяна Фоменко} Сванетію...

\iusr{Татьяна Фоменко}
\textbf{Аліса Забой} через Сванетию. А дальше Зугдиди, Тбилиси

\iusr{Renata Miroshnichenko}
\textbf{Татьяна Фоменко} именно так! Мы тоже последний раз были там лет двадцать назад

\iusr{Natasha Levitskaya}
\textbf{Татьяна Фоменко}

Я тоже была в Азау, Чегет, Терскол... Самые яркие воспоминания! Мне было 24
года... Теперь, я часто пересматриваю фильм \enquote{Завтрак с видом на Эльбрус}. Кафе
\enquote{Ай}, канатка...

Была летом и по леднику (наверху -4°, метель) поднимались на \enquote{Приют 11-ти}.


\iusr{Тина Валентина}
\textbf{Татьяна Фоменко} Даже Я была там, совсем не \enquote{горный} человек - романтика...

\iusr{Татьяна Фоменко}
\textbf{Тина Валентина} О, уже свои потянулись))

\end{itemize} % }

\iusr{Татьяна Черкасова}
Супер! Получила огромное удовольствие от чтения! Спасибо!

\iusr{Тамара Ар}
Молодец, Ваша мама! Чем сидеть без дела, так хоть горы увидели

\iusr{Римма Грановская}
Прекрасно пишите.

\iusr{Анатолий Золотушкин}
\textbf{Римма Грановская} спасибо

\iusr{Renata Miroshnichenko}
А ещё Терскол - это наше свадебное путешествие!

\iusr{Татьяна Жалнина}
84 года Высоцкий Владимир Семенович, 25.01.

\iusr{Людмила Новичкова}
Классно написали, просто и со вкусом ! СПАСИБО !

\iusr{Анатолий Золотушкин}
\textbf{Людмила Новичкова} спасибо вам

\iusr{Александр Лазаретник}
Лучше гор. Могут быть только горы. На которых еще не бывал.

\iusr{Renata Miroshnichenko}
У Чегета

\ifcmt
  ig https://scontent-frt3-1.xx.fbcdn.net/v/t39.30808-6/272834795_7007656992608898_5925701998088920026_n.jpg?_nc_cat=102&ccb=1-5&_nc_sid=dbeb18&_nc_ohc=VjfTde34Vy4AX_LdrYx&_nc_ht=scontent-frt3-1.xx&oh=00_AT9k5d6-lcAiO4FTAEnxPw-7FOA0HlixEXUrJmK0d1fbrQ&oe=61F8948B
  @width 0.4
\fi

\iusr{Renata Miroshnichenko}
Азау

\ifcmt
  ig https://scontent-frt3-1.xx.fbcdn.net/v/t39.30808-6/272831838_7007669452607652_2765674832622396187_n.jpg?_nc_cat=108&ccb=1-5&_nc_sid=dbeb18&_nc_ohc=L3nm2YAySk4AX8T_21e&_nc_ht=scontent-frt3-1.xx&oh=00_AT-3d2eCblh6onD6OuCkdAnhl3rhaQnBt2oWxyRjYZgfZw&oe=61F8200F
  @width 0.2
\fi

\iusr{Валентин Марченко}
\textbf{Renata Miroshnichenko} Канешна... Тут и без вина опьянеешь.. От такой красоты!..

\iusr{Валентина Белозуб}

Прекрасные воспоминания молодости связаны с Терсколом! Тогда студентов КПИ
танула в горы романтика и песни бардов! Порой бывало трудновато, особенно
неопытным девчонкам! Но мы шли вперёд и на привале все равно пели!  @igg{fbicon.person.raising.hand} 


\iusr{Olena Karelina}
\textbf{Валентина Белозуб} я також... там був ГЛОБУС... незабутня гірська база...

\iusr{Виктор Садовничий}

С Терсколом у меня и моей жены связаны самые прекрасные воспоминания. Февраль
1973год мы провели там свой медовый месяц. Лыжная база для военных называлась
так по одноимённому посёлку. Я, тогда молодой офицер предложил своей жене
провести там свой отпуск. Вспоминается много чего хорошего да и смешного, что
происходило с нами там. Вот один из случаев. Всей нашей группой мы решили
отпраздновать окончание отпуска в ресторане у подножья Эльбруса - \enquote{Азау}. В
конце застолья мы заказали две бутылки шампанского, не подозревая к чему это
приведёт. Было весело, и когда один из нас (бравых вояк) начал открывать первую
бутылку, то пробка вылетела как из пушки и содержимое вылетело вместе с ней,
обдав сидевших ароматным душем. Стало ещё веселее (водочку мы выпили под
шашлычёк) и тут (как зто всегда бывает) вызвался другой более опытный военспец
по открыванию шампанского. Он взял вторую бутылку, отошёл с нею в угол и начал
открывать. Пробка вылетела, струя ударила в стенку, наш военспец отпрянул от
стены, направив струю в сторону. Только в той стороне сидела такая же группа
туристов из Германии, подвыпивших немцев, держащихся под руки и распевающих
\enquote{Катюшу}. Они конечно не ожидали такого душа и их мужчины вскочили с угрозами.
Мы кое-как объяснили им, что этот конфуз связан с пониженным давлением в горах,
и что мы сами не ожидали такого. Подключилась администрация, купили ещё выпивки
и так урегулировали международный скандал.

Выставляю сохранившееся наше фото. Мы на горе Чегет, вдали Эльбрус.

Символично, что я пишу это в день нашей свадьбы - 26.01.1973.

(Путёвку брал в Киевском доме офицеров).

\ifcmt
  ig https://scontent-frt3-1.xx.fbcdn.net/v/t39.30808-6/272616982_489927252723604_2079658311322854149_n.jpg?_nc_cat=106&ccb=1-5&_nc_sid=dbeb18&_nc_ohc=WgHzzHzXN1AAX8ocMn4&_nc_ht=scontent-frt3-1.xx&oh=00_AT-l8xFEr_L8X77yntPmcS2Uc4WW8MeCIQab3lEevuWnwQ&oe=61F97529
  @width 0.4
\fi

\begin{itemize} % {
\iusr{Людмила Гнынюк}
\textbf{Виктор Садовничий} с годовщиной свадьбы Вас!!!

\ifcmt
  ig https://scontent-frt3-1.xx.fbcdn.net/v/t39.30808-6/272295942_1533576537018927_9112212027252039158_n.jpg?_nc_cat=106&ccb=1-5&_nc_sid=dbeb18&_nc_ohc=r_l812B51n0AX88hzFa&_nc_ht=scontent-frt3-1.xx&oh=00_AT_qJb0N3Ut23xCTqvHZS1Awo0a0wolrmHUxwh-jPyp6LQ&oe=61F87849
  @width 0.2
\fi

\iusr{Виктор Садовничий}
\textbf{Людмила Гнынюк} СПАСИБО!!!! @igg{fbicon.heart.beating} 

\end{itemize} % }

\iusr{Vadim Vadim}

Мой товарищ служил в Заполярье, по приходу домой спал на балконе, зимой. Мать
сходила с ума, но обошлось,, когда он женился)))

\iusr{Tanya Podkova}

Терскол, Теберда, Домбай- прекрасная юность! В Теберде меня пастух хотел купить
у инструктора за стадо баранов)))) мне было семнадцать...

\iusr{Elena Dubovenko}
\textbf{Tanya Podkova} о! И меня в Сванетии хотели купить за 1000 рублей и сколько-то баранов, мне было 14.

\iusr{Наталия Грищенко}
\textbf{Кабардино-Балкария} 

все начиналось с Нальчика. Дорога в Чегет к двухглавому Эльбрусу. А в
Терсколе, где любил ходить по горам Высоцкий, было нереально фантастично просто
компьютерная графика. Много было туристов из Питера и бардов из НИИ, Я помогала
им настраивать гитары так как я имею муз. образование. Воспоминания зашкаливают!

\iusr{Татьяна Домаскина}
Легко написано и с юмором @igg{fbicon.thumb.up.yellow} . Читала с удовольствием!

\iusr{Ирина Архипович}
Классный рассказ, особенно финал!!  @igg{fbicon.face.grinning.sweat}  @igg{fbicon.face.happy.two.hands} 

\iusr{Олег Стеля}

Места замечательные! В аспирантские годы был инструктором горного туризма.
Всегда спешили, на поесть шашлыки и попить вина времени не хватало, а жаль)

\iusr{Михайлина Голуб}

Спасибо большое за ваши публикации, уважаемый Анатолий. Они написаны мастерски,
всегда с самоиронией. Очень приятно их читать.

\iusr{Анатолий Золотушкин}
\textbf{Михайлина Голуб} вам спасибо

\iusr{Olga Step}

Мне знакомы походы: лыжный на Драгобрат(Карпаты), лодочный где-то в
Приморско-Ахтарске и по горам Кавказа. Незабываемо! Спиртное лилось рекой!
Спасибо за интересный рассказ, как всегда с юмором!

\iusr{Владимир Глинский}
Терскол - Приэльбрусье. Когда там был украинский альплагерь.

\iusr{Аліса Забой}
\textbf{Владимир Глинский} \enquote{Ельбрус}

\iusr{Александр Бубенев}
Юмор в каждой фразе. Юмор, главное лекарство от жизненных обострений.

\iusr{Анатолий Золотушкин}
\textbf{Александр Бубенев} спасибо @igg{fbicon.face.grinning.big.eyes} 

\iusr{Renata Miroshnichenko}
Эльбрус. Станция \enquote{Кругозор}

\ifcmt
  ig https://scontent-frt3-2.xx.fbcdn.net/v/t39.30808-6/272684643_7009921482382449_7857380753711159470_n.jpg?_nc_cat=101&ccb=1-5&_nc_sid=dbeb18&_nc_ohc=etESMxvosaEAX__fvkG&_nc_ht=scontent-frt3-2.xx&oh=00_AT-sdR-BKKiw7aBLby-rylFyX5ev5TRXquMpN55b7wU-8w&oe=61F8A38B
  @width 0.4
\fi

\iusr{Олена Медведева- Прицкер}

Спасибо ! Прекрасные времена в Приэльбрусье ! Хочу напомнить всем Приют-11 (
высота 4100 над уровнем моря ) ! Побывала там в 1968 году, кинофильм
\enquote{Вертикаль} и любимые песни Высоцкого позвали в горы советскую молодежь.

\begin{itemize} % {
\iusr{Renata Miroshnichenko}
\textbf{Олена Медведева- Прицкер} 

как раз в 1968-ом вступило в строй и первое высотное здание Приэльбрусья -
восьмиэтажная турбаза-гостиница Министерства обороны СССР - \enquote{Терскол}, о
которой и написал \textbf{Анатолий Золотушкин}

\iusr{Анатолий Золотушкин}
\textbf{Renata Miroshnichenko} когда я там был, высотного здания ещё не было

\iusr{Renata Miroshnichenko}
\textbf{Анатолий Золотушкин} 

это я сейчас в книжке прочитала \enquote{Эльбрусская летопись}, 1976 год издания, с
которой смахнуть пыль побудил ваш пост). Мы в начале 90-х именно там
останавливались

\ifcmt
  ig https://scontent-frx5-1.xx.fbcdn.net/v/t39.30808-6/272300103_7010656062308991_8245442863851436473_n.jpg?_nc_cat=105&ccb=1-5&_nc_sid=dbeb18&_nc_ohc=QzccQvvwf40AX9JDoq4&_nc_ht=scontent-frx5-1.xx&oh=00_AT-nyUxSQnlTuoA3iMScncqz3qB1bSSL_cD7Lidk-lcvmA&oe=61F965B4
  @width 0.2
\fi

\iusr{Анатолий Золотушкин}
\textbf{Renata Miroshnichenko} я жил в маленьком 1 этажном корпусе. Комнаты на 8 человек. На самом деле репрессии были после того, как мы послали прапорщика - коменданта этого здания

\iusr{Renata Miroshnichenko}
\textbf{Анатолий Золотушкин} ааа! А мы были крутые: ездили без путевок, наша принимающая сторона был начальник продсклада этой базы @igg{fbicon.wink} 
\end{itemize} % }

\iusr{Олена Шелест}
Класс!

\iusr{Тамара Кизя}

1973 год. Терскол, Иткол, приют 11. Путешествие было после схождения снежной
лавины. Что-то было трагическое в этом, однако отдыху не помешало. Были молоды и
счастливы.


\iusr{Татьяна Тихонова}
Замечательно, блеск!!!

\iusr{Анатолий Золотушкин}

спасибо

\iusr{Андрей Надиевец}

В далёком Заполярье радовались месяц, все выискивали, но вы традиционно ушли
проходными дворами, бросив в преследователей бутылку \enquote{Фанты}.


\iusr{Roman Sahm}

Терскол не внес в ряды раскол @igg{fbicon.grin}{repeat=3} 

\iusr{Анна Сидоренко}
Самокритично, однако...

\iusr{Александра Заворуева}
В Терсколе инструктором горнолыжного был Белиловский Владимир Львович, кто знает ? Что с ним?

\iusr{Анатолий Золотушкин}
\textbf{Александра Заворуева} не знал

\iusr{Михаил Рабинович}

От \enquote{Терскола} до \enquote{Чегета} была прекрасная длинная пологая лыжня для полного релакса после катания

\iusr{Саша Саша}
А пить зачем было?

\iusr{Юрий Шиян}

Как будто кто-то за меня написал эту заметку. 1980-й - инструктор по пешему
туризму - легендарный Захаров, обошли все вокруг, так как лыжи пообещали только
через дней 5. Зато - ресторан \enquote{Нарзан}, шашлыки по рублю, сухое вино еще
дешевле, пивбар в Азау и много других незабываемых впечатлений...

\iusr{Irina Burmistrova Baulina}
 @igg{fbicon.beaming.face.smiling.eyes}  @igg{fbicon.fist.raised} 

\iusr{Сергей Байдак}

Прочитав про спирт вспомнил автобазу \enquote{БАМСТРОЙПУТЬ} в г. Тында 1981- 1983 годов
где в автомобили всех марок с гидравлическим приводом тормозов заливали
смесь(50х50) глицерина и спирта иначе любая \enquote{тормозуха} уже в - 30 замерзала.
Смесь эту готовили в августе - сентябре, но к первым морозам ( в ноябре к
праздникам было уже за - 30 ) всё выпивалось \enquote{ударниками} строительства БАМа и
когда подходило время менять тормозную жидкость все принимавшие участие в
\enquote{дегустации} \enquote{волшебного} напитка дружно тянулись в ближайший магазин для
покупки пищевого спирта за 9 руб. пол-литра взамен выпитого. Кстати вкус у
напитка напоминал очень крепкий ликёр. Да, в магазинах из ликёроводочных
изделий преобладали спирт и болгарское сухое вино \enquote{Медвежья кровь} которое на
мой вкус гадость. @igg{fbicon.face.grinning.squinting} 

\iusr{Анатолий Золотушкин}

С глицерином не пробовал @igg{fbicon.grin} 

\end{itemize} % }
