% vim: keymap=russian-jcukenwin
%%beginhead 
 
%%file 16_01_2022.stz.news.ua.hvylya.1.anatomia_vraga.6.effektivnost_stratagemy
%%parent 16_01_2022.stz.news.ua.hvylya.1.anatomia_vraga
 
%%url 
 
%%author_id 
%%date 
 
%%tags 
%%title 
 
%%endhead 

\subsubsection{Эффективность стратегемы выживания}

Эффективность государственной стратагемы в России, во многом, определяется
\textbf{степенью жестокости} властной вертикали. Все периоды, когда в Кремле появлялись
условные \enquote{либералы} такие, как Борис Годунов, самозванец Гришка Отрепьев, Петр
III, Павел I, Александр II, Николай II, Хрущев или Горбачев, приводили к
падению легитимности вертикали и, как следствие - к смуте или \enquote{перестройке},
угрожающей мировоззренческой дезорганизацией, территориальными и ресурсными
потерями.

Смута или \enquote{оттепель} в России - это постоянно невыученный урок истории, после
которого в Кремль, отражая коллективное бессознательное и отвергая
\enquote{классический} вариант развития государства, как создание комфортных условий
жизни для подданных, всегда возвращался \textbf{смиритель смуты}, топором, лагерной
баландой и пулей укрепляющий государственную вертикаль, направляющий ее по
\enquote{иррациональной}, но традиционной стратегеме развития \enquote{защиты от хаоса}. То
есть в русло экспансии.
