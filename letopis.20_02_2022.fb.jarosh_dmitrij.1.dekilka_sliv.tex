% vim: keymap=russian-jcukenwin
%%beginhead 
 
%%file 20_02_2022.fb.jarosh_dmitrij.1.dekilka_sliv
%%parent 20_02_2022
 
%%url https://www.facebook.com/dyastrub/posts/4919794521430726
 
%%author_id jarosh_dmitrij
%%date 
 
%%tags napadenie,rossia,ugroza,ukraina
%%title Декілька слів, виходячи із ситуації
 
%%endhead 
 
\subsection{Декілька слів, виходячи із ситуації}
\label{sec:20_02_2022.fb.jarosh_dmitrij.1.dekilka_sliv}
 
\Purl{https://www.facebook.com/dyastrub/posts/4919794521430726}
\ifcmt
 author_begin
   author_id jarosh_dmitrij
 author_end
\fi

Декілька слів, виходячи із ситуації.

У цей час Україні потрібна єдність Нації та максимальна мобілізація духовних,
матеріальних, людських ресурсів для відбиття агресії Ерефії.

Єдність - не тільки у декларативних заявах політиків, але й у конкретних
державо-захисних діях.

Враховуючи існуючі загрози, вже зараз необхідно розпочати:

-  мобілізацію, як мінімум, першої черги Оперативного резерву ЗСУ; 

- доукомплектування бойових бригад Армії; 

- розгортання, озброєння, забезпечення всім необхідним та бойове злагодження
добровольчих підрозділів (які будуть діяти, як завжди, у цілковитому
підпорядкуванні Командування ЗСУ).

- системну зачистку українських міст та сіл від колаборантської наволочі тощо.

Окрім цього, керівництву Держави необхідно звернутися із закликом до громадян
України приготуватися до ведення тотальної війни проти віковічного ворога -
Російської імперії: Украінці мають знищувати агресора скрізь, де той з’явиться,
всім, що є під рукою. 

У клятих ерефівців має і буде горіти земля під ногами!

Завдання кожного українця - вбити якомога більше окупантів…

Брати і Сестри, будьмо готові до жертовної Битви за майбутнє нашої Нації та
Держави!

За наших дітей і онуків!

За домашні огнища і вівтарі!

P.S. Українська добровольча армія перебуває у стані повної бойової готовності:
фронтові підрозділи виконують поставлені завдання в зоні ООС; резервні
батальйони та новосформовані підрозділи готові до розгортання; медбат
«Госпітальєри» працює. 

Не бійся! БИЙСЯ! @igg{fbicon.flag.ukraina}{repeat=3}
