% vim: keymap=russian-jcukenwin
%%beginhead 
 
%%file 19_07_2020.fb.lnr.6
%%parent 19_07_2020
 
%%endhead 
\subsection{Дуда с конфедератами хотят забрать Галичину --- львовский экономист}
\label{sec:19_07_2020.fb.lnr.6}
\url{https://www.facebook.com/groups/LNRGUMO/permalink/2859058107539078/}
  
Для победы на президентских выборах в Польше Анджею Дуде потребовалась
поддержка партии «Конфедерация», которая выступает за расчленение Украины и
присоединение к Польше территорий Галиции.

Об этом в эфире «NTA» заявил львовский экономист Юрий Туранский, передаёт
корреспондент «ПолитНавигатора».

«На президентский выборах в Польше немного более половины проголосовали за
Анджея Дуду. Для Украины судьбоносно, что за несколько дней у Анджея Дуды
состоялась встреча с президентом США Дональдом Трампом.

Но тревожным для Украины является момент, что как никогда большой процент на
выборах набрала радикальная партия «Конфедерация», они набрали 6,8\%.

Перед вторым туром президентских выборов для того, чтобы Дуде набрать больше
голосов, происходили переговоры между представителями и электоральными
симпатягами «Конфедерации» и Дудой, там звучали такие слова про Украину как
Малопольша», --- тревожится Туранский.

«Поэтому нам нужно выстраивать чёткую дипломатическую линию отношений с Польшей
и, собственно, отстаивать свою линию. Даже если это не является партией
большинства, и она не является выразителем воли в Польше, но она влияет на
мнение людей.

Очень важно для Украины со своими соседями выставлять флажки, за которые нельзя
заходить», --- переживает львовский экономист.

Тарас Стрельцов 
