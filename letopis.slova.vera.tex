% vim: keymap=russian-jcukenwin
%%beginhead 
 
%%file slova.vera
%%parent slova
 
%%url 
 
%%author_id 
%%date 
 
%%tags 
%%title 
 
%%endhead 
\chapter{Вера}

%%%cit
%%%cit_head
%%%cit_pic
%%%cit_text
Я не случайно обратился к проблеме религиозной. \emph{Вера} — самое святое, что есть у
человека, тем более, если речь идёт о вере в бессмертную душу и её спасение.
Если \emph{вера} тысячелетиями проповедует мир и бесконфликтное взаимодействие, её
адепты должны были бы составлять потрясающе цельное общество. Но всё происходит
с точностью до наоборот. Не осознав ведущий характер и силу Слова, не вникая в
его суть, а останавливаясь на внешней форме, люди лишь углубляют своё
разделение, вопреки воле Господа, которому молятся
%%%cit_comment
%%%cit_title
\citTitle{О Слове, Боге и разделённом человечестве и Крошке Еноте}, 
Ростислав Ищенко, ukraina.ru, 30.10.2021
%%%endcit
