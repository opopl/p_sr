% vim: keymap=russian-jcukenwin
%%beginhead 
 
%%file 23_08_2021.fb.kosachev_konstantin.1.pakt_ribbentropa_molotova
%%parent 23_08_2021
 
%%url https://www.facebook.com/permalink.php?story_fbid=4233161343431262&id=100002123135703
 
%%author 
%%author_id kosachev_konstantin
%%author_url 
 
%%tags istoria,kiev,krymskaja_platforma,pakt.ribbentrop.molotov,rossia,ukraina
%%title 23 августа 1939 года был подписан Договор о ненападении между Советским Союзом и Германией
 
%%endhead 
 
\subsection{23 августа 1939 года был подписан Договор о ненападении между Советским Союзом и Германией}
\label{sec:23_08_2021.fb.kosachev_konstantin.1.pakt_ribbentropa_molotova}
 
\Purl{https://www.facebook.com/permalink.php?story_fbid=4233161343431262&id=100002123135703}
\ifcmt
 author_begin
   author_id kosachev_konstantin
 author_end
\fi

Сегодня одна из мрачных дат в истории Европы, хотя и далеко не единственная. 23
августа 1939 года был подписан Договор о ненападении между Советским Союзом и
Германией, из-за секретных протоколов к нему известный как пакт
Молотова-Риббентропа.

Подозреваю, не случайно именно в этот день (хотя формально повод иной)
украинские власти собрали в Киеве так называемую «Крымскую платформу»,
незатейливо пытаясь провести  исторические параллели. Но если они, эти
отсылающие к диктаторскому произволу параллели и уместны в отношении Крыма, то
применительно не к 2014-му, а к 1954-му  году, когда была на тяжелые 60 лет
определена его непростая и несправедливая украинская доля. 

\ifcmt
  tab_begin cols=3
  width 0.2

     pic https://scontent-cdg2-1.xx.fbcdn.net/v/t1.6435-9/237608274_4233160403431356_6130234087703011871_n.jpg?_nc_cat=108&_nc_rgb565=1&ccb=1-5&_nc_sid=8bfeb9&_nc_ohc=ER6mxj2kbC8AX8gykYI&_nc_ht=scontent-cdg2-1.xx&oh=32e0f0f7a9fd47d1c518739cd7545bf5&oe=614B124E

     pic https://scontent-cdt1-1.xx.fbcdn.net/v/t1.6435-9/237799297_4233161233431273_5135105340196966942_n.jpg?_nc_cat=103&_nc_rgb565=1&ccb=1-5&_nc_sid=8bfeb9&_nc_ohc=QClGVCKEfLoAX_cKpkL&_nc_ht=scontent-cdt1-1.xx&oh=ac432fd408cd6bf6d35c59ccdb1f9871&oe=614C3C19

     pic https://scontent-cdg2-1.xx.fbcdn.net/v/t1.6435-9/236580844_4233160450098018_6938155622957179044_n.jpg?_nc_cat=111&_nc_rgb565=1&ccb=1-5&_nc_sid=8bfeb9&_nc_ohc=pgSwUo5ENk8AX8qlCsE&_nc_ht=scontent-cdg2-1.xx&oh=a1ab348dcd35cc8968d283201122bd44&oe=614BF95E
  width 0.3

  tab_end
\fi

Ведь тогдашнее решение Хрущева передать Крым из РСФСР в УССР на самом деле было
чисто сталинским: Советский Союз в 1954 году думал и жил на 100\% по сталинским
лекалам, до разоблачения культа личности оставалось еще два года, и к народам
по-прежнему относились как к мешкам с картошкой. И решение 1954 года, принятое
без спроса крымчан, было столь же тоталитарным, как и договоренности Сталина с
Гитлером за 15 лет до того по Польше и Прибалтике. В этом смысле сегодняшняя
«крымская платформа», как и вся политика нынешних киевских властей по крымскому
вопросу - это безуспешная и бесперспективная, но оттого не менее циничная и
неправовая попытка вернуть к жизни решения эпохи сталинского тоталитаризма,
презиравшего людей и тусовавшего народы во имя торжества «идеи». 

Но 23 августа - это еще и очень светлая, славная дата в советской истории, в
нашей общей истории с Украиной. В этот день в 1943 году был окончательно
освобожден Харьков. И именно этот день харьковчане отмечают как день города.
Поздравляю!

И именно сегодня, наряду со многими другими событиями, произошло самое главное
лично для меня. Сегодня в селе Чемужовка Змиёвского района Харьковской области
на братской могиле времен страшных битв за освобождение города были установлены
еще две мемориальные плиты, на которых - 127 имен героев Великой Отечественной.
И среди них - родной брат моей бабушки, мой двоюродный дед Карп Фасович
Санников, младший сержант, боец 152-й стрелковой дивизии, погибший в свои 37
лет в боях за Харьков 18 августа 1943 года.

По паспорту общее число захороненных в этой могиле - 420 человек, из которых
поименно до сих пор были указаны только 168. Искренние слова благодарности
районному Военно-патриотическому обществу «Ориентир» и харьковскому отделению
партии «Оппозиционная платформа - За Жизнь». Без них  добавить на плиты ещё 127
имен бойцов было бы просто невозможно (23 из которых, кстати, призывались с
территории современной Украины). Помогали и российские военные архивисты. Жаль,
что не совместно с коллегами с Украины - результат мог бы быть еще более
впечатляющим. Но и то, что удалось дополнительно увековечить 127 имен героев -
священное дело. И оно сделано, что особенно важно именно в эти непростые
времена. Времена обязательно пройдут. И мы обязательно сделаем больше. Вместе.
Россия и Украина. Все народы, победившие нацизм. 

Возвращаясь к личному. Меня, внука, тоже приглашали на церемонию открытия
нового мемориала. Несмотря на понятные риски, связанные с посещением
современной Украины российскими политиками, горел желанием приехать. Еще 6
августа наши дипломаты запросили, как положено, нотой украинские власти. При
этом было особо подчеркнуто, что моя поездка будет носить исключительно частный
и гуманитарный характер - только посещение братской могилы деда. Ответ пришел
17 августа, в нем - грубый отказ со ссылкой на бессрочный запрет в части, меня
касающейся, на въезд в Украину (о чем я, кстати, не знал). Далее, гадливо
кривляясь, украинский МИД предлагает снять упомянутый запрет на условиях
«прекращения российской агрессии», «деоккупации Крыма, Севастополя и
Юго-Востока», а также «компенсации причиненных убытков».

\ifcmt
  tab_begin cols=2

     pic https://scontent-cdg2-1.xx.fbcdn.net/v/t1.6435-9/237569896_4233160986764631_6148189325633723036_n.jpg?_nc_cat=100&_nc_rgb565=1&ccb=1-5&_nc_sid=8bfeb9&_nc_ohc=oFGUIuwI39QAX98Z0Ox&_nc_ht=scontent-cdg2-1.xx&oh=efce53cccb736ffa08976ee09e2d33c8&oe=614B25C2

     pic https://scontent-cdg2-1.xx.fbcdn.net/v/t1.6435-9/235987837_4233160583431338_517827301222492572_n.jpg?_nc_cat=102&_nc_rgb565=1&ccb=1-5&_nc_sid=8bfeb9&_nc_ohc=3T-eabGo6pcAX8tiQ_p&_nc_ht=scontent-cdg2-1.xx&oh=7299085d98476a903699daba76a29c11&oe=614DA072

  tab_end
\fi

Ну что тут скажешь… Мерзко от ощущения, что в нынешней украинской власти засели
деятели без совести и без памяти, без души и без сердца. Обнадеживает, что в
украинском обществе, в данном случае - в Харьковской области, людей с сердцем и
с душой, сохраняющих память и живущих по совести, по-прежнему много. Всем вам -
огромное спасибо! Особенно и отдельно - моему другу, настоящему патриоту своей
родины Андрей Лесик!

\ifcmt
  tab_begin cols=3

     pic https://scontent-cdg2-1.xx.fbcdn.net/v/t1.6435-9/237799288_4233161183431278_8691359508299150299_n.jpg?_nc_cat=102&ccb=1-5&_nc_sid=8bfeb9&_nc_ohc=JzwBtoJBHTQAX_d8vjL&tn=lCYVFeHcTIAFcAzi&_nc_ht=scontent-cdg2-1.xx&oh=fd7cb08853cd5729e53fd84ffd9c981a&oe=614D53A9

     pic https://scontent-cdt1-1.xx.fbcdn.net/v/t1.6435-9/238126469_4233161130097950_3783982174650661794_n.jpg?_nc_cat=101&_nc_rgb565=1&ccb=1-5&_nc_sid=8bfeb9&_nc_ohc=_cKYMSzsZxAAX9hmjnO&_nc_ht=scontent-cdt1-1.xx&oh=1e86991a661f89991d47e6bec6bc68ef&oe=614A56BC

     pic https://scontent-cdt1-1.xx.fbcdn.net/v/t1.6435-9/237681263_4233160656764664_4245290533461809060_n.jpg?_nc_cat=106&_nc_rgb565=1&ccb=1-5&_nc_sid=8bfeb9&_nc_ohc=iFdSeBDp2PAAX-GvKTu&_nc_ht=scontent-cdt1-1.xx&oh=29afcbcff9d0ff0cc917e5d563f2952e&oe=614A69D3

  tab_end
\fi

Мы обязательно приедем на могилу Карпа Санникова - его прямые потомки, живущие
в Удмуртии, и вся моя семья, вместе с которой в прошлом году, на 75-летие
Победы, мы так же открывали в подмосковном селе Спас-Заулок обновленный
мемориал с 285 новыми именами, включая Евстафия Фасовича Санникова, погибшего в
боях за Москву в декабре 1941-го. 

Два брата Санниковых. И еще сын одного из них, Мирон Санников, погибший в августе 1942-го под Сталинградом.

Санников - Москва, 1941. 

Санников - Сталинград, 1942. 

Санников - Харьков, 1943. 

285 в прошлом году и 127 - сейчас. Мой «Бессмертный батальон». 

Благодаря им и таким, как они, мы все сегодня и живем. В Москве и в Харькове, в России и на Украине. Везде.

Низкий поклон и вечная память! Вы с нами. Мы вместе. Что бы ни думали по этому поводу нынешние киевские власти. 

А мой Санников - вот он, последний в первом ряду… И опять на линии фронта.

\begin{verbatim}
  #ВеликаяОтечественнаяВойна #Харьков #БратскиеМогилы
\end{verbatim}

\ifcmt
  tab_begin cols=2

     pic https://scontent-cdg2-1.xx.fbcdn.net/v/t1.6435-9/237948166_4233160846764645_7096952403971343914_n.jpg?_nc_cat=107&_nc_rgb565=1&ccb=1-5&_nc_sid=8bfeb9&_nc_ohc=0KsosT0_kPMAX_z4MQw&_nc_ht=scontent-cdg2-1.xx&oh=9af34e80dfc026af36d5140c813596a5&oe=614A58B5
     width 0.27

     pic https://scontent-cdt1-1.xx.fbcdn.net/v/t1.6435-9/236438641_4233160936764636_2094070392082819139_n.jpg?_nc_cat=109&_nc_rgb565=1&ccb=1-5&_nc_sid=8bfeb9&_nc_ohc=W-95l1Pr5JwAX9ZFhg2&_nc_ht=scontent-cdt1-1.xx&oh=7a7efe9d1827226d8ed52e228607e532&oe=614D4DFA

  tab_end
\fi

