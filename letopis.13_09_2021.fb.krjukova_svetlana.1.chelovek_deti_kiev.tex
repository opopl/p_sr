% vim: keymap=russian-jcukenwin
%%beginhead 
 
%%file 13_09_2021.fb.krjukova_svetlana.1.chelovek_deti_kiev
%%parent 13_09_2021
 
%%url https://www.facebook.com/kryukova/posts/10159658805588064
 
%%author_id krjukova_svetlana
%%date 
 
%%tags chelovek,deti,golubi,gorod,kiev,obschestvo,ukraina
%%title Человек, даже самый одичавший, всегда ищет общества
 
%%endhead 
 
\subsection{Человек, даже самый одичавший, всегда ищет общества}
\label{sec:13_09_2021.fb.krjukova_svetlana.1.chelovek_deti_kiev}
 
\Purl{https://www.facebook.com/kryukova/posts/10159658805588064}
\ifcmt
 author_begin
   author_id krjukova_svetlana
 author_end
\fi

- Школа это детский ад! - жалуется старшая сестра младшим сестрам и братьям.

-Утренний Киев ожил так же, как и заснул, быстро и незаметно. Проголодавшиеся
воробьи и голуби как пернатые крысы кружат вокруг лавок одного из людных кафе у
 Николаевского Костела. Тут киевляне останавливаются на 10 минут, чтобы
задержать утро, сделать парочку наставлений детям перед уроками, набраться
храбрости, пошептать в чашку с кофе свою молитву и начать день. 

Человек, даже самый одичавший, всегда ищет общества. Мы выползаем по любому
поводу из своих дворцов и скворечников и направляемся в людные места,
усаживаемся, согреваемся тёплым и разглядываем друг друга, подслушиваем,
присматриваемся, переглядываемся. Большие города дарят им такую роскошь -
оставаться одинокими среди людей. Быть вместе, но молчать. 

Троллейбус притормаживает на светофоре. И все гости уличного кафе прямо у
дороги оборачиваются вправо. Пассажиров укачало и они полусонные и вялые как
мухи на летней кухне чуть приоткрыли глаза, чтобы рассмотреть сидящих под
шезлонгами. Близость расстояния, 10 секунд встречи, бесцеремонные взгляды в
упор - тот недолгий и редкий случай, когда такие прицелы позволены - не
догонишь.

Девушка в чёрной кепке с чёрным чаем сидит и хохочет над книжкой Ильфа и
Петрова, растроганная сатирой до слез, как будто она проводит это утро не с
книгой, а с закадычными друзьями, которые бессовестно щекочут ее словами.

- Ооооо, до каникул осталось ещё больше месяца! Вот бы ввели карантин, -
продолжает старшеклассница, умышленно пугая младших предстоящим тяжелым годом.
Там все очень и очень страшно! Тут учителя с дубинками, а в столовке кормят
котлетами из гвоздей! А в туалетах дырки вместо унитазов, а оттуда выглядывает
МОМО! - сказала старшеклассница и скорчила для младших форменную гримасу ужаса.
Те захихикали, подавились от смеха сдобными булками и совершенно не напугались.

- А почему у тебя такой огромный рюкзак? - обращается она к брату, когда дети
выскакивают из-за стола, чтобы не опоздать на первый урок. 

- Я все время что-то забываю. То тетрадку, то книжку. И на меня орет училка. И
чтоб она не орала, я решил носить все! 

Они опаздывают и убегают.

Отец двоих красивых дочерей медленно допивает стакан горячего молока и крошит
на площадную плитку белую булку. Дочки ласкаются около плеча рослого парня как
прирученные канарейки, а птицы прокармливаются как неприручённый, пугливо
отскакивая от каждого движения кормящих рук. Одинокие девушки с горячим кофе,
задумчивые и картинно загадочные, глазеют на детей со скукой, и на молодых
отцов - с тоской. 

Батон свежий и мягкий, поэтому птицам достаются большие куски. Они налетают со
всех сторон и уже через две минуты все гости смотрят на шумную голубятню, как
на огонь и реку, бездумно и бесконечно. Воробьи и голуби дерутся, жадничают,
воркочут себе под клюв, отхватывают большие ломти и улетают куда подальше. Вот
так и с людьми, одни отдают себя по кускам, другие отхватывают. А когда крошишь
мелкими кошками, собираются стаями, просятся на ладони, чтобы получить ещё и
ещё.

\ii{13_09_2021.fb.krjukova_svetlana.1.chelovek_deti_kiev.cmt}


