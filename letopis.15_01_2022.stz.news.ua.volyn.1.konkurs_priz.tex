% vim: keymap=russian-jcukenwin
%%beginhead 
 
%%file 15_01_2022.stz.news.ua.volyn.1.konkurs_priz
%%parent 15_01_2022
 
%%url https://www.volyn.com.ua/news/203275-10-richna-volynianka-pryzerka-mizhnarodnoho-konkursu-mystetstv-foto
 
%%author_id news.ua.volyn
%%date 
 
%%tags deti,konkurs,ukraina
%%title 10-річна волинянка – призерка Міжнародного конкурсу мистецтв (Фото)
 
%%endhead 
\subsection{10-річна волинянка – призерка Міжнародного конкурсу мистецтв (Фото)}
\label{sec:15_01_2022.stz.news.ua.volyn.1.konkurs_priz}
 
\Purl{https://www.volyn.com.ua/news/203275-10-richna-volynianka-pryzerka-mizhnarodnoho-konkursu-mystetstv-foto}
\ifcmt
 author_begin
   author_id news.ua.volyn
 author_end
\fi

\begin{zznagolos}
Наприкінці грудня у Львові відбувся Міжнародний фестиваль-конкурс мистецтв
«Happy new year-2022». Друге місце у престижному мистецькому змаганні виборола
10-річна жителька села Гірка Полонка Мілана Павлюк, викладач – Марія Лавренюк	
\end{zznagolos}

За звання кращих виконавців у номінації «Вокальне мистецтво» молодшої дитячої
вікової категорії змагалися 65 учасників. Мілана Павлюк підкорила серця журі
виконанням пісні «Нічка-новорічка», увійшовши до трійки призерів.

\ii{15_01_2022.stz.news.ua.volyn.1.konkurs_priz.pic.1}

Це вже не перша перемога школярки на конкурсі такого рівня. Незважаючи на
доволі юний вік, дівчина є кількаразовою призеркою міжнародних вокальних
конкурсів та фестивалів, фіналісткою творчого проекту «СВОЇ».

\begin{zznagolos}
\enquote{Зараз матиму трохи часу для відпочинку і знову починаю підготовку до нових
змагань.}
\end{zznagolos}

«Зазвичай я не дуже хвилююся перед виступами, але цього разу, виходячи на сцену
переживала, – розповідає Мілана. – Фестиваль «Happy new year» – це серйозний
конкурс, тому хотілося показати хороший результат і мені це вдалося. Зараз
матиму трохи часу для відпочинку і знову починаю підготовку до нових змагань».

\ii{15_01_2022.stz.news.ua.volyn.1.konkurs_priz.pic.2}

13 січня Мілана святкувала 10-річчя. Щиро вітаємо юну виконавицю з ювілеєм.
Бажаємо творчих успіхів, натхнення та нових злетів на мистецькому олімпі! Про
це йдеться на сайті Боратинської громади.
