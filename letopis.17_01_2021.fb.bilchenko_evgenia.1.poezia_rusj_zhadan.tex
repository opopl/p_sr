% vim: keymap=russian-jcukenwin
%%beginhead 
 
%%file 17_01_2021.fb.bilchenko_evgenia.1.poezia_rusj_zhadan
%%parent 17_01_2021
 
%%url https://www.facebook.com/yevzhik/posts/3561830740518677
 
%%author 
%%author_id 
%%author_url 
 
%%tags 
%%title 
 
%%endhead 

\subsection{БЖ. Мои учителя в поэзии}
\Purl{https://www.facebook.com/yevzhik/posts/3561830740518677}

\ifcmt
  pic https://scontent-bos3-1.xx.fbcdn.net/v/t1.6435-9/r270/139982311_3561817947186623_612325693745962347_n.jpg?_nc_cat=106&ccb=1-3&_nc_sid=730e14&_nc_ohc=LlJWKkAOpaYAX-5tDpZ&_nc_oc=AQmbv_hkDkHiT1wTqAYS1KPVlySmrTe6fkVqTGnjtFhSz20GkT3wrlLGYMnuRMRNhts&_nc_ht=scontent-bos3-1.xx&_nc_tp=31&oh=fa8d9439daa735c78e1f693d6f4c5e01&oe=60B47C1F
\fi


Наверное, настала пора по болезни подбить кое-какие итоги своей творческой деятельности. Недавно - и это был удар - в "Новом литературном обозрении" за 2020 г. (вроде бы серьезном российском издании, нет?) из меня опять попытались сделать чучело "поэта майдана" и проанализировать текст "Я - мальчик". На этом фоне параллельное упоминание меня Дмитрием Быковым в числе интересных ему русскоязычных поэтов Украины меня добило. 
Можно я назову три причины, почему нельзя больше говорить о тексте "Я - мальчик"?
1. Потому что я - не поэт одного стихотворения. Я не шла к и не шла от.
2. Потому что мои взгляды поменялись. 
3. Потому что текст - все-таки хороший, и я намерена отделить его нравственное ядро от его трактовок. Я сейчас прочла этот текст губами от имени Донбасса: оно получилось идеально. 
Можно я назову три причины, почему я - не "русскоязычный поэт Украины"? 
1. Я не встаю читать стихи со словами: "Извините, я буду на русском".
2. Я не говорю: "Я понимаю все уродство современной российской политики, но, пожалуйста (просительно) оставьте мне мою русскую культуру, пожалуйста".
3. Я - не лузер. 
А теперь о сокровенном. Я очень люблю лирику Жадана. Вот до боли и посинения. Мне очень больно, что взгляды наши разошлись, но ярый патриотизм Сергея... мне милее фашистского либерализма. Будь Жадан анархически патриотичен, я бы приняла, не разделила бы, но приняла. Он говорил мне: "Революция убивает своих детей". Я предпочла быть убитой, он - взойти на пьедестал. Это его выбор. Быть им убитой - честь.
Я все думала: как можно писать ТАК красиво? Кто способен творить этот литературный джаз на языке МОЕЙ культуры? Пока я не увидела Аню Долгареву и не втрескалась в это чудо так, что она стопроцентно на меня на позднюю повлияла - на всю позднюю меня. И по форме, и по контенту, и по Логосу. Если уж быть эпигоном, то её эпигоном. Но я стараюсь держать свой голос. Иногда получается, иногда - нет. Можно я покажу один текст Жадана, который раньше бы меня свел с ума?
СЕРГІЙ ЖАДАН
Люди самотні, - говорить, - це ж було помітно відразу.
Навіть більше – їхня самотність для них і є головною.
Тому хай буде пора, коли їм важливо буде збиратися разом.
І хай цю пору вони називають, скажімо, зимою.
Доки чоловіків тримає тепло ведмеже,
доки жінки тихо звіряються книгам,
хай у них буде межа, за якою все буде як вперше.
І хай цю межу вони, скажімо, називатимуть снігом.
Доки тут ще діється щось незвичайне,
доки я сам бодай чимось тут володію,
треба їм дати що-небудь радісне і печальне.
Радісне і печальне. Скажімо, надію.
Но потом я увидела ЭТО, народ, и всё. Я хотела новую форму, но старую Русь-матушку, полная гармония. 
АННА ДОЛГАРЕВА:
И приходят они из жёлтого невыносимого 
     света,  
Открывают тушёнку, стол застилают 
     газетой,  
Пьют они под свечами каштанов, под 
     липами молодыми,  
Говорят сегодня с живыми, ходят с 
     живыми.  
И у молодого зеленоглазого капитана  
Голова седая, и падают листья каштана  
На его красивые новенькие погоны,  
На рукав его формы, новенькой да 
     зелёной.  
И давно ему так не пилось, и давно не 
     пелось.  
А от водки тепло, и расходится 
     омертвелость,  
Он сегодня на день вернулся с войны с 
     друзьями,  
Пусть сегодня будет тепло, и сыто, и 
     пьяно.  
И подсаживается к ним пацан, молодой, 
     четвёртым,  
и неуставные сапоги у него, и форма 
     потёртая,  
птицы поют на улице, ездят автомобили.  
Говорит: «Возьмите к себе, меня тоже 
     вчера убили».
И еще. Я  вижу по комментариям дурачков, что иностранных агентов в России, работающих в масках ура-патриотов, сильно беспокоит извечный соросовский вопрос: а вдруг БЖ приедет в Россию лечиться и, не дай Бог, поможет воссоединению Украины с Россией и восстановлению единства восточнославянской цивилизации? Они же боятся, что их западные кураторы, которые спонсируют проект "За Русь утрусь", сильно по этому поводу расстроятся. Хочу сказать товарищам глобалистам, разряженных в лапти и гербы с орлами: не посылайте своих дурачков, я - не в состоянии восстановить Русь, мне имперских силенок на это не хватило, это не я сделаю... И приехать я тоже не могу - еле встаю с постели и на улицу не выхожу, лежу. "Ойся, ты ойся, Ты меня не бойся, Я тебя не трону, Ты не беспокойся".
