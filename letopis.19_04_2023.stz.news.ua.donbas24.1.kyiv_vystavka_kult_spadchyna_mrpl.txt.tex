% vim: keymap=russian-jcukenwin
%%beginhead 
 
%%file 19_04_2023.stz.news.ua.donbas24.1.kyiv_vystavka_kult_spadchyna_mrpl.txt
%%parent 19_04_2023.stz.news.ua.donbas24.1.kyiv_vystavka_kult_spadchyna_mrpl
 
%%url 
 
%%author_id 
%%date 
 
%%tags 
%%title 
 
%%endhead 

Ольга Демідко (Маріуполь)
Маріуполь,Україна,Мариуполь,Украина,Mariupol,Ukraine,Kiev,Kyiv,Київ,Киев,date.19_04_2023
19_04_2023.olga_demidko.donbas24.kyiv_vystavka_kult_spadchyna_mrpl

У Києві відкрилася виставка, присвячена культурній спадщині Маріуполя (ФОТО)

У столиці відкрили виставку, присвячену пам’яткам архітектури та
монументального мистецтва Маріуполя

18 квітня у День пам’яток історії та культури в Києві відкрилася маріупольська
фотовиставка «Вписані в історію. Зруйновані війною». Вона проходитиме у
Національному музеї історії України. Головна мета виставки — показати
зруйновану російськими окупантами культурну спадщину українського Маріуполя.

Читайте також: Сталевар, Вежа та море: у Львові презентують писанки з
маріупольськими мотивами (ФОТО)

Як відомо, до повномасштабного вторгнення рф саме в Маріуполі було зосереджено
одне з найбільших в Україні зібрань мозаїчної монументалістики. Станом на 24
лютого 2022 року в Маріуполі налічувалось: 206 історичних будівель, 6 з яких зі
статусом памятника архітектури; 106 памятників історії та монументального
мистецтва, 24 мозаїчниї полотна, 33 об'єкти археології. На фотовиставці були
представлені фото до повномасштабного вторгнення і після. Внаслідок авіаударів
і постійних обстрілів майже вщент знищені знакові будівлі центру міста, які
формували історичний ареал Маріуполя. Організаторами виставки виступили
Маріупольська міська рада, Національний музей історії України, проєкт «Я —
Маріуполь. Культура», ККП mEHUB.

«Виставку почали готувати з 17 березня. Ідея належить саме Департаменту
культурно-громадського розвитку Маріупольської міської ради. Ми розуміли, що в
цей день — 18 квітня — ми маємо на національному рівні говорити про втрачені
історичні місця Маріуполя», — зазначила Діана Трима.

Читайте також: Nava повернулася — з полону звільнили захисницю Маріуполя
Валерію Суботіну (ФОТО)

На виставці були присутні директор Національного музею історії України Федір
Андрощук та перший заступник міського голови Маріуполя Михайло Когут. На
відкриття завітав і Міністр культури та інформаційної політики України
Олександр Ткаченко, який уважно ознайомився з усіма світлинами.

«Навіть зараз Маріуполь продовжує боротись за свою культуру, традиції та
історію. Місто, колись відоме своєю історичною мультикультурністю, зараз ми
згадуємо із великим болем у серці. Проте, такі фотовиставки несуть в собі
велику цінність, бо дають нам можливість запам’ятати місто таким, яким воно
було, аби в майбутньому його відродити», — наголосив Олександр Ткаченко.

Читайте також: Маріупольський театр «Conception» показав у Тернополі свою
виставу (ФОТО)

Фото для виставки надали: муніципалітет Маріуполя, архіви Маріупольського
краєзнавчого музею, ККП mEHUB, маріупольські фотографи — Євген Сосновський та
Іван Станіславський. Фіксація руйнувань — з відкритих джерел.

«Світлини, представлені на виставці, зроблені протягом 2015−2021 років. Ці
світлини залишились, як важливі документи і наша пам’ять. Дуже боляче на все це
дивитись. Не хочеться порівнювати фото до і після», — підкреслив Євген
Сосновський. 

Виставка «Вписані в історію. Зруйновані війною» у Національному музеї історії
України триватиме до 1 травня 2023 року.

Раніше Донбас24 розповідав, що фільм про Маріуполь переміг на кінофестивалі у
США.

Найсвіжіші новини та найактуальнішу інформацію про Донецьку й Луганську області
також читайте в нашому телеграм-каналі Донбас24.

ФОТО: з архіву Донбас24
