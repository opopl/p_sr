% vim: keymap=russian-jcukenwin
%%beginhead 
 
%%file 06_11_2021.fb.fb_group.story_kiev_ua.1.bereznjaki_aktery_buchma_brondukov_mykolajchuk.cmt
%%parent 06_11_2021.fb.fb_group.story_kiev_ua.1.bereznjaki_aktery_buchma_brondukov_mykolajchuk
 
%%url 
 
%%author_id 
%%date 
 
%%tags 
%%title 
 
%%endhead 
\subsubsection{Коментарі}

\begin{itemize} % {
\iusr{Елена Сидоренко}

Знаменитые актёры, великие и талантливые люди, жили в обыкновенных
квартирах....

\begin{itemize} % {
\iusr{Наталья Клочкова}
\textbf{Елена Сидоренко} останні роки свого життя Борислав Брондуков проживав в будинку на розі Інститутської і Шовковичноі

\begin{itemize} % {
\iusr{Татьяна Сирота}
\textbf{Наталья Клочкова} В останні роки він проживав у Биківні.Там і помер...

\iusr{Наталья Клочкова}
\textbf{Татьяна Сирота} на будинку меморіальна табличка

\iusr{Татьяна Сирота}
\textbf{Наталья Клочкова}  @igg{fbicon.thinking.face}{repeat=3} 

\iusr{Наталья Клочкова}

\ifcmt
  ig https://scontent-frx5-1.xx.fbcdn.net/v/t39.30808-6/253711637_1774458712753584_7682488256257529163_n.jpg?_nc_cat=111&ccb=1-5&_nc_sid=dbeb18&_nc_ohc=xjVDNV-gyfwAX_M-w_R&_nc_ht=scontent-frx5-1.xx&oh=e05cb4c7888af276664e4b2939097091&oe=61BA512D
  @width 0.5
\fi

\iusr{Елена Сидоренко}
\textbf{Наталья Клочкова} 

я знаю, потом квартиру сдавали, и жили за городом. Жена очень хорошая, и дети
заботились... всё-таки лучше в частном доме...

\iusr{Наталья Клочкова}
\textbf{Елена Сидоренко} так, я знайома з біографією Борислава Брондукова. Великий артист!

\iusr{Татьяна Сирота}
\textbf{Наталья Клочкова} 

Практически полностью парализованный, актер был прикован к кровати последние
шесть лет жизни. Работать он уже, конечно, не мог, и семья оказалась в полной
нищете. Семья вынуждена была переехать в небольшой частный домик в Быковне, а
свою четырехкомнатную квартиру в центре сдать в аренду (возможно это квартира в
том доме, о котором говорите Вы).

Из Википедии:
Дата смерти -
10 марта 2004 г.
Место смерти -
Быковня, Киев.

\end{itemize} % }

\iusr{Elena Kravchenko}
\textbf{Елена Сидоренко} улица Бучмы, но сам актёр жил вовсе не на Берещнякпх, а на пересечении Владимирской и БЖ
\end{itemize} % }

\iusr{Елена Заславская}
17 этажка! И Криницкая жила на 17 а мы с родителями на 9

\begin{itemize} % {
\iusr{Татьяна Сирота}
\textbf{Елена Заславская}  @igg{fbicon.thinking.face}{repeat=3} 

\iusr{Игорь Новус}
\textbf{Татьяна Сирота} 

При всем уважении, вы забыли Алима Федоринского... живёт по Бучмы5а... часто его
вижу... а в молодые годы часто видел Брондукова с ребенком, гуляли на озере...


\iusr{Татьяна Сирота}
\textbf{Игорь Новус} 
Я не забыла! Я об этом не знала ☺ ️ 
Спасибо за дополнение к публикации!

\iusr{Ness Maria}
\textbf{Елена Заславская}
А мы жили на 3- ем этаже.
\end{itemize} % }

\iusr{Ирина Нищимная}
Моя родина, здесь прошло моё детство и юность,,, люблю Березняки,, это особый
рай, родом из детства,,,

\begin{itemize} % {
\iusr{Tamila Leychkis}
\textbf{Ирина Нищимная} и мое тоже. Серафимовича 6

\iusr{Наталия Сорока}
\textbf{Ирина Нищимная} привет, Ирочка! Да, Березняки это наше детство и юность)

\iusr{Ирина Нищимная}
\textbf{Наталия Сорока} Приветик, где ещё встретиться, очень рада, сколько лет не живём, а родина есть родина!
\end{itemize} % }

\iusr{Ол Ткаченко}

Район забудовувався у 70-х роках і ліфти з того часу не мінялися. В 90-х роках
вони часто не працювали, виходили з ладу. На схилі віку в Маргарити Криніциної
ускладнилися проблеми із ногами і вона старенька, живучи під дахом не могла ані
спуститися ані піднятися додому і сиділа днями як ув’язнена у своїй квартирі.

\iusr{Татьяна Сирота}
\textbf{Ол Ткаченко} Да... я про это читала. @igg{fbicon.face.pensive}{repeat=3} 

\iusr{Larisa Lappo}
Спасибо, Таня!

\iusr{Татьяна Зубко Маркина}

Спасибо. Давно, много лет не была. Несколько лет была квартира возле озера.
Какие артисты. Миколайчук, глыба, талант и его трепетная, любящая жена

\begin{itemize} % {
\iusr{Татьяна Сирота}
\textbf{Татьяна Васильевна Зубко} Маркина Она до сих пор живёт в этом доме.

\iusr{Татьяна Зубко Маркина}
\textbf{Татьяна Сирота} прекрасная женщина
\end{itemize} % }

\iusr{Vadim Vadim}
Вышел летом из гостей и упал в Тельбин))!

\begin{itemize} % {
\iusr{Татьяна Сирота}
\textbf{Vadim Vadim} И такое может быть @igg{fbicon.face.upside.down}{repeat=3} 

\iusr{Vadim Vadim}
\textbf{Татьяна Сирота} лето, ночь, жара))

\iusr{Татьяна Сирота}
\textbf{Vadim Vadim} Ну да...\enquote{Тополиный пух, жара, июль...}😉
\end{itemize} % }

\iusr{Вахтанг Кварелашвили}
Лучший район Киева.

\iusr{Татьяна Сирота}
\textbf{Вахтанг Кварелашвили} Мне тоже нравится ☺ ️ 

\iusr{Петр Кузьменко}

Прекрасный пост рассказ о знаковых для Киева и Украины замечательных и любимых
актёрах! И великолепные фотографии уютного района родного Города! Благодарю!
 @igg{fbicon.hands.applause.yellow}  @igg{fbicon.heart.sparkling} 

\iusr{Анна Сидоренко}
Спасибо.

\iusr{Сергій Савелій}

Бучмы, 8, квартира 108, милости просим... Так говорили, коли там жив Боря
Савлохов... На жаль, нині це й місце суму... Адже у березні 2000- го саме біля
цього будинку було вбито Тимура Савлохова, світла їм пам'ять...

\begin{itemize} % {
\iusr{Татьяна Сирота}
\textbf{Сергій Савелій} Да...лихие 90-е.

\iusr{Мария Иванова}
\textbf{Сергій Савелій} , хорошие бандиты были, у богатых, наверное отнимали и бедным раздавали. Робин Гуды

\iusr{Сергій Савелій}
Маріє, Борис - заслужений тренер СРСР, Тимур - заслужений тренер України. Так,
наша Асоціація спортивної боротьби України, президентом якої був Борис
Сосланович, допомагала дитячим будинкам, школам - інтернатам для дітей-
сиріт...

\begin{itemize} % {
\iusr{Мария Иванова}
\textbf{Сергій Савелій} , что не мешало им создать одну из самых больших ОПГ.

\iusr{Сергій Савелій}
\textbf{Mariia Ivanova} Мені здається, що нині ОТГ справжні ОПГ... А те, що пишуть і говорять... То переважно ті, хто і знать не знав ні Боріка, ні Тимура...

\iusr{Мария Иванова}
\textbf{Сергій Савелій} , ну как не знал. Вся коммерция Киева знала, кому платят. Кто Савлоху, кто Киселю, кто Авдышеву.
\end{itemize} % }

\iusr{Андрей Гаврилко}
\textbf{Сергій Савелій} ,живу всю жизнь в этом доме, такого фольклора не слышал))

\begin{itemize} % {
\iusr{Мария Иванова}
\textbf{Андрей Гаврилко} , здрасти, каждой год цветы лежат в месте убийства. Напротив второго подъезда.

\iusr{Андрей Гаврилко}
\textbf{Мария Иванова} ,я про поговорку

\iusr{Сергій Савелій}
\textbf{Андрей Гаврилко} Так говорили борці між собою...

\iusr{Любовь Кулакова}
\textbf{Андрей Гаврилко} 

Я не пам'ятаю який це був день, але пам'ятаю цю ніч. Коли його розстрілювали, а це
було десь о першій ночі, то я прокинулася від грохоту салюту. Тоб то так все
було сплановано. Я Савлохова старшого бачила, його жінку, двоє хлопців. Знала, що
тренер, більш нічого. Це вже пізніше, коли його посадили, почали писати багато чого
про його життя.

Так, в нашому районі проживало і проживає багато знаменитих людей, крім тих що
згадали, ще Могилевська, Бужинська, Зевйо, а ще хлопці з відомих гуртів, але забула
їх назву вже. Може ще хтось згадає про когось.
\end{itemize} % }

\iusr{Ольга Патлашенко}
\textbf{Сергій Савелій} Рада приветствовать, Сергей! Присказка про Савлохова очень понравилась.
А еще на Березняках жил хорошо нам с вами знакомый Леня Вайнерман...

\begin{itemize} % {
\iusr{Сергій Савелій}
\textbf{Ольга Патлашенко} 

Навзаєм! Льоня любив баскетболістів і борців. Завжди проводжав і зустрічав нас
в аеропорту, щоб все було за вищим розрядом!...

\end{itemize} % }

\end{itemize} % }

\iusr{Любовь Добровольская}

Дружина Івана Миколайчука, Марія Евгенівна Миколайчук, теж відома акторка(в
місті Чернівці, на площі біля театру, на алеї зірок, є зірка і на її честь), і
зараз мешкає в тій квартирі. На жаль вона дуже хвора і вже не виходить з
квартири. А раніше, коли мала сили, то виходила і казала, що йде посидіти до
каменюки і має надію дочекатись на пам'ятник (це камінь в сквері, який помпезно
відкрили років 6 тому, і встановили камінь з написом, що незабаром тут буде
пам'ятник Івану Миколайчуку).

Мабуть з такими темпами та Марія Евгенівна того пам' ятника може і не
дочекатись.

А ще трохи і забудовники які мають владу в Київі і сквер забудують.  @igg{fbicon.cry} 

\begin{itemize} % {
\iusr{Людмила Ива}
\textbf{Любовь Добровольская} очень печально...

\iusr{Татьяна Сирота}
\textbf{Любовь Добровольская} Хочеться сподіватися, що пам'ятника дочекається...
\end{itemize} % }

\iusr{Людмила Ива}
Еще здесь проживает знаменитая украинская певица Алла Кудлай!))

\begin{itemize} % {
\iusr{Татьяна Сирота}
\textbf{Людмила Ива} Ух ты!!! Не знала.А где именно?

\iusr{Людмила Ива}
\textbf{Татьяна Сирота} 

на Днепровской набережной, в доме, где раньше размещался, на первом этаже,
Яготинский магазин. Очень часто встречала ее на Днепре, в районе, не доходя до
общего пляжа. Очень приятный человек в общении.)

\end{itemize} % }

\iusr{Лариса Сидоренко}

Жила на П. Тычины 1975-1980гг - новые 16-этажные дома, песок, ветер... очень не
уютно. Мы молодые, нам на работу на правый берег - переезд трамваем через Днепр до
метро - целое дело: конец света! Автобус упал с моста Патона, аварии машин... Мечтали
переехать в центр - переехали. Постарели-какой замечательный зелёный красивый
стал массив! И Днепр рядом! Как бы было хорошо в нем сейчас жить: курорт!!! С
возрастом меняются вкусы и желания.

\begin{itemize} % {
\iusr{Татьяна Зубко Маркина}
\textbf{Лариса Сидоренко} и я прожила несколько лет возле озера, Березняковская, З0, уехавши с Леси..Чудесно, плавай прямо с парадного. Страдала за центром. Теперь Оболонь, не любила. А как теперь радуюсь. Все рядом и озера метро,познаем в сравнении. Удачи
\end{itemize} % }

\iusr{Фарид Степанов}

Моя малая родина: Тычины 3.

Учился в 191 школе. Ходил в кинотеатр \enquote{Десна}. С пацанами бегали купаться на
Тельбин. Когда открывали музей ВОВ, стояли напротив моста Платона и кричали
\enquote{Ура!}.

\begin{itemize} % {
\iusr{Маргарита Рассказова}
\textbf{Фарид Степанов} абрес Оз. Тельбин придумал знаменитый Киевский Архитектор Вадим Михайлович Гечина ,один из авторов Украинского дома ...))

\iusr{Валентина Горковенко-Спицына}
\textbf{Маргарита Рассказова} Гречина В.М. И, наверное, абрИс? Я об этом не знала. Хотя долго работала с В.М. и долго жила на Березняках.
\end{itemize} % }

\iusr{Михайлина Голуб}

Во-первых, огромное спасибо, уважаемая Татьяна, за рассказ о моих Березняках. Я
прожила тут 29 лет, дольше, чем где-либо в своей жизни. Более того, я жила на
Русановке с самого начала ее существования. Мы переехали сюда 14 февраля 1964
года. А училась в первом выпуске первой русановской 182-ой школе. И начиная с
9-го класса у нас учился целый класс с Кухмистерской слободки - деревне в черте
города на месте современныз Березняков. И Березняки появились и росли на моих
глазах. Помню я и печально знаменитое противостояние между молодежью Русановки
и Березняков. Эти драки, заканчивающиеся часто серьезными трав. мами.  И
замечательные березняковские 2 маленьких кинотеатра, такие уютные. Сейчас, увы,
"Десны" уже много лет нет. Осталась только остановка автобуса "Кинотеатр
"Десна". Судьбу "Десны" разделили и гастроном "Харьков", и замечательный
книжный магазин. Дольше всех сопротивлялся магазин "Культовары". Сначала его
все урезали и урезали, а потом и вовсе заменили супермаркетом "АТБ". Зато на
Березняках есть и свой ВУЗ. Вначале был институт усовершенствования учителей, а
потом в этом же здании разместили часть пединститута им. Б.Гринченко.

О Березняках можно говорить еще много и много. Еще раз спасибо, Татьяна, за ваш
пост.

\begin{itemize} % {
\iusr{Татьяна Сирота}
\textbf{Михайлина Голуб} Тоже люблю этот район города. Тут живут мои дети и внуки ☺ ️ 

\iusr{Юлия Никандрова}

Кухмистерская Слободка была там, где Левобережный массив (Окипной, Туманяна,
Флоренции). И был хутор Березняк-Зательбин, его имя и дало название массиву.
Большинство этой территории стало пригодно для строительства только когда
насыпали много песка, а так... воды было тут много, низина, до того как.

\begin{itemize} % {
\iusr{Михайлина Голуб}
\textbf{Юлия Никандрова} 

Извините, госпожа Никандрова. Но нынешняя Левобережная - Никольская слободка.
Кстати, кроме класса с Кухмистерской слободки у нас в 9-ом классе был класс и с
тогдашней Никольской слободки, из восьмилетки 208 школы. Это сейчас она
уважаемая школа, даже лицей, а тогда была обычная сельская 8-летка. И
Никольская слободка постепенно уходила в прошлое уже при мне. Кстати, именно
там в НИКОЛЬСКОЙ церкви венчались Ахматова и Гумилев в начале ХХ века. А
Кухмистерская слободка тоде осталась в памяти. Остались деревья, родом оттуда,
в том числе и тополь - самое старое дерево на всем Левом берегу. И массив, как
и Русановка, был построен именно на насыпном песке. Именно поэтому у нас гибнут
березы - их корни глубоко уходят в грунт, и достигнув намытой песчяной
подушки, погибают. Так что проверьте ваши сведения. Они, увы,не во всем верны.

\end{itemize} % }

\iusr{Larisa Golubchyk}
\textbf{Михайлина Голуб} Где же здесь был гастроном Харьков, живу на Березняках больше 30 лет

\begin{itemize} % {
\iusr{Михайлина Голуб}
\textbf{Larisa Golubchyk} 

Уважаемая госпожа Голубчук. Гастронлм \enquote{Харьков} был в том доме, где сейчас
аптека, молодежный клуб, библиотека, офис нотариуса. Это ул. и. Миколайчука
(бывшая Серафимовича), по-моему номер 7. Насчет номера не уверена,а все
остальное правильно. Но старые березняковцы, а семья моего мужа здесь жила с
1969 г.по-прежнему называют этот дом Харьковом.

\end{itemize} % }

\iusr{Dimitri Statnikov}
\textbf{Михайлина Голуб} 

Михайлина Ильинична много лет работала учителем Истории в 141 школе на
Русановке я все 10 лет учился в ней с начала 80-ых до начала 90-ых. Это
случайно не Вы?) А насчёт резни пацанов Русановки и Березняков подтверждаю
лично я в них участвовал. Самым страшным местом был периметр Тельбина.
Березняковские там избивали там наших Арматурными прутами. Я как то тащил
оттуда до Энтузиастов окровавленного товарища.. Последние 10 лет живу как раз в
новом доме на Тельбина и когда гуляю вокруг озера иногда вспоминаю это дикое
время. Позже там ещё был знаменитый штаб дяди Бори Савлохова банный комплекс,
он и жил там недалеко..

\iusr{Михайлина Голуб}
\textbf{Dimitri Statnikov} Это я, Дима.

\iusr{Виктория Киселева}

А я помню, как возвращаясь с поездки на дедушкину родину, село Погребы, все
машины заезжали на Тельбин. Купались, мыли машины и через мост Патона вьезжали
в Киев. А жила я на Ярославовом валу в коммунальной квартире Мой сын не может
понять, с какой радостью мы в 1965 году покинули \enquote{это счастье} и переехали в
двухкомнатную квартиру на Энтузиастов 37. Возвращаясь вечером с университета,
мы с друзьями собирались и купались в канале. Это действительно был рай!

\begin{itemize} % {
\iusr{Михайлина Голуб}
\textbf{Виктория Киселева} 

Да, Виточка, в 60-е на Русановке был рай. Мы тоже купались в канале, вода в нем
тогда была чистой, и он еще не был одет плитами. И муж рассказывает, какой
чистой была вода в Тельбине в начале 70-х, когда они перебрались с Печерска на
Березняки.

\end{itemize} % }

\iusr{Dimitri Statnikov}
\textbf{Михайлина Голуб} 

как же я рад! Не знаю вспомните ли Вы меня, я тогда был на фамилии отца -
Кашакашвили, я учился в классе Виталия Петровича Борисенко. Я Вам желаю долгих
лет, здоровья и Благополучия!! Вы потрясающе передавали Ваши Знания!!


\iusr{Михайлина Голуб}
\textbf{Dimitri Statnikov} Огромное спасибо. И тебе здоровья и счастья.

\iusr{Виктория Киселева}
\textbf{Михайлина Голуб}

Димочка, и я тебя прекрасно помню, ваш класс, особенно Женю и Лену. А с
Виталием Петровичем мы проникали на премьеры в Оперный театр. Он был в своей
любви к балету и опере моим единомышленником.

\begin{itemize} % {
\iusr{Dimitri Statnikov}
\textbf{Виктория Киселева} спасибо, очень приятно, извините а какой предмет Вы вели у нас а то я что-то не могу вспомнить?))

\iusr{Виктория Киселева}
\textbf{Dimitri Statnikov} я преподавала русский язык и литературу - Виктория Владимировна.

\iusr{Dimitri Statnikov}
\textbf{Виктория Киселева} конечно вспомнил, всех благ Вам и Вашим Родным! Как говорится, нам не дано предугадать как слово наше отзовётся!
\end{itemize} % }

\iusr{Ирина Парасій-Вергуненко}
\textbf{Михайлина Голуб} я тоже училась в первом выпуске 182 школы. Первая учительница - Лидия Владимировна Ружницкая. На Русановке живу и сейчас

\iusr{Михайлина Голуб}
\textbf{Ирина Парасій-Вергуненко} Я окончила школу в 1967 году. В 182-ой училась со дня ее основания, с 8-го класса.

\iusr{Ирина Парасій-Вергуненко}
\textbf{Михайлина Голуб} А вы случайно не Елены Михайловны Бороденко дочка?

\begin{itemize} % {
\iusr{Михайлина Голуб}
\textbf{Ирина Парасій-Вергуненко} Да, это моя мама.

\iusr{Ирина Парасій-Вергуненко}
Значит мы давнишние соседи по дому 4/1, я Ира с 40 квартиры, а ваша семья жила в 41 квартире. Я живу там и сейчас.

\iusr{Михайлина Голуб}
\textbf{Ирина Парасій-Вергуненко} Как приятно. Я прекрасно помню вас, Ирочка.

\iusr{Ирина Парасій-Вергуненко}
Где вы сейчас живёте? Какие у вас новости?
\end{itemize} % }

\iusr{Виктория Киселева}
\textbf{Dimitri Statnikov} 

Я сейчас работаю в школе, которая находится в здании 141 школы, но это теперь
целый учебный комплекс по изучению информационных технологий. Мне очень приятно
, что я уже учила детей ребят из твоего класса - Гуляеву Машу, Шульгина Никиту
Кияницу /Журавлева /Ивана. Чудесные, талантливые, современные ! Вот так все
вернулось и откликнулось ! Я счастлива !


\iusr{Dimitri Statnikov}
\textbf{Виктория Киселева} 

да я иногда когда гуляю по Русановке подхожу и смотрю на этот комплекс, конечно
все очень изменилось, но наверное это и к лучшему! Отлично что поколения
сменяются но по прежнему получают знания именно в этих стенах! Юрий
Григорьевич, вел у нас Алгебру и Геометрию мы были как раз одним из первых его
классов, когда он пришел в школу! Дай Вам всех Благ и Здоровья!!  .

\iusr{Виктория Киселева}

Он создал замечательную школу и проявил себя современным руководителем. Это уже
совсем другая история ! Мы вместе 22 года создаём совсем другую школу, с
другими подходами к детям и преподаванию. Заходи, когда закончится локдаун, и
ты будешь приятно удивлён! Вам такие условия и возможности обучения и не
снились!

\begin{itemize} % {
\iusr{Dimitri Statnikov}
\textbf{Виктория Киселева} 

да конечно! Но есть и кое-какие минусы, например нет больше уроков труда для
ребят разрушили и не восстановили теплицу которую создавал наш учитель
Биологии, срезали все деревья со стороны нашего класса в котором преподавал
Виталий Петрович, но конечно внутри я с тех как выпустился, никогда больше не
был. Спасибо за Приглашение!

\end{itemize} % }

\iusr{Татьяна Годынская}
\textbf{Михайлина Голуб} 

я родилась на Русановке в 1966, Березняки помню как застраивались, как
поднимался песок во время ветра и будучи школьницей мне было там как
жутковато... помню противостояния молодежи. Потом березняки стали рости и
делиться на свои микрорайоны и там делить территорию... много чего было... Кафе
\enquote{Березка}, чего только там не происходило... внизу овощной и \enquote{инвалидный}
магазины... помню хозяйственный магазин, \enquote{Десну}... всё меняется, всё течёт...

\end{itemize} % }

\iusr{Sergey Chernov}
Да, пам'ятаю, бо жив тоді поряд.

\iusr{Елена Квашенко}

Спасибо за рассказ о наших любимых Березняках и наших знаменитых жителях.
Любимый район - Березняки.

\iusr{Tatyana Poturnak}

Танюша, ещё на Русаковской набережной жил Лесь Сердюк - актёр Киностудии им . Довженко.

\iusr{Татьяна Сирота}
\textbf{Tatyana Poturnak} Спасибо! Теперь об этом знаю ☺ ️ 

\iusr{Маргарита Рассказова}

Автором застройки этого жилого района был АРхитектор Виктор Иванович Сусский,
автор Дворца спорта и многих др. объектов...))

\iusr{Маргарита Рассказова}
Это было время блестящих талантов, со многими из них мы дружили ...))

\iusr{Татьяна Клубченко}

А ещё здесь жил Николай Луценко.. Наш любимый актер ТЮЗа на Липках, Карлсон и
ведущий \enquote{Миколина погода}, оф. голос телеканала ICTV-зірка телебачення України..

\iusr{Михайло Присенко}

У 1969 році молоде подружжя вимірювало кроками на суцільній пісчаній пустелі
від кута сходження залізничних колій де буде їх перша домівка в запланованому
9-ти поверховому будинку чеського проекту! Потім незабутні стукіт залізниці у
нічній тиші, Тельбін (рибальство,відпочинок літом та «моржування» в холодну
пору року), подолання відстані на роботу на правий берег і додому звідти і ...
щасливі роки у власній оселі! Дякую за публікацію!

\iusr{Татьяна Сирота}
\textbf{Михайло Присенко} Будь ласка! @igg{fbicon.hands.pray} 

\iusr{Ольга Іушина}

Рідні Березняки! Моє дитинство пройшло там у дідуся в квартирі). Та й зараз, і
завжди, жила поруч, на Лівобережній)

\iusr{Всеволод Шевчук}
Чудове оповідання!

\iusr{Елена Гирина}

Татьяна, огромное Вам СПАСИБО, что вернули меня в детство, самое счастливое и
беззаботное время!

Волею Судьбы я жила на Березняках с 1970 по 1979 г. г., училась с 1 го по 8 кл. и
самое удивительное!!!, - проживала именно на Бучмы 8!!!.

Семья Брондуковых были нашими соседями! Это были люди удивительной
простоты, чистоты, искренности, доброжелательности, - можно перечислять
бесконечно!....

Светлая Память Б. Брондукову и всем ушедшим @igg{fbicon.hands.pray}{repeat=3} 

И сейчас, когда бывает счастливая возможность, - с удовольствием гуляю по любимым
Березнякам, вокруг озера Тельбин, наслаждаюсь добрыми воспоминаниями и получаю
огромный позитивный заряд! @igg{fbicon.hearts.revolving} 

\ifcmt
  ig https://i2.paste.pics/8f102be201c3fc592d02a4362f9e133a.png
  @width 0.2
\fi

\iusr{Светлана Ветрова}

Дякую за цікаві історії нашого міста завдяки вашим прогулянкам дізнаємось
багато цікавого про людей нашого рідного Києва дякую вам

\iusr{Майрам Разакова}

\ifcmt
  ig https://i2.paste.pics/e784dc36ca8829154d5cb65173478475.png
  @width 0.2
\fi

\iusr{Dimitri Statnikov}

Последние десять лет новые дома на Тельбине это лучшее место для жизни в Киеве.
Но раньше это был мрачный район.

\begin{itemize} % {
\iusr{Татьяна Сирота}
\textbf{Dimitri Statnikov} За последние 6 лет Березняки преобразились.
Район стал комфортным для проживания.

\iusr{Dimitri Statnikov}
\textbf{Татьяна Сирота} даже ещё раньше!

\iusr{Татьяна Сирота}
\textbf{Dimitri Statnikov} Про раньше не могу сказать.У нас дети в 2015 году переехали на Березняки. И я вижу,как меняется район,как хорошеет.

\iusr{Татьяна Зубко Маркина}
\textbf{Dimitri Statnikov} и раньше такой как все новые. Озеро, Днепр украшали. Все видят по своему. Да, после центра не очень, но я с 14 этажа в дела Киев с Выдубичей...
Удачи

\iusr{Ол Ткаченко}
\textbf{Dimitri Statnikov} 

На жаль не можу погодитися. Ці нові будинки над Тельбіном — вкрадена в
березняківців частина рекреаційної зони. Збудовані в часи розгулу будівельної
мафії без погоджувальних документів із підкупом голосів навколишніх голів
кооперативів тощо. Вони псують пейзаж і затуляють краєвид своєю надмірною
висотністю.

\begin{itemize} % {
\iusr{Dimitri Statnikov}
\textbf{Ол Ткаченко} будiвельна мафiя же Ви ii бачiли взагали! Тiльки

\iusr{Dimitri Statnikov}
Задрволенни Люди!

\iusr{Татьяна Годынская}
\textbf{Ол Ткаченко} а новый желтый дом вообще имел дурную славу. Церковь правда красивая.

\iusr{Ол Ткаченко}
\textbf{Татьяна Годынская} Він збудований на цвинтарі. Розповідають, що під час його зведення один з будівельників розбився.
\end{itemize} % }

\end{itemize} % }

\iusr{Владимир Картавенко}
Отселение коренних киевлян в Черниговскую губернию

\begin{itemize} % {
\iusr{Катя Катя}
Ага всех, многих на трою @igg{fbicon.face.grinning.smiling.eyes} 

\iusr{Анна Борисенко}
\textbf{Владимир Картавенко} да, моя бабушка шутила: Вам должны дать медаль за освобождение Киева, когда мы переехали на Бойченко, возле метро Черниговская, тогда Комсомольская. Я не понимала, она поясняла: вы из Киева переехали в Черниговскую губернию.
\end{itemize} % }

\iusr{Любовь Кулакова}

Дуже цікава публікація. Діліться своїми спогадами, це так захоплююче. В МЕНе на
РУСАнІВЦІ жИлА знайома і колИ я До неЇ приЇздиЛА у гостІ, тО мені дуже
подобавсЯ цеЙ р-н. Навіть мріялА тут жити. Тепер я живу на Березняках. МРІЇ
ЗБУВАються.

\iusr{Людмила Комлик Свидынюк}

\ifcmt
  ig https://i2.paste.pics/06f53427b1fc655ac1f70377512e0c4a.png
  @width 0.2
\fi

\iusr{Наташа Гресько}

спасибо! живу тут 15-й год - уже и на правый расхотелось возвращаться @igg{fbicon.face.wink.tongue}  а фото
сегодняшние? прокатный велик на бульваре на том же месте @igg{fbicon.face.grinning.smiling.eyes} 

\iusr{Татьяна Сирота}
\textbf{Наташа Гресько} Может прокатный велик и ныне там...
А фото сделаны 13 октября ☺ ️ 

\iusr{Polina Feldman}

красота просто зеленый сад мы жили по Серафимовича где магазин Харьков @igg{fbicon.face.smiling.hearts} 

\begin{itemize} % {
\iusr{Татьяна Сирота}
\textbf{Polina Feldman} Сейчас в этом доме под номером 7 живут мои дети и внуки. ☺ ️ 

\iusr{Валентина Горковенко-Спицына}
\textbf{Татьяна Сирота} Я тоже очень долго жила на Березняках. А в перерыве - на Русановке. Муж увлекался рыбалкой и никак не хотел менять место жительства. Слишком много у него было друзей-рыбаков и друзей-киношников.

\iusr{Polina Feldman}
мы жили на 3 эт 3 парадное 2 балкона я за 50 лет небыла в Киеве но всегда вспоминаю с Любовью @igg{fbicon.face.smiling.hearts}{repeat=8} 
\end{itemize} % }

\iusr{Раиса Карчевская}
Танечка!
Большое спасибо за прекрасный пост и фотографии

\iusr{Леонид Сорока}

13 лет прожили на улице Серафимовича, 7, на втором этаже. Под нами была
библиотека и продовольственный магазин. А напротив окон у само трассы было
отдельное небольшое здание - общежитие милиционеров.

\begin{itemize} % {
\iusr{Михайлина Голуб}
\textbf{Леонид Сорока} 

А сейчас этого здания, общежития нет. Здание снесли, начали на этом месте
что-то строить, но видно деньги кончились, успели сделать фундамент и стоят
развалины.

\iusr{Татьяна Сирота}
\textbf{Михайлина Голуб} 

Уже и развалин скоро не будет. Фундамент здания демонтировали.  Эта территория
относится к скверу Миколайчука. У застройщика не получилось
\enquote{отгрызнуть} её. Громада отстояла.  Будет сквер!

\iusr{Михайлина Голуб}
\textbf{Татьяна Сирота} И это прекрасно. Спасибо. Я, увы, уже 1,5 года очень
далеко от Березняков и вообще от Киева.
\end{itemize} % }

\iusr{Валентина Белозуб}

Спасибо, Танюша. Мне нравятся Березняки, куда переехали из центра мои
одноклассницы. Раньше частенько приезжала к ним в гости, или летом на прекрасное
озеро Тельбин. Сейчас с этим сложнее.

\iusr{Valeria Tarhons'ka}
Мила Йовович, Березняковская, 26 @igg{fbicon.flame} 

\begin{itemize} % {
\iusr{Татьяна Сирота}
\textbf{Valeria Tarhons'ka} И???
Родилась в Киеве. Одно время семья жила на Русановке. Про Березняковскую, 26
ничего не знаю. Если есть чем, поделитесь ☺ ️ 

\iusr{Valeria Tarhons'ka}
\textbf{Татьяна Сирота} ну, я же не могу выложить инфо про ее дядю, он собственник @igg{fbicon.heart.red}

\iusr{Татьяна Сирота}
\textbf{Valeria Tarhons'ka}
Раз, пишете, что не можете, то не надо.
Дядя - это не Мила...

\iusr{Valeria Tarhons'ka}
\textbf{Татьяна Сирота} ок, удаляю

\iusr{Валентина Горковенко-Спицына}
\textbf{Татьяна Сирота} 

На Березняковской, 26 не Мила жила, а ее дядя. Я знала его и его семью, дружила
с ними. И жили мы почти рядом. Но тогда точно никто не знал, что Мила это
Мила!!! Она была еще совсем маленькой. Да и потом они никогда об этом не
говорили (хотя фамилия была такая же).

\end{itemize} % }

\iusr{Svetlana Loguinova}

Я хорошо помню как начинали строить этот массив. Ближе к озеру были дачи. У
наших знакомых там была дача, и мы приезжали к ним. Я ещё была тогда небольшая,
но запомнились чудесные сады. Затем вместо садов начал расти массив.

\iusr{Сергій Савелій}

А ще в будинку Бучми, 5 жила унікальна сім'я Двораків - Саша в 12 років( ще в
Макіївці) закінчив школу, а в 17- уже Київський університет, Вова і Таня -
також раніше ровесників. Вова - триразовий чемпіон Радянського Союзу з
настільного тенісу, нині живе в Іспанії, а його дочка Галя, яка народилася ще в
Києві, як настільниця захищала честь Іспанії на чотирьох Олімпіадах. Цікаві
Березняки...

\iusr{Любовь Белоцерковец}
Спасибо большое!
Березняки строились практически на моих глазах. ( в 1973 году жила на Тычины
1). И так приятно вспомнить....

\ifcmt
  ig https://scontent-frx5-2.xx.fbcdn.net/v/t39.1997-6/p480x480/91521538_1030933857302751_5093925307199520768_n.png?_nc_cat=1&ccb=1-5&_nc_sid=0572db&_nc_ohc=T84vDeO16oQAX_Hz1V5&_nc_ht=scontent-frx5-2.xx&oh=ee46446a5e4b522b6c9ec1e1c4978a45&oe=61BBBC37
  @width 0.2
\fi

\iusr{Татьяна Желдубовская}
Радует, что назвали сквер и улицу в его честь, он заслужил это, спасибо огромное за Ваш труд

\iusr{Рона Ильинична}
Очень интересный материал.

\ifcmt
  ig https://i2.paste.pics/735376c55865c2e07708004112358c9b.png
  @width 0.2
\fi

\iusr{Рона Ильинична}
Жители. Березняков
должны знать, кто живёт с ними рядом...

\iusr{Алёна Буркот}
Березняки

\ifcmt
  ig https://scontent-frx5-2.xx.fbcdn.net/v/t39.1997-6/s168x128/240595295_280280763862242_7752062239402781192_n.png?_nc_cat=1&ccb=1-5&_nc_sid=ac3552&_nc_ohc=Ii-6Fpf6JUMAX_sp4sO&tn=lCYVFeHcTIAFcAzi&_nc_ht=scontent-frx5-2.xx&oh=fc5d32d61eecc017f1057974f015145a&oe=61BA271D
  @width 0.1
\fi

\iusr{Татьяна Сирота}
\textbf{Алёна Буркот} Я тоже. @igg{fbicon.heart.red}

\iusr{Александр Дуженко}

А хто пам‘ятає пивний бар на озері?? На перехтесті стояв.. нажаль забув назва
вулиць.. Тичини і ще якась.. може помиляюсь..((

\begin{itemize} % {
\iusr{Ол Ткаченко}
\textbf{Александр Дуженко} 

Той вертеп називався \enquote{Багратіоні}. Згорів врешті-решт. Замозахват частини
бульвару. Нажаль замість того щоб вигребти те попелище дощенту і відновити
бульвар, приміщення відбудували вже в цеглі але розмістити в ньому знову
ресторан не подужали. Тому тепер там торгують рідкісною фарбою.

\begin{itemize} % {
\iusr{Игорь Билинский}
\textbf{Ол Ткаченко} Не Багратион, а Бастион @igg{fbicon.index.pointing.up} Ги-Ги знаток  @igg{fbicon.face.hand.over.mouth} 
И тот кто спрашивал, имел ввиду Пивные автоматы на месте которых сейчас стоит Новоапостальская церковь @igg{fbicon.index.pointing.up}

\iusr{Ол Ткаченко}
\textbf{Игорь Билинский} Ваша правда  @igg{fbicon.smile}  Бастіон
Ось так красиво палало:

\ifcmt
  ig https://scontent-frx5-1.xx.fbcdn.net/v/t39.30808-6/254630684_2054226528086867_4541989462822798735_n.jpg?_nc_cat=111&ccb=1-5&_nc_sid=dbeb18&_nc_ohc=VFVfDoTgYIkAX8itV1W&_nc_ht=scontent-frx5-1.xx&oh=00_AT9c-Br47QBx8716aKtLH1yDooEeMlhiyM1Mq-YLY7ulmQ&oe=61BAB6AA
  @width 0.4
\fi

\iusr{Елена Ходорко}
\textbf{Ол Ткаченко} ...А ПОСЛЕ ПОЖАРА... там сделали лако-красочный магазин или склад, ни вывески....

\iusr{Ол Ткаченко}
\textbf{Елена Ходорко} Ну, власне я про це теж написав раніше  @igg{fbicon.smile} 

\iusr{Александр Дуженко}
\textbf{Ол Ткаченко} ні!!! Це було в 90-х і він стояв на т- образному перехресті, за ним озеро!! Там була невеличка будка і приїзжала бочка, коли раз на день, коли двічі!!

\end{itemize} % }

\end{itemize} % }

\iusr{Ольга Патлашенко}

Березняки - мой район. Жила там с родителями в студенческие годы. Удачное
расположение массива - пляж на Тельбине, рядом Русановский канал и чуть дальше
- переправа с Русановки на пляж на окраине Гидропарка.

Всегда гордилась красивым сквером, который тянулся от моста Патона до моста над
железной дорогой. Увы, сейчас его уже застроили: заправка, Макдональдс, и
главное \enquote{украшение} - рынок со всеми его \enquote{красотами} !!!

А последние годы душа радуется, как благоустроили и облагородили центр
Березняков. В районе Тельбина.

О знаменитых соседях - актерах знали все жители массива. И очень гордились этим

\iusr{Roma Bal}

\ifcmt
  ig https://i2.paste.pics/6ae9cccbb7d17dd494bdcc5c4723488c.png
  @width 0.2
\fi

\iusr{Виталий Перещ}

А ще, у вісімдесятих роках, на Березняках проживала Галина Кузяєва. Нині
художній керівник Дніпровської філармонії.

\ifcmt
  ig https://scontent-frt3-1.xx.fbcdn.net/v/t39.30808-6/251462563_589391922179367_3404795161537047489_n.jpg?_nc_cat=102&ccb=1-5&_nc_sid=dbeb18&_nc_ohc=RwrZUrg-A40AX9HTGtP&_nc_ht=scontent-frt3-1.xx&oh=6bda2dcd83530229564146c594f24d41&oe=61BA7C19
  @width 0.4
\fi

\iusr{Любовь Лысенко}

\ifcmt
  ig https://scontent-frx5-2.xx.fbcdn.net/v/t39.1997-6/p480x480/91629713_1030933093969494_5303695177838231552_n.png?_nc_cat=1&ccb=1-5&_nc_sid=0572db&_nc_ohc=wMTJpKm3CykAX_Sguve&_nc_ht=scontent-frx5-2.xx&oh=a209593b946eced41767a5a5ad1fd451&oe=61BBC3A6
  @width 0.2
\fi

\iusr{Ольга Патлашенко}

Татьяна, твой замечательный пост о Березняках вызвал такую живую дискуссию.
Столько позитива, приятных эмоций, воспоминаний... Спасибо огромное

\begin{itemize} % {
\iusr{Татьяна Сирота}
\textbf{Ольга Патлашенко} . @igg{fbicon.hands.pray}{repeat=3} 
Оленька, я очень рада, что пост понравился!

\iusr{Ольга Патлашенко}
\textbf{Татьяна Сирота}

\ifcmt
  ig https://scontent-frx5-2.xx.fbcdn.net/v/t39.1997-6/s168x128/106169513_1001060243683013_4196035040412806638_n.png?_nc_cat=1&ccb=1-5&_nc_sid=ac3552&_nc_ohc=R2UPzizWxRYAX8LAhGb&_nc_ht=scontent-frx5-2.xx&oh=09acd221d9b199e66863403326664edc&oe=61BA6734
  @width 0.2
\fi
\end{itemize} % }

\iusr{Лана Кор}
Очень интересно и немного грустно, что настоящий талант подчас оставался невостребованным и невознагражденный

\iusr{Ирина Иванченко}

Сто лет там не была, как- то ,,ветер моих странствий" не дул в ту сторону,
спасибо, Танюша, за прогулку, очень мило.

\iusr{Саша Ищенко}

У этого района есть и другая сторона я имею ввиду наркотики
@igg{fbicon.syringe}{repeat=3} я сам родился на Березняках и прокололся 12 лет
пока не сехал от туда атак впринцыпе район красивый.

\begin{itemize} % {
\iusr{Татьяна Сирота}
\textbf{Саша Ищенко}
Разве только Березняки страдают от этой напасти?
А Вы молодец! Справились с этой заразой!
\end{itemize} % }

\iusr{Gary Sorokin}
У меня гараж был 4 этажный там первый возле жел дороги

\iusr{Ирина Иващенко}

Моя любовь ..мой район детства... коньки зимой на замершем тельбине... 81 школа
..потом 30... а потом работа в 228... забирала брата со 191 и бежали купаться и
уроки учить на тельбине... целыми днями на улице... красота... пешком можно было
прогуляться до метро левобережки...

Золотое детство ...наверное потому что было спокойно ...и молоды были..

Люблю березики!!!

\iusr{Eric Cantona}
Бурбулаевка левобережная

\iusr{Елена Ходорко}

...а ещё на березняках жил Народный художник Украины Грох Н. Н., которого не
стало, увы... в октябре 2021 года (мой сосед по подъезду)... Светлая память....

\iusr{Lana Mitnitsky}

Как всегда, с большим удовольствием прочитала ваш рассказ о знаменитых людях,
чья жизнь была связана с Березняками, ну и фотографии вернули меня в мою
молодость. Оттуда, с Серафимовича 13/2, переехал ко мне муж. Родители его жили
там до самой смерти и мы часто навещали их.

\iusr{Elena Shevtsova}

Родина с 2 лет до 32 лет потом иммиграция в Канаду тут 23 года каждый год на
Березняках

\iusr{Ирина Иващенко}
\textbf{Elena Shevtsova} 

леночка шевцова ...привет.. увидела в твоем профиле танюху жураховскую... девочки
...жду возможно встречи.... @igg{fbicon.tropical.drink}  @igg{fbicon.ice.cream} 

\iusr{Нина Гордийчук}
Спасибо за интересный рассказ.

\iusr{Larisa Factorovich}

1. Березняков не знаю. Детство на чкалова, потом Отрадный и затем уже не в
Киеве

То что «За двоив зайцами» - классика, это ясно!

Но одно предложение

«То титка Секлета просила Вам передати, Проня Прокопивна, що ви падлюка!»

Украинский характер с лукавством, юмором, и красотой языка!

Всегда считала что украинский язык очень полно передаёт музыкальность, юмор и
такую нежность (матусенька, риднесенька и т д)

Извините что пишу по русски - нет украинского алфавита

\iusr{Лариса Манзюк}
Спасибо

\iusr{Ирина Лейбман}
А ещё там жил драматург Александр Каневский - брат знаменитого актера

\iusr{Андрей Шубин}

Прекрасный микрорайон Киева, который еще не очень тронула рука строителей, но у
них же до все красивых и комфортных мест для жизни руки дойдут, ведь правда?
Даешь везде Харьковский и Теремки - каменные джунгли наше усе(

\begin{itemize} % {
\iusr{Татьяна Сирота}
\textbf{Андрей Шубин} По поводу \enquote{прекрасный район} - согласна на  @igg{fbicon.100.percent} .
По поводу \enquote{руки дойдут} - думаю,что не очень \enquote{дойдут}, так как на Березняках особо строить негде.
И это радует.

\iusr{Андрей Шубин}
\textbf{Татьяна Сирота} Даже там где не где - они находят, но вот поверьте, мне
тоже сильно бы этого не хотелось

\iusr{Андрей Шубин}
\textbf{Татьяна Сирота} ну что еще немного спасает - дальность от метро, так бы
они на голове построили, окно в окно и не смутились бы

\iusr{Олена Чепега}
\textbf{Татьяна Сирота} . 

Я на Русановен живу- тоже курорт. И вот с ужасом думаю, что кому в голову
придёт какой-то дом сделать аварийным, а дальше все по схеме. Через канал от
нас уже наступают во всю новые дома. А места здесь злачные. Одно хочу, спокойно
быть здесь до конца

\iusr{Татьяна Годынская}
\textbf{Олена Чепега} 

так Русановку и так застройки уплотнили. И красный дом впихнули по центру. Хотя
набережную облагородили, кафешки разные. ( кроме бывшего туалета - старожилы
поимут.) Только уж больно тесно там стало... Краков жалко... был замечательный
кинотеатр, потом - злачное место, хорошо, что теперь там хоть театр открыли.
Родной район, мое детство юность... взрослые годы... теперь я там редкий гость.

\end{itemize} % }

\iusr{Наташа Хлизова}
Березняки и Тельбин это чудесное место в Киеве, одно из самых любимых и родных!

\ifcmt
  ig https://scontent-frx5-2.xx.fbcdn.net/v/t39.1997-6/s168x128/93025159_222645582401279_8207007965157261312_n.png?_nc_cat=1&ccb=1-5&_nc_sid=ac3552&_nc_ohc=GMY8RvQrrmcAX_f6FQ3&_nc_ht=scontent-frx5-2.xx&oh=d71a8596c70c4d38d561615e337d7c9d&oe=61BB5F42
  @width 0.2
\fi

\iusr{Светлана Ралло}

А на Бучмы 6-в жила я, 23 года, частенько видела Борислава, идущим по улице, а
в детской поликлинике с его женой, симпатичной блондинкой, сидели с детьми на
приём к одному педиатру... С Маргаритой Васильевной и её мужем нас познакомила
кардиолог из взрослой п-ки... не забываемые годы общения с хорошими людьми...
Есть, что вспомнить...


\iusr{Виктория Кохан}
А ещё жил Алим Владимирович Федоринский (\enquote{В бой идут одни старики} ).
Дата рождения
14.10.1941 (80 лет)
Место рождения
Кирнасовка (Украина)
Образование
Киевский национальный университет театра, кино и телевидения имени И. К. Карпенко-Карого.

\iusr{Всеволод Цымбал}
Имел счастье быть у Маргариты Васильевны дома

\iusr{Татьяна Сидорук}
Спасибо огромное!

\ifcmt
  ig https://scontent-frt3-1.xx.fbcdn.net/v/t39.30808-6/254552684_356980499558752_6237882115606043385_n.jpg?_nc_cat=102&ccb=1-5&_nc_sid=dbeb18&_nc_ohc=sf11EzDoKbcAX8JTGQ0&_nc_ht=scontent-frt3-1.xx&oh=00_AT9CtTAkz7qkek0ANPsJJ486pzSibgfWICjchwIlLcEOdw&oe=61BA8CDD
  @width 0.4
\fi

\iusr{Юрий Павленко}

На Амвросия Бучмы 8, не 16-ти, а 17-ти этажка!.. это конечно не меняет
\enquote{исторического} значения, но ведь истина - дороже  @igg{fbicon.wink} 


\iusr{Татьяна Сирота}
\textbf{Юрий Павленко} Пусть будет 17 @igg{fbicon.wink} 

\iusr{Ирина Иващенко}

Леночка шевцова каждый год приезжает?? Ждем встречи... вообще отзовитесь
одноклассники!!

10 -а.. закончил школу 30. в 1984 году.. последний год директора Бащинского... меня
долго не было на украине... сейчас живу почти на березняках родных... из
однокласников закончивших со мной 8 классов и ушедших из школы только андрей
денисенко... @igg{fbicon.clinking.glasses} куку однокашники!! Встречи не было вообще... скоро бабушки и
дедушки не узнают друг друга... знаю что многие покинули страну давно...

\begin{itemize} % {
\iusr{Elena Shevtsova}
Приезжаю и мы с 10б встречаемся присоединяйся

\iusr{Ирина Иващенко}
\textbf{Elena Shevtsova} 

здорово.. лет 35 уже жду и никого не видела давно... с огромным удовольствием
...!! Если пригласите!!!! Уже мечтаю... здоровье правда уже не то... но душа
молодая...!! Спасибо огромное... настроение на встречу.. не пропадайте..


\iusr{Elena Shevtsova}
Напишу если будет встреча

\iusr{Ирина Иващенко}
\textbf{Elena Shevtsova} предупредите хоть заранее... может помощь в организации встречи... кого нибудь подтянем...
\end{itemize} % }

\iusr{Ирина Иващенко}

Спасибо огромное за интересные и содержательные истории про любимый киев... вы
исключительная и замечательная бабушка... мечтаю составить вам компанию со своми
внуками... это очень важно ..здорово и правильно.. очень нужна
компания... боагодарю заранее @igg{fbicon.hands.pray}  @igg{fbicon.hands.raising} 

\begin{itemize} % {
\iusr{Татьяна Сирота}
\textbf{Ирина Иващенко} Ирина,Вы на Березняках живёте?

\iusr{Ирина Иващенко}
Здаю там квартиру свою... часто бываю... живу сейчас на троещине... часто езжу
к дочке на харьковский... все рядом!!


\iusr{Татьяна Сирота}
\textbf{Ирина Иващенко} Ирина, если захотите прогуляться в нашей с внуком компании - пишите в личку
\end{itemize} % }

\iusr{Natta Sky}

Большое спасибо. Но в этом доме жили не только большие артисты, также и
спортсмены, которые защищали честь Советского спорта.

Спасибо большое за статью, душевная и тёплая очень. С Брониславом Брондуковым
и его семьей была лично знакома. Восстанавливала его после первого
перенесённого инсульта. Замечательный был человек и очень скромный. Ещё раз
благодарю


\iusr{Наталья Борчанинова}

Я знала Березняки ще з початку 60-х років! Бувала на дачі на березі оз. Тельбін
у бабусі своєї подруги.

В 80-х жила на вул. Березняківська 30-а (майже на місті тієї дачі), інколи на
вихідних днях зустрічалася в чергах магазинів з Бориславом Брондуковим.

\iusr{Greg Trosman}
\textbf{Наталья Борчанинова} подскажите как ввести украинский язык в Фейсбуке, хочу користуватися.

\iusr{Лідія Самоха}

\ifcmt
  ig https://scontent-frx5-1.xx.fbcdn.net/v/t39.1997-6/p370x247/11057010_872065779521933_915357462_n.png?_nc_cat=110&ccb=1-5&_nc_sid=0572db&_nc_ohc=01R2UJaW8xUAX-t8nHK&_nc_ht=scontent-frx5-1.xx&oh=683d79390212ac6692308d6f3f71d152&oe=61BADF68
  @width 0.2
\fi

\iusr{Алексей Гришевскийй}

Очень люблю Киев. Постоянно на связи с друзьями. Помню многие старые районы,
тихие улочки. Мои живут в Святошино, на Оболони, на Бальзака, на Татарке. К
сожалению не могу посещать Киев до коренных перемен там. Я из КРЫМА. @igg{fbicon.face.sad.but.relieved}{repeat=3} 

\iusr{Влад Гро}

Украинское плэтическое кино - это миф, которым прикрывали нудоту и серость
фильмов.

Почему то брондуков раскрывался в фильмах других студий.

\iusr{Natalia Yatsunenko}

\ifcmt
  ig https://i2.paste.pics/9b2306a66be4645ead3b2ed0a0d32c40.png
  @width 0.2
\fi

\iusr{Мария Глухова}

Березняки найкращий район міста Києва. Багато зелених насаджень, озеро Тельбін,
Русанівський канал з красивими фонтанами, річка Дніпро, чудові для прогулок
набережні, сквери. Курортне місце. А також на Березняках проживали такі
персонажі, як \enquote{Штепсель та Тарапунька}, \enquote{Кролики}. Люблю Березняки.


\end{itemize} % }
