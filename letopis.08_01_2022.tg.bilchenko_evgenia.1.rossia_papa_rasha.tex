% vim: keymap=russian-jcukenwin
%%beginhead 
 
%%file 08_01_2022.tg.bilchenko_evgenia.1.rossia_papa_rasha
%%parent 08_01_2022
 
%%url https://t.me/bilchenkozhenya/5616
 
%%author_id bilchenko_evgenia
%%date 
 
%%tags rossia,ukraina,travlja,bilchenko_evgenia,odin_narod,rusmir,semja
%%title БЖ. Мне помогла Россия, папа. Не Rasha
 
%%endhead 
\subsection{БЖ. Мне помогла Россия, папа. Не Rasha}
\label{sec:08_01_2022.tg.bilchenko_evgenia.1.rossia_papa_rasha}

\Purl{https://t.me/bilchenkozhenya/5616}
\ifcmt
 author_begin
   author_id bilchenko_evgenia
 author_end
\fi

БЖ. Мне помогла Россия, папа. Не Rasha.

И сказал: посему оставит человек отца и мать и прилепится к жене своей, и будут
два одною плотью... (Мф.19:5,6).

Наверное, нельзя это писать, потому что это - слишком личное, но я попытаюсь
быть максимально сдержанной. Настолько, насколько позволяет совесть. Я полагаю
этот пост поможет украинцам, которые переехали в Россию и пережили не только
потерю страны, но и потерю близких. Только ради этого.

Не так давно я написала пост \enquote{Мы все уронили} о том, что, может быть,
правы скептики, которые говорят, что на Украине царит такая атмосфера ненависти
к русским, что ходить туда и кого-то освобождать не стоит, потому что идти
придется по живым людям. Я-то скептиков не жаловала, ибо считала, что они тоже
подыгрывают Западу. 

Но циничная мысль: \enquote{Не стоит!} - всякий раз приходят ко мне после
захода в Фейсбук. Мигранты из Украины, если хотите никогда не ностальгировать
по малой родине, - зайдите на страницу Бильченко в ФБ.

Я так много лет отстаивала тезис о братском народе, я ТАК в него верила, что
полагала, что надо бы России ещё побороться за Украину. По факту, я повторяла
слово в слово, статью ВВП (нет, не агент). Только я это все говорила намного
раньше, и мне вообще показалось, что это я писала эту статью. Впрочем, нет, я
говорила ещё эмоциональнее. 

Русские скептики типа Ростислава Ищенко все время боятся, что я из покаявшейся
сторонницы майдана превращусь в рупор русской весны или что-то подобное. Во мне
действительно много русской весны по части единой Руси, одного СССР. Дедушкино
воспитание пальцем не раздавишь. И товарищей своих не предашь. Выбор сделан.
Где-то Ищенко прав, наверно.

Все самое прекрасное весеннее во мне кончается, когда я захожу в этот проклятый
Фейсбук. Да, я знаю, вы скажете, что там - \enquote{боты}, что не все украинцы - такие
звери, я соглашусь. Не все. Но, когда среди ботов узнаешь лица друзей детства,
бывших коллег на работе, вахтерш в университетах и прочих \enquote{ребят из нашего
двора},  становится по-настоящему СТРАШНО. 

А, когда в дружном ряду ботов я узнаю отца, который не жил со мной и ни разу за
время адской травли даже не позвонил мне, то возникает сомнение в адекватности
соотечественников. 

Сначала вылазит очередной ноунейм. Там много всякого интимно грязного, приведу
лишь пристойные для читателя отрывки:

Ноунейм: \enquote{Я в шоке. Перетащила из Киева все свое потрепанное барахло, включая
исторический затертый и замызганный дедушкин коврик неопределенного от грязи
цвета; дешевые книжки за пять копеек сотня, за которые еще и пошлину на таможне
заплатила. Теперь донаты собирает с фанов, чтобы на тортик к Рождеству
набрать... Зато с \enquote{родными} - \enquote{ты, да я, да мы с тобой} последователями...
печалька}.

Отвечаю: да, ноунейм, я горжусь дедушкиным затертым ковриком. Это не хлам - это
храм. Меня дедушка воспитал так. Да, у меня много книг, они все - старые, но не
антиквар, потому дешёвые: Пушкин, Лермонтов, Толстой, Достоевский, Бунин... У
моей бабушки была очень большая библиотека русской и зарубежной классики. Я на
ней выросла. Меня воспитали в трепетном уважении к книге. Это, наверное,
старомодно, но для меня книга - это святое. Я редко об этом говорю, но
пришлось. Да, я уважаю законы РФ и за свое \enquote{барахло} заплатила пошлину,
превышающую стоимость книг в разы. Наверное, это в системе ваших ценностей
называется \enquote{лошара}. Хорошо, что я честно собираю донаты своим творчеством:
нельзя меня упрекнуть, что Кремль возит меня на черном воронке и я приехала в
Россию за выгодой. Мои родные - это русские люди. Замечательные, лучшие,
русские люди, преследуемые на Украине.

Ноунейм обращается напрямую к моему отцу: \enquote{Простите, а Вы, случайно, не отец
Евгении? Если да, то я Вам сочувствую. У меня тоже дочка Женечкиного возраста,
но я даже в страшном сне не могла бы себе представить, чтобы она так
привселюдно позорила себя и семью... Можно было бы не обращать на нее
внимания... Но она пытается влиять на общественное мнение, а это уже гораздо
опаснее, поскольку ее поведение порождает у молодежи желание следовать данной
тупиковой модели поведения...}

Отвечаю: что позорного в том, что русский поэт любит Россию и ни за что не
отдаст свою большую Родину на съедение миру? 

Но это - ещё полбеды. Дальше - больнее. Включается мой папа: \enquote{Я не Тарас
Бульба... но иногда прорывает... и что? Помогли тебе твои Раши?}

Мой папа - наполовину русский... Отвечаю ему словами из песни 25/17:
\enquote{Для вас Россия - Раша, для нас - Мария, Маша}. Да, папа. Россия у
человека в сердце - одна, и, независимо от того, фиговая она или нет, - она
Самая Прекрасная в Мире. Потому что нет никого прекраснее нее. Точка.

Никогда не ранят чужие - ранят свои. Никогда не ранят нацисты - ранят манкурты. 

Вернёмся к теме. Если это правда и украинский народ массово сошел с ума, кого
там спасать? Оставшихся в живых мучащихся единиц? Или все же есть ещё люди, не
потерявшие чести и любви? 

Учтите, что все, что я здесь написала, висит в моем паблике в ФБ, а не из
личной переписки. Я не Павлик Морозов и личное НИКОГДА бы не обнародовала:
родитель сам вылез в публичное пространство осудить дочь на всенародном суде
украинской нации. Такой советский в худшей версии вариант. 

Может, теперь многим станет ясно, что я реально вылазила  не из мифического
майдана, а из близких, средних и дальних кругов своего жизненного пространства.
Понимаете, ЧЕРЕЗ ЧТО я прошла? Впрочем, это неважно, это не попытка
разжалобить. Сама захотела - сама прошла.

Завершаю, уважаемые читатели. Ноунейм сказала, что я влияю на общественное
мнение и призываю украинскую молодежь к определенным моделям поведения. 

Что ж, что ж... Сделаю небольшое обращение:

\begin{zzquote}
"Дорогая украинская молодежь! Если ты меня ещё слышишь, пожалуйста, верь мне.
Россия - прекрасна. Да, не все в этой стране идеально, как и в любой иной. Да,
у мигрантов случаются срывы и неудачи. Но она прекрасна по двум причинам: по
сравнению с Украиной - она свободная, по сравнению с Европой - родная. Если
воля и покой соединяются, ты обретаешь счастье. 

Делайте не так, как я. Это значит: не грузите Россию собой сразу, она этого не
любит. Тут считают: \enquote{Насрали - сначала уберите за собой, а потом к мамке за
конфетами}. Ощутите этот посыл. 

Уберите, блин, за собой для начала на месте. Я убирала пять лет, последний год
- мясом и кровью, мой отъезд - прямое следствие того, что моя искупительная
деятельность на Украине исчерпала себя лишением меня моего призвания и
предназначения - учить вас добру. Здесь я ещё надеюсь как-то прорваться на
Донбасс, здесь это в перспективе реально ценой полной потери Украину, ну, и
родителей, потому что они не поймут мой выбор.

Из меня сделали фрика благодаря жесточайшей травле, которой не подвергались
даже Зощенко и Ахматова в СССР. Потерять  жизнь я боялась гораздо меньше, хотя
такая угроза в 2021 году стала явью. Меня просто пытались уничтожить морально,
но я выстояла, я ушла от них. Стойте за свою совесть. 

Стойте за Украину и Россию, не верьте тому бреду, которому вас учат дома и в
школе".
\end{zzquote}

Немного пафосно, простите. И репостните. И пишите мне, если вам на пути из
Украины в Россию пришлось пройти сквозь общественную травлю и потерю родных и
близких. Помогу, как смогу, словом.

PS. Наши в Казахстане. Ура, хоть так.
