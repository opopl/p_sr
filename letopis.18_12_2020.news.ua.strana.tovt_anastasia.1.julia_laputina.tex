% vim: keymap=russian-jcukenwin
%%beginhead 
 
%%file 18_12_2020.news.ua.strana.tovt_anastasia.1.julia_laputina
%%parent 18_12_2020
 
%%url https://strana.ua/news/307379-julija-laputina-ministr-po-delam-veteranov-biohrafija-henerala-sbu-deklaratsija-o-dokhodakh-muzh-vorotnjuk-.html
 
%%author Товт, Анастасия
%%author_id tovt_anastasia
%%author_url 
 
%%tags 
%%title Воевала в рядах СБУ против УПЦ, Гоголя и Булгакова. Чем известна министр по делам ветеранов Лапутина
 
%%endhead 
 
\subsection{Воевала в рядах СБУ против УПЦ, Гоголя и Булгакова. Чем известна министр по делам ветеранов Лапутина}
\label{sec:18_12_2020.news.ua.strana.tovt_anastasia.1.julia_laputina}
\Purl{https://strana.ua/news/307379-julija-laputina-ministr-po-delam-veteranov-biohrafija-henerala-sbu-deklaratsija-o-dokhodakh-muzh-vorotnjuk-.html}
\ifcmt
	author_begin
   author_id tovt_anastasia
	author_end
\fi

\ifcmt
  pic https://strana.ua/img/article/3073/julija-laputina-ministr-79_main.jpeg
  caption Юлия Лапутина в зоне АТО 
  width 0.5
\fi

\index[names.rus]{Лапутина, Юлия!Украина!министр по делам ветеранов, 18.12.2020}

Сегодня, 18 декабря, Верховная Рада утвердила Юлию Лапутину на должность
министра по делам ветеранов.

Это одна из немногих в Украине женщина-генерал. Она служила в СБУ, участвовала
в АТО, а потом занималась информационной кампанией против России и Украинской
православной церкви, разъезжая с "передвижной выставкой" агитплакатов по
Донбассу. 

Генеральские погоны Лапутиной вручал лично президент Зеленский. 

"Страна" рассказывает, что известно о новой министерше.  

\subsubsection{Биография Юлии Лапутиной}

Юлии Лапутиной 53 года, она уроженка Киева. Ее покойный отец - академик
Анатолий Лапутин - специалист по атлетической гимнастике, был в числе первых
исследователей допинга в СССР.

У министра по делам ветеранов два образования - педагогическое и
психологическое. По второй специальности есть ученая степень кандидата наук.  

Почти всю жизнь она работает в органах, куда пришла в 1992 году. Службу начала
опером СБУ в подразделении контрразведки. 

В течение 2010-2012 годов занимала должность замначальника Центра специальных
операций "А" (Альфа) СБУ. Участница АТО. С сентября по декабрь 2014 года была
командиром группы СБУ на Донбассе, которая задерживала сепаратистов. 

\subsubsection{Пропаганда против России и УПЦ}

После АТО перешла на должность замначальника Департамента контрразведывательной
защиты интересов государства в сфере информационной безопасности СБУ.

Ее задача, среди прочего, заключалась в том, чтобы готовить материалы по
запрету Вконтакте и Яндекса. Также она читала лекции о "методах ведения
информационной войны России против Украины и участии в ней Украинской
православной церкви Московского патриархата". 

Вот с такой выставкой плакатов Лапутина ездила по Донбассу. На одном из
плакатов, которые она привезла три года назад в Покровск, уже до 2013 года в
Украине шла "информационная агрессия РФ". Она заключалась в "усилении
ностальгии за жизнью в СССР, единством народов бывших советских республик". 

Как примеры такой агрессии СБУ устами нового украинского министра называла
фильмы "Тарас Бульба" и "Белая гвардия", снятые по произведениям Гоголя и
Булгакова. Оба писателя - уроженцы Украины.

Также "агрессивным" Лапутина почему-то признала фильм "Стиляги", где вообще-то
критикуется советский строй и украинцы вообще никак не упоминаются. 

\ifcmt
pic https://strana.ua/img/forall/u/0/92/institut_SBU_03[1].jpg 
\fi

В 2018 году будущая министр тоже во время поездки в Покровск рассказывала, как
СБУ изучает литературу Украинской православной церкви на предмет призывов к
сепаратизму. И сообщила, что монахи Святогорской лавры прячут за иконами
"пророссийские молитвы". 

"Расследование (деятельности УПЦ - Ред.) проводится путем научных экспертиз,
выявление документов, свидетельства людей, которые видели или были участниками
мероприятий, которые осуществляла УПЦ МП. Осуществляется опрос свидетелей и
сбор доказательств, исследуется церковная литература на предмет наличия
манипулятивных текстов", - говорила она.

На идеологической ниве Юлия трудится давно. В 2008 году она предлагала
концепцию "ментальной интеграции крымчан в Украину". А уже с началом войны на
Донбассе писала публикацию на тему: "Манипуляция понятием гендерного равенства
- текущий тренд когнитивной войны".

Отметим, что ее продвижение на столь высокий пост комментаторы также называют
именно "гендерным назначением".

\subsubsection{Генеральские погоны от Зеленского}

В 2020 году от Владимира Зеленского Лапутина получила звание генерал-майора.
СМИ пишут, что она то ли первая, то ли вторая женщина-генерал. На самом деле
это не так. Подобное звание в Украине для представительниц прекрасного пола
редкое, но не настолько. 

Входит в состав Консультативного совета по делам ветеранов войны и семей
погибших. По совместительству - доцент кафедры Национальной академии СБУ. 

Юлия Лапутина замужем, воспитывает дочь Ольгу. Ее муж - полковник СБУ Сергей
Воротнюк, тоже участник АТО. 

Политические взгляды супругов оценивают не как умеренные, а как воинственные.
Что в общем-то видно и по общественной деятельности новой министерши. 

По некоторым данным, супруги дружны с заместителем секретаря СНБО Сергеем
Кровоносом, который был назначен при Порошенко. Недавно он отличился
высказыванием про Зеленого и Путина, где президента Украины сравнил с зайцем, а
главу России - с медведем. И указал, что "медведь все равно съест зайца" - так
Кривонос оценил шансы договориться с Владимиром Путиным.

\subsubsection{Дома в Козине. Чем владеет министр Лапутина}

Как для бюджетников, пусть и с хорошими должностями, семья Лапутиных-Воротнюков
обеспечена солидно. Информация следует из декларации министра за 2019 год. 

Юлия Лапутина вместе с родственниками владеет квартирой в Киеве площадью 165,8
метров. А ее мама - 700 метровым домов в элитном поселке Козине. Эта
недвижимость - не наследство от отца-академика, а была куплена в 2005 и 2007
годах.

Давно приобретенное имущество только у Воротнюка - 36-метровая киевская
однушка, собственником которой он стал в 1998 году. 

Сейчас семья тоже располагает доходами - строит новый дом в Козине площадью
322,5 метров. 

Автопарк не очень пестрый: "Фиат" 2016 года выпуска, "Смарт" 2001 года,
"Фольксваген" 2007 года. 

Зарплата Лапутина в СБУ - 540 427 гривен (45 тысяч гривен в месяц), мужа
Воротнюка - 403 185 гривен (по 33,5 тысяч в месяц). 

Деньги семья предпочитает хранить только наличными и только в иностранной
валюте. Лапутиной принадлежит 86 000 долларов, ее матери - 22 000 долларов,
мужу - 32 000 долларов. 

\subsubsection{\enquote{Сейчас везде назначают людей СБУ}. Что говорят ветераны }

Опрошенные "Страной" участники боевых действий удивлены назначением Лапутиной.
И усматривают в нем попытку усилить влияние СБУ на ветеранское движение. 

"Не знаю, у кого и как родилось это назначение. Отношение у бойцов к Лапутиной
разное. Долгое время вообще считали, что новым министром станет Александр
Порхун – герой Украины, десантник, бывший комбат 13 батальона 95-й Житомирской
десантной бригады. Он был замом у экс-министра по делам ветеранов Бессараба. В
кругу моего общения военные и добровольцы были уверены, что Порхуну предложат
возглавить министерство, так как у него уже есть опыт заместителя. Но у нас
сейчас везде назначаются люди, поработавшие в СБУ или на СБУ", – считает один
из участников боевых действий.

"Не знаю, за какие заслуги ей дали звание генерала. Может, готовили к министру.
Знаю, что она дружит близко с генералом Сергеем Кривоносом, замглавы СНБО. Еще
известно, что у нее в декабре 2019 погиб зять, Денис Волочаев, полковник ЦСО
"А" СБУ, герой Украины (посмертно). Громкое было событие", - рассказывает
собеседник "Страны" среди ветеранов.

Ветеран АТО Сергей Ульянов говорит "Стране", что для него назначение Лапутиной
также стало сюрпризом.

"Я очень удивился, что назначили Лапутину. Я раньше о ней ничего не слышал. И
не знал, что ее вообще рассматривали как кандидата в министры. У нее была очень
узкая специфика службы, и она непубличный человек. Буквально несколько дней
назад я впервые узнал о Лапутиной. До этого в кулуарах говорили, что собираются
назначить Александра Порхуна. Он неплохой чиновник, вменяемый, контактный,
понимающий. Но, знаете, после того как в министерство при Порошенко назначили
Ирину Фриз, потом – Оксану Коляду, а затем – Сергея Бессараба, мы уже ничему не
удивляемся", – отмечает ветеран.

При этом Сергей Ульянов отмечает, что почти никто из предыдущих руководителей
ведомства также не имел отношения к АТО.

"Единственно, что я знаю о Лапутиной, – она контрразведчица. Не знаю, как она
связана с боевыми действиями. Но Ирина Фриз тоже никакого отношения к боевым
действиям не имела. Оксана Коляда имела статус участника боевых действий, но
фактически она не участвовала в боевых действиях. Так или иначе, Лапутина
теперь имеет статус министра. Человек занимался достаточно серьезной службой –
контрразведкой. Как она проявит себя как топ-менеджер именно нашего
министерства – здесь будем смотреть. Пока рано делать прогнозы. Для меня, как
для военнослужащего, важен конечный результат. А все команды, которые приходили
в министерство до этого, к сожалению, результата не показали", – говорит Сергей
Ульянов.

Опрошенные "Страной" ветераны надеются, что новый министр продемонстрирует
открытость к диалогу и сможет объединить военное сообщество. А также решить
вопросы с финансированием по основным статьям расходов для помощи ветеранам.

"В первую очередь мы рассчитываем на открытость. Когда чиновники приступают к
своим обязанностям, они часто закрываются от той категории людей, которыми
должны опекаться. Так делали "попередники" в нашем министерстве. А проблем
накопилось очень много. Многие постановления недофинансируются в полном объеме.
Об этом пока публично не говорят, а тем временем трудности накапливаются, как
снежный ком. Например, недофинансируется постановление №280 Кабмина –
компенсации за жилье участникам боевых действий с оккупированных территорий. И
постановление №719 – жилье инвалидам первой и второй группы. Если, не дай Бог,
военнослужащий получил тяжелое ранение и группу инвалидности вследствие этого
ранения, то квартиру от государства раньше он получал в тот же год. А сейчас
уже очереди на три-четыре года вперед. Количество инвалидов увеличивается, а
финансирование остается на том же уровне. С этим нужно системно разобраться", –
объясняет Сергей Ульянов.

Ветеран АТО Андрей Бойко также говорит, что ничего не слышал о Лапутиной до
того, как она стала министром по делам ветеранов. 

"У меня по Лапутиной вообще нет информации. Вы первая, кто говорит мне о ее
назначении. Для меня это личность абсолютно неизвестная. По поводу ее
поддержки, в силу специфики ее профессии, вряд ли кто-то из ветеранов ее знает.
Это такая служба, где работают не слишком публичные люди. Судя по официальным
данным, она занималась информационной борьбой – это никакие не боевые действия.
На войне женщины, тем более в таких званиях, непосредственно не принимали
участие. И не должны. При мне даже простого сержанта снимали с "передка" в силу
ее гендерной принадлежности, женщин туда не пускали", – отмечает Бойко.

Однако уточняет, что для него не важно, насколько далека или близка была
Лапутина к боевым действиям. Главное – искреннее желание помогать ветеранам.

"По предыдущему министру сразу было понятно, зачем он пришел и что он хочет. А
вообще, ветераны же не продолжают участвовать в боевых действиях, и министру не
надо продумывать тактику и стратегию войны. Здесь важно проникнуться знанием
проблем ветеранов. Время покажет, насколько правильное это назначение", –
отмечает Андрей Бойко.

\subsubsection{\enquote{Юлия Анатольевна была на передовой с апреля 2014 года}}

Сергей Кривонос, заместитель секретаря Совета национальной безопасности и
обороны Украины, прокомментировал «Стране» назначение Юлии Лапутиной министром
по делам ветеранов.
 
Сергей Кривонос отметил, что лично знает Юлию Лапутину. И утверждает, что она
принимала активное участие в боевых действиях на востоке Украины.
 
Он также удивлен, что ветераны, опрошенные «Страной», не знают, какое отношение
Лапутина имела к боевым действиям.
 
"Мы знакомы с Юлией Анатольевной, сотрудничали с ней долгое время. Работали
вместе на востоке Украины во время войны с РФ. Я знаком с ней по фронту. И
многие бойцы тоже ее знают. Понятно, что ее не могут знать все ветераны, фронт
большой – 450 км шириной. Она физически не могла со всеми бойцами
познакомиться. Но Юлия Анатольевна была на передовой с апреля 2014 года – под
Славянском, в Краматорске. Первой, кто приехал на Краматорский аэродром после
деблокировки, была ее группа во главе с Юлией Анатольевной. Еще даже
Вооруженных сил не было, а она уже приехала. Она была на самом «передке», не
один и не два раза", – сообщил «Стране» Сергей Кривонос.
 
Сергей Кривонос описывает Лапутину как профессионала, умеющего объединять возле
себя людей и мотивировать их к работе.
 
"Я знаю Юлию Анатольевну как профессионального человека, который много сделал
для установления мира на Донбассе. Огромную работу она проводила во время своей
службы в СБУ. Очень профессиональная, систематичная, умеет объединять и
организовывать людей вокруг себя. У нас много тех, кто умеет говорить, а она
умеет работать. И ей многое предстоит сделать в Министерстве ветеранов.
Во-первых, решить вопрос со зданием для Министерства – пока что оно находится
на месте бывшего Госагентства по делам ветеранов и не может полноценно
функционировать. Во-вторых, нужно объединить патриотично настроенных ветеранов,
чтобы они чувствовали поддержку со стороны государства. Что они не брошены и не
забыты. И работа с молодежью, которой сами ветераны могут передать свои взгляды
на ситуацию в стране и ее защиту. Опытные волки будут учить молодых волчат
защищать свою стаю. Думаю, Юлия Анатольевна сможет организовать этот процесс.
Главное, чтобы была поддержка не только со стороны ветеранов, а и со стороны
руководства государства", – считает Кривонос.
