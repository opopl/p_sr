% vim: keymap=russian-jcukenwin
%%beginhead 
 
%%file slova.usa
%%parent slova
 
%%url 
 
%%author 
%%author_id 
%%author_url 
 
%%tags 
%%title 
 
%%endhead 
\chapter{США}

Сбежавшие из приюта в \emph{США} дети нашли в чужом доме автоматы и полчаса отстреливались от полиции,
strana.ua, 03.06.2021

\emph{Посольство США} в Киеве призвало власти Украины привлечь к формированию
Высшего совета правосудия (ВСП), Высшей квалификационной комиссии судей (ВККС)
и руководящих органов Национального антикоррупционного бюро (НАБУ) независимых
экспертов.  Пост на эту тему появился на странице посольства в Фейсбуке 21 мая
по итогам встречи представителей США и ЕС с председателем Верховной рады
Дмитрием Разумковым и заместителем главы Офиса президента Украины Андреем
Смирновым,
\citTitle{Посольство США рассказало, как защитить антикоррупционные органы Украины}, Наталья Полулях, strana.ua, 21.05.2021

Зараз українці, які продовжують вірити в пустопорожні заяви і декларації наших
західних партнерів, чимось нагадують африканських туземців в яких європейці
обмінювали золото та платину на скляні буси. Всі демократичні цінності
номіновані в доларах і євро. Недарма в чудовому фільмі \enquote{Пограбування казино}
було сказано: \enquote{ Зв'язковий мафії: Ти цинічна сволота, ти це знаєш? Обама по
телевізору: Утвердитись в фундаментальній істині, що нас багато, але ми єдині!
Зв'язковий мафії: Чув що він сказав? Замислись. Джек Коган: Не сміши мене \enquote{ми
один народ}. Цей міф створив Томас Джефферсон – американський святий, бо він
написав, що всі люди створені рівними. У що він явно не вірив, якщо залишив
своїх дітей підневільними. Заможній сноб, який не бажав платити податки
Британії. Він написав красиві слова і підняв людей, які пішли помирати за ці
слова. А він сидів, пив дороге вино і шпекав свою рабиню. (Вказує на телевізор
з Обамою) Він говорить мені, що ми живемо в суспільстві? Я живу в Америці, а в
\emph{США} кожен сам за себе. \emph{Америка} – це не країна, це лише бізнес. Тож трясця,
плати мені}, 
