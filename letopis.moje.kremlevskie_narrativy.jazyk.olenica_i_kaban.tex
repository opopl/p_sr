% vim: keymap=russian-jcukenwin
%%beginhead 
 
%%file moje.kremlevskie_narrativy.jazyk.olenica_i_kaban
%%parent moje.kremlevskie_narrativy.jazyk
 
%%url 
 
%%author_id 
%%date 
 
%%tags 
%%title 
 
%%endhead 

\paragraph{Басня 28 - Оленица и Кабан}

В польских и венгерских горах Оленица, увидев домашнего Кабана:

— Желаю здравствовать, господин Кабан, — стала приветствовать, — радуюсь, что
вас...

\ifcmt
  tab_begin cols=2,no_fig,center
     pic https://avatars.mds.yandex.net/i?id=b767ab5a0259a30724a5875691fb443d-5734795-images-thumbs&n=13
		 pic https://avatars.mds.yandex.net/i?id=2ff2220f615a06f4d4c10d80a5c7a521-4827941-images-thumbs&n=13
  tab_end
\fi

— Что ж ты, негодная подлость, столь не учтива! — вскричал, надувшись, Кабан.—
Почему ты меня называешь Кабаном? Разве не знаешь, что я пожалован Бараном? В
сем имею патент, и что род мои происходит от самых благородных Бобров, а вместо
епанчи для характера ношу в публике содраную с овцы кожу.

— Прошу простить, ваше благородие,— сказала Оленица,— я не знала! Мы, простые,
судим не по убору и словам, но по делам. Вы так же, как прежде, роете землю и
ломаете плетень. Дай бог вам быть и Конем!

Сила. Не можно довольно надивиться глупцам, пренебрегшим и поправшим премного
честнейший и бесценный добродетели бисер на то одно, чтоб продраться в чин,
совсем ему не сродный. Какой им змий в ухо нашептал, что имя и одежда
преобразят их в бытии, а не жизнь честная, достойная чипа? Вот точные граки
Эзоповы, одевающиеся в чужие перья. Из таковых сшитое жительство подобное
судну, в котором ехали морем одетые по-человечески обезьяны, а ни одна править
не умела. Если кто просвещенное око имеет, какое множество видит сих Ослов,
одетых в львиную кожу! А на что одеты? На то, чтоб вполне могли жить по рабским
своим прихотям, беспокоить людей и проламываться сквозь законов гражданских
заборы. А никто из достойных чести на неучтивость скорее не сердится, как сии
Обезьяны с Ослами и Кабанами. Древняя эллинская пословица: «Обезьяна обезьяною
и в золотом характере».

Вспоминает и Соломон о свинье с золотым в ноздрях ее кольцом (Притчи, гл. II,
стих 22). Знаю, что точно он сие говорит о тленных и бренных фигурах, в которых
погрязло и скрылось кольцо вечного царствия божия, а только говорю, что можно
приточить и к тем, от которых оное взято для особливого образования в Библию.
Добросердечные и прозорливые люди разными фигурами изображали дурную душу сих
на одно только зло живых и движимых чучел. Есть и в Малороссии пословица:
«Далеко свинья от коня». 

