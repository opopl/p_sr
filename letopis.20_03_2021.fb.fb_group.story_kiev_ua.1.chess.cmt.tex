% vim: keymap=russian-jcukenwin
%%beginhead 
 
%%file 20_03_2021.fb.fb_group.story_kiev_ua.1.chess.cmt
%%parent 20_03_2021.fb.fb_group.story_kiev_ua.1.chess
 
%%url 
 
%%author_id 
%%date 
 
%%tags 
%%title 
 
%%endhead 
\zzSecCmt

\begin{itemize} % {
\iusr{Alla Zinovych}

\ifcmt
  ig https://scontent-frt3-1.xx.fbcdn.net/v/t1.6435-9/162987467_804227950178912_6986750370145612985_n.jpg?_nc_cat=107&ccb=1-5&_nc_sid=dbeb18&_nc_ohc=AbD7H9R4OJQAX9vILpY&_nc_ht=scontent-frt3-1.xx&oh=00_AT9e_oOPHyIwcVRohHXKRSVKCWGOhc8UDgPpocbHMHKE5A&oe=61E4D165
  @width 0.4
\fi

\iusr{Alla Zinovych}
И летом, и зимой...

\iusr{Вадим Вересюк}
Только и слышно, мат, мат..

\begin{itemize} % {
\iusr{Михайло Наместник}
\textbf{Вадим Вересюк}, дык, не надо ж лошадью ходить куда ни попадя!

\iusr{Ирина Иванченко}
\textbf{Михайло Наместник} , и слоном...

\iusr{Игорь Мезецкий}

Да..., было время! И сам когда-то возле КГУ поигрывал... В центре и справа,
если стоять спиной к КГУ, все лавки были заняты, а некоторые сдвигались вместе
и подойти из-за скопления \enquote{болельщиков} было сложно... @igg{fbicon.thinking.face} 

\end{itemize} % }

\iusr{Vicki Seplarsky}
Гроссмейстер Штейн был из Киева.

\begin{itemize} % {
\iusr{Юрий Безродный}
\textbf{Vicki Seplarsky} Штейн родился в Каменец-Подольском

\iusr{Marianna Shebotnova}
\textbf{Vicki Seplarsky} Леонид Штейн из Львова.

\iusr{Vicki Seplarsky}
\textbf{Marianna Shebotnova} он жил и работал в Киеве, я работала вместе с его женой.

\iusr{Vera Faynberg}
\textbf{Marianna Shebotnova} Совершенно верно! Гроссмейстер Эдуард Гуфельд родился и вырос в Киеве, потом уехал в Тбилиси.
\end{itemize} % }


\iusr{Нина Черадионова}
Там такая трогательная обстановка, что хочется опять радоваться жизни, как в детстве.

\iusr{Bob Voronkov}
ПодзабьІл ...
в Октябрьском : Спасский - Корчной ?...

\iusr{Ольга Лубягина}

\ifcmt
  ig https://paste.pics/8a352974c11bf55ce87ada118ff91791
  @width 0.2
\fi

\iusr{Таня Сидорова}
Батько моєї подруги там грав.

\iusr{Татьяна Грицай}

Конечно, помню эту атмосферу соперничества и драйва борьбы. Мой муж не имел
титулов, но играл неплохо, а я в сторонке за него болела. Как правило, он
выигрывал несколько партий и мы шли обедать в ресторан «Украина», а вечером в
кино.

\iusr{Valentyna Venglovska}

...Коробка шахматной доски распахивалась как дверь в прекрасное будущее, и в
Советском Союзе несть числа было шахматным кружкам и секциям...

\iusr{Valentyna Venglovska}
Ключевое слово \enquote{было}...

\iusr{Евгения Зайончковская}
Алешин папа в 10м классе выиграл у Корчного.

\iusr{Виктория Угрюмова}
\textbf{Евгения Зайончковская} феерическая история)))) а хоть кусочек написали бы? А?

\iusr{Наталя Кудря}
не сильно то на любителів розраховуйте- там менше першого ....
@igg{fbicon.wink} хто не вірить - сідайте грати - на гроші..

\iusr{Татьяна Зубко Маркина}
Парк Шевченко, сколько раз проходила мимо игроков. Казалось, что они живут там

\begin{itemize} % {
\iusr{Виктория Угрюмова}
\textbf{Татьяна Зубко} вероятно, так оно и есть - отчасти. Там проходит большая часть их жизни. А в Мариинском -
\end{itemize} % }

\iusr{Виктория Угрюмова}

В Мариинском у шахматного павильона было дерево, на котором сидела сова. Она
сидела не в дупле - а у дупла. И когда грохотали шахматами по доске - открывала
обиженно глаза и подхмыкивала

\iusr{Неонила Сваток}
\textbf{Виктория Угрюмова} Вот это да!!!

\iusr{Анна Загорулько}

Постоянно была свидетелем этих турниров в период учебы в КГУ да и после, хотя
шахматами не увлекалась. Меня больше привлекали растения, особенно весной.
Основные фишки парка Шевченко - магнолии и декоративные райские яблони, но их
красота заслуживает не коммента, а отдельного поста

\iusr{Irina Somova}

Странное противопоставление \enquote{не Пушкин, а .......}
Радует одно, что советская школа всех научила писать....
А место действительно прекрасное!

\begin{itemize} % {
\iusr{Alexander Khalaim}
Зато Facebook уравнял всех! Здесь даже самый безграмотный может активно участвовать, ставя лайки и смайлики!  @igg{fbicon.wink} 
\end{itemize} % }

\iusr{Valeriy Yanovskiy}

\ifcmt
  ig https://scontent-frx5-2.xx.fbcdn.net/v/t39.1997-6/s168x128/104115671_1000342980421406_8422357867973340691_n.png?_nc_cat=1&ccb=1-5&_nc_sid=ac3552&_nc_ohc=RMqkMjOuPWsAX9yKokX&_nc_ht=scontent-frx5-2.xx&oh=00_AT9hcXcjIGtxYJ7uRLeSchTEe-hLd0v3q_WuMTgLrXhpGQ&oe=61C192C5
  @width 0.1
\fi

\iusr{Alexander Khalaim}

Кто помнит, в советское время, в начале ул. Володарского (слева от Цирка) был
даже шахматный клуб!

\end{itemize} % }
