% vim: keymap=russian-jcukenwin
%%beginhead 
 
%%file letters.mariupol.demidko_olga.2.ohmatdyt
%%parent letters.mariupol.demidko_olga
 
%%url 
 
%%author_id 
%%date 
 
%%tags 
%%title 
 
%%endhead 

%07:52:42 17-06-23

оце от Ви напевне дивуєтесь... чому я написав про себе в моїх листах до Вас якось як про лікаря...

а справа то в тому...

що крім того що я фізик та програміст...

в дитинстві я також цікавився медициною...

але якось не склалось...

також... у мене тітка моя Таня була медиком...

а сестра Віка фармацевт... короче, щось в тому є...

але декілька років тому мені в голову влізла в голову одна божевільна думка...

а чому б мені не поступити на мед...

я ж походами займаюсь... а в походах знаєте...

все треба зараннєє готувати... і в горах розраховувати тільки на себе...

ну от... хотів медицину підучити, щоби якщо шо на місці все знати що і як робити...

але зрозуміло, що це все було дуже наївно...

але тим не менш... я в дорослому вже віці двічі складав ЗНО, і з другого

разу склав ЗНО на потрібний бал - українська мова + математика + біологія + хімія

щоби поступити на контракт на медичний університет Богомольця - НМУ тобто...

але... от поступив... але потім стало зрозуміло... що я не можу виділити стільки часу...

хоча б на один рік навчання... мені в принципі диплом не цікавить... хотілось просто от

знаєте... повивчати... в анатомічке в трупах покорписатись... що там як влаштовано...

цікаво ж дуже!!!

ну от... здав я ЗНО... потім...

мене зарахували на НМУ на перший курс на контракт... ну в принципі конкурсу на контракт

навіть на один із топових вишів по медицині... так... конкурсу такого немає як на бюджет...

ось на бюджет звісно так йде шалена боротьба так!!! але... на бюджет в Україні можна провчитись тільки один в житті...

бюджетне навчання в Україні... це як знаєте... перший поцілунок із дівчиною... колись було... але добре памьятаєш як воно було...

блін... і я тут думаю... чесати собі голову... починаю рахувати свої гроші...

треба кинути роботу... ооооооойййййй нашо я всім цим зайнявся...

цілий рік от значить сидіти слухати лекції короче знову студентом стати...

але якось не сходиться!!! бо значить грошей було явно недостатньо

щоби ось так цілий рік побути студентом як би мені того не хотілось би...

ну і короче... мої документи там дотепер лежать в НМУ - шкільний сертифікат... все таке...

я до них заходив... кажу... такі справи... вчитись у вас я не буду на контракті... бо хотів...

але зрозумів що не сходиться... ну... вони пожали плечами... але кажуть... треба короче

весь університет обійти позбирати бумажки... обхідний лист короче... я такий, аааааааа!!!!!!

ще тут обходити все на кампусу... облом!!! і от так... короче, мій шкільний сертифікат дотепер лежить в НМУ

ім. Богомольця, той, що біля метро Шулявська, і поруч також Зоопарк і парк Пушкіна... До речі... той Пушкін...

а шо той Пушкін... а нічого... книжечку я Вам на вході залишив... сподіваюсь... вахтьорша її не викинула ще...

бо Ви ж на роботу не ходите.. правильно... так... хм...


ну от... а ось вам трохи ще публікацій про Охматдит...

от... Ви ставите свічки... ходите в церкву... навіть за мене поставили Свічку, хоча я дуже-дуже поганий, справжній Бармалєй,

і Ви на мене написали заяву... а значить ми таки скоро побачимось в суді... я вже почав готуватись знаєте так...

обзвонив адвокатів... короче... готуюсь...

от! відволікся... якщо Ви вже ставити свічки за неймовірно поганих людей як от мене... крім того... оця погана людина тобто я вже живе і живе...

ото Ви напевне думаєте... коли вже той Саша відчепиться або просто помре... а я бачте вже живу...

а Оксани то вже немає... їй просто не повезло... їхала собі на роботу і загинула прямо в центрі Києва... розумієте...

от майже так як Ви їдете з Оболоні на Соломянку в універ...

і то я думаю...  можете також поставити свічку за без сумнівів... хорошу людину... дитячого лікаря Оксану Лєонтьєву...

вона загинула від ракетного удару по Києву 10 жовтня 2022 року... по центру був удар.. пам'ятаєте...

в дитячу площадку в парку Шевченка впав снаряд і зробив воронку таку величезну... яма короче там була

наче кратер якийсь метеоритний... залишився 5ий річний хлопчик її син сиротою... а хлопчик той майже як Ваш синочок...

і тільки дідусь у нього і є... печаль..


Дитяча Лікарня Охматдит - Публікації

01.02.2022 Охматдиту 128 рокiв

https://archive.org/details/01_02_2022.fb.ohmatdet.bolnica.128_let

22.12.2022 Кожна дитина заслуговує на диво: в Охматдитi розпочався цикл новорiчних свят для дiтей!

https://archive.org/details/22_12_2022.fb.ohmatdet.bolnica.dytyna_svjato

02.12.2022 35 операцiй за 8 мiсяцiв лiкування

https://archive.org/details/02_12_2022.fb.ohmatdet.bolnica.35_operacij_divchynka

24.11.2022 Найбiльша дитяча лiкарня України продовжує працювати та надавати допомогу дiтям

https://archive.org/details/24_11_2022.fb.ohmatdet.bolnica.pracja

11.10.2022 Внаслiдок ракетного обстрiлу Києва загинула лiкарка Охматдиту Оксана Леонтьєва

https://archive.org/details/11_10_2022.fb.ohmatdet.bolnica.oksana_leontjeva
З болем у серці повідомляємо, що внаслідок ракетного обстрілу Києва загинула лікарка Охматдиту Оксана Леонтьєва.🙏
▪️Оксана Григорівна загинула по дорозі на роботу, поспішаючи до своїх пацієнтів. Трагедія трапилась зранку 10 жовтня.
▪️Оксана Леонтьєва — лікар-гематолог НДСЛ «Охматдит». Вона працювала у відділенні трансплантації кісткового мозку 11 років. Оксана рятувала дітей з раком крові. Ця молода жінка була справжнім професіоналом і опорою для своїх пацієнтів та колег. Вона була самовідданою та відповідальною людиною і лікарем.
▪️Вранці 10 жовтня Оксана завезла 5-річного сина Григорія в дитячий садок. Востаннє. Далі поїхала в Охматдит, аби рятувати пацієнтів, як вона це робила вже багато років. Життя Оксани забрала російська ракета, її машина згоріла вщент в середмісті Києва. Через цей жахливий терористичний обстріл, загинуло і постраждало багато невинних людей, в тому числі дітей.
▪️Маленький син Гриша залишився сиротою. Його батька також не стало більше року тому. Тепер хлопчика буде виховувати дідусь. Ми можемо підтримати родину. Номер картки дідуся — 5375414117404300 Леонтьєв Григорій Олександрович.
Наші співчуття родині, близьким та колегам Оксани Лентьєвой. Це велика втрата для Охматдиту та усієї України.
Світла пам‘ять!🕯
1️⃣ Номер картки дідуся (монобанк): 5375414117404300
Леонтьєв Григорій Олександрович.
2️⃣ Допомога за реквізитами
Отримувач: Леонтьєв Григорій Олександрович
IBAN: UA973220010000026206310541304
ІПН/ЄДРПОУ: 2204024891
Призначення платежу: Поповнення рахунку
3️⃣ Реквізити SWIFT єврової картки
IBAN: UA253220010000026203331749230
Account No: 26203331749230
Receiver: LEONTIEV HRYHORII, 02140, Ukraine, c. Kyiv, st. Larysy Rudenko, build. 3A, fl. 50
Account with Institution (Банк Бенефіціара)
Bank: JSC UNIVERSAL BANK
City: KYIV, UKRAINE
Swift code: UNJSUAUKXXX
4️⃣ Рахунок PayPal: a8112934@gmail.com
МОЗ України Oleksandr Lysytsia Oksana Leontieva Оля Дащаковська Alexander Istomin Andriy Budzin Анна Брудна
