% vim: keymap=russian-jcukenwin
%%beginhead 
 
%%file slova.narod
%%parent slova
 
%%url 
 
%%author 
%%author_id 
%%author_url 
 
%%tags 
%%title 
 
%%endhead 
\chapter{Народ}
\label{sec:slova.narod}

Что я хочу предложить и почему. Сначала почему. Я считаю, что \emph{Народ}
совершенен. Любой \emph{Народ}. Его менталитет всегда и везде сформирован
внешними условиями. История, традиции действует на людей только благодаря
пропаганде власти. Наши дети уже не знают, кто такой Ленин, не говоря о
Брежневе и Хрущеве. После 1991 уже родилось два поколения новых людей,
\textbf{Не нужно никого лишать избирательного права}, Павел Себастьянович,
strana.ua, 27.05.2021

...Одна из проблем, которую нельзя решить высокоточными ракетами, - миллиарды
недорослей, недоучек, недоразвитков. Примитивные \emph{народы} умели
воспитывать своих мальчиков и девочек. Простая культура целиком влезала в одну
голову, и в каждой голове были необходимые элементы этики и религии, а не
только техническая информация. Культура была духовным и нравственным целым.
Естественным примером этой цельности оставались отец с матерью. Сейчас они
банкроты. Тинейджер, овладевший компьютером, считает себя намного умнее деда,
пишущего авторучкой.  Мир изменился, каждые пять лет он другой, и все старое
сбрасывается с корабля современности. Растут миллиарды людей, для которых
святыни, открывшиеся малограмотным пастухам, не стоят ломаного гроша. Полчища
Смердяковых, грядущие гунны, тучей скопились над миром. И они в любой день
готовы пойти за Бен Ладеном или Баркашовым. Записку Иконникова гунны не прочли
(а если б и прочли - что им Иконников? Что им князь Мышкин?). Судьбу Другого
они на себя не возьмут… Одно из бедствий современности - глобальная пошлость,
извергаемая в эфир.  Возникает иллюзия, что глобализм и пошлость - синонимы. И
глобализм уже поэтому вызывает яростное сопротивление. Не только этническое. Не
только конфессиональное...,
Григорий Соломонович Померанц (13 марта 1918, Вильно, Литва — 16 февраля 2013, Москва, Россия)

Как можно проводить пир во время чумы. Люба ты высказала мысли 90\% \emph{народа},
\citComment{Валентина Костюченко}, 
\citTitle{Cpoчнo! PE3KOE oбpaщeниe Tитapeнкo к 3eлeнcкoмy ШOKИPOBAЛO Kиeв!},
youtube.com, 07.06.2021

В общем, президентский законопроект о \emph{коренных народах} является
правильным, но к нему есть существенное замечание – в нем четко не очерчено то,
что именно украинцы являются коренным народом. Именно украинцы укоренились
здесь и выросли на этой земле. Разве они – не древний \emph{народ}? Если так
пойдет дальше, мы, украинцы, рискуем остаться без своей территории, без своей
земли.  Закон является правильным, но слишком либеральным – нужно было четко
определиться, кем являемся мы сами, украинцы, потому что пока что в
национальном вопросе у нас полная вакханалия. Мы думали, что вокруг нас
дружественные и братские \emph{народы}, но как только \enquote{старший брат}
посягнул на нашу территорию, то сразу зашевелились и Донбасс, и Крым, и
Закарпатье,
\citTitle{Как законопроект о коренных народах Украины выбил почву из-под ног Москвы - Главред}, 
Григорий Перепелица, opinions.glavred.info, 10.06.2021

%%%cit
%%%cit_pic
%%%cit_text
\emph{Народные} песни. Ночь эта лунная, зорькою ясная, Ночь эта лунная,
зорькою ясная!  Видно, что ночь для чудес.  Выйди любимая, за день уставшая,
Хоть на минуточку в лес.  Сядем, обнимемся тут под калиною, И над панами я пан!
Глянь же ты рыбонька – волной серебристою Стелется полем туман.  Лес
зачарованный, нимбом овеянный, Толь задремал, толи спит. Лишь на высокой и
стройной осине Ветер листвой шелестит.  Небо бескрайнее осыпано зорями – Какая
же божья краса!  Искрами ясными там под деревьями, Брызжет, играет роса.  Ты не
пугайся, что ножками босыми, Станешь в холодную росу, Я же тебя верная, в дом
по тропиночке, сам на руках донесу.  Ты не пугайся замерзнуть лебёдушкой, Тепло
– и нет в небе хмарь.  Я же прижму тебя близко к сердечку, Сердце пылает как
жар. Ты не пугайся, что могут подслушать, О чём ты шептала в лесу.  Ночь
уложила всех, снами окутала – За нами следить не досуг.  Спят твои недруги,
бытом замучены, Не потревожит их смех.  Разве обделенным нам долею нашею
Минутка свидания грех?
%%%cit_comment
%%%cit_title
\citTitle{Украинские Песни Русскими Словами}, 
БРАТИНА, zen.yandex.ru, 15.12.2020
%%%endcit

%%%cit
%%%cit_pic
%%%cit_text
Ох, труден в осмыслении триединый и (ныне трехглавый) \emph{русский народ}. Одни себя
в пограничники записали, окраины от москалей стерегут, другие никак не решат:
из каких именно краев себе на холку «русь» варяжскую призвали и что это вообще
значит
%%%cit_comment
%%%cit_title
\citTitle{Белоруссия: две загадки одного названия...}, 
Исторические Напёрстки, zen.yandex.ru, 28.03.2021
%%%endcit

%%%cit
%%%cit_head
%%%cit_pic
%%%cit_text
Донёс до нас в своих романах Достоевский, Что ложь - как зло, а преступленье -
это грех! Литературная культура - повод веский, Тем самым к лучшему он изменил
нас всех. А если честно это глупое сравненье, Объединяет здесь нас всех -
славянский род, Хотя и разное имеем поведенье, Но москали, да и хохлы - один
\emph{народ}!
%%%cit_comment
%%%cit_title
\citTitle{Как отличить украинца от русского?}, 
Уткин, zen.yandex.ru, 14.06.2021
%%%endcit

%%%cit
%%%cit_head
%%%cit_pic
%%%cit_text
И Рада, и Кабмин, и Офис президента враз опустеют. Истосковавшись по вниманию
СМИ, Лизавета Богуцкая внесла законопроект о введении уголовной ответственности
за надругательство над украинским языком. Тепла цветочку не достает. Понять
можно.  Если наш «ЦК» не планирует какой-нибудь стройки века с привлечением
осужденных, то лишь полоумные отдаст свой голос за этот глупенький
законопроект.  У нас ведь почти вся госслужба и две трети политиков говорят на
чудовищном суржике. Это ж сколько народа за надругательство по этапу пустить
придется? И Рада, и Кабмин, и даже Офис президента враз опустеют
%%%cit_comment
%%%cit_title
\citTitle{Депутат Богуцкая подала закон о надругательстве над украинским языком}, 
Максим Могильницкий, strana.ua, 19.06.2021
%%%endcit


%%%cit
%%%cit_head
%%%cit_pic
%%%cit_text
Именно поэтому, кстати, нас пытаются убедить в том, что сразу после майдана
власть являлась учредительной и честно исполняла волю \emph{народа}.  Правовую оценку
действиям парламента в 2014-м, а также прокуроров, которые расследуют
сфабрикованные дела и ставят решение КСУ под сомнение, дает мой коллега
Владимир Богатырь в своей новой статье. Почитайте. Она интересная.  Меня же
сегодня занимает вопрос отсутствия у Турчинова полномочий для подписания
законов, принятых в период с 23-го февраля по 2-е марта 2014-го. Тогда ведь
Конституция образца 96-го действовала и обязанности президента должен был
исполнять премьер-министр, а не какой-то Турчинов
%%%cit_comment
%%%cit_title
\citTitle{Ранее, чтобы выглядеть прилично, власть меняла Конституцию / Лента соцсетей / Страна}, 
Максим Могильницкий, strana.ua, 28.06.2021
%%%endcit


%%%cit
%%%cit_head
%%%cit_pic
%%%cit_text
Мне все происходящее напоминает период падения Великого Инки с развращенной
\enquote{аристократией}, оторванной от простого \emph{народа}. Впереди - вполне возможно,
пришествие \enquote{конкистадоров}....  Моя цитата из статьи: \enquote{Украина оказалась в
своего рода демографической воронке.  У нас низкая рождаемость и высокая
смертность, в итоге мы ежегодно теряем фактически население одного областного
центра. Плюс - высокая миграция за границу.  В итоге по темпам сокращения
населения мы бьем все возможные антирекорды}, — говорит экономист Алексей Кущ
%%%cit_comment
%%%cit_title
\citTitle{Происходящее в Украине напоминает период падения Великого Инки / Лента соцсетей / Страна}, 
Алексей Кущ, strana.ua, 29.06.2021
%%%endcit

