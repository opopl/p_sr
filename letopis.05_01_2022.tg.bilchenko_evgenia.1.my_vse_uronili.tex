% vim: keymap=russian-jcukenwin
%%beginhead 
 
%%file 05_01_2022.tg.bilchenko_evgenia.1.my_vse_uronili
%%parent 05_01_2022
 
%%url https://t.me/bilchenkozhenya/5586
 
%%author_id bilchenko_evgenia
%%date 
 
%%tags bilchenko_evgenia,ukraina,narod,obschestvo,nacionalizm,rossia
%%title БЖ. Мы все уронили
 
%%endhead 
\subsection{БЖ. Мы все уронили}
\label{sec:05_01_2022.tg.bilchenko_evgenia.1.my_vse_uronili}

\Purl{https://t.me/bilchenkozhenya/5586}
\ifcmt
 author_begin
   author_id bilchenko_evgenia
 author_end
\fi

БЖ. Мы все уронили

Русские... Мы - грубые, но добрые по природе своей, и нам постоянно хочется
верить во всякую фигню, ибо любит народ наш всякое дерьмо, как пел Летов. И все
мы, русские, - и те, что остались на Украине, и те, что уехали, дети одного
духовного пространства, - верим, что не все потеряно. Ну, живёт в нас
метафизика братства, живёт...

\ii{05_01_2022.tg.bilchenko_evgenia.1.my_vse_uronili.pic.1}

Из меня сочувствующие прямо вот выбивают ответ, что на Украине большинство
людей не сошло с ума, не истерит и не ксенофобит. Мол, царствует антинародный
режим, а народ-то в большинстве - за Россию, за дружбу, за ягоду- малину.

Видит Бог, я над этим тезисом билась на Украине лет пять: что народ, де, не
отравлен ядом нацпропаганды, что он стонет под гнетом власти, что ему надо
помочь освободиться...

Ничего подобного. Большинство людей на Украине именно сошло с ума. Массово.
Восемь лет зачистки школ и телеканалов не прошли даром. Если бы народ не был
отправлен пропагандой, он бы не выбрал русофоба Замир  Зеленского, а выбрал бы,
там, Бойко, или Медведчука, или кто-там ещё поприличнее из нашей скудной
кукольной и прочей ретро оппозиции? Ну, хотя бы их. Пусть бы их.

Но дело в том, что сдвиги произошли на уровне коллективного бессознательного, и
сдвиги - отнюдь не в нашу пользу. Чтобы прочувствовать фантазм, надо посмотреть
на него изнутри психбольницы, а потом извне.

Украинцы реально верят, что на них напали русские - и напали первыми. Они
реально верят, ребята. Они не осознали, как начали уничтожать внутреннюю Россию
в лице своих сограждан на Юге и Востоке, потом западнее и повсеместно.  Они
сейчас на апогее религиозного экстаза праведности. Любая вера предполагает
червь сомнения, но они его тщательно запихивают подальше и от этого истерят ещё
больше. Я понимаю всю трагедию того, о чем говорю. Целый народ сошел с ума.
Целый народ свели с ума? Такое возможно.

И дело не в том, что я сужу по группе боевых маргиналов социальных сетей. Не
сужу я о стране по Фейсбуку: ни о какой. Ни об Украине, ни о России, ни о
Франции, ни о Зимбабве. Виртуальный мир создаёт спектакль, вывеску, картинку.
Да, это берет нас на эмоции, но это не повод поливать грязью страну, будь то
Украина, Франция или Россия.

Я о другом - о жизни. Я пять лет жила рядом с истерящими в адрес России
реальными людьми: родственниками, друзьями, коллегами по работе, ребятами с
нашего двора и творческой интеллигенцией, о которой, как о покойном, - либо
ничего, либо правду. 

Помните, как в Германии 30-х зиговали школьники и старушки? Вот то же самое.

Конечно, кому-то больше повезло, кто с Востока или Юга, кому-то меньше. Но, чем
шире круг твоих знакомых (а он у меня на Украине сестрой и соседом не
ограничивался, поверьте), тем страшнее жить. И страшнее всего от русскоязычных
ненавистников России, те начали зверствовать с особенным изяществом, что ли. С
рогулями-то и так всё ясно.

Нужны ли России выруси и конформисты? Я не имею полномочий и знаний отвечать на
этот вопрос однозначно. Я, со своей верой в единый братский, даже не ожидала,
что я этот вопрос поставлю. 

Короче, очень нехороший вывод. Я полагаю, что Украине нужна принудительная
денацификация, или, как там Хабермас говорил по отношению к агрессивному
национализму, "принуждение к космополитизму". Пусть хоть либералы пацификацию
проводят (хозяин - барин), если уже иначе никак нельзя. Но кто-то же должен
привести в чувство страну, которая давно представляет опасность для своих и
чужих граждан?

Никак не хотела становиться скептиком, но, простите, - факты. Ясное дело, что
на Украине осталось много людей, думающих, как я, но их запугали. Их не видно.
Это не resistance, это Бог знает что. И нет ничего пока, что бы давало надежду,
что мы своих людей на Украине как-то раскрепостим. В первую очередь, их надо
освобождать от последних сомнений внутри себя, потому что даже внутри себя они
стесняются сказать: "Я люблю тебя, Россия", - а тем более выступить открыто.
Это можно понять.Но это все усложняет.

Донбасс не в счёт, Крым не в счёт, Одесса уже контролируется, отдельные
смельчаки и герои, покаявшиеся и полупокаявшиеся, а также трупы и юноши без
оружия тоже не в счёт. Ресурсов мало. Это я пытаюсь быть объективной, извините,
ничего личного, наоборот: наступаю себе на горло. 

Там страшно, ребята. Там реально страшно жить, и дело не в том, что убьют.

\#мывсеуронили \#вставай \#полежиеще
