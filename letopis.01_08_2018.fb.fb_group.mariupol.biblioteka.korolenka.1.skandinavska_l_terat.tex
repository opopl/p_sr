%%beginhead 
 
%%file 01_08_2018.fb.fb_group.mariupol.biblioteka.korolenka.1.skandinavska_l_terat
%%parent 01_08_2018
 
%%url https://www.facebook.com/groups/1476321979131170/posts/1784961861600512
 
%%author_id fb_group.mariupol.biblioteka.korolenka,lunina_tetjana.mariupol
%%date 01_08_2018
 
%%tags skandinavia,literatura,kniga
%%title Скандинавська літературна мандрівка
 
%%endhead 

\subsection{Скандинавська літературна мандрівка}
\label{sec:01_08_2018.fb.fb_group.mariupol.biblioteka.korolenka.1.skandinavska_l_terat}
 
\Purl{https://www.facebook.com/groups/1476321979131170/posts/1784961861600512}
\ifcmt
 author_begin
   author_id fb_group.mariupol.biblioteka.korolenka,lunina_tetjana.mariupol
 author_end
\fi

Скандинавська літературна мандрівка

Користувачі Центральної бібліотеки ім. В.Г. Короленка продовжують свою
літературну подорож країнами світу! Цього разу, в рамках циклу щомісячних
книжкових виставок читацьких уподобань «Літературна карта світу», вони вирушили
до країн Скандинавії – Норвегії, Швеції та Данії. Голосуючи за улюблені книжки
на бібліотечних сайті, блозі, сторінці у «Фейсбуці», а також безпосередньо у
залі абонементу, читачі до складу найкращих включили наступну п'ятірку:

1. Стіг Ларссон «Дівчина з татуюванням дракона»

Важко знайти людину, яка ніколи не чула про відому книгу швецького письменника
С. Ларсона «Дівчина з татуюванням дракона», яка була екранізована у 2011 році.
До речі мало хто знає, що спочатку творець цієї видатної трилогії дав їй зовсім
іншу назву, а саме «Чоловіки, що ненавидять жінок». 

Ця книга пронизана динамічним розвитком подій та неймовірною інтригою, що не
відпускає читача. А головні герої відрізняються вкрай яскравими характерами. То
що, Ви вже готові розгадати загадку 40-річної давнини? Тоді поспішайте до
бібліотеки!

2.  Ю Несбьо «Кров на снігу»

Норвезьський майстер детективу Ю Несбьо не дарма відомий на батьківщині як
композитор і вокаліст рок-групи Di Derre, а не просто як автор романів про
інспектора Харрі Холе. 

Роман «Кров на снігу» - історія найманого вбивці, яку письменник ухитряється
розповісти всього за 200 сторінок. За цей час трохи безглуздий, що страждає на
дислексію Улав, який став кілером тільки тому, що інших здібностей у  нього в
банді не виявлено, забезпечує читачеві сеанс психоаналізу і трохи любові в
кращих традиціях скандинавських романів. 

Твір починається з того, що головний герой отримує замовлення - вбити дружину
боса. Гарну, беззахисну і, на перший погляд, безневинну. Зачарований кілер
береться дамочку захистити ...А що далі зазирніть під обкладинку.

3. Геннінг Манкелль «На крок позаду» 

Третю ланку займає детектив «На крок позаду» Геннінга Манкелля, який отримав
міжнародну відомість завдяки серії книг про інспектора Курта Валландера.  

Вже з перших сторінок історія захоплює всю увагу читача, а головний герой
чепляє не тільки як розумний та винахідливий слідчий, але і як людина зі своїми
думками та почуттями. Головною родзинкою книги є те, що Ви начебто поринаєте у
думки головного героя, та ниточка за ниточкою розлутуєте павутиння, яку на крок
попереду плете злочинець.

4. Кнут Гамсун  «Містерії» 

Не оминула наш рейтинг і книга видатного представника літератури Норвегії К. Гамсуна «Містерії»

Приїхавши в маленьке містечко Юхан Нагель викликає у місцевих жителів жвавий
інтерес до своєї персони, що підігрівається його вчинками (наприклад, порожній
футляр від скрипки, яскравий костюм, отрута в кишені на випадок раптового
бажання померти). Нагель має свою окрему думку з усіх важливих питань буття:
про смерть, кохання, мистецтво, він з азартом вступає в суперечки. Любов
Нагеля, також як і він сам, сумбурна, не зовсім логічна, з надривом.

В чому ж криється загадка цієї неординарної людини? Автор залишає
недомовленість, пунктирність, дає читачам право пофантазувати, представити свій
варіант розвитку подій.

5. Інгрід Юхансен. «Фіорди. Крижане серце»

Замикає п'ятірку читацьких уподобань роман молодої норвежської письменниці
«Фіорди. Крижане серце».  Це хвилююча історія любові і смерті на тлі
грандіозних пейзажів норвезьких фіордів.

Коли на чудовому лайнері, що здійснює круїз по норвезьких фіордах, гине
дівчина-стюарт, служба безпеки переконує поліцію Осло, що трагедія сталася
через халатність її колеги - Льоні Ольсен. Але молода жінка відмовляється
визнати свою провину і починає власне розслідування, щоб довести: це був не
нещасний випадок, а навмисне вбивство! Чим їй доведеться пожертвувати заради
викриття невловимого маніяка? Які порочні пристрасті і криваві таємниці
ховаються за розкішними  інтер'єрами плавучого палацу? Чи варто вірити
похмурому красеню з дивними смаками і досконалим тілом...? І чи зможе Льоні
розтопити його крижане серце, що холодніше холодний води полярних фіордів?
..Щоб знайти відповіді на всі ці запитання варто прочитати цю книгу.
