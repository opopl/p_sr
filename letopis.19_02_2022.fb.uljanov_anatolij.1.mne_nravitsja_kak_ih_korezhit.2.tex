% vim: keymap=russian-jcukenwin
%%beginhead 
 
%%file 19_02_2022.fb.uljanov_anatolij.1.mne_nravitsja_kak_ih_korezhit.2
%%parent 19_02_2022.fb.uljanov_anatolij.1.mne_nravitsja_kak_ih_korezhit
 
%%url 
 
%%author_id 
%%date 
 
%%tags 
%%title 
 
%%endhead 

\subsubsection{2}

Противопоставляя Европу Мордору, они мнят себя её частью, хотя ничто в их
практике не даёт на то оснований. Гуманизм, к которому их ресторанная элита
апеллирует в своей апологетике произвола, является лирикой, наброшенной на
порыв в чужую сытость, на шопинг и жопинг, под зонтики «белых людей», прочь от
«скота» и «скотного двора», в котором эти «гуманисты» не узнают себя.

Это – тайна их ненависти. Они ненавидят нас именно потому, что мы – это они. Не
европейцы, а дети совков. Которые, в отличие от них, отказались отказаться от
себя, своих родителей, культуры и истории. Люди, чьё достоинство выражается в
этом отказе. Люди, не узнающие себя в американцах и французах, но узнающие себя
в осколках радуги советских народов. Наше сожительство в этом плацкарте истории
породило связи, которые не исчезают просто потому, что кто-то флаг на домике
сменил, и начертил полосочки на карте. По нам бегают все наши крови.

Неужели вы думаете, что мы будем стрелять в самих себя? Что слова британцев
или американцев, – людей из далёких, чужих миров, – могут нокаутировать нашу
эмпатию, отменить региональные исторические отношения между этносами?

Можно по-разному оценивать эти отношения. И желать разного их развития. В том
числе прекращения. Однако делаться это должно с пониманием нежности
человеческого существа. С уважением к его правде.

Нет, мы отказываемся ненавидеть. Ненависть – это злобная беспомощность. Кто
ненавидит – тот уже проиграл. Но и молчать в ответ на ненависть мы не можем.
