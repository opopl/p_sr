% vim: keymap=russian-jcukenwin
%%beginhead 
 
%%file 23_11_2020.news.ua.pravda.1.vaccine
%%parent 23_11_2020
 
%%url https://www.epravda.com.ua/publications/2020/11/23/668273/
%%author 
%%author_id 
%%tags 
%%title 
 
%%endhead 

\subsection{Перші вакцини від COVID-19: на що можуть розраховувати українці і коли}
\Purl{https://www.epravda.com.ua/publications/2020/11/23/668273/}

\index[rus]{Коронавирус!Вакцины}

Світові гіганти фарміндустрії завершили випробування вакцини від коронавірусу.
Першими імунізацію отримують громадяни розвинених країн. А коли на вакцинацію
очікувати українцям?

Понеділок, 23 листопада 2020, 07:15 - Всеволод Некрасов, Микола Максимчук

Прогнози Всесвітньої організації охорони здоров'я (ВООЗ) щодо розробки вакцини
від коронавірусу не справдились. В березні 2020 року організація заявляла, що
цей процес триватиме не менше року. Проте фармкомпанії справились швидше.

Уже в листопаді пролунали перші заяви про успішне випробування одразу двох
вакцин від COVID-19. Їх винайшли дві американські та одна німецька компанія,
які вже до кінця 2020 року поставлять на ринки розвинених країн десятки
мільйонів доз противірусних препаратів.

Ще чотири вакцини знаходяться на фінальній стадії випробувань.

Українська влада вже розпочала перемовини з виробниками про закупівлю
препаратів, які виробляють імунітет від COVID-19. Партію вакцини, якої
вистачить для проведення щеплення 4 мільйонів українців, наша країна отримає
безкоштовно через спеціальний фонд ООН.

\subsubsection{Якою є ефективність препаратів та коли саме розпочнеться вакцинація українців?}

Прогнози Всесвітньої організації охорони здоров'я (ВООЗ) щодо розробки вакцини
від коронавірусу не справдились. В березні 2020 року організація заявляла, що
цей процес триватиме не менше року. Проте фармкомпанії справились швидше.

Уже в листопаді пролунали перші заяви про успішне випробування одразу двох
вакцин від COVID-19. Їх винайшли дві американські та одна німецька компанія,
які вже до кінця 2020 року поставлять на ринки розвинених країн десятки
мільйонів доз противірусних препаратів.

Ще чотири вакцини знаходяться на фінальній стадії випробувань.

Українська влада вже розпочала перемовини з виробниками про закупівлю
препаратів, які виробляють імунітет від COVID-19. Партію вакцини, якої
вистачить для проведення щеплення 4 мільйонів українців, наша країна отримає
безкоштовно через спеціальний фонд ООН.

Якою є ефективність препаратів та коли саме розпочнеться вакцинація українців?
Рятівна вакцина

До листопада 2020 року увесь світ боровся лише з наслідками коронавірусу.

Введені карантинні заходи були направлені виключно на стримування поширення
вірусу. Маска, рукавички та дотримання соціальної дистанції були єдиними
інструментами захисту для мільярдів людей по всьому світу.

Все змінилось за один тиждень.

\subsubsection{Pfizer-BioNTech}

\ifcmt
pic https://eimg.pravda.com/images/doc/f/3/f3cf2a6-phizer-vaccina.jpg
\fi

9 листопада попередні результати досліджень спільно розробленої вакцини від
COVID-19 оприлюднили американська Pfizer і німецька BioNTech. Дослідження
препарату, яке проводилося на 44 тисячах добровольцях, показало його
ефективність на рівні 90\%.

\enquote{Відмінні новини від Pfizer і BioNTech Group про успішні результати їхнього
останнього клінічного випробування вакцини від COVID-19. Європейська наука
працює! Європейська комісія незабаром підпише з ними контракт на 300 мільйонів
доз. А поки давайте продовжимо захищати один одного}, \dshM заявила після перших
заяв компанії президентка Єврокомісії Урсула фон дер Ляєн.

Що відомо про вакцину? Вартість вакцини становить 20 дол (565 грн). Для
успішної імунізації, з інтервалом у 21 день, необхідно зробити дві
ін’єкції. Тобто курс вакцинації однієї особи обійдеться в 40 дол (1130
грн).

Препарат має певні особливості зберігання, які ускладнюють його
транспортування. Для зберігання протягом шести місяців вакцину необхідно
тримати при температурі -70 градусів Цельсія. Якщо зберігати її при
температурі від +2 до +7 градусів Цельсія, то термін придатності
зменшується до п'яти днів.

Незважаючи на це, Pfizer-BioNTech планують кожен день завантажувати ящики
розміром з валізу з місць розподілу в місті Каламазу, штат Мічиган (США), та
місті Пуурс, Бельгія на 20 вантажівок. До найближчих аеропортів щодня будуть
перевозити приблизно 7,6 млн доз.

Розробники вже уклали угоди на поставку вакцини з декількома країнами по всьому
світу. Найбільше замовив Європейський Союз: 11 листопада підтверджено
замовлення 300 млн доз, Японія замовила 120 млн доз, США – 100 млн.
Великобританія, Канада, Австралія та Чилі разом узяті придбали щонайменше 10
млн доз препарату.

Загалом, передбачається, що до кінця поточного року буде випущено близько
50 млн доз вакцин. Плани на 2021 рік – 1,3 мільярда доз.

\subsubsection{Moderna}

\ifcmt
pic https://eimg.pravda.com/images/doc/7/3/735ce88-moderna-vaccine.jpg
\fi

За тиждень після оприлюднення інформації Pfizer-BioNTech про успішне
випробування вакцини від COVID-19 оголосила американська біотехнологічна
компанія Moderna.

В дослідженнях приймали участь 30 тисяч людей, які отримували діючий препарат
або плацебо. З них коронавірусом заразились 95 людей, з яких лише п'ятеро були
вакциновані препаратом, а інші – плацебо. Таким чином, ефективність вакцини
становила 94,5\%.

Що відомо про вакцину? Як і розроблена Pfizer-BioNTech вакцина компанії Moderna
складається з двох ін’єкцій, які вводяться з інтервалом у чотири тижні.

Але остання має важливу перевагу – для її зберігання не потрібна сильна
заморозка. Представники Moderna запевняють, що в холодильнику препарат може
зберігатись до 30 днів, і півроку не буде втрачати властивостей при зберіганні
за температури -20 градусів Цельсія.

Країни світу, крім США, купуватимуть вакцину компанії Moderna за ціною, яка
вдвічі більша за препарат Pfizer-BioNTech. Вартість складатиме 32-37 доларів
(904-1045 грн), в той час як для внутрішнього американського ринку ціна складе
24,8 дол за дозу (700 грн).

Саме тому Глобальний фонд доступу до вакцин проти COVID-19 (COVAX) при
Організації Об'єднаних Націй заявив про намір субсидувати отримання вакцини
проти коронавірусу в країнах з низьким рівнем доходу.

Тобто, незалежно від фінансових можливостей країн, кожна отримає певну
кількість цього препарату безкоштовно.

Moderna заявила, що до кінця 2020 року для потреб ринку США виготовить близько
20 млн доз вакцини. До кінця 2021 року компанія зобов’язалася поставити в США
100 млн доз, а за потреби американці можуть отримати додатково ще 400 млн доз.

Озвучені Moderna плани виробництва вакцини на 2021 рік становлять 0,5-1 млрд доз.

Уряд Канади вже замовив 56 млн доз, Великобританія \dshM 50 млн, а Швейцарія \dshM 4,5
млн. Японія, Швейцарія, Катар та Ізраїль також підписали угоди про поставки
цієї вакцини. Європейська комісія має \enquote{потенційну угоду про закупівлю} об’ємом
80-160 млн доз.



4 перегляди
1
		В закладкиВ закладки

Прогнози Всесвітньої організації охорони здоров'я (ВООЗ) щодо розробки вакцини від коронавірусу не справдились. В березні 2020 року організація заявляла, що цей процес триватиме не менше року. Проте фармкомпанії справились швидше.

Уже в листопаді пролунали перші заяви про успішне випробування одразу двох вакцин від COVID-19. Їх винайшли дві американські та одна німецька компанія, які вже до кінця 2020 року поставлять на ринки розвинених країн десятки мільйонів доз противірусних препаратів.

Ще чотири вакцини знаходяться на фінальній стадії випробувань.

Українська влада вже розпочала перемовини з виробниками про закупівлю препаратів, які виробляють імунітет від COVID-19. Партію вакцини, якої вистачить для проведення щеплення 4 мільйонів українців, наша країна отримає безкоштовно через спеціальний фонд ООН.

Якою є ефективність препаратів та коли саме розпочнеться вакцинація українців?
Рятівна вакцина

До листопада 2020 року увесь світ боровся лише з наслідками коронавірусу.

Введені карантинні заходи були направлені виключно на стримування поширення вірусу. Маска, рукавички та дотримання соціальної дистанції були єдиними інструментами захисту для мільярдів людей по всьому світу.

Все змінилось за один тиждень.
Pfizer-BioNTech
 
джерело: Pfizer

9 листопада попередні результати досліджень спільно розробленої вакцини від COVID-19 оприлюднили американська Pfizer і німецька BioNTech. Дослідження препарату, яке проводилося на 44 тисячах добровольцях, показало його ефективність на рівні 90%.

"Відмінні новини від Pfizer і BioNTech Group про успішні результати їхнього останнього клінічного випробування вакцини від COVID-19. Європейська наука працює! Європейська комісія незабаром підпише з ними контракт на 300 мільйонів доз. А поки давайте продовжимо захищати один одного", – заявила після перших заяв компанії президентка Єврокомісії Урсула фон дер Ляєн.

    Що відомо про вакцину? Вартість вакцини становить 20 дол (565 грн). Для успішної імунізації, з інтервалом у 21 день, необхідно зробити дві ін’єкції. Тобто курс вакцинації однієї особи обійдеться в 40 дол (1130 грн).

    Препарат має певні особливості зберігання, які ускладнюють його транспортування. Для зберігання протягом шести місяців вакцину необхідно тримати при температурі -70 градусів Цельсія. Якщо зберігати її при температурі від +2 до +7 градусів Цельсія, то термін придатності зменшується до п'яти днів.

Незважаючи на це, Pfizer-BioNTech планують кожен день завантажувати ящики розміром з валізу з місць розподілу в місті Каламазу, штат Мічиган (США), та місті Пуурс, Бельгія на 20 вантажівок. До найближчих аеропортів щодня будуть перевозити приблизно 7,6 млн доз.
Ірландія – на крові, Північна Корея – на годинниках. На чому тримається експорт країн світу?

Розробники вже уклали угоди на поставку вакцини з декількома країнами по всьому світу. Найбільше замовив Європейський Союз: 11 листопада підтверджено замовлення 300 млн доз, Японія замовила 120 млн доз, США – 100 млн. Великобританія, Канада, Австралія та Чилі разом узяті придбали щонайменше 10 млн доз препарату.

    Загалом, передбачається, що до кінця поточного року буде випущено близько 50 млн доз вакцин. Плани на 2021 рік – 1,3 мільярда доз.

Moderna
 
Компанія Moderna базується в Кембриджі
www.washingtonpost.com

За тиждень після оприлюднення інформації Pfizer-BioNTech про успішне випробування вакцини від COVID-19 оголосила американська біотехнологічна компанія Moderna.

В дослідженнях приймали участь 30 тисяч людей, які отримували діючий препарат або плацебо. З них коронавірусом заразились 95 людей, з яких лише п'ятеро були вакциновані препаратом, а інші – плацебо. Таким чином, ефективність вакцини становила 94,5%.

    Що відомо про вакцину? Як і розроблена Pfizer-BioNTech вакцина компанії Moderna складається з двох ін’єкцій, які вводяться з інтервалом у чотири тижні.

    Але остання має важливу перевагу – для її зберігання не потрібна сильна заморозка. Представники Moderna запевняють, що в холодильнику препарат може зберігатись до 30 днів, і півроку не буде втрачати властивостей при зберіганні за температури -20 градусів Цельсія.

    Країни світу, крім США, купуватимуть вакцину компанії Moderna за ціною, яка вдвічі більша за препарат Pfizer-BioNTech. Вартість складатиме 32-37 доларів (904-1045 грн), в той час як для внутрішнього американського ринку ціна складе 24,8 дол за дозу (700 грн).

Саме тому Глобальний фонд доступу до вакцин проти COVID-19 (COVAX) при Організації Об'єднаних Націй заявив про намір субсидувати отримання вакцини проти коронавірусу в країнах з низьким рівнем доходу.

Тобто, незалежно від фінансових можливостей країн, кожна отримає певну кількість цього препарату безкоштовно.

Moderna заявила, що до кінця 2020 року для потреб ринку США виготовить близько 20 млн доз вакцини. До кінця 2021 року компанія зобов’язалася поставити в США 100 млн доз, а за потреби американці можуть отримати додатково ще 400 млн доз.

    Озвучені Moderna плани виробництва вакцини на 2021 рік становлять 0,5-1 млрд доз.

Уряд Канади вже замовив 56 млн доз, Великобританія – 50 млн, а Швейцарія – 4,5
млн. Японія, Швейцарія, Катар та Ізраїль також підписали угоди про поставки
цієї вакцини. Європейська комісія має "потенційну угоду про закупівлю" об’ємом
80-160 млн доз.

\subsubsection{Хто ще?}

На даний момент в світі розробляються більше 200 вакцин від COVID-19. За
інформацією ВООЗ, до третьої фази клінічних досліджень, крім Pfizer-BioNTech та
Moderna, дійшли ще чотири препарати.

Це розробки:

\begin{itemize}
  \item Оксфордського університету (спільно з AstraZeneca).
  \item Китайської компанії Sinovac.
  \item Уханського інституту вірусології (спільно з Sinopharm).
  \item Пекінського інституту вірусології (спільно з Sinopharm).
\end{itemize}

Результати третього етапу досліджень розробники вакцини AZD1222 – британська
фармкомпанія AstraZeneca і Оксфордський університет – отримають до кінця 2020
року.

Вона також складатиметься з двох ін’єкцій, які поки що зберігаються в
замороженому стані. Проте, коли вакцина наблизиться до моменту затвердження для
застосування, до її флаконів виробник додасть кінцевий інгредієнт, який
дозволить зберігати препарат при звичайній температурі в холодильнику.

Незважаючи на те, що офіційно випробування AZD1222 ще не завершене, розробники
вже уклали ряд угод з декількома країнами на поставку вакцини. США та Індія
домовились про закупівлю 500 млн доз, ЄС придбає 400 млн доз, а організація ООН
COVAX замовила 300 млн доз.

Великобританія, Японія, Індонезія, Бразилія та Латинська Америка, за винятком
Бразилії, підтвердили замовлення щонайменше на 100 млн доз разом узяті. В
угодах на постачання вартість вакцин становить 3-4 дол (84-113 грн) за одну
дозу.

Вакцина від коронавірусу \enquote{Спутник V}, яку ще влітку розробив російський
Національний дослідницький центр епідеміології та мікробіології імені
М.Ф.Гамалії, коштує близько 13 дол (367 грн). Проте Україна її не закуповувала і
не планує цього робити, оскільки препарат не пройшов необхідної стадії
клінічних випробувань.

\enquote{У світі не існує підтверджень того, що та вакцина, про яку говорять у
Російській Федерації, пройшла всі фази клінічних випробувань, особливо третю...
Третя стадія (випробувань. \dshM ЕП) передбачає велику кількість людей, які є
добровольцями. Зараз навіть немає предмета розмови для дискусій \dshM купувати чи
ні}, \dshM ще в жовтні заявляв міністр охорони здоров'я Максим Степанов. 

\subsubsection{Українці отримають вакцину, але не раніше травня}

За інформацією МОЗ, уряд веде перемовини з компаніями Moderna, та
Pfizer-BioNTech щодо поставок вакцини в Україну.

За вимогами виробників (договір нерозголошення конфіденційної інформації,
ініційований цими компаніями, – ЕП) результати перемовин не можуть бути
оголошені публічно, зазначили ЕП в Кабінеті міністрів.

Постачання вакцин до України може здійснюватись трьома шляхами:

\begin{itemize}
  \item за програмою COVAX;
  \item закупівлі за кошти державного бюджету;
  \item комерційні постачання через дистриб'юторську мережу.
\end{itemize}

\enquote{Як повідомляє ВООЗ, вакцини найбільш вірогідно стануть доступними у
травні-червні 2021 року. За програмою COVAX гарантовано буде доставлено 20\%
від кількості населення доз вакцин, а решту вакцин держава буде закуповувати
самостійно}, \dshM зазначили в уряді.

Максим Степанов повідомив, що перший транш для України становитиме 1,2 млн доз.
Загалом же за програмою COVAX ми отримаємо до 8 млн доз, яких вистачить для
щеплення 4 мільйонів громадян. Йдеться про отримання препарату компанії
Moderna.

\enquote{Перш за все, ми плануємо вакцинувати людей з групи ризику: медичних працівників, громадян старших за 65 років і людей з хронічними захворюваннями}, –додав Степанов.

Раніше повідомлялось, що для закупівлі вакцини в державному бюджеті 2021 року
передбачається до 19,4 млрд грн. Додатково 50 млн євро, які Європейський
інвестиційний банк виділив на розвиток муніципальної інфраструктури, Україна
перенаправить на купівлю вакцини.

Закуповувати, швидше за все, Україна буде вакцину від Pfizer-BioNTech.

Співрозмовник в уряді повідомив, що минулого тижня президент Володимир Зеленський провів переговори з канцлером Німеччини Ангелою Меркель з приводу придбання препарату, одним з розробників якого є саме німецька компанія.

Якщо заявлені для купівлі вакцини кошти будуть повністю виділені, то Україна зможе закупити близько 38 млн доз препарату від Pfizer-BioNTech. Цього вистачить для вакцинації 19 мільйонів громадян.

Найбільш оптимальним рішенням було б отримання українськими фармакомпаніями
технології виробництва вакцини проти коронавірусу. \enquote{Але зазначені технології є
комерційною таємницею компаній-розробників вакцини. Дозволів на їх використання
Україна не отримувала}, \dshM повідомили в уряді.
