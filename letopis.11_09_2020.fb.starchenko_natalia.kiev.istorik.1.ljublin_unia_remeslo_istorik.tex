% vim: keymap=russian-jcukenwin
%%beginhead 
 
%%file 11_09_2020.fb.starchenko_natalia.kiev.istorik.1.ljublin_unia_remeslo_istorik
%%parent 11_09_2020
 
%%url https://www.facebook.com/natalya.starchenko.1/posts/2985835954854742
 
%%author_id starchenko_natalia.kiev.istorik
%%date 
 
%%tags istoria,ljublin.unia,ukraina
%%title ЩЕ РАЗ ПРО ЛЮБЛІНСЬКУ УНІЮ ТА РЕМЕСЛО ІСТОРИКА
 
%%endhead 
 
\subsection{ЩЕ РАЗ ПРО ЛЮБЛІНСЬКУ УНІЮ ТА РЕМЕСЛО ІСТОРИКА}
\label{sec:11_09_2020.fb.starchenko_natalia.kiev.istorik.1.ljublin_unia_remeslo_istorik}
 
\Purl{https://www.facebook.com/natalya.starchenko.1/posts/2985835954854742}
\ifcmt
 author_begin
   author_id starchenko_natalia.kiev.istorik
 author_end
\fi

ЩЕ РАЗ ПРО ЛЮБЛІНСЬКУ УНІЮ ТА РЕМЕСЛО ІСТОРИКА

Певним тригером цих роздумів послужило інтерв’ю знаного історика Віктора
Брехуненка на ZAXID.NET, в якому серед іншого є й таке:  

на середину XVII ст. «традиційна еліта повністю втратила державотворчий
інстинкт. Люблінська унія – це вже свідчення втрати цього інстинкту. Вони не
поставили питання третього члена Речі Посполитої – Великого князівства
Руського, хоча вони мали золоту акцію, коли литовська магнатерія пішла з
Любліна». 

І таке: «попри документи Люблінського сейму, шляхта, особливо з кінця XVI
століття, почала культивувати міфи – нібито Русь увійшла до Речі Посполитої на
засадах угоди, хоча це були реституційні привілеї насправді. Князі погодилися
на реституційні привілеї, це страшенно контрастує із позицією, яку обіймали
литовські магнати». 

І тут я змовчати не можу. Дуже мені образливо за наших хлопців, які вистояли на
Люблінському сеймі й таки отримали той максимум, на який і сподіватися було
годі. Відтак хай вибачать мене мої давні читачі за повтор, який для тих, хто
потрапив до кола друзів значно пізніше від моменту, коли я про це розлого і
багато писала. Відтак \enquote{коротко і страшно}  @igg{fbicon.smile}. Щоб
принаймні мої читачі у своїх роздумах про минуле мали альтернативу, а не лише
схему Грушевського, а чи модернізовану схему Грушевського, де замість
зрадницької шляхти постає шляхта, що втратила «державницький інстинкт».  

Отож я як той Катон, що всіх переконував у необхідності зруйнувати Карфаген.
Тільки я  про те, що стару схему української історії треба віддати для
дослідження історіографам, які детально проаналізують, чому автори вибудували
її у XIX-на початку XX ст. саме на такий спосіб, як її підлаштувала під свої
потреби радянська ідеологія, та чому вона виявилася такою живучою до сьогодні.
Кесарю кесареве, історіографам – стару історіографію, сьогоднішньому – свій
погляд на минуле, збагачений новими питаннями і новими дослідженнями. Однак для
початку трошки про ремесло історика з неминучим елементом камінґауту. 

Історіографії, яка від кінця XVIII ст. дуже прагнула потрапити до клубу
«справжньої» точної науки, не дуже щастило. Бо минуле не надавалося для
спостереження, експерименту та чіткої перевірки результатів. Біда полягала
також у тому, що суб’єктивність історика, від якої позитивісти намагалися
активно відкараскатися, є насправді частиною інструментарію історика. Його
власний досвід, його вміння за маленькими деталями побачити загальну картину,
його здібності (аналогічні здібностям слідчого/медика/слідопита), себто його
інтуїція (читай професіоналізм), складають важливу основу інтерпретації
джерела. 

На позір виглядає, що історики, активно розпочавши від початку 70-х років XX
ст. професійну розмову про суб’єктивність своїх результатів, підважують їх
значущість.  Скажімо, найдошкульніший удар по історіографії як науці, після
якого ще довго йшов поголос про кінець історії, завдав із середини історичного
цеху Гайден Вайт своєю книгою «Метаісторія» (1973 р.). Він буквально препарував
кілька великих історіографічних наративів, указавши в своєму діагнозі на
літературну природу будь-якого історичного тексту. Історик завжди додає «щось»
до опису подій; це «щось» – літературність і уява, заявляє Вайт. Він не може
безпосередньо спостерігати за подією, він хіба може її уявити. Отож історик
вибирає найважливіші на його думку факти, укладає їх в певні послідовності та
оповідає про них за допомогою літературних прийомів. Постане під пером автора
драма, комедія, трагедія чи сатира, говорить Вайт, залежить від його вибору,
але передусім від персональних характеристик дослідника. Для такого способу
історичного письма джерела не більше, аніж ілюстрація авторських побудов. Вайт,
аналізуючи великі історичні наративи переважно XIX ст., поширював свої висновки
на всю історичну науку, і у своєму запалі утвердити історію як варіант красного
письменства дещо перебільшував. 

Що зчинилося у професійному колі! Відгомін, щоправда, краєчком, зачепив і наші
терени, зокрема у критиці постмодернізму чи не як різновиду фашизму. Однак з
часом гострота дискусій вщухла, а поголос про кінець історії, як це вже не раз
траплялося, відійшов у минуле. Історіографія його пережила, як напевно переживе
й наступні круті повороти. Однак цей болісний досвід пішов їй на користь.
Усвідомлення «незручних» деталей, на які спрямував свою увагу Гайден Вайт та
його прибічники, насправді відточує професіоналізм історика, змушуючи його
постійно звертати увагу на власну присутність у своїх інтерпретаціях, на оте
болюче: де я як людина зі своїм набором характеристик, а де моє джерело, писане
Іншим. І, зрештою, як мислив той Інший, який сенс надавав своїм діям, чим
керувався на щодень і в порогових ситуаціях.

Краще розуміння (слідом за Гайденом Вайтом) природи великих історичних побудов
XIX ст. як конструкту певної доби для суспільних потреб дозволяє підважити
домінування гранднаративу над конкретними науковими дослідженнями. Для
української історіографії це ще й шанс - перестати модифікувати схему
Грушевського. Адже велика пояснювальна схема тяжіє практично над будь-якою
маленькою інтерпретацією та її автором, а неповнота джерел зазвичай провокує
історика пояснити їх через велику пояснювальну схему. А це означає, що історик
розкаже про хід подій з висоти свого знання про те, чим же у підсумку серце
заспокоїться чи розхвилюється. Він сконструює причини за наслідками, він внесе
причинно-наслідковий зв’язок туди, де могло йтися про випадковість і
непрогнозований результат. Історик виходитиме з того, що правда завжди на боці
переможців і саме вона забезпечила потрібний для майбутнього прогрес, без якого
світ напевно пішов би під укіс. 

А якщо гранднаратив і дослідження поміняти місцями? Якщо почати з нормального
дослідження спірних моментів і заповнювати пазлами загальну картину? Якщо
сказати: гранднаратив – не істина в її останній інстанції, до якої допасовують
свої маленькі історії дослідники, це відкрита інтерпретація, яка перевіряється
чи не на щодень конкретними дослідженнями, підважується і змінюється? Це має
бути варіант великої комп’ютерної гри, під час якої задаються нові питання та
постають нові проблеми відповідно до прирощеного знання.

І тут я пригадаю ще свого улюбленого Умберто Еко, який у своїх літературних
текстах насправді весь час розмірковує про ремесло історика. Я якось знічев’я
знайшла низку паралелей між його науковою монографією «Відсутня структура» й
романом «Ім’я троянди». Скажімо, Еко говорить, що ми часто сприймаємо свій
спосіб впорядкування матеріалу – певну логічну структуру - як таку, що існує
насправді. А тим часом це зазвичай лише інтелектуальна операція, з допомогою
якої ми досягли певного результату. Що, утім, геть не свідчить про помилковість
результату. Біда, однак, якщо ми повіримо, що структура справді об’єктивно
існує. А от як про це говорить Вільгельм, герой роману Еко: «Початковий порядок
– це як сітка, чи як драбина, яку використовують, щоб куди-небудь піднятися.
Однак після цього драбину необхідно відкинути, адже виявляється, що хоча вона й
знадобилася, в ній самій не було жодного сенсу». 

І нарешті про Люблінську унію.  Річ Посполита постала на Люблінському сеймі
1569 року як унія двох окремих держав, кожна з яких мала свого суверена: Велике
князівство Литовське великого князя на чолі і Польське Королівство – короля.
Для тих людей байдуже, що великий князь і король був однієї смертною особою,
йшлося про символічне тіло сюзерена. Три українські воєводства, які
Зиґмунт-Август вилучив із Князівства і приєднав до Корони своєю волею, не мали
жодних ознак автономії у складі Великого князівства Литовського і приєднувалися
як давня частина Королівства, яка нарешті поверталася до свого питомого цілого.
Аргументація цієї єдності для учасників сейму насправді мало важила. Отож ні
для Великого князівства, ані для Королівства, у цих областей не було жодної
легітимації для якихось претензій на свою окремішність, уже не кажучи про
членство у майбутній федерації. Скажу вам більше – більшість волинських
князів-урядників не входили до ради великого князя Великого князівства
Литовського. З клопотаннями їх туди увести періодично зверталася волинська
шляхта на сеймах до великого князя, але намарне. Отож приготувавши на початку
березня 1569 р. під час Люблінського сейму універсал про приєднання Волині і
Підляшшя «до прав і свобод» Корони, король надіслав його до шляхти цих теренів,
вимагаючи прибути до 4 квітня в Люблін для складання присяги на вірність королю
і Польському Королівству. Однак щось пішло не так. Волинська шляхта зібралася
на свій з’їзд і підготувала 29 березня петицію до короля, в якій висловлювала
своє велике здивування. Сенс петиції зводився до того, що волиняни не проти
унії з Короною, але позаяк вони люди вільні, то їх ніхто ні до чого не може
примусити декретом. Отож вони пропонували скликати інший сейм на кордоні Волині
і Корони, сісти за стіл переговорів і виробити умови такої «спілки». Як ви
розумієте, ніхто присягати до Любліна не квапився. 

Терпець учасникам сейму поступово уривався, вони ж бо сиділи у Любліні від 12
січня, тож вони почали вимагати від короля покарати порушників його волі –
конфіскувати маєтки і позбавити урядів. Ситуація погострювалася і напевно про
це були поінформовані волиняни, адже присутність їхніх представників фіксується
у квітні в Любліні. В кінці травня на сейм підтягується волинська шляхта і
князі. І от, коли 23 травня закликають присягати першу групу волинян, ті
заявляють, що присягнуть, якщо король, сенатори і посли присягнуть їм навзаєм.
Автор «Щоденника Люблінського сейму» - основного нашого джерела – лаконічний:
їм відмовили і вони присягли. А от ситуацію, що склалася у сеймовій світлиці
наступного дня, коли мали присягти князі, він описує значно ширше. Князі теж
вимагають взаємної присяги, яка мала засвідчити двосторонню угоду між
суб’єктами. Коронярі не розуміють потреби такої присяги, адже вони, люди
вільні, приймають волинян як рівних собі до всіх своїх свобод, готові й надалі
ділитися своїми здобутками, про що й заявляється. На це волиняни твердять
вустами князя Костянтина Вишневецького, що вони мають власні вольності, які й
потребують гарантії від тих, до кого волиняни приєднуються. Автор «Щоденника»
коротко цитує промову князя, де згадувалося про існування князів на Волині,
себто потенційних правителів, і окремої православної Церкви, а також
зауважувалося, що волиняни не погоджуються на централізоване провадження
ревізії своєї нерухомості. І от що важливо, ніхто з істориків, які пишуть про
невелику кількість волинських вимог, не зауважує фрази: «Ще багато чого
вимагали, про що писати не потрібно». Що ж на думку автора було такого у
волинських вимогах, чого фіксувати не варто було? Цікаво, чому історики
ставляться до “Щоденника” як до  стенографічного протоколу?  

Все це відбувається на фоні вмовлянь, шаленого тиску і погроз, спрямованих на
волинян з вимогою присягнути. Врешті їм обіцяють не лише підтвердити їхні
окремі вольності, а й збільшити їх, якщо їм чогось бракуватиме, а також
письмово все це засвідчити, аби лише вони присягнули. Обіцяють не лише 24
травня, а й 1 червня, коли присягав гетьман князь Роман Сангушко. Він нагадав
про обіцянку від короля і учасників сейму його «братам» 24 травня. Архієписком
і головуючий у посольській палаті (маршалок Посольської ізби) вдруге
підтверджують готовність підтвердити привілеєм усе, чого волиняни
потребуватимуть. 12 червня волиняни подали свій проєкт привілею, який вони самі
для себе написали, який у підсумку з невеликою редакцією був затверджений. Там
були пункти про недоторканність кордонів, збереження у судочинстві свого права
– ІІ Литовського статуту, який не міг порушити жоден інший державний акт,
руська мова як мова суду і адміністрації, рівність вірних православної і
католицької Церков, надання королем урядів виключно місцевій шляхті і ще чимало
чого. Отакий привілей буквально вирвала волинська шляхта для себе, а також для
Брацлавського і Київського воєводств. І це в ситуації, коли ані представники
Польської Корони, ані еліта Великого князівства Литовського не бачила до
певного часу жодних підстав для автономії цих теренів. 

Додам, що в преамбулі привілеїв українським воєводствам вказувалося, що на
приєднання погодилися дві сторони – всі стани Королівства і всі стани цих
воєводств, себто йшлося про два суб’єкти. Привілеї затвердив король своєю
присягою, пообіцявши дотримуватися його пунктів вічно, пообіцяв від себе і від
своїх наступників. Надалі привілеї Волинській землі та Київському князівству
гарантуватимуть при коронації усі королі, тож вони увійшли в пакет основних
актів, які затверджуватимуться на всіх коронаціях. Цікаво, що українська шляхта
від початку почала говорити про унію з Короною, власне так цей акт називали
інколи й самі коронярі. Це про так званий «реституційний» характер привілеїв
для українських воєводств. А от привілей для Підляшшя справді мав саме такий
характер, де в преамбулі наголошувалося на «реституції» та «реінтеграції», а
король затверджував його лише підписом і печаткою. 

І наостанок. У мисленні тих людей унію могли укласти лише правителі держав.
Сюзерен не міг укласти унію з населенням певної частини іншої держави. Тут за
приклад може служити ситуація з Пруссією, яка з’явилася на землях Ордену
хрестоносців. Одна її частина була просто приєднана як провінція до Корони, бо
про це просили її стани, а володар другої частини, колишній магістр Ордену,
уклав з польським  королем ленну угоду.  Отож українські воєводства просто не
могли претендувати на рівноцінне членство у Речі Посполитій поруч із Великим
князівством Литовським і Польською Короною, бо це виглядало так, якби під час
договору двох суверенних держав три області однієї з них заявили, що вони
хочуть теж бути рівноправним суб’єктом перемовин. А от отримана на Люблінському
сеймі українськими воєводствами широка автономія таки стане з часом підставою
для певності української шляхти, що Русь є третім членом Речі Посполитої, який
увійшов до новопосталої 1569 р. країни на умовах рівноправної угоди. 

Отож відповідати на питання, які ми ставимо минулому, варто з позиції
тогочасних людей, їхніх уявлень і обставин. Інакше історик впадає в
найстрашніший професійних гріх – гріх анахронізму, осучаснення минулого.
Скажімо, про яку модель державотворення могло йтися волинянам на 1569 рік?
Великого князівства Руського, що дорівнювало Київській Русі? Волиняни мали
вступити у війну з великим князем і литовською елітою за всі землі Київської
Русі, якої давно не було? А чи сказати, що от Волинь і є Великим князівством
Руським, бо вона тотожна території Київської Русі? 

Відтак перш ніж братися за писання нової синтези, варто дослідити те, що
досліджене не було, та переосмислити те. що досі засвоюється за зразком більше
аніж столітньої давності  @igg{fbicon.face.smiling.eyes.smiling} 
