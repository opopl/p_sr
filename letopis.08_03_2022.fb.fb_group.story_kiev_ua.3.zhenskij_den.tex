% vim: keymap=russian-jcukenwin
%%beginhead 
 
%%file 08_03_2022.fb.fb_group.story_kiev_ua.3.zhenskij_den
%%parent 08_03_2022
 
%%url https://www.facebook.com/groups/story.kiev.ua/posts/1876895969173862
 
%%author_id fb_group.story_kiev_ua,stepanov_farid
%%date 
 
%%tags 
%%title ХРОНИКИ НЕОБЪЯВЛЕННОЙ ВОЙНЫ, Или: день тринадцатый. Женский день
 
%%endhead 
 
\subsection{ХРОНИКИ НЕОБЪЯВЛЕННОЙ ВОЙНЫ, Или: день тринадцатый. Женский день}
\label{sec:08_03_2022.fb.fb_group.story_kiev_ua.3.zhenskij_den}
 
\Purl{https://www.facebook.com/groups/story.kiev.ua/posts/1876895969173862}
\ifcmt
 author_begin
   author_id fb_group.story_kiev_ua,stepanov_farid
 author_end
\fi

ХРОНИКИ НЕОБЪЯВЛЕННОЙ ВОЙНЫ.

\ii{08_03_2022.fb.fb_group.story_kiev_ua.3.zhenskij_den.pic.1}

Или: день тринадцатый. Женский день.

Говорил, писал и ещё раз повторю: никогда не понимал этот праздник. Считал и
считаю, что мужчина может сделать каждый день для женщины особенным. И не нужно
ждать именно этого дня. Просто берите и делайте.


Знаю, что сам не идеален. Что не так часто говорю жене: \enquote{Люблю тебя} и дарю
цветы. 

Пусть и считаю, что делами, а не словами, но каюсь.

Мы все, последние дни, живём одним днём. Бывает, что и одним часом. Позавчера
подарил жене орхидею. Белоснежную, как сегодняшний утренний снег. Не стал ждать
... Мы все стали немного фаталистами. 

Когда сегодня утром увидел фотографии, которые прикрепил к этому тексту, долго
молчал. Подбирал слова. Он провожал жену и сына в эвакуацию, а сам остался
защищать их дом. Не знаю, кто автор фото, но они - вызывают сильные эмоции. Тот
случай, когда слова излишни. Очень хочу, чтобы эта семья вновь встретилась: она
дождалась его, он - дождался её. Чтобы все семьи воссоединились. И тогда будут
совсем другие слёзы. Слёзы радости и счастья. А праздник, спросите вы? Будет и
праздник.

Я помню рассказ бабушки, как мой дедушка вернулся с фронта в их полуподвальную
квартиру на Трёхсвятительской. Тогда, по малолетству, я не понимал всю важность
её рассказа. Сейчас - понял.

А за окном идёт дождь со снегом. Словно, сама природа плачет. Но дождь не может
идти вечно.

\ii{08_03_2022.fb.fb_group.story_kiev_ua.3.zhenskij_den.cmt}
