% vim: keymap=russian-jcukenwin
%%beginhead 
 
%%file 29_11_2022.fb.ohmatdet.bolnica.1.nyrky
%%parent 29_11_2022
 
%%url https://www.facebook.com/ndslohmatdyt/posts/pfbid0YWNVx9fcJQiqKC1FsSytHDmYtMEai28QQ5NEVJMgBMMrhd9QsXSfvznLYtahEQFrl
 
%%author_id ohmatdet.bolnica
%%date 
 
%%tags 
%%title Нирки не працювали з народження: лікарі Охматдиту змогли виходити новонароджену дитину з 0 до 2-х років та успішно пересадити нирку
 
%%endhead 
 
\subsection{Нирки не працювали з народження: лікарі Охматдиту змогли виходити новонароджену дитину з 0 до 2-х років та успішно пересадити нирку}
\label{sec:29_11_2022.fb.ohmatdet.bolnica.1.nyrky}
 
\Purl{https://www.facebook.com/ndslohmatdyt/posts/pfbid0YWNVx9fcJQiqKC1FsSytHDmYtMEai28QQ5NEVJMgBMMrhd9QsXSfvznLYtahEQFrl}
\ifcmt
 author_begin
   author_id ohmatdet.bolnica
 author_end
\fi

Нирки не працювали з народження: лікарі Охматдиту змогли виходити новонароджену
дитину з 0 до 2-х років та успішно пересадити нирку🙏🏻

20 років тому тільки почався зароджуватись такий напрям, як замісна ниркова
терапія у новонароджених. Тоді в Україні не існувало лікування для таких діток,
як 2-річний Богданчик. З народження його нирки не працювали — в Охматдиті
діагностували хронічну хворобу нирок 5 ступеня. Дитина потребувала замісної
ниркової терапії з перших місяців життя, аби заміняти функції непрацюючих
нирок.💔

\enquote{Такі маленькі пацієнти мають безліч ускладнень. У 2019 році Богданчика у
важкому стані взяли в Охматдит: тут провели \enquote{ювелірну роботу}, сформували
діалізний доступ та підібрали програму терапії. Перетоніальний діаліз дозволив
хлопчику рости вдома, а періодично перевірятися в лікарні},— розповідає Ольга
Бабічева, завідувачка відділення інтенсивної та еферентної терапії хронічних
інтоксикацій.💪🏻

Весь час організм дитини лікарі підтримували в чудовому стані, аби, коли він
підросте, серце та інші органи були готові для ниркової трансплантації.
Пересадку роблять дітям, які важать від 10 кілограмів. Зазвичай цю вагу дитина
має вже у рік, але при непрацюючих нирках, на замісній терапії, Богданчик важив
стільки у 2 рочки.⚡️

Ніхто з родини не підходив хлопчику як донор. Тому врятувати Богдана міг лише
посмертний донор. Ним став 37-річний чоловік, якому констатували смерть мозку.
Його родина дала згоду на трансплантацію органів. Шана донору та його родині.
Їх рішення врятувало життя 2-річного Богдана.🙏🏻

Трансплантація тривала всю ніч та пройшла успішно. Зараз 2-річний Богданчик
безтурботно бігає по коридору, радіє іграшковому автомобілю та навіть не
здогадується про те, який важкий шлях боротьби за життя він подолав. Тепер
дитина їде додому та більше не потребує діалізу.💛

🔻Усе лікування в Охматдиті було безкоштовним. Охматдит надає допомогу діткам з
усієї України. При потребі замісної ниркової терапії звертайтеся за телефоном:

044 236 52 13 — відділення інтенсивної та еферентної терапії хронічних
інтоксикацій;

У відділення лікують пацієнтів від 0 років з нирковою недостатністю.

🇺🇦Наша команда готова допомогти українським дітям, які потребують органних
пересадок. За допомогою звертатись до:

хірург-трансплантолог Олег Святославович Годік — 0 (50) 543 37 81

\ii{29_11_2022.fb.ohmatdet.bolnica.1.nyrky.pics}

%\ii{29_11_2022.fb.ohmatdet.bolnica.1.nyrky.orig}
%\ii{29_11_2022.fb.ohmatdet.bolnica.1.nyrky.cmtx}
