% vim: keymap=russian-jcukenwin
%%beginhead 
 
%%file 06_07_2020.fb.fb_group.story_kiev_ua.1.fotografia_vospominania.pic.6
%%parent 06_07_2020.fb.fb_group.story_kiev_ua.1.fotografia_vospominania
 
%%url 
 
%%author_id 
%%date 
 
%%tags 
%%title 
 
%%endhead 

\ifcmt
  ig https://scontent-frx5-1.xx.fbcdn.net/v/t1.6435-9/106985339_3380588828641367_2252510605577857949_n.jpg?_nc_cat=110&ccb=1-5&_nc_sid=b9115d&_nc_ohc=du2JSL1TN2kAX-eTtvg&_nc_ht=scontent-frx5-1.xx&oh=66974aac0aff17537b19ab7217e0352b&oe=61B85E3C
  @width 0.4
\fi

\iusr{Ирина Петрова}
Песочница)

\iusr{Luda Abramova}
С пасочами, это ностальгия за детством!

\iusr{Ирина Петрова}
Помню, что ведёрко было зелёное  @igg{fbicon.wink} 

\iusr{Luda Abramova}

И почему некоторым не нравится период нашего детства: кружков море, поездок -
море, гастроли со Дворцом пионеров! Я с мая по конец августа по гастролям,
отдых+ концерты. Но ведь было здорово и все б е с п л а т н о! Только когда
ехали в Германию (1969г.) Помню, с собой 20 р. И оплата 20 руб. Всё. Детей на
улице шатающихся не было, да и за ручку нас никто не водил, как мы наших! Да
что говорить, Ностальжи.

\iusr{Наталія Крюкова}
\textbf{Luda Abramova} 

И мне есть, что вспомнить!

Любимый театр кукол "Барвинок" Киевского Дворца пионеров под руководством
вечнолюбимых Всеволода Александровича и Евгении Яковлевны Канаш!

Этот детский театр существует по сей день - благодаря ученикам Канашей и
теперешнему бессменному руководителю - нашей Калинке, Леночке Коваленко!

Объездили с гастролями полСоюза и побывали в соцстранах.  Все это незабываемо!
Но не всем так везло.  Основная масса маршировала в пионерских лагерях или
трудилась на колхозных полях в лагерях труда и отдыха.  Тогда главное было -
вырваться из дому, от надзора родителей.
