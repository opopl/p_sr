% vim: keymap=russian-jcukenwin
%%beginhead 
 
%%file 26_10_2021.fb.ivanova_viktoria.doneck.1.covid_privivka_tromboz.cmt
%%parent 26_10_2021.fb.ivanova_viktoria.doneck.1.covid_privivka_tromboz
 
%%url 
 
%%author_id 
%%date 
 
%%tags 
%%title 
 
%%endhead 
\subsubsection{Коментарі}

\begin{itemize} % {
\iusr{Роман Чуни Муни}
Вот как бы и я ждун....... И каждый отвечает самолично за свой нос

\begin{itemize} % {
\iusr{Виктория Иванова}
\textbf{Роман Чуни Муни} абсолютно с тобой согласна, но согласись, от антиваксеров нет такой агитации и истерик, как от прививочных

\iusr{Роман Чуни Муни}
\textbf{Виктория Иванова} совершенно верно

\iusr{Юджин Иванов}
\textbf{Виктория Иванова} есть, и гораздо бОльшая...

\iusr{Ольга Бо}
\textbf{Виктория Иванова} 

Викуль, упоротость сама по себе мерзкая штука. Я, например, постоянно натыкаюсь
на безобразную агрессию с переходом на личности от антиваксеров. Я сделала
прививку, но не призываю и тем более не навязываю. Через месяц, видимо, таки
заболела, недели 3 не было обоняния. Я даже не заметила такой неприятности))) А
по поводу тромбов, нужно все-таки людям самим о себе заботится, то, что мы на
фиг не нужны никаким докторам, это уже аксиома. Я 3 дня до и 3 дня после пила
кардиомагнил и антигистаминные. Но пошла не на Спутник, а на эпивак. Спутник
прекрасно переносят пожилые люди, кто помоложе имеет вакцинную реакцию.

\iusr{Виктория Иванова}
\textbf{Ольга Бо} 

клиенту 60, он систематически пьёт Кардиомагнил, укололся Спутником Лайт.

Я не говорю, что тромбоз появится, у всех без исключения, его деев перенесла
вакцину прекрасно, но риск слишком велик и никто не знает на кого упадёт
жребий. Или это нужно воспринимать как естественный отбор? Не слишком ли
жестоко?

\iusr{Ольга Бо}
\textbf{Виктория Иванова} 

Вика, я не пропагандирую. Каждый выбирает сам. Я написала только свои действия
перед прививкой. Спасибо донецкой тетушке, рассказала что нужно делать. У нас
даже температуру не меряют. Укололи и вали. А что будет дальше неважно

\iusr{Виктория Иванова}
\textbf{Ольга Бо} та я ж не наезжаю, тоже мысли вслух высказываю))

\iusr{Елена Ланская}
\textbf{Виктория Иванова}, 

поэтому я и не иду вакцинировать. Недавно меня прооперировали. Перед операций
брали кровь на анализ... Еле.. еле смогли взять... такая густая. На RW даже
повторно сдавала... свернулась. А, что такое густая кровь ? Это предпосылки к
тромбами. К тому же у меня вены появились на ногах. Вот как вакцинироваться ?!
Это, я к тому, что люди, прежде всего, должны сами просчитывать слабые места в
своем организме. И конечно же учитывать свои хронические болячки.

\end{itemize} % }

\iusr{Юджин Иванов}

Вик, такое себе. Из разряда «привился, набухался, попал под машину - 100\%
прививка виновата».

\begin{itemize} % {
\iusr{Виктория Иванова}
\textbf{Юджин Иванов} Женя, не беси меня! Тромбоз и прививка, факт давно признанный. Вы реально не видите дальше своего носа.

\iusr{Юджин Иванов}
\textbf{Виктория Иванова} 

не говори за все вакцины подряд. Да и не все они одинаково полезны. Давай
начнём с того, какой именно вакциной он и его брат вакцинировались и была ли у
них предрасположенность к тромбозу? Или ты сразу бегом всё в одно смешала и все
вакцины обвинила? Просто подумай спокойно, без лишних эмоций.

\iusr{Виктория Иванова}
\textbf{Юджин Иванов} они в Донецке, вакцина у нас одна - Спутник Лайт

\iusr{Виктория Иванова}

Жень, вакцинируйтесь на здоровье. Я не понимаю дончан, мы 8 лет живём во лжи,
неужели это ничему не научило? Как можно колоть в себя не понятную, не
прошедшую толком исследования, бурду? Это моё мнение. Выше факт, который
отрицать глупо. И таким тромбозов полно, просто наша статистика делает вид, что
это не взаимосвязано. У меня на днях просили личный пример, вот он
@igg{fbicon.index.pointing.up}, привела

\iusr{Людмила Денисенко}
\textbf{Юджин Иванов} 

Жень, ключевое в твоём мессендже: « была ли предрасположенность к тромбозу?» А
кто это проверял? Вот меня тоже некоторые обвиняют, что я антиваксер. Никогда
не была, только всегда выступала за разумный подход к прививкам! Прививать
можно только здоровых, хотя бы даже для того, чтобы не давать повода говорить о
смертях после прививок. А кто проверяет уровень здоровья перед прививкой?
Температуру, давление и сатурацию в лучшем случае. А если ты переболел
бессимптомно и у тебя достаточный титр антител? Молчу об уровнях тестостерона,
витамина Д, ферритина, омега-3... И никто не пропагандирует неспецифические
методы профилактики, и никто не говорит, что молодые и здоровые, переболевшие
легко и бессимптомно, создадут более полноценную иммунную прослойку... В общем,
вопросов больше, чем ответов.

\iusr{Юджин Иванов}
\textbf{Людмила Денисенко} отчасти согласен. Но лишь отчасти. Перед вакцинацией заполняется опросный лист, там есть масса вопросов, если на них честно ответить как есть - думаю отводов было б в разы больше.

\iusr{Виктория Иванова}
\textbf{Юджин Иванов} 

пример из личной жизни. Моя мама в свои 55 была на удивление здоровой женщиной,
ни тебе давления, ни сердце, она не знала с какой стороны печень, и в один миг,
как гром среди ясного неба, онкология 4 стадия, а ведь ничего не предвещало. И
в больницы она ходила, и более менее систематически обследовалась, она пищевик.
О каком опроснике речь? Ты цепляешься за соломину,лишь бы не признать нашу
правоту. Если всё так безопасно, почему перед прививкой вы подписываете бумагу
об вашей ответственности, почему врачи её на себя не берут?

\iusr{Людмила Денисенко}
\textbf{Юджин Иванов} 

Жень, смешно, реально, опросник! Ты знаешь свой уровень хотя бы гемоглобина,
эритроцитов и тромбоцитов? Витамина Д, омега-3, тестостерона? Мои пациенты
процентов 90 даже не в курсе, что это такое, а кто и в курсе, проверяли пару
лет назад. Ко мне приходят уже с такими дефицитами, а вроде считают себя
здоровыми людьми. О чём мы спорим? О том, надо ли поголовно и насильственно
всех прививать? Так это даже не для Донецка тема, каждый день в ленте вопросы,
где есть вакцина. Ещё не всех желающих привили, а уже нападки на нежелающих.
При этом с агрессией, мол, всем и немедленно...

\iusr{Людмила Денисенко}
\textbf{Юджин Иванов} И да, по поводу отводов. Если сейчас нет отводов для онкобольных, с аутоиммунными заболеваниями, сахарным диабетом, беременным... О каких отводах по «опроснику» речь?

\iusr{Юджин Иванов}
\textbf{Людмила Денисенко} при мне нескольким людям в вакцинации отказали... причём одна даже возмущалась - мол сама из медиков, «могли б своим и сделать...»

\iusr{Андрей Медведев}
\textbf{Виктория Иванова}, 

на счёт бурды я написал отдельно. Что касается рисков, то это сотые доли
процента, а у ковида это порядка 3\% летальности, ну и последствия тоже ещё тот
хоррор.

\iusr{Людмила Денисенко}
\textbf{Юджин Иванов} Ну, на всех всё равно вакцины пока не хватает!  @igg{fbicon.wink} 

\iusr{Юджин Иванов}
\textbf{Людмила Денисенко} и как это меняет тот факт, что таки не всех без разбору вакцинируют как любят писать антиваксеры?  @igg{fbicon.wink}  Я лично наблюдал реальный пример когда нескольким людям отказали по медпоказаниям.

\iusr{Виктория Иванова}
\textbf{Юджин Иванов} Жень, я привела конкретный пример. Решил человек сделать вакцину, милости просим, колим всех и без разбора. в ТЦ, где массово делают прививки, даже температуру не всём меряют

\iusr{Юджин Иванов}
\textbf{Виктория Иванова} а где у нас в ДНР колят в ТЦ? Ты ж говорила что человек из твоего рассказа тут вакцинировался? Или таки в РФ? Или это уже совсем о другом речь?

\iusr{Виктория Иванова}
\textbf{Юджин Иванов} да причём тут мой человек! В РФ штампуют всём подряд в торговых центрах. Что это меняет, у нас или не у нас? Давно мы стали самостоятельными или у нас какой то другой Спутник?

\iusr{Виктория Иванова}
Это был комментарий к твоему утверждению * что всё таки не всех без разбора вакцинируют*

\iusr{Юджин Иванов}
\textbf{Виктория Иванова} ясно. Смешались в кучу кони, люди...

\iusr{Людмила Денисенко}
\textbf{Юджин Иванов} А я, как врач, ежедневно сталкиваюсь с теми, кому прививаться нельзя было совсем... И да, на мой вопрос о том, знаешь ли ты показатели своего здоровья, ответа не получила. Красную кровь, как донор, понимаю, сдаёшь регулярно, а остальное?  @igg{fbicon.face.flushed} 

\iusr{Виктория Иванова}

И вот ещё что. Я могла привиться ещё во вторую волну, когда поступил первый
Спутник Лайт, но каким то внутренним чутьём отказалась, не пошла, а ведь тогда
ещё не было такой статистики о тромбозе, вызванным вакциной. Так вот, у меня
предрасположенность, варикоз нижних конечностей, так что сделав прививку,
совсем не факт, что я бы сейчас с тобой разговаривала

\iusr{Елена Ланская}
\textbf{Виктория Иванова}, аналогично..

\iusr{Екатерина Карманова}
\textbf{Виктория Иванова} Я тоже Ждун. 3 года назад во время медосмотра у меня обнаружили предраковые клетки. Прооперировалась. Кто мне даст гарантию, что после прививки не пойдет новый рост? Никто.
\iusr{Людмила Денисенко}
\textbf{Екатерина Карманова} Ну, сейчас говорят о том, что онкология- не поотивопоказание!  @igg{fbicon.face.flushed} 

\iusr{Виктория Иванова}
\textbf{Екатерина Карманова} никто, они даже здоровым людям никаких гарантий не дают, заставляют это письменно подписывать

\iusr{Виктория Иванова}
\textbf{Людмила Денисенко} Светлана Саф говорит, что у неё отвод, вроде бы

\iusr{Людмила Денисенко}
\textbf{Виктория Иванова} Повезло ей с думающим врачом, который дал отвод.

\iusr{Виктория Иванова}
\textbf{Людмила Денисенко} это да, она у хороших докторов наблюдается

\iusr{Сергей Владимиров}
\textbf{Людмила Денисенко} Чтобы тромбозы исключить после прививок ,да !? Кстати, Вы привились?

\end{itemize} % }

\iusr{татьяна шатрова}
такого же мнения.

\iusr{Халиль Чудин}

Вика, мои друзья, муж и жена, перед поездкой в Геленджик вакцинировались, по
приезду на курорт оба почувствовали себя хреново, обратились к врачу, сдали пцр
тест который показал корону. Две недели провели в номере безвылазно на
таблетках

\begin{itemize} % {
\iusr{Виктория Иванова}
\textbf{Халиль Чудин} самый распространённый аргумент от ваксеров, это : ну вы же не противитесь прививкам от оспы, столбняка и т.д... Ну что тут скажешь, я не медик, но точно знаю, что после этих прививок человек не брокет оспой, ни в лёгкой, ни в тяжёлой форме. Может потому что пока эту вакцину довели до совершенства прошло ни одно десятилетие?

\iusr{Халиль Чудин}
\textbf{Виктория Иванова} я уже не вступаю с ними в споры. бесполезное занятие.

\iusr{Виктория Иванова}
\textbf{Халиль Чудин} просто на днях один или одна ваксерша потребовала у меня пример из личного опыта, вплоть до фамилий, фамилии я конечно называть не буду, но пример, вот он, пожалуйста

\iusr{Иола Гар}
\textbf{Виктория Иванова} , 

дело не только в "недоведении до совершенства" вакцин, но и в том, что это
делается в разгар эпидемии. В первые дни после вакцинации количество лимфоцитов
падает и встреча с вирусом может привести к тяжелому течению или даже смерти (в
том числе и потому, что протромботическое действие вакцины и вируса
суммируются).

\iusr{Андрей Медведев}
\textbf{Виктория Иванова}, оспой последний раз заразились и болели в 1977 году. Сейчас вакцинируют для поддержания иммунной прослойки у населения .

\iusr{Людмила Денисенко}
\textbf{Андрей Медведев} Против оспы не вакцинируют с 1980 года. И не сравнивайте вакцину против оспы и нынешние «уколы».

\iusr{Виктория Иванова}
\textbf{Иола Гар} я рассуждаю, как дилетант, я не медик, но честное слово, уже не хватает аргументов, народ тупо не хочет видеть очевидного

\iusr{Сергей Владимиров}
\textbf{Виктория Иванова} 

ага.. "никто не видит...! (((( Министерство здравоохранения ДНР информирует,
что 02 ноября проведено 4060 исследований, из них выявлено 1800 случаев
заболевания (44,3 \%) и у 37 пациентов диагноз COVID-19 установлен
клинико-эпидемиологически.

Всего по состоянию на 10:00 02 ноября заболевших инфекцией COVID-19 на территории Донецкой Народной Республики –88660
на лечении (стационарном и амбулаторном) –16293 пациентов,
выписаны –65569 человек,
летальных случаев – 6798 (101 за трое суток)
За сутки выявлено 250 случаев пневмоний, что на 34 случаев больше, чем за прошедшие сутки. Из них госпитализировано 161. Всего на лечении 3556 пациентов с пневмониями, из них 1766 на кислородной поддержке (+103 за сутки).
Источник: 

\href{https://mzdnr.ru/news/po-sostoyaniyu-na-1000-02-noyabrya-vsego-88660-zabolevshih-infekciey-covid-19-na-territorii}{%
По состоянию на 10:00 02 ноября всего 88660 заболевших инфекцией COVID-19 на территории Донецкой Народной Республики, %
mzdnr.ru, 02.11.2021%
}

© mzdnr.ru
\end{itemize} % }

\iusr{Некрасова Ольга}

Вот тоже странность, которая меня жутко возмущает. Почему, зная что после прививки
может быть тромбоз, заранее не назначают антикоагулянты? Казалось бы, чего
проще.. И, кстати, с самого начала везде и всюду вещали что аспирин пить нельзя, а
ведь он тоже кровь разжижает.. Все больше вопросов, все сильнее сомнения.

\begin{itemize} % {
\iusr{Виктория Иванова}
\textbf{Некрасова Ольга} хочешь ржаку?)) на днях прочитала отчёт медиков))
Оказывается, они не отрицают того, что в следствии прививки может развиться тромбоз, НО! оказывается только в случае, если вакцина попадает напрямую в кровяной сосуд  @igg{fbicon.man.facepalming} 
Это пипец просто

\iusr{Виктория Иванова}
Факт есть, его отрицать невозможно, придумали вот такой бред))

\iusr{Иола Гар}
\textbf{Виктория Иванова} , конечно же, содержимое вакцины попадает в кровеносное русло в любом случае.

\iusr{Виктория Иванова}
\textbf{Иола Гар} настаивают на прямом попадании, то есть когда проколов кожу, игла попадает непосредственно в кровеносный сосуд.)) ну вот они, медики, несут такой бред))

\iusr{Иола Гар}
\textbf{Виктория Иванова} , да я поняла, о чем речь.
\end{itemize} % }

\iusr{Юрий Алейников}

У нас в принципе то же самое. Только выбор - четыре вакцины. Многие, если не
большинство, их делает без обследования. Осложнений и летальных случаев
дохрена, как и повторных заболеваний. Чтобы узнать, какие осложнения могут дать
прививки, и кому противопоказано, я позвонил в свою поликлинику МВД. Вот там
да, все объяснили подробнейшим образом. Но это единичные случаи. Я буду
держаться до последнего, потому как в "чудодейственную вакцинку" не верю. Там
только бабло. А чего ж лекарств нет? Правильно, лекарства потом будут лечить
осложнения после вакцинки.

\begin{itemize} % {
\iusr{Виктория Иванова}
\textbf{Юрий Алейников} лекарств нет потому, что никто нихрена толком об этой гадости не знает, зато вакцину придумали за три месяца, бабло же стричь надо

\iusr{Юрий Алейников}
\textbf{Виктория Иванова} почему ж, не верю, что нет, только сейчас гробят вакцинкой, а потом всех будут лечить от её последствий. То же бабло.

\iusr{Андрей Медведев}
\textbf{Виктория Иванова}, 

Вика, вирус это внутриклеточный паразит в отличии от бактерии которая снаружи
клетки. Соответственно если бактерию мы можем убить десятком способов, а как
убить вирус который в строился в нашу клетку? Убить клетку? Поэтому и нет
практически именно препаратов ПРОТМВОВИРУСНЫХ, убивающих именно вирус, вне
клетки это не живой объект, те противовирусные препараты, что продаются в
аптеке это по существу симптоматические препараты и профилактирующие
осложнения. Так что очередной тебе лично и остальным неуд.

\end{itemize} % }

\iusr{Дмитрий Карский}
Пишу Вам с того света. Умер от тромбоза

\begin{itemize} % {
\iusr{Виктория Иванова}
\textbf{Дмитрий Карский} я всегда считала вас умным человеком, Дмитрий. Это не выдуманная мной история, это реальность, добрее надо быть, человек всё таки в реанимации. Или вы впервые слышите о взаимосвязи между тромбозом и прививкой?

\iusr{Дмитрий Карский}
\textbf{Виктория Иванова} такая связь действительно есть, о ней знают все группы риска, как знают и о вакцинах, которые чаще всего вызывают тромбоз, как есть и средства для разжижения крови, которые тромбозу препятствуют

\iusr{Виктория Иванова}
\textbf{Дмитрий Карский} человек в возрасте, 60 лет, кардиомагнил принимает регулярно, вакцина Спутник лайт.

\iusr{Юджин Иванов}
\textbf{Виктория Иванова} врачей предупреждал о том, что кардиомагнил принимает? Ведь его не от нечего делать пьют, не так ли? Или промолчал, «авось сойдёт»? Ибо в противном случае возможно б в прививке ему просто отказали.

\iusr{Виктория Иванова}
\textbf{Юджин Иванов} кардиомагнил, это практически модный аспирин, его рекомендуют пить всём после 45-50, кровь разжижает. Не ищите кошку в чёрной комнате.

Его родной брат младше его на 8 лет, ничего не принимал, на сердце не жаловался.

Женя, это факт. Почитай исследования, год уже говорят о взаимосвязи между
вакциной и тромбозом. Я тебе больше скажу, российские медики признали этот
факт, вот только очень смешно оправдываются, мол, да, связь есть, но только в
том случае проявляется тромбоз, если вакцина попала в кровяной сосуд.

@igg{fbicon.face.tears.of.joy} 

\iusr{Дмитрий Карский}
\textbf{Виктория Иванова} 

РФПИ объявляет данные Министерства здравоохранения Парагвая, демонстрирующие
высокие показатели безопасности однокомпонентной вакцины "Спутник Лайт" и
эффективность 93,5\% в ходе массовой вакцинации в стране»,— сообщается на сайте
вакцины.

При расчете эффективности учитывались данные 320 тыс. человек, которые сделали
прививку до 30 июля. По итогам их вакцинации не зафиксировано случаев смерти,
церебрального венозного тромбоза, синдрома Гийена-Барре, повышенной
проницаемости капилляров, миокардита или перикардита.

\end{itemize} % }

\iusr{Юля Николаевна Кордюкова (Филиппская)}

Я ещё один ждун...  @igg{fbicon.hands.shake} 

Мне показалось что ты упустила ещё один очень важный факт.... Ответственность
за последствия после прививки лежит на прививаемом! То есть эксперимент за свой
счёт (а здоровье, это личный счёт каждого) продолжается и настолько успешно,
что в школах непривитых гнобят уже даже учителя....не говоря о шантаже
увольнения и паспортов прививок при путешествиях.... Лично я наблюдаю вокруг
меня (а большинство моего круга общения люди привитые) лютую агрессию, даже
среди самых спокойных и флегматичный людей, какой то такой постоянный стресс,
агрессивные поступки, которые люди не могут объяснить.... Какие то приливы
агрессивных форм, потом их попускает на определенный период, потом снова.  @igg{fbicon.thinking.face}  @igg{fbicon.frown} 

\begin{itemize} % {
\iusr{Виктория Иванова}
\textbf{Юля Николаевна Кордюкова (Филиппская)} абсолютно мои мысли излагаешь, но если писать все пункты, вечера не хватит. У меня на днях просили личный пример, чуть ли не с фамилиями, я им его привела. Но даже в этой ситуации они ерничают и стебутся.

\iusr{Юля Николаевна Кордюкова (Филиппская)}
\textbf{Виктория Иванова} Ничё смеётся тот, кто смеётся последним!  @igg{fbicon.wink} 
\end{itemize} % }

\iusr{Некрасова Ольга}

.Впервые в жизни встретила препарат, не имеющий противопоказаний. Даже
люди, находящиеся на иммуносупрессии, после массивных трансплантаций, оказывается
тоже должны быть вакцинированы. Беременные, для которых масса препаратов просто
запрещены, тоже должны быть вакцинированы. Кто изучил последствия и риски для
ребенка?

\begin{itemize} % {
\iusr{Виктория Иванова}
\textbf{Некрасова Ольга} никто! Беременных и кормящих заставляют колоться, нонсенс

\iusr{Некрасова Ольга}
Жуть
\end{itemize} % }

\iusr{Наташа Скрипка}

И я ждун. Даже болея сейчас, я ни ЗА ни ПРОТИВ. Вот не могу твердо сказать :
за, против. Все это очень серьезно.

\begin{itemize} % {
\iusr{Виктория Иванова}
\textbf{Наташа Скрипка} именно потому что очень серьёзно, нельзя бездумно совершать такие поступки, тем более, что никто из врачей ответственность на себя не берет, как мне выше Юля напомнила

\iusr{Наташа Скрипка}
\textbf{Виктория Иванова} да.
\end{itemize} % }

\iusr{Наталия Королёва}

Вогнали народ в прострацию полнейшую. Да, Викуль, согласна полностью.
Вакцинироваться, не имея полного представления состояния своего здоровья, бред
бредовый. Это ровным счётом, как моя сестра,.... Жила с " бомбой замедленного
действия" в голове. Пока не коснулось ...и знать не знали, что у неё такой
диагноз. Уверена, что ей бы прививка точно бы " вышла боком"....

\begin{itemize} % {
\iusr{Виктория Иванова}
\textbf{Наталия Королёва} у меня мама была очень здоровым человеком, в 55 ни давление, ни голова не болела, она не знала с какой стороны печень, и тут в один миг, как гром среди ясного неба, онкология 4 стадия. Ничего не предвещало, вот ну совсем

\iusr{Наталия Королёва}
\textbf{Виктория Иванова} и как на бум делать прививку?  @igg{fbicon.thinking.face}  А на МРТ и всякую байду, чтобы вакцинироваться и пройти полную диагностику откуда у людей деньги? Какой-то трэш

\iusr{Виктория Иванова}
\textbf{Наталия Королёва} дам какие исследования, если фигачат прям в ТЦ
\end{itemize} % }

\iusr{Евгения Михайлова}
И я ждун. Даже с лёгкими медицинскими знаниями я пока ждун.

\begin{itemize} % {
\iusr{Виктория Иванова}
\textbf{Евгения Михайлова} я далеко не медик, даже на уровне санитарки, но интуиция и здравый смысл меня останавливают
\end{itemize} % }

\iusr{Наталия Королёва}

В дополнение @igg{fbicon.thinking.face} ... У супруга на работе сделал прививку его коллега. Итог- через
4 дня температура под 40. Но всё бы ничего, если бы не пневмония. Совпадение
или нет.... не могу судить. Вот так то.

\begin{itemize} % {
\iusr{Виктория Иванова}
\textbf{Наталия Королёва} ваксеры скажут, чтот нет или сам виноват))

\iusr{Наталия Королёва}
\textbf{Виктория Иванова} угу))))  @igg{fbicon.face.tears.of.joy} 

\iusr{Ольга Левченко}
\textbf{Наталия Королёва} тоже самое у моего свекра...(...сейчас в реанимационном отделении с 2стор пневмонией

\iusr{Виктория Иванова}
\textbf{Ольга Левченко} к сожалению, такие случаи довольно частые, только статистика молчит, делает вид, что ничего такого нет.

\iusr{Ольга Левченко}
\textbf{Виктория Иванова} естественно врачи говорят, что связи с прививкой нет! типа раньше уже заразился

\iusr{Наталия Королёва}
\textbf{Ольга Левченко} и те же самые врачи не спешат делать прививку.

\iusr{Сергей Владимиров}
\textbf{Ольга Левченко} А что, не мог раньше заразиться?...
\end{itemize} % }

\iusr{Юрий Афендулов}

А я люблю овощи фрукты и употребляю, а там витамины и всякое полезное. Они
повышают иммунитет, вот пусть и борется, зря что ли его кормлю.

\begin{itemize} % {
\iusr{Виктория Иванова}
\textbf{Юрий Афендулов} вот даже возразить нечего, я свой тоже балую периодически, пусть отрабатывает)))

\iusr{Александр Валентинович}
\textbf{Юрий Афендулов} к овощам рекомендую вакцину  @igg{fbicon.face.smiling.halo} 

\ifcmt
  ig https://scontent-frt3-1.xx.fbcdn.net/v/t39.30808-6/247553569_593713398719936_3755566532133177107_n.jpg?_nc_cat=106&ccb=1-5&_nc_sid=dbeb18&_nc_ohc=gtUzMHnVZH0AX_h2lsj&_nc_ht=scontent-frt3-1.xx&oh=fb6ff0720d682a42ad9387e7c97845d6&oe=61A12ACA
  @width 0.4
\fi

\iusr{Юрий Афендулов}
У меня свой рецепт вот это всё что на этикетках всё в одном. Киндер сюрприз называется.

\iusr{Юрий Афендулов}
А как холодеет иммунитет начинает сало просить, скоро угощу.

\iusr{Виктория Иванова}
\textbf{Юрий Афендулов} надо попробовать, вакцину именно этой марки я ещё не исследовала))

\end{itemize} % }

% -------------------------------------
\ii{fbauth.volik_irina.doneck.dnr.sobaki.prijut}
% -------------------------------------

Вот и я из за тромбоза боюсь.... Ещё тот ждун я!!!

\begin{itemize} % {
\iusr{Евгения Михайлова}
\textbf{Ирина Волик} тебе нельзя тромбоз. И мне нельзя. Ишь ты, как легко от жизни можно отделаться..

\iusr{Виктория Иванова}
\textbf{Евгения Михайлова} так и у меня тромбоз) а давайте создадим свою партию?  @igg{fbicon.face.tears.of.joy} 

\iusr{Ирина Волик}
\textbf{Виктория Иванова} неее, секту!) Так круче звучит)))

\iusr{Виктория Иванова}
\textbf{Ирина Волик} супер, точно. Будем там главными ведьмами)))

\iusr{Ирина Волик}
\textbf{Виктория Иванова} ага!!!)

\iusr{Наталия Королёва}
\textbf{Ирина Волик} жуй фрукты. Имунку береги. Меньше нервничай.

\iusr{Ирина Волик}
\textbf{Наталия Королёва} вот именно, не нервничать на грани фантастики @igg{fbicon.face.tears.of.joy}  За имунку спасибо
@igg{fbicon.exclamation.mark}
\end{itemize} % }

\iusr{Abraimov Leonid}

Нкоим образом нельзя подменять реальную статистику личным и не личным -
косвенным опытом и эмоциями. Все, конечно, индивидуально. И если я скажу ( и
это так) что я, моя семья, десятки знакомых привились и живы-здоровы, это Вас
убедит привиться? Авторитеты говорят, последствия прививок непредсказуемы в
будущем (опять таки, индивидуально) Но, пока надо выжить. Или - или, другого
пути(пока) нет

\begin{itemize} % {
\iusr{Виктория Иванова}
\textbf{Abraimov Leonid} то есть, если привившись человек умирает от тромбоза, то это так, всего лишь побочный эффект? вы предлагаете играть в рулетку, пронесёт-не пронесёт? А ничего что мы о медицине тут говорим?

\iusr{Юля Николаевна Кордюкова (Филиппская)}
\textbf{Abraimov Leonid} Сегодня статистика это политика, какую надо, такую и напишут!!!!

\iusr{Виктория Иванова}
\textbf{Юля Николаевна Кордюкова (Филиппская)} и деньги. Но ты ж понимаешь, люди укололись, а вдруг что-то пойдёт не так) вместе с прививкой, по моему, народу и шоры надевают, иначе я не знаю, как можно отрицать очевидное

\iusr{Юля Николаевна Кордюкова (Филиппская)}
\textbf{Виктория Иванова} 

до поры, до времени.... Никто не знает чем их начиняют и какая реакция от этого
будет...такая себе бомба замедленного действия... поэтому только ожидание
расставит все по местам... Ждём, наблюдаем, фиксируем...  @igg{fbicon.wink} 

\iusr{Виктория Иванова}
\textbf{Юля Николаевна Кордюкова (Филиппская)} ждём) и набоюдаем

\iusr{Abraimov Leonid}

ЕЕсли у Вас есть опасения, тщательно обследуйтесь Поскольку все индивидуально.
И опираться надо на реальную статистику, коей не личный опыт, не круг знакомых,
и тем более соцсети не являются. Рулетка? Да. Или от прививки, или от болезни.
Вероятности оцените сами, по возможности без эмоций. Но даже при всем
скептицизме к оф.стастике....

\iusr{Abraimov Leonid}
\textbf{Юля Николаевна Кордюкова (Филиппская)} но "личный" опыт и соцсети ещё хуже. На информационной войне, как на войне.

\iusr{Alina Makalish}
\textbf{Abraimov Leonid} хорошо я сдала анализы и врач мне сказал вы как будто болели совсем недавно ... Может быть бессимптомно но отвод от прививки мне никто не дал ))) так что анализы до одного места если есть команда всех вакцинировать

\iusr{Abraimov Leonid}
Нет такой команды-всех! ( В отличии от многих стран, гдеобстановка гораздо лучше)

\iusr{Виктория Иванова}
\textbf{Abraimov Leonid} официально нет, но как относиться к факту, что людям грозят увольнением, если не привьются? Это мы назовём добровольно-принудительно?

\iusr{Abraimov Leonid}

Если вы уверены, что вакцина вас убьет или покалечит, чтож, выбирайте.
Повторяю, это не ваше личное дело, это война и мобилизация,
добровольно-принудительная. А вот у многих полная (обязательная вакцинация). А
на войне есть раненные и убитые.

\iusr{Виктория Иванова}
\textbf{Abraimov Leonid} 

давайте закончим эту бессмысленную дискуссию, иначе я теряю нить разговора,
наблюдая какой то диссонанс. То вы утверждаете, что никто никого не заставляет
насильно, то война и мобилизация, а значит всё таки принудиловка  @igg{fbicon.man.facepalming} 

Не болейте и пусть у вас всё будет хорошо, ну а я всё таки подожду  @igg{fbicon.wink} 

\iusr{Abraimov Leonid}

Не вижу диссонанса. Да, мобилизация, но дезертиров никто не ловит и не
расстреливает. И если вы азартный человек - играйте в рулетку дальше, ваше
право. Может, повезет, тем более выигравших пока больше проигравших. А если вы
на передовой, будьте добры, наденьте каску и бронежилет.

\iusr{Виктор Ю. Артамонов}
\textbf{Виктория Иванова} от бодуна тож лекарства нет, и чо?

\end{itemize} % }

\iusr{Abraimov Leonid}

Ждунам. Ваше полное право ждать полной ясности (если повезет). Но не
подвергайте опасности других (бессимптомные носители - фак). Изолируйтесь,
носите маски, противогазы и ... Помните, в мире -городе вы не одни, и другие
тоже хотят жить и не болеть.

\begin{itemize} % {
\iusr{Виктория Иванова}
\textbf{Abraimov Leonid} а кто сказал, что мы пренебрегаем способами защиты? И маски носим, и руки моем. А ещё иммунитет свой повышаем. И так, на всякий случай уточню, привитые в такой же степени могут быть заразными, и не факт, что привитый заразившись, выживет

\iusr{Abraimov Leonid}
Ваша работа предполагает активное общение. Т.с. группа риска, берегитесь и берегите других.

\iusr{Виктория Иванова}
\textbf{Abraimov Leonid} спасибо, применяю всё, чтобы обезопасить себя и других.
\end{itemize} % }

\iusr{Андрей Медведев}
\textbf{Виктория Иванова}

Про сроки.... Говард Флори и Эрнст Чейн в 1940 г. выделили препарат
(пенициллин) в чистом виде и изучили его терапевтические свойства. 12 февраля
1941 г. инъекция пенициллина впервые была сделана человеку.

Прошло не больше года всего. Наука с 1940 года прошлого столетия шагнула в
перед?

И да, векторному методу уже что-то порядка 30 лет.

\begin{itemize} % {
\iusr{Виктория Иванова}
\textbf{Андрей Медведев} пенициллин, антибиотик практически природного происхождения, это лекарство. От Ковид лекарства нет, зато вакцина есть. Смешно даже.

\iusr{Андрей Медведев}
\textbf{Виктория Иванова}, я тебе ответил и на этот вопрос в ветке

\iusr{Андрей Медведев}
\textbf{Виктория Иванова} в данном случае это про "жижу" и сроки разработки и начала применения, опять же чем вирус менее природен нежели плесень?

\iusr{Виктория Иванова}
\textbf{Андрей Медведев} 

Андрюш, я не имею медицинского образования, но у меня хорошо развита интуиция и
я поклонник здравого смысла, всё что я вижу, мне не нравится. Моё твёрдое
убеждение, что это непонятная бурда, если быть честными, то врачи очень мало
знают ещё об этой заразе, чтобы изобрести от неё панацею. Поэтому я подожду, а
учитывая кучу моих хронических заболеваний, в том числе и тромбоз нижних
конечностей, 200 раз подожду. Я не мышь подопытная и не самоубийца, играть в
рулетку желания ну совсем нет. А ваша волшебная пилюля, на сегодняшний день, не
что иное, как русская рулетка


\iusr{Андрей Медведев}
\textbf{Виктория Иванова}, панацеи нет, верно, но и ситуация не "мирного времени"

\iusr{Андрей Медведев}
\textbf{Виктория Иванова}, тромбозы это медотвод

\iusr{Елена Ланская}
\textbf{Виктория Иванова} ... что-то, кажется, с плесенью связано..?

\iusr{Виктория Иванова}
\textbf{Андрей Медведев} 

да ладно, и давно это отвод? Когда мне первый раз предложили уколоться, тогда
тромбофлебит не был в списке отводов, это сейчас, когда по всему миру
прокатилась волна побочки. И если бы я тогда не послушала свою интуицию, не
факт, что сейчас мы бы с тобой разговаривали.

И потом, как относиться к бреду, который несут медики? Понятно, что отрицать
взаимосвязь между тромбами и вакциной смешно и глупо, они придумали отмазку :
тромбы образуются только в том случае, если вакцина попала непосредственно в
кровеносный сосуд.))

То есть даже при таком раскладе, получается таки рулетка, повезёт - не повезёт?

\end{itemize} % }

\iusr{лариса орлова}

Вика, а я вообще молчу, но у меня достаточно фактов и не столь слухи, как, где,
когда и что происходит , посему не мешало бы неким усиленным прививочникам и
головы включать.. не всё т просто

\begin{itemize} % {
\iusr{Виктория Иванова}
\textbf{лариса орлова} бесполезно, они не хотят даже очевидного видеть, не говоря о здравом смысле.

\iusr{лариса орлова}
это их путь я каждый день почти слушаю из разных стран достойных простых враче, и не только просто специалистов в свое области и....

\iusr{лариса орлова}
\textbf{Виктория Иванова} стадность преобладает

\iusr{Виктория Иванова}
\textbf{лариса орлова} 

проблема в том, Лариса, что мы верим разным врачам. И я не шучу, когда говорю,
что после прививки люди меняются, ген агрессии возбуждается. Ну укололись вы,
вам так легче? На здоровье. Но нет же, они крови хотят. Расстреливайте,
увольняйте, сажайте тех, кто не с нами. Миф о коллективном иммунитете, при
постоянно мутирующем вирусе, это нонсенс какой то, но они верят.

\end{itemize} % }

\iusr{лариса орлова}
\url{https://www.facebook.com/watch?v=597741001428986}

\iusr{лариса орлова}
имеющий уши и глаза, может, сделает какой-то вывод

\iusr{Aleksey Anatolievitsch}
Лекарства есть.
Гидроксихлорохин.
Или Ивермектмн

\begin{itemize} % {
\iusr{Виктория Иванова}
\textbf{Aleksey Anatolievitsch} так просто? А народ то гибнет пачками..  @igg{fbicon.face.smiling.sunglasses} 

\iusr{Victoria Francis}
\textbf{Виктория Иванова} потому ,что тех врачей, (которые приписывают эти 2 высокоэффектных для короны, но очень дешевых препарата) ГНОБЯТ, вплоть до потери работыи отбирания лицензии ( диплома)
\end{itemize} % }

\iusr{Людмила Денисенко}

Виктория, не тратьте силы, нервы и время, пытаясь кому-то что-то доказать.
Отсылайте на круглый стол на странице Марии Шукшиной. Там профессора и
академики, вирусологи, иммунологи, эпидемиологи, военные медики, организаторы
ЗО и масса других спецов доходчиво всё объясняют. Правда, я мало верю, что те,
кто с пеной у рта пытается что-то доказать, будут это слушать. Им зачем? Они
всё и так знают... \url{https://t.me/ksmariishukshinoj/172}

\begin{itemize} % {
\iusr{Виктория Иванова}
\textbf{Людмила Денисенко} 

я бы с удовольствием закрыла комментарии, вы правы, разговор слепого с глухим,
но в личной ленте, это невозможно. И да, читать не будут, почему то известные,
заслуженные люди медицины для них не авторитет, они как зомбированные.

Мы смотрим, слушаем и верим разным врачам, но, уверена, время всё расставит по
местам.

Спасибо за ссылку, с удовольствием посмотрю  @igg{fbicon.hands.pray} 


\iusr{Людмила Денисенко}
\textbf{Виктория Иванова} Там четыре часа, я смотрела с перерывами, но там люди, которых знаю и уважаю, которые не боятся говорить правду.  @igg{fbicon.heart.eyes} 

\iusr{Виктория Иванова}
\textbf{Людмила Денисенко} вот за это я их тоже сильно уважаю
\end{itemize} % }

\iusr{Victoria Francis}

Виктория, Вы - совершенно правы. Связь между прививкой о короны и тромбо -
образованием уже известна и ВНЕСЕНА в список последствий, и предствлена на
ОФИЦИАЛЬНЫХ веб- сайтах. Большинство привитых получают тромбозы в микро-
циркуляции, что ведет в multiple organ failure = отказ множественных органов (
кажется так это звучит по- русски.. простите, я оставила русскую медицину 20
лет назад, подзабыла терминологию)..... Почему диапазон реакции ( от совсем
хорошо до совсем плохо), могу обьяснить, так как была вовлечена в прививочных
процесс огромной корпорации и имела доступ к спец информации..... но это, как
нибудь в другой раз....


\iusr{Евгения Ковалец}

На тебе еще поленьев для интеллектуального костра. Только что прочла. И да,
тоже говном не кидалась.

\href{https://news.mail.ru/society/48533184/}{%
Коронавирус: ученые опасаются мутации А.30, обходящей вакцины, 28.10.2021, news.mail.ru%
}

\iusr{Alla Lisenko}

Витунь, я сегодня приехала с дачи, с деревни. Сегодня днем захожу в наш местный
магаз, спрашиваю про деревенские новости. Так вот женщина 40 лет работает
санитаркой в больнице... сделала первую прививку \enquote{короноваком} (говорили, что он
самый щадящий) ..Вот тоже на ТРЕТИЙ ДЕНЬ ей стало плохо, приехала скорая, забрали
в ее же больницу-инфаркт, в 40ЛЕТ.. слава Богу вовремя... Сейчас от больницы
направляютее в санаторий и пока оформляет группу...

\begin{itemize} % {
\iusr{Андрей Медведев}
\textbf{Alla Lisenko}, ох уж эти мне новости из магаза  @igg{fbicon.laugh.rolling.floor}{repeat=3} 

\iusr{Андрей Медведев}
\textbf{Alla Lisenko}, ладно, сделаю сегодня публикацию с инфой из ковидариев, моих коллег

\iusr{Alla Lisenko}
\textbf{Андрей Медведев} хорошо... и мне ж скажите...

\iusr{Сергей Владимиров}
\textbf{Андрей Медведев} Что скажете, Доктор? Что непривитые в большинстве поступают тяжёлыми и уходят чаще...?

\iusr{Андрей Медведев}
\textbf{Alla Lisenko}, так опубликовал

\iusr{Андрей Медведев}
\textbf{Сергей Владимиров}, согласно общению с докторами от туда, да.

\iusr{Сергей Владимиров}
\textbf{Андрей Медведев} А где опубликовали? Не вижу...

\iusr{Андрей Медведев}
\textbf{Сергей Владимиров}, открыл публикацию
\end{itemize} % }

\iusr{Михай Коростель}

Вика, все правильно. К сожалению это только начало. И показателем нетерпения
устроителей является ревакцинация. Быстрей-быстрей. Пока не одумались многие


\iusr{Сергей Владимиров}
Это где произошло тоГ? И чем прививались люди?

\iusr{Сергей Владимиров}
Из серии..." я напишу о том, о чем только что услышала по телефону. "... Кто-то где-то чего-то кому-то звонил...((((

\begin{itemize} % {
\iusr{Виктория Иванова}
\textbf{Сергей Владимиров} читайте комментарии, Коль уж решили отметиться и поумничать. Там всё описано и написано сто раз, повторять для одарённых сто первый не буду.
И для тех, кто читает жопой, это мой клиент, которого только сегодня перевели из реанимации, с которым я каждый день на связи.
И я не живу в ФБ, чтобы моментально реагировать на *умные* замечания

\iusr{Сергей Владимиров}
\textbf{Виктория Иванова} 

Я не умничаю, в отличии от противников вакцинации - привился в сентябре
спутниковым лайтом и жив пока. А что касается "спецов" рижских, то попросил у
друзей-рижан про него разузнать! Отвечают: - А врач-то типичный антиваксер! Вот
потому его и удалили из фб. И лекарство - это пищевая добавка,он рекламирует.
Тот ещё типчик! На сайте "жалобы" (есть у нас такой на лат.языке, где жалуются
в любой области) так на этого Лицитиса много жалоб, он сдирает бешеные деньги,
халатно относится, выписывает всякие бады и добавки, а работает в нескольких
частных клиниках ( и в своей личной тоже), 50 евро визит, 5 минут приём, не
интересуется больным, грубит и т.д." Продолжать верить таким "спецам" ?!
Ну-ну.. \url{https://www.facebook.com/watch/?v=597741001428986}


\iusr{Виктория Иванова}
\textbf{Сергей Владимиров} 

угу, и министр здравоохранения тупень, судя по-всему, заявил, что Спутник Лайт
подходит только для ревакцинации...

В общем ваше тело-ваше дело, я посмотрю, чем эти вакцины в итоге закончатся,
как только я пойму, что они безопасны для моего здоровья, возможно,
вакцинируюсь, но только после тщательного обследования, а не как слепой
котёнок, подставляя плечо

\end{itemize} % }

\iusr{Андрей Бурлаченко}

\href{https://vlast.kz/avtory/45905-tromboz-i-vakcinacia.html}{%
Тромбоз и вакцинация - Что нужно знать перед получением векторных вакцин, vlast.kz, 19.07.2021%
}

\begin{itemize} % {
\iusr{Андрей Бурлаченко}

Тут важно понимать, что тромбообразование может происходить вследствие многих
болезней и факторов. В данном случае мы видим несколько признаков, которые
отличаются от других форм тромбообразования. Этот синдром, возникающий при
векторных вакцинах, уже получил свое название: вакцино-индуцированная иммунная
тромботическая тромбоцитопения (VITT). Ученые выяснили, что это иммунная
реакция, это состояние вызывается антителами, которые распознают тромбоцитарный
фактор-4 (PF4).

\iusr{Андрей Бурлаченко}

Важно понимать, что это происходит очень редко, примерно 2-4 раза на миллион. И
это связано только с векторными вакцинами, после РНК-вакцин таких осложнений не
наблюдается. Ученые еще ничего не знают о том, кто именно находится в зоне
риска. Первоначально, в феврале-марте, когда только начиналась история с
AstraZeneca, считалось, что этому состоянию больше подвержены молодые женщины,
но сейчас в этом нет уверенности.

Самый высокий уровень VITT был в Норвегии: 5 случаев на 130 тысяч человек,
привитых AstraZeneca. В целом по Европе было 222 случая на 15 млн. доз
AstraZeneca. По вакцине Janssen статистика выглядит так: 15 случаев на 8 млн.
человек. Недавно Аргентина опубликовала свои данные по прививке «Спутник V»: на
2,8 млн привитых было зарегистрировано два случая иммунной тромбоцитопении.

\iusr{Андрей Бурлаченко}

Сейчас часто задают вопрос, почему мы видим такие последствия только при
массовой вакцинации, не поторопились ли мы с регистрацией вакцин. Нет, объясню,
почему. Напомню, что в третьей фазе тестирования AstraZeneca участвовали 70
тысяч человек, по «Спутнику V» есть данные о 20 тысячах добровольцах. В таких
масштабах мы не можем увидеть очень редкие побочные действия подобного рода.
Для того, чтобы заметить их, правильно зарегистрировать, связать с вакциной,
нужна очень строгая система поствакцинального мониторинга и огромное число
привитых. Людей с такой реакцией очень мало для того, чтобы понять, как это
работает.

\iusr{Андрей Бурлаченко}

Симптомы VITT начинают проявляться не сразу после вакцинации. Как правило, это
происходит в промежутке в 5-14 дней после вакцинации векторной вакциной. Если
два-четыре дня после инъекции у вас головная боль, усталость, это нормально.
Насторожиться нужно, если это происходит через неделю после прививки.


\iusr{Андрей Бурлаченко}

Нужно понимать, что этот синдром хорошо поддается лечению, если его вовремя
диагностировать. Синдром напоминает аутоиммунную гепарин-индуцированную
тромбоцитопению, которая иногда возникает у пациентов, получающих лечение
гепарином.

\iusr{Андрей Бурлаченко}

Не нужно ничего измерять перед прививкой, потому что механизм VITT иммунный и
дело именно в антителах, которые вырабатываются после инъекции. Есть люди,
которые перед вакцинацией на всякий случай пропивают антикоагулянты,
пожалуйста, не нужно так делать. Такие препараты должен назначать врач, иначе,
при бездумном применении они могут привести к кровотечениям.

\iusr{Андрей Бурлаченко}

И наконец, самое важное — взвесить риски болезни и вакцинации. Если вы боитесь
тромбозов после прививки, то необходимо понимать, что риски тромбообразования
при самой коронавирусной инфекции в разы выше, чем редкие побочные действия при
получении векторных вакцин. Поэтому, когда вы вакцинируетесь, фокусируйтесь в
первую очередь на этом. Есть препараты, которые влияют на тромбообразование
больше, чем вакцины, в том числе это оральные контрацептивы и даже курение.

\end{itemize} % }


\end{itemize} % }
