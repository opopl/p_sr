%%beginhead 
 
%%file 15_01_2023.fb.fb_group.story_kiev_ua.1.khron_ki_neogoloshen
%%parent 15_01_2023
 
%%url https://www.facebook.com/groups/story.kiev.ua/posts/2116594891870634
 
%%author_id fb_group.story_kiev_ua,stepanov_farid
%%date 15_01_2023
 
%%tags 
%%title ХРОНІКИ НЕОГОЛОШЕНОЇ ВІЙНИ  День триста двадцять шостий. Мовчання
 
%%endhead 

\subsection{ХРОНІКИ НЕОГОЛОШЕНОЇ ВІЙНИ  День триста двадцять шостий. Мовчання}
\label{sec:15_01_2023.fb.fb_group.story_kiev_ua.1.khron_ki_neogoloshen}
 
\Purl{https://www.facebook.com/groups/story.kiev.ua/posts/2116594891870634}
\ifcmt
 author_begin
   author_id fb_group.story_kiev_ua,stepanov_farid
 author_end
\fi

ХРОНІКИ НЕОГОЛОШЕНОЇ ВІЙНИ 

День триста двадцять шостий. Мовчання. 

Коли у 1931 році Олександр Олесь писав \enquote{Пам'ятай} він і у страшному сні не
міг собі уявити, що і через більше ніж 90 років Європа буде мовчати.
Спостерігати, вичікувати, виказувати занепокоєння і стурбованність і...
мовчати. Бо є свої геополітичні інтереси, наступні вибори, ціна на газ для
місцевого електорату, криваві гроші в конверті від кремля за лобіювання і...
мовчання. Да багато чого є, що примушує Європу мовчати. 

Європа мовчала в 1917-18, в 1932-33 і 1938 мовчала, як мовчала і в 2014. І
14 січня 2023 мовчить. Занепокоєно та стурбовано мовчить. І буде мовчати. Бо
є ціна на газ, вибори, конверти... 

А нам треба пам'ятати, що наше майбутнє - виключно наша справа. І не дай нам
Боже, після Перемоги, забути всі жертви цієї війни. Забути і втратити
Перемогу. Як втратили 30 років незалежності. Не сформулювавши національної
ідеї, не збудувавши ефективних державних інституцій, не створивши потужної
армії. 30 років ми ділили спадок урср, пиляли літаки і ракети, ділили владу,
руйнували заводи і фабрики, продавали надра України. 

Залишу ці фото з Дніпра тут. Як нагадування для тих, хто і після Перемоги
буде мати бажання продавати і продаватися, ділити і пиляти. Вірю, що кров
невинних жертв пробудить у них совість. А якщо не пробудить, то є ми, які не
будуть мовчати. 

\obeycr
Коли Україна за право
життя
З катами боролась, жила
і вмирала,
І ждала, хотіла лише
співчуття,
Європа мовчала.
Коли Україна в нерівній
борьбі
Вся сходила кров'ю і
слізьми стікала
І дружної помочі ждала
собі,
Європа мовчала.
Коли Україна в залізнім
ярмі
Робила на пана і в
ранах орала,
Коли ворушились і скелі
німі,
Європа мовчала.
Коли Україна криваві
жнива
Зібравши для ката, сама
умирала
І з голоду навіть згубила
слова,
Європа мовчала.
Коли Україна життя прокляла
І ціла могилою стала,
Як сльози котились і в
демона зла,
Європа мовчала.
PS:
Україна в вогні.
Дніпро, 14 січня 2023. 
Фото: автор невідомий.
\restorecr
