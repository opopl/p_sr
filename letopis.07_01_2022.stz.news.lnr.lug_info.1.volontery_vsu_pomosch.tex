% vim: keymap=russian-jcukenwin
%%beginhead 
 
%%file 07_01_2022.stz.news.lnr.lug_info.1.volontery_vsu_pomosch
%%parent 07_01_2022
 
%%url https://lug-info.com/news/volontery-peredali-vsu-maskhalaty-kvadrokoptery-i-voennoe-imushestvo-narodnaya-miliciya
 
%%author_id 
%%date 
 
%%tags ukraina,vojna,donbass,volonter,vsu
%%title Волонтеры передали ВСУ маскхалаты, квадрокоптеры и военное имущество – Народная милиция
 
%%endhead 
\subsection{Волонтеры передали ВСУ маскхалаты, квадрокоптеры и военное имущество – Народная милиция}
\label{sec:07_01_2022.stz.news.lnr.lug_info.1.volontery_vsu_pomosch}

\Purl{https://lug-info.com/news/volontery-peredali-vsu-maskhalaty-kvadrokoptery-i-voennoe-imushestvo-narodnaya-miliciya}

Волонтеры передали бригадам ВСУ, действующим в зоне проведения \enquote{операции
объединенных сил}, маскхалаты, квадрокоптеры, тепловизоры и приборы ночного
видения. Об этом на брифинге сообщил официальный представитель Народной милиции
ЛНР Иван Филипоненко.

\enquote{Волонтерами было поставлено в (79-ю) бригаду 20 зимних легких маскировочных
костюмов Ghillie, 4 прибора ночного видения Armasight Spark Core и 3
квадрокоптера типа Mavic. Волонтеры также доставили в 30-ю бригаду 2
квадрокоптера типа Mavic и 5 тепловизоров Thermal-Eye X320}, - сказал он.

Филипоненко предположил, что в ближайшее время киевские силовики совершат
провокации.

\enquote{Мы оставляем за собой право на ответные действия для защиты жизни и здоровья
мирных граждан Республики}, - добавил представитель оборонного ведомства
Республики.

Власти Украины начали силовую операцию против Донбасса в апреле 2014 года.
Урегулирование конфликта базируется на Комплексе мер по выполнению Минских
соглашений, подписанном 12 февраля 2015 года в белорусской столице участниками
Контактной группы и согласованном с главами стран - участниц \enquote{нормандской
четверки} (Россия, Германия, Франция и Украина). Документ, в частности,
предусматривает прекращение огня и отвод тяжелых вооружений от линии
соприкосновения. 
