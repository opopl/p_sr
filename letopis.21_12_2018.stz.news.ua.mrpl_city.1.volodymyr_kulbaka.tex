% vim: keymap=russian-jcukenwin
%%beginhead 
 
%%file 21_12_2018.stz.news.ua.mrpl_city.1.volodymyr_kulbaka
%%parent 21_12_2018
 
%%url https://mrpl.city/blogs/view/volodimir-kulbaka-vidatnij-mariupolskij-arheolog-i-ulyublenij-vikladach
 
%%author_id demidko_olga.mariupol,news.ua.mrpl_city
%%date 
 
%%tags 
%%title Володимир Кульбака: видатний маріупольський археолог і улюблений викладач
 
%%endhead 
 
\subsection{Володимир Кульбака: видатний маріупольський археолог і улюблений викладач}
\label{sec:21_12_2018.stz.news.ua.mrpl_city.1.volodymyr_kulbaka}
 
\Purl{https://mrpl.city/blogs/view/volodimir-kulbaka-vidatnij-mariupolskij-arheolog-i-ulyublenij-vikladach}
\ifcmt
 author_begin
   author_id demidko_olga.mariupol,news.ua.mrpl_city
 author_end
\fi

\ii{21_12_2018.stz.news.ua.mrpl_city.1.volodymyr_kulbaka.pic.1}

Завжди усміхнений, веселий, талановитий, енергійний і позитивний викладач...
Таким Володимир Костянтинович Кульбака запам'ятався студентам Маріупольського
державного університету. Він не тільки прищеплював інтерес до таємничої науки
археології, але й розумів молодь, тому завжди з легкістю знаходив спільну мову
зі студентами. Особисто для мене (а мені пощастило бути студенткою В. К.
Кульбаки) пари Володимира Костянтиновича були найулюбленішими, час летів
непомітно. У студентські роки моя група не усвідомлювала, як нам пощастило,
адже у нас викладав не просто цікавий лектор, а й видатний вчений, який значно
розширив знання про історію Приазов'я. \emph{Він довів, що наш регіон, і зокрема
Маріуполь, має багату та унікальну давню історію, а не виник кілька століть
тому.} Тільки людина з великим серцем, безмежно віддана обраній справі буде
завжди в доброму гуморі, незважаючи на всі життєві труднощі. Адже Володимир
Костянтинович пов'язав своє життя з наукою, що не мала фінансової підтримки з
боку держави. Розраховувати залишалося на власні сили та невичерпний ентузіазм...

\textbf{Читайте також:} \emph{Фортеця Кальміус, Ольга Демідко, mrpl.city, 08.11.2018}%
\footnote{\url{https://mrpl.city/blogs/view/fortetsya-kalmius}} %
\footnote{Internet Archive: \url{https://archive.org/details/08_11_2018.olga_demidko.mrpl_city.fortecja_kalmius}}

Завдяки науковій діяльності Володимира Кульбаки в Маріупольському краєзнавчому
музеї з'явився сектор археології, на базі університету почала працювати
археологічна лабораторія, спеціальна аудиторія для проведення лекцій з
археологічної тематики. Йому вдалося підготувати ряд навчальних посібників,
створити демонстраційний ряд, зручний для студентів-істориків, які під час
літньої практики робили зі своїм викладачем унікальні відкриття.

Біографічних відомостей про Володимира Костянтиновича залишилося небагато.
Систематизувати всю наявну інформацію вдалося завдяки доповіді завідувача
відділу найдавнішої історії Маріупольського краєзнавчого музею \textbf{Лідії Іванівни
Кучугури}.

Народився В. К. Кульбака в м. Маріуполь (тоді – Жданов) 22 березня 1954 р.

У 1971 р. закінчив тут же курс середнього навчання в школі № 51.

У 1972 – 1974 рр. служив у лавах Радянської армії.

Сьогодні важко визначити, коли саме у молодого Володимира з'явився інтерес до
археології. Однак відомо, що освоювати ази польових досліджень, розширювати
свій кругозір, захоплюватися дослідженням пам'яток катакомбної культури епохи
бронзи він почав з 1976 року, коли працював лаборантом Сіверсько-Донецької
експедиції Інституту археології АН УРСР.

У 1980 р. В. К. Кульбака почав навчання в Донецькому державному університеті.

Перші самостійні розкопки він провів у 1984 році як керівник створеного ним
археологічного гуртка при тоді ще Орджонікідзевському районному Будинку
піонерів м. Жданова. Гурток переріс у Жданівську археологічну експедицію.
Рятувальні розкопки частково зруйнованого кургану \enquote{Виноградники} на західній
околиці м. Жданова фінансувалися Донецьким відділенням Товариства охорони
пам'яток.

Після завершення навчання в 1986 році Володимир Кульбака працював у складі
науково-дослідного сектора кафедри археології, історії стародавнього світу та
середньовіччя ДонНУ (1986–1987), де він виконував обов'язки інженера.

\textbf{Читайте також:} \emph{На глубине почти 300 метров в Донбассе зазвучала европейская классика}%
\footnote{На глубине почти 300 метров в Донбассе зазвучала европейская классика, Роман Катріч, mrpl.city, 20.12.2018, %
\par\url{https://mrpl.city/news/view/na-glubine-pochti-300-metrov-v-donbasse-zazvuchala-evropejskaya-klassika-foto}
}

Важливим етапом становлення для молодого спеціаліста як організатора
археологічних робіт було створення власної експедиції. У 1989–1991 роках В. К.
Кульбака очолив відділ археології Донбасу в Республіканській лабораторії
охоронних досліджень пам'яток археології та культури Українського Фонду
культури (м. Київ). А в 1991–1993 рр. очолив відділ археології Півдня України.

Протягом 1989–1993 років Маріупольська археологічна експедиція під його
керівництвом провела охоронні розкопки ряду курганів Приазовського краю.

У 1993 – 1995 рр. В. К. Кульбака завідував відділом археології Півдня України
кооперативу \enquote{Археолог}, що працював під егідою Інституту археології НАН
України. У 1993 р. він почав досліджувати порушений будівельними роботами
курган \enquote{Дід} на західній околиці Маріуполя. \emph{90-ті роки стали кризовими для
розвитку археології, далеко не всім у ті часи вдавалося без жодної підтримки та
фінансування залишатися відданими своїй справі. Але Володимир Костянтинович
зміг впоратися з цим випробуванням.}

\ii{21_12_2018.stz.news.ua.mrpl_city.1.volodymyr_kulbaka.pic.2}

З 1997 р. В. К. Кульбака починає працювати старшим викладачем кафедри
історичних дисциплін історичного факультету Маріупольського державного
гуманітарного коледжу, а пізніше - інституту й університету. Експедиція
Маріупольського гуманітарного інституту під його керівництвом завершила в 1998
році розкопки ґрунтового золотоординського могильника, розпочаті в 1987-му.
Тоді ж у мікрорайоні \enquote{Східний} були розкопані два кургани зрубного часу
(період бронзи).  А в 1999 році він продовжив розкопки кургану \enquote{Дід}.
Тут дослідив всього три поховання, одне з них - з моделюванням особи, а також
парне поховання епохи бронзи. Археолог очікував сенсацій від цієї потужної
стародавньої архітектурної споруди, розраховував знайти й скіфське золото...

\textbf{Читайте також:} \emph{На Донетчине нашли каменную бабу, которой жертвуют деньги}%
\footnote{На Донетчине нашли каменную бабу, которой жертвуют деньги, Анастасія Папуш, mrpl.city, 27.09.2018, \url{https://mrpl.city/news/view/na-donetchine-nashli-kamennuyu-babu-kotoroj-zhertvuyut-dengi-foto}}

Варто відзначити, що Володимир Костянтинович мав альтернативний погляд на
історію рідного краю, що було вкрай небажаним в СРСР. Працюючи в експедиціях з
1976 року і дослідивши 67 курганів, він отримав можливість видати свої
монографії тільки на початку 2000-х років. Всього було видано чотири книги, в
яких зібрані результати роботи над 274 похованнями з епохи енеоліту до пізнього
середньовіччя та одним могильником золотоординського часу.

Однією з унікальних знахідок Володимира Костянтиновича Кульбаки є дерев'яні
чотириколісні вози з суцільними дерев'яними дисковими колесами. Вони були
виявлені в кургані бронзової доби на околиці нашого міста – біля дороги, що
веде в селище Володарське. \emph{Ці знахідки є не тільки одним з найдавніших видів
колісного транспорту нашого регіону, але й одними з найдавніших возів у світі.}

\ii{21_12_2018.stz.news.ua.mrpl_city.1.volodymyr_kulbaka.pic.3}

Видатний археолог зробив ще одне відкриття, яке залишилося просто непоміченим
нашими земляками. Під час розкопок під Артемівськом він знайшов, на перший
погляд, звичайний горщик зрубної культури, яка відноситься до бронзового віку,
але при подальшому дослідженні виявилося, що цей горщик містить на собі \emph{напис
на протосанскритській мові.}

Впродовж 1984–2000 рр. В. К. Кульбака досліджував кургани з похованнями
енеоліту, епохи бронзи, раннього залізного віку, кочівників і ґрунтовий
могильник періоду Золотої Орди. А здобуті яскраві матеріали лягли в основу його
розробок з індоєвропейської тематики. Вчений неодноразово робив спроби
організувати регулярні розкопки в Приазов'ї й водночас – повноцінну практику
для майбутніх істориків-студентів МДУ. Так у 2000 р. вдалося дослідити другу
ділянку золотоординського могильника на східній околиці Маріуполя. Але
відсутність фінансування зривало ці плани.

\textbf{Читайте також:} \emph{Популярная украинская группа осуществила мечту 10-летней мариупольчанки}%
\footnote{Популярная украинская группа осуществила мечту 10-летней мариупольчанки, Олена Онєгіна, mrpl.city, 19.12.2018, \par%
\url{https://mrpl.city/news/view/populyarnaya-ukrainskaya-gruppa-osushhestvila-mechtu-10-letnej-mariupolchanki-video}%
}

У 2009 році Володимир Костянтинович передчасно пішов з життя. Це стало
справжнім шоком не тільки для його сім'ї, але й для його колег, студентів. Всі,
хто знав цю Людину особисто, не міг повірити, що більше не почує цікавих лекцій
улюбленого викладача, що не буде з ним душевних розмов... Володимир Костянтинович
вже давно став душею історичного факультету. Учень і послідовник вченого,
старший викладач кафедри історичних дисциплін МДУ В'ячеслав Олегович Забавін,
на щастя, продовжив справу Володимира Костянтиновича. Він наголосив, що \emph{\enquote{важко
переоцінити внесок Володимира Костянтиновича у світову науку. Захопленим своєю
справою, талановитим, відкритим і веселим ми запам'ятали його назавжди}}...
