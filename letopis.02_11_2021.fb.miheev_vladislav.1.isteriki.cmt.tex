% vim: keymap=russian-jcukenwin
%%beginhead 
 
%%file 02_11_2021.fb.miheev_vladislav.1.isteriki.cmt
%%parent 02_11_2021.fb.miheev_vladislav.1.isteriki
 
%%url 
 
%%author_id 
%%date 
 
%%tags 
%%title 
 
%%endhead 
\subsubsection{Коментарі}

\begin{itemize} % {
\iusr{Вячеслав Бутко}

Если заговор против Украины существует, то в нем участвуют все элиты/контрэлиты
и критически подавляющее большинство населения страны.


\iusr{Анатолій Скринник}

"коммуницировать-то особо и не с кем - общества, как реального, взрослого,
ценностно ориентированого и самоосознангого субьекта в Украине НЕТ" Когнитивный
диссонанс — это несоответствие между тем, во что верят люди, и тем, как они
себя ведут и ...люди испытывают дискомфорт, когда они придерживаются
противоречивых убеждений. " Если бы в мире был спрос на коллективную панику -
мы бы стали мировым монополистом." @igg{fbicon.hands.applause.yellow}{repeat=3}  @igg{fbicon.thumb.up.yellow}  @igg{fbicon.cry} 

\iusr{Максим Ряпулов}

щодо проваленої комунікації, не зовсім згоден.

в 19му просто прийшла нова цікава пересічним громадянам комунікація, коли з
ними спілкувався "капітан країнаУ".

бачив, як на одному окрузі в області розгромно програв кандидат, який 2
каденції поспіль вигравав вибори без зайвих зусиль, і доволі адекватно працював
на електорат між виборами.

і це був зовсім не представник ЄС.

просто проти кіношного супергероя звичайна людина, навіть з ідеальною
комунікацією, стопудово програє.

апд. знов таки, про фізиків і ліриків: некоректно, з позиції фізики,
стверджувати, що одна комунікація провальна, а інша - ідеальна. коректне
твердження - одна була ефективнішою за іншу. точно так, як не бува холодного
ітеплого. бува одне з вищою температурою, ніж інше.

\iusr{Никола Ра Дчук}
Еще одна офигительная история.  @igg{fbicon.thinking.face}  @igg{fbicon.beaming.face.smiling.eyes} 

\iusr{Евгений Вагнер}
Уважаемый автор, добавь букву "р" в фразу "всё попало", а то смысл гуляет)))

\begin{itemize} % {
\iusr{Йосеф Блудофф}
\textbf{Евгений Вагнер} или убрать букву "п"? "всё опало"  @igg{fbicon.face.smiling.eyes.smiling} 

\iusr{Евгений Вагнер}
\textbf{Йосеф Блудофф} будет пол-шестого..))

\iusr{Йосеф Блудофф}
\textbf{Евгений Вагнер} а ты предпологаешь, что там всегда полдень?

\iusr{Евгений Вагнер}
\textbf{Йосеф Блудофф} надо быть оптимистом !!!
\end{itemize} % }

\iusr{Александр Хургин}

Грустно, Влад. Очень. Знаешь, я, читая, вдруг почему-то вспомнил. Когда
Рабинович купил ТД и тут же перепродал его Теленеделе, нам из последней пришла
инструкция. Убей, не помню, что она требовала, но написана была "образно", и
запомнил я из неё всего одну фразу. Зато на всю жизнь. Фраза такая: "Труп
умирающего делает рефлекторные хватательные движения". Что там было дальше, не
помню.


\iusr{виталий теплов}
Михеич, вот и выдал ты врагам, как Мальчиш-Плохиш, нашу главную военную тайну)))

\iusr{Евгений Вагнер}

Зная Михеева-человека, иногда закрадывается мысль о подражателе
Михееве-писателе)))). Влад в обычной жизни так много букв не произносит... а
тут показывает.. красивое...)))

\iusr{Yartsev Anatoliy}

Ну примеров, когда правители пугали свои народы - предостаточно в истории
человечества. Зря вы так именно про украинцев тут расписались в таком виде. За
последнее столетие - море таких примеров привести можно коллективного
помешательства. Фашизм, сталинизм, франкизм, маккартизм...

\begin{itemize} % {
\iusr{Владислав Михеев}
\textbf{Yartsev Anatoliy} 

примеров много, это бесспорно. Только вряд ли кто-то назовет Сталина, Гитлера и
Франко людьми без стратегии и цели. Они реализовали проекты, были реальными
субъектам истории. Где у нас это все? Людей зомбировали, предьявляя им образ
будущего. Какой образ будущего, транслируя истерику в массы, предъявляет
украинская власть?

Кроме того, даже Байден, помниться, в эпоху Обамы выступая в ВР, говорил :
завязывайте создавать плохую копию роспропаганды, начинайте, помимо манипуляций
общественным мнением, делать что-то реальное для развития ...

Найдите эту речь Байдена, если получится. Она прекрасна. ) Но , как водится,
население колоний белым господам подчиняться еще как-то может, но учиться у них
не в состоянии.


\iusr{Yartsev Anatoliy}
\textbf{Владислав Михеев} 

тут вы разумеется правы. Наши клоуны-компрадоры всех мастей за тридцать лет
научились только воровать в основном, сбегать от ответственности, ну а образ
как сырьевой колонии, увы, уже неоспоримый факт. Но снова, Украина в этом не
уникальна. Море стран мира оказались в такой роли.

\iusr{German Viktorov}
\textbf{Владислав Михеев} 

"Людей зомбировали, предьявляя им образ будущего." Эти люди зомбируют себя
сами, в первую очередь, своим постоянным лицемерием. В ловушку собственного
лицемерия они и попали, захотев себе свою страну, якобы для того, чтобы
самостоятельно заниматься собственной жизнедеятельностью, но это была просто
очередная, привычная тут ложь, а обратного пути нет, вот и приходится теперь
жить только в манипуляциях. Ничего ведь другого нет, как и желания иметь
что-либо другое...

\iusr{Владислав Михеев}
\textbf{Yartsev Anatoliy} есть много таких стран - старых колоний, неоколоний... Но Сомали не позиционирует себя как "це Европа" и т.д и т. п.

\iusr{Yartsev Anatoliy}
\textbf{Владислав Михеев} 

не, упражняться в терминах и определениях можно долгое и нудно. И даже Сомали
может вдруг найти свои корни в Европе, для своего населения, каждый день
проговаривая вслух везде об этом. Не в том речь. Тут вот именно у вас упор на
выделение украинцев, неспособных что-то сотворить стоящее за эти тридцать лет.
Не факт, что они не могут. Вот корейцы, арабы ещё 70 лет назад нищими
кочевниками и рыбаками жили, а сейчас? Всякое возможно. И украинцы не самый
плохой пример, в общем то....

\iusr{Владислав Михеев}
\textbf{Yartsev Anatoliy} 

я с вами согласен! Более того, неоднократно утверждал, что проект развития
может быть успешен на базе разных культурных традиций и менталитетов! Хотя со
мной многие и не соглашались... Украинцы, конечно же, не обречены на неуспех
государственности своей религией , историей и культурой. Но на сейчас в стране
просто нет тех условий для роста, которые были в Сингапуре или Эмиратах... Но
об этом нужен большой разговор... Сейчас не я - статистика! - говорит что по
уровню бардака и бедности мы в Европе такие одни. Это факт. Статистически мы
Тунис, а не европейская страна... Если кратко о причинах - нет волевого
субьекта изменений. В Сингапуре, Корее , Польше, ОАЭ, Финляндии он был... Можно
сказать , что успешные социальные, экономические управленческие технологии хотя
и контекстны, но вместе с тем и универсальны. В Украине понятно что делать и
как делать. Но главный вопрос о субьекте не решен: кто это будет делать?

Но это опять-таки длинный разговор...

\end{itemize} % }

\iusr{Татьяна Букланова}

Изложено разумно, со вкусом... но если кратко: нация ущербна.( И никакие
отдельные здравомыслящие или прославленные индивидуумы, увы, не меняют общей
печальной картины.

\iusr{Полина Дьяченко}

А как еще можно воздействовать на менталитет хатаскрайников? Только эмоциями

\end{itemize} % }
