% vim: keymap=russian-jcukenwin
%%beginhead 
 
%%file 13_01_2018.stz.news.ua.mrpl_city.1.staryj_novyj_rik
%%parent 13_01_2018
 
%%url https://mrpl.city/blogs/view/starij-novij-rik-osoblivosti-svyatkuvannya-traditsii-i-prikmeti
 
%%author_id demidko_olga.mariupol,news.ua.mrpl_city
%%date 
 
%%tags 
%%title Старий Новий Рік: особливості святкування, традиції і прикмети
 
%%endhead 
 
\subsection{Старий Новий Рік: особливості святкування, традиції і прикмети}
\label{sec:13_01_2018.stz.news.ua.mrpl_city.1.staryj_novyj_rik}
 
\Purl{https://mrpl.city/blogs/view/starij-novij-rik-osoblivosti-svyatkuvannya-traditsii-i-prikmeti}
\ifcmt
 author_begin
   author_id demidko_olga.mariupol,news.ua.mrpl_city
 author_end
\fi

Якщо в ніч з 31 на 1 січня вам не вистачило веселощів, ви можете продовжити
святкування, адже в ніч з 13 на 14 січня в Україні відзначають Старий Новий
рік. Незважаючи на те, що це свято не відноситься до офіційних, багато хто з
нас любить і шанує цей день.

Що саме варто знати маріупольцям про Старий Новий рік?!

Ну, по-перше, те, що це свято з'явилося в результаті зміни літочислення, тому
воно відзначається переважно в тих країнах, де православна церква продовжує
користуватися юліанським календарем. До ХХ століття юліанський календар
Російської імперії (до складу якої входила територія України) на 13 діб
відставав від Європи, яка давно перейшла на григоріанський календар. Щоб
скоротити цей розрив в 1918 році декретом Ради Народних Комісарів був
здійснений перехід на григоріанський календар – новий стиль, а 14 січня – день
Святого Василя – став Старим Новим роком. З тих пір у жителів України з'явилася
можливість влаштовувати собі два Нових роки – з 31 грудня на 1 січня і з 13 на
14 січня.

По-друге, існує безліч цікавих прикмет, які відомі далеко не всім, але про які
слід дізнатися. Так, вважається, що 13 і 14 січня не можна говорити слово
тринадцять, інакше вас цілий рік будуть супроводжувати невдачі. На свято не
можна рахувати копійки, інакше потім будете гірко плакати. Також існує повір'я,
що дитина, народжена 14 січня, стане дуже багатою. Крім того, не варто виносити
сміття 14 січня, оскільки ви винесете своє щастя. Якщо приготовлена святкова
каша вийде пишною, красивою і смачною, значить треба чекати гарного року. Якщо
ж горщик в печі тріскався або каша виходила несмачною – треба чекати поганого
(худого) року.

По-третє, з давніх-давен традиційно у дні Старого Нового року українці також
відзначали день Маланки (13 січня) і день Святого Василя (14 січня). Ці свята
завжди супроводжувалися народними гуляннями, переодяганнями і ворожінням.
Головні обряди святкування – щедрування і водіння кози Маланки. З вечора і до
самої ночі компанії щедрувальників ходять по домівках, співають обрядові
пісні-щедрівки, танцюють і жартують. Компанію супроводжує Коза Маланка, в яку
найчастіше переодягаються молоді хлопці.

Вранці 14 січня в Україні прийнято засівати. Згідно з цим ритуалом на Старий
Новий рік хлопчики і молоді хлопці наповнюють свої рукавиці зерном і засівають
спочатку в своєму будинку, потім в будинках хресних батьків, родичів і сусідів.
При цьому засівальники бажають щастя, достатку і врожаю в новому році. У деяких
регіонах України вранці 14 січня спалюють \enquote{дідухів} – заздалегідь підготовлені
снопи, які на Щедрий вечір стояли в будинку на покуті. Вважається, що цей обряд
допомагає позбутися від нечисті, бід і неприємностей.

Однак одним лише спалюванням соломи цей обряд на Старий Новий рік не
закінчувався: щоб очиститися від усіх бід і неприємностей, набраних за минулий
рік, люди стрибали через багаття.

Серед давніх українських звичаїв на Старий Новий рік варто виділити і сватання.
Тільки 13 січня ті молоді хлопці, які раніше отримали від коханої дівчини
\enquote{гарбуза}, могли повторити спробу і знову відправити до будинку дівчини сватів.
Свататися на Маланку можна і вперше, адже вважається, що якщо погодитися на
шлюб в Щедрий вечір, то і життя молодої пари буде щедрим і радісним.

На Старий Новий рік за столом прийнято збиратися всією сім'єю. Родичі просять
один у одного вибачення за образи і загадують бажання, які обов'язково повинні
здійснитися в новому році.

В Україні на Старий Новий рік щедро накривають стіл. На ньому обов'язково
повинна бути кутя. Святковий стіл дуже схожий на різдвяний, однак страви тут
вже не пісні. На стіл ставлять ковбаси, м'ясо, вареники, пироги, млинці та
багато іншого.

Водночас ніч з 13 на 14 січня вважалася магічною, тому невід'єм\hyp{}ною частиною
святкування були також ворожіння на Старий Новий рік.

Цікаво, що Старий Новий рік є доволі популярним святом і у різних народів має
свої особливості. Так, в країнах колишньої Югославії в ніч з 13 на 14 січня
обов'язково збираються за сімейним столом. Причому називається свято
по-різному: в Сербії відзначають \enquote{Мале Різдво}, а в Чорногорії \enquote{Права нова
година}, що перекладається як \enquote{Правильний новий рік}. У Греції широко прийнято
відзначати Старий Новий рік, оскільки це день пам'яті святителя Василія
Великого, особливо шанованого греками.

Не відмовляються від свята і в \enquote{григоріанській} Європі: для жителів деяких
районів Швеції і Румунії ніч на 14 січня вважається \enquote{казковою}. А в деяких
громадах Уельсу ця ніч відзначається як свято \enquote{добросусідства і відкритих
дверей}.

Як це не дивно, але Старий Новий рік святкують навіть у ряді африканських
країн, де живуть за своїм берберським календарем, який має багато спільного з
юліанським. Тому в Алжирі, Марокко, Тунісі Старий Новий рік відзначають 12
січня.
