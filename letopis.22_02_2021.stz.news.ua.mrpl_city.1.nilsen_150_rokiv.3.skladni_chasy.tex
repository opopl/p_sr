% vim: keymap=russian-jcukenwin
%%beginhead 
 
%%file 22_02_2021.stz.news.ua.mrpl_city.1.nilsen_150_rokiv.3.skladni_chasy
%%parent 22_02_2021.stz.news.ua.mrpl_city.1.nilsen_150_rokiv
 
%%url 
 
%%author_id 
%%date 
 
%%tags 
%%title 
 
%%endhead 

\subsubsection{Складні часи}

На жаль, за радянської влади ставлення до архітектора погіршилося. Його вважали
успішним представником царського періоду. З 1935 року почалася чистка кадрів,
які мали дворянське чи неробітниче походження. Архітектор був позбавлений права
працювати архітектором-практиком і працював у технічному відділі нового заводу
\enquote{Азовсталь}. В 1936 році також була зруйнована церква Костянтина та Єлени в
місті, створена і вибудувана архітектором.

Відомий краєзнавець Аркадій Дмитрович Проценко у своїй статті наводить спогади
старожила Маріуполя \emph{\textbf{Р. Д. Чарфаса}}, який згадував, що Віктор Олександрович був
людиною надзвичайно працелюбною і талановитою. 

\begin{quote}
\em\enquote{У тридцяті роки, працюючи разом
з ним проєктно-технічному відділі будівництва \enquote{Азовсталі}, – ми навіть не
здогадувалися, що цей літній, але повний енергії інженер колись був головним
архітектором Маріуполя. Енергію свою він ніколи не економив, а всю віддавав
улюбленій справі. І, здається, її нітрохи не зменшувалося. Так, незважаючи на
свій похилий вік – йому було вже 72 роки, в 1943 році він активно включився в
роботу з відновлення зруйнованого війною міста}.
\end{quote}

Однак похилий вік не міг не позначитися на загальному стані. Це і змусило В. О.
Нільсена вийти на пенсію. У 1948 році Віктор Олександрович поїхав до родичів
(за однією з версій до доньки) в місто Душанбе, а наприкінці 1949 року, у віці
78 років, його не стало.
