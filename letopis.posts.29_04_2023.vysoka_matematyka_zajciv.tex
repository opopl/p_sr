% vim: keymap=russian-jcukenwin
%%beginhead 
 
%%file posts.29_04_2023.vysoka_matematyka_zajciv
%%parent posts
 
%%url 
 
%%author_id 
%%date 
 
%%tags 
%%title 
 
%%endhead 

Висока Математика Зайців та Писанок

До речі...

От щодо системи #TeXLaTeX

Щось засвербіло всередині і вирішив написати.

#ЩосьЗасвербіло

Як воно виглядає всередині... Маєте текстовий файл + окремо зображення. Є
компілятор, який бере оцей файл + зображення, і зшиває все це разом. Потім на
виході буде файл PDF (який можна зконвертувати в зображення або ж викласти
онлайн, наприклад як ось оцей збірник з культури Маріуполя
https://acrobat.adobe.com/.../urn:aaid:sc:EU:4ecce413...), або ж файл HTML
(коли Ви переглядаєте сторінку будь-якого сайту, ви бачите HTML - HyperText
Markup Language - мова розмітки, яку винайшов свого часу Тім Бернерс Лі в ЦЕРН
- Європейському Центрі Ядерних Досліджень (CERN = Conseil Européen pour la
Recherche Nucléaire) в Женеві, Швейцарія - з неї власне і розпочалась ера WWW -
World Wide Web - Всесвітня Паутина - тобто Інтернет такий, яким ми його бачиво
в візуальному сенсі. Хоча сам по собі Інтернет як мережа бере свій початок ще з
70 років в США, коли військові там почали замислюватись над тим, як можна було
б об'єднати всі свої компьютери в одну мережу. До речі, у нас в Україні теж був
подібний проєкт, проєкт Мережі, піонером напрямку цього був академік Віктор
Глушков, - яка б об'єднувала б СРСР електронним чином, він називався
електронний соціалізм але... йому не було суджено бути здійсненим. Може й
добре, що не був здійснений, бо СРСР був тоталітарною державою... а це означало
можливо б в кінці тотальний контроль над населенням, як от зараз в Китаї... і
було б дуже сумно так жити, я так думаю, бо тоді б не було б ні незалежної
України, ні Майдану, ні Руслани з її Дикими Танцями та піснею Знаю Я, ні
перемог Олександра Усіка або Ольги Харлан або прекрасних пісень Пономарьова або
ж шахових перемог також Пономарьова (Руслан, Пономарьов, український шаховий
чемпіон світу), ні ні всього того, що є зараз, напевне... всі б ходили строєм і
маршами під отой совітський гімн союз нерушімий... і кожного ранку школяри б
пов'язували собі оті червоні галстучки... і за всім пильно б слідкувало б
КДБ... жах, короче  ). 

До речі, od там вчора проводила екскурсію на вулиці Ярославів Вал, і там свого
часу жив Віктор Глушков, на цій вулиці. Також, на Байковому кладовищі на
центральній алеї він похований. І ще... там неподалеку є кав'ярня по вул
Золотоворітська 2А. Там я був здається ще зимою, зайшов. Присів, бачу на
підвіконні книжки стоять! Одна з тих книжок була книжка Наші Уляни Скицької,
дуже цікава, рекомендую! Там короче все про українців, в найширшому розумінні,
тобто, також, з кола емігрантів, які можливо самі по собі не особливо
замислювались про те, що... якась їхня бабуся колись жила десь в далекому
Золочеві під Тернополем, але... які, тим не менш, змінили світ докорінним чином
(як-от наприклад, винахідник мобільного телефону Мартін Купер, цього року його
винаходу вже 50 років, бо ж перший дзвінок з того самого телефону він зробив ще
в 1973 році... )

А щодо математики - творцем системи #TeXLaTeX є американський математик Дональд
Кнут, який, крім створення системи Tex/LaTeX (зараз де-факто стандарт в
науковому світі для запису публікацій та книжок із формулами, але... як показує
мій приклад... можна записувати також Зайців за допомогою цієї системи... )
