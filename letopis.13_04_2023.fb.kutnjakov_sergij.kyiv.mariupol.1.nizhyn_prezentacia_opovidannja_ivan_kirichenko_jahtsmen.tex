%%beginhead 
 
%%file 13_04_2023.fb.kutnjakov_sergij.kyiv.mariupol.1.nizhyn_prezentacia_opovidannja_ivan_kirichenko_jahtsmen
%%parent 13_04_2023
 
%%url https://www.facebook.com/permalink.php?story_fbid=pfbid0zbKnxFNnVtcsnwc53TBZq6H7SQHTXsPvLhJwDNiMuG8sF9zvdz85uvpq42f1ti9Rl&id=100000943996304
 
%%author_id kutnjakov_sergij.kyiv.mariupol
%%date 13_04_2023
 
%%tags jahta,more,planeta,puteshestvie,literatura,kultura,prezentacia,nizhyn.misto.ua
%%title Презентація у Ніжині - оповідання "Людина–вітрила, людина–вітер" - про маріупольського яхтсмена Івана Кириченка
 
%%endhead 

\subsection{Презентація у Ніжині - оповідання \enquote{Людина–вітрила, людина–вітер} - про маріупольського яхтсмена Івана Кириченка}
\label{sec:13_04_2023.fb.kutnjakov_sergij.kyiv.mariupol.1.nizhyn_prezentacia_opovidannja_ivan_kirichenko_jahtsmen}

\Purl{https://www.facebook.com/permalink.php?story_fbid=pfbid0zbKnxFNnVtcsnwc53TBZq6H7SQHTXsPvLhJwDNiMuG8sF9zvdz85uvpq42f1ti9Rl&id=100000943996304}
\ifcmt
 author_begin
   author_id kutnjakov_sergij.kyiv.mariupol
 author_end
\fi

Презентація у Ніжині в музеї українського мореплавця~навколосвітника Юрія
Лисянського \enquote{Морського сборнику}, де надруковане моє оповідання
\enquote{Людина–вітрила, людина–вітер}. Це про мого друга, маріупольського яхтсмена
Івана Кириченка, який вперше в Україні і на пост-радянському просторі пройшов
наодинці Середземне море і Атлантику, а потім і всю планету.

Презентували книгу письменник-мариніст, організатор
\href{https://www.facebook.com/matelotUA}{Мателот - Конкурс української
морської прози ім Лисянського}
\href{https://www.facebook.com/profile.php?id=100001205690197}{Антон Санченко}
і член журі конкурсу, капітан другого рангу Сергій Шумський. Цікаві розповіді,
безліч подарунків, жваве знайомство з чудовими експозиціями морського музею на
Чернігівщині. Щирий респект господарям і організаторам заходу за гостинність.
Кілька слів на презентації сказав і я. 

Свої враження від Ніжина і туристичної привабливості його історичного центру
викладу з фото пізніше..
