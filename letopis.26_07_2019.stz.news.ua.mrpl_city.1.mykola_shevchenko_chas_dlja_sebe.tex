% vim: keymap=russian-jcukenwin
%%beginhead 
 
%%file 26_07_2019.stz.news.ua.mrpl_city.1.mykola_shevchenko_chas_dlja_sebe
%%parent 26_07_2019
 
%%url https://mrpl.city/blogs/view/mariupolets-mikola-shevchenko-znahodte-chas-dlya-sebe
 
%%author_id demidko_olga.mariupol,news.ua.mrpl_city
%%date 
 
%%tags 
%%title Маріуполець Микола Шевченко: "Знаходьте час для себе!"
 
%%endhead 
 
\subsection{Маріуполець Микола Шевченко: \enquote{Знаходьте час для себе!}}
\label{sec:26_07_2019.stz.news.ua.mrpl_city.1.mykola_shevchenko_chas_dlja_sebe}
 
\Purl{https://mrpl.city/blogs/view/mariupolets-mikola-shevchenko-znahodte-chas-dlya-sebe}
\ifcmt
 author_begin
   author_id demidko_olga.mariupol,news.ua.mrpl_city
 author_end
\fi

\ii{26_07_2019.stz.news.ua.mrpl_city.1.mykola_shevchenko_chas_dlja_sebe.pic.1}

Мабуть, у Маріуполі складно знайти більш неординарну, суперечливу і яскраву
особистість, ніж \textbf{Микола Шевченко}. Таке насичене життя як у Миколи може
викликати у пересічного мешканця Землі багато питань. Як він все встигає?
Звідки черпає натхнення? Як в одній людині поєдналося стільки талантів? Адже
він і громадський діяч, і художник, і поет, і письменник, і філософ, і
археолог, і краєзнавець, і екскурсовод, і актор... Цей список можна продовжувати
довго. Головне, що він всю свою творчу і шалену енергію зміг спрямувати в
конструктивне русло. Активний і небайдужий маріуполець став координатором
великої кількості загальноміських проектів, зміг посприяти
суспільно-культурному розвитку міста і активізації маріупольської молоді.

\ii{26_07_2019.stz.news.ua.mrpl_city.1.mykola_shevchenko_chas_dlja_sebe.pic.2}

Микола народився в Маріуполі в українській родині. Батьки завжди опікувалися
хлопцем. Завдяки спільним подорожам розширювали його уявлення про існуючий
світ. Має дві вищі освіти, які отримав в ПДТУ на Метфаку, та ДонНУ на Кафедрі
Міжнародної Економіки. Він встиг попрацювати і електриком, і викладачем
інформатики. Водночас хлопець отримав і художню освіту, закінчивши Художню
школу в Маріуполі. Найбільше Миколу надихають люди і природа. Однак, якщо
навколо нічого не відбувається і хочеться змін, він може і сам себе надихнути і
створити щось незвичне і унікальне. Вважає найвищою цінністю свободу, завдяки
якій людина може реалізувати власний потенціал повною мірою. Кожна подія в його
житті – це те, що його наповнює.

\textbf{Читайте також:} \emph{\#МАГНИТДОБРА: мариупольцы участвуют в благотворительной лотерее в помощь Виктории Луцык}%
\footnote{\#МАГНИТДОБРА: мариупольцы участвуют в благотворительной лотерее в помощь Виктории Луцык, Анастасія Папуш, mrpl.city, 25.07.2019, \par%
\url{https://mrpl.city/news/view/magnitdobra-mariupoltsy-uchastvuyut-v-blagotvoritelnoj-loteree-v-pomoshh-viktorii-lutsyk-foto}}

Громадська діяльність Миколи почалася з дитячого садочка. І це зовсім не жарти.
Він завжди відчував свою відповідальність за те, що відбувається навколо нього,
вважав важливим приносити користь суспільству. Громадський діяч став
організатором багатьох маріупольських проектів, проводив літературні вечори.
Взяв участь як екскурсовод і актор у загальноміському проекті \textbf{\enquote{Маріуполь – це
Україна}}. Став одним з координаторів фестивалю \textbf{\enquote{З країни в Україну}}, очоливши
театральну локацію. Взяв участь як організатор і художник у фестивалі графіті
\textbf{\enquote{6 континентів}}. Діяльність маріупольця охоплює багато різних напрямів, тому у
Миколи є багато однодумців, при цьому в кожному напрямі вже давно є \enquote{свої
люди}. Водночас його ентузіазм і натхненна діяльність змогли активізувати
маріупольську молодь, яка довгий час залишалася пасивною.

Разом з тим наш герой є членом місцевого осередку краєзнавства. Часто проводить
екскурсії для школярів і студентів. Є співавтором відкриттів \textbf{Приазовської
археологічної експедиції} під керівництвом В. М. Горбова. Микола намагався
систематизувати поховання на найстарішому кладовищі Маріуполя. Також
маріуполець пише вірші і прозу. Деякі його вірші можна знайти у збірці \textbf{\enquote{По
живому. Околовоенные дневники}}. Сатиричні вірші поета можна знайти і в
інтернеті. Пише він під псевдонімом \textbf{ПанадолЪ РубильникЪ}.

\ii{26_07_2019.stz.news.ua.mrpl_city.1.mykola_shevchenko_chas_dlja_sebe.pic.3}

Микола встигає реалізувати і свій художній потенціал. Зокрема, він неодноразово
проводив персональні виставки як в Маріуполі (Центр сучасного мистецтва ім.
Куїнджі), так і інших містах України. Він є автором панно на Центральній
міській дитячій бібліотеці, присвяченому ідеї загального прагнення людей до
знань, які можуть всіх об'єднати. Батьки підтримують єдиного сина в усьому,
розділяють його позицію і розуміють спосіб життя Миколи.

\ii{26_07_2019.stz.news.ua.mrpl_city.1.mykola_shevchenko_chas_dlja_sebe.pic.4_7}

Маріуполь громадський діяч любить з дитинства, при чому не шкодує ані сил, ані
часу задля його розвитку. Має багато улюблених місць, зокрема Приморський парк.

\ii{insert.read_also.demidko.arhangelska}
\ii{26_07_2019.stz.news.ua.mrpl_city.1.mykola_shevchenko_chas_dlja_sebe.pic.8}

\textbf{Улюблена книга:} найбільше полюбляє твори О. Дюма, братів Стругацьких, К. Буличова.

\textbf{Улюблений фільм:} \enquote{Пропала грамота} 1972 року кінорежисера Бориса Івченка.

\textbf{Хобі:} колекціонує марки, значки, мінерали.

\textbf{Порада маріупольцям:} 

\begin{quote}
\em\enquote{Знаходьте для себе час, для свого власного розвитку. Якщо ви знайдете
цей час, то ви станете кращими для себе}. 
\end{quote}
