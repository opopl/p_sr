% vim: keymap=russian-jcukenwin
%%beginhead 
 
%%file 30_11_2021.fb.fb_group.story_kiev_ua.1.chaes_piknik_puscha_vodica
%%parent 30_11_2021
 
%%url https://www.facebook.com/groups/story.kiev.ua/posts/1808935922636534
 
%%author_id fb_group.story_kiev_ua,fedjko_vladimir.kiev
%%date 
 
%%tags 1986,avaria_na_chaes,chaes,chernobyl,kiev,kiev.puscha_vodica,piknik
%%title 26 квітня 1986-го... Пікнік в Пуща-водицькому лісі
 
%%endhead 
 
\subsection{26 квітня 1986-го... Пікнік в Пуща-водицькому лісі}
\label{sec:30_11_2021.fb.fb_group.story_kiev_ua.1.chaes_piknik_puscha_vodica}
 
\Purl{https://www.facebook.com/groups/story.kiev.ua/posts/1808935922636534}
\ifcmt
 author_begin
   author_id fb_group.story_kiev_ua,fedjko_vladimir.kiev
 author_end
\fi

26 квітня 1986-го... Пікнік в Пуща-водицькому лісі.

«...Все бегут, бегут, бегут, бегут, бегут, бегут, бегут, бегут, бегут, бегут, бегут, 
а он горит...»

(З популярноі пісні...)

***

Ми живемо в чудовому районі Києва – на площі Шевченка! Перейшовши через
Вишгородську, навпроти Інституту геронтології, можна піти по Сошенка до
старовинного парку «Кинь Грусть» (Дача Куліженка). А можна пройти пару сотень
метрів вздовж траси на Вишгород і повернути у Пуща-Водицький ліс. Гуляй і
досхочу насолоджуйся красою дерев, чистим лісовим повітрям та співом пташок! З
літа 1974-го, коли я побрався з Надійкою, це були улюблені місця наших
прогулянок і пікніків вихідного дня з друзями. Тут ми майже щоденно гуляли в
часи вагітності дружини та після народження доньки Юлії. 

25 квітня 1986 року, п’ятниця

Надійчині батьки планують на вихідні дні поїхати на дачу – трохи прибратися в
будиночку після зимової відсутності. Пропонують поїхати і нам, але у нас інші
плани.  

Зранку ми хочемо провести сімейний пікнік в лісі, а в другій половині дня
поїхати в місто та посмакувати морозивом і кавою в кафе. Надійка звечора
маринує м’ясо для шашликів, а я вийшов до магазину, знайшов і розбив пару
дерев’яних ящиків на дрова для вогнища. 

26 квітня 1986 року, субота 

Надійчині батьки зранку поїхали на дачу. А ми швиденько попили чай з канапками,
одяглися і рушили в ліс. Вибрали гарне місце із зручними пеньочками для
сидіння. Я рубаю дрова, розпалюю вогнище... Мої дівчата, Надійка і Юля, нанизують
на шампури м’ясо і овочі...

Традиційне приготування шашликів. Процес документуємо, фотографуючи по черзі.

Погода хоча і похмура, але тепла. 

Смакуємо шашликами… Руки жирні, тому я і сам не беру камеру в руки, і своїм
жінкам не дозволяю. Та й що цікавого в процесі поїдання шашликів?!

Трапезу закінчено. Заливаю кострище водою, присипаю землею. Години три гуляємо
лісом.

Повертаємося додому. Мої дівчата бажають з годинку подрімати… Та і я не проти…
Відпочивали години дві, потім одягнулися і поїхали в центр… 

Поласувати морозивом можна було в спеціалізованих кафе. На Хрещатику
розташовувалися п'ять подібних закладів: одне біля площі Ленінського комсомолу
(нинішньої Європейської площі), два двоповерхових – біля входу в Пасаж (по
обидві сторони від сучасного фонтану, працювали тільки в теплу пору року) і ще
два – навпроти Бессарабського ринку. З цієї п'ятірки найбільш незвичайним було
кафе «Морозиво» у напівпідвалі будівлі по Хрещатику, 52 – праворуч від входу в
Театральний інститут. В народі його називали «Льох».

Це була справжня Мекка для любителів морозива! Приємний інтер'єр, стіни до
стелі «зашиті» дзеркалами в білих рамах, над столиками – бра з канделябрами.
Значний асортимент: кульки білого, крем-брюле, шоколадного, горіхового,
рожевого морозива, политі (за бажанням) сиропом, а то і подвійним сиропом;
напої в діапазоні від газованої води або соку до молочного коктейлю або
шампанського. 

Мої дівчата набрали собі різних кульок і по молочному коктейлю. Я ж взяв
шоколадне морозиво і бокал шампанського. 

Посмакували морозивом, пройшлися Хрещатиком до Піонерського парку, помилувалися
краєвидами Дніпра і десь годині о дев’ятій вечора поїхали додому.

27 квітня, неділя 

Зранку пішли гуляти в парк «Кинь Грусть», до озер. Обійшли всі три озера, потім
лісом перейшли до будівлі колишнього маєтку Куліженка. На той час в ньому
розміщувалися майстерні Спілки художників УРСР. Лісом пройшли до кабмінівських
дач, вийшли на вишгородську трасу біля фонтану, і через Мінський масив
поверталися додому. По дорозі зайшли в «Гастроном», де Надійка купила щось з
продуктів. Дома були годині о четвертій дня. 

Приблизно о 18-й годині з дачі повернулися Надійчині батьки. Жінки приготували
їжу. Сидимо на кухні, вечеряємо… Дзвонить телефон… Іду в кімнату, піднімаю
слухавку… Телефонує Віктор Шевчук, мій давній приятель. Схвильовано говорить,
що вночі 26-го вибухнув реактор Четвертого блоку ЧАЕС. Почалася повна евакуація
Прип’яті. Міліцію переводять в режим посиленої готовності. Віктор радить щільно
закрити вікна і поменше виходити на вулицю, оскільки вітер дме в сторону Києва
і несе радіоактивний пил. 

Дякую Віктору за інформацію і прошу тримати мене в курсі подій.

(Віктор – офіцер міліції, начальник відділу карного розшуку. Наприкінці 1969-го
– початку 1970-х ми разом служили в спецпідрозділі МВС по охороні державних
установ; разом вчилися в спеціальному учбовому закладі; у 1970-му, після угону
літака в Турцію, ми два місяці літали напарниками, супроводжуючи літаки під
виглядом пасажирів.) 

Повертаюся на кухню і розповідаю новину. Тесть, в минулому старший офіцер МВС
(і комуніст), вибухає гнівом, звинувачуючи мене у розповсюдженні брехні
«ворожих радіоголосів». Я кидаю виделку і демонстративно покидаю кухню. Через
пару хвилин дружина приносить мені на підносі незакінчену вечерю.

Через деякий час, повечерявши, в кімнату приходять дружина і донька. 

Я напружено розмірковую, з ким можна ще переговорити, щоб уточнити інформацію.
І, головне, що треба робити? Як вести себе в цих умовах?

В результаті роздумів вимальовується список:

- Людмила, інститутська подруга Надійки. Її батько – полковник міліції,
начальник штабу УВС м. Києва.

- Олег Байдін, сокурсник Надійки по навчанню в інституті, наш приятель. Після
інституту пішов по комсомольській лінії, працював секретарем райкому комсомолу.
Зараз – інструктор Шевченківського райкому партії. Нормальний мужик, не одну
чарку з ним випито. 

- Сашко Іванов, мій добрий знайомий, заступник декана факультету Інституту
іноземних мов; керівник спеціалізованої студентської дружини (ДНД); його мати –
Антоніна Миколаївна, – працює в ректораті інституту і має впливові зв’язки.

- Альвіра Кальченко, викладач хореографії в КДХУ; дружина Володимира Кальченко,
професора, сина Н.І. Кальченко, партійного і державного діяча.

- Валентин, мій приятель, за фахом геофізик, працює завідуючим фотолабораторією
Київської геофізичної експедиції на Автозаводській (метрах в 400-х від нашого
будинку). В експедиції, я точно знаю, ведеться моніторинг забрудненості
повітря, рівня радіації і радіоактивного фону.

Кажу Надійці, щоб вона зателефонувала Людмилі і обережно вияснила все, що
можливо, відносно вибуху реактора. 

А сам вирішив надрукувати фотографії з пікніка (плівку я проявив ще ввечері
26-го).

Закрився у ванній і друкую фотографії. Стукає Надійка… Впускаю її. Нічого від
Люсі дізнатися не вдалося – вона нічого не знає. Але важлива деталь – батько
телефонував і сказав, щоб вони не хвилювалися, бо йому потрібно затриматися на
роботі до ранку. 

28 квітня, понеділок 

Вранці поснідали. Юля пішла в школу, а Надійка поїхала на роботу. Я ж вирішив
сходити до Валентина поглянцювати фотографії і переговорити щодо Чорнобиля.
Оскільки давно до нього не заглядав, то зайшов у «Гастроном» і взяв пару пляшок
«Портвейн Таврический» (1 р. 62 к., 0,5 л.). Валентин радо зустрів мене… А коли
побачив портвейн, то його настрій різко підвищився… 

Я включив барабан нагріватися (АПСО-5м), розлили вино по гранчакам, смакуємо,
розмовляємо про життя… 

Розкладаю фотографії на полотно, вони закатуються на барабан. Декілька обертів
і фотографії вискакують сухими і блискучими. Я знімаю їх з полотна і кладу на
стіл. 

Валентин бере фотографії і розглядає… Потім питає: 

- Це ти коли в лісі був?

- В суботу, 26-го. Ми гуляли і 27-го, але я не фотографував.

Валентин знову роздивляється фотографію і мене, переводячи очі з фото на мене,
з мене на фото. Потім каже:

- Бери фотографії і ходімо. Пляшку бери з собою.

Ми допиваємо залишки вина, другу пляшку я ховаю в сумку. Туди ж кладу і
фотографії. 

Валентин веде мене по території в окремий будиночок. По дорозі пояснює, що це
лабораторія, яке веде контроль за радіацією, радіоактивним фоном і забрудненням
повітря. 

Заходимо, вітаємося... Валентин представляє мене, як свого давнього приятеля,
якому цілком можна довіряти. Потім просить мене показати фотографії. Я слухняно
дістаю фото... 

- Ось цей красавєц, – каже Валентин Валерію – в суботу і неділю довго гуляв
лісом в цій одежі... Поміряй його!

Валерій підносить датчик приладу до моїх кросівок... Жах! Прилад сильно тріщить!
Валерій піднімає датчик по джинсам вгору... Частота тріску зменшується... Біля
талії вже ледь тріщить. Вердикт, виголошений замість тосту під склянку
портвейну, звучить невмолимо грізно:

- Кросівки, джинси, сорочку, светр і труси знімаєш, кладеш в поліетиленовий
пакет і викидаєш в контейнер для сміття! Крапка!

- Друзі, ви шо, здуріли! Я джинси купив у березні за 150 «дерев’яних» і
кросівки за 100! Светр імпортний... 

- Ну, як хочеш! Імпотенти країні теж потрібні! А якщо хочеш залишитися
здоровим, то зараз ідеш додому, збираєш речі дружини і доньки – все, що було на
вас одягнуто під час прогулянок, і несеш сюди. Не забудь по дорозі прикупити
пару пляшок «Мадери». Після замірів нам потрібно провести «колективну
дезактивацію»!

Що таке радіоактивне забруднення я знаю – в армії служив!

Приходжу додому, роздягаюся на балконі… Джинси, кросівки, светр, сорочку,
шкарпетки складаю в спортивну сумку. Ту сумку, з якою ходили на прогулянку.
Туди ж складаю речі дружини і доньки, в яких вони ходили в ліс. Окремо в газету
загортаю туристичну сокирку і каструлю, кладу їх в «авоську». 

Ретельно мию голову, приймаю повний душ. Одягаюся і йду в «Гастроном». Беру дві
пляшки «Мадери (по 0,7) і повертаюся в експедицію. Заходжу до Валентина, разом
йдемо в лабораторію. Валерій міряє речі – все дзвенить! А сокирка і каструля не
дзвенять!

Валерій поливає складені в сумку речі якоюсь рідиною… На мій здивований погляд
пояснює:

- Сірчана кислота! Це щоб мусорщики, які вивозять наші відходи і сміття, не
спокусилися на речі. За будинком стоїть закритий контейнер для сміття. Викинь
туди сумку.

Майже із сльозами на очах прощаюся з купою дорогих речей! Скільки часу, зусиль
і грошей було витрачено на пошуки і придбання модних речей. І в одну мить все
пропало! 

[В ті роки слово «дістав» було набагато більш уживаним, ніж «купив». Гарну річ
можна було придбати тільки через зв’язки або, як тоді казали, «по блату».]

За «колективною дезактивацією» Валерій пояснює, що вітер дме в сторону Києва і
несе радіоактивний пил, який осідає на дерева і землю. Рівень радіації в
десятки разів перевищує звичайний радіоактивний фон. Цифр він не називає –
секретність, але що ті цифри мені дадуть!? Головне – високий рівень радіації!

Трапезу закінчено. В хореографічне училище їхати не має сенсу – від мене за
версту тхне вином! Тому домовляюся, що товариші будуть тримати мене в курсі
подій і йду додому. 

***

Донька вже прийшла зі школи і запитує куди подівся її спортивний костюм і
кросівки. Обіцяю все пояснити, коли з роботи приїде Надійка. 

Годині о сьомій вечора повертається з роботи Надійка. Я розповідаю дівчатам про
долю забруднених речей… З сумом підраховуємо наші втрати:

- Джинси (Wrangler i Levi Strauss) – 2;

- Кросівки (Adidas) – 2;

- Курточка (імпортна) – 1;

- Светр (імпортний) – 2;

- Сорочка (імпортна) – 1;

- Спортивний костюм (імпортний) – 1;

- Туфлі жіночі (імпортні) – 1;

- Білизна, колготки, шкарпетки.

- Сумка спортивна.

КапецЪ! Добра пропало рублів на 800! 

***

Пізно ввечері телефонує Віктор… Розповідає, що триває повна евакуація населення
з 10-кілометрової зони. Я в свою чергу, розповідаю йому про свої пригоди. 

Буквально хвилин через десять після нашої розмови телефонує Сашко. Каже, що
мати (Антоніна Миколаївна), відправляє його вагітну дружину до своїх родичів у
Росію. Наступного дня він з Оленою від’їжджають. Повернеться у Київ через
тиждень. Радить якнайшвидше відправити Надійку з Юлею в село до родичів, бо
коли у Києві почнеться паніка, то з квитками буде величезна проблема.

Розмовляю з Надійкою. Вирішуємо, що наступного дня вона напише заяву на
відпустку з 5 травня (планова відпустка у неї намічалася з кінця травня) і з
Юлею поїдуть до родичів в Говтву, на Полтавщині. 

29 квітня, вівторок

Надійка – на роботу, Юля – в школу. А я до Валентина і Валерія. Рівень радіації
високий!

Вдень телефонує Надійка… Начальство на нарадах. Заяву на відпустку підписати
нікому. Питання вирішиться або завтра, 30-го, або вже після свят, 5-го травня. 

***

Ввечері замовляємо телефонну розмову з Говтвою. Розмовляємо з сестрою тестя,
Марією… Надійка питає чи можна їй приїхати з Юлею на місяць відпочити?
Заперечень немає! Запитує, що їм треба привезти з міста… Повторює вголос, а я
записую…

Нашу розмову почула теща і тут же донесла тестю. 

Черговий скандал! Тесть звинувачує мене, що я «антирадянщик» і наслухавшись
«ворожих голосів» провокую паніку, зриваю Надійку з роботи, а доньку зі школи… 

Нічого страшного! Вдень у них були партійні збори (пенсіонерів, що стоять на
партійному обліку при ЖЕКу), на яких сказали, що «ситуація під контролем» і 1
травня, як завжди, буде демонстрація на Хрещатику.

Я не витримав і обізвав його «тупим комунякою з партквитком замість душі і
серця...». Ледь не дійшло до бійки... Але тесть вчасно зрозумів, що ми в різних
категоріях... Йому 71, а мені 39 і я в набагато кращій фізичній формі. Тому він
різко збавив свою агресивність і перестав розмахувати руками. 

***

Чотири святкових дні тягнулися як чотири тижні… Практично ми ці дні просиділи
вдома. Надійка і Юля ходили в кімнату батьків дивитися телевізор, а я з
батьками Надійки принципово не спілкувався. Вікна закриті, двері на балкон
закриті. На кухні їмо по черзі. Спочатку батьки, потім ми.

Ввечері 4 травня зателефонував Віктор. Сказав, що будуть евакуйовані всі
населені пункти в 30-кілометровій зоні навколо ЧАЕС.

***

5 травня, понеділок

Надійка зателефонувала, що їй підписали заяву на відпуску з 12 травня
(понеділок). Відпускні отримає 7-го, в середу

Я одразу помчав на вокзал за квитками. Вистояв довжелезну чергу у каси
попереднього продажу квитків і взяв квитки на потяг ввечері 10 травня. 

6 – 9 травня

Надійка на роботі. За складеним списком купую гостинці для родичів у селі. Беру
також «універсальну валюту» – горілку, бо від залізничної станції до села можна
дістатися або автобусом, який ходить нерегулярно, або попутною автівкою. За
пляшку «казенки» тебе довезуть куди потрібно. 

В ці дні вдалося живцем поспілкуватися з Альвірою Кальченко і Олегом Байдіним.
Масштабів катастрофи вони не знають, але все дуже серйозно. Йде повна евакуація
жителів всіх населених пунктів у 30-кілометровій зоні навколо ЧАЕС!

Надійка побувала в школі, де вчиться Юля… Багатьох дітей батьки не пускають в
школу і планують вивезти з Києва. Щодо від’їзду Юлі в село класний керівник не
заперечує.

10 травня, субота

Зранку складаємо валізи… Вирішуємо, що дівчата з речей беруть з собою тільки
саме необхідне, а також подарунки для родичів. Решту речей, по мірі
необхідності (невідомо скільки їм доведеться там жити), я відправлю посилками.
Супроводжувати їх у Говтву я не можу, оскільки мені в силу певних обставин
потрібно залишатися в Києві.

Ввечері проводжаю рідних на потяг… На серці сумно. Прощаємося… Потяг
відправляється… Хвилину йду поруч з потягом… Надійка і Юля махають мені рукою
крізь скло вікна… Я відповідаю… Коли ми побачимося знову?!

Додому їхати не хочеться.

З таксофона телефоную Сашку, який вже повернувся у Київ, і пропоную провести
«колективну дезактивацію». Він радіє, бо вже кілька разів телефонував мені
додому, маючи запропонувати мені аналогічне. Домовляємося зустрітися біля
«Центрального гастроному» (на розі Хрещатика і вулиці Леніна), щоб узяти запас
«дезактиватора».

Стрибаю в метро, виходжу на Хрещатику. Біля гастроному мене чекають Сашко і
Вадим, ще один наш приятель, колишній співкурсник Сашка, а нині викладач
англійської мови і завуч середньої школи. 

Беремо «Мадеру», «Портвейн Таврический», свіжоморожений хек, батони, вершкове
масло і йдемо до Сашка, який живе поруч, на вулиці Енгельса. Антоніна
Миколаївна допомагає нам накрити «поляну» – смажить рибу. 

Чудова вечеря. Десь о другій годині ночі Антоніна Миколаївна стелить нам на
підлозі і ми лягаємо спати.

***

Надійка з Юлею пробули в селі до кінця червня. 

Мені зателефонували із школи і повідомили, що починається плановий вивіз дітей
на оздоровлення до моря. Розказали які речі треба приготувати для Юлі, щоб вона
взяла їх з собою.

Я поїхав у Говтву, забрав Юлю в Київ. Через день повернулася і Надійка. 

Юлін клас відправили в піонерський табір «Красная гвоздика» під Ждановим (зараз
Маріуполь). Там діти пробули десь до 20 серпня.

***
\ii{30_11_2021.fb.fb_group.story_kiev_ua.1.chaes_piknik_puscha_vodica.cmt}
