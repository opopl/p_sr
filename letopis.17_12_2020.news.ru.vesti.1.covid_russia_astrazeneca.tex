% vim: keymap=russian-jcukenwin
%%beginhead 
 
%%file 17_12_2020.news.ru.vesti.1.covid_russia_astrazeneca
%%parent 17_12_2020
 
%%url https://www.vesti.ru/article/2500028
 
%%author 
%%author_id 
%%author_url 
 
%%tags covid_vaccine,vaccine,covid,astrazeneca,sputnik_v,russia
%%title COVID-19: сотрудничество Центра Гамалеи и AstraZeneca важно для всего мира
 
%%endhead 
 
\subsection{COVID-19: сотрудничество Центра Гамалеи и AstraZeneca важно для всего мира}
\label{sec:17_12_2020.news.ru.vesti.1.covid_russia_astrazeneca}
\Purl{https://www.vesti.ru/article/2500028}

\video{https://player.vgtrk.com/7b15f157-2194-4a4d-ba73-84bdc9085519}

\ifcmt
  pic https://cdn-st1.rtr-vesti.ru/vh/pictures/xw/307/883/5.jpg
  width 0.5
  fig_env wrapfigure
\fi

Сотрудничество российского Центра имени Гамалеи и компании AstraZeneca в работе
над вакциной против коронавируса COVID-19 важно для всего мира. С таким
заявлением выступил на Ежегодной пресс-конференции президент России Владимир
Путин.\Furl{https://www.vesti.ru/doc.html?id=291713}

"Наши коллеги иностранные, слава богу, тоже повернулись к нам лицом, готовы к
сотрудничеству, – сказал президент. – Что-то у них не получается, компания
AstraZeneca готова с нами работать, подписывают сейчас соответствующие
соглашение".

По словам Путина, такое сотрудничество – "это очень хорошо".

"Меня это очень радует, когда специалисты на таком высоком уровне – а это
крупная хорошая компания с мировым именем – когда они объединяют усилия в том
числе и с российскими партнерами", – подчеркнул глава государства.

Президент выразил уверенность в том, что "результат будет очень хорошим не
только для наших граждан, но и для всего мира".

По словам Путина, он рассчитывает на то, что "уже в начале 2021 года у России
будут миллионы доз вакцины".

"Для того, чтобы произвести эту вакцину, нужны соответствующие заводы,
предприятия, оборудование – это будет наращиваться, – сказал Путин на ежегодной
пресс-конференции. – Исхожу из того, что все планы в этом смысле будут
исполнены, в следующем году, уже в начале, у нас будут миллионы доз вакцины".

Глава государства напомнил, что Россия первой создала и зарегистрировала
вакцину против коронавируса: это был препарат "Спутник V" Центра имени Гамалеи.
Позже свою вакцину зарегистрировал новосибирский научный центр "Вектор".

"У нас хорошая вакцина: и безопасная, и эффективно действующая – 95 с лишним
процентов, под 96-97 процентов уровень защиты, говорят специалисты", – отметил
президент.

Российский лидер добавил, что поддерживает всеобщую вакцинацию. По его словам,
"это нужно делать".

"То же самое говорят не только наши отечественные вирусологи, но и их коллеги
практически во всем мире, – заявил президент. – Один из немногих путей
преодоления всех вопросов, связанных с пандемией, это массовая вакцинация".

Ранее в Российском фонде прямых инвестиций (РФПИ), который участвовал в
разработке вакцины, рассказали о сотрудничестве\Furl{https://smotrim.ru/article/2497667} Центра имени Гамалеи с
AstraZeneca.

