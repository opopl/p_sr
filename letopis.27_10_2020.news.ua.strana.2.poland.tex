% vim: keymap=russian-jcukenwin
%%beginhead 
 
%%file 27_10_2020.news.ua.strana.2.poland
%%parent 27_10_2020
%%url https://strana.ua/news/297502-v-polshe-vvodjat-voennye-patruli-iz-za-karantina-na-fone-antipravitelstvennykh-protestov.html
%%author aleksandra harchenko
 
%%endhead 

\subsection{"Уеб@вайте!". Почему Польшу в карантин охватили протесты под матерным лозунгом}
\label{sec:27_10_2020.news.ua.strana.2.poland}

\url{https://strana.ua/news/297502-v-polshe-vvodjat-voennye-patruli-iz-za-karantina-na-fone-antipravitelstvennykh-protestov.html}

Александра Харченко 22:49, 27 октября 2020 
\index[authors.rus]{Харченко, Александра}

\ifcmt
img_begin 
  url https://strana.ua/img/article/2975/2_main.jpeg
  caption Краков стал одним из десятков городов во главе с Варшавой, где протестующие перекрыли движение. Фото: Instagram 
  width 0.7
img_end
\fi

Несмотря на ужесточение карантинных мер, Польша уже почти неделю охвачена
многомиллионными протестами с блокадами улиц.

Поводом стал принятый закон о практически полном запрете абортов, в том числе
по медицинским показаниям, но митинги переросли в массовый
антиправительственный и антицерковный бунт с тенденцией к эскалации, к которому
присоединяются и другие недовольные. 

Полиция применяет против протестующих слезоточивый газ и повсеместно жестко
разгоняет акции.

В среду, 28 октября, в Варшаве фермеры планируют перекрыть трассы тракторами, у
них свои заботы - новый закон о защите животных, который сейчас рассматривается
в парламенте страны и который, как опасаются они, отнимет рентабельность у
животноводства. В среду же многие грозятся в знак протеста не выйти на работу.

Также на улицы выходят таксисты и рестораторы, недовольные почти полным
локдауном, который на днях объявили в Польше.

"Протесты очень смешанные, хаотичные, выходят по разным причинам. Многие еще
озлоблены из-за повторного карантина, и все эти локдауны и запреты абортов
считают вмешательством в частную жизнь со стороны государства. А в Кракове на
митинге против запрета абортов еще и поддерживали Беларусь. Народ завелся,
некоторые видят в этом попытку реванша оппозиционных сил", - говорит "Стране"
польский журналист Марчин Ковальчик.   

Девизом протестного движения стал "Это война" и нецензурное "Wypierdalać!", что
в переводе значит "Уе@ывайте!", адресованный правящей консервативной партии
"Право и справедливость" во главе с Ярославом Качиньским, негласным лидером
страны.

В оппозиционной среде, кстати, дискутируют, этично ли использовать нецензурную
лексику как главный лозунг демонстраций. Но, как говорят сами протестующие, их
"прорвало". Также недовольные прозвали Польшу "первым халифатом Европы" и
негодуют, что страна "погружается в средневековье".

\ifcmt
img_begin 
  url https://strana.ua/img/forall/u/10/85/122657983_10225665888099892_5139022311259959138_n.jpg
img_end
\fi

Забастовщики помимо перекрытия дорог атакуют храмы и срывают службы в костелах
католической церкви, очень влиятельной в стране (запрет на аборты был введен,
во многом, по инициативе церкви).

Некоторые противники запрета абортов называют цели церкви иезуитскими, обвиняя
ее в том, что ксендзы рассчитывают на получение выгоды за счет пожертвований от
родителей тяжело больных детей, которым не позволили аборт, и за счет дотаций
от государства за пожизненный уход за теми, кого оставят в роддоме.

Власти уже отреагировали на протесты. 

27 октября вечером Качиньский выступил с обращением к народу, в котором назвал
протестующих "преступниками", призвал защищать церкви и оградить детей от
"тлетворного влияния нигилизма и экспрессии протестующих".

"Эта атака на церковь - это то наступление, которое должно уничтожить Польшу,
привести к триумфу силы, власть которых закончит историю польского народа, как
мы его до этого понимали", - сказал Качиньский.

"Оптимистичные комментарии из соцсетей: это так Качиньский, который всегда
хотел быть Пилсудским, а стал Ярузельским, таким образом хочет отвлечь внимание
от все больших косяков власти в борьбе с Ковидом. Пессимистичные комментарии:
это прямой призыв к гражданской войне, мы или они. Уступок не будет", - пишет
на своей странице в Facebook живущая в Польше журналистка Елена Бабакова.

\ifcmt
img_begin 
  url https://strana.ua/img/forall/u/10/85/122673064_350736496180757_4643480391216941734_n.jpg
  caption Польша восстала - и не только против запрета абортов
img_end
\fi

Тем временем власти с 28 октября еще больше усиливают карантин (об этом чуть
ниже) и вводят на улицы военные патрули. В Польше это восприняли в штыки,
посчитав, что настоящей причиной ужесточения карантина стали протесты, а не
коронавирус. 

"Это уже военное положение или нужно дождаться, пока Качиньский наденет военную
форму и тёмные очки?" - пишет в Facebook украинский журналист, который живет в
Польше, Олег Хавич. "Темные очки" - это намек на правителя коммунистической
Польши генерала Ярузельского, который ввел в начале 80-х годов чрезвычайное
положение в стране на фоне протестов движения "Солидарность". 

\ifcmt
img_begin 
  url https://strana.ua/img/forall/u/10/85/122680129_679342099377427_4116561231708356882_n(1).jpg
  caption В названии города Варшава протестующие обыграли слов war (в переводе с английского - война)
img_end
\fi

В экспертной среде не исключают усиления протестов и перерастания их в стычку
между польскими правыми консерваторами и леваками. 

"Польский протест последних дней вообще не похож на восточноевропейские
революции достоинства последних лет. Это чисто Париж 1968 год (социальный
кризис во Франции, начавшийся с леворадикальных студенческих выступлений и
вылившийся d массовые беспорядки - Ред.)", - пишет Бабакова. 

В комментариях под постом напоминают, что парижский протест был марксистским.

Протесты совпали с ужесточением ограничений из-за коронавируса, которые
вступили в силу с минувшей субботы, 23 октября, когда вся Польша вошла в
"красную зону".

В стране закрыли рестораны, оставив только еду на вынос и доставку, санатории,
а учеников школ, начиная с 4-го класса, перевели на дистанционное обучение.

Дети до 16 лет днем смогут выходить из дома только под присмотром родителей, а
сборы более пяти человек запретили.

Власти Польши не исключают полного локдауна.

Частичный карантин уже повлиял на украинских заробитчан, которые снова жалуются
на потерю работы.

"Только устроился снова в ресторан, как его закрыли. Весной меня уже уволили из
другого кафе из-за карантина, не рассчитались до конца. Пришлось вернуться
домой, но потом снова приехал в Польшу, дома вообще никакой работы нет", -
рассказал нам житель Львовщины Олег Маслюк.

В других сферах тоже начались сокращения зарплат.

"Летом собирала урожай, но сезонная работа закончилась, устроилась на завод по
упаковке и сортировке, но нам сказали, что ставки урежут на 20\% из-за нового
карантина, хотя как нас касается закрытие ресторанов, неясно. Просто "отмазка",
чтобы меньше платить. Теперь буду получать 14 тысяч гривен на наши деньги, а не
20 тысяч, как раньше", - призналась "Стране" Галина Стахурская, которая
трудится под Краковом.
