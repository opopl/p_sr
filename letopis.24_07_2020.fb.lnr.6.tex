% vim: keymap=russian-jcukenwin
%%beginhead 
 
%%file 24_07_2020.fb.lnr.6
%%parent 24_07_2020
 
%%endhead 
\subsection{Спасибо Путину! Потеря Крыма и Донбасса стала счастьем для нас – украинский писатель}
\url{https://www.facebook.com/groups/LNRGUMO/permalink/2866596016785287/}

\index{Авторы!Максим Карпенко}
\index{Люди!Лесь Подеревянский}
\index{FB!GROUPS!LNRGUMO}

Спасибо Путину! Потеря Крыма и Донбасса стала счастьем для нас – украинский писатель

События 2014 года, в результате которых Украина лишилась значительной части
своей территории, пошли на пользу Украине.

Об этом в эфире меджлисовского телеканала ATR заявил украинский писатель Лесь
Подеревянский, передает корреспондент «ПолитНавигатора».

«Не отобрали бы у нас Крым и Донбасс, могли ли мы, имея электорат Крыма и
Донбасса, провести хоть какие-нибудь реформы? Мы сидели бы в такой же жопе.
Очень трудно классифицировать какие-то события, когда нет временного люфта. Мы
часто воспринимаем какие-то вещи, что это очень плохо, а на самом деле, потом,
по прошествии времени, мы видим, что именно в этом был единственный выход и это
было хорошо. В перспективе хорошо, хотя, может быть, сейчас нам кажется, что
это очень херово», – заявил Подеревянский.

По его мнению, с российской стороны было неправильно ограничиваться Крымом –
вся Украина могла бы стать частью РФ, если бы постмайданные элиты были куплены
Россией.

«В итоге кому мы обязаны независимостью и взлетом национального сознания, этими
томосами? Всем мы обязаны только одному человеку – Путину. Не нам, а Путину, он
все это замутил. Если бы он это не замутил, если бы он был человеком каким-то
умным. Вот у нас революция, допустим, Янык сбежал. Он бы сказал: «Какие вы
замечательные наши братья, вы прогнали этого коррупционера, вот вам деньги на
развитие – все, элита была бы разложена тут же. Мы стали бы тут же прекрасной
большой областью этого кацапстана… Они очень плохо считают, буквально каждый их
шаг – в штангу. Они отобрали Крым, но потеряли Украину – очень умный поступок»,
– считает писатель.

Максим Карпенко
  
