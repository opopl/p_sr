% vim: keymap=russian-jcukenwin
%%beginhead 
 
%%file slova.veter
%%parent slova
 
%%url 
 
%%author 
%%author_id 
%%author_url 
 
%%tags 
%%title 
 
%%endhead 
\chapter{Ветер}
\label{sec:slova.veter}

%%%cit
%%%cit_head
%%%cit_pic
%%%cit_text
\enquote{Как ветер засвистит — аж кровь закипает}.
Как оказалось, квартиры в полупустых пятиэтажках все же закреплены за своими
хозяевами, которые разъехались кто куда. Приобретают недвижимость в Цукроварове
обычно местные жители, которые хотят переехать в более просторные апартаменты.
Расценки на жилье, по словам местных, примерно такие: за
\enquote{двушку-трешку} — 3000-4000 гривен, а за четырехкомнатную — 5000
гривен.  Фантастически мизерные цены по меркам столицы и даже областного центра
— Кропивницкого.  — Только у нас никто не хочет и даром эти квартиры. Жутко.
Хоть я и живу здесь всю жизнь, но очень страшно находиться здесь после 20:00.
Бывает ведь такое, что одна жилая квартира на весь дом, а \emph{ветер} как
начнет свистеть — аж кровь в жилах закипает. А рядом еще и руины завода, — с
ужасом в голосе рассказывает пенсионерка Надежда, одна из немногих, кто
продолжает жить в Цукроварове, несмотря на разруху
%%%cit_comment
%%%cit_title
\citTitle{\enquote{Страна} съездила в город-призрак, который когда-то был одним из самых значимых промышленных центров страны}, 
Юлия Корзун; Полина Пронина, strana.ua, 13.06.2021
%%%endcit

