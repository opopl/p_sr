% vim: keymap=russian-jcukenwin
%%beginhead 
 
%%file 16_01_2022.stz.news.ua.hvylya.1.anatomia_vraga.7.endogennye_sily_razvitia_strategii
%%parent 16_01_2022.stz.news.ua.hvylya.1.anatomia_vraga
 
%%url 
 
%%author_id 
%%date 
 
%%tags 
%%title 
 
%%endhead 

\subsubsection{Эндогенные силы развития стратегии}
\label{sec:16_01_2022.stz.news.ua.hvylya.1.anatomia_vraga.7.endogennye_sily_razvitia_strategii}

В процессе экспансии Москвы на север, юг и восток, сработал исторический
парадокс - колоссальное увеличение \enquote{фронтира} и своей суверенной территории,
даже после \enquote{величайшей геополитической катастрофы} - краха Золотой (или
Большой) Орды в 1480 году, \enquote{не решило проблему выживания}.

Хотя непосредственная внешняя угроза столице стала существенно меньше, огромные
пространства \enquote{территории хаоса} на востоке, севере и юго-востоке продолжали
существовать и это вынуждало Москву, по примеру античного Рима продолжать
экспансию \textbf{в своей стратегеме \enquote{защиты от хаоса} через завоевание новых
территорий}, покорение и мировоззренческую ассимиляцию \enquote{стабильностью} этнически
чуждых народов - от взятия Казани и \enquote{замирения} черемисов до покорения Чукотки
и от завоевания Кавказа до многочисленных кровавых войн с Османской империей и
Персией.

Таким образом, московский экспансионизм, как \textbf{непрерывный экспорт мировоззрения
\enquote{единоначального порядка} и постоянная борьба с \enquote{территорией хаоса}} стали
самоцелью. Последующее появление уже имперских геополитических амбиций и
создание сфер своего влияния методом \enquote{защиты} через нападение стали эндогенной
силой российской государственной стратегии. Причем, этот \enquote{грубый}, на первый
взгляд, метод работает, вне всякой зависимости от исторического периода и формы
идеологемы, будь то воинствующее православие, марксизм, превращенный русскими
большевиками в суррогат религии или же современный путинский неоимперский
конструкт, построенный на ресентименте по отношению к странам запада и культе
\enquote{недовоеванной} войны.
