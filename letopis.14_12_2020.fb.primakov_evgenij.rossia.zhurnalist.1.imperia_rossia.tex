% vim: keymap=russian-jcukenwin
%%beginhead 
 
%%file 14_12_2020.fb.primakov_evgenij.rossia.zhurnalist.1.imperia_rossia
%%parent 
 
%%url https://www.facebook.com/permalink.php?story_fbid=10222134411261599&id=1602777439
 
%%author 
%%author_id primakov_evgenij.rossia.zhurnalist
%%author_url 
 
%%tags ideologia,imperia,rossia,rusmir
%%title 
 
%%endhead 
\subsection{Империя – это сверхкультура}
\Purl{https://www.facebook.com/permalink.php?story_fbid=10222134411261599&id=1602777439}
\ifcmt
 author_begin
   author_id primakov_evgenij.rossia.zhurnalist
 author_end
\fi

Помните неприличный анекдот про «но есть нюансы» - вот его я вспоминаю всё
чаще, когда в дискуссии с очередным профессиональным «имперцем» тот заявляет,
что Российская Империя должна опираться на принуждение, силу и грубое
доминирование.

Вчитаться в эти нюансы – дело интеллектуально сложное: «многобуков», читать
лень, думать и того хуже, возьмёмся за руки и поорём. 

Но я попробую занудно разложить по полкам.

Империя – это сверхкультура. Империя объединяет многие народы, обеспечивая их
права и процветание. Ядро, маяк, на который равняются, в любой империи – народ,
который обладает самой великой культурой, дающий империи общий язык, стандарты
науки, примеры великой литературы, определяющей сознание многих прочих народов,
создаёт смысл для развития.

Именно этот народ и называют государствообразующим. Его процветание, его
развитие – гарантирует процветание и развитие многих прочих народов в империи.
Его величие – не в угнетении других. Это, мне кажется, аксиоматика. Империя,
которая душит народы, входящие в неё, обречена. Ровно так же обречена империя,
которая душит основной, государствообразующий народ, расплачиваясь за его счёт
с другими. И это тоже аксиоматика.

В России – государствообразующий народ, народ дающий всему государству язык и
величие культуры – русский. При всех блистательных примерах прочих национальных
культур – их процветание только и возможно под прикрытием, при поддержке
русского народа и русской великой культуры. Как в имперской короне каждый
камень сияет ярче, делая корону блистательной.

Зачем я занудно пишу вот это.

По совершенно практическим причинам. 

У нас есть три подхода.

Первый, очевидно, капитулянтской. Империя – зло, надо платить и каяться, надо
обязательно «всех отпустить», сколько можно притеснять. 

Порой капитулянтской подход выражается не уважением чужого суверенитета, а
ограничением своего собственного ради «уважаемых партнёров»: а давайте мы вам
много денег дадим, скидок на газ, может и территорию свою прирежем, а если вы
закроете русские школы и будете устраивать военное партнёрство с нашими врагами
– ну так вы же в своём праве. «Русские врывались в кишлаки и оставляли за собой
дымящиеся развалины».


Иногда этот же капитулянтской подход пытаются продать аудитории как
«отстаивание наших национальных интересов» - мы за мир ради мира, не будем
раздражать. «Любим, ценим, уважаем» до самоуничижения – в интернет-публицистике
это называется «новиопство». Суверенитет этих вот стран воспринимается либо как
самоценность, либо как анти-российский (для чего были основания). «Русские
врывались в кишлаки, оставляли за собой школы, больницы и библиотеки.
Возрадуемся!»

Это порочный, негодный, дрянной подход.

Второй, казалось бы противоположный, но совершенно родственный: на что нам эти
национальные государства-недоразумения, мы – Великая Русская Империя, прочно
стоим на костях всех прочих, должны прямо говорить этим прочим как они
никчёмны, подлы, неполноценны, ура экспансии и принуждению. Подход
профессионального тролля, что бы вы ни вкладывали в это слово. Легализует этот
подход очевидные неудачи, провалы, слабость России в пространстве бывшего СССР.

Не всюду, но там, где такие неудачи заметнее всего, они обнуляют все
относительные успехи, перекрывают их в разы. Кто вспомнит «выселение»
американцев с базы Манас под Бишкеком и размещение там наших сил, когда голос
одно недоумка о том, что русского языка в Киргизии не должно быть, по эффекту
оказывается громче. Бойтесь, киргизы. Бойтесь белорусы, армяне, таджики и
узбеки, бойтесь все – «русский мир» грядет так, что мало не покажется. Я почти
уверен, что Think-tanks в Британии и Штатах, если ещё не додумались заносить
поддержку вот таким мыслителям, то должны всерьёз подумать об этом. Для них это
перспективное направление: злой пубертат потому что мамка запретила журнал
Playboy, или злостная провокация. Суверенитет этих вот стран воспринимается как
однозначно анти-российский (и для этого раньше часто были основания). «Русские
врывались в кишлаки и оставляли за собой школы, больницы и библиотеки, а вместо
этого надо было гарнизоны и губернатора». Делает ли такой подход наш внешний
периметр безопаснее в реальности?  Это порочный, негодный и дрянной подход.

И есть третий, который только-только появляется на практике: дружите с Россией,
объединяйтесь с Россией, равняйтесь на Россию, потому что с Россией безопасно,
хорошо, выгодно, потому что те блага, которые вам даёт содружество и
сотрудничество с Россией – традиционные ценности в сочетании с технологическим
модернизмом, безопасность и независимость от разлагающих государство мягких и
жёстких атак извне – возможны только с Россией. 

Этот подход предполагает
бухгалтерский учёт наших затрат – и мы исходим из того, что в прекрасной России
будущего это выгодно и другим странам, потом что такая поддержка укрепит их
гражданское общество и суверенитет и сбережёт их. 

Не за счёт России, а вместе с
Россией. Вопросы общей безопасности являются гарантией и внешней границы
России. Привилегии российских компаний усиливают местную экономику. Привилегии
российского образования, медицины, технологий – обеспечивают единое культурное
пространство. 

Посмотрите, что делали британцы и французы, с некоторой спецификой, но что
позволяет им и сегодня говорить о том, как хорошо, приятно и полезно
сотрудничать с Францией и Британией? «Русские врывались в кишлаки и оставляли
за собой школы, больницы и библиотеки, почту, телеграф, телефон, газету,
телеканал, двадцать Telegram-каналов и военную базу» 

Многое предстоит переосмыслить – российская помощь не должна доставаться в
безраздельное владение правящим элитам, важна её ощутимость для людей, важно
понимание того, что дороги и мосты, рабочие места, лучшее образование и
медицина – не дар местного князя холопам, или не откуп от России, который выбил
местный князь, а проявление братской любви, содружества с Россией – и потому
отказаться от России, отвернуться от России плохо, означает, что жизнь станет
хуже.

Именно поэтому нужно объяснять преимущества содружества и сотрудничества с
Россией. Что только Россия гарантирует развитие, безопасность, суверенитет.
Суверенитет как защиту от разложения, архаизации, навязанных чужих ценностей,
колониального правления. Суверенитет и мир при поддержке России и при условии
поддержки Россией, которая защищает свой периметр и обеспечивает своё развитие.
