% vim: keymap=russian-jcukenwin
%%beginhead 
 
%%file 03_01_2022.fb.fb_group.story_kiev_ua.1.ev_lobanovskij
%%parent 03_01_2022
 
%%url https://www.facebook.com/groups/story.kiev.ua/posts/1832619896934803
 
%%author_id fb_group.story_kiev_ua,krysjkov_sergij.kiev
%%date 
 
%%tags kiev,lobanovskij_valerij.trener.sssr
%%title Мои воспоминания о Е. В. Лобановском, брате Валерия Лобановского
 
%%endhead 
 
\subsection{Мои воспоминания о Е. В. Лобановском, брате Валерия Лобановского}
\label{sec:03_01_2022.fb.fb_group.story_kiev_ua.1.ev_lobanovskij}
 
\Purl{https://www.facebook.com/groups/story.kiev.ua/posts/1832619896934803}
\ifcmt
 author_begin
   author_id fb_group.story_kiev_ua,krysjkov_sergij.kiev
 author_end
\fi

Мои воспоминания о Е.В. Лобановском, брате Валерия Лобановского. Написано по
просьбе Дениса М., интересующегося историей семьи Лобановских.

Летом 1974 года, после окончания Киевского Института Пищевой Промышленности
(КТИПП) по специальности «промышленная теплоэнергетика», я был распределён на
работу в Укргипросахар – государственный институт по проектированию предприятий
сахарной промышленности, в теплотехнический отдел. То, что начальником отдела
был человек по фамилии Лобановский, не вызывало у меня эмоций до того момента,
пока я не узнал, что он – родной брат великого и легендарного Валерия
Лобановского. Естественно, такой начальник был предметом гордости сотрудников
отдела.

\ii{03_01_2022.fb.fb_group.story_kiev_ua.1.ev_lobanovskij.pic.1}

Скажу, что на протяжении моего трудового стажа, мне везло с начальниками, за
редким исключением. Попадались откровенные жлобы и самодуры, и на такой работе
я долго не задерживался. Точно так же, когда Евгений Васильевич стал главным
инженером, а на его место пришёл другой человек, я уволился. Это событие
совпало с окончанием моего трёхлетнего пребывания в статусе молодого
специалиста. Выпускники институтов обязательно распределялись на работу по
предприятиям и учреждениям, подавшим соответствующие заявки в учебное
заведение. Распределённые таким образом специалисты, обязаны были отработать 3
года на том месте, куда были направлены. В течение этих трёх лет, работник
является "молодым специалистом" и не может быть уволен ни при каких условиях.
Просто не существовало такой юридической нормы и, любое увольнение до окончания
этого срока, считалось нарушением закона.

Молодых специалистов обычно брали на оклад 100руб. в месяц, обладатель красного
диплома мог рассчитывать на 110 р. У меня диплом был обычный, оценки - 4 и 5,
кроме истории КПСС, политэкономии, диалектического и прочего материализЪма,
которые я органически не переваривал. Таким образом, з/п была мне назначена
100р. в месяц, минус подоходный налог, минус налог за бездетность. Получал на
руки чуть меньше 90 р, точно уже не припомню. Случались также и премии.

Теплотехники, занимались проектированием новых и реконструкцией существующих
котелен и тепловых хозяйств сахарных заводов, которых в Украине было тогда
более 400. Так что, работы хватало. Реконструировались котельные, переводимые с
мазута на газообразное топливо (в связи с тогдашним нефтяным кризисом), а также
те, где требовалось нарастить производственную мощность.

Институт полностью занимал 6-этажное административное здание на улице Щорса
(теперь Евгения Коновальца) 31, а наш отдел  - половину левого его крыла, на
третьем этаже. В другой половине был расположен сметный отдел, отгороженный от
нас рядом шкафов с документацией. Каждому сотруднику был выделен стол и
кульман, 2 ряда которых протянулись вдоль просторного и светлого помещения.
Рабочее место начальника располагалось у входной двери и было отгорожено от
общего объёма стеклянным «аквариумом».

Теплотехнический отдел был разбит на несколько бригад по 6-8 специалистов в
каждой, с руководителем бригады во главе. Бригада занималась одним большим, или
несколькими мелкими проектами, бригадир же распределял задачи между
сотрудниками, помогал начинающим, корректировал и контролировал выполнение.
Сотрудников могли переводить из одной бригады в другую, что бывало редко, либо
временно переподчинять - в случае аврала или штурмовщины - что бывало часто.
Поэтому, первый вопрос сотруднику, явившемуся к начальнику, Лобановский
озвучивал довольно  оригинально:

- Вы на кого работаете?

Имелось в виду, "какому бригадиру вы подчинены в настоящий момент?".

С самого начала, Евгений Васильевич мне понравился и ни разу меня не
разочаровал. Никакого начальственного высокомерия, справедлив, уважителен и
внимателен к подчинённым, грамотнейший специалист, решителен и быстр в
решениях, умён, при необходимости - строг. Даже - не побоюсь этого слова –
талантлив, несомненно у Лобановских это семейная черта. Недаром, он потом стал
главным инженером, а затем и директором института.

Работа, коллектив, начальник – мне всё нравилось. Никаких дрязг и серьёзных
конфликтов не припомню. Корпоративки (тогда это называлось «праздничные
вечера»), стенгазета, выезды на сельхозработы, субботники, уборка территории
института и микрорайона, добровольно-принудительное членство в ДОСААФ и в
добровольной народной дружине (помощь милиции), долгие перекуры с болтовнёй и
анекдотами (когда мало работы), коллективное участие в "Спортлото" с порученным
мне вычислением и прогнозированием цифр, художественная самодеятельность, в
которой не гнушался принимать участие и наш уважаемый начальник. Помнится, мы
изображали футбольную команду, эдакий прототип киевского «Динамо», а Евгений
Васильевич – тренера, т.е. своего брата. Он же предложил дать нашей команде
название «Дымок», в котором удачно сочетались буква «Д» и дым из котельной. Эта
команда и запечатлена на прикреплённой фотографии. Мне, благодаря высокому
росту, досталась роль вратаря – тогда Е.В. Рудакова – но меня на этом снимке
нет, т.к. я фотографировал.

Кроме работы за кульманом, изредка были командировки на объекты, для проведения
обмеров, либо для согласований и корректировок проектов, поскольку вдалеке от
объекта трудно предусмотреть всё, до мелочей.

Атмосфера в коллективе во многом зависит от руководителя, и Лобановский умел
поддерживать её на должном уровне. Держал нас в курсе футбольных событий,
приносил в отдел соответствующие фотоснимки, а я, как фотолюбитель, делал с них
репродукции и раздавал сослуживцам. Естественно, все сотрудники, самым
искренним образом, болели за «Динамо» и даже я, ранее равнодушный к футболу.

Одним из фактов, характеризующих почитание сотрудниками Евгения Васильевича,
было создание шуточной картины «Теплотехники пишут письмо Лобановскому». За
основу взяли известную картину И. Репина «Запорожцы пишут письмо турецкому
султану», где лица козаков заменили на лица сотрудников. Я этого коллажа не
видел, но коллеги о нём иногда вспоминали и рассказывали. Лобановский был тогда
2 года на Кубе, в рамках помощи кубинским товарищам (это было до того, как я
попал в отдел). СССР посылал туда специалистов, а наш институт был задействован
в этой программе. О такой командировке мечтали все сотрудники, т.к. там люди
неплохо зарабатывали и, практически каждый, возвращаясь с Кубы, вёз с собой
новенький автомобиль «ГАЗ-24 Волга». В середине 1970-х эта программа была
закрыта.

В общем, все эти воспоминания эмоционально положительны. Работал я с увлечением
и был на хорошем счету. В своём воображении, я всегда старался представлять
воплощение моих чертежей в металле.

Когда нам поставили другого начальника, атмосфера в отделе понемногу стала
меняться. Заканчивался мой 3-летний срок работы по распределению, и я подал
заявление об увольнении по собственному желанию. Реакцией администрации был
удивлён: мне предложили должность старшего инженера и повышение месячного
оклада, со 100 сразу до 140 рублей! Возможно, Евгений Васильевич приложил руку…
Однако, я уже принял решение и, одержимый увлечением фотографией, пошёл
работать фотографом на киностудию хроникально-документальных фильмов, на оклад
120р., о чём и сейчас не жалею. Но это уже совсем другая история.

2020, (с)Айнцвайдрайченко

\ii{03_01_2022.fb.fb_group.story_kiev_ua.1.ev_lobanovskij.cmt}
