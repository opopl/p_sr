% vim: keymap=russian-jcukenwin
%%beginhead 
 
%%file 26_10_2020.fb.fb_group.mariupol.nekropol.1.kem_byl_konstantin_eberling.cmt
%%parent 26_10_2020.fb.fb_group.mariupol.nekropol.1.kem_byl_konstantin_eberling
 
%%url 
 
%%author_id 
%%date 
 
%%tags 
%%title 
 
%%endhead 

\qqSecCmt

\iusr{Сергей Штамбур}

Признаюсь честно, по немцам проживающим в Мариуполе, ( кроме известных всем
фамилий ) у меня сведений нет. Собрал данные почти по всем колониям
Мариупольского уезда. И дочерним колониям в Области Войска Донского. Конечно,
сведения не полные и не точные ( из разных источников ). Постараюсь в течении
недели прошерстить информацию.

\begin{itemize} % {
\iusr{Andrei Marusov}

о! будет просто здорово! ибо ну явно это наши приазовские немцы переехали из сельской колонии в город...

\iusr{Сергей Штамбур}
\textbf{Андрей Марусов} Очень возможно. Но если приехали напрямую из Германии, или Австрии, и основали предприятие - тоже НАШИ.

\iusr{Andrei Marusov}
\textbf{Сергей Штамбур} да, конечно... просто ж было старожильческое население - насколько долго бок о бок жили... скажем, c XIX века... ну, мы с тобой тогда говорили, что немцев-старожильцев при Сталине извели под корень почти полностью 🙁

\iusr{Сергей Штамбур}
\textbf{Андрей Марусов} Да, лишь немногим удалось затеряться на гишанстких стойках социализма.
\end{itemize} % }

\iusr{Виктор Евгеньевич}

Ну есть ещё один потомок немецкого колониста из Петербурга. Правда умер в 1941 м в младенчестве((

\begin{itemize} % {
\iusr{Andrei Marusov}

кто? где похоронен в Некрополе?

\iusr{Виктор Евгеньевич}
\textbf{Андрей Марусов} да, в некрополе. Умер из за болячки осенью 1941 го. С медикаментами напряг был вот и не смогли спасти мвлыша

\iusr{Andrei Marusov}

тот же Валентин Мантель прожил всего три года (1887-1890)... Но у его отца Фердинанда из Шлиссельбурга наверняка были еще дети...

\iusr{Andrii Poltavskyi}

Есть одно немецкое кладбище, не доезжая Розовки. Очень сильно заросло.. Не смог пролезть сквозь кусты лес

\iusr{Andrei Marusov}
\textbf{Andrii Poltavskyi} ну, в Новокрасновке, кажется, есть.. никогда там не был, только проездом...

\iusr{Andrii Poltavskyi}
\textbf{Андрей Марусов} Там даже местные не лазят,

\iusr{Andrii Poltavskyi}

Да, очень сильно заросло. Весной не смог пролезть. Листва спадет, попробую
\end{itemize} % }

\iusr{Кристина Лернер}

А как же мы \enquote{Лернер}? Могила деда есть на старом кладбище. Я сама её найти не смогла. знаю что умер в 71 году.

\begin{itemize} % {
\iusr{Сергей Штамбур}
\textbf{Кристина Лернер} Мой дед Штамбург то же похоронен в Мариуполе.тНо родился в Екатерингоф Верхнеднепровского уезда.

\iusr{Andrei Marusov}

а хотя бы приблизительно - где находилась могила деда? ибо если это 1971 год,
то она может быть где угодно, старые, дореволюционные захоронения более-менее
локализированы, а вот советские, особенно под закрытие кладбища (закрыли в
1972), могут быть везде...

\begin{itemize} % {
\iusr{Кристина Лернер}
\textbf{Андрей Марусов} без понятия. я родилась спустя 10 лет. Поэтому даже не могу сказать
\end{itemize} % }

\iusr{Andrii Poltavskyi}

Кристина, у нас с Андреем есть одна очень Уважаемая знакомая, которая искала бабушку1, 5 года. Она её нашла, но какой титанический объем работы она сделала!!!

\begin{itemize} % {
\iusr{Кристина Лернер}
\textbf{Andrii Poltavskyi} я не говорю ,что кто-то обязан искать.Но если вдруг найдётся случайно(что мало вероятно)я буду рада.

\iusr{Andrii Poltavskyi}
\textbf{Кристина Лернер} Кристина в С М мы все обсуждали знаковые места города.Никрополь как-то стороной обходили.Настоятельно рекомендую,если у Вас будет время посетите экскурсию.!!!

\iusr{Кристина Лернер}
\textbf{Andrii Poltavskyi} я бы с удовольствием. Но я не живу в Мариуполе. Бываю один раз в год в городе

\iusr{Andrii Poltavskyi}
\textbf{Кристина Лернер} все ок!!

\end{itemize} % }

\end{itemize} % }

\iusr{Сергей Штамбур}

К сожалению, в моем архиве нет фамилии Эберлинг (Eberling). В связи с этим
вопрос- просьба к Helga Buzlami поделиться источником информации.

\iusr{Сергей Штамбур}

Эберлинг Иван Адамович (1896) род. хутор Бахмутск Мар. уезда.\par
Эберлинг Август Адамович (1918) род с. Балобановка\par
Эберлинг Эмануил Адамович (1914) с. Балобановка\par
Эберлинг Александр Адамович (1899) с. Балобановка.\par

\iusr{Сергей Штамбур}

В 1937-работники колхоза им. Тельмана с. Ново-Орловка. Арестованы и приговорены к расстрелу..

\iusr{Сергей Штамбур}

Так же : ЕБЕРЛІНГ Іван Якович 1907 с. Балабанівка Мілерівського р-ну Ростовської обл.

ЕБЕРЛІНГ Карл Якович 1903

\iusr{Сергей Штамбур}

заарештовані с. Мало-Орлівка Чистяківського р-ну Сталінської (Донецької) обл. та засуджені до рострілу в 1938 р.

\iusr{Герман Гиллер}

Может быть у кого нибуть есть информация о госхозе \enquote{Корнталь} в Каховке?
Интересует село Богдановка. 42-43г. Огромная благодарность!

\iusr{Марина Кохан}

Очень интересно, не находили ли захоронение Якуб Гербст тоже немец, (Украина,
Мариупольский окр., Буденовский р-н, хут. Бливерница) это мой прапрадедушка. Или
Александр Гербст 1898 г.р

\iusr{Архі-Місто}

нет, не находили... А кода он умер? Точно ли похоронен в Некрополе? И где - хотя бы приблизительно?
