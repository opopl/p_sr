% vim: keymap=russian-jcukenwin
%%beginhead 
 
%%file 31_01_2022.fb.zaborin_dmitrij.1.jedinaja_nacia
%%parent 31_01_2022
 
%%url https://www.facebook.com/d.zaborin/posts/5082884558398833
 
%%author_id zaborin_dmitrij
%%date 
 
%%tags edinstvo,nacia,nacidea,nacionalizm,obschestvo,ukraina
%%title А куда подевалась единая нация в дни тревожных ожиданий?
 
%%endhead 
 
\subsection{А куда подевалась единая нация в дни тревожных ожиданий?}
\label{sec:31_01_2022.fb.zaborin_dmitrij.1.jedinaja_nacia}
 
\Purl{https://www.facebook.com/d.zaborin/posts/5082884558398833}
\ifcmt
 author_begin
   author_id zaborin_dmitrij
 author_end
\fi

Что меня действительно интересует в связи с невнятной военной истерией, так это
куда подевалась новорожденная нация? Все Это начиналось восемь лет назад как
процесс \enquote{націєтворення}, ведь формируется данное явление, как сказал
один умный человек, либо в процессе войны, либо в процессе созидания. Созидание
представлялось слишком сложным и скучным, а вот война отлично ложилась в
интересы западных партнеров, совпадая с текущей целью галицких идеологов -
полная зачистка старого мира с лепкой новой державности. 

Все последующие убийства и создание из них культа героев, поливших своей кровью
корни древа Святой Ненависти, должны были подтвердить, что нация родилась. В
огне и пепле войны, сопротивляясь агрессии огромного монстра, на передовом
рубеже Европы, защищая западные ценности, отрекаясь от мрака в пользу света и
прочее в таком духе. 

Нам постоянно предъявляли фото новорожденной в виде воспрявшей мовы (без
которой, как известно, тоже нет нации), разного рода неумерших киборгов,
торжества декоммунизации, могуче-бородатых ветеранов, которым на протезы
собирали крышечки от бутылок, прекраснодушных волонтеров, светлооких
активистов, ставших грудью на пути реванша, готовых насиловать, вешать и жечь
ради торжества Добра и Любви. Ужасы прошлого с расстрелами, голодом и
несправедливостью не должны повториться в новом мире. Просто пока мы не там -
придется кое-кого расстрелять и кое-кому заткнуть рты. Но это Другое. А пока
послушайте, как дети поют \enquote{Батько наш Бандера}.

Но вот стоило пару месяцев повыходить каким-то стремным новостям, как нация
куда-то делась. 

Как-то не ощущаю тугой волны единения, когда один вдруг ни с того, ни с сего
крикнет где-то на базаре \enquote{слава Украине}, а миллионы подхватят, бросив авоськи
с кабачками ради вскинутых в небо сжатых кулаков. Не наблюдаю корзинок с
печеньем для воинов на передовой в супермаркетах. Перила никто не красит.
Ажиотажа со сбором у военкоматов и возвращением добробатов нет. Наоборот, то и
дело из новостей узнаешь: тот ветеран на заработках в Польше, этот уехал. Герои
Майдана вон в России всплывают, это как вообще?

Здесь же от соли земли нашей, белой кости, патритической интеллигенции, все
какие-то панические разговоры про бегство во Львов, куда враг забоится совать
своё суконное рыло: даже за песни на улицах бьют. А за вторжение бригадных
тактических групп вообще прям не знаю что сделают. Только там не нация, а
неизменно пассионарные львивцы. Когда давать отпор собираются партизанские
отряды, это не \enquote{нация воюет}. И когда Британия обещает помочь - тоже. 

Ведь другой умный человек, французский историк Эрнест Ренан высказался куда
точнее, определив нацию как \enquote{согласие и желание жить совместной жизнью}. Не
мова, не вира и не рубашка с вышивкой, а субъективное ощущение членов нации
своей принадлежности к одному сообществу определяет её. Если я не хочу иметь
ничего общего с Остапом Дроздовым - у нас не нация. Даже если напялю
вышиваночку. 

Тут, говорит Ренан, идет ежедневное голосование. И большинство уже
проголосовало против. Война не помогла.

Пометьте там себе.

\ii{31_01_2022.fb.zaborin_dmitrij.1.jedinaja_nacia.cmt}
