% vim: keymap=russian-jcukenwin
%%beginhead 
 
%%file 28_02_2023.fb.dedova_natalia.mariupol.1.avtobusy_koridor_evakuacia.cmt
%%parent 28_02_2023.fb.dedova_natalia.mariupol.1.avtobusy_koridor_evakuacia
 
%%url 
 
%%author_id 
%%date 
 
%%tags 
%%title 
 
%%endhead 

\qqSecCmt

\iusr{Olena Ulanova}

Тони гуманітарки на Комунальнику? Чергова брехня від команди мера? Я не вірю.
Так просто зараз все списати на обстріли та все згоріло. В мера Дніпра чи кого
там навчилися?

\begin{itemize} % {
\iusr{Натали Кина}
\textbf{Olena Ulanova} 

Якщо навіть так, то якесь дивне \enquote{стратегічне} рішення - знаючи про бомбування,
звести все в одне місце, не в бомбосховища (якщо їх було стільки, скільки він
казав).

\iusr{Микола Шевченко}
\textbf{Olena Ulanova} ага, і все у Комунальник... Там такі бої йшли, там все було трупами завалене... Що там зберігати можна було?

\end{itemize} % }

\iusr{Olena Ulanova}

За водія чула. Людина до останнього возила воду на Бахчик. Казали якась тварь
кинула в машину щось, вона загорілась, водій загинув, світла йому пам'ять.😰

\begin{itemize} % {
\iusr{Ирина Куцан}
\textbf{Olena Ulanova} 💯 

інформація, ми жили на Бахчике під час блокади міста до 22 березня. Дійсно
водій кожного дня привозив воду... Води було мало, людей дуже багато, всім не
вистачало, були великі черги, але ж він робив, що в його силах. Одного дня його
не стало, він так і загинув у своєму автомобілі в метрах 50 від того місця,
куди привозив воду ... Він справжній герой і мужній Чоловік.

Потім ми ходили по воду до гуртожитку, який був неподалік. Люди шикувалися в
великі черги, спочатку воду давали студентам ...там було дуже багато дітей. Це
навіть страшно уявити, що відчували їх батьки, адже не могли забрати їх з цього
пекла. Нам пощастило під страшними обстрілами набрати там води, простоявши на
морозі години 3 і поспілкуватися з дітьми і вахтером. Так ось ця жінка теж
справжній герой. Вона не залишала свого робочого місця і своїх студентів. Вона
сказала такі слова, я їм останнє віддаю, а вони мені.

Маріупольці Ви справжні герої.

\iusr{Svetlana Marflyuk}
\textbf{Ирина Куцан} 

щодо батьків, чиї діти лишилися у Маріуполі: ми виїхали 19 березня, повезли людей
в Мангуш, там мені зустрілася жіночка, яка намагалася заїхати в Маріуполь, бо саме
в тому гутрожитку, про який ви розповідаєте, був її син. Чи вдалось їй знайти
сина? тобто батьки намагались забрати дітей, думаю, що та мама дісталася до
Маріуполя, вона була дуже наполегливою.

\end{itemize} % }

\iusr{Evgenia Salnikova}

А коли і де був обстріл складу
\enquote{ЯМаріуполь} з теплим одягом для дітей?

\begin{itemize} % {
\iusr{Inna Mijlkovskai}
\textbf{Евгения Сальникова} писали прилёт был на склад, может ошибаюсь Одесса
\end{itemize} % }

\iusr{Татьяна Комиссарова}

Давайте жить дальше, а не постоянно вспоминать этот ад.

\begin{itemize} % {
\iusr{Yuliia Mahda}
\textbf{Татьяна Комиссарова} 

это не вспоминание ада, это реконструкция событий. Многие про что-то узнают
впервые, кто-то делится тем, что мало кто знал. Врядли через год кто-то
вспоминать так будет. Но Мариуполь слишком быстро забыли, при том что сами
мариупольцы по сравнению с жителями Днепра, Запорожья и Киевской области за год
кукиш с маслом получили, просто приравняли ко всем переселенцам с 2 тыс грн

\iusr{Danylo Rebro}
\textbf{Татьяна Комиссарова} 

ви можете що завгодно робити. Особисто ви. Але немаєте ніякого права вказувати
людям забути геноцид. Це навіть не смішно. Всі спогади повинно зібрати і треба
судити керівництво РФ.

\iusr{Натали Кина}
\textbf{Тесса Виконтесса} Ви так легко про це кажете, ніби це забути колишнього парня, а не безпрецедентний геноцид 21 століття в центрі Європи!

\iusr{Olga Ivanova}
\textbf{Тесса Виконтесса} 

а всех, кто не выжил в этом аду, кого уби*л русский мир тоже прикажете забыть?
С вашей стороны это звучит ужасно. Прекрасно если вас обошло стороной, но это
не просто \enquote{что то плохое} чтоб вот так взять и забыть и жить дальше как вы
выразились... У нас отобрали жизнь. У нас забрали любимых. Думайте когда пишите
публичные комментарии.

\iusr{Tania Tainna}
\textbf{Натали Кина} она переименовалась быстро что ли ....?

\iusr{Shvedkina Oksana}
\textbf{Tania Tainna} мимикрировала

\iusr{Iryna Iryna}
\textbf{Yuliia Mahda} 

именно, и наша городская \enquote{власть}, ни мариупольские депутаты в раде не
пошевелились, чтобы инициировать отдельный особый статус для мариупольцев. И
нам надо платить, явно, не 2000 грн.

\iusr{Натали Кина}
\textbf{Tania Tainna} Я встигла побачити обидва імені)) так, мабуть

\iusr{Микола Шевченко}
\textbf{Iryna Iryna} зате піаряться так, наче в них вибори на носу, а не суд мають бути.

\iusr{Svetlana Marflyuk}
\textbf{Danylo Rebro} 

звісно, що треба судити і все таке... але мені сьогодні, завтра і післязавтра не
стане легше жити від того, що якесь умовне пуйло сяде на нари колись. Може краще
почати з того, щоб полегшити наше життя якось просто зараз? бо даже ті нещасні 2
тис-то не для нас, вони плавно перетікають з наших карток до рук тих, хто нам
здає житло в оренду, тобто до тих, в кого і так все добре, а буде ще краще з тими
грошима. А ми як були з нічим, так і залишилися.

\iusr{Yuliia Mahda}
\textbf{Iryna Iryna}

абсолютно согласна. В январе познакомилась с переселенцами из Купянска. Город
деокупирован. Не знаю, есть ли выплаты, но за 4 месяцы, как они выехали, они и
бизнес вывезли в Киев, и 4 раза домой проведать собачку ездили. У меня стоял
ком в горле, я не сказала, что уже год скоро не дома. Может, петиция нужна?
Чтоб конкретно выделили нас, мариупольцев, а не всех подряд.

\iusr{Yuliia Mahda}
\textbf{Svetlana Marflyukі} 

знаєте, що такі кажуть з приводу того, що в них житло? \enquote{Ми заслужили}, тобто
мати житло. А ми? Ми теж заслужили, будь Маріуполь ближче до центру чи заходу
країні і досі був би живий і цілий.

\iusr{Svetlana Marflyuk}
\textbf{Yuliia Mahda} 

в кого заслужили? власне я своє житло заробила, а не заслужила.. а вони
служать, як слуги комусь... кріпаки за своєю сутністю

\iusr{Svetlana Marflyuk}
\textbf{Yuliia Mahda} 

у нас тоже есть такие, которые туда-сюда ездили... я знаю одного, который трижды
ездил в Мариуполь, потом из Мариуполя через рашу в ЕС, потом оттуда через
Запорожье снова в Мариуполь, и там и остался, бизнес какой-то начал с
какцабами.. не все, далеко не все, так как мы выехали и скитаемся, как цыгане.

\end{itemize} % }

\iusr{Yuliia Mahda}

Ще 24-25.02 на 0629 був запит від громадськості чи якоїсь ГО, щоб хтось взявся
за евакуацію, навіть апплікаційна форма була. Потім окрема стаття, хто буде цим
займатися (тобто людина, що подалася). Я пишу цій людині в кінці 25.02 і
дізнаюся, що вона вже виїхала, чоловік вивіз, але нащо було анонсуватися на
0629 - загадка. Потім, мабуть, я пишу, Наташа, під твоїм постом про поїзди,
будуть ли ще потяги. Десь так пригадується. Як ми повинні були дізнатися про
евакуацію 5.03, не зрозуміло - ні інтернета, ні моб.зв'язку

\begin{itemize} % {
\iusr{Анна Мурликіна}
\textbf{Yuliia Mahda} 

я була на зв'язку з Вершиніним. Він обіцяв їздити по місту і кричати в
гучномовець. У декого з маріупольців був зв'язок. Дуже поганий, але інколи
вдавалось прорватись. Далі - з вуст у вуста. Проте нема про що було
попереджати. Бо 5 березня росіяни не відкрили коридор. Навпаки. Вони почали
обстрілювати маршрут руху колони - той самий маршрут, який запросили у
української сторони напередодні. Ось так було. Це була російська провокація.
Автобуси знищили не всі. Декілька залишилось. Але і вони не змогли виїхати -
нема куди було їхати. Росіяни не відкрили коридор

\iusr{Oksana Klimenko}
\textbf{Анна Мурликіна} 

в гучномовець казали, їздили по місту .. але це про відбій евакуації, я лічно
чула .. а про початок почули із уст в уста, так і було !..

\iusr{Yuliia Mahda}
\textbf{Анна Мурликіна}

я була в центрі міста, неподалік, нічого не було чути. Може, й їздили. Єдиний
раз чутно було гучномовець після 20.03 чи вже 25.03 від днрівців про гум
допомогу на 17 мкрн

\iusr{Микола Шевченко}

В тих, хто виїхав рано, ніякого розуміння не було, що в місті відбувається, що
зв'язку нема, повідомлень нема, ніяких майже з вуст у вуста нема, бо люди по
підвалах сидять, вулиці прострілюються, літаки бомблять, гради сиплять, одиниці
бігали з одного району в інший, електрики нема, тож марафон цей грьобаний, який
собою всі канали інформації замінив, ніхто в Маріуполі не бачить і не чує.
Бісить, що всі причетні роблять вигляд, наче не розуміють, що росіянам ті
коридори до одного місця, особливо від української влади. А люди в ізоляції
навіть один будинок від іншого, не кажучи вже про цілі райони

\iusr{Раиса Ильинична}
\textbf{Оксана Клименко} 

Який гучномовець ні чого чути не було для кого він казав по якому місту він
їздив не було чути ні сірен ні чого хоча я жила на Куїнджи72, тільки одні
обстріли.

\iusr{Oksana Klimenko}
\textbf{Раиса Ильинична} 

мабуть на лівому теж не було, я з лівого .. а я кажу про те, що я чула і не
одну машину чула .. в моєму районі, де була під час війни дом бита відносно на
той час ще спокійно було мабуть тому чули, багато людей шли пешком з багажами
без ., це я бачила, я їхала на машині за рулем повертаю голову на один бік
гради торчали .. на інший бік трупи лежали .. жах вже був по всьому місту, але
приморський відносно спокійно якщо порівнювати мабуть .. але їздили казали
допомогали, кажу, тільки, що сама бачила на свої очі .. спілкувалась питала
, коли кінець цьому аду, вони нажаль тільки очі відводили, відповіді не будо
не в кого!!!!(

\iusr{Раиса Ильинична}
\textbf{Оксана Клименко} У мене дочка з Азостальскої під заводом гатили трощили все. Дома немає, і наш теж потрощили Куінджи.

\iusr{Oksana Klimenko}
\textbf{Раиса Ильинична} 

так мого тєж немає .. Я на манежі Азовсталь жила , такий гарний район був
...але ми живі дякую Богові кожного дня , по усьому світу , і мабуть ,
розуміємо смисл життя по іншому вже .. я зараз в Європі і часто кажуть мені
европейці , як так в тебе що не спитай все гуд .. роботою задоволена , будинком
так хозяйка і та найкраща , але як так можливо .. це те що чую часто , кажу ,
тут не стріляють , і жити можно і тому і радію кожної секунди життя будь то
кава будь то море будь що , тому що знаю , що і цього може і не бути !!!...

\iusr{Раиса Ильинична}

Головне що не стріляють, спасибі людям що прихистили нас, дай Боже їм здоров'я.

\end{itemize} % }

\iusr{Марина Шарпай}

Бойченко брехун 😡

\iusr{Натали Кина}

Про сповіщення та про те, як дізнатись про \enquote{оголошення}. Я про коридор знала по
чуткам. І дивом була додзвонилась до доньки та колеги, вони сказали, що 6 буде
коридор, про це оголосили на ТБ. Йти туди пішки мені не проблема, але як дійти
батькам, донести тварин? Поки думала, дізналась, що коридор не відкрили. І так
потім майже кожного дня. Я підійшла до патрульного поліцейського, коли
зустріла, спитала: А як дізнатися про евакуацію? Він відповів: 1) чекайте
тільки офіційного оголошення, тоді можна, без цього ніні, сидіть де сидите; 2)
як воно буде, ми оголосимо. Я спитала ЯК?? Як ви оголосите, щоб всі почули? Він
сказав, що поліція буде їздити з гучномовцем по всьому місту. Щоб всі почули.
Якби я чекала на це, то точно б не виїхала. Оголошення через ТБ та інтернет в
місті без світла та зв'язку це ні про що. Так і поїхали на свій ризик.

\iusr{Elena Elena}

Вадіка в смітник

\begin{itemize} % {
\iusr{Раиса Ильинична}
\textbf{Елена Елена} Він ще буде керувати Маріуполем.

\iusr{Elena Elena}
\textbf{Раиса Ильиничная}, ні, ні, ні

\iusr{Раиса Ильинична}
\textbf{Елена Елена} Вы так уверены.

\iusr{Elena Elena}
\textbf{Раиса Ильинична} , чомусь впенена
\end{itemize} % }

\iusr{Ольга Єсипчук}

Чесно кажучи, мені не зрозуміла концепція \enquote{ЯМаріуполь}, Сєвєродонецьк, Бахмут
і т.д мають ту ж історію, міст немає практично, але чому ці центри не працюють
для їх теж?

\begin{itemize} % {
\iusr{Таня Лебедкина}
\textbf{Ольга Єсипчук} тому що, ці центри відкрила Маріупольська міська рада. Чому немає таких центрів для ваших міст, Вам треба запитати, у міських голів цих міст.

\iusr{Iryna Iryna}
\textbf{Таня Лебедкина} а те, що не всі маріупольці мають доступ до тої допомоги, це ви як поясните?

\iusr{Таня Лебедкина}
\textbf{Iryna Iryna} я повинна Вам щось пояснювати? Чи може ви зателефонуєте на колл-центр Маріупольської міської ради ( номер можна подивитись в інтернеті) і запитаєте у них? Хоча, там більше питань, чим відповідей, але може Вам пощастить.

\iusr{Iryna Iryna}
\textbf{Таня Лебедкина} так ви ж взялися пояснювати про Бахмут

\iusr{Таня Лебедкина}
\textbf{Iryna Iryna} 

при чому коментар, був дан по містах, які не всі знаходяться навіть у Донецькій
області, і у ж точно не відносяться к Маріупольському району. А з приводу
роботи центрів для маріупольців, то я теж можу задати багато питань і ще не
одне з них не отримала нормальної відповіді від керівництва цих центрів.

\iusr{Таня Лебедкина}
\textbf{Iryna Iryna} так при чому Бахмут к маріупольским центрам?

\iusr{Iryna Iryna}
\textbf{Таня Лебедкина} який сенс такої допомоги не для всіх?
Я теж була платником податків у Маріуполі.
Але читати пихаті коменти представників центрів перестала давно.
Це така собі допомога, як кажуть, щоб було.

\iusr{Таня Лебедкина}
\textbf{Iryna Iryna} 

тут я з Вами згодна, і з цього приводу. Але, я не можу відповісти на Ваше
питання про сенс, тому що по перше сама його ставила, прям у центрі, людині яка
їм керує, але крім хамства нічого не отримала у відповідь, по друге, я не є
представником центра.

\iusr{Iryna Iryna}
\textbf{Таня Лебедкина} ось і чудово, гадаю це і для вас, і для мене приклад того, що ми маємо моральну шкоду. І будемо подавати в суди, тому зо рівного доступу до допомоги немає, механізм не пропонувався. Лише хамство.

\iusr{Tetyana Dzhurik Levitska}
\textbf{Ольга Єсипчук} є Я Луганщина, там дуже гарна допомога. Не так часто дають, як маріупольцям, але набагато кращу

\iusr{Anhen Maria}
\textbf{Ольга Єсипчук} А як в Бахмуті де ідуть запекли бої буде працювати центр?🤔

\iusr{Anhen Maria}
\textbf{Ольга Єсипчук} В Северодонецькі Я Маріуполь? Це ж Луганська область.

\iusr{Victoria Rohozhyna}
\textbf{Anhen Maria} 

так в Маріуполі він також не працює, якесь дивне, щоб не сказати дурне
запитання. Центри \enquote{яМаріуполь} працюють по регіонах України в тих містах, де
найбільші маріупольські громади (земляцтва)

\iusr{Svetlana Marflyuk}
\textbf{Tetyana Dzhurik Levitska} а хто дає і де? в мене сусідка з Луганщини, вона нічого ніде не отримує, розкажіть, я їй передам

\iusr{Svetlana Marflyuk}
\textbf{Anhen Maria} 

по-перше, на окупованих територіях звертайтесь про допомог до окупаційної
влади, по-друге, ЯМаріуполь обслуговує тільки маріупольців і мабуть щей
маріупольській район-тих людей, що виїхали на підконтрольну Україні територію.

\iusr{Anhen Maria}
\textbf{Svetlana Marflyuk} 

Світик, то не до мене. Я ні в Маріуполі., а в Одесі, все знаю, то Ви мабуть мене з кимось переплутали.

\iusr{Svetlana Marflyuk}
\textbf{Anhen Maria} мабуть, тоді пардон)

\end{itemize} % }

\iusr{Kateryna Stinberg}

З 8 березня були чутки про евакуацію, про коридор.

Кожного дня чекали на новини. І лише 17 березня отримали достовірну інформацію,
що люди виїжджають та доїжджають... Щось чули по радіо, щось - від людей, трохи
доходили СМС, іноді можна було дозвонитися. Потім у наших військових дізналися
про більш-менш безпечний шлях з міста.

\begin{itemize} % {
\iusr{Юлія Соколова}
\textbf{Kateryna Stinberg} с 5 марта.

\iusr{Натали Кина}
\textbf{Kateryna Stinberg} 

15 був перший 100\% коридор. Але ми не знали, просто побачили машини, сусідка
нас взяла в свою, вагались, потім поїхали навмання.

\iusr{Svetlana Obedins}
\textbf{Натали Кина} Я виїхала 14-го. Знаю людину, яка навіть 13-го змогла дістатися Бердянстка

\iusr{Натали Кина}
\textbf{Svetlana Obedins} Маю на увазі коридор з Маріуполя до самого Запоріжжя.
\end{itemize} % }

\iusr{Наталья Ушакова}

Мені кума надсилала постійно смс про те, що домовляються про зелений коридор
для маріупольців. Так тривало кілька днів. Потім,ура, нарешті. Домовились.
Скинуто було три адреси, де людей мали забирати автобусу протягом трьох днів.
Ми жили на Нептуні. Одна з точок собору була Іллічівський ісполком. Вже був
тваринний страх, розуміння безвиході, зневірення, страх за дитину. Я кричала,
що не можу більше так, що боюся. Розуміла, що загинемо якщо не від зброї, то
від голоду точно. Кожен день біля ісполкому збиралися люди. Багато людей. З
грудними дітьми, з маленькими дітьми, з літніми людьми, з тваринами. Кожного
дня в визначений час, в визначеному місті. ДАРМА!!!!! Ні автобусів, ні будь
якої інформації щодо них та евакуації. Поки ще був хоч поганий, але зв'язок.
Кума писала, чого ви там сидите. Сідайте в автобус та виїзжайте. Вона не могла
зрозуміти та повірити, що евакуації просто не було, бо по всіх новинах
розповідали про "зелений коридор". Вже потім ми узнали, що людей вивозили
спочатку з Мангушу, а вже потім з Бердянську. Евакуації з Маріуполя не було!!!!
Люди виїзжвли, виходили пішки різними способами та свій страх і риск Нам
вдалося виїхати після того, як скинули бомбу на Нептун, де було облаштовано
щось типу видачі продуктів з вцілілих складів ( продукти це звісно гучно
сказано, в основному це було печиво,) та жіночі доктори добровольці
консультували вагітних. Наші чоловіки побігли розгрібати завали та витягувати
вагітних звідти. Спочатку нічого нам не казали, але після цього. Мій чоловік
прийшов і сказав: Збирайся, нехай краще в дорозі розстріляють, хоч якийсь є
шанс вижити". На Аістах потрапили під жорсткий обстріл. Моя дитина запала
куртку та з диким плачем вистрибувала з машини. Аж руки тремтять, коли пишу це.
В вже потім він розповів, що після прильту в басейн, там була тьиа трупів.
Останньою витягали вагітну, у якої нога була, майже не було вже цієї ноги. То
був ЖАХ!!!!!!!!!

І ніхто не змусить мене це забути. Це просто неможливо!!!!! Я дуже хочу всей
цей жах забути, та жити далі, як тут пишуть жіночки. Але це не можливо!!!!!
Коли тебе прийшли вбивати, коли ти пройшов через пекло, коли в тебе забрали
ВСЕ, навіть спогади про щасливі моменти життя( я зберігала першу одіжку, перше
взуття, перший зубик своєї дитини, фото, відео та взагалі ......,), коли ти не
живеш, а існуєш, бо просто не знаєш, як тобі далі жити, то не можливо просто
взяти і забути!!!!!!!!

\begin{itemize} % {
\iusr{Natalya Dedova}
\textbf{Наталья Ушакова} розкажіть, будь ласка, свою історію. Для проекту Голоси мирних. ❣️

\iusr{Наталья Ушакова}
\textbf{Наталя Дєдова} Не знаю, чи зможу. Вже минув рік, а воно болить так, як будто це було вчора. Зараз пишу, а сльози вже залили весь телефон)

\iusr{Tania Tainna}
\textbf{Наталья Ушакова} это и читать невозможно((

\iusr{Nadya Panfilovich}
\textbf{Наталья Ушакова} возле Аистов, какого числа было?

\iusr{Наталья Ушакова}
\textbf{Надежда Панфилович} не можу відповісти на Ваше питання. Для мене весь цей період злився в один день, який тривав цілу вічність (((

\iusr{Nadya Panfilovich}
\textbf{Наталья Ушакова} така ж проблема( ми 16 виїжджали від Аістов, Ви напевно пізніше

\iusr{Раиса Ильинична}
\textbf{Наталья Ушакова} 

У Мангуші, Володарську було тихо і спокійно, як кажуть і волосина не впала люди
жили своїм життям, а Мариуполь растрощили в шматки.

\iusr{Наталья Ушакова}
\textbf{Раиса Ильинична} 

Я знаю. Ми як вибрались з Маріуполя поїхали спочатку через Мангуш в село, де в
мого чоловіка був будинок, в 15 км від Маріуполя, бо ми взагалі не розуміли яка
ситуація в країні, куди їхати безпечно, а куди ні. Я,чесно кажучи думала, що
таке діється по всій Україні. На той час це селище було вже окуповане ДНР та
ордою Так от кожного ранку вся техніка, що там була виїзжала на трасу та до
пізнього вечора гатила по Маріуполю. Просто так...........

\iusr{Раиса Ильинична}
\textbf{Наталья Ушакова} 

Я теж думала що так б'ють не тільки Маріуполь а всюди але коли мене привезли до
Мангуші, а там тихе життя тільки гатили ракети на Маріуполь, то хтось трохи
передохнули, потім перевезли у Ялту маріупольську у морі стояли кораблі і з них
теж пуляли по Маріуполь ми це бачили. Бідному Маріуполь і людям досталось як ні
кому цого горя.

\iusr{Наталья Ушакова}
\textbf{Раиса Ильинична} 

Але про це чомусь замовчують. Я так чекала на Новорічне звернення президента.
Була вкрай розчарована. Він про Маріуполь лише \enquote{вскользь упомянул}, ні єдиного
дня не було объявлено в країні жалоби по загиблим людям в місті, якого спіткала
участь ягня на закланні. Дуже боляче це освідомлювати. Дуже!!!!!!

\iusr{Natalya Pakhomenko}
\textbf{Наталья Ушакова} 

так похоже на нашу историю. Каждый день ждали эвакуацию, ходили к горисполкому,
а вечером по новостям слышали, что автобусы не пропустили в Мариуполь. Выезжали
тоже, когда попали в Нептун, проходили мимо него буквально за 20 минут до
попадания. Когда проходили, видели много людей, возле него, в основном женщины
с детьми. Видать давали что-то, но кто-то принял глупое решение, чтобы что- то
получить на ребенка, недостаточно было документы, нужно было идти с ребенком. И
буквально через некоторое время, когда проезжали мимо, уже было все разрушено.
Первый вопрос, что с людьми, которые там стояли? Хотелось верить, что
разошлись, что успели спрятаться, но по вашему комментарию, получается, что
там погибло много. Я до этого, честно не встречала информацию о погибших там.
Хотелось верить в лучшее.

\end{itemize} % }

\iusr{Afina Khadzhynova}

Мародерити почали на початку березня.

\iusr{Nadya Panfilovich}

не пам'ятаю число, вийшло додзвониться родичам у Запоріжжі, вони сказали, що по
телебаченню наш мер сказав, що з Маріуполя від Іллічівця будуть автобуси,
поїхали туди, повна площа машин, люди пішки з усіх районів, те ж по чутках
прийшли. Приїхали військові, сказали розходитися, що як буде коридор вони по
рупору повідомлятимуть ... коли запитали, як можна самим виїхати, сказали, що
запорізька траса замінована, у бік Мелекіно знизали плечима і сказали чекати на
інформацію.

\iusr{Irina Polyushkina}

А мене окрім Бойченка бентежить ще одне питання: чому мені досі на електронну
пошту йдуть листи з Нафтогазу з нагадуванням про борги за період з початку
війни та проханням їх сплатити? Хотілося б знати, чи я одна така і що з цим
робити.

\begin{itemize} % {
\iusr{Yana Lupinova}
\textbf{Irina Polyushkina} пусть приедут отключат

\iusr{Irina Polyushkina}
\textbf{Яна Лупинова} Вже відключали газ, а оплату - ні.

\iusr{Наталья Ушакова}
\textbf{Ирина Полюшкина} Ні, я маю борги за світло)

\iusr{Вікторія Крель}
\textbf{Irina Polyushkina} Вы не одна така. У нас з 17 березня будинка немає, а листи приходять на пошту і досі. Хай їдуть відключають.

\iusr{Natali Korovanenkova}
\textbf{Ирина Полюшкина} ні, ви не одна, теж ...

\iusr{Irina Polyushkina}
\textbf{Вікторія Крель} З цим треба щось робити, щоб скасувати борг.

\iusr{Iryna Iryna}
\textbf{Irina Polyushkina} а тому що міська рада Маріуполя, її виконком засідає, але ухвалити звернення до відповідних постачальних компаній часу не знаходить.

\iusr{Яна Граборова}
\textbf{Iryna Iryna} , 

а яким чином міська рада може зробити звернення до приватних компаній?
Маріуполь півмільйоник. В середньому за газ близько 1-3 тис за зимовий місяць
повинні сплатити з кожної квартири. То чому газова компанія має втрачати
гроші?в лютому ж газом користувались і на опалення в тому числі.

\iusr{Elena Katkova}
\textbf{Ирина Полюшкина} тоже самое, долг висит за февраль. Как быть тем, кто вообще не платил годами)) пощастило

\iusr{Iryna Iryna}
\textbf{Яна Граборова} 

таким чином, як і ми, чого маємо втрачати житло і майно? Але я про інше. Мені
досі приходять листи не про борги, а про те, що треба передати показання
лічильника✌️

Виконком мав би давно написати звернення, що діє військовий стан, місто
окуповане і т.д

\iusr{Irina Polyushkina}
\textbf{Яна Граборова} 

Мій чоловік в перші дні війни все оплатив через Приват 24. Ми таке пережили,
залишились без житла і ще повинні сплачувати за послугу, яку не отримували?

\iusr{Яна Граборова}
\textbf{Ирина Полюшкина} 

я теж з Маріуполя. Теж сплатили всі послуги. За лютий. То які питання?
Найчастіше пишуть чому ми повинні за лютий платити

І якщо що то я теж втратила все майно і бізнес.

\iusr{Irina Polyushkina}
\textbf{Яна Граборова} До Вас ніяких питань.

\iusr{Наталья Бабич}
\textbf{Яна Граборова} за лютий мер скасував сплату. була об'ява ще коли зв'язок був. сьогодні в стрічці бачила у когось у спогадах

\iusr{Яна Граборова}
\textbf{Наталья Бабич}, 

комунальні ( тобто ті підприємства що на балансі міста)

Ясно, Нафтогаз - це приватні компанії. Мер не може скасувати ці платежі. Я теж
спочатку обурювалися, поки не розібралась

\iusr{Наталья Бабич}
\textbf{Яна Граборова} я не обурююсь Ян, я кажу що сьогодні прочитала. значить минулого року то був піар виходить з його боку. хз

\iusr{Наталья Бабич}
\textbf{Яна Граборова}

\ifcmt
  igc https://scontent-fra3-1.xx.fbcdn.net/v/t39.30808-6/334002982_5883026808442102_5576638815123899382_n.jpg?_nc_cat=105&ccb=1-7&_nc_sid=dbeb18&_nc_ohc=Q0uBHJJ1JBsAX9CenyN&_nc_ht=scontent-fra3-1.xx&oh=00_AfDghr66kwNCYkHbnSW5O4EfgQxUFzE_3bL2GxZ5uH5iqA&oe=64372C6D
	@width 0.5
\fi

\iusr{Наталья Бабич}

Диви

\iusr{Svetlana Marflyuk}
\textbf{Irina Polyushkina} не тільки вам і не тільки за газ) мені ще й Ясно прислало рахунок)

\iusr{Євдокія Фрушичева}
\textbf{Irina Polyushkina} спочатку побачила нагадування про борг, а потім прочитала ваш коментар. Ви не одна.

\url{https://m.facebook.com/story.php?story_fbid=6118127314919461&id=100001667583471}

\iusr{Irina Polyushkina}
\textbf{Евдокия Фрушичева} Я вже це зрозуміла.

\end{itemize} % }

\iusr{Nadya Panfilovich}

7 березня у місті було не звично тихо (відносно), пройшла інформація, що
вивозили іноземних студентів

\begin{itemize} % {
\iusr{Юлія Соколова}
\textbf{Надежда Панфилович} 

я помню 5 марта, когда мы с левого вырвались на \#\#эвакуацию\#\#\# ( пишу, а ком к
горлу поступает и слезы).. возле Драма , сказали, чтоб расходились. Пытались
проехать на Краснофлотскую, на дачи... это была ещё бОльшая ошибка.
Заехали... сразу разбитые дома и алабай у разрушенного дома... начался обстрел и
мы не доехали, развернулись ...начали возвращаться.. а на дороге уже была
воронка ( туда мы ехали, не было). Мы смогли вырваться опять вглубь, на
Ильичевец( ещё одна точка сбора). Полиция ездила и с рупором, оповещала чтоб
все расходились... А мне бросилась в глаза, стайка иностранных студентов, с
чемоданами и в коротких куточках... они бедные вообще были в шоке. Я даже не
представляю их состояние... мы хоть в своем городе, а они в чужой стране.... и с
этими чемоданчиками на колёсиках.... жутко...

\iusr{Юлія Соколова}
\textbf{Надежда Панфилович} 7 -го марта исчез газ на Кировском..

\iusr{Nadya Panfilovich}
\textbf{Юлия Соколова} жутко было всем...

\iusr{Nadya Panfilovich}
\textbf{Юлия Соколова} 4 марта в Приморском районе газа уже не было, к удивлению газ был последним

\iusr{Євдокія Фрушичева}
\textbf{Юлія Соколова} у нас исчез 5 марта...

\iusr{Юлія Соколова}
\textbf{Евдокия Фрушичева} на левом ещё в феврале в одном районе, а в другом ещё и свет с водой был...

\end{itemize} % }

\iusr{Olena Kryvun}

Я пам'ятаю 16 березня. Наше, до того пусте місто ожило побитими машинами,
переляканими людьми. І від цього стало ще страшніше. В нашому місті вже на той
час не вистачало продуктів, хліба, медикаментів.

Ми дивилися на маріупольців і казали собі, що вони якісь \enquote{припорошені}.
Ми не могли знати, що вони пережили. \enquote{У нас немає більше аптек} чуєш і
насилу втримуєш сльози. \enquote{Я не витримаю ще одну ніч на морозі} чуєш від
легко вдягненої жінки на десятиградусному морозі і з жахом думаєш що завтра на
її місті можеш опинитися ти.

Я на власні очі бачила як відправлявся з мого міста перший гуманітарний
коридор. Це було настільки страшно. Бо стільки людського горя ти не можеш
відразу в себе умістити. В той день я відправляла з чужими людьми свою доньку і
молила бога, щоб всі добралися живі.

А ще переповнена лікарня, поранені люди і жахливі фото на їх мобільних
телефонах.

І ще. Люди не могли зрозуміти за яким часом живе наше місто, бо три колори в
ньому з'явились пізніше.

І ще згадалась історія. Сім'я з Маріуполя. У знайомих на п'ятому поверсі. Їх
собака прив'язаний на балконі. І тут у собаки починається паніка через літак
бомбардувальник. Вони тоді літали низко морем бомбити Миколаїв.

\begin{itemize} % {
\iusr{Oleksandr Poleopulo}
\textbf{Olena Kryvun} Бердянськ?

\iusr{Olena Kryvun}
\textbf{Александр Полеопуло}

Бердянськ.

\iusr{Виктория Бахалова}
\textbf{Olena Kryvun}, 

мешканцям Бердянську різкий уклін. Вони самі вже були під окупацією, відчували
дефіцит і ліків, і гігієни, але допомогали нам, маріупольцям. Надали прихисток.
Я пам'ятаю, як одна літня жіночка безкоштовно хотіла дати нам металеву каністру
під бензин (це був великий дефіцит, бо і бензин був на вагу золота). Ми все ж
таки заплатили, вона довго відмовлялася.

\iusr{Yuliia Mahda}
\textbf{Olena Kryvun}

можна ваш коментар взяти собі...? Дякую за нього❤️

\end{itemize} % }

\iusr{Лідія Муглі}

Про тони продуктів на Комунальнику - не вірю. Маріупольська влада багато чого
розповідає але ніколи не прикладає факти. Наприклад, щоб підтвердити свій
казковий міф про тони підготовлених запасів продуктів на складі треба просто
вказати на закупку у системі Прозоро. Таких закупок виконкомом на передодні
війни не було. Жодної. На передодні війни вони ще закуповували на сотні тисяч
готельні та ресторанні послуги. Так 18.02 вони замовили на 300000грн ресторанні
послуги та на 400000 грн готельні послуги, 23.02.22 на 332тис телекомунікаційні
послуги, 11.02 на 250тис. послуги агенства створення інформаційних роліків ,
17.02.22 - призи на 47000грн та інше. Ніяких продуктів виконком не купував.

Виконком - це бюджетна установа, це не приватний бізнес. Закупка такого об'єму
продуктів для тисяч громадян - це великі грошові бюджетні витрати, яки не могли
обійти систему Прозоро.

\begin{itemize} % {
\iusr{Natalya Dedova}
\textbf{Лідія Муглі} може, не тони. Але хоча б кілограми. Чи це міф? Бачила у когось відео, де завозять воду на Комунальник.

\iusr{Лідія Муглі}
\textbf{Наталя Дєдова}, 

воду? на Комунальник? Закупок води теж не бачила на Прозоро.

Це було до війни? Чи під час війни? А яка вода, бутілірованая?

\iusr{Natalya Dedova}
\textbf{Лідія Муглі} бутильована. Щось мені підказує, що про цей склад треба поговорити окремо. 😶

\iusr{Людмила Ковалевская}
\textbf{Наталя Дєдова} это не для мариупольцев, а для рашистов!
\end{itemize} % }

\iusr{Olha Lavreniuk}
\textbf{Татьяна Комиссарова} а, може, ще й промінчики добра окупантам послати?

\iusr{Irina Kozka}

Дійсно планувалася евакуація автобусами, їх заправляли на Кірова на заправці
біля \enquote{амстора} і водії підтвердили, що плануть з лівого забирати людей. Але,
нажаль, не сталося. Автобусів було багато

\iusr{Irina Kozka}

З приводу продуктів, води та ліків, також не вірю. Нічого не було приготовлено

\iusr{Світлана Добровольська}

Моя донька працювала у 4 лікарні. Вже 24 лютого зранку не було керівництва. Не
було підготовлено укриття для хворих та співробітників. Не було запасів
води... і не тільки. Діти та дорослі були підключені до апарату ШВЛ, які звичайно
перестали працювати без світла. Лікарі працювали в дуже важких умовах, загинули
колеги.

\begin{itemize} % {
\iusr{Olena Perkova}
\textbf{Світлана Добровольська} у нас в МЛ 9 теж саме! Головний лікар 24 і 27 ще був, як приїхав одразу побіг до підвалу! А ми на 5 поверсі працювали! І хворих не мали змоги куди подіти, бо вони були на кісні! Було жахливо і геть не підготовлено!

\iusr{Natalya Dedova}
\textbf{Світлана Добровольська} це пан \textbf{Владимир Жигиль} там керував? \textbf{Владимир Жигиль} \textbf{Владииир Жигиль}

\iusr{Світлана Добровольська}
\textbf{Наталя Дєдова} так.
\end{itemize} % }

\iusr{Yana Kutana}

Так, війна показала справжні обличчя людей. Хто мародерив. А хто воду возив людям.

\iusr{Екатерина Кибкало}

А чей был приказ, не выпускать людей из города, т. к на трассах опасно? Мне
кажется, это было все таки ошибкой, может, больше людей бы спаслось.... Но это
просто мои мысли, Я не военный эксперт.

\begin{itemize} % {
\iusr{Нина Муравченко}
\textbf{Екатерина Кибкало}

Вам кажется, что это было ошибкой?

А мне кажется, что это было стратегией.

Как это не противно - есть правила, ведения боевых действий.

По мирным стрелять нельзя. Надеялись, что не будут? Или прикрывались? Или
политикам нужна была "козырная карта" (жертвы), чтоб козырять потом на
саммитах?

Эвакуации небыло. И город закрыли.

Почему кто-то решил, что он вправе решать где нам лучше погибнуть.

Нас просто слили, как пешку в шахматной партии.

Когда ещё можно было перемещаться по городу, мы КАЖДЫЙ ДЕНЬ ходили к Драмтеатру
узнавать будет ли зелёный коридор.

И помню, как однажды, один военный всем кричал, что идёт подмога. К нам идут
наши, с Мелитополя....

Он врал, или ему врали?

Врали нам, говорили эвакуацию ЖДИТЕ. И не выпускали

Врали защитникам - говорили, подмога близко.

Ошибка?

\iusr{Екатерина Кибкало}
\textbf{Нина Муравченко} 

а самое для меня ужасное, но почему-то все об этом молчат, что из города начали
выезжать люди когда? Когда пошел слух, что разбили укр. блок пост и можно
пытаться выехать. В моей ситуации было именно так. 14 марта я смогла на Танке
поймать связь, Я дозвонилась подружке и она сказала, что ее знакомые вчера смогли
выехать, т. к укр. блок пост разбит, пробуйте....

\iusr{Нина Муравченко}
\textbf{Екатерина Кибкало} 

\obeycr
мы выезжали ... даже не выезжали.
Мы просто бежали, куда глаза глядят.
Мы не знали куда можно, по какой трассе..
Над нами бой прошёл 16.03
Разбомбили дом, над подвалом где мы прятались.
Нас спасло чудо.
Вернее два чуда.
1. Подвал был надёжный. Выдержал. Выжили все. А нас было три семьи с детьми.
2. Уцелел гараж, а в нём машина.
За тем, что осталось от дома стоял русский танк, и мы понимали, что у нас всего несколько минут.
Мы бежали в отчаянии.
И многие бежали вот так.
Из под завалов, из руин. Когда было непонятно доживёшь ли до завтра.
Тогда всем было не до мирных
Тогда ещё не придумали фильтрацию.
Тогда мы вырвались из ада
\restorecr

\iusr{Наталья Бабич}

это был не план, эвакуацию нужно было проводить заранее, уже выехать с 26 числа
не было возможным. обстреливали

\url{https://www.facebook.com/100033310509512/videos/763310181298439/}

\iusr{Любовь Товкач}
\textbf{Нина Муравченко} Господи...

\iusr{Iryna Iryna}
\textbf{Екатерина Кибкало} 

убивали людей. Возле поврота к Осипенко стояла желтая расстрелянная копейка,
убили людей, они ехали из Бердянска в село

\iusr{Нина Муравченко}
\textbf{Iryna Iryna} 

Убивали. По дорогам встречались сгоревшие машины

И ты понимаешь, что в этой машине осталась чья-то жизнь. Убивали в дороге,
убивали в городе... и продолжают убивать

\iusr{Natalya Dedova}
\textbf{Iryna Iryna} 

у мене є історія. 27 лютого розстріляли автівку. Чоловік 44 роки загинув.
Дружина та донька отримали поранення. Потім кадирівці розстріляли й автівку
кума, який приїхав забирати тіло. Так двох і поховали. Від кума залишилася
тільки рука.

Осипенко.

\iusr{Svetlana Marflyuk}
\textbf{Екатерина Кибкало} 

вы когда выезжали, видели расстрелянные и сгоревшие гражданские машины? они были
по всей трассе-от Мариуполя и в Запорожской обл. В бомбике намного больше шансов
уцелеть.

\iusr{Екатерина Кибкало}
\textbf{Svetlana Marflyuk} мы не в сторону Запорожья выезжали, Слава Богу, не видела.

\iusr{Svetlana Marflyuk}
\textbf{Екатерина Кибкало} 

ну конечно, зачем какцабам расстреливать машины, которые едут к ним. Еще одно
подтверждение какие они конченые твари.

\iusr{Екатерина Кибкало}
\textbf{Svetlana Marflyuk} Я не в Россию ехала.

\iusr{Iryna Iryna}
\textbf{Natalya Dedova} так. І ще й тіла посто так не віддали

\end{itemize} % }

\iusr{Людмила Иванова}

В Ровно привезли гумманитарку... позор... сам Вадик, пробовал? Вопросы нет, сразу же
охрана под ручки выводит... И он собирается ещё сессию открывать в
Мариуполе??? Ненавижу...

\begin{itemize} % {
\iusr{Любовь Товкач}
\textbf{Людмила Иванова} поддерживаю!
\end{itemize} % }
