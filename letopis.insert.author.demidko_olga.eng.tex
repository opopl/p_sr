% vim: keymap=russian-jcukenwin
%%beginhead 
 
%%file insert.author.demidko_olga.eng
%%parent insert
 
%%url 
 
%%author_id 
%%date 
 
%%tags 
%%title 
 
%%endhead 

%\par\noindent\rule{\textwidth}{0.4pt}

\ifcmt
  igc https://donbas24.news/storage/authors/1b9riuj839ggc5uf.jpeg
  @width 0.4
\fi

\href{https://donbas24.news/author/olga-demidko}{Демідко Ольга}
%\footnote{%
%\small\em Демідко Ольга

Historian, Lecturer, Excursion Guide, Public Activist of Mariupol.
%історик, викладач, екскурсовод, громадська діячка Маріуполя.

Graduate of Mariupol State University, Masters in History (2013).
%В 2013 році закінчила Маріупольський державний університет, магістр історії.

Worked as a history teacher at secondary school 66 (Mariupol), tv

Працювала вчителем історії у СШ № 66, телеведучою Маріупольського телебачення,
на посаді асистента кафедри соціальних комунікацій Маріупольського державного
університету.

Координаторка проектів \enquote{Маріуполь – це Україна}, \enquote{Малюємо
заради миру}, \enquote{Маріуполь туристичний}. З 2017 р. – голова арт-центру
\enquote{Театр всередині тебе}.  Здобула
\href{https://archive.org/details/book.2019.demidko_olga.mariupol.dissertacia}{науковий
ступінь кандидата історичних наук} (прирівнюється до диплома доктора
філософії). Наразі доцент кафедри культурології Маріупольського державного
університету, який тимчасово розташувався у Київському національному
університеті будівництва і архітектури.

Є автором 120 публікацій наукового та публіцистичного стилю та монографії
\href{https://archive.org/details/book.2017.demidko_olga.monografia.illustrovana_istoria_teatr_kultury_mariupolja}{%
\enquote{Ілюстрована історія театральної культури Маріуполя}} (2017 р.); а також
редактором і упорядником декількох маріупольських видань, спрямованих на
підтримку талановитої молоді Маріуполя: \enquote{Альманах конкурсних робіт учнівської
та студентської молоді загальноміського конкурсу \enquote{Маріуполь – це Україна}} (2016
р.); Маріупольська розмальовка \enquote{Розфарбуй місто} (2016 р.); Збірка творчих
робіт маріупольців \enquote{Мелодії незримих почуттів} (2016 р.). 

Фейсбук: \href{https://www.facebook.com/profile.php?id=100009080371413}{Olga Demidko}\par
Електронна Пошта: \textit{demidko.olga1991@gmail.com}\par
Блог: \url{https://mrpl.city/blogs/olga-demidko}\par
\linkDemidkoWorksMDU\par
\linkDemidkoWorksIA\par
\linkDemidkoWorksGoogleScholar\par
Ольга Олександрівна Демідко\par
кандидат історичних наук, доцент кафедри культурології\par
Маріупольський університет\par
Преображенська 6, місто Київ 03037\par
%\url{mailto:o.demidko@mdu.in.ua}
\textit{o.demidko@mdu.in.ua}
%}

\ifcmt
  igc https://i2.paste.pics/7975c3ec35742aa68240948c60541d13.png
	@width 0.4
\fi

%\par\noindent\rule{\textwidth}{0.4pt}

%\href{https://archive.org/search?query=creator%3A"Ольга+Демідко+(Маріуполь)"}{Список публікацій (Інтернет-Архів)}

% pereadresacia
%59 0009 8344 8146 

% file:///data/out/html/p_sr/letopis/author.demidko_olga/jnd_ht.html

% todo 
% 
% http://history.mdu.in.ua/index/kk/0-16

%https://archive.org/search?query=creator%3A"Ольга+Демідко+(Маріуполь)"

%https://archive.org/search?query=creator%3A"Оксана+Стоміна+(Маріуполь)"
%https://archive.org/search?query=creator%3A"Євген+Сосновський+(Маріуполь)"
%https://archive.org/search?query=creator%3A"Наталія+Дєдова+(Маріуполь)"
%https://archive.org/search?query=creator%3A"Надія+Сухорукова+(Маріуполь)"
%https://archive.org/search?query=creator%3A"Олена+Сугак+(Маріуполь)"

%https://archive.org/search?query=creator%3A"Кіпчарський%2C+Віктор+(Маріуполь)"

%https://archive.org/search?query=creator%3A"Маріуполь+Довоєнний"
%https://archive.org/search?query=creator%3A"Сергій+Буров+(Маріуполь)"
%https://archive.org/search?query=creator%3A"Наталія+Лоскутова+(Маріуполь)"
%https://archive.org/search?query=creator%3A"Мирослав+Кобилянський+(Маріуполь)"
%https://archive.org/search?query=creator%3A"Євген+Шишацький+(Маріуполь)"
%https://archive.org/search?query=creator%3A"В'ячеслав+Слободянський+(Маріуполь)"
%https://archive.org/search?query=creator%3A"Марія+Подибайло+(Маріуполь)"
%https://archive.org/search?query=creator%3A"Маріупольський+Державний+Університет"
%https://archive.org/search?query=creator%3A"Місто+Марії"
%https://archive.org/search?query=creator%3A"Алевтіна+Швецова+(Маріуполь)"
%https://archive.org/search?query=creator%3A"Еліна+Прокопчук+(Маріуполь)"

%https://archive.org/search?query=subject%3A"Данило+Подибайло"

%Національна академія керівних кадрів культури і мистецтв


% plan 29/06/2023
%https://donbas24.news/news/teatromaniya-prodovzuje-rozvivati-ta-pidtrimuvati-kulturu-mariupolya
%https://donbas24.news/news/trimajemos-razom-mariupolskii-teatr-conception-predstavit-novu-vistavu-foto
%https://donbas24.news/news/yaki-teatralni-projekti-ta-kulturni-zaxodi-prisvyaceni-mariupolyu-realizuyutsya-zakordonnimi-mitcyami
%https://donbas24.news/news/lyudi-yaki-zdivuvali-za-rik-viini-svoyim-dobrim-sercem-foto
%https://donbas24.news/news/u-kijevi-vidnovili-mariupolsku-vistavu
%https://donbas24.news/news/novi-projekti-teatromaniyi-20-ta-zittya-kolektivu-za-kordonom
