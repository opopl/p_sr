% vim: keymap=russian-jcukenwin
%%beginhead 
 
%%file 27_11_2020.sites.ru.zen_yandex.yz.bolshe_znaesh_menshe_spish.1.drevnjaja_rus
%%parent 27_11_2020
 
%%url https://zen.yandex.ru/media/id/5e56bd29e998506d361653ce/gde-beret-svoe-nachalo-drevniaia-rus-ili-o-chem-molchat-istoriki-5fc088c1c9a19d0e1cc1ccac
 
%%author Больше Знаешь, Меньше Спишь (Яндекс Zen)
%%author_id yz.bolshe_znaesh_menshe_spish
%%author_url 
 
%%tags russia,ancient
%%title Где берет своё начало Древняя Русь или о чем молчат историки
 
%%endhead 
 
\subsection{Где берет своё начало Древняя Русь или о чем молчат историки}
\label{sec:27_11_2020.sites.ru.zen_yandex.yz.bolshe_znaesh_menshe_spish.1.drevnjaja_rus}
\Purl{https://zen.yandex.ru/media/id/5e56bd29e998506d361653ce/gde-beret-svoe-nachalo-drevniaia-rus-ili-o-chem-molchat-istoriki-5fc088c1c9a19d0e1cc1ccac}
\ifcmt
	author_begin
   author_id yz.bolshe_znaesh_menshe_spish
	author_end
\fi

\index[rus]{Русь!История!Древняя Русь, истоки, 27.11.2020}

На просторах интернета нашел очень интересную информацию, на одном из
зарубежных форумах возник спор между людьми откуда произошли славяне. Хочу Вам
мой читатель передать основную мысль, так как на мой взгляд она довольна
информативна и интересна для общего развития.

\subsubsection{Где берет начало Древняя Русь}

Где берет начало русская история - неизвестно до сих пор. Слишком неоднозначны
находки археологов и молчаливы источники. Истоками Древней Руси можно считать
«Историю государства российского» Карамзина М. К, увидевшую свет двести лет
назад. Однако, многие из его аргументов не имеют доказательной базы и
опровергаются более молодыми экспертами.

\ifcmt
pic https://avatars.mds.yandex.net/get-zen_doc/203431/pub_5fc088c1c9a19d0e1cc1ccac_5fc09197b1f92632baa383d2/scale_1200
caption Древняя Русь Источник: Яндекс картинки
\fi

\subsubsection{Место жительства славян до начала славянской истории}

В 6 веке в Византии возникли первые упоминания о славянах, затем в Западной
Европе и на Востоке. Причем у славян настолько широкий ареал расселения, что
возникает вопрос: откуда они взялись и не прятались ли под другими названиями.

Согласно мнению византийских историков, первое имя славян звучало как «венеды»
или «анты». И римские ученые также о них упоминают. Но эксперты в области
истории до сих пор не выяснили, где они проживали и всех ли славян так
называли.

В результате исследования данных самых различных наук(лингвистики, археологии и
т.п.) можно сказать, что прародиной современных славян могли быть:

\begin{itemize}
  \item Средний Днепр;
  \item Северные области Карпат;
  \item Дунай
\end{itemize}

Также, существуют и другие гипотезы, но за 150 лет ученые так и не пришли к
единому мнению.

\ifcmt
pic https://avatars.mds.yandex.net/get-zen_doc/3985976/pub_5fc088c1c9a19d0e1cc1ccac_5fc0956ab1f92632baa7ce82/scale_1200
caption Карта Древнерусского Государства
\fi

\subsubsection{Первые русы}

Согласно теории западных ученых, русами были скандинавы, которых восточные
славяне пригласили на княжение или завоевавшие их. В этом случае русы и варяги
— это одно и то же.

Однако и здесь присутствуют противоречия, не имеющие объяснений в истории.
Загадка в том, что кроме «Повести временных лет» ни в одной древнерусской
летописи нет отождествления варягов и русов. Нет ни одного примера о проживании
на территории Скандинавии племен с названием рус или рос.

Поэтому родилась еще одна гипотеза появления русов, в которой утверждается, что
прародителями русо стали хазары, финские племена или ираноязычные народности (к
примеру, «роксоланы» или «аорсы»). Эта версия также не нашла подтверждения и в
итоге изначальное местоположение Руси, большой страны и многочисленного народа,
остается загадкой.

\ifcmt
pic https://avatars.mds.yandex.net/get-zen_doc/1873182/pub_5fc088c1c9a19d0e1cc1ccac_5fc097d01080732360194c0c/scale_1200
caption Русы покоряют Рим Источник: Яндекс картинки
\fi

\subsubsection{Русский каганат}

Впервые Русь как государство упоминается в 838 году. Именно тогда русские послы
(русский каганат) просили Людовика I разрешить им возвратиться на родину через
его территории, потому что основной путь был захвачен варварами. Этих послов
пленили и после допроса, оказалось, что они шведы и могли служить кагану русов.

Отмечают, что Ярослава Мудрого в 11 веке и его преемников тоже иногда называли
каганами, но при этом местоположение русского каганата до сих пор не
определено.

\ifcmt
pic https://avatars.mds.yandex.net/get-zen_doc/2420191/pub_5fc088c1c9a19d0e1cc1ccac_5fc099b2c9a19d0e1cd30f1e/scale_1200
caption Русский каганат так и остался загадкой истории Источник: Яндекс картинки
\fi

\subsubsection{Остров русов}

Согласно летописям арабских ученых, русы жили на острове. Есть немало версий о
его нахождении:

\begin{itemize}
  \item низовья Дуная;
  \item о. Рюген в Балтийском море;
  \item низовья Волги и т.п.
\end{itemize}

А персидский ученый 11 века ал - Марвази вообще упомянул о русах, проживающих в
«междуречье» и обратившихся в христианство, а затем принявших ислам.

\subsubsection{Правители}

Даты правления Русью первыми князьями легендарны, если верить летописям, то
детей они рождали практически до глубокой старости, а взрослыми становились
довольно поздно. К примеру:

\begin{itemize}
\item Игорь Рюрикович — единственный сын Рюрика, увидел свет, когда его отец
				был глубоким стариком, а вступил в наследование в 34 года (согласно
								другим источникам в 44 года). До этих пор же по неизвестной
								причине у власти находился Олег.

\item В 25 лет Игорь женится на Ольге, которая тогда была 13 - летним ребенком!
				Однако наследника Святослава Ольга родила лишь в 52 года. Им был
				Святослав. Когда Ольге было 65 лет, к ней сватался 50-летний Константин
				Багрянородный, император Византии. Кроме того, полноценно править
				страной Святослав начинает лишь после кончины матери, когда ему было
				26(а по некоторым данным-48!) лет.
\end{itemize}

\textbf{Противоречия в русской хронологии встречаются на каждом шагу. И разобраться в
ней ученые пока не в состоянии. Возможно, многие настоящие имена правителей
канули в Лету и никогда не всплывут на поверхность.}

\subsubsection{Крещение Руси}

Параллельно с признанной гипотезой принятия христианства во времена правления
князя Владимира существуют доказательства того, что Ярополк, его старший брат,
задолго до этого принял христианство. Об этом говорят некоторые немецкие
источники, а также Новгородские летописи 16-17 веков.

{\color{orange}\em
К тому же, сами термины «поп», «церковь» явно западного происхождения.
Получение церковью десятины от доходов государства - также традиция, перешедшая
к нам с Запада. К тому же, русским была неизвестна манера отделять в церкви
женскую территорию.
}

Также до сих пор не понятно и вызывает удивление, кому–же, всё-таки поклонялись
русы до принятия христианства. Совершенно отсутствуют доказательства
сколько-нибудь серьезного поклонения идолам и капищам, и наличию языческих
обычаев у русов.

\ifcmt
pic https://avatars.mds.yandex.net/get-zen_doc/1889318/pub_5fc088c1c9a19d0e1cc1ccac_5fc09cc0210b317d1e16fee4/scale_1200
caption Крещение Руси Источник: Яндекс картинки
\fi

\textbf{Легендарная Русь Изначальная неясна и туманна. Она полна сомнений и
противоречий. Немногие же факты, которыми обладают ученые, дают возможность
трактовать их довольно широко, вплоть до полного вымысла.}

\begin{leftbar}
	\em
Вот такую тему обсуждают на зарубежных форумах и негодуют от какой расы мы
произошли и чьими потомками являемся... А вы, что думаете по этому
поводу и есть ли у Вас своя история и предположения появления русской
нации? Пишите в комментариях, обсудим!
\end{leftbar}

