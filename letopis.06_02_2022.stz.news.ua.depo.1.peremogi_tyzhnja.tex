% vim: keymap=russian-jcukenwin
%%beginhead 
 
%%file 06_02_2022.stz.news.ua.depo.1.peremogi_tyzhnja
%%parent 06_02_2022
 
%%url https://www.depo.ua/ukr/war/peremogi-tizhnya-v-oos-zabaneni-prorosiyski-internet-propagandisti-i-sudovi-viroki-posibnikam-boyovikiv-l-dnr-202202061420860
 
%%author_id 
%%date 
 
%%tags ato,donbass,oos,ukraina,vojna
%%title Перемоги тижня в ООС: Забанені проросійські пропагандисти і судові вироки посібникам бойовиків
 
%%endhead 
 
\subsection{Перемоги тижня в ООС: Забанені проросійські пропагандисти і судові вироки посібникам бойовиків}
\label{sec:06_02_2022.stz.news.ua.depo.1.peremogi_tyzhnja}
 
\Purl{https://www.depo.ua/ukr/war/peremogi-tizhnya-v-oos-zabaneni-prorosiyski-internet-propagandisti-i-sudovi-viroki-posibnikam-boyovikiv-l-dnr-202202061420860}

\begin{zznagolos}
Війна проти російських окупантів триває. Цього тижня Кремль недорахувався на
Донбасі 10 своїх бойовиків, трьох затримали українські спецслужби і ще двоє
отримали за злочини проти України 8 і 9 років в'язниці
\end{zznagolos}

У 2022 році Depo.ua продовжує відслідковувати перемоги нашої армії. Цього тижня
Збройні сили України разом зі спецслужбами та дипломатами можуть записати собі
в актив наступні звитяги:

\subsubsection{Втрати живої сили бойовиків \enquote{Л-ДНР}}

Щодня на прицільні обстріли російських бойовиків українські захисники надавали
адекватну відповідь.

Упродовж тижня на окупованому Донбасі втрати російських найманців склали,
щонайменше, 10 осіб. З них 5 безповоротні (ліквідовані, вантаж 200) і 5
санітарні (поранені, вантаж 300).

Зокрема, втрати російсько-окупаційних військ 31 січня склали три особи. Двоє
безповоротні (9 омсп 1 АК та 2 омсбр 2 АК) і один санітарний (1 омсбр 1 АК).

1 лютого безповоротно вибув зі строю терорист 9 омсп 1 АК.

2 лютого поранили трьох окупантів (один з 2 омсбр 2 АК і двоє з 9 омсп 1 АК).

Втрати російських бойовиків 3 лютого склали одну особу. Більше ніколи не
турбуватиме українських захисників терорист 1 омсбр 1 АК.

4 лютого одного бойовика ліквідували  (9 омсп 1 АК) і один отримав поранення (2
омсбр 2 АК).

\subsubsection{Підтримка України у світі}

2 лютого уряд Швеції ухвалив виділення близько 5,4 мільйона доларів Україні для
зміцнення стійкості на тлі агресії Росії.

Зокрема, Україна отримає кошти протягом наступних трьох років у межах спільного
з Великою Британією, Канадою, Швейцарією та США фонду.

Фонд зміцнить стійкість до російської агресії на Сході та Півдні України,
сприяючи поліпшенню надання державних послуг, розширенню діалогу та
інклюзивному економічному розвитку.

3 лютого Україна вперше взяла участь в міністерській зустрічі Бухарестської
дев'ятки НАТО – політичного та безпекового формату взаємодії держав-членів
східного флангу Альянсу.

Міністр закордонних справ України Дмитро Кулеба взяв участь в міністерській
зустрічі Бухарестської дев’ятки НАТО на запрошення свого румунського колеги
Богдана Ауреску.

Кулеба детально розповів про концентрацію російських військ поблизу державного
кордону та на тимчасово окупованих територіях України. Міністри погодилися, що
агресія Росії загрожує безпеці не лише України, але й інших країн Центральної
Європи і євроатлантичному простору загалом.

Низка членів Бухарестської дев’ятки НАТО поінформувала, що на запит нашої
держави готує рішення, які допоможуть посилити оборонні спроможності,
кіберзахист та економічну стабільність України.

4 лютого Сенат Польщі одноголосно ухвалив резолюцію про підтримку України. За
проголосували всі 92 сенатори.

Сенат підтримав політичну та фінансову допомогу Україні, постачання зброї та
участь солдатів з країн-членів НАТО у підготовці українських військових.

\enquote{Сьогодні, коли Україна стоїть перед смертельною загрозою – загрозою чергового
вторгнення, – особливо важливо, щоб влада Республіки Польща надійно
підтримувала нашого сусіда. Без незалежної України немає безпечної Польщі.
Вільна Україна – одна з гарантій безпеки Польщі}, – йдеться у резолюції.

\subsubsection{Блокування пропагандистів Росії в інтернеті}

4 лютого YouTube заблокував канали \enquote{Перший незалежний} та UkrLive, які
пов'язані з депутатами від проросійської партії \enquote{Опозиційної платформи
– За життя}.

Як зазначає платформа, їх заблокували через \enquote{порушення правил спільноти
YouTube}.

Обидва канали було створено пропагандистами заблокованих рік тому телеканалів
ZIK, \enquote{112 Україна} та NewsOne.

Також 4 лютого відеохостинг YouTube заблокував канали \enquote{міністерства
інформації} та \enquote{народної міліції} \enquote{ДНР}.

Так само YouTube заблокував канали \enquote{ЛНР}: видалив пропагандистський
канал ЛІЦ і \enquote{народна міліція} \enquote{ЛНР} і заблокував канал
\enquote{ДТРК ЛНР}.

Того ж дня незаконне військове угрупування "Народна міліція" "ЛНР" заявила про
блокування своєї сторінки в соцмережі Facebook.  Озброєння для України від
західних партнерів

2 лютого Чехія та Україна підписали угоду про передачу зброї ЗСУ.

Нам передадуть артилерійські боєприпаси та медичну допомогу.

Також 2 лютого влада Польщі ухвалила рішення про передачу Україні оборонного
озброєння та боєприпасів, серед яких офіційно називаються ПЗРК \enquote{Перун}
(Piorun).

ПЗРК \enquote{Перун} призначений для ураження повітряних цілей, що низько
летять – як на зустрічних, так і на наздоганяючих курсах.

Читайте також: Його \enquote{батько} збивав росіян: Що за ПЗРК \enquote{Перун}
дає Україні Польща

1 лютого до України прибув шостий літак з військовою допомогою від США.

У черговій партії військової допомоги – 84 тонни боєприпасів.

3 лютого в Україні приземлився сьомий літак з військовою допомогою від США.

У цій партії Україна отримала 85 тонн бойових боєприпасів для гранатометів.

Читайте також: \enquote{Стінгери}, \enquote{Джавеліни}, вогнемети і
гранатомети: Яку зброю вже дають Україні західні партнери

\subsubsection{Викриття та затримання бойовиків \enquote{Л-ДНР} і їхніх
посіпак}

2 лютого в Донецькій області встановили місцезнаходження і затримали учасницю
угруповання \enquote{ДНР} \enquote{Пятнашка}.

За даними слідства, затриманою є мешканка Мирнограду на Донеччині, яка у 2014
році добровільно приєдналась до незаконних збройних формувань ОРДЛО. Там вона
отримала автоматичну зброю та боєприпаси.

Після \enquote{курсу} підготовки у лавах бойовиків, ватажки залучали її до
виконання завдань проти підрозділів Збройних сил України. Згодом зловмисниця
покинула угруповання і тривалий час переховувалась від правосуддя.

Наразі слідчі СБУ повідомили затриманій про підозру за ч. 1 ст. 258-3 (участь у
терористичній групі чи терористичній організації) Кримінального кодексу
України. Вирішується питання щодо обрання судом міри запобіжного заходу.

3 лютого СБУ затримала бойовика, який з гранатомета обстрілював українські
позиції в районі Станиці Луганської.

Встановлено, що зловмисник проходив бойову підготовку під керівництвом
військових інструкторів з РФ на полігоні поблизу тимчасово непідконтрольного
українській владі міста Лутугине.

Затриманим є мешканець Лисичанська, якого раніше вже притягували до
кримінальної відповідальності. У січні 2015 року він добровільно приєднався до
угруповань бойовиків. Там зловмисник отримав зброю та військову форму
російського зразка.

Після курсу вогневої підготовки його призначили гранатометником одного з
незаконних збройних формувань ОРДЛО. Його \enquote{підрозділ} брав активну участь у
бойових діях проти українських військ у районі Станиці Луганської.

Згодом зловмисник втік з лав бойовиків через конфлікт з ватажками, які
знущались над своїми підлеглими та \enquote{вибивали} з них гроші, і тривалий час
переховувався від правосуддя.

Перевіряється інформація щодо причетності зловмисника до вчинення інших тяжких
та особливо тяжких злочинів.

4 лютого Служба безпеки України задокументувала злочинну діяльність бойовика
так званої "МДБ "ДНР".

Він обвинувачується у жорстокому катуванні військовополонених та цивільного
населення, що є порушенням законів та звичаїв війни.

У межах розпочатого кримінального провадження оперативники спецслужби
встановили, що злочинні дії зловмисник вчиняв на території катівні \enquote{Ізоляція}.

Керував \enquote{закладом} його кум на прізвисько \enquote{Палич}, якого контррозвідники СБУ
затримали його в Києві у листопаді минулого року.

Слідчі органів безпеки заочно повідомили бойовику про підозру у: ч. 2 ст. 28
(вчинення злочину групою осіб, групою осіб за попередньою змовою, організованою
групою або злочинною організацією); ч. 1 ст. 258-3 (створення терористичної
групи чи терористичної організації); ч. 1 ст. 438 (порушення законів та звичаїв
війни).

Санкції статей передбачають покарання у вигляді позбавлення волі на термін до
15 років.

Також 4 лютого в Луганській області прокуратура повідомила про підозру
депутатці Золотівської міської ради Попаснянського району.

Депутатка Золотівської міської ради у травні 2014 року брала активну участь в
організації та проведенні незаконного \enquote{референдуму}.

Підозрювана, виконуючи функції \enquote{секретарки виборчої комісії}, здійснювала облік
осіб, видавала \enquote{бюлетені}, слідкувала за процесом і порядком голосування.

Результати такого незаконного \enquote{волевиявлення} передали до так званої
\enquote{центральної виборчої комісії} для подальшого відокремлення Луганської області
від території України та проголошення так званої \enquote{ЛНР}.

За вчинення протиправних дій підозрюваній загрожує позбавлення волі на строк до
10 років з конфіскацією майна або без такої.

\subsubsection{Судові вироки бойовикам \enquote{Л-ДНР} і їхнім посіпакам}

31 січня Рівненський міський суд засудив на 9 років позбавлення волі командира
розвідувально-диверсійної групи, який воював проти ЗСУ.

Бойовик є громадянином Чехії. Він вступив до \enquote{ДНР} у квітні 2014 року.
Пройшов спеціальний вишкіл з бойової підготовки і згодом брав участь у боях
проти українських військових, зокрема, під час активної фази боїв за Дебальцеве
у 2015 році.

Окрім цього, бойовик збирав розвіддані щодо підрозділів ЗСУ на Донбасі.
Інформацію щодо кількості, розташування позицій та передислокації українських
військ він передавав через \enquote{кураторів} представникам спецслужб Росії.

За завданням російських \enquote{кураторів} бойовик займався також підбором
кадрів для розвідувально-диверсійних груп угруповання \enquote{Оплот}. Він
навчав учасників терористичних організацій веденню розвідувально-підривної
діяльності проти українських військових.

За матеріалами СБУ, визнав зловмисника винним у вчиненні злочину за ч. 1 ст.
258-3 (участь у терористичній групі чи терористичній організації) Кримінального
кодексу України та призначив покарання у вигляді 9 років позбавлення волі.

4 лютого набув чинності вирок суду колишній співробітниці органів прокуратури
України, яка зрадила присязі і перейшла на бік окупантів.

У тимчасово окупованому Алчевську вона очолила підрозділ так званої
\enquote{генеральної прокуратури \enquote{ЛНР}}. За даними слідства, жінка
отримала від бойовиків керівну посаду у \enquote{прокуратурі} у 2015 році.

Слідчі СБУ зібрали беззаперечні докази того, що вона забезпечувала виконання і
дотримання незаконно виданих окупантами "законів".

Зокрема, за вказівками ватажків терористичної \enquote{ЛНР} здійснювала
контроль за перереєстрацією захоплених українських підприємств в органах
окупаційної адміністрації та стягненням з них податків до \enquote{бюджету
ЛНР}; чинила тиск на підприємницькі структури, санкціонувала незаконні
затримання та арешти громадян України.

У 2017 році, на підставі зібраних СБУ доказів незаконної діяльності, поплічницю
терористів було оголошено у загальнодержавний розшук.

Суд, розглянувши обвинувальні матеріли за заочною процедурою, прийняв остаточне
процесуальне рішення щодо вини правопорушниці та засудив її до 8 років 6
місяців позбавлення волі з конфіскацією майна. Термін покарання
обчислюватиметься з дня затримання засудженої

\subsubsection{Нова зброя \enquote{Укропів}}

3 лютого на брифінгу міністр оборони України Олексій Резніков повідомив, що
Міністерство оборони спільно зі Збройними силами України розробили ракетну
програму, яка має сприяти появі в українському війську сучасних систем ППО та
ПРО (протиповітряної і протиракетної оборони).

За словами Резнікова, програма розрахована на 10 років та передбачає
фінансування за цей період на рівні близько 250 мільярдів гривень.

Читайте також: Нова ракетна програма: Чи створить Україна свій Patriot

4 лютого українські військові провели практичні стрільби з гранатомета M141.

Фахівці 184-го Навчального центру під час навчання поєднали умови виконання
вправ з цієї зброї та тактикою дій Збройних сил України.

Під час практичних стрільб з американських гранатометів військовослужбовці не
тільки уражали мішені, а й показали тактичні дії парами.

Ці інструктори згодом проводитимуть аналогічну підготовку в інших навчальних
Центрах, своїх бойових підрозділах та у вищих військових навчальних закладах.

Читайте також: Для знищення нападників і ворожих укріплень: Що за
гранатомети-вогнемети SMAW-D отримала Україна від США

Отож, як бачимо, і цей тиждень виявився досить вдалим: як на полі бою, так і за
його межами Україні вдалося здобути низку локальних перемог, які ще на кілька
кроків наблизили головну перемогу у війні з Росією.
