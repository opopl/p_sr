% vim: keymap=russian-jcukenwin
%%beginhead 
 
%%file 20_01_2022.fb.uljanov_anatolij.1.psiho_terror
%%parent 20_01_2022
 
%%url https://www.facebook.com/dadakinder/posts/5163721576980297
 
%%author_id uljanov_anatolij
%%date 
 
%%tags napadenie,psihika,psihologia,rossia,ugroza,ukraina
%%title Психологический террор
 
%%endhead 
 
\subsection{Психологический террор}
\label{sec:20_01_2022.fb.uljanov_anatolij.1.psiho_terror}
 
\Purl{https://www.facebook.com/dadakinder/posts/5163721576980297}
\ifcmt
 author_begin
   author_id uljanov_anatolij
 author_end
\fi

Круглосуточная бомбардировка жителей социально депрессивной страны сообщениями
о том, что на них вот-вот нападут – это форма террора. То как это делается,
характеризует тех, кто это делает, и их отношение к тем, с кем это делается. 

\ii{20_01_2022.fb.uljanov_anatolij.1.psiho_terror.pic.1}

Подобные сообщения предполагают ответственность сообщающего. Ими нельзя
произвольно разбрасываться, ввергая и без того задёрганных людей в панические
состояния. Подобные сообщения не должны поступать без сопровождающей их
программы действий и соответствующих телодвижений властей, которые давали бы
понять, что всё серьёзно. В противном случае, это просто опасная болтовня.

Людей поливают угрозами и страхом, что, очевидно, является психологической
подготовкой к какой-то провокации, либо же прочим манёврам нашей американской
метрополии в контексте её отношений с российскими партнёрами по дележу
постсоветского пространства.

Если мне не изменяет память, то, согласно официальной версии, Путин уже давно
напал. Но тут они, вдруг, меняют нарратив, и говорят: «нападёт». То есть, ещё
не напал? Тогда возникает множество вопросов по поводу всего, что происходит на
Востоке, где умирают люди…

Так или иначе, от рядового гражданина при нынешней конфигурации власти, ничего
не зависит. И так будет до тех пор, пока этот гражданин не изменит конфигурацию
власти. 

Не надо вестись на каждый возглас «Волки!». Поберегите психику. Потому что она
очень нужна в ситуации реального волка. По моменту её нет, а есть кабинетные
танцы политиков с бубном, цель которых выгодный договорняк. Ему на помощь
призывается игра мускулами и прочий «театр военных действий». Пока торгуются,
ничего не будет.
