% vim: keymap=russian-jcukenwin
%%beginhead 
 
%%file 25_01_2021.fb.proshkovskaja_maria.1.bilchenko_zajavlenie_studentka_magistratura
%%parent 25_01_2021
 
%%url https://www.facebook.com/permalink.php?story_fbid=4087393264606431&id=100000073933417
 
%%author 
%%author_id proshkovskaja_maria
%%author_url 
 
%%tags bilchenko_evgenia,obrazovanie,otchislenie,travlja,ucheba,ukraina,universytet_dragomanova
%%title Заява на відрахування - студентка - магістратура - допис професора Євгенії Більченко
 
%%endhead 
 
\subsection{Заява на відрахування - студентка - магістратура - допис професора Євгенії Більченко}
\label{sec:25_01_2021.fb.proshkovskaja_maria.1.bilchenko_zajavlenie_studentka_magistratura}
 
\Purl{https://www.facebook.com/permalink.php?story_fbid=4087393264606431&id=100000073933417}
\ifcmt
 author_begin
   author_id proshkovskaja_maria
 author_end
\fi

Публічна заява від мене, як студентки 1 курсу магістратури  факультету
філософії і суспільствознавства (спеціальність культурологія) Національного
педагогічного університету України ім.Драгоманова. Прошу керівництво
університету відрахувати мене за власним бажанням в зв’язку з повною
відсутністю реакції на події останніх днів, пов’язані з публічними
висловлюваннями професорки культурологіі Євгенії Більченко. 

Професорка Більченко не поважає країну в якій живе, працює і заробляє, мову
цієї краіни, вважає війну Росії проти України громадянською війною,
демократично обрану владу - неонациським режимом. 

Крім того вона публічно заявляє про дружбу з Захаром Прилєпіним, виступає на
телеканалах ДНР, Клименка та російському телебаченні, де розповідає про те, як
її тут пригнічують. 

Прямо в фейсбук заявляє також про те, що вона разом з керівництвом університету
«підтирається скаргами», які на неї надходять. І питання тут не тільки в її
особистій позиції, а ще і в тому, що вона доволі м’яко, але системно промиває
мізки студентам. 

Викладаючи гуманітарні дисципліни філософського спрямування не
важко транслювати ті чи інші особисті погляди як істину. Достатньо подивитись
на YouTube її лекції, що є частиною учбової програми на теми, пов’язані в
сучасністю. 

Я пройшла Революцію Гідності, допомагала армії і цивільним, виховую свого сина
в повазі і любові до України, а тому просто не можу вчитись далі у Євгенії
Більченко та в університеті, який паплюжить всі принципи державної вищої освіти
і на 7 році війни закриває очі на подібні резонансні речі. 

Як заключення можу сказати, що я проти будь-яких цькувань, погроз, оцінок
зовнішності чи постановки віртуальних діагнозів і пані Більченко має право на
будь-яку думку, окрім тої, що суперечить законодавству України як приватна
особа, але як викладач вишу має добирати слова і поважати суспільство, в якому
вона живе. 

P.S. Фізично заяву подам завтра. 

Слава Україні! 

UPD Заяву зареєстровано в канцелярії ректорату 26.01.2021, фото додала в коментарі.
