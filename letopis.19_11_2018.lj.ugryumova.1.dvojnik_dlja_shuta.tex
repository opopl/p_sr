% vim: keymap=russian-jcukenwin
%%beginhead 
 
%%file 19_11_2018.lj.ugryumova.1.dvojnik_dlja_shuta
%%parent 19_11_2018
 
%%url https://ugryumova.livejournal.com/249049.html
 
%%author_id ugrjumova_viktoria.kiev.pisatel
%%date 
 
%%tags 
%%title ЗАНИМАТЕЛЬНАЯ БИБЛИОГРАФИЯ - ДВОЙНИК ДЛЯ ШУТА
 
%%endhead 
\subsection{ЗАНИМАТЕЛЬНАЯ БИБЛИОГРАФИЯ - ДВОЙНИК ДЛЯ ШУТА}
\label{sec:19_11_2018.lj.ugryumova.1.dvojnik_dlja_shuta}

\Purl{https://ugryumova.livejournal.com/249049.html}
\ifcmt
 author_begin
   author_id ugrjumova_viktoria.kiev.pisatel
 author_end
\fi

Благоденствие бессюжетно. Только перемены счастья двигают судьбу.

Самуил Лурье. Счастливый Тициан

Далеко не про все свои книги знаешь, где и когда они написаны. Вот где и когда
записаны – совсем другое дело, потому что адрес и прочие мелочи с годами не
меняются, разве что модели компьютера, начиная с пишущей машинки «Любава».

Обычно книги живут рядом с тобой так долго, просачиваются в твою жизнь так
постепенно, с течением времени обрастая персонажами и подробностями, что почти
невозможно вспомнить тот самый момент, когда хлопнул себя по лбу: Семен
Семенович! А напишу-ка я роман, и будет ему название такое-то, а сюжет такой-то
и закончится он тем-то и тем-то, и вот тут именно прольем мы несколько слез, а
здесь понимающе улыбнемся и захлопнем томик. Нет, не бывает такого на свете,
господа.

\ii{19_11_2018.lj.ugryumova.1.dvojnik_dlja_shuta.pic.1}

Тем более странно, что о «Двойнике для шута» я знаю на редкость много, как для
своей собственной книги, над которой еще не работали въедливые литературоведы и
строгие критики. Другое дело,  сколько абсолютно разных вещей вспоминаются по
этому поводу – к месту и не к месту.

«Двойник для шута» стал «Двойником для шута» прекрасным весенним днем прошлого
века и тысячелетия у станции метро «Лукьяновская» на последней (она же первая)
остановке 23 троллейбуса в тот благословенный миг, когда у троллейбуса
заклинило двери.

Вообразите картину: довольно раннее утро, будний день, энергичный народ,
видимо, стремится на трудовую вахту; кто-то, судя по крикам, опаздывает на
важную встречу, кто-то на свидание, кто-то рвется на базар – он уже виден из
окна троллейбуса – рукой буквально подать, - а двери не открываются. А народ
темпераментный. И двери выламывает могучим женским плечом. А водитель нервный.
Он материально ответственный за этот троллейбус, а все материально
ответственные – нервные. Или просто так болеет душой за своего рогатого
питомца. Словом, он уже не пытается справиться с техникой и выпустить буйных
пассажиров на волю, а покидает водительское место и становится грудью на защиту
своего многоколесного друга и удивительно легко, как родной, вписывается в
конфликт. И конфликт очень быстро переходит в новую стадию, потому что
опаздывающие граждане требуют вернуть им деньги, потраченные на билеты, и
упирают на то, что не столько деньги дороги, сколько принципы. И тут у водителя
окончательно сдают нервы, и он принимается буквально на пальцах показывать, как
долго он сюда ехал и вез на собственном горбу всех этих неблагодарных людей, и
доказывать, что одно это чего-нибудь да стоит. А снаружи другие активные
граждане, желающие быстро ехать в обратном направлении, тоже не стоят, сложа
руки, громко колотят в стенки троллейбуса и привлекают к себе внимание
разнообразными телодвижениями и рожи корчат.

А бедному призраку куда податься?

Так что я скромно сидела на заднем сидении, и крутилась у меня в голове
почему-то одна мысль: двойник для – кого? Несколько вариантов помню по сей
день: двойник для убийцы, двойник для мертвеца, двойник для покойника...
подумать только, согласись тогда моя причудливая мысль следовать этой тропою –
и быть бы книге детективом. Но не случилось. Воображение перескочило в миры
иные.  Появились варианты: двойник для короля, двойник для полководца – совсем
никуда не годится. Потом двойник для императора, потом пару вариантов, которые
сейчас и не вспомнятся толком. А потом, невесть откуда, выскочил двойник для
шута.

Не так уж много моих книг начиналось именно с названия. Всего-то две – еще
«Некромерон», про который сразу стало ясно, что он такое, ибо это название
двояких толкований не допускает. Но в то время с магией названий я еще не
сталкивалась. Просто к тому времени, когда нас все-таки выпустили из средства
передвижения, загадочный шут уже полностью занимал мое воображение.

Шуту, согласитесь, зачем бы иметь двойника. Разве что он не очень-то кот, простите – не очень-то шут.

Думаю, без Булгакова тут не обошлось. Думаю, что не обошлось и без некоторой
чертовщины. Ведь выйди я в тот день из троллейбуса, как собиралась, закрутилась
бы в делах, и не вспомнила бы о незнакомом двойнике. А так минут сорок пять –
не меньше – размышляла только о нем.

Одна из важных особенностей моей работы над всякой книгой – явление мира. Даже
персонажи не так важны, как те тропинки, деревушки, озера, реки, парки, домики,
кабачки, гостиницы, театры, дворцы, замки, магазинчики, степи, пустыни… словом,
что угодно, лишь бы мир проступил сквозь ткань реальности, чтобы наше
пространство стало такой себе кисеей, сквозь которую явственно просвечивает
цельная картина. И чем лучше она видна, тем вернее знак, что новая книга
состоится. Потом персонажи начинают обживать это пространство, заглядывать в
самые пыльные, темные и дальние его уголки – это и есть сюжет. Далее я не
беспокоюсь, они сами расскажут каждый свою историю – только успевай записывать.

И тут мир явился целиком и полностью – сразу весь, что, согласитесь, редкий
случай: карту, например, Варда я зарисовывала, следуя за воображением, в
течение месяца; просто эпоха Великих Географических открытий. Только-только
начертишь, ан нет – снова какой-нибудь непредвиденный город обнаружится или
даже целая страна вынырнет из тьмы веков: наша песня хороша – начинай сначала.

Лунггар этих хлопот не доставлял. Ровно как в анекдоте:

Бог придумал труд и обезьяну, чтобы получился человек. А с пингвином он не
экспериментировал. Тот сразу прекрасно получился.

Этот мир я узнала моментально, даже запоминать специально не пришлось.

Как говорил Александр Грин: я и теперь, спустя несколько десятков лет могу
пройти из Зурбагана в Лисс и описать вам каждую тропинку, каждый поворот,
каждый пенек и каждый домик. Ну, или как-то приблизительно так он говорил.

Нет, без Булгакова точно не обошлось.

На дворе, как писал именно Михаил Афанасьевич, стоял год от Рождества Христова
1999. Последний год тысячелетия, думали тогда все приличные граждане. Я
работала над «Мужчинами ее мечты», их, кстати, очередную подачу и везла в
издательство - какие еще, к шутам, шуты, какие двойники?

Но книги прорастают сквозь все слои жизни, как дерево – сквозь все слои земли,
в которой живет, а там что только не лежит одно над другим: обломки мечей и
копий, щиты, черепки какие-то глиняные, медяки, гвозди, гильзы, осколки
снарядов и кости, кости, кости… И как эта пышная липа или там клен не похожи на
то, что когда-то было семечком, упавшим возле бронзовой загогульки – а ей,
страшно сказать, сколько лет, и что-то от нее дереву непременно через состав
почвы передалось.

Моя бабушка про тридцатые годы прошлого века рассказывала не слишком много и не
чересчур подробно. Но рассказчица она была блистательная, так что я все равно
составила себе довольно ясное представление о том, как у черных окон в черных
квартирах над черными ущельями улиц стояли люди, ожидая неизбежного. Как
раздавался звук мотора, подъезжала машина – не обязательно, кстати,
увековеченный в сотнях мемуаров «черный воронок», порой машина могла быть
вполне прозаическая – вроде той, что развозила хлеб или молоко. И как люди
исчезали каждую ночь почти беззвучно: загорался свет в какой-нибудь из квартир,
- и все. Как у многих стояли наготове собранные вещи – и это было совсем не так
весело, как читать у Ильфа и Петрова про «допровскую» корзинку. «И никто не
отстреливался», - бросила однажды бабушка как бы невзначай. В нашей семье к
этому году отстреливаться было уже некому и главное - нечем.

Ей удалось когда-то в нескольких фразах передать этот вязкий липкий ночной
кошмар, в котором металась вся страна. И ощущение ярости и беспомощности, от
которого мог и кондрашка хватить.

В середине XVIII века Уильям Питт, государственный секретарь и лорд-хранитель
печати Великой Британии заявил: «Последний бедняк в своем жилище может
оказывать сопротивление силам Короны. Дом может разваливаться, крыша может
трястись, ветер может продувать его насквозь, буря может ворваться в него, и
дождь может ворваться в него, но Король Англии не имеет права войти. Вся его
сила не дает ему власти переступить порог этих развалин».

В нашей жизни такого надежного порога никогда не было.

А вот был зато Булгаков, «Мастер и Маргарита» - великий роман, как раз и
написанный в то время и в той стране, когда все замирали по ночам у
неосвещенных окон.

Первый, он же главный, в моей жизни экземпляр «Мастера» выглядит настоящим
раритетом. В сущности, он и есть теперь, в своем, неповторимом роде раритет.
Это копия, сделанная (Ксерокса тогда еще в помине не было, тот зверь-машина
звался Ротапринтом) на серой бумаге, с одной стороны, по книжному развороту на
страницу, сшитая нитками и переплетенная в довольно мягкую обложку с зеленым
дермантиновым корешком. Больше всего похожа она на огромный тяжелый альбом.
Никакое другое издание «Мастера и Маргариты» не трогает так мое сердце, как это
старинное самиздатовское чудище, которое и теперь-то весит немало, а в детстве
казалось мне совершенно неподъемным. Я не знаю, сколько раз перечитывала его.

Об этой чудесной книге написано с тех пор, как ее разрешили издавать, немало
других книг – воспоминаний, исследований, критик. Да что там немало – чертова
прорва, ведь она зацепила за живое, как минимум, два, а то и три поколения.
Многое из того, что было написано, я прочитала. Но мне кажется, никто так и не
написал, во всяком случае нигде я не встретила мысль, комментарий, трактовку –
что угодно бы – одного очень важного момента. Мне представляется, что роман –
вся его московская, во всяком случае, часть - и подводил нас к этому моменту,
чтобы он звучал, как правда, легко и приятно. Чтобы ни секунды не усомниться,
что так оно и было и иначе быть не могло. И так хорошо вышло это у Михаила
Афанасьевича, что поколения читателей пролетают этот фрагмент с огромным
удовольствием, но на крейсерской скорости, заданной ритмом самой книги,
вероятно, не замечая, что вот она – одна из кульминаций. Может быть, самая
важная.

- А что это за шаги такие на лестнице? – спросил Коровьев, поигрывая ложечкой в
чашечке с черным кофе.

- А это нас арестовывать идут, - ответил Азазелло и выпил стопочку коньяку.

- А-а, ну-ну, - ответил на это Коровьев.

Это, видите ли, такая неосуществимая, такая заветная мечта всякого, кто жил под
кровавым дамокловым мечом ночных исчезновений; в стране, где есть один довольно
известный город, стоящий на воспетой классиками реке – так вот в одном месте
дно этой реки красное, и вода, соответственно, тоже, потому что какие-то
природные премудрые сработали механизмы; но горожане в природу верить
отказываются, а говорят, что это въелась в камни кровь, потому что она – в
самом деле так было – стекала из стоков, из помещений с красным асфальтовым
полом, в которых ежемесячно превращали в неживых сорок тысяч живых людей. «Я
вас любил как сорок тысяч братьев», - нет, совершенно неверная ассоциация.
Вовсе не отсюда.

И вот когда живешь в мире, все еще находящемся под властью этого ночного
кошмара, нелепо как-то ожидать, чтобы люди, с которыми так лихо обращались
десятилетиями, как-то иначе обращались с окружающим пространством. Ненависть
въедается во все, до чего дотянется. Это вам не кровь, которой требуются
предметы материальные, обладающие хотя бы минимальной гигроскопичностью, порами
там какими-нибудь, чтобы в них проникнуть. Ненависть может быть элементарно
разлита в воздухе и прямо из него генерировать невероятную энергию разрушения.
А безумно хочется все-таки попробовать этот мир сохранить. И тогда выясняется,
что для этого нужен какой-нибудь ultima ratio regnum – водородную бомбу не
предлагать. Но что делать, что делать, если ultima ratio всю нашу историю –
только вариации на тему пушек?

К тому времени, когда начался и невероятно быстро воплотился на бумаге мой
роман с новым романом, я уже написала и опубликовала свою любимую киевскую
тетралогию: «Путеводитель для гнома», «Баллада о зонтике в клеточку», «Дом там,
где ты» и «Почти что эпитафия». Эти рассказы – одновременно и киевский миф, и
реквием по старому Киеву, по его удивительным обитателям, по паркам и
библиотекам, по фонтанам и клумбам, белкам, совам и скворцам. С тех пор этих
горестных вздохов издано так много, так много, но, вряд ли, это что-то меняет в
нашей реальности. В ней, как уже стало привычным за долгие годы, постоянно
что-то исчезает: то вдруг спилят древнее дерево, то снесут дом – даже и с
табличкой «Памятник архитектуры. Охраняется государством». То от великолепной
Владимирской горки оттяпают кусок на музей Ленину – а кто против советской
власти?; то ручных белок употребят на шапки, набив ими полный чемодан; то целый
скверик за ночь растворится в пространстве - не было здесь ничего, сплошная
резанная бумага вместо денег. То киевские зеленые склоны, воспетые Булгаковым
же, побреют налысо.

А вот на Лунггаре все обстояло иначе.

Великий Роан, империя Агилольфингов – моя воплощенная мечта о мире, тишине,
покое; о таком земном приюте для слабых и беззащитных; о месте, где Корона не
только не имеет власти вторгнуться в твое жилище, но и не помышляет об этом. О
стране, где такая вот как бы Александрийская библиотека до сих пор в целости и
сохранности; великие памятники древности радуют глаз подрастающему поколению
спустя многие века; где грандиозные города – вовсе не кладбища мертвых животных
и невинно убиенных деревьев, а – напротив – надежное прибежище красоты; где
государи не просто пекутся о благе государства, но платят за это самую высокую
цену, и где при этом живут гордые и свободные люди, готовые в любой момент
постоять за сохранность своего мира. Мир, в котором – когда и если кто-то
умирает, то не бессмысленно, не зря и не сдуру. А эта смерть означает, что
порядок сохранится еще лет на тысячу.

Это такая надежда на то, что в мире есть сила, по сравнению с которой привычные
нам, неистребимые в нашем мире, всемогущие кошмары – не значат ровным счетом
ничего. И какое это наслаждение, какая радость, какая безумная отчаянная
надежда – выводить на бумаге слово за словом:

- А что это за шаги такие на лестнице? – спросил Коровьев, поигрывая ложечкой в
чашечке с черным кофе.

- А это нас арестовывать идут, - ответил Азазелло и выпил стопочку коньяку.

- А-а, ну-ну, - ответил на это Коровьев.

* * *

Из всех выдуманных стран, от Утопии и Лилипутии до Хоббитании и Нарнии Гондал –
одна из самых загадочных. Не сохранилось ни отчетов путешественников,
побывавших там, ни исторических хроник...

Григорий Кружков, «Принцесса Гондала»

Свою статью об Эмилии Бронте Григорий Кружков назвал очень красиво – «Принцесса
Гондала»...

Нет, не могу. Буду сначала рассказывать о Кружкове и его «Пироскафе». Не к
месту? Пожалуй, не совсем. Впрочем, я ведь еще в самом начале заявила о своем
намерении вспоминать все, что касается книги, подряд, без разбора. Никто не
говорил, что будет легко.

Григорий Кружков – вообще-то переводчик. Если на то пошло – очень хороший
переводчик, каких мало. Он мало того, что знает, как переводить, он еще и любит
то, что переводит. Оттого выходят у него изумительные тексты.

До огромного этого «Пироскафа» (690 страниц нестандартного, приятного большого
формата) он написал такое же восхитительное чудище – «Лекарство от Фортуны:
поэты при дворе Генриха VIII, Елизаветы Английской и короля Иакова». Ну а
«Пироскаф» - это английская поэзия следующего века. Только в обоих случаях
вышли замечательные книги: переводы перемешаны со статьями, статьи с
открытками, рисунками, фотографиями, картинками. Тут и марки, и фрагменты
писем, и экслибрисы. И все это так прелестно, легко, дышит полной грудью. И
утверждается право автора рассказывать не как положено, а как ему, автору,
хочется. И выходит, что про всемирно известного Эдгара По мы обнаруживаем
краткий столбец скупой информации хорошо если знаков на 200 максимум. А о Клэре
– внушительная статья с солидным иллюстративным материалом. Вы не знаете, кто
такой Джон Клэр? Зеленый человечек английской поэзии? Вот теперь и узнаете. Так
что в каком-то смысле, эта книга – еще и дань справедливости. И чистый
авторский произвол. Как по мне – шедевр.

И теперь я с чистой совестью могу возвращаться к Гондалу.

Итак, на чем мы остановились? На том, что свою – большую – статью об Эмилии
Бронте Григорий Кружков назвал «Принцесса Гондала».

Популярность сестер Бронте в Англии удивительна – до 1 миллиона туристов в год
в их доме-музее, масса сувениров, марок, переизданий их романов. Есть даже кафе
«Бронтозавтрак». То есть, они стали классиками английской литературы. То есть,
их книги, почти гарантированно, почти никто не читает.

Эмили написала знаменитый «Грозовой перевал» и много чудесных стихов; Шарлотта
– ставшую канонической «Джен Эйр». А была еще Анна, она написала роман «Агнесс
Грей»; и брат Брануэлл – он много рисовал. Еще две сестры Мэри и Элизабет
умерли в 10-летнем возрасте. Это не имеет никакого отношения к остальной части
рассказа; и я говорю об этом только потому, что мне представляется жестоким да
и нечестным не упоминать, что они жили на свете лишь по той причине, что они
ничего не успели написать.

Дети Бронте, а после уже не дети, а вполне взрослые люди сочиняли, создавали до
мельчайших подробностей историю Гондала и соседнего государства Гаальдин всю
свою не очень долгую жизнь. Когда все умерли, и осталась только Шарлотта, она,
по всей видимости, уничтожила основную часть работы. Все, что осталось -
несколько обрывков истории, пару дневниковых записей да сколько-то
стихотворений.

Эмилия умерла в 28, Анна – в 29, а Шарлотта всю оставшуюся жизнь творила миф о
сестрах Бронте – и этот миф сверкал и искрился на солнце как первый снег. В нем
было много христианской морали, чистоты и непорочности, рассказов о суровых
испытаниях и тяжелой жизни, масса назидательности и поучительности. Гондал, не
подходивший к этому мифу ни по одному из этих пунктов, был вымаран начисто.

Но, несомненно, что именно в таком виде, в виде легенды о легенде, он достигает
наивысшего совершенства. Законченный роман о событиях в вымышленном мире (если
судить по тем отрывкам, которые до нас дошли), наверняка уступил бы и
Хоббитании, и Нарнии, и Лилипутии.

Но когда кто-то поет невыносимо печально о стране, которой нет, о людях,
которые ушли навсегда, о ее принцах и принцессах, рыцарях и разбойниках, героях
и предателях, - кажется, что прекраснее истории на свете нет. Жаль, те, кто ее
помнил – умерли, а остальные – позабыли.

* * *

Итак, «Двойник для шута» был написан очень быстро и быстро издан. И даже
переиздан почти сразу – с теми же ошибками и чуть ли не с теми же опечатками.

Но история желала быть продолженной. С историями иногда такое случается. Они не
заканчиваются, когда ты ставишь последнюю точку. И всплыли внезапно из небытия
старые хроники, которые дотошно и подробно объясняли, почему Агилольфинги берут
в жены только дочерей Майнингенов; почему Далихаджар так скорбел о смерти
императрицы Арианны; о том, что случилось на Бангалорах за двести лет до
описываемых событий. И о том, как жил дальше без своей единственной
возлюбленной величайший император Великого Роана Ортон Агилольфинг. Вот оно –
«Зеркало для шута».

Перебираю потертые на сгибах листки разного формата, с потрепанными краями, в
линейку и клеточку, и просто белые (вот даже меню какого-то кафе пострадало за
изящную словесность), перечитываю время от времени – как письма из страны,
которой уже нет на карте, от людей, которых уже нет на свете.

А хранитель императорской библиотеки Олден Фейт все-таки умер, прямо над своими
любимыми книгами, и Сивард Ру так переживал из-за этого, что буквально
переселился в осиротевшую библиотеку; стал помногу читать; начал с книг, над
которыми умер старик – и однажды нашел от него весточку. А Аластер влюбился, и
в кого! Подумать только, какой мезальянс. А Аббон Сгорбленный уехал в далекое
путешествие, что-то там такое весьма важное произошло с ним вдали от Роана… К
йеттам внезапно вернулся их давно потерянный бог Терей… Очень многое случилось
с моими персонажами.

Но я, - хотя еще не читала тогда о принцессе Гондала – не поверила в
возможность продолжения. Как сказала когда-то Татьяна Лиознова о последних
кадрах своего великого фильма: 

- Я знаю, что случится со Штирлицем потом, когда он вернется в Берлин, но это
только мое. Этого никому другому знать не нужно.

С другой стороны, так порою хочется петь невыносимо печально об этих дамах и
рыцарях, королях и героях. Потому что если выбросить однажды все эти стопки
исчерканных листков, все эти обрывки историй, старые хроники и легенды, и
фрагменты  карт, многих людей не станет, как если бы и не было их никогда. И
только потом, вдруг вспоминаешь, что их никогда и не было.
