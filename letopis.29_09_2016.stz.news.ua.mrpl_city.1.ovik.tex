% vim: keymap=russian-jcukenwin
%%beginhead 
 
%%file 29_09_2016.stz.news.ua.mrpl_city.1.ovik
%%parent 29_09_2016
 
%%url https://mrpl.city/news/view/istoriya-o-tom-kak-v-mariupole-poselilis-lisa-letuchaya-mysh-i-ryby-foto
 
%%author_id news.ua.mrpl_city
%%date 
 
%%tags 
%%title История о том, как в Мариуполе поселились лиса, летучая мышь и рыбы (ФОТО)
 
%%endhead 
 
\subsection{История о том, как в Мариуполе поселились лиса, летучая мышь и рыбы (ФОТО)}
\label{sec:29_09_2016.stz.news.ua.mrpl_city.1.ovik}
 
\Purl{https://mrpl.city/news/view/istoriya-o-tom-kak-v-mariupole-poselilis-lisa-letuchaya-mysh-i-ryby-foto}
\ifcmt
 author_begin
   author_id news.ua.mrpl_city
 author_end
\fi

\ifcmt
  ig https://i2.paste.pics/1fa13cb55370232d4e1e5ba8ce055736.png
  @wrap center
  @width 0.9
\fi

29 вересня 2016 в 14:00

Эта история о небольшой команде людей, которые решили сделать Мариуполь ярче.
\enquote{ОВИК} - так называют они себя среди своих. Аббревиатура образована от имен
преподавателя истории МГГУ Ольги Демидко, художника Вадима Ланового и юриста
Игоря Смеречинского. \enquote{Ольга, Вадим, Игорь и компания} - вот полная расшифровка.

В эксклюзивном интервью MRPL.CITY они рассказали о том, как и почему появилась
идея украшать Мариуполь детскими рисунками. 

\ii{29_09_2016.stz.news.ua.mrpl_city.1.ovik.pic.1.ovik}

Всего на их счету 6 красочных изображений, украшающих улицы города,
отреставрированный фасад Художественной школы им. Куинджи, альманах с эссе,
стихотворениями и рисунками, посвященными Мариуполю, раскраска-антистресс
\enquote{Розфарбуй місто} с историческими объектами города и еще множество
интереснейших проектов.

\ii{29_09_2016.stz.news.ua.mrpl_city.1.ovik.pic.2.rozfarbuj_misto}

\subsubsection{Все началось с \enquote{Мариуполь – это Украина}}

На вопрос, когда пришла идея преобразить родной город, команда рассказала
длинную историю. Она как нельзя лучше показывает, что на свете ничего не
происходит случайно.

Реализовывая проект \enquote{Мариуполь – это Украина}, Ольга Демидко влилась в
общественную жизнь.

\enquote{Мы провели конкурс среди учеников и студентов на знание истории родного края –
мы принимали стихотворения, эссе, рисунки, посвященные городу. У нас было 90
финалистов из учеников 5-11 классов и студенты. Результатом конкурса и проекта
стал альманах. Также провели несколько экскурсий по археологическим объектам},
- говорит она.

Чтобы показать жизнь, которая была до возникновения Мариуполя, была придумана
фотосессия в костюмах половцев. Их для детей воссоздала историк Наталя Чиркун.
Затем Ольга Демидко познакомилась с мариупольской художницей Оксаной Гнатышин.
Она предложила открыть в городе художественное училище. И хоть училище так и не
открыли, в результате этой идеи состоялись знакомства с Игорем Смеречинским и
художником Вадимом Лановым. Затем команда подружилась с Украинской
Миротворческой Школой, где поддержали реализацию конкурса \enquote{Малюємо заради
миру}.

\ii{29_09_2016.stz.news.ua.mrpl_city.1.ovik.pic.3.konkurs_deti}

Рассказывая о реставрационных работах в школе, Ольга Демидко немного смущается.
Ведь ей, хрупкой девушке, был не под силу тяжелый физический труд.

\ii{29_09_2016.stz.news.ua.mrpl_city.1.ovik.pic.4.1972}

\enquote{Школа была отреставрирована усилиями двух мужчин – Вадима Ланового и Игоря
Смеречинского. Я приходила к ним с пиццей, чтобы морально поддержать. Так, наша
команда сработалась}, - отмечает она.

Вадим Лановой признается: все было не так гладко, как хотелось бы. То мешали
погодные условия – май оказался холодным, то человеческий фактор давал о себе
знать. Переживали, что жилой дом перейдет в ОСМД и тогда работы будут
бессмысленны. В итоге работы начались только в июне.

\ii{29_09_2016.stz.news.ua.mrpl_city.1.ovik.pic.5.shkola}

\enquote{Сначала мы хотели выровнять фасад. Но здание историческое – сталинский ампир
50-х годов. Ребята посовещались и решили восстанавливать. И реставрационные
работы оказались самыми сложными - приходилось болгаркой вырезать, воссоздавать
разрушенные элементы. Нам было обидно, что за зданием никто не ухаживал:
во-первых, там дети, во-вторых – художественная школа, одна из немногих в
городе. Да и я сам когда-то в ней недолго учился. Идешь и думаешь: когда, когда
кто-нибудь обратит внимание и что-то сделает?}, - рассказывает он.

В итоге, кроме обновленного окрашенного фасада, на школе появились рисунки
\enquote{Голубь} Александры Бондаренко, морские мотивы Алексея Гусельникова,
\enquote{Подсолнух} Влада Курбакова и летающие шарики Евгения Щербака.

\subsubsection{Лиса и летучие мыши}

Широкую известность среди мариупольцев получил рисунок лисы на пересечении пр.
Нахимова и Итальянской. Яркий мурал притягивает внимание – горожане возле него
фотографируются и любуются им. Оказывается, случайности были не только в
знакомствах и дружбе, но и в творчестве.

По словам Игоря Смеречинского, рисунок школьницы Анны Харабат изначально должен
был быть нанесен на библиотеку им. Короленко. Но узнав, что здание будет
реставрироваться, решили не спешить.

\ii{29_09_2016.stz.news.ua.mrpl_city.1.ovik.pic.6.anna_habrat_lisa}

\enquote{Нам предложили забор по пр. Нахимова. Конечно, всех смущает, что за ним
развалины, но это уже – вопрос к хозяину}, - говорит Игорь.

\ii{29_09_2016.stz.news.ua.mrpl_city.1.ovik.pic.7.igor_lisa_zabor}

Затем, после яркой лисы ребята пошли дальше: Вадиму Лановому пришла идея
сделать океанариум в одном из дворов, чтобы скрасить игры детворы.

\enquote{Так как мы искали объект, в управлении культуры и туризма городского совета
нам предложили двор домов 113 и 115 по пр. Победы. Через него проходит
теплотрасса, все серое, мрачное. Недавно там поставили новые качели. Мы же
изобразили на теплотрассе обитателей моря, а на серой стене постройки – летучих
мышей Даниеллы Зубаревой}, - рассказывает Ольга Демидко.

\ii{29_09_2016.stz.news.ua.mrpl_city.1.ovik.pic.8.rebenok_stena_zajcy_derevo_pomidory}

По словам Вадима, эта их последняя работа вызвала большой интерес жителей. К
ним подходили дети и родители.

\ii{29_09_2016.stz.news.ua.mrpl_city.1.ovik.pic.9}

\enquote{Это нас вдохновило и воодушевило на дальнейшую деятельность. 99\% наблюдателей
были \enquote{за} и радовались. Дети гуляли, один мальчик даже пробовал вместе с нами
что-то делать}, - признается он.

I ♥ MRPL

Только любовь к родному городу, энтузиазм и жизнерадостность стали причиной
ярких добрых дел. Для команды важно, что в основе каждого мурала – рисунки юных
мариупольцев. Нарисованные работы всегда подписывают.

Вадим Лановой говорит, что разница между серыми, невзрачными поверхностями и
затем преображенными – велика.

\enquote{Ведь, всегда, когда появляются красивые вещи, они вызывают радость}, - делится
он.

\ii{29_09_2016.stz.news.ua.mrpl_city.1.ovik.pic.10}

А Ольге Демидко запомнились слова военного из Чернигова во время экскурсии на
день города: \enquote{Мы говорили с ним о муралах и фасадах. Он сказал:
\enquote{Когда мы заезжали, Мариуполь показался очень серым. Сейчас мы ходим по
нему, видим его и уже любим, потому что он яркий. Я сравниваю ваш город с
Черниговом, и он мне нравится уже больше моего города. Ведь здесь есть люди,
которые готовы что-то менять. У нас такого нет} . Такие отзывы вдохновляют}.

Планов у команды \enquote{ОВИК} множество. Они уверяют, что, если осенью погода
изменится, украсят в Мариуполе что-нибудь еще.
