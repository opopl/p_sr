% vim: keymap=russian-jcukenwin
%%beginhead 
 
%%file 12_01_2013.fb.fb_group.prosveschenie_radi_svobody.gruppa_vladimira_dubrovskogo.1.vlast
%%parent 12_01_2013
 
%%url https://www.facebook.com/groups/223130817802885/posts/327088357407130/
 
%%author_id fb_group.prosveschenie_radi_svobody.gruppa_vladimira_dubrovskogo,lukjanec_sergej
%%date 
 
%%tags obschestvo,slovo,vlast
%%title ВЛАСТЬ
 
%%endhead 
 
\subsection{ВЛАСТЬ}
\label{sec:12_01_2013.fb.fb_group.prosveschenie_radi_svobody.gruppa_vladimira_dubrovskogo.1.vlast}
 
\Purl{https://www.facebook.com/groups/223130817802885/posts/327088357407130/}
\ifcmt
 author_begin
   author_id fb_group.prosveschenie_radi_svobody.gruppa_vladimira_dubrovskogo,lukjanec_sergej
 author_end
\fi

ВЛАСТЬ

Слово и само понятие являются исконно славянскими (не заимствованными из
каких-то языков)

Суть,  вкладываемая в это понятие, появилась за долго до того, как появилось
современное понимание ПРАВ, СВОБОД и ОСОБЕННО ИХ РАВЕНСТВО ДЛЯ ВСЕХ.
Философское понимание ВЛАСТИ уходит глубоко корнями в ВАРВАРСТВО и уже из века
в века, из поколения в поколение , удерживает восточных славян в варварских
традициях. Если обратиться ко всевозможным словарям

\url{http://бравика.org/search/Власть}

очевидно, что ВЛАСТЬ есть абсолютная возможность тотально навязывать свою волю
большинству, как правило с помощью силы, по праву сильнейшего, как абсолютное
право находиться над ЗАКОНОМ, как абсолютная возможность устанавливать любые
ЗАКОНЫ для всех без оглядки на ПРИНЦИПЫ цивилизационные.

Так толкует Власть словарь Даля-«- ж. право, сила и воля над чем, свобода
действий и распоряжений; начальствование; управление;»

«ВЛАСТЬ, -и, мн. -и, -ей, ж. 1. Право и воз-можность распоряжатьсякем-чем-н.,
подчинять своей воле. ..»,-объясняет Ожегов.

Российский энциклопедический словарь :  «- власть, в общем смысле способность и
возможность оказывать определяющее воздействие на деятельность, поведение людей
с помощью каких-либо средств — воли, авторитета, права, насилия (родительская
власть, государственная, экономическая и др.); политическое господство, система
государственных органов.»

Уникальные трактовки дает Юридический словарь: «- - способность и возможность
осуществлять свою волю, оказывать определяющее воздействие на деятельность,
поведение людей с помощью таких инструментов, как авторитет, право, насилие,
даже вопреки сопротивлению, и независимо от того, на чем такие способность и
возможность основаны. Властвующий может достигать своих целей, действуя
различными методами: демократическими и авторитарными, честными и нечестными,
насилием и местью, обманом, провокациями, вымогательством, стимулированием и
т. д. В. как явление необходима. Она призвана обеспечивать потребности
человеческого общежития. Специфической разновидностью В. является политическая
власть - способность определенной социальной группы, страты (слоя) или
определенного класса осуществлять свою волю, оказывать определяющее
воздействие на деятельность, поведение людей посредством авторитета, права,
насилия.»  и   «Власть
- - особые общественные отношения господства и подчинения, при которых воля и
действия одних людей, организаций преобладает (доминирует) над волей и
действиями других людей, организаций (подвластными). Властвующие имеют
возможность повелевать. В первобытном обществе В. носила сугубо общественный
характер. Такими властными структурами являлись родовые органы управления.
Всякая В. реализуется в определенных отношениях - властных отношениях.
Потребность в установлении властных отношений возникает в силу необходимости
регулирования поведения людей. В. старейшин и вождей основывалась исключительно
на авторитете, на глубоком уважении всех членов рода к старшим, их опыту,
мудрости, храбрости охотников, воинов. Огромную роль в родовой общине играли
обычаи, с помощью которых регулировалась жизнедеятельность рода и его членов. С
появлением государства организация и осуществление В. сосредоточиваются в руках
специальных органов, постоянно занимающихся управлением общественными делами,
т.е. осуществляется через "аппарат власти" и основывается на институтах
организованного, государственного принуждения. В. призвана служить обществу,
обеспечивать его целостность, надлежащее функционирование, служить личности,
обеспечивая и охраняя права и свободы граждан.», а так же: «Власть
- - возможность управлять, командовать, распоряжаться людьми, материальными
объектами, территориями. Осуществляется посредством авторитета, слова, акта,
действия, в том числе принуждения (см. также: Государственная власть.
Законодательная власть. Исполнительная власть. Политическая власть.
Непосредственная демократия. Представительная демократия. Судебная власть). (С.
А.)»

Еще интересней трактовка Словаря русских синонимов: 

«- Власть, владычество, господство, держава, сила; могущество, полномочие,
право (полное); престол, царство. Бразды правления. Конечно, Царь, сильна твоя
держава. Пушк. Все дело в его руке. Власти, правительство, начальство,
администрация, правление, управа. Власти предержа-щие. Ср. Главенство, Сила,
">Право Начальник. господство, господствовать, право, сила быть во власти
кого-л. быть под властью власти во власти иметь под своею властью давать полную
власть ни в чьей-л. власти превысить власть у власти">предоставлять полную
власть, у власти».

Ну и наконец довесок из Википедии: «Вла́сть — возможность и способность
навязать свою волю, воздействовать на деятельность и поведение других
людей, даже вопреки их сопротивлению[1]. Суть власти не зависит от того,
на чём основана такая возможность. Власть может базироваться на различных
методах: демократических и авторитарных, честных и нечестных, насилии и
мести, обмане, провокациях, вымогательстве, стимулировании, обещаниях и т.
д.[2]

Считается, что власть появилась с возникновением человеческого общества и будет
в той или иной форме всегда сопутствовать его развитию. Она необходима для
организации общественного производства, которое требует подчинения всех
участников единой воле, а также для регулирования других взаимоотношений между
людьми в обществе.[3]

Специфической разновидностью является политическая власть — способность
определённой социальной группы или класса осуществлять свою волю, оказывать
воздействие на деятельность других социальных групп или классов. В отличие от
иных видов власти (семейной, общественной и др.), политическая власть оказывает
своё влияние на большие группы людей, использует в этих целях специально
созданный аппарат и специфические средства. Наиболее сильным элементом
политической власти является государство и система государственных органов,
реализующих государственную власть

Но проблема, конечно, не в словарях, они лишь отражают определенное
философское восприятие масс. Проблема в том, что массы готовы соглашаться с
этой философской трактовкой, готовы отдаваться сильному и соглашаться с
навязываемыми условиями, а меньшая часть, тех же масс, соглашаясь с этой
философией, готова вступить в борьбу на тех же условиях, ради тех же целей.  

Таким образом, пока не измениться философское понимание ВЛАСТИ, пока сила и
воля сильнейшего будут над ПРИНЦИПАМИ и над законами, пока понятие Власть не
канет в лету, а на его место придет  ИНТЕЛЛЕКТУАЛЬНОЕ УПРАВЛЕНИЕ,
подчиненное ПРИНЦИПАМ и ЗАКОНАМ, контролируемое и не абсолютное, все
конституции, как общественные договора, вся система права, будут формальными
и фиктивными. Таким образом, не буду отрицать, что утверждение составителей
Википедии, что Власть "будет в той или иной форме всегда сопутствовать его
развитию", будет, но в другом философском понимании, с иными функциями и на
иных принципах, а общности, которые не понимают этого будут волочиться за
прогрессом и стоять на грани утраты государственности, интеллектуальной
деградации своих членов.

Очень опасно Главное заблуждение, что ВЛАСТЬ может принадлежать народу (а в
Украине, так это полный абсурд, народу не принадлежит НИЧЕГО, не только
власть), этого не может быть по определению тех же словарей, хотя об этом
прямо не говориться. Абсолютное заблуждение, что народ может чем-то
управлять, потому что управление процесс интеллектуальный, требующий
глубоких специальных знаний в многих сферах, ведь никто не идет к
электрику для того, чтобы пошить костюм, а к портному, чтобы провести
свет, так же абсурдно управление невеждами (людьми не имеющими знаний). 

В завершении хочу привести письмо посвященное первому выходу передачи
Политклуб на TVi. В который я постарался вложить определенный философский
смысл: «...вчерашний первый эфир Политклуба нельзя считать первым
блином, который комом. Много звучало здравых мыслей, но в передаче
отсутствовала фундаментальность мысли, именно то, чего нет в элитном
политикуме, и ждать от них подобного не стоит. Основной задачей проекта, я
предполагаю, является пропаганда(разъяснение демократических ценностей, однако
во многих эфирах и на интернет ресурсах эта тема также основная, но понятие
демократии и демократических ценностей трактуется по-разному и
неоднозначно. Получается, как у хорошей собаки, понимают все, а точно рассказать
не могут. Поэтому очень важно прийти к одному знаменателю, особенно важно для
тех, кто собирается нести суть принципов в массы. Для этого нужно обратиться к
античным классикам. Платон и Аристотель рассматривали ДЕМОКРАТИЮ как одну из
форм государственного управления, причем форму не совсем правильную, при этом
считая правильными формами автократию и аристократию. Аристократию не следует
отождествлять с олигархией. 

Выделяя демократию, как самую справедливую форму,
древние мыслители уже тогда отмечали недостаточную ее
жизнеспособность. Обязательным условием демократического уклада является
значительное число в обществе СВОБОДНЫХ людей, обладающих некоторыми правами и
свободами, при чем равенство всех во всех свободах и правах не
предусматривалось, а лишь в некоторых. В этой связи выскажу вообще несколько
крамольную мысль: ДЕМОКРАТИЯ не является формой правления вообще. Если
считать, что демократическое общество путем демократических же выборов
делегирует определенные полномочия кому-либо, то это по сути является
ВЫБОРОМ(ЭЛИТЫ) СВОЕОБРАЗНОЙ АРИСТОКРАТИИ, АРИСТОКРАТИИ на определенную
каденцию. С момента оглашения результатов выборов общество становится
АРИСТОКРАТИЧЕСКИ управляемым. Но с античных времен утекло много времени и
сегодня стоит говорить о демократии как о наборе принципов, сформированных
обществом на основе, существующей в нем морали, о возможности общества
контролировать избранную временную АРИСТОКРАТИЮ в том, как она в своих
функциях придерживается обозначенных обществом принципов. Наша демократия
должна начинаться с РАВЕНСТВА. В первой передаче очень точно были приведены
примеры Бен-Гуриона, В. Гавела и других. Очень важно понимать тем, кто собирается
объяснять людям, усвоить самим, что РАВЕНСТВО-не всеобъемлющее, что невозможно
выровнять раз и на всегда людей по материальному
достатку, интеллекту,.... РАВЕНСТВО в обществе, без «постоянных репрессивных
"подравниваний" возможно в ПРАВАХ и СВОБОДАХ, в равных возможностях
реализовать свои права и свободы .На этой основе и должны строиться
государственные демократические институты, включая судебную систему. Если
демократия делегирует определенные полномочия ВРЕМЕННОЙ АРИСТОКРАТИИ для
управления и осуществления общественно-регуляторных функций, то встает вопрос
качества ВРЕМЕННОЙ АРИСТОКРАТИИ. Если ДЕМОКРАТИЯ не будет способна избрать
интеллектуальную аристократию, отдавая предпочтение или соглашаясь с другими
ее качествами, то она будет подтверждать выводы древних о своей временности и
неэффективности. Процесс управления-процесс высокоинтеллектуальный, требующий
знаний, знаний системных, требующий стремления к познанию нового, к
совершенствованию. Сегодняшняя ЭЛИТА не является интеллектуальной. Она ввела
материальный ценз для возможности быть избранным, оттеснив потенциальную
интеллектуальную элиту. Для будущего развития Украины очень важным является
замена существующего МАТЕРИАЛЬНОГО ЦЕНЗА на ЦЕНЗ интеллектуальный. Необходим
КУЛЬТ ЗНАНИЙ, который придет на смену культу силы и наглости, культу формальных
дипломов. Демократия будет обречена, если не будут точно сформулированы
демократические принципы как рамки, в которых общество через механизмы
контроля, обеспеченные СВОБОДОЙ сможет контролировать избранную им же
временную управляющую АРИСТОКРАТИЮ.»

\begin{itemize} % {
\iusr{Константин Соловьев}

У Хосе Ортега-и-Гассет я встречал удивительную по глубине мысль о том, что
народ это нация организованная аристократией ( Духовной аристократией,
связывающей народы и эпохи). Не менее интересные мысля я встречал у С.С.
Алексеева, какой выделял разновидности власти, как то власть сильного, сласть
силы ( доминирующей группы), власть государства ( византийство, СССР) и власть
Гражданского общества, находящегося в правовом состоянии ( по сути верховенство
права). Правда, при существующей норме принципе ( ст. 19 Конституции) власть
имеет легитимный характер при условии если действует на основании и способом
предусмотренным Конституцией и законами Украины. Я давно пишу о том, что
современные диктатуры, тирании, фашизмы и тому подобное, т.е. власть силы,
власть государства, какое принадлежит силам - осуществляются посредством
неправовых законов. ( народ учится на примерах, а не познании ( теориях),
дающих видеть ( умозрительно, предусмотрительно) - будущее. Тиранические формы
власти, наряду с парадностью и лицедейством благопристойности это тайные
сообщества ( общаки, подельники), где задумываются и осуществляются
преступления, за счет чего власть имущие удовлетворяют свои ( в основном
жлобские интересы потребительского уровня). В зависимости от моральных качеств
тех, кто представляет собой власть, - уровень преступной составляющей во власти
определяется соответствующим большинством. Если больше криминалитета - то
больше негатива ( преступности) во власти. Если меньше, есть возможность
минимизировать криминалитет. Это банальные истины, но их почему то не замечают,
поскольку в массе не читают уголовный кодекс и толковые учебники по уголовному
праву. Переход в понимании власти, как данности, т.е. силы к необходимости
власти, как должного ( на современном уровне верховенства права, по сути власти
Гражданского общества) требует усвоения принципиальный различий таких типов
власти и политической воли масс ( большинства). А это означает необходимость и
многоплановые усилия той части граждан, какие смогут исполнить миссию
аристократии. На этот счет у меня были заметки:

\url{http://ciacivicua.org/artikles/5-nauka/69-2012-05-30-01-15-11.html}

\iusr{Константин Соловьев}

Ув. Сергей! Я, как и Вы всегда пытаюсь найти возможность пообщаться с
политиками. Конечно, всегда бесполезно, поскольку они практически всегда "
заняты" настолько, что им некогда думать. Нет они конечно думают и быстро
соображают. Но, делают они это весьма поверхностно, легковесно. легковесность
не новое свойство так называемых "образованных", но митрофанов по сути. ( см. 

\href{https://rvb.ru/saltykov-shchedrin/01text/vol_07/01text/0189.htm}{%
ЛЕГКОВЕСНЫЕ, М. Е. Салтыков-Щедрин, rvb.ru%
}

). И это "причина многих наших несчастий" Даже здесь на ФБ иногда бывают
случаи, когда затронутые проблемы рождают мысли. Конечно, на уровне ФБ трудно
развить суждения и просвещения - относительно природы власти, правления,
управления, регулирования и т.д., что связано с общественными процессами, в том
числе отношениями власти и граждан. Пока еще не удалось выйти на какой либо
рабочий уровень рабочих "отношений". Увы!

\end{itemize} % }
