% vim: keymap=russian-jcukenwin
%%beginhead 
 
%%file poetry.rus.evgenija_bilchenko.gorod_geroi
%%parent poetry.rus.evgenija_bilchenko
 
%%url https://t.me/bilchenko_z/156
%%author 
%%tags 
%%title 
 
%%endhead 

\subsubsection{БЖ. Город-герой}
\label{sec:poetry.rus.evgenija_bilchenko.gorod_geroi}
\Purl{https://t.me/bilchenko_z/156}

Никогда не спрашивай у меня формальности ради: "Как дела?", -
Если я болею, бодрюсь, храбрюсь, трясусь, как трусло и гадина.
Город мой, насмерть дождём пробитый, в намордниках гонит в арку тела:
Они там курят и ждут, мусоля гаджеты. Я ж в Аркадию

Хочу ехать на блаблакаре: на автобусе не тяну уже.
В Аркадии - море, песочек, чайки и всё запредельно нужное,
Прочищающее духовным ИВЛ выдыхающие натужно
Дым бензина и никотина лёгкие, жизнью суженные

До предела последней рифмы. Остаточной монографии.
Я научилась писать сквозным единым потоком трафика.
Биография - это бред по сравненью с движеньем рафика.
Картину моря творит Христос: синих артерий графику.

Бог пишет иконы на веках, худых медицинских венах.
Я видела эту Европу: ни с Берлином её, ни с Веною
У меня не сложилось. Я - слишком русская, слишком славянско-генная,
Слишком советская для эпохи постмодернистских гениев.

Может, они и гении. И прогрессом в пустоты катятся.
Но мне милее песня Шульженко "Закурим" и та, про Катеньку,
Где сизый орёл превращает небо в Левитановский купол радио
И где по утрам, простужаясь, море кашляет на Аркадию.

1 ноября 2020 г.
