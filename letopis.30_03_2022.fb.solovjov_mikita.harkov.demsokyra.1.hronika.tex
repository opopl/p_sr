% vim: keymap=russian-jcukenwin
%%beginhead 
 
%%file 30_03_2022.fb.solovjov_mikita.harkov.demsokyra.1.hronika
%%parent 30_03_2022
 
%%url https://www.facebook.com/Mikita.Solovyov/posts/7328610360542647
 
%%author_id solovjov_mikita.harkov.demsokyra
%%date 
 
%%tags 
%%title Хроника Харькова, 30-е марта
 
%%endhead 
 
\subsection{Хроника Харькова, 30-е марта}
\label{sec:30_03_2022.fb.solovjov_mikita.harkov.demsokyra.1.hronika}
 
\Purl{https://www.facebook.com/Mikita.Solovyov/posts/7328610360542647}
\ifcmt
 author_begin
   author_id solovjov_mikita.harkov.demsokyra
 author_end
\fi

Хроника Харькова, 30-е марта. 

Обстрелов очень много за последние сутки. Если перед этим несколько дней
количество обстрелов шло на убыль, то с вчерашнего вечера как с цепи сорвались.
Обстреливают много, обстреливают разным. Куча обстрелов Градами, ствольной
артой и минометами окраин. Салтовка (не только северная), Жуки, Пятихатки
(особенно сильно), ХТЗ, Холодная гора, Алексеевка. Точнее называть нельзя.

Много прилетов тяжелого, скорее всего ракет. Причем ПВО явно эффективно
работает и это только прорвавшиеся ракеты. Был после долгого перерыва одинокий
самолет. Удар нанести успел, а вот улететь нет, насколько я понимаю. Несколько
достаточно крупных пожаров. Пока все пожары удавалось достаточно оперативно
тушить. Но вот опять горит сильно, буквально час-два назад очередной начался.

С коммунальной сферой на этом фоне все на удивление прилично. То есть я бы не
удивился при таких обстрелах новым крупным отключениям света, воды, газа и т.д.
Но их нет. Коммунальщики и спасатели делают не просто все возможное, а намного
больше моих представлений о возможном. И тем более моего о них довоенного
мнения, если уж быть честным.

Обстрелы идут не только по городу, но и по многим населенным пунктам области.
Как пригородам Харькова, так и более отдаленным. Сильные обстрелы в Дергачах и
Чугуеве. Не только там, но об остальных пока в паблике не было, кажется.
Разрушений много, пострадавших в этот раз почти нет. Жертв нет.

При этом наземную активность русские проявляют только в районе Изюма. В
пригородах и ближайших окрестностях Харькова они уже особо и не пытаются. Зато
ЗСУ постепенно их переселяют с украинской земли в украинскую землю.
Продолжаются постепенно зачистки пригородов. Но пока у врага остается
достаточно контролируемых зон, чтобы обстреливать Харьков из вооружений
ближнего радиуса.

Кроме не очень значительного установления контроля над территорией, ЗСУ весьма
активно борются с техникой врага. Сегодня очередную партию техники не просто
уничтожили, а смогли отжать. По тяжелой технике, в том числе по танкам, на
харьковском направлении отжато ЗСУ было с начала войны намного больше техники,
чем потеряно в боях. Захвачена еще одна установка РЭБ. На изюмском направлении
сбито за сутки еще два самолета и беспилотник (не знаю один или нет).

Обычно я не очень много значения придаю словам и заявлениям наших властей. Но
вот сегодня прямо бальзам на душу. Синегубов много говорил о перезапуске
экономики. Причем в Харькове даже было в этой связи какое-то выездное
заседание. Я очень рад, что этот важнейший вопрос наконец-то попал в фокус
внимания властей. Хотя обычно я считаю, что лучшая помощь бизнесу со стороны
государства это не мешать, а тихонечко в лес пойти, но сейчас ситуация иная. Во
многих вопросах бизнесу без помощи государства не обойтись. Например, сейчас
без прямого содействия властей никак не обеспечить логистику по ж/д и таких
вопросов много.

Что касается Харькова, то я все так же считаю ключевым вопросом для
возобновления работы бизнеса запуск общественного транспорта. В первую очередь
метро. Потому что если бизнес еще как-то в состоянии наладить себе логистику
товаров и сырья, то вот обеспечить подвоз сотрудников уже нет. А метро самый
эффективный и безопасный вид общественного транспорта. Но для этого нужно
начать использовать его по прямому назначению. 

Да, и моя постоянная рубрика  @igg{fbicon.smile}  Нашел в центральном районе
города еще одно открывшееся кафе. Адрес пока не называю по их просьбе. Две
заработавшие пиццерии в режиме доставки и один открывшийся магазин. Это за день
и только из известного мне. Город начинает возвращаться к жизни.

Харьков стоит!

Слава Украине!

Низкий поклон нашим защитникам!

\textbf{\#ХроникаХарькова}

\ii{30_03_2022.fb.solovjov_mikita.harkov.demsokyra.1.hronika.cmt}
\ii{30_03_2022.fb.solovjov_mikita.harkov.demsokyra.1.hronika.cmtx}
