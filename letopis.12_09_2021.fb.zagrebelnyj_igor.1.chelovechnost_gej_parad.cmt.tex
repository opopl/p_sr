% vim: keymap=russian-jcukenwin
%%beginhead 
 
%%file 12_09_2021.fb.zagrebelnyj_igor.1.chelovechnost_gej_parad.cmt
%%parent 12_09_2021.fb.zagrebelnyj_igor.1.chelovechnost_gej_parad
 
%%url 
 
%%author_id 
%%date 
 
%%tags 
%%title 
 
%%endhead 
\subsubsection{Коментарі}

\begin{itemize} % {
\iusr{Валентин Шермірзаєв}
Тобто ми нелюди чи що, а гомошизофреніки людяні?

\begin{itemize} % {
\iusr{Igor Ivanchenko}
\textbf{Валентин Шермірзаєв} ми фошизди

\iusr{Igor Ivanchenko}
\textbf{Валентин Шермірзаєв} алє прікол в тому шо горжусь бути фошиздом
\end{itemize} % }

\iusr{Роман Богданов}
Ми з вами це застанемо.
Третій Храм уже будується.
Уже скоро.

\begin{itemize} % {
\iusr{Oleh Kaminskiy}
\textbf{Роман Богданов} Чим скоріше це все станеться ,тим швидше прийде Христос)Хто витримає до кінця ,той спасеться.
\end{itemize} % }

\iusr{Юрій Хорт}
"Самий чєловєчний чєловєк" - відомо хто)

\iusr{Олеся Горгота}
На рахунок цього є прекрасна цитата Аль Пачіно у фільмі "Адвокат диявола", коли диявол вже розкрив себе перед адвокатом, то говорить: "Я - найбільший гуманіст у світі! Я найбільше люблю людей! Саме я даю людині все, що їй потрібно!"
Чому дивуватися?

\begin{itemize} % {
\iusr{Євгенія Мироненко}
\textbf{Олеся Горгота} кіномистецтво вміє увести, запутати, внушити. та, певно, як і інші види мистецтва
\end{itemize} % }

\iusr{Ігор Шандрігон}

гуманізм від початку був профанацією, антонімом до гуманності. мені найкращий
символ Антихриста - це чоловічок АHA

\begin{itemize} % {
\iusr{Ігор Шандрігон}

\ifcmt
  ig https://scontent-mxp1-1.xx.fbcdn.net/v/t1.6435-9/241732609_4371252006325599_6546210682373354290_n.jpg?_nc_cat=107&ccb=1-5&_nc_sid=dbeb18&_nc_ohc=JKGmfZkZERsAX_KpfWA&_nc_ht=scontent-mxp1-1.xx&oh=b970629f9ffda59b8b782bb609b4ccee&oe=61919AA6
  @width 0.4
\fi

\end{itemize} % }

\iusr{Oleh Kaminskiy}
Ідеологія ЛГБТ і є злочином проти людяності.

\iusr{Ігор Демур}
Творець/Бог визначає міру кожній речі

\iusr{Andrii Bezuhlyi}

Точніше: толерантність. Толерантність до всього нового та нищення старого. Це
"толерантний футуризм", що котиться за інерцією до своїх збочених форм...

І це закономірно, в епоху без війн в Європі.

\iusr{Євгенія Мироненко}

\obeycr
от-от, є дж поняття "злочин проти людяності".
те, що ти маєш на увазі- я вловила, але я б не поспішала списувати слово "людяність".
той же Василь Симоненко, очевидно, захопливо писав про ЛЮДИНУ.
Я сама часто використовую це поняття, що "людяність" - це універсально, тобто: бути добрим, намагатися зрозуміти іншу людину, розібратись в мотивах, вчиняти добросовісно.
проте не втрачати здорового глузду- називати речі своїми іменами.
\restorecr

\begin{itemize} % {
\iusr{Ігор Загребельний}
\textbf{Євгенія Мироненко} 

"бути добрим, намагатися зрозуміти іншу людину, розібратись в мотивах, вчиняти
добросовісно". З чого це випливає? Звісно, люди так чи інакше добрі. Навіть
люди, яких заведено вважати дуже лихими, відзначаються якоюсь добротою. Але
водночас усі люди по-своєму є злими. Чому "бути людяним" - це бути добрим, а не
бути злим? Чому така вибірковість?

\iusr{Irina Shuhtueva}
\textbf{Євгенія Мироненко} тому щоби бути добрым треба вольове рішення, а щоби злим - ні)

\iusr{Євгенія Мироненко}
\textbf{Ігор Загребельний} 

я не знаю, чому повелося так вважати. можна , звісно, переглянути цей термін,
але нічого шкідливого в цьому не бачу.

всі добрі люди часом вчиняють зле, але різниця в тому, що добра людина може
визнати і виправити, має на це силу характеру, а дріб"язкові характери намірено
чинять лихе, чи то в захист, чи то від заздрощів, чи то від власної болі.
Напевно тому так спрощено за поняттям "людяність" мається на увазі сила
характеру, бо доброта- одна з найточніших ознак сили характеру. Бути добрим
часто більше ресурсозатратно  @igg{fbicon.smile} 

Цікаву ти тему зачепив, це може бути довга дискусія. Нариклад, чи є вибір: бути
добрим чи бути злим  @igg{fbicon.smile} 

Проте тут я маю на увазі загальноприйняте поняття "людяність", яке переважно
сприймається як щось добре, світле, надійне.

\iusr{Євгенія Мироненко}
\textbf{Irina Shuhtueva} щось в цьому є. справді)
\end{itemize} % }

\iusr{Іван Петрущак}
ЛГБТ це комуністи, бо вони домагаються щоб людей саджали за їх критику в тюрму.. і при цьому вони умудряються сміятись з інших країн (Афган) де схожі закони проти содомії

\iusr{Юрій Хорт}
Підказка

\ifcmt
  ig https://scontent-mxp1-1.xx.fbcdn.net/v/t1.6435-9/241744846_217905880381009_4153897590570742743_n.jpg?_nc_cat=109&ccb=1-5&_nc_sid=dbeb18&_nc_ohc=PfgKVaqaHUcAX_1pF0W&_nc_ht=scontent-mxp1-1.xx&oh=be060771f1c7b09dfe67d12dae1667ea&oe=61919FC2
  @width 0.4
\fi

\iusr{Погоріленко Артур}
Майже геніально))

\iusr{Марина Хаперська}
Так а як же християнське милосердя та заклик любити ближнього?

\begin{itemize} % {
\iusr{Ігор Загребельний}
\textbf{Марина Хаперська} 

Який стосунок християнське милосердя та заклик любити ближнього мають до
гуманізму? Християнське милосердя випливає передусім із любові до Бога й існує
в світлі усвідомлення цінності людини, яке диктує релігія. Етика відносин з
іншими людьми легітимізується загальною теоцентричною метафізикою.

\iusr{Марина Хаперська}
\textbf{Ігор Загребельний} та я розумію, що людина завжди знайде собі виправдання і всі догми під себе підлаштує))

\iusr{Ігор Загребельний}
\textbf{Марина Хаперська} Догми передують людині, вони вищі за мене, Вас, усе людство.
\end{itemize} % }

% -------------------------------------
\ii{fbauth.jasenec_jaroslav.kiev.ukraina.pcu.svjaschennik}
% -------------------------------------

Якщо ці істоти - атеїсти, то для них людина - тільки випадковий продукт
випадкової еполюції, немає безсмертної душі та сенсу буття. Приб'єш його -
нічого не зміниться((((

\begin{itemize} % {
\iusr{Ігор Загребельний}

Зазвичай люди, які заперечують Бога й об'єктивність моралі, спалахують гнівом,
коли під час дискусії моделюється ситуація застосування щодо них їхніх же
принципів.

\iusr{Yaroslav Yasenets}
\textbf{Ігор Загребельний} Так. Улюблена фраза: "Ви не розумієте, ЦЕ ІНША РІЧ". Це вже мемчиком стало.
Щоб зрозуміти сенс будь-якої ідеї, треба довести її до кінця, змоделювати світ, в якому ця ідея панівна цілком, а не наполовину. І тоді подумати, чи хочеться в ньому жити
\end{itemize} % }

\iusr{Ігор Спокій}

Якщо людина має душу, а душа може бути одухотворена Животворящим Святим Духом,
то людина - теоцентрична істота. Мавпи і тваринки інші також можуть бути
людяними...

\iusr{Irina Shuhtueva}

Якраз в сьогоднішньому читанні Ісус говорить Петру - ти говориш про людське, а
не про Боже. Геть від мене, сатано!‘ в проповіді почула - що тут якраз йдеться
про атеїстичний гуманізм, де в центрі всього людина, а не Бог. І тому та
‚душа‘, чи самість, або індивідуальність, яку людина намагається зберегти усіма
силами - вона і губить людину в людині...

\iusr{Андрій Подільський}
А якщо вони - біороботи (без свідомості і глузду розрізняти добро і зло)!??

\iusr{Yury Likhota}
ось так охороняли "людяність" у Харкові

\ifcmt
  ig https://scontent-mxp1-1.xx.fbcdn.net/v/t39.30808-6/241807630_1000220824154076_10479552576710431_n.jpg?_nc_cat=103&ccb=1-5&_nc_sid=dbeb18&_nc_ohc=IUGR9Sz9rQsAX-hgLL8&_nc_ht=scontent-mxp1-1.xx&oh=f2aa142e4542ba021099dd2c97c6de49&oe=6172313B
  @width 0.4
\fi

\begin{itemize} % {
\iusr{Микола Пекний}
Радує, що ця "людяність" ніде не може відбуватися без такої охорони.

\iusr{Yury Likhota}
\textbf{Микола Пекний} покищо(((
\end{itemize} % }

\iusr{Oleksandr Rd}
комуняки.

\iusr{Іван Сута}
Людяність і сатанизм - різні речі! Самці тварин, сексуально не "трахаються".

\iusr{Homa Pryskiplyvy Pryskiplyvy}
Біблійний епізод - був Богом посланий дощ горячої сірки на мешканців міст Содом і Гомора

\iusr{Сергій Комаренко}

Це гасло прикриття знищення людяності і людства з його закономірностями та
цінностями. До того ж це відоме гасло грабіжника "тримайте грабіжника". У
гомодиктатури або гомоагресії немає людяності, хіба що її агресія і ненависть
до всього традиційного і здорового.

\iusr{Олег Гриб}

Є гріх,і є грішник, я проти гріха, я не проти грішника - це і є людяність. Але
бути гордим тим що я грішник або чиню гріх - вибачте, про яку людяність мова?
Якщо вже копати глибше - любов є тим що сатана намагається спотворювати як
умога сильніше, бо любов фундвмент віри, фундамент здорової сім’ї. Любов і
відносини між чоловіком і жінкоє є благословенні Богом, але є і інше, любов
спотворена сатаною, сатана вкладує до людських сердець жахливе: содомію,
педофілію, зоофілію, нікрофілію... Сьогодні світом крокує содомія, на черзі
легалізація педофілії (дзвіночки вже є), а що завтра, нікрофілія? Теж такими
народилися і має викладатися у школах??? Впевнений, бо знаю приклад, що Бог
зціліть і змінить всіх хто до Нього щиро звернеться за допомогою, і те що
сатана вкладує у серця людей - зникає як темрява на сонці. Є лише один шлях,
шлях через Ісуса Христа, через Його безмежну і святу любов до нас. Горе тим хто
цього не розуміє.

\end{itemize} % }


