% vim: keymap=russian-jcukenwin
%%beginhead 
 
%%file 14_11_2021.stz.news.ua.tyzhden.1.znaki_skovoroda
%%parent 14_11_2021
 
%%url 
 
%%author_id ljutyj_taras
%%date 
 
%%tags 
%%title Розкодовуючи знаки
 
%%endhead 
\subsection{Розкодовуючи знаки}
\label{sec:14_11_2021.stz.news.ua.tyzhden.1.znaki_skovoroda}

\ifcmt
 author_begin
   author_id ljutyj_taras
 author_end
\fi

Матеріал друкованого видання \textSelect{№ 45 (729) від 11 листопада}

\textSelect{Останнім часом, розміщуючи написані для Тижня колонки про Григорія Сковороду у
фейсбук, я ставлю гештеги, що вказують на впливи, яких зазнав наш любомудр.}

Серед них є прямі й непрямі. Зараз я хочу навести найочевидніші. Для цього
звернуся до розвідки італійської славістки Марії Ґрації Бартоліні «Пізнай
самого себе». Дослідниця відстежує неоплатонічні джерела творчості Сковороди й
за центральну точку відліку бере ідейний задум Плотіна, доповнений
юдео-християнськими візіями Александрійської школи (Філон, Климент, Оріген),
апофатичної теології (Діонісій Ареопагіт, Максим Сповідник), кападокійців
(Василій Великий, Григорій Назіанзин, Григорій Нисський), Авґустина, Йоана
Златоуста й інших. Авторка зазначає, що в українській літературно-бароковій
традиції поєднання філософських і релігійно-містичних аспектів було чимось геть
звичним, і розкриває концептуальні моменти, які спонукали Сковороду по-своєму
адаптувати це вчення. У засаді його розмислу лежить теза, що кожна людина
проходить випробування, позначені як дияволові наскоки. Протистояти вторгненню
можна лише якщо очистити розум в акті споглядання Бога. У пригоді тут і стає
неоплатонічна доктрина.

Згідно з Плотіном, Єдине — це всеосяжне начало Всесвіту, яке шириться врозтіч,
щоб знову повернутися до витоку. Найвіддаленіші терени, де інтенсивність його
поширення спадає, можна порівняти з матерією. Розпізнавши в собі дієву
розумність, людина здатна виборсатися з цієї твані. У християнстві Єдиному
відповідає Бог, матерії — тлінний світ, а Христос є посередником між ними й
посланцем одкровення про спасіння. Людська драма полягає в перебуванні на
віддаленій периферії від благого Господа. Святі отці навчали: демонічні напади,
а вони неабияк схожі на панічні атаки, з’являються тоді, коли людині здається,
що для врівноваженого існування їй більше не потрібна єдність із Богом. А тому
вона й узалежнюється від якогось із численних дрібних бісів.

Читайте також: \href{https://tyzhden.ua/Columns/50/253562}{Наука вільних %
мулярів, Тарас Лютий, tyzhden.ua, 07.11.2021}

Одначе втіленого зла не існує, воно виникає через відпадання від добра та
дезорієнтує людину. Як показує Бартоліні, гаспидські витівки схожі на крадіжку.
Нечистий постає як трикстер, якому до снаги зводити наклепи, заплутувати,
вдаватися до підміни й ціннісного перевороту. Так Ориґен ототожнює чортівню з
грабіжництвом. Тільки-но душа позбудеться самовладання, їй легко піддатися
спокусі. За ним і Сковорода визнає, що чортяче викрадення думки спричиняється
до втрати хисту чинити морально. Вражене супостатовими вибриками мислення
скидається на навіженство, марево, імітацію, перетворюється на машинерію
(тенета, пастка, ятір) і призводить до махінації. Додаймо, що все це є
різновидом симуляції. Відтак спотворені думки обертаються пристрастями. І
людина невдовзі знудиться, не знаючи, як їх спекатися.

\begin{zznagolos}
Сковорода наполягає на боротьбі з бісом нудьги. Бо щойно гріховні помисли
потраплять у серце, яке у східному християнстві ототожнено з розумом, стане
несила вистояти в натиску журби. У безодні серця вирує ціле сонмище дум. І не
всі вони втішні, а надто коли зіпсовані дідьком, читай — власним зухвальством 	
\end{zznagolos}

Як і Ніцше, який через століття казав про «веселу науку» подолання страждань,
Сковорода наполягає на боротьбі з бісом нудьги. Бо щойно гріховні помисли
потраплять у серце, яке у східному християнстві ототожнено з розумом, стане
несила вистояти в натиску журби. У безодні серця вирує ціле сонмище дум. І не
всі вони втішні, а надто коли зіпсовані дідьком, читай — власним зухвальством.
Отже, поглянути на себе означає відкрити простір внутрішнього світу та
зрозуміти, де хибиш. Слід збудувати не паноптикум самопригнічення, а
перетворитися на готового до пильнування себе вартового власного серця.

Постійне віддалення від Єдиного (Бога) означає поступове розпадання,
потрапляння в закамарки світу, де плодяться нечестиві думки. Бартоліні вживає
метафору спускання з височини донизу та знову підйому вгору або почергового
відцентрового й доцентрового руху. Зрештою, будь-які мандри мають завершитися
прямуванням до першоджерела, а відхилення — виправленням порушеної єдності. Ще
недостатньо відсторонитися, щоб збагнути свою окремішність, адже вона
допроваджуватиме до пригніченості. Осягання себе увінчується замиканням кола й
поверненням у вічність. Отже, дистанціювання та єднання створюють головні
фігури такого діалектичного шляху. Хто опанував його, той і є мудрецем.

Читайте також: \href{https://tyzhden.ua/Columns/50/253010}{%
Повсякдення Гетьманщини, Тарас Лютий, tyzhden.ua, 09.09.2021}

Але як розбудити зацькований демонами розум? До того ж жодні буквальні вказівки
не скерують до невидимого центру. Доводиться розшифровувати коди. Біблійні
тексти є зібранням таких знаків. Одначе це не мапа, якою користуються
механічно. Потрібна інтерпретація. Недаремно Сковорода водночас є поетом і
філософом. З низки таємних повідомлень необхідно сконструювати силу-силенну
нових змістових дороговказів. У цьому й полягає мистецтво творення. Ситуацію
описує давня легенда. Якогось схимника так зачарував прегарний спів чудного
птаха, що не міг відірватися від тих звуків упродовж трьохсот років. Хоч
скільки лови того птаха, нізащо не впіймаєш. Він недосяжний, як і сам Бог.
Прийти до Бога не означає привласнити. Кожне наближення не означає остаточного
прибуття до пункту призначення. Тому й мудрування не дає практичних
результатів. За його допомогою хіба підійматимешся вгору, а з тієї височини ще
більше чудуватимешся, не втомлюючись від постійної забави, не задовольняючи
бажання блага достоту, досягаючи невимовного щастя в невичерпності таємниці. 
