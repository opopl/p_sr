% vim: keymap=russian-jcukenwin
%%beginhead 
 
%%file 16_02_2017.stz.news.ua.mrpl_city.1.113_dnej_v_istorii_kombinata_imeni_iljicha
%%parent 16_02_2017
 
%%url https://mrpl.city/blogs/view/113-dnej-v-istorii-kombinata-imeni-ilicha
 
%%author_id burov_sergij.mariupol,news.ua.mrpl_city
%%date 
 
%%tags 
%%title 113 дней в истории комбината имени Ильича
 
%%endhead 
 
\subsection{113 дней в истории комбината имени Ильича}
\label{sec:16_02_2017.stz.news.ua.mrpl_city.1.113_dnej_v_istorii_kombinata_imeni_iljicha}
 
\Purl{https://mrpl.city/blogs/view/113-dnej-v-istorii-kombinata-imeni-ilicha}
\ifcmt
 author_begin
   author_id burov_sergij.mariupol,news.ua.mrpl_city
 author_end
\fi

Оперативная сводка Совинформбюро за 10 сентября 1943 г.: \enquote{Наши войска,
наступающие вдоль побережья Азовского моря, сломили упорное сопротивление
противника и овладели городом и портом Мариуполь}.

\ii{16_02_2017.stz.news.ua.mrpl_city.1.113_dnej_v_istorii_kombinata_imeni_iljicha.pic.1}

Из сообщения Государственной Чрезвычайной Комиссии по установлению и
расследованию злодеяний немецко-фашистских захватчиков о разрушениях в городе
Мариуполе: \enquote{На заводе им. Ильича оккупанты полностью разрушили доменный цех,
взорвали все три мартеновских цеха, прокатные и трубосварочный цехи, вывели из
строя энергетическое хозяйство, здания трех электростанций, турбогенераторы,
паровые котлы, сило­вые трансформаторы, газогенераторную станцию. Были
разрушены также фасонно-литейный, чугунолитейный и сталелитейные, модельный,
прессовый, термический, ремонтно-механические и другие вспомогательные цеха,
Сартанская водокачка и насосная станция второго подъема. Гитлеровцы разобрали
значительную часть железнодорожных путей и рельсы вывезли, угнали большую
часть железнодорожного состава, а оставшиеся паровозы, вагоны и краны
взорвали. Общий ущерб, нанесенный оккупантами заводу им. Ильича, составил 319
347 тысяч рублей}.

\ii{16_02_2017.stz.news.ua.mrpl_city.1.113_dnej_v_istorii_kombinata_imeni_iljicha.pic.2}

В правительственном постановлении \enquote{О неотложных мерах по восстановлению
промышленности в освобожденных районах страны} от 21 августа 1943 года
указывалось, что \enquote{быстрейшее восстановление Мариупольского завода им. Ильича
следует считать важнейшей народнохозяйственной задачей}.

Хроника восстановления завода имени Ильича по материалам книг Дмитрия
Грушевского \enquote{Имени Ильича}. Донецк, \enquote{Донбасс}, 1966, Юрия Некрасовского
\enquote{Огненное столетие. 1897 – 1997}. Изд. 2-е. Мариуполь, ММК им. Ильича, 1999, а
также из газет.

\ii{16_02_2017.stz.news.ua.mrpl_city.1.113_dnej_v_istorii_kombinata_imeni_iljicha.pic.3}

11 сентября в Мариуполь прилетели на военно-транспортном самолете директор
завода им. Ильича Александр Фомич Гармашев и несколько руководящих работников
завода, находившихся в эвакуации.

Из воспоминаний Ивана Никифоровича Аграмакова, бывшего работника мартеновского
цеха №3: \enquote{На следующий день, как из Мариуполя выгнали немцев, объявили у нас в
Ильичевском районе, чтобы все мужчины, подлежащие призыву, собрались на площади
у первых ворот завода им. Ильича. Я пошел туда. Офицеры нас построили, что-то
стали объяснять. Тут появился человек в военной форме, но без знаков отличия.
Громко сказал, скорее, выкрикнул: \enquote{Кто работал в горячих цехах, выйти из
строя!} Я вышел. К тому времени, хоть годов мне было мало, я уже успел
поработать электриком в литейном цехе. Таким, как я, сказали: \enquote{Идите в отдел
кадров, оформляйтесь на работу}}.

11 сентября был создан совет женщин-общественниц завода. По инициативе этого
совета организовывались женские бригады. Впоследствии женщины-общественницы
оборудовали госпиталь для раненых воинов, достали матрацы, подушки, посуду и
мебель. Большую помощь оказали они в оборудовании детского дома им. Крупской и
в определении в него детей-сирот.

12 сентября начались восстановительные работы, в которых приняли участие
рабочие из горячих цехов завода имени Ильича, которые не подлежали мобилизации
в армию, ветераны предприятия, пережившие оккупацию, домохозяйки,
старшеклассники.

13 сентября на завод уже вышло десять тысяч рабочих и несколько сот
инженерно-технических работников. Не было электричества, подъемных средств,
пара, воды, машин и механизмов. Люди взяли в руки кирки, лопаты, ломы и
разбирали завалы, извлекали и восстанавливали металлические
кон­струкции. Это было только начало многомесячного изнурительного труда.

13 сентября 1943 года начал работать отдел рабочего снабжения завода. Был
составлен график первоочередных работ по организации столовых на предприятии.

20 сентября дал ток восстановленный турбогенератор, что позволило запустить
металлорежущие станки в механических цехах.

25 сентября начали работать цех водоснабжения и хлебозавод – важнейшие
источники жизнеобеспечения трудящихся предприятия и их семей.

29 сентября директор завода А.Ф. Гармашев сделал доклад активу ильичевцев, в
котором были определены мероприятия по восстановлению завода и график пуска
восстановленных объектов. Мероприятия и график были утверждены. Еще шло
восстановление цехов, а завод уже начал выдавать продукцию.

30 сентября завод и Ильичевский район получили электроэнергию.

15 октября заработал первый блок кислородной станции.

10 ноября после восстановления дала первую плавку стали мартеновская печь №5
мартеновского цеха №1.

15 ноября отремонтированы мостовые краны первой очереди.

25 ноября стала в строй действующих мартеновская печь №1 в мартеновском цехе
№1.

28 ноября начали плавить сталь в восстановленной мартеновской печи №2
мартеновского цеха №1.

10 декабря заработала мартеновская печь №6 мартеновского цеха №1.

К концу 1943 года на заводе имени Ильича было произведено 25 тысяч тонн стали,
более тысячи тонн проката. Для действующей армии изготовлено 1380 автодеталей,
выполнен ремонт двадцати танков.

31 декабря 1943 года делегация 9-й гвардейской Мариупольской истребительной
авиаци­онной дивизии в составе начальника политотдела соединения полковника
Дмитрия Мачнева и Героя Советского Союза Николая Лавицкого вручила переходящее
Красное знамя за трудовые успехи коллектива завода имени Ильича директору
предприятия Александру Гармашеву.

113 дней и ночей в только что освобожденном от гитлеровцев Мариуполе люди
поднимали из руин предприятие, люди, ослабленные от недоедания, люди, плохо
одетые, люди, ютившиеся в перенаселенных коммуналках, бараках, хатенках
близлежащих поселков. Ведь оккупантами была сожжена половина жилого фонда
города.

А впереди у них было еще очень много работы...
