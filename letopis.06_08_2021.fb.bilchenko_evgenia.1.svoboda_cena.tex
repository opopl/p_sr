% vim: keymap=russian-jcukenwin
%%beginhead 
 
%%file 06_08_2021.fb.bilchenko_evgenia.1.svoboda_cena
%%parent 06_08_2021
 
%%url https://www.facebook.com/yevzhik/posts/4129204633781282
 
%%author Бильченко, Евгения
%%author_id bilchenko_evgenia
%%author_url 
 
%%tags bilchenko_evgenia,chelovek,svoboda,zhizn
%%title БЖ. Цена свободы
 
%%endhead 
 
\subsection{БЖ. Цена свободы}
\label{sec:06_08_2021.fb.bilchenko_evgenia.1.svoboda_cena}
 
\Purl{https://www.facebook.com/yevzhik/posts/4129204633781282}
\ifcmt
 author_begin
   author_id bilchenko_evgenia
 author_end
\fi

БЖ. Цена свободы.

Если бы мне кто-нибудь в апогее моей популярности сказал бы, что я буду жить
вот так, я бы не поверила. Нет больше университета. Двадцать лет был - и нет.
Нет больше сцены - пятнадцать лет была и нет. Нет больше студентов, которые
гроздьями висели на руках, шурша на открытых лекциях, сплевывая в курилках,
восхищённо заглядывая в глаза. Нет переполненных залов с читателями, когда люди
встают, хлопают стоя, окружают тебя после выступления душным говорливым
кольцом. Нет больше самостоятельных прогулок - узнаваемость и большой риск. 

\ifcmt
  pic https://scontent-cdg2-1.xx.fbcdn.net/v/t1.6435-9/230171882_4129204587114620_7697358100660346126_n.jpg?_nc_cat=100&ccb=1-4&_nc_sid=8bfeb9&_nc_ohc=VRcc7NHlx60AX97qQ42&_nc_ht=scontent-cdg2-1.xx&oh=8cb6f55c6024e9fc100b2cfbfd4887de&oe=61351F22
  width 0.4
\fi

Нет больше тела - его хватает до ближайшего квартала, любое превышение
расстояния равняется раненой, бессильной муке. Нет будущего - ты не знаешь,
насколько хватит этого тела. Нет родителей: бабушка и дедушка умерли, большего
Бог не дал. Нет никого более старого, более мудрого, более сильного, чем ты,
кто бы сам нашел тебе больницу и приют. Нет помощи - только социальные сети, но
по ним нельзя жить. Нет утра - больно просыпаться, больно телесно так, что не
до души. Нет друзей - они стесняются, что дружат с тобой. Нет имени - ты
работаешь анонимным редактором, консультантом, переводчиком, воспринимая все с
терпением и благодарностью чернорабочего. Какая-то вечно скорченная
скрюченность перед монитором ради грядущего, которого ты уже не увидишь. Лишь
бы не думать об этом и о теле. Абсурд? Высший смысл?

Есть только - дни, недели и месяцы в собственной квартире. Есть постель для сна
днем. Есть монитор для труда ночью. Есть лекарства на тумбочке. Тонны лекарств.
Аспирин горстями, ломота в костях, улыбка. Вымученная улыбка. Есть двадцать
шагов до парка. Вот и все события. Есть, правда, твоя единственная любовь. Это
ещё есть. 

Иногда я считаю, сколько ещё протяну. Абсолютно серьезно, запоминая любую
малость. Шелест ветки. Травинку. Даже непрерывная боль в теле стала родной, по
ней хотя бы помнишь, что жив. Я смотрю, как страдают русские поэты в роскоши
этой золотой и смарагдовой цивилизации, между забавами и премиями, и думаю: "А
вы вот так смогли бы?" Любое движение, любая поездка, любой выход из дома
вызывают физический ужас. Просто - больно. Больно даже умыть лицо и
накраситься. Больно держать в руках смартфон. Больно стоять. Понять это могут
лишь болящие. И выброшенные на обочину. Даже выпить нельзя - тело не принимает
ни капли, не может, не в состоянии, тошно, противно.

Самое главное - это не пост жалобы. Возможно, это даже пост какого-то странного
счастья. Я уже смирилась, что сделать больше ничего нельзя. С каждым днём у
меня меньше и меньше сил. Я не жалею ни о чем и понимаю, что меня сплюнули и
предали забвению именно тогда, когда я сама ощутила, что я ухожу. Очень бы
хотелось жить, даже в гетто, даже в ужасном страдании: человеку всегда хочется
Жить. Но обмануть плоть, как бы ни был силен дух, невозможно. А плоть говорит:
"Это край". Или, если веселее: "Вскрытие покажет". 

Слишком я полагалась на своих покойных старших в семье, которые меня
вытаскивали, слишком щедро тратила всё, не думая, что наступит момент, когда
ресурсов больше не будет. И вот этот момент пришел. Я угасаю, я это знаю. Я
покидаю мир. Я даже не тороплюсь: мне не успеть допеть эту песню. Я смирилась.
Я не знаю, психосоматика ли это или сомопсихотика: меня сначала обидели, а
потом я заболела, или я сначала заболела, а потом обидели и обиделась. 

Но черная туча скорбей и немощей навалилась разом, и топлива уже выруливать не
хватает. Лежать. Единственное, что радует: когда лежишь. И ждёшь, когда
следующее обострение будет последним. Или когда уже не сможешь вставать. 

И вот такая жизнь - это цена, которую я оплатила, сказав "Нет" году 2014. Вот
так выглядит настоящая, неплощадная свобода. Вот такого она стоит. Так она
убивает, когда твое "Нет!" не предполагает за спиной своих. Я смотрю на тех,
кто "за": за ними - свои. На тех, кто "против": за ними - свои. "За мной -
пустота". Никто не воспользовался твоим волевым жертвенным решением, точнее, ты
никому не дал себя использовать, ты - слишком целое для того, чтобы дробиться
на удобные фрагменты для наскакивающих друг на друга орд, стай, сообществ,
говорящих нужные предсказуемые слова, повторяющих одно и то же годами, чтобы
понравиться своим.

Красивая смерть - это удел книг. В жизни это долго, мучительно и страшно.
Смерть - очень навязчивая бытовая вещь. Очень. Она в самые поры просачивается.
Она ослабляет тело так, что твой дух просто смотрит на тебя издалека. И -
вспышками - идут стихи, строки, странные мечты о яркой вещи, мягкой игрушке, и
- я это очень люблю - о кофе на заправке, когда едешь по шоссе и внезапно
просыпаешься: остановка. 

Не знаю, даст ли мне Господь всё это пережить, но, если бы мне предложили
прожить жизнь по-другому, я бы все равно выбрала свою. Я сделала по совести. Я
сделала, как подсказывало сердце. Это ведь главное?

\ii{06_08_2021.fb.bilchenko_evgenia.1.svoboda_cena.cmt}
