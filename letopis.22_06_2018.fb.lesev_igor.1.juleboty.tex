% vim: keymap=russian-jcukenwin
%%beginhead 
 
%%file 22_06_2018.fb.lesev_igor.1.juleboty
%%parent 22_06_2018
 
%%url https://www.facebook.com/permalink.php?story_fbid=1962909177073509&id=100000633379839
 
%%author_id lesev_igor
%%date 
 
%%tags politika,timoshenko_julia
%%title Интересно наблюдать за телодвижением юлеботов
 
%%endhead 
 
\subsection{Интересно наблюдать за телодвижением юлеботов}
\label{sec:22_06_2018.fb.lesev_igor.1.juleboty}
 
\Purl{https://www.facebook.com/permalink.php?story_fbid=1962909177073509&id=100000633379839}
\ifcmt
 author_begin
   author_id lesev_igor
 author_end
\fi

Интересно наблюдать за телодвижением юлеботов, которых, оказывается, достаточно
и в моей френд-ленте.

Хотя, признаюсь, юлеботы на порядок меньше раздражают, нежели порохоботы.
Во-первых, потому что Юля какая никакая, а оппозиция. Ватники могут сказать,
что куйовоая оппозиция, но тут уже с чем сравнивать. Если со шлюхами из
Оппозиционного блока, то Юля вполне еще ничего. Если с оппозицией домайданного
времени, то и здесь у Юли не очень большой фарватер. Крен влево или вправо –
мель, а то и торпеда в жопу.

\ifcmt
  ig https://scontent-frx5-1.xx.fbcdn.net/v/t1.6435-9/36063404_1962908993740194_5896807930665107456_n.jpg?_nc_cat=105&ccb=1-5&_nc_sid=730e14&_nc_ohc=VonYQK0hKMUAX8EvgEE&_nc_ht=scontent-frx5-1.xx&oh=3eef3ed02798fcc07bf9d55856c286ae&oe=61BA5CC1
  @width 0.4
  %@wrap \parpic[r]
  @wrap \InsertBoxR{0}
\fi

Во-вторых, юлеботы косят под объективных аналитиков. Типа, как художник, вижу,
что она не идеал, но если в сравнении, то покатит под рыбку и пиво. Смотрится
все равно несколько коряво, но такого зачетного лизания, как у Петра
Алексеевича нет. И этому тоже есть свои объяснения. Все знают, что Юля может
кинуть, недоплатить и даже не расплатиться. К тому же, Юля еще не президент. А
вдруг все переиначится, а ты уже законтачился с Юлей? Поэтому, осторожничают.

Но в любом случае, все кто пишет за деньги – я их понимаю и критикую разве что
только топорную методу. Не более того. Но когда начинают вылезать бесплатные
почитатели Юли, уверовавшие в изменения «после Петра», вот тут у меня возникает
недоумение.

Какая может быть Украина без Петра, но при Юле? Вот точно такая же, как прямо
сейчас, разве что на год все старше станем. Юля – это и есть Петя, со всей его
командой, методой и замашками. Ну, может быть, будет говорить чуть адекватнее и
с меньшим пафосом.

А так я бы нанятым юлеботам посоветовал быть более напористыми. То что Юля
будет следующим президентом, лично у меня сомнений все меньше. Но вы – платные
и бесплатные любители Юли – через год вокруг оси обернуться не успеете, как
окажетесь снова на обочине, а штатные сегодня порохоботы будут следующие 5 лет
облизывать новую хозяйку. Почему? Потому что Юля – это равно Петя. И
«конструктивная критика» ей будет нужна разве что на отрезке избирательного
забега. А дальше все то же – благоговейное почитание и безальтернативная любовь
до следующего президентского цикла. Ну упустите свой шанс. Места ограничены.

\ii{22_06_2018.fb.lesev_igor.1.juleboty.cmt}
