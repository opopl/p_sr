% vim: keymap=russian-jcukenwin
%%beginhead 
 
%%file topics.vojna.my.11.filmy.1.mind_games.tetjana_rudenko
%%parent topics.vojna.my.11.filmy.1.mind_games
 
%%url 
 
%%author_id 
%%date 
 
%%tags 
%%title 
 
%%endhead 

Тот вариант, когда начало было многообещающее, чем финал...

В самом начале нас знакомят с военным психологом, молодой девушкой, готовящейся
к свадьбе. И все вроде бы гладко и не предвещает беды, но... Ее похищает
неизвестный, и в круг подозреваемых, в первую очередь, попадают два близких ей
человека, жених и подруга. Но телефонный разговор этих двоих по поводу
исчезновения главной героини сужает этот круг вдвое, причем, как окажется, не в
правильном направлении. С этой точки зрения, сценарий не плох, так как фильм с
самого начала интригующий.

С момента появления в трейлере Брюса, начинается самое интересное. Они не
знакомы были до этого, кто он? Родственник одного из ее покончивших с собой
пациентов? Нет, более похож на психа. Тогда почему он выбрал ее? Зачем позволил
ей пользоваться рацией? Он же псих, но не дурак.

В том, что на "том конце провода" все было зараннее спланировано и шло по
сценарию маньяка, не было сомнения. Зачем? Этот вопрос не покидает ни нас, ни
главную героиню. Практически весь фильм состоял из воспоминаний из прошлого, в
которых она хотела найти на него ответ.

Зритель невольно выискивает в этих отрывках из прошлого обьяснение
случившемуся.  Но он еще не догадывается, что так далек от разгадки...

До сих пор неясными остаются взаимоотношения между главной героиней и ее
возлюбленным.  Оказывается, не все так гладко. Она неуравновешена, он слишком
безэмоциональный ( я долго думала, это ответ на главный вопрос или актер просто
плохо играет? ) Она погрузла в работу, он не оказывает ей должной поддержки.
Откуда ни возьмись - повод для ревности, в виде СМС от незнакомого мужчины.
Зритель невольно начинает пророчить им расставание. Но это же не повод ее
похищать?

И снова улики уводят нас от ее суженого. О подруге тоже ничего не слышно.  Мы
потихоньку начинаем о ней вообще забывать. Остается Брюс?

Цепляясь за любую ниточку из ее воспоминаний, мы параллельно с ней ищем
взаимосвязь между злодеем и людьми из ее прошлого. Но нет никаких зацепок...
Нас подвели к тому, что на ней "косвенно" лежит вина за гибель 90 (!!!) \% ее
пациентов, прошедших войну в спецподразделениях. Так может, все-таки месть за
чью-то смерть? Одни вопросы...

Очевидным стало только то, что она глубоко несчастная женщина, потерявшая
ребенка и живущая в постоянном страхе потерять и возлюбленного, несостоявшаяся
как психолог и осознающая свою вину за гибель многих своих "подопечных", как бы
тщательно она не старалась себя оправдать в этом.

Финал оказался неожиданным, в прямом смысле этого слова. Все ожидания крутейшей
развязки "психологического" триллера сводятся на "нет". Да и триллер ли это
вообще, как было заявлено?

Причиной всему оказалось банальное наркотическое вещество. Серьезно? Спасибо
хоть интригу сохраняли на протяжении всего фильма. Но если быть до конца
откровенной, то задумка все-таки не плохая. Главной героине приходилось лечить
"разум" людей, которые, пройдя через все круги ада на войне, были готовы
покончить с собой. Труд не из легких.  В ее случае - безуспешный. Да и разве
можно вылечить разум и душу человека, который по приказу "свыше" и по долгу
службы вынужден был убивать? Думаю, нет.

В народе бытует поговорка: "с кем поведешься, того и наберешься". Не минует эта
участь и психологов и психотерапевтов. Наверное, именно поэтому для этого
эксперимента выбрали именно ее - уязвимую и отчаявшуюся.  

А вот какова цель разработки этого вещества? Хотели ли эти люди по принципу
"русской освободительной" армии создать при помощи этого вещества свою такую
же? А, может, автор еще сам не придумал, для чего оно было разработано?  Тогда,
вся надежда на сиквел :)

