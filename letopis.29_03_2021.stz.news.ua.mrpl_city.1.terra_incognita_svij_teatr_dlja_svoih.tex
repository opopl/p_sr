% vim: keymap=russian-jcukenwin
%%beginhead 
 
%%file 29_03_2021.stz.news.ua.mrpl_city.1.terra_incognita_svij_teatr_dlja_svoih
%%parent 29_03_2021
 
%%url https://mrpl.city/blogs/view/pershij-nederzhavnij-teatr-mariupolya-terra-incognita-svij-teatr-dlya-svoih
 
%%author_id demidko_olga.mariupol,news.ua.mrpl_city
%%date 
 
%%tags 
%%title Перший недержавний театр Маріуполя "Terra Incognita" (Свій театр для Своїх)
 
%%endhead 
 
\subsection{Перший недержавний театр Маріуполя \enquote{Terra Incognita} (Свій театр для Своїх)}
\label{sec:29_03_2021.stz.news.ua.mrpl_city.1.terra_incognita_svij_teatr_dlja_svoih}
 
\Purl{https://mrpl.city/blogs/view/pershij-nederzhavnij-teatr-mariupolya-terra-incognita-svij-teatr-dlya-svoih}
\ifcmt
 author_begin
   author_id demidko_olga.mariupol,news.ua.mrpl_city
 author_end
\fi

\ii{29_03_2021.stz.news.ua.mrpl_city.1.terra_incognita_svij_teatr_dlja_svoih.pic.1}

Є в нашому місті один театральний колектив, який вражає своєю працездатністю.
Він за досить короткий час зміг не тільки завоювати визнання глядачів, але й
отримати високу оцінку від театральних критиків. Перший недержавний театр
Маріуполя \enquote{Terra Incognita} (Свій театр для Своїх) відкрився \emph{27 жовтня 2019}
року і наразі показує вистави на сцені Маріупольського театру ляльок за адресою
вул. Харлампіївська, 17. Керівником театру є український театральний режисер та
сценограф \emph{\textbf{Анатолій Левченко}}.

За перші 3 місяці свого існування театр випустив 4 прем'єри. Репертуар
складається з наступних вистав: \enquote{Голос тихої безодні} Неди Нежданої, \enquote{Селфі зі
склерозом} Олександра Володарського, \enquote{Сватання in Gon4arivці} за
Г. Ф. Квітка-Основ'яненком, \enquote{Твори Й. Бродського}, \enquote{Палата № 6} Антона Чехова,
\enquote{НІЧ проти РІЗДВА} (реп-мюзикл за твором М. Гоголя), \enquote{Дорослі ігри завзятих
дівчат за п'єсою сучасної драматургині Наталі Уварової, \enquote{DELETE} за п'єсою
нобелівського лауреата Харольда Пінтера. Наразі театр готує нові прем'єри –
\enquote{Доктор Лектор} за п'єсою Кая Хензеля \enquote{Кламмова война}, та \enquote{Янголи – теж Люди}
за п'єсою сучасної драматургині Світлани Конощук. До ювілею Лесі Українки
глядач побачив поезовиставу \enquote{На полі крові. Вірші}, в якій було поєднано \enquote{живе}
читання творів поетки з показом спеціально створеного мікротелесеріала \enquote{На полі
крові}.  Це не перший поетичний досвід театру, для якого просвіта підростаючого
покоління і робота з різними верствами населення, спрямована на
самоідентифікацію себе українцями – реальна, постійна робота. Зокрема, театр
здійснив постановки до 130-річчя Остапа Вишні \enquote{Блазень з сумними очима},
130-річчя Павла Тичини \enquote{Все на світі - від примружених очей}, а до 160-ї
річниці смерті Тараса Григоровича Шевченка колектив показав поезовиставу \enquote{Назар
Стодоля}. Вірші. Щоденник}. Театр взяв на себе \enquote{культурне шефство} над
захисниками українського неба - виїжджає з виставами і приймає у себе
військовослужбовців зенітного ракетного Теребовлянського полку. Спектакль \enquote{НІЧ
проти РІЗДВА} – перша театральна постановка, створенна спільнокоштом
маріупольців.

\ii{29_03_2021.stz.news.ua.mrpl_city.1.terra_incognita_svij_teatr_dlja_svoih.pic.2}

Театр сучасного спрямування, тяжіє до постмодерну, та так званого \enquote{мислячого}
театру, шанує класику. Вистави ставляться українською та російською мовами.

\enquote{Terra Incognita} є членом асоціації \emph{\enquote{Незалежний театр України}}. На двох
прем'єрах, поставлених за творами українських сучасних драматургів, були
присутні автори. Також в театрі проходив всеукраїнський проєкт \enquote{Весна}.
Спектакль \enquote{Голос тихої безодні} за п'єсою Неди Неждани став дипломантом 50-го
Всеукраїн\hyp{}ського театрального свята \enquote{Вересневі самоцвіти} та ІІ Відкриттого
театрального фестивалю \enquote{SCENE FESN}. Колектив співпрацює з Першою
Маріупольською театральною школою-студією (художній керівник – заслужена
діячка естрадного мистецтва України Ірина Руденко), учні старших груп беруть
участь у виставах. Один з цікавих проєктів – Різдвяний Вертеп, показаний на на
подвір'ї \href{https://www.facebook.com/st.panteleimon}{Храму Великомученика i Цiлителя Пантелеймона Православної Церкви
України}.%
\footnote{\url{https://www.facebook.com/st.panteleimon}}

\ii{29_03_2021.stz.news.ua.mrpl_city.1.terra_incognita_svij_teatr_dlja_svoih.pic.3}

В рамках просвітницького проєкту ютуб-каналу \enquote{Мистецький хаб  \enquote{ВО!}} (видатні
особистості та сучасні митці) професійні актори разом із творчою молоддю
читають та записують твори видатних письменників і поетів. Під час карантину
театр залучив до цього в он-лайн режимі сучасних поетів з Ураїни і зарубіжжя.
Вийшло 125 випусків, серед них – 20 циклів, присвячених маловідомим поетам
\enquote{розстріляного Відродження}. Загалом колектив складається як з досвідчених
акторів-резидентів (заслужений артист України \emph{\textbf{Анатолій Шевченко}}), так і з юних
акторів-початків\hyp{}ців. Встигли полюбитися глядачам молоді актори \emph{\textbf{Михайло Загреба}}
та \emph{\textbf{Катерина Просолупова}}.

\ii{insert.read_also.demidko.teatromania}

У 2020 році колектив брав участь у  першому міжнародному театральному
інтернет-конкурсі \enquote{Театр.NET}, де отримав Диплом лауреата Першого ступеня за
виставу \enquote{Сватання in Gon4arivці} \textbf{в номінації} \enquote{Професійний театр}. Художній
керівник театру Анатолий Левченко удостоєний нагород – медалі \enquote{За жертовність і
любов до України} Православної Церкви України і ордена \enquote{За внесок у розбудову
України в гуманітарній сфері}.

За своєю формою театр є частиною громадської організації, всі творчі і технічні
співробітники беруть участь в проєктах на благодійній основі.

\ii{29_03_2021.stz.news.ua.mrpl_city.1.terra_incognita_svij_teatr_dlja_svoih.pic.4}

З нагоди Всесвітнього дня театру, 27 березня, на сцені Першого недержавного
театру Маріуполя \enquote{Terra Incognita} (Свій театр для Своїх) відбудеться вдруге
прем'єрна вистава за п'єсою нобелівського лауреата Харольда Пінтера \enquote{DELETE},
яка, до того ж, присвячена пам'яті славетного українського режисера, що
нещодавно пійшов за обрій, - Романа Віктюка. Підготувавши стільки прем'єр,
наразі колектив театру цілком готовий приємно вражати глядачів своєю
майстерністю і багатством репертуару. В планах-проведення фестивалю, обмінні
гастролі та участь у двох міжнародних проєктах.

\ii{29_03_2021.stz.news.ua.mrpl_city.1.terra_incognita_svij_teatr_dlja_svoih.pic.5}
