% vim: keymap=russian-jcukenwin
%%beginhead 
 
%%file 12_09_2021.fb.semesjuk_ivan.1.chelovek_prividenie
%%parent 12_09_2021
 
%%url https://www.facebook.com/ivan.semesyuk/posts/4678711652179365
 
%%author_id semesjuk_ivan
%%date 
 
%%tags chelovek,filosofia,mysli,prividenie
%%title Людина - явище очевидно тимчасове, фактично ж це привид
 
%%endhead 
 
\subsection{Людина - явище очевидно тимчасове, фактично ж це привид}
\label{sec:12_09_2021.fb.semesjuk_ivan.1.chelovek_prividenie}
 
\Purl{https://www.facebook.com/ivan.semesyuk/posts/4678711652179365}
\ifcmt
 author_begin
   author_id semesjuk_ivan
 author_end
\fi

Трохи про метаантропологію. Пафосно, але інакше про це не виходить. 

Людина - явище очевидно тимчасове, фактично ж це привид. Сьогодні є, завтра
нема, і все у неї тимчасове. Руки, ноги, вуса, почуття, усе це тимчасове.
Навіть надгробки, які можуть нагадувати про свого носія значно довше ніж він
проіснував, теж вельми тимчасові, особливо у наш час.

\ifcmt
  ig https://scontent-frt3-1.xx.fbcdn.net/v/t1.6435-9/241676904_4678720568845140_1113618724181814469_n.jpg?_nc_cat=104&_nc_rgb565=1&ccb=1-5&_nc_sid=8bfeb9&_nc_ohc=yyMZSXlriDkAX-zmqGB&_nc_ht=scontent-frt3-1.xx&oh=84e2df07aaca4ab61975edc4670edcde&oe=616533EF
  @width 0.4
  %@wrap \parpic[r]
  @wrap \InsertBoxR{0}
\fi

Та все ж таки не все так однозначно, тому іноді, коли якомусь вумному привиду
стає зрозуміло, що він привид, його починає сильно тягнути до нетимчасового.
Або вічного, або хоча би дійсно довготривалого.

Він починає прагнути чогось більшого ніж є сам. Втім вусів нетимчасових не
буває, але такими бувають почуття, адже їх можна пронести крізь усе життя і
скерувати при тому у майбутнє, перетворити на твір і поділитися ними з
подальшим, але не лише це. Так і виходить, що у своїй людськості спертися
можемо лише на почуття, тому емоційна сфера є єдиним людиноутворюючим фактором.

Коли ми говоримо про почуття, то цілком можемо оперувати таким терміном як
якість. Якісне почуття – звучить може й  незвично, але тут є про що подумати,
бо за якістю причаїлася цінність.

Яке воно, якісне почуття? Вочевидь таке, в яке вкладено душевні сили,
виплекане, збережене, тобто довготривале. Таке почуття може еволюціонувати,
змінюватися, зростати, дорослішати, мати свою історію буття, і насправді якісне
почуття є ні чим іншим як живою істотою, сутністю зі своїми надіями,
розчаруваннями, звершеннями та врешті особистістю. 

Чим довше існує почуття, тим живішим у прямому смислі воно стає з плином часу.
Почуття варто оберігати саме як живу істоту, виховувати, розвивати, навчати і
спілкуватися з ним (як би дивно це не прозвучало), бо в них є сама наша суть,
яка по-перше робить нас саме людиною, а по-друге робить нас конкретною людиною,
саме цією, а не взагалі, і не якоюсь іншою. І звичайно не варто ставитися до
них легковажно.

Привид же або все життя страждає від невихованих, диких, недовготривалих,
необробених, інфантильних почуттів, а потім зникає в нулі, або через співпрацю
зі своїми почуттями знаходить вхід у той світ, одного якого й варто прагнути –
світ реального існування, куди привиду не потрапити ніколи.

Цей вхід він завжди і зовсім поруч, не дарма тема переходу між світами так
кріпко і так давно турбує людство. Він сховався в почуттях, однак вхід – це
процес, а не діра в реальності.

Тема складна, для когось й болюча, але варта того щоби торкнутися її під
недільну каву на ґанку.

Та й по всьому.
