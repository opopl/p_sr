% vim: keymap=russian-jcukenwin
%%beginhead 
 
%%file 12_12_2016.fb.uljanov_anatolij.1.travlja_kabanov.cmt
%%parent 12_12_2016.fb.uljanov_anatolij.1.travlja_kabanov
 
%%url 
 
%%author_id 
%%date 
 
%%tags 
%%title 
 
%%endhead 
\subsubsection{Коментарі}

\begin{itemize} % {
\iusr{Александр Кабанов}

Спасибо, Толя. И картина - яркая, фамильная.))

\begin{itemize} % {
\iusr{Татьяна Щербина}
травлю пропустила, но осуждаю)

\iusr{Борис Бергер}

САША! а давай закажем Ройтбурду картину в стиле Слепцов Брейгеля - как мы с
тобой с завязанными чорными или жовтоблакитными лентами глазами гуляем по ул.
Бандеры.

\iusr{Лена Мандель}
Саша, первый раз услышали про это про все. Молодец! Ты там не унывай, держись.
\end{itemize} % }

\iusr{Алеся Шаповалова}

Саш, подождем еще лет двадцать и этот дурацкий мост (может и не дурацкий, не
знаю, не видела) будет назван твоим именем!  @igg{fbicon.smile}  а сейчас береги себя!

\begin{itemize} % {
\iusr{Сергей Косяшников}
Мост и сейчас называется Московским.
\end{itemize} % }

\iusr{Hanna Protasova}

Пане Анатолію, правильніше було б сказати \enquote{певна частина літературної формації
традиційно є найконсервативнішою...} і далі за текстом. І консервативною є не
сама література, а її представники, які успішно транслюють агресивні слогани у
своїх фб. І питання велике, чи це консерватизм узагалі.

\iusr{Анатолий Ульянов}
да, это важные уточнения. спасибо

\iusr{Владимир Юмашев}
Спасибо.

\iusr{Олеся Маркова}
А почему вы решили, что высказывание господина Кабанова не понравилось только националистам?

\iusr{Анатолий Копейкин}

\obeycr
Со стороны эта буча вообще не понятна.
Ну проспект Бандеры - и что?
Да хоть Романа Шухевича.
Или Лаврентия Берии.
Или Андрея Власова.
Это же всего лишь БУКОВКИ на табличке.
\restorecr

\begin{itemize} % {
\iusr{Андрей Кабо}
Ну фашизм и чё?

\iusr{Vladimir Shamray}
Копейкин ваша мама была моей наложницей. Только не обижайтесь это всего лишь буковки да и не правда наложником был ваш папа

\iusr{Григорий Цветков}
Ну да, и зачем вообще задумываться?
Ну вломятся потом почитатели того, чье имя на табличке, ну пробьют размовляющих на неправильной мове, ну и чё?
Делов-то...
\end{itemize} % }

\iusr{Аслан Ногиев}
Я уверен что Украина никогда такой не была и не будет как сегодня в том числе и благодаря поэту Кабанову

\iusr{Тетяна Монтян}

Ото нема чого робити - звертати увагу на упоротих їбанашок  @igg{fbicon.smile} 

\begin{itemize} % {
\iusr{Volodimir Bondal}
подтянулась выдра

\iusr{Тетяна Монтян}

\textbf{Volodimir Bondal}, нах ото пішов хутенько, лішенець кастрюльний !  @igg{fbicon.smile} 

\iusr{Серж Базилик}
\textbf{Volodimir Bondal} Эта бешеная корова просила - не обращать на неё внимания. Думаю, отличный план!  @igg{fbicon.wink} 
\end{itemize} % }

\iusr{Іван Гонта}

\enquote{Народ только нєдавно обрєтший свою гасударствєность..}
Завдяки таким як ви цей народ кров'ю спливав за те, щоб цю саму державність відновити.
А Бандера в таборі сидів..
За те, щоб повернути усе, що у нас \enquote{братья} забрали.

\begin{itemize} % {
\iusr{Ирина Морозовская}

Какие у вас имя-фамилия говорящие! Натуральные или псевдоним? Ясное дело - не
из потомков изверга, он своих детей не пощадил. Если в его честь - то очень
странно, а если само по себе досталось - то, похоже, имя таки влияет на
характер, как некоторые считают...


\iusr{Taras Shuplat}
\textbf{Ирина Морозовская}, а по существу? Без антропонимических изысков? А то уж очень детсадовская аргументация у вас, \enquote{Наташка-какашка} какая-то...

\iusr{Ирина Морозовская}

По существу и в силу профессии - с человеком, которого зовут Иван Гонта в
частной жизни я бы не рискнула иметь дело, хотя посочувствовала бы такому
имени, если оно вдруг от родителей досталось. Потому что корни имён не просто
уходят в коллективное бессознательное, а и подпитываются оттуда связью с теми,
кто их носил. Поэтому стараются называть детей в честь достойных людей и
опасаются называть в честь злодеев. А уж если человек решил сам
идентифицироваться с нелюдью посредством псевдонима - то это очень много
говорит о нём самом неприятного. Это - способ выражения неприсвоенной агрессии.

\iusr{Іван Гонта}
\textbf{Ирина Морозовская}

Ну так Вам і Шевченко \enquote{нелюдь}, коли вже почнете палити \enquote{Кобзарі} і на
Куликовому вимахувати \enquote{гєоргієвской лєнтой}? Коли Ваша українофобія почне
реалізовуватись у дії?


\iusr{Ирина Морозовская}
\textbf{Іван Гонта} 

Чому ж ви Тараса Григоровича сюди приплели? Він як раз про Гонту розповів те й
так, що ніякіх сумнивів що до нелюдскої поведінки того не залишається. Якщо
забули - то ось, нагадую

\href{https://www.ukrlib.com.ua/books/printit.php?tid=723&page=4}{%
Гайдамаки, Тарас Шевченко, сторінка 4, ukrlib.com.ua%
}

\begin{multicols}{2}
\obeycr
...
Смеркалося. Із Лисянки
Кругом засвітило:
Ото Гонта з Залізняком
Люльки закурили.
Страшно, страшно закурили!
І в пеклі не вміють
Отак курить. Гнилий Тікич
Кров'ю червоніє
Шляхетською, жидівською;
А над ним палають
І хатина, і будинок;
Мов доля карає
Вельможного й неможного.
А серед базару
Стоїть Гонта з Залізняком,
Кричать: "Ляхам кари!
Кари ляхам, щоб каялись!"
І діти карають.
Стогнуть, плачуть; один просить,
Другий проклинає;
Той молиться, сповідає
Гріхи перед братом,
Уже вбитим. Не милують,
Карають, завзяті.
Як смерть люта, не вважають
На літа, на вроду
Шляхтяночки й жидівочки.
Тече кров у воду.
.Ні каліка, ані старий,
Ні мала дитина
Не остались, — не вблагали
Лихої години.
Всі полягли, всі покотом;
Ні душі живої
Шляхетської й жидівської.
А пожар удвоє
Розгорівся, розпалався
До самої хмари.
А Галайда, знай, гукає:
"Кари ляхам, кари!"
Мов скажений, мертвих ріже,
Мертвих віша, палить.
"Дайте ляха, дайте жида!
Мало мені, мало!
Дайте ляха, дайте крові
Наточить з поганих!
Крові море...
\restorecr

\end{multicols}

\iusr{Ирина Морозовская}

та сцена вбивства Гонтою своїх синів:

\href{http://www.ukrlib.com.ua/books/printit.php?tid=723&page=6}{
Гайдамаки, Тарас Шевченко, сторінка 6, ukrlib.com.ua%
}

\begin{multicols}{2}
\obeycr
...
Сини мої, сини мої!
Чом ви не великі?
Чом ви ляха не ріжете?.."
"Будем різать, тату!"
"Не будете! не будете!
Будь проклята мати,
Та проклята католичка,
Що вас породила!
Чом вона вас до схід сонця
Була не втопила?
Менше б гріха: ви б умерли
Не католиками;
А сьогодні, сини мої,
Горе мені з вами!
Поцілуйте мене, діти,
Бо не я вбиваю,
А присяга". Махнув ножем —
І дітей немає!
Попадали зарізані.
"Тату! — белькотали, —
Тату, тату... ми не ляхи!
Ми..." — та й замовчали.
"Поховать хіба?"
"Не треба!
Вони католики.
Сини мої, сини мої!
Чом ви не великі?
Чом ворога не різали?
Чом матір не вбили,
Ту прокляту католичку,
Що вас породила?..
...
\restorecr
\end{multicols}

\iusr{Ирина Морозовская}

За те, що він там каже про свою дружину, мати своїх дітей - природно соромитися
а не пишатися отаким. Чи то його одружили насильством? Та про мою українофобію
дуже цікаво від вас почути, бо я двомовна, до украінської літератури непогано
освідчена та знаю ії краще, ніж більшість тих, хто про свій националізм
розповідає у фейсбуці.


\iusr{Volodymyr Kryshenyk}
\textbf{Ирина Морозовская} Вибачте, але в українській мові відсутнє слово освідчений. Краще вже одукований.

\iusr{Taras Shuplat}

По существу - ни одного слова. Всё любопытно, и психоанализ выбора имени, и
литературно-исторический экскурс. Но... Что вы можете конкретно возразить ФБ
пользователю \enquote{Иван Гонта}, кроме того, что вам не нравится его имя?

\iusr{Ирина Морозовская}
\textbf{Volodymyr Kryshenyk} Так, це я літеру Д загубила на початку слова. Хоча "одукований" буде влучніше.

\iusr{Ирина Морозовская}
\textbf{Taras Shuplat} 

Я ФБ пользователь Ивану Гонта высказала в ответ на его коменты именно то, что и
хотела сказать - что мне не нравится ни его имя ни люди (человек) взявшие себе
это имя ввиду моего личного и крайне негативного отношения к историческому
прототипу. Ссылки на часть источников моего отношения и того автора, который он
сам упомянул - тоже указала. Благо помню эту поэму со школы. И что он
оскорбляет \textbf{Александр Кабанова} мне тоже не понравилось.

\iusr{Іван Гонта}

\enquote{до украінської літератури непогано ДосвіДчена} ? ))

це через гуглоперекладач?

Що ж, давайте влаштуємо поетичний вечір та почитаємо як Шевченко відповідає
планарному літературознавцю Ірині на її закиди щодо особи Гонти (адже Гонта,
безсумнівно, непроста особистість):

\ifcmt
  ig https://scontent-frt3-1.xx.fbcdn.net/v/t1.18169-9/15390923_10211739013940935_300991360993687768_n.jpg?_nc_cat=102&ccb=1-5&_nc_sid=dbeb18&_nc_ohc=YyrRB-o4vbUAX9VgEwn&_nc_ht=scontent-frt3-1.xx&oh=0d85c2dd15e2f74136c3bf395f5f6a4e&oe=61D1281F
  @width 0.4
\fi

\iusr{Ирина Морозовская}

Як цікаво. Що загубилась літера д на початку слова досвідчена - то ви помітили,
а що до особистості ката та мерзотника Гонти - то якись обмежені клаптики з
Шевченка замість того, щоб цілком поєму прочитати та зробити висновки, що він
був таки рідка падлюка, злочинець и кровопивця. И що видати його ляхам для
тортурів та катування було найкраще для усіх, крім його самого.

\iusr{Іван Гонта}

Та ж при чому там \enquote{загубилась буква} якщо я чітко показав - там речення
недолуге у Вас, неначе з російської перекладене.

Підсумок буде простий - спочатку вчіться писати до ладу, потім про Шевченка
дописуйте. В Гонті всі бачать щось своє: кати, кровопивці що катували мій народ
(ба, й досі катують) - бачать ката. Мерзотники бачать мерзотника. І так далі.

\iusr{Ирина Морозовская}
\textbf{Іван Гонта} 

Ви за мою мову не хвилюйтесь, якось сама собі розраду дам, а що Гонта - злодій,
то я залишусь при своїй думці а вам навіть співчувати не буду по тому, що ви з
ним отожнюєтесь. Хоча впевнена, что усе навпаки - мерзотники в ньому бачать
свого героя.

\end{itemize} % }

\iusr{Валерій Верховський}

Увы, как раз либералы не приемлют других взглядов, если они на йоту отличны от
либеральных. Кабанов это продемонстрировал. Вопрос почему и зачем? И почему
одновременно с ним украинофобские высказывания полились от еще нескольких
\enquote{ущемленных}. Совпадение?

\iusr{Наталья Аксенова}

Для живущих в языке всё это кажется мелкими словесными драчками. Все много
серьезнее, думаю. Поберегите его, гении - а это так, Саша гений, кто поспорит?
- уязвимы более любого из нас. А все человечество из-за них уязвимо тоже.

\iusr{Борис Артинюк}

Бандера некоторым горе ...... . , как быку красная тряпка, значить в правильном
направлении идем товарищи.

\iusr{Александр Самарцев}

Ага. Оставить лениных, блеа -ть. А язык - стержень национального самосознания
токмо у "отсталых народов... Нда... ё*нуться не запретишь.

\iusr{Юрий Юхман}

а по улицам Ленина товарища Кабанова не тошнило ездить?

\iusr{Anna Novina}

Тошнит не от названия улиц, любезный... А от ездоков... Поэт призван небом
писать и читать и никому ничего не должен. Сто лет жизни Александру Кабанову
ещё. А собравшимся... Вы лично теперь будете литературу подбирать...? Тогда,
извольте, на вручение Нобелевской премии Кабанову, как одному из лучших
современных поэтов Украины не попадаете.... Только троньте его... Вот, значит,
как вы собрались в Европу?. Или вы надеетесь, что потянете в грядущем на
название проспекта... Не тревожьтесь напрасно.

\iusr{Marusya Nechuray}

Та ладно жалеть сашу. Про него теперь хоть узнали. Забудут конечно, но плохая
реклама графоману-тоже реклама.

\begin{itemize} % {
\iusr{Alexey Akulovich}

Город про него и так знал. Причем не один и не в одном государстве. Дремучая
деревенщина не знала. Так деревне и Шекспир - графоман, ибо рядовой житель
дальше букваря не ходил. Мама мыла раму - предел. Вот вы - деревня. Типичная.

\iusr{Елена Васюнина}
Вы деревню то не обижайте!

\iusr{Marusya Nechuray}
Акулович, вы свой аул городом зовете?
\end{itemize} % }

\iusr{Vladimir Shamray}

А ваты сколько в комментах ггг. Толя, нравится ватка? Любишь нюх нюх ватку?

\begin{itemize} % {
\iusr{Анатолий Ульянов}

Мне кажется, что из всей этой риторики про ватку ничего, кроме "иди на хуй",
"та сам иди на хуй" и пошли ломать клавиатуры и носы, получится не может. Кто
вата, а кто марля – уже как-то не интересно. Интересно как сделать так, чтобы
бодающееся перестало бодаться. Во многом это языковая задача. Писателя тем паче
такое должно заводить, разве нет?

\iusr{Vladimir Shamray}

Ох, Толя, нет. Я понял за это время важную вещь. Примирение невозможно, только
смерть. И не только между ватой и укропом но и между остальными илеологическими
противниками. Только смерть. А вата не люди зря ты к ним так спустя сквозь
пальцы относишься.

\iusr{Vladimir Shamray}
Я тебя уже посылал нахуй, или с почином?

\iusr{Виктор Че}
\textbf{Vladimir Shamray} но в чем-то Толик прав))):

\ifcmt
  ig https://scontent-frt3-2.xx.fbcdn.net/v/t1.18169-9/15400427_961951270603908_7911980408173101506_n.jpg?_nc_cat=101&ccb=1-5&_nc_sid=dbeb18&_nc_ohc=Jih17lew8ogAX_3tO75&_nc_ht=scontent-frt3-2.xx&oh=74863deb98c50267ae85403b3fb04e24&oe=61CEC812
  @width 0.4
\fi

\end{itemize} % }

\iusr{Alexander Kabanov}
Согласен полностью. Оставьте Кабанова в покое!

\iusr{Олександр Малицький}

Кабанов сморозил чушь. Выступил в роли провокатора. Понятно - нужна реклама.
Получил по делу. За все надо платить. Поступок подлый со стороны так
называемого поэта.

\begin{itemize} % {
\iusr{Alexey Akulovich}

Кабанов - поэт настоящий и очень редкий. Будь вы чуть пообразованнее, то знали
бы, что не Кабанов бегает за рекламой, а реклама - за Кабановым. А вы -
диканька во всей своей красе.

\iusr{Олександр Малицький}
\textbf{Alexey Akulovich} Я, как диканька, тебе и отвечу. Иди на йух!
\end{itemize} % }

\iusr{Светлана Сухарева}

Я из России ( ежели так ближе - то вата-ватная)), и мне очень жаль, что
украинцы так небрежны со своим национальным достоянием. Вся эта политическая
пена схлынет, а стихи Кабанова останутся, и , похоже, вспоминать будут вашу
страну именно благодаря его творчеству, и не только в Украине. Не так много
сейчас творцов такого уровня, поберечь бы ....Хотя бы как оправдание перед
потомками.

\begin{itemize} % {
\iusr{Светлана Сухарева}
\textbf{Ivan Petrov} хочется верить в лучшее....

\iusr{Andrey Yatsunenko}
Шли бы в жопу, Петров. Там вам и место - вместе с Москвой.
\end{itemize} % }

\iusr{Lora Soroka}
Спасибо. Бродского из СССР/России выгнали. И кто потерял больше? Поэтов нужно беречь, особенно таких - с талантом и душой.

\iusr{Lora Soroka}
\textbf{Ivan Petrov} да уж не Бродский.

\iusr{Мария Чернова}
бузина-2

\iusr{Виталий Пациора}
хня какая-то...(((

\iusr{Євген Кірдан}

\enquote{Кабанов строил культурные мосты...} То что сделано- это часть информационноц
войны. И никакой он не ватник и не графоман, а вполне себе кремлевский боец,
глубоко законспирированный крот. Профи. Не говоря о том, что убежденный
украинофоб.

\begin{itemize} % {
\iusr{Ирина Легонькова}
@Кирдан Евгений (Kirdan Evgenij) Вы дурак?

\iusr{Ирина Морозовская}
\textbf{Ирина Легонькова} 

Нет, Ир, просто явный безумец. Много народу с ума посходило, вот и этот.

\iusr{Игорь Волков}

Прежде чем бросаться оскорблениями, познакомьтесь с творчеством Кабанова,
пропитаным неприятием украинских реалий и злобой на. Кабановский флэшмоб -
ловко состряпаная сурковская провокация, чистый Шатун, подливающий масла,
вносящий разделение озллобление, оболванивание масс, которые не читая
ппервоисточных кабановских глупостей, бросаются со своим "я - за!" на мнимую
амбразуру. Но нет худа без добра - фильтрация мутной среды - тоже дело.

\end{itemize} % }

\iusr{Volodymyr Kuzmenko}

\enquote{Поэтому хорошо бы воздержаться от спорных переименований, которые это
разобщение только усугубляют.} - хорошо бы не писать всякую чушь!!!!!


\iusr{Игорь Волков}

А почему вы решили, что дело в высказываниях? Если бы только. А \enquote{...наша
война...} это не кабановское перо? Какая такая наша война? Украина ее ведет, но
это не наша война и сын так не скажет. А собрать у себя любительниц называть
возмущенных украинцев \enquote{народцем} это не кабановская страница? Рука руку моет. О
каких мостах вы говорите на третьем году войны?! Сожжены дотла и не нами. И
если и тянет над пожарищами запахом крови, так это от Иловайска.


\iusr{Viktor Kinski}

уже травля ) изгой в радном отечестве ) не смішіть, панове. Звичайна реакція
рашистського сообчества. Захищайте Кисельова. І реакція Маскви мало цікава.
Ніхто вашого Кабанова вбивати не буде. Але жалувати теж, він просто нам не
цікавий і слави великого пііта ніхто у нього відбирати не буде. Хай крапає, в
цьому вся різниця української ментальності від собачозвірілої сутності
російського бомонду. Кабанов хитрюга, його відношення особисте мало інакшу
ціль, і він її досяг.


\iusr{Константин Медведев}
кивске провокацЮки якы... ви не в маскви?

\iusr{Наталия Горильская}
Иногда лучше промолчать...

\iusr{Татьяна Конькова}

Я думала такое только в России возможно. Что Украина борется за свободу. В том
числе и за свободу слова.

\begin{itemize} % {
\iusr{Andrey Yatsunenko}
Собственно, поэт Кабанов и его демарш обсуждается в соответствии с этой самой
свободой. Весь спектр представлен, как можно заметить.
\end{itemize} % }

\iusr{Бджілка Рання}
Поети, музиканти, художники-люди поціловані Богом і чіпати їх НЕ МОЖНА!

\iusr{Наталья Аксенова}
Что Кински, что Крук - эти лица своего не кажут. Боятся? Или с детства привыкли заходить со спины? Или уроды?

\iusr{Женя Киршенблат}

Он что против Украины? Против майдана ? Сколько хлопцев погибло с именем
Бандеры на устах ! Понаехало оккупантов после голодомора.

\begin{itemize} % {
\iusr{Alexander Kabanov}
Тролль, сволочь и му-к. Причем российский.

\iusr{Женя Киршенблат}
\textbf{Alexander Kabanov} Нас таких вся Киеврада, иди и скажи.

\iusr{Елена Синявская}
о, еще один бот кремлевский...

\iusr{Женя Киршенблат}

Придумали бы уже что-то новое,украинофобы, вы нас кремлетитушками называете с 1
декабря 2013, за то что посмели требовать национальную власть. Только ваше
время уже ушло, вы теперь - ничтожное меньшинство, стараниями вашего же Путина.
Или вы подчинитесь воли украинского народа или уедете отсюда в Калугу, к
соотечественикам, там и наслаждайтесь московскими проспектами.

\end{itemize} % }

\iusr{Костя Корабель}
\textbf{Alexandr Kabanov} 

- уникальный поэт! А обвинять его в отсутствии патриотизма - абсурдно!

Это цитата из его стихотворения:

\obeycr
"Иловайская дуга,
память с видом на руину:
жил - на языке врага,
умирал - за Украину."
\restorecr

\begin{itemize} % {
\iusr{Krajzer Ferlay}

Не надо вырывать из контекста. А, контекст там: "ну, что петро, помогли тебе
ляхи, пиндосы..."? Вспомните, сколько радостных взвизгов слышалось оттель при
очередном "подписании" "турецкого потока".


\iusr{Александр Кабанов}

\textbf{Krajzer Ferlay} не надо клеветы про радостные взвизги. это подло.

\iusr{Krajzer Ferlay}
\textbf{Александр Кабанов} отчего же клевета? мы с Вами знатно по-препирались, тогда. Впрочем, воля Ваша: если считаете, что я не прав, что радость была не по поводу ТП, а по поводу "верности" Вашего предсказания - могу и взять назад эти свои слова.

\iusr{Александр Кабанов}

\textbf{Krajzer Ferlay} а с какого хрена мне радоваться ТП? Это ж во вред Украине!

\iusr{Krajzer Ferlay}
\textbf{Александр Кабанов} мне не с руки искать тогдашнее с Вами препирательство. Если хотите, я тотчас эту ветку комментов удалю, если Вы считаете, что я тут не полностью достоверен. Помню, но "доказательств нет".

\iusr{Александр Кабанов}

\textbf{Krajzer Ferlay} да мне уже все равно

\iusr{Alexander Kabanov}

Абсолютно гениальное стихотворение и абсолютно проукраиское. А насчет газа вы
лучше что-то конструктивное придумайте вместо прокачки газа. Например хайтек
какой-то. Зачем Вам Ферлай вражеский газ?

\end{itemize} % }

\iusr{Олександр Гунько}
* * *
\obeycr
Трипільський день гортає смуток
Над вогнищами злих утіх.
Не знає, як свій час добути
Й куди сховати вічний гріх.
А поза ним серед пожарищ
Стоїть незреченості дім.
І сонце відчайдушно шкварить,
І вічність переходить в дим...
Іще богів жага несмертна,
А вже моква тумани тче.
І волхв сидить у позі вепра
Із кам’яним різдвом очей.
Та білу стужу не пригубиш,
Неначе келих забуття.
Лиш душу вихопиш зі згуби,
Немов із полум’я дитя...
\restorecr


\iusr{Андрей Явный}
Кабанова Бандерой не испортишь

\iusr{Наталья Макеева}
Как интересно наблюдать-то (поправляя значок с флагом ДНР)...  @igg{fbicon.wink} 

\iusr{Salko Ann}
Как это глупо и затратно - объяснять, что НЕ. Объяснять вообще.

\iusr{Олексій Мартиненко}

Рекомендую в следующий раз более рационально использовать приступ
немотивированной агрессии и ненависть неокрепших \enquote{умов} на пользу общества. А то
шо ж получается - бунт есть, а смысла нету). Александр Кабанов должен понимать,
шо его слово не воробей)


\iusr{Зьміцер Дрозд}

Ситуация так и просится на РТР и пр. росс. каналы - и, увы, на радость врагам
Украины....

\begin{itemize} % {
\iusr{Viktor Kinski}
Їм не звикати, дай не дай, одинаково про двох рабів розкажуть!

\iusr{Оксана Королева}

Кремлевские пропагандисты - дети по сравнению с ярыми бандерафилами,
устроившими этот шабаш. Так опустить имидж страны в глазах тысяч иностранных
френдов Саши не смогут никакие Киселевы-Соловьевы.


\iusr{Зьміцер Дрозд}
Падение ниже Соловьёва-Киселёва невозможно - на их совести тысячи жизней - в том числе и русских.

\iusr{Оксана Королева}
Так они за деньги, а эти - от всей души. Тем правдивее выглядит и тем ужаснее звучит.
\end{itemize} % }

\iusr{Антон Картавцев}
Проспект Голосов

\iusr{Anna Novina}
ясненько, судя по тону, дети выступающих будут вынуждены обучаться по букварю в картинках....

\iusr{Vladimir Shamray}

Прочел все комменты. \textbf{Alexandr Kabanov} оставить в покое, в Бандеры переимеровать
улицу Гонты, ВАТА НЕ ЛЮДИ. Ну а наши черти, кобАне -дети природы, как Виннету,
через одного лечить фистингом.


\iusr{Volodymyr Kryshenyk}

Ляпнув чоловік дурницю, а з того діла вже ліплять нещасну жертву - такого собі
громадянина Мерсо зі "Стороннього". Нате, маємо - злетілись одразу
культур-трегери й цивілізатори з-за поребрика і по всьому світу розсіяних.

\begin{itemize} % {
\iusr{Сергей Косяшников}
И его ж подставляют. Бо захотят "раскрутить" как Бузину. Не дай Бог, конечно.
\end{itemize} % }

\iusr{Александр Воловик}
Саша! Держись! Ты во всём прав. Полностью тебя поддерживаю.

\iusr{Anna Novina}

Вообще, Саша не держитесь и не рискуйте собой. Те, кому не нужны Вы- не нужны
Вам... Такие как Вы, для меня и есть - цивилизованная Украина. А мир велик,
особенно для поэта.


\iusr{Константин Минко}

Какими бы намерениями не было продиктовано желание отдельной части общества
стоять на Майдане / защищать страну от российских захватчиков/ не вести
переговоров с террористами, нужно понимать, что измученный украинский народ
сегодня разобщен.


\iusr{Елена Есина}
Я увидела только маленький кусочек травли...
То что увидела, отвратительно.. и правда псы..
Александр Кабанов поэт..
А кто эти люди, которые смеют?

\iusr{Себастьян Парейро}
Дякую боже що не поет

\iusr{Игорь Волков}

предмет дискуссии дословный, отодвинут в сторону, защищается Александр,
защищается свобода слова, идеалы, поэзия. Кто против? Да никто не против.
Кто-то увидел \enquote{маленький кусочек травли} и возмущен. Ок, давайте увидим не
травлю, а стих с поминанием \enquote{...нашей войны...} , с поминанием Порошенка, не
могу его найти на странице Кабанова, был еще позавчера там. А то как то
голословно все...

\iusr{Роман Филатов}

забавно, культурологическая эссеистика Анатолия не вызывает такого шквала
комментариев, а как только чуть ангажполит - набежали всем селом - уровень


\end{itemize} % }
