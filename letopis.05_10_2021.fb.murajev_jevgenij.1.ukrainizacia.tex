% vim: keymap=russian-jcukenwin
%%beginhead 
 
%%file 05_10_2021.fb.murajev_jevgenij.1.ukrainizacia
%%parent 05_10_2021
 
%%url https://www.facebook.com/yevgeniy.murayev/posts/4065252870245192
 
%%author_id murajev_jevgenij
%%date 
 
%%tags jazyk,mova,muraev_evgenii,obschestvo,ukraina,ukrainizacia
%%title Я начну говорить на украинском, когда перестанут преследовать тех, кто говорит на русском
 
%%endhead 
 
\subsection{Я начну говорить на украинском, когда перестанут преследовать тех, кто говорит на русском}
\label{sec:05_10_2021.fb.murajev_jevgenij.1.ukrainizacia}
 
\Purl{https://www.facebook.com/yevgeniy.murayev/posts/4065252870245192}
\ifcmt
 author_begin
   author_id murajev_jevgenij
 author_end
\fi

Я начну говорить на украинском, когда перестанут преследовать тех, кто говорит
на русском. Но я общаюсь на украинском языке, с теми людьми, которые обращаются
ко мне на украинском. Это видно и в моих поездках по стране. Но я не приемлю
любую форму насилия. И когда под видом украинизации на самом деле запрещают и
поражают в правах миллионы граждан – это не допустимо.

Украинизация – это процесс, в котором государство должно выделять средства на
развитие кинематографа на украинском языке, когда есть программы,
поддерживающие издательство литературы, и когда телеканалам, создающим
украиноязычный контент, дают налоговые льготы либо стимулируют. Но у нас всего
этого нет. Обратите внимание, что у нас даже закон об образовании разный для
украинского языка, европейских языков и русского. Там разный переходный период.

Да, я считаю, государственным языком должен быть украинский язык. Но всё, что
сейчас происходит – это форма надругательства над русскоязычными гражданами
Украины. И эта проблема должна быть решена с помощью механизма местных
референдумов, которые защитят права всех граждан страны, на каком бы языке они
не говорили.

\ii{05_10_2021.fb.murajev_jevgenij.1.ukrainizacia.cmt}
