%%beginhead 
 
%%file 05_01_2023.fb.ohmatdet.bolnica.1._noga_bula_vzhe_siny
%%parent 05_01_2023
 
%%url https://www.facebook.com/ndslohmatdyt/posts/pfbid02u5HRc21R7h8Eu7Wow5SZWnAtCny3gi9ZXyYLRKoCZBRLk3PJNHBgmkjj9Xd3Tqa7l
 
%%author_id ohmatdet.bolnica
%%date 05_01_2023
 
%%tags medicina,herson
%%title «Нога була вже синя і холодна»: в Охматдиті врятували кінцівку 14-річному хлопцю з Херсону
 
%%endhead 

\subsection{«Нога була вже синя і холодна»: в Охматдиті врятували кінцівку 14-річному хлопцю з Херсону}
\label{sec:05_01_2023.fb.ohmatdet.bolnica.1._noga_bula_vzhe_siny}

\Purl{https://www.facebook.com/ndslohmatdyt/posts/pfbid02u5HRc21R7h8Eu7Wow5SZWnAtCny3gi9ZXyYLRKoCZBRLk3PJNHBgmkjj9Xd3Tqa7l}
\ifcmt
 author_begin
   author_id ohmatdet.bolnica
 author_end
\fi

⚡️«Нога була вже синя і холодна»: в Охматдиті врятували кінцівку 14-річному
хлопцю з Херсону⚡️

Артем пережив окупацію Херсонщини, а коли здавалося, що найстрашніше вже
позаду, його дім обстріляли росіяни. Це сталося наприкінці листопада. Хлопчик
постраждав найбільше. В момент обстрілу він якраз йшов до підвалу, але не встиг
повністю сховатися. Снаряд розірвався прямо у дворі його будинку, дитина
отримала множинні осколкові поранення ніг та живота.💔

Батько хлопчика відвіз Артема до місцевої лікарні. Там йому надали першу
медичну допомогу, врятувавши життя. Через складність поранень, відсутність
світла у херсонській лікарні, через два дні хлопця евакуювали в Охматдит. «Нога
була вже синя і холодна, коли ми приїхали у Київ»,— згадує мама хлопчика.⚡️

Через важкий стан, спочатку хлопець знаходився у відділенні інтенсивної
терапії. Команда спеціалістів Охматдиту відновили хлопцю артерію у нозі,
вилучила множинні осколки, відновила непереривність кишківника. Через тромби в
артерії, кровообіг у нозі був мінімальний. Тривалий час у хлопця зберігалась
загроза ампутації через те, що значна частина м'язів ноги загинула. Але лікарі
змогли зберігти кінцівку хлопцю. Зараз він ще знаходиться на етапах лікування,
спеціалісти борються, аби повернути функції кінцівки. На хлопця чекає довга
реабілітація.🙏🏻

За здоров'я Артема в Охматдиті боролася мультидисциплінарна команда фахівців
різних відділень. Це мікрохірурги, абдомінальні хірурги, реаніматологи,
анестезіологи, спеціалісти відділення Emergency, психологи, діагности,
лабораторна служба. Ми продовжуємо працювати далі, приймати дітей з усіх
куточків України, приймати поранених та повертати пацієнтам здоров'я.🇺🇦

Артем точно знає, що найстрашніше вже позаду. У стінах лікарні він вже
повернувся до улюбленої справи та знову взяв до рук кларнет. 💪🏻
