% vim: keymap=russian-jcukenwin
%%beginhead 
 
%%file 04_02_2022.stz.news.lnr.lug_info.1.youtube_lnr_cenzura
%%parent 04_02_2022
 
%%url https://lug-info.com/news/you-tube-bez-obuyasneniya-prichin-udalil-kanal-lugansk-inform-centra-glavred-agentstva
 
%%author_id news.lnr.lug_info
%%date 
 
%%tags cenzura,donbass,internet,lnr,youtube
%%title YouTube без объяснения причин удалил канал ЛуганскИнформЦентра – главред агентства
 
%%endhead 
 
\subsection{YouTube без объяснения причин удалил канал ЛуганскИнформЦентра – главред агентства}
\label{sec:04_02_2022.stz.news.lnr.lug_info.1.youtube_lnr_cenzura}
 
\Purl{https://lug-info.com/news/you-tube-bez-obuyasneniya-prichin-udalil-kanal-lugansk-inform-centra-glavred-agentstva}
\ifcmt
 author_begin
   author_id news.lnr.lug_info
 author_end
\fi

Видеохостинг YouTube без объяснения причин удалил канал государственного
информационного агентства \enquote{Луганский Информационный Центр}. Об этом сообщил
главный редактор ЛИЦ Сергей Мешковой. 

\ii{04_02_2022.stz.news.lnr.lug_info.1.youtube_lnr_cenzura.pic.1}

\enquote{Минувшей ночью был удален канал \enquote{Луганского Информационного
Центра} в популярном видеохостинге YouTube, при чем это было сделано крайне
бесцеремонно, без предупреждений и уведомлений. За что, почему, как –
непонятно}, - сказал он.

В настоящее время руководство агентства разбирается в сложившейся ситуации,
направив письмо в техподдержку видеохостинга.

\enquote{Удивляет то, что канал был удален так, что его вроде и не существовало в
поиске YouTube: когда запрашиваешь (в поисковой строке канал) \enquote{Луганский
Информационный Центр}, он просто ничего не выдает, то есть такого канала вроде
бы и не было, хотя он создан был почти семь лет назад}, - отметил Мешковой. 

Главред ЛИЦ сообщил, что минувшей ночью YouTube также удалил каналы
Государственной телерадиокомпании ЛНР и пресс-службы управления Народной
милиции Республики. 

\enquote{Мы ранее сталкивались с ситуацией, когда блокировали канал или отдельное
видео, но хотя бы объясняли за что это было сделано, а в данном случае причины
такого поступка руководства YouTube нам пока не понятны. Очень надеемся сегодня
это выяснить. Будет очень жаль, если мы не восстановим канал, поскольку там
несколько тысяч достаточно популярных видеороликов – в основном брифинги
\enquote{Луганского Информационного Центра}, начиная с 2015 года. Будет обидно, если мы
их потерям навсегда}, - добавил он.
