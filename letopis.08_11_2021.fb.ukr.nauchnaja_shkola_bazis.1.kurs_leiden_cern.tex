% vim: keymap=russian-jcukenwin
%%beginhead 
 
%%file 08_11_2021.fb.ukr.nauchnaja_shkola_bazis.1.kurs_leiden_cern
%%parent 08_11_2021
 
%%url https://www.facebook.com/sciencebasis/posts/1211213979373703
 
%%author_id ukr.nauchnaja_shkola_bazis
%%date 
 
%%tags cern,fizika,gollandia,leiden.gollandia,nauka,obrazovanie,shkola,ukraina,universitet.gollandia.leiden
%%title Курс «Чим займається сучасна фізика» - Лейденський Університет та CERN
 
%%endhead 
 
\subsection{Курс «Чим займається сучасна фізика» - Лейденський Університет та CERN}
\label{sec:08_11_2021.fb.ukr.nauchnaja_shkola_bazis.1.kurs_leiden_cern}
 
\Purl{https://www.facebook.com/sciencebasis/posts/1211213979373703}
\ifcmt
 author_begin
   author_id ukr.nauchnaja_shkola_bazis
 author_end
\fi

5 учнів школи «Базис» успішно пройшли курс «Чим займається сучасна фізика» від
вчених Лейденського університету та Європейської організації з ядерних
досліджень (CERN). 

Талановиті діти відвідали курс теоретичних та практичних занять, який тривав з
жовтня 2020 року до травня 2021 року, склали іспит та нещодавно отримали
сертифікати!

\ifcmt
  ig https://scontent-lga3-1.xx.fbcdn.net/v/t39.30808-6/254981262_1211212416040526_7375811625581744911_n.jpg?_nc_cat=110&ccb=1-5&_nc_sid=730e14&_nc_ohc=5aM869rvoIkAX_net6l&_nc_ht=scontent-lga3-1.xx&oh=3a469c907cbf1f7a3f3be3bd620a8b1d&oe=61A00710
  @width 0.4
  %@wrap \parpic[r]
  @wrap \InsertBoxR{0}
\fi

Сьогодні, напередодні старту нового курсу, вони діляться своїми враженнями від
занять з науковцями у сфері сучасної практичної фізики.

Уля Ронська, учениця 11-А класу розповідає: «Цей курс проводився за участі
нашої школи, тому я вирішила до нього приєднатися, адже хотіла дізнатися більше
про сучасну фізику від людей, які цим займаються та живуть. Через це його й не
можна порівнювати зі звичайними уроками фізики, тут усі були завжди
зацікавленими, і цим створювали класну атмосферу. Дуже сподобались теми та
задачки. Особливо цікаво було слухати про темну матерію та початки квантової
хімії. Задачі та запитання були одночасно і не заскладними, і не простими, що
підтримувало інтерес. Звісно хочу порекомендувати цей курс усім знайомим, які
цікавляться фізикою». 

Враженнями ділиться і Софія Шпортюк, учениця 11-А класу: «Я обрала цей курс
тому, що мені було цікаво послухати про сучасну фізику, хоча я і не планую
серйозно займатись нею у майбутньому. Цей курс сильно відрізняється від
звичайних уроків фізики. По-перше, лекторами в ньому є не вчителі, а науковці
та аспіранти-фізики, які є експертами галузі про яку вони розповідали.
По-друге, ми розглядали цікаві теми, що не входять до шкільного курсу, але
важливі у сучасній фізиці.

Мені сподобалось багато що: теми, подача матеріалу, атмосфера. Лекції були
адаптовані для того, щоб ми могли їх зрозуміти. Хоча це і було іноді важко, але
ми завжди могли ставити запитання, лектори охоче пояснювали незрозумілі
моменти. А ще раз повторити та краще зрозуміти матеріал допомагало
розв’язування задач. До заліку ми мали змогу підготуватись, тому скласти його
було не надто складно. Але для цього потрібно було зрозуміти чимало речей, тому
й простим залік назвати не можу. 

Звісно рекомендую цей курс усім, кого цікавить сучасна фізика. Але потрібно
бути готовим вчитися, а також мати певну базу знань у фізиці, тому все ж краще
проходити цей курс в 10-11 класах, не  раніше».

Нагадаємо, що реєстрація на курс «Чим займається сучасна фізика» триває до 10
листопада: \url{https://forms.gle/1gpY32zwefy6ywjh6}

\#Basis \#цікава\_фізика
