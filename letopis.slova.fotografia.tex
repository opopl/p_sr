% vim: keymap=russian-jcukenwin
%%beginhead 
 
%%file slova.fotografia
%%parent slova
 
%%url 
 
%%author 
%%author_id 
%%author_url 
 
%%tags 
%%title 
 
%%endhead 
\chapter{Фотография}
\label{sec:slova.fotografia}

%%%cit
%%%cit_head
%%%cit_pic
\ifcmt
  tab_begin cols=3
		width 0.33

     pic https://strana.ua/img/forall/u/0/36/233941014_100725945642855_4626010813299551360_n(1).jpg

     pic https://strana.ua/img/forall/u/0/36/2021-08-09_11h21_23.png

		 pic https://strana.ua/img/forall/u/0/36/2021-08-09_10h46_07.png

  tab_end
\fi
%%%cit_text
Что известно про совместное \emph{фото}.  7 августа легкоатлетка Ярослава
Магучих завоевала бронзовую медаль по прыжкам в высоту, показав результат в 2
метра.  Более успешным стало выступление австралийской спортсменки Николы
Макдермотт, которая выдала результат в 2,02 м. И россиянки Марии Ласицкене,
которая прыгнула на 2 метра 4 сантиметра и заняла "золото".  Сразу после
исполнения прыжков легкоатлетки не сдерживали своих эмоций, взяли флаги своих
стран и сделали совместные \emph{дружеские фотографии}, которые опубликовали
мировые \emph{фотоагентства}
%%%cit_comment
%%%cit_title
\citTitle{\enquote{Поддерживают дискурс вражды}. Как власти кошмарят олимпийскую призерку Магучих за фото с россиянкой}, 
Анна Копытько, strana.ua, 09.08.2021
%%%endcit

%%%cit
%%%cit_head
%%%cit_pic
\ifcmt
  pic https://strana.ua/img/forall/u/0/36/rue96bf96f967.jpg
  width 0.4
\fi
%%%cit_text
Тогда сеть заполонили не только \emph{фото}, но видео и гиф двукратного чемпиона
Европейских игр Верняева с российским спортсменом.
Белявский тогда сказал, что Верняев "победил на все сто процентов" и рассказал
о том, что они дружат с украинскими спортсменами. "Мы действительно с ними
хорошо общаемся, дружим, нам не обязательно смотреть друг на друга косо.
Показушничать для СМИ нам смысла не было", - заявил Давид Белявский
%%%cit_comment
%%%cit_title
\citTitle{\enquote{Поддерживают дискурс вражды}. Как власти кошмарят олимпийскую призерку Магучих за фото с россиянкой}, 
Анна Копытько, strana.ua, 09.08.2021
%%%endcit

%%%cit
%%%cit_head
%%%cit_pic
\ifcmt
  pic https://strana.ua/img/forall/u/0/34/imgonline-com-ua-Resize-F2z1Hw7tMaRpz4(1).jpg
  width 0.4
\fi
%%%cit_text
\emph{Фото} со Скабеевой.  Но началось все еще вчера, когда совместное \emph{фото} с
украинским спортсменом и депутатом "Слуги народа" Жаном Беленюком у себя в
Telegram-канале обнародовала российская журналистка и ведущая Ольга Скабеева.
Она это сделала после того, как легкоатлетку Ярославу Магучих вызвали в
Минобороны после \emph{фото}, где она обнималась с российской спортсменкой на Играх в
Токио.  "Было бы справедливо вызвать в Минобороны нардепа Жана Беленюка.
Смотрите, какая зрада", - подписала Скабеева \emph{фотографию}, где Беленюк держит
медаль. При этом в руках у Скабеевой микрофон телеканала "Россия".  Судя по
форме Беленюка и серебряной медали у нег в руках, \emph{фото} было сделано на
Олимпиаде в Рио-де-Жанейро в 2016 году
%%%cit_comment
%%%cit_title
\citTitle{\enquote{Украинец ли я для своего государства?} Как после фото со Скабеевой на Беленюка напали в Киеве}, 
Оксана Малахова, strana.ua, 13.08.2021
%%%endcit

%%%cit
%%%cit_head
%%%cit_pic
\ifcmt
  pic https://lgaki.info/wp-content/uploads/2021/10/MMM_4627-768x919.jpg
  @width 0.4
\fi
%%%cit_text
\emph{Фотография} витража, созданного студенткой отделения изобразительного искусства
колледжа Академии Матусовского Олесей Скитеевой, с изображением старейшего
храма Луганска была продемонстрирована в Белграде на встрече, посвященной
историческим связам России и Сербии.  Участница встречи Ольга Милунович
отметила: — В своем выступлении я рассказала о дружбе России и Сербии, о
деятельности сербов на Донбассе, об исторических связях и о современном
положении дел. Я рассказала о луганских вузах, об Академии Матусовского. И
лучшей иллюстрацией моего рассказа стала работа Олеси Скитеевой – витраж
«Родные мотивы». На нем изображен Свято-Петропавловский кафедральный собор.
Инициатором строительства церкви апостолов Петра и Павла в Камброде в далеком
1760 году стал сербский премьер-майор Петр Иванович Шевич
%%%cit_comment
%%%cit_title
\citTitle{Работу студентки Академии Матусовского показали в Белграде}, , lgaki.info, 12.10.2021
%%%endcit

%%%cit
%%%cit_head
%%%cit_pic

\ifcmt
  tab_begin cols=2
     pic https://sharij.net/wp-content/uploads/2021/10/gbat.jpg?x47180
     pic https://sharij.net/wp-content/uploads/2021/10/250203555_1314392768990683_1451125085070350726_n.jpg?x47180
  tab_end
\fi
%%%cit_text
На рынках Украины стали появляться специальные предложения для народного
депутата от «Слуги народа» Николая Тищенко, который недавно заявил, что
покупает на Бессарабском рынке говядину по 20 грн за 1 килограмм. Об этом
сообщает Instagram «Ху*вый Киев».  Так, на \emph{фото} видны мясные обрезки с ценником
«Для Тищенко, 20 грн». Рядом на прилавке лежит более дорогое мясо
%%%cit_comment
%%%cit_title
\citTitle{«Говядина по 20» — на рынке в Киеве начали продавать мясо «для Тищенко» (фото)}, 
, sharij.net, 31.10.2021
%%%endcit

%%%cit
%%%cit_head
%%%cit_pic
\ifcmt
  pic https://avatars.mds.yandex.net/get-zen_doc/3431006/pub_617ec161fc4ba2579f66efa1_617ec207a1b9e12034c37635/scale_1200
  @width 0.4
\fi
%%%cit_text
waiting86186: Я думаю, что Алена борется с трудностями, но еще рано сбрасывать
ее со счетов. У Алены есть реальный шанс на серебро на следующем этапе (и отбор
в Финал Гран-При), и я думаю, что ее компоненты вернутся в форму вместе с ее
выносливостью. Доверие Алене от Этери - это 40-минутный спринт на беговой
дорожке.  Это был Триумф. Посмотрите на \emph{фото} ниже – какой же прекрасный
пьедестал.  Надеюсь, что трех российских девушек на этих ступенях мы увидим еще
не раз. И побитые мировые рекорды тоже
%%%cit_comment
%%%cit_title
\citTitle{Прокаты российских девушек на Скейт Канада. Отзывы иностранных болельщиков}, 
Данделайн, zen.yandex.ru, 01.11.2021
%%%endcit

%%%cit
%%%cit_head
%%%cit_pic
\ifcmt
  tab_begin cols=2
     pic https://avatars.mds.yandex.net/get-zen_doc/39788/pub_6183704f17cb4e669a6e14c1_6183817ccdb808610964973c/scale_1200
     pic https://avatars.mds.yandex.net/get-zen_doc/3006682/pub_6183704f17cb4e669a6e14c1_618383bbafa6a85265cfafb6/scale_1200
  tab_end
\fi
%%%cit_text
Татьяна из Якутска, небось, уже вкусно отобедала. А я только утренний кофе
ставлю. Благодаря Татьяне, я знаю, что в Якутске под рукой иногда не бывает
кинзы, но на днях ей отлично удались спринг-роллы: "экспериментировала на своей
компании. Блинчики были с обжаренным говяжьим фаршем, с индейкой, и сладкие с
творогом и карамелизированной айвой. Кривоватые, но вкусные".  А когда она
ставит \emph{фото} северного сияния ("не знаю чьи, ватсаппом принесло"), ее
кто-нибудь обязательно спросит - Неужели есть счастливцы, видевшие такую
красоту?  - Неоднократно))! - отвечает она.  С Анной Файфкиной мы стали
друзьями на ФБ, когда у них в Иркутской области была беда. Наводнение. И люди
остались людьми, спасая братьев наших меньших. И мы видели ролики, когда
потерявшие дома и скарб хозяева заливались слезами при встрече со спасенным
котом и собакой.  Благодаря Ане я вижу, как выглядит осень в той части России.
Украшает природу, как всегда, её Лёля
%%%cit_comment
%%%cit_title
\citTitle{Люблю свою страну}, Наталия Ефимова, zen.yandex.ru, 04.11.2021
%%%endcit

%%%cit
%%%cit_head
%%%cit_pic
%%%cit_text
Переді мною знову зблиснуло обличчя. — Два роки тому він вирушив одинцем в
експедицію до Південної Америки.  Повернувся звідти торік. Безсумнівно,
Південну Америку він одвідав, але категорично відмовляється сказати, де саме
побував. Він почав був оповідати про свої пригоди, та побачивши, що йому не
вірять, заховався в черепашці, неначе слимак. З ним, мабуть, і справді
трапилося щось незвичайне, а може, — і це ймовірніше — він просто незвичайний
брехун. Є в нього кілька зіпсованих \emph{фотографій}. Каже, що робив їх з
натури. Він став такий вразливий, що ображає кожного, хто насмілюється
розпитувати його, і виганяє з дому всіх репортерів.  На мою думку, це просто
небезпечний божевілень, хворий на манію величі і з деяким нахилом до науки. Це
якраз людина для вас, містере Мелоуне. Тепер ідіть і подумайте, що з ним можна
зробити. Ви вже досить дорослий, щоб подбати про себе, та й силою природа вам
не поскупилася. Крім того, вас захищає закон про недоторканість особи.
Усміхнене червоне обличчя знов обернулося в рожевий овал, облямований пухом.
Аудієнція скінчилася
%%%cit_comment
%%%cit_title
\citTitle{Утрачений світ}, Артур Конан Дойл
%%%endcit

%%%cit
%%%cit_head
%%%cit_pic
\ifcmt
  pic https://aif-s3.aif.ru/images/025/658/c8806d35de3e84c343f0fdd7c74becfb.jpg
  @width 0.4
\fi
%%%cit_text
Сотрудники заповедника под Воронежем напомнили жителям о необходимости
подкормки пернатых. Конкурс пройдет в онлайн-формате. В группе заповедника во
«ВКонтакте» создан \emph{фотоальбом}, в который участники могут загружать свои работы.
Кормушки должны быть из натуральных материалов. Изделия из вторсырья к участию
в экологической акции не принимаются.
\emph{Фотографии} этапов работы и самих кормушек нужно прислать до 1 февраля 2022
года. Результаты объявят 1 марта.  Победители (1,2 и 3 места) получат дипломы.
Остальным участникам выдадут сертификаты
%%%cit_comment
%%%cit_title
\citTitle{Хоперский заповедник под Воронежем объявил конкурс кормушек}, 
, vrn.aif.ru, 06.11.2021
%%%endcit
