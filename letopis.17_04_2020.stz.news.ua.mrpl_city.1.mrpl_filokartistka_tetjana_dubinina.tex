% vim: keymap=russian-jcukenwin
%%beginhead 
 
%%file 17_04_2020.stz.news.ua.mrpl_city.1.mrpl_filokartistka_tetjana_dubinina
%%parent 17_04_2020
 
%%url https://mrpl.city/blogs/view/mariupolska-filokartistka-tetyana-dubinina
 
%%author_id demidko_olga.mariupol,news.ua.mrpl_city
%%date 
 
%%tags 
%%title Маріупольська філокартистка Тетяна Дубініна
 
%%endhead 
 
\subsection{Маріупольська філокартистка Тетяна Дубініна}
\label{sec:17_04_2020.stz.news.ua.mrpl_city.1.mrpl_filokartistka_tetjana_dubinina}
 
\Purl{https://mrpl.city/blogs/view/mariupolska-filokartistka-tetyana-dubinina}
\ifcmt
 author_begin
   author_id demidko_olga.mariupol,news.ua.mrpl_city
 author_end
\fi

\ii{17_04_2020.stz.news.ua.mrpl_city.1.mrpl_filokartistka_tetjana_dubinina.pic.1}

Я не втомлююся повторювати, що Маріуполь відрізняється сво\hyp{}єю неповторністю та
унікальністю, але коли бачиш наше місто на старих листівках, закохуєшся в нього
ще більше... Пропоную познайомитися з чарівною і дуже цікавою особистістю, чия
діяльність не може не викликати інтерес. \emph{Філокартистка, фотограф і просто
небайдужа та кмітлива маріупольчанка \textbf{Тетяна Дубініна}}. Її колекція складається з
різних листівок, які є цінними фотодокументами та містять важливу інформацію з
містобудування, історії, етнографії Маріуполя. Спілкуватися з Тетяною
надзвичайно цікаво і приємно, вона з легкістю заражає своїм оптимізмом та
неабияким ентузіазмом.

Народилася Тетяна в Донецькій області (село Гранітне). Батько закінчив
Харківський політехнічний інститут. Спочатку за перерозподілом його направили
до Рязанської області, куди він і переїхав разом із сім'єю. Однак за станом
здоров'я маленької Тетянки вони були вимушені переїхали до Маріуполя. Батьки
багато часу займалися з донечкою. Таня росла всебічно розвиненою особистістю,
відрізнялася начитаністю та ерудованістю, займалася живописом. Через поганий
зір Тетяни батьки вирішили, що телевізор вдома непотрібен. Найулюбленішим
заняттям для дівчин було – переглядати листівки, які збирав її батько. Вона із
захопленням могла передивлятися їх годинами. Мабуть, саме це заняття стало
першим поштовхом до її майбутнього захоплення.

\ii{17_04_2020.stz.news.ua.mrpl_city.1.mrpl_filokartistka_tetjana_dubinina.pic.2}

Освіту дівчина отримала в Київському національному університеті культури і
мистецтв (тоді – ім. О.Є. Корнійчука). Отримавши спеціальність
бібліотекаря-бібліографа, спочатку працювала у заводській бібліотеці на
\enquote{Азовсталі}. Саме під час роботи бібліотекаркою у жінки виник великий
інтерес до історії Маріуполя. Після розформування бібліотеки вона продовжила
працювати на заводі, але вже машиністкою насосних установок. Тетяна заміжня,
має чарівного сина. Водночас у неї безліч хобі і захоплень. Проте найбільшим з
них – залишається колекціонування листівок, яке зацікавило її з дитинства.
Можливо, це у Тетяни генетичне, адже колекціонерами були і батько з тіткою. До
речі, свої колекції вони передали нашій героїні. Цікаво, що саме з листівок
маріупольчанка вчилася перемальовувати, можна сказати, що вони стали основою
для реалізації її художньо-творчого потенціалу.

Колекція включає в себе декілька напрямів: \emph{міста і містечка України, святкова
тематика, балет, квіти, полюванні та рибалка}. Зокрема, міста України включають
до 200 листівок, міста Донецької області – 150, Маріуполь – 290. Тетяна знайшла
багато однодумців. В бібліотеці ім. Короленка двічі проводилася виставка з її
листівками. Також у 2017–2018 роках в Маріупольському механіко-металургійному
коледжі проводилася творча зустріч з нею на тему \emph{\enquote{Історія в листівках}}. Вона не
просто збирає листівки, а вже досить добре знається на їх специфіці,
особливостях, історії створення. На виставках Тетяна з легкістю передає
учасникам любов до свого захоплення, відкриває двері в несподівано цікавий і
дивовижний світ листівки. Маріупольчанка розповідає про історію поштової
листівки, рух філокартистів, про те, як її інтерес до історії рідного міста
переріс у колекціонування листівок. 

\ii{17_04_2020.stz.news.ua.mrpl_city.1.mrpl_filokartistka_tetjana_dubinina.pic.3}

Сім'я підтримує захоплення нашої героїні. Зокрема, чоловік часто розповідає
друзям і знайомим про унікальне хобі дружини, після чого приносить додому нові
листівки, які передають Тетяні. Батьки чоловіка також намагаються допомагати
поповнювати колекцію. На роботі співробітниць таке хобі вражає, вони
намагаються на старих листівках впізнати знайомі вулиці рідного міста. Інколи,
дізнавшись про колекцію Тетяни, колеги приносять листівки, що знаходять вдома.
Захоплюючись своїм заняттям, вона почала більше цікавитися минулим Маріуполя,
тому найбільш цікавими для неї наразі є листівки з Маіруполя дореволюційного,
що є дійсно раритетом. Нещодавно жінка поповнила колекцію унікальною серією
\enquote{Маріупольські метаморфози} (Маріуполь у 1930-х роках). Також мріє доторкнутися
до листівки часів Другої світової війни, де одну половину займає зображення, а
іншу – текст. Сподівається поповнити свою колекцією і яскравими та самобутніми
листівками російської художниці \emph{Єлизавети Бьом}. Вони відрізняються неповторним
стилем, а завдяки зображенням дітей і мудрим словам несуть потужну духовну
складову та мають позитивну енергетику.

\ii{17_04_2020.stz.news.ua.mrpl_city.1.mrpl_filokartistka_tetjana_dubinina.pic.4}

Втім у Тетяни є й інші великі захоплення. Вона вже давно професійно  займається
фотографією та обожнює подорожувати. Щодо останнього хобі, то завдяки її
організаторським здібностям колеги по роботі отримали можливість подорожувати
частіше. Якщо б не було карантину, вони і зараз відправилися б до Дніпра... Але
життя вносить свої корективи. Подорожі Тетяну дуже надихають і вона щиро радіє,
що знайшла на роботі однодумців та може тепер подорожувати не одна. Також
маріупольчанка обожнює фотографувати природу та маріупольські вулиці, будинки.
Вона неодноразово була учасницею і переможницею міських конкурсів, що
проводилися Центральною бібліотекою імені Короленка, чи комбінатом \enquote{Азовсталь}.

\ii{17_04_2020.stz.news.ua.mrpl_city.1.mrpl_filokartistka_tetjana_dubinina.pic.5}
\ii{17_04_2020.stz.news.ua.mrpl_city.1.mrpl_filokartistka_tetjana_dubinina.pic.6}

\begingroup
\em
\textbf{Улюблена книга:} \enquote{Три товариші} Еріх Марія Ремарка

\textbf{Улюблений фільм:} \enquote{Реальна любов} (2003 рік)

\textbf{Порада маріупольцям:} 

\begin{quote}
\enquote{Любіть і бережіть наше місто. Правильно обирайте владу, яка буде намагатися
працювати задля розвитку Маріуполя. А ще – частіше фотографуйте не себе, а
місто. Фото з Маріуполем 1950-х чи 1960-х років не вистачає. Хочеться, щоб в
майбутньому наші нащадки знали, я виглядало наше місто...}.
\end{quote}

\textbf{Курйозний випадок:} 

\begin{quote}
Запам'яталося як їхала з відомою маріупольською активісткою Ларисою Петровою,
яка уважно слухала про хобі Тетяни. А зранку Тетяна побачила, що Лариса вийшла
раніше, але залишила на згадку про зустріч доволі незвичну сучасну листівку із
зображенням Маріуполя (фотограф Євген Сосновський). Філокартистка дуже зраділа
раптовому поповненню колекції і такому цікавому знайомству.
\end{quote}
\endgroup
