% vim: keymap=russian-jcukenwin
%%beginhead 
 
%%file 07_12_2022.fb.hrystenko_hrystyna.ternopil.1.istyna
%%parent 07_12_2022
 
%%url https://www.facebook.com/anna1056/posts/pfbid0AAGdBEsLoipBEJThkd9Nj2JwkZM5JS2NuBs4fd9mk8QkPxxGQYYC2B71pk9kAaMAl
 
%%author_id hrystenko_hrystyna.ternopil
%%date 
 
%%tags 
%%title Для чого шукати істину на цій землі? Її тут немає?
 
%%endhead 
 
\subsection{Для чого шукати істину на цій землі? Її тут немає?}
\label{sec:07_12_2022.fb.hrystenko_hrystyna.ternopil.1.istyna}
 
\Purl{https://www.facebook.com/anna1056/posts/pfbid0AAGdBEsLoipBEJThkd9Nj2JwkZM5JS2NuBs4fd9mk8QkPxxGQYYC2B71pk9kAaMAl}
\ifcmt
 author_begin
   author_id hrystenko_hrystyna.ternopil
 author_end
\fi

Для чого шукати істину на цій землі? Її тут немає?

Все, що ми шукаємо і не знаходимо тут, аж до кінця життя, насправді ніколи не
знайдеться. 

Вічна дилема щастя у багатьох залишається дилемою, яку розгадати, а тому
пізнати не можливо. 

Все ті ж питання, все ті ж пошуки, і недолугі, неповні, несправжні відповіді,
які не здатні задовільнити духовний голод людини. 

Так, людина тлінна, недосконала, слабка, але її дух прагне справжнього,
вічного, істинного, досконалого - що називається щастям. 

Всі хочуть бути щасливі, але кожен називає щастя по-своєму, більше того, і в
різний час життя, щастя змінює свій лик. 

Дивно, чи не так? Одне обличчя щастя замішає інше, а те інше ще інше… і ще
інше, бо голод вгамувати не вдається. Це як «їсти» повітря, дим, максимум воду,
але хліба як не було, так і немає щоби насититися. Коли свідомість приймає те,
що дає їй людина, а підсвідомість добре знає, що не те. Вона штовхає людину
шукати той хліб, але його немає в магазині, та і як він виглядає людина не
знає… 

На землі немає нічого справжнього і поживного для людської душі окрім «Неба». 

Вчора мені самовпевнено написали в приват, аби я спускалася з неба... Я
жартівливо відповіла, що мені там добре і посміхнулася в душі. А тоді зловила
себе на думці, що я сказала правду. 

Щоб жити на небі, не обов'язково вмирати, чи майструвати халабуду високо між
хмарами, або літати безперервними рейсами по світу...Достатньо дивитися на світ
через призму неба. 

Все стає по-іншому. Зрозуміло. Справжньо. Глибоко. З Любов'ю. 

Приходить усвідомлення, що вже не зникає, але дає розуміння щастя, яке не
знайти на землі. Його можна лише набрати «повні кишені» в небі, а тоді
«спуститися», лише для того, аби пожити і для фізичного тіла. 

Знаєте, що таке щастя для мене?

Це легкість, мир, радість, всеможливість і вседосяжність за будь-яких обставин. 

Головне, що з Ним. 

\#психотерапевтпохристиянськи

\ii{07_12_2022.fb.hrystenko_hrystyna.ternopil.1.istyna.orig}
\ii{07_12_2022.fb.hrystenko_hrystyna.ternopil.1.istyna.cmtx}
