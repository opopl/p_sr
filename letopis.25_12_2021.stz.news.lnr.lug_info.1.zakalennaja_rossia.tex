% vim: keymap=russian-jcukenwin
%%beginhead 
 
%%file 25_12_2021.stz.news.lnr.lug_info.1.zakalennaja_rossia
%%parent 25_12_2021
 
%%url https://lug-info.com/news/luganchane-prinyali-uchastie-v-obsherossijskoj-akcii-zakalennaya-rossiya-zdorovaya-strana
 
%%author_id 
%%date 
 
%%tags 
%%title Луганчане приняли участие в общероссийской акции "Закаленная Россия – здоровая страна"
 
%%endhead 
\subsection{Луганчане приняли участие в общероссийской акции \enquote{Закаленная Россия – здоровая страна}}
\label{sec:25_12_2021.stz.news.lnr.lug_info.1.zakalennaja_rossia}

\Purl{https://lug-info.com/news/luganchane-prinyali-uchastie-v-obsherossijskoj-akcii-zakalennaya-rossiya-zdorovaya-strana}

Жители Луганска присоединились к общероссийской акции \enquote{Закаленная Россия –
здоровая страна}, приняв участие в заплыве в пруду у села Стукалова Балка
Славяносербского района. Об этом с места события передает корреспондент ЛИЦ.

\ii{25_12_2021.stz.news.lnr.lug_info.1.zakalennaja_rossia.pic.1}

В акции приняли участие представители луганского клуба закаливания и зимнего
плавания \enquote{Морж}, казаки регионального отделения общероссийской общественной
организации по развитию казачества \enquote{Союз казаков-воинов России и зарубежья},
спортсмены Республиканского физкультурно-спортивного общества (РФСО) \enquote{Динамо},
любители активного образа жизни.

\ii{25_12_2021.stz.news.lnr.lug_info.1.zakalennaja_rossia.pic.2}

\enquote{Сегодня по городам Российской Федерации проходит акция
\enquote{Закаленная Россия – здоровая страна}, и второй раз подряд в ней
принимают участие жители Луганской Народной Республики, потому что Луганск – за
здоровый образ жизни}, - рассказал советник главы ЛНР по вопросам казачества,
кадетского образования и военно-патриотического воспитания молодежи, морж со
стажем Сергей Юрченко.

\ii{25_12_2021.stz.news.lnr.lug_info.1.zakalennaja_rossia.pic.3}

Так как с ночи началась оттепель и лед на водоеме успел подтаять и истончиться,
любители закаливания решили в этот раз прорубить не прорубь, а полынью от
берега к центру водоема. Необычная ледяная дорожка, температура воды плюс 1
градус и хорошее настроение сделали заплыв, по словам участников акции,
\enquote{идеальным} и \enquote{комфортным}.

\ii{25_12_2021.stz.news.lnr.lug_info.1.zakalennaja_rossia.pic.4}

\enquote{В Луганской Народной Республике акция проходит при поддержке заместителя
министра культуры, спорта и молодежи ЛНР Корниенко Александра Николаевича.
Сегодня окунулись в прохладу зимнего пруда порядка 30 человек разного возраста
– от трех месяцев до 70 лет}, - проинформировала руководитель луганского клуба
закаливания и зимнего плавания \enquote{Морж} Оксана Бондарь.

\ii{25_12_2021.stz.news.lnr.lug_info.1.zakalennaja_rossia.pic.5}

Луганчанка Зоя приехала с трехмесячной дочкой Машей. Первый этап закаливания
девочки – это \enquote{свежий воздух и смех людей вокруг}. Ее мама уже несколько зим
является членом луганского клуба моржей, и на собственном опыте \enquote{ощутила пользу
для здоровья организма от плавания в холодной воде}.

\ii{25_12_2021.stz.news.lnr.lug_info.1.zakalennaja_rossia.pic.6}

Студентка выпускного курса Института физического воспитания и спорта Луганского
государственного педагогического университета, фитнес-инструктор и постоянная
участница заплывов в ледяной воде Ирина Гущина организовала для товарищей по
увлечению разминку до и после заплыва.

\ii{25_12_2021.stz.news.lnr.lug_info.1.zakalennaja_rossia.pic.7}

\enquote{Плотность воды выше, чем воздуха, и в воде ты быстрее охлаждаешься. И если ты
сразу нырнешь – для организма может быть шок: больно, плохо, сосуды сузились. А
тут ты уже тело подготовил, тело акклиматизировалось, оно знает, что \enquote{сейчас мы
пойдем нырять}. И уже идешь спокойно и так же спокойно выходишь. То есть
разминка – обязательна!} – пояснила инструктор.

\ii{25_12_2021.stz.news.lnr.lug_info.1.zakalennaja_rossia.pic.8}

После заплыва Бондарь выдала участникам акции дипломы, подтверждающие участие в
общероссийской акции, а самые активные из них получили от РФСО \enquote{Динамо}
спортивные костюмы.

Завершилось мероприятие чаепитием на берегу озера.

Общероссийская акция \enquote{Закаленная Россия – здоровая страна} проводится ежегодно
в преддверии Нового года во всех регионах России. В рамках акции все любители
моржевания, закаливания, зимнего плавания и здорового образа жизни окунаются в
холодную воду местных водоемов и обливаются холодной водой.

Напомним, в начале марта 2021 года спортсменки луганского клуба закаливания
\enquote{Морж} впервые представили Луганск на международных соревнованиях по зимнему
плаванию \enquote{Кубок Большой Невы}, который в пятый раз прошел в Санкт-Петербурге. 
