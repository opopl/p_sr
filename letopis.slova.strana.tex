% vim: keymap=russian-jcukenwin
%%beginhead 
 
%%file slova.strana
%%parent slova
 
%%url 
 
%%author 
%%author_id 
%%author_url 
 
%%tags 
%%title 
 
%%endhead 
\chapter{Страна}
\label{sec:slova.strana}

Авіація давно лежить в основі економічного успіху багатьох країн світу, з
перших днів польоти допомагали налагоджувати міжнародні торгові зв’язки та
створювати життєво важливі внутрішні зв’язки, які \enquote{зшивають} \emph{країну}. В
авіації, як і в інших галузях, які знаходяться не в фокусі топ-чиновників, за
роки накопичилося багато проблем.  Сьогодні ми отримали реальний шанс
кардинально покращити ситуацію в галузі, адже Міністром інстраструктури став
Олександр Кубраков, людина яка не на словах знає, що таке дійсно будувати
\emph{країну} кожного дня, під керівництвом якого було збудовано і реконструйовано
близько 9 тис. км доріг тільки державного значення,
\citTitle{Сім кроків до державної політики розвитку та розбудови української авіації}, Георгій Зубко, www.epravda.com.ua, 08.06.2021

\citEntry{%
По той же причине: нет ясности со \emph{страной}. Будет или нет? В поведении
власти прочитывается четкий сигнал: она сама не верит в жизнеспособность
государства, которые управляет. А потому занимается тотальным ограблением
остатков, распродавая все за бесценок.  Четвертое - все отвлекающие
общественное внимание маневры власти, типа санкций, суда над Медведчуком и
партией Шария, референдумы по Донбассу и статусу олигархов и пр. и пр. и пр.
имеют очень короткий ресурс эффективности. И к осени, когда люди увидят рост
цен на все и вся в полтора раза, едва ли станут предохранителем от массовых
выступлений. Возможно - бунта. Ресурс терпения у пересичных близится к нулю. С
переходом в отрицательные значения по отношению к Зе-банде. Так что вангую
смену власти осенью этого - зимой-весной следующего года
}{%
\citTitle{На глазах нарастает грандиозный кризис украинской государственности}, Валерий Песецкий, strana.ua, 09.06.2021
}

Прямо на наших глазах происходит беспрецедентный дерибан \emph{Страны}.  Одноходовые
схемы Блогера. Здравствуй, Люба. Я вернулся. Три месячных бана в этом году, а
ведь на календаре только июнь. Каждый второй месяц в карцере. 100\%-е
показатели пятилетки Петра.  Хотя какие тут могут быть претензии? «Нормы
сообщества ФБ» - это правила гопников. Это их площадка и их район, и ты всегда
виноват уже тем, что не вышел мордой.  Только давайте не забывать базовое.
Черные списки неугодных формируют в Офисе Зеленского. И это его шарашка
непосредственно стоит за банами оппонентов. А не забывать будем ради главного.
Когда «настанет день, настанет час», и мы это чучело, как ранее его
предшественника, выкинем на помойку.  Правда, когда этот «день и час»
настанет, подозреваю, у нас тут уже останется Мозамбик.  Пацанва даже не
утруждает себя выдумыванием хотя бы двухходовых схем по распилу \emph{страны}. Толпа
мародеров, которая выносит торговый центр, на их фоне смотрится более
респектабельно,
\citTitle{Эта власть построила нам кунсткамеру в смартфоне}, Игорь Лесев, strana.ua, 09.06.2021

Вот так \emph{Страна} и скатывается к краю финансовой пропасти. СУПЕРХЛЕБНАЯ
ИДЕЯ ДЛЯ УКРАИНСКИХ АГРОБАРОНОВ, КАК ПО-МАКСИМУМУ ОБОГАТИТЬСЯ И ОДНОВРЕМЕННО
ОБНУЛИТЬ ГОСБЮДЖЕТ-2021. Они нас тащат к краю самой глубокой финансовой
пропасти; такое ощущение, что завтра уже не будет. Судите сами, только один ряд
беспристрастных цифр из открытой отчетности крупных экспортеров агропродукции,
свидетельствующий о минимальных размерах уплаченных ими налогов и максимальных
заработках на возмещении НДС. Вот данные Государственной налоговой службы о
том, насколько больше денег из госбюджета вернули экспортерам АПК, и сколько
они заплатили налогов,
\citTitle{Государственная налоговая служба обнародовала шокирующие цифры возмещения НДС}, Александр Гончаров, strana.ua, 09.06.2021


%%%cit
%%%cit_pic
%%%cit_text
Дело в том, что маленьким по территориям \emph{странам} значительно легче организовать
свою жизнь, найти общественный консенсус, разработать привлекательные
инвестиционные идеи. Крохотный Сингапур с населением 5 млн чел генерирует 102
тыс дол на душу. Еще эффективнее такие \enquote{малыши} как Катар и Кувейт.
Процветают и сравнительно небольшие Швейцария и Тайвань.  \emph{Маленькой стране}
проще создать эффективную налоговую систему и привлечь капитал.  Несколько лет
назад российская IT-компания разработчик игр Game Insight перенесла офис в
Литву. Ее выручка — \$100 млн в год и почти вся она в твердой валюте. Для Литвы
с ее 2.8 млн чел населения это была огромная сумма. Если бы этому примеру
последовала \enquote{Лаборатория Касперского} с выручкой в \$667 млн (2013
год), то Литва закрыла бы треть дефицита торгового баланса (€1,9 млрд). Еще
пара таких компаний, и нет проблемы
%%%cit_comment
%%%cit_title
\citTitle{Мы живем в мире, когда размер перестал иметь значение}, 
Андрей Головачев, strana.ua, 13.06.2021
%%%endcit

%%%cit
%%%cit_head
%%%cit_pic
%%%cit_text
Я думаю, \emph{Страна}, в которой сегодня не удалось еще создать глупого
человека - робота, это Россия. Нам постоянно нужно поделать что-то для души,
потому что хочется. Находясь на работе, в большинстве своем мы думаем только об
одном, когда этот чертов день закончится и побыстрее уже свалить отсюда. Для
нас работа находится далеко не на первом месте. В первую очередь для нас важно
пожить, вечно поспорить, что-то обсудить, в целом чувствовать себя живым.
Россия, как и во все времена, сегодня полна яркими и живыми личностями, каждая
из которых имеет свой уникальный характер. Начиная от простого работяги и
вплоть до нерадивых чиновников. Вот это всегда и напрягало запад - русская
непокорность и Воля. Иначе говоря - проявление характера.  Вот друзья, в чем
заключается по мнению брата главное отличие человека русского от человека
запада. И я с ним пожалуй соглашусь
%%%cit_comment
%%%cit_title
\citTitle{После 11 лет жизни в США понял самое важное на мой взгляд отличие между американцами и русскими}, 
За Пределами Онлайна, zen.yandex.ru, 06.06.2021
%%%endcit

%%%cit
%%%cit_head
%%%cit_pic
\ifcmt
  pic https://avatars.mds.yandex.net/get-zen_doc/3414416/pub_60cf26e0b4fa377376e18e2f_60cf3048b4fa377376fd744f/scale_1200
	width 0.4
\fi
%%%cit_text
Прошло ровно 80 лет с того дня, когда мир для нас раскололся на \enquote{до} и \enquote{после}.
Сегодня день памяти и скорби. Сегодня мы скорбим обо всех жертвах этой
чудовищной бойни. 22 июня на нашу родину напали враги. А уже через несколько
дней на перроне Белорусского вокзала впервые прозвучало: ВСТАВАЙ, \emph{СТРАНА}
ОГРОМНАЯ!
%%%cit_comment
%%%cit_title
\citTitle{Когда у тебя мурашки побегут от этой песни, тогда ты поймёшь...}, 
СтихиЯ Высоцкого, zen.yandex.ru, 22.06.2021
%%%endcit

%%%cit
%%%cit_head
%%%cit_pic
%%%cit_text
Украина – \emph{страна} непредсказуемого будущего. Мы привыкли ставить над собой
эксперименты. Мы гордимся своей непрогнозируемостью, но это качество вряд ли
относится к нашим достоинствам.  Потому что в той цивилизации, частью которой
мы хотим стать, прогнозируемость считается преимуществом. Договора принято
выполнять. Обещания – сдерживать. А потому любое демократическое общество
отличает наличие институтов. Тех самых, что способны сдержать эксцентрику
любого калифа на час
%%%cit_comment
%%%cit_title
\citTitle{Ключи от мира в Украине находятся в руках Кремля}, Павел Казарин, news.obozrevatel.com, 24.06.2021
%%%endcit

%%%cit
%%%cit_head
%%%cit_pic
%%%cit_text
Мы должны запустить политический процесс эффективного управления экономикой,
социальной сферой, государством. Это вопрос украинской независимости и
государственного суверенитета. Нынешняя власть оказалась беспомощной в тарифной
и социальной политике, вакцинации населения и оборонной стратегии. А основа
фиаско ‒ в полном непонимании экономических и социальных процессов в \emph{стране}.
Реальный анализ подменяется политическими штампами и охотой на ведьм.  СМИ,
которые честно анализируют ситуацию, закрываются, а политики, журналисты и
эксперты, которые говорят правду о реальном положении дел, преследуются. А ведь
о чем говорят те, кого репрессируют? Власть управляет неэффективно, \emph{стране},
народу, государству эта власть приносит одни убытки.  Вот мы и должны научиться
хозяйствовать у себя в \emph{стране}, а не запускать в наше хозяйство жуликов и
проходимцев.  Вот мой рецепт ответа на вызовы, о которых говорит Владимир
Путин. Мы должны делом доказать состоятельность своей \emph{страны}. Состоятельность
экономическую, правовую, политическую. И тут нет ничего нового. Каждое
независимое государство должно каждый день доказывать свою состоятельность, и
доказывать веско, а не популистскими заявлениями и демаршами
%%%cit_comment
%%%cit_title
\citTitle{О будущем украинского и русского народов}, 
Виктор Медведчук, strana.ua, 15.07.2021
%%%endcit

%%%cit
%%%cit_head
%%%cit_pic
\ifcmt
  pic https://img.strana.ua/img/article/3463/hod-v-istorii-43_main.jpeg
  width 0.4
	caption Смерть Гонгадзе запустила цепочку событий, которые в итоге привели к первому Майдану. Фото УП 
\fi
%%%cit_text
"Страна" рассказывает об истории современной нам Украины в проекте
"Украина.30". Он посвящен каждому из 30 лет независимости.  Сегодня мы говорим
о последнем годе второго тысячелетия - 2000-м.  Для Украины он и правда стал
переломным.  С одной стороны - закончился длившийся почти десять лет
экономический спад "лихих 90-х".  С другой - \emph{страна} сделала первый шаг в
сторону майданов, которые за последующие 15 лет несколько раз резко поменяют
ход истории Украины, приведя к войне и потере территорий.  Цепочка событий по
дестабилизации ситуации в \emph{стране} начнется с убийства Георгия Гонгадзе и
кассетного скандала - оба события случились в 2000-м году. Этот  же год дал
старт большой политической карьере Виктора Ющенко и Юлии Тимошенко
%%%cit_comment
%%%cit_title
\citTitle{Убийство Гонгадзе и первый шаг к Майдану. Чем жила Украина в 2000-м году}, 
Игорь Гужва; Олеся Медведева; Максим Минин, strana.ua, 04.08.2021
%%%endcit

%%%cit
%%%cit_head
%%%cit_pic
\ifcmt
  pic https://img.strana.ua/img/article/3476/taras-kremin-protiv-17_main.jpeg
  width 0.4
\fi
%%%cit_text
Языковый омбудсмен Тарас Креминь, который недавно призвал всех недовольных
политикой украинизации уезжать из \emph{страны}, пошел дальше и увидел зраду в
трансляции на ТВ-каналах русскоязычных сериалов и фильмов. По его мнению это
"масштабная угроза национальной безопасности Украины".  Об этом он публично
заявил в эфире ток-шоу "Первые лица" на "24 канале".  По его мнению трансляция
сериалов на русском языке нарушает принятый Радой закон о тотальной
украинизации, согласно которому с 16 июля абсолютно все фильмы на телевидении
должны демонстрироваться с украинским дубляжом.  "Если исходить из позиции
Конституционного Суда Украины, то можно говорить, что это угроза
нацбезопасности \emph{страны}", - заявил омбудсмен
%%%cit_comment
%%%cit_title
\citTitle{Креминь назвал угрозой нацбезопасности фильмы на русском языке}, 
Александр Максюк, strana.ua, 07.08.2021
%%%endcit

%%%cit
%%%cit_head
%%%cit_pic
%%%cit_text
Водночас це завдання і навіть копняк в одне місце для українців та наших еліт
нарешті дати відповідь на запитання «Україна – це…?». Ми досі не можемо
відповісти, якою ми є \emph{країною} по суті, за формою. Не можемо розібратись з
ідентичністю, культурою та минулим. Хоча війна мала б сприяти нашій
ідентифікації, але насправді тільки відкрила рани, які наразі ніхто не хоче
лікувати.  У питанні визначеності ми – \emph{країна безвідповідальних людей}. Вважати,
що, внісши до Конституції бажання вступу в Європейський Союз та НАТО, ми
поставили крапку в розумінні того, хто ми є та чого хочемо – велика помилка
%%%cit_comment
%%%cit_title
\citTitle{Україна це – ...}, Станіслав Безушко, zaxid.net, 02.08.2021
%%%endcit

%%%cit
%%%cit_head
%%%cit_pic
%%%cit_text
Украинская полиция освободила 120 человек, которых держали в трудовом рабстве.
"Пострадавших подыскивали на железнодорожных вокзалах Киева, Днепра и Одессы.
Потенциальных сотрудников "работодатели" обещали обеспечить жильем, питанием и
заработной платой от 3000 до 4000 гривен в месяц. Завербованных на работу лиц
правонарушители доставляли на микроавтобусах и расселяли в полуразрушенных
домах, где отсутствовали надлежащие условия для проживания. Рабочих заставляли
работать от 12 до 16 часов в производственно-складских помещениях и на полях в
сезон сбора урожая. Питание граждане получали из испорченных продуктов, а
обещанные деньги им вообще не выплачивались", - рассказывают представители МВД.
Недавно я написал статью об этой проблеме - но ее масштабы гораздо больше, чем
можно себе представить. \emph{Страна} рабов, \emph{страна} господ - это как раз
про современную Украину
%%%cit_comment
%%%cit_title
\citTitle{Страна рабов, страна господ - это про современную Украину / Лента
соцсетей / Страна}, Андрей Манчук, strana.news, 10.09.2021
%%%endcit

%%%cit
%%%cit_head
%%%cit_pic
\ifcmt
  pic https://img.strana.news/img/article/3584/ukrainoj-upravljaet-ne-20_main.jpeg
  @width 0.4
\fi
%%%cit_text
Как считает Путин, ситуация в Украине находится в тупике. По его словам, у
власти в Киеве стоит не президент, а группа людей, которая руководит \emph{страной}.
"Складывается впечатление, что народу Украины не дадут легальными способами
сформировать органы власти, которые отвечают их чаяниям", - заявил глава
соседнего государства.  Он добавил, что народного депутата от ОПЗЖ Виктора
Медведчука в Украине пытаются привлечь к ответственности за открытую
политическую позицию, направленную на стабилизацию в \emph{стране}
%%%cit_comment
%%%cit_title
\citTitle{Ситуация тупиковая". Путин заявил, что Украиной правит не президент, а народу не дают выбрать желаемую власть}, 
Наталья Полулях, strana.news, 21.10.2021
%%%endcit

%%%cit
%%%cit_head
%%%cit_pic
%%%cit_text
В общем, куда ни кинься, везде названия \emph{стран} происходят или от названия
племен, или от имени царей или богов. Исключений практически нет. Даже название
«Швейцария» происходит не от современной профессии «швейцары», а от более
благородного слова – «корчевщики». Да и «Канада» на языке индейцев ирокезов
означает «край». Но никогда ирокезы не додумались бы назвать свою \emph{страну}
«окраиной».  Вы представляете себе страну, которая называется «Окраина»?
%%%cit_comment
%%%cit_title
\citTitle{Переименуем Украину из Окраины снова в Русь и заживем гордо и богато!}, 
Исторический понедельник, zen.yandex.ru, 24.10.2021
%%%endcit

%%%cit
%%%cit_head
%%%cit_pic
\ifcmt
  tab_begin cols=2
     pic https://avatars.mds.yandex.net/get-zen_doc/5235961/pub_615f6b3f88d7d5437bfac53f_615f6b4456520d1b9be91fc3/scale_1200
     pic https://avatars.mds.yandex.net/get-zen_doc/4600043/pub_615f6b3f88d7d5437bfac53f_615f6b4221504518faaca068/scale_1200
  tab_end
\fi
%%%cit_text
Или когда в реалии одной \emph{страны} неожиданно врываются реалии другой - и
не сразу успеваешь переменить культурную линзу.  Так и тут - вроде я в Украине,
где, особенно на контрасте с Россией, быстро привыкаешь к расслабону - никакого
тебе театра безопасности, всех этих бесконечных держиморд, охранников,
вахтеров, сторожей, ментов с необъятными пузами, гвардейцев с запотевшими
забралами, рамок, обысков, рентгенов, собак, проверок паспортов, перегороженных
входов и выходов, пугалок со всех сторон о терроризме и врагах, которые
подползают по всем границам, дабы изобидеть больно вновь народ русский, подолы
бабам оборвать, запасы солёных рыжиков отобрать.  И вдруг - в \emph{стране}
победившего расслабона, в безалаберном, но благодатном и гедонистическом городе
Киеве, да не разразит тебя, град благословенный, Рагнарёка гром, такой
беспардонный пассаж
%%%cit_comment
%%%cit_title
\citTitle{Украина. Киев. Арт-завод Платформа - флагман несвободы, или как я внезапно нашел островок России в украинской столице}, 
Александр "haydamak" Бутенко, zen.yandex.ru, 08.10.2021
%%%endcit

%%%cit
%%%cit_head
%%%cit_pic
%%%cit_text
Я хочу сказати, що мова здатна постійно змінюватися. І еліта повинна вирішувати
як ці зміни застосовувати. З однієї сторони, є запит частини населення України
за українізацію (і це непоганий процес). Але є молодь, яка думає вже зовсім
іншими категоріями ніж розвиток рідної мови. Елітам (в першу чергу політичним)
без різниці що відбувається в \emph{країні}, вони лише виявляють і розвивають ті
тренди які допомагають їм отримати владу.' і "викачувати" ресурси зі свого
народу, але це вже питання до нашої еліти. Треба памятати, що історію, в чергу
чергу, "робить" не пасивна більшість, а активна меншість. Тому у 2014 році
після Революції гідності "патріотичні сили" підхопили тренди на патріотизм і
лише використали їх для отримання та утримання влади. Наприклад, рейтинг партії
"Народний Фронт" за два роки після виборів впав в 55 разів. Як тільки люди
побачили лицемірство політиків - їх рейтинг почав швидко знижуватися. Але
представники цієї політичної сили були значною частиною влади всі 5 років
%%%cit_comment
%%%cit_title
\citTitle{Мова - це зброя? Чому я розмовляю українською, але проти "мовної диктатури"}, 
Максим Данильченко, analytics.hvylya.net, 31.10.2021
%%%endcit

%%%cit
%%%cit_head
%%%cit_pic
%%%cit_text
Біблійний Арон так пояснив Мойсеєві, чому його неслухняний народ знову
поклонився золотому тельцю: «Ти сам знаєш цих людей, як їх тягне до зла» (Вих.
32:22). Не здивуюся, якщо хтось скаже так і про нас. Але я переконаний, що наш
народ робить вибір на користь зла не тому, що він якийсь органічно порочний, а
тому що вибір на користь добра видається йому програшним. Ось чому всі хочуть,
щоб \emph{країна} змінилася, але... нема дурних: хай міняються спершу інші. Журавель у
небі прекрасний, але я вже якось і з синицею в руках переб’юся
%%%cit_comment
%%%cit_title
\citTitle{Чи зміниться чинний суспільний договір без наших зусиль?}, 
Мирослав Маринович, zbruc.eu, 06.11.2021
%%%endcit

%%%cit
%%%cit_head
%%%cit_pic

\ifcmt
  tab_begin cols=2
     pic https://avatars.mds.yandex.net/get-zen_doc/40456/pub_6184f5d149c7c41b209fdc08_6184f77abb6ab00e7633d522/scale_1200
     pic https://avatars.mds.yandex.net/get-zen_doc/1640581/pub_6184f5d149c7c41b209fdc08_6184f78ab50152750dd37aac/scale_1200
  tab_end
\fi
%%%cit_text
Говорят, дьявол ничего сам не может создать, поэтому извращает божественные
творения.  Не хочу проводить параллелей, но есть у нас рядам \emph{страна}, где
усиленно перелицовывают все, до чего могут дотянуться. Сносят памятники,
переименовывают города и села, сбивают символы с зданий, переписывают историю и
заставляют детей в школах называть чужой язык родным. Вместо того чтобы
создавать что-нибудт аутентичное, активисты этой \emph{страны} активно крадут
чужое и выдают за свое. Так было с историческими личностями, с культурными
ценностями, даже символами... В результате рождается нация манкуртов, которые
забыли свои корни и живут в выдуманном мире, полном ненависти к братьям
%%%cit_comment
%%%cit_title
\citTitle{Небратья обокрали русского художника}, Бунтовский, zen.yandex.ru, 08.11.2021
%%%endcit

%%%cit
%%%cit_head
%%%cit_pic
\ifcmt
  tab_begin cols=3
     pic https://avatars.mds.yandex.net/i?id=ea139535cf1c02c82fce6f6c4147c217-4570724-images-thumbs&n=13
     pic https://avatars.mds.yandex.net/i?id=41223ceda3cf7faec2ce00c2e98f6f9f-4937208-images-thumbs&n=13
		 pic https://avatars.mds.yandex.net/i?id=9268602cfdb69ada89c81656ed09b892-5544440-images-thumbs&n=13
  tab_end
\fi
%%%cit_text
Проведення російського перепису населення в Криму і Севастополі засудили у
Великій Британії, ЄС, США, Міністерстві закордонних справ України та низці
інших країн. В українському Міністерстві закордонних справ заявили, що такі дії
Росії в Криму — це «недружній акт» проти України і що офіційний Київ залишає за
собою право застосування заходів у відповідь.  Речник Міністерства закордонних
справ України Олег Ніколенко в ефірі радіо Крим.Реалії заявив: «Ми розглядаємо
такі дії Росії як підрив державного суверенітету і територіальної цілісності
нашої \emph{країни}, це подальша спроба легітимізувати анексію Криму, поширивши
на його територію дію російського законодавства. Ми активно задіяли інструменти
«Кримської платформи» для реагування на всеросійський перепис на території
окупованого Криму і отримали швидку реакцію міжнародних партнерів, які також
засудили проведення перепису: дуже чітку позицію заявили США, Євросоюз,
Великобританія, Німеччина та інші країни. Таким чином, Україна і міжнародна
спільнота вважають російський перепис в Криму незаконним, а його результати —
юридично нікчемними. Ми також інформуємо міжнародні організації і продовжимо
тиск на Росію за її протиправні дії...»
%%%cit_comment
%%%cit_title
\citTitle{Чим небезпечний російський перепис населення в Криму}, 
Микола Семена, day.kyiv.ua, 18.11.2021
%%%endcit

%%%cit
%%%cit_head
%%%cit_pic
%%%cit_text
Вот это все – наверное и есть какое-то специфическое прохождение этого самого
«конца света». В том смысле, что в толкотне за выживание в «темные годы» будет
еще долго идти поиск спички, фонарика, костра, который бы хоть на время
«осветил». А значит, это еще надолго, и с пока еще ненаписанной утопией про
завтра.  К чему это я? О своей \emph{стране}, об Украине. \emph{Страны},
которую я знал еще лет 20 назад, и с которой связано было мое становление,
работа, устремления, уже нет.  И уже такой не будет. И дело не майданных
революциях, войнах и территориях.  Радикально изменился, трансформировался
«мир», которого хватило на полтора десятилетия, да и то, судя по дальнейшему,
инерционного. Последние годы страны лишь подтверждают мои ощущения, наблюдения
и предположения, что с тем старым украинским «миром» - все, конец.
Одновременно, рождается и какой-то новый «мир». Еще не родился, но – рождается.
Тот самый новый мир, в котором становится нормой и возможностью все то, что я
наблюдаю сейчас. И он, этот новый мир, для меня чужой.  Один мой товарищ (нет
его с нами, вечная память), с тонкой душой и какой-то гиперчувствительностью к
окружающему миру (а он относился к тем редким исключениям, которые были
способны «переживать настоящее» по-настоящему) заметил: холод космический
какой-то вокруг и внутри. Я тогда, в минуты произнесенного, не проникся, или
как сейчас говорят в обиходе, «не врубился». А пришло понимание позже.  Когда
сам почувствовал.  Холод у нас в \emph{стране} – космический
%%%cit_comment
%%%cit_title
\citTitle{Пришел конец старому теплому украинскому миру. Начинается что-то другое / Лента соцсетей / Страна}, 
Андрей Ермолаев, strana.news, 29.11.2021
%%%endcit
