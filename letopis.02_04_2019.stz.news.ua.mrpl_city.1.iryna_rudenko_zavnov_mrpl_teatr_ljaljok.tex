% vim: keymap=russian-jcukenwin
%%beginhead 
 
%%file 02_04_2019.stz.news.ua.mrpl_city.1.iryna_rudenko_zavnov_mrpl_teatr_ljaljok
%%parent 02_04_2019
 
%%url https://mrpl.city/blogs/view/irina-rudenkozasnovnitsya-mariupolskogo-teatru-lyalok-1
 
%%author_id demidko_olga.mariupol,news.ua.mrpl_city
%%date 
 
%%tags 
%%title Ірина Руденко - засновниця Маріупольського театру ляльок
 
%%endhead 
 
\subsection{Ірина Руденко - засновниця Маріупольського театру ляльок}
\label{sec:02_04_2019.stz.news.ua.mrpl_city.1.iryna_rudenko_zavnov_mrpl_teatr_ljaljok}
 
\Purl{https://mrpl.city/blogs/view/irina-rudenkozasnovnitsya-mariupolskogo-teatru-lyalok-1}
\ifcmt
 author_begin
   author_id demidko_olga.mariupol,news.ua.mrpl_city
 author_end
\fi

\ii{02_04_2019.stz.news.ua.mrpl_city.1.iryna_rudenko_zavnov_mrpl_teatr_ljaljok.pic.1}

Завдяки неймовірній наполегливості та вражаючій енергійності корінної
маріупольчанки \textbf{Ірини Анатоліївни Руденко} у Маріуполі було відкрито перший
ляльковий театр. Сьогодні маленькі маріупольці з легкістю можуть поринути у
загадковий і чарівний світ театрального мистецтва, але не всім відомо, скільки
було покладено зусиль на таку важливу і корисну для міста справу. Наша героїня
– заслужений діяч естрадного мистецтва України, художній керівник
Маріупольського театру ляльок \enquote{Бібо}, директорка обласного ТО
\enquote{Арлекіно} – вміло поєднує в собі необхідні професійні якості та рідкий
ентузіазм, без яких досягати поставленої мети набагато складніше. З нагоди
Всесвітнього дня театру, який відзначався 27 березня, пропоную вам, дорогі
читачі, відкрити для себе ще одну сильну і унікальну особистість.

\ii{insert.read_also.demidko.kozhevnikov}

Народилася Ірина Анатоліївна в Маріуполі. З дитинства була творчою і здібною
дитиною, яка відвідувала різні секції та навіть ставила п'єси на подвір'ї біля
свого будинку. Батьки в усьому підтримували маленьку Іру. Мама сама вигадувала
казки, допомагала шити костюми. Для Ірини мама залишається головним Вчителем і
справжнім педагогом, якій, не маючи педагогічної освіти, вдавалося з легкістю
навчати дітей. Батьки виховували Іру дуже мудро, вони привчали її до свободи в
усьому. Їй здавалося, що вона завжди робить вибір самостійно, але не помічала,
що за нею уважно спостерігає \enquote{приховане око} батьків, які в разі потреби
оточували юну дівчинку потрібними порадами і своєчасною допомогою.

\ii{02_04_2019.stz.news.ua.mrpl_city.1.iryna_rudenko_zavnov_mrpl_teatr_ljaljok.pic.2}

Після закінчення школи Ірина вступила до ПДТУ. Незабаром вийшла заміж за
однокласника, який був військовим. Оскільки чоловіка відправили служити у
Московський військовий округ, молоде подружжя поїхало до Москви. Саме там Ірина
стала студенткою університету мистецтв за спеціалізацією \enquote{Режисер масових
театральних видовищ}. Ще у студентські роки наша героїня часто відвідувала
ляльковий театр і дуже уважно спостерігала за процесом роботи.

\ii{insert.read_also.burov.istoria_staryj_korpus}

\ii{02_04_2019.stz.news.ua.mrpl_city.1.iryna_rudenko_zavnov_mrpl_teatr_ljaljok.pic.3}

Згодом Ірина Руденко приїхала до рідного Маріуполя. Почала працювати в Міському
парку культури та відпочинку культурним працівником на дитячому майданчику.
Тоді вона разом зі своєю колегою почали показувати лялькові вистави. У 1992
році Ірина Анатоліївна разом з групою ентузіастів створила перший в Маріуполі
ляльковий театр. Загалом в Ірини було багато однодумців, серед них \emph{Віктор
Грамматиков, Валентина Саврасова, Анжеліка Ряпалова, Лариса Колеснікова, Євген
Сосновський та інші}. Стараннями справжніх подвижників театральної справи у тому
ж 1992 році була поставлена перша вистава \enquote{Три веселих п'ятачка} про
братиків-поросят, улюблених і відомих у всьому світі. Ірина разом з колегами
майстрували ляльок, робили декорації, шили собі костюми для ігрових
розважальних програм, обійтися без яких було ніяк не можна – потрібні були
кошти для творчості. Труднощів, звичайно ж, було багато. Але, мабуть, найбільша
проблема після фінансової – відсутність власного приміщення. Театру доводилося
ставити вистави то в старій залі ігрових автоматів в міському парку, то в
кінотеатрах. Засновниця лялькового театру зізнається, що її кар'єрний шлях був
досить важкий, але вона не шкодувала ані сил, ані вільного часу, щоб ляльковий
театр у Маріуполі посів своє особливе місце в культурному житті Приазовського
краю. Водночас Ірина Анатоліївна свою роботу зуміла поєднати з отриманням ще
однієї вищої освіти. У 1994 році вона закінчила Бердянський державний
педагогічний університет.

\ii{02_04_2019.stz.news.ua.mrpl_city.1.iryna_rudenko_zavnov_mrpl_teatr_ljaljok.pic.4}

Завдяки наполегливості Ірини Анатоліївни у липні 1997 р. Марі\hyp{}упольський
ляльковий театр був зареєстрований міською радою та отримав статус професійного
муніципального театру, ставши таким чином першим офіційним дитячим театром
Приазов'я. Незабаром театр почав ставити вистави у будівлі колишнього
кінотеатру \enquote{Маріуполь}.

\textbf{Читайте також:} \emph{В Мариуполь возвращается \enquote{Отель Континенталь}}%
\footnote{В Мариуполь возвращается \enquote{Отель Континенталь}, Роман Катріч, mrpl.city, 01.04.2019, \par%
\url{https://mrpl.city/news/view/v-mariupol-vozvrashhaetsya-otel-kontinental}%
}

Репертуар театру складався з яскравих дитячих вистав: \enquote{Одного разу в
Африці} за мотивами казок Остера, \enquote{Три бажання}, \enquote{Операція
\enquote{Чебурашка}}, \enquote{Ай, да Мицик} Є.  Чеповецького,
\enquote{Маленький принц} за повістю Екзюпері, \enquote{Дюймовочка} за казкою
Андерсена, \enquote{Коза-Дереза}, \enquote{Білосніжка і сім гномів} тощо. Ірина
Руденко, маючи педагогічну і режисерську освіту, всіма силами створювала,
відроджувала і підтримувала життя унікального жанру театрального мистецтва. У
2005 р. театр ляльок почав нове життя для найменших маріупольців і отримав
назву \textbf{\enquote{Бібо}}. При ньому була створена школа акторської
майстерності.

\ii{02_04_2019.stz.news.ua.mrpl_city.1.iryna_rudenko_zavnov_mrpl_teatr_ljaljok.pic.5}

Тендітна і сонячна, але сильна характером Ірина виховала двох чудових синів –
\emph{Олексія та Романа}. Олексій – майстер спорту України з ралі, а Роман –
здібний актор лялькового театру, член Національної спілки журналістів України,
успішний київський телеведучий. Три роки назад в Олексія народився син, онук
став справжнім джерелом натхнення і радості для Ірини Анатоліївни. Дуже
розвинений маленький \emph{Кирюша} відвідує всі вистави лялькового театру, він вже
давно став найголовнішим суддею і найбільш вибагливим глядачем для художнього
керівника театру.

\ii{02_04_2019.stz.news.ua.mrpl_city.1.iryna_rudenko_zavnov_mrpl_teatr_ljaljok.pic.6}

\textbf{Читайте також:} \emph{Цена жизни маленьких мариупольцев}%
\footnote{Цена жизни маленьких мариупольцев, Елена Калайтан, mrpl.city, 30.03.2019, \par%
\url{https://mrpl.city/blogs/view/tsena-zhizni-malenkih-mariupoltsev} }

Ірина дуже любить рідний Маріуполь і підкреслює, що маріупольці відрізняються
від інших особливим менталітетом і самобутністю. Найбільше полюбляє Міський
сад, який зберігає свою неповторну атмосферу.

\ii{02_04_2019.stz.news.ua.mrpl_city.1.iryna_rudenko_zavnov_mrpl_teatr_ljaljok.pic.7}

Зараз Ірина Анатоліївна посідає відповідальний пост голови правління творчого
об'єднання \enquote{Арлекіно}, до складу якого входить Театр ляльок, Перша театральна
школа, молодіжний театр \enquote{Персонаж} та інші відділення. У школі театрального
керівника виховується близько 100 дітей. Всі вони різного віку, але єдині у
прагненні навчатись та самовдосконалюватись. Перша театральна школа-студія
міста Маріуполь неодноразово ставала лауреатом всеукраїнських, європейських і
міжнародних фестивалів-конкурсів. Ірина Анатоліївна дуже любить своїх учнів і
щиро хоче, щоб вони розвивали власні таланти та не зупинялися на досягнутому.
Вона вважає, що кожна дитина має свій винятковий талант, просто потрібен час,
щоб його розкрити. Завдяки вірі у своїх учнів та постійній праці з ними вона
змогла виховати справжніх акторів, які сьогодні стали справжніми професіоналами
і прославляють Маріуполь та свого викладача в інших містах України.

\ii{02_04_2019.stz.news.ua.mrpl_city.1.iryna_rudenko_zavnov_mrpl_teatr_ljaljok.pic.8}

\textbf{Читайте також:} \emph{Співуча й неповторна: українські слова, які не перекласти російською}%
\footnote{Співуча й неповторна: українські слова, які не перекласти російською, Кіра Булгакова, mrpl.city, 01.04.2019, \par%
\url{https://mrpl.city/news/view/spivucha-j-nepovtorna-ukrainski-slova-yaki-ne-pereklasti-rosijskoyu}
}

\textbf{Улюблена книга Ірини Руденко:} \enquote{Маленький принц} Антуана де Сент-Екзюпері.

\textbf{Улюблені фільми:} класика радянського кіно.

\textbf{Курйозний випадок з життя:} веселих і незабутніх випадків у житті
бадьорої і життєрадісної Ірини Анатоліївни було дуже багато. Зокрема, якось їй
довелося грати Шапокляк. Ніс у неї був на гумці, і, коли вона чекала свого
виходу, щоб легше було дихати, ніс натягувала на лоб. Але, поспішаючи на сцену,
Ірина забула поставити ніс на місце. Вийшовши у новому образі, адже тепер у
Шапокляк виріс справжній ріг, вона дуже здивувалася, що раптом її вихід
викликав такий сміх і неабияке пожвавлення в залі. Про себе подумала, як же
гарно граємо, дітям так весело, ще й Чебурашка щось очима показує їй, мабуть,
сам не вірить в такий успіх.

\ii{02_04_2019.stz.news.ua.mrpl_city.1.iryna_rudenko_zavnov_mrpl_teatr_ljaljok.pic.9}

\textbf{Порада маріупольцям:} 
\begin{quote}
\em\enquote{Хочеться побажати не бути байдужими, навчитися
об'єднуватися заради однієї важливої мети – змінювати місто на краще. Жити за
принципом: \enquote{Де народився там і згодився!}. Багато маріупольців вважають своє
місто трампліном, з якого все починається, але для нашого міста краще, щоб ми
залишалися і будували майбутнє саме тут і зараз!}
\end{quote}

\textbf{Читайте також:} \emph{ЭКСКЛЮЗИВ: Все, что нужно знать о ГогольFest: голова Гоголя, спецпоезд, спектакль о буллинге и 150 проектов за 6 дней}%
\footnote{ЭКСКЛЮЗИВ: Все, что нужно знать о ГогольFest: голова Гоголя, спецпоезд, спектакль о буллинге и 150 проектов за 6 дней, mrpl.city.tilda.ws/gogolfest}
