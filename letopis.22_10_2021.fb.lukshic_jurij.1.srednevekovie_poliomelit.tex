% vim: keymap=russian-jcukenwin
%%beginhead 
 
%%file 22_10_2021.fb.lukshic_jurij.1.srednevekovie_poliomelit
%%parent 22_10_2021
 
%%url https://www.facebook.com/permalink.php?story_fbid=1114522399080387&id=100015679115504
 
%%author_id lukshic_jurij
%%date 
 
%%tags bolezn,civilizacia,medicina,poliomelit,srednevekovie,strana,ukraina
%%title В Украине, похоже, наступает настоящее Средневековье - Вспышка Полиомелита
 
%%endhead 
 
\subsection{В Украине, похоже, наступает настоящее Средневековье - Вспышка Полиомелита}
\label{sec:22_10_2021.fb.lukshic_jurij.1.srednevekovie_poliomelit}
 
\Purl{https://www.facebook.com/permalink.php?story_fbid=1114522399080387&id=100015679115504}
\ifcmt
 author_begin
   author_id lukshic_jurij
 author_end
\fi

В Украине, похоже, наступает настоящее Средневековье, а сама страна всё дальше
и дальше от цивилизации. Вопиющий случай произошёл в Ровенской области, на
территории одной из ОТГ. Там зафиксировали вспышку полиомиелита. Болезнь
обнаружили у полуторагодовалого ребёнка, затем у шести его братьев и сестёр.
Малыша парализовало, медики стали проводить профилактические мероприятия и
приглашать местных делать прививки детям.

\ifcmt
  pic https://scontent-frt3-1.xx.fbcdn.net/v/t1.6435-9/248177412_1114522325747061_2122080867082064210_n.jpg?_nc_cat=106&ccb=1-5&_nc_sid=8bfeb9&_nc_ohc=dHb39_YGBakAX8m4hW5&_nc_ht=scontent-frt3-1.xx&oh=1a173e2ab324ab74885fb18a4580a9f5&oe=619A7C3F
  @width 0.8
\fi

Знаете, как их там встретили?? Это просто шок. Местные жители закрывали ворота
перед медиками, отпускали во двор собак, убегали огородами от них. В начале
октября там подтвердили случай паралича у 1,5-летнего ребёнка. Родители
сознательно отказывались от прививок из-за религиозных убеждений.

Что же будет через 10 лет?? Будут на костры отправлять медиков? Лечить
молитвами? Отказываться от кипячения воды, прочих основ гигиены? Такое
ощущение, что цивилизация скатывается в новые Средние века. Полиомиелит
считается опасным инфекционным заболеванием, вызываемым полиовирусом.

Заболевание можно предотвратить с помощью вакцины, однако для её эффективности
требуется несколько доз. Вакцинация победила эпидемию полиомиелита в 20
столетии, и количество диагностированных случаев сократилось на 99,9\%, с
примерно 350 000 случаев в 1988 г. до 42 случаев в 2016-м.  

На фото - зарегистрированные случаи полиомиелита в 2019 году.

\ii{22_10_2021.fb.lukshic_jurij.1.srednevekovie_poliomelit.cmt}
