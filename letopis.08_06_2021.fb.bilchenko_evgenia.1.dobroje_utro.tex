% vim: keymap=russian-jcukenwin
%%beginhead 
 
%%file 08_06_2021.fb.bilchenko_evgenia.1.dobroje_utro
%%parent 08_06_2021
 
%%url https://www.facebook.com/yevzhik/posts/3963060330395714
 
%%author Бильченко, Евгения
%%author_id bilchenko_evgenia
%%author_url 
 
%%tags 
%%title БЖ. Доброе утро!
 
%%endhead 
 
\subsection{БЖ. Доброе утро!}
\label{sec:08_06_2021.fb.bilchenko_evgenia.1.dobroje_utro}
\Purl{https://www.facebook.com/yevzhik/posts/3963060330395714}
\ifcmt
 author_begin
   author_id bilchenko_evgenia
 author_end
\fi

БЖ. Доброе утро!

Доброе... У меня был очень странный день. Я пошла в НИИ, собрала нервы в кулак
и в который раз правила гранки новой книги \enquote{Язык ненависти. Язык толерантности.
Язык любви}. Тендер она выиграла. Но я не знаю, сколько будет авторских. Я не
писала, чтобы не ныть, но, только не ругайте меня, у меня так долго, двадцать
дней, опять держатся горло, нос, сонливость вперемежку с бессонницей и
субфибрилка с такой ломотой в костях, что я ещё подобного не испытывала. Только
не ругайтесь, но 12 дней антибиотиков реально помогли, бросила - все по новой:
это - точь-в-точь, что было со мной зимой, когда это длилось полгода, и все с
ног сбились, пока инфекционист не проколол гормоны (стало мягко, но легче), а
лор, которого мне нашли, назначил операцию в 30 тысяч (от шока я поправилась).
Вот вы не поверите: после ковида мужа пульс 120 и сатурация до 90 меня уже не
пугают, а к врачам с тем, что в народе зовут \enquote{пролонгированный ковид}, -
обращаться нет смысла. Хорошая сатурация (96 \%) - у меня только в научных
работах, и наци это ужасно бесит.

Я жду свою родню на квартирнике 12 июня в 17.30 у метро Берестейская. Подробности в личку. Но я изменилась: 

\begin{itemize}
\item А. Спиртного нет. Я всё. Я могу выпить лишь с близким товарищем 100 боевых раз в полгода. Б. Мата нет, тусы с воплями - тоже. 
\item В. Я очень жду не только литераторов, но тех людей, что были со мной
после 18 января, как в окопе. Священников, журналистов, слесарей,
врачей, инженеров. 

\item Г. Я - другая. Вы к этому привыкнете. 
\item Д. Я допускаю, что следующий квартирник будет очень нескоро, если вообще будет, официальных сцен мне здесь - нельзя (и я сама не хочу: ни одних подмостков от сего режима, ни одной награды). 
\end{itemize}

\ifcmt
  pic https://scontent-lga3-2.xx.fbcdn.net/v/t1.6435-0/p526x296/198094883_3963060240395723_961902534587942864_n.jpg?_nc_cat=104&ccb=1-3&_nc_sid=8bfeb9&_nc_ohc=br_6hCAOZMcAX_Yma-F&_nc_ht=scontent-lga3-2.xx&tp=6&oh=89bc5b744dcf805c11671825e995c170&oe=60E48EE5
\fi

Есть моя \enquote{Пьета}. Очень нужно, чтобы кто-то держал мой телефон со стримом: в
других странах меня смотрят сотни, людей надо уважать. И, если Господь даст,
единственное, чем мы можем здесь когда-нибудь лупануть, - открою тайну - это
жёстким дуэтом с Максим Кабир. Но не факт, что мы выживем после него. Потому,
может, после Победы. Или это сделают наши внуки: с листа нас друг другу
почитают.  Это был пост благодарности Богу за всё. Вы очень хорошие, простые
люди. Не в смысле: \enquote{простые}, а в смысле - не из научной и  литературной
богемы. Я ее боюсь.
