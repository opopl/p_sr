% vim: keymap=russian-jcukenwin
%%beginhead 
 
%%file 14_04_2021.fb.respublikalnr.4.vojna_nachalo_marochko_mnenie
%%parent 14_04_2021
 
%%url https://www.facebook.com/groups/respublikalnr/permalink/796329054336221/
 
%%author 
%%author_id 
%%author_url 
 
%%tags 
%%title 
 
%%endhead 

\subsection{МНЕНИЕ. АНДРЕЙ МАРОЧКО: «Не было никаких правовых оснований для начала войны в Донбассе»  }
\label{sec:14_04_2021.fb.respublikalnr.4.vojna_nachalo_marochko_mnenie}
\Purl{https://www.facebook.com/groups/respublikalnr/permalink/796329054336221/}

\ifcmt
  pic https://scontent-bos3-1.xx.fbcdn.net/v/t1.6435-9/174202426_122253349954773_923831221627406719_n.jpg?_nc_cat=102&ccb=1-3&_nc_sid=825194&_nc_ohc=SJfH-adkAmIAX-eJIWm&_nc_ht=scontent-bos3-1.xx&oh=a9f452c4a6bb44ebfaa948cec1074bbb&oe=609AB104
\fi

Киевский режим, возникший на Украине по результатам государственного
переворота, начал войну против Донбасса, чтобы укрепить свою власть. Об этом в
эксклюзивном комментарии корреспонденту телеканала «Луганск 24» в День жертв
украинской агрессии рассказал военный эксперт, общественный деятель,
подполковник запаса Народной милиции ЛНР Андрей Марочко.

«После государственного переворота новым киевским властям нужно было усидеть в
своих креслах и, так сказать, закрепить успех. То, что не было никаких правовых
оснований для начала войны в Донбассе, говорит как конституция Украины, так и
устав ВСУ. Он не позволял использовать армию против собственного населения. Но
Турчинов со своими подельниками, которые участвовали в государственном
перевороте, совершил военное преступление и подписал злополучный закон, ввёл
решение СНБО в действие, что послужило началом так называемой
«антитеррористической операции»», – напомнил Андрей Марочко.

Военный эксперт подчеркнул, что причиной начала вооруженной украинской агрессии
послужили русская идентичность Донбасса, память о подвигах предков, а также
наше желание дружить с Россией.

«Безусловно, это была одна из основных причин, поскольку та риторика и
политика, которую марионетки Запада воплощали в жизнь на Украине, подразумевала
обязательную ликвидацию неугодных. Предстоящие после переворота выборы свели бы
на нет все «достижения майдана», поскольку Юго-Восток всегда голосовал за
русскую идентичность, за русский язык и за память о своих предках», – добавил
общественный деятель.

По его мнению, события 2014 года продемонстрировали единство жителей Донбасса.

«Есть такие популярные строки: «Донбасс никто не ставил на колени и никому
поставить не дано». Это, наверное, самое яркое выражение, которое характеризует
всех жителей Донбасса. Это, прежде всего, народ крепкий, сплочённый, который
привык не языком болтать, а доказывать всё на практике и своими действиями», –
подчеркнул Андрей Марочко.

Военный эксперт отметил, что за Донбассом – правда, которая позволила ему
выстоять.

«Мы никуда не наступали, не уходии, мы находимся на своей земле. Мы защищали
свои семьи, свои дома, свою землю, а Украина пришла убивать. Нам было некуда
отступать, потому что позади – наши семьи и дома», – напомнил Андрей Марочко.

Он также оценил успехи в восстановлении Республики после завершения активной
фазы боевых действий, заверив, что «несмотря на все тщетные старания и Запада,
и Киева Республика хоть и малыми шагами, но развивается».

«Есть своя законодательная база, запускаются предприятия, люди работают. Не всё
так гладко, как хотелось бы, но можно учесть, что мы живём в условиях войны,
которая тянет и финансовые средства, и человеческие ресурсы. Но вместе с тем мы
восстанавливаемся во время войны», – сказал Андрей Марочко.

