% vim: keymap=russian-jcukenwin
%%beginhead 
 
%%file 23_12_2020.fb.bilchenko_evgenia.1.nastuplenie_2_0
%%parent 23_12_2020
 
%%url https://www.facebook.com/yevzhik/posts/3501048369930248
 
%%author Бильченко, Евгения
%%author_id bilchenko_evgenia
%%author_url 
 
%%tags bilchenko_evgenia
%%title БЖ. Наступление 2.0
 
%%endhead 
 
\subsection{БЖ. Наступление 2.0}
\label{sec:23_12_2020.fb.bilchenko_evgenia.1.nastuplenie_2_0}
\Purl{https://www.facebook.com/yevzhik/posts/3501048369930248}
\ifcmt
 author_begin
   author_id bilchenko_evgenia
 author_end
\fi

БЖ. Наступление 2.0
Значит, что? Значит, умирай, но всё же - вертись ужом?
Ряди истину в маску лжи (рука с букетом, карман с ножом)?
Ты говоришь, "нападения время пришло" и оно пришло
В виде кривого танца, где ноги режутся о стекло?
Значит, что? Значит, подыхай, боец, но всё равно ползи?
И будь доволен любой подмогой, сидя в своей грязи?
Такая работа, да? Такая работа, да?
Каждый радист обладает большим, чем шифры и провода.
Каждый охотник желает знать, где фазаны сидят.
Война и охота имеют сроки: мир не имеет дат.
И эта война за правду не закончится никогда...
Говоришь, "такая работа", да? 
- Такая работа, да.
23 декабря 2020 г.

\ifcmt
  pic https://scontent-lga3-2.xx.fbcdn.net/v/t1.6435-9/132649796_3501050256596726_4572971398280292551_n.jpg?_nc_cat=103&ccb=1-3&_nc_sid=8bfeb9&_nc_ohc=mxhvM8d3buwAX-JAWgq&_nc_ht=scontent-lga3-2.xx&oh=e35fd743daf4a1b86e807c6f8ff05cce&oe=60CA8EFF
	caption БЖ. Наступление 2.0, Илл.: кадр из к/ф \enquote{Джокер}.
\fi

\emph{Лідія Свачій}

Красивий ритм вірша.. Хто б ще пояснив таким особливо тупим, як я, що вірш цей
означає... Двічі прочитала, не зрозуміла

\emph{Евгения Бильченко}

Лідія Свачій Ритм називається \enquote{класична тоніка}: найбільш ранній ствродавній
ритм, яким написані твори Конфуція та \enquote{Отче наш}. Тоніка за часом передує
золотій ері Байрона та Пушкіна і наслідує їх у срібній ері Маяковського. Тоніка
- це джаз. Вважається вишим виявом майстерності. Я лише на стадії учня. Спроби
створити тоніку - завідомо провальні, якщо автор не володіє хореєм та ямбом.
Мої класичні ритми можна побачити на poezia. org й будь-де в гуглі за
прізвищем. Я не відслідковую своєї пуболічності. Щодо змісту... Автор не
коментує контент. Кожен розуміє по-своєму, у кожного поета свря аудиторія
приходить на концерти. Це природно).

\emph{Лідія Свачій}

Неймовірно красива, наспівна ця класична тоніка ) Нею можна описувати зоряне небо. Дякую!

\emph{Евгения Бильченко}

\textbf{Лідія Свачій} Зоряне небо Канта))

\emph{Евгения Бильченко}

Астрономія слова

\emph{Сергей Никонов}

Стих написан потому, что Евгения Бильченко жила, живет и , дай Бог, долго будет
жить осмысленной жизнью, стремясь к цели. Я считаю, что он о тяжелой борьбе
раненого или измученного человека. С той же идеей снят российский фильм
Подольские курсанты. Он о схватке дивизии из молодых ребят с нацистскими
войсками на подступах к Москве. Приглядитесь к его персонажам и учитесь как
любить, жить. Они победили в тяжелейших боях и страданиях от ран и душевной
боли. Из трех тысяч выжили 500, но удержали позиции.

\emph{Евгения Бильченко}

Сергей Никонов Да, о тяжелой борьбе раненого человека в условиях оккупации. Я
не сторонник парадигмы фильма \enquote{Джокер}: просто гениальная игра актера вывела
картину из постмодерной истерии в нечто большее. В скоморошество и мессианство,
в трагедию клоуна у трона диктатора, коим предстает мир денег и цензуры.

\emph{Сергей Никонов}

Я не смотрел Джокер, я смотрел Подольские курсанты.

\emph{Евгения Бильченко}

Сергей Никонов Хороший выбор.

\emph{Тереза Славович}

Похожи как две песчинки,
и ртутные капли
Умыться и сдаться с поличным
в еще одно утро
Мои минотавры выходят из ванной в солдаты
Пугая прохожих задумчивым взглядом...
Полундра!!!
Нагая с наганом придет Навсикая к Орфею
крутись извивайся ужом
Умирай, но не падай
Священник служил панихиду по истинным змеям
что даже в гробу так свежи и полны желчным ядом
Ведь все демагогия- клятвы и обещания
довольны подмогой и теми,что срама не ищут
философы принимающие подаяние
Исключительно
в виде духовной Пищи!
А толпа восхищенно глазела...молилась и каялась
плоть свою неистово теребя
Люди по разным причинам
Приходят к богу
Это легче, чем
приходить в себя

Евгения Бильченко

Тереза Славович ... Ибо мы были, и были вместе, и были преступно счастливы.  И
скрипела под нами морской пивнухой земля дощатая.  Язычник в спортивках и
камуфле, православный с исконной верой, Девочка-эзотерик, обольщавшая офицера.

\emph{Евгения Бильченко}

Какой потрясающе солидарный и цивилизационно многообразный интернационал в
ветке комментариев. Радуюсь.

\emph{Юлия Балабей}

Очень красиво и оптимистичненько
