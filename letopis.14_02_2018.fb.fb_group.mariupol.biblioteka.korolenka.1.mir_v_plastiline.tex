%%beginhead 
 
%%file 14_02_2018.fb.fb_group.mariupol.biblioteka.korolenka.1.mir_v_plastiline
%%parent 14_02_2018
 
%%url https://www.facebook.com/groups/1476321979131170/posts/1585860314844002
 
%%author_id fb_group.mariupol.biblioteka.korolenka,kibkalo_natalia.mariupol.biblioteka.korolenko
%%date 14_02_2018
 
%%tags mariupol,plastilin,kultura
%%title Мир в пластилине
 
%%endhead 

\subsection{Мир в пластилине}
\label{sec:14_02_2018.fb.fb_group.mariupol.biblioteka.korolenka.1.mir_v_plastiline}
 
\Purl{https://www.facebook.com/groups/1476321979131170/posts/1585860314844002}
\ifcmt
 author_begin
   author_id fb_group.mariupol.biblioteka.korolenka,kibkalo_natalia.mariupol.biblioteka.korolenko
 author_end
\fi

Мир в пластилине

Многие мариупольцы даже не подозревают, что пластилином можно рисовать, как
красками! С работами, выполненными в технике «пластилинографии» - относительно
новой, нетрадиционной художественной технике - можно познакомиться в Мобильной
галерее «Мир увлечений» Центральной библиотеки им. В.Г. Короленко. 17 февраля в
1500 в галерее открывается выставка «Мир в пластилине». Ее инициатор – аматор
Жанна Юрковская, более известная в городе как «Тетушка Жаннет». Она
предоставила для экспозиции 90 творческих работ: собственные, а также работы
учеников – детей и взрослых.

Разнообразие пластилино-творческих идей, представленных на выставке, можно
условно разбить на тематические серии: «Цветочная феерия», «Братья наши
меньшие», «Пернатые друзья», «Петриковская роспись», «Садок вишневий коло
хаты». Большой интерес, особенно в преддверии 240-летнего юбилея города,
вызывают краеведческие серии «Старый Мариуполь» и «Азовское море». 

«Пластилиновые» картины – настоящее «energy»! Они – искрометные, эмоциональные,
насыщенно-яркие, креативные и в то же время излучают  доброту, мягкость, дарят
зрителю спокойствие, гармонию, Замечательно, что у каждого автора – свои
индивидуальные приемы нанесения пластилина, свой почерк.

Мобильная галерея приглашает всех любителей прекрасного! Увлекательная выставка
«Мир в пластилине» будет открыта до 4 марта и непременно принесет большое
эстетическое удовольствие своим посетителям!

Вход свободный: с 1030 до 1900, кроме понедельника, вторника и санитарного дня
(22.02.2018).
