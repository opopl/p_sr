% vim: keymap=russian-jcukenwin
%%beginhead 
 
%%file 29_11_2020.fb.volga_vasilii.1.upa_mihail_pogrebinskii
%%parent 29_11_2020
 
%%url https://www.facebook.com/Vasiliy.volga/posts/2765371290446952
 
%%author Волга, Василий Александрович
%%author_id volga_vasilii
%%author_url 
 
%%tags 
%%title УПА И МИХАИЛ ПОГРЕБИНСКИЙ
 
%%endhead 
 
\subsection{УПА и Михаил Погребинский}
\label{sec:29_11_2020.fb.volga_vasilii.1.upa_mihail_pogrebinskii}
\Purl{https://www.facebook.com/Vasiliy.volga/posts/2765371290446952}
\ifcmt
	author_begin
   author_id volga_vasilii
	author_end
\fi

Очень немного осталось на Украине ученых гуманитариев, чье мнение для меня так
же авторитетно, как мнение Михаила Борисовича Погребинского. 

Таких людей я могу перечесть по пальцам одной руки --- это Кость Бондаренко,
Евгений Копатько, Петр Толочко, Дмитрий Выдрин и Михаил Борисович Погребинский. 

Я дорожу дружбой с этими людьми и восхищаюсь ими. Для меня важен не только их
высокий интеллект и энциклопедические знания, для меня так же важно то, что они
самих себя не продали изменившимся обстоятельствам и не поддались давлению
новых хозяев новой украинской действительности. Не испугались их.

Их пытались купить, запугивали, лишали возможности работать. Двоим из них
пришлось оставить Украину, когда их начали преследовать по сумасбродным
уголовным обвинениям. 

Эти люди --- ученые с большой буквы «У», потому что нет ничего более важного в
гуманитарной науке, чем честность ученого, его независимость, объективность.
Но именно объективность иногда может сыграть с нами злую шутку, и особенно это
возможно, когда общество входит в режим внутреннего противостояния. 

Михаил Борисович высказался в одном из интервью о том, что российская
пропаганда очень однобоко оценивает деятельность УПА во время Великой
Отечественной Войны. Мол, УПА --- это не только служение Гитлеру и не только
зверства против украинцев, поляков, евреев, русских, а это еще и отказ служить
Гитлеру и военные действия против немецких оккупантов. 

Также высказался Михаил Борисович в отношении того, что мы, те кто не принимает
сегодня бандеризации Украины, обязаны найти хоть какой-то способ
взаимопонимания с теми миллионами граждан Украины, которые считают УПА героями,
боровшимися за независимость Украины. Иначе де, мы не сможем мирно
сосуществовать с ними в одной стране. 

И первое утверждение Михаила Борисовича объективно, и второе и третье, но тут
же поднялся против него вал возмущенных комментариев от тех людей, которые
прям-таки с болью восприняли эту его позицию. 

Друзья, но ведь Михаил Борисович прав!

Мы и в самом деле не сможем жить в одной стране с теми, для кого Бандера герой
и для кого УПА --- освободители. 

И ведь не мы в этом виноваты, а они - эти новые наци, которые в день рождения
Бандеры бродят под стилизованной свастикой и факелами по улицам наших городов,
угрожая зарезать ножами всех тех, кто считает себя русским. 

Это не мы, а они гадят на наши памятники. Это не мы пришли к ним в дом. Это они
приперлись в наши города и уродуют души наших детей своей бандеровской лагерной
культурой. Это нам угрожает заместитель Секретаря СНБО тем, что за любовь к
Пушкину или Толстому с нами расправятся, как с пятой колонной. Это они свой
сельский менталитет привнесли в систему государственного управления, уничтожив
пять отраслей промышленности - основу существования городов, и выгнав десять
миллионов человек на заработки. Это не мы хотим их перевоспитать --- это они
хотят, чтобы мы стали ими. А когда мы этого делать не хотим, они нас убивают,
как Олеся Бузину, сжигают, как одесситов, отправляют танки, как на Донбасс,
сажают по тюрьмам или в лучшем случае дают время бежать, предупреждая, что
«вешать будут потом». 

\ifcmt
pic https://scontent.fiev6-1.fna.fbcdn.net/v/t1.0-9/128684171_2765410113776403_4564522145105986631_n.jpg?_nc_cat=101&ccb=2&_nc_sid=730e14&_nc_ohc=RMR3LmS-Y9kAX-EgDz-&_nc_ht=scontent.fiev6-1.fna&oh=4ae8a23388f9b463990c5a223059f46f&oe=5FEA3517
fig_env wrapfigure
width 0.5
\fi

Михаил Борисович, будьте и далее объективны --- совместно жить с ними уже не
получится. Это не наша вина - это их условия. 

С большим к Вам уважением, Василий Волга.
