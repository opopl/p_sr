% vim: keymap=russian-jcukenwin
%%beginhead 
 
%%file 06_11_2021.fb.chugajenko_tatjana.kiev.1.den_osvobozhdenia_kiev.cmt
%%parent 06_11_2021.fb.chugajenko_tatjana.kiev.1.den_osvobozhdenia_kiev
 
%%url 
 
%%author_id 
%%date 
 
%%tags 
%%title 
 
%%endhead 
\subsubsection{Коментарі}

\begin{itemize} % {
\iusr{Инна Темирова}
Внучка легендарного танкиста! С днём освобождения Киева от нацистов!

\begin{itemize} % {
\iusr{Татьяна Чугаенко}
\textbf{Инна Темирова} , спасибо, Инночка! И тебя с праздником!

\iusr{Инна Темирова}
\textbf{Татьяна Чугаенко} Спасибо, Танюша!
\end{itemize} % }

\iusr{Евгений Минченко}
От молодых киевлян.

\ifcmt
  ig https://scontent-frx5-1.xx.fbcdn.net/v/t39.30808-6/254608566_5146438998718744_1448866364085048240_n.jpg?_nc_cat=111&ccb=1-5&_nc_sid=dbeb18&_nc_ohc=GQzuLik7xo0AX9araCd&_nc_ht=scontent-frx5-1.xx&oh=f65643a11638a353ba098b929c0f909b&oe=61A2AEAC
  @width 0.25
\fi

\iusr{Valentina Prokopenkova}
С праздником!

\iusr{Наталья Александровна}

За форсирование Днепра бабушка получила орден "Красной звезды". Она была
связисткой. Наши были уже на том берегу, а связи не было. Соорудили плот и
выбрали самую мелкую. Бабина с кабелем весила наверно больше, чем она. Вся эта
история есть у меня в майских публикациях. К 30 летию Победы статья вышла в
журнале Чаян


\iusr{Татьяна Наумова}

Когда переехала в Киев, удивлялась, почему здесь не отмечают день освобождения.
В Одессе 10 апреля - значимый праздник, о котором знала с раннего детства. В
Киеве выяснилось, что праздником этот день фактически считать не получается.
Для того, чтобы город был освобождён к прздничным датам (а именно к 7-8 ноября)
было положено огромное количество человеческих жизней, чего можно было
избежать, освободив город на несколько дней позже  @igg{fbicon.frown} 

\begin{itemize} % {
\iusr{Виктор Виктор}
\textbf{Татьяна Наумова} не слушайте резунов-бебиков! а самое главное - не повторяйте за ними всякую чушь
не было задачи освободить к празднику.
а каждый час промедления давал фашистам время на перегруппировку и усиление обороны. Что, как следствие, вело только к росту наших потерь

\iusr{Аркадий Шухман}
\textbf{Татьяна Наумова} А подождали бы ещё пару лет и союзники открыли бы третий фронт и освободили бы нам наш Киев.
\end{itemize} % }

\iusr{Алла Жарова}
С Днём освобождения Киева! @igg{fbicon.heart.suit}

\iusr{Маркиса Лихтман}
Вечная память и благодарность тем, кто освободил мой город от фашистов!

\iusr{Елена Седова}

Мои родители освобождали Киев в составе 317 артелеристского
полка. награжденыорденами за освобождение. Всегда со своими однопочанами
встречались в этот день в школе 90 и возлагали цветы к Памятнику Славы. Сегодня
мои сын и дочь со своими уже взрослыми детьми несут цветы воинам-
освободителям. Мы помним и любим вас, Герои! Вечная память и низкий поклон

\iusr{Виктор Виктор}
ну здесь хоть имя более менее приличное дали...
деду - земной поклон и вечная слава!

\begin{itemize} % {
\iusr{Елена Седова}
\textbf{Виктор Виктор} Виктор! Я реальный человек и родители мне имя дали. Тут выдумывать нечего
\end{itemize} % }

\iusr{Ирина Грицай}
С праздником!

\iusr{Александр Кайманн}
а как сейчас Танковая называется?

\begin{itemize} % {
\iusr{Виктор Виктор}
\textbf{Александр Кайманн} Авиаконструктора Игоря Сикорского
но это хоть человек достойный

\iusr{Александр Кайманн}
\textbf{Виктор Виктор} согласен.. Хоть не Петлюры
\end{itemize} % }

\iusr{Аркадий Шухман}

Слава героям, уничтожившим нацизм, освобождавших Киев и другие города от зверей
двуногих! Вечная память погибшим! Никто не забыт, ничто не забыто!


\iusr{Вавилова Елена}
Спасибо деду за Победу! С праздником, Танюш!

\iusr{Анатолий Чурсин}
А по российскому (второму) каналу звучат красивые мелодичные украинские пеcни. С Праздником тебя, славный град Киев!!!

\iusr{Анатолий Чурсин}
И тебя, Таня и всех твоих многочисленных друзей с Праздником, с Днём освобождения Киева!!!!

\iusr{Дима Жуков}
Киев - город герой, всех с праздником

\iusr{Нина Свечкарева}
Присоединяюсь!

\iusr{Валентина Мезенцева}
С Праздником!!!

\iusr{Ян Таксюр}

Танюша, прости старика, запутался в твоих аккаунтах и не знаю, где поздравлять.
Словом, с Днём рождения тебя сердечно поздравляю! Желаю здравия духовного и
телесного, мира и благополучия семье, успеха во всех добрых делах и начинаниях.


\iusr{Grigori Rybak}
С Днём Рождения! @igg{fbicon.rose} 

\iusr{Мила Дени}

Поздравляю с днём рождения, Татьяна! Желаю крепкого здоровья, успешных
реализаций новых проектов, новых ресурсов, интереснейших впечатлений и
событий, новых статусов, вдохновения в каждый миг! Любви, нежности, заботы,
внимания, терпения в семье, верных друзей, которые всегда поддержат и
порадуются Вашим успехам!

\end{itemize} % }
