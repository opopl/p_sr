% vim: keymap=russian-jcukenwin
%%beginhead 
 
%%file 04_09_2021.fb.karevin_aleksandr.1.kuleba_moskovia.cmt
%%parent 04_09_2021.fb.karevin_aleksandr.1.kuleba_moskovia
 
%%url 
 
%%author_id 
%%date 
 
%%tags 
%%title 
 
%%endhead 
\subsubsection{Коментарі}
\label{sec:04_09_2021.fb.karevin_aleksandr.1.kuleba_moskovia.cmt}

\begin{itemize}
%%%fbauth
%%%fbauth_name
\iusr{Dmitriy Vysotskyi}
%%%fbauth_url
%%%fbauth_place
%%%fbauth_id
%%%fbauth_front
%%%fbauth_desc
%%%fbauth_www
%%%fbauth_pic
%%%fbauth_pic portrait
%%%fbauth_pic background
%%%fbauth_pic other
%%%fbauth_tags
%%%fbauth_pubs
%%%endfbauth
 

Как были обозначены эти земли на иностранных картах, по моему, не столь уж
важно, по моему, гораздо важнее другое, употреблялся ли когда-либо термин
"Московия" в качестве официального названия Великого княжества Московского или
иного государственного образования с центром в Москве и
существовал/использовался ли в тот период термин "Украина" в качестве
официального названия какого-либо государственного образования на территории
современной Украины? Вопрос к автору.

\begin{itemize}
%%%fbauth
%%%fbauth_name
\iusr{Александр Каревин}
%%%fbauth_url
%%%fbauth_place
%%%fbauth_id
%%%fbauth_front
%%%fbauth_desc
%%%fbauth_www
%%%fbauth_pic
%%%fbauth_pic portrait
%%%fbauth_pic background
%%%fbauth_pic other
%%%fbauth_tags
%%%fbauth_pubs
%%%endfbauth
 
\textbf{Dmitriy Vysotskyi} Термин "Московия" в качестве официального названия на Руси не употреблялся. С Украиной - аналогично..

%%%fbauth
%%%fbauth_name
\iusr{Dmitriy Vysotskyi}
%%%fbauth_url
%%%fbauth_place
%%%fbauth_id
%%%fbauth_front
%%%fbauth_desc
%%%fbauth_www
%%%fbauth_pic
%%%fbauth_pic portrait
%%%fbauth_pic background
%%%fbauth_pic other
%%%fbauth_tags
%%%fbauth_pubs
%%%endfbauth
 
\textbf{Александр Каревин} Вот и я о чем. По моему, именно на этом и нужно делать акцент в дискуссиях такого рода, особо обращая внимание на тот факт, что люди, проживавшие на землях к востоку от Киева, всегда считали себя частью единого русского народа.

%%%fbauth
%%%fbauth_name
\iusr{Пётр Гринёв}
%%%fbauth_url
%%%fbauth_place
%%%fbauth_id
%%%fbauth_front
%%%fbauth_desc
%%%fbauth_www
%%%fbauth_pic
%%%fbauth_pic portrait
%%%fbauth_pic background
%%%fbauth_pic other
%%%fbauth_tags
%%%fbauth_pubs
%%%endfbauth
 
\textbf{Dmitriy Vysotskyi} 

ну и в титуле московского князя, а потом и царя значилось \enquote{Всея Руси(России)}. По
термину \enquote{Украина} рекомендую замечательную монографию Фёдор Гайда \enquote{ИСТОРИЧЕСКАЯ
СПРАВКА О ПРОИСХОЖДЕНИИ И УПОТРЕБЛЕНИИ СЛОВА «УКРАИНЦЫ»}.


%%%fbauth
%%%fbauth_name
\iusr{Александр Каревин}
%%%fbauth_url
%%%fbauth_place
%%%fbauth_id
%%%fbauth_front
%%%fbauth_desc
%%%fbauth_www
%%%fbauth_pic
%%%fbauth_pic portrait
%%%fbauth_pic background
%%%fbauth_pic other
%%%fbauth_tags
%%%fbauth_pubs
%%%endfbauth
 
\textbf{Dmitriy Vysotskyi} Безусловно. Но в данном случае речь идёт об ответе на бредовый тезис, распространяемый укропропагандистами.

%%%fbauth
%%%fbauth_name
\iusr{Dmitriy Vysotskyi}
%%%fbauth_url
%%%fbauth_place
%%%fbauth_id
%%%fbauth_front
%%%fbauth_desc
%%%fbauth_www
%%%fbauth_pic
%%%fbauth_pic portrait
%%%fbauth_pic background
%%%fbauth_pic other
%%%fbauth_tags
%%%fbauth_pubs
%%%endfbauth
 
\textbf{Александр Каревин} Именно бредовый. И на их бред я всегда отвечаю: Московия? А что это? В каком официальном межгосударственном документе зафиксировано это название? Зачастую их это на некоторое время успокаивает.
\end{itemize}

%%%fbauth
%%%fbauth_name
\iusr{Татьяна Давыдова}
%%%fbauth_url
%%%fbauth_place
%%%fbauth_id
%%%fbauth_front
%%%fbauth_desc
%%%fbauth_www
%%%fbauth_pic
%%%fbauth_pic portrait
%%%fbauth_pic background
%%%fbauth_pic other
%%%fbauth_tags
%%%fbauth_pubs
%%%endfbauth
 
Даже англичане путешествуя к Ивану Грозному, знали, что Московия - московское княжество.

\begin{itemize}
%%%fbauth
%%%fbauth_name
\iusr{Алексей Щинников}
%%%fbauth_url
%%%fbauth_place
%%%fbauth_id
%%%fbauth_front
%%%fbauth_desc
%%%fbauth_www
%%%fbauth_pic
%%%fbauth_pic portrait
%%%fbauth_pic background
%%%fbauth_pic other
%%%fbauth_tags
%%%fbauth_pubs
%%%endfbauth
 
\textbf{Татьяна Давыдова} княжеств было много на Руси. Но Московское стало новым центром объединения. Был бы Киев, иностранцы бы называли Киевия

%%%fbauth
%%%fbauth_name
\iusr{Татьяна Давыдова}
%%%fbauth_url
%%%fbauth_place
%%%fbauth_id
%%%fbauth_front
%%%fbauth_desc
%%%fbauth_www
%%%fbauth_pic
%%%fbauth_pic portrait
%%%fbauth_pic background
%%%fbauth_pic other
%%%fbauth_tags
%%%fbauth_pubs
%%%endfbauth
 
\textbf{Алексей Щинников} Конечно же - московское - одно из многих. Источник: Английские путешественники в Московском государстве в 16 веке. Ленинград, 1937

%%%fbauth
%%%fbauth_name
\iusr{Алексей Щинников}
%%%fbauth_url
%%%fbauth_place
%%%fbauth_id
%%%fbauth_front
%%%fbauth_desc
%%%fbauth_www
%%%fbauth_pic
%%%fbauth_pic portrait
%%%fbauth_pic background
%%%fbauth_pic other
%%%fbauth_tags
%%%fbauth_pubs
%%%endfbauth
 
\textbf{Татьяна Давыдова} если мы будем себя называть как нас зовут иностранцы, то устанем от постоянного переименования

%%%fbauth
%%%fbauth_name
\iusr{Татьяна Давыдова}
%%%fbauth_url
%%%fbauth_place
%%%fbauth_id
%%%fbauth_front
%%%fbauth_desc
%%%fbauth_www
%%%fbauth_pic
%%%fbauth_pic portrait
%%%fbauth_pic background
%%%fbauth_pic other
%%%fbauth_tags
%%%fbauth_pubs
%%%endfbauth
 
\textbf{Алексей Щинников} Правильно. Но почему мы в родной стране позволяем себя называть чужими именами и коверкаем изначальные имена - это главное.

%%%fbauth
%%%fbauth_name
\iusr{Алексей Щинников}
%%%fbauth_url
%%%fbauth_place
%%%fbauth_id
%%%fbauth_front
%%%fbauth_desc
%%%fbauth_www
%%%fbauth_pic
%%%fbauth_pic portrait
%%%fbauth_pic background
%%%fbauth_pic other
%%%fbauth_tags
%%%fbauth_pubs
%%%endfbauth
 
\textbf{Татьяна Давыдова} например?

%%%fbauth
%%%fbauth_name
\iusr{Татьяна Давыдова}
%%%fbauth_url
%%%fbauth_place
%%%fbauth_id
%%%fbauth_front
%%%fbauth_desc
%%%fbauth_www
%%%fbauth_pic
%%%fbauth_pic portrait
%%%fbauth_pic background
%%%fbauth_pic other
%%%fbauth_tags
%%%fbauth_pubs
%%%endfbauth
 
\textbf{Алексей Щинников} мыколы ганны cвiты, олэксы. Только не надо говорить, что это другое.

%%%fbauth
%%%fbauth_name
\iusr{Алексей Щинников}
%%%fbauth_url
%%%fbauth_place
%%%fbauth_id
%%%fbauth_front
%%%fbauth_desc
%%%fbauth_www
%%%fbauth_pic
%%%fbauth_pic portrait
%%%fbauth_pic background
%%%fbauth_pic other
%%%fbauth_tags
%%%fbauth_pubs
%%%endfbauth
 
\textbf{Татьяна Давыдова} это польские или чешские имена?

%%%fbauth
%%%fbauth_name
\iusr{Татьяна Давыдова}
%%%fbauth_url
%%%fbauth_place
%%%fbauth_id
%%%fbauth_front
%%%fbauth_desc
%%%fbauth_www
%%%fbauth_pic
%%%fbauth_pic portrait
%%%fbauth_pic background
%%%fbauth_pic other
%%%fbauth_tags
%%%fbauth_pubs
%%%endfbauth
 
\textbf{Алексей Щинников} перевод русских имён на украинский. Т.е. человек Кирилл, а он Кырыло, Елена - Олена и т.д. У меня подруга в суде доказывает, что она это она - такие коллизии тоже происходят.
\end{itemize}

%%%fbauth
%%%fbauth_name
\iusr{Пётр Гринёв}
%%%fbauth_url
%%%fbauth_place
%%%fbauth_id
%%%fbauth_front
%%%fbauth_desc
%%%fbauth_www
%%%fbauth_pic
%%%fbauth_pic portrait
%%%fbauth_pic background
%%%fbauth_pic other
%%%fbauth_tags
%%%fbauth_pubs
%%%endfbauth
 

Очень точно высказался по этому вопросу Сигизмунд Герберштейн, посещавший Москву
в начале XVI века. Русью ныне владеют три государя, написал Герберштейн: это
король польский, князь литовский и великий князь московский. Наиболее
могущественный - это князь московский. Ну а "Московия" это всего лишь отсылка к
столице государства. Кстати, французский путешественник (имя подзабыл) говорил, что
жителей России неверно называть московитами, потому как сами себя они называют
русскими.

\begin{itemize}
%%%fbauth
%%%fbauth_name
\iusr{Александр Каревин}
%%%fbauth_url
%%%fbauth_place
%%%fbauth_id
%%%fbauth_front
%%%fbauth_desc
%%%fbauth_www
%%%fbauth_pic
%%%fbauth_pic portrait
%%%fbauth_pic background
%%%fbauth_pic other
%%%fbauth_tags
%%%fbauth_pubs
%%%endfbauth
 
\textbf{Пётр Гринёв} Имя француза - Жак Маржерет.
\end{itemize}

%%%fbauth
%%%fbauth_name
\iusr{Юрий Симонов}
%%%fbauth_url
%%%fbauth_place
%%%fbauth_id
%%%fbauth_front
%%%fbauth_desc
%%%fbauth_www
%%%fbauth_pic
%%%fbauth_pic portrait
%%%fbauth_pic background
%%%fbauth_pic other
%%%fbauth_tags
%%%fbauth_pubs
%%%endfbauth
 
Ну почему над московитами стебаются, а москвичам- завидуют?))

\begin{itemize}
%%%fbauth
%%%fbauth_name
\iusr{Стефан Машкевич}
%%%fbauth_url
%%%fbauth_place
%%%fbauth_id
%%%fbauth_front
%%%fbauth_desc
%%%fbauth_www
%%%fbauth_pic
%%%fbauth_pic portrait
%%%fbauth_pic background
%%%fbauth_pic other
%%%fbauth_tags
%%%fbauth_pubs
%%%endfbauth
 
Первое — следствие второго!

%%%fbauth
%%%fbauth_name
\iusr{Юрий Симонов}
%%%fbauth_url
%%%fbauth_place
%%%fbauth_id
%%%fbauth_front
%%%fbauth_desc
%%%fbauth_www
%%%fbauth_pic
%%%fbauth_pic portrait
%%%fbauth_pic background
%%%fbauth_pic other
%%%fbauth_tags
%%%fbauth_pubs
%%%endfbauth
 
\textbf{Стефан Машкевич} как правило- да.
\end{itemize}

% -------------------------------------
\ii{fbauth.pokrass_jurij.kiev.ukraina}
% -------------------------------------
 
Монета Василия III.16 век, читаем.

\ifcmt
  ig https://scontent-frx5-1.xx.fbcdn.net/v/t39.30808-6/241254045_1235688483524689_1106077495912596_n.jpg?_nc_cat=110&_nc_rgb565=1&ccb=1-5&_nc_sid=dbeb18&_nc_ohc=e1k4biXHoJMAX-IjrnJ&_nc_ht=scontent-frx5-1.xx&oh=39500b3ca237ae637cf01d88092a13b8&oe=6138CA1D
  width 0.3
\fi

%%%fbauth
%%%fbauth_name
\iusr{Стефан Машкевич}
%%%fbauth_url
%%%fbauth_place
%%%fbauth_id
%%%fbauth_front
%%%fbauth_desc
%%%fbauth_www
%%%fbauth_pic
%%%fbauth_pic portrait
%%%fbauth_pic background
%%%fbauth_pic other
%%%fbauth_tags
%%%fbauth_pubs
%%%endfbauth
 
Очень интересно. Не знал. Спасибо!

%%%fbauth
%%%fbauth_name
\iusr{Евгения Сокор}
%%%fbauth_url
%%%fbauth_place
%%%fbauth_id
%%%fbauth_front
%%%fbauth_desc
%%%fbauth_www
%%%fbauth_pic
%%%fbauth_pic portrait
%%%fbauth_pic background
%%%fbauth_pic other
%%%fbauth_tags
%%%fbauth_pubs
%%%endfbauth
 
Срашкой была и Срашкой осталась.

%%%fbauth
%%%fbauth_name
\iusr{Виталий Гаркуша}
%%%fbauth_url
%%%fbauth_place
%%%fbauth_id
%%%fbauth_front
%%%fbauth_desc
%%%fbauth_www
%%%fbauth_pic
%%%fbauth_pic portrait
%%%fbauth_pic background
%%%fbauth_pic other
%%%fbauth_tags
%%%fbauth_pubs
%%%endfbauth
 
Спасибо за фактаж! Историю надо учить! Это дано не всем


\end{itemize}

