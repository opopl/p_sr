% vim: keymap=russian-jcukenwin
%%beginhead 
 
%%file slova.atmosfera
%%parent slova
 
%%url 
 
%%author_id 
%%date 
 
%%tags 
%%title 
 
%%endhead 
\chapter{Атмосфера}

%%%cit
%%%cit_head
%%%cit_pic
%%%cit_text
Ресентимент возможен в искусственно созданной враждебной среде, в
\emph{атмосфере} подозрения и злонамеренности. Ресентимент может быть весьма
изобретательным, то есть враждебность врага должна полагаться как не всегда
открытая и ясная.  Выражение «враг коварен» означает, что мы любое действие
врага можем проинтерпретировать как хитрый злонамеренный замысел.  Иногда
состояние враждебности может быть доведено до мирозлобия: весь мир суть наш
враг, мы в глобальной опасности. Такое доведение опасности и страха до абсурда
Йозеф Геббельс описывал так: «Имей мужество жить в опасности».  Враг должен
быть равный или более сильный. Вражда с более слабым врагом недостойна. Победа
более ценна, когда это была победа над превосходящим врагом или в
продолжительной равной схватке. Победа над слабым врагом или добровольная сдача
врага в плен-подчинение-оккупацию унижает и приуменьшает собственное
достоинство
%%%cit_comment
%%%cit_title
\citTitle{Враг}, Сергей Дацюк, analytics.hvylya.net, 18.11.2021
%%%endcit
