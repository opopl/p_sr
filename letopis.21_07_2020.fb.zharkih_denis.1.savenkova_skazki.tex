% vim: keymap=russian-jcukenwin
%%beginhead 
 
%%file 21_07_2020.fb.zharkih_denis.1.savenkova_skazki
%%parent 21_07_2020
 
%%url 
 
%%author 
%%author_id 
%%author_url 
 
%%tags 
%%title 
 
%%endhead 

\subsection{Рецензия - Савенкова - Жарких - Сказки}

Вторая рецензия  - Дениса Жарких. Наверное, такую рецензию может написать лишь
настоящий сказочник. И так оно и есть: Денис пишет сказки. Я рада, что есть
такие люди, которые пишут добрые и хорошие истории. Они должны  быть. Потому
что если уходят добрые волшебники, на их место приходят злые. Спасибо Денису за
рецензию.

Мудрость и возраст

В моем детстве была целая индустрия добрых сказок: тут и "Сказка о потерянном
времени", и "Приключения желтого чемоданчика", и "Буратино"... да разве хватит
сил перечислять книги, фильмы, мультфильмы. И все это у детей забрали. 

Нельзя сказать, что уж совсем ничего не делается, что-то мастерят, только явно многие творцы потеряли секрет ремесла. 

Могу сказать, что утеряно - любая сказка для детей дело весьма трудное и туда
необходимо заложить весьма недетскую мудрость. Будет мудрость - будет сказка, а
нет, то все будет забавно и прикольно, но эффект будет не тот, не пойдет
работа, не сохранится. 

Работать со сказками - работать с вечностью, не каждый сумеет найти материал,
не каждый сумеет справиться. Множество авторов не понимает ни композиции, ни
драматургии, ни замысла, пишут, что в голову придет, а другие снимают, что
снимается. 

И тут присылает мне Фаина Савенкова (11 лет) свою пьесу "Умри, чудовище". Я
занят до умопомрачения, и первая мысль: "Что ты понимаешь, малявка, в
драматургии? Тут взрослые не справляются. Одно дело эссе, то есть, почти стихи
в прозе, а другое дело полновесное драматическое произведение. Тут кроме
таланта еще и мастерство нужно, опыт". 

Но обещал, значит прочту. И тут, признаюсь, посадила малявка меня в лужу - и
замысел, и композиция, и образы, и драматургия - все на месте. И, главное,
мудрость, вечные ценности любовь, семья, дружба - все в одном небольшом
произведении. Именно в произведении, а не в работе. 

И нет там ничего про войну, и про политику нет. Описана жизнь обычной семьи и
немного волшебства. Вот совсем немного, как у старых мастеров. 

Откуда это? Не знаю, просто чудо. Это не пьеса, а готовый фильм для совместного
семейного просмотра. О том, что взрослым надо понимать детей, а детям взрослых.
О том, что эгоизм разрушает, а любовь создает. О том, что волшебство совершенно
в простых бытовых вещах, нужно просто заметить. И если заметишь, то будешь
счастлив. Фаина Савенкова заметила и поделилась счастьем с нами. Она из тех
самых, о которых некоторые титульные политики говорят "недолюди", по которым
дают приказ стрелять. А в ответ получают талант и искусство. Так что у меня
сомнений нет, кто недолюди, и кто заслужил свою обещанную пулю в лоб. И если в
мире есть такие сказочники волшебники, то добро обязательно победит зло. Ведь
есть сказки в которые опасно не верить.


\ifcmt
  pic https://scontent-amt2-1.xx.fbcdn.net/v/t1.6435-9/109734897_159408139013460_6169998944219168900_n.jpg?_nc_cat=101&ccb=1-3&_nc_sid=730e14&_nc_ohc=PqwZR0FEY-MAX-IC7Y8&_nc_ht=scontent-amt2-1.xx&oh=4f18d9d29f00b932ba1b858b153202e0&oe=6090555D
  width 0.4
\fi

