% vim: keymap=russian-jcukenwin
%%beginhead 
 
%%file books.istoria_rusiv
%%parent books
 
%%url 
 
%%author_id 
%%date 
 
%%tags 
%%title 
 
%%endhead 

\chapter{ІСТОРІЯ РУСІВ}

\begin{zznagolos}
Давно вже малоросіяни бажали бачити друкованою «Історію Русів, або Малої Росії,
твір Преосвященного ГЕОРГІЯ КОНИСЬКОГО».

Багато слів було час від часу, що той чи другий збираються видати її, навіть
друкують; але до сих пір Ті як нема, так нема! Маючи кілька списків цієї
історії, я вибрав ліпший з-поміж них, підвів до нього з інших різночитання, а
потім запропонував Імператорському Товариству Історії і Старожитностей
Російських видати його в світ, що і здійснюється нині. Час від часу я маю намір
те саме зробити і з іншими писемними джерелами Малої Росії, літописами,
записками, описами і т. ін., напр., з Шафонським, Симановським та інш.  А тому
я просив би всіх, хто тільки має і бажає бачити за короткий час ці та подібні
до них пам'ятники надрукованими, надсилати їх мені як секретареві Товариства
для зняття з них списків і негайного поміщення в «ЧТЕНІЯХЬ». Щира вдячність і
признання сучасників, які займаються історією, а також і самого потомства, буде
найкращою нагородою тим, хто відгукнеться на цей заклик. Пора вже, давно пора
не затаювати подібного скарбу і тим, оскільки можливо, полегшити вивчення і
пізнання історії Південних Русів для всіх і кожного, особливо ж самих Руських.
Добре, що є певна і скора нагода виконати цей священний обов'язок справжнього
сина свого народу і батьківщини.

О. Бодянський

Лютого 9, 1846, Москва
\end{zznagolos}

\section{ПЕРЕДМОВА}

Історія Малої Росії до пори нашестя на неї Татар з ханом їхнім Батиєм злучена з
Історією всієї Росії або ж вона і є єдина Історія Російська; бо ж відомо, що
початок сеї історії, разом з початком правління Російського, береться од Князів
і Князівств Київських, з прилученням до них лише одного Новгородського Князя
Рюрика, і триває до навали Татар безперервно, а від сього часу буття Малої
Росії в Загальній Російській Історії ледве згадується; по визволенню ж її від
Татар Князем Литовським Гедиміном і зовсім вона в Російській Історії замовчана.

Саме тому пропонована тут Історія Малоросійська писана на два періоди, тобто до
нашестя Татарського екстрактом, а від того нашестя — широко і докладно.

Істориків та Літописців сеї доби було в Малій Росії задосить. Але як ця країна,
начеб створена або приречена на руїну од частих навал чужинців, а ще частіших
наскоків та січей од народів сусідніх і, зрештою, од ненастанних міжусобиць і
побоїщ зазнала всіляких плюндрувань, згуби та всеспалення і, так би мовити,
залита і напоєна кров'ю людською і посипана попелом, то в такій нещасній землі
чи можливо було зберегти будь-що цілим? А з тої причини взято цю Історію із
літописів і записок Білоруських, як із країни одноплемінної, сусідньої і од
руїн Малоросійських віддаленої.

Вченістю відомий і знатністю славний Депутат Шляхетства Малоросійського пан
Полетика, коли виряджався у справах Депутатства до тої великої Імперської
Комісії для створення проекту нового укладу, то мав конечну потребу роздобути
вітчизняну Історію. Він удався з приводу цього до первісного навчителя свого,
Архієпископа Білоруського Георгія Кониського, котрий був питомим Малоросіянином
і впродовж значного часу перебував у Київській Академії Префектом і Ректором.

І сей-бо Архірей передав Панові Полетиці Літопис, або ж Історію цю, запевнюючи
архіпастирськи, що вона ведена з давніх літ в кафедральному Могильовскому
монастирі тямущими людьми, які здобували потрібні відомості від учених мужів
Київської Академії і різних найповажніших Малоросійских монастирів, а найбільше
від тих, де перебував ченцем Юрій Хмельницький, колишній гетьман
Малоросійський, що полишив у них чимало записок і паперів батька свойого,
гетьмана Зіновія Хмельницького, і самі журнали достопам'ятностей і діянь
національних, та й до всього вона знову ним переглянута і виправлена.

Пан Полетика, звіривши її з багатьма іншими літописами Малоросійськими і
знайшовши її од тих найліпшою, завше дотримувався її у довідках І писаннях по
Комісії. І так ся Історія, пройшовши "стільки умів видатних, здається, мусить
бути достовірною. Лише воєнні дії видадуться, можливо, декому сумнівними, бо ж
занадто численні. Та, міркуючи про становище землі сеї з-поміж народами сливе
непримиренними, судячи про часи та обставини, в яких народ сей завжди майже був
у вогні та плавав у крові, варто зробити висновок, що сього народу все ремесло
й управа полягали у війні та убивствах. Одна Польща доказом всьому тому. Вона
лише тоді була могутньою і страшною, коли мала у себе війська Малоросійські; а
лише скоро їх позбулася, відразу занепадати почала, а наслідки ті відомі.
Історики Польські та Литовські, справедливо запідозрювані у вигадках та
самохвальстві, описуючи діяння народу Руського, що начебто у підданстві
польському пробував, затьмарювали всіляко великі подвиги його, учинюванї на
користь спільної вітчизни своєї і Польської. Навіть самі постанови та привілеї
їхні у сій вітчизні затаювали, наближаючи якомога народ сей до рабського стану
й нікчемства. 

А коли дійшла повість їхня до часів гонінь і тиранства Польських,
на народ Руський учинений з приводу вигаданої від них Унії, а саме як дійшло до
визволення народу свого з кормиги Польської власною його мужністю і майже
безприкладною хоробрістю, то тут виригнули письменники тії всі свої лайки і
всілякого роду неправди І наклепи на сей народ і на його вождів та начальників,
називаючи їх непостійним і бунтівливим хлопством, що по сваволі і буйнощах
своїх бунти і заколоти вчиняло. 

Але діла Гетьманів Руських Косинського, Наливайка, ОстряницІ і, нарешті, великі
діла Хмельницького, листування їхні і декларації доводять вельми тому противне,
і всіляка людина здорового глузду добачить в них істину несумнівну і подвиги
шляхетні і справедливі; побачить при тому і визнає розумний, що всіляке
творіння має право буття своє боронити, власність І свободу і що для того воно
споряджено самою природою, або Творцем своїм достатніми знаряддями чи способом.

Про мужність і заповзятливість народу Руського даємо пораду творцям байок та
критикам заглянути в Історії Грецькі, Римські та інші іноземні; і вони їм
покажуть Кагана, Кия, Оскольда, Святослава, Володимира, Ярослава та інших
великих Володарів, або Князів Руських, що воювали славно з воїнством Руським в
Європі, Азії, Греції і на самі столиці їхні Константинополь і Рим нападали. І
хіба такий народ, який пожив дещо в поєднанні з Поляками і Литовцями у
повсякчасних майже війнах за їхню і за свою вітчизну, чи ж міг він загубити
природну свою хоробрість, яка згодом і над самими Поляками і Литовцями зрештою
доволі себе показала?

Але, незважаючи на все те, варто з жалем сказати, що занесені деякі безглузді
речі і наклепи в самі літописи Малоросійські, на нещастя, творцями їхніми,
питомими Русами, що необачно наслідували безсоромних і злосливих Польських і
Литовських байкотворців. 

Так, приміром, в одній шкільній історійці виводиться
на сцену зі Стародавньої Русі, або нинішньої Малоросії, нова якась земля над
Дніпром, названа тут Україною, а в ній зводяться Польськими Королями нові
поселення і засновуються Українські козаки; а до того ся земля була пустельна і
безлюдна, і Козаків на Русі не бувало. 

Але, видно, пан письменник такої нікчемної історійки не бував ніде, окрім своєї
школи, і не бачив у тій стороні, що її називає він Україною, Руських міст
найдавніших або принаймні далеко давніших од його Королів Польських, себто:
Черкаса, Крилова, Мишурина та старого Кодака над рікою Дніпром, Чигирина над
Тясмином, Умані над Россю, Ладижина і Чагарлика над Бугом, Могилева, Рашкова й
Дубосар над Дністром, Кам'яного Затону і Білозерська у гирлі Лиману. З тих міст
були деякі провІнціальними та обласними Руськими містами впродовж багатьох
віків. Але в нього все те пустеля, і Князі Руські, що виводили великі флотилії
свої в Чорне море із ріки Дніпра, себто з тих самих країн, які воювали на
Грецію, Синоп, Трапезонт і на самий Царгород з військами саме тих областей, ним
у непам'ять пущено; рівно як і сама Малоросія повернута кимось з Польського
володіння без зусилля і по добрій волі, а тридцять чотири кривавих герці, що
були при тому од військ Руських супроти Поляків та Королів їхніх і посполитого
рушення, не заслуговують на те, щоб визнати за народом сим та його вождями за
подвиги їхні і геройство належну справедливість. Одначе хто що не кажи, а
кінець діло вінчає завжди. Прийди і виждь!

\section{ІСТОРІЯ РУСІВ, або МАЛОЇ РОСІЇ}

Народ Слов'янський, що походить од племені Афета, Ноєвого сина, названий
Слов'янами за родоначальником і Князем своїм Славеном, нащадком Росса Князя,
внука Афетового. Він, переселившись з Азії од часів Вавілонського змішання мов,
став мешкати од гір Поясних, або Рифейських, і від моря Каспійського на Сході
до ріки Вісли і моря Варязького на Заході; і від Чорного моря і ріки Дунаю на
Півдні до Північного океану і Балтійського моря на Півночі. Доказом тому о
Історія Преподобного Нестора Печерського і його послідовників і попередників,
які ту історію писали і які всі були Академіками, або Членами тої головної
школи, яку в Слов'янах заведено було в місті Києві Кирилом, філософом Грецьким,
невдовзі по запровадженню там релігії Християнської. А взята вона з книг
Священних Біблій і з старовинної багатої бібліотеки, в Києві зібраної, яка в
пору нашестя варварів і од колишніх руїн загинула; від чого і самі школи
ховалися в самих монастирях та підземних житлах навіть до днів Руського
обраного Князя, або Гетьмана, Сагайдачного і Митрополита Київського Петра
Могили, які стародавню Академію Київську поновили.

Не меншим доказом означених меж Слов'янських суть спорожнілі міста і румовища,
Слов'янською мовою пойменовані, і написи, їхніми літерами і наріччям писані на
каменях, цвинтарях І статуях кам'яних; також назви рік, озер, гір та улусів,
які розташовані в степах Кримських, Заволжанських і на острові Тамані, або на
стародавньому Тмутаракані, — все це очевидно свідчить про Слов'янське тамтешнє
поселення. А зауважені деякими письменниками в тих межах чужеплемінні Слов'янам
народи, а саме: Кімеріани, або Готи, Маіоти, Гунни та інші, — нарівні з ордами
Понтійського Царя Мітрідата були перехожими через Слов'янську землю та іноземні
колонії, які найшли сюди зі Сходу та Півночі і по короткому тут перебуванню в
країни південні і західні відійшли. Та і самі Греки і Генуезці, які вважалися
мешканцями над Чорним і Азовським морем, були не чимсь іншим, як купцями, які
оселилися за згодою Слов'ян на їхніх приморських землях заради обопільної
торговельної вигоди; а війни Слов'ян з містами їхніми Херсоном, Феодосією та
Босфором означають хіба короткочасні сусідські чвари, що закінчувалися згодою.

Історики суміжних зі Слов'янами народів: Птолемей, Геродот, Страбон, Діодор та
інші — приписували Слов'янам давність сиву, за 1610 років до Різдва Христового
відому, мовлять, що вони, ведучи з сусідами безнастанні війни та переслідуючи
чужоплемінні народи, які переходили їхньою землею, зайшли і переселили колонії
свої за ріку Дунай, до моря Адріатичного в Іллірії і від гір Карпатських до
ріки Одра; а на західних берегах Балтійського моря оселили всю Померанію, їхнім
наріччям так пойменовану. Але дають ці історики Слов'янським племенам
різноманітні назви, залежно від способу їхнього життя і вигляду народного,
приміром, Східних Слов'ян називали Скіфами, або ж Скіттами, тому що жили вони
мандрівним життям і часто переселялися з місця на місце; Південних — Сарматами,
по гострих ящуриних очах з примружкою, і Русами, або Русняками, — за волоссям;
Північних приморських Варягами називали через хижацтво і засідки, де чигали на
перехожих; а всередині од тих мешканців, за родоначальниками їхніми, нащадками
Афетовими, так називали: по князю Русу — Роксоланами і Россами, а по Князю
Мосоку, кочівникові над річкою Москвою, що дав їй цю назву, — Москвитами і
Москами, від чого згодом і Царство їхнє дістало назву Московського і нарешті
Російського.

Самі Слов'яни і того більше собі назв наробили. Болгарами називали тих, які
мешкали над рікою Волгою; Печенігами тих, що живилися печеною їжею; Полянами і
Половцями, що жили на полях або в степах безлісих; Древлянами — мешканців
полісних, а Козарами всіх тих, що їздили верхи на конях та верблюдах і чинили
набіги; а сю назву дістали зрештою і всі воїни слов'янські, вибрані з їх же
породи для війни та оборони вітчизни, якій служили у власній збруї,
комплектуючись та переміняючись також своїми родинами. Та коли у пору війни
виходили вони поза свої межі, то інші цивільного стану мешканці давали їм
підмогу, і задля цього заведена була у них складка громадська, чи податок,
прозваний нарешті з обуренням Даниною Козарам.

Ці воїни, часто своїм союзникам допомагаючи, а паче Грекам у війнах з їхніми
ворогами, перейменовані Царем Грецьким Константином Мономахом з Козарів на
Козаків, — і така назва назавжди вже у них залишилась. А описувані деякими
письменниками війни Слов'ян з Печенігами, Половцями, Козарами та іншими
слов'янськими народами і бездоказово чужоплемінними війнами звані, означають не
що інше, як міжусобні самих Слов'ян січі за межі обласні, за відгін худоби І за
інші домагання і чвари Князів, що їх творили; а помилки від істориків виникли з
причини множества різних назв, одному й тому самому народові приписуваних.
Справедливість цього доводиться тим, що описувані вище чужоземні народи, себто
Готи, Гунни та інші, знані з Історій та переказів, звідки вони вийшли і куди
пішли; а про сих нічого того немає, і начеб з неба вони впали І в землю
ввійшли, не залишивши й потомства свого; такого Історія ніяк терпіти не повинна
як вигаданого.

Таким чином, частина Слов'янської землі, яка лежить од ріки Дунаю до ріки Двіни
і од Чорного моря до рік Стиру, Случі, Березини, і Дінця, і Сіви, дістала назву
Русь, а народ, що на ній проживає, названо Русами і Русняками взагалі. Згодом
та ж сама земля поділилась назвою на Чермну, або Червону, Русь — за землею, що
родить барвні трави та червець у краю полуденному, і на Білу Русь — за великими
снігами, що випадають у стороні північній.  Провінційними поділами тої землі
були Князівства: Галицьке, Переяславське, Чернігівське, Сіверське, Древлянське
і чільне, або Велике, Князівство Київське, котрому вся решта підлягала. Князі,
або Верховні начальники, обирані були народом в одній особі, але на цілу
династію, і нащадки обраного володіли за спадком. Із князів сих найзначніші за
Історіями: Каган, що Грецію воював і облягав флотилією своєю та сухопутним
військом столичне місто Константинополь, що врятувалось дивом Богоматері; Кий,
засновник міста Києва і князівства свого імені; переможні у війнах Оскольд і
Дір, що славно воювали з Греками та Генуезцями на морі і суші, що зруйнували
славні міста Синоп і Трапезонт і наголову розбили війська ворожі над рікою
Осколом; Ігор, що підступно побив Оскольда і Діра і сам був убитий Древлянами;
Святослав, що підкорив собі Болгар Задунайських і жив там у місті Переяславці,
в сучасному Рущуку; і Володимир, який першим хрестив усю Ро'сію.

І сей Володимир, понад означені Князівства, з'єднав всі інші Слов'янські
Князівства, які розділилися були під різними назвами поміж його братами і
родичами, був один над ними Самодержець і звався Великим Князем Руським і
Цариком над усіма Князями; і, будучи могутнім і лихим у війнах, що безнастанно
точилися з сусідами, набув од них і від народів віддалених великої поваги, чому
ж всі Держави запобігали його дружби, а для утримування її пропонували йому
свої віри або релігії. Але він, звідавши спершу їх, слушно віддав перевагу
перед усіма Християнській Грецького, або Єрусалимського обряду і року 988-го по
Різдві Христовому, вирушивши з військом до приморського міста Херсона,
хрестився, там од Греків і побрався з Грецькою Царівною Анною; а повернувши до
Києва, хрестив родину свою і народ. Перед хрещенням же всі Слов'яни мали віру
східних поган і, визнаючи єдиного Бога Вседержителя, вважали символом і житлом
його сонце, а знаряддям гніву — його грім, або перун. Тому і вшановували сонце
запалюванням вогню як його образу, вкидаючи туди початки од всього ростучого, а
празник сей називаючи Купалою.

Хрещення Володимирове вважається третім в літописах Слов'янських; перше у них
введене за днів Апостольських, благовістям Апостола Андрія Первозванного, що
приплив був кораблем з Чорного моря і рікою Дніпром до тої Київської гори,
котра по заснуванню міста Києва завше Андрієвою горою звалася і що на ній
опісля збудовано в ім'я його церкву.  Сей же апостол рікою Десною був тоді і в
Новгороді-Сіверському, благовістив Євангелію і дивувався з того, як тамтешній
люд лазні свої уживає, де, за словами його, розпалювала кожна людина себе,
немов розпечений камінь, сікла себе хворостом до знемоги, а тоді, кидаючись з
шумом в річкову воду, виходила звідтіль жива й бадьора, начебто ніколи не
розпалювана й не бита. Річки тії до упадку Чорного моря мали води вищі од
нинішніх, і пороги на Дніпрі не були відкриті. Друге хрещення перевела баба
Володимирова, Велика Княгиня Київська Ольга, що сама хрестилася в Царгороді і
була наречена по хрещенню Оленою.

Після скону Володимира Першого незабаром скінчилося і об'єднання його царства.
Сини та небожі Володимирові поділили його на дванадцять Князівств, залишивши ж,
одначе, по-старому найвищим над усіма Велике Князівство Київське, де
найголовніші від інших Князів були: Ярослав Володимирович, який поширив і
утвердив Християнство, уложив через обраних мужів Руські закони, заснував у
Києві головну школу Богослов'я та інших красних наук з багатою, із Греції
виписаною, бібліотекою і додержував першості своєї зі славою; Володимир Другий,
названий Мономахом по дідові його з материного боку, Імператорові Грецькому
Константинові Мономахові, по якому і він визнаний од Грецької Імперії Царем
Руським і одержав на те дідівську корону з усіма іншими Царськими регаліями. Та
міжусобні війни за першість і наслідство, що постали були за розподілом
Князівств і все тривали між Князями, спершу ослабили Велике Князівство
Київське, а згодом і зовсім його розірвали, і з 1161 року назвалися Великими
Князівствами: Галицьке в Чермнорусії, Володимирське на Клязьмі і, нарешті,
Московське по місту Москві. Але й ті Князівства славилися першістю своєю до
1238 року; а від того року нашестя війною Мунгальських Татар під начальством
Хана їхнього Батия, онука Чінгісханового, всі Князівства удільні і великі
зруйнувало майже дощенту; міста їхні і села сплюндровано І багато спалено;
Князі і воїнство вбито, а ті, що лишилися, розсіялись по віддалених Північних
провінціях, і від того часу більша частина Руських Князівств підпала в
Татарську неволю. І хоч Князівства знову постали, та перебували вони з Князями
своїми в підданстві Татарських Ханів, які, стягаючи данину з народу,
настановляли в них Князів і змінювали їх на свій розсуд, що тривало аж до 1462
року, у який Князь Московський Іван Васильович, Третій сього імені,
скориставшись зі слабості Татар, які знемоглися від міжусобних війн та
розділів, відмовив ханові Ахматові щорічної данини з народу і своєї покори; а
внук сього Князя, Іван Васильович Четвертий, названий Грозним, злучивши багато
Князівств Руських воєдино, в році 1547-му перейменував себе з Князя на Царя і
Самодержця Московського, і відтоді завше вже Царство Московське і його володарі
сею назвою титулувались з перейменуванням, нарешті, Царства Московського на
Російське, яке, на відміну від Чермної і Білої Русі, звалося Великою Росією; ті
ж обидві Русі вкупі названі тоді Малою Росією.
