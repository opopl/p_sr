% vim: keymap=russian-jcukenwin
%%beginhead 
 
%%file 15_10_2022.fb.ukr.min_kultury_inf_politiki.1.kerpatenko_murder
%%parent 15_10_2022
 
%%url https://www.facebook.com/MCIPUkraine/posts/pfbid0HaJNdiAZV2tXJ8rMDNpeDX3N7PzZiG2EHNZNNsPLM5XhVMY7fchHR2X1BR6otQ2xl
 
%%author_id ukr.min_kultury_inf_politiki
%%date 
 
%%tags herson,murder,ubijstvo,ukraina
%%title Вбивство Юрія Керпатенко - головного диригента Херсонського музично-драматичного театру
 
%%endhead 
 
\subsection{Вбивство Юрія Керпатенко - головного диригента Херсонського музично-драматичного театру}
\label{sec:15_10_2022.fb.ukr.min_kultury_inf_politiki.1.kerpatenko_murder}
 
\Purl{https://www.facebook.com/MCIPUkraine/posts/pfbid0HaJNdiAZV2tXJ8rMDNpeDX3N7PzZiG2EHNZNNsPLM5XhVMY7fchHR2X1BR6otQ2xl}
\ifcmt
 author_begin
   author_id ukr.min_kultury_inf_politiki
 author_end
\fi

14 жовтня стало відомо про жорстоке вбивство головного диригента Херсонського
музично-драматичного театру імені Миколи Куліша. 

Після того, як Юрій Керпатенко відмовився співпрацювати з окупантами, російські
військові застрелили його у власному будинку. Про це повідомила журналістка
Олена Ваніна.

Юрій Керпатенко з 2000-го року працював у Херсонській обласній філармонії. У
2004-му він став головним диригентом Херсонського музично-драматичного театру
імені Миколи Куліша. Виступав головним диригентом камерного оркестру \enquote{Гілея}. 

Юрій Керпатенко відкрито проявляв свою громадянську позицію і відмовився
виїжджати з окупованого Херсона 🇺🇦

Як повідомляє сайт Херсон.онлайн, до 1 жовтня, міжнародного дня музики,
окупанти разом з колаборантами у філармонії планували у Херсоні святковий
концерт за участю знаменитого камерного оркестру \enquote{Гілея}. Цей концерт мав за
задумом окупантів продемонструвати так зване \enquote{налагодження мирного життя} в
Херсоні. Проте диригент оркестру Юрій Керпатенко категорично відмовився
співпрацювати з окупантами. 

Висловлюємо найглибші співчуття близьким та колегам Юрія. Вічна памʼять @igg{fbicon.candle} 
