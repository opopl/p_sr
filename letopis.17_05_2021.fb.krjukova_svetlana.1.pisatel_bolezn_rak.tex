% vim: keymap=russian-jcukenwin
%%beginhead 
 
%%file 17_05_2021.fb.krjukova_svetlana.1.pisatel_bolezn_rak
%%parent 17_05_2021
 
%%url https://www.facebook.com/kryukova/posts/10159400793263064
 
%%author 
%%author_id 
%%author_url 
 
%%tags 
%%title 
 
%%endhead 

\subsection{У сорокалетнего школьного учителя нашли неоперабельный рак и дали ему год жизни}
\Purl{https://www.facebook.com/kryukova/posts/10159400793263064}

У сорокалетнего школьного учителя нашли неоперабельный рак и дали ему год жизни. 

Учитель был крепких англосаксонских кровей и сурово озаботился одним: как бы
обеспечить жену и дочерей, остающихся без всяких средств к существованию...

Он преподавал язык и литературу и не сумел придумать лучшего способа заработать сносную сумму, как попробовать написать роман. 
И такой роман, чтоб его хорошо читали — раскупали. Читателей он представлял в виде своих учеников и их родителей. 
И героев представлял в таком же духе. Жизнь он представлял только в объеме родной рабочей окраины.
Дело было новым, он втянулся и увлёкся. Срок поджимал. Он спешно и отчаянно овладевал ремеслом. Высокая литература его не интересовала. Его интересовало завещать авторское право семье: на что жить.
И к концу своего года Энтони Бёрджес завершил свой роман «Заводной апельсин». 
Миллион был срублен! 
В культовом фильме сыграл юный Малькольм Мак-Дауэлл. Шпана надела котелки и стала спрашивать в барах молоко. Книгу перевели на полста языков.
Не свой от удачи и выполненного долга Бёрджес хорошо выпил и отправился к врачу. 
Врач посмотрел снимки, полистал историю болезни и вылупил глаза: рака не было. 
Бёрджес выздоровел.
Он стал писателем. 
Написал более 50 книг. 
А так же начал писать музыку и написал 175 музыкальных произведений. Даже симфонический оркестр заказывал произведения у Бёрджеса...
©️ Михаил Веллер, "Слово и профессия"
