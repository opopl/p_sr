% vim: keymap=russian-jcukenwin
%%beginhead 
 
%%file 08_12_2020.news.ru.53news.lavrova_olga.1.troickii_raskop_arheologia
%%parent 08_12_2020
 
%%url https://53news.ru/novosti/63168-zagadochnaya-kostyanaya-gramota-iz-troitskogo-raskopa-okazalas-unikalnym-svidetelstvom-drevnego-brachnogo-obryada.html
 
%%author Лаврова, Ольга
%%author_id lavrova_olga
%%author_url 
 
%%tags arheologia,russia
%%title Загадочная «костяная грамота» из Троицкого раскопа оказалась уникальным свидетельством древнего брачного обряда
 
%%endhead 
 
\subsection{Загадочная «костяная грамота» из Троицкого раскопа оказалась уникальным свидетельством древнего брачного обряда}
\label{sec:08_12_2020.news.ru.53news.lavrova_olga.1.troickii_raskop_arheologia}
\Purl{https://53news.ru/novosti/63168-zagadochnaya-kostyanaya-gramota-iz-troitskogo-raskopa-okazalas-unikalnym-svidetelstvom-drevnego-brachnogo-obryada.html}
\ifcmt
	author_begin
   author_id lavrova_olga
	author_end
\fi

\ifcmt
pic https://53news.ru/images/wsscontent/articles/2020/12/zagadochnaya-kostyanaya-gramota-iz-troitskogo-raskopa-okazalas-unikalnym-svidetelstvom-drevnego-brachnogo-obryada.jpg
\fi

{\bfseries 
Известный лингвист и текстолог Алексей Гиппиус рассказал во время онлайн-лекции о загадке новгородской «костяной грамоты».
}

В нынешнем археологическом сезоне на Троицком 16-м раскопе не нашли берестяных
грамот, зато обнаружили необычный предмет — кость с написанным на ней текстом.

По словам \textbf{Алексея Гиппиуса}, само по себе письмо на кости — интересное явление.
Известно некоторое количество новгородских костяных изделий с надписями. Так
писали и в средневековой Скандинавии — до наших дней дошли рунические надписи
на кости.

Но в нашем случае эта находка — костяная бирка, сделанная из коровьего ребра.
На ней говорится: «молвила куна соболи» и указаны числа.

Первоначально Алексей Гиппиус предложил такую версию: видимо, это запись о
торговой сделке, связанной с мехом соболя. Раз Куна «молвила», то это человек,
тем более что такое имя действительно существовало. Вспомним населенные пункты
с названием «Кунино». А дальше речь идет о соболях. Видимо, за них заплатили
112 гривен, а было этих соболей два сорочка (50 плюс 30 отлично делятся на
сорок, а мех измерялся сороками).

\ifcmt
pic https://53news.ru/images/images/2020/12/8/kuna5.jpg
\fi

Эту весьма правдоподобную версию учёный даже успел озвучить на семинаре две
недели назад. Однако теперь он уверен, что на самом деле всё совсем не так.

Правильный вариант, по его словам, позже предложила филолог Марина
Бобрик-Фремке.\Furl{http://ruslang.ru/publica/bobrik} Рассказывая о нём, Алексей Гиппиус пошутил, что прежде находился
под гипнозом собственных построений в области меховой торговли.

Марина Бобрик-Фремке напомнила, что вряд ли в одном документе «куна» и «соболь»
могут принадлежать разным семантическим сферам, где Куна — имя человека, а
соболь — мех.

\begin{leftbar}
	\begingroup
		\em По её мнению, в данном случае слова «куна» и «соболь» жители древнего
				Новгорода использовали в переносном значении. Это метафорические
				наименования жениха и невесты, которые можно встретить в русском
				свадебном обряде, в восточно-славянском песенном фольклоре
				\footnote{Иллюстрации: фрагменты презентации из лекции Алексея Гиппиуса}
	\endgroup
\end{leftbar}

Теперь Алексей Гиппиус уверен: речь в документе действительно идет о торговле,
только символической, брачной. Торг за невесту ведется в гривнах.

Он отметил, что за девушку запросили очень большую сумму — 14 гривен серебра
(цифра 112). А дальше следует ответ другим почерком — это ответ, цену
попробовали снизить. Сначала предложили 50, затем добавили еще 30. После этого,
видимо, стороны договорились.

Комментируя свою первоначальную версию, учёный рассказал одну историю,
связанную с незабвенным Андреем Анатольевичем Зализняком.\Furl{https://53news.ru/novosti/35819-umer-issledovatel-novgorodskikh-berestyanykh-gramot-andrej-zaliznyak.html} Однажды они оба долго
и безуспешно пытались понять надпись на иконе, которая была совершенно
нечитаемой. Лингвисты высказывали версии, успевая поверить в них. Но потом,
подводя итоги этих напрасных поисков, академик Зализняк написал Алексею
Гиппиусу такую фразу: «Да, сильный-таки дурман представляет собой эйфория,
которая наступает, когда решишь, что что-нибудь открыл».

Завершая рассказ о «костяной «грамоте», Алексей Гиппиус отметил, что она являет
собой уникальный документ, новое материальное свидетельство древнего брачного
обряда.

Также во время онлайн-лекции учёный рассказал о том, что житель древнего
Новгорода восхитил его красотой почерка,\Furl{https://53news.ru/novosti/63132-zhitel-drevnego-novgoroda-voskhitil-sovremennykh-uchjonykh-krasotoj-pocherka.html} а одной из женщин не очень повезло с
именем.\Furl{https://53news.ru/novosti/63165-zhitelnitse-velikogo-novgoroda-ne-ochen-povezlo-s-imenem.html}
