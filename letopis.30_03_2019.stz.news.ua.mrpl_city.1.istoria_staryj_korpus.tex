% vim: keymap=russian-jcukenwin
%%beginhead 
 
%%file 30_03_2019.stz.news.ua.mrpl_city.1.istoria_staryj_korpus
%%parent 30_03_2019
 
%%url https://mrpl.city/blogs/view/istoriya-staryj-korpus
 
%%author_id burov_sergij.mariupol,news.ua.mrpl_city
%%date 
 
%%tags 
%%title История: Старый корпус
 
%%endhead 
 
\subsection{История: Старый корпус}
\label{sec:30_03_2019.stz.news.ua.mrpl_city.1.istoria_staryj_korpus}
 
\Purl{https://mrpl.city/blogs/view/istoriya-staryj-korpus}
\ifcmt
 author_begin
   author_id burov_sergij.mariupol,news.ua.mrpl_city
 author_end
\fi

\ii{30_03_2019.stz.news.ua.mrpl_city.1.istoria_staryj_korpus.pic.1}

Это добротное, сложенное из красного кирпича старинное здание по адресу улица
Итальянская, 115 знают, пожалуй, все мариупольцы. Кто-то здесь учился, кто-то
собирался здесь учиться, да не приняли, у кого-то учились дети или даже внуки.
Нетрудно догадаться, что речь здесь идет о первом корпусе Приазовского
государственного технического университета. Есть сведения, что построен он по
проекту и (уж совершенно точно) под наблюдением \textbf{Виктора Александровича Нильсена}
– выпускника Санкт-Петербургского института гражданских инженеров,
приглашенного мариупольской городской думой на должность главного архитектора.

\textbf{Читайте также:} 

\href{https://mrpl.city/news/view/semya-nilsen-vyskazalas-po-povodu-nilsenomanii-v-mariupole}{%
Семья Нильсен высказалась по поводу \enquote{нильсеномании} в Мариуполе, Роман Катріч, mrpl.city, 29.03.2019}

Строилось здание для епархиального училища. Заглянув в Википедию, мы узнаем,
что так назывались средние женские учебные заведения в России, созданные по
Уставу 1843 года, главным образом, для дочерей священников. Епархиальные
училища содержались на средства, полученные от сборов с церквей, отчислений от
свечных заводов и т.п., находились в ведении Синода. Дочери священников
обучались бесплатно, девочки из других сословий – за плату. Учебный курс,
состоящий из шести классов, был близок к курсу женских гимназий. К числу
необязательных предметов, преподаваемых за особую плату и во внеклассное время,
относились новые языки: французский и немецкий, музыка, рисование. Окончившие
курс епархиального училища получали звание домашних учительниц и право
преподавать предметы, по которым имели хорошую успеваемость. Их принимали
учительницами начальных, по преимуществу сельских, школ (с 1884 г. –
церковно-приходских). С 1900 г. при епархиальных училищах были открыты
специальные седьмые педагогические классы. При некоторых епархиальных училищах
существовали начальные школы, в которых воспитанницы старших классов проходили
практику.

\ii{30_03_2019.stz.news.ua.mrpl_city.1.istoria_staryj_korpus.pic.2}

По данным заместителя директора Мариупольского краеведческого музея по науке \textbf{Р.
П. Божко}, в нашем городе епархиальное училище было открыто в 1910 году, а уже
на следующий год состоялся первый выпуск воспитанниц-старшеклассниц из числа
тех, кто был переведен из Екатеринослава. Помещения епархиального училища
использовались следующим образом. В двух нижних этажах находились учебные
классы, кабинет директора, учительская. На третьем этаже помещались дортуары,
то есть, говоря современным языком, спальни иногородних епархиалок.
Епархиальные училища, в том числе и мариупольское, были ликвидированы декретом
Совета Народных Комиссаров от 24 декабря 1917 года...

\ii{30_03_2019.stz.news.ua.mrpl_city.1.istoria_staryj_korpus.pic.3}

\textbf{Читайте также:} 

\href{https://mrpl.city/news/view/muzyka-spaseniya-dlya-5-mesyachnoj-dashi-romanovoj-organizovali-blagotvoritelnye-kontserty-foto}{%
Музыка спасения: для 5-месячной Даши Романовой организовали благотворительные концерты, Олена Онєгіна, mrpl.city, 29.03.2019}

Пока трудно сказать, кто занял опустевшие помещения училища после его
упразднения. Но достаточно достоверно известно, что в 1921 году в нем был
устроен губернский детский городок, а проще – детский дом. Об этом многое
довелось узнать от \textbf{Елены Ивановны Шевченко}, которая на протяжении многих лет
бережно хранила рукопись воспоминаний своей мамы, \textbf{Марии Александровны Добровой},
– одной из тех, кто непосредственно принимал участие в организации учреждения.
Вот что поведала Елена Ивановна: 

\begin{quote}
\em	
\enquote{Мама приехала в Мариуполь вместе с
воспитанниками шестидесяти детских домов, ранее разбросанных по территории
нынешних Донецкой и Луганской областей. Детей привезли где-то в августе прямо
на железнодорожный вокзал. Первое, что сделали, - повели их на море, выкупали,
постригли, переодели, в общем, привели в порядок. А потом колонна детдомовцев с
воспитателями и обслугой, с тремя духовыми оркестрами отправилась к своему
новому жилищу. Правда, выяснилось, что оно еще непригодно для проживания: там
только-только закончился ремонт, и краска еще не высохла. Пришлось
новоприбывшим разместиться во дворе в наскоро установленных палатках. Через
десять дней все уладилось, и две тысячи ребят и триста человек обслуживающего
персонала вошли в здание. Мама там работала поваром, затем - заведующей
столовой. Из губернского детского городка вышло немало специалистов, многие из
которых работали на строительстве \enquote{Азовстали}, других предприятиях}. 
\end{quote}

Известно, что последним директором губернского детского городка, пока он
находился в здании бывшего епархиального училища, был Дмитрий Александрович
Михалевский.

\ii{30_03_2019.stz.news.ua.mrpl_city.1.istoria_staryj_korpus.pic.4}

В середине 20-х годов прошлого века, скорее всего в 1925 году, здание на
Шишманке обрело нового хозяина. Это было управление 238-го Мариупольского
территориального стрелкового полка. Смысл слов \enquote{Мариупольский} и \enquote{стрелковый} в
названии полка очевидны. А вот значение \enquote{территориальный} для современных
читателей, наверное, нужно пояснить. Территориальные воинские части и
соединения формировались в Советском Союзе в 20-30-е годы. Они включали в себя
постоянный командный состав и рядовых - жителей той местности, где
дислоцировалось соответствующее воинское формирование. Их периодически
призывали на службу, где они осваивали военные специальности на протяжении
нескольких месяцев. Таким образом, в 238-м полку служили обитатели Мариуполя и
близлежащих сел и поселков. Известный журналист Семен Гольдберг много лет
изучал историю этой части, в результате он написал и опубликовал документальную
повесть \enquote{Слово о полку Мариупольском}. Позже автор включил это произведение,
существенно дополненное новыми героями и фактами, в свой сборник \enquote{Солдаты
победы}. С. Гольдберг установил, что в 1939 году 238-й полк был развернут в
80-ю стрелковую дивизию, которая вскоре покинула наш город, как позже
оказалось, навсегда.

\textbf{Читайте также:} 

\href{https://mrpl.city/news/view/chudo-rebenok-iz-mariupolya-udivivshij-parizh-oderzhal-ocherednuyu-pobedu-foto}{%
Чудо-ребенок из Мариуполя, удививший Париж, одержал очередную победу, mrpl.city, 29.03.2019}

К моменту захвата Мариуполя гитлеровцами 8 октября 1941 года помещения, где
относительно недавно находились отделы и службы управления полка, опустели. Но
ненадолго. Сюда 18 октября 1941 года по приказу оккупационных властей собрали
все еврейское население, проживавшее на тот момент в городе. Мужчин и женщин,
стариков и детей. Двое суток более восьми тысяч человек без воды, в страшной
тесноте и полном неведении о том, что их ожидает в будущем, находились в здании
полка под охраной немецких солдат. Они были полностью изолированы от внешнего
мира. Утром 20 октября их повели в сторону поселка Агробаза. Там, у
противотанкового рва ни в чем не повинные люди были расстреляны. Много позже на
стене старого корпуса была установлена скульптурная композиция в память о
мариупольской трагедии. Ее авторы – заслуженный деятель искусств Украины \textbf{Ефим
Харабет} и скульптор \textbf{Юрий Балдин}. По данным, приведенным в фундаментальном
историческом труде В. М. Зиновьевой \enquote{Чтобы жизнь продолжалась.
Приазовье в период оккупации 1941-1943} (Мариуполь, 2004), в здании бывшего
епархиального училища во время хозяйничанья в нашем городе гитлеровцев
размещался штаб немецкой 60-й мотодивизии. 

В сентябре 1943 года гитлеровцы, отступая из Мариуполя под ударами частей и
соединений Южного фронта Красной Армии и Азовской военной флотилии, сожгли
большинство городских строений, в том числе и то, о котором здесь повествуется.
Опаленные огнем стены с зияющими проемами окон оставалось собственностью
Наркомата обороны. В 1944 году в Мариупольский металлургический институт
приехал новый преподаватель – \textbf{Иван Георгиевич Казанцев}. Его назначили
заместителем директора этого учебного заведения. Институт в то время ютился в
малоприспособленных для учебных занятий помещениях на улице Вузовской в
Ильичевском районе. Тогда-то и возникла идея у тогдашнего директора вуза
Тимофея Михайловича Смирнова попросить военное ведомство передать \enquote{погорелку}
институту. Были подготовлены соответствующие бумаги, с которыми Иван Георгиевич
отправился в Москву. Там, конечно, никто с распростертыми объятиями его не
ждал. И ему пришлось довольно долго убеждать министерских чиновников в
необходимости такой передачи. Однако И. Г. Казанцев добился своего. Его сын -
заслуженный деятель науки и техники Украины, профессор, доктор технических наук
Евгений Иванович - вспоминал: \emph{\enquote{Отец лично это здание выпросил в Министерстве
обороны. И, как он мне потом рассказывал, ему там сказали, что \enquote{еще никто
ничего у нас не сумел взять, только мы брали, а вам удалось забрать такое
большое строение}}}. Восстанавливали корпус строительные организации города. В
качестве рабочей силы они использовали немцев-военнопленных. Для них были
построены бараки, которые располагались неподалеку - на теперешней улице,
носящей имя профессора И. Г. Казанцева. Частично восстановив здание,
военнопленные переселились в него...

\textbf{Читайте также:} 

\href{https://mrpl.city/news/view/studenty-iz-raznyh-gorodov-ukrainy-otkroyut-dlya-sebya-mariupol}{%
Студенты из разных городов Украины откроют для себя Мариуполь, Роман Катріч, mrpl.city, 28.03.2019}

В феврале 2005 года начался капитальный ремонт старого корпуса. Когда строители
стали снимать наслоения старой штукатурки в фойе главного входа, на стенах и
колоннах вскрылись надписи, как выяснилось вскоре, на немецком языке. Об этом
было сообщено в ректорат. Нужно отдать должное руководству университета:
надписи были со всеми предосторожностями расчищены и сфотографированы. О
находке оповестили местные средства массовой информации. Преподаватели кафедры
иностранных языков перевели их содержание на русский язык. Содержание надписей
оказалось таким: \emph{\enquote{Социализм – путь к миру во всем мире}}, \emph{\enquote{Единство рабочего
класса противостоит реакции}}, \emph{\enquote{Труд – источник всех ценностей}}, \emph{\enquote{С единой
партией за единство, прогресс и демократию}}, \emph{\enquote{Мы хотим заключить союз, который
в сотрудничестве и дружбе приведет нас к миру}}. Одна из них, состоящая всего из
трех слов, была написана крупными буквами и располагалась выше других. В
русском переводе означала – \emph{\enquote{Свободная немецкая молодежь}}. Газета \enquote{Приазовский
рабочий} 23 февраля поместила сообщение о надписи на первой полосе номера и
обратилась к читателям с просьбой сообщить все, что им известно о надписях. И
несколько горожан откликнулось на обращение газеты. Все они сообщили, что
надписи эти относятся к периоду, когда в этом здании находился лагерь немецких
военнопленных. Более того, - поделились некоторыми подробностями.

\textbf{Виктор Петрович Пшёнкин} в 1947 году окончил ремесленное училище. Жил он тогда
на Парковом поселке, а трудиться ему пришлось на строительстве цеха \enquote{Блюминг} и
рельсобалочного цеха на заводе \enquote{Азовсталь}. Пассажирского транспорта в городе
не было, а за опоздание на работу строго наказывали. В то же время
военнопленных из лагеря возили на \enquote{Азовсталь} на грузовых машинах,
приспособленных на скорую руку для перевозки людей. И вот жители поселков
Парковый, Шишманка, Клиновая балка и Виктор Петрович с ними приспособились и, с
разрешения конвоиров, рассаживались в кузовах грузовиков на пол между лавок, на
которых сидели немцы. Таким образом добирались на работу. Продолжалось это
около полутора лет, до 1949 года, пока обитатели лагеря не были отпущены на
родину, в Германию. По воспоминаниям Виктора Петровича, не все немцы из этого
лагеря использовались на стройках \enquote{Азовстали}. Часть из них занималась
восстановлением здания полка.

\textbf{Читайте также:} 

\href{https://archive.org/details/23_03_2019.sergij_burov.mrpl_city.lico_goroda_mariupol}{%
Лицо города Мариуполь, Сергей Буров, mrpl.city, 23.03.2019}

Бывший заведующий кафедрой ковочно-штамповочного производства, доцент, кандидат
технических наук \textbf{Анатолий Дмитриевич Кирицев} с 1944 по 1949 год был студентом
Мариупольского металлургического института. Тогда учебный корпус этого учебного
заведения находился в Ильичевском районе на улице Вузовской. А Анатолий
Дмитриевич жил почти рядом с нынешним ПГТУ. Иногда директор института Тимофей
Михайлович Смирнов подвозил своего заместителя И. П. Пришвина и студента
Кирицева к институту на Вузовской на автомобиле, закрепленном за ним. Но бывали
случаи, когда Тимофей Михайлович со своими спутниками, прежде чем отправиться
на службу, подъезжал к зданию полка, чтобы самому видеть, как ведутся работы.
Несколько раз вместе с директором заходил внутрь здания и Анатолий Дмитриевич.
Ему запомнился идеальный порядок в помещениях, где жили немцы. На стенах были
надписи, как бы мы сказали, по технике безопасности и пропагандистского
характера. Отложился у него в памяти также рисунок углем на стене, выполненный
очень профессионально. Были изображены фигуры людей почти в натуральную
величину. Военнопленные выполняли различные строительные работы внутри здания.
И делали все очень качественно. Они не выглядели истощенными, чувствовалось,
что кормили их достаточно хорошо.

\textbf{Александр Васильевич Панищенко} был в то время комендантом\par учебного корпуса
института на улице Вузовской. Директор института попросил его, наряду с
выполнением своих прямых обязанностей, присматривать за тем, как идут дела с
восстановлением здания полка. Поэтому А. В. Панищенко довелось там бывать
довольно часто. Он также видел на стенах внутренних помещений аккуратные
надписи по-немецки, по его мнению, они были частью пропагандистской работы,
которая велась среди пленных. Запомнились ему и рисунки, выполненные не без
юмора. Александру Васильевичу еще тогда показалось, что основная масса немцев
состояла из молодых людей.

Сопоставляя воспоминания А. В. Панищенко с содержанием надписей, вскрытых при
ремонте, можно с уверенностью сказать: надписи сделали военнопленные –
активисты прокоммунистического союза \enquote{Свободная немецкая молодежь}. Он был
основан 7 марта 1946 года на базе Антифашистского молодежного комитета, как бы
дочернего общественно-политического образования при Национальном комитете
\enquote{Свободная Германия}. Этот комитет действовал с 1943-1944 года в лагерях
немецких военнопленных. И наконец, надписи, обнаруженные при ремонте, сделаны в
промежутке времени с 1946 по 1949 год...

\textbf{Читайте также:} 

\href{https://mrpl.city/blogs/view/zakohani-mariupoltsi-stvoryuyut-rodini-v-vezhi}{%
Закохані маріупольці створюють родини в \enquote{Вежі}, mrpl.city, 28.03.2019}

Осенью 1951 года перед студентами и преподавателями металлургического института
распахнулись двери обновленного корпуса. С тех пор прошло без малого семьдесят
лет. В этих стенах работали такие видные ученые и педагоги, как профессора,
доктора наук {\bfseries Иван Георгиевич Казанцев, Владимир Федорович Зубарев, Дмитрий
Иванович Старченко, Константин Владимирович Багрянский, Борис Анатольевич
Леонтьев, Мирон Яковлевич Меджибожский, Константин Никандрович Соколов, Евгений
Александрович Казачков, Евгений Капустин} и другие ученые. Здесь подготовлено
несколько десятков докторов, сотни кандидатов наук, тысячи инженеров различных
специальностей. Выпускники института, преобразованного в технический
университет, стали руководителями производственных и научных коллективов,
государственными и общественными деятелями, крупными учеными и педагогами,
высококлассными специалистами на предприятиях и в научно-исследователь\hyp{}ских
учреждениях.

Монументальное краснокирпичное здание близ Шишманки за более чем столетнюю
историю именовалось в народе по-разному. \emph{\bfseries \enquote{Епархиалкой}}
- до революции 1917 года, \emph{\bfseries \enquote{приютом}} - в первой
половине 20-х годов прошлого века, почти до 50-х годов – \emph{\bfseries
\enquote{полком}}, затем \emph{\bfseries\enquote{институтом}}, что
соответствовало его использованию и, наконец, в последние годы –
\emph{\bfseries\enquote{старым корпусом}}, не прибавляя иногда законного слова
– \enquote{университета}.

\textbf{Читайте также:} 

\href{https://mrpl.city/news/view/studentam-mariupolya-vernuli-pravo-polnotsenno-pitatsya-v-vuzah}{%
Студентам Мариуполя вернули право полноценно питаться в вузах, Яна Іванова, mrpl.city, 26.12.2018}
