% vim: keymap=russian-jcukenwin
%%beginhead 
 
%%file 19_06_2021.fb.fb_group.story_kiev_ua.1.most_truhanov
%%parent 19_06_2021
 
%%url https://www.facebook.com/groups/story.kiev.ua/posts/1688481258015335/
 
%%author_id fb_group.story_kiev_ua,kuzmenko_petr
%%date 
 
%%tags dnepr,kiev,kiev.truhanov.ostrov
%%title Пешеходный мост
 
%%endhead 
 
\subsection{Пешеходный мост}
\label{sec:19_06_2021.fb.fb_group.story_kiev_ua.1.most_truhanov}
 
\Purl{https://www.facebook.com/groups/story.kiev.ua/posts/1688481258015335/}
\ifcmt
 author_begin
   author_id fb_group.story_kiev_ua,kuzmenko_petr
 author_end
\fi

Набрёл недавно на редкое фото строительства Пешеходного Паркового моста в сети.
И пальцы сами потянулись написать пару строк в любимых Киевских историях.
Сколько душевных и трепетных воспоминаний связано у меня с этим, знаковым для
Киева, сооружением! Не буду сейчас подробно описывать историю проектирования и
создания нашего \enquote{Пешеходного моста}. Многие киевляне её знают, а желающие могут
легко \enquote{нагуглить}. 

Напишу о личном. В первую очередь врезались в память воспоминания из раннего
детства когда родители иногда \enquote{гуляли} меня на Труханов остров по этому
удивительному мосту. Для нашей маленькой патриархальный подольской семьи это
было целое приключение. И подготовка была соответствующей.  Как правило,
происходило сие значимое мероприятие в одну из тёплых весенних или летних
суббот. 

\zzrule

\ii{19_06_2021.fb.fb_group.story_kiev_ua.1.most_truhanov.pic.1}

\zzrule

Ещё в пятницу мама предусмотрительно готовила множество съестных
припасов для пикника на свежем воздухе, а папа доставал с антресоли нашей
коммуналки на Андреевском спуске свою старую армейскую плащ-накидку, служившую
и подстилкой, и \enquote{столом}, и защитой от внезапного дождя. Спать мы ложились
пораньше. Утром, часов в 9, мы выдвигались. До Почтовой площади пешком по
Боричеву току или сразу по Жданова. Последний маршрут несколько не нравился
маме, зато устраивал нас с папой возможностью заглянуть в уже открывшийся
ликёроводочный магазинчик на углу Жданова и Андреевской, где я получал любимый
шоколадный \enquote{Гуливер}, а отец обзаводился \enquote{мерзавчиком} коньячка. 

Иногда мы шли пешком по набережной до пешеходного моста, но чаще ехали одну
остановку на трамвае. По дороге я в очередной раз выслушивал от папы историю
киевского водопровода, бывшей почтовой станции, памятника возвращения Городу
Магдебургского права и другие интересные исторические подольские подробности.
Наконец, мы поднимались на мост. Проходили его не спеша. Мама всё заботилась о
том, чтобы свежий днепровский ветер не продул её драгоценное чадо. А папа
рассказывал о строительстве моста, набережной, об истории Труханова острова. 

Мы глядели вниз, на часто снующие в то время под мостом, теплоходы, баржи,
ракеты и прогулочные катера. Незаметно для меня мы оказывались уже на острове.
Далеко вглубь не заходили. Пройдя береговую полосу и, тогда ещё исправно
функционирующие спортивные площадки, мы останавливались на полянке у протоки
\enquote{Аппендицит} на пикник. 

Тогда приходил звёздный час папиной военной плащ-палатки и маминых припасов,
которые поглощались мной с неимоверной скоростью и завидным детским аппетитом.
Я бегал к \enquote{Аппендициту}, где иногда даже видел раков улепётывающих
боком от меня по песчаному дну в прозрачной воде. Помню, с какой неописуемой
нежностью и любовью смотрели мои пожилые родители друг на друга и на своего
долгожданного отпрыска в лице вашего покорного слуги. Эти мгновения врезались в
мою детскую память и отчётливо встают перед глазами даже сейчас, когда я старше
своих тогдашних родителей и сам давно дважды дедушка. Возвращались мы довольные
и уставшие по любимому пешеходному мосту. Потом всегда на трамвае, до Красной
(ныне Контрактовой) площади, и немножко пешком до Андреевского спуска 2, моего
вечного центра мироздания. 

Позже у меня в жизни было ещё очень много прогулок по Пешеходному мосту, из
которых особо хочу отметить совместные гуляния с любимой одноклассницей,
подругой, невестой, женой, самой прекрасной женщиной в мире Нелли. Были и
походы на зимнюю рыбалку с друзьями на Матвеевский залив.  Путешествия на пляж
и просто погулять на Труханов с дочерью. И вылазки с любимыми таксами в уже
зрелом возрасте. О некоторых из них я уже писал в своих постах в нашей группе. 

Однако, те первые детские путешествия по Пешеходному мосту остались всё - же
самым яркими и эмоциональными. И, если бы какой-нибудь волшебник смог хоть на
несколько мгновений перенести меня в прошлое, я скорее всего выбрал бы именно
эти моменты своего беззаботного детства на любимом Подоле в лучшем Городе
Земли.

\ii{19_06_2021.fb.fb_group.story_kiev_ua.1.most_truhanov.cmt}
