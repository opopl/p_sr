% vim: keymap=russian-jcukenwin
%%beginhead 
 
%%file slova.mova
%%parent slova
 
%%url 
 
%%author 
%%author_id 
%%author_url 
 
%%tags 
%%title 
 
%%endhead 
\chapter{Мова}

Проблема \emph{мовной} инквизиции создана абсолютно искуственно, Андрей
Головачев, strana.ua, 20.01.2021

Шарий - пропагандист многоязычия! Я за 7 лет подписки \emph{мову} выучил!
комментарий, \textbf{Опять Шария достали. Что дальше?} Анатолий Шарий,
youtube.com, 31.05.2021

%%%cit
%%%cit_pic
%%%cit_text
Вы знаете, какая самая чтимая личность в государстве Украина? Вы думаете, что
Бандера? Ошибаетесь, Бандеру знают от силы 10\% украинцев, а для остальных он
вообще иностранец. Козак Мамай? Тоже нет, персонажи с монгольскими именами не
могут быть украинцами. Богдан Хмельницкий? Продался «москалям», и даже на
памятнике у него это написано.  Тарас Шевченко? Тоже тот еще «украинец», творил
в Петербурге на \emph{русской мове}, а его «Кобзарь» хоть и был опубликован на
украинском языке, но на русифицированном («ярыжке») и исключительно на царские
деньги
%%%cit_title
\citTitle{Почему киевский хан Владимир Креститель считается украинским князем, когда он даже славянином не был?},
Исторический Понедельник, zen.yandex.ru, 05.01.2021 
%%%endcit

%%%cit
%%%cit_pic
%%%cit_text
Есть о чем задуматься, примеряя это на нашу реальность \enquote{Чи виживе
Сингапур}. Продолжаю читать эту красивую историю экономического чуда. Автор
пишет, что двумя первыми приоритетами правительства были экономика и одинаковое
отношение ко всем этничным группам. Поскольку до прихода к власти Ли Куан Ю в
Сингапуре \enquote{відбувалися заворушення на етнічному грунтi}, правительство
работало с каждой этничной группой, чтобы \enquote{кожна група усвідомлювала
свою роль в історії розвітку Сингапуру}. \enquote{Ми навчилися на прикладі
Шрі-Ланки, де межетнічні конфлікти разпочалися ісля того, як там скасували
тамільску \emph{мову}, а єдиною офіційною зробили сингальску}. \enquote{Коли
жодної дискримінації нема - це створює відчуття належності}
%%%cit_comment
%%%cit_title
\citTitle{Секрет Сингапура - прагматизм и отсутствие дискриминации}, 
Павел Себастьянович, strana.ua, 13.06.2021
%%%endcit

%%%cit
%%%cit_head
%%%cit_pic
%%%cit_text
Также омбудсмен ревностно следил, как говорили и берегли \emph{мову} украинские
чиновники и выявил таки несколько нарушений, совершил которые, о ужас,
президент Зеленский. Президент посмел (какой негодяй) ответить некоторым
журналистам на пресс-конференциях на русском языке.  Верховной Раде омбудсмен
посвятил целую отдельную главу. \emph{Мовные претензии} предъявлены, естественно,
депутатам оппозиционных партий, который, такие негодяи, использовали русский не
только в Раде, но и в ходе личностного общения. Омбудсмен насчитал около 12
депутатов, которые посмели давать комментарии журналистам на русском языке.
Некоторые \enquote{обнаглевшие} депутаты в своей речи с трибуны, приводили цитаты
русских классиков на русском языке, а оппозиционный депутат Рабинович посмел
усилить свою речь русской песней
%%%cit_comment
%%%cit_title
\citTitle{Speak на мове please}, 
Terra Incognita, zen.yandex.ru, 30.04.2021
%%%endcit
