% vim: keymap=russian-jcukenwin
%%beginhead 
 
%%file 24_08_2021.fb.mitropolit_antonij.1.sobor_sofia_varfolomej
%%parent 24_08_2021
 
%%url https://www.facebook.com/MitropolitAntoniy/posts/369281667900884
 
%%author Митрополит Антоний
%%author_id mitropolit_antonij
%%author_url 
 
%%tags kiev,molitva,nezalezhnist,religia,sofia_sobor,ukraina,vera
%%title «Благословення України» - Свята Софія
 
%%endhead 
 
\subsection{«Благословення України» - Свята Софія}
\label{sec:24_08_2021.fb.mitropolit_antonij.1.sobor_sofia_varfolomej}
 
\Purl{https://www.facebook.com/MitropolitAntoniy/posts/369281667900884}
\ifcmt
 author_begin
   author_id mitropolit_antonij
 author_end
\fi

Тільки-що в Святій Софії відбувся традиційний захід, який кожного року
проходить в День Незалежності - «Благословення України», в якому взяли участь
глави основних конфесій України, а також Президент України Володимир
Зеленський. 

\ii{24_08_2021.fb.mitropolit_antonij.1.sobor_sofia_varfolomej.pic.1}

Хотів би відзначити, що патріарха Варфоломія на ньому не було. Це була позиція
і умова нашої Церкви, про що я говорив раніше в попередніх коментарях. А саме,
що Українська Православна Церква не братиме участь в заході у Святій Софії,
якщо там буде патріарх Варфоломій. Цю позицію ми донесли до відома як
Всеукраїнської Ради Церков і релігійних організацій, що виступила організатором
цього заходу, так і держави. Ми вдячні за те, що нас почули. 

Гості приїжджають і від’їжджають, а нам тут жити. І хочеться жити у
внутрішньому мирі і взаєморозумінні.

\ii{24_08_2021.fb.mitropolit_antonij.1.sobor_sofia_varfolomej.pic.2}
\ii{24_08_2021.fb.mitropolit_antonij.1.sobor_sofia_varfolomej.cmt}

