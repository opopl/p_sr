% vim: keymap=russian-jcukenwin
%%beginhead 
 
%%file link.15_11_2021.youtube.zharkih_katerina.fashizm_ugroza_miru
%%parent 25_11_2021.fb.zharkih_katerina.1.nasilie_v_ukraine.cmt
 
%%url 
 
%%author_id 
%%date 
 
%%tags 
%%title 
 
%%endhead 

\href{https://youtu.be/dojoAovL-6M}{%
Угрожает ли фашизм современному миру?, Екатерина Жарких, youtube, 15.11.2021%
}

\begin{multicols}{2}
Долго молчали, готовили документальный фильм для вас! Никуда не пропали, делаем
еще пару интересных проектов. Поддержите просмотром и лайком, если нравится
такой формат! На зло ютубу.
  
Мы привыкли относится к фашизму, как к явлению из исключительно из прошлого:
как к тому, что никогда уже не вернется. Более семи десятков лет учебники
истории твердят нам о том, что человечество выучило уроки, и ужасы фашизма
надежно погребены под вехами истории. Почему же тогда сегодня все чаще на
улицах европейских городов восстают его внешние проявления – символика, марши,
осквернения могил. 

Почему же правительства в 21-м веке снова прибегают к запретам и глоссариям
дабы сделать эту тему закрытой для широкого обсуждения. При этом из учебников
истории потихоньку затирается вклад огромного количества людей, стран, и
явлений восставших с беспрецедентным героизмом против коричневой чумы.
Внутренний сторож обывателя молчит и при виде исписанных корявой свастикой
братских могил и при виде новых глоссариев, и при виде учебников истории, в
которых портреты настоящих героев заменены на новых героев, угодных
политической конъюнктуре. Обыватель приспосабливается жить дальше, пытаясь
понравится власти, чтобы быть как социально одобряемое большинство. 

Данный фильм о тех явлениях, которые на первый взгляд совершенно неопасны.
Однако, в определенном порядке, используя определенные социальные механизмы, в
определенный момент, они способны свести с ума целые страны и народы. И что
важно, - неизбежно приводят эти народы разгромному поражению и разграблению. 

Так существует ли где-то в современной Европе благодатная почва и условия для
взращивания ростков ненависти, дискриминации и человеконенавистничества?
Благодаря каким механизмам фашизм остается одной из серьезных угроз мирному
сосуществованию людей – смотрите в новом документальном авторском проекте
журналиста и публициста Дениса Жарких.
\end{multicols}

