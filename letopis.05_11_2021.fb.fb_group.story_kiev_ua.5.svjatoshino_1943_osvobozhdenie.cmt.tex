% vim: keymap=russian-jcukenwin
%%beginhead 
 
%%file 05_11_2021.fb.fb_group.story_kiev_ua.5.svjatoshino_1943_osvobozhdenie.cmt
%%parent 05_11_2021.fb.fb_group.story_kiev_ua.5.svjatoshino_1943_osvobozhdenie
 
%%url 
 
%%author_id 
%%date 
 
%%tags 
%%title 
 
%%endhead 
\subsubsection{Коментарі}
\label{sec:05_11_2021.fb.fb_group.story_kiev_ua.5.svjatoshino_1943_osvobozhdenie.cmt}

\begin{itemize} % {
\iusr{Muravov Dmitriy}

Несколько поправок и дополнений:

1. Походу дела основной коммуникацией у немцев стало Васильковское шоссе, по
Житомирскому шоссе отходили незначительные части и велся в основном артминогонь
(см. ЖБД 6-го гв. тк)

2. Надо бы упомянуть про окружение части сил 240-й сд и 136-й сд между
Пущей-Водицей и Святошино в течении ночи 4/5 ноября - несмотря на подавляющее
численное преимущество наши таки умудрились попасть в мешок.

3. И про потери в период 3-5 ноября в стрелковых частях нужно сказать - они
зашкаливали на участках прорыва за 200-300 чел./сутки на дивизию убитыми,
ранеными и пленными - раза в 2-4 выше, чем во время октябрьских боёв за
Лютежский плацдарм.

\begin{itemize} % {
\iusr{Евгения Ерёменко}
\textbf{Muravov Dmitriy} 

Вот и спасибо, Дмитрий, за дополнение! Цель моего сообщения - лишь напомнить о
том, что было. И, разумеется, оно не претендует на хоть какую-либо полноту.

А вот по потерям очень хотелось уточнить, но ничего найти не смогла - увы!
Понятно, что очень много, но цифр в доступе нет

\begin{itemize} % {
\iusr{Muravov Dmitriy}
\textbf{Евгения Ерёменко} 

Есть исследование на эту тему (бои за Киев в период конец сентября - середина
ноября 1943), см. веб-сайт: kiev1943(точка)ho(точка)ua. Обратите внимание на
ссылки и список использованных источников на данном сайте и на выводы по каждой
главе исследования.

\iusr{Евгения Ерёменко}
\textbf{Muravov Dmitriy} Благодарю!
\end{itemize} % }

\iusr{Андрей Кайнаран}

Да, за село из 15-20 дворов на Киевщине за 3 дня положили 600 человек. Это
только те, кого на плиты написали, а сколько их всего там легло, одному Богу
известно.

\iusr{Валентин Тим}

А як пояснити, що кількість знищеної техніки і живої сили противника, яка
зазначалась в нагородних документах сумарно набагато позицій перевищувала
реальні втрати германських підрозділів на конкретному участкового фронта?

\iusr{Muravov Dmitriy}

Ой, Валентине Степановичу, нагородні - то така фігня. Там же було головне, щоб
начальствєнниє ліца та всякі отвєтствєнниє замполіти отримали при читанні того
папірця множєствєнний о.., ну ви зрозуміли. Таку брехню використовували і щоб
отримати незаслужену нагороду (ГРС Чібісов, командарм-38), і щоб все ж хоча б
посмертно нагородити заслужену людину (ГРС, танкіст Бочарніков).

\end{itemize} % }

\iusr{Елена Семененко}

На Чайках до сих пор находят остатки самолёта да и вообще война аукивается до
сих пор

Вот на той же Чайке била запретная зона но у моей однокласница Люби Кукси брат
в 14 полез нашел снаряд и подорвался ето в семидесятих било

\begin{itemize} % {
\iusr{Lena Pavlikova}
\textbf{Елена Семененко} То же случалось и в Ново-Беличанском лесу. Последний раз (очень надеюсь, что последний) погиб одноклассник Шурик, сбежавший с ребятами с последнего звонка, 11-й класс. 1964 г. @igg{fbicon.cry} 
\end{itemize} % }

\iusr{Катя Бекренева}
Хочу поправить) Дачный поселок Святошино стал городом Киевом в 1923-24 году.

\begin{itemize} % {
\iusr{Евгения Ерёменко}
\textbf{Катя Бекренева} 

Катя, так написано в воспоминаниях, на которые я сослалась. Вероятно еще долго
наша местность воспринималась как посёлок, пока не началась застройка
многоэтажками

\begin{itemize} % {
\iusr{Катя Бекренева}
\textbf{Евгения Ерёменко} 

Скажу честно, что моя бабушка, папа, которые прожили там почти всю свою жизнь
не называли его никак), только Лесопилка и Просеки, Северная, Южная улицы. Во
время войны они были в Святошино, прабабушка умерла летом 43-го не дожив до
освобождения.

\iusr{Евгения Ерёменко}
\textbf{Катя Бекренева} вот-вот
\end{itemize} % }

\end{itemize} % }

\iusr{Oksana Shevchuk}
Киев был освобождён к 7 ноября... почему?

\begin{itemize} % {
\iusr{Сергей Хромешкин}
\textbf{Oksana Shevchuk} была поставлена задача Сталиным освободить Киев к годовщине Октябрьской революции...

\begin{itemize} % {
\iusr{Oksana Shevchuk}
\textbf{Сергей Хромешкин}

Именно! К годовщине! А если бы Киев освоили 10 декабря или 15 января то победа была бы хуже?!
Угробили десятки тысяч человек ради того чтобы покрасоваться перед Сталиным.
Для сравнения поинтересуйтесь как " брали" Варшаву.
32 дня стояла армия в Августе ! На берегу тихой Вислы ...

\iusr{Володимир Шуневич}
\textbf{Oksana Shevchuk} 

Оксаночка, Вы приказ читали внимательно? Каждый день промедления давал шансы
противнику укрепить оборону. 7 ноября тут ни причем. Так сложилась оперативная
обстановка. Ни в коем случае не сравнивайте с Варшавой. Тогда наши войска
выдохлись после тяжелейших боев на Сандомирском плацдарме. Мой отец там воевал,
рассказывал. Нельзя недооценивать силу противника.

\iusr{Oksana Shevchuk}
\textbf{Володимир Шуневич}

Я коренная киевлянка в 5 поколении. Одна из моих прабабушек оставалась в Киеве.

Мои коментарии о том что нужно не праздновать а молится за души бессмысленно
убитых, принесених в жертву честолюбию, трусости, некомпетентности руководства.

\iusr{Володимир Шуневич}
\textbf{Oksana Shevchuk} 

Я не буду спорить с женщиной, совершенно некомпетентной в военном деле,
нахватавшейся познаний из сомнительных источников, цель которых была одна -
опорочить командование Советской Армии. Вашей прабабушке я больше верю, чем
брехунам типа Виктора Суворова. Если Вы, Оксаночка, человек верующий, прошу Вас
об одном: не судите... С уважением.

\iusr{Евгения Ерёменко}
\textbf{Oksana Shevchuk} 

простите, Оксана, сплошные непонятки. Разве кто-то призывал праздновать? Хотя
вряд ли киевляне не радовались при освобождении города. Речь о благодарности.

\iusr{Юрий Василенко}
\textbf{Oksana Shevchuk} 

Стояла 32 дня потому, что ждала пока немцы подавят польское восстание!
Англичане и американцы дальней авиацией за тысячи километров летели, сбрасывали
помощь восставшим полякам, а РККА стояла в десяти километрах и пальцем не
пошевелила ! То есть военные действия в этот период тупо не велись!

\iusr{Oksana Shevchuk}
\textbf{Сергей Хромешкин}
Именно так!
Ну а в Украине, в Киеве была ещё одна задача...
Уничтожить как можно больше ненавистных " хохлов "

\iusr{Eugen Zlyvko}
\textbf{Oksana Shevchuk} уважаемая, вы знаете , сколько в Красной Армии было украинцев? Процент скажете?

\iusr{Eugen Zlyvko}
\textbf{Сергей Хромешкин} 

бред полнейший, основная идея освобождения Киева именно к этой дате
-Тегеранская конференция, начатая 28 ноября ,взятие Киева было козырной картой
советского руководства и один из факторов открытия второго фронта

\iusr{Алексей Лисицин}
\textbf{Eugen Zlyvko} Не скажет, потому что смотрит телевизор. А там об этом не рассказывают

\end{itemize} % }

\iusr{Valeriy Levitskiy}

Не нужно все мешать в одну кучу, у людей мололдого поколения в головах сплошной
хаос. То, что советские войска не считались с потерями это известный
факт, признанный давно. Другой вопрос, можно ли было в той ситуации действовать
иначе. То что творится с памятью о войне, это третий вопрос... Каждый плетет, что
ему в голову взбредет, сказал тот, написал этот... В любимая случае это страшная
бойня и трагедия, без всяких но..

\begin{itemize} % {
\iusr{Oksana Shevchuk}
\textbf{Valeriy Levitskiy}
Я не очень " молодое поколение". Мне 60+
Так же как и все изучала " историю" по советским учебникам. ..
И мне очень больно что многие мои соотечественники не способны оценить и ужаснулся содеянному. Потери в советских войсках были 1/ 10 . Особенно "преуспели" на киевском направлении.
\end{itemize} % }

\iusr{Valeriy Levitskiy}
Написанны тысячи книг, можно разобраться. А кому нужен срач, тому факты не требуются

\end{itemize} % }

\iusr{Innessa Assenni}

Отец рассказал, что в районе нынешней школы номер 140 были военные сооружения,
в том числе подземные. Даже его одноклассник провалился в глубокую шахту,
сломал ногу. Если интересно, я поподробнее его расспрошу и запишу

\begin{itemize} % {
\iusr{Евгения Ерёменко}
\textbf{Innessa Assenni} Конечно интересно, Инесса, пока возможно расспросить - сделайте это, пожалуйста, разместите в группе  @igg{fbicon.face.smiling.eyes.smiling} 

\iusr{Innessa Assenni}
\textbf{Евгения Ерёменко} ок
\end{itemize} % }

\iusr{Евгения Ерёменко}

Почему-то не могу войти, чтобы ответить Оксане Шевчук. Ваш вопрос немного
удивил: для старшего поколения не секрет, что трудовые, военные и проч. победы в
СССР были приурочены к 7 ноября как дню революции. Ровно так и в этом случае
было.

\begin{itemize} % {
\iusr{Oksana Shevchuk}
\textbf{Евгения Ерёменко}
Именно так. В этом то и весь ужас!
Ни в одной армии мира не подстраивали военные операции к праздникам!

\begin{itemize} % {
\iusr{Евгения Ерёменко}
\textbf{Oksana Shevchuk} мы о чём? Это не было главным и вообще каким-то в моём тексте

\iusr{Vadym Shvydkiy}
\textbf{Oksana Shevchuk} подстраивали! Ещё как!!! Вон в КНДР до сих пор подстраивают! @igg{fbicon.face.grinning.smiling.eyes} 

\iusr{Eugen Zlyvko}
\textbf{Oksana Shevchuk} хватит нести этот бред  @igg{fbicon.beaming.face.smiling.eyes}{repeat=2} 
\end{itemize} % }

\iusr{Eugen Zlyvko}
\textbf{Евгения Ерёменко} бред полнейший, Киев брали в ноябре для другой даты ,отнюдь не праздничной

\begin{itemize} % {
\iusr{Евгения Ерёменко}
\textbf{Eugen Zlyvko} не стоит так горячиться, уважаемый тёзка. К завтрашнему дню разместила ещё одно сообщение - о Феодоре Пушиной. Прочитайте её письмо с фронта
\end{itemize} % }

\end{itemize} % }

\iusr{Сергей Хромешкин}

\ifcmt
  ig https://scontent-frx5-2.xx.fbcdn.net/v/t39.1997-6/s168x128/93118771_222645645734606_1705715084438798336_n.png?_nc_cat=1&ccb=1-5&_nc_sid=ac3552&_nc_ohc=HRdX71mk81wAX8-fbPt&tn=lCYVFeHcTIAFcAzi&_nc_ht=scontent-frx5-2.xx&oh=b1910353f72b9bb53e5aa5471f018ee8&oe=618B6C7F
  @width 0.1
\fi


\iusr{Vadym Shvydkiy}

И в Петропавловской Борщаговке... Я жил на 50 лет Октября (Леся Курбаса), 205
школа.

\begin{itemize} % {
\iusr{Светлана Аникина}
\textbf{Vadym Shvydkiy} А я до сих пор там живу. И 205-ю заканчивала в 1977 году

\begin{itemize} % {
\iusr{Vadym Shvydkiy}
\textbf{Светлана Аникина} я только в 82 пошёл в школу. А директор была Очкасова Тамара Григорьевна?

\iusr{Светлана Аникина}
\textbf{Vadym Shvydkiy} Да, она самая. Привет, земляк!

\iusr{Vadym Shvydkiy}
\textbf{Светлана Аникина}  @igg{fbicon.grin} 
\end{itemize} % }

\end{itemize} % }

\iusr{Катерина Прокопенко}

Герой Радянського Союзу Н .К. Кругликов загинув в боях і похований в с. Меделівка
Радомишльського району Житомирської обл. Його ім'я носить центральна вулиця
села. Це всього 100 км від Києва пройшов, жаль.

\iusr{Eugen Zlyvko}

Именно взятие Киева в ноябре 43-го стало козырной картой советского руководства
на Тегеранской конференции, перелом в войне стал очевиден, и союзников таки
уговорили открыть второй фронт

\end{itemize} % }
