% vim: keymap=russian-jcukenwin
%%beginhead 
 
%%file 11_11_2020.fb.tanya_naumchuk.1.chtoby_ljubili
%%parent 11_11_2020
 
%%url https://www.facebook.com/manya.naumchuk/posts/784226369102253
%%author 
%%author_id 
%%tags 
%%title 
 
%%endhead 

\subsection{Чтобы любили}
\Purl{https://www.facebook.com/manya.naumchuk/posts/784226369102253}
\Pauthor{Наумчук, Таня}
\index[writers.rus]{Наумчук, Таня!Чтобы любили}

\ifcmt
pic https://scontent-waw1-1.xx.fbcdn.net/v/t1.0-9/124169006_784226325768924_8055760480941972746_o.jpg?_nc_cat=104&ccb=2&_nc_sid=8bfeb9&_nc_ohc=6_kEAg-hGOEAX9XcnY7&_nc_ht=scontent-waw1-1.xx&oh=f567c51f1baca96428e00252a5da0de0&oe=5FDCFA9D
\fi

Друзья, в эти прохладные осенние дни хочу немножко подарить вам тепла... 

\begin{multicols}{2}
\obeycr
Здесь стою, как околдована
Красотой родных полей...
Так свободно и раскованно
Слышу крики журавлей.

Не луну ждала печальную,
Звёзды тоже ни к чему,
Я ждала ЛЮБОВЬ случайную...
Как: зачем и почему!?

Я ждала ее под ивою,
Под берёзой и сосной...
Ее ждала неторопливую
Теплой майскою весной.

У моей ЛЮБВИ есть солнышко,
Плеск ручья и пение птиц.
Соберу ее до донышка,
Всю, до крошек и крупиц.

Здесь, где дышится свободно,
Где ничто не подвластно годам;
Я ЛЮБОВЬ свою охотно
Подарю или отдам.

Я собрала ее у облачка,
У вишни, что роняет цвет,
У одуванчика- ципленочка;
Чтоб охватить весь белый свет.

Я берегу ЛЮБОВЬ могучую,
Чтобы заполнить души и сердца,
И, чтобы Родину мою певучую
Просто любили до конца!

10.11.2020г. (Наумчук Татьяна).
\restorecr
\end{multicols}
