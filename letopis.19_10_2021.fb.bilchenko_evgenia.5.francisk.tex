% vim: keymap=russian-jcukenwin
%%beginhead 
 
%%file 19_10_2021.fb.bilchenko_evgenia.5.francisk
%%parent 19_10_2021
 
%%url https://www.facebook.com/yevzhik/posts/4357324504302626
 
%%author_id bilchenko_evgenia
%%date 
 
%%tags bilchenko_evgenia,hristos,poezia
%%title БЖ. Франциск
 
%%endhead 
 
\subsection{БЖ. Франциск}
\label{sec:19_10_2021.fb.bilchenko_evgenia.5.francisk}
 
\Purl{https://www.facebook.com/yevzhik/posts/4357324504302626}
\ifcmt
 author_begin
   author_id bilchenko_evgenia
 author_end
\fi

БЖ. Франциск

\ifcmt
  ig https://scontent-lga3-1.xx.fbcdn.net/v/t39.30808-6/246671571_4357324590969284_6869804095096356265_n.jpg?_nc_cat=107&ccb=1-5&_nc_sid=8bfeb9&_nc_ohc=k3bqXbUslKMAX-KaA3i&_nc_ht=scontent-lga3-1.xx&oh=8a3926d830cd3af979bac513a068020e&oe=617444EC
  @width 0.4
  %@wrap \parpic[r]
  @wrap \InsertBoxR{0}
\fi

Христос... Который так высоко, что иные из Него идола,
Творят, скрывается между листьев, и ветром Его не выдуло.
Маленький Сын, богатырь Отец, Он идёт по горелой осени,
Как по предчувствию своего января нулевого. Проседью
Проступает первая изморозь. Её расплавляет изморось.
Бабьего лета настойка чачи на золоте и на извести
Разогревает ковидный воздух до кислорода сизого,
До строк Пастернака, транслитерированных из откровений Ассизского.
Мне больно с земли поднимать слова: болдинскими напалмами
Они обжигают руки. Тремор гидазепамовый
Утихомиривается, держа свечу. И горит избушкою
Мое грешное прошлое перед ликом Ксении Петербуржской.
А Ксюша из плюша - сегодня в чёрном. И тропинка - опасней слалома:
Сырою глиною сапоги пропитались от православного
Предвосхищения белизны, но пока ещё пламенеется.
Рыба сошла с византийских амфор и русским проходит нерестом
Из южных морей пароходом в реки - северные, свинцовые.
Багряною старостью покрывается тинейджерское лицо моё.
По опавшей мечте, разгребая сырость, бродят грибы резиновые...
Смерти нет, ибо, смерти нет у месторождения зимнего,
У сибирских руд моих, у волос, у речи моей - то рэперской,
То матерной, то литургийной, - уж такое устройство скреп Его:
Они свободны, как два крыла на ржавеющей прочной проволоке,
И они качают меня с лихвой - ниже обморока, выше облака -
От земли до небес, от листвы до туч, от нижнего и до горнего,
В какое-то странное счастье дурня от ума обращая горе.
И я прижимаю листву к лицу - желто-алую, черно-гравийную:
Так наклоняются над иконой, над дедовой фотографией.
19 октября 2021 г.
Фото: Grigory Belov

\begin{itemize} % {
\iusr{Лиза Высоцкая}
Процарапало по сердцу борозду. И больно, и читать хочется снова и снова. Спасибо! @igg{fbicon.heart.red}
\end{itemize} % }
