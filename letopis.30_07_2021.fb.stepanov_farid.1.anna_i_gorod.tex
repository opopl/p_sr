% vim: keymap=russian-jcukenwin
%%beginhead 
 
%%file 30_07_2021.fb.stepanov_farid.1.anna_i_gorod
%%parent 30_07_2021
 
%%url https://www.facebook.com/StepanovFarid/posts/615626586079157
 
%%author Степанов, Фарид
%%author_id stepanov_farid
%%author_url 
 
%%tags ahmatova_anna,gorod,istoria,kiev,literatura,poezia
%%title АННА И ГОРОД, Или: гуляя аллеями Мариинского парка
 
%%endhead 
 
\subsection{АННА И ГОРОД, Или: гуляя аллеями Мариинского парка}
\label{sec:30_07_2021.fb.stepanov_farid.1.anna_i_gorod}
 
\Purl{https://www.facebook.com/StepanovFarid/posts/615626586079157}
\ifcmt
 author_begin
   author_id stepanov_farid
 author_end
\fi

\index{Ахматова, Анна! Поэтесса, Россия}

АННА И ГОРОД.

Или: гуляя аллеями Мариинского парка.

«И было садов в Городе так много, как ни в одном городе мира. Они раскинулись
повсюду огромными пятнами, с аллеями, каштанами, оврагами, клёнами и липами.

Сады красовались на прекрасных горах, нависших над Днепром, и, уступами
поднимаясь, порою пестря миллионами солнечных пятен, порою в нежных сумерках
царствовал вечный Царский Сад»

Михаил Булгаков «Белая гвардия».

\ifcmt
  tab_begin cols=4
	width 0.22

     pic https://scontent-cdt1-1.xx.fbcdn.net/v/t39.30808-6/228177225_615625096079306_1212848010269233973_n.jpg?_nc_cat=103&ccb=1-5&_nc_sid=8bfeb9&_nc_ohc=c0d9pfPw4GsAX-BD59b&_nc_ht=scontent-cdt1-1.xx&oh=94c0f52eaa617a394db54a6376a0af67&oe=6124F6B4

     pic https://scontent-cdg2-1.xx.fbcdn.net/v/t39.30808-6/225377102_615625139412635_4265387093525301838_n.jpg?_nc_cat=104&ccb=1-5&_nc_sid=8bfeb9&_nc_ohc=lzYTvx1PncUAX88jXFi&_nc_ht=scontent-cdg2-1.xx&oh=521274c35f1ef1fa43d65786fc023a26&oe=61243145

		 pic https://scontent-cdg2-1.xx.fbcdn.net/v/t39.30808-6/225736523_615625176079298_8435891065990950081_n.jpg?_nc_cat=111&ccb=1-5&_nc_sid=8bfeb9&_nc_ohc=0IIeB_8I6nEAX9cJpbN&_nc_ht=scontent-cdg2-1.xx&oh=677d963e25dbcb5958d0ea864dd8df80&oe=61255BCF

		 pic https://scontent-cdt1-1.xx.fbcdn.net/v/t39.30808-6/223663069_615625212745961_2698853242675682897_n.jpg?_nc_cat=109&ccb=1-5&_nc_sid=8bfeb9&_nc_ohc=8UKlG-dzhBcAX-sobGi&_nc_ht=scontent-cdt1-1.xx&oh=201ebb5fb4368114d57b91440e7ab398&oe=6123D32F

  tab_end
\fi

Но сегодня не о Михаиле Афанасьевиче.

Вчера гулял по Мариинскому и на одной из тенистых аллей набрёл на памятник Анне
Ахматовой. Солнечные лучи играли светом на гордом профиле, создавая иллюзия
ожившего камня. Киев в жизни поэтессы сыграл важную роль. 

\ifcmt
  tab_begin cols=3

     pic https://scontent-cdg2-1.xx.fbcdn.net/v/t39.30808-6/228170005_615625309412618_6945104696007014559_n.jpg?_nc_cat=108&ccb=1-5&_nc_sid=8bfeb9&_nc_ohc=ysQtbDtO_80AX90yfXq&_nc_ht=scontent-cdg2-1.xx&oh=14689ac0336aa3e7481dc626b62a39bf&oe=61246800

		 pic https://scontent-cdt1-1.xx.fbcdn.net/v/t39.30808-6/225358266_615625356079280_4507827041032021034_n.jpg?_nc_cat=103&ccb=1-5&_nc_sid=8bfeb9&_nc_ohc=hjV9Jr-pAUQAX-7uEos&_nc_ht=scontent-cdt1-1.xx&oh=d9c5f68992b5dc5bcd22a9aedc88d4ef&oe=6124DB5E

		 pic https://scontent-cdg2-1.xx.fbcdn.net/v/t39.30808-6/223026751_615625416079274_8707090283660645764_n.jpg?_nc_cat=102&ccb=1-5&_nc_sid=8bfeb9&_nc_ohc=rGhSt5wna1sAX-SvYyC&_nc_ht=scontent-cdg2-1.xx&oh=7f79ec9b9fe87c1d181dd5f64f8ff055&oe=6124CDAA

  tab_end
\fi

В Городе, как называл
его Михаил Булгаков, с Анной Горенко, будущей Ахматовой, произошло много важных
событий. В Киеве родилась её младшая сестра Ия, здесь Анна училась, сперва в
Фундуклеевской гимназии, затем на Киевских Высших женских курсах при Киевском
университете им. Святого Владимира. В Городе она пишет свои одни из первых
стихотворений. В Киеве она приняла предложение Николая Гумилёва, здесь же с ним
и обвенчалась в Николаевской церкви на Никольской Слободке.

\ifcmt
  tab_begin cols=3

     pic https://scontent-cdg2-1.xx.fbcdn.net/v/t39.30808-6/224850777_615625469412602_1182254445832317004_n.jpg?_nc_cat=100&ccb=1-5&_nc_sid=8bfeb9&_nc_ohc=I7H3zqJWKOEAX-qINAm&_nc_ht=scontent-cdg2-1.xx&oh=0b2fc9d24558fefc23b7d70c63ef6809&oe=6123A397

     pic https://scontent-cdg2-1.xx.fbcdn.net/v/t39.30808-6/228510291_615625506079265_6285913273063812311_n.jpg?_nc_cat=102&ccb=1-5&_nc_sid=8bfeb9&_nc_ohc=pFP_7fXDt1cAX_Mcz9j&_nc_ht=scontent-cdg2-1.xx&oh=5e03cc5e82a3dcfa7728e4fe221846d3&oe=612415EE

		 pic https://scontent-cdt1-1.xx.fbcdn.net/v/t39.30808-6/224902385_615625552745927_1982653685987005955_n.jpg?_nc_cat=109&ccb=1-5&_nc_sid=8bfeb9&_nc_ohc=cAQz8INwtHAAX9vPi_g&_nc_ht=scontent-cdt1-1.xx&oh=0517056f50c8f01b6493f24a2ec4a9d9&oe=6123AC48

  tab_end
\fi

Анна Ахматова любила и не любила Город. «… я не любила дореволюционного Киева.
Город вульгарных женщин. Там ведь много было богачей и сахарозаводчиков. Они
тысячи бросали на последние моды, они и их жёны… Моя семипудовая кузина, ожидая
примерки нового платья в приёмной у знаменитого портного Швейцера, целовала
образок Николая Угодника: «Сделай, чтобы хорошо сидело»».

\ifcmt
  tab_begin cols=3

     pic https://scontent-cdg2-1.xx.fbcdn.net/v/t39.30808-6/226396328_615625609412588_6377510306606453101_n.jpg?_nc_cat=107&ccb=1-5&_nc_sid=8bfeb9&_nc_ohc=vxB_ucb6MQwAX91zKT9&tn=lCYVFeHcTIAFcAzi&_nc_ht=scontent-cdg2-1.xx&oh=f97f5e7b3208fc34a9437988e83fb970&oe=61250496

     pic https://scontent-cdg2-1.xx.fbcdn.net/v/t39.30808-6/223408162_615625656079250_2644404582191768071_n.jpg?_nc_cat=102&ccb=1-5&_nc_sid=8bfeb9&_nc_ohc=Q4EEstv4NqAAX_2cyJa&_nc_ht=scontent-cdg2-1.xx&oh=68df5da54972f5bebb0ee00f404be8e9&oe=6124FD15

		 pic https://scontent-cdt1-1.xx.fbcdn.net/v/t39.30808-6/224592594_615625706079245_3578589050242477982_n.jpg?_nc_cat=106&ccb=1-5&_nc_sid=8bfeb9&_nc_ohc=6XHxh6BdEwgAX9XhPtL&_nc_ht=scontent-cdt1-1.xx&oh=321f01071437729bdef0e8adfb50ce92&oe=6123A54E

  tab_end
\fi

И только много позже поэтесса напишет:

\obeycr
Древний город словно вымер,
Странен мой приезд
Над рекой своей Владимир
Поднял чёрный крест.
Липы шумные и вязы
По садам темны,
Звёзд иглистые алмазы
К Богу взнесены.
Путь мой жертвенный и славный
Здесь окончу я,
И со мной лишь ты, мне равный,
Да любовь моя.
\restorecr
1914

Впервые Анна Горенко попала в Киев в пятилетнем возрасте. И бонна с детьми
гуляли Царским Садом. Именно в Царском Саду с Анной произошло событие, которое
девочка запомнила на всю жизнь. На одной из аллей она нашла булавку в виде
лиры. «Это значит, ты будешь поэтом» - сказала ей бонна. Кто знает, возможно,
юная Анна на одной из аллей Царского сада, встречала, младшего её на два года
мальчика, с восхищением смотрящего на любимый Город. Малчика звали Миша
Булгаков.

Будет настроение и время, напишу отдельно о Анне и Городе. В их истории есть о
чём рассказать. Ахматова ещё не раз возвращалась в Киев. Сохранилось много мест
где она жила и бывала. Будет интересно.

А пока, хочу поделиться с вами красотой летнего Мариинского. И посмотреть на
Киев глазами поэтессы. Каким она могла бы его сегодня увидеть.

\obeycr
И в Киевском храме Премудрости Бога,
Припав к солее, я тебе поклялась,
Что будет моею твоя дорога,
Где бы она ни вилась.
То слышали ангелы золотые
И в белом гробу Ярослав.
Как голуби, вьются слова простые
И ныне у солнечных глав.
И если слабею, мне сниться икона,
И девять ступенек на ней.
И в голосе грозном софийского звона
Мне слышится голос тревоги твоей.
\restorecr

1915
