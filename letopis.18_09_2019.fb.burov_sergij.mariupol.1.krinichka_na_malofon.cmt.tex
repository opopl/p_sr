% vim: keymap=russian-jcukenwin
%%beginhead 
 
%%file 18_09_2019.fb.burov_sergij.mariupol.1.krinichka_na_malofon.cmt
%%parent 18_09_2019.fb.burov_sergij.mariupol.1.krinichka_na_malofon
 
%%url 
 
%%author_id 
%%date 
 
%%tags 
%%title 
 
%%endhead 

\qqSecCmt

\iusr{Александр Железняк}

За криничкой дом Мултыхов - моих дальних родственников!!!

\begin{itemize} % {
\iusr{Сергей Буров}
К сожалению, дома Мултыхов нет. Он разрушен.
\end{itemize} % }

\iusr{Agnessa Fidelman}
❤️❤️😍😍


\iusr{Agnessa Fidelman}
❤️❤️💖💝


\iusr{Даниил Чкония}

По этой лестнице сбегали вниз к вокзалу, там - через пути - к началу - от бани
- городского пляжа, потом разбирались: здесь оставаться или идти подальше от
слободки, скажем, к азовстальской водной станции...

\begin{itemize} % {
\iusr{Сергей Буров}

Даня, эта лестница от старого базара. Криничка на Малофонтанной.

\iusr{Alex Rud}

Не близко бежать до вокзала.

\iusr{Даниил Чкония}

Да, понял! Лестницы похожи!

\iusr{Borys Milyavskyy}

Да какая х. й разница! Наши они лестницы, ЖДАНОВСКИЕ!!!!!! Это вам, уважаемые, не
берлинские лестницы или всякие там, потёмкинские! Это родные, ЖДАНОВСКИЕ! Недалеко
от этой лестницы в 1965-м зимой подрались с гаванскими пацанами, мент увидел
засвистел. Мы в отрыв, а она скользкая (ещё не наступило глобальное
потепление!). Раз по пять упали, пока до верха дотянули. Я, лично, тогда колено
травмировал. А та лестница, Даня, которая на Слободку ведёт - она несколько
другая.... Но, тоже наша!

\iusr{Алехандро Гагавович}

\ifcmt
  igc https://scontent-fra3-1.xx.fbcdn.net/v/t39.1997-6/64422460_1231657023672314_153572813435830272_n.png?stp=dst-png_s168x128&_nc_cat=1&ccb=1-7&_nc_sid=ac3552&_nc_ohc=7bJGTN8mfYMAX_t_Imf&_nc_ht=scontent-fra3-1.xx&oh=00_AfDdyzPwjSSyzqnBTNpzN1qCScl7HqnRj7phC9WwQosonw&oe=642EA2C9
	@width 0.1
\fi

\end{itemize} % }

\iusr{Александр Железняк}

Эта лестница не к вокзалу. а на Гавань Мы поднимались по этой лестнице к цирку
Швпито который всегда стоял на месте ДОСААФа

\iusr{Nina Kiseleva}

Всё верно, это спуск на Гавань.

\iusr{Vladimir Chertushkin}

Я по этой лестнице в детстве поднимался к музыкальной школе № 1 (наверху надо
было влево повернуть). А вода в колодце была изумительной!

\iusr{Александр Железняк}

А директором была Соболевская

\iusr{Borys Milyavskyy}

Зато у нас, сука, Мялковская.

\iusr{Borys Milyavskyy}

Там, в начале лестницы столб и, по-моему, лампа на самом верху. Так вот, этот стол
для ремонтников-электриков был спорным по причине раздела границ Жовтневого и
Приморского районов именно на его реографическом стоянии. Т. е., уже не и ещё не!
Мой друг, покойный, Валя Аввакумов, классный ждановский боксёр и работник
электросети, часто решал эти вопросы с коллегами и, представьте, доказывал им
правильность географической принадлежности оного столба. Так, что место, Серёжа, ты
продемонстрировал, историческое и архиважное. Как примерно - ЦЕНТР ЕВРОПЫ или
ГРАНИЦА АЗИИ И ЕВРОПЫ. Благоларочка от меня и пожелания здоровья на долгия леты.
