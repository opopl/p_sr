% vim: keymap=russian-jcukenwin
%%beginhead 
 
%%file 20_09_2021.fb.skubchenko_aleksandr.1.pravo_jazyk_ukraina
%%parent 20_09_2021
 
%%url https://www.facebook.com/skubchenco/posts/4143481615751264
 
%%author_id skubchenko_aleksandr
%%date 
 
%%tags arestovich_aleksei,jazyk,konstitucia,obrazovanie,ukraina,zapret
%%title Право на язык — это не только право на нём разговаривать. Это право получать воспитание, образование и информацию на нём
 
%%endhead 
 
\subsection{Право на язык — это не только право на нём разговаривать. Это право получать воспитание, образование и информацию на нём}
\label{sec:20_09_2021.fb.skubchenko_aleksandr.1.pravo_jazyk_ukraina}
 
\Purl{https://www.facebook.com/skubchenco/posts/4143481615751264}
\ifcmt
 author_begin
   author_id skubchenko_aleksandr
 author_end
\fi

Представитель Зеленского в ТКГ Арестович обратился к жителям ОРДЛО: "Помните,
что никто вам не запрещает говорить на русском языке и в соответствии с 10
статьей Конституции можно пользоваться русским и другими языками так, как вам
хочется". Сам Зеленский тоже любит говорить, что в Украине "никто не запрещает
говорить на русском языке". Но так ли это? 

\textbf{Алексей Арестович}, а разве речь о "говорить", когда говорят о запрете русского
языка в Украине? Я, как и любой из 250 миллионов носителей языка, на русском
языке можем СВОБОДНО говорить не только в Украине, но и в любой точке Земного
шара: просто как иностранцы, в любой стране. И нам не нужна для этого статья 10
Конституции Украины.

И никто — ни государство, ни закон, ни один президент в мире не в состоянии
запретить "говорить" на родном языке кому бы то ни было и где бы то ни было.
Ведь право на язык каждому на планете Земля предоставляется не законами, а
рождением, и прекращается только смертью, а не желанием власти — это
неотъемлемое право человека, как и право на жизнь. 

Но разве дома, в Украине, мы все иностранцы??? Разве статья 10 Конституции
Украины о праве "говорить" на русском языке? Нет! Речь в Конституции о праве на
свободное использование русского языка во всех сферах общественной жизни! Но в
Украине и я, и даже президент, на русском языке можем ТОЛЬКО говорить и только
в быту. Как иностранцы, но только в своём государстве. 

Однако право на язык — это не только право на нём разговаривать. Это право
получать воспитание, образование и информацию на нём. Это всё в Украине
запрещёно: русский язык запрещён во всех сферах общественной жизни! На нём
остаётся только "говорить", что запретить законами просто невозможно, иначе
запретили бы и это. 

Уже лично Зеленский запретил среднее образование на русском языке, он же
запретил свободное использование русского языка в сфере обслуживания, с 16 июля
уже самому Зеленскому запрещено снимать фильмы и сериалы на русском языке, а со
следующего года за русский язык начнут штрафовать! Скоро вы на мове будете
говорить, что "в Україні ніхто не забороняє розмовляти..." 

В Конституции написано, что государство гарантирует применение государственного
языка во всех сферах жизни? Да, но разве в Конституции написано, что ТОЛЬКО
государственного? Разве свободное использование русского языка во всех сферах
жизни мешает применению госязыка? Разве украинский и русский не могут
одновременно свободно использоваться?

Дети должны изучать государственный язык? Да, но изучать и обучаться — это
разные процессы! Детям гораздо легче изучать госязык, обучаясь при этом на
своём родном языке! Но вы детей принудили обучаться, нарушив при этом их права.
Напомню, что, например, Конституция УССР гарантировала право обучения на родном
языке каждому ребёнку (даже не гражданину, а ребёнку!) в республике. Но Украина
стала тюрьмой народов.

Так о чём вы говорите, Алексей? О чём говорит Зеленский? Вы реально считаете,
что ваша ложь про "в Украине никто не запрещает говорить на русском", кого-то
впечатляет? Вместо распространения лжи на весь мир, Зеленскому стоит исполнить
своё предвыборное обещание, и прекратить запрет и дискриминацию русского языка
в Украине — родного языка миллионов граждан Украины!

\ii{20_09_2021.fb.skubchenko_aleksandr.1.pravo_jazyk_ukraina.cmt}
