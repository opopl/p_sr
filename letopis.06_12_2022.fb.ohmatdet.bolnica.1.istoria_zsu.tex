% vim: keymap=russian-jcukenwin
%%beginhead 
 
%%file 06_12_2022.fb.ohmatdet.bolnica.1.istoria_zsu
%%parent 06_12_2022
 
%%url https://www.facebook.com/ndslohmatdyt/posts/pfbid0AaqZxVHdHtjJRVfRh3ZbvShR3vqpUCYWK92XCgp1yHSudanHgptFD6Ve53yvaB3hl
 
%%author_id ohmatdet.bolnica
%%date 
 
%%tags 
%%title До Дня Збройних сил України, ділимося історією однієї з співробітниць Охматдиту
 
%%endhead 
 
\subsection{До Дня Збройних сил України, ділимося історією однієї з співробітниць Охматдиту}
\label{sec:06_12_2022.fb.ohmatdet.bolnica.1.istoria_zsu}
 
\Purl{https://www.facebook.com/ndslohmatdyt/posts/pfbid0AaqZxVHdHtjJRVfRh3ZbvShR3vqpUCYWK92XCgp1yHSudanHgptFD6Ve53yvaB3hl}
\ifcmt
 author_begin
   author_id ohmatdet.bolnica
 author_end
\fi

До Дня Збройних сил України, ділимося історією однієї з співробітниць
Охматдиту, яка долучилась до лав ЗСУ 🇺🇦

Від початку повномасштабного вторгнення чимало наших співробітників з різних
відділень та підрозділів лікарні стали на захист нашої держави.

Ольга — одна з них.❤️

% 1,2,3
\ii{06_12_2022.fb.ohmatdet.bolnica.1.istoria_zsu.pic.1}

Ольга Севрук «Колібрі» — старший бойовий медик 206 батальйон територіальної
оборони ЗСУ. Вона працювала в онкологічному відділенні Охматдиту дитячим
нейропсихологом по відновленню втрачених функцій після операцій на головному
мозку. З початку повномасштабного вторгнення росії, лікарня Охматдит приймала
поранених: як дітей, так і дорослих. Ольга працювала з дітьми та їх батьками, а
також з пораненими і медперсоналом.🏥

% 4,5,6
\ii{06_12_2022.fb.ohmatdet.bolnica.1.istoria_zsu.pic.2}

Ольга зрозуміла, що на передовій не вистачає медиків та почала тренуватися.
Закінчила повний курс тактичної медицини «від поранення до шпиталю», включно з
польовою хірургією і долучилася до спротиву. Дома на неї чекають чоловік та
діти. Ольга розповідає, що рідні непросто сприйняли її рішення вступити до
війська, але зараз вони нею пишаються, підтримують і розуміють важливість
служби. 🇺🇦

% 7,8,9
\ii{06_12_2022.fb.ohmatdet.bolnica.1.istoria_zsu.pic.3}

«Своїм рідним я сказала, що армія це те місце, де я можу максимально допомогти
і лишилася у війську. Як бойовий медик я турбуюся про всіх моїх «котиків». Це
означає слідкувати за станом їх здоров’я, працювати на позиціях, виїжджати на
евакуацію. Я не той боєць, який буде сидіти на гальорці. Я прийшла на війну.
Тому хочу і буду на нулі, надавати там першу медичну допомогу. Моя мета –
встигнути прибути якнайшвидше, щоб врятувати життя. Бо врешті, це можливість
хлопцям повернутися до своїх дітей, дружин, батьків. А тим, хто ще немає
родини, дати таку можливість: одружитися, народити дітей і продовжити наш
козацький рід»,— ділиться Ольга.🙏🏻

% 10,11,12
\ii{06_12_2022.fb.ohmatdet.bolnica.1.istoria_zsu.pic.4}

«Україна – понад усе! І вся вона в моєму серці, якому боляче за її подальшу
долю. Ми робимо важливу справу, кожен на своєму фронті: і побратими, і
волонтери, яким щира вдячність за допомогу. Все буде Україна! Ми обов’язково
переможемо!»,— додає Колібрі.❤️

Слава мужнім синам та донькам України, котрі стали на її захист! Слава
Україні🇺🇦

\ii{06_12_2022.fb.ohmatdet.bolnica.1.istoria_zsu.pic.5}

\ii{06_12_2022.fb.ohmatdet.bolnica.1.istoria_zsu.orig}
\ii{06_12_2022.fb.ohmatdet.bolnica.1.istoria_zsu.cmtx}
