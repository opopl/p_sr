% vim: keymap=russian-jcukenwin
%%beginhead 
 
%%file 13_10_2021.fb.karpec_aleksandr.1.neadekvat_ze_vlast
%%parent 13_10_2021
 
%%url https://www.facebook.com/permalink.php?story_fbid=2728031724154823&id=100008440657762
 
%%author_id karpec_aleksandr
%%date 
 
%%tags adekvatnost,neadekvatnost,obschestvo,strana,transport,ukraina,vlast,zelenskii_vladimir
%%title Зе-власть все более впадает в неадекват
 
%%endhead 
 
\subsection{Зе-власть все более впадает в неадекват}
\label{sec:13_10_2021.fb.karpec_aleksandr.1.neadekvat_ze_vlast}
 
\Purl{https://www.facebook.com/permalink.php?story_fbid=2728031724154823&id=100008440657762}
\ifcmt
 author_begin
   author_id karpec_aleksandr
 author_end
\fi

Зе-власть все более впадает в неадекват, продуцируя «решения», которые не
налезают на голову и остальные части тела! Одним из таких «решений» является
введение с 21 октября карантинных ограничений в виде необходимости
предоставлять при посадке в поезда, самолеты и автобусы документы,
свидетельствующие о наличии прививок от ковида, отрицательных тестов и так
далее.

\ifcmt
  pic https://external-frt3-2.xx.fbcdn.net/safe_image.php?d=AQFwlYPSvHk6SbQg&w=500&h=261&url=https%3A%2F%2Fenovosty.com%2Fwp-content%2Fuploads%2F2021%2F10%2Fkollaps_transporta_ed.jpg&cfs=1&ext=jpg&_nc_oe=6ee7f&_nc_sid=06c271&ccb=3-5&_nc_hash=AQEovtmFIsh1p5nb
  @width 0.8
\fi

\href{https://enovosty.com/society/full/1310-kovidnyj-kollaps-transporta-k-chemu-privedut-neadekvatnye-resheniya-vlasti}{%
Ковидный коллапс транспорта: к чему приведут неадекватные решения власти, enovosty.com, 13.10.2021%
}

Все это может привести к коллапсу на транспорте и парализовать страну.

Впрочем, сложившаяся на транспорте ситуация такова, а решение Кабмина выписано
так, что из-за внешней неадекватности торчат уши коррупционных интересов.
Ограничения фактически не коснутся железной дороги и авиации, но сильнейшим
образом ударят по автоперевозчикам и пассажирам, которые должны будут
откупаться от полиции, чтобы ехать дальше. Таким образом, предпринимаемые
Зе-властью меры очень хорошо иллюстрируются украинской поговоркой: «Хоч дурне,
але лукаве».

Итак, Кабмин принял решение о том, что с 21 октября нынешнего года любые
перевозки между областями будут разрешены только для тех пассажиров, которые
могут предоставить один из указанных далее документов:

— документа, подтверждающего получение полного курса вакцинации (или одной дозы
для регионов «желтого» зоны);

— международного, внутреннего сертификата или иностранного сертификата,
подтверждающего вакцинацию от Covid-19 одной дозой одноразовой вакцины или
двумя дозами двухдозной вакцины, разрешенных ВОЗ, а также свидетельство о
выздооровлении от коронавируса или отрицательный ПЦР-тест, активированный через
«Дию»;

— отрицательного результата тестирования на Covid-19 методом ПЦР или
экспресс-теста на определение антигена коронавируса SARS-CoV-2, действующий 72
часа.

Ограничения вводятся для в регионов с желтым, оранжевым или красным уровнем
эпидемиологической опасности. Фактически это для всей Украины, поскольку желтый
уровень является сейчас базовым для страны.

Предоставление указанных документов будет необходимым при пользовании любым
транспортом начиная от маршруток и автобусов и заканчивая поездами и
самолетами. Такая же «обязаловка» вводится для персонала транспортных компаний,
то есть для водителей автобусов, машинистов и проводников в поездах, пилотах и
стюардессах в самолетах.

Обращает внимание тот факт, что ограничения касаются не только регулярных, но и
нерегулярных перевозок пассажиров автомобильным транспортом. Иными словами, эти
ограничения распространяются также на такси, если они выполняют рейсы между
областями, «бла-бла-кары», нерегулярные автобусные перевозки и так далее.

Тот факт, что отечественная власть все делает, как говорится, «через одно
место», ярко иллюстрируется следующим обстоятельством.

Накануне Минздрав ввел приказ об обязательной вакцинации некоторых категорий
работников.

Согласно приказу, от обязательной вакцинации освобождаются те, кто имеет
медицинские противопоказания по Перечню Минздрава №595 от 16 сентября 2011
года.

Согласно этого перечня, абсолютными противопоказаниями для введения вакцин
являются следующие:

— наличие в анамнезе анафилактической реакции на предыдущую дозу вакцины;

— беременность;

— иммунодефицит;

— острые заболевания с повышением температуры выше 38 градусов;

— при наличии злокачественных образований прописана вакцинация с
предостережениями (прививка инактивироваными вакцинами через 3 месяца после
завершения курса химиотерапии);

— при трансплантации органов — не ране чем через 2 месяца;

— также имеются предостережения при вакцинации людей с проблемами сворачивания
крови, нарушением обмена веществ и т.д.

Таким образом, получается, что при вакцинировании некоторых категорий
работников действуют освобождения от вакцинации в связи с медицинскими
противопоказаниями, но при предполагаемой поголовной вакцинации транспортников
и пассажиров противопоказания уже не действуют. А ведь количество случаев
«побочки», вплоть до тяжелых и даже летальных форм, ширится с каждым днем.
Например, на днях Швеция, Дания, Финляндия и Исландия остановили использование
вакцины Модерна для вакцинируемых возрастом до 30 лет, потому что возникают
тяжелые побочные последствия в виде различных воспалений миокарда, то есть
сердечной мышцы… Ни много ни мало!

Складывается впечатление, что сочинители решения Кабмина о «поголовной
вакцинации транспорта» настолько халатно отнеслись к делу, что банально
«прощелкали» то обстоятельство, что вакцинация транспортников и пассажиров
может быть чреватой для многих из них крайне серьезными последствиями. Но если
это обстоятельство не было учтено умышленно, то тогда следует вести речь не о
преступной халатности, а о преступном умысле, а это уже другая, куда более
тяжелая статья Уголовного кодекса.

Сказать, что подобные «заезды» Зе-власти вызвали резкую реакцию как
транспортников, так и пассажиров, — это ничего не сказать.

Прежде всего, обращает внимание, что указанные меры вводятся через 9 дней после
их объявления 12 октября, когда многие, взяв уже билеты на тот или иной вид
транспорта, просто физически не успевают сделать две дозы вакцины, а наличие
одной дозы допускается только для регионов желтой зоны. Сдача теста обойдется
от 400 до 1000 гривен, и этот тест придется через три дня подтверждать,
увеличивая расходы на тестирование. Сдача билетов тоже ведет к финансовым
потерям, тем более что в «Укрзалізниці» уже заявили, что ситуация с ковидными
документами не является форс-мажором, а потому деньги возвращать никто не
собирается, и придется возвращать билеты и деньги на общих основаниях, то есть
с большими потерями. К тому же, «Укрзалізниця» может и сама начать отмену тех
поездов, которые будут слабо заполняться из-за обсуждаемых ограничений.

Кроме того, у самой «Укрзалізниці» могут возникнуть проблемы с персоналом,
подавляющее большинство которого поголовно не привито и, по слухам, делать это
не собирается. Причины более чем настороженного отношения людей к так
называемой вакцинации, причем не только на железной дороге, уже неоднократно
приходилось рассматривать.

Впрочем, норматив Кабмина выписан так, что железной дороги они вроде как и не
особо касается. Очевидно, это сделано сознательно, потому что существует
опасность полностью парализовать страну, оставив ее без железнодорожного
сообщения. Не говоря уже о том, что выполнить выдвигаемые условия до 21 октября
просто нереально.

В незначительной степени устанавливаемые ограничения коснутся авиаперевозок.
Работников авиакомпаний уже давно принудили сделать прививки, а аэропортах
отлажена система проверок ковид-сертификатов и тестов на международных рейсах,
где документы по ковиду требуются уже давно. Другое дело, что из-за слабой
заполняемости воздушных судов по причине малого количества вакцинированных,
авиакомпаниям придется сворачивать внутренние перелеты.

Хуже всего придется наиболее массовым автобусным перевозчикам и миллионам
пассажиров, которые регулярно пользуются их услугами.

Когда появилась информация о принятии Кабмином указанного решения по
«ковид-паспортизации» транспорта, автоперевозчики, прежде всего маршрутчики,
даже намеревались гнать колонны своих «железных коней» со всей страны в Киев
под Кабмин и офис Зеленского, но пока решили немного «притормозить» и
выработать стратегию и тактику борьбы.

Ведь возникает банальный вопрос: как и на каком основании водитель автобуса
будет препятствовать доступу пассажиров без соответствующих ковид-документов в
свое транспортное средство? Не говоря уже о том, что водителям надо будет
отличать фальшивые документы от настоящих, вообще, отличать одни медицинские
документы от других. Такой бред могли выдумать только персонажи, которые
абсолютно далеки от внедрения своих изощрений! Ведь дело будет каждый раз
доходить до мордобоя!

Ситуация осложняется тем, что буквально с 1 октября для автоперевозчиков
вдесятеро повысили штрафы путем внесения соответствующих изменений в закон об
автотранспорте. Эти изменения предполагают штрафы в 17 тысяч на водителя и от
34 тысяч до 170 тысяч на руководство компании-перевозчика при отсутствии
разрешительной документации или нарушении правил перевозок. Ковидные
ограничения можно привязать к таким нарушениям, и откроется непаханое поле не
только для штрафования, но также для коррупции на дорогах, причем вполне
возможно, что именно для этого все и затевалось.

Все это может усложнить и без того непростую ситуацию на автотранспорте.
Руководители автопредприятий жалуются на нехватку водителей, которая доходит до
20\%. Большинство водителей прививаться не намерены, и скорее уволятся.

Большинство же тех, что прививку получил, уже работают за границей, хотя бы
даже в соседней Польше, где за ту же работу платят совершенно другие деньги.

Если к резкому повышению штрафов и существенному подорожанию автогаза еще
добавятся ковидные ограничения, автотранспорт может просто остановиться, и
жизнь в стране окажется парализованной. Зато вместо легальных перевозчиков
могут еще сильнее активизироваться нелегалы, что приведет как к падению
поступлений в бюджет, так и росту правонарушений и резкому падению уровня
безопасности движения со всеми вытекающими последствиями для жизни и здоровья
миллионов граждан.

Словом, неадекватная публика во власти порождает столь же неадекватные решения,
результатом чего может стать паралич страны. И с этими «профессионалами» во
власти следует что-то срочно делать, иначе социальная ситуация может
окончательно выйти из-под контроля.

Читать в "Экономических Новостях"

\ii{13_10_2021.fb.karpec_aleksandr.1.neadekvat_ze_vlast.cmt}
