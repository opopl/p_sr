% vim: keymap=russian-jcukenwin
%%beginhead 
 
%%file 10_04_2021.fb.myronenko_petro.1.matematika_svoboda
%%parent 10_04_2021
 
%%url https://www.facebook.com/petro.myronenko/posts/3104085533152420
 
%%author 
%%author_id 
%%author_url 
 
%%tags 
%%title 
 
%%endhead 

\subsection{Свобода через математику}
\label{sec:10_04_2021.fb.myronenko_petro.1.matematika_svoboda}
\Purl{https://www.facebook.com/petro.myronenko/posts/3104085533152420}

В сільській школі на Донеччині, де я навчався, вчителем математики була Фрося
Андріївна, яка вчила доказувати теореми різними способами. Її учні вигравали
математичні конкурси в районі і області (серед яких був і я). Фрося Андріївна
навчила мене бути вільним через математику. 

Але більшість вчителів вчили і продовжуть вчити жити в казармі. Сьогодні
проблема управлінського апарату у тому, що чиновники з шорами на очах не
навчені логічному мисленню, що потребує сучасний світ.

Значення логіки в освіті і вихованні вільної людини розуміли ще з давніх часів.
Але логіка не потрібна тоталітарному режиму, із якого ми ще не вийшли і
продовжуємо жити в ньому, де підготувати вільну особистість майже не можливо.

Розв'язуйте задачі різними способами, бо в житті це дуже знадобиться.

З повагою П.В. Мироненко
