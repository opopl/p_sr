% vim: keymap=russian-jcukenwin
%%beginhead 
 
%%file 04_12_2020.news.ua.strana.sibircev_aleksandr.1.covid_odessa
%%parent 04_12_2020
 
%%url https://odessa.strana.ua/304833-v-odesskoj-bolnitse-objasnili-pochemu-v-koridorakh-ne-ubirajut-tela-umershikh-ot-koronavirusa.html
 
%%author Сибирцев, Александр
%%author_id sibircev_aleksandr
%%author_url 
 
%%tags 
%%title "Тела не успели вывезти, не было свободных рук". Подробности скандала с больными COVID-19 в Одессе
 
%%endhead 
 
\subsection{\enquote{Тела не успели вывезти, не было свободных рук}. Подробности скандала с больными COVID-19 в Одессе}
\label{sec:04_12_2020.news.ua.strana.sibircev_aleksandr.1.covid_odessa}
\Purl{https://odessa.strana.ua/304833-v-odesskoj-bolnitse-objasnili-pochemu-v-koridorakh-ne-ubirajut-tela-umershikh-ot-koronavirusa.html}
\ifcmt
	author_begin
   author_id sibircev_aleksandr
	author_end
\fi

\ifcmt
pic https://odessa.strana.ua/img/article/3048/33_main-v1607101117.jpeg
cpx В сутки больница принимает по 50 пациентов. Фото: izbirkom.org.ua 
\fi

\index[rus]{Коронавирус!Смерти!Одесса, 5 больница, 04.12.2020}

В Одессе разгорается скандал, связанный с повышенной смертностью от COVID-19 и
неготовностью городских больниц к пандемии.

Причиной шумихи стали фото умерших людей в коридорах одесской городской
больницы № 5,\Furl{https://odessa.strana.ua/304506--v-odesskoj-bolnitse-nomer-5-umershie-ot-covid-19-lezhat-rjadom-s-zhivymi.html} что на улице Троицкой, 37.

Снимки выложил у себя на странице в \enquote{Фейсбуке} Евгений Цяпенко.

На них запечатлены тела четырех человек, завернутых в одеяла, простыни и
пластиковые пакеты.

\ifcmt
tab_begin cols=3
	%reload 1
	caption Ужас в одесской больнице на Троицкой, 37

pic https://odessa.strana.ua/img/forall/u/11/33/127815045_10216297220457213_8227647211014657751_n.jpg

pic https://odessa.strana.ua/img/forall/u/11/33/127993076_10216297219937200_3784579694287582102_n.jpg

pic https://odessa.strana.ua/img/forall/u/11/33/128066861_10216297220137205_819407367595955758_n.jpg

pic https://odessa.strana.ua/img/forall/u/11/33/%D0%B1%D0%BE%D0%BB.png

tab_end
\fi

После публикации фото в СМИ одесский облздав дал брифинг по этому поводу. И. о.
директора ведомства Наталья Одарий–Захарьева заверила журналистов, что
информация и фото не отвечают действительности.

\enquote{У нас работают морги, да, они значительно заполнены, но автобаза санитарного
транспорта в силах вывезти тело. В этом нет проблем. Если и был какой-то
момент, когда тело не успели забрать, прошу прощения}, – объяснилась
Одарий-Захарченко.

Сейчас в больницу пройти невозможно, на входах дежурит Национальная гвардия.

Но, как выяснила \enquote{Страна}, в прошлые выходные и понедельник в больнице,
действительно, произошло несколько смертей больных COVID-19. Тела вывезти не
успели, поэтому они пролежали несколько часов в коридорах и в палатах. Что и
успели сфотографировать сами больные.

\enquote{Эти фото – правда. Просто так получилось, что из этой больницы не
успели вовремя вывезти умерших. Персонала в этой больнице мало – всего семь
человек, включая младший медицинский персонал. Всем не успевают оказать помощь.
Там не хватает персонала по штатному расписанию. Ситуация обострилась в прошлые
выходные. Особо тяжелым не успели оказать помощь – они были совсем плохи. Двое
больных умерло ночью, один умер в понедельник утром. Их тела не успели вывезти,
просто не было свободных рук}, – пояснил \enquote{Стране} сотрудник облздрава,
попросив не называть его имени.

Аналогичные пояснения дала заместитель директора по медицинской части
горбольницы Людмила Бирюкова. По ее словам, наплыв пациентов в день – 50
человек, и это очень большая нагрузка. Из-за этого персонал в панике.

Отделение начало работать с ковидными пациентами только 27-го числа, и персонал
не всегда успевает быстро увозить тела, когда одновременно умирают несколько
больных, что иногда случается.

\href{https://youtu.be/wDFuU0nTUCg}{Видео - О ситуации в 5 больнице}

Больница, прежде имевшая специализацию по лечению неврологических заболеваний,
с ноября начала принимать больных с COVID-19. Для них предназначено 90 коек,
оборудовано восемь реанимационных палат. Но, по информации \enquote{Страны} от
источников в одесском облздраве, сейчас все койки в больнице заняты больными.

\enquote{Лекарств не хватает, медсестры просто разрываются, не успевают ставить
капельницы. Не хватает элементарных лекарств и кислорода}, – пояснил
\enquote{Стране} тот же источник в облздраве.

\enquote{Страна} не раз писала о тяжелом положении в украинских больницах, где
не хватает коек для зараженных коронавирусом. Так, больницы Ужгорода
переполнены и больных COVID-19 везут в районы.\Furl{https://odessa.strana.ua/news/298580-v-bolnitsakh-uzhhoroda-ne-ostalos-mest-dlja-bolnykh-koronavirusom-ljudej-vezut-v-rajony.html}

То же самое происходит и в столице, о чем мэр Кличко говорил еще в сентябре.\Furl{https://odessa.strana.ua/news/274974-kievskie-bolnitsy-perepolneny-bolnymi-koronavirusom.html}
