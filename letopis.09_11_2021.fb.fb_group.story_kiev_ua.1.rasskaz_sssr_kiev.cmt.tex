% vim: keymap=russian-jcukenwin
%%beginhead 
 
%%file 09_11_2021.fb.fb_group.story_kiev_ua.1.rasskaz_sssr_kiev.cmt
%%parent 09_11_2021.fb.fb_group.story_kiev_ua.1.rasskaz_sssr_kiev
 
%%url 
 
%%author_id 
%%date 
 
%%tags 
%%title 
 
%%endhead 
\subsubsection{Коментарі}

\begin{itemize} % {
\iusr{Светлана Манилова}

Как же много я пропустила! Но благодаря "Киевским историям" наверстываю! Спасибо!❤

\iusr{Ирина Архипович}
Было очень интересно!! @igg{fbicon.thumb.up.yellow}   @igg{fbicon.face.happy.two.hands} 

\iusr{Владимир Картавенко}

А почему в 1958 расформировали морполит и какое значение для создания
21.01.1967 кввмпу имело значнние постанлвление пленума Верховного суда ссср от
21 сент.1966. Это Основы.

\iusr{Валентина Кудрявцева}

Я также применила этот метод на курсах повышения квалификации инженеров по
вентиляции, когда для получения удостоверения необходимо было написать полит.
реферат. Применила я передовицы из газеты «Правда». Написала не только для
себя, но и для некоторых «курсантов» из Среднеазиатских республик. Но, на самом
деле, рассказ напомнил мне слова из выступления чемпиона мира Юрия Власова на
каком-то высоком партийном собрании о раннем практицизме у детей. Здесь яркая
иллюстрация этого термина.

\begin{itemize} % {
\iusr{Галина Литвиненко}
\textbf{Валентина Кудрявцева} Это точно: практицизм высочайшего уровня. А где же знания?

\begin{itemize} % {
\iusr{Петр Кузьменко}

\textbf{Галина Литвиненко} и знания были. Но практицизмом надёжней! @igg{fbicon.laugh.rolling.floor} 

\iusr{Валентина Кудрявцева}
\textbf{Галина Литвиненко} Ну, я думаю, знания в школе мы, все же, получили. У меня, например, были знающие, ответственные, строгие и добрые учителя.

\iusr{Юля Юхневич}
\textbf{Галина Литвиненко} 

уж поверьте свидетельнице всего вышеописанного - знания присутствовали в
достаточном объёме. Наши пятерки по истории и обществоведению позволили нам с
Петр Кузьменко окончить весьма успешно идеологические вузы в своё время. Автор
статьи - весьма образованный и разносторонний человек, очень интересный
собеседник, друг, товарищ и почти брат. Кто-либо из наших будет спорить? Не
думаю.

\end{itemize} % }

\iusr{Юля Юхневич}
\textbf{Валентина Кудрявцева} я бы применила вместо неоднозначного слова «практицизм» словосочетание «здравый смысл»

\begin{itemize} % {
\iusr{Валентина Кудрявцева}
\textbf{Юля Юхневич} Полагаю, что это не здравый смысл, а умение приспособиться к условиям, которые нам навязали в данной ситуации.

\iusr{Юля Юхневич}
\textbf{Валентина Кудрявцева} Так мы ж с Подола @igg{fbicon.wink} 
\end{itemize} % }

\end{itemize} % }

\iusr{Елена Зелинская}

Хоть я и не Подоле школу заканчивала, но материалы съезда КПСС и прочая, и
прочая очень выручали в школьной жизни. Поступала на истфак Универа, но не
сложилось. Зато, возможно, вместо меня поступила 19-летняя кандидатка в ряды
партии с двухлетним стажем работы на заводе, выходец из сельской местности.
Тогда это было в тренде.

\iusr{Vladimir Umanetc}

Дааа... Изучение материалов 26 съезда КПСС , это что-то..☺ У многих тогда эта
красная книжечка была настольной, помню ее и я , правда совсем позабыл что , и
о чем в ней писалось , видимо не очень впечатлила... @igg{fbicon.face.grinning.squinting} 

\iusr{Татьяна Сирота}

Э-э-х!!! Молодость... @igg{fbicon.face.upside.down}{repeat=3} 

\iusr{Michael Yurovsky}

Благодаря "красным дисциплинам" в советском ВУЗе я научился конспектировать по
методу Корнелла, и успешно применил (и применяю - теперь уже в качестве
преподавателя - для обучения студентов) при обучении в Иерусалимском универе.


\iusr{Tatyana Krivtsova}

Вы начали с XXVII съезда КПСС, а закончили двадцать шестым. Поясните,
пожалуйста!

\begin{itemize} % {
\iusr{Петр Кузьменко}
\textbf{Tatyana Krivtsova} обычная опечатка. Благодарю. Уже исправил.

\iusr{Tatyana Krivtsova}
\textbf{Петр Кузьменко} Редко читаю большие тексты (чаще "пробегаю" глазами), а этот читала с интересом и внимательно!

\iusr{Светлана Манилова}
\textbf{Tatyana}. опечатка.
\end{itemize} % }

\iusr{דמיטרי קוגן}

История повторяется. Эта история напомнила мне как я сдавал выпускные экзамены
на шесть лет раньше но в той же лучшей в мире школе. Только некоторые учителя
были другие. Биология Елена Петровна химия Галина Кирилловна история Наталья
Петровна. Да и съезд был 25-й

\begin{itemize} % {
\iusr{Нелли Кузьменко}
\textbf{דמיטרי קוגן}
Это школа номер 100! @igg{fbicon.face.grinning.smiling.eyes}  @igg{fbicon.thumb.up.yellow} 

\iusr{דמיטרי קוגן}
\textbf{Нелли Кузьменко} фон евеар

\iusr{Петр Кузьменко}
\textbf{דמיטרי קוגן} Супер! Оказывается это школьная традиция! @igg{fbicon.hands.shake}  @igg{fbicon.laugh.rolling.floor} 

\iusr{דמיטרי קוגן}
\textbf{Петр Кузьменко} а то. Я учился у учителей которые учили ещё моего отца. Валентина Фёдоровна так и называла меня своим внуком

\iusr{Alla Zgurzhnitsky}
\textbf{דמיטרי קוגן} а у меня был 24й и все те же учителя. Наталка меня не любила за родственные связи. Всегда находила что-то, чтобы занизить оценку.
\end{itemize} % }

\iusr{Елена Сидоренко}
Спасибо, интересно, и совсем не нудно! Все мы жили во времена съездов, куда ж деваться! @igg{fbicon.beaming.face.smiling.eyes} 

\iusr{Ольга Кирьянцева}

А вот и не нудный совсем ваш рассказ. Бывают тексты, которые начинаешь читать и
уже оторваться не можешь, потому как в них есть что-то тебе знакомое и
родственное , которые рассказывают о дорогих сердцу вещах. Вот Ваши, Петр,
рассказы как раз такие. И о школе, и об училище, и о Подоле, где жили мои
родственники, и где я, пусть совсем недолго, всего около месяца, училась в
школе, правда, не в 100...

Спасибо Вам, пишите ещё, и побольше) @igg{fbicon.face.happy.two.hands} 

\iusr{Леся Лагуна}

Пётр, благодарю Вас! Совсем не нудный рассказ у Вас получился. Очень интересный
и легкий. Навеял Воспоминания моих выпускных ( этот же период). Спасибо  @igg{fbicon.hands.pray} 

\iusr{Николай Мелешко}
Это называется, - решения съезда КПСС - в жизнь.

\iusr{Татьяна Ховрич}
Спасибо! Супер!!!

\iusr{Mykhaylo Dym}

Лежит у меня пару конспектов "класиков" из универа. Не выбрасываю, чтобы дети
поняли как им повезло, что не занимаются бесполезным чистописанием и не пишут
дифирамбы тому, во что уже в 70е (уж не говоря о 80х) никто не верил.

\iusr{Vitali Andrievski}

А я выплыл на госэксамене по научному коммунизму ( и такая была хренотень в те
годы) благодаря одолженной у моего друга великого произведения не менее
великого Леонида Ильича "Малая земля". Я убедил принимающих экзамен преподов
что я ее действительно читал

\begin{itemize} % {
\iusr{Светлана Манилова}
\textbf{Vitali}, 

мне повезло еще больше. В 1989 году в КГУ им. Т.Г. Шевченко вводится госэкзамен
под названием " Марксизм- ленинизм", который включал историю КПСС, философию,
политэкономию и... научный коммунизм. Мало не показалось...Это ж очень было
важно на механико-математическом факультете... @igg{fbicon.smile} 

\iusr{Петр Кузьменко}
\textbf{Vitali Andrievski} 

а я читал всю трилогию лауреата Ленинской премии по литературе. Даже цитировал.
Ещё мы смотрели балет "Целина". Весь спектакль ждали, когда выбежит танцор в
виде трактора. По силе эстетического воздействия на зрителя его можно сравнить
лишь с оперой "Повесть о настоящем человеке". Помнится, когда смотрел и слушал
этот "шедевр", думал, что нельзя так издеваться над настоящим подвигом и
героизмом.  @igg{fbicon.face.sad.but.relieved}  Однако воспроизведу некоторые
арии или ариозо из этого творения.  Итак, представьте, по сцене на каталке 4
санитара или медбрата везут практически бездыханное тело главного героя. Один
из них протяжно поёт: "Гангрена, гангрена! Ему отрежут ноги!" Навстречу им
выбегает доктор и кровожадно вторит: "Отрежем, отрежем Мересьеву ноги!". Тут
лежащий полутруп садится на каталке и поёт: " Не надо, не надо! Я буду летать!"

\begin{itemize} % {
\iusr{Alla Zgurzhnitsky}
\textbf{Петр Кузьменко} Я получила тройку на истории партии, потому как проигнорировала. А вы молодец, если одолели это чтиво.
\end{itemize} % }

\end{itemize} % }

\iusr{Genady Goldstein}
Цили Семеновны уже около года, как не стало. Умерла в Германии.

\iusr{Alla Zgurzhnitsky}
\textbf{Genady Goldstein} не знала, что она тоже ушла.

\iusr{Лидия Гончарук}
А Вы , батенька, циник. Так выбирать профессию в молодые годы!

\iusr{Раиса Карчевская}
Петр!

Ваши воспоминания написаны прекрасным языком, что еще раз подчёркивает Ваше
образование. Написано очень живо и увлекательно, и побудило море воспоминаний о
том времени. Пишите, Ваши воспоминания всегда очень интересные

\iusr{Вахтанг Кварелашвили}
Помню, помню. Это тот съезд, на котором Брежнев начал читать свой доклад не с начала, а где-то с середины)))

\iusr{Yana Shepurova}

Интересно было читать про своих учителей. Вспоминать Цилю Семёновнау, Наум
Ильича, Аллу Анатольевна и всех перечисленных в рассказе. Спасибо за эти
воспоминания.

\iusr{Владимир Дубровский}

\obeycr
К чему ругать нам те года
Ведь мы же жили и тогда
Тогда своей мы жизнью жили
Что задавали, то учили
Друзей имели и врагов
На что-то каждый был готов
На Дни рождения ходили
Смотрели разное кино
Живём сейчас, а значит жили
Хоть это вроде и давно.
\restorecr

\iusr{Alla Zgurzhnitsky}

Какой же ерундой забивали наши головы.

\end{itemize} % }
