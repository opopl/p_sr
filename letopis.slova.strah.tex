% vim: keymap=russian-jcukenwin
%%beginhead 
 
%%file slova.strah
%%parent slova
 
%%url 
 
%%author 
%%author_id 
%%author_url 
 
%%tags 
%%title 
 
%%endhead 
\chapter{Страх}

%%%cit
%%%cit_head
%%%cit_pic
%%%cit_text
Катя і Настя познайомилися у 2013-му, тоді ж і зробили свій перший напис. Тут,
біля будинку Сікорського. – Це була така суміш почуттів. З одного боку,
\emph{страх}, що порушуємо певні норми, робимо щось заборонене, і за це нас
покарають. З іншого боку, захват від того, що змогли порушити правило. Це
спрацювало як звільнення від того, що над нами тяжіло, і що ми навіть не
усвідомлювали.  Бо ти все життя правильна дівчинка: не засмучувати старших, не
підвести родину.  Та й взагалі, що може дозволити собі 30-річна жінка? Чи може
вона дозволити собі писати на стіні? Може!
%%%cit_comment
%%%cit_title
\citTitle{Непозбувний лох, стіна-психолог та вічна молитва. Про що говорить Київ на стінах, парканах і асфальті}, 
Олена Струк, www.pravda.com.ua, 02.07.2021 
%%%endcit

