% vim: keymap=russian-jcukenwin
%%beginhead 
 
%%file 02_02_2019.stz.news.ua.mrpl_city.1.k_122_letiu_mkk_im_iljicha_vladimir_bojko
%%parent 02_02_2019
 
%%url https://mrpl.city/blogs/view/k-122-letiyu-mmk-im-ilicha-vladimir-bojko
 
%%author_id burov_sergij.mariupol,news.ua.mrpl_city
%%date 
 
%%tags 
%%title К 122-летию ММК им. Ильича: Владимир Бойко
 
%%endhead 
 
\subsection{К 122-летию ММК им. Ильича: Владимир Бойко}
\label{sec:02_02_2019.stz.news.ua.mrpl_city.1.k_122_letiu_mkk_im_iljicha_vladimir_bojko}
 
\Purl{https://mrpl.city/blogs/view/k-122-letiyu-mmk-im-ilicha-vladimir-bojko}
\ifcmt
 author_begin
   author_id burov_sergij.mariupol,news.ua.mrpl_city
 author_end
\fi

\ifcmt
  ig https://mrpl.city/uploads/posts/resize/w/750x750/amdc1oelngwdflzt.jpg
  @wrap center
  @width 0.9
\fi

У его родителей не было связей в высших сферах – они были простыми тружениками,
ему не нанимали репетиторов, у него не было влиятельных родственников, ему не
делали поблажек ни в школе, ни тем более в вузе. И то, что он стал генеральным
директором одного из крупнейших металлургических предприятий страны – результат
только его трудолюбия, природного ума, фантастической памяти и искреннего
человеколюбия.

Владимир Семенович Бойко родился 20 сентября 1938 года в г. Мариуполе. Татьяна
Владимировна Потапова, дочь легендарного генерального директора, поделилась
семейными преданиями, что впитала ее память с детских лет.

\textbf{Читайте также:} 

\href{https://archive.org/details/19_01_2017.sergij_burov.mrpl_city.istoria_sozdania_zavoda_nikopol_mariupol_obschestva}{%
История создания завода Никополь-Мариупольского общества, Сергей Буров, mrpl.city, 19.01.2017}

В год, месяц и день рождения Володи его родители, Вилина Ильинична и Семен
Иосифович, жили в доме № 13 по улице Семашко. Назвав имя бабушки, Татьяна
Владимировна вскользь заметила, что имя ее бабушки никак не связано с
революционным \enquote{имятворчеством} 20 – 30-х годов ХХ века. Да, действительно, оно
произошло от старинного русского  женского имени Вила. Почти все биографы Бойко
не забывали подчеркнуть, что его детство прошло в Садках. Так оно и было. Что
же заставило в годы немецкой оккупации Мариуполя семью с двумя детьми (в 1940
году Вилина Ильинична родила дочку)  поменять квартиру со всеми удобствами на
родительскую землянку в призаводском поселке? В памяти Татьяны Владимировны
отложилось, что произошло будто это из-за бомбежек. Может быть. Но нельзя
исключить и такую причину – 7 сентября 1943 года гитлеровцы, отступая, сожгли
дом № 13 по улице Семашко, где жила семья Бойко, как много и других строений в
Ильичевском районе.

Детство и юность Владимира Бойко, как уже сказано выше, прошли в Садках.
Родители его построили  дом. Кстати, в этом доме родилась и Татьяна
Владимировна. Семен Иосифович работал слесарем в ремонтной службе в одном из
цехов завода имени Ильича. Вилина Ильинична за детьми смотрела, но не сложа
руки. Она умела делать все: и покрасить материю, и сшить из нее одежонку, но
коньком ее была кулинария. Она пекла невероятной высоты караваи из сдобного
теста, мастерски готовила разнообразные мясные, рыбные и овощные блюда. Талант
Вилины Ильиничны не пропадал даром. Когда на поселке случалась свадьба или, не
дай Бог, похороны, Вилину Ильиничну и двух ее сестер приглашали приготовить
яства на гостевые столы. В то время еще не было \enquote{вредной} привычки праздновать
или поминать в ресторанах.

А Владимир? Он не хулиганил, не участвовал в сомнительных \enquote{мероприятиях}
поселковой братвы. Но за себя и за слабых мог с кулаками постоять. Владимир
Семенович с детства был завзятым книгочеем. Он мог целую ночь до рассвета
читать понравившуюся ему книгу. Все больше произведения литературы о морях и
моряках. До наших дней, конечно, не дошли библиотечные формуляры на его имя, но
можно предположить, что это были \enquote{Морские рассказы} К. Станюковича, \enquote{Два
капитана} В. Каверина, \enquote{Под парусами} Д. Лухманова и книги других
писателей-маринистов. Ведь именно этими книжками зачитывались мариупольские
сверстники Володи Бойко. Вот почему, окончив семь классов, Владимир подал
заявление в Школу мореходного обучения, находившуюся в Портовском районе. Но
документы у него не приняли. В это учебное заведение зачисляли  ребят не моложе
четырнадцати лет на 1 сентября текущего года. А Володе Бойко до четырнадцати
лет не хватило двадцати дней. Пришлось  продолжить учебу в своей 44-й школе. В
1955 году он окончил десять классов. Можно с большой степенью вероятности
предположить, что Владимир сделал попытку поступить в Ждановский
металлургический институт, но попытка оказалась неудачной. И не потому, что он
был плохо подготовлен. В историю ЖдМИ 1955-й вошел как год с наибольшим
количеством претендентов на одно студенческое место. \enquote{Пробиться} сквозь очередь
медалистов и почти медалистов на двести пятьдесят вакантных студенческих мест
было очень трудно.

\textbf{Читайте также:} 

\href{https://archive.org/details/25_01_2017.sergij_burov.mrpl_city.russkij_providans_god_1906}{%
\enquote{Русский Провиданс}: год 1906-й, Сергей Буров, mrpl.city, 25.01.2017}

Владимир начал свой трудовой путь в 1955 году с трубопроводчика в цехе
водоснабжения завода им. Ильича. Затем он словно испытывает себя на прочность.
Строки из послужного списка: 1956—1957 — слесарь-сантехник, подземный рабочий,
путевой откатчик, СУ № 13 \enquote{Сталиншахтострой}; матрос рыболовного траулера
РТ-142 тралфлота \enquote{Главмурманрыбпрома}. В 1957 году Владимира призвали в
Советскую армию. Ему довелось служить в Ленинградском военном округе.
Демобилизовался он в 1960 году с должности командира отделения в звании
младшего сержанта.

Бойко вернулся в родной город. Устроился резчиком металла в листопрокатный цех
№ 6 завода имени Ильича. Руководство заметило сметливого работящего парня, и он
был назначен бригадиром. Владимир женился. Его жена, Клавдия Александровна,
подарила ему дочку.

В 1964 году Владимир поступил на вечерний факультет Ждановского
металлургического института на специальность \enquote{Прокатное производство}. Татьяна
Владимировна рассказывала, что отец, обучаясь в институте, не пропускал
занятий, будь то лекция, а тем более, лабораторная работа. Перед
экзаменационной сессией ему, как и всем студентам-вечерникам, предоставлялся
отпуск. В это время он с невероятной скоростью прочитывал конспекты лекций,
заглядывая время от времени в учебники. Люди, близко знавшие Владимира
Семеновича, отмечали его поразительную память. Он успешно сдавал зачеты и
экзамены, успешно защитил дипломный проект в 1970 году и получил звание
инженера-металлурга.

Дальнейшая карьера инженера Бойко на заводе имени Ильича пошла по восходящей. С
1971 по 1973 год  он работал старшим мастером листопрокатного цеха № 6. В
1974-м его назначили  заместителем начальника производственного отдела завода
по специальному производству. И, наконец, с 1976 года становится начальником
листопрокатного цеха № 6.

До 1971 года Владимир Семенович с семьей жил в родительском доме в Садках. Но в
1971 году, подкопив денег и дождавшись очереди, он приобрел кооперативную
двухкомнатную квартиру. В апреле 1973 года его пригласили на завод \enquote{Азовсталь}
на должность помощника начальника листопрокатного цеха \enquote{3600}. Цех находился в
стадии завершения монтажа оборудования, и большой производственный опыт и
глубокое знание особенностей прокатного производства Владимира Семеновича очень
пригодились. Особенно когда настало время освоения оборудования. Его весомый
вклад в это дело был отмечен правительственной наградой – орденом \enquote{Знак
Почета}. Его просили остаться на \enquote{Азовстали} - он отказался, вернулся на свой
завод.

\textbf{Читайте также:} 

\href{https://archive.org/details/02_02_2017.sergij_burov.mrpl_city.poezdka_na_zavod}{%
Поездка на завод, Сергей Буров, mrpl.city, 02.02.2017}

Следующие ступени служебной лестницы с 1983 по 2012 год включительно:
заместитель начальника производственного отдела по прокатному производству, и.
о. начальника производственного отдела,  начальник производственного отдела,
заместитель генерального директора по производству, генеральный директор, глава
правления-генеральный директор Мариупольского металлургического комбината имени
Ильича.

Анкетные данные, послужные списки – лишь канва для жизнеописания простого
инженера или руководителя. Другое дело - воспоминания о нем человека, близко
знавшего его. Причем воспоминания, состоящие не только из прилагательных
\enquote{хороший}, \enquote{умный}, \enquote{деловой} и т.п. Такой  человек – Сергей Анатольевич
Матвиенков, при Бойко - главный инженер комбината им. Ильича.  Несмотря на
занятость обязанностями народного депутата, Сергей Анатольевич, узнав, что
разговор пойдет о Владимире Семеновиче, без всяких условий согласился на
встречу. И вот что он рассказал.

С 1982 год завод им. Ильича возглавил Николай Алексеевич Гуров, человек
твердого характера и решительной натуры. Однажды посетив листопрокатный цех №
6, он тут же принял решение закрыть его. Действительно, и здание, и
оборудование цеха были старыми. В это время начальником этого подразделения был
Владимир Бойко. Он понимал, что цех не блещет красотой, но понимал и другое –
продукция его важна для государства. Да и заводу цех приносит немалую прибыль.
Спорить с директором бесполезно. Владимир Семенович собрал коллектив, сообщил о
решении директора и предложил в кратчайшие сроки навести порядок. Люди
поддержали своего руководителя. И началась \enquote{генеральная уборка}. Вскоре на
завод приехал министр. Заехал он со свитой и в 6-й цех. Гуров докладывает –
закрывать будем. А министр: Зачем? Все работает, все блестит, пользу стране
приносит. Цех не закрыли, но Владимир Семенович был переведен в
производственный отдел...

В 1990 году В. Бойко был назначен генеральным директором предприятия. Но в
следующем году произошел развал Советского Союза. Рушились производственные
связи, прекратилась более или менее установившиеся системы
материально-техничес\hyp{}кого обеспечения производства. Всеобщий хаос. Новолипецкий
металлургический комбинат перестал поставлять слябы на толстолистовой стан
\enquote{3000}. Шахтеры бастуют, нет угля, не из чего выжигать кокс для доменных
печей. Предприятие было на грани краха, вот-вот несколько тысяч трудящихся
могли остаться за воротами комбината, остаться без куска хлеба.

Именно в это смутное время Бойко принимает беспрецедентное решение. Будем
строить МНЛЗ – машину непрерывного литья заготовок! МНЛЗ – сложнейший агрегат.
И все-таки ильичевцы взялись за дело. Инженеры проектно-конструкторского отдела
разработали проектно-техническую документацию, ученые ЦЗЛ подготовили
технологию, хорошо поработали снабженцы, в ремонтных цехах изготовили
заготовки. И МНЛЗ вступила в строй. Завод продолжал работать. Владимир
Семенович видел перспективу и верил в людей.

\textbf{Читайте также:} 

\href{https://archive.org/details/15_02_2017.sergij_burov.mrpl_city.istoria_direktor_mmk_im_iljicha_aleksandr_garmashev}{%
История: директор ММК им. Ильича Александр Гармашев, Сергей Буров, mrpl.city, 15.02.2017}

Сущностью характера Владимира Бойко была независимость. Он не дружил с
областным руководством, твердо охраняя комбинат от вмешательства. Мог отчитать
министра за невыполнимое указание. Он добился приватизации комбината. И
искренне любил людей, и верил людям. Собственно, для людей жил и трудился
Владимир Семенович. Он смело шел на риск: подставил плечо оказавшимся на грани
краха колхозам  и совхозам близлежащих районов. Взял под опеку молокозавод,
швейную фабрику, другие предприятия. А сколько им сделано для города, сколько
отремонтировано дорог, школ, детских садов, жилья, скольким конкретным людям,
попавшим в беду, им оказана помощь. Похоже, что юношеская влюбленность к морю и
морякам не оставила его. Когда капитан тральщика \enquote{Мариуполь} попросил помочь
отремонтировать корабль, Владимир Семенович дал команду привести тральщик в
порядок. И это задание, конечно, было выполнено.

Владимир Семенович длительное время ездил на работу на УАЗике, по дороге
подбирал рабочих комбината. Некоторые пользовались этим, чтобы обратиться к
генеральному директору с просьбой. Вообще к нему обращалось множество
просителей. Это стало мешать его работе, а УАЗик заменили более комфортабельным
автомобилем. Бойко не пользовался благами, которые предоставлялись ему как
генеральному директору абсолютно законно. Маленький пример. В Киеве при
представительстве комбината имени Ильича были апартаменты для ночевки. Владимир
Семенович там никогда не ночевал, шел в гостиницу.

Вот что еще рассказал Сергей Анатольевич. Сотрудников, близко знавших Владимира
Семеновича, поражала его феноменальная память. Он помнил множество
цифр-показателей производства, даты событий, происходивших на комбинате и в
связи с комбинатом в Бог знает в какие годы. Он досконально знал не только
практику прокатного производства, но и теорию этого производства. И все это
было результатом его цепкой памяти. Это знали некоторые руководители научных
институтов и вузов и предлагали \enquote{помочь} оформить его познания в виде
диссертации. Бойко всегда отказывался от таких предложений, иногда в очень
резкой форме. Он скептически относился к тем руководителям предприятий, которые
окольными путями старались \enquote{украсить} свои фамилии учеными степенями и
званиями. Владимир Семенович ценил науку и ученых. Авторитетами для него были
Георгий Иванович Налча и Эдуард Николаевич Шебаниц. Оба защитили свои
диссертационные работы по результатам  исследований, принесших большую пользу
комбинату, и одновременно обогатили теорию и практику прокатного производства.
Он находился в приятельских отношениях с Борисом Евгеньевичем Патоном.

\textbf{Читайте также:} 

\href{https://archive.org/details/10_03_2017.sergij_burov.mrpl_city.istoria_direktor_mmk_im_iljicha_nikolaj_gavrilenko}{%
История: директор ММК им. Ильича Николай Гавриленко, Сергей Буров, mrpl.city, 10.03.2017}

Владимир Семенович был совершенно равнодушен к своим вполне заслуженным
наградам, званиям и отличиям. Кажется, он один раз на людях появился с \enquote{Золотой
Звездой} Героя Украины. Вернувшись из Донецка, он встретился с ильичевцами.
Поздоровавшись, он снял пиджак со Звездой и повесил на спинку стула. Просмотрев
множество фотографий с его изображением, лишь на одной из них он изображен с
\enquote{Золотой Звездой} Героя Украины. Но, присмотревшись, можно заметить, что
награда \enquote{пришпилена} на лацкан пиджака с помощью фотошопа.

Владимир Семенович Бойко - Герой Украины, кавалер ордена \enquote{За заслуги} всех трех
степеней, народный депутат Украины 4, 5 и 7-го созывов, председатель правления
и генеральный директор Мариупольского металлургического комбината имени Ильича,
почетный президент футбольного клуба \enquote{Ильичевец}, почетный гражданин города
Мариуполя ушел в мир иной 10 июня 2015 года.
