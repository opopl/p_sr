% vim: keymap=russian-jcukenwin
%%beginhead 
 
%%file 13_10_2020.fb.oleg_kovalenko.1_ussr
%%parent 13_10_2020

%%url https://www.facebook.com/shpik.inet
%%tags ussr,fb 
%%author oleh kovalenko
 
%%endhead 

\subsection{ФАКТ. В Советском Союзе, в УССР за 70 лет построили}

Олег Коваленко
\url{https://www.facebook.com/shpik.inet}

ФАКТ. В Советском Союзе, в УССР за 70 лет построили:
Штеровскую ГРЭС
Углегорскую ГРЭС
Запорожскую ГРЭС
Криворожскую ГРЭС
Бурштынскую ГРЭС
Змиёвскую ГРЭС
Ворошиловградскую ГРЭС
Старобешевскую ГРЭС
Приднепровскую ГРЭС
Славянскую ГРЭС
Ладыжинскую ГРЭС
Трипольскую ГРЭС
Кураховскуюю ГРЭС
Чернобыльскую АЭС
Ровенскую АЭС
Южноукраинскую АЭС
Запорожскую АЭС
Хмельницкую АЭС
Мощнейший даже по советским меркам промышленный и угольный комплекс на Донбассе, разработка Днепровского буроугольного бассейна, множество построенных нефтепроводов, крупнейшие из которых: 
нефтепровод «Дружба», Долина — Дрогобыч, Куйбышев — Лисичанск — Кременчуг, Гнединцы — Глинско - Розбишевское — Кременчуг — Херсон, Херсон — Одесса, Ахтырка — Глинско -Розбишевское. 
Газопроводы: «Союз» (Оренбург — западная граница СССР), Уренгой — Помары — Ужгород, «Братство» и другие.
Заводы:
«Запорожсталь», «Азовсталь», «Криворожсталь», «Днепроспецсталь» (первый электрометаллургический завод качественных сталей в СССР и один из первых в мире), Никопольский Южнотрубный завод, Запорожский ферросплавный завод и другие.
Металлургические заводы:
имени Г. И. Петровского и имени Коминтерна, Днепровский металлургический комбинат им. Ф.Э. Дзержинского, «Запорожсталь» и «Днепроспецсталь»,«Криворожсталь» имени В. И. Ленина,
Донецкий металлургический завод имени В. И. Ленина, Макеевский металлургический комбинат имени С. И. Кирова, Енакиевский металлургический завод, Коммунарский металлургический комбинат, Краматорский металлургический завод, «Энергомашспецсталь», Константиновский металлургический завод, металлургические комбинаты имени Ильича и «Азовсталь» имени С. Орджоникидзе,
а также Броварской завод порошковой металлургии.
Запорожский ферросплавный завод, Никопольский ферросплавный завод, Стахановский ферросплавный завод.
В результате политики советского руководства по объёму промышленной товарной продукции и стоимости основных производственных фондов УССР занимала 2-е место после РСФСР, а по производству ртути, ильменитовых, рутиловых и циркониевых концентратов — 1-е место в СССР.
Перечисленное – лишь капля в море ( чего стоит например «Южмаш» и не только, далеко не только) нескольких тысяч промышленных предприятий и уникальных наукоёмких производств, которые были созданы в период с 1945 г. по 1991 г

ВОПРОСЫ ЕСТЬ? К ТОМУ, ЧТО ЗА ОКНОМ?

