% vim: keymap=russian-jcukenwin
%%beginhead 
 
%%file 18_11_2021.fb.logvinova_dina.doneck.dnr.1.jenakievo_mamina_kolybeljnaja
%%parent 18_11_2021
 
%%url https://www.facebook.com/dina.logvinova/posts/4252227001570484
 
%%author_id logvinova_dina.doneck.dnr
%%date 
 
%%tags dnr,donbass,enakievo,gorod
%%title Енакиево живет во мне, как мамина колыбельная
 
%%endhead 
 
\subsection{Енакиево живет во мне, как мамина колыбельная}
\label{sec:18_11_2021.fb.logvinova_dina.doneck.dnr.1.jenakievo_mamina_kolybeljnaja}
 
\Purl{https://www.facebook.com/dina.logvinova/posts/4252227001570484}
\ifcmt
 author_begin
   author_id logvinova_dina.doneck.dnr
 author_end
\fi

Говорят, девушку можно вывезти из села, а вот село из девушки – никогда. Есть в
этой сермяжной правде какая-то щемящая нежность. Ведь, по сути, то, что мы
впитываем там, где родились, в дальнейшем становится фундаментом, на котором
базируется наша духовность. 

\ifcmt
  pic https://scontent-frx5-1.xx.fbcdn.net/v/t39.30808-6/257952744_4252226471570537_4247089627014567241_n.jpg?_nc_cat=111&ccb=1-5&_nc_sid=730e14&_nc_ohc=5HZ4bbc6u8kAX9IewbG&_nc_ht=scontent-frx5-1.xx&oh=cc4955aaf785bd64eda0b722c2368b57&oe=61A4AE62
  @width 0.8
\fi

Я родилась и выросла в маленьком рабочем городке, носящем имя инженера-путейца
Федора Енакиева. Удивительный был человек Федор Егорович. Дворянин, статский
советник, промышленник СОЗИДАТЕЛЬ! В его активе не только строительство нового
Петровского чугунноделательного завода, вокруг которого потом вырастет город
его имени. Енакиев принимал активное участие в создании разработок
серебросвинцовых рудников в Азербайджанской провинции Персии. И в Петербуржском
метрополитене тоже есть частичка его ума и души...

Я все это к чему рассказываю? Да к тому, что в генный код моего маленького
рабочего города изначально было заложено добро созидания, и я горжусь этим.
Енакиево живет во мне, как мамина колыбельная, как первый несмелый поцелуй
одноклассника, как урок старого журналиста, поверившего в меня...Вот пишу, а в
памяти отчетливо всплывают его слова после прочтения моей первой заметки в
газете об ударных строительных буднях: «Будет из тебя толк...» Сейчас с высоты
прожитых лет я понимаю, что тогда он говорил не только о профессиональном
становлении. Это был его наказ всегда оставаться человеком и помнить свои
корни. Я помню... Собственно поэтому и родился этот пост. Наконец-то сбылась
моя мечта быть хоть чем-то полезной своей малой Родине. Раньше, до войны, (как
же больно делить жизнь на такие временные промежутки) так вот, раньше я
частенько ездила в енакиевскую школу-интернат №2 с разными подарками для детей
или с людьми, которые могли решить и решали многие проблемы. А вот в войну, в
это лихое безвременье, как-то не случилось. И душа болит, потому что Енакиево
живет во мне, живет!

Но вот, благодаря единомышленникам, людям с высочайшим интеллектом и доброй
душой, которые собираются в культурном салоне «Куприн» на игре BRAIN TEAM
наконец-то удалось попасть в Енакиево к особенным деткам, порадовать их
обновками и, конечно же, сладостями-вкусностями. А помогла мне в этом наша
очаровательная «незабудка» Наталия Майтамалова, как оказалось, тоже рожденная в
Енакиево. В интернат приехали к обеду. Там тепло, уютно и вкусно пахнет
борщиком. В душе сразу же запели серебряные колокольчики, и звучат они во мне
до сих пор...

Помните свои корни. Питайте их добром, удобряйте любовью, и жизнь будет
улыбаться вам...

\ii{18_11_2021.fb.logvinova_dina.doneck.dnr.1.jenakievo_mamina_kolybeljnaja.cmt}
