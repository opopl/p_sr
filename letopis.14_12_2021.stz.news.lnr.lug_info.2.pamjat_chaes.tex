% vim: keymap=russian-jcukenwin
%%beginhead 
 
%%file 14_12_2021.stz.news.lnr.lug_info.2.pamjat_chaes
%%parent 14_12_2021
 
%%url https://lug-info.com/news/lugancane-zazgli-sveci-v-pamat-o-pogibsih-likvidatorah-posledstvij-avarii-na-caes
 
%%author_id 
%%date 
 
%%tags chaes,chernobyl,pamjat,lnr,lugansk,donbass
%%title Луганчане зажгли свечи в память о погибших ликвидаторах последствий аварии на ЧАЭС
 
%%endhead 
\subsection{Луганчане зажгли свечи в память о погибших ликвидаторах последствий аварии на ЧАЭС}
\label{sec:14_12_2021.stz.news.lnr.lug_info.2.pamjat_chaes}

\Purl{https://lug-info.com/news/lugancane-zazgli-sveci-v-pamat-o-pogibsih-likvidatorah-posledstvij-avarii-na-caes}

Фото: Марина Сулименко / ЛИЦ

Акция \enquote{Помним героев}, приуроченная ко Дню чествования участников ликвидации
последствий аварии на Чернобыльской атомной электростанции (ЧАЭС), прошла в
Луганске у мемориала \enquote{Чернобыльский лелека}. Об этом с места события передает
корреспондент ЛИЦ.

\ii{14_12_2021.stz.news.lnr.lug_info.2.pamjat_chaes.pic.1}

В мероприятии приняли участие депутаты Народного Совета ЛНР, активисты проектов
\enquote{Мы помним!} и \enquote{Дружина} общественного движения (ОД) \enquote{Мир Луганщине}, депутаты
Молодежного парламента ЛНР и участники общественной организации
"Республиканское объединение ветеранов и инвалидов "Союз-Чернобыль".

Молодые люди в честь погибших героев-ликвидаторов зажгли перед памятником сотни
свечей.

\ii{14_12_2021.stz.news.lnr.lug_info.2.pamjat_chaes.pic.2}

\enquote{Сегодня мы собрались у памятника ликвидаторам последствий на Чернобыльской
АЭС. Я являюсь ликвидатором (аварии) 1986 года и я искренне благодарна, что
сегодня сюда пришла молодежь Луганской Народной Республики – молодежь, на
которую мы можем опереться и которая участвует в строительстве нашей молодой
Республики}, - отметила депутат Народного совета ЛНР Светлана Алешина.

Она рассказала представителям молодого поколения о трагедии, которую видела
своими глазами, и призвала молодых людей быть внимательными, чтобы не допустить
подобных ошибок в будущем.

\ii{14_12_2021.stz.news.lnr.lug_info.2.pamjat_chaes.pic.3}

По ее мнению, такие мероприятия памяти необходимо проводить в учебных
заведениях, начиная с детского сада, потому что \enquote{нынешнее поколение считает эти
события многолетней давности уже сказкой}.

\enquote{Относитесь ко всему бережно, с осторожностью, с чувством хозяина за свою
землю. Желаю вам мира, добра и удачи}, - пожелала Алешина.

Участники акции возложили к памятному знаку цветы и почтили память погибших при
ликвидации последствий трагедии минутой молчания.

\enquote{Молодежь черпает память своего государства у старшего поколения, принимает и
победы, и горечь. История не должна забываться, поэтому луганская молодежь
поддерживает такие акции, чтобы не забывались наши герои}, - сказал заместитель
координатора проекта \enquote{Мы помним!} Артем Дыченко.

Ранее в луганском Дворце культуры состоялось торжественное собрание,
посвященное Дню чествования участников ликвидации последствий аварии на
Чернобыльской атомной электростанции.

День чествования участников ликвидации последствий аварии на Чернобыльской АЭС
был установлен на Украине в 2006 году. В Луганской Народной Республике был
установлен указом главы ЛНР в 2016 году, отмечается ежегодно 14 декабря.

Крупнейшая в истории человечества техногенная катастрофа – авария на
Чернобыльской АЭС – произошла 26 апреля 1986 года. Благодаря самоотверженности
ликвидаторов последствий катастрофы, многие из которых заплатили за свой
героизм жизнями и здоровьем, авария была локализована. 30 ноября 1986 года было
закончено строительство саркофага над разрушенным энергоблоком, а спустя две
недели, 14 декабря, центральные издания СССР опубликовали извещение ЦК КПСС и
Совета министров о том, что государственной комиссией был принят в эксплуатацию
комплекс защитных сооружений четвёртого энергоблока ЧАЭС. Именно дата
публикации этого сообщения, по прочтении которого страна смогла вздохнуть
спокойно, и стала Днем чествования участников ликвидации последствий аварии на
ЧАЭС.

Несмотря на заплаченную цену, сами ликвидаторы-чернобыльцы нередко называют 14
декабря праздником. По самым скромным оценкам, в ликвидации катастрофы приняли
участие не менее 90 000 человек из всех республик Советского Союза.
