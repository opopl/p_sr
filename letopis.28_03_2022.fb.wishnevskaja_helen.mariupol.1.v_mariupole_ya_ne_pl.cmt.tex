% vim: keymap=russian-jcukenwin
%%beginhead 
 
%%file 28_03_2022.fb.wishnevskaja_helen.mariupol.1.v_mariupole_ya_ne_pl.cmt
%%parent 28_03_2022.fb.wishnevskaja_helen.mariupol.1.v_mariupole_ya_ne_pl
 
%%url 
 
%%author_id 
%%date 
 
%%tags 
%%title 
 
%%endhead 

\qqSecCmt

\iusr{Tati Di}

Лен, а как сыночек переносил все? он, наверное не понял, что случилось? сильно перепугался? как же это страшно 💔

\iusr{Helen Ivchenko}

Какая ты находчивая! Боец 💪

\begin{itemize} % {
\iusr{Helen Wishnevskaya}
\textbf{Helen Ivchenko} не, я просто трусливая)
\end{itemize} % }

\iusr{Helen Wishnevskaya}

Да, он этого не понимал. И не понимал, почему я вдруг его хватаю и бегу в
коридор на пол, где холодно и неудобно. Испугался, когда первый раз под подъезд
прилетело и дом зашатался. Потом этих шатаний было не счесть, уже не пугался, в
отличие от нас с мамой.

\begin{itemize} % {
\iusr{Tati Di}
\textbf{Helen Wishnevskaya} будь они прокляты, сволочи(((
\end{itemize} % }

\iusr{Kateryna Stinberg}

Я тоже в Мариуполе не плакала... Первые дни слезы стояли в глазах по но не
хотела пугать сына. А потом было не до того. Зато с Бердянска ревела несколько
дней. Слезы просто катились из глаз, особенно когда в Фейсбук зашла и увидела
фото и рассказы других людей. Через несколько дней пропустило. Держитесь!
Сложно говорить, что теперь все будет хорошо, но вы выбрались из ада, и теперь
приходится привыкать к тому, что жизнь идёт дальше...

\begin{itemize} % {
\iusr{Helen Wishnevskaya}
\textbf{Екатерина Штайнберг} да, Катюша. Главное, что остались живы. Значит, жизнь продолжается!

\iusr{Kateryna Stinberg}
\textbf{Helen Wishnevskaya}, в первое время это вызывало дикий диссонанс - в Мариуполе реальный ад, а тут люди ходят, гуляют, продукты покупают...

\iusr{Helen Wishnevskaya}
\textbf{Екатерина Штайнберг} 

в Днепре воздушная сирена - а люди спокойно по улицам ходят. Мы когда приехали,
сидели в машине и ждали заселения, находились как раз под динамиком. Днепряне
молодцы - у них сирена то, что надо сирена! А не то, что было у нас. И в
Мариуполе, и в Днепре я её слышала в центре города, поэтому могу сравнивать.
Так вот, нас с мамой затрясло, как сирена начала раздаваться, а люди идут себе
спокойно по своим делам, никто в убежище не бежит. Дай Бог, чтобы они никогда
не узнали того, что пережили мы

\iusr{Kateryna Stinberg}
\textbf{Helen Wishnevskaya}, я в Виннице в парикмахерскую пошла )) и пока меня девочка стригла, начала сирена выть. Парикмахерша - как же достала выть эта сирена! А я не выдержала и сказала, что когда сирен воет - это хорошо, это просто замечательно. А вот когда прилетает просто так - вот это плохо
\end{itemize} % }

\iusr{Vadim Popov}

А мне вспомнились фильмы, там где в древности племена нападали на деревеньки из
хижин и деревянных домиков и выжигали их до тла, а людей убивали всех без
разбора. Вот что изменилось с тех далёких времён?

\begin{itemize} % {
\iusr{Oleg Shuklin}
\textbf{Vadim Popov} именно это я недавно осознал и думал что люди на таком уровне осознания и развития что это невозможно...но увы...все так как вы написали(((
\end{itemize} % }

\iusr{эдуард рябушев}

Леночка, напиши, как окна в квартирах(у нас тещина квартира на 3 этаже)

\iusr{Helen Wishnevskaya}
\textbf{эдуард рябушев} 

окна, которые выходят на входы в подъезды практически все повылетали, балконы
сильно побило осколками ещё 6 марта, когда под мой второй подъезд прилетел
град. 21 марта бомбанули с самолёта электроподстанцию, которая с другой стороны
дома. Окна повылетали в основном в квартирах, что над бывшим Струмком, но
осколками досталось многим балконам вплоть до первого подъезда. Но не так
сильно, как с той стороны дома. Мы уезжали 25 марта. Дом был потрепан, но цел.
Что сейчас - не знаю.....

\iusr{Eugenia Zlamanuk}

Дякую, що вижили.

\iusr{Helen Wishnevskaya}

Что-то задерживается бумеранг. Пора бы уже
