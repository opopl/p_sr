% vim: keymap=russian-jcukenwin
%%beginhead 
 
%%file 01_06_2022.fb.saparov_nikolaj.mariupol.1.ofitsialno_o__skorbo.cmt
%%parent 01_06_2022.fb.saparov_nikolaj.mariupol.1.ofitsialno_o__skorbo
 
%%url 
 
%%author_id 
%%date 
 
%%tags 
%%title 
 
%%endhead 

\qqSecCmt

\iusr{Михайло Верескун}

Благодарю Вас, Николай. От имени всех мариупольцев. За все 25 лет, в течение
которых Скорбота помогала родственникам ушедших в самые трудные моменты...
😢😢😢 Благодарю за этот трудный, но очень честный и нужный пост... А ещё
благодарю за тот подвиг, который Вы совершили 14 марта, прорвавшись в
осажденный город (да, все мечтали вырваться ИЗ города, а Николай прорывался В
город!) и дав нам самое главное: информацию и надежду! Именно благодаря Вам мне
удалось 15.03 под обстрелами вывезти всех родных и близких в безопасное место.
Уверен, что у меня ещё будет возможность пожать Вам руку и поблагодарить Вас
лично. Спасибо Вам 🙏🙏🙏

P.S. Николай, я должник...

\begin{itemize} % {
\iusr{Николай Сапаров}
\textbf{Михаил Верескун} Вы ничего не должны.

\iusr{Михайло Верескун}
\textbf{Николай Сапаров} Это Ваше решение. Мое осталось прежним. В любом случае знайте, что есть на свете человек, который с радостью сделает для Вас все, что будет в его силах. Если Вам это потребуется...

\end{itemize} % }

\iusr{Таня Чернышева}

Мой сын, Чернышёв Максим, находился на складе Торговая, 65, и как теперь жить,
воспринимая эту ,,ужасную участь,, и ,,обгоревшие останки,,???...

Господи, за что?

Соболезную всем людям, чьи близкие были вместе с сыном, и верю, что мы найдём их захоронения,🖤

\iusr{Mykola Udot}

Подскажите пожалуйста можно ли узнать место захоронения человека (ищу где
похоронен отец), хоронили 02.03.2022 из того что знаю тело отвезли на
володарскую трассу и похоронили в старом Крыму. Может быть остались у кого то
контакты сотрудников у кого могут быть документы и информация и т д?

\iusr{Лилия Жукова}

Коля, спасибо! Нам все ясно Все знаем, понимаем! Суки, пусть и на их могилх никто не положит цветов!

\iusr{Владимир Праводелов}

\ifcmt
  ig https://scontent-ams4-1.xx.fbcdn.net/v/t39.30808-6/286214664_1663713273990928_6043806845918865437_n.jpg?_nc_cat=111&ccb=1-7&_nc_sid=dbeb18&_nc_ohc=4yeGerYvyR4AX9SSV2L&_nc_ht=scontent-ams4-1.xx&oh=00_AfC8MxAzFz0QOrsKPT-PR6qO5Oh3-FQkd5f5eLgzGlWLEw&oe=63E21CD6
  @wrap center
  @width 0.6
\fi

\iusr{Nadin Nadin}

2 января похоронила маму, помогала Скорбота. А вот папа 28 марта ушёл с дома и
больше не вернулся. Был с паспортом, но как сказали "местные власти" никаких
учетов никто не вёл. И если я маму оплакала и отпустила, то мысль что я не знаю
как погиб папа и где похоронен меня не отпускает. И таких как я тысячи. Наверно
шансов нет, что когда нибудь мы это узнаем.

\begin{itemize} % {
\iusr{Николай Сапаров}
\textbf{Nadin Nadin} верьте в лучшее. Не нашли- значит жив.
\end{itemize} % }

\iusr{Maine Helen}

Подскажите пожалуйста, я думаю эта информация может пригодится многим. Если
человек похоронен в огороде своего дома, стоит ли срочно перехоранивать на
кладбище? Может некорректно пишу, извините. Насколько понимаю сейчас в
Мариуполе коллапс в плане эксгумации, освидетельствования, документов.
Насколько опасно, безопасно захоронение в огороде... Стоит ли платить окупантам
сумму в размере 25000 грн. Где-то читала о таких ценах. Хотелось бы по -
человечески похоронить, после освобождения. Или лучше сейчас?

Подскажите пожалуйста.

\begin{itemize} % {
\iusr{Николай Сапаров}
\textbf{Maine Helen} Каждый сам принимает решение. Если не беспокоит,то что могила рядом и она глубоко выкопана, я бы тогда не спешил. Обычно,перезахоронение делают минимум через три года.

\iusr{Maine Helen}
\textbf{Николай Сапаров} Спасибо.

\iusr{Lori Djella}
\textbf{Николай Сапаров} подскажите пожалуйста ! Муж погиб 6 марта, успели сообщить в полиции об этом, не известно внесено это в Ердр, похоронен у себя дома частный сектор. Как потом доказать что смерть была насильственной?

\iusr{Николай Сапаров}
\textbf{Lori Djella} здравствуйте. Попробуйте обратиться в отделение полиции по месту своего нахождения

\iusr{Lori Djella}
\textbf{Николай Сапаров} Дякую!

\end{itemize} % }

\iusr{Владимир Праводелов}

Я был недавно в Мариуполе, искал маму. Она была на Торговой в лакированном
гробу. Я был на этом складе, общался со сторожем. Видел оставшиеся доски от
лакированного гроба. Был в Ритуале, это они вместе с мчс в начале мая вывозили 19
тел оттуда в морг за Метро. Судмедэксперты взяли днк, тела были захоронены на
старокрымском. Вернее обгоревшие останки. Есть у них фото в базе, но это отдельная
группа по обгоревшим телам. Она еще не собрана, находится на разных носителях у
судмедэкспертов. Сказали будет готова к началу июня, но опознать кого то на 99,9\%
будет невозможно. Только по днк, которые будут брать у живых где то в середине
июня.

\begin{itemize} % {
\iusr{Анна Николаенко}
\textbf{Владимир Праводелов} 

Владимир, я вам глубоко соболезную! Мы тоже потеряли сестру и не знаем о судьбе
местонахождения её тела. Были какие-то списки умерших на Торговой 65? Что вам
известно?

\iusr{Владимир Праводелов}

Списки есть, если тело идентифицировано, т. е. если были какие либо документы у
умершего или погибшего, есть еще фотобаза
\end{itemize} % }
