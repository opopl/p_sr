% vim: keymap=russian-jcukenwin
%%beginhead 
 
%%file 27_04_2022.fb.tomilenko_sergej.1.broniki_nszhu
%%parent 27_04_2022
 
%%url https://www.facebook.com/sergiy.tomilenko/posts/5052042531547237
 
%%author_id tomilenko_sergej
%%date 
 
%%tags 
%%title Захищати життя журналістів будуть якісні бронежилети українського виробництва, - НСЖУ
 
%%endhead 
 
\subsection{Захищати життя журналістів будуть якісні бронежилети українського виробництва, - НСЖУ}
\label{sec:27_04_2022.fb.tomilenko_sergej.1.broniki_nszhu}
 
\Purl{https://www.facebook.com/sergiy.tomilenko/posts/5052042531547237}
\ifcmt
 author_begin
   author_id tomilenko_sergej
 author_end
\fi

Захищати життя журналістів будуть якісні бронежилети українського виробництва,
- НСЖУ

У Києві відбулася передача 30 бронежилетів та 60 турнікетів для журналістів за
запитом Національної спілки журналістів України. Вони виготовлені благодійним
фондом «Захисти життя» бізнесмена Євгена Черняка. Захід пройшов у редакції
«Вечірнього Києва», де наразі НСЖУ облаштувала один з пунктів видачі
працівникам ЗМІ захисного спорядження, аптечок, жилетів «ПРЕСА» і інших
предметів на підтримку діяльності воєнкорів.

\ifcmt
  ig https://scontent-mxp1-1.xx.fbcdn.net/v/t39.30808-6/278576928_5052036358214521_5623723022384293623_n.jpg?_nc_cat=100&ccb=1-5&_nc_sid=730e14&_nc_ohc=-Rk4RwuVRVgAX_V3fNE&_nc_oc=AQn2hwZK68QyGACPxZacbdMKbfjErfXoJnb_02THRos8NEWTNGZa8aWPZZ-_TuKgIRI&_nc_ht=scontent-mxp1-1.xx&oh=00_AT8r2kpaLkMlVZwkD0c_lBa9RLMdj90ZnQsgXQzhWTGSwg&oe=626F35A3
  @wrap center
  @width 0.8
\fi

«Ми виготовляємо бронежилети класу захисту 4+ та безкоштовно передаємо їх
Збройним Силам України. Безумовно і журналісти для всіх нас дуже важливі, адже
завдяки вам світ дізнався про Бучу, Маріуполь, інші події війни. Ви працюєте у
небезпечних місцях і тому на звернення Національної спілки журналістів України
ми передаємо сертифіковані бронежилети та турнікети», - сказав директор БФ
«Захисти життя» Денис Кульчицький.

\ifcmt
  tab_begin cols=2,no_fig,center

     pic https://scontent-mxp1-1.xx.fbcdn.net/v/t39.30808-6/278608061_5052036368214520_4522268348692004230_n.jpg?_nc_cat=104&ccb=1-5&_nc_sid=730e14&_nc_ohc=9qPnWBiq7dAAX-xKV9G&_nc_ht=scontent-mxp1-1.xx&oh=00_AT83-9VzM-WuFeXuIMVinoXDq1zAv9BJN5Nw-BGLX8PsPw&oe=6270C01A

		 pic https://scontent-mxp1-1.xx.fbcdn.net/v/t39.30808-6/278582113_5052036501547840_2415486254695127699_n.jpg?_nc_cat=100&ccb=1-5&_nc_sid=730e14&_nc_ohc=kwVfhFrG3hYAX89u70x&_nc_ht=scontent-mxp1-1.xx&oh=00_AT9FAi_sva55-nja1499eN4V4C2ag4CN2keLkeqzwt0low&oe=626F0873

  tab_end
\fi

У фонді наголошують, що захист є надійним:

- Захисні пластини бронежилетів – зі сталі, яку ми загартовували, гнули,
роблячи ці бронепластини. Ми відстрілювали бронежилети, віддавали на балістичну
експертизу, самі зшивали плитоноски, - розповідає Денис Кульчицький. - Я
сподіваюся, вони вам не стануть у нагоді, але нехай краще вони будуть у вас,
захищаючи ваші життя.

Член правління НСЖУ, керівник КП «Київінформ» Євген Лопушинський подякував
представникам благодійного фонду за убезпечення журналістів та за проведений
інструктаж, а також передав отримані комплекти 14 журналістам за заявками їхніх
ЗМІ, Незалежної медіа профспілки. Також бронежилети надсилаються за запитами
журналістів у регіони.

Голова Національної спілки журналістів України Сергій Томіленко відзначає, що
завдяки громадянській позиції українського бізнесу, як і всього народу, Україна
дає гідну відсіч окупантам. Важливою є і журналістська солідарність, адже про
надання працівникам ЗМІ додаткових бронежилетів та турнікетів домовилася
керівник видання «ГОРДОН» Алеся Бацман:

- Журналісти фіксують воєнні злочини загарбників, викривають брехню
кремлівських пропагандистів, саме тому мішенями та жертвами атак росіян є
працівники ЗМІ, - наголошує голова НСЖУ Сергій Томіленко. – Наш спільний
обов’язок захистити і професію, і мужніх журналістів.

\ii{27_04_2022.fb.tomilenko_sergej.1.broniki_nszhu.cmt}
