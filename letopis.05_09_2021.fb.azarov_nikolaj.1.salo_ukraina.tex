% vim: keymap=russian-jcukenwin
%%beginhead 
 
%%file 05_09_2021.fb.azarov_nikolaj.1.salo_ukraina
%%parent 05_09_2021
 
%%url https://www.facebook.com/nikolayjanovich/posts/353432426520392
 
%%author_id azarov_nikolaj
%%date 
 
%%tags eda,ekonomika,import,kuhnja,salo,torgovlja,ukraina
%%title В Украину начали завозить сало
 
%%endhead 
 
\subsection{В Украину начали завозить сало}
\label{sec:05_09_2021.fb.azarov_nikolaj.1.salo_ukraina}
 
\Purl{https://www.facebook.com/nikolayjanovich/posts/353432426520392}
\ifcmt
 author_begin
   author_id azarov_nikolaj
 author_end
\fi

В Украину начали завозить сало.

Украина превратилась в импортера продукции животноводства. 

\ifcmt
  pic https://scontent-frx5-1.xx.fbcdn.net/v/t39.30808-6/241185932_353431736520461_4688007329844127332_n.jpg?_nc_cat=100&_nc_rgb565=1&ccb=1-5&_nc_sid=730e14&_nc_ohc=h3NNmyvdP3kAX_4_srN&_nc_ht=scontent-frx5-1.xx&oh=bc0f7d94965a94a7349b2f977a0096b7&oe=61392F10
  width 0.8
	name eda.salo.kiev
\fi

У нас, представьте себе, сало стали завозить. Важная часть традиционного
домашнего стола в Украине. Если не важнейшая. Продукт легендарный, воспетый в
местном устном народном творчестве. А когда вдруг это вкуснейшее лакомство
грозит исчезнуть из рациона, становится тревожно. 

Началось снижение поголовья скота. А еще есть проблемы в молочной сфере.

Падение собственного производства объясняют низкой рентабельностью выращивания
свиней, которая сокращается год от года. Дорожают комбикорма. Малый и средний
бизнес, связанный с животноводством, скоро совсем разорится, а украинцы
столкнутся с дефицитом сала и свинины. 

Сегодня украинцы недоедают свинины, по сравнению с нашими европейскими
соседями, а за последние пять лет потребление мяса уменьшилось на 7 килограммов
на душу населения, что составляет чуть более 30\%.

Так вот, еще по итогам 2016 года украинский экспорт свиного сала рухнул на
98\%. И власть ничего с этим не сделала. 

А теперь украинский рынок вовсю заполняет сало датское, польское да
нидерландское. 

Дореформировались!
