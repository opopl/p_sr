% vim: keymap=russian-jcukenwin
%%beginhead 
 
%%file 27_10_2018.stz.news.ua.mrpl_city.1.istoria_pokrovskaja_ploschad_mariupolja
%%parent 27_10_2018
 
%%url https://mrpl.city/blogs/view/istoriya-pokrovskaya-ploshhad-mariupolya
 
%%author_id burov_sergij.mariupol,news.ua.mrpl_city
%%date 
 
%%tags 
%%title История: Покровская площадь Мариуполя
 
%%endhead 
 
\subsection{История: Покровская площадь Мариуполя}
\label{sec:27_10_2018.stz.news.ua.mrpl_city.1.istoria_pokrovskaja_ploschad_mariupolja}
 
\Purl{https://mrpl.city/blogs/view/istoriya-pokrovskaya-ploshhad-mariupolya}
\ifcmt
 author_begin
   author_id burov_sergij.mariupol,news.ua.mrpl_city
 author_end
\fi

\ii{27_10_2018.stz.news.ua.mrpl_city.1.istoria_pokrovskaja_ploschad_mariupolja.pic.0}

От когда-то обширной Покровской площади осталась только ее часть. На картах
нашего города, изданных в 1988 и 1993 годах, эта часть именуется
Административной площадью. Вместе с тем здание \enquote{Гипромеза} имеет адрес:
проспект Мира, 68, а Мариупольский городской совет: проспект Мира, 70. О домах
на нечетной стороне и говорить не приходится, они изначально имели адреса
главной магистрали города. Там же - \textbf{Административная площадь}, а теперь
\textbf{Греческая площадь.}

Но обратимся в прошлое. Как и все дореволюционные городские объекты Мариуполя,
Покровская площадь получила свое официальное имя вовсе не случайно. Здесь
ежегодно 1 октября, на Покров день, устраивались ярмарки. Главным предметом
торга на них были фураж и зерно. Постепенно фуражом – кормом для скота – стали
торговать не только в ярмарочные дни.. Вероятно, с тех пор среди мариупольских
обывателей укрепилось, так сказать, народное наименование – \textbf{Сенная площадь.} 

Было у нее еще прозвание – \textbf{Выгон.} Этим названием пользовались те мариупольцы,
которые покинули наш бренный  мир глубокими стариками лет семьдесят назад. Они,
будучи детьми, видели, как в промежутках между ярмарочными днями жители
близлежащих кварталов выгоняли сюда на пастбище своих коров и лошадей. Кстати о
коровах, овцах и лошадях - на Сенной площади торговали ними даже в первые
послевоенные годы. Внук главного врача городской поликлиники Леонида
Михайловича Шапиро рассказывал, как дед приходил сюда, чтобы выбрать для
руководимого им учреждения лошадей. Ведь в послевоенное время весь транспорт
мариупольской медицины был на конной тяге. На линейках лошадки развозили
бригады скорой помощи, тяжелобольных в больницы.

\textbf{Читайте также:} 

\href{https://mrpl.city/news/view/novaya-grecheskaya-ploshhad-mariupolya-s-vysoty-ptichego-poleta-foto}{Новая Греческая площадь Мариуполя с высоты птичьего полета, Яна Іванова, mrpl.city, 19.09.2018}

Интересен Выгон еще и тем, что на площадке близ городской больницы в начале
второго десятилетия XX века между первыми мариупольскими  футбольными командами
проходили ожесточенные матчи.  А 19 октября 1910 года с площадки на  Выгоне
известный на всю Россию летчик Адам Габер-Влынский впервые в Мариуполе при
большом стечении публики поднял в небо аэроплан. Выгон, Сенная, или Покровская
площадь - как хотите, так и называйте - еще в начале 30-х годов занимал
территорию от нынешнего главного и самого старого корпуса Приазовского
государственного технического университета до технического училища на
Митрополитской улице. В некотором отдалении виднелись корпуса городской
больницы. То, что это было именно так, подтверждается планом Мариуполя,
составленным по топографической съемке 1930 года. 

Перед самой войной на Покровской площади построили здание четырехэтажного
общежития совпартшколы, возвели школу, а неподалеку от нее - многоквартирный
жилой дом, современный его адрес – проспект Металлургов, 56. Он стал первенцем
на четной стороне улицы Франко, переименованной позже в проспект Металлургов.
Нечетная сторона в виде приватных ломиков и амбаров, которые аборигены
Мариуполя называли хлебными ссыпками, существовала еще с царских времен. В
книге историка и краеведа Валентины Михайловны Зиновьевой \enquote{Мариуполь 1941 –
1943} приведена цитата из воспоминаний командира разведывательного батальона
Курта Майера, который первый ворвался в Мариуполь 8 октября 1941 года. Вот ее
содержание: \emph{\enquote{В отличие от других советских городов, на окраине Мариуполя были
высокие многоэтажные здания. Поскольку там не было не единого дерева или
маленького дома, резкая смена обстановки угнетала, даже подавляла}}. Что же
увидел гитлеровец, наступавший со стороны Мангуша? Названные выше школу, жилой
дом и общежитие совпартшколы, которое находилось на площадке по четной стороне
улицы, как раз против того места, где находится сейчас Дворец бракосочетаний,
встроенный  в  многоквартирный дом. В отдалении высилась так называемая
азовстальская пятиэтажка (ул. Энгельса, 39).  Действительно, на площади тогда
деревьев не было.

\textbf{Читайте также:}

\href{https://mrpl.city/news/view/mariupolskaya-ploshhad-stala-mestom-vstrech-i-svidanij-foto}{%
Мариупольская площадь стала местом встреч и свиданий, Анастасія Папуш, mrpl.city, 28.08.2018}

Отступая из города в сентябре 1943 года, оккупанты предали огню множество
мариупольских домов, в том числе и перечисленные выше строения, кроме общежития
совпартшколы, оно сгорело от точного попадания зажигательной бомбы. Враг был
изгнан из Мариуполя, и почти немедленно началось восстановление. Пожалуй,
первой на Сенной площади была возвращена к жизни школа. Ей присвоили №7. В наши
дни ее помещения занимает городской лицей. Постепенно были восстановлены и
другие упомянутые здесь здания. Лишь \enquote{погорелка} общежития  совпартшколы
некоторое время стояла неприкаянно на краю площади пока ее не снесли.

С начала пятидесятых годов площадь начала постепенно застраиваться жилыми, в
основном четырех-пятиэтажными домами. В первом этаже 75-го дома, если не
изменяет память, устроили кассу и диспетчерскую междугородной автостанции.
Проход к кассам более не существует, в нем устроили мастерскую по ремонту
чего-то. Пассажиры садились в автобусы, отправляющиеся в Донецк (тогда он
назывался Сталино), Ростов-на-Дону, Бердянск и близлежащие районные центры, на
том месте, где совсем недавно была остановка городского транспорта,
направляющегося в сторону Ильичевского и Орджоникидзевского районов.

По периметру площади к тому времени возникло много новых зданий, но сама она
сохраняла довольно непрезентабельный вид. Примерно там, где сегодня возвышается
\enquote{Гипромез}, сквозь заросли бурьяна проглядывали железобетонные глыбы
фундамента, начатого, но так и не построенного Дворца культуры. Пространство,
теперь занятое горсоветом, было огорожено дощатым местами покосившимся забором,
за которым располагались базы строительных организаций. Ближе к посадочной
площадке междугородных автобусов стояло деревянное строеньице закусочной, где
водители между рейсами могли перекусить свиной котлетой общепитовского
кулинарного искусства из молотого хлеба с добавлением мясного фарша.

\textbf{Смотрите также:} \href{https://mrpl.city/news/view/otkrytie-grecheskoj-ploshhadi}{Открытие Греческой площади, mrpl.city, 29.09.2018}

\ii{27_10_2018.stz.news.ua.mrpl_city.1.istoria_pokrovskaja_ploschad_mariupolja.pic.1}

В начале 60-х годов, прежде чем начать комплексную застройку бывшей Покровской
площади, по заданию архитекторов местного отделения \enquote{Гипрограда} ее
сфотографировали. Снимки эти сохранились. Они запечатлели мебельную фабрику,
которая располагалась на углу проспекта Ленина и улицы Франко, первые
многоквартирные дома на будущем проспекте Нахимова. Одиноко стоит Центральный
универмаг в окружении тарной базы, автомобильного гаража и бараков городской
тюрьмы.

\ii{27_10_2018.stz.news.ua.mrpl_city.1.istoria_pokrovskaja_ploschad_mariupolja.pic.2.univermag}
\ii{27_10_2018.stz.news.ua.mrpl_city.1.istoria_pokrovskaja_ploschad_mariupolja.pic.3.1966_god}

% 4 - Универмаг и дом со сказкой
\ii{27_10_2018.stz.news.ua.mrpl_city.1.istoria_pokrovskaja_ploschad_mariupolja.pic.4}

По проекту застройки образца 1967 года предусматривалось строительство на
площади  четырнадцатиэтажного здания для гостиницы - его потом занял \enquote{Гипромез}
- здание городского совета и концертный зал с куполообразной крышей. Все это к
началу 80-х годов было осуществлено, за исключением концертного зала. К тому
времени площадь приобрела тот вид, который она имела, кажется совсем недавно.
Ее наиболее живописные уголки сняты для почтовых открыток. А телевизионщики
часто использовали вид с крыши Гипромеза в своих произведениях, а иногда
выносили на заставки передач. Символично, что рядом с городским советом была
построена детская площадка. Размеренная жизнь на бывшей Покровской площади
продолжалась, пока не произошли трагические события весны 2014 года...

\textbf{Читайте также:} 

\href{https://mrpl.city/news/view/mariupoltsy-uznali-kak-budet-vyglyadet-sgorevshij-gorsovet-posle-rekonstruktsii-fotofakt}{Мариупольцы узнали, как будет выглядеть сгоревший горсовет после реконструкции, mrpl.city, 02.10.2018}

Сейчас на Греческой площади (таково современное название бывшей Покровской
площади) произведены масштабные работы по ее реконструкции. Но это уже другая
история...
