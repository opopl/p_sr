% vim: keymap=russian-jcukenwin
%%beginhead 
 
%%file 26_01_2022.fb.fb_group.story_kiev_ua.2.terskol
%%parent 26_01_2022
 
%%url https://www.facebook.com/groups/story.kiev.ua/posts/1847988272064632
 
%%author_id fb_group.story_kiev_ua,zolotushkin_anatolij.hajfa
%%date 
 
%%tags gory,kavkaz,kiev,sssr
%%title Терскол
 
%%endhead 
 
\subsection{Терскол}
\label{sec:26_01_2022.fb.fb_group.story_kiev_ua.2.terskol}
 
\Purl{https://www.facebook.com/groups/story.kiev.ua/posts/1847988272064632}
\ifcmt
 author_begin
   author_id fb_group.story_kiev_ua,zolotushkin_anatolij.hajfa
 author_end
\fi

Терскол.

Моя мама всегда хотела, чтобы я стал человеком. Семья и две школы не смогли с
этим справиться. Даже Советская армия не помогла. Я вернулся оттуда слегка
ошеломлённым, но в человека так и не превратился.
\ii{26_01_2022.fb.fb_group.story_kiev_ua.2.terskol.pic.1}

Оставалась единственная надежда - горы. Вслед за Снежным Человеком и Алитетом
именно туда потянулась интеллигенция средней руки. Опять же Высоцкий из каждого
утюга пел про скалолазок.

Однажды мама проходила мимо Дома офицеров на Печерске и прочитала объявление,
что горы зовут в Терскол, на лыжную базу министерства обороны и можно за
наличный расчет там оказаться. Очевидно, среди военных романтиков было немного.

Так я попал в горы Кавказа. В моем номере жили лётчики. Они приехали не одни. У
каждого с собой было по канистре спирта, но он удивительно быстро скончался.
Поэтому, когда альпинисты и примкнувшие к ним девушки штурмовали снежные
вершины, нам приходилось спускаться вниз к людям для закупки ароматной
грузинской водки и шашлычной закуски. За срок отпуска мы явно прошли больше
километров вниз, чем те, кто ходил вверх. Шашлыки в те далёкие годы почему-то
жарили на солярке и мы подозревали, что при их изготовлении ни один баран не
пострадал.

Когда срок отпуска подошёл к концу, особо злостных нарушителей решили примерно
наказать. Нас вызвали к командиру базы отдыха и сообщили, что по нашему месту
службы направят обличительные письма. Я с радостью сообщил номер части, в
которой отслужил год назад. Пусть порадуются в далёком Заполярье, что армейская
школа не прошла даром.

\ii{26_01_2022.fb.fb_group.story_kiev_ua.2.terskol.cmt}
