% vim: keymap=russian-jcukenwin
%%beginhead 
 
%%file 14_07_2020.fb.fb_group.story_kiev_ua.1.bajka_bessarabki
%%parent 14_07_2020
 
%%url https://www.facebook.com/groups/story.kiev.ua/posts/1402173703312760/
 
%%author_id fb_group.story_kiev_ua,petrova_irina.kiev
%%date 
 
%%tags gorod,kiev,rasskaz
%%title Байка Бессарабки
 
%%endhead 
 
\subsection{Байка Бессарабки}
\label{sec:14_07_2020.fb.fb_group.story_kiev_ua.1.bajka_bessarabki}
 
\Purl{https://www.facebook.com/groups/story.kiev.ua/posts/1402173703312760/}
\ifcmt
 author_begin
   author_id fb_group.story_kiev_ua,petrova_irina.kiev
 author_end
\fi

Байка Бессарабки

Эту историю рассказал мой свёкр Всеволод Николаевич  про своего деда.

Гнат Григорович ЛопатЫна (так было правильное ударение) был  точильщиком.

Его мастерская была  неподалеку от Бессарабского рынка.

Точил он всё - ножи, косы, сапки и прочий крам. С годами стал
совершенствоваться, точил даже скальпели для хирургов, тонкие ножнички
маникюршам, прочие деликатные изделия.

\ifcmt
  tab_begin cols=2

     pic https://scontent-frx5-1.xx.fbcdn.net/v/t1.6435-9/109831465_3405485239485059_661562181056562240_n.jpg?_nc_cat=110&ccb=1-5&_nc_sid=b9115d&_nc_ohc=K-NfJBINKpcAX-WaKkq&_nc_ht=scontent-frx5-1.xx&oh=c0dc1deea439ed06157c1807ae2dc355&oe=61B83814

     pic https://scontent-frx5-2.xx.fbcdn.net/v/t1.6435-9/107841182_3405485466151703_236224606854823450_n.jpg?_nc_cat=109&ccb=1-5&_nc_sid=b9115d&_nc_ohc=UEWEtkqXcuAAX9Z75uU&tn=lCYVFeHcTIAFcAzi&_nc_ht=scontent-frx5-2.xx&oh=06adc2ea0ee24b9ccb941375f7d185b5&oe=61B81772

  tab_end
\fi

Стал известным, зарабатывал  хорошо, заказчики текли рекой.

Как-то пришел заказчик, представился доктором,  хирургом, объяснил, что нужно
заточить кусачки, маленькие, использует их при операции на костях. Главное
требование - чтобы они при перекусе нитки не щелкали, т.е. заточить особым
образом кромки, чтобы не стукались друг об друга, а мягко соприкасались.

Гнат Григорович немало потрудился над заказом. Заказчик пришел, проверил на
нити, на тонкой проволочке, которой скреплялись переломанные кости, на полоске
толстой грубой кожи. Работали кусачки отменно.

Заказчик похвалил мастера, расплачивался не скупясь, пожал руку, откланялся.

Довольный Гнат Григорович почувствовал, что пора уж и отобедать, решил
взглянуть на свой  точный хронометр Omega, но... из карманчика жилета торчал
маааахонький обрывочек толстой золотой цепочки.

Зато кусачки не щелкнули!

Еще к этой истории: дочери Гната Григоровича очень переживали из-за
неблагозвучности своей фамилии, они были воспитанницами Института благородных
девиц. И решили преобразовать фамилию:  стали мадемуазель Мария ЛопАтина,
Лидия Лопатина и Антонина Лопатина.

Баба Муся (в прошлом мадемуазель Мария) пошла дальше и стала Липатиной.

Вышла замуж, как доверительно и строго конфиденциально сообщала всем,  за сына
от морганатического брака Государя императора, но... это уже другая байка.

Мадемуазель Антонина удачно вышла замуж за орловского мещанина, у неё родился
очаровательный сыночек Севушка.

Мадемуазель Лидия очень любила вышивать. Вышивание стало делом всей её жизни,
была удостоена звания Заслуженного мастера народного творчества УССР.

Покоятся все на Байковом.

Царство им Небесное!

\ii{14_07_2020.fb.fb_group.story_kiev_ua.1.bajka_bessarabki.cmt}
