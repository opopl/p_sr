% vim: keymap=russian-jcukenwin
%%beginhead 
 
%%file 26_12_2021.fb.fb_group.story_kiev_ua.1.jelochnye_igruski.cmt
%%parent 26_12_2021.fb.fb_group.story_kiev_ua.1.jelochnye_igruski
 
%%url 
 
%%author_id 
%%date 
 
%%tags 
%%title 
 
%%endhead 
\zzSecCmt

\begin{itemize} % {
\iusr{Vadim V. Bandourin}
Ага. Хранятся ещё такие. Некоторые буквально - такие же!

\iusr{Алла Беркова}
У меня были такие игрушки

\iusr{Петр Кузьменко}

А наши легендарные игрушки детства (и мои и Нелины) ждут нас дома, на Подоле.
Бережно упакованные в два больших фанерных ящика, которые сами по себе уже
исторические раритеты, они хранятся на антресоли подольской квартиры. Уже
второй год, в связи с пандемией и сопутствующими неудобствами, мы не встречаем
Новый Год и Рождество на любимом Подоле, с детьми, внуками и близкими друзьями.
Всё, что с нами здесь, это ГДР-овская искусственная ель, приобретённая моими
родителями в 1977 году после переезда с Андреевского спуска на Автозаводскую.
Ещё, как и много лет жизни и службы на Дальнем Востоке, с нами старенький Дед
Мороз, купленный в Киеве в 50-е годы прошлого века, ещё Нелиными дедушкой и
бабушкой. @igg{fbicon.snowflake} @igg{fbicon.hands.shake}  @igg{fbicon.heart.sparkling} 

\ifcmt
  ig https://scontent-frx5-1.xx.fbcdn.net/v/t39.30808-6/270028500_4869322753119970_3107114138548323990_n.jpg?_nc_cat=105&ccb=1-5&_nc_sid=dbeb18&_nc_ohc=MfVBB8fDXKEAX-i1ixW&_nc_ht=scontent-frx5-1.xx&oh=00_AT_kWt_FZkw-Xw2vL0q1_K2SNBsqezD0Nhb0K8YPKN66VQ&oe=61CDA005
  @width 0.3
\fi

\begin{itemize} % {
\iusr{Алла Беркова}
\textbf{Петр Кузьменко} мы тоже жили на Подоле. По улице Ниже Юрковская напротив кирпичного завода. Жаль только игрушки нас не ждут

\iusr{Светлана Манилова}
\textbf{Петр}, я уже несколько ночей во сне брожу по своей подольской квартире... И мне кажется, что чувствую запах елки и вижу блеск еще новых тогда игрушек. И, конечно, жду Деда Мороза...@igg{fbicon.heart.red}

\begin{itemize} % {
\iusr{Алла Беркова}
\textbf{Светлана Манилова} а на какой улице ?

\iusr{Светлана Манилова}
\textbf{Алла}, улица Хорива. @igg{fbicon.smile} 

\iusr{Алла Беркова}
\textbf{Светлана Манилова} да не далеко

\iusr{Петр Кузьменко}

Светлана, мне сегодня ночью, тоже, приснилась наша подольская квартира на
Почайнинской. И любимый пёс, такса Арис, который так славно играл со мной во
сне. Мы бегали, прыгали и дурачились. Давно нет с нами Ариса, ещё более долго я
не могу бегать и прыгать. А во сне совершаются чудеса. Ещё, в последнее время,
мне всё чаще снится наша комнатушка в коммуналке на Андреевском спуске и
настоящие зимы детства, в сердце Подола и родного Города. Пусть с нами всегда
будут яркие сны и воспоминания счастливых времён! Будем верить в новогодние
чудеса! @igg{fbicon.hands.shake}  @igg{fbicon.heart.sparkling} @igg{fbicon.snowflake}

\iusr{Алла Беркова}
\textbf{Петр Кузьменко} наше детство настоящая зима запах елочки всегда останутся с нами. Хорошего и веселого Праздника 🤶 @igg{fbicon.santa.claus}  @igg{fbicon.christmas.tree} 

\iusr{Светлана Манилова}
\textbf{Петр}, 

поскольку мы пока не можем встретиться, надеюсь, мы встретимся во снах на нашем
любимом Подоле и в нашей 100-й школе! Предлагаю новогодние наряды не менять. Ты
будешь всё так же охотником, а я снежинкой... А Неллочка, по- моему, была
Лисичкой? Или я ошиблась? @igg{fbicon.smile} 


\iusr{Петр Кузьменко}
\textbf{Светлана}! Для нас Хоревая! Была и будет. \enquote{Чинно-благородно. По-старому!}, как говорил герой Вицина в любимом гайдаевском \enquote{Не может быть.}

\iusr{Петр Кузьменко}
\textbf{Алла}! Взаимно! Здоровья и счастья, мира и добра!

\iusr{Ирине Вильчинская}

как интересно! и мне сегодня снился мой дом на Зверинце, которого уже
нет... наверное, места нашего детства тоже скучают по нам и приходят к нам уже
только во сне... до слез!


\iusr{Светлана Манилова}
\textbf{Ирине}, предполагаю, с годами мы все чаще думаем о них.@igg{fbicon.heart.red}
\end{itemize} % }

\iusr{Петр Кузьменко}
\textbf{Светлана}, точно! Именно лисичкой. Как добыл этот \enquote{трофей} тогда, так и наслаждаюсь им всю жизнь. @igg{fbicon.laugh.rolling.floor} 
\end{itemize} % }

\iusr{Олена Коноваленко}
ІГРАШКИЄ, Я НИМИ КОРИСТУЮСЬ І ЗГАДУЮ ДИТИНСТВО

\ifcmt
  ig https://scontent-frt3-2.xx.fbcdn.net/v/t39.30808-6/270059489_777321243664438_7124982836541176379_n.jpg?_nc_cat=101&ccb=1-5&_nc_sid=dbeb18&_nc_ohc=Ji9LTQdGSdcAX8fHZWq&_nc_ht=scontent-frt3-2.xx&oh=00_AT-kSFnLZvkdwJB0YDY2dI7MCFUg3M8qA9N_9eLT9D1-YA&oe=61CE05E9
  @width 0.2
\fi

\iusr{Nina Ivanco}

А у меня в прошлом году разбилась моя любимая игрушка, моя ровесница. Даже
пыталась найти на Петровке и нашла, но увы не купила, оказывается это очень
редкая коллекционная игрушка и цена заоблачная. Решила пойду куплю современную
стеклянную она в десять раз дешевле.

Для каждого поколения новогодняя игрушка своя. У моей дочки девочка ангел
которую я купила для её когда ей было пять лет, сейчас дочкиному сыну столько
же, а ангелочек весит на почетном месте каждый год. Всех поздравляю с
новогодними праздниками!

\begin{itemize} % {
\iusr{Светлана Манилова}
\textbf{Nina}, спасибо! Здоровья Вам и Вашим родным и новогодних чудес! @igg{fbicon.smile} 

\iusr{Светлана Власенко}
\textbf{Nina Ivanco} что же это за игрушка такая?
\end{itemize} % }

\iusr{Раиса Карчевская}

\ifcmt
  ig https://scontent-frx5-1.xx.fbcdn.net/v/t39.30808-6/270106259_1255961678242696_3981721138681207297_n.jpg?_nc_cat=105&ccb=1-5&_nc_sid=dbeb18&_nc_ohc=DdS5p8RYx9YAX-jrc9g&_nc_ht=scontent-frx5-1.xx&oh=00_AT8Bagfdgy1HrRGjxcKTraQ9ej9mbwxgDGPBYdXg8O5ySw&oe=61CD36F1
  @width 0.3
\fi

\iusr{Валентина Кравченко}

А стеклянные бусы сегодня достала и вспомнилаааааа

\iusr{Владимир Аксенов}

Очень тепло напомнили о прошлом, которое уже никогда не вернуть. С возрастом
стал слишком сентимнтальным, читал и глаза стали влажные, если не сказать
больше.


\iusr{Светлана Манилова}
\textbf{Владимир}, спасибо! А я вот боюсь с годами, чтоб не растерять свою сентиментальность. @igg{fbicon.smile} 

\iusr{Микола Веселий}

Помню их толстенькие, крепенькие. Не то, что сейчас - китайские. И хлопушки
были качественные и бенгальские огни, безотказно горели и хлопали  @igg{fbicon.smile} 

\begin{itemize} % {
\iusr{Светлана Манилова}
\textbf{Микола}, я по части \enquote{гореть} и \enquote{хлопать} была всегда не очень..., а вот игрушки радовали всегда.  @igg{fbicon.face.partying} 

\iusr{Микола Веселий}
\textbf{Светлана Манилова} Ну, я тоже хлопушками не очень хлопать любил, а вот как бенгальские огни горят любил долго смотреть. И запах от них такой приятный помню. А еще на елку- дождик, вату и конфети бросали  @igg{fbicon.smile} 

\iusr{Светлана Манилова}
\textbf{Микола}, я до сих пор боюсь хлопка открывающегося шампанского, поэтому и хлопушки не зашли. Огни были более приятны, конечно. @igg{fbicon.smile} 
\end{itemize} % }

\iusr{Светлана Ветрова}
Дуже душевно описано іграшки на фото в мене були а деякі ще є всіх зі святом

\iusr{Валентина Кравченко}
\textbf{Светлана Ветрова} девчонки у их витаю

\iusr{Светлана Манилова}
\textbf{Светлана}, спасибо! Хорошего Нового Года!

\iusr{Светлана Федорик}
А этот шар мне достался от бабушки, ему больше 50 лет @igg{fbicon.christmas.tree} 

\ifcmt
  ig https://scontent-frt3-2.xx.fbcdn.net/v/t39.30808-6/269944631_426349762518820_239221279527378962_n.jpg?_nc_cat=103&ccb=1-5&_nc_sid=dbeb18&_nc_ohc=CAAmtPN8bhMAX-XQgfP&_nc_ht=scontent-frt3-2.xx&oh=00_AT_7DSkKPMCr5dendOo8-Gan6r9rwbBZiXnYlmdyc5msww&oe=61CD4C95
  @width 0.3
\fi


\iusr{Светлана Гурневич}
Да игрушки детства греют  @igg{fbicon.heart.beating}  они живые !

\ifcmt
  ig https://scontent-frx5-2.xx.fbcdn.net/v/t39.30808-6/269692854_953489325571445_4876141621202338336_n.jpg?_nc_cat=109&ccb=1-5&_nc_sid=dbeb18&_nc_ohc=ACb5zigraToAX_89bbH&_nc_ht=scontent-frx5-2.xx&oh=00_AT85UoC6h1GP3uPBFt4Akn6_dGGKuadln5tBY8zaNpD_xA&oe=61CE57F8
  @width 0.3
\fi

\begin{itemize} % {
\iusr{Светлана Манилова}
\textbf{Светлана}, красота! @igg{fbicon.heart.eyes} 

\iusr{Светлана Гурневич}
\textbf{Светлана Манилова} жаль уходят по чуть-чуть ( то вдруг елка  @igg{fbicon.christmas.tree}  упадёт, то по древности) таких душевных сейчас нет, есть красивые...

\iusr{Светлана Гурневич}
Благо дарю @igg{fbicon.snowflake} @igg{fbicon.christmas.tree} @igg{fbicon.snowflake} @igg{fbicon.christmas.tree} 
\end{itemize} % }

\iusr{Kate Levenko}

Ось моя колекція від бабусі і дідуся, цього року дуже старі навіть і не
діставали, окрім ватяних мухоморів  @igg{fbicon.smile}, хочу якось зробити для них спеціальну
шафку і зберігати просто як колекцію на поличці, деякі іграшки ще з 50-х. Бо
щоразу б'ються, якщо використовувати за призначенням, дуже жаль  @igg{fbicon.frown} 

\ifcmt
  ig https://scontent-frt3-1.xx.fbcdn.net/v/t39.30808-6/270113705_6784107781664093_1873288399126452010_n.jpg?_nc_cat=108&ccb=1-5&_nc_sid=dbeb18&_nc_ohc=Z6YEKHq3OxoAX-Kfe9X&_nc_ht=scontent-frt3-1.xx&oh=00_AT_UndsQ-1qj81fjjrt6nRLcm7nuXk-ceRfEMab7FKxgIA&oe=61CEBAA7
  @width 0.4
\fi

\begin{itemize} % {
\iusr{Светлана Гурневич}
\textbf{Kate Levenko} впізнаю знайомі @igg{fbicon.hands.applause.yellow}  бережіть  @igg{fbicon.christmas.tree} 

\iusr{Алла Беркова}
\textbf{Kate Levenko} вот это дааа раритет винтаж . Спасибо что сохранили и напомнили мои игрушки все точно

\begin{itemize} % {
\iusr{Kate Levenko}
\textbf{Алла Беркова} 

Було 3 ящики, лишився 1  @igg{fbicon.frown}  Колись накупив іграшок, солодощів цілу коробку,
поставив на поріг і попросив когось з сусідів в двері подзвонити, там було на
коробці написано \enquote{Катюші від діда Мороза}, і хоч я здогадувалась, що то
просто від діда, бо почерк був його  @igg{fbicon.wink}  але спогади дуже приємні  @igg{fbicon.smile}  Дідусь дуже
любив свята і скупав всякі красиві іграшки все життя  @igg{fbicon.smile} 


\iusr{Алла Беркова}
Очень душевно и трогательно
\end{itemize} % }

\iusr{Светлана Манилова}
\textbf{Kate}, прекрасная коллекция! Берегите их, пусть радуют еще долгие годы!

\end{itemize} % }

\iusr{Александр Сандул}

\ifcmt
  ig https://scontent-frx5-1.xx.fbcdn.net/v/t39.30808-6/269852047_1010533786196150_7085803099300897807_n.jpg?_nc_cat=110&ccb=1-5&_nc_sid=dbeb18&_nc_ohc=f603_g07BmMAX_ZQRqj&_nc_ht=scontent-frx5-1.xx&oh=00_AT_wEsOFZc8s53dzn2NicKOEAYsnLm-fQoQAuSEZXyfStw&oe=61CDECC5
  @width 0.3
\fi

\iusr{Наталія Гуменюк-Гуральська}

\ifcmt
  ig https://scontent-frt3-1.xx.fbcdn.net/v/t39.30808-6/269768709_6915696525168537_8974357060169007381_n.jpg?_nc_cat=102&ccb=1-5&_nc_sid=dbeb18&_nc_ohc=FVMB2Mb55u0AX-9oSX2&_nc_ht=scontent-frt3-1.xx&oh=00_AT8_3b3pfJE1T19Y-wQ6kfNdwApdJ5hAXee_7D7GYVCVJQ&oe=61CCEE3C
  @width 0.4
\fi

\iusr{Светлана Манилова}
\textbf{Наталія}, просто радует глаз! @igg{fbicon.heart.eyes} 

\iusr{Надежда Владимир Федько}
Спільними зусиллями вже можемо створити історію ялинкових прикрас)

\iusr{Инна Косянчук}
Як я любила іграшки тих часів,!

\iusr{Igor Volod}
Помню

\iusr{Ирина Дубчак}
и у меня такие есть... память

\iusr{Ковальская Татьяна}

Завтра я открою коробку с игрушками и буду наряжать ёлку, встречаю уже 85 раз в
жизни, уже нет дудочки моего отца, нет светящегося клоуна, но ещё есть
дореволюционные шары, большие часы и виноград память о моей тете, умершей на
лесоповале!

\begin{itemize} % {
\iusr{Алла Беркова}
\textbf{Ковальская Татьяна} 

пусть в вашей жизни еще будет украшаться еще и еще много елочек. Будьте
здоровы. С наступающим Новым Годом  @igg{fbicon.santa.claus} 🧑
@igg{fbicon.christmas.tree}{repeat=2} 


\iusr{Светлана Манилова}
\textbf{Ковальская Татьяна}, желаю Вам наряжать елку и встречать Новый Год еще долгие годы!
\end{itemize} % }

\end{itemize} % }
