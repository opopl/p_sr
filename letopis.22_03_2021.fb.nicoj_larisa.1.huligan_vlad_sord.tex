% vim: keymap=russian-jcukenwin
%%beginhead 
 
%%file 22_03_2021.fb.nicoj_larisa.1.huligan_vlad_sord
%%parent 22_03_2021
 
%%url https://www.facebook.com/nitsoi.larysa/posts/884688148764225
 
%%author 
%%author_id 
%%author_url 
 
%%tags 
%%title 
 
%%endhead 
\subsection{Беру на поруки «хулігана», який розбив вікно в Офісі Президента}
\Purl{https://www.facebook.com/nitsoi.larysa/posts/884688148764225}

Ще зранку, подивившись світлини розписаного офісу з розтрощеним склом, я була
проти вандалізму. Це НАШЕ добро, а не котрогось із президентів. Українське
добро треба берегти… Ще зранку я обговорювала в одній з патріотичних груп, що
всі мирні протести себе вичерпали, але й такі дії мені не подобаються, то якими
вони повинні бути, протести?..

\ifcmt
  pic https://scontent-iad3-1.xx.fbcdn.net/v/t1.6435-0/p180x540/164008038_884688078764232_8243472883665951967_n.jpg?_nc_cat=100&ccb=1-3&_nc_sid=730e14&_nc_ohc=Idj-LnWdmykAX_RaBqi&_nc_ht=scontent-iad3-1.xx&tp=6&oh=6d05268e345e2f4ba7c3fd78532d2d3b&oe=60CA5E36
\fi

Увечері, дізнавшись, хто затриманий – без вагань беру його на поруки і прошу
підтримки українців, а хто проти, будь ласка, просто промовчте. 

Знаю особисто затриманого Влада Сорда. Талановитий поет, прозаїк, видавець.
Позитивний. Добрий. Патріот України. Пише неймовірні книжки. Аби в нас була
країна, яка має гуманітарну політику, його б книжку «Безодня» обговорював би
кожен канал ТБ, кожне радіо. Такої прози в нас ще не було. 

Ветеран війни, який побував у Іловайському котлі, єдиний, хто врятувався від
вибуху серед своїх побратимів, які загинули, але був поранений, паралізований,
заново учився ходити… 

Повернувшись до мирного життя, знайшов своє кохання і вже разом з коханою
дружиною, письменницею Вікторією Гранецькою, створили у Вінниці неймовірно
цікаве видавництво «Дім химер». Стали видавати високоякісну прозу сучасних
українських авторів.

Україна, на жаль, - країна, яка мало читає і мало купує книги. Попри складну
для видавництва фінансову ситуацію, Влад з Вікторією ухвалюють рішення 10\% від
проданих книг перераховувати дружині Ріфмайстра, яка опинилася ще у більшій
скруті, ніж вони. Її звільнили з роботи після  походів по судах на підтримку
чоловіка. Влад постійно приїздив з Вінниці в Київ на акції підтримки Ріфа.

Приїхав і цього разу. Мав квиток на вечірній потяг додому… 

Дізнавшись про арешт Влада Сорда, я одразу зателефонувала його дружині, моїй
подрузі Вікторії Гранецькій. Колись я прочитала роман «Тіло» і була впевнена,
що такий зрілий і «жорсткий» (не плутати з жорстоким) текст міг написати лише
чоловік. Яке ж було моє здивування від знайомства з автором, коли виявилося, що
роман написала маленька, тоненька, очі на пів обличчя, неймовірно талановита
дівчина, лауреатка міжнародних літературних конкурсів. Сьогодні ця дівчина –
дружина Влада. Вона дуже турбується про нього. Каже, що Влад кожного ранку
страждає на дуже сильні головні болі, вони постійно звертаються до лікарів, і
п’є від цього ліки.  Як він там витримує цей біль без ліків в ІТТ?  І разом з
тим пожартувала: «Заберу його додому, він в мене ще попроситься назад в
ізолятор».

Я знаю одне, друзі, дуже легко засуджувати на відстані. Але я можу лише уявити,
як допекла ситуація, якщо не витримав інтеліґент і письменник Влад Сорд.

Якщо для вас моє слово щось важить, допоможіть витягти Сорда. Його треба взяти
на поруки і відправити додому. Це звичайне хуліганство, адміністративне
порушення, а не злочин, який йому шиють на 7 років тюрми.
