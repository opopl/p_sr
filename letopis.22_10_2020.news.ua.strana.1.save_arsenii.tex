% vim: keymap=russian-jcukenwin
%%beginhead 
 
%%file 22_10_2020.news.ua.strana.1.save_arsenii
%%parent 22_10_2020
%%url https://strana.ua/articles/rassledovania/296371-kto-nastojashchaja-zhertva-a-kto-prestupnik-v-dele-arsenija-nevhodovskoho.html
 
%%endhead 

\subsection{#СпаситеАрсения. Что известно об истории подростка, который напал с ножом на тетю, и почему его защищают}
\label{sec:22_10_2020.news.ua.strana.1.save_arsenii}

\url{https://strana.ua/articles/rassledovania/296371-kto-nastojashchaja-zhertva-a-kto-prestupnik-v-dele-arsenija-nevhodovskoho.html}

Анастасия Товт 10:43, сегодня 

\ifcmt
img_begin 
	tags arsenii,crime
	url https://strana.ua/img/article/2963/71_main.jpeg
	caption Подробный разбор дела подростка Арсения Невгодовского, который напал с ножом на тетю и ее сына, а затем обвинил свою мать и тетю в домашнем насилии 
	width 0.7
img_end
\fi

В соцсетях набирает обороты хештег #СпаситеАрсения. Это история 15-летнего
Арсения Невгодовского, который долгое время жил в семье своей тети Юлии
(сестры матери). В марте 2020 года подросток напал на нее и ее сына с ножом. К
счастью, травмы оказались не смертельными - и тетя, и ее сын выжили, хоть и
получили серьезные ранения. 

Арсения отправили в СИЗО, обвинив в покушении на убийство. Но вскоре семья его
одноклассника приняла Арсения на время круглосуточного домашнего ареста.

Последние полгода Арсений живет у них. Его мать, Елена Невгодовская,
утверждает, что мальчик психически неуравновешенный, и настаивает на том, чтобы
его отправили в психбольницу. 

Но за то время, которое мальчик провел в патронатной семье, история приняла
новый неожиданный оборот. Опекуны утверждают, что Арсений психически здоров и
адекватен, а на преступление пошел из-за того, что мать и тетя с детства
издевались над ним. А теперь женщины якобы хотят замять дело о домашнем
насилии, упрятав подростка в сумасшедший дом. 

Даже отец мальчика, на которого Арсений напал с ножом, принял сторону
подростка: он публично оправдывает его поступок и говорит, что простил его.

“Страна” решила разобраться в семейной драме Алексея Невгодовского и собрала
мнения всех сторон конфликта.

\subsubsection{Версия тёти Арсения: "Мальчик-аутист был вовлечён взрослыми людьми в преступление"}

Арсений Невгодовский жил в доме своей тети, Юлии, с 13 лет. 1 марта 2020 года
Арсений, по словам Юлии Невгодовской, взял нож и напал на нее и ее сына. Они
жили втроем в Крюковщине Киево-Святошинского района.

Юлия Невгодовская записала видео-обращение к президенту Украины, где рассказала
свою трактовку событий того дня, когда на нее напал племянник Арсений. 

Юлия полагает, что на такой шаг подростка подговорил отец ребенка Юлии и ее
бывший сожитель - Святослав Козырев, с целью присвоить себе ее имущество. А
самого Арсения тетя называет "мальчиком-аутистом".

Мой племянник, мальчик-аутист, был вовлечён взрослыми людьми в страшное
преступление --- попытку убийства меня и моего маленького ребёнка. Я обоснованно
подозреваю в случившемся отца моего ребёнка, который незадолго до организации
моего убийства, переоформил по поддельной доверенности мое имущество на своих
подельников.

Для того чтобы замести следы своего преступления, он и подговорил моего
племянника --- аутиста  на преступление, зная, что никакой ответственности такой
подросток не понесёт. Племянник напал на нас с маленьким сыном с ножом, и нанёс
нам десятки ударов. Мы 100\% должны были умереть, но чудом выжили", - говорит
Юлия Невгодовская в видео-обращении. 

Помимо грабежа, Юлия Невгодовская обвиняет экс-сожителя в том, что тот украл у
нее ребенка, когда она лежала в реанимации после нападения. Таким образом,
параллельно с делом Арсения разворачивается еще одна семейная драма. Но к этому
вопросу мы еще вернемся позднее. 

Пока что разберемся, что об этом говорит сам Арсений Невгодовский. 

\subsubsection{Версия Арсения: “Я пострадавший”}

Арсений утверждает, что Святослав Козырев здесь вообще ни при чем.

Мальчик объясняет, что стал жертвой домашнего насилия. По его словам, тетя и
мать систематически издевались над ним с самого детства - и физически, и
морально.

В какой-то момент мальчик просто не выдержал и не нашел другого выхода, кроме
как напасть на тетю с ножом. Он утверждает, что это была попытка защитить себя. 

“Я пострадавший”, - считает мальчик. 

\ifcmt
pic https://strana.ua/img/forall/u/0/25/%D0%BD%D0%B5%D0%B2%D0%B3%D0%BE%D0%B4.jpg
\fi

Арсений утверждает, что напал на свою тетю и двоюродного брата в тот день,
когда Юлия Невгодовская била его шнуром от компьютера. 

“...Я просто не мог выдержать эту боль и бросился на вас с ножом. Я хотел не
жить в семье, я не мог терпеть унижения и то, что вы с матерью меня били... Я
хотел решить все без насилия, но я не знал, как”, --- рассказал он.

Арсений признает, что после СИЗО он попал в нормальную семью. Добавляет, что
если бы мог отмотать время назад, “ни на кого не бросался бы с ножом”.

“Надо мной издевались всю жизнь, и в итоге я попытался себя защитить своими
силами, из-за чего сразу же попал в тюрьму. А что будет с теми, кто мучил меня
годами? Ничего?”, - спрашивает подросток.

Он через Фейсбук просит помочь восстановить дело против тёти и добиться решения
суда против мамы. 

“Я не хочу никакого приговора или наказания для них. Я хочу, чтобы было
доказано, что я говорю правду. Чтобы люди поняли, что я не только монстр,
напавший на свою тётю, а что я ещё и жертва, которая пыталась себя защитить”, -
пишет Арсений на своей странице в Фейсбук. 

\subsubsection{Версия опекунов Арсения}

Полгода назад Олег Шиндер и Арсений Невгодовский впервые увидели друг друга в
зале заседаний Киево-Святошинского суда. С тех пор они живут вместе. 

Олег Шиндер - отец одноклассника Арсения. Узнав о том, что подросток совершил
преступление и оказался в СИЗО, а родственники Невгодовского фактически от него
отказались, Олег Шиндер решил помочь ему. 

“Моя жена об этой истории узнала на родительском собрании в школе”, -
вспоминает в разговоре со “Страной” глава семьи Олег Шиндер. Арсений -
одноклассник его младшего сына. 

“Жена после родительского собрания позвонила мне и рассказала, что Арсений
совершил преступление, парня закрыли в СИЗО, потому что ему негде отбывать
домашний арест - мать сказала, что не хочет его больше видеть. Тогда я сказал
жене: “Подойди к учительнице и скажи, что мы готовы принять Арсения”, -
рассказывает “Стране” Олег Шиндер. 

Олег Шиндер пришел на следующее судебное заседание по делу Невгодовского. После
короткого обмена репликами судья спросил у Арсения, не против ли он пожить в
семье Шиндеров, и тот согласился. Так Арсению изменили меру пресечения на
круглосуточный домашний арест.

\ifcmt
img_begin 
	tags oleg shinder,arsenii nevgodovskii,save_arsenii
	url https://strana.ua/img/forall/u/0/25/%D0%BE%D0%BB%D0%B5%D0%B3_%D1%88%D0%B8%D0%BD%D0%B4%D0%B5%D1%80.jpg
	caption Олег Шиндер с женой Кариной, опекуны Арсения Невгодовского
img_end
\fi

На вопрос, почему он согласился приютить у себя в доме человека, который напал
на свою тетю и брата с ножом, Олег Шиндер отвечает, что осознавал все риски, но
желание помочь и разобраться в причинах поступка Арсения оказалось сильнее
страха.

“Конечно, и я, и жена первое время с опаской к нему относились. Хотя старались
не показывать это Арсению. Наш сын, Тимофей, первое время немного ревновал. Да
и Арсений вел себя настороженно. Если возле него громко хлопнешь - пугается,
прикоснешься - его передергивает всего. Но со временем отошел, стал улыбаться.
Понятно было, что им все детство никто не занимался. Наша задача - отогреть
ребенка. Я решил показать Арсению пример другой жизни. В возрасте 15 лет уже
непросто поменять что-то в человеке. Но мы над этим работаем”, - говорит Олег
Шиндер. 

У него два сына: старшему 22 года, он уже живет отдельно, младшему 15, он ровесник Арсения.

“Я давно говорил жене, что хотел бы усыновить ребенка. И тут такой случай”, - признается Олег Шиндер. 

“Я верующий человек, - добавляет отец семейства. - А кроме того, мое глубокое убеждение: ребенок в таком возрасте просто так ни на кого с ножом не нападет. Должны быть весомые причины. Как оказалось, причина в том, что Арсений постоянно подвергался домашнему насилию”.

По его словам, о том, что Арсения дома бьют, говорили даже на родительских собраниях. Мальчик успел сменить несколько учебных заведений, год учился в закрытой школе на Мальте, а после - в кадетском корпусе. 

Учителя неоднократно обращались в соцслужбы с заявлениями о том, что Арсений приходит на учебу с побоями. Педагоги замечали синяки на теле мальчика, а тот старался всегда ходить в закрытой одежде, чтобы скрыть ссадины.

Но все эти жалобы заканчивались ничем. 

“Мать просто откупалась от любых претензий. У нее частный дом, прислуга, живет
она богато. Соцслужбы приходили, видели дорогую обстановку, мать давала им
денег - и проблемы нет”, - утверждает Шиндер.

Те, кто работал в доме матери Арсения, публично подтверждали, что мама
издевалась над сыном.

“Когда Арсению исполнилось 6 лет, мама запретила персоналу его кормить, а сыну
сказала, что он должен готовить еду себе сам. Если он, допустим, неправильно
держал нож - подзатыльник. И оскорбления вроде “ты что, дебил?”. Если начинал
плакать - за шкирки, и в холодный душ”, - передает слова домработницы в доме
Невгодовских Олег Шиндер. 

В то, что у Арсения есть психические отклонения, как уверяет его мать, опекуны
не верят.

“Мать Арсения говорит, что он умственно отсталый и недееспособный. Нет. Он
просто флегматик, немного медлительный. Мой родной сын тоже такой, они очень
похожи. Арсений считает его своим другом, они нормально уживаются”, - говорит
Олег Шиндер. 

По его словам, за полгода жизни в патронатной семье у них с Арсением не было
конфликтов. 

“В общении он подкупает прямотой, откровенностью, даже иногда излишней - может
сказать в лоб то, что в социуме обычно говорить не принято. Стараемся его
социализировать. Он к нам попал недокормленный, неухоженный. Его развитием
никто не опекался. При этом Арсений очень старается быть послушным и полезным”,
- говорит опекун. 

\subsubsection{Версия Святослава Козырева}

Бывший сожитель Юлии Невгодовской, отец мальчика, которого Арсений изрезал
ножом, играет отдельную роль в этой истории.

Интересно, что вначале Козырев и сам называл подростка убийцей. Сразу после
случившегося он говорил, что замечал за Арсением агрессивное поведение и даже
тягу к холодному оружию.

Но спустя некоторое время он простил мальчика и публично занял его сторону.
Кроме этого, как сообщил Козырев "Стране", его сын готов общаться с Арсением и
уже его не боится.

Информацию о домашнем насилии в доме Невгодовских “Стране” теперь подтверждает
и Козырев. Он прожил с тетей Арсения 8 лет, но год назад после очередного
скандала ушел из семьи. 

По его словам, он неоднократно был свидетелем того, как издевались над
Арсением. Поэтому понимает, что толкнуло его на такой поступок.

“Ребенок всегда был лишним в доме, был унижен и оскорблен мамой. Жил как
отброс. Ему стиральной машинкой не разрешали пользоваться - он стирал свои вещи
в тазике. С 10 лет он полностью сам себя обслуживал”, - добавляет Козырев. 

\ifcmt
img_begin 
	tags svjatoslav kozyrev,fb
	url https://strana.ua/img/forall/u/0/25/%D0%BA%D0%BE%D0%B7%D1%8B%D1%80%D0%B5%D0%B2(1).jpg
	caption Святослав Козырев, Фото: Фейсбук
	width 0.7
img_end
\fi

Он также считает, что Арсений не аутист.

“Арсений всегда рос под лейбой “придурок”, “аутист”, “даун”, “недалекий”,
“дебил” - эти фразы звучали в доме постоянно. Я сам это слышал. Сказать, что он
умственно отсталый или аутист, не могу. Я видел, как он читал книжки,
энциклопедии. Он многими вещами интересуется, много знает про всяких
динозавров, историю, Атлантиду. Арсений очень закомплексованный, замкнутый, у
него не было друзей - он так вырос, один в своей комнате. Эту асоциальность ему
привили мама с тетей, постоянно называя его ненормальным. Признаков явных
психических отклонений я не видел. Он просто привык ни с кем не общаться, и
все”, - говорит “Стране” Святослав Козырев. 

Юлия Невгодовская подала иск на Козырева - за то, что тот якобы ее обокрал,
"заказал" нападение на нее и похитил у нее сына.  

Козырев эти обвинения отвергает. Их сын Михаил действительно живет сейчас с
Козыревым - отец забрал его из реанимации после ранения. Но Святослав
утверждает, что не препятствует общению сына с матерью - ребенок, по его
словам, сам не хочет ее видеть из-за сильной психологической травмы. А Юлия
якобы таким образом мстит бывшему за то, что он ушел.

\ifcmt
pic https://strana.ua/img/forall/u/0/25/%D0%BA%D0%BE%D0%B7%D1%8B%D1%80%D0%B5%D0%B2.jpg
\fi

“Целый год, что я не жил с Юлей и Мишей (так зовут сына - Ред.), мой ребенок
жил в неизвестной мне обстановке. Юлия запрещала мне видеться с сыном. Только в
садике через забор я мог с ним поговорить. И вот я его увидел 2 марта в
реанимации - изрезанного, плачущего, шокированного. И я увидел, что он к маме
не хочет вообще. У него мама стерта. Он жил в атмосфере скандалов, криков и
насилия - своей мамы против своего двоюродного брата. Так что из реанимации я
его забрал к себе. Имею право - не лишен родительских прав. Когда мы ехали из
больницы и проезжали мимо дома, он испугался и закричал: "Там Юля дерется с
Арсением!, - он даже "мама" не сказал, а "Юля". - Там кровь, я туда не хочу!".
Он видел, как его мама била его брата. У ребенка психологическая травма.
Ассоциация с мамой - агрессия, драки, крики, кровь, боль. Я постоянно
спрашиваю, хочет ли он к маме - еще ни разу он не сказал “да”. И я не вижу от
Юли искреннего желания видеть ребенка. В своем видео она дважды упоминает
имущество, которое у нее якобы украли, а сына - лишь один раз, и то вскользь.
Это не похоже на искренние эмоции мамы, у которой отобрали ребенка”, - считает
Козырев.

По словам Святослава, это способ надавить на него как на свидетеля по делу о
домашнем насилии против Арсения. Козырев утверждает, что не только его, но и
других свидетелей по этому делу пытаются вынудить отказаться от показаний,
путем угроз, обвинений и возбуждения различных уголовных дел. Кстати,
следователя, открывшего дело на мать за насилие над сыном, после этого уволили
из органов. 

\subsubsection{Что известно о матери Арсения: "Арсений жил изгоем"}

Как только Арсения поместили в СИЗО, его мать, Елена Невгодовская, заявила, что
ее сын - психически неуравновешенный. И что она больше ничего об этом ребенке
слышать не хочет. По словам опрошенных "Страной" свидетелей, на суде Елена
Невгодовская говорила, что уже подобрала сыну "заведение закрытого типа".

Елена Невгодовская сказала "Стране", что отправила сына жить к сестре потому,
что он учился в школе рядом с ее домом. А свои отношения с Арсением его мать
описывает так: "Как у всех подростков с родителями". Более подробно отвечать на
вопросы она отказалась. 

Охотнее об отношениях матери с Арсением рассказывают его нынешние опекуны.

“Мама для Арсения - больной вопрос, - говорит Олег Шиндер, и передает слова
мальчика: - Он ее напрямую давно спрашивал: “Зачем ты меня родила, если ты меня
не любишь?” Отвечала: “Так получилось”. 

\ifcmt
img_begin 
	tags 
	url https://strana.ua/img/forall/u/0/25/%D0%A1%D0%BD%D0%B8%D0%BC%D0%BE%D0%BA_%D1%8D%D0%BA%D1%80%D0%B0%D0%BD%D0%B0_2020-10-22_%D0%B2_01.02_.32_.png
	caption Елена Невгодовская, мать Арсения (На Фейсбуке подписана Елена Подольская) 
	width 0.7
img_end
\fi

Арсений рос без папы. С отцом Арсения отношения у Елены Невгодовской не
сложились. 

“Прислуга в доме Невгодовской говорит, что папа Арсения был недостаточно богат,
чтобы обеспечить тот уровень жизни, который она хотела. Так что она выгнала
папу, лишила его родительских прав. Потом нашла нового ухажера и разыгрывала
перед ним спектакль, что у нее на руках больной ребенок. Ухажер давал на его
содержание по 100 тысяч долларов в год”, - рассказывает “Стране” Олег Шиндер. 

Этот “ухажер” - Павел Погиба, который, по информации собеседников "Страны",
состоит в дружественных отношениях с Романом Говдой, первым замгенпрокурора. 

\ifcmt
img_begin 
	tags 
	url https://strana.ua/img/forall/u/0/25/pohuba.jpg
	caption Павел Погиба
	width 0.7
img_end
\fi

Павел Погиба имеет интересную биографию. Сейчас его называют в СМИ
председателем Киевского отделения Ассоциации налогоплательщиков Украины. Но
ранее вместе с опальным политиком Сергеем Думчевым Погиба возглавлял партию
"Рух за реформи". Осенью 2015 года Погиба выдвигался в Киевраду от этой партии.
Думчев, кроме прочего, - кум бывшего генпрокурора Виталия Яремы.

Ранее Павел Погиба и Сергей Думчев были связаны еще через одну структуру –
коммерческий банк "Премиум". По информации НБУ, на 2016 год у Погибы было там
9\% акций. А Думчев до 2015 года возглавлял Набсовет "Премиума". 

С этим же банком "Премиум" связана и Елена Невгодовская.

\ifcmt
pic https://strana.ua/img/forall/u/0/25/%D0%BD%D0%B5%D0%B2%D0%B3%D0%BE%D0%B4%D0%BE%D0%B2.jpg
\fi

У Елены Невгодовской с Павлом Погибой пять лет назад родилась дочка, Полина.
Тогда отношения между Еленой и Арсением Невгодовскими окончательно испортились.

“Арсений стал “отработанным материалом”. Когда приходил Павел, Арсению не
разрешали выходить из комнаты, его не пускали за общий стол. И вот с момента,
когда родилась дочь, его отправляли то в закрытую школу, то жить к тете”, -
говорит Олег Шиндер, ссылаясь на показания свидетелей по делу. 

“Арсений всю жизнь жил изгоем у матери, это подтверждают все свидетели -
учителя и домашние сотрудники”, - говорит Святослав Козырев. 

По его словам, Арсений рос “в жесткой феминистической обстановке”. 

При этом со своей дочерью, по словам Козырева, мать Арсения обращается
совершенно иначе.

“Для сравнения - у Елены Невгодовской сегодня пятилетняя дочка Полина - у нее 2
няни, 3 домработницы, и все, что малышка хочет, ей на блюдечке подается. И тут
Арсений - как собака, не питается с общего стола. Там жесткая феминистическая
обстановка в доме. Мама ненавидит мужчин: своего отца, отца Арсения и самого
Арсения. А Арсению достается за всех мужчин в мире. Елена Невгодовская также
ужасно отзывается о папе Полины, но он состоятельный человек и дает ей много
денег. Сама Елена нигде не работает, ее содержит Павел Погиба”, - говорит
Козырев.

По его словам, Арсений с матерью не уживались вместе. Елена Невгодовская всегда
пыталась сына куда-то “отправить”.

“Мать сама переселила Арсения к тете. Так и жили втроем: Юлия, ее сын Михаил и
Арсений. После неудачной попытки мамы выселить Арсения на Мальту и в
суворовское училище, тетя была его последней надеждой”, - говорит “Стране”
Святослав Козырев.

\subsection{\enquote{Все из-за царапин на ламинате}}

Первое время с тетей Арсению жилось хорошо. Но с тех пор, как год назад от Юлии
Невгодовской ушел Святослав Козырев, у тети резко поменялось отношение к
племяннику. 

“Еще при мне были незначительные ссоры. Юля могла дать Арсению пинок,
подзатыльник, обозвать “чмом”, “придурком” за то, что он зашумел, что-то уронил
или не убрал. При мне до битья не доходило, но за тот год, что я в этом доме не
жил, там вполне могло быть более серьезное рукоприкладство. У Юли -
нестабильный гормональный фон из-за удаленной щитовидки, перепады настроения.
Она могла всерьез выйти из себя, если поцарапаешь вилкой ложку, накалывая
спагетти. У Юли бывали срывы, и Арсений отхватывал за любые мелочи”, -
вспоминает Козырев.

Олег Шиндер со слов Арсения также утверждает, что тетя систематически избивала
племянника. А перед тем как мальчик пошел на нее с ножом, Юлия Невгодовская
якобы била его шнуром от компьютера.

“Тетя заходит в комнату, находит дырку на ламинате и спрашивает: “Откуда
дырка?”, - “Я не знаю”. “Ах не знаешь!..” - и начинает его бить. Арсений думал,
что все, его сейчас забьют, и если он что-то не сделает, просто погибнет. И в
этот момент то ли позвонили, то ли что-то тетю отвлекло. И он воспользовался
моментом, выхватил нож и спрятался за углом. И когда она шла обратно, он ее
ударил”, - рассказывает Олег Шиндер, передавая слова Арсения. 

И добавляет, что и мама, и тетя Арсения - в два раза крупнее него. В обычной
драке, по словам Шиндера, шансов у Арсения защититься от женщин нет, они явно
физически сильнее. 

Святослав Козырев считает, что случившееся вовсе нельзя называть нападением -
это, по его мнению, бытовой конфликт. 

“В комнате Арсения Юля обнаружила какие-то царапины или вмятины на ламинате.
Зная, что они могли поругаться из-за грязной ложки - это вполне могло стать
причиной скандала. В доме свежий ремонт, так что Юля вполне могла сорваться
из-за вмятин на ламинате. Она предъявила претензии, Арсений возразил, спор
перерос в драку, в результате - поножовщина. Вдумайтесь - это всё произошло
из-за того, что тетя увидела царапины на ламинате!”, - говорит Козырев.  

Он не сомневается в том, что Юлия избивала Арсения.

“В одной из драк он просто не выдержал”, - говорит Козырев. 

“Экспертиза подтвердила, что Арсений был в состоянии аффекта. Хотел порезать
всех, кто подвернется - поэтому поранил и своего брата, о чем сожалеет. Арсений
хотел покончить жизнь самоубийством. У Арсения остался шрам на шее - он просто
не достал ножом до артерии”, - утверждает Олег Шиндер. 

\subsubsection{\enquote{Из больницы Павлова Арсений не выйдет}}

Сразу после нападения Арсения на тетю и двоюродного брата была проведена
экспертиза, которая зафиксировала множественные побои на Арсении - как давние,
так и свежие. 

Следователь, которая занималась делом Арсения, на основании этой экспертизы
вручила подозрения матери и тете Невгодовского - за домашнее насилие и
невыполнение родительский обязанностей, - прямо на заседании суда, спустя две
недели после нападения.

Но эти дела вскоре (как утверждают Козырев и Шиндер, по поручению Романа Говды)
были переданы в Оболонский райсуд и там "похоронены". 

“Товарищ Паши Погибы - первый замгенпрокурора Роман Говда. Как только он зашел
на должность, через 5-6 дней это дело выхватывают из Киево-Святошинского суда и
передают в Оболонский суд. По неофициальной информации, прокурор Оболонского
района - человек Говды. Там подозрения и по маме, и по тете отменили, а сами
дела о домашнем насилии ушли в небытие”, - рассказывает Святослав Козырев.

А тем временем против Арсения возбудили еще уголовное одно дело - за покушение
на жизнь матери. Елена Невгодовская через два месяца после мартовской
“поножовщины” подала заявление на своего сына.

“Понятно, что и Юлия, и Елена просто пытаются защитить себя. Ну это же смешно -
мама только два месяца спустя поняла, что в день, когда Арсений ранил Юлию и
моего сына, он оказывается хотел убить и свою мать! Хотя на видео видно, что
Арсений не только не нападал на мать, он стал убегать от нее, как только увидел
во дворе. Он ее боится всю жизнь, испугался ее и тогда. Он не то что не
поранил, он даже не подходил ближе чем 2 метра. И за это ему сейчас “катают”
подозрение”, - утверждает Козырев. 

Тем не менее под предлогом этого дела назначили новую психологическую
экспертизу. На этот раз - стационарную. Арсений должен провести 2 месяца в
лечебнице для комплексной проверки на вменяемость. 

Опекуны опасаются, что если Арсений туда попадет, то уже из больницы не выйдет. 

“Мама не скрывает, что ее сына надо закрыть в психушке до конца жизни и сделать
из него овоща - это ее слова”, - говорит Олег Шиндер.

Когда Арсений должен был явиться на экспертизу, вся семья заболела, о чем
предупредила правоохранителей и предоставила соответствующие справки. Тогда
Арсения, который находится по известному адресу на круглосуточном домашнем
аресте, подали в розыск. Полиция пыталась силой отвезти Арсения на экспертизу. 

“Сейчас полиция всеми методами пытается заполучить Арсения, чтобы отвезти его
на экспертизу в Павлова, где у мамы уже все договорено. Есть свидетели, которые
видели, как приходили мамины адвокаты и давали деньги персоналу больницы”, -
сообщает “Стране” Олег Шиндер.  

Опекуны Арсения опасаются, что мальчику “нарисуют” несуществующие диагнозы,
лишь бы “замять” дела о домашнем насилии, возбужденные против его тети и мамы.

“В рамках нового дела Арсению зачем-то назначают стационарную психологическую
экспертизу. Арсений прошел уже две экспертизы еще по первому делу. Так там
поножовщина, а во втором деле ничего нет, кроме фантазий. Если он сейчас
попадет в Павлова, то как заявитель по делу о домашнем насилии становится
недееспособным. И выходит, что заявлять о домашнем насилии некому - дело
спокойно закрывается. Так как потерпевший стал овощем и живет в Павлова”, -
рисует перспективу Святослав Козырев.

“Страна” обратилась за комментариями к Елене и Юлии Невгодовским, но на момент
публикации материала ответов не получила. Мы готовы опубликовать их в любой
момент.

В эту историю уже лично включился Уполномоченный Президента Украины по правам
детей Николай Кулеба, а Офис Омбудсмена Украины и Минсоцполитики держат этот
вопрос под контролем. 

По словам Николая Кулебы, дело это “непростое”, вокруг него много противоречий.

"Из-за этой истории я считал необходимым лично разобраться в ситуации в
управлении полиции Оболонского района. Да, ребенок обвиняется в совершении
преступления. Но это не значит, что он бесправный", --- добавил Кулеба.

По его словам, ни при каких обстоятельствах ребенок не должен быть заложником
конфликтов взрослых, а процессуальные решения в ходе расследования должны
приниматься в интересах ребенка.

Но пока что суды идут только над Арсением. До тех пор, пока следователи не
дадут хода делам против его матери и тети, сложно объективно сказать, кто в
этой истории преступник, а кто - жертва. 
