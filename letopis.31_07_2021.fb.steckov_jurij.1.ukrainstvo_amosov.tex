% vim: keymap=russian-jcukenwin
%%beginhead 
 
%%file 31_07_2021.fb.steckov_jurij.1.ukrainstvo_amosov
%%parent 31_07_2021
 
%%url https://www.facebook.com/justets/posts/2324036394397182
 
%%author Стецков, Юрий
%%author_id steckov_jurij
%%author_url 
 
%%tags amosov_nikolaj,identichnost',nacia,ukraina,ukrainstvo
%%title ПОЗДРАВЛЯЮ УКРАИНЦЕВ С "ПЕРСИДСКОСТЬЮ" а вообще-то НУ КАКАЯ НАХ*Й РАЗНИЦА?
 
%%endhead 
 
\subsection{ПОЗДРАВЛЯЮ УКРАИНЦЕВ С \enquote{ПЕРСИДСКОСТЬЮ} а вообще-то НУ КАКАЯ НАХ*Й РАЗНИЦА?}
\label{sec:31_07_2021.fb.steckov_jurij.1.ukrainstvo_amosov}
 
\Purl{https://www.facebook.com/justets/posts/2324036394397182}
\ifcmt
 author_begin
   author_id steckov_jurij
 author_end
\fi

ПОЗДРАВЛЯЮ УКРАИНЦЕВ С "ПЕРСИДСКОСТЬЮ" 🙂 а вообще-то НУ КАКАЯ НАХ\_Й РАЗНИЦА?

Фамилию АМОСОВ слышал (хоть раз 🙂 ) каждый украинец.

В 2008 году в телевизионном проекте «100 великих украинцев» на одном из
украинских телеканалов в результате народного голосования Амосов занял второе
место, пропустив вперёд только Ярослава Мудрого.

Хирург от Бога, сперва торакальный, а затем и кардио, он является автором
новаторских методик в кардиологии и торакальной хирургии, автор системного
подхода к здоровью ("метод ограничений и нагрузок"), работ по геронтологии,
проблемам искусственного интеллекта и рационального планирования общественной
жизни ("социальной инженерии").

\ifcmt
  pic https://scontent-cdg2-1.xx.fbcdn.net/v/t39.30808-6/227779155_2324037491063739_1074691959519765742_n.jpg?_nc_cat=102&ccb=1-4&_nc_sid=730e14&_nc_ohc=K4Y9gmY0nnYAX_ltj5-&_nc_oc=AQml48NXEACotEvtaWfBFLGSKcT6_PuLuUZ0vzJ9UHpVIE5kTHk-L7fIPwjpZX5ijIQ&_nc_ht=scontent-cdg2-1.xx&oh=eeaa7253ec8a944d0aa929a2ebc38e1d&oe=611A183C
  width 0.4
\fi

Ещё я упомяну то, что мне, как компьютерщику, близко.

В 1960-1970 годах Амосов возглавлял отдел биокибернетики института кибернетики
украинской академии наук, сейчас это институт имени  Виктора Глушкова.

Этот институт кибернетики - первый в Украине, а отдел биокибернетики в этом
институте появился только после того, как академики Амосов и Глушков тесно
познакомились и обменялись идеями.

Проще говоря, отдел биокибернетики был создан "под Амосова..." 🙂

Друзья, когда вы в первый раз услышали слово "компьютер", и когда впервые его
увидели? В каких годах?


Просматривая сегодня написанное (он и писатель) и сказанное Николаем
Михайловичем Амосовым, я обнаружил ранее мной не читанное интервью, данное им в
2001 году.

Вот цитата из этого интервью.

Отвечая на вопрос корреспондента о «недопонимании» между украинцами и русскими,
Украиной и Россией (ещё в 2001 году!), Амосов сказал:

"Как выяснилось, у нас есть различия - генетические. Я был уверен, что славяне
одинаковы хотя бы потому, что 350 лет прожили вместе. И в первой, и во второй
половине жизни я, русский человек, не замечал, что украинцы - некий особый
народ. Но когда расшифровали наши геномы, выяснилось: у россиян вклинились гены
угро-финских народов, у украинцев - южные, персидские. При желании можно
сказать, что мы - разные".

Ну и какая, блин, разница?

Кто лучше - угро-финские народы или персидские?

Мы все - потомки HOMO SAPIENS, возникших 70 тысяч лет назад на северо-востоке
африканского континента, и расселившихся по всему миру.

К сегодняшнему дню часть из нас побелели, часть - пожелтели, часть -
почернели... У одних глаза стали узкими, у других носы стали широкими, и так
далее, и так далее...

Но если новорожденного ребёночка одного из племён из современной Папуа - Новая
Гвинея перенести в японскую семью, то для него саке с крепостью 20 оборотов и
подогретое до шестидесяти градусов будет нормой, если налить его в блюдечко.

А если перенести в украинскую семью, то борщ и сало... 🙂

44-й президент США Барак Обама, Barack Hussein Obama - с фамилией африканского
и средним именем арабского происхождения - тому пример.

И, кстати, АйКью Обамы гораздо выше, чем у Джорджа Буша - младшего, предыдущего
президента США, чисто белого...

Тырындёж о том, кто братья, а кто нет, начал Пукин. 

А кто подхватил этот тырындёж в Украине?

Самый главный подхвативший - это Петя Пятый.

Петя Пятый к этому тырындежу добавил и язык, и веру, и армию.

Добавил и армию, на которой воровал сам и позволял воровать другим.

И в заключение.

А "чистый" ли украинец я сам?

Мои предки по маме, Зандберг Альвине Ансовне - семья немецкого
мастера-краснодеревщика, которого Пётр Первый привёз вместе с семьёй и поселил
в Киеве. Её фамилия и отчество - Зандберг Ансовна - от моего деда, латыша Анс
Зандберг, латышского стрелка (погуглите "латышский стрелок").

Мой отец - Иван Михайлович Стецюк - украинский крестьянин из села Головчинцы,
сейчас это село в Летичевском районе Хмельницкой области.

Именно Стецюк, а не Стецков - папа, увы, как и его брат, изменил фамилию на
более "русскую", и говорил только на русском, но года с 1993-го проклял и СССР,
и КПСС, и говорил только на украинском.

И ещё.

Я служил с несколькими татарами, не буду называть фамилий. Так вот, они иногда,
за рюмкой чая, шутили, что, мол, все вы - немножко татары... Ведь сколько веков
длилось татаро-монгольское иго?

Так что, скорее всего, и я немного татарин, а не только украинец, латыш и
немец. Да и, насколько я помню, в роду у маминых предков была пара ну явно
еврейских фамилий... 

И?

И кому я брат?

Тырындеть о генах, о языке и вере - это обман, это попытка пустить пар в
свисток, а не в колёса.

***************************************

Друзья, год или полтора назад я в сети наткнулся на сайт проекта, названия
которого не запомнил, но скачал несколько видео.

Проект на волонтёрских началах организовала группа генетиков из разных стран.
Всем желающим предлагалось пройти генетический тест, чтобы определить свои
генетические корни.

Запомнился немец (настоящий немец, крепкий до лёгкой полноватости), сказавший
во время предварительного интервью, что уж кого-кого, а англичан в числе его
предков быть ну абсолютно не может... Ведь Англия и Германия веками воевали
друг с другом.

И нужно посмотреть на его лицо после того, как тест показал, что он на 30\% -
англичанин... 

Это видео, и не только это, где-то лежит на дисках моих компьютеров, и, если
будет запрос, я его поищу.

***************************************

***************************************

Николай Михайлович Амосов

\ii{31_07_2021.fb.steckov_jurij.1.ukrainstvo_amosov.cmt}
