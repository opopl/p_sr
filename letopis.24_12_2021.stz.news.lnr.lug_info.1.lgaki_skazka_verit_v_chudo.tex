% vim: keymap=russian-jcukenwin
%%beginhead 
 
%%file 24_12_2021.stz.news.lnr.lug_info.1.lgaki_skazka_verit_v_chudo
%%parent 24_12_2021
 
%%url https://lug-info.com/news/studenty-lgaki-predstavili-prem-eru-novogodnej-skazki-nuzno-verit-v-cudo
 
%%author_id 
%%date 
 
%%tags 
%%title Студенты ЛГАКИ представили премьеру новогодней сказки "Нужно верить в чудо!"
 
%%endhead 
\subsection{Студенты ЛГАКИ представили премьеру новогодней сказки \enquote{Нужно верить в чудо!}}
\label{sec:24_12_2021.stz.news.lnr.lug_info.1.lgaki_skazka_verit_v_chudo}

\Purl{https://lug-info.com/news/studenty-lgaki-predstavili-prem-eru-novogodnej-skazki-nuzno-verit-v-cudo}

Студенты Луганской государственной академии культуры и искусств (ЛГАКИ) имени
Михаила Матусовского представили на сцене творческого центра \enquote{Красная площадь,
7} премьеру сказки \enquote{Нужно верить в чудо!}. Об этом сообщила пресс-служба
Министерства культуры, спорта и молодежи ЛНР со ссылкой на вуз.

\ii{24_12_2021.stz.news.lnr.lug_info.1.lgaki_skazka_verit_v_chudo.pic.1}

\enquote{В главном творческом вузе Донбасса и елка – особенная. Не просто праздник, а
комплексный экзамен по режиссуре, сценарному и актерскому мастерству для
студентов третьего курса кафедры театрального искусства, будущих режиссеров
театрализованных представлений и праздников. В этом году это курс народной
артистки ЛНР, профессора кафедры Веры Евдокимовой}, – говорится в сообщении.

\ii{24_12_2021.stz.news.lnr.lug_info.1.lgaki_skazka_verit_v_chudo.pic.2}

Педагог кафедры и куратор курса Надежда Рубель рассказала, что сценарий
представления был подготовлен студентами, которые также выступили актерами и
режиссерами сказки.

\enquote{Ребята сами выступили в роли актеров и режиссеров – у каждого был свой эпизод.
Преподаватели оценят, как студенты продумали и представили своих персонажей,
как у них получилось организовать взаимодействие со зрителями, особенно – с
маленькими зрителями. Для этого в действо вплели игровые моменты}, – отметила
она.

В представлении задействованы Дед Мороз, Снегурочка и работники Фабрики
подарков – трудолюбивые гномы, снеговик Снегослав. Есть и отрицательные
персонажи – Бяка, Бука и Ледяная Борода, которые пытаются сорвать праздник.

\enquote{Но даже отрицательные персонажи этой новогодней истории знают, что зимой нужно
носить шапку, шарфик, а в наши дни еще и маску, а также пользоваться
антисептиком. И учат этому маленьких зрителей}, – рассказали в ЛГАКИ.

Исполнительница роли Бяки Алина Горшкова отметила, что \enquote{название сказки говорит
само за себя}.

\enquote{Мы призываем всех наших маленьких и взрослых зрителей верить в чудо, в Новый
год, в Деда Мороза. И обязательно читать сказки. Самое главное в любой сказке –
мораль, воспитательный момент. У нашей он тоже есть. В конце мы – Бука, Бяка,
Ледяная Борода – исправляемся, начинаем говорить хорошие слова, совершать
хорошие поступки. Учимся верить в чудо!} – отметила студентка.

В ЛГАКИ добавили, что студенты вуза планируют провести еще несколько показов
сказки до наступления Нового года.

\enquote{В том, что студенты получат заслуженные \enquote{пятерки} на экзамене, нет
сомнений.  Восторга маленьких зрителей от соприкосновения с придуманным ими
чудом режиссерам и актерам добиться уже удалось. А это дороже любых оценок}, –
отметили в академии.
