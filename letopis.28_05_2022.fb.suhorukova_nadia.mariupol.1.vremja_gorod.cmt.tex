% vim: keymap=russian-jcukenwin
%%beginhead 
 
%%file 28_05_2022.fb.suhorukova_nadia.mariupol.1.vremja_gorod.cmt
%%parent 28_05_2022.fb.suhorukova_nadia.mariupol.1.vremja_gorod
 
%%url 
 
%%author_id 
%%date 
 
%%tags 
%%title 
 
%%endhead 
\qqSecCmt

\begin{itemize} % {
\iusr{Олександра Фоя}

Це так дивовижно, що ви знайшли в собі сили описувати весь цей кошмар.
Свідчення очевидців - це дуже важливо. І ви дуже цікаво розповідаєте (якщо,
звісно, доречні такі компліменти враховуючи, що описується пекло).

Дякую вам за ваші пости, хай вам щастить. Та іншим людям, хто пройшов крізь все
це пекло

\begin{itemize} % {
\iusr{Надія Сухорукова}
\textbf{Олександра Фоя} спасибі!

\iusr{Наталия Трокачевская}
\textbf{Надія Сухорукова} 

мы ночевали в тамбуре с соседями. После каждого авианалета дом шатало так, что
мы думали сложится. Казалось что мы в поезде и его заносит на поворотах(

Сказать что было страшно, это ничего не сказать......

\iusr{Надія Сухорукова}
\textbf{Наталия Трокачевская} это страхом назвать трудно. Какой-то ужас животный

\iusr{Наталия Трокачевская}
\textbf{Надія Сухорукова} это точно

\iusr{Anzhelika Ismusova}
\textbf{Надія Сухорукова}  @igg{fbicon.face.crying.loudly}  @igg{fbicon.hands.pray}{repeat=3}  @igg{fbicon.flag.ukraina}

\end{itemize} % }

\iusr{Natali Korovanenkova}

Мы жили на пятом по пр Металлургов, я каждый вечер в занавешанное одеялами
разбитое окно, тогда оно еще у меня было, с 7 этажа смотрела как бомбят и горит
Кировский, там мои друзья... это так рядом и так далеко через мост.. мои друзья
работали в театре.. и тогда очень надеялась, что они там спрятались, за наш
театр я узнала только потом, когда смогла выехать ...

\iusr{Yulia Farafonova}

бессмыссленные циничные убийства людей. Ужас ужаснее которого быть не может.

Как могут люди которые это делали и одобряли продолжать жить в этом мире(...

живу надежной на справедливую кару

\iusr{Георгий Сочивко}
\textbf{Юлия Фарафонова} а она придет @igg{fbicon.anger}  @igg{fbicon.face.smiling.horns} ☠

\iusr{Инна Полякова}

Мне говорят отпусти и просто живи..., мол время лечит. Нет, не могу, не
получается отпустить. Как хочется пролететь над моим домом и хоть одним глазком
увидеть и услышать его жизнь, его голос. Надюша, пишите, я вместе с вами опять
переживаю всю эту боль ...Спасибо вам.

\iusr{Anzhelika Likavilnius}

Я очень хочу знать и видеть морды тех млекопитающих, кто отдавал приказы. Все
цепочку этих морд. На суде. Потому что у меня нет даже точки опоры, чтобы
понять: как они отдавали приказы, как они их исполняли.

\begin{itemize} % {
\iusr{Ionel Cuibus}
\textbf{Anzhelika Likavilnius} din pacate nu vor raspunde, cred... Slava Ucraina !

\iusr{Alyona Pevneva}
\textbf{Anzhelika Likavilnius} 

Да. Я тоже хочу на них посмотреть. И на их самок, соседей, которые поцокают «ну
надо же, а с виду совсем приличный, не маньяк, да и жену почти не бил»...  @igg{fbicon.face.angry.horns}{repeat=3} 

\iusr{Valentina Tretiouk}
\textbf{Alyona Pevneva} Только Трибунал! И на скамье пусть сидят те, кто это придумал, исполнял и пропагандировал!

\iusr{Инна Белая}
\textbf{Anzhelika Likavilnius} это бандиты, все будут наказаны

\iusr{Donatas Tauterys}
\textbf{Valentina Tretiouk} так уже готовят трибунал.В Донецке

\iusr{Elina Ozerskiy}
\textbf{Anzhelika Likavilnius} спросите в Киеве. Может скажут

\iusr{Людмила Гунько}
Все знают эти морды. Путин. Исполнители петушилины
\end{itemize} % }

\iusr{Анна Запрудина}
Россия-позор человечества.

\iusr{Dina Tarasenko}
Неможливо дивитись... @igg{fbicon.face.crying.loudly}  Горіть у пеклі рашисти прокляті! Ніколи не пробачимо!

\iusr{Ivanka Usmanov}

Я дуже сильно хочу щоб Ви були! Були такі люди як Ви! Нехай Бозя Вас береже і
благословить це все пережити і знайте Ви перемогли, бо вас не здолав страх, а
навпаки- Ви одолали його...

Пишіть! Не переставайте! Правда світить особливим світлом!!!!!

\iusr{Татьяна Китанина}

Невыносимо представлять себе эти жизни - обычные жизни обычных людей, с
радостями, любовью, заботой о детях - вот так вдруг, ни зачем, ни за что,
разрубленные, уничтоженные... И оторопь берет от того, что человечество уже три
месяца на это смотрит и ничего не может сделать.

\begin{itemize} % {
\iusr{Надія Сухорукова}
\textbf{Татьяна Китанина} 

Танюша, это же не заканчивается. Никаких ресурсов внутренних не хватит, чтобы с
этим жить. То что было - очень страшно, но гораздо страшнее, что ад
продолжается. Для Мариуполя, для Лимана, для Харькова, для Северодонецка, для
Бахмута, Николаева, для других городов Украины. В том числе и оккупированных

\iusr{Татьяна Китанина}
\textbf{Надія Сухорукова} 

Надiя, да, я именно об этом - и это просто невыносимо понимать, что ад
продолжается а мир как будто к этому привыкает. Спасибо Вам за то, что не
позволяете привыкать и выбрасывать из головы - иначе человечество просто
перестанет быть человечеством.

\iusr{Светлана Иванова}
\textbf{Татьяна Китанина} просто ужасно!
Что ни кто не может это остановить!
Зато рашисткие новости пишут о возобновлении работы порта Мариуполь!
Как так?
Кровь из ушей

\iusr{Татьяна Китанина}
\textbf{Светлана Иванова} да, ужасно. И да, кровь из ушей - а больше из сердца... Не знаю, как с этим жить.

\iusr{Tatiana Ifraimov}
\textbf{Надія Сухорукова} Мне кажется или нет, вы перепутали группу. Здесь группа Русские

\iusr{Надія Сухорукова}
\textbf{Tatiana Ifraimov} то есть в этой группе правду рассказывать нельзя, правильно?

\iusr{Tatiana Ifraimov}
\textbf{Надія Сухорукова} Ваша правда с дивана, поживём увидим.

\iusr{Надія Сухорукова}
\textbf{Tatiana Ifraimov} серьёзно? И вам такой же правды с дивана, милая старушка, желаю

\iusr{Yuliia Bilous}
\textbf{Tatiana Ifraimov} какая же бездна у вас в душе. Как же тяжело вам с этой мрачной пустотой внутри. Ничего не спасает от самого себя. Ни бегство от правды, ни желчь, которая сочится на людей, способных открыть вам правду, которой вы избегаете.

\iusr{Светлана Иванова}
\textbf{Татьяна Китанина} у меня в друзьях 3 родственников из России.
Я хочу чтоб они видели эти фото. Просто ни кто не верит в это

\iusr{Светлана Иванова}

\ifcmt
  ig https://scontent-frx5-1.xx.fbcdn.net/v/t39.30808-6/285368288_5469714566423928_6004012408902291267_n.jpg?_nc_cat=110&ccb=1-7&_nc_sid=dbeb18&_nc_ohc=FKdRV7ODxDkAX-VilZ4&_nc_oc=AQl3NmYNPI7JOgGRKBgq5SWGmXfNgcMTl2STuA-iTS_EsHOy3PC5dy2w4wwPgb2Tchk&_nc_ht=scontent-frx5-1.xx&oh=00_AT8FLhw_qOSYkR-sZIvwTTmr7Y419n41msa-rH8SO2dItg&oe=629B5516
  @width 0.3
\fi

\iusr{Татьяна Китанина}
\textbf{Светлана Иванова} 

как можно не верить, я не понимаю - столько живых голосов, рассказывающих свои
страшные истории, столько фотографий и видео... Как же нужно не желать видеть,
чтобы жить без этого! И да, я тоже из России - и живу три месяца в
непрекращающемся кошмаре, не могу думать ни о чем другом - как и почти все мои
близкие и друзья.

\end{itemize} % }

\iusr{Nasib Taskinbayeva}

Миру і благополуччя стійким, волелюбним і мужнім українцям!
@igg{fbicon.heart.blue}  @igg{fbicon.heart.yellow}  @igg{fbicon.sun.with.face}
@igg{fbicon.dove} @igg{fbicon.flag.ukraina}  @igg{fbicon.hands.pray} @igg{fbicon.heart.red}

\iusr{Леонова Жанна}
До 24.02.22 я ничего не знала о страхе, о потерях и о горе.... У меня все.... Мариупольцы поймут.....

\iusr{Раиса Курочкина}
Как это страшно, жестоко. Все время задаю вопрос " за что,? Почему?

\iusr{Светлана Капцова}

Тяжке випробовуваання на долю Маріуполя випало, як важко і страшно все це було
пережити, і я всім серцем Обіймаю, Вас, Надюшо! З повагою і всім світлим і
добрим до Вас!

\begin{itemize} % {
\iusr{Надія Сухорукова}
\textbf{Светлана Капцова} дякую. Обіймаю Вас
\end{itemize} % }

\iusr{Ирина Журавко}
Цей жах неможливо забути і пробачити.

\iusr{Мария Бойко}

Самым страшным для меня был гул самолёта..., а потом бомбопад -ближе, дальше... И
если не прямое попадание, то выжить еще можно... Выжили.

\begin{itemize} % {
\iusr{Ольга Ниякая}
\textbf{Мария Бойко} 

тоже самое... очень страшен был этот гул самолета... и ожидание -куда, в какой
двор упадет.. взох облегчения, что не в наш, но... он же делал второй круг(

\iusr{Anna Kolomytseva}
\textbf{Мария Бойко} 

кошмар и ужас.. все мариупольцы как один рассказывают про эти авиаудары и гул
самолетов как самое страшное  @igg{fbicon.face.crying.loudly}{repeat=2}  это читать невыносимо, а как там было
выживать - вообще немыслимо...

Держитесь, пожалуйста  @igg{fbicon.hands.pray}
@igg{fbicon.hands.pray}{repeat=2}  обнимаю, хоть мы и не знакомы

\iusr{Milena Bobkova}
\textbf{Мария Бойко} 

Да, гул самолёта и потом тот момент, когда он выпустил ракеты. Свист этих
ракет, их приближение, не понятно же,икуда они упадут... В доме через пару
домов от нашего застряли две такие, с меня ростом, но так и не разорвались,
наверное, до сих пор там торчат. Хозяйка дома умерла в ту ночь от сердечного
приступа...

\iusr{Svetlana Lysenko}
\textbf{Мария Бойко} 

да, точно! Авиаудары самое страшное. У меня все внутри сжималось, когда самолет
заходил с моря на нас. Мы на Левом, каждые три минуты по два удара. Жесть
просто! Пусть будут прокляты!

\iusr{Tatyana Dzhurik Levitskaya}
\textbf{Svetlana Lysenko} 

мы тоже засекали время. Два броска, три минуты, следующий. Сначала их было два-
три, потом штук семь. Минут сорок, берут новые и уже летят с Ейска.

\iusr{Svetlana Lysenko}
\textbf{Tatyana Dzhurik Levitskaya} да, так и было

\iusr{Roman Kovalchyk}
\textbf{Светлана Лысенко} Авиаудары по жилым кварталам. Просто не верится, что кто-то способен на такую жестокость.  @igg{fbicon.cry}  @igg{fbicon.face.crying.loudly} 

\iusr{Svetlana Lysenko}
\textbf{Anna Kolomytseva} Спасибо! Мира вам!

\end{itemize} % }

\iusr{Nadiia Art}

Важко це читати і бачити фото, набагато важче це писати, а як важко це
пережити?  @igg{fbicon.heart.broken}  Знаєте, що найнезбагненніше в цій ситуації- люди, які це чинили,
які нищили і підтримувли знищення - вважають себе героями. І готові повторювати
подвиг з усіма українцями

\iusr{Svetlana Lysenko}

Читаю вас постоянно. Как точно вы описываете наши чувства. Когда нас на Левом
обстреливали градами три часа без перерыаа, то 10 минут казались вечностью. Я
каждые 10 минут спрашивала сына который час. Время остановилось и казалось, что
это никогда не закончится.

\iusr{Arina Antonenko}

Все ваши посты нужно собрать и издать книгу. Перевести на все языки мира. С
фотографиями нынешнего Мариуполя. Пусть весь мир знает, что с Вами произошло.
Пусть весь мир увидит этот ужас!!!!!!

\iusr{Георгий Сочивко}
\textbf{Arina Antonenko}  @igg{fbicon.100.percent}  @igg{fbicon.thumb.up.yellow} 

\iusr{Valentina Tretiouk}
Совершенно невозможно это читать. Какими нужно быть выродками, нелюдями, чтобы творить эти ужасы. Нет им прощения на века.

\iusr{Pavel Rodion}

Такие преступления россиян не останутся безнаказанными, а потом на россии
начнутся стихийные бедствия, пожары, аварии, теракты. А они будут спрашивать за
что это нам? Обнимаю Вас!

\iusr{Анна Растягаева}

Мы тоже думали, что находимся в какой-то страшной компьютерной игре, но с одним
большим НО, в в этой игре Жизни не восстанавливаются(((( Когда спрашивали
\enquote{который час} мы отвечали \enquote{Война}....

\iusr{Тетяна Мєчева}
\textbf{Анна Растягаева} 

саме так! Відібранне життя ніколи не повернути, ніколи(( Який жах, що у 21
столітті з'явилися особини, які захотіли погратися у людське життя( немає
виправдяння цим гравцям, це навіть не люди для мене((

\iusr{Людмила Соболева}

Надежда, подписываюсь под каждым вашим словом, все это пережили вместе с
внучкой. Мы жили в 300 метров от перинатального центра на п. Украина. Сначала
утюжили роддом, потом завод...Ад.

\iusr{Юлия Щекина}

Благодарю Вас, что находите в себе силы описывать этот ужас.

Каждый раз, читая Вас, надеюсь, что соседи, ,которые \enquote{люди мира},
\enquote{лично мы ни на кого не нападали}, \enquote{мы -за мир и против войны},
\enquote{мы ничего не можем сделать потому что мы - маленькие люди и от нас
ничего не зависит и нас могут посадить...} тоже прочтут эти строки и всё же
хоть что-то, даже несмотря на пропаганду, сделают, чтоб прекратить весь этот
ужас, который их государство создало в чужом для них, когда-то братском (как
мы, украинцы, вообще когда-то могли верить в \enquote{братскую любовь}
!?)государстве...  Ещё очень верю в возмездие, - не должны эти зверства,
которые творит у нас в Украине рашистская армия, бесследно пройти для их страны
!

\begin{itemize} % {
\iusr{Ольга Шейнис}
\textbf{Юлия Щекина} Будет возмездие, но вот когда? Может пройти много десятилетий...

\iusr{Ionel Cuibus}
\textbf{Юлия Щекина} nu vor fi represalii, din pacate, cred eu... Slava Ucraina !
\end{itemize} % }

\iusr{Елена Симоненко}

Как хорошо что вы можете описать весь тот ужас, что был в Мариуполе  @igg{fbicon.face.sad.but.relieved}{repeat=4}  у
меня не хватает слов, мне сложно передать всю степень страха и боль потерь, но
такое ощущение, что вы говорите устами всех мариупольцев! Ваши заметки должен
увидеть, услышать и прочитать весь мир!!!!!!!!!!!

\iusr{Irina Paul}

Как точно всё описано. Будто снова оказываешься там и сердце сжимается...

\iusr{Таня Ларіонова}
Дякую що Ви вижили! Дякую! Все буде Україна! Обіймаю!

\iusr{Alla Gerkalyuk}

Моя подруга каже: Не читай. Просто живи.
А я весь час ніби там, пролітаю.

\begin{itemize} % {
\iusr{Надія Сухорукова}
\textbf{Alla Gerkalyuk} так хоч читай, хоч ні, все одно війна йде. Її треба зупинити. Подруга як ми у Маріку ховається від війни під ковдрою

\iusr{Alla Gerkalyuk}
\textbf{Надія Сухорукова} так.
\end{itemize} % }

\iusr{Юлия Травинчева}

Самое точное описание нашего состояния и нашей жизни в погибающем городе....

\iusr{Наталья Бейда}

Читая ваш рассказ, каждая частичка тела вздрагивала от воспоминаний всего того
ужаса, который пережили мы, Мариупольцы.... Эти раны в сердце и в душе всех нас
никогда не заживут.... Я верю, что всё будет Украина, что наш любимый город
Мариуполь отстроят, но все пережитое нами останется в сердце и в душе
навсегда....

\iusr{Ольга Цыбуцынина}

А мы жили на левом берегу, на бульваре Меотиды, где мы ещё были отрезаны и от
центра города, обстрелы были и днём и ночью, этот ад снится мне до сих пор, но
мы выжили всем врагам назло...

\iusr{Оля Меканорова}
Жуткий, липкий страх и ужас жил в нас, все слова-правда!!!!!

\iusr{Tatyana Dzhurik Levitskaya}

Когда летели самолёты, я смотрела вверх и пыталась представить: кто там сидит,
видит ли он нас, как бросает свои бомбы? Куда захочет? Или есть план? Или не
понравился дом, или заметил людей? Что-то вроде стрелялки, кто быстрее? А потом
отбросается, заработает денежек для своих деток, это же боевой вылет, хорошо
платят. Пообедает, отдохнёт. Примет душ, у него же есть вода, и не только
попить. И полетит снова. Хороший мамин сын и отличный отец. Да, это такая игра.
В кальмара.

\iusr{Танюша Маслова}

С одной стороны мы мерзли в эти ужасные дни, а с другой стороны, может и к
лучшему, трупы то начали разлагаться с приходом тепла. Мы находились в
невыносимом фильме ужасов...

Бойня над нашими головами и гул самолета, как рулетка- Куда и кто?!!!

\iusr{Tanya Hrabovska}
Как все точно описано, спасибо, что нашли в себе силы это писать.

\iusr{Vovan Vovan}

Всё правда, подписываюсь под каждым словом! И даже про одеяло, я то же так
делала, пряталась, ну ещё так теплей было. А когда летели самолёты, мои нервы не
выдерживали и я спускалась в подвал, там было тише и спокойней, хотя сыро и
прохладно. С собой в подвал я затаскивала свою собаку(ротвейлер), ей то же было
очень страшно! Я то же боялась ночи, очень хотелось спать, а тут бомбят, и тдрожишь
,как осиновый лист, и подбрасывало при каждом взрыве! Молились, просили Бога, чтоб
уберёг нас, повторяла -самолёту: улетай, улетай!!!

\iusr{Valentina Buti}

Love and prayer from Italy. Justice and Peace!

\ifcmt
  ig https://scontent-frx5-2.xx.fbcdn.net/v/t39.1997-6/16781161_1341101952618574_7704631035023065088_n.png?stp=cp0_dst-png_s110x80&_nc_cat=1&ccb=1-7&_nc_sid=ac3552&_nc_ohc=GeupXBkePbkAX-jWI1J&_nc_ht=scontent-frx5-2.xx&oh=00_AT8_TbuiBApetads1vmAI7Z_8SfEmpLcsssWhMe41ltgxA&oe=629B8414
  @width 0.1
\fi

\iusr{Bogena Petrovaj}

Но каким быстрым, молниеностым, был путь с первого на пятый, когда муж дома
кормил кошку, а тут прилет( и дверь в квартиру закрыта....

\begin{itemize} % {
\iusr{Liya Koreisha}
\textbf{Bogena Petrovaj} рада видеть что вы тут и отвечаете. Когда видишь знакомых и радуешься что живы... Жаль что не все...

\iusr{Надія Сухорукова}
\textbf{Bogena Petrovaj} и я тоже мировые рекорды ставила, когда бежала вниз во время обстрелов
\end{itemize} % }

\iusr{Elena Romantsova}

Всё, кто прошёл этот ад созвучны душами и мыслями... Я благодарна вам за наши
общие слезы, потому что душа окаменела ещё там...

\iusr{Tania Frolova}
Скажите, Надя, а вот 24 февраля ещё была возможность выехать из Мариуполя или уже нет?

\begin{itemize} % {
\iusr{Marcina Antonova}
\textbf{Tania Frolova} 

я посмотрела интервью Чичваркина и поняла, все сволочи врут, и та сторона и
эта . А кто думает об людях ? Нахер никто, как и Ковид изящез как видео Бори
нашего показали ! Так все врут ЭЛИТА !  @igg{fbicon.anger}{repeat=2} 

\iusr{Ніна Орловська}
\textbf{Marcina Antonova} про що каже неправду українська сторона?

\iusr{Наталья Калинич}
\textbf{Tania Frolova} 

да возможно уехать была, мы уехали 24 в 13:00 из Мариуполя, в соседнее село
Мангуш, потому что оставаться уже было не выносимо, уезжали колонны машин, на
блок посту в Мангуш ещё стояли два человека, никто ни кого не останавливал, они
показывали- проезжайте быстрее. А 25 уже выезд из города был закрыт. И нас
начали окружать со всех сторон...

\iusr{Слава Черненко}
\textbf{Marcina Antonova} не ври

\iusr{Александр Лосев}
\textbf{Tania Frolova} 

нас заверяли в том, что Мариуполь защищён тремя линиями обороны, что его взять
нереально. Перед тем, как окончательно пропала связь, мэр закрыл выезды из
города, \enquote{оставайтесь дома, идёт зачистка дрг} а потом возможность выехать была
у очень немногих. Мне это удалось только в мае и в Россию

\iusr{Elena Ukrainska}
\textbf{Tania Frolova} да, через БП на Мангуш, второй накрыли около 11-12, мне так кажется. Его обстреляли и все машины разворачивали на мангуш.

\iusr{Tania Frolova}
\textbf{Александр Лосев} 

вот я потому и спрашиваю. Я была в Киеве накануне и в первые дни войны- тоже
Кличко уверял, что будет государственная эвакуация рассчитанная на 5 миллионов
человек ( в случае необходимости). На самом деле все было несколько иначе. А
Мариуполь получается сразу стал невозможным для выезда...

\iusr{Александр Лосев}
\textbf{Tania Frolova} сразу. было очень маленькое окошко. пару дней. эвакуационные поезда ушли пустые, потому что о них никто не знал

\iusr{Tania Frolova}
\textbf{Наталья Калинич} 

у меня был рефлекс в первый же день выезжать из Киева. Правильно сделали те
люди, кто выехал сразу. Не все это могли сделать. Для пожилых людей это вообще
нереально, если нет семьи. Это ужас, что люди пережили и ещё переживают.

\iusr{Tania Frolova}
\textbf{Александр Лосев} как пустые? И вот кто за это в ответе? Столько людей пострадало , а могли бы спастись.

\iusr{Александр Лосев}
\textbf{Tania Frolova} а кто в ответе за разминированый Чонгар? как так вышло, что россия за пару дней оккупировала южные регионы? У меня есть версии, но озвучивать не буду, слишком мрачно тогда все получается

\iusr{Tania Frolova}
\textbf{Александр Лосев} 

я понимаю, что ни в жизни, ни тем более в войне нет справедливости. В Киеве
тоже не было никакой информации поначалу. Очень много легло на плечи
волонтёров.

\iusr{Александр Лосев}

мне тяжело об этом говорить. по радио, для которого я искал батарейки, менял их
на вино, Арестович заверял, что еще 2 недели и все изменится. Что Мариуполь
успешно обороняется. по факту выехать было невозможно, связи не было, обстрелы
со всех сторон, люди выезжали на машинах и их расстреливали. Волонтеров не
было, по крайней мере в моем районе. Я считаю, власть самоустранилась. Да, удар
был внезапный, но можно же было сделать хоь что-то. И народ, даже зная, что их
бомбит россия, возненавидел Украину.

\iusr{Svetlana Lysenko}
\textbf{Александр Лосев} да, о них мы тоже не знали

\iusr{Надія Сухорукова}
\textbf{Tania Frolova} наверное была возможность, но никто из нас не выезжал. Никто не знал, что будет такой кошмар. Мы были уверены, что город защитят

\iusr{Наталья Калинич}
\textbf{Елена Украинская} а второй это какой? Который по Донецкой трассе?

\iusr{Elena Ukrainska}
\textbf{Наталья Калинич} нет, конечно. На Запорожскую трассу.

\iusr{Наталья Калинич}
\textbf{Елена Украинская} я поняла

\iusr{Marina Stezhka}
\textbf{Александр Лосев} потому что его бомбили с воздуха, вероятно этого не ждали, основные разрушения из-за этого. Это ужас, что люди пережили господи

\iusr{Marina Stezhka}
\textbf{Tania Frolova} 

многие не видели никакой угрозы, что-то отключается у таких людей. Я в первый
же день настояла на отъезде дочери из Мелитополя, умоляла ее, а она меня
убеждала что ничего страшного, да его не бомбили но вечером 24-го сразу как
только мы ушли туда зашли чеченцы. Я знаю многих кто потом так же убеждал своих
уезжать из Ирпень, Херсон, Мариуполь.

\iusr{Marina Stezhka}
\textbf{Александр Лосев} к сожалению никто не ответит. Этот вопрос не даёт покоя всем.

\end{itemize} % }

\iusr{Елена Ефимик}

Все, кто знал или знает меня, считали меня( простите за тавтологию) сильной
женщиной, кремень-бабой. Убегая от артобстрелов на левом, пережив весь ужас
авианалёта на Троицкой, познав чувства голода, жажды и страха, во мне что-то
надломилось, они уничтожили не только мой дом, квартиру, быт, они РАСТОПТАЛИ
меня @igg{fbicon.face.sad.but.relieved}  не знаю, как жить дальше
@igg{fbicon.face.sad.but.relieved}  без жилья, без копейки обещаных выплат
@igg{fbicon.face.sad.but.relieved}  и не говорите о гуманитарке
@igg{fbicon.face.sad.but.relieved}  Спасибо асвабадителям
@igg{fbicon.face.sad.but.relieved}{repeat=3} 

\iusr{Сергей Забогонский}
Всё в точку
...

\begin{itemize} % {
\iusr{Надія Сухорукова}
\textbf{Сергей Забогонский} Сережа((( все войны кончаются. Эта тоже кончится обязательно
\end{itemize} % }

\iusr{Maria Batova}

Позавчера говорила с одной бывшей жительницей Петербурга. «Мариуполь помогут
отстроить. Грозный же отстроили. Там теперь все хорошо».

На вопрос, зачем было уничтожать Мариуполь - стандартные «они сами» и «там же
были боевики, они убивали людей». (( Я не знаю, как устроен мозг у вроде бы
умных людей с высшим образованием, да ещё уже несколько лет не живущих в
России...

\begin{itemize} % {
\iusr{Kateryna Bilotkach Kurashkevich}
\textbf{Maria Batova} 

там теперь все хорошо?

Пожалуй это ответ на вопросы «зачем».

Можно ее еще потролить, спросив хороший ли Кадыров лидер.

Очень жаль чеченцев. Этот гордый народ «стерилизуют», превращая в безвольных
танкистов-бурятов.

\iusr{Diana Belova}
\textbf{Maria Batova} 

самое страшное читать примерно это у той роженицы, помните, Марианна
Вышемирская, которую, как на грех, сфотографировал фотограф Associated Press.
Даже в самой Украине пропагандой создалась какая-то своя картина мира у части
людей. Что уж про россиян.

\iusr{Roman Kovalchyk}
\textbf{Maria Batova} 

Наверное здесь не вполне корректное сравнение с Грозным. В Чечне Россия так и
не смогла одержать победу. Поэтому решили вопрос путём подкупа клана Кадырова.
Самого Рамзана Россия сейчас щедро заваливает деньгами, званиями и регалиями в
обмен на то, что Кадыровцы держат в узде чеченцев. Что-то и Грозному из этих
денег перепадает. В Мариуполе иная ситуация. Здесь сейчас никто особо
сопротивляться не станет, и своего Кадырова в Мариуполе нет. Да и не нужен. Им
вон баню сделали, банку паштета консервированного дали - и они успокоились. Со
временем сделают свет и водопровод. Ну, на этом все улучшения и закончатся.
Может быть какая-то работенка появится на территории комбината им. Ильича.
Поэтому как Чечню Мариуполь заливать деньгами не станут. И Мариуполь скорее
будет второй Горловкой или Донецком, чем вторым Грозным.

\iusr{Maria Batova}
\textbf{Роман Ковальчик} логично. Это сравнение было в ответ на заявление Беднова о городах-побратимах.
\end{itemize} % }

\iusr{Наталия Склярова}

Я тоже кричала разбудите меня, это страшный сон мне сниться, но у вы это была
реальность ! Или просила дайте мне таблетку чтоб я уснула и проснулась когда
всё это закончиться.

\iusr{Julia Garkusha}
бляяяя фото на подбор... жесть

\iusr{Екатерина Кагал}

Я никак не могу понять. За ЧТО? Почему? Откуда такая жестокость и жЕскость?
Такая ненависть именно к этому городу, к этим людям. Простите, простите...
Надежда, пишите, мы должны все знать, все, все...

\begin{itemize} % {
\iusr{Алия Нургали}
\textbf{Екатерина Кагал}

За желание и волю быть свободными и независимыми, за желание хорошо и по
человечески жить и др.. например расширить границы империи в масштабах бывшего
союза

\iusr{Екатерина Кагал}
\textbf{Алия Нургали} 

нет, я не могу этого понять. Это чудовище, нелюдь, в словаре нет такого
определения, которым можно назвать ЭТО. Болезненные фантазии уничтожили ВСЕ.
Какое горе, какое горе...

\iusr{Людмила Гунько}

Нужен порт и все, на людей этому карлику плевать, он так решил, это Сатана
\end{itemize} % }

\iusr{Margarita Vnt}
Подло украли у людей весну 2022 года.

\begin{itemize} % {
\iusr{Olesya Wehlau}
\textbf{Margarita Vnt} не только это, у них украли будущее и прошлое, а у многих - отняли жизнь и не будет ни будущего, ни прошлого.
Страшно

\iusr{Ирина Сариджа}
\textbf{Margarita Vnt} разве только весну.........

\iusr{Александр Лосев}
\textbf{Margarita Vnt} у нас украли душу. Наш дом, наш город, наше море.

\iusr{Margarita Vnt}
\textbf{Александр Лосев} так. Весну украли у здравомыслящей части планеты Земля.
\end{itemize} % }

\iusr{Лена Варварич Дудник}

Прочитала, а слезы сами катятся, я не с Мариуполя, но также с такого ада,
просто очень страшно. Вы россияне должны остановить войну. Заберите с Украины
своих военных.

\iusr{Sasha Sasha}

А мне говорят родственники их росии: вы должны "благодарить Азов, если бы он не
базировался в Мариуполе, никто бы город не бомбил. Что??? Азов сейчас в плену,
а нелюди продолжают обстреливать наши города с особой жестокостью и садизмом.

\begin{itemize} % {
\iusr{Roman Kovalchyk}
\textbf{Sasha Sasha} 

В Северодонецке и Рубежном нет Азова. Но в этих городах повторяется сейчас
трагедия Мариуполя. Просто эти населенные пункты меньше Мариуполя, и подобная
варварская тактика захвата городов уже не шокирует

\iusr{Dina Tarasenko}
\textbf{Роман Ковальчик} 

Не змогли перемогти Україну за три дні, а тепер, як кажуть експерти, другий
етап під назвою \enquote{випалена земля}. Як було в Іраку та Сірії. Ці виродки тепер
руйнуватимуть до заснування міста та села. Геноцид українського народу! Жах!  @igg{fbicon.face.crying.loudly} 

\end{itemize} % }

\iusr{Алёна Пороховая}

Да, я тоже не знала день недели.. Только знала число.... И считала дни, потом
недели сколько мы находимся в этом аду... Тоже укрывались одеялом... И,
наверное первый раз я поняла, что под одеялом не спрячешься, когда я стояла на
коленях в подвале и укрывала сына одеялом и сама с ним, нагнулась к нему, сосед
по подвалу спокойно так сказал "не стой так, а то если рухнет, то позвоночник
перебьет, лучше плашмя ложись, так привалит, но меньше шансов переломов, может
достанут и целая будешь, тогда я поняла, что одеяло не спасёт...

\iusr{Оксана Спивак}
 @igg{fbicon.heart.broken} 🥲 @igg{fbicon.anger}  @igg{fbicon.hands.raising}  @igg{fbicon.hands.pray} @igg{fbicon.hand.victory}


\ifcmt
  ig https://scontent-frx5-2.xx.fbcdn.net/v/t39.30808-6/284478813_395707372482502_8458762945800969204_n.jpg?_nc_cat=109&ccb=1-7&_nc_sid=dbeb18&_nc_ohc=kMGC8rZxmKcAX9HKKZZ&_nc_ht=scontent-frx5-2.xx&oh=00_AT-_OkyUBor5bMe0DwhGI68OuPEDMTo1Y2blsu8UMyAYFA&oe=629BCBF1
  @width 0.3
\fi

\iusr{Lelya Palna}

Надія, дуже дякую, за те що ділитесь цим. Це важливо. Я дуже співчуваю Вам і
кожному маріупольчанину. Мені довелося бувати у Маріуполі, як і у Донецьку,
Луганську, Краматорську, Лісічанську. Може у Вас є думки, чому люди з інших
міст де є можливість євакуюватись не виїзджають? Я розумію, що є старенькі,
маломобільні. На останні автобуси у Попасній приходило по 2 родини, не вже це
така недовіра?

\begin{itemize} % {
\iusr{Надія Сухорукова}
\textbf{Lelya Palna} 

люди не вірять, що може бути пекло. Ми теж до останнього трималися за своє
місто, свій будинок. Дуже важко було наважитися. Хоч нас бомбували постійно

\end{itemize} % }

\iusr{Людмила Петровская}

\ifcmt
  ig https://scontent-frx5-1.xx.fbcdn.net/v/t39.30808-6/285168544_329867952634689_3607350032166624038_n.jpg?_nc_cat=110&ccb=1-7&_nc_sid=dbeb18&_nc_ohc=zXZEq7Goj-wAX_xVDU3&_nc_ht=scontent-frx5-1.xx&oh=00_AT-A1P0CdpS6_mgWgUKlgqXQN772P2AZKaM-MwCdxHQozw&oe=629A2CDA
  @width 0.3
\fi

\iusr{Наталія Беспалова}

Кожною часткою тіла я відчуваю Ваші слова.. Кожною. Знову там. Знову чую свист
міни.. Знову вилітають вікна ... Знову літак, я падаю на коліна.. закриваю
голову руками...

\iusr{Iryna Maksymuk}
Серце розривається. Маріуполь @igg{fbicon.heart.broken} 

\iusr{Ruslan FoXx}

За каждую слезинку ребенка, за каждого убитого и раннего человека - за всех
невинноубиенных - будут прокляты те, кто прямо или косвенного был причастен к
этому. По делам или одобрению, по знанию или незнанию, согласно утверждению
Соломона и книге Судей, будут прокляты сии до седьмого колена, и дети их, и все
родоплеменные колени и их дети. Черная меланхолии поразит их души и да не будет
в них мира отныне. Аминь.

\iusr{Танюша Маслова}
Надя, пишите хроникальную книгу и комментарии туда же пишите.

\iusr{Ядвига Ядвіга}

Надюша, Ви пережили, та продовжуєте переживати ад, який тяжко дивитися навіть на
фото... Не повинна, не може жіноча душа переносити страшні жахіття.... Не повинна
жодна людина переживати те, що переживає Велике Місто Маріуполь ...Я не знаю, як
допомогти маріупольцям, серце від цього рветься на частини!

Страждаю, плачу, молюся за всіх маріупольців.

Вінниця з вами, дорогі люди...

\begin{itemize} % {
\iusr{Надія Сухорукова}
\textbf{Ядвига Ядвіга} дякую Вам. Дякую Вінниці
\end{itemize} % }

\iusr{Юліана Ніколіца}

З першої спроби, не змогла прочитати цей текст повністю. А як тим, хто через це
пройшов? Дякую за те, що розказали.

\iusr{Irina Grigalevičienė}

Это ужасно! Не укладывается в голове, как это возможно. Те кто это совершил-
нелюди! Те, кто говорит, что это фейки и оправдывают этот ужас такие же уроды,
орки. Нет им прощения !

\iusr{Владимир Платонов}

Саме у тій \enquote{Школі мистецтв} ми й переховувалися майже два тижні. Перший
авіаудар по нам був 15 березня, приблизно о третій годині дня. Скоріше за все,
то була ракета, бо бомба знищила б будівлю з одного разу (я бачив ту жахливу
вирву від авіабомби біля дитячого відділення третьої поліклініки). Ракета
влучила у вікно третього поверху, праворуч від центрального входу, та пробила
всі міжповерхові перекриття аж до цокольного поверху. Приміщення сховища майже
миттєво заповнилося пилом, димом та порховими газами. На той час там перебувало
приблизно 150 людей. Було багато дітей різного віку (одній манюні було трохи
більше місяця). А ще з нами там були котики та собачки. Одну собачку вбило
уламком міни трьома днями раніше...

Приблизно через півтори години літак русофашистів здійснив ще одну атаку на
нас, але вже з протилежної сторони. Ледь не влучив у вхід до сховища. Уламками
побило багато автівок, які були припарковані поруч (деякі так там й залишилися
через побиті колеса та радіатори).

На щастя, тоді ніхто не загинув. Лише мені дісталося, бо на той час перебував у
дворі: отримав легку контузію та відносно легке поранення у груди.

Сенсу чекати на міфічний зелений коридор більше не було, тож наступного ранку
ми вже тікали від неминучої смерті.

А вже через кілька годин літак розбомбив Драматичний театр, біля якого на площі
величезними літерами було написано \enquote{ДЕТИ}...

\iusr{Marina Stezhka}

А те кто сейчас в Мариуполе, что они думают? Как они налаживают свою жизнь?

\iusr{Anni Krasova}

\obeycr
\bfseries
Anonymous
Boris Johnson
United Nations Human Rights
Recep Tayyip Erdoğan
OSCE – The Organization for Security and Co-operation in Europe
Reuters UK
Olaf Scholz
International Committee of the Red Cross
ElonMusk
Emmanuel Macron
Angelina Jolie.
\restorecr

\iusr{Irina Golovach}
Дякую

\iusr{Elena Nowak Panashchenko}

Я розумію по фото, що там жили тільки \enquote{фашисті і націоналісти}, бо як
можна таке зробити з містом?!!...

Якийсь ген у руzkіх гніє...


\end{itemize} % }
