% vim: keymap=russian-jcukenwin
%%beginhead 
 
%%file topics.ukraina.rossia.ugroza_napadenia.preface.3.slova
%%parent topics.ukraina.rossia.ugroza_napadenia.preface
 
%%url 
 
%%author_id 
%%date 
 
%%tags 
%%title 
 
%%endhead 
\clearpage
\subsubsection{Слова, слова, слова...}

Февраль 2022, Киев, обычная киевская квартира. Написано еще до признания
Путиным республик Донбасса. Касательно мнимого вторжения России в Украину, то
бишь, Российской Федерации в Малую (Киевскую) Россию и словесном потоке, его
сопровождающем.  Знаете, это даже интересно за этим наблюдать. Какой-то
массовый психологический эксперимент происходит, и самое удивительное, что в
нем принимают участие самые могущественные страны и люди, - США,
Великобритания, президент США, премьер-министр Британии... Так и хочется
ущипнуть себя за щеку и спросить - божечки, неужели это таки происходит на
самом деле, они там что, совсем все уже ку-ку? Чего вы там все так подвисли на
Украину и Россию, у вас что, своих проблем не хватает?

\ifcmt
  tab_begin cols=2,no_fig,center

     pic https://kiev.sq.com.ua/img/news/2017/11/11/352975_900.jpg
     pic https://my-kiev.com/wp-content/uploads/2019/09/1.jpg

  tab_end
\fi

Вроде с осени 2021 пишут, что вот-вот уже сейчас вторгнется, а сейчас уже
февраль 2022 года, - и поразительно! - в Киеве спокойно и тихо. Спокойно
работает метро, ездят машины, ходят пешеходы. Влюбленные парочки рассматривают
Днепр, а компания подростков что-то там увлеченно обсуждает.  Как же так? Уже
столько написано, уже как бы ясно - что все это - один гигантский
информационный пузырь, а до сих пор находятся люди, которые во всю эту чушь
неистово верят. 

\ifcmt
  tab_begin cols=2,no_fig,center
     pic http://bestbridge.net/data/upimages/pedestrian_bridge9.jpg
     pic https://i.pinimg.com/originals/2d/34/d7/2d34d7f0964d19321e7aa8b3594780fa.jpg
  tab_end
\fi

На границе страшный враг, он обязательно хочет нас уничтожить, злобный карлик в
Кремле день и ночь не спит, думая, как бы захватить бедную несчастную
Украиночку. В чем же дело?  Почему постоянная ложь и нагнетание со стороны
Западных сми до сих пор проходит в Украине? Наверное, тут есть такие причины:

(1) В информационном обществе 21 века ценность слова как такового, да, именно
Слова, с большой буквы напишем, вот, - стала очень низкой, и вообще
отсутствующей. Люди в фейсбуке, телеграме и СМИ пишут что попало и как попало.
Какой-то сплошной базар, один кричит одно, другой кричит другое, и громче всего
кричат те, у кого совести и разума меньше всего.  А в обычной жизни как? Если
ты что-то пообещал, своему другу например, - то надо исполнить обещанное, - а
то можно и друга потерять, нажив вместо него врага. Можно и по морде получить,
и вообще могут прибить, если что-то не то сказал. Это называется по-простому
отвечать за свой базар, или же, пацан сказал, - пацан сделал. А в фейсбуке
наоборот, - можешь писать ужасные вещи, и ничего тебе за это не будет. Можно
постоянно оскорблять людей по языку, вере, - и нет средства, как утихомирить
безумца. Вот, например, напишет человек \enquote{вата}, или же
\enquote{москаль}, и что? - а вот попробуй ты, например, войти в утреннюю
маршрутку, и крикнуть, - эй! кто здесь ватник?  или же - вата, на выход! -
слабо, да? Или же \enquote{москаль}, - ты вот написал где-то, - всех москалей
на ножи! Ура! - а теперь, представь себе... ты выходишь на Красную Площадь в
Москве, и орешь - москаляку на гилляку! - слабо, да?

\ifcmt
  tab_begin cols=2,no_fig,center
     pic https://culttourism.ru/data/photos/8/1/81dda7a7727963c89e5042e38efb41e0.jpg
     pic http://myukr.com/images/photos/medium/8382ad7ee7fa01a4a61a908777a67d71.jpg 
  tab_end
\fi

И в итоге, что касается ценности Слова, возникает катастрофический разрыв между
объективной реальностью, и тем, какие именно слова описывают эту реальность.

\ifcmt
  tab_begin cols=2,no_fig,center
     pic https://avatars.mds.yandex.net/i?id=fbcf526e23ae10246b5aa8cf6334a090-5877774-images-thumbs&n=13
     pic https://pbs.twimg.com/media/CH9WWwNW8AQOcSY.jpg
  tab_end
\fi

Как следствие, это ведет к тому, в частности, что Украина как государство
потеряло весь свой международный престиж, потому что Украина как государство не
в состоянии выполнять то, что было подписано Украиной как государством,
например, те же пресловутые Минские договоренности. Это просто какой-то позор.
Это ж международный договор, - если подписал, изволь исполнять.  Может быть это
довольно и неприятно, но раз подписал, - нужно исполнять. А то подписали, и не
исполняем совершенно. Торжественно пожали руки, подписали, и уже столько
времени не можем исполнить. Хаха, думаем, какие все лохи вокруг, а мы -
конечно, самые умные, и пушистые. Но в итоге, какое впечатление могло сложиться
вокруг? Серьезных людей и государств, имеем в виду. Бумага, на которой
подписаны соглашения с Украиной, не стоят ничего. Это государство-призрак,
которое как бы существует, а по факту - просто Дикое Поле.

\ifcmt
  tab_begin cols=3,no_fig,center
     pic https://i.pinimg.com/736x/29/cd/45/29cd453cd53f31a77417ee351aebdcc3.jpg
     pic https://avatars.mds.yandex.net/get-zen_doc/1671806/pub_5fedaec8d1a90641ca6009b0_5fedaee7f906b16872500758/scale_1200
     pic https://mayak.org.ua/wp-content/uploads/2020/12/pole.jpg?resize=2048%2C1362
  tab_end
\fi

(2) Сон разума порождает чудовищ. Это, вроде, Ницше сказал. И да,
действительно, к сожалению, ты, Украина, на почве собственных исторических
поражений и горестных испытаний взрастила в своей душе чудовищ, в которых ты
веришь и которых ты неистово боишься/ненавидишь. Мы имеем в виду демонизацию Российской Федерации,
граждан России, и лично президента Владимира Путина. Уж сколько желчи вылилось
в Украине в отношении России, русского языка, россиян и Путина! \enquote{рабы},
\enquote{кремлевский карлик}, \enquote{мокшане}, \enquote{московиты},
\enquote{мордор}, \enquote{орки}, \enquote{орда}, \enquote{хто не скаче, той
москаль}, \enquote{путин - лалала}, и тому подобное! Просто удивительно, как
ты, Украина, так загадила свою душу за такое короткое время!

\ifcmt
  tab_begin cols=3,no_fig,center

     pic https://st3.depositphotos.com/1116619/12490/i/950/depositphotos_124904410-stock-photo-toddler-boy-holding-russian-flag.jpg
     pic https://avatars.mds.yandex.net/i?id=e5c157d79bc9e5d2f887aca8872d3314-5608305-images-thumbs&n=13
     pic https://avatars.mds.yandex.net/i?id=03e23deac64312b4c9f62bb63463a099-5495958-images-thumbs&n=13

  tab_end
\fi

Ты придумала себе злобного врага в своем
воображении - Россию, и ты в это веришь. Да, конечно, много всего происходило в
прошлые времена между Москвой и Киевом, между Московским Царством и
Малороссией, между Россией и Украиной.  Но это еще не повод так сильно и
неистово ненавидеть. Портос сказал - дерусь! - потому что дерусь, а ты,
Украина, ненавидишь Россию... потому что... ну, потому что это Россия, и
украинцам ее полагается ненавидеть, это ж и ослику понятно! Ты ненавидишь само
слово Россия, разве не так? Как только скажешь - Россия! - такая реакция идет в
так называемом патриотическом лагере, как будто красной тряпкой помахали перед
разъяренным быком! 

\ifcmt
  tab_begin cols=3,no_fig,center
     pic https://avatars.mds.yandex.net/i?id=db301da57610d65bb352fe9260c2add6-4327755-images-thumbs&n=13
     pic https://sharij.net/wp-content/uploads/2021/08/232836328_252285156710407_3016163210169121650_n.jpg?x47119
     pic https://playmaker24.ru/wp-content/uploads/2021/08/licensed-image-1-6.jpg
  tab_end
\fi

Магучих обнялась с Ласицкене - такое возбуждение понеслось кипятком у
истеричных патриотиков! А скажи, откуда это все взялось? Разве Тарас Шевченко,
Леся Украинка и Иван Франко призывали ненавидеть Россию, всю, целиком, - от
Ярославля до Владивостока, от Эрмитажа до Иркутска? Откуда вообще эта
ненависть, всепоглощающая, убийственная ненависть к России взялась? - Разве не
потому, что ты - Украина - сама есть Россия - более того, самая центральная,
изначальная ее часть? Можно сказать даже так, самая российская Россия, Россия в
квадрате, даже в кубе? - потому что пословицы о России как раз очень точно
ложатся и на жизнь в Украине - потому что закон как дышло, куда повернул - туда
и вышло; потому что строгость российских законов смягчается необязательностью
их исполнения; потому что если в России воруют, то в Украине воруют втройне;
если в России плохие дороги, то в Украине воронки на дорогах способны просто
убить машину напрочь; и если в России есть дураки, то Украина - это просто
нескончаемый дурдом, палата номер шесть?

\ifcmt
  tab_begin cols=2,no_fig,center

     pic http://s3.fotokto.ru/photo/full/290/2908475.jpg
     pic https://travel-guia.com/wp-content/uploads/2017/11/que-ver-en-kiev-ucrania-5.jpg
     %pic https://img-fotki.yandex.ru/get/6408/30777698.18/0_6ef2c_13e9c2b2_XL.jpg

  tab_end
\fi

(3) Потеря Украиной понимания слов Россия, Русь, Русский, Русин, утрата своего
исторического имени. Русь - это же Украина. Южная Русь, Южная Россия, Малая
Россия, Киевская Русь, правда?  Откуду есть пошла Руская Земля, знаете, такой
есть камень с такой надписью возле исторического музея в Киеве, возле развалин
Десятинной Церкви?  А значит Украина = Русь = Россия = Russia, потому что
Россия - это греческий вариант слова Русь. И не надо этого стесняться,
совершенно! 

Говорят, что Московия украла у Украины это слово, говорят, так Петр Первый
сделал. Это как украл, как это вообще возможно украсть имя?! в карман что ли
исподтишка положил, вынув наверное из саркофага с Ярославом Мудрым в Софии
Киевской, пока охрана дремала? Вынул воришка - и бросился бежать в Московию,
взяв в заложники с собою Феофана Прокоповича... Так что, насчет кражи, мы
думаем, что это чушь собачья, все это какая-то бессмыслица. На самом деле,
конечно, Московское Государство всегда было Русью, если брать летописи и
исторические документы. И чрез века, через столетия Москва сохранила
посредством своей государственности, иногда очень жестокой и кровавой, если
вспомнить Ивана Грозного и других московских царей, само слово - Русь (Россия),
а это дорогого стоит! Слово Русь (Россия), не исчезло, не растворилось, а
наоборот, сохранилось и окрепло... И если Вы начнете внимательно анализировать
современную Россию, то придете к выводу в итоге, что Россия - это Россия, а
никакая не орда, не мордор, не оркостан. Но что верно, это то, что как раз
Украина как общество потеряло в себе ценность слова Русь, в конце концов
Украина просто напросто забыла, что она - тоже Русь, а значит Россия. Ну,
исторически так сложилось. Поэтому и такой истерический страх перед огромным
Русским Медведем, и на Западе тоже, потому что в плане слов кажется, что Украина
- это совсем другое, чем Россия. На самом деле, конечно, Украина, то есть Южная
Россия различается в чем-то от Северной (Московской) России, прежде всего в
традиции государственности, и конечно в том, что в Украине сформировался также
второй язык, второй русский язык, - украинский, удивительно певучий и
волшебный, - но различия в итоге все-таки не так велики и фундаментальны, как
кому-то хочется.  И поэтому возможное нападение (вторжение) России на Украину -
это не нападение одного мира на совершенно другой мир, это конфликт двух частей
одного организма. А касательно единого организма, так подтверждений этому
тысячи, - украинец Гоголь писал на русском для всей необъятной России, а Марко
Вовчок - наоборот, как раз для народа Украины писала на украинском, болея душой
за украинский язык и культуру.

\ifcmt
  tab_begin cols=2,no_fig,center
     pic https://avatars.mds.yandex.net/i?id=48726179bd68044823ce73cf1ec03db5-5520851-images-thumbs&n=13
     pic https://i.pinimg.com/originals/1a/53/5f/1a535fb3238265e04072a4af0ec07263.jpg
  tab_end
\fi

Далее, Елизавета Антонова, российская певица, с удовольствием поет
\textbf{\emph{Ой у вишневому саду}}, - и ее исполнение, возможно, самое лучшее
из того, что мы видели или слышали! - совершенно волшебное исполнение! - а с
другой стороны, - фильмы студии Квартал 95 постоянно крутят по всей России.
Одни переезжают из бандеровского Киева в Москву за порядком и имперским
величием, а другие, наоборот, бегут от ужасного путинского режима в свободную и
демократическую Украину, - ну, кто во что верит... Ну а Роман Цымбалюк,
постоянно поливая грязью русских и Россию, тем не менее, упрямо держится руками
и ногами за Москву, - это ж парадокс из парадоксов! И так далее, и так далее,
до бесконечности. 

\ifcmt
  tab_begin cols=2,no_fig,center
     pic https://i2.paste.pics/1a73df6aa1c0f4f4ad7ae049e995ab5c.png
     pic https://i2.paste.pics/fcbf09a7d7a0a07d755fe93742026e17.png
  tab_end
\fi

\raggedcolumns
\begin{multicols}{2} % {
\setlength{\parindent}{0pt}
\obeycr
Ой, у вишневому саду, там соловейко щебетав.
Додому я просилася, а ти мене все не пускав.
Додому я просилася, а ти мене все не пускав.
\smallskip
Ти, милий мій, а я твоя, пусти мене, зійшла зоря.
Проснеться матінка моя, буде питать де була я.
Проснеться матінка моя, буде питать де була я.
\smallskip
А ти їй дай такий отвіт, яка чудова майська ніч.
Весна іде, красу несе, а тій красі радіє все.
Весна іде, красу несе, а тій красі радіє все.
\smallskip
Доню моя, не в тому річ, де ти гуляла цілу ніч.
Чому розплетена коса, а на очах горить сльоза?
Чому розплетена коса, а на очах горить сльоза?
\smallskip
Коса моя розплетена - її подруга розплела.
А на очах горить сльоза, бо з милим розлучилась я.
А на очах горить сльоза, бо з милим розлучилась я.
\smallskip
Мамо моя, прийшла пора, а я весела-молода.
Я жити хочу, я люблю, мамо, не лай дочку свою.
Я жити хочу, я люблю, мамо, не лай дочку свою.
\restorecr
\end{multicols} % }

\ifcmt
  tab_begin cols=2,no_fig,center

     pic https://i2.paste.pics/e49efd6480c086ac3b80a14729b1dea3.png
     pic https://i2.paste.pics/0d57e068ffc52f936230ebcaf0dff9ad.png

  tab_end
\fi

В общем, слово такое есть, междуусобица.  Москва и Киев выясняют отношения,
причем Украина - молодое Киевское государство - только в начале своего
осмысления, своего пути, своего предназначения... и к сожалению, еще очень мало
осознающее силу Киева, да, собственно Киева, тысячелетнего Города Киева в своей
внутренней государственной архитектуре, и в своем будущем. Поразительно, но
факт. Вы так часто поете, пишете об Украине, но так мало о Киеве... о самом
обычном Городе Киеве, которому Украина, в общем-то, всем обязана, начиная от
крещения Руси и кончая собственно своей нынешней государственностью. Вместо
Киева, который по идее должен быть на первом месте в Вашем сознании, который
есть самый главный Город, где происходит все самое главное и интересное, у Вас
Украина, Украина, и еще раз Украина. Карпаты, красивые песни, вишиванки,
Бандера, Петлюра, жовто-блакитные флаги величиной с футбольное поле, вареники
величиной с вагон метро... все в общем, все кроме Киева! Вот, кстати,
как пример, песня Натальи Бучинской, \emph{\textbf{Україна - це ми!}},
вдумайтесь, прослушайте - все об Украине, о Карпатах, Черном море и так далее,
но вообще ни слова о Киеве, просто удивительно! Так что... Киев для украинцев,
пока что - ну... столица как столица. Удобный город, есть метро, аэропорт
Борисполь; много возможностей для работы, учебы и развлечений. Можно пойти на
байдарке покататься в парк Дружбы народов, а можно - на ВДНХ на каток. Да,
еще... раз в году споем про Киев на День Киева, ну и хватит. Торжественно
признаемся в любви к Киеву, мэр Кличко даже созвонится с постаревшим железным
Арни, потом, весело потусим на Софиевской площади, или же на Спивочем поле, и
тут же забудем, бросая окурки на Крещатике, и прижимаясь друг к друг где-нибудь
в переполненной маршрутке на Троещине. И это кстати интересно проявилось также
в том, что западные посольства так легко слиняли из Киева во Львов, -  а все
почему? А потому что у этих всех иностранцев нет совершенно осознания, что их
посольство находится в Стольном Граде, в Столице, в совершенно уникальном
Городе. Ну, Киев - ну и что?  Взяли, переехали во Львов, хотя Львов - это всего
лишь один из многих городов, Киеву - совершенно не ровня, ну, есть там кофе,
ну, есть там старуха Фарион, ну и что? а Киев...  чудесный, неповторимый Киев
один такой на всей Земле. Это не Нью-Йорк, не Москва, это Киев, тысяча чертей!
Нет другого Киева, как нет и другого Днепра!

\ifcmt
  tab_begin cols=2,no_fig,center
     pic https://vesti-ua.net/uploads/posts/2018-01/1516176721_2105450.jpg
		 pic https://wozap.ru/uploads/posts/2018-02/15196437181211.jpeg 
  tab_end
\fi

Возвращаясь к вопросу о вторжении, что касается обилия карт о массированных
ударах по Украине со всех сторон, -  в общем-то, Москве совершенно не нужно
прямое вооруженое нападение на Киев, это ж не времена 11-12 века, когда
совершенно нормально было, что один князь шел на другого, киевляне шли на
новгородцев, а суздальцы (владимирцы) шли на киевлян, - ну вот как Андрей
Боголюбский разграбил Киев, - времена же были дикие и жестокие, - России вообще
не нужно уничтожение Украины, поверьте. Все эти вопли про злобного врага, это,
извините, просто клиника. 

Ну ок, а что ей, Москве и вообще всей огромной России, нужно от нас, от Киева,
на самом деле сейчас?  Нормальные дружественные отношения, построенные на
здравом смысле и взаимопонимании. Вот и все, в общем, что Москве нужно.
Говорите, идет гибридная (информационная) война? Мы честно не знаем, что такое
есть гибридная (информационная) война. Что такое велосипед-гибрид, мы знаем
точно, а что такое гибридная агрессия (война), - это нам неведомо. 

А то, что на границе сейачс стоит стотысячная армия, - ну так ты, Украина, уже
здорово таки постаралась, постоянным вылизыванием и преклонением перед так
называемыми Западными партнерами, постоянным нытьем о вступлении в НАТО, что
совершенно является незаконным согласно Декларации о Суверенитете, - и вместе с
конфликтом на Донбассе, чтобы это произошло, чтобы Москва начала видеть в тебе,
Киев, действительную угрозу своей безопастности как государству, и начала
предпринимать какие-то шаги, чтобы немного остудить зарвавшихся обитателей
Банковой.

И Российская Федерация тут находится в сложном положении, можно тут примерно
вообразить, что там думают в Кремле, - с одной стороны, напрягает вся эта
накачка оружием, все эти крики, что мы направим ракеты на Москву, сожжем
Кремль, - вся эта антироссийская истерия, - а с другой стороны, это ж тоже наши
люди, единый народ. Да, люди заболевшие, люди опустившиеся, но все-равно наши
люди.  И вон, второй человек в РФ - Валентина Матвиенко, - председатель Совета
Федерации, - и фамилия украинская, - и родилась в Шепетовке, корни то здешние,
украинские. И не бить же ракетами по Киеву, это ж вообще Матерь Городам
Русским! - как можно вообще думать о таком? И что в итоге делать, не очень
понятно. Как-то хочется приструнить Украину, чтобы она стала нормальной
адекватной страной и перестала трепать нервы, - а как это сделать, совершенно
не ясно.

\ifcmt
  tab_begin cols=3,no_fig,center
     pic https://anews.com/upload/post/2020/11/04/116731351/gallery/tn/943287415.jpg
		 pic https://anews.com/upload/post/2020/11/04/116731351/gallery/tn/555144511.jpg
		 pic https://anews.com/upload/post/2020/11/04/116731351/gallery/tn/247635349.jpg
  tab_end
\fi

Ну и какие выводы изо всего этого потока сознания, скажете Вы? Ты тут, товарищ,
наговорил с три короба, а что ты, собственно, хотел сказать?

Нужно учиться здраво и неторопливо рассуждать; успокоить эмоции, и понять - что
нужно договариваться обо всем, восстанавливать ответственность за сказанное, и
конечно, - прежде всего закрыть вопрос с Крымом, и достичь мира на Донбассе, а
также восстановить добрососедские дружественные отношения с Российской
Федерацией. Это очевидно в наших интересах. Вы хотите сказать, что это зрада
зрадная? Ну конечно зрада, но также и перемога. Как Гегель еще говорил - тезис
- антитезис - синтез.  Совершенно полная зрада, и одновременно полная перемога.
Переможная зрада, или же - зрадная перемога. Потому что это в конце концов,
победа разума над глупостью, мудрости над идиотизмом.  И было бы хорошо
все-таки достичь этой победы, особенно учитывая то, что как раз Григорию
Сковороде, нашему философу-любомудру, скоро стукнет (или уже стукнуло - смотря
когда это будет прочтено) 300 лет. Надо ж чем-то порадовать старика! А то он
смотрит на нас с небес, знаете, и наверное думает...  ууу...  как тут все
запущено... Я в свое время столько всего понаписывал, всю Украйну пешком
обошел, столько всего осмыслил и понял! - а никто этого не читает, и мало кто
следует моей философии в этой нынешней Украйне... Запретили язык, на котором я
писал свои сочинения, извратили столько всего... Бедная несчастная Русь,
потерявшая разум и совесть... Как жаль... И раз мы его тут вспомнили походу,
позвольте вставить сюда и его слова.

\raggedcolumns
\begin{multicols}{2} % {
\setlength{\parindent}{0pt}
\obeycr
Всякому городу нрав и права;
Всяка имеет свой ум голова;
Всякому сердцу своя есть любовь,
Всякому горлу свой есть вкус каков,
\smallskip
А мне одна только в свете дума,
А мне одно только нейдет с ума.
Петр для чинов углы панские трет,
Федька купец при аршине все лжет.
\smallskip
Тот строит дом свой на новый манер,
Тот все в процентах, пожалуй, поверь!
А мне одна только в свете дума,
А мне одно только нейдет с ума.
\smallskip
Тот непрестанно стягает грунта,
Сей иностранны заводит скота.
Те формируют на ловлю собак,
Сих шумит дом от гостей, как кабак, —
\smallskip
А мне одна только в свете дума,
А мне одно только нейдет с ума.
Строит на свой тон юриста права,
С диспут студенту трещит голова.
\smallskip
Тех беспокоит Венерин амур,
Всякому голову мучит свой дур,
А мне одна только в свете дума,
Как бы умрети мне не без ума.
\smallskip
Смерть страшна, замашная косо!
Ты не щадишь и царских волосов,
Ты не глядишь, где мужик, а где царь, —
Все жерешь так, как солому пожар.
\smallskip
Кто ж на ея плюет острую сталь?
Тот, чия совесть, как чистый хрусталь...
\restorecr
\end{multicols} % }

\ifcmt
  tab_begin cols=2,no_fig,center
     pic https://avatars.mds.yandex.net/i?id=b346d4fbf91207103d8d7e9cb6502761-4936279-images-thumbs&n=13
     pic https://avatars.mds.yandex.net/i?id=77819ddb0ae4982d576cb6551fd63c99-5875539-images-thumbs&n=13
  tab_end
\fi

Так что, в общем так, было бы хорошо сделать следующее:

(1) признать в полном объеме результаты крымского референдума 2014 года,
признать Крым частью Российской Федерации, и закрыть этот вопрос навсегда; если
для Москвы этот вопрос закрыт, то он должен быть закрыт и для Киева, потому
что...  черт возьми! Стольный Град Киев старше Москвы, древнее Москвы, - если в
Москве Кремль, то в Киеве Лавра, София, и ладья с Кием, Щеком, Хоривом, и их
сестрой Лыбедью, и поэтому Киев должен использовать свои тысячелетние Мудрость
и Разум, чтобы... чтобы все было хорошо в итоге.  Да... Кто бы там что ни
говорил про зеленых человечков и принудительное голосование под ружьями, а
крымский поезд уже давно ушел в Россию, Украина, а ты все сиротливо стоишь на
платформе, машешь флажками и орешь бессмысленные лозунги, размахивая
бесполезными бумажками. Ну так люди решили, понимаешь ли ты Украина, что есть
такое слово: \enquote{Люди}? Крымчане захотели в Россию на фоне горящего Киева,
подобно тому как майдановцы восстали в свое время против Януковича, избиения
студентов и так далее - ну а вежливые молчаливые люди им (крымчанам) очень
вовремя помогли, и это уже история, это никак не изменить. Увы, шо упало, - то
пропало, и надо научиться в конце концов принимать реальность такой, какая она
есть. Ты ж не хочешь продолжать плакать за Крымом еще сто лет, Украина? Ну так
вытри слезы, Украина, и займись Киевом для начала, сделай его настоящим центром
своей государственности, культуры и науки, столько ж еще работы впереди! 

\ifcmt
  tab_begin cols=2,no_fig,center

     pic https://avatars.mds.yandex.net/i?id=18787a154af3f61cfa15d38829a29d88-5661278-images-thumbs&n=13
     pic http://900igr.net/datai/geografija/Istorija-SNG/0039-029-Istorija-SNG.jpg 

  tab_end
\fi

(2) признать в полном объеме независимость Донецкой и Луганской Народной
Республик, установить с ними взаимные дипломатические отношения, перестать
стрелять по селам и городам, и вернуть войска домой, начав процесс всеобщего
примирения и прощения, чтобы дончане могли без проблем ездить в гости в Киев, а
Киевляне - наоборот в Донецк или Луганск. И это будет лучшей реинтеграцией
Донбасса в Украину, разве не так?
