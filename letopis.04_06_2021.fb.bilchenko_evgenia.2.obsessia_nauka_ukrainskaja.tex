% vim: keymap=russian-jcukenwin
%%beginhead 
 
%%file 04_06_2021.fb.bilchenko_evgenia.2.obsessia_nauka_ukrainskaja
%%parent 04_06_2021
 
%%url https://www.facebook.com/yevzhik/posts/3952510274784053
 
%%author Бильченко, Евгения
%%author_id bilchenko_evgenia
%%author_url 
 
%%tags 
%%title БЖ. Украинские формы обсессии в науке
 
%%endhead 
 
\subsection{БЖ. Украинские формы обсессии в науке}
\label{sec:04_06_2021.fb.bilchenko_evgenia.2.obsessia_nauka_ukrainskaja}
\Purl{https://www.facebook.com/yevzhik/posts/3952510274784053}
\ifcmt
 author_begin
   author_id bilchenko_evgenia
 author_end
\fi

БЖ. Украинские формы обсессии в науке.

Только не смеяться. Можно узнать, какое отношение к проблематике серьезной
научной конференции \enquote{Человек в цифровом мире: эффекты смерти, театра,
\enquote{занавеса} и Реального как феномены современного искусства} имеют
идеологические темы \enquote{русско-украинской войны}, \enquote{ложной
концепции ВОВ в России} и \enquote{что прячут \enquote{левые} культурологи под
неомарксизмом?} Я уже не говорю об абсурдности тем. У меня тупой вопрос: почему
больное колено лечат в стоматологии? 

Мои русские, европейские и американские друзья, простите. 

Не, ребята, если бы я знала, что это стадион, а не академическая конференция, я
бы вышла с алой звездой по теме Победы 9 мая, стадион, так стадион. На мечах  я
тоже умею. Но вот такой биполярки я не поняла. Я думала: Кант, Гегель, Лотман,
Лакан, Хичкок... Наука же? Или все, похороны науки? \verb|#поржать|
