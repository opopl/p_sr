% vim: keymap=russian-jcukenwin
%%beginhead 
 
%%file 02_11_2021.stz.news.ua.mrpl_city.1.juzef_vladislav_nikolchenko
%%parent 02_11_2021
 
%%url https://mrpl.city/blogs/view/yuzef-vladislav-nikolchenko-vikladach-yakij-nadihae-vchitisya
 
%%author_id demidko_olga.mariupol,news.ua.mrpl_city
%%date 
 
%%tags 
%%title Юзеф-Владислав Нікольченко: викладач, який надихає вчитися
 
%%endhead 
 
\subsection{Юзеф-Владислав Нікольченко: викладач, який надихає вчитися}
\label{sec:02_11_2021.stz.news.ua.mrpl_city.1.juzef_vladislav_nikolchenko}
 
\Purl{https://mrpl.city/blogs/view/yuzef-vladislav-nikolchenko-vikladach-yakij-nadihae-vchitisya}
\ifcmt
 author_begin
   author_id demidko_olga.mariupol,news.ua.mrpl_city
 author_end
\fi

\ii{02_11_2021.stz.news.ua.mrpl_city.1.juzef_vladislav_nikolchenko.pic.1}

Цього разу я хочу розповісти про викладача, чия унікальна діяльність на ниві
вузівської освіти надихнула багатьох студентів, зокрема і мене, стати
викладачами та займатися наукою, прагнути до самовдосконалення і постійно
вчитися. Ім'я \textbf{Юзефа-Владислава Мойсейовича НІКОЛЬЧЕНКА} вже багато років відоме
далеко за межами не тільки Маріуполя, а й України, воно є символом порядності,
патріотизму, вірності своїй справі. В одній особі поєдналися непересічний
талант і велика працелюбність, відповідальність і професіоналізм,
інтелігентність і гострий розум, вражаюча ерудиція та мудрість.

\ii{02_11_2021.stz.news.ua.mrpl_city.1.juzef_vladislav_nikolchenko.pic.2}

Юзеф-Владислав Мойсейович народився 2 листопада 1946 року в містечку Сарни
Рівненської області. Батько був військовим лікарем в Сарнах, а мама стала
першою лікаркою-педіатром, яка прибула в Сарни в 1944 році після його
звільнення від німецько-нацистських загарбників. Подвійні імена дітей були
пов'язані з католицькими традиціями сім'ї батька. У брата Юзефа Мойсейовича
(так звертаються до нього колеги і студенти) також було подвійне ім'я –
Борис-Євген. До речі, вони з братом були двійнятками, хоча інтереси у
майбутньому у них були різними – Борис більше захоплювався природничими
науками, а Юзеф – гуманітарними.

\ii{02_11_2021.stz.news.ua.mrpl_city.1.juzef_vladislav_nikolchenko.pic.3}

У 1951 році сім'я переїхала до Рівного, мама – \textbf{Олександра Митрофанівна
Нікольченко} – була призначена завідувачкою ди\hyp{}тячого відділення Рівненської
обласної лікарні. А через рік отримала посаду головного дитячого лікаря
Рівненської області, яку обіймала до смерті у 1979 році У 1969 році вона
отримала почесне звання Заслужений лікар України. Саме мама робила все можливе,
щоб її сини після закінчення середньої школи отримали вищу освіту за їх
особистим вибором і виросли гідними людьми. Адже батько у 1963 році залишив
сім'ю.

Любов до історії у Юзефа-Владислава виникла у 5 класі, коли історію
стародавнього світу почала викладати блискуча та неймовірна вчителька \textbf{Євгенія
Володимирівна} \textbf{Клібанова}. Він й досі пам'ятає її слова на уроці з історії
Стародавнього Єгипту: \enquote{Ось з цієї хвилини ви, дорогі діти, будете на все життя
закохані у історію...}. Завдячуючи \textbf{Євгенії Володимирівні}, її методиці викладання
та глибоким знанням хлопець закохався в історію і одразу ж визначивcя з
майбутньою професією. З 5 класу майбутній історик почав багато читати. Саме
бібліотеки – міські і домашня – формували його як гуманітарія і сприяли
постійному самовдосконаленню. Та й не лише читання поглинало весь вільний час
майбутнього історика. Було ще й захоплення спортом. Між шурхотом книжних
сторінок проявилося ще одне уподобання – фехтування. У 1960–1970-х роках він
був неодноразовим учасником республіканських та всесоюзних змагань із
фехтування на шпагах.

\ii{02_11_2021.stz.news.ua.mrpl_city.1.juzef_vladislav_nikolchenko.pic.4}

Перша спроба вступити на історичний факультет у 1964 р. не вдалася, оскільки
тоді історія сприймалася як ідеологічна дисципліна, була певна квота, яка не
перевищувала 20\% для випускників шкіл. Юзеф-Владислав тоді склав історію СРСР
на 5, російську мову та літературу теж на 5, але твір з літератури – на 4. І,
як це не дивно, цих високих балів йому не вистачило для вступу до університету.
Але юнак завжди був дуже наполегливим та цілеспрямованим. У 1965 році він
вступає на I курс Харківського державного університету ім. М. Горького (тепер –
Харківський національний університет імені В. Н. Каразіна). Всі три іспити юнак
склав на \enquote{відмінно}. На курсі було 50 студентів, серед яких переважна більшість
були вже дорослі люди – 1933, 1935, 1938, 1939, 1940, 1942, 1943 та 1944 років
народження, які досить добре пам'ятали трагічні події Другої світової війни та
страшні часи німецько-фашистської окупації. Курс, у переважній більшості,
складався з наполегливих та успішних студентів.

А вчитися було нелегко: один нескладений у сесію залік – недопуск до іспиту,
два нескладених іспити без поважних причин – негайне відрахування. Із 50
студентів на першому курсі семеро не змогли дійти до захисту дипломів через
академічну неуспішність. Була досить жорстка система складання іспитів та
заліків. У ті часи ніхто з кураторів ніколи не телефонував батькам студентів.
Кожен відповідав за себе. Серед колег-студентів Юзефа Мойсейовича, випускників
курсу, були нині видатні вчені-історики України: доктор історичних наук,
професор \textbf{Пиріг Руслан Якович} та український історик і громадський діяч, доктор
історичних наук, завідувач кафедри новітньої історії України Запорізького
національного університету \textbf{Турченко Федір Григорович}.

\ii{02_11_2021.stz.news.ua.mrpl_city.1.juzef_vladislav_nikolchenko.pic.5}

На думку Юзефа Мойсейовича, його як майбутнього професійного археолога,
історика і культуролога сформував історичний факультет Харківського державного
університету та його блискуча викладацька і наукова когорта, зокрема професори
\textbf{В. І. Астахов, П. І. Гарчев, К. Е. Гриневич, В. М. Довгопол, А. П.
Ковалевський, С. М. Королівський, О. О. Кучер, Б. К. Мигаль, І. К. Рибалка, С.
І. Сідєльніков, І. Л. Шерман, К. К Шиян, Б. А Шрамко, А. Г. Слюсарський.} На той
час харківська історична школа була найкращою в УРСР.

На четвертому курсі Юзефа обрали головою студентської ради інтернаціонального
гуртожитку, у якому поруч з радянськими студентами жили іноземні студенти
підготовчого факультету для іноземних громадян, переважно з африканських та
азійських країн. На студраду гуртожитку покладалися обов'язки сприяти
порозумінню між нашими студентами та іноземцями, створювати атмосферу
доброзичливості і взаємоповаги. Студрада працювала у тісному контакті з
деканатом підготовчого факультету та земляцтвами іноземних студентів. На той
час заступником декана підготовчого факультету з виховної роботи була блискуча
випускниця філологічного факультету, чарівна \textbf{Тамара Марківна Донцова} – майбутня
дружина нашого героя.

\ii{02_11_2021.stz.news.ua.mrpl_city.1.juzef_vladislav_nikolchenko.pic.6}

Перша зустріч Юзефа і Тамари відбулася у 1968 році. Коли Тамара вперше зайшла
до студради, Юзеф зрозумів, що перед ним його майбутня доля! Це було кохання з
першого погляду. Але попереду його чекав тернистий шлях боротьби за серце
Тамари. Конкуренція була велика! Серед претендентів на руку майбутньої дружини
були її колеги, серед яких і син генерала. Проте це не зупинило наполегливого
юнака. \enquote{Per aspera ad astra} – \enquote{Через тернії до зірок} він таки зміг завоювати
серце і руку своєї коханої. 4 квітня 1970 року відбулося грандіозне
викладацько-студентське весілля у харківському ресторані \enquote{Люкс}: гості
сплачували по 5 карбованців за участь у цьому урочистому заході – це і було
подарунком для нареченої і нареченого. Сьогодні, згадуючи ті \enquote{буремні} часи,
подружжя посміхається.. Вони разом вже 52 щасливих роки!

\ii{02_11_2021.stz.news.ua.mrpl_city.1.juzef_vladislav_nikolchenko.pic.7}

Наприкінці 1970 р., коли Юзеф і Тамара вже чекали свою першу дитинку, вони
переїхали до Рівного, де жили разом з мамою та сім'єю брата. У лютому 1971 року
народилася Олександра.

У Рівному, з грудня 1970 року, Юзеф Мойсейович почав працювати у \textbf{Рівненському
обласному краєзнавчому музеї}, де очолив археологічну експедицію та відділ
охорони пам'яток історії та культури. Музейний період його трудової діяльності
був позначений значними досягненнями у дослідженні археологічних пам'яток на
території Рівненської та Волинської областей, у музейній справі, краєзнавстві
та науковій діяльності. З 1973 року по червень 1981 року він працював
заступником директора музею з наукової роботи.

\ii{02_11_2021.stz.news.ua.mrpl_city.1.juzef_vladislav_nikolchenko.pic.8}

У липні 1976 року у подружжя Нікольченків народилася друга донька – Марія.

Особливу вдячність за роки роботи у Рівненському обласному краєзнавчому музеї
Юзеф Мойсейович висловлює на адресу його директора, ветерана Другої світової
війни \textbf{В. Я. Сидоренка} та свого вчителя і наукового керівника,
видатного українського археолога, історика, музеєзнавця і культуролога –
львівського вченого, доктора історичних наук \textbf{І. К. Свєшнікова}.

Тамара Марківна спочатку працювала разом з чоловіком у Рівненському обласному
краєзнавчому музеї як організатор і учасник спеціальних
фольклорно-етнографічних експедицій в Рівненській області. Пізніше вона була
запрошена на викладацьку роботу до Рівненського педагогічного інституту, а в
червні 1991 р. захистила кандидатську дисертацію з поліського фольклору періоду
Другої світової війни.

\ii{02_11_2021.stz.news.ua.mrpl_city.1.juzef_vladislav_nikolchenko.pic.9}

У червні 1981 року Юзефа Мойсейовича було призначено заступником начальника
Рівненського обласного управління культури. А в жовтні 1991 році, вже після
проголошення незалежності України, він був на сесії обласної ради одноголосно
обраний начальником управління культури Рівненської обласної державної
адміністрації. Працював на цій посаді до червня 1998 року. Це були важкі часи.
Юзеф Мойсейович робив усе можливе і неможливе, щоб зберегти в області мережу і
кадри установ та закладів культури і мистецтва; розвивати професійне мистецтво
і традиційну культуру, бібліотечну і музейну справу, забезпечити збереження і
вивчення культурної спадщини нашого народу. Його сумлінна праця була високо
поцінована державою: 23 березня 1998 року Указом Президента України Л. А. Кучми
йому було присвоєне почесне звання Заслуженого працівника культури України.

Працюючи в музеї і обласному управлінні культури, Юзеф Мойсейович за
сумісництвом працював на посаді доцента у Рівненському державному інституті
культури, де викладав навчальні дисципліни \enquote{Музеєзнавство і краєзнавство},
\enquote{Історію культурно-освіт\hyp{}ньої роботи}, \enquote{Організацію діяльності установ культури}
та інші. У 1998 році йому було присвоєне звання доцента з культурології.

У 1995 р. Тамара Марківна і Юзеф Мойсейович отримали привабливу пропозицію від
ректора Маріупольського гуманітарного інституту \textbf{Костянтина Васильовича
Балабанова} переїхати до\par\noindent Маріуполя і працювати викладачами інституту. Особиста
зустріч, спочатку Тамари Марківни, а пізніше Юзефа Мойсейовича з Костянтином
Васильовичем викликала у них відчуття захопленості від його високого
професіоналізму як організатора і керівника авторитетного в Україні вишу,
самовідданого ентузіаста розвитку вищої освіти в державі. Ніяких сумнівів не
було – треба їхати до Маріуполя! Це рішення приймалося Юзефом і Тамарою
Нікольченками у зеніті їхнього авторитету на Рівненщині, у віці, коли їм обом
вже було за 50! Вони вірили у те, що їхня подальша доля повинна бути пов'язаною
з Маріуполем і Маріупольським гуманітарним інститутом.

Спочатку, у січні 1995 року, до Маріуполя переїхала Тамара Марківна працювати
доцентом кафедри української мови та літератури, а згодом, з 28 серпня 1998
року ї Юзеф Мойсейович розпочав свою викладацьку і наукову діяльність у
Маріупольському гуманітарному інституті на кафедрі історичних дисциплін, де
викладав курси \enquote{Давня історія України} та \enquote{Історія української та зарубіжної
культури}. З січня 2000 року до травня 2002 року він виконував обов'язки
завідувача кафедри історичних дисциплін.

У 1999 році К. В. Балабановим було прийняте рішення щодо створення факультету
заочного навчання. Його деканом призначили Юзефа Мойсейовича. Він очолював цей
факультет з вересня 1999 року до вересня 2009 року. Це був час його плідної
співпраці з деканатами і кафедрами. За 10 років значно зросла кількість
спеціальностей і студентів заочної форми навчання. З багатьма з них і сьогодні
він підтримує добрі відносини.

\ii{02_11_2021.stz.news.ua.mrpl_city.1.juzef_vladislav_nikolchenko.pic.10}

У 2003 році на історичному факультеті було створено кафедру культурології та
інформаційної діяльності, на яку Юзеф Мойсейович перейшов працювати і де працює
донині. Він високопрофесійний і авторитетний викладач і науковець. За
статистикою Google Academy – автор понад 200 наукових праць і навчальних
видань, з високим показником рівня їхнього цитування, учасник багатьох наукових
форумів різних рівнів, постійний керівник науковою діяльністю студентів. З
першого року роботи у навчальному закладі Юзеф Мойсейович користується великою
і заслуженою повагою студентів, у в якому вони вбачають не тільки
доброзичливого викла\hyp{}дача-професіонала, а й старшого друга і порадника.

Сьогодні у Юзефа Мойсейовича щаслива велика родина: дві доньки, два зяті, дві
онучки та правнук. Його старша донька – \textbf{Головко Олександра Владиславівна} –
доцент, кандидат історичних наук, доцент Харківського національного
університету внутрішніх справ. Її чоловік – \textbf{Головко Олександр Миколайович},
доктор юридичних наук, професор, Заслужений юрист України, проректор
Харківського національного університету ім. В. Каразіна. Дочка Анастасія –
економіст, працює у казначействі Харкова. Правнук Богдан народився у 2015 році.

Молодша донька – \textbf{Нікольченко Марія Владиславівна} – кандидат філологічних наук,
доцент Маріупольського державного уні\hyp{}верситету. Її чоловік \textbf{Володимир Михайлович
Деліман}, юрист, випускник Маріупольського гуманітарного інституту. Їхня дочка –
\textbf{Владислава Деліман} навчається в Маріупольському державному університеті на
спеціальності \enquote{Культурологія} і за високі досягнення в навчанні та у наукових
студіях отримує стипендію Президента України.

\ii{02_11_2021.stz.news.ua.mrpl_city.1.juzef_vladislav_nikolchenko.pic.11}

Юзеф Мойсейович завжди наголошує: сім'я – це головне, що є у кожної людини.
Вона завжди була і є поруч, вона завжди підтримує і надихає. Щасливий, що має
вірних друзів в Україні і щирих колег у Маріупольському державному університеті
– на кафедрі культурології, на історичному та інших факультетах і, безумовно,
вдячних студентів. Він до нестями закоханий в улюблену справу, якій віддав
майже 52 роки життя. Звертаючись до студентів, особливо першокурсників, Юзеф
Мойсейович завжди наголошує, що без улюбленої справи життя не цікаве.

\textbf{Улюблена книга:} \enquote{Енеїда} І. П. Котляревського.

\textbf{Улюблені фільми:} \enquote{Тіні забутих предків} Сергія Параджанова (1965 рік), \enquote{В бій
ідуть тільки старі} Леоніда Бикова (1974 року), \enquote{Форрест Ґамп} Роберта Земекіса
(1994 року).

\textbf{Хобі:} Читання. Юзеф Мойсейович і Тамара Марківна мають власну велику
бібліотеку, у якій переважає наукова література археологічного, історичного,
філологічного та культурологічного\par\noindent спрямування.

\textbf{Улюблені місця в Маріуполі:} автобусні та тролейбусні зупинки у центрі біля
магазину \enquote{1000 дрібниць}, біля Маріупольського державного університету та біля
ринку \enquote{Денис} на Східному.

Після обстрілу терористами мікрорайону \enquote{Східний} у січні 2015 році, під який
потрапив і будинок, у якому живуть Юзеф Мойсейович, Тамара Марківна і сім’я
доньки Марії, вони без вагань відкинули пропозиції друзів з Рівного повернутися
назад. Незважаючи на всі випробування останніх років, їхнє місце саме в
Маріуполі, що став для них за чверть століття рідним містом.

\textbf{Побажання маріупольцям:} 

\begin{quote}
\em\enquote{У Маріуполь я закохався. Обожнюю Східний, де живу вже
23 роки. Вважаю, що маріупольці пережили багато яскравих подій і вистояли в
найтрагічніші хвилини. Тепер настав час рухатися вперед та пам'ятати, що
Маріуполь є невід'ємною частиною великої держави. А в умовах пандемії Covid-19
звертаюсь до всіх маріупольців – особливо друзів, колег і студентів: бережіть
себе, своїх рідних і близьких. \textbf{Все буде добре!}}.
\end{quote}
