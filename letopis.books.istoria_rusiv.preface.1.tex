% vim: keymap=russian-jcukenwin
%%beginhead 
 
%%file books.istoria_rusiv.preface.1
%%parent books.istoria_rusiv.preface
 
%%url 
 
%%author_id 
%%date 
 
%%tags 
%%title 
 
%%endhead 

The History of Kiev Russia, before the Invasion of Tatar headed by Batu-Khan,
is joined together with the history of whole Russia, or it is the only history
of Russia; for it is known that the beginning of this History, together with
the beginning of the Russian rule, has its source from the Princes of Kiev and
the Princedoms of Kiev, joined together with the Prince of Novgorod Ryurik.
This unified history has been lasting without interruption till the Invasion of
Tatars, and after that, rarely has been mentioned in the General Russian
History; and after the Liberation from Tatars, by the Lithuanian Prince
Gedimin, is silenced at all.  And because of this, the presented History of Kiev 
Russia is split into two parts, the first part shortly covering
the period of time before the Invasion of Tatars, and the second part covering historical 
events after that Invasion in a much more detailed fashion.

Історія Малої Росії до пори нашестя на неї Татар з ханом їхнім Батиєм злучена з
Історією всієї Росії або ж вона і є єдина Історія Російська; бо ж відомо, що
початок сеї історії, разом з початком правління Російського, береться од Князів
і Князівств Київських, з прилученням до них лише одного Новгородського Князя
Рюрика, і триває до навали Татар безперервно, а від сього часу буття Малої
Росії в Загальній Російській Історії ледве згадується; по визволенню ж її від
Татар Князем Литовським Гедиміном і зовсім вона в Російській Історії замовчана.
Саме тому пропонована тут Історія Малоросійська писана на два періоди, тобто до
нашестя Татарського екстрактом, а від того нашестя — широко і докладно.
