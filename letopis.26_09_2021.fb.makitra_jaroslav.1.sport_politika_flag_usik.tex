% vim: keymap=russian-jcukenwin
%%beginhead 
 
%%file 26_09_2021.fb.makitra_jaroslav.1.sport_politika_flag_usik
%%parent 26_09_2021
 
%%url https://www.facebook.com/yaroslav.makitra/posts/4470921686327736
 
%%author_id makitra_jaroslav
%%date 
 
%%tags __sep_2021.usik.pobeda.dzhoshua,boks,obschestvo,pobeda,politika,sport,ukraina,usik_aleksandr
%%title Спорт - це політика, тому прапор і гімн під якими виступає спортсмен і є визначальними!
 
%%endhead 
 
\subsection{Спорт - це політика, тому прапор і гімн під якими виступає спортсмен і є визначальними!}
\label{sec:26_09_2021.fb.makitra_jaroslav.1.sport_politika_flag_usik}
 
\Purl{https://www.facebook.com/yaroslav.makitra/posts/4470921686327736}
\ifcmt
 author_begin
   author_id makitra_jaroslav
 author_end
\fi

Спорт - це політика, тому прапор і гімн під якими виступає спортсмен і є
визначальними! 

Вибачте, та дискурс навколо перемоги Усика гарно демонструє, хто мислить
сьогоденням, а хто завтрашнім днем. Трішки спортивно-політичної історії. 

Перемоги ФК "Динамо" у ЛЧ в кінці 90-тих. Скільки людей зараз пам'ятає в яку
партію входили власники клубу і як це допомогло цій партії? Де вона, ця партія?
Які почуття тоді викликали перемоги у підлітків, дітей і не тільки, в часи,
коли тільки формувалася Україна? І що кажуть про це зараз?

Перемога ФК «Шахтар» в останньому Кубку УЄФА і продовження виступів донецького
клубу за Україну (!!!) об’єднує багатьох. Безвідносно ставлення до власника і
симпатиків інших клубів. 

ЧС-2006 по футболу. Хто вийшов в 1/4: Україна чи комуніст Блохін? Де він і його
партія? А які емоції зараз і спомини про успіхи збірної? 

Фантастична кар'єра футболіста Андрія Шевченка 20 років тому і тренера збірної
на ЧЄ-2020 зараз. Це успіх партії "Україно, вперед!" і віп-агітатора Януковича?
Де партія і цей горе-політик? А що відчувають ті, хто ріс та росте на перемогах
Шевченка? 

Феєричні перемоги боксера Віталія Кличка і таке собі мерство. Від того тодішні
емоції та гордість за державу стали меншими?

Пропоную усім дивитися трішки вперед, а не мислити сьогоденням через призму
теперішніх власних політичних переконань. Ось зараз діти, підлітки дивляться на
перемоги Олександра Усика і щиро горді, що вони українці. Це також бачать за
кордоном. І це десятки мільйонів, хто побачив силу України. А його громадянську
позицію чують окремі категорії і то по-різному, бо він цікавий, насамперед, як
боксер-чемпіон, а не його уявлення про інші галузі. 

Насамкінець, згадайте зараз, що говорив про Америку Мухаммед Алі і чому він
змінив ім'я та прізвище і прийняв мусульманство? А скільки людей знає, що він
легендарний американський боксер, що прославив США?  

Спортивні перемоги говорять сильніше в інтересах держави, ніж ті слова, що
говорять автори перемог про політику. Та і більшість українців гарно навчилися
відрізняти перемогу Держави і поліитчні позиції її авторів! 

Тому вітання усім нам з вчорашнім днем, який укріпив чи народив патріотів
України.

\ii{26_09_2021.fb.makitra_jaroslav.1.sport_politika_flag_usik.cmt}
