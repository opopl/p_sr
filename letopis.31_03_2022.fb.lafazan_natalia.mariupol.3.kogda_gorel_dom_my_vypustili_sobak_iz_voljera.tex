%%beginhead 
 
%%file 31_03_2022.fb.lafazan_natalia.mariupol.3.kogda_gorel_dom_my_vypustili_sobak_iz_voljera
%%parent 31_03_2022
 
%%url https://www.facebook.com/permalink.php?story_fbid=pfbid0qmUaKvie1Wvf5FJvXmvQr5fe1whkV7AuHaaJYUA7DjJt1NmTincawfLzVcAGMfFSl&id=100030592628843
 
%%author_id lafazan_natalia.mariupol
%%date 31_03_2022
 
%%tags mariupol,mariupol.war
%%title Когда горел дом и все искали укрытие, мы выпустили собак из вольера
 
%%endhead 

\subsection{Когда горел дом и все искали укрытие, мы выпустили собак из вольера}
\label{sec:31_03_2022.fb.lafazan_natalia.mariupol.3.kogda_gorel_dom_my_vypustili_sobak_iz_voljera}

\Purl{https://www.facebook.com/permalink.php?story_fbid=pfbid0qmUaKvie1Wvf5FJvXmvQr5fe1whkV7AuHaaJYUA7DjJt1NmTincawfLzVcAGMfFSl&id=100030592628843}
\ifcmt
 author_begin
   author_id lafazan_natalia.mariupol
 author_end
\fi

В тот момент мне казалось это гуманно. Когда горел дом и все искали укрытие, мы
выпустили собак из вольера. Они его разрывали. Перед обстрелами так было каждый
раз. Мне казалось, что выпустив их, я дам им шанс на жизнь. Последнее время они
не ели. Еду давали, но им она была не интересна. Только вода. Мы дали им
свободу. Две черепашки остались в доме. Мы долгое время пытались их согревать,
подливая воду нагретую на костре. Фикция. Она становилась теплее лишь на
маленький период. Из горящего дома мои обе Доченьки выносили своих котов. Они
не взяли куртки, телефоны и не обувались. Но я счастлива, что они выбрали живые
существа а не гаджеты. В квартире папы сгорели заживо два кота. Коты были
подобраны и вылечены. У них шансов на жизнь не было когда мы их нашли. Но мы их
тогда спасли. А тут, в горящей девятиэтажке их жизнь закончилась. Это не
сравнимо с жизнью людей. Но. Таки допишу. Собаки остались на улице. Людей
хоронить некому. Трупы лежат на улице. Как было гуманнее, оставить собак в
заперти либо отпустить на волю. Я никогда не узнаю, какой выбор гуманнее и
правильнее.

%\ii{31_03_2022.fb.lafazan_natalia.mariupol.3.kogda_gorel_dom_my_vypustili_sobak_iz_voljera.cmt}
