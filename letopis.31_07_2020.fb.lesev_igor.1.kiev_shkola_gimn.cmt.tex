% vim: keymap=russian-jcukenwin
%%beginhead 
 
%%file 31_07_2020.fb.lesev_igor.1.kiev_shkola_gimn.cmt
%%parent 31_07_2020.fb.lesev_igor.1.kiev_shkola_gimn
 
%%url 
 
%%author_id 
%%date 
 
%%tags 
%%title 
 
%%endhead 
\subsubsection{Коментарі}

\begin{itemize} % {
\iusr{Оксана Высоцкая}
Как в воду глядел великий Михаил Афанасьевич. Может, эти картинки всплывали в его наркотических снах? "Суровые годы приходят....."

\iusr{Денис Рафальский}
Вот поэтому памятник Ватутину и хотят снести, чтобы у них не было когнитивного диссонанса.

\begin{itemize} % {
\iusr{Тимур Филоненко}
\textbf{Денис Рафальский} сносить хотят радикальнее, мы перенесем в музей ВОВ.

\iusr{Денис Рафальский}
\textbf{Тимур Филоненко} вы - это кто? Телеканал "Прямой"?

\iusr{Тимур Филоненко}
\textbf{Денис Рафальский} вы не оригинальны. Мы это киевляне.

\iusr{Владимир Егоров}
\textbf{Денис Рафальский}, Да, и "труженики" Прямого канала считают себя прямее всех прямых.
\end{itemize} % }

\iusr{Olexander Berdychevsky}
памятник Ватутину стоит временно, не вижу проблемы

\iusr{Юрий Скалицкий}

"Приём в бандерята" - гениально. 2 года был октябрёнком и 2 пионэром, но гимн
СССР по утрам никто не пел, хотя другой херни хватало. Но вот какой нюанс. В
детстве говорили "держись за красное" и если пробалаболил, предъявляли :"ты
Ленина обманул". Сейчас проходя по дворам на Печерсе ( !) шпана лет по 7-8
кричит другу другу "ответь на пи..раса!" , причём маленькие девочки базарят с
пацанами на одной фене. А вот как должны "следить за базаром" бандерята?
"Забожись на бандеру! " ? Так половина детворы по наслышке дома знает, что он
того, "евроинтегратор" , если по культурному. Дилемма

\iusr{Геннадий Овчаренко}

У Булгакова Коровьев тоже занимался хоровым пением с членами зрелищной
комиссии... Все помнят чем это закончилось??  @igg{fbicon.face.tears.of.joy}  Правда у нас Супрунша дурдомы
позакрывала

\iusr{Igor Tenetko}
И животину надо приобщить к гимнопению...

\ifcmt
  ig https://scontent-frt3-1.xx.fbcdn.net/v/t1.6435-9/116584736_2688570934796182_8627701477904957076_n.jpg?_nc_cat=106&ccb=1-5&_nc_sid=dbeb18&_nc_ohc=H3GQ6sBtIJ8AX8BLJ_U&_nc_ht=scontent-frt3-1.xx&oh=e38720c9e81e68d82b5431577ac7c1fb&oe=61A5892B
  @width 0.4
\fi

\iusr{Олег Резник}

\ifcmt
  ig https://scontent-frx5-1.xx.fbcdn.net/v/t1.6435-9/116718138_747360586051952_6048101397571135814_n.jpg?_nc_cat=111&ccb=1-5&_nc_sid=dbeb18&_nc_ohc=i-6Ux-4PN3YAX-hkKDT&_nc_ht=scontent-frx5-1.xx&oh=9b78c8331571965242fcbedb231a9193&oe=61A42C01
  @width 0.4
\fi

\iusr{Jora Mihaylovskiy}

И правда, госсимволы давно не объединяют, а разьединяют. Гимн , герб и флаг
вызывает оргазм у одних и отрыжку у других

\begin{itemize} % {
\iusr{Тимур Филоненко}
\textbf{Jora Mihaylovskiy} ну так у кого они вызывают отрыжку, могут срыгнуть.

\iusr{Jora Mihaylovskiy}
Так и делаем. В 6 утра и в 12 ночи регулярно

\iusr{Тимур Филоненко}
\textbf{Jora Mihaylovskiy} я имею ввиду из страны. Балалайка рядом.

\iusr{Jora Mihaylovskiy}
Я купил билеты на это шоу, фриков, хочу посмттреть чем кончится этот балаган
\end{itemize} % }

\iusr{Тимур Филоненко}
Я бы не говорил о огромном количестве людей. 13\% судя по рейтингам ОПЗЖ.

\begin{itemize} % {
\iusr{Михаил Якобсон}
А если сравнить рейтинг ОПЗЖ с порошенковским?

\iusr{Тимур Филоненко}
\textbf{Mihail Jakobson} сейчас немного порох опережает

\iusr{Jora Mihaylovskiy}
Рейтинг опзж не может быть индикатором. Условный регионал может на дух не переносить Леву или Медведчука. Голосовать не за кого

\iusr{Тимур Филоненко}
\textbf{Jora Mihaylovskiy} ну это вообще маргинал.

\iusr{Михаил Якобсон}
\textbf{Тимур Філоненко} Вот именно - немного.
Маргиналы - они и есть маргиналы, с обеих сторон.
Нормальных людей всегда больше.

\iusr{Тимур Филоненко}
\textbf{Mihail Jakobson} осталось только определить дефиницию нормального человека.

\iusr{Михаил Якобсон}
\textbf{Тимур Філоненко} Да, это каждый решает для себя.
Я главным признаком нормальности считаю неушибленность скрепами.

\iusr{Natasha Varetskaya}
\textbf{Тимур Филоненко} все проще - спросите десяток окружающих вас персонажей и статистика готова. 2 к 8 . понятно что восьмерых тошнит. и не спорьте.

\iusr{Тимур Филоненко}
\textbf{Antara Bird} буду спорить, поскольку у каждого свое окружение. Мы же тянимся к людям которые близки к нам во взглядах?

\iusr{Natasha Varetskaya}
\textbf{Тимур Филоненко} тогда скажите прямо - "я патриот по-украински" и тогда все от вас отстанут

\iusr{Natasha Varetskaya}
\textbf{Тимур Филоненко} правда, скорее " я провокатор по-украински"...

\iusr{Тимур Филоненко}
\textbf{Antara Bird} простите, а с риторики моей с самого начала не видно было, что я занимаю проукраинскую позицию?

\iusr{Тимур Филоненко}
И я если и провоцирую, то только на диалог. Всеж хотят диалога? А кто сказал что они будут приятными?

\iusr{Natasha Varetskaya}
\textbf{Тимур Филоненко} диалог? диалог - это когда на равных... в любом ином случае - это поза, истерика, крики и лозунги , а еще агрессия и убийства. о каком диалоге речь?... здесь - насилие в чистом виде... счастье лишь в одном - ура-патриоты слабы ментально и ничего поделать и инакомыслящими не могут.. поэтому убивают... но , и это пройдет. причем скоро. гимн - это такое.. на посошок...

\iusr{Natasha Varetskaya}
\textbf{Тимур Филоненко} а вот это: "что я занимаю проукраинскую позицию" - такая размытая поза, которая в любой момент может растаять... 25 градусов и все - ничего нет...

\iusr{Михаил Якобсон}
\textbf{Тимур Філоненко} Вопрос только в том, что считать проукраинской позицией.
Нынче право называться патриотами пытаются монополизировать люди, ненавидящие реальную Украину и мечтающие её уничтожить - и создать на её месте нечто, соответствующее их представлениям.

\end{itemize} % }

\iusr{Петр Шелест}
А петь будут в начале каждого урока?

\begin{itemize} % {
\iusr{Игорь Лесев}
\textbf{Петр Шелест} я бы после каждой двойки заказывал хоровое исполнение

\iusr{Ирина Кузнецова}
Не знаешь что сказать, пой гимн. Учитель не проникся и поставил кол-учитель сепар))
\end{itemize} % }

\iusr{Владимир Скляров}
Так и идём: по життю з гімном!

\begin{itemize} % {
\iusr{Эмма Банникова}
З гIмнОм.
\end{itemize} % }

\iusr{Павло Бубенко}

Разные миры, Игорь, в этом всё дело! К слову, пение гимна в школах сегодня
происходит исключительно в коммунистических странах типа Северной Кореи и
Киеве.



\end{itemize} % }
