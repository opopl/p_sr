% vim: keymap=russian-jcukenwin
%%beginhead 
 
%%file archeo.sagajdak_chernjakov_kiev_vorota_2001
%%parent body
 
%%endhead 

\subsection{Лядські ворота Києва - оберт колеса світової історії}

М.А. Сагайдак, І.Т. Черняков Інститут археології НАН України 

6 грудня 1240 р. - взяття штурмом та руйнація Києва ордами хана Батия -
важлива дата не тільки в історії Києва, України, але й всієї Східної Європи.
Історичні наслідки цієї події виявились більш значними, ніж сумна доля столиці
Київської Русі, зникнення одної з найбільших середньовічних держав у Європі.
Ця подія мала свій вплив на весь подальший хід історичного процесу на широких
теренах майже всього Старого Світу у Азії та Європі. Вирішальні події взяття
Києва ханом Батиєм у 1240 р. проходили біля Лядських воріт Києва, які
знаходились на території сучасного Майдану Незалежності.

Хоча Хрещатик центральна вулиця сучасного Києва, але він виник порівняно з
містом нещодавно - у ХІХ ст. В часи Київської Русі і майже до ХІХ ст. тут по
широкій долині, що густо заросла кущами та лісом, протікав Хрещатицький
струмок. Київські князі навіть полювали тут звіра, а від влаштування загонів
для ловлі дичини долина носила назву "Перевісище". Стародавній Київ як раз у
районі сучасного Майдану Незалежності спускався до Хрещатицької долини, яка в
цьому місці має більш пологі схили.

Рис. 1

Відомо, що в ХІ ст. Київ став величним містом-фортецею. Його територія сягала
близько 80 га, була оточена валами, на яких розташовувались дерев'яні стіни та
башти. Висота земляних валів сягала 12 м. Попереду них були викопані глибокі
рови. Довжина оборонних споруд дорівнювала 3.5 км. Крім того фортифікаційні
споруди були побудовані навколо числених посадів та монастирів Києва, які
охоплювали територію у 350-500 га. За підрахунками академіка П.П. Толочка на
початку ХІІІ ст. у Києві мешкало біля 45-50 тисяч населення [Толочко 1972, с.
174; 1976, с. 202]. Київ був не тільки найбільшим містом Східної Європи, але й
одним з більших на всьому Європейському континенті. Нагадаємо, що такі великі
на той час європейські міста, як Гамбург чи Лондон навіть у ХІV ст. були значно
меншими від Києва (відповідно біля 20 і 35-40 тисяч мешканців). І тільки Париж
за кількістю населення був удвічі більший від Києва.

Рис. 2

Початок ХІІІ ст. не тільки у вітчизняній, але й світовій історії вважається
незвичайним, бо колесо історії повернулось у бік глобальних змін, пов'язаних з
появою нових світових імперій. Почалось утворення від Тихого океану до Дунаю
найбільшої за всю попередню історію людства євразійської імперії монголів
Чингізхана [Черпнин 1977, с. 209]. У 1223 р. монгольські полководці вийшли в
степи Північного Причорномор'я : майже вся Азія була підкорена, а попереду
лежала Європа. Самою першою і великою європейською державою на їх шляху була
Київська Русь, на той час ослаблена міжфеодальними чварами.

 Рис. 3

Перша битва з монголами відбулась у 1223 р. на р. Калці, де руські князі у
союзі з половцями зазнали нищівної поразки. Онуку Чингізхана Батию було
доручено підкорити всю Європу. Але він тільки у 1236 р. розгромив Волзьку
Булгарію і вдерся на Русь : були розорені Рязань, Коломна, Володимир, Суздаль
та інші міста. У 1239 р. він знову здійснив похід на Русь, де завоював
Чернігів, Переяславль, а його війська підійшли до Києва. Монголи споглядали
столицю Київської Русі з лівого берега Дніпра и дивувались її красоті та величі
: "…видив град, удивися красоте его и величеству его…" [Літопис Руський 1989,
с. 394]. Монголи надіслали послів до Києва, але кияни вже знали про їх страшні
наміри та звичаї, а тому рішуче відмовились здати місто. Монголи не наважились
на штурм такого великого міста і повернули назад.

Дійсно, таке велике місто з великою кількістю населення, міцну фортецю, що була
розташована на високих кручах Дніпра, розділених глибокими урвищами, штурмувати
було важко. На підготовку до цього знадобився рік. Восени 1240 р. хан Батий
зібрав під Київ військові сили всіх монгольських ханів. Це був безпрецедентний
випадок у практиці монгольських завоювань. З ханом Батиєм під Києвом були
найвідоміші монгольські хани і воєводи : Урдюй (засновник Білої Орди), Байдар,
Бирюй, Кайдан, Бечак, Менгу, Куюк. Військовими операціями безпосередньо
керували Себідяй-богатир, Бурондай-богатир. У битві на р. Калці монголів було
біля 200 тисяч, а під Київ прийшло ще більше. На кожного киянина було до
десятка ворожих воїнів. Літописець записав про це : "У той же рік прийшов Батий
до Києва з великою силою многим-множеством сили своєї, і окружив город. І
обступила Київ сила татарська, і був город в облозі великій. І пробував Батий
коло города, а вої облягали город. І не було чути нічого од звуків скрипіння
теліг його, ревіння безлічі верблюдів його, і од звуків іржання коней його, і
сповнена була земля Руськая ворогами". [Літопис Руський... 1989, с. 395].

Але навіть з такою кількістю військ штурмувати місто з неприступними схилами
Дніпра зі сходу, чи болотистою долиною річки Либіді з заходу було неможливо.
Спочатку були взяті монголами посади та монастирі навколо міста, а його
центральна частина залишалась неприступною завдяки вигідному унікальному
топографічному розташуванню та міцним оборонним спорудам.

Давньоруські літописи містять різні дані про строки облоги Києва. На наш
погляд, така плутанина свідчить лише про те, що в них зафіксовані різні етапи
облоги та вирішального штурму. Найбільш вірогідне повідомлення Псковського
літопису про значний строк облоги - 10 тижнів і 4 дні, тобто 74 дні, починаючи
з вересня до грудня, вірогідніше, з 23 вересня по 6 грудня. Довготривалість
облоги Києва підтверджує і посланець римського папи до хана Батия Плано
Карпіні, котрий проїздив через зруйноване місто через 6 років у 1246 р. [Ивакин
1982, с. 14].

З літописів відомо, що Київ був узятий штурмом ханом Батиєм 6 грудня. Ця
груднева дата взяття, мабуть, і пояснює факт довгої облоги та стратегії штурму.
Саме в цей період починались великі морози, внаслідок яких кригою вкривався
Дніпро, Либідь, всі струмки та болота. В умовах міцної фортифікації і складної
для бойових дій топографії місцевості чисельна перевага у кілька десятків разів
не мала б вирішального значення. Хан Батий прийняв раціональне стратегічне
рішення зосередити війська в одному місці з вигідною для них місцевістю і,
зруйнувавши оборонні споруди, вдертись у місто.

Його план за логікою подій повинен був включати декілька етапів у взятті
столиці Київської Русі : 1 - повне оточення міста та його ізоляцію від всіх
князівств, щоб виключити приток військової допомоги; 2 - захоплення всіх
околишніх приміських центрів Києва; 3 - захоплення торгово-ремісничих посадів
Києва, у тому числі і Подолу; 4 - зосередження облоги та військових акцій проти
центральної найбільш укріпленої та неприступної частини міста; 5 - штурм одної
з ділянок укріплень центральної частини; 6 - прорив у місто після штурму
окремої ділянки; 7 - повне заволодіння містом та придушення окремих осередків
опору.

Найбільш доступними для штурму виявились Лядські ворота. Замерзання Дніпра
дозволило монголам перевезти на правий берег тяжкі машини для кидання каміння і
по закритій кригою Либіді та Хрещатицькому струмку доставити їх до самих
Лядських воріт. Розташоване напроти воріт Козине болото теж промерзло і
представляло ідеальну поверхню для встановлення машин для кидання каміння -
пороків. Сюди замерзлим Дніпром була доставлена і необхідна кількість каміння
для цих машин, вірогідно, з під Трахтемирова та Канева. За повідомленням
"Хроніки" Матфея Паризького, який переказав розповіді біженців з Русі, монголи
поставили 32 машини, що кидали важке каміння, яке могли підняти 4 людини
[Матфей Парижский ... 1979, с. 142].

Хрещатицька долина була найбільш вигідною для зосередження війська і штурму, бо
чагарники та дерева закривали нападаючих і захищали їх від стріл. Відома
легенда, повязана з назвою Батиєвої гори у Києві, має, вірогідно, під собою
історичне підгрунтя. Вершина цєї гори розташована майже напроти Хрещатицької
долини, що впадає в р. Либідь. З неї добре спостерігати за долиною Либеді, і
Хрещатицькою долиною. Відстань не велика. До берега Либеді - 750 м, а до
Лядських воріт - 3.5 км. З вершини гори зручно вести постійний контроль та
керівництво військовими резервами в долині Либеді та координувати дії штурмових
загонів біля Лядських воріт. Там, мабуть, і знаходилась, бойова ставка хана
Батия.

Як переказують літописці, "і поставив Батий пороки під город коло воріт
Лядських, - бо тут підступили були дебрі, - і пороки без перестану били день і
ніч. Вибили вони стіни, і вийшли городяни на розбиті стіни, і було тут видіти,
як ламалися списи і розколювалися щити, а стріли затьмарили світ переможеним…"
[Літопис Руський 1989, с. 396]. Це були найбільш жорстокі і кровопролитні бої і
для киян, і для монголів, бо коли вони вже стіни і Лядські ворота взяли, то
змушені були зробити відпочинок : "…вийшли татари на стіни і сиділи там того
дня і ночі…"

Крізь пролом, зроблений у Лядських воротах, монголи увірвались до міста. Далі
були взяті Софійські ворота, що з тої пори звуться "Батиєвими". Розпочався
погром Києва. Останнім оплотом киян стала Десятинна церква, яка, як відомо,
під вагою сотень людей обвалилась. За словами літописця "взяше Києв татарове и
святую Софью разграбиши, и монастыри все, и иконы и кресты, и все узорочье
церковное взяша, а люди от мала и до велика вся убыша мечем".

Плано Карпіні у 1246 р. писав, що монголи "провели велике побиття у країні
Русі, зруйнували міста і фортеці, і вбили людей, обложили Київ, який був
столицею Русі, і після тривалої облоги взяли його і вбили мешканців міста;
звідси, коли ми їхали через їх землю, ми знаходили безчисленні голови і кістки
мертвих людей, що лежали у полі, бо це місто було дуже великим, дуже
багатолюдним, а зараз воно зведено нанівець, ледве існує там 200 будинків, а
людей там тримають вони у тяжкому рабстві" [Плано Карпини 1911, с. 45]. Жахливу
картину винищення населення підтверджує і сучасник цих подій архімандрит
Києво-Печерської Лаври Серапіон : "Кров отець і братія наша, акі вода множі
землі напоя…".

Оповідання літописців та сучасників цих жахливих для Києва подій не
перебільшені. Археологічні розкопки, проведені у Києві у ХІХ та ХХ століттях,
підтвердили їх. Були відкриті численні братські могили захисників і мешканців
міста. В районі Дорогожичів знайдена могила з 2 тисячами скелетів. На вул.
Великій Житомирській у 1892 р. І. Хойновський під час земляних робіт виявив
півтораметровий шар людських скелетів на протязі 14 м. Продовження цієї
величезної братської могили знайдено Д. Мілеєвим на розі Володимирської та
Десятинної вулиць. В. Хвойка виявив братську могилу на схід від Десятинної
церкви біля Андріївського узвозу [Каргер 1949, с. 55-102; Толочко 1976, с.
196-203].

Розкопки М. Каргера в радянські часи довели достовірність літописної оповіді
про події у Десятинній церкві. Численні руїни церков, палаців, будинків, зариті
скарби дорогоцінних речей цього часу - німі свідки тієї давньої трагедії, що,
безумовно, вплинула на розвиток світової історії того часу. П. Толочко вважає,
що після "Батиєвого погрому" у Києві залишилось не більше 2 тисяч мешканців
[Толочко 1976, с. 202]. Але подальше успішне завоювання Європи не могло
відбутися, бо монгольське військо було знесилене підкоренням Київської Русі і
довгою облогою її укріпленої столиці Києва.

Відомий англійський вчений В. Рокхілл писав : "Тільки слух про вторгнення їх на
Русь досягнув Західної Європи і сучасники залишили нам лише короткі згадки про
це" [Райт Дж.К. 1988, с. 240; Rockhill 1900, p. ХІІІ]. І хоча Батий після
взяття Києва зробив похід далі на захід, пройшовши західноукраїнські землі,
Малопольщу, Моравію, Угорщину, Хорватію, та він вже не мав сил там закріпитись
і, оминувши Карпати з півдня, у 1242 р. повернувся в степи Північного
Причорноморя та Прикаспію. В пониззі Волги він заснував столицю своєї
новоствореної імперії Золотої Орди - Сарай.

Штурм Лядських воріт - то було підкорення і руйнація Києва, а разом з тим
загибель столиці держави і зникнення з історичної арени власне європейської
держави Київської Русі. Закономірний історичний процес феодального роздроблення
Київської Русі на відміну від інших європейських держав був насильницьки
перерваний і вже природно-історичним шляхом не міг відбутися акт централізації
держави, який досить скоро пережили європейці. З Лядських воріт починається
життя могутньої держави монголів Золотої Орди на терені Європи. На її уламках
потім виникло Кримське ханство і татарські угруповання по всьому Північному
Причорноморю. Згодом вони стали васалами Туреччини і створювали атмосферу жаху
у всій Східній і Центральній Європі аж до кінця ХVІІІ ст.

Лядським воротам, де відбулась кривава прелюдія подій історії світового
значення, повезло у тому відношенні, що під час підготовки до святкування
1500-річчя Києва у 1981 р. була зроблена реконструкція Майдану Незалежності з
побудовою великого фонтану. Під час земельних робіт археологами М.А. Сагайдаком
та В.О. Харламовим, В.А. Ленченко були відкриті і досліджені залишки оборонного
валу, Лядських воріт і побудованих на їх місці пізніше Печерських воріт
[Сагайдак, Харламов, Ленченко 1982, с. 18 - 19 1982; Сагайдак 1982, с. 32-48;
!984, с. 110-114].

Треба зазначити, що археологічне відкриття точного місця розташування Лядських
воріт поставило остаточну крапку у досить давній дискусії істориків та
археологів, що займалися топографією стародавнього Києва. Так, ще в плані
Києва, опублікованому у 1957 р., Лядські ворота розташовувались значно вище від
сучасного Майдану Незалежності. [Нариси 1957, с. 467]. Така ж іх позиція
показана і в плані Києва, опублікованому у 1960 р. [Історія ... т. І]. Автори
цих реконструкцій виходили з того, що близькість "Козиного болота" була
перешкодою стародавнім киянам для побудови брами на цьому місці.

Але ще у 1950 р. О.М. Тихонович та М.М. Ткаченко, спираючись на дані топографії
мап ХVІІ ст., визначили місце розташування Лядських ворот біля сучасного
поштамту поблизу Хрещатика [Тихонович О.М., Ткаченко 1950, с. 43]. Такого ж
висновку дійшов і визначний дослідник археологічних памяток Києва М.К. Каргер
[Каргер 1959, с. 250] та П.П. Толочко на основі детального дослідження
історичної топографії стародавнього Києва [Толочко 1972, с. 84; 1976, с. 60].

Дискусійним було не тільки питання місця розташуання, але й походження назви
Лядських воріт. Ще у ХІХ ст. дослідники історії Києва повязували походження
цієї назви з "ляхами"-поляками, які ніби то жили біля них в часи Київської Русі
[Закревський Н. 1868, с. 439; Соловьев С.М. 1872, с. 39]. Але цікаве пояснення
висунув П.О. Юрченко ще в позаминулому столітті [Юрченко П.А. 1878, с. 60]. На
його думку, назва повязана з такими словами, як "ляда", "лядина", що означали
розчищені від кущів та дерев площі. Деревяна кришка льоху до сьогодення в
українській мові має назву "ляда". Отже назва Лядських воріт повязана з
розчищенням лісовини у Хрещатицькій долині та побудовою до цієї низини міцної
деревяної брами, що мала надійно захищати з цього боку вхід до міста. Це
пояснення назви Лядських воріт прийнято зараз всіма сучасними дослідниками
історії Києва [Толочко 1972, с. 93; Сагайдак 1982, с. 40].

Найдавніші літописні повідомлення про Лядські ворота відносяться до 1151 р. і
повязані з описом усібної боротьби за Київ між військами київського князя
Ізяслава Мстиславича та дружиною Юрія Володимировича (Долгорукого), де детально
розповідається про розташування навколо окремих військових підрозділів князів.
[Літопис Руський 1989, с. 243-245]. Вираз літописця "… а другі навпроти
Лядських воріт на пісках билися", безумовно свідчить про їх розташування внизу
у Хрещатицькій долині.

Ефективність оборонних споруд стародавнього Києва та доцільність їх
розташування за топографічними особливостями було доведено у ХVІІ ст., коли
після приєднання міста до Російської держави почали будувати фортецю. Російські
військові інженери використали оборонні вали часів Київської Русі, які були
відновлені та досипані. Саме на місці Лядських воріт були збудовані так звані
"Печерські ворота" фортеці. Вони, як і Лядські, були деревяними, що добре видно
на одному з найстаріших планів Києва 1695 р. [Алферова Г.В., Харламов 1979, с.
61-75; Алферова Г.В., Харламов 1982]. У 50-х роках ХVІІІ ст. вони були замінені
новими, побудованими з цегли.

Під час охоронних дослідницьких робіт при реконструкції площі Жовтневої
Революції (Майдан Незалежності), археологічна експедиція Інституту археології
НАН України виявила залишки деревяної нижньої частини Печерських воріт ХVІІ ст.
- досить міцну деревяну споруду довжиною біля 30 м, розвернену фасадом до
Хрещатика. Але особливу увагу дослідники приділили залишкам земляного валу,
який був помітний з обох боків у розрізі Печерських воріт. Було виявлено, що
вал завширшки 35 м, зроблений на основі прямокутних деревяних зрубів розмірами
33.1 м. Археологи простежили залишки 9 таких клітей-зрубів, зроблених з
деревяних кантованих брусків, розташованих поперек насипу. Ці елементи нижньої
частини валу біля Печерських воріт ХVІІ-ХVІІІ ст. мали повну подібність до
конструкції валу ХІ ст. біля Золотих воріт, тобто його нижня частина була
споруджена в часи Київської Русі. Дослідники встановили незаперечний факт, що
досліджені ними залишки оборонних споруд у нижній частині представляють
відрізки єдиної системи оборонної лінії стародавнього Києва, спорудженої під
час княжіння Ярослава Мудрого у 30-ті рр. ХІ ст. [Сагайдак 1982, с. 46].

Оборонний вал, що починався біля Золотих воріт, продовжувався на схід уздовж
сучасної вул. Прорізної та трохи вище неї до перетину з вул. Новопушкінською,
роблячи поворот до сучасного Майдану Незалежності, де і повинні були
знаходитись Лядські ворота. Дослідникам вдалося вияснити, що фундаменти
Печерських воріт були закладені вище основи стародавнього валу на 3 м. Під ними
був виявлений стародавній засипаний проїзд завширшки 4.2 м, стінки якого були
обличковані міцними деревяними брусами, покладеними один на другий. Це і все,
що залишилось від Лядських воріт ХІ-ХІІІ ст. Але ці неефектні на перший погляд
залишки представляють значну історичну цінність.

Дослідження археологами залишків Лядських воріт супроводжувалось іншими
відкриттями цього найважливішого для Києва вузла оборонних споруд. З боку
Хрещатика вдалося виявити оборонний рів глибиною до 5 м, схили якого мали кут у
60 градусів. З боку міста на площі, прилеглій до Лядських воріт, були
досліджені залишки декількох давньоруських жител, а також піч виробничого
призначення, що розташовувались на березі давнього струмка, який протікав на
місці сучасного провулка Т. Шевченка. Цікавим було і те, що вдалося встановити
ту частину цього струмка перед міським валом, яка була взята у своєрідний короб
з деревяних дошок. Він проходив під основою оборонного валу, упереджуючи
загрозу розмивання. Це одна з найдавніших гідротехнічних споруд Києва, що
забезпечувала осушування та зберігання оборонних споруд від паводків.

Крім відкриття точного місця розташування Лядських воріт, констатування їх
будівництва одночасно з "Градом Ярослава" ці археологічні дослідження
відзначені і першою спробою в Києві і в Україні консервації виявлених деревяних
споруд на місці знахідки та постійної музеєфікації їх залишків.

На відміну від сумної долі руйнування деревяних будівель, виявлених
археологами під час будівництва метро на Контрактовій площі [Гупало 1981, с.
136-158; 1982], цей археологічний обєкт значної історичної ваги був
збережений. У підземному переході був зроблений музей, але з часом його
закрили.

У 2001 р. почалися будівельні роботи по реконструкції Майдану Незалежності та
спорудженню памятника на честь 10-річчя незалежності України. В Києві на
Майдані Незалежності, де були колись Лядські ворота, потрібно було б поставити
хоч невеличку символічну "Браму скорботи".

У двохтисячолітній історії християнства, ювілей якого зараз відзначає все
людство, руйнування великої кількості храмів у Києві у 1240 р. займає особливе
місце. Вчені по-різному оцінюють ті далекі події, їх вплив на вітчизняну
історію та історію інших держав Європи і Азії [Івакін 1996, с. 31-40]. Настала
пора провести Міжнародну конференцію істориків, культурологів, діячів різних
напрямків релігійних церков, присвячену цій важливій події історії.

ітература

Алферова Г.В., Харламов В.А. Крепостные укрепления Киева во второй поовине XVII века // Вопросы истории.- 1979.- № 7.
Алферова Г.В., Харламов В.А. Киев во второй половине XVII века : Историко-архитектурный очерк.- К. : "Наукова думка", 1982.
Гупало К.Н. Деревянные постройки древнекиевского Подола // Древности Среднего Поднепровья.- К., 1981.
Гупало К.Н. Подол в древнем Киеве.- К. : "Наукова думка", 1982.
Закревский Н. Описание Киева.- Т. І, ІІ.- М., 1868.
Ивакин Г.Ю. Киев в ХІІІ - ХV веках.- К. : "Наукова думка", 1982.
Івакін Г.Ю. Історичний розвиток Києва ХІІІ - середини ХVІ ст.- К., 1996.
Історія Києва.- Т. І.- К., 1958.

Каргер М.К. Киев и монгольское завоевание Киева.- СА.- 1949.- Т. 9.
Каргер М.К. Древний Киев.- М.-Л. : Изд-во АН СССР, 1959, 1961.- Т. 1-2.
Літопис Руський. За Іпатським списком переклав Л. Махновець.- К. : "Дніпро".- 1989.

Матфей Парижский. Великая хроника / Матузова В.Н. Английские средневековые источники.- М., 1972.
Нариси стародавньої історії Української РСР.- К., 1957.
Плано Карпини де Иоанн. История Монголов. Рубрук де Вильгельм. Путешествие в восточные страны.- Спб., 1911.

Райт Дж. К. Географические представления в эпоху крестовых походов.- М. : "Наука", 1988.
Сагайдак М.А. Великий город Ярослава.- К. : "Наукова думка", 1982.
Соловьев С.М. История России.- Т. ІІІ.- 1872.
Тихонович О.М., Ткаченко М.М. Древній Київ-град. Створення відтворення плану Верхнього Києва ХІ-ХІІ ст. // Архітектурні памятники.- К., 1950.

Толочко П.П. Історична топографія стародавнього Києва. К. : "Наукова думка", 1972.
Толочко П.П. Древний Киев.- К. : "Наукова думка", 1976.
Черепнин Л.В. Монголо-татары на Руси (ХІІІ в.) // Татаро-монголы в Азии и Европе.- М., 1977.

Юрченко П.О. О происхождении названия Лядских ворот в Киеве // Труды ІІІ археологического сьезда.- Т. ІІ.- К., 1878.

Rockhill W.W. The Journey of William of Rubruck to the Eastern Parts of the World, 1253-55. As Narrated by Himself, with Two Acciunts if the Earlier Journey of John of Pian de Carpine (Hakluyt Society Publications, sar. 2, vol. IV).- London, 1900.
