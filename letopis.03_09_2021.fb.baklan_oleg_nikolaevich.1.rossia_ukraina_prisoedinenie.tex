% vim: keymap=russian-jcukenwin
%%beginhead 
 
%%file 03_09_2021.fb.baklan_oleg_nikolaevich.1.rossia_ukraina_prisoedinenie
%%parent 03_09_2021
 
%%url https://www.facebook.com/permalink.php?story_fbid=4639095592813844&id=100001403463502
 
%%author_id baklan_oleg_nikolaevich
%%date 
 
%%tags politika,prisojedinenie,rossia,ukraina
%%title Присоединение России к Украине как части Украины
 
%%endhead 
 
\subsection{Присоединение России к Украине как части Украины}
\label{sec:03_09_2021.fb.baklan_oleg_nikolaevich.1.rossia_ukraina_prisoedinenie}
 
\Purl{https://www.facebook.com/permalink.php?story_fbid=4639095592813844&id=100001403463502}
\ifcmt
 author_begin
   author_id baklan_oleg_nikolaevich
 author_end
\fi

... Неэвклидова геометрия украинской политики и нашего влияния на внешний мир,
в первую очередь на тех, от кого мы зависим как в хорошем (наши
друзья/союзники), так и в плохом (наши враги) смыслах 

Такое длинное и невзрачное название моего программного выступления является
"длинным" и "невзрачным", не зря я назвал его длинным и невзрачным, но на ум
приходят слова одного из братьев Кличко, мировых чемпионов: "Я боксирую
незрелищно, зато результативно"

Результативность!

Моё программное выступление излагает качественно новые идеи и цели нашей, как
страны и народа, результативности

Осуществление реформ (нормализация, модернизация, движение в будущее), защита
независимости, суверенности, государственности, победа в противостоянии с
Россией - все это начинается здесь, в моем программном выступлении, в
качественно новых идеях и целях для нашей, как страны и народа,
результативности

Готовы?
Начинаю! 

... Свое программное выступление я хочу начинать с аналогии, которая
присутствует в названии моего программного выступления

Кратчайшее расстояние между двумя точками всегда является прямой линией в
евклидовой геометрии.

Это геометрия, которую обычно изучают в школе, где фигуры двухмерны и
представлены на плоской поверхности, как лист тетради.

В реальной жизни кратчайшее расстояние - это кривая, называемая геодезической.
Это потому, что наша планета не плоская! Таким образом, используется
неевклидова геометрия, а именно: риманова геометрия

Это концепция, которую специалисты по планированию полетов используют для
построения маршрутов полета, чтобы сэкономить время и топливо.

С практической точки зрения в большинстве случаев геодезическая - это кривая
самой короткой длины, соединяющая две точки.

Этот эффект имеет интересные последствия; например, когда вы летите на
самолете, траектория, необходимая для перехода от одного пункта назначения к
другому, не следует по «прямой линии», как многие думают. Он следует за
«кривизной» Земли, делая небольшие корректировки в направлении движения, чтобы
преодолеть как можно более короткое расстояние.

Если самолёт  ✈️  летел бы просто «по прямой», он бы в конечном итоге двигался
по более длинной траектории, чем при следовании кривизне Земли

... Если эта аналогия понятна, перейду к главной сути моего программного выступления, к его ключевому положению

Оно состоит в следующем :

1. Политическая траектория, которая в Украине необходима для перехода из
существующего положения с его проблемами в будущее идеальное положения требует
"кривизны", как в случае неэвклидовой геометрии, о которой я только что
говорил.

2. Движение "по прямой", а не "по кривой" не только будет длиннее и дольше, оно
будет НАСТОЛЬКО длиннее и НАСТОЛЬКО дольше, что будет невозможным, то есть
никогда не приведёт Украину из существующего положения с его проблемами в
будущее идеальное положения

3. Осуществление реформ (нормализация, модернизация, движение в будущее),
защита независимости, суверенности, государственности, победа в противостоянии
с Россией - все это требует движения по кривой в политическом пространстве
Украины

Тезис, что движение в политическом пространстве Украины "по прямой", а не "по
кривой" не только будет длиннее и дольше, но оно будет НАСТОЛЬКО длиннее и
НАСТОЛЬКО дольше, что будет невозможным, то есть никогда не приведёт Украину из
существующего положения с его проблемами в будущее идеальное положения - очень
серьёзный тезис, и для его понимания я сделаю два разъяснения

Во-первых, движение по прямой в политическом пространстве Украины - это те или
иные действия, решение о которых принято нами при помощи привычного и прекрасно
зарекомендовавшего себя мышления на основе логики Аристотеля с её законом
исключенного третьего, то есть, с её или-или

Во-вторых, движение по кривой в политическом пространстве Украины - это те или
иные действия, решение о которых принято нами при помощи непривычного и
усложненного мышления на основе логики Лотфи Заде с её ОГРАНИЧЕНИЕМ закона
исключенного третьего, то есть, с её или-или

Воспользуюсь ещё одной аналогией, на этот раз не из геометрии, а из физики 

Нельзя создать работающую атомную электростанцию или ускоритель на встречных
пучках, нельзя объяснить перегилий Меркурия или энергетический спектр теплового
излучения, нельзя сделать много аналогичных действий, если делать это через
использование классической физики и её понятий, её формул.

Хотя можно создавать самолёты, автомобили, ракеты делать расчёт полета ракет на
Луну и Марс и обратно, строить дома, делать аналогичные и привычные действия
(через использование классической физики и её понятий, её формул)

Но!

Чтобы создать работающую атомную электростанцию или ускоритель на встречных
пучках, чтобы объяснить перегилий Меркурия или энергетический спектр теплового
излучения, чтобы  сделать много аналогичных действий НАМ ПОТРЕБУЕТСЯ:

\begin{itemize}
  \item 1. Постклассическая физика с её понятиями и формулами
  \item 2. Люди, способные  понимать и применять постклассическую физику с её понятиями и формулами
  \item 3. Смелость, решительность, готовность сотрудничать с такими людьми, решительность финансировать этих людей и их работу, решительность оказывать любое другое содействие этим людях и их работе
\end{itemize}

В противном случае, у нас не будет работающих атомных электростанций или
ускоритель на встречных пучках, не будет объяснений перегилия Меркурия или
энергетического спектр теплового излучения, не будет много другого аналогичного

Наш мир будет не полным, и, главное, будет не просто что-то отсутствовать, а
будет отсутствовать что-то ключевое, сощдающее наш прогресс, наше движение в
будущее как человечества

... Мы не результативны в нашей внутренней и внешней политике, в наших реформах
и создании мощной экономики, мы не результативны во всем, в чем хотим создать
наш прогресс и движение в будущее, потому что:

А) Там особая территория, где нужно применять особое мышление, аналогично тому
как на особой  территории создания атомных электростанций, ускорителей на
встречных пучках, объяснений перегилия Меркурия и энергетического спектра, для
подобного другого требуется особая (постклассическая!) физика и её особые
понятия, её особые формулы, а не обычная (классическая) физика и её обычные
понятия, её обычные формулы

Б) Таким особым мышлением является мышление на основе логики Лотфи Заде с её
ОГРАНИЧЕНИЕМ закона исключенного третьего, с её и-и, и мы это мышление не
применяем

В) На этой особой территории [в нашей внутренней и внешней политике, в наших
реформах и создании мощной экономики,  во всем, в чем хотим создать наш
прогресс и движение в будущее] мы должны отказаться от привычного нам обычного
мышления, на основе логики Аристотеля с её законом исключенного третьего, с её
или-или, а мы от него не отказываемся. 

... Второе программное положение состоит в том, что:

А) Переход к такому новому особому, качественно новому мышлению должен
сопровождаться насыщением наших действий деголленизацией, то есть насыщением
элементами единоличной абсолютной власти и лидерства, разумеется без сползания
в диктатуру и насилие, без выхолащивания нашего разнообразия к унылому и
гнетущему тоталитаризму

Б) Такое насыщенные легче и лучше всего осуществить через восстановление в
Украине созданной сто три года назад государственности в виде Украинского
Гетьманского Государства, основанного Павлом Скоропадским 29 апреля 1918 года в
ответ на просьбу 3000 делегатов Всеукраинского съезда хлеборобов

В) Для обсуждения всего этого друг с другом и с внешним миром, как нейтральным
внешним миром, так и не нейтральным внешним миром, нашими друзьями и врагами
нам потребуется три площадки, ключевая во Франции, как форум "Большой разговор
в Париже", и две дополнительные, в США, как форум "Большой разговор в
Вашингтоне", и в Китае, как форум "Большой разговор в Пекине"

... В этом месте своего программного выступления хочу сделать реплику и сожаление.

В недрах нашего народа, среди  интеллектуалов, бизнесменов, людей политически
активных с хорошей проукраинской позицией, идеалами, целями всё время рождаются
группы, обьединения, формируются политические движения и даже партии, и
стараются, стараются, стараются изменить Украину к лучшему

Но на деле все заканчивается ничем, очередной неудачной попыткой и разочарованием.

Внешне это напоминает, как из недр бурлящего энергией Солнца время от времени
возникает очередной проьуберанец, и отрывается от поверхности Солнца, но потом,
как бы высоко он не поднялся, он падает обратно, отрыв не происходит

Всё эти группы, объединения, движения и партии, все эти попытки - ещё один
"протуберанец", который  "отрывается от поверхности Солнца и потом падает
обратно"

Внутри таких групп, объединений, движений и партий начинаются споры, кроме
того, они   очень слабо связаны с реальной жизнью, даже если декларируют некие
жизненно важные для Украины и украинцев цели

Всё это проявление силы тяжести обычного мышления, которое своей негибкостью,
игнорированием реальных смыслов тянут в небытие все эти группы, объединения,
движения и партии.

Увы, это так

Я,  моя команда, которая начинает формироваться, и моё партнёрское окружение,
которое также начинает формироваться - единственное исключение.

А почему?

Мы не просто "отрывается от поверхности Солнца и не возвращаемся обратно", мы
"взрывам Сонце изнутри", когда "все Солнце становится разлетающимися
протуберанцами" , свободными от возвращения назад, к прежнему состоянию
политического и государственного устройства Украины, к состоянию негатива,
проблем, прозябания

... Третий пункт моего програмного выступления - мы можем стать гигантом, если встанем на плечи гигантов

В частности, получить 2 трлн долларов США инвестиций, и 20 кратно увеличить в
течение 2 лет наш валовый национальный продукт с уровня 200 млрд долларов США в
год до уровня 4 трлн долларов США в год

В новом мышлении нуждаемся не только мы, но также мир, и в первую очередь два
гиганта, США и Китай, на плечи которых мы  можем встать, путем оказания им
консультационных и организационных услуг по преодоления их конфликта,
противостояния

Реальной основы у этого конфликта нет, он следствие применения США мышления на
основе логики Аристотеля с её законом исключенного третьего, то есть, с её
или-или, а Китай не осознает, что он мыслит качественно иначе, на основе логики
Лотфи Заде с её ОГРАНИЧЕНИЕМ закона исключенного третьего, то есть, с её и-и.

Выступив консультантом и организатором диалога, организатором новой
аналитической работы в правительствах обоих стран, Украина прекратит этот
конфликт, США и Китай начнут сотрудничать.

... Четвертый пункт моего программного выступления касается дальнейшей судьбы России

Аналогично тому, как дерево - не ветвь дерева, но ветвь дерева - это дерево,
так Украина - не Россия, но Россия - это Украина, и России пора обратно, в
Украину

Взращенная Украиной и потом отломленная украинская ветвь, а это долгорукая
берущая свое начало от киевских истоков Москва, Московия, Россия, должна быть
привита обратно, к Украине

Иначе говоря, Россия должна быть присоединена к Украине как часть Украины

Это объединение, но качественно другое объединение, не путем присоединения
Украины к России как части России (о чем мечтает Россия, и агрессивно
принуждает Украину к этому, без малейшего шанса на успех), но наоборот, путем
присоединения России к Украине, как части Украины (когда  RU обретает смысл
Region Ukrainian) 

Почему я в своём программном выступлении касаюсь России, да ещё в таком
варианте, как присоединение России к Украине как части Украины?

Россия - более сильный враг, чем мы и мир себе думаем.

Смотреть на Россию надо с позиций продолжения на новом уровне и с новой целью
стенфордского (Стенфордский университет, США) в 1988-1994 годах исследования,
открывшего феномен построенных навечно компаний, некоммерческих организаций и
стран

На этой основе, мы и мир должны осознать, что обуздать Россию можно только
изнутри, а не извне, но не изнутри России, а изнутри Украины, когда Россия
станет частью Украины

Новое мышление поможет этому, в том  числе поможет россиянам и властям России
понять неизбежность такого финала России как отдельной страны и её начала на
новом уровне как части Украины

Кораблю пора вернуться в родную гавань, а отломленной ветви пора быть привитой
обратно к дереву, из которого эта ветвь выросла и отломилась

... Пятый пункт моего программного выступления - все  самые правильные мысли есть ноль, если нет действий.

Мы должны не только правильно мыслить, мы должны действовать.

Действие, действие и ещё раз действие

Разумеется, действие  при участии правильных мыслей

Объявляю о создании политического объединения и движения, а также политической
партии, которая изложенные мысли будет воплощать в жизнь, и которая:

\begin{itemize}
  \item 1. Возьмёт власть путем восстановления Украинского Гетьманского Государства, основанного Павлом Скоропадским 29 апреля 1918 года в ответ на просьбу 3000 делегатов Всеукраинского съезда хлеборобов
  \item 2. Правильно этой властью распорядится
	\item 3. Правильно распоряжаясь этой властью, осуществит реформы
	(нормализации Украины, модернизация Украины, движение Украины в будущее),
	защитит независимость, суверенность, государственность, победит в
	противостоянии с Россией, сделает другие действия, которые необходимо сделать
	для безопасности, процветания, прогресса Украины, в частности, после победы
	над Россией присоединить Россию к Украине как часть Украины, когда RU
	обретает значение Region Ukrainian
\end{itemize}

Спасибо за внимание

Это моё программное выступление

Я Баклан Олег Николаевич
