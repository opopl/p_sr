% vim: keymap=russian-jcukenwin
%%beginhead 
 
%%file 27_06_2022.stz.news.ua.donbas24.2.arheolog_ekspedicii_mdu_pauza.txt
%%parent 27_06_2022.stz.news.ua.donbas24.2.arheolog_ekspedicii_mdu_pauza
 
%%url 
 
%%author_id 
%%date 
 
%%tags 
%%title 
 
%%endhead 

Алевтина Швецова (Маріуполь)
27_06_2022.alevtina_shvecova.donbas24.arheolog_ekspedicii_mdu_pauza
Маріуполь,Україна,Мариуполь,Украина,Mariupol,Ukraine,Маріуполь.МДУ,Mariupol.MSU,Мариуполь.МГУ,В'ячеслав Забавін,
Archeology,Археологія,Археология,Expedition,Експедиція,Экспедиция,date.27_06_2022

Археологічні експедиції Маріупольського державного університету поставлено на
паузу (ФОТО)

Сьогодні, 27 червня 2022 року, традиційно повинна була розпочатися щорічна
археологічна експедиція Маріупольського державного університету під
керівництвом В'ячеслава Забавіна, кандидата історичних наук, старшого
викладача, директора музею історії та археології МДУ.

Про це науковець повідомив на своїй сторінці в Facebook, додавши світлини з
минулорічної експедиції. Він зазначив, що кожного року польовий сезон
розпочинався саме з підняття державного прапора.

«До тих пір, поки ворог топче нашу землю, археологічні дослідження в околицях
Маріуполя поставлено на паузу... Але все в нас ще буде, все буде Україна!» —
підкреслив В'ячеслав Забавін.

Нагадаємо, щороку археологічна експедиція Маріупольського державного
університету продовжувала багаторічну і плідну роботу, спрямовану на виявлення,
вивчення й охорону археологічних пам'яток Приазов'я. Вона є єдиною в Донецькій
області, що має офіційне право проводити розкопки.

Влітку 2021 року учасники археологічної експедиції Маріупольського державного
університету розбили наметовий табір поблизу села Комишувате (село в
Мангушській селищній громаді Маріупольського району Донецької області України).
У цій місцевості почав дослідження ще в 1989 році засновник археологічної
експедиції МДУ — Володимир Кульбака, який був вчителем для сучасних археологів
Приазов'я — В'ячеслава Забавіна та Сергія Небрата. Торік кандидат історичних
наук, доцент кафедри історичних дисциплін та керівник Археологічної експедиції
МДУ В'ячеслав Забавін зіставив карти і з'ясував, що з п'яти курганів, які
значилися в цьому курганному могильнику було досліджено всього три. Цікаво, що
на базі цього могильника минуле нашого краю вивчає вже третє покоління
археологів.

Читайте також: Переселенці з Рубіжного відкрили затишне кафе у Кропивницькому
(ФОТО)

Раніше Донбас24 розповідав про те, як склалася доля восьмирічного автора
«Маріупольського щоденника».

Ще більше новин та найактуальніша інформація про Донецьку та Луганську області
в нашому телеграм-каналі Донбас24.

ФОТО: fb Vyacheslav Zabavin
