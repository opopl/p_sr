% vim: keymap=russian-jcukenwin
%%beginhead 
 
%%file slova.priroda
%%parent slova
 
%%url 
 
%%author 
%%author_id 
%%author_url 
 
%%tags 
%%title 
 
%%endhead 
\chapter{Природа}
\label{sec:slova.priroda}

%%%cit
%%%cit_head
%%%cit_pic

\ifcmt
  tab_begin cols=1
     caption Когда-то здесь жили почти триста человек. Фото: Facebook/zapovidnyk2018
     width 0.4

     pic https://img.strana.ua/img/article/3415/chernobylskij-zapovednik-pokazal-11_main.jpeg
     width 0.35

     pic https://strana.ua/img/forall/u/11/33/%D1%87%D0%B5%D1%80%D0%BD(1).png
     width 0.7

  tab_end
\fi

%%%cit_text
Чернобыльский радиационно-экологический биосферный заповедник показал,
как \emph{природа} и время без следа уничтожают то, что когда-то создал человек.  В
Фейсбуке пресс-служба заповедника опубликовала фото села Новоселки, в котором
некогда, до аварии на Чернобыльской АЭС, жили почти три сотни людей. А сейчас
на их присутствие почти ничего не указывает. \enquote{Теперь там - только
полуразрушенные временем дома и тишина. Деревянные домики окутывает паутина
времени, которая каждый сезон маскируется в разные цвета и формы. Летом она
пышная и зеленая. Издалека даже может показаться, что ничего не изменилось за
несколько десятилетий. Но изменилось. Здесь свое забирает \emph{природа}...} -
говорится в посте заповедника
%%%cit_comment
%%%cit_title
\citTitle{Чернобыльский заповедник показал фото, как природа побеждает цивилизацию}, 
Наталья Полулях, strana.ua, 01.07.2021
%%%endcit
