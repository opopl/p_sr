% vim: keymap=russian-jcukenwin
%%beginhead 
 
%%file 22_09_2018.stz.news.ua.mrpl_city.1.ko_dnju_mashinostroenia_vladimir_karpov
%%parent 22_09_2018
 
%%url https://mrpl.city/blogs/view/ko-dnyu-mashinostroitelya-vladimir-karpov
 
%%author_id burov_sergij.mariupol,news.ua.mrpl_city
%%date 
 
%%tags 
%%title Ко Дню машиностроителя: Владимир Карпов
 
%%endhead 
 
\subsection{Ко Дню машиностроителя: Владимир Карпов}
\label{sec:22_09_2018.stz.news.ua.mrpl_city.1.ko_dnju_mashinostroenia_vladimir_karpov}
 
\Purl{https://mrpl.city/blogs/view/ko-dnyu-mashinostroitelya-vladimir-karpov}
\ifcmt
 author_begin
   author_id burov_sergij.mariupol,news.ua.mrpl_city
 author_end
\fi


Владимир Федорович Карпов был не первым и даже не вторым директором Ждановского
завода тяжелого машиностроения, позднее производственного объединения
\enquote{Ждановтяжмаш}. Но именно его бронзовый бюст возвышается у главных проходных
ворот \enquote{Азовмаша}. Такое проявление уважения к его памяти совсем не случайно, а
вполне закономерно.

В. Ф. Карпов принял под свое руководство предприятие, образованное за три года до
того из машиностроительных цехов, выделенных из металлургического завода имени
Ильича. Да, в этих цехах выпускали очень серьезную и важную для страны
продукцию: железнодорожные  цистерны, топливозаправщики для авиационной и
ракетной техники, бронекорпуса и башни для боевых машин. Но в то же время
совокупность цехов можно было лишь с большой натяжкой отнести к отрасли
тяжелого машиностроения.

Когда же в 1980 году Владимир Федорович сдавал дела своему преемнику Игорю
Дмитриевичу Нагаевскому, производственное объединение \enquote{Ждановтяжмаш} входило в
тройку самых крупных предприятий тяжелого машиностроения Советского Союза,
наряду с такими  индустриальными гигантами как \enquote{Уралмаш} и Ново-Краматорский
машиностроительный завод. 

%\ii{22_09_2018.stz.news.ua.mrpl_city.1.ko_dnju_mashinostroenia_vladimir_karpov.pic.1}

В том же 1980 году кинодокументалисты из Киева создали фильм
\enquote{Командармы индустрии} о нескольких руководителях крупнейших
предприятий, в том числе и о В. Ф. Карпове. Монологи этого удивительного
человека и руководителя записаны в фильме. Вспоминая о становлении тяжелого
машиностроения в нашем городе, он сказал тогда: \emph{\enquote{Если отбросить
все то, что говорят, какой я хороший, то о главном можно сказать так:
практически завода не было, а была группа разрозненных цехов. Сейчас это
крупнейший завод, тридцать девять тысяч народа, семь с половиной тысяч
инженерно-технических работников, выпуск продукции перевалил за триста
миллионов рублей. Значит, если я при этом участвовал вместе с коллективом
завода, это действительно стало делом моей жизни}.}

\ii{22_09_2018.stz.news.ua.mrpl_city.1.ko_dnju_mashinostroenia_vladimir_karpov.pic.1}

Именно при В. Ф. Карпове была начата масштабная реконструкция ЖЗТМ. Были
построены новые производственные корпуса, станочный парк был увеличен за счет
нового современного оборудования. Все это позволило существенно нарастить
объемы ранее выпускавшихся изделий, расширить номенклатуру, организовать новые
не только для нашего города, но и для всей страны отрасли машиностроения. На
предприятии начали массово изготавливать специальные  подъемно-транспортные
машины с грузоподъемностью несколько сотен тонн, уникальное металлургическое и
горнорудное оборудование, реакторы для нефтехимической промышленности...

\textbf{Читайте также:} \href{https://mrpl.city/news/view/vydayushhemusya-mariupoltsu-vladimiru-bojko-ispolnilos-by-80-let}{%
Выдающемуся мариупольцу Владимиру Бойко исполнилось бы 80 лет, Ігор Романов, mrpl.city, 20.09.2018}

Владимир Федорович Карпов родился 4 ноября 1911 года в большом селе Молочанске
Запорожской области, которое в 1938 году получило статус города. Его отец –
Федор Васильевич – был горным инженером, а мать – Раиса Ивановна —
учительницей. Федор Васильевич Карпов до Октябрьской революции работал на
машиностроительном заводе в городе Бежица, позже слившемся с Брянском. В
советское время он занимал руководящие инженерные должности на заводах Большого
Токмака, Харькова, Коломны, Краматорска, той же Бежицы. Умер в 1939 году после
тяжелой продолжительной болезни на шестьдесят втором году жизни в Краматорске.
Владимир Федорович рано потерял мать. Она ушла из жизни, когда ему было семь
лет, а его младшему брату Николаю – пять.

Детство и юность, да и молодые годы Владимира Федоровича совпали с тяжелейшими
периодами истории страны. Октябрьская революция со сломом устоявшегося
общественного строя, гражданская война с беспрерывной сменой властей,
эпидемиями сыпного тифа и испанки, голод 1921 года, послевоенная разруха,
голодомор в Украине 1932-33 годов, наконец, Великая Отечественная война.

Он пошел в школу в Харькове в голодном 1921 году. После окончания семилетки
поступил в Харьковский автотранспортный техникум. Завершив обучение в техникуме
и получив диплом в 1931 году, сразу же устроился в отдел технического контроля
Харьковского тракторного завода. Это была отнюдь не \enquote{конторская}
должность. Его рабочее место было в цехе, где нужно было дотошно проверять
точность изготовления деталей для тракторов.

С 1932 года молодой техник становится студентом вечернего отделения
Харьковского механико-машиностроительного института, продолжая работать на ХТЗ.
Сейчас трудно объяснить, почему он в 1933 году переехал из Харькова – тогда
столицы Украины – в провинциальный Краматорск. То ли в надежде, что там легче
будет перенести голод, то ли в связи с переездом отца к новому месту службы,
может быть были и какие-то другие причины.

В Краматорске на НКМЗ Владимир Федорович довольно быстро освоил сложную и
ответственную профессию фрезеровщика-зуборезчика, став одним из лучших на
заводе специалистов этого далеко не простого дела. Но совмещать работу в цехе с
учебой становилось все труднее, поэтому в июне 1936 года он отправляется в
Харьков и на последнем – пятом —  курсе вуза обучается на стационаре. В июле
следующего года блестяще защищает дипломный проект и получает диплом
инженера-механика с отличием.

Новоиспеченному инженеру довелось всего полгода поработать мастером участка в
одном из цехов НКМЗ. В декабре 1937 года его призывают в армию, где пришлось
служить курсантом в 4-м танковом корпусе, части которого дислоцировались в
Киеве и Бердичеве. В январе 1939 года его увольняют в звании младшего
лейтенанта в бессрочный отпуск. Это означало, что по приказу командования он
должен был немедленно явиться в часть для продолжения службы.

\textbf{Читайте также:} \href{https://mrpl.city/blogs/view/donbass-i-mirnye-my-ne-molchim}{%
\enquote{Донбасс и Мирные}: мы не молчим, Кирилл Папакица, mrpl.city, 20.09.2018}

После армии Владимир Федорович возвращается на Ново-Краматор\hyp{}ский
машиностроительный завод. Его назначают старшим инженером-тех\hyp{}но\hyp{}логом, а чуть
позже заведующим бюро технологических работ в цехе №6. К этому времени у него
накопился большой опыт. Он не только теоретически, но и практически освоил
холодную обработку металлов, знал ее тонкости и сложности. Перед ним
открывались большие перспективы на инженерном поприще. Но началась война...

Уже 23 июня 1941 года, на второй день войны, младший лейтенант Карпов был
мобилизован в действующую армию. Его назначают командиром разведывательного
взвода отдельного танкового батальона 227-й стрелковой дивизии. Внезапность
нападения гитлеровской Германии на Советский Союз сыграла свою пагубную роль.
Красная Армия отступала, а с ней и дивизия, в которой служил Карпов. Судьба
этого соединения оказалась трагичной: она потерпела жестокое поражение в районе
Белгорода. Уцелевшие остатки личного состава были рассредоточены по другим
частям и соединениям. Владимир Федорович был назначен заместителем начальника
оперативного отдела штаба 21-й армии. В автобиографии В.Ф.Карпов о военных
годах пишет так: \enquote{Занимал различные командные и штабные должности, участвовал в
боевых действиях Юго-Западного, Сталинградского, Воронежского и Прибалтийского
фронтов}. В середине 1943 года  он написал наставление \enquote{Охрана тыловых частей и
учреждений штаба армии в боевых условиях}. Штабным офицерам, в том числе
В. Ф. Карпову, приходилось, особенно в первый год войны, не раз и не два отражать
с оружием в руках вылазки противника. Этот опыт был и изложен в его труде по
теории военного дела. Ратный труд В. Ф. Карпова был отмечен орденами
Отечественной войны 1-й и 2-й степеней, Красной Звезды, а также медалями.
Великая Отечественная война окончилась, а капитан Карпов еще год оставался в
армии. Ему предложили поступить в военную академию – он отказался.

После демобилизации Владимир Федорович возвратился на НКМЗ и приступил к работе
в качестве заместителя главного технолога завода. Талантливого инженера,
обладающего вдобавок недюжинными организаторскими способностями и неукротимой
волей в достижении намеченных целей, приобретенными на фронте, заметили и
предложили должность начальника производства. Карпов согласился, но на этом
посту работал недолго. В апреле 1949 года его назначают главным инженером
завода. На этом высоком посту он находился пять лет и семь месяцев. Ровно через
год после назначения главным инженером В. Ф. Карпов был удостоен ордена Трудового
Красного Знамени, как сказано было в наградных документах \enquote{за выполнение
специальных заданий правительства}. В этот период времени он, наряду с
исполнением своих прямых обязанностей, принимает активное творческое участие в
проектировании и изготовлении первого отечественного блюминга – стана для
прокатки стальных заготовок, из которых при последующих переделах
изготавливаются рельсы, швеллеры, балки и тому подобные изделия для
строительства и машиностроения. За эту работу он вместе с группой товарищей в
1951 году был удостоен Государственной премии.

В ноябре 1955 года В. Ф. Карпова командируют в Германскую Демократическую
Республику, где он работает до марта 1957 года советником министерства тяжелого
машиностроения страны. В те годы специалистов, находившихся в длительных
зарубежных командировках, по их возвращении  на родину, как правило, назначали
на должность, на ступеньку низшую по сравнению с той, что они занимали до того.
Так случилось и с Карповым. Теперь он - только заместитель главного инженера
НКМЗ. Правда, в этом качестве он находился всего три месяца. Следующим этапом в
карьере Владимира Федоровича была работа главным инженером Управления
машиностроения Донецкого Совнархоза, которая продолжалась в течение четырех
лет. Его, привыкшего к живой заводской деятельности, во многом тяготила
рутинная кабинетная деятельность. Поэтому, когда  освободилась должность
директора Ждановского завода тяжелого машиностроения, он попросился занять эту
вакансию. О результатах работы Владимира Федоровича Карпова на этом посту с
1961 по 1980 год рассказано выше.

\textbf{Читайте также:} \href{https://mrpl.city/news/view/lyudi-kak-knigi-mariupoltsy-smogut-prochitat-armyanina-grechanku-i-azerbajdzhanku}{Люди как книги: мариупольцы смогут \enquote{прочитать} армянина, гречанку и азербайджанца, Ігор Романов, mrpl.city, 21.09.2018}

Его плодотворный самоотверженный труд был отмечен Государственными премиями
СССР и Украины, орденом Ленина, тремя орденами Трудового Красного Знамени, ему
было присвоено звание Героя Социалистического Труда. Он был депутатом
Верховного Совета УССР трех созывов.

Сложив с себя в 1980 году полномочия генерального директора, В. Ф. Карпов до
самой своей внезапной кончины в сентябре 1987 года работал ведущим
инженером-консультантом Научно-исследовательского отдела патентов, информации и
новой техники ПО \enquote{Ждановтяжмаш}.

* * *

К сожалению, в наши дни от бывшего индустриального гиганта, который был создан
при В. Ф. Карпове, остались лишь жалкие остатки. Еще в начале 90-х годов
появились первые тревожные признаки его разрушения. На то были свои причины -
как во всей стране, так и на \enquote{Азовмаше}. В 2000 годах предприятие возглавил
инициативный хозяйственник Александр Владимирович Савчук. Положение дел
улучшилось. Было модернизировано литейное производство. Расширился портфель
заказов. Но 11 ноября 2012 г. после внезапной и тяжелой болезни скончался А. В.
Савчук. И теперь на предприятии жизнь едва теплится...
