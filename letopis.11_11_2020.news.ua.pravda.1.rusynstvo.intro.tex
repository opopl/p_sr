% vim: keymap=russian-jcukenwin
%%beginhead 
 
%%file 11_11_2020.news.ua.pravda.1.rusynstvo.intro
%%parent 11_11_2020.news.ua.pravda.1.rusynstvo
 
%%url 
%%author 
%%tags 
%%title 
 
%%endhead 

\subsubsection{Вступ}
\label{sec:11_11_2020.news.ua.pravda.1.rusynstvo.intro}

\ifcmt
pic https://img.pravda.com/images/doc/e/0/e0a0912-rusyny.jpg
\fi

\enquote{Українство на Закарпатті \dshM це вірус, що потрапив сюди наприкінці 19 століття з
Галичини і заразив русинів}, \dshM каже мій випадковий попутник у потязі Львів \dshM
Ужгород.

Усі свої хитромудрі алегорії він чомусь конструює на термінах з вірусології.

Це не дивує \dshM за останній рік пандемія багатьох перетворила на стихійних
вірусологів.

Дивує хіба що категоричність \enquote{адепту русинського радикалізму}, як він себе
презентує.

За всеукраїнським переписом 2001 року на Закарпатті проживало близько 10 тисяч
русинів. Якщо розглядати сучасне \enquote{підкарпатське русинство} не в етнографічному,
а в політичному сенсі, воно нагадує казан с булькотючою юшкою. Зі смаком то
українського борщу, то російських щів, то угорського гуляша, то чеської
часникової полівки.

Хто і навіщо підкладає дрова у багаття під тим казаном?

Чи насправді є ризик, що на українському західному кордоні може виникнути ще
одна так звана \enquote{народна республіка} під парасолькою Москви?

Чому зводити русинське питання до недолугого сепаратизму \dshM це ще одна гілка
хмизу у вогнище?

На ці питання багато відповідей, і усі вони суперечать одна одній.

\enquote{Українська правда} вирушила на Закарпаття, щоб з’ясувати, що таке \enquote{русинська
карта}, яку в прикордонному регіоні України періодично розігрують Чехія,
Угорщина і Росія.

