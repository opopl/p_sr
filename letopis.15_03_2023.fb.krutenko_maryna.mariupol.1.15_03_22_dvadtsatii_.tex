%%beginhead 
 
%%file 15_03_2023.fb.krutenko_maryna.mariupol.1.15_03_22_dvadtsatii_
%%parent 15_03_2023
 
%%url https://www.facebook.com/marinakrytenko/posts/pfbid0pGjRcqmEbes77gVmraKAVSYJMmamN8iRHQxbNWvwUaYYfhi7yeX3R5eohEkRiW4Yl
 
%%author_id krutenko_maryna.mariupol
%%date 15_03_2023
 
%%tags mariupol,mariupol.war,dnevnik
%%title 15.03.22 ДВАДЦАТЫЙ ДЕНЬ ВОЙНЫ - Всю ночь бомбили, квадрат за квадратом
 
%%endhead 

\subsection{15.03.22 ДВАДЦАТЫЙ ДЕНЬ ВОЙНЫ - Всю ночь бомбили, квадрат за квадратом}
\label{sec:15_03_2023.fb.krutenko_maryna.mariupol.1.15_03_22_dvadtsatii_}

\Purl{https://www.facebook.com/marinakrytenko/posts/pfbid0pGjRcqmEbes77gVmraKAVSYJMmamN8iRHQxbNWvwUaYYfhi7yeX3R5eohEkRiW4Yl}
\ifcmt
 author_begin
   author_id krutenko_maryna.mariupol
 author_end
\fi

15.03.22 ДВАДЦАТЫЙ ДЕНЬ ВОЙНЫ!!!

Всю ночь я не спала, а дремала....

Всю ночь бомбили, квадрат за квадратом. Мне казалось, ужасней ночи ещё не
было.... Я спала и слышала как все ближе и ближе ложатся снаряды. И....  в 5:15
снаряд от \enquote{Града} прилетает прям под наше окно, разбиваются стёкла, (через час
я выглянула в разбитое окно, сгорело дерево, а под деревом лежала сгоревшая
ворона).... В окно врывается огненный шар, все что я успела, но голову накинуть
одеяло.... Мы вскочили и побежали в тамбур. Возле лифта уже сближались все жители
этого подъезда, которые жили на первом этаже.... Жители верхних этажей жили в
подвале. Мы были на адреналине.... Попили кофе, благо что у нас была портативная
газовая горелка, с ней мы ходили в походы. 

Ридик 🐕 тоже проснулся. Захотел гулять, я его отпустила самого, он под
обстрелами погулял и быстро вернулся.

После каждого обстрела все эти дни, мы выглядывали на улицу с надеждой, что
машина цела.... Да, СЛАВА БОГУ она была цела (не считая пару дырок) !!!! Из-за
обстрелов загорелась девятиэтажный дом который стоял перед нашим домом. 

В 8 утра, когда бой немного утих, мы быстро загрузили вещи, сели в машину. В
пятиместную машину сели 6 человек и собака и у каждого было по одной сумке, и у
некоторых были ещё рюкзаки....

Мы выезжали с открытыми окнами, чтоб слышать обстрелы.... Как же мы молились и
кричали.... \enquote{БОЖЕ СОХРАНИ НАС....} Я в первый раз за неделю наверное, увидела
город.... На проспекте лежали провода, строительный мусор и деревья, дома были
разбиты.....

Мы приехали к друзьям в порт. Там мы собирались колонной выезжать из города.
Нас немного покормили. 

Позвонил, Костя и попросил Женю и Сергея забрать тёщу, которая жила рядом с
нами (наш дом до 11.03). 

Женя и Сергей выложили из багажника вещи и документы, поехали....

Дом, где мы остались их ждать был на горе. Мы смотрели на город...., город горел,
а над городом летал бомбардировщик.... Мы молились, в надежде, что они вернутся.
Через час они приехали и привезли маму Юли. Она сидела в подвале дома. Им
пришлось переехать через забор, чтоб попасть к ней домой. 

Все, время ехать!!!!

Было очень много машин. Это был первый день массового выезда людей из
блокадного города. Люди решили, пусть лучше нас убьют в дороге, чем в городе.
Машины ехали, без зеркал, лобовых, задних и боковых стёкл. На пробитых
колёсах.... Все машины, что можно было завести, все пытались выехать. Были люди
которые пешком шли в сторону Бердянска. Были те которые ехали в грузовиках, в
кузове для грузов.

На блокпосту на посёлке Моряков ещё были наши военные. Я выезжала и подумала:
\enquote{у меня есть хоть шанс выехать... А у них нет. Неужели они смертники????}

Первые российские войска, мы встретили в Мангуше 20 км от Мариуполя. Там же и
был первый блокпост. 

Очень долго мы стояли на кругу Мариуполь-Бердянск-Мелитополь-Васильевка. 

Это был первый день так званного \enquote{зелёного коридора} (гуманитарного коридора),
документы, содержимое багажника и телефоны особо не проверяли. На следующий
день уже раздевали мужчин на наличие проукраинских наколок. Проверяли телефоны,
есть ли там компроментирующие фото....

Всего до Запорожья мы проехали 18 блокпостов. 

Долго стояли на блокпосте в Токмаке.

ХОЧУ СКАЗАТЬ ОТДЕЛЬНО ЗА ТОКМАК!!!! Когда мы проезжали Токмак, люди выходили со
своих дворов, плакали, крестили и махали нам рукой. Они уже были под
оккупацией. Я знаю, что многие жители этого городка принимали у себя дома
незнакомых людей, тех кто не успел до комендантского часа проехать блокпост.
Они давали одежду, возможность помыться, кормили людей!!!! 

На одном из блокпостов мы увидели две машины, справа и слева. В них видны были
трупы людей и вещи разбросанные вокруг машины.... Я увидела, маленькие детские
кроссовки....

В тот момент я подумала: \enquote{как хорошо что я два раза не выехала из Мариуполя, ни
кто не давал гарантии, что я доеду до Запорожья....} ещё я подумала: \enquote{ведь кто-то
ищет этих людей и ждёт}.

В Токмаке мы увидели машину наших соседей, тогда мы узнали, что они живы...

Наша машина от перегруженности закипала. Нам периодически приходилось
останавливаться или открывать окна и включать на всю печку....

На одном из блокпостов, русские стояли под памятником, на котором было написано
\enquote{русский военный корабль иди..... (в пучину морскую))))}

И смех и слёзы....

Мы выехали из Токмака, нам нужно было проехать Васильевку до 18:00, потом
комендантский час....

Под Васильевкой нас на трассе остановили два военных, мы подумали, что это
украинские военные и остановились...., но это были молодые парни РФ. Они просили
у нас сигарет. Мы сказали, что мы не курим и поехали дальше. Васильевку мы
проезжали уже в 18:30. Подъехали к \enquote{серой зоне} буфер между \enquote{адом} и \enquote{раем}....
Многие машины разворачивались и ехали назад в ближайший населённый пункт,
только чтоб не ночевать в буферной зоне. 

Мы дозвонились друзьям в Запорожье, чтоб узнать что нам делать. В Запорожье
тоже уже начался комендантский час. Друзья сказали, что за минным полем нас
ждут украинские патрульные машины. Сформировалась колонна, около 200 машин
поехали через минное поле, друг за другом. Там были метки и ориентиры, но это
уже было очень темно. Все машины ехали на габаритных фарах. Слево горело поле,
после обстрела. 

На украинском блокпосту, машина задымилась....., но мы доехали до Запорожья. 

В мирное время дорога от Мариуполя до Запорожья, составляла 2,5-3 часа. В этот
день мы ехали 15 часов. 

В Запорожье нас сопроводили к Эпицентру (типо ИКЕЯ) напоили чаем с печеньем,
дали воду....  Переписали наши данные, вдруг нас кто-то из родственников ищет. 

Когда я вышла из машины и увидела наших военных.... мне хотелось плакать от
счастья и целовать нашу украинскую землю....

МЫ ТАКИ ВЫБРАЛИСЬ ИЗ АДА!!!!!

Когда мы приехали в центр ночевки, было уже 01:30 16.03.22

Я НЕ СПАЛА С 5:15 (почти сутки), я толком не мылась две недели, я не ела в 9
утра. ЗНАЕТЕ ЧТО Я ВЫБРАЛА....? Я ВЫБРАЛА СПАТЬ!!!! Пусть я буду вонять ещё пару
часов, ничего что я хочу кушать! Я ПРОСТО ХОЧУ ЗАСНУТЬ И НЕ ПРОСЫПАТЬСЯ НОЧЬЮ
ОТ БОМБЕЖЕК!!!! 

Продолжение следует....

(Потерпите немного моих рассказов... уже конец той страшной жизни)

%\ii{15_03_2023.fb.krutenko_maryna.mariupol.1.15_03_22_dvadtsatii_.cmt}
