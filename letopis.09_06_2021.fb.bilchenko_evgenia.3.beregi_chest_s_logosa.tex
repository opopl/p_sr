% vim: keymap=russian-jcukenwin
%%beginhead 
 
%%file 09_06_2021.fb.bilchenko_evgenia.3.beregi_chest_s_logosa
%%parent 09_06_2021
 
%%url https://www.facebook.com/yevzhik/posts/3966689583366122
 
%%author Бильченко, Евгения
%%author_id bilchenko_evgenia
%%author_url 
 
%%tags 
%%title БЖ. Береги честь с Логоса
 
%%endhead 
 
\subsection{БЖ. Береги честь с Логоса}
\label{sec:09_06_2021.fb.bilchenko_evgenia.3.beregi_chest_s_logosa}
\Purl{https://www.facebook.com/yevzhik/posts/3966689583366122}
\ifcmt
 author_begin
   author_id bilchenko_evgenia
 author_end
\fi

БЖ. Береги честь с Логоса.

Поздравьте меня, родня: не каждый день учёный получает такой подарок. 

Гугль-академия сообщила, что меня процитировали на их Google Academy более 10
раз учёные с мировыми именами за очень краткий срок последнего месяца. И я
вошла в \enquote{премиум}. У меня теперь  обновленный научный аккаунт. 

Вот какой его девиз (у них это называется месседж):

To keep honest in Logos.

Ссылка: \url{https://yevzhik80.academia.edu/}

Без вас ничего бы не было. Вы очень мне помогаете, делая заказы на книги,
редактуру и мастер-классы, донаты я трачу, если не на лечение, то на наше общее
дело. Мне не карьера нужна, а истина. Пусть хоть со всех украинских вузов
изгонят. Спасибо, родные мои люди! Храни вас Бог за помощь! Я бы умерла без
вас.

А вот с обновленной биографией я постаралась: с тех пор, как культурология в
Украине стала приложением к глобальной американистской идеологии, мои имена -
Русский Поэт и Философ Мира. Надёжнее. Честнее. По любви.

Doctor of cultural studies, philosopher, writer, Russian poet, professor,
bloger, human wrights defender, the author of 220 publications, the laureat of
two international premia for research and literature achviments in the affair
of dialogue of cultures and civilizations and human universal solidarity.

\ifcmt
  pic https://scontent-lga3-2.xx.fbcdn.net/v/t1.6435-9/198561049_3966691716699242_68889050491605896_n.jpg?_nc_cat=106&ccb=1-3&_nc_sid=730e14&_nc_ohc=B3Fxjux6mLYAX_qb0xw&_nc_ht=scontent-lga3-2.xx&oh=fccad2e82851255c07afdcb9dd39c8db&oe=60E4BB8C
\fi

Отдельное спасибо националистам, которые меня увольняли-увольняли, да так и не
увалили. Дорогие сыны и дочери Ария Галициевича Креатора-Бандеровича из села
Торба Автобусного района Гастарбайтерской области!  Если бы не вы, я бы так и
тусила на хуторских конфах об описании Геродотом в книге \enquote{Греко-перские
войны} \enquote{московских людоедов} и отвечала бы на ваши вопросы:
\enquote{Почему вы не указали имя и отчество... Платона и Конфуция в своей
монографии?} (это не шутка, это - ипать и плакать, - это наука в нашей
несчастной колонии, занимавшей некогда ведущее место в советском
профессиональном академизме, инженерии, мысли, строительстве...)

\emph{Сергей Никонов}

Поздравляю. А с националистами, особенно с крайними или хуторскими, все
естественно. Они бросают вызов или вызывают отвращение. Как результат человек
ведет поиск иных жизненных путей.

\emph{Евгения Бильченко}

Сергей Никонов Именно Пути - не как пройти - а сквозь продраться, не себя
продвигать, а братцев. Я к смерти готова каждый день. Родные ещё сохраняют мои
дипломы - пустую тень.

\emph{Сергей Никонов}

Диплом никогда не пустая тень, если он получен честно. В нем и знания, и
умения. Вот личность или мировоззренческие взгляды, научный кругозор могут
меняться, как и нахождение или выход, вытеснение человека из системы
образования или научных школ.

\emph{Евгения Бильченко}

Сергей Никонов особенно, если его вручали за высокий профессионализм те, кто
после 18 января завел дело о моей профнепригодности).
