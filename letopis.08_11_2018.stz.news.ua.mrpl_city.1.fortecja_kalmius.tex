% vim: keymap=russian-jcukenwin
%%beginhead 
 
%%file 08_11_2018.stz.news.ua.mrpl_city.1.fortecja_kalmius
%%parent 08_11_2018
 
%%url https://mrpl.city/blogs/view/fortetsya-kalmius
 
%%author_id demidko_olga.mariupol,news.ua.mrpl_city
%%date 
 
%%tags 
%%title Фортеця Кальміус
 
%%endhead 
 
\subsection{Фортеця Кальміус}
\label{sec:08_11_2018.stz.news.ua.mrpl_city.1.fortecja_kalmius}
 
\Purl{https://mrpl.city/blogs/view/fortetsya-kalmius}
\ifcmt
 author_begin
   author_id demidko_olga.mariupol,news.ua.mrpl_city
 author_end
\fi

\ii{08_11_2018.stz.news.ua.mrpl_city.1.fortecja_kalmius.pic.1}

Історія Кальміуської фортеці не втрачає своєї актуальності, адже з нею
пов'язана одна з версій заснування Маріуполя. Тему фортеці Кальміус вивчали \textbf{В.
Веренікін, В. Забавін, В. Коробка, П. Мазур, М. Руденко, Г. Тимошевський.}

Нещодавно Маріуполь відзначав 240 років, адже за версією, затвердженою міським
самоврядуванням, місто дійсно було засновано в 1778 році. Однак згідно з
козацькою версією місто заснували козаки Запорізької Січі як містечко Кальміус,
яке стало центром величезної Кальміуської паланки. На жаль, не так багато
документів збереглося, які можуть підтвердити цю версію заснування міста.

\textbf{Читайте також:} \emph{240 фактов о Мариуполе к юбилею города}%
\footnote{240 фактов о Мариуполе к юбилею города, Олена Онєгіна, mrpl.city, 27.09.2018, \url{https://mrpl.city/news/view/240-faktov-o-mariupole-k-yubileyu-goroda}}

Звернемося до загального визначення Паланки часів Нової Січі (1734–1775 рр.) –
фортеця, а також і місто навколо неї, центр округи чи області Війська
Запорозького і сама округа. Кальміуська паланка була однією з дев'яти паланок,
найбільшою за розміром території та найменшою з паланок за кількістю жителів.
Відомо, що паланка займала землі від верхів'я річки Вовча до берега Азовського
моря від Кривої коси до Бердянської коси. Простір паланки як округи поширювався
між річками Вовчою, Кальміусом і Азовським морем.

Кальміуська паланка (фортеця) охороняла рибні промисли від донських козаків і
загалом Приазов'я від набігів татарів з Кримського ханства або Ногайської орди.
Військо паланки складалося з 600–700 козаків. Кальміуська паланка особливо
часто згадується в документах Нової Січі, або останнього Коша Запорозького
(1734–1775).

Такі міста як Слов'янськ, Красний Лиман, Артемівськ починали свою історію саме
з фортець. Статус же міста вони отримали лише через сто – двісті років після
свого заснування. Саме тому і для Маріуполя важливо відновити факти з історії
містечка Кальміус, яке дозволяє припустити, що історія міста є більш давньою та
недостатньо дослідженою.

Зовнішні ознаки існування фортеці наразі відсутні, тому що залишки валів і
ровів були знищені в 1845 р Завдяки картографічним спостереженням
маріупольських дослідників, стало відомо, що фортеця розташовувалася на
високому правому березі біля гирла р. Кальміус.

До цих пір залишається дискусійним питання щодо дати створення Кальміуської
фортеці. Один з авторів книги \enquote{Мариуполь и его окрестности} (1892 р.)
Г.  Тимошевський зазначав, що створення фортеці, чи козацького селища
відноситься до XVI ст., \enquote{до часу Предcлава Ланцкоронського, одного з
перших запорізьких героїв}. На думку краєзнавця В. Веренікіна, фортеця існувала
вже наприкінці ХVII століття. Краєзнавець М. Руденко відзначав, що в XVII
столітті фортеця Кальміус виконувала важливу роль в період походів запорожців
проти татар і турків, зокрема, в 1615, 1616, 1617, 1 620, 1622, 1624 і 1630
роках. Вона забезпечувала запасний вихід по Дніпру – Самарі – Вовчій –
Кальміусу в Азовське і Чорне моря. Фортеця на Кальміусі була також зручним
місцем для з'єднання запорізьких і донських козаків в період походів проти
спільних ворогів. З огляду на те, що у XVII ст. фортеця вже існувала і мала
важливе значення для козаків, краєзнавці припускають, що вона була заснована
наприкінці XVI ст.  Ймовірно це сталося не пізніше 1593 року. Саме до цього
року відноситься грамота московського царя Федора Івановича, в якій було
згадано \enquote{місто Кальміус} в гирлі річки Кальміус (проте цей документ не
опублікований, збереглися лише посилання на нього).

\textbf{Читайте також:} \emph{До 240-річчя Маріуполя: Легенди міста. Частина I.}%
\footnote{До 240-річчя Маріуполя: Легенди міста. Частина I., Ольга Демідко, mrpl.city, 12.09.2018 \par 
\url{https://mrpl.city/blogs/view/do-240-richchya-mariupolya-legendi-mista-chastina-i}, \par
Internet Archive: \url{https://archive.org/details/12_09_2018.olga_demidko.mrpl_city.240_rokiv_legendy_mrpl_1}}

П. Мазур вважає, що гарнізон фортеці становив 500 осіб. Козаки укомплектовували
на ввіреній їм території проміжні станції на Кош і на Дон, забезпечували
безпеку тракту до фортець Петровської та Св. Дм. Ростовського. Козаки фортеці
забезпечували також переправу через Кальміус в районі сучасного селища Сартана.
Виконувалися також слідчі, судові та інші дії, передбачені писаними і
неписаними законами Запорізької Січі.

У 2010–2017 рр. на місці фортеці Приазовською археологічною експедицією
(керівник В. Горбов) спільно з Маріупольським краєзнавчим музеєм, експедицією
МДУ (керівник В. Забавін), за підтримки козацьких організацій (кер. А.
Безручко), маріупольських краєзнавців та волонтерів були проведені археологічні
розкопки, які виявили різночасові шари. Проведені археологами розкопки істотно
розширюють інформацію про історію краю до XVIII ст. включно. За будівлею
колишнього ДОСААФу було знайдено комплекс, який визначили археологи як
господарське приміщення Азовського козацтва. Воно мало глибокий котлован з
опорними стовпами, було досить добротним, хоча й тимчасовим явищем. Містило
фрагменти корабельної обшивки. У верхніх, більш пізніх шарах заповнення
котловану була знайдена мідна монета – 1 гріш 1747 р. (Росія), що дозволяє
віднести час спорудження комплексу до першої пол. XVIII ст. Унікальними
знахідками були фрагменти дев'яти червоноглиняних трубок. Ймовірно, всі
червоноглиняні трубки є турецьким імпортом. Турецькі трубки в XVIII ст.
виготовлялися з червоних глин, привезених з району озера Ван, в той час як в
Україні їх робили з білих глин. У колекції Маріупольського краєзнавчого музею
зберігається червоноглиняна трубка турецького виробництва, знайдена під час
розкопок М. Макаренком на лівому березі Кальміусу.

\textbf{Читайте також:} \emph{До 240-річчя Маріуполя: Легенди міста. Частина II.}%
\footnote{До 240-річчя Маріуполя: Легенди міста. Частина II., Ольга Демідко, mrpl.city, 19.09.2018 \par
\url{https://mrpl.city/blogs/view/do-240-richchya-mariupolya-legendi-mista-chastina-ii}\par
Internet Archive: \url{https://archive.org/details/19_09_2018.olga_demidko.mrpl_city.240_rokiv_legendy_mista_2}
}

На думку маріупольського археолога В. Забавіна, знахідка трубок (яскравого
індикатора матеріальної культури запорізького козацтва) і монети
(хронологічного індикатора) в поєднанні з відомостями картографії дозволяє
стверджувати, що досліджуваний комплекс відносився до часу існування фортеці
Кальміус.

Водночас була знайдена колекція парадного поливного посуду XVIII ст., яка
представляла велику цінність. Наявність фрагментів такого посуду в комплексі
запорізьких козаків свідчить про її приналежність людині, що займала високе
соціальне становище.

Фортеця Кальміус припинила своє існування, розділивши загальну долю Війська
Запорозького. У 1778 р в гирлі р. Кальміус з ініціативи губернатора Азовської
губернії В. Черткова почалося будівництво міста Павловська, населення якого
було прийшлим. У 1780 році відбулася зміна населення міста Павловська. На його
території оселилися греки-переселенці з Кримського ханства. У зв'язку з цим
місто було перейменовано в Маріуполь.

Дискусії навколо давньої історії Маріуполя не припиняються й сьогодні, але
сподіваюся, що майбутні дослідження і нові письмові та матеріальні джерела
посприяють об'єктивному висвітленню історії фортеці Кальміус.

\textbf{Читайте також:} \emph{До 240-річчя Маріуполя: Легенди міста. Частина III}%
\footnote{До 240-річчя Маріуполя: Легенди міста. Частина III, Ольга Демідко, mrpl.city, 24.09.2018 \par
\url{https://mrpl.city/blogs/view/do-240-richchya-mariupolya-legendi-mista-chastina-iii},\par
Internet Archive: \url{https://archive.org/details/24_09_2018.olga_demidko.mrpl_city.240_rokiv_legendy_mista_3}%
}
