% vim: keymap=russian-jcukenwin
%%beginhead 
 
%%file 23_07_2021.fb.muharskij_antin.1.zrosijscheni
%%parent 23_07_2021
 
%%url https://www.facebook.com/AntinMykharskyi/posts/1664703327251861
 
%%author Мухарский, Антин
%%author_id muharskij_antin
%%author_url 
 
%%tags jazyk,mova,ukraina,ukrainizacia
%%title НЕ "РОСІЙСЬКОМОВНІ", а "ЗРОСІЙЩЕНІ"
 
%%endhead 
 
\subsection{НЕ \enquote{РОСІЙСЬКОМОВНІ}, а \enquote{ЗРОСІЙЩЕНІ}}
\label{sec:23_07_2021.fb.muharskij_antin.1.zrosijscheni}
 
\Purl{https://www.facebook.com/AntinMykharskyi/posts/1664703327251861}
\ifcmt
 author_begin
   author_id muharskij_antin
 author_end
\fi

НЕ "РОСІЙСЬКОМОВНІ", а "ЗРОСІЙЩЕНІ".

Термін "російськомовний українець" - це фейк, брехня, ілюзія, що прив'язує нас
до "русскаго міру", як і "російськомовний патріот України". У такий спосіб він
легітимізує саме існування та звучання російської мови в контексті нашої
національно-визвольної боротьби. Дає підстави Путіну заявляти про "адін народ"
і "громадянську війну", коли росіяни демократичного штибу воюють з росіянами
штибу імперського, творячи модерну історію України, вписану в загальний
контекст російської імперської історії. 

\ifcmt
  pic https://scontent-cdg2-1.xx.fbcdn.net/v/t1.6435-9/222474240_1664703303918530_351575227943042601_n.jpg?_nc_cat=102&ccb=1-3&_nc_sid=730e14&_nc_ohc=cJCGTBYj5cMAX8VJMV2&_nc_ht=scontent-cdg2-1.xx&oh=05412703a01189ca28871fc20e0a8a46&oe=612B3C03
  width 0.4
\fi

В той час як вживання терміну "зросійщений українець" буде термінологічно
правильним, оскільки чітко визначає ментальний стан багатьох українців, котрі
знайшли в собі сили постати проти російської агресії фізично, але ще й досі не
спромоглися здолати Росію в собі. Бо перемога над власною духовною лінню,
звичками, соціальними та поведінковими штампами - найскладніша. 

Тому уникаймо на всіх рівнях терміну "російськомовний патріот", нав'язаного
гібридними сенсами "русскаго міра". 

А на провокаційні запитання Пальчевського чи Наташі Влащенко про "русскоязичних
патріотов в окопах"  відповідати - вони не "російськомовні", вони "ЗРОСІЩЕНІ",
а тому поки ментально слабкі. 

Бо зросійщення передусім передбачає вольову та духовну деградацію людини, її
деморалізацію та нівеляцію національних чеснот, притлумлення критичного
мислення, перетворення людини на раба соціальних кліше та поведінкових штампів
з метою легкого маніпулювання нею різними пропагандистськими інституціями
російського штибу, і передусім, через належність цієї людини до категорії
"російськомовного населення". 

Ось чому питання зміни термінології в країні, де навіть президент належить до
численної категорії ментально хворих  "зросійщених" громадян, є вкрай
необхідним. 

На сьомому році російсько-української війни, знаючи історію нищення української
мови та України Московією упродовж століть, віднині для дорослих відповідальних
громадян України  розмовляти російською в публічному просторі не тільки
легковажно, але й АМОРАЛЬНО!

Вйо до деокупації свого внутрішнього життєвого простору від російської зарази.
Називаєш себе українцем - будь ним!

УКРАЇНІЗАЦІЯ АБО СМЕРТЬ!

Це був уривок із нашої з Elizabeth Bielska спільної книжки "Як перейти на
українську", котра побачить світ у перших числах листопада.  Якщо ви маєте
можливість і бажання підтримати видання, то зробити це можна у два способи:

\begin{itemize}
\item 1. Передзамовити книжку за 300 грн зараз, щоб отримати з автографом серед перших.
\item 2. Підтримати сумою від 1000 грн, тоді ваше ім'я буде вказане на спеціальній
сторінці подяк у книзі, яку отримаєте, щойно вона вийде друком, разом з
автографом, дипломом-подякою та браслетом "Шляхетні люди говорять українською".
\end{itemize}

Вичерпна інформація на сайті UKRIDEABOOK. Посилання в коментарях та на
світлині, за яку дякуємо львівському фотографу Віталій Воробйов.

PS Поширення допису теж є великою допомогою! 🙂

\ii{23_07_2021.fb.muharskij_antin.1.zrosijscheni.cmt}
