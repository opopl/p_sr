% vim: keymap=russian-jcukenwin
%%beginhead 
 
%%file 11_01_2022.fb.zharkih_denis.1.ljubov_i_politika.cmt
%%parent 11_01_2022.fb.zharkih_denis.1.ljubov_i_politika
 
%%url 
 
%%author_id 
%%date 
 
%%tags 
%%title 
 
%%endhead 
\zzSecCmt

\begin{itemize} % {
\iusr{Николай Воронин}

А может и пусть гибнет? Всё, что строится на землях исторической России, так
или иначе получается Антироссия. Да и зачем нам с вами русским два государства,
Денис?  @igg{fbicon.beaming.face.smiling.eyes} 

\iusr{Сергей Вартанян}

Денис, вспомнил тут анекдот навеянный твоим постом: В середине 70-х годов на
сельском клубе появляется объявление:

\enquote{Лекция \enquote{Виды любви}. Показ слайдов.}

На лекцию приходит вся деревня.

На трибуну выходит лектор и начинает:

Первый вид любви - это любовь мужчины и женщины.

Народ: Слайды! Слайды!

Лектор: Слайды будут позже. Второй вид любви - это любовь мужчины и мужчины.

Народ: Слайды! Слайды!

Лектор: Слайды будут позже. Третий вид любви - это любовь женщины и женщины.

Народ: Слайды! Слайды!

Лектор: Четвёртый вид любви - это любовь к Родине. И ВОТ ТЕПЕРЬ БУДУТ СЛАЙДЫ!

\iusr{Сергей Вартанян}
а вообще лень источник прогресса

\iusr{Татьяна Жалнина}

Между прочим, при СССР любовь друг к другу была, люди огромной страны были рады
встрече, ездили в разные уголки СССР, работали, учились и верили в будущее, а
теперь взаимная ненависть и недоверие

\iusr{Олег Крило}
ну бывают любят, матросят, а потом бросят

\iusr{Ирина Чернобай}
Не могут Родину любить ... предавшие ...

\iusr{Кира Берёзкина}
Они любят деньги - ничего личного. А те отвечают им взаимностью.
\end{itemize} % }
