% vim: keymap=russian-jcukenwin
%%beginhead 
 
%%file 22_02_2022.fb.menendes_enrike.1.sobratsja_s_mysljami.cmt
%%parent 22_02_2022.fb.menendes_enrike.1.sobratsja_s_mysljami
 
%%url 
 
%%author_id 
%%date 
 
%%tags 
%%title 
 
%%endhead 
\zzSecCmt

\begin{itemize} % {
\iusr{Павел Климов}
\textbf{Энрике Менендес} 

Байден подписал указ о санкциях против ЛДНР. Означает ли это, что он признал их
субъектом международного права?


\iusr{Sergii Kostezh}

Ахах, прям раздвоение личности. И все это в одном посте. То «Украина виновата»,
то «верил, что Россия хочет мира Донбассу»))))) так сложно усидеть на двух
стульях, чтобы собирать лайки со всех сторон....

\begin{itemize} % {
\iusr{Volodymyr Fomichov}
\textbf{Sergii} я все равно лайкнул и особенно концовку. Мне тоже в какой-то момент даже
хотелось, чтобы мирным путем удалось все решить и нормально вернуться домой. Но
наши хотения к сожалению не коррелируют с реальностью.


\iusr{Volodymyr Fomichov}
ну и не время сейчас ругаться на самом деле. все и так уже понятно.
\end{itemize} % }

\iusr{ILona Logvin}

По пункту 7, все кто хотел уехать, у них было 8 лет, остались те, кто ждёт
@igg{fbicon.flag.rossia}, те кто вчера с салютами был ) и страна, которая поддерживает войну 8
лет, не может принести мир, это очевидно

\iusr{Максим Глухов}
Ничего что ты перечислил не будет

\iusr{Михалыч Птичкин}
Тебя обманули? А кто ты, \enquote{типичный украинец}?

\begin{itemize} % {
\iusr{Михалыч Птичкин}

Ты не та мразь, что радуется обстрелам Жонбасса?? Это не ты, мразь, скакала
\enquote{кто не скачет тот москаль}, москалей на ножи?. Энрике, ты до 14го года от
русских слышал такие слова?????? Может это вы просто освинячились‽?


\iusr{Оксана Петрова}
\textbf{Михалыч Птичкин} 

брат - отставной военный в России.в мае 2013 они приезжали в гости.
Наслушалась. Было очень жалко отца, который в два раза старше. Вот эта вся
риторика уже была. До майдана было ещё полгода. Я своим тогда сказала, что
Родина прикажет - он в нас стрелять будет. Не гипотетический. Свой, родной. В
нашем доме.

\end{itemize} % }

\iusr{Pavlo Lenets}
Вот это поворот!

\iusr{Алексей Горниенко}

2. А почему только газопровод ? Надо запретит любой транзит через Украину в/из
России- это вообще для Украины будет очень полезно.

\iusr{Олекса Стасевич}
Все догори дриґом переставили... Як можна любити було ґвалтівника?

\iusr{Katya Grishina}
От кого?

\iusr{Светлана Костина}

Посыл не верный в предпоследнем абзаце. Вы застряли с ним в 90-х. Это совковый
совок. И возможно ошибка современных украинцев в понимании современной России.
Такие глупости эти ваши любит не любит.

А сейчас мы увидим реальную декоммунизацию на марше.

\iusr{Тетяна Александрова}
Вы ли это? Ой.

\iusr{Гаяне Авакян}
Обманули, как неожиданно)))

\iusr{Wowa Lys}

То есть, если бы Россия ещё 8, 12 или 28 лет ждала выполнения Минска всё было
бы ок ? Лично мне решение ВВП тоже очень не нравится, но и сидеть столько лет и
ждать с моря погоды тоже не выход. Или ждать ещё одно поколение, выросшее на
войне ?

\iusr{Катя Тарабукина}
От куда этот маразм у меня в ленте? Как в друзья попали

\iusr{Stanislav Tsykalovskyi}
Какой ты кровожадный.

\iusr{Ольга Самойлова}

Вот вы непонятный. Ну как можно метаться? Ну просрала Украина все что могла.
Кончились разговоры в пользу бедных. Какая вы Украина? Вы товарищ с пониженной
социальной ответственностью вот и все.

\begin{itemize} % {
\iusr{Вадим Щетинин}
\textbf{Ольга Самойлова} мы рады за орков, но посмотрим на цыплят по осени. Главное, чтоб раша не всё просрала.

\iusr{Кирилл Полевой}
\textbf{Ольга Самойлова} во вторник санкции, ждём долл по 200

\iusr{Владимир Токарев}
\textbf{Кирилл Полевой} В России бакс в 14 был 73 теперь 80. И как изменилась гривна? Прежде чем писать посмотрите.
\end{itemize} % }

\iusr{Stepan Demin}

шизофрения, Украина это территория и безвольная пешка в игре нескольких центров
силы, а люди на этой территории живут разные и всем не угодишь, да и не нужно,
смотря сквозь призму истории так всегда было

\iusr{Вероніка Міронова}
Да уж! Лучше поздно, чем никогда. И похоже, у вас эти выводы искренны.

\iusr{Сергей Яценко}
Энрике, хотеть не вредно.

\iusr{Сергей Васильев}

Вам во Львов. Пройдите, товарищ.

\iusr{лара новикова}
Фу и фи.

\iusr{Ганна Карнаух}
Мендес що за птиця залітна. І віщує та віщує. Крим і Донбас - це Україна. Мендес - путіноїд, ворог.

\iusr{Олекса Стасевич}
Ви жартуєте, як виїхати? Все перекрито, а молодь хапають окупанти.

\iusr{Вадим Щетинин}
\textbf{Олекса Стасевич} так надо было сопли не жевать и определяться то ли раша, то ли Бандерстан.

\iusr{Олекса Стасевич}
Я більше переживаю за те, що шапіто та режисура тепер коштуватимуть життя українцям.

\iusr{Ольга Кравченко}

Ну ну! Иждивенцы! Плохо слушали Путина. Это шанс для Украины стать
дружественной, но вы им пользоваться не хотите. Гнитесь дальше под наших
врагов. Пестуйте свою нацистскую власть. Очень давно Вы уехали и очень оторвались
от действительности. Рады Вы как же. Ни хрена Вы не рады. В бессильной злобе
корчитесь. Насрать вам всем на людей. Подавитесь нашими пенсиями.

\iusr{Максим Дудник}

Минские нужны были только для остановки полноценной войны в тот момент, и для
того, чтобы тянуть время. Имхо, они себя отжили.

\iusr{Вадим Щетинин}
\textbf{Максим Дудник} только Энрике Менендес в связи со своей начитанностью этого не понимал.

\iusr{Денис Левин}

\enquote{5. Полная заморозка всех социальных выплат и пенсий на неподконтрольных
территориях Украины до выяснения обстоятельств.}- Вот лицемер, этот пункт и так
выполняется уже с 2016 года

\iusr{Михаил Волков}
Совсем с ума сошли?

И очень прошу, не называйте всем Донбассом эти огрызки. И не пишите про
симпатии к России на подконтрольной территории.

\iusr{Владимир Скляров}
Эмоции

\iusr{Александр Луганский}

Никаких Минских договорённостей Украина не собиралась выполнять. Ни вчера, ни 6
лет назад, ни через 6 лет. Это должно было неизбежно произойти. 8 лет было
достаточно, чтобы что-то здесь продвинуть. Вы обманули сами себя..

\iusr{Ирина Новикова}

8 лет обстрелов, бесконечного вранья, унижений на КПП, разного рода блокады,
ограничения в вывозе своего же имущества и денег, дискриминаций по прописке в
тех же банках для кредитов, оскорблений «сами хотели» — это то, что мы слышали
и терпели от Неньки. Россия это другая страна, все, что она делает, она делает
ради себя и для себя, ненавидеть ее или любить, дело такое, неблагодарное, ей
от этого никак, а вот к Неньке претензий гораздо больше у меня, чем к России.
По сути то, что произошло, всецело на совести Зеленского и смещать акценты не
получается. Донбасс выпихнули и этим гордятся

\iusr{Максим Дудник}
К неньке больше, чем к России - это ненадолго. Мне так кажется

\iusr{Людмила Зараковская}

Вы серьезно, мы вас обманули? А вы нас предали, взяли и объявили русских
врагами. Избави, Боже, от таких друзей.

\begin{itemize} % {
\iusr{Aleksandr Baulin}
\textbf{Людмила Зараковская} никто не обьявлял, просто мы не хотим жить как вы

\iusr{Людмила Зараковская}
\textbf{Aleksandr Baulin} Продолжайте жить в своем мире, где \enquote{кто не скачет, тот москаль}, и где жители Луганска - лугандоны.

\iusr{Aleksandr Baulin}
\textbf{Людмила Зараковская} та нет, вы все неправильно поняли
Лагандон это ЛУГАНскДОНецк, а не жители луганска

\iusr{Людмила Зараковская}
\textbf{Aleksandr Baulin} 

Ну вам видней, что мы спорим, вы же белые, пушистые, цивилизованные, одна
Россия злая, варварская, агрессивная. Живите с этим, другой России нет. Засим
откланиваюсь.


\iusr{Aleksandr Baulin}
\textbf{Людмила Зараковская} та нет, мы не былые и пушисты а почти такие как вы, но просто хотим стать другими.

\iusr{Людмила Зараковская}
\textbf{Aleksandr Baulin} Почти - не считается.
\end{itemize} % }

\iusr{Юрий Ватник}

Да, это Ваша позиция и Вы имеете на неё право. Может быть Вы и найдкте в ней
логику, соыесть и здравый смысл.

Я - нет. А жаль.....

Как по мне, то, что сделано, дрлжео было бы быть сделано весной 14-го. А
роковой водороздел возник не вчера, а в том же 14-м в Киеве. И в Одессе.

Вы не заметили?

Мы - не Россия и не Украина.

Мы - прежде всего, люди.

И \enquote{не желай другому того, что не хотел бы для себя}.

Происходящее да можно толковать по-разному.

Но вот законы Украины, принятые после 14-го двойному толкованию не поддаются.

И законы эти многое из происходящего в эти годы объясняют. С этим попробуйте
поспорить.

\begin{itemize} % {
\iusr{Aleksandr Baulin}
\textbf{Юрий Ватник} что за законы такие?

\iusr{Юрий Ватник}
\textbf{Aleksandr Baulin} Вы из Украины пишете?
\end{itemize} % }

\iusr{Oksana Dutchak}
Пункт 2 і 3 - неможливо реалізувати одночасно

\iusr{Пётр Скоробогатый}
Слили Донбасс. Коллективно слили все. Если не будет подвижек, сольют остальную Украину. Россия, конечно, заплатит.

\iusr{Сергей Киселев}
 @igg{fbicon.lightning} ️ Признание ЛНР и ДНР будет в границах Донецкой и Луганской областей — глава
 комитета Госдумы Калашников

\iusr{Светлана Топалова}
На зло бабушке уши отморожу.. Эх бедолага, мне жаль вас.

\iusr{Lena Vitalievna Ivanova}
О! Гюльчатай открыла личико!  @igg{fbicon.grin} 

\iusr{Daniel Tchikin}
Шо делается - даже Энрике бЕндеровцем стал!)

\iusr{Микола Якубович}
Энрике, не буду рассуждать над раскладами и размышлениями - это сугубо твоё восприятие.

А вот по пунктам реальных действий полностью согласен с семью первыми, по
восьмому пункту уточнение - Украина стреляет без предупреждения в случае любой
провокации с территории \enquote{ордло} или \enquote{северного соседа}.

\iusr{Вадим Щетинин}
\textbf{Энрике Менендес}, 

п. 7 - дырку от бублика (если сказать очень культкрно). Если кто-то из орков
приедет, то они не должны получить больше, чем получили переселенцы выехавших в
2014.

И как-то Вы болтаетесь как поплавок в проруби. Любить значит прощать. А яаы то
любите рашу, то теперь не любите.  @igg{fbicon.face.grinning.smiling.eyes} 

И прав Членограй, что не пошёл на минский договорняк. Вы как супер эксперт, ни
разу не спрогнозировали, что бы началось на подконтрольной если бы Членограй
запустил механизм по особому статусу ордло.

Ну что посмотрим, что получат орки.

\iusr{Юрий Нефёдов}

за комменты под твоими постами меня банили 3 раза из трех. можешь теперь мою
историю банов вернуть назад? нет? отож молчи лучше. это не тебя предали. это ты
предал

\iusr{Alek Davis}
\enquote{Передайте товарищу Сталину - произошла чудовищная ошибка.}

\iusr{Валентина Гаташ}
Не узнаю вас в гриме!

\iusr{Андрей Куликов}
Россия! Обманула! Вчера!)))... Ну как же так...

\iusr{Юрий Тимчик}

Ближе к концу потерял логическую цепочку повествования, но не об этом...

Что сегодня мешает выполнять Минск 2? Фактический выход из формата РФ? Так по
нему она не сторона конфликта. Что может помешать Раде срочно принять законы,
касающиеся политической части? По факту, введение регулярной армии РФ
обеспечило выполнение первого пункта минских соглашений.

\begin{itemize} % {
\iusr{Кирилл Полевой}
\textbf{Юрий Тимчик} 

ну да, а миллион рос паспортов, главари члены Единой России, российская армия,
им ещё и выборы, чтобы в ВР и правительство попали большим числом....

\iusr{Юрий Тимчик}
\textbf{Кирилл Полевой}, 

сотни паспортов \enquote{стран-партнёров} в Раде, иностранные инструкторы,
присутствующие и ожидаемые после вступления в НАТО полноценные военные базы...

Всё это не смущает?

\iusr{Ольга Суркова}
\textbf{Юрий Тимчик} 

можно еще поинтересоваться наличием польских и венгерский паспортов у граждан
Украины. Я уж молчу про израильские.

\end{itemize} % }

\iusr{Марин}

Комменты показывают, что за 8 лет так никто ничего и не понял. И не поймут
никогда.

\begin{itemize} % {
\iusr{Кирилл Полевой}
\textbf{Марин} написала москвичка

\iusr{Марин}
\textbf{Кирилл Полевой} ну какая я москвичка. Кривой рог - малая родина. И таких в РФ знаете сколько
\end{itemize} % }

\iusr{Илья Королёв}

Особенно улыбнуло про ГТС и транзит газа @igg{fbicon.face.tears.of.joy}  И ещё,
пункт 3. Какое отношение Украина имеет к этому трубопроводу?

\iusr{Михаил Педченко}
А я думал 1-8 (в соответствии с официальной позицией украинской власти) должно было произойти в 2014-м?

\iusr{Кирилл Полевой}
То есть, если идёт обстрел Мариуполя из российских градов, стрелять в ответ не надо? Я вас правильно понял?

\begin{itemize} % {
\iusr{Александр Полянский}
\textbf{Кирилл Полевой} а есть обстрел мариуполя из российских градов и можно себе представить такой обстрел - если это не ответный огонь?
\end{itemize} % }

\iusr{Светлана Топалова}

Мерзкий скользкий тип, противоречащий себе в каждой строчке. Трус! Как уж на
сковородке. А потому что не определился. Те кто определился мужественно ждали
своего часа и дождались вчера! Мы счастливы мы верили все эти 8 тяжких лет в
свою победу и дождались.

\iusr{Андрій Дейнеко}
Яка прикра несподіванка. Злодій повів себе як злодій.

\iusr{Ольга Ткаченко}

А мне пофиг, смотрю на события как сериал, просто интересно что дальше. Все
сломали 8 лет назад, на Площади. Тогда я кричала на весь интернет:
Остановитесь, вы хоть понимаете какие будут последствия!

А сейчас я просто наблюдаю за последствиями: давайте следующую серию, че там
дальше.

Один минус, когда бахкают, очень страшно. Ну значит сериал в стиле триллера и
ужастика

\iusr{Сергей Сергеев}

В целом все верно кроме надежд на Минские соглашения и на то, что украинские
власти что-то могли решить переговорами и дипломатией. Все Ключи решения
находились в руках России, которая это все и начала и явно не хотела закончить
мирным путём.

И вот ещё новость: госдума заявила что признание лднр будет в границах
областей. Т.е. пойдут в наступление на ВСУ. И что не конец, думаю всем понятно.

\iusr{Сергей Яценко}

Энрике, собирай чемоданы. ЛНР и ДНР будут признаны в границах Донецкой и
Луганской областей, сообщил глава комитета Госдумы по СНГ Калашников.

\iusr{Андрей Лимарев}
Пункт н.2 идёт в + сертификации Северного Потока-2

\iusr{Александр Полянский}

противоречивый текст. на данный момент это всего лишь оформление де-юре того,
что существовало де-факто, и как раз отказ от войны, а не ее начало. по
риторике это да отказ украинскому государству в праве на существование,
возможно он выльется в отъем других кусков - но в реальных шагах этого нет

\iusr{Кирилл Полевой}
Народ, но вы хоть уже сейчас спорить не будете, что Путин психопат - террорист?

\iusr{Тимур Чудутов}
А если со стороны Запада реакции не будет?
Что касается КНР, то они могут Тайвань возвращать и что тогда?

\iusr{Ирина Корчагина}
Ну и каша у вас в голове..

\iusr{Кирилл Полевой}

Вся суть в 9 пункте, выборы.. Путин может нарисовать в ОРДЛО 3,5 млн нужных ему
голосов.. Это куча Пушилиных в ВР, в правительстве... Этого никогда не будет

\iusr{Сергей Основин}
Там, где стоят войска - там не мир. Там война.

\iusr{Елена Лобанова}

Обманутый мальчик осьмнадцати лет. Ничего более смешного, думаю, сегодня я уже
не прочту. Энрике, идите в украинскую политику вы полностью созрели. Эта
публикация яркое тому подтверждение. Театральный жест в конце повествования
бесподобен.

\iusr{Владимир Токарев}
А пенсии тут причём. Россия выплачивает своим пенсии которые живут за границей как и все страны.

\iusr{Igor Gordiychuk}
Несподівано. Вітаю.

\iusr{Андрей Лимарев}

Я также скептически настроен касательно перспектив признанных ДНР и ЛНР, однако
ни России, ни жителям неподконтрольных территорий не оставили другого выбора.
Как и с Крымом, реакция России была не упреждающей, но ответной. Теперь поздно
пить Боржоми.

\iusr{Дмитрий Баевский}

\textbf{Пизде\#}, конечно.

Создала войну Россия, поводы надумала Россия, признала половину Донбасса серой
зоной - россия.

А «довыделывались» - наши.

И в Молдове довыделывались молдаване.

И в Грузии - грузины.

Одна россия - невиноватая, они сами довыделывались.

Энрике, у вас разумные мысли во второй половине поста.

Не ищите оправданий россии. Не выдумывайте причины. То, что произошло - именно
то, что россия хотела. Более того, хотела больше. Речь \textbf{\#уйла}
записывали для другого. И дописали последние пять минут. Все предупреждения
Джонсона и др.  были обоснованы. Шумное инфополе и явные поставки оружия пока
что сорвали более масштабные планы. Солдаты в посещениях жд станций в 5 км от
границы с Харьковской областью на учениях не ночуют. Это была подготовка к
другому. Пока проскочили.

Выводы делать надо. Ошибки отмечать надо.

Но россии не нужны поводы. Для неё повод - альтернативная «история», которую
\textbf{\#уйло} вчера озвучил. И все.

А нам - своё делать. В том числе делать свободную часть Донбасса (исторического
украинского региона) - пространством нормальной жизни. Чтобы как на Кипре. Где
граница между его частями - портал времени на лет 10

\iusr{Юрий Тимчик}
\textbf{Дмитрий Баевский} , камень в сторону Реджепа, нашего, Эрдогана?

\iusr{Natalya Grin}

Вы либо великого гуманиста из себя стройте, либо «а я и в политику тоже могу».
Нельзя быть матерью Терезой и Ганнибалом одновременно. Попуститесь.

\iusr{Игорь Кочетков}

Какое разгромное поражение?) Это начало ползучей оккупации, Энрике, очнитесь)
Все посольства западников во Львове, какое поражение?) Выдыхайте))

\iusr{Дмитри Синайко}
Обманули ахаха.

\iusr{Сергей Перевозчиков}

Давайте я расскажу вам, что будет дальше. На фронтах перестали стрелять, потому
что ЛДНР это теперь не нужно. Скажут, что это результат решения путина, но это
не так. Теперь в окопах будет не 10\% российской армии, а все 100\%. Дальше до
нового года будет полная тишина, чтобы мы ещё расслабились. Они будут насыщать
эти территории военной техникой, боеприпасами, проводить доразведку. После
нового года начнётся опять обострение, опять российские провокации. Как сейчас.
Российские войска перейдут в наступление с задачей выйти на бывшие границы
луганской и донецкой области. Ведь Путин признал ЛДНР, а согласно их
конституции у них полная территория этих областей. Потом опять будет затишье, а
потом наступление на харьковскую, Запорожскую, Днепропетровские области. И так
дальше и дальше, пока Украины не останется. А мировое сообщество будет выражать
озабоченность и вводить санкции, которые ничего не значат.


\iusr{Николай Гастелло}
2 и 3 пункты исключают друг друга.

\iusr{Александр Кузнецов}
спасибо, отписка

\iusr{Михаил Бурлаков}

Не с этого нужно начинать, уважаемый Энрике. Вы опять слушаете только себя.
Ведь давно известно, что делать, чтобы все у нас наладить и чтобы не произошло
то, что случилось всера. А сейчас нужно исправлять ситуацию, а не усугублять
ее, как Вы предлагаете.  @igg{fbicon.anger}{repeat=3} 

\iusr{Юлия Коломойцева}
К сожалению, украинские \enquote{элиты} и Вы это разные люди.

\iusr{Ирина Белачеу}
Как можно вообще верить России?! И предлагать аналитику!!!

\end{itemize} % }

