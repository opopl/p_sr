% vim: keymap=russian-jcukenwin
%%beginhead 
 
%%file 12_06_2021.fb.bilchenko_evgenia.2.anekdoty
%%parent 12_06_2021
 
%%url https://www.facebook.com/yevzhik/posts/3972180789483668
 
%%author Бильченко, Евгения
%%author_id bilchenko_evgenia
%%author_url 
 
%%tags anekdot,selo,ukraina
%%title БЖ. Подборка сельских анекдотов
 
%%endhead 
 
\subsection{БЖ. Подборка сельских анекдотов}
\label{sec:12_06_2021.fb.bilchenko_evgenia.2.anekdoty}
\Purl{https://www.facebook.com/yevzhik/posts/3972180789483668}
\ifcmt
 author_begin
   author_id bilchenko_evgenia
 author_end
\fi

БЖ. Подборка сельских анекдотов

Родня, сегодня не 1 апреля, но несколько блистательных  перлов из травящих меня
\enquote{профпригодных} учёных человеков не могу как \enquote{профнепригодная} неученая
\enquote{дилетантка} не привести. Настроение хорошее.

\ifcmt
  pic https://scontent-lga3-2.xx.fbcdn.net/v/t1.6435-9/199823824_3972180719483675_8484117415626339317_n.jpg?_nc_cat=104&ccb=1-3&_nc_sid=8bfeb9&_nc_ohc=Arim6VbHWtoAX_vpZaC&_nc_ht=scontent-lga3-2.xx&oh=0f3b992bd8f873148fe9df2481bcf2a2&oe=60C98965
\fi

- Кто основал письменность на Руси?
- Кирилло-Мефодиевское братство.
- Как интересно! А кто туда входил?
- Конечно, Шевченко, Костомаров и Кулиш.
- И в каком веке они основали древнерусскую письменность?
- Ну, в XI или XII - это же Русь.
- А ещё каких-то писателей Киевской Руси вы знаете?
- Конечно. Несторы.
- И много их было?
- Да, много их ходило по Руси... (Укрфил НПУ).
- В системе управления над митрополитами в православной церкви кто стоит?
- Филареты (религиоведение, НПУ).
- Культуральность - это... cultural studies, а еще там марксизм, это политология, очень вредная вещь. Русской культурологии вообще не было. Бахтин, Маркарян? Какой Маркарян? (кандидат  философских наук, НПУ). 
- Достоевский вообще предсказал Делеза, русское тут - не при чем (доктор культурологии, НПУ).
- Главный метод культурологии - книга Вятровича (оттуда же).
- Сакрального треугольника в Египте не было в символике пирамиды (высшее руководство вуза).
- В... славяноарийских Ведах... описаны традиционные ценности древних украинцев (травести-аспирант, НПУ).
- "Бытие-в-себе" - это ваш неологизм?
- Нет, это Гегель. 
- Что, так и пишется? (НИИ культурологии).
- А какое имя и отчество у Платона, почему вы не упомянули? (оттуда же).
- В лице "андрофагов" в книге "Греко-перские войны" Геродот описал жителей будущей Москвы и их диктаторские нравы (оттуда же).
- В Бабьем Яру евреев расстреляли власовцы (оттуда же).
- "Бытие к смерти" - это вы такое придумали?
- Нет, Хайдеггер.
- Интересно...
- "Россия придумала Победу 9 мая" (на конференции по цифровым эффектам... в экранном искусстве).
- Жители Шумера происходят от пришельцев со звезды Сириус (доктор филос. наук, НПУ).
- Бретон не сотрудничал с коммунистами и не говорил о политической революции в сюрреалистическом манифесте (оттуда же).
- Ален Бадью - фашист (оттуда же).
На сегодня - Зощенко во мне исчерпан. Всем хорошего настроения у Горького на дне.
