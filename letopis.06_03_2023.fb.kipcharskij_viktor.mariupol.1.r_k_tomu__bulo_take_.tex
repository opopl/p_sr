%%beginhead 
 
%%file 06_03_2023.fb.kipcharskij_viktor.mariupol.1.r_k_tomu__bulo_take_
%%parent 06_03_2023
 
%%url https://www.facebook.com/permalink.php?story_fbid=pfbid08QmCM4pPoftd1pK8kGkCMqtrhPFLAHWsPNKtLLg7duRtPy1AtQ7E9JTXjgtj3q9Sl&id=100006830107904
 
%%author_id kipcharskij_viktor.mariupol
%%date 06_03_2023
 
%%tags mariupol,mariupol.war,dnevnik,06.03.2022
%%title Рік тому  було таке:  День 11 - 06.03.22. Неділя
 
%%endhead 

\subsection{Рік тому  було таке:  День 11 - 06.03.22. Неділя}
\label{sec:06_03_2023.fb.kipcharskij_viktor.mariupol.1.r_k_tomu__bulo_take_}

\Purl{https://www.facebook.com/permalink.php?story_fbid=pfbid08QmCM4pPoftd1pK8kGkCMqtrhPFLAHWsPNKtLLg7duRtPy1AtQ7E9JTXjgtj3q9Sl&id=100006830107904}
\ifcmt
 author_begin
   author_id kipcharskij_viktor.mariupol
 author_end
\fi

Рік тому  було таке: 

День 11 - 06.03.22. Неділя.

Прокинувся опівночі в тиші чи від тиші. Світліє щілина між шторам і це дивно,
бо зазвичай в цей час ще зовсім темно - не видно навіть руки біля носу. Думав,
що десь горить, а насправді - випав сніг.

0:54. Почали гупати "відліти". Це ж треба: звук гарматного пострілу заспокоює.

2:45. З-за правої доменної печі (\#2) видні спалахи, а секунди за три лунає звук
пострілу. Три секунди - це близько трьох кілометрів. Тобто стріляють зі старого
копрового цеху? (Копровий цех - це площадка під відкритим небом за старим
мартеном, на якій подрібнюють крупногабаритний металобрухт).

6:30. Гупає дуже голосно.

9:00. Перебирав спінінги - заспокоював нерви. Шкода буде їх залишати, але якщо
брати, то це ще дві-три сумки катушок та приманок.

9:40. Син виносив сміття. Каже, що снаряд прилетів у приватні будинки дорогою
до гаражів. Ще один - дорогою до школи. Новостепова? 

10:00. Зайшли свати. Вони "гуляли на районі". На подвір'ї школи \#42 стоїть
сгоріла військова техніка. Ракета влучила у п'ятий поверх над Лазур'ю-Каскадои
на Повороті. Їхати вони не збираються - залишаться.

14:00. Сусідка з 4-го поверху принесла дві палки копченої ковбаси: приїхала
машина і якійсь чоловік просто роздавав ковбасу з кузова - вона взяла на нас.
Дякуємо.

Поки є газ, невістка пече кекси-бісквіти. Оля зварила суп. Я лущів горіхи та
розповідав онукам про Куму Лису. Потім пішов шукати цю книгу у барикаді на
підвіконні. Якось напередодні син заклеїв нахрест скотчем вікна, виклав на
підвіконня книжки і посунув до вікна книжкову шафу, аби захиститися від уламків
скла. Я відібрав кілька дитячих книжок, а син потім після мене відновлював
порядок.

13:00. По радіо сказали, що Зеленого коридору не буде, але ви тримайтесь.

О 16:00 газ ще був. О 17:45 онукам вечеряти - газу вже нема. Зник останній
атрібут цивілізації.

Завтра будемо гріти їжу на запальничках або на вогнищі.

%\ii{06_03_2023.fb.kipcharskij_viktor.mariupol.1.r_k_tomu__bulo_take_.cmt}
