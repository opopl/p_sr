% vim: keymap=russian-jcukenwin
%%beginhead 
 
%%file 06_01_2017.stz.news.ua.mrpl_city.1.rozhdestvo_v_priazovje
%%parent 06_01_2017
 
%%url https://mrpl.city/blogs/view/rozhdestvo-v-priazove
 
%%author_id burov_sergij.mariupol,news.ua.mrpl_city
%%date 
 
%%tags 
%%title Рождество в Приазовье
 
%%endhead 
 
\subsection{Рождество в Приазовье}
\label{sec:06_01_2017.stz.news.ua.mrpl_city.1.rozhdestvo_v_priazovje}
 
\Purl{https://mrpl.city/blogs/view/rozhdestvo-v-priazove}
\ifcmt
 author_begin
   author_id burov_sergij.mariupol,news.ua.mrpl_city
 author_end
\fi

\ii{06_01_2017.stz.news.ua.mrpl_city.1.rozhdestvo_v_priazovje.pic.1}

Из двунадесятых православных праздников Рождество Христово - самый веселый и
нарядный. С гаданиями в ночь в сочельник, с колядками и ряжеными, с
мистическими историями. Вспомним хотя бы \enquote{Ночь перед Рождеством}, так живо и
ярко описанная Николаем Васильевичем Гоголем. Этот праздник породил особый жанр
в литературе – святочный рассказ. Ведь святки – время надежд на лучшее будущее,
на милосердие,  на неминуемую победу добра над злом. Святочные рассказы
написали в разное время Федор Достоевский и Александр Куприн, Леонид Андреев и
Валерий Брюсов, Антон Чехов и Владимир Короленко, Николай Лесков и Иван Шмелев.

\ii{06_01_2017.stz.news.ua.mrpl_city.1.rozhdestvo_v_priazovje.pic.2}

Тема празднования православного Рождества огромна. И  здесь придется ее сузить
до границ Мариупольщины. То есть территории как самого Мариуполя, так и
прилегающих  к нему сел. В наших краях православие исповедовали и исповедуют
украинцы, русские, греки, потомки выходцев из Крыма. Есть в наших краях
представители и других народов, принявших православие или сами, или их предки.

\ii{06_01_2017.stz.news.ua.mrpl_city.1.rozhdestvo_v_priazovje.pic.3}

Более семи десятилетий борьбы с религией на государственном уровне привели к
тому, что, к сожалению, многие народные рождественские обычаи были утрачены
навсегда. И все же благодаря трудам местных историков-этнографов можно
представить, как наши предки отмечали Рождество.  Нина Алексеевна Тыркалова -
историк и мариупольский краевед, в мельчайших подробностях знающая быт народов
Приазовья.  Вот что запомнилось из ее рассказов. В украинских семьях к
Рождеству  готовились заранее. Все, что могло напоминать о работе, выносилось
из хаты. Выносили прялку и мотовило, ткацкий станок, а если это было жилище
гончара – гончарный круг, если в доме трудился сапожник – верстак, колодки,
лапку. В доме производилась уборка, стирка. Перед Рождеством резали свиней.
Украинцы считали, что Рождества без колбас не бывает. Из детства запомнилось,
как дома к Рождеству варили узвар из заранее высушенных яблок, вишен, жерделей
– плодов от абрикосов-дичек. И, конечно, готовили кутью. Для нее отваривали
пшеницу - \enquote{арновку}, к ней добавляли поджаренную муку и все это сдабривали
растворенным в теплой воде медом.  Вот тогда пионер Сережа, озираясь, чтобы
никто не заметил, шел с Торговой улицы на Слободку с узелком из хустки, где
были тарелка с кутьей, кусок домашней \enquote{малороссийской} колбасы, шматок сала и
пирожки с начинкой из кабака. Это была \enquote{вечеря} его крестной Матрене
Григорьевне.

\ii{06_01_2017.stz.news.ua.mrpl_city.1.rozhdestvo_v_priazovje.pic.4}

В домах русских людей в канун Рождества (в сочельник) происходила обрядовая
трапеза. Она начиналась с того, что глава семьи разбрасывал зерна пшеницы или
ржи, приговаривая: \enquote{На здоровье человечье, на коровье и овечье}. Под скатертью
на столе он же раскладывал  крестом зерна всех злаков, которые возделывала
семья. На стол ставилось двенадцать блюд - по числу апостолов. В рождественскую
подготовку входила и поездка на ярмарку. Для всей семьи - от стариков до
младенцев - надо было приобрести обновки. Для детишек покупали  подарки и
сладости. 

Доктор исторических наук, профессор Ирина Семеновна Пономарева  в своей
фундаментальной монографии \enquote{Етнічна істория греків Приазов'я (кінець ХVIII –
початок ХХI століть)} пишет, что  Рождество в греческих селах Приазовья, как в
самой Греции, отмечаются не так торжественно по сравнению с другими странами».
Некоторые колядки греков сохранились  до нашего времени. Из-за отсутствия
письменности греки передавали колядки из уст в уста, часто не понимая смысл
отдельных слов. Религиозные обряды мариупольские греки, как и другие
православные,  соблюдали до 20-х годов ХХ столетия. То есть до варварского
разрушения храмов и навязчивой антирелигиозной пропаганды. На рубеже ХIХ и ХХ
веков в греческих селах, да и в Мариуполе, молодые парни и девушки в сочельник
до рассвета начинали колядовать. Колядовали практически весь день. Они пели,
почти выкрикивали, колядку. Вот ее перевод на русский язык: \enquote{Христос родился,
людям радость! Хозяйка, дай свечку!} Но чаще вместо свечки колядникам дарили по
бублику или мелкие монеты.

До гонений на церковь и священнослужителей в рождественскую ночь во всех
православных храмах службу правили почти беспрерывно. Но далеко не всем
молящимся  хватало сил выстоять всенощную. Однако большим грехом это не
считалось. Так, во всяком случае, утверждала бабушка...
