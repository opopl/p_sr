% vim: keymap=russian-jcukenwin
%%beginhead 
 
%%file 21_10_2020.fb.david_statnikov.1.science_ukraine
%%parent 21_10_2020
 
%%url https://www.facebook.com/dstatnikov/posts/3649176291761908
%%author 
%%tags 
%%title 
 
%%endhead 

\subsection{Ученые - Украина}
\label{sec:21_10_2020.fb.david_statnikov.1.science_ukraine}
\Purl{https://www.facebook.com/dstatnikov/posts/3649176291761908}

Наконец удалось собраться "умникам" из 9 стран в ZOOMе. 
С радостью отметили, что никто не переболел COVID.
Обсудили:

- странные выборы Президента в США, впервые за всю её историю;
-"присутствие" России во всех неспокойных точках мира, её влияние на хаос и радикализм в мире,...;
- "просыпание" Европы, но медленное, от исламского радикализма;
- СOVID: реальности, политика, ослабление экономик, разные точки зрения, взгляды и "выгодность" некоторым, вакцины,...;
- прорыв в арабо-израильских отношениях, дающий миру новые шансы;
- переход развития стран мира от общебазовых гуманитарных основ, на внутренние,
протекционистские, научно-технические и концентрация развития и  существования
стран на совершенно новом образовании, на биотехнологиях, медицине, на участии
в космических исследованиях и проектах, на мировой интеграции в этом, …;

И, наконец, об Украине.

Разговор был долгий (несколько часов), с осмыслением каждого аргумента. Было бы
бессмысленно стараться передать все обсуждения, постараюсь сконцентрироваться
на наших, украинских проблемах.

Итак, главное - Украина, так или иначе, возникает в клубке большинства (не
всех) европейских и мировых проблем политики и интересов стран, которые влияют
на развитие мира и его "переделы" - США, ЕС, Китай.

Украина, за 30 лет так и не нашла своё место в цивилизации 21 века,
практически, уничтожив свои уникальные потенциалы: георасположение, уникальные
природные богатства, не так давно, уникальную науку и авиакосмическую
промышленность, большую (ударение на О) часть профессионально - трудолюбивого
населения (от рабочих, до учёных и уезжающей талантливой молодёжи).  Возможно и
скорее всего, эти потери невосполнимы в ближайшие десятилетия. Нужно приложить
огромные профессиональные усилия, нужна чёткая программа восстановления
потенциала и нужны большие, целевые, средства.

А этого, как раз, нет в современной Украине и не ощущается, что Украина (в т.ч.
в лице её властей, последние 20 лет и сейчас) способна, понимает как, может и
хочет изменить вектор развития и сохранения страны, как часть современного,
высокотехнологичного и социального мира. При всём его разнообразии политических
систем.

Решения политических властей Украины последние 20 лет и сейчас, всё более
провинциальны и уводят Украину из цивилизованного, современного мира.

Исходя из общей ситуации, страны "гегемоны" и международные структуры сами
ОТВОДЯТ Украине только РОЛЬ ИНСТРУМЕНТА/объекта мировой политики и их
интересов. 

Всего не перескажешь в небольшом посте.
