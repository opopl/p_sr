% vim: keymap=russian-jcukenwin
%%beginhead 
 
%%file 16_11_2021.fb.fb_group.story_kiev_ua.1.dengi_mirvojna2
%%parent 16_11_2021
 
%%url https://www.facebook.com/groups/story.kiev.ua/posts/1799030106960449
 
%%author_id fb_group.story_kiev_ua,lushelskij_boris
%%date 
 
%%tags dengi,mirvojna2
%%title ВОЕННЫЕ ДЕНЬГИ ВТОРОЙ МИРОВОЙ ВОЙНЫ
 
%%endhead 
 
\subsection{ВОЕННЫЕ ДЕНЬГИ ВТОРОЙ МИРОВОЙ ВОЙНЫ}
\label{sec:16_11_2021.fb.fb_group.story_kiev_ua.1.dengi_mirvojna2}
 
\Purl{https://www.facebook.com/groups/story.kiev.ua/posts/1799030106960449}
\ifcmt
 author_begin
   author_id fb_group.story_kiev_ua,lushelskij_boris
 author_end
\fi

Добрый вечер, мои дорогие киевляне. Привет, земляки!

Я хочу показать вам свои новые поступления и, вкратце рассказать о них. Итак:

ВОЕННЫЕ ДЕНЬГИ ВТОРОЙ МИРОВОЙ ВОЙНЫ.

Накануне Второй мировой войны в третьем рейхе была разработана программа
неотложных мер по снабжению немецких войск на территории захваченных государств
необходимыми денежными средствами.

\ii{16_11_2021.fb.fb_group.story_kiev_ua.1.dengi_mirvojna2.pic.1}

Эта программа стала осуществляется с момента вторжения фашистских войск в
Польшу в сентябре 1939 года, когда в созданном Генерал-губернаторстве \enquote{Польша}
стали выпускаться билеты имперских кредитных касс Германии.

Весной 1940 года в Германии была завершена подготовка к крупномасштабному
выпуску и внедрению в обращение на территориях захваченных стран военных марок
единого образца.

\ii{16_11_2021.fb.fb_group.story_kiev_ua.1.dengi_mirvojna2.pic.2}

Денежные билеты имперских кредитных касс выпускались номиналами в 5 пфеннигов,
1; 2;5;20 и 50 рейхсмарок ( они были пронумерованы, но не имели даты выпуска).

Механизм появления билетов имперских кредитных касс был следующим: как только
затихали бои, на захваченной территории появлялись так называемые \enquote{подвижные
экономические отряды}, которые вели учёт захваченных материальных и культурных
ценностей, вывозимых под видом трофеев в Германию. Составной частью этих
\enquote{отрядов} были имперские кредитные кассы ( военно-полевые банки ), которые
бесперебойно  снабжали наступающие войска Вермахта военными деньгами, а так же
занимались кредитованием местных правительств и банков.

\ii{16_11_2021.fb.fb_group.story_kiev_ua.1.dengi_mirvojna2.pic.3}

Курс одной марки равнялся десяти рублям.

В 1942 году,на территории оккупированной Украины были созданы такие же
имперские банки. В одном из таких банков, в городе Ровно, оккупационное
правительство начало выпускать Военные деньги для Украины ( оккупационки), так
называемые карбованцы. Этими же деньгами пользовалось население южных областей
Белоруссии.

\ii{16_11_2021.fb.fb_group.story_kiev_ua.1.dengi_mirvojna2.pic.4}

Итак мы видим, что на период оккупации, на территории Украины и Киева, в
обращении было три вида денег:

Советские купюры от 1 до 10 рублей

Украинские карбованцы ( военного образца( от 1 до 500.

Немецкие ( военные) рейхсмарки от 1 до 50 

Исчезли из обращения в 1944-ом.

\ii{16_11_2021.fb.fb_group.story_kiev_ua.1.dengi_mirvojna2.cmt}
