% vim: keymap=russian-jcukenwin
%%beginhead 
 
%%file poetry.rus.evgenija_bilchenko.kolokola
%%parent poetry.rus.evgenija_bilchenko
 
%%url https://t.me/bilchenko_z/104
%%author 
%%tags 
%%title 
 
%%endhead 

\subsubsection{БЖ. Монолог колокола}
\Purl{https://t.me/bilchenko_z/104}

Если нам не отлили колокол,
Значит, здесь время колокольчиков.
Александр Башлачев

Смесь меди с трипольским оловом,
Спина в расписных узорах...
Я — древний центральный колокол
От Храма, который взорван.

Там девы в Святую Троицу
Сплетали венки тугие
И пели о Богородице
Небесные литургии.

Там дети светлоголовые
Пускали из рук кукушек,
Но кто-то хотел столовую:
Ведь надо же людям кушать

И в красочный телек пялиться,
Кроша об орешки пломбы...
Так, в ночь со среды на пятницу
В алтарь подложили бомбу.

Напрасно я выл: «Пожалуйста,
Помилуйте богомолок!» -
Полвека потом пожарище
Пугало глухой поселок.

Теперь я лежу на мусорке:
Во мне копошатся мыши.
Мою неземную музыку
Никто на земле не слышит.

Тараньками и тарелками
Себя веселит эпоха.
Девчушка-речушка мелкая,
Нет-нет, да издаст по вздоху.

Бубенчики-колокольчики
Меня размножают в «мини»;
И только на самом кончике
Иголки (иголка — в мине,

А мина — в гробнице байковой)
С какой-то пустой надеждой
Меня окликает Батюшка...
По имени.
Как и прежде.

Фото: Аркадий Веселов
