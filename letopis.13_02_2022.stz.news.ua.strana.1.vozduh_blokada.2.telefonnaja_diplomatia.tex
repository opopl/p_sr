% vim: keymap=russian-jcukenwin
%%beginhead 
 
%%file 13_02_2022.stz.news.ua.strana.1.vozduh_blokada.2.telefonnaja_diplomatia
%%parent 13_02_2022.stz.news.ua.strana.1.vozduh_blokada
 
%%url 
 
%%author_id 
%%date 
 
%%tags 
%%title 
 
%%endhead 

\subsubsection{Телефонная дипломатия}
\label{sec:13_02_2022.stz.news.ua.strana.1.vozduh_blokada.2.telefonnaja_diplomatia}

Параллельно шел целый вал звонков с Запада в Россию. 

Министру обороны РФ Сергею Шойгу позвонил коллега из Пентагона Ллойд Остин, а
Владимиру Путину - Эммануэль Макрон.

Остин заявил о \enquote{наращивании Россией сил в Крыму и вокруг Украины}. А Макрон
поговорил с Путиным о Минских соглашениях и стабильности в Европе.

Также французский президент заявил о том, что \enquote{искренний диалог несовместим с
эскалацией}. В Париже сообщили, что нужно \enquote{попытаться ослабить растущую
напряженность в российско-украинском кризисе}. Кроме того, отмечено, что Макрон
и Путин намерены продолжить диалог о выполнении минских соглашений и
стабильности в Европе.

Интересно, что после этого разговора президент Франции заявил, что не видит
угроз со стороны РФ. О том же, кстати, сообщил отдельно и МИД Турции.

Наиболее важные беседы прошли между Россией и США. Сначала созвонились министры
иностранных дел, а потом и президенты. 

По версии Вашингтона, Байден в основном только угрожал Путину разными карами за
предполагаемое \enquote{вторжение}. В Москве же рассказали, что санкции звучали, но
акцента на них американская сторона не делала. И дали свою трактовку разговора.

Помощник президента РФ Юрий Ушаков заявил, что президент США выдвинул
\enquote{несколько идей по безопасности}. Теперь Москва \enquote{рассмотрит предложения Байдена
и по возможности учтет в своей реакции на ответ НАТО}.

Однако Путин в свою очередь заявил Байдену, что Россия даже в этих новых
предложениях не получила по существу ответа по гарантиям безопасности
(напомним, Москва требует гарантий нерасширения Альянса на восток и свертывания
военной инфраструктуры НАТО в Восточной Европе).

По данным Ушакова, уже закончены последние согласования того, как Москва будет
реагировать на предложения США и НАТО. И скоро этот ответ будет передан. 

В Кремле также при разговоре с Байденом назвали информацию о \enquote{вторжении}
заведомо ложной. Причем в последние дни американские власти ее \enquote{довели до
абсурда}, сливая в СМИ.

Далее заявили, что под этим предлогом украинскую армию начали резко накачивать
оружием. Чем \enquote{создаются предпосылки для возможных действий украинских военных}.

Последний момент сегодня повторяют со многих российских площадок. Путин по сути
обвиняет США в ускоренной милитаризации Украины - чего бы не стали делать без
информационной кампании с \enquote{вторжением}.

Также Путин упрекнул Байдена, что США не осуществляют должного нажима на
Украину для выполнения Минских соглашений. Прозвучало это так, будто Вашингтон
действительно обещал это сделать. 

Но самое важное из прозвучавшего - некие новые идеи Байдена, которые Кремль
рассмотрит. То есть, как и предполагала \enquote{Страна}, разговоры о \enquote{вторжении} могут
активизировать дипломатическую составляющую.

При этом говорить о прорыве пока оснований нет. По крайней мере, внешне он
никак не проявляется. 

