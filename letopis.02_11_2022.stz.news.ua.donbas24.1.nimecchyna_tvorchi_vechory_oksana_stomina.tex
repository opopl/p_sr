% vim: keymap=russian-jcukenwin
%%beginhead 
 
%%file 02_11_2022.stz.news.ua.donbas24.1.nimecchyna_tvorchi_vechory_oksana_stomina
%%parent 02_11_2022
 
%%url https://donbas24.news/news/oksana-stomina-u-mistax-nimeccini-provodit-tvorci-vecori-prisvyaceni-mariupolyu
 
%%author_id demidko_olga.mariupol,news.ua.donbas24
%%date 
 
%%tags 
%%title В Німеччині проходять творчі вечори, присвячені Маріуполю
 
%%endhead 
 
\subsection{В Німеччині проходять творчі вечори, присвячені Маріуполю}
\label{sec:02_11_2022.stz.news.ua.donbas24.1.nimecchyna_tvorchi_vechory_oksana_stomina}
 
\Purl{https://donbas24.news/news/oksana-stomina-u-mistax-nimeccini-provodit-tvorci-vecori-prisvyaceni-mariupolyu}
\ifcmt
 author_begin
   author_id demidko_olga.mariupol,news.ua.donbas24
 author_end
\fi

\ii{02_11_2022.stz.news.ua.donbas24.1.nimecchyna_tvorchi_vechory_oksana_stomina.pic.front}

\begin{center}
  \em\bfseries\color{blue}\Large
Оксана Стоміна на творчих вечорах у Німеччині розповідає про трагедію Маріуполя
та маріупольців 
\end{center}

31 жовтня у Білефельді відбулася творча зустріч з відомою маріупольською
поетесою і прозаїком, лауреаткою літературної премії ім. Юрія Каплана,
літературної премії \enquote{Слов'янські традиції}, громадською діячкою, засновницею
громадської організації \enquote{Паперові сходи} та волонтеркою Оксаною Стоміною і
спільний літератур\hyp{}но-музичний вечір, на якому українські і німецькі актори та
акторки читали твори \href{https://www.facebook.com/oksana.stomina}{Оксани}%
\footnote{\url{https://www.facebook.com/oksana.stomina}}
і новоспеченого лауреата німецької літературної премії миру Сергія Жадана.

\textbf{Читайте також:} \href{https://donbas24.news/news/mariupolski-xudoznici-predstavili-novu-vistavku-v-dortmundi}{\emph{Маріупольські художниці представили нову виставку в Дортмунді}}%
\footnote{Маріупольські художниці представили нову виставку в Дортмунді, Ольга Демідко, donbas24.news, 01.11.2022, \par%
\url{https://donbas24.news/news/mariupolski-xudoznici-predstavili-novu-vistavku-v-dortmundi}%
}

\begin{leftbar}
\emph{\enquote{Я була вражена кількістю гостей і якістю організації зустрічі, щирою
небайдужістю, глибокою увагою, підтримкою кожного, хто долучився до
події. Я дякую німкеням, які плакали, наче йдеться про своє... Дякую
їхнім чоловікам, які ловили кожне слово. Дякую за всі розкуплені
книжки, бо ці гроші відправляться на добру й важливу справу!}}, —
наголосила Оксана.
\end{leftbar}

\ii{02_11_2022.stz.news.ua.donbas24.1.nimecchyna_tvorchi_vechory_oksana_stomina.pic.1}

Після виїзду з України до Німеччини Оксана за допомогою своїх віршів
намагається за кордоном розповідати про те, що відбувається зараз з кожним
українцем, адже вони написані з безпосереднього досвіду та переживань. Протягом
місяця від початку повномасштабного вторгнення рф в Україну жінка принципово
залишалася в Маріуполі, аби у міру сил допомагати військовим захищати місто й
цивільним — вижити в ньому в оточенні і під щільними обстрілами. \textbf{Дмитро
Паскалов}, чоловік Оксани, 16 березня наполіг, щоб вона поїхала з Маріуполя, при
цьому буквально змусив її сісти в машину.

Поетеса виїхала з Маріуполя з мінімумом речей, з малесеньким, на п'ять літрів,
наплічником. Наразі Дмитро у російському полоні. Він один із в'язнів
металургійного заводу \enquote{Азовсталь}, був начальником азовстальського порту та
займав дуже активну громадську позицію. Дмитро брав участь у багатьох
загальноміських проєктах у Маріуполі. Оксана не знає, що наразі з її чоловіком,
в яких умовах він знаходиться. На творчому вечорі маріупольчанка розповідала
свою історію та історію всіх маріупольців та загалом українців через власні
вірші.

\ii{insert.read_also.demidko.donbas24.mrpl_art_rezidencia_postmost_vidnov_dijalnist}

\ii{02_11_2022.stz.news.ua.donbas24.1.nimecchyna_tvorchi_vechory_oksana_stomina.pic.2}
\ii{02_11_2022.stz.news.ua.donbas24.1.nimecchyna_tvorchi_vechory_oksana_stomina.pic.3}

На творчому вечорі виступали музиканти та актори, серед яких була і
маріупольська актриса \href{https://www.facebook.com/profile.php?id=100001752890543}{Ольга Самойлова}.%
\footnote{\url{https://www.facebook.com/profile.php?id=100001752890543}}
Для неї ця подія була водночас і приємною, і дуже важкою.

\begin{leftbar}
\emph{\enquote{Після сьогоднішнього творчого вечора я прийшла геть виснажена з того, що я
весь час стримувала сльози и бажання кричати, а ця тендітна жінка
їздить по різних містах Німеччини і проводить такі творчі вечори і
розповідає свою історію, історію нашого міста її очами, вона живе і
плаче віршами, як вона сама каже. Чекає свого чоловіка з полону, вірить
в світле майбутнє і посміхається! Мені дуже приємно було обійняти її і
мати за честь прочитати декілька її віршів!}}, — розповіла Ольга.
\end{leftbar}

Читати вірші актрисі було досить непросто, адже говорити про те, що так сильно
болить можуть далеко не всі.

\ii{insert.read_also.demidko.donbas24.futbolki_nimecchyna_mrpl}
\ii{02_11_2022.stz.news.ua.donbas24.1.nimecchyna_tvorchi_vechory_oksana_stomina.pic.4}

Також вірші читали німецька акторка Brit Dehler і українські митці театральної
і музичної сцени Наталія Ховда, Найра Арзуманян та Василь Хотсько. Всі вони
посприяли створенню особливої атмосфери та цілісного емоційного дійства.

\begin{leftbar}
\emph{\enquote{Поетеса і письменниця з Маріуполя Оксана Стоміна довела вчора на вечорі, що її
літературний голос не завмер від пережитого, а може ставати жарким
полум'ям, яке запалює інших на українську тему}}, — зазначила Олена
Туров.
\end{leftbar}

\ii{02_11_2022.stz.news.ua.donbas24.1.nimecchyna_tvorchi_vechory_oksana_stomina.pic.5}
\ii{02_11_2022.stz.news.ua.donbas24.1.nimecchyna_tvorchi_vechory_oksana_stomina.pic.6}
\ii{02_11_2022.stz.news.ua.donbas24.1.nimecchyna_tvorchi_vechory_oksana_stomina.pic.7}

\begin{leftbar}
\emph{\enquote{Це була зустріч, сповнена сліз, печалю, але й надій та сподівань на нашу перемогу!}}, — підкреслила Ірина Гринько.
\end{leftbar}

Також на творчому вечорі можна було почути музику українського письменника та
автора-виконавця власних пісень В'ячеслава Купрієнка і побачити світлини Євгена
Сосновського, які йому вдалося вивезти з Маріуполя.

Раніше Донбас24 розповідав, як маріупольські художниці \href{https://donbas24.news/news/mariupolski-xudoznici-predstavili-novu-vistavku-v-dortmundi}{\emph{представили нову виставку у Дортмунді}}.%
\footnote{Маріупольські художниці представили нову виставку в Дортмунді, Ольга Демідко, donbas24.news, 01.11.2022, \par\url{https://donbas24.news/news/mariupolski-xudoznici-predstavili-novu-vistavku-v-dortmundi}}

Ще більше новин та найактуальніша інформація про Донецьку та Луганську області
в нашому телеграм-каналі Донбас24.

ФОТО: з відкритих джерел.

\ii{insert.author.demidko_olga}
%\ii{02_11_2022.stz.news.ua.donbas24.1.nimecchyna_tvorchi_vechory_oksana_stomina.txt}
