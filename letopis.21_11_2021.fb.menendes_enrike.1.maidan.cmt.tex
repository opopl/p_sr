% vim: keymap=russian-jcukenwin
%%beginhead 
 
%%file 21_11_2021.fb.menendes_enrike.1.maidan.cmt
%%parent 21_11_2021.fb.menendes_enrike.1.maidan
 
%%url 
 
%%author_id 
%%date 
 
%%tags 
%%title 
 
%%endhead 
\subsubsection{Коментарі}

\begin{itemize} % {
\iusr{Энрике Менендес}
Комментарии в фейсбуке существуют для того, чтобы напоминать сколько в мире кретинов.

\begin{itemize} % {
\iusr{Alex Kriloff}
\textbf{Энрике Анатольевич Менендес} и ватной погани


\iusr{Сергей Перевозчиков}
\textbf{Энрике Менендес} не только комментарии, но и посты.

\iusr{Михаил Альперович}
\textbf{Энрике Менендес} тебя кто то обидел сегодня?

\iusr{Ihor Chvankin}
\textbf{Энрике Анатольевич Менендес} как и посты. Считал Вас умнее. Т.н. РФ в пятницу ушла в технический дефолт по госкорпорации. Да, 1944 - самое время было мириться с Гитлером.

\iusr{Сергей Основин}
\textbf{Ihor Chvankin} не «мириться», а «нормализовать отношения».  @igg{fbicon.wink} 

\iusr{Вадим Щетинин}
\textbf{Михаил Альперович}, а может он по жизни обиженный.

\iusr{Сергей Прокопенко}
\textbf{Энрике Менендес} хотел подписаться под каждым словом, а теперь не знаю, шо и делать... @igg{fbicon.grin} 

\iusr{Олег Матковський}
\textbf{Энрике Анатольевич Менендес} да и посты тоже.

\iusr{Andre Bo}
\textbf{Alex Kriloff}

\ifcmt
  ig https://scontent-frt3-1.xx.fbcdn.net/v/t39.30808-6/259680105_253176670210783_1095598962196818733_n.jpg?_nc_cat=108&ccb=1-5&_nc_sid=dbeb18&_nc_ohc=5ONjViHvTNQAX9C7ech&_nc_ht=scontent-frt3-1.xx&oh=a74b804db48180ac945a751b803dbbd4&oe=619F8E49
  @width 0.4
\fi

\iusr{Andre Bo}
\textbf{Ihor Chvankin}

\ifcmt
  ig https://scontent-frt3-1.xx.fbcdn.net/v/t39.30808-6/259640050_253176966877420_6203643775332615033_n.jpg?_nc_cat=104&ccb=1-5&_nc_sid=dbeb18&_nc_ohc=u9L5Bnp6XSsAX8GcMtC&_nc_ht=scontent-frt3-1.xx&oh=f84fbc7e03d6e0a7c6c47aa0229d1d17&oe=619F9275
  @width 0.4
\fi

\iusr{Евгений Агапов}
Все правильно написано. Молодец Энрике.

\iusr{Олександр Костюченко}
\textbf{Энрике Анатольевич Менендес} Ну ваші тези прямо протирічать одна одній. Ви вже визначіться за росію чи за Україну?

\iusr{Павел Климов}
\textbf{Ihor Chvankin} 

Дефолт объявили в Роснано, которое до недавнего времени возглавлял прозападный
либерал Чубайс, которого ненавидят все ватники. Тем временем золотовалютный
резерв России за год увеличился на 34 миллиарда до 626 миллиардов долларов.
Капитализация Роснано 2,5 миллиарда баксов. Для сравнения прибыль
государственного Сбера 13 млрд. баксов за 3-й квартал 2021. Так что пэрэмога
отменяется. \url{https://www.cbr.ru/hd_base/mrrf/mrrf_7d/}

\iusr{Anton Vernigor}
\textbf{Энрике Анатольевич Менендес} но твой пост показывает это гораздо красноречивее.

\iusr{Вячеслав Беленький}
\textbf{Ihor Chvankin}
Такую ересь можно вбросить только в информационно изолированное общество. Для чего украинцам и построили железный занавес.
Резервы РФ пробили потолок, макро стабильность железобетонная даже по признанию госдеповских ястребов.

\iusr{Irina Earley}
\textbf{Энрике Анатольевич Менендес} наконец-то за 8 лет высказался без обиняков и ПРЯМО. Почему только через 8 лет ? Если многим было понятно ещё в феврале 2014.

\iusr{Эдуард Рубин}

\textbf{Энрике Менендес}, ты первый 


\end{itemize} % }

\iusr{Энрике Менендес}

Час не заходил в комментарии, надо сделать апдейт:
не просто перепись кретинов, я ошибся.
Перепись злых и обиженных кретинов. Так будет вернее.
Большое счастье, что есть и вменяемые люди. Спасибо вам друзья! Вы помогаете не терять веру в человечество.

\begin{itemize} % {
\iusr{Sviatoslav Orobchuk}
\textbf{Энрике Менендес} по другому людей оставляющих такие комментарии и не назовешь

\ifcmt
  ig https://scontent-frt3-1.xx.fbcdn.net/v/t39.30808-6/259840562_6395907687117000_1641876235308533797_n.jpg?_nc_cat=108&ccb=1-5&_nc_sid=dbeb18&_nc_ohc=M95sEYT_WV4AX-t7wUm&_nc_ht=scontent-frt3-1.xx&oh=0578205d0b327de449c3fe9f2f38d54a&oe=619F54F2
  @width 0.4
\fi

\iusr{Anton Vernigor}
\textbf{Энрике Анатольевич Менендес} только вменяемых людей ты называешь обиженными кретинами (потому что с твоим высером не согласны, видимо), а биомусор - вменяемыми. Но если это учитывать, то все становится на свои места.


\iusr{Сергей Перевозчиков}
\textbf{Энрике Менендес} Пожалуйста, Энрике.

\iusr{Геннадий Щербак}
\textbf{Энрике Менендес} они мешают работать, их цель сбить, у них свой путь и своя история пусть живут паралельно, отвлекаясь на этих ....... Только теряешь драгоценное время которое можно использовать для того ради чего Ты пришел в этот мир.
\end{itemize} % }

% -------------------------------------
\ii{fbauth.german_andrej.kiev.ukraina}
% -------------------------------------

Реакция части людей в комментариях говорит о том, что они боятся и признают
существование значительного количества людей в стране с противоположными
взглядами. Это их беспокоит, они чувствуют опасность и каждый дизлайк,
оскорбление, попытки унизить и высмеять считают своим вкладом в борьбу добра со
злом в их понимании. Даже не так, правильного и неправильного. Они так решили
за других, и свобода мышления их раздражает и пугает.

Консолидации не видно и не будет в ближайшее время. Слишком мало времени
прошло, недостаточно потрясений для понимания, слишком мало светлых умов.

\begin{itemize} % {
\iusr{Энрике Менендес}
\textbf{Андрей Герман} Крутой коммент. Респект!

\iusr{Michail Vidiassov}
\textbf{Андрей Герман} если это глупость, но работает - это не глупость
\end{itemize} % }

\iusr{Юрий Антонов}

Энрике, к сожалению, майданную Украину в таком виде как сейчас уже не спасти -
давайте работать для будущей Украины.

\begin{itemize} % {
\iusr{Irina Earley}
\textbf{Юрий Антонов} 

будущая Украина, это в составе России. Или люди будут жить все хуже и хуже пока
не распадётся на Львовское, Днепропетровское
княжество, что уже и происходит.
\end{itemize} % }

\iusr{Игорь Гомольский}
Менендес - ватник!!!)))

\begin{itemize} % {
\iusr{Энрике Менендес}
\textbf{Игорь Гомольский} Нашёл, чем удивить. Ватник и сепар ))

\iusr{Игорь Гомольский}
\textbf{Энрике Менендес}, не, гля! Я же первый тебя разоблачииииил!)

\iusr{Павел Климов}
\textbf{Игорь Гомольский} Вот кому кастрюля на мозг не давит! Ахахах!

\iusr{Михаил Омский}
\textbf{Игорь Гомольский} И как всвязи с этим быть! Расстреляете  @igg{fbicon.smile} ?

\iusr{Игорь Гомольский}
\textbf{Михаил Омский}, вот хуй знает даже. Ну ладно эти всерьёз лайкают. А вы чего? Табличку «сарказм» нужно поднять?) Не расстреляю каэш. Съем просто)
\end{itemize} % }

\iusr{Iryna Palona}

ого, крепкую психику нужно иметь, чтоб выдерживать все эти хейтерские комменты,
мне от антиваксеров и то меньше прилетает  @igg{fbicon.smile} 

\begin{itemize} % {
\iusr{Сергей Основин}
\textbf{Iryna Palona} чота прочитал и ни одного «хейтерского коммента» не увидел... Хотя, если вы считаете хейтом оппонирование на любом уровне...

Ps. Не. Парочку таки увидел. Но, впрочем, как и а обратную сторону.


\iusr{Энрике Менендес}
\textbf{Iryna Palona} У меня ОЧЕНЬ крепкая психика )

\iusr{Anton Vernigor}
\textbf{Iryna Palona} можно было выразительнее - все эти незаслуженные, лицемерные, беспочвенные хейтерские коментарии (нет).

\iusr{Vasiliy Psiola}
\textbf{Iryna Palona} , так ведь ковидиоты и устроившие майдан, одного поля ягоды.

\iusr{Vasiliy Psiola}
\textbf{Iryna Palona} , вы не родственица менгеле?

\iusr{Vasiliy Psiola}
\textbf{Iryna Palona} , в бога играете? Имейте в виду, гордость самый большой грех, если не покаетесь - умрете!
\end{itemize} % }

\iusr{Olga Kosse}
Начался шабаш "хазяев" Украины, которые сейчас будут рассказывать, куда тебе собирать чемоданы))

\begin{itemize} % {
\iusr{Anton Vernigor}
\textbf{Olga Kosse} и вы еще скажите что будут неправы.

\iusr{Olga Kosse}
\textbf{Anton Vernigor}, конечно, скажу. Украина, вроде все еще демократичная страна, согласно Коституции. Жаль таких ушлепков как вы нельзя призвать к ответсвенности за оскорбления в соц.сетях. Хотя, вы себе рамку однажды поставили, что вы "порохобот". Собственно говоря, жизнь вас и так уже наказала)
\end{itemize} % }

\iusr{Artem Trumpet}
Добре сказали! Дякуємо!

\iusr{Max Nefyodov}

Іронічно, що саме завдяки Майдану \textbf{Энрике Анатольевич Менендес} може
писати подібне. А в ордло, де нормалізовані стосутнки з росією, де захід нікого
не використовує, а східні регіони представлені в політиці, його посадили «на
подвал» Непокаране зло повертається, доведено Менендесом  @igg{fbicon.smile} 

\begin{itemize} % {
\iusr{Энрике Менендес}
\textbf{Max Nefyodov} О, да. Именно майдан дал мне право голоса.
Чушь в квадрате, учитывая, что мне твой майдан не дал проголосовать в 2014, а теперь я также не могу выбрать себе даже облсовет.
Либерализаторы хреновы

\iusr{Лиля Веревкина}
\textbf{Энрике Анатольевич Менендес} коли ти виїхав із окупації? Жив при РФ?

\iusr{Энрике Менендес}
\textbf{Лиля Веревкина} Меня выдворили в 2016 году. "При РФ" я не жил ни дня.

\iusr{Лиля Веревкина}
\textbf{Энрике Анатольевич Менендес} ти жив в донецьку 14 та 15? Ну! Ти значить жив при днр та РФ! Не міг не бачити РФ!

\iusr{Alex Samorukov}

Вот да, я тоже не понимаю как же так вышло, что в русскомирном лднр вам жить не
хочется. Нормализации выше крыши, живут за счет подачек рф, паспорта выдают,
суркову и прочей погани отчитываются. Что же пошло не так?

\iusr{Сергей Гайдаш}
\textbf{Max Nefyodov} как всегда у патриотов. Комментировать то, чего не знаешь и не видел. Вы писали в Донецке и вас посадили за это на подвал?

\iusr{Alex Samorukov}
\textbf{Sergey Gaydash} т.е. пытки в "изоляции" враги придумали?

\iusr{Вячеслав Беленький}
\textbf{Max Nefyodov}
Бузине расскажите.

\iusr{Сергей Гайдаш}
\textbf{Alex Samorukov} и вы там были? У того, человека, который пытал в изоляции, торчат уши СБУ. Мало того, он разыскивается как в ДНР, так и в России. При этом, спокойно жил два года в Киеве. Подобных провокаций сотни. Но факт один. Вы там не были и достоверно не можете знать, что происходит.

\iusr{Alex Samorukov}
\textbf{Sergey Gaydash} нет, и не планирую. Как и не был к КНДР, например. Факты пыток "на подвалах" документированы многократно, и не видеть их может только слепой. И уж точно проверять качество подвалов не входит в мои планы. Это еще более глупо чем совать пальцы в розетку, утверждая что смерть от электричества придумали враги.

\iusr{Сергей Прокопенко}
\textbf{Max Nefyodov} думаю, Вам пізно писати 'патрійотичні месаги', Вас викинуто з тусівки... @igg{fbicon.face.tears.of.joy} 

\iusr{Александр Лапин}
\textbf{Энрике Менендес} , 

в плане гражданских прав и свобод вообще в Украине - ты не прав, при Януковиче
оппозицию и инакомыслящих преследовали как минимум так же, как при Порошенко
(просто власть и оппозиция поменялись местами). Но что касается гражданских
прав и свобод людей с донецкой, луганской и крымской регистрацией ( в том
числе, ВПЛ), ты прав на сто процентов! В современной Украине они стали аналогом
афроамериканцев в США до Мариинка Лютера Кинга - вроде бы и такой же гражданин,
но сегрегация при приёме на работу, выдаче кредитов и прочих банковских
услугах, подтверждении гражданства и других документов - присутствует, и
большинство украинцев это абсолютно не колышет. Даже голосовать ВПЛ разрешили
по фактическом месту жительства только год назад. А на подконтрольной части до
сих пор выборы отменены в отдельных местах...


\iusr{Артём Непомнящий}
\textbf{Max Nefyodov} шо ты мелешь, чучело? Майдан дал срок бабке за лайк флага СССР в одноклассниках.

\iusr{Олексій Романов}
\textbf{Max Nefyodov}

\ifcmt
  ig https://scontent-frt3-1.xx.fbcdn.net/v/t39.30808-6/259438453_4474148296040926_6851538324289516524_n.jpg?_nc_cat=104&ccb=1-5&_nc_sid=dbeb18&_nc_ohc=eQb1akYalSUAX8K0gIO&_nc_oc=AQnFjFBbHYJZEt9nkdlBQYpmY2E650xSOoJz42x-hhlaEG3XTVDVjQxYkNxBxLyW9LY&_nc_ht=scontent-frt3-1.xx&oh=2971e5d37707167ea0d9fb5b1a246426&oe=619FC89F
  @width 0.4
\fi

\iusr{Max Nefyodov}
\textbf{Энрике Анатольевич Менендес} 

ти цілком можеш вибрати «облсовєт» в рідному Донецьку, хіба ні? Там де
відкинули «госпєрєворот Майдана», повернулись до омріяних часів до 2014-го? Що
не так? Чого тебе так тягне до клятих бандерівців?

\iusr{Max Nefyodov}
\textbf{Vyacheslav Belenkiy} ще і Гіві с Моторолкой. Земля їм скловатою.

\iusr{Виталий Лукащук}
\textbf{Max Nefyodov} 

как можно так врать? И ведь, наверное, не краснеешь. Результат госпереворота
2014 года - десятки тысяч убитых украинцев (своими же вооруженными силами),
миллионы беженцев от нищеты и политических преследований. Я не говорю о падении
экономики, уровня жизни населения Украины, неизбежной деградации последующих
поколений. В России, как и любой стране, немало проблем. Тем не менее, нынешней
Украине далеко до свободы чего-либо в России. А твой комментарий - попытка
бездарной манипуляции. Впрочем, чему удивляться, сборище позорников, именуемых
"политической элитой" Украины уже седьмой год не вызывает ничего, кроме
недоумения и смеха.

\iusr{Sergey Moroz}
\textbf{Сергей Прокопенко} он в резерве. Там использованных не выбрасывают, подкармливают грантами, чтоб при случае нам на довольствие поставить. Для этого и рвет вышиванку на груди;)

\iusr{Sergey Moroz}
\textbf{Энрике Менендес} какой он либерализатор? Корупционер и иностранный агент.

\iusr{Max Nefyodov}
\textbf{Виталий Лукащук} 

я то знаю, як ви можете так брехати. Ви просто щирі вороги українського народу.
Тому як і Менендес, не ховаючись, працюєте на агресора, виправдовуєте його дії,
кажете будь-яку брехню (ви в неї і самі не вірите - просто ж війна, тому
розповідати казки - це обов‘язок кожного піонєра).

\iusr{Олег Дьяченко}
\textbf{Виталий Лукащук} а вы точно профессор? А то у нас был один, очень похоже говорил...

\iusr{Виталий Лукащук}
\textbf{Max Nefyodov} 

и снова никаких фактов, просто поток негативных эмоций. А записывать во "врагов
народа", тех, кто не бьется, как пришибленный, в истеричных припадках
псевдолюбви к Бандере, не верит в "Святой Майдан" - это очень демократично и
по-еврпейски. В Европе всегда именно так и поступают. И Да, я был пионером.
Возможно, был бы и комсомольцем, и членом КПСС, потому что я коммунист. Только
вот проблемы на Украине не из-за коммунистов, а от вороватой неумной власти, от
околовластной националистической шоблы, от украинских олигархов, обворовывающих
украинский народ, от западных господ. И большинство людей это прекрасно
понимают. Просто им нужно объединиться и организоваться.

\iusr{Виталий Лукащук}
\textbf{Олег Дьяченко} а вы точно адекватный? А то, те, кто поддерживают нынешний политический режим на Украине, с его агрессивными комплексами и фобиями вызывают вопросы.

\iusr{Олег Дьяченко}
\textbf{Виталий Лукащук} Во первых не режим, а демократически избранную власть, во вторых у профессоров каких наук защита территориальной целостности страны в международно признанных границах называется агрессивными комплексами? Что скажете "проффесор"?

\iusr{Виталий Лукащук}
\textbf{Олег Дьяченко} для начала, постарайтесь научиться писать грамотно и, желательно, по сути. Не имею желания доказывать вам очевидные вещи. Всего хорошего.

\iusr{Victoria Moroz}
\textbf{Энрике Анатольевич Менендес} , это неправда, облсовет уже года три как вы можете выбирать.

\iusr{Николай Марков}

\ifcmt
  ig https://scontent-frx5-2.xx.fbcdn.net/v/t39.30808-6/258143838_4901258796572215_6172700717981482339_n.jpg?_nc_cat=109&ccb=1-5&_nc_sid=dbeb18&_nc_ohc=sF-Cvasrd-QAX9zkmtt&_nc_ht=scontent-frx5-2.xx&oh=111ef02812e35ab6e95b3910845b89d9&oe=619F08A7
  @width 0.4
\fi

\iusr{Олег Дьяченко}
\textbf{Виталий Лукащук} написано правильно , не переживайте за мою грамотность. В Украине это имя нарицательное. Именно такие проффесора как Виктор Федорович и вы , ему подобный, и разожгли костер войны . А с аргументами то у вас не густо...не тех наук профессор?

\iusr{Rostyslav Pelekhovych}
> миллионы беженцев от нищеты
Виталий, а чому бєжєнци лише там, куди прийшов рускій мір? Чому рускіє люді в Слов'янську живуть краще ніж в Донецьку ?

\iusr{Сергей Гайдаш}
\textbf{Rostyslav Pelekhovych} 

в Славянске по их домам не стреляет ВСУ. Когда были события в Славянске, город
покинули 70\% населения. Иногда очень странно отвечать на такие вопросы. Если по
вашему дому бьёт артиллерия, вы будете оставаться дома, или же всё-таки
покинете его?


\iusr{Олексій Романов}
\textbf{Max Nefyodov} О, вже "ворогів народу" знайшов. Респект, "правильной дорогой идете товарищи!"

\iusr{Павел Климов}
\textbf{Александр Лапин} 

Инакомыслящих преследовали при Януковиче? Это кого конкретно преследовали? Это
Тимошенко, которая подняла цену на газ со 149 до 500 баксов? Я лично помню как
свободовцы и Ляшко каждую пятницу у Шустера разоблачали вышки Бойко и
Межигорье, которое Ющенко вполне официально переписал на Януковича, за что при
Порошенко его тягали по судам. Все СМИ во время Майдана были против Януковича,
однако ни одно из них не было закрыто. Судя по тому, что Бойко на свободе,
значит с вышками не было никакого криминала.

\url{https://www.bbc.com/ukrainian/news-russian-48540767}

\iusr{Iaroslav Gonchar}

Макс, да с ними бестолку общаться, к сожалению есть ещё йобнутые, по другому их
нельзя называть, и они будут, и с этим придётся жить. Слава Україні!) Пусть их
чуть подергает)))

\end{itemize} % }

\iusr{Олександр Михельсон}

ну, можно со многим поспорить. скажем, невозможно же не быть антизападником и в
то же время выступать за "нормализацию отношений с россией", ведь россия видит
условием "нормализации" именно антизападный курс. впрочем, со многим можно и
соглашаться. например, насчет перемен - в принципе  @igg{fbicon.smile} 

\begin{itemize} % {
\iusr{Энрике Менендес}
\textbf{Олександр Михельсон} 

С чего ты это взял, что РФ хочет видеть Украину антизападной? Я так понимаю,
что хотят видеть нейтральной или хотя бы не анти-российской

\iusr{Олександр Михельсон}
\textbf{Энрике Анатольевич Менендес} 

из практики взял. "вставание" с колен предсказуемо превратилось в
империализм-лайт, который в качестве raison d'etre имеет антизападный курс, и
совершенно ничего более. путин тыщу раз прямым текстом говорил, что россия
считает себя вправе руководить определенной частью света, куда украина входит
по умолчанию. какие на фиг нейтралитеты, не будем детьми.

\iusr{Petro Protsyk}
\textbf{Энрике Анатольевич Менендес}

В. Сурков: "Экспорт хаоса дело не новое. «Разделяй и властвуй» — древний
рецепт. Разделение — синоним хаотизации. Сплачивай своих+разобщай чужих=будешь
править и теми, и другими. Разрядка внутренней напряженности (которую Лев
Гумилев расплывчато называл пассионарностью) через внешнюю экспансию. Римляне
делали это. Все империи делают это. На протяжении веков Русское государство с
его суровым и малоподвижным политическим интерьером сохранялось исключительно
благодаря неустанному стремлению за собственные пределы. Оно давно разучилось,
а скорее всего, никогда и не умело выживать другими способами. Для России
постоянное расширение не просто одна из идей, а подлинный экзистенциал нашего
исторического бытия.

Имперские технологии эффективны и сегодня, когда империи переименованы в
сверхдержавы. Крымский консенсус — яркий пример консолидации общества за счет
хаотизации соседней страны. Жалобы Брюсселя и Вашингтона на вмешательство
Москвы, невозможность урегулирования значимых конфликтов по всему земному шару
без российского участия показывают, что наше государство не утратило имперских
инстинктов"

Подробности от АК:

\href{https://actualcomment.ru/kuda-delsya-khaos-raspakovka-stabilnosti-2111201336.html}{%
Куда делся хаос? Распаковка стабильности, Владислав Сурков, actualcomment.ru, 20.11.2021%
}

\iusr{Юрий Лосев}
\textbf{Энрике Менендес} +100500!!!

\iusr{Кирилл Лебедев}
\textbf{Petro Protsyk} 

Ну, во-первых, Сурков сейчас не занимает каких-либо постов. Он - частное лицо,
и может высказывать своё мнение. Это не означает, что политика государства
будет такой, как он написал. Можно даже предположить обратное - что по этой
причине он и вынужден был уйти с госслужбы, т.к. его точку зрения не разделяли
те, кто остался во власти. А, во-вторых, про крымский консенсус он
пофантазировал. Возврат Крыма на какое-то время объединил и сплотил людей по
той причине, что это был акт справедливости, с точки зрения россиян. И он
случился после вопиющий несправедливости, допущенной Ельциным при распаде СССР
- когда Крым оставили Украине. Украина не имеет никакого отношения к Крыму, и
он просто остался ей на шару, из-за нежелания Ельцина конфликтовать в самом
начале своего правления. И в этой связи я не понимаю, к какому консенсусу может
привести возврат Львова или Тернополя? Они никому не сдались тут в России.
Одессу, Николаев - ещё можно попытаться вернуть. Но к этим регионам россияне
относятся не так трепетно, как к Крыму, да и в Одессе противников интеграции
будет больше, чем в Крыму. Сейчас в российском обществе совсем другие запросы -
развитие внутренней инфраструктуры, экономический рост, рост благосостояния.
Присоединение Львова или Одессы эти запросы не решит.

\iusr{Petro Protsyk}
\textbf{Kirill Lebedev} 

"Великий договір про дружбу....". Ґельсинські угоди.... А ще - Україна явно має
до Криму більше відношення, ніж московія

\iusr{Petro Protsyk}

\ifcmt
  ig https://scontent-frt3-2.xx.fbcdn.net/v/t39.30808-6/259056628_10224435708631750_5349766595141240876_n.jpg?_nc_cat=103&ccb=1-5&_nc_sid=dbeb18&_nc_ohc=3m4AnY1Ct0oAX_0oDZu&_nc_ht=scontent-frt3-2.xx&oh=b7afa2103deefe868d35972113044658&oe=619F691E
  @width 0.4
\fi


\iusr{Petro Protsyk}
\textbf{Kirill Lebedev} як пан не пам'ятає, то путін якраз напав на "нейтральну" країну
\iusr{Petro Protsyk}
\textbf{Kirill Lebedev} що, це таке правило?

\iusr{Вадим Щетинин}
\textbf{Энрике Менендес} и с какого перепугу рашедерьму потокать и и дти тем курсом, которым им хочется?

\iusr{Michail Vidiassov}
\textbf{Энрике Менендес} 

а разве обострение не началось с того, что "богатовекторный" Янукович оказался
"пророссийским"? В условиях курса Запада на интеграцию Украины позиция
нейтралитета оказалась антизападной, теперь же, когда этот курс уже
"статус-кво", и подавно. РФ не единственный внешний игрок и не её зона контроля
расширяется.


\iusr{Олег Алексеев}
\textbf{Олександр Михельсон} 

Для России условие - не антизападный курс, а просто не антироссийский. Это
вообще мелочь в плане реализации, но увы без этого Запад не поддерживает и не
даёт безвизы

\iusr{Artem Duvanov}
\textbf{Энрике Менендес} 

с точки зрения России модель идеального соседа это Финляндия. Страна, очевидно,
не то что прозападная, а просто западная. Но тем не менее, они тщательно
избегают быть антироссийскими и много сил тратят на поддержание хороших
отношений с РФ. Хотя ведь были в истории отношений и очень сложные периоды.
Если бы Украина могла стать такой страной, было бы идеально.

\iusr{Олександр Михельсон}
\textbf{Олег Алексеев} 

а что они считают "антироссийским"? позволю себе напомнить, что блокада
украинского импорта в россию началась уже летом 2013-го, причем россияне не
скрывали, что это - инструмент давления на "дружественную", заметим, страну,
чтобы заставить ее отказаться от торговой ассоциации с ес. которая россии ничем
не угрожала, да и вообще, что это за способ выстраивания отношений с
"дружественным" государством? я уж молчу о повальном вранье российских
пропагандистов о жутких ужасах, которые якобы прописаны в соглашении об
ассоциации, каковое, мол, скрывают от украинцев (это в то время, как полный
текст висел на сайте верховной рады!) увы, не способная сойти с рельс
"вставания с колен" россия не умеет дружить - только руки выкручивать  @igg{fbicon.frown} 

\iusr{Michail Vidiassov}
\textbf{Олександр Михельсон} 

насколько помню, речь была не об отказе, а о трехсторонних переговорах по
евроассоциации. Какие-то консультации даже были формально проведены, но уже
после победы Майдана, подписания Ассоциации, её ратификации и т.д. - ну чисто
посмеяться с РФ. Ну а про ограничения на импорт - так вот только что идея "no
deal Brexit", без согласования с ЕС, упиралась в угрозы ЕС ухудшить условия
торговли, недовольство в Северной Ирландии, которой особо не хотелось жесткой
границы с Ирландией и т.д.

\iusr{Rostyslav Pelekhovych}
> с точки зрения России модель идеального соседа это Финляндия
\textbf{Artem Duvanov} , а Фінляндія член ЄврАзЕС чи ТС? Бо Росія назвала антросійськими діями в 2013 саме відмову від євразійської економічної інтеграції на користь європейської. Так що не треба тут втирати про Фінляндію
це і на Ваше питання відповідь, \textbf{Энрике Анатольевич Менендес}, про те що насправді треба Росії

\iusr{Konstantin Isikov}
\textbf{Олександр Михельсон} "торговой ассоциации с ес. которая россии ничем не угрожала" Зачем же врать?
Ещё как угрожала! Хотя бы тем, что между Украиной и Россией существовала беспошлинная торговля, а ассоциация привела бы к тому, что поток европейских товаров лавиной полился через украинскую границу.
И как можно быть дружественным государством, к государству, которое ведёт себя не дружественно упрямо нарушая 6-й пункт Договора о дружбе, воруя газ, прославляя нацистских коллаборационистов, разжигая вокруг Голодомора искажает совместную историю?

\iusr{Олег Алексеев}
\textbf{Konstantin Isikov} не надо отнимать у них веру, им и так погано) а если ещё и правду узнают, ради чего их под пули на майдан гнали, вообще с катушек съедут

\iusr{Konstantin Tarasenko}
\textbf{Olexander Mykhelson} не думаю что автор топика ответит по существу о Вами упомянутых событиях и политики РФ уже в 2013 году. Я уж не говорю о рос.пропаганде еще далеко до Майдана, Крыма и войны на востоке.

\end{itemize} % }

\iusr{Михаил Альперович}
Как сочетается «захват Крыма Россией» и «нормализация отношений с Россией»?

\begin{itemize} % {
\iusr{Евгений Языков}
\textbf{Михаил Альперович} так же, как и предложение России продлить контракт на прокачку российского газа на 15 лет.

\iusr{Peter Bashkatov}
Нормально сочетается. Это ж Украина, страна победившего двоемыслия. Что нам г-н Менендес постоянно демонстрирует.

\iusr{Энрике Менендес}
\textbf{Михаил Альперович} Нормально сочетается. В мире взрослых так бывает

\iusr{Михаил Альперович}
\textbf{Энрике Менендес} в мире взрослых встречаются также случаи раздвоения личности, нет?:)

\iusr{Галина Куриленко}
\textbf{Михаил Альперович} Привести всё в соответствие с международным правом и двигаться дальше. А не пятую точку показывать друг другу десятилетиями.

\iusr{Михаил Альперович}
\textbf{Галина Куриленко} есть рецепт применительно к Крыму? Или к Донбассу? ...

\iusr{Михаил Кузнецов}
\textbf{Galina Kurilenko} Так приведите. Верните Украине Крым, и освободите Донбасс. Выплатите Украине компенсацию за ущерб. Публично принесите извинения. Потом добрососедские отношения.

\iusr{Alexey Lebedev}
Раздвоение личности, это когда требуем российский газ и воюем с Россией. Как это сосчитается?

\iusr{Михаил Альперович}
\textbf{Alexey Lebedev} И тоже плохо сочетается.

\iusr{Alexey Lebedev}

И Крым, это как раз следствие майдана и госпереворота. Вы России спасибо должны
сказать, что не далитвам совершить ещё более ужасные преступления против своего
народа.

\iusr{Энрике Менендес}
\textbf{Михаил Альперович} Ты врач? У меня положим, есть личный врач. И это не ты. И он говорит, что у меня точно нет раздвоения личности.

\iusr{Михаил Альперович}
\textbf{Alexey Lebedev} Ну да, ну да, если бы не Россия... старая песня.

\iusr{Alexey Lebedev}
Действительно старая песня и каждый раз вы убеждаетесь в правоте и каждый раз удивляетесь )).. аквариумные рыбки гуппи )

\iusr{Павел Климов}
\textbf{Михаил Альперович} 

Как сочетается мексиканский флаг с музеем истории в захваченном американцами
Техасе? Очень даже сочетается. Также, как украинский государственный язык в
российском Крыму. Разница только в том, что в Техасе не было референдума.

\ifcmt
  ig https://scontent-frt3-2.xx.fbcdn.net/v/t39.30808-6/259461910_4846935125351145_522728067364022928_n.jpg?_nc_cat=101&ccb=1-5&_nc_sid=dbeb18&_nc_ohc=OMTn9cbBQmIAX8kKcpC&_nc_ht=scontent-frt3-2.xx&oh=2378cf55c72569f0f11c395789f6dd1e&oe=61A01E54
  @width 0.4
\fi

\iusr{Михаил Альперович}
\textbf{Энрике Менендес} а почему ты оправдываешься? Я же не тебе ставил диагноз?
Есть мир взрослых. Есть мир детей. Есть люди с раздвоением личности. Есть люди без него.
Есть люди за нормализацию отношений с агрессором. Есть люди для которых это неприемлемо.
Что такого раздражительного для тебя в таком устройстве мира.

\iusr{Stepan Demin}
\textbf{Михаил Альперович} Россия забрала у брата-алкаша то что подарила, чтобы алкаш в ломбард не сдал

\iusr{Галина Куриленко}
\textbf{Stepan Demin} Брат снабжающий "алкаша" оружием, чтобы скорее застрелился и оставил наследство? Да, такое есть.

\iusr{Stepan Demin}
\textbf{Галина Куриленко} этот алкаш сам на чужом наследстве сидит, ничего своего нет, на всё готовое пришёл, такой алкаш давно выпрашивает эвтаназию, чтобы не мучился и других не раздражал

\iusr{Сергей Основин}
\textbf{Stepan Demin} так-то это даже в таком ключе выглядит мерзко.  @igg{fbicon.wink} 

\iusr{Михаил Кузнецов}
\textbf{Stepan Demin} Когда русский называет алкашом кого-то, это смешно.
\iusr{Вадим Щетинин}
Энрике, а я то думаю, ково Вы мне напомнили в маске - ужа на сковородке. И в ответе так-же.

\iusr{Вадим Щетинин}
\textbf{Энрике Менендес} это последствия ПТСР. (((

\iusr{Stepan Demin}
\textbf{Михаил Кузнецов} мог бы назвать наркоманом сути это не меняет, Украина фэйл стейт с инфантильными гражданами и недальновидным ворьём во главе

\iusr{Михаил Кузнецов}
\textbf{Stepan Demin} У тебя учётная запись фальшивая. Ты бот.
\iusr{Светлана Ивановна}
\textbf{Сергей Основин} а как верно!

\iusr{Michail Vidiassov}
\textbf{Михаил Кузнецов} посмотрите графики ожидаемой продолжительности жизни на Украине - Егор Кузьмич на несколько лет тренд на её снижение переломил

\iusr{Михаил Кузнецов}
\textbf{Michail Vidiassov} Вы сначала в своей России навeдите порядок. Потом смотрите "чётамyx\#xлов".

\iusr{Александръ Балагановъ}
А когда Украина агрессором Россию называет, тут ничего не смущает, Михаил Кузнецов?

\iusr{Andre Bo}
\textbf{Михаил Альперович} как и Техас и нормальные отношения Мексики и США!

\iusr{Олег Дьяченко}
\textbf{Stepan Demin} не нужно свою болезнь алкоголизм на других вешать , это не поможет , уважать больше не станут.

\iusr{Andre Bo}
\textbf{Михаил Кузнецов} ты когда перестал быть русским?

\iusr{Олег Дьяченко}
\textbf{Andre Bo} вы ещё Британию с Римом в пример приведите

\iusr{Лариса Макарова}
\textbf{Stepan Demin} Россия трезвенник? Это очень смешно.

\iusr{Павел Климов}
\textbf{Михаил Кузнецов} Боярышник на третьем месте по продажам в украинских аптеках после хлорида натрия (соли) и цитрамона. Тем временем в России запретили продавать Боярышник дешевле водки.

\ifcmt
  ig https://scontent-frt3-1.xx.fbcdn.net/v/t39.30808-6/259947566_4847230915321566_3325129405572492924_n.jpg?_nc_cat=107&ccb=1-5&_nc_sid=dbeb18&_nc_ohc=yv3H9tT_QJoAX-e6zPy&_nc_ht=scontent-frt3-1.xx&oh=b07f1f9ea81ae1858e86bc3681d3acab&oe=619FBBA5
  @width 0.4
\fi

\iusr{Александр Бучинский}
\textbf{Михаил Кузнецов} когда русский, судя по имени и фамилии, считает,что все русские алкаши,то это тоже смешно и вызывает в памяти античный анекдот или парадокс: Критянин сказал:Все критяне-лжецы

\iusr{Вадим Щетинин}
\textbf{Галина Куриленко} пани международник, просветите, что по Крыму и ордло и что это право глаголит?Потеште, плыз!

\iusr{Вадим Щетинин}
\textbf{Михаил Альперович} кто-то говорил, что оправдывается виновный. Хотя может это и оценочной суждение....

\iusr{Anton Vernigor}
\textbf{Энрике Анатольевич Менендес} про «нормально сочетается в мире взрослых» - есть противоречие. Может у кого-то «мир взрослых» это то же самое что «мир лицемерных продажных гнид», но не надо считать, что это нормально  @igg{fbicon.smile} 

\iusr{Іван Ложка}
\textbf{Энрике Анатольевич Менендес} в міре взрослих тобі б запхали швабру в дупу в Ростові. Чого ми тобі й бажаємо.

\end{itemize} % }

\iusr{Дмитрий Баевский}

Долго писать не буду, но есть вещи субъективные и объективные. Например, вы
против НАТО, а я сторонник вступления Украины в НАТО. Это два субъективных
мнения. Или вы считаете Януковича лучше последующих президентов, а я считаю его
худшим, что пока что случалось в истории Украины. Это тоже субъективные мнения.

Но есть объективные вещи - например, Россия готовила аннексию Крыма,
раскачивала Донбасс, в лагерях типа Валдая даже символику показывала еще до
2013. А еще мы живем лучше, чем до 2013 - несмотря на не лучшие власти,
несмотря на привнесенную Россией войну и коррупцию. Это тоже объективная вещь,
измеряемая численно - зарплаты, пенсии, они же относительно цен, объем
приобретения товаров и услуг в стране, количество поездок зарубеж, состояние
автодорог и т.д., и т.п.

Есть еще вещи, которые сложно измерить. Поэтому о них я и не писал, хотя они
важнее. Эти 8 лет - очень сложное время. Очень тяжелое. Но мы идем по
правильной дороге.

И еще. Концовка поста противоречит всему, написанному перед этим.

\begin{itemize} % {
\iusr{Энрике Менендес}
\textbf{Дмитрий Баевский} "Лагерь типа Валдая" - Вы наверное Селигер имели ввиду. Валдай это чуть про другое

\iusr{Юрий Лосев}
\textbf{Дмитрий Баевский} Янукович, вне зависимости от его оценки, единственный президент, избранный народом Украины дважды!

\iusr{Кирилл Лебедев}
\textbf{Дмитрий Баевский} Про качество дорог насмешили.

\iusr{Кирилл Лебедев}
\textbf{Дмитрий Баевский} Да и про уровень жизни - тоже. ВВП Украины упал и так и не дотянулся до ВВП времён Януковича.

\iusr{Александр Малюк}
\textbf{Юрий Лосев} Дважды? Четырежды! Умиляют эти "эксперты" из РФ.  @igg{fbicon.smile} 

\iusr{Rostyslav Pelekhovych}
> единственный дважды избирался
Данілович в цей момент поперхнувся

\iusr{Rostyslav Pelekhovych}
\textbf{Kirill Lebedev}

\ifcmt
  ig https://scontent-frt3-1.xx.fbcdn.net/v/t39.30808-6/259976354_4891711860862859_4508834130878637420_n.jpg?_nc_cat=106&ccb=1-5&_nc_sid=dbeb18&_nc_ohc=6waM1AoGkEwAX9Aud_X&_nc_ht=scontent-frt3-1.xx&oh=57aa84386a62cfb5ce6cf494cbb922eb&oe=619F2A77
  @width 0.4
\fi

\iusr{Анжелика Василевская}
\textbf{Кирилл Лебедев} это он мечтает.

\iusr{Татьяна Шноль}
\textbf{Дмитрий Баевский} , Украина сама готовила аннексию Крыма. В Севастополе Украина присутствовала номинально, типа, никуда он уже не денется, а Россия никогда не забывала Севастополь , с него все и началось и закончилось....

\iusr{Денис Зимин}
\textbf{Дмитрий Баевский} вы действительно думаете, что сейчас пенсии, степендии и прочие социальные выплаты, выше чем в 2013?

\iusr{Николай Лиховид}
\textbf{Rostyslav Pelekhovych} 

Ну да, как раз с 16 года начался массовый выезд к соседям с запада на
заработки. Кстати, в массе люди с Западной Украины неожиданно оказываются
гораздо адекватнее в общении с "москалями", чем жители других регионов Украины.
Думаю это оттого, что у них была реалистичная цель - получить свою "Румынию" с
возможностью ездить на подработки к богатым соседям. Они её на майдане добились
и теперь у них в душе мир и порядок ). На остальных территориях сложная каша в
большой долей советских и имперских "пережитков", задрапированных "шведским"
флагом и, местами, вышиванками. И, соответственно, хронический когнитивный
диссонанс, с вытекающими последствиями для психики. Но всё-равно, имхо, идея
"румынской" "элиты" заглотить и переварить Юго-Восток выглядит авантюрой,
которая уже принесла много бед и может кончиться совсем плохо.

\iusr{Вадим Щетинин}
\textbf{Кирилл Лебедев} 

По случаю годовщины начала Майдана в ближайшее время будет много спекуляций.
Члены недобитой шайки Януковича будут ныть о том, каким хорошим руководителем
он был, и как много потеряла Украина после его бегства из страны. Поэтому
сегодня самое время вспомнить о том, как на самом деле управляли страной
Янукович и Азаров.

Недавно я специально сделал видео, в котором подробно разобрал экономические
предпосылки Майдана. О них у нас говорят обычно куда меньше, чем об
идеологической стороне. Особенно не любят эту тему экс-регионалы. Ведь даже
поверхностный анализ экономических показателей Украины 2011-2013 гг показывает,
что Янукович и Азаров разорили страну и поставили ее на грань банкротства.

Падение ВВП Украины началось со второго квартала 2012 года. За год до Майдана.
Хотя в то время не было ни войны, ни оккупации, ни разрыва экономических связей
с РФ. А золотовалютные резервы с 2012 до начала 2014 года сократились в 2 раза
- с 31 до 15 млрд долларов.

К 2014 году резервы Украины иссякли, и поддерживать иллюзию стабильности больше
было нечем. Стране грозило банкротство (что признавал Азаров в своих интервью).
Пришлось в срочном порядке бежать за деньгами к Путину. А тот согласен был их
давать только в обмен на изменение внешнеполитического курса.

Не будь Азаров и Янукович такими бездарными управленцами, им бы не пришлось
торговать суверенитетом и европейским курсом Украины. Следовательно, не было бы
и причин для Майдана.

Каждому жулику, который лжёт об эффективном управленце Януковиче» нужно просто
показывать беспристрастные цифры статистики.

\url{https://youtu.be/tHok48eVegs}

\iusr{Konstantin Isikov}
\textbf{Дмитрий Баевский} 

Но есть объективные вещи - например, США готовила аннексию Крыма, и выдворение
оттуда российской базы, а Россия естественно готовилась опередить США. США с
90-х раскачивали Украину как Антироссию, расчитывая в будущем стравить их в
братоубийственной войне.

\iusr{Michail Vidiassov}
\textbf{Rostyslav Pelekhovych} корректнее было бы показать график в постоянных ценах

\iusr{Rostyslav Pelekhovych}
> корректнее было бы показать график в постоянных ценах
чому, Michail Vidiassov ? нам же цікаі не цифри, а можливості ці "цифри" використати, чи не так?

\iusr{Sviatoslav Grigorhuk}
\textbf{Kirill Lebedev} ВВП.
2013, млрд. \$
@igg{fbicon.flag.ukraina} - 183
@igg{fbicon.flag.rossia} - 2300
2021, млрд.\$:
 @igg{fbicon.flag.ukraina}- 181 (99\% від 2013 року)
 @igg{fbicon.flag.rossia} - 1700 ( 73\% від 2013 року)
 @igg{fbicon.shrug} 

\iusr{Кирилл Лебедев}
\textbf{Sviatoslav Grigorhuk}

\ifcmt
  ig https://scontent-frx5-1.xx.fbcdn.net/v/t39.30808-6/258219444_4724389867641097_7796712264402763938_n.jpg?_nc_cat=100&ccb=1-5&_nc_sid=dbeb18&_nc_ohc=7obEoZpWAJAAX8xkuoY&_nc_ht=scontent-frx5-1.xx&oh=2cef67053075c354d683c88244d61f11&oe=619E9FE1
  @width 0.4
\fi

\iusr{Orest Rybak}
\textbf{Konstantin Isikov} росія ніколи братом Україні не була, саме росія винна у смертях та убивствах. Росія і Україна то як Нацизм і євреї.

\iusr{Ольга Горлатая}
\textbf{Дмитрий Баевский} Может вы и живете теперь лучше, но большинство, гораздо хуже чем до майдана!

\end{itemize} % }

\iusr{Vladislav Maistrouk}
\textbf{Энрике Анатольевич Менендес}, 

если бы не Майдан, и если бы он был только танцы, то было бы как в Белоруссии.
Не дай бог такого любой стране.

Цена свободы и достоинства всегда высоки, кто не готов платить, тот живет в
оковах или как недостойный

\begin{itemize} % {
\iusr{Вячеслав Беленький}
\textbf{Vladislav Maistrouk}
Пытаетесь напугать украинцев низкими тарифами, дешёвым топливом, работающими заводами, отсутствием запретов на актёров и фильмы, свободой учить детей на родном языке?

\iusr{Vladislav Maistrouk}
\textbf{Vyacheslav Belenkiy} скажи спасибо россии, за все «хорошее», что есть сейчас в Донецке и Луганске. Ну и местным люмпенам. И учи украинский язык, и пусть твои дети учат. Или русскоязычные умственно отсталые и ещё один язык не в состоянии выучить, тем более что это язык страны где живешь?

\iusr{Вячеслав Беленький}
\textbf{Vladislav Maistrouk}
Мои дети уже осведомлены, что государство-деградант отобрало у них право, на их же родине, на образование на родном языке и пытается их загнать в культурно нищий мовный вольер.
Выводы делаются.
Вы не ответили на вопрос, чем пугаете сограждан.

\iusr{Vladislav Maistrouk}
\textbf{Vyacheslav Belenkiy} тамбовский волк твой сограждан.

\iusr{Vladislav Maistrouk}
\textbf{Vyacheslav Belenkiy} если тебе так противна Украина и ее культура, переезжай туда где тебе нравится.

\iusr{Вячеслав Беленький}
\textbf{Vladislav Maistrouk}
Хоть горшком называйте, если душу греет. Мне фиолетово.
Вы - не Украина. Вы только её больная майданная часть, захватившая власть над почти всей страной.

\iusr{Orest Rybak}
\textbf{Vyacheslav Belenkiy} ні, хворі то якраз такі як ви котрі віддаються перевагу диктатурі над свободою. Не забудьте їм розказати як ваш любий совок русифіковував та винищував все українське. От Росія, Білорусь і СССР ото держави деграданти які дійсно не давали жодних прав.

\iusr{Вячеслав Беленький}
\textbf{Orest Rybak} 

малолетний галичанин-гастарбайтер в Словакии, ни разу не бывавший ни в
Беларуси, ни в России, а про СССР знающий только по "напевам вятровича", нам
будет рассказывать про "вкус устриц", увиденных по телевизору?

Раз вмешались без приглашения - ответьте уж вы за автора ветки, так чем пугаете
украинцев, конкретно? можете и ботоферму своих лайков в помощь пригласить

\iusr{Orest Rybak}
\textbf{Vyacheslav Belenkiy} ахах, ото пішла Галичанофобія. Мені не треба бувати в Третьому Рейху щоб знати що то зло. Як і в СССР, Росії чи Білорусі. Росія і Білорусь, включно з їх попередником СССР то найогидніше що коли-небудь було на планеті Земля. У Вятровича не наспіви, а документи, докази, він вчений історик. Ахах, так то ви любителі залякувати українців, верещати москвофільські тези і фашизм. Ахах, кому що болить той на тому і палиться.

\iusr{Вячеслав Беленький}
\textbf{Orest Rybak} Я вырос в 40 км от Галичины и знаком с "предметом" не понаслышке, что позволило на лично опыте убедиться в верности оценки менталитета региона, данной П.Скоропадским еще сто лет назад.
При этом не отрицаю наличия толковых и здравомыслящих, но увы, в виде исключений.
Так вы собираетесь на вопрос отвечать или дальше будете демонстрировать свою профанацию "вятровичем-ученым"?

\iusr{Orest Rybak}
\textbf{Vyacheslav Belenkiy} ти знайомий з російською пропагандою. Скоропадський то така ж людина як усі, схильний помилятися. Я якраз відповів на всі питання, причім фактами і здоровим глуздом, від чого в тебе знатно бомбануло. Так гарно ти виписував брєд російського геббельс тв, а тут якийсь хлопець з Галичини всю малину обламав @igg{fbicon.wink} 

\iusr{Вячеслав Беленький}
\textbf{Orest Rybak} понятно, ответа не будет. попугай-радиоточка.
Щасты

\end{itemize} % }

\iusr{Gorobets Volodymyr}
Не хватает еще одной строки - Да я не не украинець. Разве нет?

\begin{itemize} % {
\iusr{Энрике Менендес}
\textbf{Gorobets Volodymyr} Я горжусь своим донбасским происхождением.

\iusr{Anton Vernigor}
\textbf{Gorobets Volodymyr} а это и без отдельной строки очевидно.
\end{itemize} % }

\iusr{Аня Степаненко}

Энрике, с вашей родословной , вы давно уже могли жить счастливо где-нибудь в
Барселоне или на Канарах... чего вы так мучаетесь в Украине?

\iusr{Энрике Менендес}
\textbf{Аня Степаненко} А где сказано, что я мучаюсь? Всё в моих руках и я не хочу покидать свою страну

\iusr{Александр Лапин}

А я поддерживал Евромайдан, пока он был мирным. Большой ошибкой Януковича и
Азарова было два года топить за евроинтеграцию, а потом свернуть её за
несколько дней, даже не потрудившись выяснить мнение людей об этом. В декабре я
думал, что теперь он затихнет, как в своё время сдулись «Налоговый Майдан» и
«ЮлеМайдан» - холодно, скоро новогоднее затишье, у студентов на носу сессия.
Потом, в январе-феврале был оглушён тем насилием и смертями людей, а весной у
нас в Донецке началось то же самое. Да и продолжается до сих пор. Если бы
тогдашняя ВР удосужилась хотя бы начать процедуру импичмента Януковича, то не
считал бы его смещение госпереворотом. А что касается векторов, то я -
западник, но принимающий во внимание интересы соседа с востока (хотя и
считающий его внешнюю политику неадекватной и агрессивной). Но Финляндии ведь
как-то удаётся сосуществовать с РФ после взаимных исторических обид и войны.
Думаю, что и нам в будущем это удастся. 

А любой немирный протест - это почти всегда трагедия, отбрасывающая страну в
развитии назад, независимо от успеха или подавления. Вон в Беларуси протесты
жестоко подавлены, но стало ли лучше от этого белорусам? И я сейчас даже не о
гражданских правах, а об экономике под санкциями и без эмигрировавших. Хотя
лично для меня гражданские права и возможность их мирно отстаивать важнее
холодильника. Поэтому воспринимаю Евромайдан как трагедию своей страны, но не
считаю, что её можно было бы избежать, подавив его...

\begin{itemize} % {
\iusr{Сергей Основин}
\textbf{Александр Лапин} никогда они не «топили за евроинтеграцию». Они просто блефовали в торговле с Путиным. И выиграли. Как они тогда понимали этот выигрыш.  @igg{fbicon.smile} 

\iusr{Александр Лапин}
\textbf{Сергей Основин} , 

ответственно заявляю, что подавляющее большинство украинских законов было
изменено под европейские директивы в 2011-2013 именно Януковичем и ПР.
Благодаря им потом Порошенко удалось пробить подписание Соглашения об
ассоциации с ЕС, потому что львиная доля «домашней работы» уже была сделана до
него. В частности, с 2012 года благодаря Виктору Фёдоровичу мы вынуждены
«блефовать» новыми на тот момент УПК и законом об адвокатуре...

\iusr{Alex Samorukov}
\textbf{Sergey Osnovin} все так. + была инерция со времени Ющенко

\iusr{Сергей Основин}
\textbf{Александр Лапин} это ничего ровным счетом не доказывает. Даже если это правда, в чем я сильно сомневаюсь.

\iusr{Сергей Основин}
\textbf{Александр Лапин} это ничего ровным счетом не доказывает. Даже если это правда, в чем я сильно сомневаюсь.

\iusr{Александр Лапин}
\textbf{Сергей Основин}, ну так погуглите даты принятия законов в рамках «дорожной карты» на сайте ВР, это же публично доступная информация, сами проверите.

\iusr{Сергей Основин}
\textbf{Александр Лапин} 

я написал «даже если это правда».  @igg{fbicon.wink}  чтобы выяснить, какая ВР больше приняла
проевропейских законов, надо тупо сидеть и считать. Долго и нудно, попутно
оценивая «техническое или гуманитарное» значение этих принятых законов. Я этим
заниматься не собираюсь. Кроме того, для блефа он должен был принимать законы.
Иначе никто бы не поверил.  @igg{fbicon.wink} 

\iusr{Александр Лапин}
\textbf{Сергей Основин}, 

вот по ссылке приказ Минюста, изданный на следующий день после инаугурации
Януковича об утверждении плана мероприятий по реализации Плана действий
ассоциации Украина-ЕС. Его никто не отменил (чтобы вы не говорили, что это
подгадили «попередники»), и он был выполнен при Януковиче, хотя и не на сто
процентов. Хотите поспорить с человеком, который в этом разбирается, сначала
потрудитесь изучить матчасть  @igg{fbicon.wink} 

\url{https://zakon.rada.gov.ua/laws/show/v0151323-10#Text}

\iusr{Сергей Основин}
\textbf{Александр Лапин} я не понимаю, что именно вы хотите мне доказать, что Янукович ПРАВДА хотел подписать соглашение с Европой ? И в доказательство вы мне вот это ???  @igg{fbicon.face.flushed}  @igg{fbicon.face.smiling.eyes.smiling} 

\iusr{Сергей Основин}
\textbf{Александр Лапин} послушайте, вы можете разбираться в чем угодно, но когда вы говорите «Янукович хотел...» - у вас нет этому доказательств. Никаких. А у меня - есть. Янукович НЕ ПОДПИСАЛ СОГЛАШЕНИЕ. А значит - не хотел.  @igg{fbicon.wink} 

\iusr{Александр Лапин}
\textbf{Сергей Основин} , до определённого момента хотел. Потом, когда понял, что больше получит от РФ, резко перестал хотеть. Да только поздно было, весь 2012 год по «Евро-2012» по телеку крутили социальную рекламу с подтекстом о нашем сколом европейском будущем. Вот и докрутились. Ему надо было не самому отказываться от подписания в Вильнюсе, а сыграть тоньше, чтобы ЕС сам перенёс подписания из-за невыполнения нами «дорожной карты» на 100 процентов. Но что уж теперь говорить...

\iusr{Сергей Основин}
\textbf{Александр Лапин} ещё раз. все это ваши фантазии против моих. Результат подтверждает мою версию. Янукович отказался подписывать и получил от Путина кучу миллиардов денег.  @igg{fbicon.wink} 

\iusr{Александр Лапин}
\textbf{Сергей Основин} , ещё раз, хотите спорить со мной в тех вещах, в которых я разбираюсь, предъявляйте доказательства своим словам, как это делаю я. А тратить своё воскресенье на болтологию мне неинтересно.

\iusr{Сергей Основин}
\textbf{Александр Лапин} смотрите, вы же по сути со мной согласились.  @igg{fbicon.wink} 
«Потом, когда понял, что больше получит от РФ...»
Вы согласны, что он не в Европу хотел, а денег побольше , ведь так ?  @igg{fbicon.wink} 

\iusr{Александр Лапин}
\textbf{Сергей Основин} , 

сложно сказать, хотел ли только для себя лично, или и для страны тоже. Чтобы
закрыть эту тему, вот сюжет российского «Первого канала» о саммите в Вильнюсе,
в сюжете есть прямое включение Януковича. Можете ещё поднять другие видео его
речей того периода. Он нигде не говорит, что он в принципе против ассоциации с
ЕС, просто торгуется об условиях.
\url{https://static.1tv.ru/eump/embeds/public_vod.html?v=247284:1}

\iusr{Сергей Основин}
\textbf{Александр Лапин} но это даже не так важно, для себя или для страны тоже. В любом случае, речь идёт о деньгах. А Европа - это не про деньги. Точнее, не так про деньги, как про гуманитарные ценности.
И да, он «торговался об условиях». Просто я думаю, что торговался он с Путиным. С самого начала.

\iusr{Денис Зимин}
\textbf{Александр Лапин} а вм не казалось странным, что Янукович якобы должен был проводить проевропейскую политику, с учётом того, что его основной электорат, был за мной вектор развития?

\iusr{Александр Лапин}
\textbf{Денис Зимин}, 

так то же самое сейчас происходит с Зеленским. Или вы считаете, что все 73
процента его избирателей проголосовали за «АрМоВир», продолжение конфликта в
ОРДЛО, тотальную украинизацию и продолжающееся переименование городов и улиц, в
том числе в честь коллаборантов с немецкими нацистами? Заказывает музыку тот,
кто платит...

\iusr{Александр Лапин}

Если бы Януковича купили ЕС и США, а не РФ, не было бы Евромайдана, и не факт,
что были бы «Л/ДНР», но наши «дорогие партнёры» пожадничали, а Путин - нет.
Хотя в итоге не получил того, за что платил...


\iusr{Денис Зимин}
\textbf{Александр Лапин} 

Зеленский показал, что у него нет намерений ломать тут систему, которая
сложилась при его предшественнике, да и сила традиции- каждый президент
проводит такую политику, что бы максимально не выполнить обещания данные своим
избирателям, - просто кинуть.

\iusr{Сергей Основин}
\textbf{Александр Лапин} 

73 процента там были не «за него», а просто против Порошенко. За него были те,
кто в 1м туре проголосовал за него. Там были разные группы людей, но в основном
это были люди инфантильные и те, что «по приколу» и «хуже не будет». А против
Порошенко они были в основном по причине того, что «он у них деньги украл».
@igg{fbicon.smile} 

В общем, там все сложно, но те, кто понимал, что Зеленский это реинкарнация
Януковича - тот за него не голосовал. Причём с обеих сторон полит.
спектра(снова таки, я про первый тур)

\iusr{Вадим Щетинин}
Александр, Вы что были на Майдане, а с какой группой?

\iusr{Сергей Основин}
\textbf{Александр Лапин} вот вы опять про «купили». И это снова совпадение наших позиций. Шёл торг на счёт цены продажи, а не о выборе полит. курса страны.

\iusr{Александр Лапин}
\textbf{Сергей Основин} , вы согласны, что те, кто поддерживали "АрМоВир" во втором туре проголосовали за Порошенко?

\iusr{Александр Лапин}
\textbf{Вадим Щетинин} , с группой туристов в декабре 2013, зашёл посмотреть. Не стоял там, естественно (вообще не люблю массовые мероприятия), но поддерживал своих коллег и знакомы, вышедших на площадь, пока это был мирный протест. Субъективно, Евромайдан выглядел приличнее и безопаснее, чем Антимайдан в Мариинском.

\iusr{Александр Лапин}
\textbf{Сергей Основин} , шёл торг обо всём, чем Янукович мог торговать. Ничего не изменилось особо и сейчас, просто нынешние покупатели с 2014 не дают выйти с предложением на свободный рынок, если понимаете, о чём я...

\iusr{Konstantin Isikov}
\textbf{Александр Лапин} Евроинтеграцию не свернули, а приостановили на опасениях, что она приведёт к потери промышленности, , после Майдана Яценюк сделал тоже самое, на тех же основаниях, но Майдана почему-то не случилось

\iusr{Александр Лапин}
\textbf{Konstantin Isikov} , Яценюк оттягивал подписание гораздо хитрее, руками наших "партнёров". Но и то, полностью спалил свой рейтинг...

\iusr{Konstantin Isikov}
\textbf{Александр Лапин} 8 лет уже прошло, и любому вменяемому человеку стало понятно, что курс на евроинтеграцию был разрушителен для Украины. Сотни предприятий были закрыты после разрыва экономических связей с Россией, а Европа не открыла, да и не собиралась открывать свои рынки.

\iusr{Александр Лапин}
\textbf{Konstantin Isikov} , ЕС-то открыл свой рынок, и квоты, возможно, были бы больше, будь наша продукция качественней и если бы она подходила под евростандарту. А так остаётся только торговать сырьём, и у тех, кто это делает (например, в ягодном бизнесе) до пандемии (которая ударила в принципе по бизнесу) всё было хорошо.

\iusr{Konstantin Isikov}
\textbf{Александр Лапин} Да враньё это всё и отмазки. Я в Мюнхене два года прожил, там любому чиновнику становится плохо от одной только мысли, что Украина может стать частью Евросоюза и граница ЕС будет проходить по Ростовской области.
Евросоюз настолько защищает свой рынок, что на одну капусту у них 2 тысячи пунктов описывающих стандарт её округлости, хрустящности и прочего бреда, призванного законно никого не пускать на свой рынок без их желания.

\iusr{Александр Лапин}
\textbf{Konstantin Isikov} , мой хороший товарищ работает (не в поле) в этом бизнесе (ягодном), и говорит, что полностью доволен возможностью торговать в ЕС напрямую без посредников. Его мнение для меня достовернее, извините.

\iusr{Michail Vidiassov}
\textbf{Александр Лапин} а зачем были нужны посредники до евроассоциации? Где-то попадалась статья про митинги и потасовки весны 2014 - там статистически, по данным машинной обработки СМИ, проверялась гипотеза о том, что АнтиМайданить склонны были сильнее там, где была несырьевая промышленность.

\iusr{Александр Лапин}
\textbf{Michail Vidiassov} , затем, что наши стандарты в сельхоз отрасли (и в других отраслях тоже) не соответствовали европейским, поэтому нужны были посредники из третьих стран, чьи сертификаты качества признавались в ЕС.

\iusr{Александр Лапин}
\textbf{Michail Vidiassov} , 

и да, логично, что если РФ закрыла рынок сбыта, чтобы наказать нас за
внешнеполитический выбор, а ЕС открыла свой рынок только той продукции, которая
соответствует их стандартам, те предприятия, которые десятилетиями
модернизировались их хозяевами, оказались в заднице. Только кто же в этом
виноват?

\iusr{Ольга Горлатая}
\textbf{Александр Лапин} 

И Азаров и Янукович из кожи вон лезли, чтобы доказать что хваленая
евроинтеграция на таких условиях не выгодна! Они хотели выторговать лучшие
условия! Но вам же хотелось всего и сразу.. Получили?

\end{itemize} % }

\iusr{Александр Малюк}
Не знаю, ПТСР это или нет, но явно какое-то расстройство психическое.

\begin{itemize} % {
\iusr{Энрике Менендес}
\textbf{Александр Малюк} Ты врач? У меня есть личный врач, который занимается моей психикой. И у меня всё в норме.

\iusr{Слава Коваль}
\textbf{Энрике Менендес} офигеть ....личный врач по психике...я флюрографию не могу сделать

\iusr{Peter Korobov}
\textbf{Слава Коваль} , что так? Дорого?


\iusr{Энрике Менендес}
\textbf{Слава Коваль} Здобули!

\iusr{Слава Коваль}
\textbf{Peter Korobov} Да... в совке заставляли...,а сейчас в Молдове надо до хрена платить

\iusr{Peter Korobov}
\textbf{Слава Коваль} , сочувствую.
Ну, зато не в "совке".
Не кашляйте там.
\end{itemize} % }

\iusr{Rostyslav Pelekhovych}
Вітаю з камінг-аутом, ріспект!

\begin{itemize} % {
\iusr{Энрике Менендес}
\textbf{Rostyslav Pelekhovych} А я скрывал?

\iusr{Rostyslav Pelekhovych}
по багатьом позицім , Энрике, ваша позиція була незрозуміла. тепер я бачу чітку програму анти-Майдану. так завжди простіше
\end{itemize} % }

\iusr{Венди Белова}
Вы абсолютно не понимаете, где находитесь. В какой точке развития истории.

\begin{itemize} % {
\iusr{Энрике Менендес}
\textbf{Венди Белова} Я всё прекрасно понимаю.

\iusr{Венди Белова}
\textbf{Энрике Менендес} нет.

\iusr{Anton Vernigor}
\textbf{Энрике Анатольевич Менендес} нет
\end{itemize} % }

\iusr{Татьяна Иванова}

Я только понять не могу, почему вы не в Донецке? Где по Ваше мнению Вас никто
не будет угнетать и где Ваш дом!

\begin{itemize} % {
\iusr{Olga Kosse}
\textbf{Татьяна Иванова}, а вы загуглите.

\iusr{Alex Radfox}
\textbf{Olga Kosse} Так отож.

\iusr{Татьяна Иванова}
\textbf{Olga Kosse}, нет необходимости - знаю, как облупленого.


\iusr{Энрике Менендес}
\textbf{Татьяна Иванова} Вы претендуете на знание, которым не обладаете. Насколько я помню - мы не знакомы.

\iusr{Татьяна Иванова}
\textbf{Энрике Анатольевич Менендес} , ще и память короткая. Не свойственно для молодого возраста.


\iusr{Энрике Менендес}
\textbf{Татьяна Иванова} И какой мой любимый фильм? )))))

\iusr{Olga Kosse}
\textbf{Татьяна Иванова}, а какой его любимый цвет? а любимая еда? Важно понять, что входит в критерий "облупленности"! А то такое смелое заявления.

\iusr{Галина Мудра}
\textbf{Татьяна Иванова} А в чому він не правий? З 1991 року мінус 15 000000 українців. 15 мільонів українців вимерли , втекли з України. Це трагедія. Повне винародовлення.
Невже Вам це влаштовує?
Про все інше мовчу.
Україна має бути НЕЙТРАЛЬНОЮ країною!!!

\end{itemize} % }

\iusr{Alex Samorukov}
(судя по "програмности" поста у одной из гусскомирных партий - пополнение)


\iusr{Галина Куриленко}
\textbf{Alex Samorukov} Вы хотели сказать - уткомирных партий?


\iusr{Олександр Мельник}

Не буду спорить с отдельными моментами (например, не думаю, что Зеленский хуже
Януковича. По крайней мере коррупции в Украине сейчас меньше и на повседневном
уровне с государством легче иметь дело). Но я понимаю, что это вопрос
субъективный и зависит от обстоятельств разных людей. Фундаментальная проблема
все таки в том, что Украина в границах 1991г. создавалась даже не СССР, а в
значительной степени сталинским режимом и в условиях фактической власти
олигархата продолжать сушествовать могла только с согласия РФ и при условии
наличия авторитарного режима, который бы подавлял и националистические и
демократические поползновения. В обшем-то тоже самое можно сказать и про
политический режим образовавшийся после Майдана, только в нем теперь нет
Донбасса и Крыма, зато есть долгосрочный конфликт с РФ.

\begin{itemize} % {
\iusr{Иван Шашлов}
\textbf{Олександр Мельник} коррупция выросла и перетекла вверх, хотя и по низам её тоже достаточно осталось. Но вверху она возросла неимоверно и приняла масштабы равно эпические и циничные. Собственно, не даром говорится о реализации латиноамериканской модели в Украине.
\end{itemize} % }

\iusr{Станислав Печенкин}
Типовой манніфест антизахідника - інтелектуала .

\iusr{Олександра Макарова}

Если бы это был госпереворот, то сейчас бы вас за подобный пост расстреляли.
Ну, или минимум вызвали бы на допрос в КГБ, как в у соседей.

Попробуйте что-то подобное в РФ написать в контактике.

\begin{itemize} % {
\iusr{Павел Фомичёв}
\textbf{Олександра Макарова} вы серьёзно думаете, что у нас на каждый пост выезжает чёрный воронок?

\iusr{Александр Лапин}
\textbf{Олександра Макарова} , так и у нас при Порошенко за свои убеждения сажали. Вспомните дело Коцабы. Сейчас, при Зеленском, Слава Богу, только с работы выгоняют, как Бильченко...

\iusr{Олексій Романов}
\textbf{Олександра Макарова} Бузину расстреляли

\iusr{Олександра Макарова}
\textbf{Олексій Романов} о да, конечно.

\iusr{Олександра Макарова}
\textbf{Павел Фомичёв} на этот бы выехали )
Жду что-то подобное у вас на странице  @igg{fbicon.wink} 

\iusr{Павел Фомичёв}
\textbf{Олександра Макарова} 

у нас костерят в сети Путина на чём свет все, кому не лень. Вся оппозиция сидит
в соцсетях и нещадно критикует государство. Это же не запрещено.

В посте Энрике я не вижу вообще ничего такого, что бы нарушало собственные
украинские законы.

\iusr{Олександра Макарова}
\textbf{Павел Фомичёв} 

заметьте, в своих комментариях я ни разу не упомянула ни РФ, ни Путина, а КГБ
можно также отнести к СБУ, как и к ФСБ. Но вы интуитивно поняли, о чем идет
речь. Как по мне, это уже красноречиво.

По сути, мне вообще фиолетово, что там происходит в России, потому что, в
отличие от Энрике, я вижу единственный путь нормализации отношений с русскими
(как минимум, при нынешней политической системе) - в построении стены и рва с
крокодилами.

Речь в моем первом комменте лишь о том, что существует непреодолимый
когнитивный диссонанс между утверждением, что произошел националистический
госпереворот, и самим фактом возможности писать подобные вещи.

О том, что через пять лет после госпереворота проходят выборы, на которых
действующая власть терпит оглушительное поражение и в тот же момент мирно
переходит в оппозицию, я вообще молчу.

Чтоб везде так "переворачивали".

\iusr{Andre Bo}
\textbf{Олександра Макарова}

\ifcmt
  @width 0.4

  ig https://scontent-frx5-1.xx.fbcdn.net/v/t39.30808-6/259544664_253178563543927_5320736589288037591_n.jpg?_nc_cat=100&ccb=1-5&_nc_sid=dbeb18&_nc_ohc=SDOsc0PFoFwAX8vcnmW&_nc_ht=scontent-frx5-1.xx&oh=08da92e2b2311ec67b7ca3493b20e852&oe=619F2746

	ig https://scontent-frx5-1.xx.fbcdn.net/v/t39.30808-6/259750560_253178946877222_2251102432029478850_n.jpg?_nc_cat=110&ccb=1-5&_nc_sid=dbeb18&_nc_ohc=cZQY4W2SlOsAX9ifEkT&_nc_ht=scontent-frx5-1.xx&oh=6eca6f5ec7c296534dfea226b47b91a6&oe=61A02FCF

\fi

\iusr{Олександра Макарова}
\textbf{Andre Bo} о, у вас, вижу, целая коллекция. Впечатляет ))
Только не в ту сторону со своим узелком ползете. Вам явно на восток.

\iusr{Andre Bo}
\textbf{Олександра Макарова} мы уже давно на востоке! Вот вам сюда, и только, это именно ваше!

\ifcmt
  ig https://scontent-frt3-1.xx.fbcdn.net/v/t39.30808-6/260032567_253183200210130_9180830519272705844_n.jpg?_nc_cat=104&ccb=1-5&_nc_sid=dbeb18&_nc_ohc=PChLEc4ZJMkAX_Kmzum&_nc_ht=scontent-frt3-1.xx&oh=1dccd322cd2e6ee5aed74bb5de7f1cae&oe=619F201C
  @width 0.4
\fi

\iusr{Alexey von Alexeyev}
\textbf{Олександра Макарова} вы так пишете, как будто мало инакомыслящих бы убито после Майдана за свои взгляды

\iusr{Alexey von Alexeyev}
\textbf{Олександра Макарова} и кстати, в РФ половина контактика совершенно безнаказанно критикует Путина и Россию
\end{itemize} % }

\iusr{Максим Ялі}

\obeycr
Не хватает:" Да, я не считаю, что, Россия на нас напала"
"Да, я считаю что путин красавчик"
"Да, я считаю, что у нас гражданская война "
"Да, я считаю, что путин имел право нападать на Украину и отбирать Крым".
Список, конечно, не полный.
\restorecr

\begin{itemize} % {
\iusr{Olga Kosse}
\textbf{Максим Ялі}, а вы точно политолог?)

\iusr{Kostyantyn Tsertsvadze}
\textbf{Olga Kosse} Не-а, он - прохфесар...)))))


\iusr{Энрике Менендес}
\textbf{Максим Ялі} Да, я считаю, что у нас гражданская война

\iusr{Максим Ялі}
\textbf{Энрике Анатольевич Менендес} 

да кто же сомневается. И Путин тоже ни причём. И защищает интересы
русскоязычных. И далее по списку. Даже не понимаю, как вы у меня в друзьях
оказались и я это проглядел. Ничего, исправлю сейчас. Кстаи, мне тут недавно
один фб-друг сказал, что вы паспорт украинский грозилились порвать или сжечь?
Если верно, то чего передумали? Российский с вашими взглядами получить не
проблема. Ума не приложу, за что вас на подвал посадили любители "русского
мира". Вы же в доску их.

\iusr{Konstantin Tarasenko}
\textbf{Энрике Анатольевич Менендес} а по 1му, 2му и четвертому пункту.... согласны?

\iusr{Александр Ковалев}
\textbf{Энрике Анатольевич Менендес}, 

а обстрелы из-за границы - это тоже гражданская война? А поставки оружия, гсм,
продовольствия, осуществление общего командования из Ростова - вот это вот все
насколько согласуется с вашими представлениями о гражданской войне?

Вдруг откуданивозьмись силы "ополченцев" в Новоазовске? Это все относится к
"гражданской войне"? Вопросы риторические.

\end{itemize} % }

\iusr{Serhiy Yaryhin}
І знов "Да, я хочу перемен."
А яких змін ви хочете? Зможете нарешті сформулювати? Чи знов, як і раніше лише гучні фрази?

\iusr{Антон Морозов}
Ти поц
© Подерв'янський

\iusr{Sergii Kostezh}

Самое смешное, что пишете вы это в жутком западном Киеве, где угнетают и варят
в котле русскоязычных младенцев, но не в Донецке, где полная нормализация з
Россией, антизападничество и тотальное уважение русского языка))) парадокс!

\begin{itemize} % {
\iusr{Olga Kosse}
\textbf{Sergii Kostezh}, так Энрике не в Киеве живет))

\iusr{Sergii Kostezh}
\textbf{Olga Kosse} а шо, в Луганске?

\iusr{Olga Kosse}
\textbf{Sergii Kostezh} и опять ошибка) ещё варианты будут?

\iusr{Кирилл Лебедев}
\textbf{Sergii Kostezh} Насколько я понимаю, Энрике не в Киеве, а в Краматорске. И потом - нет никакой заслуги в том, если вы повышаете уровень жизни Киева путём обстрела Донецка.

\iusr{Sergii Kostezh}
\textbf{Кирилл Лебедев} охохо, єто топ коммент))) я его заскриню пожалуй))))

\iusr{Кирилл Лебедев}
\textbf{Sergii Kostezh} Признак д...ны - скрин без причины.

\iusr{Sergii Kostezh}
\textbf{Кирилл Лебедев} да как уж без причины))) раньше было про Донбасс кормит Украину, а теперь - Киев кормиться обстрелами Донецка)) эволюция мысли прям!

\iusr{Кирилл Лебедев}
\textbf{Sergii Kostezh} раз уж Вы ответили нормально, поясню более развёрнуто. Модельно, вместо того, чтобы построить мост через Днепр, без которого Киев стоит в пробках, или хотя бы отремонтировать уставший мост, вы обстреливаете Донецк и потом ехидно спрашиваете - ну в Киеве же лучше? Ну так не обстреливайте Донецк - и там будет не хуже, чем в Киеве.

\iusr{Андрій Дейнеко}
\textbf{Sergii Kostezh} не кормите неутоленную жажду товарища из северной помойки учить других что им делать и как жить

\iusr{Алексей Зубов}
\textbf{Sergii Kostezh} , он сидит в «захваченном хунтой» Краматорске)) Куда вынужден был бежать из «освобождённого от Майдана» Донецка

\iusr{Igor Tkalich}
\textbf{Sergii Kostezh} самое смешное, что это в Донецке взрываются снаряды и это в Донецке, за прекращение боевых действий. А в Киеве - за "войну до победного конца"!

\end{itemize} % }

\iusr{Sergiy Gerasimenko}

Категорически не согласен со всеми утверждениями после слова "да" и решительно
сметаю их в сторону. Главное достижение - мы избежали участи России и Беларуси.
Мы избежали антропологической катастрофы, которая произошла на этих территориях
за последние 30 постсоветских лет. Да, Россия стратегически проиграла в 2014
году, но Украина пока не выиграла. Самое главное - наши векторы разошлись. Мы -
на правильном пути. И 1990, 2000, 2004 и 2014 года были важнейшими вехами этого
пути. Нам остался последний шаг, последний рывок.

\begin{itemize} % {
\iusr{Лариса Мерк}
\textbf{Sergiy Gerasimenko} куда? в пропасть?

\iusr{Alexandra Alexandra}
Последний рывок ... и ?? Заживем ??!

\iusr{Кирилл Лебедев}
\textbf{Sergiy Gerasimenko} Как можно считать, что страна находится на правильном пути, если половина населения разбежалась?

\iusr{Igor Tkalich}
\textbf{Sergiy Gerasimenko} дорогу осилит идущий! Правильность выбранного пути можно узнать только в конце этого пути! А пока что ВЫ просто "идете в направлении".
\end{itemize} % }

\iusr{Mykola Stasiuk}
\textbf{The Aspen Institute} ого які у вас люди з цінностями потрапляють на навчання @igg{fbicon.man.facepalming} 

\begin{itemize} % {
\iusr{Александр Лапин}
\textbf{Mykola Stasiuk} , в лучших традициях вышиватного «патриотизма» обязательно напишите им жалобу на Энрике, пусть отберут сертификат  @igg{fbicon.face.tears.of.joy} 

\iusr{Mykola Stasiuk}
\textbf{Олександр Лапін} проходь мимо, вата.

\iusr{Olga Kosse}
\textbf{Mykola Stasiuk}, тегните ещё и сбу, чтобы полноценный донос получился.))

\iusr{Александр Лапин}
\textbf{Mykola Stasiuk} , позвольте осведомиться, с каких пор мы с Вами перешли на «ты»  @igg{fbicon.wink} ?
\end{itemize} % }

\iusr{Serhiy Yaryhin}

З приводу "Да, я считаю, что восточные регионы страны проиграли больше и мнение
их жителей плохо представлено в политике страны." А чому східні регіони повинні
мати більший впливи на політику країни ніж будь який інший регіон?


\iusr{Daniel Tchikin}

Что ж, у такой точки зрения есть много сторонников. И сам факт того, что ты и
твои единомышленники можете выражать свою позицию открыто, говорит о том, что
власть, пришедшая после 2014-го, вполне демократична

\end{itemize} % }
