% vim: keymap=russian-jcukenwin
%%beginhead 
 
%%file 17_11_2020.news.ru.aif.1.covid_death_vjacheslav_chudimov
%%parent 17_11_2020
 
%%url https://altai.aif.ru/society/utrata_ot_covid_skonchalsya_zasluzhennyy_vrach_rossii_vyacheslav_chudimov
 
%%author 
%%author_id 
%%author_url 
 
%%tags covid_death
%%title Утрата. От covid скончался заслуженный врач России Вячеслав Чудимов
 
%%endhead 
 
\subsubsection{Утрата. От covid скончался заслуженный врач России Вячеслав Чудимов}
\label{sec:17_11_2020.news.ru.aif.1.covid_death_vjacheslav_chudimov}
\Purl{https://altai.aif.ru/society/utrata_ot_covid_skonchalsya_zasluzhennyy_vrach_rossii_vyacheslav_chudimov}

\ifcmt
  pic https://aif-s3.aif.ru/images/021/927/7047c36dd749b391dfa46a67cd135100.jpeg
  caption  Вячеслав Чудимов / личный архив 
  width 0.5
  fig_env wrapfigure
\fi

\index[deaths.rus]{Чудимов, Вячеслав!Заслуженный врач России, 68 лет, Барнаул, Россия, коронавирус, 17.11.2020}

В ночь на 17 ноября в возрасте 68 лет ушел из жизни \textbf{Вячеслав Чудимов -
невролог, рефлексотерапевт, врач лечебной физкультуры и спортивной медицины.}

Вячеслав Федорович 45 лет проработал врачом в Барнауле, 36 лет преподавал в
медицинском университете. В 2016 году Чудимов уехал жить в Сочи, так как
больницу, где он работал в Барнауле, ликвидировали, при этом известному врачу
ничего достойного не предложили. Супер-профессионала вскоре пригласили работать
в престижный оздоровительный центр у моря. В Сочи Вячеслав Федорович курировал
по УМО (углубленное медицинское обследование) детские сборные по художественной
гимнастике,  футболу и хоккею.

Год назад, когда Вячеслав Федорович приезжал на Алтай с коротким визитом, его
коллеги отправляли своих сложных пациентов к нему на консультацию. Пока, мол,
сам Чудимов здесь, пользуйтесь случаем. В эти дни пациенты выхватывали его даже
из посиделок с друзьями, и Чудимов никому не отказывал. Тогда мы  встречались с
ним  и разговаривали на тему дефицита врачей в больницах Барнаула. Чудимов
считал, что для того, чтобы врачи вернулись, нужно дать им преимущества:
повысить имидж, дать наиболее лучшие условия для работы, проживания, достойную
зарплату.

«Важно, чтобы человек был доволен, получал удовольствие», - подчеркнул тогда
заслуженный врач России.

Со своими коллегами и друзьями из Барнаула он переписывался до последнего. В
одном из недавних сообщений из больницы написал, что сатурация 96 и он
нагнетает оптимизм…

«Вячеслав Федорович был УЧИТЕЛЕМ для многих из нас, его подход к исцелению
пациентов был всегда индивидуальным и комплексным, – написал в одной из групп в
сетях Александр Газаматов, врач по медицинской реабилитации. - Умение сочетать
высокие достижения академической медицины и тысячелетние знания традиционной
медицины, натуропатии и натуротерапии приносили успех в лечении пациентов с
самыми тяжелыми заболеваниями. Этот опыт бесценен для нас. Светлая память
нашему врачевателю от Бога».

«Сложно передать мои чувства! – говорит журналист Ирина Заречнева, с которой
Чудимов работал над собственной книгой. - Вячеслав Федорович был для меня не
только учителем, но и хорошим собеседником, и даже другом. Его бесценные труды
собраны в книгу, которая ждет своего выпуска. Вячеслав Федорович искренне
надеялся, что его опыт будет служить коллегам на благо пациентов. Светлая
память и царствие небесное...»

\subsubsection{Из досье Вячеслава Чудимова}

\begin{itemize}
\item - tВ 1984 году за достойные успехи в развитии народного хозяйства награжден бронзовой медалью ВДНХ СССР.

\item - В 1985 году за цикл работ в области науки и техники присвоено звание «Лауреат премии Ленинского комсомола Алтая».

\item - В 1988 году защитил кандидатскую диссертацию в ученом совете ЦИУВ (Москва).

\item - В 1991 году присвоено ученое звание доцент.

\item - В 2002 году указом президента РФ присвоено почетное звание «Заслуженный врач Российской Федерации».

\item - С 1975 года - врач невролог медсанчасти работников текстильной промышленности и Алтайской краевой клинической больницы.

\item - С 1980 по 2016г. - Алтайский государственный медицинский университет, доцент кафедры ЛФК и факультета повышения квалификации, руководитель мастер-классов: мануальная терапия, массаж, лечебная физкультура, физиотерапия, рефлексотерапия, игло-, гирудо-, апи-, фитотерапия; эстетическая медицина: косметология (по лицу и телу, классическая и аппаратная), мезотерапия, пирсинг, таттуаж.

\item - Являлся главным внештатным специалистом по лечебной физкультуре и спортивной
медицине Главного управления по здравоохранению и фармацевтической деятельности
Администрации Алтайского края, проводил (18 лет) краевой консультативный прием
по вопросам комплексного реабилитационного лечения больных Алтайского края на
базе Алтайского краевого врачебно-физкультурного диспансера, а для больных г.
Барнаула специализированный вертеброневрологический прием на базе Детской
больницы № 5, где разработал и самостоятельно реализовал комплексную программу
восстановительного лечения вертеброневрологических больных и больных
недифференцированной соединительнотканной дисплазией.
\end{itemize}
