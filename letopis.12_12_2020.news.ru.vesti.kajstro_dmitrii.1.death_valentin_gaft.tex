% vim: keymap=russian-jcukenwin
%%beginhead 
 
%%file 12_12_2020.news.ru.vesti.kajstro_dmitrii.1.death_valentin_gaft
%%parent 12_12_2020
 
%%url https://www.vesti.ru/television/article/2497985
 
%%author Кайстро, Дмитрий
%%author_id kajstro_dmitrii
%%author_url 
 
%%tags gaft_valentin,death
%%title Гафт был народным артистом в самом полном смысле этого слова
 
%%endhead 
 
\subsection{Гафт был народным артистом в самом полном смысле этого слова}
\label{sec:12_12_2020.news.ru.vesti.kajstro_dmitrii.1.death_valentin_gaft}
\Purl{https://www.vesti.ru/television/article/2497985}
\ifcmt
	author_begin
   author_id kajstro_dmitrii
	author_end
\fi

На 86-м году от нас ушел Валентин Гафт – замечательный артист и человек.
Глубокие соболезнования супруге Гафта Ольге Михайловне Остроумовой выразил
президент Путин и отметил необыкновенный талант и уникальное чувство слова
Валентина Иосифовича, его актерский темперамент и обаяние. 

\index[names.rus]{Гафт, Валентин!12.12.2020, 86 лет}

\ifcmt
pic https://cdn-st1.rtr-vesti.ru/vh/pictures/xw/307/804/9.jpg
\fi

Гафт не фамилия, а диагноз, скажет однажды Григорий Горин. Особое состояние
организма, когда нервы обнажены и гонят через себя кровь, слова и мысли.

Далекий 1956-й. Его кинодебют. Небольшая, но яркая роль в драме Михаила Ромма
\enquote{Убийство на улице Данте}.

Тонкий, сложный, мудрый артист. С невероятно чутким отношение к мастерству.
Отсюда и знаменитое гафтовское \enquote{самоедство}.

\enquote{Я обыкновенный средний артист. Иногда мне кажется, что что-то получается}, –
говорил Валентин Гафт, народный артист РСФСР.

"Среди того, что "получилось", сам Гафт выделял роль раненого лейтенанта в
драме Петра Фоменко "На всю оставшуюся жизнь".

"Он лежал на полке госпитальной в вагоне. Произносил только одно слово –
"прелестно", – вспоминает Олег Басилашвили, народный артист СССР. – Несколько
лет войны, окопов. Все в одном слове – \enquote{прелестно}.

Он был народным артистом в самом полном смысле этого слова. Менялись время,
политика, поколения, а Гафт продолжал играть. Обаятельного и инфернального
товарища Сатанеева из новогодней сказки про чародеев. Криминального властителя
курортного города – в белоснежном костюме. Глубокого, яркого, одинокого.
Неотразимого с хрипотцой и грустными глазами циника Сидорина.

\enquote{Как он мне помог в 
\enquote{Гараже}, я вам передать не могу. Потому что те глаза,
которыми он на меня смотрел, – вспоминает Светлана Немоляева, народная артистка
РСФСР, – как он точно играл мужа в ее воображении. Я как сейчас это вижу}.

Он служил в разных театрах. Но главным театральным домом Валентина Гафта стал
\enquote{Современник}. \enquote{Я не могу сказать, что я избалован кино. Я вообще театральный
артист}, – отмечал Гафт.

\enquote{Он всегда был прямым, открытым и очень откровенным}, – подчеркнул Александр
Розенбаум, народный артист России.

Знаменитые едкие эпиграммы Гафта тоже стали символом его творческого признания.

\enquote{Эпиграммы родились совершенно случайно у нас в театре, когда я что-то такое
там пописывал, кого-то поздравлял. И Олег Николаевич Ефремов сказал, что это
все пропадает, пиши для капустников}, – говорил Гафт.

А еще он писал лирические стихи, нежные и беззащитные, в которых так
пронзительно сказал о времени, судьбе и прожитой жизни.
