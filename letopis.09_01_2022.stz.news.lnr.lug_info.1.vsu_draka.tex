% vim: keymap=russian-jcukenwin
%%beginhead 
 
%%file 09_01_2022.stz.news.lnr.lug_info.1.vsu_draka
%%parent 09_01_2022
 
%%url https://lug-info.com/news/zhitel-soledara-poluchil-pulevoe-ranenie-v-drake-s-p-yanymi-bojcami-vsu-narodnaya-miliciya
 
%%author_id news.lnr.lug_info
%%date 
 
%%tags donbass,vojna,vsu,ukraina,pjanstvo,draka,ranenie,nm_lnr
%%title Житель Соледара получил пулевое ранение в драке с пьяными бойцами ВСУ – Народная милиция
 
%%endhead 
\subsection{Житель Соледара получил пулевое ранение в драке с пьяными бойцами ВСУ – Народная милиция}
\label{sec:09_01_2022.stz.news.lnr.lug_info.1.vsu_draka}

\Purl{https://lug-info.com/news/zhitel-soledara-poluchil-pulevoe-ranenie-v-drake-s-p-yanymi-bojcami-vsu-narodnaya-miliciya}
\ifcmt
 author_begin
   author_id news.lnr.lug_info
 author_end
\fi

Драка с пьяными военнослужащими ВСУ закончилась для двух жителей Соледара
огнестрельным ранением и переломами ребер. Об этом на брифинге сообщил
официальный представитель Народной милиции ЛНР Иван Филипоненко.

\enquote{По данным наших источников, 7 января боевики 3-го батальона 24-й
бригады отмечали Рождество в ресторане \enquote{Аркадия} на улице Парковая
города Соледар.  Находясь в состоянии сильного алкогольного опьянения,
подрались с местными жителями, в результате чего один местный житель получил
огнестрельное ранение из пистолета и еще один доставлен в городскую больницу с
переломами ребер}, - рассказал официальный представитель.

Он отметил, что причастные к преступлению киевские силовики скрылись.  

\enquote{Командование ОТГ (оперативно-тактической группировки) \enquote{Север}
направило комиссию в бригаду для установления личностей, а командование
соединения предпринимает меры по сглаживанию конфликта на местном уровне}, -
добавил Филипоненко.

Власти Украины начали силовую операцию против Донбасса в апреле 2014 года.
Урегулирование конфликта базируется на Комплексе мер по выполнению Минских
соглашений, подписанном 12 февраля 2015 года в белорусской столице участниками
Контактной группы и согласованном с главами стран - участниц
\enquote{нормандской четверки} (Россия, Германия, Франция и Украина). Документ,
в частности, предусматривает прекращение огня и отвод тяжелых вооружений от
линии соприкосновения.
