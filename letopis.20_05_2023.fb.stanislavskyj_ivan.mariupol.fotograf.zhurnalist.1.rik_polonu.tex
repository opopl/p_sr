%%beginhead 
 
%%file 20_05_2023.fb.stanislavskyj_ivan.mariupol.fotograf.zhurnalist.1.rik_polonu
%%parent 20_05_2023
 
%%url https://www.facebook.com/I.Stanislavsky/posts/pfbid0em4UrwqWf2PmMzQp96z74X8mn1hmMuuSX4eHshb3mp9By9o4QFUu2E6Yj5U5cnmWl
 
%%author_id stanislavskyj_ivan.mariupol.fotograf.zhurnalist
%%date 20_05_2023
 
%%tags 
%%title РІК ПОЛОНУ
 
%%endhead 

\subsection{РІК ПОЛОНУ}
\label{sec:20_05_2023.fb.stanislavskyj_ivan.mariupol.fotograf.zhurnalist.1.rik_polonu}

\Purl{https://www.facebook.com/I.Stanislavsky/posts/pfbid0em4UrwqWf2PmMzQp96z74X8mn1hmMuuSX4eHshb3mp9By9o4QFUu2E6Yj5U5cnmWl}
\ifcmt
 author_begin
   author_id stanislavskyj_ivan.mariupol.fotograf.zhurnalist
 author_end
\fi

РІК ПОЛОНУ

Через рік після наказу про вихід з \enquote{Азовсталі}, на площі перед пам'ят\hyp{}ником
Кобзареві у Львові відбулась мирна акція.  

16 травня 2022 року українські військові, які захищали Маріуполь, отримали
наказ припинити оборону \enquote{Азовсталі} та скласти зброю. Гарнізон бився в оточенні
протягом 86 днів. 20 травня 2022 року останні захисники залишили територію
меткомбінату.

Досі у російському полоні перебувають понад 1900 військових-оборонців
Маріуполя, з них – 650 бійців полку \enquote{Азов}. За свідченнями тих, хто повернувся
з полону саме \enquote{азовці} піддаються найбільшим тортурам з боку росіян.

\enquote{В той момент коли ми тут думаємо чи взяти лате, чи еспресо там в полоні хтось
ховає хліб під футболку, щоб поділитися зі своїм побратимом який не встиг
пообідати, бо йому зуби вибили} - цитата однієї учасниці акції
