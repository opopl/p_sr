%%beginhead 
 
%%file 30_12_2022.fb.vdovenko_maksim.odesa.1.khersonshchina__den_
%%parent 30_12_2022
 
%%url https://www.facebook.com/snarkfog/posts/pfbid0265qhFMbjHGtxWJDwAnYjfLqLxGUaAPMEUbNBYTie1ynTTY5GQt7efd5t4udewR26l
 
%%author_id vdovenko_maksim.odesa
%%date 30_12_2022
 
%%tags herson
%%title Херсонщина. День четвертый
 
%%endhead 

\subsection{Херсонщина. День четвертый}
\label{sec:30_12_2022.fb.vdovenko_maksim.odesa.1.khersonshchina__den_}

\Purl{https://www.facebook.com/snarkfog/posts/pfbid0265qhFMbjHGtxWJDwAnYjfLqLxGUaAPMEUbNBYTie1ynTTY5GQt7efd5t4udewR26l}
\ifcmt
 author_begin
   author_id vdovenko_maksim.odesa
 author_end
\fi

Херсонщина. День четвертый.

Объезжаем села, находящиеся на бывшей линии фронта между Николаевом и Херсоном,
предлагаем помощь или эвакуацию оставшимся жителям. Искалеченная земля. Села
деляться на два типа: целое, но без электричества и прочих благ. И разрушенное
под ноль. Оба села могут находиться в 5 минутах езды друг от друга. Всё очень
просто: если село целое, значит, оно было сходу захвачено русскими. Наша армия
не обстреливала захваченные рашистами села, ведь это НАШИ села, где живут наши
люди. А если село превращено в руины... Думаю, вы поняли.

Посад-Покровское. Село разрушено до основания, ведь русским не удалось просто
захватить его, они были остановлены нашей армией. И они начали просто ровнять
его. Артиллерией, градами, даже авиацией. Везде валяются осколки от НАРов.
Местные (да, там всё еще живут люди) говорят, что их обстреливали с вертолетов.
В полях и дорогах торчат хвосты градов. Попадается сгоревшая русская техника.
Дороги в воронках от прилётов. Сходить с асфальта на обочину и тем более в поле
опасно. Могут быть мины и неразорвавшиеся боеприпасы. Осколки валяются повсюду.
Постапокалиптические пейзажи. Людей почти нет. Очень тягостно.

Соседнее село, Солдатское, не пострадало, так как было в оккупации.
Электричества нет с 8 марта. Общаемся с местными. От гуманитарки и эвакуации
они отказываются.

- Мы 8 месяцев прожили в оккупации, наконец-то мы снова в Украине, зачем нам
эвакуация? И к посевной нужно готовиться.

- А вы будете сеять?

- Конечно.

- В полях ведь полно неразорвавшихся боеприпасов.

- Мы ждем, когда наши сапёры почистят поля и будем сеять.

- А как к вам оккупанты относились?

- У нас тут кцпы стояли. Ходили тут с автоматами, но вели себя довольно
спокойно, в дома почти не вламывались. Вот в соседнем селе были буряты, там
село пострадало. Выносили из домов всё. Вы лучше им гуманитарку предложите, а у
нас всё что надо и так есть.

- А вы пробовали с ними как-то общаться?

- Разговаривал с ними. Но они как зомби. Повторяют слово в слово пропаганду
свою. Что защищают нас от бандеровцев, что всё будет россия, путин, вот это
всё. Я ему говорю - ты видишь здесь бандеровцев? Хоть одного? Он говорит - нет.
Так что ты здесь делаешь? А он по новой - путин сказал, мы освобождаем, всё
будет россия.

В другом населенном пункте встречаем местную жительницу. Рассказывает, что
наводила нашу артиллерию на вражеские колонны. Враг знал, что наводчик где-то
здесь и бил сюда, но она пряталась в убежище. Невероятная женщина.

На следующий день выезжаем домой в Одессу. Остальная команда остается работать
в Херсоне. А мы ограничены по времени, нужно возвращаться работе, курсам,
привычной жизни. Хотя бы на какое-то время. Вернуться окончательно скорее всего
не получится уже никогда.

По дороге узнаём, что под Бахмутом погиб наш друг. Молчим. У нас слёзы.

Возле Коблево пробуем подъехать к морю. Везде окопы, таблички "Мины!". Здесь
ждали высадку русского десанта и здесь же собирались его уничтожить. Десант так
и не явился, за исключением каких-то мелких групп еще в самом начале вторжения,
которые были тут же ликвидированы. 

Все таки находим возможность подойти к краю обрыва. Смотрим на наше море,
вспоминаем, как ездили сюда с палатками, купались ночью и сидели с гитарами у
костра, как было весело и беззаботно.

И мы всё это вернем. И наши дети будут сидеть у палаток на песке и петь песни,
будут купаться в ночном море и жарить мидий на костре. И будут знать о войне
только из наших рассказов, как мы знали о ней из рассказов наших дедов.

Потому что мы победим.

А пока обсуждаем возможность новой поездки. Нужно будет отвезти гуманитарку и
помочь с эвакуацией...

