%%beginhead 
 
%%file 28_02_2023.fb.suhorukova_nadia.mariupol.2.mikita
%%parent 28_02_2023
 
%%url https://www.facebook.com/permalink.php?story_fbid=pfbid02J5XwpTe9NTM8U9yhjiiwUPjoKTbiZXSGqVkz42h4gmZbgkxZyzfy7ebTow7c8nmYl&id=100087641497337
 
%%author_id suhorukova_nadia.mariupol
%%date 28_02_2023
 
%%tags mariupol,mariupol.war
%%title Микита
 
%%endhead 

\subsection{Микита}
\label{sec:28_02_2023.fb.suhorukova_nadia.mariupol.2.mikita}

\Purl{https://www.facebook.com/permalink.php?story_fbid=pfbid02J5XwpTe9NTM8U9yhjiiwUPjoKTbiZXSGqVkz42h4gmZbgkxZyzfy7ebTow7c8nmYl&id=100087641497337}
\ifcmt
 author_begin
   author_id suhorukova_nadia.mariupol
 author_end
\fi

Юлька пішла до пологового будинку в середу. 
І відразу ж зник зв'язок. 
Набирати її не мало сенсу, і ми з мамою вирішили відвідати Юлину маму, мою подругу. 
Завтра. 
Це було якесь безглузде слово.
Завтра могло й не настати. Але ми з мамою вперто планували наступний день.
Тоді ми ховались від обстрілів у тамбурі.  
Дві стіни трохи приглушували пекельні звуки. 
Десь біля ліфта кричали діти: вони не встигли спуститися на перший поверх і тому сховалися на четвертому поверсі. 
Я обіймала себе й хиталася на табуретці. 
Добре, що ніхто не бачив: у тамбурі хоч в око стрель. 
Темрява взагалі стала нашою постійною супутницею.
Темно було і вдень. 
Небо в  диму не пропускало сонця, а вікна ми заклали подушками та завісили ковдрами. 
Коли прийшли морози й випав сніг, виявилося, що нам нічим укриватися, а в квартирі було страшенно холодно. 
Довелося познімати ковдри.
Через кілька днів вибуховою хвилею вибило шибки на лоджії, а двері на балконі викривило так, що вони більше не зачинялися.
Мама весь час пропонувала кудись піти, когось провідати. 
Спочатку ми відвели  Енджі до знайомих.
Вигулювати її коло нас було небезпечно. 
Енджі не розуміла, що відбувається, і намагалась утекти якнайдалі від під'їзду, аби знайти чисту землю -  без смиття  та скла.
Вона піднімала лапи й дивилася на мене. Їй не було куди ступити.
Потім ми пішли провідати мою подругу в дев'ятиповерхівці поряд. 
Мамі було важливо дізнатися, як почувається її дочка і кого вона народила.
Юлька народила Микиту. 
У неї був кесарів розтин другого березня.
Четвертого вона пішла з пологового будинку. 
Шостого прийшла з немовлям у дім наших знайомих.
На її сьомому поверсі було занадто страшно.
Юлина сестра, мама, бабуся і крихітний Микита спали в коридорі на підлозі. 
Воду для суміші розігрівали на сухому пальному: маленький принц не міг їсти суміш, якщо її розбавляли водою, нагрітою на багатті.
У Юльки дуже болів шов. 
Ми обробляли його перекисом. Я казала їй: 
"Навіщо ти пішла з пологового будинку? Там же професіонали. Ти дуже рано виписалася"
Юлька спочатку відповідала, що хвилювалася за своїх, а потім сказала:
"Там надто великі вікна. Мені було дуже страшно"
За кілька днів, дев'ятого березня, рyzкий льотчик скинув на пологовий будинок авіабомбу. 
А там залишалися жінки, які не могли поїхати додому. 
Одна з них мешкала в селі під Маріуполем. 
Її навіть ніхто не відвідував. Вона народила дівчинку.
Медперсонал годував породіллю зі своїх запасів.
Крихітний Микита практично жив у підвалі. 
Іноді його виносили на поверхню, і він розглядав усе довкола винятково дорослим поглядом.
Микитку всі  любили. 
Жінки носили малого на руках і заколисували. 
Юлька зітхала: вона хотіла, щоб синочок був самостійнішим.
Я пам'ятаю його першу прогулянку.
З комбінезона стирчить лише ніс.
Малюк  вдихає повітря, що пахне порохом. 
Олена, моя подруга й бабуся Микити, тримає його на руках, немовля спить, і тут лунає скрегіт заліза, ніби хтось тягне величезний  каркас...
Ми знали, що одразу після цього стрілятимуть. 
Перша прогулянка Микити тривала п'ять хвилин.
Мабуть, він вважає, що народився у похмурому й немилосердному світі. 
Микита постійно був  в темряві.
У підвалі світить тільки  ліхтарик;
у кімнаті на першому поверсі, куди його приносять, щоб обробити пупок, теж напівтемно:
вікна затулені мішками з піском.
Тато Микити воює за Україну, тому ми називали малюка сином полку. 
Юлька іноді тікає дзвонити чоловікові зі сходів будинку №105. 
Коли з'являється рідкісний зв'язок. 
У цей час завжди обстріл. 
Ми вмовляємо Юльку нікуди не ходити, але вона дуже хоче почути голос чоловіка.
Микита, здається, передчуває обстріли.
За кілька хвилин до  нього  -   ридає немовлячим басом, а потім одразу починають стріляти. 
Микита нас попереджає.
Ми лягаємо на диван, накриваємо голови подушками і слухаємо гуркіт літака.
***
Цьому красеню через два дні виповниться рік.
Його чудова бабуся зуміла врятувати.
Вона вивезла Микиту та свою родину  із мертвого міста 15 березня.
Вони їхали під обстрілами.
Малюкові тоді  було менше двох тижнів.
