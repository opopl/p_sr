% vim: keymap=russian-jcukenwin
%%beginhead 
 
%%file 17_01_2022.stz.news.ua.radiosvoboda.1.ukraina_rik_foto
%%parent 17_01_2022
 
%%url https://www.radiosvoboda.org/a/foto-ukraina-2021/31656794.html
 
%%author_id nuzhnenko_sergіj
%%date 
 
%%tags 
%%title Рік у фотографіях. Україна
 
%%endhead 
\subsection{Рік у фотографіях. Україна}
\label{sec:17_01_2022.stz.news.ua.radiosvoboda.1.ukraina_rik_foto}

\Purl{https://www.radiosvoboda.org/a/foto-ukraina-2021/31656794.html}
\ifcmt
 author_begin
   author_id nuzhnenko_sergіj
 author_end
\fi

\begin{zznagolos}
2021 рік для України був багатим на важливі події. Боротьба з коронавірусом,
розпочалася вакцинація населення, Україна відсвяткувала 30-річчя Незалежності,
Росія нарощувала війська біля кордону, президент Володимир Зеленський
запроваджував санкції.

Проте низка викликів, які стояли перед владою та суспільством рік тому, все ще
актуальні.

Це історія 2021 року, розказана візуально – універсальною мовою фотографії.
\end{zznagolos}

\ifcmt
  globals
  @mlc 3
  dbg
\fi

% 1-2
\ii{17_01_2022.stz.news.ua.radiosvoboda.1.ukraina_rik_foto.pic.1}

% 3-4
\ii{17_01_2022.stz.news.ua.radiosvoboda.1.ukraina_rik_foto.pic.2}

% 5-6
\ii{17_01_2022.stz.news.ua.radiosvoboda.1.ukraina_rik_foto.pic.3}

% 7-8
\ii{17_01_2022.stz.news.ua.radiosvoboda.1.ukraina_rik_foto.pic.4}

\ii{17_01_2022.stz.news.ua.radiosvoboda.1.ukraina_rik_foto.pic.9_10}
\ii{17_01_2022.stz.news.ua.radiosvoboda.1.ukraina_rik_foto.pic.11_13}

\ifcmt
  globals
  @mlc 0
\fi
