% vim: keymap=russian-jcukenwin
%%beginhead 
 
%%file slova.kniga
%%parent slova
 
%%url 
 
%%author 
%%author_id 
%%author_url 
 
%%tags 
%%title 
 
%%endhead 
\chapter{Книга}

%%%cit
%%%cit_pic
%%%cit_text
Это мы разговариваем цитатами из \emph{Книг}, мультфильмов и кино, а в глубине души
сочувственно смотрим на тех, кто спрашивает: «А откуда это?».  Это для нас
работала целая индустрия детских киностудий, и лучшие актёры изощрялись в
умении петь, танцевать и изображать наших любимых героев – лишь бы мы росли
нормальными людьми!  Мы ещё умеем дружить и не прощаем предательства.  Мы ещё
помним, что любить человека – отнюдь не синоним слова «спать» с ним.  Мы
пережили развал страны, 90-е, парочку мировых кризисов и несколько воин
(особенно внутренних).  До нас, наконец, дошло, что наши родители делали для
нас всё, что могли. Даже если не делали ничего.  Мы теперь уже точно осознали,
что «круто» – это не клубы каждые выходные, не градусы в стакане, а здоровый
цвет лица, крепкий сон и дорогие люди рядом
%%%cit_comment
%%%cit_title
\citTitle{Советскому поколению повезло родиться до того, как у детей отобрали право выбора}, 
Елена Лукаш, strana.ua, 13.06.2021
%%%endcit


%%%cit
%%%cit_pic
%%%cit_text
Про найдавнішу \emph{рукописну книгу} періоду України-Русі в російських
джерелах, а також у перекладах інформації з російської в англомовних джерелах
можна прочитати наступне: «Остромирово Евангелие – памятник культуры мирового
значения – хранится в Санкт-Петербурге в Российской национальной библиотеке.
Эта \emph{рукописная книга}, созданная в XI в., занимает совершенно особое
место в ряду самых важных памятников культурного наследия, составляющих
бесценное достояние России.  ...Остромирово Евангелие действительно стоит у
истоков русской письменности и культуры». Це інформація з офіційного сайту
Російської національної бібліотеки і в ній представлено дві традиційні для
ідеології «русского мира» міфологеми. Спростуємо їх почергово
%%%cit_comment
%%%cit_title
\citTitle{Чому написане у Києві Остромирове Євангеліє – «достояние России»?},
Ірина Костенко; Ірина Халупа, radiosvoboda.org, 13.06.2021
%%%endcit

%%%cit
%%%cit_head
Сергей Жадан - Новые Книжки
%%%cit_pic
\ifcmt
  pic https://life.pravda.com.ua/images/doc/6/a/6a3bd1a-jadan-serhiy-4.jpg
	caption Збірка поезії Сергія Жадана \enquote{Псалом авіації}
\fi
%%%cit_text
Наприкінці весни видавництво Meridian Czernowitz порадувало прихильників
творчості Сергія Жадана одразу двома його поетичними \emph{Книжками}: новою збіркою
поезії \enquote{Псалом авіації} та перекладами вибраних віршів класика німецької
літератури Бертольта Брехта \enquote{Похвала діалектиці}.  Олег Поляков спеціально для
\enquote{Української правди. Життя} порозмовляв з автором про його стосунки з релігією;
з'ясував, що значать ландшафти і краєвиди в його віршах і в його житті;
подискутував про вік ліричного героя. А також розпитав, чому Сергій обрав для
перекладу вірші Брехта і чи вплинув поет, драматург і прозаїк Брехт на поета,
драматурга і прозаїка Жадана
%%%cit_comment
%%%cit_title
\citTitle{Письменник Сергій Жадан: Мені цікаво говорити про речі прості й очевидні}, 
Олег Поляков, life.pravda.com.ua, 13.06.2021
%%%endcit

