% vim: keymap=russian-jcukenwin
%%beginhead 
 
%%file 21_11_2021.fb.myhelson_oleksandr.1.gidnist_svoboda
%%parent 21_11_2021
 
%%url https://www.facebook.com/olexander.mykhelson/posts/4561242023989254
 
%%author_id myhelson_oleksandr
%%date 
 
%%tags dostoinstvo,maidan2,svoboda,ukraina
%%title Маймо Гідність – збережемо і Свободу
 
%%endhead 
 
\subsection{Маймо Гідність – збережемо і Свободу}
\label{sec:21_11_2021.fb.myhelson_oleksandr.1.gidnist_svoboda}
 
\Purl{https://www.facebook.com/olexander.mykhelson/posts/4561242023989254}
\ifcmt
 author_begin
   author_id myhelson_oleksandr
 author_end
\fi

Майдан – обидва Майдани – це ж не про те, як має бути в принципі. Це про
реакцію в кризових обставинах. Майдан – феномен рідкісний за означенням. 

І один із проявів цієї унікальності – єднання тих, хто за інших обставин на
одному гектарі не сів би. 

На обидва Майдани виходили ультраправі – й ліваки, голубі-рожеві – і вишивані
скрєпоносці, ті, хто зараз топить за повальне щеплення – і ті, хто вбачає у
ньому всесвітню змову. 

І, звичайно, ті, хто сьогодні вважає Зелю абсолютним злом – і ті, котрі
оголошують цих ворогами народу й агентами Путіна, Порошенка, Медведчука і
Ахметова в одному флаконі.

В екстремальні моменти історії так завжди буває. А потім усі починають
ділитися. Згадайте, як весело мочили одне одного ірландські патріоти після
здобуття незалежності. І досі поділені на чотири частини: за ставленням до
Лондона + за конфесіями, що далеко не завжди збігається. А ізраїльтяни? Досі
з’ясовують, хто кому більше єврей. І це ж іще цивілізовані країни.

У нас от тепер різко почали ділити пам’ять про Майдан. І спробуй же тільки
ляпнути шось тіпа «обніміться, брати мої»: громадськість не пробачить! З обох
боків  @igg{fbicon.smile}  То Шевченкові можна було, а тобі – зась, розіпнуть прям у білому
пальті  @igg{fbicon.smile} 

Но я шось вперто думаю, шо це мине.

А щодо власне дати, то сказав би так десь: маймо Гідність – збережемо і
Свободу.

\ii{21_11_2021.fb.myhelson_oleksandr.1.gidnist_svoboda.cmt}
