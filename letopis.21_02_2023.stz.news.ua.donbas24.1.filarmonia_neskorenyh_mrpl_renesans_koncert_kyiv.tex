% vim: keymap=russian-jcukenwin
%%beginhead 
 
%%file 21_02_2023.stz.news.ua.donbas24.1.filarmonia_neskorenyh_mrpl_renesans_koncert_kyiv
%%parent 21_02_2023
 
%%url https://donbas24.news/news/filarmoniya-neskorenix-mariupolskii-renesans-vistupiv-z-koncertom-u-kijevi-video
 
%%author_id demidko_olga.mariupol,news.ua.donbas24
%%date 
 
%%tags 
%%title "Філармонія нескорених" — Маріупольський "Ренесанс" виступив з концертом у Києві
 
%%endhead 
 
\subsection{\enquote{Філармонія нескорених} — Маріупольський \enquote{Ренесанс} виступив з концертом у Києві}
\label{sec:21_02_2023.stz.news.ua.donbas24.1.filarmonia_neskorenyh_mrpl_renesans_koncert_kyiv}
 
\Purl{https://donbas24.news/news/filarmoniya-neskorenix-mariupolskii-renesans-vistupiv-z-koncertom-u-kijevi-video}
\ifcmt
 author_begin
   author_id demidko_olga.mariupol,news.ua.donbas24
 author_end
\fi

20 лютого у День вшанування пам'яті Героїв Небесної Сотні у Національній
філармонії України відбувся концерт \href{https://donbas24.news/news/mariupolskii-renesans-onovlenim-skladom-vistupit-u-kijevi}{\emph{Камерного оркестру \enquote{РЕНЕСАНС}}}%
\footnote{Маріупольський \enquote{Ренесанс} оновленим складом виступить у Києві, Яна Іванова, donbas24.news, 19.02.2023, \par\url{https://donbas24.news/news/mariupolskii-renesans-onovlenim-skladom-vistupit-u-kijevi}}

Маріупольської камерної філармонії на чолі з Василем Крячком — Заслуженим
діячем мистецтв України, талановитим диригентом та директором Маріупольської
філармонії. Концерт став частиною авторського проєкту Національної філармонії
України \enquote{Філармонія нескорених}, який спрямований на підтримку музикантів з
окупованих територій.

\textbf{Читайте також:} \emph{Міцніші за залізо: як ЗСУ героїчно боронять Донеччину в умовах зими}%
\footnote{Міцніші за залізо: як ЗСУ героїчно боронять Донеччину в умовах зими, Яна Іванова, donbas24.news, 20.02.2023, \par%
\url{https://donbas24.news/news/micnisi-za-zalizo-yak-zsu-geroyicno-boronyat-doneccinu-v-umovax-zimi-foto}%
}

Після довгої перерви Маріупольський камерний оркестр \enquote{Ренесанс} вперше виступив
з оновленим складом. Директорка Департаменту культурно-громадського розвитку
Маріупольської міської ради Діана Трима наголосила, що це дуже важливий крок,
який потрібен для збереження Маріупольської філармонії в Україні.

\begin{leftbar}
\emph{\enquote{У січні було оголошено набір до складу Маріупольської камерної філармонії,
вже 20 лютого на сцені Національної філармонії України можна було
побачити всіх нових учасників, які пройшли конкурс. Звісно, серед
учасників є і маріупольці. Але, на жаль, не всі змогли долучитися.
Насправді, це єдина філармонія в Україні, яка фінансувалася містом,
вона досить молода, але вже має свою історію. Для нас дуже важливо, щоб
Маріуполь, попри всі складнощі та низку проблем, продовжував \enquote{звучати}}},
— розповіла Діана Трима.
\end{leftbar}

Відомо, що директор Маріупольської філармонії Василь Крячок у березні 2022 року
перетворив приміщення філармонії на прихисток. Він щодня ходив під обстрілами,
адже знав, що там чекають люди, яким потрібна допомога. Будь-яку гуманітарку
від окупантів Василь Михайлович відмовлявся отримувати. Диригент бачив,
наскільки зруйноване місто, його намагалися підштовхнути до співпраці, але він
не зрадив своїх цінностей. 20 лютого у Києві після окупації Маріуполя відбувся
перший концерт Камерного оркестру \enquote{РЕНЕСАНС} Маріупольської камерної
філармонії.

\begin{leftbar}
\emph{\enquote{Я дуже сподіваюся, що і в Києві ми знайдемо свого слухача. Дякуємо кожному,
хто прийшов нас підтримати у такий непростий час}}, — підкреслив Василь
Крячок.
\end{leftbar}

\textbf{Читайте також:} \emph{Постійні обстріли Часового Яру — як змінилося місто за рік війни}%
\footnote{Постійні обстріли Часового Яру — як змінилося місто за рік війни, Тетяна Веремєєва, donbas24.news, 17.02.2023, \par%
\url{https://donbas24.news/news/postiini-obstrili-casovogo-yaru-yak-zminilosya-misto-za-rik-viini-foto}%
}

\ii{21_02_2023.stz.news.ua.donbas24.1.filarmonia_neskorenyh_mrpl_renesans_koncert_kyiv.pic.1}

До концерту також долучився чудовий музикант, блискучий скрипаль, Заслужений
артист України Назарій Пилатюк, який вразив своєю винятковою майстерністю.
Перший концерт провели в День пам'яті Героїв Небесної Сотні. Загиблих вшанували
хвилиною мовчання. Після цього пролунав саундтрек до одного з найбільш
драматичнх фільмів про голокост — \enquote{Список Шиндлера}. У репертуарі \enquote{Ренесансу} є
і пісні гурту \enquote{Океан Ельзи} у класичній варіації. Загалом концертна програма
складалася з відомих творів Дж. Перголезі, Т. Віталі, Б. Фроляк, А. Вівальді,
Є. Петриченка.

\textbf{Читайте також:} \emph{Незламний Покровськ: як змінилося місто за рік війни}%
\footnote{Незламний Покровськ: як змінилося місто за рік війни, Тетяна Веремєєва, donbas24.news, 16.02.2023, \par%
\url{https://donbas24.news/news/nezlamnii-pokrovsk-yak-zminilosya-misto-za-rik-viini-foto}%
}

\href{https://archive.org/details/video.21_02_2023.donbas_novyny.koncert_kamern_orkestru_renesans_nac_filarmonia_ukrainy}{%
Відео: Концерт Камерного оркестру \enquote{РЕНЕСАНС} в Національній філармонії України, Донбас Новини, 21.02.2023%
}%
\footnote{\url{https://www.youtube.com/watch?v=_Z5ROJ8VPz0}} %
\footnote{Internet Archive: \url{https://archive.org/details/video.21_02_2023.donbas_novyny.koncert_kamern_orkestru_renesans_nac_filarmonia_ukrainy}}

\ifcmt
  ig https://i2.paste.pics/PRH8Y.png?trs=1142e84a8812893e619f828af22a1d084584f26ffb97dd2bb11c85495ee994c5
  @wrap center
  @width 0.9
\fi

Глядачі у залі не стримували своїх емоцій від концерту та аплодували за кожен виступ.

\begin{leftbar}
\emph{\enquote{Браво музикантам, браво маестро, те що ви робите, дуже важлива справа. У
вашому мистецтві продовжує жити Маріуполь}}, — наголосив маріуполець
Максим Миргородов.
\end{leftbar}

\begin{leftbar}
\emph{\enquote{Для маріупольців — це справжнє свято. Концерт вийшов незабутній. Я дуже вірю,
що вже незабаром ми зможемо почути цей талановитий оркестр і в нашому
звільненому Маріуполі}}, — зазначила Катерина Осипенко.
\end{leftbar}

\textbf{Читайте також:} \emph{Для маріупольців до Дня закоханих провели концерт у Києві}%
\footnote{Для маріупольців до Дня закоханих провели концерт у Києві, Ольга Демідко, donbas24.news, 14.02.2023, \par%
\url{https://donbas24.news/news/dlya-mariupolciv-do-dnya-zakoxanix-proveli-koncert-u-kijevi}%
}

\ii{21_02_2023.stz.news.ua.donbas24.1.filarmonia_neskorenyh_mrpl_renesans_koncert_kyiv.pic.2}

Наразі Маріупольський камерний оркестр \enquote{Ренесанс} виступатиме у Києві, але вже
найближчим часом представить свою програму і в Європі.

\begin{leftbar}
\emph{\enquote{Музика — це дуже гарний бартер для Європи, адже немає мовного бар'єру. Саме
тому зараз у розробці окремі проєкти, які дозволять представити
Маріуполь і за кордоном}}, — поділилася планами директорка Департаменту
культурно-громадського розвитку Маріупольської міської ради Діана
Трима.
\end{leftbar}

Раніше Донбас24 розповідав, про \href{https://donbas24.news/news/miznarodnii-den-ridnoyi-movi}{\emph{Міжнародний день рідної мови}}.%
\footnote{Міжнародний день рідної мови, Ольга Демідко, donbas24.news, 21.02.2023, \par\url{https://donbas24.news/news/miznarodnii-den-ridnoyi-movi}}

Ще більше новин та найактуальніша інформація про Донецьку та Луганську області
в нашому телеграм-каналі Донбас24.

Фото: з відкритих джерел

\ii{insert.author.demidko_olga}
%\ii{21_02_2023.stz.news.ua.donbas24.1.filarmonia_neskorenyh_mrpl_renesans_koncert_kyiv.txt}
