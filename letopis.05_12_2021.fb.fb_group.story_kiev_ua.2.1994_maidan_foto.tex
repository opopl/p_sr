% vim: keymap=russian-jcukenwin
%%beginhead 
 
%%file 05_12_2021.fb.fb_group.story_kiev_ua.2.1994_maidan_foto
%%parent 05_12_2021
 
%%url https://www.facebook.com/groups/story.kiev.ua/posts/1812364478960345
 
%%author_id fb_group.story_kiev_ua,fedjko_vladimir.kiev
%%date 
 
%%tags foto,kiev,maidan
%%title Липень 1994-го... З фотографами на Майдані у мене давня дружба
 
%%endhead 
 
\subsection{Липень 1994-го... З фотографами на Майдані у мене давня дружба}
\label{sec:05_12_2021.fb.fb_group.story_kiev_ua.2.1994_maidan_foto}
 
\Purl{https://www.facebook.com/groups/story.kiev.ua/posts/1812364478960345}
\ifcmt
 author_begin
   author_id fb_group.story_kiev_ua,fedjko_vladimir.kiev
 author_end
\fi

\begin{multicols}{2} % {

Липень 1994-го... З фотографами на Майдані у мене давня дружба, ще з середини
70-х. Сергій Орлов, Віктор Побединський, Валік Кармінський... Щоразу, коли їхав
по справах через Майдан, знаходив 20-30 хвилин для спілкування з друзями за
чашкою кави в «трубі». Розмови на фотографічні теми – нова техніка, перегляд
фотографій, розповіді про цікаві фотозйомки... З Віктором у нас спільна тема –
балет! Дружина Віктора – балерина театру опери і балету. Я ж постійно
фотографую в хореографічному училищі і театрі. 

\vspace{3.5cm}
\medskip
%\vspace{\baselineskip}

\setlength{\parindent}{0pt}

\ii{05_12_2021.fb.fb_group.story_kiev_ua.2.1994_maidan_foto.pic.1}
\ii{05_12_2021.fb.fb_group.story_kiev_ua.2.1994_maidan_foto.pic.1.cmt}

\ii{05_12_2021.fb.fb_group.story_kiev_ua.2.1994_maidan_foto.pic.2}
\ii{05_12_2021.fb.fb_group.story_kiev_ua.2.1994_maidan_foto.pic.2.cmt}
\end{multicols} % }

За кавою Сергій демонструє свій новий інструмент – \enquote{Поляроїд}, чудо фототехніки
ХХ століття! Клац!.. І через хвилину отримуєш кольорову фотографію! Попит на
фотографії великий і Сергій та ще дехто з фотографів на Майдані працюють
«Поляроїдами». Оскільки я не заробляв на майданній фотографії, то не запам’ятав
скільки коштувала фотографія для клієнтів. Про «Поляроїд» я багато читав, але
живцем побачити його не доводилося.  Після кави повертаємося на Майдан і Сергій
демонструє \enquote{чудо} в роботі... В розмові виясняється, що у нього таких
фотоапаратів два. Мені камера сподобалася і домовляємося про натуральний обмін:
мені «Поляроїд» з двома фотокомплектами, а Сергію імпортний фотоспалах.

\ii{05_12_2021.fb.fb_group.story_kiev_ua.2.1994_maidan_foto.pic.3}

(Іноземна фототехніка була у великому дефіциті, тому серед фотографів існував
своєрідний \enquote{чорний} ринок, на якому фотоапарати і оптика частково продавалися,
а частково обмінювалися на інші моделі з договірними доплатами.)

«Обмиваємо» обмін за кавою і їду додому, щоб показати придбання Надійці і
Юлі...

\ii{05_12_2021.fb.fb_group.story_kiev_ua.2.1994_maidan_foto.pic.4}

***

Мої дівчата готуються до фотосесії... А я беру свою маленьку подружку Маргариту
на руки і прошу Надійку нас сфотографувати. Марго демонструє характер – не хоче
сидіти на руках і фотографуватися. Я теж демонструю характер – міцно тримаю її
двома руками і ніжно притискую до себе. Клац!.. і ми вже в сімейній історії...
Відпускаю Марго... Вона обурено дивиться на мене, потім стрибає з колін і
моститься на своє місце – у кріслі. Знаючи крутий характер Марго, я
вибачаюся... Бо дівчинка може відплатити, наприклад, написати в капці. Причому,
в обидва. Таке вже бувало...

\ii{05_12_2021.fb.fb_group.story_kiev_ua.2.1994_maidan_foto.pic.5}

(Марго член сім’ї з 1990-го. Коли ми її взяли, то кицюлі був один місяць. Миле
створіння з великими очима і незалежним характером. Вона прожила довге життя –
23 роки і 16 днів. А коли померла, то я поховав її в лісі під великим дубом.
Там де ми постійно гуляли з Надійкою.)

\ii{05_12_2021.fb.fb_group.story_kiev_ua.2.1994_maidan_foto.pic.6}

Фотографую Надійку, Юлю, Надійку з Юлею... Вже не пам’ятаю, як ми
сфотографувалися втрьох... Здається у «чуда» був автоспуск.

***

Тиждень погрався новою іграшкою і виміняв її на «Minox» з доплатою... 

***

Сьогодні ми не уявляємо життя без смартфонів і цифрової фотографії. Хочеш
запам'ятати момент – нема проблем... Хочеш селфі – нема проблем... Тицьнув в
екран і готово! Тицьнув ще раз... і поділився фотографією з усім світом))

А всього чверть віку тому назад все було по-іншому.

***

\ii{05_12_2021.fb.fb_group.story_kiev_ua.2.1994_maidan_foto.cmt}
