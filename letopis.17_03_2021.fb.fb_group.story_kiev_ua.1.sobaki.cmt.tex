% vim: keymap=russian-jcukenwin
%%beginhead 
 
%%file 17_03_2021.fb.fb_group.story_kiev_ua.1.sobaki.cmt
%%parent 17_03_2021.fb.fb_group.story_kiev_ua.1.sobaki
 
%%url 
 
%%author_id 
%%date 
 
%%tags 
%%title 
 
%%endhead 
\subsubsection{Коментарі}

\begin{itemize} % {
\iusr{Любовь Линник}

\ifcmt
  ig https://scontent-frx5-2.xx.fbcdn.net/v/t39.1997-6/s168x128/12057035_446661978873394_934856931_n.png?_nc_cat=1&ccb=1-5&_nc_sid=ac3552&_nc_ohc=0hf2SHcaUE8AX8Qb5fz&_nc_ht=scontent-frx5-2.xx&oh=00_AT-HR-F3N6PTA24_OR_a_8bGMk4a9sErUozjB1Vin_gCuQ&oe=61BCDF02
  @width 0.1
\fi

\iusr{Irena Visochan}
Так трогательно! Спасибо! @igg{fbicon.hearts.two} 

\iusr{Наталья Писная}

Замечательный рассказ! Так сразу и я вспомнила о своих пуделях, тоже проживших
с нами десятилетие и так много приятных минут радости нам подаривших !

\iusr{Галина Кардаш}
Спасибо. Очень добрый рассказ собачника))

\iusr{Всеволод Шевчук}
Петрику ти пам'ятаєш як вони заплигуввли до мене на руки коли я приїжджав в гості?

\begin{itemize} % {
\iusr{Петр Кузьменко}
\textbf{Всеволод Шевчук} було!

\iusr{Всеволод Шевчук}
\textbf{Петр Кузьменко} чудові були час, а в подальшому будуть, ще кращі!
\end{itemize} % }

\iusr{Lesik Machynsky}

Уважаю серьезных собачников. Всегда с симпатией присматривался к безстрашным и
энергичным таксам. У самого были и маленькая дворняжка, в детстве, и лайка
Полкан, и бассет Патти-Кальмина фон Йоги-Амфитрион с дочерью, и овчарка
полукровка Малыш, гроза котов, и приблукавшийся гигантский бордосс Рекс...
Последнее время, один за другим мопсы. Абсолютно не собачье поведение,
психология и повадки.

\iusr{Ludmila Teslenko-ponomarenko}

Потрясающий рассказ! Спасибо, Петр! Я на собаку так и не решилась, но ангорская
белая кошка Соня прожила с нами 17 лет.

\iusr{Анна Невская}

\ifcmt
  ig https://scontent-frt3-1.xx.fbcdn.net/v/t39.1997-6/p480x480/28775943_155886991792975_937989508480106496_n.png?_nc_cat=102&ccb=1-5&_nc_sid=0572db&_nc_ohc=sICZT74X2XwAX8GD_n9&_nc_ht=scontent-frt3-1.xx&oh=00_AT-mtPSDwN_wklokoSDf7Y-Hmrc1Bmf4L1Vkjky5L_DzXg&oe=61BD3A9F
  @width 0.2
\fi

\iusr{Анна Невская}
Спасибо большое, Пётр за такой интересный рассказ о братьях наших меньших.

\iusr{Светлана Волошина}

моя самая первая такса была приобретена в УООРе. Это и неудивительно. В то
время таксу можно было только в охот.обществах приобрести. И Веру Андреевну
Масс помню, бывала у неё дома, в гостях, воднокомнатной квартире на первом
этаже на Печерске. И на притравчные станции в Бучу ездили, и на Турухановом
острове помню нору, тоже туда приезжали. Там моя такса взяла не по месту
лису..... Норный мастер расслабился, закурил аппетитно... осенний день,
солнышко, красота.... вот и потерял бдительность. \enquote{Содрали} с меня тогда за
лису аж 70 американских рублей...

\ifcmt
  ig https://scontent-frt3-1.xx.fbcdn.net/v/t1.6435-9/161674630_3646595815469130_4485721917423633934_n.jpg?_nc_cat=106&ccb=1-5&_nc_sid=dbeb18&_nc_ohc=QRwJ-S7bu_0AX-PXXq-&_nc_ht=scontent-frt3-1.xx&oh=00_AT_HoYVptvGialKSwKwzUhifyioD0cDw4XFKrC8s_stLAQ&oe=61DE8CED
  @width 0.5
\fi

\begin{itemize} % {
\iusr{Светлана Волошина}

вот ещё одно архивное фото.... норный мастер со своей воспитанницей..... Ныне
на этом месте - многоквартирные и частные дома, всё застроено.

\ifcmt
  ig https://scontent-frx5-1.xx.fbcdn.net/v/t1.6435-9/161621617_3646700368792008_2836057526281897760_n.jpg?_nc_cat=110&ccb=1-5&_nc_sid=dbeb18&_nc_ohc=WewJ5D3yMI8AX9PM7fJ&_nc_ht=scontent-frx5-1.xx&oh=00_AT__NwmIN0UAij_VyYV4S1jKajhccTkbP7g6alEV44HK4A&oe=61DCFF76
  @width 0.4
\fi

\end{itemize} % }

\iusr{Semyon Belenkiy}
С душой написано, спасибо

\ifcmt
  ig https://i2.paste.pics/d68b73f042bef16a625080d3dbe9cede.png
  @width 0.1
\fi

\iusr{Нелли Кузьменко}

\ifcmt
  ig https://scontent-frx5-2.xx.fbcdn.net/v/t39.1997-6/p240x240/851592_490565377723574_924545120_n.png?_nc_cat=1&ccb=1-5&_nc_sid=0572db&_nc_ohc=WC_AOwZHBXcAX_jVdba&tn=lCYVFeHcTIAFcAzi&_nc_ht=scontent-frx5-2.xx&oh=00_AT8JOn9wClGYOnm4_2IgV7ZiNwmH5mT-J8D1YcomSE_i5g&oe=61BD5E6F
  @width 0.2
\fi

\iusr{Раиса Карчевская}
Пётр!
Спасибо большое за потрясающий рассказ о Ваших собаках

\iusr{Дарья Кирносова}
Ужас. Не понимаю подобных развлечений.
Умиляют комменты про \enquote{трогательность}...
Притравочные станции и подобные натаскивания собак - живодерство в чистом виде.

\begin{itemize} % {
\iusr{Петр Кузьменко}
\textbf{Дарья Кирносова} 

живодёрство в чистом виде это стаи беспризорных животных. Это уничтожение собак
и кошек, как брошенных \enquote{сердобольными дюдьми} так и домашних. Более
ответственных собаководов, чем заводчики охотничьих собак трудно найти. А
деятельность притравочных станций была чётко регламентированна. Звери для
притравок и натаски разводились специально и жили лучше их диких собратьев.
Теперь по требованиям \enquote{защитников природы} станции закрыты. Собак охотники
обучают в лесу сами, теряя массу зверья подранками, калеча и губя собак.
Недостаток поголовья норных собак в Украине обусловил небывалую вспышку
бешенства плотоядных, болезни опаснейшей для человека и животных. Основными
разносчиками бешенства являются лисы и енотовидные собаки. Когда-то они были
объектами норной охоты... @igg{fbicon.face.sad.but.relieved} 

\begin{itemize} % {
\iusr{Дарья Кирносова}
\textbf{Петр Кузьменко}, ожидаемый ответ.
Особенно про бешенство. Стандартные отписки охотников.
В нормальных странах с бешенством борются превентивными мерами, а н как не отстрелом.
Про замечательные условия жизни притравочных животных - пишите сказки дальше, кто-то да поверит...
Отвечать больше не буду...
За все каждый когда-то ответит... По заслугам...

\iusr{Иванна Зацнова}
\textbf{Петр Кузьменко} В клетках покалеченые животные жили лучше чем на воле! Жестокий циник.

\iusr{Нелли Кузьменко}
\textbf{Иванна Зацнова} 

Вы бывали на притравочных станциях? Видели животных здоровых, привитых и
откормленных? Они обучены работать с собаками. Есть фото нормастера с
притравочной лисой выше в комментариях. Если Вы так хорошо в этом разбираетесь,
ответьте мне, пожалуйста, на простой вопрос какова продолжительность жизни
дикого зверя в нашей нынешней природе? На нашей притравочной станции лисовин
прожил пять лет и был отправлен на заслуженный отдых в зооуголок охотхозяйства
в вольер. Так в чьих комментариях и поступках больше цинизма?

\end{itemize} % }

\iusr{Петр Кузьменко}
\textbf{Дарья Кирносова} ожидаемая реакция диванного \enquote{защитника природы} @igg{fbicon.face.sad.but.relieved} 

\end{itemize} % }

\iusr{Владимир Новицкий}

\ifcmt
  ig https://scontent-frx5-2.xx.fbcdn.net/v/t39.1997-6/s168x128/93027172_222645632401274_7176243611145601024_n.png?_nc_cat=1&ccb=1-5&_nc_sid=ac3552&_nc_ohc=5bx1DLdOCosAX8NNX4d&_nc_ht=scontent-frx5-2.xx&oh=00_AT8Sa4aezLqwImEmpvaei8DWAPTRqU3PnZHlng6ieN3gAA&oe=61BE3A70
  @width 0.1
\fi


\iusr{Леся Фандралюк}

Спасибо, Петр!) Прочитала на одном дыхании. И в конце начала реветь.. У меня тоже
была собака(( Больше завести не могу(. И я помню, какой кошмар был, когда травили
собак. Это делают и сейчас, уродов хватает ((

\begin{itemize} % {
\iusr{Римма Риммская}
\textbf{Леся Фандралюк} , наш друг стаффордширский терьер так погиб. Мучительно уходил... Зачем это делать? Что в голове у этих людей?

\iusr{Леся Фандралюк}
\textbf{Римма Риммская} в голове-думаю там пусто, как и в душе ((Вернее там яд((
\end{itemize} % }

\iusr{Ирина Архипович}
Ах, какой чудесный рассказ!!!  @igg{fbicon.hand.ok}
@igg{fbicon.thumb.up.yellow}  @igg{fbicon.face.happy.two.hands} 

\iusr{Наталия Калатозишвили}
Спасибо за интересный рассказ

\iusr{Римма Риммская}

С огромным удовольствием читала! Просто наслаждалась чтением!

А у нас горе... Вчера вечером умерла наша собака... Карликовый пудель Буся (
Бусинка) - воспитатель моего сына, любовница (я совсем не ревновала) моего
мужа, моя подруга и центр Вселенной моего папы. Дождалась его из больницы и
через пару часов умерла. Она болела и постепенно готовилась в Собачий Рай. Если
модераторы позволят, я сделаю пост - метафору о печальном клоуне и его собаке,
посвященный моему папе и его самой драгоценной Бусинке на земле.

\begin{itemize} % {
\iusr{Светлана Манилова}
\textbf{Римма}, конечно, позволят! Сочувствую Вам...

\iusr{Петр Кузьменко}
\textbf{Римма Риммская} 

как я Вам сочувствую. Наша семья скорбит вместе с Вашей. Мы знаем, как это
больно и тяжело терять ДРУЗЕЙ... Приходиться утешать себя тем, что все собаки
попадают в Рай. @igg{fbicon.face.sad.but.relieved} 

\iusr{Леся Фандралюк}
\textbf{Римма Риммская} Скорблю и обнимаю вас, Римма! Знаю, что такое потерять Друга((

\iusr{Natasha Levitskaya}
\textbf{Римма Риммская}
Римма, очень жаль! Сочувствую и понимаю! Мы летом потеряли нашего любимца Барника...
Пишите обязательно!
\end{itemize} % }

\iusr{Наталия Водяницкая}

Спасибо за такой замечательный рассказ!@igg{fbicon.heart.red}Только у таких
людей, как Вы - добрых и ответственных, должны жить животные! Ваши маленькие
четвероногие друзья прожили с Вами замечательную, интересную, насыщенную жизнь.
И все - благодаря Вам. @igg{fbicon.smile}  Прочитала, как фильм посмотрела.
@igg{fbicon.face.happy.two.hands}  Очень талантливо пишите, Петр.
@igg{fbicon.thumb.up.yellow}  Одно удовольствие - читать!  @igg{fbicon.smile}
@igg{fbicon.heart.red}

\begin{itemize} % {
\iusr{Петр Кузьменко}
\textbf{Наталия} благодарю за столь лестную оценку моих скромных мемуарных опытов в нашей тёплой и душевной группе!😍

\begin{itemize} % {
\iusr{Наталия Водяницкая}
\textbf{Петр Кузьменко} полностью с Вами согласна. Группа у нас действительно очень теплая, душевная и атмосферная - благодаря таким замечательным людям, как Вы, Светочка Манилова, её муж Феликс, Сергей Векслер, Наташенька Левицкая, Женечка Бочковская и её муж ВикТор, Галина Васильевна Демина и многие-многие-многие другие славные люди-киевляне и не только (всех не перечислить!), которых я уже знаю очно-заочно и многих люблю! Благодаря - КИ!  @igg{fbicon.smile}  @igg{fbicon.face.blowing.kiss}  @igg{fbicon.face.happy.two.hands}  @igg{fbicon.heart.red}

\iusr{Светлана Манилова}
\textbf{Наталия}, ты забыла упомянуть Наташеньку Водяницкую, человека необыкновенной доброты и искренности!

\iusr{Наталия Водяницкая}
\textbf{Светлана Манилова} мой дорогой человечек - с невероятно чуткой душой!!!@igg{fbicon.heart.red} Я даже ещё ничего не написала и не поделилась в КИ!!! Благодарю за такие слова. Я тронута и смущена. @igg{fbicon.smile}  @igg{fbicon.face.blowing.kiss}  @igg{fbicon.face.happy.two.hands} @igg{fbicon.heart.red}

\iusr{Natasha Levitskaya}
\textbf{Наталия Водяницкая}
И я присоединяюсь к Светлана Манилова и упоминаю Наташеньку Водяницкую, редкого по своим душевным качествам человека! @igg{fbicon.heart.red}

\iusr{Наталия Водяницкая}
\textbf{Natasha Levitskaya} спасибо большое, Наташенька. Ох, дорогие мои девочки!... @igg{fbicon.smile}  @igg{fbicon.face.blowing.kiss}  @igg{fbicon.face.happy.two.hands} @igg{fbicon.heart.red}
\end{itemize} % }

\iusr{Нина Опольская}

трогательный рассказ о братьях наших меньших, душевная радость...

\begin{itemize} % {
\iusr{Михайло Дяченко}
\textbf{Нина Опольская} О братьях меньших - это вы о лисе и медведе на притравочной станции? Действительно, сплошная душевная радость...

\iusr{Петр Кузьменко}
\textbf{Михайло Дяченко} Вы там бывали? Знаете как они у нас появились, как жили какова их судьба?

\iusr{Михайло Дяченко}
\textbf{Петр Кузьменко} 

Жили лис и медведь замечательно, от всей души желаю вам и вашим близким такой
судьбы и жизни. Может тогда вы что-то поймёте.

\iusr{Петр Кузьменко}
\textbf{Михайло Дяченко} 

не наблюдал Вас на нашей притравочнрй станции. Наш лис был привит, упитан и
откормлен. Он был обучен работе с собаками. Попал он к нам лисёнком, найденым в
тайге. Его маленького бросила мать. Так поступают лисы, чтобы сохранить
здоровую часть потомства. Он прожил на станции 5 лет и был отправлен в вольер
зооуголка охотхозяйства на заслуженный отдых. Выше, в одном из комментариев,
есть фото киевского нормастера с \enquote{змученной} притравочной лисой. Потрудитесь
посмотреть. Кстати, в дикой природе лисы живут 3 - 4 года. Наш мишка жил в
вольере. На поляне, во время притравки, собаки не работают с медведем в прямом
контакте, а лишь облаивают. Попал он к нам подростком, мать которого убили
браконьеры. И прожил долгую, сытую и счастливую по сравнению с дикими
собратьями жизнь. Прошу Вас, уважаемый диванный защитник природы, рассказать,
что сделали лично Вы, чтобы помочь диким и одичавшим домашним животным.
Читателям, полагаю, будет очень интересно.

\iusr{Михайло Дяченко}
\textbf{Петр Кузьменко} Как могу, борюсь против притравочных станций и горе-охотников. Мое пожелание вам судьбы животных из притравки остаётся в силе. Написать отзыв подробнее не смогу, поскольку после вашего поста отписался от группы.
\end{itemize} % }

\iusr{Михайло Дяченко}
\textbf{Наталия Водяницкая} 

Особенно интересная жизнь была у лиса и медведя, на которых тренировались
собаки. С вырванными зубами и когтями, каждый раз израненные до полусмерти, а
когда раны подживут, снова затравленные собаками. И некуда скрыться, спрятаться
- душевные охотники вновь и вновь натравливают своих питомцев. Ведь это
страшное издевательство над живыми существами, а вы автору благодарности
раздаете !


\iusr{Петр Кузьменко}
\textbf{Михайло Дяченко} 

Вам могу в ответ пожелать судьбу нынешних обитателей дикой украинской природы.
Их, конечно оберегают, борются за увеличение поголовья и подкармливают зимами
такие \enquote{зщитники}, как Вы. Поинтересуйтесть, какие зоотехнические мероприятия
проводят охотхозяйства, егеря и охотоведы. Какова стоимость для охотника
путёвки, отстрелочного талона, лицензии и куда направляются средства от их
реализации?

\end{itemize} % }

\iusr{Елена Романская}

Замечательный, полный любви рассказ. Очень тяжело терять своих питомцев. На
днях ушёл наш кот. Это так непросто, видеть, как постепенно утекает жизнь из
любимого сущестаа, смотреть на последний вздох... Помогает любимая собачья
мордаха, вылизывающая твои слёзы.

\iusr{Елена Романская}

\ifcmt
  ig https://scontent-frt3-1.xx.fbcdn.net/v/t1.6435-9/161999134_3667393716689168_5298057794839314654_n.jpg?_nc_cat=104&ccb=1-5&_nc_sid=dbeb18&_nc_ohc=_jvW4hZKVjwAX_rc3sk&_nc_ht=scontent-frt3-1.xx&oh=00_AT9Q9KKHuMCf-sFZODSGthsuotF8SBVPpvNf3whrefU-gQ&oe=61DE1311
  @width 0.5
\fi

\iusr{Светлана Поклад}

Тут прям нахлынуло!!! Вот прям лавиной! Потому что... Первой моей собакой была
такса по имени Пиццули (я тогда в Югославии жила и это имя можно было перевести
, как Малышка (с итальянского))). Ох... Это не собака была! Гений! Питалась у
соседей, куда прорыла ТРИ хода под забором! Любимица улицы! А потом к нам жених
пришёл! Сам Шикарнючий такс Бакс! И жил у нас месяц! А потом щенки!)))Которые
газеты своим хозяевам носили из киосков! Ооооох! Это было нечто! А по поводу
выбрать щенка)))Ведь когда ко мне навстречу выбежала орава лабрадоров, все
бросились играть с висюльками на джинсах подруги, а один..., Бусенька,
вскарабкался ко мне на колени! Мне надо было выбирать?! Он меня выбрал!!!

\begin{itemize} % {
\iusr{Петр Кузьменко}
\textbf{Светлана} таксы - суперсобаки! Благодарю за тёплые и весёлые воспоминания!
\end{itemize} % }

\iusr{Михайло Дяченко}

А Пиню и Кешу, и всех, на кого натаскивали собак не было жаль? Они не живые, у
них раны не болят? Нельзя любить собаку и заставлять страдать других животных.
Притравочные станции - это извращение и садизм в чистом виде. Лучше бы автору
своим участием в кровавом деле не хвалиться.

\begin{itemize} % {
\iusr{Петр Кузьменко}
\textbf{Михайло Дяченко} 

не наблюдал Вас на нашей притравочнрй станции. Наш лис был привит, упитан и
откормлен. Он был обучен работе с собаками. Попал он к нам лисёнком, найденым в
тайге. Его маленького бросила мать. Так поступают лисы, чтобы сохранить
здоровую часть потомства. Он прожил на станции 5 лет и был отправлен в вольер
зооуголка охотхозяйства на заслуженный отдых. Выше, в одном из комментариев,
есть фото киевского нормастера с \enquote{змученной} притравочной лисой. Потрудитесь
посмотреть. Кстати, в дикой природе лисы живут 3 - 4 года. Наш мишка жил в
вольере. На поляне, во время притравки, собаки не работают с медведем в прямом
контакте, а лишь облаивают. Попал он к нам подростком, мать которого убили
браконьеры. И прожил долгую, сытую и счастливую по сравнению с дикими
собратьями жизнь. Прошу Вас, уважаемый диванный защитник природы, рассказать,
что сделали лично Вы, чтобы помочь диким и одичавшим домашним животным.
Читателям, полагаю, будет очень интересно.

\end{itemize} % }

\end{itemize} % }
