% vim: keymap=russian-jcukenwin
%%beginhead 
 
%%file 16_09_2021.fb.uli_natalia.1.dostoinstvo_ukraina
%%parent 16_09_2021
 
%%url https://www.facebook.com/nataliya.ulia/posts/10223296058687608
 
%%author_id uli_natalia
%%date 
 
%%tags chelovek,dostoinstvo,obschestvo,strana,ukraina,vlast,zelenskii_vladimir
%%title О чувстве СОБСТВЕННОГО ДОСТОИНСТВА...
 
%%endhead 
 
\subsection{О чувстве СОБСТВЕННОГО ДОСТОИНСТВА...}
\label{sec:16_09_2021.fb.uli_natalia.1.dostoinstvo_ukraina}
 
\Purl{https://www.facebook.com/nataliya.ulia/posts/10223296058687608}
\ifcmt
 author_begin
   author_id uli_natalia
 author_end
\fi

О чувстве СОБСТВЕННОГО ДОСТОИНСТВА...

Я как-то писала о том, что случайно довелось увидеть программу канала Food
Network. Была приятно удивлена тем, как мастерски они обыгрывали сюжеты о
приготовлении различных блюд. Все эти люди любят свою работу и уважают себя, -
подумала я. Когда ты вкладываешь душу и сердце в то, что делаешь, когда ты
доволен плодами своего труда, когда своей деятельностью доставляешь
удовольствие и приносишь пользу другим, ты, тем самым показываешь - я  уважаю
себя и окружающих.

Лично для меня, человек хорошо и честно делающий свое дело, однозначно обладает
чувством собственного достоинства.

Много ли таких людей? Не берусь судить. Но......

Какую же картину мы наблюдаем, глядя на тех, кого называют правящей элитой
общества. Есть ли у них чувство собственного достоинства? Мой ответ однозначный
- нет.

"Человек не становится достойнее, когда к его имени приставляют словечко
«сир»."

Так  сложилось в моей жизни, что я работаю с разными людьми.  Я всегда несла и
несу ответственность за то, что делают другие люди в моих проектах. Случалось,
люди не умели или не хотели работать. Но мне никогдa не приходила в голову
мысль о том, что я не справлюсь или не решу тех или иных задач. Все, чего мне
хотелось, так это  чтобы людям было интересно и комфортно работать.. И работа
была сделана действительно хорошо. Меня часто называют перфекционисткой. Да,
возможно, я чрезмерно стараюсь контролировать процессы. Но я точно знаю, что
если ты взялся за дело, какое б оно ни было, будь добр, выполняй на совесть. 

Поэтому я  не устаю поражаться, тем существам, которые находятся у власти много
лет, имели и имеют возможность созидать и гордиться тем, что сделали для
общества и процветания страны, но, увы, эти особи настолько не уважают себя,
что заботятся "не столько о собственном достоинстве, сколько о достойной
собственности."

Ложь, манипулирование, воровство вот их вечный конек.  

Помните, раньше писали сочинения: " Если бы директором (президентом)....был я"?
Интересно, что каждый из вас написал бы в таком сочинении?. Я точно написала
бы, что это неописуемый драйв и радость просто профессионально  делать свое
дело, быть довольным тем, что сумел построить что-то  важное и полезное. Что
ты, дав обещания, стремишься обязательно выполнять их, даже если не все сразу
получается.  Идешь дальше, зная, что тебе верят и поддерживают   Что тебя
уважают не как высшее должностное лицо, а как Личность. 

 Ответственность — одно из лучших достоинств человека.

„Ничто не может быть более похвальным, чем сознание собственного достоинства в
тех случаях, когда мы действительно обладаем ценными качествами.“ 

Томас Карлейл

На днях посмотрела кусочек интервью нынешнего гаранта, которое он давал на
публике одному западному журналисту. На вопрос о коррупции  первое лицо
государства ответило, в хамско-шутовской манере,   что мол хватит обвинять нас
в коррупции, она существует во всем мире, а вы прицепились только к нам, и
только нам  клеите ярлык супер коррумпированного государства. "Достойнейший"
ответ государственного мужа. . 

"Даже смеяться над ним было ниже собственного достоинства."

Мне неоднократно приходилось слышать фразу от вполне казалось бы адекватных
людей: "Власть нужно уважать". Ну во-первых, вы сами наделили их этой властью.
Уважать их можно, конечно, но только если они сами себя уважают, совершая
достойные поступки. А во-вторых,  неплохо бы вам, уважаемые,   относиться к
самим себе уважительно. Раз и навсегда зарубив  на носу: государство, как и
любой чиновник - это сервис, а не власть. Прекратить терпеть произвол, не
давать  унижать себя хамским отношением чинуш всех мастей. "Стерпишь
несправедливость раз, два — и ты сломлен. Можно стерпеть нищету, но только не
то, что твое человеческое достоинство унижает какой-нибудь влиятельный говнюк." 

Идя на выборы, задумайтесь, вы действительно хотите в очередной раз отдать
голос за людей без чувства собственного достоинства?

Всегда стоит помнить о том что Достоинство - это привилегия.
