% vim: keymap=russian-jcukenwin
%%beginhead 
 
%%file slova.karta
%%parent slova
 
%%url 
 
%%author 
%%author_id 
%%author_url 
 
%%tags 
%%title 
 
%%endhead 
\chapter{Карта}
\label{sec:slova.karta}

%%%cit
%%%cit_head
%%%cit_pic
\ifcmt
  pic https://avatars.mds.yandex.net/get-zen_doc/4888095/pub_60bdf74e7b0ba72b6fd0d2cb_60be012bbcbf42494e3b1c20/scale_1200
\fi
%%%cit_text
Хочу добавить к перечисленному \emph{карту} Африки. Хорошо очерченное
побережье, с указанием городов.  В Сахаре достаточно много небольших озёр и
рек. Обозначен пролив Гибралтар.  То есть \emph{карта} написана не из фантазий
автора, а по, достаточно точным, измерениям.  В Африке в это время идёт борьба
с работорговлей против европейских колонизаторов. Как и во всём мире, в Африке
16 века сформировано несколько государств. Развита металлургия, животноводство
и земледелие.  Но с приходом европейских колонизаторов эти государства пришли в
упадок
%%%cit_comment
%%%cit_title
\citTitle{Карты 1570 года. Тартария, Америка и материк на Северном Полюсе}, 
Нешкольная История, zen.yandex.ru, 08.06.2021
%%%endcit

%%%cit
%%%cit_head
%%%cit_pic
\ifcmt
  tab_begin cols=3
     pic https://avatars.mds.yandex.net/get-zen_doc/1880126/pub_60f29964cf9db26dda7a8641_60f2a1f54b5d8b6f29283db2/scale_1200
     pic https://avatars.mds.yandex.net/get-zen_doc/3414453/pub_60f29964cf9db26dda7a8641_60f2a2856de09944845431f5/scale_1200
     pic https://avatars.mds.yandex.net/get-zen_doc/916951/pub_60f29964cf9db26dda7a8641_60f2a42b4b5d8b6f292d6bc5/scale_1200
  tab_end
\fi
%%%cit_text
И это, заметьте, \emph{карта} 1375 года!  А вот \emph{карта} Николая Кузанского
1491 года, которую можно найти в Национальной библиотеке Франции. И тут тоже
Россия. Причем к России тут относят и милый сердцу каждого уважающего себя
украинского националиста Львов. А Киев обозначен как Chiovea - Киовия,
соответственно, жить там должны "киовиты".  А вот на Херефордской \emph{карте}
(изготовлена до 1300 года) из Херефордского собора, Британия можно обнаружить
одновременно и Россию и славян и медведя (ну само собой, медведь же наше все).
Ориентация \emph{карты} - восток сверху, в центре Иерусалим. Это самая большая
существующая средневековая \emph{карта}, она нарисована на пергаменте из
цельной телячей кожи 158 на 133 см, её автор - Ричард из Хардингема (есть
подпись в левом нижнем углу).  И таких \emph{карт} можно еще много найти. То
есть "московия" появляется на \emph{картах} именно во время и благодаря
антироссийской литовско-польской пропаганде, что в принципе довольно наглядно
видно по тому, когда вообще впервые появились \emph{карты} с термином
"московия". И любой, кто прежде всего человек увлекающийся историей, а не
нацист, это прекрасно знает
%%%cit_comment
%%%cit_title
\citTitle{300-летию \enquote{московии} посвящается}, Илья Duke, zen.yandex.ru, 19.07.2021
%%%endcit

%%%cit
%%%cit_head
%%%cit_pic
%%%cit_text
А вот в Киевраде, в отличие от Верховной, есть не только
"национально-сознательное" большинство, но и, похоже, пронацистское. Только
этим можно объяснить тот факт, что с \emph{карты} столицы одну за другой
убирают улицы, которые не подпадают ни под закон о декоммунизации, ни даже под
какие-либо антироссийские тенденции.  К примеру, со вчерашнего дня больше не
существует улицы Народного ополчения. В чем провинились ополченцы, защищавшие
Киев от немецких захватчиков, понять сложно.  То же касается улицы
Молодогвардейцев – переименовать ее могли только те, кто считает деятельность
юных подпольщиков Краснодона диверсией против вермахта.  Ну, а улицу
легендарного советского разведчика Николая Кузнецова и вовсе переименовали в
честь писателя-коллаборациониста Олеся Бабия, сбежавшего с нацистами в 1944-м.
И это еще не все. Уже известно, что на очереди у Киеврады находятся улицы
Героев Сталинграда и Маршала Тимошенко, начались разговоры и в отношении улицы
Героев Днепра и площади Героев Севастополя. Судя по тенденциям, доберутся и до
них. Вопрос только в том, какую из этих улиц назовут именем гауляйтера Украины
Эриха Коха
%%%cit_comment
%%%cit_title
\citTitle{СБУ на службе у "активиста", Рада "разгромила" ЗСТ с ЕС, Киеврада уничтожила Молодогвардейцев. Итоги "Страны"}, 
, strana.news, 05.11.2021
%%%endcit

%%%cit
%%%cit_head
%%%cit_pic
\ifcmt

\ifcmt
  tab_begin cols=3
    pic https://strana.news/img/forall/u/0/36/photo_2021-11-07_12-18-33.jpg
    pic https://avatars.mds.yandex.net/i?id=04bc9e48345c0e50fcda4bf5e874a016-5008692-images-thumbs&n=13
    pic https://politikus.ru/uploads/posts/2016-03/1458309955_256.jpg
  tab_end
\fi
%%%cit_text
Компания BlaBlaCar запустила рекламу в Google, где использовали \emph{карту} Украины
без Крыма. На это обратил внимание один из пользователей, после чего за ошибку
извинились, а ошибку обещали исправить.  "А это ок такую рекламу в Google
получать BlaBlaCar, а чей Крым? Интересно, что еще никто об этом не говорил", -
возмутился Александр Остапа на своей странице в Facebook.  Позже он дописал
сообщение, в котором размещено извинение от представителей BlaBlaCar.  "Добрый
день, Александр. Спасибо за тег. Это ошибка, мы оперативно выясним, откуда
вообще эта реклама могла появиться. Естественно, украинский офис BlaBlaCar
никогда бы не сделал такую рекламу, это противоречит позиции нашей компании", -
ответили пользователю
%%%cit_comment
%%%cit_title
\citTitle{BlaBlaCar запустил рекламу с картой Украины без Крыма. Фото}, 
Анна Копытько, strana.news, 07.11.2021
%%%endcit

%%%cit
%%%cit_head
%%%cit_pic

\ifcmt
  tab_begin cols=3
     pic https://avatars.mds.yandex.net/get-zen_doc/901899/pub_61854afbee117612a305bdb5_61854b218716dc0d2fe84e43/scale_1200
     pic https://avatars.mds.yandex.net/get-zen_doc/3524532/pub_61854afbee117612a305bdb5_61854b2ff93c223e7ca5a956/scale_1200
		 pic https://avatars.mds.yandex.net/get-zen_doc/1246934/pub_61854afbee117612a305bdb5_61854b3a6001fe1da5fbf43e/scale_1200
  tab_end
\fi
%%%cit_text
Поговорим сегодня о \emph{картах}. Нет, ни в коем случае речь не будет идти об
игровых картах. Я в этом хорошо разбираюсь, знаю, что такое крап, раздача,
сброс, да вообще много чего знаю. Вот только бесполезные это знания и ненужные.
Посему рассказывать о них нет решительно никакого смысла. Лучше мы с вами
посмотрим на то, какой Россия представлялась нашим западным партнерам, когда
ещё была империей. У нас тут есть тенденция утверждать, что если бы не
революция 1917 года, то мы бы сейчас дружны были бы со всей Европой. Такие
тезисы идут от банального незнания истории. В рамках небольшой подборки эту
самую историю изучать невозможно, но бросить взор на импортные изображения
родной страны очень даже можно. Вот и бросим.  На мой взгляд, заглавное фото
очень красноречивое. Эта \emph{карта} датируется 70-ми годами 19 века. То есть
вы видите, как изображали тогда нашу страну. Судя по языку, эта карта из
Италии, которая тогда себя ещё видимо, считала наследницей Римской империи и
всерьез думала о том, что так и будет влиятельной страной.  Итальянцы оказались
наивными ребятами. Уж в ту пору они мало что решали в Европе. И всё на что были
способны это на изображение нашей страны вот в таком виде
%%%cit_comment
%%%cit_title
\citTitle{Исторические карты Европы и как там изображали Россию}, 
soullaway soullaway, zen.yandex.ru, 06.11.2021
%%%endcit
