% vim: keymap=russian-jcukenwin
%%beginhead 
 
%%file saint_russia.chitalnja_ru.lebedia
%%parent saint_russia
%%url https://www.chitalnya.ru/work/1584078/
 
%%endhead 

\url{https://www.chitalnya.ru/work/1584078/}

Лебедия - страна Русов
[Наталья Котомина]
 Версия для печати

Корни Руси уходят в недосягаемые глубины прошлого.
Мир был чище и  откровенней,   Дух предков хранил прямых потомков, в мире была ГАРМОНИЯ!
                                                                            
   Лебеди не случайно стали священной птицей, символом  возрождения, чистоты, целомудрия, благородства, мудрости, пророческих способностей, поэзии, верности и мужества далеких предков.
ЛЕБЕДЬ --- СИМВОЛ ГИПЕРБОРЕИ.
Гиперборейская цивилизация является самой древнейшей в мире и примерно в 2 раза старше Шамбалы на Тибете.
ЛЕБЕДИ - это духовная суть русов, а Дух предков - это и есть "РУССКИЙ ДУХ".
Дух основанный на внутренней чистоте,  на Прави,
Царство Лебедия  не  призрачная  сказка, а  красивая быль.
 "Сказка ложь, да в ней намек, добрым молодцам урок"!  У  РОК!
Рок - предназначение!
Слишком много  артефактов, чтобы не обратить внимание на культ лебедя.  
Среди археологических находок на территории России встречаются древние колесницы запряжённые лебедями, первобытная утварь с их изображениями, вышивки в виде лебедей, амулеты, другие малые художественные формы.
ЛЕБЕДИЯ была страной русов.
 Род остался верен предкам и пронес силу духа  и чистоту лебедя через сотни тысячелетий, выдержав катаклизмы, катастрофы, войны.   
В мифах бурятов, ненцев, румынов, славян, Девица-Лебедь сбросывала лебединое одеяние, превращаясь в волшебных красавиц.
Культ Лебедя шел с севера, т.к. мифы бурят древнее европейских!
Пазыринская культура с изображениями лебедей - V тыс. до н.э. - факт особого внимания,  изображения  на древних камнях Онежского озера -  IV тыс. до н.э.,  лебедь в раскопках III—II тыс. до н.э. на Среднем Урале, другие, древнейшие ковши,  петроглифы лебедей Онежского озера !
Лебеди на скалах Кольского полуострова.
В Х веке, когда византийский император заносил на пергамент скупые сведения о таинственной Лебедии, ей  было уже несколько тысячелетий. 
Единственный народ, который имел и сохранил культ лебедя до сих пор,  не только духовно, но и материально, - это русские!                      
Только на Руси величали девушку "лебедушкой", только у русских была царевна-Лебедь, только русские витязи изначально одевали шлемы в виде лебедя. Это поже появились аналоги у других народов.
ЛЕБЕДИНАЯ ДЕВА - Царевна-лебедь - культ Девы;
Лебединые девы Гиперборейские были светлыми божественными существами.
Лебединый культ АППОЛОНА  Гиперборейский в  Древнегреческих хрониках  очевиден. Апполон прилетал в колеснице запряженной белыми лебедями;
Древние петроглифы лебедей; шлемы, 2-х метровые статуи белокрылых птиц, русских ковшей в виде лебедя, сказания,  др. артефакты, говорят о большой  значимости  этого культа на Руси.  Всё лебеди! 
Их дискредитация будет позже,  как и всего русского, приписывание противоположных смыслов, демонизация как политических противников  безвинных культов предков.   Не пощадят дев, а лебедя побоятся тронуть - слишком светлая птица. Ума хватило на мизер осознания священного.
Этот милый сердцу ОБРАЗ ЛЕБЕДЯ доказывает сакральный смысл русского культа!!!
В славянской мифологии Лебедь относится к почитаемым, «святым» птицам.
ИРИЙСКИЕ горы - Алтай - стали  пристанищами  лебеде-людей. На алтайской реке Лебеди, притоке Бии, до сих пор проживает народ Лебединцы - ассимилированные потомки светлых предков.
Самого Архангела Михаила издревле изображали Лебедем!
На Большом государственном гербе Российской Империи 1882 года, в его четвёртой части, «в лице» герба Стормарнского помещен серебряный лебедь, лебедь на печатях, флагах как символ преданности и бессмертия.
то не выдумки бывалого сказочника.Есть хроники: чекист и парапсихолог А.В. Барченко в 20-х годах прошлого века, со своей группой, посещал Кольский п-ов и ставил себе задачу распознать как удавалось гиперборейцам перемещать по воздуху тяжёлые объекты, перемещаться в пространстве и раскрыть секрет расщепления атомного ядра. Они долго жили на Кольском полуострове, сотрудничали с саамскими шаманами, но тайны полетов гиперборейцев остались нераскрытыми.
У  каждого русского генетическая память хранит образ о священной птице, её безмерной преданности и величия, напоминая всем своим царственным видом о достоинстве, красоте и величии борейкого рода.
ЛЕБЕДЬ  остался сердцем России! 
                                          На старинном гербе Москвы шлем Георгия Победоносца венчает ЛЕБЕДЬ, как и на некоторых иконах!
ЛЕБЕДИНАЯ ВЕРНОСТЬ известна всем, но еще лебеди перед смертью поют!!
Это красивая грустная песня! Выводить птенцов Лебеди летят на север - домой!!!
Самого Архангела Михаила издревле изображали Лебедем!
На Большом государственном гербе Российской Империи 1882 года, в его четвёртой части, «в лице» герба Стормарнского помещен серебряный лебедь, лебедь на печатях, флагах как символ преданности и бессмертия.
Русы имеют могущественную Силу данную Родом Белой Расы, оставаясь душой чистым и светлым как древний символ прародины - ЛЕБЕДЯ.
Сказка маленькой девочки из интерната потрясает воображение: « Далеко- далеко на севере есть прекрасная страна Лебедия. Там высокие сосны и глубокие озера. Там живут одни лебеди. В стране Лебедии никто никогда не плачет, там все очень счастливы, потому что у каждого ребенка-лебеденка есть своя мама-лебедушка. Она защищает его большими крыльями от холода и ветра, от злых насекомых. Кормит его из клюва, пока он маленький, а когда ребенок-лебеденок вырастает, он также заботится о своей маме-лебедушке». Она нашла свою семью - помоги духи Лебедии.
Россию именуют «Лебедией Будущего» - ЛЕБЕДИЯ - это Русь
Пока существует Русь --- у человечества есть шанс избежать вымирания.

Белопенно-белокрылая Русь - лебедушка моя,
Проплыла царевной лебедью, да по глади бытия,
Как без края и без времени льется песнею душа,
Ввысь взлетела белой лебедью воскрешая небеса!

Пролетали гуси-лебеди, да в родимые края,
Да на северную родину, клином вечности паря.
Всколыхнулась из безвременья наша русская краса -
Величалась на созвездие, претворяя чудеса!
 
