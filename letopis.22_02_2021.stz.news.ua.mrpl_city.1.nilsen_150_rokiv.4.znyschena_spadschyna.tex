% vim: keymap=russian-jcukenwin
%%beginhead 
 
%%file 22_02_2021.stz.news.ua.mrpl_city.1.nilsen_150_rokiv.4.znyschena_spadschyna
%%parent 22_02_2021.stz.news.ua.mrpl_city.1.nilsen_150_rokiv
 
%%url 
 
%%author_id 
%%date 
 
%%tags 
%%title 
 
%%endhead 

\subsubsection{Знищена спадщина}

Німецькі окупанти знищували не лише культурні та архітектурні пам'ятки, вони
знищили і альбом замальовок архітектора, куди той вносив малюнки фасадів
запроєктованих споруд та їх тогочасного оточення. Будинок Віктора
Олександровича та його приватний архів постраждали від пожеж, які вчинили
німецькі загарбники в часи відступу з міста. Втрата малюнків архітектора –
значно погіршила вивчення періоду пожвавлення забудови міста через відсутність
копій зниклих малюнків.
