% vim: keymap=russian-jcukenwin
%%beginhead 
 
%%file 30_07_2021.fb.buzhanskii_max.1.valuev_cirkuljar.cmt
%%parent 30_07_2021.fb.buzhanskii_max.1.valuev_cirkuljar
 
%%url 
 
%%author 
%%author_id 
%%author_url 
 
%%tags 
%%title 
 
%%endhead 
\subsubsection{Коментарі}
\label{sec:30_07_2021.fb.buzhanskii_max.1.valuev_cirkuljar.cmt}

\begin{itemize}
%%%fbauth
%%%fbauth_name
\iusr{Кирилл Рыжанов}
%%%fbauth_url
%%%fbauth_place
%%%fbauth_id
%%%fbauth_front
%%%fbauth_desc
%%%fbauth_www
%%%fbauth_pic
%%%fbauth_pic portrait
%%%fbauth_pic background
%%%fbauth_pic other
%%%fbauth_tags
%%%fbauth_pubs
%%%endfbauth
 
1863

\begin{itemize}
%%%fbauth
%%%fbauth_name
\iusr{Макс Бужанский}
%%%fbauth_url
%%%fbauth_place
%%%fbauth_id
%%%fbauth_front
%%%fbauth_desc
%%%fbauth_www
%%%fbauth_pic
%%%fbauth_pic portrait
%%%fbauth_pic background
%%%fbauth_pic other
%%%fbauth_tags
%%%fbauth_pubs
%%%endfbauth
 
\textbf{Кирилл Рыжанов} , испрвавил
\end{itemize}

%%%fbauth
%%%fbauth_name
\iusr{Стефано Ди Лоренцо}
%%%fbauth_url
%%%fbauth_place
%%%fbauth_id
%%%fbauth_front
%%%fbauth_desc
%%%fbauth_www
%%%fbauth_pic
%%%fbauth_pic portrait
%%%fbauth_pic background
%%%fbauth_pic other
%%%fbauth_tags
%%%fbauth_pubs
%%%endfbauth
 
А до 1863 многие книги были написаны на украинском языке?
\begin{itemize}
%%%fbauth
%%%fbauth_name
\iusr{Макс Бужанский}
%%%fbauth_url
%%%fbauth_place
%%%fbauth_id
%%%fbauth_front
%%%fbauth_desc
%%%fbauth_www
%%%fbauth_pic
%%%fbauth_pic portrait
%%%fbauth_pic background
%%%fbauth_pic other
%%%fbauth_tags
%%%fbauth_pubs
%%%endfbauth
 
\textbf{Stefano Di Lorenzo} , а количество имеет значение?)

%%%fbauth
%%%fbauth_name
\iusr{Николай Нико}
%%%fbauth_url
%%%fbauth_place
%%%fbauth_id
%%%fbauth_front
%%%fbauth_desc
%%%fbauth_www
%%%fbauth_pic
%%%fbauth_pic portrait
%%%fbauth_pic background
%%%fbauth_pic other
%%%fbauth_tags
%%%fbauth_pubs
%%%endfbauth
 
\textbf{Стефано Ди Лоренцо} до 1863 большинство населения жило в селе, и никаких книг не читало

%%%fbauth
%%%fbauth_name
\iusr{Стефано Ди Лоренцо}
%%%fbauth_url
%%%fbauth_place
%%%fbauth_id
%%%fbauth_front
%%%fbauth_desc
%%%fbauth_www
%%%fbauth_pic
%%%fbauth_pic portrait
%%%fbauth_pic background
%%%fbauth_pic other
%%%fbauth_tags
%%%fbauth_pubs
%%%endfbauth
 
\textbf{Макс Бужанский} да, считаю что имеет значение. а) чтобы понять, до
1863-ого, без запретов, на каких языках были опубликованы книги в Российской
Империи? б) чтобы оценивать явление украинскоязычной литературной традиции,
насколько она была распостраненная итд итп в) чтобы понять, какой смысл
действительно имел запрет: хотели ли все в Украине до 1863 писать на украинском
языке и им жестоко и грубо запретили, или было что-то другое? г)....

%%%fbauth
%%%fbauth_name
\iusr{Макс Бужанский}
%%%fbauth_url
%%%fbauth_place
%%%fbauth_id
%%%fbauth_front
%%%fbauth_desc
%%%fbauth_www
%%%fbauth_pic
%%%fbauth_pic portrait
%%%fbauth_pic background
%%%fbauth_pic other
%%%fbauth_tags
%%%fbauth_pubs
%%%endfbauth
 
\textbf{Stefano Di Lorenzo} , то есть, вы подозреваете царизм в том, что запретили то, чего не было?))

%%%fbauth
%%%fbauth_name
\iusr{Стефано Ди Лоренцо}
%%%fbauth_url
%%%fbauth_place
%%%fbauth_id
%%%fbauth_front
%%%fbauth_desc
%%%fbauth_www
%%%fbauth_pic
%%%fbauth_pic portrait
%%%fbauth_pic background
%%%fbauth_pic other
%%%fbauth_tags
%%%fbauth_pubs
%%%endfbauth
 
\textbf{Макс Бужанский} нет, вы неправиьно поняли, извините.

%%%fbauth
%%%fbauth_name
\iusr{Vadim Ryzhkov}
%%%fbauth_url
%%%fbauth_place
%%%fbauth_id
%%%fbauth_front
%%%fbauth_desc
%%%fbauth_www
%%%fbauth_pic
%%%fbauth_pic portrait
%%%fbauth_pic background
%%%fbauth_pic other
%%%fbauth_tags
%%%fbauth_pubs
%%%endfbauth
 
\textbf{Стефано Ди Лоренцо} 

Конечно, публиковали украинскую литературу. Все творчество Шевченко было
опубликовано. Циркулярное письмо для цензоров министра внутренних дел Валуева
это не запрет украинского языка вообще. В нем говорится о приостановке издания
до согласования с министром просвещения, Синодом и др. - духовных книг и
букварей, но не художественной литературы. Кстати, циркулярное письмо это не
закон, а просто ведомственное распоряжение

%%%fbauth
%%%fbauth_name
\iusr{Алексей Шимановский}
%%%fbauth_url
%%%fbauth_place
%%%fbauth_id
%%%fbauth_front
%%%fbauth_desc
%%%fbauth_www
%%%fbauth_pic
%%%fbauth_pic portrait
%%%fbauth_pic background
%%%fbauth_pic other
%%%fbauth_tags
%%%fbauth_pubs
%%%endfbauth
 
\textbf{Стефано Ди Лоренцо} 

Котляревский, Гулак-Артемовский, Гребёнка, Квитка-Основьяненко, Шевченко.
Вполне себе современная тому времени литература.

%%%fbauth
%%%fbauth_name
\iusr{Татьяна Давыдова}
%%%fbauth_url
%%%fbauth_place
%%%fbauth_id
%%%fbauth_front
%%%fbauth_desc
%%%fbauth_www
%%%fbauth_pic
%%%fbauth_pic portrait
%%%fbauth_pic background
%%%fbauth_pic other
%%%fbauth_tags
%%%fbauth_pubs
%%%endfbauth
 
\textbf{Макс Бужанский} и качество тоже...

\end{itemize}

%%%fbauth
%%%fbauth_name
\iusr{Дмитро Хрістов}
%%%fbauth_url
%%%fbauth_place
%%%fbauth_id
%%%fbauth_front
%%%fbauth_desc
%%%fbauth_www
%%%fbauth_pic
%%%fbauth_pic portrait
%%%fbauth_pic background
%%%fbauth_pic other
%%%fbauth_tags
%%%fbauth_pubs
%%%endfbauth
 
Точно в 1963?
Валуєв не дожив...

%%%fbauth
%%%fbauth_name
\iusr{Maksym Levutskyi}
%%%fbauth_url
%%%fbauth_place
%%%fbauth_id
%%%fbauth_front
%%%fbauth_desc
%%%fbauth_www
%%%fbauth_pic
%%%fbauth_pic portrait
%%%fbauth_pic background
%%%fbauth_pic other
%%%fbauth_tags
%%%fbauth_pubs
%%%endfbauth
 
Я уже запутался: кто против чего, за что и с кем \Smiley[1.0][yellow]

%%%fbauth
%%%fbauth_name
\iusr{Ирина Ольховка}
%%%fbauth_url
%%%fbauth_place
%%%fbauth_id
%%%fbauth_front
%%%fbauth_desc
%%%fbauth_www
%%%fbauth_pic
%%%fbauth_pic portrait
%%%fbauth_pic background
%%%fbauth_pic other
%%%fbauth_tags
%%%fbauth_pubs
%%%endfbauth
 
Чему удивляться?! По-моему, у нас, кляня Сталина, уже приняли все законы тех времен.)

%%%fbauth
%%%fbauth_name
\iusr{Петро Иванов}
%%%fbauth_url
%%%fbauth_place
%%%fbauth_id
%%%fbauth_front
%%%fbauth_desc
%%%fbauth_www
%%%fbauth_pic
%%%fbauth_pic portrait
%%%fbauth_pic background
%%%fbauth_pic other
%%%fbauth_tags
%%%fbauth_pubs
%%%endfbauth
 

\index{Валуевский циркуляр}

Кстати, для тех кто не читал этот документ: 18 июля, 1863 г.

\url{https://ru.wikisource.org/wiki/Валуевский_циркуляр}

Давно уже идут споры в нашей печати о возможности существования самостоятельной
малороссийской литературы. Поводом к этим спорам служили произведения некоторых
писателей, отличавшихся более или менее замечательным талантом или своею
оригинальностью. В последнее время вопрос о малороссийской литературе получил
иной характер, вследствие обстоятельств чисто политических, не имеющих никакого
отношения к интересам собственно литературным. Прежние произведения на
малороссийском языке имели в виду лишь образованные классы Южной России, ныне
же приверженцы малороссийской народности обратили свои виды на массу
непросвещенную, и те из них, которые стремятся к осуществлению своих
политических замыслов, принялись, под предлогом распространения грамотности и
просвещения, за издание книг для первоначального чтения, букварей, грамматик,
географий и т. п. В числе подобных деятелей находилось множество лиц, о
преступных действиях которых производилось следственное дело в особой комиссии.

В С.-Петербурге даже собираются пожертвования для издания дешевых книг на
южно-русском наречии. Многие из этих книг поступили уже на рассмотрение в
С.-Петербургский цензурный комитет. Не малое число таких же книг представляется
и в киевский цензурный комитет. Сей последний в особенности затрудняется
пропуском упомянутых изданий, имея в виду следующие обстоятельства: обучение во
всех без изъятия училищах производится на общерусском языке и употребление в
училищах малороссийского языка нигде не допущено; самый вопрос о пользе и
возможности употребления в школах этого наречия не только не решен, но даже
возбуждение этого вопроса принято большинством малороссиян с негодованием,
часто высказывающимся в печати. Они весьма основательно доказывают, что
никакого особенного малороссийского языка не было, нет и быть не может, и что
наречие их, употребляемое простонародием, есть тот же русский язык, только
испорченный влиянием на него Польши; что общерусский язык так же понятен для
малороссов как и для великороссиян, и даже гораздо понятнее, чем теперь
сочиняемый для них некоторыми малороссами и в особенности поляками, так
называемый, украинский язык. Лиц того кружка, который усиливается доказать
противное, большинство самих малороссов упрекает в сепаратистских замыслах,
враждебных к России и гибельных для Малороссии.

Явление это тем более прискорбно и заслуживает внимания, что оно совпадает с
политическими замыслами поляков, и едва ли не им обязано своим происхождением,
судя по рукописям, поступившим в цензуру, и по тому, что большая часть
малороссийских сочинений действительно поступает от поляков. Наконец, и
киевский генерал-губернатор находит опасным и вредным выпуск в свет
рассматриваемого ныне духовною цензурою перевода на малороссийский язык Нового
Завета.

Принимая во внимание, с одной стороны, настоящее тревожное положение общества,
волнуемого политическими событиями, а с другой стороны имея в виду, что вопрос
об обучении грамотности на местных наречиях не получил еще окончательного
разрешения в законодательном порядке, министр внутренних дел признал
необходимым, впредь до соглашения с министром народного просвещения,
обер-прокурором св.синода и шефом жандармов относительно печатания книг на
малороссийском языке, сделать по цензурному ведомству распоряжение, чтобы к
печати дозволялись только такие произведения на этом языке, которые принадлежат
к области изящной литературы; пропуском же книг на малороссийском языке как
духовного содержания, так учебных и вообще назначаемых для первоначального
чтения народа, приостановиться. О распоряжении этом было повергаемо на
Высочайшее Государя Императора воззрение и Его Величеству благоугодно было
удостоить оное монаршего одобрения.

%%%fbauth
%%%fbauth_name
\iusr{Ilya Mordkovich}
%%%fbauth_url
%%%fbauth_place
%%%fbauth_id
%%%fbauth_front
%%%fbauth_desc
%%%fbauth_www
%%%fbauth_pic
%%%fbauth_pic portrait
%%%fbauth_pic background
%%%fbauth_pic other
%%%fbauth_tags
%%%fbauth_pubs
%%%endfbauth
 

извените)))))))) в вашем законе ограничения и на издания которые принадлежат к
области изящной литературы.

\begin{itemize}
%%%fbauth
%%%fbauth_name
\iusr{Макс Бужанский}
%%%fbauth_url
%%%fbauth_place
%%%fbauth_id
%%%fbauth_front
%%%fbauth_desc
%%%fbauth_www
%%%fbauth_pic
%%%fbauth_pic portrait
%%%fbauth_pic background
%%%fbauth_pic other
%%%fbauth_tags
%%%fbauth_pubs
%%%endfbauth
 
\textbf{Ilya Mordkovich} , в ихнем) я полтора года как внес законопроект об отмене закона о языке)

%%%fbauth
%%%fbauth_name
\iusr{Людмила Смирнова}
%%%fbauth_url
%%%fbauth_place
%%%fbauth_id
%%%fbauth_front
%%%fbauth_desc
%%%fbauth_www
%%%fbauth_pic
%%%fbauth_pic portrait
%%%fbauth_pic background
%%%fbauth_pic other
%%%fbauth_tags
%%%fbauth_pubs
%%%endfbauth
 
\textbf{Ilya Mordkovich} что мелочиться, политика полумер есть беспринципная, соглашательская политика)

%%%fbauth
%%%fbauth_name
\iusr{Ilya Mordkovich}
%%%fbauth_url
%%%fbauth_place
%%%fbauth_id
%%%fbauth_front
%%%fbauth_desc
%%%fbauth_www
%%%fbauth_pic
%%%fbauth_pic portrait
%%%fbauth_pic background
%%%fbauth_pic other
%%%fbauth_tags
%%%fbauth_pubs
%%%endfbauth
 
\textbf{Макс Бужанский} с уточнением согласен)
\end{itemize}

%%%fbauth
%%%fbauth_name
\iusr{Sergiy Apostos}
%%%fbauth_url
%%%fbauth_place
%%%fbauth_id
%%%fbauth_front
%%%fbauth_desc
%%%fbauth_www
%%%fbauth_pic
%%%fbauth_pic portrait
%%%fbauth_pic background
%%%fbauth_pic other
%%%fbauth_tags
%%%fbauth_pubs
%%%endfbauth
 
Та же история с политикой полонизации

\begin{itemize}
%%%fbauth
%%%fbauth_name
\iusr{Інна Мельникова}
%%%fbauth_url
%%%fbauth_place
%%%fbauth_id
%%%fbauth_front
%%%fbauth_desc
%%%fbauth_www
%%%fbauth_pic
%%%fbauth_pic portrait
%%%fbauth_pic background
%%%fbauth_pic other
%%%fbauth_tags
%%%fbauth_pubs
%%%endfbauth
 
\textbf{Sergiy Apostos} а Вы им попробуйте докажите. У меня не получилось ни разу

%%%fbauth
%%%fbauth_name
\iusr{Sergiy Apostos}
%%%fbauth_url
%%%fbauth_place
%%%fbauth_id
%%%fbauth_front
%%%fbauth_desc
%%%fbauth_www
%%%fbauth_pic
%%%fbauth_pic portrait
%%%fbauth_pic background
%%%fbauth_pic other
%%%fbauth_tags
%%%fbauth_pubs
%%%endfbauth
 
\textbf{Інна Мельникова} а смысл?

%%%fbauth
%%%fbauth_name
\iusr{Інна Мельникова}
%%%fbauth_url
%%%fbauth_place
%%%fbauth_id
%%%fbauth_front
%%%fbauth_desc
%%%fbauth_www
%%%fbauth_pic
%%%fbauth_pic portrait
%%%fbauth_pic background
%%%fbauth_pic other
%%%fbauth_tags
%%%fbauth_pubs
%%%endfbauth
 
\textbf{Sergiy Apostos} смысл в том, что не должны доминировать фальшивые смыслы, они приносят вред

%%%fbauth
%%%fbauth_name
\iusr{Sergiy Apostos}
%%%fbauth_url
%%%fbauth_place
%%%fbauth_id
%%%fbauth_front
%%%fbauth_desc
%%%fbauth_www
%%%fbauth_pic
%%%fbauth_pic portrait
%%%fbauth_pic background
%%%fbauth_pic other
%%%fbauth_tags
%%%fbauth_pubs
%%%endfbauth
 
\textbf{Інна Мельникова} взрослых людей с блоками в мозгах не переубедить

%%%fbauth
%%%fbauth_name
\iusr{Юрий Токаренко}
%%%fbauth_url
%%%fbauth_place
%%%fbauth_id
%%%fbauth_front
%%%fbauth_desc
%%%fbauth_www
%%%fbauth_pic
%%%fbauth_pic portrait
%%%fbauth_pic background
%%%fbauth_pic other
%%%fbauth_tags
%%%fbauth_pubs
%%%endfbauth
 
А вы пробовали что то доказать Свидетелям Иеговы? Вот та же фигня
\end{itemize}

%%%fbauth
%%%fbauth_name
\iusr{Владимир Орлов}
%%%fbauth_url
%%%fbauth_place
%%%fbauth_id
%%%fbauth_front
%%%fbauth_desc
%%%fbauth_www
%%%fbauth_pic
%%%fbauth_pic portrait
%%%fbauth_pic background
%%%fbauth_pic other
%%%fbauth_tags
%%%fbauth_pubs
%%%endfbauth
 

Да, царские министры отличались редкой непоследовательностью! Что, в
результате, и сгубило царский режим. Например, тому же Валуеву и в голову не
пришло назначить языкового омбудсмена. Который зорко бы приглядывал, что бы в
шинках и лавках обслуживали строго на русском языке.))


%%%fbauth
%%%fbauth_name
\iusr{Oleg Kokotov}
%%%fbauth_url
%%%fbauth_place
%%%fbauth_id
%%%fbauth_front
%%%fbauth_desc
%%%fbauth_www
%%%fbauth_pic
%%%fbauth_pic portrait
%%%fbauth_pic background
%%%fbauth_pic other
%%%fbauth_tags
%%%fbauth_pubs
%%%endfbauth
 

Ну валуевский циркуляр, хоть художественную литературу оставил и театр, сейчас
театр, кино, концерты и все остальное попадает под запрет.


%%%fbauth
%%%fbauth_name
\iusr{Сергей Сахань}
%%%fbauth_url
%%%fbauth_place
%%%fbauth_id
%%%fbauth_front
%%%fbauth_desc
%%%fbauth_www
%%%fbauth_pic
%%%fbauth_pic portrait
%%%fbauth_pic background
%%%fbauth_pic other
%%%fbauth_tags
%%%fbauth_pubs
%%%endfbauth
 
Мне казалось что на польском всегда.

%%%fbauth
%%%fbauth_name
\iusr{Олег Ротанов}
%%%fbauth_url
%%%fbauth_place
%%%fbauth_id
%%%fbauth_front
%%%fbauth_desc
%%%fbauth_www
%%%fbauth_pic
%%%fbauth_pic portrait
%%%fbauth_pic background
%%%fbauth_pic other
%%%fbauth_tags
%%%fbauth_pubs
%%%endfbauth
 
Макс, ну какой в 1863 году украинский язык? В этом циркуляре вообще шла речь о малороссийском языке.

\ifcmt
  ig https://scontent-mia3-2.xx.fbcdn.net/v/t39.30808-6/224777114_4675004239199755_6478193956108929885_n.jpg?_nc_cat=105&ccb=1-3&_nc_sid=dbeb18&_nc_ohc=C4kKiJ7gZS8AX8ZZ9Ts&_nc_ht=scontent-mia3-2.xx&oh=944ec238329efa7c844625be112cf890&oe=610F1CA8
  width 0.4
\fi

%%%fbauth
%%%fbauth_name
\iusr{Михаил Будниченко}
%%%fbauth_url
%%%fbauth_place
%%%fbauth_id
%%%fbauth_front
%%%fbauth_desc
%%%fbauth_www
%%%fbauth_pic
%%%fbauth_pic portrait
%%%fbauth_pic background
%%%fbauth_pic other
%%%fbauth_tags
%%%fbauth_pubs
%%%endfbauth
 
На малороссийском наречии.

%%%fbauth
%%%fbauth_name
\iusr{Андрей Симончук}
%%%fbauth_url
%%%fbauth_place
%%%fbauth_id
%%%fbauth_front
%%%fbauth_desc
%%%fbauth_www
%%%fbauth_pic
%%%fbauth_pic portrait
%%%fbauth_pic background
%%%fbauth_pic other
%%%fbauth_tags
%%%fbauth_pubs
%%%endfbauth
 

Правильно император запретил, ибо Запад с помощью малороссийского наречия
навязывал сепаратистскую пропаганду, свою версию истории и т.д.

Запрещались только учебники, художественная литература не запрещалась.

Прошло полтора века, а ситуация примерно такая же.

%%%fbauth
%%%fbauth_name
\iusr{Стефан Машкевич}
%%%fbauth_url
%%%fbauth_place
%%%fbauth_id
%%%fbauth_front
%%%fbauth_desc
%%%fbauth_www
%%%fbauth_pic
%%%fbauth_pic portrait
%%%fbauth_pic background
%%%fbauth_pic other
%%%fbauth_tags
%%%fbauth_pubs
%%%endfbauth
 

1) Изучив украинский "закон" о языке, Валуев в гробу, безусловно, бьёт себя
костями пальцев по черепу и остатками челюстей артикулирует: "А что, так можно
было???"

2) Валуевский циркуляр НЕ ЗАПРЕЩАЛ украинский язык.

Вообще никак, ни в каком контексте, ни полностью, ни частично.

Убедиться в этом очень легко: прочитать первоисточник, т.е. текст циркуляра.
(Здесь выше и ссылку выложили.)

В тексте есть фразы о запретах, и есть фраза "украинский язык".

Они в разных частях текста. Никак вообще не связаны между собой.

\begin{itemize}
%%%fbauth
%%%fbauth_name
\iusr{Макс Бужанский}
%%%fbauth_url
%%%fbauth_place
%%%fbauth_id
%%%fbauth_front
%%%fbauth_desc
%%%fbauth_www
%%%fbauth_pic
%%%fbauth_pic portrait
%%%fbauth_pic background
%%%fbauth_pic other
%%%fbauth_tags
%%%fbauth_pubs
%%%endfbauth
 
\textbf{Stefan}, я ж не про циркуляр тут, а про людей, которые провозгласили его жупелом, и нарисовали свой, похлеще)

%%%fbauth
%%%fbauth_name
\iusr{Miroslava Berdnik}
%%%fbauth_url
%%%fbauth_place
%%%fbauth_id
%%%fbauth_front
%%%fbauth_desc
%%%fbauth_www
%%%fbauth_pic
%%%fbauth_pic portrait
%%%fbauth_pic background
%%%fbauth_pic other
%%%fbauth_tags
%%%fbauth_pubs
%%%endfbauth
 

Особенно, когда почитаешь - КАКОЙ язык реально запрещался в развитие
Валуевского циркуляра на местах: "Штраф за употребление служащими польского
языка

На днях управлением Юго-Западных железных дорог разослан приказ, из которого
видно, что кондуктор пассажирских поездов на ст. «Брест» Антон Стодульский
подвергнут 25-рублевому штрафу за то, что в поезде разговаривал с пассажиром на
польском языке. (КиевлянинЪ. — 1886. — №68).

Циркуляр Волынского губернатора мировым посредникам

До сведения моего доходит, что лица польского происхождения обращаются к
волостным и сельским урядникам с просьбами и жалобами на польском языке.
Признавая совершенно неуместным разбор дел в крестьянских управлениях на языке,
принадлежащем меньшинству населения, неудобопонятном для массы его, покорно
прошу вас, милостивый государь, внушить урядникам вверенного вам мирового
участка, что они не только не обязаны принимать к разбирательству дел по
жалобам, приносимым на ином языке, как на русском, но отнюдь не должны
допускать этого послабления, естественно возбуждающего недоверие крестьян и
поддерживающего в то же время у поляков притязание на преобладание на Волыни
польской речи. (Московские ведомости.— 1865. — №44).

Бердичевский календарь

Закрыли бердичевский календарь, который выходил огромным тиражом и был
распространителем польского печатного слова и латино-польских преданий даже в
массе сельского населения. Новый народный календарь на русском языке, при всех
недостатках, во всяком случае, и дешевле и неизмеримо выше уничтоженного
бердичевского... (КиевлянинЪ. — 1866. — №10).


%%%fbauth
%%%fbauth_name
\iusr{Стефан Машкевич}
%%%fbauth_url
%%%fbauth_place
%%%fbauth_id
%%%fbauth_front
%%%fbauth_desc
%%%fbauth_www
%%%fbauth_pic
%%%fbauth_pic portrait
%%%fbauth_pic background
%%%fbauth_pic other
%%%fbauth_tags
%%%fbauth_pubs
%%%endfbauth
 
\textbf{Макс Бужанский} Да, насчет "похлеще" — иначе и не скажешь
\end{itemize}

%%%fbauth
%%%fbauth_name
\iusr{Антон Лицис}
%%%fbauth_url
%%%fbauth_place
%%%fbauth_id
%%%fbauth_front
%%%fbauth_desc
%%%fbauth_www
%%%fbauth_pic
%%%fbauth_pic portrait
%%%fbauth_pic background
%%%fbauth_pic other
%%%fbauth_tags
%%%fbauth_pubs
%%%endfbauth
 
Ваш законопроект 5822-1 посмотрел, надо письма в поддержку писать. Пусть хоть в первом чтении примут

%%%fbauth
%%%fbauth_name
\iusr{Oleg Garkusha}
%%%fbauth_url
%%%fbauth_place
%%%fbauth_id
%%%fbauth_front
%%%fbauth_desc
%%%fbauth_www
%%%fbauth_pic
%%%fbauth_pic portrait
%%%fbauth_pic background
%%%fbauth_pic other
%%%fbauth_tags
%%%fbauth_pubs
%%%endfbauth
 
Авала хороший отель👍

%%%fbauth
%%%fbauth_name
\iusr{Слава Полянский}
%%%fbauth_url
%%%fbauth_place
%%%fbauth_id
%%%fbauth_front
%%%fbauth_desc
%%%fbauth_www
%%%fbauth_pic
%%%fbauth_pic portrait
%%%fbauth_pic background
%%%fbauth_pic other
%%%fbauth_tags
%%%fbauth_pubs
%%%endfbauth
 

Но мы же осколок Российской империи( СССР пытался восстановить разрушенное в
1917г)вот и копируем старину . Плохо получается потому что осколок.


%%%fbauth
%%%fbauth_name
\iusr{Александр Сошенко}
%%%fbauth_url
%%%fbauth_place
%%%fbauth_id
%%%fbauth_front
%%%fbauth_desc
%%%fbauth_www
%%%fbauth_pic
%%%fbauth_pic portrait
%%%fbauth_pic background
%%%fbauth_pic other
%%%fbauth_tags
%%%fbauth_pubs
%%%endfbauth
 
Апплодировать стоя нарушителям закона - это ваша сущьность?

\ifcmt
  ig https://scontent-mia3-1.xx.fbcdn.net/v/t39.30808-6/225635257_4165784976835666_7153413180919369862_n.jpg?_nc_cat=101&ccb=1-3&_nc_sid=dbeb18&_nc_ohc=Drh4nMFOG-EAX_lE55a&_nc_ht=scontent-mia3-1.xx&oh=aae12e1ea0b2917c1141192369d9f5b2&oe=61103D07
  width 0.4
\fi

\begin{itemize}
%%%fbauth
%%%fbauth_name
\iusr{Vitaliy Sv}
%%%fbauth_url
%%%fbauth_place
%%%fbauth_id
%%%fbauth_front
%%%fbauth_desc
%%%fbauth_www
%%%fbauth_pic
%%%fbauth_pic portrait
%%%fbauth_pic background
%%%fbauth_pic other
%%%fbauth_tags
%%%fbauth_pubs
%%%endfbauth
 
\textbf{Александр Сошенко} Спасибо Шарию за то что дал многим понять кто есть кто
\end{itemize}

%%%fbauth
%%%fbauth_name
\iusr{Марина Прохорова}
%%%fbauth_url
%%%fbauth_place
%%%fbauth_id
%%%fbauth_front
%%%fbauth_desc
%%%fbauth_www
%%%fbauth_pic
%%%fbauth_pic portrait
%%%fbauth_pic background
%%%fbauth_pic other
%%%fbauth_tags
%%%fbauth_pubs
%%%endfbauth
 

За 2 года монобольшинство так и не удосужилось отменить Парубиёвский закон про
мову, приняло дискриминационный закон про освиту, Шантажом вынудило КС признать
всё это конституционным, а теперь рассказывает нам про каких-то "тех людей". Вы
во главе с вашим президентом и есть "те люди".

\begin{itemize}
%%%fbauth
%%%fbauth_name
\iusr{Макс Бужанский}
%%%fbauth_url
%%%fbauth_place
%%%fbauth_id
%%%fbauth_front
%%%fbauth_desc
%%%fbauth_www
%%%fbauth_pic
%%%fbauth_pic portrait
%%%fbauth_pic background
%%%fbauth_pic other
%%%fbauth_tags
%%%fbauth_pubs
%%%endfbauth
 
\textbf{Marina Prokhorova} , ага.

А кто есть вы, умудрившаяся написать это единственному человеку, который пытался это изменить и отменить, а?

%%%fbauth
%%%fbauth_name
\iusr{Марина Прохорова}
%%%fbauth_url
%%%fbauth_place
%%%fbauth_id
%%%fbauth_front
%%%fbauth_desc
%%%fbauth_www
%%%fbauth_pic
%%%fbauth_pic portrait
%%%fbauth_pic background
%%%fbauth_pic other
%%%fbauth_tags
%%%fbauth_pubs
%%%endfbauth
 
\textbf{Макс Бужанский} 

Я русская гражданка Украины, дискриминируемая в собственной стране, за
независимость которой голосовала. Обманутая президентом, за которого
голосовала. А ваши попытки - это фиговый листочек, который ничего не
прикрывает. И вы это знаете лучше меня )


%%%fbauth
%%%fbauth_name
\iusr{Макс Бужанский}
%%%fbauth_url
%%%fbauth_place
%%%fbauth_id
%%%fbauth_front
%%%fbauth_desc
%%%fbauth_www
%%%fbauth_pic
%%%fbauth_pic portrait
%%%fbauth_pic background
%%%fbauth_pic other
%%%fbauth_tags
%%%fbauth_pubs
%%%endfbauth
 
\textbf{Marina Prokhorova} , 

меня за мои попытки раз в три дня обещают убить.

И это не фиговый листочек, а то, зачем я пришёл в парламент и то, что обещал.

Обычно я спрашиваю у людей, подобных вам, кто их мажоритарщик, и что они
сделали для того, чтобы он голосовал за эти инициативы.

Но вы даже лицо свое прячете, куда вам что то от кого то требовать кроме
претензий к тем, кто реально что то пытается сделать.

Вот и всё.
\end{itemize}

%%%fbauth
%%%fbauth_name
\iusr{Александр Коваленко}
%%%fbauth_url
%%%fbauth_place
%%%fbauth_id
%%%fbauth_front
%%%fbauth_desc
%%%fbauth_www
%%%fbauth_pic
%%%fbauth_pic portrait
%%%fbauth_pic background
%%%fbauth_pic other
%%%fbauth_tags
%%%fbauth_pubs
%%%endfbauth
 

Мой отец в 65 году учился в Днепропетровске, на физтехе, так москали, свлочи -
оккупанты, так боролись с укр. Мовою, что аж студенческие билеты писали на
украинском языке! У меня до сих пор сохранился это студенческой билет.

\begin{itemize}
%%%fbauth
%%%fbauth_name
\iusr{Валерий Чернолуцкий}
%%%fbauth_url
%%%fbauth_place
%%%fbauth_id
%%%fbauth_front
%%%fbauth_desc
%%%fbauth_www
%%%fbauth_pic
%%%fbauth_pic portrait
%%%fbauth_pic background
%%%fbauth_pic other
%%%fbauth_tags
%%%fbauth_pubs
%%%endfbauth
 
\textbf{Александр Коваленко} Я закончил ДГУ в 83-м. Насчёт билета и зачётки не помню, кажется всё же писались на русском. А вот диплом выписан на двух языках: на русском и украинском.
\end{itemize}

%%%fbauth
%%%fbauth_name
\iusr{Valeriy Sosnitsky}
%%%fbauth_url
%%%fbauth_place
%%%fbauth_id
%%%fbauth_front
%%%fbauth_desc
%%%fbauth_www
%%%fbauth_pic
%%%fbauth_pic portrait
%%%fbauth_pic background
%%%fbauth_pic other
%%%fbauth_tags
%%%fbauth_pubs
%%%endfbauth
 
Валуевский циркуляр запрещал не издательство, а ввоз издаваемых в Австрии
учебников и другой литературы.

\end{itemize}

