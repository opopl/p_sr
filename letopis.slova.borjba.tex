% vim: keymap=russian-jcukenwin
%%beginhead 
 
%%file slova.borjba
%%parent slova
 
%%url 
 
%%author 
%%author_id 
%%author_url 
 
%%tags 
%%title 
 
%%endhead 
\chapter{Борьба}
\label{sec:slova.borjba}

%%%cit
%%%cit_pic
%%%cit_text
Всю жизнь \emph{боролся}, боролся - и без результата... 14 червня 1891 року
народився Євген Коновалець. Забув, коли писав про Трампа та Че Гевару. Такий
же революціонер і \emph{борець}, як Че Гевара. Народився на території
Австро-Угорщини, воював у складі австро-угорської армії під час Першої світової
війни. Потравив у полон пді час бою на горів Маківка. З кінця 1917 року
полковник Армії УНР, командир Січових Стрільців, член Стрілецької Ради. Після
сепаратного миру галичан з Денікіним залишався вірним УНР. Покинув Петлюру лише
після того, як той уклав договір з Пілсудським про військову допомогу від
Польщі в обмін на західні землі. Командант УВО, голова Проводу українських
націоналістів (1927), перший голова ОУН (з 1929), один із ідеологів
українського націоналізму
%%%cit_comment
%%%cit_title
\citTitle{Судьба Евгения Коновальца типична для революционеров того времени}, 
Владимир Воля, strana.ua, 14.06.2021
%%%endcit

%%%cit
%%%cit_head
%%%cit_pic
%%%cit_text
К середине XII в. многочисленное семейство Рюриковичей разделилось на несколько
самостоятельных, хотя и родственных между собой «династий», четырьмя
крупнейшими из которых были — Суздальская («Юрьевичи»), Черниговская
(«Ольговичи»), Волынская («Изяславичи») и Смоленская («Ростиславичи»).
Собственно борьба за Киев вплоть до татарского нашествия велась между ними.
«Киевский стол» и титул «Великого князя Киевского» должен был принадлежать
старейшему в роде. Но «лествичная» система наследования (не от отца к сыну, а
от брата к брату) делала вопрос крайне запутанным. Поэтому занятие «киевского
стола», как правило, сопровождалось усобицами, чаще всего — столкновениями
дядьёв с племянниками и войнами кузенов
%%%cit_comment
%%%cit_title
\citTitle{Синдром киевской исторической неполноценности}, 
Олег Сапожников, regnum.ru, 21.07.2021
%%%endcit
