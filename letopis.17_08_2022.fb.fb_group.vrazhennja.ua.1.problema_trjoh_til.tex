% vim: keymap=russian-jcukenwin
%%beginhead 
 
%%file 17_08_2022.fb.fb_group.vrazhennja.ua.1.problema_trjoh_til
%%parent 17_08_2022
 
%%url https://www.facebook.com/groups/vrajenniya.ua/posts/1684888958536203
 
%%author_id fb_group.vrazhennja.ua,dzhurdzha_denys
%%date 
 
%%tags chtenie,fantastika,fizika,kniga,literatura,nauka
%%title Лю Цисінь - Проблема трьох тіл
 
%%endhead 
 
\subsection{Лю Цисінь - Проблема трьох тіл}
\label{sec:17_08_2022.fb.fb_group.vrazhennja.ua.1.problema_trjoh_til}
 
\Purl{https://www.facebook.com/groups/vrajenniya.ua/posts/1684888958536203}
\ifcmt
 author_begin
   author_id fb_group.vrazhennja.ua,dzhurdzha_denys
 author_end
\fi

Лю Цисінь

Проблема трьох тіл

От починаючи читати книгу я знав про неї не так вже й багато, лиш те, що це має
бути крута тверда наукова фантастика, і те, що мова там іде про інопланетне
вторгнення. І от уявіть моє здивування, коли з цією інформацією про книгу, я
почав читати першу главу і побачив що в ній іде мова про китайську революцію і
загони хунвейбінів…

І хоча потім події, все ж, перенеслися в дуже недалеке майбутнє, здивування не
закінчились і навіть навпаки. Коли в одного з головних героїв, Ван Мяо, починає
показуватись на фотографіях зворотній відлік, то виникає підозра, що я читаю
якесь фентезі, в котрому можливий магічний влив на реальність, бо як ще, окрім
магії, пояснити те, що коли на один і той же фотоапарат робить знімки Ван Мяо,
то зворотній відлік є, а на тих кадрах, коті робить його дружина – вже нема? А
коли з'являється комп'ютерна гра «Три тіла», то виникає підозра, що я читаю
якийсь кіберпанк, в котрому мова іде про додаткові реальності, а не про
інопланетне вторгнення… Але оці всі підозри, оці всі незрозумілості приводять
до того, що читати книгу неймовірно цікаво, і сторінки просто пролітають одна
за одною.

Особисто мені найбільше сподобалася саме загадка того зворотного відліку,
оскільки раз і сам автор, і всі літератори стверджують, що дана книга – це
наукова фантастика, то це ж має бути якось науково пояснено, ні? І саме тому
мені хотілося чим пошвидше дочитати до того моменту, де буде пояснено яким же
це чином Трисоляріани змогли так швидко, з такої величезної відстані впливати
на матерію на Землі, щоб транслювати відлік на фото, а потім навіть і напряму
на сітківку отого Вана Мяо. А ще більше зацікавлення з'явилося, коли вони
змусили замиготіти реліктове випромінювання! Мозок просто кидався із сторони в
сторону в спробах зрозуміти – як?? Але після опису діалогу Ван Мяо із Дін Ї, в
якому останній поділився своїми результатами, абсолютно алогічними
результатами, досліджень руху елементарних частин, я признав свою поразку, і
зрозумів що сам точно не додумаюся до пояснення.

І от коли таки прийшов час пояснення, то воно не те що не розчарувало, воно
викликало абсолютний дикий захват! Та от ідея з Софонами, з квантовими
комп'ютерами зробленими на «розгорнутих» протонах – це просто шедевр, вона так
елегантно пояснює всі ті дивні речі, що відбувалися на Землі, що це викликає
просто неймовірний захват! Я про це читав, і в голові просто волав сам до себе:
«Так! Саме так! Тепер все зрозуміло!». І як на мене – це дійсно геніально, я
схиляю голову перед інтелектом, обізнаністю, фантазією і логікою Лю Цисіня.

До речі, щоб краще зрозуміти і навіть уявити собі і теорії фізиків про
дванадцятивимірність всесвіту, в якому ми живемо, і ті загравання з вимірами
протонів, які описав Цисінь, я б порадив почитати книгу «Флатландія» Едвіна
Еббота, бо хоч вона і написана більш ніж століття тому, але ті всі нюанси з
різницями вимірів описані в ній так, що читач дійсно, не лиш розуміє, а й
уявляє як це працює.

Ну і ще один момент, який викликав в мене неймовірний захват – це побудова
комп'ютера з людей, ці от мільйони солдат, яких вистроїли в певному порядку,
навчили елементарним командам, тим самим сформувавши щось, на подобі
комп'ютера, основаного на архітектурі фон Неймана – це шедевр, давно мені не
траплялися в книгах подібні оригінальні ідеї.

Хоча, стосовно логіки, був один момент котрий мене трішечки стурбував – це така
собі тупенька невеличка помилочка, причому настільки тупенька, що в мене навіть
виникла підозра, що це помилка не автора, а перекладачів. Коли розроблялися
плани по захопленню корабля ОЗТ «Судний день» і висловилася ідея про
використання так званого «Літаючого леза», надтонкого і надміцного
наноматеріалу, то спочатку Ван Мяо був заявив, цитую: «За сьогоднішнього
розвитку технологій єдина форма, яку ми можемо виготовити – це нитка», але
через дві сторінки він уже говорить: «у нас є певна кількість листового
«Літаючого леза». Нестиковочка трішки)

А в загальному ж, книга залишила по собі виключно позитивні враження, я не
пам'ятаю коли востаннє я читав щось настільки захопливе і я з захватом берусь
за прочитання наступної частини.

\#Цисінь \#ЛюЦисінь \#ПамятьПроМИнулеЗемлі \#ПроблемаТрьохТіл \#проблема \#атака \#вторгнення \#наука \#фантастика \#НауковаФантастика \#фізика \#життя \#всесвіт
