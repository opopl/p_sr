% vim: keymap=russian-jcukenwin
%%beginhead 
 
%%file 26_07_2020.news.ua.1
%%parent 26_07_2020
 
%%endhead 
  
\subsection{``Влада чомусь повірила, що Україну можна здати ворогам''. Ярош звернувся до керівництва держави та військових}
\label{sec:26_07_2020.news.ua.1}
\url{https://espreso.tv/news/2020/07/26/quotvlada_chomus_poviryla_scho_ukrayinu_mozhna_zdaty_vorogamquot_yarosh_zvernuvsya_do_kerivnyctva_derzhavy_ta_viyskovykh}

\index{СМИ!Украина!espreso.tv}
  
Командир Української добровольчої армії (УДА), політичний та громадський діяч
Дмитро Ярош оприлюднив звернення до вищих посадових осіб та вояків на передовій
і в тилу з приводу призначеного з 00:01 понеділка, 27 липня 2020 року, "повного
і всеосяжного" перемир’я з росіянами на Донбасі 

Про це передає Еспресо.TV.

"Влада чомусь повірила, що Україну можна здати ворогам. Звідти всі процеси, що
йдуть крайніми днями на Фронті російсько-української війни. Пам’ятаю, як в
14-му зруйновану Армію теж роззброювали і деморалізовували. Тоді вперед вийшли
добровольці. І ми зупинили ворога, і відкинули його на сотні кілометрів із
займаних позицій. Зараз, якщо політичне керівництво держави, що тимчасово
перебуває при владі, захоче стати "Зрадниками", що ж - ми дамо цьому раду", -
зазначив він.

За словами Яроша, для того щоб російські війська не зайняли нових позицій,
українські воїни мають "не виконувати злочинних антиконституційних і
антидержавних наказів", а все суспільство має виступити проти капітуляції.

"Брати-воїни: військовослужбовці Збройних Сил України, Національної гвардії,
Служби безпеки України, працівники МВС, добровольці, держава знову у
небезпеці!" - наголосив політик.

У тому випадку, якщо влада не відмовиться від своїх нинішніх планів, додав
Ярош, воювати знову вирушають добровольці.

"Люди війни, всі патріоти, які вміють тримати у руках зброю та готові знову
стати на захист самостійності та соборності нашої держави", - впевнений
командир УДА.

\begin{itemize}
		\item Нагадаємо, 22 липня 2020 року відбулося чергове засідання Тристоронньої
		контактної групи у форматі відеоконференції. Згідно з ухваленим рішенням, з
		00:01 понеділка, 27 липня, тобто вже за день, має дотримуватися режим
		повного та всеосяжного припинення вогню.
								\url{https://espreso.tv/news/2020/07/22/trystoronnya_grupa_domovylasya_pro_quotpovne_prypynennya_vognyuquot_na_donbasi_z_27_lypnya}

		\item Як заявив ввечері у п'ятницю, 24 липня, засновник та перший командир
		добровольчого полку "Азов" Андрій Білецький з українського боку вже відвели
		аеророзвідку, спецназ та снайперів.

		\item Окрім того, за словами Білецького, нові обмеження стосуються не лише ЗСУ, а
		для стримування активності бійців "на кожному взводно-опорному пункті
		сідають штабні офіцери з завданням не допускати вогонь, залякувати офіцерів
		та особовий склад підрозділів".

		\item Сьогодні, 26 липня, у Міноборони підтвердили, що за дотриманням перемир'я з
		боку України стежитимуть спеціальні офіцери.

		\item Попри це обстріли українських позицій з боку росіян досі продовжуються.
		Протягом учорашнього дня, 25 квітня, вони 15 разів відкривали вогонь зі
		стрілецької зброї, гранатометів та мінометів. Подібне мало місце поблизу 9
		сіл та селищ.
								\url{https://espreso.tv/news/2020/07/26/za_den_do_quotvseosyazhnogo_prypynennya_vognyuquot_z_rosiyskogo_boku_15_obstriliv_z_strileckoyi_zbroyi_granatometiv_i_minometiv_poblyzu_9_sil_ta_selysch}

		\item Українське ветеранське і націонал-патріотичне середовище вважає умови
		перемир'я несправедливим, а те, що за ними послідує, підступною тактикою
		Кремля.
		\url{https://espreso.tv/news/2020/07/26/quotza_nayivnist_zelenskogo_budut_platyty_nashi_soldatyquot_patrioty_z_kharkova_pro_nove_peremyrya}

		\item На завтра Рух Опору Капітуляції скликає віче під стінами Офісу президента у
		Києві (ву. Банкова, 11).
								\url{https://espreso.tv/news/2020/07/26/quotprezydentu_dovedetsya_abo_pity_na_nashi_vymogy_abo_pakuvaty_valizy_v_rostovquot_rukh_oporu_kapitulyaciyi_klychne_nebayduzhykh_na_bankovu}
\end{itemize}

Слідкуйте за подіями в Україні та світі разом з Еспресо! Підписуйтесь на
Telegram-канал: \url{https://t.me/espresotb}
