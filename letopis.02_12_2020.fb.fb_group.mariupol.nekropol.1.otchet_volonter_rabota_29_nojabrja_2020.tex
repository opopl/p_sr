%%beginhead 
 
%%file 02_12_2020.fb.fb_group.mariupol.nekropol.1.otchet_volonter_rabota_29_nojabrja_2020
%%parent 02_12_2020
 
%%url https://www.facebook.com/groups/278185963354519/posts/424249218748192
 
%%author_id fb_group.mariupol.nekropol,arximisto
%%date 02_12_2020
 
%%tags 
%%title Отчет о волонтерской работе в Некрополе 29 ноября 2020
 
%%endhead 

\subsection{Отчет о волонтерской работе в Некрополе 29 ноября 2020}
\label{sec:02_12_2020.fb.fb_group.mariupol.nekropol.1.otchet_volonter_rabota_29_nojabrja_2020}
 
\Purl{https://www.facebook.com/groups/278185963354519/posts/424249218748192}
\ifcmt
 author_begin
   author_id fb_group.mariupol.nekropol,arximisto
 author_end
\fi

\textbf{Отчет о волонтерской работе в Некрополе 29 ноября 2020}

В прошлое воскресенье мы воспользовались теплой погодой и плодотворно поработали в Некрополе...

\textbf{Открытия и находки}

Елена Сугак вместе с Наталия Шпотаковская продолжили расчистку старинной
усыпальницы рядом со склепом Александра Хараджаева, городского головы в 1860-64
гг., купца и мецената. Обломки ранее найденных плит с надписями позволяют
предположить, что эта усыпальница принадлежала роду Хараджаевых. Однако их
недостаточно, чтобы сделать окончательные выводы...

Александр и Наталья Шпотаковские обследовали восточную часть Некрополя и
обнаружили целый ряд старинных плит и надгробий. Вот список фамилий (фотографии
опубликуем чуть позже в отдельном посте):

\begin{itemize}
	\item Анастас Иосифович Шоте-Четенов, 1863-1918
	\item Исай Степанович Каракаш, 1845 - 1 мая 1891
	\item Константин Пичахчи, 1809 – 1891
	\item Кириякия ... Юрьева, ум. 1884
	\item ... Карпович Юрьев, ум. 1906
	\item Прасковья Антоновна Хара
	\item Екатерина Михайловна Волошина, 23 лет от роду (без дат)
\end{itemize}

Если вы узнали своих родственников, мы сможем показать, где находятся эти
захоронения.

\textbf{Благоустройство}

Maryna Holovnova вместе с Андреем Марусовым покрасила практически все кресты на
заброшенных могилах древнего участка! Посмотрите фото \enquote{до} и \enquote{после} - вид
участка сразу же преобразился!

\textbf{Дальнейшие планы}

Расчистка центральной части Некрополя от зарослей (от древнего участка вверх -
к памятному Кресту), покраска, уборка мусора и, конечно же, поиск и
обследование старинных могил – ключевые приоритеты волонтерской работы.

Из-за холодов мы вынуждены считаться с погодой. Любые теплые выходные – и мы
возобновляем работу!

Огромная благодарность всем волонтерам!

\textbf{Пожертвования и их использование}

На днях мы получили еще одно пожертвование от Барановой Ольги Михайловны!
Огромное Вам спасибо!

Таким образом, на сегодня 15 человек, а также команда Вежа - Vezha Creative
Space перечислили 6 250 грн. на восстановление Некрополя. Мы потратили 1 396,92
грн. (см. о расходах
\footnote{\url{https://www.facebook.com/groups/278185963354519/posts/410948413411606}} %
\footnote{Internet Archive: \url{https://archive.org/details/13_11_2020.fb.arximisto.mrpl_nekropol.pozhertvovania_otchet_1}}
).

Большое спасибо всем благотворителям!

На этой неделе мы планируем закупить все расходные материалы (масло,
растворитель, банку белой краски и т.п.) ввиду возможного роста цен с началом
следующего года...

\#mariupol\_necropolis\_report
\#mariupol\_necropolis\_spending
