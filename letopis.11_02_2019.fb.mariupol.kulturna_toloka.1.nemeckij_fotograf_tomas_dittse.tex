%%beginhead 
 
%%file 11_02_2019.fb.mariupol.kulturna_toloka.1.nemeckij_fotograf_tomas_dittse
%%parent 11_02_2019
 
%%url https://www.facebook.com/100029107552223/posts/pfbid02Bj4oD57gtdEhgfAo1K7C74QpZAt6bRqmKpE26ngNaC4KvhPECqrFzQDWUuzSE6ycl
 
%%author_id mariupol.kulturna_toloka
%%date 11_02_2019
 
%%tags 
%%title Немецкий фотограф Томас Дитце
 
%%endhead 

\subsection{Немецкий фотограф Томас Дитце}
\label{sec:11_02_2019.fb.mariupol.kulturna_toloka.1.nemeckij_fotograf_tomas_dittse}

\Purl{https://www.facebook.com/100029107552223/posts/pfbid02Bj4oD57gtdEhgfAo1K7C74QpZAt6bRqmKpE26ngNaC4KvhPECqrFzQDWUuzSE6ycl}
\ifcmt
 author_begin
   author_id mariupol.kulturna_toloka
 author_end
\fi

Никому не нужный заброшенный завод или современный арт-объект? Опытом
\enquote{оживления} заброшенных индустриальных объектов поделился немецкий фотограф,
куратор фестиваля современного урбанистического искусства ibug Томас Дитце. Уже
более 10 лет художники из многих стран мира дарят вторую жизнь старым заводским
помещениям, превращая их в модные креативные пространства.
