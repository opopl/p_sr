% vim: keymap=russian-jcukenwin
%%beginhead 
 
%%file slova.cenzura
%%parent slova
 
%%url 
 
%%author 
%%author_id 
%%author_url 
 
%%tags 
%%title 
 
%%endhead 
\chapter{Цензура}
\label{sec:slova.cenzura}

%%%cit
%%%cit_head
%%%cit_pic
%%%cit_text
Из того же смыслового ряда - 15-я статья Конституции. В ней говорится, что
никакая идеология не является обязательной, а \emph{цензура} запрещена.
Однако в Украине события развиваются в последние годы совсем в обратном
направлении.
\emph{Цензуру} осуществляют, во-первых, националисты, которые атакуют неугодных им
артистов, блогеров или простых людей, высказывающих публично или в соцсетях
\enquote{неправильную} позицию. Радикалам в этом помогает власть - хотя бы тем, что
закрывает глаза на их действия и не привлекает к ответственности.
Второй пласт \emph{цензуры} - это списки неугодных лиц, куда входят писатели, актеры,
музыканты. Их ведут Минкульт и СБУ. Таких людей запрещено не только пускать в
страну, но и показывать по телевидению. Это \emph{прямая цензура}
%%%cit_comment
%%%cit_title
\citTitle{День Конституции Украины 28 июня - какие статьи нарушаются сильнее всего}, 
Оксана Малахова; Максим Минин, strana.ua, 28.06.2021
%%%endcit

