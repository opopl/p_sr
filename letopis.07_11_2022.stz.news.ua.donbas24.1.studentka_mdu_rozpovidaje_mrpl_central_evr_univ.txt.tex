% vim: keymap=russian-jcukenwin
%%beginhead 
 
%%file 07_11_2022.stz.news.ua.donbas24.1.studentka_mdu_rozpovidaje_mrpl_central_evr_univ.txt
%%parent 07_11_2022.stz.news.ua.donbas24.1.studentka_mdu_rozpovidaje_mrpl_central_evr_univ
 
%%url 
 
%%author_id 
%%date 
 
%%tags 
%%title 
 
%%endhead 

Ольга Демідко (Маріуполь)
07_11_2022.olga_demidko.donbas24.studentka_mdu_rozpovidaje_mrpl_central_evr_univ
Маріуполь,Україна,Мариуполь,Украина,Mariupol,Ukraine,Mariupol.MSU,Маріуполь.МДУ,Мариуполь.МГУ,Освіта,Образование,Education,Europe,Європа,Европа,date.07_11_2022

Студентка МДУ розповідає про Маріуполь в Центрально-Європейському університеті
(ФОТО)

Студентка спеціальності «Культурологія» Маріупольського університету вже вдруге
стала учасницею програми «Invisible University for Ukraine»

Студентка 3 курсу спеціальності «Культурологія» Маріупольського державного
університету Софія Шустенко, виїхавши з Маріуполя до Вільнюса, одразу ж почала
брати активну участь в освітніх заходах за кордоном. Зокрема, з 10 до 19 липня
дівчина відвідувала літню школу Будапешта. Наприкинці квітня вона завітала до
Центрально-Європейського університету на спеціальний курс для студентів та
науковців з України, обравши напрям «Збереження культурної спадщини».
Мотиваційний лист-есе Софії про те, що вона хотіла б досліджувати в рамках
літньої школи, приємно вразив організаторів, які запросили її стати учасницею
школи.

Читайте також: Студент МДУ Олександр Денисов завоював 14 медалей на двох
чемпіонатах Німеччини

Які можливості відкривають українським студентам Літня та Зимова школи
Центрально-Європейського університету?

Протягом семестру студенти з різних університетів України та різних ступенів
освіти мають змогу навчатися та отримувати актуальні знання від професорів
європейських та американських університетів за такими напрямами, як: політика,
соціологія, історія та культурологія.

Для Софії участь в Літній школі стала чудовою можливістю не тільки доповнити
свої знання та покращити англійську, а й поспілкуватися особисто з професорами,
студентами-менторами ЦЄУ, що супроводжували навчання, доєднатися до дискусій.
До того ж, були запропоновані зустрічі з академічної англійської, де викладачі
допомагали включитися у процес, розібратися в лекції, якщо через низький рівень
щось незрозуміло, та покращити рівень знання мови.

«Для мене програма стала мотиватором рухатися далі, взятися за дослідження,
замислитися, а головне, розібрати актуальні процеси сучасного світу. Не можу
сказати, що зараз замало міжнародних програм для культурологів та істориків,
але, з мого досвіду, точно можу сказати, що це однозначно неймовірна можливість
для тих, хто хоче стати спеціалістом або поринути у наукову діяльність і
додатково зрозуміти, як викладають іноземні університети», — розповідає Софія.

Читайте також: Як під час війни працює Маріупольський державний університет?

Крім того, програма надає можливість поїхати до Літньої або Зимової школи ЦЄУ
до кампусу в Будапешті та отримати стипендію на всі витрати поїздки. Програма
таких шкіл дуже насичена: лекції починалися о 10 ранку, а до гуртожитку
студенти поверталися десь о 23, адже після лекцій можна було неформально
поспілкуватися з викладачами та менторами школи, зі студентами, обговорити
найрізноманітніші питання і поділитися своїми думками або досвідом. Крім
теоретичних питань, також розв'язували і практичні кейси. Наприклад, в один з
днів школи відбулося обговорення історії Маріуполя, його суспільного життя, а
також одне з найболючіших та все ж найактуальніших питань: «Як меморалізувати
завод „Азовсталь“?».

«У своїй команді я мала змогу більш детально розповісти про життя міста до
повномасштабного наступу, поглибитись в історію міста та самого заводу. Це
допомогло студентам, багато з яких, нажаль, почули про місто тільки після 24
лютого, сформувати більш чітке розуміння Маріуполя та запропонувати ідеї
стосовно „Азовсталі“. Для мене ж цей кейс сприяв розумінню бачення іншими твого
рідного міста, а також що саме може бути у перспективі відновлення та чому все
ж важливо залучати місцевих діячів», — зазначила Софія.

Докладніше про програму Літньої та Зимової можна ознайомитися за посиланням.

Читайте також: В Україні запускають Google Знання: чому можна навчитися на
освітньому хабі

В яких ще заходах взяла участь Софія?

З 14 до 25 вересня студентка взяла участь у проєкті Еразмус+, темою якого стало
ментальне здоров'я. Організаторами виступали латвійська молодіжна організація
YoungFolks та українська громадська організація «Крок. ВПЕРЕД»

Учасниками заходу стали молодь з України, Латвії та Італії.

Протягом двох тижнів національні команди проводили тематичні тренінги на
покращення стану ментального здоров'я та командоутворення. Найголовніша мета
молодіжного обміну — допомогти підліткам позбутися стресу і усвідомити, що люди
побороли свої проблеми, стали витривалішими завдяки своїм знанням. Також одним
із завдань було познайомити молодь з неформальною освітою і дати зрозуміти, що
всю інформацію можна отримати з інтересом і на практиці.

Водночас команди проводили тематичні вечори, присвячені культурам та традиціям
країн-учасниць, мали змогу послухати екскурсію містами Італії, відвідати
Адріатичне та Іонічне море.

Пам'ятка культуролога-переселенця від Софії

Також студентка 3 курсу Софія Шустенко створила унікальну Пам'ятку
культуролога-переселенця:

1. Спостерігай за всім, що є у місті. Стоячи на площі або вулиці тебе оточують
не тільки будівлі, а й люди. Їхні особливості у зовнішньому вигляді, стилі
тощо, допоможуть тобі зрозуміти особливості регіону.

2. Доки їдеш до міста, почитай трішки про саме місто. Коротка інформація
допоможе тобі, культурологу, орієнтуватися та визначати важливі пам'ятки нового
міста.

3. Де б ти не був, звертай увагу на архітектуру. Вона може розповісти набагато
більше про розвиток міста та його життя, аніж тобі здається.

4. Ходи пішки. Так ти зможеш поглинутись у ритм міста та відчути його дух.

Читайте також: Школи, виші та садочки України переходять на дистанційку: деталі

5. Відвідай хоча б один музей. Зупиняєшся більше ніж на декілька днів?
Обов'язково пошукай та відвідай місцевий краєзнавчий музей, адже можна
подивитися наживо пам'ятки та навіть поспілкуватися з робітниками музею, які
можуть розповісти щось цікаве, що не знайдеш в інтернеті.

6. Не бійся мандрувати містом. Відвідай спочатку найбільш туристичні місця, а
потім, потроху вивчаючи місто, обери те, що вважаєш визначним саме ти.

7. Читай місцеві новини. З них можна дізнатися про культурні події, перфоманси,
тощо.

8. Використовуй соціальні мережі. Якщо зупиняєшся в місті надовго, пошукай
місцевих митців, особливо сучасного мистецтва.

9. Поділись тим, що ти дізнався про місто. Не бійся зробити невеликий пост у
фейсбуці або сторіс в інстаграм. Як культуролог, що вимушено перебуває в іншому
місті, ти можеш стати очима інших людей, звернути увагу, навіть якщо саме місто
досить невелике.

Нагадаємо, що у Дії з'являть електронні версії дипломів та атестатів, які можна
буде показати за місцем вимоги.

Ще більше новин та найактуальніша інформація про Донецьку та Луганську області
в нашому телеграм-каналі Донбас24.

ФОТО: з особистого архіву Софії Шустенко.
