% vim: keymap=russian-jcukenwin
%%beginhead 
 
%%file 01_11_2022.stz.news.ua.donbas24.1.mrpl_hudozhnyci_nova_vystavka_dortmund
%%parent 01_11_2022
 
%%url https://donbas24.news/news/mariupolski-xudoznici-predstavili-novu-vistavku-v-dortmundi
 
%%author_id demidko_olga.mariupol,news.ua.donbas24
%%date 
 
%%tags 
%%title Маріупольські художниці представили нову виставку в Дортмунді
 
%%endhead 
 
\subsection{Маріупольські художниці представили нову виставку в Дортмунді}
\label{sec:01_11_2022.stz.news.ua.donbas24.1.mrpl_hudozhnyci_nova_vystavka_dortmund}
 
\Purl{https://donbas24.news/news/mariupolski-xudoznici-predstavili-novu-vistavku-v-dortmundi}
\ifcmt
 author_begin
   author_id demidko_olga.mariupol,news.ua.donbas24
 author_end
\fi

\ii{01_11_2022.stz.news.ua.donbas24.1.mrpl_hudozhnyci_nova_vystavka_dortmund.pic.front}
\begin{center}
  \em\color{blue}\bfseries\Large
У німецькому місті Дортмунд відкрилася виставка, де можна побачити роботи,
присвячені життю в Маріуполі під час війни
\end{center}

28 жовтня в Шахті-музеї \enquote{Цехе Цоллерн}, у Дортмунді, відкрилася виставка
PostMost, яка була створена учасницями \href{https://archive.org/details/11_10_2022.olga_demidko.donbas24.mrpl_art_rezidencia_postmost_vidnov_dijalnist}{\emph{попередніх проєктів}}%
\footnote{Маріупольська арт-резиденція PostMost відновлює свою діяльність, Ольга Демідко, donbas24.news, 11.10.2022, %
\par\url{https://donbas24.news/news/mariupolska-art-rezidenciya-postmost-vidnovlyuje-svoyu-diyalnist}, \par%
Internet Archive: \url{https://archive.org/details/11_10_2022.olga_demidko.donbas24.mrpl_art_rezidencia_postmost_vidnov_dijalnist}%
} в період арт-резиденції в гостинному просторі ArToll міста Бедбург-Хау. Про це
\href{https://www.facebook.com/profile.php?id=100002254840110}{повідомила} маріупольська художниця Олена Украінцева.

\begin{leftbar}
\emph{\enquote{Мета виставки — з одного боку вкотре привернути увагу світу до подій, що
відбуваються в Україні, емоційно включити глядачів в ситуацію. З іншого
боку — це цілюща зустріч для маріупольських художниць та важливий крок
в їхній реабілітації після пережитих жахів}}, — наголосила Олена.
\end{leftbar}

\ii{01_11_2022.stz.news.ua.donbas24.1.mrpl_hudozhnyci_nova_vystavka_dortmund.pic.1}
\ii{01_11_2022.stz.news.ua.donbas24.1.mrpl_hudozhnyci_nova_vystavka_dortmund.pic.2}

\ii{insert.read_also.demidko.donbas24.futbolki_nimecchyna_mrpl}

Учасниці резиденції здебільшого є маріупольськими художницями, що через війну
залишили рідне місто та наразі перебувають в Європі. На виставці у вигляді
постерів були представлені і роботи маріупольських художників, які перебувають
в Україні.

У виставці взяли участь Антоніна Дурнєва, Анастасія Шишкіна, Олександр
Малаховський, Hilmenna, Simona, Інна Абрамова, Марина Черепченко, Ivetta Becker
та Олена Украінцева. Перед виставкою художниці протягом двох тижнів працювали
разом в арт-резиденції АрТолл в Бедбург-Хау.

\ii{01_11_2022.stz.news.ua.donbas24.1.mrpl_hudozhnyci_nova_vystavka_dortmund.pic.3}
\ii{01_11_2022.stz.news.ua.donbas24.1.mrpl_hudozhnyci_nova_vystavka_dortmund.pic.4}
\ii{01_11_2022.stz.news.ua.donbas24.1.mrpl_hudozhnyci_nova_vystavka_dortmund.pic.5}

\textbf{Читайте також:} \href{https://donbas24.news/news/okupanti-v-mariupoli-maize-znishhili-mural-milana-navishho-voni-ce-roblyat-foto}{\emph{Окупанти в Маріуполі майже знищили мурал \enquote{Мілана}: навіщо вони це роблять}}%
\footnote{Окупанти в Маріуполі майже знищили мурал \enquote{Мілана}: навіщо вони це роблять, Яна Іванова, donbas24.news, 01.11.2022, \par%
\url{https://donbas24.news/news/okupanti-v-mariupoli-maize-znishhili-mural-milana-navishho-voni-ce-roblyat-foto}%
}

\ii{01_11_2022.stz.news.ua.donbas24.1.mrpl_hudozhnyci_nova_vystavka_dortmund.pic.6}
\ii{01_11_2022.stz.news.ua.donbas24.1.mrpl_hudozhnyci_nova_vystavka_dortmund.pic.7}

Також на виставці можна побачити інсталяцію-розповідь про життя в Маріуполі під
час війни.

\begin{leftbar}
\emph{\enquote{Це окрема болюча тема як для митців, так і для глядачів. Троє з наших
художниць виживали в Маріуполі в підвалах. Це жахіття їм довелося
згадувати, аби відтворити умови побуту, в яких знаходилися деякий час}},
— розповіла Олена Украінцева.
\end{leftbar}

Звуки обстрілів доповнювали загальне враження. Так художники намагалися більш
детально донести сенси, що закодовані в їхніх роботах. На відкриття виставки
прийшло багато українців-біженців з різних міст України. Також захід відвідали
й ті українці, які приїхали жити до Німеччини ще до повномасштабного вторгнення
рф в Україну. Всі глядачі залишилися під сильним враженням від побаченого.

\textbf{Читайте також:} \href{https://donbas24.news/news/xudoznik-denis-metelin-dolucivsya-do-blagodiinogo-projektu-zi-zboru-kostiv-dlya-zsu-video}{\emph{Художник Денис Метелін долучився до благодійного проєкту зі збору коштів для ЗСУ}}%
\footnote{Художник Денис Метелін долучився до благодійного проєкту зі збору коштів для ЗСУ, Алевтина Швецова, donbas24.news, 05.09.2022, \par%
\url{https://donbas24.news/news/xudoznik-denis-metelin-dolucivsya-do-blagodiinogo-projektu-zi-zboru-kostiv-dlya-zsu-video}%
}

\ii{01_11_2022.stz.news.ua.donbas24.1.mrpl_hudozhnyci_nova_vystavka_dortmund.pic.8}
\ii{01_11_2022.stz.news.ua.donbas24.1.mrpl_hudozhnyci_nova_vystavka_dortmund.pic.9}
\ii{01_11_2022.stz.news.ua.donbas24.1.mrpl_hudozhnyci_nova_vystavka_dortmund.pic.10}

\begin{leftbar}
\emph{\enquote{Я сам з України, до Німеччини приїхав багато років назад. Для мене все, що
відбувається в Україні — велика трагедія. Ця виставка — це вперше за
всі 8 місяців така потужна рефлексія-відповідь від маріупольців. Роботи
художників дозволили по-справжньому зрозуміти, що пережили люди. Ми
бачимо на прикладі маріупольців, які виїхали, що українці намагаються
висвітлювати правду. І це дає велику надію нам тут}}, — поділився
думками відвідувач виставки Олексій Кісліцин.
\end{leftbar}

Виставка експонуватиметься до \textbf{24 березня 2023 року.}

\ii{01_11_2022.stz.news.ua.donbas24.1.mrpl_hudozhnyci_nova_vystavka_dortmund.pic.11}

Раніше Донбас24 розповідав, як маріупольський художник\par\noindent\href{https://archive.org/details/24_10_2022.olga_demidko.donbas24.kartyny_oleksandr_lukjanov}{Олексій Лукьянов відтворює трагедію Маріуполя}%
\footnote{Поранене та ув'язнене місто у роботах маріупольського художника, Ольга Демідко, donbas24.news, 24.10.2022, \par%
\url{https://donbas24.news/news/lorem-ipsum-is-placeholder-text-commonly-used-in-the-graphic}, \par%
Internet Archive: \url{https://archive.org/details/24_10_2022.olga_demidko.donbas24.kartyny_oleksandr_lukjanov}%
} у своїх картинах.

Ще більше новин та найактуальніша інформація про Донецьку та Луганську області
в нашому телеграм-каналі Донбас24.

ФОТО: з архіву учасниць проєкту.

\ii{insert.author.demidko_olga}
%\ii{01_11_2022.stz.news.ua.donbas24.1.mrpl_hudozhnyci_nova_vystavka_dortmund.txt}
