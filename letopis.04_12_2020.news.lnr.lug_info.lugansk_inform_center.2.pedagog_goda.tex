% vim: keymap=russian-jcukenwin
%%beginhead 
 
%%file 04_12_2020.news.lnr.lug_info.lugansk_inform_center.2.pedagog_goda
%%parent 04_12_2020
 
%%url http://lug-info.com/news/one/mer-luganska-nagradil-pobeditelei-vtorogo-etapa-konkursa-pedagog-goda-luganschiny-62520
 
%%author ЛуганскИнформЦентр
%%author_id lugansk_inform_center
%%author_url 
 
%%tags lnr
%%title Мэр Луганска наградил победителей второго этапа конкурса "Педагог года Луганщины"
 
%%endhead 
 
\subsection{Мэр Луганска наградил победителей второго этапа конкурса \enquote{Педагог года Луганщины}}
\label{sec:04_12_2020.news.lnr.lug_info.lugansk_inform_center.2.pedagog_goda}
\Purl{http://lug-info.com/news/one/mer-luganska-nagradil-pobeditelei-vtorogo-etapa-konkursa-pedagog-goda-luganschiny-62520}
\ifcmt
	author_begin
   author_id lugansk_inform_center
	author_end
\fi

\index[rus]{Конкурсы!ЛНР!Педагог Года, 04.12.2020}
\index[names.rus]{Пилавов, Манолис!ЛНР!Глава администрации Луганска}

Глава администрации Луганска Манолис Пилавов наградил победителей и призеров
второго этапа республиканского конкурса \enquote{Педагог года Луганщины}. Об этом
сообщила пресс-служба столичной мэрии.

Церемония награждения состоялась в администрации Луганска, в финал вышли 11
педагогов столичных учреждений образования.

\enquote{\enquote{Учителем года}, а также абсолютным лидером второго этапа республиканского
конкурса стала учитель географии специализированной школы № 54 Инна Дыченко,
которая отдала любимому делу более 20 лет}, – говорится в сообщении.

В номинации \enquote{Воспитатель года} лучшей была признана сотрудница яслей-сада № 117
\enquote{Ладушки} Елена Ильченко, \enquote{Практическим психологом} года стала Ольга Плямина из
учебно-реабилитационного центра № 135. В номинации \enquote{Педагогический дебют}
победу одержала Анастасия Салманова, которая преподает в средней
общеобразовательной школе № 38 имени маршала Ворошилова.

Дипломы участников и призеров второго этапа республиканского конкурса \enquote{Педагог
года Луганщины} финалистам вручили начальник управления образования столичной
администрации Валентина Кияшко и директор Луганского
информационно-методического центра Ольга Комаровой. Кроме дипломов четверка
лучших педагогов столицы Республики получила почетные грамоты и цветы из рук
главы администрации Луганска.

\enquote{Мы сегодня гордимся вами. Город Луганск, являясь столицей Республики, должен,
безусловно, представлять большее количество участников, а самое главное –
должен показывать качество знаний, качество профессионального мастерства. И это
у вас получается}, – обратился Пилавов к финалистам второго этапа конкурса.

Дыченко призвала луганских педагогов в будущем активно участвовать в конкурсе
\enquote{Педагог года Луганщины}.

\enquote{Очень часто педагоги, может быть, стесняются, может быть, боятся таких
конкурсов. Но я считаю, что не стоит этого бояться, нужно преодолевать, идти
вперед и добиваться победы}, – сказала она.

Напомним, церемония награждения победителей и призеров республиканского
конкурса "Педагог года Луганщины" 5 октября состоялась в здании Федерации
профсоюзов ЛНР.

Республиканский конкурс \enquote{Педагог года Луганщины} впервые прошел в 2018 году.

12 мая 2018 года тогда еще исполняющий обязанности главы ЛНР Леонид Пасечник
обратился к представителям законодательной и исполнительной власти Республики,
представив пятилетнюю программу развития ЛНР \enquote{Наш выбор}. В частности, власти
Республики планируют повысить среднюю заработную плату педагогов до 10,5 тыс.
руб. Также до 2023 года планируется полностью ликвидировать очереди в
дошкольные учреждения. 

\textbf{ЛуганскИнформЦентр — 04 декабря — Луганск}

