% vim: keymap=russian-jcukenwin
%%beginhead 
 
%%file 04_05_2022.fb.topolja_taras.1.harkov_rusmir.cmt
%%parent 04_05_2022.fb.topolja_taras.1.harkov_rusmir
 
%%url 
 
%%author_id 
%%date 
 
%%tags 
%%title 
 
%%endhead 
\zzSecCmt

\begin{itemize} % {
\iusr{Taras Topolya}

\ifcmt
  ig https://scontent-mxp1-1.xx.fbcdn.net/v/t39.30808-6/279399409_5143475665732132_6934601867594801818_n.jpg?_nc_cat=109&ccb=1-6&_nc_sid=dbeb18&_nc_ohc=FY87idEErpMAX_sJb2w&_nc_ht=scontent-mxp1-1.xx&oh=00_AT8R1PjcjqkOe4MCsKBDKAjx-ql8bsMtwwxZd4yMnqcNkA&oe=627B81DE
  @width 0.3
\fi

\begin{itemize} % {
\iusr{Сергій Яцук}
\textbf{Taras Topolya} можна зняти і приспособити таврувати ту худобу ...

\iusr{Людмила Кожухарь}
\textbf{Тарас Тополя} і таких, нажаль, все ще багато. Сидять у підвалах і очикують на русню((((

\iusr{Олена Терещук}
\textbf{Taras Topolya} мабуть у старих людей

\iusr{Шрамко Ксения}
\textbf{Тарас Тополя} Такий він був Харків, в 2014 не даремно створювали референдуми....((((

\iusr{Valentyna Hradova Girina}
\textbf{Олена Терещук} а може за цими дверима живе керегувальник ракет, які падають нам на голову, по всьому Харківу?

\iusr{Tetiana Usiuk}
\textbf{Taras Topolya} який жах((( патріотів таких, з їх ручками....

\iusr{Олена Терещук}
\textbf{Valentyna Hradova Girina} може бути і таке

\iusr{Наталия Кушильная}
\textbf{Taras Topolya} 

Тарасе, на жаль! І серед знайомих є такі екземпляри. Не розуміла і не розумію
їх.... Сумно і боляче, що навіть сьогодні до них не доходить...

\iusr{Анна Сидорук}
\textbf{Taras Topolya} зараз за таке треба карати. Хоча б штрафом.

\iusr{Yury Kovtun}
\textbf{Наталия Кушильная} 

все просто, як правило це - росіяни, які переїхали до нас, або мають родичів на
росії, так ведуть себе або етнічні росіяни або совкодроти.

\iusr{Олена Здоровець}
\textbf{Олена Терещук} 

та ні. Старі, хто старший 65- 70 ще добре пам'ятають страшні часи савєтів. Або
самі або з розповідей батьків, бабів, дідів. А от деякі молодші... Які були в
силі, при кормушці, при блаті і дефіциті саме в молоді роки, ті найзапекліші.
Бо втратили прості \enquote{рицарі вмліянія} у вигляді ковбаси, шоколаду і консервів. А
піднятися вище КШТ інтересів не змогли...

\iusr{Любовь Сивцова}
А чому в Черкаській Лозовій написано, адже Лозова в Харківській обл.?

\iusr{Анатолій Чорний}
\textbf{Taras Topolya} Умань - усім все пох.

\ifcmt
  ig https://scontent-mxp1-1.xx.fbcdn.net/v/t39.30808-6/279016569_945164269499687_8093746770096043751_n.jpg?_nc_cat=100&ccb=1-6&_nc_sid=dbeb18&_nc_ohc=rqwImfnKWesAX-uemFB&_nc_ht=scontent-mxp1-1.xx&oh=00_AT-EejJmpGbJTZYXGTEsfl3_lc2eGOGRxQ8BgXHieUNV2A&oe=627D401F
  @width 0.3
\fi

\iusr{Merja Nahkiaisoja}
\textbf{Taras Topolya} what is this.. @igg{fbicon.face.sad.but.relieved} ?

\iusr{Katerina Mishchenko}
\textbf{Тарас Тополя} Аааааааа  @igg{fbicon.face.symbols.mouth} 

\iusr{Ирина Сумцова}
\textbf{Taras Topolya} Надіюсь це ручка в туалет на дворі, такий із діркою в підлозі...

\iusr{Наталія Сухецька}
\textbf{Taras Topolya} Як же треба поклонятися царю, щоб поставити собі таку ручку. Манкурти і злочинці...

\iusr{Юля Добрынина}
\textbf{Тарас Тополя} не розумію, чому з таким настроєм не забрати ту ручку та не валити на ту сторону  @igg{fbicon.thinking.face}  @igg{fbicon.anger} 

\iusr{Ira Snijunska}
\textbf{Тарас Тополя} такі сєпари в нас у кожному регіоні є...

\iusr{Альона Леонова}
\textbf{Taras Topolya} 

Нічого, хороші наші, найкращі Вояки після перемоги ручки з дверей змінимо )))
нехай гріє думка, що за Україну більшість. Люди ще не зрозуміли який мір несе
росія, але і до них дійде. Бережіть Себе !!!

\iusr{Наталия Кушильная}
\textbf{Yury Kovtun} Може й просто... 

Але жити все життя в Україні і не бачити очевидного, то занадто. Щось не
подобається, то ніхто не тримає, скатертиною доріжка. Через таких недалеких
деякі думають, що на сході та півдні всі тільки й ждуть \enquote{русский мир}.

\iusr{Наталия Кушильная}
\textbf{Любовь Сивцова} Черкаська Лозова -це назва села під Харковом.

\iusr{Шевченко Ірина}
\textbf{Тарас Тополя} 

повірте, таких зрадників залишилося не багато! Так, вони є, але такі щури є в
кожному місті! Не тільки у Харкові! Ми їх переможемо! Очистимо нашу країну і
націю! Все буде Україна!@igg{fbicon.flag.ukraina}

\iusr{Mihail Berezhnoy}

Не все, що з двома головами - руZZкий герб  @igg{fbicon.wink} 
\url{https://uk.wikipedia.org/wiki/Двоголовий_орел}

\raggedcolumns
\begin{multicols}{2} % {
\setlength{\parindent}{0pt}

Двоголовий орел. Матеріал з Вікіпедії — вільної енциклопедії.

\textbf{Двоголо́вий оре́л} — фантастичний птах, гербова фігура в геральдиці. Зображається
у вигляді орла, який має дві голови. Використовується в декоративному мистецтві
та геральдиці. Двоголовий орел походить зі Стародавнього Близького Сходу.
Найдавніші знахідки зображення орла датуються III—II тисячоліттям до н. е.
Пізніше використання символу відродилося у високому середньовіччі, особливо він
використовувався сельджуками, артукідами, зангідами, огузами та монголами. З
XIII століття двоголовий орел стає все поширенішим у Європі, зокрема,
використовувався у Візантійській імперії, Священній Римській імперії, та
Московському князівстві (з 1547 року Московське царство). Нині двоголовий орел
є гербом Албанії, Сербії, Чорногорії та Росії.

\ifcmt
  ig https://upload.wikimedia.org/wikipedia/commons/thumb/2/28/Imperial_Coat_of_Arms_of_the_Empire_of_Austria.svg/486px-Imperial_Coat_of_Arms_of_the_Empire_of_Austria.svg.png
  @caption @label <++>
\fi

\end{multicols} % }


\end{itemize} % }

\end{itemize} % }
