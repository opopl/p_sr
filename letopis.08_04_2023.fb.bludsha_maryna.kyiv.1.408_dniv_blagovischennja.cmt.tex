% vim: keymap=russian-jcukenwin
%%beginhead 
 
%%file 08_04_2023.fb.bludsha_maryna.kyiv.1.408_dniv_blagovischennja.cmt
%%parent 08_04_2023.fb.bludsha_maryna.kyiv.1.408_dniv_blagovischennja
 
%%url 
 
%%author_id 
%%date 
 
%%tags 
%%title 
 
%%endhead 

\qqSecCmt

\iusr{Владимир Трегубов}

Дуже фiлосовська та змiстовна iсторiя про вiдродження марiуполя. Я цiлком
згоден з чим хлопцем iсториком що про марiуполь не можна забувати. Про
розквiтаючi квiти теж було цiкаво читати. Особливо менi сподобалося друге фото
з пташиним гнiздом. На мою думку весна цього року приходить дуже повiльно i
постiйно зi змiнами, то вона тепла зi сонечком то холодна зi зливами та
сирiстю. На останок хочу вам побажати тiльки добрих та хороших новин у вашому
життi i щоб таких негативних випадкiв от як з банком та новою поштою було
якомога менше. Бережiть себе.

\iusr{Olexandra Gryn}

Ого! Оце так день!

\iusr{Yaroslav Areshkovych}

На рахунок "просто йшла по вулиці", якщо була галочка про автосписання то при
отриманні/відправки чи що там було через деякий час те і намагалося відбутися,
все по системі, треба уважно правила і налаштування читати і не обов'язково
щось треба робити щоб без відома було списання в даному випадку так.

Вчора в мене теж з НП історія, треба було відправити вже оплачену посилку, було
пів години, встигаю на пошту, прийшов - вона закрита, побіг в іншу, є ще 15хв
чекаю в черзі і тут вимикають світло, а до закриття відновити роботу не
можливо, є ще 10 хв біжу в наступну, за 4хв до закриття відправив)

За пів години майже 4км але не здався)

\begin{itemize} % {
\iusr{Maryna Bludsha}
\textbf{Yaroslav Areshkovych} не було ніякої галочки, просто повідомлення про нову послугу
\end{itemize} % }
