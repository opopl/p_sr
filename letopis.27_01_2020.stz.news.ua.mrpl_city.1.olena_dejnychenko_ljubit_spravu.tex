% vim: keymap=russian-jcukenwin
%%beginhead 
 
%%file 27_01_2020.stz.news.ua.mrpl_city.1.olena_dejnychenko_ljubit_spravu
%%parent 27_01_2020
 
%%url https://mrpl.city/blogs/view/olena-dejnichenko-lyubit-spravu-yakoyu-zajmaetes
 
%%author_id demidko_olga.mariupol
%%date 
 
%%tags istoria,krajeznavstvo,mariupol,mariupol.pre_war
%%title Олена Дейниченко: «Любіть справу, якою займаєтесь!»
 
%%endhead 
 
\subsection{Олена Дейниченко: \enquote{Любіть справу, якою займаєтесь!}}
\label{sec:27_01_2020.stz.news.ua.mrpl_city.1.olena_dejnychenko_ljubit_spravu}
 
\Purl{https://mrpl.city/blogs/view/olena-dejnichenko-lyubit-spravu-yakoyu-zajmaetes}
\ifcmt
 author_begin
   author_id demidko_olga.mariupol
 author_end
\fi

Вона краще за інших розбирається в деталях маріупольських будинків, знає безліч
унікальних фактів та цікавих легенд про наше місто. Завдяки краєзнавчим
прогулянкам з \textbf{Оленою Василівною Дейниченко} у багатьох містян значно виріс
інтерес до історії та архітектури Маріуполя. Пропоную познайомитися з людиною,
яка по-справжньому закохана у рідне місто та справу, якою займається...

Народилася Олена в Маріуполі (тоді ще у Жданові). Особливе ставлення до
Маріуполя у нашої героїні виникло поступово. І. М. Гревс, один із засновників
шкільного краєзнавства, зазначав, що саме вивчення деталей міста сприяє
вихованню любові до \enquote{малої батьківщини}, так сталося і з Оленою. Змалечку вона
часто разом з мамою кілька разів на тиждень їхала до центру міста, щоб
прогулятися вздовж старовинних вулиць, помилуватися вишуканим цегляним
орнаментом чи ліпниною, кованими навісами, дерев'яними різьбленими дверима...
Трохи згодом дівчина почала замислюватися, що, окрім моря та індустріального
пейзажу, можна показати друзям, які приїжджають до нас зі Львова, Харкова,
Києва.

\ii{27_01_2020.stz.news.ua.mrpl_city.1.olena_dejnychenko_ljubit_spravu.pic.1}

За освітою Олена Василівна вчителька російської мови і літератури. Однак працює
в університетській бібліотеці за комп'ютером... Два роки тому отримала другу вищу
освіту у виші, з яким у неї пов'язане все доросле життя: навчання на
підготовчих курсах для абітурієнтів, диплом філолога – тоді ще МГІ ДонНУ,
диплом магістра з бібліотечної справи й інформаційної діяльності – МДУ.
Розпочинала бібліографом, наразі завідує відділом наукової бібліотеки, який
відповідає за часткове наповнення електроних баз даних, за сайт бібліотеки.
Навіть краєзнавство – одночасно і робота, і хобі... Сьогодні, коли саме почалася
діяльність нашої героїні, як краєзнавця, сказати важко. У 1996–1997 рр.
редакторка газети \enquote{Привет, ребята!} запропонувала дівчині написати про те, що
її цікавить. Та, мабуть, почалося все ще раніше – з, так би мовити,
аналітичного ставлення до зовнішнього оздоблення будинків, коли Олена
показувала татові, що, на її думку, варто сфотографувати. З позиції нинішнього
досвіду розуміє, що зробили-таки пару цінних кадрів: фасад будинку навпроти
сучасного ПК \enquote{Молодіжний}, рослинні елементи якого зникли під шаром штукатурки,
і дерев'яні двері з вазами неподалік від краєзнавчого музею. А ось плівка із
старовинними кованими навісами, балконами, парканами, на жаль, \enquote{засвітилася}...
Збиралася відновити добірку вже з кольоровою \enquote{мильницею}, та кілька років було
не до того: навчалася, завдяки чудовій компанії – багато подорожувала Україною.
Здавалося, Маріуполь почекає, та ні, деякі елементі будинків втрачено назавжди.
Іншою важливою  краєзнавчою темою Олена також зацікавилася завдяки оточенню і
наявності фотоапарата. Її бабуся підлітком кілька років мешкала поблизу старого
цвинтаря у центрі міста, розповідала деякі історії і місцеві легенди;
відвідуючи могили родичів, завжди з мамою додатково перевіряли, чи стоять ще
залишки каплиці і мармуровий хрест з трояндами, зроблений наче з дерева, які
нагадували львівський Личаків. Десь на початку 2000-х випадково виявилося, що
один з друзів також знає місце знаходження кількох старовинних пам'ятників.
Вони змогли зацікавити незвичайною екскурсією інших, бо хоча центральною алеєю
завжди користувалися перехожі, блукати безлюдними хащами у непомінальні тижні
було не так вже й небезпечно, і пішли обмінюватися краєзнавчим досвідом.
Пам'ятає, як вона з друзями намагалася перевернути голову ангела, що лежала
поруч з пустим постаментом... До речі, знаходила цей куточок і пізніше, та більше
вона цю голову ніколи не бачила. Наповнювати такими місцями фотоплівки на 36
кадрів було і не зручно, і якось неправильно. Та до того часу, як в активне
використання увійшла цифрова фототехніка, стало зрозуміло, що це – чи не єдиний
спосіб їх зберегти. 

\ii{27_01_2020.stz.news.ua.mrpl_city.1.olena_dejnychenko_ljubit_spravu.pic.2}

Олена є авторкою багатьох статей наукового та науково-популярного характеру, а
також учасницею низки краєзнавчих проєктів. Зокрема, завдяки Ірині Федотовій і
\textbf{\enquote{Школі гідів}} вона вперше розповідала про цікаві деталі маріупольської
архітектури не родичам, друзям, колегам, а тим, кого вперше бачила; а завдяки
проєкту \textbf{\enquote{Маріуполь – це Україна}} вона чи не вперше взяла участь у проведенні
саме екскурсії. А от тематичні «краєзнавчі прогулянки» Олена проводила частіше.
Мінімум історичних фактів, максимум візуалізації, і маршрут – лише орієнтовний,
який може змінитися в залежності від того, чим саме зацікавиться конкретна
аудиторія. Звісно, у віртуальному форматі менше простору для імпровізації, бо в
основі – заздалегідь підготовлена добірка слайдів, у пішохідному форматі
остаточно вирішеною залишається основна локація – стежки старого цвинтаря чи
вулиці старої частини міста. До речі, з тими, хто каже, що цвинтарі – не місце
для прогулянок, маріупольчанка згодна стовідсотково... Та якщо завдяки таким
прогулянкам зростає кількість тих, хто час від часу піклується не тільки про
поховання родичів, а й про ті, що цікаві в історичному плані, тим більше, якщо
завдяки комусь з учасників Маріуполь врешті-решт отримає власний відновлений
Личаків, – воно того варте.

У майбутньому Олена мріє нарешті систематизувати перелік тих осіб, над чиїми
похованнями ще стоять пам'ятники кінця ХІХ – першої половини ХХ ст. на старому
центральному цвинтарі. Було б непогано позначити координати якомога більшої
кількості поховань, а в подальшому – об'єднати більш-менш систематизовану
добірку інформації з тим, що напрацьовано іншими небайдужими маріупольцями, і з
даними, які містяться у метричних книгах, наявних в Інтернет, та у книгах з
переліком похованих на закритих цвинтарях, що зберігаються у міському архіві
(рукописний формат і перших книг, і других не дозволяє повноцінно працювати з
ними, коли необхідно швидко відшукати у тексті конкретні прізвища). Та поки що
все це – тільки мрії. З часом Олена Василівна хотіла б об'єднати напрацювання
тих, хто небайдужий до цієї теми, все ж таки зібрати єдину електронну
загальнодоступну картотеку похованих на закритих цвинтарях нашого міста – хоча
б на одному з найстаріших з них. 

Поточні ж плани укладає саме життя. Наприклад, рік тому (у лютому 2019 р.)
обговорювався новий історико-архітектурний опорний план, що спонукало жінку
ретельно порівняти переліки потенційної культурної спадщини з тими, що були
розроблені на початку 1990-х років. Так на сторінці Олени Дейниченко у \underline{\textbf{Calaméo}}
з'явилася добірка цікавих і дійсно корисних даних
(\url{https://www.calameo.com/accounts/2979071}). На цій сторінці можна
ознайомитися і з іншими тематичними добірками Олени, які стануть в нагоді всім,
хто прагне пізнати історію Маріуполя і Приазов'я загалом.

Найголовніше для нашої героїні, що її підтримують і розуміють найближчі люди.
Це відбувається не тільки на рівні \enquote{Молодець, так тримати!}, вони ще й активно
допомагають: беруть безпосередню участь у локальних краєзнавчих розвідках,
постійно забезпечують цікавою інформацією. Також вони – завжди перші, хто дає
їй зрозуміти, що те, що вона робить, ще комусь цікаво.

\ii{27_01_2020.stz.news.ua.mrpl_city.1.olena_dejnychenko_ljubit_spravu.pic.3}

Надихають відомі маріупольські краєзнавці. Окрім того, один з позитивних
яскравих прикладів – \textbf{Людмила Андріївна Проценко}, історик, архівіст,
некрополіст, активна громадська діячка, завдяки якій Лук'янівський цвинтар було
визнано історико-меморіальним заповідником; більше 40 років вона працювала над
іменною картотекою похованих у Києві осіб, до складу якої увійшло 40000 карток.

Як \textbf{Лада Лузіна} каже, що пише романи – не просто про Київ, а саме про своє
Місто, як один з гідів зазначав у публікації в Інстаграм (\textbf{Фаіг Гусейнов},
@fa\_huu), що намагається показати туристам ту Грузію, у якій живе, а не ту, про
яку розповідають у путівниках, – так і Олена намагається висвітлювати Маріуполь
таким, яким його бачить, передавати захоплення тим, що свого часу її вразило
найбільше...

Коли Олена починає байдикувати більше, ніж треба, одразу згадує тих, хто вражає
своєю вибуховою активністю. Коли ж відчуває, що, навпаки, необхідно кинути все
і відпочити, згадує негативний приклад: вивчаючи цвинтарі свого рідного міста і
регіону, \textbf{А. Ю. Москвін} зібрав 900 епітафій, склав картотеку на 10000 поховань,
але він залишиться неоднозначною постаттю в нижегородській історії; і не має
значення, з яких причин його життя склалося так, як склалося, необхідно лише
пам'ятати, що найважливіше у житті – ті, хто тебе оточують, жваве спілкування з
родичами, друзями, а не те, що тебе цікавить...

\textbf{Улюблені місця:} \enquote{Багаті на джерела природної води схили між Парком ім.
Петровського і Кальчиком. Це – загадкове, чарівне місце, та, на жаль, вже
зовсім не таке, яким воно було, коли вперше мене вразило. Невеличкий, в кілька
каменів, в цілому десь півметра заввишки, але дуже бурхливий водоспад, перед
ним – заводь з кришталево чистою, завжди холодною водою, позаду нього – обрив
(два-три метри вапняку), ліворуч і праворуч взимку – одночасно і товстий шар
льоду, і молода зелена трава, – все це років 10–15 тому сховано за
залізобетонною огорожею, не знаю, чи щось збереглося. Ще одне джерело було
відновлено, і виглядає доволі цікаво, проте казкова арка, оплетена коренями
дерева, зникла.

До речі, подібним чином змінилося і джерело на Малофонтанній неподалік від
історичного центру міста: змінилося, безумовно, на краще, та цегляний орнамент
з хрестами, який починав руйнуватися, остаточно втрачено... Залишилось тільки
кілька фото в альбомі, які вдалося зробити своєчасно.

Трохи на південь від Парку Петровського, за вкритими ожиною схилами було ще
одне джерело, подібне до того, що з водоспадом, тільки не таке бурхливе, із
вдвічі нижчою вапняною стіною, з кількома невеличкими заводями... Маю надію, що
хоча б воно залишилось таким, як  було}...

\textbf{Улюблена книга:} \enquote{Дорога} з циклу \enquote{Бездна голодных глаз} Г. Л. Олді.

\textbf{Улюблений фільм:} \enquote{Будинок з піску і туману} (фільм, який, немов
вітраж, весь складається з дзеркальних парних моментів, і який допомагає
відчути: щоб не відбувалось в житті, все не так ще й жахливо...).

\textbf{Хобі:} \enquote{Обожнюю невеличкі подорожі (бажано, 3-5 днів), як новими, так і вже
добре знайомими містами. Час від часу відволікаюсь від краєзнавства на пазли,
в'язані серветки, зібрані за власною фантазією намиста... Колекціонувала, майже
не поповнюю, та із задоволенням зберігаю добірки листівок, тематичних
конвертів, кишенькових календарів, значків, марок, етикеток з сірників. Кілька
років тому, коли ці птахи ще були не так популярні, завдяки друзям почала
збирати сов}.

\textbf{Порада маріупольцям:} \enquote{Не наважуся щось порадити, тільки побажати:
здоров'я тим, для кого ви, незважаючи на всі успіхи чи невдачі, перш за все –
рідна людина, а також чудових друзів, які завжди підтримають, і яких завжди
приємно підтримати, задоволення від справи, якою займаєтеся}.

\textbf{Курйозні випадки:} \enquote{Дві історії, бо вони розповідають, якими різними можуть бути
незнайомі люди: одні викликають жах, інші вражають щирістю.

\textbf{Історія перша.} Потяг до Харкова, плацкартний вагон. Більшість пасажирів –
непосидячі хлопці-школярі 3-5 класу, в кінці вагону – наша невеличка, але трохи
шумна компанія, всередині – кілька оскаженілих від ґвалту бабусь. Школярі по
черзі бігають то руки помити, то сміття викинути, то у відкрите вікно
зазирнути, і оскільки їх багато, здається, що це – постійний неперервний
процес. Якщо ж хтось з них зупинявся біля нашої компанії, задоволений тим, що
можна пожартувати разом з дорослими, всі, не помічаючи цього, говорили ще
голосніше... Нарешті одна з бабусь не витримала, на півдорозі схопила за руку
найспокійнішого хлопчика, бо за іншими вона не встигала, і злобно прошепотіла:
\enquote{Ще раз підеш до туалету, я тебе вб'ю}. Навіть нам стало моторошно від думки,
чи не виникнуть у хлопчика проблеми з походом до туалету на кілька наступних
років...

\textbf{Історія друга,} пов'язана з тим, як ми незаплановано затримались у старій
частині міста, хоча про туристичні магніти, від яких, як хтось пошуткував,
важко відірватися, тоді ще і мова не йшла. Під час прогулянки раптово почалась
розмова (начебто, ні про що) з мешканцями будинку Гампера. Вже хвилин через
п'ять один з них помив старовинну кахлю з клеймом заводу Епштейна, щоб віддати
її до фонду краєзнавчого музею, потім пішов за драбиною, щоб разом з нами
оглянути горище і загадкову кімнату під баштою. Його сусідка вела нас до
власної квартири, щоб показати, як старовинні вікна зі ставнями виглядають з
середини (а головне, як стіни не витримують випробування часом і недбайливим
ставленням співвласників, та це неважливі зараз деталі)... Хоча за годину до того
ми з однією чудовою людиною зібралися просто прогулятися Гамперським спуском і
помилуватися хвилями на пляжі неподалік від залізничного вокзалу. Невдовзі ми
одружилися, і за багато років разом він вже звик, що від спілкування з
незнайомцями можна очікувати чого завгодно, а тоді думав, що я його розігрую,
бо не міг повірити, що щирі, небайдужі мешканці до того моменту, як запросити
нас до будинку, ніколи раніше мене не бачили}
