% vim: keymap=russian-jcukenwin
%%beginhead 
 
%%file 06_06_2021.fb.bilchenko_evgenia.2.margarita
%%parent 06_06_2021
 
%%url https://www.facebook.com/yevzhik/posts/3958194744215606
 
%%author Бильченко, Евгения
%%author_id bilchenko_evgenia
%%author_url 
 
%%tags 
%%title БЖ. Маргарита
 
%%endhead 
 
\subsection{БЖ. Маргарита}
\label{sec:06_06_2021.fb.bilchenko_evgenia.2.margarita}
\Purl{https://www.facebook.com/yevzhik/posts/3958194744215606}
\ifcmt
 author_begin
   author_id bilchenko_evgenia
 author_end
\fi

БЖ. Маргарита.
Она была не такой, как все: изысканная, ершистая.
Майя Плисецкая в роли Бетси: бытия отличник, не хорошистка.
Душистая маргаритка на средневековом кладбище,
Проросшая через мшистый склеп. На ней, - не касаясь клавиш, -
Одной тишиною играть хотелось. Служить ей, не рефлексируя,
В детском восторге от тонкой смеси сирости и красивости.
Казалось, что носит она корсет под платьем и пропасть синюю
Неба в глазницах: пристань, достань, из облачной бездны вынь её.
Я же была для неё - лесом. И безо всякой лести,
Заблудившись во мне, она поняла, что не выйдет, но сможет лес нести,
На собственной хрупкой шее, не сетуя и не ябедничая:
Поделилась со мной бы, я б не стерпела, ввязалась бы в бой за яблони.
А потом, - сначала едва заметно, как спотык в дилетантском дактиле, -
Из чуда начали проступать то ли трусость, то ли предательство.
Грань между ними - тонка настолько, что методом лаборантским
Её не вычислить:  в этом нежность монстра коллаборации.
Раньше я верила в мысль Сократа о миссии Просвещения.
Верила в talking cure Фрейда, проскальзывая меж щелями
Их одержимых нацизмом душ, вставляла туда нарциссы, и
Всё становилось ещё ужасней: без помощи экзорцизма,
Стало понятно, не обойтись: кадилом по рылу, вот те как.
Обратная сторона попа - снайперское сквозь оптику
Прищуривание в точку цели: чёрное авторалли.
Мы так не хотели, мы так не играли, вы сами нас собирали.
Собирали нас, как закаляли сталь, - грамотную конструкцию, -
Посредством травель, убийств, арестов, иронии и обструкции.
Мы - Майи Плисецкие, и княгини Тверские, и ночи Кабирии,
И маргаритки на стенах склепов, которые вы убили.
6 июня 2021 г.

\ifcmt
  pic https://scontent-cdt1-1.xx.fbcdn.net/v/t1.6435-9/197328000_3958194674215613_3770550897140458808_n.jpg?_nc_cat=110&ccb=1-3&_nc_sid=8bfeb9&_nc_ohc=tl-2DgYE3l8AX9qmb17&_nc_ht=scontent-cdt1-1.xx&oh=204ac24c563bd0bf48e046be8d271a5a&oe=60E4AE83
\fi
