% vim: keymap=russian-jcukenwin
%%beginhead 
 
%%file 16_11_2020.news.ua.5ua.1.kokush_titanik
%%parent 16_11_2020
 
%%url https://www.5.ua/kultura/bez-noho-tytanik-stav-by-inshym-iak-ukrainets-zdiisnyv-tekhnichnu-revoliutsiiu-v-hollivudi-229187.html
%%author 
%%tags 
%%title 
 
%%endhead 

\subsection{Без нього "Титанік" став би іншим: як українець здійснив технічну революцію в Голлівуді}
\label{sec:16_11_2020.news.ua.5ua.1.kokush_titanik}
\Purl{https://www.5.ua/kultura/bez-noho-tytanik-stav-by-inshym-iak-ukrainets-zdiisnyv-tekhnichnu-revoliutsiiu-v-hollivudi-229187.html}

\index[names.rus]{Кокуш, Анатолий!Кино, Голливуд}

\href{https://www.5.ua/spetstemy/154/}{Видатні українці}

{\bfseries
Не лише "Титанік", а й "Гаррі Поттер", "Троя", "Війна світів" і багато
інших --- як часто люди переглядають улюблені стрічки і захоплюються
неймовірними кадрами! Багато з них з'явилися завдяки титанічній праці
одного українського інженера, вихідця з Криму, талант якого завоював серце
Джеймса Кемерона і Стівена Спілберґа
}

\ifcmt
pic https://www.5.ua/media/pictures/1140x641/200730.jpg
caption Ukrainian Arm in Hollywood
\fi

Анатолій Кокуш --- українець, винахідник операторського обладнання, що
створило найкращі кадри "Титаніка", "Гаррі Поттера та Ордена Фенікса",
"Форсажу" та багатьох інших кінострічок.

Згадуючи фільм "Титанік", одразу спадає на думку легендарний кадр, коли
Джек та Роза стоять на носі корабля. 1997 року, коли фільм вийшов на
екрани, ця сцена змусила плакати мільйони глядачів.

За цим стоїть титанічна операторська робота. Глядач то зблизька
спостерігає за емоціями героїв, то "підлітає" над закоханими. Ефекту
польоту та плавних рухів камери допомогло досягти операторське обладнання,
вироблене українською компанією "Фільмотехнік".

\url{https://youtu.be/8MDPeL8lpzo}

Виходець із Криму Анатолій Кокуш за радянських часів навчався в
Ленінградському інституті кіноінженерів. Після його закінчення (1974 р.)
влаштувався конструктором на кіностудію імені Довженка в Києві. Відтоді
виношував ідею "відірвати" знімальну камеру від ока оператора. Та його
розрахунки й дослідні моделі сприймали іронічно.

{\bfseries
У 1980-х роках він із колегами розпочав роботу над ідеєю крана з
гіростабілізованою панорамною головкою Flight Head, яка змогла б
захистити камеру від зовнішніх коливань. Камера на такому крані мала
здатність плавно обертатися на 360 градусів навколо автомобіля, що
рухається на великій швидкості.
}

Паралельно йшла розробка спеціальних довгих кранів із вильотом стріли до
25 метрів, які дозволили знімати складні сцени в масштабних декораціях.

Утім розкрити свій конструкторський потенціал Кокуш зміг уже після розпаду
Союзу. На початку 90-х заснував компанію "Фільмотехнік", яка стала
унікальним поєднанням конструкторського бюро з майстернею. Однією з
найпрогресивніших розробок її став операторський кран "Каскад".

"Каскад" --- це надлегка конструкція, стріла якої здатна вилітати на 21
метр. Завдяки крану режисери отримали можливість розмістити камеру у
найбільш недоступних місцях.

Незважаючи на нестачу фінансів, колеги намагалися брати участь у всіх
спеціалізованих міжнародних виставках.

{\bfseries
На одній із таких виставок до Кокуша підійшов представник режисера
Джеймса Кемерона. Він похвалив техніку на стенді і пообіцяв
передзвонити. Так обладнання Анатолія потрапило в багатомільйонний
проєкт "Титанік".
}

Момент, коли герої Кейт Уінслет і Леонардо Ді Капріо "летять" на носі
корабля як і багато інших кадрів знімалися за допомогою обладнання
Анатолія. Його винаходи виявилися суперзатребуваними за кордоном --- там
нічого подібного на той момент не було.

Ще однією сенсаційною розробкою в галузі кінозйомок став "Авторобот" –
чотириметровий кран із камерою, що монтується на даху будь-якого авто.
Кран та камера можуть плавно обертатись на 360 градусів довкола машини,
навіть коли та рухається на шаленій швидкості. Така система дозволяє
фільмувати карколомні ракурси без поштовхів і коливань.

Це перше пристосування, за яке удостоївся нагороди особисто Анатолій
Кокуш, стало відоме в Голлівуді як Russian Arm ( "Російська рука"). Вперше
українські інженери показали це обладнання в 1991 році на нью-йоркській
виставці, викликавши хвилю публікацій, у яких їхні ідеї пов'язували з
космічними технологіями росіян --- тому цю українську "руку" в Америці й
назвали "російською". 

\ifcmt
pic https://www.5.ua/media/pictures/820x546/200835.jpg
\fi

Амбіції привели Анатолія до США. Голлівудський ринок сприйняв українця не
відразу. Та через кілька років участі у виставках та співпраці з Джеймсом
Кемероном поступово до Кокуша вишикувалася черга замовників.

{\bfseries
Фільми, зняті за допомогою техніки Кокуша: "Форсаж", "Код да Вінчі",
"Місія нездійсненна-3", "Люди Ікс", "Містер та місіс Сміт", "Гаррі Потер
та Орден Фенікса", "Бетмен", "Війна світів", "Титанік", "Троя", "Код да
Вінчі", "Бетмен", "12 друзів Оушена", "Пограбування по-італійськи",
"Чарлі і шоколадна фабрика", "Казанова", "Форсаж-3", "Люди Ікс-3",
"Хоббіт", "Тарзан", "Книга джунглів", "Зоряні війни", "Примарний
гонщик".
}

Наприклад, ось так обладнання компанії Андрія Кокуша працювало під час
зйомок фільму "Примарний гонщик":

Режисер Стівен Спілберґ був вражений видовищною картиною від'їзду з міста
у фільмі "Війна світів", яку знімали українською технікою.

У 2005 році на одній із чергових виставок в Лос-Анджелесі до Анатолія
підійшли члени технічної ради Кіноакадемії і сказали, що рекомендують його
на здобувача технічного "Оскара". Він став одним із більш ніж 100
претендентів на премію.

У 2006 році українець Анатолій Кокуш, його колега Юрій Поповський і
оператор Олексій Золотарьов отримали дві інженерно-технічної нагороди
(Scientific and Technical Academy Award) американської кіноакадемії
"Оскар". Одну --- за операторський кран зі стабілізованою панорамною
голівкою, а другу --- за крани серії "Каскад" і операторський кран із
рухомою кареткою "Тревелінг каскад". "Крейн (кран --- англ. --- ред.) фром
Юкрейн", так обіграла тоді схоже звучання ведуча церемонії нагородження,
молода актриса Рейчел Мак-Адамс.

Юрій Поповський та Анатолій Кокуш  Юрій Поповський та Анатолій
Кокушukrainethebest.com

\ifcmt
pic https://www.5.ua/media/pictures/820x546/200733.jpg
\fi

На рахунку Анатолія Кокуша також дві нагороди Баварської кіноакадемії --- це
європейський аналог американської премії.

{\bfseries
На техніку Анатолія знімали великомасштабні заходи: від візиту Папи
Римського в Україну до концертів Майкла Джексона, Мадонни, Стінга, Роббі
Вільямса, Deep Purple і інших зірок. Українська "Фільмотехніка"
вигравала тендери на зйомки на Олімпійських іграх або "Євробачення".
}

Навіть здобувши світове визнання, Кокуш і далі працює в Києві. Каже: тут
дешевше і комфортніше. Отримані від замовлень кошти вкладає у нові
проєкти. Зараз в асортименті "Фільмотехніка" --- понад 20 видів
операторського обладнання. Фірма постачає техніку для зйомок у десятки
країн світу.

\textbf{\emph{Інформацію взято з відкритих джерел}}


