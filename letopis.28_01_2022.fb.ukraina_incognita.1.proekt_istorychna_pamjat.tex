% vim: keymap=russian-jcukenwin
%%beginhead 
 
%%file 28_01_2022.fb.ukraina_incognita.1.proekt_istorychna_pamjat
%%parent 28_01_2022
 
%%url https://www.facebook.com/ukrainaincognita/posts/5350485971646454
 
%%author_id ukraina_incognita
%%date 
 
%%tags istoria,pamjat,ukraina
%%title Давайте відтворимо історичну пам'ять українців разом. Підтримайте проект
 
%%endhead 
 
\subsection{Давайте відтворимо історичну пам'ять українців разом. Підтримайте проект}
\label{sec:28_01_2022.fb.ukraina_incognita.1.proekt_istorychna_pamjat}
 
\Purl{https://www.facebook.com/ukrainaincognita/posts/5350485971646454}
\ifcmt
 author_begin
   author_id ukraina_incognita
 author_end
\fi

Нам потрібна ваша підтримка! Фінансова та інформаційна. Поширюйте, будь ласка,
цей пост.

На фото  старовинний козацький цвинтар у селі Глибоке на Одещині. Він
підмивається водами озера-лимана Сасик. У воду падають кам'яні хрести, гроби та
людські кістки. Цвинтар потрібно рятувати, але про його існування мало хто
знає, адже він не включений у Реєстр пам'яток, інформація в інтернеті дуже
скупа, а туристи тут не буваютЬ. адже дорога у село страшна... 

\ii{28_01_2022.fb.ukraina_incognita.1.proekt_istorychna_pamjat.pic.1}

Це наша спадщина, наша пам'ять. Щоб її врятувати ми робимо проект  \enquote{Старовинні
цвинтарі України}. Починаємо із козацьких. Будемо дуже вдячні фінансовій
підтримці, а також інформаційній, волонтерській, та поширенню цього посту.

\ii{28_01_2022.fb.ukraina_incognita.1.proekt_istorychna_pamjat.pic.2}

Давайте відтворимо істричну пам’ять українців разом. Підтримайте проект.

Карти для благодійних внесків:

ПриватБанк:

5363542016957863

Iban:
UA513052990000026209691037349

Монобанк:
5375414125538636

Номери карт винесемо у коменти, для зручності копіювання.

Також ви можете стати нашими патронами у системі Patreon – лінк ми надаємо у
коментарі.

\ii{28_01_2022.fb.ukraina_incognita.1.proekt_istorychna_pamjat.pic.3}

Ми почали нашу акцію десять днів тому і люди відгукнулися. 

Рекордний благодійний внесок вніс Олександр Пархоменко - 1100 грн. 

\ii{28_01_2022.fb.ukraina_incognita.1.proekt_istorychna_pamjat.pic.4}

Нас дехто питав - нащо вам гроші? Мовляв, усе можна робити у вільний час. Тому
акцентуємо увагу на завданнях, які потребують значних коштів:

Основні завдання проекту:

\begin{itemize}
  \item 1. Створення реєстру старовинних цвинтарів України.
  \item 2. Краєзнавчі експедиції із пошуку старовинних цвинтарів.
  \item 3. Створення інформаційної системи, сайту та карти \enquote{Старовинні цвинтарі України}.
  \item 4. Сприяння внесенню старовинних цвинтарів у Державний реєстр пам’яток.
  \item 5. Робота (запити, консультації, постійна комунікація, влаштування
          толок) з місцевою владою, щодо впорядкування старовинних кладовищ.
\end{itemize}

Давайте відтворимо істричну пам’ять українців разом. 

Підтримайте проект.  Карти для благодійних внесків:

ПриватБанк: 5363542016957863

Iban: UA513052990000026209691037349

Монобанк: 5375414125538636

Також ви можете стати нашими патронами у системі Patreon – лінк ми надаємо у коментарі.

Ми відразу створимо список наших благодійників, куди будемо вносити імена та
прізвища людей, які внесли благодійні внески. Обіцяємо щомісячні детальні
звіти, щодо нашої роботи.

Один із найбільших краєзнавчих сайтів, а нині громадська організація “Україна
Інкогніта” розпочинає нову історію. Для цього нам потрібна ваша підтримка.
Підтримайте нас фінансово, надсилайте інформацію про старовинні цвинтарі. 

Поширюйте цей пост – це дуже важливо!

Далі подаємо список наших донорів і виносимо їм величезну подяку. Маємо лідера
за благодійними внесками - Олександр Пархоменко пожертвував нам 1100 грн.
Дякуємо!

Ось перелік наших благодійників:

\raggedcolumns
\begin{multicols}{2} % {
\setlength{\parindent}{0pt}

Приват:

\obeycr
Жигадло Андрій Євгенович 200 грн.
Юськів Роман Володимирович 100 грн.
Ткачук Сергій Іванович 200 грн.
Чирук Оксана Клавдіївна 500 грн.
Калініченко Ганна Олегівна 100 грн.
Коркішко Ольга Дмитрівна 200 грн.
Константюк Ірина Вікторівна 500 грн.
Скобель Наталія Олегівна 100 грн.
Таращук Любов Степанівна 100 грн.
Кінайло Віта Олександрівна 300 грн.
Омеляненко Оксана 300 грн.
Дашковська Марія 200 грн.
Довбня Вікторія 200 грн.
Танасієнко Сергій 150 грн.
Танасійчук Станіслав 100 грн.
Сомік Світлана 1000 грн.
Рубан Лідія 500 грн.
Камаралі Тетяна 200 грн.
Вячеслав Миронов 200 грн.
Тхорик Василь 150 грн.
Портнягін Дмитро 200 грн.
Сворак Олена 200 грн.
Новіцька Наталія 200 грн.
Архіпенко Олександр 500 грн.
Швачко Сергій 20 грн.
Гавінський Іван 70 грн.
Ковтун Юрій 100 грн.
Вівчар Вячеслав 500 грн.
Живаго Олександр 500 грн.
Чедолума Ілля 200 грн.
Репінська Оксана 100 грн.
Синюта Назар 200 грн.
Пушкова Зоя 115 грн.
Величко Андрій 500 грн.
Мотрич Юрій    280 грн.
Пархоменко Олександр 1100 грн.
Моно
Олексій Дзюба 159 грн.
Юрій Калініченко 100 грн.
Юрій Гарбар 500 грн.
Олена Пряхіна 150 грн.
Олексій Дутчак 100 грн.
Наталія Мельник 200 грн.
Вікторія Закорко 100 грн.
Владислав Панченко 250 грн.
Олексій Нікіфоров 100 грн.
Олексій Березюк 200 грн.
Дмитро Васильчук   500 грн.
Ганна Коваленко 200 грн.
Артур Єнчке 150 грн.
Олександр Кшуташвілі 400 грн.
Людмила Рилякова 200 грн.
Patreon (патрони, які надають щомісячну підтримку в міжнародній системі Патреон - лінк на систему в коменті)
Алекс Пелеш 10 доларів
Sergiy Shevchenko 5 доларів
Марія 10 доларів
Віталій Чухліб 5 доларів
Inna Kudriavkina 10 доларів
Ярослав Гаврилович 5 доларів
Кирило Сніжко 5 доларів
СтелВіт 5 доларів
Taras Nebeluk 10 доларів
Сергій 20 доларів
\restorecr
\end{multicols} % }
