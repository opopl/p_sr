% vim: keymap=russian-jcukenwin
%%beginhead 
 
%%file 18_09_2021.fb.bryhar_sergej.1.deti_sadik_mova.cmt
%%parent 18_09_2021.fb.bryhar_sergej.1.deti_sadik_mova
 
%%url 
 
%%author_id 
%%date 
 
%%tags 
%%title 
 
%%endhead 
\subsubsection{Коментарі}

\begin{itemize} % {
\iusr{Юлія Кравченко}
 @igg{fbicon.100.percent} 
Дуже радію, що в нашому садочку в Миколаєві весь освітній процес українською, і вихователька теж спілкується українською:)
А от на майданчиках так..діти звертають увагу, що син спілкується українською, і здивування таке, наче ми не в Україні)

\iusr{Лиза Ребар}
Моя Віра навчилася в садочку говорити : "навіщо" замість домашнього "нашо".

\iusr{Елена Савчук}
А як дитина почувається в школі, хтось подумав? Який це стрес? Садочки мають
бути україномовні, обов'язково!!!!!
\iusr{Tetyana Rozumenko}

я побачила цікавий момент. дітей виховують у дитсадку українською, вдома з
батьками вони розмовляють російською. і от, минаю я дитячий садок на районі і
чую як ганою українською діти розмовляють з вихователями, тут їх забирають
батьки і вони з ними щебечуть російською. діти засвоюють, що у зовнішньому
світі треба говорити українською. це ситуація навпаки, аніж у совєцькі часи.


\iusr{Татьяна Шабалина}

"російський садочок - це не біда, бо діти підуть до школи, там навчаться, і
говоритимуть". - біда, бо вони як розмовляли російською, так і розмовлятимуть.
Бо на перервах говорять російською, звикли так. Мій україномовний онук у школі
почав говорити трохи російською і тепер розмовляє суржиком. Виправляю, але...


\iusr{Olesia Makarenko}

Мої колеги, які вчилися в російських школах, доводили мені, що деяких
українських слів не існує. Бо вони їх ніколи не чули.

\begin{itemize} % {
\iusr{Наталка Стус}
\textbf{Olesia Makarenko} Теж із цим стикалася. Мій російськомовний начальник колись намагався мені довести, що слова "неприйнятно" не існує. І радив перейти на "нормальний язик" замість "ковєркать" українську, якою я буцімто не дуже добре володію.

\iusr{Olesia Makarenko}
\textbf{Наталка Стус} на «нормальний язик» мені пропонували перейти буквально пару місяців тому. Послала нах.... Вибачте.
\end{itemize} % }

\iusr{Максим Меркулов}
Поки у Києві не набереться хоча б 60-70\% україномовного населення, процес не
зрушить з місця. Ні в Одесі, ні в Харкові, ні в Дніпрі.  @igg{fbicon.frown} 

\iusr{Анна Бутова}

Такі діти приходять до школи і не можуть скласти речення українською . Читають
з російським наголосом ...це жахливо на жаль. До україномовних дітей
звертаються - а ти можеш гаваріть нормально ? Нормально для них це російською .
Українська це іноземний предмет у школі .

\begin{itemize} % {
\iusr{Serhii Bryhar}
\textbf{Анна Бутова} Та навіть деякі мої однолітки й досі читають з російським акцентом).

\iusr{Анна Бутова}
\textbf{Serhii Bryhar} це на жаль реалії життя.
\end{itemize} % }

\iusr{Maryna Zelenyuk}

Всі цивілізовані країни навчають дітей іноземців державної мови саме в
САДОЧКАХ, а в школу діти ідуть вже з знанням державної мови. Чим пізніше
починаєш вивчати мову, тим гірше вона засвоюється.

\begin{itemize} % {
\iusr{Serhii Bryhar}
\textbf{Maryna Zelenyuk} В моє у випадку якесь більш-менш вивчення почалося десь уже в середніх класах, рочків так в 11-12. До того вона була такою номінальщиною, що ніхто й не переймався..
\end{itemize} % }

\iusr{Georgiy Rusanovsky}
Це освітній процес,десь так

\iusr{Ольга Сербіна}

Ми живемо теж в Одесі на Поскоті і менша донька ходить в звичайний дитсадок. То
ж всі ранки і навчання проходить українською, а вихователі і помічниця
вихователів в побуті з дітьми лише російською з проблесками української. В
школі 49й теж саме і навіть на уроках більше російської, ніж української. Курси
іноземних мов теж викладаються москальською. Власникам намагалася добитися
виконання закону "Про мову". Але нічого не добилася "патамушта большинство
желаєт на язикє". Дуже і дуже мені шкода, тому що моя сім'я УКРАЇНОМОВНА і я не
хочу щоб вони підлаштовувались під обставини, були індивідуальністю. Я
принципово, проживаючи в Одесі, і в області спілкуючись по своїй роботі і в
особистому житті категорично не використовую язика. Тяжко боротись з "вітряними
мельницями".

\begin{itemize} % {
\iusr{Ярослав Попович}
\textbf{Ольга Сербіна} написав вам пп

\iusr{Serhii Bryhar}
\textbf{Ольга Сербіна} Говорите, не хочете, щоб діти підлаштовуватися? Я десь до цього року теж щиро вірив, що це важливо, що можна щось придумати, якось подолати, вирішити, перемогти). А тепер - не вірю. Тут це неможливо. Якщо ми готові миритися - добре, якщо ні - зіюрали валізи, і на виїзд... Немає сенсу заперечувати очевидне.

\iusr{Serhii Bryhar}
До речі, мої рідні місця. Мешкав на Бочарова, навчався у 67-й школі). Але зараз би туди вже не повернувся...

\iusr{Анна Бутова}
\textbf{Serhii Bryhar} на жаль Харків , Одеса , Маріуполь ...там шансів на українську немає. Просто битися , як риба об лід. Або формувати сильний осередок в містах і просто притягати до закону всіх у сфері обслуговування , навчання , виконавчої влади . Іншого шляху немає . Тільки переїзд або у менш зросійщене місто або до Львову.

\iusr{Georgiy Rusanovsky}
\textbf{Serhii Bryhar} тут має працювати,держава,але чинуши які сидять у освітніх органах,не відповідають вимогам,відписки що садки українські ч вже отримував і булінг дитини теж мав місце так що Пані Буйневич ,має щось сказати з цього приводу

\iusr{Georgiy Rusanovsky}
В цілому це саботаж не кваліфікаційними вихователями і керівниками

\end{itemize} % }

\iusr{Тереза Мама}

П. Сергію, я вас дуже поважаю і з цікавістю читаю ваші дописи, але не
погоджуюся з тезою, що наша держава навіть не намагається пояснювати дітям і
далі за текстом. На мою думку пора перестати все перекладати на державу, вона
та наша держава вже всім оскомину наїла, бо постійно чогось не робить.Не
розказує що треба щепитися від ковіду, не розказує що за руль не можна сідати
на підпитку ,не розповідає, що треба дотримуватися законів і ще багато чого не
розказує, бо то така держава. А висновок у народу який: треба валить с етой
страни. Хоч насправді наша держава ні не найбідніша в Європі, ні не
найбайдужіша до своїх громадян. А ось громадяни, саме вони, не виконуючи своїх
прямих обов"язків щодо держави України та один до одного , далеко не в кращому
вигляді . саме громадяни порушують закони ,де лиш можна. На жаль влада, як
центральна ,так і на місця явно не дає собі раду або, як це зараз, саботує
виконання законів про мову тощо.  

\iusr{Андрій Смолій}

Сьогодні пройшовся центром Києва. Десь до 40\% говорили українською. При чому
це були люди 18 - 40 років

\begin{itemize} % {
\iusr{Лиза Ребар}
\textbf{Андрій Смолій} це вже прогрес?

\iusr{Serhii Bryhar}
\textbf{Андрій Смолій} Знаєш, я от коли ходжу центром Львова, чую десь те ж саме). В спальних районах, звісно, краще, але загалом псуються навіть Сихів і Винники (ну так, власне, там житло дешевше - мекки "восточгих репатріантів"..). Плюс ще - "работнікі технологічєской сфєри". От і маємо..

\iusr{Orest Kochan}
\textbf{Serhii Bryhar} я мешкаю у "спальному районі Львова". У 1990-ті російську тут я чув раз на тиждень, а може і рідше. А зараз майже кожен вихід з хати її чую.

\iusr{Serhii Bryhar}
\textbf{Orest Kochan} вул. Наукова. АТБ (ближче Стрийської). Якесь бідове місце. Як не підеш, так помітна кількість, часом просто десятки "русскоязичниє граждан". Навіть не знаю, з чим це пов'язано..

\iusr{Orest Kochan}
\textbf{Serhii Bryhar} там нові будинки побудовані на вулиці Трускавецькій і кілька офісів ІТ-фірм. Видно з тим пов'язано

\iusr{Андрій Смолій}

\textbf{Ліза Ребар} ну мабуть так. В часи коли я був школярем, а це початок 00х, в Києві рідко чув українську. В мене в класі говорили російською всі. На вулицях було «диковинкою» її почути. Тепер я чую українську майже в кожному другому випадку.

\iusr{Максим Меркулов}
\textbf{Ліза Ребар} А чому б і ні? Якщо порівняти з Києвом початку нульових - точно прогрес.

\iusr{Максим Меркулов}
\textbf{Serhii Bryhar} Ця тенденція, на жаль, була помітна ще до війни.  @igg{fbicon.frown} 

\iusr{Лиза Ребар}
\textbf{Максим Меркулов} ви ж пишете,шо треба 70\%

\iusr{Максим Меркулов}
\textbf{Ліза Ребар} Так не все ж відразу. Для початку і 40\% згодиться. Тим паче
слід пам'ятати, що центр столиці значно русифікованіший за околиці.

\iusr{Лиза Ребар}
\textbf{Максим Меркулов} для мене це незвично,я ж з Тернополя

\iusr{Максим Меркулов}
\textbf{Ліза Ребар} То ви мешкаєте у раю. Вам навіть львів'яни позаздрять.

\iusr{Oxana Ljashenko}
\textbf{Андрій Смолій} оце щойно пройшли з чоловіком 4 км По Позняках (мешкаємо
тут), все, що чули дорогою, було російською, і якісь аніматори в парку волали
на все горло теж російською

\iusr{Андрій Смолій}

\textbf{Oxana Ljashenko} я тому і вказав, що центром. На Позняках буваю дуже рідко. Хоча дивно, там мала б бути українська, бо є багато новобудов.

\iusr{Oxana Ljashenko}
\textbf{Андрій Смолій} ні, тут української мало, особливо це стосується молодих батьків і дітей

\iusr{Лиза Ребар}
\textbf{Максим Меркулов} дякую за розуміння

\iusr{Українською Будь Ласка}
\textbf{Андрій Смолій} багато новобудов - багато приїжджих - всі хочуть здаватися "гарадскімі" - катастрофічно мало української. Серед малих дітей значно менше, ніж було 20 років тому. Гурток українською знайти практично неможливо
\end{itemize} % }

\iusr{Max Wolf}
Почему разговаривать на русском это неправильно?  @igg{fbicon.face.smiling.eyes.smiling} 

\begin{itemize} % {
\iusr{Максим Меркулов}
\textbf{Max Wolf} Хто сказав, що це неправильно?? У сучасних реаліях це - САМОГУБСТВО.

\iusr{Світлана Трембач}
\textbf{Max Wolf} 

Не правильно мерця в труну поперек ложити, а освіта в Україні повинна
надаватись українською мовою. Тому що "вот пайдьот в садік, виучіт укрáінскій",
"пайдьот в школу виучіт укрáінскій", "пайдьот в інстітут, виучіт укрáінскій" і
виходять з інститутів спєци, які впевнені, що української термінології не
існує, що "спалахуйка" українське слово, що "горобець" смішніше за "варабєй" і
що всі, од пуцьвірінка до старця, розуміють їхню балачку.

Що стосується приватного спілкування, говорите російською чи німецькою нікого
не обходить.

\iusr{Max Wolf}
\textbf{Світлана Трембач} ну так я и не говорю чтоб учили на русском, в посте было сказано что разговаривать на русском это неправильно.
Вот и спросил.

\iusr{Світлана Трембач}
\textbf{Max Wolf} Допис, взагалі-то, про мову виховних занять в дошкільному закладі)

\iusr{Serhii Bryhar}
\textbf{Max Wolf} Давайте все ж розділяти приватну і публічну сфери. Жоден
закон не в праві регулювати приватний простір. Але якщо людина працює в
освітній сфері, у сфері торгівлі, або просто приходить на телебачення, радіо,
то було б їй дуже непогано таки розмовляти державною мовою. Власне, у нас уже
щось подібне: україномовні часто-густо розмовляють українською десь на кухні, а
в публічному просторі переходять на російську. Цікава схема, але мало би бути
навпаки. Бо це Україна, і публічний простір тут має бути наповненим
українськими наративами.


\iusr{Сергій Білоног}
\textbf{Max Wolf} 

на територіях де російської занадто багато, рано чи пізно приходить Росія, і до
самої Росії ставляться лояльно, відповідно для України як держави це шкідливо,
бо створює міну уповільненої дії, коли з часом ці території відійдуть до Росії
або відділяться від України.

У 20 столітті Україна таким чином втратила Кубань, де раніше розмовляли
українською, але перестали, а в 21 столітті Крим і частину Донбасу

\iusr{Max Wolf}
\textbf{Serhii Bryhar} 

про публічний простір питань нема. Но мне не нравится формулировка-государство
должно объяснять детям что разговаривать на русском это неправильно.

Чего ж неправильно? Ну он так с детства научен, в семье говорит, что ж теперь с
ним делать? Вряд ли такими формулами можно привить любовь до української.

\iusr{Max Wolf}
\textbf{Сергій Білоног} по большей части согласен.

\iusr{Serhii Bryhar}
\textbf{Сергій Білоног} 

На жаль, це цілком про моє і ваше, Сергію, місто. Так,
можна говорити про різне, і не все вже аж так однозначно, але розслаблятися і,
як деякі, казати: "у нас в місті все нормально", як на мене, не варто. Отже,
все просто: або української більшає, або напрям руху і, на жаль, кінцевий
результат, зрозумілі...

\iusr{Serhii Bryhar}
\textbf{Max Wolf} Я теж народжений в Одесі, мене теж виховували російською. І
все довкола тоді, на початку 90-х, було російською. Просто нормально вивчити
мову я спромігся лише ближче до повноліття. Тепер розмовляю нею постійно.. Але
якийсь дебільний, чесно кажучи, шлях. Складний.. Часи великих змін, так би
мовити.. Все може бути простіше і органічніше.. Що ж до "пояснень": держава -
не людина, яка сяде поряд і пояснить, тобто це слово слід сприймати як
можливість держави з найменшого віку продемонструвати людям, що в українській
публічній сфері повинна домінувати українська мова - це нормально. Відповідно,
якщо це нормально, то ненормальна зворотна ситуація - коли там домінує
російська.

\iusr{Max Wolf}
\textbf{Serhii Bryhar} 

ну это ваше право, а я например не собираюсь переходить
на украинский. Хотя за государственный язык один-украинский.

\iusr{Надія Тимошенко}
\textbf{Max Wolf} 

Пояснювати треба батькам, а не дітям. Недавно зустріла маму двох чудових
випускників, у обох уже діти. Жіночка в статусі бабусі в розпачі - онука
виховувалась вдома в російськомовному середовищі, в школі не тільки не розуміє
матеріал, але й опирається вивченню української. Шкільний стрес для
першокласниці супроводжується ще й мовним.

Якби батьки були свідомими того, що дитина матиме освіту державною мовою, то
мусили б змалку формувати мовне середовище.

Ніхто не забирає право на материнську мову. Але багато залежить від батьків. По
сусідству циганочка переживала, що доня не зможе вчитися в українській школі,
бо домашні знають лише ромську та російську. Я дещо порадила, і мала циганочка
пішла в школу вже підготовленою навіть без садочка.

\iusr{Serhii Bryhar}
\textbf{Max Wolf} Так ніхто від вас, чи будь-кого іншого цього не вимагає.
Відмовлятися чи ні - приватна справа кожного. Але серед наших співгромадян є
чимало тих, хто державною мовою і двох речень зв'язати якось не дуже вміє, і
оце вже питання, в якому держава мала би дати собі раду, але не може. Чи не
хоче.. А ще є очевидний факт: там, де таких людей забагато, починається щось
дуже нехороше. В нас теж починалося. І, дякувати богові та всім небайдужим,
припинилося. Принаймні до якогось часу..

\iusr{Max Wolf}
\textbf{Serhii Bryhar} ну ничего, припиним ещё раз, если понадобится.

\iusr{Serhii Bryhar}
\textbf{Max Wolf} Та воно так, але хотілося б уже відвернути саму небезпеку як таку.

\iusr{Лариса Данілевська}
\textbf{Max}! тому, що знову виросте покоління москворотих.

\iusr{Тетяна Лук'янова}
\textbf{Max Wolf} , ну так і спілкуйтеся тоді тільки з такими, як ви.

\iusr{גיל מוזיקה}
\textbf{Max Wolf}

В Україні, будучи російськомовним, сказати "я принциплво не переходитиму на
українську, це моє право, і все, їжте це" не вимагає сміливості чи зусіль,
тому-що російська і її носії знаходяться в Україні у привелийованому становищі,
а українці дискриміновані. От змогли б ви сказати у Сполучених Штатах чи у
Мексиці "я принципово не переходитиму на місцеву мову, мене так батьки у
дитинстві навчили, і тому це моє святе право, а всі кому це не подобається
можуть відвалити"?


\iusr{Max Wolf}
\textbf{Лариса Данілевська} они и так и так вырастут, в семьях русскоязычных такие же дети.

\iusr{Max Wolf}
\textbf{גיל מוזיקה} 

как-то не замечаю дискриминации.

И сам ни кого не дискриминирую. Поэтому и не понимаю зачем кого-то заставлять
переходить на другой язык. Только потому что диванным патриотам так хочется?

\iusr{Max Wolf}
\textbf{Тетяна Лук'янова} а я с вами и не говорил @igg{fbicon.face.smiling.eyes.smiling} 

\iusr{Suzanne Protsenko}
\textbf{Max Wolf} логічно, що російськомовний не помічає дискримінації

\iusr{Тетяна Лук'янова}
\textbf{Max Wolf} , от і не говоріть. Дуже дивно, коли люди собі спілкуються українською, а тут влізає таке принципово рсукоєзичне, якому неодмінно треба довести до відома усіх, що він на українську переходити не буде.

\iusr{Max Wolf}
\textbf{Suzanne Protsenko} а в чём она выражается?

\iusr{Suzanne Protsenko}
\textbf{Max Wolf} перейдіть на українську і побачите

\iusr{Max Wolf}
\textbf{Тетяна Лук'янова} а вы делите людей по принципу языка?  @igg{fbicon.face.smiling.eyes.smiling} 

\iusr{Max Wolf}
\textbf{Suzanne Protsenko} я тут уже спрашивал. Спрошу у вас-зачем мне переходить?

\iusr{Тетяна Лук'янова}
\textbf{Max Wolf} , ну то хай вони забудуть, що інші діти мають обов'язково знати російську.

\iusr{Suzanne Protsenko}
\textbf{Max Wolf} ви на ПолітКлуб підписані. Віталій Портников чудово пояснює, чому необхідно переходити на українську мову. Не лінуйтесь, пошукайте. Тут вас інформаційно обслуговувати ніхто не зобов’язаний.

\iusr{Лариса Данілевська}
\textbf{Max}! Так хоч мову буду знати, спілкуватися зможуть, сто наче тупі - "не знаю, не вчила, не понімаю" ( стикалась з такими туупарями)

\iusr{Max Wolf}
\textbf{Suzanne Protsenko} я говорил на эту тему с Портниковым.
И в большинстве с ним согласен.

\iusr{Тетяна Лук'янова}
\textbf{Max Wolf} , після деяких подій -так. Один з таких епізодів, це коли в Акерманській фортеці місцевий коваль-реконструктор почав щось моїй доньці розповідати російською. Я, вважаю, зробила ласку, перейшла на ту саму мову і попросила його говорити повільніше, щоб я могла перекласти... Вам сказати, що я за свої гроші, заплачені за квиток, вислухала потім? Всі, хто живе в Україні, зобов'язані вміти розмовляти українською, а от знання російської не обов'язкове.

\iusr{Suzanne Protsenko}
\textbf{Max Wolf} перейдіть на українську хоча б для того, щоб відповісти на власне запитання: «в чьом проявляєтся діскрімінация?»
Тоді ви зрозумієте хто «дєліт людєй по принципу язика».

\iusr{Max Wolf}
\textbf{Suzanne Protsenko} манипулируете  @igg{fbicon.face.smiling.eyes.smiling} 

\iusr{Тетяна Лук'янова}
\textbf{Max Wolf} , після деяких подій -так. Один з таких епізодів, це коли в Акерманській фортеці місцевий коваль-реконструктор почав щось моїй доньці розповідати російською. Я, вважаю, зробила ласку, перейшла на ту саму мову і попросила його говорити повільніше, щоб я могла перекласти... Вам сказати, що я за свої гроші, заплачені за квиток, вислухала потім? Всі, хто живе в Україні, зобов'язані вміти розмовляти українською, а от знання російської не обов'язкове.

\iusr{Suzanne Protsenko}
\textbf{Max Wolf} перейдіть на українську хоча б для того, щоб відповісти на власне запитання: «в чьом проявляєтся діскрімінация?»
Тоді ви зрозумієте хто «дєліт людєй по принципу язика».

\iusr{Max Wolf}
\textbf{Suzanne Protsenko} манипулируете  @igg{fbicon.face.smiling.eyes.smiling} 

\iusr{Світлана Трембач}
\textbf{Suzanne Protsenko} 

Пан не розуміє, що обмежує у праві на рідну мову тих українців, що йому тут
відповідають, хто це читає, також ображає тих, хто не знає російської, або з
певних причин не хоче чути й розуміти її. Облиште. Ви ж не вступаєте в
дискусії, коли звертаються мовою, яку не знаєте чи погано розумієте (хоча,
зауважте, інші нації й не дозволяють собі писати в публічних обговореннях
рідними мовами). Два-три речення і відвертаємось. Будь-де на державну мову
ввічливість передбачає відповідати державною, а ось ці говорилки іноземною
-звичайне свинство.

\iusr{Костянтин Писаренко}
\textbf{Max Wolf} 

Тому що це мова правічного лютого та підступного ворога.  @igg{fbicon. tongue} "Язик" є носієм
скаліченого генного коду московитів , нацистського вірусу... Не секрет що
московити з самого зародження ,як держави ,у 13ст були за своїми діями
нацистами.. Згодом червоний нацизм породив брунатного.. Питається, якого хріна це
штучно утворена калічка має бути в Україні?

\iusr{Max Wolf}
\textbf{Костянтин Писаренко} закусывайте.

\iusr{Олена Вілівчук}
\textbf{Max Wolf} московське нарєчіє має мати кордони в межах свого мокшанського болота .

\iusr{Степан Сторонський}
\textbf{Max Wolf} Потому што Путін прийде тебе визволяти а нас завойовувати щоб і ми заговорили московською.

\iusr{Max Wolf}
\textbf{Степан Сторонський} так он уже пришёл @igg{fbicon.face.smiling.eyes.smiling} 

\iusr{Max Wolf}
\textbf{Олена Вілівчук} сильный аргумент  @igg{fbicon.face.smiling.eyes.smiling} 

\iusr{Степан Сторонський}
\textbf{Max Wolf} Завдяки російській мові. ..

\iusr{Андрій Глух}
\textbf{Max Wolf} можливо тому що Україна це не Росія і якщо хтось незнає нагадую у нас в країні війна йде і взагалі це мова окупантів

\iusr{Max Wolf}
\textbf{Степан Сторонський} думаете другого повода бы не нашёл

\iusr{Max Wolf}
\textbf{Андрій Глух} и что теперь что это язык оккупантов? Путин его приватизировал?

\iusr{Андрій Глух}
\textbf{Max Wolf}

\ifcmt
  ig https://scontent-frx5-1.xx.fbcdn.net/v/t39.30808-6/242143184_856877571682743_9060154112474220284_n.jpg?_nc_cat=105&_nc_rgb565=1&ccb=1-5&_nc_sid=dbeb18&_nc_ohc=ghqBmqn9BjIAX_AEwrf&_nc_ht=scontent-frx5-1.xx&oh=cd8a77ea770932a7823c7501a3098495&oe=614C391D
  @width 0.3
\fi

\iusr{Max Wolf}
\textbf{Андрій Глух} понятно, когда аргументов нет в ход идут псевдо цитатки.

\iusr{Андрій Глух}
\textbf{Max Wolf} якщо ти неповажаєш мову країни в якій ти живеш то що тобі тут робити їдь туди де тебе зрозуміють а ми вже достатньо натерпілися від тих кому какая разніца

\iusr{Галина Бабицька}
Москва оддвічний кат народів і ворог Україні, злодії-крадії
Все крадуть!
Кирилиця первісно була Руси, а Русь була Київською
Стибрили все ,щей зазіхають на територію

\iusr{Андрій Глух}
\textbf{Max Wolf} чи ти може один з цих що поза політикою?

\iusr{Max Wolf}
\textbf{Андрій Глух} а хто сказал что неповажаю. Поважаю. Но разговариваю на том каком привык.

\iusr{Андрій Глух}
\textbf{Max Wolf} можливо вже пора щось міняти в своєму житті якщо хочеш жити краще

\iusr{Max Wolf}
\textbf{Андрій Глух} я что-то неуверен что переход на украинский улучшит мою жизнь  @igg{fbicon.face.smiling.eyes.smiling} 

\iusr{Андрій Глух}
\textbf{Max Wolf} у світі стільки багато мов а ти обрав мову народу який намагається нас винищити

\iusr{Max Wolf}
\textbf{Андрій Глух} так не я обрав, в детстве научили.

\iusr{Андрій Глух}
\textbf{Max Wolf} я думаю що ти вже дорослий щоб розрізняти що добре що погане і яка рідна мова а яка мова окупантів які вбивають наших людей на сході

\iusr{Max Wolf}
\textbf{Андрій Глух} я разговаривал и до оккупантов. И вообще как язык может мешать убивать тех же оккупантов? @igg{fbicon.face.smiling.eyes.smiling} 

\iusr{גיל מוזיקה}
\textbf{Max Wolf}
"На каком привьік". А чому це у вас повинні бути більші права ніж в мене? Я теж так хочу, ось я приїду в Україну і принципово буду поважати місцеву мову. Глибоко у серці, як ви. А розмовляти буду якою звик, тобто на івриті. І хочу, щоб мені у житті в Україні було так само зручно, як вам. А що, ви іноземною, і мені теж можна іноземною. Чому ні?

\iusr{Олена Вілівчук}
\textbf{Max Wolf} змалку ти звик срати в штани ,продовжуй в тому ж дусі !

\iusr{Андрій Глух}
\textbf{Max Wolf} спочатку знищать нашу мову і нашу культура а потім і нас якщо мова неважлива то чого її стільки разів намагались знищити і заборонити?

\iusr{Олена Вілівчук}
\textbf{Max Wolf} війна прийшла саме туди де не знають правдивої історії України та не чути української мови!

\iusr{Max Wolf}
\textbf{גיל מוזיקה} ваше дело @igg{fbicon.face.smiling.eyes.smiling} 

\iusr{Max Wolf}
\textbf{Андрій Глух} кто сказал что неважлива.

\iusr{Oleg Sass}
\textbf{Max Wolf} патамушта, ти живеш в Україні

\iusr{Max Wolf}
\textbf{Oleg Sass} и?

\iusr{גיל מוזיקה}
\textbf{Max Wolf}
Ось це і є то що ви безсумлінно насолоджуєтесь вашим привелийованим становищем як російськомовного

\iusr{Oleg Sass}
\textbf{Max Wolf} маєш знати мову країни в якій проживаєш і чий хліб їш, а не то пи@дуй в мокшу і там живи

\iusr{Max Wolf}
\textbf{גיל מוזיקה} насолоджуюсь и шо теперь? @igg{fbicon.face.smiling.eyes.smiling} 

\iusr{Max Wolf}
\textbf{Oleg Sass} уж точно не твой хлеб ем

\iusr{Костянтин Писаренко}
\textbf{Max Wolf} Вчіть історію не по моцкальських "пісюльках.".

\iusr{Oleg Sass}
\textbf{Max Wolf} я б тебе послав, але бачу ти тільки звідтам повернувся @igg{fbicon.face.tears.of.joy} 

\iusr{גיל מוזיקה}
\textbf{Max Wolf}

Тобто ви самі погоджуєтесь з тим що ваше становище привелийовано у порівнянні з
тими хто знає тількі українську, без російської, і також погоджуєтесь з
несправедливістю яку це становище викликає?

\iusr{Max Wolf}
\textbf{גיל מוזיקה} так а я при чём, это их проблемы. Я знаю оба языка.

\iusr{Світлана Літвінова}
\textbf{Max Wolf} З 14го-так!

\iusr{גיל מוזיקה}
\textbf{Max Wolf}

Але ж не будете розмовляти українською коли почуєте її, і навіть коли вас
попрохають. Навпаки, ви сподіваєтесь що кожен хто почує вашу російську відразу
ж переде на неї.


\iusr{Max Wolf}
\textbf{גיל מוזיקה} ну вы ж не перешли и я вас прекрасно понимаю.

\iusr{גיל מוזיקה}
\textbf{Max Wolf}
Може і зараз зрозумієте:
אולי גם עכשיו תבין אותי?

\iusr{Oleg Sass}
\textbf{גיל מוזיקה} та воно ж туге, і щелепа не та @igg{fbicon.face.tears.of.joy} 

\iusr{Max Wolf}
\textbf{גיל מוזיקה} я ж не в Израиле @igg{fbicon.face.smiling.eyes.smiling} 

\iusr{גיל מוזיקה}
\textbf{Max Wolf}
Але ж також і не в Росії

\iusr{Людмила Смолякова}
\textbf{Max Wolf} жаль Вас, нічого не допоможе

\iusr{Max Wolf}
\textbf{Людмила Смолякова} себя пожалейте @igg{fbicon.face.smiling.eyes.smiling} 

\iusr{Max Wolf}
\textbf{גיל מוזיקה} тем не менее в Украине меня понимают

\end{itemize} % }

\iusr{Inna Mytsa}

Біда в тім, що діти в усіх садочках не розуміють української бо батьки вдома
розмовляють іншою мовою, мультики вони дивляться іншою мовою. Доводиться
перекладати кожне слово і так цілий день. Але маю надію що через кілька місяців
почнуть розуміти і хоч через слово говорити. Хоча батькам це зовсім не
потрібно...

\begin{itemize} % {
\iusr{Максим Меркулов}
\textbf{Інна Мица} Не так давно спілкувався з дівчиною, яка вперше приїхала до України з Підмосков'я. Вона добре розуміла українську та білоруську, хоч і не вивчала ці мови раніше.

\iusr{Inna Mytsa}
\textbf{Максим Меркулов} коли ви говорите чотирирічній дитині "сідай на стілець"- а дитина питає мене "вьі со мной всегда по-немецки говорить будете? Я не понимаю что делать" - я розгубилася, чесно кажучи. А доросла дівчина, маючи більш-менш нормальну освіту і досвід, звичайно має розуміти і українську, і білоруську, і англійську.

\iusr{Вікторія Лопата}
\textbf{Інна Мица} 

побуду вашим мікропромінчиком надії @igg{fbicon.face.grinning.sweat} я навмисне обмежила ворожомовний контент,
вчуся не реагувати на звернення російською, бо виявилося, що це дуже незручно.
Коли ти на майданчику, як дурепа відповідаєш чужим дітям українською на їх
російську, вони чомусь думають, що твоя дочка її теж розуміє, а це не так, бо
їй от от 3 і не ми не розмовляємо з нею російською.

Навіть якщо Київ не порозумнішає за найближчі 15-20 років, то це не завадить
моїм дітям отримати пристойні оцінки в школі з мови, та продовжити освіту, а
про що ті батьки-креоли собі досі думають, я не уявляю.

\end{itemize} % }

\iusr{Євген Шпичка}

Завідуюча садочку сказала мені про виховательок: "Ну ані же каренниє адессітки,
оні па украінскі так говорят, что лучше уже на рускам"

\begin{itemize} % {
\iusr{Georgiy Rusanovsky}
\textbf{Євген Шпичка} коренні адесітки в більшості з області

\iusr{Євген Шпичка}
\textbf{Georgiy Rusanovsky} так отож, просто ж так усім хочеться бути "городськими"
\end{itemize} % }

\iusr{Georgiy Rusanovsky}
\textbf{Наталка Стус} маючи обмеження вони навчають других

\iusr{Ntina Ntoubrova}

Так, це погано, коли діти навіть НЕ РОЗУМІЮТЬ мову. Це означає, що сидять вони
на російських українофобських + совкодрочерських росіянських соцмережах. Бо
інших в РФ немає - а в нас ніхто не робить.

Я от знайшла навіть одного нашого ютубера, що веде канал російською про
українську нашу школу. Вже хотіла рекомендувати знайомим замість росіянського
лайна гайпових лайливих топ-блогерів для підлітків в СНГ, бо дійсно дуже цікаво
та модерново і БЕЗ МАТЮКІВ (!!!) робить контент, я дійсно аж зачарувалася...
Тому для переведення підлітків у НАШУ інфосферу такий канал був би в допомогу..

Аж тут виявилося, що цей український блоххер понаробив ще й купу відео "нашим
дітям про життя в СССР і як класно тоді жилося". Бо, вочевидь, є запит від
росіян, плюс він підсвідомо орієнтується на стандарт рос.блогерів - от він і
рад старатися робити те, про що в нас загалом взагалі мало хто обговорює чи
цікавиться.. (( І тим самим він ВЖИВЛЮЄ цей трешовий дискурс, цю архаїчну
тематику, в мізки ж НАШИХ дітей та підлітків, що його дивляться.... ((

Тому що через одну мову він не має почуття кордонів та розуміння відмінності
своєї країни від росії в нього немає... Тому тема закрита ((

\begin{itemize} % {
\iusr{Максим Меркулов}
\textbf{Ntina Ntoubrova} 

Спілкувався з росіянкою, що без проблем розуміла білоруську та українську, хоч
і не вивчала ці мови ніколи.

\iusr{Ntina Ntoubrova}
\textbf{Максим Меркулов} 

Бо в неї немає родичів, які русифікувалися щоб не відчувати себе в РФ людьми
другого сорту. От вона й вивчила мову.

В нас та в Білорусі українці з дитинства себе відчувають вторинними. А оскільки
здорова амбітність бере своє - то вони, будучи гнучкими та open-minded, не
намагаються ламати світ під свої правила - а швидесенько знаходять шпаринки в
ньому щоб просочитися і зайняти місце в ієрархії повище. Якщо заради цього
треба перейти на російську вони це зроблять. Бо гроші та статус для них є
ціннішими за свою гідність ( = вміння настояти на своєму).

\end{itemize} % }

\iusr{Олег Коваленко}

В мене вікна виходять на приватний садочок.

Чистіше там лунають російськомовні сучасні пісні і московячі народні

\begin{itemize} % {
\iusr{Georgiy Rusanovsky}
\textbf{Олег Коваленко} московські народні що хозяїна знали,а українських і більше і вони наші,навіть там де мову забули у Румунії,Парагваї,пісні ще пам'ятають,а тут червона рута і Одесі звучала чи не на кожному кроці
\end{itemize} % }

\iusr{Danya Kononov}
кожного ранку прохожу дит садок, чую українську і від дорослих і від дітей

\iusr{Володимир Олівець}

Важливо, але ж де його взяти у самостійній Україні! В Києві спробуй знайти
гурток українською!

А в Коцюбієві мовна Білорусь. Знаю про що кажу, гірше лише в Криму.

І мова тут не про закони, а про людей, яким зручно окупантити й нема на то ради

\iusr{Марина Макогон}
Все раніше починається! З колискової! Або ні: на думку фахівців майбутня дитина
вже відчуває все до свого народження!

\iusr{Надія Рудько}

От читаю коментарі і здається мені , що обговорюється біда якоїсь зовсім іншої
держави.. В своїй країні ми рахуємо відсотками людей, які розмовляють
українською.. В УКРАЇНІ - УКРАЇНСЬКОЮ у відсотках!!! Це НОНСЕНС!!!

\iusr{Dmytro Dzyuba}

Наче й про Одесу мова, а таке як про якусь рязанщіну.

Освіта в Україні має бути всюди державною українською мовою, за замовчуванням.

У нас на Запорожжі дитсадки і школи українськомовні.

І коли \textbf{Іван Мелобенський} розповідає, як його змушують перекладати
затверджений українською мовою предмет інформатики на "общєпанятньій узкій
язьік" в адєсскай школє №1, це щонайменше дивно.

\begin{itemize} % {
\iusr{Ярослав Попович}
\textbf{Іван Мелобенський} а хто змушує? Адміністрація школи чи рускоязикаті батьки?

\iusr{Ivan Melobensky}
\textbf{Ярослав Попович} у нас типу дистанційний 4 клас відкрили російською мовою, всі предмети так викладають
\end{itemize} % }

\end{itemize} % }
