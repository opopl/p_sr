% vim: keymap=russian-jcukenwin
%%beginhead 
 
%%file 04_10_2021.fb.gorovyj_ruslan.1.jazyk_posty_facebook.cmt
%%parent 04_10_2021.fb.gorovyj_ruslan.1.jazyk_posty_facebook
 
%%url 
 
%%author_id 
%%date 
 
%%tags 
%%title 
 
%%endhead 
\subsubsection{Коментарі}

\begin{itemize} % {
\iusr{Ярослав Матюшин}

Аби писати українською, мені треба замислюватись та зосереджуватися. Але ж
іноді пишеш з бодуна, і тоді лінькі вмикати башню.

\begin{itemize} % {
\iusr{Вікторія Валевська}
Ярославе, не придурюйтесь. З вашими то здібностями до мов...
\end{itemize} % }

\iusr{Соня Поліщук}
Інколи фб нагадує пости тих врємьон,соромно бляха і якісь вони бездушні як і той язик....

\iusr{Інна Долгушина}
Коли знаходжусь в дуже емоційному стані. Наприклад, обурена або розлючена, можу написати російською. Мова агресора, фіглі.

\iusr{Natalya Lyutikova}
Теж не розумію цього.

\iusr{Mariya Syemashkina}

Бо думаю все одно російською. Хоча нею вже майже не читаю книжок. Але є одне
але) пости з відгуками на книжки пишу все ж виключно українською, так що це
питання часу, щоб і все інше писати українською)

\begin{itemize} % {
\iusr{Евгений Кузин}
\textbf{Mariya Syemashkina} 

Це питання внутрішнього переналаштування. Коли я почав курс Upper Intermediate
англійської, почав перемикати звукову доріжку в кіно на англійську, читати
англомовні джерела новин, книги. Перші півроку було важко капець, потім почав
ловити себе на тому, що свої внутрішні діалоги часто веду англійською, не
помічаючи цього, дуже прикольне відчуття. Звісно, англійська в мене -
необхідність, мої робочи дні проходять виключно англійською. Тому думаю, що це
питання умов, якщо середовище сприяє, мова розвинеться до вільного використання
і на рівні підсвідомості.


\iusr{Mariya Syemashkina}
\textbf{Eugene}, 

моя робоча мова теж англійська) і німецькою я думаю іноді, проте мій внутрішній
голос думає переважно російською і, так, живу в російськомовному середовищі

\end{itemize} % }

\iusr{Vitally Sladko}

Все просто. Разговорный язык повседневного общения - русский. Посты в фб - это
как обычное общение, потому пишут на том языке, на котором разговариваю.

В Украине живу, вижу ее исключительно как независимое государство, а РФ как
врага, который пытается в течение веков уничтожить Украину.

Рус язык, как и англ язык на мой взгляд не является маркером принадлежности к
ценностям современной РФ.


\iusr{Ruslan Shloma}

Хто знa. Я от дивлюся по своїм друзям. Всі спілкуються між собою українською. Я
навіть і не згадаю, чи чув від них російську мову. Але у фейсбуці, чи в
інстаграмах всяких пишуть виключно російською. Навіть між собою у вайбері. Щоб
було по-городському. Бо і ми, Химко, люди. Я із-за цього вийшов з групи нашої
компанії.

\begin{itemize} % {
\iusr{Юрій Горішний}
\textbf{Ruslan Shloma}, прикро таке спостерігати... Дійсно, люди, яким проблемно вимовити російською дві фрази, пишуть все російською з жахливими помилками...

\iusr{Halyna Rudyka}
о, і я таке бачила!
\end{itemize} % }

\iusr{Евгений Кузин}

Mother tongue, переважно. Особисто для себе будь-яку мову вважаю перш за все
інструментом обміну інформацією, тому загальний принцип маю - відповідати, за
можливості, мовою того, хто звертається. Пишу першим російською просто тому, що
з усіх мов максимально вільно нею володію. Але, мабуть, не зовсім відповідаю
вимогам, бо в частині "хочу щоб дитина залишалась жити тут" - абсолютно ні, і
хочу щоб дитина мала достатньо знань і вмінь, щоби існувати поза державними
кордонами і обирати місце комфортного існування незалежно від того як
намальована політична карта світу. В контексті мови тому більше переживаю за
знання англійської.

\begin{itemize} % {
\iusr{Наталія Гурошева}
\textbf{Евгений Кузин} Ви вжили слово, яке означає язик як орган , а "мова" англійською - language. Вибачте)

\iusr{Евгений Кузин}
Це усталений англійський вираз, що означає "рідна мова"

\url{https://dictionary.cambridge.org/ru/словарь/английский/mother-tongue}

\iusr{Елена Данкина}
\textbf{Eugene Kuzin} От насправді - вік живи, вік учись) Я теж сіпнулася, а тепер завдяки вам знаю щось нове для себе!

\iusr{Евгений Кузин}

Оооо, англійські ідіоми - це просто прекрасно, і дуже весело ))) Рекомендую
сайт urbandictionary.com, та канал "Fucking English" в Телеграмі - жива мова,
особливо в порівнянні з прямим перекладом слів виразу - дуже весело щось таке
нове дізнаватись.

\iusr{Наталія Гурошева}
\textbf{Евгений Кузин} дякую за науку! Була така підозра...)) Ви іі підтвердили, спасибі, знатиму!

\end{itemize} % }

\iusr{Станислав Кушнаров}
Пишу української. За винятком жартів, коли гра слів виглядає такою саме російською.

\iusr{Лана Рисенко}

Дехто ніяковіє писати українською через те, що у френдах є реальні знайомі, які
знають вас, як російськомовного.

Наприклад, в кулінарній групі до мого френда приколупався його реальний
знайомий типу "а с какіх ето пор, Саньок, ти так на мовє пісать начял? Я етава
нє панімаю, твоя мать к тєбє как абращалась?" Ну і в такому стилі. Я як читала,
то аж кипіла. Але френд дуже гарно і спокійно все пояснив. Не всі витримують
оцих підйопок.

\iusr{Лана Рисенко}

\ifcmt
  ig https://scontent-frx5-1.xx.fbcdn.net/v/t1.6435-9/244448007_1725656014297590_8293952530889128628_n.jpg?_nc_cat=105&_nc_rgb565=1&ccb=1-5&_nc_sid=dbeb18&_nc_ohc=JrJwlfhUWFgAX-QxHW0&_nc_ht=scontent-frx5-1.xx&oh=59b51863e4547e219bb49c8b875dac7d&oe=617FDA6E
  @width 0.4
\fi

\iusr{Ruslan Gorovyi}
\textbf{Лана Рисенко} достойно

\iusr{Valeria Petrova}

Намагаюся писати українською, але певна кількість дописів таки рос. У таких випадках:

1) коли дууууже поспішаю

2) коли щось мене сильно вразило, і є бажання якомога точніше транслювати свої
відчуття, та і взагалі детально розповісти, а українською я буду писати три
години зі словником, і не впевнена, що напишу саме те, що відчуваю

\iusr{Yulia Klochko}

Ще є такий феномен: людина зі мною спілкується українською, а повідомлення пише
виключно російською. Правда, дехто після запитання "чому так?" переходять на
рідну в листуванні.

\iusr{Александр Вектор}

Все розумію, все можу. В розмові деколи незручно, бо голова думає російською і
доводиться підбирати слова. Якщо коротко відповідати, то це звичка. Якщо більш
розлого, то це становиться куди легше, коли навколишнє середовище україномовне.
Якщо ні (а це й досі доволі значний відсоток, менше ніж було, але все одно є),
то відбуваються рецидиви. Мозок сам по собі становиться на старі зручні рельси
і так далі.

Щодо написання російською і чому... Ну поперше таж сама звичка. Іноді не
встигаєш замислитися. Є думка і ти її ізлагаєш. Крім того, деякі речі, як на
мене, російською мені більш реальні (якщо можна так сказати). Але все одно.
Основна проблема, це звичка)


\iusr{Natalia Gnydiuk}

Я можу зрозуміти, чому пишуть пости російською, бо не хочуть писати з
помилками, але отчого взагалі не можу зрозуміти, чому багато хто, хто пише
пости українською з дітьми розмовляє російською.

\begin{itemize} % {
\iusr{Мария Скакунова}
\textbf{Natalia Gnydiuk} бо коли наші батьки з нами розмовляли російською, то і до своїх дітей так. Якщо родина в побуті російською.. Це автопілот. Почати українською хоча б писати, це вже маленький крок до змін  @igg{fbicon.smile} 
\end{itemize} % }

\iusr{Вікторія Валевська}
От зараз читала Бутусова і думала про це. Хоча загалом мені подобається, що він пише.

\iusr{Андрей Свистула}
Буття визначає свідомість.

Нажаль, у повсякденному бутті в російськомовному середовищі і з купою побутових
проблем, проблем зі здоров'ям , роботою і таке інше, іноді, в ситуаціях де те
має принципових ознак, вже не вистачає сил і натхнення перебувати у стані
війни.

Прошу не перекручувати і не додавати моєму коментарю зміст, котрий я в нього не
вкладав

\begin{itemize} % {
\iusr{ірина болгарова}
\textbf{Андрій Свистула} 

навіщо бути у стані війни - просто бути у українському стані.. можна просто у
громадських місцях ,частіше , у магазинах та у фб розмовляти державною..


\iusr{Ruslan Gorovyi}

не зрозумів останній посил... якщо ви переживаєте, що хтось щось не зрозуміє і
перекрутить, сформулюйте так, щоб все було однозначно і зрозуміло...


\iusr{Андрей Свистула}
\textbf{Ruslan Gorovyi}

Гарний приклад наш діалог. Я, людина з російськомовного регіону і майже на 100\%
російськомовного оточення шукаю для себе нішу, де я можу відчути себе українцем
на своїй землі. Для цього я приєднуюся до україномовних спільнот і підписуюся
на україномовних лідерів думок. Але що я маю на виході коли вступаю в діалог чи
ділюся своїми сумнівами? Майже зі 100\% верогідностю я чую у відповідь, що я
недостатньо принциповий, недостатньо наполегливий, що я якийсь неповноцінний
недоукраінець. І все це у середовищі, котрому я не ворог і в якому шукаю сенси
для себе. А що коїться на львівських майданчиках? Там, якщо ти не 770 років
родом зі Львова і в тебе прізвище не Фаріон або Дроздові, то ти маєш тільки
слухати не відкриваючи рота...

А тепер уявіть собі ситуацію, коли ти знаходишся у недружньому середовищі, у
нашому випадку за мовними ознаками. Негативом тебе спалять за лічені дні, може
тижні, може місяці.

Про те я і написав, що в ситуаціях, які не носять принциповий характер,
наприклад у спілкуванні з російськомовними товаришами, використовую російську
не замислюючись про те

\iusr{Ruslan Gorovyi}
\textbf{Андрій Свистула} 

тепер все чітко і зрозуміло. Дякую вам за відповідь. Оскільки я сам був з
такого регіону і середовища, то дуже добре вас розумію.

\end{itemize} % }

\iusr{Віктор Ярмолюк}

У мене більше питань до земляків з Волині, що після переїзду до Києва стали не
тільки розмовляти російською, а й писати свої пости. Не розумію, чому? Це
якийсь прояв совкового комплексу неповноцінності? Поясніть, не розумію.

\begin{itemize} % {
\iusr{Ruslan Gorovyi}

тут комплекс всього,.. на моїй пам'яті дівчина з прекрасною українською в Києві
за 5 років зросійщилася повністю. це було ще до війни.... як зараз я не знаю.

\iusr{Олена Бабенко}
\textbf{Віктор Ярмолюк} 

Моя племінниця навіть не виїжджала нікуди з Волині, блін. 17 років дитині.
Російської навіть у школі не вчила. І при цьому вся її інста в узкоязичних
цитатах штибу ОК/ВК. Звідки це гівно в її голові - не уявляю.  @igg{fbicon.face.woozy} 

\iusr{Віктор Ярмолюк}
\textbf{Олена Бабенко} , 

якщо подивитись тренди Ютубу, то одразу все стає зрозумілим. У мене донька 8р.
недавно питала, чи знаю я про Моргенштерна, бо її однокласники слухають, а там
багато слів "які напевне матюки".. А усі блогери з великою багатомільйонною
аудиторією - російськомовні, тож діти просто тягнуться до того, що їм здається
ознакою популярності та успішності.

\iusr{Катерина Ващук}
\textbf{Віктор Ярмолюк} оо та сама фігня...) мене аж підкинуло як побачила пости однієї землячки

\iusr{Ольга Щербак}
\textbf{Віктор Ярмолюк} 

Щодо Ютубу, то там український контент — це невтомна праця з очищення від
російського. Є такий фокус: здавалося б очистила пропозиції рашаконтенту, але
не дай Боже, навіть українське щось російською подивишся — кришка, лізтиме в
усі щілини. Що цікаво, то ні з англійським, ні з польським, ні з німецьким чи
французьким такого не траплялося ні разу (уточню, французькою не володію,
просто музику якусь слухала). Тому для подивитися щось із підозрілого контенту
у мене на комп'ютері стоїть браузер, де на тому ж Ютубі ніхто не залогований,
щоб не загажувати профіль.  @igg{fbicon.face.rolling.eyes} 


\iusr{Тетяна Бурлак}
\textbf{Олена Бабенко} від блохерів, яких вони дивляться.

\end{itemize} % }

\iusr{Ирина Мастакова}

Записую думку в тому вигляді, як вона виринула у свідомості. Тобто лише
занотовую свій внутрішній голос. Як воно в голові прозвучало - так і буде,
переклад не роблю. Іноді воно українською, а іноді російською. Буває і
мішанина, суржик.

\begin{itemize} % {
\iusr{Виктория Кононова}
\textbf{Ирина Мастакова} 

у мене теж саме так. Іноді у випадках "російськомовної" думки вдається
переключити її на українську, іноді - ніяк слова не підбираються.

Ще помітила, що важко говорити й писати українською, коли дуже втомлена або зпросоння.

\end{itemize} % }

\iusr{Вячеслав Раєвський}

Повністю підтримую, за час війни не скажу що сто відсотків але переважна
більшість російськомовних друзів стали не тільки писатии але і спілкуватись
українською

\iusr{Людмила Храпачевская}

Чому, коли я пишу українською мовою до людини, яка спілкується рузькою, ( до
колеги, до подруги) , я відчуваю провину, що цим її ображаю.... Що робити з
цим?

\begin{itemize} % {
\iusr{Ruslan Gorovyi}
це все говорить про вас, а не про них... вони ж вам не казали, що ображаються?

\iusr{Лана Рисенко}
\textbf{Lyudmila Hrapachevskaya} бо ви відчуваєте себе кращою, ніж вони.

\iusr{Людмила Храпачевская}
\textbf{Ruslan Gorovyi} Я їх не звинувачую. Але ,що зі мною! Чому?

\iusr{Ольга Щербак}
\textbf{Lyudmila Hrapachevskaya} Нас привчали до того, що до тих, хто говорить російською обов'язково переходити на російську, а не роблячи цього Ви наче порушуєте правила. З тої ж причини (точніше її віддзеркалення) деякі російськомовні так негативно ставляться до українізації публічного простору і волають про насилля над ними, хоча ж жоден закон не втручається в особисту сферу — бояться втратити ексклюзивне становище. Просто українцям досі незвично мати право бути українцем в Україні.  @igg{fbicon.eyes} 
\end{itemize} % }

\iusr{Andrii Kryvolap}
Десь з 2016-го став постити солов'їною. Хоча в побуті ще відсотків на 70 спілкуюсь свнсбчою(

\iusr{Natali Blashchyshena}
Мене теж завжди це цікавило, послухаю.
Тим більше тут не катять відмови "боюсь сказать нє правільно" тому що акценту тут не видно, а в гуглі завжди можна глянути потрібне слово якщо вже дуууже хочеться "правильно"

\iusr{Антон Селин}

Язык это личное, его использование не требует пояснений. Исключения прописаны в
языковом законе - это публичные выступления, госслужба и сфера обслуживания,
если я не ошибаюсь.

\begin{itemize} % {
\iusr{Ruslan Gorovyi}
то навіщо ви написали комент? тобто якісь пояснення такі треба?

\iusr{Антон Селин}
\textbf{Ruslan Gorovyi} вы спросили, я ответил. Кажется, в посте было приглашение к дискуссии

\iusr{Ruslan Gorovyi}
\textbf{Anton Selin} я говорив про власний досвід, а ви мені вирішили розповісти про закон.

\iusr{Антон Селин}
\textbf{Ruslan Gorovyi} я ж не про закон писал, а про то, что язык общения это личное, не требующее пояснений.
Проте, коли не очікують пояснень, то зазвичай я переходжу на мову співрозмовника, тобто моя російська це не принцип, це просто моя рідна мова.

\iusr{Ruslan Gorovyi}
\textbf{Anton Selin} дякую. зрозуміло
\end{itemize} % }

\iusr{Оксана Лоза}

Пишу виключно українською , стала помічати , що мені складно іноді знайти
російський відповідник тому чи іншому слову .

\begin{itemize} % {
\iusr{Елена Данкина}
\textbf{Оксана Лоза} Те саме. Але найдивніше коли моя донька, що живе за кордоном і її мови - англійська та іврит, так само в розмові чомусь вставляє українські слова, дуже природньо і невимушено)
\end{itemize} % }

\iusr{Яків Бриль}

Коли ще задовго до війни почав переводити діловодство на українську, дуже
помічним був MS Word з увімкненим жорстким контролем правопису. А от зараз вже
мені значно важче писати російською, аніж українською. Англійською ще важче,
але то таке

\iusr{Натали Натадзе}

У мене багато родичів в Європі і в Грузії, які не розуміють українську

\begin{itemize} % {
\iusr{Яків Бриль}
\textbf{Natali Natadze} ეს ახლა პრობლემაა?
\end{itemize} % }

\iusr{Tatyana Misiyuk}

Звичка перш за все. Бо з російськомовного Запоріжжя (хоча українська була у
бабусі в Умані, тож і з 4 років українською вже читала). А потім освіта
роійською на англійською, робота та спілкування також постійно англійською
була, до Майдану україномовного середовища практично не було. Коли почало
з'являтись, мене за акцент за діаспору вважали). Зараз спілкування письмового
набагато більше, а в житті - ні. Тому звичка залишається, хоча все частіше
стала українською писати, інколи навіть з россійськомвними в месенджері
перехожу. Треба час. Не хочу це робити в вигляді принципу " а от від сьогодні
ніякої російської". Має бути натуральним

\begin{itemize} % {
\iusr{Ruslan Gorovyi}
все має бути так, як вам хочеться. це головне

\iusr{Tatyana Misiyuk}
\textbf{Ruslan Gorovyi} 

ну просто дууууже рідко хтось власне може заявити мені на моїй де сторінці,
мовляв, чого не державною. От тут я на рога встаю. Бо інакше нормально
спілкуюсь українською. Але аби мені не тикали, як мені у себе писати. Реально
дух протиріччя вмикається. А ще у одного друга, у якого глибинне розуміння теми
націі, цікава теорія на тему, що насправді не всім і треба розмовляти
українською, бо це також свого роду маркер. От я, наприклад, точно не можу
сказати, що вишиванка для мене - це мій генетичний код. Бо більше польське
коріння мало значення, українського набагато менше. Тому воно і вкладається в
певну схему


\iusr{Tatyana Misiyuk}
Треба ж не забувати, що дійсно не для всіх українська - рідна мова

\iusr{Ольга Щербак}
\textbf{Tatyana Misiyuk} Ой, леле! Мене уродженку Полтавщини, яка прожила тут усеньке життя регулярно називають западенкою, бо, бачте, не послуговуюся словами: поняв, обідився тощо. Я про що? Хай собі вважають.  @igg{fbicon.wink}  @igg{fbicon.face.smiling.halo} 

\iusr{Tatyana Misiyuk}
\textbf{Ольга Щербак} я і западенців знаю з такими словами  @igg{fbicon.smile}  та чого тут мірятись взагалі?
\end{itemize} % }

\iusr{Oksana Kotich}

як показує практика, українську чудово розуміють навіть самі російськомовні, що
письмову, що розмовну. Більш того, за півроку роботи у переважно
російськомовному колективі мої колеги навчились відповідати мені українською.
Без примусу чи якогось там ультиматуму з мого боку. Просто завжди спілкуюсь
українською.

\begin{itemize} % {
\iusr{Tatyana Misiyuk}
\textbf{Oksana Kotich} ну так мова ж не про діалог, я також українською відповідаю завжди, але особисті дописи - то інше

\iusr{Oksana Kotich}
\textbf{Tatyana Misiyuk} може я просто хотіла похвастатись?

\iusr{Tatyana Misiyuk}
\textbf{Oksana Kotich} ну таке) коли нема примусу і тикання носом. люди поступово звикають використовувати украінську. Бо більшості дійсно не вистачає практики. Кіно. ТБ, Інтернет - це пасивне використання. Тому і те, що в магазинах українською обслуговують - це також на користь і іншим - звичка відповідати українською поступово у багатьох з;являється
\end{itemize} % }

\iusr{Anna Radysh}
Іноді - розраховуючи на френдів з інших країн, що російською володіють, а українською - ні.

\begin{itemize} % {
\iusr{Олександр Рощин}
\textbf{Anna Radysh} для цього є перекладач. Мене читають багато людей з інших країн що української не розуміють

\iusr{Лариса Прядкіна}
\textbf{Олександр Рощин}
Перекладач настільки спотворює текст і зміст, що мої рідні досить часто мене не розуміють зовсім.
Так перекладають, що іноді лишається тільки покрутити пальцем біля скроні.
Хороших автоперекладачив мало, у мене є один адекватний, але в ньому нема української.
Виникає друге питання: чому в українському сегменті ФБ досі нема суто українських адміністраторів та модераторів?
Від кого це залежить?
При вирішенні цього питання, певно, і перекладачи покращалися б.
\end{itemize} % }

\iusr{Андрей Свистула}

Руслане, кожного літа їдемо з сім'єю в Тернопільську область у відпустку (теща
переїхала туди в 14му з Донецька). Намагаюсь якнайбільше спілкуватися з
місцевими, придивляюся і прислухаюся. Іноді волосся на голові шевелиться:

1. Стільки шансону, як там, в Дніпрі в транспорті не почуєш.

2. Ми ходимо до церкви МП тому, що туди завжди ходили наші батьки.

3. Отам, між Кременцем і Збаражем проходить кордон, котрий розділяє галичан і
волинян. Якщо щось і буде, то ми повернемося під Польщу і будемо жити як і
раніше до 1939го року.

4. Ми, тут у Тернопіллі недолюблюемо Львів'ян, тому що вони були під австріяками...

От таке я чую в землях, котрі мають буди колискою всього українського.

І я уявляю відношення тих людей до мене, українця зі сходу. Відкрито, багато
хто з заходу України готові відмовитися від своїх східних співвітчизників разом
з тими ватанами.

Мої спостереження дають підстави мені вважати, що в нас досі не сформульована
національна ідея, котра дає поняття, що нас об'єдную і поєднує.

Але все і не так погано, як могло бути

\begin{itemize} % {
\iusr{Ruslan Gorovyi}

ми всі вражені імперією. хтось більше, хтось менше. хтось чинить опір, хтось
ні... хтось мав змогу говорити українською з дитинства, а хтось повертається
свідомо... такі часи.


\iusr{Оксана Атаманчук}
\textbf{Андрій Свистула} 

є різні люди з різними думками)і все,... можливо ви думаєте що ці землі МАЮТЬ
БУТИ колискою.. та я думаю що ці землі і Є колискою, і не тільки ці землі ,вся
наша земелька є колискою)))бурянчику просто багацько, тому ,полоти-сапати))))і
все буде добре),ну я так думаю))

\end{itemize} % }

\iusr{Валерий Шамрай}
Вчора видалив трьох ,,друзів",яких знаю особисто, за листівки російською мовою, надоїли.
А можливо вже нерви.

\begin{itemize} % {
\iusr{Ruslan Gorovyi}
якщо ви людей не знаєте, то нема чого й жаліти. якщо ж знаєте - тут ви самі вирішуйте. це ваша зона комфорту

\iusr{Валерий Шамрай}
\textbf{Ruslan Gorovyi} Просто я їм натякав, що таких картинок в інтернеті на українській мові повно. Мабуть какая разніца.
\end{itemize} % }

\iusr{Olga Malyshko Demydova}

якщо не заради срача, а заради дослідження, то - я не знаю. інколи слова просто
так складаються у речення - російською. інколи німецькою. інколи англійською.

\iusr{Євгенія Гуржій}
І мені цікаво!
Поділюсь)

\iusr{Святослав Воробченко}

1. Привычка. Всю жизнь жил в русскоязычной семье. Русский знаю лучше
украинского.\par Переучиваться-доучиваться лень. Хотя иногда бывает, что пишу на
украинском. Но пока что это исключение из правил, и больше, как уважение к
собеседнику, или если из пени слов не выкинешь, или если есть настроение. Почти
всегда проверяю свои тексты на отсутствие ошибок (не хочу, чтобы проскакивало
что-то типа "останівка"). Уже если писать, то писать правильно. На это нужно
время и желание.

2. Не дорос ментально.


\iusr{Oksana Ruban}
Патамушо какаяразніца.

\iusr{Петр Москалец}

Часто роблю репост інформації мовою оригіналу(московитською), бо вважаю більш
актуальним зміст, а не форму. Багато шанованих авторів пишуть московитською, і
перекладати не бачу сенсу. По великому рахунку, мова, це засіб спілкування, і
від людини залежить як цей засіб використовувати. Яби теж хотів, щоб в Україні
була українська мова, але в найближчому майбутньому так не буде. Може через
покоління.

\end{itemize} % }
