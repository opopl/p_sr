% vim: keymap=russian-jcukenwin
%%beginhead 
 
%%file poetry.rus.lina_kostenko.translations
%%parent poetry.rus.lina_kostenko
 
%%url 
%%author oleho,lina kostenko
%%tags 
 
%%endhead 

\subsubsection{Переклади}

\Purl{https://stihi.ru/2013/02/18/1378}

* * *
Умирают маэстро, оставляя нам памяти раны.
В барельефах печаль высекает мгновения плеть.
Подмастерья еще не успели взрасти мастерами.
А работа не ждет. Нужно многое в жизни успеть.
 
Неизвестно откуда приходят безликие боссы.
Потирая ручонки, берутся за всё и всегда.
Пока гений стоит, утирая усталости слёзы,
бесталанная рать засидела святые места.

Очень странный пейзаж: косяками гуляют таланты.
Свод священный небес наклоняет к себе суета.
С мастерами все проще. Они – как земные Атланты.
Держат небо века. И стоит оттого высота.

* * *
Моя любовь, я вновь перед тобою.
Бери меня в блаженство своих снов.
Не сделай лишь послушною рабою
под непосильной тяжестью оков.
Не допусти, чтоб свет сошелся клином
и не приспи, зачем я жизнь живу.
И подари над трактом тополиным
старинную от солнца булаву.
Не дай по жизни мелко суетиться,
убереги от выбоин своих,
чтоб кости не ворочались в гробницах
забытых гордых прадедов моих.
У них любовь была хмельнее мёда,
у них от страсти меркнул  небосвод.
Их женщины цеплялись за стремёна
и плакали, но только до ворот.
А там, а там… ужасный клекот боя
и звон мечей в ристалищах весны. 
Моя любовь, я вновь перед тобою.
Веди меня в свои ночные сны.

* * *
Мой первый стих написан был в окопе,
на взрывами потрёпанной земле,
когда затмило звёзды в гороскопе
моё больное детство на войне.

Лилась пожаров огненная лава.
В седых стояли кратерах сады.
И захлебнулась кровью переправа
под натисками вздыбленной воды.

И белый свет казался ночью чёрной.
Зарницы тьмы соперничали с днём.
Окопчик тесный стал подводной лодкой
средь моря чувств, растерзанных огнём.

Не сказки  мир с зайчишкой или волком –
кровавый час, где плавилась земля.
И я писала, чуть ли не осколком,
большие буквы, как из букваря.

Мне бы играть в считалки или прятки,
летать во сне на крылышках страниц.
А я стихи в замасленной тетрадке
таила от осколочных убийц.

Ужасна боль недетских впечатлений.
Рубцы на сердце от её ножа.
В плену страданий вилась нить сомнений,
не станет ли немой моя душа.

Душа в словах – как море в перископе
и память та – как блики на челе.
Мой первый стих, написанный в окопе,
печатался на раненной земле.

* * *
И скажет мир: - Ты малое во мне,
пустая боль в житейском буреломе.
Твоя любовь – мираж в бредовом сне
и вера – мысль, затерянная в слове.

Чем можешь ты, наивное дитя,
содействовать всемирным переменам?
- Могу я пот и слёзы бытия
пожертвовать неправедным победам.

Но знаю я: исчезнет пелена,
история насытит свою злобу.
Захочется не крови, а вина
ей выпить за любовь и за свободу!

* * *
А жизнь идёт и всё без корректур
и шаг времен всё ближе до галопа.
Давно уж нет маркизы Помпадур
и мы  живем давно после потопа.

Не знаю я, что будет после нас,
какие платья выберет природа.
Единственно бессмертный – это час,
а нас зовет к свершениям работа.

Спеши творить, пускай не чудеса.
Мы все уйдём, исчезнем словно тени,
лишь бы небес глазурные глаза
всегда смотрели на земли цветенье.

Еще бы мир не стал планетой бурь
и ссудный день не стал кончиной судеб.
Ведь жизнь идёт и всё без корректур
и как напишешь, так оно и будет.

Не бойся слога грешного строки
и боли чувств, что увлажняет веки.
Не бойся истин, пусть они горьки,
и тех печалей, что текут как реки.

Святую душу бойся обмануть.
Солгавши раз – окажется навеки.

* * *
В летах ты, фрау Магда иль Луиза.
Над миром злым – холодные ветра.
У нас еще распахивают ныне
железо Круппа в поле трактора.

Ну как там вальс – еще танцуют в Вене?
Как доктор Фауст – борется со злом?
Уж сколько лет ребят убитых тени
живут в домах за рамочным стеклом.

Я не скажу тебе плохого слова.
Твой тоже не вернулся с той войны.
- За что он дрался, Магда? – Вопрошаю снова.
- Он не кричит «Хайль Гитлер!» со стены?

* * *
Уже началась, да, началась
мольба за новь  грядущих дней,
а то, что с прошлого осталось,
торопит вдаль своих коней.

Не обезличивайте ценность,
не затеряйтесь в толчее.
И не меняйте совершенство
на сто нелепостей в себе.

Уходят фронды и жиронды,
уходят слава, звон фанфар.
Ищите таинство Джоконды,
её улыбка – вечный дар.

* * *
Боюсь я слов, когда они молчат,
когда они безмолвно затаились,
когда не знаешь, как они звучат.
Они другим давно уже приснились.

Ведь ими кто-то воспевал любовь,
их чудный хор другими растревожен.
Людей незримо и не счесть их слов,
а ты впервые их промолвить должен.

Все было уж: уродство и красивость,
асфальт дорог и тишина глуши.
Поэзия – всегда неповторимость,
бессмертное касание души.

* * * 
Не допускай мизерных мыслей.
Бессмертие местами есть.
Коль искру Божью фатум высек,
прими призвание за честь.

Чужды поэту лесть и слава.
Его удел – будить народ.
Ему нужна одна награда –
высоких Слов нетленный свод. 
