% vim: keymap=russian-jcukenwin
%%beginhead 
 
%%file 06_01_2022.fb.gorovyj_ruslan.2.padenie_mamonta_rossia.cmt
%%parent 06_01_2022.fb.gorovyj_ruslan.2.padenie_mamonta_rossia
 
%%url 
 
%%author_id 
%%date 
 
%%tags 
%%title 
 
%%endhead 
\zzSecCmt

\begin{itemize} % {
\iusr{Наталка Пасхіна}
Саме так. Мать спілкується, я - фашистка))

\begin{itemize} % {
\iusr{Ruslan Gorovyi}
\textbf{Наталка Пасхіна} та я навіть не намагаюся. У кожного свій шлях

\iusr{Ксеня Мальована}
\textbf{Наталка Пасхіна} доць, ти фашист, бо ходиш на марші Бандери. З Великоднем!

\iusr{Українською Будь Ласка}
я б не зміг спілкуватися навіть крізь зуби про погоду

\iusr{Наталка Пасхіна}
\textbf{Сергiй Одаренко} я і не спілкуюся. А мать звідтам, як і батько.
\end{itemize} % }

\iusr{Тетяна Жданова}

рускій мір - це монголо-татарська навала, тільки у наш час. І складається
враження, що крім Укрвїни протистояти їм поки що ніхто не в змозі

\begin{itemize} % {
\iusr{Yeseniya Yesenia}
\textbf{Тетяна Жданова} не ображаєте татар

\iusr{Yeseniya Yesenia}
І монголів також  @igg{fbicon.wink} 

\iusr{Тетяна Жданова}
\textbf{Yeseniya Yesenia} та так
\end{itemize} % }

\iusr{Марина Черевань}

Так само, дві двоюрідні сестри там. Дійсно, нема про що говорити. А ще багато
вихваляються. Зомбували маму дзвінками жахіттями, особливо на дідивоєвалі. Але
вона була стійкою, сміялася тай годі. Так і пішла у засвіти українкою. Якісь
вони примітивні, бояться і слова сказати. Жалію.

\iusr{SvitLana Horosh}

\ifcmt
  ig https://i2.paste.pics/95387ab397771a8e117c10473b9936f8.png
  @width 0.2
\fi

\iusr{Oksana Skrypniuk}

Двоюрідний брат мами за совка віїхав жити в Іркутськ. Останнє, що чула про
нього, їхав воювати під Донецьк... Сподіваюсь, там і лишився.

\begin{itemize} % {
\iusr{Ruslan Gorovyi}
\textbf{Oksana Skrypniuk} йолоп, блядь... вибачте

\iusr{Oksana Skrypniuk}
\textbf{Ruslan Gorovyi} так і є, чого там
\end{itemize} % }

\iusr{Таня Ярошенко}

Наша родина припинила всі стосунки з родичами і з Мордору, і з Криму ... Зараз
навіть не знаємо, чи живі ще ...

\iusr{Valentyna Boyko}

Я спілкуюся з російськими родичами, хоч і не часто. Навчила їх говорити \enquote{в
Україні}. Пояснила, \enquote{чем провинился Горький} (саме це спитав у мене дядько з
Нижнього Тагілу, коли я сказала йому, що мою вулицю перейменували, і тепер це
вулиця Володимира Антоновича). Ну і іншу роз'яснювальну роботу потрошку веду.

Річ у тім, що вся моя російська рідня - це нащадки мого розкуркуленого і
висланого з родиною в Мєдвєж'єгорськ прадіда. Лише моя бабуся після заслання
повернулася на рідну Полтавщину. Один з її братів осів в Оренбурзі, інший - в
Севастополі (ось з його нащадками у мене зараз зв'язку немає), сестра - в
Нижньому Тагілі. Але всі вони підтримували дуже тісні зв'язки все життя:
листувалися, їздили в гості. Далі ці зв'язки підтримували діти, тепер черга
онуків, хоча тут вже значно складніше, бо в гості вже давно ніхто не їздить.

Таких розкиданих по \enquote{1/6 частині суші} сімей дуже багато. І якщо є
можливість спілкуватися з ними, розказуючи їм те, чого вони не знають зі своїх
ЗМІ (і вони все це не просто слухають, а й розуміють, підтримують), то це
потрібно робити.

Не хочеться втрачати зв'язки, які так берегли мої бабуся і мама.

\iusr{Pudlyk Svitlana}
@igg{fbicon.heart.red} @igg{fbicon.heart.black}  тримаймося.

\iusr{Myroslava Prots}

Маю ФБ друга з Сирії. \textbf{Жизнь Новый}. Він більше розуміє про проблеми
московитів, ніж вони самі. Бо й там той демон наробив такої біди, що ніхто з
місцевих жителів ніколи не пробачить... Чому воно лізе всюди, чому не займеться
своїм спитим зубожілим народом???

\iusr{Зинаида Василенко}
Я також.

\iusr{Helga V. Topolska}

троюрідна тітка живе в Москві, вийшла заміж у 1986 і переїхала до чоловіка.
причому їхала туди з сильно задертим догори носом бо вона ж типу в сталіцу їде,
а ті хто в Києві лишився, то вони хуторяни. іронічно шо в усіх \enquote{хуторян} все в
житті склалося краще ніж у неї. ми не спілкуємося взагалі. в неї є син Женя,
який типу є мені кузеном, але я з ним навіть не знайома. ну й фіг з ними.

\iusr{Тамара Свеженцева}
Аналогічно

\iusr{Галина Поворознюк}

Та мабуть у більшості є там родичі. Такий безжальний совковий міксер позмішував
все. Але в мене вже навіть і в соцмережах спілкування нема. Останнє відвалилось
року з півтора як. Чи маю потреба у спілкуванні? Ні.

\iusr{Любов Крижановська}

Я маю унікальний випадок бо моя свекруха пані Потоцька, поїхала туди жити, а ми
поїхали звідти.

Брат мого чоловіка, його жінка їхня донька, моя свекруха всі вони народжені у
Львові. Їхні батьки і прадіди були львів'янами. Тепер вони всі расіянє,
масквічі. Звичайно вони не такі кінчені, як переважна більшість насєлєнія, але
замбування дуже сильне. Спілкується чоловік, але все менше. Нема про що
говорити...

\iusr{Ксеня Мальована}

Я думала, чому не сумую за мамою. За сестрою - так. Повинна ж за мамою.

От. Повернулась. Якщо б я ходила до психолога - в нього б додалось роботи.

А так мені просто хочеться скинутись під поїзд метро після канцу тнмк, щоб все
добре було.

Та сподіваюсь, що вийде знайти роботу.

Бо якщо ніт, і оце продовжиться вже в рашці, то краще вже точно під поїзд

\begin{itemize} % {
\iusr{Лидия Ткач-Слиденко}
\textbf{Ксеня Мальована} 

Варто триматися, допоки вітер в лице, тобто допоки живі. Під поїзд ніколи не
буде пізно, хоча згідна з вами - мовчанка, ізоляція, то важко. Кріпімось.

\end{itemize} % }

\iusr{Тетяна Ігорівна}
Кожне слово підтримую.

\iusr{Полина Кривцова}

Скільки тут історій... Батько чоловіка в Росіі взнав, що ми не святкуємо 23
февраля, питав що може ми боїмось, нам забороняють, \enquote{притисняют}, може ми в
Росію б переїхали. Щоб пояснити, довелось би цілий курс по історіі читати. Мої
предки по мамі з розкулачених і частково розстріляних. Вже й пояснювати не
хочеться, якщо досі не цікавились.

\begin{itemize} % {
\iusr{Ruslan Gorovyi}
\textbf{Polina Kryvtsova} розумію тебе

\iusr{Лариса Кожемякина}
Здається, що у всіх нас живуть там рідні Чужі...
\end{itemize} % }

\iusr{Iryna Yatsyshyn Danyluk}
Цікаво, з Вашого досвіду, наскільки відрізняються українці в Росії від росіян

\begin{itemize} % {
\iusr{Ruslan Gorovyi}
\textbf{Iryna Yatsyshyn Danyluk} 

ні на скільки. Не можна жити, ходити на роботу, всмоктувати наративи коаїни яка
вбиває Україну і бути українцем. Можна бути громадянином урср, але не України

\iusr{Лидия Ткач-Слиденко}
\textbf{Iryna Yatsyshyn Danyluk} 

Запитайте ліпше, наскільки українці в Україні відрізняються від українців в
Україні. З рідних вже в живих лише одна сестра. Путінопріді-стка.
Демонстративно перейшла з української на російську. Жде. Зі мною розірвала всі
зв"язки. Церква МП зробила свою справу (((

\iusr{Iryna Yatsyshyn Danyluk}
\textbf{Лидия Ткач-Слиденко} це інше питання, теж маємо таких \enquote{родичів}

\iusr{Андрей Королевский}

Наскільки сильно відрізняються українці з прикордонного села в Харківській
області від українця з села в Білгородської?

\end{itemize} % }

\iusr{Женя Малахова}
Моя найбільша мрія теж саме така: побачити смерть цієї імперії зла

\begin{itemize} % {
\iusr{Ruslan Gorovyi}
\textbf{Євгенія Лисак} сила силенна людей цього хоче

\iusr{Valentyna Boyko}
\textbf{Ruslan Gorovyi} 

Ось думаю: може, хоч об Казахстан зуби обламають? Не тому, що він дасть гідну
відсіч, а тому, що просто не подужають стільки маріонеткових режимів
підтримувати: Придністров'я, Абхазія, Південна Осетія, ОРДЛО з Кримом, тепер ще
й Казахстан.

Там Байконур для росіян таке ж сакральне місце, як і Севастополь.

\iusr{Ruslan Gorovyi}
\textbf{Valentyna Boyko} не думаю, що так все просто... потоплять в крові

\iusr{Valentyna Boyko}
\textbf{Ruslan Gorovyi} 

Радянська імперія і на межі 80-90-х в крові топила - країни Балтії, Закавказзя.
Після чого монстр СРСР розвалився.

Дуже хочеться вірити, що так само розвалиться, не подужавши переварити, і
нова-стара росімперія.

\iusr{Ruslan Gorovyi}
\textbf{Valentyna Boyko} тоді совок був слабкий... зараз рейх кристалізований

\iusr{Valentyna Boyko}
\textbf{Ruslan Gorovyi} 

Але надто багато намагається захапати і втримати за своїми межами. \enquote{Жадность
фраера сгубила} - сподіваюся, що і з ними таке станеться ще на моєму віку.

\end{itemize} % }

\iusr{Тимофей Кулишенко}

А я б не хотів до цього дожити... Краще хай тихенько живцем зогниє... Да, довше
воняти буде, да, некомфортно, але хвиля від падіння наробить лиха більше...

\begin{itemize} % {
\iusr{Ruslan Gorovyi}
\textbf{Timofey Kulishenko} воно повалиться як картковий будиночок... історія свідчить, що відбувається лише так

\iusr{Тимофей Кулишенко}
\textbf{Ruslan Gorovyi} повірю наслово)
\end{itemize} % }

\iusr{Татьяна Снопок}
+1

\iusr{Сашко Олександр}
Паскуди змінили танки на снайперів.

\iusr{Олексій Дручок}
Ментальна прірва

\iusr{Ruslan Gorovyi}
\textbf{Олексій Дручок} так і є

\iusr{Iryna Safonova}

Наші родичі, які живуть на рашці, подзвонили нам у 2014 році запитати, \enquote{что там
у вас происходит?}. Після того, як ми їм популярно все пояснили, вони навіть
видалились у мене із друзів у фейсбуці. З того часу жодного разу не
спілкувались. Та, навіть, не знаю, про що можна було б говорити, бо дійсно ми -
два різних світи. Спілкуючись з ними до 2014 року ми на власні очі бачили, як
пропаганда змінювала їх світогляд і відношення до України

\begin{itemize} % {
\iusr{Ruslan Gorovyi}
\textbf{Iryna Safonova} розумію

\iusr{Tatyana Misiyuk}
\textbf{Iryna Safonova} мене рідна сестра - кримська вата- давно забанила...

\iusr{Yeseniya Yesenia}
\textbf{Tatyana Misiyuk} а я свою кримську родичку з українським корінням забанила за березневе 2014 «ми вєрнулісь до дому»
\end{itemize} % }

\iusr{Наталя Гриценко}

Я це гарно наблюдаю у випадку з однокласником. Льотчик. Живе в Іваново, ще й
має дядьку в Луганську...

\begin{itemize} % {
\iusr{Ruslan Gorovyi}
\textbf{Natalie Gritsenko} у мене однокласників дохуя теж там... військове містечко. Та більшість з них ніколи і не були упраїнцями

\iusr{Дмитро Вагабунд}
Бабусина сестра з Горлівки мамина хрещена, пиздила її як снопа і кляла. Бо відітє не так живе. русскій мір блять...
\end{itemize} % }

\iusr{Олексій Смірнов}
Таж сама історія.

\iusr{Любовь Свиридова}

Мені тітка з Москви заявила, що я вже сміюсь як бандерівка. І якби ж була
росіянкою, а то до сорока років жила в Україні, народилась в селі Харьківської
області. От вам чим відрізняється українець в Росії від росіянина: нічим


\begin{itemize} % {
\iusr{Yeseniya Yesenia}
\textbf{Любовь Свиридова} це комплімент  @igg{fbicon.sunflower} 
\end{itemize} % }

\iusr{Svitlana Shabarova}
То правда говорити нема про що питаю у брата про здоров'я тітки іф сьо

\iusr{Алена Лесечко}
Сказав як Боженька !

\iusr{Tanja Engel}
У нас з 2014 жодного контакту, ні днів народжень, взагалі повна тиша з рускими родичами

\iusr{Julia Dembitska}
+1
не спілкуюся зовсім після того як мамин брат з москви в 2014 сказав: «путін вас уму-разуму научіт». Потім він плакався і хотів спілкуватися: «я ж твой радной дядя!».
Ні, дякую.

\iusr{Володимир Східняк}

Нехай здихає зі своїм народом. Від малого до великого. Нах нам біженці з росії?
Хоча рибу чимось годувати треба, як це роблять багато продуманих країн.


\iusr{Marta Marta}
Нудить від всіх родичів москальских

\iusr{Єгор Кравець}
російські родичі - мертві, бо живі

\iusr{Тетяна Лук'янова}

Інша родина з батькового боку спілкується з тими родичами, що живуть на расєї.
Я хіба що привіт можу передати. Мені ще до війни з ними не було про що
балакати, бо мову не зберегли. Із сім'єю маминого брата, який на відміну від
сестри \enquote{понавернувся} у рідний Пітер, тим паче. Намагалися спочатку не чіпати
політики, але коли я зателефонувала привітати з днем народження, а це старе
опудало почало ляпати \enquote{я твой враг, ти мой враг}... Та пішли вони всі нахуй,
якщо не виздихали. І дядько, і дядина, яка під час останнього приїзду у 2010
тринділа, що це на Волзі голод був, а у нас у 1947 не було. І брат, у якого
\enquote{жена хахлушка} і в тому, що тоді ж якесь смоленське шмаркате чмо до нього,
50-річного кандидата наук, почало тикати (бо він народився у нашому місті) і
хохлом обзиватися, теж винна Україна. І все їхнє кодло теж.

\iusr{Halyna Kaiser}

Дякую за правду, яку ви розповсюджуєте по світу. Ну і ваше гасло: \enquote{....і не
забувайте хто х..... ло!} - це просто, як домашнє завдання кожній свідомій
людині))

\iusr{Oleg Futuyma}

Влучно підмічено про вбивць, що ходять поруч. Пам’ять - вона без виміру в часі.
Моїм першим авто в штатах ще двадцять років тому був старенький Мерседес, який
мало не вбив мене. Одного дня при з’їзді з швидкісної траси в нього відмовили
гальма. Взагалі! Чудом обійшлося без наслідків. Пізніше на халяву припав ще
один старенький турбодізєль. Та сама ситуація, відмовляють гальма, на двох
колесах роблю викрутаси і уникаю аварії. А потім згадую, що прадіда мого
Тихона, який до війни був трохи у Гамериці, спалили живцем німці разом з
худобою. Прадіду мабуть з Небес не дуже подобалось, що ганяю на німецьких
машинах. Так само ні один кореєць не придбає японський автомобіль чи телевізор,
бо у них добре працює пам’ять про мільйони загиблих.

\iusr{Ольга Выросткова}

Я з 14го року не спілкуюсь з сестрою та батьком
@igg{fbicon.face.smiling.sunglasses}  @igg{fbicon.face.rolling.eyes}  Кінець
13го/початок 14го ще намагалась щось пояснити на їхнє \enquote{дочєчька,
пріїжайтє всє, ми вас прімєм, мєста многа, а то у вас там рускіх прітєсняют}...
Намагалась щось пояснити, але марно. А коли чоловіка мобілізували, вони там
образились чомусь: \enquote{Ета он чо, в нашіх будєт стрілять?}

- Та вас же ж там нет, йопта!  @igg{fbicon.face.smiling.sunglasses}
@igg{fbicon.beaming.face.smiling.eyes}  Ну й все. Ми для них бандерівці та
фашисти, а вони нам на всі наші пояснення про рос.агресію: \enquote{А чо
такова?}

Тому, нема у мене більше рідні  @igg{fbicon.frown} 

\iusr{Максим Повненький}

ех((( тут далеко ходити не потрібно. Повно люди який в Україні втратили
ідентичність і не розуміють хто кого обстрілює... Дивляться рашуТВ і будь які
намагання спробувати переконати, натикаються на агресію

\iusr{Anatolii Dybets}
Ти не один.

\iusr{Катруся Косілкова}

На московії батьки. Спілкуємося раз-два на тиждень про \enquote{битовуху}. Зі знайомими
якось не складається, вони чомусь коли чують, що рф має бути знищена як
найкінченіше утворення, як держава-терорист і окупант, не хочуть далі
спілкуватися... @igg{fbicon.shrug} 

\iusr{Ольга Пашковская}

Чим далі, тим більше пишаюся, що ні з маминої ні з татової сторони у мене немає
російського коріння. Мама чистокровна українка, у тата по лінії батька польське
коріння, а по лінії матері українське. Не хочу мати нічого спільного з цим
недолугим утворенням під назвою росія!

\begin{itemize} % {
\iusr{Ruslan Gorovyi}
\textbf{Olga Pashkovska} генетика така штука, що може бути все шо хош. Однак я теж не хочу мати з ними нічого спільного

\iusr{Ольга Пашковская}
\textbf{Ruslan Gorovyi} Так, від генетики нікуди не дінешся, згодна. Просто радію, що і генетично далека від них. Наскільки знаю про своїх пра-пра

\iusr{Ruslan Gorovyi}
\textbf{Olga Pashkovska} я зробив аналіз... виявилося що в мене багато збігів з тернополянами... хоча я й гадки не мав. Отак

\iusr{Stepan Kravets}
\textbf{Olga Pashkovska} Ну, місце народження та батьків не вибирають, а от світогляд та свої прагнення так) Тому я хоч і по крові наполовину росіянин, але повністю за Україну)

\iusr{Зоряна Берездецька}
\textbf{Ruslan Gorovyi} цікаво, шо то за аналіз? Теж би хотіла зробити такий:)
\end{itemize} % }

\iusr{Тетяна Носир}
Росіян настільки зомбують їхні ЗМІ ( причому дуже давно), що вони українців просто ненавидять!

\iusr{Oleksiy Sokil}
Честь Друже

\iusr{Ольга Прокурашко}
Чекаю теж!!! А я вмію чекати!

\end{itemize} % }
