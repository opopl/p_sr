% vim: keymap=russian-jcukenwin
%%beginhead 
 
%%file 09_12_2020.news.ua.strana.2.smert_pfizer_vaccine
%%parent 09_12_2020
 
%%url https://strana.ua/news/305747-smert-pri-ispytanii-amerikanskoj-vaktsiny-pfizer.html
 
%%author 
%%author_id 
%%author_url 
 
%%tags covid_vaccine,pfizer
%%title В США заявили о смерти людей при испытаниях вакцины Pfizer от коронавируса
 
%%endhead 
 
\subsection{В США заявили о смерти людей при испытаниях вакцины Pfizer от коронавируса}
\label{sec:09_12_2020.news.ua.strana.2.smert_pfizer_vaccine}
\Purl{https://strana.ua/news/305747-smert-pri-ispytanii-amerikanskoj-vaktsiny-pfizer.html}

\index[rus]{Коронавирус!Вакцина!Pfizer, смерть людей, 09.12.2020}

\ifcmt
pic https://strana.ua/img/article/3057/smert-pri-ispytanii-47_main.jpeg
caption При испытании вакцины были летальные исходы. Фото: freepik 
\fi

Разработанная американским концерном Pfizer \Furl{https://strana.ua/news/305027-amerikanskaja-kompanija-pfizer-vypustit-menshe-vaktsin-ot-koronavirusa-chem-planirovala.html} и немецкой компанией BioNTech
вакцина-кандидат против коронавируса может быть причиной смерти добровольцев,
принимавших участие в клинических испытаниях препарата.

Об этом сообщает\Furl{https://www.fda.gov/media/144245/download} управление по санитарному надзору за качеством пищевых
продуктов и медикаментов (FDA) США в своём отчёте.

Так, в нем упоминается 6 смертельных случаев из среди порядка 38 тысяч
добровольцев, согласившихся опробовать на себе новую разработку. Из 6 умерших -
половина старше 55 лет и только 4 из скончавшихся получили вакцину, остальным
вводили плацебо.

"У двоих причиной смерти стали инфаркт и инсульт, ещё у двоих причины смерти
устанавливаются - это добровольцы, получившие плацебо", - заявили в документе. 

Сообщается, что треть смертельных случаев пришлась на тех, кого действительно
вакцинировали: один, страдавший атеросклерозом и ожирением, умер спустя 3 дня
после введения первой части, а второй скончался от остановки сердца после
получения второй части вакцины, через 60 дней была зафиксирована остановка
сердца, а ещё через 3 дня - смерть.

В FDA считают, что, несмотря на результаты отчета, нет никаких серьёзных угроз
для здоровья вакцинируемых и препарат Pfizer может использоваться для
применения в экстренном порядке.

Напомним, что у двух сотрудников Национальной службы здравоохранения
Великобритании (NHS) были выявлены аллергические реакции после вакцинации
препаратом от коронавирусной болезни производства Pfizer/BioNTech.\Furl{https://strana.ua/news/305671-vaktsina-ot-koronavirusa-proizvodstva-pfizer-mozhet-vyzyvat-allerhicheskuju-reaktsiju.html} Британские
медики предупредили, что людям, которые ранее страдали от аллергии, не стоит
прививаться от Covid-19.

Также напомним, что в США заявили о дефиците вакцины компании Pfizer.\Furl{https://strana.ua/news/305396-v-ssha-defitsit-vaktsiny-ot-koronavirusa-kompanii-pfizer.html} Так,
американские власти заказали 100 миллионов доз лекарства от Covid-19 и
планировали добрать еще 500 млн доз, однако теперь эти поставки под вопросом.
