%%beginhead 
 
%%file 21_01_2023.fb.kusch_pavlo.ua.zhurnalist.writer.1.deyak__rukopisi_spra
%%parent 21_01_2023
 
%%url https://www.facebook.com/pavlo.kushch.58/posts/pfbid0edm9v9vBMTmzB41ayPHzMFPtcipacn8LA3AfmxETLzSA3ZpcnFWpiPMhHVvnt1aNl
 
%%author_id kusch_pavlo.ua.zhurnalist.writer
%%date 21_01_2023
 
%%tags mariupol,kniga
%%title Деякі рукописи справді не горять. Навіть у знищеному Маріуполі…
 
%%endhead 

\subsection{Деякі рукописи справді не горять. Навіть у знищеному Маріуполі...}
\label{sec:21_01_2023.fb.kusch_pavlo.ua.zhurnalist.writer.1.deyak__rukopisi_spra}

\Purl{https://www.facebook.com/pavlo.kushch.58/posts/pfbid0edm9v9vBMTmzB41ayPHzMFPtcipacn8LA3AfmxETLzSA3ZpcnFWpiPMhHVvnt1aNl}
\ifcmt
 author_begin
   author_id kusch_pavlo.ua.zhurnalist.writer
 author_end
\fi

Деякі рукописи справді не горять. Навіть у знищеному Маріуполі...

Вона, за словами людей, які у ті жахливі маріупольські дні і ночі були поряд,
казала: «Пережили ту війну, переживем і цю». На жаль, сама журналістка і
письменниця Наталія Харакоз вже другу окупацію свого рідного Маріуполя не
витримала...

Не пережили цинічного і жорстокого «визволення» також більшість примірників її
книг. Квартири письменниці та її родичів  вигоріли дотла. 

Останньою прижиттєвою книжкою нашої колеги стала колективна збірка «Біля лінії
зіткнення», що вийшла у світ у лютому 2022 року і яку вона так і не встигла
побачити. А нещодавно завдяки зусиллям рідних, близьких та просто добрих людей
побачила світ книга Наталії Харакоз «Новели Азовського узбережжя». 

До видання увійшли також і ще ніде не опубліковані твори письменниці. Оскільки
окремі рукописи пані Наталії дійсно не згоріли... 

Також у книжці є спогади рідних, друзів і колег про Наталію Харакоз, яка чимало
кілометрів рядків написала у газеті «Приазовский рабочий» і водночас
опікувалася молодими авторами, очолюючи літературне об'єднання «Азов'є». 

Світла пам'ять нашій колезі. Величезна вдячність Аня Кот за підготовку цієї
книг
