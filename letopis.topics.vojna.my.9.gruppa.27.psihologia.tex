% vim: keymap=russian-jcukenwin
%%beginhead 
 
%%file topics.vojna.my.9.gruppa.27.psihologia
%%parent topics.vojna.my.9.gruppa
 
%%url 
 
%%author_id 
%%date 
 
%%tags 
%%title 
 
%%endhead 

\paragraph{12:49:07 20-10-22 Николай Кондауров}

Диагноз расейских путинистов.Парано‌идная шизофрени‌я — тип
шизофрении, характеризующийся доминированием галлюцинаций и (или) бреда, при
этом разорванность бреда, аффективное уплощение и кататонические симптомы могут
присутствовать в лёгкой форме, но не являются основными в клинической картине
бредового расстройства. Параноидный тип шизофрении — наиболее часто
встречающийся. Особенность этого типа — обязательное наличие бреда
парафренного, параноидного или паранойяльного типов. Характерно преобладание
галлюцинаторно-параноидных клинических картин, менее выраженные дефицитарные
симптомы и более позднее по сравнению с другими формами шизофрении начало
(обычно около 25—35 лет, но может быть и позже. Поведение характеризуется
враждебностью и агрессивностью, подозрительностью, напряжённостью,
нетерпимостью, раздражительностью.


% -------------------------

%Всё это правильно,но,к большому сожалению,они не читают большие тексты и
%совершенно не хотят думать. К ним очень трудно достучаться... Общалась с
%некоторыми в личке. Всё равно не доходит. Была переписка с одной,которая
%пережила события в Грозном,сама была под бомбежкой,и все равно оправдывает
%путлера.

да читают они тексты, они все читают. Я уже довольно много времени все это
анализирую, все ОНИ ЧИТАЮТ. Но естественно, посколько содержимое текстов
противоположно их мировоззрению, то они отпихиваются от них, защищаются. А сами
по себе тексты, они конечно читают. А как же не прочесть текст, если он
занимает сразу же все место?

а насчет думать. Это же ЗОМБИ. Они и не хотят думать в ТОМ НАПРАВЛЕНИИ КОТОРОЕ
НУЖНО НАМ, ЭТО ЖЕ ЕСТЕСТВЕННО. Понимаете, не ХОТЯТ ОНИ ДУМАТЬ НЕ ПОТОМУ ЧТО ОНИ
ТУПЫЕ, А ПОТОМУ ИМ ЭТО ЯВНО НЕВЫГОДНО. Их мировоззрение противоположно нашему,
так зачем им думать в нашем направлении? Вот когда дело касается того, как бы
уничтожить Украину, они думать умеют будь здоров. А когда их пытаешься убедить
в наших смыслах - естественно, они сопротивляются, потому что их убеждения уже
сформированы их виртуальным лживым миром, их многолетней пропагандой, и
поэтому, естественно, они и не хотят менять своих убеждений. Здесь все очень
завязано на законах психологии. Поэтому здесь дело не в том, что они якобы "не
читают", или же что они "тупые", а в том, что они суть жители ЗомбиЛенда,
собственного виртуального мира, который их ежедневно накачивает и накачивает, в
котором они живут. И естественно, что концепции их мира, их глубинные убеждения
- что Украина во всем виновата, что они воюют с НАТО и так далее, - уже
довольно глубоко сидят в них. Поэтому, когда мы вступаем с ними в контакт, нам
кажется, повторюсь, - только кажется, что они якобы тупые. На самом деле мы
имеем дело с зомби, которые уже сформированы, и поэтому их так трудно пробить,
понимаете. Поэтому здесь особенно важно иметь четкую стратегию поведения с
ними, потому что это не обычные люди, это реально ЗОМБИ. Как бы так сказать.
Вообразите, что в голову человека методично вбиваются гвозди - один гвоздь -
это "Запад угрожает россии", второй гвоздь - "Украина во всем виновата", третий
гвоздь - "россия богоизбранная страна, россия ни в чем не виновата", и так
далее. Это называется ГЛУБИННЫЕ УБЕЖДЕНИЯ в психологии. Загуглите, что это
означает. А как только глубинные убеждения сформированы - образованием, школой,
пропагандой, самой жизнью - то человек начинает все, что он видит, - нанизывать
на эти самые глубинные убеждения, интерпретировать реальность согласно этим
самим глубинным убеждениям. А когда реальность начинает реально противоречить
этим убеждениям - человек сопротивляется, начинает придумывать, уходить от
ответа, и так далее. А клинический случай - это те самые российские зомби, в
которых глубинные убеждения вбили пропагандой, и довольно сильно, так что эти
самые гвозди, которые в них уже вбили, мешают, не дают им думать в том
направлении, которе нужно нам. И естественно, чтобы начать вытаскивать эти
гвозди, нужно приложить реально большую психическую и информационную мощность.

Поэтому здесь кстати и метод повторяющихся смысловых текстов. А почему
смысловые и почему повторяющиеся. Ну, смысл простой. Это не просто атака на
конкретного зомби, это атака на виртуальный мир зомби, атака на его глубинные
убеждения, сформированные в его виртуальном мире. И повторюсь, все мои тексты
они читают, вся информация поступает к ним в голову, а дальше что? А дальше
идет обработка информации, и естественно, вся поступившая информация
(содержание текста ) то есть - что на самом деле россия напала на Украину, что
на самом деле россия разбомбила Мариуполь, и так далее - является абсолютно
негативной по отношению к базовым установкам зомби (что это Украина во всем
виновата) => далее, следует попытка отторжения поступившей информации - и зомби
выдается ругательство какое нибудь - или комментарий с возражением - на выходе.
А реакции бывают самые разные. Так что тексты они читают, весь смысл сказанного
понимают, только естественно, принять никак не могут, поэтому идет отрицание
как реакция. А зачем повторять снова и снова. Ну потому что здесь в Сети, чтобы
что то донести, реально нужно продавливать, ПРОДАВЛИВАТЬ, СИЛОЙ, то, что ты
хочешь донести. Написал текст - потом снова - читай, - не хочешь? вот тебе
снова. И снова. И снова.

И также, очень важный момент ПРОСТРАНСТВА. Сами по себе тексты - они выбивают у
зомби пространство, они теряют чувство контроля над информационным
пространством, а это им очень неприятно
