% vim: keymap=russian-jcukenwin
%%beginhead 
 
%%file 27_12_2021.fb.fb_group.story_kiev_ua.3.onuk_komandarma.cmt
%%parent 27_12_2021.fb.fb_group.story_kiev_ua.3.onuk_komandarma
 
%%url 
 
%%author_id 
%%date 
 
%%tags 
%%title 
 
%%endhead 
\zzSecCmt

\begin{itemize} % {
\iusr{Evgen Zachepylenko}
Остер май 1969 год.

\begin{itemize} % {
\iusr{Надежда Владимир Федько}
\textbf{Evgen Zachepylenko} Я в цей час дослужував на Кубі (1967-1969).

\iusr{Evgen Zachepylenko}
\textbf{Надежда Владимир Федько} в мое время тоже туда отправляли дослуживать...

\iusr{Надежда Владимир Федько}
\textbf{Evgen Zachepylenko} Я знаю. У мене один з приятелів служив у 1971-1973-му. Але вони вже служили у напіввійськовій формі. А ми ще були в цивільному.
\end{itemize} % }

\iusr{Надежда Владимир Федько}

Ця історія трапилася на початку грудня 1966-го, тому і згадалася напередодні 2022-го.

\iusr{Сергій Савелій}

Я служив пізніше ( 73- 75) і не в Острі, але так згадалися і курилка ( я також
не курив, але сидів за компанію), і наряди на кухню - і чергові, і
позачергові... Особливо, коли зміну приймали вредні кабельники, які не любили
нас - радіорелейників. Проводили пальцем по верхній мисці - \enquote{жирна} - всю
пірамідку з 10 мисок руйнували і на всіх столах - \enquote{перемивати...}

\begin{itemize} % {
\iusr{Надежда Владимир Федько}
\textbf{Сергій Савелій} 

Знайоме! На двох - 500 мисок, 500 кружок, 500 ложок, 50 кастрюль для супу, 50
мисок для \enquote{бацил}, 50 блюдець для масла... Посуд з алюмінію. Тільки сніданок
закінчився... починаємо мити... Майже перед обідом закінчуємо. З миючих засобів
- господарське мило і сода!

\iusr{Сергій Савелій}
\textbf{Nadegda Volodymyr Fedko} Інколи можна було випросити у кухарів порошок гірчиці...

\iusr{Надежда Владимир Федько}
\textbf{Сергій Савелій} 

А котли мити... Одягаєш старе (підмінне) х/б... Скидаєш чоботи... залазиш у
котел і драїш... Потім вилазиш... з шлангу вимиваєш.

Приходить повар... проводить пальцем по стінці котла...

Жирно! Перемити!

\iusr{Игорь Морочковский}
\textbf{Сергій Савелій} 

75-77, Гатчина, нас, кабельщиків, при передачі чергувань \enquote{сношали} стаціонщикі,
а радіорелейниками у нас була дружба. Доречі, дідовщини у нас також майже не
було.

\iusr{Вадим Скоп}
73-75, Р-408. Учебка под Полтавой, а затем Монголия.

\begin{itemize} % {
\iusr{Сергій Савелій}
\textbf{Vadim Skop} А нам в полк урядового зв'язку прийшли Р- 410 М \enquote{Альбатрос}, тропосферки...

\iusr{Вадим Скоп}
\textbf{Сергій Савелій} 

Р-410 к нам тоже пришли в конце 74-го. Но перед дембелем не было желания
переучиваться. На них работала уже \enquote{молодёжь}. У нас, на \enquote{точке} были ещё
линейщики и спутник. Со всеми были хорошие отношения, жили в одной казарме.

\iusr{Сергій Савелій}
\textbf{Vadim Skop} 

Я написав пісню \enquote{Армейская}, з таким приспівом \enquote{Мы тропосферками стреляем в
небосвод, На разворотах замесив гектары грязи, Сигналу трассу без проблем
пробьет Гвардейский полк правительственной связи!} Є що згадати... З прийдешніми
святами, годок!

\iusr{Вадим Скоп}
\textbf{Сергій Савелій} Дякую. І Вам не хворіти.
\end{itemize} % }

\end{itemize} % }

\iusr{Надежда Владимир Федько}

Десна (Остер) літо 2017-го... Я в жовтій \enquote{британці}.

\ifcmt
  ig https://scontent-frx5-1.xx.fbcdn.net/v/t39.30808-6/270202444_4855536254505794_1834943867048460512_n.jpg?_nc_cat=100&ccb=1-5&_nc_sid=dbeb18&_nc_ohc=UisDsbrXGp0AX-mDQzK&_nc_ht=scontent-frx5-1.xx&oh=00_AT8FcemCiqCjuD1EgGCEf1ocR50FwXkF9wvcxrdc8Z-FRw&oe=61E4C450
  @width 0.3
\fi

\begin{itemize} % {
\iusr{Alex Vines}
\textbf{Надежда Владимир Федько} 

Привет, я 2 года прослужил в Остре, 300й полк. И на губу меня на 7м суток
посадил зам комдива. Уже дембеля, 2го Мая.

\end{itemize} % }

\iusr{Надежда Владимир Федько}

У 2018-му я тиждень читав лекції молоді... Намагався знайти свою казарму... НЕ
знайшов. Все дуЖЖе змінилося за піввіку.

\begin{itemize} % {
\iusr{Evgen Zachepylenko}
наша была по центру одноэтажная саперы инженерные войска ВУС 0501.

\iusr{Evgen Zachepylenko}
когда была СА меня как офицера запаса гоняли на переподготовку в КИСИ и летом в лагерь на 40 дней ...
\end{itemize} % }

\iusr{Alexey Novozhylov}

Навіть і не знаю що сказати, я служив в тій армії у 83-85 роках, але до сих
пір, коли мені інколи сниться армія, я прокидаюсь в шоковому стані. @igg{fbicon.beaming.face.smiling.eyes}{repeat=2} 

\begin{itemize} % {
\iusr{Надежда Владимир Федько}
\textbf{Alexey Novozhylov} 

Ми щодня після підйому і туалету бігали 1 кілометр. Спочатку це було важко..
Потім звикли. Можу сказати, що фізична підготовка плюс добре харчування,
отримані в Острі, дали мені на все життя гарне здоров'я.

\begin{itemize} % {
\iusr{Alexey Novozhylov}
\textbf{Nadegda Volodymyr Fedko} я теж можу так сказати, але як зараз люди кажуть, ніколи знову!  @igg{fbicon.face.tears.of.joy} 

\iusr{Надежда Владимир Федько}
\textbf{Alexey Novozhylov} 

Я до пізньої осені 2019-го постійно їздив в Десну. Там у нас підшефний
підрозділ. Брав участь у нічних стрільбах із ЗУ. Фантастичне відчуття!

Але у зв'язку з важкою хворобою дружини із зими 2019- 2020 року змушений був
сидіти вдома.


\iusr{Džeradž Libor}
\textbf{Надежда Владимир Федько} 

в учебке мы бегали 3 до завтрака, 3 до обеда и 3 перед ужином. Не считая
подтягиваний и отжиманий на «делай раз », ходьбы гусиным шагом по пол-км,
рукопашки и марш-бросков пару раз в неделю. Питались в учебке пюре из сушёной
картошки, гороховым супом и кашей из того же гороха)

Есть хотелось всегда)

Но, это дало такую закалку, что боевуха с постоянными рубками с воинами ислама
но при этом нормальной кормёжкой показалась мне цветочками на озарённом солнцем
поле.

Хорошее было время )


\iusr{Надежда Владимир Федько}
\textbf{Džeradž Libor} Ми бігали щодня зранку 1 км.; двічі на тиждень, через годину після обіду, 3 км.; і раз на тиждень марш-кидок у повній викладці на 5 км.

\iusr{Džeradž Libor}
\textbf{Надежда Владимир Федько} первый километр-полтора самый трудный. Потом открывается второе дыхание и можно бежать хоть 15. У вас был просто курорт какой-то)
Хотя слухи об уставщине в Десне доходили даже до Прибалтики. А это мрак)

\iusr{Надежда Владимир Федько}
\textbf{Džeradž Libor} Та ніякого мраку не було)) Це слабаки поширювали чутки...
\end{itemize} % }

\iusr{Džeradž Libor}
\textbf{Надежда Владимир Федько} возможно)

\end{itemize} % }

\iusr{Pavlo Bondarenko}
Наряд вообще то суточный, да и котлы у нас мыли сами повара, видно наряду не
было доверия.. правда это конец 80-х

\begin{itemize} % {
\iusr{Надежда Владимир Федько}
\textbf{Pavlo Bondarenko} Плановий наряд - на добу. А позачерговий - від відбою до підйому.

\iusr{Василий Артеменко}
В 1976 році позачергові наряди (наряды вне очереди) були скасовані, принаймі, у ГРВН (ГСВГ).
\end{itemize} % }

\iusr{Виталий Богатько}

\ifcmt
  ig https://scontent-frt3-1.xx.fbcdn.net/v/t39.30808-6/269920340_3209233409200638_9168533182015431809_n.jpg?_nc_cat=108&ccb=1-5&_nc_sid=dbeb18&_nc_ohc=B-H2KKhJLTUAX9-OduF&_nc_ht=scontent-frt3-1.xx&oh=00_AT_B6rAHM5w2TyLjWbL2Hu97ILdKhGdCGxZ8Vn-RA-PqpA&oe=61E5532E
  @width 0.3
\fi

\begin{itemize} % {
\iusr{Evgen Zachepylenko}
альбом покажи ...

\ifcmt
  ig https://scontent-frt3-2.xx.fbcdn.net/v/t39.30808-6/270084354_1057955898111843_6695318600633302195_n.jpg?_nc_cat=103&ccb=1-5&_nc_sid=dbeb18&_nc_ohc=xHiWrlh6nZwAX-y6uXG&_nc_ht=scontent-frt3-2.xx&oh=00_AT-oqyLN-SCU124dmo80uZQDIt-areDxSAqXm8lwb5C1vA&oe=61E59C74
  @width 0.3
\fi

\iusr{Надежда Владимир Федько}

Я з закінчував службу на Кубі. Альбоми були категорично заборонені! Перед
посадкою в автобуси, щоб їхати в порт (на теплохід), на стадіоні влаштовувався
огляд. Стоїмо... перед кожним лежить розкрита валіза. Особіст переглядає всі
речі... дає дозвіл сісти в автобус.

\end{itemize} % }

\iusr{Simon Bricklin}

Тим часом я проходив службу з 1978 по 1980 рік. У 1981 році отримав повістку,
щоб прийти в призов і бути готовим до служби в армії. Коли я прийшов, я сказав
їм, що я вже служив і зареєстрований у тому самому офісі 7 місяців тому. Вони
не повірили мені, поки я не показав їм свої військові документи. Вони не могли
повірити, бо забули оновити мої особисті документи в 1980 році.


\iusr{Юрий Онищенко}
Забагато нафантазовано

\begin{itemize} % {
\iusr{Надежда Владимир Федько}
\textbf{Юрий Онищенко} Я не літератор заради отримання лайків у Фейсбуці. А пишу про те, що пережив.

\begin{itemize} % {
\iusr{Юрий Онищенко}
\textbf{Надежда Владимир Федько} 

Радянська армія була філією буцугарні. Бо всі хто там служив кожен день були
під шаленим тиском кримінальної відповідальності, як що порушать якісь установи
або статут. Я знаходячись там у 70-х роках минулого сторіччя. Не зустрів
жодного патріота СРСР, жодного симпатика КПРС. Усі строковці службовці мали
свій \enquote{срок} ув'язнення, усі офіцери добре утримання та бестурботне життя
ледарів. Так що за армію великих держави трохи стриманіше будь ласка, якщо не
боти.


\iusr{Надежда Владимир Федько}
\textbf{Юрий Онищенко} 

В армії США, у Вермахті, в японській армії і т.д. за порушення статутів і норм
законів була і є дисциплінарна та кримінальна відповідальність.

\iusr{Юрий Онищенко}
\textbf{Надежда Владимир Федько} я згоден, але там армія це захосники а не загарбники власного народу. Новочеркаськ пам'ятаєте?

\iusr{Надежда Владимир Федько}
\textbf{Юрий Онищенко} 

Я все чудово знаю! Крім Новочеркаська була Чехословаччина 1948-го, НДР 1953-го;
Угорщина 1956-го; Чехословаччина 1968-го; Афганістан 1979-го...


\iusr{Юрий Онищенко}
\textbf{Надежда Владимир Федько} Так чим пишалися?

\iusr{Надежда Владимир Федько}
\textbf{Юрий Онищенко} Тим, що я був чоловіком, а не слюнтяєм і слабаком!

\iusr{Надежда Владимир Федько}
\textbf{Юрий Онищенко} 

Таке враження, що Ви не були ні жовтенятком, ні піонером, ні комсомольцем... І
ваші батьки воювали в СС Дивізії \enquote{Галичина} (яку я поважаю).

\iusr{Юрий Онищенко}
\textbf{Надежда Владимир Федько} 

По перше бути людиною це головне, чи чоловіком чі жінкою. По друге сприймати
дійсність, якою вона є, це дуже складно за для тих хто створив собі неіснуючі
ілюзії. По трете для того щоб висловлювати якісь припущення, треба спочатку,
скоштувати той хліб про який складаєте враження. А так це побрехеньки, мені це
розповідали. Дякую

\iusr{Петро Злидень}
\textbf{Юрий Онищенко} 

... ги ги, сміх сміхом, а в зиму з90 на 91й внас в батальйоні один штрих
скажем так написав заяву в капеерес, в комунари ннах. А літом гекачпе, і
корито лопнуло .... і таке буває. @igg{fbicon.face.smirking} 

\iusr{Надежда Владимир Федько}
\textbf{Петро Злидень} Я під час служби на Кубі прийняв принципове рішення не вступати в комуністичну партію. І не вступив.
\end{itemize} % }

\iusr{Naumova Svitlana}

Написано правдиво. У мене було багато знайомих хлопців, які служили в ті роки і
армію згадували з вдячністю. Вона на все життя залишила відчуття отриманої
мужності. Мій брат двоюрідний служив на кордоні з Китаєм в період заворухи. Мої
однокласники служили пізніше на 10 років, теж, ще ніякої дідовщини не було. Жах
розпочався під час афганської війни, коли пішли перші гроби з афгану і матері
пробували за будь-яку ціну відкупити дітей. Це рік народження 1960 і далі.

\end{itemize} % }

\iusr{Олена Шелест}
Дякую. Цікаві спогади.

\iusr{Саша Саша}
Что после армии было? поженились с той девушкой?

\begin{itemize} % {
\iusr{Надежда Владимир Федько}
\textbf{Саша Саша} Ні. Армія багато в чому змінила мене. Я став дорослим, зазнав корекції світогляд, ставлення до життя, життєві пріоритети...
\end{itemize} % }

\iusr{Надежда Владимир Федько}

2019-й... Десна...

\ifcmt
  ig https://scontent-frx5-1.xx.fbcdn.net/v/t39.30808-6/269916319_4855645391161547_1440036589741282503_n.jpg?_nc_cat=100&ccb=1-5&_nc_sid=dbeb18&_nc_ohc=KlL3YUN-cGsAX9yChVs&_nc_ht=scontent-frx5-1.xx&oh=00_AT_JgSoJG9tTj2bzQLvY2iPgUqKEM2TCxZoj1GOd3EQPEA&oe=61E4D55A
  @width 0.3
\fi

\iusr{Надежда Владимир Федько}

2019-й, Десна...  В нинішній час треба знайти своє місце в обороні рідної
землі. Чи на фронті, чи в тилу. Це не патетика, а життєва філософія.

\ifcmt
  ig https://scontent-frt3-1.xx.fbcdn.net/v/t39.30808-6/269814546_4855658567826896_2988544868896058227_n.jpg?_nc_cat=107&ccb=1-5&_nc_sid=dbeb18&_nc_ohc=b4kibjfawUYAX-DP9N1&_nc_ht=scontent-frt3-1.xx&oh=00_AT-LSG6Wd5xuxtXmUD5IDRXE4nnIucr_ONnBp-9vtgqDCQ&oe=61E43B85
  @width 0.2
\fi

\iusr{Алексей Шевнюк}
Форма раз : трусы и противогаз @igg{fbicon.face.smiling.eyes.smiling} 

\iusr{Džeradž Libor}
Служил я позже и слава богу не в «Десне» ))
В армии было много всякого. Воспоминания остались очень тёплые.

\iusr{Павел Кондя}
БАМ 1980-1982р.
Маревая, Діпкун)) ЖДВ.

\iusr{Надежда Владимир Федько}

Зараз почую недовірливі голоси))

В \enquote{учебці} не було \enquote{дєдовщини} та позастатутних відносин, які в 70-х роз'їдали
армію, як ракова пухлина.

І пояснюється це просто...

Учбова дивізія \enquote{Остер} була школою підготовки сержантського складу СА для
всього Радянського Союзу. Всього \enquote{учебок} було дві. Друга - в Гончаров Круг.

Наша \enquote{учебка} мала постійний склад - офіцери-викладачі різних дисциплін та
сержанти, які пройшли вишкіл та були залишені для роботи з курсантами. І
курсантів, відібраних з призваних на службу. Вишкіл сержанта в ті часи займав,
грубо кажучи, десять з половиною місяців. Тобто, від осіннього призову до
наступного осіннього призову. Курсанти в ротах були з одного призову. Так
званих \enquote{дідів} просто не існувало. Всі курсанти були рівними.

А за якісь нестатутні відносини - образи, травлю чи бійку, можна було миттєво
потрапити під трибунал і відправитися у дисциплінарний батальйон на строк від 3
місяців до 2-х років. Після чого треба було дослужувати строкову службу вже
рядовим у військах.

\iusr{Надежда Владимир Федько}

На Кубі була тотальна секретність! Я служив у \enquote{Бригаді} (Республіка
Куба, Учбовий центр №12-в /Нарокко - Манагуа/. Поштова адреса: Москва-400).

Цю фотографію і ще одну, відзняті 26 березня 1969 року, я акуратно заклеїв під
внутрішню обшивку валізи. І таким чином привіз їх додому.

\ifcmt
  ig https://scontent-frt3-1.xx.fbcdn.net/v/t39.30808-6/270059033_4855771037815649_1582129985377872737_n.jpg?_nc_cat=106&ccb=1-5&_nc_sid=dbeb18&_nc_ohc=s3pQwTY5ZlgAX8lkBVG&_nc_ht=scontent-frt3-1.xx&oh=00_AT-A65aIWcxowNBpv8cS-zmsOt_3OsmkksT9LqmY7OdB4Q&oe=61E535E1
  @width 0.2
\fi

\begin{itemize} % {
\iusr{Марія Голуб}
\textbf{Nadegda Volodymyr Fedko} а тепер яка пам'ять! Чекаю розповідей про Кубу)

\iusr{Надежда Владимир Федько}
\textbf{Марія Голуб} Розкажу...
\end{itemize} % }

\iusr{Dmitriy Gernak}
Как тогда говорили - Лучше в костер, чем в Остер.

\begin{itemize} % {
\iusr{Надежда Владимир Федько}
\textbf{Dmitriy Gernak} Остер тоді ще називали \enquote{Дика дивізія})

\iusr{Надежда Владимир Федько}
\textbf{Dmitriy Gernak} 

При зрості 172 см я важив 54 кг... А снарядний ящик на \enquote{смузі перешкод} важив
60 кг. Підбігаєш... хапаєш... піднімаєш і біжиш далі... 25 метрів... кладеш на
землю (кидати не можна, бо за кидок отримаєш наряд поза чергою) і біжиш далі. У
нас \enquote{смуга перешкод} називалася \enquote{могила 20-го століття}.

\end{itemize} % }

\iusr{Дмитрий Сигнеев}
Всё как и у меня @igg{fbicon.face.wink.tongue} 

\iusr{Надежда Владимир Федько}

З веселих пригод можна ще згадати \enquote{похорони недопалка} (\enquote{похороны окурка})...

\iusr{Юрий Панчук}
Я колись проходив практику в 22му інструментальному, вчився в арсенальському технікумі.

\iusr{Юрий Панчук}
С командармом смішно получилося. Неприглядно виглядав не шутнік, а замполіт ))) За це він і впаяв 10 діб ))), не за жарт 100\%

\iusr{Valentyn Shcherbachev}
За три \enquote{з лишком} роки відходив предостатньо і на \enquote{губу} і в наряди...

\begin{itemize} % {
\iusr{Надежда Владимир Федько}
\textbf{Valentyn Shcherbachev} \enquote{Дисидент})

\iusr{Valentyn Shcherbachev}
Вже тоді...
\end{itemize} % }

\iusr{Natali Gudyrenko}
Дякую. Дуже цiкаво. Чекаю нових спогадiв.

\iusr{Вячеслав Мусиенко}
А я помню варили картошку и заправляли подсолнечным маслом в наряде на кухне.

\end{itemize} % }
