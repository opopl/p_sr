% vim: keymap=russian-jcukenwin
%%beginhead 
 
%%file 26_10_2021.stz.news.ua.misto_marii.1.80_rokiv_tragedia_na_agrobazi
%%parent 26_10_2021
 
%%url https://mistomariupol.com.ua/uk/mariupol-shanuye-pamyat-80-oyi-richnyczi-tragediyi-na-agrobazi
 
%%author_id news.ua.misto_marii
%%date 
 
%%tags 
%%title 80-та річниця трагедії на Агробазі
 
%%endhead 
 
\subsection{80-та річниця трагедії на Агробазі}
\label{sec:26_10_2021.stz.news.ua.misto_marii.1.80_rokiv_tragedia_na_agrobazi}
 
\Purl{https://mistomariupol.com.ua/uk/mariupol-shanuye-pamyat-80-oyi-richnyczi-tragediyi-na-agrobazi}
\ifcmt
 author_begin
   author_id news.ua.misto_marii
 author_end
\fi

\ifcmt
  ig https://i2.paste.pics/PIQF7.png?trs=1142e84a8812893e619f828af22a1d084584f26ffb97dd2bb11c85495ee994c5
  @wrap center
  @width 0.9
\fi

\begin{quote}
\em
20, 22 та 23 жовтня 1941 року було розстріляно понад 16 тисяч представників
єврейської діаспори Маріуполя. Дітей, жінок, старше покоління зібрали та
примусили пройти більше як 9 км пішки від Маріуполя до селища Агробаза, де
потім під звуки пальби жорстоко обірвали життя багатьох родин! Сьогодні
Маріуполь вшановує пам'ять жертв цієї страшної трагедії та згадує Праведників
народів світу, які ризикували життям та допомагали пораненим. Від сьогодні та
на два тижні в старовинній синагозі на Георгіївській встановлені інсталяції та
пам'ятні банери. Пам'ятаємо свою історію! 
\end{quote}

\subsubsection{Страшні сторінки історії}

\ii{26_10_2021.stz.news.ua.misto_marii.1.80_rokiv_tragedia_na_agrobazi.pic.1}

Наприкінці жовтня кожного року єврейська спільнота Маріуполя пригадує страшну
трагедію їхнього народу та шанує пам'ять жертв розстрілу під селищем Агробаза.
У 1941 році, під час Другої світової війни, німецько-фашистські загарбники
розстріляли понад 16 тисяч маріупольців єврейського походження. Їх збирали під
стінами старовинного корпусу ПДТУ, а потім змусили йти пішки до селища
Агробаза, де у протитанковому рву обірвалося життя тисячі євреїв Маріуполя.

\ii{26_10_2021.stz.news.ua.misto_marii.1.80_rokiv_tragedia_na_agrobazi.pic.2}

На одному з банерів ми побачили фрагменти щоденника Сари Глейх – дівчина якимсь
дивом змогла врятуватися та вижити. Вона описувала все, що сталося з євреями
Маріуполя та весь жах цієї трагедії. Посилання на щоденник Сари – тут На самому
банері можна побачити QR-код на сайт, де розташований щоденник єврейської
дівчини.

\ii{26_10_2021.stz.news.ua.misto_marii.1.80_rokiv_tragedia_na_agrobazi.pic.3}

Маріупольці – Праведники народів світу:

1. Алабушеві Марія Абрамловна. Померла. Врятувала Ірину Карпович (Цікурская).\par
2. Петренко Ганна Савеліївна.\par
3. Олейниченко Поліна Лаврентіївна. Померла. Врятувала Борисковского Веніаміна.\par
4. Олейниченко Аким. Помер. Врятував Борисковского Веніаміна.\par
5. Олейниченко (Пономарьова) Валентина Якимівна Врятувала Борисковского Веніаміна.\par
6. Олійниченко Віктор Якимович. Помер. Врятувала Борисковского Веніаміна.\par
7. Сіроштан Парасковія. Врятувала Мельцен Захарія.\par
8. Сіроштан Андрій. Врятував Мельцен Захарія.\par
9. Сіроштан Іван. Врятував Мельцен Захарія.\par
10. Етьемез Дмитро. Помер. Врятував Беренгольц Зіну.\par
11. Етьемез Варвара. Врятувала Беренгольц Зіну.\par
12. Тимофєєва (Людвінская) Валентина Лаврентіївна. Врятувала Шварцберг Галю і Риту.\par
13. Ломакіна (Олійник) Віра Іллівна. Померла. Врятувала Шварцберг Галю і Риту.\par

\ii{26_10_2021.stz.news.ua.misto_marii.1.80_rokiv_tragedia_na_agrobazi.pic.4}

\subsubsection{Про підготовку у синагозі}

Команда ККП \enquote{m.EHUB} під час підготовки до пам'ятних інсталяції та
інформаційних банерів інформаційно співпрацьовувала з єврейським товариством
Маріуполя. Тому кожна інформація, яку ви побачите на стінах синагоги, правдива
та документально підтверджена.

\emph{\enquote{Ця інсталяція присвячена трагедії, яка відбулася 80 років тому. 24 жовтня
відбувся пам'ятний мітинг у селищі Агробаза, а сьогодні ми продовжуємо
вшановувати пам'ять всіх загиблих. Ця інсталяція нагадує нам і імена 5809
загиблих, саме їхні імена встановили фахівці ізраїльського інституту Яд Вашем.
Пошуки інформації про інших загиблих ще тривають. На іншому банері ми бачимо
сторінки зі щоденника дівчини, яка пережила цю трагедію, також ми дізнаємося
інформацію про Праведників народів світу, які допомагали людям, ризикуючи
власним життям. Ми повинні вшановувати цю історію, аби така трагедія більше
ніколи не повторилася}}, – розповіла директорка ККП \enquote{m.EHUB} Марія Сльота.

Протягом 14 днів ця інформація та інсталяції будуть знаходитися у стінах
старовинної синагоги. А сьогодні увечері можна побачити світлові стовбури – як
символ страченого життя кожного загиблого єврея.

\ii{26_10_2021.stz.news.ua.misto_marii.1.80_rokiv_tragedia_na_agrobazi.pic.5}

\emph{\enquote{Наше основне завдання як міста зберегти пам'ять про цю страшну трагедію, коли
розстрілювали маленьких дітей, жінок, людей похилого віку. Ми вважаємо, що ця
трагедія рівносильна трагедії Бабиного Яру під Києвом, адже тільки офіційно – в
Агробазі було розстріляно 16 тисяч євреїв, але ми вважаємо, що ця цифра значно
більша. Важливо, аби цей геноцид ми не забували та ніколи не дозволили цим
подіям повторитися ще. Коли ми пам'ятаємо – ми цінуємо людське життя. Адже саме
цінне для нас – це наша історія}}, – підкреслив мер Маріуполя Вадим Бойченко.

Старовинна синагога була обрана як основна локація для інсталяції через її
світлу енергетику та символічність цього місця для єврейської спільноти міста.

\ii{26_10_2021.stz.news.ua.misto_marii.1.80_rokiv_tragedia_na_agrobazi.pic.6}
\ii{26_10_2021.stz.news.ua.misto_marii.1.80_rokiv_tragedia_na_agrobazi.pic.7}

\subsubsection{Про будівлю}

Поява синагоги датується 1882 роком. Друга за хронологією, вона стала
хоральною, що свідчить про її значущість. Одним з ініціаторів відкриття
синагоги був Йосип Авербах, батько видатного офтальмолога Михайла Йосиповича
Авербаха, про що йдеться в книзі \enquote{Маріуполь та його околиці}.

У користування радянської влади синагога перейшла в 1930-х роках і, на подив
місцевих жителів, не перетворилась на руїни, як більшість сакральних об'єктів в
того часу. В період німецької окупації в будівлі знаходився госпіталь і пункт
збору остарбайтерів для відправки в Німеччину на примусові роботи. Довгий час в
будівлі розміщувалося медичне училище, після нього – заочна школа. Служила
будівля також гімнастичним залом, конторою Гіпромеза і морської школою.

Мало хто знає, але тут могла розташовуватися картинна галерея. Ідею не вдалося
реалізувати, оскільки будівля не пережила зиму, після сильного снігопаду дах
обвалився під чималою вагою.

\subsubsection{Плани на майбутнє}

У 2021 році в рамках пакету ініціатив \enquote{Діалог мовою мистецтва} ці стіни
прийняли 2 фестивалі, концерти класичної музики \enquote{Mariupol Classic}, спектаклі,
екскурсії та перформанси. Тільки \enquote{Ніч музеїв} привернула увагу до синагоги
понад 10 тисяч маріупольців.

Наразі планується вдихнути в будівлю нове життя. Як розповів головний равин
єврейської спільноти Маріуполя Менахем Мендел Коєн, маріупольська синагога, яку
побудували наприкінці ХІХ століття, була дуже гарною будівлею. Тому єврейське
товариство та представники міської влади дуже хотіли б зберегти та відтворити
цю величність. Саме тому на цьому місті буде відтворена синагога, яка стане
чинною молитвеною будівлею для всіх євреїв міста.

\ii{26_10_2021.stz.news.ua.misto_marii.1.80_rokiv_tragedia_na_agrobazi.pic.8}
\ii{26_10_2021.stz.news.ua.misto_marii.1.80_rokiv_tragedia_na_agrobazi.pic.9}
\ii{26_10_2021.stz.news.ua.misto_marii.1.80_rokiv_tragedia_na_agrobazi.pic.10}
\ii{26_10_2021.stz.news.ua.misto_marii.1.80_rokiv_tragedia_na_agrobazi.pic.11}
\ii{26_10_2021.stz.news.ua.misto_marii.1.80_rokiv_tragedia_na_agrobazi.pic.12}
