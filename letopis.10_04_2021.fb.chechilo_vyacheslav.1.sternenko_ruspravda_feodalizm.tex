% vim: keymap=russian-jcukenwin
%%beginhead 
 
%%file 10_04_2021.fb.chechilo_vyacheslav.1.sternenko_ruspravda_feodalizm
%%parent 10_04_2021
 
%%url https://www.facebook.com/vyacheslav.chechilo/posts/3989409094454655
 
%%author 
%%author_id 
%%author_url 
 
%%tags 
%%title 
 
%%endhead 

\subsection{Феодализм - Стерненко - Судебная система - Правда Ярослава}
\Purl{https://www.facebook.com/vyacheslav.chechilo/posts/3989409094454655}

Когда в случае Стерненко говорят о политически мотивированном правосудии –
делают большую ошибку. Причем как его сторонники, так и его противники. Потому
что политически мотивированное правосудие было, когда Янукович сажал Тимошенко.

Но с тех мы далеко продвинулись вперед, вглубь веков, и у нас теперь другие
общественные отношения, в рамках которых слова «политика» и «правосудие» имеют
совершенно другие значения. Отношения, которые лучше всего описываются термином
«феодальные».

«Патриоты» являются одним из привилегированных сословий. И, исходя из своего
статуса, они требует вполне понятные и правильные для них вещи. 1) чтобы
патриотов не судил суд, которым судят остальных и тем более, чтобы патриотов не
судил судья-«ватник». Ведь это абсурд, учитывая, что ватники находятся ниже
всех по социальному статусу.  2) чтобы государство официально признавало
статус-кво - неравноценность жизни патриота и представителя более низких
сословий.

Как заявил вчера сам Стерненко на суде, Сергей Щербич, которого он похищал,
состоял в партии «Родина», а потому является сепаратистом. Т.е., де-факто,
парией в новой системе отношений, неприкасаемым. Судить патриота за похищение
ватника так же неуместно, как и судить за «жаренных колорадов». То, что закон
формально пока еще не делает здесь различий – исключительно проблема закона,
оставшегося от оккупационной администрации.

Да, закон нельзя исправить по вполне понятным причинам – 21 век, ЕС и
Венецианская комиссия. Но его можно и нужно игнорировать, чтобы хотя бы таким
образом приблизить к реально существующим общественным отношениям. Именно этого
и требуют от Зеленского. И именно это он, в итоге, и сделает, если хочет и
дальше управлять этим племенем.

«За княжеского сельского старосту или за полевого старосту платить 12 гривен, а
за княжеского рядовича 5 гривен. А за убитого смерда или холопа 5 гривен. А за
княжеского коня, если тот с пятном, 3 гривны, а за коня смерда 2 гривны».
