% vim: keymap=russian-jcukenwin
%%beginhead 
 
%%file 30_01_2021.fb.bilchenko_evgenia.1.feminizm_ukraina_zapad_travlja
%%parent 30_01_2021
 
%%url https://www.facebook.com/yevzhik/posts/3598266576875093
 
%%author Бильченко, Евгения
%%author_id bilchenko_evgenia
%%author_url 
 
%%tags bilchenko_evgenia,feminizm,nacionalist,obschestvo,ukraina
%%title БЖ. Кто здесь западных феминистов хотел? Вам их есть
 
%%endhead 
 
\subsection{БЖ. Кто здесь западных феминистов хотел? Вам их есть}
\label{sec:30_01_2021.fb.bilchenko_evgenia.1.feminizm_ukraina_zapad_travlja}
\Purl{https://www.facebook.com/yevzhik/posts/3598266576875093}
\ifcmt
 author_begin
   author_id bilchenko_evgenia
 author_end
\fi

БЖ. Кто здесь западных феминистов хотел? Вам их есть. \verb|#нормальныйфеминизм|

Кажется, против меня высказывалась украинская феминистка? Здорово, - вот ответ
представителя феминистского дискурса США. Я - не сторонник политического
феминизма, я склонна к традициям, но диалог с его представителями в Берлине у
меня был очень плодотворным. Мне радостно видеть, как гуманисты разных стран и
взглядов объединяются в борьбе против неонацизма. Thank you for friendship!

"Во всех этих украинских историях с \enquote{патриотической} травлей
примечательно то, что её жертвами обычно становятся женщины: Александра Чистова
(поздравила россиян с Новым годом), Таисия Маламан (выразила желание съездить в
Россию), курсантки Харьковской академии МВД (за то, что полайкали видео с
приглянувшимся им пацанчиком в форме российского мента), Ирина Машкевич
(хозяйка одесского ресторана еврейской кухни, пошутившая на тему принудительной
укринизации сферы услуг), врачи Галина Лекунова (за протест против языковой
политики) и О.Ткаченко (ответившая вне рабочего времени на русском языке
журналистам, обратившемся к ней на русском языке), профессор Евгения Бильченко
(за критику языкового законодательства), адвокат Tetiana  Montian (защищающая
пацифиста Коцабу) и т.д.

В правых сообществах выделяются, как минимум, две группы женщин, которые, в глазах фашни, подлежат преследованиям:

1. Феминистки. В основном, из Киева, Одессы и Харькова. Возраст жертв – от 14 до 19.

2. Девушки, встречающиеся с парнями неарийской внешности. География – Николаев,
Одесса, Киев.

Травля ВСЕГДА направлена на тех, кто слабее. В консервативной Украине это
женщина. Ром. Квир. Темнокожий… Что, казалось бы, должно вызывать реакцию
возмущения у нашего гражданского общества. Однако, как верно отмечает Volodymyr
Ishchenko, те, кто сегодня травят украинских женщин и вообще неугодных – это и
есть наше гражданское общество. 

Пока одна его часть, в лице уличных людоедов, запугивает насилием нарушителей
патриотических скреп (например, анти-конституционного языкового закона), другая
его часть, в лице \enquote{национал-демократов} и колониальных грантососов, сидящих в
\enquote{неолиберальных НГО}, вообще никак не освещает в медиа все вышеописанные случаи
травли украинских граждан: из страха, из нежелания \enquote{кормить} кремлёвскую
пропаганду, из своеобразного патриотического отрицания (\enquote{да, в Украине есть
фашня, но ведь есть и стартапы}), а иногда, как в случае с рошеновской
интеллигенцией, попросту поддерживая и оправдывая репрессии… Всё это
нормализует травлю и смягчает картинку реальности в глазах общественности. 

\ifcmt
  pic https://scontent-lga3-2.xx.fbcdn.net/v/t1.6435-9/143833073_3598266536875097_6657230639316703171_n.jpg?_nc_cat=101&ccb=1-3&_nc_sid=8bfeb9&_nc_ohc=felOUig34ngAX8zFESI&_nc_ht=scontent-lga3-2.xx&oh=32db84b3fd774e18ae56dd5722c9e908&oe=60C9D17E
\fi

Вместо освещения произвола швондеров, которым языковой закон, как и закон о
декоммунизации, дал легальную причину для нападок на украинских граждан; вместо
акцента на том, что этот и подобные ему законы анти-конституционны, нарушают
права человека, и разжигают меж-этнические конфликты, способствуя развитию
геноцидальных тенденций в Украине, наше гражданское общество объясняет тем, кто
чувствует дискриминацию, что никакой дискриминации в законе нет. Объясняет
точно так же, как мужики, квакающие мошонкой из пива, объясняют феминисткам,
что никаких проблем у женщин нет – особенно в стране, так и не ратифицировавшей
Стамбульскую конвенцию против насилия в отношении женщин".

Анатолий Ульянов, Львов, нынче - политический беженец в США.
