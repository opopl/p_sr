% vim: keymap=russian-jcukenwin
%%beginhead 
 
%%file slova.serdce
%%parent slova
 
%%url 
 
%%author 
%%author_id 
%%author_url 
 
%%tags 
%%title 
 
%%endhead 
\chapter{Сердце}

%%%cit
%%%cit_head
%%%cit_pic
%%%cit_text
Камнем преткновения среди украинских поклонников игры стало название на русском
языке - \enquote{\emph{Сердце} Чернобыля}. Его разработчики перевели на
английский путем транслитерации как \enquote{\emph{Heart} of Chernobyl}.
Националисты настаивают, что по законам украинского языка правильно было бы
писать Chornobyl. Они недоумевают: почему украинские разработчики позволили
себе выпустить трейлер на русском языке и допустили подобный перевод, который
может создать впечатление того, что игра разработана в России, а не Украине.  В
Twitter даже запустили специальный хэштег \verb|#ChornobylNotChernobyl|.
\enquote{При этом весь трейлер на русском. Теперь на Западе игру считают
российской, а русня бегает по сети и кричит про \enquote{отечественный}
геймдев}, - пишет пользователь Даниил Максименко.  \enquote{Потому что
Чернобыль - это Украина, и название должно быть украинское}, - считает еще один
пользователь соцсети
%%%cit_comment
%%%cit_title
\citTitle{Сталкер 2 Сердце Чернобыля - почему хейтят видеоигру украинских разработчиков}, 
Екатерина Терехова, strana.ua, 16.06.2021
%%%endcit

%%%cit
%%%cit_head
%%%cit_pic
%%%cit_text
Глотнув конскую дозу узбагоительного, шпрехен-фюрер далее в отчете указал, что
в \emph{сердце страны} Киеве, молчу уже про юго-восток, более половины учителей
позволяют себе на уроках говорить на русском языке и, о ужас, также
предпочитают русский вне школы и школьных занятий
%%%cit_comment
%%%cit_title
\citTitle{Speak на мове please}, 
Terra Incognita, zen.yandex.ru, 30.04.2021
%%%endcit
