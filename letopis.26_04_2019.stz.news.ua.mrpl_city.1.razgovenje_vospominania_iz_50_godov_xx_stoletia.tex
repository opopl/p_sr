% vim: keymap=russian-jcukenwin
%%beginhead 
 
%%file 26_04_2019.stz.news.ua.mrpl_city.1.razgovenje_vospominania_iz_50_godov_xx_stoletia
%%parent 26_04_2019
 
%%url https://mrpl.city/blogs/view/razgovene-vospominaniya-iz-50-h-godov-hh-stoletiya
 
%%author_id burov_sergij.mariupol,news.ua.mrpl_city
%%date 
 
%%tags 
%%title Разговенье: воспоминания из 50-х годов ХХ столетия
 
%%endhead 
 
\subsection{Разговенье: воспоминания из 50-х годов ХХ столетия}
\label{sec:26_04_2019.stz.news.ua.mrpl_city.1.razgovenje_vospominania_iz_50_godov_xx_stoletia}
 
\Purl{https://mrpl.city/blogs/view/razgovene-vospominaniya-iz-50-h-godov-hh-stoletiya}
\ifcmt
 author_begin
   author_id burov_sergij.mariupol,news.ua.mrpl_city
 author_end
\fi

\ii{26_04_2019.stz.news.ua.mrpl_city.1.razgovenje_vospominania_iz_50_godov_xx_stoletia.pic.1}

Бабушка Таня, несмотря на то, что большая часть ее жизни прошла в Мариуполе,
так и не стала \enquote{городянкою}. Все ее привычки, весь ее образ жизни, все ее
поступки, все ее миропонимание шло оттуда, - с хутора Руда, что прилепился к
местечку Варва Лохвицкого уезда Полтавской губернии. Оставленный в тарелке
недоеденный борщ был преступлением, равным по тяжести с грабежом со взломом. В
то время, когда была еще при относительном здоровье, ее указания нужно было
претворять в действительность немедленно, как говорится, бежать в одном ботинке
для исполнения. Была она хлебосольной, всегда готова поделиться с ближним, в
самом широком понимании этого слова, но, Боже спаси, выбросить в мусор даже
скорынку хлеба. Недоеденные кусочки отправлялись на противень в духовку.
Торбынка с сухарями всегда висела в углу за печкой. Сухари же в нужное время
становились сырьем для приготовления кваса. Ее главным критерием трудолюбия
человека было то, как он ест. Если ест хорошо - то и работник хороший.

\textbf{Читайте также:} 

\href{https://mrpl.city/news/view/zakon-o-gosudarstvennom-yazyke-prinyat-chto-izmenitsya-v-zhizni-mariupoltsev}{%
Закон о государственном языке принят: что изменится в жизни мариупольцев, Роман Катріч, mrpl.city, 25.04.2019}

Вспоминается начало 50-х годов ушедшего ХХ столетия. Уже отправился в историю
голодный сорок седьмой год. Все меньше в Мариуполе оставалось погорелок, многие
дома восстановлены, на пустырях строились новые здания. У властей еще не дошли
руки до закрытия церквей, открытых в приспособленных помещениях в конце
Отечественной войны.

К этому времени произошло отложившееся в памяти семейное событие. Бабушка,
вернувшаяся из церкви, выкладывает на стол освященные паски, окрашенные в
луковой шелухе яйца, шматок сала и еще что-то, не сохранившееся в памяти. На
столе - суповые тарелки, рядом с ними ложки, вилки, стопки для хмельных
напитков, в центре стола подставка для кастрюли. Тут же присутствуют блюда с
оселедцями, кольцом домашней колбасы с сохранившимся смальцем, в котором
находился этот продукт домашнего кулинарного искусства с начала зимы,
нарезанное пластиками сало, зажаренный кусок свинины, зажаренный же небольшой
судак килограмма на три с половиной. А также соленья: огурчики в пупырышках,
толстощекие помидоры с прилипшими к ним веточками укропа, синенькие, начиненные
морковкой. В центре бутылка с \enquote{казенной} водкой – это для
присутствующих немногочисленных мужчин, графин с самодельной наливкой – для
женщин, и еще одна самая большая прозрачная емкость с узваром – это для детей.
Им в стопки, а взрослым - в стаканы и кружки.

Один из мужчин разливает напитки в стопки, кому что положено, мама заносит
огромную кастрюлю с борщом \enquote{зі старим салом, затовченим разом з часником}.
Бабушка поднимает стопку и произносит: \enquote{Христос воскрес!}. Все присутствующие
нестройным хором провозглашают: \enquote{Воистину воскрес!}. После первой стопки гости
и хозяева начинают есть борщ, между ложками расхваливая кулинарное искусство
бабушки и ее дочерей.

Разговенье началось.

\textbf{Читайте также:} 

\href{https://mrpl.city/news/view/v-mariupole-pashalnye-vyhodnye-projdut-s-solntsem-i-grozoj}{%
В Мариуполе пасхальные выходные пройдут с солнцем и грозой, Анастасія Селітріннікова, mrpl.city, 26.04.2019}
