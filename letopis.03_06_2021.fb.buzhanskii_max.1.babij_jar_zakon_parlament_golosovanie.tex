% vim: keymap=russian-jcukenwin
%%beginhead 
 
%%file 03_06_2021.fb.buzhanskii_max.1.babij_jar_zakon_parlament_golosovanie
%%parent 03_06_2021
 
%%url https://www.facebook.com/permalink.php?story_fbid=1959976274166842&id=100004634650264
 
%%author 
%%author_id buzhanskii_max
%%author_url 
 
%%tags 
%%title В этом году, у нас жуткий юбилей, 80 лет трагедии Бабьего Яра
 
%%endhead 
 
\subsection{В этом году, у нас жуткий юбилей, 80 лет трагедии Бабьего Яра}
\label{sec:03_06_2021.fb.buzhanskii_max.1.babij_jar_zakon_parlament_golosovanie}
\Purl{https://www.facebook.com/permalink.php?story_fbid=1959976274166842&id=100004634650264}
\ifcmt
 author_begin
   author_id buzhanskii_max
 author_end
\fi

Я, всё таки, позволю себе ещё пару слов по поводу сегодняшнего голосования за
запрет героизации эсэсовцев, и, не забудьте этот момент тоже, лиц,
непосредственно принимавших участие в Холокосте.

В этом году, у нас жуткий юбилей, 80 лет трагедии Бабьего Яра.

И можете не сомневаться, огромное количество народных депутатов будет скорбеть,
давать грустные интервью, повторять никогда снова и тд.

Вот тогда и наступит момент вспомнить, как они проголосовали за то, чтобы те,
кто убивал в Бабьем Яре не имели право на название улицы, памятную табличку,
свое имя, в качестве почётного названия для школы и тд.

Эти люди вполне могли это запретить.

Но предпочли разрешить сейчас, и скорбеть потом, осенью.

\ii{03_06_2021.fb.buzhanskii_max.1.babij_jar_zakon_parlament_golosovanie.cmt}




