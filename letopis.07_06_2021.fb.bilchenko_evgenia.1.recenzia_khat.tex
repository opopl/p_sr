% vim: keymap=russian-jcukenwin
%%beginhead 
 
%%file 07_06_2021.fb.bilchenko_evgenia.1.recenzia_khat
%%parent 07_06_2021
 
%%url https://www.facebook.com/yevzhik/posts/3956774101024337
 
%%author Бильченко, Евгения
%%author_id bilchenko_evgenia
%%author_url 
 
%%tags 
%%title БЖ. Последняя рецензия на КХАТ
 
%%endhead 
 
\subsection{БЖ. Последняя рецензия на КХАТ}
\label{sec:07_06_2021.fb.bilchenko_evgenia.1.recenzia_khat}
\Purl{https://www.facebook.com/yevzhik/posts/3956774101024337}
\ifcmt
 author_begin
   author_id bilchenko_evgenia
 author_end
\fi

БЖ. Последняя рецензия на КХАТ.

\enquote{Луч солнца золотого}... Щурьтесь: лето в Киеве светит жёлтым поцелуем Климта,
- если не с неба, то стенами, розами, худи на моей худобе. А потом мы попали на
новую постановку пьесы \enquote{Иоганна} Леси Украинки в КХАТ по постсобытиям распятия
Иисуса Христа, и мне понравилась, как о забвении своих имён в пользу
чужестранцев-господ из Рима говорила старая мать в роли актрисы Алены
Кожухаровой. Родина-мать. Чудесное перевоплощение. Я оценила. Я это восприняла
как голос Паоло Пазолини, разоблачающий манкуртизм. Он хотел снять фильм об
апостоле Павле, где в роли Рима выступил бы американский политикум. Но не
успел. 

И на сцене все было оранжевым, жёлтым, золотым, как купол храма.  Красивое
колористическое решение. Но я поверила только голосу матери - и за сим все. 

А вот гротеск, который продемонстрировали Виктор Кошевенко и та же Алена
Кожухарова в роли гордых римлян, - неуместен и провинциален, как и обилие
кабаре-танцев между рваными эпизодами незавершенной пьесы: театр не может брать
серией оргазмов  от созерцания фигурок полуголых танцовщиц со стороны сидящих в
зале мужчин. Максим Кабир бы сказал: \enquote{Базарные декадентки}.

Ещё о реализме игры. Римляне - гордые имперцы с честью virtus, даже, когда
кормят рыбу мясом образованных рабов (Борис Пастернак): в этом плане Татьяна
Лозина  и Валерий Рождественский играли римлян намного убедительнее. Они не
пшикали по-сельски, не поднимали голос до визга, а столично язвили. Можно
перечитать Цицерона, Толстого о кружке Элен Курагиной. 

Мне понравился из актерских цельных номеров только  Виктор Кошель - он
гениальный актер и глубоко страдающий человек, оголяющий жёсткую неторопливость
реалистической чеховско-станиславской спокойной игры без патетики: как будто в
Питере побывала, а не на селе. Потому жёлтая роза Катарина Синчилло  досталась
Вите как итоговая, прощальная и - одна на двоих. Тем более, что приму завалили
цветами, а лучшего актера - нет. 

Всего вам доброго. Спасибо за контрамарки. Мы - реально нищие. Но мы -
свободные, потому трагически счастливы. Мой вам этический и культурологический
долг воздан десятками похвальных рецензий до драматических событий со мной 18
января 2021 года. Отныне пошел новый качественный виток - это правило внучки
военного по ту сторону принципа наслаждения. Храни вас Бог, господа актеры.

\ifcmt
  tab_begin cols=3
     pic https://scontent-cdt1-1.xx.fbcdn.net/v/t1.6435-9/196177472_3956773867691027_4490400412874788332_n.jpg?_nc_cat=105&ccb=1-3&_nc_sid=8bfeb9&_nc_ohc=mb9-zLxhpiAAX-oHsmG&_nc_ht=scontent-cdt1-1.xx&oh=973d6579f58aa6444996188d2780d996&oe=60E56835

     pic https://scontent-cdt1-1.xx.fbcdn.net/v/t1.6435-9/196276258_3956773947691019_1253546264862878880_n.jpg?_nc_cat=109&ccb=1-3&_nc_sid=8bfeb9&_nc_ohc=vCtn44txvt8AX8TXmZc&tn=ntrKbsW_7ChXu3v-&_nc_ht=scontent-cdt1-1.xx&oh=8a4b42d210036414fcd422616b34adce&oe=60E1FB9D

     pic https://scontent-cdt1-1.xx.fbcdn.net/v/t1.6435-9/196609253_3956774031024344_3024363722226139593_n.jpg?_nc_cat=101&ccb=1-3&_nc_sid=8bfeb9&_nc_ohc=JxABf7o8UgIAX-e8WG5&tn=ntrKbsW_7ChXu3v-&_nc_ht=scontent-cdt1-1.xx&oh=1553ae86cdd48adcc169dd2d971fd303&oe=60E1F22A

  tab_end
\fi

\ii{07_06_2021.fb.bilchenko_evgenia.1.recenzia_khat.cmt}
