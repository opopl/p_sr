% vim: keymap=russian-jcukenwin
%%beginhead 
 
%%file topics.vojna.my.6.collect.poezia
%%parent topics.vojna.my.6.collect
 
%%url 
 
%%author_id 
%%date 
 
%%tags 
%%title 
 
%%endhead 

Подпольная кличка Окурок.
Я слышал, что Моль тоже.
Никто и не мог подумать.
Что это он это всё сможет.

А миру насрать в кашу,
Введя всех вокруг в заблуждение.
Никто и понять не может,
Какое-то наваждение.

Устроил в стране лагерь,
Народ обобрал до нитки.
И все кто с умом дружит
Собрали свои пожитки.

А кто не успел - на нары!
Чтоб сказать не могли слово.
А сам ведь уже старый,
Ну что ж ты творишь, Вова?

Войну развязал. Сволочь!
С Бандеровцами чтобы драться.
Людей погубил много.
Ночами они не снятся?

О  чём это я, право?
У тебя ведь души нету.
Там яд и сплошная отрава,
Но тебя призовут к ответу!



Где вы сейчас, почитатели Клиберна?
Элвиса Пресли и ранней Таганки...д
Ваша страна, сущность мерзкую вывернув,
Тащит в Донбасс отморозков и танки!
Где эти юноши, певшие Галича,
И от Хрущёва Манеж заслонившие?
Нешто вы все поразбиты параличом?
Или всё в прошлом, и вы уже "бывшие"?
Где эти девушки с песнями Визбора,
Ночи стоявшие в кассы Вахтангова?
Путь сатанинский страна ваша выбрала
И в Украину поехала танками!
Где вы, ценители Бакста и Рушевой,
Хармса, Синявского, Тэффи и Грина?
Всё это в Грозном, в Цхинвали разрушено...
Танком раздавлено...кражею Крыма...
Вы ли гордились тогда самиздатом,
Пряча в матрасах страницы "МетрОполя"?
Что ж вы молчите? Ведь ваши солдаты
Прут в Украину звериными толпами!
Стоило слушать ночами Высоцкого,
В жёлтых тонах рисовать субмарину,
Чтобы взрастить поколение скотское
И созерцать как бомбят Украину?
Стоило слушать Миллера Гленна
И обсуждать делово Мураками?
...Друг мой вернулся из русского плена,
Где был избит и затоптан ногами.
Стоило слушать музыку Шнитке,
Мудро стоя у тарковских зеркал,
Если вернулись доносы и пытки,
Сталин, война и " пойдёшь на подвал"?


21:53:12 20-08-22
Валерий Мартыненкоreplied to олег
20:19

"Колись на росії, десь під Брянськом,

В колгоспі з назвою Радянський

На краю самої клоаки,

На ланцюгу жив був собака.

Весь чорний, тільки вухо жовте,

Господар звав його "Пішов ти".

Пес жив, бо вбити було гріх,

Їв тільки те, що вкрасти міг.

В колгоспі тім, які розваги,

Щодня господар жлуктить брагу, хропе у дворі

Пес сцяв на нього, бувало мав його у ногу, бувало мав і у лице...

Але розмова не про це.

На речі час він має вплив, ланцюг одного разу згнив.

Біжи, тікай, мчи стрім голів... – свобода,

Пес пішов у хлів, там у хліві у стилі ню узрів в кутку наш пес свиню,

У грудях почастішав стук,

Цицьок аж чотирнадцять штук.

Ми тут пропустимо частину, ну пес зробив свині дитину,

І у хліві де так воня, з'явилось свиноцуценя.

Такий гібрид на божу злість –

В гівні і спить, гівно і їсть,

Найкраще взявши з мами, з тата,

Він виріс в руського солдата, засунув рило в БТР,

В Ізюм приїхав і помер.

Мораль:

Коли свиню *бе собака, то є кому ходить в атаку,

Ну, а коли наоборот, то руській мовчазний народ,

А толк із того тільки буде, у кого мама й тато люди".
