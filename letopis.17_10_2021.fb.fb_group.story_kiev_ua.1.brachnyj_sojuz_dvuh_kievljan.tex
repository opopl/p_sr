% vim: keymap=russian-jcukenwin
%%beginhead 
 
%%file 17_10_2021.fb.fb_group.story_kiev_ua.1.brachnyj_sojuz_dvuh_kievljan
%%parent 17_10_2021
 
%%url https://www.facebook.com/groups/story.kiev.ua/posts/1776708532525940/
 
%%author_id fb_group.story_kiev_ua,novickij_vladimir
%%date 
 
%%tags chelovek,gorod,kiev,kievljane,ljubov,pamjat,zhizn
%%title Брачный Союз Двух Киевлян
 
%%endhead 
 
\subsection{Брачный Союз Двух Киевлян}
\label{sec:17_10_2021.fb.fb_group.story_kiev_ua.1.brachnyj_sojuz_dvuh_kievljan}
 
\Purl{https://www.facebook.com/groups/story.kiev.ua/posts/1776708532525940/}
\ifcmt
 author_begin
   author_id fb_group.story_kiev_ua,novickij_vladimir
 author_end
\fi

Пару дней тому назад я опубликовал коротенький рассказ,даже не рассказ а
заметку о пожаре на нашей улице Дмитриевской.  Я просто не ожидал со стороны
читателей нашего  клуба такого интереса к жизни старого Киева. Отзывы меня
поразили и порадовали — как много людей болеют за наш родной Киев и хотят
сохранить его исторический облик. 


Многие в своих отзывах просили подробней описать жизнь в нашем любимом Киеве в
тот период и конечно рассказать о свадьбе , которую я упомянул. 

Сегодня  как  раз - ровно  62 года тому назад, 17 октября
1959 года на углу улиц Тургеневской и Полтавской города Киева   в районом загсе
Шевченковского района был заключён брачный союз двух коренных Киевлян,
родившихся  и выросших на Евбазе - он  на улице Дмитриевской,  в №42, а она в
№21 и этот брачный союз - благополучно продолжается до наших дней. Согласитесь
— это  дата достойная, чтобы о ней вспомнить и рассказать молодым людям,
родившимся  несколько позже, как  Киевляне  жили в те далёкие  времена и как
отмечались  тогда свадьбы — без дорогих платьев,  фаты, смокингов и лимузинов.

\ifcmt
  tab_begin cols=3

     pic https://scontent-lga3-1.xx.fbcdn.net/v/t1.6435-9/245521710_4416359331744652_3932576990601279845_n.jpg?_nc_cat=108&ccb=1-5&_nc_sid=b9115d&_nc_ohc=HSnfjsg3VS4AX-fSPxL&_nc_ht=scontent-lga3-1.xx&oh=0c0cc1a64816fe5011e50c3dc0581f26&oe=6192A240

     pic https://scontent-lga3-1.xx.fbcdn.net/v/t1.6435-9/246390182_4416359991744586_3900876142873360956_n.jpg?_nc_cat=107&ccb=1-5&_nc_sid=b9115d&_nc_ohc=ou6LOUhJeMsAX9kS9i4&_nc_ht=scontent-lga3-1.xx&oh=c1f9fb4b082bb7e64a6ea28099fa442a&oe=6195E109

     pic https://scontent-lga3-1.xx.fbcdn.net/v/t1.6435-9/245527552_4416360491744536_4368739241927947574_n.jpg?_nc_cat=101&ccb=1-5&_nc_sid=b9115d&_nc_ohc=oKV4Qs61EvgAX_gK-4s&_nc_ht=scontent-lga3-1.xx&oh=f573309833c4960406a9c40d09e4278c&oe=61938B24

  tab_end
\fi

Заранее прошу извинения у тех членов нашего клуба, которые могли читать мои
предыдущие рассказы в которых я  уже  рассказывал о Киевской жизни того времени
и эпизоды из которых я использовал в этом рассказе. Но поскольку , я начал
публиковать свои рассказики ещё с 2019 года, когда наша группа была
немногочисленной, а сейчас нас слава Богу  более 100 тысяч членов и как я понял
из комментарий на мой предыдущий рассказ о пожаре ,  найдётся много новых
членов клуба, которым будет интересно узнать о жизни нашего любимого Киева в те
далёкие и трудные годы, о которых пойдёт речь в этом рассказе.

Конечно время стёрло некоторые детали, но я постараюсь вспомнить и по
возможности  подробно рассказать Вам вначале о том времени в котором мы жили и
конечно же о той свадьбе, которая проходила прямо во дворе нашего дома под
музыку звучащую из старого патефона.

Отличительной частью того времени, без преувеличения можно сказать было то, что
люди были  дружны и стремились помогать друг другу, гордились успехами  детей,
вместе радовались и вместе грустили.

\ifcmt
  tab_begin cols=2

     pic https://scontent-lga3-1.xx.fbcdn.net/v/t1.6435-9/245504889_4416362785077640_4764638901328058748_n.jpg?_nc_cat=103&ccb=1-5&_nc_sid=b9115d&_nc_ohc=UXuDtPCrzKkAX9mxtrw&_nc_ht=scontent-lga3-1.xx&oh=06d07604f44b84b531ea6653be7a8383&oe=61936288

     pic https://scontent-lga3-1.xx.fbcdn.net/v/t1.6435-9/246358711_4416364398410812_8773482534315768357_n.jpg?_nc_cat=107&ccb=1-5&_nc_sid=b9115d&_nc_ohc=adglabuLQlEAX-2yZ8c&_nc_ht=scontent-lga3-1.xx&oh=6b8be762e8c3ba2baa323a4c4ae55776&oe=61955A3E

  tab_end
\fi

Раньше улицу, на которой ты жил, (все проживающие на ней) - считали своим
домом и знали друг друга в лицо, в отличии от теперешнего  времени, когда
иногда  не знают даже соседей по лестничной клетке.  Тогда ещё можно было
услышать проходя по родной улице мимо соседних дворов такое :   “да это же
Вовка, внук Сорокина с сорок второго»,  или покупая  что - нибудь  из
продуктов,  на углу в    «Басяцком» магазине  - я мог сказать продавщице - « у
меня тут не хватает  немного денег  заплатить за покупку, я занесу попозже» и
мне  верили, так как меня знали, потому что я  был свой « с нашей улицы».  Ну а
двор  - это вообще был как  квартира , его большая комната. 

\ifcmt
  tab_begin cols=3

     pic https://scontent-lga3-1.xx.fbcdn.net/v/t1.6435-9/245565259_4416365885077330_9021050026972532022_n.jpg?_nc_cat=109&ccb=1-5&_nc_sid=b9115d&_nc_ohc=xnnKcotae-8AX9aqvdK&_nc_ht=scontent-lga3-1.xx&oh=66fc65088579cbc11ff0c99d69b2194b&oe=6195F998

     pic https://scontent-lga3-1.xx.fbcdn.net/v/t1.6435-9/245585253_4416366131743972_668267981213811938_n.jpg?_nc_cat=111&ccb=1-5&_nc_sid=b9115d&_nc_ohc=ANXv888y5I4AX8_L3BC&_nc_ht=scontent-lga3-1.xx&oh=2f4808f0fdaad8c13f9e3c90921ee3a0&oe=619250D2

     pic https://scontent-lga3-1.xx.fbcdn.net/v/t1.6435-9/246123701_4416367931743792_7430479838727940806_n.jpg?_nc_cat=102&ccb=1-5&_nc_sid=b9115d&_nc_ohc=n0v1xQdsux8AX8gD7Q2&_nc_ht=scontent-lga3-1.xx&oh=e1833ed62a48fdda6484ebd845e3df59&oe=619407C1

  tab_end
\fi

Вдоль всего двора были сараи , где хранили дрова и уголь на зиму, а также
многие держали курочек и кроликов.  На чердаках, которые были общими — сушили
белье в дождевую погоду. Чердаки были запретным и вожделенным местом наших
мальчишеских игр, их «штабом» и               « наблюдательным пунктом». Кроме
того мы там держали голубей. Из детских забав  запечатлелись также наши налёты
на фруктовые сады на прилегающей к нам Тургеневской улице. А , чтобы нас не
покусали собаки, мы с крыш сараев, специальными (шополками) длиной палкой с
прикреплённой к ней мешочком срывали яблоки, груши и сливы, которые так были
нужны нашему неокрепшему детскому организму.

\ifcmt
  tab_begin cols=4

     pic https://scontent-lga3-1.xx.fbcdn.net/v/t1.6435-9/245506806_4416368225077096_4294170569021546050_n.jpg?_nc_cat=106&ccb=1-5&_nc_sid=b9115d&_nc_ohc=nAJGysV9NmcAX9WR_eY&_nc_ht=scontent-lga3-1.xx&oh=d6a1372ee96ce25f681ce3a7d531d206&oe=61940904

     pic https://scontent-lga3-1.xx.fbcdn.net/v/t1.6435-9/246238485_4416369101743675_6126284117117020724_n.jpg?_nc_cat=111&ccb=1-5&_nc_sid=b9115d&_nc_ohc=qlWMtpk5t_8AX92knof&_nc_ht=scontent-lga3-1.xx&oh=40287652ec60755d1b6ab8f358f56ff7&oe=6192B7EB

     pic https://scontent-lga3-1.xx.fbcdn.net/v/t1.6435-9/245539499_4416370905076828_8737889186139247810_n.jpg?_nc_cat=109&ccb=1-5&_nc_sid=b9115d&_nc_ohc=7mzGMSF9-2YAX_iepbW&_nc_ht=scontent-lga3-1.xx&oh=26c337f54fe75a1962ab5ce8a3ffae36&oe=619427F9

     pic https://scontent-lga3-1.xx.fbcdn.net/v/t1.6435-9/245532457_4416371528410099_7334394583365415420_n.jpg?_nc_cat=109&ccb=1-5&_nc_sid=b9115d&_nc_ohc=CdYHJhJizYUAX-Rz7GY&tn=lCYVFeHcTIAFcAzi&_nc_ht=scontent-lga3-1.xx&oh=d3648b1065bd694ef7e6c96b31384125&oe=6193EBD0

  tab_end
\fi

Центром двора  был водопроводный кран — колонка. Возле которого полоскали
бельё, купали замурзанных детей, чистили рыбу, чтобы не воняло дома. Зимой,
чтобы он не замёрз , его закутывали в  тряпки  и старые одеяла. Вода зимой
должна была всё время течь, чтобы не замёрз кран. И если кто-то по
невнимательности закрывал кран и вода замерзала, ему приходилось нанимать
слесаря с паяльной лампой, что бы он отогрел кран и люди могли набрать домой
воды.

Но возле крана собирались не только ради воды — Боже упаси. Это был клуб,
судилище, форум, своеобразный вариант знаменитого « Гайт — парка» в Лондоне,
где говорили обо всём и обо всех, не становясь даже на символическую трибуну.
Во дворе был прекрасный обычай, угощать соседей, которые были как члены большой
семьи, своими коронными блюдами. Никого совсем не удивляло, когда одна соседка
могла попросить другую, приготовить своё коронное блюдо, которое только она так
прекрасно могла приготовить. Или тётя Надя говорила тёте Фане « Фаничка - Ваш
Бог не обидится , если Вы скушаете кусочек вот этой чудесной пасхи, которую я
испекла к празднику». Дворы и сами являли собой колоритную картину,
специфически озвученную с утра до вечера. Утро начиналось пронзительными
криками молочниц «Молоко-о-о, Молоко-о-о!!!» затем появлялись уличные мастера,
нищие с гармошками и без, старьевщики. Раздавались крики: «Стекла вставляем!
Стеклы!!!,Подайте Христа ради!, Старые вещи покупаем! Старывещ! Старывещ!
Липкая бумага! Купите и можете спокойно спать без мух! Липкая бумага!!! Швейные
машинки чинить, вёдра, чайники паять и лудить! Точить ножи , ножницы и
мясорубки! Сольные партии гастролёров звучали на фоне многоголосого дворового
хора: « тётя Соня вы были уже на Евбазе, почём сегодня мясо? Раичка идите же
быстрей — в Босяцком выбросили  головы и ножки. Зиночка, сбегай и посмотри —
открыли ли керосин? Дядя Гриша киньте мне пару спичек, Фира Борисовна, это не у
Вас горит тряпка? , мадам Новицкая, который уже час?   Вот в такой обстановке
взаимной выручки проходила наша с Лилей свадьба.  На этой свадьбе  люди сидели
не семьями, а домами, верней дворами. Так как чаще всего раньше говорили  не
«из нашего дома» а из «нашего двора». Больше  гостей конечно было из «нашего
двора», но и с Лилиного двора, гостей было тоже достаточно.

\ifcmt
  tab_begin cols=2

     pic https://scontent-lga3-1.xx.fbcdn.net/v/t1.6435-9/245828096_4416372001743385_536954348621327662_n.jpg?_nc_cat=111&ccb=1-5&_nc_sid=b9115d&_nc_ohc=zZaWd8L71WsAX8GYZ2w&_nc_ht=scontent-lga3-1.xx&oh=03de63ada80c4ca4f3941d69dd26e75b&oe=6194F3DC

     pic https://scontent-lga3-1.xx.fbcdn.net/v/t1.6435-9/246345288_4416373268409925_5137104621148349358_n.jpg?_nc_cat=111&ccb=1-5&_nc_sid=b9115d&_nc_ohc=CmHxZlvmqeEAX9MZhcA&tn=lCYVFeHcTIAFcAzi&_nc_ht=scontent-lga3-1.xx&oh=66db78bdf214df5db4cbc9bc1242c1d2&oe=619271F5

  tab_end
\fi

Но сначала расскажу, как я с ней познакомился. Постепенно , по мере того, как я
становился  старше, детские забавы становились уже нам не интересны и мы с
ребятами стали выходить на улицу и наблюдать как  прохаживались девочки по
другой стороне нашей улицы. Там я и увидел впервые Лилю, которая прохаживалась
с подружками Зинкой с девятнадцатого  и Светкой с  тридцать восьмого. Они были
все очень симпатичные — имели прекрасные фигурки, но  я почему то сразу начал
больше смотреть на Лилю — она мне понравилась  больше других. Но подойти к ней
и познакомиться я  ещё стеснялся. Просто обменивались взглядами. Видно было,
что мы друг другу нравимся.  Ближе мы познакомились, когда она стала заходить к
моему другу  Фимке, с которым она училась в одном классе 141 школы на Речной
улице. Фимка жил в 5 квартире , а я в 6-ой.   Потом мы стали вместе ходить на
стадион « Старт» кататься на коньках. 


\ifcmt
  tab_begin cols=2

     pic https://scontent-lga3-1.xx.fbcdn.net/v/t1.6435-9/245508949_4416374345076484_526658819214799403_n.jpg?_nc_cat=110&ccb=1-5&_nc_sid=b9115d&_nc_ohc=PiyK-rujZ74AX8xKL-C&_nc_ht=scontent-lga3-1.xx&oh=ff30e724a872140d7ab6969175fe7383&oe=619444DF

     pic https://scontent-lga3-1.xx.fbcdn.net/v/t1.6435-9/245504887_4416373921743193_6071925567350763339_n.jpg?_nc_cat=108&ccb=1-5&_nc_sid=b9115d&_nc_ohc=fPw6DZIJqtoAX9mFOtY&_nc_ht=scontent-lga3-1.xx&oh=03421f2b22c6ea6e4f7f9a6b26582d5f&oe=6195E3F5

  tab_end
\fi

Как то я решился пригласить её пойти вместе со мной в кино «Ударник», где
показывали трофейный фильм.  Детство видимо безвозвратно уходило, начиналась
юность с другими интересами и развлечениями. Первое свидание у нас было у
памятника Щёрсу, она пришла такая красивая, что влюбиться в неё было просто
невозможно и я влюбился - как оказалось на всю жизнь.

Я стал заходить к ней домой. Жили они очень бедно — папа погиб на фронте, мама
работала швеёй на двух работах, чтобы прокормить и как то одеть двух девочек,
чтобы они  выглядели не хуже других детей.

Заходила она и к нам, моей маме она понравилась сразу. Мама часто , увидев её
из окна , как она идёт со школы, кричала мне во двор. Вова , твоя статуэточка (
она так её называла) - уже пришла со школы. Иди , а то её точно кто-то украдёт.
И я бежал через дорогу в её двор.  Ухажёров у неё действительно было много, так
что надо было держать ухо востро. Увидев, что у неё нечего  кушать, я старался
привести её в наш дом и мама , она понимала меня с полу слова, тут же
уговаривала, не смотря на её отказ, покушать с нами.

Потом мама садилась за пианино и играла , она любила и умела играть популярные
в то время мелодии и песни, а Лиля пела, у ней был прекрасный голос, потом она
даже пела в вокально -инструментальном ансамбле завода « Большевик», куда мы
пришли работать после окончания института. Вот так, задолго до свадьбы у Лили с
мамой, а значит и со всей нашей семьей, так как главой семьи — была мама,
сложились прекрасные и доверительные отношения.  В связи с этим, я часто
задавал  себе вопрос, почему так произошло, ведь моя мама ,  была, как говорили
- « не подарок»,  довольно суровой и требовательной женщиной. Примером может
служить случай, когда по окончании школы — она заказала мне на выпускной  вечер
чудесный костюм и сказала при этом : «Это последнее, что я тебе сделала, если
ты не пойдёшь дальше учиться, а будешь шалопайничать — не получишь от меня
больше ни копейки ». И так оно бы и было, я силу её слова знал.  Кроме того она
часто нам с братом говорила , что мечтала иметь дочь, думала, что вторым
ребёнком, вместо меня родится девочка, чтобы научить её всему , что она умела.
К сожалению с Жориной, моего старшего брата первой женой, когда они жили у нас,
доверительных отношений  у неё к сожалению не было.. Слишком характеры у них
были разными. Поэтому , когда через 4 года, Жора с ней разошёлся и женился на
Ире Скляренко, которая  вместе с нами  училась в Хореографическом училище и
переехал  жить к ней на улицу Ленина. В результате  освободилась отдельная
комната где он жил с прежней женой и  мы  с Лилей  подумали , а не
воспользоваться ли нам этим благоприятным случаем  и  пожениться.  Встречались
мы уже 4 года и  несмотря на то, что нам было всего по  20 лет, со стороны мамы
возражений не было. Она только сказала - вижу Вы у меня сильно мэдэкованные,
увидели , что освободилась отдельная комната и сразу заспешили жениться. Но это
и для меня и для Жоры думаю будет хорошо, не будет бегать туда — сюда. Будет
держаться за новую семью.  Видимо за эти годы пока мы встречались с Лилей,  она
успела хорошо изучить её и поняла, что Лиля  будет ей не просто невесткой , а ,
как она и хотела — дочкой. В итоге , жизнь показала, что так оно и вышло.
Больше 20 лет мы прожили с родителями в одной квартире, пока построили
кооператив и я не помню случая, чтобы между ними было какое -то недопонимание,
я уже не говорю о конфликтах..  Единственным требованием со стороны мамы —
было,  чтобы пока мы учимся в Институте не иметь детей. Так и сказала, как
отрезала « Если не знаете как — я  Вас научу». Вот почему у нас Мишенька
родился в 1964 году, ровно через 5 лет после свадьбы. После того, как мама
выдала нам эту тираду, я  набрался смелости и обратился к ним двоим - сказал «
Вот что  девочки, выслушайте и моё наставление. Это прекрасно, что у Вас сейчас
такие чудесные отношения, но в жизни бывает всякое и поэтому я прошу, если у
Вас в дальнейшем будут какие то трения  между собой, прошу не приходить ко мне
и жаловаться друг на друга . Слушать Вас в отдельности я  не собираюсь.  Отвечу
только -« Бачили очi, що купували». Маме на её претензии, если они будут,  я
скажу, что — это моя жена и я сам с ней разберусь.. А Лиле — скажу просто  ,
что — это моя  МАМА. Поэтому разбирайтесь сами и нечего меня в это вмешивать.
Слава Богу — такого никогда не было. Но молодым людям я хочу посоветовать.
Вступая в брак, постарайтесь не вступать в полемику между мамой и женой. Ничего
из этого хорошего не выйдет. Они сами разберутся. Так будет верней и брак
сохранится на долгие годы, тому примером наш.  Продолжу по поводу маминого
наставления. Ослушаться маму мы не могли, да и сами не хотели, так как мы были
слишком молодыми и на её полном обеспечении. Стипендию правда мы ей отдавали,
но кормила и , что главное красиво одевала нас она. Да и на кино и на прочие
расходы нам давала. Заработать она умела ещё с времён оккупации, когда смогла
сама  содержать , без кормильца отца, семью из 6 человек, двух детей и трёх
стариков. Она и сейчас, несмотря, на то что папа получал  хорошую зарплату, как
зам директора завода « Большевик « и пенсию — как инвалид войны, могла выйти в
воскресенье , когда ещё была толкучка на Евбазе и за день заработать больше,
чем он получал за месяц. Кроме того  у неё была куча заказчиц, которым она шила
по выкройкам из журнала «Бурда». Расскажу такой случай, как то мы всей семьёй
пошли в кино на фильм « Девушка моей мечты», в нём пела и плясала знаменитая
венгерская актриса Марика  Рёкк. Мама спросила Лилю понравилось ли ей
красивейшее платье в котором она танцевала и если ей оно понравилось , то она
ей сошьёт такое же, что она к величайшей радости Лили и сделала за  неделю.
Когда Лиля одела  его и вышла в нём на нашу улицу, - фурор был огромный, ведь
многие запомнили это платье ещё из фильма,  который все тоже смотрели.  Да ,
мама была не только рукодельницей, она ещё и многому нас научила, что
пригодилось нам в дальнейшей нашей жизни.  Она научила нас никогда не
отчаиваться, собираться в трудную минуту и находить выход из любых  положений.
Научила, прежде чем что - то сделать или сказать,  хорошо подумать, как это
отразиться на твоей судьбе в дальнейшем. Не быть наивным и уметь  разбираться а
людях. Она всегда мне говорила и это подтвердила вся последующая моя  жизнь,
что не всякий кто тебе красиво говорит — обязательно хороший человек. А тот,
кто явно тебе льстит —  если не враг, то уж точно плохой человек и наверняка
его надо опасаться. 

Научила уважать старших и прислушиваться  к их мнению. Научила быть
благодарным, не быть Фомой, не помнящим добра и родства и ещё многому, многому
другому. Кстати она научила меня хорошо готовить пищу и это стало в дальнейшем
моим хобби. А папа научил меня быть — обязательным. Если дал слово — выполни.
Не можешь сделать в обещанный срок, позвони , извинись и скажи когда сделаешь.
Он страшно не любил сам опаздывать куда либо и не уважал людей не обязательных,
которые  не понимали, что опоздание — это  прежде всего  не уважением к
человеку, который тебя ждёт.

Если , как Вы поняли мои родные не возражали против нашей свадьбы, то Лилина
мама была против и расскажу почему. У её сестры был русский муж и когда в 1953
году начались репрессии против врачей евреев и Сталин готовил переселить всех
евреев в Биробиджан, он испугался и её бросил. Она боялась, что подобное я могу
тоже сделать. Но потом , увидев как мы любим друг друга и познакомившись с
моими родителями, она дала согласие и даже пообещала дать мне в приданое -
один стул и подушку, которое так мне и не досталось, за что , я её постоянно
подначивал, спрашивая «так где же стул и подушка?», на что она отвечала «я же
тебе дала такую прекрасную девочку». По поводу девочки я с ней соглашался, но
отвечал, что  договор — есть договор, где стул и подушка? Лилина мама была
прекрасным человеком, оставшись в 28 лет вдовой, всю жизнь  посвятила детям и
помогла нам  вырастить наших детей. Я её очень любил и похоже она меня тоже.
Недаром же она меня называла ласково - «Нахес» и «Герунт золт зайн», что  я
понимал как «Радость моя, чтоб ты был мне здоров».

Братья Лилиного отца, который погиб на войне,  дали нам деньги, на которые мы
купили тахту- лиру в  мебельном  магазине напротив «Босяцкоого», а ребята
дворовые принесли её на руках, так как на извозчика денег не хватило. Дед
Сорокин подарил две золотые монеты из которых сделали обручальные кольца. Мама
с папой взяли на себя все расходы по свадьбе и подарили Лиле  золотые часики и
модные туфельки . Мама пошила ей новое платье и шубку со шляпкой. Получилось
так, что мне , кроме самой Лили, да ещё без приданого, никто ничего и не
подарил. Нет, вспомнил, мой школьный друг Вася Сирош, подарил хрустальный
графинчик с шестью рюмочками, который мы привезли в Америку . Рюмочек правда
осталось только четыре. Но каждый раз на 17 октября мы с Лилей из этого
графинчика  наливаем себе по рюмочке коньячка и вспоминаем тот счастливый для
нас день 17 октября 1959 года.

Как прошла свадьба не помню, так как был сильно взволнован Родственники сидели
в нашей большой комнате, причём всю Лилину родню - «мишпуху»  посадил кто -то
прямо под дедушкиным  иконостасом. У него были очень красивые  старинные иконы.
Думаю, что кого - то это возможно и смутило, но вида никто не подал. Ну а
основная свадьба была, как я и говорил во дворе. Там веселье шло полным ходом .
Выпивки и закусок хватало, так как Жора отвечал за  то, что бы спиртное не
заканчивалось, а закуски подносили постоянно сами же соседи, которые эти
закуски и готовили. Столы  вынесли из квартир и приставили к тому дворовому
столу, где обычно  играли в домино и карты. На два стула ложилась доска, так
что сидеть всем было где. Был патефон и танцы. На шум свадьбы заходили
проходящие по улице люди и тоже присоединялись к празднованию. 

Короче свадьба удалась — также -  как и наша дальнейшая жизнь.

Не даром же сегодня отмечаем  - 62 ую годовщину.  Бриллиантовую (60 лет) — дети
и внуки нам устроили уже с лимузином и рестораном. А золотую свадьбу (50лет)
мы отмечали всей семьёй в Лас Вегасе. Дочь Таничка нам сняла специальный номер
для молодожёнов и когда мы зашли в него, вдруг включились телевизоры , на
которых  зажглось  «Добро пожаловать Лилия и Владимир Новицкие», а на столе уже
стояло шампанское и клубника в шоколаде.

Собственно говоря  этот рассказ я задумал и написал  совсем не  о свадьбе , а
скорей о том, как жили люди в то время, когда отношения между людьми были
другими, чем сейчас. Я не ортодокс и вовсе не осуждаю людей, которые стремятся
сейчас жить лучше чем мы жили тогда — это вполне естественно. Новое время и
новое поколение  естественно должно жить по другому и лучше чем жили  мы . Но
хотелось бы , чтобы люди были немножечко мягче и добрей. Очень много  сейчас
негатива , и это меня очень тревожит.

Толкнуло  меня собственно  написать этот рассказ — первое  фото, которое я
выставил, полученное  от нашей соседки Ирочки Парасий — Вергуненко  из
квартиры №7 и которая нашла нас именно благодаря рассказам на  «Кивских
историях». Она прислала мне это  очень старенькое фото, где  видна счастливая
Лиля на крылечке нашего дома и смотрящая на неё  с улыбкой моя мама. С обратной
стороны фото надпись : Лиля  и тётя Клава 1958 год.  Это фото, кем - то
сделанное за год до нашей свадьбы,  отчётливо показывает те отношения , которые
я описал в этом рассказе между моей мамой и моей будущей женой Лилей.
Посмотрите ещё раз на это фото, оно очень старенькое, но оно  о многом говорит. 

На втором фото видно второе крылечко, где  уже я её сфотографировал . Это год
нашего знакомства 1955. Совсем девочка, но очень  хорошенькая. Она жила на
первом этаже этого дома, в квартире  из одной комнаты с земляным полом  -
покрытым чем то на подобии линолиума и маленькой кухоньки  без окон.   Жила она
там с мамой и старшей сестрой Ирой.

Насколько  она стала ещё красивей, чем когда я на ней женился  и как мы   по
прежнему любим друг друга  на последующих снимках.
