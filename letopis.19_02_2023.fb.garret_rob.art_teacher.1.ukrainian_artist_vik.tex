%%beginhead 
 
%%file 19_02_2023.fb.garret_rob.art_teacher.1.ukrainian_artist_vik
%%parent 19_02_2023
 
%%url https://www.facebook.com/robgarrettartteacher/posts/pfbid03vVNSugkmEe8MzRWUpFduTcgmD9h3UbVwAMZqDXD5pUBGRtnnRhxxPuc9mZtjhudl
 
%%author_id garret_rob.art_teacher
%%date 19_02_2023
 
%%tags isskustvo,hudozhnik
%%title Ukrainian artist Viktor Zhmak - український художник Віктор Жмак
 
%%endhead 

\subsection{Ukrainian artist Viktor Zhmak - український художник Віктор Жмак}
\label{sec:19_02_2023.fb.garret_rob.art_teacher.1.ukrainian_artist_vik}

\Purl{https://www.facebook.com/robgarrettartteacher/posts/pfbid03vVNSugkmEe8MzRWUpFduTcgmD9h3UbVwAMZqDXD5pUBGRtnnRhxxPuc9mZtjhudl}
\ifcmt
 author_begin
   author_id garret_rob.art_teacher
 author_end
\fi

DAY 361 💙💛 Ukraine in my heart and on my mind! 💙💛 Today's post features
Ukrainian artist Viktor Zhmak (b.1954) from Lviv, born in the village of
Pershotravensk, Zhytomyr region. Education: Lviv Polytechnic University,
Faculty of Architecture. Between 1990 and 2002, Viktor Zhmak was engaged in
architectural activities, and since 1991, the artist has been participating in
all-Ukrainian and foreign exhibition projects, in particular, a solo exhibition
at the National Museum named after A. Sheptytskyi (Lviv; 1995), in the Lviv
Palace of Arts (2015), \enquote{Flowers, wine and mares} (Madame Palmgren Gallery,
Lviv; 2018), \enquote{In anticipation of the holiday} (Proart Gallery, Lviv; 2018),
etc. The influence of architectural education and profession can be traced in
Viktor Zhmak's work. The author's creative work consists of landscapes, still
lifes, figurative and abstract painting.

\enquote{Sometimes you start to paint one picture, and the result is a completely
different one. From time to time, you must wait until the desire to draw
returns... One of the works, for example, I painted for about ten years. I
often must discipline myself - force myself to draw, because I always face a
dilemma: drawing or architecture? And although I do both, I sometimes want to
switch to one thing.}

💙💛
Першотравенськ Житомирської області. Освіта: НУ «Львівська політехніка»,
факультет архітектури. Протягом 1990-2002 років Віктор Жмак провадив
архітектурну діяльність, а з 1991 року художник бере участь у всеукраїнських та
закордонних виставкових проєктах, зокрема персональна виставка у Національному
музеї ім. А. Шептицького (Львів; 1995), у Львівському палаці мистецтв (2015),
«Квіти, вино та кобіти» (Галерея Мадам Пальмгрен, Львів; 2018), «У передчутті
свята» (Proart Gallery, Львів; 2018) та ін. У творчості Віктора Жмака
прослідковується вплив архітектурної освіти. Творчий доробок автора складають
пейзажі, натюрморти, фігуративний та абстрактний живопис.

\enquote{Іноді починаєш малювати одну картину, а в результаті виходить зовсім інше. Час
від часу потрібно зачекати, поки повернеться бажання малювати… Одну з робіт,
наприклад, малював близько десяти років. Часто доводиться себе дисциплінувати –
змушувати малювати, адже я завжди стою перед дилемою: малювання чи архітектура?
І хоч займаюся і тим, й іншим, хочеться іноді перейти на щось одне.}

💙💛
See more of his artworks here:

Дивіться більше його робіт тут:

Web: \url{http://viktor-zhmak.com/gallery}  

Web: \url{https://arts.center/jmaniko} 

FB Page: \url{https://www.facebook.com/profile.php?id=100066714781405} 

Other:

\url{https://sverediuk.com.ua/viktor-zhmak/} 

\url{https://proartgallery.com.ua/event/віктор-жмак/}  

\url{https://sverediuk.com.ua/hudozhnik-viktor-zhmak-natyurmort/} 

\url{https://goods-for-art.livejournal.com/194799.html} 

💙💛 People matter! Freedom matters! Ukraine Matters! Slava Ukraini! Heroyam
Slava! Слава Україні! Героям Слава! Glory to Ukraine! Glory to heroes! Все буде
Україна. Everything will be Ukraine.
