% vim: keymap=russian-jcukenwin
%%beginhead 
 
%%file 13_05_2021.fb.lnrgumo.2.bog_donbass_svjatograd
%%parent 13_05_2021
 
%%url https://www.facebook.com/groups/LNRGUMO/permalink/3683780788400135/
 
%%author 
%%author_id 
%%author_url 
 
%%tags 
%%title 
 
%%endhead 
\subsection{Святоград Луганский}
\label{sec:13_05_2021.fb.lnrgumo.2.bog_donbass_svjatograd}
\Purl{https://www.facebook.com/groups/LNRGUMO/permalink/3683780788400135/}

Не робкого десятка. Это определение достаточно живо характеризует традицию
Крестных ходов в Святограде Луганском.  Они сродни подвигу 300 спартанцев,
вставших на защиту мира, когда мир пребывал в "замешательстве". 

\ifcmt
  pic https://scontent-bos3-1.xx.fbcdn.net/v/t1.6435-9/185807733_4189703351074568_2600185448938138680_n.jpg?_nc_cat=102&ccb=1-3&_nc_sid=825194&_nc_ohc=W6lir1TBGHUAX90EGGx&_nc_ht=scontent-bos3-1.xx&oh=681d05055040393c612155d515507364&oe=60C5C734
\fi


Еще ближе всем нам пример наших "спартанцев" –  вставших в Славянске на пути многочисленных орд сланцедобытчиков в 2014 году. 

А еще  жива память в Русском мире  о  Боже, его детях и 70-ти боярах, которые
встали на защиту Бога, когда от него – Бога Живого Единосущного в 4-м веке
отрёкся даже Папа Римский, увлёкшийся арианской ересью. Тогда, в 375 году на
нашу землю ополчились орды готтов т.е."богов", кем они себя возомнили,
отвергнув Христа, как Единосущного Бога, а себя возвысив, как ранее сделали
языческие императоры Римской империи. 

Вдумайтесь! Судьба Вселенской Христианской Церкви решалась даже не в Ватикане
или Константинополе , а в Донбассе, где наши неробкого десятка предки Анты не
предали Бога.  И пошли с Ним на распятие. Сначала был распят Бож и его сыновья,
затем бояре. По преданию, распятие произошло весенним майским днём где-то на
восточной окраине антского мира. Ныне археологи границей Антов определяют
Донецкую область, село Богородичное Славянского района, известного
Святогорской Лаврой. Анты подобно святому великану-мученику Христофору на своих
плечах перенесли Христа и его вселенную через мутный поток  времени.

 Ныне в Луганске идёт подготовка к крёсному ходу, посвящённому чествованию
 иконы Богородицы Луганской. Веряне вспоминают предание о явлении Богородицы в
 Луганске старцу Филиппу Луганскому, заповедавшей помнить, как Пасху знамение о
 Святограде Луганском откуда прольётся свет по всей земле и куда будут
 стекаться люди со всего мира. Это может стать своего рода собором христиан
 всего мира.

А был и Первый Вселенский Собор. Епископы были на него созваны императорским
указом Константина I в МАЕ 325 г. Прогоны, почтовые лошади – все это было
бесплатно предоставлено Империей. Константин звал всех, всех, всех.
Приглашались делегаты не только из Империи, но и заграничный епископат: из
Сирии, Армении, Кавказа, Персии. К тому времени соборная практика уже была
всеобщим правилом. Но то были соборы местные: в Африке, в Александрии, в Сирии,
в Азии. Даже соседние области, например Египет и Антиохия, ни разу не
собирались вместе.

Вообще это первое собрание такого рода в истории. Единство Римской империи было
весьма умозрительным понятием. Ни разу ее представители с разных концов не
собирались вместе, не совещались, не съезжались, почти даже не знали друг
друга. Мысль о всеобщей личной встрече, неком светском, культурном
«соборовании», была чужда Империи.

Только христианская Церковь, переросшая уровень двух миров – иудаизма и
эллинизма, породила и осмыслила саму идею всеобщности, вселенскости,
всемирности человеческой истории, сознательно оттолкнувшись от всех обветшавших
местных национализмов. «Нет ни эллина, ни иудея, но всё и во всем Христос».
Константин потому и стал Великим, что эта идея пленила его. 

Закладывая в основу перерождаемой Империи новую религиозную душу, он творил
историческое дело выше дела самого августа. Рождалась подлинная вселенскость,
которую осознал не епископат, а римский император. Император, империю которого
унаследомал "ТРЕТИЙ РИМ". 

Память Первого Вселенского Собора празднуется Церковью Христовой с древнейших
времен. Празднуется результат – достижение всеми определённости в осознании
Символа веры. Господь Иисус Христос оставил Церкви великое обетование: "Создам
Церковь Мою, и врата ада не одолеют Ее" (Мф. 16, 18). В этом радостном
обетовании находится пророческое указание, что, хотя жизнь Церкви Христовой на
земле будет проходить в трудной борьбе с врагом спасения, победа на Ее стороне. 

Святые мученики засвидетельствовали истинность слов Спасителя, претерпев
страдания за исповедание Имени Христова, и меч гонителей склонился перед
победоносным знамением Креста Христова,  и осознание того, что нет больше той
любви, чтобы положит душу свою за други своя .

С IV века прекратились преследования христиан, но внутри самой Церкви возникли
ереси, на борьбу с которыми Церковь созывала Вселенские Соборы. Одной из
опаснейших ересей было арианство. Арий, александрийский пресвитер, был
человеком безмерной гордыни и честолюбия. Он, отвергая Божественное достоинство
Иисуса Христа и Его равенство с Богом Отцом, ложно учил, что Сын Божий не
Единосущен Отцу, а сотворен Отцом во времени. 

Поместный Собор, созванный по настоянию Александрийского Патриарха Александра,
осудил лжеучение Ария, но тот не покорился и, написав многим епископам письма с
жалобой на определение Поместного Собора, распространил свое лжеучение по всему
Востоку, ибо получил поддержку в своем заблуждении от некоторых восточных
епископов. Для расследования возникшей смуты святой равноапостольный император
Константин (память 21 мая) направил епископа Осию Кордубского и, получив от
него удостоверение, что ересь Ария направлена против самого основного догмата
Христовой Церкви, решился созвать Вселенский Собор.

 По приглашению святого Константина в город Никею в 325 году собрались 318
 епископов - представителей христианских Церквей из разных стран.

Среди прибывших епископов было много исповедников, в том числе из Приазовья,
пострадавших во время гонений и носивших на телах следы истязаний. Участниками
Собора были также великие светильники Церкви - святитель Николай, архиепископ
Мир Ликийских (память 6 декабря и 9 мая), святитель Спиридон, епископ
Тримифунтский (память 12 декабря), и другие, почитаемые Церковью святые отцы.

Александрийский Патриарх Александр прибыл со своим диаконом Афанасием,
впоследствии Патриархом Александрийским (память 2 мая), названным Великим, как
ревностный борец за чистоту Православия. Равноапостольный император Константин
присутствовал на заседаниях Собора. В своей речи, произнесенной в ответ на
приветствие епископа Евсевия Кесарийского, он сказал: "Бог помог мне
низвергнуть нечестивую власть гонителей, но несравненно прискорбнее для меня
всякой войны, всякой кровопролитной битвы и несравненно пагубнее внутренняя
междоусобная брань в Церкви Божией".

Арий, имея своими сторонниками 17 епископов, держался гордо, но его учение было
опровергнуто и он отлучен Собором от Церкви, а святой диакон Александрийской
Церкви Афанасий в своей речи окончательно опроверг богохульные измышления Ария.
Отцы Собора отклонили символ веры, предложенный арианами. 

Был утвержден православный Символ веры. Равноапостольный Константин предложил
Собору внести в текст Символа веры слово "Единосущный", которое он часто слышал
в речах  светильников церкви епископов из Приазовья.

Отцы Собора единодушно приняли это предложение. В Никейском Символе святые отцы
сформулировали апостольское учение о Божественном достоинстве Второго Лица
Пресвятой Троицы – Господа Иисуса Христа. 

Ересь Ария, как заблуждение гордого разума, была обличена и отвергнута. После
решения главного догматического вопроса Собор установил также двадцать канонов
(правил) по вопросам церковного управления и дисциплины. Был решен вопрос о дне
празднования Святой Пасхи. Постановлением Собора Святая Пасха должна
праздноваться христианами не в один день с иудейской и непременно в первое
воскресенье после дня весеннего равноденствия (который в 325 году приходился на
22 марта). 

Мало кто помнит, что христианство и христиане в Приазовье были россіяні еще
Андреем Первозванным, а среди первых святых из среды тех, кто были россияни
были Пинна, Римма и Инна. Они почитаются церковью дважды в году. Вцелом же
словенских святых за их подвиг исповедников-фисонитов Церковь чтит 18 мая.

(По материалам книги "Доля Луганска быть Святоградом").
