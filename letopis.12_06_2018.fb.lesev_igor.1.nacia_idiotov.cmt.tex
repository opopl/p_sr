% vim: keymap=russian-jcukenwin
%%beginhead 
 
%%file 12_06_2018.fb.lesev_igor.1.nacia_idiotov.cmt
%%parent 12_06_2018.fb.lesev_igor.1.nacia_idiotov
 
%%url 
 
%%author_id 
%%date 
 
%%tags 
%%title 
 
%%endhead 
\subsubsection{Коментарі}

\begin{itemize} % {
\iusr{Станислав Бочкур}

Ну, когда в сегодняшний день смотреть стыдно, а в завтрашний - страшно, только
и остаётся, что мриять о грядущих перемогах лет через 10-15, по принципу "а
вдруг?"

\iusr{Сергей Куприянов}

Подозреваю, что если на Кубани какого-нибудь Петренко назвать украинцем, то
можно отгрести и в морду. (рассуждения по поводу картинки).

\begin{itemize} % {
\iusr{Станислав Бочкур}
Откуда взяться украинофобии на Кубани?

\iusr{Сергей Куприянов}
Дело не украинофобии, а в идентификации себя, как русского, особенно на фоне событий на Украине.

\iusr{Станислав Бочкур}
\textbf{Сергей Куприянов} О, нет. Многие на Кубани и не только идентифицируют себя как украинцы. Но не как жёлтоголубисты, конечно.

\iusr{Сергей Куприянов}
\textbf{Станислав Бочкур} На этнографическом уровне, не более.

\iusr{Станислав Бочкур}
\textbf{Сергей Куприянов} Ага. А другого и не надо.

\iusr{Таня Кушнир}

Вы были на Кубани?  @igg{fbicon.smile}  уверена, что все строго бы наоборот случилось. Этот
кубанский Петренко бы вам радостно рассказал, сколько у него в роду украинцев,
кто где воевал и кем он гордится. Они простые и наивные. И совершенно
естественно быть русским и с дедом-бабкой украинцами.

\end{itemize} % }

\iusr{Иван Данилов}

Так вся политика властей Украины после 91 года основана на этом. Простому
народу впаривается национализм, а элита под шумок дерибанит советское
наследство

\iusr{Надежда Топоровская}
Вернет, когда попадет в состав нового союза.

\iusr{Михайло Бойченко}
Между прочим русские тоже мечтали вернуть крьім полвека... Дождались

\begin{itemize} % {
\iusr{Матвей Кублицкий}
только до этого русские им двести лет владели и кровушки за него пролили не меряно

\iusr{Михайло Бойченко}
Кто им только не владел и не мечтал! Думаю, Турция и сейчас не против его вернуть @igg{fbicon.wink} 

\iusr{Станислав Бочкур}
\textbf{Михайло Бойченко} Кто бы ещё спрашивал ту Турцию...

\iusr{Матвей Кублицкий}
\textbf{Михайло Бойченко} она им практически не владела. Крымское ханство просто присягнуло османам, сохранив в какой-то мере свободу

\iusr{Михайло Бойченко}
Обьічно забирают без спроса @igg{fbicon.wink} 

\iusr{Станислав Бочкур}
\textbf{Михайло Бойченко} Таким мы традиционно говорим "молон лабе"...

\iusr{Михайло Бойченко}
Ну вот, не хотел ханов вспоминать... А они думаете не хотят? @igg{fbicon.wink} 

\iusr{Станислав Бочкур}
\textbf{Михайло Бойченко} Увы, такую валюту, как "хотелка", можно обменять разве что на другую "хотелку", посмелее. Нам тоже много чего хочется...

\iusr{Михайло Бойченко}
Поживем - увидим, чья хотелка более живучая... @igg{fbicon.wink} 

\iusr{Матвей Кублицкий}
\textbf{Михайло Бойченко} вот как раз через 15 лет и надо смотреть. Украина перед серьезной проблемой в виде кредитов оказалась. проблема то с годами пройдет, но лет 10 не до развития будет. Тут как раз Рабинович прав - надо сделать так чтоб крымчане сами назад захотели

\iusr{Марина Прохорова}

Русские никогда не считали Крым украинским. Нелепый подарок безумного Хрущёва в
рамках единого Союза никто не воспринимал всерьёз. И даже замена СССР на СНГ
поначалу виделась чистой формальностью. Но потом взыграла нэзалэжнисть, теперь
ещё долго придётся расхлёбывать последствия ((

\iusr{Матвей Кублицкий}
\textbf{Игорь Лесев} нет. не соглашусь. Игорь. тут совпало. если б россия не забрала крым - там бы щас амеровская база была. они эти планы не скрывали

\iusr{Станислав Бочкур}
\textbf{Игорь Лесев} Там была поначалу призрачная надежда, что ваши резко включат заднюю и сделают вид, что переворота никакого не будет - так, встали с ног на голову, да и обратно с головы на ноги. Но - не включили, там уж и нам пришлось заходить по-настоящему.

\iusr{Матвей Кублицкий}
\textbf{Игорь Лесев} согласен со второй частю - КАК это происходило. но база НАТО там - реалные планы. просто щас представь себе там базу НАТО - этоже полный контроль всего черного моря. мечта всех натовцев во все времена. вспомни войну в крыме в середине 19 века...

\iusr{Игорь Лесев}
ну, плюс/минус тоже склоняюсь к такой версии

\iusr{Матвей Кублицкий}
\textbf{Станислав Бочкур} будте любезны по четче писать. из вашего текста куча непонятного

\iusr{Станислав Бочкур}
\textbf{Matvey Kublitsky} Чего именно непонятно?

\iusr{Игорь Лесев}
\textbf{Матвей Кублицкий} не преувеличивай значение Черного моря. Кроме того, Россия последовательно готовилась выходить из Севаса, а значит, и из Крыма. Новороссийск объявили главной базой ЧФ. На Кубани стали строить аналог крымской Нитки. Короче, ушли бы потихоньку, добазарились бы о нейтральном статусе ЮА и никакой трагедии. Но все все ускорили в 14-м

\iusr{Матвей Кублицкий}
\textbf{Игорь Лесев} ну не знай. четкости этих планов не было никогда. наши и давили на цены газа чтоб оставаться в крыму. новороссийск имеет кучу проблем для того чтоб стать базой. порт то ограниченный. а там и гражданский флот и военный. в общем находимся в зоне предположений и домыслов

\iusr{Игорь Лесев}
\textbf{Матвей Кублицкий} это были не планы, а проходила последовательная работа. Строились пирсы, углублялось дно, в Крыму сокращался личный состав, ЧФ не обновлялся, русские РЕАЛЬНО готовились к уходу из Крыма

\iusr{Матвей Кублицкий}
\textbf{Игорь Лесев} ты был в новороссийском порту? я был. он географически ограничен. половину моря выкапывать? опять же дело же не где будет российский флот. проблема - чтоб в крыму не было флота НАТО. Чисто стратегия. какой смысл от порта Новороссийска - если его можно раздолбать малой кровью из крыма

\iusr{Игорь Лесев}
\textbf{Матвей Кублицкий} так и от Севаса какой смысл, есть проливы контролирует НАТО?))

\iusr{Матвей Кублицкий}
опять же. разборки эти международные сколько продлятся то? ну пять лет, десять... потом кому разница будет чей крым? лишь бы отдыхалось комфортно

\iusr{Матвей Кублицкий}
проливы турки контролируют. по особому договору. там много что прописано

\iusr{Матвей Кублицкий}
кстати, менно из-за турок на украину никогда не придет сжиженный газ - вот и нато.

\iusr{Игорь Лесев}
сжиженный газ не придет не из-за турок, а из-за отсутствия бабла на него) это совсем разные вещи

\iusr{Станислав Бочкур}
\textbf{Игорь Лесев} видимо, бабла хватает лишь на купить ребят, которые расскажут, что газа нет из-за турок  @igg{fbicon.wink} 

\iusr{Матвей Кублицкий}
\textbf{Игорь Лесев} я про турецкий запрет на проезд газгольдеров по проливам. но и твоя мысль в тему. денег нет. с другой стороны - нафтогаз показал в прошлом году дикую прибыль в полтора лярда. хотя тут я могу и соврать....

\end{itemize} % }

\iusr{Елена Несветайлова}
Последнее предложение - молодец.)

\begin{itemize} % {
\iusr{Елена Несветайлова}
Так и знала, что так напишешь. То, что выше - обсуждаемо. А вывод = аксиома.

\iusr{Елена Несветайлова}
Импровизируй.) Смелее!)
\end{itemize} % }

\iusr{Евгений Отовчиц}
Вернем. Сперва Донбасс, потом Крым, потом Волынь и Закарпатье, Буковину вернем, Львовщину возвращать будем смело...

\iusr{Евгений Водовозов}

При сильном желании называть украинцев нацией идиотов любой повод хорош, как я
вижу. Не вернётся Крым? Нереально?

На наших глазах произошли более нереальные вещи - распад совка, например. Кто
бы мог подумать, что такое возможно?

Ну а примеров, подобных возврату аннексированному Крыму, можно привести массу.

Судетские немцы спаковали чемоданчики и мирно потопали в Германию.
Аннексированные Латвия, Литва и Эстония сейчас свободные страны. Если копнуть
глубже, то можно вспомнить и Донецко-Криворожскую, почившую в бозе, исполнив
свою функцию. Из того же времени можно вспомнить и Финляндию, ценой Карелии
обретших свободу.

Можно и немного натянутые аналогии - падение Берлинской стены. Чем ГДР не
аннексированная территория?

А тут гляди ещё Кореи объединятся - ваще из области фантастики.

Вроде норм набросал примеров? Только вот идиотов искать не хочется)

Таким образом дураком может оказаться не тот, кто думкою богатие)

\end{itemize} % }
