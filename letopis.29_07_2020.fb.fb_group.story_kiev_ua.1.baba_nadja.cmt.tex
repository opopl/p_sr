% vim: keymap=russian-jcukenwin
%%beginhead 
 
%%file 29_07_2020.fb.fb_group.story_kiev_ua.1.baba_nadja.cmt
%%parent 29_07_2020.fb.fb_group.story_kiev_ua.1.baba_nadja
 
%%url 
 
%%author_id 
%%date 
 
%%tags 
%%title 
 
%%endhead 
\zzSecCmt

\begin{itemize} % {
\iusr{Наталья Рыбкина}
злобно

\begin{itemize} % {
\iusr{Андрей Журавлев}
я бы это назвал скорее сарказмом.

\iusr{Андрей Шиян}
\textbf{Наталья Рыбкина} не согласен! Очень человечно и с юмором!

\iusr{Володимир Шуневич}
\textbf{Наталья Рыбкина} Вы правы. Я бы сказал резче - говнистенько как-то...

\iusr{Андрей Журавлев}
\textbf{Володимир Шуневич} огромное спасибо за ваш отзыв. Я серьезно. Кто то в комментах рассказ хвалит, а кто то ругает. И это нормально. Значит рассказ получился!)

\iusr{Наталья Рыбкина}
\textbf{Володимир Шуневич} Следующая публикация этого автора подтверждает Ваше наблюдение.

\iusr{Андрей Журавлев}

Человек без чувства юмора - это тяжёлый случай. Жизнь видать не сильно балует.
А насчёт юмора в рассказах - я заранее именно с ВАМИ согласен, что там его и
близко нет)))

\end{itemize} % }

\iusr{Олег Московко}

Школа \#66 возле Индустриального моста - тоже английский с первого класса, а в
десятом - до десяти уроков в неделю... те же \enquote{допросы военнопленных} и изучение
структуры американской армии, правда директора были \enquote{нормальными}...но доставало
по 6-7 человек в подгуппе на уроках английского, приходилось \enquote{готовиться}. Хотя
со знанием английского выше \enquote{Ландон - из зе кепитал...} оканчивали не многие. @igg{fbicon.smile} 

\begin{itemize} % {
\iusr{Андрей Журавлев}

я вам скажу так - подобные школы давали очень хорошую БАЗУ языка. Я закончил
школу 40 лет назад, по английскому у меня было 4. Язык мне понадобился в начале
2000-х (я - инженер). Буквально за месяц-другой практики я вспомнил почти все.
Я сам был очень удивлен. В итоге спикаю, пишу и т.д. очень даже гут.


\iusr{Олег Московко}
\textbf{Андрей Журавлев} 

ключевое - \enquote{понадобился}, я закончил в 77, нженерил, керувал... но английским
пользовался только дабы узнать сколько ликера надо для коктейля.


\iusr{Андрей Журавлев}
\textbf{Олег Московко} все правильно. Но без хорошей языковой базы я бы \enquote{учил} все по новой неизвестно сколько времени. А так хватило пару месяцев.
\end{itemize} % }

\iusr{Alla Alla}

Прочла с мыслью, что наша директриса была точный клон бабы-Цеткин, значит их
делали под копирку и садили на директорские места, чтобы школьное время
запомнилось, как ужас всей жизни @igg{fbicon.laugh.rolling.floor}{repeat=3} 

\begin{itemize} % {
\iusr{Володимир Шуневич}
\textbf{Alla Alla} 

Очень Вам соболезную. Для нас школьные годы были лучшим, самым счастливым
временем жизни. Обьездили с учителями и директором на школьном автобусе
полСоюза от Ленинграда до Сухуми и от Бреста до Волгограда. Спортом занимались
весьма успешно, учителя были интересные, кружки. Директор тоже был по-своему
оригинал, но все любили его как отца. И школа была одной из лучших в
Житомирской области... Многие киевские - хлевы по сравнению с ней. Кстати, у
нас английский преподавали со 2-го класса...

\iusr{Маргарита Мышанская}

Вспоминаю школу с содроганием. Бесконечное унижение личности детей и
подростков. Бесконечные вычитывания по любому поводу (как в анекдоте \enquote{почему в
шапке, почему без шапки?}), бестактность и наглость учителей. С облегчением
вздохнула, когда все закончилось. Институт (КПИ, ТЭФ) показался раем. Стараюсь не
думать и не вспоминать школу вообще... Знакомимся-школа №4, Соломенка. @igg{fbicon.face.pensive} 

\end{itemize} % }

\iusr{Ольга Даненко}

Все бы ничего, но украинский язык в исполнении такой личности как вы ее описали
звучит весьма неправдоподобно.

\begin{itemize} % {
\iusr{Олег Московко}
\textbf{Ольга Даненко} все \enquote{спецшколы} были україномовными.

\iusr{Hardashnyk Irene}
\textbf{Олег Московко} Неправда! 

Я окончила 32-ю, т. н. английску. ю базовую школу ин яза, на ул. Федорова, школа
была русская, а вот недалеко находящиеся 56-я и 130-я, действительно были
украинскими.

\iusr{Андрей Журавлев}
\textbf{Hardashnyk Irene} Все вы правильно говорите, но 32-ю вряд ли можно назвать спецшколой. Программа по инглиш там была хорошая, но менее насыщенная, чем, например, у нас в 87-й. У вас не было техн. перевода и военного перевода.

\iusr{Олег Московко}
\textbf{Hardashnyk Irene} исключения только подтверждают правила! @igg{fbicon.smile} 

\iusr{Ольга Даненко}
\textbf{Andriy Zhuravlyov} 

я не знаю сколько вам лет и настолько ли вы старый как я, но я прекрасно помню
отношение к украинской мове в советские годы и тем более в Киеве. Так что
извините, но я вам не верю.

\iusr{Ольга Сахарова}
\textbf{Ольга Даненко} 

Были украинские школы, были говорящие на украинском языке учителя, дети,
семьи... \enquote{Бабу Надю} помню плохо, но была свидетелем, как говорили между собой
другие учителя по-украински. Отношение было чаще к конкретным людям.

\iusr{Андрей Журавлев}
\textbf{Ольга Даненко} 

это ваше личное дело. Кстати, в комментах выше есть 2 человека, которые учились
в моей школе. Они почему то поверили. А я в этой школе отучился с 1969 по 1979
годы. все было именно так. Другое дело - отношение к укр языку в быту. тут я с
вами соглашусь.

\iusr{Андрей Журавлев}

Ольга, у нас была украинская школа и почти все учителя говорили по украински
даже на переменках. В том числе и баба Надя. Как они и она общались в быту - я
не знаю.

\iusr{Олег Московко}
56, 130, 87, 66, 80 ...судя по комментам - україномовні.

\begin{itemize} % {
\iusr{Андрей Журавлев}
\textbf{Олег Московко} 

56, 130, 87 - точно украинские. Про остальные не знаю. Вообще в центре (тогда -
Ленинский р-н) было 12 школ. В 70-х из них русских было только 3 или 4 (33, 21,
57 и вроде все). Когда я вернулся из армии в 1983, зашел в школу, общался с
учителями. Узнал, что из нашей 87-й и еще некоторых убрали английский язык
(начали учить как везде, с 5 класса) и несколько школ района перевели на
русский язык обучения. Украинских школ осталось меньшинство. Вот такая была
\enquote{русификация по тихому}.

\iusr{Ирина Петрова}
\textbf{Andriy Zhuravlyov} 

русскоязычные - 94-я, 86-я, 49-я, 90-я...это навскидку. А вот украинскую помню
из нашего района Ленинского только 117-ю на Энгельса. С директрисой, конечно,
не повезло... Нашу 94-ю Бог миловал, ни об одном из учителей не могу вспомнить
негативно)

\end{itemize} % }

\iusr{Белла Кушнир}
ошибаетесь такие личности говорили на шдиш и украинском

\iusr{Alik Perlov}
\textbf{Белла Кушнир} почему вы так решили??))

\iusr{Андрей Журавлев}

До 1980 в Ленинском районе укр школ было больше чем русских. Потом все
поменялось. Да и 49 и 90 школы - это другие районы


\iusr{Маргарита Мышанская}

На Соломенке - шк. №115 украинская с усиленным английским. Я возле нее жила. И мои
соседи в нее ходили. Там говорили по-украински. И это было в 70-х. Так что не надо
тра-ля-ля об отношении к украинскому языку при СССР. Просто у родителей был
выбор куда отдавать ребенка учится. А в школах с обучением на русском языке было
2 урока украинской мовы и 2 урока украинской литературы в неделю. Расказывать об
ущемлении украинского языка будите рассказывать молодому поколению (дурить их), а
нас не надо... Мы знаем все по своей жизни в Киеве.

\begin{itemize} % {
\iusr{Андрей Журавлев}
\textbf{Маргарита Мышанская} 

а вы знаете, что например, в 1980 году 3 или 4 киевские школы в тогдашнем
Ленинском районе, в котором всего было 12 школ, перевели на русский язык
обучения? Просто взяли - и перевели. Никого не спрашивая. Это в самом маленьком
районе города. И нормальную карьеру в СССР мог сделать только человек
русскоязычный. Украинские и другие языки считались второстепенными, для
\enquote{селюков}. Так что русификация была. Естественно, не только в Украине. Она в те
времена была уже даже не насильственная. Но целенаправленная.

\end{itemize} % }

\iusr{Маргарита Мышанская}

А Вы знаете, что в 90-х, подбирая для дочери школу в Святошинском районе я не
нашла ни одной с русскоязычным обучением. Еле нашла школу, где было 2 урока в
неделю русского. Это что? Мое мнение-должен быть выбор и карьера не должна
зависеть от мовы. Вы учились в КПИ. Кто-то кого-то притеснял за разговоры на
украинском языке? У нас в группе полно было ребят говоривших на украинском. Это
никак не влияло на их успеваемость и получение диплома. Хватит уже этого
всего. Бесконечная ложь и профанация. И, кстати, именно мои однокурсники из сел и
маленьких городков преуспевали в обучении и сейчас занимают высокие посты на
энергетических предприятиях.

\begin{itemize} % {
\iusr{Андрей Журавлев}
\textbf{Маргарита Мышанская} 

На тему языка можно спорить до бесконечности. И я бы наверное согласился с вами
лет 9-10 назад. Но в свете военной агрессии РФ против Украины считаю, что чем
быстрее наши дети и внуки забудут русский язык, тем лучше. Тем более есть
гораздо лучшая и перспективная альтернатива - английский. И да, в Украине один
государственный язык - украинский. Все. Точка.

\end{itemize} % }

\iusr{Маргарита Мышанская}
Можете начинать забывать сию же секунду уже! @igg{fbicon.laugh.rolling.floor}{repeat=4} 

\iusr{Маргарита Мышанская}
...И точка. @igg{fbicon.face.upside.down} 

\iusr{Маргарита Мышанская}

А тогда почему Вы свои опусы пишете на русском, а не на украинском или
английском? Как-то лицемерно получается! @igg{fbicon.face.tears.of.joy}{repeat=5} 

\begin{itemize} % {
\iusr{Андрей Журавлев}
\textbf{Маргарита Мышанская} 

все просто - родной язык для меня русский. Благодаря школе знаю украинский на
5+ и английский 4+. Мог бы и на этих языках написать. Но мне кажется, что
получилось бы хуже. Да и читателей было бы меньше. Не понимаю при чем тут
лицемерие. Да, я считаю что в Украине должны обязательно изучаться укр и англ
языки. За государственный счет. Остальные языки - только по желанию, в том
числе и русский.

\end{itemize} % }

\iusr{Маргарита Мышанская}

Вы пишете \enquote{чем скорее дети и внуки забудут русский язык-тем лучше}. Вы имеете
русскую фамилию и родной язык для Вас русский. Вы хотите, чтобы дети и внуки
предали свой род, своих предков и мыслили на английском или украинском? О чем
Вы? \enquote{Да и читателей было бы меньше..} А Вы задумайтесь почему их было бы меньше?

\begin{itemize} % {
\iusr{Андрей Журавлев}
\textbf{Маргарита Мышанская} 

так не о чем тут задумываться - понятно, что в Киеве пока еще из читающей
публики большинство читает на русском. Я не буду вдаваться в детали почему так
сложилось, но эту ситуацию надо менять. Естественно, не насильственно. А менять
примерно так, как русификация делалась в СССР в 80-е. Потихоньку. И при чем тут
язык предков? Мои родители - русскоязычные, родились в РФ. Но сознательно
отдали меня в укр школу. Потому что считали, что я должен знать укр язык. Тем
более, что вы прекрасно знаете, что в быту у нас можно говорить на любом языке.
Но государственным и обязательным для изучения должен быть один язык. А русский
язык... Его изучение в ближайшей перспективе не принесет практической пользы
нынешним первоклассникам. Так что лучше пусть учат английский.

\end{itemize} % }

\iusr{Маргарита Мышанская}

Страны с несколькими государственными языками: Канада-англ. и франц.,
Индия-англ. и хинди, Швейцария-немецкий, франц. итал, ретороманский(0. 1\%
населения), Финляндия-фин., шведский(шведы 5\% населения), Мальта-мальтийский,
англ., Кипр-греческий, турецкий, Люксембург-франц., немецкий, люксембургский
(франсик мозепан). Список можно продолжить... Откуда, из какой тьмы и
преисподней вырвались идеологи того, о чем Вы говорите? Каким же узколобым
мышлением надо обладать, чтобы поддерживать то, о чем Вы пишете?  @igg{fbicon.face.worried}  Мне жаль
людей, которые думают так как Вы... Убожество это все, ущербность

\begin{itemize} % {
\iusr{Андрей Журавлев}
\textbf{Маргарита Мышанская} 

узколобое мышление и ущербность - это как раз непонимание того, что наш сосед
РФ - это враг, желающий уничтожить Украину как государство. Как минимум, враг
на нашем отрезке времени. Язык - это как говориться \enquote{одно из}. Крепко в вас
засела мнимая советская \enquote{дружба народов}. Вы не обратили внимание, что
например, Франция почему то не отправляет свои войска в Канаду или Люксембург
для защиты франкоязычного населения? Или вы не в курсе, что в США живут даже,
наверное, марсиане и люди говорят там на миллионе разных языков, но
государственный язык почему то один.

\end{itemize} % }

\iusr{Маргарита Мышанская}

Отто фон Бисмарк: \enquote{Могущество России может быть подорвано только отделением от
нее Украины... необходимо не только оторвать, но и противопоставить Украину
России. Для этого нужно лишь найти и взрастить предателей среди элиты и с их
помощью изменить самосознание одной части великого народа до такой степени, что
он будет ненавидеть все русское, ненавидеть свой род, не осознавая этого. Все
остальное-дело времени...}

\begin{itemize} % {
\iusr{Андрей Журавлев}
\textbf{Маргарита Мышанская} 

вряд ли это сказал Бисмарк, но со смыслом цитаты я согласен. Только маленькая
поправка - почему надо взрастить \enquote{предателей}? Кого и что они предают?
Вас к Соловьеву, в Москву, на эфир не приглашали?)))

\end{itemize} % }

\iusr{Маргарита Мышанская}

А Вас на эфир \enquote{Великий Львів} не приглашали?
@igg{fbicon.face.grinning.squinting}  Если нет, то есть над чем работать....
Хотя нет, не пригласят. Фамилия москальская. С такой фамилией в новом
украинском мире Вам карьеры не видать. Как впрочем, и денег. Каким бы
\enquote{свідомим} Вы не бы @igg{fbicon.beaming.face.smiling.eyes}
@igg{fbicon.face.grinning.squinting}  @igg{fbicon.face.tears.of.joy} ли.

\begin{itemize} % {
\iusr{Андрей Журавлев}
\textbf{Маргарита Мышанская} 

карьеру я давно сделал, денег тоже хватает, так что все - мимо))) Тем более,
что я не сталкивался с дискриминацией по поводу фамилии или русского языка
(хотя свободно владею украинским). А вот \enquote{дискриминацию} по поводу отсутствия
мозгов наблюдал неоднократно. Так что все таки так называемый \enquote{укр национализм}
в московском понимании и в Киеве, и во Львове отсутствует. Украинцы -
здравомыслящий и практичный народ. если что, я имею ввиду не этнических
украинцев (к которым я не отношусь), а украинцев по гражданству (к которым я
без сомнения принадлежу). Есть такое понятие - полезные идиоты. Это те, которые
за все хорошее против всего плохого. Им \enquote{какаяразница} на каком языке говорить,
они считают что в Украине пылает гражданская война, они считают, что надо
оставить все как было до 2014 года... Только так, как до 2014 уже не будет.
Есть всего лишь 2 варианта: или Украина отстоит свою независимость, или сюда
придет Россия. И тогда они не спросят, на каком языке вы говорите. Вы и вам
подобные, наравне с такими, как я, будете для них всего лишь \enquote{продажными
хохлами}. Вот тогда вы и объясните им то, что впариваете тут мне.

\end{itemize} % }

\iusr{Маргарита Мышанская}

Конечно! Так как до 2014 года уже не будет! Потому что совершен кровавый
госпереворот, от которого выиграли только олигархи ( многократно увеличили свои
состояния), а \enquote{здравомыслящий и практичный народ} стал пятикратно
беднее, еще и кровь пролил за толстосумов, которые обрели власть в результате
кровавой бойни в центре Киева, и продолжает этот народ детей своих посылать на
бойню...

\begin{itemize} % {
\iusr{Андрей Журавлев}
\textbf{Маргарита Мышанская} с цифрами в руках очень просто опровергнуть ваш бред. Но скорее всего дискутировать с вами бесполезно. И очень плохо, что такие как вы имеют право голоса. И гореть вам в аду за ваши мысли и слова о своих соотечественниках.
\end{itemize} % }

\iusr{Маргарита Мышанская}

Я никогда не голосую-не за кого... Я всегда работаю в комиссиях-секретарем. А
насчет ада -не горячитесь. Пути Господни неисповедимы... @igg{fbicon.smile} 

\iusr{Андрей Журавлев}

я не горячусь, я просто назвал вещи своими именами. Человек, ненавидящий свою
родину добром не кончит.

\iusr{Маргарита Мышанская}

Дело не в моей любви к Родине, дело в оценке того, что происходит на моей
Родине... А происходят события, которые отбрасывают ее развития на десятилетия
назад. И маячит весьма реально колониальное будущее..

\iusr{Маргарита Мышанская}

Как говорится: \enquote{Не рой яму другому-сам в нее попадешь!} За сим
разрешите откланяться...

\end{itemize} % }

\iusr{Раиса Карчевская}

Мне очень понравился ваш рассказ с юмором и очень правдивый. Я проходила
практику английского языка в 32 школе рядом с институтом иностранных. языков, в
котором я училась и мне было очень комфортно с детьми. Школа была очень
сильная, английский учили с 1го класса, а в группе было 10 человек, но дети были
хорошего уровня. Англичанка в классе была замечательная и она всегда оставляла
меня студентку'-практикантку абсолютно спокойно с детьми. Мне очень нравилась
практика и я 2 года подходила в одном и том же классе. Я переводчик, но учу
своих родственников и их детей, и друзей английскому

\begin{itemize} % {
\iusr{Андрей Журавлев}

32 школу отлично знаю, там много друзей детства училось. Я живу там рядом. Но в
32-й программа была пожиже нашей 87-й. Там не было техперевода и военки.


\iusr{Раиса Карчевская}
\textbf{Андрей Журавлев} Да там не было тех.перевода и военки

\iusr{Hardashnyk Irene}

Да, тех перевода и военки у нас не было, но были очень сильные учителя. Я без
репетиторов и блата в 77-м поступила в ин яз после 32-й. У нас каждый год туда
поступали по 2-3 выпускника.

\iusr{Алексей Усенко}
\textbf{Раиса Карчевская} 

а дериктриса - гоффно..... меня не взяли в 4 -й клас именно из-за не желания
моих родителей дать взятку... мой отец сделал глупость ...приехал на \enquote{Волге}..

\iusr{Алексей Усенко}

Меня отдали в самую \enquote{хулиганскую} школу микрорайона... 37..... о чем я ни
сколько не жалею.... единственная школа где директор был Мужик... Николай
Александрович Витьман.. Да Бог ему - здоровья!!

\iusr{Андрей Журавлев}
\textbf{Алексей Усенко} знаю 37-ю. У меня сын там учился. Директор тоже был Витьман.

\iusr{Алексей Усенко}
\textbf{Андрей Журавлев} это он и есть)
Преподаватель математики..

\iusr{Андрей Журавлев}
\textbf{Алексей Усенко} да, он самый
\end{itemize} % }

\iusr{Violeta Tetiana Stetsenko}

Я проучилась в 87 школе с 1961 по 1971 год. Это на моей памяти Надія Сергіївна
стала директриссой и именно наш класс дал ей прозвще Баба Надя. Она преподавала
у нас Обществоведение, но тогда она не была такой вредной. А про курцов в
туалете. это точно, дверь нараспашку \enquote{Ану, виходь!} Мы её абсолютно не боялись,
относились с юмором. И Ивана Захаровича прекрасно помню, тогда еще заливали
каток на спортивной площадке, а онк по радио из спортзала руководил секцией
фигурного катания. Типа \enquote{Ира, Наташа, еще разок повторить поворот} и при этом
было слышно булькание разливаемого в стаканы.

\begin{itemize} % {
\iusr{Андрей Журавлев}
да, насчет бульканья в стаканы Захарыч был чемпионом)))

\iusr{Natasha Levitskaya}
\textbf{Violeta Topy}
Вот другой взгляд на одну и ту же школу и на одного и того же человека!
\end{itemize} % }

\iusr{Наталя Піпаш}
Згадав....

\iusr{Ольга Морева}

Мда, такую страну пооср@ли, с мороженым, с карательной педагогикой, с линейками
и политинформациями. Вот сразу тоской повеяло от воспоминаний о
пионэрско-комсомольском детстве. Теперь кошмары будут сниться.

\begin{itemize} % {
\iusr{Ольга Сахарова}
Феномен \enquote{бабы Нади} очень популярен и ныне. Просто несколько слов в лозунгах поменяли...

\iusr{Maxim Kostenko}
\textbf{Ольга Морева} нууу... не знаю...
Сейчас, наверное, процветаем!)))

\begin{itemize} % {
\iusr{Андрей Журавлев}
\textbf{Maxim Kostenko} так ведь сейчас у каждого есть выбор - процветать или нет. Остальное - от лукавого...

\iusr{Maxim Kostenko}
\textbf{Andriy Zhuravlyov} ну да, ну да...
Самая бедная страна в Европе...
Видать выбрать никак не могут!

\iusr{Ольга Морева}
\textbf{Maxim Kostenko} кто именно выбрать не может?

\iusr{Андрей Журавлев}
\textbf{Maxim Kostenko} ну вы может и не можете выбрать. Я лично свой выбор еще в конце 80-х сделал.

\iusr{Maxim Kostenko}
\textbf{Ольга Морева} украинский народ!

\iusr{Maxim Kostenko}
\textbf{Andriy Zhuravlyov} )))
В Украине больше 40М жителей!!!

\iusr{Андрей Журавлев}
\textbf{Maxim Kostenko} и что из этого следует? То что 40 млн человек могут сделать свой выбор. Выбор как им жить. Но кто то этот выбор может и не делать, а плыть по течению и считать, что ему все должны)

\iusr{Maxim Kostenko}
\textbf{Andriy Zhuravlyov} вот видите, не все жк украинцы такие гении чедлвечесьва как Вы!)))

\iusr{Julia Ablamska}
\textbf{Maxim Kostenko} вы не ощущаете разницы между \enquote{бедная страна Европы} И \enquote{республика в Совке}? Если так, примите мои соболезнования

\iusr{Maxim Kostenko}
\textbf{Julia Ablamska} ну как сказать...
ВВП УРСР был где-то 20ке мира.

\iusr{Julia Ablamska}
\textbf{Maxim Kostenko} по мнению советской статистики? :)) ну-ну

\iusr{Андрей Журавлев}
\textbf{Maxim Kostenko} конечно не все. Особенно безграмотные с потугами на чувство юмора

\iusr{Андрей Журавлев}
\textbf{Julia Ablamska} не, ну танков на заводе Малышева точно больше почти всех делали. За исключением танковых заводов в РСФСР)))

\iusr{Maxim Kostenko}
\textbf{Julia Ablamska} нет, конечно!

Мировой банк для Вас авторитет???

http://en.classora.com/.../ranking-of-the-worlds-richest...

Украина - на 30м месте...
Где сейчас напомнить???

\iusr{Maxim Kostenko}
\textbf{Andriy Zhuravlyov} та вот... все 40 миллионов почти!

\iusr{Maxim Kostenko}
\textbf{Andriy Zhuravlyov} да не только танков...
Хотя, конечно, определенный дефецит был.

\end{itemize} % }

\iusr{Aleksandr Vasyliev}
\textbf{Ольга Морева} , сочувствую

\iusr{Андрей Журавлев}

В 1989 возможно Украина и была в тридцатке. По валовому продукту. Если оценить
реальную стоимость выпускаемого то показатели сильно рухнут вниз. Пример -
автор Запорожец ценою в районе 3000 рублей. По искусственному курсу 1 долл=65
копеек. Это советская цена. Реальная цена такого автор долларов 150 по
реальному курсу 1 долл=2 рубля. И так по всем видам продукции

\begin{itemize} % {
\iusr{Maxim Kostenko}
\textbf{Andriy Zhuravlyov} Чего????
Это Вы сами посчитали?)))

\iusr{Андрей Журавлев}
\textbf{Maxim Kostenko} любой человек, читающий в советское время газету Известия с курсом валют Госбанка СССР, и при этом знающий курс тогдашнего черного рынка, может это посчитать с достаточной степенью точности. Ну а госцена Запорожца никогда секретом не была.
\end{itemize} % }

\end{itemize} % }

\iusr{Светлана Алексеееко}
\textbf{Иван Захарович} был настоящий физрук. Он старался для детей-каток-был лучший !А школа была украинская.

\iusr{Татиана Корнейко}

Моя дочка училась в этой школе! Сейчас там самый лучший директор - Любовь
Ивановна! Каждого ученика знает по имени!

\begin{itemize} % {
\iusr{Андрей Журавлев}
к сожалению, с 1980 года, наша школа уже не английская(((

\iusr{Елена Царовская}
\textbf{Татиана Корнейко} 

Категорически не согласна. Знать имена учеников в маленькой школе- не велика
заслуга. А вот то что при ней школа совсем \enquote{скатилась} - это точно. Держалась
наша 87 только благодаря стараниям Наталии Витальевны Галкиной. Она ещё как-то
традиции держалась, но, увы, в минувшем году и она ушла

\end{itemize} % }

\iusr{Эльвира Земблевич}

Я училась в 30 школе с \enquote{преподаванием ряда предметов на английском языке} в
60-70 годы и даже обидно, ничего отвратительного об учителях припомнить не
могу. Была противная вахтерша баба Вера. Учителя физики Соломона Борисовича я
разыскала через ФБ в Израиле и на встрече одноклассников устроили видио
встречу. До слез. Школа дала все и инглиш, и физику, и математику, и
литературу, и спорт. Лучшее время!

\begin{itemize} % {
\iusr{Андрей Журавлев}
вы считаете, что я написал об учителях что то отвратительное? И в мыслях такого не было.
\end{itemize} % }

\end{itemize} % }
