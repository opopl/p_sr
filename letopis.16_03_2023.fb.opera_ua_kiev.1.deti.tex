%%beginhead 
 
%%file 16_03_2023.fb.opera_ua_kiev.1.deti
%%parent 16_03_2023
 
%%url https://www.facebook.com/National.opera.of.Ukraine/posts/pfbid0298k4ZMuG2GhgQQCZzt2a1r83H1KJ5A4pduLNPwBbhA3DM3TEHUbxwqz2NHbsPf54l
 
%%author_id opera_ua_kiev
%%date 16_03_2023
 
%%tags 
%%title ДЕТИ
 
%%endhead 

\subsection{ДЕТИ}
\label{sec:16_03_2023.fb.opera_ua_kiev.1.deti}

\Purl{https://www.facebook.com/National.opera.of.Ukraine/posts/pfbid0298k4ZMuG2GhgQQCZzt2a1r83H1KJ5A4pduLNPwBbhA3DM3TEHUbxwqz2NHbsPf54l}
\ifcmt
 author_begin
   author_id opera_ua_kiev
 author_end
\fi

Рік тому, 16 березня, російські військові скинули дві надпотужні авіабомби на
драматичний театр у Маріуполі.

На той час його підвали слугували сховищем для понад тисячі цивільних з дітьми.

Аби вціліти, на площі перед театром люди лишили величезний надпис зрозумілою
окупантам мовою - ДЕТИ. 

Але покидьків це не зупинило...

Сьогодні, у роковини трагедії, українці у різних містах влаштували спомини тих,
хто загинув у руїнах.

Ось так сьогодні ввечері, 16 березня, виглядала площа перед Національна опера
України імені Т. Г. Шевченка

Учасники акції виклали слово \enquote{ДЕТИ}, утім читати його треба \enquote{ДЕ ТИ?}. Адже це
те саме запитання, котре рідні та друзі безвісті зниклих у Маріуполі і досі
ставлять собі.

Кількість загиблих під уламками Маріупольського драматичного театру і досі
достеменно невідома...

Світлини Сергія Макарова.
