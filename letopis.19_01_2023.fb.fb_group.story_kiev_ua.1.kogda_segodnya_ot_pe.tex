%%beginhead 
 
%%file 19_01_2023.fb.fb_group.story_kiev_ua.1.kogda_segodnya_ot_pe
%%parent 19_01_2023
 
%%url https://www.facebook.com/groups/story.kiev.ua/posts/2119273108269479
 
%%author_id fb_group.story_kiev_ua,chudnovskij_roman.kiev
%%date 19_01_2023
 
%%tags kiev
%%title Когда сегодня от печальных новостей становится совсем безрадостно, иногда спасает творчество
 
%%endhead 

\subsection{Когда сегодня от печальных новостей становится совсем безрадостно, иногда спасает творчество}
\label{sec:19_01_2023.fb.fb_group.story_kiev_ua.1.kogda_segodnya_ot_pe}
 
\Purl{https://www.facebook.com/groups/story.kiev.ua/posts/2119273108269479}
\ifcmt
 author_begin
   author_id fb_group.story_kiev_ua,chudnovskij_roman.kiev
 author_end
\fi

Когда сегодня от печальных новостей становится совсем безрадостно, иногда
спасает творчество. Взял, например, только мобильный - и идешь в милый с
детства уголок родного города \enquote{останавливать} прекрасные мгновенья... 

Сегодня с утра выглянуло солнышко, а потом Киев окутал туман. Значит, много
акварельных видов можно найти. Наверно, красиво сейчас, думаю, на Андреевском,
- не пробежаться ли по нему до работы?..

Грустна сегодня \enquote{самая киевская} улица: и война в стране, и погода ни
рыба ни мясо: ни снежинки - в январе! Только листвы "почерневшая кровь" на
склонах холмов да голые кроны, простёршие окоченевшие пальцы веток к небесам...
И на тебе: ещё и сирена воздушной тревоги завыла где-то неподалеку!..

Но азарт художника ещё жив в тебе - щёлкаешь, почти "на автомате", виды, уже
запечатлённые тыщи раз ранее, пытаясь увидеть их по-новому... 

Взобравшись на одну из  лестниц, видишь трогательный ракурс с крышами
Андреевского и замком Ричарда в дымке, и только хочешь его снять, как тут
голубь садится на нижнюю ступеньку. Далековато, пожалуй, для моей композиции...
Просишь его: ну поднимись, будь человеком, на несколько ступенек ближе ко
мне!.. И голубок понимает - как это важно для искусства: прыг-скок - и он
передо мной на переднем плане!.. Кадр сделан, а у меня в кармане  очень кстати
остаток горбушки завалялся... 

Ставлю точку в моей очередной фотосессии любимой улицы и бегу, несмотря на
вселенскую грусть вокруг, на работу.
