% vim: keymap=russian-jcukenwin
%%beginhead 
 
%%file 30_12_2020.fb.bilchenko_evgenia.1.otchet_poezia_god
%%parent 30_12_2020
 
%%url https://www.facebook.com/yevzhik/posts/3517843951584023
 
%%author Бильченко, Евгения
%%author_id bilchenko_evgenia
%%author_url 
 
%%tags bilchenko_evgenia,chelovek,facebook,god,otchet,poezia,ukraina
%%title БЖ. Поэтический отчет за год
 
%%endhead 
 
\subsection{БЖ. Поэтический отчет за год}
\label{sec:30_12_2020.fb.bilchenko_evgenia.1.otchet_poezia_god}
\Purl{https://www.facebook.com/yevzhik/posts/3517843951584023}
\ifcmt
 author_begin
   author_id bilchenko_evgenia
 author_end
\fi

БЖ. Поэтический отчет за год.

Вот сейчас и у меня начнется - вывешивание фоток с елочками и подведением
итогов. Итак, плюсы и минусы. Во время чумы моя гастрольная деятельность была
полностью парализована. О выступлениях в Украине еще до пандемии не могло быть
и речи в силу цензуры: я привыкла к большим легитимным залам, страдала
сценической избалованностью, а после моей маргинализации андеграундные кружки
испуганных русскоязычных украинских литераторов \enquote{не той} позиции, в которых
половина из десяти человек уже не читает \enquote{Буря мглою}, - перестали меня
вдохновлять. 

\ifcmt
  pic https://scontent-lga3-2.xx.fbcdn.net/v/t1.6435-9/s1080x2048/134091691_3517843811584037_5447677902732354807_n.jpg?_nc_cat=100&ccb=1-3&_nc_sid=8bfeb9&_nc_ohc=wW9lY9DXow4AX8vwGyc&_nc_ht=scontent-lga3-2.xx&tp=7&oh=089ac4aef633abd3555c57450e729b88&oe=60CB5AE6
\fi

Я перешла на международные поездки, где у меня чаще всего - всё
хорошо/комфортно, а также на поездки в Россию, где мне - всегда хорошо/духовно
даже в самых тяжелых условиях с адскими тратами и потерями. Начала предпочитать
не тусовки, а сольники по Украине, куда может приходить не местная, прости
Господи, \enquote{элита}, а нормальный народ, который - мне... и которому - я...

Пандемия позволила: провести 2 сольника в Киеве и 2 сольника в Одессе. В их
рамках была презентована антология Победы. Мой первый проект: #Ценасирени,
антология 9 мая, посвящена нашим дедам и прадедам, включает 7 стран. Раздается
бесплатно. Сделана на личные деньги. Второй - не мой - литпроект, участием в
котором я горжусь: \enquote{Возврату не подлежит}, антология о коронавирусе, продюсер
Стефания Данилова. Презентован продюсером в Москве. Среди многих прекрасных
стихов прозвучали и скромные мои.

Принято участие в 2 международных фестивалях онлайн: проект памяти Победы 9 мая
(Франкфурт-на-Майне) и питерский проект \enquote{МОСТЫ}. Из отечественного продукта,
который я \enquote{типа} завоевала в депрессивном полузабытье: на своем маршевом
кубиковом стенд-ап старье типа \enquote{Зубов поэтессы} (не могу уже выговаривать)
выиграла, разделив победу с другим поэтом, слэм на фесте \enquote{Интерреальность}
(новое мое хавает или народ очень молодой, или очень пожилой, или очень
военный, или очень сетевой, или очень русский, или очень языческий, или очень
православный: разброс - непостижим, это - тысячи людей, которые мне важнее
мнения украинских русскоязычных поэтов, застрявших на раннем Шевчуке и позднем
Макаревиче: короче, это все - мы, родня, от Луцка до Сыктывкара).

Напечатаны 2 сборника: памяти деда \enquote{Море на парапете} (Киев) и \enquote{Беседка: стихи
о Питере и его людях} (Таганрог). За \enquote{Беседку} спасибо Стефании Даниловой: она
как продюсер вложила ваши и свои финансы в меня и сама занимается теперь
продажей, гонорары я отчуждаю в ее пользу - я умею быть благодарной. В Украине
я давно публикуюсь за свои деньги: несмотря, родня, на наши с вами продажи я
считаюсь диссидентом, а вклад во \enquote{врага народа} - не в коня корм.

... 2 концерта в Одессе, 2 вечера в Киеве, 2 антологии, 2 сборника и 2 ценных
для меня фестиваля: 2-2-2-2-2. Остальное не вспомню, но могло быть и хуже...

PS. А! Опубликована в сетевых журналах: \enquote{Твоя глава} (Москва) и
\enquote{Перископ} (Волгоград). Вообще, что-то архизначимое происходит между
Волгоградом Великим и крохотной мной, но об этом - в другом посте. 

\begin{verbatim}
	#любите #взмываетвнебо #замоимзаокном #непобежденнаястрана
\end{verbatim}

\emph{Роман Поломодов}

Женя по хорошему уже сейчас надо называть в честь тебя улицы и переулоки в
Киеве и особенно в Москве..

Мне кажется в этом стремительно летящим мире в задницу дьявола.. мы не
увидемся.. Но я бы с удовольствием погулял бы по улице Евгения Бильченко.. на
маяковской рядом с музеем Булгакова. ещё одного гениального киевлянина.

Но наверное первой улице с твоим именем будет все таки Питер, город поэтов и
философов, твой город.

Но в пантеоне русских поэтов и философов, ты уже вошла в историю..  Живая
русская наша классика!!

\emph{Евгения Бильченко}

\textbf{Роман Поломодов} Это пока живая. Больно нужны в Москве улицы с моим
именем: это обидит непростивших. Да я и сама была бы первая против.

\emph{Роман Поломодов}

\textbf{Евгения Бильченко}

Для меня ты уже классика, живая или не живая, ты уже история нашей жизни.

Я человек который воплощает фантазии человеческих чувств, в камень наших
городов, готовься к тому, что тебя будут изучать последущие поколение
литераторов и романтиков, свою жизнь ты прожила превосходна уже сейчас. Дальше
ты должна по идеи наслаждаться плодами своих трудов. Заслужила!

\emph{Евгения Бильченко}
\textbf{Роман Поломодов} Служу народу.

\emph{Роман Поломодов}
\textbf{Евгения Бильченко}
Трудовому народу милая, трудовому!))

\emph{Евгения Бильченко}

Роман Поломодов Трудовому, русскому и украинскому, одному. А какому ж еще?

\emph{Евгения Бильченко}

\textbf{Роман Поломодов} Трудовому, русскому и украинскому, одному. А какому ж еще?

\emph{Роман Поломодов}

\textbf{Евгения Бильченко}
Служу советскому Союзу!))

Представь в параллельном мире, союз не распался, но преобразился.. и сразу же
поднялся на несколько уровней во вселенной..

Я давно понял, почему в нашем мире постоянно несправедливость и замечательные
люди типа тебя постоянно страдают недостатками здоровья

Если бы все было хорошо. Этот мир бы просто пропал с радаров, тех кто здесь нас
контролирует.

К сожалению правила игры требуют постоянного уничтожения добра и счастья
здесь.. Баланс соблюдается явно не нами.

Мы всего лишь сканируем действительность.. и охрениваем

Но это маленькая матрица..

Просто песчинка того, что мы можем увидеть. И походу увидеть это.. придется..

\emph{Евгения Бильченко}
\textbf{Роман Поломодов} В этом есть смысл.

\emph{Мирослава Александровна Бердник}

\textbf{Роман Поломодов}, лучше улицы в честь деда Евгения Бильченко и живая-здоровая Бильченко

\emph{Апалькова Ирина}

Женя, успехов и здоровья. Я счастлива, что была на твоем сольнике, надеюсь, что
в Новом году мое желание исполнится снова услышать твои твореня.

\emph{Татьяна Лозина}

Женя, ты молодчина!!!! Будь счастлива в Новом году!!! И здорова!!!

\emph{Olga Koltsova}

Женя, у Вас - дивные стихи, а Ваша культурология - заставляет извилины
вращаться. Здоровья Вам - и всяческого успеха! И - спасибо!

\emph{Славко Артеменко}

\enquote{От её ума с ума сходят} Цагарели
