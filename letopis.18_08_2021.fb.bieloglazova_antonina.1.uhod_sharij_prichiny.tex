% vim: keymap=russian-jcukenwin
%%beginhead 
 
%%file 18_08_2021.fb.bieloglazova_antonina.1.uhod_sharij_prichiny
%%parent 18_08_2021
 
%%url https://www.facebook.com/antonina.bieloglazova/posts/5061313170563574
 
%%author Белоглазова, Антонина
%%author_id bieloglazova_antonina
%%author_url 
 
%%tags beloglazova_antonina.kiev.stranaua,partia_sharija,sharij_analotij
%%title О причинах моего ухода от Анатолия Шария
 
%%endhead 
 
\subsection{О причинах моего ухода от Анатолия Шария}
\label{sec:18_08_2021.fb.bieloglazova_antonina.1.uhod_sharij_prichiny}
 
\Purl{https://www.facebook.com/antonina.bieloglazova/posts/5061313170563574}
\ifcmt
 author_begin
   author_id bieloglazova_antonina
 author_end
\fi

Думаю, что пришло время рассказать о причинах моего ухода от Анатолия Шария. 

Потому что то, что происходит у них по конфликтам внутри партии - это
закономерность, которая следует от их изначального отношения к своим же людям. 

Сразу скажу: я уходила без каких-то гневных постов и комментариев. Я ничего не
говорила даже в ответ на выпады Шариёв в мой адрес. Но теперь, видя как моих
друзей поливают грязью, считаю необходимым рассказать. 

Я не знаю, кому на самом деле в голову пришла идея создать партию, но лично я
считаю, что это была ошибка. Партией должны управлять люди эмоционально
уравновешенные. Иначе получится то, что и получается. Мне бы очень хотелось
чтобы Ольга и Анатолий были именно такими людьми. Но мы видим постоянные войны,
розжиг и ненависть. 

И это происходит не только вовне, но и внутри партии. Люди слово боятся
сказать, потому что их сразу уволят. 

Многие помнят, что меня отстранили от руководства партией после известного
видео из Москвы про Крым. Потом я ещё пару месяцев проработала журналистом,
параллельно закрывая вопрос с передачей управления партией на другого человека. 

И уволилась в мае 2020 года. Никто меня не увольнял. Я ушла сама и написала об
этом мирный пост. Анатолий в свою очередь тоже написал пост о том, что я ухожу
и что это было моим личным решением. 

Почему я ушла? 

Много всего накопилось во время моей работы с ними, я морально сильно устала.
Устала от их отношения к людям.

Помните как меня похищали СБУшники? Как на меня нападали С14? Как на меня
нападали представители братства Корчинского? Всеми этими делами фактически
занималась я сама.  

Как сказал Никита Роженко «а где мои адвокаты?», вот и у меня такой же вопрос
«где мои адвокаты?». Почему мне не помогали с моими делами? 

Почему я сама должна была бегать в полицию, по прокурорам, судам, пытаясь дать
ход моим делам? Я была сама себе адвокат по сути, который пытался добиться хоть
какого-то результата. Но мои внутренние ресурсы на такие походы не безграничны,
особенно когда тебе дают понять, что мол «что ты там бегаешь, иди лучше
поснимай что-то». 

И это только один пример.

Но была и последняя капля. Ею стала ситуация с тем знаменитым уже видео про
Вуйчича. Ещё до того, как видео с Вуйчичем увидели подписчики Анатолия, ко мне
обратилась Карина Богачева из СММ партии и сказала о том, что видео с Вуйчичем
хотят выдать в блог Анатолия с манипуляцией (нарезав нужные моменты).

Разумеется, я написала об этом Ольге, чтобы предостеречь о том, что отдел медиа
что-то там нахимичил с Вуйчичем и что это нельзя выдавать, потому что это ложь
и манипуляция. 

Думаете, мне сказали «спасибо Тоня, что предупредила»? Не угадали. На меня
обрушился гнев Оли мол, это меня не касается и т.д. 

Исходник с Вуйчичем у меня есть и я понимаю, почему видео порезали. Потому что
там были фразы как, например, "I don't know who are you and I've not seen your
blogs personaly". 

А надо же было, получается, показать что Вуйчич чуть ли не великий наш друг. Но
это же не правда. 

Это ложь. 

Меня тогда еще это очень сильно поразило, что они вообще не боятся врать. Не
боятся, что их разоблачат. Потому что убеждены, что их сторонники поверят во
все, что они скажут. А на тех, кто не поверит, можно еще и хайп поднять.

А Карину которая, как я, хотела тогда помочь и предостеречь, попытались уволить
за то, что она вообще ко мне обратилась. Но тогда мне и Ване Мамчуру как-то
удалось её отстоять. 

Это все стало для меня последней каплей. Я ушла. Тихо и мирно. 

Но сразу после моего ухода я начала замечать странные комментарии под постами
от Ольги и Анатолия, что мол я херовый сотрудник и вообще говорить не умею. 

Потом в блоге Анатолия я услышала упоминание обо мне что мол я ему вообще не
интересна и чем я сейчас занимаюсь ему тоже не интересно. Ну не интересно и
ладно. Но зачем ты так отзываешься обо мне в своём блоге? 

Чем я заслужила такое отношение с вашей стороны? Я когда уходила, ни одного
слова плохого нигде не написала о вас и о работе с вами. 

Но это еще не все.

Я когда уходила, у меня уже было место (одно СМИ, но не «Страна»), куда я
собиралась устраиваться на работу и там у меня уже были договорённости. Я
попросила только об одном, чтоб мне сначала дали время отдохнуть и настроится
после прошлого места работы. Мне сказали «Да, ок. Отдыхай, набирайся сил.
Ждём». 

И вот в июле месяце внезапно от меня начинает морозиться руководство этого издания. 

Я весь июль пыталась понять, что происходит, а оказывается, что и тут без
Анатолия не обошлось. Я об этом узнала от троих разных людей. Они сказали, что
туда был звонок от Шария, что в дальнейшем и повлияло на отказ мне в новой
работе. Это что за отношение такое? Или после ухода с Шарий.нет мне больше
нельзя нигде работать? 

Где-то в начале подготовки к местным выборам я узнаю, что в партии Шария было
распоряжение со мной не общаться и не пересекаться. Только с личного их
согласия. 

И началось нечто удивительное. Тата (лидер днепровской ячейки ППШ), которую я
считала своей подругой, вдруг начала записывать наш разговор на диктофон при
нашей с ней дружеской встрече и затем ещё и обвинила меня в работе на СБУ, ОП и
кого-то там ещё. 

Потом я спросила у неё тогда «Тата, зачем ты это делаешь?». Она со слезами на
глазах ответила «у меня нет выбора, меня они вынудили». 

Я уже не знаю, где здесь правда, а где ложь. Вынуждали её или нет. 

Но правда в том, что партия превратилась в место, где все друг друга предают,
тайно пишут, ставят прослушки. 

Ну и конечно же подставы. 

В августе на меня вышел один из моих знакомых, который помогал Олегу Зангиеву
(он же Олег Малежа) по вопросам безопасности в Партии Шария. Он сказал, что
ушел уже оттуда. Мы некоторое время общались друг с другом. 

И в какой-то момент он мне внезапно предложил слить за деньги каким-то людям
информацию о партии. Мол, он же знает как я отношусь к Анатолию и Ольге. 

Я не стала участвовать в этой истории. 

И тогда он взял в оборот с тем же предложением моего друга Колю Гладенького,
который тоже ушел из партии. Коля даже встречался с ним. Но сказал мне потом,
что, по его мнению, это подосланный провокатор. 

Позже мы получили подтверждение этой информации. А Шарий записал блог, что
якобы Гладенький продал властям информацию о партии. Еще и на меня там намекал,
не называя фамилии. Хотя это от начала и до конца была подстава. 

Очень много подобных ситуаций было. Как, например, когда я поехала в Мариуполь
делать опрос и там меня позвали на свадьбу партийного СММщика. И ему потом не
выплатили месячную зарплату за фотки со мной. Он теперь мне в шутку говорит,
что «это были самые дорогие фотографии в его жизни». Поэтому я вполне верю
рассказу Никиты Роженко, что и Сироту могли потребовать уволить за фотку с
Нагаткиным. 

Долгое время я ничего не писала об этом и где-то глубоко-глубоко в душе
надеялась, что всё наладиться и разум восторжествует. 

Но столько грязи льётся с их стороны на моих коллег и друзей, что это уже край.
Никаких стопов у них нет. 

Цель моего поста призвать к разуму всех, кто привык слепо кому-то верить. Не
создавайте себе кумиров. 

И напоследок – исходник видео Ника Вуйчича

\ii{18_08_2021.fb.bieloglazova_antonina.1.uhod_sharij_prichiny.cmt}
