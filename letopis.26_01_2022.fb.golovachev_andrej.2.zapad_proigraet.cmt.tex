% vim: keymap=russian-jcukenwin
%%beginhead 
 
%%file 26_01_2022.fb.golovachev_andrej.2.zapad_proigraet.cmt
%%parent 26_01_2022.fb.golovachev_andrej.2.zapad_proigraet
 
%%url 
 
%%author_id 
%%date 
 
%%tags 
%%title 
 
%%endhead 
\zzSecCmt

\begin{itemize} % {
\iusr{Serhiy Dovbnya}

Да, я тоже об этом давно думал. \enquote{Героїчна козацька доба} - это и есть
культурный код. Это приговор. Пока мы будем гордиться, что нашими истоками
является сообщество разбойников и рекетиров (чем по сути и было запорожское
казачество; оно даже принципиально отличается этим от Донского), до тех пор на
будущем можно ставить жирный крест. Украинское общество мало того что
православное - оно крайне ксенофобное, чем сильно отличается от других
православных народов: оно выковыряло евреев и поляков - до 1940-х они
составляли более 1/3 населения запада Украины. Фактически, шанс построения другой
Украины был потерян уже тогда

\begin{itemize} % {
\iusr{Нина Постой}
\textbf{Serhiy Dovbnya} кому що- а курці просо...

\iusr{Serhiy Dovbnya}
Армія, мова, віра! Ура!!!
\end{itemize} % }

\iusr{Виктор Марченко}
Я называю это явление генетическим кодом (или генетической памятью).

\iusr{Слава Марков}

Многое не сходится. Междоусобицы между славянскими княжествами были
исключительно кровопролитными. Турция со дня основания была светским
государством ( в отличии от Османской империи), а православный Константинополь
сгинул в лета и даже через три поколения не вернулся


\iusr{Даша Маркова}

Жаль, вы не поняли. Нет давно Украины, за долги куплена. И ща будет продана
саудитам. Летом запустили оттудова лоукосты, семьями прилетали - выбирать земли
себе.

Населению, кто еще не сбежал, расскажут - \enquote{почем вода в пустыне!}

\iusr{Татьяна Грицай}

Европейская демократия в ближайшем рассмотрении оказалась настолько аморальной
и опять же коррумпированно-продажной, да и внедрять ее хотели не из
бескорыстных соображений, а в обмен на бусики за большие деньги.


\iusr{Alexandr Barsukov}

Все это гадания на кофейной гуще. Какой православный мир, если самая
религиозная часть Украины- западная. А там католики и греко-католики. А на
востоке традиций на выходные в церковь ходить нет уже. Не так уж много людей в
Украине религиозных. И то, что в Украине один человек править будет как в
России, мне кажется тоже маловероятно. У нас бунт в крови. 2 майдана за 10
лет. И Украина разная от востока до запада. Неоднородное общество.

\begin{itemize} % {
\iusr{Ольга Терещенко}
\textbf{Alexandr Barsukov} Вот так и распадется по цивилизации и верованию.

\iusr{Alexandr Barsukov}
\textbf{Ольга Терещенко} 

мы живём в 21 веке, в веке науки и информационных технологий. Наши дети уже
другие. Какая Россия если спросить любого школьника, хотел бы он продолжить
учёбу в Европе или США, он скажет на 90\% что да. И едут массово. От пристижных
вузов до ПТУ где-то в Польше... бредни это все. Пока не умрут те, кто на этой
войне воевал, ни какого сближения с РФ не будет. Даже если риторика изменится и
президент поменяется в РФ.

\iusr{Ольга Терещенко}
\textbf{Alexandr Barsukov} Это тоже только Ваши предположения.
Еще совсем не известно что будет с самой Европой, а она тоже разная и нашествие мусульман скажется со временем....

\iusr{Александр Громыко}
\textbf{Alexandr Barsukov} Вы правы - казаки выбирали гетмана. Вольность в крови у нас. Знакомый россиянин еще до войны поражался этому .

\iusr{Alexandr Barsukov}
\textbf{Ольга Терещенко} 

конечно разная, но ни одна страна не похожа на РФ ментально. Есть сменяемость
власти, свобода слова, этика. Со своими национальными особенностями. И РФ где
чекисты правят так как хотят и сколько хотят. И все подчинено одному человеку.
Как в старые времена-царю.


\iusr{Ольга Терещенко}
\textbf{Alexandr Barsukov} Вы хотите сказать, что мы ближе к западной цивилизации?

\iusr{Alexandr Barsukov}
\textbf{Ольга Терещенко} 

конечно ближе. И чем ближе мы к ней будем, тем лучше. Я ещё 20 лет назад был
удивлён насколько Львов чище чем мой родной Днепр, когда первый раз туда попал.

Чем ближе к западной границе, тем больше это ощущается. Людей много которые
жили, работали в Европе. Дома похожие у себя строят. Ёлочки, цветочки во
дворах. Элементарное дело съездить за покупками в Польшу к примеру. Во Львове
украинский с польскими словами часто. В Берегово на венгерском люди
разговаривают. Они что ментально ближе к РФ? И какое отношение православие тут
имеет. Если большинство людей вообще атеисты.

\iusr{Ольга Терещенко}
\textbf{Alexandr Barsukov} 

Так я же и написала, что распадается. Эта часть как раз другая по менталитету,
и Вы мне не рассказывайте про Берегово, я, как раз тут живу, и многое бы Вам
рассказала.

И с мужем местным 42 года прожила и сама родилась на Закарпатье и про разность
менталитет и то, что не сливаются они на своем опыте познала.

Видели границы двух морей, я видела, вот как то так.


\iusr{Alexandr Barsukov}
\textbf{Ольга Терещенко} 

с другой стороны немцы к примеру ведь тоже разные. Те кто жил в ГДР отличаются
от тех кто в ФРГ, но только старшее поколение. Те кто родился в обьединенной
Германии уже другие. Но от региона к региону есть особенности. И язык
отличается и какие-то привычки. Но тем не менее все вместе они построили
сильнейшее государство. Вот и мне бы хотелось, чтобы не смотря на все наши
отличия мы состоялись как нация и построили государство где будет комфортно и
безопасно жить. Где каждый может реализовать свои способности при желании. На
самом деле за 30 лет сделано не мало, но если бы не наши вороватые правители,
то могли бы больше.

\iusr{Ольга Терещенко}
\textbf{Alexandr Barsukov} Ваши слова да Богу в уши....
К сожалению я в это не верю...

И еще добавлю, что местные русины 1000 лет были оторваны от славян, но
сохранили дух и сегодня отличаются от венгров. В статье верна главная мысль, а
Германия как раз эту мысль и подтверждает.

\iusr{Михаил Иванов}
\textbf{Alexandr Barsukov} 

ключевое слово здесь - \enquote{мне бы хотелось}! И да, за 30 лет действительно сделано
немало - просрали всё что только можно! Осталась только так называемая
\enquote{государственность}, но и ей по всей вероятности скоро придёт конец! Всё к
тому идёт семимильными шагами!

\iusr{Alexandr Barsukov}
\textbf{Михаил Иванов} 

о ты смотри, ботик с аватаркой Шварцнегера. Все у нас нормально. Лучше чем у
многих стран восточной Европы. А будет ещё лучше, когда война закончится.


\end{itemize} % }

\iusr{Игорь Хорст}
почти все страны вышли из язычества... и не вернулись...

\iusr{Юрий Дмитриевич}
\textbf{Игорь Хорст} Да и Константинополь и вся современная Турция когда-то были христианскими. Все основополагающие вселенские соборы (Халкидонский и пр.) были на её территории, но сегодня там почти нет христиан

\iusr{Tetiana Barosa}
Франция почти потеряла свой код, становясь мусульманской. И скандинавские страны на этом же пути.

\iusr{Светлана Никитина}
\textbf{Tetiana Barosa} Но она становится мусульманской не потому что французов массово обращают в ислам.

\iusr{Tetiana Barosa}
\textbf{Светлана Никитина} а Вы из моего текста поняли, что я утверждаю обратное??

\iusr{Денис Коканов}
Обо всем этом писал еще Лев Гумилев в 70-х годах.

\iusr{Сергей Комплектов}
Согласен на все 100\%

\iusr{Дмитрий Колосовский}

Беспочвенные рассуждения. Автор или плохо знает историю упоминаемых стран, или
специально манипулирует фактами.

\iusr{Андрей Манин}
\textbf{Дмитрий Колосовский} \enquote{армиямовавира}.

\iusr{Олег Анатольевич}
Классно написано, четко и по сути

\end{itemize} % }
