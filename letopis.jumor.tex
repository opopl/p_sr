% vim: keymap=russian-jcukenwin
%%beginhead 
 
%%file jumor
%%parent body
 
%%url 
 
%%author 
%%author_id 
%%author_url 
 
%%tags 
%%title 
 
%%endhead 

\url{https://t.me/Medvedeva_Olesya/3379}
Анекдот! На тему запрета омбудсмена жестикулировать по-русски для людей с нарушениями слуха и речи:

А Бог спрашивает
-А как по-украински нога?
- нога
- а рука?
- рука
- а голова?
- голова..
- блин, ну а жопа?
- дупа
- так шож вам из-за одной жопы теперь страну целую давать?

Так и живем...

%https://uposter.ru/blog/yumor/11507.html

 Сегодня нашёл точную формулировку такому явлению, как «лень»:
«Лень — это привычка отдыхать до того, как ты устанешь».

Юмор: Смешные анекдоты


Скажи бабе «Купи самое необходимое!», и она купит плед, рамку для фотографий и косметику.
Скажи мужику «Накупи всякой х@йни!», и он растеряется.

сегодня мне гибэдэдэшник
сказал мадам ёпвашумать
автомобиль отнюдь не ступа
какого ху@ так летать

А я с Фридрихом Энгельсом в один день родился. Хорошо хоть не в один год. А то помер бы уже.

Пpиxодит pаботник к диpектоpу и говоpит:
— Иван Петpович, я xотел бы обpатить ваше внимание, что моя заpплата не соответствует моим способностям.
— Знаю, знаю, но не могу же я позволить умеpеть вам с голоду…

Объявление:
Молодая семья (16 и 17 лет) ищет друга семьи с паспортом для покупки спиртного и сигарет.

— Ну подумаешь, девушка бросила?! Они как электрички: за одной всегда идет другая…
— Нет. Настоящая жена как электричка: попал под одну — второй уже не будет.

сегодня я тевтонский рыцарь
но не с копьём не на коне
и не со знаменем а просто
сегодня я пришёл свиньёй

Выписка из протокола: со слов задержанного: «меня били все, даже ПОНЯТЫЕ»…

Гадаю на ромашках, помогу наполнить верхнюю чакру водкой, заштопаю карму.
Недорого.

– какой сегодня день?
– сегодня сегодня.
– о, это мой любимый день.

Объявление:
— Хочу маленькую собачку.
Подпись: «Большая собачка».

Молодая вдова пожилого олигарха на похоронах не выдержала и все-таки сделала селфи с покойным мужем.

Начинающий писатель решил использовать в своем произведении игру слов и проиграл.

Стоит ли идти к психиатру, спросил я себя.
Мнения разделились.

поэт коллектором работал
придёт бывало к должнику
и вслух свои стихи читает
пока все деньги не вернёт

Чем моложе блогер, тем хуже ему жилось при Сталине и лучше в перестройку.

Дочка просила День рождения в стиле «Золушки». Мы все так и сделали. Сами пошли праздновать, а ее оставили дом убирать!

Самым коротким сочинением на тему «В гостях у бабушки» было «Мы приехали к бабушке, а она нам не открыла».

портретных сходств не обещаю
я начинающий хирург

Если вы не можете определить по шагам в коридоре, кто это из членов вашей семьи, то, может, вы и не семья вовсе?

Маленькой дочке в американской школе задали домашнюю работу о родине. Посидела она в интернете, приносит мне на проверку. Читаю:
— В Москве живут азербайджанцы, татары, и узбеки.
Я прям весь нервничать, мне даже русская речь начала отказывать:
— Слюшай, ты что написала? У тебя отец в Москве родился, у тебя мать в Москве родился! Армяне тоже живут в Москве, да!?

«Как жаль, что век мой так недолог,
Я б много поменял в своей судьбе» — Вот так я думал в минус сорок,
Прильнувший языком к трубе…

Чтобы хоть как-то привлечь внимание школьников, преподавательница написала на доске «Жесть. Смотреть всем».

Секреты маркетолога: Колбаса «Филейная» – звучит гораздо лучше, чем «из задницы».

крокодил напрасно
уползает в тень
я уже решила
сумка и ремень

У респектабельного человека машина должна быть немецкая, часы швейцарские, парфюм французский, а «Доширак» говяжий.

В деревне, после первой брачной ночи, по традиции, жених выносит окровавленную простынь. А следом выходит невеста с разбитым носом.

— А я в детстве мечтал не о компьютере, а о подземном бункере, только рыть его мне было негде. И когда родители купили дачу, сказали одна сотка земли в углу участка твоя, рой свой бункер. Я на радостях вырыл там офигенную яму куба на три, но обустроить не успел… Через неделю возвели над моим бункером сортир. До сих пор не заполнился.

Приказ по Аду № 666: «Собаководов в Ад не брать! Дрессируют чертей и таскают их по выставкам!»

Продолжайте улучшать свой английский до тех пор пока не перестанете слышать от окружающих — «о, какой у вас хороший английский».

Чтобы пережить нынешний кризис, надо быть крупным бизнесменом. Причем настолько крупным, чтобы застрять в оконном проеме даже с разбега.

Проснулся однажды римский композитор Корсаков от какого-то Шумана (как оказалось, Глюк), продрал Глазунова, почесал в Бородине. Потом выпил Чайковского с Бизе, съел Штрауса под Сметаной, закусил Хренниковым с Мясковским, и такое у него в животе Пуччини сделалось! Он понял, что Стравинский. Вскочил, накинул Шуберта и выбежал во Дворжака, чуть на Глинке не поскользнулся. Посереди Дворжака сел на на Могучую Кучку Мусоргского. Бах! Брамс! — навалил Гуно. Запахло Паганини. Римский композитор Корсаков взял Листа с капельками Россини — и Скрябиным по Шопену.

— Здравствуйте!
— Здравствуйте!
— У вас Мальборо есть?
— Есть.
— Какое?
— Есть только лёгкие. Это же аптека.

Вчера я хотел дать денег бездомному. Но потом я увидел табличку на его шее «В любой день, даже завтра, ты можешь оказаться на моем месте».
Я положил деньги обратно в карман. Ни в чем нельзя быть уверенным в наши дни.

олег купил презервативы
проник в семидесятый год
и подарил их папе с мамой
такой нелепый суицыд

Здесь сейчас будет жуткая история о том, как силовики кошмарят малый и средний бизнес.
В общем, в столице жил предприниматель (я намерено не называю имени), у которого был собственный магазин одежды, аксессуаров и личной гигиены. Дела шли неплохо, бизнесмен подумывал о расширении. А еще у него была в центре столицы квартира, которую он решил сдавать, дабы не простаивала. Все его проблемы начались в тот момент, когда в эту квартиру въехал совсем еще молодой сотрудник службы охраны — уроженец, к слову, одного из южных регионов страны. Этот силовик, как потом выяснилось, попал в органы по блату, благодаря кумовству и личной дружбе его родственников с высокопоставленным человеком в погонах из службы охраны.
Юный силовик вел себя отвратительно, за квартиру не платил месяцами, зато систематически закатывал у себя пьянки с друзьями. И тут, вдруг, на мелкого бизнесмена очень круто наехали. Для начала следствие арестовало его жену. Для верности, чтоб не мелочиться, ей пришили госизмену. Госизмену, Карл! Жене мелкого предпринимателя. На квартире и в магазине устроили шмон, а потом арестовали еще и самого бизнесмена. А его жилец, у которого тот, от отчаяния, просил защиты, еще сам и сдал предпринимателя в заботливые руки своих коллег.
И, что бы вы думали? Бизнесмену тоже стали шить госизмену! Еще раз, мелкого предпринимателя обвинили в госизмене! Полный абсурд и произвол. Обращались, разумеется, чудовищно. Держали в адских условиях. Допрашивали по несколько раз в сутки, в том числе, и среди ночи. Устраивали очные ставки с каким-то силовиком, который, как выяснилось, выдавал себя за другое лицо.
В конце концов, правда, бизнесмена выпустили, вынудив того сотрудничать со следствием. Когда он вернулся, то выяснилось, что его жилец домогался до жены (ее тоже отпустили) и склонил ее к сексуальной связи. На свободе бизнесмен пробыл недолго. Вскоре органы опять арестовали его жену, а затем и его самого.
Он отсидел несколько лет без приговора суда. Его просто мурыжили в СИЗО, ничего не предъявляя. Когда бизнесмена, все-таки, отпустили, он уже был полностью разорен. Магазин отжали, квартиру отобрали. Это был наглый рейдерский захват, с которым честный предприниматель не мог бороться.
Так преуспевающий столичный бизнесмен превратился в городского бомжа, жившего на милостыню.
Жилец, кстати, сделал фантастическую карьеру в службе охраны. Жену бизнесмена убили, чтоб не путалась под ногами.
Такова грустная история парижского галантерейщика по фамилии Бонасье…

смотри вон лебеди машуля
а это уточки уть уть
пойдём мы дяденьке мешаем
тонуть

бледную поганку
выловил в борще
ты меня не любишь
милая вапще

день в сети без лайков
ощущенье что
шла родной деревней
не кивнул никто

я охуительный фристайлер
как коршун я вознесся ввысь
я жду медалей в лыжном спорте
еблысь

— У Вас сигареты не будет? Люблю, знаете ли, покурить после секса.
— Но у Вас же не было секса.
— Ах, да, еще одна просьба.

Художник еще раз окинул взглядом обнаженную модель, вздохнул и начал уверенными крупными мазками наносить лубрикант.

спросил у ветврача о рисках
заболеваний пикачу
а он послал меня к другому
врачу

и снова в нашей комуналке
проходит рыцарский турнир
кто побеждает тот проходит
в сортир

Учитель ОБЖ смотрит фильмы катастрофы со словами: « Я же говорил».

Принцип жизни прост:
замёрзшего — согреть, голодного — накормить, утопающего — спасти, охуевшего — отпиздить.

И сказал царь сыновьям:
— Идите и стреляйте!
Старший сын вернулся с «беломориной», средний — с «примой», а младший обкурился на месте и женился на жабе.

— Мам, мам, я сегодня с аргентинцем познакомилась. Он мне даже гимн их пел на своем аргентинском
— Да? И что за гимн?
— Ще не вмерла Аргентїни і слава, і воля…

Лег вчера спать, ну и спи спокойно, так нет же, в голову пришел вопрос: «Интересно, как называют жителей города Осло?».

    Теги: юмор, анекдоты	

Поделитесь с друзьями:

    djamix djamix
    +12
    1295 

1 комментарий
kir

    kir
    25 июня 2018, 10:06
    +1
    Ответить

Учёные проводили социологический эксперимент.
На один остров отправили двух французов и одну
фрацуженку, на второй двух англичан и одну англичанку,
на третий двух русских и одну русскую. Прилетают
через месяц к французам- две могилы, два креста,
две шпаги и француженка вся в слезах. Спрашивают:
— Что здесь произошло?
— Они друг друга из-за меня на дуэли убили.
Приезжают к англичанам — тоже самое, только два пистолета.
Прилетают к русским — у мужиков застолье, бухают от души,
а женщина вся в синяках сидит рядом и стирает грязное белье.
Спрашивают:
— А как же вы ее делите?
— Понедельник, вторник, среда она мне жена, ему —
любовница. Четверг, пятница, суббота — наоборот.
— А в воскресение?
— А в воскресение мы ее пиздим за блядство.

