% vim: keymap=russian-jcukenwin
%%beginhead 
 
%%file 18_11_2021.fb.medvedeva_olesja.1.donbass_detstvo.cmt
%%parent 18_11_2021.fb.medvedeva_olesja.1.donbass_detstvo
 
%%url 
 
%%author_id 
%%date 
 
%%tags 
%%title 
 
%%endhead 
\subsubsection{Коментарі}

\begin{itemize} % {
\iusr{Евгений Таран}
"жужалка" - наша девочка, с Донбасса!  @igg{fbicon.wink} 

\begin{itemize} % {
\iusr{Олеся Медведева}
\textbf{Евгений Таран} надела мастерку, взяла тормозок и пошла

\iusr{Евгений Таран}
\textbf{Олеся Медведева} Моя учительница по Русскому языку приехав к нам на ПМЖ из России, когда услышала это слово говорила что специально ходила смотреть что это, пришла увидела, а это шлак обыкновенный))

\iusr{Лариса Проценко}
\textbf{Евгений Таран} И у нас это жужалка, Запорожская область))

\iusr{Гостевой Волоконовка}
И у нас на Кубани

\iusr{Вадим Константинович}
\textbf{Олеся Медведева} , а тремпель забыли?)
\end{itemize} % }

\iusr{Антонина Белоглазова}

Если бы не твоё закалённое холодом детство, мы бы не сняли клип «Город» или
сцены с ванной в клипе «Глаза врага», где ты лежишь в холодной воде ))

\iusr{Олеся Медведева}
\textbf{Антонина Белоглазова} ахаааа, точно

\iusr{Надежда Опалюк}

Я правильно поняла, что ты сейчас в Горловке, Олеся???
Давно не слышала и читала тебя... и уже запереживала... всё ли у тебя хорошо???!!!

\begin{itemize} % {
\iusr{Олеся Медведева}
\textbf{Надежда Опалюк} нет, я в Киеве

\iusr{Надежда Опалюк}
\textbf{Олеся Медведева} смотрю и читаю тебя постоянно!
Пусть Ангел Хранитель будет всегда с тобой рядом!!!


\iusr{Олеся Медведева}
\textbf{Надежда Опалюк} спасибо огромное

\iusr{Владимир Колмыков}
Сочувствую Олеся, это какой год если не секрет?
\end{itemize} % }

\iusr{Aleksandr Sibircev}
Эт что, я в центре Одессы от печи грелся в детстве.

\iusr{Olga Altey}
\textbf{Aleksandr Sibircev} мы одесситы вообще уникумы в плане жилищных условий. По моему не в одном городе так не жили.

\iusr{Юрий Загребельный}
То ж не тільки світло вимикали, а й угля попробуй виписати в 90-тих...

\iusr{Олеся Медведева}
\textbf{Юрий Загребельный} это да

\iusr{Vetal Pechaev}

В точности та же история, до сих пор помню всю ту жесть и поражаюсь, что ничему
те времена людей не научили. Еще вот тогда многие люди были не в состоянии
платить за свет, воду, долги копились, но никто не отключил, позже стало легче
с деньгами и народ потихоньку погасил задолженность. Если же до подобного мы
дойдем сейчас, боюсь такой роскоши у нас уже не будет.


\iusr{Марченко Светлана}

"Счастливые" годы молодой украинской независимости, принесшей с собой в том
числе и веерные отключения. А бани? Разве не было бань? И на каждой шахте
имелись душевые.


\iusr{Александр Курка}

У нас в Бессарабии так люди до сих пор живут. Так как газа нет, електричество в
сёлах отключают временами, а уголь дорогой 400\$ тонна, за зиму на дом площадью
90 кв.м нужно 2 тонны угля. Работ нет все на заработках

\begin{itemize} % {
\iusr{Татьяна Зурнаджи- Супрунюк}
\textbf{Александр Курка} 

Разница в том, что уголь, в Горловке, во времена СССР, стоил копейки
@igg{fbicon.face.open.mouth} . Семья шахтеров. Это значит- выписывали себе
уголь, хороший уголь на всю зиму. И этот уголь грел и прогревал дома хорошо.
Если дом был построен правильно. А дрова @igg{fbicon.face.open.mouth} . Дрова,
а во времена СССР, их много не нужно было. Был уголь, Антрацит. Им топили дома.
Щепки, газета , уголь. Печь растопили. Дом отапливается, хороший, тёплый дом,
держал тепло. Особый дух в доме, тепло. Нет, не так- жарко в доме.

\iusr{Оксана Дяченко}
\textbf{Татьяна Зурнаджи- Супрунюк} у нас особо углём не топили, у нас были дрова и брикет

\iusr{Татьяна Зурнаджи- Супрунюк}
\textbf{Оксана Дяченко} В Горловке??? @igg{fbicon.face.open.mouth} 

\iusr{Оксана Дяченко}
\textbf{Татьяна Зурнаджи- Супрунюк} а у нас воспоминания исключительно о Горловке?

\iusr{Татьяна Зурнаджи- Супрунюк}
\textbf{Оксана Дяченко} Я говорю о Донбассе, а не о Чернигове. Горловка- это город, где шахты и следовательно уголь. Где шахтеры выписывали себе несколько тонн ( в зависимости от кВ метров дома) угля. По низким ценам.

\iusr{Татьяна Зурнаджи- Супрунюк}
\textbf{Оксана Дяченко} Статья о Горловке. Свои воспоминания Вы можете рассказать в своём рассказе.

\iusr{Aleksandr Sibircev}
\textbf{Татьяна Зурнаджи- Супрунюк} что ж вы так жестко? Автор поста как бы не запрещала делиться воспоминаниями(

\iusr{Тимур Нигамов}
На Кузбасе уголь стоит дешевле чернозема на грядки. Там о газе никто и не помышляет. У нас коллеги работают с Кузбаса.
\end{itemize} % }

\iusr{Алексей Пикаев}
Олеся, Вы ни капли не изменились за прошедшие годы  @igg{fbicon.smile} 


\iusr{Олеся Медведева}
\textbf{Алексей Пикаев} такое же «пузико»?

\iusr{Алексей Пикаев}
\textbf{Олеся Медведева} Я в основном про характерные черты лица и взгляд  @igg{fbicon.smile} 

\iusr{Alex Meleshko}
Знакомое чувство

\iusr{Яков Гольдарб}

Здорово описали воспоминания . У нас (жили в Кировске недалеко от Вашего
детства) тоже были эти отключения и мы всей семьей , сидя рядом (так теплее и
уютнее) темными вечерами играли в города. А ещё воды не было и ее нужно было
носить двумя вёдрами на 5 этаж .

\iusr{Солодухов Олег}
Т.е. в полтора года ты уже прямо была обескуражена?)))

\begin{itemize} % {
\iusr{Олеся Медведева}
\textbf{Солодухов Олег} именно! Да ты глянь на это лицо

\iusr{Солодухов Олег}
\textbf{Олеся Медведева} я «обескураженным» стал ближе к 30.


\iusr{Олеся Медведева}
\textbf{Солодухов Олег} говорят, что девочки быстрее развиваются)))
\end{itemize} % }

\iusr{Яна Свиридова}

Моё детство описала... У нас в доме ни воды, ни канализации небыло. Тот же
квест с печью на дровах, но вместо электроотопления было обычное АГВэ. Газ был
дешёвый и мы топили не экономя, но дом без печи всё равно был холодный. В
темноте, в свете свечей было видно, как печь раскаляется докрасна.

\iusr{Евгений Мальков}

А мы и сейчас этот челендж продолжаем, но сейчас есть бойлер и туалет в доме,
но так же и на улице. Печь топится углём и дровами. Очень много времени
отнимает....

\iusr{Дмитрий Фомин}
Спасибо, что поделились интересными детскими воспоминаниями

\iusr{Любовь Желтышева}
Намучились

\iusr{Владимир Пильщик}
Сурово

\iusr{Любовь Желтышева}
Помню эти веерные отключения, через 2 часа, просто ужасно, вот научились..

\iusr{Олеся Ярмак}
Да, печь, уголь, откл. света... Напомнили... (Живу рядом, в Дзержинске (Торецке).

\iusr{Роман Носок}
Да, всё примерно так и меня было, помню в году так 95-ом не было света 2 недели!! а сейчас интернет на час пропал и уже паника

\iusr{Andrey Antonov}
В детстве все советские лети волком выли без интернета.

\iusr{Людмила Селезнева}

Олеся трудное детство у вас конечно, сочувствую. У меня этот период на 20 лет
ранее был в Харькове. Всё таки огромная разница была между крупным городом и
пгт . Ну а насчёт борща - как его можно не любить ?!?!?!  @igg{fbicon.face.open.mouth} 
\iusr{Serega Amg}
Ещё семки в темноте под разговоры хорошо заходили

\iusr{Mohamed Faruk Elassar}
wonderful young lady until becomes a stunning girl

\iusr{Сергей Жуков}
Да, бывало, по два - три ведра угля в день уходило. Сейчас зимы не те, иногда еще дровами от обрезки сада топлю, намного меньше надо.

\iusr{Игорь Швыгин}
Выросшим в большом городе с паровым отоплением и тёплым толчком трудно понять это... Поэтому так мало среди успешных городских , мы не получили такой жизненной закалки, умению выходить из самых трудных ситуаций ... Старик Дарвин был прав...

\iusr{Надежда Горбачёва}
Спасибо за рассказ!
Это тоже мое воспоминание о бабушкином доме в Макеевке.
На дворе 2021г.и такие дома ещё есть.

\iusr{Denis Nebotov}
Да, подтверждаю. В 90-ые годы очень часто жил у бабушки на 5-ой шахте - через Балку от Комсомольца!  @igg{fbicon.wink} 
Все так и было! Та еще романтика! В обоих смыслах слова!  @igg{fbicon.smile} 
Очень скучаю за Горловкой...

\iusr{Снежана Егорова}
\textbf{Олеся Медведева}

Сложно представить, что это возможно в 21 веке, людям не нюхавшим пороха....
Но, к сожалению, ВОЗ и ныне там и все больше погружается в пучину Средневековья....
Просто люди в основной своей массе-оптимисты и все время помнят только лучшее, а все плохое склонны забывать до тех пор, пока это плохое вновь к ним с двери не постучится....
 @igg{fbicon.flame}{repeat=3} 
@igg{fbicon.face.smiling.tear}{repeat=3}

\begin{itemize} % {
\iusr{Марченко Светлана}
\textbf{Снежана Егорова} это наверняка где-то про середину 90-х. В союзное время веерных отключений не было, и далее, где-то с конца 90х уже началось улучшение.

\iusr{Снежана Егорова}

\textbf{Марченко Светлана} 
@igg{fbicon.face.smiling.tear}{repeat=3}
\end{itemize} % }

\iusr{Сергей Васильев}

Горловка - она большая. Мое детство прошло в Центрально-городском районе, в
пятиэтажке. Там было зимой всегда тепло. И газовые конфорки на всех кухнях.

\begin{itemize} % {
\iusr{Олеся Медведева}
\textbf{Сергей Васильев} провести газ было очень дорого, а жили рядом одни бабульки

\iusr{Сергей Васильев}

\textbf{Олеся Медведева} да. В частном секторе вокруг Горловки был именно так, как вы пишите.
\end{itemize} % }

\iusr{Igor Tenetko}

Мне повезло больше, я вырос в маленьком хруще, но в нем все было и самое
главное была газовая колонка, которую можно было зажигать и горячая вода была,
потому что газ поступал не "из Словакии"..))

Но зато потом я с лихвой наверстал все ледяное и неприятное в военном училище в
Донецке))

\iusr{Олеся Медведева}
\textbf{Igor Tenetko} добро пожаловать! Донбасс гостеприимный )

\iusr{Igor Tenetko}
\textbf{Олеся Медведева}

Это армия, она везде обнимала в свои холодные объятия)

\iusr{Влад Винница}

Теперь, все стало ясно и понятно, откуда такая целеустремленность и хваткость

\iusr{Олеся Медведева}
\textbf{Влад Винница} дайте мне топор

\iusr{Вадим Калмиков}

Тот же Донбасс только лет на двадцать раньше. Шахтеры получали уголь
бесплатно, ещё и выбирали, брали антрацит. У всех были открытые форточки.
Дома были построены датчанами. С тех времён не люблю разжигать печи, камины.

В СССР не было веерных отключений, это было чп.


\iusr{Марченко Светлана}
\textbf{Вадим Калмиков} да, именно!

\iusr{Александр Гайволя}
Какое страшное детство...  @igg{fbicon.frown} 

\begin{itemize} % {
\iusr{Олеся Медведева}
\textbf{Александр Гайволя} почему же? Я выросла в любви и была очень счастливым ребёнком

\iusr{Alex Wo}
\textbf{Александр Гайволя}
Страшное детство - это когда снаряды рвутся или мир через колючую проволоку.
\end{itemize} % }

\iusr{Оксана Женижу}
Всё именно так и было. Макеевка.

\iusr{Наталья Беда Ражева}

Олеся! А где же пиджачок? ?Штаны штанами, трусы трусами, а пиджаки как положено!
@igg{fbicon.face.grinning.smiling.eyes} По суровому Вашему детству надо
добавить, что Вы описали ещё относительно нормальную жизнь по сравнению с тем,
что нас всех ждёт!  ВЕДЬ У ВАС БЫЛ УГОЛЬ, У НАС ЕГО СЕГОДНЯ НЕТ !!!

\iusr{Olga Altey}
Вспомнила и свое детство. Очень похоже, хотя выросла в центре Одессы.
Самое тяжело было для нас мыть мою парализованную бабушку, мы то в баню ходили. А ей ставили посредине комнаты корыто, таскали воду с общего водопровода на этаже и из коммунальной кухни. Вода закипела на примусе.
Выливать воду было легче.
Но все это как то отдалённо помнится, и не считается героизмом.
Столько тёплых добрых воспоминаний

\iusr{Алексей Всемогущий}

А почему уехали из Горловки? Зачем нужен этот Киев? Я вот как родился во Львове
так и живу и не накакой другой не поменял. Не смотря на то что мне начинают не
только хамить а даже угрожать стали за мой Великий и могучий русский язык!

\iusr{Тимур Нигамов}

Тяжелое детсво железные игрушки. Уфа, БАССР зимы, тогда были не чета нынешним.
-40-50С недельку другую было скорей нормой чем исключением. Мы жили в старом
доме которому было 130 лет. Когда то был часовней во дворе уфимского
губернатора. Купол снесли и разделил на комнаты. Удобства во дворе, но был газ
и наша печка топилась газом. Вернее ее переделали под газ с дров. И каждую
весну дед перебирал эту печь, кирпич был обычный и выдерживал ровно одну зиму
испытание природным газом. Мылись в общественной бане раз в неделю. А нас
мелких были прямо в печи по мере загрязнения. Это было целое приключение мы
всей семьей собирались с утра в субботу и шли в баню на Чернышевского. Летом же
у нас в каретнике была оборудован летний душ. Летом мылись там. Благо уральское
лето хоть и короткое, но теплое.

\end{itemize} % }
