%%beginhead 
 
%%file 05_02_2021.fb.arximisto.1.devjat_istor_zdanij_pamjatki_arhitektury
%%parent 05_02_2021
 
%%url https://www.facebook.com/arximisto/posts/699511190747010
 
%%author_id arximisto
%%date 05_02_2021
 
%%tags 
%%title Только девять исторических зданий Мариуполя официально признаны памятками архитектуры
 
%%endhead 

\subsection{Только девять исторических зданий Мариуполя официально признаны памятками архитектуры}
\label{sec:05_02_2021.fb.arximisto.1.devjat_istor_zdanij_pamjatki_arhitektury}

\Purl{https://www.facebook.com/arximisto/posts/699511190747010}
\ifcmt
 author_begin
   author_id arximisto
 author_end
\fi

Только девять исторических зданий Мариуполя официально признаны памятками архитектуры

Всего девять исторических зданий Мариуполя имеют статус памяток архитектуры
местного значения, согласно ответу Департамента культурно-общественного
развития Мариупольского горсовета на информационный запрос А.Марусова,
директора ГО \enquote{Архи-Город}.

\#новости\_архи\_города

Шесть получили этот статус еще при Советском Союзе, в 1983 году. Это:

\begin{itemize}
  \item 1. Водонапорная башня, 1910 г.
  \item 2. Дом со шпилем (восточная башня, 1953-54 гг.)
  \item 3. Дом со шпилем (западная башня, 1953-54 гг.)
  \item 4. Донецкий областной академический драматический театр, 1960 г.
  \item 5. Бывшая Александровская мужская гимназия (\enquote{индустриальный техникум}, 1899 г.)
  \item 6. Отель \enquote{Континенталь} (бывший ДК \enquote{Молодежный} \textbackslash~\enquote{Азовсталь}, 1897 г.)
\end{itemize}

Осенью 2020 года еще три здания получили статус недавно выявленных объектов культурного наследия в сфере архитектуры:

\begin{itemize}
  \item 7. Хоральная синагога, 1882 г.
  \item 8. Казенный винный (акцизный) склад, 1902 г.
  \item 9. Дом В. Нильсена на улице Семенишина, 49, начало XX в.
\end{itemize}

Здание имеет правовой статус недавно выявленного объекта культурного наследия
до принятия решения о его внесении в Государственный реестр недвижимых памяток
Украины. После принятия такого решения оно приобретает статус памятки
архитектуры местного (или национального) значения.

Обязательным условием для придания такого статуса является разработка учетной
документации объекта, согласно Порядка учета объектов культурного наследия
Министерства культуры Украины.

Учетная документация и охранные зоны для синагоги, дома В. Нильсена и казенного
винного склада были разработаны ООО \enquote{Научно-исследовательский проектный центр
\enquote{Историко-архитек\hyp{}турное наследие}} в сентябре-декабре 2020 г. (ООО основано в
Краматорске в 2019 году архитектором Алексеем Губановым).

Компания стала победителем тендера Департамента культурно-общественного
развития (всего было три участника). Благодаря открытому аукциону стоимость
услуг снизилась почти в два раза (с 160 000 до 87 492 грн.). 

Таким образом, средняя стоимость пакета учетных документов по одному зданию
составила 29 000 грн. (около тысячи долларов).

\enquote{Придание трем историческим зданиям статуса нововыявленного объекта культурного
наследия и радует, и печалит одновременно, по мнению А. Марусова, директора ГО
\enquote{Архи-Город}. Печалит – потому что за треть века независимой Украины мэрия
Мариуполя смогла взять под защиту всего лишь три исторических здания! В 2018
году киевский Институт культурного наследия определил в городе, как минимум, 74
здания, потенциальные памятки архитектуры (см. проект историко-архитектурного
опорного плана г. Мариуполя, разработанный Институтом).

Сколько денег нужно, чтобы взять под защиту те же 74 здания? Около 2,1 млн.
грн. (или 76,6 тыс. долларов). Это составляет около 0,05\% доходной части
городского бюджета в 2021 году. 

Велика вероятность, что стоимость будет ниже, если мэрия закажет изготовление
учетной документации сразу же для нескольких десятков зданий...

Между тем, вчера в Киеве мэр Мариуполя официально представлял город как Большую
культурную столицу Украины в 2021 году. Культурную \enquote{столицу}, в которой
архитектурное наследие разрушается на глазах...?!}

==================================

\begin{minipage}{0.9\textwidth}
Письмо-ответ Департамента культурно-общественного развития Мариупольского
городского совета № 425-520-421 от 17.12.2020 см. \url{https://cutt.ly/nkxIj9X}

Договор между Департаментом культурно-общественного развития и ООО
\enquote{Научно-исследовательский проектный центр \enquote{Историко-архитектурное наследие}} от
17.09.2020 и другие документы о закупке услуг по разработке учетной
документации и охранных зон см. \url{https://cutt.ly/rkxho7j}

Порядок обліку об'єктів культурної спадщини см. \url{https://cutt.ly/BkxjHML}

Проект \enquote{Историко-архитектурный опорный план г. Мариуполя Донецкой области по
определению границ и режимов использования зон охраны памятников и исторических
ареалов} см. \url{https://mariupolrada.gov.ua/ru/page/proekti-detalnogo-planu}

Потенциальные памятки архитектуры местного значения см. на карте Мариуполя
здесь h\url{ttps://cutt.ly/ukxUmwc}
\end{minipage}
