%%beginhead 
 
%%file 13_11_2020.fb.fb_group.mariupol.nekropol.1.pozhertvovania_otchet_1
%%parent 13_11_2020
 
%%url https://www.facebook.com/groups/278185963354519/posts/410948413411606
 
%%author_id fb_group.mariupol.nekropol,arximisto
%%date 13_11_2020
 
%%tags 
%%title О пожертвованиях на восстановление Некрополя. Отчет № 1
 
%%endhead 

\subsection{О пожертвованиях на восстановление Некрополя. Отчет № 1}
\label{sec:13_11_2020.fb.fb_group.mariupol.nekropol.1.pozhertvovania_otchet_1}
 
\Purl{https://www.facebook.com/groups/278185963354519/posts/410948413411606}
\ifcmt
 author_begin
   author_id fb_group.mariupol.nekropol,arximisto
 author_end
\fi

\textbf{О пожертвованиях на восстановление Некрополя. Отчет № 1}

Друзья, мы рады сообщить, что 14 человек пожертвовали 4 650 грн. на восстановление Некрополя!

Огромное спасибо благотворителям – Yaroslav Fedorovskyi,\par\noindent Maryna Holovnova, Алла
Штенкер, Алиса Молчанова, Александр Черномор, Владиславу Довгалюку%
\footnote{\url{https://www.facebook.com/vladislav.dovgalyuk}}, Александру Мингалеву, Виктор Евгеньевич, Марина
Богородченко, София Мещерская, Анне Кучеренко, Наталье Гончаровой, Феликсу
Кашарскому%
\footnote{\url{https://www.facebook.com/kasharsky}}
и анонимному жертвователю!

Нам ценен каждый взнос, вне зависимости от его суммы!

Нас особо порадовало, что среди благотворителей оказались не только
мариупольцы, но и – трое киевлян! 🙂

\textbf{Как мы использовали ваши средства?}

На момент опубликования этого отчета мы потратили 1 396,92 грн. Еще на прошлых
выходных мы закупили 120 клубней мускарей и сразу же посадили их вокруг склепа
Спиридона Гофа. Успели до холодов! Один клубень обошелся всего в 2 грн.

Для покраски оград и крестов закупили по три литра черной и белой краски,
кисточки, железные скребки (чтоб зачищать поверхность), уайт-спирит
\textbackslash растворитель (а также мешки для мусора). Одну ограду успели
покрасить на прошлых выходных. Сегодня купили одну лопату (одна из трех
сломалась на прошлых выходных).

Мы ведем переговоры с мариупольским питомником о получении саженцев клена. Их
результаты нас приятно удивили. Но – об этом позже, после их завершения.

Более полная информация о сделанном на прошлых выходных (7-8 ноября) - в
отдельном отчете, который мы вскоре опубликуем.

\textbf{Что мы планируем сделать на этих выходных - 14-15 ноября?}

Покрасить оградки \textbackslash\ кресты на древнем участке. Выкопать ямы для
деревьев и, возможно, их посадить. Восстановить разваленные советские
памятники. Продолжить разбор мусорных куч возле склепа Александра Хараджаева,
городского головы в 1860-64 гг. Если успеем – сделать разметку для прокладки
дорожек из агроволокна и гравия.

Мы приглашаем всех желающих присоединяться к волонтерам! Мы работаем в субботу
и воскресенье с 10 утра. Встреча – в центре Некрополя, у белого Памятного
креста. Возьмите обязательно маски и перчатки. Контактный телефон – 096 463 69
88.

Мы не опубликовали отдельный анонс о волонтерской экспедиции на этих выходных
из-за введения \enquote{локдауна выходного дня}. Надеемся, что вскоре все
ограничения будут четко понятны.

Еще раз – огромная благодарность всем благотворителям! Мы продолжим публиковать
отчеты об использовании ваших пожертвований.

И – до встречи в Некрополе! 🙂
