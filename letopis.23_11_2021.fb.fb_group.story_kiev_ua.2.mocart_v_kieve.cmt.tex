% vim: keymap=russian-jcukenwin
%%beginhead 
 
%%file 23_11_2021.fb.fb_group.story_kiev_ua.2.mocart_v_kieve.cmt
%%parent 23_11_2021.fb.fb_group.story_kiev_ua.2.mocart_v_kieve
 
%%url 
 
%%author_id 
%%date 
 
%%tags 
%%title 
 
%%endhead 
\zzSecCmt

\begin{itemize} % {
\iusr{Таня Сидорова}
Дякую, сподобалась і ваша прогулянка прозою написана, але поетична і вірш сподобався.

\begin{itemize} % {
\iusr{роман чудновский}
\textbf{Таня Сидорова} дякую @igg{fbicon.smile} 

\iusr{Таня Сидорова}
\textbf{роман чудновский}  @igg{fbicon.hands.applause.yellow} 
\end{itemize} % }

\iusr{Evgen Zachepylenko}
Моцарт Вена ...

\ifcmt
  ig https://scontent-frx5-1.xx.fbcdn.net/v/t39.30808-6/258870347_1037817386792361_5866281461053885943_n.jpg?_nc_cat=100&ccb=1-5&_nc_sid=dbeb18&_nc_ohc=pSlASYOpdVIAX_DNlXQ&_nc_ht=scontent-frx5-1.xx&oh=00_AT_jpSy9twztNy0JjZAbQf23BjIDY_CgiSbbLzv8aoIySg&oe=61D85BF6
  @width 0.3
\fi

\iusr{роман чудновский}
\textbf{Evgen Zachepylenko} 

спасибо @igg{fbicon.smile} 

\iusr{Наталия Королева}
Спасибо! И за Моцарта, и за рассказ, и за стихотворение (по-моему, прелестное
@igg{fbicon.wink} ).

\iusr{роман чудновский}
\textbf{Наталия Королева} 

благодарю Вас @igg{fbicon.smile} 

\iusr{Natasha Levitskaya}

Браво, Роман! Чудесная музыкальная прогулка!
@igg{fbicon.hands.applause.yellow}  @igg{fbicon.musical.notes}
@igg{fbicon.musical.score} 

\enquote{Музыка Моцарта –
Вечность ожившая,
Льется, играет, летит!
Музыка Моцарта,
Миру служившая,
Время она победит!}

\ifcmt
  ig https://scontent-frt3-1.xx.fbcdn.net/v/t39.30808-6/259468902_4658899887506988_2175765083901358410_n.jpg?_nc_cat=108&ccb=1-5&_nc_sid=dbeb18&_nc_ohc=32CkhiXDWlgAX95kvUd&_nc_ht=scontent-frt3-1.xx&oh=00_AT8GjPIWt_fuYkWtRvvvLLZUtFRvbnW--O0OTw2hVEgYOw&oe=61D780D4
  @width 0.2
\fi

\iusr{роман чудновский}
\textbf{Natasha Levitskaya} спасибо, Наташа, хорошо, что прогулка обошлась без жертв @igg{fbicon.smile} 

\iusr{Анна Сидоренко}
Спасибо, приятно читается ваш рассказ, тем более жизненный.

\iusr{роман чудновский}
\textbf{Анна Сидоренко} благодарю @igg{fbicon.smile}  @igg{fbicon.sun.with.face} 

\iusr{Evgen Zachepylenko}
Моцарт Вена...

\ifcmt
  ig https://scontent-frt3-1.xx.fbcdn.net/v/t39.30808-6/260450349_1037827896791310_409155755380513366_n.jpg?_nc_cat=107&ccb=1-5&_nc_sid=dbeb18&_nc_ohc=7pj3zSI1Zc4AX-kYiXu&_nc_ht=scontent-frt3-1.xx&oh=00_AT_9WF6htPANZy92_NyKkg5s9CLbqtb053KYnFmR5fI-Nw&oe=61D7F97B
  @width 0.3
\fi

\iusr{роман чудновский}
\textbf{Evgen Zachepylenko} спасибо @igg{fbicon.sun.with.face}{repeat=2} 

\iusr{Татьяна Гурьева}
Как скучно мы живем

\iusr{роман чудновский}
\textbf{Татьяна Гурьева}  @igg{fbicon.smile}  @igg{fbicon.sun.with.face} 

\iusr{Татьяна Гурьева}
\textbf{роман чудновский} но ы сердце не скудеет нежность)

\iusr{Leonid Dukhovny}

А где на Саксаганского, Роман, вы жили? Очень знакомые описания района"! Я с
1944 по 1964 гг. жил на ул. Горького в квартале между Саксаганского и
Жилянский. И фамилия мне ваша знакома. Может мы пересекались?

\begin{itemize} % {
\iusr{роман чудновский}
\textbf{Leonid Dukhovny} увы, не пересекались @igg{fbicon.smile}  я в дом 78 жил, напротив Музея Леси Украинки, а по проходным дворам через дом 76 можно было добраться до училища имени Глиера, чей адрес - улица Толстого @igg{fbicon.smile} 

\iusr{Olga Ponomarskaya}
\textbf{роман чудновский} разве училище не всегда было на углу ул. Воровского и ул. Тургеневской?

\iusr{роман чудновский}
\textbf{Olga Ponomarskaya} я помню то, что было на Толстого @igg{fbicon.smile}  хотя вроде его ремонтировали, может переезжало?.. @igg{fbicon.face.upside.down} 

\iusr{роман чудновский}
\textbf{Leonid Dukhovny} а родился я в 1965 году - выходит, Вы уже тогда же там не жили?

\iusr{Георгий Майоренко}
\textbf{роман чудновский} Рем, а ты не встречал Леонида у Майи Марковны?

\iusr{роман чудновский}
Ой, не помню уже @igg{fbicon.face.upside.down} 

\iusr{Георгий Майоренко}
\textbf{Leonid Dukhovny} Вы могли с Романом пересекаться у Майи Марковны Потаповой.
\end{itemize} % }

\iusr{Наталия Вигерина}
Великолепно написано, впрочем, как всегда! Отдельно благодарю за стихи! @igg{fbicon.maple.leaf} 

\iusr{роман чудновский}
\textbf{Наталия Вигерина} благодарю Вас @igg{fbicon.smile} 

\iusr{Nadiya M Shana}
Красота!

\iusr{роман чудновский}
\textbf{Nadiya M Shana} спасибо @igg{fbicon.sun.with.face} 

\iusr{Natalie Michael}

А Сливко случайно не ЛОР был? Когда-то меня водили на домашнюю консультацию по
поводу гланд. Я вас там не встречала?

\begin{itemize} % {
\iusr{Наталия Лаврентьєва}
\textbf{Natalie Michael} 

Профессор Сливко жил на Саксаганского, был педиатр и консультировал меня по
поводу плеврита в детстве. Помню суматоху в доме когда его пригласили. Какое
уважение. Это были конец 50 годов. Внешности не помню, а вот фамилия застряла в
голове

\end{itemize} % }

\iusr{Галина Грищенко}
 @igg{fbicon.hands.applause.yellow} 

\begin{itemize} % {
\iusr{роман чудновский}
\textbf{Галина Грищенко} спасибо @igg{fbicon.sun.with.face} 

\iusr{Галина Грищенко}
\textbf{роман чудновский} Если найду, напишу своё стихотворение о Моцарте - маге и чародее))

\iusr{роман чудновский}
\textbf{Галина Грищенко} спасибо, обязательно найдите! @igg{fbicon.smile}  @igg{fbicon.sun.with.face} 

\iusr{Галина Грищенко}
\textbf{роман чудновский} Постараюсь! Давно это было))
\end{itemize} % }

\iusr{Татьяна Оржеховская}
Благодарю. Музыка вечна при любом настроении она окрыляет

\iusr{Анатолий Таран}
Спорне про МОЦАРТА

\iusr{Татьяна Желдубовская}
Очень душевно, спасибо большое, всех благ

\iusr{роман чудновский}
\textbf{Татьяна Желдубовская} благодарю Вас @igg{fbicon.sun.with.face} 

\iusr{Татьяна Желдубовская}

\ifcmt
  ig https://i2.paste.pics/eca9b00f227692416295da7154e2c2d1.png
  @width 0.2
\fi

\iusr{Galyna Stryzhak-Svitzer}
И слушали вы Фантазию из окон детской музыкальной школы \#3 им. В. Косенко.

\iusr{роман чудновский}
\textbf{Galyna Stryzhak-Svitzer} именно @igg{fbicon.smile} 

\iusr{Rimma Turovskaya}

Когда-то давно смотрела фильм, в котором симфонический оркестр (дирижера играет
Роберт Тэйлор) исполняет 40-ю симфонию Моцарта \enquote{Юпитер}. Прошли десятки лет, не
помню ни название фильма, ни содержание, но не могу забыть то впечатление,
которое произвела на меня эта МУЗЫКА. А недавно перечитала книгу Д,Вейса
\enquote{Возвышенное и земное} и как обидно было, что у этого гения была такая короткая
и тяжелая жизнь.

\begin{itemize} % {
\iusr{роман чудновский}
\textbf{Rimma Turovskaya} спасибо, и я читал эту книгу, и вообще спасибо небесам, что они подарили нам Моцарта @igg{fbicon.smile}  @igg{fbicon.musical.notes}  @igg{fbicon.sun.with.face} 
\end{itemize} % }

\iusr{Ирина Архипович}
Спасибо!!@igg{fbicon.heart.red}

\iusr{роман чудновский}
\textbf{Ирина Архипович} не за что @igg{fbicon.smile}  @igg{fbicon.sun.with.face} 

\iusr{Olga Step}
По-видимому Вы проходили мимо 3й музыкальной школы им Косенко, это старая школа с давними традициями.

\iusr{роман чудновский}
\textbf{Olga Step} да, это она @igg{fbicon.smile}  @igg{fbicon.sun.with.face} 

\iusr{Людмила Бобошко}

\ifcmt
  ig https://scontent-frx5-2.xx.fbcdn.net/v/t39.1997-6/s168x128/118211260_659965514618313_1559572347261946985_n.png?_nc_cat=1&ccb=1-5&_nc_sid=ac3552&_nc_ohc=r28znXNS9iYAX_8N4KX&_nc_ht=scontent-frx5-2.xx&oh=00_AT_VaoKXjWuAuG8CoUs1Ln-N5q-GMjNtY3CoqJmkHRQ3Ig&oe=61D7A2B7
  @width 0.1
\fi

\iusr{Лариса Шелест}
Как всегда - замечательный, прекрасный рассказ!

\iusr{роман чудновский}
\textbf{Лариса Шелест} спасибо @igg{fbicon.sparkles}  @igg{fbicon.sun.with.face} 

\iusr{Лариса Шелест}

\ifcmt
  ig https://i2.paste.pics/794ce88d32b05220065e62e28a0e7fce.png
  @width 0.2
\fi

\iusr{Мария Бойчук}
Спасибо за Моцарта! Он наш везде...

\iusr{роман чудновский}
\textbf{Мария Бойчук}  @igg{fbicon.sun.with.face}  @igg{fbicon.smile}  @igg{fbicon.musical.notes} не за что

\iusr{София Михайлик}
Чудесный рассказ! И любимый Моцарт! Спасибо!

\ifcmt
  ig https://scontent-frx5-2.xx.fbcdn.net/v/t39.1997-6/s168x128/104115261_1001176463671391_1421272797233986940_n.png?_nc_cat=1&ccb=1-5&_nc_sid=ac3552&_nc_ohc=o93nWNieoUYAX-QnTuI&_nc_ht=scontent-frx5-2.xx&oh=00_AT9Z-2hxMfO-oLxP0eC2nhotsaonY_nSGTX6IOZx3vL-iA&oe=61D709B4
  @width 0.1
\fi

\iusr{роман чудновский}
\textbf{София Михайлик} спасибо @igg{fbicon.smile} 

\iusr{Gennady Henry Sergienko}
Прекрасно

\iusr{роман чудновский}
\textbf{Gennady Henry Sergienko} благодарю @igg{fbicon.smile} 

\iusr{Victoria Novikov}
Чудесно!

\iusr{роман чудновский}
\textbf{Victoria Novikov} благодарю @igg{fbicon.sun.with.face} 

\iusr{Нина Гордийчук}
Спасибо за замечательный рассказ.

\iusr{роман чудновский}
\textbf{Нина Гордийчук} спасибо @igg{fbicon.smile} 

\iusr{Ирина Иванченко}

Увы, хвалебные эпитеты все \enquote{расхватали} раньше, скажу лишь так:  \enquote{Вы просто
молодец!} С женой \enquote{разборки}, значит, до сих пор ей мИлы, ревнует к прошлому
- и это хорошо!

О Моцарте, Шопене и прогулке - бесподобно, а вот стихи... Простите, думаю, не
Пастернак.

\begin{itemize} % {
\iusr{роман чудновский}
\textbf{Ирина Иванченко} спасибо @igg{fbicon.smile} 

\iusr{Евгения Ерёменко}
\textbf{Ирина Иванченко} и правильно, это не Пастернак, а Чудновский  @igg{fbicon.face.grinning.smiling.eyes} 

\iusr{Ирина Иванченко}
\textbf{Евгения Ерёменко}, \enquote{о вкусах не спорят},- говоря, так говорят французы... я вам мешаю ? Наслаждайтесь! У каждого свой вкус.
\end{itemize} % }

\iusr{Татьяна Горовенко}

Приветствую. Вы художник, художник в мыслях и чувствах. Поэзия это состояние
души, без этого состояния не может создаться творение, будь то музыкальное или
литературное. Ваша душа вибрирует на любое высокое творение. Вы так устроены и
а этом Ваше счастье

\iusr{роман чудновский}
\textbf{Татьяна Горовенко} спасибо @igg{fbicon.smile} 

\iusr{Irena Visochan}
 @igg{fbicon.hands.applause.yellow}{repeat=3} @igg{fbicon.heart.suit}

\begin{itemize} % {
\iusr{роман чудновский}
\textbf{Irena Visochan} спасибо @igg{fbicon.sun.with.face}{repeat=2} 

\iusr{Irena Visochan}
\textbf{роман чудновский} Восхитительно! Хорошего дня!

\iusr{роман чудновский}
\textbf{Irena Visochan} спасибо, и Вам @igg{fbicon.smile}  @igg{fbicon.sun.with.face} 
\end{itemize} % }

\iusr{Tatiana Thoene}
Замечательный рассказ. Вы так изысканно пишите.

\iusr{роман чудновский}
\textbf{Tatiana Thoene} благодарю @igg{fbicon.sun.with.face}{repeat=2}  @igg{fbicon.musical.notes} 

\iusr{Валентина Киндрась}
Очень красивый рассказ, как и осень. Стихотворение тоже. Пишите. Ждём.

\iusr{роман чудновский}
\textbf{Валентина Киндрась} спасибо @igg{fbicon.smile} 

\iusr{Tetiana Khomenko}
Прекрасная история @igg{fbicon.heart.red}

\iusr{роман чудновский}
\textbf{Tetiana Khomenko}  @igg{fbicon.smile}  @igg{fbicon.musical.notes}  @igg{fbicon.smile} благодарю Вас

\iusr{Евгения Ерёменко}

Спасибо, Роман! Прекрасное ощущение и, конечно, Моцарт вечен и принадлежит
всем. Совершенно киевская история, с большой любовью Вами написанная  @igg{fbicon.face.smiling.eyes.smiling}  И стихи
от сердца

\iusr{Людмила Старовойтенко}
Спасибо замечательная прогулка и рассказ

\iusr{роман чудновский}
\textbf{Людмила Старовойтенко} спасибо @igg{fbicon.smile} 

\iusr{Людмила Митина}

\ifcmt
  ig https://i2.paste.pics/9bf1491cdb9830187ddc792618e036d1.png
  @width 0.2
\fi

\iusr{Леила Габриадзе}
Прекрасно! Моцарт везде и всегда! Forever!

\iusr{роман чудновский}
\textbf{Леила Габриадзе} спасибо @igg{fbicon.musical.notes}  @igg{fbicon.smile} 

\iusr{Майя Зигель}

Спасибо большое за память, за необыкновенного Рихтера За музыку Моцарта ! Я
тоже могу слушать до бесконечности и Моцарта и Шопена ! Это великая радость !

\iusr{роман чудновский}
\textbf{Майя Зигель} спасибо @igg{fbicon.sun.with.face} 

\iusr{Валентина Ненько}
Спасибо Вам и за Моцарта и за прекрасное стихотворение @igg{fbicon.hands.pray}{repeat=3} @igg{fbicon.heart.red} @igg{fbicon.hand.ok}  @igg{fbicon.thumb.up.yellow} 

\iusr{роман чудновский}
\textbf{Валентина Ненько} спасибо @igg{fbicon.smile} 

\iusr{Валентина Ненько}

\ifcmt
  ig https://scontent-frt3-1.xx.fbcdn.net/v/t39.30808-6/260621753_440986867424453_4485202406626575331_n.jpg?_nc_cat=102&ccb=1-5&_nc_sid=dbeb18&_nc_ohc=YbVIxYCcSekAX-sVN7J&_nc_ht=scontent-frt3-1.xx&oh=00_AT-oTeGLcY2pq6EpQDjeA_XY75DgZmaZN571dA4-NAAY1g&oe=61D859A5
  @width 0.2
\fi

\iusr{роман чудновский}
\textbf{Валентина Ненько} спасибо @igg{fbicon.smile} 

\iusr{Татьяна Тесля}

@igg{fbicon.musical.score} Моцарт на старенькой скрипке играет

Моцарт играет, а скрипка поёт

Моцарт отечества не выбирает

Просто играет всю жизнь напролёт.

Б. Ш. Окуджава  @igg{fbicon.violin} 

\iusr{роман чудновский}
\textbf{Татьяна Тесля} спасибо, обожаю Окуджаву @igg{fbicon.heart.orange} 

\iusr{Лариса Новосёлова}

У меня маленький внук, я включаю ему колыбельную музыку Моцарта!!! Это
прекрасно!!! И он слушает и засыпает, я слушаю и получаю огромное удовольствие


\iusr{роман чудновский}
\textbf{Лариса Новосёлова} замечательно @igg{fbicon.musical.notes}  @igg{fbicon.smile} 

\iusr{Наталия Ковалева}

Как замечательно, я сейчас не в Киеве, но как будто с вами прогулялась по
Саксаганского Красноармейской, Толстого и осенним паркам.

Я жила на Горького в коммуналке и эти места мне очень знакомы!

Спасибо за прогулку и за Моцарта!

\iusr{роман чудновский}
\textbf{Наталия Ковалева} 

спасибо @igg{fbicon.heart.orange}  @igg{fbicon.sun.with.face}
@igg{fbicon.musical.notes}  @igg{fbicon.smile} 

\iusr{Наталия Ковалева}

\ifcmt
  ig https://scontent-frx5-2.xx.fbcdn.net/v/t39.1997-6/s168x128/104115261_1001176463671391_1421272797233986940_n.png?_nc_cat=1&ccb=1-5&_nc_sid=ac3552&_nc_ohc=o93nWNieoUYAX-QnTuI&_nc_ht=scontent-frx5-2.xx&oh=00_AT9Z-2hxMfO-oLxP0eC2nhotsaonY_nSGTX6IOZx3vL-iA&oe=61D709B4
  @width 0.1
\fi

\iusr{Olena Klymenko}
Прекрасный рассказ, душевный. Замечательно пишите. Спасибо.

\begin{itemize} % {
\iusr{роман чудновский}
\textbf{Olena Klymenko} спасибо @igg{fbicon.sun.with.face}  @igg{fbicon.musical.notes}  @igg{fbicon.smile} 

\iusr{Margarita Kaminsky}
\textbf{роман чудновский}

Спасибо, Роман. Я жила на Жилянской, и все, описанные Вами места, мне хорошо
знакомы. Плюс, я занималась на вокальном отделении эстрадно-циркового училища
(помните, в помещении цирка на Площади Победы), а муж - профессиональный
музыкант, контрабасист. Так что музыка Моцарта всегда была в нашей жизни. Мы
были в Страсбурге в музее Моцарта, в Брюгге в Бельгии зашли в ресторан,
почему-то носящий имя Моцарта.

В Европе есть много мест, связанных с именем этого великого человека. Его
музыка - бальзам на сердце.

\ifcmt
  ig https://i2.paste.pics/f85514b163fe699cd49e97e81881b250.png
  @width 0.2
\fi

\iusr{роман чудновский}
Спасибо большое @igg{fbicon.face.upside.down}  @igg{fbicon.musical.notes}  @igg{fbicon.smile} 

\end{itemize} % }

\iusr{Татьяна Ксенофонтова}
СПАСИБО! ПРЕКРАСНО!

\iusr{роман чудновский}
\textbf{Татьяна Ксенофонтова} благодарю @igg{fbicon.musical.notes}  @igg{fbicon.smile}  @igg{fbicon.sun.with.face} 

\iusr{Надія Чуй}

\ifcmt
  ig https://scontent-frx5-2.xx.fbcdn.net/v/t39.1997-6/s168x128/240645880_280280507195601_1513516822806232993_n.png?_nc_cat=1&ccb=1-5&_nc_sid=ac3552&_nc_ohc=0ELHESBuXjcAX-Iepty&_nc_ht=scontent-frx5-2.xx&oh=00_AT8VTbfTXvTqp9B_Jhf2Z06xEiUPs8Pz-cUkAWtV_1m24Q&oe=61D83C09
  @width 0.1
\fi

\iusr{Polina Zagranichnaya}

Как же все мне близко и знакомо. Доктор Сливко был нашим семейным доктором. Он
лечил меня, сестру и моего сына. музыкальная школа на Владимирской была предметом
мечтаний. Туда я даже не пробовала поступать а училась игре на фортепиано в
клубе ТЭЦ на Жадановского. Моцарта я полюбила гораздо позднее.

\begin{itemize} % {
\iusr{роман чудновский}
\textbf{Polina Zagranichnaya} мир тесен @igg{fbicon.musical.notes}  @igg{fbicon.smile}  @igg{fbicon.sun.with.face} 

\iusr{Alexander Koifman}

И 3-я музыкальная школа, где я учился ( преп.Виленская Е.О.) и ул.
Саксаганского, Владимирская, Горького, Толстого - всё это моё детство и юность.
Спасибо Вам, что дали возможность окунуться в эту атмосферу.


\iusr{роман чудновский}
\textbf{Alexander Koifman} спасибо @igg{fbicon.musical.notes}  @igg{fbicon.smile}  @igg{fbicon.sun.with.face} 
\end{itemize} % }


\end{itemize} % }
