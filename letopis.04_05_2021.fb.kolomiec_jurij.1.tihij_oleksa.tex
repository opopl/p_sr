% vim: keymap=russian-jcukenwin
%%beginhead 
 
%%file 04_05_2021.fb.kolomiec_jurij.1.tihij_oleksa
%%parent 04_05_2021
 
%%url https://www.facebook.com/permalink.php?story_fbid=944797259645917&id=100023469540550
 
%%author 
%%author_id 
%%author_url 
 
%%tags 
%%title 
 
%%endhead 
\subsection{Тихий Олекса}
\Purl{https://www.facebook.com/permalink.php?story_fbid=944797259645917&id=100023469540550}

\ifcmt
  pic https://scontent-bos3-1.xx.fbcdn.net/v/t1.6435-9/182438716_944796882979288_1104961961786757461_n.jpg?_nc_cat=110&ccb=1-3&_nc_sid=730e14&_nc_ohc=seryUqHlVEIAX_qzW1-&_nc_ht=scontent-bos3-1.xx&oh=b4ce194edcd14fdd4c975c6679a5d116&oe=60B77628
\fi

5 травня минає 37 років з дня відходу у вічність (1984) українського дисидента,
правозахисника, педагога, мовознавця, члена-засновника Української гельсінської
групи, що виступав на захист української мови Олекси (Олексія) Івановича Тихого
(27 січня 1927 р., хутір Їжевка, Краматорський район, Артемівська округа – 5
травня 1984 р., тюремна лікарня м.Перм).

Життєпис. Закінчив філософський факультет Московського університету.

У 1948 р. вперше засуджений військовим трибуналом Сталінської (тепер Донецької)
області за критику кандидата в депутати на 5 років позбавлення волі, але
військовий трибунал Міністерства внутрішніх справ Українського округу замінив
покарання на умовне.

З 1950 р. працював учителем біології у Новокостянтинівській школі (тепер
Приазовського району Запорізької області). З 1953 р. мешкав у селі Рубцове
(тепер Лиманського району Донецької області), викладав у школі. З 1954 р. –
вчитель історії у рідному селі.

Заарештований у лютому 1956 р. за листа, надісланого до ЦК КПРС із протестом
проти введення військ Варшавського договору в Угорщину. 18 квітня 1957 р. на
закритому засіданні Сталінського обласного суду (м.Донецьк) на підставі ст.
54-10 ч. 1 Карного кодексу УРСР «за антирадянську агітацію та пропаганду»
засуджений на 7 років таборів і 5 років позбавлення громадянських прав. Термін
покарання відбував у Володимирській тюрмі та Дубравлагу.

Після звільнення 15 лютого 1964 р. О.Тихий, не маючи можливості влаштуватися на
роботу за фахом, працював вантажником, слюсарем-механіком,
оператором-дефектоскопістом, пожежником. Водночас проводив велику роботу з
укладання словника української мови, розробив «метод навчання без школи» (за
домашніми завданнями). У своїх публіцистичних творах виступав за відродження
української мови та національної культури на Донеччині.

У січні 1972 р. О.Тихий надіслав до редакції газети «Радянська Донеччина»
статтю «Роздуми про українську мову та культуру в Донецькій області», а на
початку 1973 р. – до Президії Верховної Ради УРСР листа під назвою «Думки про
рідний донецький край» (у квітні листа було відправлено на адресу Голови
Президії Верховної Ради УРСР І.Грушецького). У 1974 р. написав нарис «Сільські
проблеми» та роздуми «Ви і ми», в яких виступив проти процесу русифікації та на
захист української мови. У листопаді 1976 р. О.Тихий разом з М.Руденком,
О.Мешко, П.Григоренком, Л.Лук'яненком, О.Бердником та ін. виступив
членом-засновником однієї з найперших правозахисних асоціацій – Української
громадської групи сприяння виконанню Гельсінських угод і підписав перші
документи УГГ «Декларацію Української громадської групи сприяння виконанню
Гельсінських угод» та «Меморандум №1».

Літературна і правозахисна діяльність О.Тихого стала причиною його другого
арешту на початку лютого 1977 р. Справу О.Тихого було об'єднано зі справою
колишнього секретаря парткому Спілки письменників України М.Руденка. У
червні-липні 1977 р. під час судового процесу («процес по справі
Руденка-Тихого») в Дружківці (Донецької області) О.Тихого було звинувачено в
«антирадянській агітації і пропаганді» та у «незаконному зберіганні зброї»
(було підкинуто гвинтівку). 21 липня 1977 р. оголошено вирок: О.Тихого
позбавлено волі на 10 років у виправно-трудовій колонії особливого режиму з
засланням на 5 років. Суд визнав його «винятково небезпечним рецидивістом».
Місцем покарання визначено табір особливого режиму ЖХ-385/1, с.Сосновка в
Мордовії, звідки його етапували до лікарні в м. Нижній Тагіл.

На захист О.Тихого виступили визначні правозахисники П.Григоренко, А.Сахаров,
Л.Лук'яненко, О.Подрабінек та ін.

У березні 1980 р. О.Тихого перевели у табір особливого режиму для політв'язнів
у с. Кучино (Чусовський район Пермської області, Росія). Кілька разів
оголошував голодування (найдовше – 52 дні). З 1981 р. тяжко хворів. Помер 5
травня 1984 р. у тюремній лікарні Пермі.

Постановою Пленуму Верховного суду УРСР від 7 грудня 1990 р. вироки щодо
О.Тихого скасовано і справу закрито «за відсутністю складу злочину».

19 листопада 1989 р. його прах перепоховано на Байковому кладовищі в Києві
поряд з прахом В.Стуса та Ю.Литвина.

Життєва позиція. Тихий Олексій Іванович: «Сьогодні думаю: 

1) Я – українець. Не лише індивід, наділений певною подобою, умінням ходити на
двох кінцівках, даром членороздільної мови, даром творити та споживати
матеріальні блага. Я громадянин СРСР, і як «советский человек», і, передусім,
як українець, я – «громадянин світу», не як безбатченко-космополіт, а як
українець... Люблю свою Донеччину. Її степи, байраки, лісосмуги, терикони.
Люблю і її людей, невтомних трударів землі, заводів, фабрик, шахт. Любив
завжди, люблю і сьогодні, як мені здається, в годину негоди, асиміляції,
байдужості моїх земляків-українців до національної культури, навіть до рідної
мови...

2) Я – для того, щоб жив мій народ, щоб підносилась його культура, щоб голос
мого народу достойно вів свою партію в багатоголосому хорі світової культури. Я
– для того, щоб мої земляки-донбасівці давали не лише вугілля, сталь, прокат,
машини, пшеницю, молоко та яйця. Для того, щоб моя Донеччина давала не тільки
уболівальників футболу, учених-безбатченків, російськомовних інженерів,
агрономів, лікарів, учителів, а й українських спеціалістів-патріотів,
українських письменників, українських композиторів та акторів.

3) Я, очевидно, поганий патріот, слабодуха людина, бо, бачачи кривди рідного
народу, примітивізм життя людей, усвідомлюючи гіркі наслідки сучасного навчання
й виховання дітей, випадання з кола культурного розвитку мільйонів моїх
одноплемінців, задовольняюся ситістю, маніловськими мріями, крихтами культури
тільки для себе. І не маю ні мужності, ні волі активно боротися за розквіт
національної культури на Донеччині, за прийдешнє.

Не біда, а вина кожного інтелігента, кожного, хто здобув вищу освіту, займає
керівні посади, а живе тільки для натоптування черева, байдужий, як колода, до
долі свого народу, його культури, мови».

Вшанування пам`яті. В грудні 1997 р. в Дружківці було створено фонд пам'яті
Олекси Тихого.

Указом Президента України від 8 листопада 2006 р. за громадянську мужність,
самовідданість у боротьбі за утвердження ідеалів свободи і демократії та з
нагоди 30-ї річниці створення Української Громадської Групи сприяння виконанню
Гельсінкських угод Олекса Тихий нагороджений орденом «За мужність» І ступеня
(посмертно).

26 січня 2007 р. відбулося урочисте відкриття пам'ятної дошки, присвяченої
Олексі Тихому в місті Дружківка, та пам'ятника на подвір'ї школи №14 в смт.
Олексієво-Дружківці Донецької області, де О.Тихий навчався у 1940-ві рр., а
потім й викладав. 

16 травня 2007 р. в Донецьку, вперше на обласному рівні, в приміщенні Донецької
обласної філармонії відбувся великий меморіальний захід з нагоди 80-річчя від
народження, 23-ї річниці його загибелі, а також презентації виходу з друку
книги «Олекса Тихий. Мова – народ. Висловлювання про мову та її значення в
житті народу». Окрім цього, присутнім представили двотомне майбутнє видання
творів Тихого та спогадів про Олексу «Страдницький шлях українського
правозахисника. Методичний посібник для вчителів» та «Словник неправильностей
українських говорів Донбасу». Ініціатором та організатором заходу була Донецька
обласна організація Всеукраїнського Товариства «Просвіта» ім. Тараса Шевченка
за участю обласної бібліотеки ім. Крупської, Донецького національного та
Слов'янського державного педагогічного університету. 

У травні 2007 р. при ДОО ВУТ «Просвіта» імені Т.Шевченка, у вигляді тематичного
об'єднання почало роботу Донецьке обласне товариство імені Олекси Тихого.
Голова – Євген Олексійович Шаповалов.

У 2008 р. за результатами міського конкурсу «10 знаменитих дружківчан», що
проводився міськими газетами «Наша Дружківка» та «Дружківка на долонях»
О.Тихого визнали найвідомішим дружківцем.

Товариством ім. Олекси Тихого за сприянням ДОО ВУТ «Просвіта» імені Тараса
Шевченка, Донецького осередку Союзу Українок, редакцій газет «Наша Дружківка»,
«Провінція», видаються книги, та брошури про Тихого, та з його статтями. Так у
2007-2009 рр. вийшли 5 книг з серії «Хто ж такий Олекса Тихий?», 2008 рік
книжка «Не можу більше мовчати». У 2009 р. нарешті побачив світ довгоочікуваний
«Словник мовних покручів», матеріали до якого О.Тихий збирав майже все життя.

Починаючи з 2007 р. ДОО ВУТ «Просвіта» імені Т.Шевченка, у якому тематичним
об'єднанням діє товариство Олекси Тихого, проводяться щорічні «Олексини
читання», у яких беруть участь учні старших класів загальноосвітніх шкіл
Донеччини, студенти, молоді педагоги. Переможці читань преміюються поїздкою до
Києва.

Проводяться велопробіги під девізом «Стежками Олекси Тихого». У 2011 р.
велопробіг організовувала Костянтинівська райдержадміністрація.

У 2010 р., за сприянням Товариства О.Тихого, вийшли книжки Олекси Тихого
«Шевченко про мову» і «Думки про виховання». До 85-ти річчя зі дня народження
О.Тихого Донецькою ОДА, під редакцією В.Овсієнка, готується до друку двотомник
до якого включені як статті самого Олексія Івановича, так і свідчення про нього
видатних людей.

У 2016 р. Український інститут національної пам'яті в рамках відзначення 83-х
роковин Голодомору в Україні вніс його ім'я до проекту «Незламні», як
відзначення на державному рівні 15 видатних людей, що пройшли через страшні
1932-33 рр. та змогли реалізувати себе.

Вулиці, названі на честь Олекси Тихого, існують в кількох населених пунктах України. 

27 січня 2017 р. на державному рівні в Україні відзначено ювілей – 90 років з
дня народження Олекси (Олексія) Тихого (1927-1984), правозахисника, політв'язня
радянського режиму, поета, мовознавця, педагога, члена Української гельсінської
групи. 

У листопаді 2019 р. в Києві вулицю Виборзьку перейменовано на честь Олекси Тихого. 

З 16 березня 2020 р. в Києві станцію швидкісного трамваю «Польова»
перейменовано на станцію «Олекси Тихого».
