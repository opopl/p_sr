% vim: keymap=russian-jcukenwin
%%beginhead 
 
%%file 13_05_2021.fb.bilchenko_evgenia.1.babushka
%%parent 13_05_2021
 
%%url https://www.facebook.com/yevzhik/posts/3888438801191201
 
%%author 
%%author_id 
%%author_url 
 
%%tags 
%%title 
 
%%endhead 

\subsection{БЖ. Бабушка, я не плачу}
\Purl{https://www.facebook.com/yevzhik/posts/3888438801191201}

Лет до пяти я плакала в транспорте, спрятав в рукав лицо:
Меня убивали чужие люди, контролёры, билеты, сдачи.
А потом, как рукой сняло, как выкатали яйцом.
И я в счастье шепнула бабушке: 
"Я не плачу".

\ifcmt
  pic https://scontent-bos3-1.xx.fbcdn.net/v/t1.6435-0/p526x296/185737676_3888438621191219_6845515577122396345_n.jpg?_nc_cat=104&ccb=1-3&_nc_sid=8bfeb9&_nc_ohc=xZ8xYOghUt0AX-iDJhX&_nc_ht=scontent-bos3-1.xx&tp=6&oh=bf5cd12af3d1f96e98d117597490b70a&oe=60C5E79A
\fi

Теперь меня убивают вежливо, с наслаждением,
В грудину вбивая уже не гвоздь и даже не кол, а столб -
Столп истины - это ствол: стреляя по дребедени,
Он превращает меня в ничто, выпевающееся в стол.

Меня убивают за те стихи, за которые и любили
Те, кто считал, что стихи о них, но вышло, что не о них.
Теперь стихи мои - в секретерах Питера и Сибири:
Строки мои невечные, ибо вечность - она же миг.

И вот, я улицами бреду, бывшими мне своими,
Чисто физически среди них ещё обитая.
Вот яблоня расцветает... Вот оживает имя
На одной из табличек... Вот лепесток, летая, в полёте тает.

Вот кладбище, где хоронила, убитая третьей горстью
Вязкой, сырой, родной - безвозвратно родной - земли.
На ней лежат мои гроздья с венками, под нею - мои же кости:
Ещё лежали бы сотни лет, но истлели и не смогли.

И стали они лепестка невесомее, и взлетели: туда, к комодам и секретерам, 
Где спрятаны в стол стихи.
Да святится имя твоё, Любовь, Надежда и Вера.

И глаза мои - страшные, очень страшные,
Ибо они сухи.

14 мая 2021 г.
