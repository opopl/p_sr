% vim: keymap=russian-jcukenwin
%%beginhead 
 
%%file 07_06_2021.fb.zharkih_katerina.1.ukraina_rossia_mir
%%parent 07_06_2021
 
%%url https://www.facebook.com/kate.zharkih.5/posts/6257241677635217
 
%%author Жарких, Катерина
%%author_id zharkih_katerina
%%author_url 
 
%%tags 
%%title ХОТЯТ ЛИ МИРА УКРАИНЦЫ С РОССИЕЙ?
 
%%endhead 
 
\subsection{ХОТЯТ ЛИ МИРА УКРАИНЦЫ С РОССИЕЙ?}
\label{sec:07_06_2021.fb.zharkih_katerina.1.ukraina_rossia_mir}
\Purl{https://www.facebook.com/kate.zharkih.5/posts/6257241677635217}
\ifcmt
 author_begin
   author_id zharkih_katerina
 author_end
\fi

ХОТЯТ ЛИ МИРА УКРАИНЦЫ С РОССИЕЙ?

(Ссылка на опрос в первом комментарии)

2 года назад, когда экс-президент Порошенко досиживал свой срок, я проводила
опрос киевлян, нужен ли нам мир с Россией. Ответы были очень вдохновляющие - из
13 случайных граждан только 2 сказали, что нет. А остальные говорили о
переговорах, мирном урегулировании, что этим вопросом должен заниматься
президент. 

Собственно на это и давил будущий президент Зеленский. Все помнят, как он, по
его словам, с чертом лысым готов был договориться, лишь бы не гибли наши
граждане. На этих обещаниях и народной антипатии к Порошенко он и победил. 

Как видим 2 года спустя, для достижения мира не сделано почти ничего. Даже
наоборот. После первых робких попыток Зеленский скатился в какой-то треш и
накаляет обстановку похлеще Порошенко, но с молодыми и красивыми лицами. 

Выходит обманул, пошутил наш комик-президент. А что скажут сегодня киевляне,
если задать им тот же вопрос? Поменялось ли отношение? Хотят ли мира так же как
2 года назад, или \enquote{миролюбивой} зелёной власти удалось настроть людей
на войну ещё больше? И кто из политиков может его установить? 

Спрашиваем сегодня на том же месте у простых прохожих, погнали!

\ifcmt
  pic https://scontent-cdg2-1.xx.fbcdn.net/v/t1.6435-9/197154082_6257337067625678_3300461341104958729_n.jpg?_nc_cat=108&ccb=1-3&_nc_sid=8bfeb9&_nc_ohc=YqFUThKB188AX8Gv2xp&_nc_ht=scontent-cdg2-1.xx&oh=8fa51e49b1767388022307ed01cf30e6&oe=60E3072A
\fi

\begin{itemize}

\iusr{Екатерина Жарких}
\url{https://youtu.be/ZwPpGkI-gwk}
Опрос о мире Украины с Россией, Киев 2021. Екатерина Жарких
\iusr{Екатерина Жарких}
А это опрос 2019 года \url{https://youtu.be/o7SP0g0TxKg}

\iusr{Александр Соловйов}
Так путін ще зовсім не лисий. Довго чекати...
\iusr{Василий Январев}
Мир конечно же нужен. Вопрос в том на каких условиях....

\begin{itemize}

\iusr{Aleksandr Perekhodchenko}
Василий Январев, на условиях безоговорочной капитуляции, как с нацистами в 45 )
\iusr{Василий Январев}
Aleksandr Perekhodchenko глупая шутка
\iusr{Екатерина Жарких}
Василий Январев на каких?
Пока тягомотина политическая продолжают умирать люди.
\iusr{Aleksandr Perekhodchenko}
Василий Январев , а с нацистами других вариантов нет
\iusr{Василий Январев}
Екатерина Жарких Это очень сложный вопрос, но очевидно не на минских, поскольку
это капитуляция в чистом виде на которую не пошло бы никакое государство.
\iusr{Valery Staroverov}
Василий Январев Условие очень простое: украинская власть признаёт независимость
республик. И всё, война полностью прекращается со всеми выгодами для всех.
Главное: люди перестанут гибнуть от войны, мирная жизнь у всех будет
по-настоящему, у Украины появится возможность полноценно развивать свою
экономику, тоже и у донбассовцев. Приведите свои контраргументы против
изложенного.
\iusr{Василий Январев}
Валерий Староверов 1. Никаких \enquote{республик} не существует - эти образования обладают нулевой политической субъектностью. 2. В Кремле этот вариант не одобрят.
\iusr{Valery Staroverov}
Василий Январев Это не контраргументы, а детский лепет. До вас ещё не дошло,
что Кремль терпеть не будет натовскую страну у себя под боком, для этого и Крым
ушёл от Украины и республики появились. Придётся с этим смириться.
\iusr{Василий Январев}
Валерий Староверов То есть Вы признаете, что никакой субъектности у "республик" нет:)
\iusr{Aleks Skela}
Василий Январев у Украины нет никакой субъектности.

\end{itemize}

\iusr{Максим Диканенко}

Дорогие друзья,то что происходит,последние лет 7 это какой-то треш,угар,адское состояние.

Причем,все уже забыли,что вначале был разговор о подписании евроассоциации,которая якобы улучшит жизнь простых украинцев.

Скажите прямо,братья и сестры,лучше стали жить?

Пришла долгожданная свобода от \enquote{москалей и ватников}?

А с ней и обещанные промайданными \enquote{атаманами} -благополучие и спокойствие???

И в чем это выражается,если забыть о \enquote{достижении} в виде вытаскивания из могил
профашистских полицаев и бандюг,и попыток сделать их национальными героями?

А Ковпака,Павличенко,Кожедуба,Ватутина,Жукова и прочих наших общих Героев
попытались низвергнуть и забыть???  Кому от этого стало лучше???

\iusr{Alexander Myro}

У меня одного возникают вопросы: когда это Украина объявила, что начала воевать
с Россией? В какую российскую область они зашли за последние 5 лет?  Или они
уже считают граждан Украины на востоке и Украинские восточные территории Донбан
и Луганск - Россией?  Ох уж этот телевизор...вспоминается: "Дайте мне средства
массовой информации, и я из любого народа сделаю стадо свиней“ - Йозеф
Геббельс.

\iusr{Вася Николаев}

Дурнуватий опрос, росіяни не хочуть миру з Україною, тому що їх вождь не хоче
вести розмову з Зеленським відносно війни на Донбасі. Це сказав сам Путлер.

\iusr{Michael Kot}

Вася Николаев Дурнуватий пост, який свідчить що той, хто його написав, не
думає, а тільки споживає пропаганду укрТВ.

\iusr{Serg Timo}

Вася Николаев ну ты Вася и ваааася....)))))

\iusr{Сергей Никонов}

Спасибо за интересный опрос. Терпения Вам, ко всем подходы находите. Терпение
нужно ибо фанатиков в обществе хватает. За себя скажу. Нам нужен не просто мир,
а нормальные отношения с РФ. И это возможно. Видно даже по ответам отвечавших
против мира. Не потому, что они плохие люди, а потому что их там обработали
пропагандой или есть личные особенности восприятия мира. Люди найдут общий
язык. Но политиканам и заживающимся на войне это не нужно.

\iusr{Sergey Gaydash}

Не понимаю, откуда в голове у людей берется, что надо договариваться с Россией.
Мир утвердил минские соглашения в которых четко написано, кто и с кем
договаривается. Украина может либо следовать этим соглашениям, либо отказаться
и предложить свой вариант решения.

\iusr{Денис Логинов}

Мира? А у нас война, да? За российский транзит, видимо? За ТВЭЛы, уголь и
ГСМ... Люди, бегите с Украины. Рагулизм заразен

\iusr{Денис Логинов}

Украинцы одного не могут понять: Россия говорила с ними в 13-ом году.
Переговоров с Россией не будет. Украина давно продемонстрировала свою
неадекватность и недоговороспособность. В.В., конечно, человек вежливый, но на
Украине разговаривать уже не с кем...

\end{itemize}
