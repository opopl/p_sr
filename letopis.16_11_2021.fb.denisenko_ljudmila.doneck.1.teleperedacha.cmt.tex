% vim: keymap=russian-jcukenwin
%%beginhead 
 
%%file 16_11_2021.fb.denisenko_ljudmila.doneck.1.teleperedacha.cmt
%%parent 16_11_2021.fb.denisenko_ljudmila.doneck.1.teleperedacha
 
%%url 
 
%%author_id 
%%date 
 
%%tags 
%%title 
 
%%endhead 
\subsubsection{Коментарі}
\label{sec:16_11_2021.fb.denisenko_ljudmila.doneck.1.teleperedacha.cmt}

\begin{itemize} % {
\iusr{Андрей Налётов}

У Вас получается по-настоящему качественный продукт, который интересен и
пациентам, и докторам... На сегодняшний день достойных передач на нашем тв (и
это не только в Республике) практически нет..., поэтому лично я надеюсь, что
перерыв - временный...

\iusr{Рамаз Джоджуа}
Думаю после творческого отпуска появятся идеи, а мы поддержим!

\iusr{Nataliia Maslova}

У відпустці і зрозумієте - Ваше це чи ні! Хоча я бачу, так, однозначно Ваше! В
ТБ є як \enquote{+}, так і \enquote{-}. Але, якщо у Вашій передачі великий рейтинг, то
зупинятися точно не варто!

\iusr{Виктор Гинькут}
С нетерпением будем ждать Вашего возвращения!

\begin{itemize} % {
\iusr{Ксения Шадрина}
\textbf{Виктор Гинькут} После Нового года будем хором звать: "Сне-гу-ро-чка! (Док-тор-Три-Дэ!)  @igg{fbicon.heart.eyes} 
\end{itemize} % }

\iusr{Светлана Лебеденко}
Если вообще участие помогло хотя бы нескольким людям решить свои проблемы со здоровьем, пойти к врачу, пересмотреть питание,то стоит продолжать.

\iusr{Ксения Шадрина}

Людмила, не успели мы обрадоваться, а кина уже не будет?..  @igg{fbicon.face.pensive}  На одном дыхании
все передачи! Две последние еще не смотрела, не хочу походу слушать, смотрю,
исключительно без отрыва глаз. Важно всё, и темы, и знакомство с докторами,
которые трудятся на сегодняшний день! Важно то, это всё это происходит в нашем
родном городе Донецке, в нашем Доме родном! Людмила, планка взята! Разве ж вы
сможете жить без этого проекта? Я подсела! Все мои друзья, разбросанные по
свету ,смотрят и очень благодарят за своевременность и пользу от этих
замечательных передач! Нужно продолжать!!! Просим! Благодарим за труды!
 @igg{fbicon.face.blowing.kiss}  @igg{fbicon.tulip}{repeat=3} 

\iusr{Елена Качанова}
Конечно продолжать. После отпуска внесете еще новую струю.

\iusr{Оксана Ковалева}
Возвращайтесь.

\iusr{Анна Ястримская}

Людмила Эдуардовна, уникальный и полезный продукт вы делаете. Не оставляйте,
пожалуйста, эту затею! Просветительство - благое дело. Когда это касается
здоровья - втройне полезнее. В любой форме, какую бы вы не выбрали, не
оставляйте, пожалуйста, тех, кого приручили и тех, кто ещё должен проникнуться.

\iusr{Яна Жуковская}
Конечно продолжать!

\iusr{Наталия Чернецкая}

Режим напряженный. Понятно - устала. Вот отдохнешь в кругу близких и родных
людей, вернешься вся такая позитивная  @igg{fbicon.wink}  @igg{fbicon.face.upside.down}  @igg{fbicon.face.blowing.kiss}  и будет не хватать этой суеты и
усталости. Впереди отпуск - заряжайся позитивом, отдыхай (ты уж точно заслужила
этот отдых) , а потом сама поймешь, продолжать или нет. А передача нужная,
однозначно.


\iusr{Екатерина Фомичева}

Людмила Эдуардовна, мы с нетерпением ждём Вашего возвращения! Верю в то, что
эти "Рождественские каникулы" рассеют все сомнения, родятся новые идеи, а
воплощение их будет лёгким! Вы действительно умеете мотивировать и вдохновлять,
в наше время это очень важно!! Всё получится!

\iusr{Lika Ishchenko}

Возвращайтесь, пожалуйста! Это единственная передача на местных каналах,
которую интересно смотреть...

\iusr{Татьяна Сидоренко}

Как можно сомневаться в том,что приносит пользу жителям Донбасса!
Хорошего отдыха и много новых идей!

\iusr{Дмитрий Мундштуков}

Я считаю нужно! Что нам Малышевы и Мясниковы?! У нас должна быть Денисенко!
@igg{fbicon.wink}   @igg{fbicon.face.eyes.star}  @igg{fbicon.thumb.up.yellow} 

\iusr{Anna Plyushcheva-Kolesnik}
Людмила Эдуардовна, как же мы без нашего любимого доктора( Будем ждать Вашего возвращения.

\begin{itemize} % {

% -------------------------------------
\ii{fbauth.schadrina_ksenia.doneck.dnr.hudozhnik.zhivopisec.hramy}
% -------------------------------------

\textbf{Anna Plyushcheva-Kolesnik} Слетает в космос и вернется! Но совсем Другой!  @igg{fbicon.smile} 
\end{itemize} % }

\iusr{Галина Цветочная}
Не уходите! Вы очень всем нужны!!!

\iusr{Наталья Ольшевская}
Отдохнуть нужно 100\%. После отпуска ждем продолжение и новеньких идей !

\iusr{Elena Koval}

Передача нужная и полезная. В последнее время я убедилась, что её смотрят очень
многие. И даже конспектируют советы и рекомендации. Очень хорошие отзывы
пациентов. А значит, передача интересна людям, они нуждаются в таком материале!

\iusr{Татьяна Голос}

Передача замечательная. И полезная, и актуальная. Но по себе знаю:
эмоциональное выгорание очень истощает. Так что делай, как лучше тебе, Людочка.

\iusr{Оксана Мартынова}

Прекрасная передача. Каждый раз жду с нетерпением!!!!!!!!!!
Продолжайте, пожалуйста!!!!!

\iusr{Елена Цхакая}

Отдых необходим. Снимется напряжение , придут новые мысли, и Ваши преданные
зрители увидят Вас в другом качестве. Надо уходить от однообразия, и Вы как
творческий и пишущий человек знаете это лучше всех. Хорошего отдыха.

\iusr{Виталий Фролков}

Консилиум в режиме реального времени, с разбором коморбидного пациента (может
играть актер), поиск индивидуального подхода с учетом современных стандартов по
лечению отдельных нозологических форм, маршрутизацией пациента в реалиях нашей
диагностической базы, калькуляцией расходов на обследование и лечение, т.е.
поиск оптимальных вариантов по наиболее распространённым сочетаниям
заболеваний. Хороший пример для коллег и демонстрация возможностей для
пациентов. Должно быть интересно.

\begin{itemize} % {
\iusr{Людмила Денисенко}
\textbf{Виталий Фролков} Вот сейчас мои зрители прочитали и подумали: с кем это доктор разговаривал? Снимаем передачу для пациентов в первую очередь!  @igg{fbicon.wink}  Но мысль Ваша понятна. Стоит подумать и над такой формой.

\iusr{Виталий Фролков}
\textbf{Людмила Денисенко}, спасибо)
\end{itemize} % }

\iusr{Фёдор Лимарев}

К сожалению не могу делать выводы о самой передаче, т. к. ни одной не смотрел,
но есть же такое понятие, как рейтинг программы. Если он довольно высокий, то
какой смысл закрывать передачу? Не знаю, может у нас сетка ТВ программ по
другим принципам верстается, тогда мнение зрителей тут вообще ничего не решает.
Поэтому вывод следующей, если программа рейтинговая, то конечно, продолжать.
Тема здоровья людей это вечная тема и всегда актуальна.

\begin{itemize} % {
\iusr{Людмила Денисенко}
\textbf{Фёдор Лимарев} А вот зря не смотрел!  @igg{fbicon.wink} 

\iusr{Фёдор Лимарев}
\textbf{Людмила Денисенко} я бы с удовольствием, но местные каналы у меня не настроены, а для ютьюба не хватает времени.
\end{itemize} % }

\iusr{Александр Зубов}
Движуха Это движение Духа! Медицина без интеллектуального креатива - тусклое ремесло! Переходите на второе дыхание!

\iusr{Марина Титенок}
Если откроется голосование, я однозначно напишу ДА!!!! Так хочется красоты, знаний, грамотности подачи с наших экранов!)

\iusr{Елена Лимарева}

Фёдор опередил меня! Поздно ночью уже не хотела формулировать. Однозначно
считаю, что стоит опираться на объективные показатели : рейтинг в динамике,
количество просмотров в соцсетях и прочие.

Современному искушённому телезрителю помимо основной информации нужна динамика,
смена яркой картинки, наличие аудио- и видеоряда и прочее, чтобы сделать
передачу менее статуарной. Ведущей ещё бы профессиональную команду и солидный
бюджет в помощь ...

\begin{itemize} % {
\iusr{Людмила Денисенко}
\textbf{Елена Лимарева} Команда у меня и так замечательная. С бюджетом - это чисто риторическое желание. Видеоряд, как по мне, в тему и без зауми. А вот по поводу динамики... Ну, разве что, как у Малышевой, дебильные костюмы ведущей и гостям...  @igg{fbicon.face.wink.tongue}  И изображать из себя сперматозоидов или язву...  @igg{fbicon.face.wink.tongue} 

\iusr{Елена Лимарева}
\textbf{Людмила Денисенко} , в таком случае стоит продолжать без оглядки на любое мнение.


\iusr{Людмила Денисенко}
\textbf{Елена Лимарева} Я просто не представляю, какая динамика может быть в беседе с доктором? Мы же не полчаса тупо обсуждаем, у нас сюжеты, вопросы, плазма. Что ещё? Да, хорошо бы иметь макет человеческого тела с внутренними органами, но это как раз в бюджет и упирается... А какая ещё динамика? Порекомендуй!

\iusr{Елена Лимарева}
\textbf{Людмила Денисенко} , нет уж, уволь. Я не обладаю специальными знаниями в области возможностей современного телевидения и могу судить только как зритель. Дилетантов и так достаточно во всех сферах деятельности.
Я поняла, что твой искренний вопрос к себе и зрителям вызван пониманием того, что нужно что-то менять , совершенствовать. Но, видимо, нужна была исключительно поддержка.
Я искренне поддерживаю тебя в желании просвещать и направлять людей в области доступных не специалисту медицинских знаний.
\end{itemize} % }

\iusr{Маргарита Фом}

После перезагрузки, пожалуйста возвращайтесь обязательно! Передача очень нужна.
Возможно у Вас появятся новые идеи, новые формы. Дончане должны знать, что у
нас есть хорошие врачи, серьезные ученые. Мне очень нравится, как Вы работаете
в кадре, как внимательно и дружелюбно ведете беседу. Особенно понравилась
передача в которой гостем программы была моя дочь....

\begin{itemize} % {
\iusr{Людмила Денисенко}
\textbf{Маргарита Фом} А кто дочь?  @igg{fbicon.heart.eyes}  Екатерина? Надо же, какой наш Донецк маленький!  @igg{fbicon.face.wink.tongue}  @igg{fbicon.face.flushed} 

\iusr{Маргарита Фом}
\textbf{Людмила Денисенко} Да! Приятно, что Вы догадались?

\iusr{Людмила Денисенко}
\textbf{Маргарита Фом} А я просто на фото Ваше внимательно посмотрела, Вы ведь очень похожи с дочкой! А она у Вас умница! Я с нею не только на передаче встретилась. Вернее, на передачу пригласила после того, как познакомилась в клинике, где она работает. Привела к ней внука. Очень внимательный и вдумчивый врач! И очень доброжелательный и открытый человек. Мама может гордиться!

\iusr{Маргарита Фом}
\textbf{Людмила Денисенко} @igg{fbicon.heart.red} Спасибо большое
\end{itemize} % }

\iusr{Дина Логвинова}

Рано или поздно, но этот пост должен был случиться. Человек творческий, а ты,
Людочка, из таких, - человек всегда сомневающийся и ищущий. Теперь, по
существу. Нужна ли в эфире программа о здоровье? Однозначно - ДА! Другой
вопрос, какой она должна быть. Ты, Людочка, предложила свой вариант, и
аудитория его приняла. Но если у тебя возникли сомнения продолжать ли дальше,
значит ты понимаешь, что нужно что-то менять. Телевидение – слишком живой
организм, и он требует постоянного развития. Потенциал для развития твоей
передачи есть. Прежде всего – он в твоих знаниях и компетенциях. Очень надеюсь,
что после каникул добрый доктор Денисенко через экраны телевизоров и мониторы
компьютеров вернется в наши дома и будет помогать нам быть здоровыми.

% -------------------------------------
\ii{fbauth.zhuravljev_vadim.doneck.dnr}
% -------------------------------------

Если кого-то интересует моё мнение, то считаю - продолжать надо!

\iusr{Елена Емельянова}
Однозначно, Да!

\iusr{Анна Пашкова}

Очень нравится ваша передача, интересно, беседа в дружеской атмосфере располагает
к доверию. Здорово было бы увидеть формат передачи, похожей на \enquote{я стесняюсь
своего тела} ( в более узкой направленности, конечно). Знаю многих
людей, которые, может из- за неграмотности, лени, страха, не доверия к нашей
медицине, но, в большей степени, из-за отсутствия денег, не идут к врачам, имея
достаточно серьезные проблемы со здоровьем, теряя драгоценное время. А после
пошаговой \enquote{инструкции} на примере такого же Дончанина и его истории у многих
появиться Надежда, та самая, прекрасная \enquote{женщина}, дающая силы идти вперёд )))


\iusr{Рашит Шехмаметьев}

Удивлен! Я думал, что будет новый сезон после устранения шероховатостей.
Делайте то, что нравится и что любите. Если нравится - делайте.

\iusr{Маргарита Фом}

\ifcmt
	ig https://scontent-frx5-1.xx.fbcdn.net/v/t39.30808-6/258544903_6546752052062697_9051341986222171725_n.jpg?_nc_cat=105&ccb=1-5&_nc_sid=dbeb18&_nc_ohc=1_X7phuWA7YAX_5M82y&_nc_ht=scontent-frx5-1.xx&oh=b7f51aa802a29d58ce23cb4d2d386bfc&oe=61A631DB
	@width 0.4
\fi

\iusr{Людмила Позднякова}

Нести в массы "медицинскую грамотность" это очень нужное дело! Мне нравится как
ты это делаешь грамотно, доходчиво! Гости, уверена, лучшие! После отдыха будет
видно, как фишка ляжет! Решение принимать придется самостоятельно! @igg{fbicon.heart.red}

\iusr{лариса орлова}

Люда, не буду писать заумные фразы, тут их, вижу, достаточно, я бы сказала, что
стоит после отдыха послушать свое сердце и душу -только свои - что они
подскажут..... советчиков много, чувства свои - вернее

\iusr{Ирина Киница}

Всё очень просто - отдохнуть с месяц, если выдержите, а потом решить.

% -------------------------------------
\ii{fbauth.poltorackaja_irina.torez.dnr.doneck.volonter}
% -------------------------------------



Людмилочка, продолжать необходимо, но соглашусь: нужна смена формата: выезды в
больницы, общение с больными и т.д.

\begin{itemize} % {
\iusr{лариса орлова}
\textbf{Ирина Ивановна} ну еще по больницам ездить Людмиле.... не уверена

\iusr{Ирина Ивановна}
\textbf{лариса орлова} это будет интересно и самой Людмиле.


\iusr{Людмила Денисенко}
\textbf{Ирина Ивановна} Не факт. У нас всё-таки передача о здоровье, а не о болезнях и не о здравоохранении.

\iusr{лариса орлова}
\textbf{Ирина Ивановна} не думаю
\end{itemize} % }

\iusr{Азизахон Ходжаева}

Ток шоу в медицинских передачах это не удачный формат. Вы владеете знаниями и
ими делитесь. Токать не надо, надо просвещать. Это всего лишь мнение. Я тоже
веду медицинскую передачу и всячески ухожу от формата ток шоу. Лучше приглашать
врачей, владеющих своим предметом и пусть они щедро делятся знаниями

\iusr{Инна Воля}

Продолжать обязательно!!!!! Всегда смотрела с интересом! А в виде шоу, мне
кажется, супер идея!! Отдохните, и с новыми силами просвещайте нас !!!! У Вас
очень интересно! @igg{fbicon.thumb.up.yellow} 

\iusr{Татьяна Михайлова}

Людмила, я повторюсь: надо продолжать. Год для программы это очень мало. Я
понимаю, что нужна поддержка команды, поощрение и вдохновение. Это очень важно.
Я верю, что все получится. Маленькая перезагрузка собраться с мыслями и
продолжать. Помните, мы когда-то говорили о формате с пациентами, и о худеющих
в программе? Тоже было бы классно. Но если вдруг чего... - сохраните эту идею
для личного блога)))

% -------------------------------------
\ii{fbauth.andrienko_julia.doneck.dnr.zhurnalist}
% -------------------------------------

Если не с вами делать передачи, то с кем??? Вы шикарный спикер - грамотный,
харизматичный, уверенный. Для любого телевидения было бы почётно делать
программу с вами. Так почему же нет!? И пусть это будет в нашем Донецке!

\iusr{Самвел Назаров}

Обязательно продолжать, Людмила. Мало на нашем ТВ нужных и грамотных передач.
Творческих сил Вам и вдохновения.

\end{itemize} % }
