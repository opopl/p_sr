% vim: keymap=russian-jcukenwin
%%beginhead 
 
%%file 23_08_2021.fb.uljanov_anatolij.1.nezavisimost_bezhenec
%%parent 23_08_2021
 
%%url https://www.facebook.com/dadakinder/posts/4652715388080921
 
%%author Ульянов, Анатолий
%%author_id uljanov_anatolij
%%author_url 
 
%%tags bezhenec,nezalezhnist,politika,ukraina
%%title Когда Украина праздновала 20 лет своей независимости, я уже был политическим беженцем
 
%%endhead 
 
\subsection{Когда Украина праздновала 20 лет своей независимости, я уже был политическим беженцем}
\label{sec:23_08_2021.fb.uljanov_anatolij.1.nezavisimost_bezhenec}
 
\Purl{https://www.facebook.com/dadakinder/posts/4652715388080921}
\ifcmt
 author_begin
   author_id uljanov_anatolij
 author_end
\fi

\begin{multicols}{2}
  
Когда Украина праздновала 20 лет своей независимости, я уже был политическим
беженцем. 10 лет спустя я по-прежнему беженец. 12 из 30 лет украинского
суверенитета я прожил в изгнании. И если вчера моя точка зрения была опасной,
то сегодня она ещё и уголовно наказуема. В Украине.

Я с интересом послушаю рассказы о репрессиях в советском прошлом, но мой личный
опыт, и моя собственная шкура знакомы с репрессиями в украинском настоящем. Не
сомневаюсь, что ситуация с правами и свободами в России или Беларуси – ужасна.
Но я – гражданин Украины, и меня первоочерёдно волнует ситуация в моей стране. 

Рассказы о демократичности украинского общества, возможно, убеждают тех, кто
никогда не сталкивался с Украиной, но не меня – человека, который с ней
столкнулся, и каждый день переживает последствия этого столкновения –
невозможность быть дома собой – быть своим и по-своему в своей стране.

Я – живое доказательство того, что никакой свободы в Украине нет. И таких, как
я – миллионы. Мы – свидетельство того, что в действительности представляет из
себя «демократия по-украински». И именно потому мы так раздражаем «патриотов»,
что уже одним своим бытием мешаем им развидеть реальность, отвернуться от неё,
не видеть это кладбище мечты и надежды. 

Я люблю людей. И знаю, что в любом, даже самом беспросветном болоте, можно
встретить порядочных, честных, добродушных. Но я не люблю Украину. Моя страна
не дала мне для этого повода. За что мне её любить? За отсутствие прав и
свобод? За принуждение к единомыслию и насилие за неправильную точку зрения? За
нищету, разрушенные перспективы и невозможность реализовать себя по-своему, а
не согласно ГОСТу? Что Украина может предложить молодому человеку, который
хотел бы жить в мультикультурном обществе, где разные – возможны, а не
наказуемы?

И, тем не менее, мы навсегда связаны. Следовательно, обречены на диалог с
родиной, которую можно не любить, но от которой невозможно сбежать. Потому что
она – внутри.

PHOTO: \href{https://www.facebook.com/oleg.yasinsky}{Oleg Yasinsky}

%\columnbreak

\begin{minipage}{0.5\textwidth}
\ifcmt
  ig https://scontent-cdg2-1.xx.fbcdn.net/v/t1.6435-9/240468713_4652707214748405_8402993565762829785_n.jpg?_nc_cat=108&_nc_rgb565=1&ccb=1-5&_nc_sid=730e14&_nc_ohc=MI7mZpiDTl0AX_G8OoF&_nc_ht=scontent-cdg2-1.xx&oh=e5090cad32594b0c3e16c58afd78ea0f&oe=614F9536
  %width 0.4
  tex \captionof{figure}{Памятник Ленину, Трезубец, Киев}
\fi
\end{minipage}

\end{multicols}

