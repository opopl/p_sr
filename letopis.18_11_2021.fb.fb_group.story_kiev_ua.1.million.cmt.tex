% vim: keymap=russian-jcukenwin
%%beginhead 
 
%%file 18_11_2021.fb.fb_group.story_kiev_ua.1.million.cmt
%%parent 18_11_2021.fb.fb_group.story_kiev_ua.1.million
 
%%url 
 
%%author_id 
%%date 
 
%%tags 
%%title 
 
%%endhead 
\zzSecCmt

\begin{itemize} % {
\iusr{Dmytro Nohal}

Чтобы эти расчеты работали, брежнев должен был быть до сих пор живым @igg{fbicon.face.grinning.sweat}  еще чуть
больше 30 лет назад сказка бы окончательно завершилась бы)))

\begin{itemize} % {
\iusr{Надежда Владимир Федько}
\textbf{Dmytro Nohal} У мене немає ностальгії по СРСР!

\begin{itemize} % {
\iusr{Dmytro Nohal}
\textbf{Nadegda Volodymyr Fedko} не вижу связи

\iusr{Надежда Владимир Федько}
\textbf{Dmytro Nohal} 

На той час (1969) перспектив розпаду СРСР я не бачив. Ілюзій щодо
\enquote{щасливого життя} також не було. Перспектив виїхати з країни не було -
я українець, а не єврей.

Залишалося жити, працювати, самовдосконалюватися, будувати сім'ю.

А розрахунок я зробив, оскільки в перший раз у житті побачив мільйон рублів))

\iusr{Dmytro Nohal}
\textbf{Nadegda Volodymyr Fedko} не еврей, дальше не интересно все, абсолютно.

\iusr{Надежда Владимир Федько}
\textbf{Dmytro Nohal} Нічого образливого я не сказав. З початку 1970-х право на еміграцію мали тільки євреї. І вони масово покидали СРСР.

\iusr{Dmytro Nohal}
\textbf{Nadegda Volodymyr} Fedko да да

\iusr{Нина Светличная}
\textbf{Надежда Владимир Федько} 

У сусідів на одного померлого єврея наприкінці 70-х рр. виїхало 7
українців-руських. Євреєм був голова родини - людина поважного віку, дружина -
українка. Діти з родинами - невістки, зяті, онуки. Яким чином він отримав візи
на всіх - не питайте, не знаю, але більше НІКОГО з євреїв в родині не було. І
от перед самим від"їздом він помирає! Поховали його чи не наступний день і ще
через день вся ця галаслива родина поїхала на \enquote{історичну Батьківщину} свого
єдиного єврейського родича.


\iusr{Надежда Владимир Федько}
\textbf{Нина Светличная} На цю тему ходив анекдот в ті часи... Але розповідати не буду, бо знову хтось образиться.

\iusr{Нина Светличная}
\textbf{Надежда Владимир Федько} Свого часу переконалась, що багато тих \enquote{анектодів} мали цілком реальних персонажей)

\iusr{Надежда Владимир Федько}
\textbf{Нина Светличная} 

Щодо еміграції 70-х маю цікавий спогад...

***

1974-й... Ми з майбутньою дружиною працювали в одному інституті, але в різних
відділах. Надійка в конструкторському, а я завідуючим фотолабораторією у
відділі науково-технічної інформації. В обідню перерву Надійка приходила до
мене в лабораторію і ми разом обідали і пили каву.

Якось Надійка сказала, що інженер із їхнього відділу, Давид Мойсейович,
підходив до неї і тихенько запитував, чи можна зі мною поговорити щодо якихось
фотографій. Я відповів їй, що можна. Ми попили чай с канапками, порозмовляли.
Надійка пішла працювати, я теж взявся за друк фотографій.

Хвилин через двадцять дзвінок в двері лабораторії... Спускаюсь, відчиняю... Біля
дверей стоїть чоловік років п’ятдесяти. Інтуїція підказує мені, що це і є Давид
Мойсейович.

Розмову починає здалеку... Що Надійка йому дуже подобається... Він знає, що ми
зустрічаємося і подали заяву в ЗАГС... Я терпляче чекаю коли він перейде до
справи.

Він говорить, що у нього дуже делікатна справа, яка повинна залишитися тільки
між нами обома і переходить до конкретики... Він, дружина – лікар і донька –
лікар, усією сім’єю збираються подавати документи на виїзд з СРСР в Ізраїль.
Йому потрібно дуже якісно сфотографувати усі документи: свідоцтва про
народження; паспорти; дипломи; військові квитки; профспілкові квитки; його
партійний квиток і комсомольський квиток доньки; різні посвідчення і довідки.
Хоча він має гарний німецький фотоапарат «Praktica super TL», але всі спроби
якісно сфотографувати документи були невдалими. У державну фотографію він,
звісно, піти не може.

Я мовчу і думаю... Оскільки наш інститут не займається проектуванням військової
техніки, то ризику потрапити під пильну увагу КДБ немає. Мабуть Давид
Мойсейович розцінює мою мовчанку трохи інакше – як обмірковування ціни роботи.
Тому він порушує мовчанку...

- За якісну роботу я готовий заплатити по 1 рублю за кожну сторінку документу.
Але мені потрібно два комплекти відзнятих і проявлених плівок.

Пахне солідним гонораром!

Домовляємося, що наступного дня він, не заходячи у свій відділ, занесе мені усі
документи, а потім вже піде до себе. Також попереджаю, щоб він нікому, навіть
Беллі (дружині), не розповідав про нашу домовленість!

***

Наступного ранку я приїхав на роботу раніше звичайного... За декілька хвилин до
дев’ятої прийшов Давид Мойсейович... Віддав мені пакет з документами. Домовилися,
що він зайде за відзнятими плівками о 16:30.

Я розклав документи по порядку, сфотографував усі один раз, потім повторно.
Репродукування документів зайняло години три.

[В одному тільки військовому квитку є 39 сторінок, плюс обкладинка. Три квитки
– вже 123 сторінки... Всього було сфотографовано 407 сторінок документів. 24
плівки по 36 кадрів кожна.]

Проявив плівки, промив водою, потім у спирті, повісив сушитися.

***

Прийшов Давид Мойсейович... Я взяв одну з плівок і вставив у фотозбільшувач... На
екрані чіткий документ, видно водяні знаки на папері. Переглядаємо ще декілька
документів... Все нормально, замовник задоволений!

Виходимо з темної кімнати... Я показую Д.М. на плівки і кажу, щоб він скручував
їх в рулончики. Він слухняно скручує плівки, кладе в пластмасові стаканчики, що
залишилися від касет з плівкою.

***

Давид Мойсейович витягає з кишені пачку 10-рубльовок і простягає мені...

- Я все порахував! Тут 40 штук, 400 рублів. Не будемо дріб’язковими!


\iusr{Нина Светличная}
\textbf{Надежда Владимир Федько} 

Оце халтурка підвалила) Дійсно, в домашніх умовах якісно зняти документи важко.
Але навіщо йому фотокопії?

\iusr{Надежда Владимир Федько}
\textbf{Нина Светличная} 

Я не запитував)) Пізніше, коли Д.М. виключили з партії і профспілки, він
приносив мені ще пару документів, одним з них була характеристика для ОВІРу.
Вона позитивно оцінювала його роботу в інституті, але в підсумку зазначалося,
що оскільки він виїжджає на постійне проживання в Ізраїль, то являється
\enquote{сіоністом} і \enquote{зрадником Батьківщини}.


\iusr{Нина Светличная}
\textbf{Надежда Владимир Федько} 

Чудово пам'ятаю цю процедуру \enquote{позбавлення всього}- партквитка, профквитка,
роботи, квартири. І при цьому має бути характеристика! В якому ж сюрі ми жили)

\iusr{Надежда Владимир Федько}
\textbf{Нина Светличная} 

Працюючи в архіві Київської області (ДАКО) я переглянув справи райкомів по
виключенню з партії тих, хто виїжджав... Були цікаві виключення. Наприклад,
виключають N з партії за те, що його донька виїжджає з чоловіком. Мотивація -
\enquote{утратил политическую бдительность}!


\iusr{Нина Светличная}
\textbf{Надежда Владимир Федько} 

Всі ці \enquote{органи} іноді ловили велику гаву) В 70-ті рр. у моєї подруги мати (для
мене тетя Таня) з вітчимом їхали на \enquote{історичну Батьківщину} - взагалі то в США.
А мати мала рідну сестру яка разом з чоловіком - УВАГА!! - працювала в Москві в
кремлівській лікарні! Уявляєте, які наслідки для них мав би цей від"їзд?? Що
робить тетя Таня - вона НІЧОГО не пише про сестру, в жодній анкеті. Взагалі! І
пройшло)

\end{itemize} % }

\end{itemize} % }

\iusr{Александр Асатуров}
И я видел.... сторублевыми купюрами.... ровно один метр

\iusr{Владимир Писаренко}

Зворушлива історія, але наскільки оманливими можуть бути наші мрії! Мільйону
рублів, не вкладених у валюту, нерухомість чи інші цінності, вистачило б на
небідне життя рівно до 1991 року. Але, за умови будь-якої конвертації, це вже
не був би мільйон рублів.))

\begin{itemize} % {
\iusr{Dmytro Nohal}
\textbf{Владимир Писаренко} уже в 88 не хватило бы на расчитанные сроки

\iusr{Светлана Мороз}
деньгами надо пользоваться -это инструмент для достижения целей а не икона Божей матери

\begin{itemize} % {
\iusr{Надежда Владимир Федько}
\textbf{Светлана Мороз} Так у моїх мріях і було ними користатися.

\iusr{Светлана Мороз}
\textbf{Надежда Владимир Федько} пользоваться СЕЙЧАС а не когда нибудь
\end{itemize} % }

\iusr{Надежда Владимир Федько}
На початку 90-х ми пішли в ресторан і нам це обійшлося трохи більше 2 мільйонів
купонів... Розплачувався я купюрами по 200 000.

\iusr{Александр Костюкевич}
а еще счастливие прожил с лямом.. @igg{fbicon.smile} ... но это другая история...

\begin{itemize} % {
\iusr{Надежда Владимир Федько}
\textbf{Александр Костюкевич} Не в деньгах счастье! А в их количестве))

\iusr{Александр Костюкевич}
И в своевременности...

\iusr{Надежда Владимир Федько}
\textbf{Александр Костюкевич} Це точно!

\end{itemize} % }

\end{itemize} % }

\iusr{Сергій Євдоченко}
Чудова людина....

\iusr{Виктор Николаев}
Генадий головкин

\iusr{Олена Шелест}
Дякую. Цікава публікація.

\iusr{Александр Петренко}
\enquote{Я прожив 50 років щасливо і без мільйона.} А был бы тот мульён возможно и не прожил бы.

\iusr{Надежда Владимир Федько}
\textbf{Александр Петренко} Цікавий поворот))

\end{itemize} % }
