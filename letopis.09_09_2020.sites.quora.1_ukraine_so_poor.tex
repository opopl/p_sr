% vim: keymap=russian-jcukenwin
%%beginhead 
 
%%file 09_09_2020.sites.quora.1_ukraine_so_poor
%%parent 09_09_2020
%%url https://www.quora.com/Why-is-Ukraine-so-poor
 
%%endhead 

\subsection{Quora - Why is Ukraine so poor}
\label{sec:09_09_2020.sites.quora.1_ukraine_so_poor}

\url{https://www.quora.com/Why-is-Ukraine-so-poor}

It’s not.

Ukraine could be considered poor only within international trade. If average
Ukrainian tries to live in EU or USA on his Ukrainian salary, of course he
couldn’t afford it. But Ukrainians spend their salary in Ukraine, and for most
part, the prices are very low, so all the basic needs are satisfied with even
the minimal salary.

What is most important to consider is land and agriculture. Ukraine is big, and
any citizen can receive a piece of land for free from the government. Its size
depends on what he’s planning to do with it, but if he wants to use it for
farming or agriculture, he could get a large field. But even the smallest land
he gets is more than enough for a family to support itself - an average village
can be completely autonomous. Note that Ukraine soil is highly fertile, with
two-thirds of all the soil being chernozem, which is considered the most
fertile type of soil in the world.

So not only Ukraine can fully support itself with food with even current sloppy
organization (we could feed five times current population), it is among the
world top in agriculture export. In the world, Ukraine production amount is \#1
in sunflower, \#4 in potatoes, \#5 in honey, \#4 in pumpkin, \#3 in currant,
and is among top-10 wheat and rye producers (and corn, if I’m not mistaken).

Food prices are extremely low. For minimal monthly salary, I can buy 500 kg of
wheat and 25 kg of beef. Or I can just move to countryside and make it all by
myself, trading with other peasants for everything I don’t focus on producing -
I can just grow potatoes and chickens, exchanging eggs and chicken meat for
beef, milk, honey, and moonshine, and selling what is left on a local market.

\ifcmt
	pic https://qph.fs.quoracdn.net/main-qimg-c97109d3294300d5537d4687fcd6e20a
	pic https://qph.fs.quoracdn.net/main-qimg-56369c82bbe1c46eed35bf1f8d15ec89
\fi

Electricity prices are low - it takes no more than 10\% of minimal wage to pay
for the bills. Internet prices are ridiculously low.

Housing prices depends on where you live, but outside the capital and largest
cities it would cost about 25\% of an average salary to rent a flat, or even a
small house. You can even buy a house for less than 1000\$ in the countryside.

But those who live in cities usually live in their own apartment with their
spouse or family, since during Soviet times almost every family received a flat
from the government, and most families own a dacha in the countryside, where
they spend weekends and holidays.

\ifcmt
	pic https://qph.fs.quoracdn.net/main-qimg-dbe32c0b22a05622fb923647a55e2827
\fi

There is one thing Ukraine, along with other post-soviet countries, is actually
ahead of most “rich” countries, especially USA. Free healthcare. This is a part
of soviet legacy, where healthcare and education were considered a basic need
that must be provided to everyone for free. It is something so deeply fixed
within Soviet people’s mind that no capitalism could take away - it’s just
unthinkable for people even born after the USSR collapsed that basic healthcare
and education are not secured and provided by the government. It would be the
end of career for any politician who tried to take away free healthcare. Yes,
the state hospitals are not the best, the doctors can be rude, walls are shabby
and tiles on the floor are cracked, but everyone gets all the basic treatment
for free. They wont charge you for calling an ambulance, for being
hospitalized, or talking to a doctor. I’ve never in my lifetime paid anything
to see a physician. When I had my appendix removed, they only asked me to buy
syringes, a mask for a surgeon and sterile gloves, but I didn’t have to. After
the surgery, I was put in a ward and for the week I was there they brought me
food three times a day, which was basically porridge or soup, and the doctor
visited me few times a day. When I broke my leg, an ambulance took me to a
hospital where they did an x-ray, set me a plaster cast and I only paid for the
x-ray film. I also had a few minor surgeries and only paid for alcohol wipes.
When I was a kid, every time I felt sick my mom called our family doctor and
she came within few hours. We used to thank her with box of chocolate or a
bottle of liquor, which is traditional gift for doctors in ex-USSR (a doctor is
guaranteed to have a lifetime supply of sweets and alcohol).

Healthcare is not completely free - you still have to buy medicine, and you
have to pay for dental services (only if it’s an emergency they’ll remove your
tooth for free). But I honestly believe state healthcare is much better in
Ukraine and ex-USSR than it is in USA, where getting sick is always expensive
and calling an ambulance is considered a big decision. But still, most of the
pharmacy is produced in Ukraine and is relatively cheap, and seniors sometimes
can claim up to 50\% discount. But given that visiting a doctor is free, it is
possible to just not getting sick in the first place, monitoring your condition
and do what the doctor tells you to stay healthy.

\ifcmt
	pic https://qph.fs.quoracdn.net/main-qimg-73f931f41d6691c893712e1bc459be9b
\fi

General education is free. Once in a while the teachers will ask parents to
give some money for building renovation and other minor stuff, and though it is
not legal, it is generally accepted. After you finish school, you can pass a
test, and if your results are above average, you can apply to a state
university in any city, provided with dormitory room and stipend. Many state
universities give you a very decent education. Note that Ukraine was USSR
richest and wealthiest republic, with some of the top technical facilities and
universities being here (soviet rockets and planes were built in Ukraine).

Though liberal education quality was average in USSR, the technical education
was very serious, and its momentum still holds. If I were to receive education
in physics, math or engineering, I would apply to Igor Sikorsky Kyiv
Polytechnic Institute (KPI) without any hesitation.

\ifcmt
	pic https://qph.fs.quoracdn.net/main-qimg-daa2c1bb35ff38ae1c5497cfaba4b084
\fi

KPI territory is so big and has so many buildings, It would take 20 minutes to
go from one end to another.

If you asked an average Ukrainian, 9 of 10 will tell you they are barely
hanging on, the country is a disaster, the whole parliament should be hanged
and all the ministers put in prison without trial. But if you visit their
homes, their fridges would be full, their closets wouldn’t hold all their
dresses, their kids would be pink and happy and there would be a plasma TV on
the wall in at least two rooms.

Now, here’s an independent opinion I would value more.

This is Hitoshi Nakamura.

\ifcmt
	pic https://qph.fs.quoracdn.net/main-qimg-19d050a77101bcd3af4e18d624daa87f
\fi

He’s the chief of Japanese-Ukrainian Cultural Center, which is located on the
third floor of KPI’s library.

Nakamura-san was born in Tokyo and came to Ukraine in the early 2000s and
studied in the Ukraine’s top university, Taras Shevchenko National University
of Kyiv:

\ifcmt
	pic https://qph.fs.quoracdn.net/main-qimg-8a1324ca7a1e65f43e6fd78bc7552fbc
\fi

He speaks fluent Russian and a bit of Ukrainian. He’s the library manager, and
for 10 years already he’s there almost every day behind the counter.
Nakamura-san is polite, intelligent and for those who study Japanese or learn
its culture he is the one who represents Japan in Ukraine. I had a few
conversations with him as I was curious about his view on Ukraine. And one
thing he said I remember most. He said “you don’t even realize how rich you,
Ukrainians, are; you have everything, and many things that you all have and
don’t even appreciate, for us, Japanese, would be a luxury - the vast lands,
the food that is almost free, the house in a countryside that every family owns
and the people themselves, all of it outweighs everything else which is
fixable.” I completely agree with him.

Personally, I wouldn’t want to live anywhere else and those of my friends who
fled to EU for money I consider ignorant.

So, in terms of the number of iPhones Ukrainians can buy, or how much Russian
gas we can pour into our expensive foreign cars - yes, Ukraine is poor. But in
terms of living - we are without doubt above most of the world.

And hey - the pot is also cheap here. ;)

(pls point out grammar and punctuation errors in comments)
