%%beginhead 
 
%%file 29_04_2022.fb.galushko_denis.1.kolis__tse_vse_zak_n
%%parent 29_04_2022
 
%%url https://www.facebook.com/HalushkoDenis/posts/pfbid02iXVWcZ6m7HKBGPgcYejz5iVBDmNyYf58Q6UJeoWmHUQ2GkLBy6fn6i3jBgW9THdsl
 
%%author_id galushko_denis
%%date 29_04_2022
 
%%tags 
%%title Колись, це все закінчиться…
 
%%endhead 

\subsection{Колись, це все закінчиться...}
\label{sec:29_04_2022.fb.galushko_denis.1.kolis__tse_vse_zak_n}

\Purl{https://www.facebook.com/HalushkoDenis/posts/pfbid02iXVWcZ6m7HKBGPgcYejz5iVBDmNyYf58Q6UJeoWmHUQ2GkLBy6fn6i3jBgW9THdsl}
\ifcmt
 author_begin
   author_id galushko_denis
 author_end
\fi

Колись, це все закінчиться…

Але мені вже дивно, що люди кажуть, ми повернемося до свого звичного життя, ми
будемо так само жити як і до...

Пишу це не для того щоб в вас прокинулося якесь відчуття провини, за будь які
ваші дії або бездіяльність в ці важкі для кожного з нас часи.

Прекрасно розумію, ми різні, комусь було і необхідно врятувати тварину, хтось
рятував чуже, але людське життя. 

Я зустрічав і тих і тих, на нулі, повірте на слово мені, ризик для власного
життя там був однаковий. 

Хтось рятувався сам, забувши про усі зв’язки які пов’язували його до
повномасштабного наступу ворога, не засуджую, просто байдуже за тих людей, так
само як і їм за всіх.

Але я за інше, так ми різні, та чи будемо ми такими самими, після війни?

Мені іноді навіть огидно про це думати, що ті люди які складали моє оточення,
мої друзі, мої колеги з ресторанного бізнесу, мої рідні, якось не збираються
мінятися після цієї війни, а просто хочуть повернутися до буденного життя, щоб
все було як раніше.

Я не маю зараз часу знаходячись на сході пояснювати, що це не можливо, бізнес
мовою, як звик, за різних макроекономічних показників.

Невідомо скільки ще буде шкоди, але та що заподіяна для нашої економіки і людей
у масштабах країни і так достатньо, щоб відновлюватися не один десяток років.  

А за інше, за той саме ваш побут, ваше будене життя.

Його може змінити кожен і варто за це подбати зараз, бути українцем, це не
вишиванка, на день вишиванки. 

Ваші діти і далі будуть підписані на російські ютюб канали і тік-ток сторінки і
говорити переважно російською?

Ви і далі будете продовжувати дивитися в сторону російських акторів, співаків,
реперів, слухати, читати, надихатися і вихвалятися один перед одним як якийсь з
«ліберальних» поглядів покидьок, і «нашим і вашим» виїхав з росії і тепер несе
свою дичину з закордону, який він «герой», постав проти режиму.

Для мене константа, з 14- го року, записувати людей в герої які не тільки
словом, а і своєю справою доказали, хто вони з ким вони і за що вони боряться в
своєму житті, аби ви мали ту свободу на яку заслуговує кожна людина в цьому
світі.

А ви і далі накидаєте на себе це ярмо, вам треба «хазяїн думок», ви думаєте
вашим дітям треба «хазяїн» їх думок і дій в майбутньому?

Навіщо ви породжуєте далі це? 
