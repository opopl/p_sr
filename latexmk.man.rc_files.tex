% vim: keymap=russian-jcukenwin
%%beginhead 
 
%%file man.rc_files
%%parent man.errors
 
%%endhead 
  
\section{CONFIGURATION/INITIALIZATION (RC) FILES}

\vspace{0.5cm}
 {\ifDEBUG\small\LaTeX~section: \verb|man.rc_files| project: \verb|latexmk| rootid: \verb|p_saintrussia| \fi}
\vspace{0.5cm}
  
In this section is explained which configuration files are read by latexmk.
Subsequent sections "How  to  Set  Variables  in  Initialization Files",
"Format  of  Command  Specifications",  "List of Configuration Variables Usable
in Initialization Files", "Custom  Dependencies",  and "Advanced Configuration"
give  details  on what can be configured and how.

Latexmk can be customized using initialization files, which are read at startup
in the following order:

\begin{itemize}
	\item 1) The system RC file, if it exists.
On a UNIX system, latexmk searches for following places for its system RC file, in the following order, and reads the first it finds:
\begin{itemize}
  \item \verb|"/opt/local/share/latexmk/LatexMk"|,
  \item \verb|"/usr/local/share/latexmk/LatexMk"|,
  \item \verb|"/usr/local/lib/latexmk/LatexMk"|.
\end{itemize}

On a MS-Windows system it looks for \verb|"C:\latexmk\LatexMk"|.  On a cygwin
system (i.e., a MS-Windows system in which Perl is  that of
cygwin), latexmk reads the first it finds of

\begin{itemize}
  \item \verb|"/cygdrive/c/latexmk/LatexMk"|,
  \item \verb|"/opt/local/share/latexmk/LatexMk"|,
  \item \verb|"/usr/local/share/latexmk/LatexMk"|,
  \item \verb|"/usr/local/lib/latexmk/LatexMk"|.
\end{itemize}

In addition, it then tries the same set of locations, but with the file
name replaced "LatexMk" replaced by "latexmkrc".

If the environment variable \verb|LATEXMKRCSYS| is set, its value is  used  as
the name of the system RC file, instead of any of the above.

	\item 2) The user's RC file, if it exists.  This can be in one of two places.
The traditional one is \verb|".latexmkrc"| in the user's home
directory.   The other  possibility  is "latexmk/latexmkrc" in
the user's XDG configuration home directory.  The actual file
read is the first  of
\verb|"$XDG_CONFIG_HOME/latexmk/latexmkrc"|  or
\verb|"$HOME/.latexmkrc"| which exists.  (See
\url{https://specifications.freedesktop.org/basedir-spec/basedir-spec-latest.html}
for details on the XDG Base Directory Specification.)

Here  \verb|$HOME|  is  the  user's  home  directory.  [Latexmk determines the
user's home directory as follows:  It is the value of  the
environment variable  HOME,  if this variable exists, which
normally is the case on UNIX-like systems (including Linux and
OS-X).  Otherwise  the  environment  variable USERPROFILE is
used, if it exists, which normally is the case on MS-Windows
systems. Otherwise a blank string is used instead of \verb|$HOME|, in
which case latexmk does not look for an RC file in it.]

\verb|$XDG_CONFIG_HOME|  is  the  value  of  the environment variable
\verb|XDG_CONFIG_HOME| if it exists.  If this environment variable
does  not  exist, but  \verb|$HOME|  is  non-blank,  then
\verb|$XDG_CONFIG_HOME| is set to the default value of \verb|$HOME/.config|.
Otherwise \verb|$XDG_CONFIG_HOME| is blank,  and  latexmk does not
look for an RC file under it.

\item 3)  The  RC  file  in  the current working directory.  This file can be
named either \verb|"latexmkrc"| or \verb|".latexmkrc"|, and the first of these to  be
found is used, if any.

\item 4) Any RC file(s) specified on the command line with the -r option.

Each RC file is a sequence of Perl commands.  Naturally, a user can use this in
creative ways.  But for most purposes, one simply  uses  a
sequence of assignment statements that override some of the
built-in settings of Latexmk.  Straightforward cases can be
handled without  knowledge  of  the  Perl  language by using
the examples in this document as templates.  Comment lines are
introduced by the \verb|"#"| character.

Note that command line options are obeyed in the order  in  which  they are
written; thus any RC file specified on the command line with the \verb|-r|
option can override previous options but can be  itself overridden  by later
options on the command line.  There is also the \verb|-e| option, which allows
initialization code to be specified in latexmk's command line.

For possible examples of code for in an RC file, see the directory
\verb|example_rcfiles|  in  the  distribution  of  latexmk (e.g., at
\url{http://mirror.ctan.org/support/latexmk/example_rcfiles}).
\end{itemize}

