% vim: keymap=russian-jcukenwin
%%beginhead 
 
%%file 26_11_2020.news.ua.fakty.bahareva_taisia.1.komarov
%%parent 26_11_2020
 
%%url https://fakty.ua/361729-gospodi-skolko-ecshe-vyderzhim---otkroveniya-dmitriya-komarova-o-proekte-mir-naiznanku
 
%%author Бахарева, Таисия
%%author_id bahareva_taisia
%%author_url 
 
%%tags 
%%title «Господи, сколько еще выдержим?!» --- откровения Дмитрия Комарова о проекте «Мир наизнанку»
 
%%endhead 
 
\subsection{«Господи, сколько еще выдержим?!» --- откровения Дмитрия Комарова о проекте «Мир наизнанку»}
\label{sec:26_11_2020.news.ua.fakty.bahareva_taisia.1.komarov}
\Purl{https://fakty.ua/361729-gospodi-skolko-ecshe-vyderzhim---otkroveniya-dmitriya-komarova-o-proekte-mir-naiznanku}
\ifcmt
	author_begin
   author_id bahareva_taisia
	author_end
\fi

13:40 --- 26 ноября 2020, Таисия БАХАРЕВА, «ФАКТЫ»

\ifcmt
pic https://fakty.ua/photos2/article/36/17/361729w540zc0.jpg?v=231800
caption Дмитрий Комаров
\fi

\index[names.rus]{Комаров, Дмитрий!Путешественник, Киев}

\begin{leftbar}
	\bfseries
За десять лет популярного проекта «Мир наизнанку» («1+1») \textbf{известный
путешественник Дмитрий Комаров} лишь однажды сделал большой перерыв в работе —
из-за пандемии коронавируса. И хотя ситуация с COVID-19 во всем мире остается
очень напряженной, Комаров в начале осени вместе с неизменным оператором
проекта Александром Дмитриевым вновь отправился в путешествие. Страну, в
которой проходят съемки, Дмитрий пока не называет, признался только, что это
уникальное место.
\end{leftbar}

\subsubsection{«Инвестировал личные сбережения в первые экспериментальные съемки»}

--- Дмитрий, не каждый телепроект может похвастаться историей длиной
в десять лет.

--- Наверное, секрет в том, что «Мир наизнанку» для меня не работа, это --- моя
жизнь. Как я говорю, ребенок, который уже сильно повзрослел. Хотя на самом деле
возраст --- вещь условная. Если проект не развивается, то 10 лет --- это много. Но
когда постоянно появляются новые идеи, есть мечты, цели и желания, то совсем
немного. А ведь в мире есть такое количество мест, где я еще не бывал, что моих
планов хватит надолго. Так что в случае с «Миром наизнанку» 10 лет --- просто
круглая дата.

\ifcmt
pic https://fakty.ua/user_uploads_new/images/articles/2020/11/26/361729/%D0%94%D0%A0%20%D0%9C%D0%B8%D1%80%D0%B0%20%D0%BD%D0%B0%D0%B8%D0%B7%D0%BD%D0%B0%D0%BD%D0%BA%D1%83%202016%20%D0%BF%D1%80%D0%B0%D0%B7%D0%B4%D0%BD%D0%BE%D0%B2%D0%B0%D0%BB%D0%B8%20%D0%B2%20%D0%9D%D0%B5%D0%BF%D0%B0%D0%BB%D0%B5.JPG
caption В 2016 году день рождения «Мира наизнанку» праздновали в Непале
\fi


--- Помните свои мысли, переживания, когда все только начиналось?

--- Конечно! Сказать по правде, меня тогда вообще не интересовали ни финансовые
вопросы, ни ежедневный график. Я готов был жить-спать-есть на работе, только
чтобы делать «Мир наизнанку». Все, что меня волновало, --- это сам проект.
Хотелось создать что-то новое, уникальное, чего на тот момент наш зритель еще
не видел. Я инвестировал свои личные сбережения в первые экспериментальные
съемки. При этом совершенно не зная ничего о том, как создаются телепрограммы.
Действовал наудачу. Снимал так, как, на мой взгляд, было бы интересно смотреть.

\ifcmt
pic https://fakty.ua/user_uploads_new/images/articles/2020/11/26/361729/%D0%98%D0%BD%D0%B4%D0%B8%D1%8F,%20%D1%81%D1%8A%D0%B5%D0%BC%D0%BA%D0%B8%20%D0%BE%D0%B4%D0%BD%D0%BE%D0%B3%D0%BE%20%D0%B8%D0%B7%20%D0%BF%D0%B5%D1%80%D0%B2%D1%8B%D1%85%20%D1%81%D0%B5%D0%B7%D0%BE%D0%BD%D0%BE%D0%B2%20%D0%BF%D1%80%D0%BE%D0%B3%D1%80%D0%B0%D0%BC%D0%BC%D1%8B.jpg
caption Индия, съемки одного из первых сезонов программы
\fi

--- Было тяжело?

--- Когда начинал снимать первые сезоны, действительно, было очень сложно.
Знаете, в моменты, когда ты вдали от дома, иногда посещали мысли:
«Господи, сколько еще выдержим? Как же это тяжело». Помню, как
мы с оператором зачеркивали дни на календарике и мечтали побыстрее
вернуться домой. Но после возвращения через какое-то время снова
появлялось желание куда-нибудь поехать.

--- Стали зависимы?

--- Именно. Кстати, коронавирус и пандемия только подтвердили это. Я никогда
так долго не находился дома. И просто реально чувствовал физическую тягу,
чтобы как можно быстрее взять свой рюкзак, чемодан и отправиться
в экзотическую страну снимать новый сезон. Да, у меня зависимость
от съемок, от проекта, но я это очень люблю.

\ifcmt
pic https://fakty.ua/user_uploads_new/images/articles/2020/11/26/361729/%D0%9A%D0%B8%D1%82%D0%B0%D0%B9.JPG
caption  В Китае
\fi

\subsubsection{«Нам с Сашей Дмитриевым для общения часто даже не нужно слов»}

--- С вами в экспедициях всегда находится еще один человек, который часто
остается за кадром, --- оператор Александр Дмитриев. Легко вместе работать?

--- Наши взаимоотношения с Сашей я бы назвал идеальным тандемом. Нам для
общения часто даже не нужно слов. Можем едва заметным для окружающих
жестом подать друг другу сигнал. Микродвижение пальца, брови, глаза —
и Саня понимает, что нужно поменять объектив, переместиться, включить
запись или срочно ее остановить.

\ifcmt
pic https://fakty.ua/user_uploads_new/images/articles/2020/11/26/361729/%D0%94%D0%B8%D0%BC%D0%B0%20%D0%9A%D0%BE%D0%BC%D0%B0%D1%80%D0%BE%D0%B2%20%D0%B8%20%D0%A1%D0%B0%D1%88%D0%B0%20%D0%94%D0%BC%D0%B8%D1%82%D1%80%D0%B8%D0%B5%D0%B2%20%D0%9D%D0%B5%D0%BF%D0%B0%D0%BB.JPG
caption  Дмитрий Комаров и Александр Дмитриев. Непал
\fi

--- И никаких спорных моментов?

--- Если взять целый год, 365 дней, то мы с Сашей проводим вместе больше
времени, чем со своими семьями. И в экспедициях видимся каждый день,
потому что работаем без выходных. Например, в Бразилии провели 204
съемочных дня, периодически возвращаясь в Украину. Да, иногда это
непросто. Но то, что мы продолжаем работать столько лет, доказывает, что
наши рабочие взаимоотношения налажены идеально.

Конечно, случаются какие-то острые моменты на съемках, разногласия или
эмоциональный разговор. Но это рабочий процесс. Никто не обижается,
поскольку мы оба знаем, что хотим сделать работу идеально. Если днем
возникает ситуация, когда пришлось пообщаться на повышенных тонах, то уже
вечером мы о ней вспоминаем с улыбкой.

\ifcmt
pic https://fakty.ua/user_uploads_new/images/articles/2020/11/26/361729/%D0%94%D0%B8%D0%BC%D0%B0%20%D0%9A%D0%BE%D0%BC%D0%B0%D1%80%D0%BE%D0%B2%20%D0%B8%20%D0%A1%D0%B0%D1%88%D0%B0%20%D0%94%D0%BC%D0%B8%D1%82%D1%80%D0%B8%D0%B5%D0%B2%20%D0%B2%20%D0%9A%D0%B8%D1%82%D0%B0%D0%B5.jpg
caption Дмитрий Комаров и Александр Дмитриев. Китай
\fi

--- Наверное, вы чаще бывали в джунглях или горах, чем в комфортных для
жизни мегаполисах. Как справляетесь с бытовыми трудностями в путешествиях?

--- Мы настолько привыкли, что не обращаем внимания на бытовые неурядицы.
Нам абсолютно все равно, где спать --- в гамаке, в гостевом доме
на неудобном матрасе или в отеле в городе. Конечно, больше любим, когда
получается ночевать в удобных отелях, но это быстро забывается.
Запоминаются какие-то особенные места, например, ночевка в доме на дереве
в неконтактном племени на острове Новая Гвинея, или в Амазонии, когда
спишь и боишься, что может схватить анаконда. У меня в голове «винегрет»
из отелей. Я уже не отличу гостиницу в Китае от отеля где-нибудь
в бразильском городе.

--- Вы сравнили «Мир наизнанку» с десятилетним ребенком. А каким был Дима
Комаров в свои 10 лет?

--- Я хорошо себя помню с пятилетнего возраста, эпизодически —
с трехлетнего. А какие-то яркие вспышки, моменты зафиксировались в памяти
даже с более раннего возраста. Помню, например, как выступал в садике,
когда мне было два года. Я читал стих. Хорошо помню апрель 1986 года,
когда произошла трагедия в Чернобыле. Мы летели к бабушке практически
в пустом самолете. Я шел по салону и говорил: «Электростанция взорвалась».
Естественно, тогда еще не понимал масштаба трагедии. Уже в десять лет
я обожал фотографировать и бегал с пленочной любительской видеокамерой.
Вместе с папой мы делали какие-то ролики. Помню, как снимал в Киевском
зоопарке животных. Потом проявляли эти пленки и на проекторе дома
смотрели.

\ifcmt
pic https://fakty.ua/user_uploads_new/images/articles/2020/11/26/361729/%D0%94%D0%B8%D0%BC%D0%B0%20%D0%9A%D0%BE%D0%BC%D0%B0%D1%80%D0%BE%D0%B2%20%D0%B2%20%D0%B4%D0%B5%D1%82%D1%81%D1%82%D0%B2%D0%B5.jpeg
caption  Дима Комаров в детстве
\fi

--- Мечтали о путешествиях?

--- Конечно! Я вырос на фильмах Юрия Сенкевича, смотрел Стива Ирвина --- это
были мои любимые телепрограммы. Обожал фильмы про Индиану Джонса, вообще
все картины, где что-нибудь происходило в джунглях. Исследователи,
путешественники летали на самолетах, открывали новые места, искали
сокровища и попадали в какие-то племена. От всего этого я приходил
в восторг!

«Готов вывернуться наизнанку, чтобы оказаться там, где наметил, и снять то, что
хочу»

--- Немногие знают, что у вас газетная закалка.

--- С 17 лет я работал журналистом --- как папарацци, корреспондент и фотограф
ежедневной газеты. Постоянно находился в гуще событий: политических,
экономических, даже трагических. Тогда интернет был еще не так сильно
развит, нужно было всю информацию и фотографии добывать лично. Были
экстремальные и даже секретные задания. Так что у меня старая газетная
школа.

Приобретенный тогда навык быстро принимать решения помогает и на съемках
«Мира наизнанку». Как правило, мой каждый день в экспедиции отличается
тем, что происходят какие-то непредвиденные ситуации. И нет времени долго
думать --- я должен действовать, мгновенно решать, как подстроиться под
новые обстоятельства.

--- Эти экстремальные ситуации остаются за кадром «Мира наизнанку»?

--- Что-то входит в выпуски, но многое просто не помещается в формат
программы. Со мной происходило столько курьезного, оставшегося за кадром,
что может хватить на целый фильм или книгу. Думаю, даже не одну.
Естественно, если когда-нибудь мой график позволит этим заняться. Моя
жизнь очень насыщенна. Мы с Сашей Дмитриевым часто говорим, что за год
в экспедициях испытываем столько эмоций, сколько в обычной жизни
не получишь и за 10 лет.

\ifcmt
pic https://fakty.ua/user_uploads_new/images/articles/2020/11/26/361729/%D0%98%D0%BD%D0%B4%D0%BE%D0%BD%D0%B5%D0%B7%D0%B8%D1%8F,%20%D0%BF%D1%80%D0%BE%D0%B2%D0%B8%D0%BD%D1%86%D0%B8%D1%8F%20%D0%9F%D0%B0%D0%BF%D1%83%D0%B0,%20%D1%81%D1%82%D1%80%D0%BE%D0%B8%D1%82%D0%B5%D0%BB%D1%8C%D1%81%D1%82%D0%B2%D0%BE%20%D0%B4%D0%BE%D0%BC%D0%B0%20%D0%BD%D0%B0%20%D0%B4%D0%B5%D1%80%D0%B5%D0%B2%D0%B5.JPG
caption Индонезия, провинция Папуа. Строительство дома на дереве
\fi


--- Какая из ваших экспедиций была самой дорогостоящей?

--- Если сравнивать страны по стоимости жизни, то самой дорогой была Япония.
Что касается съемок, то это зависит не от страны, а от контента, который
делаешь, и от сложившейся ситуации. Самые дорогие съемки, уверен, еще
впереди. Потому что чем дольше снимаешь, тем более экстремальные
и необычные сюжеты приходится придумывать, а для их реализации нередко
требуются приличные деньги. Например, на организацию, информаторов,
этнографов, историков, которые помогут разобраться и найти то, чего нет
на поверхности и о чем не пишут в интернете.

--- Вы по нескольку месяцев живете в одной стране. Успеваете хоть немного
выучить местный язык?

--- В каждой поездке учим основные фразы на языке народа, о котором снимаем.
Вне зависимости от того, работаем ли в Китае, где полтора миллиарда
населения, или в племени бывших людоедов, язык которого знают 100 человек
на планете. Это наше правило. Набор базовых фраз и слов --- «здравствуйте»,
«до свидания», «спасибо», «пожалуйста», «мы вам благодарны» --- помогает
наладить коммуникацию. Людям очень приятно услышать что-то на своем языке
от иностранца.

\ifcmt
pic https://fakty.ua/user_uploads_new/images/articles/2020/11/26/361729/%D0%98%D0%BD%D0%B4%D0%BE%D0%BD%D0%B5%D0%B7%D0%B8%D1%8F,%20%D0%BF%D1%80%D0%BE%D0%B2%D0%B8%D0%BD%D1%86%D0%B8%D1%8F%20%D0%9F%D0%B0%D0%BF%D1%83%D0%B0.JPG
caption Индонезия, провинция Папуа
\fi


--- Есть места на планете, куда вам так и не удалось попасть?

--- Признаюсь, я готов вывернуться наизнанку, чтобы оказаться там, где
наметил, и снять то, что хочу. Наш девиз: если закрыта дверь, мы заходим
через окно. Для нас нет слова «невозможно». Всегда пытаемся снять через
эти «невозможно», «нельзя» и «запрещено».

\ifcmt
pic https://fakty.ua/user_uploads_new/images/articles/2020/11/26/361729/%D0%94%D0%BC%D0%B8%D1%82%D1%80%D0%B8%D0%B9%20%D0%9A%D0%BE%D0%BC%D0%B0%D1%80%D0%BE%D0%B2%20%D0%B8%20%D0%90%D0%BB%D0%B5%D0%BA%D1%81%D0%B0%D0%BD%D0%B4%D1%80%D0%B0%20%D0%9A%D1%83%D1%87%D0%B5%D1%80%D0%B5%D0%BD%D0%BA%D0%BE%20(1).JPG
caption Дмитрий Комаров и Александра Кучеренко
\fi



