%%beginhead 
 
%%file 24_03_2023.fb.korovanenkova_natalia.mariupol.1.podval_24_03_2022
%%parent 24_03_2023
 
%%url https://www.facebook.com/natali.korovanenkova/posts/pfbid0ACdA1DQfioS7LfvKfn4iUckXKqtKTa9pQVAxrq7onBYAb4Z1JgFmMh6Hyr34Eh3Hl
 
%%author_id korovanenkova_natalia.mariupol
%%date 24_03_2023
 
%%tags mariupol.war,mariupol,dnevnik,24.03.2022
%%title Подвал. 24.03.2022
 
%%endhead 

\subsection{Подвал. 24.03.2022}
\label{sec:24_03_2023.fb.korovanenkova_natalia.mariupol.1.podval_24_03_2022}

\Purl{https://www.facebook.com/natali.korovanenkova/posts/pfbid0ACdA1DQfioS7LfvKfn4iUckXKqtKTa9pQVAxrq7onBYAb4Z1JgFmMh6Hyr34Eh3Hl}
\ifcmt
 author_begin
   author_id korovanenkova_natalia.mariupol
 author_end
\fi

\#маріуполь\_2022 \#як\_це\_було

Подвал. 24.03.2022

Сына положили в подвале  на земляной пол у входа, места в подвале было ровно
столько,чтобы люди могли спать сидя, у сына был свой табурет на котором он спал
до этого все ночи, 

Соседи пошли искать еще один стул, чтобы сделать лежачее место не на земле .

\ii{24_03_2023.fb.korovanenkova_natalia.mariupol.1.podval_24_03_2022.pic.1}

Еще в подвале стоял  мой разобраный деревяный щенятник, мы его заранее принесли
для костра, доски с него мы положили на два табурета и получилась "кровать", на
него перенесли сына.

Пока соседи обустраивали место для сына, который был без сознания, у него была
разбита голова, глаза, и все лицо было залито кровью, я пошла искать врача,
сказали, что где то через двор в соседнем доме есть военврач.

Как шла не помню, обстрел не прекрашался ни на минуту, помню  как мне с
подьездов/ укрытий кричали \enquote{ложись}, но у меня на это не было времени.

Врача я нашла, он был в соседнем дворе, в подвале дома.

Полный подвал раненых, под'ехал "масквич пиражок", туда загрузили раненых и
увезли, куда незнаю. Военврач сказал: "приносите, если возможности нет, тогда
дам ампулу обезбаливающего"

Еще я выпросила перекись.

Я шла обратно,  я  обязана  была дойти, у меня был самый ценный груз в моей
жизни, ампула противошокового, ампула обезболивающего и перекись)

Без сознания сын был три дня, я постояно  щупала пульс на шее, на очень
короткое время он приходил  в создание, говорил, что очень холодно и просил
есть.

Мои соседи,  мои  дорогие люди, мама Таня готовила Сергею горячую похлебку,
Марина обрабатывала раны, делала уколы,  Лешка был  помощником, держал фонарик
и следил чтобы Сергей не падал с "кровати".

С Лешкой в подвале было двое детей, пацаны, и старенькая мама.

Наш Лешка тихий и не заметный всюжизнь, в сложнейшей ситуации-человек стоИк)

В подвале было много детей, был подросток с ДЦП, совсем не ходящий, он все
время кричал, были пожилые соседи после инсульта,был сосед инсулино залежний,
без воды, еды и лекарств у него начались галлюцинации. Он попросил, чтобы его
вывели с подвала,поставил стул посреди тротуара и  сидел так под обстрелами,
смотрел на солнце. Его супруга Наташа, каким то чудом добралась до гаражей, за
машиной, приехала, посадила его, сына и своою чихуашку и они уехали под
обстрелами, что с ними я не знаю

Я с трудом втиснула свой табурет между стеной и сыном.

Вечером Надя, забралась на 7 этаж к себе в квартиру, нашла свою кошку и мою
собаку принесла в подвал, остальные оставались под завалами, выйти из подвала
уже было не реально, обстрел продолжался.

%\ii{24_03_2023.fb.korovanenkova_natalia.mariupol.1.podval_24_03_2022.cmt}
