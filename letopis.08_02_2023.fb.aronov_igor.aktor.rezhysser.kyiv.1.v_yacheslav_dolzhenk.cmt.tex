% vim: keymap=russian-jcukenwin
%%beginhead 
 
%%file 08_02_2023.fb.aronov_igor.aktor.rezhysser.kyiv.1.v_yacheslav_dolzhenk.cmt
%%parent 08_02_2023.fb.aronov_igor.aktor.rezhysser.kyiv.1.v_yacheslav_dolzhenk
 
%%url 
 
%%author_id 
%%date 
 
%%tags 
%%title 
 
%%endhead 

\qqSecCmt

\iusr{Ihor Aronov}

\href{https://mrpl.city/blogs/view/vyacheslav-dolzhenko-hochu-shhob-mariupol-stav-najkrashhim-mistom-na-zemli}{%
В'ячеслав Долженко: «Хочу, щоб Маріуполь став найкращим містом на землі!»,%
Ольга Демідко, mrpl.city, 26.05.2021%
}

\par\noindent\rule{\textwidth}{0.4pt}

\ifcmt
  ig https://i.paste.pics/4219e82e0c6ed20f164734a27449ac76.png
  @wrap center
  @width 0.8
\fi

\ifcmt
  tab_begin cols=2,no_fig,center,separate

		 pic https://mrpl.city/uploads/posts/resize/w/750x750/sgme7xzlobqybjcb.jpg
		 pic https://mrpl.city/uploads/posts/resize/w/750x750/c1tczas7u3fxbtft.jpg

		 pic https://mrpl.city/uploads/posts/resize/w/750x750/yukexpsh2phyrljf.jpg
     pic https://mrpl.city/uploads/posts/resize/w/750x750/x5f5onmbgnmoybri.jpg

  tab_end
\fi

\raggedcolumns
\begin{multicols}{2} % {
\setlength{\parindent}{0pt}

Цьогоріч Маріуполь отримав статус Великої культурної столиці. Нещодавно в
музеях міста маріупольці ознайомилися з унікальними  експонатами. Але не всім
відомо, що в Маріуполі є ще один незвичайний музей, створений зусиллями
справжнього ентузіаста.

Розташований музей за адресою проспект Миру, 43. Саме в цьому будинку мешкає
В’ячеслав Анатолійович Долженко, який вже 17 років збирає дуже цікаві та
незвичні речі. Все життя В'ячеслав прожив саме в цьому будинку, в підвалі якого
і створив справжній музей.

В'ячеслав народився в Маріуполі в сім'ї електрика та бібліотекарки. Його мама
48 років пропрацювала в бібліотеці Жданівського металургійного інституту
(сьогодні – ПДТУ), який і закінчив Славік за спеціальністю – обробка металів
тиском. Однак за перерозподілом не поїхав. Вирішив попрацювати вантажником в
театрі. Вважає це найщасливішим періодом в житті. Саме за 3 роки роботи в
театрі він познайомився з багатьма цікавими людьми: художниками, артистами. В
цей же час з братом починає продавати речі, зроблені власноруч: чоловічі
козирки, метелики, жіночі кулончики. Насправді В'ячеслав – майстер на всі руки.
Все, за що він береться отримує унікальний і неповторний вигляд. Після роботи
вантажником маріуполець вирішує закінчити курси пожежників у Луганську.
Пропрацював 13 років інспектором пожежної охорони. Був і офіцером, і капітаном.
Мабуть, В'ячеслав – єдиний інспектор пожежної охорони в країні, який штрафував
людей і сам платив за них гроші. Насправді він не міг інакше. Чоловік завжди
знаходив спосіб допомагати тим, хто цього потребує.

Ще в юності маріуполець мріяв облаштувати своє місце роботи прямо в будинку.
Так часто робили голландці, які створювали в підвалах будинків майстерні, а на
першому поверсі магазини. Маючи технічну освіту В'ячеслав не зміг працювати на
заводі, адже вважає, що це не зовсім його покликання. Для нього важливо, щоб
робота приносила справжнє задоволення. Його робочий день починається о 6 ранку
і закінчується об 11 ночі. Завдяки невтомній праці маріупольця в нашому місті
з'явився ще один незвичайний туристичний об'єкт.

Спочатку у підвалі була котельня. В'ячеслав ще дитиною бігав з друзями туди
погрітися. Опалювався будинок на вугіллі. Усюди була темрява. Пізніше котельню
закинули. Стояло це приміщення внизу. Воно було по пояс залито каналізацією. 

В'ячеслав вирішив приватизувати підвальне приміщення і почав дуже старанно та
ретельно все відновлювати. Він поставив собі за мету перетворити підвальне
приміщення на справжній клуб, де можна прийти на екскурсію, знайти однодумців
та просто провести корисно час. Облаштовувати і підвал, і двір В'ячеславу
допомагали його друзі. Сьогодні маріуполець шкодує, що не займався цією
благородною справою з молодості. Згадує, що коли ще навчався в школі, сусід
почав ремонт у квартирі і дав йому на збереження ковану решітку. Тоді Славік
подумав, навіщо чоловікові потрібен цей мотлох. Але зараз його світогляд
змінився кардинально. Після закінчення ремонту сусід вирішив, що решітка йому
не потрібна і тепер вона прикрашає камін в музеї В'ячеслава. 17 років він
збирав десятки експонатів. Кожен зі своєю історією. У музеї можна побачити і
половецьку бабу, яку колекціонер обожнює, привіз її з селища Агробаза,. Є тут і
ваги, виготовлені понад 100 років тому, в 1908-му році, які В'ячеслав знайшов
серед металевого брухту, і фортепіано, віком 150 років (колекціонеру подарувала
знайома). Кінське ярмо він знайшов біля покинутого приватного будинку. Є тут і
радіоприймачі, друкарські машинки, верстати різних років, німецькі каски часів
Другої світової війни, німецький алюмінієвий чайник, радянський морський
телефон, картини та ескізи видатних театральних художників Маріуполя і навіть
фрагменти фундаменту від Харлампіївського собору. Місцезнаходження фрагменту
В'ячеславу підказав його друг, видатний краєзнавець – Аркадій Дмитрович
Проценко.  Колекціонер замовив табличку і встановив унікальний фрагмент
фундаменту біля свого музею. Перевозити допомагали друзі.. В'ячеслав сам
змайстрував стійку для музейних експонатів, використавши для цього старі
віконні рами. Щось чоловік знайшов на сміттєзвалищі, щось йому подарували.
Зокрема, паркан, біля будинку, який вважить 500 кг, він знайшов на
металобрухті, зміг його викупити за 1,5 тонни металу та завдяки роботі ковалів,
зумів перетворити на новий. Біля паркана часто фотографуються молодята. Двері
будинку прикрашає козирок з купецького будинку. За парканом можна побачити
фонтан. Влітку В'ячеслав туди запускає жабенят. Багатьох маріупольців
діяльність В'ячеслава Анатолійовича надихає. Голова ОСББ Мир-43 Віра Гутнікова
підкреслила, що «він та людина, на яку повинні рівнятися всі, хто дійсно любить
своє місто і прагне його розвивати та вдосконалювати». За відповідальне
ставлення до історії рідного краю, збереження пам'яток старовини і надання
можливості безкоштовно відвідувати музей В'ячеслав  неодноразово отримував
подяки і грамоти. 

Ніякої фінансової підтримки  В'ячеслав не має.  Все робить за власні кошти.
Однак дуже вдячний за сприяння Сергію Васильовичу Магері, який допоміг з
чорноземом для упорядкування подвір'я. Він же нагородив маріупольця за найкраще
упорядкування двору. З роками ентузіазм і бажання щось змінювати не зникають.
Але є й такі сусіди, що пишуть скарги на В'ячеслава... Складно зрозуміти їхні
мотиви і дуже шкода, що такі випадки псують настрій нашому герою. Але, коли
приїжджають комісії, щоб штрафувати чоловіка, їхні представники розуміють, що
В'ячеслава Анатолійовича потрібно підтримувати, а не карати. У подвір'ї чоловік
облаштував дуже затишний куточок. Насправді він любить кожен камінчик, який сам
і встановив. Вважає, що граніт має особливу енергетику. Дуже шанує старовинний
гранітний тротуар на вулиці Фонтанній. Такий же хоче облаштувати і у себе на
подвір'ї.  Є  у чоловіка одна мрія – зробити маріупольський сквер найкращим у
світі: встановити там бронзовий фонтан, все озеленити, зробити гранітний
тротуар. Але для цього чоловік повинен знайти скарб і дуже великий. Наголошує,
що на себе нічого б не витрачав, все віддав би на облаштування улюбленого
міста.

У маріупольця ще дуже багато планів. Незабаром запрацює колоритний фонтан,
автором якого є художник Анатолій Манохін. Такий фонтан повинен був стояти біля
універмагу. Але плани залишилися на папері. Тому художник вирішив подарувати
його колекціонеру. Фонтан прикрашають азовські бички. Восени фонтан біля
будинку вже буде працювати.

Наразі чоловік розлучений, мешкає з мамою, якій вже 90 років і улюбленою
собакою – Асею. Саме вона допомагає колекціонеру знаходити нові унікальні речі
на сміттєзвалищах. Цікаво, що Асю, ще зовсім маленьку, чоловік купив на ринку у
безхатченків за 10 грн. Вважає, що розумнішої і красивішої собаки, просто не
знайти. Доглядати за музеєм і подвір'ям – його улюблена справа. Сьогодні до
незвичайного музею можна прийти на екскурсію і побачити всі експонати на власні
очі. В'ячеслав наголошує, що якщо кожен мешканець Маріуполя буде дбайливіше
ставитися до міста, намагатися зберігати все навколо і займатися благоустроєм
своїх будинків та вулиць, наше місто стане найкращим у світі.

Улюблена книга: твори Александра Дюма.

Улюблений фільм: фільми Леоніда Гайдая

Улюблене місце: щонеділі обожнює разом з Асею гуляти на вулиці Фонтанній і
спускатися вниз на залізничний вокзал.

\end{multicols} % }

\par\noindent\rule{\textwidth}{0.4pt}

\iusr{Ihor Aronov}

\href{https://www.youtube.com/watch?v=JvcZCQV7w5M}{%
\#місце\_сили: музей загублених речей / Музей Вячеслава Долженко / Место силы МТВ, %
Алевтина Швецова, youtube, 07.02.2022%
}

\ifcmt
  tab_begin cols=3,no_fig,center,separate

	   pic https://i.paste.pics/f115a648d7337e31042507309f7b1873.png
     pic https://i.paste.pics/e8945a1682d057ee7a4cfb9d17ff4b45.png
		 pic https://i.paste.pics/5d0b6e36a32b3b69b2ea408ab2312585.png

		 pic https://i.paste.pics/aafbc6b7bfa96858377c1f73931d4360.png
		 pic https://i.paste.pics/8f4a73bf9e22b1b8809fcfbdabde952d.png
		 pic https://i.paste.pics/2f6f54f783cc35b1e90665e05d78fb50.png

  tab_end
\fi

\par\noindent\rule{\textwidth}{0.4pt}

\begin{itemize} % {
\iusr{Олена Сугак}
\textbf{Ігор Аронов} 

о! Тут він культурно з Алею спілкується🤣. А з нашою компашкою яка несподівано
приперлася у вихідний день , він фольклорно- народною мовою з приказками та
матюком ... І так у нього ладно і смішно виходило розповідати про кожен
експонат! Талант.

\iusr{Алевтина Швецова}
\textbf{Олена Сугак} головне, що цікаво, незважаючи на формат спілкування.
Олено, обіймаю)
\end{itemize} % }

\iusr{Olga Demidko}

Ігорю, дякую вам за те, що збираєте свідчення і за те, що були сьогодні поруч з
В'ячеславом Анатолійовичем. Він у захваті від розмови з вами і дуже вдячний, що
ви взяли в нього інтерв'ю. )

\begin{itemize} % {
\iusr{Ihor Aronov}
\textbf{Olga Demidko} Дякую, що познайомили)
\end{itemize} % }

\iusr{Ольга Чайко}

А він у Києві? Так круто, що ти цим займаєшся! Дякую!!!

\begin{itemize} % {
\iusr{Ihor Aronov}
\textbf{Ольга Чайко} Так в Києві, і дуже потребую спілкування і занять, бо він супер активний

\iusr{Ольга Чайко}
\textbf{Ihor Aronov} виклянчу контакти якось)))

\iusr{Ihor Aronov}
\textbf{Ольга Чайко} Залюбки)

\iusr{Олена Сугак}
\textbf{Ігор Аронов} може йому сосватати якусь жіночку...

\iusr{Ihor Aronov}
\textbf{Олена Сугак} Якщо відгукнуться)

\end{itemize} % }

\iusr{Evelina Ganska}

Дякую

\iusr{Олена Сугак}

Славік то наша легенда... У мене на сторінці цілий альбом, присвячений його
музею. Головне, повернутись додому, я б залюбки допомогла збирати експонати.

\begin{itemize} % {
\iusr{Ihor Aronov}
\textbf{Олена Сугак} О, дякую, ви перша про нього мені сказали і ось ця зустріч відбулась) після неї в мене особисто дуже додалось віри що це повернення відбудеться

\iusr{Олена Сугак}
\textbf{Ігор Аронов} він чудовий. Ти правильно сказав, йому спілкування потрібно.
\end{itemize} % }

\iusr{Hanna Lemberska}

Я хочу долучитись до Ваших мистецько-соціальних проектів.

\begin{itemize} % {
\iusr{Iryna Kudria}
\textbf{Hanna Lemberska} +1
\end{itemize} % }

\iusr{Natalya Dedova}

Були у грудні. Спілкувалися. 

\href{https://civilvoicesmuseum.org/stories/kamennaya-baba-ya-s-nee-pylinki-sduval-a-ona-rassypalas}{%
\enquote{Кам'яна баба. Я з неї порошинки здував. А вона розсипалася}, %
civilvoicesmuseum.org, 05.01.2023%
}

Разом із 92-річною мамою В'ячеслав під час обстрілів був удома. Стекол у
квартирі вже не було. Навколо бігали \enquote{ДНР}-вівці. З-під даху пішов дим. \enquote{Нам
нема, чим дихати. Відкриваю люк. Все, що лежало на ліжку, кидаю туди, у підвал
музею. 3,5 метри глибина, - розповідає мешканець Маріуполя, засновник
приватного музею В'ячеслав Долженко. - Маму з ліжка підтягую, кидаю вниз.
Закрив люк. А в музеї – теж дим}.

\ifcmt
  ig https://i.paste.pics/7baa60f3c3d21be0a7bfe16f49fbee27.png
  @wrap center
  @width 0.7
\fi

\iusr{Natalya Dedova}

\href{https://civilvoicesmuseum.org/stories/kogda-legla-na-krovat-i-nakrylas-teplym-odeyalom-kak-v-raj-popala-azh-ne-verilos-chto-takoe-mozhet-byt}{%
"Коли лягла на ліжко і накрилася теплою ковдрою, як до раю потрапила. Аж не вірилося, що таке може бути", %
Тамара Долженко, civilvoicesmuseum.org, 05.01.2023%
}

Коли прийшли німці, Тамарі було 10 років. Коли прийшли росіяни, Тамарі Павлівні
виповнився 91 рік. Каже, так страшно, як цього разу, ніколи не було. "Люди
тягли мене з підвалу. Нога підверталася. Боляче було. Я кричала. Зубний протез
згорів. У підвалі на дошках, цементі лежали. Голодували, - згадує Тамара
Павлівна. - Страшно було їхати з Маріуполя. А інакше не можна було».

\ifcmt
  ig https://i.paste.pics/b3d6f238bbb59913681101f6d5846077.png
  @wrap center
  @width 0.7
\fi

\iusr{Iryna Kudria}

Неймовірна історія. Дякую. Пішла далі дивитися

\iusr{Iryna Kudria}

Ігоре, а чия це ініціатива Архів війни? Чи це вже власна назва проєкту?

\begin{itemize} % {
\iusr{Ihor Aronov}
\textbf{Iryna Kudria} \url{https://ukrainewararchive.org}
\end{itemize} % }

\iusr{Оскар Валерійович}

Оххх

\iusr{Alina Shapran}

Дякую за це!

\iusr{Olena Shevchuk}

Ох, реву

\begin{itemize} % {
\iusr{Ihor Aronov}
\textbf{Olena Shevchuk} Я теж сьогодні двічі

\iusr{Ольга Чайко}
\textbf{Olena Shevchuk} та теж. Вдруге перечитую і...
\end{itemize} % }

\iusr{Iryna Voloshyna}

Дякую, Ігоре, мій дорогий, за твою роботу! Це надіажливо!!!!!!

\iusr{Yevheniya Bohdanova}

Щемко...💔 дякую що поділився

\iusr{Nata Kozhevnikova}

😔господи🙏🌅Дякую! Написала в пп

\iusr{Angelina Solnzewa}

Ігоре, дякую. Поки історія просто цифри у підручниках, покоління не будуть
розуміти її, відчувати її запах, чути крики, сльози. Коли за цими сухими
фактами нема історії, життя ....а ти змінюєш цю парадигму. Оживає історія, ми
можемо відчути, зрозуміти, цінувати, пам'ятати.

\iusr{Hanna Harnyk}

Які люди! Дякую, Ігоре❤️

