% vim: keymap=russian-jcukenwin
%%beginhead 
 
%%file 23_04_2023.fb.bludsha_maryna.kyiv.1.423_teplyj_sonechnyj_den_cvyntar_bajkove.cmt
%%parent 23_04_2023.fb.bludsha_maryna.kyiv.1.423_teplyj_sonechnyj_den_cvyntar_bajkove
 
%%url 
 
%%author_id 
%%date 
 
%%tags 
%%title 
 
%%endhead 

\qqSecCmt

\iusr{Anna Atamanchuk}

Я любила в дитинстві поминальні дні. Мені подобалось ходити на кладовища і
роздивлятися обличчя на пам'ятниках. Завжди було цікаво знати причину смерті
всіх цих людей, особливо молодих і дітей. Я б хотіла аби десь на плитах
зазначали її, може це когось наштовхнуло на думку пройти обстеження, піти від
аб'юзера, чи дотримуватися ПДР.

Я питала у батьків, а вони казали \enquote{Не знаю} тепер мені Ілля ставить ті самі
запитання, а я відповідаю, як мої батьки 😞

\begin{itemize} % {
\iusr{Maryna Bludsha}
\textbf{Anna Atamanchuk} і мені завжди хочеться знати більше про людину та історію її життя і смерті, шкода, що так мало пишуть на пам'ятниках
\end{itemize} % }

\iusr{Іван Іван}

дуже добре, що написали! Я теж був вчора на Байковому кладовищі. Було прощання
з Володимиром Мельниченко - автором Стіни Пам'яті. Вічна йому пам'ять, це один
із наших найвизначніших митців! До своїх також зайшов. Дуже-дуже багато людей!
Я аж здивувався, бо зазвичай в усі мої відвідини Байкового людей та машин було
значно менше... А вчора потік машин стояв аж від низу від вулиці Казимира
Малевича-знизу-під-залізничним-мостом-через-Грінченка аж до верху по всій
Байковій! А щодо центральної алеї, де поховані багато із наших поетів,
визначних людей із різних напрямів людської діяльності (список там величезний
насправді...) як-от Леся Українка або Остап Вишня як-от патріарх української
кібернетики Віктор Глушков, на проспекті якого розташовані корпуса універу
Шевченка, в які попала ракета під новий рік (було дуже сумно це бачити, вибите
скло усюди лежало!) або Євген Патон, чиє ім'ям названий цільнозварний міст
(перший у світі цільнозварний міст!) через Дніпро, відкритий в 1953 році, то
треба було при вході притискатись до стіни, бо машини там, машини тут, прям як
на Хрещатику у годину пік!

\ifcmt
  igc https://scontent-fra3-1.xx.fbcdn.net/v/t39.30808-6/342899550_967864497730987_183387687686441963_n.jpg?_nc_cat=103&ccb=1-7&_nc_sid=dbeb18&_nc_ohc=koErX8wjCk4AX9FjOa1&_nc_ht=scontent-fra3-1.xx&oh=00_AfAMIfkKOzrx18CBuZSZE7vgbTteIoDvQWw_qR4uiEl7KA&oe=644A2742
	@width 0.4
\fi

\iusr{Іван Іван}

Володимир Мельниченко, творець Стіни Пам'яті на фото зверху (поки що лише один
із барел'єфів звільнено з бетонного полону, в який Стіну закатала радянська
влада)

\ifcmt
  igc https://scontent-fra5-2.xx.fbcdn.net/v/t39.30808-6/342864111_1445337786210968_7345553943504144304_n.jpg?_nc_cat=106&ccb=1-7&_nc_sid=dbeb18&_nc_ohc=7R_NEAXmeHIAX8PzCHK&_nc_ht=scontent-fra5-2.xx&oh=00_AfALkEW0F9RNuMsXPPcsOvX3RoMTAwDS5OFd9prbB-jsJA&oe=644A2A9A
	@width 0.8
\fi
