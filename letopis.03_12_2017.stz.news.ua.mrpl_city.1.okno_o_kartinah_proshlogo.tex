% vim: keymap=russian-jcukenwin
%%beginhead 
 
%%file 03_12_2017.stz.news.ua.mrpl_city.1.okno_o_kartinah_proshlogo
%%parent 03_12_2017
 
%%url https://mrpl.city/blogs/view/okno-o-kartinah-proshlogo-kotorye-vse-chashhe-beredyat-dushu
 
%%author_id burov_sergij.mariupol,news.ua.mrpl_city
%%date 
 
%%tags 
%%title Окно: о картинах прошлого, которые все чаще бередят душу
 
%%endhead 
 
\subsection{Окно: о картинах прошлого, которые все чаще бередят душу}
\label{sec:03_12_2017.stz.news.ua.mrpl_city.1.okno_o_kartinah_proshlogo}
 
\Purl{https://mrpl.city/blogs/view/okno-o-kartinah-proshlogo-kotorye-vse-chashhe-beredyat-dushu}
\ifcmt
 author_begin
   author_id burov_sergij.mariupol,news.ua.mrpl_city
 author_end
\fi

\ii{03_12_2017.stz.news.ua.mrpl_city.1.okno_o_kartinah_proshlogo.pic.1}

Окна в дедушкином доме были большие. Если подставить стул или табуретку, можно
было взобраться на широкий подоконник одного из них, задернуть за собой гардину
и смотреть на то, что происходило на Торговой улице. Бывало, что взрослые
сгоняли с подоконника и сами взволнованно смотрели на происходящее за окном, но
почему-то - сквозь сетку гардины. Увиденное в детстве из окна осело в памяти,
как кадры кинофильма, никогда никем не снятого. Один из эпизодов: вниз по
мостовой бегут красноармейцы, они так не похожи на тех, что нарисованы в
детских книжках. На них испачканные гимнастерки, на многих - окровавленные
бинты. Кто-то тащит за собой пулемет, громыхающий железными колесами по
брусчатке. Газик, набитый людьми, бешено сигналя, мчится в том же направлении,
что и красноармейцы, те разбегаются в разные стороны...

Немцы в городе. Непрерывным потоком катятся тяжелые грузовики, кузова закрыты
тентами. Этот грязно-зеленый поток непрерывен и нескончаем. Через время дед
Мирон – муж бабушкиной сестры Оришки – полушепотом произносит: \enquote{Так вони вже
утретє тут їдуть! Ось зараз буде їхати рудий німець, у нього у кабіні
теліпається мавпочка з плюшу. Я його вже кілька разiв бачив. Це вони - щоб ïх
пранці поïли -  коло з машин зробили, щоб нас налякати}...

Идут люди, занимая всю ширину улицы, идут молча, лишь изредка кто-нибудь из них
перебрасывается короткими фразами с идущим рядом. Лица у них отрешенные. Много
детей. Малышей несут на руках, кто-то толкает перед собой детскую коляску. Еще
сравнительно тепло, но почти все они в зимней одежде, некоторые женщины даже в
шубах. У кого-то за плечами наскоро сделанная котомка, кто-то несет чемодан, у
детей, что постарше, в руках узелки, авоськи, кошелки, с которыми мариупольские
хозяйки ходят на базар. \enquote{Мама, куда они идут?} \enquote{Они сами не знают, сыночка}. На
глазах у нее слезы. Это потом станет известно, что мариупольские евреи будут
расстреляны у противотанкового рва близ поселка Агробаза...

Еще шествия на Торговой улице, уже в послевоенные годы. На восстановление
\enquote{Азовстали} или коксохимического завода ведут пленных немцев. Они бодро шагают,
строго соблюдая равнение. На некоторых из них свои шинели, на других – шинели
красноармейские, на которых видны бурые пятна: то ли следы огня, то ли крови.
Одеяние их сильно изношено, но дыры на нем тщательно заштопаны. Оно застегнуто
на все пуговицы и крючки. Немцы обуты в видавшие виды ботинки, такие же, как и
на ногах их редких конвоиров. Конвоиры – все больше молодые низкорослые солдаты
– едва поспевают за строем, то и дело поправляют сползающие с плеча ремни
карабинов...

По нашей улице ведут заключенных. Продолжительность их печального шествия столь
велика, что около часа местные жители не могут перейти улицу с одной стороны на
другую. Здесь конвой куда более серьезный, чем у пленных немцев. Солдаты держат
винтовки наперевес, время от времени заглядывают в нестройные шеренги,
покрикивают на отстающих бедолаг. Сплошная серая людская масса с угрюмыми
изможденными лицами, облаченная в серые потрепанные армейские бушлаты, в серые
телогрейки, стеганые белыми нитками... 

В сорок шестом или в сорок седьмом году в Мариуполе возобновились демонстрации
в первомайские и октябрьские праздники. Работники завода \enquote{Азовсталь}, треста
\enquote{Азовстальстрой}, других предприятий, находившихся в Орджоникидзевском районе,
а также трудящиеся мелких предприятий центрального района города, прежде чем
продемонстрировать свою преданность партии и правительству перед трибуной,
установленной на проспекте Республики у здания Госбанка, шли по Торговой улице.
Первыми у дедушкиного дома появлялись колонны учащихся ремесленных училищ.
Подростки в синих гимнастерках, заправленных под широкие ремни, в черных
брюках, на головах фуражки с лакированными козырьками и перекрещенными
молотками на околышах, были похожи на хорошо вымуштрованных маленьких
солдатиков. Затем маршировали ученики и ученицы школ фабрично-заводского
обучения, их шеренги были не так стройны, как у ремесленников. А уже потом
двигались взрослые участники демонстрации со знаменами, обязательными
портретами Сталина и других руководителей страны, с красными транспарантами, на
которых белой или желтой краской были начертаны лозунги. Каждый праздничный
строй предприятия возглавлялся его руководителями, за которыми следовал духовой
оркестр, если предприятие было крупным, и один или два баяниста у коллективов
помельче.

Иногда через два-три часа после окончания праздничной демонстрации духовой
оркестр клуба металлургов завода \enquote{Азовсталь} (так в то время назывался  недавно
закрытый дворец культуры этого предприятия) собирался в полном составе с
инструментами близ пересечения Торговой и Фонтанной улиц. Благо что большинство
музыкантов, так уж получилось, жили во дворе \enquote{колбасни}, на Банном спуске и в
других близлежащих домах. Оркестранты выстраивались в привычный порядок,
впереди них становился горбатенький Ваня с малым барабаном. Он исторгал из
своего инструмента дробь, затем поднимал и резко опускал палочки. В этот момент
из всех труб одновременно вылетали оглушительные звуки музыки, и тотчас же
начиналось движение оркестра ко всеобщему восторгу мальчишек и девчонок... 

Понятно, что наблюдать из окна шествия большого количества людей приходилось
нечасто. Но и в обычные дни можно было увидеть немало интересного. Каждый день,
в одно и то же время, проезжал по улице на мотоцикле с коляской грузный
человек, плотно затянутый в черную кожаную куртку, кожаные же галифе и
пилотский шлем, с лицом, почти полностью закрытым огромными защитными очками.
Иногда появлялся велосипедист, сухопарый, со смуглым, словно прокопченным
лицом, черными, как смоль, гладкими блестящими волосами и узенькими усиками под
крючковатым носом. Велосипедист изо всех сил нажимал на педали, чтобы
преодолеть немалую крутизну улицы. Это был дядя Жора Жанжурист – лучший в
округе настройщик роялей и пианино и большой оригинал. На его \enquote{веломашине}, так
по старинке он называл своего двухколесного коня, было навешано много нужных и
ненужных устройств и приспособлений. Клаксон от дореволюционного автомобиля,
фонари на руле и на крыле заднего колеса с питанием от солидных размеров
аккумулятора, самодельное зеркало дополняло оборудование руля, на багажнике с
двух сторон были приторочены сумки из потертой кожи.

Можно было наблюдать подпольного маклера Беляева, спешащего совершить сделку по
купле-продаже домостроения и вместе с тем не упускающего случая остановиться и
поболтать со встретившимся знакомым. На Беляеве надет видавший виды долгополый
плащ, болтающийся на его костлявых плечах. Голова была накрыта засаленной
кепкой с длинным – по моде начала тридцатых годов – козырьком. Дела у него
далеко не блестящи. Еще картины. Вихляющей походкой идет воришка Генка, недавно
освободившийся после очередной отсидки. Он по привычке озирается по сторонам и
вежливо, с поклоном, здоровается с бабушками, ожидающими подвоза товара в
продовольственный магазин. Школяр-малолетка в пальтишке, скроенном из немецкого
трофейного мундира, с холщовой сумкой через плечо неспешно следует на учебу.
Его перегоняют две девчонки – ученицы женской школы в одинаковых красных
вязаных шапочках, из-под которых выглядывают черные бантики. Такие шапочки
имели тогда название \enquote{менингитки}, поскольку прикрывали лишь темя маленьких
модниц. А вот рабочие и инженерно-технические работники в поле зрения
глазеющего на улицу мальчишки не попадались: те шли по Торговой с зарей, до
первого гудка \enquote{Азовстали}, возвещавшего о начале первой смены, и возвращались с
работы затемно. 

Самым интересным было разглядывание по улице транспорта. Он был представлен
главным образом телегами, дрогами, двуколками, линейками – все на конной тяге.
Кстати,  линейка – конный экипаж, ныне забытый: возница и три седока сидят по
двое с каждой стороны спинами друг к другу. На линейках ездило начальство и
врачи скорой помощи, сопровождаемые, как и теперь, медицинскими сестрами. Что
еще запомнилось. Низкорослые монгольские лошадки, запряженные в груженые
телеги, без особой натуги преодолевающие подъем улицы. Отечественных одров, для
того, чтобы они вывезли дроги с поклажей на более-менее пологое место,
извозчики нещадно полосуют кнутами по ребрам. Автомобилей было во много-много
раз меньше, чем в наши дни, а разнообразия – тем более. Тряслись по мостовой
трехтонки ЗиС-5, трудяги полуторки, уверенно двигались американские
\enquote{Студебекеры} и \enquote{Форды}, иногда можно было увидеть трофейные \enquote{Фольксваген-Жук},
\enquote{БМВ} или \enquote{Опель}. Это уж потом появились отечественные \enquote{Победы}. Шустро
проскальзывали новенькие \enquote{Москвичи}, как две капли воды похожие на немецкий
\enquote{Опель Кадет}. Лязганьем листов потрепанного кузова и чиханием изношенного
мотора предвещал свое появление длинный довоенный автобус, чуть ли не
единственный в городе...

Давно продан дедушкин дом, переплавлены в мартенах автомобили, некогда
пробегавшие по Торговой улице, покинули наш бренный мир те, о ком здесь шла
речь, а картины прошлого все чаще бередят душу. К чему бы это?
