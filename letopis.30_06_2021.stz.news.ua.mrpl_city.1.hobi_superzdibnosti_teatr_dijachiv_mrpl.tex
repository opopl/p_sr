% vim: keymap=russian-jcukenwin
%%beginhead 
 
%%file 30_06_2021.stz.news.ua.mrpl_city.1.hobi_superzdibnosti_teatr_dijachiv_mrpl
%%parent 30_06_2021
 
%%url https://mrpl.city/blogs/view/nezvichni-hobi-ta-superzdibnosti-teatralnih-diyachiv-mariupolya
 
%%author_id demidko_olga.mariupol,news.ua.mrpl_city
%%date 
 
%%tags 
%%title Незвичні хобі та "суперздібності" театральних діячів Маріуполя
 
%%endhead 
 
\subsection{Незвичні хобі та \enquote{суперздібності} театральних діячів Маріуполя}
\label{sec:30_06_2021.stz.news.ua.mrpl_city.1.hobi_superzdibnosti_teatr_dijachiv_mrpl}
 
\Purl{https://mrpl.city/blogs/view/nezvichni-hobi-ta-superzdibnosti-teatralnih-diyachiv-mariupolya}
\ifcmt
 author_begin
   author_id demidko_olga.mariupol,news.ua.mrpl_city
 author_end
\fi

Нещодавно в рубриці \emph{\textbf{\enquote{Маріуполь театральний}}} ранкової роз\hyp{}важально-інформаційної
програми \emph{\textbf{\enquote{Ранок} Маріупольського телебачення}} з'явилася нова традиція – ходити
додому чи в гримерки до акторів та знайомитися з ними ближче. Думаю, ця
традиція є унікальною можливістю подивитися на життя режисерів і артистів поза
театром, дізнатися про них незвичні та цікаві факти та стати трохи ближчими до
театральних митців.

На мою думку, абсолютно всі театральні діячі, з якими я встигла познайомитися в
рамках програми, мають неповторний внутрішній світ, адже вони дійсно дуже
талановиті та креативні. До того ж їхні хобі можуть не тільки приємно вразити,
а й здивувати. Саме про те, чим займаються маріупольські актори та режисери
поза сценою і які \enquote{суперздібності} допомагають їм в повсякденному житті хочу
розповісти окремо.

\ii{30_06_2021.stz.news.ua.mrpl_city.1.hobi_superzdibnosti_teatr_dijachiv_mrpl.pic.1}

Недарма кажуть, що обдарована людина – талановита в усьому. Це твердження
справедливе і для неординарного, дуже працьовитого актора Донецького
академічного обласного драматичного театру (м. Маріуполь) \textbf{Валерія Капінуса}. Так
артист має багато різних хобі. Зокрема, любить працювати з деревом. Може і
полички, і макети, і декорації для вистави створити. Він дійсно є справжнім
майстром на всі руки. Сьогодні у актора вже є власна столярна майстерня. Коли в
театрі працювали над виставою \enquote{Біла ворона} Валерію не вистачало дзвону
справжніх мечів. Він отримав дозвіл від режисерки вистави \textbf{Анжеліки Добрунової}
і сам зробив мечі. Цікаво, що ще одним незвичним хобі Валерія є в'язання.
Чоловік вважає, що спиці та гачок – це його стихія.

\ii{30_06_2021.stz.news.ua.mrpl_city.1.hobi_superzdibnosti_teatr_dijachiv_mrpl.pic.2}

До речі, в'язати любить і провідна майстриня сцени Донецького академічного
обласного драматичного театру (м. Маріуполь) \textbf{Лариса Колесник}. Актриса
підкреслює, що її це заспокоює. Водночас це дуже зручне та корисне хобі, адже
син давно ходить в речах, створених нею.

\ii{30_06_2021.stz.news.ua.mrpl_city.1.hobi_superzdibnosti_teatr_dijachiv_mrpl.pic.3}

Репетиторка з техніки мови Донецького академічного обласного драматичного
театру (м. Маріуполь) та режисерка Театральної артілі \enquote{Драмком} \textbf{Наталя
Гончарова} обожнює готувати. Раніше вона могла цілими днями не виходити з кухні,
але зараз часу не так багато. Цікаво, що Наталя Гончарова та Валерій Капінус,
одружившись влаштовували справжні костюмовані вечірки. Їхні яскраві та дружні
свята завжди запам'ятовувалися креативним підходом. Мабуть, свій режисерський
хист Наталя реалізовувала і вдома.

\ii{30_06_2021.stz.news.ua.mrpl_city.1.hobi_superzdibnosti_teatr_dijachiv_mrpl.pic.4}

Деякі актори вміють літати і не тільки від надихаючої ролі, а й в прямому сенсі
слова. Зокрема, маріупольська актриса \textbf{Анна Німайєр} вже три роки займається
аеростретчингом. Відчуття польоту у повітрі актрису заворожує та піднімає
настрій. Через травму спини Анна довгий час намагалася знайти саме той вид
спорту, який би їй найбільше підійшов і сьогодні вона радіє, що відкрила для
себе аеростретчинг.

\ii{30_06_2021.stz.news.ua.mrpl_city.1.hobi_superzdibnosti_teatr_dijachiv_mrpl.pic.5}

Справжнім супертатком і суперчоловіком є актор Донецького академічного
обласного драматичного театру (м. Маріуполь) \textbf{Дмитро Нестеренко}, який з легкістю
може впоратися з будь-якою хатньою справою. Дружина Дмитра актриса \textbf{Надія
Лаврененко} наголошує, що у чоловіка виходить і смачніше готувати, і краще
прибирати, і за сином доглядає дуже ретельно. Він справжній перфекціоніст і
таких помічників ще треба пошукати. Цікаво, що під час прибирання актор може
зробити спонтанну перестановку чи вимити до блиску всі стіни. Також Дмитро
Нестеренко любить рибалити, грати у футбол та обожнює готувати.

\ii{30_06_2021.stz.news.ua.mrpl_city.1.hobi_superzdibnosti_teatr_dijachiv_mrpl.pic.6}

Є серед акторів і колекціонери. Зокрема, у актриси \textbf{Наталі Квятковської} вже
назбиралася унікальна та різноманітна колекція янголів. Є у неї і в'язані, і
рукодільні, і мармурові янголи. До речі, досить часто актрисі дарували їх саме
глядачі.

\ii{30_06_2021.stz.news.ua.mrpl_city.1.hobi_superzdibnosti_teatr_dijachiv_mrpl.pic.7}

Дуже різнобічною і талановитою людиною є \textbf{Ігор Курашко.} Інколи він пише вірші.
Має диплом психолога. З 2003 року актор займається бігом, йогою та медитацію.
Актор вважає біг піснею тіла. Ігор брав участь у всеукраїнських марафонах з
бігу, має і відзнаки. Також артист записує аудиокниги, дуже надихається
діяльністю індійського релігійного діяча, філософа та спортсмена, художника,
поета і драматурга Шрі Чинмоя.

\ii{30_06_2021.stz.news.ua.mrpl_city.1.hobi_superzdibnosti_teatr_dijachiv_mrpl.pic.8}

Незвичне хобі у актриси маріупольського драматичного театру та режисерка
Народного театру \enquote{Театроманія} \textbf{Ольги Самойлової}, яка найбільше любить  їздити
на своєму мопеді вулицями Маріуполя. Також Ольга обожнює подорожувати разом з
сім'єю.

% telbizov
\ii{30_06_2021.stz.news.ua.mrpl_city.1.hobi_superzdibnosti_teatr_dijachiv_mrpl.pic.9}

Багато хобі і у талановитого режисера Народного театру \enquote{Театроманія} \textbf{Антона
Тельбізова}. Проте, на мою думку, головною суперздібністю Антона є вміння
доглядати за надзвичайно примхливими квітами – фіалками. Режисер \enquote{Театроманії}
вивчив їхні види (торік у нього було 106 видів фіалок)  і дуже ретельно
доглядає за квітами. А вони зі свого боку відповідають дбайливому Антону своїм
квітучим виглядом у будь-яку пору року. У кожної фіалки є своє ім'я. Режисер їх
збирав з усієї України. Всіх гостей Антона Тельбізова завжди вражає його оселя
взимку, яка завдяки фіалкам, дарує весняний настрій.

% vira shevcova
\ii{30_06_2021.stz.news.ua.mrpl_city.1.hobi_superzdibnosti_teatr_dijachiv_mrpl.pic.10}

А талановита актриса \textbf{Віра Шевцова} полюбляє монтувати та малювати за допомогою
різних програм. Також актриса дуже любить займатися спортом.

Звісно, я встигла познайомитися не з усіма нашими театральними митцями, але з
впевненістю можу сказати, що їх всіх об'єднує бажання створювати щось нове,
самовдосконалюватися та експериментувати.
