% vim: keymap=russian-jcukenwin
%%beginhead 
 
%%file 01_07_2017.stz.news.ua.mrpl_city.1.mariupol_kakim_on_byl_kakoj_on_est
%%parent 01_07_2017
 
%%url https://mrpl.city/blogs/view/mariupol-kakim-on-byl-kakim-on-est
 
%%author_id burov_sergij.mariupol,news.ua.mrpl_city
%%date 
 
%%tags 
%%title Мариуполь: каким он был, какой он есть
 
%%endhead 
 
\subsection{Мариуполь: каким он был, какой он есть}
\label{sec:01_07_2017.stz.news.ua.mrpl_city.1.mariupol_kakim_on_byl_kakoj_on_est}
 
\Purl{https://mrpl.city/blogs/view/mariupol-kakim-on-byl-kakim-on-est}
\ifcmt
 author_begin
   author_id burov_sergij.mariupol,news.ua.mrpl_city
 author_end
\fi

\ii{01_07_2017.stz.news.ua.mrpl_city.1.mariupol_kakim_on_byl_kakoj_on_est.pic.1}

Ежегодно 1 июля в нашей стране отмечается День архитектуры Украины. Не этот ли
повод \enquote{пробежаться} по истории архитектуры и градостроительства места, где
прожита жизнь. Начнем с того, что изображения домов и церкви административного
центра Кальмиусской паланки не дошли до нас. А может быть, таких изображений и
не существовало. Но чтобы их представить, достаточно посмотреть их на офортах
Тараса Шевченко, на пейзажах Архипа Куинджи, Владимира Орловского, Сергея
Васильковского, Григория Светлицкого и других мастеров кисти.

\ii{01_07_2017.stz.news.ua.mrpl_city.1.mariupol_kakim_on_byl_kakoj_on_est.pic.2}

В 1778 году глава Азовской губернии Василий Алексеевич Чертков распорядился
заложить церковь во имя святой равноапостольной Марии Магдалины во вновь
образованном городе Павловске. Расположение храма было определено согласно
плану, который разработал губернатор собственноручно. Были уже возведены стены,
но летом 1780 года вместо поселения казаков, ставших подданными российской
императрице, в город Павловск под предводительством митрополита Игнатия пришли
крымские греки. Город получил новое имя - Мариуполь. Митрополит почему-то не
захотел достраивать начатую церковь. Вместо этого на месте паланки он освятил
закладку городской соборной церкви св. Харлампия. В предместьях также было
предусмотрено строительство храмов. В Карасу-Базаре - церковь Рождества
Пресвятой Богородицы, в Мариинске – Успенскую церковь, в Кафайском квартале –
храм Феодоро-Стратилатовский, который сгорел еще при жизни Митрополита и не был
восстановлен. Церковь же, заложенная В.А. Чертковым, была достроена украинцами
Горленским, Пилипенко, Велигурой и другими. Обряд ее освящения произошел 4 июня
1791 года. Эти храмы стали первыми архитектурными украшениями Мариуполя...

\ii{01_07_2017.stz.news.ua.mrpl_city.1.mariupol_kakim_on_byl_kakoj_on_est.pic.3}

К началу Первой мировой войны наш город обзавелся рядом интересных в
архитектурном смысле строениями. В первую очередь это культовые здания.
Величественный Харлампиевский собор, Екатерининская церковь (старый собор св.
Харлампия), Марии-Магдалининская церковь в Александровском сквере,
реконструированные храмы на Карасевке и в Марьинске, кладбищенский храм. Были
также римско-католическая церковь близ пересечения Итальянской и Торговой улиц,
хоральная синагога на Георгиевской улице, еврейские молитвенные дома на
Харлампиевской, Николаевской и Митрополитской. Довольно пристойно выглядели
торговые заведения, дома состоятельных обывателей, уездное земство, гостиница
\enquote{Континенталь} и др. А жилища судовладельца Петра Регира на Бондарной улице и
Давида Хараджаева на Таганрогской даже называли дворцами. Солидный вид имели
учебные заведения – Духовное, Епархиальное и Реальное училища, здания гимназий,
особенно Александровской мужской. Приличным был железнодорожный вокзал,
приходские школы и городские училища. Вид Мариуполя с моря украшали колокольня
Харлампиевского собора и купол Марии-Магдалининской церкви, стоявшей на самой
высокой точке тогдашнего города. Между прочим, для этих храмов нашли
утилитарное назначение. Рыбаки, находясь в море, использовали их в качестве
ориентиров. При помощи простого и вместе с тем остроумного устройства они
определяли расстояние своих суденышек от берега.

\ii{01_07_2017.stz.news.ua.mrpl_city.1.mariupol_kakim_on_byl_kakoj_on_est.pic.4}

Первое ощутимое потрясение архитектура города испытала в середине 30-х годов
прошлого столетия, когда один за другим были взорваны все церковные здания.
Чтобы придать этой акции пристойный вид, близ мест, где были церкви, возвели
школы. На Марьинске – школу № 36, на Карасевке – № 11, на Слободке - 37-ю
школу. Нужно сказать, что в Мариуполе перед Отечественной войной было построено
несколько заметных многоэтажных домов. Дом рядом со сквером, в котором были
ресторан и гастрономический магазин, а еще так называемый дом с хлебным
магазином, проспект Республики, 22/15, пятиэтажки на проспекте Республики, № 43
и № 45, дома на улице Энгельса, №1 и № 39, а также великолепный Дворец
пионеров.

\ii{01_07_2017.stz.news.ua.mrpl_city.1.mariupol_kakim_on_byl_kakoj_on_est.pic.5}

Смертельный удар городу нанесли немецко-фашистские захватчики. События
разворачивались так. В 1943 году, 5 сентября, был расклеен приказ
фельдкоменданта Мариуполя генерала Гофмана, в котором предписывалось населению
уходить в западном направлении. На следующий день оккупанты и их пособники
покинули город. А 7 сентября его передали специальным командам жандармерии,
попросту говоря, поджигателям. Три дня горел город. Передовому отряду одной из
частей Красной Армии 10 сентября удалось захватить факельщиков. Они были
повешены на импровизированной виселице перед восточным входом в сквер. Все
строения в два этажа и выше, за очень малым исключением, превратились в
погорелки. На некоторых улицах были преданы огню и одноэтажные дома.

\ii{01_07_2017.stz.news.ua.mrpl_city.1.mariupol_kakim_on_byl_kakoj_on_est.pic.6}

К восстановлению жилищ приступили немедленно. Среди руководителей города и
предприятий шли споры. Одни считали, что нужно по возможности восстанавливать
здания исторической части города. Другие говорили, мол, дешевле будет снести
погорелки, а новый город строить на Левом берегу Кальмиуса. Но победили те, кто
считал необходимым сохранить по возможности исторический облик города.
Восстановлением занимались строители треста \enquote{Азовстальстрой}, поселки при
заводе имени Ильича поднимали из руин трудящиеся Управления капитального
строительства этого предприятия. С 1946 года на протяжении двадцати семи лет
главным архитектором Мариуполя был Александр Митрофанович Веселов. Он много
сделал для украшения наших улиц и площадей. Это и дома со шпилями, драмтеатр.
Это и жилые дома на проспекте Республики к западу от Театрального сквера. А
также здания на том же проспекте - 12, 13, 24, 42. И еще одна страничка из
истории. С середины 60-х годов развернулось строительство панельных жилых
домов. В то время появился Владимир Михайлович Цыбулько, который формально не
был руководителем города. Тем не менее, он добился в Киеве, чтобы в необластном
городе разрешили строить дома повышенной этажности. Все остальное об истории
архитектуры нашего города его современные жители знают.
