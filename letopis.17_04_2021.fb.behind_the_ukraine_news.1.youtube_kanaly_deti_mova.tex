% vim: keymap=russian-jcukenwin
%%beginhead 
 
%%file 17_04_2021.fb.behind_the_ukraine_news.1.youtube_kanaly_deti_mova
%%parent 17_04_2021
 
%%url https://www.facebook.com/behindtheukrainenews/posts/910234876467313
 
%%author_id behind_the_ukraine_news
%%date 
 
%%tags deti,internet,jazyk,mova,ukraina,youtube
%%title Список україномовних ютуб каналів для ДІТЕЙ
 
%%endhead 

\subsection{Список україномовних ютуб каналів для ДІТЕЙ}
\label{sec:17_04_2021.fb.behind_the_ukraine_news.1.youtube_kanaly_deti_mova}

\Purl{https://www.facebook.com/behindtheukrainenews/posts/910234876467313}
\ifcmt
 author_begin
   author_id behind_the_ukraine_news
 author_end
\fi

Список україномовних ютуб каналів для ДІТЕЙ

Просимо якнайбільшого поширення та вподобайок, навіть якщо ви дорослі й у вас
немає дітей@igg{fbicon.heart.yellow}@igg{fbicon.heart.blue}

Клятий раціоналіст, Майкл Щур, Рагулівна, Тайлер Андерсон, Остап Українець,
Євгенія Кузнєцова… Досить з нас цих старих блогерів, подумали ми. Треба щось
запропонувати і дітям. Запалімо нові зірочки на українському ютубі!;-)

Наш критерій — дітям має бути цікаво. А якщо вони з цього ще й отримають
користь і знання — пречудово! Хоча можна й просто подуркувати;-) В кожному
разі, що дивитися — вирішувати тільки вам і вашій дитині. Ми лише хочемо
полегшити вибір.

\ifcmt
  ig https://scontent-frx5-1.xx.fbcdn.net/v/t1.6435-9/174279182_910234349800699_419170747384146811_n.jpg?_nc_cat=105&ccb=1-5&_nc_sid=730e14&_nc_ohc=_A4MWUu3nrIAX-36pEj&_nc_ht=scontent-frx5-1.xx&oh=dc9c5e8214f611ea1419a32b76a1bed5&oe=61620D7C
  @width 0.4
  @wrap \InsertBoxR{0}
  %@wrap \parpic[r]
  %@wrap minipage{0.5\textwidth}
\fi

Минулого разу ми пропонували список україномовних ютубканалів
\url{https://bit.ly/3gVeRr8} Допис поширили понад 10 тисяч людей, тисячі вподобань,
охоплення — взагалі мільйони. Сподіваємося, дитяча версія перевершить рекорд.
Все залежить тільки від вас, ваших вподобайок, поширень і коментарів з
улюбленими каналами для дітей. А ще — збережіть цей допис собі у закладки;-)

Ви також можете підтримати авторів каналів, які вам сподобалися. Як саме —
шукайте в їхньому описі. Підтримати нашу роботу можна тут
 \url{https://www.patreon.com/behindthenews} Дякуємо, що ви з нами @igg{fbicon.heart}🖤

@igg{fbicon.right.arrow.curving.down} А ось і наш список:

🦉Paleo Legends \url{https://bit.ly/3wOf4FH}

Хто з дітей не любить динозаврів та інших доісторичних істот? Хлібом не годуй — дай потеревенити про троодонів, тиранозаврів та решту героїв Спілберґа. Хлопець, який ховається за ніком Paleo Legends, вирішив спрямувати цю порожню зацікавленість в наукове русло і створив канал про палеонтологію.

Проєкт стартував у далекому 2018 році, тож якщо дивитися відео в хронологічному порядку, можна прослідкувати, як дорослішає голос ведучого;-) Наразі на канал підписалося менше тисячі людей. Це незаслужено мало — динамічні розповіді не залишать байдужими не тільки дітей, а й дорослих. Головне — анітрохи не нудно і на одній хвилі з однолітками.

🦉ХабблТраббл \url{https://bit.ly/3tuCQ7H}

Богдану сім, він ходить у другий клас і мріє стати астрофізиком. Але не тільки
мріє, а й цілеспрямовано йде до своєї мети. Ось лише короткий перелік тем, які
його цікавлять: чорні діри відригують, на Меркурії є вода, Сатурн має
шестикутник, що змінює колір, Термінатор — це не тільки I'll be back, а й межа
між днем і ніччю, квантова фізика цікавіша за мультики, Ньютон відкрив не те,
про що всі думають, між Марсом і Юпітером є планета-Церера! А ще про Ейнштайна,
Гокінґа, мікроби на МКС і як вони там опинилися, шатли, космічний туризм,
метеорити і місця для пошуку життя в космосі. Мерщій підписуйтеся на канал і
підписуйте своїх дітей. Разом пізнаємо Всесвіт)

🦉Агент Маргарита \url{https://bit.ly/3gb7wXA}

Хочете зацікавити дітей наукою, але щоб вони при цьому не почувалися, як на уроці? Тоді підпишіть їх на канал їхньої ровесниці Маргарити. Школярка в буквальному сенсі відкриває світ, збирає зразки для досліджень, розглядає рослини під мікроскопом, захоплено розповідає про живу природу, охорону рослин і тварин. Ваші діти неодмінно подружаться з нею. Наукову частину всіх випусків рецензує батько Маргарити — незрівнянний ботанік у кедах — Олексій Коваленко, відомий також, як автор Довколаботаніки \url{http://bit.ly/2NJsX2N}

🦉Новак Володя \url{https://bit.ly/3x13o2y}

Володя не лише блогер, а й справжній пластун — з рою Чорні хлопці. На своєму каналі він покаже, як готувати чіабату та пляцки. Навчить грати в «загадкові малюнки», «надувний кран» і «дрозда». Розкаже, як захиститися від коронавіруса по-пластунськи та як мандрувати під час карантину, коли табори не проводяться. А ще він знає «кричалки», які єднають друзів і роблять прогулянки веселішими.

🦉Лука в ефірі \url{https://bit.ly/3ge78HF}

Луці Колибі десять років і він читає книжки. Але не просто читає, а й робить на них короткі відеоогляди. В середньому по одному випуску кожні два тижні. Зазвичай це дитяча та підліткова література в жанрі фентезі та детективи. Але інколи на Луку сходить натхнення і він розповідає цікаві факти про класиків — Лесю Українку, Івана Франка, Тараса Шевченка. Сподіваємося, приклад цього хлопця заразить і ваших дітей любов’ю до читання. Наразі на каналі усього 199 підписників. Побачимо, скільки буде після нашого допису;-)

🦉Природа ТВ \url{https://bit.ly/3dlCdYe}

Усе почалося з простого оголошення на стовпі, де невідома авторка просила підписатися на її канал. Любов до природи, зворушлива історія і маркетинг сотень людей дали дитині тисячі підписників протягом кількох днів. Котик Пушинка, папуга Кеша, чотири курки — Біла, Коні, Мері, Конюшина, а ще рибка Карась — головні герої коротких відео десятилітньої Віталіни. Дівчинка нічого не коментує, не з’являється в кадрі — бо вважає, що саме її тваринки є популярними, а не вона. Цю неймовірну історію читайте за посиланням \url{https://bit.ly/3mQyAwp}

🦉Казки Лірника Сашка \url{https://bit.ly/3wRjtYm}, \url{https://bit.ly/2QlNvTh}

Сашка Лірника називають найкращим казкарем України. Можливо так і є — важко
пригадати, хто ще в нашій країні набув слави рівня Діда Панаса. До того ж
Лірник не переповідає народні казки, він їх сам і вигадує. На каналі їх
сила-силенна, слухати  — не переслухати! Отож щовечора о 21:00 вмикайте ютуб,
щоб наживо послухати нову оповідку. Але це ще не все! З автором можна навіть
потеревенити. Діти можуть поставити йому запитання, на які він залюбки
відповість. Де ви ще таке бачили?

🦉Наука для дітей \url{https://bit.ly/3a6onXG}

Гей! Не лякайтеся — в науці, про яку розповідають на цьому каналі, немає і
близького того, що можна почути на іспитах або контрольній роботі. Автори
проводять експерименти, які можна повторити вдома. Цілком безпечні — хату не
спалять. Хіба ж не цікаво? Але ці дослідження не просто «для фану» — розумахи
просто й доступно пояснюють, як і чому відбуваються різні природні явища.
Підписники каналу геть по-новому подивляться на науку. Дорослим, обіцяють, теж
має бути цікаво.

🦉Ukranimaua \url{https://bit.ly/3uSKdFV}

УкрАніма — канал для всієї родини. Повна колекція мультфільмів студій
«КиївНаукФільм» та «УкрАнімаФільм», починаючи з 1961 року. Серед них: Як
Козаки..., Вій, Котигорошко, Енеїда, Капітошка, Івасик-Телесик, Як Петрик
П'яточкин слоників рахував, Ходить гарбуз по городу і багато інших. Чимало з
них ще совєцьких часів, тому не завжди українською мовою. Але ж ви можете
обрати саме те, що вам потрібно;-) Познайомте дітей з мультфільмами, на яких ви
самі зростали.

🦉Valera i анімуємо українською \url{https://bit.ly/3djh94O}

Мультфільми можна не тільки дивитися, але й створювати. На своєму каналі Валера
розкаже про все, що стосується анімованих фільмів і їх виробництва. «Якщо ти
думаєш, що чогось не можеш — це тому, що ти дивишся не на тих людей.
Невпевненість в тебе заклали рідні і друзі, але не зі зла. Просто хтось колись
зробив те саме з ними. Неважливо якій саме справі ти присвячуєш життя. Якщо
маєш бажання, ти можеш створити власний мультфільм, — запевняє Валера. — На це
ніяк не впливає вік, фах або освіта. Лише ти маєш той унікальний досвід, яким
можеш поділитися зі світом. Забудь, що ти чогось не вмієш, чи не знаєш. Забудь
про інших і як це має бути «правильно». Валера тобі допоможе.

🦉Хмаринка \url{https://bit.ly/3sf5h80}

Можливо саме підказки Валери надихнули подружжя Тученків із Лисичанська на
створення мультиків. За останні три роки вони випустили близько сотні
україномовних серій. І на цьому не збираються зупинятися. Олена пише сценарії,
Сергій працює з відеомонтажем. Допомагає їм маленький син Максим — наприклад,
озвучує казкових героїв. Сюжети для історій беруть у знайомих дитячих
письменників. Головна авдиторія каналу — малюки від двох до семи років. Відео
невигадливі, але саме ця щирість і підкуповує. Подружжя каже, що намагається
вдосконалити свої навички.

🦉Янко Гортало — цікавинки для дітей \url{https://bit.ly/3acx82G}

Навчально-розважальний канал для, як кажуть самі автори — чистеньких і замазур,
слухняних і неслухів, мрійників і дзиґ. Чого тут тільки немає: пісні, віршики,
мультики, відео про складні речі простими словами. Наприклад, тут вам розкажуть
про види транспорту та його будову, з чого складається аеропорт, що таке
антоніми, склади та ономатопея (те саме, що няв-няв та му-му). Пояснять, як
визначати час, навчать готувати піцу і навіть робити орігамі. А ще — з цим
каналом можна робити руханку! Гасло — «Люби українське, друже! 💛💙».

🦉Ya Victoria \url{https://bit.ly/2Q6Epdd}

На своєму каналі Вікторія розказує про своє життя, собачку, уроки, мандри,
монтує кліпи та скетчі про школу. Інколи їй в цьому допомагає молодша сестра. А
ще вона дуже любить розпаковувати на камеру свої покупки — тому будьте
обережні, можлива реклама. Одне слово — цей канал — дитяча версія шоу «за
склом». Ваші діти зможуть побачити, чим живе їхня ровесниця.

🦉HelloAnn \url{https://bit.ly/3dk993m}

Як і Віка, Аня любить знімати смішні скетчі а-ля «очікування-реальність», відео
в стилі «зроби сам», влаштовує кумедні челенджі. Захоплюється малюванням,
спортом, подорожами та вивченням англійської мови, любить читати книжки,
подорожувати та ділитись усім цим на своєму каналі. Інколи в кадрі з’являються
назви брендів — тож може бути, що це «продакт плейсмент». Загалом канал Ані
нагадує сторінку якоїсь «селебрітіз» — у неї навіть є свої фани, з якими вона
регулярно влаштовує стріми.

🦉Вишиванчик — дитячий ютуб канал (на жаль, цей канал потрапиав у спам-фільтри
фейсбуку, тому лінк не можемо дати, але спробуйте знайти його на ютубі).

В описі до каналу написано, що його створено для «виховання патріотизму,
популяризації української мови серед дітей та дорослих в інтернеті». Насправді
на каналі немає ніякого пафосу та «шароварщини» — розповідають про аномальну
поведінку тварин, порівнюють, як звучать найпопулярніші мови світу, показують
відео, яке тренує пам’ять, добірки веселощів, ігри «відгадай пташку за її
голосом», озвучують дитячі комп’ютерні ігри українською мовою та багато іншого.

🦉Книга-мандрівка. Україна \url{https://bit.ly/3sd0Hat}

Персонажі «Книги-мандрівки» ожили в українському YouTube. Як українцю розгадати
писемність стародавньої цивілізації майя, не виходячи з кабінету? Чому ніколи
не варто викликати на дуель Нобелівського лавреата? Хто допоміг Миколі Гоголю
створити перше українське фентезі? Як харків’янка змінила світ балетної моди та
ще й «Оскар» отримала? Відповіді на ці та інші питання можна знайти на цьому
каналі. Його герої — всесвітньо відомі українці, об’єкти та події — на жаль, не
завжди добре знані на батьківщині. Події перших п’яти випусків розгортаються в
Харкові та Римі, в Одесі та Гватемалі, на Полтавщині та Нью-Йорківщині. Крім
того автори проєкту створили мобільний застосунок, який допомагає познайомитися
з цікавими локаціями України.

🦉Anitube \url{https://anitube.in.ua/}

Не зовсім ютуб, але хочеться порадити цей некомерційний ресурс. Аніме з кожним
роком набуває популярности. Як правило, це серіали, що складаються з декількох
серій. Головними героями можуть бути як дорослі, так і підлітки — це робить
аніме одним із найзатребуваніших жанрів серед підліткової авдиторії. Основою
для аніме часто служить манґа (японські комікси). Anitube — либонь найбільша
збірка аніме українською мовою. Увага — на сайті використовується безкоштовний
відеоплеєр, який може пропонувати рекламу. Автори каналу не мають з цього
жодного зиску і не контролюють зміст реклами.

🦉Dream School \url{https://bit.ly/3gdiCvm}

Dream School — «школа мрії», яка займається альтернативною освітою в Одесі. На каналі реалізують проєкт «Дистанційка», який під час карантину зробили доступним для всіх. Не беремося судити про ефективність такої освіти, але для загального розвитку та просто «позалипати» канал ідеально підходить. Креативні відео з цікавезною інформацію про все на світі. Чому не варто боятися крові? Як дружити з роботом? Митися чи не митися (привіт Лесю Подерв’янському)? Як покинути незалюднений острів? Про це та багато іншого розкажуть вчителі-блогери.

🦉Хоробрі зайці \url{https://bit.ly/2RDhTcz}

Українсько-іспанський розвиваючий мультсеріял про мандрівки Хоробрих Зайців. Розраховано на дітей від трьох до п’яти років. Невгамовні та допитливі брат і сестра Боп та Бу разом зі своєю зайцеродиною мандрують зайцебусом через джунглі, савану, ліс, поле і гори. Завжди сповнені геніальних ідей, як потоваришувати попри відмінності та весело грати разом. Девіз Хоробрих Зайців: ми можемо здаватися один одному різними, але все ж знайдемо спосіб стати друзями».

🦉Веселі Саморобки \url{https://bit.ly/2QxgzHK}

Назва каналу говорить сама за себе — веселий та добрий канал, який розважить ваших дітей. Вони навчаться власноруч виготовляти собі іграшки, листівки-трансформери та конверти до них, зошити-сумки, неповторні упаковки для подарунків, кімнатні прикраси, оригінальні закладки для книг, календарики, блокноти з секретом, фоторамки та ще багато цікавого. Канал регулярно оновлюється.

🦉Лепетуни \url{https://bit.ly/2REyJrt}

Головні герої каналу...сім’я язиків на прізвище Лепетуни. Вони потрапляють у різні лінгвістичні пригоди, в кінці кожної — коротке пояснення, як правильно говорити те чи інше слово. Усього одинадцять випусків. Це освітній проєкт для тих, хто хоче вдосконалити свої знання української мови. Як казав поет-гуморист Глазовий, пишатися незнанням рідної мови не слід, бо «якраз така біда в корови: має, бідна, язика і не знає мови».

🦉Дитинство TV \url{https://bit.ly/3dkDkHz}

Канал для малюків від двох до восьми років. Автори осучаснюють класичні народні дитячі пісні, перекладають світові хіти українською мовою, створюють власні колискові. На каналі також можна послухати найрізноманітніші казки. Усе це оздоблено авторською анімацією та 3Д графікою. Контент давно не оновлювався, але за попередні роки назбиралися десятки відео. Радимо на прохання наших читачів.

❗️Україномовних каналів для дітей на ютубі виявилося так багацько, що ми й половини з них не оглянули. З нетерпінням чекаємо на ваші доповнення в коментарях! Діліться улюбленими ютуб каналами ваших дітей, найцікавіші з них потраплять в оновлений допис.

💛💙Просимо якнайбільшого поширення та вподобайок, нехай всі знають про український ютуб. Може нарешті припинять крутити своїм дітям російський контент.

