% vim: keymap=russian-jcukenwin
%%beginhead 
 
%%file 22_12_2021.fb.fb_group.story_kiev_ua.2.novogodnie_igrushki.pic.6.cmt
%%parent 22_12_2021.fb.fb_group.story_kiev_ua.2.novogodnie_igrushki
 
%%url 
 
%%author_id 
%%date 
 
%%tags 
%%title 
 
%%endhead 

\iusr{Ольга Писанко}
И сосулька красная ещё осталась ...

\iusr{Лариса Зарицкая}
И такие были. Ностальгия.

\iusr{Анна Шустерман}
Кукурузка и две сосульки сохранились.

\iusr{Татьяна Соловьева}

Можно было сосульку крутануть и она закручивалась на нитке, а потом сама долго
раскручивалась в противоположную сторону. Почему-то невозможно было глаз
оторвать)


\iusr{Валерий Зиновьев}
\textbf{Татьяна Соловьева} И не только сосульку  @igg{fbicon.smile} 

\iusr{Наталья Иванова}
\textbf{Татьяна Соловьева}

О, и я люблю это проделывать! Осталось несколько таких сосулек из детских
ёлочных игрушек. До сих пор как нарядим ёлку, как крутану сосульку, а она потом
быстро-быстро раскручивается! Класс! Люблю!


\iusr{Віта Ілейко}
у мeня 2 зeлeные и 2 розовые сосульки

\iusr{Irena Visochan}
\textbf{Анна Шустерман} и у меня были точно такие, точь в точь.

\iusr{Anna Portman}
\textbf{Irena Visochan} и у меня  @igg{fbicon.face.smiling.hearts} 

\iusr{Irena Visochan}
\textbf{Anna Portman} Как говорит мой муж \enquote{мы все из одного песочника}

\iusr{Галина Гурьева}
Красненькие сосульки живы еще!

\iusr{Анна Сидоренко}
У меня и сейчас есть такие игрушки только дед мороз разбился.

\iusr{Арт Юрковская}

У меня сосульки и кукуруза есть. Это Хрущевская игрушка. Помните \enquote{Кукуруза -
царица полей}. Тогда их выпустили в виде новогодних игрушек.

\iusr{Tamara Pysarenko}

У меня тоже есть такие

\iusr{Лариса Шелест}
И у нас ещё и другие

\iusr{Алла Волонтирець}
Початок кукурузы...
Наверное, эта игрушка в память о кукурузной кампании Генсека Хрущёва.

\iusr{Алла Вайнерман}
А у меня целый ящик таких игрушек. Только из-за кошки уже 5 лет на ёлку не вешаем.

\iusr{Зинаида Ткаличева}
Эти игрушки у нас есть все!  @igg{fbicon.face.smiling.hearts}{repeat=2} 

\iusr{Лариса Волошина}
Папин шарик, вроде бы ещё довоенный.

\ifcmt
  ig https://scontent-frx5-1.xx.fbcdn.net/v/t39.30808-6/269843617_2647792622183203_1565828920723815913_n.jpg?_nc_cat=111&ccb=1-5&_nc_sid=dbeb18&_nc_ohc=p35bbMSYItMAX9r3CYl&_nc_ht=scontent-frx5-1.xx&oh=00_AT9TiPCBvt6oTJ8IBRGhKtl48UWRqix_8pbJcG3tKbx9mQ&oe=61CBFF54
  @width 0.3
\fi

\iusr{Светлана Александренко}
И у нас есть игрушки с 50-х и 60х.
Скоро и я их выставлю.

\iusr{Юлія Качура}

Это более современные игрушки. Моя мама покупала такие моему сыну, 94 года
рождения, т. е. они приблизительно 97-го, 98-го года выпуска. У мамы добрая
традиция добавлять ксждый год новые игрушки в коллекцию. Самые старые в этой
коллекции еще из ее детства, 50-х. Мой самый любимый - белый гипсовый медведь.

\iusr{Анна Шустерман}
\textbf{Юлія Качура}, клубничка, кукурузка, гроздь винограда, шишка и сосульки были ещё в моем детстве.

\iusr{Юлія Качура}
\textbf{Анна Шустерман} 

сосульки здесь, кстати, из более старых, 70-80-х, и да, конечно, были и раньше
такие игрушки. Просто конкретно эти(они мелкие, на небольшую елочку, еще
стеклянные, дутые, как положенно -мне близко знакомы @igg{fbicon.face.eyes.star} 
