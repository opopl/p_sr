% vim: keymap=russian-jcukenwin
%%beginhead 
 
%%file 22_06_2021.fb.goncharenko_aleksej.1.vov_22_jun_pamjat
%%parent 22_06_2021
 
%%url https://www.facebook.com/alexeygoncharenko/posts/4222506224455249
 
%%author Гончаренко, Алексей
%%author_id goncharenko_aleksej
%%author_url 
 
%%tags germania,istoria,nacizm,sssr,ukraina,vov,vov.22.06.1941
%%title Сьогодні День скорботи і вшанування пам'яті жертв війни. Вічна пам'ять!
 
%%endhead 
 
\subsection{Сьогодні День скорботи і вшанування пам'яті жертв війни. Вічна пам'ять!}
\label{sec:22_06_2021.fb.goncharenko_aleksej.1.vov_22_jun_pamjat}
\Purl{https://www.facebook.com/alexeygoncharenko/posts/4222506224455249}
\ifcmt
 author_begin
   author_id goncharenko_aleksej
 author_end
\fi

80 років тому нацистська Німеччина напала на СРСР. Розпочалася німецько-радянська війна. 

Для України її наслідки були катастрофічними. Сам факт того, що фронт двічі
повністю прокотився територією України вже багато про що говорить. 

Понад 7 млн українців воювали в складі армій СРСР, кожен другий з них загинув. 

Сукупні демографічні втрати України внаслідок війни оцінюються в 8-10 мільйонів
осіб. Центральна, Південна і Східна Україна втратила 30\% населення, Галичина -
22\%, Волинь і Полісся - 12\%.

Сьогодні День скорботи і вшанування пам'яті жертв війни. Вічна пам'ять!

\ifcmt
  pic https://scontent-mxp1-2.xx.fbcdn.net/v/t1.6435-9/203770657_4222506077788597_2019262092153606867_n.jpg?_nc_cat=104&ccb=1-3&_nc_sid=730e14&_nc_ohc=eVZJAS4hm1oAX-k4GHH&_nc_ht=scontent-mxp1-2.xx&oh=ba1983bbd77d072f239d3f37b2aed1b1&oe=60D75D54
	caption Німецькі солдати перетинають кордон Радянського Союзу. Джерело — УІНП
\fi

P.S. На фото німецькі солдати перетинають кордон Радянського Союзу. Джерело — УІНП
