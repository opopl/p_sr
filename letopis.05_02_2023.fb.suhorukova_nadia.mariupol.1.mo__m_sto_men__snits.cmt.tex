% vim: keymap=russian-jcukenwin
%%beginhead 
 
%%file 05_02_2023.fb.suhorukova_nadia.mariupol.1.mo__m_sto_men__snits.cmt
%%parent 05_02_2023.fb.suhorukova_nadia.mariupol.1.mo__m_sto_men__snits
 
%%url 
 
%%author_id 
%%date 
 
%%tags 
%%title 
 
%%endhead 

\qqSecCmt

\iusr{Ania Ivanova}

Нарешті я Вас читаю.

\begin{itemize} % {
\iusr{Nadia Sukhorukova}
\textbf{Ania Ivanova} в мене є телеграм канал. Я там в той час писала, коли була вигнана з фб. І зараз пишу

\iusr{Ania Ivanova}
\textbf{Nadia Sukhorukova} дякую. Обов'язково замовлю книгу, шкода, що не має можливості отримати автограф.
Я Вас читала з самих перших дописів, і не могла дихати від жаху і болю. І страшенно раділа, коли Вам вдалося виїхати. Обіймаю.
\end{itemize} % }

\iusr{Helen Plotnikova}

З поверненням, Надія ❤️

\iusr{Оксана Стомина}

Наче цей текст писала я... Підписуюсь під кожним словом, під всіма Вашими
відчуттями й почуттями. Дякую!

\iusr{Виктория Костоглодова}

Надія з поверненням дочекалися ♥️♥️😍😍🌹🌹🌹🌹

\begin{itemize} % {
\iusr{Nadia Sukhorukova}
\textbf{Виктория Костоглодова} дякую! Але не знаю на довго чи ні)

\iusr{Виктория Костоглодова}
\textbf{Nadia Sukhorukova}, 

Будемо надіятися що назавжди ♥️♥️♥️

Хоча ці потвори правди не люблять, але ми її будем говорити та розповідати
усюди, і завжди!!!!

\iusr{Настя Гонгало}
\textbf{Nadia Sukhorukova} спасибо вам большое за всё, что вы делаете, я рада, что вы вернулись

\iusr{Nadia Sukhorukova}
\textbf{Настя Гонгало} и я рада. Но делаю я не только хорошие вещи))) за некоторые лучше не благодарить)))

\iusr{Настя Гонгало}
\textbf{Nadia Sukhorukova} вы лучик света во тьме, вы символ )))

\iusr{Nadia Sukhorukova}
\textbf{Настя Гонгало} спасибо, Настя, это неожиданно и приятно!

\iusr{Настя Гонгало}
\textbf{Nadia Sukhorukova}

\ifcmt
  igc https://scontent-fra3-1.xx.fbcdn.net/v/t39.1997-6/47270791_937342239796388_4222599360510164992_n.png?stp=cp0_dst-png_s110x80&_nc_cat=1&ccb=1-7&_nc_sid=ac3552&_nc_ohc=PK3BGAKvqoQAX-CRRgJ&_nc_ht=scontent-fra3-1.xx&oh=00_AfBm_3U8H4uVmH4dLwGdFNGjrBjskrE-x-r5Mq98o-5c4Q&oe=63EC64B2
	@width 0.2
\fi

\iusr{Ельвіра Та Григорій Ілюшенко}
\textbf{Nadia Sukhorukova} Вы всегда правильно все делаете, не наговаривайте на себя. Вы справедливая.
\end{itemize} % }

\iusr{Alena Pakhomova}

Я так рада, что ты вернулась.

\begin{itemize} % {
\iusr{Nadia Sukhorukova}
\textbf{Alena Pakhomova} не знаю сколько здесь продержусь... Если что у меня есть телеграмм канал

\iusr{Alena Pakhomova}
\textbf{Nadia Sukhorukova} да, я подписана с самого начала 🙂

\iusr{Ольга Вингольц}
\textbf{Nadia Sukhorukova} можно ссылку на канал?
\end{itemize} % }

\iusr{Наталия Трокачевская}

Такі ж почуття були, коли покидали наше місто....

\enquote{Поїхати неможливо залишитись} - де поставити кОму?

Ми поїхали, коли залишатися було страшніше, ніж їхати в невідомість....

Виїхали 14-го березня, а 15-го вранці був прильот в наш будинок (Миру, 105)

Загинули сусіди....

\begin{itemize} % {
\iusr{Nadia Sukhorukova}
\textbf{Наталия Трокачевская} 

це ж поряд. Ми були у 99 та у будинку знайомих, на Осипенко. Коли прилетіло у
дах знову перейшли у підвал Миру, 99. Там ховалися. У 105 горіли поверхи.
Полум'я жерло квартири. Ми бачили як горить, але нічого не могли подіяти

\iusr{Alena Hriy}
\textbf{Наталия Трокачевская} саме такі почуття були, залишатися страшніше, ніж йти пішки з міста в невідомість

\iusr{Alena Hriy}
\textbf{Nadia Sukhorukova} моя подруга з 90 будинку, ховалися через дорогу в 103, бачила на власні очі як р\_снявий танк вистрелив у вікна її квартири на другому поверсі...

\iusr{Наталия Трокачевская}
\textbf{Nadia Sukhorukova} так, було страшно коли щось горіло. Ти розумів, що там можуть бути люди, а ти нічим не можеш допомогти,
бо води нема
\end{itemize} % }


\iusr{Натуля Захарова}

Наш рідненький, ми повернемось!

\iusr{Olexandra Hutnik}

Рада знову тебе читати, моя мила Надія) І чекаю з нетерпінням твоєї книжки!

\begin{itemize} % {
\iusr{Nadia Sukhorukova}
\textbf{Olexandra Hutnik} я рада тебе чути Сашенька! Дуже рада!

\iusr{Olexandra Hutnik}
\textbf{Nadia Sukhorukova} книгу я замовила. А як потім підпис автора отримати?))

\iusr{Nadia Sukhorukova}
\textbf{Olexandra Hutnik} я підпишу! Знайду можливість, Сашенька і обов'язково підпишу
\end{itemize} % }

\iusr{Ольга Кашпор}

Це біль, якому немає межі, Надь. Він не слабне, не минає, не еволюціонує у
щось, що можна пере-жити. Це одвічна каменюка - не в гролі, ні на серці - на
все тіло.

\begin{itemize} % {
\iusr{Nadia Sukhorukova}
\textbf{Ольга Кашпор} так, Оля, це точно. Одвічна каменюка

\iusr{Алла Алимова}
\textbf{Ольга Кашпор} 100\%
\end{itemize} % }

\iusr{Ельвіра Та Григорій Ілюшенко}

А ми думали: « Як буде шкода помирати в дорозі після того, як отсиділи 3 тижні
безвилазно у підвалі і вижили»...

\begin{itemize} % {
\iusr{Nadia Sukhorukova}
\textbf{Ельвіра Та Григорій Ілюшенко} а мені було вже всеодно. Тільки би не в підвалі. Хоча, коли виїхали з міста відразу жити захотілося

\iusr{Ельвіра Та Григорій Ілюшенко}
\textbf{Nadia Sukhorukova} так і є. Жити хотілося. Але не давали, тzzzарі, до гаража добігти аж до 25 березня, носа висунути не могли на сонечко глянути...

\iusr{Nadia Sukhorukova}
\textbf{Ельвіра Та Григорій Ілюшенко} вам здається є що розповісти про ці дні в Маріуполі. \textbf{Наталя Дєдова} записує історії нашіх маріупольців. Щоб вони збереглися. Розкажіть їй, будь ласка.

\iusr{Natalya Dedova}
\textbf{Nadia Sukhorukova} вже записалися. Ще влітку. Дякую, Надюша!

\iusr{Nadia Sukhorukova}
\textbf{Наталя Дєдова} дай посилання, будь ласка

\iusr{Nadia Sukhorukova}
\textbf{Наталя Дєдова} мені дуже цікаво. Вони так пишуть сильно. Наче про мене

\iusr{Natalya Dedova}
\textbf{Nadia Sukhorukova}

\href{https://civilvoicesmuseum.org/stories/%22kvartira-dacha-uliki.-kak-eto-brosit%22}{%
"Квартира, дача, вулики. Як це залишити?", Ельвіра Ілюшенко, 13.07.2022, civilvoicesmuseum.org%
}

\ifcmt
  igc https://i.paste.pics/491c3fd7f1863da480615d2708c8b8ea.png
	@width 0.6
\fi

\iusr{Ельвіра Та Григорій Ілюшенко}
\textbf{Nadia Sukhorukova} 

я їй давала інтерв'ю через 3 тижні, як приїхала і прийшла до тями. Не хотіла
згадувати ні про що, а їй відмовити не змогла. Ч

Знаю її ще до війни, і тут таку трагедію вони пережили... Дуже шкода талановитого
Віктора..

\iusr{Nadia Sukhorukova}
\textbf{Наталя Дєдова} спасибі, Наташік

\iusr{Ельвіра Та Григорій Ілюшенко}
\textbf{Natalya Dedova} ой, не все тоді розповіла... дуже було боляче, але, Наталочко, Вам ще гірше. Так хочеться Вас побачити, обійняти і підтримати!

\end{itemize} % }

\iusr{Алла Алимова}
💔

\iusr{Yuliya Mikhaylovskaya}

Ми вирішили вийти з підвалу, коли вже усі будинки поруч були зруйновані - я,
діти, мама та відчим на інвалідній колясці... Дві години йшли, мене ледве не
застрелив на очах у дітей безумний орк, це шлях жахіття, який досі сниться.
Через три дні дім зруйнували рашиські танки.

\begin{itemize} % {
\iusr{Nadia Sukhorukova}
\textbf{Yuliya Mikhaylovskaya} Юлія, це жах. \textbf{Наталя Дєдова} Наталя Дєдова, ти це чула? Є в тебе ця історія?
\end{itemize} % }

\iusr{Людмила Миснянкина}

Спасибо Вам большое за Ваши публикации. В каждом слове - жизнь, прожитая в тот
период. Это невозможно забыть и простить. Понять может только тот, кто это
пережил. Ужас, боль, страх не за себя, а за детей и внуков. И основной вопрос -
ЗА ЧТО нам такое горе?

\iusr{Оксана Спивак}

\ifcmt
  igc https://scontent-fra3-1.xx.fbcdn.net/v/t39.1997-6/318693632_688883395960150_8201081418578648568_n.webp?stp=dst-webp_s168x128&_nc_cat=107&ccb=1-7&_nc_sid=ac3552&_nc_ohc=h31oTMBjHYcAX8kdFAO&_nc_ht=scontent-fra3-1.xx&oh=00_AfBXzX4aUDoKw-05jTIlkHU0udXh2nEkZzl6PSIG1lC_uA&oe=63ED49B1
	@width 0.2
\fi
