% vim: keymap=russian-jcukenwin
%%beginhead 
 
%%file 25_03_2018.stz.news.ua.mrpl_city.1.belaja_akacia
%%parent 25_03_2018
 
%%url https://mrpl.city/blogs/view/belaya-akatsiya
 
%%author_id burov_sergij.mariupol,news.ua.mrpl_city
%%date 
 
%%tags 
%%title Белая акация
 
%%endhead 
 
\subsection{Белая акация}
\label{sec:25_03_2018.stz.news.ua.mrpl_city.1.belaja_akacia}
 
\Purl{https://mrpl.city/blogs/view/belaya-akatsiya}
\ifcmt
 author_begin
   author_id burov_sergij.mariupol,news.ua.mrpl_city
 author_end
\fi

\ii{25_03_2018.stz.news.ua.mrpl_city.1.belaja_akacia.pic.1}

Какие только деревья не встречаются сегодня на мариупольских улицах, в парках и
скверах. Стройные тополя и плакучие ивы, красавцы каштаны, украшающие себя в
конце мая свечками-соцветиями, и обладательницы широченных листьев -  катальпы,
гостьи из далекой Мексики, а еще несколько разновидностей елей и сосен и совсем
уж экзотичные для наших южных мест белоствольные березы. Но здесь перечислена
лишь очень малая часть из всего многообразия зеленого украшения нашего города.
По данным биологов краеведческого музея, на территории Мариуполя произрастает
более ста шестидесяти видов деревьев и кустарников, включая и плодовые.

Выделить молодым людям из этого многообразия самое \enquote{мариупольское} дерево,
конечно, трудно. Иное дело старожилы, чьи семейные корни далеко уходят в
историю нашего приморского города. Те без особых раздумий скажут: \enquote{Белая
акация, конечно!} Длительное время эта уроженка африканских саванн безраздельно
царила на мариупольских улицах и дворах горожан. Именно с нее в конце ХIХ века
начиналось озеленение города.

1896 год. В уездном городе Мариуполе открылось отделение газеты \enquote{Приазовский
край}, издававшейся в Ростове-на-Дону. В его штате всего один сотрудник –
Александр Попов, в будущем известный писатель Александр Серафимович, автор
романа \enquote{Железный поток}. Это произведение школьники в свое время изучали на
уроках литературы. К периоду жизни Серафимовича в нашем городе относятся такие
вот строки из его корреспонденции: \enquote{На горе над морем раскинулся прекрасный
сад. Шумят деревья, акации качают над тенистыми аллеями свои ветви, украшенные,
как гроздьями, белыми цветами}.

Образ цветущей в начале лета белой акации стал с давних пор как бы символом
нашего города. Обостренные чувства поэтов это уловили, что и отразилось в
поэтических строках. Душное лето сорок первого года. Поэт и воин Николай
Берилов запомнил его таким:

\begin{qqquote}
Нас провожали,

не было оваций.

Лишь на перроне –

Грусть во всю длину,

под плач родных

и тихий шум акаций

мы уходили молча

на войну.
\end{qqquote}

Мариупольский поэт и прозаик Анатолий Заруба, вспоминая свое военное детство, пишет:

\begin{qqquote}
Пусть ночами тебе не снится

время дикое оккупации,

изможденные серые лица

и стволы обгоревших акаций.	
\end{qqquote}

Полистаем поэтическую антологию Мариуполя. Образ акаций встречается в
творчестве Григория Кисунько и Алексея Шепетчука, и Владимира Смирного,
прошедшего суровые испытания Великой Отечественной войны, тем не менее,
сохранившего любовь к жизни и оставшегося до конца жизни оптимистом. Вот его
стихи:

\begin{qqquote}
Осенний ветер рвет

листву акаций,

А улица не молкнет никогда.

По ней идут на смену

Азовстальцы –

Великие художники труда.	
\end{qqquote}

Татарский поэт Максуд Сюндюкле так изобразил в своих стихах наш город:

\begin{qqquote}
Стоїть воно, підсвічуючи небо,

Все в шелесті акацій, у цвіту

Приморське місто.	
\end{qqquote}

Мариупольская Слободка в стихах Даниила Чконии предстает в следующих строках:

\begin{qqquote}
Цветут акации. Осядет пыль,

Которая лениво проплыла за трамваем,

А еще иногда такое бывает:

В улочку заглянет автомобиль.	
\end{qqquote}

Анатолий Белоус в одном из своих поэтических произведений, предворенного
строкой из стихотворения шахтерского поэта Николая Анциферова \enquote{И в оккупации
цвела акация...}, использовал образ любимого мариупольцами дерева для передачи
своих детских впечатлений:

\begin{qqquote}
Как она тогда цвела

Меж домами и у моря,

В белоснежном уборе

От верхушки до ствола!

Щелкал кованый сапог,

Конь в обозе тихо форкал –

У ее недвижных ног

Лепестки ссыпались горкой.

Ствол винтовки, вздетый вверх

Отшибал от веток локон –

Свет акации не мерк

В стеклах затемненных окон.

И ее медовый дым,

Исходя из снежных чащей,

Был не буйным, а грустящим,

Древним, но не молодым.

От верхушек до ствола.

И сейчас – как в снежной пыли...

Как же вырубить забыли?

В оккупации ж была...

Да еще там и цвела!	
\end{qqquote}

Но, конечно, не у всех поэтов акация вызывает печальные воспоминания. Ярко,
мажорно звучат строфы Ивана Битюкова:

\begin{qqquote}
Блестя от влаги дождевой,

Пройдя через ворота радуг,

В наш южный город, как домой,

Спешит \enquote{прямой} из Ленинграда.

Измерив сутками страну,

Он пролетел немало станций

И машут улицы ему

Платками белыми акаций.
\end{qqquote}

Белая акация на мариупольских улицах... Жаль только, что это неприхотливое и
по-своему красивое дерево, приобретающее особое очарование в пору своего
цветения, постепенно исчезает в Мариуполе. Постаревшие экземпляры
выкорчевываются, а вместо них высаживают пусть более броские, но порой
нестойкие и подверженные болезням породы деревьев. Хочется призвать горожан:
давайте же сохраним белую акацию – поэтический символ нашего города.
