% vim: keymap=russian-jcukenwin
%%beginhead 
 
%%file slova.agressia
%%parent slova
 
%%url 
 
%%author 
%%author_id 
%%author_url 
 
%%tags 
%%title 
 
%%endhead 
\chapter{Агрессия}
\label{sec:slova.agressia}

%%%cit
%%%cit_head
%%%cit_pic
\ifcmt
  tab_begin cols=2
		width 0.4
     pic https://strana.ua/img/forall/u/0/36/8870e29b30de20a65403b30c30b4b37560dc7956d45f7795881761.jpg
		width 0.46

     pic https://strana.ua/img/forall/u/0/36/2021-07-01_09h35_48.png
  tab_end
\fi
%%%cit_text
Парень рассказал о том, что многие украинские фанаты подходили к ним с
возмущениями, поэтому людям приходилось объяснять, что означают цвета их
одежды. При этом, по словам Толкача, только один человек подошел с целью
затеять драку. Сергей предполагает, что нападавший был под градусом.
Полиция увела \emph{агрессора} со стадиона.
Кроме того, по словам Толкача, многие украинские болельщики защищали их
%%%cit_comment
%%%cit_title
\citTitle{На Евро 2020 избили российского фаната Дениса Смоленского, что известно об инциденте}, 
Анна Копытько, strana.ua, 01.07.2021
%%%endcit

%%%cit
%%%cit_head
%%%cit_pic
%%%cit_text
Ні, ти кажеш, що держава втрачає монополію на насилля… Смотри, она дальше потом
ее возвращает, когда проходит. А у нас этого не произошло. И в результате мы
имеем то, что это вот такое насильственное решение каких-то конфликтов, оно
становится частью повседневности. И это касается не только каких-то
политических споров – это касается даже споров на дороге. Мы все прекрасно
знаем, что куча людей ездит с пистолетами, с автоматами, с битами, с газовыми
баллончиками, и периодически друг другу что-то делают, какие-то увечья наносят.
И этот уровень \emph{агрессии} он как раз же о чем говорит – что государство оно же не
спускает эту \emph{агрессию}, не дает ей выход, оно не наказывает хулиганов, да, оно
позволяет, оставляет им пространство, в котором они могут таким образом
самореализовываться.  У нас во дворе есть чувак, который терроризировал, сейчас
затих из-за того, что построили стоянку под домом. Два дома соседних построили
стоянку, и дальше начинается традиционный конфликт украинский в духе
рязановского фильма «Гараж» - начинается выяснение отношений кто же может
поставить авто, кто же не может поставить. И там одно быдло почувствовало, что
он там царь горы, с пистолетом бегал, выскакивал
%%%cit_comment
%%%cit_title
\citTitle{Украина в плену консервативного мышления: почему 30 лет шли не туда и что делать}, 
Сергей Иванов; Юрий Романенко, hvylya.net, 15.07.2021
%%%endcit

%%%cit
%%%cit_head
%%%cit_pic
%%%cit_text
То есть, иными словами, высшее руководство Украины публично называет Путина
\emph{газовым агрессором} и тут же предлагает этому \emph{агрессору}, у которого, вообще то
планы по уничтожению украинской государственности, а газовый кризис в Европе
является одной из промежуточных целей реванширующей империи, какую то смешную
"скидку" на транзит и какой то хитровымученый "план", чтобы европейских "лохов"
Путину было веселее "обувать" и еще самим с газом в ГТС остаться и какие то
деньги за транзит от агрессора сохранить.  Представляете, насколько "власна
добыча" из "трубы" важна для наших гавриков, если они \emph{агрессору} почти открытым
текстом транзитные "скидки" обещают))))))) Пришли козлята кремлевского
крокодила учить охотиться)))
%%%cit_comment
%%%cit_title
\citTitle{Руководство Украины публично называет Путина газовым агрессором / Лента соцсетей / Страна}, 
Богдан Савруцкий, strana.news, 24.10.2021
%%%endcit

%%%cit
%%%cit_head
%%%cit_pic
%%%cit_text
Гражданам Украины (прежде всего госслужащим и СМИ) предписывается сверху, как
называть то или иное событие и/или явление. Правда, пока вроде как
рекомендательно. Это пока еще не Геббельс и не «1984» Оруэлла. Но направление
движения — именно туда.  Делается это все во имя священной цели — борьбы с
российской \emph{агрессией} в информационном пространстве.Здесь следует сделать
замечание. Желающих повоевать на Донбассе, по слухам, все меньше и меньше.
Поскольку это опасно для жизни и здоровья. Зато бороться с информационной
\emph{агрессией} Кремля в столичных офисах — желающих хоть отбавляй. Более
того, эти желающие только тем и занимаются, что навязывают обществу свои, как
нынче модно говорить, нарративы, не обладая при этом соответствующими
интеллектуальными и моральными качествами
%%%cit_comment
%%%cit_title
\citTitle{Краткий словарь грантоедов под редакцией СНБО / Лента соцсетей / Страна}, 
Александр Карпец, strana.news, 26.10.2021
%%%endcit
