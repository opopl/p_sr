% vim: keymap=russian-jcukenwin
%%beginhead 
 
%%file 07_12_2022.fb.ryzhenko_sergij.ua.likar.dnepr.1.syn
%%parent 07_12_2022
 
%%url https://www.facebook.com/rsa010963/posts/pfbid0efpb6JbAvmQoJQVYpNPqLypajymosyTH9tV1Hy3cX734yS835btbf5Hibh5jKwmDl
 
%%author_id ryzhenko_sergij.ua.likar.dnepr
%%date 
 
%%tags medicina
%%title Єдиний, рідний, завжди любимий син...
 
%%endhead 
 
\subsection{Єдиний, рідний, завжди любимий син...}
\label{sec:07_12_2022.fb.ryzhenko_sergij.ua.likar.dnepr.1.syn}
 
\Purl{https://www.facebook.com/rsa010963/posts/pfbid0efpb6JbAvmQoJQVYpNPqLypajymosyTH9tV1Hy3cX734yS835btbf5Hibh5jKwmDl}
\ifcmt
 author_begin
   author_id ryzhenko_sergij.ua.likar.dnepr
 author_end
\fi

\obeycr
Єдиний, рідний, завжди любимий син...
Роман, 26 років, шахтар із Марганцю.
Щодня на зв'язку із мамою, яка заробляла копійку на життя в Америці.
У Марганці роботи за фахом не знайти.
П'ять днів тому зв'язок з сином зник.
Роман із важким пораненням живота, переломом таза та кінцівок, у комі доставлений до реанімації Мечникова.
Серце матері через океани відчуло лихо.
Кинула все, діставалася усіма видами транспорту до Дніпра.
Всю дорогу вона заплющувала очі і молилася, щоб син був живий.
Допомогли волонтери – знайшли у Мечникова.
Поруч з Романом була молода лікарка-анестезіолог Яна Токарик.
Виконувала перев'язки, переливала кров і була самою необхідною та найріднішою.
Матуся, побачивши сина, впала йому на груди.
Жити, жити, просто жити, - казала мама, доторкуючись до сина, який лежав весь перев'язаний бинтами до п'ят. 
P.S.: Вдячні волонтерам Мечникова, особливо тим, хто безмірно віддає тіло та душу найважчим пацієнтам - Ірина Ванжа, Вікторія Павлова, Тетяна Бондар, Андрій Ісаєв.
І тим, хто перебуває за межами нашої країни та допомагає апаратурою і ліками.
Це Олег Ястребов із Гамбургу зі своєю командою, компанія АТБ, яка стала для нас символом порятунку, і багато наших друзів, які допомагають лікарні Мечникова.
Це і є Україна - коли ми поруч, коли ми разом...
\restorecr

\obeycr
He is the only one, the closest and always beloved son...
Roman is 26 years old. He is a miner from Marganets.
He was always in touch with his mother, who earned a living in America.
You cannot find a job by profession in Marganets.
She lost contact with his son five days ago.
Roman with severe injury to the abdomen, fractured pelvis, and limbs, has been brought to the intensive care unit of Mechnikov hospital. He was in a coma.
Mother's heart sensed tragedy across the ocean.
She gave up everything and got to the Dnipro by all means of transport.
She closed her eyes the entire trip and prayed for her son to be alive.
Volunteers helped her; they found Roman in the Mechnikov hospital.
Yana Tokaryk, a young anesthesiologist, was around Roman all the time.
She changed bandages, transfused blood, and was the most necessary and the closest person to him.
Having glanced at her son, mother fell on his chest.
\enquote{You should live, live, just live}, - said mother, touching her son, who was lying all bandaged up from the head to the toes.
P.S.: We are grateful to the volunteers, who help Mechnikov hospital, especially those who immeasurably give their body and soul to the most diseased patients - Iryna Vanzha, Viktoriya Pavlova, Tetyana Bondar, Andriy Isayev.
We appreciate those who are outside our country and help with equipment and medicines.
One of them is Oleg Yastrebov from Hamburg and his team, and the \enquote{ATB Company} that has become a symbol of survival for us, and many other our friends who help Mechnikov hospital.
This is all about Ukrainians – we all are close, and we all are together...
\restorecr

\ii{07_12_2022.fb.ryzhenko_sergij.ua.likar.dnepr.1.syn.orig}
\ii{07_12_2022.fb.ryzhenko_sergij.ua.likar.dnepr.1.syn.cmtx}
