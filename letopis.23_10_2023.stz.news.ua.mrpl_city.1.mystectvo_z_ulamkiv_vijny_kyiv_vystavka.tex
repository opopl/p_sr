% vim: keymap=russian-jcukenwin
%%beginhead 
 
%%file 23_10_2023.stz.news.ua.mrpl_city.1.mystectvo_z_ulamkiv_vijny_kyiv_vystavka
%%parent 23_10_2023
 
%%url https://mrpl.city/news/view/mistetstvo-z-ulamkiv-vijni-u-stolitsi-vidkrilas-unikalna-vistavka-z-ulamkiv-zbroi-ta-tehniki
 
%%author_id demidko_olga.mariupol,news.ua.mrpl_city
%%date 
 
%%tags 
%%title Мистецтво з уламків війни - у столиці відкрилась унікальна виставка з уламків зброї та техніки
 
%%endhead 
 
\subsection{Мистецтво з уламків війни - у столиці відкрилась унікальна виставка з уламків зброї та техніки}
\label{sec:23_10_2023.stz.news.ua.mrpl_city.1.mystectvo_z_ulamkiv_vijny_kyiv_vystavka}
 
\Purl{https://mrpl.city/news/view/mistetstvo-z-ulamkiv-vijni-u-stolitsi-vidkrilas-unikalna-vistavka-z-ulamkiv-zbroi-ta-tehniki}
\ifcmt
 author_begin
   author_id demidko_olga.mariupol,news.ua.mrpl_city
 author_end
\fi

\begin{quote}
\bfseries
20 жовтня в рамках проєкту War Artists Union у Києві розпочала роботу
незвична виставка артоб'єктів, створених сучасними українськими митцями з
уламків зброї та техніки, які зібрали волонтери в українських містах, що
зазнали масованих обстрілів російської армії.
\end{quote}

Головна мета Проєкту War Artists Union - нагадати, що мистецтво може
перетворювати знаряддя вбивства на символи відродження. Водночас цілями проєкту
War Artists Union є культурна дипломатія у світі. Команда спілки митців прагне
відновити та посилити увагу до війни в Україні через мистецтво, показати, як
знищення може бути перетворене на символ відродження та миру. WAU має за мету
змінювати імідж країни, прагне, щоб Україну сприймали не як безпомічних жертв,
а як цінну частину світової спільноти. На меті також й розвиток бізнесу,
необхідний для стабільного існування України, WAU створює робочі місця, сплачує
податки та розвиває експорт. WAU дає надію і сприяє формуванню стійкого
ставлення до викликів війни.

\textbf{Читайте також:} \emph{У Нью-Йорку та Вільнюсі відкрилася виставка про трагедію розбомбленого росією Маріуполя}%
\footnote{У Нью-Йорку та Вільнюсі відкрилася виставка про трагедію розбомбленого росією Маріуполя, Наталія Сорокіна, %
mrpl.city, 11.10.2023, \par%
\url{https://mrpl.city/news/view/u-nyujorku-ta-vilnyusi-vidkrilasya-vistavka-pro-tragediyu-rozbomblenogo-rosieyu-mariupolya}
}

War Artists Union (WAU) – це команда українських митців з усієї України, яка
спеціалізується на створенні ексклюзивних артоб'єктів для дизайну інтер'єру з
уламків військової техніки та зброї. Проєкт втілює філософію трансформації: із
залишків війни народжується мистецтво, з засобів руйнації народжується
українське відродження. Як наголошують організатори виставки, кожен витвір
колекції має історичну, художню та утилітарну цінність, а усі матеріали, з яких
створені артоб'єкти,  пройшли перевірку, визнані безпечними та мають необхідну
документацію.

\ii{23_10_2023.stz.news.ua.mrpl_city.1.mystectvo_z_ulamkiv_vijny_kyiv_vystavka.pic.1}

Загалом експозиція колекції мистецтва War Artists Union налічує понад 20
артоб'єктів. Зокрема, це колекція прикрас з латунних гільз, що були зібрані
волонтерами біля Ізюма, від відомого ювеліра та скульптора Володимира
Балибердіна, чиї роботи зберігаються в Музеї історичних коштовностей
Національного історичного музею України. До того ж, на виставці представлено
стіл, що був виготовлений з уламків гелікоптера Алігатор, збитого нашими
військовими біля Ізюма. Серед інших, робота Стаса Кадочнікова – скульптура
Мадонни з уламків ракети Х-31, яка впала у Києві в травні 2022 року та багато
інших цікавих і унікальних експонатів. Колекція включає й меблі з залишків
зброї Івана Стецюка, скульптури з металу Романа Велігурського та скульптури з
металолому Сергія Демченка. Всі роботи можна подивитись на \href{https://wau.art/uk/}{сайті проєкту}.
\footnote{\url{https://wau.art/uk}}

\begin{quote}
\em\enquote{Україна ніколи не буде такою, як раніше, але навіть з понівечених уламків
українці здатні творити прекрасне. Прийде час і ми переможемо, прийде час і
Україна трансформується і стане ще кращою, ніж була. І саме мистецтво має
відображати цю трансформацію, демонструвати світу нову Україну та формувати
новий імідж країни}, – зауважила Світлана Білик, засновниця проєкту  War
Artists Union. 	
\end{quote}

\textbf{Читайте також:} \emph{У Львові відкриється виставка про блокаду Маріуполя - де відвідати}%
\footnote{У Львові відкриється виставка про блокаду Маріуполя - де відвідати, Еліна Прокопчук, mrpl.city, 02.10.2023, \par%
\url{https://mrpl.city/news/view/u-lvovi-vidkrietsya-vistavka-pro-blokadu-mariupolya-de-vidvidati}
}

\ii{23_10_2023.stz.news.ua.mrpl_city.1.mystectvo_z_ulamkiv_vijny_kyiv_vystavka.pic.2}

Завдяки продажу артоб'єктів War Artists Union планує зібрати кошти на допомогу
армії, ДСНС та постраждалим від війни – 15\% від продажу йде на благодійність.
Надаючи митцям незвичайний матеріал для роботи, проєкт стимулює їх творчість та
створює нове українське мистецтво, яке відповідає трендам сучасної України. Цей
проєкт підкреслює значення відновлення екології та перероблення матеріалів, що
зменшує негативний вплив війни на довкілля. WAU виступає як платформа для
збереження пам'яті про війну, героїзм воїнів та наслідки конфліктів через
мистецтво, створення мистецького спадку, який буде відображати культурний,
історичний та емоційний аспекти війни.

Виставка проходитиме в експопросторі БЦ City Zen Park (вул. Сумська, 1, метро
Васильківська) та працюватиме з середи по неділю з 15.00 до 19.00. Триватиме
експозиція на постійній основі. Вхід вільний.

Раніше розповідали, що на початку жовтня в двох українських містах відкрилася
виставка \href{https://mrpl.city/news/view/vistavka-polk-azov-yangoli-mariupolya-vidkrilasya-v-dvoh-mistah-ukraini-de-vidvidati}{%
\enquote{Полк Азов – Янголи Маріуполя}}.%
\footnote{Виставка \enquote{Полк Азов – Янголи Маріуполя} відкрилася в двох містах України – де відвідати, Наталія Сорокіна, %
mrpl.city, 09.10.2023, \par%
\url{https://mrpl.city/news/view/vistavka-polk-azov-yangoli-mariupolya-vidkrilasya-v-dvoh-mistah-ukraini-de-vidvidati}}

Фото: з відкритих джерел
