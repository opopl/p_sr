% vim: keymap=russian-jcukenwin
%%beginhead 
 
%%file 02_12_2021.fb.smolij_andrij.1.kumiry.cmt
%%parent 02_12_2021.fb.smolij_andrij.1.kumiry
 
%%url 
 
%%author_id 
%%date 
 
%%tags 
%%title 
 
%%endhead 
\zzSecCmt

\begin{itemize} % {
\iusr{Oleksii Kliashtornyi}

Це - щось типу дорікань, що \enquote{українці апльодували Сталіну}, Андрій Смолій?..
Вони деструктивні, бо іґнорується природа явища, суть якого у вкладанні
московських навтодолярів у культурну експансію до України.

\begin{itemize} % {
\iusr{Volodymyr Lyvynskyi}
\textbf{Oleksii Kliashtornyi} 

якби ці \enquote{українці} не мали таки москальські смаки та вподобання, то і
ефективність вкладення доларів була б нижчою. Потрібно визнати певна провина за
таку ситуацію лежить на таких недогромалянах, які 41 відсоток адін народ з
Хутіним... Андрій просто гірку правду пише.

\iusr{Андрій Смолій}
\textbf{Oleksii Kliashtornyi} ну багато українців ще часи приходу більшовиків радо плескали в долоні Лєніну.
І що з того вийшло? Голодомори, колективізація, терор, знищення.
Зараз щось схоже просто вже в іншому вимірі.
Долари то йдуть. Але і свої мізки треба мати. Ми не в совку і вибір теж завжди є.

\iusr{Богдан Дмитерчук}
\textbf{Андрій Смолій} 

Моя покійна бабця шоворила так: \enquote{Були поляки, ходили голосракі, еоли прийшли
маскалі - появився хліб на столі} Від себе додам: зібралися до гейропи, знову
будем голожопі.

\iusr{Oleksii Kliashtornyi}
\textbf{Volodymyr Lyvynskyi}, 

це - хибна теза, бо вона створює \enquote{хибне коло}, типу \enquote{Чому бідний? - Бо дурний!}
Який сенс має пошук \enquote{винних}?!. Це схоже на фразу \enquote{в неврожаї винні
довгоносики}.

Ситуація, яку в своєму дописі описує Андрій Смолій, почалася ЗА ЮЩЕНКА, коли
Москва скупила весь FM-радіопростір України. Ми, як держава, відповіли
запровадженням квот на FM-радіостанціях через ДУЖЕ довгий час після цього, коли
ФМ-и вже відійшли на марґінес.

Apple music з'явилася в 2015, коли \enquote{Відсіч} якраз розгорнула в нас боротьбу за
радіоквоти, Spotify дійшов до наших теренів \enquote{колишнього СССР} у минулому
2020-му.

Москва серйозно та системно займається цим. Перша спроба Spotify \enquote{зайти} в РФ
датується 2014 роком, але тоді їх там ТУПО ЗАКРИЛИ. І дозволили працювати ЛИШЕ
коли компанія \enquote{пішла на всі умови} Кремля. Неофіційна частина цих умов фактично
передбача передачу України до \enquote{зони впливу} Москви.

Формально окремі українські \enquote{плей-листи} Spotify – це фактично московські. Ще
раз: «УКРАЇНСЬКІ» плей-листи – це московські плей-листи.

Постановка питання «чому українці не слухають на Spotify українських пісень?»
така ж завідомо абсурдна, як і питання «чому українці не дивляться телесеріялів
українською?»

Відповідь: БО ЇХ ТАМ НЕМА!

На українських олігархічних телеканалах НЕМА \enquote{драм за жізнь}
українською, а на Спотифаї – українських плів-листів.

Самоочевидно, що з московських плей-листів можна вибрати лише наявні у них до
вибору московські пісні.

\enquote{Чому українці не купують в МакДональдзі українських національних
страв?} – \enquote{Ви здивуєтесь, але їх там не продають!}

\iusr{Elena Lytvynenko}
\textbf{Богдан Дмитерчук}
А моя покійна прабабуся Мотря говорила:
Серп і молот- Смерть і голод !

\iusr{Ліна Макарова}
\textbf{Богдан Дмитерчук} Дуже цікаво, а куди ж дівався хліб зі столу 1923, 1932,1933 і т. д? Мабуть Польща з'їла?

\iusr{Еміль Дубров}
\textbf{Богдан Дмитерчук} 

Не знаю, де жила Ваша бабуся, мої батьки жили в західній Україні \enquote{під Польщею},
то там ні в двадцятих ні в 32–33 голоду не було, а коли нашу сім'ю переселили в
південні степи, то тут села пусті стояли, люд від голоду вимер!


\iusr{Еміль Дубров}
\textbf{Андрій Смолій}! 

Біда наша в тому, що нашим олігархам з їх мільярдами, Україна чужа, а держава –
бідна, тож нікому вкладатися в українське національне...

\iusr{Oksana Marets}
\textbf{Богдан Дмитерчук} 

ваша бабця напевно була з тої когорти комуняк, які забирали хліб в господарів
УКРАЇНЦІВ, бо щоб хліб був на столі то треба працювати при любій владі.
Голосракі були ті та і є зараз які чекають що їм дадуть

\iusr{Олег Алоха}

А хіба українські пєвуни співають українською?

Хіба вони не заглядають на мокшостан?

У нас банально окрім декількох виконавців, як от Вінник, ТІК, \enquote{лісапєт}, і
\enquote{кропива}, \enquote{голос Ельзи}, більше немає що слухати. В Україні, ще не народилась
естрада.

А \enquote{гулять, бухать і ваєвать},- це і є привілегія мокшів.

Ми ще сотні років не відійдемо від совка,з такими виконавцями. І будемо співати
\enquote{ой чиє ж то жито}, ще 75 років. Більшого у нас немає.

\iusr{Парасковія Джурбій}
\textbf{Богдан Дмитерчук} 

Ваша покійна бабця напевно була із тих, які в Україні радянську владу
встановлювали, та напевно й поселилася в будинок тих людей, які померли від
голоду. Для них завжди був хліб на столі.

\iusr{Volodymyr Lyvynskyi}
\textbf{Oleksii Kliashtornyi} 

чомусь я всі ці роки обходжуся без того спотіфая та лайносеріалів на ТБ, бо в
мене немає свідомості москаля, а у багатьох є. Той хто прагне знайти серіал або
пісеньку українською - знайде, питання в тому чи є прагнення це зробити. Брак
національної свідомості це факт та велика проблема в Україні, а Андрій
правильно відзначає, що держава яка би мала це виправляти просто бездіє, в
кращому разі. Ти справді сповідуєш принцип, що от в усьому винні політики,
якась вища сила, тільки не бідолашні рядові \enquote{українці} - вони невинні
янголятка, а точніше Овечки та баранчиики), вони не можуть нічого вдіяти?


\iusr{Aleks Kost}
\textbf{Андрій Смолій} 

Погоджуюсь з Вами на 90\%. 10\% - необєктивність необ'єктивність щодо
розвішування ярликів! \enquote{95 квартали}, \enquote{свати}, російськомовні президенти
(ППО)/міністри - не віяння сьогодення..... це наслідки нашої (в тому числі і
вашої) байдужості/бездіяльністі за всі роки незалежності...

\iusr{Ruslan Kozlov}
\textbf{Олег Алоха} 

в Україні є багато сучасної, якісної музики, яка набагато краща за все те
трендове російське лайно.

Назвіть мені хоч один російський гурт, який за рівнем шоу переплюне The
Hardkiss. Або назвіть російські етно-гурти, подібні до ДахаБраха чи Dakh
Daghters, які гастролюють по всьому світу. Той же Бумбокс і досі в Росії
слухають і ніхто там подібної музики за всі роки існування гурту не написав.

В плані хіп-хопу у нас теж все набагато краще. Хто може зрівнятися в плані
текстів та змісту з Паліндромом? Чи хтось робить настільки ж унікальне поєнання
етніки та хіп-хопу як Аліна Паш? Ті ж Альона Альона та Калуш мають свій
унікальний саунд, який ніколи не сплутаєш.

Рухаємося далі - інді. Тут теж повно чудової уінкальної музики - Один в каное,
Vivienne Mort, Христина Соловій, KRUTЬ. Що в Росії є з подібного?

У важкій музиці ми взагалі переганяємо на кілька десятиліть вперед. Stoned
Jesus, Motanka, 1914, Space of Variation, Jinjer (не хотів їх включати у цей
список через їхню ватну позицію, проте це гурт з України). Всі ці гурти
підписані на один з найбільших метал-лейблів Napalm Records і дають навіть
більше концертів за кордоном, ніж в Україні. А що там в Росії слухають? Далі
Арєю, Карольішут, Сєктар Газа і Гражданскую Абарону?)) То що, у нас далі нема
що слухати?))

\iusr{Надія Бачинська}
\textbf{Андрій Смолій} 

пане Андрій, згідна з вами... наші думки збігаються. Ви їх впорядкувати і
виклали. А я ні. Дякую. Тільки що з цим РОБИТИ? Вже виросло нове покоління. І
прагне грошей і доброго життя.. на шлюбі гроші. І їде за кордон. В ту саму
Росію. А хто має нам державу будувати? В душі не може бути пустоти. Якщо там
відсутня національна доктрина там поселяються дурниці. Треба чіткої мети-
будівництво сильної України. Це має бути чіткий план, програма. І має вона
охопити усі сфери життя. Починаючи з роддому і закінчуючи пенсією, книжками,
абетками, підручниками, вивісками, вітринами магазинів, шкільними програмами,
релігіями, газетами, музикою, танцями, піснями, обрядами, одягом, їжею, весілля
ми. І вихованням. Бо таки українці були скромнішими. І в часи війни думали не
про вечірки і тусовки. Має бути якась помірність. А не банкнот на кістках людей
які безвинно загинули. І ніхто не журиться, що поряд пенсіонери живуть
напівголодним життям. А скільки безробітних. Як їм далі жити? А поряд працюють
ті, які мають набагато меншу компетентність.

\iusr{Наталия Соловьёва}
\textbf{Олег Алоха} 

Ну це у Вас немає, а в мене є - "Без обмежень", "Антитіла", "Друга ріка"... та
багато інших талановитих гуртів та виконавців, на жаль їх не
популяризують... шкода.

\iusr{Євгенія Романенко}
\textbf{Наталия Соловьёва} ну чого, антитіла доволі радісно популяризуються на виступах у кварталі

\iusr{Наталья Брюх}
\textbf{Олег Алоха} Kozak System, Тінь Сонця, Скай, БЕZ ОБМЕЖЕЬ...
Слухайте на здоров'я

\iusr{Lina Kacmar}
\textbf{Олег Алоха} розуму у вас не має, а пісні є!
\url{https://youtu.be/LOgZJPTz9dU}

\iusr{Марія Корчинська}
\textbf{Oleksii Kliashtornyi} .. Смолій за то й пише!! @igg{fbicon.index.pointing.up}

Вкладаються велики гроши в впровадження рос.культури в українські голови..

\iusr{Mirek Photos}
\textbf{Олег Алоха} жах, як примітивно

\iusr{Павло Гай-Нижник}
\textbf{Андрій Смолій} брехня не плескали лєніну.

\iusr{Юрій Арнаутов}
\textbf{Богдан Дмитерчук} ти достойний внучок своєї тупої бабці! @igg{fbicon.face.tears.of.joy} 

\iusr{Babka Yozhka}
\textbf{Андрій Смолій} , 

замало вибору, на жаль... Молодь захоплюється репом, а в нас майже немає
«розкручених» українських реперів... На жаль, перемагає та культура, в «розкрутку»
якої вкладають великі гроші. І як можна заборонити Інтернет?

\iusr{Jaroslav Datzik}
\textbf{Oleksii Kliashtornyi} 

Коли, ти вживаєш слово \enquote{типу}, означає, що ти маєш освіту вісім класів. Це я пишу не тобі, а тим хто тебе читає.


\iusr{Олег Ковалишин}
\textbf{Олег Алоха} кконтужений???

\iusr{Олег Алоха}
Насолоджуйтесь.
А я буду слухати те, що мені подобається.

\iusr{Андрій Василюк}
\textbf{Ruslan Kozlov} 

це все так, але трохи не так... цей ринок працює по іншому. Я б провів паралелі
між західним ринком і російським. звичайно західна музика
цікавіша, професійніша. Російська поп музика завжди йшла за копією західної. І це
мабуть нормально, так було завжди, з з часів Моцарта. Тому російська попса, це не
тільки моргенштерн, це ціла ідеологія, це, якщо хочете машина, яка працює на
ідеологію. крім того, вона жанрово різна і тому конкурентно здатна. А тих наших
виконавців, яких ви назвали, часто навіть музикантами не є. Яка музична освіта в
Альони, чи Калуш.? Чи може нове ім'я попасти в ротацію на радіо? І т п.

\iusr{Ruslan Kozlov}
\textbf{Андрій Василюк} 

так і у Моргенштерна також немає музичної освіти. У світі поп-музики її
наявність не грає жодної ролі. Ніхто з учасників Бітлз також не мав музичної
освіти, що не завадило їм стати найвідомішим гуртом світу.

У нас також є багато жанрово різної музики. Єдина проблема в тому, що у нас
більше розкручують в інформпросторі російську та російськомовну музику, а не
українську. І питання не в якості, а в комплексі меншовартості, який нам
нав'язували ще з радянських часів - типу, усе найкраще в Росії, всі діячі
мистецтва та шоу-бізнесу переїжджають у Москву і т.д. І в цьому причина, що ми
встидаємося свого, вважаючи його гіршим за російське. Хоча в тому, що російська
музика завжди була блідою копіпастою західної музики, ви праві. Я скажу навіть
більше, російська музика була ще й копією української. Тільки дуже поганою
копією

\iusr{Андрій Василюк}
\textbf{Ruslan Kozlov} 

та звичайно справа не в освіті. Тай хрін зним з моргенштерном. а назвіть мені
аналог Бітлз у нашій поп музиці, чи Б. Адамса, чи Рода Стюарта, чи десятки
інших, які збирають стадіони. Я працюю у цій сфері більше 20 років і знаю цю
проблему мабуть на всі 100. Проблема не в популяризації російського, а в тому,
що в цю сферу зайшло багато не профі. не з цієї галузі. А приходиться конкурувати
з профі. Коли ви бачили концерт по загально національних каналах Т Петриненка.?
,Чи О Тищенка.? Тому про багатожанровість, я б подумав. Наші, ще за совка їхали
в Москву, бо там було більше свобод і можливостей. а ми, на жаль застрягли в
,,,полонинах, ріках, калинах, малинах,,, про рок музику, я просто мовчу, бо це
якраз стосовно освіти, необов'язково консерваторської.

\iusr{Ruslan Kozlov}
\textbf{Андрій Василюк} 

дивіться, ті імена, які ви назвали це академічна музика, яка потребує вищої
освіти. Ми ж говоримо про шоу-бізнес. Це зовсім інший світ.

Щодо аналогів я відповідав вище і повторюся. Безглуздо шукати у нас аналоги
музики Браяна Адамса чи Рода Стюарда, Бітлз, Лед Зеппелін чи Кінг Крімзон.
Західна музика не створювалась в умовах тоталітарного режиму. У той час, коли в
80-х гриміли Металліка, Пантера і Слеєр, нас годували Аллою Пугачьовою та
Едітою П'єхою з Кобзоном. І намагаються годувати досі, адже, якщо ви увімкнете
російські телеканали, то там всі ті ж самі \enquote{пєсні года} і
\enquote{новогодніє огонькі}.  Нам потрібно вибудовувати власну сцену, власне
звучання і власний стиль, а не копіювати західні аналоги. А тим більше
російські.

\iusr{Андрій Василюк}
\textbf{Ruslan Kozlov} 

але ж цю музику придумали не ми. просто треба грати, як на заході, а моментами і
краще. А на рахунок академізму.... 2019 рік, день незалежності, м Калуш ,Західна
Україна .Великий концерт за участю,, цвіт кульбаби,, ,,полюс,,. Т Петриненко з
легендарною піснею Україна. і як не дивно завершує програму ,,цвіт кульбаби,,. З
піснею,, всі баби, як баби, а ти королева,, от до чого призводить відсутність
професійності, чи хочете освіти. що дає українському шоу бізу, такі виконавці і
пісні? А їх більшість серед вище згаданих, або тотальні труси, або примітиви. я
дуже хотів би, щоб ми мали абсолютно свій шоубіз. але це справа профі і фанатично
відданих цій справі людей. на жаль ми зараз програємо і то сильно.


\iusr{Ruslan Kozlov}
\textbf{Андрій Василюк} 

типовий споживач (наголошую саме на слові споживач, а не слухач) завжди буде
слухати щось просте і примітивне. І так є на Заході. Умовно кажучи той же Род
Стюард також програє в плані популярності Джастіну Біберу чи Аріані Гранде.
Аналогічно і у нас. Візьміть, до прикладу, того ж Михайла Хому. Які пісні у
нього популярніші? Дурнуваті, які він співає в образі бородатого Дзідзя, чи
класичні естрадні пісні у його власному виконанні? І так завжди буде.
Академічна чи класична естрадна музика ніколи не буде масовою. Вона має свою
аудиторію і з цим доведеться змиритися. Щоби робити популярний масовий продукт,
потрібно дивитися на світові тренди в музиці і працювати в цьому напрямку.
Проте це вже тема окремої дискусії)

\iusr{Андрій Василюк}
\textbf{Ruslan Kozlov} 

не погоджуюся... категорично. бо така логіка і призвела до тотального обидлення
.дурнуваті пісні ніколи не сформують національно свідоме суспільство. Вони і
формують \enquote{какая разніца}, Я ДЕСЬ ЗГОДЕН з приводу світових трендів. Але в
Британії кожен горобець знає хто такийПол Маккартні і Бібер тихо розмовляє в
його присутності. Гляньте на план концертів Рода Стюарта, якому під 80. Оце
просте і примітивне в нас надто просте і надто примітивне. я думаю це комусь
дійсно треба, щоб українське асоціювалося з простим і дурненьким.. цікаво, то
хто в цьому зацікавлений?


\iusr{Ruslan Kozlov}
\textbf{Андрій Василюк} 

я здається зрозумів в чому проблема. Справа в тому, що примітивна музика буде
завжди існувати і мати свою стабільну аудиторію. Вона навряд пересічеться з
аудиторією класичного прогресів року, наприклад. Навіть той приклад, що ви
навели. Аудиторія Тараса Петриненка явно не та, яка слухає пісню \enquote{всі баби як
баби}. Це явно дивне поєднання на одному концерті і свідчить про
непрофесіоналізм його організаторів. Адже навіть на Заході фанати Джастіна
Бібера навряд слухають Пола Макартні. Чули мабуть, але для них це якийсь
дядько, якого слухали їхні бабусі та дідусі в молодості.

Справа в тому, що потрібно розвивати сцену у межах свого напрямку серед своєї
аудиторії. Молодь не буде слухати Тараса Петриненка, але якщо ви хочете
виховати серед молодих людей національно свідоме суспільство, ви маєте їм
запропонувати сучасного молодого артиста, який і буде просувати українську
культуру у молодіжному популярному стилі. І це не обов'язково має бути
примітивна музика. Яскравий приклад ті ж учасники Євробачення Go A. Їхня пісня
має неабияку популярність серед молоді і до того ж це не тупа примітивна попса,
а яскравий приклад того як українська етніка може звучати водночас модерново і
стильно

\end{itemize} % }

\iusr{Лілія Григорович}

Потрібно щось на кшталт «Теритоіі А» в ребрендингу.

Цікаво, який канал відмовиться від якоїсь «маски» на користь такого проекту?

На разі маємо 3-4 проукраїнські телеканали, які б змогли... Інакше,
моргенштерни не відлипнуть.

\begin{itemize} % {
\iusr{Андрій Смолій}
\textbf{Лілія Григорович} 

питання в тому, що молодь не дивиться телевізор. Це минуле.

Держава повинна попіклуватись, щоб Ютюб, Еппл Мюзік, Ютюб мюзік та інші сервіси
оокально готували списки музики тут в Україні. І щоб в рейтингу «рекомендоване»
теж була українська, а не російська.

У нас є ціле «міністерство цифрової транформації», Мінкульт. Це їхнє завдання
комунікувати, їхати, вирішувати ці питання.

Але хіба їх це цікавить, якщо Зелені заробляють гроші на росії?

Без проукраїнської влади і реальної стратегії - не буде нічого. На жаль.


\iusr{Юра Грицик}
\textbf{Андрій Смолій} треба розповідати дітям, що це все дно і чому. Держава наврядчи тут допоможе.

\iusr{Леся Бандурович}
\textbf{Лілія Григорович}, пізно. Треба починати з укр. мультиків.

\iusr{Алла Бойко}

Як каже мій брат: - я б послухав українське, тільки де воно.

Раша і Китай - будь ласка в радіопріймачі. Телек в селі багато дивитись нема
коли

\iusr{Микола Миколайович}
\textbf{Андрій Смолій}

Згідний з Вами, починати треба з української влади. Але.... . Як вкласти в
голови українцям, що треба обирати патріотів-українців.

\iusr{Олександр Якименко}
\textbf{Юрій Грицик} 

розповіді теж не допоможуть.

Якщо Україна нездатна продукувати конкурентоспроможного контенту для дітей та
молоді ( а це тільки той, який молодь обиратиме абсолютно добровільно без всяких
\enquote{рекомендацій} ) - то вважайте що наша молодь вже втрачена.

\iusr{Олександр Якименко}
\textbf{Алла Бойко} 

по деяких розділах сучасного українського, цікавішого ніж російське для
основної маси - просто нема. Як такого.

\iusr{Степан Гавриляк}
\textbf{Лілія Григорович}. Хто при владі? Членограй рояльний.
Ота т.зв. \enquote{мододь} і вибрала його, якій до вподоби \enquote{Мордоштерни}.
Інформаційна війна працює проти нас.

\iusr{Степан Гавриляк}
\textbf{Лілія Григорович}. Потрібна масована інформаційна атака проти Раші, а як її зробити, то є спецтехнології.

\iusr{Euhenia Pysmenna}
\textbf{Андрій Смолій} як гарно що вони не дивляться тв. Чудово

\iusr{Леся Бандурович}
\textbf{Юрій Грицик}, батьки не мають часу розповідати - або ФБ або робота.

\iusr{Юра Грицик}
\textbf{Леся Бандурович} не тільки батьки, школа, старші товариші, більш свідомі люди навколо, той же ютуб з проукраїнським контентом. Я маю декілька прикладів саме ютуб блогерів.

\iusr{Олександр Буєвич}
\textbf{Микола Миколайович} Відбивши телевізор в купки багатіїв
\end{itemize} % }

\iusr{Роман Кузнєцов}

Я вже нещодавно про це писав. Сумно все це..

\url{https://m.facebook.com/story.php?story_fbid=4713105258748693&id=100001477956740}

\begin{itemize} % {
\iusr{Андрій Смолій}
\textbf{Роман Кузнєцов} це дійсно дніщє...
Але що робити? Ми ж не взмозі ідіотам мізки вправити. Це могла б хіба держава..

\iusr{Роман Кузнєцов}
\textbf{Андрій Смолій}  @igg{fbicon.100.percent} 

\iusr{Наташа Заболоцкая}
\textbf{Андрій Смолій} Держава? це як у 30 роках Сталi нiзму?

\iusr{Семен Гавура}
\textbf{Андрій Смолій} .Починається все \enquote{з пелюшок}. Потрібна реформа школи. І чим скоріше. В же втрачено 30 років.

\iusr{Нина Орлова}
\textbf{Семен Гавура} , 

потрібно починати з громадянства. В інших державах громадянином стає людина,
коли здає екзамен по мові та історії. В нас і тут 'какая разніца'. В серці
України, в Києві - 80\% лунає язік. Весь спорт російськомовний! Мені теж
огидно...

\iusr{Семен Гавура}
\textbf{Nina Orlova} .

Повністю підтримую вашу думку. Але наша розхристаність не дозволила нам це
зробити тоді, коли це зробили країни Балтії. А зараз ми накопичили стільки
невирішених проблем, що потрібно садити за кермо державою мабуть самого бога, щоб
їх розрулив. Бо якщо суспільство буде викидати з свого життя по 5 років і
далі, то подальша доля його незавидна. Але потрібно бути оптимістами і
об*єднаними.

\iusr{Irina Michlewicz}
\textbf{Роман Кузнєцов} і писали ви російською... @igg{fbicon.face.confused} 

\iusr{Роман Кузнєцов}
\textbf{Irina Michlewicz}, и что теперь? Я в Украине не имею права писать на иностранном языке?

\iusr{Irina Michlewicz}
\textbf{Роман Кузнєцов} ну так і молодь скаже, ми що, не маємо права слухати пісні на іностранном язике?

\iusr{Роман Кузнєцов}
\textbf{Irina Michlewicz}, а в моем посте идет речь о песнях на иностранном языке? Или вы по принципу \enquote{не читал, но осуждаю}?
\end{itemize} % }

\iusr{Владимир Воробей}
\textbf{Андрій Смолій} 

хто винен що в нас одне і те саме. Частушки етно якесь і брєд завиваючій...,
конопель-конопель Конопелєчка ))). WTF??? От не буде мододь цього слухати і
все. Вангую. Це музика... Вона або є модною або нею не є. Без мов. Без кордонів.

\begin{itemize} % {
\iusr{Андрій Смолій}
\textbf{Vladimir Bird} пфффф, ви мабуть не чули української музики, якщо таке говорите.
Я слухаю україномовну, англомовну, іспаномовну музику. Серед україномовної жодні етно не слухаю і жоден «брєд» теж.
Брєд це московський непотріб, який нам вкладають у вуха.

\iusr{Владимир Воробей}
\textbf{Андрій Смолій} чув на Євро баченні 2021. І не зрозумів чому на весть світ виставили ОЦЕ.

\iusr{Владимир Воробей}
\textbf{Андрій Смолій} 

\url{https://youtu.be/lqvzDkgok_g}

\iusr{Владимир Воробей}
\textbf{Андрій Смолій} ну це ж хуторянщина якась. мавпування з «Бурановськіх бабушек»

\iusr{Виталий Беланов}
\textbf{Vladimir Bird} 

не буде хліба, будете лайно їсти? Мені теж, мало що подобається з українського,
але ж, кацапс... кої ахінеї, консаної, дешевої, примітивної музики я не слухаю.
В мене є вибір: AC/DC, Metallica ..Wanda Jacson, Jony Cash... ABBA,
BONI-M... Sash, Darude, B Plan, dj Alligator.... одним словом, живемо не у 80 их,
вибір шикарний, доступний! Так що, не треба заганяти пиз... на! Так і скажіть,
що Вам це дешеве, примітивне ЛАЙНО подобається!


\iusr{Леся Полиш}
\textbf{Владимир Воробей} , тому що ви не вивчали й не знаєтеся на фольклорі! @igg{fbicon.shrug} 

\iusr{Владимир Воробей}
\textbf{Виталий Беланов} 

лайна не їм. І навіть не згадую як ви у кожному реченні. Слухаю. Молюсь. І
роблю інші речі в житті без націоналістичних гасел і @ба@утих порад. Так
виховали.

\iusr{Владимир Воробей}
\textbf{Леся Полиш} і знатись не хочу. Це регрес. Давайте на конях в шапках петлюровських їздити будемо по вулицях

\iusr{Леся Полиш}
\textbf{Владимир Воробей} , 

не знаючи історії, не пам'ятаючи свого коріння, у нас не буде майбутнього... От
саме через це й танцюємо на граблях з революції 1917 року, бо витруїли совіти
справжню історію й пам'ять в українців...

\iusr{Владимир Воробей}
\textbf{Леся Полиш} ну да. ну да. саме тому

\iusr{Владимир Воробей}
\textbf{Леся Полиш} 

а ще тому що кожен дивиться на свою халабуду, і свій садочок коло хати. А на
вулиці, чи на східцях грязі по коліна., і всім байдуже...

\iusr{Валентина Бугаєнко}
\textbf{Владимир Воробей} 

Якби ви були Украінцем, пане Горобець, та жили б і м. Рівне, та спілкувались
державною украінською мовою, то може б і слухали б украінську музику. А
хуторянщина, то до Соні Ротару, вам з нею по дорозі в ту сторону, до рашки.

\iusr{Леся Полиш}
\textbf{Владимир Воробей}, 

\enquote{Якби ми вчились так, як треба, то й мудрість би була своя}...

\enquote{Учітесь, читайте! І чужому научайтесь й свого не цурайтесь!}...

\enquote{Всі наші біди через невігластво!}

\iusr{Владимир Воробей}
\textbf{Леся Полиш} найбідніше і найбезперспективніше населення Європи.

\iusr{Владимир Воробей}
\textbf{Валентина Бугаєнко} пані Худобенко. вчіть свого діда рибу ловити @igg{fbicon.index.pointing.up}

\iusr{Владимир Воробей}
\textbf{Валентина Бугаєнко} і вам і той проміжок

\iusr{Yevhen Khomiak}

Повністю згодний про \enquote{московський непотріб}. Української музики на будь-який
смак вистачає (підкреслюю - на будь-який смак, від камерної та класичної до
панка!), от хоча би свіжий приклад сьогодні:

\url{https://www.facebook.com/621354521632435/posts/1323369234764290/}

Треба себе дуже не поважати, щоб слухати абсолютно вторинний московський непотріб

\iusr{Valentina Dovgan}
\textbf{Андрій Смолій} 100\%

\iusr{Lana Rössner}
\textbf{Vladimir Bird},

не згодна з Вами. Мій чоловік із Італії і він із задоволенням слухає українські
пісні і каже, що наші українські пісні також чудові, як і пісні Італії.

\iusr{Хелен Велен}
\textbf{Vladimir Bird} 

Це все слуха бидло-молодь по великому рахунку. Інша молодь також є. Наприклад у
Дніпрі російська мова - це мова бидла потроху стає. Це процес невідворотній вже.
І московітське лайно нічого не зробить. Це тимчасово,хоча і неприємно
безумовно.

\iusr{Prokhor Ruban}
\textbf{Vladimir Bird} 

ну якщо що, тримайте трохи: THE HARDKISS Vyacheslav Drofa ONUKA Alina Pash
Jamala | Джамала Koloah KAZKA MÉLOVIN TVORCHI Pianoboy高至豪, і ще про всяк
випадок — проєкт «Спалах» з випуском про нову українську музику на Ютубі, де
розуміють цінність «етно та брєду завиваючого». Може допоможе дивитися трохи
далі того, що пропонує вам телебачення та радіо.

\iusr{Валера Шевченко}
\textbf{Андрій Смолій} згоден!

\iusr{Adrijan Gnatjuk}
\textbf{Vladimir Bird} 

а у мордора класна музика? Щось не видно, щоб її признавали і слухали в світі.
Чи може вони щорічно Греммі виграють. Не вангуйте, бо можна в халепу вскочити.

\iusr{Igor Puzhak}
а чому Владімір?

\iusr{Adrijan Gnatjuk}
\textbf{Igor Puzhak} тому що Володимир, то по українському. А воно йому треба?

\iusr{Олег Ровінець}
\textbf{Helen Velen} ооо, гарні новини із Дніпра! Дякую!

\iusr{Eugenia Stanis}
\textbf{Vladimir Bird} не повірите, але «частушки етно і брєд» молодь ще й як слухає.

\iusr{Zyma Vasyl}
\textbf{Vladimir Bird} 

до речі те, що ви називаєте брєдом, тобто, конопелочка, були в топі по
прослуховуванню досить довгий час в Європі. І європейський глядач дав
конопелочці найвищі бали. А ви слухайте далі \enquote{не брєд}...


\iusr{Володимир Фесюк}
\textbf{Андрій Смолій} Цікаво а що ви слухаєте з української музики, і до чого тут англомовна і іспаномовна ?

\iusr{Igor Kiianchuk}
\textbf{Vladimir Bird} конопелечки - просто афігенна музика. Щоправда, для підготовленого слухача. Тому й зайшла в Європі на ура.

\iusr{Kote Goroshko}
\textbf{Vladimir Bird} 

Звісно, ви лайно не істе, бо вже нікуди, воно лізе з вас з кожноі дірки. Ви є
класичний зразок малороса-манкурта \enquote{какаяразніца, ліш би фанарі гарєлі}. Вам
аби корито, з якого хлєбати, і все буде чудово. Через таке бидло, як ви,
Украіна і не може відбутися як держава

\iusr{Yaroslav Isay}
\textbf{Vladimir Bird} яка музика у фріка мангєрштєрна? @igg{fbicon.grin}  @igg{fbicon.face.smirking} 

\iusr{Валентина Тимощук}
\textbf{Vladimir Bird} 

До речі, \enquote{конопелечки} це веснянка це етнос України це фольклор. Так це
джерело нашої культури. Є у нас гарна сучасна музика, але за руською попсою не
може пробитись. Виховувати треба молодь на українському, а не втюхувати всякий
непотріб. Риба воняє з голови.


\iusr{Олег Ортинський}
\textbf{Vladimir Bird} та конопелечка то маразм...

\iusr{Елена Сыс}
\textbf{Владимир Воробей} 100\%

\iusr{Ludmyla Ukranienne}
\textbf{Vladimir Bird} Ну, так. Москальське лайно краще.

\iusr{Ilya Girzha}
\textbf{Владимир Воробей} 

вы слушали Даху Браху, О, Джамалу? Это просто как пример украинской музыки,
интересной и высочайшего качества. Я понимаю молодежь (школьники/студенты) -
там обычно кисель в голове и действительно слушают то что в топах чартов. Но вы
вроде спелый... С молодежью (детьми) нужно говорить и рассказывать о таких
вещах как самоидентификация, ну, и конечно, о истории... А так же почему в
Украине многие говорят по-русски.

\iusr{Артемко Шифанерко}
\textbf{Vladimir Bird} the doox наприклад...

\iusr{Vera Bobb}
\textbf{Владимир Воробей} 

ти глухий, нічого не чуєш ..Є Вакарчук, є Друга ріка, єПивоваров, є
Тополя, Барських, Дорн. Що ти несеш, любитель балалайки.

\iusr{Alex Wagis}
\textbf{Владимир Воробей} Антитіла, Широкий Лан, Точка Опори, Шабля, Тінь Сонця. Це так, навскидку... з упором на рок і фолк.

\iusr{Евгений Незенко}
\textbf{Vladimir Bird} вангую. Если подключи мозги и разшириш свой кругозор, ты будеш приятно удивлён, сколько у нас прекрасных исполнителей и музыки.

\iusr{Tetiana Galinska}
\textbf{Vladimir Bird} не \enquote{в ту стєпь ви вангуєте}. Треба цікавитись тим, про що пишете.

\iusr{Олександр Каленський}

Нажаль багато часу втрачено, треба було з 1991року робити українську Україну, як
колись казали: - \enquote{Свій до свого, по своє}. Але обрали спершу гниду
комуняцьку-Кравчука, потім Кучму, які плазували перед московітами, а
проукраїнських Ющенко та Порошенко зацькували, а наразі і зовсім обрали
впровадженого агента Московії, ось і маємо, те що маємо.


\iusr{Vasil Martynuk}
\textbf{Vladimir Bird} 

питаєш хто винен, сядь будь ласка та заспокойся і добре подумай, а що я роблю
про що думаю про родину про близьких, чим я, вони живуть, як живемо чому так
,кого я поважаю, як до мене  @igg{fbicon.thumb.up.yellow}  відносяться і я до них чому так мало заробляю,
і яка зарплата у інших країнах, як вони живуть, а чому так, так багато
запитань, і як на це все дати відповідь, а з хто у цьому винен, і що я, ми
усі щось не так робимо ч ому влада така погана і хто її нам подарував таку
неспроможну владу які дурні нам її підсунули і які вороги нас хочуть знищити
невже ми самі себе хочемо знищити це ж не правда ми ж усі хороші а де ті кляті
вороги невже ми самі себе знищуємо, як так ми усі хороші та дуже мудрі і дуже
, дуже хитрі, наша хата скраю, як щось не так то не ми а коли піде на краще,
тоді перший прибіжу та битимусь у груди і кричатиму це все я, я перший

\iusr{Вуйко Місь}
\textbf{Vladimir Bird} ти, тямиш в музиці, як вовк у звіздах. @igg{fbicon.face.tears.of.joy}  Частушки, невігласе лаптєногі виють.

\iusr{Зінаїда Вавілова Статник}
\textbf{Vladimir Bird} бо молодь диградована тупоголова, часткшек в нас немає а конопелтки молодь європейська слухає

\iusr{Зінаїда Вавілова Статник}
\textbf{Андрій Смолій} уявляєте яка в нас молодь і взагалі нарід деградований

\iusr{Євген Бердін}
\textbf{Vladimir Bird} 

повністю з вами згоден, що музика без мов, без кордонів. І якщо ви про ті
конопелечки, що всьому світу зайшли, то цікаво які ж у вас смаки. Вангую, якщо
не моргенштерн, то алєгрова з кіркоровим

\iusr{Янина Родзевич}
\textbf{Владимир Воробей} , 

конопелечки були на вершинах хіт-парадів всього світу! Цію піснею захоплювались
переможці Євробачення італійці. Вони, доречі, молодь, чи не так? І потім, пан
Смолій не дискутує щодо музичних жанрів. Не подобається етно-до вашої уваги
безліч інших жанрів УКРАЇНСЬКОЇ музики. Чудовий джаз, доречі.


\iusr{Тетяна Поліщук}
\textbf{Ilya Girzha} 

не знаю, що там слухають діти, а у мого сина в плейлист потрапили GO-A, Kalush
і Петро Сказків )


\iusr{Nadya Goroshochek}
\textbf{Євген Хом'як} 

ви про джаз говорите, а треба про попкультуру, щоб з кожної праски лунало, в
транспорті, таксі, на вулицях... тоді буде результат. Щоб відбулась
популяризація, продукт має бути абсолютно доступним, а не вишуковувати на
ютьюбі щось цікаве годинами ....

\iusr{Sergii Chepurnyi}
\textbf{Андрій Смолій} 

Андрій, поділіться назвами якісних українських рок груп? Бо мені щось не
щастить - не бачу цікавих і талановитих. Або безголосі, або етно, або нудота.

ОЕ задовбав вже.

Металу нормального нема - все чомусь з етно присмаком.

Може я не там шукаю?

\iusr{Bohdan Kondrusik}
\textbf{Владимир Воробей} 

не дякуйте:

\url{https://www.youtube.com/watch?v=hQt7T4J0iQw}

було б бажання шукати. Але держава відстає в плані, що не пропагує та заохочує
до українського.


\end{itemize} % }


\end{itemize} % }
