%%beginhead 
 
%%file 22_06_2022.fb.rogozhyna_viktoria.kyiv.1.nad_ya_sukhorukova__
%%parent 22_06_2022
 
%%url https://www.facebook.com/vrogozhina/posts/pfbid021QTWmK7tX51nduy6CBEFoGUEvU5fWjnJvkN8KrKUJpvz4fQd7BKRSFvmxaouftNWl
 
%%author_id rogozhyna_viktoria.kyiv
%%date 22_06_2022
 
%%tags mariupol
%%title Надія Сухорукова - Все продолжается
 
%%endhead 

\subsection{Надія Сухорукова - Все продолжается}
\label{sec:22_06_2022.fb.rogozhyna_viktoria.kyiv.1.nad_ya_sukhorukova__}

\Purl{https://www.facebook.com/vrogozhina/posts/pfbid021QTWmK7tX51nduy6CBEFoGUEvU5fWjnJvkN8KrKUJpvz4fQd7BKRSFvmxaouftNWl}
\ifcmt
 author_begin
   author_id rogozhyna_viktoria.kyiv
 author_end
\fi

Автор наша Надя Надія Сухорукова, которую фейсбук в очередной раз забанил на
месяц. Но мир должен знать правду о трагедии Мариуполя. Трагедии всех
украинцев, которых безжалостно, бесчеловечно каждый день убивает фашистская
россия. Поэтому расшаривем Надины публикации, помогаем правде пробиться сквозь
тьму и запреты.

\#мариуполь \#надежда Все продолжается. В Николаеве взрывы, дома Северодонецка
стали близнецами разрушенных многоэтажек Мариуполя. Нас убивают   по одному и
дюжинами.  Даже если погибаем вместе, то все равно,  каждый общается со
смертью один на один. 

Никто не знает, что будет завтра. Меня забанили на месяц.  Это не страшно. Это
же не бомбы. Можно пережить. Через месяц читать этот пост будет ещё интереснее.
Я думаю продажный Фейсбук через месяц исчезнет. Его покинут все адекватные
люди. Лично я перейду в телеграмм. Интересно, как мой пост будет звучать через
месяц, после победы? Наверное, как привет из тяжёлого прошлого. 

Только что семь ракет прилетело в Николаев.  Вот это страшно.  Рашисты убивают
мирных людей. Направляют смерть  по спальным районам. Бьют по тем, у кого нет
оружия, кто не может ответить. Иногда отчаяние запускает сопротивление. Я уже
готова сопротивляться убийцам. Думаю, многие люди тоже к этому готовы. Увидим
через месяц.  Или раньше? 

Моя семилетняя племянница Варя спросила  в мариупольском  подвале: "Если нас
всех убьёт одна бомба, мы полетим к боженьке вместе? Будем держаться за ручки?"

Ещё одна маленькая девочка, которую  привела Лена в Ноев ковчег, выясняла в
какую игру я хочу играть. Я ни в какую не хотела, но пришлось, потому что
малышка была грустная, у нее  только что разбомбили дом и она вместе с мамой
бежала под обстрелами. 

Мама ее тянула за руку, она за ней не успевала, мама нервничала и кричала.
Малышка думала: мама за что-то злится на нее, а мама злилась на войну. 

У девочки  какое-то необычное и красивое имя. Я почему-то  не могла его
запомнить. Все время переспрашивала и опять забывала.  Большеглазая и кудрявая
малышка  очень походила на ангела и я про себя назвала ее Ангелиной. 

\enquote{Моя мама тоже боится}, - призналась мне шёпотом Ангелина. Для нее это стало
открытием. Мама сидела в общей комнате, качалась из стороны в сторону  и ждала,
когда Наташа найдет ее дочери хоть какую-то  теплую одежду. 

Когда вынесли курточку, штанишки и свитер, все это  тут же стали надевать на
ребенка поверх прежней одежды. Мама Ангелины отвечала на какие-то наши  вопросы
и вертела дочь как куклу. Малышка не сопротивлялась. Мне почему-то было ее
очень жаль. 

За час до прихода в ковчег  над ними летал самолёт.  Бомбы скидывал где-то
поблизости  каждые  десять минут.  Взрывная волна била по их  дому. Стены
подвала плавали и вибрировали.

Потом часть  пятиэтажки  рухнула. Их подъезд устоял. Для меня это было
открытием. Я не знала, что можно выжить после авиаудара. Мама Ангелины одевала
дочку и все время вздрагивала и  присаживалась на корточки, когда слышала
прилеты. Мы говорили \enquote{Это не к нам, это далеко}. И сами себе не верили. 

Две напуганные девочки - маленькая и взрослая  рассказали, что когда
выбрались, бежали непонятно куда. Впереди видели чьи-то спины. Ориентировались
по ним. Потом спины исчезли. Это была катастрофа. Но на пути в никуда
встретилась Лена. 

У малышки дома остались игрушки и карандаши.  Когда сильно стреляли она
рисовала принцесс. Рисунок освещался фитильком из ваты и масла.  На него
откуда-то дул ветер. Из-за этого казалось, что принцессы на  рисунке   дрожат
от страха. 

\enquote{Принцессы  умерли. Их убила  бомба. Им не больно}. Малышка ещё  не понимала,
что смерть - это навсегда. 

Мы играли в "шла собака по роялю. Единственная игра, которую я знаю с детства.
Это когда говоришь цифры и,  шутя, ударяешь ладошкой по ладошке. Ангелина
выигрывала. Она хорошо считала и вовремя убирала ладошку. 

Ангелина мама -  подруга Лены. Лена вела их дальше,  к родственникам. В Ноевом
ковчеге больше не было мест. Они зашли сюда, чтобы пересидеть жёсткий обстрел. 

30 человек, собаки и кошки ютились на первом этаже, под лестницей, в подвале.
Ковчег превращался в теремок. Это чувствовали все. 

Девочка с мамой шли на Новоселовку. Там тоже сильно стреляли. Авианалеты не
прекращались. Мы не знали чей дом будет следующий.  Такая рашистская рулетка.
Попадет или пролетит мимо?

Маленькая большеглазая девочка не хотела уходить. Она схватила моя руку.  Я
здесь ничего не решала. Впервые Наташа не предложила им  остаться. Наверное,
что-то чувствовала. Через день в крышу дома случилось прямое попадание. 

\enquote{Как ты думаешь, лето скоро придет?} Тогда я думала, что лета у нас не будет
совсем. Но я жива. Надеюсь, лето  пришло к Ангелине и ее маме. Мне ничего
неизвестно о судьбе девочки. 

Этой весной я сказала малышке \enquote{Мы скоро увидимся}. Ангелина не поверила. Я не
хотела её обманывать. До этого дня  я ее больше  не видела, но мы обязательно
увидимся. Я же обещала. 

Малявка держалась за мою руку, а с другой стороны ее тянула мама. Немного
затихло.  Им нужно было срочно  бежать. 

Стойкая Лена сидела под столом и плакала.  Я впервые видела ее слезы. 

\enquote{Я больше не могу. У меня  нет сил}. 

Самолёт снова гудел. В  нашем доме  не было  ПВО. К сожалению. Мы стреляли в
летающий  катафалк -   словами. Но наши слова не спасали от бомбёжек. 

На следующей день Лена сказала, что родственник мамы и дочки посадил их в
погреб и не выпускает. Тот  район очень сильно бомбили и он пытался спасти им
жизнь. Малышка плакала и просилась выйти. В погребе  темно и страшно. Наверху
был ад.
