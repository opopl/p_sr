% vim: keymap=russian-jcukenwin
%%beginhead 
 
%%file 21_07_2020.fb.lnr.1
%%parent 21_07_2020
 
%%endhead 
  
\subsection{СТРАНА, КОТОРУЮ МЫ ПОТЕРЯЛИ}
\url{https://www.facebook.com/groups/LNRGUMO/permalink/2864592280318994/}

\index{Авторы! Сергей Мальцев}

Просто до боли ностальгический и очень сильный очерк, под каждой строчкой готов
подписаться!

На одной из улиц, как-то остановили меня два мальчика и неожиданно спросили:
«Дядя, а вы жили в Советском Союзе? Расскажите, как вам жилось? А, то нам в
школе говорят, что тогда было очень плохо, а знакомые взрослые говорят, что
жилось хорошо. И мы не знаем, кому верить?».

Я предложил ребятам присесть на скамейку и часа два рассказывал о Советском
Союзе. О заводах и фабриках, которые стабильно работали, о колхозных полях, где
всегда колосилась пшеница и рожь. О бесплатных школах и институтах, бесплатных
больницах и квартирах.  О пионерских лагерях, где весело и полезно для здоровья
проводили время школьники. Рассказал и о бесплатных спортивных секциях, которые
работали в каждом районе.  Затаив дыхание, слушали меня ребята.

«Дядя, а если такая хорошая была страна Советский Союз, как же вы позволили её
разрушить?» - спросили мальчики.

«Народ обманули, - с горечью и болью ответил я.

- Нам пообещали еще лучшую жизнь, чем была, а мы из-за жадности и глупости
поверили лжецам и предателям. Теперь локти кусаем...»

Один из мальчиков сказал: - «Даже не верится, что вы, обычный человек, в
советское время могли слетать в Москву, чтобы посмотреть в театре спектакль!

А мои родители вкалывают с утра до вечера, но денег в семье вечно не хватает. В
Москву мы никогда не летали, а в театр ходили лет пять назад...».

И на меня нахлынули воспоминания - я рассказал мальчикам, как в столовых на
столе стояли бесплатные тарелки с хлебом – бери, сколько хочешь и кушай на
здоровье! А, как интересно мы проводили отпуска!

Бывало так- вначале поедешь отдохнуть в Карпаты, потом оттуда летишь на Кавказ,
по дороге домой заедешь на Черное море и лишь потом возвращаешься домой.

И зарплата позволяла так отдохнуть, и страна была другая – большая и дружная.
Верно подмечено, большое видится на расстоянии.

Уже прошли годы, как не стало нашей Родины – СССР, а страна, которую мы
потеряли, всё явственнее и величественнее предстает перед нами.

Как седая мать, которую сын убил в пьяном кураже по непонятной причине. Теперь,
в утреннем, тяжелом похмелье, сын начинает постигать страшный смысл содеянного.
И до конца своих дней, образ матери будет являться ему с немым укором - что ж
ты наделал, сынок?  Сгинули старые оракулы и пророки, оказавшиеся неприлично
голыми перед насмешливым взглядом истории. А новых, как говорится, нет и не
будет. По очень простой причине. Так эффективно построить такое сложное
многонациональное государство, каким являлся СССР, как сделали это большевики,
никому, и ни на каких иных принципах не удастся, даже применительно к его
урезанному "правопреемнику" - эРэФии.

Сколько бы не золотили орлов и не придумывали бы замысловатых штандартов. Не
тот уровень, не тот калибр, не та элита...

Где сейчас все эти витии и краснобаи, "специалисты" по экономике и
межнациональным отношениям, властители дум и возмутители спокойствия, все эти
солженицыны и сахаровы, старовойтовы и шмелевы, нуйкины и пияшевы? Иных уж нет,
а те далече...

Такое впечатление, что они и сами сейчас не прочь зачеркнуть то свое
перестроечное словотворение, миллионными тиражами утюжившее общественное
сознание. Как-то хотел найти в Сети, хоть что-нибудь из публицистики того
времени, но без успеха.

Бесследно исчезли и шмелевские "Авансы и долги", и "Где пышнее пироги" Ларисы
Пияшевой, и прочие сочинения перестроечных сирен и трубадуров.

Советский Союз не был страной в традиционном смысле этого слова. Он был великой
советской цивилизацией, материализацией Красного проекта, олицетворением мечты
и надежды человечества.

Его право на существование было оплачено жизнями миллионов лучших сынов и
дочерей отечества и не подвергалось никакому сомнению, с чьей бы то ни было
стороны. Подходить к нему с мерками буржуазного права, значит
продемонстрировать свой безнадежный местечковый провинциализм, полное
непонимание его сути.

И его особый статус молчаливо признавался мировым сообществом. Советский Союз
даже в ООН был единственной страной, имевшей три места, которую кроме
собственно СССР, представляли еще Украина и Белоруссия.

Главным доводом так называемых "демократов" было утверждение, что СССР есть
империя и, как и всякая другая империя, исторически обречена на исчезновение.

Это ложь, умышленный обман народа кучкой самозваных и, зачастую, небескорыстных
"специалистов". Советский Союз никогда не был империей! Он был построен на
совершенно иных идеологических принципах. На принципах пролетарского
интернационализма.

И как бы не корежило это чей-то слух, Советский Союз был государством рабочих и
крестьян.  Что находило свое отражение и в государственной символике в виде
скрещенных серпа и молота, и в Конституции, законодательстве, и в
господствующей общественной морали.

Партийная и хозяйственная бюрократия, при всей своей косности, вынуждена была
руководствоваться интересами трудящихся классов. Правда, все более и более, не
забывая при этом, свои собственные. При отсутствии оздоровляющей обратной
связи, процесс разложения элиты зашел довольно далеко. И это не замедлило
сказаться и на обществе.

Стала происходить трансформация общественного сознания в сторону
индивидуализма, потребительства, личного успеха. А в этой системе координат,
где мерилом всего становятся деньги, материальное благополучие, советская
система явно уступала развитым странам Запада. Хотя, строго говоря, уступала не
всем, и не во всем...

Когда в кранах есть практически даровая холодная и горячая вода, газ по 20 коп.
в месяц, теплое жилье, электричество в розетке, бесплатное образование в
голове, можно и пофантазировать на кухне про "скромное обаяние буржуазии",
плюрализм и рыночные отношения. Кухонные мыслители не понимали, что они уже
живут в самой свободной и процветающей стране мира.

Со всеми ее очередями и идиотскими дефицитами.

Это была свобода, недоступная западному обывателю.

Свобода от безысходности, от бесправия, от голода, от изнуряющей работы, от
эксплуатации, от нищеты, от страха перед будущим. Раньше все это казалось само
собой разумеющимся, незыблемым порядком вещей.

Нынешняя молодежь уже с трудом во все это верит. Система больше не устраивала
"верхи" чем "низы".

Вполне допускаю, что большая часть элиты совершенно искренне хотела перемен к
лучшему. В годы войны немцы с самолетов сбрасывали над расположением частей
Красной армии листовки. Они являлись пропуском для беспрепятственной сдачи в
плен.

Наверное, в них расписывались преимущества сытой и веселой жизни в лагерном
бараке с салом и шнапсом, как альтернатива неизбежной гибели под немецкими
бомбами на фронте. Но одно дело, когда подобную агитку противника поднимает не
слишком грамотный солдат, и другое, когда эту листовку держит в руках генерал.

И не только держит, но и верит тому, что в ней написано.

И не только верит, но и организует сдачу доверенной ему армии врагу.

Так высшее руководство СССР, в силу своего невежества, некомпетентности,
вопиющего дилетантизма, вопреки воле своих народов, выраженной на мартовском
референдуме, и сдало нашу Родину.

После "победы демократии" в августе 91-го, первое, что вспоминаю, это
истеричные вопли по телевизору про коммунистов, которые через Амур тысячами
бегут в Китай.

Это прием уже не просто пропаганды, а сознательной информационной войны,
применяемой для ошеломления и дезориентации противника на поле боя.

Это формирование новой реальности при помощи информационного оружия. Подобный
прием повторился, когда после ельцинского мятежа в октябре 93-го сообщалось,
что арестованных участников сопротивления "тысячами свозят на стадион", явно
стараясь вызвать ассоциации с пиночетовским путчем 1973 года, и дать установку
на желательное дальнейшее развитие событий.

Подобная "журналистика" весьма характерна для боевых операций американской
армии, в частности в Ираке. Так что аналогия с немецкой листовкой не так уж и
фантастична. На войне, как на войне...

Иногда, можно слышать что, дескать, в Советское время были очереди за колбасой
по 20, 30 человек, a сейчас очередей нет. Да было. Было, зачем стоять.

Да если сейчас положить на прилавок ту советскую колбасу, то очередь была бы по
200 и 300 человек, а, то и более. Да сейчас очередей нет. Потому что колбасы
нет. Разве это колбаса, если в ней нет мяса?

Тоже самое можно говорить и о других нынешних демократических продуктах, если
их можно назвать продуктами, а не всеобщей отравой.

Советский Союз кормил своих граждан натуральными продуктами, пусть и с
дефицитом Советские люди жили по законам, где их интересы нигде и никогда не
пересекались с деньгами и выгодой, кумирами их были не банкиры и дилеры, не
брокеры и иже с ними.

Героями их были мужественные и сильные люди из поколения их отцов, из 20-х и
40-х годов.  Они равнялись на тех, кто строил Волховскую электростанцию и
Днепрогэс, Магнитку, Комсомольск-на-Амуре, Новокузнецк, кто осваивал Сибирь и
Дальний Восток, строил заводы и дома, кто выиграл страшную войну и освободил
Европу от фашизма.

Конечно, все это прописные истины, но, к сожалению, они теперь напрочь забыты.
Из истории нашей страны вычеркнуты семьдесят лет, семьдесят лет жизни советских
людей, которых теперь можно сравнивать с «пылью» и называть «совками», хотя до
сих пор Россия и все мы живем тем, что было создано поколениями 20-70-х годов.

Стыдно и страшно, что те, кто пинает прошлое нашей Родины, забыли, что все они
родом из СССР, что все они хоть и бывшие, но советские люди, что история СССР –
история их Родины, которая не бывает бывшей, что «совки» – это их деды и отцы,
которые не продавались и не предавали, а были людьми чести и долга.

Стыдно за наш народ, который молча позволяет плевать в свое прошлое. Мы молчим,
как будто и не было в нашей жизни 70 лет побед и строек, прекрасных песен и
стихов, замечательных спортсменов и блестящих ученых, как будто не было нашей
Родины – Советского Союза.

Мы забыли, что все, что мы имеем и чем пользуемся, досталось нам от Советского
Союза: наше жилье, больницы, поликлиники, детские сады, школы, санатории, дома
отдыха, пионерские лагеря, нефть, газ, электроэнергия...

И молча смотрим, как абрамовичи и прочие нагло проматывают наше наследство,
достояние нашей страны, созданное трудом поколений 20-70-х годов.

Теперь у нас превыше всего – доллар, он и в сердце, и в глазах, все продается и
распродается оптом и в розницу. Что ж удивляться тому, что о нашей истории
рассказывают иностранцы: англичане, американцы, немцы (фильм «Враг у ворот» о
Сталинградской битве; энциклопедия «Асы Сталина» о наших летчиках, защищавших
небо Родины). Низкий поклон этим людям за их память и интерес к советской
истории.

А моя особая благодарность английским журналистам Томасу Полаку и Кристоферу
Шоурзу, составителям энциклопедии «Асы Сталина», за их огромный труд (в
энциклопедии более тысячи имен советских летчиков). Именно в этой энциклопедии
я нашёл сведения о 127 ИАП Ленинградского фронта, где воевал и погиб родной мне
человек.

Из этого полка названо девять фамилий. За первые четыре месяца Великой
Отечественной войны шестеро из этих 9 сбили 38 самолетов, 2 – тараном; 38
фашистских самолетов не долетели до Ленинграда. И это только один полк и всего
6 летчиков-истребителей, мужеству и мастерству которых отдают дань английские
журналисты.

А что же наши писатели, историки, журналисты?

Молчат. Табу на все 70 лет советской истории; или ничего, или только плохо.

Мне больно оттого, что исчезла моя Родина – Советский Союз, и это не ностальгия
по дешевой колбасе, как модно сейчас определять суть прошлого.

Это ностальгия по той жизни, где милиция была без дубинок и черных масок, где
не было стальных дверей с глазком и десятью запорами и решетками на окнах, где
можно было гулять ночами, не боясь, что тебя ограбят или прибьют, где жизнь
человека, хоть он и был «винтиком», что-то стоила.

Там, в той жизни, не расстреливали закон в центре столицы и не бомбили
собственные города, не взрывали людей в театрах и метро, в электричках и
собственных домах.

Там была уверенность в завтрашнем дне и в завтрашнем дне детей, там люди много
смеялись, спорили, мечтали и не делились на людей кавказской, азиатской и
прочей национальности, там двери дома были открыты для каждого.

Но... Но мы выбрали свой путь сами, отреклись от прошлого и предали своих
отцов. А история и жизнь такого не прощают. Без прошлого нет будущего.

Сейчас, даже высшим руководством России признается, что разрушение СССР было
крупнейшей геополитической катастрофой. Поубавилось любителей щегольнуть
геростратовой славой своей причастности к Беловежскому сговору. Все вдруг,
оказывается, были ярыми сторонниками сохранения Союза и если бы не ГКЧП, да не
партократы, да не то, да не это...

Вся эта публика боится не восстановления Союза.

Она смертельно боится неотвратимого возмездия за содеянные государственные
преступления. Ей не дает покоя страх предстать перед суровым судом народов и
понести ответственность соразмерную содеянному.

Дамоклов меч, висящий над их головами, мешает нормальной праздной жизни
отставных предателей. Председательствовать в фондах, читать "лекции" в
университетах, красоваться в европейских салонах, пристраивать своих отпрысков
к кормушкам.

Не знаю, как насчет ума или сердца, но насчет совести могу сказать определенно.
Ее нет у людей, которых после разрушения своей страны не преследуют эти глаза
седой матери, принявшей мученическую смерть от рук любимого сына, с немым
вопросом – что ж ты наделал, сынок?

Автор: Сергей МАЛЬЦЕВ.


