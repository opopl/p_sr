% vim: keymap=russian-jcukenwin
%%beginhead 
 
%%file 23_02_2019.stz.news.ua.mrpl_city.1.k_122_letiu_mmk_im_iljicha_jakov_privezencev
%%parent 23_02_2019
 
%%url https://mrpl.city/blogs/view/k-122-letiyu-mmk-im-ilicha-yakov-privezentsev
 
%%author_id burov_sergij.mariupol,news.ua.mrpl_city
%%date 
 
%%tags 
%%title К 122-летию ММК им. Ильича: Яков Привезенцев
 
%%endhead 
 
\subsection{К 122-летию ММК им. Ильича: Яков Привезенцев}
\label{sec:23_02_2019.stz.news.ua.mrpl_city.1.k_122_letiu_mmk_im_iljicha_jakov_privezencev}
 
\Purl{https://mrpl.city/blogs/view/k-122-letiyu-mmk-im-ilicha-yakov-privezentsev}
\ifcmt
 author_begin
   author_id burov_sergij.mariupol,news.ua.mrpl_city
 author_end
\fi

Только началось строительство нового мартеновского цеха на заводе имени Ильича,
а его начальника уже назначили. Им стал Яков Григорьевич Привезенцев. Делалось
так, а может быть и сейчас делается, при сооружении новых цехов. Кто-то должен
был формировать штат нового подразделения предприятия, следить за правильным
исполнением проектных решений, а иногда и подправлять эти решения, используя
производственный опыт. А опыта этого было у Привезенцева немало. Кстати, до
перевода на новую должность, он руководил мартеновским цехом №2, тем самым, где
в довоенные годы ставил свои рекорды \href{https://archive.org/details/16_02_2019.sergij_burov.mrpl_city.k_122_letiu_mmk_im_iljicha_rekordy_makara_mazaja.}{Макар Мазай}. Но запускать ни одну из печей
нового цеха Якову Григорьевичу не пришлось. Через год его сменил Н. Г. Зарубин...

Накануне 50-летия мартеновского цеха Юрию Яковлевичу Некрасовскому, на то время
журналисту газеты
\href{https://mrpl.city/blogs/view/geroi-novyh-formatov}{\enquote{Ильичевец}},
предложили написать книгу об истории этого подразделения завода им. Ильича. В
процессе работы над ней возникло немало трудностей. В том числе, крайне скупы
были сведения о Я. Г. Привезенцеве. Автор будущей книги обратился в отдел
кадров комбината, но там ответили, что личные дела умерших руководителей
отправлены в Донецк в какой-то архив. Ю. Я.  Некрасовский потратил много
времени, но вышел на родственницу Якова Григорьевича, которая помнила героя
нашего очерка и, кроме того, сохранила некоторые его документы.

\ii{23_02_2019.stz.news.ua.mrpl_city.1.k_122_letiu_mmk_im_iljicha_jakov_privezencev.pic.1}

Этими документами воспользовался и автор этих строк. Итак, в автобиографии Яков
Григорьевич пишет, что родился он 16 октября 1896 года в селе Тайдаково
Покровской волости Алексинского уезда Тульской губернии. Семья его отца
состояла из четырнадцати человек. Работать он начал в четырнадцать лет в одной
из московских швейных фабрик. В 1915 году его мобилизовали на войну. Тогда
Якову было девятнадцать лет. На обороте одной из фотографий с изображением
Якова Григорьевича начертан такой текст: \emph{\enquote{На добрую и долгую память дорогому
Саше, 16/IX 66}} и подпись. А ниже уже другим почерком написано: \emph{\enquote{Участник
Брусиловского прорыва в 1-й империалистической войне, за что награжден
Георгиевским крестом. Участник гражданской войны 1917 – 1921}}. 

\textbf{Читайте также:} 

\href{https://archive.org/details/02_02_2019.sergij_burov.mrpl_city.k_122_letiu_mkk_im_iljicha_vladimir_bojko}{%
К 122-летию ММК им. Ильича: Владимир Бойко, Сергей Буров, mrpl.city, 02.02.2019}

После окончания первой мировой войны Привезенцев оказался в городе Острогожске,
ныне входящем в Воронежскую область. Здесь он некоторое время трудился в
швейной мастерской рабочего кооператива, а затем добровольно вступил в Красную
Армию. Ему довелось участвовать в боях с деникинцами и мамонтовцами. После
демобилизации - год службы в острогожской милиции, три года киномехаником
Политпросвета.

Мариупольский период жизни молодого ветерана двух войн начался 9 февраля 1926
года, когда его приняли в мартеновский цех на площадке \enquote{Б} завода им. Ильича.
Яков Привезенцев стал там работать канавщиком. Прежде чем продолжить
повествование, наверное, нужно объяснить, что такое \enquote{канава}, и чем занимались
там \enquote{канавщики}. Канава – это углубление, - а иногда просто площадка, - в
литейном пролете мартеновского цеха, где устанавливаются чугунные поддоны, а на
них чугунные изложницы – емкости для отливки слитков. Канавщики показывают
машинистам кранов, куда ставить поддоны и изложницы, предварительно очищенные и
смазанные. Самая тяжелая работа у канавщика, когда приходится чистить горячие
поддоны, подготавливая их к очередной разливке.

\vspace{0.5cm}
\begin{minipage}{0.9\textwidth}
\textbf{Читайте также:}

\href{https://archive.org/details/09_02_2019.sergij_burov.mrpl_city.k_122_letiu_mmk_im_iljicha_dmitrii_grushevskii}{%
К 122-летию ММК им. Ильича: Дмитрий Грушевский, Сергей Буров, mrpl.city, 09.02.2019}
\end{minipage}
\vspace{0.5cm}

В ноябре 1930 года при заводе им. Ильича был открыт Вечерний металлургический
институт. В него поступил Яков Григорьевич. К тому времени ему исполнилось
тридцать четыре года. Учиться ему было нелегко. С 1940 года он начал работать
мастером в мартеновском цехе. Но вопреки сложившимся неблагоприятным условиям,
он шел к цели. И в июле 1935 года ему вручили диплом, где было записано: \emph{\enquote{Гр.
Привезенцев Я. Г. защитил в Государственной квалификационной комиссии дипломный
проект на тему: \enquote{Мартеновский цех с качающимися печами в условиях завода
Азовсталь, рассчитанного на производство болванок для Маннесмановского цеха
завода им. Ильича} с оценкой \enquote{отлично}. На основании изложенного гр.
Привезенцеву Я. Г. присвоена квалификация инженера-металлурга по производству
стали}}.

Новоиспеченного инженера направили в мартеновский цех завода \enquote{Азовсталь}, где
он около восьми месяцев работал начальником смены. С конца марта 1936 года
начинается стремительное движение Якова Привезенцева вверх по служебной
лестнице. До 1938 года - начальник литейного пролета цеха №2 завода имени
Ильича, с 1938 по 1939 год там же - заместитель начальника цеха, с 1939 по
сентябрь 1941 года – начальник этого же цеха. Эвакуация на Южный Урал в город
Челябинск. Якова Григорьевича назначают заместителем начальника мартеновского
цеха №1 завода \enquote{200} Наркомата танковой промышленности. За освоение выплавки
броневой стали для легендарного танка Т-34 был награжден орденом Трудового
Красного Знамени.

\textbf{Читайте также:} 

\href{https://mrpl.city/news/view/na-mariupolskom-kombinate-imeni-ilicha-poyavilsya-virtualnyj-trenazher-foto}{%
На мариупольском комбинате имени Ильича появился виртуальный тренажер, Анастасія Папуш, mrpl.city, 07.02.2019}

В автобиографии Яков Григорьевич пишет: \enquote{жена Мария Федоровна 1896 года
рождения. Сын Игорь, 1923 года рождения}. Игорь Яковлевич перед войной поступил
в Ейское военно-морское авиационное училище. Еще курсантом попал на фронт, был
тяжело ранен. Лишь искусство мариупольских врачей спасло его от инвалидности.
После войны окончил Ждановский металлургический институт, стал преподавателем,
проректором института. Он увлекался спортом и туризмом. Был председателем
институтского клуба туристов.

Но вернемся к жизнеописанию его отца – Якова Григорьевича.

За самоотверженный труд, высокие показатели в производственной деятельности
Яков Григорьевич Привезенцев был награжден двумя орденами Трудового Красного
Знамени, орденами Ленина и \enquote{Знаком почета}. У Якова Григорьевича было три
жизненных приоритета: семья, здоровый образ жизни – он делал зарядку, обливался
холодной водой круглый год, и главное – производство. Он работал на заводе
имени Ильича до своей кончины в 1974 году.

\ii{23_02_2019.stz.news.ua.mrpl_city.1.k_122_letiu_mmk_im_iljicha_jakov_privezencev.pic.2}

\textbf{Читайте также:} 

\href{https://mrpl.city/news/view/na-mmk-imeni-ilicha-dostavili-dva-e-ksklyuzivnyh-150-tonnyh-e-lektrodvigatelya-foto}{%
На ММК имени Ильича доставили два эксклюзивных 150-тонных электродвигателя, Анастасія Папуш, mrpl.city, 22.02.2019}
