% vim: keymap=russian-jcukenwin
%%beginhead 
 
%%file 29_04_2022.fb.solovjov_mikita.harkov.demsokyra.1.hronika.cmt
%%parent 29_04_2022.fb.solovjov_mikita.harkov.demsokyra.1.hronika
 
%%url 
 
%%author_id 
%%date 
 
%%tags 
%%title 
 
%%endhead 
\zzSecCmt

\begin{itemize} % {
\iusr{Евгений Бурцев}

Салтовку сегодня обстреливали меньше, чем обычно. Может даже ощутимо меньше.

Поехавшим никто не может нормально донести информацию. Все начинается и
заканчивается на словах \enquote{в то направление} или вас поселят в
заброшенной деревне и дадут чужой огород капать. Многие живут по садикам,
школам и тп.  Огромная проблема алкоголизма, отсутствия работы, возможности
даже снять пенсию.

По Салтовке, в частности в ГТ, могу как первоисточник дать много инфы

\begin{itemize} % {
\iusr{Микита Соловйов}

Я говорю о конкретных районах и эвакуации оттуда. Я немного помогаю волонтерам
там и знаю, какие варианты предлагают. Особенно когда речь идет о возрастных и
маломобильных людях, там и близко речь не идет о направлении. Только вывоз в
конкретное место и поселение там.

\iusr{Микита Соловйов}
Да, а за детали в личку буду благодарен )

\iusr{Евгений Бурцев}
\textbf{Микита Соловйов} пенсы, то отдельный случай. За которыми ездят обычные пацаны
под большой риск. А потом оказывается, многие из тех ждали zz флаг @igg{fbicon.face.neutral} 

В метро тоже прикормились неплохо на харчах. Время от времени устраивают бойни
между собой за очередь, в сторону волонтеров. Все для них, по их мнению, возит
наш мер, а по факту мы сами наладили пути. Двое-трое выпало из раздачи и
кормежка 3-5 в день прекращается.

К чему это все. Пускай те живут своей жизнью. Дали два автобуса и хватит. А
ресурсы лучше направить на тех, кто в этом действительно нуждается. Цинично. Но
так правильнее.

\iusr{Roman Frolov}
\textbf{Евгений Бурцев} 

может тогда едой приманивать - тут больше кормить не будут (и сделать перебои)
а вот там куда повезет автобус - там бесплатно кормят и обустраивают.

Почти как пропаганда про \enquote{вернем ссср} вещает.

Возможно для некоторьіх сработает.

\iusr{Elena Medvedeva}
\textbf{Микита Соловйов} 

у меня двое старых родственников живут в Черкасских тишках. Грады орковские
ставят у них за забором. В доме нет ни одного окна уже. Куры и кроли дохнут от
страха от постоянных взрывов. Когда они мне могут дозвониться то у меня телефон
выпадает из рук от взрывов возле их дома. Они готовы эвакуироваться в сторону
Харькова. Мы их готовы забрать и поселить у себя на другом конце Харькова. Как
это сделать? Как им оттуда выехать? Там только один выезд сейчас на Белгород.
Они сказали, что лучше погибнут под руинами своего дома, но в рашку не поедут.

\iusr{Микита Соловйов}
\textbf{Елена Медведева} Оттуда я вариантов не знаю.


\end{itemize} % }

\iusr{Shersh Stas}

не могу никак найти вариант вывоза мамы из Изюма...

там вообще глухо? ни за какие деньги?

\iusr{Микита Соловйов}
Написал в личку.

\iusr{Ivan Varonen}
А из моего окна я вижу АХ.

\iusr{Roman Frolov}
\textbf{Микита Соловйов} 

про теплоснабжение вьі упустили в каком контесте бьіло сказано - "Відвідав
сьогодні Дергачівську та Малоданилівську територіальні громади.

На власні очі побачив наслідки дій окупантів в цих регіонах. Ворог наробив
багато лиха, але все відновимо!" .....

Також уже зараз необхідно готуватися до відновлення систем теплопостачання. Нам
треба розуміти, до яких наслідків можуть призвести воєнні дії та готуватися до
наступного опалювального сезону.

Речь шла не о городе а про прилегающие населенньіе пунктьі, где разрушена
инфраструктура.

\iusr{Yulia Denisyuk}

Доставать градами и ураганами Харьков могут даже из Белгорода. Поэтому
освобождение Циркунов, Тишков и других населенных пунктов никак не решит
ситуацию с обстрелами, к сожалению(

\begin{itemize} % {
\iusr{Юрій Матюхін}
\textbf{Юлия Денисюк} 

градами бм 21 не могут, ствольной артой, минометами не смогут. ураганом
смерчем пионом - да, но это уже плотность и интенсивность обстрелов уменьшается
в разы, п лолюс для орков это на порядок дороже. Это очень не дешевое
удовольствие и колличество таких боеприпасов на порядки меньше

\iusr{Микита Соловйов}
\textbf{Юлия Денисюк} 

Смотря что вы понимаете под решением ситуации с обстрелами. Если полную
невозможность обстрелов, то это вообще только победа и ликвидация такого
явления как Россия. А если заметное снижение плотности обстрелов, то конечно
станет намного легче после создания пояса безопасности в 15-20 км. Уже после 10
км каждый следующий километр снижает риски для Харькова.

\end{itemize} % }

\iusr{Юрій Матюхін}
Спасибо

\iusr{Lyubov Ostrenko}
Можливо в Харкові.

\ifcmt
  ig https://scontent-mxp1-1.xx.fbcdn.net/v/t39.30808-6/278587797_3188880028063675_1445734330431027937_n.jpg?_nc_cat=100&ccb=1-5&_nc_sid=dbeb18&_nc_ohc=BD6Gq7F8msEAX8QJ0VO&_nc_ht=scontent-mxp1-1.xx&oh=00_AT_wYm5TkXG1APLjJRfi5Fmmcj1JLk2TgnFCIhyMH3ya8A&oe=62736B5B
  @width 0.3
\fi

\end{itemize} % }
