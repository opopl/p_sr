% vim: keymap=russian-jcukenwin
%%beginhead 
 
%%file 16_01_2014.fb.dykyj_evgen.1.zaklyk
%%parent 16_01_2014
 
%%url https://www.facebook.com/evgen.dykyj/posts/10152100692083808
 
%%author_id dykyj_evgen
%%date 
 
%%tags maidan2,ukraina
%%title ПУБЛІЧНИЙ ЗАКЛИК ДО ВІДНОВЛЕННЯ КОНСТИТУЦІЙНОГО ЛАДУ
 
%%endhead 
 
\subsection{ПУБЛІЧНИЙ ЗАКЛИК ДО ВІДНОВЛЕННЯ КОНСТИТУЦІЙНОГО ЛАДУ}
\label{sec:16_01_2014.fb.dykyj_evgen.1.zaklyk}
 
\Purl{https://www.facebook.com/evgen.dykyj/posts/10152100692083808}
\ifcmt
 author_begin
   author_id dykyj_evgen
 author_end
\fi

ПУБЛІЧНИЙ ЗАКЛИК ДО ВІДНОВЛЕННЯ КОНСТИТУЦІЙНОГО ЛАДУ. МАКСИМАЛЬНИЙ ПЕРЕПОСТ,
ХТО ЩЕ НЕ ЗЛЯКАВСЯ.

Сьогодні остаточно завершився антиконституційний заколот, який почався у
вересні 2010 з відміни Конституції-2004. Тепер ми не маємо ні Конституції, ні
законного президенту, ні законного парламенту. Як тільки ухвалені путчистами
\enquote{закони} запрацюють, ми не матимемо навіть залишків громадянських прав та
свобод. Наш єдиний шанс прокинутись завтра не в Білорусі чи РФ, а в Україні -
вже сьогодні всім разом вийти на барикади та своєю волею, волею громадян
України, ВІДНОВИТИ ПОВАЛЕНИЙ КОНСТИТУЦІЙНИЙ ЛАД. Так, це означає ПОВАЛЕННЯ
РЕЖИМУ В БУДЬ-ЯКИЙ СПОСІБ.  Але повалення режиму, який незаконно узурпував
виконавчу, судову та законодавчу владу - не є злочином, а є відновленням
Конституції. Захист Конституції - обов'язок громадян.

На виконання цього обов'язку у нас лишились лічені години. Ми і так дали режиму
оговтатись після грудневого повстання, і от тепер маємо результат. Якщо ми не
здатні зараз захистити Конституцію - нас чекає довга-довга боротьба в умовах
диктатури.

Сьогодні-завтра диктатура ще не окріпла, ми ще маємо шанс, якщо будемо дуже
швидкими та рішучими.

Сьогодні - масовий збір киян на барикадах Майдану, оголошення про те, що ця
влада більше нами не визнається, є путчистами та кримінальними злочинцями.
Рішення громади про відновленя конституційного ладу та арешт заколотників,
включно з громадянином Януковичем, який видає себе за президента України, так
званими \enquote{міністрами} так званого \enquote{уряду}, та депутатами які брали участь у
ліквідації парламентаризму сьогодні в Раді.

Залежно від кількості киян, готових захищати Конституцію - або самі сьогодні ж
це реалізовуємо, або тримаємось на барикадах і розбудовуємо Майдан як єдиний
поки що плацдарм, звільнений від окупаційного режиму, і чекаємо підмоги з усієї
України.

Не-кияни - кидаймо все і вперед на столицю! Як тільки завтра досягаємо
критичної кількості рішучих відповідальних громадян - див. вище, придушуємо
антиконституційний путч \enquote{регіоналів}.

Якщо ми здатні так мобілізуватись та за добу протиставити режиму сотні тисяч
захисників конституційного ладу - вже у суботу ми будемо жити у іншій, нашій
Україні. Якщо не здатні - також вже до суботи будемо жити в іншій Україні, вже
надовго не нашій. Вибір за нами, відлік часу пішов.
