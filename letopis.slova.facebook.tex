% vim: keymap=russian-jcukenwin
%%beginhead 
 
%%file slova.facebook
%%parent slova
 
%%url 
 
%%author_id 
%%date 
 
%%tags 
%%title 
 
%%endhead 
\chapter{Фейсбук}

%%%cit
%%%cit_head
%%%cit_pic
%%%cit_text
The Washington Post прошелся по событиям 6 января 2021 года, когда трамписты
штурмовали Капитолий. По их мнению, \emph{Facebook} подстрекал их к этому -
хотя после и забанил аккаунт действующего президента.  Согласно внутреннему
отчету за 6 января, резко усилились жалобы на фейковые новости, их число
достигало почти 40 000 в час. В Instagram чаще всего сообщалось о
подстрекательстве к насилию под официальным аккаунтом президента
@realdonaldtrump.  При этом указывается, что компания отклонила рекомендацию
своего собственного Наблюдательного совета изучить, как ее политика
способствовала насилию.  CNN также пишет о штурме Капитолия и считает, что
\emph{Facebook} то ли намеренно, то ли случайно проигнорировала призывы к
штурму Капитолия
%%%cit_comment
%%%cit_title
\citTitle{Разжигает ненависть от Азии до Африки. Почему Facebook одновременно атаковали крупнейшие мировые СМИ}, 
Максим Минин; Екатерина Терехова, strana.news, 26.10.2021
%%%endcit
