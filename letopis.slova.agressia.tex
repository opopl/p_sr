% vim: keymap=russian-jcukenwin
%%beginhead 
 
%%file slova.agressia
%%parent slova
 
%%url 
 
%%author 
%%author_id 
%%author_url 
 
%%tags 
%%title 
 
%%endhead 
\chapter{Агрессия}
\label{sec:slova.agressia}

%%%cit
%%%cit_head
%%%cit_pic
\ifcmt
  tab_begin cols=2
		width 0.4
     pic https://strana.ua/img/forall/u/0/36/8870e29b30de20a65403b30c30b4b37560dc7956d45f7795881761.jpg
		width 0.46

     pic https://strana.ua/img/forall/u/0/36/2021-07-01_09h35_48.png
  tab_end
\fi
%%%cit_text
Парень рассказал о том, что многие украинские фанаты подходили к ним с
возмущениями, поэтому людям приходилось объяснять, что означают цвета их
одежды. При этом, по словам Толкача, только один человек подошел с целью
затеять драку. Сергей предполагает, что нападавший был под градусом.
Полиция увела \emph{агрессора} со стадиона.
Кроме того, по словам Толкача, многие украинские болельщики защищали их
%%%cit_comment
%%%cit_title
\citTitle{На Евро 2020 избили российского фаната Дениса Смоленского, что известно об инциденте}, 
Анна Копытько, strana.ua, 01.07.2021
%%%endcit

%%%cit
%%%cit_head
%%%cit_pic
%%%cit_text
Ні, ти кажеш, що держава втрачає монополію на насилля… Смотри, она дальше потом
ее возвращает, когда проходит. А у нас этого не произошло. И в результате мы
имеем то, что это вот такое насильственное решение каких-то конфликтов, оно
становится частью повседневности. И это касается не только каких-то
политических споров – это касается даже споров на дороге. Мы все прекрасно
знаем, что куча людей ездит с пистолетами, с автоматами, с битами, с газовыми
баллончиками, и периодически друг другу что-то делают, какие-то увечья наносят.
И этот уровень \emph{агрессии} он как раз же о чем говорит – что государство оно же не
спускает эту \emph{агрессию}, не дает ей выход, оно не наказывает хулиганов, да, оно
позволяет, оставляет им пространство, в котором они могут таким образом
самореализовываться.  У нас во дворе есть чувак, который терроризировал, сейчас
затих из-за того, что построили стоянку под домом. Два дома соседних построили
стоянку, и дальше начинается традиционный конфликт украинский в духе
рязановского фильма «Гараж» - начинается выяснение отношений кто же может
поставить авто, кто же не может поставить. И там одно быдло почувствовало, что
он там царь горы, с пистолетом бегал, выскакивал
%%%cit_comment
%%%cit_title
\citTitle{Украина в плену консервативного мышления: почему 30 лет шли не туда и что делать}, 
Сергей Иванов; Юрий Романенко, hvylya.net, 15.07.2021
%%%endcit

%%%cit
%%%cit_head
%%%cit_pic
%%%cit_text
То есть, иными словами, высшее руководство Украины публично называет Путина
\emph{газовым агрессором} и тут же предлагает этому \emph{агрессору}, у которого, вообще то
планы по уничтожению украинской государственности, а газовый кризис в Европе
является одной из промежуточных целей реванширующей империи, какую то смешную
"скидку" на транзит и какой то хитровымученый "план", чтобы европейских "лохов"
Путину было веселее "обувать" и еще самим с газом в ГТС остаться и какие то
деньги за транзит от агрессора сохранить.  Представляете, насколько "власна
добыча" из "трубы" важна для наших гавриков, если они \emph{агрессору} почти открытым
текстом транзитные "скидки" обещают))))))) Пришли козлята кремлевского
крокодила учить охотиться)))
%%%cit_comment
%%%cit_title
\citTitle{Руководство Украины публично называет Путина газовым агрессором / Лента соцсетей / Страна}, 
Богдан Савруцкий, strana.news, 24.10.2021
%%%endcit

%%%cit
%%%cit_head
%%%cit_pic
%%%cit_text
Гражданам Украины (прежде всего госслужащим и СМИ) предписывается сверху, как
называть то или иное событие и/или явление. Правда, пока вроде как
рекомендательно. Это пока еще не Геббельс и не «1984» Оруэлла. Но направление
движения — именно туда.  Делается это все во имя священной цели — борьбы с
российской \emph{агрессией} в информационном пространстве.Здесь следует сделать
замечание. Желающих повоевать на Донбассе, по слухам, все меньше и меньше.
Поскольку это опасно для жизни и здоровья. Зато бороться с информационной
\emph{агрессией} Кремля в столичных офисах — желающих хоть отбавляй. Более
того, эти желающие только тем и занимаются, что навязывают обществу свои, как
нынче модно говорить, нарративы, не обладая при этом соответствующими
интеллектуальными и моральными качествами
%%%cit_comment
%%%cit_title
\citTitle{Краткий словарь грантоедов под редакцией СНБО / Лента соцсетей / Страна}, 
Александр Карпец, strana.news, 26.10.2021
%%%endcit

%%%cit
%%%cit_head
%%%cit_pic
\ifcmt
  tab_begin cols=2
     pic https://mir-lug.info/wp-content/uploads/2021/10/img_0051.jpg
     pic https://mir-lug.info/wp-content/uploads/2021/10/img_0021.jpg
  tab_end
\fi
%%%cit_text
В Луганске на мемориальном комплексе «Не забудем, не простим!» 29 октября
перезахоронили останки мирных жителей – жертв украинской агрессии, поднятых в
рамках деятельности Межведомственной рабочей группы по розыску захоронений
жертв украинской \emph{агрессии}, их идентификации и увековечению памяти.  В
торжественной обстановке перезахоронили 50 останков мирных жителей – жертв
украинской \emph{агрессии}, поднятых в ходе первого цикла работ по вскрытию массового
захоронения жертв украинской \emph{агрессии} вблизи трассы Т-13-01 в районе поселка
Видный-1. Почтить память пришли представители Администрации Главы ЛНР,
Администрации города Луганска, Межведомственной рабочей группы, представители
Общественного движения «Мир Луганщине» и активисты проекта «Мы помним!» ОД «Мир
Луганщине»
%%%cit_comment
%%%cit_title
\citTitle{В Луганске перезахоронили останки мирных жителей – жертв украинской агрессии}, 
, mir-lug.info, 29.10.2021
%%%endcit

%%%cit
%%%cit_head
%%%cit_pic
%%%cit_text
Общая идеология – только в статусе колонии у Украины есть шанс выжить, а сами
мы обречены. Украинцы иногда спорили, но главный аргумент – факты. Собственно,
если бы российская пропаганда не врала и наша страна действительно не могла
жить самостоятельно – Кремлю не пришлось бы переходить к прямой \emph{агрессии}.
Достаточно было подождать. Но в Москве прекрасно видели реальную картину и
понимали – без войны по доброй воле Украина в этот концлагерь не вернется.
Тем не менее миф про богатую и щедрую Россию прочно укоренился в некоторых не
самых умных головах наших соотечественников. Это неизбежно при таком напоре
пропаганды. И на Донбассе, и в Крыму в 2013-2014 годах хватало разговоров про
то, что при Украине живут бедно, а вот придет Россия и наступят жирные времена.
Ну, Донбасс "наелся" щедрот русского мира уже по полной программе. Безработица,
смешные с украинской точки зрения зарплаты, дороговизна – вот и пробиваются
правдами и неправдами жители оккупированных территорий за украинскими пенсиями
%%%cit_comment
%%%cit_title
\citTitle{Украина больше не вернется в русский концлагерь}, 
Кирилл Сазонов, news.obozrevatel.com, 30.10.2021
%%%endcit

%%%cit
%%%cit_head
%%%cit_pic
\ifcmt
  pic https://storage.lug-info.com/cache/4/8/09edf3ae-c4bd-4794-8420-46e961540eda.jpg/w700h474%7Cwm
  @width 0.4
\fi
%%%cit_text
В результате украинской \emph{агрессии} пострадала гражданская инфраструктура.
Всего с 00:00 часов 21 июля 2019 года режим всеобъемлющего устойчивого и
бессрочного прекращения огня со стороны вооруженных формирований Украины был
нарушен 962 раза, из них 628 раз – после вступления в силу допмер; 39
защитников Республики погибли, 23 получили ранения. Среди мирного населения 5
человек погибли, 37 получили ранения, поврежден 181 объект гражданской
инфраструктуры
%%%cit_comment
%%%cit_title
\citTitle{НЕДЕЛЯ ГЛАЗАМИ ЭКСПЕРТА: Ротация блокираторов, экобомбист и комедия Гайдая}, 
, lug-info.com, 07.11.2021
%%%endcit

%%%cit
%%%cit_head
%%%cit_pic
\ifcmt
  tab_begin cols=4
		 pic https://i1.wp.com/giraffebig.ru/wp-content/uploads/2018/08/275190-Sepik_2048x1152-e1535021997346.jpg?fit=1200%2C869&ssl=1
		 pic https://ds02.infourok.ru/uploads/ex/00df/00003b76-3fb941ed/1/img8.jpg
     pic http://www.smsport.ru/image/dialcon/olymp/o20.jpg
     pic https://avatars.mds.yandex.net/i?id=2a00000179febe260bfa894217e950d2bcdb-5167369-images-thumbs&n=13
  tab_end
\fi
%%%cit_text
На сей раз речь идет о законопроекте № 4188 от 5 октября 2020 года о внесении
изменений в закон о телевидении и радиовещании относительно противодействия
информационной войне. Основным моментом этого яркого образчика нормотворчества
является запрет на отрицание «факта военной \emph{агрессии} Российской
Федерации», за что предлагается лишать телеканалы лицензий на право
телевещания.  К слову, если кто не в курсе. Нацсовет по телевидению назначил
телеканалу «Интер» проверку из-за показа советских кинокомедий «Служебный
роман» и «Кавказская пленница». Очевидно, по причине того, что через эти
шедевры Эльдара Рязанова и Леонида Гайдая Кремль в чьем-то больном воображении
осуществляется \emph{коварная информационная агрессия}. Этак скоро и мультик
про Чебурашку запретят...
%%%cit_comment
%%%cit_title
\citTitle{Украину хотят окончательно превратить в патриотический зоопарк / Лента соцсетей / Страна}, 
Александр Карпец, strana.news, 11.11.2021
%%%endcit

%%%cit
%%%cit_head
%%%cit_pic
%%%cit_text
Война предполагает готовность воевать. Кроме враждебности и \emph{агрессии},
война требует материальной, экономической, технической, стратегической,
информационной и моральной подготовки. Если одна из сторон не готова воевать,
то есть продолжить свою враждебность в \emph{агрессивной} форме войны, она проигрывает
войну и подчиняется другой стороной.  Враждебность без подготовки к войне суть
чистый ресентимент, обреченный на погибель.  Война суть прямое столкновение с
противником, позволяющее вначале сильно усилить враждебность, а затем через
уничтожение или подчинение врага ослабить враждебность.  Война не закончена,
пока враждебность не ослаблена или стороны не остановлены третьей силой.
Конечно же это определение войны не является одним из классических, поскольку
оно сделано лишь в контексте врага и вражды.  Состояние войны суть ситуация
онтологического вызова, ответ на который требует мыслительно-цивилизационной
мобилизации. В этом смысле национально-культурная и
идеологически-пропагандистская мобилизации суть необходимые, но весьма
примитивные и не достаточные формы мобилизации
%%%cit_comment
%%%cit_title
\citTitle{Враг}, Сергей Дацюк, analytics.hvylya.net, 18.11.2021
%%%endcit
