% vim: keymap=russian-jcukenwin
%%beginhead 
 
%%file slova.muzej
%%parent slova
 
%%url 
 
%%author 
%%author_id 
%%author_url 
 
%%tags 
%%title 
 
%%endhead 
\chapter{Музей}

%%%cit
%%%cit_pic
%%%cit_text
В Мелитополе откроют \emph{Музей} черешни.  Таким образом в город хотят
завлекать туристов.  Наверное, дело в том, что люди постоянно между Львовом,
Одессой и Мелитополем, выбирают не Мелитополь. Спросите, почему? - Там мало
\emph{музеев} и не было ни одного \emph{музея} черешни. Скорее всего так видит
ситуацию мэр города.  Уважаемые мэры городов. Хотите, чтобы в ваш город ехали
современные туристы, тогда стройте современные гостиничные комплексы, устричные
и страусиные фермы, эко-отели, парки активности и прочие современные объекты,
привлекающие людей и в которых вы сможете продать большое количество других
дополнительных услуг и тем самым развивать местный бизнес.  Что должен человек
купить в \emph{музее} черешни? Килограмм черешни? Магнитик или футболку с
изображением черешни?
%%%cit_title
\citTitle{В Мелитополе решили создать музей черешни}, Василий Апасов, strana.ua, 12.06.2021 
%%%endcit

