% vim: keymap=russian-jcukenwin
%%beginhead 
 
%%file 16_05_2021.fb.nicoj_larisa.1.stryj_kiev_ukraina_mova.cmt
%%parent 16_05_2021.fb.nicoj_larisa.1.stryj_kiev_ukraina_mova
 
%%url 
 
%%author 
%%author_id 
%%author_url 
 
%%tags 
%%title 
 
%%endhead 
\subsubsection{Коментарі}
\label{sec:16_05_2021.fb.nicoj_larisa.1.stryj_kiev_ukraina_mova.cmt}

\begin{itemize}

%%%fbauth
%%%fbauth_name
\iusr{Олена Броварська}
%%%fbauth_url
%%%fbauth_place
%%%fbauth_id
%%%fbauth_front
%%%fbauth_desc
%%%fbauth_www
%%%fbauth_pic
%%%fbauth_pic portrait
%%%fbauth_pic background
%%%fbauth_pic other
%%%fbauth_tags
%%%fbauth_pubs
%%%endfbauth
 

Слава Україні! І слава Богу, що Стрий був, є і завжди буде саме таким
націоналістичним, з нього починається кровообіг нашої нації! Немає тут суму!
Шана і честь! Це приклад для усієї України!

%%%fbauth
%%%fbauth_name
\iusr{Володимир Бердянський}
%%%fbauth_url
%%%fbauth_place
%%%fbauth_id
%%%fbauth_front
%%%fbauth_desc
%%%fbauth_www
%%%fbauth_pic
%%%fbauth_pic portrait
%%%fbauth_pic background
%%%fbauth_pic other
%%%fbauth_tags
%%%fbauth_pubs
%%%endfbauth
 

А я не вірю, що стрийскі діти не дивляться російських блогерів і не слухають
роспопсу. З усіма наслідками з цього.

%%%fbauth
%%%fbauth_name
\iusr{Victory Turyanitza}
%%%fbauth_url
%%%fbauth_place
%%%fbauth_id
%%%fbauth_front
%%%fbauth_desc
%%%fbauth_www
%%%fbauth_pic
%%%fbauth_pic portrait
%%%fbauth_pic background
%%%fbauth_pic other
%%%fbauth_tags
%%%fbauth_pubs
%%%endfbauth
 

\textbf{Володимир Бердянський} Ви чули передачі «Радіо Стрий - радіо нашого міста»?
«Наші слухачі вітають свого друга з днем народження піснею Віталія Козловського
..., та за умовами нашого радіо пісні російською мовою не звучать в його ефірі.
Тому вітаємо вашого друга іншою піснею улюбленого співака!» Це було ще 2010. І
Дзідзя вперше почула на цьому радіо!

%%%fbauth
%%%fbauth_name
\iusr{Володимир Бердянський}
%%%fbauth_url
%%%fbauth_place
%%%fbauth_id
%%%fbauth_front
%%%fbauth_desc
%%%fbauth_www
%%%fbauth_pic
%%%fbauth_pic portrait
%%%fbauth_pic background
%%%fbauth_pic other
%%%fbauth_tags
%%%fbauth_pubs
%%%endfbauth
 

\textbf{Вікторія Туряниця} тобто діти знають російську попсу не з радіо і не зі школи.
От і все. Третьосортне л-но перемагає, бо "модне". Сучасні мас-медіа
ефективніші у 1000 разів у русифікації, ніж сталінські концтабори.
Запропонувати щось натомість не можу, ідей не маю. Лише констатую.

%%%fbauth
%%%fbauth_name
\iusr{Victory Turyanitza}
%%%fbauth_url
%%%fbauth_place
%%%fbauth_id
%%%fbauth_front
%%%fbauth_desc
%%%fbauth_www
%%%fbauth_pic
%%%fbauth_pic portrait
%%%fbauth_pic background
%%%fbauth_pic other
%%%fbauth_tags
%%%fbauth_pubs
%%%endfbauth
 

\textbf{Володимир Бердянський} Так. Але не з радіо і не зі школи - і це вже неабищо! Та
інет і ТБ-помийка задньощелепні у переважній більшості. І потрібна комплексна
державна політика. А отут ми... у повному лайні... І загроза в ньому
потонути... що дуже страшно...

%%%fbauth
%%%fbauth_name
\iusr{Таїса Поцілуйко}
%%%fbauth_url
%%%fbauth_place
%%%fbauth_id
%%%fbauth_front
%%%fbauth_desc
%%%fbauth_www
%%%fbauth_pic
%%%fbauth_pic portrait
%%%fbauth_pic background
%%%fbauth_pic other
%%%fbauth_tags
%%%fbauth_pubs
%%%endfbauth
 

Які ж молодці там! Якби ж по всій Україні отак... Мрії вголос...

%%%fbauth
%%%fbauth_name
\iusr{Лиска Лілія}
%%%fbauth_url
%%%fbauth_place
%%%fbauth_id
%%%fbauth_front
%%%fbauth_desc
%%%fbauth_www
%%%fbauth_pic
%%%fbauth_pic portrait
%%%fbauth_pic background
%%%fbauth_pic other
%%%fbauth_tags
%%%fbauth_pubs
%%%endfbauth
 

Дякую пані Ларисо, що пишете нам такі бомбезні розповіді!!! Дуже вражена,бо
чомусь думала,що Стрий,звичайне містечко,як всі інші.Але ж молодці мешканці та
влада їхня,що так прекрасно зберегли своє Українське місто!!! Є з кого приклад
брати!!! Як каже мій чоловік:"о,то не все пропало!" І не пропаде! Україна
жила,жиє і буде жити вічно! І,оці стрийські школярики,приклад для українських
діточок!!!

%%%fbauth
%%%fbauth_name
\iusr{Viktor Marchenko}
%%%fbauth_url
%%%fbauth_place
%%%fbauth_id
%%%fbauth_front
%%%fbauth_desc
%%%fbauth_www
%%%fbauth_pic
%%%fbauth_pic portrait
%%%fbauth_pic background
%%%fbauth_pic other
%%%fbauth_tags
%%%fbauth_pubs
%%%endfbauth
 

Браво, пані Ларисо! Давно нічого подібного не читав. Гордий з того, що в нас в
Україні є хоч Стрий. Молодці!

%%%fbauth
%%%fbauth_name
\iusr{Nina Popel}
%%%fbauth_url
%%%fbauth_place
%%%fbauth_id
%%%fbauth_front
%%%fbauth_desc
%%%fbauth_www
%%%fbauth_pic
%%%fbauth_pic portrait
%%%fbauth_pic background
%%%fbauth_pic other
%%%fbauth_tags
%%%fbauth_pubs
%%%endfbauth
 

Як завжди, велика подяка вам пані Лариса! Ваші дописи виховують і дають
насолоду від чудового слова! Наснаги вам і СИЛЬНОГО ДУХУ!

%%%fbauth
%%%fbauth_name
\iusr{Степан Шніцар}
%%%fbauth_url
%%%fbauth_place
%%%fbauth_id
%%%fbauth_front
%%%fbauth_desc
%%%fbauth_www
%%%fbauth_pic
%%%fbauth_pic portrait
%%%fbauth_pic background
%%%fbauth_pic other
%%%fbauth_tags
%%%fbauth_pubs
%%%endfbauth
 

А Ви ще завітайте в Комарно. З комарнівських ткачів походить родина Коновальця.
Завітайте в Малу академію мистецтв, в місцеві церкви. Комарно - перше і єдине
місто, якому вдалося цілком легально здійснити перепоховання енкаведистського
цвинтаря з центру міста на старий польський. Потім комуняки внесли зміни в
закон і ця практика не набула поширення.

%%%fbauth
%%%fbauth_name
\iusr{Ntina Ntoubrova}
%%%fbauth_url
%%%fbauth_place
%%%fbauth_id
%%%fbauth_front
%%%fbauth_desc
%%%fbauth_www
%%%fbauth_pic
%%%fbauth_pic portrait
%%%fbauth_pic background
%%%fbauth_pic other
%%%fbauth_tags
%%%fbauth_pubs
%%%endfbauth
 

Залишилося в Стрию виховати талановитих україномовних макроекономістів та
фінансистів, інженерів та топ-менеджерів - і тоді дійсно щось в країні
зміниться. Бо знаннями народних пісень та православними молитвами країну не
нагодуєш і товари з високою додатковою вартістю не створиш.

А от наші західні українці, отримуючи класну економічну, інженерну, Ай-Ті,
управлінську та фінансову й політологічну освіту в Львівських вишах, чомусь
їдуть піднімати економіку будь-якої іншої країни (росії зокрема)... І їм все
класно вдається.. Там вони, засукавши рукави та показавши свої 32 гострі зуби,
не бояться з самого низу, з рівня імігранта (за замовчуванням - людини НЕ
першого сорту) підніматися чи не до головних міністерських посад. А в нашій
країні чомусь безпомічно пасують свій м,яч - безмізковим нахабним нащадкам та
викормишам комуняк та "братків". І де ота войовничість дівається?

%%%fbauth
%%%fbauth_name
\iusr{Василь Коружак}
%%%fbauth_url
%%%fbauth_place
%%%fbauth_id
%%%fbauth_front
%%%fbauth_desc
%%%fbauth_www
%%%fbauth_pic
%%%fbauth_pic portrait
%%%fbauth_pic background
%%%fbauth_pic other
%%%fbauth_tags
%%%fbauth_pubs
%%%endfbauth
 

Стрий - найкраще місто і це є правда.

Інакше можуть стверджувати лише люди, які не знають життя поза кордонами
Галичини та й в інших наших містах.

%%%fbauth
%%%fbauth_name
\iusr{Надія Собко-Федорів}
%%%fbauth_url
%%%fbauth_place
%%%fbauth_id
%%%fbauth_front
%%%fbauth_desc
%%%fbauth_www
%%%fbauth_pic
%%%fbauth_pic portrait
%%%fbauth_pic background
%%%fbauth_pic other
%%%fbauth_tags
%%%fbauth_pubs
%%%endfbauth
 

Моя дочка проживає у Києві і викладає у школі. Не правда що у Києві такі
русофоби. Різні люди і відношення різне до мови до національних символів. Хоча
є люди провокатори. Але я рада що Україна відновлюється оживає і хоч малими
кроками з бідами війною ми вже відзначаємо 30років незалежності України. І
головне Любити і поважати хоч яка не є але наша Україна.

%%%fbauth
%%%fbauth_name
\iusr{Daniel Los}
%%%fbauth_url
%%%fbauth_place
%%%fbauth_id
%%%fbauth_front
%%%fbauth_desc
%%%fbauth_www
%%%fbauth_pic
%%%fbauth_pic portrait
%%%fbauth_pic background
%%%fbauth_pic other
%%%fbauth_tags
%%%fbauth_pubs
%%%endfbauth
 

Илон Маск основывает первую колонию на Марсе, исследователи СRISPER
приблизились к излечению рака, а в Стрыю ходят в вышиванках, бьют в бубен и
поклоняются лысому дядьке. Ужас. Как хорошо что нам не по пути.

%%%fbauth
%%%fbauth_name
\iusr{Олексій Устяк}
%%%fbauth_url
%%%fbauth_place
%%%fbauth_id
%%%fbauth_front
%%%fbauth_desc
%%%fbauth_www
%%%fbauth_pic
%%%fbauth_pic portrait
%%%fbauth_pic background
%%%fbauth_pic other
%%%fbauth_tags
%%%fbauth_pubs
%%%endfbauth
 

Читав уважно текст , виявив декілька російських слів , які дуже мулять око : -
говорить ( розмовляти ), прекрасно ( чудово) , робочий день ( день праці ,
граждан ( громадян )...  Якого милого в Державній школі находяться класи
християнської етики ,чому так нагло порушується Конституція України , яка
наголошує на тому , що церква відокремлена від держави і не допустимо надавати
преференції одній релігійній організації у збиток іншим !!!!  Прошу свою
корупційну релігію вивчати в приватних закладах , а не вдержавному навчальному
закладі !!!  Ще одне питання ,а за кого голосували ваші священники в першому
турі президенських виборів , часом не за промосковських рошенівських регіоналів
,які рясно засівають усі церкви корупційним баблом вже більше 15 років ??!

%%%fbauth
%%%fbauth_name
\iusr{Наталія Катерняк Трух}
%%%fbauth_url
%%%fbauth_place
%%%fbauth_id
%%%fbauth_front
%%%fbauth_desc
%%%fbauth_www
%%%fbauth_pic
%%%fbauth_pic portrait
%%%fbauth_pic background
%%%fbauth_pic other
%%%fbauth_tags
%%%fbauth_pubs
%%%endfbauth
 

ДЯКУЮ Вам, пані Ларисо, за такий влучний, щемливий і правдивий допис про наш
Стрий. Так, ми такі, і як би тут тролики і ватніки не намагались спаплюжити і
принизити наш край, ми і надалі продовжуємо жити з нашою ПРАВДОЮ !!! Творимо,
працюємо, мислимо з нашим баченням життя,з нашим знанням історії , дуже часто
не з історичних книжок чи досліджень, а з розповідей наших бабусь і дідусів, це
передається з покоління в покоління.

І так, стрияни - унікальні!!! Нас знищувавали у тюрмах, вивозили на Сибір,
щедро годували совковими "цінностями", але Ми вистояли і Ми є, хоч і як би це
не дратувало московитів.

Слава Україні!!!

%%%fbauth
%%%fbauth_name
\iusr{Микола Степаненко}
%%%fbauth_url
%%%fbauth_place
%%%fbauth_id
%%%fbauth_front
%%%fbauth_desc
%%%fbauth_www
%%%fbauth_pic
%%%fbauth_pic portrait
%%%fbauth_pic background
%%%fbauth_pic other
%%%fbauth_tags
%%%fbauth_pubs
%%%endfbauth
 

Стрий без сумніву славне місто. Та я знав Стрий коли української мови почути
було там зась. Лише коли Мирон Лисишин з друзями добре випивали то лунали
українські пісні і мова. Та хвала стрийчанам що вони змогли так поставити це
питання і українська мова стала звучати всюди. Нажаль Стрий під московським
окупантом був лише 78 років а моя Кіровоградщині більше трьохсот років, де
навіть московський матюк вкорінився так що не викорчуєш.

%%%fbauth
%%%fbauth_name
\iusr{Олег Щербан}
%%%fbauth_url
%%%fbauth_place
%%%fbauth_id
%%%fbauth_front
%%%fbauth_desc
%%%fbauth_www
%%%fbauth_pic
%%%fbauth_pic portrait
%%%fbauth_pic background
%%%fbauth_pic other
%%%fbauth_tags
%%%fbauth_pubs
%%%endfbauth
 

Я теж дитиною не знав, що у Києві розмовляють російською, бо в той період
телебачення було українськомовним. І я з Стрийщини якщо що)

\end{itemize}

