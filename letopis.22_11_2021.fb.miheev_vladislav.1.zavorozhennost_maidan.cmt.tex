% vim: keymap=russian-jcukenwin
%%beginhead 
 
%%file 22_11_2021.fb.miheev_vladislav.1.zavorozhennost_maidan.cmt
%%parent 22_11_2021.fb.miheev_vladislav.1.zavorozhennost_maidan
 
%%url 
 
%%author_id 
%%date 
 
%%tags 
%%title 
 
%%endhead 
\zzSecCmt

\begin{itemize} % {
\iusr{Влад Войцеховский}

Суть поста: человечишко - знай свое место, маленькое говно! Шустер и Гордон
тебе в помощь...

\begin{itemize} % {
\iusr{Владислав Михеев}
\textbf{Влад Войцеховский} суть поста в буддийской истине: сердце мудрого человека есть деяние без борьбы! )
Хотя маленькое говно, согласен, должно знать свое место. Хотя бы потому, что место повыше занимает, как правило, большое говно)

\iusr{Влад Войцеховский}
\textbf{Vladislav Mikheev} Жизнь не буддийский монастірь. И как ты быстро разделил людей на два сорта говна) Это не по-буддистски)

\iusr{Владислав Михеев}
\textbf{Влад Войцеховский}
Я просто перешёл на твою терминологии) нас Будда учил, что можно разными путями достить просветления)

\iusr{Влад Войцеховский}
\textbf{Vladislav Mikheev} Слава Будде, что я на твою терминологию не переходил. Так и оскотиниться недолго - или по-твоему: достичь просветления)

\iusr{Владислав Михеев}
\textbf{Влад Войцеховский} вот и не надо, должен же кто-то оставаться гностическим светочем среди оскотинившегося мира )

\iusr{Валентина Гаврилова}
\textbf{Влад Войцеховский} 

Слепой не увидит, глухой не услышит... Каждый воспринимает поданную информацию
соответственно своей нозологии, т. е. патологии (глухой/слепой и т. п.). И это
тоже имеет право на существование... Увы...

\iusr{Влад Войцеховский}
\textbf{Валюша Гаврилова} 

Тобто вы оставляете каждому право воспринимать информацию только соответственно
нозологии/патологии? Воспринимать информацию через интеллект, опыт, чувства -
это вы, стало быть, отвергаете?) Так у кого тут нозология/патология?

\iusr{Валентина Гаврилова}
\textbf{Влад Войцеховский} 

Нет. Вы снова всё гипертрофировали, позволив своим эмоциям, вычленить не суть,
а \enquote{найти виноватых} якобы в чем-то Вас обвинить или унизить...Ну Вы же сами
обозначили утвердительно/осудили (что не позволительно, на мой взгляд) цитирую:
\enquote{Суть поста: человечишко - знай свое место, маленькое говно!} Что это, если не
провокация/манипуляция и ...Далее я свое мнение высказала. Имею право на СВОЁ
мнение! Т.к. в посте Владислав Михеев ничего о \enquote{человечешках} и \enquote{говне} я
лично не увидела. А о нозологии/патологии-это профессиональное считывание... Не
эмоционируйте. Вредно.

\iusr{Влад Войцеховский}
\textbf{Валюша Гаврилова} 

Умение закрывать глаза на очевидное и пытаться закрыть их другим - это
неоценимо!) Искренне желаю вам жить среди таких, как вы

\iusr{Валентина Гаврилова}
\textbf{Влад Войцеховский} 

Да это Ваше \enquote{очевидное} !!! Что ж Вы навязываете то, что видите Вы и только
Вы???!!! Ваша правда и зрячесть - это только ваше \enquote{зрение} ЛИЧНОЕ!!! Поэтому
меня и возмутило ваше вколачивание \enquote{человечишки } и \enquote{говно}. Оставьте себе
Ваше личное жизнеощущуние!!! Это и есть \enquote{нозология/патология} на моём
профессиональном языке.

\iusr{Влад Войцеховский}
\textbf{Валюша Гаврилова} 

Не стоит так нервничать) Понятия не имею о вашем \enquote{профессиональном языке}, но
на моем языке профессионального медийщика пост Влада - это сплошь подмена
понятий, игра слов и манипуляции. Как и ваши комменты. Только вы пишете не так
искусно, как Михеев и, в отличие от сабжа, похоже, сами искренне верите в
написанное)

\iusr{Валентина Гаврилова}
Нужно слушать, чтобы понять, а не слушать, чтобы ответить... Вот в чём различие между нами...
\end{itemize} % }

\iusr{Евгений Вагнер}
Ну слава Богу.... Ожил...
А то я думал: ну как же Владислав \enquote{день гидности} пропустит..)))

\begin{itemize} % {
\iusr{Олег Климов}
\textbf{Евгений Вагнер} сам-то не забыл)))

\iusr{Евгений Вагнер}
\textbf{Олег Климов} та не... я такое на всю жизнь запомнил...

\iusr{Олег Климов}
\textbf{Евгений Вагнер} Ты не один)
\end{itemize} % }

\iusr{Станислав Шпиль}

А вот я, если и завороженный, то не майданом - этим аутодафе испанского стыда,
а беспрецендентной ковидной глобализьмой... До сих пор под впечатлением от
масштабов операции... Особенно умиляет, что для наших сервильных хуторянских
экспердов, такой темы и проблемы вообще не существует...

Такие лапусики... Все о своём... О национальном. О зе.. И конечно о ненавистном
пу...

\iusr{Baolisy Yalishanta}
Далеко ли ты от песчинки? Фразы грамотно строить, возможно тоже революция, но не та, Vuladico)

\iusr{Сергей Безгин}
То торнадо, а то убитые люди

\begin{itemize} % {
\iusr{Владислав Михеев}
\textbf{Сергей Безгин} именно поэтому использована эта метафора - ураганы, землетрясения и цунами убивают людей

\iusr{Сергей Безгин}
\textbf{Владислав Михеев} совершенно разные вещи - стихия, в которой гибнут люди, и спланированные насильственные действия с целью погубить

\iusr{Владислав Михеев}
\textbf{Сергей Безгин} 

диалектически это все очень условно. Более того социальные и экономические
явления носят иногда стихийных, а иногда и циклический характер. Участвует в
них и человек своей проектной, плановой деятельностью. О степени это участия
можно споорить, но это именно степень. Классическая диалектика случайности и
необходимости и тд и тп.

\iusr{Сергей Безгин}
\textbf{Владислав Михеев} 

для меня это не тонкая, а толстая и ясная красная линия- планирование нанесения
человеку ран, увечий-это сознательное нарушение не только этических, но и
социальных законов, ведущее к разрушению предыдущей жизни. При этом я не
пацифист, я, скорее, миротворец, и хочу, чтобы и внуки пожили в мире и теплоте


\iusr{Сергей Безгин}
А Майдан и был демонстрацией - мы перешли все границы и забыли все законы. И
понеслось. Не только на Украине понеслось

\iusr{Владислав Михеев}
\textbf{Сергей Безгин} с этим-то я как раз согласен, но могут ли лемминги себя контролировать, когда несутся в пропасть? )

\iusr{Сергей Безгин}
\textbf{Владислав Михеев} их кто-то направил. Найти и - Нюрнберг. С неизбежностью. Иначе нам к миру не вернуться
\end{itemize} % }

\iusr{Владислав Вареник}
Напишіть щось про «зачарованість Холокостом, «зачарованість Голодомором»...

\iusr{Полина Дьяченко}

Маленького человека нужно направить в созидательное русло. Историей доказано,
что сам он не способен самонаправиться. А у нас тут проблема - некому направить.
Созидателей наверху нет, только толпа разрушителей.


\iusr{Владимир Иконников}
Именно так.

\iusr{Олег Климов}
Какие интересные комментарии. Сразу видно, кто по какую сторону бури был и кто выводы сделал или нет. Прямо лакмус.

\begin{itemize} % {
\iusr{Елена Скачко}
\textbf{Олег Климов} А зачем лакмус? Никто вроде и не скрывает

\iusr{Олег Климов}
\textbf{Елена Скачко} Может, Влад проводит контрольную \enquote{перепись} населения в процессе пандемии. Вдруг на кого-то укольчики повлияли как-то необычно)))
\end{itemize} % }

\iusr{Дмитрий Левченко}
Всё бы хорошо бы.. и текст прекрасен, но кто то должен ответить за всё это

\iusr{Елена Скачко}

Как дипломатично ты однако)) Не, я пас. Никакой дипломатии. Жуткие годы,
которые расхлёбывать ещё долго. И расхлебать ли?

\begin{itemize} % {
\iusr{Владислав Михеев}
\textbf{Елена Скачко} да, это большая
коллективная психотравмма. А лечить её как? Мифологизация Майдана - это рубец на ране. Она затянулась, но не исчезла и болеть не перестала. На мой взгляд выхода два: не ковырять её вообще, забыть как о страшном сне. И выход второй : деконструкция мифа.

\iusr{Дмитрий Коломийченко}
\textbf{Владислав Михеев} Перекрывается другой травмой.

\iusr{Владислав Михеев}
\textbf{Дмитрий Коломийченко} тоже вариант

\iusr{Елена Скачко}
\textbf{Владислав Михеев} ну, наша официальная политика этот рубец постоянно поливает и удобряет)) из него уже баобаб вырос)
\end{itemize} % }

\iusr{Александр Опанасенко}

Горе от \enquote{ума}.)) Разочаровал. Во все времена увлечения подобными модними
\enquote{соплями} заканчивались одинаково грустно и печально: или пеньком или маузером
или пером в подворотне. (Я уже не говорю о Голгофе))).

Пока кто то совершенствуется в хитросплетениях умозаключений, мир (а это
основная масса) продолжает оставаться в статике, т.е. таким же, каким был
тысячи, миллионы лет.

\begin{itemize} % {
\iusr{Владислав Михеев}
\textbf{Александр Опанасенко} а что обещал кого-то очаровать и нарушил обещание? ))

\iusr{Александр Опанасенко}
\textbf{Владислав Михеев} 

да нет, простите за некоторую резкость, я вообще противник очаровываться, легче
жить.) Нередко Вас читаю, во многом мы единомышленники. Просто именно этот Ваш
пост мне показался как то уж слишком пространственным. На мой \enquote{пересічний}
бытиё до безобразия банальней. И жестоко. Как только расслабишся, так тут же и
вжарят).

\end{itemize} % }

\iusr{Олексій Романов}
секта

\iusr{Stanislav L. Kovtun}

Ого! Ничего себе. Дискуссия однако.

Мона я скажу за свой лайк?

Спасибо.

В мировом масштабе \enquote{Майдан-2014} это, скорее всего последняя
\enquote{революция постмодерна} как явление начавшаяся в 60-х.

\enquote{Общество спектакля} и \enquote{Человек-масса} с его восстанием ушли в
прошлое и так называемая \enquote{пандемия} доконала их.

Свято место заполняет \enquote{Общество сетей} и \enquote{цифровой человек}.

И человек в массе таких же как он - по общности целей (путь и эфемерных)
уступает человеку для лайков и количества просмотров и подписчиков.

И мир площади человеку заменит мир тик-тока.

В этом случае его значительность и сопричастность будет выражена в подписке
любимцев и хейте противников.

\iusr{Максим Ряпулов}

судження ваші північно-східним менталітетом тхнуть.

бунт...

малєнькій чєловєк...

найменша людина на Майдані - більша за овоча.

отам був той ваш літературно класичний \enquote{малєнькій чєловєк}.

випадкові люди на чолі ДніпроОДА зимою 2013-2014 року - теж ті самі малєнькіє.

чи у вашій логіці, той, хто має багато бабла, той і велика людина?

тоді всі ваші судження - такі ж \enquote{вєлікіє}, як і олегоцарьов і янукович, і
азіров, хто там ще був великим, кого маленькі люди вигнали нафіг?

знаковий пост для вас, ящетаю. вино чомусь виявилося оцетом.

пічалька.

\begin{itemize} % {
\iusr{Владислав Михеев}
\textbf{Максим Ряпулов} 

а Янукович великим никогда и не был. Его в ранг \enquote{великого зла} возвели на
Майдане.

И деньги-то то тут причём, да еще и в моей логике? Где это у меня была такая
логика?

Перечитайте то, что вы написали внимательно: я именно это и написал, почти
вашими же словами - любой маленький человек на Майдан чувствовал себя
сверхчеловеком.

\iusr{Максим Ряпулов}
\textbf{Vladislav Mikheev}, 

а як ви тоді визначили, що Майдан - це був бунт \enquote{малєнького чєловєка}?

на підставі чого ви зробили цей висновок?

я, наприклад, стверджую, що \enquote{малєнькій чєловєк} під час Майдану сидів
вдома, бо він не спроможний на таку дію, як Майдан.

а там були Люди. і жоден з них не був Акакієм Акакійовичем.

а от азіров, яник, царьов та інша пошесть - оті самі гоголівські сморчки.

а друге, що я стверджую, що то не був бунт. був свідомий протест проти дій
керівництва країни.

акакійовичи на протест ще більше не спроможні, ніж на бунт.

\iusr{Владислав Михеев}
\textbf{Максим Ряпулов} 

здесь вы опять говорите ровно то же, что и я ) просто на удивление этого не
замечаете ) я с вами полностью согласен и в посте это превращение в \enquote{воинов
света} описал, и о том, что скучные и жалкие людишки по домам сидели, в огорода
ковырялись тоже написал) а вот о том, что это чисто гностическая
мировоззренческая установка, я думаю, вы, как образованный человек и без меня
понимате)

Да, и кстати, Путина, Гоголя, Януковича, российских нарративов и нас с вами во
времена гностиков еще не было, а установки уже были ) конечно, только
избранные, только титаны духа могут подняться на бунт против лежащего во зле
мира! ) уже больше 2000 лет установке этой! Конечно, в компании Прометея,
Люцифера и Каина намного приятней, чем в компании Акакия Акакиевича, кто б
спорил? )

\iusr{Андрей Бока}
\textbf{Максим Ряпулов} 

як ви тоді визначили, що Майдан - це був бунт \enquote{малєнького чєловєка}?

===================

По отсутствию конструктивной программы.

\enquote{За всё хорошее против всего плохого!}

\enquote{Выгоним плохих и назначим хороших!}

\enquote{Весь мир за нас!}

Это лозунги очень маленького мышления.

Социальная почва для Майдана вызрела из-за накопленных в обществе противоречий
политических, идеологических, социальных, экономических.

Однако не ставились цели разрешить эти противоречия или хотя бы их озвучить для
дальнейшего поиска обществом путей их разрешения.

Напротив, кураторы Майдана намерено обостряли многие противоречия для подогрева
страстей.

\iusr{Максим Ряпулов}
\textbf{Andrey Boka}, 

ви впевнені, що оце - дійсно цілі тих людей? чи може це ваша ретрансляція?

наприклад, Антігона хотіла поховати брата, згідно зі звичаями, а для Креона -
то був бунт маленької дівчинки проти виданого закону.

так, ви впевнені, що вірноретранслюєте події?

\iusr{Андрей Бока}
\textbf{Максим Ряпулов} 

Ви впевнені, що оце - дійсно цілі тих людей?

Чи може це ваша ретрансляція?

===============

Начало: \enquote{Требуем подписать Ассоциацию с ЕС!}

Финал: \enquote{Долой Януковича!}

Покажите кто и где озвучивал системные требования - может я упустил.

Требование вернуть в госсосбвтенность украденную у народа в 90-х собственность.

Требование жёсткого контроля естественных монополий.

Требование изменить принципы распределения добавочной стоимости, создаваемой
экономикой.

Требование внедрить финляндизацию внешнеполитической доктрины.

Требование начать общественный диалог по гуманитарным вопросам.

И многие-многие-многие другие фундаментальные вопросы.

\iusr{Максим Ряпулов}
\textbf{Andrey Boka}, 

тобто, ви вважаєте, що оці фундаментальні (солідарний з вами, що такими вони і
є) вимоги яник послухав би і сказав: ок, я розумію, що це важливі речі, і я їх
виконаю, а от коли б вимагали асоціацію з ЄС, тоді б я наказав беркуту фігачити
цей бунт.

чи як?

як на мене, \enquote{яника геть} - то була зрозуміла кожному митингарю і кожному ригу
вимога, щоб потім була можливість поробити перелічені вами фундаментальні
штуки.

єдине, про що з вами не згодний, то це фінляндизація.

в нас ще за Кучми з островом Тузла було зрозуміло, що за такого сусіда воно не
працює.

\end{itemize} % }

\iusr{Irina Gladuniak}

а войну и потерю территорий куда вынести? разве они не связаны как процессы,
где одного нет без другого?. торнадо? наверное. главное, чтоб не смерч.

\iusr{Игорь Соседов}
время все лечит!

\iusr{Анатолий Опалатенко}

\obeycr
Я так понимаю это тезисный пересказ монолога Гамлета... 
Мириться лучше со знакомым злом,
Чем бегством к незнакомому стремиться!
Так всех нас в трусов превращает мысль,
И вянет, как цветок, решимость наша
В бесплодье умственного тупика,
Так погибают замыслы с размахом,
В начале обещавшие успех,
От долгих отлагательств.
\restorecr

\iusr{Игорь Потысьев}
\textbf{Владислав Михеев}, разве Гаврилюк, Булатов, Лещенко, Найем ... не стали богаче?

\begin{itemize} % {
\iusr{Владислав Михеев}
\textbf{Игорь Потысьев} 

в теории элит это на совсем народ ) у Парето народ определен как неэлита. А
вами перечисленные это антиэлита. Почему бы при смене элиты на контрэлиту
обслуживающему эту смену персоналу не стать богаче?

Помните в \enquote{Убить дракона}: это не народ, это лучше народа.... ))

\end{itemize} % }

\iusr{German Viktorov}

\enquote{Меняться самому - опять не вариант: это требует рефлексии, больших
энергозатрат и времени, а кроме того больно бьёт по тщеславию} Так что у
украинцев всё традиционно-стабильно и всё и так \enquote{за скобками}, на самом то
деле, тем более \enquote{...реальное проектное мышление и реальная картина мира это
реальная скукотища! То ли дело гностический миф об избранничестве, магия толпы
и постреволюционные каргокульты.} Тем и живём.

\iusr{Ольга Терещенко}

Людей губят иллюзии, мы верили, мы боролись, а спросите каждого из участников
за что?

И вот момент для манипуляторов, они то знают за что.

\iusr{Сергей Юшин}

Ленин утверждал, что «всякое (!!!) государство несвободно и ненародно». И цели
у него были республиканскими, пока он шел к власти, а получив власть, он
произнес: «государство – это мы, ... коммунисты». Вот здесь и образовался раскол
между мечтой и реальностью, который спустя 70 лет развалил СССР. Интересна
аналогия, упомянутая Ж.Энкосом (Папюсом) в Каббале: за 32 века до Р.Х.
(священные книги браманов), принц Иршу, крайне честолюбивый и лишенный
перспектив достигнуть верховной власти законно и легально, организовал
революцию с ложно-революционными намерениями до момента овладения военной
властью, чтобы вернуть цезаризм во имя бесконечной власти: личной,
монархической, олигархической; право древних подменил закон с характером
общественной анархии; был установлен синтез наук и искусств, а белый цвет
заменен на красный. Так что прав мудрец: «нет ничего нового под солнцем» [Еккл.
1: 9].

\iusr{Сергей Юшин}

\obeycr
Павло Тичина – На майдані коло церкви...
На майдані коло церкви
революція іде.
— Хай чабан! — усі гукнули, —
за отамана буде.
Прощавайте, ждіте волі, —
гей, на коні, всі у путь!
Закипіло, зашуміло —
тільки прапори цвітуть...
На майдані коло церкви
посмутились матері:
та світи ж ти їм дорогу,
ясен місяць угорі!
На майдані пил спадає,
Замовкає річ...
Вечір.
Ніч.
1919
\restorecr

\iusr{Виктор Шкуро}

Забастовки, марши, пикетирования, стачки и другие майданы - это
распространенные и даже банальные инструменты диалога групп исполнителей
(трудящихся) с администрацией систем в прозападном демократическом мире по
достижению компромисса по разделению заработанного и достижению паритета
интересов. Для трудящихся Запада участие в стачке есть не бездумный захват
торнадо и даже не стихийное проявление отчаяния - а конкретная форма работы на
себя - организованное сознательное участие в переговорном процессе между трудом
и капиталом (и вряд ли отдельный участник протеста считает (ощущает) себя
песчинкой в разгулявшейся стихии). На пространствах бывшего Советского Союза
так и не сформировалась культура профессиональных и уличных протестов и поэтому
Майдан и народу и элитам показался как некий феномен, тогда как такое
взаимодействие - обычное для систем с большим количеством людей. Феномен нашего
Майдана в том, что власть не пошла на переговорный процесс с протестующими
(опять же в силу отсутствия культуры диалога через протесты), тогда как в
западном мире переговоры и достижение компромисса являются неотъемлемой частью
и финалом протеста. Именно власть, а не участники Майдана превратили эту форму
диалога в стихийное бессмысленное торнадо.

\iusr{Yartsev Anatoliy}
Довольно забавно, когда шокоолигарха сделали вождём переворота кланового.

\iusr{Sergey Kuchinsky}

Можно ли было \enquote{вынести за скобки} переворот Салазара в Португалии или переворот
Франко в Испании и \enquote{просто жить}? В математике \enquote{вынесенный за скобки} множитель
определяет масштаб результатов всего выражения и их позитивную либо негативную
природу - их экономическую и социальную прибыльность или убыточность.(имхо)


\end{itemize} % }
