% vim: keymap=russian-jcukenwin
%%beginhead 
 
%%file 23_11_2021.stz.news.ua.misto_marii.1.mini_proekty_plany
%%parent 23_11_2021
 
%%url https://mistomariupol.com.ua/uk/yaki-mini-proyekty-planuyut-realizuvaty-u-mariupoli-do-nastupnogo-turystychnogo-sezonu
 
%%author_id news.ua.misto_marii
%%date 
 
%%tags 
%%title Які міні-проєкти планують реалізувати у Маріуполі до наступного туристичного сезону?
 
%%endhead 
 
%\subsection{Які міні-проєкти планують реалізувати у Маріуполі до наступного туристичного сезону?}
\subsection{Міні-проєкти}
\label{sec:23_11_2021.stz.news.ua.misto_marii.1.mini_proekty_plany}
 
\Purl{https://mistomariupol.com.ua/uk/yaki-mini-proyekty-planuyut-realizuvaty-u-mariupoli-do-nastupnogo-turystychnogo-sezonu}
\ifcmt
 author_begin
   author_id news.ua.misto_marii
 author_end
\fi

\ifcmt
  ig https://i2.paste.pics/PI0NI.png?trs=1142e84a8812893e619f828af22a1d084584f26ffb97dd2bb11c85495ee994c5
  @wrap center
  @width 0.7
\fi

\begin{quote}
\em
Нова серія міні-скульптур, галерея просто неба, історичний фестиваль
реконструкції, реставрація дверей та інші цікаві ініціативи. Які міні-проєкти
планують реалізувати у Маріуполі до наступного туристичного сезону?	
\end{quote}

22 листопада було проведено фінальне засідання конкурсної комісії Програми
міні-грантів сприяння розвитку культури, туризму, спорту та молодіжної політики
щодо проєктів на 2022 рік. Загалом за напрямом \enquote{туризм} було опрацьовано 8
грантових заявок, з яких комісія відібрала 7 проєктів-переможців для подальшої
реалізації.

Метою кожного з проєктів є розвиток туристичної привабливості міста, створення
нових туристичних магнітів та популяризація історико-культурного спадку міста
та регіону. Проєкти цього року досить різнопланові: від створення мобільного
додатку до проведення освітньої програми для гідів.

\subsubsection{\enquote{Життя дверей має значення}}

Проєкт полягає у реставрації дверей будинків, що знаходяться в історичній
частині міста. Автор Ярослав Федоровський зазначає, що реалізація ідеї
дозволить створити 10 нових туристичних точок, які будуть привертати увагу
мешканців міста та його гостей до  збереження історичної забудівлі Маріуполя.

\ii{23_11_2021.stz.news.ua.misto_marii.1.mini_proekty_plany.pic.1}

\subsubsection{Історичний мікро-фестиваль \enquote{Битва на Калці}}

Дводенна подія, присвячена роковині битви князівської коаліції Київської
держави з монголо-татарами на річці Калка у 1223 році. Проєкт передбачає
проведення турніру з середньовічного бою, створення міського туристичного
об'єкту для ознайомлення зі стародавньою історією України та Приазов'я, як
одного з її регіонів.

\ii{23_11_2021.stz.news.ua.misto_marii.1.mini_proekty_plany.pic.2}

\subsubsection{Англомовний табір \enquote{Mariupol Guide Camp 2022}}

Мета проєкту – підготовка кваліфікованих англомовних гідів для розвитку
екскурсійної індустрії Маріуполя. Для цього буде розроблена авторська освітня
програма навчання та практичний курс роботи над англомовними екскурсіями для
гостей міста.

\ii{23_11_2021.stz.news.ua.misto_marii.1.mini_proekty_plany.pic.3}

\subsubsection{Мобільний додаток \enquote{Mariupol Travel}}

Проєкт, що передбачає розробку туристичного універсального інструменту для
орієнтування в місті, пошуку інформації та придбання послуг за програмою
лояльності, який буде сприяти підтримці місцевого бізнесу. Застосунок надасть
можливість аналізу туристичного попиту на окремі види інформації, налагодження
партнерських відносин із зацікавленими закладами харчування, культури,
гостинності та розваг.

\ii{23_11_2021.stz.news.ua.misto_marii.1.mini_proekty_plany.pic.4}

\subsubsection{Створення тематичної серії міні-скульптур \enquote{Гастрономічні цікавинки Маріуполя}}

Виготовлення та встановлення трьох міні-скульптур, присвячених тематиці
гастрономічної спадщини регіону на популярних туристичних локаціях міста. У
попередній пропозиції було запропоновано такі варіанти страв: чир-чир (або
більш популярна назва – маріупольський чебурек), псалтир (обрядовий хліб
приазовських греків) та борщ з кількою. Остаточні варіанти будуть визначені
шляхом відкритого голосування.

\ii{23_11_2021.stz.news.ua.misto_marii.1.mini_proekty_plany.pic.5}

\subsubsection{OPEN AIR KUINJI GALLERY}

Проект переслідує мету популяризації творчості та особистості видатного
маріупольського живописця Архипа Куїнджі, шляхом створення нової туристичної
локації просто неба. Практична реалізація передбачає створення на фасаді
будинку однієї з головних вулиць відкритої галереї міста репродукцій
найвідоміших картин Архипа Івановича Куїнджі.

\ii{23_11_2021.stz.news.ua.misto_marii.1.mini_proekty_plany.pic.6}

\subsubsection{Створення професійного довідника та промороликів кінолокацій Маріуполя}

Даний проєкт передбачає дослідження, створення інформаційної бази та каталогу
локацій для кінозйомки, промоцію потенціалу Маріуполя в Україні та за її
кордонами. Метою проєкту є ознайомлення потенційних українських та закордонних
партнерів з потенціалом Маріуполя, як одного з найконтрастніших та
найколоритніших міст України, цікавих для кіноіндустрії.

\ii{23_11_2021.stz.news.ua.misto_marii.1.mini_proekty_plany.pic.7}
