% vim: keymap=russian-jcukenwin
%%beginhead 
 
%%file slova.pobeda
%%parent slova
 
%%url 
 
%%author 
%%author_id 
%%author_url 
 
%%tags 
%%title 
 
%%endhead 
\chapter{Победа}
\label{sec:slova.pobeda}

%%%cit
%%%cit_pic
%%%cit_text
Борьба за \emph{победу} действительно была острой. И противник - сборная Нидерландов -
на последних минутах забил гол и сорвал уже казалось бы железобетонную ничью.
В итоге Украина потерпела поражение, но можно убедительно сказать, что почетное
- со счетом 3:2. Чему, кстати, порадовались даже в России - показав, что многие
болельщики не повелись на попытку раздуть политическое противостояния вокруг
чемпионата. В Украине тоже многие хвалили сборную за волю к \emph{победе}. Но
политику в обсуждения внес внезапно сам Владимир Зеленский, который зачем-то
вспомнил о скандале с формой украинской сборной, поздравляя ее с игрой.
Интересно, что за украинскую сборную болели и многие россияне. В том числе и
комментаторы на госканалах. И это резко контрастирует с тем, как в Украине
некоторые восприняли недавнее поражение российской сборной от Бельгии.
\enquote{Страна} публикует отзывы на первый матч Украины в еврочемпионате
%%%cit_comment
%%%cit_title
\citTitle{Украина - Нидерланды на Евро 2020. Отзывы на матч в Украине и России}, 
Оксана Малахова, strana.ua, 14.06.2021
%%%endcit

%%%cit
%%%cit_pic
%%%cit_text
Взять хотя бы последнюю историю с формой для игры в мяч. Какая была \emph{победа},
когда УЕФА по недопониманию утвердила её. Какие словесные гопаки выплясывали
хлопцы и девчата в появившейся возможности потроллить соседа. И тут, бац, —
убрать немедленно! Зрада? Ещё бы. Вой неистовый в сторону продажного УЕФА не
слышат разве что на далёких квазарах. Но они упёртые. Где-нибудь в укромном
месте намалюют что хотят, потому что нация, потому что независимые, потому что
треба. И снова будет \emph{перемога}. А потом опять будет вой. Цикличность
%%%cit_comment
%%%cit_title
\citTitle{От \enquote{псеродактиля} в \enquote{андертальца} и далее в венец человечества: рассуждения никудышного антрополога}, 
Дмитрий Жук, zen.yandex.ru, 13.06.2021 
%%%endcit

