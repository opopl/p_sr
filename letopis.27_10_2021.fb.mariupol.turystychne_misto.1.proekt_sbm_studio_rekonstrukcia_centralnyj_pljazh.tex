%%beginhead 
 
%%file 27_10_2021.fb.mariupol.turystychne_misto.1.proekt_sbm_studio_rekonstrukcia_centralnyj_pljazh
%%parent 27_10_2021
 
%%url https://www.facebook.com/mistoMarii/posts/pfbid032vjTRJwCnG5rtUSRHKApQm9UHpDfAtKsBcrDk8kyoGnvbk63tVx5TpiAULEprERBl
 
%%author_id mariupol.turystychne_misto
%%date 27_10_2021
 
%%tags 
%%title Проєкт SBM Studio - реконструкція центрального пляжу Маріуполя
 
%%endhead 

\subsection{Проєкт SBM Studio - реконструкція центрального пляжу Маріуполя}
\label{sec:27_10_2021.fb.mariupol.turystychne_misto.1.proekt_sbm_studio_rekonstrukcia_centralnyj_pljazh}

\Purl{https://www.facebook.com/mistoMarii/posts/pfbid032vjTRJwCnG5rtUSRHKApQm9UHpDfAtKsBcrDk8kyoGnvbk63tVx5TpiAULEprERBl}
\ifcmt
 author_begin
   author_id mariupol.turystychne_misto
 author_end
\fi

🌊 Трішки раніше ми писали про завершення Міжнародний архітектурний конкурс на
реконструкцію Центрального пляжу Маріуполя. Переможцем став проєкт української
компанії SBM Studio. Вчора відбулася зустріч з французькими експертами, які
представили свої технічні концептуальні напрацювання.  У поєднанні з ідеями
переможця архітектурного конкурсу вони стануть основою для вироблення фінальної
версії проєкту реконструкції Приморського бульвару та Центрального пляжу.

Конкурс проводився з урахуванням напрацювань французьких експертів з компанії
Beten, а також серед професійного журі конкурсу була представниця компанії -
Алізе Моро

Так Центральний пляж буде поділено на 4 зони: 

🔷 Перша зона ( район мосту через залізничну колію) - тут зроблять сучасну зону відпочинку. \par
🔷 Друга (пірс), де пропонують облаштувати сучасний яхт-клуб, фестивальну зону та скейтпарк.\par
🔷 Третя (територія хвилерізів) - їх пропонують переробити під пірси з місцями для риболовів. \par
🔷 Четверта зона – тераса з ресторанами.\par

Також французькі експерти пропонують оновити не тільки пляжну лінію, але й
бульвар. Для цього планується організувати дві паркові зони біля північного та
південного входів, щоб прибрати автівки уздовж бульвару. Це вплине на
розширення пішохідної зони, а також посилить громадський та велотранспорт. 

Щоб з'єднати пляж, бульвар та паркову зону організують три маршрути. Один –
кільцевий для піших прогулянок, другий – через оглядові майданчики у верхній
частині пляжу, третій – за оглядовими майданчиками Приморського парку. Усі вони
об'єднаються в районі центрального пірса.
