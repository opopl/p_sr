% vim: keymap=russian-jcukenwin
%%beginhead 
 
%%file 30_01_2022.stz.news.ru.lenta.1.leonid_kuravljev
%%parent 30_01_2022
 
%%url https://lenta.ru/articles/2022/01/30/kuravlevrip
 
%%author_id ruzaev_denis
%%date 
 
%%tags kino,kultura,kuravljov_leonid,rossia,smert,sssr
%%title Умер Леонид Куравлев. Он был любимым актером нескольких поколений россиян
 
%%endhead 
 
\subsection{Умер Леонид Куравлев. Он был любимым актером нескольких поколений россиян}
\label{sec:30_01_2022.stz.news.ru.lenta.1.leonid_kuravljev}
 
\Purl{https://lenta.ru/articles/2022/01/30/kuravlevrip}
\ifcmt
 author_begin
   author_id ruzaev_denis
 author_end
\fi

\begin{zznagolos}
В Москве в возрасте 85 лет скончался Леонид Куравлев — не только один из самых
прославленных актеров отечественного кино, но и, возможно, самый любимый,
причем для нескольких поколений россиян сразу. Такие фильмы, как «Афоня» и
«Иван Васильевич меняет профессию», «Золотой теленок» и «Живет такой парень»,
продолжают находить своих зрителей и навсегда останутся частью российского
культурного кода. Все они немыслимы без обаяния Леонида Куравлева, о жизни и
карьере которого рассказывает материал «Ленты.ру».
\end{zznagolos}

\ii{30_01_2022.stz.news.ru.lenta.1.leonid_kuravljev.scr.1}

Очень не хотелось бы, чтобы поток печальных новостей о состоянии Леонида
Куравлева в последние месяцы его жизни — помещение в пансионат для пожилых
людей, одиночество, страшные болезни, а теперь и уход из жизни — затмил, как
это нередко бывает, его значение не только для отечественного кино, но и для
самоощущения, самоидентификации российского человека. Да, Куравлев чаще всего
играл комедийные роли — но при этом никогда не был классическим комедийным
артистом, скорее уж воплощал на экране ту иронию, без которой здраво проживать
русскую жизнь просто невозможно. Тем показательнее, что и судьба его выдалась
во многом для этой территории характерной.

\ii{30_01_2022.stz.news.ru.lenta.1.leonid_kuravljev.pic.1}

Куравлев родился на рабочей окраине Москвы — и уже в раннем детстве познал не
только бедность, но и всю тяжесть того обращения со своим народом, которая была
свойственна сталинскому государству. В 1941-м его мать арестовали по печально
известной 58-й статье («пропаганда или агитация, содержащие призыв к свержению,
подрыву или ослаблению Советской власти или к совершению отдельных
контрреволюционных преступлений»). Следующие пять лет она провела в Караганде,
после чего еще довольно долго не могла вернуться в Москву как ссыльная — и
Леонида и его сестру в ее отсутствие воспитывала тетя. Тем не менее сам актер,
несмотря ни на что, свое детство считал счастливым.

\ii{30_01_2022.stz.news.ru.lenta.1.leonid_kuravljev.pic.2}

Во ВГИКе после окончания школы будущий актер оказался более-менее случайно: по
легенде, сестра посоветовала Леониду, в детстве не дружившему с точными
науками, поступать в главный киноинститут страны, «потому что там точно не
придется сдавать математику». Приняли Куравлева, впрочем, только со второго
раза — и он успел целый год проработать на фабрике елочных украшений и линз для
оптики. Во ВГИКе же состоялось самое судьбоносное для его карьеры знакомство —
Куравлева приметил, разглядел в нем нечто одновременно уникальное и
универсальное заканчивавший режиссерский факультет Василий Шукшин, человек с
особенным чувством и эпохи, и русского народа. Куравлев сыграл характерного
комбайнера-заику в дипломной работе Шукшина «Из Лебяжьего сообщают» — и больше
недостатка в ролях уже не испытывал никогда.

«С легкой руки Шукшина я стал очень много сниматься, он как бы предложил меня
режиссерскому сообществу: \enquote{Обратите внимание на этого артиста — Куравлев его
фамилия. Может быть, вам он пригодится}», — скромно вспоминал потом сам артист.
Факт, впрочем, заключается в том, что, увидев игру Куравлева в первом полном
метре Шукшина «Живет такой парень», где он изобразил молодого алтайского шофера
Пашку Колокольникова, душа которого распахнута настолько широко, что он даже не
замечает собственного героизма, не заметить его таланта было решительно
невозможно.

В середине и второй половине 1960-х Куравлев стремительно превращается в одного
из самых востребованных актеров своего поколения. Следующий — и, возможно,
самый недооцененный — фильм Шукшина «Ваш сын и брат». Эпическая
производственная драма легендарного Михаила Швейцера «Время, вперед».
Пронзительная «Старшая сестра» Георгия Натансона. Подлинно народной же звездой
Куравлева сделал «Вий» — упоительный советский блокбастер, увидевший в
гоголевском первоисточнике материал для эффектного и остроумного зрительского
фэнтези. Проводником же для аудитории в этот сказочный мир послужил именно
Куравлев в роли Хомы Брута.

\ii{30_01_2022.stz.news.ru.lenta.1.leonid_kuravljev.pic.3}
