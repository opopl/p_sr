% vim: keymap=russian-jcukenwin
%%beginhead 
 
%%file 07_12_2020.fb.fb_group.story_kiev_ua.2.magia_chisel.cmt
%%parent 07_12_2020.fb.fb_group.story_kiev_ua.2.magia_chisel
 
%%url 
 
%%author_id 
%%date 
 
%%tags 
%%title 
 
%%endhead 
\subsubsection{Коментарі}

\begin{itemize} % {
\iusr{Людмила Каштан}
Очень красивая романтичная история!

\begin{itemize} % {
\iusr{Ирина Петрова}
\textbf{Людмила Каштан} спасибо) в предверии новогодных волшебств захотелось вспомнить

\iusr{Людмила Каштан}
\textbf{Ирина Петрова} Всё настолько реально, что хочется пожелать им счастья!
\end{itemize} % }

\iusr{Ольга Кирьянцева}
Ах, как романтично... @igg{fbicon.heart.sparkling} 
Неужели такое случается в жизни?... @igg{fbicon.face.happy.two.hands} 

\begin{itemize} % {
\iusr{Ирина Петрова}
\textbf{Ольга Кирьянцева} это немного "олитературенная" ситуация, некий симбиоз двух реальных историй из жизни нашей семьи. полностью придумать сюжет я бы не смогла - не профессиональный я писатель. Спасибо за отклик)

\iusr{Юлия Науменко}
\textbf{Ирина Петрова} благодарю Вас. Очень трогательно.

\iusr{Olena Safroniychuk}
\textbf{Ирина Петрова} Какая волнительная и чудесная история, ка будто просмотрела фильм представляя персонажей, улочки, балкон, цветы и все чувства. Благодарю!

\iusr{Оксана Гриневич}
\textbf{Ирина Петрова}, не сомневайтесь, Вы- писатель))) Почти не дыша, до последних букв и точек, слёзы в глазах и странная улыбка... пишите ещё @igg{fbicon.face.happy.two.hands} 

\iusr{Ирина Петрова}
\textbf{Oksana Grinevich} это отбежавшая в сторону частичка брата - профессионального писателя) я просто рядом с ним)))

\end{itemize} % }

\iusr{Мирослава Ткачук}
Хоть кино снимай @igg{fbicon.face.happy.two.hands}. Готовый сценарий  @igg{fbicon.thumb.up.yellow} 

\begin{itemize} % {
\iusr{Ирина Петрова}
\textbf{Мирослава Ткачук} думаю, на небольшую киноновеллу пошло бы)

\iusr{Елена Кулагина}
\textbf{Мирослава Ткачук} кстати, да! только актеров надо взять с улицы, не профессионалов. Чтобы сыграли как прожили, как написано и прочитано. Рассказ на одном дыхании.
\end{itemize} % }

\iusr{Наталия Гетьман}
Невероятно !!!! Неужели правда ?...@igg{fbicon.heart.red}

\iusr{Марина Резничук}
Просто изумительный рассказ

\begin{itemize} % {
\iusr{Ирина Петрова}
\textbf{Марина Резничук} очень рада, что удалось подарить немного радости)
\end{itemize} % }

\iusr{Влада Собуцька}
Чудово! Проникає аж до дна душі...

\begin{itemize} % {
\iusr{Ирина Петрова}
\textbf{Влада Собуцька} дякую, бо завжди почуття , якщо вони справжні, знаходять відклик у душах)
\end{itemize} % }

\iusr{Татьяна Кец}
Очень красивая история. Пусть на Вашем пути всегда будут счастливые семёрки!

\begin{itemize} % {
\iusr{Ирина Петрова}
\textbf{Татьяна Кец} спасибо! Я верю в магию чисел)))

\iusr{Людмила Ткаченко}
\textbf{Ирина Петрова} 

И я верю в магию чисел! И число семь для меня тоже счастливое. Жила в квартире
на седьмом этаже, училась в седьмой группе, на двух разных работах работала на
седьмом этаже и то были мои счастливые моменты жизни. Старшая дочь родилась в
июле ( 7 месяц) 21 числа ( три семёрки). А пять лет назад я впервые прилетела в
Израиль именно седьмого числа. Когда мой друг прислал мне электронные билеты на
Боинг 737, каково же было мое удивление, когда я увидела номера рейсов... В
Тель Авив рейс 777, а обратно в Киев 778. А в аэропорту меня встречал друг на
машине и на ее номерах было тоже три семёрки. И была я в Израиле семь дней. Это
была для меня самая незабываемая поездка. Ну как тут не верить в магию чисел, а
тем более числа семь!

\iusr{Ирина Петрова}
\textbf{Людмила Ткаченко} 

вот это даааа!!!! Ну, вот жизнь иногда такой расклад преподнесет; и самый
фантазийный писатель не выдумает! И, исходя из Вашего рассказа, обязательно
следующий, трисемёрочный год должен принести Вам и всем близким побольше
позитива! чего я искренне Вам желаю!

\iusr{Людмила Ткаченко}

Ой, как же приятно! Искренне Вам благодарна, Ирина, за добрые пожелания! И я
Вам желаю в будущем году всех благ, крепкого здоровья и много радости! Очень
надеюсь, что 2021 год однозначно будет для меня счастливым ещё и потому, что
это мой год - Год Быка!  @igg{fbicon.cow.face} Уррра! @igg{fbicon.face.blowing.kiss}  @igg{fbicon.wink} @igg{fbicon.hand.victory}

\iusr{Ольга Березюк}
Прочла на одном дыхании, очень трогательный рассказ, но слава Богу - хеппи энд @igg{fbicon.thumb.up.yellow}  @igg{fbicon.grin} 

\end{itemize} % }

\iusr{Vita Narodetsky}
Какая красивая история!

\begin{itemize} % {
\iusr{Ирина Петрова}
\textbf{Vita Narodetsky} любовь всегда красива, правда?

\iusr{Irina Ahieieva}
\textbf{Ирина Петрова} история для фильма
\end{itemize} % }

\iusr{Світлана Ліщинська}
Красивая сказка! И добрая!!!

\begin{itemize} % {
\iusr{Ирина Петрова}
\textbf{Світлана Ліщинська} кое-что в этой истории абсолютная правда, вернее, как бы две даже правды;)
\end{itemize} % }

\iusr{Ольга Корвацкая}
Как хочется, чтобы это было правдой! Такое действительно бывает рах в жизни.

\iusr{Ирина Петрова}
\textbf{Ольга Корвацкая} да, настоящее всегда бывает нечасто, иногда и один всего лишь раз в жизни)

\iusr{Тетяна Мар'янова}
Замечательная история!

\iusr{Ирина Петрова}
\textbf{Тетяна Мар'янова} спасибо, тронута откликами тёплыми)

\iusr{Марина Киорогло}
Замечательный рассказ!!! Ждём ещё...  @igg{fbicon.bouquet} 

\iusr{Ирина Петрова}
\textbf{Марина Киорогло} спасибо. Предновогоднее настроение, думаю, этот рассказик подарил читателям)

\iusr{Irena Visochan}
Красивый рассказ! Такой душевный!

\begin{itemize} % {
\iusr{Ирина Петрова}
\textbf{Irena Visochan} спасибо) любовь, верность всегда трогают наши души)

\iusr{Irena Visochan}
\textbf{Ирина Петрова} Конечно! Я всегда радуюсь за тех, кому хорошо.
\end{itemize} % }

\iusr{Helen Rybka}
Очень трогательный рассказ

\iusr{Ирина Петрова}
\textbf{Helen Rybka} спасибо)

\iusr{Мария Андрушко}
Вірю, це все правда. До сліз зворушливо.

\iusr{Ирина Петрова}
\textbf{Мария Андрушко} коли є віра - навіть, неможливе стає можливим...

\iusr{Мария Андрушко}
\textbf{Ирина Петрова} дякую, шановна пані Ірина.

\iusr{Ирина Гукасян}
Хотелось плакать! Хороший рассказ!!

\iusr{Ирина Петрова}
\textbf{Ирина Гукасян} Благодарю. Очень трогает сердце чувство читателей.

\iusr{Наталия Ковалева}
Я так рада за вас!!!!

\iusr{Ирина Петрова}
\textbf{Наталия Ковалева} спасибо! Вы угадали участником одного из эпизодов была я)))

\iusr{Tetiana Tetiana}
Я тоже верю в магию чисел. Сократ тоже согласен. Спасибо за историю.

\iusr{Ирина Петрова}
\textbf{Tetiana Tetiana} Спасибо за отклик! О! Сократ! Какая замечательная компания @igg{fbicon.wink} 

\iusr{Наталия Ковалева}

\ifcmt
  ig https://scontent-frx5-2.xx.fbcdn.net/v/t39.1997-6/p480x480/100387475_2708175719423460_6431293155835904000_n.png?_nc_cat=1&ccb=1-5&_nc_sid=0572db&_nc_ohc=5dJiUV_ZgScAX8qfimT&_nc_ht=scontent-frx5-2.xx&oh=777c10e47ca3201969d0ef53e76161ff&oe=6194F88A
  @width 0.2
\fi

\iusr{Елена Мороз}
На одном дыхании

\iusr{Ирина Петрова}
\textbf{Елена Мороз} спасибо! Очень приятно, что трогает душу)

\iusr{Елена Мороз}
\textbf{Ирина Петрова} очень! Вам спасибо

\iusr{Natasha Sutulova}
Спасибо

\iusr{Ирина Петрова}
\textbf{Natasha Sutulova} Спасибо)

\iusr{Наталия Калинина}
Потрясающий рассказ! Сердечно Вам спасибо за него! И ёлка на фото сказочная! Такие тогда ставили на площади.

\begin{itemize} % {
\iusr{Ирина Петрова}
\textbf{Наталия Калинина} Спасибо за теплые слова! Да, ёлочку тогда ставили на площадке, где потом будет памятник, а потом стали ставить на противоположной тороне плащади)
\end{itemize} % }

\iusr{Вера Вера}
Как трогательно, прямо до слез...

\iusr{Ирина Петрова}
\textbf{Вера Вера} спасибо за светлые чувства)

\iusr{Татьяна Рубан}
Реву....

\begin{itemize} % {
\iusr{Ирина Петрова}
\textbf{Татьяна Рубан} так всё ж хорошо закончилось, возможно, и медицина оказалась не бессильна..будем так думать)

\iusr{Татьяна Рубан}
\textbf{Ирина Петрова} так и реву от счастья! Хоть и не своего.. Ведь в 77г мне было 18 лет и, читая рассказ я окунулась в свою молодость, в радости и горечи того времени.

\iusr{Ирина Петрова}
\textbf{Татьяна Рубан} очень рада, что доставила Вам несколько проникновенных минут!
\end{itemize} % }

\iusr{Alla Biden}
Вот, что значит! Не стандартное мышление... Суперрррр!

\begin{itemize} % {
\iusr{Ирина Петрова}
\textbf{Alla Biden} тут Вы совершенно правы - было такое дело, нестандартный подход к поискам ёлочки @igg{fbicon.wink} 

\iusr{Alla Biden}

\ifcmt
  ig https://scontent-frx5-2.xx.fbcdn.net/v/t39.1997-6/s168x128/88352243_204096957500194_4694650943906512896_n.png?_nc_cat=1&ccb=1-5&_nc_sid=ac3552&_nc_ohc=vnjZD98TcCoAX9yVL5X&_nc_ht=scontent-frx5-2.xx&oh=3c924d44c9c50191543776ce4d1665fa&oe=6194DBDB
  @width 0.1
\fi

\end{itemize} % }

\iusr{Нинель Кузницкая}
Чудесная история! Спасибо!  @igg{fbicon.hands.pray}  @igg{fbicon.bouquet} 

\iusr{Ирина Петрова}
\textbf{Нинель Кузницкая} спасибо, приятна всегда похвала

\iusr{Катерина Силина}
Очень трогательно.

\iusr{Ирина Петрова}
\textbf{Катерина Силина} благодарю)

\iusr{Елена Галинская}

Благодарю автора  @igg{fbicon.hands.pray}  Желаю всем чудес и окрыляющей любви от всей души! Пусть
каждый новый день будет наполнен радостью пробуждения, а не душевной болью!

\iusr{Ирина Петрова}
\textbf{Елена Галинская} Спасибо за теплые пожелания для всех нас!

\iusr{Iryna Naidonova}
Настоящий Новогодний рассказ, о любви, чудесах и судьбе! Благодарю!

\iusr{Ирина Петрова}
\textbf{Iryna Naidonova} да, вот поэтому он и вспомнился мне сейчас, в предверии новогодних чудес!

\iusr{Татьяна Тамер}
Какая чудесная история, очень трогательно, спасибо

\begin{itemize} % {
\iusr{Ирина Петрова}
\textbf{Татьяна Тамер} спасибо! Тронута...

\iusr{Татьяна Тамер}
\textbf{Ирина Петрова} у Вас чудесный слог, и история тоже трогают до глубины души  @igg{fbicon.face.smiling.eyes.smiling} 
\end{itemize} % }

\iusr{Любовь Жук}
Изумительно!

\iusr{Ирина Петрова}
\textbf{Любовь Жук} благодарю)

\iusr{Лилия Швец}
Душевно до сліз @igg{fbicon.face.sleepy} 

\begin{itemize} % {
\iusr{Ирина Петрова}
\textbf{Лилия Швец} да, искренние чувства всегда очень трогательны, правда?

\iusr{Лилия Швец}

З першої хвилини читання я так хотіла, щоб вони були разом. І щоб кожного 31
грудня у них була зайва ялинка, яку б вони дарували людям.

\end{itemize} % }

\iusr{Nadejda Popova}
Спасибо, прямо волшебная история))

\iusr{Ирина Петрова}
\textbf{Nadejda Popova} жизнь - это самая настоящая волшебница)

\iusr{Оксана Куртова}
Я тоже верю в числа. Спасибо!

\iusr{Ирина Петрова}
\textbf{Оксана Куртова} Благодарю! Числа - это не просто так, правда!

\iusr{Виктория Глущенко}
Как же проникновенно Вы пишите! Просто невероятная , сказочная история! Спасибо Вам!

\begin{itemize} % {
\iusr{Ирина Петрова}
\textbf{Виктория Глущенко} так получается, наверное, потому, что это не вполне придуманная история, вернее, не придуманные несколько жизненных историй) Спасибо!)
\end{itemize} % }

\iusr{Ирина Черныш}

Есть надежда, что 21 год будет хорошим, ведь это сумма трёх семёрок. Я тоже
верю в магию чисел. И поверила в историю. Спасибо!!!

\iusr{Ирина Петрова}
Ирина Черныш вот именно - ведь ТРИ семёрки - это даже не две @igg{fbicon.wink} 

\iusr{Elena Kobinskaya}

Такой трогательный рассказ... Пусть у всех сбываются мечты, и сердце верит в
чудеса! Благодарю вас за сказку @igg{fbicon.face.happy.two.hands} 

\iusr{Ирина Петрова}
\textbf{Olena Kobinska} Благодарю за отклик и Вам чудес!

\iusr{Elena Kobinskaya}
\textbf{Ирина Петрова} благодарю и принимаю во благо  @igg{fbicon.hands.pray} 

\iusr{Ирина Клименко}
До слез , с верой в чудеса и любовь

\begin{itemize} % {
\iusr{Ирина Петрова}
\textbf{Ирина Клименко} благодарю! Пусть слезы у нас будут только от прекрасных чувств!

\iusr{Iryna Hoffmann}
\textbf{Ирина Петрова} и я реву, как белуга. Спасибо за рассказ.

\iusr{Ирина Петрова}
\textbf{Iryna Hoffmann} девочки, милые, там потом всё-всё было хорошо, я точно знаю!
\end{itemize} % }

\iusr{Наталья Иванова}

Ну почему я не режиссер? Ваш необыкновенный рассказ - это же готовый сценарий
для доброго, новогоднего фильма. Ирина, Вы написали чудесно, я даже
расплакалась, когда финал оказался вдруг счастливым! Написано настолько
искренне, сердцем, что не оставляет сомнений в Вашем личном участии в этой
истории, иначе так классно написать невозможно! УМНИЧКА!!! Говорю, как филолог
и как женщина, конечно.

\begin{itemize} % {
\iusr{Ирина Петрова}
\textbf{Наталья Иванова} Благодарю Вас! Мне очень ценен отклик специалиста, потому что - сущий технарь, пишется иногда по настроению Реальные истории, подчас, невероятнее самых изощренных придумок.)

\iusr{Наталья Иванова}
\textbf{Ирина Петрова}

Вы - сущий технарь? Ну нет! Вы так мастерски владеете словом, которое задевает
самые потаённые уголки души, что далеко не каждому писателю под силу. И,
кстати, написано очень грамотно в смысле пунктуации и орфографии, что тоже
немаловажно, т.к. сейчас встретить на просторах интернета грамотно написанный
пост - весьма большая редкость. А Вам, Ирина, ОБЯЗАТЕЛЬНО НУЖНО ПИСАТЬ, это
Ваше. Удачи и благосклонности фортуны!

\iusr{Ирина Петрова}
\textbf{Наталья Иванова} 

благодарю))) уже говорила, что в семье профиписатель - старший брат, я "на
растирке красок". Даже в школе всегда любила сочинения на вольные темы.
Придумать полностью сюжет, с завязкой, кульминацией, развитием, концовкой мне
не под силу. А вот немного описать реальные события как-то получается @igg{fbicon.face.happy.two.hands}  и тепло
от того, что могут принести мои новеллки приятные минуты многим людям @igg{fbicon.heart.eyes} 

\iusr{Наталья Иванова}
\textbf{Ирина Петрова} Пишите как велит сердце! @igg{fbicon.heart.red}

\end{itemize} % }

\iusr{Олександра Пеліховська}

Так трогательно. Слёзы.
Спасибо большое, что поделились, это очень важно и нужно.

\begin{itemize} % {
\iusr{Ирина Петрова}
\textbf{Олександра Пеліховська} Благодарю) светлые чувства очень драгоценны в наше время, а их у читателей так много. и это радует!)
\end{itemize} % }

\iusr{Olga Velichko}
СПАСИБО !!! Будьте здоровы и долгих, счастливых лет !!!

\begin{itemize} % {
\iusr{Ирина Петрова}
\textbf{Olga Velichko} Благодарю Вас! И вам всех предновогодних радостных ощущений!

\iusr{Olga Velichko}
\textbf{Ирина Петрова} СПАСИБО БОЛЬШОЕ.
\end{itemize} % }

\iusr{Юлия Лоцман}
Як зворушливо! І які повороти долі нас чекають, ми не знаємо, але так важливо чути своє серце! Дякую за прекрасну історію!

\iusr{Ирина Петрова}
\textbf{Юлия Лоцман} Дякую! Так, життя - то такий примхливий режисер)

\iusr{Елена Банит}
Спасибо за душевность и сердечность вашего чудесного рассказа.

\iusr{Ирина Петрова}
\textbf{Елена Банит} Спасибо) так приятны похвалы читателей)))

\iusr{Evgenya Podtepa}
И я реву

\begin{itemize} % {
\iusr{Ирина Петрова}
\textbf{Evgenya Podtepa} не, не, не, сонечко, всё замечательно, правда, это не выдумка! Это новогодние чудеса❄️
\end{itemize} % }

\iusr{Лилия Перехрест}
Прелесть!

\iusr{Люсянка Балашова}
tft

\begin{itemize} % {
\iusr{Люсянка Балашова}
это у меня от волнения... ваш рассказ тронул меня до слёз... спасибо
\end{itemize} % }

\iusr{Nina Gerasymenko}
Дякую! До сліз!

\begin{itemize} % {
\iusr{Ирина Петрова}
\textbf{Nina Gerasymenko} І Вам дякую за почуття!

\iusr{Nina Gerasymenko}
\textbf{Ирина Петрова}

\ifcmt
  ig https://scontent-frx5-2.xx.fbcdn.net/v/t39.1997-6/s168x128/16781161_1341101952618574_7704631035023065088_n.png?_nc_cat=1&ccb=1-5&_nc_sid=ac3552&_nc_ohc=7VVbYisllDsAX_Wd3Vk&_nc_ht=scontent-frx5-2.xx&oh=26fbb3769d691fc4de785a969568c35c&oe=61946C6B
  @width 0.1
\fi

\end{itemize} % }

\iusr{Ольга Топчий}

Всё - таки Новый год - это сказочный праздник и волшебство! Люблю истории со
счастливым концом!

\begin{itemize} % {
\iusr{Ирина Петрова}
\textbf{Ольга Топчий} я тоже, очень люблю, тем более, что они есть в жизни!)

\iusr{Ольга Топчий}
Спасибо Вам!!!
\end{itemize} % }

\iusr{Olga Kyivska}
Спасибо. Вот готовый сюжет из жизни для блокбастера.

\iusr{Ирина Петрова}
\textbf{Olga Kyivska} ой, спасибо, я, конечно, на такое и не замахиваюсь @igg{fbicon.face.grinning.squinting}  хотя бы студенческую дипломку на 15 минут)))

\iusr{Тамара Федоренко}
Я поверила...

\iusr{Ирина Петрова}
\textbf{Тамара Федоренко} благодарю! Честно, тут реальные историйки

\iusr{Катя Бекренева}

Девушки, реву) )

А можно немного смешного, но тоже из жизни, моей?

Жили мы в Святошино, было 30 декабря, канун 76го года, 8 января у мамы юбилей,
50, ждали много гостей, раньше в ресторанах редко отмечали. Было принято
решение елку не ставить, т.к. отмечать будут в большой комнате. Еле уговорила
маму, что Новый год без елки,это не праздник.

Мама сдалась. Вышла к электричкам, там всегда можно было купить с рук, минут
через десять торжественно внесла довольно большую. сосну завернутую в бумагу.
Мы с братом развернули, и все ахнули... Был сосновый столб, сверху веточки,
снизу ветки, а вся средина голая, но прилагалось к этому четыре огромные
пушистые ветки. Мы с братом, вооружившись молотком с гвоздями и проволокой,
соорудили прекрасное новогоднее дерево, и поставили в моей комнате.

Праздник удался)

\begin{itemize} % {
\iusr{Ирина Петрова}
\textbf{Катя Бекренева} воооот, это же тоже сюжет для новогодней новеллы!!! @igg{fbicon.laugh.rolling.floor}{repeat=3} 

\iusr{Катя Бекренева}
Ирина, мне было 14, а брату 21, это был последний мамин юбилей, через три года её не стало (
\end{itemize} % }

\iusr{Людмила Яфаева}
Тронули до слёз . Огромное спасибо. Жизнь, все-таки, дарит чудеса и нужно верить в это

\begin{itemize} % {
\iusr{Ирина Петрова}
\textbf{Людмила Яфаева} жизнь - мешок сюрпризов, кто верит - всегда с подарком!))
\end{itemize} % }

\iusr{Людмила Стасевич}
Благодарю, так понравился рассказ- нет слов!

\iusr{Ирина Петрова}
\textbf{Людмила Стасевич} благодарю! Иногда и слов не надо, автор чувствует)))

\iusr{Виктория Иванюшина}

Очень тепло и уютно @igg{fbicon.heart.red}  @igg{fbicon.christmas.tree}  @igg{fbicon.heart.red} ️ Следующий
год 2021 @igg{fbicon.heart.suit} ️  @igg{fbicon.hearts.two}
@igg{fbicon.heart.red} ️ Это уже 3 семёрки (7 @igg{fbicon.heart.red} ️ 7
@igg{fbicon.heart.suit} ️ 7 @igg{fbicon.heart.red} ️ ) Пусть эти волшебные
семёрки принесут нам всем чудеса, в которые очень хочется верить
@igg{fbicon.heart.red} ️  @igg{fbicon.hearts.two}  @igg{fbicon.heart.suit} ️
!!!!!!

\begin{itemize} % {
\iusr{Ирина Петрова}
\textbf{Victorya Ivanushina} верим! Ждём! Надеемся! Ну, 21-й только попробуй не...  @igg{fbicon.face.grinning.squinting} 

\iusr{Виктория Иванюшина}

\ifcmt
  ig https://scontent-frx5-2.xx.fbcdn.net/v/t39.1997-6/s168x128/66138183_615506415623488_5774385587314229248_n.png?_nc_cat=1&ccb=1-5&_nc_sid=ac3552&_nc_ohc=WS7dz0IKzN4AX9_Z6p2&_nc_ht=scontent-frx5-2.xx&oh=0149211abcaa83c3277c0e7885e0b937&oe=6194645A
  @width 0.1
\fi

\end{itemize} % }

\iusr{Надежда Дунаева}
 @igg{fbicon.dove}  @igg{fbicon.hands.pray}{repeat=2}  @igg{fbicon.dove}  @igg{fbicon.face.sad.but.relieved} 

\iusr{Ирина Петрова}
\textbf{Надежда Дунаева} спасибо!

\iusr{Петр Кузьменко}
Нумерология интересная штука! Спасибо!

\iusr{Ирина Петрова}
\textbf{Петр Кузьменко} спасибо! Дааа, это всё непросто! Ждём три семёрки! И пусть только попробует быть не таким, как мы ждём!!!))

\iusr{Марина Яроцкая}
Благодарю Вас  @igg{fbicon.hands.pray}  очень трогательно !

\iusr{Ирина Петрова}
\textbf{Marina Yarotskaya} благодарю! Приятно знать, что нашла история отклик в душах)

\iusr{Tania Gumeniuk}
Волшебный 1977 @igg{fbicon.heart.red}

\iusr{Ирина Петрова}
\textbf{Tania Gumeniuk} да, это была ещё такаааая юность!!! Ух!!!

\iusr{Lana Green}
Великолепный рассказ  @igg{fbicon.hands.applause.yellow}{repeat=3} 

\iusr{Ирина Петрова}
\textbf{Lana Green} спасибо! Сердцу автора тепло и приятно!

\iusr{Ольга Лубягина}
Замечательный, трогательный, романтический рассказ о любви. Очень понравился

\iusr{Ирина Петрова}
\textbf{Ольга Лубягина} спасибо! Приятно, что мой рассказик принес пару милых минут)

\iusr{Валентина Федоровна}
Дякую. Так зворушливо, так неймовірно красиво!

\iusr{Ирина Петрова}
\textbf{Валентина Федоровна} дякую! Дуже важливо чути відгук читача! А якщо він при цьому і схвальний @igg{fbicon.heart.eyes}{repeat=3} 

\iusr{Ирина Шимановская}
Спасибо

\iusr{Ирина Петрова}
\textbf{Ирина Шимановская} и Вам спасибо за отклик)

\iusr{Алла Зубкова}
Не могла оторваться! Это очень нужная, хорошая литература! Так ее не хватает! Я уж думала, не пишут уже)))

\begin{itemize} % {
\iusr{Ирина Петрова}
\textbf{Алла Зубкова} пишууут, много, интересно, душевно, но, не хватает времени почитать из-за Фейсбука @igg{fbicon.wink}  ведь есть вещи, о которых никогда не перестанут писать)
\end{itemize} % }

\iusr{Галина Хомченко}

Вы - молодец! сказать тронули до слез - ничего не сказать... Пишите и делитесь
своими историями: в наше время так не хватает искренности, любви и веры! Веры в
жизнь! Спасибо!

\begin{itemize} % {
\iusr{Ирина Петрова}
\textbf{Галина Хомченко} да, вы правы, времена не выбирают, и вокруг нас то, что мы сами и несём в каждый день!
\end{itemize} % }

\iusr{Ляля Пышная}
Тронуло до слез. Очень душевно. Спасибо.

\iusr{Ирина Петрова}
\textbf{Ляля Пышная} спасибо, дорогая, уверена, что это чистые слезы нежности)

\iusr{Ирина Саркисова}
Спасибо, очень трогательно, нежно, Ирония судьбы киевского разлива, здоровья всем и радости в 2021 году трёх семерок

\begin{itemize} % {
\iusr{Ирина Петрова}
\textbf{Ирина Саркисова} спасибо, сравнение крайне лестное, три семёрки просто обязаны исполнить свою важную роль!!!

\iusr{Марина Ермилова}
\textbf{Ирина Саркисова} только где вы нашли три семерки?
\end{itemize} % }

\iusr{Валентина Камко}
ВОТ ОНА МАГИЯ: СРАЗУ ВСЁ РОДНОЕ, ТВОЁ.... ПОКАЗАЛОСЬ И ИСЧЕЗЛО.... НАВСЕГДА...

\begin{itemize} % {
\iusr{Ирина Петрова}
\textbf{Валентина Камко} нееет, нет, оно всегда с нами, и неважно ведь где - на родной улице или в уголке памяти, в любимом скверике или в сердце - всегда и везде с нами наше родное и прошлое @igg{fbicon.heart.eyes} 
\end{itemize} % }

\iusr{Ольга Кочеткова}
Класс!

\iusr{Ирина Петрова}
\textbf{Olga Kochetkova} спасибо!!!

\iusr{Ольга Кочеткова}
\textbf{Ирина Петрова} очень хороший рассказ!

\iusr{Ирина Забродская}
До слез...

\iusr{Ирина Петрова}
\textbf{Irina Zabrodskaya} я не ожидала такого впечатления, трогательного, душевного от своей маленькой истории. Спасибо!

\iusr{Татьяна Лобачевская}
Спасибо за рассказ

\begin{itemize} % {
\iusr{Ирина Петрова}
\textbf{Татьяна Лобачевская} и Вам спасибо за теплые впечатления

\iusr{Татьяна Лобачевская}
\textbf{Ирина Петрова} вы и представить не можете, какие тёплые! Вот прослезилась, как прожила кусочек жизни. Можно ведь и фильм поставить, это будет замечательно! Спасибо, пишите ещё!

\iusr{Ирина Петрова}
\textbf{Татьяна Лобачевская} есть ещё кое-что в папочке  @igg{fbicon.face.happy.two.hands}  как-нибудь... да и здесь уже есть немного, были случаи, @igg{fbicon.wink} 
\end{itemize} % }

\iusr{Галина Шаронова}
Душу тронули, спасибо, замечательный рассказ, а можно продолжение?

\begin{itemize} % {
\iusr{Ирина Петрова}
\textbf{Галина Шаронова} продолжением была обычная счастливая жизнь, семья, дети, внуки, старость милая...сама жизнь @igg{fbicon.face.happy.two.hands} 
\end{itemize} % }

\iusr{Ольга Шишак}

Как будто художественный фильм посмотрела-невероятная и трогательная история.

\begin{itemize} % {
\iusr{Ирина Петрова}
\textbf{Ольга Шишак} 

спасибо! Были предложения сделать небольшую ленту
@igg{fbicon.face.happy.two.hands} но, это уже с мастерам фильма)

\iusr{Татьяна Зоценко}
\textbf{Ольга Шишак}, а мне почему то сразу вспомнился фильм "Не отрекаются любя". Что-то в них есть похожее...
\end{itemize} % }

\iusr{Мария Кобыжская}

\ifcmt
  ig https://scontent-frx5-2.xx.fbcdn.net/v/t39.1997-6/s168x128/47270791_937342239796388_4222599360510164992_n.png?_nc_cat=1&ccb=1-5&_nc_sid=ac3552&_nc_ohc=1KT2uRRam-gAX-9m-Lq&_nc_ht=scontent-frx5-2.xx&oh=c91bae6d0d6a1cf662261763eb93b5c6&oe=61938A36
  @width 0.1
\fi

\iusr{Лариса Полякова}
Замечательная история- простая и понятная всем. Спасибо

\iusr{Ирина Петрова}
\textbf{Лариса Полякова} спасибо!)

\iusr{Людмила Губанова}
 @igg{fbicon.thumb.up.yellow}  @igg{fbicon.hands.applause.yellow}  @igg{fbicon.beaming.face.smiling.eyes} 

\iusr{Rimma Turovskaya}
Потрясающая история. Какое счастье, что ОН выжил.

\begin{itemize} % {
\iusr{Ирина Петрова}
\textbf{Rimma Turovskaya} да, и жизнь продолжилась во всей её полноте)
\end{itemize} % }

\iusr{Наталия Кулик}
@igg{fbicon.heart.red}

\iusr{Надежда Кожушная}
Дякую

\iusr{Мария Аронова}
Чудо

\iusr{Larisa Kirova}
Какая чудесная, волшебная и трогательная история...

\begin{itemize} % {
\iusr{Ирина Петрова}
\textbf{Larisa Kirova} любая история любви - это волшебно и чудесно, правда?

\iusr{Larisa Kirova}
да, правда. Но бывают и особенно волшебные - вот как Ваша :))
\end{itemize} % }

\iusr{Тетяна Романовська}

Просто милая и трогательная история, прочитала на одном дыхании! Сама потеряла
в Афганистане свою первую любовь, очень тяжело это пережила, а у вас такой
прекрасный конец, огромное вам спасибо и много вам счастья!

\begin{itemize} % {
\iusr{Ирина Петрова}
\textbf{Тетяна Романовська} 
Благодарю... Вам было непросто вспомнить свою историю... мои сочувствия... Вы нашли силы жить дальше, жить во имя любви...
\end{itemize} % }

\iusr{Антонина Юхта}
Читается на одном дыхании, супер и очень хочется верить в чудеса спасибо будьте здоровы.

\iusr{Ирина Петрова}
\textbf{Антонина Юхта} чудеса - они и происходят для того, чтобы в них верили @igg{fbicon.face.happy.two.hands} 

\iusr{Алла Квасницкая}

Так красиво, романтично. И дарит надежду на счастье! Хотя уже не в моем
возрасте. Для меня 77 год - это год моей свадьбы. И вот уже скоро 44 года
счастья. Разного, конечно. Но рассказ навеял такие щемящие воспоминания.
Спасибо огромное!

\begin{itemize} % {
\iusr{Ирина Петрова}
\textbf{Алла Квасницкая} благодарю! Как прекрасно, что маленькая история напомнила Вам о счастливой дате Вашей жизни! И верю, что Ваше счастье будет очень долгим @igg{fbicon.heart.eyes} 
\end{itemize} % }

\iusr{Наталия Королева}
Слёзы! Но - счастливые!

\begin{itemize} % {
\iusr{Ирина Петрова}
\textbf{Наталия Королева} благодарю именно за "счастливые" @igg{fbicon.face.happy.two.hands}  они часто нужны, омывают пыль повседневности @igg{fbicon.face.happy.two.hands} 
\end{itemize} % }

\iusr{Олексій Мачинський}

Не верю... Хотя....
Неужели с кем-то, кроме меня, чудеса случаются?
Есть Он... истинно есть!

\begin{itemize} % {
\iusr{Ирина Петрова}
\textbf{Олексій Мачинський} знаете, ведь чудеса есть , они есть не потому, что кто-то верит в них , или не верит... им всё равно... они просто происходят... со всеми @igg{fbicon.wink} 
\end{itemize} % }

\iusr{Оксана Трощановская}
Главное-не пропустить свою судьбу. Такой шанс онп дарит только однажды!

\iusr{Роза Миколаївна Кириченко}
Замечательный рассказ.

\ifcmt
  ig https://scontent-frt3-1.xx.fbcdn.net/v/t1.6435-9/130488990_163334578809383_6861591331204635082_n.jpg?_nc_cat=108&ccb=1-5&_nc_sid=dbeb18&_nc_ohc=Q9U8O98_s4cAX9C-8Ym&_nc_ht=scontent-frt3-1.xx&oh=44512368e302c47fedc98da36d8b7052&oe=61B5B892
  @width 0.4
\fi

\begin{itemize} % {
\iusr{Ирина Петрова}
\textbf{Роза Кириченко} благодарю! Это мои любимые цветы!

\iusr{Роза Миколаївна Кириченко}
\textbf{Ирина Петрова} Очень рада. Любуйтесь.
\end{itemize} % }

\iusr{Виктория Глебова}
очень трогательно, спасибо

\iusr{Татьяна Литвина}
Чудесная история и очень позитивная.... Благодарю за талант и за такую редкость -душевность....

\iusr{Ирина Петрова}
\textbf{Татьяна Литвина} и Вас благодарю за теплые ободряющие слова @igg{fbicon.heart.eyes} 

\iusr{Надежда Сенчило}
Чудо история о любви с первого взгляда, какое счастье, что события позитивные, а любовь не проходит, спасибо, что поделились.

\begin{itemize} % {
\iusr{Ирина Петрова}
\textbf{Надежда Сенчило} спасибо за теплые слова, всё очень верно!
\end{itemize} % }

\iusr{Юлия Чернышева}
Очень трогательно...

\iusr{Неля Бондарчук}

\ifcmt
  ig https://scontent-frx5-2.xx.fbcdn.net/v/t39.1997-6/s168x128/93118771_222645645734606_1705715084438798336_n.png?_nc_cat=1&ccb=1-5&_nc_sid=ac3552&_nc_ohc=4g5LOmLaxfQAX_jWmi2&tn=lCYVFeHcTIAFcAzi&_nc_ht=scontent-frx5-2.xx&oh=69e96d339a18683eb4fa3efb75e5e835&oe=61954FBF
  @width 0.1
\fi

\iusr{Катя Клюева}

Дааа, мыльносериальная чушь про любовь. Сладкий сиропчик вместо жизни. Стыдно!
Никаких газетных статей про боевых товарищей быть не могло, СССР скрывал
участие в войне в Афгане. Остальное тоже видимо личная авторская фантазия.

\begin{itemize} % {
\iusr{Татьяна Петрюк}
\textbf{Катя Клюева} , не фантазия автора , а ТАЛАНТ !! Почувствуйте разницу , детка ...

\iusr{Ирина Петрова}
\textbf{Katya Klueva} конечно, фантазия, конечно, не могло быть, конечно сиропчик... забудьте... в мире столько правды и истины - на всех хватит.

\iusr{Алена шевченко}
\textbf{Катя Клюева} может и фантазия, но не чушь

\iusr{Raisa Prisetska}
\textbf{Katya Klueva} ссср скрывал полтора года 79—80. А дальше все было в газете Правда. Поднимите архив и не пишите чушь по малолетству. Неуки!

\iusr{Ирина Петрова}

В группе запрещены ссылки на сторонние ресурсы. Это жесткое и правильное
требование. Поэтому придется дать многословную выдержку, понимаю, что автор
комментария её не сочтет нужным прочесть, пусть... однако, правда, какой бы
горькой она не была, имеет право быть увиденной...

Итак (конечно, не могу привести документ полностью, он ооочень объемен) :

"......секретная записка отдела внешнеполитической пропаганды (Л. Замятин) и
отдела административных органов (Н. Савинкин) ЦК КПСС «О публикациях в
средствах массовой информации материалов относительно действий ограниченного
контингента советских войск в Aфганистане». В ней сообщалось о проработке
Министерством обороны, Министерством иностранных дел и КГБ перечня сведений,
«разрешаемых к открытому опубликованию, относительно действий ограниченного
контингента советских войск на территории Демократической Республики
Aфганистан». Говорилось, что «отделы ЦК КПСС считают возможным согласиться с
этим перечнем и поручить средствам массовой информации руководствоваться им при
освещении вопросов, связанных с пребыванием советских войск в Aфганистане».

К записке прилагался документ, завизированный руководителями трех силовых
ведомств государства: Министерства обороны (С. Aхромеев), Министерства
иностранных дел (Г. Корниенко) и КГБ (В. Крючков), в котором излагалась их
согласованная позиция в отношении событий в Aфганистане и освещения их в СМИ. 

С
этого момента разрешалось показывать «организацию и ход боевой подготовки,
размещение во временных городках воинских частей, их повседневную деятельность,
проведение совместно с подразделениями ВС ДРA тактических учений в масштабе не
выше батальона; посещение советских частей руководителями партии и
правительства ДРA, другими афганскими делегациями и проведение мероприятий
партийно-политического характера и культурно-массовой работы; привлечение
летательных аппаратов и автотранспорта для перевозок грузов местному населению
и выделение боевых подразделений для сопровождения колонн и охраны отдельных
строящихся объектов; наличие и работу советских военных специалистов по
оказанию помощи афганским военнослужащим в освоении поставляемой боевой
техники; применение одиночными советскими военнослужащими, отделениями
(экипажами, расчетами) и взводами штатного вооружения в целях самообороны при
нападении на них мятежников в ходе занятий и учений, в период передвижения и
патрулирования, при выполнении других повседневных задач, охране и обороне
своих и совместно с воинами ДРA афганских объектов, разминировании, доставке
грузов, сопровождении транспортных колонн, проведении повседневных полетов
боевых вертолетов и самолетов; отдельные единичные факты (не более одного в
месяц) ранений или гибели советских военнослужащих при исполнении воинского
долга, отражении нападения мятежников, выполнения заданий, связанных с
оказанием интернациональной помощи афганскому народу; строительство,
эксплуатацию, вооруженную охрану и оборону трубопровода, построенного
советскими подразделениями, и их повседневную деятельность; работу советского
военного госпиталя по оказанию врачебной медицинской помощи местному населению,
раненым афганским и поступившим на излечение советским военнослужащим;
присвоение советским военнослужащим звания Герой Советского Союза с показом их
мужества и героизма, проявленных при оказании интернациональной помощи ДРА, без
приведения сведений об участии подразделений и частей, где они служили, в
боевых действиях» (Ляховский, 1995: 296-298)." как-то так...

\begin{itemize} % {
\iusr{Raisa Prisetska}
\textbf{Ирина Петрова} нЕуки, позиционирующиеся себя, как великие спецы с умными словами такое не читают. И даже не слышали об этом. Я в шоке очередной раз. Одна в комменте спрашивает, что такое «интернациональный долг»(((. Докатились...
\end{itemize} % }

\iusr{Ирина Петрова}

ну, возможно, люди, младше тридцати лет, не знают этого словосочетания. Знаете,
когда выносишь что-либо в сеть, рассказ, просто комментарий, то понятно, что ты
- не доллар, всем не понравишься. И это надо или принять, или, если не
принимать, то можно ж и не писать в сети. Поэтому, по возможности, я стараюсь
объяснить всегда и всем свои моменты. Получается или нет - это уже другое дело.
Тем более, что это просто новогодняя история, просто захотелось красивого,
просто когда-то написалось под настроение @igg{fbicon.face.smiling.eyes.smiling} 

\begin{itemize} % {
\iusr{Анна Корнева}
\textbf{Ирина Петрова} спасибо Вам

\iusr{Raisa Prisetska}
\textbf{Ирина Петрова} Ирина, я написала реплику на Ваши слова. А не знать историю страны 25 летней давности- этоткак? Или сознание выборочно? Ленина помним , а Афганистан забыли?((. Политесс здесь не к месту - мое мнение. И оно не отрицает Ваше.
\end{itemize} % }

\iusr{Тетяна Оксюченко}
\textbf{Katya Klueva} какая вы грубиянка, фу. Не понравился рассказ, листайте дальше. "Чужого жука каждый обидеть может..."

\end{itemize} % }

\iusr{Uliya Dobreva}
Прекрасный пост! Душевный, человечный и немного волшебный! Чудеса и любовь должны править миром!

\begin{itemize} % {
\iusr{Ирина Петрова}
\textbf{Uliya Dobreva} благодарю) они и правят @igg{fbicon.wink} 
\end{itemize} % }

\iusr{Татьяна Петрюк}
Так хочется , чтобы история была взята из жизни ...,

\iusr{Ирина Петрова}
\textbf{Tatyana Petryuk} там много реальных моментов, честно @igg{fbicon.face.happy.two.hands} 

\iusr{Неля Архипова}
Спасибо.

\iusr{Lyudmila Zhuchkova}

Благодарю! @igg{fbicon.hands.pray}  Так душевно написали, не могла оторваться, на одном
дыхании... Пишите. Молодец! @igg{fbicon.thumb.up.yellow}  Всем интересно, это реальная история???

\iusr{Ирина Петрова}
\textbf{Lyudmila Zhuchkova} благодарю @igg{fbicon.face.happy.two.hands}  есть несколько реальных моментов, честно)

\iusr{Любовь Облат}
Читала иплакала, что хорошо так закончилась история. Спасибо.

\iusr{Ирина Петрова}
\textbf{Любовь Облат} спасибо за светлые слезы) история и продолжалась хорошо, поверьте...

\iusr{Oleksandr Tarashchuk}
Высший пилотаж !!!!!  @igg{fbicon.thumb.up.yellow} 

\iusr{Ирина Петрова}
\textbf{Oleksandr Tarashchuk} благодарю @igg{fbicon.face.happy.two.hands} 

\iusr{Tatyana Tumanova}
Трогательно до слез...

\iusr{Ирина Петрова}
\textbf{Tatyana Tumanova} спасибо @igg{fbicon.face.happy.two.hands} 

\iusr{Victoria Gryshchenko-Zhurba}
 @igg{fbicon.hands.applause.yellow}{repeat=3} 

\iusr{Евгения Бочковская}

\obeycr
Очень искренне ! Сижу и тоже... @igg{fbicon.sweat.droplets}{repeat=3} 
Я не просто поверила... Я ПОВЕРИЛА !!!
Зв всё спасибо - Богу, случаю, судьбе...
Счастья Вам, радости и тепла !
Спасибо Вам за искренность!
Жизнь учит нас ! И мы, уже душой,
очень ценим подарки судьбы...
И, тем более, к Новому году...  @igg{fbicon.sparkles}  @igg{fbicon.heart.sparkling}  @igg{fbicon.heart.beating} 
\restorecr

\begin{itemize} % {
\iusr{Ирина Петрова}
\textbf{Евгения Бочковская} благодарю! Очень приятно знать, что немного минут радости и светлых эмоций приносит твое творение людям, особенно тем, которых знаешь @igg{fbicon.heart.eyes}  @igg{fbicon.face.happy.two.hands} 
\end{itemize} % }

\iusr{Елена Руденко}
Не верю  @igg{fbicon.face.smiling.eyes.smiling} 
В магию @igg{fbicon.thumb.down.yellow} 
Это
Не от Бога @igg{fbicon.thinking.face} 

\begin{itemize} % {
\iusr{Ирина Петрова}
\textbf{Елена Руденко} конечно, с магией дело иметь как-то... непросто... да и лучше быть поосторожнее
\end{itemize} % }

\iusr{Наталия Маслова}

Ирина, большое Вам спасибо! Очень талантливо и даже на душе стало светлее.
Спасибо и не слушайте тех, кто пишет про сиропчики и мыльные сериалы. Это
зависть. Удачи Вам и пишите ещё!

\begin{itemize} % {
\iusr{Ирина Петрова}
\textbf{Наталия Маслова} 

спасибо) знаете, когда выпускаешь на страницы сети что либо, рассказик, очерк,
просто комментарий, то осознаешь, что ты - не доллар, всем нравиться не
можешь @igg{fbicon.laugh.rolling.floor}  если это можешь принять - тогда ОК, если можешь абстрагироваться от
нудных и дотошных - ОК, если сможешь спокойно воспринимать серьезные
доказательства того, что Деда Мороза не существует и не вступать в полемику -
тогда дважды ОК. Я не профилитератор, мне не сильно ранят сердце и тщеславие
неодобрямсы, поэтому я и выставляю свои миниэтюды. Не скрою - конечно, милые и
теплые отзывы читателей - бальзам для ранимого сердца @igg{fbicon.laugh.rolling.floor}{repeat=3} за что я Вам и
многим благодарна @igg{fbicon.heart.eyes} 

\end{itemize} % }

\iusr{Олена Кислицька}
Як зворушливо, неначе побувала в тому часі, побачила як то було...

\iusr{Ирина Петрова}
\textbf{Олена Кислицька} щиро дякую, приємні слова! @igg{fbicon.heart.eyes} 

\iusr{Marina Lavrow}
Просто дух захватило. Счастья вам всем.

\iusr{Ирина Петрова}
\textbf{Marina Lavrow} спасибо, добрые и теплые слова всегда приятны @igg{fbicon.heart.eyes} 

\iusr{Татьяна Беринцева}
Какая красивая трогательная история. Правда, до слёз. Спасибо!

\begin{itemize} % {
\iusr{Ирина Петрова}
\textbf{Татьяна Беринцева} спасибо Вам за отклик сердечный @igg{fbicon.face.happy.two.hands} 
\end{itemize} % }

\iusr{Tatiana Klotchko}
А что ж такое интернациональный долг ?

\begin{itemize} % {
\iusr{Ирина Петрова}
\textbf{Tatiana Klotchko} 

Интернациональный долг — термин, который использовало руководство СССР и других
стран социалистического блока, основываясь на принципе пролетарского
интернационализма, вводя свои войска в другие страны, как правило, для оказания
помощи в наведении порядка (подавления восстаний) или для обеспечения перехода
власти к коммунистам под предлогом отражения внешней агрессии. С точки зрения
других стран, интернациональный долг является одним из видов интервенции.

Постаралась ответить @igg{fbicon.wink} 

\iusr{Tatiana Klotchko}
\textbf{Ирина Петрова} Спасибо, знала. Абсолютно правильная точка зрения у других стран

\iusr{Ирина Петрова}
\textbf{Tatiana Klotchko} вот тут ППКС.
\end{itemize} % }

\iusr{Наталия Герус}
На одном дыхании... спасибо за эмоции

\begin{itemize} % {
\iusr{Ирина Петрова}
\textbf{Наталия Герус} спасибо Вам ! Очень приятно, что рассказик доставил радость @igg{fbicon.heart.eyes} 
\end{itemize} % }

\iusr{Людмила Петрик}

\ifcmt
  ig https://scontent-frx5-2.xx.fbcdn.net/v/t39.1997-6/s168x128/64354091_1420710664738666_6729115645658529792_n.png?_nc_cat=1&ccb=1-5&_nc_sid=ac3552&_nc_ohc=lLkPB9t-8M4AX8zmxAF&tn=lCYVFeHcTIAFcAzi&_nc_ht=scontent-frx5-2.xx&oh=1e2aa1da3e6dd28acb0e953aef63fec6&oe=6194F2B5
  @width 0.1
\fi

\begin{itemize} % {
\iusr{Ирина Петрова}
\textbf{Людмила Петрик} спасииибо! Такое новогоднее настроение!!!
\end{itemize} % }

\iusr{Татьяна Ларченко}

Ух! Мороз по коже. Почему - то такая реакция...
Спасибо за красивый рассказ !

\begin{itemize} % {
\iusr{Ирина Петрова}
\textbf{Татьяна Ларченко} спасибо) всё потом было хорошо! @igg{fbicon.wink} 

\iusr{Татьяна Ларченко}
\textbf{Ирина Петрова} да. Я поняла. И я счастлива от этого.  ☀ ️ 
\end{itemize} % }

\iusr{Наталия Бджілка}
Чудесный рассказ! Затронул до слез.

\iusr{Ирина Петрова}
\textbf{Наталия Бджілка} благодарю)

\iusr{Tetiana Denisova}
Як у казці! Дякую!

\iusr{Ирина Петрова}
\textbf{Tetiana Denisova} дякую, у житті буває, що й в казці не зустрінеш @igg{fbicon.wink} 

\iusr{Юта Кухарева}
Здорово! Хочется жить!

\iusr{Ирина Петрова}
\textbf{Юта Кухарева} спасибо! Конечно, будем жить!

\iusr{Ирина Архипович}
Потрясающая история!!  @igg{fbicon.hand.ok} Прям почувствовала!!! @igg{fbicon.face.happy.two.hands}  @igg{fbicon.hearts.two}  @igg{fbicon.heart.eyes} 

\iusr{маргарита ярославская}
Красивая сказка о любви, спасибо! @igg{fbicon.heart.red}

\iusr{Роксолана Рудяка}
Благодарю! Чудесный рассказ! С таким удовольствием и надеждой читала!

\iusr{Іллона Зейкан}
Як шкода, що через умовності можна було втратити все (

\begin{itemize} % {
\iusr{Ирина Петрова}
\textbf{Іллона Зейкан} пробачте, якщо не складно - трішечки поясніть свою думку) дякую.

\iusr{Іллона Зейкан}
\textbf{Ирина Петрова} чому не можна було самій знову прийти? Чекали ж. Маму привітати хоть з чим. Треба жити, а не чекати. Життя таке коротке

\iusr{Ирина Петрова}
\textbf{Іллона Зейкан} якщо б Ліза прийшла - що б мені було писати? @igg{fbicon.wink}  (якщо чесно - я б прийшла, та шо там - прибігла б)

\iusr{Іллона Зейкан}
\textbf{Ирина Петрова} 

так, це ж художній задум! Мені так шкода їх було - до сліз, а в кінці
посердилася на них і пораділа ) Дякую вам! Таке коротке
оповідання, а така хвиля емоцій!

\end{itemize} % }

\iusr{Ирина Иванченко}

Очень мило... Затронуло струны душевные, благодарю!

\iusr{Tamara Sologub}

Стефан Цвейг отдыхает!!! Пишите еще! @igg{fbicon.face.smiling.hearts} @igg{fbicon.hearts.two}  @igg{fbicon.heart.sparkling} 
@igg{fbicon.heart.red} @igg{fbicon.dizzy} 

\begin{itemize} % {
\iusr{Ирина Петрова}
\textbf{Tamara Sologub} смущаете) Ведь роман или повесть мне все же не по силам))) так, скорее, зарисовки из реальной жизни)
\end{itemize} % }

\iusr{Ярослав Новицький}
Спасибо

\iusr{Луцкер Изабелла}
СПАСИБО!!!!!!!!!!

\iusr{Неоніла Ганюк}
Прочла «на одном дыхании»...

\iusr{Людмила Черненко}
Спасибо

\iusr{Елена Ярчук}

Очень душевно написано! Спасибо. А я вспомнила как на работу пришли к нам
учениками два парня, один был блондин а второй брюнет покрытый сединой. Они были
очень молоды, но уже прошли Афган. Однажды рассказали малый эпизод из этой
непонятной войны. Они только оттаивали и я с удовольствием помогала им вернуться
к обычной жизни.

\begin{itemize} % {
\iusr{Ирина Петрова}
\textbf{Елена Ярчук} да, это был ужасный, трагический период жизни. Его раны болят до сих пор...молодые, полные жизни, будущего ребята уходили на гибель непонятно ради чего и кого... всем невернувшимся - вечная память...
\end{itemize} % }

\iusr{Наталья Ваховская}
Спасибо

\iusr{Татьяна Щербина}
Хочется, чтобы этот рассказ не был выдумкой!

\begin{itemize} % {
\iusr{Ирина Петрова}
\textbf{Татьяна Щербина} это не чистая выдумка, я бы и не смогла такое выдумать. Есть моменты реальной жизни, поверьте, конечно, чуть-чуть "подправленные" для связи событий. Однако, точно - жизнь продолжилась очень счастливо.
\end{itemize} % }

\end{itemize} % }
