% vim: keymap=russian-jcukenwin
%%beginhead 
 
%%file 01_06_2021.fb.bilchenko_evgenia.2.tochka_nevozvrata
%%parent 01_06_2021
 
%%url https://www.facebook.com/yevzhik/posts/3945037008864713
 
%%author Бильченко, Евгения
%%author_id bilchenko_evgenia
%%author_url 
 
%%tags 
%%title БЖ. Точка невозврата
 
%%endhead 
 
\subsection{БЖ. Точка невозврата}
\label{sec:01_06_2021.fb.bilchenko_evgenia.2.tochka_nevozvrata}
\Purl{https://www.facebook.com/yevzhik/posts/3945037008864713}
\ifcmt
 author_begin
   author_id bilchenko_evgenia
 author_end
\fi

БЖ. Точка невозврата
Хочешь меня ослепить? Давай. Вот - глаза мои. Оба. Выколи, на.
Я их протягиваю, как щёки: щуки реки Любовь.
Поколение лет-сигарет меж нами, друг мой, ещё не выкурено.
А дождь слепой бьёт о стекло, льёт и ткёт лён, льнёт в бок.
Но мы с тобой уже не покурим: ты - зрячий, а я - незрячая.
Потому и вижу я огоньки зажигалки, а ты не зришь их.
Нет более видящих, чем слепые: ведь пламя, оно горячее,
И я беру его, как Маяк - Татьяну, "вдвоём с Парижем".
Все удивляются: почему я - отныне покорна цензорам?
Всё очень просто: эта цензура - детсад по сравнению с тем, что
Пройдено мной через Рубикон, вслед за Пушкиным, вслед за Церковью.
Нельзя посадить за неуплату налогов мечту с надеждой.
Мелкие игры - уже не в счёт ввязавшемуся по-крупному.
Берегись ослепших: они, смеясь, небо несут под нёбом:
Это ноги мои вслепую идут - не по кругу, а сквозь округу, и,
Как бы ты, милый мой, не хотел, - ничего уже не вернётся.
1 июня 2021 г.

\ifcmt
  pic https://scontent-cdt1-1.xx.fbcdn.net/v/t1.6435-9/193962084_3945036938864720_250204899096924541_n.jpg?_nc_cat=101&ccb=1-3&_nc_sid=8bfeb9&_nc_ohc=vozRZsFc0HUAX97v1Zq&_nc_ht=scontent-cdt1-1.xx&oh=d9059157aa967486d57d5beb2fd2e065&oe=60DDAC43
\fi

\emph{Elena Solobaenko}

НО ОСТАНЕТСЯ СВЕТ...

\emph{Vera Romanova}

Интересный творческий лабиринт. Интересное исследование вдогонку
\enquote{Черным птицам} \enquote{Наутилуса} и \enquote{Слепая}
(\enquote{Снега}).

