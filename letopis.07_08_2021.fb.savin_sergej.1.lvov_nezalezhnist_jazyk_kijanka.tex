% vim: keymap=russian-jcukenwin
%%beginhead 
 
%%file 07_08_2021.fb.savin_sergej.1.lvov_nezalezhnist_jazyk_kijanka
%%parent 07_08_2021
 
%%url https://www.facebook.com/sergio.don.14/posts/2077882789019595
 
%%author Савин, Сергей
%%author_id savin_sergej
%%author_url 
 
%%tags jazyk,lvov,mova,nezalezhnist,ukraina
%%title Наш концерт  Незалежності збило російське мовлення під галицьким небом
 
%%endhead 
 
\subsection{Наш концерт  Незалежності збило російське мовлення під галицьким небом}
\label{sec:07_08_2021.fb.savin_sergej.1.lvov_nezalezhnist_jazyk_kijanka}
 
\Purl{https://www.facebook.com/sergio.don.14/posts/2077882789019595}
\ifcmt
 author_begin
   author_id savin_sergej
 author_end
\fi

Вчора на Личаківському цвинтарі зробив для подруги концерт Незалежності – перед
Івасюком співали його "Червону руту", перед Скориком – "Намалюй мені ніч" ,
перед Ігорем Римаруком вивчили його вірш, де янгол читає при місяці жіночу
ксерокопію.

\ifcmt
  tab_begin cols=3
		width 0.3

     pic https://scontent-cdt1-1.xx.fbcdn.net/v/t39.30808-6/233377401_2077882579019616_4719509523679180968_n.jpg?_nc_cat=103&ccb=1-4&_nc_sid=8bfeb9&_nc_ohc=KgQ14QkRbhkAX9Ub38s&_nc_ht=scontent-cdt1-1.xx&oh=15bf85aeaabdc212ffdbe9381c005fa6&oe=6117AD9D

     pic https://scontent-cdg2-1.xx.fbcdn.net/v/t39.30808-6/231224551_2077882615686279_8663379323802557531_n.jpg?_nc_cat=104&ccb=1-4&_nc_sid=8bfeb9&_nc_ohc=M7YGfagSa4gAX8oUILl&tn=lCYVFeHcTIAFcAzi&_nc_ht=scontent-cdg2-1.xx&oh=34630cbfcc28fb97103369d0e7917ad0&oe=61183717

		 pic https://scontent-cdg2-1.xx.fbcdn.net/v/t39.30808-6/227276550_2077882652352942_6252351944828745226_n.jpg?_nc_cat=102&ccb=1-4&_nc_sid=8bfeb9&_nc_ohc=yqx4_m8xdFwAX-qDBpD&_nc_ht=scontent-cdg2-1.xx&oh=d1c3e4dd7aad4b7da6286cea98767ed8&oe=61166002

  tab_end
\fi

З Грицьком Чубаєм довго говорили про його геніальну поему "Марія"(1971),
згадували Вас, як його донечку,  чудова Соломія Чубай і те, яка Ви спрагла і
чуттєва у мистецтві.

І цей весь наш концерт  Незалежності збило російське мовлення під галицьким
небом. 

Я по очах бачив, що мої (кієвлянкі! Кієвлянкі!Кієвляночки!). Довелось
познайомитись, заспівати разом "Червону руту" і лишити їх наодинці, але уже з
нашою мовою❗. Бо спершу, як почув російську на Личаківському, згадав що
нобелівський лауреат Бродський перед своїм кабінетом в американському
університеті, де викладав російську літературу,  лишив табличку :"Русские
пришли" , бо часто тоді професори жартували, мовляв рускіє прідут.

\ifcmt
  tab_begin cols=3
		width 0.3

     pic https://scontent-cdg2-1.xx.fbcdn.net/v/t39.30808-6/232245669_2077882702352937_1521701921696659955_n.jpg?_nc_cat=104&ccb=1-4&_nc_sid=8bfeb9&_nc_ohc=-xSHbPLvQUMAX8xo4Hz&_nc_ht=scontent-cdg2-1.xx&oh=073c8650df19b4d4359186e81f0b3c27&oe=61169797

     pic https://scontent-cdt1-1.xx.fbcdn.net/v/t39.30808-6/231111484_2077882752352932_8896873579345534646_n.jpg?_nc_cat=106&ccb=1-4&_nc_sid=8bfeb9&_nc_ohc=1vxryojqiuIAX-xUPz-&_nc_ht=scontent-cdt1-1.xx&oh=397a7e2a476a031b550246e7aa20ee02&oe=6117379E

		 pic https://scontent-cdt1-1.xx.fbcdn.net/v/t39.30808-6/233377401_2077882579019616_4719509523679180968_n.jpg?_nc_cat=103&ccb=1-4&_nc_sid=8bfeb9&_nc_ohc=KgQ14QkRbhkAX9Ub38s&_nc_ht=scontent-cdt1-1.xx&oh=15bf85aeaabdc212ffdbe9381c005fa6&oe=6117AD9D

  tab_end
\fi

Зранку ще веселіше стало. 

Година одинадцята. Я крізь сон чую, що в моїй хаті львівській російська мова
десь взялася. Крізь сон прокидаюся. Думаю, ще цього не вистачало. І хто такий
невіглас.

Завіса падає. На концерт до мого друга приїхала група музикантів з Києва.
Дівчина одного з них гаваріт російською. Підходжу, як чемний хлопчик, говорю
італійською – панєнка образилась. 

Я думаю: "Не перша, не остання". 

Найдурніше, що бачив у світі – це принципове не бажання говорити українською.
При мені цього бути не може. Ми не в печері живемо. В державі, як і в квартирі
має бути Конституція, щоб хата не згоріла. От, якщо ми турбуємось про країну –
не треба принципово у місті Франка і Данила приїжджати зі своїми правилами.

\ifcmt
  tab_begin cols=4

     pic https://scontent-cdg2-1.xx.fbcdn.net/v/t39.30808-6/231224551_2077882615686279_8663379323802557531_n.jpg?_nc_cat=104&ccb=1-4&_nc_sid=8bfeb9&_nc_ohc=M7YGfagSa4gAX8oUILl&tn=lCYVFeHcTIAFcAzi&_nc_ht=scontent-cdg2-1.xx&oh=34630cbfcc28fb97103369d0e7917ad0&oe=61183717

     pic https://scontent-cdg2-1.xx.fbcdn.net/v/t39.30808-6/227276550_2077882652352942_6252351944828745226_n.jpg?_nc_cat=102&ccb=1-4&_nc_sid=8bfeb9&_nc_ohc=yqx4_m8xdFwAX-qDBpD&_nc_ht=scontent-cdg2-1.xx&oh=d1c3e4dd7aad4b7da6286cea98767ed8&oe=61166002

		 pic https://scontent-cdg2-1.xx.fbcdn.net/v/t39.30808-6/232245669_2077882702352937_1521701921696659955_n.jpg?_nc_cat=104&ccb=1-4&_nc_sid=8bfeb9&_nc_ohc=-xSHbPLvQUMAX8xo4Hz&_nc_ht=scontent-cdg2-1.xx&oh=073c8650df19b4d4359186e81f0b3c27&oe=61169797

     pic https://scontent-cdt1-1.xx.fbcdn.net/v/t39.30808-6/231111484_2077882752352932_8896873579345534646_n.jpg?_nc_cat=106&ccb=1-4&_nc_sid=8bfeb9&_nc_ohc=1vxryojqiuIAX-xUPz-&_nc_ht=scontent-cdt1-1.xx&oh=397a7e2a476a031b550246e7aa20ee02&oe=6117379E

  tab_end
\fi


Кілька місяців в освітньому центрі у мене була студентка, як і всі, кияни, бо я
живу і працюю в Києві. Так у нас з нею була одна розмова : на моїх уроках лише
українська, без російськомовних вставних конструкцій. Вона стабільно це
забувала. Телефонувала, казала: "Здраствуйте", я вимикався. Телефонувала ще
раз, говорила "Добрий день" і урок починався. Це було усі наші три місяці.
Щоуроку вона забувала – щоуроку я вимикався.

Але ця принциповість неукраїнськомовності в Україні немає жодного підґрунття,
бо всі ми вростаємо в єдиний корінь – і цей корінь українська мова.

Дівчина-киянка, яка мене розбудила сьогодні – сидить зараз навпроти,
усміхається і говорить прекрасною українською. Посиділа – поображалася і
зрозуміла, що хоче бути щасливою.

А я щасливий, що жінки хочуть бути щасливими. 

І добре пам'ятаю, що українська мова одного разу стала однією з причин мого
розлучення з жінкою, яку любив по-справжньому. 

І я не шкодую. Бо навіщо бути поетом і писати книжки цією мовою, якщо твоя
жінка поруч говорить не мовою твоїх віршів? 

Коли я молюся, я прошу лише одного (не за канонічною молитвою) – аби моя жінка
була подібна на Олену Телігу.

Вона, народжена на Петербургщині, говорила з усіма українськими поетами 20-х.
рр. лише російською. До одного випадку (про нього самі прочитаєте!) Після цього
випадку вона сказала: "Тепер українська – моя мова" і написала 28 віршів лише
українською. І це прекрасні тексти. 

Тому я щороку святкую день народження Олени.

А жінка, яка вчора була зі мною на Личаківському цвинтарі – навіть в погляді
схожа з Телігою. 

\begin{verbatim}
	#ЯЛюблюТелігу
	#БезУкраїнськоїМовиСтрашноПрокидатися
\end{verbatim}

\ii{07_08_2021.fb.savin_sergej.1.lvov_nezalezhnist_jazyk_kijanka.cmt}
