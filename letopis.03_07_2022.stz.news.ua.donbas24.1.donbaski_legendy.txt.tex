% vim: keymap=russian-jcukenwin
%%beginhead 
 
%%file 03_07_2022.stz.news.ua.donbas24.1.donbaski_legendy.txt
%%parent 03_07_2022.stz.news.ua.donbas24.1.donbaski_legendy
 
%%url 
 
%%author_id 
%%date 
 
%%tags 
%%title 
 
%%endhead 

Донбаські легенди, які передаються з покоління в покоління

У кожного краю є власні легенди, які переходять з покоління в покоління,
зберігаючи власну самобутність, загадковість і унікальність

Донбас має не тільки багату історію, але й велику кількість цікавих легенд, які
вдалося зібрати завдяки статтям краєзнавців, журналістів та розповідям
мешканців. Проте деякі з легенд можуть бути справжнісінькою правдою, а не
вигадкою... Сьогодні ми можемо лише здогадуватися, що ж відбувалося насправді і
як виникла та чи інша легенда про Донбас.

Дух Шубіна

Ця містична історія про привид шахтаря Шубіна виникла ще у часи заснування
перших шахт на Донбасі. Вона має декілька версій. Перша оповідає про те, що
Шубін працював в одній із шахт спалювачем газу метан. Була така окрема
шахтарська спеціальність років 150 тому, коли ще не існувало датчиків для
виявлення метану. Тоді людина з факелом в руках спалювала газ, що збирався у
повітрі. Ця робота була дуже небезпечною, і Шубін врешті-решт загинув від
вибуху, а його дух став хранителем шахт і шахтарів. За другою версією Шубін
працював звичайним гірником разом зі своїми товаришами, і одного разу його
бригада загинула від вибуху, а вижив тільки він. Після цього Шубін не витримав
такого горя і втік під землю, після чого більше його ніхто живим не бачив. А
потім шахтарі з різних куточків Донбасу нібито бачили дивну фігуру у чорному
кожусі і з вогнем у руці, яка попереджала їх про небезпечні ділянки під землею.
Так і почали вважати, що Шубін оберігає шахтарів від смерті.

Донбаський Шубін — це ніби прототип уральської міфічної Хазяйки Мідної гори.
Серед шахтарів навіть існує повір'я, що якщо побачив Шубіна, потрібно
звільнятися з шахти поки живий і здоровий.

«Маріупольські катакомби»

За розповідями старожилів, від центральних вулиць Маріуполя під Міським садом
проходять тунелі, що виходять на Слобідку. За одними чутками, вони існували ще
з козацьких часів, за іншими — вириті перед Кримською війною 1853−1856 років.
Під час Кримської війни й обстрілу міста маріупольці ховалися в цих тунелях.
Але, де входи та виходи цих тунелів? На думку, видатного краєзнавця Аркадія
Проценка, тунелів було два. І в післявоєнні роки він сам особисто, ще в
підлітковому віці проходив по ним. Обидва мали виходи на Слобідку, а входи — в
нинішньому Старому місті. Один з тунелів дійсно проходив під Міським садом. На
жаль, і входи, і виходи в підземні тунелі були засипані будівельниками та
ремонтниками. Довше протримався другий тунель, вхід в який знаходився в одному
з дворів на вулиці Італійській, але і цей вхід зараз засипаний. Тобто тунелі й
сьогодні існують, але входів і виходів в них немає.

Камінь старця Пилипа Луганського

У Луганську шанують диякона Пилипа, якого місцеві православні віряни вважають
святим, хоча він й досі не канонізований. Пилип Єлисейович Горбенко народився
22 листопада 1858 року в селянській родині на Чернігівщині. До Луганська він
приїхав наприкінці XIX століття, прожив у місті до 1956 року, і помер у віці 98
років. Кажуть жив скромно, у келії. У часи після кривавої революції більшовиків
Пилип бував на околиці Луганська в Олександрівську, де були великий панський
маєток Юзбаша з садом і могутніми дубами, а також старовинний
Свято-Вознесенській собор, який пережив кривавий терор та працює і досі.

Легенда розповідає про те, що у 1938 році Пилип поспішав у собор, коли його
зупинили комсомольці і почали знущатися з його віри та набожності, вимагаючи
вчинити диво. Поруч з ними лежав стовбур поваленого дуба. Пилип став перед ним
навколішки і промовив молитву, а потім вдарив палицею і вимовив: «Дуб оцей
обернеться в камінь на згадку про мене і на свідчення чудес Божих, а для тих,
хто не буде вірити сказаному про мене, камінь цей буде видимим свідченням життя
мого. Бо камінь цей буде лежати до Другого Пришестя, зміцнюючи немічних у вірі.
Допоки не розквітне». За переказами у цей же момент дуб обернувся на камінь. І
хоча Пилип наказав не ворушити камінь з місця, інакше буде біда, його
переносили двічі. Вперше камінь до свого маєтку переніс місцевий бізнесмен
1990-х Карен Андреасян. У 2003 його знайшли мертвим у будівельному вагоні.
Потім дерево-камінь перевезли у Луганськ і поставили біля могили старця Пилипа
на недіючому вже кладовищі, де згодом збудували капличку та церкву. Про Пилипа
говорять, що він нібито тричі бачив Богородицю, а невідомий художник написав з
цього образу ікону. Вона стала прообразом пам'ятного знаку в центрі Луганська,
а також храму біля нього.

Легенди найстарішого цвинтаря Маріуполя

Існує легенда, що на старовинному цвинтарі Маріуполя, яке вважається недіючим,
на одному з пам'ятників чи склепів заховано унікальне і заповітне дзеркало.
Варто в нього подивитися людині похилого віку і до неї повернуться втрачені
роки, а разом з ними — надія на спробу виправити скоєні помилки. Також за
легендою завдяки дзеркалу можна повернути кохану людину, втрачений успіх і
здійснити будь-яке бажання. За іншою версією, погляд в таємниче дзеркало
приносить людині низку невдач. Серед місцевих поширена історія про маріупольця
Олега Дубодовку, якому вдалося знайти те саме дзеркало, після чого чоловіка
почало переслідувати невезіння. «Біла смуга» змінила чорну тільки після того,
як він поглянув на себе в дзеркало на могилі повторно. Також на цьому цвинтарі
є величний склеп, в якому, за однією з версій, було поховано сімейство купця
Пілічова. Оповідка свідчить, що зловмисники замкнули в склепі дочку охоронця
кладовища і вона провела там добу. Коли дівчину знайшли, вона була вже мертвою,
за іншою версією, вона втратила розум. Залишився на території кладовища і
унікальний склеп родини купця Найдьонова. Тут розташовані пам'ятники купця
Івана Найдьонова і його дружини Параски. За легендою, внизу була похована
незаміжня дочка Найдьонова. Коли злодії спробували пограбувати її могилу через
кілька днів після поховання, тіло дівчини нібито розсипалося — залишилася
тільки весільна сукню, в якій її поховали.

Легенди про Азовське море 

Далеко не всім відомо, що про Азовське море складено безліч легенд. Одна з них
присвячена двом сестрам, які жили зі старим батьком — рибалкою біля нашого
моря. Їхня мати давно померла. Старшу сестру звали Азою, а другу, меншу, —
Золотокосою Піщанкою. Сестри були такі красиві, що хто їх бачив, той про сон
забував: все про них думав. А дівчата шукали своїх суджених перебірливо, ніхто
із місцевих парубків не припадав їм до серця. Аза щодня сиділа на березі моря,
на високій кручі, та все виглядала когось. Мабуть, свого судженого, який поплив
у далекі світи і там, як переказували люди, загинув від ворожої шаблі. І ось
одного разу, коли дівчина сиділа на березі моря, трапився сильний вітер-буран.
На морі піднялися височенні хвилі. Бігли вони до берега, били в кручі й страшно
стогнали. Аж раптом відколовся від кручі великий шмат землі і разом із Азою
впав у розбурхані хвилі. Побачила це Золотокоса Піщанка — та й кинулась з гори
у море, щоб врятувати старшу сестру. Так вони обидві й загинули. Ранком
наступного дня, коли море заспокоїлося, повернувся з гостей старий рибалка,
вийшов на берег і побачив, що нема його дочок на кручі, а на тому місці, де
любила сидіти Аза, — свіжий обвал. Глянув батько униз — а там, під самою
кручею, такий золотий пісок іскриться на сонці, що аж очі засліплює! А море —
тихе-тихе і таке лагідне, як його діти... Тоді зрозумів усе нещасний та й гірко
заплакав...

З того часу і море наше почали називати Азовським, бо ж утопилася в ньому
красуня Аза. А довгих піщаних кіс у цьому морі тому так багато, що разом з Азою
втопилася її молодша сестра — Золотокоса Піщанка.

Таємнича заплакана дівчина в маєтку Мсциховського

На Луганщині і досі зберігся єдиний на Донбасі готичний замок — маєток
польського інженера Казимира Мсциховського 1840 року. Він проводив тут зі своєю
дружиною багато часу. Все село чекало на приїзд господарів як на свято, бо коли
вони були вдома з особняка лунала класична музика, бариня гуляла доглянутим
парком, сиділа з гостями в альтанці і обов'язково відвідувала хворих і бідних.
При церкві вона організувала парафіяльну школу, де іноді і сама викладала.
Через багато років, коли господарі замку вже давно померли, якось вночі сторож
через вікно побачив жінку, яка ходить будівлею із запаленою свічкою, плаче і
голосить. Жінку бачили у закритому приміщенні кілька разів, жителі вирішили, що
це сама Борогодиця плаче за колись щасливими часами. Або ж це могла бути донька
Мсциховського, яка за легендою трагічно померла. В одній з кімнат зберіглася
картина із зображенням дівчини — за легендою, це та сама донька інженера.

Маріупольський скарб Нестора Махна

Існує легенда, що батько Махно, після того, як взяв Маріуполь, нібито закопав
на вулиці Італійській, де мешкав його друг, Григорій Лепетченко, два (з трьох)
ящики білогвардійського золота.

Чи існують ті два ящики, чи ні і де вони закопані, — про це історики та
краєзнавці сперечаються й досі. Як втім і про те, на якій будівлі в місті
знаходиться балкон, з якого Нестор Махно звертався до жителів Маріуполя в 1919
році. Є версія, що цей балкон знаходиться на будівлі, в якому розташовувалася
редакція газети «Приазовський робочий». А загублений скарб так і не був ніким
знайдений.

Нагадаємо, раніше Донбас24 розповідав про подробиці перебування «Тайри» у
російському полоні.

ФОТО: з відкритих джерел.
