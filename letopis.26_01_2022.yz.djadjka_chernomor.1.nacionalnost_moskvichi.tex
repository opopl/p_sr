% vim: keymap=russian-jcukenwin
%%beginhead 
 
%%file 26_01_2022.yz.djadjka_chernomor.1.nacionalnost_moskvichi
%%parent 26_01_2022
 
%%url https://zen.yandex.ru/media/chernomorr/est-takaia-nacionalnost--moskvichi-61f11edb11e9eb013e73080e
 
%%author_id yz.djadjka_chernomor
%%date 
 
%%tags moskva,moskvichi,rossia
%%title Есть такая национальность – москвичи!
 
%%endhead 
 
\subsection{Есть такая национальность – москвичи!}
\label{sec:26_01_2022.yz.djadjka_chernomor.1.nacionalnost_moskvichi}
 
\Purl{https://zen.yandex.ru/media/chernomorr/est-takaia-nacionalnost--moskvichi-61f11edb11e9eb013e73080e}
\ifcmt
 author_begin
   author_id yz.djadjka_chernomor
 author_end
\fi

Готов поспорить о том, что москвичи — это отдельная нация в составе России.
Пусть такой нации нет в мировом списке, но в нашем отечественном она на самом
первом месте. Это высшая российская раса, такая как, например было прежде белое
население в ЮАР. И их процент по отношению к остальному населению примерно тот
же – 10\%.

Москвичей отличает от других рост – два с половиной метра, повальная
способность к телекинезу, и голубая кровь в розовых жилах. Это и выводит их на
первое место среди остальных россиян.

\ii{26_01_2022.yz.djadjka_chernomor.1.nacionalnost_moskvichi.pic.1}

Стоп! Вот тут моя восьмилетняя внучка подсказывает мне, что я ошибаюсь, и что
москвичи вполне обычные люди, которые просто живут в Москве.

\enquote{Позвольте сударыня, - говорю я ей, - мала ты, чтобы подсказывать деду}.
Согласно непроверенной статистике, в отличие от остальных россиян, 48\%
процентов москвичей никогда не видели живой коровы, считают, что Подмосковье –
это московская подземка, а Турция – это дачный посёлок для летнего отдыха. \enquote{Это
всем известный факт, правда, - добавляю я, - не в этом заключается главное их
отличие от простых смертных}.

Конечно, не все москвичи одинаковы. Тут приходится делить их на коренных
москвичей и приезжих. Хотя, можете мне верить или не верить, но любой человек
перебравшийся в Москву, сразу становится сверхчеловеком, такова уж волшебная
сила здешней земли. Гордый джигит из аула сразу становится бессмертным горцем
из известного голливудского сериала – всемогущим и непобедимым. Привыкший к
горному серпантину он лихо водит маршрутки по малознакомым улицам,
безбоязненно, хотя не безопасно. Хорошо владеющий кетменем садык из далёкого
кишлака готовит строительный раствор быстрее бетономешалки хотя, может быть, и
не так качественно. А больше всего среди них виртуозов в совершенстве владеющих
метлой и лопатой. Этого факта уж никто оспорить не посмеет.

Скажу вам по секрету, что на самом деле этот слой переселенцев всего лишь
переходный от простого человека к настоящему москвичу. И здесь существует
только два варианта дальнейшей эволюции. Те, кому удаётся получить российское
гражданство и московскую прописку становится всё-таки не всамделишними
москвичами, а лишь простыми россиянами проживающем в Москве, но отличаются
особенностью к быстрому размножению: они начинают подтягивать в Москву своих
многочисленных родственников, которые вскоре тоже становятся в том же ряду
претендентов на небожительство. И их уже реально очень много, а Москва-то не
резиновая, поэтому властями регулярно принимаются новые законы и правила,
ограничивающие их безудержное размножение.

\ii{26_01_2022.yz.djadjka_chernomor.1.nacionalnost_moskvichi.pic.2}

Другие возвращаются в родные кишлаки и аулы к своим семьям, где их считают
кастой просветлённых, побывавших в заповедных местах, и они периодически
выезжают туда, как паломники, для поддержания своего высокого статуса.

Но не о них речь, то есть о них тоже, конечно, но всё же совсем не о них. Тьфу
ты! Совсем запутался.

\enquote{Настоящие москвичи – это мы родившиеся в Москве,} - поддерживает запутанный
разговор мой тринадцатилетний внук.

\enquote{Усы ещё не подросли а уже споришь с дедом. Паспорт с московской пропиской
сперва получи, потом получишь право голоса}. Разница в том, что настоящим
москвичам не пристало трудится за тридцать-сорок тысяч рублей. Для этого как
раз и существуют те самые приезжие, проживающие в Москве. Настоящие москвичи
трудятся на должностях, где ценится умение правильно подшивать в папочку
бумажки, предварительно созданные ими же в компьютере, знание документооборота
и цифровых технологий или, на худой случай, умение работать на заковыристом
высокотехнологичном оборудовании. Но уже за другие деньги. И такой работы в
Москве – хоть пруд пруди. Вот!

Всем от мала до велика известно, что москвичи – это уже не те люди, кто за
подобную работу, где-нибудь в Ульяновске получает вдвое меньшую зарплату. Даже
неработающие пенсионеры в Москве почти вдвое обеспеченнее провинциальных
пенсионеров. Уж таков высокий статус москвича! И с этим не поспоришь!

\enquote{А вот и поспорю}, - отвечает мне моя сорокалетняя дочь, считающая себя
настоящей москвичкой, ввиду двадцатипятилетнего стажа проживания в Москве.
\enquote{Здесь рубль совсем другой. То есть с виду он, конечно, такой же точно, но по
содержанию сильно отличается}.

Собственно, примерно также как и доллар в Москве от доллара в Нью-Йорке. По
своей покупательской способности. В Нью-Йорке ты с двумя тысячами дохода почти
нищий, а в Москве – вполне обеспеченный человек. В Ижевске ты до работы и
обратно доберёшься за полчаса по времени и полтинник по стоимости, ещё и сдача
останется. А в Москве за три часа по времени и за сто пятьдесят по деньгам не
всегда уложишься. Хотя дорогу до работы в трудовой стаж не засчитывают, но и к
свободному времени его тоже вроде как не отнесёшь.

И жильё в Москве за те же деньги что в Йошкар-Оле, или даже Казани не купишь и
не снимешь. Разница будет, пожалуй, поболее чем в проезде. Примерно раза в три.
Да и другие расходы тоже.

Вот так, втроём, они разрушили мою глубокую веру в особое предназначение
москвичей. Хотя! Наверное онр правы. В Москве находятся большинство лучших, да
и не лучших тоже, ВУЗов России, здесь же наибольшее число учёных на квадратный
метр площади, 100\% федеральных министров и их заместителей, и больше половины
крупных чиновников всей нашей необъятной Родины. Это о чём-то же говорит.

А вот олигархи и крупные бизнесмены бывают в Москве только проездом, а жить они
предпочитают в Лондоне, на Лазурном берегу, или в Испании, хотя Кипр или Мальта
тоже сгодятся. Такая она, святая московская земля – не терпит нечестно нажитых
денег, правда сначала позволяет их нажить.

\ifcmt
  ig https://avatars.mds.yandex.net/get-zen_doc/4697586/pub_61f11edb11e9eb013e73080e_61f13c08c41680325201da50/scale_1200
  @caption Картинка с сайта kaifolog.ru
	@wrap center
	@width 0.7
\fi

\enquote{А вот взять бы да уравнять всех жителей России в возможностях} – высказал я
вслух крамольную мысль.

\enquote{Папа, Вы, наверное, перегрелись сидя у батареи, выйдите на балкон и
остудитесь! - вмешался в спор мой зять – урождённый москвич. Если вы хотите,
нижегородцев попытаться подтянуть по качеству жизни к москвичам, то, уверяю
Вас, что получится все ровно наоборот, это уровень жизни в Москве скатится до
уровня жизни в Саратове. Поэтому оставьте в покое Москву, как образец, к
которому надо тянуться}.

И ведь действительно, он прав! Пусть те, у кого к этому есть знания и
способности перебираются в Москву и добиваются карьерного роста или
коммерческого успеха, а менее квалифицированную и престижную работу выполняют
те у кого знаний и навыков для ответственной работы недостаточно. Не будем
завидовать москвичам, оказывается их жизнь тоже не прогулка в райском саду.

Может быть у кого-то другое мнение по этому поводу?

ПОДПИСЫВАЙТЕСЬ и будем спорить до хрипоты.

Или хотя бы поставьте лайк за глубокое научное исследование актуальной
проблемы.
