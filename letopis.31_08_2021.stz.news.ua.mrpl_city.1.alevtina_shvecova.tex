% vim: keymap=russian-jcukenwin
%%beginhead 
 
%%file 31_08_2021.stz.news.ua.mrpl_city.1.alevtina_shvecova
%%parent 31_08_2021
 
%%url https://mrpl.city/blogs/view/moe-osobiste-mistse-siliporuch-z-dorogimi-ta-kohanimi-lyudmi
 
%%author_id demidko_olga.mariupol,news.ua.mrpl_city
%%date 
 
%%tags 
%%title "Моє особисте місце сили – поруч з дорогими та коханими людьми!"
 
%%endhead 
 
\subsection{\enquote{Моє особисте місце сили – поруч з дорогими та коханими людьми!}}
\label{sec:31_08_2021.stz.news.ua.mrpl_city.1.alevtina_shvecova}
 
\Purl{https://mrpl.city/blogs/view/moe-osobiste-mistse-siliporuch-z-dorogimi-ta-kohanimi-lyudmi}
\ifcmt
 author_begin
   author_id demidko_olga.mariupol,news.ua.mrpl_city
 author_end
\fi

\ii{31_08_2021.stz.news.ua.mrpl_city.1.alevtina_shvecova.pic.1}

Цього разу я хочу познайомити читачів ближче з неймовірно позитивною і відомою
маріупольчанкою, чия посмішка здатна вмить підняти настрій. Її програма \emph{\textbf{\enquote{Місце
сили}}} стала чудовим туристичним путівником для багатьох мешканців і гостей
Маріуполя. Вона душа і серце Маріупольського телебачення, надзвичайно
талановита ведуча та креативна блогерка, яка просто обожнює свою роботу та
виконує її на дуже  професійному рівні. Нескладно здогадатися, що це чарівна
\emph{\textbf{Алевтина Швецова}}, яка розкрила мені багато унікальних і дуже цікавих фактів зі
своєї яскравої біографії...

\ii{31_08_2021.stz.news.ua.mrpl_city.1.alevtina_shvecova.pic.2}

Народилася Алевтина у Маріуполі і насправді вже не уявляє свого життя без
рідного міста. Вона змалечку жила  на 23 мікрорайоні, бачила його на різних
етапах розвитку і дуже любить та всім серцем вболіває за свій район. У школі
дівчина навчалася на \enquote{відмінно}. Любила  читати. Завдяки педагогічному таланту
класної керівниці \emph{\textbf{Соф'ї Петрівні Чапни}} улюбленим предметом була математика.
Після 9 класу Аля вступає в індустріальний технікум, де отримувала стипендію,
що дозволило стати трохи незалежнішою.  З дитинства ведуча ходила до різних
гуртів, але ніде довго не затримувалася. Лише у свідомому віці, у 16 років,
почала ходити на Тхеквондо, де навіть отримала білий пояс.  Цікаво, що Алевтина
не вважає себе творчою людиною. Вона наголошує, що живе і насолоджується  своїм
життям без якихось суперздібностей, просто така як є. І все ж є декілька
факторів, що істотно  вплинули на нашу героїню. Перший – це любов до моря, яка
виникла у Алі з дитинства завдяки мамі. Вони з матусею починали  сезон  з
квітня і закінчували у жовтні. Приїжджали, коли зовсім не було людей і
засмагали та відпочивали. Ведуча з теплом згадує ці поїздки і дуже вдячна мамі
за такі незабутні враження.

\ii{31_08_2021.stz.news.ua.mrpl_city.1.alevtina_shvecova.pic.3}

Наразі Алевтина свою любов до моря  передає власному синочку. Другий фактор,
пов'язаний з роботою мами Алі. Оскільки матуся працювала продавчинею, у дівчини
була можливість багато спілкуватися з покупцями, адже вона інколи відправлялася
на роботу разом з мамою. Часто спілкувалася і з бабусями, і з вантажниками та
іншими покупцями. Загалом таке спілкування посприяло розвитку навички
комунікації. І дійсно Алевтина зовсім не боїться спілкуватися з незнайомими
людьми, навпаки вона здатна в співбесіді досить швидко розговорити будь-кого.
Варто лише згадати, як Аля вела програму \enquote{Ранок}, де чарівна ведуча підкорювала
гостей своєю щирістю і відкритістю. Попри те, що мама працювала продачинею, все
ж сім'я у Алевтини складається суто з металургів. І батьки, і бабусі та дідусі
працювали на комбінаті ім. Ілліча. Це вплинуло на те, що маріупольчанка
спочатку бачила своє майбутнє лише на заводі. У 2005 році, коли Аля поверталася
додому з індустріального технікуму, вона побачила  оголошення про набір до
театральної студії. Раніше дівчина ніколи не уявляла, що наважиться спробувати
свої сили на сцені, але подумала, чому б і ні та вирішила піти на
прослуховування. Конкурс був досить великий, охочих потрапити до Студії
виявилося чимало.

Ала подумала, що потрібно підготувати щось одне:  вірш, пісню чи байку. І
підготувала лише вірш В. Маяковського \enquote{Вірші про радянський паспорт}. Як
виявилося, дівчина переплутала –  підготувати потрібно було все. І коли
\emph{\textbf{Костянтин Володимирович та Анжеліка Арганівна Добрунови}}, яким сподобалося
декламування вірша,  спитали, що вона буде співати, креативна Аля не
розгубилася. Обрала пісню з Фабрики зірок (яку часто тоді наспівувала), а байку
– \enquote{Ворона і лисиця} (розповідала, що пам'ятала). Сміливість і завзятість
Алевтини не могли не вразити Добрунових, завдяки чому вона потрапила до
основного складу Театральної студії. Заняття Алі дуже подобалися, вона з
нетерпінням чекала наступних та дуже ретельно почала вивчати особливості
театрального дійства. Ця театральна студія допомогла зрозуміти Швецовій, що
життя може бути набагато цікавішим і не обмежуватися заводом, є ще й творчі
професії. Під час навчання вона потоваришувала з \emph{\textbf{Мариною Леонович}}, яка у 2010
році запропонувала Алі спільно провести кінофестиваль, на що дівчина
погодилися. Це був кінофестиваль \enquote{Відкрита ніч}. Алевтина дуже вдячна
Костянтину та Анжеліці Добруновим за те, що вони подарували їй творчі навички,
прищепили любов до театру та допомогли зрозуміти саму себе і власний потенціал.
Та незважаючи на таку закоханість у творчість, після закінчення індустріального
технікуму на червоний диплом, Ала вступила до ПДТУ  на машинобудівний
факультет. Але саме тоді сталася  переломна подія в житті дівчини. Вона
дізналася, що група \emph{\textbf{Quest Pistols}}, творчістю якої Алевтина захоплювалася, у
Києві записує кліп на пісню \enquote{Я твой наркотик} і запрошує всіх охочих дівчат
взяти участь. Аля вирішила, що треба їхати і разом з подругами відправилися до
столиці. Від рідних цю авантюру Швецова приховала, адже всі гроші, що бабуся
давала на навчання, популярна ведуча витратила. І все ж воно було того варте.
Поїздка подарувала неймовірні враження. Крім участі в кліпі, Алевтина
познайомилася з відомими українськими виконавцями та потоваришувала з багатьма
яскравими людьми. Щодо Маріупольського телебачення, Аля завжди любила цей
канал. З дитинства дивилася програму \enquote{М+}. Декілька разів її запрошували в
ефір, де питали, як можна бути такою позитивною. І все ж дівчина мріяла побути
в ролі ведучої, але як це зробити, не знала, тому вважала цю мрію
нездійсненною. Під час навчання в університеті маріу  польчанка працювала
адміністраторкою в суші-барі. Згадує з посмішкою, як говорила відвідувачам
привітним голосом: \emph{\enquote{Администратор Аля. Самурай! Здрастуйте!}}. Університет Аля
закінчувала у 2013 році вагітною. За перерозподілом до Володарська працювати не
пішла, народила синочка і відправилася у декрет. Коли маленькому виповнилося 7
місяців, Аля почала активно  займатися  у спортзалі. Якось під час тренувань
вона зустріла знайому, яка шукала тренера, Алевтина запропонувала позайматись з
нею. Так, коли Глібчику виповнилося 1,5 рочки, вона почала працювати
фітнес-тренеркою. Дуже ретельно готувалася до кожного заняття.

\ii{31_08_2021.stz.news.ua.mrpl_city.1.alevtina_shvecova.pic.4}

На Маріупольське телебачення Алетина потрапила випадково. Вона почала
товаришувати з \emph{\textbf{Марією Фрей}}, на той час ведучою програми \enquote{Ранок в прямому
ефірі}, яка запропонувала Алі спробувати свої сили в програмі, оскільки сама
переїжджала до Києва. Алевтина погодилася. Їй дали випробувальний термін
–тиждень. Вона згадує, що це був шалений час: спала декілька годин, готувала
дуже багато матеріалів. Зараз Аля вдячна своєму колезі \emph{\textbf{Денису Мініну}}, який
підтримав її та дуже допомагав на перших ефірах. Повірила в Алевтину і
редакторка програми \emph{\textbf{Олена Ткачова}}. Та все ж перші 2 роки, дівчина не вірила, що
мрія здійснилася. Їй здавалося, що будь-якої миті  директор телеканалу
повідомить, що знайшли іншу ведучу. На думку Алі, саме ця невпевненість в собі
дозволяла їй не розслаблятися і бути постійно в тонусі, працювати над власними
помилками та самовдосконалюватися.

\ii{31_08_2021.stz.news.ua.mrpl_city.1.alevtina_shvecova.pic.5}

Завжди усміхнена Алевтина Швецова підкреслює, що бувають різні ситуації і
позитивною залишатися не завжди вдається, але всі посмішки і позитивна
енергетика – це віддзеркалена вдячність всім людям, які дарують їй величезне
натхнення. Своєму коханому Аля дуже вдячна за постійну  підтримку, розуміння і
натхнення. Синочок часто поруч з нею на заходах, інколи це нелегко, але і
дитина розширює кругозір і Алевтині спокійніше. Цікаво, що коли Алі було 20
років, у неї народився братик. Різниця між її братом та синочком становить 4
роки і наразі вони є гарними друзями. У ведучої я велика любов до закинутих
будівель. Це захоплення не розділяє синочок, але Алевтина сподівається, що все
ж в майбутньому Гліб любитиме своє місто, як і вона.

\ii{31_08_2021.stz.news.ua.mrpl_city.1.alevtina_shvecova.pic.6}

Від своєї програми \enquote{Місце сили} Аля в захваті. Найбільше їй подобається шукати
нову локацію та спікерів, сам процес пошуку її дуже захоплює, знаходити щось
неймовірне та найбільш особливе. Вона вдячна Маріупольському телебаченню за
таку унікальну програму, до якої може долучитися і кожен маріуполець,
відкриваючи всім власні місця сили. Особисте місце сили Алевтини – поруч з
дорогими та коханими їй людьми.

\ii{31_08_2021.stz.news.ua.mrpl_city.1.alevtina_shvecova.pic.7}

\emph{\textbf{Хобі:}} 

\begin{quote}
колекціонувати враження. Для Алевтини важливо сфотографувати місце, де
вона була, тоді це обов'язково закарбується в пам'яті. Загалом сфотографувати
чи зробити відео - це для Алі зберегти емоції та враження, які вона пережила,
тому цей процес вона обожнює.
\end{quote}

\emph{\textbf{Улюблена книга:}} 

\begin{quote}
\enquote{Щиголь} американської письменниці Донни Тарт, \enquote{Ті, що співають
у терні} австралійської письменниці Коллін Мак-Каллоу..
\end{quote}

\emph{\textbf{Улюблений фільм:}} 

\begin{quote}
\enquote{Мовчання ягнят} (1991).
\end{quote}

\emph{\textbf{Улюблені блогери:}} 

\begin{quote}
Супер Сус та Дмитро Громов – київські блогери, які знімають
потужні блоги про закинуті будівлі. Алю найбільше надихає приклад Дмитра
Громова, адже йому вдалося поєднати улюблену справу зі своєю професією.
\end{quote}

\emph{\textbf{Побажання маріупольцям:}} 

\begin{quote}
\em\enquote{Добра, адекватності та розуміння! Потрібно намагатися
уникати незрозумілих ситуацій (наприклад, не кричати марно на дітей через те,
що їхній м'яч залетів на клумбу) намагатися говорити одне з одним і шукати
ключ до розуміння в будь-якій ситуації}.
\end{quote}
