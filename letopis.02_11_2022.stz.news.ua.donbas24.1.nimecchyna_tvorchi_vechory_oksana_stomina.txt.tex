% vim: keymap=russian-jcukenwin
%%beginhead 
 
%%file 02_11_2022.stz.news.ua.donbas24.1.nimecchyna_tvorchi_vechory_oksana_stomina.txt
%%parent 02_11_2022.stz.news.ua.donbas24.1.nimecchyna_tvorchi_vechory_oksana_stomina
 
%%url 
 
%%author_id 
%%date 
 
%%tags 
%%title 
 
%%endhead 

Ольга Демідко (Маріуполь)
02_11_2022.olga_demidko.donbas24.nimecchyna_tvorchi_vechory_oksana_stomina
Маріуполь,Україна,Мариуполь,Украина,Mariupol,Ukraine,Німеччина,Сергій Жадан,Оксана Стоміна,date.02_11_2022

В Німеччині проходять творчі вечори, присвячені Маріуполю (ФОТО)

Оксана Стоміна на творчих вечорах у Німеччині розповідає про трагедію Маріуполя та маріупольців

31 жовтня у Білефельді відбулася творча зустріч з відомою маріупольською
поетесою і прозаїком, лауреаткою літературної премії ім. Юрія Каплана,
літературної премії «Слов'янські традиції», громадською діячкою, засновницею
громадської організації «Паперові сходи» та волонтеркою Оксаною Стоміною і
спільний літературно-музичний вечір, на якому українські і німецькі актори та
акторки читали твори Оксани і новоспеченого лауреата німецької літературної
премії миру Сергія Жадана.

Читайте також: Маріупольські художниці представили нову виставку в Дортмунді
(ФОТО)

«Я була вражена кількістю гостей і якістю організації зустрічі, щирою
небайдужістю, глибокою увагою, підтримкою кожного, хто долучився до події. Я
дякую німкеням, які плакали, наче йдеться про своє... Дякую їхнім чоловікам, які
ловили кожне слово. Дякую за всі розкуплені книжки, бо ці гроші відправляться
на добру й важливу справу!», — наголосила Оксана. 

Після виїзду з України до Німеччини Оксана за допомогою своїх віршів
намагається за кордоном розповідати про те, що відбувається зараз з кожним
українцем, адже вони написані з безпосереднього досвіду та переживань. Протягом
місяця від початку повномасштабного вторгнення рф в Україну жінка принципово
залишалася в Маріуполі, аби у міру сил допомагати військовим захищати місто й
цивільним — вижити в ньому в оточенні і під щільними обстрілами. Дмитро
Паскалов, чоловік Оксани, 16 березня наполіг, щоб вона поїхала з Маріуполя, при
цьому буквально змусив її сісти в машину.

Поетеса виїхала з Маріуполя з мінімумом речей, з малесеньким, на п'ять літрів,
наплічником. Наразі Дмитро у російському полоні. Він один із в'язнів
металургійного заводу «Азовсталь», був начальником азовстальського порту та
займав дуже активну громадську позицію. Дмитро брав участь у багатьох
загальноміських проєктах у Маріуполі. Оксана не знає, що наразі з її чоловіком,
в яких умовах він знаходиться. На творчому вечорі маріупольчанка розповідала
свою історію та історію всіх маріупольців та загалом українців через власні
вірші.

Читайте також: Маріупольська арт-резиденція PostMost відновлює свою діяльність
(ФОТО)

На творчому вечорі виступали музиканти та актори, серед яких була і
маріупольська актриса Ольга Самойлова. Для неї ця подія була водночас і
приємною, і дуже важкою.

«Після сьогоднішнього творчого вечора я прийшла геть виснажена з того, що я
весь час стримувала сльози и бажання кричати, а ця тендітна жінка їздить по
різних містах Німеччини і проводить такі творчі вечори і розповідає свою
історію, історію нашого міста її очами, вона живе і плаче віршами, як вона сама
каже. Чекає свого чоловіка з полону, вірить в світле майбутнє і посміхається!
Мені дуже приємно було обійняти її і мати за честь прочитати декілька її
віршів!», — розповіла Ольга.

Читати вірші актрисі було досить непросто, адже говорити про те, що так сильно
болить можуть далеко не всі.

Читайте також: Як у Німеччині створюються футболки, присвячені Маріуполю —
деталі

Також вірші читали німецька акторка Brit Dehler і українські митці театральної
і музичної сцени Наталія Ховда, Найра Арзуманян та Василь Хотсько. Всі вони
посприяли створенню особливої атмосфери та цілісного емоційного дійства.

«Поетеса і письменниця з Маріуполя Оксана Стоміна довела вчора на вечорі, що її
літературний голос не завмер від пережитого, а може ставати жарким полум'ям,
яке запалює інших на українську тему», — зазначила Олена Туров. 

«Це була зустріч, сповнена сліз, печалю, але й надій та сподівань на нашу
перемогу!», — підкреслила Ірина Гринько.

Також на творчому вечорі можна було почути музику українського письменника та
автора-виконавця власних пісень В'ячеслава Купрієнка і побачити світлини Євгена
Сосновського, які йому вдалося вивезти з Маріуполя.

Раніше Донбас24 розповідав, як маріупольські художниці представили нову
виставку у Дортмунді.

Ще більше новин та найактуальніша інформація про Донецьку та Луганську області
в нашому телеграм-каналі Донбас24.

ФОТО: з відкритих джерел.
