% vim: keymap=russian-jcukenwin
%%beginhead 
 
%%file 18_01_2022.fb.miheev_vladislav.1.pafos_tomos_bitva
%%parent 18_01_2022
 
%%url https://www.facebook.com/vladislav.mikheev.5/posts/4870819762984760
 
%%author_id miheev_vladislav
%%date 
 
%%tags adekvatnost,bitva,pafos,politika,poroshenko_petr,tomos,ukraina,zelenskii_vladimir
%%title Зепо и чай: о битве пафоса с томосом
 
%%endhead 
 
\subsection{Зепо и чай: о битве пафоса с томосом}
\label{sec:18_01_2022.fb.miheev_vladislav.1.pafos_tomos_bitva}
 
\Purl{https://www.facebook.com/vladislav.mikheev.5/posts/4870819762984760}
\ifcmt
 author_begin
   author_id miheev_vladislav
 author_end
\fi

Зепо и чай: о битве пафоса с томосом

Когда-то нью-йоркский раввин замечательно сказал о войне Ирака с Ираном:

- Я искренне желаю победы обеим сторонам!

И, интеллигентно завершив дискуссию, попросил чашечку чаю...

Конечно, \enquote{украинские интеллектуалы} считают себя  умнее нью-йоркского
раввина, поэтому им непременно надо выбрать: за По или за Зе?! 

Логика самопонимания и самоопределения в нашей стране должна  быть традиционно
не отличима от самоидиотизации.

Иначе какие-же они после этого \enquote{украинские интеллектуалы}?

Ситуация напоминает незамысловатый советский анекдот, героя которого непременно
пороли, потому что он каждый раз неправильно отвечал на вопрос: за белых он или
за красных?  Когда в дверь опять постучали, он по привычке снял портки и
наклонился. А это просто соседка зашла предложить чайку попить... 

Логика гражданской войны, даже если она латентная, не только опасна. Она ещё и
не отличима от глупости и позора.

Поэтому в ответ на предложение подискутировать и самоопределиться в битве
ганьбы с позором и пафоса с томосом, просто пригласите соседку в гости,
заварите хорошего чаю... Ну, чтоб занять полость рта чем-то действительно
полезным и важным. 

В идеале, попивая чай, можно вместо дискуссии с \enquote{украинскими
интеллектуалами} вслед за ньюйоркским раввином послушать Жиля Делеза: 

\enquote{У настоящего философа очень мало вкуса к дискуссиям.  Услышав фразу
\enquote{давайте подискутируем}, любой философ убегает со всех ног. Спорить
хорошо за круглым столом, но философия бросает свои шифрованные кости на совсем
иной стол. Самое малое, что можно сказать о дискуссиях, это что они не
продвигают дело вперед, так как собеседники никогда не говорят об одном и том
же. Какое дело философии до того, что некто имеет такие-то взгляды, думает так,
а не иначе, коль скоро остаются невысказанными замешанные в этом споре
проблемы? А когда эти проблемы высказаны, то тут уж надо не спорить, а
создавать для назначенной себе проблемы бесспорные концепты. Коммуникация
всегда наступает слишком рано или слишком поздно, и беседа всегда является
лишней по отношению к творчеству}.

Ага, вот теперь понятно, почему украинское дело годами не продвигается вперёд.
Почему у нас нет ни настоящей философии, ни созидания, ни творчества?

Нам не до этого - \enquote{украинские интеллектуалы} заняты другими важными
вещами - \enquote{интеллектуальными дискуссиями}...

Предлагаемый нынче выбор -  это не выбор между Зе и По. Это выбор между
\enquote{зепо} и адекватностью. 

Поэтому лично я выбираю чай, реальное дело и творчество. При этом искренне
желаю в битве пафоса с томосом победы обеим сторонам.

\ii{18_01_2022.fb.miheev_vladislav.1.pafos_tomos_bitva.cmt}
