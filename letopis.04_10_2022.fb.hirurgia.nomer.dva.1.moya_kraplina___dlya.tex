%%beginhead 
 
%%file 04_10_2022.fb.hirurgia.nomer.dva.1.moya_kraplina___dlya
%%parent 04_10_2022
 
%%url https://www.facebook.com/permalink.php?story_fbid=pfbid0zoHf9hhTBsDxevMudfh35YUoU6tLMmzzSG33rt9U3Zz7YGdQwyaehaSWSoyTQFSVl&id=100025547674356
 
%%author_id hirurgia.nomer.dva
%%date 04_10_2022
 
%%tags 
%%title Моя краплина є для когось морем
 
%%endhead 

\subsection{Моя краплина є для когось морем}
\label{sec:04_10_2022.fb.hirurgia.nomer.dva.1.moya_kraplina___dlya}

\Purl{https://www.facebook.com/permalink.php?story_fbid=pfbid0zoHf9hhTBsDxevMudfh35YUoU6tLMmzzSG33rt9U3Zz7YGdQwyaehaSWSoyTQFSVl&id=100025547674356}
\ifcmt
 author_begin
   author_id hirurgia.nomer.dva
 author_end
\fi

«Не потрібно думати, що це зробить за тебе хтось інший»

«Моя краплина є для когось морем» 

Саме з такими думками сьогодні прийшли здавати свою дорогоцінну кров в наш час
співробітники та студенти університету ім. Олеся Гончара та університету
внутрішніх справ. 

На чолі з генеральним директором Сергій Риженко медичним директором Сергій
Тимчук заступником медичного директора з медсестринства Natalija  Shulginaу
Дніпропетровська Обласна Клінічна Лікарня ім. І.І. Мечникова 4 жовтня пройшов
День донора. 

Центр хірургії гнійно-септичних ускладнень дуже часто робить переливання крові
та плазми  пацієнтам, особливо пораненим солдатам. За місця в середньому
використовується до 300х флаконів. Від імені всіх пацієнтів хочемо безмежно
подякувати людям, котрі дарують надію і можливість іншим людям жити. А нам,
медикам, продовжувати рятувати їхні життя. Низьким Вам уклін!

