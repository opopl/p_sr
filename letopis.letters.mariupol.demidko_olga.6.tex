% vim: keymap=russian-jcukenwin
%%beginhead 
 
%%file letters.mariupol.demidko_olga.6
%%parent letters.mariupol.demidko_olga
 
%%url 
 
%%author_id 
%%date 
 
%%tags 
%%title 
 
%%endhead 

... а щодо Бурова, Сергія Давидовича. Це - людина-легенда, до речі ваша людина
легенда Маріуполя.  Я от киянин його не знав, але зробив книжку гарну. Бо це
справа України! Так! Справа України - зберегти творчу спадщину Маріуполя!  А Ви
- корінна маріупольчанка - його знали персонально, виступали на його передачах,
але Вам ліньки навіть чітко вчасно написати номер НП і перевірити роботу!!!  Вам
ваш Буров Сергій Давидович - Царство йому Небесне і пам'ять про нього взагалі
потрібна чи ні????  Бляха муха!!! Ви що думаєте я Бурова писав місяць - кожен
день - пост за постом - задля того щоби вразити Вас чи що???  Що типу Оля
подивиться на мою роботу і типу краще буде про мене думати чи що???  Або ж типу
показати що я крутіший за Олю чи що???  Да нічого подібного!!! ... Я це роблю
задля України і задля її перемоги!!! Задля відродження Маріуполя!!! Бо
Маріуполь вже теж моє місто, хоча я там ніколи й не був!!! І ваші роботи я
записую тому що Ваші старі роботи - дуже талановиті і красиві!! - це теж
невід'ємна частина культурної спадщини Маріуполя яку потрібно обов'язково
зберегти а не через якісь персональні штуки!!! І от знаєте бісять мене бісять
реально ось такі штуки, коли реально якісь прості елементарні  речі - як от
подивився - почитав - зробив відгук - надіслав назад із списком своїх зауважень
та пропозицій - а це виявляється неможливо зробити через відсутність
нормального людського контакту і взаєморозуміння !! І тут моя совість чиста, бо
я вже зробив все що міг щоби налагодити якесь нормальне взаєморозуміння!  І це
якраз Ви морозитесь на повну що з Вами навіть складно просто зустрітись і
побалакати за чашкою кави хоча в одному і тому ж місті живемо! І Ви ж
науковець, кандидат історичних наук, автор
офігезної серьозної монографії начебто серьозна людина, а ведете себе зараз як
якесь примхливе дівчисько!!! Типу нехай отой Саша зачекає пару днів, а я
подумаю, чи давати йому свою адресу НП! Або ж типу як я гарно йому cказала "ні"
Фуууууууууу Бурову було б соромно за Вас, от чесно!!  Війна ж іде, кожен день
гинуть люди на війні, а на війні культурно-інформаційній - цифрова спадщина
минулого Маріуполя зникає день за днем, - ось наприклад - сайту газети
Приазовский Рабочий - його взагалі вже немає!!! Немає сайту Телеканалу Сігма!!
Немає вже сайту Бібліотеки Короленка старого!!! Сайту vidsebe.media теж
немає!!!  А випусків Мариуполь Билое - їх я знайшов штук 150 - а не 2000 - як
пишуть їх було!!!  Розумієте!!! Кожень день ви втрачаєте своє власне минуле!!!
Так!!! А без минулого не буде і майбутнього!!!  Без збереженого минулого, без
чіткої хронології подій та збережених фактичних даних як воно все було в тому ж
2021 році - не буде ніякого відродження Маріуполя а буде просто якась
депресивна чорна діра із величезною негативною енергетикою на десятиліття
вперед от і все!!!  і треба більш серьозно підходити до таких речей, або не
займатись цим зовсім!!!  Ну от дійсно, нащо Вам Буров, - сидіть собі вдома,
кохайтесь там із своїми коханими, кладіть своїх дітей хоч на бампер спереду -
весело ж буде, так!!!  А я пішов займатись далі серьозними речами! Оце от
дійсно Маріуполь задачка мені видалась! Але мені це і важливо і цікаво!
Різними я задачами займався - і теорія згортання білків і теорія струн і теорія
магнетизму!  Ну а зараз я займаюсь Маріуполем!  Повірте, дуже мені цікаво
займатись Маріуполем!!! А Ви ворочайте якось бистріше мізками своїми окей!

... і оце знаєте ще. Я от живу в реальному світі і фейсбуком і мережами
користуюсь мало. В реальному світі знаєте прийнято спілкуватись по людському, а
не через смайлики, розшарення і так далі. У мене є друзі, я з ними спілкуюсь
або по телефону або при зустрічі. Також, коли я займався наукою, так само, все
було в реальності. Також, коли я їжджу по місту Києві, в маршрутці наприклад,
знаєте, я теж спілкуюсь в реальності. Коли мне треба заплатити за проїзд, я
дістаю гаманець і плачу гроші, а не ставлю смайлик або сердечко сторінці водія.
Розумієте до чого я веду. У нас тут якась аномалія, коли я змушений писати Вам
якісь довжелезні листи, які мені насправді самому досить вже таки набридло
писати, - і  які напевне Вам досить таки сильно впливають на свідомість -
вибачайте - замість того, щоби те ж саме, абсолютно те ж само сказати спокійним
тоном по телефону або десь за чашкою кави, пояснити свою позицію. Розумієте.
Оце от все віртуальне спілкування між нами, також, те, що ми не є знайомі в
реальності - це все є суцільна психологічно-інформаційна аномалія, особливо з
точки зору надзвичайної важливості речей, які обговорюються. А проблема
збереження пам'яті, проблема відродження Маріуполя - це надзвичайно важливі
речі, як персонально для Вас як маріупольчанки, яка втратила свій рідний дім,
свій цілий величезний світ - в якому Ви жили, -  і зазнали великого лиха через
війну, так і для мене, тому що я не можу допустити, щоби в моїй країні просто
так взяли знищили ціле величезне місто, і ніхто за це не був покараний, і
важливі також для України, тому що Маріуполь - це найбільша рана України, і
вона повинна бути вилікована. Розумієте, для мене це принципово, і тому я цим
займаюсь. Але! Блін... Я вже стільки всього поробив, але проблема в тому, що ви
- маріупольці - всі так страшно зафейсбучені, що щось донести вам, щось
конструктивно обговорити, це виявляється капєц як складно. Я вже стільки бігаю
по виставкам, щось там розсказую, люди оживляються, коли бачать книжки або
постери, а потім - розбігаються - і все, забувають... Просто капец якийсь. Ви
навіть свій власний день міста - 10 вересня - не святкуєте, не робите ніякої
двіжухи по Києву - все чисто по фейсбукам!!! От бахмутяни взяли і висадили рози
на Оболоні, знаєте, в свій день міста, що 9 вересня!! Зібрались натовпом,
зробили як треба, молодці!!! А маріупольці що... пошуміли по фейсбукам і все...
А потім дивуєтесь, а чого це всі забули про Маріуполь!!! Ну правильно. якщо ви
нічого не робите в реальності в масштабах міста Києва, то ніхто ж про вас і не
знає, логічно ж, так! А щодо Вас і мене, це взагалі суцільний парадокс, тому
що, наприклад, я зробив дві товсті книжки Бурова, а Ви - ведучий спеціаліст по
культурі Маріуполя, може найбільший, найталановитіший спеціаліст, реально так
кажу. І у Вас ціла програма по публікації Бурова українською. Але... оці книжки
вже навіть київські таксисти або мої друзі в моєму дворі під моїм будинком на
Максима Кривоноса бачили, а Вам показати їх, відправити на перевірку
виявляється така величезна проблема!!! Просто капец. Невже... невже так складно
позвонити мені (яке страшне слово для Вас напевне ), домовитись про зустріч,
показати Вам все, що я зробив, блін, обговорити все це. Це ж не тільки мені
персонально для моєї втіхи потрібно, це потрібно і Маріуполю, і Україні,
розумієте, так!!! Невже це так складно, я от не розумію! Вам напевне легше
набрати двірника у вашому домі, якщо якийсь зайвий мусор лежить, або ж визвати
аварійну службу, - якщо там десь ліфт зламався або проблема із газом, ніж
позвонити мені і обговорити Маріуполь і Бурова!!! Просто капец якийсь! Всі от
кажуть Маріуполь - це Україна, труляля, все звільнимо, відбудуємо... Культурна
деокупація і так далі. Щось я у це не вірю. Щось мені підказує, що все це поки
що якийсь пустий трьоп, коли люди взагалі не усвідомлюють складності всього
цього питання, і наскільки воно складніше за те, що зараз говориться...

Все це пусті лозунги, знаєте, оскільки на практичному рівні маріупольці явно не
демонструють якогось великого бажання щось там дійсно відбудовувати,
об'єднуватись, робити якусь спільну двіжуху... Дуже шкода все це бачити,
знаєте, дуже боляче мені за цим спостерігати... Вашу спільнота так розсердилась
на фільм Юрик, але про те, що купа архітектурних бюро дуже інтенсивно працюють
над проєктами відбудови Маріуполя, в яких від колишнього Маріуполя вже нічого
не має, - якась взагалі суцільна футуристична фігня, - то нікого не хвилює...
Шкода!!! З такими темпами, з таким байдужим відношенням Маріуполю нічого вже не
світить, а жаль! Якщо вже так складно просто поговорити нам, - хоча технічної
складності в цьому ніякої нема, - це все залежить лише від Вас і Вашої Волі
перемогти свої внутрішні персональні таракани щодо мене, - то що вже казати про
якісь більш масштабні проєкти по відбудові цілого міста! Київ після війни років
10 відбудовували, Хрещатик прийняв свій сучасний вид вже під кінець 50х, а це ж
столиця!!! 15 років пішло на те, щоби стерлись сліди війни в Києві! А що ж вже
казати про Маріуполь! Напевне, років так 50 треба буде, щоби щось там почало
знову жити і працювати! А щодо персональних штук, повірте, у мене в пріорітеті
вони не стоять на першому місці, і то Ваше життя, з ким хочете, з тим і живіть,
то все Ваше персональне, я маю повагу до цього, і мені насправді шкода, що тут
все так переплуталось, що в результаті я навіть я не можу Вам свого Бурова і
також інші книжки (листівки, літопис, купа-купа всього) показати і
розсказати!!! А основне тут - мене цікавлять в першу чергу книжки,
Інтернет-Архів, збереження пам'яті, цифрової спадщини, особливо довоєнного
мирного Маріуполя, важливі, фундаментальні штуки для Маріуполя і для всієї
України, так!!!
