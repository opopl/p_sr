% vim: keymap=russian-jcukenwin
%%beginhead 
 
%%file 18_05_2019.stz.news.ua.mrpl_city.1.mariupol_professor_kazancev
%%parent 18_05_2019
 
%%url https://mrpl.city/blogs/view/mariupolskij-professor-kazantsev
 
%%author_id burov_sergij.mariupol,news.ua.mrpl_city
%%date 
 
%%tags 
%%title Мариупольский профессор Казанцев
 
%%endhead 
 
\subsection{Мариупольский профессор Казанцев}
\label{sec:18_05_2019.stz.news.ua.mrpl_city.1.mariupol_professor_kazancev}
 
\Purl{https://mrpl.city/blogs/view/mariupolskij-professor-kazantsev}
\ifcmt
 author_begin
   author_id burov_sergij.mariupol,news.ua.mrpl_city
 author_end
\fi

\ii{18_05_2019.stz.news.ua.mrpl_city.1.mariupol_professor_kazancev.pic.1}

Из двух десятков мариупольских улиц, носящих имена отечественных деятелей
науки, только одна названа в честь ученого, имеющего непосредственное отношение
к нашему городу. Речь идет о заслуженном деятеле науки и техники Украины,
докторе технических наук, профессоре Мариупольского (Ждановского)
металлургического института \textbf{Иване Георгиевиче Казанцеве}. Его имя также
присвоено кафедре \enquote{Металлургия стали} Приазовского государственного
технического университета. Даже эти два факта свидетельствуют о значимости
этого незаурядного человека в истории Мариуполя.

В шестом томе Украинской советской энциклопедии в статье, посвященной И. Г.
Казанцеву, лишь в общих чертах отражены направления его исследований. Сказано,
что он разработал основные теоретические положения процессов оседания липкой
пыли в металлургических печах, организации факельного процесса горения при
плавке, динамике газовых струй, внедряющихся в жидкость, в том числе и в жидкую
сталь, кинетической теории жидкой и твердой стали. На самом деле круг его
научных интересов и достижений был значительно шире. Характеризуя научную
деятельность Ивана Георгиевича, доктор технических наук, профессор А. М.
Скребцов подчеркивал, что одной из первых работ в области микронеоднородности
строения железоуглеродистых расплавов является его публикация, в которой даны
понятия \enquote{пульсирующая скорость} и \enquote{оптимум перегрева}. Доктор технических наук,
профессор Е. А. Капустин, отмечая приоритет ученого в области химической
кинетики и взаимодействия микрочастиц применительно к сталеплавильным
процессам, в докладе на международной конференции, посвященной 100-летию со дня
рождения профессора И. Г. Казанцева, в частности сказал: \emph{\enquote{Иван Георгиевич
использовал современные данные химической кинетики для расчетов скорости
реакции в теории металлургических процессов, ввел поправочные коэффициенты,
которые учитывают влияние температуры и растворителей, силовых полей в
сталеплавильных процессах, а также применил положения химической кинетики при
разработке методов переходного состояния, свойственного металлургическим
расплавам}}.

Непосвященному перечисленные работы могут показаться сугубо теоретическими и
весьма далекими от потребностей практики. На самом деле это не так. Иван
Георгиевич ставил себе научные задачи, исходя из нужд производства и, как
показало время, нашедшие в производстве применение. В годы Великой
Отечественной войны он принимал активное участие в разработке технологии
выплавки стали оборонного назначения в большегрузных мартеновских печах. Это
было чрезвычайно важно и необходимо для страны в тот тяжелейший период времени.
Он уделил много внимания исследованию особенностей свойств мышьяковистых
сталей, выплавленных из руд Керченского железорудного месторождения. Профессор
Казанцев занимался решением научных и технических проблем, связанных с
проектированием и строительством сталеплавильных агрегатов, усовершенствованием
мартеновских печей, разработкой новых технологических процессов, направленных
не только на повышение производительности сталеплавильных печей, но и на
улучшение качества стали. Его труды не принадлежат только истории, они
оказались востребованы и сейчас. Об их значимости нынешний заведующий кафедрой
металлургии стали ПГТУ, академик высшей школы, доктор технических наук,
профессор П. С. Харлашин писал: \emph{\enquote{В связи с созданием кислородно-конвертерных
процессов, чрезвычайно актуальными и важными явились научные работы,
выполненные И. Г. Казанцевым, они стали классическими в этой области}}. Наверное,
стоит вспомнить, что в 1957 году Иван Георгиевич в составе немногочисленной
делегации самых известных и авторитетных ученых представлял Украину, как члена
ООН на 2-м Международном конгрессе металлургов в Чикаго (США). Это ли не
признание его научных заслуг?

Иван Георгиевич Казанцев родился 20 июля (по ст.ст.) 1899 года в деревне
Комонино Орловской губернии, теперь Липецкой области Российской Федерации в
большой крестьянской семье. Его отец сам, занимаясь тяжелым трудом хлебопашца,
стремился дать детям образование, поощрял стремление к просвещению, наряду с
этим воспитывал в них трудолюбие, неприятие праздности и невежества. Однако в
его силах было обеспечить им лишь начальное образование – удел абсолютного
большинства крестьянских детей дореволюционной России. Иван Казанцев в 1913
году окончил двухклассное училище, которое представляло собой начальную школу
с четырехлетним сроком обучения. Как было заведено в крестьянских семьях, он с
малолетства участвовал в трудовых процессах семьи, насколько позволял его
возраст. В шестнадцать лет уходит на заработки, устраивается учеником слесаря.

В 1919 году Ивана Георгиевича призывают в Красную Армию, где он служит
мотористом в военном мостопоезде. Задачей личного состава мостопоезда было
восстановление и ремонт мостов, взорванных в ходе боевых действий во время
гражданской войны. После демобилизации в 1921 году недавний красноармеец
начинает работать помощником слесаря в паровозоремонтных мастерских
железнодорожной станции в Екатеринославле, который в 1926 году был переименован
в Днепропетровск. Одновременно он учится в вечерней школе рабочей молодежи, а
затем на рабфаке – курсах по подготовке в ВУЗ.

В 20 – 30-е годы прошлого века в молодом советском государстве существовала
практика направлять способных молодых людей из числа рабочих на учебу в высшие
учебные заведения. Занимались этим профсоюзы. При назначении будущей
специальности в меньшей степени учитывали пожелание поступающего в ВУЗ, а в
большей – руководствовались потребностями народного хозяйства в тех или иных
специалистах. Так, скорее всего случилось и с Иваном Георгиевичем. Вряд ли
изучение именно металлургии было его мечтой. Но вот острое желание учиться
конечно было, стать инженером – тоже.

В 1923 году профсоюз рабочих железнодорожного транспорта командирует его в
Екатеринославский горный институт, основанный в 1899 году. В стенах этого
учебного заведения было подготовлено немало высококвалифицированных горных
инженеров и инженеров-металлургов. Некоторые из них стали видными учеными.
Кстати, в свое время профессором этого ВУЗа был будущий академик Михаил
Александрович Павлов – крупнейший специалист в доменном производстве, автор
трудов по конструкции металлургических печей.

1928 год ознаменовался для И. Г. Казанцева окончанием учебы в ВУЗе и получением
диплома инженера-металлурга. Его направили работать инженером на
Днепропетровский металлургический завод имени Петровского. А в 1929 году Ивана
Георгиевича, как молодого специалиста, проявившего склонность к научной
деятельности, профессор Фортунатов пригласил на должность ассистента на кафедру
металлургии стали Днепропетровского металлургического института, выделенного
незадолго до этого из Горного института. До 1931 года он одновременно
продолжает работать на заводе имени Петровского. Такое совместительство,
несмотря на почти полное отсутствие свободного времени, было чрезвычайно
полезно для становления ученого: результаты научных исследований можно было
сразу же проверить в цехах завода. Да и сами темы для исследований черпались из
заводских проблем.

Напряженный труд дал свои плоды. В 1936 году на ученом совете
Днепропетровского металлургического института инженер И. Г. Казанцев ус\hyp{}пешно
защищает диссертацию на тему: \enquote{Механика газов в промышленных печах}.
Квалификационной комиссией Наркомтяжпрома СССР 25 февраля 1937 года ему
присуждается ученая степень кандидата технических наук, а чуть более, чем
через год – 7 мая 1938 года – он был утвержден в ученом звании доцента кафедры
металлургии стали.

С началом Великой Отечественной войны доцент Казанцев, как и другие
специалисты-металлурги его возраста и уровня квалификации, был освобожден от
призыва в армию. В августе 1941 года, когда гитлеровские полчища приблизились к
Днепропетровску, он был эвакуирован с семьей на Урал в город Магнитогорск. Там
И. Г. Казанцев работает руководителем исследовательской лаборатории
Магнитогорского металлургического комбината, совмещая с преподаванием в
качестве доцента Магнитогорского горно-металлургического института. В это время
он занимается разработкой новых и совершенствованием уже имеющихся
технологических процессов выплавки стали для нужд фронта.

Ивана Георгиевича часто привлекали для решения проблем, которые, казалось бы,
были далеки от его специальности. Сын И. Г. Казанцева – \textbf{Евгений Иванович}, теперь
он заслуженный деятель науки Украины, профессор и доктор технических наук
рассказывал: \emph{\enquote{На Магнитке в какой-то период военных лет не стало нефти.
Хлебозаводы, печи которых отапливались мазутом, остановились. В городе стали
выдавать вместо хлеба муку. Но получать двести граммов муки вместо четырехсот
граммов хлеба – не одно и то же. Ситуация была очень сложная, можно сказать –
критическая. Городские власти обратились на комбинат с просьбой, чтобы выделили
теплотехника, который помог бы найти альтернативу мазуту. Вышли на Ивана
Георгиевича. Что он сделал? Он обошел практически всю территорию
Магнитогорского комбината и нашел ямы, заполненные пеком. Этот пек твердый, но,
если его разогреть, он превращается в жидкотекучий и его можно использовать как
заменитель мазута. Иван Георгиевич дал технические рекомендации, каким образом
использовать этот пек в хлебопекарных печах. Пек пошел на хлебозаводы. Магнитка
ожила: стали выдавать на карточки не муку, а хлеб}}.

Во время войны правительство поручило руководству Академии наук УССР, которая
была эвакуирована в Уфу, обеспечить научно-техническое курирование производства
танков, сосредоточив особое внимание на качестве. Такое решение не было
случайным. Ведь до войны разработка конструкций этих боевых машин и их
изготовление было сосредоточено в Харькове на паровозостроительном заводе.
Украинские ученые, в том числе и из академических институтов, в той или иной
мере принимали участие в решении задач, которые ставились танкостроителям. По
поручению Академии наук УССР академик Евгений Оскарович Патон был назначен
ответственным за сварку корпусов и башен танков, а также самоходных орудий.
Академик Николай Николаевич Доброхотов отвечал за качество броневой стали.

По рекомендации академика Н. Н. Доброхотова Народный комиссариат танковой
промышленности СССР назначил И. Г. Казанцева государственным контролером
качества броневой стали на заводах Южного Урала. Иван Георгиевич на каждом
заводе скрупулезно изучил сырьевую базу, конструкции и состояние
сталеплавильных печей, технологию выплавки броневой стали и ее качество.
Результатом этого изучения стал подробнейший доклад о состоянии дел с выводами
и рекомендациями. С этим докладом он выступил в Наркомате. После доклада к нему
обратился начальник главного управления учебных заведений НКТП с предложением
занять пост заместителя директора (именно так называлась в те времена эта
должность в технических ВУЗах) Мариупольского металлургического института по
научно-учебной работе и одновременно возглавить кафедру \enquote{Металлургия стали}.
И. Г. Казанцев согласился.

8 марта 1944 года Иван Георгиевич приступил к исполнению новых обязанностей.
Тогда институт ютился в неприспособленном одноэтажном здании, преподавателей с
учеными степенями было очень мало. Не хватало оборудования, материалов для
обеспечения учебного процесса и проведения научных исследований. Существовала
острая нужда в учебниках и научно-технической литературе. И. Г. Казанцев
приложил немало сил, чтобы наладить работу. Он добился в Москве решения о
передаче здания бывшего епархиального училища, а затем штаба полка институту.
Теперь это первый корпус Приазовского государственного технического
университета.

Иван Георгиевич Казанцев посвятил работе в Мариупольском металлургическом
институте более двадцати лет. Работая проректором по научной и учебной работе и
заведующим кафедрой \enquote{Металлургия стали}, он внес значительный вклад в
организацию учебного процесса и развитие научных исследований. Он и сам активно
занимался научной работой. При этом, как вспоминают современники, все
лабораторные эксперименты выполнял собственноручно. Порой для постановки опытов
использовал простейшие, на первый взгляд, приборы, почти бытовые предметы. Но
природная наблюдательность, интуиция, нестандартность мышления, свободное
владение математическим аппаратом позволяли ему делать далеко идущие выводы,
которые позже находили применение в теории и практике металлургического
производства. В конце 40-х – начале 50-х годов И. Г. Казанцев, совмещая большую
административную, педагогическую работу, плановые научные исследования и
консультативную помощь местным металлургическим заводам, интенсивно работал над
докторской диссертацией. В 1953 году он ее успешно защитил на авторитетнейшем
ученом совете Института металлургии Академии наук СССР. Диссертация носила
название \enquote{Основные вопросы кинетики обезуглероживания металла в мартеновских
печах}. Крупный ученый теплотехник и металлург, профессор, доктор технических
наук Е. А. Капустин охарактеризовал ее так: \enquote{Она является образцом докторской
диссертации. Иван Георгиевич сделал ее своим умом и своими руками. Прошло более
полувека, а она остается такой же актуальной, какой была в свое время}. 3 июля
1954 года решением Высшей Аттестационной Комиссии И. Г. Казанцев был утвержден в
ученой степени доктора технических наук и в ученом звании профессора кафедры
\enquote{Металлургия стали}.

Профессор Казанцев опубликовал по современным меркам не так уж и много работ –
чуть больше восьмидесяти. Но вот что интересно: во-первых, большинство этих
работ написано без соавторов, а во-вторых, основные из них цитируются до сих
пор.

Иван Георгиевич занимался и общественной деятельностью. С 1951 года он был
председателем городского общества по распространению политических и научных
знаний, известного впоследствии как общество \enquote{Знание}. В разные периоды времени
он избирался депутатом областного, городского, районного советов, при этом
депутатскими делами занимался очень ответственно.

За высокие достижения в научной, производственной и педагогической деятельности
И. Г. Казанцев был удостоен почетного звания \enquote{Заслуженного деятеля науки и
техники УССР}, а также награжден орденами Ленина и Трудового Красного Знамени,
медалью \enquote{За доблестный труд в Великой Отечественной войне 1941-1945гг}.

Профессор Казанцев скончался после тяжелой болезни 23 января 1966 года.

\textbf{Читайте также:} 

\href{https://mrpl.city/blogs/view/zabavin-vyacheslavprofesijnij-arheolog-mariupolya}{%
Забавін В'ячеслав – професійний археолог Маріуполя, Ольга Демідко, mrpl.city, 11.05.2019}
