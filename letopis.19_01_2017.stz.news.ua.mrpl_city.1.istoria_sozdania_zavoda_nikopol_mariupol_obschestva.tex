% vim: keymap=russian-jcukenwin
%%beginhead 
 
%%file 19_01_2017.stz.news.ua.mrpl_city.1.istoria_sozdania_zavoda_nikopol_mariupol_obschestva
%%parent 19_01_2017
 
%%url https://mrpl.city/blogs/view/zavod-nikopol-mariupolskogo-obshhestva
 
%%author_id burov_sergij.mariupol,news.ua.mrpl_city
%%date 
 
%%tags 
%%title История создания завода Никополь-Мариупольского общества
 
%%endhead 
 
\subsection{История создания завода Никополь-Мариуполь\hyp{}ского общества}
\label{sec:19_01_2017.stz.news.ua.mrpl_city.1.istoria_sozdania_zavoda_nikopol_mariupol_obschestva}
 
\Purl{https://mrpl.city/blogs/view/zavod-nikopol-mariupolskogo-obshhestva}
\ifcmt
 author_begin
   author_id burov_sergij.mariupol,news.ua.mrpl_city
 author_end
\fi

История Мариупольского металлургического завода имени Ильича неплохо
представлена в краеведческой литературе. Но в ней отражены, главным образом,
экономические и социальные аспекты, что же касается технической стороны,
особенно первых десятилетий его существования, то тема эта обозначена как бы
вскользь. Редкое издание – журнал \enquote{Горное и заводское дело}, а точнее, обширная
статья горного инженера А. И. Тиме, опубликованная на его страницах, в
значительной степени заполняет этот пробел.

\ii{19_01_2017.stz.news.ua.mrpl_city.1.istoria_sozdania_zavoda_nikopol_mariupol_obschestva.pic.1}

Изначально предприятие было спроектировано и построено как завод с полным
металлургическим циклом -  доменное, сталеплавильное, листопрокатное и трубное
производство. Трубное производство было введено в эксплуатацию в то время,
когда еще шло строительство доменного и мартеновского цехов. Работали на
привозном листовом металле Путиловского и других заводов для выполнения
срочного заказа в 115 верст (около 123 километров) стальных труб для
Новороссийского трубопровода...

\ii{19_01_2017.stz.news.ua.mrpl_city.1.istoria_sozdania_zavoda_nikopol_mariupol_obschestva.pic.2}

С самого начала предполагалось пользоваться только богатыми железными рудами из
Кривого Рога. Коксом доменное производство обеспечивалось из Горловки. К началу
ХХ века было построено две доменные печи с наклонными подъемниками, что в ту
пору было новшеством. В целом они были копиями американских печей.
Производительность каждой -  280 тонн в сутки. Отмечается как большое
техническое достижение наличие крыши над литейным двором. Автор пишет, что тем
самым \enquote{рабочие защищены от непогоды, работа идет успешнее и лучше сохраняется
кладка горна}.

Мартеновский цех  состоял из пяти мартеновских печей с основной футеровкой
(набойка  пода – магнезит). Вместимость каждой из печей – 30 тонн. Отопление
печей производилось генераторным газом. Газогенераторы располагались за
пределами здания цеха. Основное грузоподъемное оборудование цеха состояло из
одной завалочной машины кранового типа (шаржир-машины) и четырех мостовых
кранов, один из них грузоподъемностью 50 тонн, остальные по 15 тонн каждый.
Разливка стали в изложницы производилась сверху из ковша вместимостью 50 тонн
жидкой стали, подвешенного цепями к крюку мостового крана.

Прокатное производство было представлено трехвалковым прокатным станом с длиной
валков 2 500 миллиметров. При необходимости станины могли быть  раздвинуты, при
этом применялись валки длиной 3 300 миллиметров. Как известно, от длины валка
зависит ширина прокатываемого листа. А.И.Тиме пишет, что на стане катались
котельный, корабельный листовой металл, а также броневые плиты. Двигателем
служила паровая машина. При стане имелось четыре газовых нагревательных печи.

В трубопрокатном цехе изготавливали сварные трубы. Технология их производства
состояла в следующем.  Листы из низкоуглеродистой  стали поступали в
заготовочную печь, по достижении необходимой температуры  пропускались через
волочильный стан, при этом получались заготовки труб. Их вновь нагревали  и
после выхода из печи пропускали через специальные валки с оправкой. При этом
труба сваривалась настолько прочно, что выдерживала гидравлическое испытание
при давлении в 110 атмосфер. В сутки изготавливалось 200 штук труб.

К вспомогательным цехам относились литейная, модельная, столярная,  кузнечная и
механическая мастерские, а также кирпичный завод для производства огнеупоров.
Имелась также электростанция с генератором номинальной мощности 225 кВт и
напряжением 250 В. Генератор приводился в действие двумя водотрубными котлами.
Водоснабжение завода осуществлялось из реки Кальмиус.

На приложенном к статье плане завода Никополь, кроме действующих объектов,
нанесены и проектируемые. В том числе, еще шесть доменных печей, пять
мартеновских печей, цех с бессемеровскими конвертерами, рельсопрокатный цех,
коксовые батареи. Время показало, что многое из намеченного так и осталось
только на бумаге.

И последнее. Горный инженер А.И.Тиме по ходу изложения своей статьи не раз
подчеркивал, что завод Никополь-Мариупольского общества как по проектному
решению, так и по воплощению был одним из лучших в Российской империи.
