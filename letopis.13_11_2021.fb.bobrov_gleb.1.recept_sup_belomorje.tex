% vim: keymap=russian-jcukenwin
%%beginhead 
 
%%file 13_11_2021.fb.bobrov_gleb.1.recept_sup_belomorje
%%parent 13_11_2021
 
%%url https://www.facebook.com/glebbobrov0665926209/posts/4530325523747533
 
%%author_id bobrov_gleb
%%date 
 
%%tags donbass,eda,kuhnja,lnr,recept,zhizn
%%title Рецепт зимнего рыбного супа родом с Беломорья
 
%%endhead 
 
\subsection{Рецепт зимнего рыбного супа родом с Беломорья}
\label{sec:13_11_2021.fb.bobrov_gleb.1.recept_sup_belomorje}
 
\Purl{https://www.facebook.com/glebbobrov0665926209/posts/4530325523747533}
\ifcmt
 author_begin
   author_id bobrov_gleb
 author_end
\fi

Как часто работающий на кухне практик, я уверен, что высший пилотаж, это не
хитро жаренное мясо и даже не фаршированная утка, а первые блюда – лицо вашей
семейной кухни, если можно так выразиться. Пока еще не начался долгожданный
Рождественский пост наша традиционная семейная рубрика "простая\&вкусная" еда
побалует вас рецептом зимнего рыбного супа родом с Беломорья. 

\ifcmt
  ig https://scontent-frt3-2.xx.fbcdn.net/v/t39.30808-6/255967248_4530325477080871_6041144178650754388_n.jpg?_nc_cat=101&ccb=1-5&_nc_sid=730e14&_nc_ohc=hNhdzwRIWYAAX-WneeQ&_nc_ht=scontent-frt3-2.xx&oh=b146169d7b69885ac385e315056308cd&oe=61A8B92E
  @width 0.4
  %@wrap \parpic[r]
  @wrap \InsertBoxR{0}
\fi

Рецепт из расчёта на шестилитровую кастрюлю, там корректируйте от своей посуды. 

Первый секрет – рыба. Брать надо несколько сортов и типов. У меня в этот раз
была голова, хвост и молоки горбуши, плюс обезглавленная тушка наваги (та самая
северная рыбка из семейства тресковых). Молоки, кстати, показывают себя в супе
с самой лучшей стороны. У нас на рынке продаются лососёвые, думаю и у вас
магазинах и на рынках их тоже можно найти. Также можно смело брать головы
лосося – по деньгам выходит чуть ли не бесплатно, а мяса в голове немало, плюс
с неё получается отличный навар. 

Рыбу помыть, положить в кастрюлю, залить на две трети водой, кинуть туда черный
и душистый перец горошком, пару листиков лавра и поставить на сильный огонь. 

Секрет номер два – пена. Это тот самый суп, где пену при варке надо обязательно
снимать. Причем снимать не только с поверхности, но и стирать салфеткой со
стенки кастрюли, где пена будет откладываться налётом на два-три сантиметра
выше среза воды. Это важно для отсутствия неприятного рыбного запаха и
красивого, золотистого цвета бульона.

Пока рыба закипает, зажариваем в сотейнике одну большую луковицу и пару средних
морковок. Моркови должно быть, как минимум на треть больше лука. Жарить на
смеси подсолнечного и сливочного масла, пополам на пополам. 

Как только зажарка зазолотилась, добавляем в неё щепотку соли, и по чайной
ложке следующих специй: куркуму, молотые чёрный перец в смеси с молотым
кориандром (т.е. по половине чайной ложки каждого), ложку любого готового
рыбного набора специй и два-три зубчика мелко покрошенного чеснока. Специи
здесь – секрет номер три. 

Все это заливаем несколькими половниками бульона и даем пару минут покипеть,
после чего оставляем настаиваться на теплой плите.  

Как рыба закипела, засекаем 20-30 минут и чистим тем временем пять небольших
картофелин размером, например, с киви.

Через 20-30 минут варки рыба начнёт развариваться и её надо достать шумовкой,
выложить на блюдо и вынести на холодок остывать. Как остынет – разделать на
кусочки.

Мелко режем картофель, закидываем в бульон, даем прокипеть пару минут и следом
всыпаем 200 гр. мелких макаронных изделий для заправки супов. В супермаркетах
продают макароны "орзо" это такая микро-паста похожая на зёрна риса – идеальная
заправка, но и обычные суповые звездочки или ракушки тоже пойдут на ура. Кроме
макаронных изделий сюда отлично пойдёт рис, или пшено.

Даем закипеть, ждем ровно минуту и высыпаем туда зажарку и очищенную от костей
и плавников рыбу. Как только бульон закипит, то сразу вливаем в него один литр
жирного молока (у меня базарное, но и магазинное 3,2\% жирности тоже хорошо
зайдёт). Молоко здесь – последний секрет. 

Накрываем крышкой и ждём, когда суп вновь закипит. Как закипело, даём пару
минут побурлить, добавляем десертную ложку с горкой соли и пробуем. Если надо
досаливаем, или доливаем воды. После отправляем в бульон еще пару листиков
лавра, выключаем и даём минут десять настояться.

В результате получается средней густоты насыщенный и сытный рыбный суп с нежным
сливочным вкусом и золотистым цветом, не имеющий к тому же неприятного рыбного
запаха. Традиционно такие супы принято украшать зеленью, но я не вижу в этом
смысла.

В целом, блюдо готовится намного быстрее борща, рассольника, фасолевого или
горохового супа из-за быстрой варки рыбы. По вкусу же даст фору многим мясным
первым блюдам. 

Приятного аппетитоса!

\ii{13_11_2021.fb.bobrov_gleb.1.recept_sup_belomorje.cmt}
