% vim: keymap=russian-jcukenwin
%%beginhead 
 
%%file slova.respublika
%%parent slova
 
%%url 
 
%%author 
%%author_id 
%%author_url 
 
%%tags 
%%title 
 
%%endhead 
\chapter{Республика}
\label{sec:slova.respublika}

%%%cit
%%%cit_head
%%%cit_pic
%%%cit_text
А в 2021 году ровно теми же словами о «нецелесообразности» объединения в период
переговорного процесса с Украиной говорит новый глава ДНР Пушилин. Итак, версия
вторая, которую можно условно назвать «рука Кремля». Если честно, то аргументы
касательно неслыханной самостоятельности Донецка и Луганска вообще не
впечатляют. Достаточно вызвать «из отпусков» людей, занимающихся военными
вопросами, вернуть на склады бронетехнику, закрыть границу с российской
стороны, и тогда эти \emph{республики} смогут просуществовать в лучшем случае
несколько дней. Возможно, «кураторы» и не лезут во все местечковые вопросы, но
общая генеральная линия однозначно доводится с самого верха. А она такова: ДНР
и ЛНР – это часть Украины, и они должны вернуться в ее состав по исполнению
Минских соглашений. Вновь процитируем господина Царева
%%%cit_comment
%%%cit_title
\citTitle{Нежелание Донецка и Луганска объединяться ставит под угрозу существование независимого Донбасса}, 
Сергей Маржецкий, voskhodinfo.su, 02.07.2021
%%%endcit

%%%cit
%%%cit_head
%%%cit_pic
%%%cit_text
Тридцатиление своей независимости Украина встречает в неприглядном виде. Кризис
\emph{Второй Украинской республики} настолько всеобъемлющий, что самое время говорить
о ее агонии. Однако поиск путей выхода требует рефлексии того, что произошло за
эти тридцать лет.  В беседе Сергея Иванова с Юрием Романенко в проекте
"Антиподы" шла речь о прошлом, настоящем и будущем Украины. Выкладываем первую
часть беседы
%%%cit_comment
%%%cit_title
\citTitle{Украина в плену консервативного мышления: почему 30 лет шли не туда и что делать}, 
Сергей Иванов; Юрий Романенко, hvylya.net, 15.07.2021
%%%endcit

