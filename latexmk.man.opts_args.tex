% vim: keymap=russian-jcukenwin
%%beginhead 
 
%%file man.opts_args
%%parent body
 
%%endhead 

\section{LATEXMK OPTIONS AND ARGUMENTS ON COMMAND LINE}
  
\vspace{0.5cm}
 {\ifDEBUG\small\LaTeX~section: \verb|man.opts_args| project: \verb|latexmk| rootid: \verb|p_saintrussia| \fi}
\vspace{0.5cm}

In general the command line to invoke latexmk has the form

\begin{verbatim}
  latexmk [options] [file]
\end{verbatim}

All options can be introduced by single or double \verb|"-"| characters, e.g.,
\verb|"latexmk -help"| or \verb|"latexmk --help"|.

\subsection{Note 1}

In addition to the options in the list  below,  latexmk  recognizes almost
all the options recognized by the latex, pdflatex programs (and their
relatives) in their current TeXLive and  MiKTeX  implementations.  Some of
the options for these programs also trigger special action or behavior by
latexmk, in which case they have specific  explanations  in  this document.
Otherwise, they are just passed through to a called latex or pdflatex program.
Run latexmk  with  the  -showextraoptions  to  get  a list of the options
that latexmk accepts and that are simply passed through to latex or pdflatex
(etc).  See also the  explanation of the -showextraoptions option for more
information.

\subsection{Note 2}

In this documentation, the program pdflatex is often referred to.
Users of programs like lualatex and xelatex should know that  from latexmk's
point  of view, these other programs behave very like pdflatex, i.e., they
make a pdf file from a  tex  file,  etc.   So  whenever pdflatex is mentioned
without mention of the other programs, the statements apply equally to
lualatex, xelatex, and any  other  similar  programs.  Latexmk can be easily
configured to use whichever of these programs is needed.  See the
documentation  for  the  following  options: -pdflua,  -pdfxe,  -lualatex, and
-xelatex, and also see the documentation for the \verb|$pdflatex|,
\verb|$lualatex|,  and  \verb|$xelatex|  configuration  variables.   At
present  latexmk  does not do automatic detection of which program is to be
used.

\ii{man.opts_args.definitions}


