% vim: keymap=russian-jcukenwin
%%beginhead 
 
%%file 08_10_2021.fb.fb_group.protydia_zrosijschennju_volynj.1.magazin_salut_mova_luck
%%parent 08_10_2021
 
%%url https://www.facebook.com/groups/446250459425417/posts/861052357945223
 
%%author_id fb_group.protydia_zrosijschennju_volynj
%%date 
 
%%tags jazyk,luck,magazin,mova,ukraina,ukrainizacia,zhaloba
%%title Магазин Салют - Мова Обслуговування
 
%%endhead 
 
\subsection{Магазин Салют - Мова Обслуговування}
\label{sec:08_10_2021.fb.fb_group.protydia_zrosijschennju_volynj.1.magazin_salut_mova_luck}
 
\Purl{https://www.facebook.com/groups/446250459425417/posts/861052357945223}
\ifcmt
 author_begin
   author_id fb_group.protydia_zrosijschennju_volynj
 author_end
\fi

08.10.2021 року о 2130 год. я зайшов у магазин "Салют", що в м. Луцьку на пр.
Грушевського 20. Обираючи товари звернув увагу на миючі засоби, що вони
промарковані незрозумілою мені мовою. Повернув упаковку іншою стороною, щоб
подивитися, що там написано. Оскільки шрифт був дрібний то довелося
сфотографувати його й збільшити в телефоні. Текст українською є, але він
недоступний для прочитання без спеціальних засобів. Це порушує мої права,
передбачені законодавством, ставить у нерівні умови з тими, кому зрозуміле
написане крупнішими літерами. Це називається дискримінацією. Будь-яке обмеження
законних прав заборонене Законом про протидію дискримінації та карається за ст.
161 ККУ. 

\ifcmt
  ig https://scontent-frx5-1.xx.fbcdn.net/v/t1.6435-9/244867389_6154353824639656_8751785784974381892_n.jpg?_nc_cat=105&ccb=1-5&_nc_sid=825194&_nc_ohc=RAaTVfDVI4cAX_rBHS9&_nc_ht=scontent-frx5-1.xx&oh=627b2291b73f74c4b4ddd3fc524465a4&oe=618ADC25
  @width 0.4
  %@wrap \parpic[r]
  @wrap \InsertBoxR{0}
\fi

Зазначу, що законодавство не зобов'язує мене знати інші мови, крім державної,
але гарантує право отримувати інформацію про товари та послуги українською . 

Прийшовши з товаром до каси, я звернув увагу на те, що касирка обслуговує
покупців іноземною мовою. Коли настала моя черга оплачувати,  привітався.
Знаючи, що людині може не сподобатися публічне зауваження, я трохи нахилився й
сказав: "прошу пані, на касі ви маєте обслуговувати українською".

Що тут почалося! Касирка відповідає:

"а я нє в Вєрховнай радє! Вот ви гаварітє на укрАінскам і всє в Польщє. Мой муж
ваєнний трідцать лєт - на каком язикє хачу, на таком і буду гаваріть! “

Раптом підходить інша жіночка, в светрі, подібному до того, в якому сидить
касирка на іншій касі - припускаю, що адміністраторка залу, - і починає
російською сварити мене, що чіпляюся до касирки, а в тої низька зарплата.

Одним словом, на зауваження нахамили, про вимоги закону чути не схотіли. А я не
розвивав це в конфлікт, забрав решту й пішов писати вам заяву.

Прошу перевірити на предмет виконання вимог законодавства пп "Торговий дім"
Салют", м. Луцьк, пр. Грушевського 20,тел. (0332)231122, керуюча Гандзюк
Світлана Миколаївна

Заява надіслана \textbf{Уповноважений із захисту державної мови}
