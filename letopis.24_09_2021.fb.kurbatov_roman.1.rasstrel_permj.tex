% vim: keymap=russian-jcukenwin
%%beginhead 
 
%%file 24_09_2021.fb.kurbatov_roman.1.rasstrel_permj
%%parent 24_09_2021
 
%%url https://www.facebook.com/rkurbatov/posts/10216922572095035
 
%%author_id kurbatov_roman
%%date 
 
%%tags __sep_2021.terakt.rossia.permj.universitet,permj.rossia,rossia,shooting,smert,studenty,tragedia,universitet
%%title Где-то там в универе расстреляли людей
 
%%endhead 
 
\subsection{Где-то там в универе расстреляли людей}
\label{sec:24_09_2021.fb.kurbatov_roman.1.rasstrel_permj}
 
\Purl{https://www.facebook.com/rkurbatov/posts/10216922572095035}
\ifcmt
 author_begin
   author_id kurbatov_roman
 author_end
\fi

Не могу понять вот этой всей негативной дичи, которой нас пичкают с целью
вызвать сочувствие. Тем более, что это вроде бы как наши СМИ, а новости вроде
бы как про наших любимых соседей. Но даже если без соседства.

Где-то там в универе расстреляли людей. С одной сторон, чего хотели, того и
получили, вы создаете депрессинвые регионы и годами транслируете ненависть по
телевизору, это закономерный результат. Но есть действительно жертвы - люди,
которые просто оказались не в то время и не в том месте, там еще можно как-то
включить сочувствие (не не так, будто бы это трагедия, произошедная минимум со
знакомыми - я не знаю этих людей и не собираюсь их знать).

И совсем другая новость - про восхождение на Эльбрус или там Эверест. Куча
дебилов, рискуя жизнями, прется туда, где заведомо может погибнуть. Потому что
им скучно и нефиг делать. И, что характерно, ПОГИБАЕТ (правда, только часть).
Или где-то срывается со скалы во время съемок селфи. Или попадает под поезд,
или получает поражение током на поезде. Кому сочувствовать, зачем?
