% vim: keymap=russian-jcukenwin
%%beginhead 
 
%%file 03_01_2022.stz.news.de.berliner_telegraph.1.deti_vojna
%%parent 03_01_2022
 
%%url https://berliner-telegraph.de/ru/kogda-deti-vojny-pishut/obshhestvo-3
 
%%author_id najdenova_veronika
%%date 
 
%%tags savenkova_faina,donbass,vojna,ukraina,deti,literatura
%%title Когда дети войны пишут
 
%%endhead 
\subsection{Когда дети войны пишут}
\label{sec:03_01_2022.stz.news.de.berliner_telegraph.1.deti_vojna}

\Purl{https://berliner-telegraph.de/ru/kogda-deti-vojny-pishut/obshhestvo-3}
\ifcmt
 author_begin
   author_id najdenova_veronika
 author_end
\fi

\begin{zznagolos}
Во времена войны дети пытаются защитить себя в меру своих возможностей, когда
взрослые не справляются.	
\end{zznagolos}

%\ii{03_01_2022.stz.news.de.berliner_telegraph.1.deti_vojna.pic.1}

Одна девочка пряталась в доме в Амстердаме (Нидерланды) в течение двух военных
лет, и теперь последующие поколения могут прочитать о том, что ей пришлось
испытать во время войны. По мере возможностей она записывала свою историю в
блокнот. Впоследствии это стало «Дневником Анны Франк».

В ту же войну, но годом раньше, другая девочка, находившаяся примерно в 2000
километрах от Амстердама в окружённом Ленинграде (Санкт-Петербург, Россия),
написала то, что она могла написать в тех тяжёлых обстоятельствах: на
нескольких листах блокнота, который был в её распоряжении, она вела реестр
смертей своей семьи. Теперь любой может увидеть эту «Тетрадь смерти» Татьяны
Савичевой в Городском музее Санкт-Петербурга.

Эти две девочки одного возраста, находящиеся по разные стороны от Германии,
делали записи, фиксируя окружающую их реальность в попытке примириться с
войной, внутри которой они внезапно оказались. Эта война была давным-давно. С
точки зрения сегодняшнего дня. Сегодня мы уже мало связаны с этими событиями,
потому что война идёт не здесь и не сейчас.

\ii{03_01_2022.stz.news.de.berliner_telegraph.1.deti_vojna.pic.2}

Война идёт сегодня. Несмотря на то, что это происходит на расстоянии около 2000
километров от Германии и за пределами Европейского союза, война бушует в
географической Европе, от нас её отделяет всего одна страна. Это страна, о
которой мы мало слышали ранее в обычных СМИ, и, благодаря широкому освещению в
тех же СМИ, не позднее 2014 года знаем, где она находится и насколько она
разорвана на части. Информация об этой войне неоднозначна и противоречива,
каждый должен сам ознакомиться с ходом событий этого военного конфликта, чтобы
выработать собственный взгляд на вещи. Пришла ли война извне или разгорелась
из-за внутренних беспорядков – факт остаётся фактом: на востоке упомянутой
страны, Украины, где сегодня двенадцатилетняя девочка пишет о той реальности, в
которой она живёт, идёт реальная война.

\ii{03_01_2022.stz.news.de.berliner_telegraph.1.deti_vojna.pic.3}

Сегодня уже не 1940-е годы, в нашем распоряжении совсем другие технические
возможности, чем были тогда у девочек Тани и Анны. Сегодня одна девочка может
добиться того, что ранее было невозможно в условиях военного времени. Она может
написать больше, чем семейную книгу смерти или личный дневник. Сегодня девочка
может написать письмо прямо с войны генеральному секретарю ООН Антониу
Гутерришу, пытаясь защитить себя и других таких же детей, как она, когда
взрослые не справляются:

\begin{zzquote}
«Уважаемый господин Гутерриш,

Здравствуйте! Меня зовут Фаина Савенкова, мне 12 лет и я живу в Луганске. К
сожалению, более половины моей жизни прошло на войне, и я уже почти не помню,
что такое мирная жизнь. Именно об этом я пишу в своих эссе, которые переводятся
на множество языков и публикуются в разных уголках земного шара. В своих
текстах я призываю к окончанию войны в Донбассе и прекращению страданий мирных
жителей, в том числе детей. Я знаю, что взрослые должны защищать тех, кто из-за
возраста не может защитить себя сам, поэтому делала обращение в ООН ко Дню
защиты детей с просьбой остановить обстрелы и спасти жизни. Но война
продолжается.

Недавно украинский сайт «Миротворец» выложил мои данные и данные моих близких в
публичный доступ, внеся в список врагов Украины. Меня обвинили в участии в
антиукраинских пропагандистских мероприятиях и распространении «фейковой»
информации. Это ложь, с помощью которой пытаются вызвать ко мне ненависть
населения Украины, запугать и очернить моё имя. Теперь я боюсь за свою жизнь и
здоровье и безопасность семьи.

13 октября 2021 года я сделала публичное обращение к президенту Украины
Владимиру Зеленскому с просьбой заблокировать сайт «Миротворец» из-за
нарушения Закона о персональных данных, а также игнорирования
рекомендаций UNICEF о персональных данных детей (The Case for Better
Gover- nance of Children’s Data: A Manifesto,
\url{https://www.unicef.org/globalinsight/reports/better-governance-childrens-data-manifesto}),
о чём говорилось в СМИ. Тем не менее, сайт «Миротворец» продолжает
работать, постоянно обновляет и дополняет информацию о моей семье,
нанося вред мне и близким. Страшнее всего то, что я не единственный
ребёнок, ставший жертвой агрессии взрослых руководителей сайта
«Миротворец». Уже не первый раз «Миротворец» без письменного согласия
публикует личные данные таких же детей, как я, оказывая на них
психологическое давление и неся угрозу их безопасности.

Хочу попросить Вас о помощи и повлиять на правительство Украины по вопросу
скорейшего удаления личной информации несовершеннолетних и блокировки сайта
«Миротворец», его «зеркальных» копий и аккаунтов во всех социальных сетях для
обеспечения безопасности детей и соблюдения их прав. Также прошу ООН
контролировать соблюдение прав детей и немедленно реагировать при появлении
сайтов, подобных «Миротворцу». Пожалуйста, помогите сделать так, чтобы
правительство Украины исполняло законы своей страны и не подвергало опасности
жизни несовершеннолетних, не реагируя на работу сайта «Миротворец». Дети не
должны становиться жертвами травли и насилия из-за политических взглядов
взрослых.

С уважением, Фаина Савенкова»
\end{zzquote}


Она может даже обмениваться письмами с администрациями высокопоставленных
политиков Европейского союза, например, с администрацией президента Франции
Макрона. Но чем это поможет в реальности, когда бомбы продолжают взрываться, а
метатели бомб преследуют гонца, который сообщает о происходящем?

Со 2 июня 2014 года эта девочка живёт в условиях войны, которая грозит стать
войной против детей. Похоже, что теперь недостаточно заранее списать детей как
сопутствующий ущерб, как это обычно бывает; сегодняшняя война объявляет их
врагами. Вот выдержка с официальной страницы украинского так называемого
«миротворца», этот текст размещается в качестве подписи под каждым розыскным
объявлением: «Миротворческий центр просит правоохранительные органы
рассматривать данную публикацию на сайте как сообщение о совершении данным
гражданином умышленных действий против национальной безопасности Украины, мира,
безопасности человечества и международного правопорядка, а также других
уголовных преступлений».

Молодая писательница нашла в её юные годы отдушину в писательстве и добивается
первых успехов в местном литературном сообществе (с августа 2019 года она
является самым молодым членом Союза писателей Луганской Народной Республики),
также находя поддержку и помощь, чтобы рассказать миру свои истории ребёнка
войны. Её эссе, сказки и даже пьесы были опубликованы в различных литературных
журналах, газетах, антологиях и на онлайн-платформах на русском, английском,
французском, итальянском, сербском, болгарском и чешском языках и уже получили
несколько наград.

Несмотря на условия военных действий на Донбассе, семья Фаины поддерживает её в
её увлечении, за что некоторые члены семьи также указаны на сайте «Миротворец»
в рубрике «По данному адресу прописаны: [...]». Но её храбрую маму Наталью и
старшего брата Матвея анонимные осведомители не пугают. Они водят молодую
писательницу на литературные и журналистские мероприятия и поддерживают её в
совместной работе с российским автором научной фантастики Александром
Конторовичем. В сотрудничестве с ним в мае этого года Фаина Савенкова
опубликовала свою первую настоящую книгу – роман «Стоящие за твоим плечом».

Этот первый роман привил Фаине интерес к более крупным публикациям, и поэтому
она уже работает над следующим. Дальнейшие проекты находятся в стадии
разработки и, безусловно, будут реализованы с помощью людей, которым
небезразлична судьба якобы потерянного поколения, списанного как сопутствующий
ущерб.

Таким образом, мы надеемся, что в недалёком будущем юная писательница сможет
отбросить все опасения за собственную жизнь, вспомнить своё сегодняшнее прошлое
и написать суммарную рецензию. Но пока Фаина пишет о детях войны в своём эссе
«Детский смех победы» (2020):

\begin{zzquote}
«Сейчас мне 12 (ред) лет. Я живу в Луганске и мне известно, что такое
артобстрел или авианалёт. Половина моей жизни – война. Я не знаю, что
чувствовали такие же дети, как я, в том тяжёлом и страшном 41-м году, но мне
кажется, что это похоже на всё то, что испытывают сейчас дети Луганска и
Донецка. Иногда мне очень хочется написать моим сверстникам из 1941 года
письмо. Сказать им много слов поддержки, но потом я вспоминаю их жизненный
путь, и понимаю, что их стойкости и самоотверженности может позавидовать
каждый. Жизнь детей войны – это не истории об отчаянии, это истории о надежде,
даже если они полны трагедий. И пусть, что тогда, что сейчас мы не знаем, что
нас ждёт завтра и будет ли оно у нас – это «завтра», но мы уверенно шагаем
вперёд. Мы не сломались и с каждым днём становимся только сильнее, потому что
сила духа у нас в крови».
\end{zzquote}
