% vim: keymap=russian-jcukenwin
%%beginhead 
 
%%file 09_01_2022.fb.fb_group.story_kiev_ua.2.krasavica_zima
%%parent 09_01_2022
 
%%url https://www.facebook.com/groups/story.kiev.ua/posts/1836864253177034
 
%%author_id fb_group.story_kiev_ua,sirota_tatjana.kiev
%%date 
 
%%tags foto,kiev,zima
%%title Сегодня здесь царит красавица Снежная Зима!
 
%%endhead 
 
\subsection{Сегодня здесь царит красавица Снежная Зима!}
\label{sec:09_01_2022.fb.fb_group.story_kiev_ua.2.krasavica_zima}
 
\Purl{https://www.facebook.com/groups/story.kiev.ua/posts/1836864253177034}
\ifcmt
 author_begin
   author_id fb_group.story_kiev_ua,sirota_tatjana.kiev
 author_end
\fi

Сегодня здесь царит красавица Снежная Зима!

Аллеи, лесные тропинки, овраги и Днепровские склоны засыпал белый, пушистый снег! 

Малолюдно...

\raggedcolumns
\begin{multicols}{3} % {
\setlength{\parindent}{0pt}

\ii{09_01_2022.fb.fb_group.story_kiev_ua.2.krasavica_zima.pic.1}
\ii{09_01_2022.fb.fb_group.story_kiev_ua.2.krasavica_zima.pic.1.cmt}

\ii{09_01_2022.fb.fb_group.story_kiev_ua.2.krasavica_zima.pic.2}

\ii{09_01_2022.fb.fb_group.story_kiev_ua.2.krasavica_zima.pic.3}

\end{multicols} % }

Видимо, киевляне продолжают наслаждаться красотой праздничных улиц и площадей.

А мы с моей давней знакомой и участницей группы \enquote{КИ} Ольга Патлашенко
едем любоваться зимней сказкой в ботанический сад.

По заснеженным аллеям  направились к Тибетскому саду, полюбовались зимней
красотой Корейского и Японского садов.

Поднялись к Ионинскому монастырю, а потом посетили и Выдубицкий.

Как же здесь здорово! Настоящее зимнее чудо!

Нагуляли километров восемь (а может и больше)  и получили массу удовольствия.

Зимний ботанический сад имени Гришко.

9 января 2022 года.
