%%beginhead 
 
%%file 11_02_2023.fb.syniepalov_ivan.mariupol.1.ya_povernus_dodomu
%%parent 11_02_2023
 
%%url https://www.facebook.com/ivan.syniepalov/posts/pfbid02Z8sxqj6vb7FiGJAmNYXFsnLrX7r6Yj8muysJYRTdaVt2hW7Jx61UjQzZkce1WkS8l
 
%%author_id syniepalov_ivan.mariupol
%%date 11_02_2023
 
%%tags mariupol
%%title Я повернусь додому
 
%%endhead 

\subsection{Я повернусь додому}
\label{sec:11_02_2023.fb.syniepalov_ivan.mariupol.1.ya_povernus_dodomu}

\Purl{https://www.facebook.com/ivan.syniepalov/posts/pfbid02Z8sxqj6vb7FiGJAmNYXFsnLrX7r6Yj8muysJYRTdaVt2hW7Jx61UjQzZkce1WkS8l}
\ifcmt
 author_begin
   author_id syniepalov_ivan.mariupol
 author_end
\fi

Я повернусь додому.

Час від часу треба собі просто нагадувати: я повернусь додому. До дому, якого
немає.

Вісім років жив із цією самою думкою про Сніжне. Повертатися назавжди туди,
звісно, не збирався, але то було моє місце сили – приїжджав туди щоразу з
великою охотою.

Дорога від мого дому до бабусиного навпростець – десь із кілометр, і на кожному
кроці зі мною була якась історія. Якби взявся їх всі написати, це зайняло би
пів фейсбука.

Місто, яке завжди вважав рідним, хоч народився і не там, – місто, яке втратив
так давно, що воно встигло для мене померти разом зі зрадою друзів дитинства.

Але я повернусь додому.

До іншого дому. До нашого Моря-у-полі. Скільки має минути часу, щоб наші з ним
шляхи розійшлися, як розійшлися зі Сніжним? Що має статися в моєму житті, де я
маю осісти, щоб перехотіти повертатися? Сумніваюсь, що це можливо.

Місто залишилося тільки на ґуґлівських мапах – але це моє місто! І мені не
потрібні ті мапи, щоби перед сном подумки гуляти по його вулицях – по вулицях,
по яких я виходив сотні і тисячі кілометрів, – по вулицях, яких більше не існує
і, мабуть, ніколи вже не існуватиме.

Фотографії з мітингу 22 лютого вже не стимулюють дофамін, фотографії знищеного
й захопленого Маріуполя вже не вганяють в шал. На місці справжнього міста
дедалі сильніше проявляється якийсь чужий і далекий образ. 

Вкладаю собі цю просту думку, як персонажі Inception, щоб із неї проросло все
інше: Маріуполь буде українським, буде зовсім інакшим і сучасним – і я туди
повернуся.

Навіть якщо нас залишиться не пів мільйона, а сто тисяч. Навіть якщо весь актив
залишиться там, де вже облаштувався. Я знаю людей, які точно повернуться, – і я
повернуся разом із ними.

Я повернуся туди, де був дім. І він там буде знову.

%\ii{11_02_2023.fb.syniepalov_ivan.mariupol.1.ya_povernus_dodomu.cmt}
