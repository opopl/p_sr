% vim: keymap=russian-jcukenwin
%%beginhead 
 
%%file 23_07_2021.fb.fb_group.story_kiev_ua.2.skulptor_igor_lysenko.pic.2.cmt
%%parent 23_07_2021.fb.fb_group.story_kiev_ua.2.skulptor_igor_lysenko
 
%%url 
 
%%author_id 
%%date 
 
%%tags 
%%title 
 
%%endhead 

\iusr{Наталия Попова}
Чудесным экскурсоводом была Элеонора Рахлина. Светлая память ей и её талантливому сыну

\iusr{Олена Шинкаренко}
Ее экскурсии до сих пор у меня в памяти! Она владела такими знаниями о родном ей городе как никто другой!

\iusr{Марина Кузнецова}
В юности довелось побывать на ее экскурсиях. Впечатления остались на всю жизнь. Светлая память им.

\iusr{Світлана Кропивницька}
Спасибо огромное!
Светлая память!

\iusr{Римма Риммская}
Мне посчастливилось побывать на её экскурсиях.

\iusr{Георгий Майоренко}
Наверное, по уровню знаний и эрудиции, она была лучшим экскурсоводом нашего города.

\begin{itemize} % {
\iusr{Татьяна Гурьева}
\textbf{Георгий Майоренко} я тоже так думаю. А еще колоссальная харизма!

\iusr{Римма Риммская}
\textbf{Георгий Майоренко}, 

я ее уже помню совсем-совсем... Еле ходила. Помню, как на одной экскурсии
далеко за Киев в монастыре ей стало плохо и мы её отпаивали сердечными... А
потом помню, как на одной передаче она отвоевывала какой то памятник
архитектуры, а над ней глумились... А потом, мне из Америки позвонили друзья и
сказали, что Элеонора умерла... А в Киеве об этом ни в одной газете. Только под
давлением кто-то и где-то написал, что умерла экскурсовод. Это был не
экскурсовод, это была наша Совесть и наше богатство...

\begin{itemize} % {
\iusr{Татьяна Гурьева}
\textbf{Римма Риммская} ужасно

\iusr{Георгий Майоренко}
\textbf{Римма Риммская} Да, судьба Элеоноры Натановны была далеко не безоблачной. Но мы храним о ней добрую память. Игорь книгу издал о маме.

\iusr{Римма Риммская}
\textbf{Георгий Майоренко} , я видела его скульптуры!

\iusr{Георгий Майоренко}
\textbf{Римма Риммская} Игорь был талантливым парнем.

\iusr{Георгий Майоренко}
\textbf{Римма Риммская} Мама заставляла Игоря осваивать фортепиано, а его тянула лепка, скульптура.

\iusr{Татьяна Гурьева}
\textbf{Георгий Майоренко} в семье жили традиции от поколения к поколению. Сейчас это утеряно

\iusr{Георгий Майоренко}
\textbf{Татьяна Гурьева} 

Это грустная тенденция. Она наблюдается, когда люди выставляют
фотографии благородных предков, а рассказывают о них в стиле
заправский деревенщины.

\end{itemize} % }

\iusr{Татьяна Гурьева}
\textbf{Георгий Майоренко} имеет место

\end{itemize} % }

\iusr{Svetlana Naulko}

Невозможно забыть - слышу её умные речи, которые пробуждали в душе невиданное
состояние @igg{fbicon.heart.red}{repeat=3}

\begin{itemize} % {
\iusr{Георгий Майоренко}
\textbf{Svetlana Naulko} 

Яркая, харизматичная была женщина! Человек - энциклопедия! Но мы её все же
воспринимали как маму одноклассника. Тут у нас было личное. Элеонора Натановна
даже специально для нашего классе однажды провела экскурсию по Киеву.

\iusr{Люся Киевская}
\textbf{Георгий Майоренко} 

Бывала на экскурсиях Лены Натановны ( все ее звали Леной в кругу знакомых)., и
на занятиях / вечерах в истор. клубе « Клио» ., также - когда осталась без
помещения - мы с М. М. Потаповой - устроили ее вечер в библиотеке искусств ( Б.
Житомирская , 4)., приходилось слушать Э. Н. и в других залах ... Наш клуб «
Экслибрис» дружил с Э. Н., мы посещали ее встречи / экскурсии / лекции, вечера
... Незабываемая и непревзойденная - не только, как экскурсовод, но и как
искусствовед, историк, человек больших знаний, и большой души !! Человек с
открытым сердцем ! Светлая память ! ...

\iusr{Svetlana Naulko}
\textbf{Люся Киевская}  

@igg{fbicon.heart.with.ribbon} @igg{fbicon.face.smiling.tear}
@igg{fbicon.heart.with.ribbon} 

\iusr{Георгий Майоренко}
\textbf{Люся Киевская} И Игорь - сын ее был замечательным парнем!

\iusr{Георгий Майоренко}
\textbf{Люся Киевская} Игорь книгу о маме издал прекрасную.
\end{itemize} % }

\iusr{Leonid Dukhovny}

У меня с незабвенной Элеонорой Рахлиной связано слишком много воспоминаний,
чтобы они вместились в краткие заметки на ФБ. Светлая ей память!

\iusr{Георгий Майоренко}
Да, о ней в двух словах не скажешь!

\begin{itemize} % {
\iusr{Irena Visochan}
\textbf{Георгий Майоренко} Семья Рахлин - достояние Киева!
\end{itemize} % }

\iusr{Елена Сидоренко}

А помните, как она выиграла миллион? Я точно не помню, но по-моему, заложила
квартиру, а потом выиграла миллион благодаря своему уму и интеллекту и квартиру
выкупила...

\iusr{Георгий Майоренко}
Да, была какая-то история, связанная с выкупом квартиры.

\iusr{Роза Щупак}

Элеонора Натановна заложила квартиру, чтоб установить доску-памяти ее отцу
Натану Рахлину. Эту квартиру выкупили для нее наши люди, покинувшие
Родину, которая не ценила таких людей, как Натан Рахлин. Помню ее замечательные
экскурсии и лекции о Киеве. В перестроечное время Элеонора Натановна приезжала
к нам на работу и бескорыстно рассказывала о Киеве. Забыть это невозможно.
Светлая память ей и ее cыну.

\iusr{Люся Киевская}
\textbf{Роза Щупак} это не внук, а сын !
