% vim: keymap=russian-jcukenwin
%%beginhead 
 
%%file 06_04_2021.fb.promovugroup.1.mova_jazyk
%%parent 06_04_2021
 
%%url https://www.facebook.com/groups/promovugroup/permalink/947998205774041/
 
%%author 
%%author_id 
%%author_url 
 
%%tags 
%%title 
 
%%endhead 

\subsection{Тут Юлія Мендель, прес-секретар Зеленського про мову заговорила}
\Purl{https://www.facebook.com/groups/promovugroup/permalink/947998205774041/}

Тут Юлія Мендель, прес-секретар Зеленського про мову заговорила. Цитую без
перекладу, оскільки Юлія виступала на каналі « Дом» російською мовою : 

“От принятого в разгар предвыборной гонки 2019 года закона о языке вся страна
так и не заговорила на украинском” "Россия не является монополистом русского
языка. И нам пора давно самим демонополизировать русский язык. И громко
заявить, что в Украине есть украинский русский язык"

По-перше, української стало значно більше у публічному просторі. По-друге ,
закон ще навіть не повністю почав діяти. Він приймався не для того, щоб після
голосування усі раптом заговорили у публічному просторі українською, а для
того, щоб за 10 років українську у Харкові і у Львові було однаково чути
повсюдно. 

Росія, звичайно не є монополістом російської. Як і Україна - української. Але
Росія захищає і розвиває російську мову, а Україна - українську. І це єдиний
спосіб забути про колоніальне минуле. Замість того, щоб турбуватися за
«украинский русскій» офіс Зеленського мав би дбати про українську українську).
І непогано б захистити теж російську українську, польську українську, румунську
українську і розвивати різні українські мови, а не різні російські )))
