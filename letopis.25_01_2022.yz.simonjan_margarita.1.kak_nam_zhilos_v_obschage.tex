% vim: keymap=russian-jcukenwin
%%beginhead 
 
%%file 25_01_2022.yz.simonjan_margarita.1.kak_nam_zhilos_v_obschage
%%parent 25_01_2022
 
%%url https://zen.yandex.ru/media/m_simonyan/po-sluchaiu-dnia-studenta-publikuiu-tut-otryvok-iz-svoei-pervoi-knigi-v-moskvu-o-tom-kak-nam-jilos-v-obscage-61efbd4b3615c85db55639a9
 
%%author_id simonjan_margarita
%%date 
 
%%tags kniga,obschezhitie,prazdnik.tatjanin_den,studenty
%%title Как нам жилось в общаге
 
%%endhead 
 
\subsection{Как нам жилось в общаге}
\label{sec:25_01_2022.yz.simonjan_margarita.1.kak_nam_zhilos_v_obschage}
 
\Purl{https://zen.yandex.ru/media/m_simonyan/po-sluchaiu-dnia-studenta-publikuiu-tut-otryvok-iz-svoei-pervoi-knigi-v-moskvu-o-tom-kak-nam-jilos-v-obscage-61efbd4b3615c85db55639a9}
\ifcmt
 author_begin
   author_id simonjan_margarita
 author_end
\fi

Общежитие номер два, вечно полное надежд и не знающее разочарований, провожало
минувшие сутки.

\ifcmt
  ig https://avatars.mds.yandex.net/get-zen_doc/1587710/pub_61efbd4b3615c85db55639a9_61efbd555d05e66deb7cc924/scale_1200
	@wrap center
	@width 0.7
\fi

В комнате триста пятнадцать было, как обычно, сыро. На столе на газетном
обрывке валялся высохший кусочек колбасы, стояли грязные чашки с заваренными по
четвертому разу чайными пакетиками. На книжных полках в беспорядке лежали
библиотечные учебники, много Борхеса, чуть-чуть Маркеса и весь Достоевский. На
книжки было навалено много разного хлама.

У шкафа висела Памела Андерсон с засиженной мухами грудью. Пол в комнате
считался паркетным, но был скрыт под слоями утоптанной грязи'.

А вы жили в общаге? Как это было? :)
