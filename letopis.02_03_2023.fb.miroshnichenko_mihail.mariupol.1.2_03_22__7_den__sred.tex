%%beginhead 
 
%%file 02_03_2023.fb.miroshnichenko_mihail.mariupol.1.2_03_22__7_den__sred
%%parent 02_03_2023
 
%%url https://www.facebook.com/mms.couch.ips/posts/pfbid0HiMDtqQENwxkH1WoyXqD8vNoq2Z3nycMumywffVJGpyiVKgWBLWZa4BywGwWgPuul
 
%%author_id miroshnichenko_mihail.mariupol
%%date 02_03_2023
 
%%tags mariupol,mariupol.war,dnevnik
%%title 2.03.22. 7 день. Среда. Ранний подъем и эвакуация в укрытие
 
%%endhead 

\subsection{2.03.22. 7 день. Среда. Ранний подъем и эвакуация в укрытие}
\label{sec:02_03_2023.fb.miroshnichenko_mihail.mariupol.1.2_03_22__7_den__sred}

\Purl{https://www.facebook.com/mms.couch.ips/posts/pfbid0HiMDtqQENwxkH1WoyXqD8vNoq2Z3nycMumywffVJGpyiVKgWBLWZa4BywGwWgPuul}
\ifcmt
 author_begin
   author_id miroshnichenko_mihail.mariupol
 author_end
\fi

«На месте каждой жизненной пропасти, когда-то был очень прочный фундамент, на
котором стоял очередной уверенный в себе человек.»

2.03.22.

7 день. Среда.

➡️Ранний подъем и эвакуация в укрытие.

В подвале немного сыро и прохладно, места мало, сидим на лавках укрывая ноги
одеялами, чтобы не замёрзнуть. Поехал на автомобиле в квартиру на ул.
Зелинского. Свет дали, прочитал новости в интернете и узнал, что на Кирова был
прилёт и пострадала девятиэтажка. Новости с Волновахи, это 50 км. от Мариуполя
в сторону Донецка обстреливали и были авиаудары. 

Когда был в квартире, пришла информация, что бы все шли в укрытие, спустился
первый раз в подвал, там стояли стулья, вода,  было темно но тепло, никто из
соседей в подвал не пришёл. Подвал без запасного выхода, прятаться в нем не
безопасно, если завалит вход то он станет для спасающихся братской могилой. Наш
дом одноподъездный, ухоженный мне очень нравится в нем жить, очень хорошие и
дружные соседи, я беспокоился о их безопасности, мы обменивались информацией,
какие стены несущие, где лучше прятаться, если прилетит. 

Решили, что самое безопасное место в подъезде, так как там шахта лифта и вокруг
несущие стены. Позвонил маме, успокоил, сказал, что мы в «безопасности» и ей не
стоит беспокоиться. Мои родители живут в Мангуше, у них был прилёт, пострадал
дежурный магазин и рядом стоящие 5 этажные дома. Они живут далеко от этого
места но Я все равно очень  переживал за родителей и беспокоился о их
безопасности. Около часа дня я позвонил сестре, Она сказала, что очень страшно,
там где они находятся идёт обстрел и они прячуться в подвале. Опять пропала
связь, надеюсь позже появиться и я смогу связаться с родными.

Вокруг громыхало, каждый день стоновиться только хуже и я даже не представляю
чего ожидать.

Вечером мы с Джеком, моим боевым товарищем, поднялись на 7 этаж в квартиру. Шли
пешком, лифт не работал уже неделю. Света, интернета и  связи с семьёй не было.

После подвала, где мы прятались на поселке Западном, в котором возникла мнимая
безопасность, в квартире было страшновато.

Я перетощил кресло в спальню сына, перевернул его и получился защитный кокон,
меня окружали несущие стены и закрытое окно столешницей,  на пол постелил
подушки из того же кресла и взял одеяло сына. 

Слышны автоматные очереди, к окнам я опасаюсь подходить ради собственной
безопасности.

Джек тоже боялся и не отходил от меня.

Он помогал мне справиться с волнением и мне было не так одиноко.

В телефоне я нашел скаченную книгу Дейла Корнеги "Как перестать беспокоиться."

Чтение успокаивало и отвлекало  от  шума боевых действий за окном квартиры.

Постараюсь уснуть, надеюсь проснутся.
