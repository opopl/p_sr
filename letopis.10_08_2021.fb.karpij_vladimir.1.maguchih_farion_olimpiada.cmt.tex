% vim: keymap=russian-jcukenwin
%%beginhead 
 
%%file 10_08_2021.fb.karpij_vladimir.1.maguchih_farion_olimpiada.cmt
%%parent 10_08_2021.fb.karpij_vladimir.1.maguchih_farion_olimpiada
 
%%url 
 
%%author_id 
%%date 
 
%%tags 
%%title 
 
%%endhead 
\subsubsection{Коментарі}
\label{sec:10_08_2021.fb.karpij_vladimir.1.maguchih_farion_olimpiada.cmt}

\begin{itemize}
%%%fbauth
%%%fbauth_name
\iusr{Михаил Мищишин}
%%%fbauth_url
%%%fbauth_place
%%%fbauth_id
%%%fbauth_front
%%%fbauth_desc
%%%fbauth_www
%%%fbauth_pic
%%%fbauth_pic portrait
%%%fbauth_pic background
%%%fbauth_pic other
%%%fbauth_tags
%%%fbauth_pubs
%%%endfbauth
 

Тут глузд у тому, щоб поділити українців та їх землю. Поширити Крим та Донбас
із їх чітким розмежуванням людей на всю Україну. Той, хто веде себе, як Литва,
Латвія чи Естонія - рано чи пізно стане таким же. За розмірами. Між Германією
та Росією вцілити та розвиватись може щось нейтральне. На кшталт Фінляндії,
Швеції, Норвегії, Ісландії тощо. Усе решта приречене на дружнє або недружнє
розшматування та поглинання. Як недружньо Європа поглинула Югославію. Як
декілька разів Берлін та Москва ділили Польщу, яка напередодні Другої світової
вела себе досить агресивно, брала участь із німцями у поділі Чехословаччини.
Але виявилась неспроможною протистояти Німеччині. І це дуже нагадує сьогодні
нашу українську позицію та очікування поділу Росії Заходом. Але годі це
пояснити нашим патріотам. Україну не раз вже били та окуповували, коли ці
селяни із ступенями докторів та академіків замість того, щоб сіяти та збирати
хліб, що вони вміють краще від багатьох, починали засідати у Радах та займатись
політикою. Але це, що в-очевидь та наочно випливає з нашої історії, нам знову і
знову незрозуміло. Що ж, значить, матимемо ще гірки уроки. Дай Бог, щоб маляр
Хівренко який намалював апокаліптичний Київ, помилився. Дай Бог.

\index{Хивренко, Иван! Киев, Апокалипсис, Картины}

% -------------------------------------
\ii{fbauth.ageeva_svetlana.jagotyn.ukraina}
% -------------------------------------

До яких пір ота баба-пенсіонерка буде каламутити воду!? Тепер вам зрозуміло,
як почалася війна, і хто винуватиця її початку!? Все наше свідоме життя ми
читали лозунги - " Спорт поза політики" , " О спорт, ти - мир"!!! Але це не для
Ірини Фаріон! Вона живе в своєму, створеному нею ж західно- бандерському світі,
і отруює цією речовиною все довкола! Але, як ми знаємо, один в полі не воїн,
хоча й всигла, зізнаємося собі, наробити горя в нашій країні!!!

\ifcmt
  ig https://scontent-frx5-2.xx.fbcdn.net/v/t39.1997-6/s168x128/17634213_1652591098100624_731967241620291584_n.png?_nc_cat=1&ccb=1-5&_nc_sid=ac3552&_nc_ohc=kf1tD45CW_MAX82IC27&_nc_ht=scontent-frx5-2.xx&oh=62bd55c79abe637f73a9c4a68c02dbfe&oe=613970F7
  width 0.2
\fi

\begin{itemize}
%%%fbauth
%%%fbauth_name
\iusr{Михаил Мищишин}
%%%fbauth_url
%%%fbauth_place
%%%fbauth_id
%%%fbauth_front
%%%fbauth_desc
%%%fbauth_www
%%%fbauth_pic
%%%fbauth_pic portrait
%%%fbauth_pic background
%%%fbauth_pic other
%%%fbauth_tags
%%%fbauth_pubs
%%%endfbauth
 

\textbf{Светлана Агеева} 

Як галичанин, скажу, що львів'яни - то переважно
українські селяни, яких порозселяли у польські та єврейські доми у Львові, коли
відбулося возз'єднання Західної України із Східною у 1939 році. Зрозуміло, що
для поляків, яким належав Львів до того (хоча були часи, коли уся Польща
входила до складу Росії чи, навпаки, поляки захоплювали, якщо не помиляюсь,
навіть Київ), так от оту болісну для поляків дату (та радісну для нас,
українців) під час розпаду СРСР ми почали...засуджувати, критикуючи пакт
Молотова Ріббентропа. Вже тоді здоровий глузд моїх земляків - галичан викликав
великі сумніви. Тому що ми жили у Львові, що відійшов Україні від Сталіна та
згідно пакту Молотова-Ріббентропа. І одночасно...засуджували одного й друге. Це
чисто польська позиція, за якою стоїть прагнення Польщі повернути собі
Галичину. Але у дивний спосіб ми, галичани, фактично виступаємо за це,
критикуючи те, що дало нам Львів (чи повернуло), відстоюючи фактично польский
інтерес під українським прапором. Так що нічого дивного у пані Фаріон не є. На
жаль, усі ми, галичани, трошки викручені та засмучені життям:). Говорю це з
любов'ю та вірою у земляків, що все так не буде.\Smiley[1.0][yellow]. Але Україна - це її Схід та
Центр. А не Захід. Там усе понівечене та викручене польським довоєнним ярмом,
проти якого повстали бандерівці. На жаль, у свідомість українців та,
передовсім, галичан якось вбили, що Львів - це Україна. А усе решта - так собі.
Гадаю, що вже зрозуміло, кому це вигідно. Аж ніяк не Україні, як на мене.

%%%fbauth
%%%fbauth_name
\iusr{Володимир Карпій}
%%%fbauth_url
%%%fbauth_place
%%%fbauth_id
%%%fbauth_front
%%%fbauth_desc
%%%fbauth_www
%%%fbauth_pic
%%%fbauth_pic portrait
%%%fbauth_pic background
%%%fbauth_pic other
%%%fbauth_tags
%%%fbauth_pubs
%%%endfbauth
 
\textbf{Михаил Мищишин} 

Скажу відверто, галичан нам зрозуміти важко. Свого часу
у Яготині у мене був сусід-галичанин з Тернопільщини. Коли у нього померла
жінка і він, залишившись сам, вирішив повернутися на Західну Україну і всім
говорив: Поїду додому на Западну (так він казав), житиму там, бо там люди кращі,
ніж у вас тут. Що мене тоді вражало: проживши на Наддніпрянщині 25 років, він
все одно своїм не став, у нього залишився поділ ВИ-МИ. І це явище я спостерігав
часто. " Ми справжні українці, а ви-східняки, сурогантні". Дивно спостерігати
протягом всього життя картину, коли приїжджали люди з інших місць
України - Вінниччини, Одещини, Херсонщини,Сумщини і ніхто не проводив лінії
ВИ - МИ. Складалось враження, що між галичанином і наддніпрянином більша
різниця, аніж між німцем і австрійцем. Може, мав рацію Табачник, вважаючи, що
галичани - ослов'янені кельти? Я не хочу бути схожим на галичан в цьому питанні,
але все ж зауважу, що Україна - це не "Борислав сміється" чи "На панщині пшеницю
жала " Це - не туга і вічне ниття про тяжку долю. Україна - це "Енеїда"
Котляревського, якою зачитувався цар Микола Перший. Україна - це відсутність
національного і релігійного угноблення з боку Росії, яка вважала малоросів
одним народом. Україна - це практично відсутність серйозної кількості
національних образ і травм. Україна - це"Спалю пана і втечу на Січ". Це степове
приволля і особиста відповідальність за безпеку родини, бо татари можуть
наскочити будь-коли. В цьому ми відрізняємось від галичан, які продовжуть жити
старими травмами, образами, люблять роз'ятрювати старі рани і дутися на весь
світ, ніби їм хтось щось винен. Пардон, якщо я зачепив Вашу галицьку струну,
Михайле. Але мені дуже шкода, що нині Україна вибудовує своє життя за лекалами
галицького розуміння і сприйняття дійсності. І з цієї точки зору реакція пані
Фаріон на рядову подію спотивного життя є симптомом серйозної душевної хвороби.
Її Ви визначили давно: сутність Людини сьогодні підім'ята і розтоптана
націонал-патріотизмом. Такі Фаріон є скрізь, але ТАМ  - вони маргінали. Зате у
нас - ньюсмейкери і лідери думок...

%%%fbauth
%%%fbauth_name
\iusr{Михаил Мищишин}
%%%fbauth_url
%%%fbauth_place
%%%fbauth_id
%%%fbauth_front
%%%fbauth_desc
%%%fbauth_www
%%%fbauth_pic
%%%fbauth_pic portrait
%%%fbauth_pic background
%%%fbauth_pic other
%%%fbauth_tags
%%%fbauth_pubs
%%%endfbauth
 

\textbf{Володимир Карпій} Та чого ж. Я з Вами повністю згоден, Володимире.
Східняків так не гнобили, як нас, галичан. В них менше національних комплексів.
І ніхто не робить ніякого поділу між українцями. Це дуже вірна та правильна
проукраїнська позиція, як на мене. Чому тепер пхають допереду галичан? Бо ми
ділимо Україну. А це влаштовує і німаків, і руських, і американців. Так що Ви
висловились дуже і дуже по суті, Володимире, як на мене.

%%%fbauth
%%%fbauth_name
\iusr{Валерий Веревкин}
%%%fbauth_url
%%%fbauth_place
%%%fbauth_id
%%%fbauth_front
%%%fbauth_desc
%%%fbauth_www
%%%fbauth_pic
%%%fbauth_pic portrait
%%%fbauth_pic background
%%%fbauth_pic other
%%%fbauth_tags
%%%fbauth_pubs
%%%endfbauth
 
\textbf{Володимир Карпій} Истинна правда, солидаинн!

\end{itemize}

% -------------------------------------
\ii{fbauth.verevkin_valerij.feodosia.krym.rossia.gorlovka}
% -------------------------------------

А кто дал право фрау Форион, или мадам ТолиЦой Толи Нецой, судить за всю
страну? "Не сеяли, не пахали", а только попиз...ми занимались. А при "советах
", еще и "стучали " по подписке на своих. Ан нет теперь мы судим! Угомонитесь
КГБишные подстилки, вспомните блжьи заповеди: " Не судите и ...",
\end{itemize}

