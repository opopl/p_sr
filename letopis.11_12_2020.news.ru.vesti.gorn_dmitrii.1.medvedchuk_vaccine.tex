% vim: keymap=russian-jcukenwin
%%beginhead 
 
%%file 11_12_2020.news.ru.vesti.gorn_dmitrii.1.medvedchuk_vaccine
%%parent 11_12_2020
 
%%url https://www.vesti.ru/article/2497239
 
%%author Горн, Дмитрий
%%author_id gorn_dmitrii
%%author_url 
 
%%tags vaccine,russia,ukraine
%%title США запретили Украине покупать российскую вакцину и отказались поставлять свою
 
%%endhead 
 
\subsection{США запретили Украине покупать российскую вакцину и отказались поставлять свою}
\label{sec:11_12_2020.news.ru.vesti.gorn_dmitrii.1.medvedchuk_vaccine}
\Purl{https://www.vesti.ru/article/2497239}
\ifcmt
	author_begin
   author_id gorn_dmitrii
	author_end
\fi

\video{https://player.vgtrk.com/iframe/video/id/2249078/start_zoom/true/showZoomBtn/false/sid/vesti/isPlay/true/mute/true/tid/2366/?acc_video_id=2359474}

\ifcmt
pic https://cdn-st1.rtr-vesti.ru/vh/pictures/xw/307/699/1.jpg
\fi

\index[rus]{Вакцина!Спутник V, Украина-Россия, 11.12.2020}

Карантин выходного дня вместо вакцинации. Гарантировать украинцам безопасность
от коронавируса власти решили без помощи соседа. Договоренности о поставках и
возможности производства на своей территории "Спутника V" проигнорировали, а
отказ Украины от российской вакцины озвучил Запад.

"За нашу власть ответил временно поверенный США. Он сказал, что Украина не
будет закупать", – констатирует факт Виктор Медведчук, руководитель партии
"Оппозиционная платформа — За жизнь".

Вслед за Западом отметился и весь верхний эшелон украинской власти. Зеленский,
как выразился, ждет "настоящую вакцину", в его понимании любую, лишь бы не
российскую. Премьер-министр попытался найти более вескую причину. Сославшись на
отсутствие результатов третьего этапа клинических испытаний. Но, так или иначе,
оба свели претензии к геополитике.

"Геополитические моменты для нашего премьера и президента основные – это
аксиома. Но заявление премьера – это лишь отговорка. Не прошла вакцина третью
стадию испытаний, но и другие тоже не прошли! Я бы хотел напомнить нашим
умникам, что третья стадия нужна для того, чтобы определить исключительно
эффективность вакцины", – говорит Виктор Медведчук, руководитель партии
"Оппозиционная платформа — За жизнь".

Борьба с коронавирусом на Украине, по всей видимости, откладывается до весны.
Глобальный фонд доступа к вакцинам против COVID-19 обещал помочь, напомнил
Шмыгаль. А пока украинцев вместо вакцинации отправят в заточение. Правительство
уже объявило о полном январском карантине. Кафе, рестораны, магазины
практически весь месяц работать не будут, школы и вузы уходят на каникулы.

"Министерство здравоохранения уже направило заявку для получения вакцины в
рамках программы COVAX, по которой Украина сможет получить восемь миллионов доз
вакцины для четырех миллионов украинцев", – похвастался Денис Шмыгаль,
премьер-министр Украины.

Но это лишь 10 процентов всего населения Украины. Да и придет вакцина только в
марте и то не вся, а лишь один миллион доз. Тогда как к этому же моменту
собственное производство можно было бы вывести на показатели в разы выше. Об
этом в интервью телеканалу "Россия 24" рассказал Виктор Медведчук.

"У нас радикальная в этом плане позиция: донести до людей что вакцина есть, и
она действует. Чем больше узнают, тем больше шанс вынудить власти идти по пути
приобретения или производства вакцины", – считает Виктор Медведчук.

У украинцев, впрочем, мнение о российской вакцине уже сложилось:

- Нужно обязательно эту вакцину вводить и не надо пиариться где-то там, а надо
свою иметь голову и прислушиваться к людям, которые говорят, как надо делать, а
не слушать Запад.

- Если есть возможность производить этот продукт, эту вакцину, то почему бы и
нет. Я не понимаю, почему они стоят.

- Вакцину необходимо срочно ввозить на Украину и прививки делать всем.

- Чтобы делали у нас только российскую вакцину. Сейчас выступают и говорят, что
и ту, и ту, но не российскую. А для нас российская – самая лучшая.

- Американской не доверяем! Только российская!

Но диалога власти и народа нет. От того, вероятно, и на Донбассе предновогодняя
тишина вновь разорвана.

"Говорить сегодня о плане "Б", об альтернативе Минским соглашениям, о замене
участников нормандского или минского формата. Это все от того, что люди не
хотят достигать договоренностей. И ищут путь, чтобы отказаться от реализации
минских, и нести политику, направленную на то, чтобы мир на Донбасс так и не
пришел", – считает Виктор Медведчук.

А не нужен мир, считают политологи, по одной простой причине. Электоратом
действующей власти жители неподконтрольных сегодня Киеву территорий никогда не
станут.

