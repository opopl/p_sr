% vim: keymap=russian-jcukenwin
%%beginhead 
 
%%file 19_07_2021.fb.aganjan_aleksandr.sevastopol.1.rossia_mentalitet
%%parent 19_07_2021
 
%%url https://www.facebook.com/aganyan.alexander/posts/3866184856836863
 
%%author Аганян, Александр (Севастополь)
%%author_id aganjan_aleksandr.sevastopol
%%author_url 
 
%%tags chelovek,dusha,mentalitet,nacia,rossia,rusmir,sevastopol
%%title Специфика российского менталитета
 
%%endhead 
 
\subsection{Специфика российского менталитета}
\label{sec:19_07_2021.fb.aganjan_aleksandr.sevastopol.1.rossia_mentalitet}
 
\Purl{https://www.facebook.com/aganyan.alexander/posts/3866184856836863}
\ifcmt
 author_begin
   author_id aganjan_aleksandr.sevastopol
 author_end
\fi

Специфика российского менталитета.

У  нас особенно явно сказывается правота евангельских слов об отсутствии пророка в своем Отечестве. 

Русские первыми в мире изобрели вакцину от коронавруса. Однако привито всего
14\% населения (впереди нас Албания, Бразилия, Сальвадор) Русские учёные первыми
в мире заявили об опасности глобального потепления, вызвав бурю негодования со
стороны своих западных коллег (академик М.И Будыко, 1971 г, симпозиум по
динамической климатологии), однако именно у нас узкая прослойка
конспирологически настроенных идиотов убедила широкие слои общества в том, что
никакого потепления нет (но это уже совсем маргинальная точка зрения). При том,
что сами идеи антиваксерства, как и климатический скептицизм, пришли к нам из
главного рассадника конспирологических концепций, - из США. Ну и напоследок,
немного любопытных фактов - из воспоминаний академика М.И. Будыко, климатолога
и геофизика, о всемирном симпозиуме 1971 г.

\href{https://indicator.ru/earth-science/globalnoe-poteplenie.htm}{%
«Это больше не является предметом дискуссии»: ученые о глобальном потеплении, indicator.ru, 21.02.2017}

\ifcmt
  pic https://external-lga3-1.xx.fbcdn.net/safe_image.php?d=AQFPx4WSKBRgPMpB&w=500&h=261&url=https%3A%2F%2Findicator.ru%2Fthumb%2F1200x628%2Ffilters%3Aquality%2875%29%2Fimgs%2F2019%2F08%2F05%2F10%2F3488159%2Fbb9ed8de899e991ebd80849ada3440b56da7c52f.jpg&cfs=1&ext=jpg&ccb=3-5&_nc_hash=AQGTI6EQ5ZSEkPJE
  width 0.4
\fi

"Обычно в конце организаторы говорят какие-то общие слова. После этого выходит
кто-либо из именитых участников и тоже говорит общие слова, благодарит
организаторов за гостеприимство и т.п., после чего конференция закрывается и
все идут на официальный банкет. Я же вместо общих слов сформулировал идею,
показавшуюся всем абсолютно неприемлемой, сказав, что глобальное потепление
неизбежно. Я представил некоторые количественные оценки и выразил надежду, что
все попытаются заняться этой проблемой, потому что она очень важна. Это вызвало
взрыв негодования, все явно демонстрировали крайнее нерасположение к
докладчику. Разумеется, никто и не думал больше о том, чтобы благодарить
организаторов. Несколько очень известных ученых выступили, сказав, что
человеческая деятельность не может оказать никакого влияния на климат, что
изменения климата непредсказуемы и что внедрять в умы такие идеи совершенно
недопустимо".....

Впрочем нет дыма без огня. Степень влияния человечества на процесс потепления тема очень непростая, как и эффективность прививок.

Только что встретил  старого твоварища привился в апреле мае спутником. Обе
прививки. Поехал в отпуск в июне в Питер - поймал индийский штамм наградил им
жену антипрививочницу и дочку.Дочка переболела в легкой форме товарищ в
среднелегкой ( боли в мышцах четыре  дня  тепературы  и несколько дней
слабости) кашля и одышки не было. У ждены было хуже - пошла одышка 30\%
поражения легких болела две недели. Сейчас еще легкая слабость. он полностью в
форме.

Вывод не делай нет статистики.

Гипотеза:  те, кто привит должны соблюдать осторожность, а не чуствовать себя
суперменами. Они могут быть и носителями и переносчиками. Вакцина работает, но
эффективность с поялением новых штаммов постепенно  снижается.
