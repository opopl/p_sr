% vim: keymap=russian-jcukenwin
%%beginhead 
 
%%file 22_12_2020.news.ua.strana.sibircev_aleksandr.1.ukraina_respublika.slavskaja_otg
%%parent 22_12_2020.news.ua.strana.sibircev_aleksandr.1.ukraina_respublika
 
%%url 
 
%%author 
%%author_id 
%%author_url 
 
%%tags 
%%title 
 
%%endhead 

\subsubsection{Славская ОТГ}

После резонанса часть селян открестилась от земляков, дескать, отделяться
они не хотят.

"Это возмутились те, у кого могут землю отобрать, нас к сепаратизму не
приплетайте. Мы - за единую Украину. А то, что курорт построят, это
хорошо, туристы будут", - оправдывается с разговоре с нами селянин
Николай.     

Во время собрания присутствовал уроженец Днепропетровска Анатолий
Балахнин, который называет себя общественным активистом и борцом с
карантинными ограничениями. Сантехник по специальности, он заявляет, что
занимается "возвращением украинскому народу права государственности".

Балахнин не признает действующих органов центральной и местной власти в
Украине, поскольку, по его мнению, они являются правопреемниками СССР.

На Западной Украине некоторые связывают днепровца с людьми Коломойского,
сам он это отрицает, заявляя, что не знаком с ними. От этих слухов, к
слову, пошла версия, настоящей причиной "отделения" Верхней Рожанки стала
попытка помешать строительству конкурента "Буковеля", который принадлежит
партнерам Коломойского.

Однако подобные собрания прошли не только в Верхней Рожанке, но и во
Львове, Хмельницком и на Ровенщине (тоже при участии Балахнина). То есть
речь вряд ли может идти о том, что имеет место бизнес-подоплека, связанная
с горнолыжными курортами.

В Хмельницком, например, 21 октября состоялась "конференция
самоуправляющегося Запада", на которой несколько десятков человек создали
"городской совет". Также "ликвидировали" действующий исполнительный
комитет и выбрали "депутатов" и "председателя совета".

По этому случаю в СБУ сообщили: "В Хмельницком СБУ задокументировала
противоправную деятельность организованной группы, участники которой
пытались создать фейковый горсовет областного центра для расшатывания
общественно-политической ситуации в регионе". По хмельницким сборам завели
дело по статье 109 (действия, направленные на насильственное изменение или
свержение конституционного строя или на захват государственной власти).

Собрание рожанцев освещал в соцсетях львовянин Виктор Палас, который
представляется журналистом и членом организации "Стоп коррупции".  


Львовянин Виктор Палас вел стрим с собрания селян, на котором они провозгласили
свой орган власти. Видео: Facebook Виктора Паласа

Палас также записывал стрим с собрания во Львовском дворце искусств 8
декабря, на котором два десятка человек под председательством того же
Балахнина объявили о создании "органа местного самоуправления Львовского
городского совета".

У группы львовян - свои мотивы, часть из них - противники и Садового,
которого в четвертый раз переизбран мэром, и Зеленского. 

\ifcmt
pic https://strana.ua/img/forall/u/10/85/%D0%A1%D0%BD%D0%B8%D0%BC%D0%BE%D0%BA_%D1%8D%D0%BA%D1%80%D0%B0%D0%BD%D0%B0_2020-12-22_%D0%B2_13.54_.52_.png
caption Львовские "сепары" под предводительством днепровца Балахнина (на кадре крайний слева) собрались в местном дворце искусств. Фото: стоп-кадр с видео в Facebook Виктора Паласа
\fi


"Я голосовал за Синютку, сколько можно Садовому править. И еще мне не
нравится эта зеленая власть, которая поддерживала Садового. Я - за
Порошенко. Узнал через знакомых, что во Львове будут создавать свой
горсовет, и решил поучаствовать", - поделился с нами участник львовского
мероприятия, попросивший об анонимности.  
