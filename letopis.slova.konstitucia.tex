% vim: keymap=russian-jcukenwin
%%beginhead 
 
%%file slova.konstitucia
%%parent slova
 
%%url 
 
%%author 
%%author_id 
%%author_url 
 
%%tags 
%%title 
 
%%endhead 
\chapter{Конституция}
\label{sec:slova.konstitucia}

%%%cit
%%%cit_head
%%%cit_pic
%%%cit_text
Почему защита русского языка - лишь пустая декларация в \emph{Конституции}?
Почему свободное использование, развитие и защита русского языка в Украине —
это лишь пустая декларация в \emph{Конституции Украины}?  Почему у Зеленского
хотят ввести уголовную ответственность за оскорбление украинского языка, но на
оскорбление русского языка у Зеленского всем плевать?  Почему нам —
украинизация всех сфер общественной жизни, а \enquote{Кварталу 95} и сериалам
Зеленского — спецзакон, позволяющий русский язык?  Есть ли проблемы с русским
языком и гуманитарными правами в Украине и какой статус для русского языка
нужен в Украине?
%%%cit_comment
%%%cit_title
\citTitle{Почему на оскорбление русского языка у Зеленского всем плевать?}, 
Александр Скубченко, strana.ua, 21.06.2021
%%%endcit

%%%cit
%%%cit_head
%%%cit_pic
%%%cit_text
В політичній сфері, мусимо розпочати процес інклюзивної Конституанти. Бо в
існуючому політико-правовому дизайні Республіки не збудуєш. Так само варто
говорити про відповідальне громадянство і нові принципи побудови управління в
країні. Ми маємо нарешті розірвати пуповину з пострадянською спадщиною у
вигляді \emph{Конституції} 1996 року та ухвалити новий Основний Закон!  Хто зробить
все це? Нові політичні суб'єкти.  І не потрібно мені тикати під ніс \enquote{зеленими}
та різними потойбічними \enquote{голосами}.
Існуючі політичні гравці в ситуації кризи та потужних викликів демонструють
повне ідейне та організаційне банкрутство
%%%cit_comment
%%%cit_title
\citTitle{Маніфест реаліста або час збирати каміння. Воно нам скоро знадобиться - ХВИЛЯ}, 
Виталий Кулик, analytics.hvylya.net, 17.06.2021
%%%endcit


%%%cit
%%%cit_head
%%%cit_pic
%%%cit_text
Сторонники введения этого механизма говорят, что в больницах и так идут
огромные теневые поборы с пациентов, и лучше, чтобы эти средства платились
легально. Видимо, опции существенно поднять зарплаты медикам и потом вынудить
их прекратить поборы с помощью правоохранительной системы - власти даже не
рассматривают. И готовят решения, которые прямо нарушают \emph{Конституцию}.  К слову,
о медицине косвенно говорит еще и 22 статья. Там написано, что при утверждении
новых законов не допускается сужение содержания и объема существующих прав и
свобод. Однако медицинская реформа урезает такие права. Как, собственно, и
пенсионная реформа. Да и языковая
%%%cit_comment
%%%cit_title
\citTitle{День Конституции Украины 28 июня - какие статьи нарушаются сильнее всего}, 
Оксана Малахова; Максим Минин, strana.ua, 28.06.2021
%%%endcit

%%%cit
%%%cit_head
%%%cit_pic
%%%cit_text
С Праздником друзья! Желаю всем, чтобы когда-то \emph{Конституция} стала действующим
Законом, а не мечтой и символом!  Но, пока-что, мы в Уип Уип проанализировали и
насчитали, что из 161 статьи \emph{Основного Закона Украины} на сегодня системно и
последовательно нарушаются 125 - около 78\%. 36 статьи \emph{Основного Закона},
которые \enquote{условно выполняются}, содержат в основном номинальные или отсылочные
нормы. Из первых двух разделов \emph{Конституции} полностью выполняется только статья
20 \enquote{Государственные символы}.  \emph{Конституция} - это очень тонкий инструмент, и
если хотя бы одна её статья цинично и публично нарушается, то это ставит под
угрозу разрушения весь документ
%%%cit_comment
%%%cit_title
\citTitle{Из 161 статьи Конституции Украины системно нарушаются 125 / Лента соцсетей / Страна},
Руслан Бортник, strana.ua, 28.06.2021
%%%endcit

%%%cit
%%%cit_head
%%%cit_pic
%%%cit_text
А уже в 2014-м на волне хаоса и беззакония нардепы «вернули» \emph{Конституцию}
образца 2004-го постановлениями парламента. Затем они избавились от тех судей
КСУ, которых могли уволить по своей квоте, а ГПУ дали указание преследовать
всех, кто имел отношение к Решению КСУ о возврате к \emph{Конституции} образца 96-го
года.  Так «революционеры» сперва надругались над \emph{Основным Законом}, а после
растоптали его. И это логично! Если оценивать акты «майданного» периода на
предмет \emph{конституционности}, то выяснится, что решения, принятые незаконно
назначенными руководителями, не являются легитимными, а многих «уважаемых»
людей следует немедля отправить в барак
%%%cit_comment
%%%cit_title
\citTitle{Ранее, чтобы выглядеть прилично, власть меняла Конституцию / Лента соцсетей / Страна}, 
Максим Могильницкий, strana.ua, 28.06.2021
%%%endcit



%%%cit
%%%cit_head
%%%cit_pic
%%%cit_text
Парадоксально, но \emph{Конституция} выглядит сегодня самым радикальным и подрывным
документом, который только можно придумать в нынешней Украине. Сборник
формальных положений, прописанных когда-то группой кабинетных юристов, выглядит
сейчас как манифест сопротивления и призыв к массовому протесту. Власть явно
видит это противоречие и пытается выправить ситуацию, постоянно штампуя новые
реакционные законы, полностью выдержанные в собственных интересах.  Хотя все
эти акты, включая декоммунизационный закон Вятровича, полностью незаконны с
точки зрения действующих \emph{конституционных норм}. Ведь \emph{конституционные права} и
свободы не могут быть упразднены, а при принятии новых законов или внесении
изменений в действующие законы не допускается сужение содержания и объема
существующих прав и свобод
%%%cit_comment
%%%cit_title
\citTitle{Конституция выглядит сегодня самым радикальным и подрывным документом / Лента соцсетей / Страна}, 
Андрей Манчук, strana.ua, 28.06.2021
%%%endcit

%%%cit
%%%cit_head
%%%cit_pic
%%%cit_text
Сегодня празднуем \emph{День Конституции Украины}. Официальные, приближенные и
пригретые люди будут топить народ в реках патоки и розовых соплей. А что
празднуем? От той самой – принятой в 1996-м, остались рожки да ножки. Потому,
что кроили ее несчастную все, кому не лень.  В 2004-м на фоне первого майдана
серьезнейшие изменения вносили без согласования с \emph{Конституционным} Судом
Украины. Вдарили по \emph{Основному Закону} пакетным голосованием. Не удивительно, что
позднее в связи с нарушением процедуры закон тот был признан \emph{неконституционным}
%%%cit_comment
%%%cit_title
\citTitle{Над Основным Законом страны сначала надругались, а потом растоптали / Лента соцсетей / Страна}, 
Максим Могильницкий, strana.ua, 28.06.2021
%%%endcit

%%%cit
%%%cit_head
%%%cit_pic
%%%cit_text
Сьогодні ми святкуємо чверть століття з дня ухвалення \emph{Конституції України}.
Втім, дата ця надзвичайно умовна. З одного боку, хоча чинним вважається саме
цей документ, реально діє його редакція 2004 року, скасована 2010-го та
відновлена 2014-го. А з іншого – \emph{конституційний} процес в Україні нараховує,
залежно від розмаху думки дослідника, сто, триста чи тисячу років. Або навіть
2300 років, тому що Крим – це Україна! І хоча поняття \enquote{\emph{конституції} держави}
тоді й зараз суттєво відрізняються, все ж у кожній епосі можна знайти закон,
який по праву називався б Основним
%%%cit_comment
%%%cit_title
\citTitle{Сергій Громенко: Українській Конституції – 2300 років}, 
, gazeta.ua, 26.06.2021
%%%endcit

%%%cit
%%%cit_head
%%%cit_pic
%%%cit_text
Подозревает ли, что ее оригинал хранится в России?  "Во время выступления в
Верховной Раде в честь 25-летия \emph{Конституции Украины} Зеленский отметил, что \enquote{к
30-летию независимости в Украину впервые за 311 лет привезут оригинал
\emph{Конституции} Пилипа Орлика}
%%%cit_comment
%%%cit_title
\citTitle{Знает ли Зеленский суровую правду о конституции Пилипа Орлика? / Лента соцсетей / Страна}, 
Дмитрий Заборин, strana.ua, 29.06.2021
%%%endcit

%%%cit
%%%cit_head
%%%cit_pic
%%%cit_text
И как результат - \emph{де-конституирование} нации, \enquote{Малая Украина} для лоялистов,
шельмование и выталкивание - для несогласных, \enquote{сепаратизация}. Это потом уже
майданы, аннексии, войны, переселенцы, мигранты...  Новый конституционный
процесс - это не про \enquote{текст}. Это новое \emph{конституирование} политической нации -
как нации со-граждан - через диалог, компромиссы, \enquote{общественные договора} - как
дальше жить вместе, как сосуществовать культурам и укладам в одном
пространстве, как организовать государство как республику (а не \enquote{вертикаль} с
гетманом во главе), какое общее будущее возможно, как организовать регионы из
нынешних бюрократических областей, как вернуть доверие и желание жить вместе
наших соотечественников на Донбассе и в Крыму, и много других \enquote{как}. Без этого
\emph{конституционного} процесса - как нового национального диалога - никакие тексты
не спасут
%%%cit_comment
%%%cit_title
\citTitle{Весь путь нашей независимости - путь потерь, бесконечных конфликтов и даже войны / Лента соцсетей / Страна}, 
Андрей Ермолаев, strana.ua, 30.06.2021
%%%endcit

%%%cit
%%%cit_head
%%%cit_pic
\ifcmt
  pic https://gdb.rferl.org/1FB3D19F-841F-4E54-9E63-7F2D24C2E4AD_w1597_n_r0_st.jpg
	caption Банер «Руки геть від закону про мову!» на мосту над Алеєю Героїв Небесної сотні, вивішений активістами спільноти «Мова об’єднує» під час акції «Руки геть від закону про мову!». Київ, 28 червня 2020 року
	width 0.4
\fi
%%%cit_text
Це лише малесенька частина спроб щось зробити з мовним законом, щоб він так не
«напружував» владний політикум та олігархічні телеканали. Але насправді
\emph{Конституція України} і стаття 10 залишається бастіоном державного статусу
української мови. Сьогодні цей бастіон є, на жаль, вододілом між владою,
джерело якої – народ, та активними, пасіонарними представниками цього народу. І
поки що влада відстає у темпі державотворення і державобудови від своїх
найактивніших виборців
%%%cit_comment
%%%cit_title
\citTitle{Бастіон державності і мови. Конституція України витримує наскоки руйнівників}, 
Тарас Марусик, www.radiosvoboda.org, 02.07.2021
%%%endcit

%%%cit
%%%cit_head
%%%cit_pic
%%%cit_text
\emph{Конституция США} действует на основе англосаксонского, или как его еще называют
сейчас англо-американского права, а \emph{конституция Франция} была написана на основе
континентального права. Это принципиально разные системы права и такими
остаются и на сегодняшний день, хотя и в США действует большое число различных
кодексов -- что характерно, скорее для континентального права, а в система
континентального права действует принцип судебного прецедента: высшие судебные
органы принимают решения, которые трактуют законы континентального права.
Характерным примером синтеза двух систем является ЕСПЧ, которые действует на
основе написанной и принятой декларации, однако при этом, все судебные решения,
иски, требования пишутся с учетом соответствующих судебных прецедентов
%%%cit_comment
%%%cit_title
\citTitle{Украинская Конституция – внебрачное дитя Сталина}, 
Леонид Штекель, analytics.hvylya.net, 02.07.2021
%%%endcit

