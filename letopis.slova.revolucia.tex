% vim: keymap=russian-jcukenwin
%%beginhead 
 
%%file slova.revolucia
%%parent slova
 
%%url 
 
%%author 
%%author_id 
%%author_url 
 
%%tags 
%%title 
 
%%endhead 
\chapter{Революция}
\label{sec:slova.revolucia}

Смена формы правления и общественного устройства путём голосования за какую-то
бумажку невозможна. Общественное устройство меняется только путём
\emph{Революции}. Она может происходить либо \enquote{сверху}, либо
\enquote{снизу}. Для \emph{Революции} \enquote{сверху} у Зеленского нет ни
доверенных кадров, ни организационного ресурса, ни на 100\% лояльного силового
аппарата. Ну, а для \emph{Революции} \enquote{снизу} у украинского общества нет
ни сплочённости, ни способности к самоорганизации, ни чёткого видения
совместного лучшего будущего. Так что, всех всё устраивает, а потому,
\enquote{деолигархизация} отменяется,
\textbf{Никакой деолигархизации в Украине быть не может},
Даниил Богатырев, strana.ua, 05.06.2021

Теперь, после предупреждения, о сути. Последние несколько лет (точнее около
10-15) в мире происходит достаточно много \emph{революционных} событий. Нет,
скажет часть читателей – политические революции происходили и раньше.
Соглашусь. Но отмечу важную деталь: раньше речь шла о смене персоналий в
политических системах либо выборе одной из известных и апробированных систем
для отдельно взятого общества. Сейчас ситуация иная – лихорадит авторитарные и
автократические режимы, лихорадит военные хунты, лихорадит либеральные
демократии.  И \emph{революционные} процессы сегодня явно либо не явно, но
ставят вопрос о трансформации существующей политической системы во что-то
новое. То есть, старые лекала, алгоритмы, работают не всегда. В некоторых
государствах процессы идут мягко, через выборы и приход к власти
\enquote{внесистемных сил}, в некоторых через волнения и войны,
\citTitle{Как \enquote{работают} революции: этапы, результаты, сроки}, Ігар Тышкевіч, censor.net, 08.06.2021

Как \enquote{работают} революции: этапы, результаты, сроки, Ігар Тышкевіч, youtube.com, 06.06.2021

%%%cit
%%%cit_head
%%%cit_pic
\ifcmt
  pic https://img.vz.ru/upimg/m11/m1104023.jpg
\fi
%%%cit_text
Этот стратегический и даже, прямо скажем, по-настоящему геополитический шаг
логично привел к победе во Второй мировой. Нет сомнений в том, что с
троцкистской идеей «\emph{мировой революции}» у России просто не было шансов победить,
а вот с идеей защиты Отечества Пушкина, Достоевского, Андрея Рублева (мало кто
помнит, однако факт: в 1918-м, среди других памятников «плана монументальной
пропаганды», Луначарский приказал увековечить также память Андрея Рублева)
Россия выстояла и победила.  Тем более это уместно и необходимо сделать
сегодня, когда Россия окончательно освободилась от большевизма, а весь
европейский (и не только европейский) мир стоит перед угрозой нового
«квир-большевизма» (ЛГБТ, БЛМ и прочих меньшинств), еще гораздо более
варварского, дикого и безумного, чем прежний троцкистско-ленинский пожар
«\emph{мировой революции}»
%%%cit_comment
%%%cit_title
\citTitle{Россия должна жить по заветам Киплинга}, 
Владимир Можегов, vz.ru, 18.06.2021
%%%endcit

%%%cit
%%%cit_head
%%%cit_pic
%%%cit_text
Вряд ли стоит надеяться на конкурентоспособность общества и государства в
мировом масштабе, когда эта тенденция закрепится и поколение, воспитанное
«манижами», «милохиными» и «бузовыми» войдет во взрослую жизнь.  Именно поэтому
\emph{Революция Духа}, является на сегодняшний день единственной возможностью
спасения и сохранения русской цивилизации и России. Насколько бы трудной,
неосуществимой, немыслимой она не казалась на первый взгляд. Ровно такой же
казалась современникам начала прошлого века и социалистическая \emph{революция}
и, особенно, в России, на что ни теоретики марксизма, ни сами русские
\emph{революционеры} всерьез не надеялись
%%%cit_comment
%%%cit_title
\citTitle{Революция Духа – единственный путь спасения России}, 
Юрий Барбашов, voskhodinfo.su, 30.06.2021
%%%endcit

%%%cit
%%%cit_head
%%%cit_pic
%%%cit_text
О \emph{РЕВОЛЮЦИЯХ}, ЛИДЕРАХ И РОДОВОМ ПРОКЛЯТИИ.
Главным критерием свершившейся \emph{социальной революции} является смена
общественно-политического строя. Второй критерий, отчасти вытекающий из первого
— это появление новых людей, претендующих на общественное лидерство.
И по этим критериям единственной свершенной \emph{революцией} у нас было
обретение независимости, приведшее к переходу от загнивающего социализма к
дикому капитализму. Именно тогда мы по-настоящему узнали о Вячеславе Чорноволе,
Левко Лукьянено, Степане Хмаре и других патриотических лидерах.
В дальнейшем дикий капитализм превратился в олигархический, но суть системы
власти не менялась.  Ни Майдан-2004, ни Майдан-2014 не только не сменили
систему власти, после них даже олигархи, то есть, персональные выгодополучатели
системы, нисколько не пострадали. И новых лидеров как носителей новых идей,
подхваченных обществом, не появилось
%%%cit_comment
%%%cit_title
\citTitle{В Украине была только одна настоящая революция - в 1991-м}, 
Александр Кочетков, strana.ua, 16.07.2021
%%%endcit

%%%cit
%%%cit_head
%%%cit_pic
%%%cit_text
Содержание «Глоссария» колеблется от безвредно-бесполезного до откровенно
глупого и омерзительного.  Например, вместо «ближнего зарубежья», следует, по
мнению этих «умников», говорить «постсоветское пространство». Сразу же
напрашивается вопрос: а какая разница?  Вместо «киевской власти» — «Украинское
государство».  Вместо «госпереворот» — «\emph{революция} гидности».  Вместо
«ДНР» и «ЛНР» — «ОРДО/ОРЛО»
%%%cit_comment
%%%cit_title
\citTitle{Краткий словарь грантоедов под редакцией СНБО / Лента соцсетей / Страна}, 
Александр Карпец, strana.news, 26.10.2021
%%%endcit

%%%cit
%%%cit_head
%%%cit_pic
%%%cit_text
Жидок с револьвером? Пустяки!  А меж тем нет на свете зверя опаснее, ибо именно
он, жидок с револьвером, делает \emph{революцию}.  Впрочем...  Впрочем, когда он
делает \emph{революцию} — это одно. Все силы ада с ним.  Когда же он ловит
контрабанду, чтобы ее украсть, —   это совсем другое.  Это из тех низших
чертей, которым кузнецы Вакулы крутят хвосты...
%%%cit_comment
%%%cit_title
\citTitle{Три Столицы}, В. В. Шульгин
%%%endcit

%%%cit
%%%cit_head
%%%cit_pic
%%%cit_text
Возможно, эмоциональное заявление главы государства о «госперевороте», хотя бы
отчасти все-таки имело определенный профилактический эффект. Кто-то решил не
рисковать, и не влезать в авантюры, кто-то рационально посчитал, что сейчас
(особенно на фоне российских угроз) не время для новых «майданов». Наверняка у
многих оппонентов и критиков Зеленского не было симпатии и доверия к
организаторам этой акции. Зачем играть в чужую игру? Но, самое главное, - не
было массовой поддержки этой акции. В телевизоре и в среде политических элит
кипят страсти, «кризис» в головах и в воспаленном сознании, а «народ
безмолвствует», как сказал когда-то классик. Народ волнуют другие, более
насущные для него проблемы, а не разборки «элит». Да нет и особого доверия к
этим «элитам».  И в очередной раз подтвердилась давняя истина –
\emph{революции} и «госперевороты» по анонсу не происходят. Тем более, в такой
стране как наша. Все самое серьезное в Украине происходит неожиданно
%%%cit_comment
%%%cit_title
\citTitle{Революции и госперевороты по анонсу никогда не начинаются / Лента соцсетей / Страна}, 
Владимир Фесенко, strana.news, 04.12.2021
%%%cit_url
\href{https://strana.news/opinions/365396-revoljutsii-i-hosperevoroty-po-anonsu-nikohda-ne-nachinajutsja.html}{link}
%%%endcit
