% vim: keymap=russian-jcukenwin
%%beginhead 
 
%%file 03_05_2014.lj.kungurov.1.donbass_chechnja_raznica
%%parent 03_05_2014
 
%%url https://kungurov.livejournal.com/84905.html
 
%%author_id lj.kungurov
%%date 03_05_2014
 
%%tags donbass,chechnja,rossia,ato,vojna,raznica
%%title Донбас и Чечня: в чем разница (часть 1)
 
%%endhead 

\subsection{Донбас и Чечня: в чем разница (часть 1)}
\label{sec:03_05_2014.lj.kungurov.1.donbass_chechnja_raznica}

\Purl{https://kungurov.livejournal.com/84905.html}
\ifcmt
 author_begin
   author_id lj.kungurov
 author_end
\fi

Имеет ли право Украина применять войска для сохранения своей территориальной
целостности? Хотя, о чем это я... никаких войск у Украины нет. Нет того, что
называется армией. Армия может быть только у государства, а государство UCRAINA
сегодня в стадии демонтажа. Вопрос надо ставить иначе: имеет ли власть право
применять вооруженную силу (всяких там правосеков и наемников, больше-то и
некого) для удержания под своим контролем территории страны? Если смотреть
поверхностно, то сейчас на Украине происходит примерно то же самое что в Чечне
в 1994 и 1999 гг. – давится очаг сепаратизма. И если России было можно, то
почему нельзя Украине?

\ifcmt
  ig http://mtdata.ru/u4/photoA035/20496235446-0/original.jpg#20496235446
  @width 0.4
  %@wrap \parpic[r]
  @wrap \InsertBoxR{0}
\fi

И Украине можно. Это ее суверенное право. Вот только под словом «можно» в
данном случае следует подразумевать слово «возможность». Как говаривал по этому
поводу Отто фон Бисмарк, «Меня не интересуют намерения моих врагов, меня
интересуют их возможности». У РФ была возможность прессануть Чечню, и она ее
успешно реализовала, пусть и со второй попытки. У Украины нет возможности
подавить силой мятежный Донбасс в принципе. И это абсолютно очевидно любому
стороннему наблюдателю. Почему же Киев упорно  пытается сделать то, что
гарантировано приведет к краху?

Основная версия, которая хорошо усваивается общественным мнением (спасибо
путинскому зомбоящику) – то, что киевская верхушка состоит из дебилов и
оголтелых нацистов-маньяков. Должен вас огорчить – они не идиоты, и действуют
единственно верно и абсолютно правильно. Правда, при этом следует учитывать, в
ЧЬИХ ИНТЕРЕСАХ они действуют, и тогда все встанет на свои места.

Нет, они все-таки идиоты, – пытаются возражать мне почти все, – ведь те лица,
что захватили власть в Киеве,  должны понимать, что Запад толкает Украину на
путь Югославии, издохшей в кровавых соплях в 90-е. Я в таком случае задаю им
встречный вопрос: а разве югославское руководство в 91-м году не понимало, что
Запад толкает страну в гражданскую войну, в которой победить невозможно в
принципе? Очевидно же, что цивилизованный «развод» по образцу Чехии со
Словакией устроил бы всех. На худой конец возможна была и ликвидация страны в
результате номенклатурного сговора, как в случае с СССР. И никто бы не стал
тогда вырезать или насильно депортировать сербов, хорватов, боснийцев и прочих,
оказавшихся вне «исторической родины». Русских ведь в Казахстане никто не
зачищал после распада Советского Союза? Их даже в Грузии не убивали и в Литве
всем предоставили гражданство автоматически на равных с литовцами. На Украине
эти самые русские не только не подвергались гонениям, но совершенно
демократически высказались в ходе референдума за суверенитет  Украины.

Вот только у югославского руководства не было ВОЗМОЖНОСТИ контролировать
ситуацию. Их грубо толкали на применение силы организацией этнических зачисток,
а применение силы объявляли геноцидом нацменьшинств. Если,  допустим, некие
боевики вырезали сербскую деревню в Боснии и идут вырезать следующую, то что
должны были делать югославские военные? Верно, они должны выполнить свой долг и
защитить мирных граждан. Хорошо, сербскую деревню спасли, убийц уничтожили в
ходе боя или взяли в плен. Тут же в западных СМИ начинается жуткий вой, что
армия давит гусеницами мирных боснийцев, борцов за свободу от кровавой
белградской тирании. Боснийские нацисты, получив от Запада оружие, начинают
мстить за павших товарищей и идут вырезать следующую сербскую деревню. Однако
там их уже ждут местные ополченцы, которые не желают смотреть, как будут
насиловать их женщин и продавать на органы детей. Вполне возможно, они даже
первые пойдут и вырежут «вражеское» боснийское село.

Ну, и понеслось...  Потом прилетают натовские «миротворцы» творить мир путем
ковровых бомбардировок (на кладбище все обитатели ведут себя предельно мирно),
коммуникации берут под контроль голубые каски ООН, а судьбу страны решают без
ее участия солидные дяди на какой-нибудь женевской конференции. Вот и скажите
мне – тупым было югославское руководство в 1991 г. или умным? Да никакой
разницы нет, просто Белград не имел ВОЗМОЖНОСТИ влиять на ход событий, а
заступиться за Югославию после уничтожения Советского Союза было уже некому.

С Украиной дело обстоит еще проще. В Киеве сейчас сидят не идиоты, а марионетки
Запада, которые, собственно, этого не скрывают. Запад (можно даже сказать более
определенно  – США) по каким-то причинам, которые мы сейчас рассматривать не
станем (это отдельная тема), желает устроить на территории Украины "югославию".
Европейские «партнеры» США этого не очень хотят, но они сейчас находятся в
ситуации, когда возможности влиять на ход событий фактически не имеют. Почему?
Будете смеяться, но против фактов не попрешь: политическое руководство
практически всех европейских стран насыщено американскими агентами влияния,
или, говоря более деликатно – лоббистами. Это, я подчеркиваю, никто не пытается
скрывать или считать аморальным. Для Европы это настолько в порядке вещей, что
иное просто представить невозможно.

Что касается Восточной Европы, то там даже есть официально признанная зона
влияния США – Польша и прибалтийские бантустаны, где в свое время даже
президентами были лица, имеющие американское гражданство. Захотели пиндосы
иметь тайные тюрьмы ЦРУ на территории Польши и Литвы – пожалуйста! Истинная
демократия никогда не обходится без тайных тюрем. Захотели разместить свои
вооруженные силы в Эстонии – да ради бога, эстонцы сами об этом будут умолять
на коленях своих хозяев. А если кто-то где-то чем-то будет недоволен, то в этой
стране вдруг случится парламентский кризис, правительство подаст в отставку и к
власти придут лица, которых лично подберет Госдеп. Демократия – это не власть
народа, как наивно полагают придурки, это – власть демократов. Демократы – это
те, кому дают деньги на выборы. А кто дает деньги? Ну, вы, короче, поняли…
Бизнес, и ничего личного.

Итак, в Киеве у власти сидят умные и энергичные люди. Ни разу не нацисты и не
бандеровцы, а благовоспитанные доктора наук и профессора. Но они – ставленники
США, что надеюсь, доказывать никому не надо. Если кому-то очень зудит, сами
погуглите, у кого из нынешних укро-вождей члены семьи в Америке живут и кто
агент ЦРУ со стажем. В этой компании Кличко, который является крупнейшим
германским налогоплательщиком-спортсменом, почти что белая ворона.
Евроинтегрировался, ептыть! Но и он, впрочем, владеет рядом компаний в США, так
что рычаги воздействия на него имеются. И все эти «патриоты» старательно делают
то, что от них ждет э-э-э… работодатель.  А работодатель ждет от них
«югославию». Вот и вся разгадка загадочного и кажущегося идиотским поведения
всех этих ярошей-турчиновых-яценюков и прочих тимошенок.

Но не могут же украинцы быть такими идиотами, что терпят у себя на шее
откровенных предателей? – не унимаются мои оппоненты. Нет, ребята, вот это как
раз очень может быть, и даже больше скажу – иного быть не может. Януковича эти
идиоты свергли за… сейчас уже никто не помнит за что, вроде как за то, что
беркутовцы студентиков на майдане дубинками отписдили. А Турчинов с Яценюком
просрали Крым, обрушили курс гривны, угробили экономику, развязали гражданскую
войну, подняли тарифы на ЖКХ, срезали соцльготы – и чо? И ничо! Никто их не
свергает. Все предельно просто: идиоты не способны никого свергнуть. Идиоты
вообще не способны совершать какие-либо осмысленные политические действия.
Идиотов развели, как последних лохов – они вышли на майдан. Обожглись. Ерез 10
лет их снова развели, как лохов – еще раз помайданили. Лучше стало? Нет, но
никаких «стихийных народных протестов» против «правительства победителей» не
наблюдается. Ибо не проплачено. Ибо у власти как раз те правильные пацаны,
которые нужны заказчикам майдана. Стихиных революций не бывает. Надеюсь, это
объяснять не надо?

Справедливости ради отмечу, что не только украинское быдло такое тупое, в
других демократических странах быдло ничуть не разумнее. Демократия держится не
на прямом насилии и принуждении, а на манипуляции. Следовательно, для
существования демократии необходимо тупое быдло. Умными-то людьми как
манипулировать? Как умный человек способен добровольно проголосовать за
Ельцина, особенно второй раз, или променять Ющенко на Януковича?

Теперь перейдем к Донбассу. Сейчас там происходит бунт. Стихийный? Не-е-е…
Умело спровоцированный. Донецк не пытался спасать «своего» Януковича. Донецк,
как Севастополь и не вопил «Путин, возьми нас!». Донецк вполне удовлетворился
бы федерализацией – то есть возможностью самим выбирать себе губернатора, самим
распоряжаться своими налогами и самим решать, на каком языке писать вывески на
домах. Вообще-то, если бы власть в Киеве взяли демократы в истинном смысле
слова, то они должны были поддержать порыв донецких демократов. Ведь что такое
истинная демократия – это когда люди сами решают как им жить. Вот  хотят
дончане сами выбирать себе губернатора, пусть и бывшего РНЕшника – разве это не
кошерно-демократично?

Киевская хунта сейчас не в том состоянии, чтобы права качать, особенно после
того, как Крым радостно выломился из страны победившей «народной революции» в
«путинский тоталитарный концлагерь». Турчинов должен был ухватиться за слово
«федерализация», как за  последнюю соломинку, и радостно провозгласить: «Берите
себе федерализма столько, сколько сможете унести!». Для школоты поясняю: в 1991
г. этот клич, обращенный к регионам РФ, бросил Ельцин, только вместо слова
«федерализм» там присутствовало слово «суверенитет». Призыв был встречен с
энтузиазмом, началась волна «национализации» госсобственности, некоторые особо
«суверенные» регионы, например, вообще перестали платить налоги в федеральный
бюджет.  В Якутии государственным языком наравне с якутским и русским был
провозглашен английский. Но лафа длилась недолго, суверенитет у регионов стали
постепенно отжимать, и при Путине отобрали полностью, Россия федерацией
является только номинально, реально же она более унитарная страна, нежели
Украина.

По той же схеме должна была действовать и хунта – обещать всем все, тем самым
сбить протестную волну, а потом сделать вид, что ничего не обещали. Или
продолжать обещать, но на деле обещанного не давать даже после выборов, которые
бы легитимизировали режим, пришедший к власти в ходе военного переворота. Но
киевский режим не только не стал спасать себя единственно возможным способом,
но начал активно тушить пожар бензином – Верховная Рада отменила закон о языках
и День победы, власти арестовали народного губернатора Губарева и всячески
поощряли праворадикальных экстремистов, которые стали гастролировать по востоку
Украины и устраивать погромы и убийства. Перечислять нет смысла, все уже в
курсе. 39  антимайдановцев, убитых вчера в Одессе, явно качнут регион в сторну
сепаратизма

И вот, наконец, апогей – население, не желающее признавать новую укро-власть,
объявлено террористами и против них брошена боевая авиация и танки. Как
показали события 2 мая, всей своей боевой мощи Киеву не хватает, чтобы взять
под контроль 100-тысячный мятежный городок Славянск, но это было ясно и без
всяких событий. Цель так называемой антитеррористической операции – вовсе не
победить «террористов», а развязать «горячую» гражданскую войну, устранить
малейшие возможности для политического урегулирования конфликта. Ну, что же,
поздравляю укров – теперь такой возможности нет.

Еще раз подчеркиваю – это целенаправленная политика Вашингтона проводимая
руками своих киевских шестерок. Почему они не выполнили и даже не пытались
сделать вид, что выполняют женевские соглашения от 17 апреля? Ладно, вопрос
риторический. Тут показательно другое: почему США не пытаются заставить Киев
выполнять взятые на себя обязательства? Причина только одна: Штатам не желает
мирного урегулирования, им нужна война. Еще более красноречива позиция МВФ,
подконтрольной США организации: фонд заявил, что кредит будет получен Украиной
только в том случае, если Киев вернет контроль над мятежными территориями
Юго-Востока. Это, конечно, логично, потому что если промышленный Юго-Восток
отложится, то кто будет кредит отдавать - не винницкие же хуторяне натурой?
Если бы хозяева МВФ не хотели войны, то могли бы поставить иные условия: кредит
будет предоставлен только в случае скорейшего политического урегулирования
конфликта.

Хунте же война нужна, как воздух, без нее - смерть. Точнее, им нужна
интервенция РФ. Пиндосы убедили их, что если в Донбассе произойдет кровавая
мясорубка, то Путин начнет вторжение. Дело в том, что экономика Украины
стремительно летит в жопу. Пока широкие слои обывателей еще не стали
нищенствовать и голодать, пока еще в домах есть тепло, свет, вода и газ, пока
еще на улицах нет разгула бандитизма, но все это ожидает Украину в самой
ближайшей перспективе. А кто будет виноват? Власть! Так что новый майдан,
страшный неуправляемый тотальный голодный и замерзший майдан неизбежен. И лишь
только  если начнется война с Россией, то во всем виноват будет Путин. Только
внешний враг способен хоть как-то консолидировать рассыпающееся украинское
общество и  дать шанс на продление существования проекта UCRAINA. Не будет  на
Украине никаких «сторонников федерализации» или «оппозиции» хунте, они в
одночасье превратятся в изменников и коллаборационистов. А быть
коллаборационистом как-то не очень комфортно, знаете ли. Чеченцы осенью 1994 г.
вовсю воевали меж собою за власть, клан Гантамирова сцепился с кланом Дудаева,
но вторжение российских войск тут же привело к забвению междоусобных обид. Да,
и раз уж война неизбежна, то за газ можно не платить.

К счастью Киев вследствие развала государства недееспособен настолько, что даже
кровавую провокацию устроить не в состоянии, хотя тут ума много не требуется. А
время поджимает. 25 мая должны пройти выборы, на Украине появится легитимная
власть, которой мировое сообщество должно оказать помощь в отражении российской
агрессии. Следовательно, войну надо развязать в ближайшие три недели. Скорее
всего, крупные провокации пройдут 9 мая, когда Донбасс будет отмечать День
Победы, упраздненный на остальной Украине (чтоб не раздражать неоацистов,
штоле?). С одной стороны это будет особенно оскорбительно для РФ, с другой
позволит сорвать донецкий референдум, назначенный на 11 мая.

 Так что ключевой вопрос в текущей повестке дня следующий: вмешается РФ в
украинскую гражданскую войну или нет. Если в Кремле сидят не совсем тупые
идиоты, то Россия не будет воевать в открытую, даже если бандеровцы устроют в
Донбассе кровопускание в духе Волынской резни. Но уверенности в том, что в
Кремле сидят не идиоты, у меня нет. Желание «защитить братьев славян» в свое
время поставило жирный крест на Российской империи, которая в августе 1914 г.
первой атаковала  Германию. Хотя царь Николаша был не совсем уж конченным
тупицей, и в генштабе сидели не совсем дураки.

Да, тут возникает вопрос морального характера: может ли Россия допустить резню
«соотечественников» на Украине? Лично я считаю, что каждый народ должен нести
ответственность за свой исторический выбор. Народ Украины 1 декабря 1991 г. на
демократическом референдуме сделал свой исторический выбор – послал нах СССР и
поддержал акт о независимости Украины. Именно эти люди уничтожили Советский
Союз, поэтому будет справедливо, если они  ответят за свой выбор по полной
программе. Напомню, что в РСФСР не было референдума о независимости, его не
было ни в Белоруссии, ни в Казахстане. Более того, Белоруссия и Казахстан даже
не делали попыток объявить суверенитет. СССР прекрасно мог обойтись без
Прибалтики, которая де факто вышла из состава страны еще зимой-весной 1991 г.,
без Молдавии и  Закавказья, но он не мог существовать без Украины.

1 декабря жители Украины собственноручно убили советскую сверхдержаву. 8
декабря состоялась знаменитая Беловежская встреча, участники которой были
поставлены перед фактом: Советского Союза больше нет. Казахстан объявил о
суверенитете только 16 декабря, Верховный совет РСФСР ратифицировал беловежские
соглашения 22 декабря. Белоруссия вообще до 1994 г. жила по советскому
законодательству и конституции (впрочем, там и сегодня совком припахивает
изрядно, не в обиду будь им сказано). Именно Украине принадлежит сомнительная
честь официального могильщика СССР.

Во время недавних крымских событий путинская пропаганда, натянув сову на
глобус, пела песню, что, дескать, Крым порвал отношения с Украиной референдумом
20 января 1991 г, провозгласив себя независимым государством в составе СССР. Из
этого делались выводы, что Украина чуть ли не оккупировала Крым 23 года, грубо
поправ всенародно принятую крымскую конституцию. Но пропагандоны почему-то
забывают, что какбэ независимый Крым принял участие в референдуме о
независимости Украины 1 декабря 1991 г. и дружно поддержал эту самую
независимость 54\% голосов. А «город русской морской славы» переплюнул и этот
результат – там сказали Москве «гудбай» 57\% из принявших участие в референдуме.
Это сейчас они скулят «Спасибо Путину что мы снова дома», но они из «русского
дома» сами же и съипали.

А что Донбас? О, там ваще красотища – в декабре 91-го за независимость
высказались 84\% в Донецкой и Луганской областях, 86\% в Харьковской. Запорожская
и Одесская области вообще переплюнули весь Юго-Восток, там за осуществление
мечты Бандеры о независимой Украине высказались более 90\% участников
референдума. И вот, по прошествии 23 лет население вдруг решило, что украинская
государственность их не устраивает. Ну, и ладненько, демократия есть
демократия, боритесь, отстаивайте  свое право жить в свободной и богатой
стране, говорить, и учить детей в школе на родном русском языке.

Однако эти хитрованы хотят немного другого – чтобы Россия, которую они
воспринимают, как реинкарнацию СССР,  пришла и взяла их обратно на содержание.
Все бла-бла-бла про фашизм не пройдет, про русский мир и славянское единство –
не более чем идеологическое прикрытие сугубо желудочного желания сменить флаг и
получить за это более жирную пайку. Нет, я тоже люблю хорошо покушать, и никого
не осуждаю за такое желание. Но объясните, почему я должен вас кормить? Вы уже
однажды предали СССР и проголосовали желудком за капитализьму, швабоду и
демократию. Теперь, вместо того, чтобы бороться за спасение Украины, которую вы
сами создали, предаете ее и сбегаете с тонущего корабля. Где гарантия, что
завтра, когда кризис неизбежно случится в России, вы не предадите ее за банку
варенья и пачку печенья? Боюсь, вопрос  риторический.

Вам мои слова кажутся циничными? Да, если смотреть правде в глаза, то неизбежно
становишься циником. Сначала эти люди в марте 1991 г. сказали «ДА» сохранению и
обновлению СССР, через полгода они же едиными усилиями с львовскими
бандеравцеми уничтожили свою страну. Вот это – цинизм. Теперь они верещат, что
фошизды-западенцы им жизнь портят, а олигархи до нищеты довели. А кто виноват,
ась? Поэтому, если бандеровцы умоют Юго-Восток кровью, а олигархи выпотрошат
окончательно и сдадут американским корпорациям шахтеров в качестве рабов, это
будет жестоко, зато исторически справедливо. Вы ведь не думаете, что
справедливость должна быть исключительно добренькой и приятной? И грядущий крах
«новой России», кстати, тоже будет исторической справедливостью, и к этому я
отношусь совершенно спокойно, как к неизбежности, которую можно оттягивать, но
нельзя предотвратить.

К счастью, бандеровская угроза – не более, чем муха, раздутая в слона путинской
пропагандой.  Только поэтому полномасштабная гражданская война на Украине не
начинается, не смотря на деятельную поддержку Запада. У нас много пишут про
англоговорящих наемников в составе карательной группировки свидомитов под
Славянском. Скорее всего это такой же фейк, как вопли укропропаганды о том, что
под видом ополчения действует российский спецназ. Но, скорее всего, убедившись
в недееспособности хунты, пиндосам придется-таки выписать десяток снайперов,
которые повторят на бис киевский расстрел 20 февраля, только уже в Славянске.
Перережут горло и изуродуют трупы нескольких украинских военных, в самом городе
застрелят пяток детей, идущих в школу. Противоборствующие стороны начнут мстить
за убитых, в результате убитых будет еще больше, и за них снова будут мстить.
Вот вам и война.

Я никогда не скрывал своих антипатий к Путину и путинской банде мародеров.
Очень хочу, чтобы как можно скорее появилась возможность их свергнуть. Но
должен констатировать, что в результате украинского кризиса положение правящего
режима невероятно усилились, массы до такой степени охвачены
верноподданническим экстазом, что поддерживают власть  даже в условиях
экономического спада и падения уровня жизни, поддерживают не смотря на
тотальную коррупцию и сворачивание социальных гарантий.  Расстраивает ли меня
это? Да нисколько! Уж больно вся эта великоимперская истерика напоминает
аналогичное явление августа 1914 г. Не прошло и трех лет, как Российская
империя рассыпалась в прах. Если история повторяется, а она почему-то имеет
свойство повторяться, то  финита ля комедиа случится уже скоро.
(\href{https://kungurov.livejournal.com/85155.html}{продолжение}).
