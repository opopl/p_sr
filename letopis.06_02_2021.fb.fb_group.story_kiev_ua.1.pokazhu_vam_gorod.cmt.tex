% vim: keymap=russian-jcukenwin
%%beginhead 
 
%%file 06_02_2021.fb.fb_group.story_kiev_ua.1.pokazhu_vam_gorod.cmt
%%parent 06_02_2021.fb.fb_group.story_kiev_ua.1.pokazhu_vam_gorod
 
%%url 
 
%%author_id 
%%date 
 
%%tags 
%%title 
 
%%endhead 
\zzSecCmt

\begin{itemize} % {
\iusr{Людмила Ареф'єва}

Спасибо огромнейшее за то, что поделились своей работой 40-ней
давности... как-будто на 10 минут вернулась в юность... тронуло до слез...

\iusr{Федор Лебедев}
\textbf{Людмила Ареф'єва} - спасибо и Вам! Позже, загружу другой свой фильм о Киеве.

\iusr{Людмила Ареф'єва}
\textbf{Федор Лебедев} буду ждать!

\iusr{Наталия Ковалева}
Спасибо, вы вернули меня в старый Киев!

\iusr{Елена Сидоренко}

Спасибо Вам, Фёдор! Вы вернули меня в молодость, в тот Киев, который отличается от
нынешнего. Очень хороший фильм, не верится, что это работа молодого
режиссера - такие красивые панорамы города, люди, среди которых ищещь знакомые
лица... а закат какой! @igg{fbicon.heart.beating}. Благодарю! @igg{fbicon.heart.red}

\iusr{Светлана Манилова}
\textbf{Елена}, не Феликс, а Федор! @igg{fbicon.smile} 

\iusr{Елена Сидоренко}
\textbf{Светлана Манилова} исправила. Ошиблась неспроста! @igg{fbicon.beaming.face.smiling.eyes} 

\iusr{Федор Лебедев}
\textbf{Елена Сидоренко} - и Вам спасибо большое за добрые слова! Через несколько дней загружу другой фильм о Киеве, который делал с братом, - он намного лучше.

\iusr{Елена Сидоренко}
\textbf{Федор Лебедев} Фёдор! Спасибо ещё раз! И консультант Халепо, и текст он читал @igg{fbicon.heart.beating} 

\iusr{Vadim Basovskiy}
Спасибо, хороший фильм, тем более, что автору было всего 19 лет. Конечно нужно сделать поправку на время.

\iusr{Lesik Machynsky}
\textbf{Vadim Basovskiy} на времена и нравы...

\iusr{Константинова Наталия}

Именно таким я покинула Киев 40 с лишним лет назад, каждый год стараюсь
приехать, видела и вижу изменения города, не всегда в лучшую сторону, но всё
равно, пожив и побывав в Киеве, его уже нельзя забыть никогда. Спасибо за
фильм

\iusr{Alla Pavlova}
Спасибо, Фёдор, Вы показали ещё мой Киев! Сейчас это уже чужой город!

\iusr{Андрей Шиян}
Спасибо!

\iusr{Вера Хохлова}
Спасибо, Вам !)

\iusr{Людмила Волжанка}
Добрый, приветливый, теплый Киев, очень жаль, что многое изменилось....... К сожалению не в лучшую сторону

\iusr{Larisa Tverdokhleb}
Спасибо большое!!! Вы любите Киев и это чувствуется в Вашем фильме! Киев нельзя не любить - это наш Город!

\iusr{Ирина Нищимная}

Спасибо Вам, фильм прекрасен, как и наш любимый город,,как приятно вернуться в
детство, юность и пройтись родными улицами,, как же не хватает сейчас таких
фильмов на тв, они ведь показывают как нужно любить и беречь наш прекрасный
Вечный Киев!

\iusr{Федор Лебедев}
\textbf{Ирина Нищимная} - Благодарю Вас за добрые слова и желаю крепкого здоровья и благополучия!

\iusr{Светлана Блаус}

Как будто машина времени вернула меня на 40 лет назад! Какое счастье увидеть
город юности, любви и надежды! Нет слов...спасибо!

\iusr{Федор Лебедев}
\textbf{Светлана Блаус} - и Вам спасибо! Всего Вам Доброго!

\iusr{Ольга Волынец}

Вопрос к автору. Есть кадр, Святой Владимир на фоне Днепра, но нет пешеходного
моста, а он открыт в 1957 году, может другой ракурс. А в целом, молодцы, окунули в
прошлое красавца Киева! @igg{fbicon.hands.applause.yellow} 

\iusr{Федор Лебедев}
\textbf{Ольга Волынец} - 

спасибо за комментарий! Я вчера после публикации видео вспомнил,что это съёмки
не 1980, а лето 1982 года. В пешеходный мост в том фильме мелькает. В кадре с
Владимиром он не виден из-за точки съёмки. Всего Вам хорошего!


\iusr{Ольга Волынец}
\textbf{Федор Лебедев} ,спасибо!

\iusr{Петр Кузьменко}

Огромное спасибо за возвращение в юность! Целый водоворот чувств и эмоций
вызвал Ваш фильм. Благодарю!


\iusr{Федор Лебедев}
\textbf{Петр Кузьменко} - и Вам спасибо! Всего Вам доброго желаю!

\iusr{Тамара Нарижная}

Спасибо за чудесный фильм Как приятно увидеть все памятники которых сейчас уже
нет и испытываешь гордость от того что живешь всю жизнь в таком красивом городе

\begin{itemize} % {
\iusr{Lesik Machynsky}
\textbf{Тамара Нарижная} по ильичам плачь???

\iusr{Федор Лебедев}
\textbf{Тамара Нарижная} - и Вам спасибо! Всего Доброго!

\end{itemize} % }

\iusr{Kristinka Medvedeva}

Автор, огромное Вам спасибо за прекрасный фильм, красивые, трогательные кадры!!!
Аплодирую Вам и Вашим помощникам за такое памятное для всех киевлян
кинопроизведение!!!

\begin{itemize} % {
\iusr{Федор Лебедев}
\textbf{Kristinka Medvedeva} - 

и Вам спасибо за высокую оценку этой скромной работы! На неделе выставлю
намного лучший фильм - \enquote{Город Архангела Михаила}, который лет 15 назад о Киеве
сделал мой брат Михаил. Вот тем фильмом действительно, можно гордиться.

\iusr{Kristinka Medvedeva}
\textbf{Федор Лебедев} с нетерпением буду ждать фильм!
\end{itemize} % }

\iusr{Оксана Дубинина}
Фёдор, спасибо! браво!)))) Это профессиональный фильм, тем более с поправкой на
технические возможности. А с чего снимали? С вертолёта?

\iusr{Федор Лебедев}
\textbf{Оксана Дубинина} - спасибо за отзыв. Да, на час арендовали у ГАИ вертолёт. Я об этом писал в предисловии.

\iusr{Татьяна Сирота}

Спасибо, Федор!
Нахлынули воспоминания.
Как же мы были тогда молоды!
А фильм о Киеве сегодняшнем очень нужен. Снимайте его!
Наш город был, есть и будет прекрасен всегда.

\iusr{Федор Лебедев}
\textbf{Татьяна Сирота} - взаимно! Благодарю Вас и всех добрых зрителей!

\iusr{Шурик Барков}
Классно, прям окунулся в атмосферу Киева своего детства, хоть оно было то на 10 лет позже.
Спасибо!

\iusr{Федор Лебедев}
\textbf{Шурик Барков} - благодарю Вас за комментарий! Успехов желаю Вам!

\iusr{Руслана Колесник}
Спасибо огромное!!)

\iusr{Cvetlana Dubodelova}

Спасибо за воспоминания, как жаль, что многие изменения произошли не в лучшую
сторону, центр Киева, Крещатик, потерял свою индивидуальность, своё лицо, а
какие фонтаны были..., Ещё раз спасибо за прекрасный фильм о любимом городе

\iusr{Федор Лебедев}
\textbf{Cvetlana Dubodelova} - благодарю за просмотр и добрые слова! Успехов Вам!

\iusr{Нина Воронцова}

Я, хорошо помню Киев, того времени, ведь тогда мне было 18 лет, и я любила
гулять по городу. А, сейчас он изменился, и сейчас другая эпоха.

\iusr{Ольга Шамрай}
Рада была увидеть и услышать Анатолия Трофимовича Халепо. Спасибо

\begin{itemize} % {
\iusr{Федор Лебедев}
\textbf{Ольга Шамрай} - и вам спасибо. Меня тут поправили, - это съёмки не 80-го года, а 82-го...

\iusr{Ольга Шамрай}

Уважаемый \textbf{Федор Лебедев}, для меня точность даты не столь важна. Важно, что Вы
это подарили: культура памяти, съемка, уважение к материалу. Для меня это
самое важное. И память о людях, которые меня учили @igg{fbicon.heart.red}

\iusr{דמיטרי קוגן}
\textbf{Ольга Шамрай} 

и Татьяну Золоторёву тоже. Мой отец тогда работал в экскурсоводом и состоял в
исторической секции которую возглавлял Анатолий Трофимович. На все заседания
секции папа брал меня тогда девятиклассника с собой. Никогда не забуду
экскурсию по Подолу которую по приглашению Халепо проводил Брайчевский

\end{itemize} % }

\iusr{Татьяна Анисимова}

Фотографий много о Киеве у меня. Могу поделиться. Снимаю с 13 лет. Сейчас мне
59. Коренная киевлянка. Ваша тема мне интересна.

\begin{itemize} % {
\iusr{Федор Лебедев}
\textbf{Татьяна Анисимова} 

- благодарю Вас! Давайте спишемся - созвонимся. Я вообще, хочу сделать
объявление в этой группе, - чтобы люди писали или в группу или мне лично свои
небольшие воспоминания о той, прежней жизни. Я бы сделал серию роликов с их
рассказами и фотографиями. Это важно для молодёжи. Мой тел. 098-131-33-00,
почта: \url{flebedev@ukr.net}

\iusr{Татьяна Анисимова}
\textbf{Федор Лебедев} ок

\iusr{Татьяна Анисимова}
0937743260. \url{anistaten@gmail.com}
\end{itemize} % }

\iusr{Volodymyr Nekrasov}

Прекрасний фільм! Велике Вам дякую! Захоплюючі ракурси. Деякі вже стали
хрестоматійними, а деякі нові, авторські. І це у 19 років!!! Геніально!

\begin{itemize} % {
\iusr{Федор Лебедев}

Дякую Вам! Але, вже після публікації я пригадав, що ці зйомки відбувались не у
1980, а у 1982 році. Я вже був великим хлопцем, - 21 рік, на останньому курсі
навчався заочно на кінознавця.

\iusr{Volodymyr Nekrasov}
\textbf{Федор Лебедев} 

я у свої що 19 що 21 був іще дитиною  @igg{fbicon.smile}  Не можу також не
поцікавитись звідки музичний супровід? Спеціально для цього фільму чи взятий
звідкись?

\iusr{Федор Лебедев}
Музику я брав у фонотеці \enquote{Київнаукфільму}, де робив монтаж. Музика з
радянських кінофільмів, але я знаю, з яких...
\end{itemize} % }

\iusr{Лариса Кучерова}

Спасибо, Федор, что можно увидеть тот Киев, мой Киев. И люди, люди.. горожане.
Как же хорошо окунуться в воспоминания..

\iusr{Федор Лебедев}
\textbf{Лариса Кучерова} - спасибо и Вам! На след. неделе опубликую другой фильм о Киеве, - он интереснее и снят лучше.

\iusr{Людмила Крошка}
Спасибо!!! Окунулась в детство!!

\iusr{Элеонора Лозовская}
Очень, очень трогательно.. хочу продолжения... не задерживайте Федор..

\iusr{Федор Лебедев}
\textbf{Элеонора Лозовская} - благодарю Вас! На неделе размещу другой фильм о Киеве. Он намного лучший.

\iusr{Natasha Levitskaya}

Спасибо! Прекрасный фильм и Киев такой уютный и теплый!
Как хорошо и профессионально вы сняли в 19 лет! Просто, браво!

\iusr{Федор Лебедев}
\textbf{Natasha Levitskaya} - Благодарю Вас!

\iusr{Сергей Евсеенко}
Просто Супер. Щедеврально.

\iusr{Федор Лебедев}
\textbf{Сергей Евсеенко} - благодарю! Успехов Вам!

\iusr{Неля Цекалова}
Прекрасное чувство ностальгии, благодарю...

\iusr{Ольга Белозовская}
Великолепно! Спасибо большое!

\iusr{Алла Лукьянченко}
Спасибо! Это настольгия! Чудесный Город, чудесные люди, чудесное время, хоть и не легкое

\iusr{Giorgi Begeluri}

Sovershenno izumitelni film! Nikogda ne biv v kieve pdumal, esli prediotsa
pobivat v vashem chudnom gorode to vsio mne budet znakoma. Bravo!

\begin{itemize} % {
\iusr{Федор Лебедев}
\textbf{Giorgi Begeluri} - благодарю Вас за просмотр и отзыв! Желаю Вам успехов и здоровья!

\iusr{Giorgi Begeluri}
\textbf{Федор Лебедев}
Eto vam spasibo za takoi xoroshi film pro vash xoroshi gorod. Privet iz tbilisi!
\end{itemize} % }

\iusr{Сизова-Билодид Валентина}
Дякую за спогади.

\iusr{Сергей Чеховский}
Восьмидесятые .....  @igg{fbicon.thinking.face} 

\iusr{Svetlana Voloshina}
Благодарю за Киев моего детства! @igg{fbicon.hands.pray} 

\iusr{Маргарита Евтушевская}
Дякую Вам Майстре! Моє дитинство, мій рідний Киів, спогади.... Браво

\iusr{Федор Лебедев}
\textbf{Маргарита Евтушевская} - дякую і Вам за перегляд і високу оцінку! Бажаю Вам успіхів!

\iusr{Гала Алданькова}
Прекрасный фильм!

\iusr{Ирина Ермакова}
Спасибо, очень интересно

\iusr{Федор Лебедев}
\textbf{Ирина Ермакова} - благодарю Вас! Желаю успехов!

\iusr{Светлана Недайбида-Бучко}
Спасибо огромное за воспоминания! @igg{fbicon.hands.applause.yellow}{repeat=3}
@igg{fbicon.hearts.two}  @igg{fbicon.rose} 

\begin{itemize} % {
\iusr{Федор Лебедев}
\textbf{Светлана Недайбида} - спасибо Вам за просмотр! Желаю успехов и здоровья!

\iusr{Светлана Недайбида-Бучко}
\textbf{Федор Лебедев} В самом конце 80-х я начала работать в Бюро путешествий нештатным экскурсоводом. Спасибо еще раз огромное за воспоминания и эмоции!
Я и сейчас работаю гидом. С удовольствием поделюсь фотографиями.
\end{itemize} % }

\iusr{Надежда Сенчило}
Большая благодарность вам, расчувствовали до слез, как все было гармонично в
родном Киеве. Ностальгия.

\iusr{Федор Лебедев}
\textbf{Надежда Сенчило} - Спасибо за добрые слова! Всего Вам доброго!

\iusr{Георгий Майоренко}
Очень теплый фильм! Спасибо! До слез. Талантливо, профессионально.

\begin{itemize} % {
\iusr{Федор Лебедев}
\textbf{Георгий Майоренко} - благодарю Вас за высокую оценку этого скромного фильма! Желаю Вам всего Доброго!

\iusr{Георгий Майоренко}
\textbf{Федор Лебедев} Все отлично! Как говаривал один мой знакомый: - \enquote{Скромность мое оружие!})) Творческих успехов!
\end{itemize} % }

\iusr{Ирина Попова}
Спасибо большое за этот трогательный фильм. Это Киев моей молодости

\iusr{Федор Лебедев}
\textbf{Ирина Попова} - благодарю Вас! Успехов Вам желаю!

\iusr{Руслана Оксаненко}
Отлично снято, спасибо!

\iusr{Федор Лебедев}
\textbf{Руслана Оксаненко} - благодарю! Всего Вам Доброго!

\iusr{Руслана Оксаненко}
И Вам!  @igg{fbicon.love.letter} 

\iusr{Олег Сушков}
 @igg{fbicon.thumb.up.yellow} 

\begin{itemize} % {
\iusr{Федор Лебедев}
\textbf{Олег Сушков} - Спасибо за просмотр и оценку!

\iusr{Вячеслав Гарбулин}

Был зелёный, чистый и красивый город, родной и любимый Киев!

А сейчас полностью наоборот! Грязь и бездействуют службы!!! И строят что
попало! Увы, но это факты

\end{itemize} % }

\iusr{Инга Болгова}

Искренние слова любви и благодарности автору! Именно таким я впервые увидела
Киев, полюбила его, и живу здесь более лет тридцати лет Жаль только, что
современное градостроительство, а не время, с каждым годом уносит частичку того
города, из фильма. Остаётся только верить в чудесный Дух города, который
охраняет его на протяжении тысячелетий. Автору - браво!

\begin{itemize} % {
\iusr{Федор Лебедев}
\textbf{Инга Болгова} - благодарю Вас за просмотр и тёплые слова! Желаю Вам крепкого здоровья и всего наилучшего!
\end{itemize} % }

\iusr{Мила Свиридова}
Спасибо большое!

\iusr{Наталия Прусова}
Замечательный фильм! Спасибо автору за его творчество! Обязательно нужно создать картину сегодняшнего города!

\iusr{Татьяна Желдубовская}
Спасибо

\iusr{Мария Аронова}
Спасибо!

\iusr{Mykola Anfinogenov}

Спасибо за фильм за память, о том что прошло, это всегда в нашем серце. Будьте
успешным во всем, все всегда держалось на аматорах и интузиастах. Это спокон
веков.!

\begin{itemize} % {
\iusr{Федор Лебедев}
\textbf{Mykola Anfinogenov} 

- благодарю Вас за такие хорошие пожелания! И Вам успехов и здоровья желаю!

\iusr{Олена Потильчак}

Коля, я опять заплакала. Я вспомнила школу, Киевпроект, Дом архитекторов, как
мы большой компанией ездили к тебе на дачу. Я вспомнила нашу аллейку на
Липской. Институтскую. Вроде бы эти улицы есть и их уже нет. Я никак не могу
принять то, что время и родных, любимых уже не вернуть. Коля, ты держись. И я
буду держаться, потому, что нам еще надо встретиться здесь, на аллейке. Береги
свое здоровье.

\iusr{Mykola Anfinogenov}
\textbf{Олена Потильчак} Благодарю теба, я обязательно постараюсь !
\end{itemize} % }

\iusr{Анна Анна}
Спасибо, за воспоминания о нашем Киеве!

\iusr{Diana Melnik}
Спасибо Вам большое. Создайте, пожалуйста, картину сегодняшнего дня.

\begin{itemize} % {
\iusr{Федор Лебедев}
\textbf{Diana Melnik} - благодарю Вас! Планирую на своих уроках со студентами-видеоблогерами скоро начать съёмки. В прошлом году нам помешал карантин. А сейчас в планах сделать несколько оригинальных видео о современном городе.
\end{itemize} % }

\iusr{Oksana Kashenko}
Киев в вашем фильме такой тёплый....
Я как будто побывала в своём детстве
Эта смешнючая девочка с бантом и мороженым напомнила мне меня :)))

\iusr{Федор Лебедев}
\textbf{Oksana Kashenko} - благодарю Вас за просмотр! Желаю Вам всего самого хорошего!

\iusr{Ірина Кот}

Очень трогательно, прямо слезы наворачивались во время просмотра. Чувствуется
атмосфера тех лет. Очень уютно и душевно. Спасибо.

\begin{itemize} % {
\iusr{Федор Лебедев}
\textbf{Ірина Кот} - благодарю Вас за просмотр и высокую оценку! Всего Вам Доброго желаю!

\iusr{Ірина Кот}
\textbf{Федор Лебедев} Вам спасибо, за любовь к Киеву. Творческих Вам успехов и всех благ!

\iusr{Федор Лебедев}
\textbf{Ірина Кот} - взаимно! Благодарю и желаю Вам всего наилучшего!
\end{itemize} % }

\iusr{Евгения Курбатова}
Спасибо огромное за фильм!! Много, к сожалению, уже не увидим сегодня!

\iusr{Олександр Кульбашний}
Спасибо большое

\iusr{Олена Потильчак}

Это время ухода моего детства. Время юности. С одной стороны, лучше бы не
смотрела, больно. С другой стороны очень хорошо, что посмотрела. Это как опять
увидеть очень родного человека, которого уже нет. Спасибо автору поста за ваш
авторский прекрасный фильм.

\iusr{Федор Лебедев}
\textbf{Олена Потильчак} - благодарю Вас за просмотр и высокую оценку! Желаю Вам успехов и самого наилучшего!

\iusr{Радмила Васильева}
спасибо

\iusr{Людмила Гончар}

Это лучшее, лучшее видео про Киев, да и вообще про нас! Спасибо. Прекрасное
посвящение любимому городу и великим людям.

\begin{itemize} % {
\iusr{Федор Лебедев}
\textbf{Людмила Гончар} - 

большое спасибо за просмотр и очень высокую оценку нашего не большого и не
нового фильма! В пятницу я планирую загрузить в эту же группу любителей Киева
(если Администратор разрешит) другой, ещё лучший фильм о Киеве, - \enquote{Город
Архангела Михаила}. Надеюсь, что он Вам понравится! Всего Вам Доброго!

\end{itemize} % }

\iusr{Виктория Чупыра}
Спасибо, этот фильм как из прошлой жизни, ностальгия... слезы...

\iusr{Ирина Ноздренко}

Спасибо за фильм трогательный до слёз, созданный с огромной любовью к городу,
несущий ( теперь уже ностальгическую ) энергетику Киева тех лет. Спасибо за
анонс продолжения о современном Киеве. Ждём))

\iusr{Roman Tymin}
Класс

\iusr{Tim Liabakh}

Фильм - огонь. Жаль, что нищета советского обывателя не позволила иметь много
отснятого Киева тех лет. Тем ценнее такие фильмы.

\begin{itemize} % {
\iusr{Федор Лебедев}
\textbf{Tim Liabakh} - 

спасибо большое за Вашу высокую оценку! Но я с Вами совсем не согласен в том,
что советский обыватель жил в нищете. Жили простые люди в начале 80-х годов не
сравнить, с сегодняшним уровнем жизни. Я лично коммунистом не был, но получил
бесплатно 2 высших образования, бесплатную квартиру и работал режиссёром и
оператором много и интересно. До развала СССР в Киеве работало 5 киностудий.
Зайдите в кино архив им. Пшеничного - им хранилищ не хватает, чтобы уместить
старую кинохронику.

\iusr{Tim Liabakh}
\textbf{Федор Лебедев} 

я про то, что количество отснятого на бытовые кино и видеокамеры гораздо
меньше. Что касается жизни простого тогда человека, и всего бесплатного, то я
бы не хотел вдаваться в эту полемику. Я не много застал того времени, но
прекрасно помню как жили мои родители, бабушки и дедушки.

Спасибо вам за фильм. Если так много отснятого, то конечно его нужно цифровать.
Понимаю, что это безумно дорого и неокупаемо.

\iusr{Галина Якушик}
А колись знімали, і не плакали що дорого

\iusr{Tim Liabakh}
\textbf{Галина Якушик} а хто зараз плаче, шо дорого знімати? Зараз на любий телефон знімати можна
\end{itemize} % }

\iusr{Эля Коваль}
Спасибо Вам человеческое, от всех киевлян, помню город таким, а сегодня
показала свои детям.

\iusr{Федор Лебедев}
\textbf{Эля Коваль} - благодарю Вас за просмотр и добрые комментарии! Успехов всей семье и здоровья желаю!

\iusr{Людмила Порунова}
Київ - моє улюблене місто, але при чому тут \enquote{русские}?

\begin{itemize} % {
\iusr{Olena Machekhina}
\textbf{Людмила Порунова} 

У дописі зазначена дата створення фільму - 1982-й рік. Якби там не було б
тогочасних ідеологічних шаблонів і кадрів з пам'ятниками Леніну, фільму б не
було. Тим паче він був створений на замовлення, тобто сюжет і текст замовлені.

Фільм хороший, теплий, створеній з любов'ю до Києва

\end{itemize} % }

\iusr{Roman Subotin}

Bolshoe spasibo i za publikacoyu filma i za vospiminaniya! Udachi v realizacii idei !

\iusr{Аркадий-Лариса Малюга}
Здорово!

\iusr{Аркадий-Лариса Малюга}
Сбережем его для наших нащадков.

\iusr{Федор Лебедев}
\textbf{Аркадий-Лариса Малюга} - благодарю Вас за просмотр и оценку!

\iusr{Ганна Путова}

Класс. Моя мама - городской экскурсовод, я с 5 лет с ней была на колесах. Как
раз в бюро она работала с 1978 по 1993 г. Здесь - их маршруты, из коллеги
молодые.

\iusr{Федор Лебедев}
\textbf{Ганна Путова} - благодарю Вас за просмотр и отзыв!

\iusr{Галина Местечкина}

Який подарунок знову почути голос незабутнього Анатолія Трофимовича і пройти
разом з ним вулицями рідного міста! Сердечна подяка!

\iusr{Федор Лебедев}
\textbf{Галина Местечкина} - спасибо Вам за просмотр и добрые комментарии!

\iusr{Галина Местечкина}

Спасибі, Федоре, за чудовий фільм і збереження голосу нашого незабутнього
Анатолія Трофимовича! Чекаю продовження вже досвідченого Майстра!

\iusr{Федор Лебедев}
\textbf{Галина Местечкина} - благодарю за просмотр и высокую оценку! Примите от меня Добрые пожелания!

\iusr{Елена Палий}
Огромное спасибо !!!!!!

\iusr{Федор Лебедев}
\textbf{Елена Палий} - благодарю Вас за просмотр и слова благодарности! Всего Вам хорошего и доброго желаю!

\iusr{Evgenia Lazartchuk}
Спасибо, замечательный фильм!

\iusr{Zoya Nechay}

Дякую автору за теплий фільм і щемливі спогади. Окреме сердечне спасибі за
збереження голосу і можливість ще раз почути-побачити Анатолія Халепо. До
сліз..

\iusr{Федор Лебедев}
\textbf{Zoya Nechay} - и я Вас благодарю и желаю Вам всего самого Доброго!

\iusr{Zoya Nechay}
\textbf{Федор Лебедев} Дякую!

\iusr{Lena Andreeva}
Спасибо за память @igg{fbicon.hearts.revolving} 

\iusr{Федор Лебедев}
\textbf{Lena Andreeva} - спасибо и вам за просмотр! Всего Доброго Вам желаю!

\iusr{наталья колесник}
Все такое родное и далекое @igg{fbicon.heart.blue}  @igg{fbicon.heart.yellow} 

\iusr{Mila Bagry-Gaskill}
Beautiful shots, lovely and so nostalgic @igg{fbicon.heart.red}

\iusr{Федор Лебедев}
\textbf{Mila Bagry-Gaskill} - благодарю Вас! Желаю всего хорошего!

\iusr{таня козырева}
Да, это был город для жизни!

\iusr{Федор Лебедев}
\textbf{таня козырева} - спасибо за просмотр!

\iusr{Людмила Заец}
Фильм классный, увидела город детства, спасибо

\begin{itemize} % {
\iusr{Федор Лебедев}
\textbf{Людмила Заец} - благодарю! Успехов Вам!

\iusr{Людмила Заец}
\textbf{Федор Лебедев} фотографии вам надо без людей ?

\iusr{Федор Лебедев}
\textbf{Людмила Заец} - Спасибо! Мне нужны виды старого Киева (до революции 17 года).

\iusr{Людмила Заец}
\textbf{Федор Лебедев} понятно
\end{itemize} % }

\iusr{Valentyna Komerzhenko}
Це неймовірно!! Дякую!

\iusr{Федор Лебедев}
\textbf{Valentyna Komerzhenko} - благодарю Вас за комментарий! На неделе выставлю ещё один наш фильм о Киеве, - он намного лучше. Всего Вам самого наилучшего желаю!

\iusr{Ніна Панченко}
Спасибо, прекрасный фильм, приятные воспоминания.... ностальгия по городу юности....

\iusr{Yuliya Kolotilova}
Спасибо, какой был город@igg{fbicon.heart.red}

\iusr{Ян Приворотский}
Словно снова в детстве побывал...

\iusr{Валерия Руденко-Ярощук}
Браво! Очень интересно!

\iusr{Наталия Багрянцева}
Благодарю. Замечательно.

\iusr{Александр Шукевич}
Чудесный фильм, о Чудесном Родном Городе! Огромная Благодарность !

\begin{itemize} % {
\iusr{Федор Лебедев}
\textbf{Александр Шукевич} - и я Вас благодарю за просмотр и за тёплый отзыв! Успехов Вам!

\iusr{Александр Шукевич}
\textbf{Федор Лебедев} предлагаю Дружбу в F.
\end{itemize} % }

\iusr{Vitalij Vartiainen}
Как же город изменился до неузнаваемости! Замечательный исторический фильм!

\iusr{Anastasia Raint}
Невероятно!!!! Потрясающе!!!!!  @igg{fbicon.face.smiling.hearts}  @igg{fbicon.face.happy.two.hands} 
Спасибо большое за \enquote{дверцу} в прошлое @igg{fbicon.face.happy.two.hands} 

\iusr{Федор Лебедев}
\textbf{Anastasia Raint} - спасибо Вам за просмотр и такие комментарии! Успехов Вам больших желаю!

\iusr{Алла Голуб Завада}
Спасибо большое Вам за фильм, который многих вернул в детство. Очень скучаю по тому Киеву.

\iusr{Федор Лебедев}
\textbf{Алла Голуб Завада} - Благодарю Вас! Всего хорошего Вам желаю!

\iusr{Наталья Шатыко}

Какой прекрасный фильм! Спасибо вам за память! Я помню этот город именно
таким! Помню эти экскурсии! Фонтаны, площади, каштаны... Трогательно до слез,
душа щемит... Спасибо огромное за ваш профессионализм! Здоровья крепкого,
успехов!

\iusr{Федор Лебедев}
\textbf{Наталья Шатыко} 

- благодарю Вас за просмотр и за высокую оценку этой работы! Судя по
многочисленным комментариям, мы не зря с подростками таскали аппаратуру по
Киеву. Всего Вам Доброго!

\iusr{Татьяна Вальдовская}
Спасибо! Окунулась в детство и юность!

\iusr{Лариса Пекарська}
Спасибо, за встречу смоим детством.

\iusr{Клавдия Плаксий}
Великолепный красивый фильм. Вы большой молодец! Спасибо вам!

\iusr{Светлана Вербицкая}
Спасибо большое, увидела свой город детства, и места мимо которых проходила каждый день.

\iusr{Анна Яненко}
Дякую! Підкажіть, а чи є десь у доступі згаданий Ваш фільм про Києво-Печерську
лавру?  @igg{fbicon.smile} 

\begin{itemize} % {
\iusr{Федор Лебедев}
\textbf{Анна Яненко} 

- дякую Вам! Ні, фільму про Лавру в Ютубі немає. Але я виставлю його у цій
групі десь, через тиждень. Його назва: \enquote{Летопись в камне} (1980рік).

\iusr{Анна Яненко}
\textbf{Федор Лебедев} дякую! Чекатиму  @igg{fbicon.smile} 

\iusr{Анна Яненко}
\textbf{Федор Лебедев} 

на сайті Національного заповідника \enquote{Києво-Печерська лавра} є рубрика \enquote{На згадку
про Києво-Печерську лавру}, де є чимало фотографій і поштівок. За бажанням
можна сконтактувати з авторами рубрики для отримання якісніших відбитків  @igg{fbicon.wink} 

\iusr{Федор Лебедев}
\textbf{Анна Яненко} дуже Вам дякую!
\end{itemize} % }

\iusr{Anatoliy Kanevsky}
Спасибо огромное за прекрасный фильм навеявший столько приятных и незабываемых воспоминаний. @igg{fbicon.heart.red}{repeat=3}  @igg{fbicon.heart.eyes}{repeat=3}  Сохраню

\iusr{Федор Лебедев}
\textbf{Anatoliy Kanevsky} -благодарю Вас за просмотр и добрые слова! Всего Вам доброго и будьте здоровы!

\iusr{Nadya Harlash}
Очень талантливая работа. Спасибо Вам огромное  @igg{fbicon.face.smiling.eyes.smiling} 

\iusr{Людмила Серая}
Замечательній фильм. Получила удовольствие. Спасибо. Удачи Вам!

\iusr{Татьяна Барабаш}
Спасибо вам большое! Приятно вернуться во время своей юности! Мой любимый и родной город!

\iusr{Asya Postoyenko}

Фільм дійсно дуже гарний, динамічний і гармонійний, а як для 21-річного юнака -
так і зовсім дивовижний. Я його показуватиму своїм штатівцям, що вчать
россійську - тому що відео з такою чіткою дикцією знайти дуже важко.. зараз всі
лопочать)

\iusr{Федор Лебедев}
\textbf{Asya Postoyenko} - дякую! Бажаю Вам щастя!

\iusr{Татьяна Губарь}
Спасибо за экскурсию по-любому городу, там где прошло детство, юность и молодость.

\iusr{Людмила Пироцкая}
Хороший фильм. согретый любовью к родному древнему и вечно молодому Киеву!!! @igg{fbicon.hand.ok}  @igg{fbicon.hands.pray} 

\iusr{Наталия Исхакова}
Ждём с нетерпением новый фильм! Спасибо!

\iusr{Наталья Ершова}
Спасибо за прекрасный фильм!!!

\iusr{Светлана Юшина}

Получила большое удовольствие! Прекрасный фильм, большое Вам спасибо! Мастерски
снят Киев, текст непревзойдённого Анатолия Трофимовича Халепы (мой Учитель)!
Время начала наших экскурсий. @igg{fbicon.heart.suit}

\iusr{Вероника Иванова}
Київ нашої юності! Нічого краще не бачила  @igg{fbicon.face.smiling.hearts} 

\iusr{Федор Лебедев}
\textbf{Вероника Иванова} - дякую Вам за перегляд і дуже високу оцінку!

\iusr{Лиса Патрикеевна}

Замечательный город детства. После таких фильмов понимаешь, что это уже другой
мир, другие люди и другой город, которые остались в прошлом

\iusr{Валентина Шахова}

Чудовий фільм, питання чому «Киев для русского человека начало государства» (не
дослівно, ) чому «русского», не українці 40 років назад населяли, наразі, Київ

\begin{itemize} % {
\iusr{Федор Лебедев}
\textbf{Валя Шахова} - 

спасибо Вам за просмотр и высокую оценку фильма! Попытаюсь кратко ответить Вам:
фильм создавался по заказу киевского городского бюро путешествий и экскурсий в
1982г., при правлении страной Л. И. Брежнева. Соответственно, и объекты съёмок
(\enquote{ленинские памятники} и др.) снимались по предложенному заказчиками
списку. В те годы тема \enquote{Киевской Руси} пропагандировалась как история
единого исторического славянского корня, от которых произошли русские, украинцы
и белорусы. И лично я поддерживаю такое определение. И если Вы изучаете историю
по книгам, написанным серьёзными учёными (например, академиком П.П.Толочко), то
Вы согласитесь с тем, что во времена древнего Киева (\enquote{Киевской Руси})
ещё не было ни украинцев, ни русских, ни белорусов. Термин \enquote{украинец} появился
в совсем недавно, - середине 19в. Поэтому у того же Т. Г. Шевченко ни в одном
произведении Вы не найдёте слова \enquote{украинец}, \enquote{украинка}
(правда, он часто упоминал \enquote{Украйну} - но не как определение отдельной
страны, а как определение региона Российской империи). Да и сам он о своей
национальности писал: \enquote{руський малорос} (как впрочем, в те годы
\enquote{руськими} и \enquote{русинами} себя называли все жители Малоросии в
т.ч. и Галичины. Но если Вы сторонница \enquote{теорий} \enquote{профессора}
Бэбика, который утверждает, что апостол Андрей был украинцем, то конечно, - Вы
со мной не согласитесь. Лично я, - режиссёр и сценарист, много лет работавший и
в различных архивах и в Национальной академии наук Украины (будучи всю жизнь
беспартийным), остаюсь на позициях уважаемого мною академика П. П. Толочко, с
которым хорошо знаком и много общался. Сегодня, конечно, фразу: \enquote{близка
сердцу каждому русскому человеку}, я бы заменил на другую, более толерантную и
не обижающую ни украинцев (которым и я являюсь) ни белорусов. Всего Вам
доброго!

\begin{itemize} % {
\iusr{Наталья Емельянова}
\textbf{Федор Лебедев}

Фильм чудесный, так приятно смотреть на Киев, каким он был 40 лет
назад... Но........ Шевченко и не мог бы записать себя украинцем, потому что
никто в Российской империи ему бы этого не позволил. Термин... а это прежде всего
термин... \enquote{Малороссия} нам был навязан и все об этом знают. Что касается термина
\enquote{руський}..... Тарас Григорьевич не ошибся... Это одно из названий украинцев,
только более древнее.. многие путают с термином \enquote{русский}. Далее.... ежели
человек не является сторонником академика Толочко, то он необязательно
разделяет измышления Бебика, который историком не является и которого так любят
цитировать в российской Федерации... В Украине есть много других молодых
серьезных и интересных историков

\iusr{Федор Лебедев}
\textbf{Наталья Емельянова} 

- согласен с Вами, кроме одной фразы: \enquote{Шевченко и не мог бы записать
себя украинцем, потому что никто в Российской империи ему бы этого не
позволил}.  Откуда такое у Вас утверждение? Где документы, (книги, газеты,
частные письма хотя бы), которые могут подтвердить, что во времена Т. Шевченко
уже упоминались \enquote{украинцы} и \enquote{украинки}?... Мне очень интересно
будет прочитать такое упоминание.... Благодарю! Успехов Вам желаю!

\iusr{Наталья Емельянова}
\textbf{Федор Лебедев} 

Есть Википедия.. это самое простое.

\url{https://uk.wikipedia.org/wiki/Українці}

Исходя из текста в годы жизни Шевченко термин \enquote{украинцы} уже
существовал.. Не говоря уже о термине \enquote{Украина}.. Впрочем если Вы
приверженец академика Толочко/ при все моем уважении к его знаниям/ Вы вряд ли
будете согласны.

\iusr{Федор Лебедев}
\textbf{Наталья Емельянова} 

- благодарю за ссылку! К сожалению, по присланной Вами ссылке я так и не смог
обнаружить в статье, написанной анонимом, да ещё и в (украино-язычной
Википедии) ни одного документа, подтверждающего то, что слово
\enquote{украинец} во времена Т. Г. Шевченко имело хождение как определение
национальности малороссийского народа (\enquote{руських}, \enquote{русинов}), к
которым себя причислял и Тарас Григорьевич... . Буду очень признателен и рад,
если Вы сможете предоставить мне ссылки на архивные документы! В свою очередь,
предлагаю Вам разыскать в интернете запрещённую сегодня в Украине книгу
Александра Каревина \enquote{Украинский язык. История становления и развития -
как рождалась \enquote{рідна мова}}, Изд.  \enquote{Киев-2011} с предисловием
академика П. П. Толочко. Книга запрещена именно потому, что содержит архивные
документы из украинских же архивов, документы, которые подтверждают правоту
изложенного автором. В частности, там же, можно прочитать и о том, когда
впервые вошёл в употребление термин \enquote{украинец} (автор доказывает, что
этот термин начал применяться как определение национальности лишь во второй
половине 19в., т.е. совсем недавно).

\iusr{Chernysh George}
\textbf{Федор Лебедев} как не было а это что

\ifcmt
  ig https://scontent-frt3-1.xx.fbcdn.net/v/t1.6435-9/172770054_3899505230165558_8081894694697601593_n.jpg?_nc_cat=102&ccb=1-5&_nc_sid=dbeb18&_nc_ohc=wXpsTpyFurgAX8lFD8V&_nc_ht=scontent-frt3-1.xx&oh=00_AT_ecHItRtjjm3riF4raqAPG5WzbhCICgSw-ZYfTyKOp1g&oe=622225B5
  @width 0.3
\fi

\end{itemize} % }

\iusr{Наталья Емельянова}

Вы так пренебрежительно написали.. \enquote{....да еще и в украино-язычной
Википедии}.... Вы считаете \enquote{русско-язычная} Википедия более правдива?
Сомневаюсь.. Книгу нашла совершенно свободно на украинском
ресурсе. 

\href{https://booksonline.com.ua/view.php?book=54908}{
АЛЕКСАНДР КАРЕВИН: РУСЬ НЕРУССКАЯ (КАК РОЖДАЛАСЬ «РІДНА МОВА»), booksonline.com.ua%
}

Немного прочла... Ничего
нового, я такое чтиво часто встречаю на российских сайтах, его просто море,
написано для простого обывателя .. Не хочу сказать, что все в подобных книгах
ложь. Нет, конечно... есть достаточно правдивая информация. Но... видите ли... я
историк по образованию и прекрасно понимаю что, где и почему. И. главное,
зачем... Спорить не хочу... честно.. Мы же с Вами в ФБ. а не на учёном
совете.. Что касается термина \enquote{украинец}... я обязательно для себя поищу
правду... сейчас не готова ответить ибо на все нужно время.. Но..... даже если
он появился и в 19 веке, что это меняет? Мы есть. И сейчас мы называемся
украинцами.

\iusr{Федор Лебедев}
\textbf{Валя Шахова} 

- благодарю Вас за ответ! А я так и написал: \enquote{предлагаю Вам разыскать в
интернете}. В украинских магазинах Вы её не найдёте. Автор нынче не может
издать её в Украине (я с ним лично знаком). А тот вариант, который Вы
обнаружили на частном сайте, - это сокращённый вариант. Он не содержит большое
вступление рецензента, - академика П. П. Толочко. И увы, - нет никаких ссылок! У
автора в финале на нескольких страницах указаны ссылки на все цитаты, на сотни
архивных документов. Да и самого финала книги нет, он почему-то отсутствует...
Вы спрашиваете: \enquote{Но..... даже если он ( украинский язык) появился и в 19 веке,
что это меняет?}. Это очень многое меняет! Это означает, что этот язык не имеет
права считаться единственным, государственным языком! Пусть себе будет! Пусть
даже будет единственным государственным языком (хотя лично считаю, что это не
справедливо и не разумно)! Но то, что сейчас в Украине переписывают историю с
целью воспитания молодёжи в ненависти ко всему русскому, даже к русскому языку
(!), - это я считаю и есть настоящий геноцид, направленный на создание
этнических конфликтов внутри некогда весьма благополучного государства (я это
время ещё застал). И вина в этом не русскоязычных украинцев (читай: \enquote{горожан}),
это вина тех, кто по своей необразованности ходит с факелами по Крещатику и
орёт \enquote{Україна для українців!}, \enquote{Коломия, - це Європа!}, \enquote{Москалів - на
ножі!}... Хочу надеяться, что жители украинских городов всё-таки проснутся и
дадут отпор фашиствующим хлопцам с хутора. И это в их же интересах. Сами они,
своей не дальновидными поступками и агрессией, принесут стране только разруху и
раскол территорий, что собственно мы сегодня уже наблюдаем.

\begin{itemize} % {
\iusr{Наталья Емельянова}
\textbf{Федор Лебедев} 

Вы понимаете... это долгий разговор и не в формате ФБ. Я не писала, что
украинский язык появился в 19 веке.. Я писала о термине \enquote{Украинцы}. Что
касается украинского языка, то он сформировался как литературный в 18 веке.А
появился еще раньше, вернее сформировался.. Избитый пример - Котляревский
.... Становление всех языков в мире проходило сложный путь.. английский,
немецкий, голландский... Вы слышали голландский? То ли немецкий, то ли
английский. В этом становлении языки смешивались, влияло соседство народов,
завоевания одних другими и т.д. Да что языки.. каждое слово имеет свою
историю... Тот же путь прошел и украинский язык.. Особенно много в его развитие
вложили украинский писатели 19 века. Но наш с Вами спор - это разговор
дилетантов. Историей украинского языка пусть занимаются ученые лингвисты. А не
россияне, орущие со своих экранов о том, что украинский язык - это наречие, что
это смесь русского с польским и так далее... Что касается государственного
языка, то я - русскоязычная украинка, учившаяся и в школе и в институте на
русском языке /И это в Украине!?!?!?))) /, жившая в России, считаю, что
государственным должен быть только один язык - украинский. И все жители Украины
должны его знать..ВСЕ... ТОЧКА. Я презираю людей, пренебрежительно говорящих о
моем родном языке.. все эти \enquote{на мове} и т.д. Еще более противны мне этнические
украинцы, не желающие учить свой родном язык.. Потому что им ЛЕНЬ, потому что в
ОБЛОМ !!! К В остальном я ЗА широчайшее использование русского.. книги, фильмы
и т.д..много можно перечислять. Далее... что касается переписывания истории...
Знаете, история - это очень сложно... я бы даже наукой ее не
называла... Переписывание у нас имеет место... да, и не всегда в лучшую
сторону...Как и в соседней России..какой там пишут бред ... Молодцы европейские
историки - собрались, договорились, и учебники истории в европейских странах
составлены таким образом, чтобы никому не было обидно. Были такие попытки и у
нас с россиянами.....но... а потом грянул Крым, Донбасс... И все
рухнуло.... ВСЕ, по крайней мере для меня. Думаю, что если бы не Крым с
Донбассом, то и для факельных шествий не было бы почвы.. Я не их сторонник, но
..... украинский национализм будет расти с усилением попыток соседней страны к
продвижению у нас так называемого \enquote{русского мира} ....и это не Чехов с
Достоевским... Это война..... К Бандере. движению ОУН-УПА отношусь сложно... с
одной стороны это борьба за независимость, отдаю должное, а с другой.... ..цель
не оправдывает средства.....Погромы евреев во Львове в 41 и другое... Тут есть
у нас перегиб, в этом еще долго будут разбираться историки... Вот очень
вкратце...

\iusr{Наталья Емельянова}

Добавлю... Украина очень сложная страна, так уж сложилась ее история.. Путем
добавления русского как второго государственного ее не объединить.. Здесь нужны
другие подходы ..Но какие, я не знаю, я не политик... На все это нужно время,
думаю лет 30-40, если не больше... Государства и нации складывались столетиями,
что уж говорить про 30 лет.. Украина - это пока маленький ребенок, она еще на
горшке сидит, со всеми вытекающими..... Было бы проще, если бы не вмешивалась
РФ, договорились бы со временем и запад, и восток, и юг... Но увы....

\iusr{Федор Лебедев}

государственным языком Емельянова - я с Вами во всём согласен, кроме одного, -
если бы при ющенко не начали на уровне государства целенаправленно третировать
русский язык, - не было бы у нас проблем ни с Крымом, ни с Донбассом. Не нужно
быть специалистом, чтобы не понимать, что с момента развала СССР, американцы
тратят огромные деньги, направленные на раскол между некогда братскими
советскими республиками. 

И факельные шествия начались не после гос. переворота
2014г. Я лично снимал эти шествия с факелами, с дикими криками: \enquote{Москалів на
ножі!} ещё 2001 году во Львове. Уже тогда они открыто зиговали и никто им этого
не запрещал. И никто меня не убедит в том, что это правильно, - что сегодня
русский язык активно третируется и изгоняется из страны. 

Это великая глупость
для Украинского государства и очень удачная стратегия \enquote{друзей} из-за
океана. И ежегодное, стремительное падение уровня экономики, культуры и науки в
Украине, начавшиеся после развала СССР, - неоспоримое доказательство того, что
сама Украина, без экономической и научной кооперации с ближайшими славянскими
народами из этой ямы не выберется. 30 лет разрухи, - это слишком много. И
русскоязычные украинцы масла в огонь национальных отношений внутри страны
никогда не подливали. Наоборот! Читайте ту книгу А,Каревина, о которой мы выше
говорили, - в ней множество примеров того, как в 20-е годы советская власть
агрессивно проводила украинизацию в украинских городах и даже в сёлах! Мне
всегда было стыдно за тех украино-мовных украинцев, которые в УССР,
перебравшись из сёл в русскоязычные города, сами спешили по-скорей выучить
русский и забыть свой родной украинский. Я таких примеров видел много. Но
русскоязычные никогда не третировали украноязычных, потому, что не нужно быть
умным, чтобы не понимать, какое преимущество приносит стране распространение в
ней нескольких языков (пример Канада, Швейцария, Швеция и другие страны...).

\iusr{Григорий Бодун}
\textbf{Федор Лебедев} 

Указ Петра1 \enquote{О запрете книгоиздания на украинском языке} датирован 1720
годом.  Почти за 100 лет до рождения Т. Г. Шевченко. У вас шарий головного
мозга - перестаньте употреблять инфотоксины.

\iusr{Федор Лебедев}
\textbf{Григорий Бодун} 

- сбросьте мне ссылочку на сей замечательный документ, пожалуйста. Вообще-то,
при Петре Первом термина ~украинский~ ещё не существовало. Как не найдёте вы ни
в одном произведении Т. Шевченко такого термина, как ~украинка~, ~украинец~...

\iusr{Chernysh George}
\textbf{Федор Лебедев}

\ifcmt
  tab_begin cols=2,no_fig,center,resizebox=0.8

     pic https://scontent-frt3-1.xx.fbcdn.net/v/t1.6435-9/173116662_3899508840165197_3003704416479596523_n.jpg?_nc_cat=106&ccb=1-5&_nc_sid=dbeb18&_nc_ohc=616iwExu7AAAX-tSn4F&_nc_ht=scontent-frt3-1.xx&oh=00_AT-fXtM78KXg1nfmK6nimYXzr29oCgioI20Zq61Q089kMQ&oe=6220D02F

		 pic https://scontent-frt3-2.xx.fbcdn.net/v/t1.6435-9/172695055_3899508943498520_6775363909831611416_n.jpg?_nc_cat=103&ccb=1-5&_nc_sid=dbeb18&_nc_ohc=bmftkoCSBhgAX8oiQMH&_nc_ht=scontent-frt3-2.xx&oh=00_AT-BPf-1sZUFitIhUnUdB1GT0xQLiePGSdEtf2sYGB_8BA&oe=6221532C

  tab_end
\fi

\end{itemize} % }

\iusr{Татьяна Анисимова}
Бред

\iusr{Татьяна Анисимова}
Историю пишут люди. А потом переписывают.

\end{itemize} % }

\iusr{Саша Синякова}
\textbf{Федор Лебедев},
спасибо Вам !!!

\iusr{Федор Лебедев}
\textbf{Саша Синякова} - и Вам большое спасибо за просмотр и оценку!

\iusr{Елена Алексеевна Подтэпа}

Замечательный фильм, приятно вспомнить родной город времён моей молодости.
Помню много экскурсий по городу, которые проводил Халепо. Город утопал в
зелени, а сейчас... Спасибо

\iusr{Наталия Рашевская}
Дякую!

\iusr{Ирина Бахмет}

Древне-новый город моей молодости. Некоторые кадры сняты после долгого
ожидания, когда солнце, облака, движение воды найдут свое совершенство.
Невольно останавливала кадр, чтобы присмотреться и полюбоваться тем, что живет
уже только \enquote{в цифре}. Со временем уйдет Киев и сегодняшнего дня, но каждая
эпоха впитает частичку Киева во всей его истории.

\iusr{Федор Лебедев}
\textbf{Ирина Бахмет} - сердечно благодарю Вас за внимательный просмотр фильма и чуткий комментарий! Желаю Вам всего доброго!

\iusr{Anatoliy Kanevsky}

Таким он остался для меня @igg{fbicon.heart.red}{repeat=3} таким я его буду
помнить всегда. Спасибо огромное за чудесный фильм и за те воспоминания которые
возбудились в моей памяти.

\iusr{Федор Лебедев}
\textbf{Anatoliy Kanevsky} - благодарю Вас за просмотр и добрый отзыв! Успехов Вам желаю!

\iusr{Инна Савенко}
Вернулась в детство .

\iusr{Константин Титов}

Замечательный фильм!

Снято душевно, красиво, талантливо.. Вызывает множество чувств и эмоций.

Включил супруге \enquote{на секундочку} и - оба не смогли оторваться до самого конца
фильма. Браво авторам и спасибо, что сняли, сохранили и выложили!


\iusr{Федор Лебедев}
\textbf{Константин Титов} 

- сердечно благодарю Вас и Вашу супругу за душевные комментарии! Всего Вам
наилучшего желаю!

\iusr{Natalie Kalenskaja}
 @igg{fbicon.face.blowing.kiss} 

\iusr{Федор Лебедев}
\textbf{Natalie Kalenskaja} - благодарю Вас!

\iusr{Elizaveta Gorbonos}

Спасибо,!!! Прекрасные съёмки. Это мой любимый Киев времён моей юности!
Знакомо почти каждое место! ещё раз большое спасибо,!

\iusr{Виктор Иванович Кривуша}

Спасибо за путешествие в молодость. \enquote{Жили и дружили мы: славяне и евреи,
осетины, грузины и армяне, беззаветно верили в то время - национальность наша
просто киевляне} (из авторской песни киевлянина, воспитанника Дворца пионеров)

\begin{itemize} % {
\iusr{Федор Лебедев}
\textbf{Виктор Иванович Кривуша} - согласен с Вами полностью!Благодарю за поосмотр! Желаю Вам всяческих Успехов!

\iusr{Виктор Иванович Кривуша}
\textbf{Федор Лебедев} наилучшие пожелания. Здоровья и успехов!
\end{itemize} % 

\iusr{Лидия Воронина}

\ifcmt
  ig https://scontent-frx5-2.xx.fbcdn.net/v/t39.1997-6/s168x128/47270791_937342239796388_4222599360510164992_n.png?_nc_cat=1&ccb=1-5&_nc_sid=ac3552&_nc_ohc=a_eg3n54jBwAX_8oD-g&_nc_ht=scontent-frx5-2.xx&oh=00_AT9xZNHEG5FqKI6NAP0bIunNy4L52GpEX6jc-pbC182vrg&oe=62004DF6
  @width 0.1
\fi

\iusr{Гельмадина Назарчук}
Большое спасибо за это волшебное путешествие в незабываемое прошлое.

\iusr{Konstantin Shamin}

Спасибо Федор Геннадиевич!  @igg{fbicon.hand.ok}
@igg{fbicon.face.happy.two.hands}  @igg{fbicon.clapboard} 

\iusr{Наталья Шатыко}
Спасибо!

\iusr{Lidija Luchkova}
Прекрасно!!!

\iusr{Svetlana Batrak}

Федор, огромнейшее спасибо! Этим видео вы вернули меня в город моего детства.
Таким я его помню маленькой девочкой. Все в этом видео прекрасно, а голос,
текст и музыка ещё больше передают настроение и того времени. Для меня оно было
самым счастливым! Были живы все мои родные...

\iusr{Федор Лебедев}
\textbf{Svetlana Batrak} - и Вам Большое Спасибо! Успехов!

\iusr{Вікторія Святненко}
Спасибо!

\iusr{Татьяна Гурьева}
Как интересно и познавательно) @igg{fbicon.100.percent} таким Киев запомнится многим из нас.

\iusr{Oksana Krassilshchykova}
Замечательный фильм о моем городе. Душа болит от того, как его \enquote{плюндрував} Майдан!

\begin{itemize} % {
\iusr{Галина Полякова}
\textbf{Oksana Krassilshchykova} А Майдан при чем тут?

\iusr{Boris Pryshchepa}
\textbf{Oksana Krassilshchykova} ну ти і паскуда
\end{itemize} % }

\iusr{Ольга Каменская}

Да, не узнать наш Киев сейчас! А выпускные на Владимирской горке и опять таки
потом Крещатик на закуску! Nostalgia!!!

\iusr{Sudorgina Lusia}
Изумительно!!! @igg{fbicon.heart.sparkling}{repeat=3} 

\iusr{Пётр Сазонов}
это был лучший город на Земле

\iusr{Инга Крюкова}
Спасибо, прекрасный фильм

\iusr{Федор Лебедев}
\textbf{Инга Крюкова} - благодарю Вас за оценку.

\iusr{Елена Олейник}
Чудесній фільм

\iusr{Юлия Марковна}
Спасибо @igg{fbicon.heart.red}{repeat=3} 🐅

\iusr{Natalia Marevic}

Спасибо за воспоминания, за Город, который ещё многие помнят таким - теплым,
уютным, зеленым, с памятниками его истории и истории всей Украины ...

\iusr{Мария Амариллис}

Фёдор, большое вам спасибо @igg{fbicon.heart.red}{repeat=3} растрогалась до
слез. А где можно посмотреть ваш фильм про Лавру)))

\iusr{Ирина Гончарук}

Отличный фильм! Моё детство, моя юнность @igg{fbicon.face.sad.but.relieved}  Я
хочу заплакать @igg{fbicon.face.sad.but.relieved} 

\iusr{Федор Лебедев}
\textbf{Ирина Гончарук} - и Вас благодарю за просмотр и отзыв!

\iusr{Sergyi Grabar}

Дякую. Хороший фільм. Важливий для розуміння міста.

\iusr{Нелли Коваль}
Спасибо за прекрасные воспоминания, каким был Киев, утопал в зелени, фонтаны!

\iusr{Igor Ilinski}

отдельное спасибо за участие Халепо! Гениальный дядька! Ходил в его экскурсии.
А фильм - чудо! Огромный пласт моих лучших воспоминаний. Смотрел и думал - как?
Как можно было за 30 лет так изгадить любимый город? Совок, говорите? Почему
при \enquote{совке} в 50-70-х ничего не сносили и берегли лицо города?!

\iusr{Оля Артющенко}
НОСТАЛЬГІЯ... ДИВИЛАСЬ І ПЛАКАЛА, ЦЕ ДИТИНСТВО, ХОЧУ ТУДА ХОТЬ НА МИТЬ

\iusr{Светлана Блаус}
Наш родной любимый город!

\iusr{Елена Красильникова}

Дуже Теплі і Чуттєві спогади, навіяв цей фільм, багато місць змінили своє
обличчя, але головне - це Розуміння того, що це Миле Серцю Вічне Місто, Місто з
глибинною та потужною історією!

\iusr{Ольга Филенко}
Наша юність. І той Київ. який залишився в пам'яті. І зовсім молодий Халепо.

\iusr{Ирина Бузунова}

С большим интересом посмотрела фильм
Спасибо

\end{itemize} % }
