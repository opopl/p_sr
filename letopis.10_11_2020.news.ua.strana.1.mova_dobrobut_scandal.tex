% vim: keymap=russian-jcukenwin
%%beginhead 
 
%%file 10_11_2020.news.ua.strana.1.mova_dobrobut_scandal
%%parent 10_11_2020
 
%%url https://strana.ua/news/300072-klinika-dobrobut-jazykovoj-skandal-s-byvshim-merom-kieva-bondarenko.html
%%author 
%%tags 
%%title 
 
%%endhead 

\subsection{Мовный скандал в "Добробуте". Как бывший глава Киева устроил перепалку из-за русскоязычной медсестры}
\Purl{https://strana.ua/news/300072-klinika-dobrobut-jazykovoj-skandal-s-byvshim-merom-kieva-bondarenko.html}
\Pauthor{Венк, Виктория}

\ifcmt
img_begin 
	url https://strana.ua/img/article/3000/72_main.jpeg
	caption Клиника "Добробут", Владимир Бондаренко (сверху) и Вадим Елизаров. Коллаж "Страны" 
	width 0.7
img_end
\fi

В Украине произошел очередной скандал на почве русского языка.

На этот раз зачинщиком конфликта стал бывший глава КГГА Владимир Бондаренко. Он
как пациент посетил клинику "Добробут" в Киеве. Сотрудница говорила по-русски,
он потребовал перейти на мову. После чего разразился скандал, в ходе которого
Бондаренко вызвал полицию. Более того, он утверждал, что один из врачей дал ему
буквально ногой под зад.

Правда, после того как клиника вступилась за своих сотрудников и пригрозила
бывшему чиновнику судом, тот удалил пост с обвинениями. 

"Страна" рассказывает, что известно об очередном языковой скандале, которых
стало резко больше после вступления в силу закона о тотальной украинизации. 

\subsubsection{Версия Владимира Бондаренко }

Владимир Бондаренко (глава КГГА в марте-июне 2014 года, нардеп пяти созывов), о
происшествии написал в Facebook.

По его словам, 8 ноября в клинике "Добробут" ему сделали операцию по удалению
жировика на спине. Во второй половине этого же дня его пригласили на перевязку.
Он прибыл в кабинет врача, манипуляцию делал доктор Вадим Елизаров. 

"Медсестра, которая ему ассистировала, говорила только на русском языке. Я
попросил соблюдать требования Конституции Украины в лечебном учреждении и
получил ответ о том, что они будут говорить так, как захотят", – написал
Бондаренко.

Отметим, что Конституция как раз гарантирует свободное использование русского
языка. А запрещает говорить по-русски в публичной сфере закон "Об особенностях
функционирования украинского языка как государственного". Чего Бондаренко не
мог не знать. 

Политик отметил, что не захотел это терпеть и решил уйти из кабинета, однако
медики якобы начали его оскорблять и бить. 

"В спину я почувствовал оскорбления, выкрики, а потом удар ногой и
ругательство, чтобы я шел прочь. Вслед в коридор была выброшена моя одежда", –
заявил бывший глава КГГА. 

Он отметил, что оставил запись в книге жалоб, но не уверен, что она
сохранилась. Бондаренко добавил, что был вынужден обратиться в полицию, в
департамент КГГА и Ассоциацию медиков Украины.

\ifcmt
pic https://strana.ua/img/forall/u/11/1/Screenshot_12(25).jpg
\fi

Позже Владимир Бондаренко скрыл обвинительный пост в адрес доктора Елизарова и
клиники. Вероятно, экс-чиновник таким образом выполнил требование "Добробута" и
забрал свои слова назад (о чем еще будет ниже). 

\ifcmt
pic https://strana.ua/img/forall/u/10/91/%D0%A1%D0%BD%D0%B8%D0%BC%D0%BE%D0%BA_%D1%8D%D0%BA%D1%80%D0%B0%D0%BD%D0%B0_2020-11-10_%D0%B2_13.29_.30_.png
\fi

\subsubsection{Версия доктора Елизарова}

Хирургом, который делал перевязку Бондаренко, был доктор Вадим Елизаров. Ранее
он был участником событий Майдана. Потом поехал добровольцем на Донбасс -
полтора года был в составе 59-го мобильного госпиталя из Винницы.

Елизаров отреагировал на слова Бондаренко. Доктор записал видеообращение, в
котором сообщил, что не будет отступать от своей позиции и не станет извиняться
перед Бондаренко.

По словам врача конфликт произошел на почве оскорбительной формы замечания,
которое Бондаренко сделал медсестре, разговаривающей на русском во время
приема. 

"Когда при мне оскорбляют женщину и унижают ее достоинство, то эта женщина
может быть спокойна - я смогу за нее постоять", - сказал медик на видео. 

Елизаров продолжил: "Когда ко мне обращаются пациенты, они пытаются найти
коммуникацию и рассказывают мне о своих проблемах. И рассказывают на том языке,
на котором больше всего болит. И я пытаюсь отвечать им тем же. Українською,
по-английски, на русском, на турецком". 

В продолжении видеообращения Вадим Елизаров заявил, что для него есть основные
законы порядочности. 

"Иногда возникают вопросы законности. Законно ли говорить на приеме у врача не
на государственном языке? Не знаю. Я наверное плохо разбираюсь в законах, но
мне кажется, что законность, прежде всего, это порядочность. Порядочность
человеческого общежития: не воровать, не убивать, не унижать, беречь свою
страну и заботиться о своем доме, о своей семье", - сказал Елизаров.

По словам Вадима Елизарова, сейчас он лечится от рака, который с июня вошел в
стадию ремиссии. Но Елизаров говорит, что у него хватит сил постоять за
достоинство свое, своих коллег, коллектива, страны и побратимов, с которыми он
служил. 

\subsubsection{Заявление "Добробута" }

В клинике "Добробут" уже потребовали от Бондаренко опровергнуть свои слова. 

Заявление сделал представитель "Добробута" в лице бота по имени Евгений
Добромен. Указывается, что слова Бондаренко были детально изучены, собраны
комментарии сотрудников и свидетелей, которые находились 8 ноября в отделении
клиники на Святошино. В итоге доказательств тому, о чем говорил экс-глава КГГА,
не нашли. 

"Заявление о физическом и словесном насилии в отношении Владимира Бондаренко
или неуважительного отношения к его вещам не соответствует действительности.
Унижение чести и достоинства людей в "Добробуте" недопустимы как к пациентам,
так и в отношении сотрудников сети", - указано в сообщении.

В клинике потребовали официального опровержения случившегося от Бондаренко,
чтобы оправдать репутацию "Добробута". В противном случае бывшему главе КГГА
пригрозили судом. И, как говорилось выше, Бондаренко свой ФБ-пост удалил. 

\ifcmt
pic https://strana.ua/img/forall/u/10/91/%D0%B1%D0%BE%D0%BD%D0%B4%D0%B0%D1%80%D0%B5%D0%BD%D0%BA%D0%BE.jpg
\fi

Такая категоричная позиция "Добробута" может говорить о том, что в правоте
своих медиков они уверены.

Также в заявлении указано, что клиника не потерпит оскорблений в том числе в
адрес сотрудников. Исходя из этого можно предположить, что замечание Владимира
Бондаренко медсестре не входило в рамки приличий и действительно было
оскорбительным. Таким, что не сдержался даже доктор, который явно не мог, судя
по своей биографии, быть большим поборником русского языка.

Данный случай не первый, когда прикрываясь законом об украинизации (который
вводит обязательность украинского языка во многие сферы, в том числе и в
медицинскую) люди откровенно унижают русскоязычных граждан. И, вероятно, не
последний. Только вчера "Страна" рассказывала, как в Днепре под увольнение
подвели профессора философии, который читал лекцию на русском языке.

И это только первые "цветочки" языковой дискриминации, которую запустил закон о
тотальной украинизации.

\subsubsection{Кто такой Владимир Бондаренко}

Владимир Бондаренко - представитель партии "Батькивщина". На местных выборах
2020 он избран депутатом Киеврады по округу в Борщаговке. 

Бондаренко был народным депутатом пяти созывов Верховной Рады. Он руководил
киевским городским штабом блока "Наша Украина", причем, именно в период
оранжевой революции. 

Во времена правления Леонида Черновицкого Бондаренко прикинул к депутатской
группе "Пора-ПРП" и принимал активное участие в его свержении с поста. 

Широко известен Бондаренко стал благодаря тому, что занимал должность главы
Киевской горадминистрации (КГГА), пусть и на протяжении короткого срока - с 7
марта по 25 июня 2014 года. 

По данным СМИ, в 90-е Бондаренко контролировал ресторанный и базарный бизнес в
Святошино. Там же находится и филиал клиники "Добробут", который он выбрал для
посещения.

В день, когда Владимир Бондаренко писал обвинения в адрес врача, он опубликовал
стихотворение Гете. Русский язык изложения его, вероятно, не смутил. Ведь
акцент был на другом - фразе "зеленый бес резвится" с намеком на Зеленского. 

\ifcmt
pic https://strana.ua/img/forall/u/10/91/%D0%A1%D0%BD%D0%B8%D0%BC%D0%BE%D0%BA_%D1%8D%D0%BA%D1%80%D0%B0%D0%BD%D0%B0_2020-11-10_%D0%B2_13.48_.06_(1).png
\fi

Бондаренко делится позицией пользователя соцсети о том, что Зеленский "на грани". 

\ifcmt
pic https://strana.ua/img/forall/u/10/91/%D0%A1%D0%BD%D0%B8%D0%BC%D0%BE%D0%BA_%D1%8D%D0%BA%D1%80%D0%B0%D0%BD%D0%B0_2020-11-10_%D0%B2_13.52_.36_.png
\fi

Русских и русскоговорящих граждан Бондаренко, судя по его постам, называет
"мокшанами". Чиновник уверен, что их "племена" были "людоедами". Исходя из
этого можно лишь догадываться, что он мог сказать медсестре, говорящей
по-русски. 

\ifcmt
pic https://strana.ua/img/forall/u/10/91/%D0%A1%D0%BD%D0%B8%D0%BC%D0%BE%D0%BA_%D1%8D%D0%BA%D1%80%D0%B0%D0%BD%D0%B0_2020-11-10_%D0%B2_13.56_.28_.png
pic https://strana.ua/img/forall/u/10/91/%D0%A1%D0%BD%D0%B8%D0%BC%D0%BE%D0%BA_%D1%8D%D0%BA%D1%80%D0%B0%D0%BD%D0%B0_2020-11-10_%D0%B2_14.00_.41_.png
\fi

У Бондаренко множество постов про Россию и ее агрессию.

\ifcmt
pic https://strana.ua/img/forall/u/10/91/%D0%A1%D0%BD%D0%B8%D0%BC%D0%BE%D0%BA_%D1%8D%D0%BA%D1%80%D0%B0%D0%BD%D0%B0_2020-11-10_%D0%B2_14.00_.56_.png
\fi

Позитивными постами на странице Владимира Бондаренко можно считать только фото
людей в вышиванках и его поздравления с православными праздниками (цитируемые
со страниц ПЦУ). А также портреты ее главы, Епифания.
