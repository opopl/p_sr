% vim: keymap=russian-jcukenwin
%%beginhead 
 
%%file 17_05_2021.fb.savenkova_faina.1.skazka
%%parent 17_05_2021
 
%%url https://www.facebook.com/permalink.php?story_fbid=304197841201155&id=100048328254371
 
%%author 
%%author_id 
%%author_url 
 
%%tags 
%%title 
 
%%endhead 

\subsection{Пока рассказывается сказка - мир живет надеждой}
\Purl{https://www.facebook.com/permalink.php?story_fbid=304197841201155&id=100048328254371}

Это эссе я написала в прошлом году для одной сирийской девочки, по просьбе
знакомого журналис.та. Но, к сожалению, оно до Сирии не дошло. Я не знаю жива
ли  она сейчас, но остается моя сказка и надежда.

\ifcmt
  pic https://scontent-iad3-2.xx.fbcdn.net/v/t1.6435-0/p526x296/186518791_304197804534492_8237688656413991603_n.jpg?_nc_cat=107&ccb=1-3&_nc_sid=730e14&_nc_ohc=b6fAAFAGweYAX98QDos&_nc_ht=scontent-iad3-2.xx&tp=6&oh=b462abc8ec978b1af7c3c7dfe7845e9c&oe=60C9CB13
\fi

Пока рассказывается сказка - мир живет надеждой 

Почти каждую ночь я выхожу на улицу и смотрю на звездное небо. Кругом тишина.
Город отдыхает в сонной неге. Лишь вдалеке слышен грохот орудий. Прислушиваясь,
ты успокаиваешься. Это далеко…

Так я живу уже шесть лет. Когда началась война я была слишком маленькая и не
думала о том, что кто-то страдает. Что ты можешь понять в 5 лет? Запомнить-то
можно, а вот именно понять и переживать из-за других – нет. Но время идет. Я
взрослею. Мне становится страшно не только за себя, но и за других.

Родители смотрят новости, где показывают такую же войну. И я понимаю, что мы
похожи с детьми в Сирии. Мы все дети одной большой войны. Мои друзья-журналисты
рассказывали, что в одном из приютов в Европе женщина-сирийка читает
детям-сиротам сказки. Я тоже пишу сказки и рассказываю их всем. Потому что я
знаю, что война вечно продолжаться не может. Так же, как дождь, который
обязательно заканчивается, а после появляется яркое солнце.

И хоть мы, дети, живущие на войне, привыкаем к взрывам и смертям, мы продолжаем
верить в сказку. Потому что волшебство не позволяет опускать руки, даря надежду
на мир. Пока только надежду. Потому что в жизнь, в отличие, от моих историй и
рассказов той сирийки, бывает несправедлива и жестока. Но все равно каждый
вечер я выхожу на улицу и смотрю на звездное небо. Я верю, что мои мечты могут
сбываться.

Возможно, в Дамаске или Алеппо есть такая же девочка, как я, верящая в сказки и
пишущая их для детей Донбасса. И как бы ей ни было страшно, она рассказывает
свою волшебную историю. И пока она звучит - мир живет надеждой.


