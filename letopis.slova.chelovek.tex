% vim: keymap=russian-jcukenwin
%%beginhead 
 
%%file slova.chelovek
%%parent slova
 
%%url 
 
%%author 
%%author_id 
%%author_url 
 
%%tags 
%%title 
 
%%endhead 
\chapter{Человек}

У диплоїдних організмів, таких, як \emph{Людина}, присутні копії майже всієї
ДНК.  Зокрема, є пари хромосом по одній від кожного з батьків, однак для
дослідників цінність представляє сума тільки однієї копії пар хромосом.
Фактично, розмір геному \emph{Людини} являє собою приблизно 6 мільярдів пар
основ, однак вчені використовують тільки половину в якості
\enquote{репрезентативної} копії.  Вчені зазначили, що 8\% геному пропустили у
2003 році не через відсутність в них важливої інформації, а через технологічні
обмеження. Дослідники наголосили, що ці обмеження вдалося подолати завдяки
секвенуванню, яке розробили вчені з Pacific Biosciences і Oxford Nanopore,
\textbf{Вчені повністю розшифрували геном людини}, day.kiev.ua, 03.06.2021

В жизни любого \emph{человека} есть, как хорошие, так и отрицательные моменты.
Существует заблуждение, что психотерапия направлена на то, чтобы сделать ее
протекание счастливым и беззаботным. Но на самом деле, специалисты в этой сфере
работают над тем, чтобы последствия испытанного стресса, страха, неудач не
мешали вам в дальнейшем. Например, если вы не можете устроиться на хорошую
работу – психотерапевт не поможет. Но, если вы боитесь ходить на собеседования,
испытываете трудности в коммуникации с новыми людьми, рушите связи из-за
конфликтного характера – вполне,
\textbf{Для чего нужна психотерапия личности?}, beztabu.net, 02.06.2021

Фаина ищет и находит простые слова, образы, сюжеты. У нее, в отличие от
современных потребителей, болит душа. Не слышать ее может лишь тот, у кого души
нет. Для кого патриотизм это не жертвенность во имя страны, а способ
обогащения, для кого война только прибыльный бизнес, для кого своя рубашка
ближе к телу, а хата с краю.  Послушайте ее обращение в ООН. Никакой
театральности, надрыва, пафоса. Она, скорее, робеет, чем кого-то обвиняет,
требует. Ей как-то неудобно, ведь она понимает, что дети не помнят мира, а
большие дяди в политике не понимают. Точнее, все эти дяди понимают, но просто
дяди негодяи, и заняты другим. А ей, Фаине, за этих дядей неловко и стыдно.
Хотя стреляют не по им, а по ней. Вот это и есть крест настоящего
\emph{Человека}, испытывать стыд за чужую подлость. Маленький \emph{Человек},
но большая душа,
\textbf{Гибель людей на Донбассе говорит о том, что наше общество больное},
Денис Жарких, strana.ua, 04.06.2021

Ведь важно не то, что плетут политики, а важно, чтобы не стреляли, чтобы \emph{Люди}
не погибали, чтобы мы опять стали \emph{Людьми}, а не чинами и капиталами. Общество
больно, его покидает \emph{Человечность}. Очевидно, прохудилась общественная совесть.
Как вы думаете?,
\textbf{Гибель людей на Донбассе говорит о том, что наше общество больное},
Денис Жарких, strana.ua, 04.06.2021


