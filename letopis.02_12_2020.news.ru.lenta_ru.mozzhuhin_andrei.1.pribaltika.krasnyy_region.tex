% vim: keymap=russian-jcukenwin
%%beginhead 
 
%%file 02_12_2020.news.ru.lenta_ru.mozzhuhin_andrei.1.pribaltika.krasnyy_region
%%parent 02_12_2020.news.ru.lenta_ru.mozzhuhin_andrei.1.pribaltika
 
%%url 
 
%%author 
%%author_id 
%%author_url 
 
%%tags 
%%title 
 
%%endhead 

\clearpage
\subsubsection{«Красный» регион}
\label{sec:02_12_2020.news.ru.lenta_ru.mozzhuhin_andrei.1.pribaltika.krasnyy_region}

\lenta{Но чем можно объяснить такую податливость и покорность? Почему политические
элиты Литвы, Латвии и Эстонии в конце 1939 года и в начале 1940 года вели себя
так вяло и пассивно?}

Во-первых, как я уже говорила, во всех трех государствах тогда были
авторитарные режимы. Но любые авторитарные режимы крайне негибки и в
критической ситуации неспособны держать удар.

Во-вторых, прибалтийские элиты после подписания пакта Молотова – Риббентропа и
особенно советско-германского договора о дружбе и границе были растеряны.
Тягостное впечатление произвела на них и Зимняя война СССР с Финляндией — во
всех трех странах решили, что для них подобное сопротивление может лишь
привести к неоправданно огромным жертвам.

\lenta{В те дни в Прибалтике действительно были многочисленные митинги и демонстрации
в поддержку действий СССР или это выдумки советской пропаганды?}

Просоветские митинги и демонстрации в июне 1940 года действительно проходили,
но все они были хорошо организованы, о чем свидетельствуют документы,
хранящиеся сейчас в РГАСПИ. Что показательно, во всех этих шествиях почти все
лозунги уже были на русском языке.

\lenta{Имелась ли в тогдашней Прибалтике социальная база для советизации?}

Имелась, причем очень широкая, о чем сейчас там тоже не любят вспоминать. Еще
со времен революции 1917 года Прибалтика считалась «красным» регионом бывшей
Российской империи. Наибольший процент голосов за большевистскую партию на
выборах во Всероссийское учредительное собрание в ноябре 1917 года дали именно
балтийские окраины. Немало рабочих и даже часть интеллигенции сочувствовали
коммунистическим идеям. Вспомним, например, про латышских стрелков, сыгравших
важную роль в ключевых российских событиях 1918 года.

\lenta{Особенно во время выступления левых эсеров в Москве 6 июля 1918 года, когда
благодаря именно латышам большевики удержали власть.}

Конечно. Более того, после этого подразделения латышских стрелков почти в
полном составе зачислили в ВЧК. Кстати, обо всем этом очень хорошо помнил
Гитлер, относившийся к прибалтийским народам с опасливым презрением. Не забыли
в Берлине и о том, как жители Прибалтики в том же 1918 году весьма активно
сопротивлялись немецкой оккупации после заключения Брестского мира.

\begin{leftbar}
\large
Если вернуться к ситуации 1940 года, то, как я уже говорила, советизация
сначала была мягкой и постепенной
\end{leftbar}

И хотя поспешно организованные выборы прошли с грубейшими нарушениями местных
избирательных законов и с явными фальсификациями, о чем в книге тоже подробно
рассказано, ни в одной из прибалтийских стран это не вызвало сколь-нибудь
серьезных протестов.

