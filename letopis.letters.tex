% vim: keymap=russian-jcukenwin
%%beginhead 
 
%%file letters
%%parent body
 
%%url 
 
%%author_id 
%%date 
 
%%tags 
%%title 
 
%%endhead 

\ii{letters.27_01_2023}
\ii{letters.06_02_2023}
\ii{letters.14_02_2023}
\ii{letters.18_02_2023}

%https://www.facebook.com/permalink.php?story_fbid=pfbid0sbMsAukawJAnZeA8fdWBvWhfxh4pusncRbwfHWN2sQYRxNnjbGn4FkQ5fDmWm4ajl&id=100087641497337

Надіє, дуже важлива книга! Мені, якщо чесно, як людині, що на щастя не бачила
наочно всього цього пекла, хоча я багато читав і читаю постів про Маріуполь, і
я відчуваю ваш біль, який так, відчуваєш, навіть якщо всього цього не бачив,
важко коментувати, оскільки подумки питаєш себе, що і як варто казати людям, що
все це пережили. Але... я мушу зізнатись в такому. Ще до війни я бачив вже
приблизно все те, що було в Маріуполі... у фільмі. Польському фільмі. Фільм
називається Місто 44 (Город 44). Страшний фільм насправді, і я в той момент
зовсім не думав, що всі ті жахіття, що були показані у тому фільмі, колись
будуть повторені вже в Україні. І цей фільм про трагедію Варшавського
повстання. Так, трагедія такого ж масштабу...  Хоча інша країна, інша історична
епоха, інші конкретні обставини. Так, у 1944 році, коли Варшава була під
німецькою окупацією, поляки підняли повстання проти нацистів, яке було в
результаті надзвичайно жорстоко придушене. Надзвичайно жорстоко. Місто було в
вогні, зруйновано, близько 100 000 поляків було в результаті вбито. Поляки
дотепер пам'ятають про це. Але... Варшава... якщо ходиш вулицями цього міста -
красивого, гарного міста... не відчуваєш, що тут колись коїлось... Так
заведено, що Міста мають внутрішню, так, внутрішню велику Силу... яка все
подолає і переможе! І так! В кінці фільму - якщо знайдете час колись подивитись
- дуже зворушливо. По-перше... (якщо подивититесь, побачите самі... - все
розказувати наперед напевне не треба... )  І по друге, що стосується власне
Міста Варшава. Кінець фільму... Палаюче знищене місто зникає... І постає
Варшава наших днів... І оцей перетворення із палаючого міста у сучасне мирне
усміхнене європейське місто колись мене надзвичайно зворушив, аж до сліз... І
так само Маріуполь. Місто залікує свої рани, так само як і маріупольці. І
колись, через десятиліття те, що відбувалось, залишиться лише у книжках та у
відео та фільмах, та пам'яті народу. Але приклад поляків дуже показовий. Вже
скільки років пройшло, а вони зняли цей фільм зовсім недавно, хоча стільки
років вже пройшло! Але вони пам'ятають, так, і дотепер, і будуть пам'ятати
завжди! 
