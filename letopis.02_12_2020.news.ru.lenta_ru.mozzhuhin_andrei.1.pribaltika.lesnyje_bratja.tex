% vim: keymap=russian-jcukenwin
%%beginhead 
 
%%file 02_12_2020.news.ru.lenta_ru.mozzhuhin_andrei.1.pribaltika.lesnyje_bratja
%%parent 02_12_2020.news.ru.lenta_ru.mozzhuhin_andrei.1.pribaltika
 
%%url 
 
%%author 
%%author_id 
%%author_url 
 
%%tags 
%%title 
 
%%endhead 
\subsubsection{Лесные братья}
\label{sec:02_12_2020.news.ru.lenta_ru.mozzhuhin_andrei.1.pribaltika.lesnyje_bratja}

\lenta{В Вильнюсе в бывшем здании управления КГБ Литовской ССР есть Центр исследования
геноцида и резистенции жителей Литвы, отделом которого является музей оккупаций
и борьбы за свободу. Судя по представленной на его сайте экспозиции, под
геноцидом там подразумевают страдания литовцев от нацистов и особенно от
Советов. А трагедии холокоста, когда литовцы под немецким руководством убивали
евреев, отведена лишь скромная небольшая комната. Но разве не об этом геноциде
в первую очередь нужно рассказывать?}

Рассказывать нужно обо всем. И делать это честно. В Латвии и Эстонии тоже есть
«музеи оккупации».

\ifcmt
tab_begin cols=3
	caption Фотографии 

	pic https://icdn.lenta.ru/images/2020/10/20/16/20201020161215148/pic_f651d7b295a83f05f46db77b79006a5c.jpg
	caption Арест евреев литовскими националистами. 1 июля 1941 г.  Фото: Wikipedia

	pic https://icdn.lenta.ru/images/2020/10/20/16/20201020161732898/pic_e86d9c9c31f6062602c714ee04873645.jpg
	caption Здание Центра исследования геноцида и резистенции жителей Литвы в Вильнюсе, в котором расположен музей оккупаций и борьбы за свободу Фото: Wikipedia

	pic https://icdn.lenta.ru/images/2020/10/20/16/20201020162010582/pic_0fc3715cfc53094ca219b750771b0935.jpg
	caption Экспозиция в музее оккупаций и борьбы за свободу в бывшей камере №3 подвальной тюрьмы, посвященная нацистской оккупации и холокосту в Литве Фото: urokiistorii.ru
tab_end
\fi

Все эти три музея объединяет то, что в них о немецкой оккупации рассказывается
очень поверхностно и мягко. Зато очень подробно, жестко и весьма односторонне
представлен весь советский период в истории этих стран, с 1940 года по 1991-й.

\lenta{В книге очень подробно рассказано о массовом коллаборационизме в Прибалтике во
время немецкой оккупации. В то же время вы подчеркиваете, что в годы Второй
мировой войны ее жители воевали и на стороне вермахта, и в составе Красной
армии. Где их было больше?}

О процентном соотношении сказать трудно, потому что всех участников боевых
действий по объективным причинам достоверно подсчитать почти невозможно. Я
отвечу так: по обе стороны фронта прибалтов было очень много. Но хочу обратить
внимание на любопытный и показательный факт: когда советские войска в 1944-1945
годах освобождали от нацистов прибалтийские столицы, первыми в них входили
именно прибалтийские формирования Красной армии. И это было принципиально важно
в идеологическом плане.

\lenta{Вы отмечаете, что широкий размах антисоветского повстанческого движения в Прибалтике после ее освобождения от немцев для Москвы оказался полной неожиданностью. Как советская власть боролась с «лесными братьями»?}

Это было сочетание жесткости с некоторыми компромиссами. Применялись
карательные методы и тактика устрашения местного населения. Аресты, депортации
и показательные процессы над повстанцами и крестьянами, подозреваемыми в помощи
им, приняли массовый характер. Особенно это касалось Литвы, где масштаб
вооруженного сопротивления советской власти был существенно больше, нежели в
других балтийских республиках. Трупы убитых «лесных братьев» нередко выставляли
на всеобщее обозрение.

\lenta{В то же время для более эффективной борьбы с антисоветским повстанческим
движением в Прибалтике власти несколько раз объявляли амнистию для тех «лесных
братьев», кто согласится сложить оружие и сдаться. Своей цели эти меры достигли
— после 1946 года вооруженное сопротивление постепенно стало затухать.}

Однако получив прощение и возможность легализоваться, многие бывшие «лесные
братья» избежали ответственности за свои преступления, совершенные как во время
войны, так и после нее. Не будем забывать, что они убивали (нередко самым
жестоким образом) не только представителей власти, но и членов их семей, не
жалея стариков и детей. У меня в книге есть воспоминания людей о зверствах,
учиняемых повстанцами: кого-то подожгли в своем доме вместе со всей семьей,
кого-то застрелили или закололи штыками.
