% vim: keymap=russian-jcukenwin
%%beginhead 
 
%%file 28_10_2021.stz.news.ua.misto_marii.1.atmosferno_provesty_ci_vyhidni
%%parent 28_10_2021
 
%%url https://mistomariupol.com.ua/uk/yak-atmosferno-provesty-ostanni-vyhidni-zhovtnya
 
%%author_id news.ua.misto_marii
%%date 
 
%%tags 
%%title Як атмосферно провести ці вихідні
 
%%endhead 
 
\subsection{Як атмосферно провести ці вихідні}
\label{sec:28_10_2021.stz.news.ua.misto_marii.1.atmosferno_provesty_ci_vyhidni}
 
\Purl{https://mistomariupol.com.ua/uk/yak-atmosferno-provesty-ostanni-vyhidni-zhovtnya}
\ifcmt
 author_begin
   author_id news.ua.misto_marii
 author_end
\fi

\ifcmt
  ig https://i2.paste.pics/PIL2O.png
  @wrap center
  @width 0.9
\fi

\begin{quote}
\em
Стрімко наближуються останні вихідні другого місяця цієї восени й традиційно
збігаються вони зі святом Гелловіну. Кельтське свято існує вже понад 2000
років, а його популярність по всьому світу тільки зростає. У Маріуполі на цих
вихідних відбудуться заходи, присвячені атмосфері Гелловіну. А якщо ви більш
консервативних поглядів, пропонуємо вам просто насолодитися жовтневим вікендом
та провести його активно та цікаво!
\end{quote}

\subsubsection{Сходити на \enquote{Містичні хіти}}

\ii{28_10_2021.stz.news.ua.misto_marii.1.atmosferno_provesty_ci_vyhidni.pic.1}

Камерний оркестр \enquote{Ренесанс} запрошує нас відсвяткувати Гелловін разом у
Маріупольській філармонії. 30 жовтня о 19:00 відбудеться концерт \enquote{Містичні
хіти} за участю солістів філармонії: Анна Іванішко (сопрано) та Олексій
Понятовський (баритон). Навіть якщо ви не є прихильником цього свята, у
концертній програмі заявлено популярні хіти з легендарних кінофільмів, а також
культові витвори світової класичної музики.

У програмі заявлено:

Ріхард Вагнер. Політ Валькірій\par
Георг Фрідріх Гендель. Сарабанда\par
Вольфганг Амадей Моцарт. Lacrimosa із Реквієму\par
Сергій Прокоф'єв. Танок лицарів\par
Нікколо Паганіні. Скрипаль диявола\par
І.Корнелюк. Майстер та Маргарита. Бал Воланда\par
Шарль Гуно. Куплети Мефістофеля\par
Notre dame de Paris. Арія\par
Карл Дженкінс Палладіо. Танго смерті\par
Мюзикл \enquote{Привид опери}. Дует Христини і Привида.\par
Едвард Гріг. У печері гірського короля. Пісня Сольвейг\par
Мелодії із кінофільму \enquote{Гаррі Потер}\par
Жак Оффенбах. Канкан.\par

У зв'язку з входженням Донецької області в \enquote{червону зону} епідемічної небезпеки
та карантинними обмеженнями у місті Маріуполі, вхід на концертні заходи у
Маріупольську камерну філармонію буде здійснюватися тільки при наявності
\enquote{зеленого} сертифіката (повна вакцинація) та збереженням маскового режиму.

\subsubsection{Вшанувати пам'ять жертв трагедії 1941 року у старовинній синагозі}

\ii{28_10_2021.stz.news.ua.misto_marii.1.atmosferno_provesty_ci_vyhidni.pic.2}

30 жовтня, у суботу, після 18:00 стіни маріупольської синагоги на Георгіївській
20, знову запалають десятками вогників. Світлові інсталяції присвячені пам'яті
жертв трагедії у селищі Агробаза під Маріуполем. Рівно 80 років пройшло зі
страшної дати, коли було розстріляно понад 16 тисяч представників єврейського
населення нашого міста. Діти, жінки, люди похилого віку... Маріуполь вшановує
пам'ять невинних жертв нацистської політики винищення та згадує Праведників
народів світу. Саме тому, до 9 листопада на стінах синагоги розміщені
інформаційні банери та інсталяції, присвячені цій трагедії. Детальніше можна
прочитати – тут\footnote{\url{https://mistomariupol.com.ua/uk/mariupol-shanuye-pamyat-80-oyi-richnyczi-tragediyi-na-agrobazi}}

У суботу вогні пам'яті запалають червоним та білим кольорами... А ми пам'ятаємо
та шануємо пам'ять всіх загиблих.

\ii{28_10_2021.stz.news.ua.misto_marii.1.atmosferno_provesty_ci_vyhidni.pic.3}

\subsubsection{Сфотографуватися з осінньою фотозоною біля перших реконструйованих старовинних дверей міста}

З 29 по 31 жовтня біля будинку з червоними дверима на вулиці Фонтанній буде
встановлена фотозона з квітів та гарбузів. Вона прикрасить нові-старі
реконструйовані двері, кошти на відновлення яких спільними зусиллями зібрали
всі активісти міста. 110-річні двері отримали друге життя та новий колір –
червоний, разом з тим, зберігши всю свою красу та стилістику. Композиція з
осіннього врожаю тільки прикрасить нову туристичну локацію. То ж приходьте на
цих вихідних на вул. Фонтанна, 44 та робіть гарні світлини на пам'ять!

\ii{28_10_2021.stz.news.ua.misto_marii.1.atmosferno_provesty_ci_vyhidni.pic.4}

\subsubsection{Скуштувати блюдо з гарбуза}

Тут вже на ваш смак – або завітати до кав'ярні чи ресторанчика, або відчути
себе кулінаром та приготувати страву з найбільш осіннього серед овочів –
гарбуза. У маріупольських закладів що восени з'являються в меню позиції страв з
гарбузом, деякі з них потім залишаються там на постійній основі.

Гуляючи улюбленим містом ми побачили чудові прикрашені вітрини деяких кав'ярень
у центрі міста. Тож надихайтеся та обов'яз\hyp{}ково приготуйте кремовий суп з
гарбузом, ароматну кашу або шумуш за рецептами маріупольських греків. Тут вже
на ваш смак та фантазію 🙂

\ii{28_10_2021.stz.news.ua.misto_marii.1.atmosferno_provesty_ci_vyhidni.pic.5}

\subsubsection{Передивитися улюблені фільми жахів}

У кінотеатр Multiplex можна прийти подивитися довгоочікувані кінострічки, за
умовами пред'явлення \enquote{зеленого} сертифікату чи чинного ПЛР-тесту. І до свята
Гелловіну нам приготували покази \enquote{Жах на вулиці Вязов} – легендарний фільм
жахів 1984 року. Покази відбудуться 28 жовтня-3 листопада о 19:00 та 21:00.
Також ще цілий тиждень триватиме показ нового фільму \enquote{Гелловін вбиває} о 23:00.

Якщо й хочете подивитися щось страшне, але й водночас кумедне – вам на мультик
\enquote{Родина Адамсів 2}. Покази проходять цілий день. Актуальні дати,
розклади та вартість квитків можна знайти на сайті кінотеатру –
тут\footnote{\url{https://multiplex.ua/ru}}

\ifcmt
  ig https://i2.paste.pics/PIPCR.png
  @wrap center
  @width 0.9
\fi

Також улюблені фільми жахів можна завжди подивитися вдома зі заздалегідь
приготовленими смаколиками 🙂

\subsubsection{Назбирати букети з листя у парках міста}

Зараз листя на деревах має настільки гарний та насичений жовто-багряний колір,
а листопад романтично супроводжує нас під час прогулянок. З листя можна
назбирати в осінні букетики, прикрасити домівку, зробити доробки або просто
насолодитися їхнім шелестінням.

У Маріуполі кожен парк та сквер має свою неповторну затишну атмосферу, час
ніби-то сповільнюється, а розмови стають все більш щирими.

\ii{28_10_2021.stz.news.ua.misto_marii.1.atmosferno_provesty_ci_vyhidni.pic.6}

\subsubsection{Скуштувати вітамінний обліпиховий чай}

Аби наша осінь була не тільки гарною, але й приносила нам тільки гарні емоції –
вживаємо більше вітамінів. Одним з найпопулярніших напоїв осіннього сезону є
обліпиховий чай. Його можна замовити у більшості кав'ярнях або приготувати
самому. Ділимося рецептом:

Потрібно взяти склянку ягід і розім'яти його половину в кашу. 

• У чайник для заварювання необхідно додати як подрібнені, так і цілі ягоди. 

• Взяти цілий лимон підігріти в мікрохвильовці й видавити з нього сік. 

• У чайник залити 0,5 окропу. 

• Почекати 10-15 хвилин, щоб напій заварилася. Після того як чай завариться, в
нього слід додати 1-2 чайні ложки меду або цукру. Після, зверху обліпихової
напою, додати зелений листовий чай і прикрасити пелюстками м'яти.

\ii{28_10_2021.stz.news.ua.misto_marii.1.atmosferno_provesty_ci_vyhidni.pic.7}
