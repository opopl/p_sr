%%beginhead 
 
%%file 31_12_2022.fb.chornoguz_jaryna.1.teper___ya_ne_v_dmov
%%parent 31_12_2022
 
%%url https://www.facebook.com/yaryna.chornohuz/posts/pfbid0RQmpPXTZfCAUjdtq1RSaQQpibS6JFwuZuFu4SafXAayhRP8PzLQWoFL6yENaZyUCl
 
%%author_id chornoguz_jaryna
%%date 31_12_2022
 
%%tags 
%%title Тепер - я не відмовляю собі в речах, які створюють відчуття дому і затишку
 
%%endhead 

\subsection{Тепер - я не відмовляю собі в речах, які створюють відчуття дому і затишку}
\label{sec:31_12_2022.fb.chornoguz_jaryna.1.teper___ya_ne_v_dmov}

\Purl{https://www.facebook.com/yaryna.chornohuz/posts/pfbid0RQmpPXTZfCAUjdtq1RSaQQpibS6JFwuZuFu4SafXAayhRP8PzLQWoFL6yENaZyUCl}
\ifcmt
 author_begin
   author_id chornoguz_jaryna
 author_end
\fi

Колись я напишу про предмети і речі, які витягують зі стану туги, коли ведеш
постійне кочове воєнне життя. Геть неепічні речі.

Попередній новий рік зустрічала у фронтовому селі на Луганщині, а цей - у таких
же умовах, лише західніше, і на Донеччині. За рік часу, якщо не згадувати про
втрати, втому, ризик, успіхи та невдачі під час завдань, то зрозуміла, як воно
- бути воєнним кочівником. Мати дім там, де випаде, і за дві доби відчувати
себе як вдома будь-де, розстеливши спальник та поставити біля себе речі, які
створюють відчуття дому. Переїздити з місця на місце, з одного села в інше, з
одного міста в інше, з одної розполаги в іншу, розуміючи, що це триватиме ще
довго, необмежено по часу. Таке дивне розуміння, що наша держава існує там, де
є скупчення наших тіл і кожен виконує свою функцію з оборони.

І якщо спочатку я намагалася бути мінімалісткою з баулом та рюкзаком, потім -
лише з тим, що було на мені, бо цей мінімум зник у реаліях окупації, то тепер -
я не відмовляю собі в речах, які створюють відчуття дому і затишку. Тому наш
евакджип, коли ми міняємо місце проживання, забитий, крім речей, з коробками
таких кочівних, зовсім невоєнних предметів. Як- от такий скляний
чайник-заварник, що нагадує мені про дім у Києві. Ароматичні палички, які
перебивають запущений запах будь-якого приміщення. У мене збереглася якимось
чудом навіть округла склянка з місяцем та зірками, яку я купила колись навесні
проїздом у Бахмуті, розуміючи цілком всю непрактичність цього предмета в
реаліях війни, коли бої ще точилися навколо Попасної і ми були там, а той
супермаркет у Бахмуті був оазисом цивілізації, тепер же він розбомблений. 

Такі речі не шкода втратити, бо мить затишку дорожча за їхню втрату в цих
реаліях. Усі вони куплені або знайдені серед непотребу. Речі з чужих покинутих
осель, якщо нема в них гострої необхідності, залишаю там, де їх поставили
господарі. 

Ці сотні місць крутяться в голові, роблячи виклик моїй пам'яті, яка намагається
притлумлювати спогади, аби свідомість концентрувалася на миті та тому, що треба
зробити. І так знаходити спокій.

Війна стала стилем життя, дико поєднуючи елементи суто воєнного, армійського та
цивільного. Його незручності однозначно вартують задоволення, яке отримуєш,
коли вдається добре виконати завдання. Знайти квадріком позицію з живою силою
ворога, якої до того не було на мапі, кинути туди підгон, навести на них
артилерію, робити все аби розвідувальна група, ризикуючи життям, знала, що в
разі поранення будь-кого з них - ми зовсім поруч, швидко за ними прибіжимо і
винесемо. А коли хтось не виживає під час евакуації - принаймні останні хвилин,
він відчував, що його не покинули, що за його життя боролися до кінця. Хоча,
відчаю від таких ситуацій байдуже, як ти пробуєш себе розрадити. Хоча, такі
довгі бойові дії роблять тебе черствим та беземоційним. Поки ти в них
перебуваєш.

Коли дають день-два відпочити, намагаюся бігати та робити інтенсивне фізо, ніби
я собі десь на мирній території дбаю про своє здоров'я та тіло. Таке примарне,
але приємне відчуття свободи.

Пробую навіть у найстрашнішому знайти щось, що дає можливість насолоджуватися
життям і не втратити мотивацію. Бо ще багато. Бо чекають, що ми виснажимося, а
ми просто зробили оборону придатною для життя. Олюдненою. 

Поки окупація рашистів цього цілком позбавлена. Дегуманізована.

