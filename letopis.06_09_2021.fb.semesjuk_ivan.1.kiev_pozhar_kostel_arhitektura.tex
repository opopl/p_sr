% vim: keymap=russian-jcukenwin
%%beginhead 
 
%%file 06_09_2021.fb.semesjuk_ivan.1.kiev_pozhar_kostel_arhitektura
%%parent 06_09_2021
 
%%url https://www.facebook.com/ivan.semesyuk/posts/4658531440864053
 
%%author_id semesjuk_ivan
%%date 
 
%%tags __sep_2021.kostel.pozhar.kiev,arhitektura,gorod,kiev,sofia_sobor,ukraina
%%title Київ - Архітектура - Пожежа в Костелі Святого Миколая
 
%%endhead 
 
\subsection{Київ - Архітектура - Пожежа в Костелі Святого Миколая}
\label{sec:06_09_2021.fb.semesjuk_ivan.1.kiev_pozhar_kostel_arhitektura}
 
\Purl{https://www.facebook.com/ivan.semesyuk/posts/4658531440864053}
\ifcmt
 author_begin
   author_id semesjuk_ivan
 author_end
\fi

Для мене "печатка цивілізації" в міському середовищі – це готична архітектура.
Хай це буде неоготика, або "псевдо" готика, втім для мене це саме так. Тому у
моєму сприйнятті пожежа в костелі святого Миколая просякнута нехорошим
символізмом. В Києві цивілізація знаходиться під прямою загрозою і я давно не
почуваюся в місті комфортно, мені ок лише поблизу історичної забудови, точніше
її залишків. 

Історична забудова сигналізує про тяглість, а отже й про смисл, що закладений в
місті. Київ давно почав виглядати значною мірою безсмислено, як і здебільшого
його мешканці. Еталоном безсмисленості (коли нічого не зміниться у світі разом
зі зникненням міста) є Кривий Ріг. Він безсмислений, він не продукує ні ясного
минулого, ні хоч якогось майбутнього. Нині Київ балансує на межі саме такого
стану. 

Порятунок душі один - час від часу прогулюватися по території святої Софії
Київської і ще кількох невеличких локацій в центрі або на Подолі, і лише тоді
починаєш ніби бачити тінь міста, тінь того міста яким воно могло бути якби не
соціалістичні експерименти включно з Другою Світовою та  будівельний терор
сьогодення. 

Київ уже на межі того щоби луснути і він це зробить – мешканці таки почнуть
самовичавлюватися у передмістя, як юшка з чиряка. Так власне й зробив я, саме
тому мого побутового навантаження основна інфраструктура столиці майже не
відчуває, принаймні надіюся на це.

Вже у 40 кілометрах від столиці починається безмежна Терра Інкогніта України –
Полісся. Порожнє, дике і наглухо відсутнє у масовій свідомості Полісся розміром
з парочку європейських країн, така собі гуманітарна пустка посеред природних
реліктів. Ми сидимо на головах одне одного, між тим як в нетрях країни
приховалася фактично ще одна країна з дійсно мікроскопічним населенням у вимірі
на один квадратний кілометр.

В Києві ж, здається, забудовники уже цілком собі готові знести святу Софію, і
зупиняє їх не закон, не міцна громада, а якась просто неуважність, чи щось
таке. Принаймні я не здивуюся якщо під нею викопають багаторівневий
торгівельний центр. Навпаки дивно, що таких проектів наче як досі нема.

Чи може є? 

Я й зараз будував би готичні архітектурні об'єкти якби мав на це вплив, бо
готика – це втілене у форму осмислене людське устремління, а саме осмисленого
устремління так не вистачає сучасним українцям. Ну, точніше сказати мешканям
України.

\ii{06_09_2021.fb.semesjuk_ivan.1.kiev_pozhar_kostel_arhitektura.cmt}
