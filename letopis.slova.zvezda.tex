% vim: keymap=russian-jcukenwin
%%beginhead 
 
%%file slova.zvezda
%%parent slova
 
%%url 
 
%%author 
%%author_id 
%%author_url 
 
%%tags 
%%title 
 
%%endhead 
\chapter{Звезда}
\label{sec:slova.zvezda}

%%%cit
%%%cit_head
%%%cit_pic
\ifcmt
  pic https://www.nkj.ru/upload/iblock/c66/c66c72df52373083c6433a2c57fd873d.jpg
	caption Красноватый дневной свет нарастал. Бетельгейзе поднималась тихо и величественно... 
	width 0.4
\fi
%%%cit_text
Наблюдая за \emph{звёздами}, астрономы, как правило, не видят их диска. Им
крайне редко удаётся следить за какими-либо процессами на поверхности и в
окрестности небесных светил, за исключением Солнца. Отсюда огромный интерес к
тем немногим \emph{звёздам}, которые выглядят не просто как яркая точка. И
когда в декабре 2019 года одна из таких \emph{звёзд} — Бетельгейзе — неожиданно
на глазах начала сильно терять свою яркость, это сразу привлекло внимание, тем
более что \emph{звезда} находится на той стадии эволюции, когда она может
превратиться в сверхновую. Возможность такого сценария сильно подогрела интерес
к поведению звезды, породив вал публикаций в прессе c предположениями, что
\emph{звезда} вот-вот взорвётся.  Разберёмся с причинами глубокого «обморока»
\emph{звезды} и вспомним, что мы знаем о Бетельгейзе и её загадках
%%%cit_comment
%%%cit_title
\url{https://www.nkj.ru/archive/articles/38479/}
%%%endcit

%%%cit
%%%cit_head
%%%cit_pic
\ifcmt
  tab_begin cols=2
     pic https://99px.ru/sstorage/56/2014/10/image_560610142019125280511.png
     pic https://99px.ru/sstorage/56/2013/06/image_562606130131075749439.jpg
  tab_end
\fi
%%%cit_text
Полная луна встала над горизонтом, гася \emph{звезды} и заливая поляну светом. Неярко
горели костры, а на траве между лагерем и дорогой танцевали маленькие существа
под резкие звуки скрипки.  После еды Сплетник развернул свой сверток и достал
оттуда скрипку и смычок.  Теперь он стоял, зажав под подбородком скрипку,
держал левой рукой лады, а правой работал смычком. Поеденная молью ворона
по-прежнему умудрялась сидеть у него на плече, подпрыгивая, чтобы удержать
равновесие, и издавая при этом протестующие крики.  Под столом спала маленькая
хромая собачка, объевшаяся мясом, которое ей бросали пирующие. Ее лапы
вздрагивали, как будто она гонялась во сне за кроликом.  — Их так много, —
сказала Мери
%%%cit_comment
%%%cit_title
\citTitle{Зачарованное паломничество}, Клиффорд Саймак
%%%endcit

%%%cit
%%%cit_head
%%%cit_pic
\ifcmt
  tab_begin cols=3
     pic https://www.rosimperija.info/wp-content/uploads/2012/10/plehanov-georgij-valentinovich.jpg
		 pic https://udipedia.net/wp-content/uploads/Vladimir-Lenin.jpg
		 pic https://img-fotki.yandex.ru/get/6735/161887320.af/0_19d61c_223385e6_orig.jpg
  tab_end
\fi
%%%cit_text
Занималась медицинской практикой мама до начала первой мировой войны, когда
наша семья со старшими братом и сестрой (я еще не появился на свет, хотя был
уже ожидаем) переехала в Москву. В восемнадцатом году она пошла на службу
военным врачом в больницу Красной армии, и в голодные годы мы перебивались
благодаря ее пайку. Она охотно рассказывала про студенческие годы в Берне, где
делила малюсенькую квартирку с двумя другими русскими студентками, питаясь
почти исключительно хлебом с маслом, чаем и колбасой в течение шести лет. Берн
в те годы, не менее Женевы, кишел русскими революционерами всякого рода,
которые старались, каждый по-своему, возбудить революционную сознательность
русских студентов. Ей случалось бывать на выступлениях таких \emph{звезд} марксизма,
как Плеханов, Ленин и Троцкий. Она была в те дни хорошо знакома с Зиновьевым,
знакомства с которым она потом никогда не возобновляла
%%%cit_comment
%%%cit_title
\citTitle{Физик, физик, где ты был}, Анатоль Абрагам
%%%endcit

%%%cit
%%%cit_head
%%%cit_pic
\ifcmt
  tab_begin cols=3
		 pic https://img-fotki.yandex.ru/get/25826/57766531.22/0_14bf84_a1e0388c_orig.jpg
     pic https://mobimg.b-cdn.net/v3/fetch/59/59b7098cd23a4838d0a8974008b16429.jpeg
		 pic https://avatars.mds.yandex.net/i?id=b6d915658c7fafa2a9e44b615d22ae4d-5232281-images-thumbs&n=13
  tab_end
\fi
%%%cit_text
Первый Оратор давно ухе перестал говорить для Студента. Он говорил сам для
себя, как бы переживая все заново, стоя у окна и глядя на далекие \emph{звезды}
той части огромной Галактики, которая теперь навсегда была в безопасности.  —
Гари Селдон сказал про Трантор — «там, где кончаются \emph{звезды}», —
прошептал он, — и каким поэтическим воображением надо было обладать, чтобы
сказать так. Всей вселенной когда-то управляли с этой планеты, все пути со
\emph{звезд} вели сюда. «Все дороги ведут на Трантор» — говорит старинная
пословица. И здесь же кончаются \emph{звезды}!  Всего десятью месяцами раньше
Первый Оратор смотрел на те же самые \emph{звезды}, которые нигде так ярко не
сияют, как в центре Галактики, с тревогой.  Но сейчас уже не волнение, а
торжество отражалось на круглом красном лице Прима Палвера — Первого Оратора
%%%cit_comment
%%%cit_title
\citTitle{Основание}, Айзек Азимов
%%%endcit
