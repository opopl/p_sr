% vim: keymap=russian-jcukenwin
%%beginhead 
 
%%file 09_03_2018.fb.fb_group.750_hudozhnykiv.1.sostav_zaitsa.cmt
%%parent 09_03_2018.fb.fb_group.750_hudozhnykiv.1.sostav_zaitsa
 
%%url 
 
%%author_id 
%%date 
 
%%tags 
%%title 
 
%%endhead 

\qqSecCmt

\iusr{Violett Obryvko}

Девочки и мальчики... В этом году мы получили менее вредный вариант (исходя из
моего случая), чем прошлогодний ...

Материал - стеклополимер какой-то...... Исходный материал - пенополистирол...
Писала неоднократно... Промойте в горячей воде... Запах уйдёт.. Тут из серии
"кому не нравится - я не виновата"... Или работаем, или... не работаем... Можно
подумать никто эмалью ПФ не дышал.. Тут, по ходу, многим вообще не дали
заготовки.... Так что... Дышите глубже.

\begin{itemize} % {
\iusr{Елизавета Крестьянкина}

Наші арт-об'єкти виготовлені зі склопластику, як і минулого року. Оскільки вони
одразу з виробництва потрапляють до вас, тому мають запах. Помийте теплої водою
та на сутки на балкон. І все буде добре. Форма кролика дуже складна, тому вони
довго виготовляються. І ми відразу до вас їх направляємо.

\iusr{Янішевська Юлія}
\textbf{Violett Obryvko}, 

можно конечно и клеем моментом дышать и эмалью ПФ и т.д. Каждый выбирает сам,
но здоровье важнее. Очень важно, какие используют компоненты для изготовления
зайца, как это скажется на здоровье!!!

\iusr{Violett Obryvko}
\textbf{Юлия Борисова} 

на здоровье и выхлопные газы сказываются... И ГМО... и вырубка лесов...

Пентан, ещё не самое страшное)))

\iusr{Янішевська Юлія}

По фото вижу, что ты получила яйцо. Яйцо не пахнет, если бы ты получила зайца,
то говорила бы совсем по другому. Эмаль, лаки и всякая хрень просто
отдыхает.)))))

\iusr{Violett Obryvko}

Та в курсе я, как оно пахнет...))) жить можно)) =

\iusr{Елизавета Крестьянкина}

Кролик та писанка виготовлені з одного матеріалу.

\iusr{Olga Mashevskaya}
\textbf{Елизавета Крестьянкина} 

работала два года с закрытой формой яйца, более и менее. А вот кролик-открыт, и
запах очень сильный, от него болит голова. Мыла, сейчас на балконе был сутки.
Занесла, через 20 минут вынесла. Работать тяжело. Буду покупать распиратор

\iusr{Ольга Черьомушкіна}
\textbf{Юлия Борисова} готова прийняти вашого зайця у свої обійми!)))

\iusr{Eugenia Anishyna}

А мы просто помыли водой и 2-е суток на балконе стоял. Сейчас порисовала 1,5
часа и снова на балкон. И все ок. Не стоит просто дно его нюхать..

А респиратор который защищает от химического воздействия- серьезно? Вы готовы
отдать за него 300 грн ?)

\iusr{Ольга Гальчинська}

Казав мій знайомий скульптор, що так пахне полієфірка

\iusr{Янішевська Юлія}

Если даже предположить, что изделия из пенополистирола, как говорит Violetta
(анализ никто не делал) то ни в коем случае его нельзя мыть в горячей воде, как
советует \href{https://www.facebook.com/profile.php?id=100010536826533}{Violett
Obryvko}. Вред пенополистирола при воздействии на него высоких температур – ещё
один важный аспект. В этом случае выделяются токсичные вещества: пары стирола,
бензола, оксида углерода, двуокись углерода и сажа. Выделяется стирол уже при
температуре 25°С.

\iusr{Violett Obryvko}
\textbf{Юлия Борисова} 

прям, капец какой-то... Ну, нету у вас такой температуры воды, чтоб прям это
всё повыделялось. Вы ж не жечь его будете!

\iusr{Violett Obryvko}

И стирола там уже нету в чистом виде!

\iusr{Eugenia Anishyna}

Так пенополистирол - это исходный материал. В нем форму делали. А сам кроль уже
сделан из стекловолокна

\iusr{Eugenia Anishyna}

Вариации пенополистирола. У нас никак не из такого кроли !

\ifcmt
  igc https://scontent-fra5-2.xx.fbcdn.net/v/t31.18172-8/28828315_1593542817394443_8556657949008609097_o.jpg?_nc_cat=107&ccb=1-7&_nc_sid=dbeb18&_nc_ohc=vevPO5EPgBEAX-49Jtn&_nc_ht=scontent-fra5-2.xx&oh=00_AfB6VCvYwAwODo4NigcMpUNjrmBYLpFlKNHATsg3zM9wvA&oe=64748B70
	@width 0.4
\fi

\iusr{Violett Obryvko}

\ifcmt
  igc https://scontent-frt3-2.xx.fbcdn.net/v/t31.18172-8/28698955_614284145599474_3378772490154734534_o.jpg?_nc_cat=108&ccb=1-7&_nc_sid=dbeb18&_nc_ohc=9HtbNsM5HVsAX8yjq0u&_nc_ht=scontent-frt3-2.xx&oh=00_AfAGKcIeSCG4vlA_r7kyGnQ_gJkVol_roz7n-8r9Muya3g&oe=6474BBDA
	@width 0.4
\fi

\iusr{Ольга Гальчинська}

Це полієфір

\iusr{Violett Obryvko}

\ifcmt
  igc https://scontent-fra3-1.xx.fbcdn.net/v/t31.18172-8/28828061_614284522266103_6252151752294914517_o.jpg?_nc_cat=105&ccb=1-7&_nc_sid=dbeb18&_nc_ohc=McdEcx6_yf0AX-wBkEV&_nc_ht=scontent-fra3-1.xx&oh=00_AfAXEtgmzSfDIaJ2VfRXFKQreQTMhoTorhkkYC1Xzbwspw&oe=6474BD4B
	@width 0.4
\fi

\iusr{Ольга Гальчинська}

\url{https://prom.ua/ua/p8080376-smola-poliefirnaya-zhidkij.html}

\ifcmt
  igc https://i2.paste.pics/5d8374ae45381d1d7b2b96ef7f39ac23.png
	@width 0.8
\fi

\iusr{Ольга Гальчинська}

\ifcmt
  igc https://scontent-fra5-1.xx.fbcdn.net/v/t1.6435-9/29025698_1858995844173679_872006497142308864_n.jpg?_nc_cat=102&ccb=1-7&_nc_sid=dbeb18&_nc_ohc=oUeJCXCp7L8AX8wrf5H&_nc_oc=AQkovkoPSnrX5pv-QzYPX5GYKfPyIv0KqAb2XZhbxxh52FOqDIHHXEdBDhFZvfbxvYM&_nc_ht=scontent-fra5-1.xx&oh=00_AfB_kpHDFti1-nDyaEss5PUvvGbUonkbY4eTfj0dwEln_w&oe=6474A79C
	@width 0.4
\fi

\iusr{Янішевська Юлія}
\textbf{Eugenia Anishyna}, конечно готовы отдать за респиратор и 300, 400 и даже 500грн- лечение обойдется дороже)))

\iusr{Olga Mashevskaya}
\textbf{Violett Obryvko} они могут))))

\iusr{Елизавета Крестьянкина}

До речі, кролі, що катаються укрпоштою дуже гарно провітрюються і не пахнуть.
Вони пахнуть коли що з виробництва до вас ідуть, як свіжі булочки з духовки)

\iusr{Янішевська Юлія}

Вот это сравнение)))))

\iusr{Елизавета Крестьянкина}

\ifcmt
  igc https://i2.paste.pics/fa4c0bbf908487db402cba010eecd63f.png
	@width 0.3
\fi

\iusr{Eugenia Anishyna}

А мы попробовали внутри напшикать лаком для волос. Действует 😁

\iusr{Ольга Гальчинська}

О, спасибо

\iusr{Elena Vergeles}
\textbf{Eugenia Anishyna} теперь аромат разнообразился?) Пластик и лак - два в одном)))

\iusr{Eugenia Anishyna}
\textbf{Elena Vergeles} у меня идеально все. Ничего не воняет )

\end{itemize} % }

\iusr{Anna Kshanovska-Orlova}

повод работать быстрее)) я еще не получила зайца, но надеюсь на след. неделе 🙂

\begin{itemize} % {
\iusr{Yana Khachikyan}

Вот это точно! Я немного быстренько порисую - и закрываю его в другой комнате с
открытым окном ))

\end{itemize} % }

\iusr{Marina Kholodniak}

Брр... я тепер дуже налякана прочитавши це.. А якщо в мене немовля і я годую
манюню грудним молочком -це безпечно для дитини і мами?

\begin{itemize} % {
\iusr{Елизавета Крестьянкина}

Друзі, я прошу не загострювати ситуацію. Запах долається теплої водою. Ніякої
небезпеки арт-об'єкти не несуть. Помийте його та на воздух. Все буде добре

\iusr{Тетяна Горбенко}
\textbf{Елизавета Крестьянкина} спасибі )) помиємо і на повітря...

\iusr{Eugenia Anishyna}

Та нормально все ) У вас ребенок на прогулку дышит гадостью страшнее...

\iusr{Olga Mashevskaya}
\textbf{Елизавета Крестьянкина} 

це не загострення, а доречні запитання, на які організатори мають відповідати.
Щоб не кожному то можно одним постом. Те, що ми працюємо безоплатно-це не
означає, що нам всеодно, аби помалювати. Питання доречні, і нічого страшного в
них не має.

\end{itemize} % }

\iusr{Тетяна Горбенко}

Лізонько, Сертифікат безпечності матеріалу має бути в виробника і в договорі
записаний...і з середини \enquote{Кролика} приклеєним має бути! Цей матеріал
токсичний, і треба щось придумати🐰 поки що сховала \enquote{кролика} на балкон ... от
тепер думаю, як же з ним працювати?? На балконі холодно ще❄️❄️❄️

\iusr{Ольга Гальчинська}

Я торік грунтувала, і після грунту запах зник
