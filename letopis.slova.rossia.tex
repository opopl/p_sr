% vim: keymap=russian-jcukenwin
%%beginhead 
 
%%file slova.rossia
%%parent slova
 
%%url 
 
%%author 
%%author_id 
%%author_url 
 
%%tags 
%%title 
 
%%endhead 

Новости Крымнаша. Предатели в Крыму пришли к выводу, что \emph{россияне} — хуже
глистов, obozrevatel.com, 24.05.2021

Луганский академический театр кукол совместно с коллегами из \emph{РФ} и ДНР
провел онлайн-выставку рисунков юных зрителей, посвященную Дню славянской
письменности и культуры, lug-info.com, 24.05.2021

\enquote{Луганский академический театр кукол совместно с Брянским областным
театром кукол и Горловским городским театром кукол организовал онлайн-выставку
детских рисунков \enquote{Мы - славяне!}, посвященную Дню славянской
письменности и культуры, который ежегодно широко отмечается в \emph{России} и
других странах 24 мая}, - говорится в сообщении, lug-info.com, 24.05.2021.

Сейчас мы переживаем время не менее судьбоносное. И от того, вспомнит ли народ
свои изначальные культурные коды, зависит судьба наша в новом веке – судьба не
только \emph{России}, но и всей человеческой цивилизации, vz.ru, 24.05.2021

С этого времени \emph{Русь} ощущает себя уже не просто некоей «срединной
землей» между Востоком и Западом, Азией и Европой, но неким особым миром между
землей и небом, между началом и концом истории. И здесь, между небом и землей,
началом и концом истории, начинает она искать свои берега: «Открылась бездна,
звезд полна. Звездам числа нет, бездне – дна», vz.ru, 24.05.2021

Свободная и богатая за счет транзитной торговли, \emph{Русь} переживает бурный
рост, на пике которого и является «Слово о законе и благодати» (между 1037 и
1050 годами) митрополита Илариона – первое слово \emph{русского} самосознания,
сказанное первым этнически \emph{русским} митрополитом, vz.ru, 24.05.2021

Не просто Україна, а Україна-\emph{Русь}. Як Михайло Грушевський захищає
державні кордони?  radiosvoboda.org, 23.05.2021

А вот принципиального различия между укладами современной \emph{Россией} и
Украиной нет. Тут, как в Крыму или на Донбассе - были украинцы сверху, стали
\emph{русские}, а по сути, так примерно одни и те же люди. Народу, по большому
счету, начхать, сегодня \emph{русские} записываются в украинцы, завтра украинцы
запишутся в \emph{русские} - делов-то, \textbf{Дело Протасевича - еще один
повод раздуть Холодную войну}, Денис Жарких, strana.ua, 24.05.2021

Ця традиція іде ще від апостола Андрія – дві тисячі років тому він першим
провістив сакральну природу Києва. У ХІ столітті у знаменитій промові «Слово
про закон і благодать» перший \emph{руський} митрополит Іларіон возвеличив Київ
як «город святий всеславний», radiosvoboda.org, 09.10.2018
