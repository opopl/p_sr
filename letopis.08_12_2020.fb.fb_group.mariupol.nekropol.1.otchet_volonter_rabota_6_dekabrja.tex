%%beginhead 
 
%%file 08_12_2020.fb.fb_group.mariupol.nekropol.1.otchet_volonter_rabota_6_dekabrja
%%parent 08_12_2020
 
%%url https://www.facebook.com/groups/278185963354519/posts/428340738339040
 
%%author_id fb_group.mariupol.nekropol,arximisto
%%date 08_12_2020
 
%%tags 
%%title Отчет о волонтерской работе в Некрополе 6 декабря 2020
 
%%endhead 

\subsection{Отчет о волонтерской работе в Некрополе 6 декабря 2020}
\label{sec:08_12_2020.fb.fb_group.mariupol.nekropol.1.otchet_volonter_rabota_6_dekabrja}
 
\Purl{https://www.facebook.com/groups/278185963354519/posts/428340738339040}
\ifcmt
 author_begin
   author_id fb_group.mariupol.nekropol,arximisto
 author_end
\fi

\textbf{Отчет о волонтерской работе в Некрополе 6 декабря 2020}

В прошлое воскресенье мы поступили опрометчиво: дул сильный ветер, было
холодно, но мы таки пришли в Некрополь. Пришлось работать, чтобы согреться 🙂

\textbf{Открытия и находки}

Недалеко от центра Некрополя мы сделали уникальную находку – надгробие польской
семьи Опацких! Сколько раз проходили мимо, но кругом были заросли. Заросли
убрали и – вернули из забвения целую семью.

Надпись на надгробии на польском. Удалось прочитать имена Анны и Варвары. Даты
отсутствуют. Поэтому точное выяснение имен требует поисков в архивных
документах. Об их предварительных результатах – в отдельной публикации
директора \enquote{Архи-Города} Андрея Марусова.

Если вы знаете о ныне живущих мариупольцах с фамилией \enquote{Опацкие} - пожалуйста,
сообщите!

Там же в центре обнаружили памятник жандармского подполковника Василия
Сосновского. Как выяснилось, его уже находили краеведы несколько лет назад.
Подробности о подполковнике – тоже в отдельной публикации.

\textbf{Благоустройство}

Полдня мы выпиливали заросли недалеко от центра Некрополя и древнего участка
Гофов-Хараджаевых. Под конец работы бензопила в очередной раз сломалась – по
всей видимости, в кнопке включения \enquote{отошли} электрические контакты. Попытаемся
отремонтировать.

Большое спасибо за самоотверженный труд Maryna Holovnova, Елена Сугак,
Александр и Наталия Шпотаковская, Андрей Клепиков!

\textbf{Использование пожертвований}

В прошлую субботу мы закупили по 5 литров масла для двигателя и цепи бензопилы,
перчатки, а также растворитель на общую сумму 1044 грн. (см. копию чека).

Таким образом, на сегодня осталось 3 809 грн. из пожертвованных 6 250 грн. Мы
планируем до Нового года докупить еще краску, мелкий инвентарь, масло для цепи
(оно расходуется очень быстро), бензин, агроволокно...

\textbf{Дальнейшие планы}

К сожалению, отныне реализация наших планов полностью зависит от погоды. Мороз
и сильный ветер – не самые лучшие условия для обследования или очистки
Некрополя от зарослей. Как только потеплеет – мы возобновляем работу...

\#mariupol\_necropolis\_report

\#mariupol\_necropolis\_spending
