% vim: keymap=russian-jcukenwin
%%beginhead 
 
%%file slova.buduschee
%%parent slova
 
%%url 
 
%%author 
%%author_id 
%%author_url 
 
%%tags 
%%title 
 
%%endhead 
\chapter{Будущее}
\label{sec:slova.buduschee}

%%%cit
%%%cit_head
%%%cit_pic
%%%cit_text
У россиян нет ничего, кроме вранья. Вся их \enquote{история} это политические
нарративы, симулякры, чаще всего прикрывающие коричнево-кровавые пятна на
российской совести. Гомо-советикусу дают нужную ему картинку: красавец человек,
победитель всего зла на земле. Гомо-советикус должен гордиться собой,
любоваться, особенно каплями крови, тянущимися за ним.  А еще, если впереди нет
\emph{будущего}, то гомо-советикус должен иметь хотя бы светлое прошлое, желательно
историческое, желательно история должна быть старше и величавее, чем у всех,
ну, и еще человечнее и вся такая мимимишная. Не показывать же людям истинное
лицо каннибалов, снохачей
%%%cit_comment
%%%cit_title
\citTitle{Страна без исторических корней всегда будет оккупантом}, 
Олена Степова, news.obozrevatel.com, 17.06.2021
%%%endcit

%%%cit
%%%cit_head
%%%cit_pic
%%%cit_text
Это подлежит анализу и переосмыслению. З цим пантеоном героїв, не в смысле как
героев, а в смысле тех ментальных структур, которые они на себе несут, у
Украины \emph{будущего} нет. У Украины нет \emph{будущего}, если Бандера будет элементом
мыслительных структур. У Украины нет \emph{будущего}, если Петлюра будет элементом
мыслительных структур. Понимаете? От Доній каже: «Треба розширювати пантеон».
Ну не розширюється він. Понимаете? Тут надо выбирать: или Скоропадский и
Болбочан, как мощнейший военноначальник, або Петлюра
%%%cit_comment
%%%cit_title
\citTitle{Сергей Дацюк: Украина сегодня  - не просто попрошайка, она на мусорнике истории}, 
Сергей Дацюк; Людмила Немыря, hvylya.net, 28.06.2021
%%%endcit

