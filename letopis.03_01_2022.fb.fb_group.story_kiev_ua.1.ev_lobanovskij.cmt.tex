% vim: keymap=russian-jcukenwin
%%beginhead 
 
%%file 03_01_2022.fb.fb_group.story_kiev_ua.1.ev_lobanovskij.cmt
%%parent 03_01_2022.fb.fb_group.story_kiev_ua.1.ev_lobanovskij
 
%%url 
 
%%author_id 
%%date 
 
%%tags 
%%title 
 
%%endhead 
\zzSecCmt

\begin{itemize} % {
\iusr{Кучеревский Геннадий}
 @igg{fbicon.thumb.up.yellow} 

\iusr{Лариса Максименко}

\iusr{Наталия Вигерина}
Спасибо за интересный рассказ! С Новым Годом!

\iusr{Сергій Криськов}

И Вам, спасибо за оценку и, С Новым Годом!

\iusr{Тарас Єрмашов}
\enquote{... Евгений Васильевич – тренера, т.е. своего брата}. Тобто, вони були зовнішньо схожі?

\iusr{Сергій Криськов}

Та подивіться ж на фото! Звісно, схожі. Але для художньої самодіяльності це не
має значення.

\iusr{Valentina Urban}

Я прекрасно помню здание этого Института, который был построен где-то то в
середине 60- х. Я жила рядом в пятиэтажке и чтобы попасть на ул.Щорса
приходилось всегда проходить мимо этого здания.

С Новым годом, спасибо за интересный рассказ!

\iusr{Лариса Лобановська}

Хочу добавить, что Лобановский проработал там практически всю жизнь, с 18 лет
до самой смерти в 66... А институт был очень дружный и веселый, народ как
приходил туда после учебы, обычно в пищевом, так и оставался навсегда, поэтому
все друг друга прекрасно знали и дружили. Плюс неплохие зарплаты, довольно
быстро можно было получить квартиру (по тогдашним меркам), ну и, конечно, Куба!
На Кубе перебывали почти все сотрудники, а это чеки, машины... Я не знаю, как
сейчас, но еще лет 10 назад институт работал, хоть и в очень урезанном
виде,чего нельзя сказать о подавляющем большинстве проектных институтов.

\begin{itemize} % {
\iusr{Сергій Криськов}
\textbf{Лариса Лобановська} 

Лариса, спасибо и с Новым годом Вас! Теперь это частная лавочка, 44 сотрудника,
а название сохранилось: Укргіпроцукор. У них есть сайт.

\iusr{Лариса Лобановська}
\textbf{Сергій Криськов} 

это не совсем частная лавочка, еще в 90-е, как на всех предприятиях, институт
приватизировали сотрудники. Потом в году, наверное, 2006-2007 была неприглядная
история с рейдерским захватом института, но институт все-таки отстояли. Как
раз, я так понимаю, нынешний директор Шкабара Е. сумел таки сохранить институт.
Хотя что там сейчас, честно, не знаю.

\iusr{Алла Федоренко}
\textbf{Сергій Криськов} 

Скажите, пожалуйста, в каком году Вы пришли на Укркинохронику и в каком отделе
Вы работали, это не любопытство, просто я и мой муж работали на студии, муж
ассистентом оператора, я техн.редактор отдела готовой продукции, в ноябре 2021
года студии исполнилось 90 лет, созванивались, хотели встретиться, нас очень
мало осталось, к сожалению, но, увы, начался новый виток ковид и встречу
пришлось отложить. Если Вам не трудно, напишите, пожалуйста, будем ждать.
Спасибо. Пусть Новый год принесет всем здоровья и добра!

@igg{fbicon.heart.red}{repeat=2} @igg{fbicon.hands.pray}{repeat=3} 
\end{itemize} % }

\iusr{Володимир Довгуник}

Цікаво!!! Такого з Вашої біографії я не знав, а от про кіностудію знаю. Дякую.
Горжусь що був Вашим Колегою по отельні.

\end{itemize} % }
