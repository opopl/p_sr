% vim: keymap=russian-jcukenwin
%%beginhead 
 
%%file 10_02_2022.fb.fb_group.story_kiev_ua.1.pasternak
%%parent 10_02_2022
 
%%url https://www.facebook.com/groups/story.kiev.ua/posts/1858367064360086
 
%%author_id fb_group.story_kiev_ua,levickaja_natasha.kiev
%%date 
 
%%tags kiev,pamjat,pasternak_boris
%%title Вспоминаем сегодня Бориса Пастернака...
 
%%endhead 
 
\subsection{Вспоминаем сегодня Бориса Пастернака...}
\label{sec:10_02_2022.fb.fb_group.story_kiev_ua.1.pasternak}
 
\Purl{https://www.facebook.com/groups/story.kiev.ua/posts/1858367064360086}
\ifcmt
 author_begin
   author_id fb_group.story_kiev_ua,levickaja_natasha.kiev
 author_end
\fi

Вспоминаем сегодня Бориса Пастернака...

\obeycr
Ты здесь, мы в воздухе одном.
Твое присутствие, как город,
Как тихий Киев за окном,
Который в зной лучей обернут,
Который спит, не опочив,
И сном борим, но не поборот,
Срывает с шеи кирпичи,
Как потный чесучовый ворот,
В котором, пропотев листвой
От взятых только что препятствий,
На побежденной мостовой
Устало тополя толпятся.
Ты вся, как мысль, что этот Днепр
В зеленой коже рвов и стежек,
Как жалобная книга недр
Для наших записей расхожих.
Твое присутствие, как зов
За полдень поскорей усесться
И, перечтя его с азов,
Вписать в него твое соседство.
1931 Киев
\restorecr

10 февраля 1890 г родился Борис Леонидович Пастернак.

В детстве он мечтал стать композитором, сочинял и импровизировал на фортепиано.
В юности хотел быть философом, брал уроки у немецкого неокантиста Германа
Когена. Но судьба распорядилась иначе - Борис Пастернак стал писателем.

\ii{10_02_2022.fb.fb_group.story_kiev_ua.1.pasternak.pic.1}

С Киевом у Бориса Леонидовича связан небольшой период жизни - это  ирпенский
дачный сезон 1930 года и дважды он  ненадолго приезжал в 1931-м. Однако,
благодаря этому появилась одна из лучших страниц в антологии стихотворений о
Киеве.

Ирпень, ул. Пушкинская 13 - адрес, по которому Борис Пастернак  отдыхал с женой
Евгенией Владимировной, сыном Женей,  с братом и друзьями.

Летний отдых Пастернака в обществе родных и друзей подарил ему неожиданный
творческий подъем -\enquote{...мне давно, давно уже не работалось как там, в Ирпене...
Написал я своего Медного всадника, скромного, серого, но цельного и, кажется,
настоящего}.

Самым заметным событием того сезона, связанным с Киевом, был групповой выезд на
концерт Генриха Нейгауза,  который по всем источникам состоялся 15 августа 1930
года. Нейгауз в сопровождении местного оркестра исполнял на открытой площадке
нынешнего Крещатого парка концерт Шопена  и концерт Листа.

В 2008 году в Киеве по адресу Чапаева 9, ныне Липинского, на доме, где Борис
Пастернак останавливался у своего друга профессора Киевского университета
Евгения Перлина, была установлена памятная доска!

P.S. Вношу поправку... В комментариях мне подсказали - оказывается, памятной
доски Борису Пастернаку уже нет, ее сняли в 2015 году вместе с доской ещё
одному человеку с мировым именем - Владимиру Горовицу...

Простите нас, Борис Леонидович! Не ведают, что творят...

\ii{10_02_2022.fb.fb_group.story_kiev_ua.1.pasternak.cmt}
