% vim: keymap=russian-jcukenwin
%%beginhead 
 
%%file 03_02_2021.fb.loza_vladislav.1.pismo_v_zaschity_bilchenko_vchena_rada
%%parent 03_02_2021
 
%%url https://www.facebook.com/vladyslavloza/posts/1646848002186322
 
%%author Лоза, Владислав
%%author_id loza_vladislav
%%author_url 
 
%%tags bilchenko_evgenia
%%title Відкритий студентський лист на захист професора Більченко
 
%%endhead 
 
\subsection{Відкритий студентський лист на захист професора Більченко}
\label{sec:03_02_2021.fb.loza_vladislav.1.pismo_v_zaschity_bilchenko_vchena_rada}
\Purl{https://www.facebook.com/vladyslavloza/posts/1646848002186322}
\ifcmt
 author_begin
   author_id loza_vladislav
 author_end
\fi

Вчена рада факультету філософії та суспільствознавства НПУ ім. Драгоманова
провела засідання у справі професора Більченко. Рішення: тимчасово відсторонити
від викладання без зняття з посади. У справу залучать невідомих
\enquote{експертів} та \enquote{представників держави}, які проаналізують
дискурс Більченко на предмет ідеологічної крамоли. Таким чином адміністрація
університету прагне перекласти відповідальність за остаточне рішення на третю
сторону. 

Звільнити \enquote{по-тихому} не вийшло: надто сильним виявився спротив
студентів та контр-кампанія у медіа.

З одного боку, можна це розуміти як тактичну перемогу, з іншого – як
бюрократичну хитрість, яка унеможливлює пряме втручання адвоката й дає нашим
опонентам час для перегрупування та вироблення менш сміховинної позиції. 

На Вченій раді нам вдалося, попри шалений спротив адміністрації, озвучити наш
Відкритий студентський лист на захист професора Більченко, який зібрав 44
підписи (а не 25, як озвучували опоненти під час засідання). Вважаємо перемогою
вже той факт, що наша позиція прозвучала, нехай і в обстановці шизофренічного
залякування і кафкіанського структурного абсурду. 

Аргументи доносчиків від професури містили: безграмотні та поверхові
квазі-методологічні звинувачення пост-єльцинських шизоїдних бабусь, які на
лекціях говорять про походження культури від колонізації Землі рептилоїдами, у
тому, що Більченко поєднує науку та ідеологічні тренди; закиди щодо
невідповідності її наукових підходів \enquote{паспорту спеціальності} з боку
представників академічної спільноти, для яких \enquote{паспортна} культурологія
зводиться до феноменології кока-коли у серіалі \enquote{Як я зустрів вашу маму};
заклики втрутитись і \enquote{прошерстити}, адресовані представникам спецслужб, із
подальшим перерахуванням статей Кримінального та Адміністративного кодексу, які
нібито були розтоптані; сентиментальні звинувачення у нехтуванні мораллю та
професійною етикою й у нападі на сакральну символіку постмайданних режимів;
слізне апелювання до травматичної пам'яті столітньої та тисячолітньої давнини,
строго у рамках націоналістичного наративу; довгі інтелектуалізовані виступи з
посиланнями на Гусерля, де доводилось, що \enquote{критика тоталітарних тенденцій
ліберал-демократії} та \enquote{політична пропаганда, неприпустима для професора} – не
одне й те ж. 

Протягом усього засідання нам, студентським активістам, які
прийшли зачитати документ на захист свого академічного наставника, посилено
намагались стулити пельку на підставі внутрішніх нормативів засідання. 

У розпалі дискусії у відповідь на моє цитування рядків Галича проти цькування
Пастернака (\enquote{Мы поимённо вспомним всех, кто поднял руку}) якийсь
професор істерично заволав, щоб йому не погрожували. Кілька викладачів,
посилаючись на травматичний досвід своїх предків, які \enquote{пройшли Сибір},
погрожували звільненням, якщо Більченко лишиться в університеті (мабуть,
огидний сентиментальний донос в рамках панівної ідеології на боці більшості
став для них способом солідаризації з подвигом предків-нонконформістів).

Результати голосування: 1 проти, 1 утримався, решта (не згадаю числа) за.
Зразкове поєднання класичної та посткласичної тоталітарної логіки: зовнішня
форма партійного зібрання (з притаманними їй одностайністю та репресуванням
альтернативних голосів), внутрішній зміст шизоїдного постмодерного дискурсу (із
м'якою стигматизацією альтернативних голосів як \enquote{некомпетентних} та
\enquote{ідеологічних}). Знавці антитерористичного й карного законодавства,
пост-єльциністи, СБУшні феноменологи, креативщики, малоросійська номенклатура й
травматизовані бандерівці проти академічної науки в Україні – що це, клінічний
портрет неоліберального університету чи бестіарій кінця історії на наших сумних
теренах?

На наступний день відбулась дискусія між активістами та адміністрацією. Нашу
позицію можна підсумувати зверненням, яке було зачитане на цій зустрічі: 

"Ви керуєтесь бажанням захистити і зберегти факультет. Це благородне прагнення
керівництва, і ми його поважаємо. Проте в даній ситуації єдине, що можна
зробити для захисту факультету – це захистити Більченко. Будь які компромісні
рішення розв'язують руки відверто темним силам, чия мета – дестабілізація і
розвал суспільства. В рамках класичної тоталітарної логіки можна було
розрізнити прапорці, за які не можна переступати. 

Ваша помилка полягає в тому,
що вам здається, ніби ці прапорці зараз існують. Мовляв, принесемо Більченко в
жертву, щоб задобрити злих демонів. На жаль, ми всі зараз серед цих прапорців.

Неможливо переплисти за буйки, коли буйки всюди. Якщо ви демонструєте
слабкість, ви потураєте злу. Наступним можете бути ви або будь хто з
факультету. Будь-що може бути витлумачене як антиукраїнська позиція і звернене
проти вас і всіх нас. Хто наступний? Охранительство не врятує нікого, лише
переконає колективного стерненка у власній безкарності. Якщо ви приймаєте
рішення на користь факультету, захистіть Більченко, захистіть нас. Не потурайте
невігласам і злочинцям."

Наша позиція почута не була: адміністрація лишилась при своєму
\enquote{поміркованому} рішенні (яке, з одного боку, може запустити маховик
подальших чисток з боку радикалів, з іншого – звужує нам простір для
маневрування, підвішуючи ситуацію в повітрі). Натомість, нас, багаторазово
перекручуючи нашу позицію та маніпулюючи фактами, звинуватили у \enquote{лівому
радикалізмі} та \enquote{розколюванні факультету}. А ще терапевтично закликали
не \enquote{підбурювати} студентів, не розхитувати ці \enquote{незміцнілі уми}. 

Отже, ситуація застигла в невизначеності. Опонентам: спротив буде
продовжуватись, доки нашого вчителя повністю не реабілітують. Адміністрації:
якщо реабілітації не відбудеться, ми запустимо усі ті сценарії, які згадано у
нашому Відкритому листі, в якості останніх заходів проти академічної цензури.
На цьому поки все.

\begin{verbatim}
  #стопцькуваннюБЖ  #наукабезцензури
\end{verbatim}

\begin{itemize}
\iusr{Nataly Pasichnyk}

робіть позов у суд

\iusr{Влад Лоза}

Nataly Pasichnyk це не так просто, оскільки ніхто нікого не звільнив. Рішення
\enquote{призупинити} викладання, скоріш за все, буде юридично бездоганно оформлене. З
чим іти в суд конкретно проти університету (не беручи до уваги погрози, наклепи
та іншу уголовщину, за які установа \enquote{відповідальності не несе})?

\iusr{Nataly Pasichnyk}

Влад Лоза ну призупинити означає відсторонити від викладання на невизначений
термін. зарплату виплачуватимуть? якщо ні, то треба ще й робити лінгвістичну
експертизу і лекцій, і постів.

\iusr{Nataly Pasichnyk}

Влад Лоза іти проти незаконного відсторонення від викладання на підставі
написання приватного допису у фб.

\iusr{Nataly Pasichnyk}

Влад Лоза найважливіше - подавати все сухо і без емоцій, по закону.

\iusr{Влад Лоза}

Nataly Pasichnyk зарплату платять начебто. дякую за думку. передам Вашу пропозицію правозахисникам

\iusr{Nataly Pasichnyk}
Влад Лоза о, так це - канікули)

\iusr{Владислав Сушков}

Не исключено, что нагрузка будет просто сдвинута на более поздний срок.
Преподаватель университета в этом более свободен, нежели школьный учитель.

\iusr{Vladislav Petrenko}

Про малоросійську номенклатуру це певний ідеологічний маркер. Думаю що не варто
використовувати риторику опонентів. А так цілком згоден. Шкода тільки що інші
університети не зробили своїх заяв.

\iusr{Влад Лоза}

Vladislav Petrenko тут все - ідеологічні маркери. + вся медіа-артикуляція
тримається на апропріації риторики опонентів. інакше виходить, що треба діяти
за логікою \enquote{якщо означник вже юзнули, то він зашкварений}, і ми
\enquote{не маємо на нього права.} Так можна зовсім без означників лишитись, з
огляду на те, що боротьба розгортається у шизофренічній площині)

\iusr{Vladislav Petrenko}

Влад Лоза він просто методологічно не вірний. І семантично пов'язаний з малою
Грецією. Думаю що варто використовувати власну риторику. Але це не суть
важливо.

\iusr{Влад Лоза}

Vladislav Petrenko дякую, що просвітили, я ж думав, що Ruthenia Minor придумав
Путін) Ви плутаєте медійне перекидання мемами і етимологічний диспут. В теорії
я ЗА те, щоб повертати правильно артикульований означник \enquote{Мала Русь} у дискурс
про українську ідентичність, бо він апелює до привілейованого права на
цивілізаційну спадщину Русі. Але використовується націоналістами, з одного
боку, як маркер стигматизації, а рос. імперіалістами, з іншого боку - як маркер
імперської провінційної периферії. І я писав про це не раз. Але, боюся, нашим
опонентам байдужі такі філологічні тонкощі, як байдужі їм і реальні проекти
інклюзивної укр. ідентичності. Тож на даному етапі тактично доцільним є
сократичне \enquote{обертання} маркеру опонента проти нього самого - в дусі
звинувачення опонента у невідповідності своєму власному дискурсу, витісненні
того, чим є він сам. Я подивився б, як би Ви повели у цій шизофренії виважений
етимологічний диспут, із методологічно вивіреним використанням "власних"(?)
означників.

\iusr{Vladislav Petrenko}

Влад Лоза 1 я не збирався вас просвіщати. 2 я з вами вже погодився 3 я не думаю
що дискусія можлива. Судячи з коментарями під вашими дописати. 4 не будемо
\enquote{мірятися} ліпше солідаризуватися. 5 з терористами і фанатиками перемовин не
ведуть ви і самі це знаєте. 6 успіху вам у вашій справі.

\iusr{Влад Лоза}
Владислав Петренко дякую

\iusr{Stanislav Tykhonko}

А які сценарії ви запустите, якщо Більченко не реабілітують?

\iusr{Alexey Nedviga}

Бестиарий

\iusr{Евгения Финчук}

Я хоть и с другого факультета, но тоже с вами! На самом деле много тех, кто
против её увольнения только из-за её политической позиции, просто молчат..
боятся высказывать. Мои с моего факультета тоже пытались пристыдить за
поддержку Евгении, хотя явно же понимают что она верные вещи говорит!

\iusr{Сергій Іванович Запоржанин}

взагалі раджу не дочікуватися висновків невідь-кого щодо змісту лекцій, а
замовити таку ж саму експертизу паралельно, хоча би лінгвістичну у київському
лінгвістичном, зібрати висновки експертів (у програм же є рецензенти) про
програми та лекції і все) а те, що депутат від слуги народу у відео одразу
сказав \enquote{що у лекціях щось не те} - ще один доказ, що розбір лекції
медиком-фрілансером замовили, може і слуги.

\begin{itemize}
\iusr{Влад Лоза}

Сергій Іванович Запоржанин дякую

\iusr{Влад Лоза}

Сергій Іванович Запоржанин а про якого депутата і про яке відео йдеться? можете
сюди скинути, якщо під рукою?

\iusr{Сергій Іванович Запоржанин}

Влад Лоза те, що в Жені на стіні, найновіше, з каналу НАШ, послухайте його уважно до кінця)

\iusr{Влад Лоза}

Сергій Іванович Запоржанин на 2.34 вона сама про це згадує, що могло дати слузі
зачіпку. Тобто він все ж не проговорився перший, а зловив на слові. хоча ви
праві, без попереднього наміру і знання контексту таке питання не поставиш, це
секундна омовка

\iusr{Сергій Іванович Запоржанин}

Влад Лоза авторка аналізу в одному з коментарів зізналася, що цей аналіз їй, як
фрілансеру-копірайтеру, замовили. не від щирого серця вона це робила, і ще
зазначала, що її заробіток чесний. мені здається, було би варто зробити аналіз
тієї ж лекції, де розбити в пух і прах її, притягнуті за вуха тези. звісно, що
Женю замовили політики, а не бідні педагоги. Лише в політика є можливість
профінансувати цілу кампанію булінгу, включно із спеціальним відео-випуском із
Стерненком. Гадаю, що спонсор Стерненка і є замовником.

\iusr{Влад Лоза}

Сергій Іванович Запоржанин погоджуюсь, що це доцільно. Всі ваші зауваги я
передав, ідея з експертизою дискутується. Але сама ситуація поки надто
підвішена. Все здійснюватиметься вже при підготовці юридичної позиції для суду.
\enquote{Аналіз} ви маєте на увазі зробити в рамках лінгвістичної/фахової експертизи,
чи постом у фейсбуку? якщо друге, то Більченко вже спростовувала всі ці речі
публічно: там на цих аргументах ніде клейма ставити. Плюс подібні \enquote{розбори}
давно тиражуються, критика фрілансерки вже далеко не єдина, і це буде
примножуватись. Відповідати на кожен такий закид не вийде. Вони зрозуміли, що
по мовному питанню і по закону нічим крити, значить, треба перевести цькування
у вимір профпридатності.

\iusr{Сергій Іванович Запоржанин}

Влад Лоза Звісно, що все має бути офіційним і закріпленим печаткою. І навіть
звернення до преза має бути офіційним, щоб вам дали офіційну відповідь. Він не
зобов'язаний реагувати на кожен відеозапис, але мусить відреагувати на офіційне
звернення. Адресу його офісу можна дізнатися на сайті :)

\iusr{Сергій Іванович Запоржанин}

Влад Лоза всі програми, які зробила професор БІльченко, наскільки я розумію,
відрецензовані докторами наук, треба підняти ці рецензії, це вже полегшить
роботу. Крім того, як виявилося, по цих програмах читають інші викладачі. Хай і
вони напишуть відгуки, що програми - хороші. Вони ж не писатимуть самі проти
себе

\iusr{Сергій Іванович Запоржанин}

Влад Лоза просто з програмами зрікатися колеги вийде важче, бо ж доктори наук
підтвердили і неодноразово, що все чікі-пікі)

\iusr{Сергій Іванович Запоржанин}

а аналізувати можна лише ті лекції, які є в записі, все решта - необгрунтований
донос, тому експертизу треба робити усіх відеоматеріалів. це затратно, але по
факту)

\iusr{Влад Лоза}

Сергій Іванович Запоржанин затратність - один із факторів. Путін нас фінансує
за залишковим принципом) + ще той факт, що кафедри більшості київських
університетів, де таке можна замовити, окуповані нациками. Треба ретельно
підібрати заклад.

\iusr{Сергій Іванович Запоржанин}

Влад Лоза ці рецензії були написані вже давно, їх просто треба підняти,
потворної експертизи програм не потрібно. треба робити аналіз лише відеолекцій.

\iusr{Сергій Іванович Запоржанин}

Влад Лоза щодо мовної екпертизи, так її замовляють при різних кримінальних
справах, думаю, таких спеців є багато.

\end{itemize}

\iusr{Sergey Titsky}

Пане Лоза! Вам не страшно учиться на факультете, ученый совет которого
составляют, по вашим словам, шизофреники, кафкисты-асурдисты и шизоиды
(\enquote{в обстановці шизофренічного залякування і кафкіанського структурного
абсурду}, \enquote{безграмотні та поверхові квазі-методологічні звинувачення
пост-єльцинських шизоїдних бабусь}), гнобящие \enquote{гениального ученого с
мировым именем} и его не менее гениальных неофитов-учеников?...

\end{itemize}
