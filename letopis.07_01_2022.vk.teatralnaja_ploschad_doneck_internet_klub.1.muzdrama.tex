% vim: keymap=russian-jcukenwin
%%beginhead 
 
%%file 07_01_2022.vk.teatralnaja_ploschad_doneck_internet_klub.1.muzdrama
%%parent 07_01_2022
 
%%url https://vk.com/theatreclubdonetsk?w=wall-125010602_29395
 
%%author_id teatralnaja_ploschad_doneck_internet_klub
%%date 
 
%%tags kultura,donbass,doneck,dnr
%%title Муздрама, я ваша навеки! 
 
%%endhead 
\subsection{Муздрама, я ваша навеки!}
\label{sec:07_01_2022.vk.teatralnaja_ploschad_doneck_internet_klub.1.muzdrama}

\Purl{https://vk.com/theatreclubdonetsk?w=wall-125010602_29395}
\ifcmt
 author_begin
   author_id teatralnaja_ploschad_doneck_internet_klub
 author_end
\fi

От нашей участницы Юлии БУЛАВИНОЙ

Муздрама, я ваша навеки!

Если спросить, какой мой любимый мультфильм, то я, не задумываясь, отвечу -
цикл \enquote{Смех и горе у Бела моря}. Ох, уж эти Апельсин, Волшебное кольцо,
Перепилиха и Мистер Пронька - смотрю запоем. (Есть, конечно, горькая новелла о
двух братьях-моряках, но ее обычно не транслировали - уж сильно грустная.)
Потому, когда в декабрьской афише увидела премьеру \enquote{Волшебного кольца},
возликовала и поблагодарила Утонувшего Бога. Честно, я давно хотела увидеть эту
искрометную сказку, но она казалась мне довольно трудной технически для
воссоздания.

\ii{07_01_2022.vk.teatralnaja_ploschad_doneck_internet_klub.1.muzdrama.pic.1}

И как же она оказалась хороша и чудесна. Весела и интересна взрослым, ведь я
ходила с папой! Смеялся папуля не меньше меня (тем более, некоторые из артистов
были его студентами). Понравилось абсолютно все от и до :).

Начнем с братьев наших меньших. Кошка Манюня, ну до чего же мила, гибка и с
хитринкой. И шапочка с ушками - костюмерам отдельные роза, поклон-поклон.
Дружок - умереть не встать - моря обаяния, задора, а этот кувырок через плечо
хозяина. Смотришь на Юлию Шафир и Дмитрия Федорова и порой забываешь, что это
не реальные зверушки, а очень талантливые молодые актеры. Скарапея идеально
влилась в компанию - пластичная, царственная. А быть может, она и вовсе не
змея, а Просто-Мария - девушка-анимаг? Кстати, про анимагов.

Ваня, иди сюда, учить магичить буду. Вингардиум Левиоса - с кольцами не
работает. Тут уж лучше - Искрис фронтис (всем привет от Тани Гроттер и
Чумы-дель-Торт, может, кто помнит?  @igg{fbicon.wink} ). Что-то меня в магию
занесло. А так Ваня славный парень! И Матушка у него - вооот такая
@igg{fbicon.thumb.up.yellow}{repeat=5} !

Топаем по мосту и знакомимся с царской семьей. Ульянка - дЭвушка современная, в
трендах шарит, хороша! А вот и Царь, с его местным аналогом \enquote{Льда} из
валирийской стали. Помню, в \enquote{День мечты} Евгений Попов приятно удивил
глубоким бархатным голосом. В Кольце звучит совсем иначе - вот что значит
актер, актерищеее! Ну, и главная вишенка - Царь-Матушка. Женщина боевая, с
характером.  Жаль только, фирменную фразу - \enquote{Что я, одичала что ли!?},
не сказала. Но и без того было много фишек.

Жаль, что эта сказка оказалась финальной в трилогии. Вообще из этого материала
хорошо бы вышел сборник фильмов для детей.

В заключении хочу сказать. Я не люблю громогласные фразы \enquote{спектакль с
русской душой}. Но тут это высказывание именно в тему! Спасибо!

\#ИнтернетКлубТеатральнаяПлощадь \#отНашегоУчастникаТП \#муздрамаДонецк
\#ВолшебноеКольцо \#ИгорьРыбчинский \#ЮлияБулавина

\begin{itemize} % {
\iusr{Galina Gordeeva}

Да, Муздрама это вечный праздник@igg{fbicon.heart.red}{repeat=3} Низкий поклон
режиссёру Игорь Рыбчинскому, актёрам, люблю всех. Многие лета Вам служить
любимому делу и восхищать нас, зрителей. @igg{fbicon.rose}{repeat=3} 

\iusr{Natalya Oxentyuk}
Вы дарите нам праздник
\end{itemize} % }
