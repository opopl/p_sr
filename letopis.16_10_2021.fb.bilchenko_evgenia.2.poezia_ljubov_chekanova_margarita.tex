% vim: keymap=russian-jcukenwin
%%beginhead 
 
%%file 16_10_2021.fb.bilchenko_evgenia.2.poezia_ljubov_chekanova_margarita
%%parent 16_10_2021
 
%%url https://www.facebook.com/yevzhik/posts/4349795958388814
 
%%author_id bilchenko_evgenia
%%date 
 
%%tags bilchenko_evgenia,chekunova_margarita.rossia.poet,ljubov,poezia,rossia
%%title БЖ. Молодые русские поэты о Любви
 
%%endhead 
 
\subsection{БЖ. Молодые русские поэты о Любви}
\label{sec:16_10_2021.fb.bilchenko_evgenia.2.poezia_ljubov_chekanova_margarita}
 
\Purl{https://www.facebook.com/yevzhik/posts/4349795958388814}
\ifcmt
 author_begin
   author_id bilchenko_evgenia
 author_end
\fi

БЖ. Молодые русские поэты о Любви

\ifcmt
  ig https://scontent-lga3-1.xx.fbcdn.net/v/t1.6435-9/246006041_4349795918388818_513773489588829477_n.jpg?_nc_cat=100&ccb=1-5&_nc_sid=8bfeb9&_nc_ohc=cMZ6hj6cRr4AX-CgBjj&_nc_ht=scontent-lga3-1.xx&oh=e234cd73cedda993ced34ac43dfee720&oe=619230FC
  @width 0.4
  %@wrap \parpic[r]
  @wrap \InsertBoxR{0}
\fi

Я продолжаю вас знакомить с творчеством молодого поэта Маргариты Чекуновой
(Сибирь - СПб), и мне кажется, что это - очень интересное стихотворение именно
вот о Любви. Здесь собрана вразброс и дерзко вся Русская классическая Традиция,
вложена она в энергичную рэп-тонику и посему расцвела заново в модерном нашем
контексте. Это то, что я ищу для дееспособности Логоса. Иллюстрацией взяла тоже
неслучайного человека. Французский уличный художник Кристиан Гуеми, известный
под псевдонимом С215, продает репродукции своей картины, чтобы помочь
французским врачам. Он написал полотно - «Любовь во время коронавируса» на
стене одного из зданий в городе Иври-сюр-Сен в 5 км к юго-востоку от Парижа. А
по-русски - вот так:

В городе N
.
Здесь никто не читал, не бродил киосками
в жажде истин, запрятанных в книжных сполохах.
Простакова ждет очереди до осени 
на ближайший курорт и ругает окрестных олухов.
Простаков наливает в стакан фильтрованную,
с корвалолом смешанную и мутную. 
У Коробочки гости. Коробочка покороблена:
«столько лишних расходов», –  
«а хочется не минутного», – размышляет Софья, 
сбегает тропинкой округа. 
Кто-то родной преследует злостным окриком. 
Вещи собраны, платье от ливня мокрое.
 «Только не бред, – говорит себе, – только не обморок».
За сокрытые в сумке листы изящные, 
вольнодумцем созданные, античные,
матушка в крик: «в нашем городе так нельзя, мол, нам
 это – крайняя степень странности, неприличности».
Ну а Софья? А Софья сбежит к Кулигину
через улицу с пьяным, поля гречихи.
«Пусть же, гнëт, негодница, свою линию», –
говорят Простакова и Кабаниха.
А Кулигин глядит в телескоп. В привычности
спят Дикие, Ноздревы и их сподручники.
Почему дети самых жестоких язычников
порой становились мученики?
Надоело смотреть, как фонарь таращится:
решил, что луч света: горд и доволен.
Софья плачет. Кулигин достал инструмент из ящика:
«Не грусти, – говорит, – я создам нам
perpetuum mobile».
Маргарита Чекунова

\ii{16_10_2021.fb.bilchenko_evgenia.2.poezia_ljubov_chekanova_margarita.cmt}
