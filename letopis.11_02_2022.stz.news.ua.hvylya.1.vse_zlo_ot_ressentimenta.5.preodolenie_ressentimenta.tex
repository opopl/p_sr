% vim: keymap=russian-jcukenwin
%%beginhead 
 
%%file 11_02_2022.stz.news.ua.hvylya.1.vse_zlo_ot_ressentimenta.5.preodolenie_ressentimenta
%%parent 11_02_2022.stz.news.ua.hvylya.1.vse_zlo_ot_ressentimenta
 
%%url 
 
%%author_id 
%%date 
 
%%tags 
%%title 
 
%%endhead 

\subsubsection{Преодоление Ресентимента}
\label{sec:11_02_2022.stz.news.ua.hvylya.1.vse_zlo_ot_ressentimenta.5.preodolenie_ressentimenta}

Есть разные способы преодоления Ресентимента. У древних греков, как это
описывает Агамбен («Стазис. Гражданская война как политическая парадигма»,
2015), Стазис как внутренняя гражданская война в противоположность Полемосу как
внешней войне означал обязательное участие всех на той или иной стороне,
разрешение противоречия через новый гражданский мир и Амнесию и Амнезию
(прощение и забвение). В то же время это был очень затратный по отношению к
жертвам способ. Более человечным преодолением Ресентимента стало христианство.

Христианство впервые последовательно и ярко уличает иудейский Ресентимент и
пытается преодолеть его рефлексивными нарративами прощения и милосердия.
Иудаизм плохо воспринял антиресентиментальные наставления христианства. Ислам
значительно использовал Ресентимент для своего продвижения, породив явление
исламского терроризма. Лучше всего антиресентиментальные установки выражены в
индуизме (ахимса), буддизме, даосизме и конфуцианстве. В то же время
христианство плохо добивается преодоления Ресентимента из-за ограничений
психической свободы.

Еще один более точечный способ борьбы с Ресентиментом, который массово
практикуют в Европе и США, — инфраструктура психоанализа и фармакотерапия
релаксантами и антидепрессантами. Такой способ позволяет избегать Ресентимента
в его основании через предупреждение вытеснений, отрицаний, проекций. Однако
при этом ресентиментал становится психотерапевтическим человеком.

Современный Ресентимент гораздо сложнее, чем тот, с которым сталкивались
Древняя Греция, христианство или даже психоанализ. Современные консциентальные
войны суть проявление Ресентимента. Умный может кого-то оглуплять или
примитивизировать, пока не одуреет или не упростится сам. Сложные смыслы не
борются с простыми: они поглощают их. Со сложными могут бороться только простые
смыслы. Но война простых смыслов со сложными еще больше разрушает простые
смыслы.

Ресентимент в принципе не преодолевается психоанализом, как это неудачно
пытался делать Делез, потому что потенциальный пациент не просто не считает
себя больным, а, наоборот, считает больными самих психоаналитиков. Как бы
хорошо было психоанализом лечить всех коммунистов, националистов, эйджистов,
расистов и феминисток. Но — нельзя.

Ресентимент это не просто психотравма, а многомерное фундаментальное упрощение,
изменяющее мышление, волю, веру и патию, которое получает множество
психо-социальных подкреплений и становится основой идентичности и социальных
паттернов поведения, которое может простираться сколь угодно глубоко и широко —
вплоть до духовности индивидов; культуры и ментальности нации и цивилизации;
истории, коллективной памяти и политики.

Ресентимент манипулятивно извне можно использовать, корректировать, усиливать,
ослаблять, потому что увидеть эти воздействия ресентиментал не может.
Ресентименталами можно легко управлять, провоцируя маркерные темы, доводя
коммуникацию до конфронтации (45% — за, 45% — против, только 10% — способны
оставаться нейтральными). Для Украины, например, такой маркерной темой
Ресентимента является «язык». Для западных университетов такими маркерными
темами является «признание вины и своих привилегий» в отношении черных,
ЛГБТ-сообществ, женщин, мусульман и т.д.

Только добро может находить в себе зло и избывать его. Зло не видит зла, зло
никогда не признает себя злом. Вопреки Ханне Арендт — зло не банально: оно
скрывается, лжет, скрывает мотивы, оправдывается, манипулирует, то есть выдает
себя за добро, а добро выдает за зло. Но зло всегда выдает себя тем, что
практикует, поддерживает или разрешает Ресентимент.

Ресентиментальные теории преодолеваются исторической практикой. XX век показал,
что если из марксизма выбросить классовый Ресентимент, то, собственно, в
марксизме, кроме концепции капитала, ничего стоящего не останется, потому что
будет хабермасовский коммуникативизм и делиберативная демократия, даосский или
буддистский коммунизм. А если из национализма выбросить весь националистический
Ресентимент, что сущностное в нем останется? И вообще насколько нация —
сущность? Если из феминизма выбросить весь Ресентимент, что сущностное в нем
останется? Особенно если учесть, что феминизм это уже не о поле, а о гендере?

В оперативном настоящем Ресентимент почти никогда не преодолевается мышлением
или теориями, потому что он отключает мышление и всякую теоретизацию.
Ресентимент не преодолевается патией, потому что она превращается в вырожденный
эмоциональный интеллект. Ресентимент не преодолевается волей, потому что
паразитирует на ней. Ресентимент не преодолевается верой, потому что блокирует
ее трансцендентный горизонт, сосредотачиваясь на имманентных ее аспектах.

Избавиться от Ресентимента можно, изживая его, то есть, когда он заводит в
тупик обессмысливания и бесперспективности, а также надоедает своей
навязчивостью и примитивизмом. Выход из Ресентимента опасен для жизни
ресентиментала так же, как и вхождение в него или нахождение в нем.

Гуманитарное преодоление Ресентимента очевидно имеет договорную природу с
равновесными принципами. Всякая акцентируемая идентичность ресентиментальна и
сводится к группизму. Лишь сложное самоопределение и договоры между сложными
самоопределениями могут быть нересентиментальными и негрупповыми.

Рессентименту противостоит положительный психический суверенитет. В этом смысле
психический суверенитет — это не анархия и не социопатия. Психический
суверенитет — это способность к независимому самопреобразованию, к свободному
или даже спонтанному мышлению, к самопроявлению патии без эмоционального
диктата или принуждения, к необусловленной творческой воле и вере широкого
горизонта. Психический суверенитет — это когда у меня нет никакой вины, никаких
привилегий, никаких обязанностей, о которых я ни с кем не договорился или
добровольно не признал без ущерба собственному духовному преображению.

С точки зрения психического суверенитета, любое ограничение сознания, например
классовое, национальное, расовое, гендерное, эйджистское и т.п. сознание — это
извращенное сознание, несвободное сознание. Не бывает отдельного национального
или отдельного классового и т.д. достоинства, ибо достоинство, как и
справедливость, есть только многомерно.

Стоит ли бороться с Ресентиментом? Если это не превращается в фашизм или
массовую агрессию и существует на индивидуальном уровне, то нет. Если кто-то
хочет жить с блокированным мышлением, с искривленной несамостной патией, с
обусловленной волей, с верой суженного горизонта, а также с мыслевирусами и
ментальными закладками, то это его право. Но когда это становится общей бедой,
то бороться с Ресентиментом нужно обязательно.

Борьба с Ресентиментом не должна быть еще одним Ресентиментом. Ситуацию борьбы
упрощает одно очень интересное обстоятельство: ресентименталов не нужно гнать
или разоблачать, они сами себя проявляют, сами заходят в тупик своей агрессии
и, в конце концов, сами отрекаются от бывших друзей, особенно тех, кто имеет
психический суверенитет. С ресентименталом надо вести себя, как с малыми
детьми: жалеть, прощать, наказывать, наставлять, давать шанс измениться и снова
жалеть, прощать...

Тем не менее, почему Ресентимент так опасен? Потому как это не просто
негативная энергия, направленная на раскол и противостояние в обществе. Это
самоуничтожение многих талантливых людей, которые не научены приобретать и
защищать свой положительный психический суверенитет. Более того,
распространяющийся сегодня по всему миру цифровой фашизм направлен на
уничтожение психического суверенитета. Свободное мышление, необусловленная
воля, самостоятельная патия, вера широкого горизонта – сегодня это единственные
привилегии.

Но это было всегда. Не так ли?
