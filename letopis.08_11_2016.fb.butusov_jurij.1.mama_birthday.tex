% vim: keymap=russian-jcukenwin
%%beginhead 
 
%%file 08_11_2016.fb.butusov_jurij.1.mama_birthday
%%parent 08_11_2016
 
%%url https://www.facebook.com/butusov.yuriy/posts/1400055683368057
 
%%author_id butusov_jurij
%%date 
 
%%tags birthday,butusov_jurij,mater,materinstvo,semja,ukraina
%%title Сегодня 8 ноября день рождения моей мамы - Ирины Анатольевны Бутусовой
 
%%endhead 
 
\subsection{Сегодня 8 ноября день рождения моей мамы - Ирины Анатольевны Бутусовой}
\label{sec:08_11_2016.fb.butusov_jurij.1.mama_birthday}
 
\Purl{https://www.facebook.com/butusov.yuriy/posts/1400055683368057}
\ifcmt
 author_begin
   author_id butusov_jurij
 author_end
\fi

Сегодня 8 ноября день рождения моей мамы - Ирины Анатольевны Бутусовой. Она
родилась в 1947-м году. Мама сделала меня тем, кто я есть. Она была настоящим
ученым-исследователем, кандидатом биологических наук. Мама выбрала карьеру
однажды и на всю жизнь - поступила на вечерний факультет биологии в Киевский
универ, параллельно устроилась лаборантом в институт физиологии имени
Богомольца и вся ее карьера прошла там. 

\ifcmt
  tab_begin cols=4

     pic https://scontent-mxp1-1.xx.fbcdn.net/v/t31.18172-8/14976328_1399986433374982_1475985329117041_o.jpg?_nc_cat=111&ccb=1-5&_nc_sid=730e14&_nc_ohc=bBa7RAaMTi0AX-R0o7k&_nc_ht=scontent-mxp1-1.xx&oh=c1f15d31cae0348044b848747fcb75f2&oe=61AEC48F

     pic https://scontent-mxp1-1.xx.fbcdn.net/v/t31.18172-8/15003273_1400017620038530_5514348242435832382_o.jpg?_nc_cat=106&ccb=1-5&_nc_sid=730e14&_nc_ohc=jhQ4mZ9fHeUAX_-D-RO&tn=lCYVFeHcTIAFcAzi&_nc_ht=scontent-mxp1-1.xx&oh=4a2c951806a7214b09a31ceea52e535c&oe=61AE90D5

     pic https://scontent-mxp1-1.xx.fbcdn.net/v/t31.18172-8/15039626_1400017703371855_1879148823712812697_o.jpg?_nc_cat=106&ccb=1-5&_nc_sid=730e14&_nc_ohc=lxsitvha77EAX8MaJl-&_nc_ht=scontent-mxp1-1.xx&oh=d2024b106a32886db26e30261e5e9850&oe=61AF5702

     pic https://scontent-mxp1-1.xx.fbcdn.net/v/t31.18172-8/14976789_1400017786705180_283408123102412228_o.jpg?_nc_cat=108&ccb=1-5&_nc_sid=730e14&_nc_ohc=AbSKLfewZlwAX90arQl&_nc_ht=scontent-mxp1-1.xx&oh=e435d4db42e00d812a5d9ea7adba9a9d&oe=61B0B4DA

  tab_end
\fi

Я еще не родился. а в 1975-м мама уже написала первую научную работу о
свойствах воды "Нафтуся". Она работала в отделе по изучению механизмов
физиологического воздействия  минеральных вод. Только недавно, из публикации в
журнале Академии наук Украины, я узнал, что мама первой в Украине освоила
радиоимунный метод изучения гормонов и влияния воды на организм. Я в детстве не
понимал, почему  моей самой доброй и ласковой маме в мире дают молоко "за
вредность", а оказывается, метод исследования состоял в том, что она изучала
влияние воды на животных путем введения в организм специальных
маркеров-радионуклидов, и молоко давали за работу с этими опасными веществами.
Благодаря внедрению радиоимунного метода изучения удалось детально установить
как именно действует вода в организме. Мама была наверное одним из лучших
специалистов по "Нафтусе", и могла рассказывать часами о свойствах этой воды и
как она влияет на человека - причем большинство знаний она черпала не из книг,
а сама открыла, изучила и доказала.

Мама успевала вести научные исследования, при этом тянуть всю семью, уделять
нам огромное время, при этом успевать отстоять во всех очередях и все, что
возможно где-то достать, сделать со мной домашние задания, почитать новые
книги, которые у нас закупались даже при полном отсутствии денег на продукты, и
еще пообщаться с подругами - такими же учеными института. До моего рождения она
еще успевала выступать солисткой одного из ансамблей, любила петь всю жизнь, но
я был уже вторым ребенком и она пожертвовала ради меня своим увлечением,
отказалась от всех выступлений. 

Только сейчас понимаю, как мне повезло в детстве попасть в окружение настоящих
целеустремленных людей, настоящую научную среду людей высокой культуры и
образования. Людмила Стельмах, Владимир и Ольга Прокопенко, многие другие. У
мамы вообще было много друзей, она была очень коммуникабельный человек. Я
раньше не понимал, какую мотивацию она находила для фундаментальной науки -
ведь даже тогда я понимал, что ее результаты не очевидны, мама часто
рассказывала, как много приписок, как много

имитации научной деятельности существует.  А сейчас уже понимаю, что она была в
коллективе таких же одержимых наукой людей, у которых были свои принципы и свои
отношения, которые ценили друг друга по качеству реальной работы, а не
выдуманных фантазий. После развала СССР ряд ее друзей уехали в зарубежные
научные центры, приглашали и маму...

Я с детства запомнил как мама постоянно что-то делала. Помню, как я смотрел
"Международную панораму" году эдак в  81-м-82-м, мне 5 или 6 лет,  репортаж о
жизни в Америке, и советский журналист-международник показывал, что капитализм
- это плохо, показывал большие дома, автомобили, бытовую технику, но чтобы это
получить американский рабочий класс мучается, капиталисты его эксплуатируют,
надо, мол,  тяжело работать, а посудомоечные машины при этом часто ломается,
пару месяцев поработает, и надо в ремонт сдавать. Я не мог понять, зачем же
рабочие покупают поломанные вещи, если они им таким трудом достаются, и спросил
маму. Она сказала: "Работа и у нас тяжелая, а вот посудомоечной машины - какой
угодно - в доме не хватает" (ведь в СССР такие машины не продавались).Я сказал
словами диктора: "А если машина сломается через два месяца?" А мама ответила:
"Да если бы и два месяца, сынок, сколько бы времени я бы смогла отдохнуть".  Я
это запомнил на всю жизнь почему-то - столкновение пропаганды и реальности.

Мама находилась в движении круглые сутки, она постоянно была под давлением, она
не успевала себе готовить, с ней всегда была сигарета, кофе, по ночам она
успевала после проверки моих домашних заданий, прочесть и обсудить с подругами
новый роман или какой-то новый журнал, типа "Новый мир", "Иностранная
литература", "Юность", "Огонек", которые мы выписывали. Она продолжала работу с
радионуклидами, регулярно готовила новые научные  работы и статьи, готовилась к
докторской диссертации, считала необходимым достичь этого уровня как ученый. 

Ей было всего 46 лет - на 6 всего больше, чем мне сейчас, когда у нее вдруг
отнялась рука и начались перебои с речью - микроинсульт... Ее положили в
больницу, рука и речь не восстанавливались, а через два месяца она внезапно
лежа в палате ушла из жизни от обширного мгновенного инсульта, признаков
приближения которого врачи не заметили.... Мне было 17, и невозможно описать ту
пустоту, которая образовалась в моей жизни после ее ухода. 

Большая часть всего того, что у меня есть и кем я есть - это от нее. От ее ума
и заботы. Она была лучшим мотиватором и позволяла мне поверить в себя. Она
умела зажечь, собраться и заставить. Только сейчас в понимаю глубину ее труда,
когда она заставляла меня учить на память стихи и прозу - в начальной школе
меня так часто наказывали за плохое поведение, и почти каждый вечер  я
занимался зубрежкой текстов. Она во многом привила мне постоянную привычку и
тягу к чтению, и многие другие такие полезные в жизни вещи. 

В 2007-м году в журнале Академии наук Украины "Медична гідрологія та
реабілітація" № 3 вышла статья в память о маме - я об этой статье узнал
недавно, приятно, что ее помнили и помнят ее коллеги. Жаль, не подписано - а то
хочется сказать спасибо... http://dspace.nbuv.gov.ua/handle/123456789/41384

Только тут я увидел список научных работ, и только сейчас понял, какой огромный
мир был у нее в руках, как серьезно и как масштабно она думала, и при этом
отдавала столько времени семье. Миллионы людей ездят в Трускавец и пьют
"Нафтусю" - а многие методики применения и свойства этой воды были исследованы
моей мамой, и легли в основу современных знаний об этой воде. Ее знания
продолжают жить, а ее фундаментальная диссертация о "Нафтусе", две монографии,
десятки научных статей сохраняют эти знания в научных библиотеках, а все
современные монографии о "Нафтусе" выходят на основе маминых работ. Она сделала
все это своими руками, 15 лет проводя первичные исследования и выполняя
огромный массив базовой исследовательской работы.

Вот эта частичка, как и ее дети и внуки, останется навсегда в память о маме.
Первая вещь, которую я подумал купить себе в квартиру была посудомоечная
машина, я о том случае никогда не забывал...

Мама - это самое первое душевное тепло, которое познает человек, и ощущение как
ты прижимаешься к чему-то доброму и родному - это первая эмоция и первое
чувство. Потому что жизнь это и есть - передача любви от одного к другому. Вот
только тепло это божественное не вечно, и счастлив тот, кто может к этому теплу
прийти и обнять его снова... 

С днем рождения, мамуля.
