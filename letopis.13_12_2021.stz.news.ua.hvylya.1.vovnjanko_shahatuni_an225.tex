% vim: keymap=russian-jcukenwin
%%beginhead 
 
%%file 13_12_2021.stz.news.ua.hvylya.1.vovnjanko_shahatuni_an225
%%parent 13_12_2021
 
%%url https://hvylya.net/analytics/243393-kak-v-sssr-stroili-samolety-elizaveta-shahatuni-ot-an-2-do-an-225-mriya
 
%%author_id 
%%date 
 
%%tags 
%%title Как в СССР строили самолеты. Елизавета Шахатуни: от Ан-2 до Ан-225 «Мрія»
 
%%endhead 
\subsection{Как в СССР строили самолеты. Елизавета Шахатуни: от Ан-2 до Ан-225 «Мрія»}
\label{sec:13_12_2021.stz.news.ua.hvylya.1.vovnjanko_shahatuni_an225}

\Purl{https://hvylya.net/analytics/243393-kak-v-sssr-stroili-samolety-elizaveta-shahatuni-ot-an-2-do-an-225-mriya}

20-го декабря 2021 исполняется 110 лет со дня рождения выдающегося создателя
самолетов марки «Ан» Елизаветы Аветовны Шахатуни. Авиаконструктор Анатолий
Вовнянко рассказал на ее примере, как в СССР строили самолеты.

\ii{13_12_2021.stz.news.ua.hvylya.1.vovnjanko_shahatuni_an225.pic.1}

20-го декабря 2021 исполняется 110 лет со дня рождения выдающегося создателя
самолетов марки «Ан» Елизаветы Аветовны Шахатуни. К сожалению, материалов в СМИ
об Е. А. Шахатуни незаслуженно мало, а те материалы, что есть, не отображают ее
выдающийся вклад в авиастроительную отрасль СССР и Украины. Ее биографические
данные приведены в материале \href{https://vstrokax.net/istoriya/elizaveta-shahatuni-stalnaya-ledi/}{\enquote{Елизавета Шахатуни – стальная женщина}}. Оттуда же
взяты и фото.

В этом материалы приведены воспоминания автора об выдающейся женщине и
авиационном конструкторе и прочнисте Елизавете Шахатуни под руководствои
которой я проработал с 1973 по 1982 год.

Елизавета Шахатуни — первая в СССР женщина-авиаконструктор, профессор, доктор
технических наук, лауреат Ленинской премии, главный конструктор, а также жена и
соратник известного авиаконструктора Олега Антонова.

Елизавета Шахатуни единственная в мире женщина, которая достигла таких высот в
области авиастроения и участвовала в создании самолетов от Ан-2 до Ан-225
«Мрія».

В апреле 1973 года после окончания Московского авиационного института я
распределился на Киевский Механический Завод (я родом из с. Великополовецкое,
Киевской области и хотел вернуться на Родину), где генеральным конструктором
был Олег Антонов. Следует отметить, что Елизавета Аветовна тоже окончила МАИ в
1935 году.

Поскольку у нас в институте преподавали выдающиеся специалисты в области
авиации, в частности, Егер С.М. (заместитель Туполева А.Н.) по пассажирской
тематике, то мне очень хотелось попасть в отдел общих видов КО-7, где
закладываются основы будущих самолетов. Но зам. директора завода по кадрам
Рожков М. С. сказал: "Или иди в отдел прочности РИО-1, или езжай обратно в
Москву". Пришлось скрепя сердце соглашаться. И мне очень повезло, т.к. я попал
в чудесный коллектив, где руководителем была Елизавета Аветовна Шахатуни,
специалист высочайшей квалификации и замечательный Человек. Она всегда
стремилась к новым знаниям и внедряла их в прочностные расчеты, опекала молодых
специалистов, помогала и в производственных вопросах и бытовых. В отделе было
несколько бригад: нагрузки, расчеты на статическую прочность планера самолета и
шасси, ресурс, флаттер, акустические и вибрационные нагрузки. Так же под
началом Елизаветы Аветовны были отделы ЭО-14 и О-26. Первый — это огромный цех
где проводились испытания планеров самолетов и агрегатов на статическую
прочность и ресурс, а второй - где проектировали системы нагружения при
испытаниях, тензометрии (замеров напряжений), осмотры конструкции в процессе
испытаний с применением визуальных и инструментальных методов и др.

Я попал в созданную 4 месяца назад новую бригаду усталостной прочности, где был
только один руководитель Бенгус Г.Ю. Дело в том, что в 1972 году под Харьковом
потерпел катастрофу пассажирский самолет Ан-10, а также под Куйбышевом в полете
летчики услышали, как в районе центральной части крыла самолета Ан-10 что-то
трещит. Чудом не произошло катастрофы. Комиссия определила, что причиной стало
усталостное разрушение по стыку центроплана крыла с ОЧК. Были подобные проблемы
и с самолетами других конструкторских бюро. В результате приказом по
Министерству авиационной промышленности (МАП) во всех опытно-конструкторских
бюро (ОКБ) СССР были образованы такие бригады.

Ранее в СССР ресурс самолетов определялся после создания самолета и внедрения
его в эксплуатацию по результатам ресурсных лабораторных испытаний натурных
образцов планеров самолетов, которые рассчитывались только на статическую
прочность, а также по результатам эксплуатации самолетов, так называемых,
лидеров (бОльший налет и более частые и тщательные осмотры).

Задача новой бригады состояла в разработке методик расчета ресурса самолетов на
стадии проектирования. Поскольку опыта было мало, то мы старались максимально
воспользоваться доступным зарубежным опытом, и работами, которые проводились в
других ОКБ, в частности Лоима В.Б, который работал у Туполева А.Н., ЦАГИ
(центральный аэрогидродинамический институт), о также результатами натурных
испытаний самолетов КМЗ. Елизавета Аветовна подбирала специалистов в эту
бригаду и лично курировала все работы. Для этого начали проводить усталостные
испытания образцов и элементов авиационных конструкций. На основании этих
испытаний и материалов разрабатывались методики расчета крыла, фюзеляжа,
оперения и других сложных элементов конструкции планера. Позже начали проводить
расчеты и испытания на скорость роста трещин и остаточную прочность образцов и
элементов конструкции. Эти работы проводил Малашенков С.П. Все эти наработки
впервые были использованы при проектировании самолета Ан-72, а затем и Ан-74.
Причем прочнисты, с перепугу, (специалистов, которые отвечали за ресурс
самолета Ан-10, прокуратура хотела реально посадить в тюрьму и только с большим
трудом руководство ОКБ спасло их) заложили такой запас прочности, что не смогли
разрушить крыло в процессе статических испытаний. Это позволило обеспечить
максимальную грузоподъемность 10 тонн, вместо 7 тонн заданных требованиями ТЗ.

В 1976 году Елизавета Аветовна создала группу «конструкционная прочность
материалов» и назначила автора ее руководителем. Задача указанной группы
состояла в том, чтобы заказывать изготовление и испытания различных образцов и
материалов на усталостную прочность, длительность роста трещины и вязкость
разрушения, а также улучшать свойства материалов для силовых деталей планера.
Проводить анализ указанных результатов и выдавать рекомендации конструкторам и
прочнистам. Мы работали с отраслевыми институтами и металлургическими заводами
по разработке и испытаниях новых материалов и внедрению их в конструкцию
самолетов. Я набрал в группу молодых специалистов И. Воронцова, А. Ковтуна, В.
Гречко, Е. Захаренко, В. Кузнецова, а несколько позже А. Николайчика для
иследования остаточных напряжений в заготовках и деталях из них и влияние их на
усталость конструкции. Поскольку в 1979–1981 годах я по заданию главного
конструктора по самолету Ан-124 П.В. Балабуева проводил на металлургических и
авиастроительных заводах с целью освоения изготовления крупногабаритных
полуфабрикатов для обеспечения постройки первого планера самолета то группой
фактически руководила Шахатуни.

Отдельно отмечу выполненные по указанию Елизаветы Аветовны работы по выбору
сплава для сложных фрезерованных деталей из поковок и штамповок для самолетов
Ан-72 и Ан-74. В СССР для этих целей, в основном, использовался низкопрочный
(предел прочности 39 кг/мм2) сплав АК6Т1. Хотя в самолете Ан-22 уже был широко
применен сплав В93Т1 (48 кг/мм2), но большие проблемы с его низким ресурсом не
удовлетворяли Шахатуни. В США для этих целей использовался высокопрочный (56
кг/мм2) сплав 7075Т6. По результатам многих исследований было известно, что
среднепрочный (44 кг/мм2) сплав Д16Т обладает высокими характеристиками
усталостной долговечности и превосходит перечисленные сплавы, но практически
нигде не применяется в виде ковочного сплава. Однако мы нашли в литературе, что
в самолете «Каравелла» (Франция), аналог сплава Д16Т применялся для этих целей.
Всесоюзный институт авиационных материалов (ВИАМ) нас пугал, но не конкретно
какими-то последствиями, а так, в общем, что этот сплав не применяется для
поковок и штамповок. Тем не менее мы изготовили на Верхне-Салдинском
металлургическом заводе (ВСМОЗ) опытные штамповки, испытали и Шахатуни Е.А.
было принято решение о применении сплава Д16пчТ для поковок и штамповок
самолета Ан-72. Меня послали на указанный завод, чтобы я согласовал технические
условия, где мы заложили прочность несколько выше среднего уровня, потому что
проблему снижения массы в самолетостроении еще никто не отменял. Никто на
заводе не хотел подписываться под этими характеристиками. Я бегал целую неделю
между цехами и начальством, отморозил уши (дело было зимой), но нам здорово
помог зам. главного инженера Никитин Е.М., заставив начальников цехов и
подразделений подписать наши характеристики. Впоследствии руководство КМЗ взяло
его к нам на завод главным металлургом.

Более 40 лет самолеты Ан-72 и Ан-74 эксплуатируются в сложных климатических
условиях и никаких проблем с деталями из сплава Д16пчТ нет!

При создании самолета Ан-124 у Шахатуни Е.А. возникли сомнения по поводу того,
что уровень ресурсных характеристик отечественных сплавов был такой же, как у
их зарубежных аналогов, и она в 1976 году поручила мне провести сравнение
усталостной долговечности. Очень сложно это было сделать, т.к. были
существенные отличия – у нас образцы с отверстием, у них с боковыми надрезами;
у нас частота испытаний 40 Гц, у них 33 Гц. Не всегда совпадали и режимы
испытаний: пульсирующая нагрузка или симметричный цикл. Тем не менее,
перелопатив кучу иностранных источников, удалось подобрать немного убедительных
результатов, где мы показали некоторое преимущества зарубежных сплавов над
отечественными по усталостной долговечности.

После утвердения этого отчета у Антонова О.К и рассылки его в отраслевые
институты у нас началась настоящая «война» (главнокомандующим была Елизавета
Аветовна) с руководством ВИАМ и ВИЛС (Всесоюзный институт легких сплавов),
которые рассказывали, что в СССР все характеристики сплавов и полуфабрикатов из
них такие же, как и у США, и мы им ни в чем не уступаем. Особенно жесткое
противостояние было с начальником лаборатории №3 ВИАМ Фридляндером И.Н.
Руководство ЦАГИ, в лице Зам. начальника ЦАГИ по прочности Селихова А.Ф. и
начальника отделения Воробьева А.З., хотя и заняли нашу сторону, но вели себя
очень пассивно. Руководство КМЗ подняло эти вопросы на уровень Министерства. Мы
также взяли себе в союзники прочнистов с ОКБ Туполева А.Н. Со временем нас в
ВИАМ поддержали академик Кишкин С. Т. и его жена Кишкина С.И., доктор наук,
руководитель лаборатории прочностных испытаний №8. Позже, когда руководителем
ВИАМ назначили Шалина Р.Е., то началась совместная продуктивная работа.

В СССР никак не могли понять, как зарубежные самолеты В-707, В-747, ДС-8 и др.
имеют ресурс 80 000–100 000 летных часов, тогда как в СССР 15 000–30 000. Мало
того, когда проектировали самолет Ту-154, так дважды пришлось уже в
эксплуатации переделывать крыло, т. к. оно не обеспечивало требуемый ресурс
30000 летных часов. Вскоре нам представилась возможность изучить конструкцию
зарубежных самолетов. В Шереметьево под Москвой потерпел катастрофу самолет
ДС-8 японской авиакомпании, а затем на Кольском полуострове истребителями был
«посажен» самолет В-707 корейской авиакомпании, который заблудился и попал в
воздушное пространство СССР.

В ОКБ генерального конструктора Илюшина были собраны куски конструкций
указанных самолетов. Шахатуни договорилась с руководством и послала меня, чтобы
я отобрал необходимые образцы для исследований и изучения. Также испытания их
проводились и в ЦАГИ, в частности, на живучесть (длительность роста трещины и
остаточная прочность при наличии трещины).

По результатам исследований и испытаний было определено:

- в конструкции (оперение и продольный набор фюзеляжа) американских самолетов
более широко применяется высокопрочный сплав 7075-Т6 (аналог в СССР сплава
В95Т1), тогда как в отечественных самолетах для этих конструкций применялся
мене прочный, но более высокоресурсный сплав Д16Т (аналог в США 2024Т3);

- широкое применение болт-заклепок и других крепежных элементов, которые
ставились с натягом, что существенно повышало усталостную долговечность и
герметичность;

- автоматическая клепка стержнями панелей крыльев автоматами фирмы «Джемкор»,
что обеспечивало высокие усталостные характеристики и их стабильность, тогда
как в СССР большинство этих работ выполнялось вручную;

-применение твердой плакировки на листах, что повышало их усталостную
долговечность. В СССР плакировка (покрытие с целью защиты от коррозии)
выполнялась чистым алюминием;

- значительно более высокий уровень проектирования конструкции для обеспечения
высокой усталостной долговечности;

- более высокое качество изготовления элементов конструкции и тщательная
подгонка деталей в производстве;

- более низкое содержание вредных примесей железа и кремния в сплавах 2024 и
7075, чем в отечественных сплавах, что повышало живучесть конструкции;

-в конструкции шасси применялась высокопрочная (210 кг/мм2) сталь, тогда как у
нас сталь 30ХГСНА прочностью 160 кг/мм2.

Результатом этих исследований и др. стало широкое применение в конструкции
самолета Ан-124 крепежа с натягом и сплавов высокой чистоты по указанным
примесям Д16очТ, В95очТ2 и В93пчТ2, проектирование с учетом ресурса, широкое
применение титана, повышение культуры и качества в серийном производстве,
внедрение новых технологических процессов, в частности, дробеструйной обработки
панелей и деталей и др., что позволило существенно повысить ресурс и
коррозионную стойкость силовых конструкций планера самолета.

При создании самолета Ан-124 остро встала проблема выбора полуфабрикатов
(прессованные панели или катаные плиты) для изготовления силовой конструкции
крыла. Дело в том, что за рубежом для крыльев пассажирских самолетов, которые
имели огромный ресурс, применяются катаные плиты с приклепанными к ним
стрингерами, а в СССР больше применялись прессованные панели, где обшивка и
стрингер составляют одно целое. Это было связано с тем, что в СССР по
инициативе руководителя ВИЛС академика Белова А.Ф. в начале 1960-х годов для
производства самолета Ан-22 и с учетом на перспективу в отрасли были
разработаны и построены уникальные горизонтальные прессы мощностью 20000 тонн
для изготовления прессованных полуфабрикатов и вертикальные прессы мощностью
60000 тонн для изготовления крупногабаритных штамповок. Такого оборудования не
было нигде в мире. В конце 1970-х годов такой вертикальный пресс купила в СССР
даже металлургическая фирма «Пешине» Франция. В крыльях самолетов Ан-24, Ан-72,
Ан-22, Ил-62, Ил-76, Ил-86 и др. широко применялись прессованные панели и
поэтому на серийных авиационных заводах было оборудование и технологии их
изготовления.

В начале 1970-х годов в Советском Союзе рассматривалась возможность закупки у
фирмы Боинг пассажирского широкофюзеляжного самолета В-747. В г. Эверетт, где
строили эти самолеты, летала большая делегация руководителей МАП, ОКБ и
институтов. Их сильно впечатлило увиденное на производстве, особенно клепка
панелей крыла автоматом «Джемкор», а также то, что ресурс этого самолета
составлял до 100 000 летных часов. Потом специалисты фирмы Боинг прилетали с
докладами о самолете В-747 в Москву, где принимала участие и Елизавета
Аветовна. После приезда в Киев она собирала нас и рассказывала об этой встрече.
Больше всего Шахатуни поразило то, что американцы каждый день одевали новые
костюм, галстук и рубашку (всего 3 дня длились эти доклады), так как у нас
обычно был один костюм на все случаи жизни.

Впечатленные увиденным на Боинге, все отраслевые институты заняли позицию, что
надо крыло самолета Ан-124 делать сборной конструкции из катаных плит! Мы же
заняли позицию, что крыло надо делать из прессованных панелей. И тут, как
говорится, нашла коса на камень. Наши конструкторы и технологи показали, что в
случае применения прессованных панелей с законцовкой можно применить фланцевый
стык, а не срезной, что упрощает стыковку концевой и центральной части крыла и
снижает трудоемкость, упрощает герметизацию кессона крыла. Мы также показывали,
что в СССР нет производства длинномерных (до 30 м) катаных плит, как в США. Но
нам надо было еще доказать, что и ресурсные и весовые характеристики такого
крыла будут не хуже.

Елизавета Аветовна предложила провести сравнительные испытания. Мы подготовили
и согласовали с ЦАГИ, ВИАМ и ВИЛС большую программу испытаний и летом 1976 года
я полетел на Ташкентский авиационный завод, где руководителем нашего филиала
был Ермохин И.Г. В это время здесь строили самолет Ил-76, крыло которого делали
из прессованных панелей. Мы отобрали 10 прессованных панелей из сплава Д16Т и
В95Т1, которые отличались, в пределах допуска по прочности и по химическому
составу. Согласно «Программы…», завод должен был изготовить сотни различных
образцов разных размеров для испытаний на усталость и живучесть и разослать их
в ЦАГИ, ВИАМ и КМЗ. Выполнение всей этой работы, не специфичной серийному
заводу, обеспечивали И. Ермохин с К. Демидов. Потом я поехал в МАП, где
руководство КМЗ решило вопрос, чтобы меня приняли на Воронежском авиационном
заводе, а также согласовали и выполнили Программу испытаний. С Москвы я поехал
в Воронеж, где производили самолет Ил-86, в конструкции центральной части
фюзеляжа которого применялись катаные плиты сплава Д16Т. Я отобрал 3 плиты,
согласовал Программу, порешал все вопросы и ознакомился с производством
сверхзвукового самолета Ту-144 и др.

После проведения многочисленных иследований прессованных панелей и катаных плит
мы доказали, что первые значительно превосходят вторые по всему комплексу
свойств в продольном направлении – статическая и усталостная прочность и
живучесть. В результате для самолета Ан-124 были выбраны уникальные
длинномерные (30 м) прессованные панели с законцовкой которые производились на
единственном в мире металлургическом заводе ВСМОЗ – ныне АВИСМА, г. Верхняя
Салда (см. «Из истории разработки «Руслана» и «Мрии», сайт Крылья).

Благодаря огромным усилиям Шахатуни и руководства КМЗ, были выбиты в МАП
валютные средства и закуплено специальное испытательное оборудование фирмы
«Шенк» (США), на котором проводились различные испытания крупногабаритных
конструктивных образцов. Занимался этим вопросом Муратов В.В.. Было закуплено в
Украине и менее мощное оборудование и организована бригада под руководством
Ханина Г.И., которая занималась многочисленными испытаниями небольших образцов.
Потом Елизавета Аветовна создала бригаду фрактографических исследований и
«выбила» специальный импортный микроскоп для исследований трещин. Таким образом
мы понимали физику разрушения конструкции - статическое, усталостное или
коррозионное разрушение. Руководителем бригады была назначена Бурченкова Л.М.,
высококвалифицированный специалист в этой области. Во всех этих вопросах и по
уровню доверия к полученным результатам мы за очень короткий срок достигли
уровня лабораторий ЦАГИ и ВИАМ, которые считались лучшими в отрасли, а в СССР и
подавно!

Следует отметить, что требования к ресурсу со стороны Заказчика (военных) для
стратегического военно-транспортного самолета Ан-124 были очень легкие — это 16
000 летных часов и 4 000 полетов, причем около 60-70\% полетов тренировочные, то
есть без полезной нагрузки. Тем не менее Елизавета Аветовна и Олег
Константинович решили, что негоже создавать такой уникальный самолет с таким
маленьким ресурсом. И, в дальнейшем, жизнь подтвердила их предвидение. После
развала СССР генеральный конструктор Балабуев П.В. вывел этот самолет на
международный рынок грузовых перевозок крупногабаритных грузов, создав
авиакомпанию «Руслан», и вкладывал все средства в создание новых самолетов,
внедрение новых технологий, строительство жилья и многое другое. Благодаря этой
авиакомпании и сегодня живет коллектив и обогащается руководство ГП «Антонов» и
ГК «Укроборонпром». Петр Васильевич даже представить не мог, что лизоблюды,
которые окружали его, будут грабить и разрушать дело всей его созидательной
жизни (см. «Как создавали «Руслан» и «Мрию». Воспоминания к 90-летию
генерального конструктора Балабуева П.В). Сейчас ресурс самолета Ан-124
составляет 10000 полетов и 50000 летных часов. Кроме того, была увеличена
грузоподъемность самолета со 120 тонн до 150 для 15\% полетов. И это все
достижения Шахатуни!

В 1970-х годах за рубежом появилось новое поколение алюминиевых сплавов: 2124,
7175, 2048, 7475, 7010, 7050 и технологии изготовления из них полуфабрикатов, а
также новые двухступенчатые режимы старения Т76 и Т73 для сплавов серии 7000.
Это позволило повысить весь комплекс прочностных, ресурсных свойств и
коррозионной стойкости. Следует отметить, что в целом США на 10-15 лет обгоняли
СССР в этой области (см. статью Вовнянко А.Г., Дриц А.М., «Алюминиевые сплавы в
самолетостроении - прошлое и настоящее», Цветные металлы, №8, 2010).

В отделе РИО-1 Шахатуни Е.А. была организована огромная работа по изучению
зарубежного опыта в различных направлениях. Выписывались отечественные и
зарубежные научные журналы. Специально введенным в штат отдела переводчиком
Шнайдманом М.Н. проводились поисковые работы по всему новому в области
прочности, ресурса, материалов и сплавов. Все это переводилось, анализировалось
и внедрялось. Например, во время войны во Вьетнаме потерпел катастрофу новейший
тактический бомбардировщик F-111А. Результаты исследований выявили, что
причиной явился незначительный производственный дефект, от которого
преждевременно и появилась трещина. За рубежом начались работы в этом
направлении, и мы тут не отставали. На многочисленных, обычных и конструктивных
образцах проводились испытания и отрабатывались методики расчетов Малашенковым
С.П. и Семенцом А.И. Большинством работ по исследованиям на конструктивных
образцах вёл Василевский Е.Т.

Поскольку сплав Д16пчТ мы успешно применили для изготовления крупногабаритных
штампвок для самолета Ан-72 то таким же образом внедрили и на самолете Ан-124.
Летом 1979 года пришла беда из Ташкента, где изготавливались крупногабаритные
детали. Начали растрескиваться огромные заготовки деталей из поковок сплава
Д16очТ после закалки. Для первых самолетов детали делают из поковок, т.к.
изготовление штампов длительный процесс. В Министерстве собрали и срочно
отправили туда большую Комиссию из представителей ВИАМ, ВИЛС и МАП. От КМЗ -
Шахатуни и автор. Приехали мы туда, а там порядка 10 заготовок деталей уже
треснули. Поскольку поковки очень огромные, например, для силовых шпангоутов
около 4 м в длину, шириной 0.8 м, толщиной 0.3 м и массой до 3 тонн, то ее
предварительно фрезеруют, оставляя только черновой припуск. Это необходимо,
чтобы скорость охлаждения была высокой и деталь имела требуемые прочностные и
коррозионные свойства. После ознакомления с ситуацией сидим мы все члены
комиссии за большим столом и думаем, что же это за напасть, что делать? В это
время приходят все новые и новые сообщения: еще треснула заготовка и еще. Счет
пошел уже за 2 десятка!
