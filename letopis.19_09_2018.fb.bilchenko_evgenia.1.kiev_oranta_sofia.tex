% vim: keymap=russian-jcukenwin
%%beginhead 
 
%%file 19_09_2018.fb.bilchenko_evgenia.1.kiev_oranta_sofia
%%parent 19_09_2018
 
%%url https://www.facebook.com/permalink.php?story_fbid=127119631579847&id=100028454333844
 
%%author 
%%author_id bilchenko_evgenia
%%author_url 
 
%%tags bilchenko_evgenia,kiev,sofia_sobor
%%title СТУДИЯ БЖ - 7 ОКТЯБРЯ: СДЕЛАЕМ ВМЕСТЕ?
 
%%endhead 
 
\subsection{СТУДИЯ БЖ - 7 ОКТЯБРЯ: СДЕЛАЕМ ВМЕСТЕ?}
\label{sec:19_09_2018.fb.bilchenko_evgenia.1.kiev_oranta_sofia}
\Purl{https://www.facebook.com/permalink.php?story_fbid=127119631579847&id=100028454333844}
\ifcmt
 author_begin
   author_id bilchenko_evgenia
 author_end
\fi

СТУДИЯ БЖ - 7 ОКТЯБРЯ: СДЕЛАЕМ ВМЕСТЕ?

\enquote{Бог посреди него - пусть не колеблется, Бог поможет ему, когда утро наступит}
- вот что гласит легендарная надпись над Марией Орантой в главном храме Киева -
Святой Софии. Нашу долгожданную осеннюю студию, которую я - внимание, родные! -
перевожу на 7 октября, я хочу посвятить нашему прекрасному Граду. Читаю в
который раз Сергея Крымского. Вот что он пишет: "один из центров
греко-славянской православной цивилизации, арена широкого украино-русского
взаимодействия, \enquote{платоновская реальность}, освящающая философией сердца небо и
землю, Бога и человека, разум и мораль, Закон и Благодать. Для меня, как и для
Бориса Крымского, Сергея Аверинцева, Виктора Малахова, Вилена Горского и всех
наших прекрасных умов Град, коего коснулась рука Кирилла Философа, жениха
единственной невесты - Мудрости - это София. Невозможного представить Град без
фаворского света этого золочения, без солнца-зерна, из коего есть пошел колос.
Без Григория Сковороды:  "Не в зерне ли все сіе закрілось и не весною ли
віходит все сіе...". София подобна ядру, стягивающему в свой эпицентр все
коллизии Града от хипповской Пейзажки с олдовыми рокерами - до пролетарской
Шулявки и харьковского воздуха Левобережки. Ну, и для равновесия ей, как
напоминание о том, что всякая радость есть печерское усилие духа, - Лавра.  

Друзья, давайте вместе отпразднуем эту весну осенью и вспомним наш Град
настоящий. Плюс у меня днюха. Хочу провести ее: а. с талантливыми. б. с
любимыми. в. с людьми, кто разделяет мое мнение об альтруизме в искусстве и
некоммерческой поэзии.  

\ifcmt
  pic https://scontent-mia3-1.xx.fbcdn.net/v/t1.6435-9/42160568_127117084913435_6160894208291897344_n.jpg?_nc_cat=100&ccb=1-3&_nc_sid=730e14&_nc_ohc=DSCfWKW1UdMAX92BHRL&_nc_ht=scontent-mia3-1.xx&oh=d2220ad85d458c761f481b157020e8bb&oe=60D12CF4
\fi

ВХОД СВОБОДНЫЙ. 

Репост тех, кто пойдет, поможет нам всем. Я ограничена в приглашениях пока.Наша
герменевтика слова-сердца ждет нас. Если у Вас нет городской лирики, - это
НЕВАЖНО. Сердце - это вся любовь. София - это вся мудрость. Град - это вся
память. Жду Вас, хозяйка.

БЖ. Киев.

Город спит неспокойно, тараторя во сне.
Бродит пан арендатор по ничейной стране.
Камуфляжится плесень на гетманском коне.
Пляшут рыбки-консервы в привокзальном вине.
Локтевая поддержка - переулочный вгиб:
Вжался боком в булыжник - почитай, не погиб.
Сорок пятый в граните прорастает, как гриб,
Из китайских липучек и тарасовских лип.
Страх приходит под утро. Пьёт, цепляясь за стул.
Гасит свет. Погружает в электрический гул.
То ли умер в трамвае, то ли снова уснул...
Упокой его, мама. Разбуди, караул.
Лихорадка судилищ накануне Суда -
Не лиха, не начало и уже не беда.
Я давно бы уехал, да не знаю куда: 
Все дороги открыты - не в мои города.
