% vim: keymap=russian-jcukenwin
%%beginhead 
 
%%file 25_11_2021.fb.fb_group.story_kiev_ua.1.1991_gkchp
%%parent 25_11_2021
 
%%url https://www.facebook.com/groups/story.kiev.ua/posts/1805380252992101
 
%%author_id fb_group.story_kiev_ua,fedjko_vladimir.kiev
%%date 
 
%%tags 1991,istoria,sssr
%%title 19 - 21 серпня 1991-го, ГКЧП
 
%%endhead 
 
\subsection{19 - 21 серпня 1991-го, ГКЧП}
\label{sec:25_11_2021.fb.fb_group.story_kiev_ua.1.1991_gkchp}
 
\Purl{https://www.facebook.com/groups/story.kiev.ua/posts/1805380252992101}
\ifcmt
 author_begin
   author_id fb_group.story_kiev_ua,fedjko_vladimir.kiev
 author_end
\fi

19 - 21 серпня 1991-го, ГКЧП. 

\begin{multicols}{2} % {
\setlength{\parindent}{0pt}

На серпень 1991 року за моїми плечима вже був певний багаж справ на ниві
національного визволення України, починаючи з 1987 року. 

Участь в організації і роботі Українського Культурологічного Клубу (УКК,
1987-й...); підготовка і проведення демонстрації протесту проти замовчування
результатів Чорнобильської катастрофи на площі Жовтневої Революції (нині Майдан
Незалежності) у квітні 1988 р.; робота кореспондентом Української Центральної
Інформаційної Служби (УЦІС; Провід ОУН-р; Голова Проводу друг Василь Олеськів;
Лондон, Великобританія; з 1987-го), а пізніше і директором УІС – філії УЦІС в
Україні (з 1988-го); робота в Проводі Української Національної Партії (УНП;
голова партії п. Григорій Приходько) на посаді члена Проводу і Голови Служби
безпеки партії (з 1988-го). 

\ii{25_11_2021.fb.fb_group.story_kiev_ua.1.1991_gkchp.pic.1}
\ii{25_11_2021.fb.fb_group.story_kiev_ua.1.1991_gkchp.pic.1.cmt}

\ii{25_11_2021.fb.fb_group.story_kiev_ua.1.1991_gkchp.pic.2}
\ii{25_11_2021.fb.fb_group.story_kiev_ua.1.1991_gkchp.pic.2.cmt}
\end{multicols} % }

Вже у 1987-му був налагоджений щоденний стабільний зв’язок телефаксом з
Лондоном. Спочатку я пересилав політичні новини і аналітичну інформацію з
України з телефаксу колеги, а вже у 1988-му, коли я почав працювати в УНП, то з
телефаксу, який мені надав п. Приходько і який стояв у мене вдома. Періодично я
пересилав новини і для українських газет в США, які були під опікою УДП
(Українського державного правління) і УККА (Українського Конгресового Комітету
Америки).

На той час міжнародні телефонні розмови потрібно було попередньо замовляти,
тому я щоденно, з вечора, замовляв 4-5 розмов з інтервалом в 2 години на
наступний день, щоб вчасно виходити на зв’язок і передавати свіжі новини.

На початку серпня 1991-го п. Григорій Приходько (керівник УНП і УМА), п. Юрій
Шухевич (УМА) і п. Дмитро Корчинський (керівник УНА-УНСО і УНС) поїхали в
Мюнхен на нараду, яку скликала пані Слава Стецько (на той час вона перебрала на
себе керівництво Проводом ОУН-р). Пан Григорій майже щоденно телефонував мені з
Мюнхена і цікавився останніми новинами, які я йому передавав.

18-го серпня 1991 року я ввечері, як завжди, замовив декілька телефонних розмов
на Лондон. Перша з розмов була замовлена на 7:30 ранку! Підготував роздруківки
новин, поклав їх біля телефаксу, поставив будильник на 6:30 і ліг спати.

19 серпня 1991-го 

Прокинувся я не від сигналу будильника, а від дзвінків міжгородської телефонної
станції. Дивлюся на будильник – 6:15!? Що таке, я ж замовляв розмову з Лондоном
на 7:30!? Піднімаю слухавку… і чую схвильований голос Юрія Оборіна, керівника
Вінницької організації УНП: 

- Пане Володимире, в імперії військовий путч!

Я, ще перебуваючи у напівсні, пожартував: «Наші почали наступ з «Вервольфу!?» 

- Я не жартую! У Москві комуністично-військовий путч! Вмикайте радіо і
телевізор!

Короткі гудки в слухавці!!!

Вмикаю радіо і телевізор... Звучить чудова музика з «Лебединого озера». На всяк
випадок вмикаю і настроюю портативний магнітофон «Маранц», іду на кухню і
заварюю каву.

Кава смакує... Музика закінчується... Коротка пауза... Я на всяк випадок натискую
клавішу «Запис». І тут починають передавати повідомлення ГКЧП. «Маранц» записує
всі повідомлення. А я декілька разів фотографую екран телевізора на «Поляроїд».
Знову грає музика. Сиджу і розмірковую над почутим!

Дзвінки міжнародної телефонної станції… Піднімаю слухавку… Жіночий голос
ввічливо запитує: 

- Лондон заказывали!? Будете говорить!?

- Да, конечно буду говорить!

- Соединяю!

Через декілька секунд в слухавці чую знайомий сигнал телефаксу, який подає
сигнал на з’єднання. Ще декілька секунд... факси домовилися і перший аркуш з
новинами поповз на передачу. Поки «Toshiba» передає аркуш за аркушем на Лондон,
в штаб-квартиру УЦІС, пишу на чистому аркуші: «Пане Миколо чи пане Богдане, по
закінченню N аркуша підніміть слухавку!!! Термінова і надзвичайна новина!!!»
Пару разів пускаю цю записку поміж аркушів з новинами. За запискою посилаю і
фотографії... Новини закінчилися і чую в слухавці голос Миколи Матвіївського,
одного з адміністраторів УЦІС: 

- Влодку, як там навколо тебе!? Що трапилося!?

Інформую про ГКЧП… Микола питає чи записав я повідомлення? На мою ствердну
відповідь прохає, щоб я передав запис. Підношу слухавку до динаміка і вмикаю
відтворення. На другому кінці Микола записує на магнітофон. Повідомлення
великі, загалом більше ніж на півгодини. Дивно, але зв’язок не переривається.
Тим часом по телевізору знову повторюють повідомлення ГКЧП.

Нарешті все передано! Ура! Микола просить слідкувати за новинами і регулярно
виходити на зв’язок!

***

Несподівано телефонує Аркадій Кірєєв. Він був першим президентом УКК і його
Сергій Набока і Вадим Галиновський підозрювали в агентурній співпраці з КГБ.
Після 1988 року Кірєєв відійшов від УКК і працював з структурами ОУН-м
(прихильниками п. Андрія Мельника) – за дорученням Голови ПУН п. Миколи
Плав’юка випускав інформаційний бюлетень «Тризуб». 

- Що ти думаєш з приводу ГКЧП!? Що робити!?

- Завари каву, налий чарку коньяку, смакуй каву з коньяком, пали папери і
чекай!

- Що чекати!?

- Коли за тобою приїдуть!

- Ти серйозно!?

- Абсолютно! До речі, поменше спілкуйся телефоном. Наші розмови можуть
прослуховуватися!

Цікавий дзвінок! Перевірка чи я вдома?!

Іду на кухню, заварюю ще одну філіжанку кави. Паралельно палю деякі папери, в
тому числі і з переданими новинами, попіл змиваю в унітаз.

***

Годині о 12-й з Мюнхена потелефонував п. Приходько. Я розповів йому новини про
ГКЧП, передав запис повідомлень ГКЧП. Наша розмова продовжувалася більше
години. 

***

Вирішив поїхати в місто подивитися, що на вулицях. Дома нічого компрометуючого
не залишилося. Вийшов з дому, тишком-нишком перевіряю чи нема за мною стеження.
Нібито чисто, але гарантій нема. Погуляв містом декілька годин, поспілкувався з
багатьма знайомими і приятелями. Добра половина з них перелякана. Дехто вже
встиг прийняти добру порцію спиртного. Десь годині о 19-й повернувся додому і
коротко описав враження від побаченого. На мій телефакс, який стояв на
автоматичному прийомі, прийшло багато цікавих повідомлень з Москви і Ленінграда
від демократичних журналістів, з якими я підтримував інформаційний обмін. А о
22-й годині мені знову дали лінію на Лондон. Відіслав підготовлені матеріали,
замовив декілька розмов на наступний день і ліг спати. 

Перша доба правління ГКЧП закінчилася!

20 серпня 1991-го

Вранці мені дали лінію на Лондон. Трохи здивувався, але відіслав підготовлені
матеріали, спалив аркуші з переданою інформацією і поїхав у місто. Накупив
свіжих газет, поспілкувався із знайомими... Повернувся додому і почав готувати
зведення новин. 

Попрохав дружину сфотографувати мене за роботою. Швидко проявив плівку,
висушив, надрукував фотографію, висушив і додав до новин. Треба ж залишитися в
історії українського спротиву! Жартую.

Цілий день йшов насичений інформаційний обмін з Москвою, Ленінградом,
Прибалтикою, Лондоном... Телефонний зв’язок працює бездоганно!

***

Ввечері заїхав поспілкуватися давній знайомий Сашко Іванов. Приїхав не один, а
з пляшкою чудового грузинського коньяку. Разом подивилися прес-конференцію
ГКЧП. Звернули увагу, що у Янаєва трясуться руки. Приятель виніс вердикт: «не
пройде і трьох днів, як усіх повісять на балалаєчній струні»!

Смакуємо коньяк і канапки з гострим сиром, каву, ведемо розмови про змови і
перевороти.

Телефонна станція дає лінію на Лондон. Факсую величезну купу матеріалів, як
написаних мною, так і отриманих від колег в інших містах імперії; вирізки та
фотографії з газет. Надіслав у Лондон і свою фотографію. 

Перша година ночі. Приятель залишається ночувати у нас. Дружина стелить йому на
розкладачці в нашій кімнаті, а сама йде спати в другу кімнату, до батьків. Ми
ще хвилин 20 перемовляємося і спокійно засинаємо.

Друга доба правління ГКЧП закінчилася!

21 серпня 1991-го

На третій добі епоха ГКЧП закінчилася, але нікого з путчистів, на жаль, так і
не повісили!

***

Комуністичний путч провалився! Починалася нова епоха!

В Україні проголосили незалежність... Вже у жовтні – листопаді я приймав у Києві
друга Степана Олеськіва, відповідального працівника Спілки Українців Британії
(СУБ), сина друга Василя Олеськіва, попереднього Голови Проводу ОУН-р; друга
Степана Мудрика, шефа Служби Безпеки ОУН-р та інших. Але це вже інша історія.

***

\ii{25_11_2021.fb.fb_group.story_kiev_ua.1.1991_gkchp.cmt}
