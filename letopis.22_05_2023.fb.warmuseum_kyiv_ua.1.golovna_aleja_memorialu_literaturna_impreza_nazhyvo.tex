%%beginhead 
 
%%file 22_05_2023.fb.warmuseum_kyiv_ua.1.golovna_aleja_memorialu_literaturna_impreza_nazhyvo
%%parent 22_05_2023
 
%%url https://www.facebook.com/warmuseum.kyiv.ua/posts/pfbid0N4Bon3Q5iivZbH9BwSM8NRbE6ZH64PHgudKbyoL2kALdvMtC4UzWGZtFK5jCjDtPl
 
%%author_id warmuseum_kyiv_ua
%%date 22_05_2023
 
%%tags 
%%title Головна алея Меморіалу - перформативна літературна імпреза "#НаЖиво"
 
%%endhead 

\subsection{Головна алея Меморіалу - перформативна літературна імпреза \enquote{\#НаЖиво}}
\label{sec:22_05_2023.fb.warmuseum_kyiv_ua.1.golovna_aleja_memorialu_literaturna_impreza_nazhyvo}

\Purl{https://www.facebook.com/warmuseum.kyiv.ua/posts/pfbid0N4Bon3Q5iivZbH9BwSM8NRbE6ZH64PHgudKbyoL2kALdvMtC4UzWGZtFK5jCjDtPl}
\ifcmt
 author_begin
   author_id warmuseum_kyiv_ua
 author_end
\fi

В рамках триденного заходу \enquote{Маріуполь. 86 \#наЖиво}, присвяченого звитяжній
обороні Міста Марії, на головній алеї Меморіалу відбулася перформативна
літературна імпреза \enquote{\#НаЖиво} - театралізоване читання віршів у виконанні
акторів та акторок маріупольського Театр авторської п'єси \enquote{Conception}.

В основу імпрези покладено вірші  \enquote{Нави} - поетки та журналістки, колишньої
викладачки Маріупольського державного університету, кандидата наук із
соціальних комунікацій, речниці полку \enquote{Азов} @Валерія Суботіна (Карпиленко).
Авторка родом з Маріуполя. У 2015 році вона приєдналася до пресслужби полку
\enquote{Азов}, згодом стала пресофіцеркою Донецького прикордонного загону, а 24 лютого
стала на захист рідного міста. Разом з побратимами та посестрами була в облозі
на \enquote{Азовсталі}, звідки потрапила у полон. Валерія провела в неволі майже рік і
повернулась додому лише в квітні цього року. В імпрезі також прозвучали поезії
інших маріупольських авторів: @Оксана Стомина, Богдана Слющинського, @Dmitry
Gritsenko, а також твори київського поета @Петро Мага.

По завершенню свого виступу актори маріупольського театру \enquote{Conception}
запропонували присутнім долучитись до заходу у форматі відкритого мікрофону,
адже серед тих, хто прийшов у цей сонячний день послухати поетичні рефлексії
маріупольців про \enquote{місто, яке звільнили... від життя} було чимало їхніх земляків,
а також діячів культури з різних регіонів України, які висловили бажання
поділитись своїми поезіями та піснями. Це, зокрема, поетка з Рівненщини Інґі
Ґерда (Ірина Рудика / Ingi Gerda), юна акторка та волонтерка, шестиразова
\enquote{Гордість України} з Києва Вокалістка Софія Котлярова, маріупольський журналіст
та телеведучий @Макс Грабовський, кіноактриса Anna Birzul та
поетка-маріупольчанка Оксана Стоміна. Її вірші було використано в імпрезі, а
зачитані нею самою \enquote{Листи в полон (з невідправленого)}, адресовані чоловікові,
з яким від моменту виходу з Азовсталі досі немає жодного зв’язку, особливо
болісно прозвучали під безтривожним на ту мить сонячним київським небом.

Колектив маріупольського Незалежного театру авторської п'єси \enquote{Conception} було
засновано у 2019 році, тоді, коли попри випробування російською агресією,
Маріуполь потужно розвивався у культурно-мистецький царині, аби вже за рік
перемогти в номінації \enquote{Велика культурна столиця} Українського культурного
фонду. Це все залишилось в іншому житті, в якому були наскрізними море, солоний
бриз, сонце і спокій - атмосфера, яку через поезії маріупольці бодай ненадовго
відтворили на київських кручах. На зміну \enquote{золотій добі} прийшла смерть, холод і
жах. Про них говорили теж... 

Не повернути спокою та рівноваги всім, хто врятувався з блокадного Міста Марії.
Принаймні доти, доки Маріуполь не буде звільнено від ворога, а останній
полонений не повернеться на батьківщину. Про це у своїх поетичних імпрезах та
виставах говорили і говоритимуть актори маріупольського театру \enquote{Conception},
адже Музи не мовчать, коли гримить зброя - \enquote{навіть за крок до смерті Музи на
диво вперті}.

Дякуємо всім, хто долучився до заходу!

Організатори: 

Маріупольська міська рада, 
Національний музей історії України у Другій світовій війні, 
ЯМаріуполь

\#музейвійни \#warmuseum \#ЯМаріуполь
