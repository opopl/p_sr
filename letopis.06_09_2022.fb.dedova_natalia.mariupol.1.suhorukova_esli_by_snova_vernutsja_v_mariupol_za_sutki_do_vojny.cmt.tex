% vim: keymap=russian-jcukenwin
%%beginhead 
 
%%file 06_09_2022.fb.dedova_natalia.mariupol.1.suhorukova_esli_by_snova_vernutsja_v_mariupol_za_sutki_do_vojny.cmt
%%parent 06_09_2022.fb.dedova_natalia.mariupol.1.suhorukova_esli_by_snova_vernutsja_v_mariupol_za_sutki_do_vojny
 
%%url 
 
%%author_id 
%%date 
 
%%tags 
%%title 
 
%%endhead 

\qqSecCmt

\iusr{Ольга Коберник}
😪😪😪

\iusr{Екатерина Хлименкова}

😢😢😢😢😢😢🙏🙏🙏🙏🙏как же все точно сказано 😢😢😢😢😢

\iusr{Ирина Имамова Имамова}

Голос войны!

\iusr{Inna Drozdenko}

Ще дома такі цілі і швидка є..

\iusr{Natali Lozovaya}

Наташенька ....такая боль.... только Бог сможет залечить наши раны ....

\iusr{Наталья Авлиякулова}

Плачу😭. Опять всё пережила😭. Нам повезло, мои все живы. Но..., 9 марта после
бомбы на 3 роддом. Моя дочь и 3-х летний внучок чуть не погибли 😪, она только
забежала на руках с дитем в ванну, и бахххххх. Их дом был напротив 😌

\begin{itemize} % {
\iusr{Anna Skaldina}
\textbf{Наталья Авлиякулова} слава Богу, что живы ❤️

\iusr{Stefaniia Gav'iak}

Тримайтьсь, маріуполтчани. Читаю про вас і кожний раз наплачусь... Стільки біди і горя....
\end{itemize} % }

\iusr{Хитрова Раиса}

\ifcmt
  igc https://i2.paste.pics/07ffbecd171a912d2b9b27a9063351bb.png
	@width 0.2
\fi

\iusr{Tatyana Komelchik}

Скажите ваше отношение к той девушке.... с роддома. Она выжила.... и потом раз
помогает этим нелюдям, которые разрушили роддом. Мы так переживали за неё.... Как
так? Это страх?

\begin{itemize} % {
\iusr{Natalya Dedova}
\textbf{Танюша Нечипоренко} это нет мозга.

\iusr{Svetlana Svetlana}
\textbf{Танюша Нечипоренко} эта девушка из России, вышла замуж за Мариупольского парня и я думаю изначально у неё была другая сторона медали

\iusr{Tatyana Komelchik}
\textbf{Svetlana Svetlana} Спасибо. А то думала, её пытали и заставили. А оно такое изначально....
\end{itemize} % }

\iusr{Лилия Жукова}

кос в горле.. нет слов!

\iusr{Татьяна Процко}

В это было бы трудно поверить, очень трудно. Сейчас глядя на всё сердце кровью
обливается, жизни людей, дома и в конце концов наши мечты всё разрушено.. Вся наша
семья верит, что Украина победит! Мы вернёмся в Мариуполь! Дай Бог всем
здоровья.

\iusr{Ольга Косовская}

Унас таки було тоді ще кілька днів надивитися на рідне місто напередодні навали
нелюдей. Вме ж ми в більшості мали надію, що не станеться повного знищення його.
Та багато нас вижило, вибралося в вільний світ ...та ми повернемося і місто
Буде!

\iusr{Лилия Аксаментная}

За сутки до войны меня предупредили трое из моих друзей, сказали, что надо
уезжать. Я не поверила, списали на панику и нагнетание ситуации в регионе и
стране.

Спасибо, что живы и выбрались..

\begin{itemize} % {
\iusr{Natalya Dedova}
\textbf{Лилия Аксаментная} давайте запишемся, Лиля.

\iusr{Лилия Аксаментная}
Наталья, дайте мне немного времени, ок?

\iusr{Natalya Dedova}
\textbf{Лилия Аксаментная} давайте. Немного. На следующей неделе? 🙏

\iusr{Лилия Аксаментная}
\textbf{Наталья Дедова}, договорились
\end{itemize} % }

\iusr{Ирина Чумакова}

Я бы не поверила наверное. Даже перед 17 февраля муж набрал 20 литровый бутыль
воды и опустил в подвал, начал учить детей быстро туда забираться. Я просто
посмеялась, сказала, что если захотят напасть, об этом не предупреждают. Даже
когда выезжали 15 марта в Мелекино, мы ехали на 10 дней, пересидеть, ведь это
закончится, и мы вернёмся.

\begin{itemize} % {
\iusr{Natalya Dedova}
\textbf{Ирина Чумакова} расскажите Ваши историю, Ирина! 🙏

\iusr{Ирина Чумакова}
\textbf{Наталья Дедова} я пока не готова, наша история не такая ужасная, как у других мариупольцев.

\iusr{Natalya Dedova}
\textbf{Ирина Чумакова} мне не нужны ужасные. Я так с ума сойду. Мне просто нужны истории каждого, кто пережил. И кто хочет рассказать.

\iusr{Ирина Попович}
\textbf{Ирина Чумакова} что же вам пришлось пережить! Это даже в страшном сне невозможно представить! Сил вам и всем мариупольцам.
\end{itemize} % }

\iusr{Olga Kontareva}

Очень многие, с кем сейчас приходиться общаться, не за долго до начала войны
чувствовали большую тревогу, но ни кто не мог её объяснить. Ни кто и подумать
не мог, что через несколько дней в нашем городе начнётся ад. Самый страшный ад
не только для нашего города, но и для всей страны.

\begin{itemize} % {
\iusr{Natalya Dedova}
\textbf{Olga Kontareva} расскажите Вашу историю, Олечка! 🙏

\iusr{Olga Kontareva}
\textbf{Natalya Dedova} \textbf{Natalya Dedova} Не хочется. Главное что выжили. И всё стараешься забыть. На фоне других нам повезло больше. Почему-то, кроме страха, я не могу вспомнить ничего.

\iusr{Natalya Dedova}
\textbf{Olga Kontareva} моя работа. ❣️ Искать истории.🙏

\iusr{Olga Kontareva}
\textbf{Natalya Dedova} \textbf{Natalya Dedova} Понимаю. Может придёт время, и я своей поделюсь.

\iusr{Ксения Ксю}
\textbf{Olga Kontareva} 

мне муж до сих говорит, часто вспоминаю, как ты в 20 числах февоаля говорила мне
,что Рона-это наша собака, как с ума сошла, стала беспокойная очень, наверное
чувствует что то. А я совсем не помню этого. Зато 24 февраля 5 утра не забуду
никогда в жизни😢😢😢Это чувство.....

\iusr{Анна Ныщик}
\textbf{Ксения Ксю} 

наш пёс так же, ещё за пару месяцев до начала ночами выл. Где-то под утро
уже, часов после 4х. Почти каждую ночь, гавкал, а потом начинал подвывать.. но он
у нас во дворе бегал, думали может другие собаки по улице бегают. Утешали
себя, не хотели думать, что это к чему-то плохому.

\iusr{Татьяна Щербань}
\textbf{Olga Kontareva} 

согласна, но что значит на фоне други, Мариуполь пострадал не легче других,
Мариуполя по сути нет, не считая модульные дома педи₽астов😡

\iusr{Лилия Холявко}
\textbf{Olga Kontareva} 

вспоминаю, как в феврале резко стали продавать квартиры с отличным ремонтом и
сравнительно недорого. А за три дня до войны резко опустел город, машины
пропали. Я мужу тогда ещё говорила, ну что - то не то. Кто бы мог подумать!

\iusr{Olga Kontareva}
\textbf{Лилия Холявко} \textbf{Лилия Холявко} Мы и в страшном сне не могли предположить, что такое сможет произойти с нашим городом.

\end{itemize} % }

\iusr{Ольга Баранова}

О, Господиииии... Как же ж страшно на самом деле просто читать обо всём этом и
вспоминать МИРНОЕ 23-е февраля, не по зимнему солнечный и тёплый день, который
усилиями рашистов был превращён в 24-е,войну,весь этот кошмар и ледяной стылый
март в холодных неприспособленых подвалах, с рушащимися как карточные домики,
домами, с гибнущими людьми... А как ЭТО ПЕРЕЖИТЬ?! И выйти из этого ада
уцелевшим физически и морально?! Боль, страшная нечеловеческая боль и слёзы
бессилия...

\iusr{Силинская Ирина}

Если бы с верхов не утверждали что город защищен!!! Наверное многие бы выехали....

\iusr{Татьяна Тимотина}

Детей и стариков... и многих, всех остальных жителей... Но это уже никогда не
исправить, не изменить, не помочь... и не пережить... Навсегда останется это у
нас в памяти, в сердце, перед глазами - у каждого мариупольца💔💔💔

\iusr{Екатерина Воробьёва}

Просто вернуться.. в наш Украинский Мариуполь.

\iusr{Надежда Панюшкина}

Всем выжившим Мира!!! Ушедшим царствие небесное... Не забудем!

\iusr{Alexander Matchay}

Да, каждый по своему это переживает. У меня ещё в городе было стойкое ощущение,
что если бы знать что моя смерть остановит этот ужас и спасёт город и его
жителей - я бы пошёл на это. Вот уж действительно, лучше бы остаться в глазах
других безумным чудаком, но чтобы война не началась. То, что пережили мы - как
будто какой-то разрыв реальности, то, что не должно происходить.

\begin{itemize} % {
\iusr{Natalya Dedova}
\textbf{Alexander Matchay} жду Вас на интервью! 🙏❣️🇺🇦
\end{itemize} % }

\iusr{Natalia Koftarova}

А я мысленно езжу на машине по городу. Вспоминаю, свои маршруты, родные улицы,
что было здесь, что было там. Иногда успокаивает. Нам тоже говорили 24 числа -
уезжайте! Мы не верили. Типа 14 год пережили, и сейчас переживём. Я думала, что
уезжаю из дома дня на 4-5. А получилось так, что незнаю насколько 😔

\iusr{Ольга Атаманюк}

Мой двор... (((

\iusr{Olga Okara}

Как больно, как же больно... что никого уже не вернуть.. что никого уже не обнять ...

\iusr{Лилия Холявко}

Как все правильно сказано! Но услышат ли люди? А больше всего поражает, что
оставшиеся винят во всех бедах Азов. Так больно это слышать. Ведь,
действительно, деньги, шмотки- ничто по сравнению с жизнью!

\begin{itemize} % {
\iusr{Юлия Золотарева}
\textbf{Лилия Холявко} Лиличка, услышат только те, кто хотят, кто пережил и испытал на своей шкуре. Натура человеческая так устроена.

\iusr{Ирина Война}
\textbf{Лилия Холявко}, 

а самое главное, что те кто ругает Азов ни словом не вспоминает русских и
днровцев, как будто их в городе не было и весь ужас творил Азов. Меня это часто
злит и заставляет искать причину такого их поведения.

\iusr{Лилия Холявко}
\textbf{Ирина Война} 100\%
\end{itemize} % }

\iusr{Natalya Dedova}

Друзья, жду ваши истории! Пожалуйста, поддержите меня! 🙏❣️🇺🇦

\begin{itemize} % {
\iusr{Екатерина Воробьёва}
\textbf{Наталья Дедова}, Наташа, мне кажется, что по сравнению с вашей историей моя не заслуживает внимания. Но могу рассказать.

\iusr{Natalya Dedova}
\textbf{Катерина Воробйова} Катя, мы не сравниваем, у кого какая история. У каждого - своя. Когда Вы готовы?
Скайп нужен.

\iusr{Darya Pieshkova}
\textbf{Наталья Дедова} 

нам чудом удалось выехать 5го марта. Через обстрелы и в прямом смысле по
трупам, я до сих пор не избавилась от чувства вины что выехала раньше чем
большинство, я до сих пор не могу смириться, что больше никогда не вернусь в
прошлую жизнь, я каждый день все больше ненавижу рф, я каждый день все больше
думаю, что не вижу впереди дороги...

\iusr{Ірина Сергадєєва}
\textbf{Natalya Dedova} 

Натуся, я ще в середине февраля просила своих уехать, также просила это сделать
24 и 25 февраля.... но тщетно, не хотели покидать родной город, думали, что
пересидят как-то... Никто даже в страшном сне себе представить не мог, что
москали будут всех беспощадно уничтожать(((((

\end{itemize} % }

\iusr{Юлия Золотарева}

Читаю и плачу. Моя любимая 9ти этажка, моя квартира горит. И я там находилась.
И ничего не смогла сделать. Пыталась, но ничего не спасла. Мама на глазах
погибла, а я живу...с этим всем.

\begin{itemize} % {
\iusr{Екатерина Воробьёва}
\textbf{Юлия Золотарева} 

Юля, это самое удачное, что могло произойти в жизни. Сочувствия мало. Да оно
тебе боль не облегчит. Не знаю какими словами тебе помочь. Держись. Цём.

\iusr{Stefaniia Gav'iak}

Тримайтесь! Дуже вам всім співчуваю. Почитаю ваші історії і тоді дивлюсь на
саою квартирку і думаю, яке це жахіття стати бездомним. Але потрібно ще вижити.
В моєї племінниці в двохкімнатній ужгородській квартирці живе її дядько з
дружиною з Бахмута. Добре, що вчасно приїхали. Квартири трьохкімнатної їхньої
уже немає. Її матері квартира в Бахмуті була також, якраз євроремонт закічили
до

Нового року - теж уже немає. Жах просто...

\iusr{Юлия Золотарева}
\textbf{Катерина Воробйова} 

Катюша, это такая дикость. Не верится, но это не сон. На моих глазах все
происходило. Эту боль потери матери и дома при таких обстоятельствах никак не
передать.

\iusr{Natalya Dedova}
\textbf{Юлия Золотарева} примите мои соболезнования!
Если готовы - расскажите свою историю. 🙏🇺🇦

\iusr{Natalya Dedova}
\textbf{Юлия Золотарева} у меня муж погиб. Тоже на глазах. И в этом аду я живу почти полгода. 🥺

\iusr{Екатерина Воробьёва}
\textbf{Юлия Золотарева} 

я умом понимаю. Но как это можно пережить не знаю. И каждый день кажется что
это плохой сон. Но нет. Блин. Не сон. Но плохой. Держись Юлька, ради дочки,
ради папы, ради дальнейшей жизни. А Мамочке твоей Царствие Небесное.

\iusr{Екатерина Воробьёва}
\textbf{Юлия Золотарева}, 

Юля, расскажи \textbf{Наталья Дедова}, она журналист из Мариуполя. Пишет о
каждой истории. Твоя, к сожалению, тоже должна быть опубликована. Чтобы весь
мир знал о трагедии в нашем Городе и в Стране.

\iusr{Юлия Золотарева}
\textbf{Катерина Воробйова} 

Катюша, самое обидное, что тела мамы нет( все сгорело), и факт смерти не могу
доказать. То, что я видела все своими глазами, для суда не аргумент.

\iusr{Екатерина Воробьёва}
\textbf{Stefaniia Gav'iak}, 

а в мене дом уцілів. У Маріуполі. І вівчар Атос мене там чекає. І сусід, котрий
годує мого собаку. І я більш за все хочу скорішеї перемоги і повернення у
Маріуполь.

\iusr{Екатерина Воробьёва}
\textbf{Юлия Золотарева}, 

набери меня как будет возможность. Так не может быть. Хотя вчера я в ТГ видела
дом на Московской, с черепом и т.д.

\iusr{Екатерина Воробьёва}

Ты наверняка тоже видела

\iusr{Юлия Золотарева}
\textbf{Катерина Воробйова} 

Катюш, завтра наберу тебя. Я уже не могу, мозг взрывается от всего. И проблемы жизненные тоже никто не отменял

\iusr{Надежда Панюшкина}
\textbf{Юлия Золотарева} Юлия, здравствуйте. Я вам очень соболезную. Держитесь, нужно Жить... Будем Жить!

\iusr{Natalya Dedova}
\textbf{Юлия Золотарева} мы доказали в суде факт смерти. Без справки от врача. По показаниям свидетелей.

\iusr{Stefaniia Gav'iak}
\textbf{Екатерина Воробьева}. 

Я рада за вашого Атоса і за ваш будиночок! Все так і буде, як ви мрієте! Ми всі
кожний день дивимся, де то вже наші, наскільки просунулись вперед і тішимся
кожному населеному пункту. Я жила до школи в Білозерківському районі
херсонщини, до сих пір памятаю свої дитячі сни про нашу біленьку хатку там.
Херсонщина моїй сім'ї дуже рідна. Моя мама за нею дуже тужила.

\iusr{Юлия Золотарева}
\textbf{Надежда Панюшкина} Спасибо, Надежда! Жизнь продолжается. И мы живем, несмотря ни на что.... И будем жить хорошо!

\iusr{Екатерина Воробьёва}
\textbf{Stefaniia Gav'iak} 

та то така собі радість. Я вже майже півроку переймаюся за корм йому. Дякувати
Богу, є друзі, куми, котрі відвозять і навіть грошей не беруть. У мене хоча б
люди близькі та рідні живі. То все інше проблеми..

\iusr{Екатерина Воробьёва}
\textbf{Юлия Золотарева} Настроение жить. Юля, ты хочешь и можешь)

\iusr{Юлия Золотарева}
\textbf{Наталья Дедова} я свидетель. Но дело у адвоката уже 4 месяца. Сама уже все изучила, все знаю, но не получается. Если не трудно, напишите в личку

\iusr{Надежда Панюшкина}
\textbf{Юлия Золотарева} Так и будет!!! Сил, Веры, Надежды!!!

\end{itemize} % }

\iusr{Юлия Пясецкая Симонова}

Я тоже тоскую по нашему гнездышку, так хочется домой, так хочется покоя, но
дома уже нет, остался только пепел и воспоминания. Оставшиеся в городе
пресмыкаются перед рашистскими палачами, считают их освободителями. Меня это
просто убивает. Как же так? Как люди могли так быстро перестать быть людьми?
Как они могли продаться, словно Иуда. Неужели они не понимают, что с рашкой
будет путь в никуда. Они уже говорят, что у вас там на Украине клоун и
наркоман. Цитируют методичку кремля. Это массовое помешательство? Думаю, что
нет, все они всегда были предателями и ждали момента, когда можно будет
проявить свою личину. Как жить после деоккупации с этими людьми? Постоянно
оглядываться? Вопросов больше, чем ответов

\begin{itemize} % {
\iusr{Natalya Dedova}
\textbf{Юлия Пясецкая Симонова} вы абсолютно правы! На все 💯!

\iusr{Наталия Кладити}
\textbf{Юлия Пясецкая Симонова} 

А кто Вам сказал, что оставшиеся в городе пресмыкаются? Кто дал Вам право судить
людей, тех кто остался в городе. Жизненные обстоятельства бывают разными, не
судите, дане судимыми будете.

\iusr{Юлия Пясецкая Симонова}
\textbf{Наталья Дедова}, 

это ранит сердце и нет им прощения. Они говорят, что все хорошо в городе и
жизнь налаживается. А как же погибшие, как же раненные, как же разбитые судьбы?
Неужели так быстро это забывается?

\iusr{Юлия Пясецкая Симонова}

Значит, Вам повезло, что Вы не знаете таких личностей. Достаточно почитать в
Телеграмме паблики мариупольские и станет все ясно по комментариям. Ну а мне
ещё и с близкими повезло, скажем так

\iusr{Юлия Золотарева}
\textbf{Наталья Дедова} я уже давла 2 интервью. Биби си приезжало и еще одна компания из Англии. Биби си ролик по всему крутили.

\iusr{Natalya Dedova}
\textbf{Юлия Золотарева} я тоже давала. Многим. И уже устала. Морально. Очень сильно. И я Вас пойму, какое бы решение Вы не приняли. Но я очень хочу записать Вашу историю. Если Вы согласитесь. 🇺🇦

\iusr{Ирина Война}
\textbf{Юлия Пясецкая Симонова}, 

я тоже думаю, что многие оставшиеся в городе присмыкаются из за шкурных
интересов. Одни мечтают, что им быстрее дадут квартиру вместо разрушенной
старой, другиие еще чего то хотят или боятся поэтому присмыкаются. Ну не тупые
же они настолько чтобы хвалить русскиий мир, который принес в город разруху.

\end{itemize} % }

\iusr{Natalie Natalie}

18 или 19 февраля мне позвонил двоюродный брат из соседнего государства, и
сказал так просто: уезжай, будет война! Я в ответ сказала: -Ты с ума сошел,
никогда не поверю что ,, дружественная россия,, нападет??!! Не
поверила, поверила 24 февраля, когда началось, но мы ещё не знали что впереди
через неделю наступит ад.... ад, когда просто молились то ли чтоб сразу
убило, без мук, то ли спастись, спаслись чудом, наверное, там наверху у Бога свои
планы на меня, мою семью, какие - не знаю, они меня пугают, я всю жизнь жила в
своей среде, в своем окружении, а теперь меня как морковку выдернули и бросили в
море: смогу - выплыву, нет, значит нет! Нифига время не лечит!!! Это кто то
придумал так, для самоуспокоения.... Чем больше времени проходит тем тоска
страшная находит, до безумия страшная, когда до сих пор снятся обстрелы, самолёты
и квартира, которой больше нет... И кот, которого тоже больше нет... и каждое утро
мне кажется, что это все не правда, что это просто ужасный сон, и думаешь, что
проснешься - и ты дома, и все как раньше....

\begin{itemize} % {
\iusr{Natalya Dedova}
\textbf{Natalie Natalie} я тоже так думаю. И молюсь все е время, чтобы а ем нашим там, на небе, было лучше, чем нам здесь🙏🙏🙏

\iusr{Юлия Золотарева}
\textbf{Natalie Natalie} Наталия, я тоже говорю, что была в аду. Но поймут только те, кто там был. А я все прочувствовала ВСЕ и уехала только в июне

\iusr{Юлия Золотарева}
\textbf{Наталья Дедова} мы должны жить. Несмотря ни на что

\iusr{Natalya Dedova}
\textbf{Юлия Золотарева} должны. Но это так тяжело

\iusr{Natalie Natalie}
\textbf{Юлия Золотарева} да, кто прошел Маріуполь, тот поймёт...

\iusr{Юлия Золотарева}
\textbf{Наталья Дедова} на небо мы еще успеем.. нам сохранил кто то жизнь. Для чего-то.

\iusr{Natalya Dedova}
\textbf{Юлия Золотарева} я была в шаге от неба. В шаге от балкона, который рухнул после прилёта снаряда..

\iusr{Юлия Золотарева}

Как бы больно это не было, но мы живы для чего-то или ради кого-то

\iusr{Юлия Золотарева}
\textbf{Наталья Дедова} а я родилась 2 раза с 21 на 22 марта. Я там должна была умереть. Но нет.

\iusr{Ирина Война}
\textbf{Natalie Natalie}, 

а я сама поняла, что скоро будет что то серьезное после того как случайно
увидела ролик примерно числа 17 февраля о том, что из ДНР эвакуируют в Россию
женщин, стариков и детей. Такого ни разу не было начиная с 2014 года. Но из за
того, что при переезде из одной области в другую требовали делать вакцину, а я
была категорически против этих уколов, я решила никуда не ехать и надеялась,
что все быстро закончится как в 2014. Потом очень пожалела. Знай я, что такое
начнется, то лучше бы наняла частную машину и уехала из города.

\end{itemize} % }

\iusr{Ilona Lozova}

Этот кошмар никогда не забыть, наша власть нас просто бросила, хотя все знали, что будет ад!

\iusr{Татьяна Тимотина}

Я дома, в городе летом, один раз посмотрела из окна... больше не смогла, а
фотографировать тем более. Смотришь и понимаешь, там больше никого нет и не
будет, кто живой и куда уехал не знаешь, встретимся когда-нибудь или нет-не
знаешь.. Район мертв, выжжен, разбит всем чем только можно и все
чёрное, развалено: висят диваны, детские манежи, вещи и на улице кресты, кресты...

\iusr{Svetlana Snegovaya}

Наташенька, и ты думаешь тебя бы послушали, нет. Так получилось я уехала из
города 17 февраля, должна была вернуться 1 марта. Когда началась война, я
умоляла своих бросить всё и выехать, но мне сказали, скоро всё закончится.
Слава Богу живы, но , что пережили.

\iusr{Olena Lavr}
💔

\iusr{Olena Lavr}
💔

\iusr{Лариса Чаговец}

Наташа.... Как написано... До мурашек.... Не поверили б..... К сожалению ....
Так в жизни часто бывает.... Светлая память погибшим... Желания жить и сил
обрести другую новую но по своему счастливую жизнь выжившим!!!!!

\iusr{Елена Романюк}

Оля, если бы мы знали..

\iusr{Anna Kondrateva}
\textbf{Наталья Дедова} 

Мені досі складно повірити у те, що трапилось із нашим містом, із нашими
людьми... Хоча я все це бачила, я жила в цьому... Хочеться закрити очі і забути
усе те, що було, та повернутися у наше мирне минуле. Ви такий текст написали,
як із кіно. Боляче дуже від того, що нічого загального з кіно він не має, що
усе це наша важка реальність... Тримайтеся...

\begin{itemize} % {
\iusr{Natalya Dedova}
\textbf{Анна Кондратьєва} це текст моєї колежанки Надії Сухорукової.

\iusr{Anna Kondrateva}
\textbf{Наталья Дедова} вибачте, не помітила одразу...

\iusr{Natalya Dedova}
\textbf{Анна Кондратьєва} її фейсбук блокує.
\end{itemize} % }

\iusr{Яна Шкляревская}

Рыдаю над каждым словом, над каждым комментарием ((((. Как это больно, очень
больно 😢. Нужно жить, мы выжили, мы должны жить! Жить ради детей, жить для
победы в этой войне, пока мы живы, жива память о наших близких и родных людях,
о тех кого уже нет.

\iusr{Екатерина Кибкало}

А я долго была \enquote{неверующая Фома}... Вот вообще мозг отключился. Боялась только
за Сартану. Ни продукты не закупили, спокойно передвигалась по городу впервые
дни.... Поняла наверное, что близится ад, когда отключили отопление. А ещё
страшно стало, когда сестра с Сартаны хотела выехать в село к родителям, а с
города не выпускали на украинских блокпостах.... Когда сосед хотел выехать, и
вернулся, т. к впереди машины начали расстреливать на блок постах. Тогда стало
страшно... Почему не выпускают? И выехать мы смогли, когда дошла информация, что
разбили украинский блок пост.... И проезжая мины, мы смогли выбраться. Недавно
перечитывала посты с группы Маріупольська міська рада, все внушали, чтоб не
боялись, все прекрасно, мы выставим, но к сожалению, мы не выстояли(((((

\begin{itemize} % {
\iusr{Natalya Dedova}
\textbf{Екатерина Кибкало} 

нам внушали. И успокаивали. Так и было. А теперь - порядка 100 тысяч погибших.и
центры Я Мариуполь по всей стране. А люди - возвращаются в родной город, потому
как денег нет, чтобы снять жилье. За консервы с макаронами квартиру не снимешь.

\iusr{Tatyana Zehova}
\textbf{Екатерина Кибкало},

все мы верили, что нас спасут ..а нас не спасли... выходили из города пешком по
берегу моря 16 марта ..в этот день разбомбили драмтеатр, шли пешком 6 часов, по
пути друга ранило осколком в ногу, но он все равно шел.... потом началась новая
пустая жизнь и реабилитация....

\iusr{Люлька Владимировна Синдицкая}
\textbf{Екатерина Кибкало},

прочитали бы побольше людей Ваш комментарий, а то многие не верят, что нас укр. армия не выпускала...

\iusr{Stefaniia Gav'iak}

Я читаю і не може зрозуміти чому не випускали? В яку сторону не випускали? Як
таке могло бути? Мої знайомі виїхали спокійно в перші дні, а потім трохи
пізніше і решту сім'ї забрали з Маріуполя. А подруга моєї сестри в Сартані
згинула в своєму прекрасному будинку, який збудувала і він утопав в
квітах... Вони продружили двадцять років, познайомившись в Італії. Вона мала
заїхати буквально на днях до сестри по дорозі в Італію але почалась війна... І
згинула... Сартану в першу дні війни і ще два населених пункти зрівняли з землею
рашисти. То може тому не пускали туди? !Як пояснити те, що ВСУ не пускало?

\iusr{Наталя Рижкова}
\textbf{Stefaniia Gav'iak}, Стефанія, поясню в приватних !

\iusr{Stefaniia Gav'iak}

Дякую.

\iusr{Ирина Война}
\textbf{Наталья Рыжкова} 

и мне поясните если знаете. А то многие в инете говорят, что их и из подвалов
украинцы не выпускали. По нашему двору украинские военные при мне пости 2
недели ходили, но им не до нас было. Наши люди из подвала туда сюда ходили,
одни как я раз в 2-3 дня, другие по несколько раз в день ходили. К нашим
женщинам часто приходили в гости сыновья из других районов, например, с 17
микрорайона к нам на пр. Мира, военные никого не задерживали. Про украинские
блокпосты ничего не знаю.

\end{itemize} % }

\iusr{Радмила Кривоніс}
\textbf{Наталья Дедова}

Обіймаю вас💔

\iusr{Yulia Loyuk}

Мы все стали другими и прежними не будем никогда, как никогда наш прекрасный
Украинский Мариуполь не будет таким, который мы помним. Мы все прошли, через
ад ,жизнь продолжается, но она стала какой то искусственной, чужой. Не
осталось ничего, даже в безумной панике не смогла забрать альбом с фото, дом
сгорел до тла, остались лишь воспоминания о счастливой жизни. А как же хочется
домой...

\iusr{Christina Zheleznyak}
😭😭😭

\iusr{Марина Лобода}

А мы не из Мариуполя. Но я вас всех очень понимаю. Мы из Чернигова. Остались
живы, но потеряли дом. Купили осенью, переехали 17-го декабря, а 6 марта его
разрушили.

\iusr{Lena Zhazhkova}

вам бы никто не поверил. как и мне в своё время(((

\iusr{Tatyana Zehova}

Какой правдивый пост... сочувствую вам. Очень понимаю. И про ледяную весну, и
про запасы воды, и про деток, которых надо было вывезти...

Домой хочется, что сердце разрывается....

\iusr{Ира Лушпай}

\ifcmt
  igc https://i2.paste.pics/9a08d1284911118e6f3b79f752607830.png
	@width 0.1
\fi

\iusr{Алёна Романская Дикая}

23 февраля вечером после работы я покупала новые тарелки... Не верили

\iusr{Анна Попова}

Да, все правда. Ведь мы до последней минуты не верили что война начнется.
Верили в людское благоразумие, а нет. Сколько же горя принесла эта война.
Лишила всего: и дома, и друзей, и Родины. Я с Мариуполя и все ужасы пережили
всей семьей пока чудом не удалось вырваться из города

\iusr{Ольга Дудник}

Да вот только, просить чтоб мы все запасались едой и водой, должн был наш сбежавший на третий день войны мэр....

\iusr{Денисенко Юля}

Здравствуйте, Наталья! Мы можем чем-то помочь?

\iusr{Marina Chuiko}

Мы даже не верили и на 2-й, и на 3-й день... Пока эти ужасы не начали пробираться
к нам в дом, к нам в душу сначала в виде разбитых стекол, потом в виде
холода, потом в виде авиаударов, потом звуков жутких выстрелов и взрывов, потом
темного подвала, кричащих детей, леденящего душу ужаса, потом при виде
военных..., а потом потерь... своих родных💔

Теперь их нет. Ни мужа, ни мамы.. А я боюсь заходить и возвращаться домой (больше
ни одну квартиру я не могу назвать \enquote{Дом}), потому что там все, как мы
уходили.. Вещи моих родных, которых нет, наша счастливая жизнь, которой тоже нет.

Как бы я тоже хотела вернуться в 23 февраля, в срЕду. И навсегда остаться там,
в этом дне недели. Навсегда😥💔

\iusr{Багдасар Галина}

\ifcmt
  igc https://i2.paste.pics/07ffbecd171a912d2b9b27a9063351bb.png
	@width 0.2
\fi

\iusr{Ирина Деревянко}

Чайник.... Вспомнила, что у нас был такой же. Только появился позже, когда уже
вышли из подвала, в начале апреля. Мы так радовались ему, что будем, наконец-то,
кипятить чайник, а не кастрюльку носить на шестой этаж от костра. Соседи
"обзавидовались", он ещё и со свистком был.🙂 Звучал, как музыка...

Да, страшное было "кино"... и мы все в главной роли.🥺

\ifcmt
  igc https://scontent-frt3-2.xx.fbcdn.net/v/t39.30808-6/305648644_1142967572962205_5151845187723845661_n.jpg?stp=cp6_dst-jpg&_nc_cat=108&ccb=1-7&_nc_sid=dbeb18&_nc_ohc=70t6usorEwoAX808vr3&_nc_ht=scontent-frt3-2.xx&oh=00_AfANsDOf9AlfBs41RDgHmkM1X8eERDjzcuyRwFWNDmNAGg&oe=643756C7
	@width 0.5
\fi

\begin{itemize} % {
\iusr{Татьяна Амелина}
\textbf{Ирина Деревянко} а у нас в соседнем дворе был чайник с таким пронзительным свистком, что мы приседали

\iusr{Ирина Деревянко}
\textbf{Татьяна Амелина}

Ого! Мой свистел по-доброму))) Я его у родителей в подвале нашла, как ненужный,
думала, что сломан свисток и даже обрадовалась, когда он неожиданно "запел" на
весь двор.)

\iusr{Natalya Dedova}
\textbf{Ирина Деревянко} мне рассказывали в интервью историю о чайнике со свистком. Люди перепугались, когда он закипел. И больше чайник не свистел.
\end{itemize} % }

\iusr{Наталья Олейник}

Дуже важливо щоб хтось про це повідомив. Я не можу це вимовляти, як ні на папір
ні на словах. Я є Маріуполь!!!!!; Мені важко на чужбі . Я хочу до дому

\begin{itemize} % {
\iusr{Tatyana Zehova}
\textbf{Наталья Олейник} я тоже хочу....безумно хочу

\iusr{Natalya Dedova}
\textbf{Наталья Олейник} розкажіть мені, будь ласка. Я допоможу!

\iusr{Марина Григорьева}
\textbf{Наталья Олейник} 

я тоже очень очень хочу домой. Мариуполь - это Украина!!! Я верю так будет и мы
все вернемся в украинский Мариуполь. Защитники Мариуполя стояли до упора на
защите города за каждый кусочек земли,за каждую травинку. До упора ждали
подмоги......но все сложилось как сложилось... Главное я верю в нашу ПОБЕДУ и я
верю в ЗСУ!!!

\end{itemize} % }

\iusr{Tatyana Zehova}

Поддержим, обещаю! Ради твоего мужа! Лучших фотографий про город я ещё не встречала!!!

\iusr{Татьяна Тетяна}
🥺😭😢

\iusr{Ольга Гошкодеря}

можете не переживать за город, строится и восстанавливается, как говорят встает
с колен, все нормально,,

\begin{itemize} % {
\iusr{Євдокія Фрушичева}
\textbf{Ольга Гошкодеря} а кто его на колени поставил? Что не устраивало в городе до 24 февраля?

\iusr{Ольга Ященко}
\textbf{Ольга Гошкодеря} С каких колен? Наш город на колени не стал! Его уничтожили, но не поставили на колена!

\iusr{Yulia Omelchenko}
\textbf{Ольга Гошкодеря} Браво!!!

\iusr{Татьяна Гетманская}
\textbf{Ольга Гошкодеря} Вы серьезно? Не переживать!!!

\iusr{Xenia Yaylova}
\textbf{Ольга Гошкодеря} за убитых тоже переживать незачем, верно? У вас точно все нормально?

\iusr{Victoria Mironenko}
\textbf{Ольга Гошкодеря} такі мразі, як ти, будете збирати речі и валити з Маріуполя до рашки! Інакше будете сидіти, суки, за гратами!!

\iusr{Ольга Лихачева}
\textbf{Ольга Гошкодеря} 🤮

\iusr{Ольга Лихачева}
\textbf{Ольга Гошкодеря} 💩💩💩💩

\iusr{Anastasiia Bohach}
\textbf{Ольга Гошкодеря} той випадок, коли вважаєш, що нижче вже немає куди, але ні... знизу постукали 🤯
\end{itemize} % }

\iusr{Voevoda Lyuba}

Мой дом... 🔥💔

\begin{itemize} % {
\iusr{Александра Алферова}
\textbf{Люба Воевода} 😢💔

\iusr{Нина Муравченко}
\textbf{Люба Воевода} пр.Мира 105 ?
\end{itemize} % }

\iusr{Sergey Vaganov}

Я на митинге у театра 22 февраля обнимался со всеми кук будто прощался.

\begin{itemize} % {
\iusr{Татьяна Иванова}
\textbf{Sergey Vaganov} 

не была на митинге, не видела смысла, что это поможет хоть в чем то. Показуха
организационная, лучше б собрали народ на благоустройство бомбоубежищ. Сами
менты и сбушники вывезли свои архивы накануне. А нам - "идите, попросите не
начинать войну" на митинг!!!

Все для отвода глаз: еще и хозяин азовстали в мячик на публику поиграл, типа, не волнуйтесь, горожане!!!

\iusr{Sergey Vaganov}

Митинг собрала в сети \href{https://www.facebook.com/maryna.moloshna}{Марина Молошна}. С утра написала в ФБ, а вечером собрались
те, кто хотел. Поверьте, не для показухи. Тем более "организационной".
Насколько я помню, - от горсовета там не было никого. С точки зрения
рациональной и в Донецке на митинги в 2014 не было смысла выходить. Вы напхали
в кучу и сбушников с ментами, которые не просили нас: " "идите, попросите не
начинать войну" на митинг!" и Ахметоа с мячиком. Не вышли - ваше дело. Я хотя
бы обнялся напоследок с хорошими людьми.

\end{itemize} % }

\iusr{Светлана Братишко}
🥺

\iusr{Татьяна Комиссарова}

Ты очень сильная, сыночек остался с тобой. Мы ещё вернёмся домой, возможно не
скоро, но очень этого хочется.

\iusr{Екатерина Домбровская}

Да я тоже хочу 23 февраля кода мы с дочкой Полиной и Ярославом поздравляли
нашего папу с праздником. А утро 24 февраля поделила нашу жизнь до и после. Я
хочу пожелать всем нам жителям Мариуполя душевного спокойствия сил мне очень
трудно это писать потому что я сама и мои близкие пережили весь этот ужас но я
верю что рано или поздно мы все вернёмся в наш любимый Мариуполь всем здоровья
и терпения

\begin{itemize} % {
\iusr{Natalya Dedova}
\textbf{Екатерина Домбровская} Екатерина, а с каким праздником вы поздравляли мужа?
\end{itemize} % }

\iusr{Inna Lysenko}

Очень жаль 🥲

\iusr{Аня Кот}

в моїй маріупольській квартирі був такий самий чайник.

\iusr{Ольга Бранцира}

Никто не думал, что будет такой ад, не поверили бы никому!

\iusr{Александра Столяр}

После нашего с Вами интервью я долго анализировала почему многое даже не
озвучила: как две недели подвала молилась 24/7, как ужасно хотелось
помыться, как каждый вечер наступал панический страх и меня трясло как при
высокой температуре и многое другое..... и я поняла: у нас у всех одни
воспоминания, с одними и теми же эмоциями и страхами... и это всё уже до меня
мариупольцы проговорили. Это никогда не забыть, с этим нужно научиться как-то
жить 💔

\iusr{Yana Bedrataya}

Сестра умоляла меня уехать, билась о нас как о стенку горохом. А потом утопала
в слезах неделями не зная живы мы или нет... если бы мы поверили... сколько жизней
могли бы спасти...

\begin{itemize} % {
\iusr{Анна Растягаева}
\textbf{Yana Bedrataya} весь Мариуполь отказывался верить... помню, как сидели с дочкой пили чай ещё в своей целой квартире, но уже при свечке и говорила ,,Евусенька ,ещё день -два и все закончится ...."
\end{itemize} % }

\iusr{Вікторія Вовченко}

Сегодня мне сказала дочь: \enquote{Мама, я только сейчас начинаю осознавать как это было
страшно и какой ужас был вокруг нас}. Шесть месяцев понадобилось чтобы осознать
и вся жизнь чтобы забыть......

\begin{itemize} % {
\iusr{Елена Михайловна Ляхова}
\textbf{Виктория Вовченко} а забыть не получится. Я в другой стране, и на каждый звук самолета поднимаю голову..
\end{itemize} % }

\iusr{Анна Растягаева}

Я до сих пор не могу понять, как мы все это пережили..... и не разучились радоваться жизни.

\begin{itemize} % {
\iusr{Елена Терехова}

Анюта, блин... а я не могу до сих пор начать радоваться... Появляются какие-то
просветы, но буквально через мгновение возвращаюсь в реальность. Даже музыку в
машине не могу слушать...(((

\iusr{Таисия Омельянчик}
\textbf{Елена Терехова} ни музыке радуют я люди, а то что живы

\iusr{Елена Терехова}
\textbf{Таисия Омельянчик} , ну, радоваться жизни и радоваться тому, что живы, немного разные понятия...

\iusr{Таисия Омельянчик}
\textbf{Елена Терехова} жив значит жизнь

\iusr{Таисия Омельянчик}

А люди пройдя кошмарный ад будут помнить всю жизнь, но будут стараться жить

\iusr{Ирина Война}
\textbf{Анна Растягаева}, как радоваться, если до сих пор бомбят украинские города постоянно, причем не военных, а гражданских. Сидишь и не знаешь куда и когда попадет?

\iusr{Анна Растягаева}
\textbf{Ирина Война} радоваться освобождению городов, радоваться от хороших весточек друзей, радоваться каждому прожитому дню.
\end{itemize} % }

\iusr{Tatyana Zehova}

Какой красивый и грустный пост. Чувствуется боль от пережитого, хорошо, хоть
хватило мужества фотографировать. Я ни разу не воспользовалась телефоном и
камерой. Мне все время казалось, что я героиня сериала Лост, и картинка щас
должна измениться. Мозг отказывался воспринимать происходящее...

\iusr{Олечка КОЛОМИЙЧУК}

Согласно официальным данным от 24 февраля 2022 года население Мариуполя
составляло 541,300 человек.

Более 21,000 человек умерло в первые четыре месяца войны.

43,000- были депортированы на оккупированные территории и территорию Российской
Федерации.

Более 300.000 человок были вынужлены оракуибоваться.

К концу июня 2022 года почти 150,000 жителей Мариуполя остались в городе. Они
вынуждены выживать в условиях отсутствия электричества, газа и воды. В
настоящий момент в Мариуполе разворачивается ужасная гуманитарная катастрофа.

\iusr{Hanna Kis}

А мы с мамой за сутки до... Гуляли по городу. Ходили в салон, заказали рамочки
на мои картины и мамины вышивки. Зашли в кафе, купили пирожные.

А 24.02 сын вскочил и я. Четыре утра. И он сказал: \enquote{Я, мама, иду за
оружием. Я буду защищать город}
