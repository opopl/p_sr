% vim: keymap=russian-jcukenwin
%%beginhead 
 
%%file moje.prizyv.vstuplenie
%%parent moje.prizyv
 
%%url 
 
%%author_id 
%%date 
 
%%tags 
%%title 
 
%%endhead 

\subsubsection{Вступление}

Добрый день/вечер, или же доброе утро. Не знаем, как лучше начать этот текст,
но давайте все-таки как-нибудь начнем. Мы начали этот текст вечером 28 февраля
2022 года, мы находимся в городе Киеве, мы есть совершенно обычные киевляне, и
как мы все знаем, уже вовсю идет по всей территории Украины кровопролитная
война. Кровопролитная, ужасная война. Война на истощение, война просто на
убой. Война, которую начала Российская Федерация против Украины. Да, это
ужасно, что вторжение, которое многие считали абсолютно мнимым, таки
состоялось. Мы, если честно признаться, смеялись над непрерывным потоком
сообщений в западных и украинских СМИ о готовящемся вторжении, и мы ничуть не
верили, что война таки произойдет. Это мнимое вторжение уже стало просто мемом
в то довоенное время для Киева, с которого как будто прошла уже целая вечность, хотя на
самом деле прошло всего несколько дней. Да, знаете, бывает так, что иногда один
день, даже один час проносится как целый год. Да, мир резко разделился на до, и
после, так же как для Киевлян 1941 года, например. Так вот... то, во что
абсолютно не верилось, таки состоялось. И каждый час, каждый час, где-то кто-то
кого-то убивает, калечит, жжет. Оккупационные войска России наступают по всем
направлениям. Горят здания, горит земля... горит Украина... перестрелки на
окраинах Киева...  бой возле метро Берестейская... бой возле детской больницы
Охматдет... удар по площади Свободы в Харькове...  и плачут человеческие
души...  кто-то плачет в подвале, кто-то орет от боли...  кто-то просто
тихонько рыдает в платок, сидя на коврике и обернувшись пледом на станции метро
Вокзальная в Киеве... и кто-то просто умирает, потому что пуля попала в печень
или селезенку, и его не успели спасти, не успели довезти в госпиталь, и все
прекрасные планы на жизнь обнулились.  Жена, дети, работа, - все исчезло
навсегда... Обычный человек...

\ifcmt
  tab_begin cols=2,no_fig,center
		 %pic https://zkm-v.ru/wp-content/uploads/img_25981.jpg
		 pic http://goldrybak.ru/wp-content/uploads/2016/05/7334614e50a8145fb553e48a8a2d7789.jpg
		 @caption Рыбалка летом, Киев, Украина

		 pic https://golos.ua/images/albums/2014/02/04/view/6alzj1ksLPsVOQQzLqFs8P85.jpg
		 @caption Рыбалка зимой, Киев, Украина
  tab_end
\fi

Ну вот, скажем...  Иван Петрович, 42 года, живет в Киеве всю жизнь... (фотка с
рыбой это не Иван Петрович, это из интернета взято, не волнуйтесь). Мама и папа
- тоже киевляне.  Мама работала учительницей, а папа - на заводе Арсенал, еще
во времена СССР, оптику разрабатывал для космических стартов СССР.  А сам Иван
Петрович работал предпринимателем... держал велосипедный магазин... Самый
обычный мужик...  Любит рыбалку на Днепре; зимой - на льду, летом - с моста
Метро, или же в заливе Десенка, там, где парк Дружбы Народов находится; любит с
друзьями посидеть, водочки попить, и шашлыки поджарить.  Любит исполнять на
гитаре в кругу друзей, - в свое время закончил музыкальную школу, - русские и
украинские песни.  Особенно любит Червону Руту, а также песни Высоцкого и
Шаова.  Пошел в тероборону с начала войны, ведь это его родной город, он тут
вырос, на Русановке или же Вознесенке... Высокой геополитикой не интересовался,
просто примерный семьянин, трое детей, жена, и старенькая мама.  Получил
автомат, и пошел защищать Город и Отчизну. Погиб в районе метро Берестейская.
Или же вот... В Киеве семья на машине недавно пыталась уехать, их расстреляли,
- это вы можете прочитать в новостях. Может быть наши по глупости или ошибке,
может быть ваши, да какая разница... 

\ifcmt
  tab_begin cols=2,no_fig,center
     pic https://aussiedlerbote.de/wp-content/uploads/2022/03/v-telebashnyu-v-kieve-popal-snaryad.jpg
		 @caption Телевизионная вышка в городе Киеве, война, 2022

		 pic https://kievvlast.com.ua/project/resources/2021/09/dDRQDw6B.jpg
		 @caption Памятник жертвам Бабьего Яра, город Киев, мирное время
  tab_end
\fi

Или же удар по телевизионной вышке в
Киеве, в районе метро Дорогожичи, там Бабий Яр рядом, где в годы оккупации
Киева нацистами было расстреляно более 100 000 киевлян, коммунистов,
беспартийных, евреев, и военнопленных. Так вот, там гуляла семья, знаете. Вы,
военное руководство Российской Федерации, выполнили военное задание, попав в
вышку, и произведя взрыв в форме небольшого ядерного гриба, - а вот эти обычные
люди просто погибли, обычная семья была сожжена, от них остались только
обугленные тела.  И что, что тут такого, скажете? Ну, погибли люди, ну и что? А
вот и нет человека, нет людей, вот и все... Люди что-то там планировали,
мечтали, но все оборвалось в один миг...  Знаете, песня такая есть - Есть
только миг, между прошлым и будущим...  и этот миг называется Жизнь. Вы ж ее в
России тоже поете, правда?  Это песня композитора Александра Зацепина на слова
Леонида Дербенёва, написанная для советского фильма «Земля Санникова» (1973).
Звучала также в телесериале \enquote{Перевал Дятлова} (2020). Наиболее
известные исполнители песни: Олег Анофриев, Олег Даль, Николай Расторгуев,
Михаил Боярский, Лев Лещенко и Игорь Наджиев.

\raggedcolumns
\begin{multicols}{2} % {
\setlength{\parindent}{0pt}
\obeycr
Призрачно все в этом мире бушующем
Есть только миг за него и держись
Есть только миг между прошлым и будущим
Именно он называется жизнь

Вечный покой сердце вряд ли обрадует
Вечный покой для седых пирамид
А для звезды что сорвалась и падает
Есть только миг ослепительный миг
А для звезды что сорвалась и падает
Есть только миг ослепительный миг

Пусть этот мир вдаль летит сквозь столетия
Но не всегда по дороге мне с ним
Чем дорожу чем рискую на свете я
Мигом одним только мигом одним
Счастье дано повстречать иль беду еще
Есть только миг за него и держись
Есть только миг между прошлым и будущим
Именно он называется жизнь

Есть только миг между прошлым и будущим
Именно он называется жизнь
\restorecr
\end{multicols} % }

