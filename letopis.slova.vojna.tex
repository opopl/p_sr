% vim: keymap=russian-jcukenwin
%%beginhead 
 
%%file slova.vojna
%%parent slova
 
%%url 
 
%%author 
%%author_id 
%%author_url 
 
%%tags 
%%title 
 
%%endhead 
\chapter{Война}

С \emph{войной} в родной стране никак нельзя примириться. О книге, которая
приобрела новый вкус после семи лет конфликта. Ще у 2018 році на книжковому
форумі купила книгу Серые пчёлы Андрея Куркова!  Гортала, тоді не пішла. Зараз,
коли минає вже сім років цієї жахливої \emph{війни}, а світла в кінці тонелю
влада так і не засвічує, просто ковтнула цю оповідь.  Ця \emph{війна} так
болить ....  Просто замисліться і відчуйте. Не змиряймося з такою ситуацією..
Ніяк. Не можна!,
\textbf{С войной в родной стране никак нельзя примириться},
Марина Ставнийчук, strana.ua, 07.06.2021

%%%cit
%%%cit_pic
%%%cit_text
\emph{Воїн}, який захищає Україну, звертається до дівчини з окопів: \enquote{Я
запрошую тебе, твого чоловіка, офіційно до себе на позицію, де ми
\enquote{прохолоджуємося в тіньочку під деревцем, ножками махаємо} } і додає:
\enquote{Поки ти працюєш, відпочиваєш, гуляєш, спиш вдома поруч зі своїм
чоловіком}.  Ще один \emph{воїн} каже: \enquote{Я кинув роботу, батько трьох
дітей, прийшов захищати свою країну}. Жінки-\emph{військовослужбовці} з фронту
також звернулися до Яни: \enquote{Мороз, погода, непогода, мінус 25. Коли по
тобі гатять дві з половиною години, мінометний обстріл. Коли ти чуєш ці виходи,
чуєш свист, рахуєш ті секунди, доки летить ця міна}. Журналісти хотіли
передати запрошення від \emph{бійців} Яні особисто, але за місцем реєстрації
б'юті-майстрині не знайшли. Де Яна нарощує вії журналісти також не довідались,
офіційних даних про її салон і чи сплачує вона податки немає
%%%cit_comment
%%%cit_title
\citTitle{Українські воїни звернулися до горе-блогерки}, 
gazeta.ua, 14.06.2021
%%%endcit

%%%cit
%%%cit_head
%%%cit_pic
%%%cit_text
Однажды, в какой-то \emph{военный} зимний день, мне приснился сон. Очень странный сон,
должен вам признаться. Словно иду я через степь и разговариваю с железными
травами, некогда бывшими фугасами, танковой броней, минометами и пулями. И
каждая \enquote{травинка} говорит мне примерно одно и тоже - что прожила немалую жизнь
в сырости складов, мечтая о войне. Не то, чтобы мечтала, скорее четко понимала
смысл своей \enquote{жизни}. А после того, как отслужила свое, превратилась в степную
траву. И так ей жить нравится больше, чем существовать орудием для убийств.
Просто жить. В степи, без границ, национальных интересов и реющих флагов
%%%cit_comment
%%%cit_title
\citTitle{Gorlovka.ua: Город, рожденный жить - Блоги}, 
Егор Воронов, gorlovka.ua, 29.07.2016
%%%endcit

%%%cit
%%%cit_head
%%%cit_pic
\ifcmt
  pic https://gorlovka.ua/Images/Upload/NewsArticle/5qzXPLR/_k2haChT5v8dE.jpg
  caption Война присылает Рите новые \enquote{подарки} почти ежедневно
\fi
%%%cit_text
\emph{Война} присылает Рите новые \enquote{подарки} почти ежедневно. \emph{Войска ДНР} базируются
чуть не доезжая села Зайцево. \emph{Украинские силовики} разбили позицию на той
стороне поля, за деревьями — до них метров триста. Раньше их было видно хуже,
из-за сарая во дворе — но недавно дяди постарались, и теперь у сарая нет
задней стенки. Там Рита нашла ещё одну хвостовую часть мины, которая и
разворотила кирпичную кладку. Теперь под камнями лежит покорёженный Ритин
велосипед и коляска. По коляске Рита не скучает, а вот велосипед совсем другое
дело — на нём можно было кататься хотя бы днём, пока дяди опять не начали
взрывать свои взрослые игрушки. Трёхгодовалая Рита помнит, что соседний
перекрёсток им с мамой лучше быстро-быстро пробегать, потому что дяди за
кустами любят иногда устраивать \enquote{бум}, просто забавы ради. Рита понимает, что
это такие правила во взрослых играх. Девочка не знает, когда они наиграются,
но каждый день ранним утром, когда обстрелы стихают, она выбегает на улицу или
во двор искать новые подарки, которые прислала \emph{война}
%%%cit_comment
%%%cit_title
\citTitle{Жизнь в Зайцево глазами российских журналистов из Life News и 3-летней девочки Риты}, 
Gorlovka.ua, gorlovka.ua, 06.06.2016
%%%endcit

