% vim: keymap=russian-jcukenwin
%%beginhead 
 
%%file 04_01_2019.stz.news.ua.lb.1.arhitekturnyj_atlas_dorevoljucijnogo_mariupolja.1.doma_tregubova
%%parent 04_01_2019.stz.news.ua.lb.1.arhitekturnyj_atlas_dorevoljucijnogo_mariupolja
 
%%url 
 
%%author_id 
%%date 
 
%%tags 
%%title 
 
%%endhead 

\subsubsection{Дома Трегубова у Театрального сквера}

Абрам Трегубов был как раз одним из таких мариупольских бизнесменов. Ему
принадлежали разные производственные мощности, а также целый ряд лакомых
кусочков городской недвижимости. К его собственности судьба оказалась весьма
благосклонна, многие дома благополучно перешагнули рубеж ХХ І века. Справа от
драматического театра стоят два дореволюционных особняка, отреставрированных
компанией \enquote{Азовинтекс}, принадлежавших когда-то Трегубову. После революции всё
его имущество реквизировали. В особняке у дороги стало заседать уездное ЧК, а
сам купец стал одним из объектов его внимания.

\ii{04_01_2019.stz.news.ua.lb.1.arhitekturnyj_atlas_dorevoljucijnogo_mariupolja.1.doma_tregubova.pic.1}

В послевоенный период оба здания достались Министерству здравоохранения и долго
служили горожанам в качестве городской поликлиники и станции переливания крови.
Где-то на стыке ХХ и ХХI веков вся территория попала в собственность
\enquote{Азовинтексу}. Более десяти лет компания проводила реставрационные
работы, завершившиеся в 2012 году. За это время у особняка близ дороги появился
третий этаж, башенку накрыли остроконечным конусом, а вот ещё один маленький
одноэтажный дом в переулке исчез. На его месте построили громадный
торгово-развлекатель\hyp{}ный центр, не работающий до сих пор. Выполненные работы
удачно подчеркнули первоначальные черты зданий, и вписали отдельные здания в
целостный ансамбль. Сегодня особняки Трегубова можно смело назвать украшением
центральной части города.

\ii{04_01_2019.stz.news.ua.lb.1.arhitekturnyj_atlas_dorevoljucijnogo_mariupolja.1.doma_tregubova.pic.2}
\ii{04_01_2019.stz.news.ua.lb.1.arhitekturnyj_atlas_dorevoljucijnogo_mariupolja.1.doma_tregubova.pic.3}
\ii{04_01_2019.stz.news.ua.lb.1.arhitekturnyj_atlas_dorevoljucijnogo_mariupolja.1.doma_tregubova.pic.4}
\ii{04_01_2019.stz.news.ua.lb.1.arhitekturnyj_atlas_dorevoljucijnogo_mariupolja.1.doma_tregubova.pic.5}
