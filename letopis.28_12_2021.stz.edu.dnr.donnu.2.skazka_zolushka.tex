% vim: keymap=russian-jcukenwin
%%beginhead 
 
%%file 28_12_2021.stz.edu.dnr.donnu.2.skazka_zolushka
%%parent 28_12_2021
 
%%url http://news.donnu.ru/2021/12/28/nakanune-novogo-goda-studenty-profilya-slavyanskaya-filologiya-posmotreli-kinoskazku-tri-oreshka-dlya-zolushki-na-cheshskom-yazyke
 
%%author_id 
%%date 
 
%%tags skazka,kino,chechia,zolushka,donbass,doneck,studenty,dnr,obrazovanie
%%title Киносказка «Три орешка для Золушки» на чешском языке
 
%%endhead 
\subsection{Киносказка «Три орешка для Золушки» на чешском языке}
\label{sec:28_12_2021.stz.edu.dnr.donnu.2.skazka_zolu28_12_2021.stz.edu.dnr.donnu.2.skazka_zolushkashka}

\Purl{http://news.donnu.ru/2021/12/28/nakanune-novogo-goda-studenty-profilya-slavyanskaya-filologiya-posmotreli-kinoskazku-tri-oreshka-dlya-zolushki-na-cheshskom-yazyke}

В Донецкой республиканской библиотеке для молодёжи прошел показ вот уже полвека
всеми любимой киносказки «Три орешка для Золушки» («Tři oříšky pro Popelku») на
чешском языке. Мероприятие было организовано кафедрой общего языкознания и
истории языка имени Е. С. Отина при участии старшего преподавателя кафедры,
полониста Натальи Александровны Шокотко и студентов третьего курса
филологического факультета профиля «Славянская филология».

Во время просмотра фильма студенты активно обсуждали лингвистические процессы в
польском и чешском языках, находили общие праславянские корни. В очередной раз
все убедились, что для филологов нет ничего невозможного! Эксперимент показал,
что после двух семестров изучения польского языка хорошо воспринимается также и
ещё один язык западнославянской группы – чешский, и наши филологи вполне могут
плавно переходить из полонистики в богемистику!

Спасибо Донецкой республиканской библиотеке для молодёжи за отличную
организацию просмотра фильма. Каждый из зрителей унёс с собой частичку
волшебного настроения в преддверии новогоднего праздника. К слову, на просмотр
этой сказки каждый год только в самой Чехии собираются более 30 процентов всех
телезрителей страны, а зарубежных поклонников из Германии, Норвегии и других
стран и того больше.
