% vim: keymap=russian-jcukenwin
%%beginhead 
 
%%file 28_02_2022.fb.fb_group.story_kiev_ua.2.vypechka.cmt
%%parent 28_02_2022.fb.fb_group.story_kiev_ua.2.vypechka
 
%%url 
 
%%author_id 
%%date 
 
%%tags 
%%title 
 
%%endhead 
\zzSecCmt

\begin{itemize} % {
\iusr{דני כהן}
Никогда не доносил до дома эту корочку целой... ммммммммм....

\iusr{Наталия Озерова}
Улюблений хліб- паляниця!

\iusr{Александр Венге}
\textbf{Natalia Ozerova} а ещё пароль!

\iusr{Наталия Даниленко}
Давно його не коштувала, не печут!

\iusr{Дмитрий Гаранжа}

Самый простой заменитель хлеба.

Что нужно:

Мука, соль вода, любой жир (хотя можно и без жира)

Грамм 300 муки залить примерно 100 граммами кипятка, добавить щепотку соли и
начать перемешивать массу вилкой. (эстетам можно добавить специи типа перца,
куркумы и т.п.)

Когда масса остынет продолжить перемешивание руками. Разделить массу на шарики,
размером с куриное яйцо или мячик для пинг-понга. Раскатать с помощью скалки
или бутылки в блин. Нет скалки, можно руками. Бросить на сковородку с любым
маслом или жиром и обжарить с двух сторон. (Блин желательно проткнуть вилкой,
чтоб не пузырился) Вишенка на торте - выдавить пару зубчиков чеснока с помощью
чеснокодава в чашку, залить кипятком и сбрызнуть этой жидкостью готовые хлебные
лепешки

Те, кто попробует эти хлебные лепешки задумаются, а стоит ли вообще покупать
хлеб.

Если нет масла, жира и т.п. можно запечь лепешки в духовке или микроволновке,
но будет суховато и не так вкусно

\begin{itemize} % {
\iusr{Оксана Дубинина}
\textbf{Дмитрий Гаранжа} хлебные лепёшки - ням-ням)) Спасибо!
Вообще в мужские руках еда получается особо вкусной!

\iusr{Дмитрий Гаранжа}
\textbf{Оксана Дубинина} )

\iusr{Ассоль Грей}
\textbf{Dmitriy Garanzha} ...зато полезно)

\iusr{Дмитрий Гаранжа}
\textbf{Ассоль Грей} И вкусно и полезно. Я другого и не посоветую - только проверенное и самое лучшее

\iusr{Даша Андреева}

Дуже вредна їда жарена @igg{fbicon.face.sad.but.relieved} 

\iusr{Дмитрий Гаранжа}
\textbf{Даша Андреева} ну так готуйте все на пару якщо так вважаєте
\end{itemize} % }

\iusr{Ирина Клименко}

А как мой од нлкий старенький папа, 86 лет, живя на троещине, может испечь
хлеб? Мы к нему не можем с правого переправить я. Что делать!? SOS!

\begin{itemize} % {
\iusr{Оксана Дубинина}
\textbf{Ирина Клименко} попробуйте написать объявление, я думаю отзовутся немало соседей. У нас в группе много с Троещине

\begin{itemize} % {
\iusr{Ирина Клименко}
\textbf{Оксана Дубинина} все рядом соседи уехали.

\iusr{Оксана Дубинина}
\textbf{Ирина Клименко} а группе напишите небольшой пост об этом. А кто сможет с Вами в личке пообщаются

\iusr{Ирина Клименко}
\textbf{Оксана Дубинина} скажите пожалуйста, какой именно группе, он живёт на Маяковского, в конце...

\iusr{Оксана Дубинина}
\textbf{Ирина Клименко} напишите у нас, в Киевских историях. Может фото папы, можно без фото. Вот увидите, отзовутся люди. А потом надо поискать местные группы на Троещине.
Мне сегодня батюшка-священник прислал контакты, кто помогает нуждающимся. Сейчас сюда пришлю

\iusr{Оксана Дубинина}
\textbf{Ирина Клименко}

\ifcmt
  tab_begin cols=2,no_fig,center,resizebox=0.8

     pic https://scontent-lhr8-1.xx.fbcdn.net/v/t39.30808-6/274858301_7321847957855408_411215327019444033_n.jpg?_nc_cat=107&ccb=1-5&_nc_sid=dbeb18&_nc_ohc=i76kKaErOO0AX-mNkDk&_nc_ht=scontent-lhr8-1.xx&oh=00_AT8ix45_S8JO7clBkmDkFEZK6lj-IgFfhpLbuosc1wWjZQ&oe=622A1FE2

		 pic https://scontent-lhr8-1.xx.fbcdn.net/v/t39.30808-6/274877190_7321848814521989_4385911102535250748_n.jpg?_nc_cat=110&ccb=1-5&_nc_sid=dbeb18&_nc_ohc=vTuL5tq1WI0AX_coyr3&_nc_ht=scontent-lhr8-1.xx&oh=00_AT9Bm1C8xXstftVgQfwqrkZpEd5H5kIvrzWHW88vpBPnjQ&oe=622AD3BE

  tab_end
\fi

\iusr{Оксана Дубинина}
\textbf{Ирина Клименко} найдите в фейсбуке и в телеграм и подпишитесь на них

\iusr{Ирина Клименко}
\textbf{Оксана Дубинина} 

спасибо конечно, но, как он сам найдёт этот макс март? Он плохо видит и слышит.
Просто мы не думали, что будет эта ситуация, так бы забрали к себе......

\iusr{Оксана Дубинина}
\textbf{Ирина Клименко} это надо не ему, а Вам пообщаться с ребятами. Если еще увижу службы помощи. Пришлю Вам

\iusr{Ирина Клименко}
\textbf{Оксана Дубинина} да я понимаю, что надо, но с кем? Как их найти, ребят?

\iusr{Оксана Дубинина}
\textbf{Ирина Клименко} попробую найти.

\iusr{Оксана Дубинина}
\textbf{Ирина Клименко} 

есть такие группы в фб Троещина. Там пишут объявления. Подпишитесь на группу,
сможете написать объявление.  Батюшка сказал, что надо искать волонтеров
Троещины. Если мне кто-нибудь напишет, скажу Вам.

\iusr{Ирина Клименко}
\textbf{Оксана Дубинина} спасибо за помощь

\iusr{Elena Andreeva}
\textbf{Ирина Клименко} это вы что, за 4 дня ничего не сделали? Никаких попыток не предприняли?

\iusr{Ирина Клименко}
\textbf{Elena Andreeva} 

у него просто на это время были запасы. Кто ж знал, что мосты перекроют и на
чем доехать. Метро через Днепр тоже не ездит. А как добраться с Нивок на
троещине, машины нет!

\iusr{Elena Andreeva}
\textbf{Ирина Клименко} , 

слава Богу. Но сейчас надо просто в Фб найти что-то, связанное с Троещиной и
там все ближе расспрашивать про группы и тд

\end{itemize} % }

\iusr{Оксана Дубинина}
\textbf{Ирина Клименко} Ирина, попробуйте набрать эти номера завтра

\ifcmt
  ig https://scontent-lhr8-1.xx.fbcdn.net/v/t39.30808-6/275091039_7322160587824145_7924663214398015831_n.jpg?_nc_cat=107&ccb=1-5&_nc_sid=dbeb18&_nc_ohc=NY--cthQqKgAX8wGsqm&_nc_ht=scontent-lhr8-1.xx&oh=00_AT86uOmVvZ90MHkKR19PcZJv1j2KEx4J2H2CuFM1e8R19g&oe=622BA4E6
  @width 0.3
\fi

\iusr{Ирина Клименко}
\textbf{Оксана Дубинина} да, спасибо, попробую позвонить!

\iusr{Olga Kapa}
\textbf{Ирина Клименко}

\ifcmt
  ig https://scontent-lhr8-1.xx.fbcdn.net/v/t39.30808-6/274914144_3209722532686011_6757577954421940922_n.jpg?_nc_cat=108&ccb=1-5&_nc_sid=dbeb18&_nc_ohc=gfZsRl76HvsAX_cGvas&_nc_ht=scontent-lhr8-1.xx&oh=00_AT9wVb5aHWjg5K_sLkHZlfOHKkAYm_ntYAmvlnbdBJzKnA&oe=622B3155
  @width 0.3
\fi

\iusr{Svitlana Revniuk}

найпростіший хліб, який завжди вдається

3.5 звичайних чайних частки борошна без гірки (1 чашка - приблизно 230 мл)

1.5 чашки води

3 столові ложки будь-якої олії, краще без запаху

1/2 ... 1 чайна ложка солі

1/2 чайної ложки сухих дріжджів ( 1 чайна ложка = 5 мл)

1 столова ложка цукру без гірки, 1ст.л.= 15 мл

Замісити руками, поставити в тепле місце приблизно на 1.5 -3 години щоб
підійшло

Випікати при ~180 С хвилин 30-40

*****

Бережіть себе! Залишайтесь всі здоровими! Ми переможемо!

\iusr{Оксана Дубинина}
\textbf{Svitlana Revniuk} слава Україні! @igg{fbicon.heart.red}

\iusr{Svitlana Revniuk}
\textbf{Оксана Дубинина} Героям слава!

\iusr{татьяна неверовская}
\textbf{Ирина Клименко} 

неужели там нет соседей или волонтеров? Киньте клич в группу где живёт, поближе
по адресу, наверняка такие есть, может принесут


\iusr{Елена Волынец}
\textbf{Ирина Клименко}
Это объявление было в группе в ФБ (название не помню).

\ifcmt
  ig https://scontent-lhr8-2.xx.fbcdn.net/v/t39.30808-6/274812372_4847623588650979_6461826427013476119_n.jpg?_nc_cat=105&ccb=1-5&_nc_sid=dbeb18&_nc_ohc=VthWkatCVWoAX-B1sgx&_nc_ht=scontent-lhr8-2.xx&oh=00_AT_pzKBAKwpR8kR4my2nHCmJj6_emJ-kBJTCPS-wz2MTdw&oe=6229FA87
  @width 0.3
\fi

\iusr{Оксана Дубинина}
\textbf{Ирина Клименко} 

Ирина, Вам удалось папе продукты завезти?

Помните я писала о компании МІКСмарт поміч?

Они теперь и доставку делают. Найдите их в телеграм и напишите в чат, Вам
перезвонят

\ifcmt
  ig https://scontent-lhr8-1.xx.fbcdn.net/v/t39.30808-6/274791773_7329091720464365_6942068826566839126_n.jpg?_nc_cat=108&ccb=1-5&_nc_sid=dbeb18&_nc_ohc=N4serfgnpSkAX9RKePg&_nc_ht=scontent-lhr8-1.xx&oh=00_AT9zLJW3xswTB6fL7IlpvjUHZfeZWzDGkNuIJ1cNS1N-Yg&oe=622BBBF3
  @width 0.3
\fi


\end{itemize} % }

\iusr{Svitlana Plieshch}

Какая вы молодец! И публикации у вас интересные и идеи хорошие. Выпечка хлеба -
это и еда, и запах, и настроение, и голова отвлекается (о, получился, или ёлки,
не получился). Побольше простых рецептов. Спасибо.

\iusr{Оксана Дубинина}
\textbf{Svitlana Plieshch}  @igg{fbicon.wink}  @igg{fbicon.face.happy.two.hands}  @igg{fbicon.face.smiling.hearts} 

\iusr{Галина Кошмак}
Якщо є продукти Для хліба

\iusr{Олена Андурова}

яблочный пирог:пачка творога, 2 яйца, 0. 5 чл соды, полстакана сахара, 2 стакана
муки. 50 гр мягкого масла или маргарина, Смешиваем, вымешиваем ,делим
пополам. Тесто мягкое, качалкой не раскатывается, все руками. Одну половину
скатываем колобок и растягиваем мокрыми руками, у меня тефлоновая форма для
пиццы).Из второй делаю 6-7 шариков как пинг-понг и опять же пальцами разминаю
на тонкие \enquote{латочки}. Засыпаем порезанными яблоками и прикрываем
латочками, прижимая края к нижнему коржу. оставить лырочки для выхода пара. Это
порция на половину противня. Если почти не добавлять сахар, то начинка
любая. Смазать желтком. Белок я тоже добавляю в тесто, чтобы не пропал. Долго
пишется, но готовится быстро, Печется сперва 3 мин при200, потом при 180 до
золотистой корочки,. Легкий, не черствеет, да и не успеет, Приятного аппетита!


\iusr{Оксана Дубинина}
\textbf{Олена Андурова} спасибо) аж слюнки потекли ) @igg{fbicon.face.savoring.food} 

\iusr{Елена Дерябкина}

- стакан молока (250 грамм),- 3 таких же стакана муки,- 1/3 пачка дрожжей 2
столовых без горки сахара,- 1 чайная соли,- треть стакана растительного масла
(можно меньше). В деле приготовления теста отмерять молоко, масло и муку нужно
одним размером стакана. Греем до 30 молоко, засыпаем в него 2 столовых сахара и
дрожжи. Дрожжи не нужно размешивать, а нужно их только чуть- чуть смочить. Ждем
15 минут. Увидев пенную шапку на поверхности, сыпем чайную соли и размешиваем.

Засыпаем всю муку и месим тесто. В процессе замешивания понемногу добавляем 1/3
стакана растительного масла. Сильно стараться здесь не нужно, а нужно лишь
домесить тесто до состояния сбора им всей муки и масла.

Через минут 20 тесто набухнет и вот уже тогда, выложив его на стол, нужно будет
вымесить его до однородности.

Тесто возвращаем в миску, накрываем полотенцем или пленкой и путь оно постоит
минут 40 при комнатной. Через 30- 40 минут тесто должно подойти и его нужно
обмять. Подождать еще 35- 40 минут. Тесто должно увеличиться в объеме раза в
три не меньше.

Делим тесто на 12 частей, каждую часть катаем в шар и выкладываем на противень,
застеленный пекарской бумагой или смазанный маслом.

Накрываем пленкой, полотенцем, или что там у вас есть по размеру противня, и
даем минут 20-30 расстояться. Можно смазать яйцом. Красиво, но на вкус не
влияет.

Ставим в разогретую до 220 духовку минут на 20. Мы перестали хлеб покупать.
Молоко можно разбавлять водой. Только на воде не пробовала. Расписала может
слишком подробно, извините, если что  @igg{fbicon.smile} 

\begin{itemize} % {
\iusr{Оксана Дубинина}
\textbf{Елена Дерябкина} 

очень даже легко запоминается. Спасибо!)))

\iusr{Елена Дерябкина}

Главное делать как написано. Рецепт не мой, мной чуть адаптированный, но
получилось сразу

\end{itemize} % }

\iusr{Irina Sokol}

Делюсь самым простым рецептом хлеба от известного пекаря Jim Lahey что владеет
Sullivan street bakery в Нью-Йорке.

Ингредиенты:

Мука 3 стакана

Дрожжи 1/4 ч.л

Соль 1,25 ч.л

Вода чуть тёплая 1,33 стакана.

Посуда:

Миска, желательно стеклянная

Чугун

Другие вещи:

Пищевая пленка

Вощёная бумага.

1. Смешать сухие ингредиенты в миске, потом влить воду, всё перемешать руками,
минут 5, накрыть плёнкой и оставить на 14-18 часов.

2. Достать тесто, оно будет тянуться, сформировать в шар на деревянной доске,
если липнет, посыпьте ладошки мукой, когда сформировали, оставьте тесто на два
часа, накройте пелёнкой и чтоб тесто лежало швом вниз.

3. С начала второго замеса, через 1.5 часа, поставьте накрытый пустой чугун в
печку на 220 градусов, полчаса пусть нагреваются.

4. Положите тесто швом вверх на вощеную бумагу, аккуратно опустите в чугун и
закрытым в печь на час 15 минут, последние 15 минут снимайте крышку.

5. Когда готов, вытащите с чугуна и перебросьте на решетчатый лист чтоб остыл.

Очень просто и вкусно!

\ifcmt
  ig https://scontent-lhr8-2.xx.fbcdn.net/v/t39.30808-6/274980773_5097982010265777_6981353930024740919_n.jpg?_nc_cat=101&ccb=1-5&_nc_sid=dbeb18&_nc_ohc=1tZ2_0V0LWkAX9wU3NM&_nc_ht=scontent-lhr8-2.xx&oh=00_AT9hNPGAXBkkMTG0KPYGcrqarYPs5jGtpXVfoXtrnuYApw&oe=622AF404
  @width 0.3
\fi

\iusr{Оксана Дубинина}
\textbf{Irina Sokol} 

ух ты)) даже на фото почти, как наша паляниця!@igg{fbicon.heart.red}
благодарю!)

\iusr{Валентина Зражевская}

Відносно паляниці, з російського телебачення, ведуча новин, робила дуже цікавий
переклад, ніколи не здогадаєтесь  @igg{fbicon.smile}, це «клубника
@igg{fbicon.strawberry} :)»


\end{itemize} % }
