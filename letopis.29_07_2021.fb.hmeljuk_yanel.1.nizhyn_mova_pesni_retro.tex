% vim: keymap=russian-jcukenwin
%%beginhead 
 
%%file 29_07_2021.fb.hmeljuk_yanel.1.nizhyn_mova_pesni_retro
%%parent 29_07_2021
 
%%url https://www.facebook.com/groups/1678715382281320/posts/2008445482641640/
 
%%author 
%%author_id hmeljuk_yanel
%%author_url 
 
%%tags 
%%title у вухах звучить \enquote{Зємля в ілюмінаторє}, \enquote{Я буду долго гнать вєлосіпєд}, \enquote{Мєчьта сбываєцца}
 
%%endhead 
\subsection{у вухах звучить \enquote{Зємля в ілюмінаторє}, \enquote{Я буду долго гнать вєлосіпєд}, \enquote{Мєчьта сбываєцца} }

Вчора я була на дійстві, яке не можу забути... не можу розчути...  

Приготуйтеся до читання і наберіться терпіння - ТЕКСТ ДОВГИЙ!

у вухах звучить "Зємля в ілюмінаторє", "Я буду долго гнать вєлосіпєд", "Мєчьта сбываєцца" 

І Я ЗРОЗУМІЛА, ЩО Я "ЗІПСОВАНА"...

"ЗІПСОВАНА" УКРАЇНСЬКОЮ ПІСНЕЮ...

\ifcmt
  pic https://scontent-cdt1-1.xx.fbcdn.net/v/t1.6435-9/227437514_4455941794450043_7761978826107483793_n.jpg?_nc_cat=103&ccb=1-3&_nc_sid=825194&_nc_ohc=oETVgcNLwGQAX8mLNSN&_nc_ht=scontent-cdt1-1.xx&oh=f02899f885fb601e6dbe60f04054b620&oe=612D6C03
  width 0.4
\fi

...мабуть, Кам'янець мене зіпсував... 

...чи країна, в якій я живу, яка бореться за збереження територіальної
цілісності ... 

...чи події пов'язані з неоголошеною війною, яку влаштувала країна-агресор... 

...чи інформація зі слів очевидців тих подій, з якою я працюю щодня і плачу,
коли працюю з текстом, в якому молодий український  генерал розподвідає, як
збирає хлопців по деталях, яких розірвало російськими сгарядами... 

.. чи проекти "Кримський альбом", "Ультрас. Шлях до волі", "Молоді генерали
України" і т.д.

Розумію, що це був концерт ретро музики... все розумію... АЛЕ ЧОМУ на цьому
концерті звучали лише російські пісні, окрім "Червоної рути Володимира
Івасюка"???

В Україні немає пісень ретро??? В українській музиці немає історії??? Ви хочете
сказати, що в українській культурі НЕМАЄ ПІСЕНЬ РЕТРО??? В Ніжині НЕМАЄ
ВИКОНАВЦІВ, ЯКІ СПІВАЮТЬ УКРАЇНСЬКІ ПІСНІ???

Мені різало слух... нам в Кам'янці не дозволяли співати російський репертуар ще
в ❗2010 році ❗задовго до Майдану і АТО/ООС... задовго до анексії Криму і
окупації Донбасу...

Працюючи з дітьми в закладі естетичного виховання я не встигала перекладати пісні класиків українською мовою...

Мої вихованці співали Баха, Бетховена, Моцарта українською, оскільки всі
хрестоматії з вокалу радянських часів і перекладені російською... можна навіть
збірку видати ..

Пішла з подругою розвіятися... а тепер не можу це розчути... мала надію почути
хоча б одну українську пісню... але за півтори години так і не почула... 

А українська естрадна музика така багата на хіти!!!

Темою моєї магістерської роботи, було дослідження становлення української
естрадної пісні... СКІЛЬКИ Ж прекрасних виконавців УКРАЇНСЬКИХ, СКІЛЬКИ Ж
НЕПЕРЕВЕРШЕНИХ ПІСЕНЬ у Івасюка, Яремчука, ВІА МРІЯ, Зінкевича, частину з яких
записала у своїх україномовних альбомах дочка українських емігрантів з США
Квітка Цісик!!! 

Невже не можна було вибрати хоч якісь?!!!

Це нівелювання української культури... 

У моєму репертуарі, коли я співала на сцені, на якій провела 20 років свого
життя, немає жодної ВИПАДКОВОЇ ПІСНІ - ЖОДНОЇ!!! Я завжди обирала репертуар
відповідально, задаючи собі питання: А ЧОМУ НАВЧАЄ СЛУХАЧА ТА ЧИ ІНША ПІСНЯ???
Бо це МОЯ ВІДПОВІДАЛЬНІСТЬ ПЕРЕД СЛУЗАЧЕМ, ЯК СПІВАЧКИ!!! Тому що СПІВАЧКА
ВКЛАДАЄ ПЕВНІ СМИСЛИ В ГОЛОВИ СЛУХАЧІВ ЗА ДОПОМОГОЮ ПІСНІ!!! МЕТА
СПІВАКА/СПІВАЧКИ - ПРИВИВАТИ ЛЮБОВ ДО РІДНОЇ УКРАЇНСЬКОЇ ПІСНІ!!!

Натомість маємо на 95\% російськомовний концерт в українському Ніжині... 

Дуже сумно... 

Минулого року на концерт "Ми - українці" я запроаонувала пісню "Приворожи", текст якої 

\begin{multicols}{2}
\obeycr
"Приворожи мене, приворожи,
Кохана земле – материнське диво.
Незвіданим мене приворожи,
Прикуй ланцем чаклунським до душі
Свою красу святу й сором’язливу.
\smallskip
Тебе зректися – найстрашніший гріх.
Без роду жити в світі не годиться.
Хай роси прикиплять до босих ніг.
Зрони гарячу краплю чар своїх
В ковток води з батьківської криниці. 
\smallskip
Щоб не зманив мене далекий світ,
Щоб не пішла я довгими стежками,
А ті стежки, якими стільки літ
Проходили і батько мій, і дід,
Не поросли травою й будяками.
\smallskip
Щоб я не здобула багатий дім
Там, де твоє ім’я – порожнє слово,
Щоб не жила безбідно в домі тім,
Та не дай, Боже, щоб він став моїм,
Забувши материнську колискову.
\smallskip
Приворожи мене в травневім сні,
Коли від свого буйноквіття п’яна
Вирує вишня в білому вогні.
Хай сили й віри в тебе дасть мені
Та ворожба первісна та незнана.
\smallskip
Земля мені відмовила: "Дарма!
Про це не варто навіть і благати.
Такого зілля дивного нема.
Дитино люба, ти збагни сама,
Хіба ж своїх дітей чарує мати?!"
\restorecr
\end{multicols}

Не повірите - РЕЖИСЕРУ НЕ ПІДІЙШЛО для національно-патріотичного заходу!!!

Після вчорашнього заходу мені не дивно...

На державному рівні ще в 2016 році

8 листопада 2016 року набув чинності закон про квоти на українські пісні та
мову на радіо і телнбаченні. Вони запроваджувався поступово. У підсумку частка
української пісні в радіоефірі має складати не менше від 35\%, а квота на
українську мову ведення ефіру – не менше від 60\%.

Маємо заборону медведчуківських каналів...

НАВІЩО ЦЕ ВСЕ, коли тут на місцях ми годуємо мешканців міста російським
репертуаром - це схоже на ВОРОГА В ТИЛУ... 

Хочу розчути все те, що я вчора почула... хочу відмотати плівку назад і не
потрапити на концерт... хочу віднайти машину часу і повернутися до 18.00 28
липня і залишитися вдома... хоча... мабуть, я мала там бути... і все це
бачити...

Ще одне протиріччя... Вчора була 1033 річниця ХРЕЩЕННЯ КИЇВСЬКОЇ РУСІ,
православна церква вшановує Князя ВОЛОДИМИРА... 

АЛЕ в Ніжині, відбувається не патріотичний концерт, а Фієста російських пісень...

Просто В ГОЛОВІ НЕ ВКЛАДАЄТЬСЯ!!!

Найстрашніше, що представники культури своїм репертуаром формують світогляд
пересічного мешканця міста... щось, все-таки, НЕ ТАК З НАМИ... це просто
жахливо...

Дай Боже, щоб нам не довелося "долго гнать вєлосіпєд", коли зайде російський
чобіт... бо вони дуже пристально моніторять такі російські уподобання
українського населення - і це для них "МАРКЕР", яким вони ставлять хрестик на
геополітичній карті у планах перспективи окупації подальших територій...

Боляче невимовно...

\ii{29_07_2021.fb.hmeljuk_yanel.1.nizhyn_mova_pesni_retro.cmt}
