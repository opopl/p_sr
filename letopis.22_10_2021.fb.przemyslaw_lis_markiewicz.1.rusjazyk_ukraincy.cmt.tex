% vim: keymap=russian-jcukenwin
%%beginhead 
 
%%file 22_10_2021.fb.przemyslaw_lis_markiewicz.1.rusjazyk_ukraincy.cmt
%%parent 22_10_2021.fb.przemyslaw_lis_markiewicz.1.rusjazyk_ukraincy
 
%%url 
 
%%author_id 
%%date 
 
%%tags 
%%title 
 
%%endhead 
\subsubsection{Коментарі}
\label{sec:22_10_2021.fb.przemyslaw_lis_markiewicz.1.rusjazyk_ukraincy.cmt}

\begin{itemize} % {
\iusr{Ivan Vagabond}
Правда

\iusr{Igor Lypkan}
Як говорять не в брову, а в око.

\iusr{Тарас Тарасенко}
Коротко і ясно.

\iusr{Lyudmila Poshuk}
Це сумно та прикро читати. Але це правда

\iusr{Olga Wolodymyriwna}
Не існує !!!

\iusr{Костянтин Лях}

Цілком підтримую, інакше не може бути... Ці люди без Батьківщини, без прапора)
Дякую Лис Маркевич за чудову рецензію...

\begin{itemize} % {
\iusr{Przemysław Lis-Markiewicz Profil Prywatny}
\textbf{Kostyantin Lyah} Мене звати Ліс-Маркєвіч, я ж поляк. Лис-Маркевич то був би українець  @igg{fbicon.smile} 

\iusr{Сергій Коцур}

Ви більший українець чим дехто вважає...
Дякую Вам за дописи і за те що Ви є!!!!
\end{itemize} % }

\iusr{Валерій Нестерчук}
Spinnt

\iusr{Roksolana Kuzmuk}
Підтримую

\iusr{Volodymyr Buncha}

Та поза сумнівом. Де лише я їх не зустрічав-завжди переконувався що то
московська ментальність ,за рідкими винятками. І то були випадки коли людина
частково виховувалась у дідуся чи тітки або ще когось ,-українця.Якщо
,наприклад є люди котрі в Канаді прожили зо 25 років,але так і не вивчили
мови(а з нею не взнали культури) а сидять на різних москвинячих серіалах або 95
квартал з України-то яка ментальність у них може бути? Я лише не можу вирішити
,що там первинне-або тупість , і їм важко вивчати щось нове чи то так тупіють
від московського ізика?

\begin{itemize} % {
\iusr{Witali Tanczuk}
\textbf{Volodymyr Buncha} у них,,, 95/73, чи73/95 - тупа впертість... або вперта тупість... Не можу дослідити: перше... чи друге... Всім вітання від українського поляка...
\end{itemize} % }

\iusr{Ігор Данилюк}
100\% Така правда.

\iusr{Олександр Косенко}
Свята правда!

\iusr{Василь Жупаник}

Пане Przemysław Lis-Markiewicz, дякую за підтримку України, Українців та
Української мови!  Dziękuję

\iusr{Olha Gontar}
 @igg{fbicon.thumb.up.yellow} 

\iusr{Olha Gontar}
100\% Правда !
Золоті слова!

\iusr{Дмитрий Наливайко}

мене теж вкурвлюють україномовні українці, чесно, навіть більше, бо на
рускоязичних мені чхати, то совки. більшість з нас, україномовних, нічого окрім
раскава язика не знає і практично всі сидять на рускоязичному контенті. весь
Львів від малого до старого сидить на рускоязичному контеті. ви візьміть мале
дитя з села в Львівській області під Польським кордоном. воно настільки
перфектно вам па рускі буде говорити, що не відрізниш від його однолітків з
владівостока. а "чь"-кають всі як. раніше більше людей "ч" казали як польське
"cz", зараз вже фонетика майже як в російській. більшість сталих виразів і
ідіом в сучасній українській мові є такою самою, як і російській. Молодіжний
сленг теж. мені здається, що в першу чергу має змінитись україномовна частина
суспільства. вона має припинити споживати рускоязичний контент і врешті вивчити
англійську мову. бо як ми можемо від російськомовних очікувати, щоб вони
змінили свою мову спілкування, якщо ми самі лінуємось відмовитися від
споживання рускоязичного контенту? тому, я думаю, що лінь україномовних це
основне джерело мовної проблеми в Україні.

\begin{itemize} % {
\iusr{Przemysław Lis-Markiewicz Profil Prywatny}
\textbf{Dmytro Nalyvaiko} Це правда. Тому треба безкомпромісно українізувати Україну та українців. Повірте, я організовую семінари для громадян України в Польщі, вони безкоштовні, але мова - українська. Деякі малороси вперше за життя заговорили українською в Польщі, у поляка.
\end{itemize} % }

\iusr{Max Webber}
а де ці коментарі від узкоязічних?

\begin{itemize} % {
\iusr{Przemysław Lis-Markiewicz Profil Prywatny}
\textbf{Max Webber} На різних сайтах і сторінках.
\end{itemize} % }

\iusr{Олег Кобизський}
Всім кажу,що "узка язичний украінец" це як мясоїдний вегетаріанець  @igg{fbicon.thinking.face}  @igg{fbicon.face.rolling.eyes} 

\iusr{Tania Szyczanina}
Низький уклін за такі слова.

\iusr{Helena Jerypałowa}
Brawo! @igg{fbicon.hands.applause.yellow}{repeat=4} 

\iusr{Irena Jasinowska}

Або українець, або ж українець, а то все решта МОСКАЛИКИ.... До сьогодні
ненавиджу цю мову, а особливо, як поляки зачинають говорити до тебе на
російській, то аж мною натягає(

\iusr{Кость Лучків}
Браво!!!

\iusr{Інгвар Тітус}

Браво, друже!!! Повага і респект, але.... і біль @igg{fbicon.face.sad.but.relieved} Тому що це все малоросіянство
підігрівається шаленими грошима, заробленими т. зв "укрАінськім олігархатом", а по
суті-штучно вирощеними агентами впливу....  @igg{fbicon.hands.shake}{repeat=3} 

\iusr{Олексій Семенович}
\textbf{Ерік Залєвський} був німець, звісно. А Фелікс Дзержинський - москвин.

\iusr{Таня Остяк}

\ifcmt
  ig https://scontent-frx5-1.xx.fbcdn.net/v/t1.6435-9/248390714_4395115117239468_8075811385622100539_n.jpg?_nc_cat=100&ccb=1-5&_nc_sid=dbeb18&_nc_ohc=159bKuE76WoAX_Dcfcc&_nc_ht=scontent-frx5-1.xx&oh=69d0f6be3b436b5b4a2d522e2ccd16ba&oe=619CC214
  @width 0.4
\fi

\iusr{Люда Кундельська}
 @igg{fbicon.thumb.up.yellow} 

\iusr{Yuliya Azizova}

Є одне але. У Східних регіонах є мало носіїв української мови. Ми цього року
були в Болгарії і там були східняки хлопці , що служили в українському війську
добровольцями, котрі воювали за Україну. Хлопці віком від сорока до п'
ятидисяти років. Котрі родились і виросли в абсолютно російськомовному
середовищі. Знаєте, що вони мене просили? Щоб я розмовляла виключно українською
бо вони хочуть чути як звучить мова. Вони всі розуміли мову, але вони її не
чують щодня у побутовому просторі. Питали мене, а як правильно говорити ті чи
інші слова. Вони кажуть ми все розуміємо, можемо перекласти на папері, а от
говорити в них не дуже виходить хоч вони дуже стараються. Я їх питаю, чи в
школі вчили мову. Вони кажуть, що уроки мови були для галочки в ті часи і як
правило українську мову заміняли іншими предметами. Вони самі не дуже добре
розмовляють українською мовою, але своїх дітей стимулюють вчити і говорити
мовою. Кажуть, що соромляться говорити українською бо в них суржик виходить. І
вони якраз найбільше і виступають за повну українізацію в публічному просторі
бо їм тої мови бракує там на Сході.

\begin{itemize} % {
\iusr{Przemysław Lis-Markiewicz Profil Prywatny}
\textbf{Yuliya Azizova} Я вивчив українську в Польщі і самотужки. Можна? Можна. Тільки треба хотіти.

\iusr{Yuliya Azizova}
\textbf{Przemysław Lis-Markiewicz Profil Prywatny} можна, але Ви вчили мову в унівеситеті в Києві де були викладовці чули як говорять. А в них немає до того доступу.

\iusr{Przemysław Lis-Markiewicz Profil Prywatny}
\textbf{Yuliya Azizova} Ви мене з кимось плутаєте. Ніколи не вчився в Києві. Українську вивчив сам у Польщі.

\iusr{Кость Лучків}
\textbf{Yuliya Azizova} Пані Юлія . Ці хлопці знають основи мови. Вони її постійно чують . Тому, їм просто треба почати нею розмовляти. Та поступово збільшувати запас слів:)
Ось і все , що їм треба:)

\iusr{Yuliya Azizova}
\textbf{Przemysław Lis-Markiewicz Profil Prywatny} перепрошую. Я чомусь вважала що Ви вчили мову стаціонарно у Вишу.

\iusr{Yuliya Azizova}
\textbf{Кость Лучків} я то саме їм казала і казала більше читати української літератури. Бо ніщо так не покращує мову як постійне читання на мові котру вивчаєш.

\iusr{Богдан Олійник}
\textbf{Yuliya Azizova} Більшість сел і на сході України і на півночі говорять українською . Но як приїдуть до горада то нужна паруску гаваріть

\iusr{Yuliya Azizova}
\textbf{Богдан Олійник} вони були з Дніпра. З міста.

\iusr{Олег Габалевич}
\textbf{Przemysław Lis-Markiewicz Profil Prywatny} 

Ваш приклад є зразком інтелігентності та старанності в навчанні. Таких одиниці.
Я тепер у Польщі та щодня чую і спілкуюся польською, але ніяк не оволодію
польською для вільного спілкування, хоча стараюся читати та аналізувати
прочитане. Що в дитинстві вивчив, те памятаю.

\iusr{Yuliya Azizova}
\textbf{Олег Габалевич} головне говоріть і все буде виходити.

\iusr{Олена Дашко}
\textbf{Yuliya Azizova} 

Я з Дніпропетровської області, все своє життя говорила російською.Хоча в школі
вивчала українську, але ж всім відомо, що там звідки я родом всі говорять
російською, хіба що в селах звучить українська. По приїзду до Польщі мій
світогляд кардинально змінився, а разом з тим і моя мова. Тепер маю дві рідні
для мене:українську і польську. Польську вивчила, бо в Польщі без неї аніруш!А
українську спочатку почала застосовувати, коментуючи щось в соцмережах, різні
СМС, листування і т. д., а потім якось само непомітно для мене стало нормою
спілкуватися виключно українською. Якщо раніше до мене зверталися російською,
то я автоматично переходила на російську. Тепер я вже цього позбулася, чим дуже
пишаюся @igg{fbicon.face.smiling.eyes.smiling}{repeat=2} 

\end{itemize} % }

\iusr{Вітер Левко}

Приклад ополяченого німця-чиновника навіть не в Польщі, а в українській
Галичині, тоді піднімецькій (підавстрійській) є у Франкових "Перехресних
стежках".


\iusr{Вітер Левко}

Ну не сказати б, що в підросійській частині поділеної Польщі не було
зрусифікованих поляків. Наприклад, Фелікс Дзержинський. Або письменники
Станюкович, Смідович (пен-нейм Вересаєв). Або автор первинного російського
тексту пісні "Серце на снігу" Дмоховський, відомої в Польщі за двома польськими
текстами на цю ж мелодію Бабаджаняна: авторства Ядвіги Урбановіч у виконанні
Вєслави Дроєцкої та авторства Збігнєва Ставецкого Trojki dwie.

\iusr{Юлія Поберезька}
 @igg{fbicon.hands.applause.yellow}{repeat=3} 

\iusr{Богдан Олійник}

В Польщі , в медіа культурі росіяни праві завжди мають негативний образ . В
Україні росіяни , а вірніше рузький язик грає взірець успіху . Для мене (якій
довший час жив в Польщі ) таке буває смішне . Смішно бачити як в Україні
пишаються тим над чим в Польщі сміються .

\begin{itemize} % {
\iusr{Yuliya Azizova}
\textbf{Богдан Олійник} що то за фільм?
\end{itemize} % }

\iusr{Роман Сидорчук}
так і є

\iusr{Oleksii Ostrovskiu}
Знавав я російськомовних Українців і україномовних шанувальників путіна і Росії , таке нажаль буває...
\end{itemize} % }
