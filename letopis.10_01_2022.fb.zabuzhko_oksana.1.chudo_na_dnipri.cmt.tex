% vim: keymap=russian-jcukenwin
%%beginhead 
 
%%file 10_01_2022.fb.zabuzhko_oksana.1.chudo_na_dnipri.cmt
%%parent 10_01_2022.fb.zabuzhko_oksana.1.chudo_na_dnipri
 
%%url 
 
%%author_id 
%%date 
 
%%tags 
%%title 
 
%%endhead 
\zzSecCmt

\begin{itemize} % {
\iusr{Volodymyr Viatrovych}

Нас було багато завдяки тому, що тривалий час Майдан тримався ненасильницьких
засобів боротьби. Саме завдяки цим ненависним деяким радикалам «ліхтарикам» і
«танцям». Такий Майдан приваблював не лише світ, створюючи позитивний образ
протесту, але й інших українців готових долучитися до нього. Радикалізація
протесту звужує його мобілізаційні можливості.

Але з іншого боку Майдан набравши розгону як ненасильницький спромігся дати
силову відповідь на насильство влади. І це сталося не надто рано (Казахстан) і
не надто пізно (Білорусь). Ми справді пройшли по лезу.

\begin{itemize} % {
\iusr{Mila Oliynyk}
\textbf{Volodymyr Viatrovych} білоруси теж були ненасильницькі з рожевими поні. І?

\iusr{Volodymyr Viatrovych}
\textbf{Mila Oliynyk} прочитайте комент до кінця. В ньому і про готовність відповісти на силові дії влади.

\iusr{Maria Zayats}
\textbf{Volodymyr Viatrovych} так, але і люди в своїй більшості вийшли якраз реагуючи на насильство над студентами зі сторони режимної влади януковича – хто зна чи вийшло б стільки людей, якби не трагічні події 30 листопада.

\iusr{Volodymyr Viatrovych}
\textbf{Maria Zayats} 

насильство 30.11 мало викликати негайну силову відповідь Майдану за планом
влади. І це мало стати кінцем Майдану. Але цього не сталося. І навіть спроби
«пришвидшити» перетікання протесту 1.12 теж не вдалися. Тому коли Майдан таки
відповів силою це не змінило позитивного ставлення до нього з боку світу та
українців

\iusr{Мирослав Логийко}
\textbf{Volodymyr Viatrovych}

 @igg{fbicon.thumb.up.yellow}  але все ж таки без парламенту з проукраїнськими та проєвропейськими
парламентарями, які не втекли, а стали представляти Майдан у Раді, нічого б не
вийшло.

\begin{itemize} % {
\iusr{Volodymyr Viatrovych}
\textbf{Miroslav Logiyko} згоден. І опозиція в парламенті вчасно і вдало зіграла свою роль

\iusr{Мирослав Логийко}
\textbf{Volodymyr Viatrovych} 

В 2014-му році у України вийшло не тільки завдяки майдану і добровольцям, а й
завдяки Парламенту України.

У Парламенті України були проукраїнські та проєвропейські політики та партії,
які втримали державу і заклали фундамент нової.

Саме Парламент України, після втечі януковича і його влади, оголосив початок
проведення антитерористичної операції.

Парламент України провів вибори і легітимізував владу.

І тепер саме парламент України є ціллю путіна-зеленського.

ОПЗЖ вже заявили про свої наміри кинути мандати, а це в свою чергу дозволяє
зеленському розпустити парламент.

\url{https://youtu.be/Kccmy-TqmgM}

Зеленський з його монобільшістю та проросійскою підтримкою це підхватить.

\url{https://m.facebook.com/story.php?story_fbid=4932810226762610&id=100001010458434}

Цього разу Путін не повторить своєї помилки 2014-го.

Главный вывод по событиям в Казахстане - что бы Западу было кому помогать, в
том числе оружием, нужна организованная оппозиция в стране и лидер у этой
оппозиции.

Запад не будет поставлять орудие и помощь разроздненым группам сопротивления.

Именно этого сейчас добивается путин/зеленський - убрать лидера оппозиции, как
объединяющую фигуру, и уничтожить парламентаризм как таковой, что бы не было
альтернатив.

Уже год про это пишу:

\url{https://m.facebook.com/story.php?story_fbid=4932810226762610&id=100001010458434}
\end{itemize} % }

\iusr{Ntina Ntoubrova}
\textbf{Volodymyr Viatrovych} 

Треба ще й врахувати, що це для українців другий майдан (а для білорусів -
перший), і що в нас на відміну від росіян та казахів - дуже компактна територія
без значних перешкод в пересуванні із наспунктів навіть прикордоння - до Києва.
Ну і наостанок - на нас перших цю спецоперацію випробували.

\begin{itemize} % {
\iusr{Marina Mischenko}
\textbf{Ntina Ntoubrova} 

Третій. Перший був \enquote{Революція на граніті}, голодування студентів. Після
першого Майдана був офіційно прийнятий Блакитно-жовтий Державний прапор

\end{itemize} % }

\iusr{Микола Миколайович Попелуха}
\textbf{Volodymyr Viatrovych} 

Головне все таки, що майданівці не грабували магазини чи перехожих! і навіть і
не думали про це.. По собі суджу.. Якось дивно було б уявити себе з кросівками
найк чи рибок бігучим стримголов по Хрещатику!))) Ясно, що в Алмаати
грабіжниками були так звані тітушки від спецслужб, які і організували і
спровокували мародерство... тітушки, думаю були, як від токаєва-назарбаєва, так
і від путлєра...

Правосєкі і ультрас, дійсно вчиняли насильство на Груші, але Вони не грабували
магазини і всіляко це пресікали!!! Сбербанк Росії зправа від барикади в бік
Європейської площі, що по Хрещатику і якраз на розі з Інститутською працював
вплоть до розстрілу на Інститутській!!!

\begin{itemize} % {
\iusr{Игорь Швец}
\textbf{Микола Миколайович Попелуха}, не ускладнюйте, той хто зовні організовував хороводи радикалів з надією, що вони почнуть мародерствувать не розрахував на те, що там українці, свого не віддадуть, чуже не візьмуть. Інакше б вони поставили на тих хто міг би зміг зайнятися мародерством. В Казахстані це врахували.

\iusr{Тетяна Костенко}
\textbf{Микола Миколайович Попелуха} Всі кафешки, бутіки, супермаркет працювали весь час в Глобусі.
\end{itemize} % }

\iusr{Taras Zarubko}
\textbf{Volodymyr Viatrovych} ніколи би не подумав, що Вятрович таку дурню напише!

\begin{itemize} % {
\iusr{Volodymyr Viatrovych}
\textbf{Taras Zarubko} дуже аргументоване заперечення  @igg{fbicon.smile} 

\iusr{Taras Zarubko}
\textbf{Volodymyr Viatrovych} Історику про історію...

10 грудня, дрімаємо коло бочок з дошками, які з одного кінця горять- тліють....

Аж тут пішла чутка про штурм! Хапаєм дошки, палаючі з одного кінця і біжимо
вверх по Інститутській! Виявилося, що беркутівський кийок не тягне проти
палаючої дошки і вони відступають з барикади....

То тільки був початок подій тієї ночі.....

Кажете мирний протест, ліхтарики - телефончики???

\iusr{Степан Лазар}
\textbf{Taras Zarubko} о експерд...

\iusr{Taras Zarubko}
\textbf{Степан Лазар} \enquote{експерт} пишеться от так!
Можеш звертатися! Багато що тобі розповім про Майдан ! На дивані, дивлячись
телевізор, ти багато що пропустив!
\end{itemize} % }

\iusr{Ольга Багалій}
Згодна. але навіть в цей час певна організація людей на Майдані була. І це врятувало від провокацій.

\iusr{Taras Zarubko}
\textbf{Olga Bagaliy} Я думаю, що не організація врятувала від провокацій і погромів, а внутрішня ідейність учасників!

\iusr{Xenia Gerke}
Про який силовий Майдан ви говорите? 100 майданівців убиті режимом Янука - скількох януків убив Майдан?

\begin{itemize} % {
\iusr{Taras Zarubko}
\textbf{Xenia Gerke} 18 \enquote{силовиків}.

\iusr{Xenia Gerke}
\textbf{Taras Zarubko} І вони не в Сотні. Бо виконували злочинні накази.

\iusr{Taras Zarubko}
\textbf{Xenia Gerke} Ви запитали, я відповів.

\iusr{Майя Копернак}
\textbf{Xenia Gerke} Майдан убив головного ворога - КПСС, нажаль, не всі це помітили...
\end{itemize} % }

\iusr{Борис Курдибаха}
\textbf{Volodymyr Viatrovych} 

ліберали на тракторі заважали швидко вирішити проблему неадекватів у владі,
тому жертви на Майдані і потім на фронті.

Воюють радикали.

\iusr{Roman Novatchinski}
\textbf{Volodymyr Viatrovych} 

а чим радикали не вгодили, може вже буде показувати які ми толєрасти і
завалювати ворога горою наших трупів, може краще нехай вже буде гора трупів, але
наших ворогів... питання до вас як публічної особи, хто дозволяв приземлитися
літакам з кацапії з їхнім \enquote{спєцназом}, чому не судять Кучму, де розслідування по
розстрілу на Майдані... може мій батько мав танцювати і світити ліхтарі перед
москалями замість того щоб піти в ліс в далеких 40х...

\begin{itemize} % {
\iusr{Taras Zarubko}
\textbf{Roman Novatchinski} Слава Україні!
Мій дід загинув в УПА!

\iusr{Roman Novatchinski}
\textbf{Taras Zarubko} вічна Слава і Пошана, мого зловили в 50 му через зраду, дали 25 років, відсидів 6ть.

\iusr{Taras Zarubko}
\textbf{Roman Novatchinski} 

В мене на Майдані побратим був Вася Фіцич з Н.Березова. Його батько брав участь
в знаменитому бою березівської сотні, коли погромили полк НКВС, більше 300
вбитих ворогів!

\end{itemize} % }

\iusr{Олександр Гринечко}
\textbf{Volodymyr Viatrovych} Пане Володимире майже у всьому я з вами згідний, тільки от думаю що якби підчас тих подій загинув хтось із представників влади а вийшло навпаки і за це ще ніхто не поніс покарання,тоді б х оч хтось боявся народу.

\iusr{Lesia Buzhuk}
\textbf{Volodymyr Viatrovych} 

пройшли, але адекватний свій захист був завжди. \enquote{Чудом} була ефективна
спонтанна взаємодія \enquote{радикальної} і \enquote{мирної} частин у найкритичніші моменти,
можливим стало саме завдяки достатній кількості мотивованих патріотів, які мали
досвід кількарічної діяльності - і гуманітарної, і прикладної військової та
давали приклад іншим. І готовності всіх учасників оперативно діяти у
правильному напрямку, власне завдяки отій патріотичній тяглості, що передається
через покоління... Результат невпинної і часто невидимої роботі тих, хто, як
зазначає Оксана Стефанівна, ніколи не демобілізувався від боротьби за Україну
та зумів передати головне нам прийдешнім. Щодо \enquote{тактичності} по часу -
цілковито Божа допомога, бо мирна бездіяльність дала перевагу ворогам

\iusr{Тарас Гірший}
\textbf{Lesia Buzhuk} сто відсотків

\iusr{Тетяна Носир}
\textbf{Volodymyr Viatrovych} 

одним словом, на Майдані було більше (набагато більше) розумних людей! Вони
своїм прикладом стримували дуже радикально налаштованих і тому Майдан відбувся.

\iusr{Anton Beletsky}
\textbf{Volodymyr Viatrovych} 

це насправді певний прояв рабської, совкової ментальності, людей бьють, а вони танці танцюють.

\enquote{мирний майдан} це був вимушений крок через те що люди не були готові
психологічно дати справжню справедливу реакцію.

В мініаполісі коли поліція вбила людину, то люди спалили відділок поліції і
постріляли поліцейських і всім чхати було на \enquote{ооой давайте без провокаці}

\iusr{Люба Пивоварник}
\textbf{Volodymyr Viatrovych} 

нажаль \enquote{благадаря} таким утопістам ми продовжуємо мешкати в
кацапсько-єврейській резервації з навязливою \enquote{незалежністью}....


\end{itemize} % }

\iusr{Олег Гуменюк}

Що за сварки в коментарях? Чому ми знову дуріємо?

Ближче до теми яку авторка подала в порівнянні і на перспективу і вже як
приклад для інших. Але ще нам багато треба працювати над собою у несприйнятті
популізму і толерантності між собою.

Дякую, пані Оксані!


\iusr{Юрий Поляков}

Варіант парадоксального розуміння того, як проходить навчання держав (пробачте,
але є і така думка).

 @igg{fbicon.flag.triangular}  Грузія – перший курс (революція троянд), другий курс (курс внутрішнього
путіна-іванішвілі)

 @igg{fbicon.flag.triangular}  Молдова – перший курс (погром парламенту), другий курс (курс внутрішнього
путіна-додона), третій курс (євро-орієнтація)

 @igg{fbicon.flag.triangular} Киргизія - перший курс (погроми парламенту), другий курс (курс внутрішнього
путіна-...)

 @igg{fbicon.flag.triangular} Казахстан - перший курс (погроми)

 @igg{fbicon.flag.triangular} Україна - перший курс (євро-орієнтація), другий курс (курс внутрішнього
путіна-януковича),... п'ятий курс (пошуки провідника до українського дива)

 @igg{fbicon.flag.triangular} Білорусь - зразу вищий курс (самрт-революція - не противитись злу насиллям),
кандидатська (дао-очікування природного само-замикання зла)

 @igg{fbicon.flag.triangular} Росія - зразу вищий курс (самрт-революція - не противитись злу насиллям),
кандидатська (дао-очікування природного само-замикання зла)

/На фото гуру Кличко хоче зупинити вогняну фазу революції..., щоб відбулось
само-замикання зла/

\ifcmt
  ig https://scontent-frt3-2.xx.fbcdn.net/v/t39.30808-6/271701037_4788105047934078_9187218432163742412_n.jpg?_nc_cat=103&ccb=1-5&_nc_sid=dbeb18&_nc_ohc=IEn6IuqZMQIAX_ZHcxr&_nc_ht=scontent-frt3-2.xx&oh=00_AT-QX_4oCvg5OavmHK-lf3qF_UaD6nOPSxPU3BORKr33Ug&oe=61E29A55
  @width 0.3
\fi

\iusr{Taras Zarubko}
\textbf{Юрий Поляков} Я тоді був з Кличком трохи не згоден... Але виступити одному проти натовпу - це дуже сильний вчинок!

\iusr{Жека Велес}
\textbf{Taras Zarubko} він вже тоді тренувався бути мером. Да і не зробили йому нічого. Остудили трохи.

\iusr{Галина Олійник}
\textbf{Taras Zarubko} згодна з Вами,один його хибний рух...

\iusr{Sergii Sushytskyi}
Гарна сатира)

\iusr{Тетяна Костенко}
\textbf{Юрий Поляков} А якби не втримався і легенько бемцнув?  @igg{fbicon.beaming.face.smiling.eyes} 

\iusr{Роман Опацький}
\textbf{Тетяна Костенко} думаю, що після цього його б там забемцали..)

\iusr{Галина Олійник}
\textbf{Тетяна Костенко} одразу видно, що там небули.

\iusr{Тетяна Костенко}
\textbf{Галина Олійник} у вас зір -15.  @igg{fbicon.laugh.rolling.floor} 

\iusr{Boris Dan}
Тренувався бути президентом, потім порошенко умовив стати мером

\iusr{Галина Куліш}

Якщо дійсно є матеріальне втілення Бога, то Він був на Майдані. Я це знаю. Я
відчувала Його присутність через екран. Сила любові до землі предків і жага
свободи змогла тоді сфокусуватися і перемогти. А ідея про День подяки воїнам
світла є слушною. Якщо і не на державному рівні, то народ підтримає.

\begin{itemize} % {
\iusr{Роман Дашко}
\textbf{Галина Куліш} +++


\iusr{Viktoriya Horbach}
\textbf{Галина Куліш} 

те що з нами на Майдані був Бог це однозначно! Я це бачила це перебуваючи
безпосередньо в самому центрі подій, не через екран. Особливо це було особливо
наочно вночі з 18.02.14 на 19.02.14, коли ми були оточені з усіх боків, на
невеликому клаптику, за площею, оточили себе палаючими барикадами і дим йшов
врізнобіч з нашого \enquote{клаптика} на спецназ, це було просто фантастично!!!

\begin{itemize} % {
\iusr{Наталья Федосеева}
\textbf{Вікторія Горбач} і коли розуміли як нас насправлі мало і чекали автобуси з львівянами та франківцями як порятунку. Я їх зустрічала біля Володимирського собору. Палаючі профспілки, поранені в коридорах, афганська лікарня. Все таке свіже в пам'яті і так болить реваншизм...
\end{itemize} % }

\iusr{Галина Куліш}
\textbf{Вікторія Горбач} 

я була на Майдані в березні. Вже майже все прибрали, залишилася сцена, туалети,
трохи бруківки і море квітів. Їх несли і несли люди. А стан душі не описати
словами. Навіть заплющивши очі, щоб не бачити де знаходишся, сльози з очей
капають, така енергетика добра і світла, відчуття волі і захищеності... Згадую,
і плачу. Чи вдасться зберегти і примножити тодішні здобутки?

\begin{itemize} % {
\iusr{Viktoriya Horbach}
\textbf{Галина Куліш} з 2004 року зерно посіяне і це поле вже рясно колоситься, наступне зерно теж сіється, і це зерно волі ,свободи в нашому коді нації- воно незнищенне. Тримаймося, єднаймося, шануймося і будьмо гідними та розумними!
\end{itemize} % }

\iusr{Lyuba Ahmed}
\textbf{Галина Куліш} ВIН був на Майданi i у 2004 р. Я це вiдчула там. Один раз за життя так сильно вiдчула.

\begin{itemize} % {
\iusr{Оксана Євтушевська}
\textbf{Lyuba Ahmed} це різні майдани. На першому було багато грошей, менше ідеї, дуже гарна постановка і організація, а люди просто піднялися, бо кандидат Янукович перегинав палицю. Другий-це піднесення і дух свободи. Не думаю, що він був зазделегіть спланованим. Тому пішов не за сценарієм і тому ідея перемогла. Але....Бог побачив, що все-таки ідея виявилася не на стільки сильною і полишив нашу землю, швидше за все , якщо не на завжди, то на дуже довго.

\iusr{Світлана Трофимчук}
\textbf{Оксана Євтушевська} а звідки Ви взяли,що полишив?

\iusr{Клара Мухина}
\textbf{Оксана Євтушевська} не панiкуйте....
\end{itemize} % }

\iusr{Тарас Гірший}
\textbf{Галина Куліш} 

зі сцени постійно йшли молитви всіх конфесій і вір, було встановлено великого
хреста, дим від шин валив тільки в сторону беркута, ніколи незабуваймо
подвиги всіх хто там був від бабусь і панянок до молоді та дітей, вічная память
всім героям павшим на полях бою майдану та сходу країни, майданів всіх міст та
містечок за нашу свободу, боротьба продовжується й далі, ворог хитріший .Ми ще
не відірвались настільки, щоб видихнути з легкістю. З Богом до перемоги ! Якщо
до нас зайде московія, буде далеко не як в Казахстані, буде як в 39 році на
заході України, загинуть мільйони. В нас дороги назад немає.


\iusr{Александр Никулица}
\textbf{Тарас Гірший} 100%

\iusr{Taras Zarubko}
\textbf{Галина Куліш} Якщо Диявол існує, то я відчував його присутність коли гинула Небесна сотня!

\iusr{Галина Куліш}
\textbf{Taras Zarubko} вічна боротьба Добра і Зла. Але Добро ЗАВЖДИ перемагає. Бо лише воно має можливість творити.

\iusr{Taras Zarubko}
\textbf{Галина Куліш} Дякую за вашу реакцію! Я сказав правду про свої відчуття в той момент ...

\iusr{Ніна Кравченко}
\textbf{Галина Куліш} саме так! Підтверджую, як свідок!

\begin{itemize} % {
\iusr{Ніна Кравченко}
Чомусь найбільше відчула в неділю, 16 го здається. Холодно було добряче, і зі сцени повідомили рішення йти до Верховної Ради. І було зрозуміло, що люди загинуть. І це безповоротньо. Але присутність Святого Духа відчувалась майже всіма.
\end{itemize} % }

\iusr{Vadim Oliinyk}
\textbf{Галина Куліш} я пам’ятаю, особисто відчував, що ми там були в самому епіцентрі істинної Сили.

\iusr{Клара Мухина}
\textbf{Галина Куліш} НА ДЕР- ЖАВ-НО-МУ !!!

\iusr{Boris Dan}
Після розстрілу на вул. Інститутській буддійський монах з бубном проводжав душі
загиблих, це відчувалось

\iusr{Ольвія Білан}
\textbf{Галина Куліш} ..а чому б і на державному не зробити?

\end{itemize} % }

\iusr{Olga Ludewig}

Вирватися на волю з московського болота! Україні пощастило, вона скористалася
вікном можливостей, хоч і заплатила надвисоку ціну - кров'ю патріотів. Слава
Нескореним! @igg{fbicon.heart.blue}  @igg{fbicon.heart.yellow} 

\begin{itemize} % {
\iusr{Олена Єрьоменко}
\textbf{Olga Ludewig} Слава!

\iusr{Ella Vyadro}
\textbf{Olga Ludewig} дай боже щоб вирвалась навіки!

\iusr{Ольга Новохатько}
\textbf{Olga Ludewig} Героям слава!

\iusr{Tetyana Solomennyk}
\textbf{Olga Ludewig} Ми гайдамаки, ми всі однакі –
Ми ненавидим вороже ярмо!
Йшли діди на муки, підуть і правнуки –
Ми за народ життя своє дамо! Вірш і музика Осипа Маковея 

\url{https://www.youtube.com/watch?v=2DBoFMQ34XI}
\end{itemize} % }

\iusr{Vadim Oliinyk}

Пам’ятаю ці справжні погляди, пронизані добром до ближнього і глибоким бажанням
спільно допомогти врятувати нашу країну. Такої кількості щирих і щасливих
поглядів, я до того ще ніде не зустрічав. Квінсистеція патріотизму нас
об’єднала в ті дні і врятувала. Страх відступав перед невідомістю, але з вірою
в краще майбутнє.

\iusr{Галина Олійник}
\textbf{Vadim Oliinyk} туди тягло, бо знав, там брати, там справжнє плече, там віра і правда, і нема зради!

\iusr{Степан Шніцар}

Варто звернути увагу на зміну технології кремля - поведінка тітушок в Україні
та Казахстані діаметрально різна. В Україні більш розвинуте громадянське
суспільство - ми легко навчилися їх вичисляти, натомість в Казахстані
протестувальники повелися на дії провокаторів (можливо, через запальний більш
азійський темперамент).

\begin{itemize} % {
\iusr{Alexander Derkach}
не тільки вичисляти, але й проводити \enquote{розʼяснювальну роботу})

\iusr{Оля Башта}
\textbf{Alexander Derkach} добрим словом і копняками

\iusr{Tanya Alex}
\textbf{Степан Шніцар} Путло засунуло, путло зайшло. Більш геніальних послідовних провокаторів, як цей немає в усьому світі

\iusr{Oleg Mykhailov}
\textbf{Степан Шніцар} просто нас було більше, а якщо точно, \enquote{дохуя}, як казав Митець.

\iusr{Любов Несторович}
\textbf{Степан Шніцар} кримінальні ватажки в Казакстані отримали вказівки мародерствувати, що і виконали.

\iusr{Taras Zarubko}
\textbf{Степан Шніцар} 

Зібрали в Херсоні зо 2 автобуси \enquote{добрі люди} бомжів для дискредитації
майдану ... Ну і охорона Майдану оперативно їх видворила!

Йду в периметр Майдану я і побратим Вася з Городенки.

Мене пропускають, а Васю ні! Прийняли за бомжа!

До мене: \enquote{Ти йди собі!} Я до них з кулаками! \enquote{Шооооо! Василя не
пускаєте????} Подіяло ! Впустили обох.

\begin{itemize} % {
\iusr{Marina Mischenko}
\textbf{Taras Zarubko} 

В ті дні я супроводжувала нетверезого професора з новорічного корпоративу в
універі. Вийшли з метро Майдан, піднялися з переходу за сценою- і тут до нас
підрулюють ))) Я як гаркну: \enquote{Ми з метро, додому йдемо!!!})))))) Пропустили,
соколики)))


\iusr{Marina Mischenko}

Читала розповідь казахського архітектора про участь в подіях. Вразило, що так
слабо розвинуто волонтерство. Пропонували тільки воду, а не як у нас: канапки,
гарячий чай- каву, суп- кашу, ковдри, саморобні пічки з дровами... ще вразила
відсутність координації: навіть мегафона не було. У нас би чувак вже побіг
купляти мегафон, акумулятори, і тяг би це все на майдан. І ще вразило як
жорстоко били казахських солдат, які здалися протестувальникам. Наші натомість
пригощали беркутів чайком з термосів, встромляли їм квіточки в щити... І в
гарячу фазу революції тих хто потрапив в полон або здався не били всерйоз.
Звісно бажаючі були, але їм не давали.

\iusr{Taras Zarubko}
\textbf{Marina Mischenko} 

Був випадок,що загорілося кілька палаток.( Певно через необережне поводження з
вогнем але всі вважали, що підпал). Хлопці ловлять якогось наркомана під дозою
і от от буде самосуд! Навколо збирається натовп! Я туди проштовхуюсь до тієї
неадекватної людини, яка навіть нічого сказати на свій захист не може! Стаю
адвокатом! Виявилося, що толком ніхто нічого не бачив!

Правда його обшукали і конфіскували ( ограбували?) десь 1000 грн. І відпустили.
Скажу чесно, я побоявся вимагати не брати ті гроші! Бо натовп є натовп,
невідомо, як воно повернулося б....


\iusr{Степан Шніцар}
\textbf{Taras Zarubko} 

ми на своїй барикаді п'яних взагалі не впускали. Якщо бачили \enquote{клієнта} в
цікавому стані, завжди питали: \enquote{Ану скажи Кіліманджаро!} А одного під кайфом
чотири години відтісняли від барикад, а він все норовив прорватися.

\end{itemize} % }

\iusr{Iryna Lange}

Величезну роль в Україні зіграла наявність політичної опозиції! Незважаючи на
будь-які обмеження, ми мали свою політичну опозиційну еліту, яка у вирішальний
момент взяла на себе керівну роль! Мсклям не вдалося розділити народ і
політиків, незважаючи на закинуті лозунги: \enquote{Без політики і політиків!}.
Політична опозиція стала на бік народу і з народом! Саме тому Європа і світ
зважали на нас і не відпускали уваги! Українська діаспора розгорнула
активність, але головне, звичайно, безстрашний український народ! Дякую вам за
це!


\iusr{Анатолий Бутенко}
\textbf{Степан Шніцар} ніде вагонами і автобусами масово не завозили ватне бидло як це відбувалося за маршрутом донецк-київ

\iusr{Анатолий Бутенко}
Затерли силовики і прокуратура звіт про оплату бюджетникам обласних центрів за участь в анти-майдані?

\iusr{Соловйова Юлия}
\textbf{Степан Шніцар}, 

це говорить про те, що \enquote{миротворці} роблять виводи з своїх
\enquote{помилок}, чого не скажеш про нас, про українців, та наших сусідів

\end{itemize} % }

\iusr{Тетяна Носир}

Так, пані Оксана! Майдан тоді переміг, але про які дні пам'яті Ви говорите,
якщо в 2019 році його зрадили. Зрадили навіть ті, які були на Майдані, бо
обрали 95 квартал, який висміював його. А ті керівники, які посилали людей на
Антимайдан, так і працюють далі керівниками. Чому не заборонили ОПЗЖ відразу,
як колись КПРС, допустили до виборів?

\begin{itemize} % {
\iusr{Олександр Гудзенко}
\textbf{Тетяна Носир} 

Чому не заборонили ПР, в 2014 р., яку визнали злочинною ? А вони увійшли в
парламент як ОпоБлок? Можливо зрада Майдану відбулася ще в 14 році ?

\iusr{Тетяна Носир}
\textbf{Олександр Гудзенко} 

ПР ніби заборонили, перейменувавши в ОПЗЖ. А ті, хто були при владі за
Януковича, залишилися і після 2014 року. Це й була зрада Майдану ще тоді!

\iusr{Клара Мухина}
\textbf{Олександр Гудзенко} в iсторii ж, нема слова чому..

\iusr{Клара Мухина}
\textbf{Тетяна Носир} почитайте коментар вище: це навчання держави, 3 курс...

\iusr{Марян Мигала}
\textbf{Тетяна Носир} 

ти хвора??? Скоро потрох сяде і це буде найкращий подарунок майданівцям! І тим
патріотам яких знищив потрох в котлах!! І це зробить Зеленський!!!

\end{itemize} % }

\iusr{Юлія Овчиннікова}
Ох і класна стаття! Дякую!

\iusr{Нина Ващенко}
\textbf{Julia Ovchinnikova} 

Прочитала на одному диханні! Не була там, але переживала як особисту
трагедію, тому, читаю і плачу! Дякую всім і кожному хто там був, хто вистояв, хто
тягнув на санчатах дрова, хто ніс їжу, хто подавав бруківку і вічна пам'ять
...тим, хто відлетів в Небо! Тоді, українці були справжніми! В, звісно ж, Господь
і Правда, була на нашій стороні ...Як би, ж, то, всі прочитали, цю
статтю... та, ще, щоб до всіх дійшло..... Хоча... Дякую, Вам, Оксаночка....

\iusr{Галина Калашник}

Якби ми себе не критикували за якісь речі, як би сором'язливо не мовчали, коли
нас хвалять, але Господь поміг нам, бо ми того варті! Царство Небесне усім, чиї
імена знаємо і тих, про кого не пишуть, бо без вісти пропали...

\iusr{Оля Башта}
\textbf{Галина Калашник} критикувати треба, але конструктивно - щоб не повторювати помилок, і використовувати вдале

\iusr{Василь Максимов}

В Україні кремлівський карлик сподівався на своєрідний \enquote{бліцкріґ}, тобто на
тотальну сварку між всіма жителями України, ворожнечу між історичними
землями(Захід-Схід, Північ-Південь), між областями, населеними пунктами і т. п.
Але жорстоко \enquote{обламався}, бо сплеск Патріотизму перевершив сподівання навіть
найбільших оптимістів! Вже після того \enquote{облому} Кремль вирішив діяти хитріше,
поставивши за мету знищувати українську Державу зсередини... І ось на арену
вийшов відомий \enquote{джокер}... Але й тут, незважаючи на такі потуги Москви - теж
неминучий \enquote{облом}! Він вже наближається! Як, зрештою, і погибель останньої у
світі рашистської імперії...

\begin{itemize} % {
\iusr{Юрій Шульгін}
\textbf{Василь Максимов} дійсно, Україну успішно знищують зсередини, на очах отупілих жителів.

\iusr{Оксана Поплавська}
\textbf{Юрій Шульгін} менше Дмитра Спєвакова дивіться.
\end{itemize} % }

\iusr{Олена Олена}
Образно нам лише вибили зуби і поламали об нас свої...,але якби відкл.інтернет, перебили б усіх

\iusr{Ar Tem}
\textbf{Олена Олена} або знесли б зрадників ще швидше

\iusr{Anna V. Melnyk}

Погоджуюся з «числом Дельта». Але все ж питання - білоруські демонстрації теж
були дуже чисельні, чому в них не пішло? Тобто я до того - число, безумовно,
має значення, але є ще і, так би мовити, складова войовничості. Отой момент,
коли чомусь приходить розуміння, що час пісень минув, і прийшов нас бруківки,
коктейля, палиці, рушниці, you name it...

\begin{itemize} % {
\iusr{Ольга Рубан}
\textbf{Anna V. Melnyk} не пішло, бо вони були проти Лукашенка, але не проти Росії. вони у них браття не дивлячись нінащо

\iusr{Anna V. Melnyk}
\textbf{Ольга Рубан} ну ми цю мантру добре знаємо. Тому і «Майдан» в них усіх лайливе має значення

\iusr{Светлана Ковтун}
\textbf{Ольга Рубан} так і є, в яблучко.

\iusr{Helg Forester}
А у них у парламенті хіба були політики проти луки мудіщева? Чи облради? Так стихійний, неструктурований рух

\iusr{Любов Несторович}
\textbf{Anna V. Melnyk} білоруси масово нам шприкали, що наш Майдан нічого не дав, бо ми всеодно вибрали клоуна.

\iusr{Клара Мухина}
\textbf{Любов Несторович} вони не дивляться глибше.

\iusr{Гузей Валерій}
\textbf{Anna V. Melnyk} В Білорусії опозиція до Лукашенка була за духом більш проросійська, ніж прозахідна. Тому вони й здулися.

\iusr{Володимир Сімонов}
\textbf{Гузей Валерій} 100\% !!! А ще багато білорусів не знають рідної мови !!! Втрата ідентичності !!! Зомбоящик працює !!! Їх лякали нашим Майданом !!!

\iusr{Галина Олійник}
\textbf{Anna V. Melnyk} танці і бубни, нічого не вирішують, нажаль.

\iusr{Maryna Belbas Rohiv}
\textbf{Anna V. Melnyk} в них не було сильної підтримки опозиційних політиків, Тіхановській прийшлось тікати в Прибалтику.

\iusr{Клара Мухина}
\textbf{Maryna Belbas Rohiv} i нe було пiдтримки заходу, дiаспори

\iusr{Ar Tem}
\textbf{Anna V. Melnyk} бо не було Духу Божого, навідміну від Майдану
\end{itemize} % }

\iusr{Ніна Кравченко}
А я вважаю, що вихід на вулиці 1 грудня 2013 року приблизно одного мільйону киян та гостей столиці теж добряче настрашив владу!

\iusr{Les Teslij}
Впевнений, що людей розстрілювали снайпери з оркостану.

\begin{itemize} % {
\iusr{Artem Kalynychenko}
В Венесуелі теж...

\iusr{Artem Kalynychenko}

\ifcmt
  ig https://scontent-frx5-1.xx.fbcdn.net/v/t39.30808-6/271714585_607358590553780_4759984600555170464_n.jpg?_nc_cat=105&ccb=1-5&_nc_sid=dbeb18&_nc_ohc=bXii8EWSynQAX8zy8WM&_nc_ht=scontent-frx5-1.xx&oh=00_AT_nkQc7JpabbhSoYPLwAi1Jt83CyROaY-eCUNhYjsSbzQ&oe=61E2EA9E
  @width 0.2
\fi

\iusr{Володимир Титюк}
\textbf{Les Teslij} ви праві

\iusr{Mariya Terekhova}
\textbf{Les Teslij} Чому ж не стріляли по тих, що били себе в груди на тракторі?

\iusr{Людмила Цирубалко}
\textbf{Mariya Terekhova} А Ви досліджували всі обставини, співставляли час, місце цих розстрілів, що питаєтесь?! Не баламутьте тут

\iusr{Mariya Terekhova}
\textbf{Людмила Цирубалко} що тут досліджувати? Це всі бачили і це не є ніяким секретом. Баламутите мабуть, Ви, бо бачите снайперів, яких побачити не можна по причині, що вони інакше б не були снайперами. От це треба досліджувати.

\end{itemize} % }

\iusr{Лариса Юхименко}

Іду до хлопців на Майдан без особливих днів, дякую і вибачаюсь, вибачаюсь і
дякую....

Зі мною розпалений боєм хлопець, який кинув на бігу: \enquote{Людоньки, не розходьтеся,
бо тоді нам всьо!..}

Дякую, пані Оксано!


\iusr{Stepan Volynyaka}

Прикметно, що на площі Незалежності у Києві впродовж трьох Майданів навіть не
було розбите жодне вікно на Хрещатику. Вже не йдеться про мародерство чи щось
інше, хоча провокаторів і \enquote{одкб}-ешників там було чимало - боялися, мабуть,
виділитися цією гидотою серед загалу...

Будинок Профспілок- справа рук антимайданівців!


\iusr{Ігор Ясенівський}

Усім треба вчитися робити (планувати і втілювати) безкровні революції. Готувати
фахові кадри. Загартовувати дух і тіло. Мислити на декілька кроків наперед.
Гуртуватися навколо спільної ідеї - кращої долі для України та українців. У
своїх громадах творити сприятливі умови для життя та праці. І це мусить бути
національна стратегія творення держави. Коли кращі з кращих ідуть у політику,
щоб своїми діями народ підняти з колін, дати йому крила (мрію), цілі, ресурси
(засоби), можливості, завдання. Українці на Майдані та на Донбасі втратили
багато свідомих прогресивних людей, котрі могли б на керівних посадах у столиці
чи в своїх громадах принести багато добра людям та Україні. На жаль, це
непоправні втрати, свідоме знищення кращих, пасіонарних... А їх могло б і не
бути, коли б народ був нацією, а не населенням... Якби при владі були патріоти,
а не ділки, пристосуванці-хамелеони, аферисти-спекулянти,
казнокради-хабарники... 

Час показав, що важливих глобальних ментальних змін в Україні не відбулося.
Українці далі поділені за симпатіями до політиків, лідерів громадських
організацій, очільників Церков, релігійно заангажовані, слухають російські
пісні, не бачать перспектив в Україні (їздять на заробітки за кордон, виїхали
туди на постійне проживання). 

Духовно багатим, проте бідним матеріально нема часу на вищі матерії, бо треба
виживати, щодня бігати як білка в колесі, крутитися, щоби вижити... 

А матеріально багатим, але духовно убогим начхати на ментальні-моральні-духовні
цінності. Вони жирують, живуть на широку ногу, смітять грошима в елітних
ресторанах і клубах, їздять за кордон на шопінг та відпочинок, люблять та
спонсорують чуже, а не своє! 

Поки не буде правдивої єдності, спільної мети та роботи - не буде успішної
щасливої процвітаючої України!!!

\begin{itemize} % {
\iusr{Людмила Куделя}
\textbf{Ігор Ясенівський} І ще об'єднуватись і вибирати достойних лідерів-профеміоналів і патріотів до керівних постів, а не бариг-корупціонерів і навести лад у державі. Велика шана героям Майдану і захисникам Вітчизни!
\end{itemize} % }

\iusr{Мирослав Логийко}

В 2014-му році у України вийшло не тільки завдяки майдану і добровольцям, а й
завдяки Парламенту України.

У Парламенті України були проукраїнські та проєвропейські політики та партії,
які втримали державу і заклали фундамент нової.

Саме Парламент України, після втечі януковича і його влади, оголосив початок
проведення антитерористичної операції.

Парламент України провів вибори і легітимізував владу.

І тепер саме парламент України є ціллю путіна-зеленського.

ОПЗЖ вже заявили про свої наміри кинути мандати, а це в свою чергу дозволяє
зеленському розпустити парламент.

\url{https://youtu.be/Kccmy-TqmgM}

Зеленський з його монобільшістю та проросійскою підтримкою це підхватить.

\url{https://m.facebook.com/story.php?story_fbid=4932810226762610&id=100001010458434}

Цього разу Путін не повторить своєї помилки 2014-го.

Главный вывод по событиям в Казахстане - что бы Западу было кому помогать, в
том числе оружием, нужна организованная оппозиция в стране и лидер у этой
оппозиции.

Запад не будет поставлять оружие и помощь разроздненым группам сопротивления.

Именно этого сейчас добивается путин/зеленський - убрать лидера оппозиции, как
объединяющую фигуру, и уничтожить парламентаризм как таковой, что бы не было
альтернатив.

Уже год про это пишу:

\url{https://m.facebook.com/story.php?story_fbid=4932810226762610&id=100001010458434}

\begin{itemize} % {
\iusr{Тетяна Бевська}
\textbf{Miroslav Logiyko} Варто дякувати Р. Кошулинському. Саме він зібрав Верховну Раду України.

\begin{itemize} % {
\iusr{Мирослав Логийко}
\textbf{Тетяна Бевська} не вірно. Дяка декільком десяткам найактивніших парламентарів, та мільйонам на вулицях.

\iusr{Тетяна Бевська}
\textbf{Miroslav Logiyko} активні парламентарі та народ підтримали. Але заперечувати ініціативу Р. Кошулинського не вірно.

\iusr{Мирослав Логийко}
\textbf{Тетяна Бевська} 

ще раз. Я не проти Руслана. Так він був один з активних. Заради його
кандидатури на довиборах по 87 округу, Звіробій зняла свою кандидатуру та
закликала своїх прибічників підтримати Руслана.

Так, Тіщенко з СБУ, Кордом і поліцією потім підтасовували вибори і вирішували
між людиною Коломойського чи Зеленського.

Але Руслан вносить зараз розбрат в проукраїнські проєвропейські ряди.

Треба об'єднуватись, а не поливати фейками.


\iusr{Тетяна Бевська}

Р. Кошулинський - заступник голови Верховної Ради України \enquote{20 лютого 2014 року,
єдиний на той час легітимний представник влади в країні, з власної ініціативи
зібрав народних депутатів на засідання Верховної Ради, щоб припинити
кровопролиття. Того дня на пропозицію Кошулинського депутати проголосували за
усунення Януковича від влади та повернення «Беркуту» на місця дислокації.
Рішення, прийняті на тому засіданні ВРУ, стали для світової спільноти підставою
визнати зміну влади в Україні легітимною.}

Що тут не вірно?

Про розбрат, про фейки прошу аргументи.

\iusr{Тетяна Костенко}
\textbf{Тетяна Бевська} Теоретичне запиьання. Хто зараз зможе зібрати більшість ВР? Здається, що більшості для прийняття рішень не буде.

\end{itemize} % }

\iusr{Олег Зозуля}
\textbf{Miroslav Logiyko} Парламент приймав рішення під тиском Майдану.

\iusr{Мирослав Логийко}
\textbf{Олег Зозуля} звісно, але він був. І частина парламентарів навіть була на Майдані та підтримувала Майдан.

\iusr{Борис Курдибаха}
\textbf{Miroslav Logiyko} Дивно, що після майдану Порошенко викинув з ВР радикальну Свободу за допомогою охендовського.

\begin{itemize} % {
\iusr{Мирослав Логийко}
\textbf{Борис Курдибаха} 

радикальна свобода (Тягнибок) саботувала призначення міністра оборони у
найважливіший момент. Коле ще не було президента, а тільки виконуючий обов'язки
Турчинов, і комусь треба було віддавати накази.

А потім і виявилися причини цього саботажу - Свобода (не рядові члени партії,
бо вони патріоти, а верхівка партії) отримувала кожен місяць гроші від
януковича.

Вони були по суті ручною опозицією.

Саме радикалізація питання робила потрібну картинку для телебачення РФ про
фашистів, і віддавала частину українців.


\iusr{Борис Курдибаха}
\textbf{Miroslav Logiyko} 

В «черной бухгалтерии» Партии регионов фигурируют Тягнибок и Ярош. Не
забуваємо, що тоді не було війни і протистояння на Майдані.

А також не забуваємо, що робив Порошенко в уряді Азарова і т.д. і т.п.


\iusr{Мирослав Логийко}
\textbf{Борис Курдибаха} про яроша не пам'ятаю такого - нагадайте посиланням будете ласкаві

\iusr{Борис Курдибаха}
\textbf{Miroslav Logiyko} 

\href{https://www.economics-prorok.com/2016/06/blog-post_233-2.html}{%
В «черной бухгалтерии» Партии регионов фигурируют Тягнибок и Ярош?, economics-prorok.com, 26.06.2016%
}

\iusr{Мирослав Логийко}
\textbf{Борис Курдибаха} будь ласка інше джерело, бо сказав брехло Лещенко у цьому

\iusr{Борис Курдибаха}
\textbf{Miroslav Logiyko} 

Підхід змінився у 2005 році. Почали з'являтися статті, в яких прослідковували
еволюцію від \enquote{Варти Руху} до Соціал-національної партії (з її подібними
до свастики відзнаками) і далі до ультранаціоналістичного утворення
\enquote{Свобода}.  Зважаючи на нову інформацію, особи, які мали доступ до
\enquote{Папки 48}, дійшли висновку, що необхідно дізнатися, ким був агент
\enquote{Загону З} з позивним \enquote{Волонтер}. Існувала ймовірність того, що
\enquote{Волонтер} усе ще вів активну діяльність і виконував накази Москви. З
2006 до 2009 років декількох експертів попросили ідентифікувати особу, яка
очолила \enquote{Варту Руху} після Дмитра Поїзда (на що вказував документ) –
цією особою і був би \enquote{Волонтер}. Прорив відбувся на початку 2009 року,
і згодом ця інформація підтвердилася влітку. Усі джерела повідомили, що цією
особою був Олег Тягнибок! Зважаючи на те, що Тягнибок очолює \enquote{Свободу}
– організацію, яка не припиняє нагадувати всім, що вона готова захищати
національні інтереси України, доцільним було б навести докази того, що Тягнибок
не є тим, за кого себе видає, і що він може бути повною протилежністю того
образу, який він собі створює. У 2010 році з'явився консенсус щодо необхідності
опублікувати \enquote{Папку 48}.


\iusr{Борис Курдибаха}
\textbf{Miroslav Logiyko} Лещенко і по Тягнибоку фігурує. А Смішко вас не турбує?

\iusr{Мирослав Логийко}
\textbf{Борис Курдибаха} 

Смішко - це проект Гордона/Ахметова для відбирання відсотків від Порошенко
електорату, яким потрібен \enquote{типу військовий}, але \enquote{аби не Порошенко}.

Смішко відкусив добряче - значить технологія спрацювала, його добре розкрутили

\iusr{Борис Курдибаха}
\textbf{Miroslav Logiyko}
Европа дала команду Порошенку, щоб Свободи не було в ВР.
І ось що записано в Акті Росія НАТО про націоналізм:
Россия и НАТО исходят из того, что общая цель укрепления безопасности и стабильности в евроатлантическом регионе во благо всех стран требует ответа на новые риски и вызовы, такие, как агрессивный национализм,...
Тепер пазли склалися: НАТО, Європа, Росія, Порошенко. І Свободу очорнили, пропаганда топить проти, охендовському дали команду, дали ордена.
Як Свобода могла провалится після Майдану? Тільки за допомогою фальсіфікацій.
А вам вирішувати, що таке націоналізм.

\iusr{Мирослав Логийко}
\textbf{Борис Курдибаха} 

інколи, я вас засмучу, політичну силу, радикальну, таку як Свобода,
використовує влада, така як яника, для "жахливої картинки" для телебачення.

Так, рядові члени Свободи - це хороші люди і патріоти, але верхівка партії - це
запроданці, які просувають російські наративи. Наприклад:

- \enquote{Україна сама по собі, і не треба нам в Європу і НАТО}.

Це брехня, бо Україна опинилася прифронтовою державою між двох центрів сил на
планеті. Сама по собі вона не витримає, не приєднавшись до Європи та НАТО, або
РФ та ОДКБ.

Меседж \enquote{Україна сама по собі}, який часто використовує Фаріон, Тягнибок - це
пастка для того, щоб Кремлю буле легше по частинах завоювати Україну.

- Тягнибок саботував призначення міністра оборони в 20-х числах 2014-го. У
найвідповідальніший час, замість того, щоб поступитися і призначити генерала
сухопутних військ, який знав як вести військові дії, він став проти (так як
квота Свободи була) і просував адмірала (який не знав як воювати, і знав тільки
кораблі), яких на той час в України вже майже не було, бо Крим вже був
захоплений.

Саботаж та брехня політичної верхівки Свободи і зробила її токсичною і для
заходу. Вони зрозуміли, що Свобода - це проект кремля.

Але не рядові члени партії - вони дійсно ідейні і патріоти.

\end{itemize} % }

\iusr{Василь Дуткевич}
\textbf{Miroslav Logiyko} Треба 151 ....

\iusr{Мирослав Логийко}
\textbf{Василь Дуткевич} тоді тільки за допомогою Слуг

\iusr{Василь Дуткевич}
\textbf{Miroslav Logiyko} Є і без них, але продажність серед них дістала!?!

\iusr{Борис Курдибаха}
\textbf{Miroslav Logiyko} вам палець покажешь і ви будете сміятися. Не цікаві.


\end{itemize} % }

\iusr{Masha Popova}

Підтримую повністю день подяки..

НАША Армія повинна цю подяку відчувати щодня, нашими діями, і нашим відношенням
,до тих, хто в АТО і до тих хто з АТО повернувся...

Низький уклін кожному із вас.... Тим, хто мав мужність відірватися від усіх
зручностей, і піти туди..... де гАряче...

Це дійсно сильні духом і характером ЛЮДИ@igg{fbicon.heart.red} @igg{fbicon.fist.right.facing}  @igg{fbicon.fist.left.facing} 


\iusr{Olena Khalimon}

Для когось Майдани - це неможливість дожувати спокійно батона та «як ви
набридли з Майданами», а для свідомих людей - це найвище піднесення людського
духу, це прагнення найсміливіших людей боротися за правду, обирати щоразу
гідність людську, коли виборюючи свободу - отримуємо й хліб!

Хто боровся за правду, не зможе не зачаруватися єдності людей, що світилися
невимовною радістю на Майдані! Там стояли не раби, там були Воїни Світла!

\iusr{Marta Farion}

Геніяльнй аналіз пані Оксано! Все залежить від людей, Народ є ключевим в цій
битві. Не сидіти в дома. Разом себе організувати і діяти. @igg{fbicon.heart.red} Що можу вам сказати
пані Оксано, ..... I love you. @igg{fbicon.heart.red} @igg{fbicon.rose} 


\iusr{міра онищук}
Дякую, пані Оксано, захоплююсь саме Вашою аналітикою.

\iusr{Леся Кричун}
Дуже гарна стаття. Дякуємо вам Оксана. Побільше би таких тестів, щоби нарозумити наших людей.

\iusr{Юрчак Наталья}

Я весь час про це думаю з глибокою вдячністю. І про той ген, який заклали в
українців попередні покоління борців. Сподівпюсь, що він нікуди не подівся,
хоча на поверхні зараз плаває зелена піна. Дякую, пані Оксано, що Ви не
втомлюєтесь нагадувати нам про дуже важливі речі! @igg{fbicon.heart.blue}  @igg{fbicon.heart.yellow} 


\iusr{Larysa Barabash}
Низький уклін Всім, хто Вистояв.!!! Щиро дякую за Пост!!! Стоїмо. іншого Не дано!

\iusr{Іван Болеста}
Прекрасна аналітична стаття! Дякую, пані Оксано!

\iusr{Ростислав Копей}
Яка не є, а ОПОЗИЦІЯ зеленій багнюці. Іншої опозиції в нас нема. В сусідів вона відсутня взагалі, тому й терплять поразки.

\iusr{Ирина Шевченко}
Пані Оксано, дуже ДЯКУЮ...

\iusr{Марія Рисс}
Дякую п. Оксано. Город треба полоти, а народ виховувати. Дійсно, до декого почало доходити, що могло бути в Україні.

\iusr{Tatiana Solomaha}
вони відкоректували технологіі. треба вже зараз бути готовими протистояти.

\iusr{Kram Valya}
Дякую за важливий пост  @igg{fbicon.hands.pray} @igg{fbicon.flag.ukraina} @igg{fbicon.heart.red}

\iusr{Олександр Тимофійович Дробаха}
Цей ЗАКЛИК має висіти тут щоденно, розслаблятись нема як, вони пруть, як бульдозер !!!

\iusr{Svetlana Grytsenko}
Як завжди - в ціль!

\iusr{Дмитрий Безрук}
\textbf{Oksana Zabuzhko (Оксана Забужко)}

- я ось йому дуже вдячний. Це журналіст. Він у цьому ракурсі зовні на мене
схожий. Світлина 28 січня 2014 р.

\ifcmt
  ig https://scontent-frx5-1.xx.fbcdn.net/v/t39.30808-6/271681608_2116759995155892_5095780899502013644_n.jpg?_nc_cat=100&ccb=1-5&_nc_sid=dbeb18&_nc_ohc=uhdLaqMFgKgAX-8h3jC&_nc_ht=scontent-frx5-1.xx&oh=00_AT9YMnlYWq-TqErkS618HmlAr-cewOMQxpJk7iDAE54W0A&oe=61E29AEC
  @width 0.2
\fi

\iusr{Yulia Blonska}
Лавру, до речі, слід капітально зачистити.

\iusr{Марианна Маркова}
першою все ж таки була Литва, минулого року як раз була 30-та річниця штурму совєтською ще десантурою телебашні у Вільнюсі

\iusr{Lara Onyshchenko}
Дякую, пані Оксано! Слава патріотам!

\iusr{Єлизавета Євтушенко}

Дякую, пані Оксано! ви, як завжди, чітко аналізуєте історичні події і сучасний
політичний момент.

\iusr{Виктория Щигельская}

Не треба відкидати дуже важливий чинник наявності в Україні патріотичної
опозиції - в Казахстані проказахської еліти наразі немає, лишається
просовєцька...


\iusr{Лариса Бойко}
Чудова стаття! Щиро дякую пані Оксано.

\iusr{Максим Беркаль}

Мабуть, на місці РФ має бути просто глуха стіна. Жодних людиноподібних го-, ба-
чи просто рельєфів. Німа пустота, вагітна лише пустотою.

\iusr{Xenia Gerke}
\textbf{Максим Беркаль} а куди - скільки їх там - діти мільйони русскіх?

\iusr{Luba Kyyanovska}

Зараз знову такий момент, що треба збиратись і не вступитись в \enquote{годину Х}, бути
разом всім, хто розуміє страшну небезпеку чорної дикої навали.

\iusr{Андрій Нагірний}
Дуже гарний текст.
Тоді в багатьох дійсно було відчуття Дива, яке відбувалось в реальному часі

\iusr{Sergiu Devdyk}
От на 18, 19, 20 маємо можливість сказати Дякую! 8 років, нє?

\iusr{Віталій Демчук}
Все буде Україна!!!

\iusr{Konstantin Andriychuk}

Про Субтельного підтверджую  @igg{fbicon.smile}  Народжений в Донецьку, в родині комуністів, я
поступово дійшов до вступу до лав добровольчого батальйону в 2014. Але саме
Субтельний став відправною точкою. Ох, і сперечався з батьками ще в школі через
цю книгу.

\iusr{Володимир Титюк}
\textbf{Konstantin Andriychuk} дякую ВАМ,такі як ви спасуть Україну,її народ...

\iusr{Irena Sakharova}
В єдності наша сила @igg{fbicon.biceps.flexed}{repeat=3} 

\iusr{Vyacheslav Yefremenko}

Пам'ятаю, скільки зусиль було витрачено, щоб підняти маленькі майдани по
областях, щоб банкову завалювали спамом губернатори, щоб мразота зрозуміла: вся
Україна проти них, не вистачить ніяких тітушок нас залякати.

\iusr{Alla Losik}

Не лише Київ врятував, адже одночасно по всій країні почались масові акції
протесту і так, саме це не дало змоги захопити країну. В Єдності Наша Сила!

\iusr{Аліса Гончаренко}
ще не вирвались(

\iusr{Vital Babenko}

Не забувайте, що в Криму, Луганську та Донецьку програли отим, що з акцентом...
І програла саме тодішня влада, яка ділила майбутні портфелі, а не дбала про
цілісність країни...

\iusr{Алла Иванова}
Пані Оксана, дякую!

\iusr{Алла Кипоренко}

Так, ми перемогли тоді. Але не тільки тим, що нас було багато, а ще й тим, що
ми день і ніч стояли на молитві, і мені так болить, коли люди вихваляються, які
ми класні. Ні, нас Господь вів, Архангел Михаїл, Мати Божа. І це в першу чергу.
А коли ми собі присвоїли, \enquote{нашу перемогу}- отут то й почалося.... . Ну і ще- на
Майдані ніхто не стояв за тарифи, за підвищення пенсій. На Майдані кувалися
ДОСТОЇНСТВО, перемога над страхом, і поступово змінювалась свідомість.

\iusr{Алла Верба}

Пишаюсь, що в моєму житті назавжди є наш Майдан. На ньому відкрила для себе
по-справжньому українців, народ, себе...


\iusr{Оксана Лукасевич}
Мороз по шкірі...

\iusr{Марія Зелінська}

Щиро дякую вам пані Оксано за повчальну та відкриту позицію боротьби за
Україну. При перемованах з друзями ви мій кумир. Пройшовши Майдан з першого до
останнього дня, та бачучи смерті молодих патріотів, не можливо бути байдужому
до своєї землі. Герої не вмирають.

\iusr{Олександр Олександренко}

Тут чотири прапора? Але жодного прапора ми не бачили на акції в Казастані. Чому
так, хто це придумав? З прапором Казахстану грабувати магазин не підеш. А без
прапора чого було виходити на мирну акцію взагалі? Нам теж намагалися втюхати
акцію без політиків яку і жорстоко побили.... Чи не є Казахстан репетицією по
захопленню України? Де наш \enquote{бубочка}. Чи не у Путіна на прийомі? Що готує Путін
для нашої опозиції? Пам'ятаємо аварію Чорновола, отруєння Ющенка? Нас чекає
повернення лідера опозиції. Як це все буде? Всі повинні стати разом під нашим
прапором на захист України. Ворог не дрімає. Нажаль. Якось так.

\iusr{Зоя Арова}

Хай здійсниться все про що мріяли люди, виходячі на Майдан. Поки що цого немає,
але сподіваюсь що Україна обов'язково стане найкращою країною для найкращих
людей. З любов'ю і надією на це - моя пісня....

\href{https://www.youtube.com/watch?v=0o7t5FHPYf8}{%
Моя Душа-Украина, youtube, 31.12.2021%
}


\ifcmt
  tab_begin cols=3,no_fig,center
     pic https://i2.paste.pics/4fb77da23b5499eb76705fc49df21d13.png
		 pic https://i2.paste.pics/462206e06c148434c569f6c45c99a736.png
		 pic https://i2.paste.pics/fdee090e90fc2e46c658b462c70a6c55.png
  tab_end
\fi

\ifcmt
  tab_begin cols=2,no_fig,center,resizebox=0.7
	  pic https://i2.paste.pics/433db8765c95f3e3cca50765540987e1.png
		pic https://i2.paste.pics/32ff0ae97ee5ffdc3aad4ed30c6364b4.png
  tab_end
\fi


\ifcmt
  tab_begin cols=3,no_fig,center
     pic https://i2.paste.pics/3a828a14c7ba189225969d2aca09cc52.png
		 pic https://i2.paste.pics/5839cd2c993b413258d7a1f5604d060a.png
		 pic https://i2.paste.pics/508a40c3c37d1ca9757232817a0708ba.png
  tab_end
\fi


\ifcmt
  tab_begin cols=3,no_fig,center
     pic https://i2.paste.pics/722df4441a1c549394bb82c79f71d198.png
		 pic https://i2.paste.pics/bec3b58bcd2d5a29771c947a8d42c785.png
		 pic https://i2.paste.pics/1ca02d69ebdae90ff876de00ca1f9840.png
  tab_end
\fi

\ifcmt
  tab_begin cols=2,no_fig,center,resizebox=0.7
     pic https://i2.paste.pics/e95b97b6ad3cc33bb254c71ec17aae7d.png
		 pic https://i2.paste.pics/c77a2c53bf6feb9e5b61df65f21d885d.png
  tab_end
\fi

\ifcmt
  tab_begin cols=3,no_fig,center
     pic https://i2.paste.pics/74936764aa7f00cda4a3fca27cc7fcb8.png
		 pic https://i2.paste.pics/8dacad9c6724e338854043eec2cbd9e0.png
		 pic https://i2.paste.pics/1fecfd012781d2a3b603d14d6046ed48.png
  tab_end
\fi

\iusr{Дмитрий Безрук}
\textbf{Oksana Zabuzhko (Оксана Забужко)}

А це по суті й без емоцій.

"І тільки ми одні її поки що дали - не тільки порівняно з білорусами чи
казахами (що вже всім очевидно), а на всьому просторі \enquote{Европы от Лиссабона до
Владивостока} (с). Увімкненим інстинктом колективної небезпеки, так. Бережімо
цей інстинкт: нічого ціннішого в нас наразі немає. Нічого безціннішого й
потрібнішого людству на цьому етапі історії...

Так, нам випав у ній передній край. І виборче \enquote{Чорне дзеркало}, розігране в
реалі, і \enquote{шоу-уряди}, виграні \enquote{по телевізору}, - все це теж уперше в \enquote{цифрову
добу} падає саме на наші плечі: ми - фронтир, \enquote{Боже грище} (див. розмову Дарини
з Вадимом у \enquote{Музеї покинутих секретів})... І нема на те ради: така доля, така
місія. А з долею не жартують, тут, як казали наші предки - або пан, або пропав.
Тож хай наші вороги пропадають. Вище носа, \enquote{борисфенці}. ))

І не розслабляймося".

- Цілком згоден! Неймовірно важливі/довгоочікувані рядки!

Додав би ще, що нас очікують не лише-но старі/нові виклики, але й найвищого
ґатунку пригода, - унікальна нагода, що сповнює екзистенційною цінністю цілі
(споріднені) покоління; що також має декілька своїх проявів, - як от кращу
якість життя та зростання чисельності громад/и у подальшій перспективі
(Держави/поколінь).

\ifcmt
  ig https://scontent-frx5-2.xx.fbcdn.net/v/t39.1997-6/s168x128/17629525_1652590988100635_6836916378440564736_n.png?_nc_cat=1&ccb=1-5&_nc_sid=ac3552&_nc_ohc=gg6_qdbGz_YAX9CeHuZ&_nc_ht=scontent-frx5-2.xx&oh=00_AT_Pf-Zu3uRRhgCTuPwwL76I0DQpDnmmFnjNXI3nO2lfUQ&oe=61E3F6C6
  @width 0.1
\fi

\iusr{Oleg Mykhailov}
Дякую, пані Оксано. Дуже точно і влучно.

\iusr{Lyudmila Rozhkova}
Пані Оксана, дякую за ваші прекрасні думки, аналіз і оптимізм.  @igg{fbicon.face.happy.two.hands}  Міцного вам здоров’я і енергії! Разом нас багато  @igg{fbicon.hands.raising} 

\iusr{Tamara Murga}
Бозя - це вона, Божа Матір.

\iusr{Rita Naidko}

Дуже гарні думки, варто пам'ятати, якою ціною дісталася Незалежність і гідно
пошанувати Героїв, а не так, як зараз на Хрещатику. Ні минула, ні тим більше
теперішня влада цього не зробила...

\iusr{Павло Колодій}

Дайте будь ласка відповідь мені, як громадянину з сильною позицією)

У 2013-2014 році ми виходили на сцену Майдану ( безкоштовно за ідею, що не
скажеш про інших відомих агітаторів), щоб підтримати Людей у спільній боротьбі
за КРАЩЕ ЖИТТЯ, за Реформи!!!

Як так сталося, що несправедливість в багатьох сферах в Україні зараз просто
зашкалює, і всі мовчать?!

Вас влаштовує сьогоднішня ситуація?!

\iusr{Maryna Dargil}

Я не була на Майдані, живу в Запорізькій області, але молилася і просила в Бога
помочі і зараз молюся за нашу Україну і за нашу Армію. От Бог нас і почув
тоді, то почуй нас, Господи, і зараз!!!!

\iusr{Арсеній Яворський}
Казахстан поки що з рахунків не списуйте. там ще нічого не скінчилося.

\iusr{Майя Кулида}

ну, дуууже толкова стаття.... про нюанси, які виявляються рішающими, значущими
для подій і результатів.... можу тільки додати, що ота кількість народу, що
вийшла на протести тоді, ота одиниця виміру, що Лесь Подерев'янський визначив
одним словом:))))) - дофуя!.....:)))).... то все стало можливим саме через
культурний символ України - Майдан.... Майдан як зерно самоорганізації... як
центр кола, виру, завдяки якому і утворюється і структурується сила торнадо....
мало сколихнутися і вийти.... треба всередині народу мати пережитий, і не
однораз, момент, коли з,являється структура, точка запуску порядку.... можна
сказати, що нам усім повезло, що ми маємо цей код і знаємо як майданити....

....( з підручника по біології: Життя - це спосіб само-організації.... білкових
тіл:)......).... дякую.... і таки варто мати свій день для цього сим-воління:)

\iusr{Майя Кулида}

наш Майдан - це саме \enquote{порив, заклятий в міт}... і асоціація видає ось
таке:))... \url{https://www.facebook.com/OTVINTAband/videos/10154928151593645/}

\iusr{Людмила Потапушкина}

Так! Пам'ятаю: навіть бабульки чаї розносили, а пізніше подавали булижники з
бруківки, щоб відігнати озброений БЕРКУТ.

\iusr{Юлія Ткаченко}
Дякую, пані Оксано! Мій \textbf{Oleh Tkachenko} боронив Україну в ті часи

\iusr{Ира Смила}

Дякую усім учасникам і захисникам Майдану, що відстояв Україну! \enquote{...бо Україна
дихатиме доти, допоки ми за Неї стоїмо} (Тарас Петриненко )

\url{https://www.facebook.com/watch/?v=2422824327973391}


\iusr{Zoja Rom}

Після подій в Казахстані ще більше переконана, для кого в Сибіру міністр
оборони РФ хотів будувати міста - мільйонники.

Після ВОВ не змогли вивезти всіх українців, бо казали, що вагонів не
вистарчило. Зараз вони такої дурниці не зроблять, щоб залишити народ на своїй
землі, де його корні.


\iusr{Людмила Верещагіна}
Пані Оксано! Мої 62 поруч))

\iusr{Lesia Buzhuk}
Є 21.11. Але саме цей подячний акцент, дійсно, важливо підкреслювати, задля спраді масового розуміння

\iusr{Naum Pauk}

«Мабуть, найбільший урок в історії полягає в тому, що ніхто не засвоїв уроків
історії» - Олдос Хакслі- //“Quizá la más grande lección de la historia es que
nadie aprendió las lecciones de la historia” -Aldous Huxley-... @igg{fbicon.face.weary}  @igg{fbicon.face.downcast.sweat}  @igg{fbicon.face.cold}  @igg{fbicon.face.sad.but.relieved} 

\iusr{Марія Грейда}
Чудова ,повчальна стаття !
Неймовірно !
Дякую !

\iusr{Rafao Sze}
Niezwykły pomnik z mocnym przekazem...

\iusr{Оксана Іщенко}
У кожного народу своя історія, свій досвід.

\iusr{Наталія Талапова}
Дякую за Ваші слова і живу правду! Завжди надихають!

\iusr{Oleksandr Androshchuk}
Дякую Оксані за її слова, повністю згодний і підтримую. Жаль, що в нас ще дуже чисельна 5-та колона.

\iusr{Оксана Євтушевська}

Яке було щастя....а з приходом до влади Зе зникло те щастя, зникло все. Йдемо знову дружно у \enquote{совок}.

\iusr{Раиса Соло}
\textbf{Оксана Євтушевська} на жаль так, згідна (

\iusr{Inna Ivanko}

Про Субтельного згодна на 1000 в1дсотк1в. Я з того покол1ння. Були 2 п1дручники
з 1стор11 Тод1 нов1. КУЛЬЧИСЬКОГО 1 СУБТЕЛЬНОГО.Кульчицький може 1 шанований
1сторик але п1дручники як на мене тогд1шню здавалася нудним. Така соб1
пом1ркована академ1чна 1стор1я,,, осторожная,,, Я б сказала без гомтрих кут1в.
1 з, явиася п1дручники Субтельного. Брала читати взагал1 не в школ1 в
б1бл1отец1 техн1куму по знайомству на пару дн1в читався легко запоем. Зараз маю
ту його книгу Укра1на. 1стор1я врятувала з макулатури. Книга бомбезна нав1ть
зараз п1сля 20 л1ття. Правду кажуть нема пророка в своему краю але нав1ть
Субтельний 1з за океану не вбер1г нас в1д багатьоз помилок але вектор дав
правильный 1 просил в1д Рос11 треба триматись подал1 або бкдувати внутр1шню
Китайську ст1ну раз маемо такий великий кордон з нею

\iusr{Людмила Мельник}
17.01.2022. Ж У Л Я Н И. ЗУСТРІЧАЄМО ПРЕЗИДЕНТА.

\iusr{Володимир Іванович}

Тому хлопчику відірвало руку зовсім поруч на вогняній барикаді на Майдані
лівіше від стели, скоти кинули гранату з намотаними болтами... А був міцної
статури і вольовий.

\iusr{Ол Удайко: літсторінка}

Виграли борисфенці, бо народилась тоді політична нація! І взірець для
наслідування гнобленим народам! Тому й жахають останніх отим Майданом! Тому й
повага світу до України! Слава і подяка Борисфену!

\iusr{Maria Bonkowska}
Дакую пані Оксана! @igg{fbicon.heart.beating} 

\iusr{Степан Лазар}

Ще є, як на мене два фактори. 1. зимова Олімпіада у Сочі закінчилась, здається, 20
лютого. Кцпія була скута нею, ви розумієте... 2. самЕ розташування Майдану
Незалежності, його форма, підходи, рельєф дуже допоміг створенню табору,
організації оборони т. п. Я так думаю.

\iusr{Леся Ковальчук}
Дякую, пані Оксано за статтю! Народ мусить пильно бути на сторожі.


\end{itemize} % }
