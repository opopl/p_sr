%%beginhead 
 
%%file 28_12_2022.fb.fb_group.mariupol.biblioteka.korolenka.1.stor_nki__stor____r
%%parent 28_12_2022
 
%%url https://www.facebook.com/groups/1476321979131170/posts/5689962694433723
 
%%author_id fb_group.mariupol.biblioteka.korolenka,lisogor_viktoria.mariupol
%%date 28_12_2022
 
%%tags mariupol,mariupol.istoria,istoria,rizdvo,novyj_god,tradicii
%%title Сторінки історії - Різдво та Новий рік у Маріуполі початку ХХ століття
 
%%endhead 

\subsection{Сторінки історії - Різдво та Новий рік у Маріуполі початку ХХ століття}
\label{sec:28_12_2022.fb.fb_group.mariupol.biblioteka.korolenka.1.stor_nki__stor____r}
 
\Purl{https://www.facebook.com/groups/1476321979131170/posts/5689962694433723}
\ifcmt
 author_begin
   author_id fb_group.mariupol.biblioteka.korolenka,lisogor_viktoria.mariupol
 author_end
\fi

\textbf{Сторінки історії}

{\Large Різдво та Новий рік у Маріуполі початку ХХ століття}

\begingroup
\bfseries\em
Одними з довгоочікуваних та улюблених свят залишаються Різдво та Новий рік. Як
зараз, так і сто років тому городяни до урочистостей готувалися задовго до
їхнього настання. Необхідно було обрати подарунки рідним та близьким, купити
делікатеси до святкового столу, визначити місце зустрічі Нового року та,
найголовніше, придбати ялинку та ялинкові прикраси.
\endgroup

Наприкінці ХІХ – початку ХХ століть у Російській імперії публічні ялинки почали
влаштовувати у будинках дворянських, офіцерських та купецьких зборів, клубах,
театрах та інших громадських місцях. Маріуполь, будучи містом повіту, також
прагнув не відставати від інших міст у цьому питанні. Не збереглося відомостей,
коли було в Маріуполі встановлено першу публічну ялинку. Відомо лише, що
наприкінці 1897 р. письменник А. Серафимович порушував це питання перед міською
думою.

Вже на початку ХХ століття перед Різдвом у будинках маріупольців повсюдно
з'являються ялинки. Вони продавалися у магазинах на вул. Георгіївської, на розі
вул. Торгової та Італійської. Вартість хвойної красуні була від 20 копійок.

Різдвяну ялинку на той час було прийнято прикрашати яблуками, мандаринами,
сухими фруктами, квітами, стрічками, золоченими горіхами, а також гірляндами,
картонними іграшками, бонбоньєрками, діамантом-пудрою, бенгальськими вогнями,
хлопцями. Наприкінці ХIХ – на початку ХХ століть у продажу з'являються скляні
кулі німецького та російського виробництв. Вінчала різдвяну красуню Віфлеємська
зірка.

Маріупольцям у різних торгових точках пропонувався великий асортимент ялинкових
прикрас. Вони були досить дорогими. Одним із найбільш рекламованих у місцевій
газеті був магазин П. Чебаненка, там ціна набору ялинкових іграшок становила 1
рубль, 1,5 рубля, 2, 3, 5 рублів та дорожче. Від 1 до 25 рублів коштувала
колекція ялинкових прикрас у «Харківській кондитерській та булочній». Там же
продавалися ялинки, «святкові сюрпризи», іграшки нових винаходів та маскарадні
маски.

Напередодні різдвяних та новорічних свят торговельні заклади незалежно від
свого профілю прагнули нагадати про себе. Вони навмисно у грудні поповнювали
свій асортимент. Вулиці Катерининська та Торгова напередодні свят були
найвідвідуванішими городянами.

Наприкінці грудня газета «Маріупольське життя» буквально рясніла рекламою, вона
розміщувала приватні оголошення про вступ до магазинів розширеного асортименту
всіляких товарів.

\textbf{Що дарували та що отримували в подарунок маріупольці?}

Великий асортимент парфумів було завезено напередодні свят аптекарський магазин
І. Б. Дворкінда. Духи \enquote{Стікс}, \enquote{Амбре}, \enquote{Антик},
одеколони \enquote{Туга}, \enquote{Наполеон}, \enquote{Каприз},
\enquote{Де-Амур} були досить витонченим подарунком. Флакон, що сподобався, за
окрему плату упаковували в спеціальну подарункову коробку. Там же можна було
придбати чарівні ліхтарі, фотоапарати, термоси, сифони Прага, електричні
ліхтарики. У торгових закладах міста маріупольцям пропонувалося придбати такі
подарунки: дитячі іграшки, дитячі освітні ігри, гравюри, картини, книги,
розкішні дрібниці для домашньої прикраси. Актуальним подарунком був календар
наступного року. Відділення дитячих іграшок було відкрито спеціально
напередодні свят. Його господар, пан Гольденштейн, запевняв городян у
найпомірніших цінах. Англійська крамниця швейних машин пропонувала до Різдва
модну новинку — поромийку «Іона», яка прала білизну пором.

Напередодні свят пані та кавалери прагнули поповнити свій гардероб. З особливою
ретельністю вони обирали святкові туалети. Свої послуги у підборі костюма
пропонував модно-галантерейний магазин Пальмана. Харківський магазин білизни та
жіночого вбрання до передріздвяного розпродажу підготував «небувалий вибір»
готового жіночого вбрання – суконь, блузок, спідниць та ін. приладдя жіночого
туалету, а також дитяче плаття.

До різдвяного та новорічного святкового столу до торговельних закладів міста
завозилися різноманітні м'ясні та кондитерські делікатеси та вина. Напередодні
їхні власники поповнювали асортимент різними кулінарними та кондитерськими
вишукуваннями. До свят «Харківська кондитерська та булочна», магазини М.
Урбанського та І. Дворкінда отримували товари кращих гастрономічних та
кондитерських фірм Абрикосова, Ейнема, Сіу, Конрада, Бормана, Ландріна,
Кромської, до уваги покупців також пропонувався великий вибір свіжих цукерок,
швів ін Господар одного з них, пан М.Ф. Урбанський проводив відпустку товарів
під своїм особистим наглядом. У ковбасно-гастрономічний магазин до свят були
завезені копчені житомирські та тамбовські стегенця, різне фаршування – качок,
пулярок, поросят та паштети.

\ii{28_12_2022.fb.fb_group.mariupol.biblioteka.korolenka.1.stor_nki__stor____r.pic.1}

«Вина кращих російських та закордонних марок», а також великий вибір горілки,
коньяків, лікерів продавали у гастрономічній лавці М.Ф. Урбанського та у
винно-гастрономічному магазині «У Абуладзе». Там же пропонувалися фрукти та
цукерки. За помірнішими цінами вина до свят пропонувалися у винних льохах.

Свято Різдва у всьому світі прийнято відзначати у сімейному колі. Не були
винятком і маріупольці. А ось новорічну ніч городянам пропонували урочисто
провести у громадських місцях. Наприклад, у 1912 р. у Комерційних зборах
відбулися сімейно-танцювальний вечір та зустріч Нового року. Вхід був за
записом, і оплата становила 1 р. 10 копійок. Кожен гість мав право пройти лише
з однією жінкою. Члени зборів та студенти проходили безкоштовно. Подібні умови
відвідування сімейного вечора були в Маріупольських громадських зборах, що
знаходилися в будинку Д. Хараджаєва. Заможні маріупольці могли дозволити собі
урочисто зустріти Новий рік у Великій концертній залі готелю «Континенталь» на
вул. Харлампіївській. Вхід на вечір був виключно за записом. Для публіки грав
струнний оркестр під керуванням Г. Теплова. На вечорі дамам безкоштовно
роздавали бутоньєрки, а чоловікам – сигари. Святкова вечеря на 1 персону
коштувала 2 р. 50 коп. Для спостереження за його приготуванням із Харкова
спеціально було виписано шеф-кухаря Т.А. Рівненський. Слід зазначити, що на той
час особливою популярністю користувалися страви французької кухні. На новорічну
вечерю в «Континенталь» подавали консомі зі спаржі, дияблі (гострі запечені
грінки з скибочок пшеничного хліба, змащених сумішшю тертого сиру, томату,
перцю та яєць), севрюгу парову в шампанському вині, смажених молодих мулар
ендівій та соління. На десерт подавали парфі зі свіжої полуниці та фрукти. З
міцних напоїв пропонувалася горілка із закускою.

На початку ХХ століття маріупольці обмінювалися вітаннями зі своїми близькими,
знайомими. Не забували городяни вітати тих, хто особливо потребував уваги або
опинився за межею бідності. Суспільство допомоги бідним міста Маріуполя (під
головуванням Д. Хараджаєва) займалося збором коштів для допомоги нужденним. У
його правління міг прийти кожен городянин і залишити пожертву.

У зимовий час у Маріуполі особливо популярні різні зимові розваги. Наприклад, у
грудні 1910 р. газета «Маріупольське життя» повідомляла про роботу ковзанки на
озері Домаха біля біржі. Тут можна було взяти ковзани напрокат. Плата для
дорослих складала 15 копійок, для дітей – 10 копійок. Ковзанка була захищена
від вітру високим парканом.

Так готувалися та відзначали різдвяні та новорічні свята маріупольці ще сто
років тому, на початку ХХ століття.

У 1914 р. розпочалася Перша світова війна. Вона не торкнулася безпосередньо
нашого міста, але відклала певний відбиток на все його життя. Багато
маріупольців воювали на фронті, саме місто знаходилося в тилу. Сюди стікалися
біженці з прифронтових територій, приїжджали на відпочинок солдати та інваліди
з передової. На допомогу армії міська громадськість періодично влаштовували
благодійні вечори, кошти яких йшли на допомогу бійцям.

В1915 р. вже не рясніє «Маріупольське життя» рекламними оголошеннями про
великий асортимент різдвяних подарунків, вона сповнена зведення з фронту. Війна
вимагала нових і нових вкладень. Тепер різдвяні та новорічні свята були не такі
веселі та безтурботні, вони служили додатковим приводом для збору коштів. Так,
наприклад, у грудні 1915 р. Маріупольський міський комітет Всеросійського союзу
міст закликав надіслати подарунки фронтовикам до свята Різдва Христового.
Внести посильний внесок городяни могли грошима або, принісши подарунок в
окремому пакеті, куди дарувальнику пропонувалося вкласти листівку для
відповіді. Що могло втішити бійця на фронті напередодні свят? Чоботи, панчохи,
онучі, тепла білизна, рукавички, жилети, шарфи, тютюн, цигарки, сірники, мило,
свічки, цукор, чай, поштовий папір, олівці, конверти та ін. Такий перелік
припустимих речей пропонував комітет. Подарунки приймалися до 10 грудня (адже
вони мали бути доставлені до Різдва!) в конторі Д. Хараджаєва на вулиці
Таганрозької, в Міській управі і в конторі В. Вальтона на вул. Георгіївській.

Для надання допомоги російським військовополоненим, які перебувають у ворожих
країнах, Жіночий гурток Маріуполя провів благодійну акцію. Розпорядчий комітет
гуртка під головуванням Д. Хараджаєва пропонував зустріти 31 грудня 1915 р.
новорічну ніч у готелі «Континенталь» за вечерею, яка коштувала 6 рублів з
персони.

У день Різдва, 25 грудня, у місті та на заводах було відкрито кінематографи. З
дозволу Катеринославського губернатора та Катеринославської духовної
консисторії половину всього валового збору кінематографи перерахували на
користь рухливого шпиталю від населення всієї Катеринославської губернії.

\textbf{Матеріали краєзнавчого музею м. Маріуполя.}
