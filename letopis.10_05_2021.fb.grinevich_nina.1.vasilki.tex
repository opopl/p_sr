% vim: keymap=russian-jcukenwin
%%beginhead 
 
%%file 10_05_2021.fb.grinevich_nina.1.vasilki
%%parent 10_05_2021
 
%%url https://www.facebook.com/permalink.php?story_fbid=2879611012303923&id=100007651574181
 
%%author 
%%author_id 
%%author_url 
 
%%tags 
%%title 
 
%%endhead 
\subsection{Васильки}
\Purl{https://www.facebook.com/permalink.php?story_fbid=2879611012303923&id=100007651574181}

\ifcmt
  pic https://scontent-frt3-2.xx.fbcdn.net/v/t1.6435-9/184029482_2879610805637277_4964156679088977641_n.jpg?_nc_cat=110&ccb=1-3&_nc_sid=8bfeb9&_nc_ohc=Ea-pX6qSNekAX9P92Zv&_nc_ht=scontent-frt3-2.xx&oh=9b1a4ebbf9a3b9b0f9e247794e039df5&oe=60C12AAB
\fi

ВАСИЛЬКИ.
🔹Венок  из васильков был найден в саркофаге гробницы Тутанхамона, во время раскопок . 
🔹Научное название васильку было дано лишь в XVIII веке К. Линнеем. 
В древности греки василек называли «центауре», в честь мифологического существа Центавра, который после битвы с Гераклом лечил свои раны соком василька.
🔹У немцев василек пользуется популярностью и любовью. Василек был любимым цветком королевы Луизы и ее сына императора Вильгельма I. 
Во время наполеоновских войн, королева с детьми укрывалась в Кенигсберге. 
Однажды гуляя с детьми, ей предложили купить корзину васильков. 
Королева Луиза вместе с детьми перебирала и любовалась цветками, плела венки. 
Когда венок надели дочери Шарлоте на голову, королева почувствовала, что с ее семьей будет все хорошо. 
Прошло время, и сын стал императором Вильгельмом I, а дочь Шарлота стала впоследствии русской императрицей Александрой Федоровной, матерью императора Александра II. 
🔹В Бельгии василек являлся эмблемой свободы. 
Горнорабочие во время стачек прикалывали василек к своим одеждам.
🔹С давних времен он считается символом преданности и нежности. 
В 1968 году василек был объявлен национальным цветком Эстонии.
🔹Валерий Чкалов в 1937 году, когда совершил полет через Северный полюс, подарил букет васильков первым встретившим его американцам.
🔹Существует замечательная басня Крылова «Василек», в которой василек сыграл некоторую, хотя, быть может, и косвенную, но все-таки историческую роль. 
Басня эта посвящена императрице Марии Феодоровне и начинается так:
«В глуши расцветший василек
Вдруг захирел, завял до половины
И, голову склоня на стебелек,
Уныло ждал кончины…»
Когда Иван Андреевич Крылов сильно заболел, ему императрица Александра Федоровна подарила букет из васильков. 
После выздоровления Иван Андреевич написал басню о бедном васильке, а букет, подаренный императрицей, он долго хранил и любовался им. Крылов так любил эти цветы, что в завещании просил положить васильки ему в гроб.
🔹Василек является одним из лучших цветков для плетения венков. 
В засушенном виде он имеет сильный пряный запах, иногда его использовали как благовонное окуривающие средство, и плели обрядовые венки из василька.
🔹Василек использовался в обрядах хлебопашества и вызывания созидательных сил матери-природы. 
У славян связано два праздника: «пошел колос на ниву»- отмечался при появлении колосьев на ниве и «именинный сноп»- проводился в конце лета перед уборкой урожая.
🔹Образ василька — любимый элемент декора ткачих и вышивальщиц.
🔹Василек хоть и очень красивый, он любит расти во ржи, но является сорняком. 
Было доказано наукой: что если к 100 семенам ржи добавить одно семечко василька – рожь растет лучше, но большее количество семян угнетают рожь.
🔹В XVI веки он стал пользоваться такой популярностью, что предприимчивые садоводы стали разводить его у себя в садах.
🔹Василькам посвятили живописные работы многие художники, такие как Венецианов А.Г, Маковский К. Е, Грабарь И.Э, Левитан И. И. и др. 
Достаточно вспомнить полотно Игоря Грабаря «Васильки», где на фоне жаркого полдня две подруги вспоминают о своей юности перед огромной охапкой васильков.
🔹У василька есть своя тайна — распространение его семян. 
Они ползают...
На верхушке гладкой, очень блестящей семянки василька, напоминающей по форме ржаное зернышко, есть небольшой хохолок из белых волосков. 
Хохолок василька — это основной орган передвижения семянок, с его помощью они и «ползают».
Намокая, он сокращается, а высыхая, удлиняется. 
Волоски же хохолка имеют направленные в одну сторону зазубрины, которыми они упираются в неровности почвы. При сокращении или удлинении васильков семянка движется.
🔹Васильки кроме своей декоративности обладают и другими свойствами. 
Они хорошие медоносы, а лекарственные свойства растения используются уже многие века. 
У василька голубого цветы используют как противомикробное средство, при заболевании почек, печени, как мочегонное средство. 
При обработке паром цветков, васильковая вода используется при глазных заболеваниях, при конъюнктивите.
🔹Говорят, что василек обладает и волшебной силой, так как его опекает сама Венера. 
Букет васильков, лучший подарок любимого он обеспечивает взаимность.
🔹В Европе на языке цветов василек означает веселость, верность, доверие.
🔹Люди так привыкли к василькам, этим голубым цветам, что даже когда речь идет о каком- то предмете голубовато – синего цвета, описывая его, говорят что предмет василькового цвета. 
Но васильки могут иметь и другие оттенки.
🔹Сейчас василек синий используют для получения лазурно-синего красящего вещество, которое безвредное и служит для подкрашивания и украшение парфюмерных препаратов, пряностей.
Окраска васильков очень стойкая, почти не выцветает на солнце.
В старину для окраски шерстяных тканей и изготовление чернил использовали цветки василька. 
Из васильков ранее делали уксус.
Вот какое удивительное и необыкновенное растение!!!
Пост - Земфира Курбанова
