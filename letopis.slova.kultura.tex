% vim: keymap=russian-jcukenwin
%%beginhead 
 
%%file slova.kultura
%%parent slova
 
%%url 
 
%%author 
%%author_id 
%%author_url 
 
%%tags 
%%title 
 
%%endhead 
\chapter{Культура}
\label{sec:slova.kultura}

%%%cit
%%%cit_pic
%%%cit_text
Оскільки було силове впровадження росмовної \emph{субкультуру}, має бути силове витіснення її з України
%%%cit_comment
Ярослав Трінчук
%%%cit_title
\citTitle{...ми ще на стадії формування українського музичного ринку - відвойовуємо його у Росії}, 
Про Мову, facebook, 18.06.2021
%%%endcit

%%%cit
%%%cit_head
%%%cit_pic
\ifcmt
  pic https://img.vz.ru/upimg/m11/m1104023.jpg
\fi
%%%cit_text
Когда Путин в послании Федеральному собранию сказал фразу о Киплинге, которая
сразу стала мемом, он, с одной стороны, отсылал, понятно, к нравам «Книги
джунглей», прозрачно намекая на ситуацию в современной международной политике.
Но, с другой стороны (и этот более глубокий месседж также, думаю, был считан
всеми, кому он предназначался), реагировал на ту беспрецедентную атаку на
традиционную христианскую европейскую \emph{культуру}, которую мы наблюдали весь 2020
год, и которая сегодня продолжает лавинообразно нарастать
%%%cit_comment
%%%cit_title
\citTitle{Россия должна жить по заветам Киплинга}, 
Владимир Можегов, vz.ru, 18.06.2021
%%%endcit
