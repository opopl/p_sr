% vim: keymap=russian-jcukenwin
%%beginhead 
 
%%file 18_10_2021.yz.nealternativnaja_istoria.1.kiev_mti_gradom_ruskim
%%parent 18_10_2021
 
%%url https://zen.yandex.ru/media/sibguide/govoril-li-kniaz-oleg-da-budet-kiev-mat-gorodov-russkih-chitaem-lavrentevskuiu-letopis-6167bdf348ee6d4c0a1fe4f6
 
%%author_id yz.nealternativnaja_istoria
%%date 
 
%%tags 
%%title 
 
%%endhead 
\subsection{Говорил ли князь Олег \enquote{Да будет Киев мать городов русских}? - читаем Лаврентьевскую летопись}
\label{sec:18_10_2021.yz.nealternativnaja_istoria.1.kiev_mti_gradom_ruskim}

\Purl{https://zen.yandex.ru/media/sibguide/govoril-li-kniaz-oleg-da-budet-kiev-mat-gorodov-russkih-chitaem-lavrentevskuiu-letopis-6167bdf348ee6d4c0a1fe4f6}

\ifcmt
 author_begin
   author_id yz.nealternativnaja_istoria
 author_end
\fi

После статьи \href{https://zen.yandex.ru/media/sibguide/kiev--mat-gorodov-russkih-razmyshleniia-o-drevnei-sovsem-nekievskoi-rusi-615d307ed132a05135984517}{"Киев – мать городов русских? - размышления о Древней совсем
НеКиевской Руси"} мне многие пишут в комментариях, дескать, Олег, как приводит
его слова Нестор в \enquote{Повести временных лет} (ПВЛ), говорил про Киев не \enquote{Да будет
это мать городам русским}, а \enquote{Да будет это город русский}.

Даже приводятся долгие умозаключения, что в оригинальной фразе Олега \enquote{Се буди
мти градомъ руским} слово \enquote{мти} употреблено неправильно. Дескать, никак оно не
стыкуется с формой \enquote{градом}. Что если бы \enquote{мти} обозначало \enquote{мать}, то фраза
должна была бы быть построена \enquote{Се буди мти градовъ руских}. Раз так, что слово
\enquote{мти} либо лишнее, либо обозначает форму \enquote{иметь}. А так как до этого русским
Киев не был, это дает смысл фразе \enquote{Будем иметь сей город русским}.

Вот честно, надоело поправлять каждого, кто прочитал чью-то глупость и с
готовностью начал ее тиражировать.

\ifcmt
tab_begin cols=2,no_fig,center

  ig https://avatars.mds.yandex.net/get-zen_doc/5321106/pub_6167bdf348ee6d4c0a1fe4f6_6167be0db54bc66203587eb8/scale_1200

	ig https://avatars.mds.yandex.net/get-zen_doc/2468786/pub_6167bdf348ee6d4c0a1fe4f6_6167be2f3b10c81d582a64f8/scale_1200

tab_end
  %@wrap \parpic[r]
  %@wrap \InsertBoxR{0}
	%@caption фрагмент \enquote{Лаврентьевской летописи}
\fi

Давайте читать первоисточники. К сожалению, оригинальный текст \enquote{Повести
временных лет} мне найти не удалось, зато удалось найти оригинальный текст
\enquote{Лаврентьевской летописи}, которая является фактическим списком ПВЛ.

Но вот дальше, на листе 17 в повествовании о крещении Ольги в Царьграде в году
6463 (955) сказано "бѣ же речно имя си во крщньи Олена якоже и древняя црца мти
Великаго Костантина". В этом фрагменте речь идет в наречении Ольги в крещении
Еленой. И совершенно очевидно, что в сравнении ее с древней царицей, матерью
Константина, слово "мти" употреблено именно в смысле матери, а не в какой-то
форме глагола "иметь".

Читаем дальше. Говоря о рождении Святополка, автор пишет: "у грѣховънаго бо
корени золъ плодъ бываетъ понеже бѣ была мти его черницею а второе Володимеръ
залежею не по браку прелюбодѣи". И опять, совершенно очевидно, что слово "мти"
употребляется именно в значении "мать".

Таких примеров употребления слово "мти" в значении матери в "Повести временных
лет" просто масса. Я бы даже сказал, нет ни одного примера, где бы слово "мти"
употреблялось в ином значении, отличном от "мать". Отсюда напрашивается
совершенно логичный вывод, что во времена Нестора слово "мти" обозначало
"мать".

Что же касается неправильного построения фразы "мти градомъ руским", то оно
неправильно с точки зрения современной грамматики. Во всяком случае, обвинить
Нестора и последующих переписчиков ПВЛ (Лаврентьевская, Ипатьевская летописи) в
безграмотности у нас нет никаких оснований.

Вероятно, причина кроется в не совсем точном переводе. Более точным было бы не
"мать городов русских", а "мать городам русским". Проблема же замены в этом
случае одной единственной буквы "а" на "о" вполне объяснима окающим диалектом
юго-восточных славян, что наблюдается и в наши дни,

Таким образом, мы вынуждены признать, что в году 6390 (882), когда Олег сел на
Киевский престол, смысл его фразы "Се буди мти градомъ руским" заключается
именно, что будет Киев не просто городом русским, но матерью городов русских.

Любителям же высасывать сенсации из пальца, а также любителям повторять чужие
глупости посоветую – читайте сами первоисточники, а не комментарии к ним.
Читайте весь текст, а не делайте выводы на основании одной единственной
вырванной из контекста фразы.

\ii{18_10_2021.yz.nealternativnaja_istoria.1.kiev_mti_gradom_ruskim.cmt}
