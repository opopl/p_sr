% vim: keymap=russian-jcukenwin
%%beginhead 
 
%%file 05_10_2021.fb.kulik_sergej.1.ukr_nacionalizm_evrei_1918_pogrom
%%parent 05_10_2021
 
%%url https://www.facebook.com/permalink.php?story_fbid=4410598019016035&id=100001976402092
 
%%author_id kulik_sergej
%%date 
 
%%tags 
%%title ЗАДОЛГО ДО БАБЬЕГО ЯРА
 
%%endhead 
 
\subsection{ЗАДОЛГО ДО БАБЬЕГО ЯРА}
\label{sec:05_10_2021.fb.kulik_sergej.1.ukr_nacionalizm_evrei_1918_pogrom}
 
\Purl{https://www.facebook.com/permalink.php?story_fbid=4410598019016035&id=100001976402092}
\ifcmt
 author_begin
   author_id kulik_sergej
 author_end
\fi

АДОЛГО ДО БАБЬЕГО ЯРА.
С 1 по 20 марта 1918 г. украинские солдаты грабили, истязали и расстреливали евреев в Киеве, главным образом на территории Михайловского монастыря. Вмешательство городской думы, пославшей делегацию в войсковые части и обратившейся непосредственно к С. Петлюре, не привело к прекращению эксцессов. Ознакомившись с настроениями украинских солдат, член думской делегации Л. Чикаленко сказал: «Они утопят украинскую свободу в еврейской крови». Лишь председателю Центральной рады М. Грушевскому удалось (и то не сразу) прекратить погром; ни один из его участников не был наказан.
В мемуарах киевлянина и очевидца происходившего Сумского "Одиннадцать переворотов. Гражданская война в Киеве" рассказывается:
"Гайдамаки хватали на улицах евреев, уводили их с собой в
Михайловский монастырь и там убивали. Газеты пестрели траурными объявлениями, кончавшимися стереотипной фразой: <зверски убит в Михайловском монастыре>. В центре города производилась расправа, и жители боялись близко подходить к монастырю. Напротив, в Софийском соборе, служились торжественные молебны об
освобождении Киева, а в ворота монастыря таскали за бороды
кричавших в смертельном страхе евреев. В течение первой недели насчитывалось несколько десятков убитых. Эта деятельность сброда украинских янычар, нисколько не страшных для большевиков, но очень страшных для населения, особенно для еврейского, становилась неприятной и всячески мешающей восстановлению порядка.
Немцы это почувствовали и решительно положили этому конец.
Они немедленно разоружили разгулявшихся янычар, часть отправили на фронт, а нескольких человек, как говорят, даже расстреляли.
`
Но погромная идеология была не только продуктом разложения в гражданской войне. Надо прямо сказать, что в последовавшей затем резне повинны в полной мере и несут за это полную моральную, если не юридическую ответственность, руководители и вожди <украинского освободительного движения>, люди, все время уверяющие, что они - демократы и социалисты. Они сделали резню орудием своего <освобождения>, и один эпизод, случившийся как
раз в эти дни, может служить тому примером.
`
Выходила в Киеве газета <Нова рада>, орган партии украинских
социалистов-федералистов, который редактировали заслуженнейший украинский ученый И писатель С. А. Ефремов и будущий министр, один из вождей партии социалистов-федералистов, Андрей Никовский. На другой или третий день после прихода немцев г. Никовский напечатал в <Новой раде> при ближайшем участии гуманнейшего С. А. Ефремова статью, в которой говорилось: мы должны
вдуматься в психологию обыкновенного гайдамака. И дальше в
первом лице, хотя и без ковычек, но так, чтобы при случае можно было сказать - так рассуждает гайдамак, а я без комментариев воспроизвожу его рассуждения, - приводились примерно такие
мысли:
`
- Я иду в советское учреждение и там встречаю молодых
людей с длинными носами, которые нахально разговаривают с
православным населением. Над населением, его культурой, церквами измываются евреи. Во время боев жители-евреи подают сигналы Красной армии и занимаются шпионством. А когда наши части вынуждены были отступать, то евреи из окон и дверей стреляли
им в спину.
`
Когда пишущий эти строки в одной из киевских газет обозвал после этой статьи г. Никовского Андреем Чебыряком, то
г. Никовский очень обиделся, но эта идеология предопределила в будущем то, что бледнеет по сравнению с подвигами Железняка и
Гонты.
`
109
`
И разве не характерно, что свои исторические воспоминания
и идеалы украинская интеллигенция связала с гайдамаками, оселедцами, Гонтой, уманской резней? И если гуманнейший С А. Ефремов допускал, чтобы в его газете печатались такие статьи, то что было спросить с прочих политиков Украины"
