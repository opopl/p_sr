%%beginhead 
 
%%file 10_02_2023.fb.fb_group.mariupol.pre_war.3.dom_nilsena_na_ul__s
%%parent 10_02_2023
 
%%url https://www.facebook.com/groups/1233789547361300/posts/1407384730001780
 
%%author_id fb_group.mariupol.pre_war,elena_mariupolskaja
%%date 10_02_2023
 
%%tags mariupol,mariupol.pre_war
%%title Дом Нильсена на ул. Семенишина
 
%%endhead 

\subsection{Дом Нильсена на ул. Семенишина}
\label{sec:10_02_2023.fb.fb_group.mariupol.pre_war.3.dom_nilsena_na_ul__s}
 
\Purl{https://www.facebook.com/groups/1233789547361300/posts/1407384730001780}
\ifcmt
 author_begin
   author_id fb_group.mariupol.pre_war,elena_mariupolskaja
 author_end
\fi

\#історіяДовоєногоМаріуполя

Дом Нильсена на ул.Семенишина

Этот дом имеет свою мистическую тайну и легенду. Одна из них объясняет второе
название \enquote{Дом плачущих нимф}. Легенда гласит, что архитектор Нильсен посвятил
свое строение умершей дочери. На доме имеются \enquote{украшения} в виде венков. Их 16
- по числу лет умершей девочки.Она умерла от тифа и это стало трагедией для
архитектора. Он разместил на доме  женский профиль. Во время дождя по нему
стекает вода  и создается впечатление, что нимфа льет слезы о умершей..

Есть предание, что по ночам в плохую погоду можно услышать, как из дома
доносится плач.

%\ii{10_02_2023.fb.fb_group.mariupol.pre_war.3.dom_nilsena_na_ul__s.cmt}
