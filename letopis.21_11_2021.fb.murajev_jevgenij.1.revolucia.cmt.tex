% vim: keymap=russian-jcukenwin
%%beginhead 
 
%%file 21_11_2021.fb.murajev_jevgenij.1.revolucia.cmt
%%parent 21_11_2021.fb.murajev_jevgenij.1.revolucia
 
%%url 
 
%%author_id 
%%date 
 
%%tags 
%%title 
 
%%endhead 
\subsubsection{Коментарі}

\begin{itemize} % {
\iusr{Ирина Гончар}
Это было очень интересно, спасибо!

\iusr{Светлана Корзинова}
КАК МУРАЕВ ПРАВ!!!

\iusr{Андрей Лапшин}

Стрелков - это слишком просто и неверно. Это эпизод. Фрагмент. Причем странный.
События в Донецке приобрели характер противостояния, причем невооруженного,
значительно раньше.

\iusr{Валерий Староверов}
\textbf{Андрей Лапшин} Стрелков это протест против майдана. Против переворота. Против убийств после майдана.

\iusr{Денис Харчиков}

Странно. все забыли 14 год. ведь верили, надеялись были смелыми. а, что сейчас с
нами стало? почему лапки опустили? кому отдали веру, надежду, смелость? что с нами
сука произошло такого, что мы стали просто потерпевшими?

\iusr{Алексей Кузьмич}
\textbf{Денис Харчиков} Не потерпевшими. а терпилами

\iusr{Oleg Koval}
Немогу понять, у него ботокс?! А то лоб вообще не шевелится

\begin{itemize} % {
\iusr{Анна Акулова}
\textbf{Oleg Koval} шутите?

\iusr{Мария Цернова}
\textbf{Oleg Koval} Это все что вас интересует?

\iusr{Раиса Френкель}
\textbf{Oleg Koval}  @igg{fbicon.face.tears.of.joy}{repeat=3} 

\iusr{Раиса Френкель}
\textbf{Oleg Koval} очень настораживает когда подобным интересуются мальчики.
\end{itemize} % }

\iusr{Виталий Пустовойт}
Голобородько тоже все правильно говорил.

\begin{itemize} % {
\iusr{Pankratova Nadiy}
\textbf{Виталий Пустовойт} Не сравнивайте , умственный потенциал разный.

\iusr{Виталий Пустовойт}
\textbf{Pankratova Nadiy} хозяева одинаковые

\iusr{Pankratova Nadiy}
\textbf{Виталий Пустовойт} Ошибаетесь !

\iusr{Николай Верещак}
\textbf{Виталий Пустовойт} мужик не тупи. Какой голобородько?

\iusr{Ирина Макаренко}
\textbf{Виталий Пустовойт} про Голобородько пора забыть!

\iusr{Inna Volkova}
\textbf{Виталий Пустовойт} не позорьтесь. Как можно было верить актёру, сказочному персонажу? )
\end{itemize} % }

\iusr{Владимир Вахрушев}
правильно писать - на Украине

\begin{itemize} % {
\iusr{Oleg Koval}
\textbf{Владимир Вахрушев} как на Руси?

\iusr{Marina Glyva}
\textbf{Владимир Вахрушев} Правильны оба варианта написания: в Украине и на Украине.

\iusr{Владимир Вахрушев}
\textbf{Marina Glyva} кто сказал? @igg{fbicon.smile} 

\iusr{Marina Glyva}
\textbf{Владимир Вахрушев} Орфографический справочник откройте.

\iusr{Владимир Вахрушев}
\textbf{Marina Glyva} открыл, всюду - на Украине.

\href{https://orfogrammka.ru/блог/интересное/на-украину-или-в-украину/}{%
На Украину или в Украину?, orfogrammka.ru, 03.09.2020%
}

\iusr{Marina Glyva}
\textbf{Владимир Вахрушев} 

Даже в этой ссылке, которую вы добавили, написано, что вариант "в Украине"
также употребляется. Слово Украина - это название страны, а не окраина. Предлог
"на" имеет значение "на поверхности", например: на земле, на небе; а предлог
"в" - нахождение чего-либо внутри или движение вовнутрь, так что даже по
правилам русского языка логичнее употреблять: в Украине (где? В Украине) Но на
Украине звучит красиво, я чаще употребляю этот вариант. Оба варианта допустимы
и не считаются ошибкой. Написание "на Украине" считается исторической
традицией, но логики в нем нет, поскольку это написание противоречит значениям
предлогов.

\iusr{Владимир Вахрушев}
\textbf{Marina Glyva} я с вами полностью согласен. Так и решим @igg{fbicon.smile} 

\end{itemize} % }

\iusr{Валентина Дрогваль}

Правы Мураев. За злодеяния надо ответить, а то судят Я Януковича, а Турчинов,
парубий до сих пор не ответили по закону. А еще Порошенко. Только умеют грабить
и делить. Яценюк разворовали деньги, выделеные на строительство стены. Ни
стены, ни денег.


\iusr{Angelika Bilash}
Такого прнзидента надо в Украину

\iusr{Grigori Rybak}

А ведь девушка учавстующая в беседе очень похоже на ту, которая 02.05.2014
истерично радовалась в прямом эфире, когда сжигали людей в Одессе!!!

\begin{itemize} % {
\iusr{Вячеслав Беленький}
причем весьма обидчивая, за одно корректное упоминание сюжета в бан отправляет)
\end{itemize} % }

\iusr{Ирина Роман Дрогобецька}
Як він прав підтримую молодець

\iusr{Grigore Zaporojan}
Уважуха

\iusr{Зинаида Зинаида}
Всегда смотрю с удовольствием!

\iusr{Светлана Панкова}

Зато: Тлявов, Ноздровская, жертвы Зайцевой, жертвы полицейских в Кагарлике,
ребенок, убитый бабкой дилером в Черкассах, авария Скрябина, подрыв Шеремета,
реформа Супрун


\iusr{А. Л. И.}

Красавчик. Ничего хорошего в силовых массовых протестах нет. И я не понимаю
когда люди под громкие слова идут за кем то.

Если что то менять, то не людей, а методы, законы.

Вот если бы мне сказали выйти на протест для того чтобы в ту же дию внесли
систему медицинского учета, чтобы там была полная история болезни, чтобы при
попадание в больницу человек видел учет медицинских средств которые на него
расходывали. Чтобы знал что ему предоставляют бесплатно, а за что деньги надо
заплатить.

Чтобы там же был раздел официальной благотворительности, чтобы человек мог
пожертвовать больнице официально, а не наличными в карман.

Вот за это я бы вышел на протест и таких можно миллион примеров написать и про
школы и про гос службы, где можно легко все изменить избежав коррупции.

А вот эти все громкие фразы выберите меня и я буду побеждать коррупцию. За
такое здравомысляший человек точно не пойдет.

\iusr{Larisa Zrazhevskaya}
Эфир очень интересный, просто СУПЕР!

\ifcmt
  ig https://scontent-frx5-2.xx.fbcdn.net/v/t39.1997-6/s168x128/93027172_222645632401274_7176243611145601024_n.png?_nc_cat=1&ccb=1-5&_nc_sid=ac3552&_nc_ohc=2QEbDVmY9SoAX-OTnwb&_nc_ht=scontent-frx5-2.xx&oh=1d6259aad431223468adbaa9492df50a&oe=61A28AF0
  @width 0.1
\fi

\iusr{Konstantin Fomin}
Да, все верно подмечено

\iusr{Татьяна Барашкова}
Умничка!!!

\iusr{Дмитрий Пермьяков}
Євгене, коли будемо щось змінювати в країні????? Час переходити від слів до діла  @igg{fbicon.thinking.face} 

\iusr{Раиса Френкель}

\ifcmt
  ig https://scontent-frx5-2.xx.fbcdn.net/v/t39.1997-6/s168x128/17633073_1652591054767295_6333333619058147328_n.png?_nc_cat=1&ccb=1-5&_nc_sid=ac3552&_nc_ohc=yFVaSb75GMEAX8ysCVN&_nc_ht=scontent-frx5-2.xx&oh=c6616fc4b0554be24b9b0bf03d9f46ff&oe=61A392BC
  @width 0.1
\fi

\iusr{Roman Pikalov}
Поддерживаю @igg{fbicon.thumb.up.yellow} 

\iusr{Daria Petrovets}
Репост )

\iusr{Daria Petrovets}
10000000\%

\iusr{Арсен Терян}
300\%

\iusr{Анди Ас}

Господин Мураев Вы уж определитесь, захват Донецка, а может все же оборона, а
то как то не вписывается во фразу захват и сожение людей в Одессе и расстрел на
майдане. С кем Вы господин Мураев?

\iusr{Иван Демидов}

Революцию делали Пашинский, Парубий, любовница Парубия телка Черновол, Лидер
Свободыжирный забылего фамилию.., Боксер и его шайка Удар. Ну и Порох. А
пользуется всем этим СЛУГИ НАРОДА!

\iusr{Евгений Бакуменко}
Венедиктова давно должна нам ответить на эти вопросы

\iusr{Евгений Попов}

Пример со Стрелковым, совсем не удачный и не в тему. Стрелков и ему подобные
люди будут скоро освобождать Украину. А у вас Евгений ничего не получится. вы
хоть и умный человек, но... Методы у вас слишком интиллигентные. А с волками
жить - по волчьи выть. Вы должны это уже понять за 8 лет то.

\iusr{Николай Раимов}

Вот эти слова, бальзамом упали на душу Наташи. Вот тут, на этих словах,
состоялся .........................настоящий Хард-Рок........................

\iusr{Миша Кононович}
Слава Крыму и Донбассу!

\iusr{Михаил Осипенко}
... ВЫ и ответили в конце на все вопросы...

\iusr{Константин Жуматис}

Пане Мараев! Чи не хочете ви засунути той прищ, що у вас на шиї замість голови,
у свою дупу? І замовкнути... Поки ще не піздно, і це не зробили ваші палки
шанувальники

\iusr{Вячеслав Беленький}

когда говорят, что "революцию совершают романтиков", сразу приходят на память
Парасюк, Гаврилюк, и остальные Булатовы с Бляхерами.  Революцию совершают не
романтики, а "свободные радикалы", которым нечего терять, кроме своих долгов, а
в хаосе любого "движняка" появляется шанс повысить свой статус, начиная с
возможности помародерить, поесть на шару в палатке, и до получить кресло в
верховном совете.  Касается Всех революций.

\iusr{Анна Акулова}
Очень информативный эфир получился. Все четко и по полочкам.

\iusr{Вячеслав Балицкий}
Все верно и по полочкам. Даже сейчас, через 7 лет после переворота многие "оппозиционеры" боятся называть вещи своими именами.

\iusr{Сергей Неганов}

Зато политика Януковича привела к Майдану поскольку не было безоблачно в его
правление, разве что чуть лучше но это смотря для кого.

Рабочих мест людям не хватало кстати всегда как раз в его правление. Как и во
время его работы до президентства во власти ещё на заре 2000-х годов.

\obeycr
Особенно если учесть разговоры с украинцами в России приезжавшими туда на заработки и ПМЖ.
Тоже самое с газом цена которого при Януковиче росла.его безконечные политические метания Фигаро тут Фигаро там по поводу того то ли он с Западом то ли с Россией.
Перечислять есть что.
Ну вишенка на торте то что он не остановил Майдан если об этом а дал ему произойти.
Показав себя как более слабый политик.
Ну поскольку выживает сильнейший а слабый сдается или погибает то всё правильно и победителей не судят!
Лидеры Майдана победили он проиграл и его сторона в том числе силовики.
Всё таким образом честно!!!
Горе побежденным! как говорили древние.
\restorecr

\begin{itemize} % {
\iusr{Marina Glyva}
\textbf{Сергей Неганов} А вас не смущает, что это было незаконное свержение власти? О какой же победе "лидеров майдана" тогда речь, если по сути они совершили преступление. В соответствии с Конституцией люди имеют право на мирные митинги, чтобы выразить свое несогласие с действиями власти, но не на вооруженное свержение власти, которое ведет к массовой гибели людей.

\iusr{Сергей Неганов}
\textbf{Marina Glyva} А власть не всегда законно устанавливалась кем-либо в истории.
Путч в 1993-м году в Москве в России был с этой точки зрения тоже незаконным свержением власти как и Октябрьский вооруженный переворот 1917-го года.
Тогда этих тоже надо обсудить в том числе с точки зрения конституции.
По которой да незаконное свержение власти запрещено.
Так в случае майдана мирный протест перерос в вооруженный.
Меня с одной стороны смущает с другой в истории много примеров когда в том числе в Европе были такие же свержения власти.но её явно это не смутило.
Примеров весьма много.

\iusr{Marina Glyva}
\textbf{Сергей Неганов} 

Вообще мне не нравится это слово "власть": как будто люди рабы, а чиновники
владеют нами, руководство страны, их же выбирают руководить и заниматься
управленческой деятельностью, а не устанавливать свои порядки и диктатуру. 21
век уже, казалось бы, время прогресса, но нет же все те же проблемы в мире -
богатые и бедные, богатые живут за счет труда бедных, те, кто во власти,
паразитируют, распоряжаясь коллективными деньгами налогоплательщиков, войны, за
сотни лет люди так и не научились жить мирно, вечная борьба добра и зла.


\iusr{Сергей Неганов}
\textbf{Marina Glyva} Всё именно так и обстоит как Вы описали.
Поскольку выбранная власть и устанавливает свои порядки и диктатуру исходя из своих личных интересов а не народа.
Система капитализма и состоит в эксплуатации бедного богатым как и паразитировании на народе.
Человек грешен по природе поэтому иногда в его душе зло сильнее добра а отсюда и войны.

\iusr{Marina Glyva}
\textbf{Сергей Неганов} Говорят, что до 7 лет все дети - ангелы, то есть изначально человек рождается в этом мире чистым и светлым, с доброй душой, а потом уже по мере взросления узнает о существовании пороков: жадность, зависть, ненависть, но больше всего души людей искажает любовь к деньгам.

\iusr{Сергей Неганов}
\textbf{Marina Glyva} Кроме того Майдану предшествовали события поскольку он не возник из безвоздушного пространства и вдруг.

\iusr{Сергей Неганов}
\textbf{Marina Glyva} 

Как гласит фраза из любимого мною сериала «Великолепный Век»: по мере
взросления теряется невинность то есть та изначальная чистота
помыслов, души, сердца и прочего».

\iusr{Marina Glyva}
\textbf{Сергей Неганов} К счастью, не у всех чистота души теряется с возрастом, многие люди остаются со светлыми душами всю жизнь, люди, сильные духом.

\iusr{Сергей Неганов}
\textbf{Marina Glyva} может быть.

\iusr{Сергей Неганов}
\textbf{Marina Glyva} Я в том числе имел ввиду что человеку свойственно ошибаться и совершать проступки в жизни .

\iusr{Ольга Панферова}
\textbf{Сергей Неганов} А сейчас что не выходят на Майдан? Всё намного хуже или денег не дают за выход? (((

\iusr{Сергей Неганов}
\textbf{Ольга Панферова} Выходят поскольку Зе оказался вообще не способным что-то выполнять.
Так что дело не в ништяках.

\end{itemize} % }

\iusr{Сергей Неганов}

Гиркин конечно не появился просто так.

Как и митинги на Юго-Восточной части Украины у которых были свои главари типа
депутата партии ОПЗЖ Нестора Шуфрича который публично признал что это он и его
партия раскачивали лодку, призывали к неповиновению и итоге вооруженному
противостоянию.

Соответственно враг в ваших стенах.

\begin{itemize} % {
\iusr{Ольга Подщанская}
\textbf{Сергей Неганов}. Прежде, чем писать не мешало бы вспомнить как все начиналось. А начиналось, отнюдь, не в Крыму и на Донбассе. 23.0114 был одновременный захват ОДА во Львове, Ровно, Тернополе, Хмельницком, попытка захвата в Житомире и Полтаве.

\iusr{Сергей Неганов}
\textbf{Ольга Подщанская} Ясно что не народом а другими людьми.теми кто собирался оказать в случае чего вооруженный отпор.
Революции вообще не безкровны поскольку склады с оружием захватывались вполне перед самым Октябрьским переворотом в Петрограде.
Это с точки зрения революции если назвать Майдан революцией достойнства как он и называется.

\iusr{Ольга Подщанская}
\textbf{Сергей Неганов} . Народ за 3 копейки как стадо бегал за пастухами

\iusr{Сергей Неганов}
\textbf{Ольга Подщанская} Почему за 3? Сильно больше вернее даже без них если говорить о революции.

\iusr{Ольга Подщанская}
\textbf{Сергей Неганов}. Народ - за 3 коп., а пастухи - за сильно больше!

\iusr{Сергей Неганов}
\textbf{Ольга Подщанская} Ну Советы тут были бессеребриниками поскольку зарплата рабочего и директора завода были одинаковыми в том числе.
В современной России естественно в разы отличаются.
Как и депутатов государственной думы РФ.

\end{itemize} % }

\iusr{Сергей Неганов}

На майдан люди вышли в том числе чтобы выразить своё недоверие властью
Януковича тем более что они не жили как сыр в масле уже тогда ...в шоколаде в
том числе.

\begin{itemize} % {
\iusr{Наталья Мироненко}
\textbf{Сергей Неганов} А чего сейчас люди не выходят? Им сейчас хорошо живётся? Вы хоть не смешите что ЛЮДИ ВЫШЛИ. Их вывели за деньги и это факт. А сейчас не кому выводить . Денег нет столько, сколько было у тех кто выводил. А вот сейчас могут люди уже сами выйти, и то сомневаюсь. А что не говорите Сергей, а при Янеке было легче жить , и у меня работа на заводе ЮМЗ была в 2 смены, а при ПОРош. и при Зелен. работают по одному дню в неделе.Так что давайте останемся каждый при своем мнении.

\iusr{Сергей Неганов}
\textbf{Наталья Мироненко} Согласен поскольку мы и останемся каждый при своём мнении.
Если в частностях то ещё до Януковича году 1997 моей бывшей девушке и её матери пришлось уехать из Украины по причине отсутствия работы и каких-либо перспектив на будущее.
Вполне тогда когда ещё не было никакого Майдана и Порошенко.
В другом случае целой бригаде строителей из Западной Украины на заработки в Россию что было году в 2001-м.поскольку уже тогда работы в том числе не хватало.
Поскольку дело не в майдане как и его отсутствии или конкретном президенте а в условиях которые создаются правительством Рады для людей либо нет.
Я к тому что каждый случай индивидуален.
А так да будем при своём мнении.
\end{itemize} % }

\iusr{Олег Онищенко}

У меня такой вопрос возник в 2014. Если Стрелков-террорист, почему его спокойно
выпустили из Славянска, он ещё заехал в Краматорск за Бабаем, спокойно въехали
в Донецк. Пока ВСУ бомбили мирные кварталы, в т. ч. при помощи авиации. Кстати, в
Донецке сидел Кургинян и есть видео, где один из ополченцев жалуется на ржавое
оружие, которое поставили им в Славянск. То есть Стрелков абсолютно не имел
возможности сопротивляться, и политики, ВСУшники намеренно делали из него
непобедимого и оправдывали бомбежки по мирняку. Просто им нужна была война,
именно кучке майдаунов, захвативших власть. А помните многообещающий ролик
Яценюка про ЕС? Рекомендую чаще показывать и сравнивать с реалиями.

\iusr{Сергей Неганов}

В числе прочего Янукович давал обещания и не выполнял.
Вёл курс политики не ясно куда.
Насовершал кучу ошибок.

\iusr{Andrey Firsov}

Гос переворот который дорого будет стоить Украине !




\end{itemize} % }
