% vim: keymap=russian-jcukenwin
%%beginhead 
 
%%file slova.territoria
%%parent slova
 
%%url 
 
%%author 
%%author_id 
%%author_url 
 
%%tags 
%%title 
 
%%endhead 
\chapter{Территория}
\label{sec:slova.territoria}

%%%cit
%%%cit_pic
%%%cit_text
Бюрократии нужна \emph{территория} и население потому что она извлекает из этого свой
доход. Чем больше территория страны и е население, - тем выше ее доход. Самая
богатая бюрократия сейчас в Китае, Индии, США и России. Борьба за квадратные
километры своей \emph{территории} всегда объявляется священной войной и вся
пропагандистская машина государства работает на обоснование необходимой
жертвенности всей нации во имя этого принципа. Жертвует, правда, не вся нация
а, главным образом, беднейшие ее слои, а обеспеченная часть, в основном,
занимается теоретическим обоснованием этой самой священности войны. Так было
всегда и во всех странах. Последний яркий пример - это война за Карабах.  Между
тем, нет ничего более бессмысленного и антигуманного в современном мире чем
борьба, и тем более война, за \emph{территориальную целостность}. Потому что все
больше экономистов начинают осознавать тот факт, что странам с меньшей
территорией легче поднять свой уровень жизни чем странам с большой \emph{территорией}
%%%cit_comment
%%%cit_title
\citTitle{Мы живем в мире, когда размер перестал иметь значение}, 
Андрей Головачев, strana.ua, 13.06.2021
%%%endcit

%%%cit
%%%cit_head
%%%cit_pic
%%%cit_text
8. На коренной \emph{территории} империи, подчеркнул фюрер, слишком многие вещи
регламентированы законом. Этого мы ни в коем случае не должны практиковать в
оккупированных восточных областях. Для местного населения не следует издавать
слишком много законов: здесь надо обязательно ограничиться самым необходимым.
Немецкая администрация должна быть поэтому небольшой. Областному комиссару
надлежит работать с местными старостами. Ни в коем случае не следует создавать
единого украинского правления на уровне генерального комиссариата или даже
рейхскомиссариата".
Особенно интересны призывы Бормана и Гитлера ввести в Украине латиницу, к чему
призывают некоторые политики и в нынешней Украине
%%%cit_comment
%%%cit_title
\citTitle{22 июня - 80 лет нападения на СССР. Что немцы готовили для украинцев}, Максим Минин, strana.ua, 22.06.2021
%%%endcit

%%%cit
%%%cit_head
%%%cit_pic
%%%cit_text
Проблема номер один, по словам опрошенных "Страной" офицеров и солдат, - острая
нехватка личного состава. Во всех подразделениях не хватает от 40 до 70\% от
штатного расписания.  "Те, у кого заканчиваются контракты, наотрез отказываются
заключать новые и служить дальше. Причин несколько. И это даже не финансовые
проблемы. Офицеры по-скотски относятся к солдатам. Рядовой состав для
командиров тягловое быдло в буквальном смысле. Общение с солдатами происходит
только при помощи мата. Часто приказы отдаются абсурдные, в духе советской
армии. Например, нужно, чтобы к завтрашнему дню неисправные бэтээры были на
ходу. На замечание, что нет запчастей, следует примерно такой ответ: "Солдат,
меня не е@..т твое мнение, прояви солдатскую смекалку и выполни приказ", -
рассказывает "Стране" один из бойцов.  "Сейчас в частях самая популярная работа
для рядовых - парко-хозяйственная. Под этим термином подразумевается покраска
техники, заборов и бордюров, подметание плаца, уборка \emph{территорий}. Все это
вместо боевой подготовки, которая во многих подразделениях существует лишь на
бумаге, в отчетах. Солдаты ненавидят офицеров, офицеры презирают солдат", -
рассказал "Стране" сержант С.  Почти все опрошенные "Страной" офицеры отмечают,
что мотивация у большинства контрактников "ниже плинтуса" и о каком-либо
патриотизме и желании воевать не идет и речи
%%%cit_comment
%%%cit_title
\citTitle{Рабы в законе. Почему бойцы уходят из ВСУ, а в подразделениях недокомплект в 70\%}, 
Александр Сибирцев, strana.news, 11.11.2021
%%%endcit


