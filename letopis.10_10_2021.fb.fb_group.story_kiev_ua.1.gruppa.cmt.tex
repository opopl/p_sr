% vim: keymap=russian-jcukenwin
%%beginhead 
 
%%file 10_10_2021.fb.fb_group.story_kiev_ua.1.gruppa.cmt
%%parent 10_10_2021.fb.fb_group.story_kiev_ua.1.gruppa
 
%%url 
 
%%author_id 
%%date 
 
%%tags 
%%title 
 
%%endhead 
\subsubsection{Коментарі}

\begin{itemize} % {
\iusr{Вікторія Святненко}
Спасибо Вам и всем участникам чудесной и любимой группы!@igg{fbicon.heart.suit}

\iusr{Natasha Levitskaya}

Спасибо, Олег! Это была лучшая идея и группа получилась лучшая!

Я вступила в группу меньше, чем через месяц после её создания, чему бесконечно
рада, встретила здесь прекрасных людей и приобрела чудесных
друзей, которые теперь друзья в жизни! А каждая встреча с
КИшниками оставляет в душе тепло! @igg{fbicon.heart.red}

УРА!!!  @igg{fbicon.party.popper}{repeat=3} 

\begin{itemize} % {
\iusr{Олег Коваль}
Natasha, ура! @igg{fbicon.heart.red}
\end{itemize} % 

\iusr{Nadiya M Shana}
Спасибо, Олег! И даже расстояния не мешают чувствовать тепло и дух Киева и киевлян.  @igg{fbicon.hands.shake} 

\iusr{Maksym Oleynikov}
Вітаю з досягненням 100 000! Дай Боже, щоб згідно закону діалектики кількість переходила в якість @igg{fbicon.thumb.up.yellow} 

\iusr{Галина Гурьева}
Долгих счастливых лет Киевским историям!

\iusr{Марина Шкляревская}
Велике ДЯКУЮ за радість спілкування!

\iusr{Таня Головина}
Група чудова! @igg{fbicon.heart.sparkling}  Группа замечательная! @igg{fbicon.house.with.garden} 

\iusr{Elena Evgenievna Zelinka}
Вы - лучшие!!!

\iusr{Евгения Ерёменко}
Спасибо, Олег, за нашу группу и Ваше доброжелательное терпение  @igg{fbicon.face.smiling.eyes.smiling} 

\iusr{Майрам Разакова}
Спасибо за группу, чудесные истории !!! Была дважды в Киеве и очень полюбила!
Творческих успехов и ещё больше интересных историй!!!

\iusr{Татьяна Ховрич}
Спасибо!!!

\iusr{Ирина Ещенко}

Спасибо, Олег! В этой замечательной группе, я встретила своих одноклассников,
которые живут сейчас далеко, но по-прежнему любят наш любимый Киев.

\iusr{Dilettante Inlife}

Четыре года всего??? Надо отдать должное или невероятной удаче, или
организаторскому таланту, или тонкому чувству темы, но за четыре года создать
такую притягательную и такую сплоченную группу - это очень сильная работа. Не
пою панегирик, но кто понимает - тот оценит. Я рад, что конечно же случайно,
попал в нашу группу. Спасибо, Олег и все сто тысяч. Вы там отметьте как
положено.

\begin{itemize} % {
\iusr{Олег Коваль}
Благодарю, особенно за нашу, потому что это действительно наша группа.
\end{itemize} % }

\iusr{Sasha Kievskaya}
Группа сейчас интернациональная. Спасибо всем организаторам и contributors!  @igg{fbicon.heart.purple}  @igg{fbicon.heart.blue}  @igg{fbicon.heart.yellow} 

\iusr{Татьяна Сирота}

Отличная возникла идея, которая прекрасно реализовалась.
КИ -самая лучшая группа!!! @igg{fbicon.hearts.two} 

\begin{itemize} % {
\iusr{Ірина Гнатюк}
\textbf{Татьяна Сирота} Дякую Вам за Ваші неповторні дописи, які вже стали візитною карткою моєї улюбленої групи.
\end{itemize} % }

\iusr{Вадим Горбов}
Кто он? Стотысячный участник ( ца) группы?!

\begin{itemize} % {
\iusr{Олег Коваль}
Вадим, оставим в тайне...

\iusr{Вадим Горбов}
\textbf{Олег Коваль} значит, дама. Инкогнито, в секрете от мужа.)

\iusr{Олег Коваль}
Вадим, не факт.  @igg{fbicon.smile}  @igg{fbicon.face.smiling.eyes.smiling} 

\iusr{Maksim Pestun}
Стотысячному участнику полагается приз. Предлагаю книгу своих стихов. (на правах рекламы).
\end{itemize} % }

\iusr{Ирина Данькова}
Очень интересные и познавательные истории!!

\ifcmt
  ig https://scontent-frx5-2.xx.fbcdn.net/v/t39.1997-6/cp0/s110x80/16685601_1341102165951886_4721639878444449792_n.png?_nc_cat=1&ccb=1-5&_nc_sid=ac3552&_nc_ohc=FF2zI1dVoHwAX8IZ1NU&_nc_ht=scontent-frx5-2.xx&oh=5b41aaa4d06252fbee125da86d361caa&oe=6168F1FA
  @width 0.2
\fi

\iusr{Наталия Привалко}
Удачи!

\iusr{Светлана Семенец}
Всего 4 года?? Невероятно!

\begin{itemize} % {
\iusr{Олег Коваль}
Светлана, да, 24 декабря 2017.

\iusr{Светлана Семенец}
\textbf{Олег Коваль} , счастливый оказался день для всех нас.

\iusr{Светлана Манилова}
Светлана, я уже себя без КИ не помню... @igg{fbicon.face.grinning.big.eyes} 

\begin{itemize} % {
\iusr{Олег Коваль}
Светлана, и я.  @igg{fbicon.smile} 

\iusr{Светлана Семенец}
\textbf{Светлана Манилова} , мне кажется, что без КИ нет смысла в соц сети. Ну правда, зачем, если не за приятными эмоциями заходить в фб?
\end{itemize} % }

\end{itemize} % }

\iusr{Tetiana Petrovska}

Олег, благодарю за гениальную идею создания такой группы! Доброта и гармония в
группе зашкаливала! Сколько трогательных, трагичных и даже драматичных историй
нам рассказали участники. Именно в этой группе я поделилась историей своей
семьи и увидела искренний неподдельный интерес. Популярность группы растет
вместе с ее участниками, значит есть социальный запрос. Быть добру!

\begin{itemize} % {
\iusr{Олег Коваль}
\textbf{Tetiana}, благодарю за участие! @igg{fbicon.heart.red}

\iusr{Алексей Добрынин}
очень большой плюс
\end{itemize} % }

\iusr{Мария Елинская}

Олег, велика подяка за цікаві розповіді про наше Місто! Розповіді як ковток
джерельної води. Дякую Добре було б зробити розповідь про наш знищений курорт
Пущу Водицю, бо зараз цегляні мури на місті красивих дерев'яних будинків

\begin{itemize} % {
\iusr{Олег Коваль}
\textbf{Мария}, дякую!@igg{fbicon.heart.red}
\end{itemize} % }

\iusr{Мария Елинская}
Бажаю Вам творчої наснаги!

\iusr{Natalie Zhironkina}
Благодарю вас \textbf{Олег Коваль} за создание и ведение такой прекрасной группы!
Быть добру - 100\%.

\begin{itemize} % {
\iusr{Олег Коваль}
\textbf{Natalie}, благодарю!@igg{fbicon.heart.red}
\end{itemize} % }

\iusr{Людмила Старовойтенко}
Спасибо будем рады пишите нам интересно ваше мнение о историях

\iusr{Людмила Старовойтенко}
Спасибо за рассказ картина супер

\iusr{Светлана Прокопенко}
Благодарю Вас, за интересные истории , атмосферу доброжелательности и любви к городу и людям .
Быть добру!

\iusr{Maksim Pestun}

Спасибо Олег! Многие киевляне, разбросанные по свету, вернулись здесь в город
своего детства. И пусть той атмосферы уже давно нет в реале, в КИ всегда тепло
и душевно!


\iusr{Maksim Pestun}

\obeycr
Как же тянет в детство окунуться,
Все заботы хоть на миг забыть!
В дворик своей юности вернуться,
Чтобы запах детства ощутить...
\restorecr

\begin{itemize} % {
\iusr{Светлана Манилова}
\textbf{Maksim}, это лейтмотив нашей группы!
\end{itemize} % }

\iusr{Ирина Нищимная}
КИ благодарю твоих создателей,ты душа и сердце @igg{fbicon.heart.red}{repeat=2}
нашего родного Киева,,

\iusr{Dina Kozlova} БЫТЬ ДОБРУ !  СПАСИБО !

\iusr{Татьяна Кириленко} Огромное спасибо, Вам, Олег! Без вашей группы ,уже
никак! Вы-Лучшие! Читаю все! И жду новые публикации!

\iusr{Діденко Юлія}

Олег, спасибо  @igg{fbicon.hands.raising}  за возвращения веры в людей,
благодаря Вам и вашей группе  @igg{fbicon.face.happy.two.hands} Душевно, тепло,
интересно  @igg{fbicon.hands.raising}  Благодарю  @igg{fbicon.rose} 

\begin{itemize} % {
\iusr{Олег Коваль}
\textbf{Діденко Юлія}, Благодарю! @igg{fbicon.heart.red}
\end{itemize} % }

\iusr{Арно Швед}
Рад быть вместе!!!@igg{fbicon.heart.red} @igg{fbicon.hands.applause.yellow} 

\begin{itemize} % {
\iusr{Олег Коваль}
Администратор
Арно, взаимно!@igg{fbicon.hand.victory} @igg{fbicon.heart.red}
\end{itemize} % }

\iusr{Леся Лагуна}

Замечательная группа! Благодарю Вас, Олег, за воплощение мечты и создание такой
интересной и очень нужной группы. Благодарю группу «Киевские истории» за
необыкновенные история, рассказы, Воспоминания, фотографии, экскурсии - за
очень нужный ( нам киевлянам), интересный, духовный и душевный контент!

Быть добру!

\iusr{Александр Венге}

Красивая идея. Благородная идея. Очень киевская )) идея.

\iusr{Tatyana Krivtsova}
Запрос на добро - к добру!  @igg{fbicon.face.smiling.eyes.smiling}  @igg{fbicon.heart.beating}  @igg{fbicon.rose} 

\iusr{Светлана Манилова}
Спасибо, Олег, за "душу нашего клуба киевлян". Это КИ навсегда поселились в моей душе!@igg{fbicon.heart.red}

\iusr{Maksim Pestun}

Когда нас будет 500 000 мы достигнем максимального уровня населения города,
который мы все любим и о котором вспоминаем в своих экскурсиях и рассказах (467
591 человек на сентябрь 1917 года). Надеюсь доживем до этого!

\begin{itemize} % {
\iusr{Manz Ura Pochernina}
\textbf{Maksim Pestun} диживем обязательно , щоб все были здоровы  @igg{fbicon.flame}  @igg{fbicon.heart.sparkling} @igg{fbicon.exclamation.mark.double}  @igg{fbicon.100.percent} 
\end{itemize} % }

\iusr{Людмила Ива}

@igg{fbicon.heart.red}

\iusr{Tamara Khmara}

Я до 1961 года жила на Пирогова, нас выселили с подвала в Соломенский район
недалеко от КВО ( Министерство обороны), шестьдесят лет живу сдесь, а постоянно
тянет на старое место жительства. В моем прекрасном высотном историческом доме
сделали гостиницу, не видоизменили, в подвале где была наша квартира сделали
кафешку. Я периодически туда езжу.

\begin{itemize} % {
\iusr{Irena Visochan}
\textbf{Tamara Khmara} Хорошо, что ты бываешь в доме, где прошло твое детство, стены сохраняют воспоминания и тепло ваших рук.@igg{fbicon.heart.red}
\end{itemize} % }

\iusr{Лариса Шелест}
Быть добру)))

\iusr{Жанна Кулишова}
Благодарю всех участников и создателей группы за интересные рассказы  @igg{fbicon.tulip} 

\iusr{Елена Сергеева}
Прекрасная идея и создання группа!

\iusr{Nataliya Kaluzhanina}

\ifcmt
  ig https://scontent-frx5-1.xx.fbcdn.net/v/t1.6435-9/245009389_4460258804051939_2435903391645376483_n.jpg?_nc_cat=105&ccb=1-5&_nc_sid=dbeb18&_nc_ohc=wX2M69zVl4AAX-ydfZc&_nc_oc=AQmmB7n2hZ2WcJ7DMfB-m6RuJBuxCuwmuznN8tBARoplLguXcxt4miVQHDnD4qYkE3Q&_nc_ht=scontent-frx5-1.xx&oh=611b630f2c423e4e4b9c0b77e6265c9d&oe=6189CB83
  @width 0.4
\fi

\iusr{Игорь Мацак}
Спасибо большое ... Группа очень интересная ... С большим удовольствием читаю всех участников.

\end{itemize} % }
