% vim: keymap=russian-jcukenwin
%%beginhead 
 
%%file 23_06_2021.fb.bilchenko_evgenia.7.s_luschim_pedagogom_ukrainy
%%parent 23_06_2021
 
%%url https://www.facebook.com/yevzhik/posts/4003716416330105
 
%%author Бильченко, Евгения
%%author_id bilchenko_evgenia
%%author_url 
 
%%tags bilchenko_evgenia,odessa,ukraina
%%title С лучшим педагогом Украины, лишенным учительствования, как я
 
%%endhead 
 
\subsection{С лучшим педагогом Украины, лишенным учительствования, как я}
\label{sec:23_06_2021.fb.bilchenko_evgenia.7.s_luschim_pedagogom_ukrainy}
\Purl{https://www.facebook.com/yevzhik/posts/4003716416330105}
\ifcmt
 author_begin
   author_id bilchenko_evgenia
 author_end
\fi

\noindent
С лучшим педагогом Украины, лишенным учительствования, как я.

\obeycr
\noindent С тобою, сын, мы веру бережём,
Передавая правнуку от деда.
И не убить ни пулей, ни ножом
Ни нашу речь, ни совесть, ни Победу.
\restorecr

Автор: Евгений Голубенко (Одесса), \urlFriend{https://www.facebook.com/skarb.poskot}

\verb|#nasprishliubivat|


\ifcmt
  tab_begin cols=2
		 width 0.4

     pic https://scontent-lga3-2.xx.fbcdn.net/v/t1.6435-9/204825563_4003716199663460_1058399069755837427_n.jpg?_nc_cat=103&ccb=1-3&_nc_sid=8bfeb9&_nc_ohc=szcOjwNxtpwAX8n-WzQ&_nc_ht=scontent-lga3-2.xx&oh=a2ad7162e08086897035f78977fd7ea3&oe=60D8880A

     pic https://scontent-lga3-2.xx.fbcdn.net/v/t1.6435-9/205263117_4003716296330117_9188275704896713500_n.jpg?_nc_cat=102&ccb=1-3&_nc_sid=8bfeb9&_nc_ohc=nq_czVYua9EAX-zhCs1&_nc_ht=scontent-lga3-2.xx&oh=252c6ea52d6641aa4d9d5ef1aa0cf112&oe=60D8C328
		 width 0.3

  tab_end
\fi

\subsubsection{Коментарі}
\begin{itemize}
\iusr{Варвара Максимова}
Как замечательно Вы написали! И люди какие прекрасные! Утешение...

\iusr{Михаил Михайлович}
\textbf{Варвара Максимова} 

А вот и лучшие святые люди:"...Отец Иоанн Кронштадтский не имел детей, не был
связан необходимостью прокормления и воспитания. Он служил и проповедовал. С
момента прихода к нему всероссийской известности он путешествовал по всей
стране, всюду принося дух апостольской ревности и апостольского чудотворства. В
отличие от него, отец Алексий никуда не путешествовал. Храм, в котором он
служил, был одним из самых маленьких и невзрачных во всей Москве. Он был
семейный человек, и, когда отдавал последнее нищим, сердце его не раз сжималось
болью о своей семье: \enquote{Чужим помогаю, а о своих не пекусь}. Как и кронштадтский
пастырь, отец Алексий искал силы и вдохновения в молитве, наипаче — в литургии.
Каждый день год за годом в его маленьком храме звонили к литургии. Долгие годы
почти одинокого подвижничества, замешанного на нищете и борьбе с тяжёлыми
мыслями, и в храм потянулись люди. Отец Алексий был очень близок по духу
подлинному монашеству. Ведь монашество — это не только и не столько безбрачие и
чёрные одежды. Это самопожертвование и любовь, это частая молитва о людях, со
временем превращающаяся в молитву всегдашнюю. Старцы оптинские считали его
своим. Серафим Саровский, к тому времени уже вошедший в Небесный покой,
наблюдал за его деятельностью и посылал к нему за помощью людей.

Одна женщина, доведённая до отчаяния житейскими трудностями, решила свести
счёты с жизнью, для чего пошла в лес, прихватив верёвку. В лесу она увидела
сидящего на пеньке благообразного старичка, сказавшего ей: \enquote{Это ты нехорошо
задумала. Иди-ка в такой-то храм к отцу Алексию. Он тебе поможет.} В храме
первое, что она увидела — была икона старичка, спасшего её от самоубийства. Это
был Св. Серафим. Но дело не только и не столько в чудесах. Дело в любви, без
которой засыхают души человеческие; которую многие жадно ищут; которая является
отличительным признаком для безошибочного узнавания учеников Христовых.

Ничего внешне великого или грандиозного. И вместе с тем подлинная святость,
подлинное сердцеведение...Он действительно любил всех попавших в орбиту его
молитвы. Тайна подлинной святости сродни тайне истинной гениальности. Никогда
не знаешь, почему дано тому, а не этому; почему у одного хватило верности,
твёрдости, мужества, а у сотен других не хватило. Ап. Павел: И уже не я живу,
но живёт во мне Христос (Гал. 2, 20). Живут во Христе отец Алексий, отец Иоанн,
отец Серафим. И тепло, приносимое в души наши мыслями о святых, свидетельствует
духу нашему о том, что, хотя мы и далеки от всегдашнего подражания друзьям
Божиим, всё же совсем чужими мы ни им, ни Христу не являемся. Они помощники,
ранее нас прошедшие путь до конца.

Ведь действительно, если в дни земной жизни никто от них не ушёл без утешения,
то неужели сегодня, войдя в славу и умножив дерзновение, они откажут нам в
помощи...


\end{itemize}
