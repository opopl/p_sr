% vim: keymap=russian-jcukenwin
%%beginhead 
 
%%file 27_01_2021.fb.minston_marina.1.mnenie_bilchenko_ucheba
%%parent 27_01_2021
 
%%url https://www.facebook.com/marina.minston.3/posts/909411169804972
 
%%author Минстон, Марина
%%author_id minston_marina
%%author_url 
 
%%tags bilchenko_evgenia,kulturologia,nauka,presledovanie,svoboda_slova,travlja,ukraina
%%title Бильченко Евгения - травля - мнение
 
%%endhead 
 
\subsection{Бильченко Евгения - травля - мнение}
\label{sec:27_01_2021.fb.minston_marina.1.mnenie_bilchenko_ucheba}
 
\Purl{https://www.facebook.com/marina.minston.3/posts/909411169804972}
\ifcmt
 author_begin
   author_id minston_marina
 author_end
\fi

Я студентка специальности «культурология» факультета философии и
обществоведения НПУ им. М.П.Драгоманова, в связи с последними событиями я хочу
высказаться, так сказать, «о наболевшем». 

Я боюсь сейчас говорить, потому что знаю, что меня осудят, но я ещё я знаю, что
я живу в свободной стране и я имею на это право! Я пишу на русском потому что
всю жизнь говорю на русском языке. Я не считаю этот язык языком врага, потому
что мои родители говорят на этом языке, а родители с самого детства учили меня
любить и уважать свою страну. 

Сейчас вы скажете: а как можно любить и уважать свою страну, не говоря на языке
этой страны? Объясню. Я сажаю деревья, этому Дедушка меня мой приучил, а ещё я
сортирую мусор и сдаю на станции переработки вторсырья, чтоб наша с вами
Украина была чистой, я всегда кормлю бездомных животных, идя по улице, если я
вижу мусор, я поднимаю его и выбрасываю (я ужасно брезгливая, мне отвратно это
делать, но ещё куда более отвратно видеть улицы страны в мусоре), я уступаю
место в транспорте, участвую в благотворительных акциях, думаю над
образовательными проектами для нового поколения молодых людей и многое другое. 

Вот так проявляется моя любовь не только к моей державе, но и к каждому её
гражданину, независимо от того, кто на каком языке говорит. Мне это неважно. 

Но мне очень больно, когда некоторые носители украинского языка ведут себя
агрессивно по отношению ко мне, которая говорит на русском. 

Я не могу не упомянуть ситуацию, которая произошла с моим отцом, когда он
путешествовал по Украине, заехав на западную часть нашей страны. Ему просто
подставили нож к телу и сказали убираться с города, потому что услышали, что он
русскоговорящий; прошло лет 20 с тех пор, но мне страшно, что точно такое же
может произойти со мной, но нож - виртуальный в виде нападений и буллинга в
интернете. А Я думала это далеко в прошлом, ведь среди моих друзей есть много
тех, кто говорит на украинском и они уважают меня не меньше, чем я их! 

Я с первого класса училась в украинском классе, обучение в университете также
проходило/проходит на украинском языке. Мне абсолютно не сложно розмовляти
українською. Но я хочу, чтоб у меня был выбор. Правда, любовь к чему-либо может
прийти исключительно через любовь, если вы понимаете о чем я. Я понимаю, как
остро стоит вопрос украинизации населения, но я вам как гражданинка приведу
свои личные примеры повышения моего украинского самосознания. Я безумно горда,
когда на огромной трибуне собираются миллионы и играет наша команда, когда с
каждым забитым голом президент прыгает от радости и гордости за наших ребят,
когда пенсии поднимают, вот я вижу как нашей стране в лице правительства не все
равно на своих людей,  когда в Киеве поставили автомат, перерабатывающий
пластик, и взамен дающий корм для бездомных котов и собак, когда о природе
заботятся, когда народ слышат, или когда иду по Крещатику, а он зелёный,
чистый, хорошо обустроенный. Вот такие вещи поднимают мое украинское
самосознание. Но не насилие, нет. 

вот совсем недавно, я решила углубиться в украинскую культуру, сама, лично, это
не было домашним заданием или указом, нет, сама. Мне было интересно, я даже
взяла и выучила стих Лесі Українки, а потому что понравилось, потому что
захотела, тронуло вот там в сердечке. 

И украинский язык я очень люблю, есть люди, которых могу слушать часами, потому
что красиво! (Говорят на украинском), а ещё я кино дома смотрю на украинском,
ну потому что так тоже больше нравится, перевод/дубляж/озвучка иностранных
фильмов нашей стране на высшем уровне! Я хочу сказать только одно, люди должны
говорить на том языке, на котором им хочется. 

И поверьте, если через любовь, то скоро вся страна заговорит на украинском. Ну
и к слову, на работе, независимо от указаний управления, я всегда говорила на
украинском с теми гостями, которые розмовляють українською.  Из любви. 

Хочу вам рассказать о преподавательнице, которая побуждает во мне любовь к
обучению, на пары к которой, я всегда иду с улыбкой, та, которая мне помогает
даже с тем, с чем по факту не должна, но потому что я попросила, потому что
любит и ценит своих студентов. Это та преподавательница, которая объяснит тебе
закрытый/открытый шов Лакана и ты поймёшь это! Ее зовут Бильченко Евгения
Витальевна. 

На парах мы проходим Лакана, Тоффлера, Баумана, Бадью, Моргана, Тайлора,
Спенсера и других. Видите, тут нет в списках ничего, что выходило бы за рамки
образовательной программы. На наших парах Евгения Витальевна говорит
исключительно по теме и исключительно на украинском языке. Иногда, мы можем
столкнуться с Евгенией Витальевной вне пар, и знаете что? Даже вне лекций и
семинаров наши темы разговора сводятся к философии. Мне больно читать как
сейчас нашу преподавательницу оскорбляют и угрожают ей в интернете. Человек
высказал свою мысль, она имеет на это право, точно так же вы имеете право быть
с ним не согласны.

Знаете, что меня лично обижает? Обижает когда в карантин поднимают тарифы на
коммунальные услуги, когда стоимость проезда в хотят поднять больше чем в два
раза, когда старик ждал социальной еды и умер от обморожения на лавочке, когда
люди в стране от голода умирают.  

А когда я с кем-то не согласна в интернете, знаете, это не та проблема, с
которой нужно бороться. Бороться нужно с тем, что я перечислила выше. 

Я ещё раз хочу подчеркнуть, что у Евгении Витальевны мы учимся по
первоисточникам, на украинском языке и в случае малейшего несогласия с чем либо
мы всегда можем проявить своё критическое мышление, сказать своё слово против и
устроить дискуссию. Ну а то, что наши пары обсуждают люди, которые не знают,
кто такая Бильченко Евгения, не учились на нашем факультете и не присутствовали
на парах, что ж очень жаль. Жаль, что из-за вас я могу лишиться качественного
образования в лице Евгении Бильченко. 

Кричите \verb|#нпубезвати| ? 

Відповідаємо \verb|#наукабезцензури| 

\verb|#стопцькуваннюБЖ|

Я закрываю комментарии не потому что я вас боюсь, нет. А просто потому что
ужасно больно видеть как мои же сограждане меня «оплюют» только за то, что у
меня есть своё мнение, которое идёт вразрез вашему.

\ii{27_01_2021.fb.minston_marina.1.mnenie_bilchenko_ucheba.cmt}


