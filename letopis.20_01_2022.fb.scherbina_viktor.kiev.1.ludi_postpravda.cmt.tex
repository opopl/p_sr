% vim: keymap=russian-jcukenwin
%%beginhead 
 
%%file 20_01_2022.fb.scherbina_viktor.kiev.1.ludi_postpravda.cmt
%%parent 20_01_2022.fb.scherbina_viktor.kiev.1.ludi_postpravda
 
%%url 
 
%%author_id 
%%date 
 
%%tags 
%%title 
 
%%endhead 
\zzSecCmt

\begin{itemize} % {
\iusr{Сергей Мигаль}
Гоббс и Амосов однако были другого мнения на счёт обычных условий.

\begin{itemize} % {
\iusr{Viktor Shcherbina}
\textbf{Сергей Мигаль} а ты?

\iusr{Фаталиев Агарза}
\textbf{Viktor Shcherbina} Он затрудняется ...

\iusr{Сергей Мигаль}
\textbf{Фаталиев Агарза} Фатализм тут неуместен.
\end{itemize} % }

\iusr{Сергей Мигаль}

Смотря, что иметь ввиду под обычными условиями. Если догосударственное
состояние, то с ними согласиться нельзя. А если повседневную жизнь в
государстве, то можно. На то оно и классовое общество, чтобы приучать убивать,
правда каждое в разных целях.

\iusr{Viktor Shcherbina}
ясно

\iusr{Михаил Смоляной}

Дух человека - это атомарный уровень организации жизни. Его последовательное,
восходящее развитие в многообразии физических форм минерального, растительного
и животного мира накопило энергетический потенциал, необходимый для
формирования человеческого тела, как наиболее благоприятного образа для
выражения индивидуального сознания. Дух первобытного человека унаследовал из
своего ближайшего эволюционного прошлого опыт жизни в телах млекопитающих
животных, открывшийся естественно жестокими проявлениями всех людей, которые в
совокупности борьбы за выживание и личное доминирование породили тяжкие
массовые страдания. Со временем, познавая причины страданий, сами же люди
вырабатывают запреты на деяния, приводящие к ним. Нарабатывая опыт жизни в
разное историческое время, в телах мужчины и женщины, в разных расах, народах,
осознавая себя в поступках и следствиях для окружающей жизни, дух людей
избавляется от прошлых заблуждений, и новым знанием прокладывает путь в
будущее, как среду более совершенного духовного развития. Из данного знания
объективно невозможно вывести понятие изначально греха человека и его вины
перед Богом, как и все сущее библейское требование - бояться бога, внушенное
народам за для беспрекословного подчинения его уставам и сохранения библейского
образа.

\iusr{Александр Подлипьян}

Устойчивость стада нуждается в социальной норме - \enquote{справедливости}. По этой
причине стадо и существует. Сегодня это может быть любая комбинация
псевдо-религиозных и социальных \enquote{идей}. Коммунистическая, талибан, армовира...
Ученый Вадим К. называет такое (обслуживает такое) - \enquote{право выше закона}. С
этого момента беззаконие становится и приемлемым и необходимым. Превращает
социум в фашистский. Справедливость для Олеси Бузины, тем более для жертв
добробатов, садистов, мародеров... стране уже никогда не будет нужна. Так что
слово \enquote{пост} - неточно. Это непрерывная траектория и необязательно к
цивилизации. Не настаиваю. Удачи.

\iusr{Ігор Бут}
Название нового бренда водки

\end{itemize} % }
