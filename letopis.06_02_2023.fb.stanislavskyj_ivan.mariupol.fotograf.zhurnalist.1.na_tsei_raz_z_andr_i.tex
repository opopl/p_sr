%%beginhead 
 
%%file 06_02_2023.fb.stanislavskyj_ivan.mariupol.fotograf.zhurnalist.1.na_tsei_raz_z_andr_i
%%parent 06_02_2023
 
%%url https://www.facebook.com/I.Stanislavsky/posts/pfbid02c5gS4pPdqzSX3zS2JHaHYJaXwjBWnKyGvFTcS2kdxBua6qGSWiJoJwFj3nqV5rzWl
 
%%author_id stanislavskyj_ivan.mariupol.fotograf.zhurnalist
%%date 06_02_2023
 
%%tags mariupol,istoria,foto,dom
%%title На цей раз з Андрій Василенко зібрали фотоісторію будинку за адресою проспект Миру 43-45
 
%%endhead 

\subsection{На цей раз з Андрій Василенко зібрали фотоісторію будинку за адресою проспект Миру 43-45}
\label{sec:06_02_2023.fb.stanislavskyj_ivan.mariupol.fotograf.zhurnalist.1.na_tsei_raz_z_andr_i}

\Purl{https://www.facebook.com/I.Stanislavsky/posts/pfbid02c5gS4pPdqzSX3zS2JHaHYJaXwjBWnKyGvFTcS2kdxBua6qGSWiJoJwFj3nqV5rzWl}
\ifcmt
 author_begin
   author_id stanislavskyj_ivan.mariupol.fotograf.zhurnalist
 author_end
\fi

На цей раз з
\href{https://www.facebook.com/profile.php?id=100005552716002}{Андрій
Василенко} зібрали фотоісторію будинку за адресою проспект Миру 43-45. Фактично
це дві різні будівлі, з різними адресами, але збудовані вони разом у середині
30-х років і створюють єдину архітектурну групу центральної площі Маріуполя. Це
представники архітектури пізнього радянського конструктивізму, на той час вони
втілювали найсучасніші уявлення про житлову забудову. У квартирах цих двох
будинків поселилися фахівці та керівництво новозбудованого заводу "Азовсталь".
Під час німецької окупації будівлі зазнали ушкоджень, але пізніше були
відновлені, тоді й з'явилися знамениті башти на кутових фасадах. Під час
російського вторгнення будинок знову сильно постраждав.

Нам вдалося зібрати світлини що відбивають майже всі етапи історії цього
будинку. В альбомі, який ви зможете переглянути за посиланням у першому
коментарі, є ескізи його проєкту, види будинку у 30-х роках, за часів німецької
окупації, руїни будинку у 40-х роках, його відновлення та перебудова у 50-х,
фото 70-90 років, ну і звісно трагічний 2022.

Нагадаю, що з Андрієм Василенком ми розпочали проєкт по акумуляції історичних
даних про Маріуполь. Усю знайдену інформацію ми викладаємо у відкритий доступ
тим самим захищаючи від зникнення. Де факто ми хочемо зробити те, чим повинен
був займатися всі ці роки Маріупольський краєзнавчий музей. Кожен може
допомогти справі, поділившись своїми архівами.

Якщо маєте світлину якої немає у нас – надсилайте, ми додамо.

Наша поштова скринька

mrplhistory@yahoo.com

%\ii{06_02_2023.fb.stanislavskyj_ivan.mariupol.fotograf.zhurnalist.1.na_tsei_raz_z_andr_i.cmt}
