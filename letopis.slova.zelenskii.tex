% vim: keymap=russian-jcukenwin
%%beginhead 
 
%%file slova.zelenskii
%%parent slova
 
%%url 
 
%%author 
%%author_id 
%%author_url 
 
%%tags 
%%title 
 
%%endhead 
\chapter{Зеленский}

Костянтине, а що Ви хочете від неосвіченого самодура, який не запнувшись ТРИЧІ
прочитав з папірця: \enquote{КонцерТ Оборонпром} (виступ при представлені Голови
Харківської ОДА у 2019)?
Він уяви не має що таке наукове середовище. Не вважаю \emph{О. С. Зеленського} світочем
науки, але навіть він міг йому міг пояснити ідиотизм задумки,
коментар, Lon Lon, \textbf{Щодо вчорашнього указу про \enquote{Президентський університет}}, 
Костянтин Матвієнко, pravda.com.ua, 01.06.2021

Зеленський і його оточення теоретично не спроможні на ретельну наполегливу
щоденну ефективну, а головне, СИСТЕМНУ роботу. Їм в усьому потрібен хайп – по
щучому \enquote{велению}, по їхньому \enquote{хотенію}, не злазячи з печі, в'їхати у щасливе
майбутнє. Мрійники з горщика, з правом на помилку))))
У даному випадку ці зелепухи у страшному сні не знають, яка величезна і
непроста робота необхідна для утворення пристойного вишу. Схоже на те, що
Зеленський повірив у те, що він \enquote{найвеличніший хтось-там сучасності}. Підписав
височайший указ, зайняв грошенят в Європах – і все, з печі можна не злазити до
наступного хайпу)))
коментар, \textbf{Щодо вчорашнього указу про \enquote{Президентський університет}}, 
Костянтин Матвієнко, pravda.com.ua, 01.06.2021

Президент \emph{В. Зеленський} отримав унікальну в українській історії можливість
провести швидкі та ефективні реформи. Натомість його політична сила і він сам
грузнуть у внутрішніх конфліктах, примітивному меркантилізмі та
некомпетентності команди, що змушує їх до намагань відтворювати модель взаємин
влади і суспільства за прикладом сумнозвісних попередників,
\textbf{Знову \enquote{Україна не Росія}, проте багатьом дуже хочеться стерти цю відмінність},
Костянтин Матвієнко, pravda.com.ua, 02.02.2021

