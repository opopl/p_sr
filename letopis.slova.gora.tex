% vim: keymap=russian-jcukenwin
%%beginhead 
 
%%file slova.gora
%%parent slova
 
%%url 
 
%%author_id 
%%date 
 
%%tags 
%%title 
 
%%endhead 
\chapter{Гора}

%%%cit
%%%cit_head
%%%cit_pic
%%%cit_text
Було се 1241 року. Весна стояла в тухольських \emph{горах}.  Одної прегарної
днини лунали лісисті пригірки Зелеменя голосами стрілецьких рогів і криками
численних стрільців.  Се новий тухольський боярин, Тугар Вовк, справляв великі
лови на грубу звірину. Він святкував почин свого нового життя,— бо недавно
князь Данило дарував йому в Тухольщині величезні полонини і ціле одно пригір'я
Зелеменя; недавно він появився в тих \emph{горах} і побудував собі гарну хату і
оце першу учту справляє, знайомиться з довколичними боярами. По учті рушили на
лови в тухольські ліси
%%%cit_comment
%%%cit_title
\citTitle{Захар Беркут}, Іван Франко
%%%endcit

%%%cit
%%%cit_head
%%%cit_pic
%%%cit_text
Снег скрывал детали рельефа, а потому на пологих берегах исследователи не могли
понять, где кончается море и начинается суша. Плохая погода, недостаток
горючего и трудности передвижения не позволили найти площадку для взрывов в
районе Земли Бунге. Ледокол с учеными застрял во льдах. Неумолимо приближалась
полярная ночь.  Тогда исследователи и решили приступить к оборудованию
испытательной площадки на архипелаге Новая Земля. Вскоре началась подготовка
полигона к испытанию сверхмощного ядерного оружия в атмосфере. Полигон получил
название «Объект-700» и в 1959-1961 годах был значительно расширен. Появились
новые опытные поля, производственные и вспомогательные сооружения для сборки и
регулировки ракет, жилые городки.  В \emph{горах} Моисеева заложили пять штолен, а на
высоте 132 на западном берегу реки Шумилихи для управления подрывом ядерных
зарядов оборудовали командный пункт
%%%cit_comment
%%%cit_title
\citTitle{«Дом будто ножом срезало» 60 лет назад СССР взорвал «Царь-бомбу» — самую мощную в истории. Что помнят о взрыве очевидцы?: Общество: Россия: Lenta.ru}, Дмитрий Окунев, lenta.ru, 30.10.2021
%%%endcit
