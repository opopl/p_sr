% vim: keymap=russian-jcukenwin
%%beginhead 
 
%%file 04_06_2021.fb.bilchenko_evgenia.3.putnik
%%parent 04_06_2021
 
%%url https://www.facebook.com/yevzhik/posts/3952735201428227
 
%%author Бильченко, Евгения
%%author_id bilchenko_evgenia
%%author_url 
 
%%tags 
%%title БЖ. Путник
 
%%endhead 
 
\subsection{БЖ. Путник}
\label{sec:04_06_2021.fb.bilchenko_evgenia.3.putnik}
\Purl{https://www.facebook.com/yevzhik/posts/3952735201428227}
\ifcmt
 author_begin
   author_id bilchenko_evgenia
 author_end
\fi

БЖ. Путник.

Я ещё долго могу брести по ночной земле
Со свечой в руке, любовь ища до утра.
\enquote{Релятивность мнений} кончается на петле,
Где болтаются твоя Родина, мать, сестра.

Я могу ещё просвещать злодея и подлеца,
Дабы узрел он лик, дабы увидел стяг.
Но у него - уже тридцать три лица,
А на моем - даже кожи нет: 
Мясо о трёх костях.

Я ещё долгу могу говорить со злом,
У брата внутри: то ли цивила, а то ли ратного,
Чтобы сдавать в себе на металлолом
Иисусову плоть, прикрытую толерантностью.

Я ещё долго могу забывать о предлоге \enquote{а}
Между \enquote{мечом} и \enquote{миром} в словах Христа.
В диалекте варвара, 
Мыслящем на диалектике науа, -
Больше любви, чем у прогрессивных ста.

Я ещё долго могу брести, но нашёл я дом.
С красным углом в иконах под синим кровом.
Главное - преступить порог его, а потом:
Дом станет жизнью, и смертью - жизнь, 
И жизнию смерть: готово.

4 июня 2021 г.

\ifcmt
  pic https://scontent-cdg2-1.xx.fbcdn.net/v/t1.6435-9/195759549_3952735134761567_2090600334322023790_n.jpg?_nc_cat=102&ccb=1-3&_nc_sid=8bfeb9&_nc_ohc=CGeXCM6IM1cAX9mhxPv&_nc_ht=scontent-cdg2-1.xx&oh=1ada608b0e1c0b336a86ce7e7fca5282&oe=60E185EB
\fi


