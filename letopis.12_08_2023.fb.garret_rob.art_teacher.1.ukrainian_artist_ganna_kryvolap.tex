%%beginhead 
 
%%file 12_08_2023.fb.garret_rob.art_teacher.1.ukrainian_artist_ganna_kryvolap
%%parent 12_08_2023
 
%%url https://www.facebook.com/robgarrettartteacher/posts/pfbid02LXpRegyKVji4GwPsDJrm12DJ9U9jy54BDdY8Ekd8zbaBt4svvhdWmNEXRKjoNZ85l
 
%%author_id garret_rob.art_teacher
%%date 12_08_2023
 
%%tags 
%%title Українська художниця Ганна Криволап
 
%%endhead 

\subsection{Day 535 - Ukrainian artist Ganna Kryvolap - День 535 - Українська художниця Ганна Криволап}
\label{sec:12_08_2023.fb.garret_rob.art_teacher.1.ukrainian_artist_ganna_kryvolap}

\Purl{https://www.facebook.com/robgarrettartteacher/posts/pfbid02LXpRegyKVji4GwPsDJrm12DJ9U9jy54BDdY8Ekd8zbaBt4svvhdWmNEXRKjoNZ85l}
\ifcmt
 author_begin
   author_id garret_rob.art_teacher
 author_end
\fi

DAY 535 💙💛 Ukraine in my heart and on my mind! 💙💛 Today's post features
Ukrainian artist Ganna Kryvolap from Kyiv, living in both Kyiv and Podgorica,
Montenegro. She was born into the family of artists, her mother Zinayida
Vasina, an artist and theatre costume designer, and her father Anatoliy
Kryvolap (who I featured on Day 278), renowned for his landscape paintings.
Ganna Kryvolap studied at the National School of Art (1989-1995) where the
teachers who most influenced her were Lez Prezvant, Zoya Lerman and Oleg
Zhyvotkov. She subsequently studied at the National Academy of Art and
Architecture (1995-2001) in the class of set designer Danylo Danylovych Lyder.
Ganna is a Member of the Ukrainian Union of Artists and of the German Art
Association \enquote{SyrlinKunstverein}. She has presented numerous solo exhibitions
since 1999 and participated in group exhibitions since 1994, in Ukraine and
many other European countries and the USA. Her artworks are held in the USA
Embassies Collection, the Museum of Modern Art in Kyiv, the National Museum in
Zaporizhzhia, MOYA – Museum of Young Art in Vienna, Austria, and in private
collections in Austria, Germany, Poland, Holland, Canada, Norway, Sweden,
Turkey, Ukraine, and the USA.

Are you in Kyiv during August-September 2023? Go and see Ganna's exhibition
\enquote{Light through the Darkness} at the Музей Києва / Museum of Kyiv, 11 August –
17 September 2023. \url{https://www.facebook.com/kyivhistorymuseum} 

💙💛 💙💛 💙💛 💙💛 💙💛
Сьогоднішній допис присвячений українській художниці Ганні Криволап із Києва,
яка живе в Києві та Подгориці, Чорногорія. Вона народилася в родині художників,
її мати Зінаїда Васіна, художниця і художник по костюмах, і батько Анатолій
Криволап (про якого я розповідав у \enquote{Дні 278}), відомий своїми пейзажами. Ганна
Криволап навчалася в Національній школі мистецтв (1989-1995), де найбільше на
неї вплинули викладачі Лез Презвант, Зоя Лерман та Олег Животков. Згодом
навчалася в Національній академії мистецтва і архітектури (1995-2001) у класі
сценографа Данила Даниловича Лидера. Ганна є членом Спілки художників України
та німецького мистецького товариства \enquote{SyrlinKunstverein}. Вона представила
численні персональні виставки з 1999 року та брала участь у групових виставках
з 1994 року в Україні та багатьох інших країнах Європи та США. Її роботи
зберігаються в колекції посольства США, Музеї сучасного мистецтва в Києві,
Національному музеї в Запоріжжі, MOYA – Museum of Young Art у Відні, Австрія, а
також у приватних колекціях Австрії, Німеччини, Польщі, Голландії, Канади,
Норвегія, Швеція, Туреччина, Україна та США.

Ви в Києві протягом серпня-вересня 2023? Перегляньте виставку Ганни \enquote{Світло
крізь темряву} в Музеї Києва / Museum of Kyiv, 11 серпня – 17 вересня 2023.
\url{https://www.facebook.com/kyivhistorymuseum} 

💙💛 Follow the artist here:\par
FB: \url{https://www.facebook.com/ganna.kryvolap} \par
Web: \url{http://gannakryvolap.com/en} \par
Insta (portrait paintings): \url{https://www.instagram.com/ganna.portrait/} \par
Insta (personal): \url{https://www.instagram.com/ganna.kryvolap/} \par

💙💛 Look back on this 2018 magazine interview: 

Погляньте на це інтерв'ю журналу за 2018 рік: 

\url{https://artukraine.com.ua/eng/a/ganna-krivolap--moi-kolorovi-spoluchennya-ne-mayut-bukvalnikh-traktuvan/}

💙💛 \enquote{Light through the Darkness} Exhibition, Museum of Kyiv FB post: 

Виставка \enquote{Світло крізь пітьму}, Музей Києва FB post: 

\url{https://www.facebook.com/kyivhistorymuseum/posts/pfbid02nwVN3sVN3EJmB97iYWKw41KqQRqoKxzLk8qzMVCqSbA8pnFNdEeQJxCzj3eaCDHml} 

💙💛 People matter! Freedom matters! Ukraine Matters! Slava Ukraini! Heroyam
Slava! Слава Україні! Героям Слава! Glory to Ukraine! Glory to heroes! Все буде
Україна. Everything will be Ukraine.
