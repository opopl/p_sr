% vim: keymap=russian-jcukenwin
%%beginhead 
 
%%file 08_07_2021.fb.nicoj_larisa.1.podrostki_jazyk_mova
%%parent 08_07_2021
 
%%url https://www.facebook.com/nitsoi.larysa/posts/947649335801439
 
%%author Ницой, Лариса
%%author_id nicoj_larisa
%%author_url 
 
%%tags jazyk,mova,ukraina,ukrainizacia
%%title А ти можєшь па-нармальнаму пісать, па рускі?
 
%%endhead 
 
\subsection{А ти можєшь па-нармальнаму пісать, па рускі?}
\label{sec:08_07_2021.fb.nicoj_larisa.1.podrostki_jazyk_mova}
 
\Purl{https://www.facebook.com/nitsoi.larysa/posts/947649335801439}
\ifcmt
 author_begin
   author_id nicoj_larisa
 author_end
\fi

У розпал мовного скандалу з українською футбольною збірною, мені подзвонила моя
підписниця, Ірина, і розповіла, як її 11-літнього сина захейтили однолітки і
викинули з групи за його українську мову. 

Це був ігровий чат. І українські підлітки накинулися на свого ровесника, який
вчиться в Київській школі:

- А ти можєшь па-нармальнаму пісать, па рускі? 
- Піші па-чєлавєческі.
- Ти чьо? Піші как всє, на рускам ілі на англійскам. 

Хлопчик не спасував, став захищатися. Відписував українською, йому апелювали
російською. Слово за слово, дійшло до теми війни і Криму.

- Крим бил нічєйним, паетаму єво расія і забрала, - сказали українські підлітки
і виперли українськомовного з гри і з чату, заблокувавши йому доступ. 

А в цей час українські зірки футболу, на яких рівняються українці (зокрема,
діти), якими захоплюються українці, на яких моляться українці – дають
прес-конференції та інтерв’ю «на нармальнам, на чєлавєчєскам» російською чи
англійською/аби не українською.

\ifcmt
  pic https://scontent-cdg2-1.xx.fbcdn.net/v/t1.6435-9/215105645_947649275801445_1968018146997397451_n.jpg?_nc_cat=111&ccb=1-3&_nc_sid=730e14&_nc_ohc=gSW4r2IFwP8AX-uWY02&_nc_ht=scontent-cdg2-1.xx&oh=3de28d9c2e9f8a96fb97b1dd7c75a60f&oe=612F6214
  width 0.7
\fi

А ще в цей час шмат українськомовної еліти захейтив осудив з ніг до голови
письменницю Ларису Ніцой (мене), засудив  у всіх українських ЗМІ за те, що
посміла зробити зауваження українському футболісту за його інтерв’ю російською,
на якого рівняються українські підлітки.

Шмат українськомовної еліти сказали, що вони теж за українську мову, АЛЕ це
зауваження не на часі, не зараз, не тепер. Що треба було це зауваження зробити
ще в 91-му, або колись потім у майбутньому. Що не треба травмувати дорослого
спортсмена-футболіста. (Інші варіанти, звісно, не розглядаються, лише бажання
травмувати).

Що письменниця, сказавши, «французькою розмовляють французи, німецькою - німці,
а московською розмовляють московити» - дуже образила спортсменів. Що назвати
людей, які живуть в Україні, але сном і духом не знають, не читали, не чули про
мовну проблему країни, про століттями гнаних і знищених українців за мову, -
мрсіанами, ніби вони з іншої планети – це теж неприпустима образа. І запитати,
в якому вакуумі тримають таких українців, що вони не знають про ганебну мовну
ситуацію в країні – теж дуже-дуже грубо було з боку письменниці. 

Так от що я вам скажу, отій нашій «НІЖНІЙ» українськомовній еліті. Кажу дуже
лагідним тоном. Ви всі СЛАБАКИ І БОЯГУЗИ. Не 5-та колона, ні, ваша НЕРІШУЧІСТЬ
І ПРИСТОСУВАНСТВО  сприяють тому, що в нашій країні ми ніяк не вирішимо мовну
проблему. І це вашими руками провокується цькування українськомовних хлопчиків
і дівчаток по всій Україні їхніми однолітками, бо ви мнетеся з мовним питанням,
і розводите антимонії про «лагідність» у той час, коли зростає армія
змосковщених дітей. Якби всі лікарі «мнялися» з необхідністю термінового
хірургічного втручання, боячись нанести психологічну травму хворому – ми б мали
безліч летальних випадків, таких, як ми маємо в мовній площині. Бо оті всі
безповоротно змосковщені українці – це і є летальні випадки для українського
світу. 

Ви дійшли у своїй «розважливості» і «поміркованості» до того, що Футболісти вже
давно погодилися виправитися! Вони ВЖЕ перейшли на українську мову у своїх
інтерв’ю і на прес-конференціях! А ви, українськомовна еліта, й досі пишете про
Ніцой, яка вона погана і заздрісна (взагалі не розумію цього звинувачення).

Ви своїми розмовами «українська мова важлива, але не чіпайте футболістів»
зробите те, що футболісти, які стали переходити на українську, бо їх зачепили,
зроблять крок назад і подумають через вас: «А набіса нам це треба? Он, навіть
українськомовні діячі кажуть, що не обов’язково переходити зараз». Ви
перетворилися з рушіїв мовної політики у її гальмо...

- Пані Ларисо, - каже Ірина в трубку, - напишіть їм про цькування мого сина
однолітками, може вони зрозуміють, що це напряму залежить від того, якою мовою
розмовляють наші зірки і який приклад подають іншим. 

Я не знаю, чи зрозуміють, бо, як показала ця ситуація, виявляється, навіть
українськомовні борці за мову не всі розуміють. За що тоді борються? 

Дякую всім, хто розуміє і підтримує рішучі мовні кроки і рішучі
українськоцентричні кроки в гуманітарній політиці прямо зараз.

Україна буде українською!

\verb|#Футбол_Ніцой|

\ii{08_07_2021.fb.nicoj_larisa.1.podrostki_jazyk_mova.cmt}
