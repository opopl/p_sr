% vim: keymap=russian-jcukenwin
%%beginhead 
 
%%file 17_05_2021.fb.arestovich_aleksej.1.fantastika_sssr_chelovechestvo
%%parent 17_05_2021
 
%%url https://www.facebook.com/alexey.arestovich/posts/4335104169886963
 
%%author 
%%author_id 
%%author_url 
 
%%tags 
%%title 
 
%%endhead 
\subsection{СССР - Фантастика - Человечество - Эволюция}
\label{sec:17_05_2021.fb.arestovich_aleksej.1.fantastika_sssr_chelovechestvo}
\Purl{https://www.facebook.com/alexey.arestovich/posts/4335104169886963}

- За неимением возможности высказаться прямо, некоторые социально-философские
системы презентовались в СССР через советскую фантастику. 

\ifcmt
  pic https://scontent-iad3-2.xx.fbcdn.net/v/t1.6435-0/p180x540/187144936_4335100363220677_8631250237224585578_n.jpg?_nc_cat=105&ccb=1-3&_nc_sid=8bfeb9&_nc_ohc=2BBRmP24pSIAX9Xd2Rg&_nc_ht=scontent-iad3-2.xx&tp=6&oh=702d09bdabcf10d6c2b4e37b5f77e532&oe=60C9ED12
\fi

Поэтому, она перенасыщена концептами, стержневым из которых являлся принцип «активной эволюции» - человечество, осваивая производительные силы, в своём развитии доходит до решения главной задачи:
- автоэволюции, т.е. освоения эволюции, как производительной силы, управления ею, создания общества с заданными свойствами.
Я всегда смотрю на базовую антропологическую модель, заложенную в той или иной концепт: что предполагается в человеке и человечестве: больше зла или больше добра?
Автоэволюция предполагает направление добра, существующего в человеке на усиление добра в человеке, на культивирование добра в человечестве.
Кир Булычев и Ричард Викторов (авторы сценария, который писался два года!) сталкивают в фильме две модели:
- процветающей Земли, сумевшей выйти на путь автоэволюции,
- и гибнущей в экологической катастрофе Дессы, идущей тем путём, которым мы примерно идём сейчас. 
Фильм, который идёт со мной через всю мою жизнь, сформировавший мое мировоззрение.
Когда я семилетним ребёнком впервые увидел его, я ещё не подозревал, насколько актуальным эта история окажется в будущей действительности:
- гениальный учёный, сломавшийся под тяжестью зла, и решивший, что современное ему человечество безнадёжно и решивший создать новых, искусственных людей, с миссией - покончить со старой историей и начать новую,
-  вице-премьер, предавший ради власти друга-ученого, с тем, чтобы спасти его мечту о новом человечестве, и пожертвовавший жизнью за возрождение родной планеты,
- олигарх, принадлежащий к высшей жреческой касте Ревнителей, помешанный на долге перед планетой, но считающий, что высшая правда - в естественном ходе вещей, согласия с природой, отказа от антропологического бунта,
- и, наконец, искусственный человек, дочка профессора, созданная ради великой миссии спасения человечества своей планеты из клеток вице-премьера, напитанная идеалами здоровья и свободы и - оставшаяся одна в погибшем звездолета - со встроенной системой полного контроля внутри. 
Когда я подбираю фильмы на киносеминары, я не ищу популярных или новых.
Я ищу те, которые передают необходимый теме символизм.
Изучение блокирующей системы (БС) в человеке, питаемой воспитанием, образованием, концептами, картинами мира, опытом, окружением, самими обстоятельствами - тема семинара.
Как и какими средствами живые автоматы и полуавтоматы, которые мы из себя зачастую представляем, могут обойти БС, стать сильнее и выше ее, глубже в реакциях, дальновиднее в мире?
Где и, главное, в чем искать опору в попытке преодолеть внутренний контролирующий механизм, инсталлированный в нас дисциплинарным обществом, самим ходом вещей?
В этом фильме не будет ни одного героя, который не столкнулся бы с необходимостью преодолеть нечто очень глубинное и «органическое» в себе и стать Человеком, выше обстоятельств:
- Да… порядок в мире недостижим. Заштопаешь одну дыру, тут же появляется новая.
- И Вы предлагаете сложить руки и ждать?
- Может быть, естественный ход событий  надёжнее? Покориться Природе и следовать за нею?
- Тогда бы мы перестали быть людьми.
- Но вы терпите и поражения.
- Это неизбежно. Наш Мир, Галактика - это арена борьбы. Если бы мы не терпели поражения, мы не могли бы научиться собраться с силами, чтобы в следующий раз побеждать.
- А где же Предел? Цель?
- Торжество Разума. Торжество Космического Человечества над силами зла.
Торжество Разума - так называется и открывающая фильм великолепная композиция Алексея Рыбникова, написавшего к этому фильму гениальную музыку.
В конце фильма мы видим биомассу, пытающуюся сожрать планету.
Профессор Глан, отец, Нийи, даёт это массе очень точную характеристику:
- Пока в ней нет мозга, она ненасытна. Из неё мы и будем делать новых людей.
А начнём это делать с с киносеминара, 29 мая, 17-21.00, онлайн. 
—————
Записаться на семинар. 
рус:
\url{https://go.apeiron.school/cherez-ternii-ru-2905RF}
укр:
\url{https://go.apeiron.school/cherez-ternii-ua-2905RF}
