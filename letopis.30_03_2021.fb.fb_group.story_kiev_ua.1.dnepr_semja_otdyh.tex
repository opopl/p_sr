% vim: keymap=russian-jcukenwin
%%beginhead 
 
%%file 30_03_2021.fb.fb_group.story_kiev_ua.1.dnepr_semja_otdyh
%%parent 30_03_2021
 
%%url https://www.facebook.com/groups/story.kiev.ua/posts/1629465290583599
 
%%author_id fb_group.story_kiev_ua,kuzmenko_petr
%%date 
 
%%tags dnepr,kiev,kievljane,pamjat,semja
%%title Наш семейный отдых на Днепре в конце 60-х, и в 70-х годах прошлого века
 
%%endhead 
 
\subsection{Наш семейный отдых на Днепре в конце 60-х, и в 70-х годах прошлого века}
\label{sec:30_03_2021.fb.fb_group.story_kiev_ua.1.dnepr_semja_otdyh}
 
\Purl{https://www.facebook.com/groups/story.kiev.ua/posts/1629465290583599}
\ifcmt
 author_begin
   author_id fb_group.story_kiev_ua,kuzmenko_petr
 author_end
\fi

Весна в этом году, медленней чем обычно, но всё - же входит в свои права.
Апрель промелькнёт быстро, а к концу мая киевляне потянуться к Днепру. Конечно
видов отдыха у седого Славутича с древнейших времён и поныне у горожан есть
превеликое множество. Начиная с простого созерцания с набережной, мостов и
пустынных пляжей того, как древняя переменчивая и любимая каждым истинным
киевлянином река несёт свои воды. В такие минуты легко думается обо всём. И
заканчивая различными видами любимого увлечения миллионов - рыбалкой. О
днепровской рыбалке в Киеве и окрестностях нашего Города было и ещё будет
написано и рассказано немало. 

Это благодатная и захватывающая тема не одного опуса. Кто-то вспомнит и
поведает о семейном отдыхе в домиках на берегах Днепра, Десёнки, заливов,
притоков и речушек в Киеве и пригороде. Многие вспомнят пляжные приключения.
Вариантов масса. 

\begin{multicols}{2} % {
\setlength{\parindent}{0pt}
\ii{30_03_2021.fb.fb_group.story_kiev_ua.1.dnepr_semja_otdyh.pic.1}
\ii{30_03_2021.fb.fb_group.story_kiev_ua.1.dnepr_semja_otdyh.pic.1.cmt}

\ii{30_03_2021.fb.fb_group.story_kiev_ua.1.dnepr_semja_otdyh.pic.2}
\ii{30_03_2021.fb.fb_group.story_kiev_ua.1.dnepr_semja_otdyh.pic.2.cmt}

\ii{30_03_2021.fb.fb_group.story_kiev_ua.1.dnepr_semja_otdyh.pic.3}
\ii{30_03_2021.fb.fb_group.story_kiev_ua.1.dnepr_semja_otdyh.pic.3.cmt}

\ii{30_03_2021.fb.fb_group.story_kiev_ua.1.dnepr_semja_otdyh.pic.4}
\ii{30_03_2021.fb.fb_group.story_kiev_ua.1.dnepr_semja_otdyh.pic.4.cmt}

\ii{30_03_2021.fb.fb_group.story_kiev_ua.1.dnepr_semja_otdyh.pic.5}
\ii{30_03_2021.fb.fb_group.story_kiev_ua.1.dnepr_semja_otdyh.pic.5.cmt}
\end{multicols} % }

Я же хочу вспомнить один из видов нашего семейного отдыха на Днепре в конце
60-х, и в 70-х годах прошлого века. Подольская набережная Днепра (я с детства
считаю её от парка Примакова и до начала речного порта, ни больше, ни меньше) в
те времена была широкой и ухоженной. Машин было немного, никто и не думал
урезать пешеходные зоны прогулок у Днепра в историческом центре Города в угоду
многополосным широким автомобильным трассам и развязкам. Исправно ходил
удобный, экономичный и красивый обожаемый многими киевлянами речной транспорт. 

Наша семья, как и многие горожане, с нетерпением ждала ежегодного открытия
речной навигации. С любимых подольских причалов Речного вокзала, на ракетах и
метеорах мы путешествовали на экскурсии в Канев и Триполье, к родственникам в
Чернигов. На речных трамвайчиках ходили на Осокорки, где была дача друзей
семьи. 

Однако особенным удовольствием, которого я - мальчишка ждал почти весь год,
были прогулки на речном такси. Им выступал скоростной катер на подводных
крыльях \enquote{Волга}. 

Для интересующихся обмолвлюсь чуть-чуть о его технических характеристиках,
совсем не рядовых для того времени. \enquote{Волга} — общее название серии
катеров на подводных крыльях (проекты 343, 343МЕ), производившихся в СССР в
различных модификациях с 1958 по 1986 год. До 1965 года катера носили название
«Стрела»; неофициально было также распространено название «Крылатка».
Водоизмещение 1,8 тонны ширина 1,95 и 2,1 м., высота 0,98 м (на миделе), осадка
0,85 м / 0,55. Мощность двигателей 75—90 л.с., движитель винт. Скорость хода
50—70 км/ч, 1 чел. рулевой, пассажировместимость 5 чел. Ещё раз прошу пощады у
читателя за технические подробности, но обойтись без них я не мог. Ваш
рассказчик моряк, всё - таки. 

Итак, стрела (как называли её родители) была великолепна. Стремительный катер с
превосходным салоном, сидениями, ковриками и отделкой напоминающими салон
автомобиля \enquote{Волга}. Базировались речные такси, насколько мне помнится,
в разные годы на разных причалах. Однажды пришлось садиться на Красной площади
на трамвай идущий в Дарницу и ехать до парка Примакова на причал катеров. 

Но, в основном крылатки располагались у причалов Речного вокзала, до которого
мы с мамой и папой шли пешком с нашего Андреевского спуска по Боричеву току и
Жданова или сразу по Жданова, ныне Сагайдачного, до Почтовой площади. Прогулки
на речном такси (стоимость которого \enquote{кусалась} по выражению папы, 10
рублей за час) мы совершали дважды в год.  Ближе к дням рождения родителей. У
папы он был 17 мая, и навигация открывалась, как правило, чуть позже. У мамы 30
июля. Тут выбрался ближайший выходной день.  На мой декабрьский день рождения
развлечения, как читатель догадывается, были иными. Однако, вернусь к сути. 

Хочу заметить, что содержимое корзинки с перекусом мама готовила накануне и
запахи отбивных или \enquote{кручеников}, форшмака и других вкусностей ещё с
вечера будоражили обонятельные рецепторы не только наши с папой, но и соседей
по коммуналке. Корзинку мы несли бережно, предвкушая пикник на воде. 

В те времена большинство киевлян не могли себе позволить часто роскошь прогулки
на речном такси и парочка катеров со скучающими водителями непременно ожидала
нас у одного из причалов. Мы забирались на борт и размещались в салоне. Папа
сообщал рулевому маршрут, разработанный заранее на семейном совете. Обычно мы
шли к устью Десны, любуясь любимым Городом с реки.

Движение судов по Днепру в те времена было оживлённым, не чета нынешнему, и
рулевому время от времени приходилось сбрасывать казавшуюся мне невероятной
скорость. Тогда катер опускался со своих высоких подводных крыльев на днище.
Мне всё было интересно на реке и по берегам и моя белобрысая голова вертелась
во все стороны. 

Придя в устье Десны мы останавливались и пировали мамиными кулинарными яствами.
Снимали верхнюю одежду и загорали под ласковым солнышком, отражавшимся весёлыми
бликами на воде. Я с интересом наблюдал за рыбаками на берегу, особенно когда
кто-то из них тащил очередной трофей. Родители смотрели на меня и друг на друга
влюблёнными глазами. Мы разговаривали, смеялись и были счастливы. Возвращались
домой ближе к вечеру надышавшись вдоволь свежим речным воздухом и набравшись
впечатлений до следующей стремительной и яркой днепровской прогулки. 

Добравшись до своей кровати я засыпал крепко и сразу. А впечатления мои ещё
долго были основной темой рассказов друзьям по 98 детсаду у Флоровского
монастыря, а позже одноклассникам по родной 100 школе. С приятной тёплой
ностальгией и учащённым биением сердца я вспоминаю их и теперь.
