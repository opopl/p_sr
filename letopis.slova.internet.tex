% vim: keymap=russian-jcukenwin
%%beginhead 
 
%%file slova.internet
%%parent slova
 
%%url 
 
%%author 
%%author_id 
%%author_url 
 
%%tags 
%%title 
 
%%endhead 
\chapter{Интернет}
\label{sec:slova.internet}

%%%cit
%%%cit_head
%%%cit_pic
\ifcmt
  pic https://avatars.mds.yandex.net/get-zen_doc/5218808/pub_60df31aabaf4b439d35409f4_60df31f65a6af30ab28a61c8/scale_1200
	width 0.4
\fi
%%%cit_text
Двадцать первый век с его \emph{интернетом} выворачивает людей и их речь в немыслимые
звуковые баранки. Местным уже видать надоело пользоваться аглицким
заимствованием — фолловер (от англ. follower) и они, поднатужившись, придумали
— підписники (пидпысныкы рус. транскрипция.). Как бы так вам объяснить
культурно, чтобы ненароком не попасть в бан, с чем оно в моём извращённом мозгу
ассоциируется? Как только из телевизора оно раздавалось, меня словно серпом по
... Это вроде про те самые которые антагонисты у плохого плясуна. Да-да, те
самые, которые участвуют, но никуда не входят. Прям в точку придумали название
для нации — пидпысныкы. Причём в прямом и во всех переносных смыслах. Это вам
не какие-то там Табаки. Могут
%%%cit_comment
%%%cit_title
\citTitle{Яйца, но не роковые}, Дмитрий Жук (ЦИНИК), zen.yandex.ru, 02.07.2021
%%%endcit

