% vim: keymap=russian-jcukenwin
%%beginhead 
 
%%file 09_12_2020.fb.burdaniy_anatoliy.1.boris_grinchenko
%%parent 09_12_2020
 
%%url https://www.facebook.com/burdaniy.anatoliy/posts/4212365235445799
 
%%author Бурданий, Анатолій
%%author_id burdaniy_anatoliy
%%author_url 
 
%%tags grinchenko_boris,ukraine,
%%title 9 грудня 1863 року день народження Бориса Грінченка
 
%%endhead 
 
\subsection{9 грудня 1863 року день народження Бориса Грінченка}
\label{sec:09_12_2020.fb.burdaniy_anatoliy.1.boris_grinchenko}
\Purl{https://www.facebook.com/burdaniy.anatoliy/posts/4212365235445799}
\ifcmt
	author_begin
   author_id burdaniy_anatoliy
	author_end
\fi

\index[writers.rus]{Грінченко, Борис!Украина!Письменник, педагог, історик}

9 грудня 1863 року день народження Бориса Грінченка – видатного українського
письменника, педагога, лексикографа, літературознавця, етнографа, історика,
публіциста, громадсько-культурного діяча. Він народився  на хуторі Вільховий Яр
поблизу села Руські Тишки, тепер Харківського району Харківської області.

Його батько відставний штаб-ротмістр – нащадок козацького роду. Прадід
письменника, козак Грінченко, в селі Межиріччя (під Харковом) так затято й
самотньо виступив проти закріпачення селян, що був покараний нагаями й закутий
в кайдани. Мати - з московської родини полковника Літарева. Рідна бабуся по
батькові – двоюрідна сестра основоположника української прози
Г.Ф.Квітки-Основ’яненка. Батько добре знав українську мову, але спілкувався нею
тільки з селянами, дома ж розмовляли тільки російською. Але через постійний
контакт із простими селянами хлопчик усвідомив  свою національну приналежність.

А довершив формування його душі \enquote{Кобзар} Т.Г.Шевченка – він досить рано
перечитав все, що було в батьківській бібліотеці і під впливом прочитаного
почав писати вірші.

У 1874 р. поступив до Харківської реальної школи. Однак Бориса виключили разом
із гуртом революційно налаштованих народників і припровадили до тюрми. Це
означало позбавлення права здобути навіть середню освіту і  до кінця життя
жандармський нагляд. Крім того, тяжкі тюремні умови спричинили невиліковну
хворобу - сухоти, що, зрештою, і вкоротили йому віку.

Згодом Б. Грінченку все ж вдалося влаштуватися дрібним канцеляристом у
Харківській казенній палаті, а невдовзі й скласти екстерном іспит за фахом
народного вчителя. З 1881 до 1893 р.(за винятком 1886–1887 рр., коли був
статистиком у Херсонському губернському земстві) вчителював на Слобожанщині і
Катеринославщині. У 1887 році разом з молодою дружиною мешкав на Донбасі в
селищі Олексіївка (Катеринославська губ., зараз Луганська обл.) та працював у
народній школі Христі Алчевської.

1894 р. Б. Грінченко переїхав до Чернігова. Він зайняв посаду діловода
оціночної комісії губернського земства, з 1898 р. - секретаря земської управи.
Тут, разом із дружиною організував перше в Україні народне видавництво, де
вийшло друком 45 книжок народнопросвітницького характеру. Зокрема «Про грім та
блискавку», «Велика пустеля Сахара», «Жанна д'Арк», життєписи Івана
Котляревського, Євгена Гребінки, Григорія Квітки-Основ'яненка та ін. Автограф
Грінченка на одному з них можна побачити тут:

\ifcmt
pic https://ireland.apollo.olxcdn.com/v1/files/4rjt68bwrjwe2-UA/image;s=1000x700
\fi
%https://www.olx.ua/.../bograf-kotlyarevskiy-kulsh-darchiy....

Пізніше, працюючи на посаді завідувача відділу народної просвіти Чернігівського
губернського земства, у підготовлених доповідях Борис Грінченко постійно
ставить питання про створення сільських, учительських, учнівських бібліотек та
земських книжкових складів. Він домагався, щоб у кожній бібліотеці «…було усе,
що треба для такої початкової загальної просвіти, од котрої можна було б
перейти до просвіти вищої».

Великого значення Борис Грінченко надавав не тільки книзі, а й народному театру
як важливим чинникам освіти. Він провів перше соціологічне дослідження про
народні читання, вплив книги на сільського читача, яке описав у праці «Перед
широким світом». На посаді земця в Чернігові Борис Грінченко велику увагу
приділяв організації народних труп. Знаємо, що і сам він виконував роль Мартина
Борулі в однойменному спектаклі Чернігівського народного театру.

Б. Грінченко тісно співпрацював з галицькими виданнями. Він належав до активних
дописувачів критико-бібліографічного відділу журналу «Зоря». Свої зведені праці
з поточної бібліографії він публікував під назвою «Новості української
літератури» і «Нові українські книжки» під псевдонімами В. Чайненко і В.
Вільхівський, а в Україні його мало хто знав. Франко написав  про Грінченка: «
Працює без віддиху…, проявляє гарячу любов до України, щирий демократизм,
бистре око на хиби української суспільності»

1902 р. Борис Грінченко з сім'єю переїжджає до Києва щоб закінчити роботу над
словником української мови. Матеріали для видання в середині 19-го століття
почав акумулювати Пантелеймон Куліш, потім вони були доповнені багатьма
відомими українцями. З головою поринувши в робочий процес, Грінченко не тільки
упорядкував і відредагував отриманий матеріал, а й доповнив його. Внеском
Грінченка стали додаткові 19 тисяч слів в словнику. У 4 томах "Словник
української мови" виходив у світ протягом 1907-1909 років. Його Берлінське
видання ( фототипічним способом) 1926 року можна побачити тут:

%https://crafta.ua/.../6537707281-slovnik-ukrayinskoyi...
\ifcmt
pic https://crafta.ua/lots/6537707281-slovnik-ukrayinskoyi-movi-za-redakciyeyu-b-d-grinchenka?fbclid=IwAR1rUSq6-QJPbdjf7HBip33MRyHPK9Xm7wD_2FKPpwofw3emUa71xyvmFjM
\fi

До словника включено 68 тисяч українських слів, які використовуються в народній
та письмовій мові, - українська лексика з часів Івана Котляревського і до
початку 20-го століття.  Для української науки і культури словник Грінченка має
таке ж значення, як словник Даля для російської мови, словник Лінде для
польської мови або словник Гебауера для чеської мови. Письменника можна назвати
батьком апострофа в українській мові і розробником багатьох фундаментальних
правил правопису.

В київський період Грінченко працював не тільки над підготовкою до видання
словника, він займався і іншими важливими завданнями. Зокрема, чимало уваги
письменник приділяв організації роботи української преси. Письменник редагував
першу українську щоденну газету \enquote{Громадська думка}, яка потім була
перейменована в газету \enquote{Рада}.

Про особисте, приватне життя. Борис Грінченко був одружений з Марією
(1863–1928), у дівоцтві Гладиліною. Вони повінчалися 10 лютого 1884 року в
приміщенні школи в селі Нижня Сироватка (нині Сумська область), де Борис на той
час був завідувачем. Грінченко з дружиною були дуже близькі, окрім кохання, їх
єднала спільність переконань і життєвих позицій. Напередодні весілля Марія
писала своєму нареченому: «Ми любимо один одного, у нас тепер одна душа, але
любов до України і спільна праця на користь їй ще дужче з'єднають нас і дадуть
силу перемогти все. Але ти це й сам розумієш ще краще мене… Твоя Маруся».


Сучасники розповідали, що у Грінченка з дружиною Марією відносини були
дивовижними. Народжена в російській родині, вона настільки перейнялася ідеєю
любові до України, що на протязі всього подальшого життєвого шляху була не
тільки вірним товаришем літератору, але і сама усіма можливими способами
наближала краще майбутнє для країни;

У подружжя була лише одна дитина — донька Анастасія. Вона цікавилась
українським національним рухом, перекладала, пробувала писати, захоплювалась
музикою. Після закінчення гімназії у Києві, Настя Грінченко вирушила до Львова,
де записалася на філософський факультет і слухала лекції професорів
Грушевського, Колесси, Студинського. Значне враження справила на неї зустріч з
Іваном Франком. Крім того, вона познайомилася з членами РУП, що її повністю
захопило. Приїжджаючи додому, до Києва, не зважаючи на перевірки, Анастасія
привозила підпільну літературу. Архівні документи засвідчують пряму причетність
Анастасії Грінченко до збройних виступів протягом революції 1905–1907 років, в
яких вона брала участь зі своїм нареченим. Після того, як її було ув'язнено, на
початку 1906 року у Анастасії розвинувся туберкульоз. Борис та Марія докладали
великих зусиль, аби звільнити доньку за станом здоров'я. Туберкульоз розвивався
швидко і 1 жовтня 1908 року Анастасія Грінченко померла. Невдовзі помер
немовлям і її єдиний син. Ці смерті дуже підірвали здоров'я Бориса, він помер
за півтори роки після доньки. Похорони відбулися 9 травня у Києві на Байковій
горі. Дуже багато людей висловлювали свій сум з приводу втрати письменника,
широко відгукнулася українська та російська преса. Ці відгуки можна побачити

\ifcmt
tab_begin cols=4
	caption На смерть Бориса Гринченка
pic https://ireland.apollo.olxcdn.com/v1/files/94bz2e1664yf1-UA/image;s=1000x700
pic https://ireland.apollo.olxcdn.com/v1/files/wve8oikkbn6y1-UA/image;s=1000x700
pic https://ireland.apollo.olxcdn.com/v1/files/v4zhjjlvvait2-UA/image;s=1000x700
pic https://ireland.apollo.olxcdn.com/v1/files/qv0dmqk35xe41-UA/image;s=1000x700

pic https://ireland.apollo.olxcdn.com/v1/files/1qk154rgd5bm-UA/image;s=1000x700
pic https://ireland.apollo.olxcdn.com/v1/files/u42sshqa2mjs2-UA/image;s=1000x700
pic https://ireland.apollo.olxcdn.com/v1/files/ut0o5kiz629j2-UA/image;s=1000x700
pic https://ireland.apollo.olxcdn.com/v1/files/3upmdifsutq12-UA/image;s=1000x700
tab_end
\fi
%тут: https://www.olx.ua/.../vidannya-pamyat-borisa-grnchenka...

Як пам'ять про Великого Українського Трударя – один із найяскравіших творів
Грінченка -  вірш «Землякам, що раз на рік збираються на Шевченкові роковини
співати гімн» (1898 р.):

Ще не вмерла Україна,
Але може вмерти:
Ви самі її, ледачі,
Ведете до смерті!

Не хваліться, що живе ще
Наша воля й слава:
Зрада їх давно стоптала,
Продала, лукава.

Ваші предки торгували
Людськими правами,
Їх продавши, породили
Нас на світ рабами.

Не пишайтеся ж у співах
Ви козацьким родом:
Ви раби, хоча й пани ви
Над своїм народом.

Україна вам не мати,
Є вам інша пані,
Зрадних прадідів нікчемних
Правнуки погані!

Тільки той достойний щастя,
Хто боровсь за його,
Ви ж давно покірні слуги
Ледарства гидкого.

Ви ж давно не люди — трупи
Без життя і сили,
Ваше місце — кладовище,
Яма та могили.

Як живі покинуть мертвих,
Щоб з живими стати,
«Ще не вмерла Україна»
Будемо співати.

Як живі покинуть мертвих,
Прийде та година,
Що ділами, не словами
Оживе Вкраїна.

Згодіться, що слова ці звучать актуально і нині!
