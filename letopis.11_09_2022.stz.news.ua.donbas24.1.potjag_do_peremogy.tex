% vim: keymap=russian-jcukenwin
%%beginhead 
 
%%file 11_09_2022.stz.news.ua.donbas24.1.potjag_do_peremogy
%%parent 11_09_2022
 
%%url https://donbas24.news/news/potyag-do-peremogi-yak-istoriyi-regioniv-vtililisya-v-malyunkax-ukrayinskix-mitciv-foto
 
%%author_id news.ua.donbas24,shvecova_alevtina.mariupol.zhurnalist
%%date 
 
%%tags 
%%title Потяг до Перемоги: як історії регіонів втілилися в малюнках українських митців (ФОТО)
 
%%endhead 
 
\subsection{Потяг до Перемоги: як історії регіонів втілилися в малюнках українських митців (ФОТО)}
\label{sec:11_09_2022.stz.news.ua.donbas24.1.potjag_do_peremogy}
 
\Purl{https://donbas24.news/news/potyag-do-peremogi-yak-istoriyi-regioniv-vtililisya-v-malyunkax-ukrayinskix-mitciv-foto}
\ifcmt
 author_begin
   author_id news.ua.donbas24,shvecova_alevtina.mariupol.zhurnalist
 author_end
\fi

\ifcmt
  ig https://gcdnb.pbrd.co/images/1XIFVomIzve8.png?o=1
  @wrap center
  @width 0.9
\fi

\ii{11_09_2022.stz.news.ua.donbas24.1.potjag_do_peremogy.0.intro}
\ii{11_09_2022.stz.news.ua.donbas24.1.potjag_do_peremogy.1.geroi_azovstal}

\subsubsection{Луганщина. Герої: лікарі на тимчасово окупованих територіях}

Вже 8 років росія тероризує людей на сході України. Цього року, попри запеклі
та героїчні бої за Сєвєродонецьк, Лисичанськ та інші ключові точки Луганщини,
війська рф майже повністю окупували Луганську область.

Митець \textbf{Роман Синенко} зобразив лікарів, які весь цей час у найгарячіших точках
та на тимчасово окупованих територіях регіону продовжують героїчно працювати.
Станом на 15 квітня на Луганщині вже не було жодної лікарні, яка б не зазнала
обстрілів. Попри це Михайло, дитячий анестезіолог із Лисичанська, і хірург
Максим продовжували рятувати життя під кулями, робили операції в окопах та
допомагали пацієнтам у підвалах.

\ii{11_09_2022.stz.news.ua.donbas24.1.potjag_do_peremogy.pic.4}

Ескіз розповідає про найгероїчніших людей в нашому суспільстві — про медиків,
лікарів, парамедиків, госпітальєрів. Композиція базується на хресті, який
проходить через весь вагон; портретах лікарів в авторському стилі, що
піклувалися про пацієнтів під час окупації, інколи досягаючи неможливого
наполегливістю та хоробрістю. На ескізі присутній медичний символ — чаша зі
змією, де змія зображується як агресор, який атакує. 

\subsubsection{Рух опору Херсонщини}

Рух спротиву окупантам на Херсонщині — потужна зброя проти колаборантів та
окупантів. Партизани допомагають ЗСУ наближати деокупацію регіону, підтримують
бойовий дух українців, які залишились на ТОТ, нагадуючи, що Херсон — це
Україна, а ЗСУ поруч.

\ii{11_09_2022.stz.news.ua.donbas24.1.potjag_do_peremogy.pic.5}

Новокаховчанин Максим Кільдеров зображує херсонців, які чинять спротив
окупантам. Рух опору, який підриває русняві тачки та \enquote{штаби}. Інфовоїнів, які
доносять правдиву інформацію в умовах россмі та передають координати. Його
полотно стає рухомим літописом героїчних вчинків жителів Херсонщини на цій
війні. 

\subsubsection{Миколаївщина. Герої: фермери на тимчасово окупованих територіях}

Фермери на тимчасово окупованих територіях стали справжніми героями.

Вони продовжують діяльність попри погрози та шантаж із боку окупантів, гасять
поля, які намагаються спалити російські ракети та всіма можливими силами
протистоять загарбникам. Саме українські фермери стримують світ від можливої
продовольчої кризи, у яку штовхає планету армія окупантів.

\ii{11_09_2022.stz.news.ua.donbas24.1.potjag_do_peremogy.pic.6}

Мисткиня \textbf{Аліна Коник} створила малюнок, надихнувшись творчістю українських
шістдесятників та примітивістів — Марії Приймаченко, Алли Горської, Віктора
Зарецького. В основі концепції — хоробрість українських фермерів, які попри
смертельну небезпеку продовжують годувати нас та світ. З їх рук ми маємо
урожай, рятуємо економіку, вберігаємо себе та багато інших країн від голоду.

Одним з ключових елементів зображення виступають руки, що символізують
титанічну працю. Також в ескізі є символи з писанок (символи родючості) та з
рушників (елементи дерева життя). Вони є оберегами, тож фінальна робота є
прагненням захистити хоробрих фермерів та наші землі від окупантів.

\subsubsection{Крим. Герой: Богдан Зіза}

Вісім років тому росія анексувала Крим. З того часу на півострові жорстоко
придушувався будь-який спротив і проукраїнські настрої. Щоб стерти всі
асоціації з Україною. Назавжди. Проте не вийшло.

16 травня 2022 року мережею розлетілися кадри протесту Богдана Зізи —
українського митця, який облив синьо-жовтою фарбою міську адміністрацію в
Євпаторії. Його заарештували того самого дня. Наразі він перебуває у СІЗО в
Сімферополі, а сестра та друзі героя борються за його свободу. Вчинок Богдана —
це голос усіх українців у тимчасово окупованому Криму.

\ii{11_09_2022.stz.news.ua.donbas24.1.potjag_do_peremogy.pic.7.krym}

У своїй роботі митець Андрій Присяжнюк уособлює хоробрість та ідейність Богдана
Зізи, який розумів можливі ризики, та попри все, вирішив висловити свою незгоду
та непокору російському режиму в окупованому Криму та російській
повномасштабній війні в Україні в формі, звичній для активістів в усьому світі.

\subsubsection{Харківщина. Герої: залізничники евакуаційних потягів}

З 24 лютого Харківська область піддається постійним обстрілам. Під тимчасовою
окупацією перебуває 30\% регіону. Цей вагон присвячений незламній Харківщині та
залізничникам, які евакуюють тих, хто вимушений покинути тимчасово окуповані
території та зруйновані міста і містечка.

\ii{11_09_2022.stz.news.ua.donbas24.1.potjag_do_peremogy.pic.8}

Концепт художника \textbf{Дмитра Касянюка} розроблявся на основі нескінченної кількості
історій про евакуацію українського населення за допомогою евакуаційних потягів
\enquote{Укрзалізниці}.

Тут можна побачити холоднокровний погляд жінки, яка чітко розуміє, що потрібно
зберігати спокій та не піддаватися паніці у тяжкі хвилини. Відчути водночас
неймовірну людяність та героїчну хоробрість, яку наш народ проявляє кожного
дня, під вибухами, кулями та обстрілами. Автор задумав зобразити на своєму
полотні образи Провідника та Провідниці поїзда, адже залізничники — це саме ті
люди, які не показували страху тоді, коли страшно було абсолютно всім. 

\subsubsection{Запоріжжя: герої Енергодару та працівники ЗАЕС}

Майже 60\% Запорізької області знаходиться під окупацією. Зокрема Енергодар —
місто, де розташована найбільша атомна електростанція у Європі. Місто, що стало
символом героїчного спротиву його мешканців, які виходили на багатотисячні
мітинги, щоб зупинити окупантів. Наприкінці лютого росіяни кілька разів
намагалися прорватися у місто, але натикалися на барикади й розверталися.

\ii{11_09_2022.stz.news.ua.donbas24.1.potjag_do_peremogy.pic.9}

Героями цього вагону стали мешканці Енергодару, які живим щитом стояли на
в'їзді до міста та виходили на кількатисячні демонстрації, щоб зупинити
окупантів та запобігти створенню пропагандистської картинки, та працівники
ЗАЕС, яких росіяни викрадають, катують та тримають у заручниках, з метою мати
контроль над станцією та шантажувати весь світ можливою ядерною катастрофою.
Стінопис створив художник \textbf{Сергій Турнікевич}.

Нагадаємо, раніше Донбас24 розповідав, як \href{https://donbas24.news/news/teatromaniya-prodovzuje-rozvivati-ta-pidtrimuvati-kulturu-mariupolya}{\enquote{Театроманія} продовжує розвивати та
підтримувати культуру Маріуполя}.

Ще більше новин та найактуальніша інформація про Донецьку та Луганську області
в нашому \href{https://t.me/donbas24/}{телеграм-каналі Донбас24}.

ФОТО: агенція \href{https://www.facebook.com/grestodorchuk/}{Gres Todorchuk}

\ifcmt
  ig https://gcdnb.pbrd.co/images/TBpnhHDpTRVB.png?o=1
  @wrap center
  @width 0.9
\fi

\ii{insert.author.shvecova_alevtina}
