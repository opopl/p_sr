%%beginhead 
 
%%file 19_02_2022.fb.mariupol.vezha_centr.1.progulyanka__v_d_vod
%%parent 19_02_2022
 
%%url https://www.facebook.com/vezhacreativespace/posts/pfbid02h4uEogQGB8ZAKw5VonqTcW11aM8PAgU5kCo32vpRiqNqGVWUGppf9Kno5MzWiXuSl
 
%%author_id mariupol.vezha_centr
%%date 19_02_2022
 
%%tags mariupol,mariupol.pre_war,mariupol.turizm,turizm,progulka,mariupol.vezha,mariupol.ploscha.svobody,more,more.azov
%%title Прогулянка: від водонапірної вежі - до морського узбережжя та від набережної - до площі Свободи
 
%%endhead 

\subsection{Прогулянка: від водонапірної вежі - до морського узбережжя та від набережної - до площі Свободи}
\label{sec:19_02_2022.fb.mariupol.vezha_centr.1.progulyanka__v_d_vod}

\Purl{https://www.facebook.com/vezhacreativespace/posts/pfbid02h4uEogQGB8ZAKw5VonqTcW11aM8PAgU5kCo32vpRiqNqGVWUGppf9Kno5MzWiXuSl}
\ifcmt
 author_begin
   author_id mariupol.vezha_centr
 author_end
\fi

Пропонуємо вам здійснити прогулянку маршрутом нового туристичного трикутника
Маріуполя: від водонапірної вежі - до морського узбережжя та від набережної -
до площі Свободи.

⚓️Розпочати свій маршрут можна з найвищого оглядового майданчика Маріуполя, що
знаходиться на території старої водонапірної вежі.

⚓️Від \enquote{Вежі} прогулятися до Театральної площі та Театрального скверу, а дорогою
подивитись на залишки фундаменту собору Марії Магдалини.

⚓️На шляху до Міського саду, є можливість роздивитися місце майбутнього
мультикультурного центру \enquote{Порт Культур}.

⚓️Міський сад подарує безліч вражень від спостереження панорами моря та
узбережжя з оглядового майданчика центральної алеї.

⚓️Розпочинаючи свій шлях до морської набережної, не можна оминути новий пірс з
його терасою, скульптурами та кав'ярнями.

⚓️Після морської прогулянки, радимо відвідати Приморський парк та алею \enquote{Якір},
її панорамою маріупольських пляжів та розміщеним на ній військовим катером
Другої Світової війни.

⚓️Бульвар Богдана Хмельницького приведе вас на площу Свободи, з її найбільшим
пішохідним фонтаном України та лазерним шоу.

⚓️А безтурботна прогулянка проспектом Миру може запропонувати велику кількість
кафе, де можна поласувати місцевими смаколиками, або, навіть, маріупольським
тортом від кондитерської \enquote{Марія Федорівна}. ☕️

В решті-решт, дорога приведе вас до культурно-туристичного центру \enquote{ВЕЖА}, де
можна придбати сувеніри на згадку про Маріуполь, або оцінити протяжність свого
маршруту з висоти пташиного польоту.

%\ii{19_02_2022.fb.mariupol.vezha_centr.1.progulyanka__v_d_vod.cmt}
