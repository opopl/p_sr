% vim: keymap=russian-jcukenwin
%%beginhead 
 
%%file 03_10_2020.sites.ru.zen_yandex.yz.netrivialnaja_istoria.1.kitaj_gorod_moskva_tajna
%%parent 03_10_2020
 
%%url https://zen.yandex.ru/media/id/5f6a1d4eaddbe610281941b4/kitaigorod-taina-imeni-5f78613671c44f0829e3cec6
 
%%author 
%%author_id yz.netrivialnaja_istoria
%%author_url 
 
%%tags moskva,kitai_gorod
%%title «Китай-город»: тайна имени
 
%%endhead 
 
\subsubsection{«Китай-город»: тайна имени}
\label{sec:03_10_2020.sites.ru.zen_yandex.yz.netrivialnaja_istoria.1.kitaj_gorod_moskva_tajna}
\Purl{https://zen.yandex.ru/media/id/5f6a1d4eaddbe610281941b4/kitaigorod-taina-imeni-5f78613671c44f0829e3cec6}
\ifcmt
	author_begin
   author_id yz.netrivialnaja_istoria
	author_end
\fi

\index[cities.rus]{Москва!Китай-город!Тайна имени, 03.10.2020}

В мае 1534 года великая княгиня Елена Глинская, правившая в Москве от имени
малолетнего Ивана, повелела обнести стеной «Великий посад». За первый год
выкопали ров и насыпали вал. В течение следующих лет (1535-1538) возвели
каменную стену. Новопостроенные укрепления («город») повелели называть
«Китаем». Это имя появилось еще на первом этапе строительства (земляной вал).

С Китай-городом связаны два недоразумения. Первое – географическое. До начала
XX века всем москвичам было очевидно: Китай-город – это улицы Никольская,
Ильинка, Варварка и переулки вокруг них, словом – все, что находится внутри
Китайгородских стен. Однако большую часть стен разрушили в 1934 году. А в 1990
году название «Китай-город» получила станция метро, но она расположена за
пределами этого района. И теперь некоторые москвичи стали называть
«Китай-городом» все, что находится недалеко от станции метро – скажем, кварталы
рядом с улицей Забелина или Солянкой. Это нелепость.

Правильные границы исторического района «Китай-город»:

\ifcmt
pic https://avatars.mds.yandex.net/get-zen_doc/3989805/pub_5f78613671c44f0829e3cec6_5f7861af61e6d41ef5d425a0/scale_2400
caption Красной чертой – границы Китай-города, черной – границы Кремля
\fi

Второе недоразумение связано с самим именем Китай-города. Что оно означает?
Некоторым иностранцам кажется, что это «чайна-таун», как называют китайские
кварталы в западных столицах. На самом деле это просто совпадение. Московский
Китай-город никак не связан с Поднебесной. Во времена Елены Глинской русские с
китайцами не контактировали.

Каков же действительный смысл этого имени? Было предложено много гипотез.
Большинство из них оказались неубедительными.

\subsubsection{Версия первая. «Средний город»}

Еще в XVIII веке русский географ Федор Афанасьевич Полунин писал, что
Китай-город означает по-татарски «Средний город» (эту версию поддерживал и
Карамзин). Однако в татарском языке слово «китай» не значит «средний».
По-татарски это было бы «урта».

Известный краевед Петр Васильевич Сытин писал, что «китай» – это «средний»
по-монгольски. Что тоже не соответствует действительности.

Эти гипотезы нелепы и с исторической точки зрения. Крепость назвали
Китай-городом уже во время ее строительства. В те времена никаких «внешних»
стен (Белый и Земляной город) еще не существовало, они появятся через десятки
лет. А значит, Китай-город не был «средним» в тот момент, когда получил имя.

\ifcmt
pic https://avatars.mds.yandex.net/get-zen_doc/3362051/pub_5f78613671c44f0829e3cec6_5f78623a71c44f0829e55fba/scale_1200
caption Часть Китайгородской стены на фотографии 1887 года
\fi

\subsubsection{Версия вторая. «Плетенка»}

Версия историка Ивана Егоровича Забелина кажется более основательной. Он
предположил, что крепость была названа по технологии строительства земляного
вала: землю (по летописным данным) насыпали в плетенки из тонких молодых
деревьев. В словаре Даля читаем:

\begin{leftbar}
  \begingroup
    \em\Large\bfseries\color{blue}
Кита… стебель, трава повойного и долгоствольного растения… свитый кольцом
сенный вьюк… плетеница цветочная, травяная, соломенная.
  \endgroup
\end{leftbar}

Однако в этом объяснении есть несколько натяжек. Первое: трава и травяная
плетенка – все-таки не то же самое, что сплетенная из дерева корзина. Второе: в
русском языке сложно из слова «кита» образовать слово «китай». Третье, самое
важное: название «Китай-город» возникло не стихийно. Крепость назвал не народ,
а великий князь (точнее, его мать-правительница): так говорит летопись. Княгиня
не стала бы давать имя грандиозной по тому времени стене из-за каких-то
плетенок. Поэтому точка зрения Забелина кажется не слишком убедительной.

\ifcmt
pic https://avatars.mds.yandex.net/get-zen_doc/3985748/pub_5f78613671c44f0829e3cec6_5f786501952c3b370e7c2d5f/scale_1200
caption Иван Егорович Забелин. Портрет работы Репина.
\fi

\subsubsection{Версия третья. Подолия и тюркский след}

Другая версия увязывает название «Китай» с биографией правительницы. Отец Елены
Глинской в 1508 году бежал в Московскую Русь из Великого княжества Литовского.
Как иногда предполагают, Елена Глинская родилась в Подолии (юго-западная
Украина), в Китай-городке. В Подолии есть несколько других похожих названий,
которые связывают с тюркским словом «крепость». Может быть, вместе с Глинскими
это слово попало в Москву?

В этой версии опять-таки есть несколько натяжек. Елена Глинская родилась как
раз около 1508 года, когда отец ее бежал в Москву. Родилась ли она в Подолии?
Существовал уже ли в это время Китай-городок и другие украинские «китаи»?
Сведения об этих поселениях относятся уже к XVII веку. Существовало ли само
слово «китай» (не имена городков, а слово) в диалектах юго-западной Руси? И
если нет, был ли смысл называть крепость в Москве словом из тюркских языков?

Вопросы, вопросы, вопросы…

\ifcmt
pic https://avatars.mds.yandex.net/get-zen_doc/3362051/pub_5f78613671c44f0829e3cec6_5f78652f71c44f0829e9efbf/scale_1200
caption Бракосочетание великого князя Василия III с Еленой Глинской. Миниатюра из летописи
\fi

\subsubsection{Четвертая версия. Citta}

Не обязательно искать решение загадки Китай-города на Украине! Были и на Руси
другие крепости с таким названием! Постниковский летописец под 1535 годом
сообщает:

\begin{leftbar}
  \begingroup
    \em\Large\bfseries\color{blue}
«На озере на Себеже сделали земляной город Китай».
  \endgroup
\end{leftbar}

Некоторые историки сообщают также, что укрепления, возведенные в 1536 году в
Пронске, тоже назывались Китай-городом.

Эти укрепления были созданы почти одновременно. Что их объединяет? Технология:
земляной вал и имя зодчего – Петрок Малый. Именно он построил все три крепости.
«Архитектон» на службе у великого князя приехал на Русь из Италии.

Весьма вероятно, интересующее нас название произошло от итальянского citta
«город» (в XVI веке это слово могло произноситься через «к»). Эту версию
считает самой вероятной Ян Рачинский, современный специалист по московской
топонимике.

\ifcmt
pic https://avatars.mds.yandex.net/get-zen_doc/3484848/pub_5f78613671c44f0829e3cec6_5f7865b08d3ae5589b28ea38/scale_2400
caption Из трактата Альбрехта Дюрера по фортификации. 1527
\fi

\subsubsection{Пятая версия. Cita urbs}

Еще одна правдоподобная гипотеза (Игоря Кондратьева) связывает китайгородскую
стену с латинским выражением cita urbs «быстрый/скорый город». Итальянский
архитектор наверняка знал латынь.

Земляной вал с каменными «одеждами» – это последнее слово западной фортификации
начала XVI века. Каменно-земляные укрепления меньше страдали от артиллерийского
огня, чем классические стены из камня. Сначала возводился вал, потом стена –
так поступили и в Москве. Вспомним, что этот оборонительный пояс получил имя
как раз на этапе строительства вала. Насыпать вал – быстрее, чем построить
стену.

Другое старорусское название «Скородом» (имя Земляного города) кажется калькой
с латинского выражения cita urbs.

Итак, по наиболее правдоподобным версиям, название «Китай-города» связано с
либо с итальянским словом citta «город», либо с латинским прилагательным citus,
-a, um («быстрый, скорый»).

%%\begin{itemize}

%\iusr{телек}

%Укрепления из киты строили на руси тысячи лет, а назвали по латыни. И может
%корзины начали делать как раз по технологии стен, а как же плетень которым
%огораживали на украине жилье из той же киты.

%\iusr{Александр ЗАХАРОВ}

%Читал версию "кий тай" -"высокая стена" на протославянском.

%\iusr{Елена М.}

%Интересные версии. Я думала, что такое необычное название район в Москве
%получил из-за плотного заселение.

%В 70-е годы прошлого столетия район новостроек на юго-западе Ленинграда в
%народе в шутку называли Китай-город именно по этой причине.

%Да, сразу же предупреждаю - я знаю, что Китай-город в Москве название
%историческое. Островского читала, "Женитьба Бальзаминова" один из любимейших
%фильмов. Замоскворечье и Китай-город для меня названия привычные

%\end{itemize}
