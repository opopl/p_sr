% vim: keymap=russian-jcukenwin
%%beginhead 
 
%%file letters.kutnjakov_sergij
%%parent letters
 
%%url 
 
%%author_id 
%%date 
 
%%tags 
%%title 
 
%%endhead 

Добрий день, пане Сергію!

Мене звати Іван, я киянин, програміст і живу і працюю в Києві. Багато зараз
читаю про Маріуполь, у мене є вже також декілька книжок про Ваше неймовірне
Місто. На жаль, до війни не встиг побувати у Маріуполі, так що мене...  із
Маріуполем мене пов'язує тільки пам'ять про те, що я колись сів на потяг
Київ-Маріуполь, і потім зійшов на проміжній станції... Давно вже це було...
Але... Я близько приймаю до серця, те, що сталось із Маріуполем і
маріупольцями. Страшна трагедія... яка забрала життя близьких, друзів, дітей,
батьків, розкидала маріупольців по всьому світу...  Тим не менш, я абсолютно
впевнений, що кращі часи для Маріуполя ще попереду, і впадати у відчай точно не
треба...  Так, як я вже сказав, я програміст. У вільний від роботи час я також
займаюсь систематизацією різних публікацій про війну, у мене є власний проект
Літопису Війни, яким я займаюсь - поки що сам - власне кажучи, я записую та
збегіраю різні важливі пости на фейсбуці, потім розкидаю їх по авторам та
темам. Технологія, яку я використовую, називається LaTeX (нею користуються
фізики та математики для запису своїх наукових публікацій, це є де-факто
стандарт у науковому світі). Так от.  Я зберігаю у печатному вигляді те, що я
читаю, те, що я вважаю, потрібно зберігати...  Про Маріуполь я теж зберігаю
останнім часом, і доволі багато. І про довоєнний мирний Маріуполь, і про
жахіття війни. У мене є телеграм-канал https://t.me/kyiv_fortress_1, і там
багато вже чого викладено, так як також у цьому фейсбук-акаунті, також я
поступово викладаю - звісно - завжди вказуючи авторів та дату публікації
оригіналу - на https://archive.org - це є сайт Машина Часу - Архів Інтернету -
присвячений збереженню взагалі всього в Мережі - посилання на мій акаунт там
https://archive.org/details/@kyiv_chronicler (chronicler = літописець, тобто
значить Київський Літописець).

Нащо все це робиться... знаєте, конкретно щодо Маріуполя... Маріуполь фізично
то вбили...  але Маріуполь не помер. Душа Маріуполя не померла і жива.
Маріуполь залишився в Духовному Просторі - у пам'яті, фото, уцілілих книжках,
старих публікаціях, в живих людях - маріупольцях - розкиданих зараз по всьому
світу...  Але... якщо не записувати, якщо систематично не займатись Маріуполем
в духовній та культурній сфері - якщо цим не займатись... то є ризик... що
Маріуполь помре духовно ...  і це буде вже назавжди. А духовна смерть - це ще
страшніше, ніж смерть фізична. І зокрема, як деякі ознаки цього процесу... Я
зараз живу в Києві. І виявляється, дуже мало книжок про Маріуполь!  Обійшов
десятки книгарень - топових в центрі - поїхав на Петрівку - нема практично
книжок про Маріуполь!  А про це я пишу більш докладно у своєму пості від
03/03/2023
https://www.facebook.com/ivan.ivan.kyiv/posts/pfbid02X7PFBDFiFcNY9Ge7mY2XDvykevGa8KWRUyXKK5MPFsH8NyFSstBCeZPb2GoCQ4mkl

Тому... я всім цим і займаюсь. Маріуполь повинен бути звільнений не тільки
воєнним шляхом, але також і культурним, інформаційним, духовним шляхом, він
повинен на повну повернутись в інформаційний та культурний простір України. 

З повагою,

Іван.

%Доброе утро, Александр! Как Ваши дела? С наступающим Вас светлым праздником
%Пасхи! Я тут сделал пару архивчиков по Мариуполю, думал, Вам будет интересно
%(1) по группе Мариуполь довоенный, 72 страницы, выложена здесь у меня на
%страничке в альбомах, ссылки там указаны на загрузки файлов (2) в основном по
%культурной жизни Мариуполя, но есть также и посты с прерасными фотографическими
%работами, вот тут -
%https://acrobat.adobe.com/id/urn:aaid:sc:EU:4ecce413-b27c-4216-9cd7-1fce3a3bb09c
%и также тут
%https://mega.nz/file/V7JUga6T#dK6wXslj5ZI4rNaTcUSG7R1M93QH3oa54O5c5AK159Q как
%оно все выглядит, смотрите скрины ниже (файлы pdf большие !!!, нужно подождать,
%пока загрузятся полностью для просмотра и скачивания - там есть оглавление и
%возможность передвигаться по ссылкам внутри, также индексация по авторам)

Доброго ранку, пане Сергію! З наступаючим Вас світлим святом Великодня!

Я тут зробив декілька збірок по Маріуполю, думав, це Вам буде цікаво. (1) по групі Маріуполь
довоєнний, 72 сторінки, вже викладена і тут теж (2) по культурі та мистецтву, але також є фотографічні роботи,
ось тут https://acrobat.adobe.com/id/urn:aaid:sc:EU:4ecce413-b27c-4216-9cd7-1fce3a3bb09c
і також тут https://mega.nz/file/V7JUga6T#dK6wXslj5ZI4rNaTcUSG7R1M93QH3oa54O5c5AK159Q
як воно все виглядає, див. скріни нижче (файли pdf дуже великі!!! треба зачекати, поки завантажиться. Крім того,
так є можливість пересуватись всередині по лінкам, та є зведений покажчик авторів)
