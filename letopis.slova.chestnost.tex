% vim: keymap=russian-jcukenwin
%%beginhead 
 
%%file slova.chestnost
%%parent slova
 
%%url 
 
%%author 
%%author_id 
%%author_url 
 
%%tags 
%%title 
 
%%endhead 
\chapter{Честность}

%%%cit
%%%cit_head
%%%cit_pic
%%%cit_text
\emph{Чесно} говорити про свої думки, почуття і проблеми – це ознака зрілості. Мати
сміливість висловити свої складні чи незручні запитання, вміти їх сформулювати
і бути готовим почути важкі відповіді. Натомість в українців досі помітне те,
що Джадт називає синдромом стародавнього мореплавця: коли хапаєш чужинця за
плече й наполягаєш на розповіданні й переповіданні йому трагічної національної
історії.  Тому ми досі так глибоко в минулому, що не можемо нічого розгледіти в
теперішньому. Навіть дна під носом. Бути \emph{чесним} зі собою – це якраз розуміти,
що відкладати важливі розмови є найбільшою безвідповідальністю. Особливо коли
повітря дедалі менше
%%%cit_comment
%%%cit_title
\citTitle{Чесно мислити про власну країну - ZAXID.NET}, 
Володимир Молодій, zaxid.net, 22.06.2021
%%%endcit

