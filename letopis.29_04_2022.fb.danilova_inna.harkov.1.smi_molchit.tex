% vim: keymap=russian-jcukenwin
%%beginhead 
 
%%file 29_04_2022.fb.danilova_inna.harkov.1.smi_molchit
%%parent 29_04_2022
 
%%url https://www.facebook.com/inna.danilova.547/posts/2599212783543966
 
%%author_id danilova_inna.harkov
%%date 
 
%%tags 
%%title ПОЧЕМУ СМИ МОЛЧИТ!?
 
%%endhead 
 
\subsection{ПОЧЕМУ СМИ МОЛЧИТ!?}
\label{sec:29_04_2022.fb.danilova_inna.harkov.1.smi_molchit}
 
\Purl{https://www.facebook.com/inna.danilova.547/posts/2599212783543966}
\ifcmt
 author_begin
   author_id danilova_inna.harkov
 author_end
\fi

%\ii{29_04_2022.fb.danilova_inna.harkov.1.smi_molchit.eng}

ПОЧЕМУ СМИ МОЛЧИТ!?

Харьков держится из последних сил, благодаря своему духу и желанию - жить! 

Благодаря нашему меру Игорю Терехову и главам администраций районов, которые
пытаются сохранить жизни людей и сердце города, обеспечивая людей теплом,
светом, водой, едой и укрытием. 

\ii{29_04_2022.fb.danilova_inna.harkov.1.smi_molchit.pic.1}

Но почему-то в сми перестали говорить о Харькове, который бомбят с первого дня
войны. Не было ещё ни одного дня, что бы не взрывались бомбы и была тишина.
Каждый день разбиваются дома, парки, улицы, здания и гибнут люди. 

Харьковчане остаются без крыши над головой и средств для их существования. Куда
им идти? Как жить им? Чем кормить своих детей? 

Работы нет. А если бы и была, то добраться под пулями на неё невозможно.
Транспорт не ходит. Лекарств в аптеке мало. Выбор скудный. И то что есть,
невероятно дорого стоит. У людей нет денег. 

Харьков умирает. Почему об этом не говорят? 

Мы все переживаем за Одессу и Киев, но в Харькове так каждый день. Бомбежка и
разрушения города Харькова не прекращается, а наоборот с каждым днём только
увеличивается. По 30-40 ракет и бомб летит в город КАЖДЫЙ ДЕНЬ! 

Конца и края этому не видно. 

Если так будет продолжаться дальше, от города ничего не останется. Это будет
выжженная земля, с обгоревшими и опустевшими зданиями. 

О Харькове нужно говорить. Это не просто город герой, это душа нашей страны,
это первая столица Украины. 

Господи сохрани город Харьков @igg{fbicon.hands.pray}
