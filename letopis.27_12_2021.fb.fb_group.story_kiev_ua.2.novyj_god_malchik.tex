% vim: keymap=russian-jcukenwin
%%beginhead 
 
%%file 27_12_2021.fb.fb_group.story_kiev_ua.2.novyj_god_malchik
%%parent 27_12_2021
 
%%url https://www.facebook.com/groups/story.kiev.ua/posts/1827671684096291
 
%%author_id fb_group.story_kiev_ua,olejnikov_maksim
%%date 
 
%%tags deti,kiev,malchik,novyj_god,simvol,sssr
%%title Новорічний символ - маленький хлопчик
 
%%endhead 
 
\subsection{Новорічний символ - маленький хлопчик}
\label{sec:27_12_2021.fb.fb_group.story_kiev_ua.2.novyj_god_malchik}
 
\Purl{https://www.facebook.com/groups/story.kiev.ua/posts/1827671684096291}
\ifcmt
 author_begin
   author_id fb_group.story_kiev_ua,olejnikov_maksim
 author_end
\fi

В радянські часи візуальним символом Нового року був маленький хлопчик,
вдягнутий у якусь подобу лижного костюму, з цифрами на грудях (іноді на шапці),
які вказували на порядковий номер прийдешнього року. Поряд із Дідом Морозом і
Снігуронькою він неодмінно брав участь у тодішніх новорічних святах і виставах.

\begin{multicols}{2} % {
\setlength{\parindent}{0pt}

\ii{27_12_2021.fb.fb_group.story_kiev_ua.2.novyj_god_malchik.pic.1}
\ii{27_12_2021.fb.fb_group.story_kiev_ua.2.novyj_god_malchik.pic.1.cmt}

%\ii{27_12_2021.fb.fb_group.story_kiev_ua.2.novyj_god_malchik.pic.2}
%\ii{27_12_2021.fb.fb_group.story_kiev_ua.2.novyj_god_malchik.pic.2.cmt}

\end{multicols} % }

У 1959 році в дитячому садку № 158 мені випало бути отим самим «новорічним
символом». Бо на своє щастя я єдиний з хлопчиків у групі мав практично готовий
для цього сценічний костюм: китайський комплект рудо-цегляного кольору –
штанці, светр, шапочка з балабоном плюс білий шарфик з китицями, тобто у ньому
міг виглядати на новорічному святі майже так, як тоді Новий рік зображали на
поштових листівках. Батько бронзовою (!) фарбою намалював на стрічці білої
тканини цифри «1960», мене тією стрічкою обперезали навскоси від плеча до пояса
і в такому вигляді в образі Нового року я вітав своїх товаришів по дитсадку,
обіцяючи їм, собі, присутнім батькам і вихователям  світле майбутнє і всілякі
гаразди у новому, прийдешньому 1960-му році. 

\ii{27_12_2021.fb.fb_group.story_kiev_ua.2.novyj_god_malchik.pic.2}

Шкода, але ніхто тоді мене не фотографував, і документального підтвердження
цього виступу в моєму архіві немає. Але є фото з наступного новорічного свята у
грудні 1960 року. 

На те свято костюми нам видавали в дитсадку «казенні», дівчатам – сніжинок,
хлопцям – клоунів, яких чомусь називали «петрушками»: двокольоровий комбінезон
з марлевими коміром і манжетами, з величезними гудзиками, до нього ковпачок і
два металеві брязкальці. Але була проблема взуття. Вимагали до костюму білі
капці, а де їх тоді візьмеш? Те, що пізніше називали «чешки», тобто м’яке
взуття для танців і гімнастики, було ще невідомим чи недоступним – точно не
скажу за давністю часу. Тому мама сама пошила мені білі капці, добре пам’ятаю
процес зняття мірок і виготовлення їх буквально напередодні свята. Здається, з
тієї ж тканини, що і моя новорічна стрічка роком раніше, щось типу простирадла...
Взути їх можна було лише раз, бо вони якось так кріпилися на нозі (щоб не
спадали під час танцю), наче зашивалися (точно вже не пригадаю), і зняти їх
можна було, лише розпоровши шов на п’ятці...

\ii{27_12_2021.fb.fb_group.story_kiev_ua.2.novyj_god_malchik.cmt}
