% vim: keymap=russian-jcukenwin
%%beginhead 
 
%%file 04_12_2020.news.ru.lenta_ru.4.ukraine_vaccine_non_existent
%%parent 04_12_2020
 
%%url https://lenta.ru/news/2020/12/04/ukraine/
 
%%author 
%%author_id 
%%author_url 
 
%%tags 
%%title Украина назвала российскую вакцину от коронавируса несуществующей
 
%%endhead 
 
\subsection{Украина назвала российскую вакцину от коронавируса несуществующей}
\label{sec:04_12_2020.news.ru.lenta_ru.4.ukraine_vaccine_non_existent}
\Purl{https://lenta.ru/news/2020/12/04/ukraine/}

Министр здравоохранения Украины Максим Степанов назвал несуществующей
российскую вакцину от коронавируса, передает «Корреспондент.net».

\index[rus]{Коронавирус!Украина!Вакцина Спутник V, несуществующая, Степанов}

\ifcmt
pic https://icdn.lenta.ru/images/2020/12/04/09/20201204095759947/pic_82d723091fc50fdadbc68610d0620e05.jpg
cpx Фото: Александр Демьянчук / ТАСС
\fi

«Я говорил неоднократно, что не существует на сегодняшний день российской
вакцины. Когда говорят, что зарегистрировали вакцину, которая не прошла третью
стадию [испытаний], то не о чем говорить. Третья стадия — самая большая, при
которой испытывается вакцина, в том числе с точки зрения ее безопасности», —
сказал Степанов.

Он сообщил, что Минздрав уже провел встречу с дипломатами США относительно
закупки вакцин Pfizer и Moderna.

Ранее в Москве открылась\Furl{https://lenta.ru/news/2020/12/04/priv/} электронная запись на прививку от коронавируса.
Получить вакцину «Спутник V» могут граждане, которые контактируют с большим
количеством людей: медики, учителя, сотрудники городских служб.

Президент России Владимир Путин заявил, что в ближайшее время объем
произведенной в стране вакцины достигнет двух миллионов доз, что позволяет
начать масштабную вакцинацию препаратом «Спутник V».

Данный препарат первым из трех российских вакцин прошел регистрацию. Он был
разработан в Центре имени Гамалеи, клинические испытания состоялись в июне и
июле.
