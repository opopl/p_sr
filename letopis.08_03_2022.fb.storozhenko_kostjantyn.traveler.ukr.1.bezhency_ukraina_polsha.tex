% vim: keymap=russian-jcukenwin
%%beginhead 
 
%%file 08_03_2022.fb.storozhenko_kostjantyn.traveler.ukr.1.bezhency_ukraina_polsha
%%parent 08_03_2022
 
%%url https://www.facebook.com/kostyantyn.storozhenko/posts/5270932546274980
 
%%author_id storozhenko_kostjantyn.traveler.ukr
%%date 
 
%%tags 
%%title Итак, актуально  для беженцев, пересекающих границу Украина - Польша
 
%%endhead 
 
\subsection{Итак, актуально  для беженцев, пересекающих границу Украина - Польша}
\label{sec:08_03_2022.fb.storozhenko_kostjantyn.traveler.ukr.1.bezhency_ukraina_polsha}
 
\Purl{https://www.facebook.com/kostyantyn.storozhenko/posts/5270932546274980}
\ifcmt
 author_begin
   author_id storozhenko_kostjantyn.traveler.ukr
 author_end
\fi

Итак, актуально  для беженцев, пересекающих границу Украина - Польша: 

по трафику из Украины каждый день ситуация разная, нужно смотреть здесь
\url{https://www.facebook.com/zahidnuy.kordon} Бесплатные автобусы во Львове
могут менять маршрут в зависимости от ситуации на границе и трафика. Во всех
пунктах есть центры медицинской помощи, детское питание, памперсы и средства
гигиены!

Корчёва—Краковец: беженцев по прибытию на автобусах размещают и направляют в
большой центр, там два ангара.  В них тепло, много еды, есть раскладные
кровати, тоже много. Отсюда развозят  по всей  Польше на автобусах тех, кто без
своей машины!   В другие страны Европы тоже рейсы есть, но не так много. Часто
перевозкой и поселением занимаются волонтеры.

\ii{08_03_2022.fb.storozhenko_kostjantyn.traveler.ukr.1.bezhency_ukraina_polsha.pic.1}

Будомєж-Грушів: в основном  здесь проходят автомобили. Центр развернут сразу у
дороги. Эта точка только для транзита. Есть еда, вода, одежда, средства
гигиены, коляски. Но условий для отдыха нет. Возможно днем развозят автобусы.
Мы были поздно, забирали людей, автобусов не было!

Пшемысль—Мостиська: Жд, автобусы, автомобили. Трафик очень большой! Жд вокзал
уже не справляется с перевозками. Очереди за билетами  бывают на 2-4 часа! По
Польше зачастую передвижение на поезде. Билет бесплатный продают только по
Польше, без места! Поэтому например в Германию сложно добраться! Нужно брать
билет на прямой поезд, до Берлина например. Садиться с билетом по Польше, и
когда на немецкой территории в вагон заходит контролер, просить у него
бесплатный билет на вторую половину пути по Германии. На вокзале очень сложно
со сном, нет условий, туалета всего два! Зарядить телефон практически негде.
Поселиться тоже, все уже забито!

Я с \href{https://www.facebook.com/evgen.rafalovsky}{Евгений Рафаловский}
сейчас пытаюсь найти просторное жилье для расселения и адаптации беженцев. В
Пшемышле это сейчас главная проблема, люди на вокзале ночуют, на полу. Мы также
в поисках большей машины, цены взлетели, сложно такую достать по адекватной
цене!

Да, для всех очень актуально! Если едите с наличной гривной, то лучше положите
ее в Украине на карту. Обмен налички в Пшемышле 125-160 гривен за одно евро!

Ну и напоследок: у нас сейчас на 8 марта все занесло  снегом! Но мы верим, что
весна прийдет, в войне мы победим и будет Мир!

\ii{08_03_2022.fb.storozhenko_kostjantyn.traveler.ukr.1.bezhency_ukraina_polsha.cmt}
