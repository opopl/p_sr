% vim: keymap=russian-jcukenwin
%%beginhead 
 
%%file 16_01_2022.stz.news.ua.hvylya.1.anatomia_vraga.9.ukrainskie_stolpy_identichnosti
%%parent 16_01_2022.stz.news.ua.hvylya.1.anatomia_vraga
 
%%url 
 
%%author_id 
%%date 
 
%%tags 
%%title 
 
%%endhead 

\subsubsection{Украинские столпы идентичности}

С момента своей успешной экспансии на левобережную Украину в середине XVII
века, Московское государство добавило в свою стратегию экспансии
\enquote{единоначальной стабильности} идею \enquote{воссоединения русских земель временно
оторванных от пуповины} западными моделями \enquote{хаоса свободы}. Эту идею Кремль
реализовывал последовательно начиная от разделов Польши Екатериной Второй в
конце XVIII века, вплоть до раздела польского государства, осуществленного
Сталиным и одним \enquote{австрийским художником} в конце 30-х годов XX века, когда в
политическое пространство СССР вошли Западная Украина и Западная Беларусь.

Для нас сегодня крайне важно понимать, что с точки зрения врага, территория и
население Украины - это \enquote{законно завоеванный}, и \enquote{временно утерянный},
\enquote{объединительный} смысл России, который является для Москвы сакральным в
контексте самого бытия государства. Киев является одним из столпов, наряду с
Москвой, Владимиром, Новгородом, Минском и Питером, которые легитимизируют
\enquote{истоки идентичности}, \enquote{общую историю} и цельность основы восточнославянской
имперской конструкции. То есть - \enquote{один народ}, как любит утверждать дедушка
Путин. Пока Киев является столицей суверенной Украины, Кремль продолжит войну
всеми доступными мерами и не остановится ни перед какими потерями в процессе
экспансии вплоть до угроз \enquote{ограниченного} применения ядерного оружия. Пусть у
Вас не будет в этом даже тени сомнения. Поэтому Лавров после раундов
переговоров с США, НАТО И ОБСЕ акцентировал месседж о том, что:

\enquote{Россия никогда не угрожала украинскому народу, в отличие от Зеленского и его
соратников} (с)

Месседж Лаврова - стандартный имперский прием, когда русские пытаются
стравливать некий \enquote{страдающий народ} с \enquote{преступным режимом}.
Этой фразой МИД РФ, по сути, выделяет \enquote{украинский народ} в отдельный
объект, который необходимо \enquote{защитить от хаоса}, лишая его возможности
иметь свою \enquote{неправильную} государственность. Угрожать же Кремль, может,
якобы, лишь \enquote{Зеленскому и соратникам}, то есть \enquote{нелегитимной},
\enquote{не сакральной}, \enquote{неприродной} организации власти.

Осуществляя комплексный процесс уничтожения украинской субъектности, применяя
все доступные меры в современной войне, Путин, последовательно, воюет не с
украинцами, как с этносом, он ведет войну с даже гипотетической возможностью
украинцев стать (!) политической нацией (!), имеющей свою, отличную от России,
общественно-политическую организацию. Путин продолжает реализовывать старую
стратагему выживания Москвы - он воюет на территории Украины с \enquote{хаосом
свободы}, продвигая на запад экспансию \enquote{единоначальной стабильности}.

Давайте научимся с Вами говорить правду хотя бы самим себе, даже, если она нам
неприятна. Современная Украина, как модель, объективно, является
постколониальным государством, не имеющим своих долгосрочных традиций
державности, с крайне низкой, деградировавшей общественной культурой и, как
следствие - катастрофически примитивной формой политической и государственной
организации, основанной на перераспределении оставшихся ресурсов. Именно по
этой причине в Украине отсутствует стратагема классического государственного
развития и своей государственной защиты. Поэтому, в настоящий исторический
момент, Кремль имеет не только теоретическую, но и организационно-практическую
возможность осуществлять десубъективизацию Украины для последующей
территориальной экспансии с очень высокой вероятностью успеха.

Таким образом - пределами территориальной экспансии Москвы или экспансии ее
геополитического влияния - являются:

1. Уровень общественно-политической организации государственных моделей
противников

2. Способность государственных элит противников проявлять инициативу и
выстраивать долгосрочные стратегии защиты

3. Осознанная обществом противников необходимость платить ресурсную цену за
суверенитет. В том числе, в измерении человеческих жизней своих граждан.

Нынешние страны коллективного Запада, несмотря на относительно высокий уровень
общественно-политической организации, утратили мотивацию к экспансии смысловых
идей, являющихся основой их общественно-политического договора. Отсутствие
желания к экспансии не дает возможности выработать эффективную стратегию
защиты, и как следствие порождает утрату инициативы. Кремль пользуется этим, в
агрессивной форме навязывает свои инициативы \enquote{реальной политики} и
заставляет своих \enquote{партнеров} в Вашингтоне действовать в режиме реакции,
блефуя и шантажируя западных политиков \enquote{военно-техническими мерами}.
Реактивная \enquote{тактика} имитации \enquote{защиты} и боязнь военных мер,
которую сегодня демонстрируют в Вашингтоне, стратегически является
контрпродуктивной и, с большой степенью вероятности, может привести к
существенной сдаче позиций США, как минимум, в Европе.
