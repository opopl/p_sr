% vim: keymap=russian-jcukenwin
%%beginhead 
 
%%file 06_02_2022.fb.baumejster_andrej.kiev.filosof.1.chelovek_degradacia.cmt
%%parent 06_02_2022.fb.baumejster_andrej.kiev.filosof.1.chelovek_degradacia
 
%%url 
 
%%author_id 
%%date 
 
%%tags 
%%title 
 
%%endhead 
\zzSecCmt

\begin{itemize} % {
\iusr{Виктор Каган}

Слухи о моей деградации преувеличены. Ваш Ecce Homo

\begin{itemize} % {
\iusr{Андрей Баумейстер}
\textbf{Виктор Каган} даже не сомневаюсь  @igg{fbicon.smile} 

\iusr{Andriy Kryvtsun}
\textbf{Viktor Kagan} Ессе Home - это же Христос? Или так называют человека вообще?

\iusr{Maria Lypych}
\textbf{Виктор Каган} а що це означає? Поясніть, будь-ласка.

\iusr{Виктор Каган}
\textbf{Andriy Kryvtsun} се - человек. Слова Пилата о Христе, да

\iusr{Andriy Kryvtsun}
\textbf{Viktor Kagan} ну, Христос вряд ли сможет деградировать, да  @igg{fbicon.smile} 

\iusr{Andriy Kryvtsun}
\textbf{Maria Lypych} \url{https://en.wikipedia.org/wiki/Ecce_homo}

\iusr{Maria Lypych}
\textbf{Andriy Kryvtsun} я це знайшла сама, дякую!
Я питала Виктор Каган що саме означає Ваш коментар?

\iusr{Maria Lypych}
\textbf{Виктор Каган} чомусь я відчуваю у Вашому коментарі скриту агресію

\end{itemize} % }

\iusr{Наталья Доброер}

Да, меня тоже напугало, когда сын подсел на это. Со скандалом удалил. Вроде
сейчас сам смысла не видит. Надеюсь

\iusr{Olga Atman}

Как мы с вами солидарны.  @igg{fbicon.hands.shake}  Года 4 назад, когда я всем это говорила, фактически
слово в слово, мне отвечали тоже самое, и отвечают. Но с тех пор, как я
познакомилась с вами, я могу ссылаться еще и на вас  @igg{fbicon.face.smiling.eyes.smiling}  От этого мое
\enquote{субъективное мнение} выглядит не столь уж субъективно.

\iusr{Анатолий Сергиенко}

Число Данбара ограничивает размер племени до 150 ч-ек. У каждого человека в
мозгу - модель племени, в которой есть место вождю, шаману, мастеру, родителям,
родственникам и т.д. Средства массовой информации \enquote{встраивают} в эту модель
новых персонажей, которые стоят на том же уровне, что и вождь. И это ломает
психику и нормальные общественные отношения.

\iusr{Alexey Burov}

Есть легкие наркотики, вроде вина или сигарет, с которыми в общем можно жить. А
есть тяжелые, убивающие личность в самом нежном возрасте. Вот тикток относится
ко вторым — и об этом, прежде всего, важно знать. Спасибо за твой удар в
колокол, Андрей!

\iusr{Nataly Rostovtseva}
Как прекрасно, что Вы есть )))

\iusr{Александр Воструев}

Как хочется, чтобы Ваша мудрость была услышана и хоть чуть-чуть осознана...

\iusr{Alexander Chaly}

А что, если это как раз нормально. Не в логике желаемого, но необходимого. Само
становление человечества с неизбежностью приоткрывало новые возможности и
угрозы. Распад нашей \enquote{разумной части} конечно пугает и разочаровывает с одной
стороны, но с другой, наиболее рельефно проявляет важность и ценность
исчезающих сторон действительности. Девальвация значения религии делает выбор
Бога человеком более осознанным (пусть этих людей и станет меньше), девальвация
знания делает Знание более ценным и глубоким в глазах тех, кто способен его
ценить. Девальвация сложности, у способных ее оценить, порождает естественное
стремление в нее углубляться. Деградация одних становится стимулом роста
других, пусть и значительно меньшего количества людей. Выбирать нам.


\iusr{Таня Максимчук}

Тик-ток это проявление основных инстинктов )) Оказывается, человека очень
увлекает проявление и наблюдение за основными несложными инстинктами. Например,
кроме рефлекторных движений и смены эмоций на лице, очень популярно
заглатывание еды на камеру.  И, думаю, инстаграм и тик-ток - признак наличия
свободного времени.

\iusr{Olga Atman}
\textbf{Таня Максимчук} именно так) мартышечная составляющая очень сильна))

\iusr{Александр Анохин}

Философу не трудно сказать длинную речь за минуту - просто обиженных в таком
случае будет очень много) и тогда, предъявы понесут некие иные черты и характер
 @igg{fbicon.beaming.face.smiling.eyes} 


\iusr{Andriy Kryvtsun}

Выходит, к разработчикам соцсетей нужно относиться так же, как производителям
алкоголя? Запретить рекламу, не допускать до их продукции людей, не достигших
возраста взросления.


\iusr{Людмила Ярошенко}
\textbf{Андрей Олегович}, 

спасибо, что разложили все по полочкам). Сейчас от школьников можно услышать
«мысль» — зачем учиться, лучше в блогеры пойти... Деградация посредством ИКТ —
проблема, в ООН уже поднимали вопрос об агрессии в сетях, похоже все намного
сложнее...

\iusr{Lara Alia}

Спасибо за обзор, Андрей. Я наверное отстала от жизни, но ни разу не заходила в
тик ток. Более того меня пугает тенденция поздравлений с праздниками среди
близких друзей. Мало кто имеет такую роскошь, чтобы встретится или позвонить и
сказать что-то особенное человеку. Все заменено на массовую рассыпку каких-то
открыток и видео, за которыми не слышно ни голоса, ни настроения и т.а.

\iusr{Natalka Sereda}

Лише вчора мені говорила подруга, що психолог (відомий) вважає, що це
нормально! Ну, і всі радо повторюють: «нормально». Ось така нормальність з уст
дитячих психологів.

\iusr{Людмила Куклева}
Все - это некоторые. Некоторые это немногие, СКа

\iusr{Анатолий Сергиенко}

2000 лет назад книги запоминали томами. После внедрения книгопечатания
запоминали меньше - например, поэмы. С появлением калькуляторов перестали
запоминать таблицу умножения. С развитием Интернета перестали запоминать всё.

\begin{itemize} % {
\iusr{Андрей Хорсев}
\textbf{Anatoliy} , вы с большой уверенность совершаете утверждения которые невозможно проверить.
Например про запоминание книг томами.
Зато можно с уверенностью утверждать что до книгопечатания люди умели писать и читать.
Например Александрийская библиотека существовала за 1000 лет до изобретения книгопечатания.

\iusr{Анатолий Сергиенко}
\textbf{Андрей Хорсев} Новый Завет именно заучивали на память. Платона заучивали. Платон в римских школах был всесто букваря.

\iusr{Андрей Хорсев}
\textbf{Anatoliy} , сам Новый Завет состоит из почти 30 книг, которые как раз не сложно запомнить и современному человеку.
\end{itemize} % }

\iusr{Igor Chernega}

Подкасты Джо Рогана, например, где люди по 3-4 часа говорят на интересные и
глубокие темы. И это огромная аудитория. Деградация относительно какого
времени?

\begin{itemize} % {
\iusr{Andriy Kryvtsun}
\textbf{Igor Chernega} где послушать?

\iusr{Igor Chernega}
\textbf{Andriy Kryvtsun} можно хитрить с впн, если хотите из Украины слушать свежие (но так делать нельзя Ай-Ай). Ранние эпизоды можно на YouTube

\iusr{Andriy Kryvtsun}
\textbf{Igor Chernega} в смысле впн? Он разве заблокирован?

\iusr{Igor Chernega}
\textbf{Andriy Kryvtsun} 

к сожалению, его подкаст недоступен в Украине. Так решили Spotify. Комьюнити
уже пару лет репортит эту проблему, но она так и не решена до сих пор, на
сколько мне известно. Но, есть слухи, что Джо опять сказал что-то не то насчет
протестов в Канаде и по Ковиду в целом. Так что, есть шанс, что его попрут и он
уйдёт на более доступную площадку

\iusr{Andriy Kryvtsun}
\textbf{Igor Chernega} понял, спасибо!

\iusr{Andriy Kryvtsun}
\textbf{Igor Chernega} уже и на Спотифае его нет

\url{https://www.theverge.com/2022/2/6/22921203/spotify-joe-rogan-episode-removal-internal-memo}

\iusr{Андрей Хорсев}
\textbf{Igor}, у Джо хорошие подкасты, но их не стоит путать с реальными знаниями.

\end{itemize} % }

\iusr{Veronika Romanova}

Очень хочется верить, что за деградацией будет следовать развитие. Вспоминаю,
как говорил 20 лет назад замечательный мой преподаватель истории искусств
Клепалов Б.Я., мол, «сейчас молодежь находится на нижайшем уровне развития...».
Так это были тогда \enquote{цветочки}. Сегодняшнюю ситуацию он уже не увидит, к
сожалению...

Как говорится – когда ты упал на дно и вдруг услышал, что тебе с низу
постучали...

Спасибо за видео, Андрей Олегович, очень актуально. Dum spiro, spero!...

\iusr{Ольга Озерян}

Да, мне недавно сказали: «Читать книги - олдскульный формат. Надо читать что-то
более современное и лаконичное»  @igg{fbicon.face.eyes.crossed.out}
@igg{fbicon.dizzy}  Или еще лучше: «Вы не написали в начале поста что это
лонгрид. Читать долго». Деградация очевидна ((((

\end{itemize} % }
