% vim: keymap=russian-jcukenwin
%%beginhead 
 
%%file topics.vojna.my.7.matrica.vragi.ukraincy.copied.1
%%parent topics.vojna.my.7.matrica.vragi.ukraincy.copied
 
%%url 
 
%%author_id 
%%date 
 
%%tags 
%%title 
 
%%endhead 

%17:08:57 23-08-22

Яков Балыхин
Нет такой задницы, через которую евроукры что-то бы ни сделали ― это утверждение, как ни одно другое, идеально подходит для современной Украины. Что они ни делают ― не идут дела. В чём причина? Воздух неправильный? Судьба-злодейка? Почему граждане великой и незалэжной ни от кого державы ни сами не живут хорошо, ни другим не дают? Одна из теорий гласит, что всему виной менталитет: та самая загадочная украинская ментальность, от которой украинцы страдают уже много лет, и чем дальше, тем их страдания сильнее.

Яков Балыхинreplied to ivan
Если о русских говорят, что в драке не помогут, а в войне победят, то об украинцах можно сказать, что они ни в драке не помогут друг другу, ни в войне. Просто потому, что среднестатистический украинец, так уж исторически повелось, старается скрывать свой внутренний мир, убеждения и мысли. Хотя бы просто потому, что это может быть невыгодно или банально опасно. Быть принципиальным человеком в селе, которое сегодня захватили поляки, завтра немцы, послезавтра белые, а затем красные ― очень, знаете ли, вредно для здоровья. Поэтому и выработалась эта ментальная установка ― ни во что не лезть дальше своего огорода. С одной стороны, этих людей можно понять: вот подходит к такому человеку на улице журналист и интересуется его политическими взглядами. На это часто следует ответ: «Вы знаете, я вне политики». Так часто говорят публичные персоны, которые не хотят портить свой бизнес политическими высказываниями. Но с другой стороны, когда в одном месте собираются десятки миллионов таких вот «хатаскрайников», то подобное становится настоящим проклятьем для общества: общества равнодушных приспособленцев, аморфных людей без принципов. Территория, населённая такой вот серой массой, обречена, что, собственно, мы и имеем возможность наблюдать

Яков Балыхинreplied to ivan
Как писал А. Деникин в своих «Очерках русской смуты», украинцы считают, что управлять государством так же легко, как «вскопать огород после пляшки горилки». Вне зависимости от чувств к самому автору очерков не согласиться с ним нельзя. Майдан ― квинтэссенция украинства ― нёс главную мысль о том, что управлять страной легко, нужны просто честные люди из народа. Беда украинского обывателя в том, что он не верит в сильное государство и в сильных государевых людей. Более того, он считает, что ему будет лучше вообще без государства, чем с ним. Отсюда и пресловутый политический лозунг «Геть усих!». Поэтому, когда на горизонте появляется по-настоящему сильный лидер из далёких земель, украинец начинает благоговеть перед ним как перед сверхчеловеком, творящим такое, на что украинец никогда бы сам не решился. Я сейчас не говорю об украинцах, управлявших страной в рамках так называемого Русского мира. Я говорю о тех, кто старается строить собственную державность в рамках мира украинского. Такие ребята всегда и везде имели одну-единственную цель ― нацаревать 100 рублей и убежать. Менталитет украинского обывателя рассматривает власть исключительно как средство к личному обогащению, вне зависимости от высоты должности. Таковы законы хуторского сознания

Яков Балыхинreplied to ivan
Один украинец ― патриот. Два ― партизанский отряд. Три ― партизанский отряд с предателем. Кто-то очень точно подметил подобный феномен. Украинцы предавали всех и вся на протяжении всей своей истории. Причём предавали не только Россию, но и своих, так сказать, «западных партнёров». Не секрет, что Мазепа уже из Турции писал Петру Первому покаянные письма с просьбой простить, обещая выдать ему Карла XII, когда тот после разгрома у Лесной и бегства к османам собирался возвращаться в Швецию. В Великую Отечественную немцы также не могли положиться на украинских коллаборационистов. Последние не были приспособлены ни к чему, кроме карательных операций против гражданского населения, и за всю историю своего существования УПА не отметилась ни одним масштабным сражением с силами советских войск, если не считать боя под Гурбами, когда в апреле 1944-го войска НКВД СССР легко разбили остатки бандеровской банды. Теперь вот США. Американцы плохо знают историю, коль уж связались с евроукрами. Последние, как видим, отразили им карму, доведя сами США до тотального майдана по всей стране и громкого коррупционного предвыборного скандала. Тот случай, когда хвост замахал собакой. Украинцы считают себя честными и благородными, но как-то так получается, что выдвигаемые украинским обществом политические лидеры сплошь оказываются лжецами, прохвостами и предателями.

Яков Балыхинreplied to ivan
Есть такой анекдот, когда украинцу сказали, что он может пожелать что угодно, но только с тем условием, что у его соседа этого будет вдвойне. Тогда тот попросил, чтобы ему выкололи глаз. В этом состоит очередная особенность украинского менталитета: сделать свою жизнь хуже, лишь бы у «москалей» тоже настроение испортилось. Например, депутат Верховной рады Э. Леонов, как-то предложил взорвать все украинские ядерные реакторы, которых он насчитал 15 штук. Сделать это нужно, чтобы, конечно же, насолить России. Скажете, что это благородный поступок, достойный японского самурая-камикадзе? Это могло быть так, да вот только есть мнение, что сам господин Леонов в момент взрывов наверняка предпочёл бы оказаться где-нибудь подальше. Например, в той же России. Украинская ментальность отключает человеку мозг и заставляет «думать» сердцем, ну или ещё одной мягкой частью тела. И в итоге получается плохо в первую очередь для самих украинцев. Россия большая и сильная, не с такими угрозами справлялась, и, так уж случилось, евроукры не могут нагадить своему северному соседу просто так, чтобы без последствий для себя. Любая украинская каверза в адрес России обязательно пагубно отражается на самой Украине. Куда ни ткни, начиная от майдана и торговой блокады Крыма и заканчивая войной на Донбассе, всё в итоге возвращалось евроукрам в виде потерь денег, людей и территорий. Но если вы думаете, что это чему-то научит украинского обывателя, то это вряд ли. Украинцы уже доказали свою неспособность к

Яков Балыхинreplied to ivan
Не съем, так понадкусываю ― в этом состоит одно из ярчайших проявлений украинской так называемой хозяйственной домовитости. Именно ею, домовитостью, и прикрывается самая банальная и разрушительная алчность, укоренившаяся в украинском обществе так глубоко, что непонятно, честно говоря, как и чем её можно вытравить. Украсть для украинца ― это не зазорное занятие, не стыдное и где-то даже благородное. Особенно если воруешь у государства или у большого частного предприятия. Но можно и у соседа. Как там поётся в песне: «На жадину не нужен нож»? Украинцы неоднократно попадали в неприятности, когда, увлечённые богатыми посулами, творили совершенно ужасные вещи, надеясь из одной шкуры получить сорок шапок. Не умея творить малого, они хватаются за великое. То требуют от Китая провести Великий шёлковый путь через них, то собираются завалить весь мир пшеницей, то внезапно находят у себя огромные залежи алмазов, то хотят какие-то космические репарации от соседей. Всё это не корысти ради, конечно. Но в итоге, увлекаемые своей ослепляющей жадностью, евроукры дошли до продажи земли иностранцам. Вспомните об этом, когда вам начнут рассказывать об эффективных украинских собственниках.

Яков Балыхинreplied to ivan
Почему-то в народном фольклоре принято считать украинцев хитрыми: «Когда хохол родился, еврей заплакал» или «Где хохол прошёл, там татарину делать нечего». Естественно, вся эта хитрость сводится к местечковым частнособственническим подвигам, благодаря чему украинцы, мол, живут припеваючи и всё у них есть на столе. Когда-то так действительно было, но это заключение сформировалось в бытность Украинской ССР, когда на Украине жизнь действительно была неплохая. Не верите ― посмотрите на тогдашние темпы прироста населения. Однако с наступлением незалэжности всё встало на свои места. Украинская хитрость, сравнимая в народных баснях с хитростью еврейской, особенно на фоне традиционного русского простодушия, начала давать осечку за осечкой, обнажив свою выдуманную сущность. А король-то голый! И теперь, глядя на ту помойку, в которую превратилась Украина, и на ту нищету, в какую сами себя загнали «хитрые» украинцы, как-то уже не верится в украинскую смекалку. Ведь хитрость настоящая ― это когда никто не знает, что ты хитёр, а когда о твоей хитрости слагают легенды, это уже хитрость сельского дурачка, променявшего мятую сторублёвку на блестящий пятак. Вера в собственную природную хитрость сыграла с украинцами злую шутку, но вряд ли чему-то научила

Яков Балыхинreplied to ivan
Украина ― страна взяток и кумовства. Если какому-то россиянину, которому пришлось решить какой-то вопрос за деньги у себя дома, покажется, что Россия дошла в вопросе коррупции и сутяжничества до крайности, то поезжайте в Киев и спросите: кому на Руси жить хорошо? И вам ответят: тем, кто сидит на взятках. Одна из главных украинских влажных фантазий, являющихся промежуточной фантазией на пути к тому самому вишнёвому саду с пчёлами, является получение должности, на которой не нужно ходить за зарплатой, получая основной источник финансирования из подношений. В итоге за последние 30 лет украинское общество так развратилось, что дача взяток стала обыденностью, несмотря на то, что в тамошнем Уголовном кодексе за провокацию, получение и дачу взятки всё-таки предусмотрено наказание. Встреча с гаишником ― это практически стопроцентное требование взятки: «Ну дай хоть на бензин!» Поход в суд ― практически стопроцентная гарантия того, что от вас начнут требовать если не деньги напрямую для судьи, то шоколадки для секретарей: «Вы нам к чаю ничего не принесли?» Встреча с врачом, от которого нужен результат, ― гарантия того, что от вас напрямую потребуют деньги или поведут дело так, что вы их сами понесёте: «Ну, вы же всё понимаете». Часто приходилось слышать, что подобный подход к делам ― это чисто восточный обычай, однако Украина, есть мнение, даст фору любой восточной стране в этой сфере. Причём к тому, что взятка ― это нормально, украинцев приучают с детства, когда начинают носить деньги учителям или когда последние начинают намекать на это сами. Поэтому не стоит удивляться коррупции в украинском правительстве, поскольку слуги народа вышли из самого народа, так что никаких сюрпризов.

Яков Балыхинreplied to ivan
Украинский обыватель иногда может начать корчить из себя эдакого вольнолюбивого индивидуалиста, потомка каких-то там рыцарей или, как правило, казаков, которые все как на подбор были благородными и смелыми. Настоящие бессребреники, честнее которых нет и не было больше на земле. Умные люди давно это поняли, и теперь, когда им нужно в чём-то убедить украинскую «спильноту», они всенепременно нажимают на это самое «лыцарство», действуя на воспалённое чувство собственного достоинства рядового евроукра, которое как воспалилось в 1991 году, так и не думает успокаиваться. Как там говорил классик: «Не дай боже с холопа пана». Для холопа, которого унижали поколениями польские феодалы, понятие чести и достоинства ― это какие-то эфемерные и недосягаемые вещи. Но опять же, мы вспоминаем ещё одного классика, утверждавшего, что украинцы не могут творить мелкого, поэтому всегда хватаются за великое. В итоге и получается, что этот самый вчерашний украинский холоп хватается за самое великое, чего у него никогда не было, ― за достоинство, и при разговоре о достоинстве он снимет себя последние портки. Именно поэтому умные дяди из спецслужб назвали последний украинский майдан Революцией достоинства, и именно поэтому бараны, похожие на людей, у которых берут интервью украинские телеканалы, на вопрос о позитивных плодах майдана часто отвечают, что они обрели достоинство. И есть мнение, что где-то тут уже заканчивается психология и начинается психиатрия.

Яков Балыхинreplied to ivan
Парадокс, но годы упоительного самовнушения насчёт собственного достоинства нисколько не смогли добавить мужества украинскому обывателю, его общественным лидерам и большим политикам. Современное украинство ― это трусость и страх понести наказание. Поэтому не стоит удивляться такой бытовой шизофрении, когда сейчас ты рыцарь, преисполненный достоинства, а через минуту ты уже стоишь на коленях в грязи у дороги, по которой едет каток, укладывающий асфальт. Последнее ― реальный случай из Львовской области: коленопреклоненная массовая благодарность асфальтоукладчику (пускай и ироническая). Всё-таки столетия крепостничества имеют большую силу в сознании обывателя, чем какое-то там достоинство. Отсюда и хроническая трусость. Первый вал карателей, прибывших на Донбасс, в основном состоял из жителей западных областей. Но, когда их сотнями начали укладывать в землю, резко наступило просветление, и теперь в т. н. АТО (ООС) в основном участвуют простые украинизированные русские, воюющие против таких же русских по ту сторону фронта. Отважные галичане, мнящие себя нравственной элитой страны, прячутся от призыва в армию по подвалам, в лесах и даже за границей. Потому что страхов много, а жизнь одна. Именно поэтому, как уже было сказано ранее, гитлеровцы и использовали бандеровцев только в карательных операциях против безоружных. В настоящем бою они были бесполезны.

Яков Балыхинreplied to ivan
16:53
Ну и, наконец, последняя фирменная черта украинского менталитета ― местечковость. Хуторская, провинциальная местечковость, завязанная на соломе, кизяках, самогоне и сале. Опять же, это благоприобретённое наследие холопской украинской истории. Батраку не интересно искусство, науки и государственные свершения. Идеальная Украина, которую вы можете встретить в современных украинских школьных учебниках, ― это одно большое село, в котором мальчики пашут землю, а девочки варят борщи. Эту идею активно продвигают западные владельцы Украины. Или вы думаете, что стремительно сокращающееся количество школ и учителей на Украине ― это случайность? Вовсе нет: холопы должны уметь считать на пальцах, чтобы знать, сколько кусков сахара нужно положить пану в чай. Этого довольно. И самое печальное в том, что ― как-то так получилось ― подобное оказалось весьма по шерсти украинскому обывателю. Хуторянство ― это естественное состояние украинского сознания. Поэтому Украина, оставшись сама по себе, оказалась неспособна родить хотя бы одного настоящего государственного мужа, потому что из хуторянина может получиться максимум приказчик, но не президент. Кто не верит ― спросите у любого адекватного киевлянина о ползучей хуторизации его города. Аналогичное происходит и в других областных центрах страны: городских вытесняют деревенские, подстраивая города под свой хуторской ум. Поэтому вместо централизованного водопровода приходят уличные колонки, вместо централизованного отопления ― печи, в том числе в многоквартирных домах, а вместо русского языка, на котором говорили и говорят выдающиеся мировые мыслители, приходит мова, которая хорошо подходит для управления коровьим стадом и которая совсем не годится для ракетостроения. И всё бы хорошо, да вот только эти самые хуторяне, переезжающие в города, делают это не потому, что их деревни процветают и они теперь хотят нести умное, доброе и вечное городским жителям. Вовсе нет. Украинский хуторизм ― это не тяга к созданию провинциального рая под соломенной крышей со сливочным маслом. Это бардак, безответственность, невежество, хамство и большая ненависть к городским, которые живут лучше.
