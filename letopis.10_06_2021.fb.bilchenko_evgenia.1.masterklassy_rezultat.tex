% vim: keymap=russian-jcukenwin
%%beginhead 
 
%%file 10_06_2021.fb.bilchenko_evgenia.1.masterklassy_rezultat
%%parent 10_06_2021
 
%%url https://www.facebook.com/yevzhik/posts/3967873593247721
 
%%author Бильченко, Евгения
%%author_id bilchenko_evgenia
%%author_url 
 
%%tags 
%%title БЖ. Практический результат моих поэтических мастер-классов
 
%%endhead 
 
\subsection{БЖ. Практический результат моих поэтических мастер-классов}
\label{sec:10_06_2021.fb.bilchenko_evgenia.1.masterklassy_rezultat}
\Purl{https://www.facebook.com/yevzhik/posts/3967873593247721}
\ifcmt
 author_begin
   author_id bilchenko_evgenia
 author_end
\fi

БЖ. Практический результат моих поэтических мастер-классов.

Мы начинали с молодым автором Надеждой почти с нуля. Три индивидуальных занятия
с ней по 2-3 часа (500 гривен каждое), и у талантливого, но пока недостаточно
мастерски владеющего формой, начинающего поэта появляются свои: дух, слог,
стиль, ярь, такт, темп, ритм, страсть и, главное, умение не бояться работать с
аудиторией. 

У меня есть жесткие правила, которые могут подтвердить мои клиенты, ученики, работодатели: 

\begin{itemize}
  \item а. Я не трогаю чужого мировоззрения. 
  \item б. Я не меняю образно-символический ряд текста. 
  \item в. Я не ломаю размер - метрику (это концентрат темперамента): я просто привожу ритм в соответствие с правилами данного размера. 
  \item г. Я незначительно корректирую структуру и композицию текста, чтобы она не убивала авторский смысл, а подчёркивала и выгодно обыгрывала его. 
  \item д. Я жестко правлю рифму, если она - банальна. 
  \item е. Я никогда не отрываюсь от объективного грунта текста. 
  \item ж. На прикладных кейсах я показываю поэтические и философские теории от Аверинцева и Бахтина до Мандельштама и Гаспарова. 
  \item з. Мы два часа работаем только над отработкой пяти текстов. Работаем на совесть. С привлечением поэтики и эстетики.
  \item к. Иногда я даю домашние задания, но их выполнение - не обязательно. Это не принудиловка - это школа мастерства. 
\end{itemize}

\ifcmt
  tab_begin cols=2

     pic https://scontent-lga3-2.xx.fbcdn.net/v/t1.6435-0/s600x600/198238567_3967815333253547_5230152896983011690_n.jpg?_nc_cat=104&ccb=1-3&_nc_sid=730e14&_nc_ohc=_STOHSAWFTsAX_5Vs03&_nc_ht=scontent-lga3-2.xx&tp=7&oh=1bfa2998ea0d485337bd59ff9be9ea04&oe=60E7B075

     pic https://scontent-lga3-2.xx.fbcdn.net/v/t1.6435-9/198289218_3967815546586859_5031489507485638190_n.jpg?_nc_cat=103&ccb=1-3&_nc_sid=730e14&_nc_ohc=ffQYVnaoxbYAX8W6msp&_nc_ht=scontent-lga3-2.xx&oh=4eec370a7ffde416efeb38c16b970a0c&oe=60E58B9E

  tab_end
\fi

Перед вами - новое стихотворение Надежды Сточко-Бабий, написанное после трех наших занятий в Zoom
Надежда Сточко-Бабий. Хохот. 
Итог, как всегда, один: 
либо ты - господина, либо ты - господин.  
Бегай потом по кругу с одним крылом 
и новым двукрылым ври о себе былом.
А я бы и рад не жить, да нужно дитё поднять .
Рад бы и с ангелами дружить, да предадут опять.
День тишины, вот такая малость, 
такая глухость во мне осталась. 
Не жду любви. 

И не жду молвы. 

Не хочу оправдываться и выть.
Для  воя  есть время особое: 
День во храме и ночь у гроба.
"Быть ли, не быть ли" - вопрос извечный, 
а ответ завис на петле у вечности. 

Между пением  жизни и смерти хохотом...

Итог как всегда один: 

либо ты хохма, 
либо ты господин.

Я приглашаю всех желающих в свою виртуальную мастерскую. Также напоминаю, что я
умею: научные консультации, перевод текстов с украинского на русский (поэзия и
проза), полная корректура и редактура любых русскоязычных текстов, стримы с
донейшенами и продажа научных и художественных книг. 

Обращайтесь. Многие из моих клиентов могут подтвердить, что пока, слава Богу,
проколов не было: я очень долго терзаю, глажу, встряхиваю и зашиваю чужой
текст. 

Ссылка на страницу моего клиента: https://www.facebook.com/profile.php?id=100006985264990

По желанию клиента я также подбираю ему издателя. Если текст меня вдохновляет
или поражает человек, бонусом я пишу предисловие к его книге (бесплатно).
Обращайтесь в личку. Пишите друзьям. За репост - отдельное спасибо. Обращайтесь
по поводу любой из форм моего фриланса.
