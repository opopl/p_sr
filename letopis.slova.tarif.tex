% vim: keymap=russian-jcukenwin
%%beginhead 
 
%%file slova.tarif
%%parent slova
 
%%url 
 
%%author_id 
%%date 
 
%%tags 
%%title 
 
%%endhead 
\chapter{Тариф}
\label{sec:slova.tarif}

%%%cit
%%%cit_head
%%%cit_pic
%%%cit_text
Тема \emph{тарифов} вчера, как уже сказано, была отодвинута на второй план, но
совсем без нее все-таки не обошлось. Елена Зеркаль – один из главных
специалистов по конфронтационной внешней политике – ныне пересела в кресло
советницы министра энергетики, стала большим специалистом в этом направлении и
нагнетает конфронтацию уже здесь. В частности, она сообщила, что в случае
запуска "Северного потока-2" цифры в платежках украинцев резко вырастут.
Зеркаль достаточно квалифицированно объяснила, что произойти такое может, если
Россия перестанет качать газ в Европу через украинскую газотранспортную
систему, и тогда за поддержание давления в трубе придется платить исключительно
украинским потребителям
%%%cit_comment
%%%cit_title
\citTitle{Киевский полулокдаун, "анонимные" освободители Украины, украино-немецкий скандал. Итоги "Страны"}, 
, strana.news, 29.10.2021
%%%endcit

%%%cit
%%%cit_head
%%%cit_pic

\ifcmt
  tab_begin cols=2

     pic https://img.strana.news/img/article/3598/mitinh-pod-op-33_main.jpeg

     pic https://strana.news/img/forall/u/11/40/2021-10-31_12.26_.28_.jpg

  tab_end
\fi
%%%cit_text
В воскресенье, 31 октября, под зданием Офиса президента состоялась акция против
высоких \emph{тарифов}. Об этом сообщает корреспондент "Страны".  Акция,
организованная движением "Красные", проходила под видом театрализованного
действия по мотивам Гоголя и носила название "Страшная платежка".  Действующими
лицами были: украинец, украинка, страшная платежка, Вий, Вурдалак, Ведьма,
Холодная смерть, зеленый клоун.  "Сегодня праздник, он не нашего происхождения,
но все-таки мы его отметили, так, как должны это делать настоящие украинцы. Не
праздновать всякую нечисть, а прогонять ее", - рассказали участники акции
%%%cit_comment
%%%cit_title
\citTitle{Под Офисом президента провели акцию "Страшная платежка" по мотивам произведений Гоголя}, 
Владислав Бовтрук, strana.news, 31.10.2021
%%%endcit

%%%cit
%%%cit_head
%%%cit_pic
%%%cit_text
Украинские предприниматели начали получать обновлённые платежки за
электроэнергию. Несмотря на уверения правительства, что \emph{тарифы} не вырастут, по
факту платежки увеличились почти в три раза, что повергло бизнес в шок.  "Если
в сентябре мы заплатили 70 тыс грн за электроэнергию, то в октябре - 167 тысяч
гривен. Все рестораторы в шоке. Многие коллеги по рынку получили платежки по
полмиллиона гривен. Все в панике, проводят собрания и пытаются придумать, как
выжить. Никто не закладывал такие расходы на коммуналку в бюджет", - говорит
"Стране" киевский ресторатор (Kritikos) Екатерина Любчик.  Бизнес оказался не
готов к такому резкому увеличению \emph{тарифа}. Потому что многие слышали заверения
правительства, что цены расти не будут. Но это касается только населения и
(касательно стоимости газа и отопления) бюджетной сферы.  Частному бизнесу же
приходится платить по полной программе.  Теперь предприниматели спешно пытаются
придумать, где взять деньги на покрытие этих расходов.  Вариантов у них не
много. Либо повышать цены, либо искать пути экономии электричества: закрывать
часть помещения, отключать все электроприборы, которые можно отключить. Либо
попросту закрыться
%%%cit_comment
%%%cit_title
\citTitle{Платежки за свет - в три раза выше. Почему бизнесу начали приходить рекордные счета за коммуналку}, 
Анастасия Товт, strana.news, 13.11.2021
%%%endcit
