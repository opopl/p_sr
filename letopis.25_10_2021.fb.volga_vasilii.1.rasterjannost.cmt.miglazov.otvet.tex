% vim: keymap=russian-jcukenwin
%%beginhead 
 
%%file 25_10_2021.fb.volga_vasilii.1.rasterjannost.cmt.miglazov.otvet
%%parent 25_10_2021.fb.volga_vasilii.1.rasterjannost.cmt
 
%%url 
 
%%author_id 
%%date 
 
%%tags 
%%title 
 
%%endhead 
\paragraph{Иван Миглазов - У меня есть ответ! Надо бороться и Победить!}

\begin{itemize} % {

% -------------------------------------
\ii{fbauth.miglazov_ivan.odessa.ukraina.doneck}
% -------------------------------------

У меня есть ответ! Надо бороться и Победить! И плевать я хотел на то, знает
Путин что делать или не знает! Он пусть занимается своей страной! А я буду
делать всё возможное, чтобы наша Украина наконец избавилась от этой нечисти
нацистской и заокеанской и по настоящему Возродилась!!!

\begin{itemize} % {
\iusr{Василий Александрович Волга}
а как бороться будем, Иван?

\iusr{Галина Дегтярева}
\textbf{Иван Миглазов} Нашему теляти да б вовка зьисты.

\iusr{Иван Миглазов}
\textbf{Василий Александрович Волга} смело! С 2018 года, как могу, так и борюсь! А вообще с «Партии Регионов» всё началось и с ней должно всё и закончится! Вернее, «назад в будущее»! Учитывая все грехи и ошибки!

\iusr{Иван Миглазов}
\textbf{Галина Дегтярева} а и сожрём нахрен! И волка и демона!!!
\end{itemize} % }

\iusr{Евгений Рулев}

А почему Путин должен знать, что делать с такой Украиной. Он что назначенный за
тамошний беспредел? Делайте что хотите, Путин то почему должен знать всё?

\begin{itemize} % {
\iusr{Василий Александрович Волга}
не должен. я просто цитирую его.

\iusr{Галина Дегтярева}
\textbf{Евгений Рулев} Да хотя бы только потому, чтобы натовские ракеты не
стояли в пяти минутах подлёта к Москве. Я уж не говорю о брошенных русских
людях.
\end{itemize} % }

\iusr{Евгений Гарец}
Вы глубоко заблуждаетесь, утверждая Его растерянность.

\iusr{Галина Бумах}
Мне кажется , не всё так прямолинейно. И не думаю , что действительно не знает .

\iusr{Сергей Соколов}

Когда кто-то пытается проследить за мыслью Путина, советую вспомнить российские
спецслужбы, прекрасно подготовленных дипломатов, десятки институтов изучающих
международные проблемы, политологов, российских политиков, которые собирают
информацию для Путина и формируют мнение России в мире. И второе. Когда русский
царь ловит рыбу, Европа может ждать», - сказал Александр III, Российский
император.

\begin{itemize} % {
\iusr{Галина Дегтярева}
\textbf{Сергей Соколов} Последним Александром III был Брежнев.
\end{itemize} % }

\iusr{Ирина Трифонова}

Что думает Путин, знает только сам Путин.  @igg{fbicon.wink}  Щас прям вот все вам расскажи,
ага...  @igg{fbicon.face.tears.of.joy} 

\begin{itemize} % {
\iusr{Василий Александрович Волга}
" а че сказать-то хотел?"))

\iusr{Ирина Трифонова}
\textbf{Василий Александрович Волга} поживём - узнаем  @igg{fbicon.beaming.face.smiling.eyes} 
\end{itemize} % }

\iusr{Sergey Novikov Lisovsky}
Там несколько вариантов развития событий прорабатываются постоянно по логике вещей. Все знают что делать..

\begin{itemize} % {
\iusr{Василий Александрович Волга}
возмоножно. мы ведь обсуждаем ответ Путина. не более.
\end{itemize} % }

\iusr{Александр Добровольский}
Будет война. Вот и всё.

\iusr{German Gorozhanski}

Если статистика о 60-ти процентах т.н. "недовольных Путиным", приведенная
Погребинским, верна, то среди этих недовольных немало тех, кто недоволен
бездеятельностью РФ и Путина в отношении Украины. А это совсем другие
"недовольные". Плюс не каждый готов ответить о своей лояльности Путину, -
боятся люди.


\iusr{Вадим Годун}

о чем можно говорить,когда совсем недавно , Путин размышлял о красных
линиях,которые к сожалению ,уже давно перешагнуты американцами. Очень печально.
Но все близится к "большому шухеру..."

\begin{itemize} % {
\iusr{Василий Александрович Волга}
и?

\iusr{Василий Александрович Волга}
где "шухер"?

\iusr{Paul Isaev}
\textbf{Василий Александрович Волга} там же, где и вступление Украины в НАТО. Зачем раскрывать карты раньше времени? @igg{fbicon.shrug} 
\end{itemize} % }

\iusr{Андрей Василич}

Есть по крайней мере два вида ответа «не знаю». Один «не знаю» - не имею
вариантов, поэтому не знаю что делать. И другой: вариантов несколько, не знаю,
какой выбрать...

\iusr{Кирилл Ольгинский}

Это хорошо, что он не знает. Значит, вышел из зоны комфорта (если действительно
не знает и действительно ему казалось, что знал). Если вышел из зоны комфорта,
значит, открывается окно возможностей для решений.

\begin{itemize} % {
\iusr{Василий Александрович Волга}
Ну, это интересная мысль!

\iusr{Евгений Рулев}
\textbf{Кирилл Ольгинский} 

Украинцы думают, что Путин только и думает о них. Он о России думает. И
потерять равновесие недопустимо. Штурвал корабля он держит крепко.

\end{itemize} % }

\iusr{Петр Иванов}

Согласен с теми кто пишет что безусловно план есть - точнее разные алгоритмы
действий в зависимости от ситуации. Другой вопрос что форсирования однозначно
не будет. Ни с одной стороны ни с другой. И этот вопрос (вступления в НАТО)
вероятно растянется на 10ки лет. Так что не факт что при нашей жизни что то
изменится

\iusr{Евгений Рулев}

Можно предположить, что на Украине на фоне экономической катастрофы и разных
бедствий произойдёт общая деградация и провал государственности. А там хаос,
махновщина и башня рухнет. Путин будет с холма наблюдать, что будет
происходить.

\begin{itemize} % {
\iusr{Василий Александрович Волга}
уже восемь лет ждем этого, да все мимо. четвертый год экономика Украины показывает уверенный рост.
\end{itemize} % }

\iusr{Юлия Хатковская}
Сколько лет была руина в18 веке? Около 25. Вы даже не в середине пути.

\begin{itemize} % {
\iusr{Галина Дегтярева}
\textbf{Юлия Хатковская} Мерси утешили.

\iusr{Юлия Хатковская}
\textbf{Галина Дегтярева} ,а мы кувыркались с начала 90 где-то до 10 года. 20 лет. И заметьте, зрелые демократии наших борцов за свободу и независимость кормили, лечили и снабжали всем необходимым.
\end{itemize} % }

\iusr{Михаил Уханов}

Ответ уже был у Гоголя. И Россия и Украина ожесточаются и готовятся к войне.

\begin{itemize} % {
\iusr{Ирина Капырина}
\textbf{Михаил Уханов} Украина готовится к земле, так правильнее и честнее. Аминь.

\iusr{Михаил Уханов}
\textbf{Ирина Капырина} было два брата: так и Украина расколется на две части.

\iusr{Дмитрий Белоногов}
\textbf{Михаил Уханов} боюсь, что именно так.
\end{itemize} % }

\iusr{Сергей Бурналейт}

А мне вот интересно - 60\% населения за НАТО, а 40\% против. А шо, был референдум
или это спрашивали у тех, у кого кружевные трусики в кармане до эвропы лежать?

\iusr{Ирина Капырина}

Я думаю, там уже все планы прописаны, что да как и главное, когда. Приоткроется окно возможностей, и будет вскрыт нужный конвертик.

\iusr{Константин Валентинович}

Вариант только один-вторжение.. увы.. и Крым был украинским пока не заговорили о
создании там баз нато и разрыве договора аренды для чмф... не видеть этого
преступно ,но когда загоняют в угол надо защищаться ...

\begin{itemize} % {
\iusr{Aleksandr Fareniuk}
\textbf{Константин Валентинович} ПО ПРОСЬБЕ ГРАЖДАН УКРАИНЫ, А ОНА БУДЕТ ИЛИ УЖЕ ЕСТЬ.
\end{itemize} % }

\iusr{Марина Хлебникова}

Путин что сказал? Что Россия и Украина переплетены семьями, как нигде. Что тут
делать? Военными действиями..это даже не обсуждается. Уезжать в Россию? Да.
Вариантов нету.. Именно НЕТУ. Мне незачем тут писать " трактаты", россияне
видели и слышали многое за восемь лет. И если многие украинцы говорят, что
правительство украинское плохое, ай-яй, тарифы повышает и прочее.... То Россия
это агрессор, а то!

Так зачем России такое на шею? Не надо. Уж извините. Хорошая была республика,
жалко возгордилась чрезмерно. Прямо напрашивается сравнение...

\begin{itemize} % {
\iusr{Oleg Dobrolyubov}
\textbf{Марина Хлебникова} Уезжать в РФ вместе с вокзалами ,домами, дорогами,полями и городами !

\iusr{Aleksandr Fareniuk}
\textbf{Oleg Dobrolyubov} 100 \%
\iusr{Aleksandr Fareniuk}
\textbf{Марина Хлебникова} 

ПОЧЕМУ Я ДОЛЖЕН УЕЗЖАТЬ КУДА-ТО , Я ОБАЖАЮ СВОЮ РОДИНУ. ПРОВЕСТИ ЧЕРЕЗ ТРИБУНАЛ
ВСЮ ЭТУ МАЙДАННУЮ НЕЧЕСТЬ ,А ЗАБЛУДШИХ ЧЕРЕЗ ДЕНАЦИФИКАЦИЮ.

\iusr{Марина Хлебникова}
\textbf{Aleksandr Fareniuk} проведите

\iusr{Aleksandr Fareniuk}
\textbf{Марина Хлебникова} ТАК ТОМУ И БЫТЬ.

\iusr{Марина Хлебникова}
\textbf{Aleksandr Fareniuk} ну, Лейте словесную воду и дальше. Вопрос поставлен, я лично, на него ответила. Конкретно.
Можете так жить, живите. Насильно же Вас никто не гонит с Украины. Обожайте на здоровье.

\iusr{Aleksandr Fareniuk}
\textbf{Марина Хлебникова} А ВЫ УВЕРЕНЫ ЧТО УКРАИНА ОСТАНЕТСЯ УКРАИНОЙ ?

\iusr{Марина Хлебникова}
\textbf{Aleksandr Fareniuk} В каком смысле пишите?

\iusr{Aleksandr Fareniuk}
\textbf{Марина Хлебникова} В ПРЯМОМ .ПУТИН СКАЗАЛ ЧТО МЫ ОДИН НАРОД .ВЫ С НИМ НЕ СОГЛАСНЫ?

\iusr{Марина Хлебникова}
\textbf{Aleksandr Fareniuk} почему не согласна то? Вы с чего вопросы такие задаете? Я что, не написала этого выше?
Я про Фому, а Вы мне про Ерему.

\iusr{Марина Хлебникова}
\textbf{Aleksandr Fareniuk} так один народ и выход жить в России. В чем вопрос? Волшебников не ждите, они в сказках. И спокойной уже ночи.

\iusr{Aleksandr Fareniuk}
\textbf{Марина Хлебникова} НЕТ ,ВЫ ПРО ТО ЧТО Я ДОЛЖЕН КУДА-ТО ПЕРЕЕЗЖАТЬ. У МЕНЯ ТАМ ПОКОЯТСЯ ДЕДЫ ПРАДЕДЫ .
\end{itemize} % }

\iusr{Данил Пасечник}

Давайте будем объективны. Россия никогда не знала, что делать с Украиной.
Например, вступление в войну с Польшей в 1654м (спустя 6 лет как идет польская
гражданская). В чем Россия была тогда уверена? Возвращение Смоленска и
Чернигова (до смутного времени российские) - достаточно веское оправдание для
вступления в войну. В чем еще могла быть уверена? Что русские люди (бывшие
польские граждане), переприсягнувшие русскому царю будут хранить ей верность.
Присягу приняло некоторое количество русских, иные русские воевали на стороне
Польши (про присягу киевской знати и духовенства ничего не известно). Россия не
была даже уверена, чей Киев, который по андрусовскому перемирию считался
польским, но все 20 лет ею не контролировался. Россия не знала, что делать с
Киевом, но заплатила за все потери, которые потерпела Польша в связи с утратой
Киева (он просто был выкуплен). Россия точно не знала что делать с северным
причерноморьем, которое отошло к ней по итогу победы над Турцией. Россия не
знала, что делать с правобережьем, которое ей досталось по итогу развала самой
Польши. Если оценивать ретроспективно, история может выглядеть как цепь
успешных событий. Но в конкретный исторический момент, "здесь и сейчас" может
оказаться, что никто не знает, что делать, а решения принимаются как ответ на
вызовы и провокации

\begin{itemize} % {
\iusr{Марина Хлебникова}
\textbf{Данил Пасечник} с какой Украиной еще Россия в 1654 м году не знала, что делать?

\iusr{Данил Пасечник}
\textbf{Марина Хлебникова} с восставшей частью Польши

\iusr{Марина Хлебникова}
\textbf{Данил Пасечник} с какой Украиной?
\end{itemize} % }

\iusr{Владимир Симонов}

Да, не знает что делать. И это печально ... Президент такой страны как Россия не
может «не знать»... Можно предположить разочарование своей роли, или усталость от
той же взятой на себя функции мирового развития... Но там в Валдае было одно
примечательным: на вопрос о будущем мира В. Путин ответил важностью внимания к
человеку. Это контрапункт. И в этом Знание. Значительно более весомого любого
«незнания».

\iusr{Леонид Леонидис}

У Путина не бывает безвыходных ситуаций, как относиться к странам НАТО он
знает, Думает над выходом, который устроит всех и при котором Украина
сохранится.

\iusr{Владимир Симонов}

Естественно Владимир Путин знает то что должен. И ,реализуя сверхзадачу
эволюционного развития человечества, он говорит о том, что важнее и НАТО, и
Америки и всего наносного и вторичного. Относительно новейшего: индивидуальной
ответственности каждого. Каждого!!! Время упования на эгрегоры и их силу ушло.
Что на смену? Другое... Эгрегорная связь государств трансформировалась. Новейшее
-в новом мироустройстве. И здесь бессилен и Путин, и Байден ,
и все все все.

\iusr{Владимир Симонов}

Избранность не в том чтобы демонстрировать свою исключительность,
но а обретении осознанного качества прозрачневения ,
синхронно эволюционными задачам вселенной.

\iusr{Лариса Портнова}

Он столько лет на виду,что некоторые его черты характера уже не новы. И
растерянность в их перечень не входит. А осторожность входит

\iusr{Владислав Сивачук}
Ну а там сказать нельзя?

\iusr{Максимук Виктор}

Василий Александрович,

ПРОРОЧЕСТВО ПО УКРАИНЕ (о. Дмитрия Смирнова).

Церковный раскол на Украине будет на совести Константинопольского патриархата,
которым сейчас манипулируют США. В ситуации с Украиной Константинополь -
среднее звено в цепи церковного раскола, первым звеном является Госдепартамент
США.

Раскольники и их поддержавший Фанар не веруют ни в какого Бога, забывают о нем,
поэтому жесткие последствия и кровопролития в стране могут произойти. Но при
этом патриарх Варфоломей, поддержавший начало раскола, не планирует каяться,
пока выполняет приказ Запада.

Патриархат Варфоломея затеян в пику России с подачи США. Так что не будет от
него никаких извинений - разве есть в Госдепе хоть один сотрудник, который
жалеет людей?

Фактически Фанар, затеявший всю ситуацию под дудку США, не сможет выйти сухим
из воды.

Патриарх Константинополя уже оказался в патовой ситуации: он никак не сможет
выйти из конфликта, не потеряв лицо. Американские власти через Церковь хотят
устроить кровавую бойню на Украине. Последствия могут быть схожи с кровавой
бойней в Ливии.

\ifcmt
  ig https://scontent-lhr8-1.xx.fbcdn.net/v/t1.6435-9/248888273_3071857379694156_2887553251056958792_n.jpg?_nc_cat=110&ccb=1-5&_nc_sid=dbeb18&_nc_ohc=DwPmiJh5y_4AX8rTTl5&_nc_ht=scontent-lhr8-1.xx&oh=a1c20b9bcd247eb53c148560e1c01abf&oe=619E35BF
  @width 0.4
\fi

\iusr{Nataliya Makashutina}

Ну, аообще он чётко сказал, что "...легальных методов, для смены власти в
Украине, не осталось.". Т.е...

\iusr{Максимук Виктор}

Василий Александрович, вы вспомните, как до сих пор все полиНтологи куражились,
что Украину в НАТО никто не возьмёт, совершенно не понимая смысла истории, а
ведь всё наоборот, непременно возьмёт и не только Украину, но и Грузию и
Молдавию и возможно и ещё кого-то. Идёт третья мировая всеми возможными
способами с Православием и Россией, чтобы наконец создать НМП по примеру
Китая...

\begin{itemize} % {
\iusr{Maya Miyagi}
\textbf{Максимук Виктор} НМП это что ?

\iusr{Лариса Портнова}
Новый мировой порядок
\end{itemize} % }

\iusr{Светлана Жучкова}

А мне показалось что Путин говорил о самой Украине- не может он выйти и на весь
мир провозгласить планы России по Украине- хоть и формально- но независимому
госсударству???? Он сказал что не представляет что делать жителям Украины с
такой властью-правовым путем ее не убрать, оппозиция сидит под лавкой- слова
сказать не может....- естественно в такой ситуации неизвестно как убрать этих
беспредельщиков.

\iusr{Елена Иванова}
\textbf{Василий Александрович}, 

я думаю, что ответ у Путина есть. И даже не один. Если исходить из того, что
Украина сама себе помочь не может, то все ответы плохие или очень плохие для
Украины. Я рискну предположить, что прежде всего рассматривается вариант с
"традиционными штампами".

\begin{itemize} % {
\iusr{Василий Александрович Волга}
Ну, я не умею читать мыслей, Елена, не обучен. Я просто слушаю то, что люди говорят. Путин говорит, я слушаю, и пытаюсь понять сказанное им.
\end{itemize} % }

\iusr{Святослав Ющенко}

Думаю, что он знает что делать. Просто не может дать ответ в данный момент,
т.к. это перспектива будущего и планы на это у России скорее всего имеются,
просто здесь есть множестов противодействий, которые он и не озвучит. Потому,
это скорее печальные мысли, чем незнание что делать.

\iusr{Маргарита Хан}

Всё он знает, другое дело, что он показывает внешне...значит он хочет, чтобы
думали, что он в растерянности @igg{fbicon.index.pointing.up}

\iusr{Николай Григорьев}

Такое уже было. Чтобы убрали ядерные ракеты из Турции, такие ракеты привезли на
Кубу. Толпе не обязательно докладывать, что и куда привезут.

\iusr{Виктория Куколкина}

Где растерянность и где Путин. Не с его "институтами" теряться.

\begin{itemize} % {
\iusr{Владимир Коваль}
\textbf{Виктория Куколкина} 

А тем более с его оружием. Лет семь тому назад на такой вопрос он отвечал, что
украинцы сами разберутся и судя по его ответу, он начинает понимать что они не
торопятся разбираться и этого он понять не может, точно как и автор
публикации..

\end{itemize} % }

\iusr{Лариса Палагина}
отправлять войска спасать зад украины он не будет. медведев все сказал..С А М И

\iusr{Maya Miyagi}
Не верю , что не знает. План точно есть.

\begin{itemize} % {
\iusr{Василий Александрович Волга}

возможно. слишком, видать, он перспективный. 8 лет прошло, а воз и ныне там.
многие говорят, что в этом план и состоит, чтобы не иметь никакого плана.

\end{itemize} % }

\iusr{Людмила Александрова}
А есть ли силы на самой Украине, что бы было на кого опереться в помощи?
Т.н. ОПЗЖ не только себя не проявляет, но откровенно сидит "в домике"
И зачем за таких впрягаться?

\iusr{Катя Костюковская}

Россия не «не знает». Тут большинству безразлично. Союз распался давно, Украина
- во всяком случае юридически - независимое государство, которое к России имеет
отношение только границами и трубой.

\begin{itemize} % {
\iusr{Владимир Коваль}
\textbf{Катя Костюковская} Разве?
\end{itemize} % }

\iusr{Vadim Tokarev}
Не всегда нужно говорить открыто о своих планах. Зе Путину - не пример.

\iusr{Василий Коростенец}

Враньё многие мои знакомые изменили своё мнение став положительно относится к
России . Так что пизд@@@т ваши социологи и Вы вместе с ними .

\begin{itemize} % {
\iusr{Василий Александрович Волга}

везет Вам со знакомыми. Я мы все больше смотрим на социологию и результаты
голосования, которые показывают, что на прошлых выборах в горсовет Киева,
например, политики русофобы набрали в общем зачете 87\%

\end{itemize} % }

\iusr{Коля Кузнецов}

Это примерно как «не знать», что делать с чирием, который создаёт сильный абсцесс.
Перспектива, что он сам рассосётся - маловероятна. Остаётся вопрос - какого уровня будет хирургическая операция.

\iusr{Светлана Лаврухина}

Наивно думать, что у Путина нет плана или ответа. Есть. Он старается
минимизировать ущерб стране, которой он служит в качестве президента и стране,
которой плевать на собственный народ.

\begin{itemize} % {
\iusr{Василий Александрович Волга}
да мы не против. мы ведь просто обсуждаем его ответ: "Я не знаю, что делать"

\iusr{Светлана Лаврухина}
\textbf{Василий Александрович Волга} , это он надеется на дискуссию.
\end{itemize} % }

\iusr{Елена Самойленко}

Больше занят вопросом как всех насильно добровольно вакцинировать. И как
быстрее избавится от пенсионеров и бедных

\begin{itemize} % {
\iusr{Larisa Kuprina}
\textbf{Елена Самойленко} Для этого он организовал материальную помощь семьям с детьми, пенсионерам, военным и прочая!

\iusr{Елена Самойленко}
\textbf{Larisa Kuprina} да, именно для этого лишил пенсионеров пенсии на 5 лет, но сунул подачку в 10 000,0. Миллион забрал 10 000,0 сунул как собакам. Хорошая экономика
\end{itemize} % }

\iusr{Иван Алексеев}

Вопрос должен быть адресован не ему, а украинцам. И у нас есть на него ответ -
эмиграция. Печаль, беда, но это факт. Если мы будем оглядываться на то, что
скажут соседи - этот ответ останется единственным.

\iusr{Леонид Зарапа}

В данный момент и здесь - Американцы нас переиграли с Украиной, но как говорят
Русские - Ещё не вечер. Крымская война в 1854 году была проиграна Россией, но
статус КВО она себе всё равно вернула, через время. Так же мы проиграли Японцам
в 1905 году и Цусиму и Порт Артур и Сахалин, но в 1945, через 40 лет мы снова
восстановили свой статус КВО, да ещё и с контрибуцией Курильских островов.
Так-что по Украине наш статус КВО будет восстановлен. Проверено неоднократно
историей.

\iusr{Константин Филиппишин}

Его неуверенность мне говорит о том что он пока не решился. Но сценарии, думаю,
давно обработаны. Вопрос в том, как во времени будет ситуация развиваться. Но в
любом случае это будет тяжёлый выбор

\iusr{Лариса Забродина}
Не думаю, что у В.В. Нет плана Б.

\iusr{Dmitry Serebrov}

Вы плохо знаете повадки разведчика. Это первое. Важнее другое: когда жители
Украины сами начнут решать свою судьбу? Или всем нравится нынешнее положение и
состояние их страны?...

\begin{itemize} % {
\iusr{Светлана Чемоданова}
\textbf{Dmitry Serebrov} Вы себе не представляете, в каком мы восторге! Особенно русские, особенно повершившие в 2014-м.

\iusr{Dmitry Serebrov}
\textbf{Светлана Чемоданова} 7 лет терпеть ЭТО - сродни мазохизму...

\iusr{Светлана Чемоданова}
Мы тоже думали, что 7 лет предательства - должно быть неуютно на сердце. Но могли бы хоть не пинать оставшихся в этом концлагере.

\ifcmt
  ig https://scontent-lhr8-1.xx.fbcdn.net/v/t39.30808-6/249239159_277838474249287_5958270640605857727_n.jpg?_nc_cat=106&ccb=1-5&_nc_sid=dbeb18&_nc_ohc=qBpTU8ocytsAX_DcgeH&_nc_ht=scontent-lhr8-1.xx&oh=1d3d424df81970afd1adb22ff1b6cc5e&oe=617E6F6D
  @width 0.7
\fi

\end{itemize} % }

\iusr{сергей шашкин}

Не будьте так наивны - знает. Он четко сказал, что это уже не зависит от народа
Украины, а на сколько и куда шагнут упоротые сейчас не может предсказать никто.
Россия будет действовать по результатам.

\begin{itemize} % {
\iusr{Василий Александрович Волга}
мне показалось он сказал: "Я не знаю". А в какой части Вы услышали обратное?

\iusr{сергей шашкин}
Не знает в т ой части, так как дальнейший ход событий не зависит от народа Украины, а от той камарильи власти в Киеве, а они в свою очередь подобны обезьяне с гранатой. Вы можете знать, что она вытворит в следующую минуту?

\iusr{сергей шашкин}
РФ будет реагировать в зависимости от поведения Украины, в той или иной степени. А ВВП показал, что он способен на поступок.
\end{itemize} % }

\iusr{Ирина Александрова-Субботина}

Какие вы хитрые. Заверили кашу, а расхлёбывать и строить планы должен Путин? У
него за спиной 145 миллионов своих граждан. Придумывайте сами- что вам с вашей
страной делать.

\begin{itemize} % {
\iusr{Василий Александрович Волга}
\textbf{Ирина Александрова-Субботина} причём тут хитрость? Я просто цитирую Путина.

\iusr{Ирина Александрова-Субботина}
\textbf{Василий Александрович Волга} Вы ждёте планов и советов? Не ждите. Судьбу своей страны решать только вам самим.Чтобы в очередной раз не искать виноватых в ваших бедах. Ну Погребинский еще может Байдена спросить. У него точно есть ответ.

\iusr{Александр Мельников}
\textbf{Ирина Александрова-Субботина} Мадам, Погребинский - нравственное мерило современного вменяемого украинского общества. Не нужно о нём так... Так... Так слишком по-кемеровски.

\iusr{Василий Александрович Волга}
\textbf{Ирина Александрова-Субботина} я просто цитирую Путина.

\iusr{Ирина Александрова-Субботина}
\textbf{Александр Мельников} у Погребинского хватает проколов и неудобных высказываний. Не надо делать из него Небожителя
\end{itemize} % }

\iusr{Александр Коломиец}

Россия действовать будет! Только так, как ляжет карта. Нужно время. И ещё, В.В.
Путин рассчитывал на внутренние силы народа, самой страны и областей Украины.
Но увы! Товарищ Василий, люд простой не готов не к чему. Не знаю или время
такое. Если в силу самых тревожных обстоятельств, когда настанет предел, лопнет
терпение и на грянет буря, тогда может радикально всё измениться, вплоть не
жалея сил и своих жизней.

\iusr{Андрей Богатырев}

В общем что думаю:

1. Если они не говорят что-то, это еще не значит, что они не знают, что делать,
т.к. не все можно говорить.

2. С чего Путин должен "знать, что делать"? У него есть общая стратегия, он ее
реализует. Но у него нет пошагового календарного графика. Есть просто некоторые
заготовки реакций на внештатные ситуации и некоторые исходные предпосылки.

\iusr{Василий Рябченко}

Все всё знают. Оттягивают максимально.

\begin{itemize} % {
\iusr{Василий Александрович Волга}
\textbf{Василий Рябченко} или нам с Вами так хочется? Иди может просто боязно признать неприятные реалии?

\iusr{Василий Рябченко}

Мы просто не владеем объективной информацией. Реалии таковы, что падение и
разложение охватывают все духовные области, независимо от границ государств.
Все значительно хуже чем украинский национализм, точнее это небольшая часть
большого конца. Это в тонком а значит и в главном плане. А есть еще и
военно-политический план(ы), адаптируемый под изменееие обстановки в.т.ч на
основании секретных данных разведки, исследования, развития и реализации
программ вооружения.

А если просто, то все мы падаем, США печатают фантики как в последний раз,
похоже что для последнего акта.

\iusr{Светлана Чемоданова}
\textbf{Василий Александрович Волга} 

Вспоминаю, что именно так мы думали в 2014-м году. И "ХПП", и "ну не может же
он ничего не сделать...", и "ему не простит собственный народ". Ага. Народ не
только простил, но и нашел сто объяснений и даже успешно огрызается на бывших
соотечественников к радости их оккупантов.

\end{itemize} % }

\iusr{Андрей Трофимов}

В пиар-кампаниях моя функция - аналитика. Работа с базами данных, упрощение,
выводы на трёх страницах ворда. В 2012 году был заказ, что делать с Украиной?
Шуршал шестеренками три дня, написал ответ, что делать нужно "ничего". То есть,
просто забыть, забить и похоронить все контакты и связи. Вывести оттуда
российский бизнес и закрыть все долгосрочные договоры, пусть и в минусе.

И что? Думаете, Кто-то послушал советы? А у меня рекомендации были, опыт,
связи.

Более того, я должен был все подробно объяснить, да.

И, кстати, в моей биографии назначение Черномырдина рулить отношениями с
Украиной, это уже третья подстава с его участием, первая была, когда я потерял
надежду на пятое место в гонке Санкт-Петербург 2004, когда его позорно сняли с
поста за пять дней (!) до голосования МОК (я был в Комитете 2004), а потом его
ребята из молодежного Нашего Дома меня кинули с рекламным проектом.

\iusr{Юрий Афендулов}

Думаю что Путин не знает что делать. НАТО может разместится на Украине по
своему усмотрению не особо интересуясь чьим либо мнением. Вопрос с какой целью?
А цель мне кажется всё таки спровоцировать поярче конфликт между Украиной и
Россией. Расклад для НАТО не плохой. Боевые действия на чужой территории с
привлечением чужого человеческого ресурса, максимум на что придётся
раскошелится это утилизировать своё старое вооружение. Исход будет ясен и при
этом на много десятков летов о дружбе и сотрудничестве между Украиной и Россией
можно забыть и + Б. Д. хорошо откинут Россию в экономическом развитии. А там с
юга Эрдоган свою думу думает. Россию обкладывают плотно. Естественно Украина
выступает в роли расходного материала.

\begin{itemize} % {
\iusr{Александр Коломиец}
\textbf{Юрий Афендулов} В этом есть печаль тревоги. Простой вопрос: почему кукловоды сами принимают решение про "Няню", забыв с народом посоветоваться. Или народ - это уже не народ, а телки и быки в полях с пастухом и сплетью.

\iusr{Александр Мельников}
\textbf{Юрий Афендулов} Юрий, Вы в самом деле думаете, что обнаружив НАТОвские ракеты на территории современной Украины, Россия снова пожалеет Париж?
\end{itemize} % }

\iusr{Елена Фролова}

Да, наверняка, Путин знает, что делать. Только зачем он будет открывать карты
вашему нацистскому руководству? Так, что ждите сюрприза, если что...

\iusr{Николай Николаевич Бруссэ}
Есть ответ, его не озвучивают.

\iusr{Роман Волков}

А с чего вы взяли, что не имелось ввиду "не знаю какой вариант выбрать"?

Когда у вас несколько вариантов - каждый со своими плюсами и минусами. Вы
будете честны, если скажете, что не знаете что (сейчас) делать.

\begin{itemize} % {
\iusr{Василий Александрович Волга}
\textbf{Роман Волков} просто я слушаю его и цитирую.
\end{itemize} % }

\iusr{Den Block}

Играл с гроссмейстером в шахматы, пару раз его лицо за нашу короткую игру было
ну таким растерянным, что предавало мне некую радостную уверенность в себе, а
на 24 ходу с таким же растерянным лицом говорит: - Вам сударь Мат

\begin{itemize} % {
\iusr{Василий Александрович Волга}
\textbf{Den Block} да. Но сегодня Лицо гроссмейстера именно растерянное. И он говорит: «Я не знаю, что делать»(с)

\iusr{Den Block}
\textbf{Василий Александрович Волга} 

растерянность на лице одного из лучших профессиональных разведчиков-вербовщиков
по отрочеству, человека который в 90е поставил раком весь криминалитет Питера а
потом и России, включая все международные террористические синдикаты вместе
взятые???

Есть вопросы, на которые лучше сказать с потерянной улыбкой: - Я не знаю.

Все он знает, и на вопрос этот есть множество уже готовых ответов и решений,
ответ может быть только конкретным действием без слов, время разговоров
окончено.

США сейчас торгуют Украиной, хотят продать за дорого, Россия делает вид, что
этот лот ей вообще не нужен. Ставки будут то повышаться, как сейчас, то
понижаться, но заражённая территория должна отстояться, как стакан с мутной
водой, это факт.


\iusr{Василий Александрович Волга}
\textbf{Den Block} увидим

\iusr{Валентина Титова}
\textbf{Den Block} очень хочется, чтобы Вы оказались правы... но не очень верится((
\end{itemize} % }

\emph{Андрей Сергеевич Пастушик}

К сожалению, 6 лет опыта Сирии, могут пригодиться! пора кончать этот беспредел!

\iusr{Марина Хлебникова}

Сколько я читала в соц. сетях о Путине всего. И что он царь, царек, тиран, или
особо одаренные граматеи пишут тЕран... и прочее. А ведь действительно, считают
его непревзойденным по уму, политиком. Что он то должен что то придумать и
предпринять, что наворотили украинцы за тридцать лет. И ведь не только власти
или олигархи, а и сами граждане. Активные или пассивные. Хотели то, как лучше,
а получилось, как всегда, предательство. С кого спрос?

С России, конечно. А мы то, украинцы, мы глупые, какой с нас спрос? Мы же
думали, вот, сейчас предадим Россию, сделаем её врагом на радость нашим
друзьям, американцам, ии как заживем по-европейски!! Здорово! Пенсии..... Во!
Пенсионерки при опросе говорили, сама слышала... В Союзе жили с Россией раньше.
Нужна ли украинцам Россия, как нужна России Украина? Не нужна. А России нужна.
Больно, правда. Хотя лично у меня вся русская родня в Москве издавна в
поколениях. Свои же все были! Дочь, была Хлебникова, стала в замужестве
Шевченко, четыре года назад. Когда кто на фамилии и национальность в СССР
внимания обращал? Русские и украинцы точно не обращали. Поэтому и семьи
переплетены тесно. Так Путин и сказал. И нет у него волшебной палочки, и хоть
очень умный он, все же ситуация, такова, что в этой ситуации с Украиной выход
один. Россия примет всех к себе. Россиян мало, страна огромная. Иного решения
нет. Так что, кто обожает Украину, живите и обожайте. Кто то обожал Ливию, к
примеру. Кстати, давно туда замуж вышла моя троюродная сестра, были живы
родители, приезжала, жила там прекрасно. Как сейчас, не знаю.

\begin{itemize} % {
\iusr{Валентина Тарасенко}
\textbf{Марина Хлебникова} 

Вам повезло, что у вас в большой политике появился Путин, а ведь могло пойти
совсем по другому сценарию, опять же вам повезло. У нас , на Украине нормальных
людей большенство, вот этих долбо.... меньше, но они очень агрессивные,
получают гранты, и пролезли в правительство. Поэтому у нас сейчас так. А тех, кто
мог бы изменить ситуацию, или посадили или выжили из страны или потихому убили,
остальные только красиво говорят.


\iusr{Дмитрий Басов}
\textbf{Валентина Тарасенко}

"У нас , на Украине нормальных людей большенство" - где вы видите это
большинство?

Я, в течении 30 лет вижу большинство агрессивных идиотов, и совершенно не вижу
перспектив.


\iusr{Валентина Тарасенко}
\textbf{Дмитрий Басов} я общаюсь с людьми, работаю в коллективе, живу среди людей, " штрих идиотов" встречаю крайне редко.

\iusr{Валентина Тарасенко}
Щирих

\iusr{Светлана Чемоданова}
\textbf{Валентина Тарасенко} Это у вас в Одессе их малая концентрация. В Киеве их погуще будет. Прям второй Львов по национальной агрессии.

\iusr{Марина Хлебникова}
\textbf{Валентина Тарасенко} 

конечно, Вы полностью правы, России повезло, что у неё случился Путин. Я
нисколько не горжусь тут, в обсуждениях, что россияне все, как на подбор,
патриоты и адекватные люди. Все одинаково. Мы, не то, что братья или сестры, мы
один , разделенный народ. Но, Украина неоднородна. Произошла диффузия
бандеровщины. И как это можно терпеть прославления бандеровщины в стране, я не
знаю.

Но.. @igg{fbicon.index.pointing.up} В России, в худшие её времена, когда и я думала, ну все, приехали... Не
кричали ни на кого, с пожеланиями смерти, не называли агрессорами, когда Россия
после 91 года исправно платила дань Украине за русский Крым, переданный Укр.ССР
в составе СССР, в последствие по взмаху руки Ельцина. Никого не обвиняли и тихо
тонули..

\iusr{Валентина Титова}
\textbf{Валентина Тарасенко} Да, БЫЛО большинство. За последние СЕМЬ!!! лет это большинство превратилось в меньшинство, притом диванное((

\iusr{Валентина Тарасенко}
\textbf{Светлана Чемоданова} просто Львов переехал в Киев, а интилегенция вся свалила за бугор.

\end{itemize} % }

\iusr{Alexander V. Pirogov}

Ответ был дан, четкий, и неоднократно, в разное время, касательно красных линий
и потери Украиной государственности.

\begin{itemize} % {
\iusr{Василий Александрович Волга}
\textbf{Alexander V. Pirogov} а мы разве говорим о том, что было когда-то?

\iusr{Alexander V. Pirogov}
\textbf{Василий Александрович Волга}

Уверен, что у политиков такого уровня нет понятий «когда-то», да и слабостей и
нюней у них не бывает, как у нас, простых смертных. Стратегия для У готова.

\end{itemize} % }

\iusr{Дарья Митина}

и восьми лет не прошло, как Вы это поняли:)

\iusr{Alexey Moiseyenkov}

Выход простой. Признать ЛДНР, заодно и Харькрвскую, Одесскую и Херсонские
республики

\begin{itemize} % {
\iusr{Андрей Быков}
\textbf{Alexey Moiseyenkov} ну ЛНР и ДНР хоть какой-то референдум провели и Республики хоть как-то существуют, а где остальные то? Просто так признать, то, чего нету?
\end{itemize} % }

\iusr{Alexey Moiseyenkov}
Признать ЛНР и ДНР в их прежних границах (бывших областей)

\iusr{Alexey Moiseyenkov}
Кстати. Все может поменяться этой зимой. Когда города начнут замерзают. Народ скажет нахер такую НАТУ

\begin{itemize} % {
\iusr{Svetlana Belikova}
\textbf{Alexey Moiseyenkov} , не поменяется. ВВ как всегда начнет спасать всех, кроме своих... Так что никто не замерзнет и продолжат благодарно гадить

\iusr{Дмитрий Басов}
\textbf{Alexey Moiseyenkov}
Эти сказочки ни на кого не действуют.

\end{itemize} % }

\iusr{Роман Богвелишвили}

Буржуазная, морально деградирующая РФ - лишь тень России и СССР без внятной
цели бытия и программы развития. Могла ли некогда великая Византия в последние
годы перед коллапсом помышлять о внешней экспансии? Путин ясно осознал и
озвучил, что нынешний российский капитализм - это тупик, необходим срочный
разворот, иначе сомнут изнутри и снаружи. Это требует от элит самоотречения и
аскетизма,к чему те не готовы и не способны. Отсюда и замешательство во всем,
включая внешнюю политику.

\iusr{Мари Леон}

Разве это главное? Путин, слава Богу, не президент Украины, не ее гражданин и
не украинец.

Главное то, что никто на Украине, да и Украина в целом, не знает что делать
сама с собой. Летите в пропасть, но продолжаете искать виноватых  @igg{fbicon.man.facepalming} 

\begin{itemize} % {
\iusr{Василий Александрович Волга}
\textbf{Мари Леон} точено же? Украинцы очень даже знают, что делать. И делают.

\iusr{Мари Леон}
\textbf{Василий Александрович Волга} 

ну, если вы считаете, что скатывание в пещерный национализм, расползание
гражданской войны по территории страны, разгул бандитизма, воровство и
беззаконие в масштабах государства и есть тот путь, который украинцы знают и
осуществляют - то да, их трудно упрекнуть в бездействии.

Странно только, что при этом вы удивляетесь "бездействию" Путина.

Бешенных псов не обязательно отстреливать - они и без этого обречены

\end{itemize} % }

\iusr{Светлана Дремлюгина}
Путину не придется.... он исчерпал себя как политик. Пора как Меркель на отдых

\iusr{Андрей Волков}

Думаете этотрастеренность? Я думаю это хуже определенного ответа и хуже угроз.
В жизни когда нечего сказать в конфликте начинается драка.

\begin{itemize} % {
\iusr{Василий Александрович Волга}
\textbf{Андрей Волков} и кто кого сегодня бьет?

\iusr{Андрей Волков}
\textbf{Василий Александрович Волга} посмотрим, наберитесь терпения.

\iusr{Василий Александрович Волга}
\textbf{Андрей Волков} с этим проблем нет. Уже 8 лет терпим, привыкли.
\end{itemize} % }

\iusr{Александр Добровольский}

Зачем политику дан язык? Чтобы скрывать свои мысли.

Вы этого не знаете? Знаете, конечно, но тем не менее предпочитаете устраивать это обсуждение и "почесать свой Гондурас".

Дело ваше, чешите.

\iusr{Виталий Водопьян}

Путин озвучил "красные линии" от которых он не отступит, иначе это будет как
"потеря лица" лично, так и сдача национальных интересов Российской Федерации.

Уверен это понимают как Вашингтоне, так и в Брюселе... а нам расхлебывать
последствия геополитических игр, надеюсь не в виде "пушечного мяса".

\begin{itemize} % {
\iusr{Василий Александрович Волга}
\textbf{Виталий Водопьян} он озвучил нечто иное. Про красные линии он говорил в другом своём выступлении.
\end{itemize} % }

\iusr{Maximilian Aue}
Почему вы, Василий Александрович Волга считаете что НАТО на Украине опаснее для России чем НАТО в Прибалтике и Польше?

\begin{itemize} % {
\iusr{Василий Александрович Волга}
\textbf{Maximilian Aue} я слушаю Путина. А он говорит, что это крайне опасно и определяет НАТО на Украине, как красную черту.

\iusr{Андрей Быков}
\textbf{Василий Александрович Волга} 

соглашусь... и сказав о красной черте и объявив этот на весь мир, теперь уверен
на все 100\% - Запад сделает всё, чтобы её переступить и посмотреть на реакцию
Путина


\iusr{Юлия Цветкова}
\textbf{Maximilian Aue} НАТО на Украине подразумевает ещё и человеческие резервы, в Прибалтике мало населения! То что в Украине будет задействовано население в противостоянии, это ясно всём....
\end{itemize} % }

\iusr{Сергей Поддубский}
Ничего не делать, отрезать и забыть.

\begin{itemize} % {
\iusr{Юлия Цветкова}
\textbf{Сергей Поддубский} оставить разгребать всё это нашим детям?

\iusr{Юлия Цветкова}
\textbf{Сергей Поддубский} поддерживаю только то, что "игнор Украины" самое лучшее лекарство для неё. Постоянное упоминание украинских проблем, только раздувает важность у её элит
\end{itemize} % }

\iusr{Юля Клейменова}

\obeycr
Естестественно не знает, что сейчвс делать.
Детки-то быстро растут, а мы - уже вымирающие динозавры.
Деткам-то уже вдолбили за эти годы, что надо! И уже звучит их голос на арене!
Вспомните, первые годы после 14-го года. Сколько наших было.
Я по Одессе сужу. Да сколько людей не то что выступило бы против этой власти,
а снесло бы её. Только бы стоило помочь, поддержать.
А теперь уже давно поздно. Мы уже "не рулим", на наше место пришли ДРУГИЕ - ТЕМНЫЕ.
\restorecr

\iusr{Татьяна Тропина}
Я думаю, Путин знает. Просто не захотел озвучивать в данный момент.

\iusr{Рустам Азизов}
Ответ прост, ничего не надо делать, просто не лезть к нам, тем более соседям.....сами разберемся. Другой вопрос какой ценой.

\begin{itemize} % {
\iusr{Юлия Цветкова}
\textbf{Рустам Азизов} вы в курсе если СШ@ не получает то, чего хочет, оно устраивает "Сирию " или Ирак"на той территории???

\iusr{Рустам Азизов}
\textbf{Юлия Цветкова} а что єто сш?

\iusr{Рустам Азизов}
\textbf{Юлия Цветкова} 

Вот только Сирию не приплетайте сюда.... мой отец служил в сирии еще советским
офицером, там стоит база русских как правоприемника советского союза ...и они
не хотят терять контроль.... по ираку не могу сказать

\iusr{Юлия Цветкова}
\textbf{Рустам Азизов} после крушения СССР базы уже не было

\iusr{Рустам Азизов}
\textbf{Юлия Цветкова}

\url{https://ru.m.wikipedia.org/wiki/Тартус}

\iusr{Юлия Цветкова}
\textbf{Рустам Азизов} 

\href{https://youtu.be/vnW6xF8MDc8}{%
Секретный доклад глобалистов. Хаос в мире продлится до 2026-го года. Андрей Фурсов, %
Андрей Фурсов, youtube, 22.10.2021%
}

\iusr{Рустам Азизов}
Юлия, милая Юлия, не хочу спорить))))) хорошего Вам дня!!!

\iusr{Юлия Цветкова}
\textbf{Рустам Азизов} 6 лет назад вернулись в Тартус 

\href{https://youtu.be/tt4M_6Rc654}{%
Тартус – самый безопасный порт Восточного Средиземноморья, Телеканал Звезда, youtube, 19.11.2021%
}

\end{itemize} % }

\iusr{Leo Singer}

Все он знает. Но не все должно становится предметом гласности.

\begin{itemize} % {
\iusr{Василий Александрович Волга}
ага. ну, слава Богу!
\end{itemize} % }

\end{itemize} % }

