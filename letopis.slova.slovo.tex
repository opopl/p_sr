% vim: keymap=russian-jcukenwin
%%beginhead 
 
%%file slova.slovo
%%parent slova
 
%%url 
 
%%author 
%%author_id 
%%author_url 
 
%%tags 
%%title 
 
%%endhead 
\chapter{Слово}
\label{sec:slova.slovo}

Рада после \emph{слов} \enquote{свободовки} о \enquote{героях-бойцах СС} провалила
проект о запрете героизации нацизма,
Эллина Либцис, strana.ua, 03.06.2021

И вначале было \emph{Слово} и \emph{Слово} было у Бога и \emph{Слово} было Бог,
от Иоанна, Евангелие (Благая Весть)

Во-первых, Зеленский совершенно не умеет давать интервью, что странно для
человека, много лет говорящего со сцены, и исходя из этого, должного владеть
\emph{словом}. У него совершенно путанные мысли, отсутствие логических связок и
переходов при формировании ответов, фразы не выверенные, а
импровазиционно-эмоциональные, противоречивые по содержанию и сути. Такой себе
Голобородько-стайл, хотя за два года уже можно было бы освоить совершенно
другой уровень общения со СМИ – президент-стайл,
\textbf{Это странно, но Зеленский совершенно не умеет давать интервью}, Светлана Крюкова, 
strana.ua, 03.06.2021

Друзі! Тижневик \enquote{\emph{Слово Просвіти}} — один із небагатьох нині
існуючих островів незалежного українського ­ \emph{слова} в інформаційному морі
— за певного \enquote{сприяння} нашої влади опинився на межі припинення свого
існування. І тільки ми самі — ті, хто любить, шанує і цінує українське
правдиве і об'єктивне \emph{слово}, можемо допомогти тижневику. Будь-яка
фінансова підтримка видання — це вагомий внесок в українську справу. Тож
подаємо Вам наші банківські реквізити: АТ \enquote{Альфа-Банк}, р/р UA 30
300346 0000026002016994001; ­ отримувач: ГО \enquote{Всеукраїнське товариство
\enquote{Просвіта} ім. Тараса Шевченка}; код ЄДРПОУ/ІНН: 00031756.  Призначення
платежу ­заповнюється платником так, щоб надати повну інформацію про платіж.
На наше прохання про допомогу для газети \enquote{\emph{Слово Просвіти}}
відгукнулися: Кропивницька Мар'яна Андріївна – 800 грн.  Стебловська Леонілла
Петрівна – 100 грн.  Лизанчук Василь Васильович – 500 грн.  Стасик Юлія
Іванівна – 100 грн.  Мельниченко Володимир Юхимович – 1000 грн,
\textbf{Звернення до наших читачів, передплатників, просвітян}, slovoprosvity.org, 17.05.2021

%%%cit
%%%cit_pic
%%%cit_text
У Зеленского были все возможности выполнить обещание - и снизить тарифы.  Но,
видимо, ему скучно держать данное \emph{Слово} ...  Впрочем, Зеленскому давно
привычно жить по принципу одного героя известного советского кинофильма: я -
хозяин своего \emph{Слова}. Я его дал, я его обратно взял. Тем более, что еще
столько можно хорошего наобещать избирателю, что к чему исполнять? Скучно ведь
станет жить. Буднично. Серо
%%%cit_title
\citTitle{У Зеленского были все возможности выполнить обещание - и снизить тарифы}, Валерий Песецкий, strana.ua, 11.06.2021
%%%endcit

%%%cit
%%%cit_pic
%%%cit_text
Есть здесь и откровенная дичь, которую сами составители «Стратегии» в силу
своих особых способностей наверняка даже не заметили. В одном месте они
перечисляют страны, территориальную целостность и суверенитет которых
поддерживает Украина – Азербайджан, Молдавия, Грузия. Сербии в этом списке как
бы «случайно» нет. Ну нет и нет. Как говорит Блогер, \emph{Cлова} не нужны.
Главное – это действия.  И вот вам действия. В другом месте они же пишут –
Украина продолжает брать активное участие в миссиях под руководством НАТО в
Афганистане и... Косово. В Косово, блд. Но вы же дебилы сами на официальном
уровне Косово продолжаете считать территорией Сербии. И одновременно без
согласия Белграда принимаете участие в миссиях НАТО на ее территории в Косово.
И тут же на каждой страничке отражаете раз за разом ру-агрессию и декларируете
приверженность международному праву
%%%cit_title
\citTitle{Украинская власть разродилась стратегией внешнеполитической деятельности}, Игорь Лесев, strana.ua, 11.06.2021 
%%%endcit

%%%cit
%%%cit_pic
%%%cit_text
Никакой свободы \emph{Слова} больше не будет, цензура становится нормой для
информационной политики Свободного Мира, а единое пространство глобального
интернета уже вскоре разделят на огороженные друг от друга сегменты. Новая
Берлинская стена – только распространившаяся практически повсеместно.  А если в
Евросоюзе заблокируют Telegram, его немедленно запретят в свободной
демократической Украине – вдобавок к уже запрещенным VKontakte, Одноклассникам,
Яндексу, Касперскому, итп. Потому что это станет отличным поводом расправиться
с неугодными для власти тг-каналами. Свободный Мир запрещает – и мы тоже.  Не
говорите, что вас не предупреждали
%%%cit_comment
%%%cit_title
\citTitle{Власти Германии грозят заблокировать Telegram}, Андрей Манчук, strana.ua, 13.06.2021
%%%endcit

%%%cit
%%%cit_head
%%%cit_pic
%%%cit_text
\enquote{Як знаєте, мої друзі, я не є приятелем гомінких \emph{Слів}. Я не
знаю, чи і наскільки нам пощастить розвинути діло, за яке ми ось тут
прийнялися. Це залежатиме від праці, зусиль, послідовности і жертовности нас
усіх. І це залежатиме від того, наскільки ми нашу справу зуміємо зробити
зрозумілою всьому українському громадянству ...  Спротиви, які зустрінемо на
нашому шляху, будуть велетенські. Бо ж віднова Соборної Української Держави,
сама собою однозначна з ліквідацією московської імперії, як і польського
історичного імперіялізму, спричинить таку докорінну перебудову цілого Сходу
Европи і великої частини Азії, що це з конечности вплине не менш глибоко й на
політичний вигляд всієї решти світу}. (4, с.  223-224)
%%%cit_comment
%%%cit_title
\citTitle{Евген Коновалець: від життя до смерти у безсмертя (тези за телепрограмою \enquote{Ген українців})}, 
Ірина Фаріон, blogs.pravda.com.ua, 14.06.2021
%%%endcit

%%%cit
%%%cit_head
%%%cit_pic
%%%cit_text
Безусловно, надо все рычаги, а их два основных (структура госбюджета и
структура налогов), максимально задействовать, чтобы повернуть страну с
монопольно-олигархической системы на инновационный путь экономического
развития.  Но кто всё это будет делать? То есть кардинально менять
экономические отношения и повышать уровень профессиональных знаний? Хватит ли
компетенции и политической воли у представителей верхних эшелонов власти?  Или
традиционно дальше \emph{слов} дело не пойдет?
%%%cit_comment
%%%cit_title
\citTitle{Украину нужно повернуть с олигархического на инновационный путь развития}, 
Александр Гончаров, strana.ua, 17.06.2021
%%%endcit

%%%cit
%%%cit_head
%%%cit_pic
\ifcmt
  pic https://avatars.mds.yandex.net/get-zen_doc/3725151/pub_60d0b464b2ba075e653d51f9_60d0b72e5fc3481f3fca0c3b/scale_1200
\fi
%%%cit_text
Вначале было - \emph{Слово}. И \emph{Слово} было у Бога. И \emph{Слово} было Богом.  Пришли черти и
\emph{Слово} стало убого. Программа беса - смутить, свою муть из своего омута выдать
за чистую воду. В тихом омуте... смутьяны водятся.  Писания - это тоже образ.
Это мироздание, построенное на правдивой базе. Само мироздание строй как
хочешь, ведь оно - поверхностное. Чтобы оно не рухнуло, за основу бери правду.
Такое здание простоит долго и люди часть правды и ВСЮ ложь будут воспринимать
как ВСЮ правду. Хитро, не правда ли? Да, Искуситель хитёр и коварен
%%%cit_comment
%%%cit_title
\citTitle{Чтобы мир извратить, нужно просто исказить исКОНную Суть Слова}, 
Ирина, zen.yandex.ru, 22.06.2021
%%%endcit

%%%cit
%%%cit_head
%%%cit_pic
%%%cit_text
\emph{Слово} - сгоВОР - ить всё поясняет да и проясняет, кому не ЯСНО.  Это ещё
цветочки. Ягодки будут впереди, когда мы им лайки подсунем и лайкать всех
заставим, АХ..вот и ГАД ЖЕ ТЫ! Хи - хи - хи. Вот такой он - ОН лай н. Н - наш,
наше. Искусителя сила искусная - в умении да в учении. Они - то НАШЕ \emph{Слово}
учат, в отличие от нас
%%%cit_comment
%%%cit_title
\citTitle{Чтобы мир извратить, нужно просто исказить исКОНную Суть Слова}, 
Ирина, zen.yandex.ru, 22.06.2021
%%%endcit

%%%cit
%%%cit_head
%%%cit_pic
%%%cit_text
Хіба це не історія, подумали німці, вирішивши залишити \emph{слова} жити, аби й надалі
не забувати, хто вони насправді.  Злочини нацистської Німеччини були
унікальними. Залишивши \emph{слова} солдата країни-переможця на будівлі свого
парламенту, Німеччина показує, що вона засвоїла важливі уроки зі свого
минулого, помістивши все в акуратні архіви історичної пам’яті. Така вона й є.
Німці тримають історію перед очима завжди і скрізь, вивчають у школі, говорять
з незнайомцями, досі відчуваючи провину перед євреями, звільняючись таким чином
від гіршого. Вони були такими. Їх такими знав увесь світ. Сьогодні вони
змінилися. Дивися, світе, і вчися
%%%cit_comment
%%%cit_title
\citTitle{Українці не розуміють одне одного не через мову, а через небажання слухати, чути і сприймати}, 
Юлія Мендель, www.pravda.com.ua, 07.07.2021
%%%endcit

%%%cit
%%%cit_head
%%%cit_pic
%%%cit_text
Весьма примечательны \emph{слова}, которые прозвучали в Киево-Печерской Лавре во время
круглого стола «Украина в православно-католическом диалоге», где также
присутствовал и кардинал Кох. Тогда представитель УПЦ МП священник Николай
Данилевич заявил: «Сейчас назрела необходимость решать напрямую проблемы,
которые были или остаются у греко-католиков и православных на Украине. То есть
между нами самими, а не в Москве, которая далеко и, возможно, не всегда в курсе
того, что на самом деле происходит»
%%%cit_comment
%%%cit_title
\citTitle{Украина в крови. Новый крестовый поход Католической церкви. Часть 1 / Статьи}, 
Александр Вознесенский, fraza.com, 08.07.2021
%%%endcit

%%%cit
%%%cit_head
%%%cit_pic
%%%cit_text
А это и есть проявление политико-государственного рукоблудия, щедро сдобренного
патриотической \emph{словесной} шелухой и бесплодной географической настойчивостью.
Пыль в глаза. Видимость вместо реальности. \emph{Слова} вместо дел. Игра вместо
действительности. Сомнительное самоутешение вместо получения настоящего
удовольствия и настоящих поводов реально утешиться. И наконец-то начать
работать себе во благо, а не ради каких-то химер и бесплодных мифов.
Но получается пока по-другому: как и заметил как-то хитромудрый и речистый мэр
Киева Виталий Кличко, «лучше рука в небе, чем синица в журавле»…
%%%cit_comment
%%%cit_title
\citTitle{«Синица в журавле». Вокабулярная война Украины за виртуальные границы}, 
Владимир Скачко, ukraina.ru, 26.07.2021
%%%endcit

