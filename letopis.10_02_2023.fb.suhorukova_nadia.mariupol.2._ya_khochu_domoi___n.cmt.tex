% vim: keymap=russian-jcukenwin
%%beginhead 
 
%%file 10_02_2023.fb.suhorukova_nadia.mariupol.2._ya_khochu_domoi___n.cmt
%%parent 10_02_2023.fb.suhorukova_nadia.mariupol.2._ya_khochu_domoi___n
 
%%url 
 
%%author_id 
%%date 
 
%%tags 
%%title 
 
%%endhead 

\qqSecCmt

\iusr{Dana Samoilovich}

Так боляче.

\iusr{Ksenya Kostynskaya}

Ти не бачила. Ось.

\ifcmt
  igc https://scontent-fra3-1.xx.fbcdn.net/v/t39.30808-6/330777288_1178249246136009_3444779054019023918_n.jpg?_nc_cat=103&ccb=1-7&_nc_sid=dbeb18&_nc_ohc=bYXeVlnI1v4AX8DEyuW&_nc_ht=scontent-fra3-1.xx&oh=00_AfClwmXDHh9ey3BZOO3UBWVcFTGaZgd7IjdbzdTViVLgEg&oe=63EC51EC
	@width 0.4
\fi

\begin{itemize} % {
\iusr{Nadia Sukhorukova}
\textbf{Ксенія Костинська} моя Ксенія. Люблю тебе
\end{itemize} % }

\iusr{Yana Berezhna}

Это точно.. как же все точно.. и что с этим делать-неизвестно..

\iusr{Lilia Syniepalova}

Нове коріння... не коріння, а так, волосинки. Просто щоб вистояти і не впасти
тут і зараз. Хоча до того більше ніж пів життя як перекоти-поле. 17 років в
одному місці, 5 в іншому, ще 5 в ще іншому, 9 в ще іншому і майже 18 в
Маріуполі.

От ті місця, де 17 і 18, нині зруйновані вщент. Обидва. Два мої життєві якорі.

\begin{itemize} % {
\iusr{Nadia Sukhorukova}
\textbf{Lilia Syniepalova} все буде Лілія. Ми повернемося. Інакше не може буде.

\iusr{Lilia Syniepalova}
\textbf{Nadia Sukhorukova} тільки так.
\end{itemize} % }

\iusr{Наталия Руденко}

Все буде Україна!!

Повернемо всі території і все відбудуємо!

\iusr{Larisa Kunina}

Да. Всё верно. Я заставляю мужа выходить на улицу. А он мне: Там все чужое! Мне
больно на это смотреть!

\begin{itemize} % {
\iusr{Nadia Sukhorukova}
\textbf{Лариса Кунина} у нас точно также. Только муж меня заставляет
\end{itemize} % }

\iusr{Ann Ann}

Моя психика умнее меня, Надя, мы не знакомы лично, но всегда муж смеялся с
меня, когда я смотрела мариупольское тв и радовалась, когда были Ваши
репортажи)))но, сейчас моя психика заблокувала всё, связанное с родным городом и
родными местами, я Безумно хочу вернуться и очень боюсь, что просто кукуха может
съехать...

\begin{itemize} % {
\iusr{Nadia Sukhorukova}
\textbf{Анна Пикалова} в тот город, который сейчас, нет смысла возвращаться. Он чужой. Наш будет под украинским флагом.

\iusr{Alena Dubrovskaya}
\textbf{Nadia Sukhorukova} 

Когда над городом будет развиваться Украинский флаг, это будет не наш
Мариуполь, наш убили « освободители» увы 😢 Мариуполь - боль, постоянно
кровоточащая рана, он не отпускает меня.

\end{itemize} % }

\iusr{Ольга Орел}
\textbf{Nadia Sukhorukova} У нас с мужем та же история. Живём в сказочном месте в Баварии, заставляем себя выйти из дома. Надя, где Вы находите слова, чтобы описать то, что мы все чувствуем? Спасибо!!!!

\begin{itemize} % {
\iusr{Nadia Sukhorukova}
\textbf{Ольга Орел} это не я их нахожу. Это слова меня находят. Наверное, я чувствую вас
\end{itemize} % }

\iusr{Ann Ann}

Я про тот, что сейчас и не поеду... хотя, там мама и мои коты, и мой взрослый
племянник.... \enquote{мыза мирыши}, но я за ними скучвю и за дачным домом... и за моим
вишнёвым, яблочным и вообще садом... сука, как же хочется домой

\iusr{Ольга Баранова}

Абсолютно знакомые мысли хоть мне повезло несоизмеримо больше других - я
вырвалась из города или, скорее, меня вырвал буквально за шиворот 24-го февраля
мой сын, - как и почему это произошло, - это совсем другая история, но факт
остаётся в том что я не идела весь этот кошмар своими собственными глазами и не
испытала его на себе, видела только бесконечные фото и видео, и, содрогаясь от
ужаса и слёз, не могла поверить что это не какая-то чудовищная фантасмагория, а
жуткая реальность, которую нелюди сотворили в нашем прекрасном городе, и даже
сейчас практически год спустя не могу во всё это поверить хотя умом понимаю что
это правда... И безумно хочу домой в наш Украинский, родной Мариуполь...

\begin{itemize} % {
\iusr{Natalya Kiseleva}

Ольга, та же история... Последним поездом выехала с подругой 24 февраля из
Маруполя по работе, а сын, друзья, знакомые остались в этом кошмаре. А что с
нами что происходило когда из нэта не вылазила и искала, искала знакомые лица,
что б знать что с ними все нормально, и с ужасом смотрела видео происходящего в
городе, волосы дыбом становились от увиденного. Подруга телефон забирала, что б
я хоть час два за сутки поспала. У меня был свой Ад. И я честно говорю,вернусь
домой, в Марик, но не сейчас, боюсь, очень боюсь увидеть, боюсь что реально
\enquote{крыша} уедет. Единственное что на данном этапе меня может туда выдернуть, это
если скажут, что мой сын Вовка там, за него уже 10 месяцев ничего не знаю..

\iusr{Ольга Баранова}
\textbf{Наталья Киселёва-Дедухова} Понимаю и сочувствую от души...

\iusr{Natalya Kiseleva}

Ольга, чувствую, что и с вами то же происходило. Благодарю вас ❤️❤️❤️

\iusr{Ольга Баранова}
\textbf{Наталья Киселёва-Дедухова} 

Абсолютно аналогично было всё со мной, я полтора месяца не имела никакой
информации о своей единственной родной сестре ибо не было с нею связи, утро
начиналось с изучения абсолютно всех списков во всех убежищах во всех мыслимых
и немыслимых группах в поисках нужной фамилии и заканчивалось глубокой ночью
тем же самым до дичайшей рези в глазах от экрана монитора и собственных слёз и
практически чуть ли не слепоты.

\end{itemize} % }

\iusr{Ana Tomsa}

Одно из самых ярких воспоминаний детства - это парк аттракционов в Мариуполе на
Площади Свободы. Год 1997, потом некоторые из них перевезли в Экстрим парк,
если я не ошибаюсь. Я помню, как мы на машине ехали по Мариуполю и я им
восхищалась. Такой красивый и гордый.

Донецк, Ясиноватая, Мариуполь, отдых на Азовском море - все счастливые летние
воспоминания детства превратились в руины.

Очень больно Вас читать, но не читать еще больнее.

\iusr{Алла Алимова}

💔😪😪💔💔

\iusr{Алла Алимова}

100\% точно написано о нас, Мариупольцах. 😪

\iusr{Галина Солдатова}

Если бы только Мариуполь... Сейчас вся Украина в войне, везде прилеты, нет
света, горе, похоронки... Сегодня страшно за всю планету, из за
неадекватов, которые решили снести всю планету.

\iusr{Оксана Демешко}

Як точно сказано ... як же Ви відчуваєте біль маріупольців ... Хоч місто
зруйноване, але там залишилися мої найдорожчі люди ... Моя душа кричить, боліт,
вивертається ... Сльози, біль, страх за рідних, страшні передчуття і сни...Але
Надія.

\iusr{Lyubov Kopyeykina}

А нас вообще далеко судьба закинула- в Австралию к сыну, дети заставили выехать
25 го, на следующий день начались бомбежки! Но душа рвётся назад, там все нам
ближе и роднее, хоть все и снесено. Надя, спасибо вам за точные описания того,
что было и того, что каждый из нас чувствует!

\iusr{Gordana Krutii}

Я хочу тату с координатами дома. Чтобы он просто был рядом, всегда

\begin{itemize} % {
\iusr{Ludmila Kushneruk}
\textbf{Гордана Крутий} я теж! Хочеться частинку дома мати поруч. Просто щоб було! Бо дах скоро поїде!
\end{itemize} % }

\iusr{Natalia Logozinskay Belova}

\ifcmt
  igc https://scontent-fra3-1.xx.fbcdn.net/v/t39.1997-6/851586_126361977548599_392107290_n.png?stp=cp0_dst-png_s110x80&_nc_cat=1&ccb=1-7&_nc_sid=ac3552&_nc_ohc=bAfK_5MxB_0AX9P0Mi2&_nc_ht=scontent-fra3-1.xx&oh=00_AfBJeP9yb0-J6LCCSaDW4hXvVxe8OY-X0LGeQaQ4l8qIqQ&oe=63EC4B53
	@width 0.1
\fi

\iusr{Ольга Вингольц}

Надо Жить вопреки всему. Позволять Себе Жить !!! Да! Тяжело. Я знаю что такое
ужас войны не из новостей . Я и мои дети прошли Ад Мариуполя с февраля по май.
Но Жизнь Продолжается!!! Новая страна, новый язык, новые люди, новые законы и
этот список можно продолжать. Всё новое! Только вот где бы мы не находились мы
всегда находимся на едине с собой. И где бы мы не находились мы несём в этот
Мир все то чем наполнены. Не теряйте и не выключайте Свет в Себе!!! Живите
Любите Радуйтесь Развивайтесь Наполняйтесь Новым и вот со всем этим Мы вернёмся
в Свой Украинский Мариуполь и будем улучшать Его своими новыми знаниями и
умениями!!!!

Мира и Счастья Вам!!!!

Маріуполь це УКРАЇНА!!!

\begin{itemize} % {
\iusr{Тамара Белавина Белавина}
\textbf{Ольга Вингольц} Дожить бы до возвращения Так боюсь умереть не дома

\iusr{Ольга Вингольц}
\textbf{Тамара Белавина Белавина} 

Страх блокирует Жизнь. Не ставьте Себя в блок. Пишите в лс пообщаемся. Мира и
Счастья Вам!

\iusr{Тамара Белавина Белавина}
\textbf{Ольга Вингольц} 

Я реалистом была всегда Когда тебе за 70 и ты каждое утро просыпаешься с
чувством что до обеда можешь не дожить- то это не страх Это трезвая оценка
вещей и событий Молодые же как саженцы Их можно вырвать с корнем и пересадить
Переболеют и приживутся У старых корни не отростают Но очень хочется дожить до
возвращения!! Очень !!

\iusr{Ольга Вингольц}
\textbf{Тамара Белавина Белавина} Желаю Вам ЖИТЬ и Дожить!!!

\iusr{Тамара Белавина Белавина}
\textbf{Ольга Вингольц} Спасибо!!!!! Хоть ползком но домой!
\end{itemize} % }

\iusr{Людмила Момот}

Надежда, я рада, что Вы вернулись в Фейсбук, не знакома с Вами лично, но
ощущение что знаю Вас всю жизнь... Все слова в точку, правильнее и не
выразиться... Спасибо Вам, с Вами легче!!!!

\iusr{Оксана Спивак}
💖🙌🙏🙏🙏✌

\ifcmt
  igc https://scontent-fra3-1.xx.fbcdn.net/v/t39.30808-6/329777711_1205045480386151_8856449180015137932_n.jpg?_nc_cat=111&ccb=1-7&_nc_sid=dbeb18&_nc_ohc=De2VCrKHIoYAX95YN1E&_nc_ht=scontent-fra3-1.xx&oh=00_AfByxXW7O0sYRi6jdR52XGm1ZKwmlJeUm1hSJHpS-bn8-Q&oe=63EC2848
	@width 0.7
\fi

\iusr{Наталия Склярова}

Сто процентов я верю в то что он будет Украинским, пусть не завтра но будет и
мы приедем всё заново отстраивать и наводить там наш порядок, чтоб даже и не
воняло этой гнилью русской !!!!

\iusr{Татьяна Павлюченко}

\ifcmt
  igc https://i.paste.pics/8b91e5a5a1303b1cbcfc45ff26ca3dfd.png
	@width 0.2
\fi

\iusr{Наташа Дирко}

😥😥😥😥😥🥺🥺🥺🥺🥺😢😢😢😢😢💔💔💔💔💔💔💔💔

\iusr{Nataliia Volkova}

Все так...

\iusr{Татьяна Жидкова}

\ifcmt
  igc https://scontent-fra3-1.xx.fbcdn.net/v/t39.1997-6/47270791_937342239796388_4222599360510164992_n.png?stp=cp0_dst-png_s110x80&_nc_cat=1&ccb=1-7&_nc_sid=ac3552&_nc_ohc=PK3BGAKvqoQAX9ZAnhl&_nc_ht=scontent-fra3-1.xx&oh=00_AfB3KAWuS1tadAwNggM-0stLaB9r9vi7SS_FAeSOMlyTIQ&oe=63EC64B2
\fi

\iusr{Надежда Заикина}

Жизнь разделилась на до и после, очень хочется в Мариупольно что там, страшно
подумать, руины, мы пожилые люди часто жаловались что тяжеловато жить и только
когда прихо)ится жить на квартире и своего ничего нет понимаешь как жили
счастливо, ездили на море, гуляли в парках и ждешь, ждешь мира и возвращения и
очень страшно, как будет, а когда смотришь фото Мариуполядо войны и после слезв
и боль не затихают

\iusr{Natalya Kiseleva}

Надежда, благодарю, что вы есть, что вы с нами. Вы как лучик света из нашего родного Мариуполя❤️

\begin{itemize} % {
\iusr{Nadia Sukhorukova}
\textbf{Наталья Киселёва-Дедухова} спасибо огромное Вам!
\end{itemize} % }

\iusr{Лина Колесниченко}

Як ви все точно передали, наче залізли кожному в душу і прожили його життя...

Наш Маріуполь на цих фото як ніколи величний, аж дух перехоплює...

Щира вам подяка від нас!

\begin{itemize} % {
\iusr{Nadia Sukhorukova}
\textbf{Лина Колесниченко} до зустрічі у нашому українському Маріуполі

\iusr{Лина Колесниченко}
\textbf{Nadia Sukhorukova} Тааааак!!!!!
\end{itemize} % }

\iusr{Елла Мусаєва}

Нужно жить, Не вопреки, а для настоящего, для будущего. Для кого-то это уже
чужой город. Для меня он всегда мой, особенно сейчас, раненный, с дырками в
домах и в сердце. И я хочу туда вернуться, потому что после войны ему, родному,
необходима моя помощь. Было страшно, и сейчас страшно - не за себя, за детей.
Вчера были прилеты, и ты снова звонишь, пишешь детям, снова в надежде, что с
ними все в порядке. И мысленно звонишь ещё одному: \enquote{Марик, как ты?...}

\begin{itemize} % {
\iusr{Nadia Sukhorukova}
\textbf{Елла Мусаєва} а на мои сообщения он не отвечает...

\iusr{Елла Мусаєва}
\textbf{Nadia Sukhorukova} Слова замечательного героя Леонида Быкова (Титаренко), нашего земляка: \enquote{Будем жить!}
\end{itemize} % }

\iusr{Валентина Стебалова}

Меня выкопали из земли, где я прекрасно себя чувствовала и посадили в другую
землю, где я и не принимаюсь и не умираю, просто чахну.

\iusr{Yana Aborneva}
😪😪

\iusr{Nina Morozova-Vuitsyk}

А моя психика не хочет возвращаться к мирной, чужой жизни. Все ходят весело
смеются, чему то радуются, а меня душат слёзы и один вопрос « за что, куда
деваться.» Ведь все сгорело, жизнь замерла. И каждую минуту думаешь «Куда
ехать, где жить, здесь ты чужой, никому не нужный, живешь на квартире, и
помереть страшно от тоски и страха, что это все, конец,»...

\iusr{Kostishin Irina}

\ifcmt
  igc https://scontent.xx.fbcdn.net/v/t39.1997-6/105941685_953860581742966_1572841152382279834_n.png?stp=dst-png_p320x320&_nc_cat=1&ccb=1-7&_nc_sid=0572db&_nc_ohc=BV4HRjtQ6NwAX-7tCt3&_nc_ad=z-m&_nc_cid=0&_nc_ht=scontent.xx&oh=00_AfBx9-aEZRMh6CuaCLSh8N1tDMnEqzrCEZSlwFW5dmo05g&oe=63EC8502
	@width 0.1
\fi
