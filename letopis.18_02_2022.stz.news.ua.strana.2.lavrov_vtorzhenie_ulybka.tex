% vim: keymap=russian-jcukenwin
%%beginhead 
 
%%file 18_02_2022.stz.news.ua.strana.2.lavrov_vtorzhenie_ulybka
%%parent 18_02_2022
 
%%url https://strana.news/news/377560-lavrov-prokommentiroval-fejkovye-novosti-ssha-o-vtorzhenii-rossii-v-ukrainu.html
 
%%author_id romanova_kristina
%%date 
 
%%tags lavrov_sergei,rossia,ugroza,ukraina,vtorzhenie
%%title Лавров заявил, что прогнозы Запада о "вторжении России в Украину" вызывают улыбку
 
%%endhead 
 
\subsection{Лавров заявил, что прогнозы Запада о \enquote{вторжении России в Украину} вызывают улыбку}
\label{sec:18_02_2022.stz.news.ua.strana.2.lavrov_vtorzhenie_ulybka}
 
\Purl{https://strana.news/news/377560-lavrov-prokommentiroval-fejkovye-novosti-ssha-o-vtorzhenii-rossii-v-ukrainu.html}
\ifcmt
 author_begin
   author_id romanova_kristina
 author_end
\fi

Министр иностранных дел Российской Федерации Сергей Лавров сообщил о том, что
прогнозы Запада о том, когда надо ждать \enquote{вторжения} России в Украину,
вызывают улыбку.

Об этом он заявил в интервью телеканалу \enquote{RT}.

\enquote{Это (заявления Запада о \enquote{вторжении} - Ред.) вызывает улыбку.
Взрослые люди ежедневно \enquote{извергают} прогнозы, заклинания }, - сказал
Лавров. 

\ii{18_02_2022.stz.news.ua.strana.2.lavrov_vtorzhenie_ulybka.pic.1}

Глава российского дипведомства также напомнил, что издание \enquote{Politico}
уже несколько раз меняло дату \enquote{вторжения}.

\enquote{Теперь анонсирует 20 февраля. Одновременно глава Форин офиса Элизабет
Трасс сказала, что \enquote{вторжение} может последовать в любой момент в
течение многих месяцев. Они уже \enquote{отбивают} себе перспективу.
Государственный секретарь США Энтони Блинкен заявил, что \enquote{вторжение} -
вопрос недель, а может быть и дней.  Знаем, что Госдеп доверительно
проинформировал своих союзников по НАТО, что надо ждать \enquote{вторжения} до
конца февраля}, - добавил Лавров.

По его словам, генеральный секретарь альянса Йенс Столтенберг при этом
заявляет, что у НАТО нет данных о стопроцентной вероятности
\enquote{нападения}.

\enquote{Параллельно форин офис заявляет: \enquote{Что бы ни происходило, надо
быть настороже}.  Мол, даже если Россия отведет войска, угроза не пропадет. Они
закладывают \enquote{кирпичики} на будущее, - пояснил глава МИД РФ. -
Закончатся учения - войска вернутся в казармы. Процесс уже идет, они это
видели, но \enquote{забивают} тему на \enquote{веки вечные}: \enquote{Даже если
Россия вернет свои войска туда, где они постоянно дислоцируются, угроза все
равно исходит}}, - сказал глава МИД РФ.

Напомним, западные СМИ анонсировали день \enquote{вторжения} на 16 февраля,
однако оно не произошло. После этого в Госдепе США заявили, что Россия может
начать \enquote{вторжение в Украину} в любой момент.

Также мы писали, что в Белом доме заявили, что Россия увеличила военное
присутствие вдоль границы с Украиной на 7 000 человек.

Олеся Медведева в блоге \enquote{Ясно. Понятно} рассказывала, как в Украине
оправдывают фейки Запада о \enquote{вторжении России}. 

