% vim: keymap=russian-jcukenwin
%%beginhead 
 
%%file 23_01_2022.stz.kiev.bigkyiv.1.leontovich_murder
%%parent 23_01_2022
 
%%url https://bigkyiv.com.ua/stik-krovyu-sogodni-vypovnyuyetsya-101-rik-vid-dnya-vbyvstva-tvorczya-shhedryka
 
%%author_id lisichkіna_ljubava
%%date 
 
%%tags kultura,leontovych_mykola,murder,muzyka,schedryk,smert,ukraina
%%title Стік кров‘ю. Сьогодні виповнюється 101 рік від дня вбивства творця «Щедрика»
 
%%endhead 
 
\subsection{Стік кров‘ю. Сьогодні виповнюється 101 рік від дня вбивства творця «Щедрика»}
\label{sec:23_01_2022.stz.kiev.bigkyiv.1.leontovich_murder}
 
\Purl{https://bigkyiv.com.ua/stik-krovyu-sogodni-vypovnyuyetsya-101-rik-vid-dnya-vbyvstva-tvorczya-shhedryka}
\ifcmt
 author_begin
   author_id lisichkіna_ljubava
 author_end
\fi

\begin{zznagolos}
Україна подарувала світу один із найважливіших символів Різдва — «Щедрик»
Миколи Леонтовича
\end{zznagolos}

\ifcmt
  ig https://bigkyiv.com.ua/wp-content/uploads/2022/01/ifvfivifv-800x483-1.jpg
  @wrap center
  @width 0.7
\fi

Нагадала про сумну дату співачка та благодійниця Анжеліка Рудницька.
 
101 рік тому вдосвіта 23 січня 1921 року чекіст вистрелив у Миколу Леонтовича.
Це сталося у хаті Леонтовича батька, куди Микола приїхав у гості, а
мерзотник-чекіст попросився переночувати. Переночував. Зранку зробив свою чорну
справу, пограбував Леонтовича і зник.  43-літній Микола стік кров‘ю, його не
змогли врятувати.

За словами Рудницької, у цей час світ аплодував «Щедрику», західна преса
співала Миколі Леонтовичу осанну. «Щедрик» став світовим хітом, а у композитора
був шанс поїхати у Європу і прожити довге життя у комфорті і багатстві. Проте
він вибрав Україну з її вічними проблемами, зневагою до своїх геніїв і злою
сусідкою росією, що краде і вбиває все найкраще українське.
 
«Минуло сто років. Україна – незалежна держава. З «мільйоном» інституцій,
державних і приватних, а про Леонтовича і досі мало знають українці, його музей
у селі вкрай скромний, пам‘ять про нього береже жменька небайдужих людей, а
його «Щедрик» досі співає увесь світ, він звучить в «Сам вдома», «Гаррі
Поттер», «Сімейка Адамсів», «Міцний горішок 2» і т.д»,- зауважила співачка.
 
Нагадаємо, в Києві у Довженко-Центрі відкрили  мультимедійну виставку про шлях
української мелодії від фольклорної пісні до відомого на весь світ різдвяного
гімну.
 
Також, цього року відкрили пам’ятну дошку Олександру Кошицю, який представив
світові \enquote{Щедрика} (ФОТО, ВІДЕО).
