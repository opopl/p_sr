% vim: keymap=russian-jcukenwin
%%beginhead 
 
%%file 02_12_2020.news.ru.lenta_ru.mozzhuhin_andrei.1.pribaltika.lesnyje_bratja
%%parent 02_12_2020.news.ru.lenta_ru.mozzhuhin_andrei.1.pribaltika
 
%%url 
 
%%author 
%%author_id 
%%author_url 
 
%%tags 
%%title 
 
%%endhead 
\subsubsection{Лесные братья}

\lenta{В Вильнюсе в бывшем здании управления КГБ Литовской ССР есть Центр исследования
геноцида и резистенции жителей Литвы, отделом которого является музей оккупаций
и борьбы за свободу. Судя по представленной на его сайте экспозиции, под
геноцидом там подразумевают страдания литовцев от нацистов и особенно от
Советов. А трагедии холокоста, когда литовцы под немецким руководством убивали
евреев, отведена лишь скромная небольшая комната. Но разве не об этом геноциде
в первую очередь нужно рассказывать?}

Рассказывать нужно обо всем. И делать это честно. В Латвии и Эстонии тоже есть
«музеи оккупации».
