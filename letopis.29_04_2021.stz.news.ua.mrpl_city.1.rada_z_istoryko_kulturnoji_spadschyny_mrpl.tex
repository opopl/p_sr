% vim: keymap=russian-jcukenwin
%%beginhead 
 
%%file 29_04_2021.stz.news.ua.mrpl_city.1.rada_z_istoryko_kulturnoji_spadschyny_mrpl
%%parent 29_04_2021
 
%%url https://mrpl.city/blogs/view/rada-z-istoriko-kulturnoi-spadshhini-mariupolya
 
%%author_id demidko_olga.mariupol,news.ua.mrpl_city
%%date 
 
%%tags 
%%title Рада з історико-культурної спадщини Маріуполя
 
%%endhead 
 
\subsection{Рада з історико-культурної спадщини Маріуполя}
\label{sec:29_04_2021.stz.news.ua.mrpl_city.1.rada_z_istoryko_kulturnoji_spadschyny_mrpl}
 
\Purl{https://mrpl.city/blogs/view/rada-z-istoriko-kulturnoi-spadshhini-mariupolya}
\ifcmt
 author_begin
   author_id demidko_olga.mariupol,news.ua.mrpl_city
 author_end
\fi

\ii{29_04_2021.stz.news.ua.mrpl_city.1.rada_z_istoryko_kulturnoji_spadschyny_mrpl.pic.1}

28 квітня з ініціативи міського голови Маріуполя \emph{Вадима Сергійовича Бойченка}
була створена \href{https://mrpl.city/news/view/v-mariupole-otrestavriruyut-40-obektov-kulturnogo-naslediyavadim-bojchenko}{%
\emph{Рада з історико-культурної спадщини Маріуполя}}.%
\footnote{В Мариуполе отреставрируют 40 объектов культурного наследия, – Вадим Бойченко, Олександра Невська, mrpl.city, 29.04.2021, \par%
\url{https://mrpl.city/news/view/v-mariupole-otrestavriruyut-40-obektov-kulturnogo-naslediyavadim-bojchenko}
}
Оскільки я брала безпосередню участь в цьому процесі, хочу більш докладно
розповісти про цю подію. Після ситуації, пов'язаної з Міським садом і з метою
уникнення подібних протиріч мер наголосив, що хоче створити новий орган, куди
увійдуть люди, які присвятили своє життя вивченню історії міста Маріуполя та
збереженню його архітектурно-культурної спадщини. Вони ж увійдуть і до
містобудівної ради, щоб також давати свої рекомендації з точки зору
поверховості забудови та об'єктів, які мають архітектурну цінність для міста.
Представники ради системно займатимуться питаннями збереження пам'яток
архітектури та об'єктів культурної спадщини. Водночас говорили і про створення
історико-опорного плану, який затвердили на рівні області та міністерства. Він
включає 75 об'єктів історико-культурної спадщини національного та місцевого
значення. З них виділили 40 об'єктів, які мають особливу цінність для нашого
міста та вимагають термінового втручання для їхнього збереження. Саме ці
об'єкти будуть представлені в рамках програми \emph{\enquote{Велика реставрація}},
ініційованої президентом України Володимиром Зеленським. Як відомо, ця програма
ще не запущена, але в бюджеті на неї вже закладено фінансування в розмірі 1,5
мільярда гривень.

Важливо, що під час створення Ради з історико-культурної спадщини Маріуполя
було вирішено розпочати процес створення музея просто неба – \emph{\textbf{\enquote{Маріупольського
світлого некрополя}}}, який стане унікальним туристичним магнітом Маріуполя. Мер
наголосив, що історія повинна бути очищена  від політики, залишатися мають лише
факти. До того ж Вадим Бойченко зазначив, що музеї сьогодні повинні відповідати
запитам сучасного суспільства, тому їхня робота вимагає поновлення. Разом з тим
в Маріуполі очікується створення масштабного музейного проєкту \emph{\enquote{Порт Культур}},
над яким працювали 4 роки. Саме в 2021 році міський голова розраховує отримати
фінансову підтримку від європейських партнерів на його будівництво.

Окремо хочу розповісти про тих маріупольців, які увійшли до складу Ради з
історико-культурної спадщини Маріуполя. Це україн\hyp{}ський громадський діяч,
журналіст, голова громадської організації \emph{\enquote{Архі-Місто}} \emph{\textbf{Андрій Марусов}}. Саме
завдяки його та волонтерським  зусиллям вже рік проводилася робота із
збереження та упорядкування маріупольського некрополя. Членкинею ради стала і
\emph{\textbf{Олена Дейниченко}} завідувачка відділу інформаційних технологій та комп'ютерного
забезпечення бібліотеки Маріупольського державного університету та відома
краєзнавиця міста. Завдяки краєзнавчим прогулянкам з Оленою Василівною
Дейниченко у багатьох містян значно виріс інтерес до історії та архітектури
Маріуполя. До складу ради також увійшла і \emph{\textbf{Чапні Надія}} – голова Маріупольської
грецької спільноти.  За 24 роки Надією Андріївною проведена дуже велика робота
з багатьох напрямів, пов'язаних з відродженням культурно-історичних коренів
грецької діаспори України. Радію, що і я стала частиною Ради з
історико-культурної спадщини Маріуполя, адже одним з найголовніших векторів в
моїй діяльності завжди залишалася історія та культура міста, збереження
архітектури, проведення екскурсій, популяризація міської історії, організація
та участь у загальноміських проєктах, присвячених цим питанням. Водночас
діяльність громадської організації \enquote{Фонд збереження культурної спадщини
Маріуполя} (є заступницею голови) цілком і повністю відповідає меті та
завданням новоствореної ради. На жаль, вчора на зустріч не змогли прийти
відомий маріупольський мистецтвознавець \emph{\textbf{Олександр Чернов}} та єдиний професійний
археолог Маріуполя \emph{\textbf{В'ячеслав Забавін}}, які теж стануть членами першого в
Маріуполі історичного органу. Важливо, що Рада не є 

закритою і склад її може розширюватися.

Докладніше про те, які саме архітектурні  об'єкти було вирішено представити у
програмі \enquote{Велика реставрація} та якими проєктами займатиметься Рада з
історико-культурної спадщини Маріуполя найближчим часом розповім трохи згодом.
