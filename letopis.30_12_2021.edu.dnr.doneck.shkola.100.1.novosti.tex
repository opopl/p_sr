% vim: keymap=russian-jcukenwin
%%beginhead 
 
%%file 30_12_2021.edu.dnr.doneck.shkola.100.1.novosti
%%parent 30_12_2021
 
%%url https://moy100donetsk.wixsite.com/moy100/lenta-novostej-inaya-informaciya
 
%%author_id edu.dnr.doneck.shkola.100
%%date 
 
%%tags donbass,dnr,doneck,shkola,obrazovanie,ucheba,novosti
%%title ЛЕНТА НОВОСТЕЙ
 
%%endhead 
\subsection{ЛЕНТА НОВОСТЕЙ}
\label{sec:30_12_2021.edu.dnr.doneck.shkola.100.1.novosti}

\Purl{https://moy100donetsk.wixsite.com/moy100/lenta-novostej-inaya-informaciya}
\ifcmt
 author_begin
   author_id edu.dnr.doneck.shkola.100
 author_end
\fi

\headTwo{04.10.2021г. Помощь бездомным животным.}

\ii{30_12_2021.edu.dnr.doneck.shkola.100.1.novosti.pic.1}

Учащиеся школы приняли участие в акции \enquote{Всем на свете нужен дом}, собрав корма,
лекарства, моющие средства и тёплые вещи для приюта бездомных животных \enquote{ПИФ}

Все было передано приюту, руководство которого поблагодарило  учащихся за это.

\headTwo{22.09.2021г. Готов к труду и обороне}

С 16 по 22 сентября учащиеся нашей школы приняли участие в Фестивале,
посвященном 5-летию Государственного физкультурно-спортивного комплекса «Готов
к труду и обороне» Донецкой Народной Республики

Ученики 2-11 выразили желание сдать необходимые нормативы. Для этого они
испытали себя в беге на 30 метровой дистанции, выполняли прыжки в длину с
места, наклон вперед, сгибание и разгибание рук, подтягивание, поднимание
туловища за 30 секунд.

\ii{30_12_2021.edu.dnr.doneck.shkola.100.1.novosti.pic.2}
\ii{30_12_2021.edu.dnr.doneck.shkola.100.1.novosti.pic.3}

\headTwo{28.05.2021г. Последний звонок}

28 мая в нашей школе прозвенел Последний звонок. Выпускников поздравили с новым
этапом жизни, а также наградили лучших учеников грамотами и похвальными
листами. 

Выпускники исполнили поздравительные номера и поблагодарили своих учителей.

\ii{30_12_2021.edu.dnr.doneck.shkola.100.1.novosti.pic.4}
\ii{30_12_2021.edu.dnr.doneck.shkola.100.1.novosti.pic.5}

\headTwo{25.05.2021г. Наши ученики посетили пожарную часть с экскурсией}

Юнармейцы ВПК "Юность "МОУ \enquote{Школа 100 г. Донецка} Донецкого территориального
штаба ОО \enquote{ВПД МОЛОДАЯ ГВАРДИЯ-ЮНАРМИЯ}" совместно с юнармейцами Донецкого
территориального штаба приняли участие в увлекательной экскурсии в пожарную
часть номер 7 Куйбышевского района г. Донецка.

Юнармейцы увидели как живет пожарная часть - как профессионально работают и как
отдыхают пожарные.

Экскурсию провел начальник части Калинский Р.Б., который  рассказал и показал
ребятам машины и пожарное оборудование. 

\ii{30_12_2021.edu.dnr.doneck.shkola.100.1.novosti.pic.6}

\headTwo{19.05.2021г. Ученики школы приняли участие в открытом уроке, посвященном Дню Пионерии}

19 мая ученики школы № 100 и  юнармейцы ВПК \enquote{Юность} МОУ \enquote{Школа
100 г. Донецка} Донецкого территориального штаба ОО \enquote{ВПД МОЛОДАЯ
ГВАРДИЯ-ЮНАРМИЯ} провели урок, посвящённый Дню Пионерии.

Юнармейцы рассказали школьникам о том, как создавалось пионерское движение,
показали атрибутику советских пионеров. Также рассказали о пионерском галстуке
и научили повязывать его.

Ребята отметили, что Пионерия тогда и Юнармия сейчас -  яркий пример связи
прошлого и нынешнего поколений.

\ii{30_12_2021.edu.dnr.doneck.shkola.100.1.novosti.pic.7}

\headTwo{17.05.2021г. Наши ученики приняли участие в соревнованиях \enquote{Веселые старты}}

Ученики МОУ \enquote{Школа №100} приняли участие в районных соревнованиях \enquote{Веселые
старты} и заняли I место. Соревнования проводились на базе нашей школы.

Поздравляем наших ребят с победой!

\ii{30_12_2021.edu.dnr.doneck.shkola.100.1.novosti.pic.8}

\headTwo{15.05.2021г. Ученица школы стала призером в Республиканском фестивале}

Ученица 6-А класса и юнармеец ВПК \enquote{ЮНОСТЬ} МОУ \enquote{Школа 100 г.
Донецка} Донецкого территориального штаба ОО \enquote{ВПД МОЛОДАЯ
ГВАРДИЯ-ЮНАРМИЯ} Шишлова Александра приняла участие в открытом Республиканском
фестивале боевых искусств на кубок Главы администрации города Макеевки по джиу
джитсу.

Соревнования проходили в МОУ \enquote{ШКОЛА 49 г. Макеевки}

Саша очень хорошо выступила на турнире, где заняла lll место. Поздравляем! Молодец!

\ii{30_12_2021.edu.dnr.doneck.shkola.100.1.novosti.pic.9}

11.05.2021г. День Республики

11 мая юнармейцы ВПК \enquote{Юность} МОУ \enquote{Школа 100 г. Донецка} Донецкого
территориального штаба ОО \enquote{ВПД МОЛОДАЯ ГВАРДИЯ-ЮНАРМИЯ} и работники школы №100
в национальных костюмах приняли участие в праздничном шествии, посвящённом Дню
Республики.

С праздником, Донецкая Народная республика!!!

\ii{30_12_2021.edu.dnr.doneck.shkola.100.1.novosti.pic.10}
\ii{30_12_2021.edu.dnr.doneck.shkola.100.1.novosti.pic.11}
\ii{30_12_2021.edu.dnr.doneck.shkola.100.1.novosti.pic.12}

\headTwo{09.05.2021г.}  9 мая юнармейцы ВПК \enquote{Юность} МОУ \enquote{Школа 100}
Донецкого территориального штаба ОО \enquote{ВПД МОЛОДАЯ ГВАРДИЯ-ЮНАРМИЯ}
приняли участие в Параде Победы.

На главной площади Донецка ровными шеренгами прошагали наследники героев
Великой Отечественной войны, пронеся в своих рядах историческую память и
героизм своих предков, гордость и патриотизм своего народа, победившего в той
войне, стойкость несломленного и непокоренного Донбасса.

76 лет прошло с той Великой даты, со Дня Победы советского народа над фашизмом,
но память молодых, память новых поколений осталась незыблемой и она останется
вечной!

Наше Движение \enquote{МОЛОДАЯ ГВАРДИЯ-ЮНАРМИЯ} - ярчайший пример сохранения
этой Великой Памяти!  Пока мы помним, мы живем!

\ii{30_12_2021.edu.dnr.doneck.shkola.100.1.novosti.pic.13}

Подробнее о праздновании Дня Победы тут: 

\url{https://moy100donetsk.wixsite.com/moy100/9-мая}

\ii{30_12_2021.edu.dnr.doneck.shkola.100.1.novosti.pic.14}

\begin{zzquote}
Здравствуй, солдат!

Пишу тебе из когда-то мирного, красивого города миллиона роз. Мы читали о ваших
подвигах в книгах, мы смотрели фильмы о Великой Отечественной войне. На уроках
истории нам рассказывали о вашем мужестве, храбрости и отваге. Но несколько лет
назад мы, также как и вы, испытали на себе все ужасы войны. Войны, в которой
брат пошел на брата, а матери оплакивали погибших сыновей с обеих сторон.

Скажи, солдат, ты видел много крови? Ты слышал свист пуль и летящих снарядов?
Ты долго помнил запах пороха? Мне, как и тебе, довелось это услышать и увидеть.
Мы тоже не выбирали для себя такую судьбу. Но если вы с гордостью защищали нашу
Родину на передовой, то нам, детям Донецка, приходилось прятаться в подвалах,
не понимая, кто и за что в нас стреляет. Когда на наших улицах гремели снаряды,
моя мама всегда успокаивала меня. Она говорила, что души погибших солдат – это
небесные ангелы, которые охраняют и оберегают нашу землю. И я верю, что среди
этих ангелов есть и ты – наш герой и защитник!

Дорогой солдат, знай: твой подвиг не забыт и живет в наших сердцах. Мы свято
чтим память всех, кто защищал наше Отечество тогда и сейчас. Мы, молодое
поколение, будем не только помнить ваши подвиги, но и сами, если потребуется,
готовы стать плечом к плечу на защиту нашей Родины. Я верю, что, как и вы,
тогда в сорок пятом мы поднимем и гордо пронесем победоносное знамя.

Спи спокойно, воин! Мы продолжим твое правое дело! Мы выстоим, мы победим!

Ученица 9-А класса

МОУ «Школа №100 г. Донецка»

Виник Анна
\end{zzquote}

\ii{30_12_2021.edu.dnr.doneck.shkola.100.1.novosti.pic.15}
\ii{30_12_2021.edu.dnr.doneck.shkola.100.1.novosti.pic.16}
\ii{30_12_2021.edu.dnr.doneck.shkola.100.1.novosti.pic.17}

\headTwo{07.05.2021г. Наши ученики приняли участие в праздничном концерте, посвященном Дню Победы}

Ученики школы приняли участие в праздничном концерте, посвящённом Дню Победы.

В мероприятии были задействованы все классы школы. Были и песни, и танцы, и
театральные постановки.

На праздник были приглашены самые маленькие гости - воспитанники ДС
\enquote{Щелкунчик}, исполнившие знаменитую \enquote{Катюшу}.

В конце мероприятия все участники концерта исполнили песню \enquote{День
Победы}, выстроившись в звезду.

\ii{30_12_2021.edu.dnr.doneck.shkola.100.1.novosti.pic.18}
\ii{30_12_2021.edu.dnr.doneck.shkola.100.1.novosti.pic.19}
\ii{30_12_2021.edu.dnr.doneck.shkola.100.1.novosti.pic.20}

\headTwo{07.05.2021г. Наши ученики почтили память партизан.}

Сегодня ученики нашей школы почтили память партизан ВОВ в Петровском районе г.
Донецка и возложили цветы к памятной стеле.

\ii{30_12_2021.edu.dnr.doneck.shkola.100.1.novosti.pic.21}

\headTwo{06.05.2021г. Единый час, посвященный годовщине Великой Победы}

6 апреля в МОУ \enquote{Школа №100} был проведен единый час \enquote{75 лет в
Великой Отечественной войне}.

\ii{30_12_2021.edu.dnr.doneck.shkola.100.1.novosti.pic.22}
\ii{30_12_2021.edu.dnr.doneck.shkola.100.1.novosti.pic.23}

\headTwo{6.05.2021г.  Юнармейцы нашей школы приняли участие в Параде Победы}

6 мая юнармейцы ВПК \enquote{Юность} МОУ \enquote{Школа № 100 г. Донецка}
Донецкого территориального штаба ОО \enquote{ВПД МОЛОДАЯ ГВАРДИЯ-ЮНАРМИЯ}, ВПК
\enquote{Патриот} МОУ \enquote{Школа101} совместно с обучающимися школ
Петровского района приняли участие в праздничном параде, посвящённом Дню
Победы.

Юнармейцы прошли торжественным маршем по площади Петровского, приняли участие в
возложении цветов к памятнику и поздравлении ветеранов ВОВ, а также в шествии
Бессмертного полка, почтив память погибших солдат.

На параде присутствовала военная техника.

Перед ДК Петровского на площади более ста учащихся приняли участие во флешмобе
\enquote{Вальс Победы}.

На мероприятии присутствовали - Глава Петровской районной администрации
Бегуненко Н.И, заместитель Главы - Плахутин Ю.И., руководитель спортивно-
патриотического сектора района Исаев Э.Н., ветераны, жители района.

\ii{30_12_2021.edu.dnr.doneck.shkola.100.1.novosti.pic.24}

\headTwo{6.05.2021г. Ученики выпустили стенгазету в преддверии Дня Победы}

Ученики нашей школы в преддверии Дня Победы украсили праздничный баннер и
выпустили стенгазету \enquote{Культура фронту}, рассказали о поэтах и музыкантах,
участвующих во фронтовых концертных бригадах.

Эти бригады во время Великой отечественной войны поднимали боевой дух бойцов,
приближая нашу победу.

\ii{30_12_2021.edu.dnr.doneck.shkola.100.1.novosti.pic.25}

\headTwo{5.05.2021г. Нашу школу посетила многократная чемпионка и рекордсменка России, Европы и мира Марьяна Наумова}

«Нужно делиться успехом», - с этих слов начала свое обращение к ученикам нашей
школы «сильнейшая девочка планеты» Марьяна Наумова. Своим примером она показала
школьникам, что мечты осуществимы, а кумир ее и ее отца, Арндольд Шварценегер
заметил простую девушку из небольшого подмосковного города, благодаря ее силе и
упорству.

Вместе с Марьяной встречу посетила ее младшая сестра Александра, которая также
имеет награды в этом виде спорта. Кроме того, на мероприятии присутствовал
представитель Федерации ДНР по стрит-воркауту и показал несколько трюков..

Марьяна Наумова привезла в качестве подарка школе теннисный стол, ракетки и
шарики к нему. Кроме того, гости провели среди учеников турнир по русскому жиму
в память об А.В. Захарченко.

Все участники турнира получили призы в виде протеиновых коктейлей и печенья. А
победителям Марьяна решила вручить супер-призы в виде смартфонов. \enquote{Увидев вас,
ребята, именно в вашей школе я решила сделать дополнительные подарки
победителям}, - прямо посреди мероприятия заявила Марьяна.

Победительницей среди девочек стала ученица 6-А Александра Шишлова, а среди
парней победу одержал Виталий Карлин, ученик 11 класса.

\ii{30_12_2021.edu.dnr.doneck.shkola.100.1.novosti.pic.26}
%\ii{30_12_2021.edu.dnr.doneck.shkola.100.1.novosti.pic.27}
