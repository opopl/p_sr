% vim: keymap=russian-jcukenwin
%%beginhead 
 
%%file slova.mif
%%parent slova
 
%%url 
 
%%author 
%%author_id 
%%author_url 
 
%%tags 
%%title 
 
%%endhead 
\chapter{Миф}

%%%cit
%%%cit_head
%%%cit_pic
%%%cit_text
Почему стоит учитывать \emph{мифологию} противника всерьёз.  Ни советское, ни
британское, ни французское руководство не учитывала гитлеровскую
\emph{мифологию}. Они искали за этим безумием какую-то логику, и оказались в
плену своего \emph{мифа} о противнике. В свою очередь Гитлер тоже полагался на
\emph{миф}, всерьёз ожидая быстрого коллапса Советского Союза, игнорируя и
историю, и реальность. Особенно учитывая, что в его планах, которые стали
приводится в действие моментально, жизнь местного населения не принималась во
внимание. Германии требовалось вечно кого-то жрать, чтобы выжить в том виде, в
котором её создали нацисты. По другому у них просто не получалось из-за их
\emph{мифологии}
%%%cit_comment
%%%cit_title
\citTitle{Не так страшны войны за свои убеждения, как свои заблуждения}, 
Юрий Романенко, strana.ua, 23.06.2021
%%%endcit

