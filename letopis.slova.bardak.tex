% vim: keymap=russian-jcukenwin
%%beginhead 
 
%%file slova.bardak
%%parent slova
 
%%url 
 
%%author 
%%author_id 
%%author_url 
 
%%tags 
%%title 
 
%%endhead 
\chapter{Бардак}
\label{sec:slova.bardak}

%%%cit
%%%cit_head
%%%cit_pic
\ifcmt
tab_begin cols=3
	width 0.3

  pic https://img.strana.ua/img/article/3409/ukrzaliznytsja-popala-v-48_main.jpeg

	pic https://strana.ua/img/forall/u/0/36/2021-06-29_10h32_06.png

	pic https://strana.ua/img/forall/u/0/36/datchanin-kotoryj-myl-67_main.jpg
tab_end
\fi
%%%cit_text
На днях состоянию подвижного состава ужаснулась известная украинская актриса
Ольга Сумская, которой пришлось проехаться в грязном и душном купейном вагоне
без воды. Народная артистка призвала пассажиров не молчать, а писать и снимать
на видео происходящее, чтобы \enquote{вместе изменить этот \emph{бардак}}.
\enquote{\emph{Бардак}}, заметим, происходит на фоне \enquote{реформ}, за которыми
присматривают иностранцы. Результаты этих реформ выдающимися не назовешь - с
учетом миллионных зарплат, которые получает менеджмент и наблюдательный совет.
Члены правления \enquote{Укрзализныци} получают в среднем по девять миллионов
за год.  Вдвое меньше этот показатель в Набсовете, где заседают иностранцы. Но,
учитывая, что для них эта работа не связана с операционным управлением
компанией, где они вообще редко появляются, то и это вознаграждение выглядит
запредельным.  Собрали последние скандалы с состоянием поездов
\enquote{Укрзализныци}
%%%cit_comment
%%%cit_title
\citTitle{Укрзализныця попала в скандал из-за душного вагона, в котором ехала актриса Ольга Сумская}, 
Анна Копытько, strana.ua, 29.06.2021
%%%endcit

