% vim: keymap=russian-jcukenwin
%%beginhead 
 
%%file 06_05_2023.stz.news.ua.suspilne.1.ludi_knizhki_bagattja.0.intro
%%parent 06_05_2023.stz.news.ua.suspilne.1.ludi_knizhki_bagattja
 
%%url 
 
%%author_id 
%%date 
 
%%tags 
%%title 
 
%%endhead 

\ifcmt
  %ig https://cdn4.suspilne.media/images/resize/1040x1.78/5879b9fc1292b84a.jpg
	ig https://i2.paste.pics/57ae501361a32b94f26a8e31f6f49904.png
  @caption Вікторія Лісогор — директорка Маріупольської центральної бібліотеки ім. Короленка. . Фото: Суспільне Донбас
  @wrap center
  @width 0.8
\fi

\begin{qqquote}
Маріупольська центральна публічна бібліотека ім. Короленка — це саме серце
міста, його центральна частина. Поблизу драмтеатр, музей Куінджі. Зведений на
початку минулого століття будинок, що на розі вулиці Грецької та проспекту
Миру, вистояв Другу світову, але не повномасштабне вторгнення РФ: на нещодавно
оновлених зображеннях Google Maps — добре видно, що будівля потрощена. У перші
тижні війни підвали бібліотеки стали прихистком для десятків людей. Директорка
Вікторія Лісогор провела в стінах рідної книгозбірні двадцять днів війни.
Глибокі підземні сховища, міцні стіни, броньовані двері — все це створювало
певний час відчуття відносної безпеки, згадує маріуполька. \enquote{Я не думала, що
доведеться покинути бібліотеку}, — каже вона. Покинула місто поспіхом 16
березня. Шлях до Запоріжжя був складний і небезпечний. Нині відновлює
бібліотеку у Дніпрі і збирає книжки, з якими планує повернутись у рідне місто.

Яким був початок повномасштабної війни у Маріуполі, як вдалося виїхати з міста
та пройти блокпости \enquote{ДНР} та як працює бібліотека в евакуації — директорка
Маріупольскої центральної публічної бібліотеки ім. Короленка Вікторія Лісогор
розповіла в інтерв'ю Суспільному. Далі — пряма мова. 
	
\end{qqquote}
