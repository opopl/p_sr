% vim: keymap=russian-jcukenwin
%%beginhead 
 
%%file 18_12_2021.stz.news.lnr.lug_info.2.skazka_premjera
%%parent 18_12_2021
 
%%url https://lug-info.com/news/luganskij-teatr-na-oboronnoj-predstavil-prem-eru-novogodnej-skazki-planeta-pousih-elok
 
%%author_id 
%%date 
 
%%tags 
%%title Луганский театр на Оборонной представил премьеру новогодней сказки "Планета поющих елок"
 
%%endhead 
\subsection{Луганский театр на Оборонной представил премьеру новогодней сказки \enquote{Планета поющих елок}}
\label{sec:18_12_2021.stz.news.lnr.lug_info.2.skazka_premjera}

\Purl{https://lug-info.com/news/luganskij-teatr-na-oboronnoj-predstavil-prem-eru-novogodnej-skazki-planeta-pousih-elok}

Луганский академический украинский музыкально-драматический театр (ЛАУМДТ) на
Оборонной представил премьеру новогодней музыкальной сказки \enquote{Планета поющих
елок}. Об этом с места события передает корреспондент ЛИЦ.

По сюжету, в сказочном лесу накануне новогоднего праздника пропадает главная
героиня постановки Елушка. Отважный Бармалей, дежурная по скорой новогодней
помощи Снегурильда Сугробовна и маленький зайчик Люсик отправляются на ее
поиски и попадают на планету Поющих Елок, куда чары злой волшебницы
Злючки-Колючки перенесли главный символ Нового года. В поисках Елушки герои
проходят непростые испытания, которые помогают снять чары злой волшебницы. В
завершение представления добро, как обычно, победило зло, в очередной раз
подтвердив ценность дружбы, верности и честности в человеческих отношениях.

\enquote{Я сыграла Снегурильду Сугробовну, это такой собирательный образ с чертами
учительницы, комсомолки и руководителя важного учреждения. Она очень
ответственная, отвечает за все, что происходит в волшебном лесу и на его
окраинах, и способна решить любую проблему}, – рассказала актриса театра
Вероника Балджи.

Она отметила, что в новогодней постановке используются как традиционные, так и
новаторские театральные приемы.

\enquote{Все, что мы можем силами нашего театра показать для детей, – мы все
используем. В постановке невероятное освещение – все фонарики, все \enquote{блестяшки},
которые так любят дети, все присутствует. Мне кажется, за многие годы у нас
освещение впервые настолько яркое. И, конечно же, максимально использованы
какие-то технические возможности сцены, декорации. Дети на протяжении спектакля
не раз удивляются: как же так получилось?} – пояснила она.

Актриса добавила, что \enquote{главное в спектакле – очень яркие персонажи, очень
колоритные образы} и выразила надежду, что \enquote{маленькие зрители их полюбят}.

\enquote{Новогодняя сказка и новогодняя кампания – это всегда раздолье для артиста,
потому что все недоигранное, недоделанное ранее мы можем воплотить здесь. В
жизни мы серьезные строгие взрослые и не можем себе позволить чего-то лишнего,
а в сказке можем просто \enquote{балдеть} и получать удовольствие}, – сказала Балджи.

В театре отметили, что отличием новогоднего спектакля является сочетание
традиционной сказочной символики и современной атрибутики.

\enquote{Дети совершенно нормально восприняли то, что в руках Снегурильды Сугробовны
оказался мобильный телефон, не удивило их и перемещение на другую планету через
портал в дупле. Родители, привыкшие к канонам классической сказки, конечно,
отреагировали иначе. Но такое сочетание традиционного и современного стало
изюминкой постановки}, – рассказали в учреждении культуры.

Режиссером-постановщиком спектакля выступил заслуженный артист Украины,
народный артист ЛНР Анатолий Яворский, хореография заслуженного деятеля
искусств ЛНР Ольги Тарасенко, хормейстер – Ирина Юсупова, художник по костюмам
– Юлия Бурдейная.

Сказочный сюжет сопровождался высокопрофессиональным вокалом и зажигательными
танцами в исполнении артистов театра на Оборонной, которые подарили маленьким и
взрослым зрителям праздничное новогоднее настроение и яркие эмоции.

\enquote{Замечательная сказка! Мы долго ждали предновогоднего похода в театр и, как
оказалось, не зря. Ребенок в восторге, да и папы и мамы в зале точно не
скучали, на этот час мы вернулись в детство. Спасибо большое любимым в нашей
семье актерам театра на Оборонной, они прекрасны в любом жанре, и сказка не
стала исключением. Перед праздником артисты зарядили нас позитивом и энергией,
которой зимой очень не хватает}, – рассказала мама трехлетнего Максима
луганчанка Оксана.

В дальнейшем спектакль можно будет увидеть 19, 21-24, 28-30 декабря в 10:00 и
12:00, а также 25 и 26 декабря в 10:00, 12:00 и 14:00. Музыкальную сказку
покажут и в новом году: 2 и 4 января в 10:00, 12:00, 14:00, а также 3, 5-9
января в 10:00 и 12:00.

ЛАУМДТ находится по адресу: Луганск, улица Оборонная, 11. Дополнительную
информацию можно получить по телефонам (0642) 92 11 77, (0642) 92 11 81, (0642)
92 11 83, на сайте театра или его странице в социальной сети \enquote{ВКонтакте}.
