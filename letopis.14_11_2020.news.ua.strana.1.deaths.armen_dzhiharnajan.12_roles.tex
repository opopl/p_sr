% vim: keymap=russian-jcukenwin
%%beginhead 
 
%%file 14_11_2020.news.ua.strana.1.deaths.armen_dzhiharnajan.12_roles
%%parent 14_11_2020.news.ua.strana.1.deaths.armen_dzhiharnajan
 
%%url 
%%author 
%%tags 
%%title 
 
%%endhead 

\subsubsection{12 самых знаменитых ролей}
\label{sec:14_11_2020.news.ua.strana.1.deaths.armen_dzhiharnajan.12_roles}

Армен Джигарханян внесен в Книгу рекордов Гинесса, как самый снимающийся
актер. Он сыграл 283 роли в кино. Колоритная внешность, запоминающийся
голос и актерский талант привлекали режиссеров.

\begin{itemize}
\item \textbf{\enquote{Здравствуй, это я!} (1965)}. Джигарханян появился на экранах в роли
молодого физика Артема Манвеляна. Роль сделала его знаменитым.

\ifcmt
img_begin 
	url https://strana.ua/img/forall/u/0/34/54772.jpg
	caption \enquote{Здравствуй, это я!} (1965)
	width 0.7
img_end
\fi

\item \textbf{\enquote{Операция \enquote{Трест}} (1967)}. В этом фильме Джигарханян сыграл чекиста
Артузова.

\item \textbf{\enquote{Новые приключения неуловимых} (1968)}. В знаменитой трилогии Эдмонда
Кеосаяна Армен Джигарханян сыграл штабс-капитана Овечкина. Который
запомнился зрителям характерной мимикой лица.\Furl{https://youtu.be/5AWJiVxSwN8}

\item \textbf{\enquote{Здравствуйте, я ваша тётя!} (1975)}. В этой комедии актер запомнился
   зрителям, как судья Кригс, опекун двух девушек \dshM Бетти и Энни.

\item \textbf{\enquote{Собака на сене} (1977)}. В этой музыкальной комедии, актер сыграл Тристана \dshM друга и слугу молодого дворянина Теодоро в исполнении Михаила Боярского.\Furl{https://youtu.be/IvUBSF98l0M}

   Песня \enquote{Советы Тристана} о том, как правильно расставаться с женщинами,
   стала классикой и вполне может послужить инструкцией для начинающих
   пикаперов. 

\item \textbf{\enquote{Мест встречи изменить нельзя} (1979)}. В этом фильме актер сыграл главаря
   банды \enquote{Черная кошка} Горбатого (в миру \dshM Карп). Одна из самых известных
   ролей Джигарханяна. 

   Крылатые фразы \dshM \enquote{Есть у нас сомнение, что ты, мил человек, стукачок},
   \enquote{Дурилка картонная},  \enquote{За твое здоровье пить глупо \dshM есть мнение, что оно
тебе больше не понадобится}.\Furl{https://youtu.be/NShdQsnYZbc}

	 \item \textbf{\enquote{Тегеран \dshM 43} (1980)}. Джигарханян воплотил образ наемного убийцы Макса
   Ришара, который написал мемуары.

	 \item \textbf{\enquote{Паспорт} (1992)}. Актер сыграл в нем роль эмигранта Семена
					 Кляйна , бывшего спецназовца, который уехал их СССР в Израиль, но
								скучал по Родине и даже устроился работать в рефриджератор,
								чтобы помнить о \enquote{русских морозах}.\Furl{https://youtu.be/3heKNx356QA} 

	 \item \textbf{\enquote{Ширли-мырли} (1995)}. В этой комедии Владимира Меньшова Джигарханян сыграл
   крестного отца и мафиози Козюльского.

	 \item \textbf{\enquote{Королева Марго} (1996)}. В сериале актеру достался образ палача.

	 \item \textbf{\enquote{Дон Кихот возвращается} (1997)}. Сыграл Джигарханян и Санчо Пансу.

   В постсоветское время одной из самых ярких ролей актера стал образ
криминального авторитета Гургена из сериала \textbf{\enquote{Бандитский петербург}}. 
\Furl{https://youtu.be/taIx7Y4i6mg}

\end{itemize}

