% vim: keymap=russian-jcukenwin
%%beginhead 
 
%%file 27_10_2020.fb.bystrjakov_vladimir.1.pamjat_karachencov.cmt
%%parent 27_10_2020.fb.bystrjakov_vladimir.1.pamjat_karachencov
 
%%url 
 
%%author_id 
%%date 
 
%%tags 
%%title 
 
%%endhead 
\subsubsection{Коментарі}

\begin{itemize} % {
\iusr{Ольга Александрова}

А я опять вспомнила, как на кубке Дерюгиной столкнулась с Николаем Петровичем за
кулисами, прямо вот в объятия попала нечаянно, так как прямо врезалась из-за
поворота... первое, что почувствовала, это обалденный запах туалетной
воды. Извинилась, там темно было и когда услышала его неповторимый голос, то
просто забыла, куда так спешила... а его исполнение "леди Гамильтон" там же на
кубке... все как вчера... светлая память Артисту... @igg{fbicon.heart.broken}{repeat=3} 

\begin{itemize} % {
\iusr{Владимир Быстряков}
\textbf{Ольга Александрова} Он всегда употреблял "Фаренгейт" и никогда ему не изменял....

\iusr{Ольга Александрова}
\textbf{Владимир Быстряков}  @igg{fbicon.hands.raising}  @igg{fbicon.heart.sparkling} 
\end{itemize} % }

\iusr{Vasyl Hryhorov}

Очень хорошо написано Владимир Юрьевич! Такое совпадение вы закончили этот пост
почти так же, как и я написал свой в годовщину смерти моей любимой мамы. Я
написал тогда так: "Я все более начинаю увлекаться теорией реинкарнации,
которая говорит о том, что бессмертное сознание после утраты внешней телесной
оболочки переходит в иное состояние, другое тело.

Кто знает, возможно это и так. Возможно, поздним вечером 13 ноября 2018 года
где-то у счастливых родителей родилась девочка с голубыми глазами, которая
никогда не узнает меня, а я не узнаю ее, закружившись в вечном круговороте
мирозданья, в череде пространств, времен и обличий, в бесконечном круге встреч
и расставаний". Вот вспомнилось и стало вновь как-то невыносимо грустно...

\begin{itemize} % {
\iusr{Владимир Быстряков}
\textbf{Vasyl Hryhorov} это и светло и грустно.... Одновременно...

\iusr{Сергей Шевчук}
\textbf{Владимир Быстряков} свет от звёзды продолжает идти даже когда она гаснет.
\end{itemize} % }

\iusr{Юрий Березуев}
Вечная память. Великий мастер.

\iusr{Nina Zakharova}
Голос, образ-навсегда!

\iusr{Юрий Алексеевич}
Спасибо Вам, Владимир Юрьевич за это. Пишите, пишите больше.

\iusr{Ирина Саакашвили}
Дякую ! И Браво  @igg{fbicon.hibiscus}  @igg{fbicon.hands.pray} 

\iusr{Юлия Самойлова}

Николай Петрович был и остается моим самым любимым артистом. Владимир Юрьевич,
ваша Леди Гамильтон в его исполнении неподражаема и лучше Караченцова ее никто
не пел.


\iusr{Олег Переладов}
Светлая память мы будем помнить Николая Караченцова. Человек отдавшый себя на
сцене сполна!!!

\iusr{Роман Прохоровский}

Памяти Николая Караченцова, актёра театра и кино, певца, народного артиста РСФСР.

Кто напишет музыку ? Кто споёт? Видио?

Памяти

\obeycr
1. Я помню концерт юбилейный поэта,
(какого не помню, но знаю, это было)
Певцы почему-то забыли про это,
Один Николай на концерт прилетел,
Он тексты не знал, по рукописи пел.
Жил артист,
Он играл свои роли,
Как артист,
Ездил он на гастроли,
Но здоровье,
Потеряв, не вернёшь,
Знают все,
От судьбы не уйдёшь.
2. Он говорил: «Надо жить с ускореньем»,
Бывают в жизни чудные мгновенья,
Что образ какой-то создать не успеет,
И делать всё надо в три раза быстрее».
3. Удачные роли бывают не часто,
А чем в этой жизни мы можем похвастать,
Понравиться Пьеру Кардену не просто,
Нью-Йорк, был в восторге от сказанных тостов.
4. Хитрость слов: «Ваше время придёт,
И талантливых слава найдёт»,
Много в мире чудес происходит,
Только жизнь почему-то уходит.
5. Уйти надо во время, в славе и силе,
Пока о тебе, про тебя не забыли,
И кто долгожитель с ним часто бывает,
Кого там хоронят? А кто его знает........
3.11.18 г.
Кто дружит по жизни спокойно и прочно,
Не знает, что можно дружить и заочно,
Кто в славе купался, не чувствовал это,
Но песни поются словами поэта.
Роман Прохоровский
\restorecr

\iusr{Леонид Брежнев}

\href{https://www.youtube.com/watch?v=5JGye1VLH4Q}{%
~ *АЛИСА в Зазеpкалье ~ Песенка Белого Рыцаря ~, youtube, 29.06.2015%
}

\ifcmt
  ig https://i2.paste.pics/a3f2a164b5447a48b6f5d7eabb511127.png
  @width 0.4
\fi

\iusr{николай ковалёв}
Вот что делает скорость на автомобиле ДТП?

\iusr{Елена Васильевна}
А мне всегда он был красавчиком!!! С завораживающим хрипловатым голосом!!!! Я тебя никогда не забуду, я тебя никогда не увижу........

\iusr{Тома Русакова}
Великий человек и актер!
Вспоминаем ...и @igg{fbicon.heart.red}

\iusr{Леонид Сорокин}

Он остался с нами в кино, в записях, в воспоминаниях... А вот нового ничего не
появится. Как жаль! Сколько бы он еще создал при его бешеной работоспособности!

\iusr{Татьяна Гоцуленко}

Вовик, спасибо тебе за эту публикацию. Как здорово ты написал о нашем ДРУГЕ!
Слов у меня нет - зато много слёз...

\iusr{Регина Ханивецкая}
\textbf{Татьяна Гоцуленко}, а сколько замечательных воспоминаний!

\iusr{Андрей Сергеевич}
Вечная память!! Надо обязательно в сотый раз пересмотреть Досье Детектива Дубровского

\iusr{Vyacheslav Kobylyatskiy}
Приятно смотреть на таких замечательных, талантливых людей!

\iusr{Ирина Караченцова}
Володя, Владимир Быстряков, как прекрасно ты написал!!!@igg{fbicon.heart.red}

\iusr{Валерий Мякотенко}

Душа Николая Караченцева знает, что мы, грешные, помним и чтим прекрасного
артиста! Уверен: Колиной душе комфортно в Царствии Небесном от наших светлых о
нем воспоминаний! Аминь...

\emph{Андрей Лесто}

Помню, как был потрясён, когда в "Утренней почте", кажется, впервые услышал, как он пел "Барыня речка"!

И вчера у Елизаветы Комаровой из Питера прочитал вот такие воспоминания о необычайной энергетике Карачиенцева:

"ПЛАМЯ. Николай Караченцов. 27 лет назад.

\obeycr
Он был не такой уж молодой, к пятидесяти. И какой-то спектакль с ним приехал на пару дней гастролировать в наш театр на Литейном.
Некоторые работники цехов намылились в зрительный зал. А меня, как самую мышь, Галина Федоровна оставила дежурить у костюмерного цеха с иголкой и утюгом наготове: вдруг звездам что-то понадобится. Сиди, сказала она, и ни с места, и что попросят, все делай.
А их всего-то было человека три: Николай Петрович и кто-то ему подыграть.
И вот я сижу на гладильном столе и болтаю ногами. И поднимается по лестнице Караченцов.
Я, будучи дурой от рождения, абсолютно не смутилась. Но сразу культурно сказала:
- Здравствуйте!
- Привет, - отвечает он и руку протягивает. - Николай.
- Очень приятно, Лиза.
- Как дела?
- Все хорошо, - говорю. - Все вам рады. А не нужно ли вам что-нибудь пришить или погладить? Я дежурный костюмер. Вдруг что-то надо.
- Вряд ли, - говорит он. - Лучше спектакль посмотри. Или тебе неинтересно?
- Интересно. Но я дежурю. По вам, - сказала я.
- Я тебя отпускаю. С условием, что ты посмотришь.
- Спасибо, тогда посмотрю. Но потом вернусь.
- Договорились, - и он пошел к себе в гримерку переодеваться. У него, собственно, свой костюмер был.
А я осталась сидеть на столе.
Никакой ослепительной красоты в нем я не заметила: невысокий, крепкий, немолодой человек.
Однако от него пылало чем-то, как от печки. Я была маленькая девушка, но сразу это словила.
Как этот огонь называется красивыми, а главное - приличными словами, не знаю.
Изобразить его нельзя.
Представьте, что к вам подходит человек. Ненадолго. Пожимает руку. Вскоре уходит. А у вас потом все тело становится горячим, и вам опять хочется постоять с ним. Как будто вы телефон с разряженной батарейкой, а этот человек мгновенно заряжает вас.
Спектакль я посмотрела и совершенно не запомнила.
Смотрела я только на Караченцова. Огонь продолжался и со сцены. Мне даже было непривычно сидеть в зале в таком странном состоянии.
Зрители глядели на него, не отрываясь.
Конечно, все смотрели "Старшего сына", "Электроника" и все остальное.
Но кинопленка это пламя не транслирует.
Видимо, зрители были поражены. Они не понимали, что вообще происходит.
После спектакля я закрыла цех и сразу ушла. Мне было грустно и холодно. Таких людей я раньше не встречала. И что именно чувствую - тогда не осознавала.
Я и потом, кстати, таких людей не встречала. Ни разу."
\restorecr


\iusr{Виталий Осовский}
«Билли, заряжай!»...

\iusr{Алиса Осенина}
Он навсегда останется легендой. Единственным и неповторимым....

\iusr{Регина Ханивецкая}
Мы с Жориком вспоминали, вчера.

\iusr{Елена Коваленко}
Очень жаль. Светлый был человек.

\iusr{Татьяна Павлович}
Мы, не знакомые с ним лично, таким его и представляли, нет чувствовали. Вечная память. @igg{fbicon.heart.broken}{repeat=3} 

\iusr{Владимир Фролов}
Спасибо, Мастер....
За то, что Вы есть....
За то, что Он был.. и есть... и будет....
Берегите себя....

\iusr{Надежда Литвинчук}
Царствие небесное. Внликий актёр!

\iusr{Оксана Высоцкая}
Друзья всегда остаются с нами. А Ваш друг ещё великий и любимый Артист!

\iusr{Ирина Барилюк}

Гениально, Владимир! @igg{fbicon.heart.red} Наревелась уже с утра  @igg{fbicon.cry}  ... И хочется верить, что
Природе не свойственно разбрасываться человеческим материалом, уничтожая
безвозвратно всё наработанное человеком по жизни... Мы знаем, что ты СЛЫШИШЬ
нас !.

\iusr{Прокопенко Алексей Георгиевич}
Владимир Юрьевич, большое Вам спасибо за то, что поделились этими эмоциями с нами!

\iusr{Элла Воробьева}
Спасибо!

\iusr{Анна Власова}
Легенды живут вечно чтобы навсегда оставатся в памяти и в сердцах

\iusr{Натали Потапова}

Много сыграно ролей в театре и в кино, много песен спето, которые стали
популярны у народа именно благодаря голосу Николая Петровича, голосу, который
ни с каким другим не спутаешь.

И легендарная "Юнона и Авось", от которой каждый раз мурашки, сколько бы не
смотрела.

Но особой песней в исполнении Караченцова для меня навсегда есть и будет песня "Кленовый лист".
Спросите почему?
Не отвечу, не знаю потому что.
Видимо, как -то я совпала с этой песней ещё в юности и живу с ней всю жизнь.
С Днём рождения, Николай Петрович!

\iusr{Игорь Зель}
Песни в его исполнении часто слушаю. А ещё его голосом говорил Жан Поль Бельмондо. ( самая лучшая озвучка на мой взгляд)

\iusr{Alex Chernega}
Великий... R.I.P

\iusr{Дмитрий Юхновец}
С Днём Рождения, Николай! Вы - великолепный Актёр! Не \enquote{были}, а \enquote{есть}!

\iusr{Ирина Виноградова}
Какая эпоха ушла! Спасибо, Маэстро!

\iusr{Геннадий Хохлов}

Прекрасные фильмы \enquote{Трест, который лопнул}, \enquote{Белые росы}, \enquote{Человек с бульвара
капуцинов}, \enquote{Старший сын}, фантастическая постановка \enquote{Юнона и авось}... Я даже
и не знаю, появится ли у нас что-то подобное... Очень жаль, что с ним всё так
вот получилось...


\iusr{Титаренко Любовь}
Вы мои @igg{fbicon.heart.red} родные


\end{itemize} % }
