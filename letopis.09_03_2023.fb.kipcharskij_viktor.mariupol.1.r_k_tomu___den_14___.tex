%%beginhead 
 
%%file 09_03_2023.fb.kipcharskij_viktor.mariupol.1.r_k_tomu___den_14___
%%parent 09_03_2023
 
%%url https://www.facebook.com/permalink.php?story_fbid=pfbid02EyuSs9wgPH6j63x9TKucovCGuk83pnscAvpabznUbsgynRsvhhneFmYu1YnN1uLzl&id=100006830107904
 
%%author_id kipcharskij_viktor.mariupol
%%date 09_03_2023
 
%%tags mariupol.war,dnevnik,mariupol,09.03.2022
%%title Рік тому:  День 14 - 09.03.22. Середа
 
%%endhead 

\subsection{Рік тому:  День 14 - 09.03.22. Середа}
\label{sec:09_03_2023.fb.kipcharskij_viktor.mariupol.1.r_k_tomu___den_14___}

\Purl{https://www.facebook.com/permalink.php?story_fbid=pfbid02EyuSs9wgPH6j63x9TKucovCGuk83pnscAvpabznUbsgynRsvhhneFmYu1YnN1uLzl&id=100006830107904}
\ifcmt
 author_begin
   author_id kipcharskij_viktor.mariupol
 author_end
\fi

\ii{09_03_2023.fb.kipcharskij_viktor.mariupol.1.r_k_tomu___den_14___.pic.1}

Рік тому: 

День 14 - 09.03.22. Середа. 

Закінчується другий тиждень "гарячої" війни.

У кімнаті температура 16,6. 

0:15 - 1:25. Тихо

Після 2:25 заснув.

4:39. Канонада. Оля бачила у вікні у небі червоний слід. Може ракета ППО? 

Гупнуло дуже гучно. Зібралися у підвал. Онук спросоння метушився. Онука
рюмсала. Настала тиша. Червоні спалахи правіше Нептуна. Потім дуже багато
спалахів і дуже гучно. Під кінець - червоне світло - щось горить? Знову
зібралися бігти у підвал, але раптом все стихло.

Спалахи десь за молочною кухнею.

5:30. Тихо.

Близько 7:00 поснідав, зробив "намаз" і пішов до вогнища кип'ятити воду.

Дуже добре чути черги пострілів з автоматів десь біля Перших воріт,
Іллічівського ринку, площадки Б. Потім вдарили Гради, здається, з парку
Петровського. Дуже гучно і страшно. Люди від вогнища побігли під будинок. Сенс?
Щоб посікло уламками скла або завалило стіною чи балконом?

Усі каструлі стали одного кольору: чорні, закопчені:

- Чия чорна каструля кипить?

Тиша.

- Чия каструля з зеленою кришкою?

- Моя!

Накіп'ятив води, відніс додому та Едіку.

Знову дуже гучно з Петровського (?). Люди знову побігли під будинок.

Едік віддав Олі курку: "Вона зіпсувалася - віддайте котам". Ми обнюхали курку і
зробили висновок, що вона не зіпсувалася, тож зварили її і занесли Едіку. 


10:20. По радіо сказали, що відкрито коридор.

Хлопці їздили у центр шукати зв'язок, сказали, що зруйновано стратегічний
об'єкт: будку Шаурма на Покришкіна біля кафе Полісся (колишня їдальня).

11:20. До райадміністрації привезли гуманітарку: люди кинулися хапати.
Військові стріляли у повітря. Сказали що що більше не приїдуть.

Також військові сказали, що коридора не буде - росіяни стріляють в районі
блок-посту на Старий Крим. 

\ii{09_03_2023.fb.kipcharskij_viktor.mariupol.1.r_k_tomu___den_14___.pic.2}

12:30. Могли ж налагодити оповіщення: наприклад,  пустити машини з мегафонами,
інформувати людей? Влітку люди ходили з мегафонами: "Картопля, дем'янівські
кавуни!" - а місто не може? Наш приймач - єдине джерело інформації на три
будинки (крім раша ФМ та чуток, які збирають у місті "мандрівники"), тож вранці
біля вогнища я переказую новини. У приймача треба полагодити або замінити
регулятор гучності, але як це зробити без паяльника?

\ii{09_03_2023.fb.kipcharskij_viktor.mariupol.1.r_k_tomu___den_14___.pic.3}

Рома заніс нам пеленгасиків та карасів, бо без світла їх морозилка зовсім
розтанула. Оля взяла, а чистити довелося мені. Дуже довго в неприємно: це не
хижаки, які їдять рибу - ці їдять якусь муляку! Решту риби Рома виніс і віддав
людям біля вогнища. Смажили рибу під снігом, що падав: поки її поперевертаєш на
інший бік - сніг сипле у пательню і риба вже не смажиться, а вариться.

Варили соус з картоплі та борщ - син чистив овочі та тер бурак. Пообідали. По
обіді я непомітно задрімав.

17:00. Нагріли соус на вечерю, накип'ятили воду на чай та на миття посуду.
Занесли воду Едіку. Вмовляли його, аби дав курячий бульйон зварити суп - не
вмовили.

17:45. Над нами пролетів літак, скинув бомби десь на Азовмаш чи парк
Петровського. В нього пустили дві ракети, але він відстрілював пастки і
безкарно полетів перезаряджатися.

Домовилися назавтра о 6:00 йти шукати дрова в "заброшках" - кинутих будинках,
що руйнуються. 

Нервовим далі краще не читати:

Кажуть, що на Повороті було кафе, в якому безкоштовно годували людей. В нього
влетів снаряд. Місиво...

Новини з приймача: в Маріуполі офіційно встановлено 1360 вбитих, але на
Східному, на західних мікрорайонах і геть по усьому місті на вулицях лежать
рештки тіл і часом неможливо зрозуміти скількох саме там загинуло людей...
Також багато людей сгоріло у квартирах, або засипано у зруйнованих будинках.

Скинули важку бомбу на пологовий будинок у центрі міста: поранено 17
співробітників. Звичайно, "сумна коняка" лавров та скабєєва кажуть, що там були
азовці.  Поруч з тим пологовим - будинки Жердєвих та Мазурів...

На старому цвинтарі у центрі комунальники копають загальні ("братські") могили.

У підвалі під завалами будинку рятівники знайшли тіла дівчинки (6 рочків) та
жінки, яка накрила дитину собою... Жінку вбило, а дівчинка померла від
спраги... Яка мученичеська смерть... За що це дитині?

Коридор було оголошено з 9:00 до 21:00. Він протримався хвилин 10, а потім з
Мангуша на Маріуполь пішли танки.  

20:00.Гумконвой з Запоріжжя з 90 тонами їжі та медикаментів  доїхав до Оріхова
і його обстріляли з Градів. Кілька машин побито, але водії встигли заховатися
по кюветах.

Поїхали на Гуляй Поле - попереду бій. Почали шукати об'їзд - підірваний міст.
Де конвой зараз - невідомо.

В понеділок (12-го березня) з Маріуполя люди поїхали на власний ризик - без
оголошення коридору. Український блок-пост випустив, бо не було обстрілу траси
і попередили про заміноване узбіччя. Російський (можливо, перед Нікольським)
після ретельного огляду людей та автівок теж пропустив, але перед поворотом на
Федорівку "люди в іншій формі" пропустили тільки машини з жінками, дітьми, та
чоловіками непризивного віку. Інші повернулися у Нікольське, де й заночували по
хатах. Місцеві підказали, як об'їхати блок-пост - близько 400 людей дісталися
Запоріжжя. Уявляєте: місцеві надали нічліг чотирьом сотням людей!  

Це нас дуже обнадіяло. Принаймні можна спробувати вивезти онуків...

Навздогін: Пару днів тому пекли лепішки. Оля з невісткою почали шукати рецепт.
Як це зробити без інтернету? Допомогли технології минулого: книжки з кулінарії
країн світу. Які народи найбільш економні? В нас були китайська та єврейська
кухні. Китайський рецепт передбачав щось екзотичне на кшталт "луска дракона" та
інше, тому за основу взяли єврейський...

На фото:

1 - "кухня" нашого будинку.

2,3 - генератор

%\ii{09_03_2023.fb.kipcharskij_viktor.mariupol.1.r_k_tomu___den_14___.pic.saved}
%\ii{09_03_2023.fb.kipcharskij_viktor.mariupol.1.r_k_tomu___den_14___.cmt}
