% vim: keymap=russian-jcukenwin
%%beginhead 
 
%%file 13_12_2020.news.ru.vesti.zenin_sergei.1.zhiteli_donbassa_etih_ljudei_ne_ostanovit
%%parent 13_12_2020
 
%%url https://www.vesti.ru/article/2498260
 
%%author Зенин, Сергей
%%author_id zenin_sergei
%%author_url 
 
%%tags donbass,vojna
%%title Жители Донбасса: этих людей не остановить
 
%%endhead 
 
\subsection{Жители Донбасса: этих людей не остановить}
\label{sec:13_12_2020.news.ru.vesti.zenin_sergei.1.zhiteli_donbassa_etih_ljudei_ne_ostanovit}
\Purl{https://www.vesti.ru/article/2498260}
\ifcmt
	author_begin
   author_id zenin_sergei
	author_end
\fi

\ifcmt
pic https://cdn-st1.rtr-vesti.ru/vh/pictures/xw/307/833/1.jpg
\fi

Отец – итальянец. Мама – русская. В 2014-м вторая кровь взыграла. Поехал в
Донбасс.\Furl{https://www.vesti.ru/article/2496052} Третий год как гражданин республики. Никола Ранджелони – один из
первых иностранцев, получивший паспорт ДНР. Подал документы на российский.
Женился на дончанке. Сыну уже почти два года. Зовут Ярославом.

На языке матери сам Витторио говорит почти без акцента. Если не считать
южнорусское \enquote{гэ} и украинское \enquote{шо}.

Стрелял только в Италии – с отцом на охоте. А в Донбассе вместе с другими
волонтерами доставлял жителям гуманитарную помощь. Самое яркое впечатление –
как привезли продукты в детский сад. \enquote{Дети впервые, наверное, за несколько
месяцев увидели конфетку, многие просто стали обнимать, благодарить}, –
вспоминает Никола.

Ведет свой блог. Плюс пишет статьи и снимает фото и видео для мировых агентств.
Скоро закончит книгу о жизни республики, о том, какие здесь люди. Говорит, не
жалеет, что приехал и с ними остался.

\enquote{В Европе все навязано, люди стандартно должны думать, иначе идет цензура. А
здесь я ощутил дух людей сильных, которые, несмотря ни на что, Донбасс тогда
сражался против всего мира, что с оружием, что в плане информационном, но люди
не побоялись попытаться хотя бы выстоять, и у них это получилось. Российский
паспорт зачем? Это честь получить паспорт здесь, в Донецке, где люди получили
это право благодаря стойкости, храбрости и немалой пролитой крови. Сам паспорт
– это символ лично для меня}, – признается Никола.

Мы ходим по одному из самых разрушенных районов Донецка. В 200 метрах от
печально известного аэропорта. В 2014 году здесь был ад. Канонада не стихала ни
на минуту. Но звуки войны – это не только разрывы снарядов. Гуляющий среди
покалеченных зданий ветер звучит не менее страшно. Чем дольше слышен этот стон,
тем дольше война не отпускает. Поэтому каждый день затишья люди тратят на то,
чтобы быстрее все восстанавливать.

Франсуа-Леон Мод Д`Эме – студент 3-го курса Донецкой музыкальной академии. В
прошлом – младший лейтенант французской армии. Уехал из Франции, чтобы не
оказаться среди натовских военных в Афганистане. Предпочел, как сам говорит,
войну по-настоящему справедливую. В Донбассе записался в добровольцы.

\enquote{Я думаю, что здесь – центр душевных сил, центр хороших людей, которые
по-настоящему готовы ни для чего жертвовать. Просто потому, что здесь наше дело
правое, здесь Бог живет, здесь он с нами}, – говорит Франсуа-Леон.

Санитарные нормы искусственно сократили очереди в миграционную службу. Все –
строго по записи. Но пусто здесь не бывает. И уже выданы почти 200 тысяч
паспортов.

Два паспорта – Французской и Донецкой республик. Еще несколько бумаг. Месяц
ожидания – и Франсуа станет гражданином России.

Эти будущие граждане России – по-настоящему смелые люди. Большинство проверило
себя на передовой. Кто с пулеметом, а кто с блокнотом и камерой. И не за
гонорар, не в командировке от западных компаний, а сами по себе. Неравнодушные.
Действительно патриоты.

Война – это всегда страшно, к ней невозможно привыкнуть и говорят, что это –
мужское занятие. Поэтому, когда в окопе встречаешь коллегу-женщину, конечно,
хочется спросить: зачем тебе это? Чтобы говорить правду. Чтобы люди в Европе
понимали реально, что происходит здесь, потому что в Европе у них есть только
пропаганда Украины.

По ночам – обстрелы. А днем на передовой в последнее время тихо. Только
таблички предупреждают о возможной опасности.

За годы войны у Кристель Нэан – несколько ранений. Осколки задели не только ее
тело, но и душу.

- Плакать приходилось?

- Да, в 2017 году, когда была эскалация. Когда мирная женщина умерла. Она
лежала на снегу. И ее муж там сидел на поле, он хотел, чтобы она проснулась
спокойно, как если она спала. Мне было реально очень трудно. Я плакала, –
говорит Кристель.

- А личные потери здесь были?

- Да, потеряла друзей и даже женщину, которая как вторая мама. Ирина. Бабушка в
Зайцево. Я не иностранка из-за этой войны. Это уже сейчас моя война. Я была
слишком часто на кладбище за эти пять лет.

Кристель в этих местах не впервые. Она видела, как с каждым годом здесь
оставалось все меньше и меньше неразрушенных домов. Она писала об этих людях,
которые живут на переднем крае.

Раньше здесь кролики долго не жили. Первый же обстрел – и разрыв сердца. А это
уже новое поколение, адаптировались к условиям. Так же, как и люди, которые не
собираются отсюда никуда уезжать.

Когда общаешься с дончанами, понимаешь, что их невозможно победить. Буквально
все – от ребенка до старика – крепко держаться за свою землю. На ней выросли
несколько поколений сильных характером и неподкупных людей. Странно, что этого
не понимают те, которые смотрят на них через оптику прицелов. Здесь, в
Донбассе, превыше многого сохранить лицо и честь. Олег с говорящей фамилией
Гром – один из последних примеров. Ефрейтор. Служит в одном из
спецподразделений.

\enquote{За сестру есть опасения. Я не могу ее там контролировать. А здесь мне не так
страшно}, – говорит Олег.

Сестра – на Украине. Через давление на нее СБУ пыталось склонить Олега к
сотрудничеству. Боец доложил об этом командованию. Подключились спецслужбы
республики. Вот запись беседы с той стороной: "У нас есть возможность через
различные структуры, в том числе через структуры подчиненные Министерства
обороны, решить вопрос о вашей реабилитации, чтобы к вам никто не имел никаких
претензий". Судя по этим словам безымянного сотрудника СБУ, Украина все еще
рассчитывает на реванш в Донбассе. При этом невозможно понять, на чем основаны
эти расчеты. Угрозами здесь никого не испугать. Кто хоть раз спускался в шахту
или знает, что такое работа металлурга, тот понимает, что эти люди не боятся
ничего.

Температура внутри чаши с кипящим металлом – 3000 градусов. Этот человек стоит
на краю для того, чтобы подцепить на крюк мостового крана гигантский электрод,
исчерпавший свой ресурс из-за огромной температуры. Электрод из особого сплава
меняют каждые полчаса. А люди на этом месте работают по шесть часов в смену.

Солнце Донецка – расплавленный металл. Льется для строительства новой жизни.

Такой, какой она здесь была и которой обязательно будет. Как невозможно
остановить мартен, так и не остановишь этих людей.
