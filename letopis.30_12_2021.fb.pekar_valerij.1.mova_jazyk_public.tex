% vim: keymap=russian-jcukenwin
%%beginhead 
 
%%file 30_12_2021.fb.pekar_valerij.1.mova_jazyk_public
%%parent 30_12_2021
 
%%url https://www.facebook.com/valerii.pekar/posts/10217401455103006
 
%%author_id pekar_valerij
%%date 
 
%%tags jazyk,mova,obschestvo,telekanal,televizor,tv,ukraina,ukrainizacia
%%title В публічному просторі російської не має бути - це моє глибоке переконання
 
%%endhead 
 
\subsection{В публічному просторі російської не має бути - це моє глибоке переконання}
\label{sec:30_12_2021.fb.pekar_valerij.1.mova_jazyk_public}
 
\Purl{https://www.facebook.com/valerii.pekar/posts/10217401455103006}
\ifcmt
 author_begin
   author_id pekar_valerij
 author_end
\fi

Новий рік - традиційний час боротьби між минулим і майбутнім. Ця боротьба
втілюється не лише в тому, що ми підбиваємо підсумки та складаємо плани, а й у
наративах, які звучать за святковим столом (у вигляді тостів), у президентській
промові, та й загалом з екранів телебачення.

\ii{30_12_2021.fb.pekar_valerij.1.mova_jazyk_public.pic.1}

Цього року з екранів телебачення знову звучатимуть старі радянські наративи,
але до них додається трансляція фільмів та серіалів російською мовою. Я вважаю,
в умовах 8-річної безперервної війни та намагання імперії знову затягти нас у
болото минулого - це ганьба.

Я не маю нічого проти російської у побуті, та й сам в особистому спілкуванні
часто її вживаю. Але то в побуті, а в публічному просторі російської не має
бути - це моє глибоке переконання.

Адже наша війна - це війна наративів та ідентичностей, і тут мова є зброєю, за
влучним висловом Серпня, героя фільму Кіборги. Як казав видатний теоретик
націоналізму Бенедикт Андерсон у книзі \enquote{Уявлені спільноти}, нація - це плем'я,
яке читає спільні книжки. Я би додав у наш час: ... і дивиться спільні
телепрограми. Спільна мова у публічному просторі потужно працює на створення та
згуртування української політичної нації, хоч би якими мовами люди не
розмовляли приватно.

Але це не подобається ні Росії, ні олігархам. Тому що в успішній Україні немає
місця ні першій, ні другим. Тому вони намагаються протягати російську у
публічний простір, щоб потім мати можливість говорити про Новоросію, про захист
російськомовних від націоналістичної хунти, про перший-другий-третій сорти
українців, яким треба голосувати на виборах не за принципом \enquote{обери собі краще
життя}, а за принципом \enquote{голосуй за своїх проти чужих}.

Я не жартую, це все дуже серйозно. Це складова війни, яка є не менш важливою за
сотні танків та десятки тисяч солдатів на наших кордонах. Відчуття єдиної долі
з Росією, яке нам нав'язують ось уже 100 років (якщо не 350), картинка
абстрактної \enquote{російськомовної країни}, в якій ми нібито живемо, - це не жарти.

Ганьба, що випущений «Кварталом-95» російськомовний серіал «Свати» практично
одночасно крутять на російському державному телеканалі «Россия-1» і
українському «1+1».

Телебачення за законом має бути українською мовою, бо телебачення - це
сьогодні основний публічний простір. Порушення закону відбувається при прямому
потуранні  Держкіно, яке видає дозволи на показ фільмів, і Нацради з питань
телебачення й радіомовлення, яка має забезпечувати дотримання закону
телеканалами.

\#заТБукраїнською

Стежте за повідомленнями з цим тегом – боротьба за телебачення українською
потребуватиме активних дій по всій країні. Прийдешній рік має остаточно
прибрати російську мову з телебачення та звільнити простір для українських
митців. Тому буде багато заходів. Зокрема, 1 лютого о першій годині дня кияни
зберуться на акцію біля Держкіно (вул. Лаврська, 10). 

Поширюйте, якщо згодні.

ОНОВЛЕННЯ: додав картинку від шановного Нікіта Тітов.
\ii{30_12_2021.fb.pekar_valerij.1.mova_jazyk_public.cmt}
