% vim: keymap=russian-jcukenwin
%%beginhead 
 
%%file 29_12_2020.news.ua.pravda.selezneva_tatiana.1.stanislav_aseev
%%parent 29_12_2020
 
%%url https://www.pravda.com.ua/articles/2020/12/29/7278243/
 
%%author 
%%author_id selezneva_tatiana
%%author_url 
 
%%tags 
%%title Станіслав Асєєв: Всі окуповані території Донбасу — один великий концтабір "Ізоляція"
 
%%endhead 
 
\subsection{Станіслав Асєєв: Всі окуповані території Донбасу — один великий концтабір \enquote{Ізоляція}}
\label{sec:29_12_2020.news.ua.pravda.selezneva_tatiana.1.stanislav_aseev}
\Purl{https://www.pravda.com.ua/articles/2020/12/29/7278243/}
\ifcmt
	author_begin
   author_id selezneva_tatiana
	author_end
\fi

\ifcmt
  pic https://img.pravda.com/images/doc/d/c/dce1a3e-420aseev.jpg
	width 0.4
\fi

Тетяна Селезньова — Вівторок, 29 грудня 2020, 05:30

"Восени 2017 року мені виповнилося 28 років, але руки тремтіли так, наче мені
70. Вони й зараз тремтять, це залишилося. Коли я відчуваю якісь емоції —
негативні чи позитивні, не важливо. За руками можна відслідковувати, що я
переживаю", — розповідає Станіслав Асєєв.

Майже півтори години нашої розмови колишній в’язень донецького концтабору
"Ізоляція" тримав руки зчепленими в замок. 

У 2014 році він залишився в окупації з особистих причин — в Донецьку залишалися
його мама і дві бабусі. Під псевдонімом Станіслав Васін писав про життя
окупованого Донецька для "Української правди", "Радіо Свобода", "Дзеркала
тижня", "Українського тижня".

При цьому він щиро вважав, що йому потрібно там бути, аби Україна знала про те,
що відбувається в окупації. В травні 2017 року в центрі Донецька його викрали
бойовики.

У сучасному місці тортур під назвою "Ізоляція", облаштованому за сталінськими
лекалами, він провів 962 дні. Про це написав книгу "Світлий шлях: історія
одного концтабору", яку закінчив вже на волі, в Києві.

Рік тому, 29 грудня 2019 року, його звільнили з полону разом із 75 громадянами
України в рамках обміну з бойовиками.

Сьогодні йому 31 рік, і зовні він не схожий на людину, яка пройшла майже два з
половиною роки фізичних і психологічних знущань. Про це нагадують хіба що руки,
зчеплені в замок. 

Очі залишилися чисті. Навіть коли він говорить про помсту своїм катам. І від
цього, можливо, розповідь здається ще страшнішою.

В інтерв’ю "Українській правді" Станіслав Асєєв розповів, які людські якості
втратив в полоні, чому не хотів би опинитися в ситуації, коли у нього буде
можливість помститися своїм катам, і як зшити розлом між тими, хто залишився на
окупованих територіях і рештою населення України. 

Найкоротша відповідь на останнє запитання – "ніяк". 

\video{https://youtu.be/K-9jAYN1e0E}

\subsubsection{\enquote{Я все підписав через годину катувань}}

— Вас викрали у травні 2017-го. Як це було? Яким був перший допит? 

— Це було 11 травня 2017 року, день так званої "республіки". Я мав зробити
зйомку та відправити в редакцію "Радіо Свобода". Повертався додому через через
площу Леніна в центрі міста і мене зупинив патруль. Звичайний патруль, не
"МГБ". Вони почали перевіряти мій рюкзак. Оскільки я розумів, що в цей день
особливі заходи безпеки, власне, не дуже звернув увагу на цю перевірку.

В рюкзаку нічого не було. Я показав їм паспорт, вони почали кудись дзвонити. За
5-7 хвилин з’явилися люди в цивільному, які відразу почали вдягати на мене
наручники. Тобто, вони вже зрозуміли, хто я такий.

Зрозуміли через рюкзак. У них не було фото мого обличчя, але було фото зі
спини. Я засвітився на похованні "польового командира" Гіві — на одному з
репортажів. Як виявилося, вже півтора роки мене розшукували — з того моменту,
коли я з’явився в публічному інформаційному просторі у Facebook.

Вони затягли мене до машини, одягли мішок на голову і повезли до будівлі так
званого "МГБ" на Шевченка, 26. Це фактично центр Донецька.

Я не назву це допитом, мене просто били. Потім розпочався допит в сусідній
кімнаті. Спочатку ввічливо попросили на камеру розповісти свою біографію, хто я
такий, чим займаюся в Донецьку. Тільки-но камеру вимкнули, внесли "тапік"
(ТА-57 — воєнно-польовий телефонний апарат — УП). Це пристрій, дроти якого
"приєднують" до різних частин тіла. Мені ці дроти приєднали до великих пальців
рук — з цього почались катування.

Побиття, психологічний тиск, погрози — їм потрібно було вибити з мене зізнання,
що я не просто журналіст, а ще й шпигун, працюю на Головне управління розвідки
України.

Я все підписав через годину катувань. Я сказав: все, що треба, підпишу.


\ifcmt
  pic https://img.pravda.com/images/doc/4/b/4b07923-img-4865.jpg
	width 0.4
\fi

— І протягом перших днів ви точно знали, що це безнадійно і допомога не прийде?

— Про допомогу думок не було, бо я зрозумів, куди потрапив.

Це був підвал. Тобто, вас не переводять до офіційної в’язниці. До мене, як
виявилося, в цьому ж підвалі сидів Ігор Козловський — мій вчитель, також
колишній політв’язень.

Я зрозумів: якщо мене помістили у підвал, вони хочуть, аби про мене ніхто не
дізнався. Так і відбувалося протягом трьох тижнів. Вони мене примушували
дзвонити мамі, писати пости у Facebook і навіть статті, які були заплановані...
Щоб ніхто не знав, що я під арештом.

Можливо, йшли якійсь торги щодо мене, вони хотіли отримати якомога більше
вигоди з цього. Мені казали: дзвони і кажи, що все добре.

— Ви провели 28 місяців в "Ізоляції" — приміщенні колишнього заводу, пізніше
унікального культурного простору Донецька, який став концтабором на вулиці
Світлого Шляху, 3. Ви, напевно, читали табірну прозу політв'язнів радянських
часів. За вашими відчуттями, де ви опинилися в травні 2017-го — в Росії,
Радянському Союзі, чи у донецьких підвалів є своя специфіка?

— Специфіка "Ізоляції" в тому, що це не офіційна в’язниця, а саме концтабір. Це
не просто медійна назва. Це слово, яке найточніше передає те, що там
відбувається.

З іншого боку, якщо порівнювати з СІЗО і підвалом на Шевченка, про який я
згадував, в "Ізоляції" кращі побутові умови. Це колишні адміністративні
приміщення заводу. Там збереглися меблі, пластикові вікна, і навіть
кондиціонери в деяких камерах. Вони просто наварили грати, поставили металеві
двері і вирішили, що це тюрма.

Але якщо порівнювати те, що там відбувається з людьми, це класичний 1937 рік.
Я, звісно, не можу судити, як було тоді, але в "Ізоляції" немає жодних меж. Бо
навіть у тридцятих роках в СРСР, якщо "політичного" засудили, його вже хоча б
не чіпали.

В "Ізоляції" ви будете терпіти щоденні, цілодобові знущання. Мене туди перевели
через два місяці після винесеного так званого вироку, і я це відчув на власному
досвіді.

— Як залишитися людиною, яка думає, аналізує, якщо щодня терпіти тортури?
Фізичний біль стирає людську подобу?

— Я б не так поставив питання… Треба потрапити туди вже думаючою людиною, і
тоді ви зможете нею залишитися.

Найкомфортніші, якщо можна так сказати, умови існування там були для тих людей,
які не думали взагалі. Вони співпрацювали з адміністрацією, бо це давало їм
певні переваги, але вони не замислювалися, що буде далі.

Таким людям і фізично, і психологічно було легше, аніж тим, хто задумувався, що
відбувається.

"Ізоляція" це справді унікальне місце в усіх сенсах. Унікальне, звісно, зі
знаком "мінус". З психологічної точки зору, навіть з точки зору психіатрії, там
панують класичні психопати і садисти, цього не помічати було неможливо.

\ifcmt
  pic https://img.pravda.com/images/doc/9/5/9509091-img-4901.jpg
	width 0.3
\fi

— В інтерв'ю для "Росії 24", яке було записано за рік після викрадення, ви
фактично визнали, що були завербовані українською розвідкою. До слова, інтерв’ю
записували в бібліотеці. Вам давали читати в ув'язненні?

— Спочатку, коли був "на підвалі", там, звісно, не було книжок. Але десь за
тиждень мені передали, втім, там було дуже темно і я фізично не міг читати.

В "Ізоляції" перші місяці я також просидів в підвалі, там книжок не було. Потім
в камері вже була полиця з книгами. Література не дуже, але читати можна було.

— Текст так званого зізнання в тому, що вас завербувала українська розвідка,
написали самі терористи?

— Ні. Вони одразу поставили таку умову, мовляв, ніхто тобі нічого писати не
буде, ти маєш розказати все, що прописано у твоїй карній справі, але своїми
словами, щоб це виглядало природно. Вони не хотіли, аби я завчив все до літери.
Вони хотіли, аби цьому повірили.

— В одному з інтерв'ю ви говорили, що в "Ізоляції" катують людей просто тому,
що отримують владу над ними. І що ви не знаєте, стали б ви мстити, якби
трапилася така нагода. Чи відповіли ви собі на це питання за рік після
звільнення?

— Це головне питання, яке я собі ставлю…

Відповіді немає, її не може бути, бо ця відповідь не теоретична. Цю відповідь
можна отримати, якщо вони опиняться на моєму місці, на тому столі, де вони
катують людей, абсолютно безпорадні.

Що відбувається у тих підвалах? Людину роздягають догола, прив’язують скотчем
так, що вона не може поворушитися, і роблять все, що завгодно. Вони
розраховують на те, що ніхто ніколи про них не дізнається. Але ми вже
встановили більшу частину адміністрації "Ізоляції". От коли вони опиняться в
цьому положенні, а мені дадуть можливість їх катувати…

Якщо будуть такі умови, тільки тоді людина може відповісти собі, хто вона така.
Буде вона катувати іншого чи ні. Це абсолютно практичне питання.

І я сподіваюся, що ніколи не отримаю на нього відповідь. Я не хочу опинитися в
ситуації, коли у мене буде така можливість, не хочу…

\subsubsection{\enquote{Книга, яку я написав, це також моя помста}}

— Ви називали "Ізоляцію", крім іншого, інкубатором сенсів. Які сенси ви
виносили в камері 5 на 7 метрів, де крім вас ще було до 10 ув'язнених?

— Насправді їх безліч, цих сенсів. Передусім, це питання, заради чого ти
існуєш. І це питання зумовлює наступне: чи варто далі жити.

Люди там не бачать майбутнього. Якщо ти не бачиш майбутнього, а над тобою
щодня, щоночі знущаються, постає досить логічне питання: а чи варто взагалі
продовжувати? Можливо треба просто покінчити з собою? Це перше питання і перші
сенси, що виникають.

Але ж є і прагматичніші речі. Адміністрація "Ізоляції" нікого там не розділяла.
Наприклад, я сидів у четвертій камері разом із трьома людьми. Один мій "сусід"
був так званим командиром бригади "ЛНР". До початку війни 2014 року —
кримінальний авторитет, близький із "злодієм в законі" з Одеси. Інший — старший
лейтенант батальйону Мотороли "Спарта". Третя людина — рецидивіст, який до
цього ув’язнення відсидів 20 років. І я. Розумієте?

Було таке, що ми прокидалися зранку і взагалі не вимовляли жодного слова — така
атмосфера, що повітря можна було різати ножем. Один одного мовчки ненавидить.
От ми сидимо в цій камері, і маємо якось виживати один з одним.

Я з ними сидів місяці півтора. Потім мене перевели до камери, де було більше
людей, але там стало простіше, бо це були цивільні звичайні хлопці.

— Чи було між в'язнями щось на кшталт братерства? Або це була повністю тюремна
соціальна ієрархія, зі стукачами, паханами, іншими ролями?

— Ні, от ніякого братерства там точно не було.

Щодо тюремних понять, це дуже цікаве питання. Адміністрація "Ізоляції"
намагалася гратися у це, але вони самі не розумілися на цих тюремних поняттях.
Бо коли я потрапив у камеру з рецидивістами, було зрозуміло, що ці люди, як
риби у воді в цих поняттях. Але вони не розуміли, що відбувається. Бо те, що
там було, не мало жодного відношення до порядків в "офіційних" в’язницях і
таборах.

По-перше, люди, які мають кримінальне минуле, не можуть сидіти разом з людьми в
погонах. Вони вже їх мали "порізати" за цими поняттями…

Є мільйон нюансів. Форма того, що там відбувалося, нагадувала кримінальні
поняття, але за суттю це зовсім відмінне від того класичного кримінального
світу, традиція якого тягнеться ще з радянських часів.

\ifcmt
  pic https://img.pravda.com/images/doc/b/b/bbd5205-img-4920.jpg
	width 0.3
\fi

— Хто з тих, з ким ви спілкувалися в камері, вас найбільше вразили?

— Справа в тому, що там про кожну людину можна зняти окремий фільм, серйозно.

Але найбільш запам’яталася історія людини, яка залишається у полоні, це Сергій
Бешта.

Його катували разом із сином, на одному столі. Його завели до нас в камеру вже
з опіками, з поламаними ребрами. З сьомої до пів на дев’яту вечора він був
повністю дезорієнтований, не розумів, де знаходиться. Він сідав скраю на нари і
тільки-но чув кроки в коридорі, був упевнений, що разом із сином знаходиться ще
в підвалі. Тобто, людина у свідомості, але постійно повторює: "Терпи, терпи,
синок…". Ми йому кажемо: "Сергію, заспокойся, все закінчилося, ти вже в камері,
вас вже не катують".

Психологічно це було важко спостерігати навіть тим, хто там вже провів
рік-півтора, і багато чого бачив.

— Що було найболіснішим, окрім фізичних тортур та знущань?

— Постійний психологічний тиск. Вони могли постійно стукати прикладом у металеві двері, навіть вночі. Не знаю, як передати це відчуття словами, але ви просто здригається і підскакуєте. 

Одне з правил — і це також психологічні тортури – коли в будь-який час доби
відчиняються двері, ви маєте підвестися з нар, одягти пакет на голову,
розвернутися до стіни, руки звести за спиною і так стояти. І коли б’ють
прикладом у двері, ви вже напоготові.

— Ваша освіта — філософія та релігієзнавство — допомогла долати жахіття буття
під час ув’язнення? Яких філософів ви згадували в камері і чому?

— Я не просто згадував, в "Ізоляції" я написав есе "Вулкан", яке повністю
присвячене філософським проблемам.

Інколи це виглядало як абсурд для людей, які потрапили до "Ізоляції". Вони
бачили, що я щось пишу або просто повторюю собі під носа тексти — розуміючи, що
їх можуть забрати, я їх просто вчив напам’ять. На мене дивилися, як на
божевільного.

Звичайно, філософія нікуди не зникла й там, але, знаєте… екзистенціалізм я вже
згадував з певною іронією. Бо багато філософів мали теоретичний, а не
практичний досвід. А коли опиняєшся в таких умовах, то "вічні питання буття" не
просто загострюються, а постають по-новому. Особливо це стосується релігійного
екзистенціалізму. І я про це також міркував.

— Які людські якості ви втратили або здобули в полоні?

— Я не скажу, що втратив щось, що не можна було б відновити…

Напевно, головна якість, яка втрачається, це співчуття до людей. Був період в
полоні, коли мені було абсолютно байдуже до тих, кого заносили до камери
закатованими, з поломаними ребрами, з опіками, вони стогнали…

Коли бачиш це щодня, рік, півтора, два, настає атрофія почуттів. Мені потрібно
було десь півроку, аби я просто прийшов до тями і почав щось відчувати до інших
людей.

А те, що можна набути в полоні, це патологічна ненависть, жага помсти, яка не
минається з часом, лише стає раціональнішою.

Книга, яку я написав, це також моя помста. Мені, до речі, в полоні сказав один
кримінальний авторитет, який відсидів 20 років: "Найгірше, що ти можеш їм
зробити, це вийти і написати книгу. Ти про це все розкажеш, бо ти журналіст, а
нас ніхто не почує".

Але навіть після виходу книги жага помсти не відпускає. Бо від цієї книги
людям, які продовжують катувати, ні холодно, ні жарко.

— Можливо, коли вони будуть на лаві підсудних це почуття зникне?

— Я маю великі сумніви, що колись вони будуть на лаві підсудних. Я не бачу
механізмів для цього. Моя стратегія в тому, щоб зробити "Ізоляцію" міжнародним
брендом, в лапках, як Освенцим. Щоб про неї постійно говорили на Заході і
тиснули на РФ.

Коли це стане проблемою для Росії, кати просто зникнуть і все. Спишуть,
можливо, на українське ДРГ (диверсійно-розвідувальну групу — УП). Це єдине, на
що можна розраховувати в цьому випадку.

Щодо Заходу, тут треба розуміти, що ми не вийшли на той рівень, щоб змусити
партнерів дійсно запитувати і тиснути на Росію.

Я зустрічаюся з представниками ЄС, ОБСЄ, ООН, Червоного Хреста, але це зовсім
не ті люди, які б могли впливати на політику РФ. Це повинні бути топ-політики
Заходу, до яких поки що інформація про "Ізоляцію" не дійшла. 

І тут я дуже сподіваюся на англомовний переклад книжки, який, до речі,
забезпечує посольство Канади. Лімітований наклад англійською передбачається не
для комерційного розповсюдження, а для західних політиків і дипломатів.
Все-таки є шанс, що Росії доведеться щось відповідати, бо ці так звані
представники ОРДО, бойовики, вони взагалі нічого не відповідають на наші запити
по "Ізоляції".

\ifcmt
pic https://img.pravda.com/images/doc/4/9/49627e1-img-4860.jpg
width 0.3
\fi

— Рік тому вас звільнили разом із українським журналістом Олегом Галазюком,
який казав, що готовий подавати позов проти Путіна особисто. Ви обрали інший
шлях — розповідати про концтабір. Що може виявитися ефективнішим?

— Думаю, точкова інформаційна кампанія про "Ізоляцію". Чому? Бо треба розуміти,
що лише в Донецьку є більше десятка місць, де катують людей. Можна казати
загалом: когось десь катують. А можна бити прицільно і показувати, що
"Ізоляція" — це індикатор всього, що там відбувається.

Адже там повний набір воєнних злочинів: катування, вбивства, зґвалтування,
приниження людської гідності, залучення до важких примусових фізичних робіт, бо
в’язні там будують полігон, ріжуть метал.

Вже є відео зсередини цієї катівні, я впізнав навіть власну камеру, але коли ми
запитуємо про "Ізоляцію" у Мінську, вони (представники "Л/ДНР" в Трьохсторонній
контактній групі — УП) заперечують, що це місце існує. І це говорить про те, з
ким ми ведемо перемовини і з ким маємо справу.

— За вашою інформацією, скільки полонених в "Ізоляції" зараз?

— Останні дані про "Ізоляцію" особисто у мене були на травень 2020 року. Там
знаходилося більше 50 осіб.

Що там зараз відбувається, взагалі ніхто не знає. По кількості вбитих і
закатованих теж не відомо. Заступник генпрокурора Мамєдов казав, що через
"Ізоляцію" з 2014 року, якщо не помиляюся, пройшло 3,5 тисячі людей. З них
трохи більше тисячі – цивільні.

\subsubsection{\enquote{Іноді спеціально переглядаю довоєнні фото Донецька, щоб відчути якісь емоції}}

— Колись ви мріяли вступити до Французького іноземного легіону і навіть поїхали
до Франції. Що з цієї історії запам'яталося в форматі "мрія — реальність"?

— (Усміхається.) Тоді це була зовсім інша людина. 2012 рік, я тільки-но
закінчив факультет філософії університету. Це дуже вплинуло на свідомість, ще й
Макіївка, я не бачив жодних перспектив у власному місті. І все це у сукупності
спровокувало те, що я поїхав до цього легіону.

У мене з дитинства травма очей — штучний кришталик, з цим вже не візьмуть у
легіон, і я це розумів. А за 20 днів до того, як я поїхав, ще й пошкодив ногу.
А в цьому легіоні біг — це релігія, там бігають абсолютно всі. І коли я приїхав
туди кульгавий і з поганим зором, у мене не було жодних шансів.

Я не можу сказати, що був ідіотом, який цього не розумів. Я все розумів.
Доросла людина, але дитина водночас.

Коли я приїхав, мене зустрів офіцер, який був одеситом, до речі. Я звернувся до
нього англійською, він відповів російською: "Даже не начинай, понятно". Завдяки
йому я три дні там прожив, але, звісно, до лав легіону не потрапив.

Це була лише мрія, але вона настільки мене захопила, що я все одно туди полетів
би, бо мені було важливо перейти цей поріг, побачити все зсередини. Я був наче
дитина, яка ще не бачить вектору свого життя і думає, що легіон — це відповідь
на все.

— Після повернення з Франції ви змінили п'ятнадцять професій, серед яких
вантажник, стажист в банку, копач могил, оператор в поштовій компанії і
продавець-консультант побутової техніки. Яка з них найбагатша на сюжети для
письменника?

— (Сміється.) Так, це, звичайно, про могили, але у мене був недовгий досвід
копання.

Це все описано в моїй книзі, усі митарства по робочим полям Донбасу. Це цікавий
досвід, бо я побачив соціальне дно, де перебувають мої земляки разом зі мною.

Після закінчення університету мені пропонували вступати в аспірантуру, але я
розумів, що з фінансової точки зору це взагалі ні про що. Це дуже довгий шлях,
на роки, аби чогось досягти. Тому я почав обирати все, що можна було знайти в
газетах, і працював фізично.

Це були специфічні професії, із специфічними людьми. Власне, і вагони я
вантажив — це була ще фабрика "Конті" в Макіївці — також разом із людьми з
кримінальним минулим. Мені було цікаво з ними перетинатися, бо я щойно вийшов з
університетської кафедри, магістр, а тут на тобі, вагони…

— Якщо говорити про атмосферу Донецька до моменту вашого затримання в 2017
році, як її можна описати?

— Це радянське місто. Мене часто запитують, який зараз рік в Донецьку. І це
цікаво, бо з огляду на те, що відбувається в "Ізоляції" та підвалах, це,
безперечно, репресії 1930-х. Але якщо дивитися на загальну картину, то це
радянське минуле, кінець 1970-х, 1980-ті роки.

Люди, з одного боку, дуже сподіваються на РФ, пишаються, що Росія їх підтримує.
А з іншого боку, я б не сказав, що з боку Росії є якась пропаганда радянських
70-х. Але в Донецьку саме вони. 

Ще більше це відчувається в Луганську. Я там не був, але сидів з людьми з
Луганська, і вони розповідали, що насправді Донецьк це ще непогане місто
порівняно з Луганськом.

— Наче людей закинули в інший вимір. Вони не відчувають різниці, сплять, це
навіювання?

— Дуже багато людей і не виходили з того радянського стану. І коли їх занурили
у 70-ті, 80-ті роки, вони там опинилися, наче риба у воді. Бо там завжди була
ностальгія за радянськими часами.

Донецьк 2012-го та Донецьк 2015-го — це абсолютно різні міста. Не лише тому що
війна і зруйнована інфраструктура. Це зміни психологічні, ментальні. Навіть
якщо газети 2015 року відкрити, там буде таке: "высадили еще 5 кустов роз,
посадили акацию, Захарченко молодец".

\ifcmt
  pic https://img.pravda.com/images/doc/c/0/c07105b-img-4890.jpg
	width 0.3
\fi

— Ви залишилися жити в окупованому Донецьку через рідних — маму і двох бабусь.
Ігор Козловський, ваш учитель, який теж був в полоні, не міг залишити хворого
сина. Таких прикладів безліч. Дехто просто говорив: куди ми поїдемо, нас там
ненавидять, залишимося під кулями, тому що нелюбов страшніше за смерть. Певною
мірою це реакція на категоричне: хто був патріотом, ті поїхали, залишилися
зрадники. Що б ви відповіли людям, які це повторюють?

— Насправді є заручники ситуації, вони не можуть виїхати, але зберігають
проукраїнську позицію, інакше, власне, і підвалів би там не було стільки. Бо в
них утримують дуже багато людей з проукраїнською позицією.

Але з огляду на загальну кількість тих, хто залишається зараз в окупації, я б
сказав, що людей з проукраїнськими поглядами насправді невеликий прошарок.

Але не можна казати, що залишилися лише зрадники і колаборанти. Коли я робив
свої репортажі з окупованого Донецька і чув щось подібне від своїх колег у
Києві, мені було образливо. Але я розумію, що це емоція, в тому числі, тих
людей, які вимушено виїхали у 2014-му та почали життя з нового аркуша, у них не
залишилося нічого…

— Колись Росія піде з Донбасу і з Криму. З точки зору історії, це неминучий
процес, питання в тому, скільки часу він займе. Коли це трапиться, як бути з
розломом, який неминуче проляже між тими, хто жив в окупації та голосував за
приєднання Криму до Росії, і рештою населення України? Чи уявляєте ви, як цей
розлом можна зшити?

— Ніяк. Людей, які вже визначилися зі своєю позицією, не переконаєш. Це питання
наступних поколінь, на них треба робити ставку.

Власне, це те, що вже зараз робить РФ, в тому числі, на окупованих територіях.
Вони дуже масштабно промивають мізки молодих людей, починаючи з перших класів,
закінчуючи університетами. Знову ж таки, це те, що Росія розуміє, і чого не
розуміє наша держава.

І попередня українська влада, і теперішня якось поверхнево дивляться на Росію.
Вони вважають, що РФ це лише танки і гармати. А це наслідок, а не причина. І,
звичайно, потрібно донести думку, що в інформаційну безпеку ми маємо вкладати
великі кошти. Цього не відчувають, бо це неможливо помацати. Якщо танки ви
бачите, то тут не зрозуміло, на що витрачати гроші.

А щодо тих людей, які там залишилися, то це звичайно, криза ідентичності. А як
може бути інакше? Частина людей ностальгує за Радянським Союзом. Потім їм
кажуть: ви "Донецька народна республіка". Потім кажуть: ні, ви будете частиною
"Новороссии", потім Захарченко придумав "Малоросію". Потім — можливо, вас все ж
таки приєднають до Росії, ось вам паспорти, але приєднання не відбувається. 

От такий хаос в мисленні, і все це підкріплюється інформаційним простором, коли
ви вмикаєте місцеве телебачення, там жахливі речі розповідають. Звичайно, що
криза ідентичності буде лише поглиблюватися. І російські паспорти не
допоможуть. 

Людина не може відчути себе росіянином, отримавши паспорт РФ, бо є питання
приєднання території: чому ми не стаємо як Крим частиною РФ? А цього, скоріше
за все, ніколи не відбудеться. Але не маючи жодного доступу до тих територій,
ми нічого вже не вирішимо.

Це жорстокий експеримент над людьми. В широкому сенсі, всі окуповані території
— це одна велика "Ізоляція".

— Свого часу була популярна фраза "Донбас не чують". Чи бачите ви частину
провини України за те, що трапилося на Донбасі і в Криму, що дозволило Росії в
2014-му окупувати суверенну територію та влаштувати тут гібридне пекло?

— Безперечно є. По-перше, це стосується наших спецслужб і державних органів,
які були неспроможні функціонувати навесні 2014 року, коли вони мали всі
можливості розігнати тих маргіналів, коли вони ще такими залишалися, без
російських танків та градів. 

Це було можливо. Місцеві еліти, вертикаль наших спецслужб у Києві цього не
зробили. 

Але проблема ще в інформаційному просторі. Катастрофічна інформаційна політика
протягом останніх 20 років призвела до того, що зараз маємо на Донбасі. І це
проблема нашої держави.

— Ви взялися говорити про найстрашніше, незважаючи на свіжі рани. Ви зараз
більше там, в Донецьку, чи столиця вас потроху затягує?

— Більшу частину часу я мешкаю в Броварах. У Києві пересуваюся за навігатором,
і географічно, і за відчуттями, якщо можна так сказати. Але я розумію, що це
дуже потужне місто, де мені було б насправді некомфортно жити через велике
скупчення людей. Київ — це такий мурашник.

Щодо Донецька, насправді я вже не пам’ятаю, який він. Намагаюся пригадати, але
не можу. Навіть іноді спеціально переглядаю довоєнні фото Донецька, щоб відчути
якісь емоції до тих спогадів, і не можу цього зробити. Два з половиною роки в
"Ізоляції" настільки багато додали переживань, що про саме місто не йдеться.
Лише про загальнолюдські речі, з якими я досі не можу розібратися…

Розмовляла Тетяна Селезньова, для УП

