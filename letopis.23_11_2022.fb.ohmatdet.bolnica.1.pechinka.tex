% vim: keymap=russian-jcukenwin
%%beginhead 
 
%%file 23_11_2022.fb.ohmatdet.bolnica.1.pechinka
%%parent 23_11_2022
 
%%url https://www.facebook.com/ndslohmatdyt/posts/pfbid02YQh7SroNGeLGEjpHmkZ4hTzDNY1PXPY2CQ43QQyo1zEqMqbc1PUa5XiGC2iJeWFZl
 
%%author_id ohmatdet.bolnica
%%date 
 
%%tags 
%%title Дівчині з вродженою хворобою пересадили печінку від загиблого в аварії 17-річного хлопця
 
%%endhead 
 
\subsection{Дівчині з вродженою хворобою пересадили печінку від загиблого в аварії 17-річного хлопця}
\label{sec:23_11_2022.fb.ohmatdet.bolnica.1.pechinka}
 
\Purl{https://www.facebook.com/ndslohmatdyt/posts/pfbid02YQh7SroNGeLGEjpHmkZ4hTzDNY1PXPY2CQ43QQyo1zEqMqbc1PUa5XiGC2iJeWFZl}
\ifcmt
 author_begin
   author_id ohmatdet.bolnica
 author_end
\fi

«Моя печінка могла ще працювати лише декілька місяців»: дівчині з вродженою
хворобою пересадили печінку від загиблого в аварії 17-річного хлопця💔

Зараз Вероніці 19 років. Про своє тяжке спадкове захворювання, хворобу
Вільсона-Коновалова, вона дізналася у 10 років. Ця хвороба призводить до
надмірного накопичення міді в тканинах, першою страждає печінка. Стан Вероніки
різко погіршився наприкінці жовтня. Вероніка розповідає, що була дуже слабка, а
колір її шкіри змінився на жовтий.🙏🏻

До Охматдиту Вероніка потрапила з ознаками гострої печінкової недостатності і
білірубіном понад 200. Печінка у дівчини функціонувала на 20\%. Стан Вероніки
був критичний, тому фахівці вирішили проводити трансплантацію. Це був єдиний
шанс зберегти її життя.⚡️

На родинну трансплантацію погодилися родичі дівчини. Але підготовка донорів до
пересадки займає багато часу, тож паралельно дівчині шукали посмертного донора.
Український Центр Трансплант Координації (УЦТК) надав перший статус екстреності
пацієнтці Охматдиту. Це означає, що такого пацієнта система розглядає як
найпріоритетнішого для отримання донорського органу.❗️

Вже протягом трьох днів після надання статусу ЄДІСТ підібрав для Вероніки двох
посмертних донорів, але їхня печінка виявилася непридатною для трансплантації.
Зрештою донора знайшли на Одещині. Ним став 17-річний хлопець, який загинув у
автокатастрофі. Після констатації смерті мозку, родина хлопця дала дозвіл на
трансплантацію органів. Світла пам’ять донору та шана його родині🙏🏻

Для забору органів виїхала команда Охматдиту. Складна логістика та шлях понад
400 км треба було продумати до дрібниць, аби зберегти органи. Поза межами
організму печінка може бути придатною для пересадки лише 8 годин. Кожна
трансплантація — це одразу три хірургічні втручання: забір органа у донора,
back-table та пересадка реципієнту. Аби зберегти час, друга бригада наших
спеціалістів паралельно готувала Вероніку до трансплантації у Києві. Вероніка
була в операційній понад 15 годин. Разом з командою Охматдиту працювали
спеціалісти лікарні Феофанія. Трансплантація пройшла успішно.🙌🏻

Вероніка розповідає, що дізналася, хто став її донором та врятував її життя
коли була у реанімації. «Це був 17-річний хлопець. Він на два роки молодший за
мене, він загинув у аварії,» — каже дівчина та починає плакати.💔

Мужнє рішення родини цього хлопця врятувало життя одразу трьох людей. Це став
перший випадок в Україні, коли посмертним донором органів став неповнолітній. В
Охматдиті, окрім печінки, спеціалісти лікарні пересадили нирку донора 9-річному
хлопчику. Останні пів року він жив у відділенні інтенсивної та еферентної
терапії хронічних інтоксикацій, бо його життя залежало від процедури
гемодіалізу, що заміняє роботу нирок. Ще одну нирку пересадили львівські
лікарі.🏥

За два тижні після трансплантації дівчину виписали з Охматдиту. На прощання
вона намалювала лікарям-трансплантологам патріотичні малюнки, аби віддячити їм
за врятоване життя.❤️🩹

«За 2022 рік в Охматдиті провели 14 органних пересадок (нирок і печінок).
Трансплантації печінок в Охматдиті запустили під час війни. Пересадка для
Вероніки стала першою в Охматдиті трансплантацією печінки від посмертного
донора. Команда спеціалістів лікарні продовжує працювати, аби рятувати якомога
більше життів»,— каже хірург-трансплантолог Олег Годік.💪🏻

Величезна мультидисциплінарна команда Охматдиту працює з пацієнтами, які
чекають на трансплантації. Лікарі готують пацієнта до трансплантації, проводять
пересадку органу та виходжують після операції: це хірурги-трансплантологи,
трансплант-координатори, анестезіологи, реаніматологи, радіологи, діагности,
анестезистки, педіатри, нефрологи, урологи, співробітники лабораторії, команда
медсестер, лаборанти. За потребою, долучають інших фахівців.🤝

Трансплантація нирок та печінок, усе лікування пацієнтів, які потребують
пересадки органів, в Охматдиті безкоштовне. 

Дякуємо за плідну співпрацю МОЗ та УЦТК.🇺🇦

🇺🇦 ‼️Наша команда готова допомогти українським дітям, які потребують пересадку
печінки. За допомогою звертатись до:

хірург-трансплантолог Олег Святославович Годік — 0 (50) 543 37 81.

трансплант-координатор Вікторія Анатоліївна Апалькова — 0 (99) 365 90 56

🔻При потребі замісної ниркової терапії, ви можете звернутися за консультацією:

* 044 236 52 13 — відділення інтенсивної та еферентної терапії хронічних інтоксикацій;

У відділення лікують пацієнтів з нирковою недостатністю.

Oleg Godik Василь Недбала Вікторія Апалькова Olga Babicheva МОЗ України МОЗ
України Український центр трансплант-координації Феофанія клінічна лікарня

\ii{23_11_2022.fb.ohmatdet.bolnica.1.pechinka.orig}
\ii{23_11_2022.fb.ohmatdet.bolnica.1.pechinka.cmtx}
