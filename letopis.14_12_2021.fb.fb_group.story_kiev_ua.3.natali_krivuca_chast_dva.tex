% vim: keymap=russian-jcukenwin
%%beginhead 
 
%%file 14_12_2021.fb.fb_group.story_kiev_ua.3.natali_krivuca_chast_dva
%%parent 14_12_2021
 
%%url https://www.facebook.com/groups/story.kiev.ua/posts/1818920678304725
 
%%author_id fb_group.story_kiev_ua,fedjko_vladimir.kiev
%%date 
 
%%tags isskustvo,kiev,ukraina,zhizn
%%title Життя – це свято. Розмова з Наталі Кривуцею (Частина II)
 
%%endhead 
 
\subsection{Життя – це свято. Розмова з Наталі Кривуцею (Частина II)}
\label{sec:14_12_2021.fb.fb_group.story_kiev_ua.3.natali_krivuca_chast_dva}
 
\Purl{https://www.facebook.com/groups/story.kiev.ua/posts/1818920678304725}
\ifcmt
 author_begin
   author_id fb_group.story_kiev_ua,fedjko_vladimir.kiev
 author_end
\fi

Життя – це свято: твори мистецтва в інтер'єрі сучасної квартири.  Розмова з
Наталі Кривуцею (Частина II)

***

З Наталі Кривуцею – відомим мистецтвознавцем і громадським діячем, доктором
філософських наук, президентом МГО «Міжнародний центр розвитку громадянського
суспільства «Ініціативи нації», полковником миротворчих сил, організатором
фестивалів «Сонце світить всім» Міжнародної програми «З усім світом разом» і
чарівною киянкою, я познайомився у далекому 1975 році, коли Наталі після
закінчення Київського художнього інституту прийшла працювати в Дирекцію
виставок спілки художників УРСР.

З Наталі ми мали велику розмову про роль мистецтва в житті і про твори
мистецтва в інтер’єрі сучасної квартири. 

Перша частина вже опублікована в Клубі «Киевские истории», а це продовження
нашої розмови.

***

Ви знаєте, мене дуже радує, що коли я приїжджаю, наприклад, в Англію, то бачу
старовинні замки, які переобладнані на сучасний манер не зовнішнім своїм
сприйняттям, а своєю соціальною структурою. Це сучасні зручності, які внесені
до будівлі. Подивіться, як живе римлянин, як живе неаполітанець, як живе
венеціанець, як живе парижанин. Ніде нічого не валиться, і як мудро все
вплітається і переплітається з однієї епохи в іншу.

\ii{14_12_2021.fb.fb_group.story_kiev_ua.3.natali_krivuca_chast_dva.pic.1}

У нас теж є, до речі, дуже хороші приклади. Це Львів, Кам'янець-Подільський,
дуже багато таких містечок, в яких вміють зберігаючи старе, набуваючи нове, не
руйнуючи, в загальний колорит вносити ті сучасні ноти, які говорять і
зливаються в прекрасну симфонію сучасного життя. Неправильно зруйнувати старе,
побудувати нове і в той же час привнести елементи того, чого ти десь бачив в
книжці в свій домашній побут. Все дуже складно. І якщо говорити про культуру
життя як такого, перш за все потрібно подумати: хто ми, звідки, і куди ми
йдемо. І перш за все – для чого! Ми будуємо свій будинок для того, щоб ми жили
і сьогодні, і жили завтра наші діти. І повірте, то, як ми обставляємо квартиру
сьогодні, буде абсолютно непотрібно і неприйнятно нашим дітям. Я згадую, з якою
любов'ю наші батьки обставляли свої будинки тими меблями шістдесятих років, яку
зараз смішно внести в наші сучасні житла. Я пам'ятаю, як моя мама говорила:
«Боже мій, ні про що не думай, у мене все є, ось у тебе все буде після того як
...» І як би мені не було гірко, боляче і прикро, але я всього цього втратила ,
тому, що я вже інша людина, у мене інші завдання, інші смаки. І я в свій
інтер'єр принесла все своє. Очевидно для них, для моїх батьків, це було б
неприйнятно, тому що ми формуємо свою свідомість в тому середовищі і в той час,
в якому ми живемо.

\ii{14_12_2021.fb.fb_group.story_kiev_ua.3.natali_krivuca_chast_dva.pic.2}

Добре, коли ми такі розумні, коли ми такі гнучкі, коли ми стільки багато знаємо
і стільки багато бачимо, і коли ми кожні десять-двадцять років можемо собі
дозволити по нашому настрою, по наших грошах і можливостям, і по нашому
розумінню змінювати свій інтер'єр . Навіть міняти свої картини, наприклад, або
свою скульптуру, або те, що є у нас в інтер'єрі. Але для цього треба бути
колекціонером. Тому що, коли ми колекціонуємо роботи, ми можемо собі дозволити
їх міняти. Ми можемо собі дозволити робити галерею образів. Тобто, ось це
спосіб мого життя двадцять років тому; а це спосіб мого мислення і розуміння
культури живопису десять років тому, або тридцять років тому. В результаті буде
цілий розповідь про твоє життя. А якщо ми беремо просто картинку в інтер'єр для
оформлення інтер'єру, це зовсім інше.

Я дуже складно ставлюся до цих питань, і вважаю, що не можна давати рецепт. І
рецептів немає. Кожен колекціонер визначає для себе, що він збирає. Або він
збирає вінегрет – все, що попадається, для того, щоб перепродати, або просто
для того, щоб сидіти на це дивитися. Або він збирає для того, щоб вибудувати
музей і потім все це показати.

Я шалено поважаю таких людей, і у нас їх тепер стає все більше і більше.
Відкриваються музеї сучасного мистецтва, відкриваються хороші галереї і
відкриваються галереї антикварні, де люди рятують те, що було у когось раніше,
а потім стало не потрібним, і було викинуто на смітник. І я дуже поважаю тих
людей, які розбираються в цьому, і добре знають, навіщо вони це роблять. Тому
що коли людині дарується робота, він іноді не розуміє навіщо, і тим більше не
відчуває чому.

Збиральництво – дуже особистісний і дуже глибоко інтимний процес. Людина
повинна усвідомити, відчути і захотіти зайнятися цим. Це як любов. Як би тобі
не сватали якогось нареченого, поки ти не полюбиш цього чоловіка, якого ти
бачиш перед собою, ти ніколи оргазму не отримаєш з нелюбимою тобою людиною,
хоча яким би красивим і дорогим він не був. Тому внесення картини в свій
будинок – це як вийти заміж за самого-самого-самого ... Бо привносячи її в
будинок – це як заручення. Заручини і клятва вірності на все життя. Ти вносиш
цей твір до себе додому. А коли тобі дарують, це як біжутерія, яку ти можеш
любити, носити і добре до неї ставитися, а можеш до неї ставитися поверхово.
Хоча може бути так, що будуть і дорогі, і хороші, і чудові камені. Може бути не
найдорожчі, але і не найгірші. Дуже добре, що зараз почали взагалі-то один
одному дарувати хороші речі. І радяться з цього приводу. Але ... Ми всі різні.
Хтось любить блондинок, хтось – брюнеток. Хтось любить натюрморт, хтось любить
пейзаж. Хтось любить історичний живопис, хтось колекціонує ікони. І не можна
нікого засуджувати, і не можна нікому допомагати у виборі. Це може бути тільки
тоді, коли людина не будучи освіченою спеціально, в той же час усвідомлює міру
відповідальності перед собою та своїми грошима, вкладеними кудись, бере собі
консультанта, і консультанта грамотного, а не «з нових», які говорять, що все
знають, і при цьому досить одного погляду, щоб зрозуміти, що все що вони
знають, це далеко забуте минуле, яке у них не отримало сьогодення. І коли
використовуючи поради, людина все одно усвідомлено робить той вибір, який йому
допомагає в цьому житті розкрити себе, і іншим показати і свій смак, і свої
можливості.

Давним-давно, років двадцять або двадцять п'ять тому, я вмовила одну людину,
відому в певних колах, придбати кілька робіт у мене. Для себе, для свого
інтер'єру. Він чинив опір, він кричав, що це йому не треба. А я йому: «Ну ти
маєш гроші, ну візьми – це недорого». Він кричав: «Це недорого !?», – в той
час, ну коли тисяча доларів це були величезні гроші, а п'ять – це було
захмарно! Я йому говорила: «Ти не розумієш. Це варто настільки дорожче, що ти
колись будеш пишатися, що у тебе робота ця в будинку». Минуло двадцять років.
Він мені подзвонив і сказав: «Боже мій, яка ти розумна, як я тобі вдячний! До
мене приїхав англієць і коли побачив, що у мене висить ..! Він мені пропонував
сто тисяч, стояв на колінах і умовляв, що це треба для нього в його будинку. Я,
– каже, – хотів спочатку продати, а потім подумав: що я жлоб! Що я ось так от
візьму і віддам те, перед чим у мій гість стоїть на колінах! Не віддам ні за
що!» І я зрозуміла, що як важливо іноді підказати правильно людині, щоб його
подяка поширилася на довгі роки. І залишилася пам'яттю в його родині на додаток
до того культурного прогресу, який може бути, ніколи б і не здійснився, якщо б
не прийшов цей твір в їх будинок.

***

Кожна колекція має свої межі. Десять робіт це теж колекція. П'ятдесят робіт –
це величезна колекція. Сто робіт – це вже майже унікальна колекція. Коли вона
перевалює за сто-двісті-триста – це вже неприйнятно по нашому сучасному способу
життя мати таку колекцію. Виходить, що ти весь час на неї працюєш. Вона вимагає
уваги, турботи, показу, публікацій, виставок, поїздок. Це можуть собі дозволити
далеко не всі люди. Дуже важливо, що почали будувати сучасні приватні художні
музеї. Це чудово. Тому що людині завжди хочеться похвалитися тим, що він має. І
він не буде тримати у себе картинку в підвалі, як у нас багато музеїв це
тримають в запасниках. Коли всі музеї хваляться тим, що «у нас в запасниках в
десять разів там або в двадцять, або в тридцять більше, ніж в експозиції». Це
нечесно по відношенню до роботи. Вона була створена і вона повинна жити. Вона
повинна когось радувати. Вона повинна бути обов'язково публічною. Вона повинна
бути відчутна. Її потрібно бачити, нею потрібно милуватися, як красивою жінкою,
яка тоді розквітає. А коли на неї ніхто не дивиться, і вона сидить десь на
самоті – вона марніє. Точно так само і будь-яка робота. Те, що дуже багато
творів мистецтва зараз виявляються в приватних руках, я тільки вітаю, руками,
ногами, всіма лапами. Тому що людині, якій властиво це купити, властиво це
показати. І якщо він показує – робота живе. Вона набирає сил, вона не може
померти, не може померти її дух і її енергетика. 

Ви знаєте, ми якось не
замислюємося над тим, що є триєдність сприйняття і впливу роботи. Ось Ви ніколи
не замислювалися над тим, що коли художник вийшов писати, наприклад, етюд на
природу, у нього є певний настрій. Встав він з правої ноги, з лівої ноги,
поцілувала його дружина, дали йому випити кави або він пішов голодний і з
поганим настроєм. Є стан природи: ранок, вечір, день; сонце світить, дощ іде.
Певний стан. Як передав до свого настрою це стан природи художник – це вже
зовсім третє завдання, яке вже ми повинні як глядачі розглянути. 

А у нас теж є
своє – і настрій, і стан. І має відбутися злиття настрою художника, його
азарту, його можливостей в той момент, коли він написав цей пейзаж в певному
стані ранку, вечора, заходу або сходу з тим настроєм, з яким я, як глядач,
прийшла на нього подивитися. Якщо співпадуть всі три стани – буде чудове
почуття любові, яке ми відразу відчуємо до цієї роботи, тому що ми всі втрьох,
і те, що я бачу, і той, хто написав, і я, яка дивлюся, на одному диханні
сприйняли і злилися в екстазі любові до того, що ми бачимо.

***

Не можна розривати зв'язок часів і поколінь. Але раз так уже сталося, що
розірвали, то будь ласка, якщо у вас є гроші, розум і можливість, є бажання, і
є культура внутрішня, почніть писати історію свою спочатку, від себе, і будьте
основоположником того чудового роду і тієї неодмінної культури побуту вашої
родини, яку ви, як кажуть «започаткуєте сьогодні, а вона продовжиться ще й
завтра». Тому що ваші діти будуть дуже цінувати, що залишилося від вас. І якщо
їм все дістанеться просто так, і вони не буду розуміти наступності, все піде
шкереберть. Нічого нікому не дорого, якщо це їм не дорого.

Чому не купити свіжу роботу живого художника, і не привнести її з радістю як
нову річ, як нову дружину, як наречену в свій будинок. І не любити її, і не
постаріти разом з нею. І не отримати з неї свою історію. Історію своєї любові і
свого життя. Ось у мене кожна робота – це робота тільки та, яку я купувала. 

Моя колекція формувалася давно. Вона формувалася, скажу без удаваної
скромності, майже сорок років. І перша робота у мене знаходиться ось тут ...
Вона у мене одна з основоположних моїх робіт, тому що вона була подарована.
Вона була подарована мені, коли мені було сімнадцять років. Художник Коровай з
Івано-Франківська сказав: «Дівчинка, нехай це буде на пам'ять про мене».
Розумієте, ось це сказане добре слово: «Дівчинка, нехай це буде на пам'ять про
мене», визначило все, що у мене збереглося – це на пам'ять. Це не тільки на
пам'ять, але це спогади і про час, і про людей, з якими я зустрічалася. Не
можна сказати, що все що у мене було в житті збережено зараз. Це неможливо.
Дуже багато чого змінюється, дарується, продається, це колекція, вона
рухається, це живий організм. Але все, що залишилося, це вже те, що ось Ви
бачите в будинку, це ніколи не винесе звідси. Тому що це те, що дорого моєму
серцю, і те, що залишиться у мене назавжди.

У мене є прекрасні реалістичні роботи, пошукові роботи, може бути десь схожі
власні екзерсиси художників з приводу імпресіонізму, або модернізму, або ще
інших напрямків ... Мені б не хотілося завжди українців порівнювати з кимось. Є
певні напрямки у французькій художній школі, світовий напрямок мистецтва. І
художники не можуть бути відірвані від цього. Все одно щось буває первинне,
щось вторинне. Я не хочу сказати, що мистецтво України вторинне, Боже упаси.

Але просто йдучи тими шляхами, якими вже були прокладені в Європі, наші
художники в умовах саме андеграунду, створили найкраще, що вони робили не для
продажу, а робили для себе. У той час існувало соціальне замовлення, і
художники, які виконували соціальне замовлення, жили непогано. А решта, які
були не допущені до цього соціального замовлення – ні до скульптури, ні до
монументального мистецтва, ні до тієї ж книжковій графіці, – робили для себе,
як кажуть письменники, вони це робили «в стіл». 

Так ось я купувала ті роботи, щодо яких художники розуміли, що вони їх ніколи
не продадуть. Саме тому у мене зібралися ті перлини, які колись робилися для
себе, а не для продажу. Які потрапили до мене, і я їх, теж розумію, ніколи не
продам. І тепер, дивлячись на них, я згадую той час, коли ми були молоді і
щасливі. Коли ходили по майстернях не для того, щоб купити, а для того, щоб
подивитися; не для того, щоб щось вирішити, а щоб поговорити; не для того, щоб
показати себе, а для того, щоб побачити когось; і не для того, щоб розповісти
що ти можеш, а для того, щоб побачити, що можуть інші.

***

Перш за все, що мене вразило в Америці ... У кожного в будинку висять картинки.
Це чудово. Вони до цього прагнуть. Причому у студентів це постери і плакати –
така ось молодіжна культура. А вже у більш солідних є і живопис, і скульптура у
багатих. Взагалі пристрасть до колекціонування розвивається у них протягом
двохсот років. Тобто з першими переселенцями це стало потребою. І потім я була
повністю вражена, коли люди з грошима приїжджали до Європи і купували цілими
палацами, цілими кімнатами і вулицями. І перевозили в Америку. Більш того, мені
сподобалося, що коли речі перестають бути потрібними в інтер'єрі, то вони все
це передають в музеї. І там написано: це від того-то, це від Генрі, це від
Сімпсона, це від Петерсона ... 

І ти йдеш по музею, де прямо цілі розділи
присвячені людям, які передали експонати в музей. Цим пишається музей і не
приховує, що їм передано і хто передав. Цим пишається родина. Коли я потрапила
в музей в Сан-Франциско, то бачила як прийшов хтось із членів сім'ї і говорив,
показуючи на експонат: «Ось це було привезено моїм дідусем, ми передали в
музей». І всі стоять і захоплюються! Коли ти заходиш до музею в Філадельфії і
бачиш гобелени на твори Рубенса, які теж подарунки. Це з глузду з'їхати можна!
Держава цього не купує. Це все купують американці, настільки вони патріоти.
Вони купують, купують і дарують. 

І коли заходиш в Метрополітен, то половина
експонатів подарунки. Щастя, що люди до цього так ставляться. Мені подобається
що за ними не стоїть дванадцять, п'ятнадцять, п'ять тисяч років, а у них двісті
з маленьким шматочком. І вони все збирають по крихтах, і пишаються кожним днем
свого життя. Це чудово! І тому, коли звучить гімн, то вони прикладають руку до
серця. Американці дорожать тим, що у них є. А у нас занадто багато всього. Нас
дуже багато, і ми дуже давно. А оскільки у нас багато всього за дуже довго, то
ми нічим не цінуємо, тому що у нас попереду всього багато. 

Більш того, ніхто не
знає, що у нас зруйнують завтра в нашому майбутньому. Хто прийде і що забере! І
тому, мені здається, наша слов'янська недбалість взагалі якось стоїть осібно в
навколишньому світі ... Такого немає ні в Європі, ні в італійців, ні у німців,
ні у французів. У нас взагалі-то якесь ненормальне ставлення до антикваріату,
тому що у всьому світі немає поняття «антикваріат». У них є «ретро». Адже вони
живуть в цьому інтер'єрі від бабусь, прабабусь і прадідусів. Якщо їм щось не
подобається, то вони можуть це продати і замінити чимось іншим. І за річчю не
йде ніякої гидоти, бо річ належить родині. Сім'я з нею або розлучається або
залишає у себе. Це органічно. 

А в нашому ж випадку, в нашій історії, все що пов'язано з минулим – це
кримінал. Тому що це було в чиїхось родинах. Потім це все прийшли і забрали.
Передали або музеям, або вкрали ... І за кожною роботою йде або вбивство, або
пограбування, або насильство, або перепродаж, або ще якісь важкі долі. Скрізь
йде негатив. І тепер він передається, цей негатив, з покоління в покоління. Я
взагалі вважаю, що у нас чужу річ в свій будинок принести – це жахливо. У нас
все було відібрано, і за цим стояла своя трагедія.

Потім річ хтось перекуповував, або хтось крав, або хтось перевозив ... і потім
вона приходить до тебе з цієї важкої історією. Історією свого життя, своїх
поневірянь. А ти, несучи її в будинок, повинен потім думати, хто до тебе прийде
і що скаже. І що після цього подумає: як робота у тебе опинилася, і яка сволота
ти, що це виявилося у тебе. Тому що у нас не прийнято, щоб краса яку творили
колись для когось належала комусь і зараз. Тому що у нас за всім цим йде 80
років – відняти, розділити і не пустити.

***

Ставлення до мистецтва в Америці і в Японії різне. В Японії я повністю була
вражена, коли зайшовши в одну з телестудій побачила там дорогі роботи
європейського художника дуже високого класу, які не висять у кабінетах. Вони
висять в коридорах і в громадських залах будівлі, тому що це престиж фірми.
Показати якого класу роботи у них висять і як вони пишаються тим, що їхня фірма
має можливість і право їх придбати. І кожен може пишатися, не тільки з тих, хто
служить, але і ті, хто туди приходить. Це гордість фірми. Тобто це зовсім інша
позиція.

Пригадую шістнадцятий округ Парижа, ми входимо в мерію ... Я йду і розумію, що
або я зійшла з розуму або ... По дорозі в туалет, в темному коридорі висять
Мане, Пісарро, Коро. Я за своєю ментальністю розумію, що це, напевно, копії. А
коли підходжу, дивлюсь і бачу оригінали – я в жаху. Як це так? У таких
коридорах висять музейні роботи. Потім починаю згадувати, що це жах, жах, жах
для мене. І це радість, радість, радість для француза. Тому що ці художники як
жили, як працювали, там і залишилися. І ними навіть не пишаються. Їх
використовують, як користуються небом, повітрям, землею, життям. Вони жили, і
вони залишили свій слід. І цей слід нікуди не зник. І цей слід скрізь і всюди
поруч зі мною. І в тій же мерії, де роботи просто висять в коридорах, бо
художники їх жителі, це їх парижани. Це не наші Пісарро, Мане і Коро, яких ми
бачимо в книжках. Тому що для нас ось цей момент зовсім неприйнятний ...
Абсолютно неприйнятним, тому що у нас якщо Айвазовський і Шишкін – значить в
музеї. Якщо Рокотов і Боровиковський – ні в якому разі не вдома. А якщо вдома,
то як він до тебе потрапив, і звідки у тебе такі гроші? 

У нас більше питань,
ніж відповідей, на стан робіт, які повинні прикрашати наше життя. А у них
більше питань і відповідей, як не позбутися цього і як це зберегти, тому що це
їхнє життя. І вони ні перед ким не відповідають, що вони жили в нормальних
країнах і нікуди культурна спадщина не поділася. Єдина заковика: що було
викрадено під час війни і як це повернути. Ось це єдина проблема, що ще існує в
Європі. Всіх інших проблем не існує. І ще ...

Як приємно, коли я зайшла на Монмартрі в маленьку гостінічку, в якій
вісімнадцять доларів коштував нічліг зі сніданком, а не по
п'ятсот-шістсот-сімсот і так далі ... І ця маленька гостінічка на Монмартрі
була викуплена індусами, які зберегли маленьку цю гостінічку такою, як вона
була. І я йду по номерам і бачу ... Тут Гоген – НЕ Гоген, Ван Гог – НЕ Ван Гог,
тут Кандинський – НЕ Кандинський ... Картинка, затягнута целофанчіком ... 

І я
подумала що це постери і так далі ... А потім, коли ми розговорилися з
власниками, він сказав: «Ні! Це те, що тут було. Ці художники так любили цей
готель, що вони сюди приходили з дівчатками. І з дівчатками розраховувалися
картинками. А дівчаткам не було куди нести ці картинки, і вони залишали тут. І
так нам це перейшло від господаря. І він дуже просив нас нікуди не подіти. І ми
залишили кожну картинку в тому ж номері». Оце життя! Ось це добре! Бідні
художники Монмартра, які стали потім геніальними художниками для світу, висять
в маленькій гостінічкі, в яку ми маємо можливість приїхати і подивитися, і
усвідомити: життя прекрасне! Головне – нічого не переривати! Головне – відкрити
очі! Головне – озирнутися, і ми побачимо роботи в Україні. Не гірше Ван Гога,
не гірше Гогена, а набагато вище і набагато інтелігентніше, і набагато
яскравіше. Просто треба вміти бачити, що є навколо тебе. 

Коли приїжджають іноземці, то вони бачать красивих жінок, тому що вони відкрили
очі. Наші чоловіки закрили очі і вже давним-давно цього не помічають. Точно так
же, як не помічають, як багато красивого роблять наші художники і наші
талановиті люди.  Ми талановиті, ми гарні. Відкрийте очі, і ви все це побачите.
І життя ваше стане цікавіше і краще.

***

Розмова про копії – це взагалі окрема тема. Абсолютно окрема. Я трохи гублюся,
коли мені задають такі питання. Я іноді потрапляю до дуже відомих людей, і в
дуже дорогі будинки. І коли я бачу копії якихось робіт я просто гублюся. Якщо у
людини є кілька мільйонів на будівництво такого будинку, якщо по кілька сот
тисяч у нього кілька машин, і за кілька десятків тисяч у нього годинник на руці
... І висить жалюгідна підробка за тисячу доларів в будинку такого класу ... я
просто розумію, що щось не зовсім нормально в королівстві кривих дзеркал. 

І щось не зовсім зрозуміло мені з мізками тієї людини, яка витрачає стільки
грошей на все це і раптом привносить таку дешевку в свій будинок. Справа в
тому, що кожна пора відзначена талантом кого-небудь. Необхідно вміти розгледіти
цей талант ... 

Адже ти не відразу розумієш котрий годинник тобі треба купити, і яка машина
найкраща. Тобі її рекламують, тобі про неї розповідають, і ти її купуєш.
Порадься і візьми щось пристойне до себе в будинок. І це має бути не
обов'язково антикваріат. Візьми те, що тобі до душі. І те, що тобі розповість
про тебе краще чогось іншого.

Ви знаєте, у мене є дуже багато друзів, які зараз роблять роботи, потім їх
тиражують – роблять постери. Постери продаються по десять, по двадцять, по
тридцять, по сто доларів, і вони розлітаються направо і наліво. Напевно, це
реверанс і посмішка художника з приводу тих, хто нічого не розуміє. І, напевно,
це правильно. Тому що художнику все одно. Він свою роботу залишає собі або
колись дорого продасть. Або колись дорого продасть той, хто її купив перший. А
всі інші користуються ... 

Це, знаєш, як у когось є гарні груди, а хтось користується завжди підставними і
надутими. Хтось цілує красиві губи і пестить гарне волосся, а хтось завжди буде
задовольнятися підробкою і перукою. Кого що влаштовує. І у кого на що вистачає
грошей. 

Я так думаю, що в сучасному світі, де все переплелося і стало з голови на ноги,
це теж той черговий кульбіт, це те циркацтво і це той трюк нашого життя, який
говорить про те, що не все в порядку в королівстві. Чи не все в порядку з
дзеркалами в королівстві. Не так все відбивається і не те ми бачимо, що нам
здається. 

Знаєш, ми бачимо себе в дзеркалі і бачимо тільки себе, а насправді в дзеркалі
віддзеркалюємося не тільки ми, на кого ми дивимося, на своє відображення, але і
те, що стоїть біля нас за спиною. А ми на це в дзеркалі вже уваги не звертаємо.
Так ось, підробка завжди буде поганим фоном для тебе як в сьогоденні, так і в
відображенні дзеркальному.

Не беріть підробок! Не беріть постерів і не користуйтеся підробкою, тому що
вона знівелює все те дороге, що ви купили в своєму житті. Тому що на тлі однієї
підробки все інше натуральним виглядати не буде.

***

Ніколи масова культура культурою була. Завжди була певна культура відтворення
себе в певному колі. Тому що, аристократ ніколи не опускався до міської
культури, робочого міста або до культури села. Адже це теж була своя велика
культура. Як виховували дітей, як одягалися на селі, як свята справляли. Це
була одна культура, велика широка культура. 

Як жили аристократи, які у них були
правила поведінки, як виховувалися їхні діти, в яких інтер'єрах вони жили. Це
була інша культура. Як жили в містах, робочих селищах; як жило місто міщанське,
аристократичне і як жило пролетарське місто. Це все культура. Просто не треба
все змішувати. В один і той же час існують різні верстви суспільства та різні
культурні наповнення їх середовища. І по-іншому не буває. Не може хлопчик, який
отримав освіту в Лондоні і повернувся до Києва, і хлопчик з села, який ходив за
сім кілометрів до школи, а потім закінчив ПТУ, – мати одне і те ж розуміння, як
себе вести, в якому суспільстві бути, в який ресторан, ходити і який театр
відвідувати. І вже тим більше, яким мистецтвом йому цікавитися. 

Хоча,
найчастіше, саме хлопчик з простої сім'ї іноді може осягнути більше, тому що
бажання знань може переповнювати його настільки, що він рвоне як спринтер
вперед і не тільки наздожене, а й пережене. Це різні пласти, і тому не треба
нікого ні з ким порівнювати. Це все одно, що порівнювати як буде себе вести в
певній ситуації китаєць, німець і американець. В одній і тій же ситуації ці
люди поводитимуться по-різному. А вже тим більше по різному, якщо вони ще й з
різних соціальних шарів. Що б не говорили про те, як у нас був великий і
могутній радянський народ, і всі були рівні, це неправда. Хоча нас насильно
зрівнювали, але ніхто рівним ніколи не був. А тепер вже і поготів про рівність
говорити безглуздо.

Якщо говорити про масову субкультурі, то це найсумніше що є в нашому житті.
Хоча ми ще спостерігаємо за нею з усмішкою, але вона нас наздоганяє всюди. Вона
захльостує нас в громадських приміщеннях, вона захльостує нас в приватних
дорогих або закритих просторах.

***

\ii{14_12_2021.fb.fb_group.story_kiev_ua.3.natali_krivuca_chast_dva.cmt}
