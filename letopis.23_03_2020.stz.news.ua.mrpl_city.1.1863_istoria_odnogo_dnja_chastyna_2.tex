% vim: keymap=russian-jcukenwin
%%beginhead 
 
%%file 23_03_2020.stz.news.ua.mrpl_city.1.1863_istoria_odnogo_dnja_chastyna_2
%%parent 23_03_2020
 
%%url https://mrpl.city/blogs/view/1863-istoriya-odnogo-dnyakniga-znahidka-pro-mariupol-chastina-ii
 
%%author_id demidko_olga.mariupol,news.ua.mrpl_city
%%date 
 
%%tags 
%%title "1863: Історія одного дня" – книга-знахідка про Маріуполь (Частина II)
 
%%endhead 
 
\subsection{\enquote{1863: Історія одного дня} – книга-знахідка про Маріуполь (Частина II)}
\label{sec:23_03_2020.stz.news.ua.mrpl_city.1.1863_istoria_odnogo_dnja_chastyna_2}
 
\Purl{https://mrpl.city/blogs/view/1863-istoriya-odnogo-dnyakniga-znahidka-pro-mariupol-chastina-ii}
\ifcmt
 author_begin
   author_id demidko_olga.mariupol,news.ua.mrpl_city
 author_end
\fi

На початку березня я опублікувала блог, присвячений, на мою думку, одному з
найцікавіших видань Маріуполя. Пропоную продовжити ознайомлюватися з
книгою-знахідкою та її змістом. Сподіваюся, що під час карантину для багатьох
вона стане розрадою і допоможе провести час з користю. Продовжуватиму з місця
на якому зупинилася і нагадаю, що мова піде про головного героя історії
\emph{\textbf{Григорія Ільяшенка}}, який вирішив помститися за свого знайомого
\emph{\textbf{Поддубню}}. От, що саме сталося з останнім ви, дорогі читачі,
незабаром і дізнаєтеся. До речі, цікаво, що частина зборів від книги
передавалася бідним маріупольцям.

\emph{\textbf{Уривки з книги надаються мовою оригіналу:}}

\begin{quote}
\em
В 1861 году Поддубня так удачно расторговался, что выручил 5015 рублей. По
несчастной случайности в том же трактире, где отдыхал после удачной торговли
наш герой, оказался и Логафетов со своим отрядом. Там они предавались даровому
обжорству и пьянству. Отуманенный чрезмерной едой и излишне выпитым Логафетов
стал буйствовать и потребовал, чтобы музыканты, развлекавшие Поддубню,
немедленно стали играть перед ним. Как бы ни был пьян Поддубня, но он выступил
защитником своих прав, говоря, что музыкантам заплатил и, что он не отпустит их
\enquote{пока они не отыграют свого}. Произошло недоразумение, в результате которого
Поддубня оказался выброшенным с завязанными назад руками в какой-то полутемный
чулан, названный арестансткой камерой. Здесь Поддубня заснул крепким пьяным
сном, от которого очнулся часов через пять, когда какой-то неизвестный человек
стал выталкивать его на свободу. Ужас охватил узника, когда он, ощупав у себя
под жилетом, заметил отсутствие денег. Он сразу отрезвел и понял, что в конец
разорен, так как там под жилетом находилась выручка за весь распроданный товар;
Поддубня понял, что теперь не было ни товара, ни денег. Оставалась нищета. Он
обратися за помощью к местным властям, но это оказалось бесполезным. \enquote{Там
проклятое греческое царство} – объяснил он своим слушателям, – \enquote{Логафетов
богат, у него большая родня, знакомства, наш брат ничего не поделает!}... Писал
Поддубня жалобу и губернатору, что вызвало расследование особо командированого
чиновника, который исписал много бумаги, но ни к какому результату не пришел. В
общем же жалобы на Логафетова Поддубни и других лиц, вызвали отстранение его от
исполнения обязанностей члена греческого суда. От этого, конечно, не было легче
Поддубне: деньги его пропали. \enquote{По миру пустил, проклятый... грабитель}, – заключил
свой рассказ Поддубня... Когда Поддубня закончил, Ильяшенко вмешался в разговор.
Он предложил Поддубне \enquote{взяться} за его дело и пояснил, что раньше с успехом вел
перед начальством дела молокан. Действительно, бесспорными официальными данными
устанавливается, что Ильяшенко посещал колонии молокан, осматривал их как
ревизор-чиновник, выражал им иногда свое благоволение, похвалы и обещания
представительства перед высшим начальством в Петербурге, какого, конечно, он
никогда не исполнял и исполнять не мог. При этом достойно замечания, что
Ильяшенко в этом случае действовал \textbf{\enquote{безо всякой корыстной цели и даже без
особой побудительной причины}}.  
\end{quote}

\ii{23_03_2020.stz.news.ua.mrpl_city.1.1863_istoria_odnogo_dnja_chastyna_2.pic.1}

Думаю не всі розуміють про яких молокан йдеться, тому слід пояснити цей термін.
\textbf{Молокани}, початкова самоназва \enquote{Духовні християни} – одна з течій російського
духовного християнства, що виникла у Борисоглібському повіті Тамбовської
губернії центральної Росії наприкінці 1760-х років шляхом відходу парафіян РПЦ
та переходу з духоборів. Головною причиною виникнення молоканського руху став
протест проти підтримання Православною Церквою соціальних засад кріпацтва.
Назву \enquote{молокани} поширила саме православна церква. Молокани пили молоко під час
православного великого посту, коли прийом \enquote{скоромної} (жирної) їжі заборонено
православними переданнями, за що православні прозвали їх саме молоканами.

\begin{quote}
\em
\enquote{На предложение Ильяшенко взяться за его дело, Поддубня ответил отказом,
заметив: \enquote{с чем же вести дело, когда я остался гол, как сокол}! \enquote{Ну так
знайте}, возразил Ильяшенко, \enquote{я вам этого самого Логафетова в кандалах через
Бердянск отправлю в екатеринославскую тюрьму!}. До сих пор разговор происходил
сравнительно спокойно и на \enquote{вы}, но после слов, сказанных Ильяшенко, Поддубня
впал в патетический тон и перешел на \enquote{ты}: \enquote{будь благодетель, запри его в
тюрму, последнюю сорочку сниму, отдам!}... завопил Поддубня. \enquote{Ничего не надо,
только поставишь магарыч}, великодушничал Ильяшенко, после чего внезапно
познакомившиеся заключили друг друга в объятия, и стали пить водку из вновь
принесенной бутылки. Возле этого сосуда объединилась вся компания из четверых
человек, и дружно вела шопотом какой-то длинный разговор...}.
\end{quote}

Після цієї розмови всі крім головного потерпілого, вирішили їхати в місто
Маріуполь та карати Логафетова. Так, купець Мазин, міщанин Колосовський та
відставний кресляр Григорій Власов Ільяшенко, який і очолив цю важливу і
благородну справу, відправилися до Маріуполя. В книзі на декількох сторінках
розповідається про погоду 5 квітня 1863 ріку та деякі деталі їхніх зборів. Ці
моменти я пропускаю і повертаюся до місця, коли вони вже приїхали до Маріуполя.

\begin{quote}
\em \enquote{Около четырех часов пополудни Ильяшенко и его товарищи подъехали к
Мариуполю и остановились в предместьи города, на Марьинской стороне, в одной из
невзрачных изб; гостинниц тогда в Мариуполе не было и приезжающие
останавливались в частных квартирах. Того же дня вечером, когда уже стемнело,
Ильяшенко разыскал дом Логафетова и явился к нему. Как мы уже знаем, Логафетов
в то время не был властью: он был отстранен от должности члена греческого суда.
Тем не менее он жил припеваючи: был холост, богат, не особенно стар, ему было
около 50 лет, недугами не страдал, в городе имел богатых и сильных
родственников, среди которых он был свой и дорогой их сердцу человек. Словом,
жилось ему хорошо, беззаботно, спокойно; его служебные подвиги не тревожили,
ибо все расследования о его прошлых деяниях кончились совсем благополучно: ни о
каких судебных преследованиях не было и речи. Вот почему Логафетов свысока
отнесся к Ильяшенко, когда тот стал ему объяснять, что ему грозят неприятности
по жалобам Поддубни и, когда Ильяшенко предложил ему свое содействие для
улаживания этого дела. В конце концов Логафетов грубо выпроводил Ильяшенко и,
закрывая за ним дверь, пропустил мимо ушей, обращенное к нему восклицание: \enquote{ты
меня попомнишь!}... Понятно, что этим словам Логафетом не придал никакого
значения}.
\end{quote}

Далі автор книги, \textbf{Петро Валерійович Каменський}, перед тим як продовжити
розповідь, наголошує, що вся ця історія є реальною. Всі відомості П. Каменський
отримав від маріупольських старожилів, які були не тільки безпосередніми
свідками, але й навіть частково учасниками описуваних подій. У наступному блозі
на вас чекає кульмінаційний момент цієї неймовірної історії. Ви зможете
дізнатися як саме Григорій Ільяшенко помстився за свого доброго знайомого і чим
все завершилося... Дочекайтеся і, найголовніше, бережіть своє здоров'я!

Першу частину читайте \href{https://archive.org/details/04_03_2020.olga_demidko.mrpl_city.knyga_istoria_odnogo_dnja}{тут}%
\footnote{\enquote{Iсторiя одного дня} – книга-знахiдка про Марiуполь, Ольга Демідко, mrpl.city, 04.03.2020, \par%
\url{https://mrpl.city/blogs/view/istoriya-odnogo-dnyakniga-znahidka-pro-mariupol}, \par%
Internet Archive: \url{https://archive.org/details/04_03_2020.olga_demidko.mrpl_city.knyga_istoria_odnogo_dnja}
}
