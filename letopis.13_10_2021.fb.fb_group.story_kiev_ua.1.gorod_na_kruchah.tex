% vim: keymap=russian-jcukenwin
%%beginhead 
 
%%file 13_10_2021.fb.fb_group.story_kiev_ua.1.gorod_na_kruchah
%%parent 13_10_2021
 
%%url https://www.facebook.com/groups/story.kiev.ua/posts/1774155489447911
 
%%author_id fb_group.story_kiev_ua,opolskaja_nina
%%date 
 
%%tags gorod,kiev,ukraina
%%title Кручи днепровские, Город на кручах, Город сверкающий, Город могучий!
 
%%endhead 
 
\subsection{Кручи днепровские, Город на кручах, Город сверкающий, Город могучий!}
\label{sec:13_10_2021.fb.fb_group.story_kiev_ua.1.gorod_na_kruchah}
 
\Purl{https://www.facebook.com/groups/story.kiev.ua/posts/1774155489447911}
\ifcmt
 author_begin
   author_id fb_group.story_kiev_ua,opolskaja_nina
 author_end
\fi

Кручи днепровские, Город на кручах, Город сверкающий, Город могучий!
Город-твердыня немеркнущей славы, древний и юный, простой величавый!  Любуюсь и
горжусь гористой местностью нашего Города и и её красотой в любое время года.
Белоснежные горы зимой, изумрудно-зеленые весной и летом, а сейчас
- пурпурно-желтые с золотинкой! Они создают неповторимый образ Города. 

\ifcmt
  tab_begin cols=2

     pic https://scontent-frx5-1.xx.fbcdn.net/v/t39.30808-6/245516903_1789508331256780_233908828170647718_n.jpg?_nc_cat=111&ccb=1-5&_nc_sid=b9115d&_nc_ohc=hICP9jiX3a0AX_lsnub&_nc_ht=scontent-frx5-1.xx&oh=ce3a728a2076bb741a4bda01523c1981&oe=616D07DA

     pic https://scontent-frx5-1.xx.fbcdn.net/v/t39.30808-6/245183501_1789508394590107_2080308451621800783_n.jpg?_nc_cat=100&ccb=1-5&_nc_sid=b9115d&_nc_ohc=eHzvKWqe3IQAX__wH2G&_nc_ht=scontent-frx5-1.xx&oh=1b7b6f6dc120e42eb4b47ddf68fce73c&oe=616C17DC

  tab_end
\fi

Живописные возвышенности Киева очень красивы сами по себе и открывают
великолепные панорамы столицы. Попытаюсь перечислить некоторые из них. 

Самый возвышенный район Города - Печерский. Здесь, в районе аристократических
Липок, можно увидеть Левашовскую гору. Её название встречается только в
некоторых источниках и применимо к возвышенности рядом с улицей Шелковичной.
Название горы связано с генерал-губернатором В. Левашовым, которому
принадлежала эта земля. Её высота - порядка 196 м над уровнем моря. Стоит
вспомнить, что Центральный ботанический сад им. Гришко расположен тоже на горе.
Это Бусовица или Бусова гора. Между улицами А. Барбюса, Госпитальной и
Лабораторным переулком возвышается Черепанова гора. Здесь в начале позапрошлого
века находилась усадьба киевского гражданского губернатора П. С. Черепанова. 

\ifcmt
  tab_begin cols=3

		 pic https://scontent-frt3-1.xx.fbcdn.net/v/t39.30808-6/244718085_1789512287923051_5555589644943634318_n.jpg?_nc_cat=108&ccb=1-5&_nc_sid=b9115d&_nc_ohc=moTjj77gwwUAX_XSXea&tn=lCYVFeHcTIAFcAzi&_nc_ht=scontent-frt3-1.xx&oh=045dc54efaa66e02305f2f843d12f280&oe=616C037F

     pic https://scontent-frt3-2.xx.fbcdn.net/v/t39.30808-6/245418500_1789518784589068_647555880843001070_n.jpg?_nc_cat=101&ccb=1-5&_nc_sid=b9115d&_nc_ohc=CoJX8iOqtBUAX8ibOHG&_nc_ht=scontent-frt3-2.xx&oh=ca63441bc7e18e0823ee2c4a21e94bbd&oe=616BCD74

     pic https://scontent-frt3-1.xx.fbcdn.net/v/t39.30808-6/245116187_1789525457921734_401662921290274842_n.jpg?_nc_cat=102&ccb=1-5&_nc_sid=b9115d&_nc_ohc=UXCO4m_5DukAX-a-L0F&_nc_ht=scontent-frt3-1.xx&oh=11ecd676de466e5edcea6cd9dee3fbb8&oe=616CB76B

  tab_end
\fi

Старый Город находится тоже на возвышенности. В
княжеские времена его так и называли - Гора, или Верхний град. Здесь самой
известной является Старокиевская гора - колыбель Киева, а также Михайловская
гора, более известная как Владимирская горка. Это-исторический центр, их высота
187 метров. К древнейшему центру также относится гора Детинец, ориентиром на
который есть Национальный исторический музей. А также Андреевская гора - около 167
метров. Гора, к которой ведёт подъем от загадочного Замка Рычарда именуется
Уздыхальница. Есть версия, что на горе осуществляли казни... А может быть из-за
крутого подъёма её величают так. Пока заберешься - вздохнешь не один
раз... Хоревица, Флоровская, Замковая гора, Киселевка - всё это названия одной
возвышенности. Думаю, проще запомнить - Флоровская гора, у подножия которой
находится Флоровский  женский монастырь. Рядом находится Щекавица или Олегова
гора. Одна из улиц на горе называется Олеговской. Возможно, здесь на древнем
кладбище похоронен князь Олег, прозванный Вещим... Кто знает.... На северо-западе
от Щекавицы - Юрковица или Юрковец.. Далее - Кудрявская гора, между улицами
Артёма, Тургеневской и Гоголевской. Далее на запад-Кирилловская гора. Вот как
много киевских гор! Есть ещё Ветряные гора, Батыева, Багринова, Байкова. 

А как не вспомнить о Лысой горе?!! Которая возвышается над Столичным шоссе в
районе Выдубичей. Её высота - порядка 156 метров. Здесь находился Лысогорский
форт-последний объект Киевской крепости, созданной в 1870 годы для защиты
подступа к Дарницкому железнодорожному мосту. Лысая гора оглашена региональным
ландшафтный парком. Некоторые её растения занесены в красную книгу. Особый
интерес вызывает Лысая гора у поклонников творчества Михаила Афанасьевича
Булгакова. Исследователи его творчества склоняются к версии, что именно эту
гору Киева упоминает писатель в произведении "Мастер и Маргарита". Именно
киевские холмы, возвышенности создают самобытный образ Города, возвышаясь над
ним и открывая самые живописные панорамы. 

Величав и неповторим Киев любимый! Славный Город, поклон тебе! Какое счастье
жить и наслаждаться твоими красотами, особенно с вершин твоих гор! Поднимешься
и радуешься! Навстречу ветрам и солнцу!  Навстречу дождю и снегопаду! И сердце
учащенно бьётся в унисон с тобой! И радость моя сливается с твоей, ибо знаю -
ты счастлив вместе со всеми нами, такими влюбленными в тебя! И эта обоюдная
любовь бесконечна и придает силы жить и хранить твою историю и современность!

\ii{13_10_2021.fb.fb_group.story_kiev_ua.1.gorod_na_kruchah.cmt}
