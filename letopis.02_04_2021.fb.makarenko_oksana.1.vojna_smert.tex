% vim: keymap=russian-jcukenwin
%%beginhead 
 
%%file 02_04_2021.fb.makarenko_oksana.1.vojna_smert
%%parent 02_04_2021
 
%%url https://www.facebook.com/permalink.php?story_fbid=2890844391200864&id=100008259933050
 
%%author 
%%author_id 
%%author_url 
 
%%tags 
%%title 
 
%%endhead 

\subsection{Война хрипит осипшим волчьим лаем, и мир скрипит, как старая кровать}
\label{sec:02_04_2021.fb.makarenko_oksana.1.vojna_smert}
\Purl{https://www.facebook.com/permalink.php?story_fbid=2890844391200864&id=100008259933050}

Специально для Олька Громова и Мила Машнова:
*
Война хрипит осипшим волчьим лаем, и мир скрипит, как старая кровать. Когда полковник писем не читает, полковнику бессмысленно писать. Реальность так беспомощно зависла, что впору гнать хранителей взашей: во фразе на заборе больше смысла, чем в длинном описании вещей. И ни к чему эпистолярной лире бренчать о каждом пережитом дне. Жизнь ничего не слышала о мире. Смерть ничего не знает о войне. У смерти нет ни возраста, ни пола – лишь отрывной, как ноги, календарь. И бесполезно жечь сердца глаголом, когда вокруг – одна сплошная гарь и прах летит густым июньским пухом, и цвет вселенной – чёрный, как графит. И не о чем писать для сильных духом, когда у слабых не за что спросить, ведь изрекать банальность – просто глупо, что всей войной доказано не раз. Полковник понимает силу буквы, когда она оформлена в приказ.
А над землёй летят сухие листья, и зависают дроны в вышине, и почтальон бойцам разносит письма, в которых нет ни слова о войне, в которых нескончаемой тирадой описаны неважные дела, в которых вместо горькой чёрной правды – лишь запятые, точки и слова. Слова пусты, как глупый треск сороки. А где-то меж бескрайних облаков, наверно, Богу тоже одиноко – не сотню лет, а тысячи веков. И нет давно ни страха, ни тревоги, а есть лишь предсказуемый финал. Но если б знал полковник адрес Бога…
…то ничего б ему не написал.
(с)Артём Сенчило
фото автора

\ifcmt
  pic https://scontent-bos3-1.xx.fbcdn.net/v/t1.6435-9/168036956_2890844371200866_6103457113109802948_n.jpg?_nc_cat=107&ccb=1-3&_nc_sid=8bfeb9&_nc_ohc=Z4VZu32WtsUAX9tlWTw&_nc_ht=scontent-bos3-1.xx&oh=ff2c8ff450f79e1fd6ad491319782e26&oe=6092D3FE
\fi

