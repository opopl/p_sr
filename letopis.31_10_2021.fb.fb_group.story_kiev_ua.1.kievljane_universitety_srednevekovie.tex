% vim: keymap=russian-jcukenwin
%%beginhead 
 
%%file 31_10_2021.fb.fb_group.story_kiev_ua.1.kievljane_universitety_srednevekovie
%%parent 31_10_2021
 
%%url https://www.facebook.com/groups/story.kiev.ua/posts/1787392494790877/
 
%%author_id fb_group.story_kiev_ua
%%date 
 
%%tags istoria,kiev,kievljane,srednevekovie
%%title Про київських студентів середньовіччя
 
%%endhead 
 
\subsection{Про київських студентів середньовіччя}
\label{sec:31_10_2021.fb.fb_group.story_kiev_ua.1.kievljane_universitety_srednevekovie}
 
\Purl{https://www.facebook.com/groups/story.kiev.ua/posts/1787392494790877/}
\ifcmt
 author_begin
   author_id fb_group.story_kiev_ua
 author_end
\fi

\begin{cmtfront}

«Більшість тепер виставляє напоказ свої праці, святотче,
Дбаючи щонайпильніш про честолюбство та зиск.
Я ж мої книги у світ випускаю з єдиним бажанням:
Хай буде користь від них роду людському в житті». 

Юрій Дрогобич (Котермак). 1483 р.
\end{cmtfront}

Про київських студентів середньовіччя.

Як кияни XІV-XVIII ст. пов’язані з європейськими університетами та до чого тут
Галілей, Кант і Копернік?

До початку XVI cт. в Європі налічувалось біля 80 університетів. І майже в
кожному з них навчались наші земляки. 

Найперші університети Болоньї, Монпельє, Парижу і Оксфорду виникли в дусі
епохи, сповідуючи і встановлюючи на століття вперед певний образ свободи,
бунтівного характеру, чогось, що виходило за межі усталених норм і правил, що
має змінити Світ. Всього чотири університети, але саме від них беруть початок
всі інші.

\ifcmt
  ig https://scontent-frt3-1.xx.fbcdn.net/v/t1.6435-9/250918992_4469368139836766_1906212327854622052_n.jpg?_nc_cat=104&ccb=1-5&_nc_sid=825194&_nc_ohc=9eF0o0gm8bUAX8aKTx1&_nc_ht=scontent-frt3-1.xx&oh=4a6d367c3f51e9dc81da15a60cd3a4ef&oe=61A7670C
  @width 0.4
  %@wrap \parpic[r]
  @wrap \InsertBoxR{0}
\fi

На навчання в них відправлялись молоді і не дуже люди з усіх усюд, не лише
шляхетського походження і багатої вдачі, але і прості, необмежені кордонами
світосприйняття, іноді небагаті, а коли і просто авантюристи за духом. 

У найдавнішому відомому на сьогодні документі, в якому згадані кияни-студенти,
зазначено: 

Magister Petrus Cordowan et Socius eius de Ruthenia - «Магістр Петро Кордован і
його товариш з Рутенії». Цей лист написаний 6 квітня 1353 року з Парижа (з
Сорбонни) до Рима. У 1389 році у списках Паризького університету значиться
«Теобало Гнібербо з Києва». Пізніше зустрічаємо ім'я «Германа Вілевича,
ліценціата мов і бакалавра рутенської нації з Києва», та у 1369 році Івана з
Рутенії (Johases de Ruthenia), який 1391 року складає докторські іспити. 

До формування націй у нашому сьогоднішньому розумінні лишалось ще багато, але
облікові книги університетів часто мали запитання до нових студентів про їх
походження, а тому кожен підписувався як веліла його традиція: рутенами,
русинами, роксоланами, сарматами…. А з початку XVII ст. все частіше маємо
підписи наших студентів - як українців.

Перший такий відомий запис з університету Падуї (сучасна Італія) засвідчує
цілком конкретний і неспростовний доказ самоідентифікації українців XVII ст.,
які 1648 року знову здобули в бою свою державу, а значить і нові змісти
самоназв. 

Текст історичного запису 1660 року з Падуї: "Joz.Daniel Dzik, wikaryuz
katedralny lucki «natione Vkrainensis de districtu Kiiouiensi», тобто: «Йозеф
Данило Дзик, катедральний вікарій Луцький, національність — українець,
Київський дистрикт».

Як мені вдалося відшукати в архівах, Дзики (Микола) вказані у документі
"Нарахування поселень і будинків воєводства Брацлавського, для стягнення
податку "podymnogo i mlynowego" в 1664 р.". У Липинського ми знаходимо ще одну
згадку, інформацію про старого Миколу Дзика, який йде з військом по
Брацлавщині, а також синів його: "Петро служить у Війську Запорізькому, як
полковник хвастівський", а "Мелетій... ректор Могилянської академії". Можемо
зробити висновки, що походили з Брацлавщини, служили, хто у Фастові, хто в
Києві, хто у Луцьку. А Данило досить однозначно підписався в списках
Падуанського університету, що він українець з Київщини (Kiiouiensi). 

Традиційний вступний внесок в університети оплатити іноді не могли, а тому
записувались у матрикули як "прості люди". Жили у бурсах-гуртожитках, коли і по
6 на ліжку, часто не мали що їсти... але хіба спинить це тих, час кого настав і
хто покликаний змінити Світ?

Щодо всіх цих цікавих особливостей студентського побуту тих років є чудова
белетризована біографія, повість, авторства Я.Ісаєвича - «Юрій Дрогобич»
(1966). Вона передає дух епохи. Не менш корисною буде чудова книга Д.Вирського
про ще одного видатного українського просвітителя, "нашого Демосфена", -
"Станіслав Оріховський-Роксолан" (2013). А найбільш глибокі дослідження шляху
українських студентів в Європі зробив геніальний історик Г.Нудьга (на
матеріалах якого побудована ця стаття).

.

Мова, письмо, книжка, література в процесі духовного збагачення народу в
XVI—XVII століттях відігравали важливу роль. Студенти, які навчались в
західноєвропейських університетах, активно бралися за невідкладну справу
складання граматик церковнослов'янською, а далі й українською мовами. Створенню
граматик, що вперше унормовували давню українську літературну мову, виробленню
канонів віршування в українській літературі прислужилися найвидатніші вчені і
письменники, що здобували освіту за кордоном. 

Повернувшись на батьківщину, вони організовували школи, писали підручники. 

З помітних українців, вихованців Краківського університету, які залишили слід в
українській культурі, варто назвати Лукаша з Нового міста під Самбором; він
здобув ступінь магістра, опублікував перший у нас підручник з епістолографії
(1522 р.). Випускник двох європейських університетів (в тч легендарної
Сорбонни) Іван Ужевич увійшов в історію як мовознавець і письменник епохи
Ренесансу, першим описав одну із східнослов’янських мов – лінґвістично
опрацював українську/білоруську (linqua Ruthena). Повна назва твору
„Слов’янська граматика, Іваном Ужевичем слов’янином, студентом богослов’я
славної Паризької академії, написана у Парижі року від народження Сина Божого
1643”. Він познайомив західноєвропейських філологів з деякими особливостями
міжслов’янської церковнослов’янської мови, яку називав «священною», але прагнув
на основі української мови створити вступ до вивчення граматичних систем інших
слов’янських мов.

Були популярними «Адельфотес...» — граматика грецької і церковнослов'янської
мови (Львів, 1591), граматика «Лексикон» Л.Зизанія (Вільно, 1596). В 1627 році
в Києві вийшов перший словник української мови Памво Беринди, який містив
близько 7 тисяч слів: загальних та власних назв переважно тогочасної
церковнослов'янської мови з перекладом та тлумаченням їх українською
літературною мовою початку 17 століття. Праця складається з двох частин:
"Лексіконь" – церковнослов'янсько-український словник; та "…Имена свойственная"
– зібрання тлумачень топонімів й антропонімів, а також загальних назв
неслов'янського походження.

А особливої популярності набула граматика М.Смотрицького (1619). Після букваря
граматика була універсальним підручником і основним джерелом літературних знань
і вела до розуміння таких наук, як діалектика, риторика, музика, арифметика,
геометрія, астрономія... а найпаче — до уміння складати літературні твори. 

Він став знаковою фігурою не лише для українців, а і для білорусів, балканських
слов'ян та навіть румунів. М.Смотрицький, народжений на Поділлі близько 1578
р., початкову освіту отримав у Острозькій академії, яка тоді була
безальтернативним освітнім закладом всього регіону. Тут викладав його батько,
Герасим Смотрицький, а також Кирило Лукаріс, випускник Падуанського
університету і один з найосвіченіших організаторів навчання, запрошений на
українські землі К-В.Острозьким (сином видатного переможця московитів 1514 року
- Костянтином Острозьким, що похований в Києво-Печерській лаврі). 

М.Смотрицький вчився у Віленській академії, де засвоїв філософські науку, через
кілька років продовжив у Лейпцігському, Нюрнберзькому, Віттенберзькому
університетах. І став одним з найосвіченіших людей України, найвизначнішим
українським філологом. Його граматика сприйнята сучасниками як основний
підручник церковнослов'янської мови у східних слов'ян, і на думку І. Франка,
була «вихідною точкою граматичних студій» не лише для вироблення норм
української мови, а і для науковців та письменників різних країн XVIII
сторіччя. А за висловом М.Ломоносова: "Граматика Смотрицького стала вратами
учености", тому він поклав її в основу граматики московського діалекту 1755
року, який відомий тепер як "російська мова".

Як відомо, і сам Ломоносов не став би людиною, яка провела революційні зміни у
Петербурзі і російській науці, якщо б не українці і їх західна освіта. Саме
завдяки ним він відбувся. І це не перебільшення. Шанс на освіту отримав лише і
тільки завдяки тому, що колишній ректор Києво-Могилянської академії
Г.Одорський, його друг і палкий прибічник Мазепи І.Рогачевський, засланий сюди
після полтавської поразки 1709 року, а також архієпископ Варнава (Василь
Волостковський) заснували на далекій півночі школу. Саме колишній випускник
київської могилянки Варнава у своєму будинку в Холмогорах навчав дітей і
Ломоносова серед них. Він відкрив останньому граматику Смотрицького та надав
квиток у щасливе майбутнє науки і знань.

Щоб докладно не переповідати всю його біографію і не наводити десятки прізвищ і
ситуацій, де київські випускники допомогли як Ломоносову, так і загалом всій
російській науці того часу, процитую краще відому професорку Інституту історії
Санкт-Петербурзького державного університету Тетяну Таїрову-Яковлєву: 

"...влияние украинского образования и украинского духовенства в России
становится невероятным, практически весь преподавательский состав всех высших
учебных заведений России формировался из профессоров Киево-Могилянской академии
или ее выпускников, кроме того, все посты крупнейшие епископские и другие
церковные в Российской Империи в ту эпоху занимали исключительно выходцы с
Украины. Таким образом, влияние украинской культуры и традиции, а также
просвещенного духовенства очень сильно проникло в Российскую Империю.". 

Причини ще і в тому, що як пояснив на одній з лекцій інший російський історик,
д.і.н. П.Уваров, Московія тоді: "...очень далека от европейской реальности... а
западноевропейские традиции 13-14-го ст. приходят к ней в 18-19 ст.". 

І, як ми розуміємо, приходять саме з українцями. 

А на наші землі, задовго до того, приходили із заснуванням спочатку Острозької,
а потім Київської академій, де фундаторами і викладачами були саме випускники
європейських університетів.

Як пише видатний російський дослідник І.Голенищев-Кутузов, польська література
здійснила величезний і позитивний вплив на розвиток української культури, яка в
свою чергу вплинула на московську. Проте, ми бачимо і інших процес: проникнення
українсько-білоруських елементів в польську мову, поезію, живопис. Відомо, що
Ягеллони часто і з бажанням говорили такою мовою. Сігізмунд Старий писав князю
Радзівілу українською мовою. Листи Сігізмунда-Августа переповнені українських і
білоруських слів.

.

Російська наука багато в чому завдячує двоим видатним українцям: Феофану
Прокоповичу і Кирилу Розумовському. На деяких детялях їх біографії коротко
зупинюсь, бо масштаб особистостей того вимагає.

Перший став однією з найосвіченіших постатей свого часу. Видатний український
гуманіст і просвітитель XVIII століття підносив розум і творчий дух людини:
"знаменитіша від самого неба, хоч і з тієї ж, що і все інше, категорії
складається, її поетичний злет покликаний служити моральному удосконаленню
людства". 

Здобував освіту в італійському Римі, а також в німецькому Галле, став видатним
ректором Київської академії, а пізніше ще і духовним наставником Петра І,
автором ідеї перетворення Московського царства в Російську Імперію в 1721 році
та покровителем М.Ломоносова. 

Про роки навчання Феофана Прокоповича в Італії написано чимало. Зокрема,
німецький учений Роберт Штуперіх у праці «Феофан Прокопович у Римі» (1931)
першим повідомив, що в матрикулах колегії Св.Атанасія Прокопович записаний під
іменем Самійла Церейського і що це, можливо, його справжнє ім'я і прізвище, яке
він змінив при переході в унію, щоб мати змогу вчитися далі. Про навчання
майбутнього українського мислителя в Римі писав і Франко Вентурі в монографії
«Феофан Прокопович» (1953) та інші автори.

До молодого Кирила Розумовского Ломоносов потрапляє вже професором хімії. В
очолювану першим, як президентом, Російську Академію Наук. При чому, українець
- перший слов'янин, який очолив РАН (після чотирьох німців, що були її першими
керівниками). Заслугою тому західна освіта Кирила в університетах Кенігсберга,
Берліна і Геттінгена. 

А для нас він залишився в історії ще і останнім Гетьманом Війська Запорізького. 

Діти ж частково продовжили справу батька. Страсбурзький університет закінчили
сини Лев і Андрій Розумовські, останній був другом Людвіґа ван Бетховена. В
Лейденському університеті вивчав у 1781 році «натуральну історію» Григорій
Кирилович, який став відомим біологом та останні свої дні прожив у Моравії, де
і лишилась його родина. За лінією Кирило-Григорій-Камілло-Андреас-Андреас мол.-
ми виходимо на нині живого нащадка славетної династії, історика за освітою
Gregor Razumovsky, який мешкає у австрійській столиці, вивчає історію України
та часто буває на землях предків.

.

Варто окремо відзначити, що Козацтво відіграло неабияку роль у консолідації
української культури, зокрема у наданні матеріальної допомоги Київському і
Львівському братствам та їхнім школам, у постійній підтримці
Києво-Могилянського колегіуму, а потім академії. Освічену частину козацтва
складали вихованці львівської, острозької, київської та інших братських шкіл, а
потім Києво-Могилянського колегіуму і академії. Колишні студенти ставали не
лише військовими писарями, а й відважними козаками, сотниками, полковниками і
гетьманами, входили до складу дипломатичних місій як секретарі та перекладачі
насамперед з латинської і польської мов, якими вільно володіли. Але менше ми
знаємо про козаків, які вчились за кордоном, бо нам нав'язали образ
бандитів-гуляйполе. 

Козацький стан - це не збіговисько пройдисвітів, а типове європейське
лицарство, професійні воїни, рейтарі, шевальє, кабальєро. Відповідно і
ставлення до науки у них було цілком свідоме і відповідальне. Адже часто не
лише розвідка неможлива без знань різних мов, але і дипломатія, розвиток
військової справи тощо неможливі без якісної і сучасної освіти. 

Знаменитий діяч доби Хмельниччини Станіслав Морозовицький, якого козаки
називали Морозенко, закінчив кілька європейських університетів (відомо про
Краківський і Падуанський, а деякі допускають навчання у  Амстердамському і
Лейденському). Опинився при дворі короля Владислава IV, міг би податися в
елітні крилаті гусари, що були окрасою польського війська, або в артилеристи,
якими тоді пишалася будь-яка армія. Він же подався в козаки. 

У Лейденському університеті в Нідерландах навчався сподвижник Б.Хмельницького
полковник Максим Кривонос. В.Стефанович після європейського навчання став
полковим суддею у Лубнах. Степан Лукомський, після спроб навчання у Польщі, -
козацький старшина у Прилуках. Він уславився ще і історичною хронікою з деякими
цікавими фактами про Україну. 1616 року на студії в Падуї (Італія) записався
ієромонах Єзекіїл Курцевич-Булига, сподвижник гетьмана П.Сагайдачного, згодом
настоятель Трахтемирівського монастиря.

В Геттінгенському університеті навчались діти відомих українських родів,
переважно Милорадовичі, Данилевські, Базилевські, Огієвські та інші. З
вихованців Кенігсберзького університету в історію вітчизняної науки увійшов
Іван Хмельницький (нащадок Б.Хмельницького), який опублікував кілька
філософських праць. Хмельницький записався на навчання в серпні 1760, а через
два роки вже надрукував дисертацію «Міркування про підстави філософії» з
присвятою згаданому вище Кирилові Розумовському. Він же видав «Міркування про
рабство за природними законами і всенародним правом». В Англії (Оксфордський,
Кембріджський університети), де існували дещо відміні від континентальних
правила,  навчались  переважно представники аристократії, заможних, знатних
родин та козацької верхівки.

Таких студентів багато, можна прослідкувати історію аж до батьків-засновників
козацтва. Одними з таких деякі вважають Глинських-Мамаїв. Відомо, що Михайло
Глинський отримавши виховання при дворі німецького імператора, навчався в
Італії та, можливо, Іспанії. А Остафій Ружинський, з польсько-українського роду
князів, ще юнаком чимало подорожував країнами Західної Європи, навчався і
служив у Франції та Німеччині. Маємо всі підстави припускати, що найбільше він
цікавився теорією та практикою військового мистецтва і, зокрема,
організаційними засадами формування різних армій світу. Повернувшись в Україну,
брався до реформи козацького війська. Набирались знань у далеких землях Італії
і численні представники княжої родини Острозьких, некоронованих королів
України-Русі. Деякий час вищу освіту українські юнаки могли здобути тільки за
кордоном, тому такі гучні імена і довгі списки наших студентів невипадкові. 

.

Після повернення в Україну, служили вихованці університетів не тільки в
державних, судових, гетьманських, полкових і сотенних канцеляріях, а й у
середовищі простих козаків на Запорозькій Січі. Часто про нікому невідомих
освічених українців ми дізнаємось лише з козачих реєстрів XVI століття, де
зустрічаються записи: «Денис бакалавр», або «Матіас фельдшер», чи «добрий дяк
бакалавр». Взагалі, серед козацької старшини від гетьманської канцелярії аж до
сотенної вели справи ті, хто мав якщо не вищу, то середню освіту. Таким
високим, як на той час, відзначався і освітній ценз старшин Богдана
Хмельницького.

Деякі з випускників європейських університетів своїми руками творили важливі
сторінки історії України.

Наприклад, Юрій Немирич, призначений генеральним писарем після смерті
Б.Хмельницького, коли булаву отримав І.Виговський, став розробником "Маніфесту
до володарів Європи" та "Гадяцької угоди" 1658 року. Масштаб і наслідки цієї
угоди, як на мене, могли б мати колосальне значення. Це той випадок, коли
кажуть: "історія могла піти зовсім інакше".

Після новин про укладення Гадяцької угоди, московити розпочали черговий раунд
війни проти України. І хоча Військо Запорізьке розбило завойовників під
Конотопом в 1659 році, в одній із найлегендарніших битв нашого народу, але
другий фронт - дипломатичний - був програний (і не з його вини). 

За різними даними, які наводять Д.Наливайко і С.Плохій, Юрій Немирич міг
навчатись у "Лейденському, Паризькому й Оксфордському університетах", або
"Лейдені, Базелі і, за деякими відомостями, Оксфорді та Кембріджі".

Стапан-Адам Мазепа також був одним з прихильників "Гадяцької угоди", а його
син, видатний Гетьман України, І.Мазепа тим часом навчався у Ягелонському
університеті, у 1657 році вивчав артилерійську справу в голландському місті
Девентер (див. «Історія армії Нідерландських Штатів»). Можливо, також навчався
інкогніто в одному з французьких університетів. В Європі мав славу одного з
найбільш освічених людей своєї доби, бував в Німеччині і Італії. Перебування за
кордоном дало Мазепі не лише зовнішній лиск європейської культури й знання
чужих мов, але й досвід європейського політичного та культурного життя і — що
найважливіше — почуття тої європейськости, що назавжди уберегло його, людину
європейського Сходу, від впливів східньої московськости.

У середині XVIII століття у німецьких університетах здобували знання сини
полкового обозного Ханенка. Козацький старшина Микола Ханенко через свого сина
Василя щороку поповнював бібліотеку творами з різних галузей знань, не тільки
читав сам, а давав скористатися іншим, створивши одну з кращих бібліотек
регіону. А пізніше ця родина ввійде в історію як одні з найбільших
колекціонерів і поціновувачів українських старожитностей, меценатів, людей, які
вкладали великі кошти у збереження нашого історичного спадку.

Тема надзвичайно об'ємна, наших студентів в Європі лише за кілька згаданих
століть відшукали близько 5 тис, про освіту кожної з давніх козацьких родин
можна написати окрему статтю-розвідку, а тому я тут спробую лише окреслити
питання та вказати на те, що усталені сьогодні в суспільстві погляди на
козацький стан, на їх освіту і підходи до державотворення, не відповідають тим
реальним історичним подіям, які мали місце, тому положенню речей, в якому
українці зовсім не пасли задніх та аж ніяк не знаходились на узбіччі світової
історії, науки, розвитку і прогресу.

.

Більше того, як і вже показав на деяких прикладах вище, українці навчались
разом з визначними постатями світової історії, навчались у них, навчали їх,
іноді дружили.

Наприклад, вчились у відомого Галілео Галілея, який в 1592-1610 р. викладав в
Падуанському університеті, читав курси космографії, математики, геометрії,
астрономії, механіки, військової архітектури (фортифікації). З-поміж тих, хто
слухав лекції Галілея в Падуї, Г.Нудьга називає такі імена, як Фелікс
Кобринський, Яків Остроріг, Станіслав Держеняк, Василь Копеч, Іван Тичина,
Лукаш Канон, Іван Орельський, Данило Коршак, брати Острозькі. Ходили студенти з
України й на приватні лекції до Галілея додому. Це засвідчується документами,
які збереглися від тих часів. Старанним слухачем, а згодом і близьким знайомим
Галілея був Криштоф Збаразький, який разом з братом Юрієм приїхав на студії до
Падуанського університету. Вони належали до старожитного православного роду,
який виводив свою генеалогію від княжо-київських часів. У 1592 році Криштоф
слухав лекції Галілея. А 1601 року брати їдуть до Нідерландів, там їм пощастило
послухати лекції Юста Ліпсія, учня Еразма Роттердамського. Повернувшись 1602
року знову до Падуї, вони в 1604—1605 роках ще ближче сходяться з Галілеєм,
відвідують його лекції, а Криштоф навіть деякий час живе в нього. В паперах
Галілея збереглися тогочасні записи про перебування К.Збаразького в його домі.
Усього лекції Галілея в Падуї слухали не менше 52-х українських студентів, які
були першими, хто приніс ідеї геліоцентризму в Україну.

Ідеї Галілея, оповіді про його відкриття, використані в курсі натурфілософії,
прочитаному згаданим вище Ф.Прокоповичем у Києві. Були відомі киянам й імена
Паоло Фоскаріні, італійського послідовника Коперника, та знаного у той час
ботаніка й біолога — Улісса Альдровонді. Оскільки в XV — XVI I ст. італійці
вважалися кращими астрологами і лікарями у всій Європі, то книжок, дотичних до
здоров'я людини, написаних ними, в бібліотеці як Київської академії, так і її
професорів і вихованців, було особливо багато.

Заснований 1544 року Кенігсберзький університет, уславився, крім іншого,
викладанням Іммануїла Канта, який читав лекції з філософії, логіки, математики,
механіки, фізики, географії, антропології й історії. Цінні знання у нього
отримали кілька десятків студентів з України. Відомо про наступних: Андрій і
Михайло Базилевські з Полтавщини, Іван Максимович з Миргорода, Федір
Туманський, майбутній історик України, Іван Кулябка, Іван Данилевський з Києва,
Михайло Щербак з Пирятина, Григорій Милорадович, Олександр Протченко та інші.
Вони, напевно, принесли в Україну ідеї великого філософа, зокрема поширену тоді
у Європі гіпотезу про виникнення сонячної системи з розсіяних мас космічної
речовини.

Щодо Миколая Коперніка, то можна припустити, його лекції слухали у Краківського
університету, де в ті часи навчалися Микола Тичина, «Петро Іванович з
Володимирського владацтва», Варфоломей з Галича, Лукаш Лучко, Матвій Емерицький
з Києва, Іван Іванович з Кам'янця, Степан Іванович із Кременця та інші.
Найімовірніше, що з Падуї чи Кракова його теорії занесло на наші землі це
студентство. 

Цікаво також, що існують відомості про навчання наших студентів поруч з
Коперніком, а найбільш цікавим є дослідження історика Я.Ісаєвич, одного з
кращих біографів ще одного видатного українця, про студента кількох
європейських університетів, а згодом ректора легендарного Болонського
університету (внесений в список його 700 найвидатніших діячів) та викладача
Краківського університету, Юрія Дрогобича (Котермака). Він писав, що видатний
земляк викладав астрономію і медицину, якраз у ті роки, коли ці предмети
студіював у Кракові Миколай Коперник. Є підстави вважати, що майбутній
славетний астроном був у числі слухачів Юрія Дрогобича. Відомо, що у своїх
лекціях «доктор Юрій» пояснював між іншим і твори Галена. А ще одним його
видатним учнем міг бути поет-гуманіст з Німеччини Конрад Цельтіс (писав, що
його вчителем був "Роксоланус"). Знайомство молодого Коперніка з Юрієм
Дрогобичем тим більш ймовірне, якщо врахувати, що до прибуття у Краків Копернік
вивчав астрономію під керівництвом поляка Миколи Водки, разом з яким Юрій
Дрогобич вчився в Болоньї і був його близьким другом. Лише за доби Ренесансу в
Краківському університеті викладали 13 професорів з «русинських земель». 

.

Крім згаданих вище Бетховена, Моцарта, Галілея, Канта, Коперника тощо,
українські студенти були знайомі і з іншими видатними діячами культури та
науки, впливали на розвиток Європи і поширення знань. Багато з них торували цей
шлях в легендарну добу XIV-XVII ст., яку у нас прийнято називати козацькою.
Деякі стали видатними діячами науки свого часу, лишались ректорами і
професорами європейських університетів. Більшість повертались на наші землі та
ставали викладачами тут, фактично створивши наукову школу, про яку я писав на
початку статті і вплив якої на східних сусідів був колосальним. Їх внесок у
світовий освітній процес був досить вагомий, просвітителі утверджували ідеї
гуманізму та новітніх наукових знань.

Києво-Могилянська академія гідно продовжила і розвивала напрацювання Острозької
школи, користуючись високими плодами кращих європейських зразків, вийшла на
новий рівень. Курси лекцій, твори, промови професорів Києво-Могилянської
академії та їхніх вихованців рясніли іменами, пов'язаними з римською
античністю: історією, літературою, науковими ідеями. Найчастіше в них
згадувались Плавт, Теренцій, Цицерон, Катулл, Вергілій, Горацій, Тібулл,
Плінії, Проперцій, Овідій, Сенека, Ювенал, Aпулей, Марціал, Вітрувій, Лукрецій
Кар, Епікур, Клавдій Птолемей, Боецій та інші. У сучасного читача створюється
враження, що наші пращури в Київській академії знали античних письменників і
поетів, філософів і математиків, істориків та астрономів значно глибше і ширше,
ніж їх знають нині. І якщо Микола Гоголь зображував Тараса Бульбу як такого, що
не лише послав своїх синів учитися до Київської академії, а й сам читав
Горація, то в цьому фіксувалося лише пересічне явище духовної атмосфери
тогочасної України; до суцільної неграмотності вона пізніше ще довго йшла під
високою рукою пресвітлого московського володаря.

.

Про це зазначу окремо. Справжній грабунок розвернувся з московської сторони,
яка примусом, хитрощами, грошима, всіма силами знекровлювала високоосвічену
Україну. Російський синод в 1721 р. досить категорично вказував: «В Московских
школах мало учителей, а ко ученню философии весьма никого нет; а в Киеве
обретаются ко учению философии, риторики и пиитики способныи мужи... И по его
великого государя указу велено способных ко ученню персон из Киевопечерского
монастыря или где инди кто обритается, отправить к Москве обычайно на подводах
без замедления». 

У монографії «Історія Московської слов’яно-греко-латинської академії» 1855 р.
російський дослідник С.Смирнов констатував, що з початку і до часів митрополита
Платона "московська академія здебільшого мала наставників з Києва".

«Розвиток творчої думки в Російської Імперії не можна охопити і зрозуміти, не
вивчивши попередньо український гуманізм. Україна була осередком,
трансформатором, передавачем західної літератури протягом довгого часу: вона
була також лабораторією вірша. Овідій і Горацій в XVII столітті лежали на столі
українських гуманістів поруч з Біблією. Ртищев, Ордин-Нащокін, Матвєєв,
Голіцин, Котошихін зобов'язані прилученням до культури Нового Часу не тільки
полякам і німцям, а й українцям. Згадаймо, що перетворення Петра зв'язані з
великим киянином Феофаном Прокоповичем, що поєднував церковнослов'янську
пишномовність з пишною риторикою бароко. А тому варто повернутись до належної
оцінки значення України і Білої Русі у передачі далі на схід західної
літератури і освіченості» - робив категоричний висновок інший російський
дослідник І.Голенищев-Кутузов.

«І хоча перебільшенням було б стверджувати, що українські професори перебували
на рівні найвидатніших досягнень тогочасної європейської науки і культури.
Проте, «профессор Болонского університету астроном Юрій з Дрогобича, поет и
гуманіст Павло Русин з Кросна, відомий політичний письменник
Оріховський-Роксолан, засновник нового жанру літератури ("селянки") Шимонович
(Симон Сімонід), який прославив рідний Львів від Константинополя до Амстердама,
працелюбний магістр Лукаш з галицького Нового Міста і інші стали засновниками
гуманістичної культури не лише України, але і всього регіону, створили цілу
гуманістичну течію» - констатує Голенищев-Кутузов.

Тому беззаперечним є визнання факту, що саме вони заклали необхідний фундамент,
підготували платформу для народження власних ідей, підходів, знань і мудрості
не лише на наших землях. І все це стало можливим завдяки європейській освіті,
глибокому поширенню в Україні, утвердженню і переосмисленню найкращих здобутків
у галузі науки і культури інших народів. 

Саме культура й освіта відобразили притаманний українцям спосіб життя,
мислення, світобачення, естетичні уподобання, моральні і правові норми,
політичні прагнення й культурно-освітні орієнтири. Активна участь українців в
освітньо-культурному просторі тогочасної Європи включила Україну до світових
інтеграційних процесів, які у подальшому забезпечили динамічний
культурологічний практикум, спрямований на цілісне розуміння безперервності
української національної культури.

Важко зрозуміти, чому ці теми не набули популярності в Україні, чому так мало
ми знаємо про кожного з названих у статті (тисяч інших, неназваних).

Може виною тому звичка козаків вчиняти: «...бойові подвиги, без жодного наміру
уславитися серед майбутніх поколінь або поширювати про себе відомості серед
інших народів» - як писав про них француз П.Шевальє в середині XVII ст.

Чи за словами Юрія Дрогобича:

«чинити «користь роду людському» без «честолюбства та зиску».

Але окреслені у статті теми потребують широкої популяризації та відповідного
переосмислення українською нацією.

(С) 2020. Ігор Полуектов

.

Замість епілогу:

«Більшість тепер виставляє напоказ свої праці, святотче,

Дбаючи щонайпильніш про честолюбство та зиск.

Я ж мої книги у світ випускаю з єдиним бажанням:

Хай буде користь від них роду людському в житті». 

Юрій Дрогобич (Котермак). 1483 р.
