%%beginhead 
 
%%file 17_03_2023.fb.stomina_oksana.mariupol.1.daty_pamjatatymu_vichno
%%parent 17_03_2023
 
%%url https://www.facebook.com/oksana.stomina/posts/pfbid0DRVzFrdHRUSa3iX1yPqrP7P5Aj4bbH33uDCJgzo7rfMThWon8m9mqTwMj4n8F4H9l
 
%%author_id stomina_oksana.mariupol
%%date 17_03_2023
 
%%tags mariupol,mariupol.war
%%title Я майже ніколи не запам’ятовую дати, та цю пам’ятатиму вічно
 
%%endhead 

\subsection{Я майже ніколи не запам’ятовую дати, та цю пам’ятатиму вічно}
\label{sec:17_03_2023.fb.stomina_oksana.mariupol.1.daty_pamjatatymu_vichno}

\Purl{https://www.facebook.com/oksana.stomina/posts/pfbid0DRVzFrdHRUSa3iX1yPqrP7P5Aj4bbH33uDCJgzo7rfMThWon8m9mqTwMj4n8F4H9l}
\ifcmt
 author_begin
   author_id stomina_oksana.mariupol
 author_end
\fi

Я майже ніколи не запам'ятовую дати, та цю пам'ятатиму вічно. 16 березня для
мене не лише про театр. Сьогодні – рівно рік, як я не бачила своє рідне місто й
свого чоловіка. 

У нас було не більше п'яти хвилин. Ми обійнялись в під'їзді, наче школярі. Та я
навіть притулитися нормально не могла, бо заважали Дімкіни бронік і зброя. А ще
я не могла заплакати. Дуже хотіла, але не могла. 

- Не хвилюйся, - шепнув мені чоловік, - підмога вже близько. Вони стоять у
посадці за Федорівкою. Незабаром будуть тут.

- Правда? – схвильовано перепитала я.

- Сто відсотків! - відповів Діма. І ще раз обійняв, поцілував  і, змахнувши
сльози, вийшов з під'їзду. 

\ii{17_03_2023.fb.stomina_oksana.mariupol.1.daty_pamjatatymu_vichno.pic.1}

А потім була дорога зруйнованими вулицями повз розтрощені й вигорілі вщент
автівки, які намагалися, але не змогли виїхати, останній погляд на задимлений
Маріуполь, перші сльози, сніг, який здався мені попелом, автоматні черги перед
блокпостом, міни вздовж ґрунтовки перед Василівкою, кількаденний путь до
підконтрольної, розуміння, що за Федіровкою немає підмоги й нестерпний біль від
цього.  Порятунок, який не приніс полегшення, адже я виїхала, а серце
залишилось там.

************

Далі - два місяці \enquote{Азовсталі} й десять місяців полону. У Дімки.
У нас. Пояснити, що відчуваєш, коли майже в прямому ефірі бачиш,
як русня намагається вбити твого чоловіка зі всіх наявних видів
зброї, неможливо. Пояснити, що відчуваєш, коли твій чоловік не
виходить на зв'язок, неможливо. 

А ще неможливо пояснити, що відчуваєш, коли після десяти
місяців полону, попри  свідчення побратимів, які були там з
ним, твій чоловік досі вважається \enquote{зниклим без вісті}, тому що
Червоний Хрест, який зафіксував його при виводі з \enquote{Азовсталь}
і в травні повідомив мене про це, раптом передумав і чомусь
вперто не надає підтвердження у відповідні українські
інституції. 

Тож, друзі, величезна кількість оборонців Маріуполя,
героїв \enquote{Азовсталі}, яких знає і шанує весь світ, наразі
й досі військовополоненими не вважаються. На жаль і
сором, наші інституції не беруть до уваги свідчень своїх
героїв, а чекають підтвердження з боку країни-агресора й
Міжнародної організації-зрадниці. Мені одній здається це
дивним??

%\ii{17_03_2023.fb.stomina_oksana.mariupol.1.daty_pamjatatymu_vichno.cmt}
