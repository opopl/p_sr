% vim: keymap=russian-jcukenwin
%%beginhead 
 
%%file letters.comments
%%parent letters
 
%%url 
 
%%author_id 
%%date 
 
%%tags 
%%title 
 
%%endhead 

%https://www.facebook.com/urtab/posts/pfbid024GtGyWgTauBg4vAk56QR65J5RHCqancaYFYfhHNEaFfpVNeHbsWwVL6uBV77X4Xl

Дуже цікаво пишете, дякую. Знаєте, що тут згадалось. Свого часу був знятий
серіал Геній - американці знімали (National Geographic), так от, перша частина
цього серіалу про Альберта Айнштайна - фізика-теоретика, творця спеціальної та
загальної теорії відносності та також автора листа президенту США Франкліну
Рузвельту, з якого почалось створення атомної бомби в Лос-Аламосі, що в
результаті призвело до атомних бомбардувань Хіросіми і Нагасакі... ну а друга
частина - про Пікассо. Кожна частина присвячена якійсь одній людині, і в кожній
частині багато серій. Короче, суть у чому. Творці цього фільму явно не
намагались зробити із Айнштайна святого, а показали його життя у всіх його
суперечностях, де Айнштайн - крім того, що показані всі його творчі натхненні
пошуки від школяра та студента аж до створення спеціальної та загальної теорії
відносності - його найвищого творчого та наукового досягнення, - виходить як і
дуже добра, неймовірна чуйна людина, так і людина, досить таки неоднозначна з
точки зору моралі та його відносин із власною сім'єю (особливо з точки зору
його відносин з Мілевою Маріч) Короче, вийшов насправді дуже життєвий фільм,
який я із задоволенням передивлявся багато раз. І сподіваюсь, колись подібний
фільм знімуть і про нашого Генія Тараса, і краще буде, якщо це зроблять самі
українці, врешті решт, не чекаючи, поки хтось інший не зробить це за них.


%14:13:17 21-03-23
а, ну добре... LaTeX (TeX) - це просто технологія програмування документів,
де-факто стандарт у науковому світі, яку використовують фізики та математики та
вчені інших спеціальностей для запису та публікації своїх робіт. Насправді це
дуже потужна штука, за допомогою якої можна підготувати публікації або книжки
будь-якого рівня складності... А я її використовую для архівації та збереження
різноманітних публікацій, як про війну, так і не тільки. Останнім часом активно
займаюсь Маріуполем, хоча я сам киянин. Як це виглядає, можете подивитись ось
тут https://t.me/kyiv_fortress_1 або
https://archive.org/details/@kyiv_chronicler
