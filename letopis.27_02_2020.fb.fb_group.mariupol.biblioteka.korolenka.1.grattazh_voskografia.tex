% vim: keymap=russian-jcukenwin
%%beginhead 
 
%%file 27_02_2020.fb.fb_group.mariupol.biblioteka.korolenka.1.grattazh_voskografia
%%parent 27_02_2020
 
%%url https://www.facebook.com/groups/1476321979131170/posts/2775756842521004
 
%%author_id fb_group.mariupol.biblioteka.korolenka,kibkalo_natalia.mariupol.biblioteka.korolenko
%%date 27_02_2020
 
%%tags biblioteka,hudozhnik,isskustvo,kartina,mariupol,mariupol.pre_war,master_klass,risunok,tvorchestvo
%%title Цікавий граттаж своїм незвичайним ефектом і простотою. Ще одна назва техніки — воскографія
 
%%endhead 
 
\subsection{Цікавий граттаж своїм незвичайним ефектом і простотою. Ще одна назва техніки — воскографія}
\label{sec:27_02_2020.fb.fb_group.mariupol.biblioteka.korolenka.1.grattazh_voskografia}
 
\Purl{https://www.facebook.com/groups/1476321979131170/posts/2775756842521004}
\ifcmt
 author_begin
   author_id fb_group.mariupol.biblioteka.korolenka,kibkalo_natalia.mariupol.biblioteka.korolenko
 author_end
\fi

У місті Маріуполі - дуже багато творчо-активних городян, а образотворче
мистецтво викликає особливий інтерес. Тому в ізостудії
\href{https://www.facebook.com/groups/1476321979131170}{Центральна міська
публічна бібліотека ім. В. Г. Короленка м. Маріуполь} завжди багато любителів
малювати: і дітей, і дорослих.  26 лютого майстер-класом в бібліотечній
ізостудії керувала викладач Маріупольської школи мистецтв
\href{https://www.facebook.com/ElenaSydorova999}{Елена Сидорова} і присвячений
він був техніці «граттаж». Цікавий граттаж своїм незвичайним ефектом і
простотою. Ще одна назва техніки — воскографія. Чому? Бо її принцип такий:
картонна основа для малювання покривається різнокольоровими восковими крейдами,
потім фарбується і вже тоді зображення наноситься продряпуванням.  Олена охоче
розповідала про «цікавинки» граттажу. Наприклад, що професіонали часто замість
воску користуються яєчними жовтками або спеціальної глиною. А коли треба було
зафарбувати віск, скориставшись гуашшю, то за порадою майстра фарби змішували з
розчином мила, щоби вони не скочувалися. Поки пофарбована поверхня
просушувалася, учасники МК роздивлялись бібліотечні книжки, вирішуя: щоби
«такого» взяти додому почитати. Для видряпування малюнка «художники»
користувалися дерев'яними шпажками, вони дозволяють робити штрихи чіткими і
красивими. Малювали зображення квітів, рибок, котиків та зайчиків – всі малюнки
вийшли експресивно-чарівними і учасники МК зацінили неординарність кольорового
граттажу.

%\ii{27_02_2020.fb.fb_group.mariupol.biblioteka.korolenka.1.grattazh_voskografia.cmt}
