% vim: keymap=russian-jcukenwin
%%beginhead 
 
%%file slova.kompozitor
%%parent slova
 
%%url 
 
%%author 
%%author_id 
%%author_url 
 
%%tags 
%%title 
 
%%endhead 
\chapter{Композитор}

%%%cit
%%%cit_pic
\ifcmt
  pic https://avatars.mds.yandex.net/get-zen_doc/2350270/pub_5fcd3b457e300d7ccad31c73_5fcd5d6e7e300d7ccaf4e1c7/scale_1200
	caption Владимир Ивасюк
\fi
%%%cit_text
Многие из вас, не сомневаюсь помнят прекрасные украинские шлягеры \enquote{Червона
рута} и \enquote{Водограй}, которые пели на огромных просторах СССР и в странах
Восточной Европы. Некоторые слушатели с удовольствием слушали и напевали их,
даже не понимая всех слов.  Нынешний год - юбилейный для двух прекрасных
украинских песен. Им исполнилось 50 лет. Автор обоих - талантливейший
\emph{композитор} Владимир Ивасюк, несправедливо рано ушедший из жизни 
%%%cit_comment
%%%cit_title
\citTitle{Какая легенда вдохновила Владимира Ивасюка написать \enquote{Червону руту} (украинскому шлягеру - 50)}, 
Татьяна Кроп, zen.yandex.ru, 07.12.2020
%%%endcit

