% vim: keymap=russian-jcukenwin
%%beginhead 
 
%%file 28_01_2022.fb.fb_group.story_kiev_ua.1.kiev_visim_sekretiv
%%parent 28_01_2022
 
%%url https://www.facebook.com/groups/story.kiev.ua/posts/1849695205227272
 
%%author_id fb_group.story_kiev_ua,poluektov_igor
%%date 
 
%%tags kiev
%%title Невідомий Київ. Вісім секретів культурного міста
 
%%endhead 
 
\subsection{Невідомий Київ. Вісім секретів культурного міста}
\label{sec:28_01_2022.fb.fb_group.story_kiev_ua.1.kiev_visim_sekretiv}
 
\Purl{https://www.facebook.com/groups/story.kiev.ua/posts/1849695205227272}
\ifcmt
 author_begin
   author_id fb_group.story_kiev_ua,poluektov_igor
 author_end
\fi

Невідомий Київ.

Вісім секретів культурного міста.

Для когось столиця - це запах шаурми, для когось - хаотична забудова, для
когось - просто робота, буденність і сірість...

Справжній, яскравий і цікавий Київ непомітний для ока непідготовленого, хто міг
прожити тут все життя, але ніколи не заглиблювався в історію, ніколи не розумів
та не відчував. Не знає його насправді.

\ii{28_01_2022.fb.fb_group.story_kiev_ua.1.kiev_visim_sekretiv.scr.1}

Проведу маленьку екскурсію місцями, які сам люблю, але про які мало знають.
Туди, де можна пізнати секрети культурної столиці, її глибину і різноманіття. 

Не стану писати про модні кав’ярні, ресторани і подібні місця, якими насичений
фб, адже вони лише місце відпочинку перед дійсно цікавими локаціями, а аж ніяк
не мета культурного дозвілля. 

\ii{28_01_2022.fb.fb_group.story_kiev_ua.1.kiev_visim_sekretiv.scr.2}

Тож, відправляємось у інший, цікавий Київ:

1. Залишилося зовсім мало часу, щоб побачити головний український скарб -
ЗОЛОТУ ПЕКТОРАЛЬ з Товстої могили!

Лише до 30.01.2022 року в \enquote{Музеї історичних коштовностей України}
(вул.Лаврська, 9) можна побачити оригінал золотої пекторалі, а також і низку
інших красивих експонатів. 

Виставка відкрита в 2021 році до 50-річчя з визначної дати в історії
української археології, коли Борис Мозолевський досліджував курган поблизу
м.Покров і відкрив одну з найбільш цінних знахідок.

Особливу увагу до виставки привернули події із міжнародним судом, де російська
федерація намагається відсудити у України належне останній скіфське золото.
Україна перемагає в європейських судах і перші 19 експонатів вже повернуті та
презентовані на цій виставці.

Поспішіть, оригінал пекторалі буде доступний для огляду лише кілька днів! 

Але інші прекрасні експонати музею можна буде подивитись в будь-який час.

2. На території Печерської лаври обов’язково необхідно відвідати виставку:

\enquote{КНЯЗІ ОСТРОЗЬКІ: ЄВРОПЕЙСЬКИЙ ВИМІР УКРАЇНСЬКОЇ ІСТОРІЇ}.

Саме з цим князівським родом пов’язана золота доба української історії. 

Навіть більше, Острозькі відігравали значну роль в історичних процесах, що
відбувались в Європі в XV-XVII ст. 

Найвідомішими представниками українського шляхетського роду були:

Данило Острозький, який приймав участь в переможній битві 1362 року на Синіх
водах, коли литово-руські війська перемогли Ординців і звільнили більшу частину
Руси-України. 

Федір Острозький, що став пострижеником Києво-Печерського монастиря і похований
тут як один із найшанованіших святих - преп.Феодосій.

Костянтин Іванович Острозький - один з головних героїв нашої історії,
багаторічний гетьман, найкращий полководець Європи XVI ст., некоронований
король Руси-України.

Костянтин-Василь Острозький - київський воєвода, засновник Острозької академії
та друкарні, \enquote{Перше вогнище освіти, яке як факел освітило всю східну
Європу новими знаннями} - за словами М. Грушевського.  

Про родовід Острозький я писав в пості, коли подорожував в Дубно.

На виставці також можна побачити одну з моїх улюблених мап 1613 року, яка
містить назву «Україна» (Vkraina; позначаючи так Київщину і землі Подніпров‘я),
що була розроблена за участі Острозьких і кращих інтелектуалів Острозької
школи.

А в відновленому Успенському соборі, на місці де був похований князь, можна і
потрібно побачили відтворений надгробок Костянтина Івановича Острозького
(1460-1530). Символ великого європейського минулого України.

Майже 500 років тому тут з великими почестями був похований наш видатний
гетьман, славетний переможець Московського князівства у війні 1514 року на
Дніпрі, а також переможець інших численних набігів орд на землі Руси.

Чому так важливо побувати тут і віддати шану славетного українському герою, я
писав у пості в цій групі. 

3. Якщо вас втомлюють музеї, але цікаво скласти загальні уявлення про історію
України, зрозуміти звідки і як з’явились українці. Чи, наприклад, є лише день,
щоб весело і цікаво розповісти іноземцю про нашу країну, то зробити це можна в
МУЗЕЇ СТАНОВЛЕННЯ УКРАЇНСЬКОЇ НАЦІЇ (знаходиться в приміщеннях монументу
\enquote{Батьківщина-мати}).

Тут інформація подана в ігровій формі, можна осягнути базові знання завдяки
інтерактиву. А історичні персонажі оживають завдяки макетам в повний зріст,
яким позадрить знаменитий Музей мадам Тюссо. 

Проте, це не просто виглядає як такий собі історичний Діснейленд, красиво,
жваво і цікаво, але і буквально кожна довідка та інформаційний стенд підписані
певним фаховим істориком, які своїми іменами відповідають за надану інформацію.
Тож, все це ще і інформативно та корисно.

Як я писав після першого відвідання музею: це найкраще, що зі мною сталося
останнім часом - заглиблення в атмосферу максимальне, а провести там можна і
дві і три години, які пролетять непомітно. Але найбільше враження справили два
зали, які розповідають про сучасну історію України, містять документальні
свідчення російсько-української війни.


4. ДЕВ'ЯТЬ ПАМ'ЯТНИХ КНЯЖИХ ПОХОВАНЬ В ЦЕНТРІ КИЄВА, про які майже ніхто не знає.

Чи не дивина? Подивіться нижче моє відео про це місце і ви також здивуєтесь.

Багато років воджу туди своїх друзів і гостей, які приїжджають в Київ. І майже
ніхто з них ніколи не чув про це місце.

Як так може бути, що славетні князі Руси зовсім не внесені у туристичні
путівники, не встановлені скрізь відповідні вказівники, не музеєфіковано
частину колишнього монастиря, де були поховання? Загадка для мене. 

Тим паче, що місце це одне із найкрасивіших і найбільш затишних в столиці. 

Розгадка можливо в тому, що деякі з цих князів уславились як захисники Руси в
тч. від навал Ростово-Володимиро-Суздальских воєвак, а тому не вкладалися в
героїчний пантеон часів імперії/совєтів, умисно були забуті.

Від Мстислава Великого, найстаршого сина Володимира Мономаха та його першої
дружини Ґіти (дочки англійського короля Гарольда II Ґодвінсона). Який був
засновником кам'яної церкви святого Федора (1129), в якій його першим поховали
в 1133 році. Тут знайшли спокій ще вісім кзязів Руси-України: 

Ізяслав Мстиславич (на пам’ятнику написано, що спочив 1154 року), Ростислав
Мстиславич (1168 р.), Ярополк Ізяславич (1169 р.), Мстислав Мстиславич (1173
р.), Володимир Мстиславич (1173 р.), Мстислав Давидович (1187 р.), Ізяслав
Ярославич (1195 р.), Гліб Юрієвич (1198 р.). 

Як бачимо, Федорівський монастир, збудований сином Мономаха князем Мстиславом
Володимировичем, являв собою родинний монастир Мстиславичів (ще однієї
героїчної родини нашої історії, яку намагались з неї викреслити). Деякі з
представників цієї родини визнані святими. Але належного пам’ятника в Києві
жодному з них досі немає.

Знайти це місце можна кількома способами: зайшовши від пам’ятника княгині
Ольги, через арку і таємним парк готелю \enquote{InterContinental}, або через
одну з арок на вулиці Володимирській (біля буд. 9 і 5), або через арку буд.6 по
вул.В.Житомирській. 

Зараз тут ми можемо бачити згаданий пам’ятник з хрестом, невеличку сучасну
дерев’яну церкву св.Георгія, яка відома тим, що містить всередині петриківський
розпис, а також і старовинні фундаменти Федорівського монастиря (все це на
відео нижче). 

Місце наповнене історією. Але відвідуючи його треба завчасно підготуватись,
почитати.

Наприклад, хоча б про те, що Мстиславичі запам'ятались в київській історії як
захисники істинної Руси. 

В тч. від нападів Юрія Довгорукого, якого зовсім юним (як найменшого в родині)
відправили в далекі провінційні землі Залісся, які залежали від Руси, але нею
не були, а саме в Ростово-Суздальську землю. І там, на землях угро-фінського
населення, під боярським наставництвом, він виріс зовсім відірваним від
традицій Києва. Взяв собі за дружину доньку половецького хана Аєпи та фактично
складав з половцями ворожу опозицію до Руси. 

А його син Андрій Боголюбський взагалі вже сприймав Русь як ворожу територію,
сам був цілком і повністю типовим азіатом по крові та вихованню. Фактично, став
половцем з іншими домішками, якого один із найавторитетніших московських
істориків назвав першим \enquote{московитом (великоросом)}: 

«З Андрієм Боголюбським великорос вперше вийшов на історичну арену». 

І цей перший великорос, який започаткує те, що ми пізніше знатимемо під
Московією і що буде природним ворогом Руси протягом століть (та і досі),
увійшов в історію як той, хто першим розграбував і попалив Київ. Сталося це в
1169 році.

Як пише про це безпрецедентне варварство наш києво-руський літопис: \enquote{Узятий же
був Київ місяця березня у дванадцятий [день], у середу другої неділі посту. І
грабували вони два дні увесь город — Подолля, і Гору, і монастирі, і Софію, і
Десятинну Богородицю. І не було помилування анікому і нізвідки: церкви горіли,
християн убивали, а других в’язали, жінок вели в полон, силоміць розлучаючи із
мужами їхніми, діти ридали, дивлячись на матерів своїх. І взяли вони майна
безліч, і церкви оголили од ікон, і книг, і риз, і дзвони познімали… Запалений
був навіть монастир Печерський святої Богородиці поганими, але бог молитвами
святої богородиці оберіг його од такої біди. І був у Києві серед усіх людей
стогін, і туга, і скорбота невтишима, і сльози безперестаннії}. 

Як бачимо, нападників літописець русами не вважає, називає їх не просто
ситуаційними ворогами (як нам це подавали в совєцьких школах), а
\enquote{поганими} - термін який застосовувався до язичників, половців,
найбільших ворогів Руси. 

Але вже 1170 року Мстислав Ізяславич почав контрнаступ і повернув собі Київ
(невдовзі помер). Після чого Андрій Боголюбський зібрав небачене доти військо з
мокшанських боліт і половецьких степів та в 1173 році вирушив в похід з метою
остаточно зруйнувати столицю Руси. В грудні 1173 року під Вишгородом це військо
було дощенту розгромлене спільним силами під командуванням Мстислава
Ростиславича та луцького князя Ярослава Ізяславича, який після перемоги став
великим київським князем. 

От такі наші герої про яких в Києві майже нічого немає. А \enquote{першого
великороса} невдовзі після того знайшли вбитим на своїх болотах. Московити
історично не пробачають своїм вождям таких поразок. Тим паче від українців  @igg{fbicon.smile}   

Так було майже 850 років тому, так буде і зараз.

5. МУЗЕЙ КИРИЛІВСЬКА ЦЕРКВА.

Це не просто оповита легендами місцина, де билинні українські герої перемагали
зміїв гориничів, а одне із найсвятіших місць Києва.

Саме тут на Дорогожичах (вул.Теліги, 12) розміщено київський храм Святого
Кирила XII ст.  Один із найважливіших об’єктів часів Руси та і загалом нашої
історії (бо таких збереглося - на пальцях рук перерахувати).

Стіни його увібрали своїм корінням і зберігають пам’ять про те, що відбувалося
тут протягом останніх майже тисячу років. Намолений сакрум зберігає накопичену
віками інформацію, треба тільки вміти прочитати її – через архітектуру,
унікальний живопис, численні, ще не повністю розшифровані написи графіті, через
ще не досліджені потаємні підземні ходи та поховання, частина яких з'явилась
задовго до, століттями формували Кирилівський цвинтар, а відтак і історію
функціонування Києво-Кирилівської Свято-Троїцької обителі.

Про історію монастиря і церкви я писав багато, як і про важкі сторінки святині
і повзучу окупацію.

Про головних героїв завдяки яким храм зберігся до наших днів (Острозьких, Савву
Туптало, Мелетія Дзика, Івану Мазепу ін.), а також і про те, як мені вдалося
допомогти святині і відновити в ній пам’ять про видатні родини Тупталів і
Дзиків.

Церква вражає історією часів Руси, до якої можна буквально доторкнутись,
княжими похованнями, а також і неймовірною палітрою витончених художніх
стінописів і робіт, від XII ст, через плеяду видатних художників XVII-XIX ст.,
і аж до сьогодні.

Прямо зараз у стінах музею відкрилась виставки «ЗІРКА, ЩО ВІД КИЄВА ВОЗСІЯЛА»,
яка стала одним із етапів «Меценатського проекту відновлення реліквій
Кирилівської церкви», та присвячена родині Тупталів. Тут можна бачити не лише
портрети колишнього ігумена, просвітника і інтелектуала Димитрія Ростовського
(Туптала), а і його батька - Савви Туптала. Який мені вдалося відновити разом з
художником О. Ольховим і І. Марголіною. Скоро ще з’являться барельєфи на фасаді
церкви, які будуть уславлювати козацькі родини Тупталів і Дзиків. Цю роботу
також вже профінансували і чекаємо результатів.

Прошу всіх сходити в Кирилівську церкву-музей, подивитись виставку (діятиме до
травня), оглянути унікальні стінописи храму, рівних яким мало в Європі, а тоді
написати мені відгук. Це буде дуже приємно для мене, а для вас пізнавально,
цікаво і корисно.

6. Музей «КАМ’ЯНИЦЯ КИЇВСЬКОГО ВІЙТА».

Знаходиться по вул.Костянтинівська 6/8, виконаний в стилі українського бароко,
є прекрасним прикладом житлової архітектури Києва межі XVII-XVIII ст.

Належав київській купецькій та козацькій родині Биковських, двоє з яких
залишились в історії міста як війти. 

Війти - це очільники місцевого самоврядування, що пов’язані із тогочасною
традицією України, Литви, Польщі і Німеччини. Виходила ця традиція із
магдебурзького права, що з кінця XII ст. виникло в Німеччині, а з кінця XIII
ст. активно поширювалося нашими землями.

Щоб якось забути і замовчати цю багатовікову києво-європейську традицію
місцевого самоврядування, яка уславлювала самоорганізацію і незалежність від
королів та царів, за часів імперії був вигаданий міф про тз. “будинок ПєтраІ”.
Хоча цей міф давно спростований, жодного джерела про це не знайдено, а ПєтроІ
ніколи в ньому і не бував, але лише зараз вдалося нарешті позбутись цього
неправдивого кліше. І ми тут маємо яскравий приклад як імперія і совєти
намагались переписувати історію. Проте, правда завжди поверне своє. Як і
відбулось з історією цієї садиби і типової подільської історії війтів. 

Повернемось краще до неї: Київський магістрат викупив будівлю у Биковських з
метою використання у громадських потребах. Значної шкоди і запустіння споруда
зазнала після пожежі 1811 року, яка охопила більшу частину Києво-Подолу.  

Зараз тут розміщено чудовий музей історії міста. 

Цікаві музейні предмети, які доповюють києвознавчий і культурний досвід новими
фактами. Музей обіцяє, що: \enquote{Ви відчуєте, як голосом однієї споруди промовляє
ціле місто, що завжди прагло європейських стандартів життя на рівні освіти,
доброхотства, міського врядування і побуту}. 

Найбільш цінне тут те, що цінують справжніх київських героїв і меценатів, які
зберегли для нас старий добрий Київ: 

Остафія Дашковича (в 2021 році нарешті в столиці з’явилась вулиця імені цього
видатного героя нашої історії), Костянтина і Василя Острозьких (про яких я
чимало писав статей, які відновили багато київських святинь своїм коштом), мого
улюбленого Савви Туптала (портрет якого я повернув в Кирилівську церкву), а
також і легендарного гетьмана Івана Мазепи (який, певно, найбільше вклався у
відновлення і реставрацію київських будівель і збереження обличчя нашого
міста)...

Я зайшов в музей з дитиною, а тому не мав можливості належно вивчити всі
експонати. Але оцінив рівень гостинності, коли з Марком активно спілкувались
працівники, займали його, доки я вичитував тексти і намагався роздивитися як
найбільше (але точно зрозумів, що ще повернусь і не раз). А коли ми
разговорились і наглядачки дізнались, що ми з Марком самі фарбували та
доглядали за пам’ятником Магдебурзькому праву (велике фото якого при виході з
музею), то одна з них кудись зникла, за кілька секунд з’явилась у супроводі
директриси, вказала на нас: \enquote{оце вони!}  @igg{fbicon.smile}  Пані Оксана, попросила нас
розповісти більше. Коротко це зробили. Запросили описати нашу історію і
заповнити анкету. Було незручно, але ми дуже поспішали і відмовились. Я лишив
всі свої дані, щоб зі мною могли зв’язатись пізніше. Тоді вона побігла на
кухню, дістала велику шоколадку і подякувала Марку, сказавши, що вони дуже
цінують таких київських меценатів, яким не все одно до нашої історії і
пам’яток. Така була мі-мі-мішна сцена  @igg{fbicon.smile}  Марк не зрозумів, чому за цікаве
фарбування пам’ятника ще щось і дають, але був радий  @igg{fbicon.smile}  


7. НАЙДАВНІШИЙ ПАМ'ЯТНИК КИЄВА - МАГДЕБУРЗЬКОМУ ПРАВУ. 

Важлива сторінка життя Києва і України, яка майже чотири сотні років нерозривно
пов’язувала нас з Європою і цивілізацією. І яку після окупації наших земель
імперією намагались відмінити (двічі) і забути (багаторазово). 

Це найстарший досі збережений пам’ятник столиці - і вже лише у цьому його
виключна цінність. А ще тут і місце, де річка Почайна сходилась із Дніпром і
могло відбуватись хрещення Руси. Насправді, ні, це було північніше, але
пам’ятні знаки встановлені саме тут  @igg{fbicon.smile}   

Пам’ятник присвячений такій важливій сторінці нашої історії, коли самоврядні
громади з автономними судово-адміністративними інституціями почали
засновуватись у галицьких і волинських містах наприкінці XIII ст., з середини
XIV ст. і до кінця XVIII ст., одне за іншим, українські міста, від Львова до
Полтави, долучаються до прогресивного явища, яке сприяло не лише розвитку
місцевого самоврядування, але і розквіту ремесел, торгівлі, ініціативи.
Магдебурзьке право встановлювало виборну систему органів міського
самоврядування та суду, визначало їх функції, регламентувало діяльність
купецьких об'єднань та цехів, регулювало питання торгівлі, опіки, спадкування,
визначало покарання за злочини тощо. Його поширення на території України
сприяло формуванню нових рис ментальності місцевого населення, залученню до
європейської родини народів, демократизму, меншій орієнтації на центральну
владу, появі схильності будувати суспільне життя на основі правових норм тощо.
Магдебурзьке право сприяло формуванню в Україні засад громадянського
суспільства. Саме ця ознака і досі вирізняє українців. Що ми всі могли бачити
під час Революції гідності та на фронті.

І саме тому, ми два роки тому винесли десятки мішків з брудом звідти,
заштукатурили втрати і пофарбували нижній рівень, щоб він був чистеньким і
яскраво-білим. Від технагляду Охорони пам'яток були поставлені жорсткі умови по
можливим матеріалам, а також і переконували мене, що цього не треба робити, бо
вже в 2021 році буде зроблено комплексний ремонт (показували кошторис на 5,5
млн.грн.). Але його не виконано досі. А так ми хоча б на якийсь час зробили
пам'ятник охайним і доглянутим. 

Проте, вже зараз він знову і терміново потребує таки ремонту. Можновладці не
можуть розібратись у кого на балансі він перебуває, тому це все затягується.
Якщо весною не почнуть ремонт, то доведеться збирати громадськість і зробимо
знову своїми силами (Мінкульт, вважай це погрозою). 


8. ОКРАСА ПОДОЛУ - \enquote{МУЗЕЙ ГЕТЬМАНСТВА}. 

Дуже красива будівля, яка була занедбана до кінця 80х, коли допомогою діаспори
її вдалося відновити і вона постала мов Фенікс з попелу, щоб осяяти собою
Києво-Поділ.

Музей декларує, що з’явився: \enquote{з метою правдивого висвітлення проблем
державотворення в Україні, надання неупередженої оцінки її видатних діячів,
розкриття специфічної форми правління – Гетьманщини}.

Тут постійно діють наукові виставки: \enquote{Гетьман Іван Мазепа},
\enquote{Пилип Орлик – Гетьман, автор першої демократичної конституції України}
і \enquote{Павло Скоропадський та Українська держава 1918 року}. 

Хоча фонди Музею налічують понад 9000 музейних предметів, але зали дуже
невеликі і обійти їх можна менше як за годину. Цікавими є наукові зустрічі, які
регулярно проходять тут, де люди можуть обговорити питання історії нашої
Гетьманщини-України. А ще, кажуть, тут відбуваються романтичні вечори в стилі
Гетьманської епохи і задушевні розмови про цю прекрасну сторінку нашої історії.

Зняв для вас фото будівлі музею дроном, бо мало зустрічав її таких красивих
ракурсів (додаю). 

На цьому поки закінчимо прогулянку Києвом.

Сподіваюсь цей мій путівник стане багатьом у нагоді.

Пізнавайте, досліджуйте наше місто і нашу історію. Вони того варті.

З любов'ю до Києва і України.
