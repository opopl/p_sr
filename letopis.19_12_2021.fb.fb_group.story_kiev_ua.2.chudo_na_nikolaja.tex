% vim: keymap=russian-jcukenwin
%%beginhead 
 
%%file 19_12_2021.fb.fb_group.story_kiev_ua.2.chudo_na_nikolaja
%%parent 19_12_2021
 
%%url https://www.facebook.com/groups/story.kiev.ua/posts/1822263164637143
 
%%author_id fb_group.story_kiev_ua,drobot_tatjana.kiev
%%date 
 
%%tags chudo,kiev,nikolaj_chudotvorec
%%title Чудо, которое случилось с нами от Николая Чудотворца
 
%%endhead 
 
\subsection{Чудо, которое случилось с нами от Николая Чудотворца}
\label{sec:19_12_2021.fb.fb_group.story_kiev_ua.2.chudo_na_nikolaja}
 
\Purl{https://www.facebook.com/groups/story.kiev.ua/posts/1822263164637143}
\ifcmt
 author_begin
   author_id fb_group.story_kiev_ua,drobot_tatjana.kiev
 author_end
\fi

Напишу и я про чудо, которое случилось с нами от Николая Чудотворца. 

Несколько лет назад, 19 декабря, мы с мужем опаздывали на очень важную встречу. 

\ii{19_12_2021.fb.fb_group.story_kiev_ua.2.chudo_na_nikolaja.pic.1}

Вечер, крепкий мороз, сильный ветер, гололед, мы в центре города, и ни одна
служба такси не присылает нам машину. 

Отчаявшись и практически замерзнув, я начала про себя молиться Святителю
Николаю. 

Но, как каждая девочка, я рисовала себе, смахивая замерзшие слезы, радужную
перспективу, что не может же не быть в центре огромного мегаполиса машин, что
вот-вот остановится большое авто, где сиденья с подогревом, и где, непременно,
хорошо пахнет.

Машин нет. От слова совсем. Ни городских троллейбусов/автобусов (район
Лукьяновки), ни маршруток/ такси, ни просто частных авто.

Ни-че-го.

Муж говорит, что еще несколько минут и нас не спасет ни машина, ни бутылка
водки, если мы доберемся живыми, -  мы просто околеем на морозе. 

Честное слово, я начала так неистово молиться, как никогда, прошу про себя
Николая Угодника прислать нам хоть что-то, пусть старое и ржавое, пусть
водитель будет неопытный, ну пришли, пожалуйста, хоть что-нибудь!

И только я это проговорила, как едет старый чихающий москвич, ржавый и
потрепанный как тузик. Мы машем ему, ни ног, ни рук уже не чувствуя. 

Останавливается этот раритет, водитель открывает заклинившую дверь и буквально
приказывает нам садиться.

Едем. 

Муж говорит, куда ехать, водитель бурчит, что вот, мы холода ему внутрь
напустили, говорит, что он живет как раз рядом, куда и нам нужно, едет
медленно, гололед, рассказывает про плохие тормоза, говорит, что сам не знает,
как оказался в этом районе, что ему вообще сюда не надо было и т.д. 

А сам старенький, весь седой, и такой загадочно-добродушный...

Всю дорогу эта железяка, буквально спасшая нам жизнь, чихает, кашляет, в окна
дует, сиденья ледяные, но я такая счастливая @igg{fbicon.heart.eyes} 

Приехали. Света в машине, понятно, нет. Водитель включает фонарик на мобильнике
и освещает торпеду, а там одна единственная иконка одного Святителя Николая,
нашего спасителя. 

Надо было видеть наши лица.

Водитель денег с нас не взял.

Вот такое чудо.
