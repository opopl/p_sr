% vim: keymap=russian-jcukenwin
%%beginhead 
 
%%file moje.kremlevskie_narrativy.jazyk.istoki_krasoty_movy
%%parent moje.kremlevskie_narrativy.jazyk
 
%%url 
 
%%author_id 
%%date 
 
%%tags 
%%title 
 
%%endhead 

\paragraph{Истоки певучести и красоты украинского языка}
\label{sec:moje.kremlevskie_narrativy.jazyk.istoki_krasoty_movy}

Сорри, отвлеклись... продолжая тему... надо признать, что украинские
интеллектуалы, да и просто обычный народ, особенно его \enquote{патриотическая}
часть в массе своей загнали себя в тупик самого слова \enquote{Украина},
разделив свою реальность на Украину и Россию, проукраинские и пророссийские
нарративы, украинцев, малороссов, какаяразница, украина-понад-усе и тому
подобное... загнав поэтому и сам украинский язык в ловушку, в своего рода
информационное гетто, лишив его живительной силы сердца и искренности, разума и
мудрости, чего Украине так не хватает в наше время! А ведь украинский язык же
не появился просто так, - а напротив, он же ж именно такой и есть певучий и
красивый, поскольку в давние времена люди хотели вольно жить на своей земле,
хотели любить и жить в гармонии с природой, - но им приходилось все время
воевать и сражаться за право быть на своей земле, - и отсюда красота языка,
\enquote{смачность} языка,  и также божественная красота украинских песен, -
как отображение внутренних волевых, чистых порывов души и сердца. Вот,
например, такая есть песня с комментариями под видео... 

\href{https://www.youtube.com/watch?v=--iWRp8Jr6k}{%
Українські пісні онлайн. Чом ти не прийшов, youtube}

\raggedcolumns
\begin{multicols}{4} % {
\setlength{\parindent}{0pt}
\obeycr
Чом ти не прийшов,
Як місяць зійшов?
Я тебе чекала.
Чи коня не мав,
Чи стежки не знав,
Мати не пускала?
\smallskip
І коня я мав,
І стежку я знав,
І мати пускала.
Найменша сестра,
Бодай не зросла,
Сідельце сховала.
\smallskip
А старша сестра
Сідельце знайшла,
Коня осідлала:
\enquote{Поїдь, братику,
До дівчиноньки,
Що тебе чекала}.
\smallskip
Тече річенька
Невеличенька,
Схочу - перескочу.
Віддайте мене,
Моя матінко,
За кого я схочу.
\restorecr
\end{multicols} % }

\ifcmt
  tab_begin cols=2,no_fig,center
     pic https://i2.paste.pics/39b5912a65683fffc41863c68bfc5f26.png
     pic https://i2.paste.pics/dd441b74a982c23a31f5cd313bc89e3d.png
  tab_end
\fi

\begin{itemize} % {
\item Боже, как я люблю украинские песни. Я сразу возвращаюсь в детство, когда все
еще были живы-здоровы, а когда за столом собирались родственники..., украинские
песни лились рекой одна за другой.... как же было здорово!!!

\item Процветания тебе, Украина  @igg{fbicon.heart.sparkling} ... с Любовью и
уважением из России...

\item Люблю Украину и украинские песни. Они самые лучшие, душу выворачивают. Слова и
музыка завораживает. Класс!!!!!!!!!

\item Песня звучит, как звучат струны души у прекрасных и красивый людей. Хочется
слушая, подпевать, хотя  не  всё слова понимаю.  @igg{fbicon.hands.pray} 

\item Я сама русская. Но очень люблю украинские песни.

\item молодцы только украинцы так поют сказочно

\item Слушаю, подпеваю с удовольствием! У меня русско-украинские корни. Люблю Россию,
люблю Украину!!! Господь все управит, все у наших народов будет хорошо!!!
\end{itemize} % }

Чувствуете, как красиво звучит? А комментарии заметили, особенно из России?

