% vim: keymap=russian-jcukenwin
%%beginhead 
 
%%file 18_12_2020.news.ru.vesti.1.kazahstran_vaccine
%%parent 18_12_2020
 
%%url https://www.vesti.ru/article/2500440
 
%%author 
%%author_id 
%%author_url 
 
%%tags covid_vaccine,russia,kazahstan,sputnik_v
%%title В Казахстане запустят совместное с РФ производство вакцины
 
%%endhead 
 
\subsection{В Казахстане запустят совместное с РФ производство вакцины}
\label{sec:18_12_2020.news.ru.vesti.1.kazahstran_vaccine}
\Purl{https://www.vesti.ru/article/2500440}

\ifcmt
pic https://cdn-st1.rtr-vesti.ru/vh/pictures/xw/308/131/7.jpg
\fi

В Казахстане на следующей неделе будет запущено совместное с Россией
производство вакцины против коронавируса. Об этом на заседании Совета глав
государств СНГ сообщил президент республики Касым-Жомарт Токаев. Мероприятие
транслирует телеканал "Россия 24".

Казахский лидер отметил ведущую роль России в борьбе с коронавирусом нового
типа. Он напомнил, что РФ первой в мире зарегистрировала вакцину и начала
прививать население.

"Казахстан также разрабатывает собственный препарат, который демонстрирует
хорошие результаты в ходе тестирования, – рассказал Токаев. – В ближайшие дни,
буквально на следующей неделе, мы запустим совместное с Россией производство
вакцин – как российской, так и казахстанской, в том числе и на экспорт".

Ранее пресс-служба президента Казахстана сообщила, что производство препарата
"Спутник V" начнется в стране 22 декабря.

Российский Центр имени Гамалеи разработал "Спутник V" – первый в мире препарат
для вакцинации против коронавируса. 11 августа 2020 года он был
зарегистрирован. Позже была зарегистрирована вакцина научного центра "Вектор"
"ЭпиВакКорона". Ведется работа над третьим препаратом, а также над облегченным
вариантом вакцины.\Furl{https://smotrim.ru/article/2500209}

О так называемой лайт-вакцине, которая подразумевает введение всего одного
компонента, рассказал в ходе Ежегодной пресс-конференции\Furl{https://smotrim.ru/article/2500192} президент России
Владимир Путин.\Furl{https://www.vesti.ru/doc.html?id=291713}
