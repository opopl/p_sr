%%beginhead 
 
%%file 26_01_2023.fb.fb_group.vrazhennja.ua.1.451_za_farengeitom__
%%parent 26_01_2023
 
%%url https://www.facebook.com/groups/vrajenniya.ua/posts/1811157259242705
 
%%author_id fb_group.vrazhennja.ua,kotikova_oksana.mariupol
%%date 26_01_2023
 
%%tags fantastika,bradbury_ray.writer,kniga,vpechatlenie,chtenie,literatura,chelovek
%%title 451 за Фаренгейтом - Рей Бредбері
 
%%endhead 

\subsection{451 за Фаренгейтом - Рей Бредбері}
\label{sec:26_01_2023.fb.fb_group.vrazhennja.ua.1.451_za_farengeitom__}
 
\Purl{https://www.facebook.com/groups/vrajenniya.ua/posts/1811157259242705}
\ifcmt
 author_begin
   author_id fb_group.vrazhennja.ua,kotikova_oksana.mariupol
 author_end
\fi

\#читаюфантастику
\#читаюксд

451 за Фаренгейтом

Рей Бредбері

Богдан, Тернопіль

Переклад - Євгена Крижевича

272с.

Люблю Бредбері. Від того важко писати відгук на його книги. Люблю за його
суперсилу: говорити просто, образно, зрозуміло, через рот своїх героїв. В ці
моменти ти наче присутній там, з героями, і приймаєш участь. Люблю Бредбері за
щирість. Всі речі він називає своїми іменами. Без прикрас. Те, що огидне - так
і виглядатиме.

Ви скажите - йой! Так всі хороші автори так пишуть! І ви будете праві. Але щодо
Бредбері, то є щось в його стилі таке, знаєте, як в токайському вині, особливе,
ледь вловиме, але з довгим своєрідним посмаком.

Цю книгу читати завжди і важко, і легко. Важко від усвідомлення, що популісти у
своєму прагнені влади можуть йти на повідку у сірої безлікої маси, яка не хоче
знати правди, не хоче заглиблюватися в себе, не хоче чесності, а прагне лише
голосливих «родичів» та оповідок про пристрасні поцілунки і жорстокі бійки,
відкидаючи мислителів та творців, що могли б допомогти зростанню душ. І
приведуть нас до отакого суспільства з жінками із кроваво червоними нігтями і
пустими очима, що закидуються снодійним і від того вмирають. Та чоловіками, що
думають, що вони щасливі, адже є і дім, і дружина, і діти, і робота - всі
ознаки успіху чи стабільності, але чогось таки нема. І від того лікарі не
вилікують. І війна стає виходом. Або - самогубство. І дітьми, які від нестачі
любові прагнуть вбивати, бо це єдиний спосіб отримати гострі щирі емоції, на
які вони тільки й можуть розраховувати, які можливо знайти у суспільстві
біороботів.

Дорослішати завжди боляче. Ми проходимо цей шлях з дитинства до свідомого віку
(щоправда - не всі, на жаль) розтягнуто у часу і просторі. Тому цей процес є
для нас природнім і не помітним. Настільки, що ми мало замислюємося над ним. А
в цій книзі ми спостерігаємо різке, болісне, розпачливо стрімке відривання душі
Монтеґа від сталих скалічених норм суспільства, що існує серед «родичів» та
«шоу» телевізійних стін, не помічаючи при цьому тих, хто поруч, до щирої
свободи, дарованої нам Богом. Дуже красномовно зображено в цій книзі прагнення
безліких мас до стану рабства, коли хтось інший подумає за них, скаже їм як
жити. І показано ціну, яку за таке «спокійне життя» сплачено. А також -
наслідки - самогубства. Тому що людина не може жити без любові. А біоробот не
навчений любити.

Не буду цитувати, тому що там розлого і доволі таки багато, але яскраво і,
навіть, страшно для усвідомлення, написано щодо «толерантності». В мене по
спині пройшовся холодок, коли я то читала (перечитувала, бо я вже читала
російською колись в минулому житті). Невже так було завжди з тою
«толерантністю»? Звідки взагалі це слово взялося? Чому воно з’явилося? Невже
простої щирої поваги стало недостатньо? А чи не є то слово прикриттям для
брехні?

Бредбері пише легко. Так легко, що й підлітки зрозуміють і побачать, не кажу
вже про дорослих. Тому його дуже потрібно читати. Його можна, навіть, читати на
лікарняному ліжку - настільки ті слова спокійно та природньо входять вам у
свідомість, прямісінько в серце.

Навіть якщо ви вже сто разів його читали - завжди приємно перечитувати знову. І
завжди буде, наче вперше, відкриття простих істин.

Окрема подяка за чудовий переклад.

Перепрошую, інтернет не стабільний, тож не витрачатиму час на пошуки, чи є ця
книга на сайті КСД, просто надам посилання на сам сайт:

\url{https://bookclub.ua/}
