% vim: keymap=russian-jcukenwin
%%beginhead 
 
%%file 13_01_2021.fb.stepanov_farid.1.dvory
%%parent 13_01_2021
 
%%url https://www.facebook.com/StepanovFarid/posts/501776464130837
 
%%author Степанов, Фарид
%%author_id stepanov_farid
%%author_url 
 
%%tags gorod,kiev
%%title СТАРЫЕ ГОРОДСКИЕ ДВОРЫ
 
%%endhead 
 
\subsection{СТАРЫЕ ГОРОДСКИЕ ДВОРЫ}
\label{sec:13_01_2021.fb.stepanov_farid.1.dvory}
 
\Purl{https://www.facebook.com/StepanovFarid/posts/501776464130837}
\ifcmt
 author_begin
   author_id stepanov_farid
 author_end
\fi

СТАРЫЕ ГОРОДСКИЕ ДВОРЫ.

Есть в старых дворах что-то притягательное. 

Удивительно, часто неухоженные, полузаброшенные и обветшалые они манят своей
атмосферой, аурой и историей. 

\ifcmt
  tab_begin cols=3

     pic https://scontent-cdg2-1.xx.fbcdn.net/v/t1.6435-9/138231805_501776174130866_7201604545736789931_n.jpg?_nc_cat=107&ccb=1-5&_nc_sid=8bfeb9&_nc_ohc=ybnj_wHChugAX_HEbOv&_nc_ht=scontent-cdg2-1.xx&oh=569fadabb2b9d32915d716f0330e5ce2&oe=6144C8EA

     pic https://scontent-cdt1-1.xx.fbcdn.net/v/t1.6435-9/138799322_501776210797529_761284661860965696_n.jpg?_nc_cat=110&ccb=1-5&_nc_sid=8bfeb9&_nc_ohc=x4mCT5NMP6kAX_Khvws&_nc_ht=scontent-cdt1-1.xx&oh=5d62aa256bc051699541bd041181259e&oe=6143190C

		 pic https://scontent-cdg2-1.xx.fbcdn.net/v/t1.6435-9/139037264_501776250797525_7133822736912068510_n.jpg?_nc_cat=104&ccb=1-5&_nc_sid=8bfeb9&_nc_ohc=3EIvpSqusLYAX_6iQ7B&_nc_ht=scontent-cdg2-1.xx&oh=c54635fba0f1cc76d8bd72116f724cb4&oe=6143EFF1

  tab_end
\fi

Люблю гулять дворами старых улиц Города. В них чувствуешь Город. Его историю и
людей. Сам придумываешь им истории, сам же и становишься частью этих историй.

Не знаю, сколько ещё простоять этим домам и дворам. Город меняется. То тут, то
там, сносят старые дома, а на их месте возводят современные и комфортные. Со
старыми домами уйдут в небытиё и их дворы. Хотя, всё же останутся в памяти тех,
кто там родился, рос, жил. И я внесу свою лепту: оставлю их на фото.

\ifcmt
  tab_begin cols=3

     pic https://scontent-cdt1-1.xx.fbcdn.net/v/t1.6435-9/138823206_501776304130853_2360913850766623344_n.jpg?_nc_cat=101&ccb=1-5&_nc_sid=8bfeb9&_nc_ohc=7u2-jSRSX9cAX9Oysxa&_nc_ht=scontent-cdt1-1.xx&oh=04db81fab133016faae6799535a106df&oe=6144A3AB

     pic https://scontent-cdt1-1.xx.fbcdn.net/v/t1.6435-9/139122496_501776330797517_3313157647191729099_n.jpg?_nc_cat=109&ccb=1-5&_nc_sid=8bfeb9&_nc_ohc=XIDZ_qQ5rlsAX-Hkj8E&_nc_ht=scontent-cdt1-1.xx&oh=13d17111b0e494ae30c3e48396716215&oe=614494B0

		 pic https://scontent-cdg2-1.xx.fbcdn.net/v/t1.6435-9/138197847_501776357464181_5788227193385225294_n.jpg?_nc_cat=111&ccb=1-5&_nc_sid=8bfeb9&_nc_ohc=v3ni8fg_V3QAX9W7TAR&_nc_ht=scontent-cdg2-1.xx&oh=d8762e36d8d10dd14dde01da632072d9&oe=61457CC7

  tab_end
\fi

\index{Пушкинская, улица!Киев}
\index{Прорезная, улица!Киев}

PS:

Рядом двор между Прорезной и Пушкинской. В одном из домов после Второй мировой,
в коммунальной квартире на четвёртом этаже, жила моя семья. А прямо во дворе,
под балконом был Летний кинотеатр. На балконе собирались друзья и сосед. И
смотрели бесплатные киносеансы.

Но, это уже совсем другая история.

(фото 3 декабря 2020 года).
