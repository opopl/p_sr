% vim: keymap=russian-jcukenwin
%%beginhead 
 
%%file 30_05_2018.fb.lesev_igor.1.babchenko
%%parent 30_05_2018
 
%%url https://www.facebook.com/permalink.php?story_fbid=1934484673249293&id=100000633379839
 
%%author_id lesev_igor
%%date 
 
%%tags babchenko_arkadii,politika,ukraina
%%title По Бабченко
 
%%endhead 
 
\subsection{По Бабченко}
\label{sec:30_05_2018.fb.lesev_igor.1.babchenko}
 
\Purl{https://www.facebook.com/permalink.php?story_fbid=1934484673249293&id=100000633379839}
\ifcmt
 author_begin
   author_id lesev_igor
 author_end
\fi

По Бабченко.

Человек был таким себе «ватником» наоборот. Свою страну называл «территорией».
Своих коллег-журналистов – пропагандистами. Войну на Донбассе считал
карательной. Со стороны России, естественно. А в Сирии, по мнению Бабченко,
русские летчики бомбят непременно детей и женщин. Не любил Бабченко-«ватник» и
свое общество которое считал инфантильным и жестоким. Именно такое общество с
«лишней хромосомой», по мнению Бабченко, виновно в кемеровском пожаре, унесшем
жизни десятков детей.

\ifcmt
  ig https://scontent-frt3-1.xx.fbcdn.net/v/t1.6435-9/33868988_1934484596582634_7356530236399288320_n.jpg?_nc_cat=106&ccb=1-5&_nc_sid=730e14&_nc_ohc=w0EU47WzHiQAX9i2fx9&_nc_ht=scontent-frt3-1.xx&oh=f9be1048b4390e130a4c12f5f8b3944f&oe=61B82DCD
  @width 0.4
  %@wrap \parpic[r]
  @wrap \InsertBoxR{0}
\fi

В общем, русский «ватник» Бабченко не любил современную ему Россию, точно также
как среднестатистический украинский «ватник» не любит современную ему Украину.
Ему было там не комфортно и он уехал. Сюда. И вот это было его ошибкой.

В Украине быть русофобом – это такой себе товар. Да, обладая российским
паспортом, ликвидность чуть возрастает. Но не особо. Русо-ненависть в наших
краях давно уже требует какой-то уникальности и специализации. У Бабченко ее не
было. Он был ярким в России, вызывая своими постами ненависть и рефлексию
окружающего его мира. Но переехав сюда, он даже фамилией не выделялся. И в
результате его ценность свелась к реквизитному пассиву, состоящему из двух
пунктов:

а) он умеренно-публичный русофоб и 

б) он обладатель российского паспорта.

Как это можно использовать в информационном пространстве, мы наблюдаем прямо
сейчас. Но волна шизофрении уже завтра-послезавтра спадет, хотя бы потому что
завтра-послезавтра в нашей чудесной стране произойдет что-то очередное
вызывающее и мерзкое. А по итогу мы останемся со следующим:

1. Убийцу Бабченко не найдут.

2. Убийцу Бабченко не найдут, потому что искать некому, а «если чо», заказчик и
так уже объявлен.

3. Если каким-то боком убийца Бабченко всплывет, как в случае с Вороненковым,
его все равно даже с украинским паспортом и меткой «участник АТО» объявят
кремлевским агентом.

Да, и последнее. Вчера в Деснянском суде отпустили на поруки нардепов трех
погромщиков на Лесном рынке. А вечером застрелили Бабченко. Связи, конечно же,
нет. Прямой. А непрямая – вот она. На поверхности. Мне почему-то кажется, что
Бабченко с огромной вероятностью встретил бы это утро живым, будь он не большим
или меньшим русофобом, а выбери он для проживания страну, в которой бандитов и
дебоширов не отпускали бы из зала суда под давлением других бандитов и
дебоширов. А так, RIP, конечно. Наверняка теперь в раю слушает хор Александрова
и обнимается с кемеровскими детишками.

\ii{30_05_2018.fb.lesev_igor.1.babchenko.cmt}
