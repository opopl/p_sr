% vim: keymap=russian-jcukenwin
%%beginhead 
 
%%file 25_12_2020.fb.bilchenko_evgenia.2.epos
%%parent 25_12_2020
 
%%url https://www.facebook.com/yevzhik/posts/3504235736278178
 
%%author Бильченко, Евгения
%%author_id bilchenko_evgenia
%%author_url 
 
%%tags bilchenko_evgenia,epos,poezia
%%title БЖ. Эпос
 
%%endhead 
 
\subsection{БЖ. Эпос}
\label{sec:25_12_2020.fb.bilchenko_evgenia.2.epos}
\Purl{https://www.facebook.com/yevzhik/posts/3504235736278178}
\ifcmt
 author_begin
   author_id bilchenko_evgenia
 author_end
\fi

БЖ. Эпос.

Если бы ты мне хотя бы раз, вот, за это время
Сказал, что есть хоть какая-то перспектива, я бы не психовала.
Меня перестала устраивать ситуация \enquote{между всеми}:
Ведь конец имеют и \enquote{Бхагавадгита}, и \enquote{Калевала}.

\ifcmt
  pic https://scontent-lga3-2.xx.fbcdn.net/v/t1.6435-9/133028295_3504235679611517_3494023366216246938_n.jpg?_nc_cat=102&ccb=1-3&_nc_sid=8bfeb9&_nc_ohc=qvFzpeQkTmEAX8OLJWG&tn=ntrKbsW_7ChXu3v-&_nc_ht=scontent-lga3-2.xx&oh=648b30a75c3b38b8e92831eecf1d131a&oe=60CC1EDE
	caption БЖ. Эпос.
\fi

Любая летопись, любой эпос имеют лето последнее
\enquote{Под} конкретный, до нашей эры, никем не известный год.
Только Лета - не измерима отрезками многолетними,
А здесь, на земле, умирает всё - умирает, пока живёт.

Я понимаю, что ты - не Карлсон, да и тот не давал гарантии
На Вечное Возвращение в форточку Малыша.
Не в Иудее, а каждый день к тайной вечере братия
Собирается, чтобы слушать, как плачет во тьме душа.

Всё повторяется. Всякий раз. Неужели же нет линейности?
Из глубины азиатских руд Сибирью растёт молчание.
Дом созидается постепенно, жимолостью взлелеянный.
Сквозь неё продираются очи окон, мудрые и печальные.

Дом отчаливает, в Ковчег обратившись, и возвращается.
Ты нарочито не обещаешь, но это уже не надо.
Если бы ты мне хотя бы раз сказал, что такое счастье,
Ничего бы не было... Перспектива отсутствия есть награда
Бесперспективности - неумолимой, умоляемой, что есть силы.
Бессмыслицы, возведённой в особую форму ранга.

Какая разница, что есть время, если небо и поле - синие?
Не имеющий рамок держится в собственных вечных рамках.

25 декабря 2020 г.

\emph{Андрей Ильичёв}

Поэты сдают экстерном экзамен на страшный суд,
Ответ будет признан верным, другого там и не ждут.
Но только поэта время не могут врачи отмерить,
Такое поэта бремя: писать и дышать, как верить.
Никто не дает гарантий, да разве они потребны,
Здесь нет ни чинов, ни мантий, а только шальное небо,
Цвета воды и стали. Ты просто ему поверь:
Тебя еще не позвали. Еще не открыта дверь.

\emph{Евгения Бильченко}

Андрей Ильичёв Очень здорово.

\emph{Olena Kalantarova}

Линейность есть в росте скорости движения по циклам....Это производная первого
порядка... Вы уже в ней...осталось ее оседлать.... ибо скорость изменений
скорости изменений есть прохождение алхимич очистки... так держать.... и
организм перезапустит все системы....
