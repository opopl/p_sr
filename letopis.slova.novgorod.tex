% vim: keymap=russian-jcukenwin
%%beginhead 
 
%%file slova.novgorod
%%parent slova
 
%%url 
 
%%author 
%%author_id 
%%author_url 
 
%%tags 
%%title 
 
%%endhead 
\chapter{Новгород}

%%%cit
%%%cit_pic
%%%cit_text
Родившись где-то в районе 942 года, Владимир стал \emph{князем Новгородским}, потому
что Киевом правили его старшие братья – попеременно. Когда Ярополк убил своего
брата Олега, в Киев явился с дружиной Владимир и убил Ярополка. Вот так
наследный Новгородский князь стал князем Киевским. Кстати, титул \emph{Новгородского
князя} остался при нем тоже. А потом он стал и каганом, то есть ханом, равным
будущему императору Чингисхану.  Вот так русский (рус, русич) князь-каган
захватил украинский Киев и крестил всех его языческих жителей. А если бы не
захватил?  Ну, тогда пришлось бы московитам впоследствии отвоевывать Киев у
хазар или половцев, а может даже у самого Чингисхана, имевшего на него свои
виды
%%%cit_title
\citTitle{Почему киевский хан Владимир Креститель считается украинским князем, когда он даже славянином не был?},
Исторический Понедельник, zen.yandex.ru, 05.01.2021 
%%%endcit

