% vim: keymap=russian-jcukenwin
%%beginhead 
 
%%file 08_12_2022.fb.storozhuk_mihail.1.bahmut_den_pershyj
%%parent 08_12_2022
 
%%url https://www.facebook.com/permalink.php?story_fbid=pfbid0t5QLkbFvg7F4ASBhNYywd3Q7awoy6zbi17iB7B238Eb1vkonjMMWmT7dTuGazaJol&id=100002121134537
 
%%author_id storozhuk_mihail
%%date 
 
%%tags 
%%title Бахмут. День перший
 
%%endhead 
 
\subsection{Бахмут. День перший}
\label{sec:08_12_2022.fb.storozhuk_mihail.1.bahmut_den_pershyj}
 
\Purl{https://www.facebook.com/permalink.php?story_fbid=pfbid0t5QLkbFvg7F4ASBhNYywd3Q7awoy6zbi17iB7B238Eb1vkonjMMWmT7dTuGazaJol&id=100002121134537}
\ifcmt
 author_begin
   author_id storozhuk_mihail
 author_end
\fi

Бахмут. День перший.

Розвантажилися на складі в Ізюмі. Частину корму залишили тут, але більшу
частину треба доставити в Бахмут.

Дорога до Бахмуту була дуже складною. Йшов льодяний дощ, дорога була вкрита
товстим шаром криги. Як відомо, тут дороги чистити немає кому. Хіба що пройде
гусенічна техніка, та проломить кригу. Зчеплення з дорогою немає зовсім. Більша
частина техніки - військові, інколи медики. Машини заносить... Всі допомагають
одне одному, витягають застряглі машини.  Цивільного транспорту немає. Навколо
- руїни, повністю знищені села, на узбіччі подекуди стоять розбиті машини та
танки. 

Першою точкою було забрати таксу в Нікіфоровці, це не доїжджаючи міста. Але вже
звідти чутно безперервну канонаду. Зупинилися, забрали таксу, одягнули
бронежилети, та потроху поїхали далі. 

Наступна зупинка - контужений, сліпий пес Рудий. Його знайшли військові і
просять забрати на лікування. 

Заїжджаємо в Бахмут.

Складність в орієнтуванні - максимальна. Навігатор не працює, зв'язку немає
ніякого. Адреси, де шукати тварин, знаходити непросто. Вулиці зовсім не
відповідають мапі.  Місто в руїнах. Висять обривки проводів. Подекуди так
низько, що під ними треба проїжджати, чепляючись за них дахом. Повсюди ями,
воронки від ворожих бомб. Людей на вулицях майже немає. Інколи, дуже рідко,
проїде військовий пікап, а в інший час, як у кіно. Спустошення, страшні порожні
вулиці, і над цим усім - безперервний гуркіт вибухів. Щось вибухає ближче, щось
далі. 

Приїхали на місце, де в будинку нас чекають 5 котів. Будинок знайти неможливо,
точка на навігаторі показує на вулицю. Шукаємо, гукаємо- нікого.

Раптом бачимо на вулиці одиноку постать. Це бабуся, яка кудись йде.
Зупиняємося, спілкуємося. Бабуся просить допомоги кормами для тварин, отже
їдемо з нею. 

Звичайний двір, звичайної багатоповерхівки. Але відчуття якоїсь нереальності,
ніби потрапив в кіно про постапокаліпсис. Тут залишається певна кількість
мешканців. Вікна жилих квартир наглухо забиті, назовні стирчать труби.  Перед
під'їздом група людей щось готують на пічці. 

Поряд, у судісньому під'їзді, пічка вже остигла, і на ній сидить кілька котів.
Теплих підвалів тут немає, тварини шукають тепло, де можуть.

З цього будинку ми забрали двох кішок і мале руде кошеня. Велика спокійна кішка
Маркіза, та невеличка сіренька. Їм потрібно шукати господарів вже зараз.

Після тривалого спілкування ми з'ясували, що бабуся знає той будинок, де в
покинутому притулку знаходяться 5 котів, і ми поїхали до них.

Ну, тут був звичайний відлов котів в будинку, нічого особливого. Окрім котів, в
притулку знайшли ще маленьку собаку, її теж забрали.

Почались сутінки, і треба було їхати назад. І так безлюдне місто вночі стає
вкрай небезпечним.

Потім був рух по слизькій темній дорозі, і ще одна раптова тварина: маленьке
кошеня, що звідкись опинилося на блокпості. Військові, дізнавшись що ми
зоорятувальники, дуже зраділи, бо мале було повністю мокре, змерле, і могло би
не пережити ніч.

Увага. Наша місія продовжиться ще 2 дні. Якщо у вас є адреси місць, де в
Бахмуті і околицях є тварини, що потребують негайної допомоги, харчування чи
евакуації, а також якщо хочете прихистити тварин з зони бойових дій - пишіть
мені. 

Телефонувати не має сенсу, зв'язок дуже поганий.
