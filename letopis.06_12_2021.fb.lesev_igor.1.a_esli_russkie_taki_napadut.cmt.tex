% vim: keymap=russian-jcukenwin
%%beginhead 
 
%%file 06_12_2021.fb.lesev_igor.1.a_esli_russkie_taki_napadut.cmt
%%parent 06_12_2021.fb.lesev_igor.1.a_esli_russkie_taki_napadut
 
%%url 
 
%%author_id 
%%date 
 
%%tags 
%%title 
 
%%endhead 
\subsubsection{Коментарі}

\begin{itemize} % {
\iusr{Oleg Voloshin}

Если честно, как-то очень путано вышло. И причём тут Крым? Никого он реально не
волнует. Тех, кто ненавидел Россию до 2014, года не удовлетворил бы его
возврат, а остальным плевать на самом деле.

И вот то, что под боком у великой державы уже 8 лет существует слабая страна,
единственной целью которой, судя по действиям ее властей, является нанесение
всяческого вреда этой великой державе, то вот и единственный логичный мотив
разрубить Гордиев узел. Военное освоение территории Украины - причина ударить
на упреждение. Никто не станет ждать, пока здесь реально появятся американские
и британские базы. Если гарантировать России отказ от таких планов, у Москвы
исчезнет окончательно и без того весьма слабое желание трогать это болото. Все
равно Украина обречена на угасание.

\begin{itemize} % {
\iusr{Людмила Левченко}
\textbf{Oleg Voloshin} 

А зачем отказываться? Чем они мешают \enquote{великой державе} с бедным народом?
Нападут? Не смешите мои тапки, сама Россия в это не верит!

\iusr{Oleg Voloshin}

\textbf{Людмила Левченко} 

тянут сюда военных НАТО, все время грозят начать наступление на ДНР,
провоцируют учения возле Крыма, разместили здесь всю возможную разведывательную
инфраструктуру США и Британии и тд. Этого мало? В России угрозу использования
территории Украины для создания давления на РФ рассматривают более чем
серьезно. Поставить ракеты с подлётным временем до Кремля 8 мин и потом можно
прямо диктовать России условия. Военные мыслят не намерениями, а потенциалом.
Если Вы не враги, то зачем тащите под наши границы свою военную инфраструктуру?
Вот их вопрос американцам. Представьте российские ракеты и базы на Кубе. Уже
ведь был Карибский кризис, который едва не завершился ядерной войной.

\end{itemize} % }

\iusr{Алексей Корецкий}

Пока есть реальная надежда и перспектива развалить Украину изнутри, используя
любые способы внешнего и внутреннего давления, никакой "большой войны" не
будет. Это намного дешевле, а главное в разы менее рискованно и предсказуемо,
чем масштабные военные действия.

Для этого нужно элементарно понимать хотя-бы основные цели РФ. А то не
экономические или территориальные интересы, а прежде всего контроль и влияние,
война за идентичность.

Плюс любая крупная военная операция всегда рискует затянуться на многие месяцы
или годы, став вместо блицкрига вялотекущим перманентным конфликтом размазанным
во времени и по территории. Чему гарантировано будут всячески способствовать
другие внешние интересанты. Поскольку любые проблемы гуманитарного и
экономического характера, как и военные потери и соответственно ослабление РФ,
тоже выгодны для целого ряда 3-х государств. Начиная от непосредственно соседей
вроде Китая или Турции и далее по списку ...

\begin{itemize} % {
\iusr{Владимир Очеретный}
\textbf{Алексей Корецкий} 

России в принципе не нужен развал Украины. Другое дело, что в нынешних границах
она неминуемо превращается в Бандеростан - кто бы ни приходил к власти. Думаю,
РФ было бы выгодно, если бы Польша, Румыния и Венгрия забрали те территории,
которые считают исторически своими. Тогда из остатков можно было бы делать,
если не пророссийскую, то хотя бы не враждебную Украину.

\end{itemize} % }

\iusr{Arina Levina}

Если Россия перекроет газ в ЕС, то примерно 40\% европейской индустрии и
домохозяйств ожидает коллапс. Замены нет и в ближайшем будущем не будет.
Разразится кризис какого ещё не было в истории Европы

\begin{itemize} % {
\iusr{Алексей Корецкий}
\textbf{Arina Levina}

Вы хоть представляете о чем вы пишите, понимаете глобальность проблем которые
хотите охватить?

Масштабный военный конфликт между РФ и Украиной это сотни тысяч комбатантов и
резервистов ввлечённые в боевые действия, сотни, а возможно тысячи километров
линии боевых действий. Массированные удары артиллерии, РСЗО, авиации по
множеству целей ... Сотни тысяч, как минимум беженцев в сопредельные Украине
страны Европы ... И не пойми какие ещё гуманитарные последствия масштабов 35
миллионный страны. Плюс экономический коллапс и падение на многих мировых
финансовых площадках.

Тут газ для европейских домхозяйств это даже не номер 5 проблема, с которыми
Европе придётся столкнуться.


\iusr{Ольга Циброва}
\textbf{Arina Levina} 

У 2009-му вже пробували перекрить, але коли Тимошенко обрисувала Путіну
масштаби техногенної катастрофи, яка накриє не лише Україну, а ще і Європу з
Росією, то негайно були відкриті вентилі і Путін пішов на перемовини.

\end{itemize} % }


\end{itemize} % }
