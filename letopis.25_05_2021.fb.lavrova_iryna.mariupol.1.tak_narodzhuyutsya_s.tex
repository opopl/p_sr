%%beginhead 
 
%%file 25_05_2021.fb.lavrova_iryna.mariupol.1.tak_narodzhuyutsya_s
%%parent 25_05_2021
 
%%url https://www.facebook.com/permalink.php?story_fbid=pfbid0Piy4PfA8iXbUeXgaSqNzBLunN8UayZgPLonYwgXgJxSqdgSfxyXChbzXZRtzzJXyl&id=100006182827725
 
%%author_id lavrova_iryna.mariupol
%%date 25_05_2021
 
%%tags mariupol,istoria,mariupol.pre_war,igra,shkola,obrazovanie
%%title ТАК НАРОДЖУЮТЬСЯ ШЕДЕВРИ - Вчора ми тестували гру "АРТ-ІСТОРІЯ" серед учнів 9-х класів Маріупольського міського технологічного ліцею
 
%%endhead 

\subsection{ТАК НАРОДЖУЮТЬСЯ ШЕДЕВРИ - Вчора ми тестували гру \enquote{АРТ-ІСТОРІЯ} серед учнів 9-х класів Маріупольського міського технологічного ліцею}
\label{sec:25_05_2021.fb.lavrova_iryna.mariupol.1.tak_narodzhuyutsya_s}

\Purl{https://www.facebook.com/permalink.php?story_fbid=pfbid0Piy4PfA8iXbUeXgaSqNzBLunN8UayZgPLonYwgXgJxSqdgSfxyXChbzXZRtzzJXyl&id=100006182827725}
\ifcmt
 author_begin
   author_id lavrova_iryna.mariupol
 author_end
\fi

ТАК НАРОДЖУЮТЬСЯ ШЕДЕВРИ

Вчора ми тестували гру \enquote{АРТ-ІСТОРІЯ} серед учнів 9-х класів Маріупольського
міського технологічного ліцею. 

Приємно відмітили, що учні добре знайомі з технологією синквейн, яку часто
використовують на заняттях з української мови та літератури. Вони добре знають,
як складаються ці особливі вірші з 5 рядків, які не мають традиційної для вірша
рими, але несуть важливе смислове навантаження.

Володіючи навичками складання сенкан-віршів, учасники гри легко справлялися з
інтерактивними вправами з історії української культури. Вони надавали власні
характеристики пам'яткам мистецтва ХІХ-ХХ ст., які вони вивчали протягом 9
класу. І ці новостворені вірші вражали своєю влучністю висловів, яскравістю
образів, чітким усвідомленням сенсів. 

Саме так народжуються справжні шедеври. Коли ми пропонуємо учням певну форму
вираження думок, а також можливість вільно оперувати інформацією та трохи часу
на обміркування, вони дивують нас своєю креативністю.  

Не обійшлося без несподіванки. Заплановану вправу із сенкан-віршами учасники
гри неочікувано доповнили власним внеском. На намальовану карту вони не тільки
приклеювали \enquote{віршовані} стікери з пам'ятками мистецтва, а й стали підписувати
міста, де ці пам'ятки розташовуються! Так старанням гравців на порожній карті
з'явилися Київ, Одеса, Черкаси та Запоріжжя.

Цікаво було спостерігати, як учні, самі того не помічаючи, змінювали гру.
Вдосконалили її активностями, які здавалися їм природніми.

А ми отримували приємні інсайти. 

Від нових правил, які створювали гравці. 

Від віршів-шедеврів, які звучали різними голосами. 

Від таких важливих для нас відгуків учасників гри.

Друзі, а у вас бувало, що ви планували одне, а у взаємодії з іншими людьми
виходило дещо інше і значно добріше? 

Якщо не шкода, поділитеся знахідками?

Навчальна гра \enquote{АРТ-ІСТОРІЯ} розроблена нами – студентами та викладачами.

За підтримки:

ГО \enquote{Майстерня думок}

Кафедра філософських наук та історії України ДВНЗ \enquote{Приазовський державний
технічний університет}

Центр інноваційного підприємництва ДВНЗ \enquote{ПДТУ}

Стартап-школа Sikorsky Challenge

Фото \href{https://www.facebook.com/kosharnyi.andrei}{Андрей Кошарный}

Дякуємо адміністрації Маріупольського міського технологічного ліцею за
підтримку, і найтепліша подяка вчительці історії Резиновій Марині Олексіївні та
учасникам гри.

\#артісторія

\#майстернядумок

\#ПДТУ

\#КафедраФНтаІУ

%\ii{25_05_2021.fb.lavrova_iryna.mariupol.1.tak_narodzhuyutsya_s.cmt}
