% vim: keymap=russian-jcukenwin
%%beginhead 
 
%%file 03_03_2022.fb.molchanov_jurij.1.monitor_smi_rf
%%parent 03_03_2022
 
%%url https://www.facebook.com/george.molchanov.9/posts/4737241479722116
 
%%author_id molchanov_jurij
%%date 
 
%%tags 
%%title Промониторил СМИ РФ. Возмущаются. Почему-то никто не встречает, никто не рад
 
%%endhead 
 
\subsection{Промониторил СМИ РФ. Возмущаются. Почему-то никто не встречает, никто не рад}
\label{sec:03_03_2022.fb.molchanov_jurij.1.monitor_smi_rf}
 
\Purl{https://www.facebook.com/george.molchanov.9/posts/4737241479722116}
\ifcmt
 author_begin
   author_id molchanov_jurij
 author_end
\fi

Промониторил СМИ РФ. Возмущаются. Почему-то никто не встречает, никто не рад.
Ни русскоязычный, ни православный, ни стар, ни млад. А фраза про «русский
корабль» за 15 минут репостов стала железобетонным слоганом страны, скрепившим
запад, север, юг и восток. Давно нас так ничего не объединяло.

Не буду сейчас про очевидный угар пропаганды. Там даже на вопрос о развязанных
шнурках будут молоть про «а где вы 8 лет были».

Мне интересна химия мышления тех, кто кивнул гривой в знак одобрения войны.

Глядя на серию последних интервью наших «политэмигрантов» из клуба им.
януковича, смею предположить, что если не основную, то значительную смысловую
часть этого безумного апокалипсиса писали именно они.

Безнадежно застрявшая в 2013-м и движимая реваншем, эта рухлядь решила, что
паттерн восприятия РФ во время войны будет таким же, как и был во время
майдана. Все чертежи разделения по гуманитарным и политическим матрицам будут
такими же актуальными, как и восемь лет назад. Не исключено, что файл с этими
бредовыми фантазиями и был подшит к общему военному пакету.

Но война, это не революция. С первых секунд атаки на Украину 24.02.22, все
внутривидовые разделения и склоки в Украине исчезли напрочь. Даже в фб были
разбанены все те, с кем спорили о президентах, партиях, религиях и языках.
Желто-голубой флаг на паспорте стал фактически логотипом единства в трагедии,
общей для абсолютно всех.

Не буду говорить о подъеме людей. Об этом уже миллионом слов сказано. Пережив
стремительный шок, все сообразили, что напали на наш ДОМ. Не на партии, не на
политические доктрины, а ворвались банально к тебе в дом. И естественно, что
сегодня в Украине даже мыши пищат про российский корабль и абсолютно все готовы
защищать свое право на жизнь. Свои храмы, свою свободу, своих родных.

Вижу, как некоторые, оплывшие жиром «эксперты» уже призывают к «дегуманизации
спецоперации» и крыть украинцев ковровыми бомбардировками по югославскому
образцу.

Что ж, потенциал у вас есть. Но помните, что также, как и 20 миллионов погибших
в 1945-м, мы, даже погибнув, останемся победителями.

Потому что при всей пелене вашей пропаганды, при всем северо-корейском задоре,
мы у себя дома.

А вы – оккупант. Не превращайте путинско-украинскую войну в войну
российско-украинскую. 

Этот грех точно не смоется никогда...

\ii{03_03_2022.fb.molchanov_jurij.1.monitor_smi_rf.cmt}
