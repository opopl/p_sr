% vim: keymap=russian-jcukenwin
%%beginhead 
 
%%file 07_11_2021.fb.zodd_zverev.1.humanity_kto_vazhnee
%%parent 07_11_2021
 
%%url https://www.facebook.com/zodd.zverev/posts/2400243220107544
 
%%author_id zodd_zverev
%%date 
 
%%tags chelovek,humanity,nauka
%%title Кто важнее для истории Человечества?
 
%%endhead 
 
\subsection{Кто важнее для истории Человечества?}
\label{sec:07_11_2021.fb.zodd_zverev.1.humanity_kto_vazhnee}
 
\Purl{https://www.facebook.com/zodd.zverev/posts/2400243220107544}
\ifcmt
 author_begin
   author_id zodd_zverev
 author_end
\fi

То, что повестка нашего мира формируется клоунами на пососе у истинных
тридваразов, как мне кажется, ни для кого и не секрет.

Но я бы хотел заострить внимание на одной хуерге, которая лично меня приводит к
натуральному пуканическому кипению и даже сублимации. А именно, к истории.

\ifcmt
  ig https://scontent-frx5-1.xx.fbcdn.net/v/t39.30808-6/253291137_2400237766774756_942457586042500342_n.jpg?_nc_cat=105&ccb=1-5&_nc_sid=8bfeb9&_nc_ohc=QCnY9Z5m3FIAX8Qx86P&_nc_ht=scontent-frx5-1.xx&oh=840a59671a8852ce6bc1d1eb5b117650&oe=61AE3517
  @width 0.4
  %@wrap \parpic[r]
  @wrap \InsertBoxR{0}
\fi

История мира ныне крайне популярна, причём как в её классическом изводе, так и
в различных альтернативных измышлениях разной степени паршивости. Оно и
немудрено, любой человек, не поспевающий за трендами, подспудно ощущает себя
говном бытия, и испытывая всем нам известный из учения о Пламени поколенческий
диссонанс, ищет некую опору для своей жизненной позиции. И эту опору ему
услужливо продают мифологизаторы и историки.

Да, ты говно. Ну вот что теперь поделать? Так вышло, но ничего страшного! 

По аналогии с великими правителями, что создавали великие империи, а потом
передавали власть в них по праву рождения своим порой реально убогим потомкам,
ты бляць — ты-ты сука — хоть и говно сам по себе, но твои предки были великими
и завещали тебе святой мандат. Ваще похер какими они были, они уже умерли, их
не расспросить, но ТЫ бляць наследник яибал каких мощщщных дидов, которые
отважно воевали, были ариями и выкопали Чёрное море первыми колесницами.

Не, того факта, что ты говно, это вовсе не отменяет. Но раз позорным потомкам
великих царей поклонялись народы по праву рождения, то хуле? И тебе в целом
должны бы, но чот не поклоняются. Не поклоняются ведь, нет? Чот там делают и
идут вперёд, а ты хрен без соли сосешь на обочине? Это несправедливость, но
правда, ИСТИНА БЛЯЦЬ, мы-то знаем, она на твоей стороне!

Все эти безродные мудаки просто сдохнут, а ты — в рай. А как ты думал? Конечно
в рай, к своим великим, славным и древним предкам. Они призывают тебя место
твоё занять в чертогах Вальгаллы, где вечно живут… храбрецы! 

(Я специально использовал эту цитату, ибо она мне по нраву тоже. Я, как и любое
говно, падок до подобных искажений).

Собственно, наше образование дохуя шибко базируется на истории, нам постоянно
пропихивают некую ретроспективу, дабы мы в динамике развития сущности обрели
знание о её природе. И ничего худого в том подходе нет, кстати. Годный в целом
подход во многих науках.

Однако, пролистнём же учебник истории, просто истории. Заметьте, что есть
история литературы, история живописи, история музыки, история экономических
учений и вот бляць просто история. Как бы базовая от них всех. Пролистнём
учебник вот этой самой базовой истории. Чему вообще он нас должен бы научить?
Пожалуй, истории нашей цивилизации, верно?

Но каковы же в нем основные реперные точки и что составляет его, так сказать,
скелет? А? 

Битвы и войны!

О чем мы постоянно читаем в любом чтиве про историю мира? О том, как какие-то
папки (вожди, ханы, князья, императоры, короли, лорды) посрались и начали друг
друга методично пиздить с переменным порой успхом. Причём! Что характерно,
лорды не стали пиздиться сами, да нахуя? Нет, они послали хуярить другу друга
именно что наших предков, ведь мало кто из нас — потомок лорда. В основном мы с
вами потомки всякого посошного крестьянского говна чисто статистически, и не
надо там себе пиздеть. 

Даже если ваш некий предок голубых кровей, то его предок был сраным
крестьянином с грязью под ногтями, который некими трюками во служении лорду
смог из грязи подняться в князи. А вы, кстати, эти достижения проебали, раз
читаете этот текст.

Ну да похуй, что уж теперь. То есть, учебник истории — он не про вас. Он про
государства и про тех, кто прямо и логично, либо случайно и внезапно получил то
самое наследие лордов, что посылали ваших предков хуярить друг друга до смерти
по всяким своим надуманным причинам.

Отсюда вопрос: а вы тут причём?

Почему вы считаете это своей историей? А конкретней, почему вы считаете
перечень битв и завоеваний каких-то там древних и не очень древних монархов
тем, что определяет вашу нынешнюю жизнь?

А что же определяет вашу нынешнюю жизнь? 

Епта! НАУКА! НАУКА, епта!

Благодаря науке и технологиям, которые из неё проистекают, каждый из вас ныне
живёт не хуже любого исторического императора. Да, у вас нет рабов, но на вас
пашут тысячи и тысячи других людей, и делают это пиздец как эффективно,
благодаря науке.

Вы хотите жрать — вам несут еду. Вы хотите срать — ваше говно смывают. Вы
хотите пить — бляць на кухне кран. И в нем вода много чище, чем в колодце
вашего деда, например. Вам надо ехать — вас везут. У вас болит жопа — вас
лечат. Вам скучно — вас развлекают. Вам хочется ебаться — вас знакомят с
вожделеющими щелями.

Хуле вам не так? 

Для любого вашего предка — это прям предел мечтаний и даже нечто невообразимое.
Людовик Шестнадцатый не жил в таком комфорте и удобстве, как вы живете. Да, у
него золота было побольше, но хуле вам это золото? А вот мог себе Людовик через
«Самокат» заказать кило манго, например? Аааа, хуй, да? Тоже мне бляць король
Франции, лошок.

Само ваше существование в этом мире обусловлено наукой. Если бы люди не
научились нормально принимать роды и не развили бы вакцинацию и педиатрию, вас
бы вообще нихуя не было! Вы бы или не родились толком вообще, или подохли в
первые годы жизни. Посмотрите статистику детской смертности, вы охуенные
счастливчики, что родились в 20-м веке, а не бляць даже в 19-м ещё.

И вот, с вашей точки зрения, нынешний учебник истории — это полная хуета,
которая вас фактически и не касается. Из него можно лишь узнать, какая тварь
погнала вашего предка умирать за свои интересы. А вам-то хуле с того?

Нет, практически для всех людей на свете намного важнее знать не историю битв и
передела границ, а историю науки и технологий. Учебник истории должен
рассказывать нам не о королях и князьях, а об учёных и внедрении изобретений.
Научные теории и проистекающие из них реальные применения — вот что бляць
реально повлияло на вашу жизнь и даже вообще — вам эту жизнь подарило!

Не какая-то там конкретная победа в конкретной приснопамятной войне, а
изобретения — целая цепочка разных изобретений. И развитие философских и
научных взглядов, которые привели к этим изобретениям. Иначе бы вас тупо не
было, посмотрите на график роста населения Земли. Он ебанул в экспоненту из-за
каких-то там войн, штоле? Да нет, сукабляць, из-за технологий.

Влияние Второй Мировой на вашу нынешнюю жизнь, вот беспесды, много меньше, чем
влияние теории относительности и квантовой механики. Но я не видел, чтоб вы
ходили парадами с портретами Эйнштейна, Минковского, Пуанкаре, Резерфорда,
Томсона, Гильберта, Зеемана, Зоммерфедьда, Бора, Планка, Борна, Шрёдингера,
Дирака, Хайзенберга, Римана, Эрмита, Паули, Белла…

Мой друг Станислав Хабибов сказал мне: «Бляяяяяя, так Белл, который придумал
неравенство Белла, он что, умер в 1990 году???»

Да, в 1990. Мало кто знает. Зато все знают, что Гитлер умер в 1945, а Сталин в
1953.

А ведь что, собственно, может узнать Станислав из учебника истории сейчас? Что
Гитлер и Сталин не убили его предков.

А что он там НЕ прочтёт? Что Белл в частности подарил ему то качество жизни,
которым он живёт сейчас.

Так и я хочу спросить вас ныне перед лицом Пламени, мира и людей: кто важнее
для истории Человечества, какой-то там ебаный Гитлер или умница Белл?

Зачем мы учим детей говну и сами едим говно, если нас определяет как личность
совсем иное? Совсем иное.

Моё радикальное требование: учебники истории должны быть переписаны в виде
учебников истории науки, а не хроники массовых убийств во имя хуй пойми чего. 

ПС Слово «говно» намеренно использовано в тексте 10 раз как рефрен, для
создания определённого настроения у читателя.
