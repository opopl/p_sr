% vim: keymap=russian-jcukenwin
%%beginhead 
 
%%file 17_10_2021.fb.fb_group.story_kiev_ua.2.aleksandr_polischuk
%%parent 17_10_2021
 
%%url https://www.facebook.com/groups/story.kiev.ua/posts/1776969529166507/
 
%%author_id fb_group.story_kiev_ua
%%date 
 
%%tags kiev,polischuk_aleksandr.kiev.kollekcioner,smert
%%title 15 октября перестало биться сердце Великого коллекционера и Патриота Александра Полищука
 
%%endhead 
 
\subsection{15 октября перестало биться сердце Великого коллекционера и Патриота Александра Полищука}
\label{sec:17_10_2021.fb.fb_group.story_kiev_ua.2.aleksandr_polischuk}
 
\Purl{https://www.facebook.com/groups/story.kiev.ua/posts/1776969529166507/}
\ifcmt
 author_begin
   author_id fb_group.story_kiev_ua
 author_end
\fi

15 октября перестало биться сердце Великого коллекционера и Патриота Александра
Полищука. Это он создал лучший и научно разработанный Музей Трипольской
культуры, который стал одним из научных центров современной Украины. К большому
сожалению, он сейчас не существует! Скорее всего его разгром и сказался пагубно
на сердце моего Друга! Не буду сейчас разбираться - почему и как это произошло,
вернусь к этому позже… Этот музей описан в ряде моих книжек… 

\ifcmt
  ig https://scontent-lga3-2.xx.fbcdn.net/v/t1.6435-9/246505483_4562338463843684_3038345193529107149_n.jpg?_nc_cat=107&ccb=1-5&_nc_sid=825194&_nc_ohc=fpZTBref6LwAX-hWq5t&_nc_ht=scontent-lga3-2.xx&oh=bfb7076607afffa2ee30b66161361844&oe=61945027
  @width 0.4
  %@wrap \parpic[r]
  @wrap \InsertBoxR{0}
\fi

Сложно у нас, на
Украине с музеями. Много лет я борюсь, чтобы создать музей «Державотворення та
Незалежність України», провел более двадцати выставок на основе своего
собрания, а про музей пока только мечтаю. Моё собрание про Мазепу и его эпоху
после выставки в Льончопинге (Швеция) и в Украинском музее (Нью-Йорк), только
сейчас демонстрируется в Национальном военно-историческом музее. Полищуку
повезло создать свой Музей в Триполье, благодаря Президенту Виктору Ющенко. А,
коллекционная деятельность Саши подробно освещена в моей книге «Записки
собирателя». На деньги А.С. Полищука издано несколько фундаментальных
исследований по археологии и подробный каталог «Українська листівка».
Многолетняя дружба с Полищуком позволила нам побывать во многих городах Европы,
где мы выискивали украинские артефакты и привозили их на Родину, чтобы показать
их на выставках и в музеях. Сейчас Саши нет, а я – инвалид…

На фото: Геннадий Киркевич, Александр Полищук, Виктор Киркевич, Президент
Виктор Ющенко, Анатолий Гайдамака.
