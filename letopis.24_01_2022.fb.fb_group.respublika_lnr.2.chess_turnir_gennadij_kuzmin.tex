% vim: keymap=russian-jcukenwin
%%beginhead 
 
%%file 24_01_2022.fb.fb_group.respublika_lnr.2.chess_turnir_gennadij_kuzmin
%%parent 24_01_2022
 
%%url https://www.facebook.com/groups/respublikalnr/posts/960335861268872
 
%%author_id fb_group.respublika_lnr,zimina_olesja
%%date 
 
%%tags chess,lnr,lugansk,sorevnovanie,sport,turnir
%%title Шахматные турниры памяти Геннадия Кузьмина состоялись в Луганске
 
%%endhead 
 
\subsection{Шахматные турниры памяти Геннадия Кузьмина состоялись в Луганске}
\label{sec:24_01_2022.fb.fb_group.respublika_lnr.2.chess_turnir_gennadij_kuzmin}
 
\Purl{https://www.facebook.com/groups/respublikalnr/posts/960335861268872}
\ifcmt
 author_begin
   author_id fb_group.respublika_lnr,zimina_olesja
 author_end
\fi

НОВОСТИ ГОРОДОВ ЛНР - ЛУГАНСК. Шахматные турниры памяти Геннадия Кузьмина
состоялись в Луганске

Шахматные турниры, посвященные памяти выдающегося советского шахматиста,
победителя шахматной олимпиады в Ницце 1974 года в составе сборной СССР
Геннадия Кузьмина, состоялись в столичной комплексной детско-юношеской
спортивной школе (КДЮСШ) № 4. Об этом сообщили в управлении образования
Администрации Луганска.

\ii{24_01_2022.fb.fb_group.respublika_lnr.2.chess_turnir_gennadij_kuzmin.pic.1}

«Турнир среди спортсменов первого разряда и взрослых, в котором участвовали 30
человек из городов Краснодон, Зимогорье, Родаково, Суходольск, Антрацит,
Луганск, проходил в 9 туров по швейцарской системе», – говорится в сообщении.

В управлении уточнили, что среди мужчин бронзовым призером стал Георгий
Ткаченко, серебряным – Егор Демин, а первое место получил Даниил Скрипник.
Среди женщин на третьем месте – Анна Лунина, второй оказалась Марья Попова,
победу одержала Виолетта Литвак.

Среди юношей первого разряда третье место завоевал Виктор Филонов, на втором –
Алексей Лицоев, обладатель золотой медали – Роман Григорьев. Среди девушек
первого разряда победу одержала Виктория Танана.

«Кроме того, в КДЮСШ прошел шахматный турнир среди 27 спортсменов третьего и
второго разрядов, представляющих Суходольск и Луганск. Он также проходил в 9
туров по швейцарской системе», – отметили в профильном структурном
подразделении столичной Администрации.

Бронзовым призером состязаний, как проинформировали в управлении, стал Герман
Золкин, второе место досталось Глебу Гончаренко, победителем турнира стал
Владислав Бибик. Третье место среди девушек завоевала Елизавета Остапенко,
Ольга Серебрянская стала серебряным призером, первое место заняла Екатерина
Полякова.

Пресс-служба Администрации города Луганска
