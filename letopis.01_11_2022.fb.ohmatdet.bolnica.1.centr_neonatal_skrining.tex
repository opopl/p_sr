% vim: keymap=russian-jcukenwin
%%beginhead 
 
%%file 01_11_2022.fb.ohmatdet.bolnica.1.centr_neonatal_skrining
%%parent 01_11_2022
 
%%url https://www.facebook.com/ndslohmatdyt/posts/pfbid02EwDs26rq12dE2ZZVuiAixE2BM7V3dGRqiKUa9UcyfAA7Hd53umeYU2Nn8f4xQuNnl
 
%%author_id ohmatdet.bolnica
%%date 
 
%%tags 
%%title В Охматдиті запустили новий Центр неонатального скринінгу
 
%%endhead 
 
\subsection{В Охматдиті запустили новий Центр неонатального скринінгу}
\label{sec:01_11_2022.fb.ohmatdet.bolnica.1.centr_neonatal_skrining}
 
\Purl{https://www.facebook.com/ndslohmatdyt/posts/pfbid02EwDs26rq12dE2ZZVuiAixE2BM7V3dGRqiKUa9UcyfAA7Hd53umeYU2Nn8f4xQuNnl}
\ifcmt
 author_begin
   author_id ohmatdet.bolnica
 author_end
\fi

⚡️В Охматдиті запустили новий Центр неонатального скринінгу: як працюватиме
високотехнологічна лабораторія ранньої діагностики рідкісних генетичних
захворювань⚡️

🔻В Україні офіційно відкрили новий Центр неонатального скринінгу. Його мета –
ефективна рання діагностика рідкісних захворювань у дітей з перших днів життя.

🔻Неонатальний скринінг — це комплексне обстеження новонароджених для виявлення
спадкових та вроджених захворювань. Забір крові проводять в пологових у перші
три дні життя немовляти та передають на дослідження до Центру неонатального
скринінгу.

🔻Дотепер в Україні усі новонароджені безоплатно перевірялися на 4 спадкові
хвороби: гіпотиреоз, фенілкетонурія, адреногенітальний синдром, муковісцидоз. 

Тепер діагностику розширили до 21 орфанного захворювання, серед яких  -
спінальна м’язова атрофія, галактоземія, фенілкетонурія та ін. Більшість
рідкісних хвороб мають бути виявлені протягом перших 10 днів життя, адже це
забезпечує ефективне і швидке лікування. Завдяки обладнанню нового центру це
стало можливим. 

🔻Проєкт запрацює в рамках пілоту в 12 регіонах України на базі двох
регіональних центрів неонатального скринінгу.

Пілотний запуск стартує в місті Києві, Вінницькій, Волинській, Житомирській,
Закарпатській, Івано-Франківській, Київській, Львівській, Рівненській,
Тернопільській, Хмельницькій та Чернівецькій областях. Запуск здійснено за
ініціативи Президента України та підтримки Міністерства охорони здоров’я
України. 

🔻Сьогодні Міністр охорони здоров'я Віктор Ляшко спільно з заступницею
керівника Офісу Президента України Юлією Соколовською, головою Комітету
Верховної Ради України з питань здоров’я нації, медичної допомоги та медичного
страхування Михайлом Радуцьким, представниками інших органів влади, міжнародних
та громадських організацій відвідали з робочим візитом новий Центр
неонатального скринінгу Національної дитячої спеціалізованої лікарні
«Охматдит».

🔻«Цьогоріч ми зробили ще один важливий крок на шляху до зміцнення здоров'я
наших дітей. Адже своєчасний скринінг — це не просто діагностика, а можливість
ще на доклінічній стадії завчасно виявити хворобу, застосувати корекцію і тим
самим забезпечити дитині повноцінне життя. І, сподіваюся, що це лише початок.
Невдовзі також плануємо відкрити нові регіональні центри скринінгу в Харкові та
Кривому Розі. А наступним етапом — сформувати мережу центрів орфанних
захворювань та забезпечити повноцінний доступ до лікування малюкам», — зазначив
Віктор Ляшко.

🔻«У ЄС рання діагностика орфанних захворювань проводиться за 1-3 дні, а в
Україні ця процедура могла розтягуватися на 2-3 тижні. За останній рік ми
провели величезну роботу заради того, аби не втрачати цей час, адже він
буквально врятує сотні життів», — прокоментувала Наталя Ольхович, завідувачка
лабораторії медичної генетики НДСЛ «Охматдит».

🔻Зразки крові на скринінг почали надходити до Центру неонатального скринінгу
НДСЛ «Охматдит» з 19 жовтня. На сьогодні у Центрі вже обстежено 1040
новонароджених. В ході досліджень у однієї дитини у віці 11 днів було виявлено
імунодефіцит. Лікарями проведено диференційну діагностику та визначено подальшу
тактику лікування. 

🔻Розширення програми неонатального скринінгу та цифровізація процесів
дозволить своєчасно виявити ризики орфанних захворювань у немовляти та
якнайшвидше запобігти їх клінічним проявам. Адже своєчасно виявлене та вчасно
розпочате лікування дозволяє запобігти розвитку хвороби та створює умови для
тривалого та повноцінного життя пацієнтів. 

🔻 Завдяки програмі Володимира Зеленського, яку реалізували МОЗ та НДСЛ
«Охматдит», було придбано обладнання для чотирьох регіональних та експертного
центрів неонатального скринінгу. А завдяки меценатській підтримці
fintech-компанії bill\_line, компанії Метінвест, гіпермаркету підлоги FLOOR,
Всеукраїнських благодійних фондів \enquote{Крона} та \enquote{Орфанні Синиці}  було проведено
капітальний ремонт корпусу, в якому буде розміщений Центр неонатального
скринінгу НДСЛ «Охматдит».

Офіс Президента МОЗ України МОЗ України

\ii{01_11_2022.fb.ohmatdet.bolnica.1.centr_neonatal_skrining.orig}
