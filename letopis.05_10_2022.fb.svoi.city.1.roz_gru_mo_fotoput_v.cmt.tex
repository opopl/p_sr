% vim: keymap=russian-jcukenwin
%%beginhead 
 
%%file 05_10_2022.fb.svoi.city.1.roz_gru_mo_fotoput_v.cmt
%%parent 05_10_2022.fb.svoi.city.1.roz_gru_mo_fotoput_v
 
%%url 
 
%%author_id 
%%date 
 
%%tags 
%%title 
 
%%endhead 

\qqSecCmt

\iusr{Свої}

Чим вас надихав мирний український Маріуполь?

\iusr{Свої}

Вітаємо \textbf{Анна Міхньова} 💛 Рандомайзер обрав ваш коментар, напишіть нам дані для
відправки, щоб отримати фотопутівник про ваш рідний Маріуполь

\ifcmt
  igc https://scontent-fra3-1.xx.fbcdn.net/v/t39.30808-6/310975847_474707304681559_7109352372534707142_n.jpg?_nc_cat=103&ccb=1-7&_nc_sid=dbeb18&_nc_ohc=UiKUoFRbaF4AX-8iNCy&_nc_ht=scontent-fra3-1.xx&oh=00_AfDy-JWyShLtWKI0x53uZTTiF8NlzheJHH-hvJKMXGeWkQ&oe=63F8BD40
	@width 0.6
\fi

\iusr{Alla Hyrenko}

Чим надихав Маріуполь? - Тим, що саме він - дім. Надихав самим собою. Своєю
суворістю, кремезністю, але і легкістю в той час. Розмаїтістю. Правильністю і
трохи дивакуватістю. Вмінням бути гарним не дивлячись на лусочки та недоліки.
Гармонією в поєднанні заводів і парків. Кабаном і великими рибами в музеї,
картинами в галереї, вітражами у філармонії, купою відвіданих концертів у БК
\enquote{Український Дім} та Драмтеатрі, квітами в кожному дворі, квітами в бабусь на
вулицях, нагодованими та поважними котами у подвір'ях, зайнятими грою веселими
дітьми на майданчиках, новими місцями з цікавинками, фестивалями, вуличними
кухнями і ходами, бібліотеками. Людьми, що вміли працювати і святкувати,
співчувати і допомагати, любили життя, мали принципи і цінності, хотіли жити в
Україні - такі надихали. Ну слухайте! Морем! Морським повітрям!Узбережжям - з
дому дійти ледь не пішки в будь-який день року, - сонце, зливу, сніг, ожеледь
чи не все одно? Знаєте, оце сидіти на \enquote{Піщанці}, пити каву, і дивитись як
кораблі виходять з порту... Кавовими місцями, \enquote{напитими} роками. Тобі там могло
бути важко, часом ніяково, часом хотілося кудись. Але якщо подумати, то
\enquote{кудись} не на все життя. Бо Маріуполь-дім. Часом було важко дихати, але тепер
дихати ще важче без нього. Хочу додому! Ніщо так не надихає...

\iusr{Iryna Hodulian}

людьми, які трималися своїх поглядів не дивлячись ні на що

\iusr{Ольга Пуніна}

Унікальними мозаїками українських монументалістів, зробленими у 60-х роках,
Аллою Горською, Віктором Зарецьким, Галиною Зубченко та іншими.

\iusr{Kateryna Kaliukh}

Теплим ласкавим морем, азовською креветкою і таранкою. Домом, що був як рідний,
вечірніми посиденьками

\iusr{Танюша Никитина}

ми жили в улюбленому та рідному місті

\iusr{Agnieszka Buslova}

Рідним домом, солоним запахом моря, старими будинками, неймовірними людьми, що
прагнули змін, та робили неможливе... вечірніми краєвидами на море...

\iusr{Maksym Bondariev}

Надихав людьми, які хотіли змін

\iusr{Anna Kurtsanovskaya}

Цікавими подіями: MRPL City Festival, гастрофестиваль локальних фермерів,
книжковий фестиваль Open Book.

\iusr{Oksana Uskova}

Крутий Гогольфест, смачна Ветерано піцца та унікаліні вайби)

\iusr{Yana Zhuravska}

Це була моя рятівна гавань. Я завжди могла плюнути на проблеми, складності в
Києві та приїхати додому до мами. Обов'язково чір-чіри в Іллюсі та ар'ян. Потім
їсти, дивлячись на море та слухаючи шум хвилі та волання чайок. Обов'язково
прогулятися по набережній до Екстрім парку та згадувати, що там були прикольні
невеликі схили з величезним камінням, і жаль, що їх прибрали. Це ще лугопарком
називалося, здається. Зустрітися з подругою та проговорити 5-6 годин. Сходити
на Драм. Дивуватися, як змінюється місто у порівнянні з тим, що було в
дитинстві. І навіть в моменти, коли здавалося, що в житті ступор, я бачила, що
змінюється рідне місто, і зі спокійною душею їхала. Завжди була впевнена, що
наступного року повернуся.

Цей рік пропустила. Наступний має компенсувати цей пробіл.

\iusr{Natalia Mykhalchenko}

Морем, степом та теплими спогадами...

\iusr{Vorona Tetiana}

Старими вуличками міста, видом моря з селища Моряків, швидким Кальчиком біля
23. Прогулянки по місту від Вежі, як окремий вид насолоди у місті.

\iusr{Kate Ruzhylo}

А я так і не встигла там побувати, поїздка була запланована на середину
квітня... Багато чула про маріупольські мозаїки, море та вежу, яка дуже схожа
на нашу вінницьку.

\iusr{Viktor Bagayev}

Людьми, морем, атмосферою. Культурними та спортивними подіями

\iusr{Наталія Ігорівна}

Там було вільно, просторо. А який туман😍 як молоко) і ти не заблукаєш, бо
знаєш ці вулички напам'ять. Я в цьому місті відчувала себе вільною.

\iusr{Нина Савенко}

Вiдчуття дому

\iusr{Боднарчук Ольга}

Надихає кожна вулиця місто, бо це моє рідне місто в яке я обов'язково повернусь

\iusr{Елена Наджар}

Це моє місто. Воно надихало мене своїми тихими старими вулицями,закутками,
морем, рідним домом. Моє місто для дущі.❤️🩹🇺🇦

\iusr{Анна Міхньова}

Маріуполь надихав рідним будинком, рідними людьми, спогадами з дитинства,
випічкою з Семейної пекарні, голубами на вулицях

\iusr{Семен Широчин}

архітектурою і атмосферою

\iusr{Ирина Клевцова}

Парками, фонтанами, теплим морем, відвертими, добрими і щедрими людьми,
ароматом Азовської риби і пахучої соняшникової олії до салату, тихими теплими
вечорами влітку. Там надихало все і ніщо не зможе замінити рідний дім.

\iusr{Andrey Shurshilov}

Друзі, вечірнє море, парки...

\iusr{Andrii Stakhovych}

\obeycr
ВІЛЬНІ ЛЮДИ
Стара забудова
Платформа ТЮ
ГогольФест
Чір-чір
\restorecr

\iusr{Иван Рубан}

надихав своїм розвитком. немає жодного міста, яке так швидко розвивалось. а
тепер буде містом, яке дуже швидко відбудується. головне повернути Маріуполь у
вільну країну

\iusr{Анна Савенко}

надихав запахом моря та чудовими людьми

\iusr{Hanna Savenko}

Людьми. Морем. Рідним домом

\iusr{Вікторія Тарасевіч}

рідне місто. Кожна вулиця, кожен будинок, кожен проспект має свій спогад. Ці
спогади з рідного міста надихають скоріше повернутись та відбодувати місто.
Місто, яке буде тільки краще

\iusr{Юлія Хрипченко}

Надихав своєю самобутністю. Особливо зараз, перебуваючи у вимушеному вигнанні,
сумуєш за рідним містом, яке було дійсно СВОЇМ.

\iusr{Анатолий Гвоздовский}

це дім. це дитинство. це спогади. це щасливе життя. кожна мить у Маріуполі мене
надихала

\iusr{Игорь Назаров}

Маріуполь надихав своєю працьовитісттю.

\iusr{Анна Кружкова}

Люди Маріуполя 💙💛 сталеві люди Маріуполя

\iusr{Yana Zhuravska}

ООО, червоне зарево на небі вночі!)) Колись були в мене гості, які висадилися,
побачивши)). Потім жартували про Мордор. Це ще до 14го року. Тоді ще не знали,
де справжній Мордор.
