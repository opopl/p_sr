% vim: keymap=russian-jcukenwin
%%beginhead 
 
%%file 17_11_2021.fb.fb_group.story_kiev_ua.1.kino.cmt
%%parent 17_11_2021.fb.fb_group.story_kiev_ua.1.kino
 
%%url 
 
%%author_id 
%%date 
 
%%tags 
%%title 
 
%%endhead 
\zzSecCmt

\begin{itemize} % {
\iusr{Светлана Манилова}
С первой публикацией Вас в \enquote{Киевских историях}! @igg{fbicon.smile} 

\iusr{Надежда Владимир Федько}
\textbf{Светлана Манилова} Дякую!

\iusr{Alexander Bronstein}
На фоні Першого павільону.

\iusr{Светлана Ландыш}

\ifcmt
  ig@ name=scr.hands.applause
  @width 0.2
\fi

\iusr{Петр Кузьменко}

Я пам'ятаю як знімався у маленькому епізоді фільма \enquote{Хвилі чорного моря} за
мотивами твору Катаєва у 1974 або 1975 році. Це було поруч з моїм будинком
номер 2 на Андріївському узвозі. На Боричеву току. Нам видали по 5 карбованців
за участь у зйомці. Може тарифи змінились, або нам дали меньше бо ми були
дітьми. Але цю пригоду ми з друзями запам'ятали назавжди.  @igg{fbicon.hands.shake} 

\begin{itemize} % {
\iusr{Igor Rostov}
\textbf{Петр Кузьменко} Одеська кiностудiя платила по 7 руб., А Москва - по 10 руб.

\iusr{Петр Кузьменко}
\textbf{Igor Rostov} нам по 5 дали. Хорошо помню.

\iusr{Igor Rostov}
За съёмки вместо заболевшего актера (дублером в гриме и со спины) заплатили 15 руб.

\iusr{Всеволод Шевчук}
\textbf{Петр Кузьменко} 

знаєш друже в цьому фільмі мені пропонували одну з головних ролей пам'ятаєш
хлопчика Петрика одного з героїв так мені пропонували зіграти його я був
товстенький в дитинстві та батьки не відпустили на зьомки в Одесу таке життя
може якби відпустили був би автором, а так став опером!

\iusr{Петр Кузьменко}

Не знав, що ти ще й, практично, актор! З кінематографом і сценою не склалося,
але талановита людина талановита в усьому. За багато років я міг змогу
переконатися у твоїх здібностях!  @igg{fbicon.hands.applause.yellow}
@igg{fbicon.thumb.up.yellow}  @igg{fbicon.hands.shake}
@igg{fbicon.face.wink.tongue} 

\end{itemize} % }

\iusr{Марина Полякова}
Боже, як мені завжди жаль акторів, які змушені вдавати п'яних, хоч їм не наливають. Тепер буду знати, що є варіанти.

\iusr{Клим Форманчук}

Интересный рассказ ! И я \enquote{засветился} на к/студии им Довженко. Только не как
актёр, а как макетчик при цехе комбинированных съёмок. Было это с 1979-го года.
Выполнял макеты для фильмов \enquote{Высокий перевал}, \enquote{Ярослав Мудрый} и ещё много
заказов для других студий страны. В 1983 пришлось уехать из Киева, а когда
вернулся в 1994, то всё уже пошло в разнос, и было уже не до кино. А жаль.
Студия, как и коллектив, были хорошие. Вот только водку по 0,75 л. в магазинах
в то время я не помню. Везде была по 0,5 литра.

\iusr{Валентина Зражевская}

Я знімалася на кіностудії « довженка» у епізодах, фільмах пов’язаних з
медициною, ще у фільмі» дума про Ковпака», працюючи у НІІ Фізіллогіі, брали
відгуки, за працю у колгоспі, тоді усі їздили, і вийшли на зьйомку у кіно, ще
і заробляли, 10-15 рублів, в залежності від - масовка, чи епізод, іноді даже
2-3 слова  @igg{fbicon.smile}. А грим був такий « важкий», як пластилін на облича наносили  @igg{fbicon.smile} 

\iusr{Вячеслав Мусиенко}
Классный сериал был

\iusr{Віктор Шинкарьов}
На фото автора допису, наречена Надійка?)

\iusr{Никита Нечитайло}
Дистильована вода - отрута. Добре що замінили на горілку)

\iusr{Yuri Zmeelov}

Теж там знімався.) Коли згадую що в моєму житті траплялося тільки друзі вірять
що це могло статися з одною людиною в одному житті.) Бо самі \enquote{дещо} з моєю
участю бачили! ) Буде час - поста допишу. )

\iusr{Тома Храповицкая}

Які спогади гарні! Така мить залишається в Пам'яті назавжди... добра і
зворушлива... Дякую Вам, приємно читати, легко...


\iusr{Кирилл Антонов}

Десь наприкінці 90-х років, будучи студентами, з моїм другом ми їздили на
шулявку, здається за якимось принтером чи ще за чимось.

І от на зупинці тролейбуса на проспекті Перемоги, як раз напроти студії ім.
Довженка до нас підходить якийсь дядько, невисокого зросту, в пальті. На перший
погляд мені здалось, що він трохи напідпитку, але в цілому норм. І от цей
дядько пропонує нам з товаришем зніматись у кіно. Каже, що він кінорежисер, що
йому сподобався наш типаж, бо в нас, за його словами, "є дух авантюризму", а
він як раз планує знімати нову картину і таке інше. Точно не пам'ятаю, чим саме
закінчилась наша коротка розмова по суті, і чи взяв він наші контакти, бо у
кіно ми так і не знимались.

Але його візитна картка в мене залишилась.

І от що цікаво, і чому саме я це пишу: той дядько у пальто на зупинці насправді
був режисер. Режисер того самого фільму, в якому знявся автор посту - Григорій
Кохан.

\iusr{Igor Rostov}
Мене так само затягли на Довженка на \enquote{Снежная свадьба}

\begin{itemize} % {
\iusr{Igor Rostov}

А ещё я снимался из-за этого случая в фильме \enquote{Школа} с Анатолием Кузнецовым.
Играли и белых и красных по очереди. Костюмерная была на Багавутовской. И вот
перед посадкой в автобус группа людей, одетые беляками - портупеи, кокарды на
фуражках, сапоги хромовые - зашли в магазин. Один старый дедушка, увидев нас,
тихо сказал \enquote{опа, наши в городе}. Пошутил.

\end{itemize} % }

\iusr{Лариса Кучерова}
10 рублей @igg{fbicon.thinking.face}  большая сумма для статиста.

\iusr{Вахтанг Кварелашвили}

К нам в 71-ю школу, что находится буквально напротив киностудии им. Довженко,
во дворе дома \enquote{три-семь} (как называли и называют его старожилы), постоянно
приходили с киностудии и приглашали на массовку или озвучку. Наш класс был
\enquote{прямо кинозвезды}, по выражению, недовольной постоянными срывами учебного
процесса, директрисы Лианы Лукинишны @igg{fbicon.beaming.face.smiling.eyes}{repeat=3} 

\iusr{Надежда Владимир Федько}

Ця історія мала продовження...

Через деякий час, ввечері, телефонний дзвінок... Телефонує помреж і запрошує на
зйомку в епізоді у фільмі про поручика Жаданівського. Назви фільму я зараз не
пам’ятаю.

Приїжджаю на студію... В павільйоні таких як я вже десятка два. Нас ведуть в
костюмерну, переодягають в арештантський одяг. Потім приносять кайдани...
Справжні кайдани для ніг, іржаві і важкі! Нас усіх заковують у кайдани, ланцюг
від них дають в руки... Ведуть на знімальний майданчик. Там нас шикують один за
одним і пояснюють роль...

Ми повинні іти з глибоко страждальним виразом на обличчі...

Разів сім ми пройшли репетируючи роль... Кайдани труть, бо вони ж на голих ногах.
Мистецтво вимагає жертв! Зігріває думка, що мене побачать друзі на екрані, а
також те, що за епізод отримаю «червонець».

Помреж приводить наш «етап» до декорацій...

Хлопушка... і ми пішли...

Режисер невдоволений нашими обличчями! Занадто життєрадісні!

Дубль 2. Хлопушка і ми пішли...

Режисер і оператор невдоволені нашими обличчями! Занадто життєрадісні!

Дубль 12-й. Хлопушка і ми пішли... Вже ледве ноги волочемо, у декого кайдани
натерли ноги до крові...

Уся знімальна група задоволена! Нам дякують!

Асистенти розковують наші бідолашні ноги... Лікар обробляє потертості... Помреж
видає по «червонцю»! Нас ведуть в костюмерну і ми отримуємо одежу. Поки
одягаємося, стихійно сколочується група чоловік з семи осіб, бажаючих
полікуватися стаканчиком портвейну або мадери. Реалізуємо наше бажання.

***

Через тиждень мене знову запрошують... На перегляд відзнятих епізодів...

Сидимо у залі в радісному збудженні... Гасне світло і на екрані наш епізод...

В кадрі тільки ноги арештантів в кайданах! Крупний план!

Думаю, що ніяких 12 дублів не було. Нас ганяли, щоб ми натурально волочили
ноги... Грим-то на обличчя не наносили! А ми наївні і довірливі повірили...

\iusr{Ольга Почивалова}
Классная история!

\iusr{Tamara Pysarenko}

Мене також в метро підійшов режисер показав документи і запрошував на проби у
кінофільмі Оксана, я не погодилась і дуже жалкую! Був шанс

\iusr{Татьяна Ткаченко}

Да, интересная история и я вспомнила, как снималась в фильме Виктора
Михайловича Иванова \enquote{Ключи от неба}. Хорошие времена были! Есть что вспомнить!

\iusr{Владимир Васильев}
Яка саме це була серія? На ю-тьюбі він викладений. ) Цікаво глянути... )

\iusr{Виктория Зайцева}
\textbf{Владимир Васильев} И я хотела бы. А то весь фильм пересматривать времени нет.

\iusr{Алина Маркина Сыченко}
А помреж таки молодець: типажі вгледіла такі, як треба!

\iusr{Тарас Єрмашов}
Згадується опис зйомок на студії юних героїв \enquote{Тореадорів з Васюківки} В. Нестайка.  @igg{fbicon.smile} 

\iusr{Ирина Шишко}

В цей саме час я також потрапила в кадр цього фільму - просто сиділа на лавочці
біля метро КПІ. А майбутній чоловік роздивлявся книжки також біля метро.
Потім побачили себе в кадрах фільму.Знімали останні серії фільму.

\iusr{Дмитрий Даен}
Гарний текст. Дякую. Розважив.

\iusr{Таня Сидорова}
Цікаво було мати такі пригоди.

\iusr{Alusha Harlamov}
Я спочатку подумав, що той що справа Георгій Віцин.

\iusr{Игорь П.}

\ifcmt
  ig https://scontent-frx5-2.xx.fbcdn.net/v/t39.1997-6/s480x480/104990052_953857721743252_4487169675628282415_n.png?_nc_cat=1&ccb=1-5&_nc_sid=0572db&_nc_ohc=GYJecJdb-NAAX8mimK4&_nc_ht=scontent-frx5-2.xx&oh=00_AT9IAWgQ_XY0zQoT44w-8t_4YPD_WeqY4tGkx4eDxWx_OA&oe=61C8BE28
  @width 0.2
\fi

\iusr{Игорь П.}

Снимались там в массовке, изображающей танцплощадку. Потом двоих моих
приятелей, крепких ребят, пригласили на сьемку фильма про село. Я тоже просил
взять меня, на что получил ответ: вот когда нужно будет сыграть скрипача или
отличника - приходите. Но на сельском току такой \enquote{чахлик невмирущий} будет
портить всю картину) (мне было 17, метр с кепкой, да еще очочки)

\iusr{Сергей Мироненко}

Учился в мединституте, в72 снялся в массовке, форма немецкого зольдата была
настоящая, автомат везде где можно запаян, но тоже настоящий. Десятку пропили в
шашлычной вПушкинском парке.

\iusr{Виктория Зайцева}
Надо же. Смотрела это кино в детстве. Не знала, что оно в Киеве снималось.

\iusr{Lara Iwalg}
А у меня мама в массовках снималась в фильме \enquote{Есть такой парень} @igg{fbicon.heart.red}

\iusr{Светлана Миргород}

Я сама подбирали таким образом ребят для массовки и актёров в театрах. По
поручению папы. Он режиссёр на этой самой студии. Жаль, что не все они были
утверждены только...


\iusr{Марія Голуб}
Так цікаво і затишно  @igg{fbicon.face.relieved} 

\end{itemize} % }
