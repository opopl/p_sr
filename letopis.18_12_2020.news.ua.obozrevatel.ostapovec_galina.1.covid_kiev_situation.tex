% vim: keymap=russian-jcukenwin
%%beginhead 
 
%%file 18_12_2020.news.ua.obozrevatel.ostapovec_galina.1.covid_kiev_situation
%%parent 18_12_2020
 
%%url https://www.obozrevatel.com/kiyany/skoraya-ne-gospitaliziruet-vrachi-ne-berut-trubki-kak-eto-bolet-na-covid-v-kieve.htm
 
%%author Остаповец, Галина
%%author_id ostapovec_galina
%%author_url 
 
%%tags covid,kiev,ukraina
%%title Скорая едет часами, а в больницу брать не хотят. Что происходит в Киеве, где 100 тыс. случаев COVID-19
 
%%endhead 
 
\subsection{Скорая едет часами, а в больницу брать не хотят. Что происходит в Киеве, где 100 тыс. случаев COVID-19}
\label{sec:18_12_2020.news.ua.obozrevatel.ostapovec_galina.1.covid_kiev_situation}
\Purl{https://www.obozrevatel.com/kiyany/skoraya-ne-gospitaliziruet-vrachi-ne-berut-trubki-kak-eto-bolet-na-covid-v-kieve.htm}
\ifcmt
	author_begin
   author_id ostapovec_galina
	author_end
\fi

\ifcmt
pic https://i.obozrevatel.com/news/2020/12/18/364.jpg?size=972x462
\fi

Киев продолжает оставаться на первом месте по количеству выявленных случаев
заболевания коронавирусом. Однако в украинской столице до сих пор существует
множество проблем для людей, которые подхватили инфекцию и пытаются получить
необходимую медицинскую помощь.

Подробнее о борьбе с эпидемией читайте в материале OBOZREVATEL.

\subsubsection{В столице почти 100 тысяч заболевших}

Киев почти каждый день устанавливает новые рекорды по заболеваемости и они
давно перешли отметку в 1,5 тысячи случаев за сутки. Состоянием на 18 декабря в
столице Украины обнаружено более 98 тысяч подтвержденных COVID-19. При этом
победило вирус менее 1/3 пациентов, а количество смертей приближается к 1700. И
это только по официальным данным, ведь реальную картину знать невозможно.

\ifcmt
pic https://i.obozrevatel.com/gallery/2020/12/18/photo2020-12-1811-55-32.jpg
\fi

По словам главного врача Украинского центра контроля и мониторинга заболеваний
Минздрава Романа Родыны, данные по заболевшим отвечают среднему количеству
населения. В Киеве это (опять же, официально) – 3 миллиона человек. Он
рассказал, что в столице занято 53\% коек, но их количество увеличивают за счет
железнодорожных больниц. Пациенты с подтвержденным диагнозом занимают более 4,7
тысяч коек, еще более 400 – с неподтвержденным.

"С подозрениями у нас не так уж и много, это значит, что тестирование идет
быстрее. Раньше, когда на тесты стояла очередь, много коек было занято с
подозрениями. Сейчас лаборатории тестируют материал день в день", – заявил он в
комментарии OBOZREVATEL.

\ifcmt
pic https://i.obozrevatel.com/gallery/2020/12/18/screenshot110.png
\fi

\subsubsection{Скорые приезжают неохотно}

Киевская волонтер Леся Литвинова утверждает, что невзирая на так называемую
разгрузку столичных больниц, к ней продолжают обращаться те, кто не может
попасть на лечение. Ей часто приходится решать проблемы с госпитализацией "в
ручном режиме", то есть по звонку высокопоставленным чиновникам.

"Скорые неохотно едут к старикам. Надо еще сказать, что и многие из них не
хотят госпитализироваться. Это тоже правда. Каждый день говорю на эту тему с
двумя-трема пациентами. Недавно у бабушки сатурация была 70\%, в больницу ехать
не хотела. Мол, буду дома, дайте мне концентратор. Объясняю ей, если хочет,
чтобы ее вылечили, пусть немедленно отправляется в больницу, при таких цифрах
сатурации дома она не справится. Иногда получается уговорить, иногда нет", –
рассказала волонтер.

\ifcmt
pic https://i.obozrevatel.com/gallery/2020/12/18/ki.jpg
\fi

О том, что скорая в Киеве неохотно забирает больных с подозрением на COVID-19
на госпитализацию, OBOZREVATEL рассказал 32-летний киевлянин Александр. Он
вызвал бригаду на 9-й день болезни, когда температура достигала 39,5. Оператор
по телефону сразу предупредил, что врачей можно ждать по 4-5 часов, так как на
подобные вызовы по предоставлению неотложной медпомощи – большая очередь

"Бригада приехала через 40 минут, я прям удивился. Но без защитных костюмов,
хотя я предупреждал, что тестируюсь, вся симптоматика вируса на лицо. Измеряли
температуру, укололи димедрол-анальгин-но-шпу, сатурация была 88\%, но в
больницу не забрали. Посоветовали срочно сделать КТ легких и позвонить, если
будет совсем плохо. К семейному врачу целые сутки не добиться. Телефон
отключен, на Viber не отвечает. Вопрос – куда бежать и что делать? Хорошо, что
другой врач согласился лечить и наблюдать. Семейный, правда, перезвонила, но
через двое суток. Вот такая помощь", – рассказал мужчина.

\subsubsection{Никто не знает, что делать и куда бежать за помощью}

Волонтер Литвинова соглашается, что многие заболевшие до сих пор не знают, что
конкретно им делать. А власть, по ее словам, только подливает масла в огонь
оторванной от реальности статистикой, из-за чего у инфицированных создается
"иллюзия благополучия". Они прекращают заботиться о своей безопасности.

"Если людям говорят, что у нас все пошло на спад, смотрите, все хорошо, новых
мало, мест много, – они начинают вести себя значительно беспечней. Во всей этой
ситуации больше всего раздражает отсутствие правил игры. Люди не понимают
ничего: что им делать, с чем обращаться к врачу, до какой точки можно сидеть
дома. По лечению – вообще "жесть". Знакомый заболевший-астматик рассказывает:
мой друг лечился и пил 5 таблеток утром, днем, вечером и на ночь. А у меня
всего три таблетки на день. Я так, наверное, умру. Надо еще чего-нибудь
добавить". И что тут сделаешь?" – возмутилась она.

\ifcmt
pic https://i.obozrevatel.com/gallery/2020/12/18/lesya.jpg
width 0.3
fig_env wrapfigure
\fi

Кроме того, по Киеву существует еще одна проблема – из больниц выписывают
кислородозависимых пациентов, и что с ними делать – в Минздраве не решили до
сих пор. У врачей аргумент, что больные не могут оставаться в больнице 2-3
месяца. Леся соглашается, что с одной стороны, действительно неразумно по долгу
держать людей в больницах, а с другой – где им быть и куда деваться?

"Не каждый может позволить себе купить аппарат за несколько десятков тысяч
гривен. Кислородные баллончики и подушки не выход. Первые продаются в аптеке за
450 гривен, но это всего лишь 80 вдохов. Его хватает только на 4-5 минут.
Сколько их должно быть, чтобы надышаться? Конечно, за эти деньги дешевле
концентрат купить", – соглашается Литвинова.

\ifcmt
  pic https://i.obozrevatel.com/gallery/2020/12/18/ki2.jpg
  width 0.3
  fig_env wrapfigure
\fi

\subsubsection{Что происходит на Киевщине}

По Киевской области ситуация похуже, чем в самой столице, отметила Литвинова.
Одна из главных проблем – отказ скорой ехать к заболевшему, особенно если у
него нет "положительного" теста.

"У пациента сатурация 60\%, а в скорой сказали: теста нет – мы вас никуда не
повезем. Пришлось "в ручном режиме" его госпитализировать. Через два часа
родственники позвонили, сказали, что сидит в коридоре, никто к нему не
подходит. Включаю "ручной режим" во второй раз. Но нельзя каждого сопровождать
таким образом. Не знаю, почему так, но этого быть не должно", – подытожила
волонтер.

Единственное, что из положительного хотелось бы отметить по Киеву – это
возможность в быстром режиме, не затягивая на сутки-двое, сделать КТ, сдать
ПЦР-тест и анализы в частной лаборатории. Можно даже найти врача, который
согласится сопровождать пациента за деньги, и медсестру, которая в любое время
суток сделает за 100 гривен укол или за 500 гривен поставит капельницу.

А тем временем украинское правительство рассказывает о COVID-фонде, бесплатной
и безотказной медицинской помощи для заболевших коронавирусом. Например, одно
из т.н. правил – скорая обязана забрать пациента в больницу, если уровень
сатурации ниже 91\%. Это является критическим состоянием и требует экстренной
медицинской помощи.

\begin{itemize}

\iusr{Александр Плесконос}

Остался без семейного. Он в декрете. Сказали надо перезаключать декларацию.
Живая очередь к единственному врачу на новую декларацию около 50 человек в
коридоре толпа, крики. Я в ужасе если заболею. Врачи увольняются. Я не знаю что
делать если заболею

\iusr{Alexander Luschik}
в лихий час Україна обрала ЗЕлупу коломойського президентом...

\iusr{Andrey Serov}
не тобі кацапе каркать про вибори.

\iusr{Alexander Luschik}
Andrey Serov не скавучи! одне з 73\% довбайобів. кацап тут саме ти

\iusr{Rezo Ivani}
бабка не хочет в больницу ? зачем её уговаривать ... её место пригодится для
молодого молодой которые ещё родят членов общества ... а гонористые старики
пусть дальше ждут что их будут упрашивать лечиться )

\iusr{Rezo Ivani}
пенсионеры вконец оборзели

\iusr{Ольга Лебедева}

Может ли называться семейным врачём врач, к которому прикреплены от 2000 тысяч
пациентов и который не ходит на вызовы к пациентам?

\iusr{Alexander Luschik}

звичайно що "не може", бо він нічим не відрізняється від всіх інших яким також
похер...

\iusr{Rezo Ivani}
волонтёры как гиены всегда кружат на каком то горе

\iusr{Oleksandr Ponomarenko}

Галя, все как обычно- ссылка на слова волонтерки (одной! Заказ на рекламу?)) В
заглавии одно, в тексте абсолютно другое. Может быть все таки с лечащими Ковид
врачами поговорить ? Или это муторно и страшно?)
\end{itemize}
