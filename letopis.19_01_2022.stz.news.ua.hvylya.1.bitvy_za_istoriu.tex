% vim: keymap=russian-jcukenwin
%%beginhead 
 
%%file 19_01_2022.stz.news.ua.hvylya.1.bitvy_za_istoriu
%%parent 19_01_2022
 
%%url https://hvylya.net/interview/245618-bitvy-za-istoriyu-sootnoshenie-myshleniya-i-voli-debaty-sergeya-dacyuka-i-pavla-shchelina
 
%%author_id romanenko_jurij,schelin_pavel,dacjuk_sergij
%%date 
 
%%tags bitva,chelovek,ideologia,istoria,myshlenie,obschestvo,psihologia
%%title Битвы за историю: соотношение мышления и воли. Дебаты Сергея Дацюка и Павла Щелина
 
%%endhead 
\subsection{Битвы за историю: соотношение мышления и воли. Дебаты Сергея Дацюка и Павла Щелина}
\label{sec:19_01_2022.stz.news.ua.hvylya.1.bitvy_za_istoriu}

\Purl{https://hvylya.net/interview/245618-bitvy-za-istoriyu-sootnoshenie-myshleniya-i-voli-debaty-sergeya-dacyuka-i-pavla-shchelina}
\ifcmt
 author_begin
   author_id romanenko_jurij,schelin_pavel,dacjuk_sergij
 author_end
\fi

\begin{zznagolos}
Почему знание - не мышление. Публикуем дискуссию между Павлом Щелиным и философом Сергеем Дацюком
\end{zznagolos}

В декабре 2021 года состоялась дискуссия между Павлом Щелиным и Сергеем
Дацюком. Модерировал Юрий Романенко. Мы говорили о понимании истории, о роли
мышления и воли в истории. Почему знание - не мышление. \enquote{Хвиля} выставляет
стенограмму этой беседы. 

\begin{zzquote}
Юрий Романенко: Друзья! Всем привет, серферы, бездельники, девианты. Мы
продолжаем наши беседы. Сегодня у нас будет необычный эфир. Сергей Дацюк
справа, Павел Щелин слева. И это будет такая дискуссия, идея которой родилась
после одного из эфиров нашего с Сергеем Дацюком по поводу истории.

Павел Щелин, который сейчас живет в Вене, решил сделать предъяву. У него
появился ряд вопросов, и ради этого этот святой человек приехал в Киев, чтобы
честно посмотреть в глаза Дацюку и задать эти вопросы.

Вопросы по поводу истории, которые мы подавали в интервью «Как вернуть Украину
в историю», это было два с половиной месяца назад. Итак, давай, обозначим
вопросы. Что тебя порвало? 
\end{zzquote}

\textbf{Павел Щелин:} Меня порвало в тот момент, конечно, качество беседы. Оно было
очень высоким. С главным тезисом я согласен: чтобы вернуться в историю,
необходимо мышление. Без этого это процесс лишен абсолютно каких-либо
перспектив.

Заходить я буду немножко с другого конца. Меня больше интересует мнение Сергея
об одном из его тезисов, где он говорит о множественности ойкумен. И быть
точнее, как он видит возможности для взаимодействия этой множественности?

\ii{19_01_2022.stz.news.ua.hvylya.1.bitvy_za_istoriu.pic.1}

Из того что я понимаю, на международной арене нет больших ставок в игре, чем
борьба за то, чтобы твоя история была доминирующей историей на международной
арене. И эти истории не особо-то могут сосуществовать друг с другом: они могут
были поглощать друг друга, либо устанавливать режим холодной войны и перемирия.
Но полноценного мира или даже творческого существования по крайней мере
исторически таких опытов я не видел.

Поскольку с Сергей этой проблематикой занимается очень плотно и очень давно,
мне интересно, как он это себе мыслит.

\textbf{Сергей Дацюк:} Прежде всего, наверное надо договориться о том, что мы будем
понимать под историей. В моем представлении никакой «моей», «вашей» истории.
История – это нарратив, который предполагает некий способ смотрения сквозь
время. Предполагалось, что время одно. И тогда, если время одно, том ы можем
договориться, что у нас одна линейная история, линейный исторический нарратив.
Мы можем спорить о том, кто там главный в этой истории: народ, личность,
монарх, или она случайна. Тогда мы будем, поддерживая этот исторический
нарратив, в разных его вариациях, как-то об этом говорить.

Но если мы говорим, что есть «моя», «ваша», еще какая-то история, то это
значит, что мы должны написать целый ряд различающихся историй. И это путь к
войне. Сразу же.

\textbf{Павел Щелин:} К конфликту как минимум.

\textbf{Сергей Дацюк:} И война условием мира будет создание единой истории. Всякий раз,
когда у нас истории начинают различаться, т.е. по моей истории ваша территория
моя, а по вашей истории ваша территория ваша, и еще некоторые части моей. Вот
есть война.

\textbf{Павел Щелин:} Да, согласен.

\textbf{Сергей Дацюк:} Поэтому история это не событийные, не территориальные отношения.
И в общем-то не территория, ни народ. История – суть, тип сознания, или
нарратив, который пытается увязывать разные времена и разные пространства. И
это принципиально для истории.

Если вы не увязываете, то вы занимаетесь чем-то другим. Грубо говоря, если у
вас одна история, у меня другая, то это означает, что у вашей и у моей истории
много политики, и нам, чтобы выяснить, чья история круче, надо выйти на поле
брани, и в конце концов влиться в одну историю. Потому что ну победит мой
нарратив, либо ваш, и все равно останется одна.

Мне представлялось, что эти тренировки нам как человечеству уже не нужны,
потому что мы прошли одну мировую войну, вторую, третью некую квази, поскольку
это была холодная война. И в общем-то этого хватит.

Но отказ от мышления приведет к тому, что одна часть мира говорит, что «никогда
больше», а вторая часть говорит «можем повторить». И это есть распад на уровне
нарративов, поскольку мы тогда так и начинаем все выстраивать. У нас не
получается разнообразить их нарратив, и что оказывается, что тогда это не
ойкумены. Потому что ойкумены создаются не пространством, не временем, не
историей – ойкумены создаются договором. Причем уровня экзистенции, когда они
выходят на пределы, когда мы кладем некоторые принципы с основанием «вот это
ойкумена». Все остальное это случайные локации, которые в истории никак не
участвуют.

\textbf{Павел Щелин:} Смотрите, почему я говорю именно об историях во множественном
числе. Потому что для меня очевидно, что каждая именно событие по-разному
осуществляет собственную синхронизацию во времени.

\textbf{Сергей Дацюк:} А почему синхронизацию?

\textbf{Павел Щелин:} Это индивидуально. Синхронизация своей…

\textbf{Сергей Дацюк:} Кроме синхронизации есть синкаэризация, есть сенсеклия. То есть,
мы как бы устанавливаем отношения времен.

\textbf{Юрий Романенко:} Объясни, что такое сенсеклия.

\textbf{Сергей Дацюк:} Я просто помню, что коллега въехал в эту тему и принял ее.
Поэтому я могу, конечно, объяснить…

\textbf{Юрий Романенко:} Зрители ж смотрят, они ж не понимают.

\textbf{Сергей Дацюк:} Хронос – это условно говоря, если так по-простому, линейное
время. Кайрос это время разрыва, или Божественное время, или время творения.
Его можно с разных ракурсов наблюдать. Если ты смотришь, то это Бог, который
иногда по отношению к тебе злой, потопы устраивает. С твоей точки зрения, это
разрыв.

С точки зрения Бога это творение, потому что он в ситуации творения, и для него
не есть выживание отдельного индивида проблемой или отдельного народа проблемой
– он держит в своем фокусе внимание все й цивилизации. Потому для него это
творение. И в этом смысле мы должны всегда понимать, что это смотрение из
разных взглядов.

И есть циклос – некая самодостаточность, которая не нуждается не в линейности
хроноса, ни в, собственно, прерывах или творения. Циклос – это то, что моно
признать как попыткой самостоятельного или свободного Божественного
вмешательства и от времени хроноса, некоего бывания. Я даже не говорю о
существовании, оно ж привязано к линейному времени – бывание.

И в этом смысле, когда мы говорим о договоре, мы говорим о согласовании этих
разных времен. И эта история соотносима, потому что нет отдельно линейной
истории, отдельной истории творения и отдельно циклической – они есть только в
единстве.

\textbf{Павел Щелин:} Разве? Я просто почему это спрашиваю, потому что как мне
представляется, множество сообществ Востока принципиально живут только в
циклосе, там кайрос даже не заходил. Как вы будете согласовывать и
договариваться?

\textbf{Сергей Дацюк:} Это не совсем так. Потому что в каждой из существующих
цивилизаций есть свое особое соединение этих времен.

Я могу ошибаться, но насколько я помню, в той же восточной парадигме Аллах
каждое мгновение создает мир и разрушает. Там кайрос есть. Циклия там тоже
есть. Она выражена как социальная: приходит пророк, создает условия,
развивается, потом выделяется неравенство, возникает протест… Циклы так тоже,
или цикл как то, то вокруг пророка. Или если выйти уже на…

\textbf{Павел Щелин:} Там есть циклы Ибн Хальдуна знаменитые.

\textbf{Сергей Дацюк:} Нет-нет. Если выйти на теоретическое обобщение, то единицей цикла
является кудб, в котором есть стержень, или столп. И это то, что
самодостаточно. Кудбы могут экспансировать, могут сжиматься, но так или иначе
они существуют. Кудб это фактически способ теоретизации циклоса как такового.

И если так говорить – обратите внимание – все представления о временах
присутствуют, просто некий тип смотрения является доминирующим, определяющим. В
Западе это иначе. В Китае еще иначе. Индия там вообще югами мыслят, циклы
объемлют гораздо большее линейного времени, само линейное время тоже есть как
смена и т.д.

В каждой цивилизации, где мы ни возьмем, есть соотнесение этих типов времен.
Поэтому, поскольку мы можем с каждой цивилизацией принципиально договариваться…

\textbf{Павел Щелин:} Откуда вот эта уверенность? Откуда ваша уверенность, что вы можете
об этом договариваться?

\textbf{Сергей Дацюк:} Потому что принципы понятные друг другу. Мы можем у друг друга
отыскать похожие принципы организации разных времен. Это нее то, что я уверен,
что мы начнем договариваться – договоримся, я говорю, что это принципиальная
возможность. Вот если бы кто-то полностью выпадал… Ну, цивилизация выпадает,
потому что у нее нет представления о времени кайроса, ил о времени циклоса. И
все. И тогда с ней говорить очень сложно. Вот я для примера говорю, потому что
времен есть больше. Есть время творения, а есть время преображения. В западной
традиции Христос…

\textbf{Павел Щелин:} Есть время эсхатологическое, а есть время не эсхатологическое.

\textbf{Сергей Дацюк:} Да. И нам нужно все эти концепты соотносить. Поскольку мы и
находим буквально в каждой цивилизации, то есть принципиальная возможность
договориться. Мы, находясь в установке на мономодельность… Какова постановка
вопроса, она же сейчас одна. Давай так, у России лучше всего выражено, меньше
всего у Запада, США, меньше всего у Китая. Как у Дункана Маклауда, победит
только один, или выживет только один.

Каждая цивилизация старается натянуть свою сову на глобус, или себя на глобус.
Постановка вопроса о том, что мир суть договор цивилизаций, она в том или ином
виде то более ярко выступает, то менее. Сегодня таким лидером в мире оказался
Китай. Он первый, кто во всеуслышание поставил и начал отстаивать вопрос о
человечестве единой судьбы. Они это поставили.

Запад пока не ставил. Запад ставит квазивопросы: мы отправим на Марс, там
сделаем котел, и этот котел сварит нам некую цивилизацию с единой судьбой, и
мы, равняясь на нее, а это как бы будет черновик. Такая квазипостановка этого
вопроса. Россия честно говорит, что нет, нету никакого Запада, никакого Китая –
есть только Россия, и она должна быть натянута на весь мир. И вот мы в
ситуации, когда…

\textbf{Юрий Романенко:} Когда от мира камня на камне не останется.

\textbf{Сергей Дацюк:} Да, потому что мономодельность это есть базовая установка.
Переход от мономодельности к тому, что мышление – это мыслить разное по разному
является очень важным, поскольку мы тогда можем говорить: Это же не только
Россия, это у нас буквально, буквально на наших глазах. То есть, разобраться не
можем с языками, потому что нету других миров и нету других языков – только
украинский. Это же установка, такая же как имперская русская, один в один.
Никакого договора не будет, ни с кем. Или мы, или смерть всеобщая.

И вот эта мономодельность она же не позволяет странам становится цивилизациями,
а цивилизациям развиваться, двигаться, изменяться. При такой установке у тебя
очень высокая энтропия, и ты не удерживаешься дольше одного 70-летнего цикла.
При такой установке.

\textbf{Павел Щелин:} То, что вы сказали про установку на мономодельность, я как раз
здесь спорить абсолютно не буду, мы здесь, думаю, достаточно солидарны. У меня
скорее больше пессимизма находится в зоне возможностей для договора, потому что
я, наверное, больше внимания акцентирую на принципиальной разности этих
различных подходов.

\textbf{Сергей Дацюк:} Хорошо, тогда давайте поговорим о том, что с вашей точки зрения
договор. Вот ка вы себе видите договор.

\textbf{Павел Щелин:} Для меня договор предполагает наличие как минимум общего языка, на
котором вы можете… Это не договор, это то, что предшествует договору.

\textbf{Сергей Дацюк:} Смотрите, вот уже, когда мы так заговорили, у вас получается
некий барьер, когда за стол садятся только те люди, которые принимают некий
язык, некий дискурс, и какие условия вы там еще примете.

Суть договора в том, что нет никаких барьеров, и нет никаких языков, нет
никакого единого языка.

\textbf{Павел Щелин:} Как вы в этот момент будете коммуницировать? Как вы в этот момент
будете осуществлять акт коммуникации между столь различными…?

\textbf{Юрий Романенко:} Подожди. Пример: западные переселенцы в Северную Америку
приезжают, встречаются с индейцами, языка не знают. У тех и у тех есть
интересы, и вот они встречаются (моделирую то, что было в истории), и они
приходят к какому-то решению. Они же как-то приходили к решению, что мы вам
пушнину, а мы вам оружие или что-то еще.

\textbf{Сергей Дацюк:} Юра, ты договор понимаешь только как культурный или
экономический. Война — это тоже договор.

\textbf{Юрий Романенко:} Я перевожу для простых людей, чтобы наши зрители понимали.

\textbf{Сергей Дацюк:} Чтобы понимали, что война это тоже договор.

\textbf{Юрий Романенко:} Да, конечно.

\textbf{Сергей Дацюк:} Война — это тоже способ договора. И любая война, не только
горячая с использованием оружия, экономическая война — это такой же договор.
Это все договора, это все способы договариваться.

Но, конечно же, когда люди с разными языками, с разными дискурсами,
представлениями, картинами мира, теориями, традициями садятся за стол,
оказывается, любых людей посади за стол, и они быстро обнаружат, что
договариваться они не могут ни на уровне языка, ни на уровне ценностей,
видений, картин, традиций, ни на уровне ничего этого – на уровне принципов. И
они когда начинают обсуждать принципы, они приходят очень быстро к согласию.
Они их кладут, эти принципы.

\textbf{Павел Щелин:} Принципов чего?

\textbf{Юрий Романенко:} Общежития.

\textbf{Сергей Дацюк:} Смотри, даже необязательно. Поскольку мы можем положить принцип и
в этом принципе прийти, что мы, например, не готовы, но мы вам не будем мешать.
Это не общее житие, этот отдельно житие. Но это такой отдельно житие, что я
тебя вижу, я живу отдельно, я тебя вижу, и я к тебе все равно не прихожу. Это
не общее житие, это разножитие.

\textbf{Юрий Романенко:} Это модель Оттоманской империи, когда есть христианские
подданые, мы вас не уничтожаем, вы платите какой-то налог. И все. Вы нами не
занимаетесь, мы вами не занимаемся. Но своих детей в таком-то количестве
периодически в янычарский корпус поставляйте.

\textbf{Сергей Дацюк:} Смотри, тут же вопрос кардинальный, который не дает
договариваться – это вопрос причины твоих неудач, моих, или твоих, которые
каждый для себя обсуждает.

Если ты посмотришь аналитику большинства российских интеллектуалов, то причиной
неудач за распад СССР они назначили не себя, ни собственный отказ от мышления,
а то, что плохой Запад им это устроил. В моем представлении достаточно взрослый
человек это когда он берет на себя ответственность, и сам отвечает за все что
он сделал. Сам. Это взрослый человек.

Подросток – это который еще не стал взрослым, не взял на себя ответственность.
Кто виноват у подростка: мать, отец, старший брат, хулиганы во дворе, учителя и
одноклассники, все, кроме него. Человек взрослеет, когда он понимает, что в
твоих неудачах виноват только ты. И чтобы преодолеть неудачи, ты должен себя
изменить, признать. Вот этот вот момент.

И когда заканчивается война, нормальная ситуация, когда люди садятся и говорят:
давайте не воевать, давайте договоримся. Потом проходит время, несколько
десятков лет, у всех по-разному получается. Но у тех, у кого не получается,
говорят: мы сука не виноваты, это вот они нас эксплуатировали, они у нас
изымали ресурсы, не давали развиваться… То есть вся эта бодяга, которая сейчас
в России происходит. И вот это и есть отсутствие взрослости.

\textbf{Юрий Романенко:} И у нас, кстати, тоже.

\textbf{Сергей Дацюк:} И у нас, кстати, та же самая ситуация. Кто виноват? Москали. Это
не мы.

\textbf{Павел Щелин:} Справедливости ради, это происходит активно не только в Восточной
Европе. Такой тип мышления распространяется активно и в западных странах…

\textbf{Сергей Дацюк:} Смотрите, в этом ж и есть штука, что по мере того, как вы
отказываетесь от мышления, и отказ распространяется, это начинает происходить
везде.

\textbf{Павел Щелин:} Ну да, шизофрению нельзя анализировать.

\textbf{Сергей Дацюк:} Как бы я прояснил, потому что я слишком много говорю. Я вот
оппонента хотел бы послушать.

\textbf{Павел Щелин:} Я вот пытаюсь очень четко быть, в чем мы согласны, в чем мы
оппонируем. Потому что «согласны» на самом деле гораздо больше. И поэтому снова
и снова возвращаюсь в эту точку – точку договора, потому что здесь вы
выступаете с позиции для меня гораздо более оптимистической, и это вызывает мой
большой интерес. Для меня, все-таки…

\textbf{Сергей Дацюк:} Мне ж хочется, когда мы сядем за стол, мышление как-то включится,
и мы начнем говорить оп принципиальным вопросам. Вот это оптимизм. В
действительности такое бывает крайне редко. Поэтому я ж понимаю.

\textbf{Павел Щелин:} Почему у меня оптимизма меньше, объясню. Потому что подобного рода
вопросы, которые мы обсуждаем, для каждого из сообществ в конечном итоге
являются источниками их самовосприятия и самосознания, в каком-то смысле
события. В предельном это вопрос о критериях истинности, кто может для этого
сообщества определять, что есть истина.

Именно, потому что это вопрос настолько высоко, настолько в нем высоки ставки,
возможность договориться об этом представляется мне крайне сложной, потому что
каждый их участников, не без основания, подозревает собеседников в том, что
собеседник стремится навязать ему свой критерий истины. И в таких условиях
договор невозможен.

Возможно взаимодействие, одним из которых, как вы упомянули, является война.
Исторически это как правило решалось поглощением. Одно сообщество с одной
историей, то, что я называю историей, поглощало другое сообщество. В самых
примитивных исторических племенах это было синхронизировано в фигуре царя: вы
приходили, убивали царя, убивали мужчин, забирали детей и женщин в рабство –
вот, поглощение произошло.

Со временем мир усложнился, расширился, не всегда стало это возможным, проекты
усложнились, научились некоторые в истории разные сообщества объединяться в
более крупные проекты. Например, таким историческим проектом считаю Pax
Christiana, проект исламской уммы. Вот это идея в том, что власть есть нечто
подлежащее, на основе чего вы можете договариваться.

И отсюда та точка вопроса, что-то подлежащее, что может подлежать глобальном
договору в глобальном масштабе с вашей точки зрения.

\textbf{Сергей Дацюк:} Смотрите, прежде всего несколько прямых апофатических ответов.
Конечно же, истина. Потому что про истину мы договориться никогда не сможем. По
поводу истины неясно, истина одна иди их много, где истина на самом деле, или
относительно чего, или в какой реальности мы можем найти истину, которая на
самом деле есть, есть ли истина вообще, и когда ты нашел ее, что делать с
другой истиной, признавать ее, ну тогда это противоречит твоей истине, вопрос
об отношении к истине, скоро их много.

Возникает масса разных вопросов, и они выходят на понимание. Все. Они выходят
на гносеологию, затем на эпистимологию, затем на когнитологию, добираются до
пределов, до онтологических начал, и там все как бы завершается. Но там не
истина.

Поскольку мы выходим даже не на экзистенцию, не на бытийные вопросы, мы выходим
дальше. Мы разговариваем перед лицом Бытия и внебытийности. Поскольку мы
предполагаем, что есть некая позиция, неважно какая, Бога, пришельцев,
сингулярности, темной энергии, темной материи, перед лицом которой мы должны
выжить как некая бытийная сущность, или сущность бытийствующая.

Когда мы ставим вопрос о бытийствующей сущности, нет той позиции, с которой мы
можем говорить, это истина или не истина. Там ее нет. Там ее найти невозможно.
Нельзя найти что-то другое, по отношению к которому сказать Бытие истина или не
истина, или точно не истина.

Но что нам позволит туда войти? Конечно же мышление. Потому что это оно мыслит
разное по-разному, может чисто мышление, не философское мышление, как чистое
мышление, выходить за пределы Бытия, и полагать как важное Бытие, инобытие,
внебытийность. И это не вопрос про истину – это вопрос про мышление. Про то,
можем ли мы с вами говорить в мышлении.

\textbf{Павел Щелин:} И вы думаете что можем.

\textbf{Сергей Дацюк:} Я глубоко уверен, что, когда стихали войны, и людям удавалось
продвинуться, преобразиться в структуры, не только социальные, но
экзистенциальные структуры, структуры преобразования, которые были менее
агрессивны и разрушительны, они договаривались именно так. Это происходило в
мышлении. Не в философии – в мышлении.

\textbf{Павел Щелин:} Этого никогда не происходило на глобальном уровне.

\textbf{Сергей Дацюк:} Смотрите, вы же когда кладете глобальный уровень, вы его
привязываете к какой-то измеримости пространства. Мышление не в пространстве
измеримости. Мышление вне этого. Мышление даже не во времени. И в этом смысле
это всегда происходило в предельном для эпохи времени.

А уж глобальное оно, экуменальное оно, или какое – это всегда предельный для
эпохи вопрос. Вот тот предел, который вы внутри эпохи смогли обнаружить, вот на
том пределе вам и надо договариваться.

Я ж даже не уточню. Для какой-то эпохи это были мифологические силы, для
какой-то это был Бог, для какой-то это была объективная реальность. Для каждой
эпохи есть свой предел, в рамках которого мы можем договориться. Но мы можем
договориться. Почему там надо договариваться: потому что, грубо говоря, это
всегда вопрос тех пределов, куда добирается ваше мышление. Ваше, мое. Вот куда
мы добрались, там мы должны договариваться, выше этого нет ничего.

Если мы договорились на каком-то более низком уровне, а выше этого есть что-то,
между нами постоянно будут споры по поводу того что чья интерпретация этого
выше, чье выше, ваше выше, наше выше, моего или мое выше вашего, и все, и опять
война. Только на предельном уровне возможно договариваться.

А это значит, что это подвластно только мышлению. Еще раз, не философскому.
Философское это про мудрость, которая в конечном счете социальная, про истину,
которая в конечном счете не определена. Это вопрос предела эпохального, который
берется только мышлением. Не наукой, не логикой, не философией, не религией –
мышлением.

И вот только так возникает вопрос, а тут, конечно то, что вы ставите – а в
опыте это возможно? Поскольку приходят люди, садятся за стол, и ты говоришь с
человеком, и ты видишь, что ты говоришь не с человеком, а с телевизором. Ты
просишь его: Скажи, а ты внутри себя можешь выключить телевизор? Могу. Он
выключает телевизор и включил своих соседей, своих друзей, знакомых. Хорошо, а
еще друзей, соседей, знакомых моешь выключить? Ну не могу, потому что меня не
останется. И вот это и есть проблема.

\textbf{Павел Щелин:} По сути вы хотите, чтобы люди договорились на пределе, выходящем
за свои преонтологические утверждения, говоря на языке Хайдеггера. Это его
термин.

\textbf{Сергей Дацюк:} Я не читаю Хайдеггера, хотя в своей эпохе он был предельным по
мышлению. Но вообще в принципе он не добрался до пределов, потому что бытийный
предел не является бытийным. Поэтому эти персоны я бы не хотел употреблять,
потому что предельность для эпохи всегда такая штука, когда куда мы достигаем.
В эпоху Хайдеггера да, бытийный вопрос был предельным. И быстро перестал быть,
поскольку быстро стал общим, и быстро начал разрабатываться как то, что уже
лежит, что понятно.

И спекулятивный реализм или философия спекулятивного реализма пытается на этом
паразитировать, она говорит: есть вопросы, которые выходят за пределы бытия,
есть небытийные вопросы, они и на уровне там объекта онтологии себя проявляют,
на уровне смерти себя проявляют. Они находят массу таких предельностей, о
которых начинается вестись разговор.

И оказывается, что имеющегося языка, имеющегося мышления не хватает чтобы
обсуждать эти вопросы, потому что они уже суть не вопросы философии. Потому что
там нет мудрости, там нет истины. Там надо выходить во что-то другое,
какими-то, другими словами. Вот в этом возникает проблема.

\textbf{Павел Щелин:} Я согласен, что проблема возникает именно потому, что она
возникает сейчас, отсюда во многом проистекает мой вопрос. По моей интуиции и
ощущениям складывается ощущение что у вас есть мышление, представление именно о
том, как подобное могло бы произойти.

\textbf{Сергей Дацюк:} Смотрите, тут есть момент: не у меня есть мышление, а у мышления
есть я. Поскольку причастность к мышлению — это базовое представление. Это же
не вещь, о которой можно сказать «вот у меня есть стакан, а еще у меня есть
мышление». В этом смысле никто не может заявить, что у него есть мышление, не
имеет права.

Мышление – это бывает, это иногда есть, иногда нет. Когда ты в ситуации
преображения, ты ему принадлежишь. Вот когда ты оказался вне ситуации
преображения, когда ты отдыхаешь, разговариваешь в магазине, тебе не надо
мышление, ты руководствуешься чем-то другим. Ты практические это берешь,
волевым образом чего-то достигаешь, проявляешь свою веру... Мышление
необязательно тебя сопровождает по всей твоей жизни – это очень краткие
периоды. Оно связано с преображением.

Нормальный человек не будет всю жизнь преображаться, не переставая. Ему надо
растить детей, заботиться о хлебе насущном, ему надо звонить родителям,
встречаться с девушкой… Масса других вещей надо делать, где мышление, извините,
не нужно.

\textbf{Юрий Романенко:} Это затратно для ума, это вызывает огромные потребности в
ресурсах.

\textbf{Сергей Дацюк:} Как говорит Паша в отношении студентов «мыслить вообще не надо,
от этого голова болит». Когда обсуждают, кто там пошел на пляж, а кто там пошел
обсуждать. Может пойдем, подумаем? Думать вообще не надо, от этого голова
болит.

Поэтому да, непрерывно думать – от этого болит голова. Инсульты случаются. То
есть, масса нехороших вещей происходит если ты будешь напрягать мышление на
протяжении всей жизни.

Поэтому, во-первых, этого не надо делать постоянно. Но в эти минуты, когда ты
принадлежишь мышлению, и собираются люди, которые собираются преобразить мир
(или преображая себя, преобразить мир), тогда оно может вас посетить. Не вы его
поимеете, а оно вас посетит, вы станете к нему причастны. И в этом пространстве
мышления как пространстве у вас появится шанс договориться.

Поэтому поскольку мы не можем спрогнозировать, кто к нему причастен, то, как
это ни странно, оно присутствует в людях, который нельзя предсказать. Мышление
может быть у кого угодно, понимаешь: у дворника, у лифтера, у ученого. Как ни
странно – даже у философа может быть мышление. У кого угодно может. Кто угодно
может оказаться причастным к мышлению. Не «может быть», а причастным.

И вот в этом интерес. Что когда мы собираемся, и человек понимает ту
ответственность, которая на нем, что он сейчас договаривается не поп поводу как
там устроить вечеринку, или как переворот устроить – а по поводу ойкумены,
которую он представляет. И что от того, как он договорится, следующие десятки
лет зависит, в какой ситуации мы окажемся. Вот тогда оно очень хорошо
включается. У всех.

Еще хорошо, когда три тонны веса направлены на левой ноге, или хороший пендель
в одно место – все эти вещи способствуют.

\textbf{Юрий Романенко:} Предельная ситуация.

\textbf{Сергей Дацюк:} Конечно. Когда мир окажется на грани войны, очень быстренько все
вдруг включаются и причащаются к мышлению. А до этого никто даже думать ни о
чем не хочет. И вот этот есть момент.

И поэтому тут даже предсказать нельзя, у кого эта причастность окажется.

\textbf{Павел Щелин:} Допустим, нельзя предсказать, у кого эта причастность окажется. Но
вот в том, как вы описывали в предыдущие минуты, прослеживается необходимость
пространства, в котором это мышление может разворачиваться. Имею в виду именно
на практическом уровне.

\textbf{Сергей Дацюк:} Это пространство преобразования. Только там оно может проснуться.
Нет преобразования – нет мышления. По поводу деления собственности не
просыпается. По поводу деления власти не просыпается. По поводу «мы не любим
этих, мы ненавидим врагов, мы подозреваем…» нету там никакого мышления.

\textbf{Павел Щелин:} И с этим согласен. Здесь уже скорее вопрос даже на более
примитивном уровне. В современной реальности, которую я вижу, такого
пространства, где подобного рода бы вопросы обсуждались и мышление
осуществлялось бы не особо наблюдается. И для меня одна из причин…

\textbf{Сергей Дацюк:} Он чуть с ума меня не свел, доказывая мне, что меня нет. Конечно
же, мышление это только мы порождаем.

Мы плюнули на все, создали среды свои (свои софистические беседы по средам), и
там мы это пространство порождаем. В ситуации сопротивления, в том смысле, что
сопротивляется аудитория, которая не ходит, не смотрит, сопротивляются
комментаторы, сопротивляются разные умники, которые приходят и говорят, что вы
назидаете, то вы чушь порете, то вы Википедию не читаете, то еще что-то. Мы
работаем в условиях сопротивления.

Но это возможно. Мы показываем, что это возможно.

\textbf{Юрий Романенко:} А еще вы придумали язык, который понятен только вам

\textbf{Сергей Дацюк:} Это плохо? Так оно всегда бывает. Кто-то придумывает язык,
начинает на нем говорить, потом появляется очень мало людей, которые его
понимают и тоже начинают на нем говорить. Ну и заканчивается это все тем, когда
говорят: ну ты что. этого языка не знаешь? Как бы все знают, ты не знаешь. Так
оно всегда и происходит. А как? Иначе не бывает.

Причем это всегда связано с сопротивлением. Всегда.

\textbf{Юрий Романенко:} И волей, которая его преодолевает.

\textbf{Сергей Дацюк:} Во.

\textbf{Павел Щелин:} А вот про волю вопрос. Соотношение мышления и воли я объясню на
конкретном примере, что я имею в виду. Вы часто употребляли цитату Бэкона о
том, что знание — это сила, власть и возможность. Но здесь для меня эта фраза
всегда была неполной. Для меня она очень тесно связана именно с проблематикой
воли, что от абсолютизации знания, от нее на самом деле очень короткий путь к
тому, что ты абсолютизируешь собственную волю и уходишь в полный отрыв от
реальности, от того, что подлежит знанию, если угодно. Мы это активно очень
видим на примере того, что произошло с западным мышлением, особенно в контексте
современных процессов в университетах.

Соответственно у меня скорее вопрос к вам, может это просто подтверждение, или
что. Считаете ли вы, что знанию все-таки необходим контакт с реальностью, что
оно не полностью автономно, что оно должно быть связано с тем…

\textbf{Сергей Дацюк:} Смотри, во-первых, знание — это особое пространство, которое
очень точно обозначено Бэконом. Впервые работа эта была написана на латыни
Scientia potentia est, которая потом была переведена на английский более-менее
адекватно Knowledge is power. И power, и potentia переводятся на русский
четырьмя словами: сила, власть, возможность, потенция.

Из всего этого ни одно определение не является мышлением. Там дважды звучит
активность, разворачивание, возможности. То есть, возможности, которые следуют
из потенции, которая волей двигается как собственно разворачивание силы. Там
нет ничего про мышление. И в этом смысле как вы яхту назовете, так она и
поплывет. Она и поплыла. То есть, его взяла на вооружение власть, разная,
необязательно государственная, взяли власть корпораций, взяли различные
организации, все взяли на вооружение власть. Там нигде не было обещания
мышления, нигде.

Поэтому знание не мышление. Более того, знание – это запрет на мышление. Когда
ты знаешь, ты не мыслишь, у тебя есть все ответы. Почему стакан стоит на столе?
Потому что, блин, сила тяжести, ее описал Ньютон. Ты не думаешь, ты перестал
мыслить в то время, как узнал.

Если бы мы назвали это иначе, то у него бы и другая судьба была.

\textbf{Юрий Романенко:} То есть, спутник мышления – любопытство. Любопытство пробуждает
постановку вопросов.

\textbf{Сергей Дацюк:} И хоть сто ответов у тебя будет про всемирный закон тяготения,
или как он в оригинале назывался у Ньютона, будучи одолженным из метафор
алхимии «Закон всемирной симпатии». Ньютон стремился обобщить симпатию к
элементам в алхимии с симпатией как притяжением. Более того, количество книг по
алхимии гораздо больше, чем каких-либо других научных книг. Но мы его знаем как
физика-математика, но не как алхимика. А его алхимия еще ждет своего
исследования.

И в этом смысле это не позволяет вам мыслить. И хоть сто есть ответов, почему
стакан стоит на столе, вы должны снова и снова спрашивать, потому что мышление
так устроено. А при каких условиях мы можем его оторвать? Это антигравитация. А
что вы вкладываете в это понимание? И это разворачивание вопроса, вот этим
занимается мышление.

Когда вы прекратили спрашивать, вы получили знание, вы закрыли путь к мышлению.

\textbf{Павел Щелин:} Ну и получается, что в основе всего проекта Модерна лежит
прекращение мышления, если уж так совсем.

\textbf{Сергей Дацюк:} В идеале да, потому что мышление удерживалось другими социальными
ритуалами, практиками. Другими. Могу просто перечислить: чтение, раздумье
наедине… Есть целый набор традиций, которые сейчас усыхают.

Наедине человек больше не остается. Чтение уходит, заменяется Твиттерами и
всеем прочим, экраном. Мы те традиции, которые существовали, не рефлексивно на
которых держалось мышление, убили развитием знанияевых технологий. И оказалось,
что и знания их нет. И там их нет, потому что мы там убили технологиями. И
оказывается, что никакие ритуалы мышление больше не удерживают. И апеллировать
к нему теперь бессмысленно.

Когда ты говоришь мышление, люди думают, что ты говоришь про логику, или про
философию. Если ты прочел пару философских книг, ты уже умеешь мыслить. Что не
так?

\textbf{Павел Щелин:} Просто здесь оказалось больше общего, чем я, честно говоря,
ожидал. Но да, то, что вы описали, мне вспомнилось очень красивое рассуждение
Ортеги-и-Гассета, где он как раз описывает практики погружения в самого себя
как ключевой практики, в рамках которой происходит процесс мысли. Она как раз в
начале ХХ века оплакивал уход этого процесса, который он так видел отчетливо.

\textbf{Сергей Дацюк:} Я тоже интересовался этим вопросом. Мало того, меня еще
интересовало это в контексте похожести ситуации на Украину. Он тоже переживал,
что ж такое, почему у нас это было и почему у нас это ушло. Почему я здесь и не
могу докричаться до своих сограждан. А ведь еще недавно как бы отсюда шел
пассионарный толчок, и мы могли там покорять другие континенты… То есть, что
приключилось? Что ушло? Вот он задал себе такой вопрос.

\textbf{Павел Щелин:} Его ответ, если я не ошибаюсь, был в направлении того, что
называют бесхребетным человеком, он назвал это бесхребетная страна, но также
бесхребетная Испания.

У него есть важная для меня метафора – разница между живой и мертвой верой, что
живая вера это та которая является самодостаточной для нашего движения, для
нашего пребывания во времени и пространстве. А мертвая вера это та, которой мы
отдаем должную дань и ритуалы, причем с разной степенью искренности, но на
самом деле она нас совершенно не греет и руководствуется нами что-то другое.

Мне кажется, его рассуждения оказались абсолютно в течение последовавших ста
лет неустаревающими. Более того, процесс только ухудшился.

\textbf{Сергей Дацюк:} У него, насколько я помню, лекции о философии чрезвычайно
интересные, поскольку он тоже спрашивал: что там, что есть такого, вот что там
есть такого. Он задавался этими вопросами.

\textbf{Павел Щелин:} Получается, так или иначе, что не он ответы на заданные вопросы не
нашел, и последовавшее время проблему только усугубило. Потому что пространство
для мышления в социальном плане…

\textbf{Сергей Дацюк:} … мы создаем сами. Нигде не существует такого пространства, в
отношении которого мы могли бы сказать – о, пространство мышления. Нет, так не
бывает. Поскольку никто нигде не может утверждать, что вот здесь мы сделаем
место, и тут люди будут преображаться. Это невозможно. Во-первых, это
непредсказуемо, а во-вторых, не у всех есть к этому стремление. Поэтому оно
возникает, как правило, спонтанно.

\textbf{Юрий Романенко:} Человек должен быть способен оказаться наедине с самим собой и
вести этот диалог. Потому что в сегодняшней культуре, пространстве, в котором
мы находимся, человек как раз боится оставаться с самим собой.

\textbf{Сергей Дацюк:} У него возникло слишком много игрушек и развлечений, чтобы
остаться наедине с собой. И речь не только о телевидении и девайсах – речь о
том, что все вокруг превращено в игрушки. Ты приходишь на работу, в свою
корпорацию, а у тебя там куча игрушечек создано, ритуалы разные, начальник,
окружающие разные структуры от тебя требуют.

Там столько всяких игрушек, столько всего вкусненького и интересного, что ты
нее можешь остаться наедине. Ты приходишь домой – телевизор, соцсети. Ты едешь
в метро – у тебя девайс. Ты окружен игрушками, которые тебя постоянно
отвлекают, ты не можешь остаться наедине с собой. Даже когда ты выезжаешь на
природу, ты выезжаешь с шашлыками с друзьями, и они тоже не дают тебе остаться
наедине с собой.

С кем можно остаться наедине с собой? С любимым человеком можно, понимаешь.
Можно с закадычным другом остаться наедине с собой, поскольку в какой-то момент
вы начинаете мыслить друг о друге.

\textbf{Павел Щелин:} Контактировать начинаете.

\textbf{Сергей Дацюк:} Да. это можно. А вот уже когда ты входишь в толпу, в массу
безымянную, с которой у вас нет никакого опыта сомыслия, там, конечно, остаться
наедине с собой невозможно.

\textbf{Юрий Романенко:} Ты когда говорил о том, что просишь снять другого человека,
мнения друзей или чего-то, это как раз вот об этом. На самом деле сегодня люди
по большому счету боясь оказаться с самим собой, через это не приходят к тому,
а что есть на самом деле моё. И когда человек не может сформулировать, что есть
его, то дальше он не может прийти ни к осознанию целей своей жизни и каких-то
целей поменьше.

\textbf{Павел Щелин:} Он не может это сделать, потому что все системы, о которых говорил
Сергей, они устанавливают очень хороший, мощный, качественный барьер между
тобой и тем, что есть условно, даже на онтологическом уровне во многих смыслах.

В каком-о смысле это все возвращает к старом мифу о платоновской пещере: из
пещеры выходить неприятно, особенно если эта пещера наполнена комфортом. В этом
смысле я абсолютно согласен, что было бы в каком-то смысле нереалистично
ожидать, что это произойдет на массовой уровне.

Проблема другая. На мой взгляд, проблема в том, что степень агрессивности к
средам мышления не уменьшается. Она только увеличивается.

\textbf{Сергей Дацюк:} Смотри. Образование, ну, что что у нас называется образованием,
на самом деле это подготовка у обучению по чудовищам. Последовательно, первое –
это встраиваться в социальную реальность, второе – бунтовать. Ни то, ни другое
не является объявлением психического суверенитета.

Мало того, психический суверенитет как явление замалчивается. Если ты объявишь
психический суверенитет, ты будешь идиотом, отшельником, маргиналом, или на
крайняк экстремистом.

Но нигде не учат тебя, что такое психический суверенитет. Как
противодействовать телевизору, как противодействовать социальным сетям, как
оставаться с самим собой. этой компетенции нет, потому что она ведет к
мышлению.

Мышление – опасная штука. Потому что ты до такого можешь додуматься, что вдруг
ты возьмешь и скажешь ни с того ни с сего, что лидер – не лидер, страна не
страна, понимаешь. Ты до черти чего додуматься можешь. А это опасно.

Поэтому у тебя есть два выхода: ты можешь либо встроиться, либо бунтовать. И
то, и другое и есть социализация.

\textbf{Павел Щелин:} Я прошу все-таки добавить в защиту, почему мышление тем не менее
важно, в чем прикладная важность. Без вашего психического суверенитета во
многом вы обречены реагировать на мир в состоянии постоянной неожиданности. Все
события для вас происходят внезапно, как бы из ниоткуда, из полумрака.

Используя любимую метафору Юрия, отказавшись от мышления, вы неизбежно
становитесь на путь столкновения со стеной. Вопрос какая именно стена, где,
когда. Но так или иначе, отрывая себя от этого, вы оказываетесь в мире
абсолютных иллюзий.

\textbf{Сергей Дацюк:} Да. Но тут же важная вещь, что только в состоянии психического
суверенитета ты способен подлинно преображаться. Все остальные способы твоего
изменения социальны. В этом смысле надо ж переосмыслить очень многие
произведения, как художественные, кино и даже музыкальные.

Про что альбом Pink Floyd «The Wall». Это про бунт, который встроен в саму
систему. Это ничего не дает. В этом смысле эти вещи связаны.

Ты начинаешь задумываться про психический суверенитет, когда он тебе становится
нужен. Он нужен для чего? Для независимого ни от кого преображения. Или
преобразования среды, в которой ты находишься. Вот там нужен психический
суверенитет.

Так, вообще, он, конечно, не нужен. Ну зачем? Есть множество траекторий, многие
из них вкусные, приятные, по ним легко двигаться, все понятно, все не разу уже
пройдено. Бунт – это одна из траекторий, туда тоже можно заходить, за счет мамы
с папой или спонсоров бунтовать. Мало того, в современной мире это очень хорошо
оплачивается.

Как только ты объявляешь, что ты хочешь бунтовать, у тебя находится куча
спонсоров, которые говорят: Боже мой, как хорошо. Дорогая Грета, против чего ты
хочешь бунтовать? Ты говоришь: против этого. Давно ждали. Сколько тебе надо?
Вот тебе 4 экипажа и яхты – плыви. Давай бунтуй, давай больше бунтарей, хороших
и разных.

\textbf{Юрий Романенко:} Или Йошка Фишер, «зеленый» министр был в Германии. Он же был
бунтарем, в радикальной группировке был. А потом закончил очень респектабельно
и очень хорошо.

\textbf{Сергей Дацюк:} А взрослость – это когда ты уже понимаешь ограниченность бунта.
Что подлинные изменения начинаются с себя. Хочешь изменить мир – начни с себя.
Преображение себя – это первый шаг к преобразованию мира. Себя.

\textbf{Юрий Романенко:}Да. и понимание того что мое. Что реально твое, что ты реально
мыслишь, о чем ты реально думаешь сам, а не то, что ты где-то услышал и
воспроизводишь просто как попугайчик. И это на самом деле касается всего.

Это даже на примитивном уровне семейной жизни, это люди, которые доходят до
уровня, когда они начинают осмысливать, что из отношения в паре имеют вот
такие-то интересы, такие-то цели, а не то, что говорит мать, отец, общество.

\textbf{Сергей Дацюк:} И в этом смысле первый вопрос, что история это и есть
преобразование в разных временах, которые замечают другие. История – это не
твоя история, эта история никому не интересна. Если кто-то та чего-то сделал,
вышел, вот переворот на Майдане… Можете не переживать, это даже никто в мире не
услышит, потому что никакого преобразования не произошло.

История начинается с того момента, когда вы осуществляете преобразование в
разных пространствах, в разных временах, и это заметно для других. Вот только
это и есть история. это нельзя игнорировать.

\textbf{Павел Щелин:} Я это сформулировал так: чтобы войти в историю, надо давать лучшие
ответы на по-хорошему еще даже не поставленные вопросы. Только так вы станете
заметным.

\textbf{Сергей Дацюк:} Я бы перевернул. Надо ставить вопросы, на которые ни у кого нет
ответов. Вот это мощнее, потому что давать ответы – ну да, есть вопросы, а я
еще напишу на них 101 ответ. Неинтересно. А хорошо поставленный вопрос может
стоять долго. Долго. Ты умрешь, и все, кого ты знаешь, умрут, а твой вопрос,
хорошо поставленный, будет стоять. Вот это и есть, собственно, источник
преобразования.

\textbf{Юрий Романенко:} И ты уже у истории самим этим фактом, просто поставив вопрос.
Даже не дав ответа.

Ну что, друзья, я думаю мы подошли к концу и, по-моему, получилось очень
неплохо. Беседа получилась – это и есть характеристика беседы, когда
собеседники равноценные, и даже если кто-то больше говорит, но это интересно
всем.

Спасибо Сергею, спасибо Павлу. Надеюсь, у нас встреча не последняя. Я буду
делать все, чтобы такие встречи повторялись.

До новых встреч. 
