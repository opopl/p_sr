% vim: keymap=russian-jcukenwin
%%beginhead 
 
%%file 20_05_2018.fb.lesev_igor.1.pomojka_god_ukraina.cmt
%%parent 20_05_2018.fb.lesev_igor.1.pomojka_god_ukraina
 
%%url 
 
%%author_id 
%%date 
 
%%tags 
%%title 
 
%%endhead 
\subsubsection{Коментарі}
\label{sec:20_05_2018.fb.lesev_igor.1.pomojka_god_ukraina.cmt}

\begin{itemize} % {
\iusr{Василий Стоякин}
Ну вот. США ему не нравится...

\iusr{Оксана Высоцкая}
Год на помойку, это при самых оптимистичных ожиданиях.

\iusr{Serhiy Vorobey}
Я пригадую, як ти волав що син Юща мав дорогий джип і про зубожілий народ.
Чого ж про сина Яника голос не підіймав?! Адже в того був цілий автопарк таких джипів, і ще телеканальчик, і ще банк, і ще так по дріб'язку майже на мільярд.

\begin{itemize} % {
\iusr{Елена Круть}
Про сына Януковича знали все - богатый, "бизнеса отжимал", а сын Ющенко, хотелось бы от вас, хоть сейчас узнать, чем промышлял то?

\iusr{Serhiy Vorobey}
\textbf{Elena Krut}, "Ніколи не кажіть ніколи" або "всі".
Не повірите для скількох пролетарів статки Яника стали відкриттям під кінець
ЄвроМайдану. "Прості" люди не знають ніц, якщо про це не кричать топові
телеканали.  ХЗ чим заробляв син Юща, але їхня родина як не була серед багачів,
так і не потрапила в списки Forbs, навіть України. Тож це показник, що Ющ не
був бенефіціаром корупційних схем.

\iusr{Владимир Алексеев}
\textbf{Елена Круть}
Про сына Януковича знали все - богатый, "бизнеса отжимал"

Надеюсь, что бизнес, который у вас отжал сын Януковича, вам вернули.

\iusr{Елена Круть}
\textbf{Владимир Алексеев}
Это вы напрасно )). Это я клише этой публики употребила, поэтому и в кавычках )).

\iusr{Владимир Алексеев}
\textbf{Елена Круть} Тады прошу пардону.

\iusr{Елена Круть}
\textbf{Serhiy Vorobey} 

Не смешно! Деньги, начиная с собранных на больницу будущего, вывезла жена Катерина
в США. Деньги банка Украина тоже не просматриваются. А в Форбс попадают только
задекларированные состояния  @igg{fbicon.face.wink.tongue} 

\iusr{German Gorozhanski}

После того, как Майдан выкинул из политики и из страны Януковичей, на каналах
укро-ТВ должна была стоять очередь из желающих рассказать в прямом эфире, как у
них Саша-стоматолог отжимал бизнеса, но ни один терпила так и не озвучил своих
претензий к почившей в бозе панде. Разборок в судах по возврату незаконно
отжатых бизнесов тоже не слышно. Эка печалька.


\iusr{Игорь Лесев}

Мало того, что соврал, ну да это такое, для тебя и тебе подобных это норма. Но
мне никогда не был интересен личный автотранспорт детей Ющенко, Януковича или
Порошенко, хотя бы потому, что это слишком мелко. Если Андрюша оставляет на
проезжей улице автомобиль и уходит, создавая затор - это несколько другое, на
чем он конкретно ездит. Но так как ты балбес, тебе это понять трудно. Но
"волав"... давай, Серега, подбирай слова корректнее. Тут же не твоя сходка
побратимов, здесь и приличные люди встречаются.

\iusr{Вячеслав Беленький}
\textbf{Serhiy Vorobey} для многих майдановцев и таблица умножения стала "открытием"
И где янукович в Форбс,? украинский полит-вариант с полусотым местом не предлагать, на фоне первого места Порошенко - даже не смешно.

\iusr{Serhiy Vorobey}
\textbf{Vyacheslav Belenkiy}, для ще більшої к-сті немайданівців таблиця множення також стала відкриттям.
А щодо Форбсу, то тут ліпше дивитися динаміку. На скільки статки змінилися впродовж Президентської каденції. Бо ж до-президентське майно в них було різне.

\iusr{Вячеслав Беленький}

немадановцы не приводили в "депутаты законодательного собрания" Гаврилюка. 

\href{https://www.youtube.com/watch?v=W1m-XKGevJc}{%
Это хуже народа, - это лучшие люди города, youtube, 06.07.2014%
}

А статки форбс... и куда они сейчас подевались? камазами Янукович вывез?

\iusr{Serhiy Vorobey}
\textbf{Vyacheslav Belenkiy}, натомість вони привели Чечетова і Царьова. А Яника навіть зробили Презом.  @igg{fbicon.smile} 

\iusr{Вячеслав Беленький}
\textbf{Serhiy Vorobey} 

Не имею ничего к Цареву. Если бы ваше быдло не орало в раде, заглушивая
выступление Олега, а вникло в озвученное в ноябре 13 - крым бы оставлася
Украиной а доллар примерно по 12 сегодня.

Но вам по душе оказались кривляния вора Скрипина, который тырил из сейфа
грантовые подачки у своих коллег и паясничал в эфире про "шо?"

\url{https://www.youtube.com/watch?v=EP7JuVFojvE}

\iusr{Вячеслав Беленький}
Для справки, "о чем речь" у Царева фильм ИЮЛЬ 2013 - полное предупреждение и
абсолютное предсказание, вплоть до снайперов.

\href{https://www.youtube.com/watch?v=uTKZYTLDnv8}{%
ТехКемп и посольство США. Рецепт революции в Украине, TheTarTararam, youtube, 09.07.2013%
}

\end{itemize} % }

\iusr{Владимир Алексеев}

Автор - неисправимый идеалист. Потому что свято верит в то, что интересы
геополитического игрока, вложившего 5 млрд в демократию на Украине, могут быть
похоронены людишками, которые бросают какие-то бумажки в какие-то ящички с
какими-то прорезями.

Какие выборы??? Всегда есть возможность отфильтровать кандидатов еще на старте,
потом - зачморить случайно прорвавшихся (Добкина с Царевым помните?), потом
включить мозгомойку в виде СМДИ (средств массовой дезинформации), а если в
ящичках с прорезями вдруг все же окажутся не те бумажки - так на этот случай
есть технология майдана, который скорректирует "неправильный выбор" и обеспечит
нужный результат.

Поэтому верить, что "демократическое волеизъявление" действительно может что-то
изменить - так это успокаивать себя подобно тому еврею, который в Освенциме до
последнего заставлял себя верить, что сейчас он помоется в бане и дальше все
нормально будет.

И не надо шапкозакидательски оценивать сильнейшего геополитического игрока,
заявляя, что "внешне-американское кураторство... не работает". Это с точки
зрения построения нормальной страны, нацеленной на собственное развитие, оно не
работает. А с точки зрения "управляемого хаоса", когда "проблемы индейцев
шерифа не волнуют", оно очень даже эффективно работает.

Стравили Россию с Украиной? Стравили!

Пролили кровь между народами? Пролили!

Повязали сотни тысяч людей этой кровью? Повязали!

Уничтожили высокотехнологические отрасли экономики Украины? Уничтожили!

Загнали страну в неподъемные долги? Загнали!

Изымут землю и то, что тут еще ценного осталось, за эти самые долги? Изымут!

Или вы думаете, что этот последний этап, ради которого все эти "демократические
преобразования" были затеяны, может быть отменен путем опускания каких-то
бумажек в ящички с прорезями?

Выборы, выборы...

Баня, баня...

\begin{itemize} % {
\iusr{Дмитрий Серба}

Меня не покидает ощущение, что серьезные парни за океаном, наследники
колониализма бриттов, вполне понимают, что делают. А делают они грабеж того,
что после уплывёт обратно, где триста лет было. И придется заново наращивать
жирок.

\end{itemize} % }

\iusr{Сергей Колесников}

Л. Гумилев о деградирующих этносах.

«Всякий рост становиться явлением одиозным, трудолюбие подвергается осмеянию,
интеллектуальные радости вызывают ярость. В искусстве идет снижение стиля, в
науке оригинальные работы вытесняются компиляциями, в общественной жизни
узаконивается коррупция, а в армии солдаты держат в покорности офицеров и
полководцев, угрожая им мятежами.»

Л. Гумилев, докторская диссертация «Этногенез и биосфера Земли». 1974 год.

\iusr{Олег Резник}
А как смотрят на прапор - можно подумать.

\iusr{Сергей Бочкарев}

"Следующие выборы с самым любым исходом – это будет хоть какое-то очищение..."
Та вы шо. Перечислите, пожалуйста, весь список...

А то я что-то в данном писании помимо предзаказаннных точек:

1) давайте пошлем подальше этих пиндосов;

2) давайте снова любить рассеюшку, ибо мы без нее - никуда; ничего больше не
разглядел.

Вот сижу и думаю, а мне нужны в ленте пропагандисты? Даже очень умные и складно
пишущие. Но матрица заказа, за пределы которой - ни-ни, дает о себе знать.

\begin{itemize} % {
\iusr{Игорь Лесев}
Сергей, вас никто насильно у меня во френдах точно не держит. Мало того, что
делаете выводы согласно своему уму, так еще и второй раз уже вешаете клише.
Валите с богом нафиг

\iusr{Сергей Бочкарев}

Свалил. Глобальные геополитические рассуждения о судьбах "рассеюшки и пиндосов"
все меньше интересуют. А что конкретно выпускал "Заря-машпроект" или
"Мотор-Сич" практически уже никто не знает из тех, кто ныне волает по этому
поводу. Ибо труженики литературного фронта.

\end{itemize} % }

\iusr{Таня Кушнир}

В этот год мы увидим воровство и распилы всего и вся в масштабах невиданных за
все предыдущие годы. Надо успеть выпрыгнуть в последний вагон уходящего поезда.
Вот эти все надоговые , полицейские, военные генералы, чиновники всех уровней
администраций они что, будут сидеть и ждать, авось пронесет? Фигушки:) уже
никто не ждёт. Надо успеть здесь и сейчас, пока карта идёт. Им что американцы,
что китайцы вообще все равно. Даже спецэкспорт или оборонка - там же точно
такая же психология принятия решений. Выжженная земля будет через год:) без
внешних врагов -оккупантов.

\iusr{Олег Хавич}
У них даже фаллический символ - и тот покосился  @igg{fbicon.frown} 

\begin{itemize} % {
\iusr{Игорь Лесев}
это они нас роняют)

\iusr{Олег Хавич}
\textbf{Игорь Лесев}, это они на нас кладут  @igg{fbicon.frown} 
\end{itemize} % }

\iusr{Дмитрий Коломийченко}

Тут год на помойку, а на повестке дня все 26 лет независимости. Экономический,
демографический, военный и политический кризисы. Элита переводит стрелки на США
и Россию, которые должны выстраивать отношения между собой и за Украину тоже.

\iusr{Матвей Кублицкий}

разделяй и властвуй. цельная Украина не нужна никому. ни Европе, ни России, ни
штатам... И Порошенко делает всё для развала Украины. Самое страшное - что
большая часть его действий необратима.

\end{itemize} % }
