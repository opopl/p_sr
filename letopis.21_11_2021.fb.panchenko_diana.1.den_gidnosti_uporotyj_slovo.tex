% vim: keymap=russian-jcukenwin
%%beginhead 
 
%%file 21_11_2021.fb.panchenko_diana.1.den_gidnosti_uporotyj_slovo
%%parent 21_11_2021
 
%%url https://www.facebook.com/permalink.php?story_fbid=2061408560677415&id=100004248730374
 
%%author_id panchenko_diana
%%date 
 
%%tags chelovek,maidan2,revgidnosti,slovo,ukraina,uporotyje
%%title День Гидности и значение слова «упоротый»
 
%%endhead 
 
\subsection{День Гидности и значение слова «упоротый»}
\label{sec:21_11_2021.fb.panchenko_diana.1.den_gidnosti_uporotyj_slovo}
 
\Purl{https://www.facebook.com/permalink.php?story_fbid=2061408560677415&id=100004248730374}
\ifcmt
 author_begin
   author_id panchenko_diana
 author_end
\fi

В день Гидности хотела бы поговорить о значении слова «упоротый». 

В моём лексиконе оно закрепилось как раз во время событий на Майдане. 

На молодежном сленге упоротый — чересчур увлечённый чем— то, зацикленный. Ещё
так называют человека в состоянии алкогольного или наркотического опьянения. 

Объясню, как это слово понимаю я. На будущее) 

Упоротый категоричен. Не слышит и не хочет слышать противоположную точку зрения
— она его убивает. Отсюда — пена у рта и порванная на груди рубаха. 

Упоротый не любопытен и не любознателен. Не любит думать. Новую информацию
подвязывает под давно сложившийся пазл стереотипов. Поэтому упоротые не меняют
точку зрения. Любят обвинять других в «переобувании»

Упоротый зациклен на себе, ему не интересен окружающий мир. 

Упоротые — глубоко несчастные люди. Не ведитесь на браваду. Павлиний хвост
самомнения на самом деле —  замаскированный комплекс неполноценности. 

Упоротый агрессивен. Раньше упоротые выплескивали яд и желчь на лавочках,
теперь — в социальных сетях. 

Упоротый несчастен в любви. У него проблемы в сексе. Отсюда ханжество и
нетерпимость ко всему, что сексуально. Упоротые тётки любят перемывать кости
красивым женщинам. Мужики — обсуждать части тела. 

Упоротый лишен культуры уважения, в избытке — неуважение ко всем и ко всему.

Во время революций упоротые первыми бегут бросать брусчатку в оппонентов. Это
их бенефис. 

Иногда упоротые становятся политиками или лидерами мнений. 

Аудитоии упоротых посоветовала бы быть осторожнее. Это контент от которого
тупеют. 

Если такие приходят к власти, расплачиваются все. Как правило,  это случается в
бедных странах. В развитых избиратель предпочтет думающего и рационального
лидера. 

Упоротое мышление губит. 

Критичное — заставляет развиваться. 

Можно ли назвать этот текст примером упоротого мышления? 

Высказывайтесь )
