% vim: keymap=russian-jcukenwin
%%beginhead 
 
%%file 09_12_2020.news.ua.strana.3.turcia_vaccine_russia
%%parent 09_12_2020
 
%%url https://strana.ua/news/305726-kakuju-vaktsinu-ot-koronavirusa-budet-zakupat-turtsija.html
 
%%author 
%%author_id 
%%author_url 
 
%%tags 
%%title Турция предпочла российской и американской вакцине китайскую
 
%%endhead 
 
\subsection{Турция предпочла российской и американской вакцине китайскую}
\label{sec:09_12_2020.news.ua.strana.3.turcia_vaccine_russia}
\Purl{https://strana.ua/news/305726-kakuju-vaktsinu-ot-koronavirusa-budet-zakupat-turtsija.html}

\ifcmt
pic https://strana.ua/img/article/3057/kakuju-vaktsinu-ot-26_main.jpeg
caption Турция не будет закупать вакцину "Спутник V". Фото: focus 
\fi

Министр здравоохранения Турции Фахреттин Коджа заявил, что Анкара пока не будет
приобретать американскую и  российскую вакцину от коронавирусной инфекции.\Furl{https://strana.ua/news/304969-tsena-vaktsiny-ot-koronavirusa-sputnik-v-sostavljaet-1942-rublja-minzdrav-rf.html} Они
предпочтут ей китайскую Coronavac. 

Об этом сообщает\Furl{https://www.haberturk.com/yazarlar/fatih-altayli-1001/2897064-bakan-koca-hedef-yazdan-once-100-milyon-doz} газета Haber Turk.

Так, от закупки было решено отказаться, поскольку препарат не соответствует
нормам надлежащей лабораторной практики.

"Проблема возникла из-за соответствия GLP (надлежащей лабораторной практике).
Россия не смогла ей соответствовать. По этой причине она вышла за рамки наших
интересов", - рассказал глава МИД Турции.

В то же время, помимо российской вакцины, Турция рассматривала варианты закупки
нескольких других препаратов. В числе перспективных министр назвал китайскую
вакцину Coronavac. По его словам, она имеет ряд преимуществ по сравнению с
вакцинами от Moderna и Biontech: в частности, в Турции будет проще развернуть
ее массовое производство.

Пресс-секретарь российского президента Дмитрий Песков ответил на отказ Турции
от вакцины.

"Можно сказать однозначно - результаты тестирования и испытаний говорят о том,
что это очень эффективная и надежная вакцина, которая может и обязательно
сыграет очень важную роль в борьбе с пандемией", - заявили со стороны России.

Ранее "Страна" сообщала, что в России объяснили, чем отличаются вакцины от
коронавируса "Спутник V" и "ЭпиВакКорона".\Furl{https://strana.ua/news/304859-v-rossii-objasnili-raznitsu-mezhdu-vaktsinami-ot-koronavirusa-sputnik-v-i-epivakkorona.html} Сообщается, что обе вакцины
показывают высокую эффективность и безопасность. 

Также напомним, что Россия хочет производить вакцину от коронавируса в
Харькове.\Furl{https://strana.ua/news/305472-rossijskuju-vaktsinu-ot-koronavirusa-mozhet-proizvodit-ukrainskaja-kompanija-biolek.html} Так, производитель российской вакцины от коронавируса "Спутник V"
заявил о готовности делать ее на территории Украины и подать документы для ее
регистрации государственными регулирующими органами.
