% vim: keymap=russian-jcukenwin
%%beginhead 
 
%%file kiev.pogoda.01_06_2021
%%parent kiev.pogoda
 
%%url 
 
%%author 
%%author_id 
%%author_url 
 
%%tags 
%%title 
 
%%endhead 
\section{01-06-2021}

\emph{Вторник, Первого Июня 2021 Года от Рождества Христова, Лета 781-ого от Падения Киева}.

Солнце ярко светит; но довольно прохладно, когда утром идешь за пачкой сигарет
и кофе; вместо долгожданного лета - по ощущениям весна; когда же станет жарко
уже?

Заходжу на сыноптик-точка-уа... - Схід 4:49, Захід 21:02, 12$^{\circ}$ по
Цельсию, максимальная температура была в 2011 году - +31.7$^{\circ}$ по
Цельсию, а минимальная +4.1$^{\circ}$, снова ж по тому самому Цельсию - ого!
Так холодно было, невероятно! Это ж что получается, наши киевские девушки
вместо нарядных воздушных платьиц должны были укутываться в теплые осенние
пальто - куда же ж это годится? Это ж так ничего и не рассмотреть толком, а
ведь лето же!

И что ж это за года такие, 2011 и 1904?

В 2011 году в России 1 января случилась авиакатастрофа Ту-154 в Сургуте;
печально тогда год начался... Народ Русский, Народ-Богоносец после попойки
новогодней еще не опохмелился толком, а самолет уже сгорел... Трое погибли...
Еще в 2011 году Нобелевскую Премию по медицине и физиологии получили \emph{Брюс
Бетлер, Ральф Стейнман та Жуль Гоффман за роботи, пов'язані з імунною
системою}, это украинская википедия мне сообщает...  А я... а я где ж тогда
был? В консерватории учился, Мусоргского на пару с Римским-Корсаковым
осваивал...

Ну а в 1904 году меня не было еще даже в самых фантастических планах моих
прапрадедушек и прапрабабушек; что же тогда было?... о! 3 августа 1904 года
родился Клиффорд Саймак, американский писатель-фантаст. Он много всего
понаписывал, мне вот больше всего из его творчества нравится повесть о Гоблине
со Cтропил, и о путешествии к Древним... Там есть вечно голодный Сплетник, и
Зверь Хаоса, и ученый-монах, ищущий таинственный университет... Правда,
названия повести не помню, надо будет посмотреть...  Еще у него есть повесть о
Городе, правда, это не о Киеве... Саймак в Киеве вроде не бывал, а жаль! многое
ты, Клиффорд, упустил в жизни, если ни разу не был в Киеве, пойми...

\ifcmt
  pic https://i.pinimg.com/originals/70/85/6f/70856f3e22d7e686a06a22819774aa75.jpg
  width 0.2
  fig_env wrapfigure
\fi

А насчет погоды, так сыноптики (Сыны Оптики?) еще пишут такое: \emph{У Києві
протягом усього дня небо буде вкрите хмарами.  Без опадів. Народний прогноз
погоди: У перший день літа вшановується пам'ять благословенного князя Івана
Углицького і великого князя Дмитра Донського. У давні часи в цей день ще бували
заморозки на ґрунті.  Наші пращури помітили, якщо перші два дні червня йде дощ
- весь місяць сухим, а літо теплим буде. А якщо літо розпочинається з холодів,
то наступні 40 днів будуть холодними.  Спостерігали також за пташиним царством:
якщо ворони сідають дзьобами в одну сторону - до сильного вітру.}

Хм... князья Иван Углицкий и Дмитрий Донской... Донской вроде знаю, это ж река
такая Дон есть, так? А Углицкий кто это такой вообще, и \emph{чьих он будет}?
Надо бы почитать про них. Еще бы посмотреть за вороньем Печерским - я вот на
Печерске живу, кстати - куда клювы будут свои вострить? - будет ли ветер и
нужно ли брать ветровку?  А пока что собираю рюкзачок, кушаю овсянку, наливаю
себе кофей, просматриваю интернет. 

%\ifcmt
  %pic https://starpri.ru/wp-content/uploads/2019/09/Olesya-Medvedeva-v-YUnosti.jpg
  %caption Олеся Медведева, взято с \url{www.socioforum.su}, 16.07.2020
  %width 0.2
  %fig_env wrapfigure
%\fi

Вот, например, - strana.ua, - это, говорят, у нас самое страшное
руссо-пропагандисткое издание, его вроде сам Путин лично курирует, а усатый
Песков\footnote{\emph{Усы Пескова}, Остро. Честно. Смешно. Подпишись пока не
запретили!\par -
\url{https://www.youtube.com/channel/UCNYsieni0V71jCgwkka79yQ}} Новой Почтой
мятые рубли Олесе Медведевой пересылает - а Олеся намедни в таком красивом
пиджаке выступала, кстати! - \emph{В Киеве блогеры разгромили квартиру и уехали
на полицейском
авто}\footnote{\url{https://kiev.strana.ua/336264-blohersha-ttiisshhiinnaa-i-ee-paren-ustroili-pohrom-na-semnoj-kvartire.html}}.
Ого! Четкие пацаны! И там еще девчонка была, главная заводила... \emph{Тишина}
ей имя... \emph{Тихонечко} так разгромили и уехали. Неужели так бывает? И это
что - сами полицейские блоггерами, что ли, подрабатывают? Или блоггеры
разгромили квартиру, и потом походу угнали еще и ментовскую машину? Как
понимать-то? Куеве мий, чудные дела в тебе творятся! И нет ли здесь руки
Путина, кстати? 

\ifcmt
  pic https://gdb.rferl.org/af07355b-c927-4fe2-8023-b3c1517526fb_w1597_n_r0_st.jpg
  caption Палає українське село. Літо 1941 року, Взято з: \emph{Ігор Роздобудько}, radiosvoboda.org, 01.06.2021
  width 0.4
\fi

Или вот еще, radiosvoboda.org, \emph{Друга світова війна на Стародубщині та в
Березані, що на
Київщині},\footnote{\url{https://www.radiosvoboda.org/a/ukrayintsi-i-druha-svitova-viyna/31281398.html}}
автор Игорь Роздобудько, историк, сидит в той самой страшной Москве - как ни
странно...  Идет уже восьмой год страшной войны с Мордором-Агрессором,
Порошенко все надувает шоколадные щеки за Отчизну и щедро раздает
\emph{Клубничку} храбрым воинам АТО, - почем нынче Клубничка, Петя? - а Игоря
все никто не арестовывает и не интернирует как гражданина вражеской страны...
Хорошо устроился, профессор! Видно, на хорошем счету у Путина, раз он никак его
не трогает... Уж слишком это подозрительно - так может быть, Игорь Роздобудько
с radiosvoboda.org и есть та самая зловещая \emph{рука Москвы}, из-за которой у
меня вчера в туалете сливной бачок поломался? Ну ладно. Времени особо нет
разбираться сейчас, надо же позавтракать как следует. Путин-Путином, а овсянка
с маслом и медом по расписанию.

\ifcmt
  pic https://gdb.rferl.org/e1c5d7db-4baf-47bf-8e04-267edd7f3ae4_w1597_n_r0_st.jpg
  caption Радянські війська залишили Україну впродовж кількох місяців 1941 року. Мапа: Україна в Другій світовій війні (1939–1942 роки), Взято з: \emph{Ігор Роздобудько}, radiosvoboda.org, 01.06.2021
  width 0.4
\fi

Ну а насчет статьи, Игорь, не кажется ли тебе странным, что старый выживший из
ума глупый дед, ничего уже собственно не понимавший в том, что происходит
вокруг, и расстрелявший солдата за хату, явно не прав? Убил Защитника Родины за
какую-то жалкую хату...\footnote{
І от, підпалюючи хату, яка погано займалася під дощем, він поставив свою
гвинтівку під дерево, і дід його з неї пристрілив. Так розповідали люди. Хата
дідова була врятована ціною життя молодого солдата. Після війни хтось на того
діда доніс у відповідні органи, тіло червоноармійця шукали та не знайшли, тож
діда відпустили на волю через нестачу доказів. І вже коли він помер своєю
смертю, через кілька років почали зносити його хату, і тіло того солдата
розшукали закопаним десь під хатою. Так вони й жили з мерцем під долівкою все
своє воєнне та післявоєнне життя. Отже, діда це врятувало від суду та Сибіру, а
можливо й розстрілу, \emph{Ігор Роздобудько}, radiosvoboda.org, 01.06.2021} \textbf{\emph{Дед! Война
началась! Очнись!}} Зло напало, и грозит уничтожить все вокруг... Все вокруг
горит, и на кону судьба всей Украины-Нени, всей России, и вообще всего
Человечества...  а ты все в фантазиях витаешь... \emph{Моя хата - пальцем тронь
- расстреляю, а больше я ничого не знаю...} И дед странный какой-то, умерших
хоронил не на кладбище, как полагается, а прямо у себя в огороде, прям мистика
какая-то... Наверное, капуста лучше растет, и чеснок острее, если твои дедушка
с бабушкой тут же в огороде гниют...  безбожник чертов, дед. Так вот. Дело ж не
в хатах самих по себе, это всего лишь хата, а в \emph{Земле Родной}, на которую
напал \emph{Враг}. Враг, беспощадный, жестокий; расстреливающий без сожаления
женщин, детей, и стариков в Бабьем Яру...

\ifcmt
  pic https://static.life.ru/posts/2017/09/1047756/42edc78d7384f023cfabced87d05c481.jpg
  caption Взято с: Бабий Яр. Кровавый кошмар на окраине Киева, life.ru, 30.09.2017
  width 0.4
\fi

Хаты-то можно и отстроить заново - если даже сожженный и разбомбленный
нацистами Киев уже давно Коммунисты отстроили и обустроили с прекрасным Метро и
чудесной Каштановой Аллеей на Крещатике - а если потерять \emph{Отчизну}, так
потеряешь все, не так ли?

Ну ладно... Листаю интернет дальше. 

Овсянка съедена, время идти на работу... Закрываю квартиру на ключ (да, да! та самая квартира, где деньги
лежат! - пару сотен гривен, не больше; - тарифы ж у нас европейские, а дворцы
за высокими пятиметровыми заборами в Конча-Заспе покруче будут Версаля и
Тадж-Махала ) - и выхожу навстречу Солнышку, навстречу самому прекрасному
Городу на Земле... 

\ifcmt
  pic https://scontent-lga3-2.xx.fbcdn.net/v/t1.6435-9/193983110_1449248275436079_7303309907055056290_n.jpg?_nc_cat=111&ccb=1-3&_nc_sid=825194&_nc_ohc=NJ6Gq8QuHisAX8FoHSc&_nc_ht=scontent-lga3-2.xx&oh=329a9eb209136b90875d9d6a1ab686df&oe=60DCF30E
  caption Киев примерно 800 лет назад (взято у \emph{Виктор Креницкий}, facebook, 31.05.2021)
  width 0.9
\fi
