% vim: keymap=russian-jcukenwin
%%beginhead 
 
%%file 27_07_2019.stz.news.ua.mrpl_city.1.letnii_teatr_v_mariupole
%%parent 27_07_2019
 
%%url https://mrpl.city/blogs/view/letnij-teatr-v-mariupole
 
%%author_id burov_sergij.mariupol,news.ua.mrpl_city
%%date 
 
%%tags 
%%title Летний театр в Мариуполе
 
%%endhead 
 
\subsection{Летний театр в Мариуполе}
\label{sec:27_07_2019.stz.news.ua.mrpl_city.1.letnii_teatr_v_mariupole}
 
\Purl{https://mrpl.city/blogs/view/letnij-teatr-v-mariupole}
\ifcmt
 author_begin
   author_id burov_sergij.mariupol,news.ua.mrpl_city
 author_end
\fi

\ii{27_07_2019.stz.news.ua.mrpl_city.1.letnii_teatr_v_mariupole.pic.1}

В те времена, когда телевизоры были редкостью, а телевизоров с цветным экраном
вообще не существовало, жители нашего города с удовольствием ходили летними
вечерами в Городской сад. Не только для того, чтобы подышать свежим воздухом,
напоенным запахом петуний или сирени, но и побывать на спектакле заезжего
театра, послушать концерт эстрадного оркестра и певцов-кумиров того времени,
которых они слушали по радио, а видели их только на открытках. Где же
происходили эти встречи? А в Летнем театре...

Начнем, пожалуй, с его истории. Лето 1934 года в городах Донбасса
ознаменовалось гастролями Ленинградского большого драматического театра имени
Горького. Завершиться этот тур должен был в Мариуполе. Но Зимний театр, который
располагался на месте, где сейчас стоит дом №24 по проспекту Мира, обветшал
настолько, что о приеме в нем гостей не могло быть и речи. И тогда было решено
построить в Парке культуры и отдыха, - так назывался тогда Городской сад, -
летний театр. Здесь будет уместно привести цитату из статьи Т. Ю. Були из книги
\enquote{Мариуполь и его окрестности: взгляд из XXΙ века} – (Мариуполь: Изд-во
\enquote{Рената}, 2008): \emph{\enquote{Мариупольцы приняли обязательство – за 45 дней выстроить в
Городском парке культуры и отдыха летний театр и открыть его в день начала
гастролей Большого драматического театра – 25 июня 1934 года. Однако, несмотря
на энтузиазм, времени не хватило и накануне приезда гостей, 24 июня, оставалось
много недоделок. Тогда был объявлен городской воскресник. Сотни мариупольцев,
сменяя друг друга, самоотверженно трудились весь день и всю ночь. Но самое
удивительное то, что этот энтузиазм захватил ленинградских актеров. С поезда,
уставшие, они не пошли отдыхать, а дружно взялись за работу и собственноручно
клали кирпичи в здание мариупольского очага культуры. Утром 25 июня на
строительной площадке, где еще вчера валялся мусор, красовались цветочные
клумбы}}. Коль скоро состоялся разговор о строительстве театра, нельзя не
вспомнить инженера-конструктора, который принимал участие в проектировании
деревянного строения – это \textbf{Михаил Васильевич Флорищевский}. К сожалению, имена
других создателей Летнего театра поглотило время.

\vspace{0.5cm}
\begin{minipage}{0.9\textwidth}
	
\textbf{Читайте также:}

\href{https://archive.org/details/09_06_2018.sergij_burov.mrpl_city.park_kultury_i_otdyha_sezon_1958_goda}{%
Парк культуры и отдыха: сезон 1958 года, Сергей Буров, mrpl.city, 09.06.2018}
\end{minipage}
\vspace{0.5cm}

В гастрольном репертуаре театра были спектакли по пьесам Николая Погодина \enquote{Мой
друг}, Максима Горького \enquote{Егор Булычев и другие}, Карло Гальдони \enquote{Слуга двух
господ}, Марселя Паньола и Поля Нивуа \enquote{Продавцы славы}. Все представления
прошли с аншлагом. И мариупольские театралы долго еще вспоминали игру артистов
одного из ведущих театров Советского Союза с берегов Невы...

История умалчивает, кто еще, кроме Ленинградского театра, выступал до войны на
его подмостках. А вот о послевоенной гастрольной жизни в Летнем театре можно
кое-что рассказать. Каждый год, начиная с конца июня и почти до начала
сентября, сменяя друг друга, шли спектакли драмтеатров Днепропетровска, города
Шахты, Кишинева, Ростова-на-Дону, Донецкого театра оперы и балета, Украинского
музыкально-драма\hyp{}тического театра имени Артема, Киевского театра музыкальной
комедии и других.

Через двадцать один год после первых гастролей, вновь с 23 по 31 июля 1955 года
для мариупольских зрителей на сцене Летнего театра выступали артисты
Ленинградского большого драматического театра: Ефим Копелян, Виталий
Полицеймако, Владислав Стржельчик, Людмила Макарова, Изиль Заблудовский,
Валентина Кибардина и другие.

В 1959 году был создан Донецкий областной русский драматический театр (г.
Жданов) – наш театр. Пока здание для него достраивалось, труппа гастролировала
в городах Донбасса. Актриса Анна Захаровна Гащенко вспоминала: \emph{\enquote{16 мая 1960
года мы вступили на мариупольскую землю. Нам предстояло ставить спектакли в
летнем деревянном театре в Городском саду. Там и произошла наша первая встреча
с нашими будущими постоянными зрителями}}...

\textbf{Читайте также:} 

\href{https://archive.org/details/30_06_2018.sergij_burov.mrpl_city.ijul_zhara_gastroli}{%
Июль, жара, гастроли, Сергей Буров, mrpl.city, 30.06.2018}

28 августа 1960 года в Летнем театре начались шестидневные гастроли Московского
театра сатиры, в составе его труппы были Борис Тенин, Георгий Менглет, Татьяна
Пельтцер, Владимир Лепко, Вера Васильева, Анатолий Папанов, Георгий Тусузов,
Ольга Аросева, Евгений Весник. Какое созвездие имен! Почему-то запомнилась
пьеса Ноэла Кауарда \enquote{Обнажённая со скрипкой} в интерпретации столичных актеров.

Кроме театральных спектаклей нередки здесь были концертные программы. Например,
26 и 27 июня 1951 года для местной публики пела неподражаемая Клавдия Ивановна
Шульженко. 30 мая 1955 года в летнем театре выступала популярная в свое время
певица Ружена Сикора. Знаменитый джаз-оркестр Олега Лундстрема дал два концерта
- 24 и 25 августа 1957 года. 2 августа 1959 года состоялся концерт Народного
цыганского оркестра под управлением Шандора Ярока (Венгрия), 3 и 4 августа 1965
года мариупольские поклонники джаза слушали оркестр Эдди Рознера, играл здесь
также эстрадный оркестр Армении под управлением Артемия Айвазяна. Его
знаменитый \enquote{Караван} произвел такое впечатление на местных музыкантов, что
бывало, нет-нет да услышишь наигрыш на аккордеоне, кларнете или трубе этой
мелодии, а то и просто напев вполголоса.

Кто еще пел под сводами Летнего театра? Любимец публики Николай Щукин, Майя
Кристалинская, Ирина Бржевская, Алла Иошпе и Стахан Рахимов, Эдита Пьеха с
ленинградским ансамблем \enquote{Дружба} под руководством Александра Броневицкого.

В Летнем театре выступали с психологическими опытами знаменитый экстрасенс и
эстрадный артист Вольф Мессинг, Шахтерский ансамбль Донбасса \enquote{Молодая гвардия},
Ансамбли песни и пляски Краснознаменного Балтийского флота и Краснознаменного
Северного флота, Московский ансамбль \enquote{Классический балет}.

Здесь названа лишь малая часть артистов, театральных коллективов, оркестров,
ансамблей, с которыми встречалась мариупольская публика в летнем театре.
Начиная с конца мая и до начала сентября, он никогда не пустовал. Только
уезжали одни гастролеры, как сцену занимали другие. А когда гастрольный сезон
заканчивался, в летнем кинотеатре показывали кинофильмы, до самых холодов.

Но все это было в прошлом, теперь уже далеком прошлом.

\textbf{Читайте также:} 

\href{https://mrpl.city/blogs/view/svitlana-otchenashenko-najvazhlivishezberegti-zhivoyu-dushu}{%
Світлана Отченашенко: \enquote{Найважливіше – зберегти живою душу!}, Ольга Демідко, mrpl.city, 17.07.2019}
