% vim: keymap=russian-jcukenwin
%%beginhead 
 
%%file 30_10_2021.fb.bilchenko_evgenia.1.metafizika_russkoj_dushi
%%parent 30_10_2021
 
%%url https://www.facebook.com/yevzhik/posts/4391145350920541
 
%%author_id bilchenko_evgenia
%%date 
 
%%tags bilchenko_evgenia,metafizika,poezia,rusdusha
%%title БЖ. Метафизика русской души
 
%%endhead 
 
\subsection{БЖ. Метафизика русской души}
\label{sec:30_10_2021.fb.bilchenko_evgenia.1.metafizika_russkoj_dushi}
 
\Purl{https://www.facebook.com/yevzhik/posts/4391145350920541}
\ifcmt
 author_begin
   author_id bilchenko_evgenia
 author_end
\fi

\ifcmt
  ig https://scontent-lhr8-1.xx.fbcdn.net/v/t1.6435-9/249521408_4391145257587217_7723811808659966290_n.jpg?_nc_cat=110&ccb=1-5&_nc_sid=8bfeb9&_nc_ohc=h7x0xF0VAHMAX-V2oCV&_nc_ht=scontent-lhr8-1.xx&oh=26a34847303a8d659cf6fea5d2ad898a&oe=61A1525B
  @width 0.4
  %@wrap \parpic[r]
  @wrap \InsertBoxR{0}
\fi

БЖ. Метафизика русской души
А ему ведь тоже твердили: "Нико,
Не будь таким дураком.
Она - плясунья в кафе: над нею нет никакого нимба,
И скоро она уедет с покровителем далеко.
Продавай свой мёд, продавай своё молоко".
А он улыбался в ответ с простодушием той клеёнки,
На которой царапал небо цветов мелиссы.
Он был слишком жив для мёртвого гения из Нью-Йорка
И слишком мёртв для живого лавочника Тифлиса.
К тому же, мечта была слишком близко.
Проданный магазин за тележку с цветами -
Концентрат Тертуллианова Credo quia absurdum.
Это потом будут Пугачёва, Паулс и Пиросмани:
Пока же были одни пересуды,
Подобные звону чужой посуды,
Трое бессонных суток, потраченные ресурсы.
А ему ведь тоже твердили: "Ваня,
Не будь таким идиотом.
Зачем тебе доходить до Германии?
Зачем тебе погибать за кого-то?
И, вообще, хорошее пиво, у этих готов".
А он улыбался в ответ широко и весело
И выполнял приказы, не мешкая,
Превращая тело своё во Христа и месиво.
Он не знал ещё, что станет посмешищем
Для постмодерного культа потребления вещи.
Взятый Рейхстаг и флаг, на нём водружённый.
И больше - ни одного еврея в проклятой печи.
Везущие милого на коляске в шёлковых платьях жёны:
Девяносто-шестьдесят-девяносто. Простые речи,
Родные плечи, хлеб на стакане, в серванте свечи.
Танец с безногим в брачный любовный вечер.
А ей ведь тоже твердили: "Женя,
Не будь такой мазохисткой.
Не устраивай хайп из самосожжения.
Извинись, возьми этот грант хипстерский,
Кружевное бельё, звание, виски.
Одно лишь sorry, - имя твоё покинет чёрные списки".
А она улыбалась в ответ совсем уже по-дебильному,
Как улыбаются только русские, в особенности - поэты.
Быль становилась сказкой, сказка - быльём и былью.
Сотрудники безопасности - люди автомобильные.
Пёхом ходят юродивые из донбасских степей ковыльных.
Дарят цветы на последние деньги мазохисты и суицидники.
Бросаются на амбразуру танка мазохисты и суицидники.
Жертвуют жизнью за русское слово мазохисты и суицидники.
Живут с безногими инвалидами мазохисты и суицидники.
Я так горжусь, что я - мазохист, что мне лень для вас рифмовать.
29 октября 2021 г.
Илл.: Н. Пиросмани "Актриса Маргарита".
