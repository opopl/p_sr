% vim: keymap=russian-jcukenwin
%%beginhead 
 
%%file letters.18_02_2023
%%parent letters
 
%%url 
 
%%author_id 
%%date 
 
%%tags 
%%title 
 
%%endhead 

Доброго дня, Ігоре! Слава Україні!

Дуже подобаються Ваші світлини мирного Маріуполя! Дякую за Вашу роботу! 

Мене звати Іван, я програміст з Києва. У мене є проєкт Літопису Війни, яким я
займаюсь у вільний від роботи час, уже досить тривалий час... Власне кажуче, це є систематичне записування
різних публікацій по авторам, категоріям, темам. Також я записую пости про
Маріуполь, як мирний, так і про 2022 рік... Намагаюсь робити так, щоби зберігти
пам'ять навічно.  Фейсбук нажаль, в цьому сенсі є досить ненадійне середовище
(для довготривалого зберігання інформації).  Технологія, яку я використовую,
називається LaTeX - хоча це досить стара технологія (їй вже більше 30 років),
але це є де факто стандарт для публікацій у науковому світі.  Вона дозволяє,
коротко кажучи, записати в гарному читабельному вигляді будь-яку інформацію
(тексти, зображення) будь-якого розміру, хоч би і книжку у 100 або 1000
сторінок. Фактично, це є окрема мова програмування документів. Нижче як
приклад, яким чином виглядає повністю записаний пост (з коментарями, фото,
скрінами - найповніший варіант запису).  Більше Ви зможете подивитись на моєму
телеграм-каналі @kyiv_fortress_1. Також деякі роботи я виклав у себе тут на фб
сторінці.

Сподіваюсь, Вам це буде цікаво.

З повагою,

Іван.
