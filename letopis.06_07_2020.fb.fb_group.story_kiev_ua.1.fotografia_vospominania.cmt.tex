% vim: keymap=russian-jcukenwin
%%beginhead 
 
%%file 06_07_2020.fb.fb_group.story_kiev_ua.1.fotografia_vospominania.cmt
%%parent 06_07_2020.fb.fb_group.story_kiev_ua.1.fotografia_vospominania
 
%%url 
 
%%author_id 
%%date 
 
%%tags 
%%title 
 
%%endhead 
\subsubsection{Коментарі}

\begin{itemize} % {
\iusr{Nadiya M Shana}
Красиво о «больном» и близком... Столько всего помнишь  @igg{fbicon.face.hushed} 

\iusr{Lola Madino}
Фребелічки - були виховательки по системі німецького педагога Фребеля.

\begin{itemize} % {
\iusr{Ирина Петрова}
Так, я вже дорослой згадала то "утаємничине" слово, багато прочитала про цю систему.

\iusr{Irena Visochan}
\textbf{Lola Madino} А я даже не слышала это слово, моя бабушка нянечка называла словом ,,Бонна,,

\iusr{Lola Madino}
\textbf{Irena Visochan} бонна могла бути просто вихователькою, безсистемною так би мовити.

\iusr{Галина Гурьева}
А лесгафтички~по системе педагога Лесгафта. Упор делался на физическое развитие ребенка.

\iusr{Ирина Петрова}
В Петербурге университет физической культуры носит имя Лесгафта )
\end{itemize} % }

\iusr{Игорь Мезецкий}

Да... Я восхищен! @igg{fbicon.wilted.flower}  Помнить имена и отчества воспитателей детского садика, -
это потрясающе! И значит то, что эта "первая школа жизни" была знаковым
событием, оставившим очень яркий след... @igg{fbicon.face.astonished} 

\begin{itemize} % {
\iusr{Ирина Петрова}

Мы помним про наш садик всё! Это и прада был наш второй дом, мы были окружены
лаской, вниманием, вкусной и полезной (иногда для нас ббббрррр... едой, в виде
тефтелек с морковной подливкой, рыбьего жЫра... бОльшей гадости вспомнить не
могу((( или чая с молокм!) теперь отлично понимаю, что наш повар Маргарита
Оскаровна для всех нас готовила домашнюю еду! Мы обожали наш садик! Был
маленьким, две группы человек по 20-ть)

\iusr{Игорь Мезецкий}
Удивительно...

\iusr{Ирина Петрова}
\textbf{Игорь Мезецкий} 

наверное, это свойство человеческой памяти - хранить добро и забывать зло, зло
разрушает, организм сам себя ограждает. Самохранение)

\end{itemize} % }

\iusr{Iryna Naidonova}
Ваш а розповідь навіяла і мені думки про дитинство, про всі пісні і друзів з двору і садочка. Дякую!

\iusr{Ирина Петрова}
Дякую!

\iusr{Алена Горобец}
Дякуємо за яскраві спогади!

\begin{itemize} % {
\iusr{Ирина Петрова}
Дякую, що потішились трішечки)

\iusr{Алена Горобец}
\textbf{Ирина Петрова} прочитала Ваші спогади і згадала про власне дитинство. Золоті роки:)

\iusr{Ирина Петрова}
\textbf{Алена Горобец} дякую! Напевно, це така властивість дитинства - бути золотим)
\end{itemize} % }

\iusr{Maksim Pestun}

а мой садик выходил террасами на гору над площадью Франко со стороны Энгельса

\begin{itemize} % {
\iusr{Ирина Петрова}

Максим, Вы будете "бешено хохотаться", но! И ваш садик у нас тоже вызывал
чувство глубокого неудовлетворения! "Те, которые там наверху" почему-то нам
портили нервы, особенно, когда кидали в нас снежками с высоты своего положения
и кривлялись через сетчатый заборчик. Дети - не самые дружелюбные существа?
Нет?)))

\begin{itemize} % {
\iusr{Ирина Петрова}
Максим, мы производим впечатление злющих детишек?))) не, не, мы выросли белыми и пушистыми. Вы же меня видели)))
\end{itemize} % }

\iusr{Maksim Pestun}
я не кидал, честно...

\iusr{Ирина Петрова}
\textbf{Maksim Pestun} конечно, не кидали, мы уже в школе были к времени Вашего прибывания в садике)

\iusr{Maksim Pestun}
меня зимой в садик не пускали. У меня няня была

\iusr{Виктор Бухтияров}
\textbf{Maksim Pestun} я в кінці 50-х в той дітсадок ходив.

\iusr{Олег Будник}
И мой. Но это были уже 80-е. И снежками мы тоже не кидались

\begin{itemize} % {
\iusr{Ирина Петрова}
\textbf{Oleg Budnik} а в 80-х наш садик уже перевели на Институтскую, кидать было у не в кого @igg{fbicon.laugh.rolling.floor}{repeat=3}  но, думаю, что и в нас бы не кидали, вы ж были добрыми детками, да? @igg{fbicon.face.grinning.squinting} 
\end{itemize} % }

\end{itemize} % }

\iusr{Тетяна Кутишенко}
Ну... очень хорошо...
Спасибо
Сижу и улыбаюсь:))
Фото \enquote{под Котовского} - волшебное

\iusr{Ирина Петрова}
\textbf{Тетяна Кутишенко} так теперь это в тренде. Мы просто обогнали эпоху @igg{fbicon.laugh.rolling.floor} 

\iusr{Виктор Бухтияров}

Жінку, яку Ви назвали "бонною", ми - діти, які ходили у приватну групу,
називали вихователькою. У групу дітей батьки віддавали не від хорошого життя.
Просто в державних садах не вистачало місця для усіх.

\begin{itemize} % {
\iusr{Ирина Петрова}
\textbf{Віктор Бухтіяров} 

а як її звали, цю жіночку? Звісно, були причини, цілком зрозумілі, зовсім не
такі, які ми, "великорозумні" тоді малювали у свідомості. Це був стрімке
зростання післявоєнного народження, садиків, хоч і було чимало - на Енгельса
два, у нас, в Пасажі, у будинку Дружба - але, звісно, дітлахів було значно
більше, у школи йшли у сім рочків, тоді декретні відпустки не існували, мама
пішла працювати, коли мені було десь місяців шість. В ясла мені не віддавали,
бо була бабуся. Але ж не у всіх вони були. Такі групи, як Ваша і були дуже
потрібні. Це ж я зараз, не пройшло всього і п'ятидесяти років, розумію. @igg{fbicon.heart.eyes} 

\iusr{Гриша Григорий Панаид}
\textbf{Ирина Петрова} пишите, чисто киевский юмор!!! Молодец

\ifcmt
  ig https://scontent-frx5-2.xx.fbcdn.net/v/t39.1997-6/p480x480/105941685_953860581742966_1572841152382279834_n.png?_nc_cat=1&ccb=1-5&_nc_sid=0572db&_nc_ohc=I6oCfNEh60YAX8gQ6Dr&_nc_ht=scontent-frx5-2.xx&oh=81d690b08f8c72ec6432fea02b2d39f7&oe=61965F0B
  @width 0.15
\fi

\iusr{Ирина Петрова}
\textbf{Гриша Григорий Панаид} дякую! Пишу потрошку @igg{fbicon.face.happy.two.hands} 
\end{itemize} % }

\iusr{Владимир Новицкий}
Прекрасно написано. Получил огромное удовольствие. Спасибо.

\iusr{Ирина Петрова}
\textbf{Vladimir Novitsky} дякую

\iusr{Галина Гурьева}
Хорошо написано, по~домашнему и с большой любовью!

\iusr{Ирина Петрова}
\textbf{Галина Гурьева} благодарю. То, что сердцу мило, всегда так выходит @igg{fbicon.face.happy.two.hands} 

\iusr{Гриша Григорий Панаид}

\ifcmt
  ig https://scontent-frx5-2.xx.fbcdn.net/v/t39.1997-6/s168x128/93025159_222645582401279_8207007965157261312_n.png?_nc_cat=1&ccb=1-5&_nc_sid=ac3552&_nc_ohc=cx_4kk5_uMYAX9UuFn2&_nc_ht=scontent-frx5-2.xx&oh=8ee9579d045329a7efe36dbc26880996&oe=6195CC82
  @width 0.1
\fi

\iusr{Светлана Юшина}
Классные фото, отличные воспоминания! Детство помним всегда!

\begin{itemize} % {
\iusr{Ирина Петрова}
\textbf{Svetlana Yushina} 

да! И возвращаемся туда мыслями, как в тихое пристанище от бед. @igg{fbicon.heart.eyes} 

\iusr{Alik Perlov}
\textbf{Ирина Петрова} 

как ты помнишь про садик?? @igg{fbicon.face.astonished}  Я помню только, что
водили меня в садик в Пассаже . Вход, на против Аптеки был. А площадка была,
где потом были Туалеты))).

\iusr{Ирина Петрова}
\textbf{Alik Perlov} 

да, я знаю про этот садик. Помню ооочень много, может, потому что там было очень хорошо.

\end{itemize} % }

\iusr{Ирина Архипович}
Прелестный рассказ, чудесные фото!!  @igg{fbicon.hands.applause.yellow}  @igg{fbicon.face.happy.two.hands} 

\iusr{Ирина Петрова}
\textbf{Irina Arhipovich} дякую)

\iusr{Natasha Gossett}
Спасибо огромное!
У меня детский сад, как и школа всегда вызывал противоположные чувства. Но получила огромное удовольствие читая вас!

\iusr{Ирина Петрова}
\textbf{Natasha Gossett} да, многие вспоминают садики с нелучшими чувствами, нам очень просто повезло @igg{fbicon.heart.eyes} 

\iusr{Гена Устенко}
Это садик, который над площадью Франка? Там своя прекрасная площадка была.

\begin{itemize} % {
\iusr{Ирина Петрова}
\textbf{Genady Ustenko} да!!! Это тот самый! Площадка теперь общественная)

\iusr{Гена Устенко}
\textbf{Ирина Петрова} я там рядышком жил @igg{fbicon.heart.eyes} 

\iusr{Ирина Петрова}
\textbf{Genady Ustenko} о! Где?

\iusr{Гена Устенко}
\textbf{Ирина Петрова} на станиславского

\iusr{Ирина Петрова}
\textbf{Genady Ustenko} супер! Наверное, много общих знакомых?

\iusr{Гена Устенко}
\textbf{Ирина Петрова} если с 94 школы

\iusr{Ирина Петрова}
\textbf{Genady Ustenko} конечно, вполне возможно! 94-я родная!
\end{itemize} % }

\iusr{Татьяна Петрюк}

По сей день помню в своей душе чувства «зависти», когда я видела подружку,
которую водили в детский сад. Мне казалось, что она самая счастливая девочка
....


\iusr{Ирина Иванченко}

Как мне все знакомо -пенки на манной каше, аппликации на 8Марта, наша группа по
парам в парке... И слёзы на подушке от неразделённого чувства... Именно тогда я
поняла, что у мальчишек,, первым делом самолёты, (рыбалка, велик...), а все
остальное потом.

\begin{itemize} % {
\iusr{Гриша Григорий Панаид}
\textbf{Ирина Иванченко}

\ifcmt
  ig https://scontent-frx5-2.xx.fbcdn.net/v/t39.1997-6/p240x240/12385796_538995839586731_1362775053_n.png?_nc_cat=1&ccb=1-5&_nc_sid=0572db&_nc_ohc=gkvzTb8vPPoAX8cq73P&_nc_ht=scontent-frx5-2.xx&oh=16579ee2708a7c4373d2437955be5e3e&oe=61968381
  @width 0.1
\fi

\iusr{Ирина Серова}
\textbf{Ирина Иванченко} очень хорошо написано, как будто в детстве побывала. С юмором и добротой пишите.

\iusr{Ирина Петрова}

Спасибо всем за добрые слова! Мне очень радостно, что люди светлеют лицами,
вспоминают те самые чудесные дни, когда все были рядом, молодыми и весёлыми.
Да, я писала в своих "Элегиях детства", что были эти времена ооочень
дискуссионными, но, ведь дети этого не знают, было просто счастливо, слезы были
только от неразделённой любви к противному Вовке из старшей группы, да от
зелёнки. И в таких рассказиках я не оцениваю то время "в общем", я не историк,
не аналитик. Может, поэтому они пишутся легко и светло, теперь понимаю, что и
читаются также. @igg{fbicon.heart.eyes} 

\end{itemize} % }

\iusr{Ружена Кислова}

Мой садик был на Спасской. Сейчас там ресторан. по-моему Текила Хаус. У нас
была чудесная воспитательница Таисья Григорьевна и нянечка Галя. Они любили
нас, а мы любили их.

\begin{itemize} % {
\iusr{Ирина Петрова}
\textbf{Регина Яворска} 

уверенна, что все закладывается именно в детстве. Семья, конечно, самое
главное, но, и опыты общения с внешним, жёстким миром тоже зависят от отношения
воспитателей, нянечек, учителей первых классов, когда наиболее уязвима, ранима
душа. Нам повезло, наши опыты были отменными, мы счастливы.


\iusr{Ольга Гондза}
\textbf{Регина Яворска} и я ходила в садик на Спасской. Когда мимо прохожу - часто вспоминаю.
\end{itemize} % }

\iusr{Светлана Казак}

Простите, а в каком году вы "сражались с группКой неприятелей",гуляющих с
фребеличкой? Мой вопрос "не праздный".. Спасибо за милые, добрые воспоминания.. @igg{fbicon.heart.red}

\begin{itemize} % {
\iusr{Ирина Петрова}
\textbf{Светлана Казак} это было в 60, 61, 62 годах. В 63 мы пошли в школу, со многими оказались в одном классе. И, вдруг, Вы сможете добавить ещё крупинки воспоминаний, буду безмерно рада!  @igg{fbicon.heart.eyes} 

\begin{itemize} % {
\iusr{Светлана Казак}
\textbf{Ирина Петрова} 

моя младшая сестра 1961г. р. ходила "в группКу",гуляя в садике И. Франко с
фрибиличкой пока моя мама была занята домашними делами. Я же в эти годы (это
район моего детства, юности и.., и...) каждый день ходила по ул. Ольгинская в
школу 94... Всё знакомо и близко, светло и радостно... А в "группКе" аристократии
не было.. @igg{fbicon.face.wink.tongue}  Мы жили в доме "со звездой"..

\end{itemize} % }

\iusr{Ирина Петрова}
\textbf{Светлана Казак} 

да, конечно, конечно, не было, это я так, художественный образ. А Вы учились с
кем? Старших многих помню, очень дружу с классом на четыре года старше, они
собираются вместе каждый !!!! год, встреча - на площади всегда!!! Может, есть и
общие воспоминания). Ведь эта группа детишек была именно от нехватки мест в
садиках. Радостно, что есть современники!!! @igg{fbicon.heart.eyes} 

\begin{itemize} % {
\iusr{Светлана Казак}
\textbf{Ирина Петрова} 

на \enquote{доме со звездой} есть \enquote{опознавательный знак} @igg{fbicon.face.wink.tongue}, либо на моей ФБ
стр. информации много... @igg{fbicon.wink}  Удачи Вам и добрых воспоминаний @igg{fbicon.hearts.two}  @igg{fbicon.exclamation.mark.double}

\end{itemize} % }

\iusr{Ирина Петрова}
\textbf{Светлана Казак} @igg{fbicon.heart.eyes}  @igg{fbicon.face.eyes.star} 

\end{itemize} % }

\iusr{Таня Ляшенко}

А мой садик был как раз над парком Франка и мы в щели забора наблюдали за
жизнью возле театра. Садик был замечательный. Все воспитатели и даже завхоз и
дворник нас любили. Был какой-то сарайчик и там выращивали кроликов, нам
разрешали их кормить, прекрасные воспоминания! Это была ул. Энгельса, напротив
школы.

\begin{itemize} % {
\iusr{Ирина Петрова}
\textbf{Таня Ляшенко} 

да, да, на Энгельса было два садика, у меня там было много знакомых ребят, мы,
конечно, познакомились уже потом, в школах и даже в институтах)))

\iusr{Таня Ляшенко}
Да, я была в нижнем. 212

\iusr{Таня Ляшенко}
А в школе 217 у меня было тоже много друзей.

\iusr{Виктор Бухтияров}

Именно в садике на Энгельса на всю жизнь мне привили отвращение к молочным
рисовой, манной и гречневой кашам.

\iusr{Ирина Петрова}

Тоже не люблю молочные каши, но, бывает, именно один раз в год мне вдруг тааак
хочется манной с ванилькой и корицей. Как-то даже не поленились в одиннадцать
вечера бежать зимой за молоком в магазинчик под домом @igg{fbicon.face.grinning.squinting}  иначе бы не уснула @igg{fbicon.face.grinning.squinting} 

\begin{itemize} % {
\iusr{Виктор Бухтияров}
\textbf{Ирина Петрова} зимой, в одиннадцать вечера бежать за молоком?! Просто открыть холодильник ....  @igg{fbicon.smile} 
\end{itemize} % }

\iusr{Ирина Петрова}
\textbf{Віктор Бухтіяров} так в том-то и дело, что я не пью молоко, почти ничего с ним не делаю, и его у нас почти не бывает) а всегда ж хочется того, чего нет @igg{fbicon.face.grinning.squinting} 

\iusr{Виктор Бухтияров}
\textbf{Ирина Петрова} у мене в холодильнику є і сало, і шоколадка, і ....

\iusr{Ирина Петрова}
\textbf{Віктор Бухтіяров}  @igg{fbicon.laugh.rolling.floor} 

\end{itemize} % }

\iusr{Людмила Глебова}

Очень похоже на мое детство, только мой садик гулял часто в парке Шевченко...
Тоже фребеличек помню - правда, они разговаривали на немецком, группы были по
три - четыре человека, у каждого ребенка был закругленный небольшой
чемоданчик-сумочка (особый предмет моей зависти), в котором был бутерброд и
яблоко... не раз наблюдала, как группка, сидя на скамейке, "обедала"... Но с
ними мы никак не пересекались. А вообще, садик вспоминаю с отвращением, даже
убегала... Дома лучше ...и было, и есть, и будет!

\begin{itemize} % {
\iusr{Ирина Петрова}
\textbf{Людмила Глебова} 

дом человеку не заменит никто! И не должны! Дом есть Дом) но, если дитеныш идёт
в садик с нетерпением, в школу с удовольствием, на мой взгляд, это кирпичики
отношения к последующей жизни. Мой сын в конце 80-х в садик не ходил, тоже
вырос прекрасным человеком, но, вот такого рассказика он не напишет - нет
материала @igg{fbicon.laugh.rolling.floor}  (шучу)

\begin{itemize} % {
\iusr{Людмила Глебова}
\textbf{Ирина Петрова}, 

у меня две дочки - одна ходила в сад, другая нет! Та, что ходила, вспоминает с
ужасом, а младшая жалеет, что он ,,проплыл,, мимо. Старшая своего сына не
отдавала, а младшая отдала - вот такие шахматы.

\end{itemize} % }

\iusr{Nata Grim}
\textbf{Людмила Глебова} 

вы говорите про металлический чемоданчик-сумочку? У меня тоже был такой.
Изначально в нем были конфетки, может монпансье. Но я точно не помню. Может
родители сами их туда положили.

\begin{itemize} % {
\iusr{Людмила Глебова}
\textbf{Nata Grim}, да про неё... На ней был какой-то детский рисунок, а закрывалась она поворотом застежки, похожей на сердечко...

\iusr{Nata Grim}
\textbf{Людмила Глебова} именно так. Какого-то зеленоватого цвета с рисуночком и розовой, ребристой, пластиковой ручкой.
Мой папа часто бывал в командировках, и в Москве, думаю моя сумочка от туда.
\end{itemize} % }

\iusr{Ирина Петрова}

Да! У меня была! Красная в наклейки, вроде, как чемоданчик путешественника! Она
жила очень много лет, и только двенадцать лет назад, при переезде с
Заньковецкой, ушла к соседской девочке. Носила я там двух маленьких куколок,
пластмассовых девочку и мальчика и их приданное @igg{fbicon.face.grinning.squinting} 

\end{itemize} % }

\iusr{Inatova Irina}
Бесценные фотографии. Спасибо за историю.

\iusr{Ирина Петрова}
Спасибо моему папе, он фотографировал каждый мой шаг, нередко доводя просто до рёва  @igg{fbicon.face.grinning.squinting} 

\iusr{Nata Grim}

У меня было три захода в садики. Первый и самый короткий. Было лето, помню, как
нас водили на прогулку в сад или парк, рядом с частным сектором, не далеко от
садика, через дорогу. В Святошино ( ведомственный садик от з-д Антонова (
раньше это был п/я ...). Все в белых панамачках, держались за ручки по парам.
Вернувшись с прогулки нас поили сладковатым, вкусным, серым напитком, думаю
что-то из дрожжей. Мне нравилось, но многие детки фыркали и плевались. Я, в том
саду, надолго не задержалась, из-за того, что родители получили отдельную,
трёхкомнатную квартиру и мы туда переехали. Вокруг были одни новостройки и
садики только строились, по этому меня водили пешком, по нашей улице ( теперь
Туполева) тогда ещё не ходил автобус. Не взирая на погоду, и в осеннюю слякоть,
и в мороз, и сугробы выше меня, я шагала и не ныла. По тому что, водить меня
поручали, чаще всего, старшему брату, а там не поноешь и на ручки не
попросишься. И так 6 км, каждый день. 3 утром и три вечером.

Садик и сейчас там находится, просто напротив сегодняшней проходной зд Антонов.
И от таких походов я часто болела. Но главной причиной моего покидания этого
садика стали два события. 

Первое:

Гуляя на прогулке и возвращаясь в группу, я провалилась, на территории садика,
в открытую канализацию. И просто чудо, что я не утонула и не отравилась газами,
из канализации. Меня спасла моя цигейковая шуба. Я повисла руками на лесничках
и меня оттуда вытаскивали за такую же цигейковую шапочку. 

А второе проишествие:

это, когда перед новым годом, я разбила голову в туалете, куда нас погнала
воспитательница с прогулки, а нянечка ещё не успела домыть кафельные, скользкие
полы и накричав на деток, те попятились назад. А я, как раз стояла последней и
весь удар пришелся на меня и падая, я ударилась о бетонную перегородку,
облицованную таким же кирпичным кобанчиком. Между первым и вторым событиями
было не больше недели.

Понятно почему мои родители меня забрали из садика.

А третий заход в садик это подготовительная группа, без которой меня бы не
взяли в нашу переполненную русскую школу. Соседняя украинская была полупустой.

\begin{itemize} % {
\iusr{Ирина Петрова}
\textbf{Nata Grim} ой, ой, ой... ничего себе @igg{fbicon.cry}{repeat=3}  ужас((( бедный ребёнок  @igg{fbicon.cry} 

\begin{itemize} % {
\iusr{Nata Grim}
\textbf{Ирина Петрова} 

шрам на лбу со мной на всю жизнь. Я к тому шраму добавила ещё один и он стал
буквой "г". Играли с друзьями в прятки и со всей дури стукнулись лбами друг в
друга, из за угла. Помню искры из глаз. После этого меня стали звать 33
несчастья. Про мои приключения и травмы можно отдельную статью писать. Как
нибудь расскажу.

\iusr{Виктор Бухтияров}
\textbf{Nata Grim} не откладывайте, пожалуйста  @igg{fbicon.smile} 

\iusr{Nata Grim}
\textbf{Виктор Бухтияров} 

ну что же. Мой средний братик принес домой очередную живность. Это была
черепаха. Она была местного происхождения, так как рядом с нашими домами было
много болотец и луг от старой речки Нивки, которая уже ушла под землю и только
местами, в виде озера выходит на поверхность. Так вот принести то он принес, а
как ее кормить не сказал. И мы с соседскими ребятами, методом тыка пробовал чем
ее кормить. Ну не ест и не ест. Тут кто-то из прохожих подсказал, что она
водяная и ест только в воде и исключительно плоть. Я стала садить ее в таз с
водой и в воде давать ей небольшие кусочки мяса.

На улице стояла жара, а нам же захотелось ее выпустить поплавать. Кто-то из
ребят подсказал, что на стадионе, между школами, невысыхающее больтце. И мы
побежали на стадион купать черепаху. Выпустили ее, она нырнет, а через время
покажется головка, как у подводной лодки перископ.

Ну ждём мы ее, ждём не появляется. Кому-то пришло в голову ее в этом болоте
искать. Я зашла в эту жижу, грязи и наступила пяткой на донце разбитой бутылки.
И, понятно мои каникулы: больница Ахмадет, а потом до школы на костылях
проскакала все лето.

\end{itemize} % }

\iusr{Ирина Петрова}
\textbf{Nata Grim}

Наташа, атас... говорят же - с вашим счастьем @igg{fbicon.face.astonished}  как же пришлось вашим паре с
мамой все это пережить... только став мамой, поняла, что такое волноваться за
дитеныша...

\iusr{Nata Grim}

На следующее лето, как под заказ, в последний день школы, я умудрилась сломать
руку.

Иду вся такая счастливая со школы окончила 2 класс. Подружки уже сидят во дворе
ждут. Залетаю домой, бабушка ,как всегда:" мой ручки, садись кушать". А я же
спешу. Бабушка замечание делает не спеши. Прошу ее отпустить меня во двор
погулять. А она мне, как на зло и то и то, а ещё и в магазин сходить молочка
купить. Дала мне овоську с пустыми молочными бутылками, на обмен ( молодые не
поймут, а наше поколение помнит) и 1 рубль. Я его зажала в ладошку и поскакала.
Я с пятого этажа спускалась прыжками, через всю маршевую лестницу ( все
ступеньки, одним махом). Так что бедные мои соседи.

И так это ловко у меня получалось, что сама себе сей час удивляюсь. А на первом
этаже что то пошло не так. Зацепилась нога об ногу и я вылетела по инерции из
подъезда, как пуля. А возле открытой, припертым булыжником ( 19 века, которым
была устлана наша улица Туполева) входной двери. Я ударяюсь, локотком левой
руки, иии.

Реву, как белуга. Бабушка, зная мое везение, кричит что ещё случилось. А я
говорю что упала и ударилась. Плетусь с бутылками вся в слезах, опять домой.
Бабушка, видя меня заревонную, спрашивает: "а где рубль?" Я говорю что забыла.
При всем, что случилось, бутылки остались целые, а рука в локтевом суставе
сломана и локоть в другую сторону. Не показав ручку, отдав бутылки, бегу вниз,
если это можно назвать бегу. А внизу подняв свой выпавший рубль, и зажав его в
ладошку, я сползла по стеночке. В это время идёт соседка и видит, что со мной.
Вызвала скорую помощь. И моей бабушке сказала что сей час приедет скорая и меня
заберёт. У бабушки была грыжа и ей было крайне трудно спускаться и подниматься
с пятого этажа. Тогда жили все дружно, ходили в гости ( смотреть у нас первый в
подъезде, цветной телевизор).

Так что, такое поведение соседей было нормой.

Мне сделали операцию, под наркозом, поставили спицу в сустав и собрали по
осколочкам мой локоток. Мне повезло. Мне тогда делал операцию какой то, потом
стал очень знаменитый хирург-травматолог.

И моя ручка со временем полностью восстановилась.

\iusr{Nata Grim}

Но эта история с поломанной рукой так просто не закончилась.

Пролежала я в больнице, потом ходили на реабилитацию, восстанавливать ручку. И
когда нас вызвали в очередной раз к врачу на осмотр, мама решила отблагодарить
врача, огромным букетом роз. В то время у нас ещё небыло дачи и дачи были
редкость. Папин двоюродный брат дядя Коля, не имел детей, но у них с женой была
шикарная дача. И супруга выращивала красивые цветы. Мы приехали на дачу, с
дядей Колей и мамай, к нему на дачу. Но он не хотел чтобы соседи знали, что к
нему приехали мы. Мало ли что подумают. И мне и маме велено было вести себя
тихо и не привлекать внимание. Они зашли в домик ,а я крутилась у крыльца.

И не знала что прямо перед порогом у них погреб. Он зачем то его открыл. А я
\enquote{ягоза} влетаю в домик иии. Проваливаюсь в погреб. И слышу крик мамы.
Все убился ребенок. Но, и в этот раз, мой ангел хранитель, спас меня. Я попой
скользнула по лестнице и уселась на сундук с картошкой. Даже не поцарапалась.

Мы шли домой с мамой, она и плакала и смеялась. Это было что-то.  Но зато с
огромным букетом раз от дяди Коли и от соседки, которая прибежала на наш крик.

\iusr{Ирина Петрова}

Наташа, Вам никто не предлагал снять фильм про Вашу удачу? "Невезучие" с
Ришаром просто детский наивняк... я даже не могу себе такое представить!!!

\end{itemize} % }

\iusr{Nata Grim}

\ifcmt
  ig https://scontent-frt3-1.xx.fbcdn.net/v/t1.6435-9/107609912_792342894943725_5345759013270677470_n.jpg?_nc_cat=106&ccb=1-5&_nc_sid=dbeb18&_nc_ohc=zGX1gmWLz50AX-pPiRf&_nc_ht=scontent-frt3-1.xx&oh=d987b4386381ecc1418f5e929b41c36d&oe=61B77AA5
  @width 0.3
\fi

\begin{itemize} % {
\iusr{Ирина Петрова}
\textbf{Nata Grim} буфееет!!!! У всех, конечно, были такие!!!! Мы со своим расстались только двенадцать лет назад. Его забрала на дачу знакомая, провела апгрейд, теперь он такой старинный))) мне самой жалко, хнык @igg{fbicon.cry} 

\iusr{Nata Grim}
\textbf{Ирина Петрова} родители готовились к переезду и по этому стоят коробки, телевизор не на тумбочке.

\iusr{Natasha Gossett}
А я телевизора не вижу @igg{fbicon.shrug} 

\iusr{Nata Grim}
\textbf{Natasha Gossett} 

стоит на стульчике, чёрно-белый телевизор (с боковыми ручками). Один из первых
телевизоров, что тогда выпускала отечественная промышленность. Но без линзы.

\iusr{Natasha Gossett}
О, это та большая коробка с ручкой сбоку?
( Я из другого поколения)

\iusr{Nata Grim}
\textbf{Natasha Gossett} телевизор назывался КВН.

\end{itemize} % }

\iusr{Вера-Вета Колесник}

У меня самые теплые воспоминания о садике, по ул. Гарматной. назывался
"Светлячек". Утренники, храню фото. Потом мы почти всей группой, пошли в один
класс. И одноклассники для меня, просто родные люди!

\begin{itemize} % {
\iusr{Ирина Петрова}
\textbf{Вера-Вета Колесник} 

молодцы! Мы тоже с девочками из садика ещё дружим. Одна из них, мне ближе
родственников) сидели за одной партой, в одном институте учились, всю жизнь
рядом!

\iusr{Галина Гурьева}
У меня тоже есть такая подруга, Леночка!
\end{itemize} % }

\iusr{Оксана Алиева}
Очень красивая история...!!!!!

\iusr{Oksana Rode}
Ja etu Maju Antonownu ne nawizschu. Ona bila ne dobraya. A njanka Marichka detey bila skakalkoy po spine. A Klawdia bila klassnaya!

\iusr{Oksana Rode}
Ja w etom sadike bila s 1972 po 1977 god. S utra do wechera. Wospominaniya u menya ne takie swetlie kak u Autora

\iusr{Ирина Петрова}

В этих годах садик уже переехал на Институтскую, так? С момента нашего
пребывания там прошло девять лет. Возможно, за эти годы что- то изменилось в их
жизни. Не знаю... Майя Антоновна с нами всегда была классной, она приносила нам
много девчачьих ништячков, играла с нами в магазин, читала. Клавдия Викторовна
уже у нас была самой старшей, но, конечно, нам так казалось. Нянечкой у нас
была Соня. Она могла прикрикнуть... но, как-то вот именно отношение с нянечкой у
меня в памяти не отложилось. Очевидно, меня это никак не затронуло. Очень
печально, что поменялось все, не знаю, кто был заведующим в ваше время.
Возможно, дело и в этом. Бывает, что ещё зависит от восприятия ребенка. Ведь
детки разные, все. Сочувствую...

\iusr{Oksana Rode}

Sadik bil na Institutskoy. Zawedujuzschaya bila normaljnaya. Pesni mi peli, eto
bilo zdorowo. No wot njanjki.... oni krichali, bili nas, stawili w ugol, ne
dawali nam igratjsa s kuklami, widno boyalisj chto mi ich slomaem

\iusr{Ирина Петрова}

Жаль, что так тогда не повезло с нянями. В тот садик, на Институтской, ходил
младший брат моей подружки, с которой мы были в садике на Франко. Она тоже не
помнит нашу няню, может даже была и не одна. Но, так как мы их не помним,
значит особо нас они не доставали. А вот помним, что очень красивые куклы
сидели на шкафу в кабинетике заведующей, нам тоже играть не давали. Только для
фотографий. В группе были куклы попроще, была мебель кукольная, посудка))) с
подружкой вспомнили, как Майя Антоновна приносила нам в подарки коробочки и
флакончики из-под французских духов Coty.

\iusr{Татьяна Ушакова}

В 50 годы фребелички в Киеве были довольно широко распространены. Я в ожидании
очереди в садик ходила к фребеличке около полугода. Женщина (я уж и имени её не
помню набирала группу 5-6 детей. мы приходили в её квартиру на Владимирской,
она нам читала сказки, рисовали, потом водила нас гулять в парк Шевченко. К
обеду приходили назад, поедали то, что принесли из дому, каждый своё, потом
укладывались спать пару человек на диване, пару на кровати и раскладушках. Вот
и всё времяпровождение. Ни о какой избранности или французском речь не шла. А
кто ещё ходил к фребеличке?

\begin{itemize} % {
\iusr{Виктор Бухтияров}
\textbf{Татьяна Ушакова} неужели Вы эту женщину фребеличкой называли?

\iusr{Татьяна Ушакова}
\textbf{Виктор Бухтияров} Нет, конечно, но не всё, что было в 4года точно запоминается. Я благодарна памяти и за такой формат. Имена моих учителей помню с первого класса.

\iusr{Виктор Бухтияров}
\textbf{Татьяна Ушакова} мы их воспитательницами называли
\end{itemize} % }

\iusr{Kostya Blinnikov}
Дуже зворушливо!

\end{itemize} % }
